\def\year{2022}\relax
%File: formatting-instructions-latex-2022.tex
%release 2022.1
\documentclass[letterpaper]{article} % DO NOT CHANGE THIS
\usepackage{aaai22}  % DO NOT CHANGE THIS
\usepackage{times}  % DO NOT CHANGE THIS
\usepackage{helvet}  % DO NOT CHANGE THIS
\usepackage{courier}  % DO NOT CHANGE THIS
\usepackage[hyphens]{url}  % DO NOT CHANGE THIS
\usepackage{graphicx} % DO NOT CHANGE THIS
\urlstyle{rm} % DO NOT CHANGE THIS
\def\UrlFont{\rm}  % DO NOT CHANGE THIS
\usepackage{natbib}  % DO NOT CHANGE THIS AND DO NOT ADD ANY OPTIONS TO IT
\usepackage{caption} % DO NOT CHANGE THIS AND DO NOT ADD ANY OPTIONS TO IT
\frenchspacing  % DO NOT CHANGE THIS
\setlength{\pdfpagewidth}{8.5in}  % DO NOT CHANGE THIS
\setlength{\pdfpageheight}{11in}  % DO NOT CHANGE THIS
%
% These are recommended to typeset algorithms but not required. See the subsubsection on algorithms. Remove them if you don't have algorithms in your paper.
\usepackage{algorithm}
\usepackage{algorithmic}
\usepackage{threeparttable}
\usepackage{booktabs}

%
% These are are recommended to typeset listings but not required. See the subsubsection on listing. Remove this block if you don't have listings in your paper.
\usepackage{newfloat}
\usepackage{listings}
\DeclareCaptionStyle{ruled}{labelfont=normalfont,labelsep=colon,strut=off} % DO NOT CHANGE THIS
\lstset{%
	basicstyle={\footnotesize\ttfamily},% footnotesize acceptable for monospace
	numbers=left,numberstyle=\footnotesize,xleftmargin=2em,% show line numbers, remove this entire line if you don't want the numbers.
	aboveskip=0pt,belowskip=0pt,%
	showstringspaces=false,tabsize=2,breaklines=true}
\floatstyle{ruled}
\newfloat{listing}{tb}{lst}{}
\floatname{listing}{Listing}


%
%\nocopyright
%
% PDF Info Is REQUIRED.
% For /Title, write your title in Mixed Case.
% Don't use accents or commands. Retain the parentheses.
% For /Author, add all authors within the parentheses,
% separated by commas. No accents, special characters
% or commands are allowed.
% Keep the /TemplateVersion tag as is
\pdfinfo{
/TemplateVersion (2023.1)
}


% DISALLOWED PACKAGES
% \usepackage{authblk} -- This package is specifically forbidden
% \usepackage{balance} -- This package is specifically forbidden
% \usepackage{color (if used in text)
% \usepackage{CJK} -- This package is specifically forbidden
% \usepackage{float} -- This package is specifically forbidden
% \usepackage{flushend} -- This package is specifically forbidden
% \usepackage{fontenc} -- This package is specifically forbidden
% \usepackage{fullpage} -- This package is specifically forbidden
% \usepackage{geometry} -- This package is specifically forbidden
% \usepackage{grffile} -- This package is specifically forbidden
% \usepackage{hyperref} -- This package is specifically forbidden
% \usepackage{navigator} -- This package is specifically forbidden
% (or any other package that embeds links such as navigator or hyperref)
% \indentfirst} -- This package is specifically forbidden
% \layout} -- This package is specifically forbidden
% \multicol} -- This package is specifically forbidden
% \nameref} -- This package is specifically forbidden
% \usepackage{savetrees} -- This package is specifically forbidden
% \usepackage{setspace} -- This package is specifically forbidden
% \usepackage{stfloats} -- This package is specifically forbidden
% \usepackage{tabu} -- This package is specifically forbidden
% \usepackage{titlesec} -- This package is specifically forbidden
% \usepackage{tocbibind} -- This package is specifically forbidden
% \usepackage{ulem} -- This package is specifically forbidden
% \usepackage{wrapfig} -- This package is specifically forbidden
% DISALLOWED COMMANDS
% \nocopyright -- Your paper will not be published if you use this command
% \addtolength -- This command may not be used
% \balance -- This command may not be used
% \baselinestretch -- Your paper will not be published if you use this command
% \clearpage -- No page breaks of any kind may be used for the final version of your paper
% \columnsep -- This command may not be used
% \newpage -- No page breaks of any kind may be used for the final version of your paper
% \pagebreak -- No page breaks of any kind may be used for the final version of your paperr
% \pagestyle -- This command may not be used
% \tiny -- This is not an acceptable font size.
% \vspace{- -- No negative value may be used in proximity of a caption, figure, table, section, subsection, subsubsection, or reference
% \vskip{- -- No negative value may be used to alter spacing above or below a caption, figure, table, section, subsection, subsubsection, or reference

\setcounter{secnumdepth}{2} %May be changed to 1 or 2 if section numbers are desired.

% The file aaai22.sty is the style file for AAAI Press
% proceedings, working notes, and technical reports.
%

% Title

% Your title must be in mixed case, not sentence case.
% That means all verbs (including short verbs like be, is, using,and go),
% nouns, adverbs, adjectives should be capitalized, including both words in hyphenated terms, while
% articles, conjunctions, and prepositions are lower case unless they
% directly follow a colon or long dash

% \title{Randomized Stochastic Gradient Descent Ascent under NC-P{\L} Condition}
\title{Convergence Analysis of Randomized SGDA under NC-P{\L} Condition for Stochastic Minimax Optimization Problems}
% \author{Anonymous Author(s)}
%\affiliations{}

%Example, Single Author, ->> remove \iffalse,\fi and place them surrounding AAAI title to use it
\iffalse
\title{My Publication Title --- Single Author}
\author {
    Author Name
}
\affiliations{
    Affiliation\\
    Affiliation Line 2\\
    name@example.com
}
\fi

% \iffalse
%Example, Multiple Authors, ->> remove \iffalse,\fi and place them surrounding AAAI title to use it
% \title{My Publication Title --- Multiple Authors}
\author {
    % Authors
    Zehua Liu,\textsuperscript{\rm 1}
    Zenan Li, \textsuperscript{\rm 2}
    Xiaoming Yuan, \textsuperscript{\rm 1}
    Yuan Yao \textsuperscript{\rm 2}
}
\affiliations {
    % Affiliations
    \textsuperscript{\rm 1} The University of Hong Kong\\
    \textsuperscript{\rm 2} Nanjing University\\
    liuzehua@connect.hku.hk, lizn@smail.nju.edu.cn, 
    xmyuan@hku.hk,
    y.yao@nju.edu.cn
}
% \fi


% REMOVE THIS: bibentry
% This is only needed to show inline citations in the guidelines document. You should not need it and can safely delete it.
\usepackage{bibentry}
% END REMOVE bibentry

\usepackage{amsmath, bm} 
\usepackage{amssymb}
\usepackage{mathtools}
\usepackage{amsthm}
\usepackage{mathrsfs}
\usepackage{subfigure}
\usepackage{svg}
\usepackage{appendix}
\usepackage{array}

\newtheorem{definition}{Definition}
\newtheorem{assumption}{Assumption}
\newtheorem{lemma}{Lemma}
\newtheorem{theorem}{Theorem}
\newtheorem{corollary}{Corollary}
\newtheorem{remark}{Remark}
\newtheorem{proposition}{Proposition}

\makeatletter
\newenvironment{breakablealgorithm}
{% \begin{breakablealgorithm}
	\begin{center}
		\refstepcounter{algorithm}% New algorithm
		\hrule height.8pt depth0pt \kern2pt% \@fs@pre for \@fs@ruled
		\renewcommand{\caption}[2][\relax]{% Make a new \caption
			{\raggedright\textbf{\ALG@name~\thealgorithm} ##2\par}%
			\ifx\relax##1\relax % #1 is \relax
			\addcontentsline{loa}{algorithm}{\protect\numberline{\thealgorithm}##2}%
			\else % #1 is not \relax
			\addcontentsline{loa}{algorithm}{\protect\numberline{\thealgorithm}##1}%
			\fi
			\kern2pt\hrule\kern2pt
		}
	}{% \end{breakablealgorithm}
		\kern2pt\hrule\relax% \@fs@post for \@fs@ruled
	\end{center}
}
\makeatother

\def\d{\mathcal{D}}
\def\e{\mathbb{E}}
\def\l{\mathcal{L}}
\def\r{\mathbb{R}}
\newcommand{\lizn}[1]{{\color{cyan}{#1}}}
\newcommand{\yy}[1]{{\color{red} YY: #1}}
\newcommand{\lzh}[1]{{\color{blue}{#1}}}

\begin{document}

\maketitle

% \begin{abstract}
% Minimax problems are prevalent in modern machine learning applications, and % first-order algorithms such as SGDA, SGDmax, and ESGDA have become the de facto methods to solve these problems.
% Epoch Stochastic Gradient Descent Ascent (ESGDA) has become the de facto method for solving these problems. %is minimax optimization.
% % Among these first-order algorithms, ESGDA, which carries out several gradient ascent steps followed by one gradient descent step in each iteration, has achieved superior performance in practice. % implementable and has a superior empirical performance.
% Although empirically superior, ESGDA still lacks theoretical guarantees for practical problems beyond the convex-concave assumptions. % in the difficulty of analyzing its efficiency. 
% % Consequently, 
% To fill this gap, a recent pioneering work proposed to analyze the randomized variation of ESGDA, which is called Randomized Stochastic Gradient Descent Ascent (RSGDA). 
% However, the existing analysis is limited to a narrow class of problems with strong assumptions and the convergence results violate the empirical observations.
% %still unsatisfactory in several aspects, including the narrow assumptions and limited convergence results.
% In this paper, we reconsider the RSGDA algorithm by introducing a new technical framework, and establish new convergence results of RSGDA under the mild assumption requiring only % that the objective function satisfies
% the P{\L} condition for one variable.
% Our theoretical results demonstrate that RSGDA can work in the broader condition and has a better convergence rate compared with the previous analysis.  
% Furthermore, based on our analysis, we devise an (inner-)loop-size adaptation strategy in the iterative procedure of the RSGDA algorithm.
% Experimental evaluations confirm the derived convergence results and  the efficacy of our proposed strategy for solving minimax problems. 
% \end{abstract}

\begin{abstract}
    
We introduce a new analytic framework to analyze the convergence of the Randomized Stochastic Gradient Descent Ascent (RSGDA) algorithm for stochastic minimax optimization problems. 
Under the so-called NC-P{\L} condition on one of the variables, our analysis improves the state-of-the-art convergence results in the current literature and hence broadens the applicable range of the RSGDA. 
We also introduce a simple yet effective strategy to accelerate RSGDA , and empirically validate its efficiency on both synthetic data and real data. 
    
\end{abstract}

% Figure environment removed

\section{Introduction}
Automatic 3D reconstruction of clothed humans using image inputs has gained increasing significance due to its potential applications in a wide array of AR/VR scenarios. High-fidelity reconstructions typically depend on sophisticated capture systems, which are developed with dense camera arrays~\cite{collet2015high,joo2015panoptic,joo2018total}, programmable light-stages~\cite{Vlasic2009, guo2019relightables}, and depth sensors~\cite{newcombe2011kinectfusion,DoubleFusion,BodyFusion,dou2016fusion4d,newcombe2015dynamicfusion}. However, stringent capture environments equipped with complex hardware pose significant challenges for consumer-level applications.


In this context, considerable research effort has been dedicated to developing methods that allow for more flexible capture configurations, such as utilizing a few RGB inputs. Among these works, learning implicit functions \cite{iccv2020PIFu, saito2020pifuhd, hong2021stereopifu} has proven effective in achieving highly detailed reconstructions by integrating the advancements of deep neural networks. These methods employ large multi-layer perceptrons (MLPs) to predict the occupancy probability or truncated signed distance function (TSDF) value of every queried 3D point based on its associated local feature, which is extracted from images. They can recover a continuous surface at arbitrary resolutions without topology restrictions.


However, in typical MLP-based implicit networks, the occupancy or TSDF value at each location is solved independently with planar image features, rendering them less capable of addressing challenging cases such as occlusions. Consequently, these methods suffer from generalization and robustness issues, particularly when tackling strong occlusions caused by large motion or multiple interacting humans. 
Some follow-up studies  \cite{zheng2021deepmulticap,zheng2021pamir,huang2020arch} utilize an extra geometric model, SMPL~\cite{Loper2015}, to improve robustness by introducing strong shape priors. 
Their success typically relies on the assumption of geometrical similarity \cite{huang2020arch} between the shape prior and target reconstruction, making them intractable for handling complex cases with loose clothes and sensitive to errors in SMPL model fitting.



%\ping{this paragraph sounds like `TSDF is better than MLP/SMPL, and we use TSDF to solve the problem'. But in Sec 3, we are telling a different story, saying `MLP needs a 3D convolutional encoder'. We need to make these two sections consistent.}\sicong{I think in this paragraph we claim that the TSDF}


%We opt for Trucated Signed Distance Funtion (TSDF) volumetric representations as they are naturally suitable for convolution operations, which have shown remarkable performance for learning hierarchical features on 2D visual perception tasks \cite{SunXLW19}. 
%Meanwhile, TSDF also describes the gradual geometry change around shape surface, which is not reflected by occupancy volume. 

We instead revisit the 3D volumetric representation and resort to 3D convolutional neural networks (CNNs) for feature learning, due to their impressive performance in feature learning and the ability to incorporate spatial context. However, volumetric methods and 3D convolution involve discretization, which might raise concerns regarding whether a discretized volume can preserve subtle geometric details as continuous representations learned in implicit functions. We investigate the relationship between volume resolution and quantization error on synthetic data by converting target mesh objects to TSDF volumes, as shown in Figure~\ref{fig:quantization_error}. We observe that the quantization errors are significantly reduced by increasing volume resolution and become nearly negligible when reaching a relatively high resolution (e.g., 512 or higher). In other words, achieving fine-detailed reconstruction is not supposed to be restricted by the use of volume representations as long as a proper volume resolution is utilized. Therefore, we present a method with high-resolution feature volumes, e.g., 256 and 512, while traditional volumetric methods \cite{varol18_bodynet,gilbert2018volumetric} are often limited to much lower resolutions, such as 32 or 128.



On the other hand, an increase in volume resolution may lead to a cubic growth of memory overhead \cite{8100085}. Reducing memory costs while guaranteeing the granularity of volumetric representations is necessary for pursuing high-quality reconstruction. Thus, we adopt a coarse-to-fine approach and cull away irrelevant voxels to build a sparse high-resolution feature volume. At the coarse level, the network computes an initial TSDF by applying a U-Net with sparse 3D CNN \cite{3DSemanticSegmentationWithSubmanifoldSparseConvNet} on the sparse feature volume, which is carved by a visual hull. Through our experiments, it turns out that more than 95\% of the volume grids are discarded by the visual hull culling, making the sparse 3D CNN efficient. At the fine level, the network focuses on a narrow band near the zero-level set of the initial TSDF and discretizes the narrow band with smaller voxels. By employing this narrow-band culling, we further shrink the sampling space, resulting in a relatively small range of grid numbers (usually 300K--500K in our experiments) even with a high volume resolution of 512. The remaining voxels in the narrow band are associated with features that fuse high-frequency information from the computed normal maps upon the low-frequency shape from the coarse level to compute the TSDF at high resolution. The final mesh is then extracted from the TSDF using the Marching-Cube algorithm ~\cite{Lorensen87marchingcubes}.
% Different from the u-net sturcture to preserve global topology context, we then apply a shallow 3dcnn to compute the final TSDF $D_{final}$ which contain more local geometry detail.




% \ping{this paragraph can be expanded. It is an important contribution and often ignored by other works. stress on the novel idea of regressing blending weights instead of colors}

In addition to geometry, high-quality mesh texture is also a crucial factor contributing to visual appearance. Directly computing a color field in 3D space, as in \cite{iccv2020PIFu}, struggles to capture high-frequency texture details, while the neural radiance field (NeRF) \cite{yu2020pixelnerf} or the DoubleField~\cite{shao2022doublefield} require expensive per-instance optimization and are often unstable for sparse input images. In contrast, we adopt an image-based rendering approach to compute a texture atlas map, which is efficient and widely supported in existing computer graphics tools. 
Specifically, we compute a blending weight at each 3D point on the mesh surface to determine its color as a weighted average of the colors at its image projections. The blending weights can be computed at a relatively coarse resolution, e.g., 512 volume resolution in our case, and leave texture details to the high-resolution images, such as 1K or 2K. Unlike previous methods that generate blurry texturing results under sparse input, our method generalizes well on both synthetic and real data with just a few input views. 
Figure~\ref{fig:teaser} shows two examples reconstructed by our method. Despite the challenging garment, pose, and occlusion, our method recovers faithful shape, normal, and texture on the right.

%with a wide variety of poses and clothing styles, and it is also adaptive to handle input image with arbitrary resolutions.
%\sicong{For this concern we claim that when the resolution of dicretized volume meets certain threshold (which is 256 in our experiment), the quantization error can be neglected.} 



In summary, the main contributions of this paper are as follows:
\begin{itemize}
\vspace{-0.1in}
  \item 
  We revisit the 3D volumetric representation and demonstrate that it can support clothed human reconstruction with equal or even better performance compared to implicit representation. 
  \item 
  We develop a memory and computation-efficient method for high-resolution volumetric reconstruction using sophisticated sparse 3D CNN, coarse-to-fine estimation, and voxel culling by visual hull and narrow bands. 
  \item 
  We introduce a novel method to compute a texture atlas map, which captures rich appearance details from high-resolution input images.
  \item 
  We achieve impressive results on standard benchmark datasets Twindom and MultiHuman, significantly reducing the point-2-surface (P2S) precision to approximately 0.2cm from just six input views, with more than $50\%$ error reduction compared to the state-of-the-art methods, including DoubleField~\cite{shao2022doublefield} and PIFuHD~\cite{saito2020pifuhd}.
\end{itemize}
\paragraph{Unlearning.}

The naive approach to machine unlearning is to retrain a model from scratch with each data deletion request. However, retraining is not feasible for companies with many large models or organizations with limited resources. Thus, the primary objective of machine unlearning is to provide efficient approximations to retraining. Early approaches in security and privacy attempt to achieve exact removal, where an unlearned model is identical to retraining, but are limited in model class~\citep{caoMakingSystemsForget2015, ginartMakingAIForget2019}. \citet{bourtouleMachineUnlearning2020} propose SISA, a flexible approach to exact unlearning that ``shards" a dataset, dividing it and training an ensemble of models where each can be retrained separately. More recent approaches propose approximate removal, requiring the unlearned model to be ``close" to the output of retraining. Some approximate removal methods focus on improving efficiency~\citep{wuDeltaGradRapidRetraining2020} and others try to preserve performance~\citep{wuPUMAPerformanceUnchanged2022}. While these methods apply to a large class of models, they have no formal guarantees on data removal. A second group of approximate approaches provide theoretical guarantees on the statistical indistinguishability of unlearned and retrained models. These noise-based methods leverage convex loss functions to guarantee unlearning with gradient updates~\citep{neelDescenttoDeleteGradientBasedMethods2020a} and Hessian methods~\citep{guoCertifiedDataRemoval2020, sekhariRememberWhatYou, izzoApproximateDataDeletion}. We augment this second set of approximate methods to simultaneously provide strong guarantees on data protection and preserve fairness performance while targeting a common class of models.

\paragraph{Fairness.}

There are a multitude of definitions for fairness in machine learning, such as individual fairness, multicalibration or multiaccuracy, and group fairness. Individual fairness~\citep{dwork2012fairness} posits that ``similar individuals should be treated similarly" by a model. 
On the other hand, recent work has focused on multicalibration and multiaccuracy~\citep{hebert2018multicalibration, kearns2018preventing, deng2023happymap}, where predictions are required to be calibrated across subpopulations. These subpopulation definitions can be highly expressive, containing many intersectional identities from protected groups. In this work, however, we focus on the most commonly studied form of fairness, group fairness, which seeks to balance certain statistical metrics across predefined subgroups. Group fairness literature has proposed various definitions of fairness, but the three most common definitions are Demographic Parity~\citep{zafar2017fairness, feldman2015certifying, zliobaite2015relation, calders2009building}, Equalized Odds, and Equality of Opportunity~\citep{hardt2016equality}. To achieve these definitions, there are generally three approaches to achieving group fairness: \emph{preprocessing} which attempts to correct dataset imbalance to ensure fairness~\citep{calmon2017optimized}, \emph{in-processing} which occurs during training by modifying traditional empirical risk minimization objectives to include fairness constraints~\citep{lowyStochasticOptimizationFramework2022, berk2017convex, agarwal2018reductions, martinez2020minimax}, and \emph{postprocessing} which modifies predictions to ensure fair treatment~\citep{alghamdi2022beyond, hardt2016equality}. 
In this work we focus on in-processing algorithms because they simply modify an objective to account for fairness rather than requiring an additional operation before or after each unlearning request which would also have to be made unlearnable.

\paragraph{Intersections.} Despite advancements in machine unlearning, the literature still lacks sufficient consideration of the downstream impacts of unlearning methods. While recent papers have explored the compatibility of the right to be forgotten with the right to explanation~\citep{krishna2023towards}, there is little work at the intersection of unlearning and fairness. In privacy literature, a thread of work has shown the incompatibility of group fairness with privacy~\citep{esipova2022disparate, bagdasaryan2019differential, cummings2019compatibility} but these incompatibilities arise due to privacy-specific methods, such as gradient clipping and differences in neighboring datasets. Fairness literature has studied the related problem of the influence of training data on fairness~\citep{wang2022understanding}, but does not provide any methods for unlearning. In unlearning literature, recent empirical studies have shown that unlearning can increase disparity~\citep{zhang2023forgotten}, other works have demonstrated the incompatibility of fairness and unlearning for the SISA algorithm \citep{kochno}, and one work~\citep{wang2023inductive} has provided a method to achieve removal and fairness but uses a sharding and retraining algorithm over fairness-corrected graph data for GNNs. In this paper, we propose the first efficient method which achieves fairness while being provably unlearnable without requiring retraining.
\section{Preliminaries}
In this section, we describe the necessary background for automated planning and the significance of the International Planning Competition. 

% \subsection{Ontology}
% A formal ontology is typically represented as a set of concepts, relations, and axioms. A concept represents a set of objects or entities that share common properties, while a relation represents a connection or association between two or more concepts. Axioms are statements that define the relationships between concepts and relations. It is a formal representation of knowledge that is designed to facilitate automated reasoning and information processing. It acts as a structured vocabulary that describes a domain and promotes interoperability, data integration, and communication between humans and machines. Formally, an ontology $O$ can be represented as a tuple $(C, R, A)$, where $C$ is the set of concepts, $R$ is the set of relations, and $A$ is the set of axioms. Each concept \textit{c} $\in$ $C$ can be represented as a set of attributes, denoted as $Att(c)$. Similarly, each relation \textit{r} $\in$ $R$ can be represented as a set of attributes, denoted as $Att(r)$.

% Ontology is a branch of philosophy that deals with the nature of existence and being. In the field of computer science, however, ontology refers to a formal representation of knowledge that is designed to facilitate automated reasoning and information processing. It is a structured vocabulary that describes a domain and promotes interoperability, data integration, and communication between humans and machines. Various tools and methodologies, including Protege and ontology editors, are available for ontology creation. Ontologies are increasingly important in artificial intelligence, knowledge engineering, and the semantic web, and researchers are exploring their potential in diverse domains and applications.

% Figure environment removed

\subsection{Automated Planning}

Automated planning, also known as AI planning, is the process of finding a sequence of actions that will transform an initial state of the world into a desired goal state \cite{ghallab2004automated}. It involves constructing a plan or a sequence of actions that will achieve a specified objective while respecting any constraints or limitations that may be present. Formally, automated planning can be defined as a tuple $(S, A, T, I, G)$, where:
\begin{itemize}
    \item $S$ is the set of possible states of the world
    \item $A$ is the set of possible actions that can be taken
    \item $T$ is the transition function that describes the effects of taking an action on the current state of the world
    \item $I$ is the initial state of the world
    \item $G$ is the desired goal state
\end{itemize}
Using this notation, the problem of automated planning can be framed as finding a sequence of actions $\prec a_1, a_2, ..., a_k\succ$ that will transform the initial state $I$ into the goal state $G$, while respecting any constraints or limitations on the actions. 
 % In automated planning, 
 A problem is defined in terms of a domain and a problem instance. The domain defines the possible actions that can be taken and the effects of each action, while the problem instance specifies the initial state of the world and the desired goal state. 
Various techniques can be used to solve the planning problem, such as search algorithms, constraint-based reasoning, and optimization methods. These techniques involve exploring the space of possible plans and selecting the one that satisfies the objective and any constraints. Figure \ref{fig:planning_bw} illustrates an automated planning scenario for the blocksworld domain, where an initial state can be transformed into a goal state by executing a sequence of actions.

% \noindent \textbf{Attributes modeled about a domain.}
%   %\noindent \textbf{Attributes modeled in a domain file}
%  \begin{enumerate}
%      \item \textbf{Requirements:} A list of requirements that the planner must satisfy in order to solve the domain. Requirements include durative actions, conditional effects, or negative preconditions. For example, in blocksworld domain with types involved, one of the requirements is \emph{typing}.
%     \item \textbf{Predicates:} Predicates are fundamental elements in the planning domain that define the properties of the world. They are used to describe the initial and goal states, as well as the preconditions and effects of actions. Predicates are usually defined as logical expressions over a set of variables, where each variable can take on a finite number of values. In the context of planning, predicates are typically used to represent facts about the world that can be true or false, such as the location of an object or the status of a machine. For example, in blocksworld domain, the predicate \verb|(on b1 b2)| could indicate that block 'b2' is on top of block 'b1'.
%      \item \textbf{Actions:} Actions are the basic units of change in the planning domain. They represent atomic operations that can be performed to transform the world from one state to another. Each action has a name, a set of parameters, preconditions that must be satisfied before the action can be executed, and effects that describe the changes that the action makes to the world. Actions can be used to model a wide variety of operations, ranging from simple movements or transformations to complex processes such as planning or decision-making. For example, in blocksworld domain, the action \verb|unstack b2 b1| can be used to unstack block 'b2' from block 'b1'. 
     
%      \item \textbf{Preconditions:} Preconditions are the conditions that must be true before an action can be executed. They are usually defined using predicates and can involve multiple variables. Preconditions can also be negative, which means that a certain condition must not be true for an action to be executed. In planning, preconditions ensure that actions are only executed when the necessary conditions have been met, such as ensuring that a machine is turned off before it is serviced. For example, in blocksworld domain, the action \verb|unstack b2 b1| has a precondition of \verb|(on b1 b2)|, meaning that for the action to be valid, the block 'b2' should be on top of block 'b1'.
     
%      \item \textbf{Effects:} Effects describe the changes that an action makes to the world. They are usually defined using predicates and can involve multiple variables. Effects can be positive, which means that a certain condition becomes true after the action is executed, or negative, which means that a certain condition becomes false after the action is executed. In the context of planning, effects are used to model the changes that result from executing an action, such as moving an object from one location to another or turning a machine on. For example, in blocksworld domain, when the action \verb|unstack b2 b1| is executed, one of its effect is \verb|(not (on b1 b2))|, indicating that block 'b2' is no longer on top of block 'b1'.
     
%      \item \textbf{Constants:} Constants are values that are fixed and do not change during the execution of the planning problem. They are used to represent objects or entities in the world that have a fixed value, such as the speed limit on a road. Constants can be used to simplify the planning problem by reducing the number of variables that need to be considered and by providing a fixed set of values that can be used in predicates and actions. For example, in blocksworld domain, the constant \emph{table} could represent the surface on which the blocks are initially placed.
     
%      \item \textbf{Types:} Types are used to classify objects or entities in the world based on their attributes or properties. They are used to define the domain of values that a variable can take on and can be used to constrain the values that are assigned to variables. In the context of planning, types are typically used to group related objects or entities together, such as cars or bicycles, and to specify the properties that are common to all members of a type, such as their color or size. For example, in blocksworld domain with types involved, one can represent the predicate as \verb|(on ?x - block ?y - block)| stating that the parameters in the predicate are of type \emph{block}.

%  \end{enumerate}


% ######### Shorter version for AI Planning preliminaries
% \subsection{Automated Planning}

% Automated planning, also known as AI planning, finds actions transforming an initial world state into a goal state \cite{ghallab2004automated}. It involves creating a plan, respecting constraints, defined as $(S, A, T, I, G)$ where $S$ is the world states set, $A$ is the actions set, $T$ is the state transition function, $I$ is the initial state, and $G$ is the goal state. The challenge is to find actions $\prec a_1, a_2, ..., a_k\succ$ converting $I$ to $G$ under constraints. 

% A problem has a domain (defining actions and effects) and an instance (specifying initial and goal states). Various techniques can be used to solve the planning problem, such as search algorithms, constraint-based reasoning, and optimization methods. These techniques involve exploring the space of possible plans and selecting the one that satisfies the objective and any constraints. Figure \ref{fig:planning_bw} illustrates an automated planning scenario for the blocksworld domain, where an initial state can be transformed into a goal state by executing a sequence of actions.

\noindent \textbf{Attributes modeled about a domain.}
 \begin{enumerate}
     \item \textbf{Requirements:} A list of requirements that the planner must satisfy to solve the given domain, e.g., \emph{typing} in blocksworld with types.
     \item \textbf{Predicates:} Define world properties, e.g., \verb|(on b1 b2)| in blocksworld.
     \item \textbf{Actions:} Units of change with preconditions and effects, e.g., \verb|unstack b2 b1| in blocksworld.
     \item \textbf{Preconditions:} Conditions for action execution, e.g., \verb|(on b1 b2)| for \\ \verb|unstack b2 b1|.
     \item \textbf{Effects:} Post-action world changes, e.g., \verb|(not (on b1 b2))| after \\ \verb|unstack b2 b1|.
     \item \textbf{Constants:} Fixed values, e.g., \emph{table} in blocksworld.
     \item \textbf{Types:} Classifications based on attributes, e.g., \\ \verb|(on ?x - block ?y - block)| in typed blocksworld.
 \end{enumerate}

\noindent \textbf{Attributes modeled about a problem instance from a domain.}
\begin{enumerate}
    \item \textbf{Name:} The name of the planning problem.
    \item \textbf{Domain:} The name of the planning domain that the problem belongs to.
    \item \textbf{Objects:} A list of objects that are present in the planning problem. Objects are typically defined in terms of their type and name. In the example shown in Figure \ref{fig:planning_bw}, objects are b1, b2, and b3.
    \item \textbf{Initial State:} A description of the initial state of the world, including the values of all relevant predicates. Figure \ref{fig:planning_bw} represents an example initial state.
    \item \textbf{Goal State:} A description of the desired goal state of the world, including the values of all relevant predicates. Figure \ref{fig:planning_bw} represents an example goal state.
\end{enumerate}

% \vspace{2cm}
\subsection{International Planning Competition (IPC)}

% IPC serves as a significant means of assessing and comparing various planning systems. By presenting new planners and benchmark problems each year, the competitions aim to stimulate the advancement of new planning methodologies and reflect current trends and challenges in the field. The competition comprises multiple tracks, each covering various planning problems such as classical, temporal, and probabilistic planning. These tracks include benchmark problems that evaluate the performance of planners concerning parameters such as plan quality, plan length, and run time. The results of these competitions provide insights into the current state-of-the-art in planning and help identify the strengths and weaknesses of different planning systems. IPC can serve as an excellent starting point for building a planning-related ontology as the benchmark problems used in these competitions can provide a comprehensive overview of the domain and the types of problems that planners need to solve. 

IPC is pivotal for evaluating and contrasting planning systems. Introducing new planners and benchmarks, it promotes innovative planning methodologies and reflects the field's evolving challenges. The competition has multiple tracks, such as classical and probabilistic planning, with benchmarks assessing plan quality, length, and run time. IPC results offer a glimpse into the latest in planning, highlighting system pros and cons. The benchmarks from IPC are ideal for crafting a planning-related ontology, encapsulating the domain's breadth and planners' challenges.

\section{Randomized stochastic gradient descent ascent (RSGDA)} \label{sec: rsgda}
%\yy{A table summarizing the conditions and results in introduction would help readers quickly get a big picture of your work.}
In this section, 
we first discuss the motivation of RSGDA (Section \ref{sec: 4.1}),
and show that RSGDA converges to the unique Nash equilibrium of $F$ in the SCSC setting (Section \ref{sec: 4.2}).
Since it is impossible to discuss the convergence to Nash equilibriums in the nonconvex-nonconcave setting, we next show that RSGDA converges to a Nash-type stationary point  and a Stackelberg-type stationary point in the NC-P{\L} condition (Section \ref{sec: 4.3}).
Finally, based on the theoretical results, we propose a selection strategy for $p$ (Section \ref{sec: 4.4}).
    
\subsection{Motivation of RSGDA} \label{sec: 4.1}

The intuition of ESGDA is straightforward. 
On the one hand, SGDmax contains a complete theory but is impractical in applications due to the calculation of $y^*$. 
On the other hand, although SGDA is tractable, it lacks theoretical guarantees and often fails in many cases.
ESGDA takes advantage of both SGDmax and RSGDA. 
% ESGDA takes a fixed number of steps in $y$ and one step in $x$ in each iteration. 
In iterations of ESGDA, 
the multiple steps in $y$ not only provide a reasonable estimation of $y^*$ but are also tractable.

Though ESGDA performs better than SGDA in applications, its theoretical properties are still unclear. 
Analyzing its properties is challenging due to the following two technical reasons.
First, the multiple gradient steps in $y$ create several immediate variables, causing the gap between $y_k$ and $y_{k+1}$. 
Second, a theoretical analysis of the best inner update steps seems to be unachievable if it is analyzed by classical techniques.
Therefore, we focus on the randomized version of ESGDA, i.e., RSGDA, as a surrogate. 
% \lizn{Need a further conclusion sentence}

\subsection{Convergence under SCSC condition} \label{sec: 4.2} %\label{subsec: 3.1}
The existing theoretical analysis~\cite{farnia2021train} demonstrates that, 
when $F$ is SCSC and with noiseless gradients, SGDA and ESGDA can successfully strongly converge to the unique Nash equilibrium with linear convergence rate. 
Thus, we first show that, RSGDA enjoys the same convergence property with SGDA and ESGDA in the SCSC setting. 
In other words, the randomized update of $x$ and $y$ will not damage the convergence results. 
Throughout this subsection, we always assume that $\sigma = 0$, i.e., the gradient estimation is exact.
% Before further details about Algorithm \ref{algo: rsgda}, we first consider it in a simple case: $F$ is $\mu$-SCSC. As discussed above, if $F$ is $\mu$-SCSC, it must have the NC-P{\L} condition. However, due to the rich theoretical results for the SCSC function, it is meaningful to consider Algorithm \ref{algo: rsgda} under the SCSC condition separately. 
Note that when $F$ is SCSC, there is a unique Stackelberg equilibrium and a unique Nash equilibrium of $F$, and the Stackelberg equilibrium and Nash equilibrium are equal~\citep{rockafellar_convex_1970}.
For simplicity, we eliminate the random term $z$ due to the noiseless gradients setting, and assume that the step sizes are constant, i.e.,  for any $k \geq 0$, $\alpha_k \equiv \alpha$ and $\eta_k \equiv \eta$ for some given values $\alpha$ and $\eta$.  
The convergence result is concluded in the following theorem, with all the proofs deferred to the appendix. 
% For ease of notation, we define the operator $H_{\text{RSGDA}}$ as follows: 
% \begin{equation*}
%     H_{\text{RGDA}} (x, y) := 
%     \begin{dcases}
%         (x - \alpha \nabla_x F(x, y), y), & \quad \text{w.p.} \quad p, \\ 
%         (x, y + \eta \nabla_y F(x, y)), & \quad \text{w.p.} \quad 1 - p.
%     \end{dcases}
% \end{equation*}

\begin{theorem} \label{thm: rgda is contractive}
Assume that $F$ is $\mu$-SCSC and $\sigma = 0$. Let $\{ (x_k, y_k) \}$ be the sequence generated by Algorithm \ref{algo: rsgda}, $(x^*, y^*)$ be the Nash equilibrium of $F$. 
For sufficiently small $\alpha = \eta$, there exists a constant $\rho < 1$, such that $\forall k \geq 0$, 
\begin{equation*}
    \e_k [\| (x_{k+1}, y_{k+1}) - (x^*, y^*) \|^2] \leq \rho \| (x_k, y_k) - (x^*, y^*) \|^2.
\end{equation*}
\end{theorem}
% \lizn{Conclusion: vs. GDA}

\noindent{\em Remarks}.
Intuitively, Theorem \ref{thm: rgda is contractive} states that RSGDA is a quasi-contractive operator in the expectation viewpoint. 
Consequently, we can prove that RSGDA converges linearly to the minimax point for a $\mu$-SCSC function $F$ in expectation, which shares the same convergence rate as SGDA and ESGDA in the SCSC setting with $\sigma = 0$. 

\begin{corollary}
    
Consider the setting of Theorem \ref{thm: rgda is contractive}, $\{(x_k, y_k)\}$ converges to the minimax point $(x^*, y^*)$ linearly in expectation.

\end{corollary} \label{coro: 1}

% \lizn{Why randomized version? compared with ESGDA?}

\subsection{Convergence under NC-P{\L} condition} \label{sec: 4.3}

%We have proved the convergence of RSGDA for the $\mu$-SCSC function. 
Next, we switch to analyze the property of RSGDA under the NC-P{\L} condition, which is an extension of the result of the $\mu$-SCSC case.  
In general, SGDA and SGDmax can both converge to a Stackelberg-type stationary point in the NC-P{\L} condition. 
However, there is no theoretical analysis ensuring that ESGDA converges in either the NCSC (nonconvex strongly concave) setting or the NC-P{\L} condition~\cite{sebbouh2021randomized}, due to the complicated structure of ESGDA. 
Hence, we analyze the efficiency of RSGDA in the NC-P{\L} condition. 
Our analysis of RSGDA is similar to the analysis of SGDA, which makes the theory simple and transparent.
Specifically, if function $F$ admits the NC-P{\L} condition, we have the following convergence result for Algorithm~\ref{algo: rsgda}.

\begin{theorem} \label{thm: convergence of rsgda}
% Let Assumption \ref{assum: smoothness}, \ref{assum: bounded variance} $\&$ \ref{assum: ncpl condtion} hold. 
Assume that $F$ is $L_1$-smooth and satisfies the NC-P{\L} condition. 
Let $\alpha_k \leq \frac{1}{2L_2}$ and $18\kappa^2 \frac{p}{1-p}  \alpha_k \leq \eta_k \leq \frac{1}{L_1}$ for $k \geq 0$, 
and assume that the stepsizes $\alpha_k$ and $\eta_k$ are square summable but not summable, 
i.e.,
$ \sum_k \alpha_k = + \infty, \sum_k \eta_k = + \infty$, and 
$\sum_k \alpha_k^2 < + \infty, \sum_k \eta_k^2 < + \infty$.
Then for non-increasing $\{ \alpha_k \}$, we have 
\begin{equation}
    \min_{t=0, 1, \dots, k-1} h_t = o \left( \frac{1}{\sum_{j=0}^{k-1} \alpha_j} \right) \to 0, \quad \textrm{almost surely},
\end{equation}
where $h_t = \frac{1}{4} \| \nabla \phi (x_t) \|^2 + \frac{1}{20} \left( \frac{L_1}{\mu} \right)^2 \| \nabla_y F(x_t, y_t) \|^2 + \frac{11}{40} \| \nabla_x F(x_t, y_t) \|^2$.

\end{theorem}

\noindent{\em Remarks.}
% Before we provide a sketch proof of Theorem \ref{thm: convergence of rsgda}, we first discuss what $h_t$ is.
% We claim that 
Here $h_t$ can be viewed as an efficiency measure of the RSGDA.
In fact, we have $\| \nabla F(x_t, y_t) \|^2 \lesssim h_t$ and $\| \nabla \phi (x_t) \|^2 \lesssim h_t$. Thus, we obtain that 
\begin{equation*}
    \min_{t=0, \dots k} \| \nabla F(x_t, y_t) \|^2 + \| \nabla \phi (x_t) \|^2 \lesssim \min_{t=0, \dots, k} h_t,
\end{equation*}
which means that $h_t$ is an upper bound of $\| \nabla F(x_t, y_t) \|^2 + \| \nabla \phi (x_t) \|^2$.
Hence, to obtain a Nash-type stationary point and a Stackelberg-type stationary point simultaneously, it is sufficient to ensure that $\min_{t=0, \dots, k} h_t \to 0$ as $k \to \infty$.
%Hence, roughly speaking, 
In other words, Theorem \ref{thm: convergence of rsgda} essentially states that RSGDA converges to a Nash-type stationary point and a Stackelberg-type stationary point simultaneously under some suitable conditions. 

A detailed proof is provided in Appendix~\ref{app: rsgda}. 
In a nutshell
, to prove Theorem \ref{thm: convergence of rsgda}, inspired by the work \cite{yang2021faster}, we introduce a Lyapunov function 
\begin{equation*} 
V (x, y):= \phi (x) + C (\phi (x) - F (x, y)), 
\end{equation*}
where $C > 0$ is a constant to be determined later. 
Note that for any $(x, y)$, we have $\phi (x) \geq F(x, y)$ according to the definition of $\phi$.
Hence, $V$ is bounded from below.
Now, for any $k \geq 0$, we define
\begin{equation*}
    V_k := V(x_k, y_k) = \phi (x_k) + C(\phi (x_k) - F(x_k, y_k)).
\end{equation*}
Due to the random term $p$ and the stochastic term $z_k$ in the definition of $(x_{k+1}, y_{k+1})$, comparing $V_k$ and $V_{k+1}$ directly is meaningless.
However, from the stochastic process perspective, one can discuss the gap between $\e_k[V_{k+1}]$ and $V_k$, where $\e_k[\cdot]$ is the conditional expectation.
Next, we show that $\{ V_k \}_k$ is similar to a submartingale which means that it is ``non-increasing'' in the conditional expectation meaning.
In particular, we provide an inequality connecting $\e_k[V_{k+1}]$ and $V_k$.
Finally, by applying the Robbins-Siegmund theorem~\cite{robbins1971convergence} to this inequality, we can obtain the almost surely convergence of RSGDA.

% We claim that we can obtain a Stackelberg-type stationary point and a Nash-type stationary point simultaneously from the sequence $\{ x_k, y_k \}_k$ by Theorem \ref{thm: convergence of rsgda}. In fact, on one side, we have $\| \nabla F(x_t, y_t) \|^2 \lesssim h_t$ and $\| \nabla \phi (x_t) \|^2 \lesssim h_t$. Thus, we obtain that 
% \begin{equation*}
%     \min_{t=0, \dots k} \| \nabla F(x_t, y_t) \|^2 + \| \nabla \phi (x_t) \|^2 \lesssim \min_{t=0, \dots, k} h_t.
% \end{equation*}
% On the other side, note that for any given $k$, $\min_{t=0, \dots, k} h_t$ is bounded by $1 / \sum_{j=0}^k \alpha_j$ and $\sum_j \alpha_j = +\infty$. Hence, $\min_{t \geq 0} h_t = 0$ almost surly. Combining the two observations, we get our claim.

Furthermore, we can get the convergence rate of Algorithm \ref{algo: rsgda} from a straightforward observation of Theorem \ref{thm: convergence of rsgda}, which is concluded in the following theorem.

\begin{theorem}[Convergence rate] \label{thm: convergence rate}

Consider the setting of Theorem \ref{thm: convergence of rsgda}, for any $\epsilon > 0$, there are sequences $\{ \alpha_k \}$ and $\{ \eta_k \}$, such that
\begin{equation}
    \min_{t = 0, 1, \dots, k-1} h_t = o \left( k^{- \frac{1}{2} + \epsilon}  \right),
\end{equation}
almost surely.

\end{theorem}

Finally, we analyze two specific versions of RSGDA, i.e., randomized gradient descent ascent (RGDA) and constant step RSGDA.

\noindent\textbf{RGDA}.
We first consider RGDA, where we use the exact gradients in Algorithm \ref{algo: rsgda}.

\begin{corollary} \label{coro: 2}

Let Assumption \ref{assum: smoothness}, \ref{assum: bounded variance} \& \ref{assum: ncpl condtion} hold with $\sigma^2 = 0$. Assume that  $\alpha_k \equiv \alpha$ and $\eta_k \equiv \eta$ for all $k \geq 0$. Moreover, we assume that $\alpha \leq 1 / (2L_2)$ and $18 \frac{p}{1-p} (L_1/\mu)^2 \alpha = \eta \leq 1 / L_1$, where $L_2 = L_1 + \frac{L_1 \kappa}{2}$ and $\kappa = L_1 / \mu$. Then
\begin{equation}
    \min_{t=0,\dots, k} \e [h_t] = O \left( \frac{1}{k} \right).
\end{equation}
    
\end{corollary}

\noindent\textbf{Constant step RSGDA}.
Another variant of RSGDA is choosing constant step sizes for RSGDA. Though RSGDA with constant steps does not converge, we can provide the computation complexity of obtaining an approximate local solution to problem (\ref{problem: stochastic minimax problem}).

\begin{corollary} \label{coro: 3}
    
Let Assumption \ref{assum: smoothness}, \ref{assum: bounded variance} \& \ref{assum: ncpl condtion} hold. Assume that for any $k \geq 0$, we have $\alpha_k \leq 1 / (2L_2)$ and $18 \frac{p}{1-p} (L_1/\mu)^2 \alpha_k = \eta_k \leq 1 / L_1$, where $L_2 = L_1 + \frac{L_1 \kappa}{2}$ and $\kappa = L_1 / \mu$. Moreover, assume that $\alpha_k \equiv \alpha$ and $\eta_k \equiv \eta$ for all $k \geq 0$. Then for any $\epsilon > 0$, if $k = O (\epsilon^2)$, then
\begin{equation}
    \min_{t=0, \dots, k} \e [h_t] \leq \epsilon.
\end{equation}
    
\end{corollary}

%\begin{remark}
\noindent{\em Remarks}.
Corollary \ref{coro: 3} indicates that the computation complexity of RSGDA is the same as SGDA. 
In other words, RSGDA is as fast as SGDA.
In fact, for any $\epsilon > 0$, SGDA (see like \citet{yang2021faster}) provides a point $x$ such that $\| \nabla \phi (x) \|^2 \leq \epsilon$ in $O(1/\epsilon^2)$ steps.
On the other side, note that $h_k \geq\frac{1}{4} \| \phi (x_k) \|^2$ for any $k \geq 0$.
Hence, Corollary \ref{coro: 3} states that RSGDA provides a point $x$ such that $\| \nabla \phi (x) \|^2 \leq \epsilon$ in $O(1/\epsilon^2)$ steps, which coincides with the results of SGDA.
It means that RSGDA shares the same theoretical computation complexity as SGDA.
%\yy{this paragraph seems to be a mismatch to the above clrollary.}

%\end{remark}

%\yy{overall, the above two sections are still messy to me. consider to add a figure showing the logic flow (this might be rare but I guess a figure may help the readers to follow), and a table to summarize the assumptions/results/etc. of existing methods and your method.}

\subsection{Selection of p} \label{sec: 4.4}

In the last part, we discuss about the value $p$ in applications.
To the end of this subsection, we always assume that the stepsizes are constants.
Theoretically, for any $p \in (0, 1)$, RSGDA converges under some suitable choices of stepsizes. 
However, the empirical experiments show that RSGDA performs better in some $p$ than others.
We propose an intuitive idea to determine the value $p$ in this part.

Recall that $h_k$ is introduced to measure the efficiency of RSGDA.
Thus, for any fixed $n$, the smaller value $\sum_{k=1}^n h_k$, the faster RSGDA converges numerically.
Hence, the basic idea of determining $p$ is to minimize the upper bound of $\sum_{k=1}^n h_k$ by choosing suitable parameter $p$.
Specifically, we choose $p$ according to the following rule.

\begin{proposition} \label{prop: 1}

Consider the setting of Corollary \ref{coro: 3}.
For any initial point $(x_0, y_0)$, There are constant $M_1, M_2 > 0$ independent of $\{ z_1, x_1, y_1, \dots \}$ (the explicit form of $M$ is given in the Appendix), such that the optimal $p$ is given as follow.
\begin{equation}
p =
\begin{dcases}
    \min \left\{ \frac{M_1}{\sqrt{n}} - \frac{M_2}{n} ,\frac{L_2}{9 L_1 \kappa^2 + L_2} \right\}, & \quad \sigma > 0; \\
    \frac{L_2}{9 L_1 \kappa^2 + L_2} & \quad \sigma = 0.
\end{dcases}
\end{equation}
where $n$ is the step number.

\end{proposition}

We give a proof sketch here.
Throughout a complicated computation, one can show that 
\begin{equation} \label{equ: 4.11}
    \sum_{k=1}^n h_k \leq \frac{1}{\alpha p} (V_1 - \inf V) + \frac{18^2 p n}{2 (1-p)} \kappa^4 L_1 \alpha \sigma^2 + \mathrm{const},
\end{equation}
where $V_1$ and $\inf V$ are two constants determined by the initial values, and $\mathrm{const}$ is a constant.
We aim to find the minimization of RHS of \eqref{equ: 4.11} with respect to $p$.
The basic theory of calculus shows that the minimization of RHS of \eqref{equ: 4.11} can be obtained at the value offered in Proposition \ref{prop: 1}.

Roughly speaking, Proposition \ref{prop: 1} states that one should choose the probability $p$ obeying the following rules.
In case the estimation of the gradient is exact, i.e., there is no variance, the hyperparameter $p$ should be as large as possible.
In case $\sigma > 0$, $p$ is the smaller one between $O(1 / \sqrt{n})$ and a constant $\frac{L_2}{9 L_1 \kappa^2 + L_2}$.
Hence, if $n$ is small, $p$ should be a constant; if $n$ is sufficiently large, $p$ should decrease in the rate $O(1/ \sqrt{n})$.
In other words, RSGDA prefers a large probability $p$ when the iteration number $n$ is small and a small $p$ with order $O(1/\sqrt{n})$ in the case $n$ is large.
Hence, we propose an intuitive method called Adaptive-RSGDA (AdaRSGDA) to adjust the parameter $p$.
First, we choose an integer $N_1 > 0$, a constant $N_2 > 0$, and initial probability $p_0$.
For the iteration step $n < N_1$, we fix $p = p_0$.
For the case $n \geq N_1$, we choose $p = 1/([(n-N_1) / N_2] +1)$ for each $N_2$ steps, where $[x]$ denotes the largest integer smaller than $x$.
We use this simple notation to estimate the term $O(1/\sqrt{n})$.
A straightforward observation shows that AdaRSGDA performs like SGDA at the starting $N_1$ steps and naturally changes to ESGDA from the $N_1 + 1$ step.


% In summary, an intuitive choice of $p$ is that for the first several iterations, $p$ should be large enough, then $p$ decreases gradually.
% In other words, RSGDA should take more gradient descent steps at the beginning of the iteration and gradually takes more gradient ascent steps.



\section{Experiment}

\subsection{Datasets and metrics}

% \noindent\textbf{Dataset.}
\subsubsection{Dataset}
% We adopt three datasets in our experiments, i.e., ClearGrasp \cite{sajjan2020clear}, TransCG \cite{fang2022transcg} and ClearPose \cite{chen2022clearpose}. The ClearGrasp dataset is the pioneering large-scale synthetic dataset that specifically focused on transparent objects. It provids a large-scale synthetic dataset as well as a real-world benchmark. The TransCG dataset comprises 57K RGB-D images from 130 different real-world scenes. 
% ClearPose dataset contains 350K RGB-D images of 63 household objects in real-world settings. Depth completion experiments and generalization verification (reported respectively in Section \ref{sec:depth} and \ref{sec:generalization}) are conducted on ClearGrasp, TransCG and ClearPose. Ablation study (reported in Section \ref{sec:ablation}) is performed on TransCG.
We use three datasets including ClearGrasp \cite{sajjan2020clear}, TransCG \cite{fang2022transcg}, and ClearPose \cite{chen2022clearpose}. The ClearGrasp dataset is a pioneering large-scale synthetic dataset that specifically focuses on transparent objects. It provides a large-scale synthetic dataset as well as a real-world benchmark. The TransCG dataset comprises 57K RGB-D images from 130 different real-world scenes. The ClearPose dataset contains 350K RGB-D images of 63 household objects in real-world settings. 
% We conducted depth completion experiments and generalization verification on ClearGrasp, TransCG, and ClearPose, reported respectively in Section \ref{sec:depth} and \ref{sec:generalization}. We performed an ablation study on TransCG, which is reported in Section \ref{sec:ablation}.

% ClearGrasp\cite{sajjan2020clear} is the first large-scale synthetic dataset as well as a real-world test benchmark focusing on transparent objects. TransCG\cite{fang2022transcg} is a large-scale real-world dataset, which contains 57K RGB-D images from 130 different scenes. ClearPose\cite{chen2022clearpose} is a recentily proposed real-world dataset, containing 350K RGB-D images covering 63 household objects.

% \newgeometry{letterpaper,top=60pt,bottom=43pt,left=48pt,right=48pt}
% \begin{table*}[!t]
% \caption{Ablation study. We show the impact of progressively substituting the components of the DFNet with ours. \label{tab:table1}
% }
% \centering
% \resizebox{\linewidth}{!}{%
% \begin{tabular}{cccccccccc}
% \toprule
% Model/Metric    & RMSE  & REL   & MAE   & $\delta$1.05 & $\delta$1.10 & $\delta$1.25          & Inference time (s)& Parameters & Size (MB)   \\ \midrule
% DFNet\cite{fang2022transcg}          & 0.018 & 0.027 & 0.012 & 83.76 & 95.67 & 99.71          & 0.0244s        & 1.25M & 4.819MB \\ \midrule
% New Loss        & 0.017 & 0.026 & 0.012 & 84.42 & 96.30 & \textbf{99.81} & 0.0244s        & 1.25M & 4.819MB \\ \midrule
% Shortcut Fusion & 0.017 & 0.024 & 0.011 & 86.18 & 96.67 & 99.79          & 0.0218s        & 1.02M & 3.919MB \\ \midrule
% Ours(slim) & 0.016          & 0.024          & 0.011          & 86.22          & 96.64          & \textbf{99.81} & \textbf{0.0143s} & \textbf{0.39M} & \textbf{1.518MB} \\ \midrule
% Ours       & \textbf{0.015} & \textbf{0.022} & \textbf{0.010} & \textbf{88.18} & \textbf{97.15} & \textbf{99.81} & 0.0153s          & 1.25M          & 4.803MB          \\
% \bottomrule
% \end{tabular}%
% }
% \end{table*}
\begin{table}[!t]
\renewcommand{\arraystretch}{1.05}
\setlength{\tabcolsep}{5pt}
\caption{Ablation study. We show the impact of progressively substituting the components of the DFNet with ours. \label{tab:table1}
}
\centering
\resizebox{\linewidth}{!}{%
\begin{threeparttable}
\begin{tabular}{cccccccccc}
\toprule
Model   & RMSE  & REL   & MAE   & $\delta$1.05 & $\delta$1.10 & $\delta$1.25          & Time(s)& Para(M) & Size (MB)   \\ \midrule
DFNet\cite{fang2022transcg}          & 0.018 & 0.027 & 0.012 & 83.76 & 95.67 & 99.71          & 0.0244        & 1.25 & 4.819 \\ \midrule
Huber Loss &0.017   &0.027  &0.012  &84.10  &95.82  &99.74 &0.0244  &1.25   &4.819  \\ \midrule
New Loss        & 0.017 & 0.026 & 0.012 & 84.42 & 96.30 & \textbf{99.81} & 0.0244        & 1.25 & 4.819 \\ \midrule
SF* & 0.017 & 0.024 & 0.011 & 86.18 & 96.67 & 99.79          & 0.0218        & 1.02 & 3.919 \\ \midrule
Ours(s)* & 0.016          & 0.024          & 0.011          & 86.22          & 96.64          & \textbf{99.81} & \textbf{0.0143} & \textbf{0.39} & \textbf{1.518} \\ \midrule
Ours       & \textbf{0.015} & \textbf{0.022} & \textbf{0.010} & \textbf{88.18} & \textbf{97.15} & \textbf{99.81} & 0.0153          & 1.25          & 4.803          \\
\bottomrule
\end{tabular}%
% \multicolumn{10}{l}{Note: NL* represents New Loss, SF* represents Shortcut Fusion and Ours(s)* represents Ours(slim).}
\begin{tablenotes}
\footnotesize
\item Note: SF* represents Shortcut Fusion and Ours(s)* represents Ours(slim).
\end{tablenotes}

\end{threeparttable}
}


\end{table}
% \vspace{-0.5cm}
\subsubsection{Metrics}
For evaluating the performance of our depth completion model, we employ four common metrics: RMSE, REL, MAE and Threshold $\delta$ (where $\delta$ is set to 1.05, 1.10, and 1.25). These metrics are calculated only on the transparent areas, as determined by transparent masks.
% Me use common metrics RMSE, REL, MAE and Threshold $\delta$ ($\delta$ is set to 1.05, 1.10 and 1.25) to evaluate our model. All metrics are calculated on the transparent areas according to transparent masks.


% We use three metrics to evaluate performance on pose estimation task. The average closest point distance (ADD-S)\cite{xiang2017posecnn} calculates the mean distance from each 3D model point to its closest neighbor on the target model. Followed DenseFusion\cite{wang2019densefusion} we report the area under the ADD-S curve (AUC) and the percentage of ADD-S smaller than 2cm ($<$2cm).

\subsection{Implementation Details}
% \noindent
% \textbf{Network configuration.}
\subsubsection{\bf Network Configuration}
% \textcolor{blue}{
In the network architecture, the number of hidden channels, \textbf{$C$}, is set to 64. Each FFEB/DFCB contains a single OSA module. Each OSA module is composed of 5 layers with stage channels of 20. The SFM module maintains \textbf{$C$} channels throughout the pipeline, while cross-layer shortcuts have only 1 channel. Residual connections between the encoder and decoder retain only \textbf{$C$} channels. The input head module and output head module use $3\times3$ convolution to adjust the number of channels and resolution (with resolution changes only occurring in the input head module). For the slim version, \textbf{$C$} is set to 32, and the OSA block contains 4 layers with stage channels of 16.
% }
% The hidden channels \textbf{$C$} in the network is set to 64. Each FFEB/DFCB contains one OSA module, in which, we use 5 layers per block and set stage channels \textbf{$C'$} to 20. SFM keeps \textbf{$C$} channels throughout the pipeline while cross-layer shortcuts take 1 channel only. Residual connections between encoder and decoder just keep channel \textbf{$C$}. $3\times3$ convolution is used in the input head module and the output head module to modify channels and resolution (resolution modified in the input head module only). For slim version, \textbf{$C$} is set to 32, \textbf{$C'$} is set to 16 and uses 4 layers per OSA block.

\subsubsection{\bf Training Details}
% \noindent
% \textbf{Training details.}
All experiments are carried out using the AdamW optimizer with an initial learning rate of $10^{-3}$. The learning rate is reduced by half after 5, 15, 25, and 35 epochs, and training continues for a total of 40 epochs with a batch size of 32. The threshold $\delta$ is kept constant at 0.1 during the training process. The weights $\alpha$ and $\beta$ for the loss function are set to 0.1 and 0.001, respectively. The images are resized to $320\times240$ for both training and testing. The experiments were conducted using an NVIDIA GeForce RTX 3090 GPU.
% We use AdamW optimizer with initial learning rate of $10^{-3}$ and multi-step learning rate scheduler which decays the learning rate by half after 5, 15, 25, 35 epochs. We train the model for 40 epochs with the batch size of 32. Threshold $\delta$ keeps 0.1 during training. Considering loss, we set $\alpha=0.1$, $\beta=0.001$. For all methods, we scale the images to $320\times240$ during training and testing. We use NVIDIA GeForce RTX 3090 for training and testing. 

 % Depth completion task and generalization ability are tested on ClearGrasp, TransCG and ClearPose. Pose estimation task is carried out on the set1 of ClearPose, since Clearpose has an accurate pose annotation without sticker. We use typical network DenseFusion\cite{wang2019densefusion} as pose estimation network. Following the learning strategy of DenseFusion, we train the network on 12G NVIDIA TITAN Xp GPU for 5 epochs with batch size of 128. The margin of refinement is set to 0.03. For fair comparison, we evaluate others works using their released source codes and optimal hyper-parameters or statistics reported in their paper.

\subsection{Ablation study} \label{sec:ablation}
We conduct an ablation study to investigate the effectiveness of our proposed components, including  new loss function, fusion branch, cross-layer shortcut and backbone structure. We take DFNet as baseline method since it is constructed following UNet structure. We  gradually replace its original components by our proposed ones and show the influence of using our proposed components. All the experiments of the ablation study are conducted on TransCG dataset.

% In view that DFNet is also constructed based on UNet, We here gradually replace its original components by our proposed. This study is conducted on TransCG dataset.
% To study the impact of each component in our proposed method, we perform experiments with different configurations of loss functions, network architecture, and backbones. Our method is compared against the recent transparent object depth completion work DFNet, which serves as our baseline. The ablation study experiments are all performed on the TransCG dataset.
% To verify the effectiveness of each component in our method, we evaluate the performance w.r.t. different configurations of loss functions, network architecture, and backbones. We use recently proposed transparent objects depth completion work DFNet as baseline. Ablation study is carried out on TransCG.




\subsubsection{\bf Loss Function}
The training of DFNet employs the mean squared error (MSE) and smooth loss as its loss function. However, these simple loss functions can lead to overfitting to local features, which makes the model more sensitive to the noise from low-level features such as edges and positions, negatively impacting its accuracy. To validate our proposed loss function, we first replaced the MSE loss with Huber loss in DFNet and termed it as Huber Loss. And then we replaced the loss function of DFNet with ours, leaving all other aspects unchanged and termed it as New Loss in Table \ref{tab:table1}. It can be observed by comparing New Loss with DFNet that all metrics showed improvement without requiring any additional parameters. 

% Qualitatively, the use of our proposed loss function can let the network to concentrate on the global structure rather than local details. By comparing the rows 3 and 4 of Figure \ref{fig:figure5}, the boundaries become smoother and even less distinct.
% The training of DFNET uses MSE and cosine distance. The simple loss function may lead to overfit to local features during training. This makes the model more sensitive to the noise of low-level features such as edge and position, which in turn affects its accuracy. So we propose a loss function consisting of Huber loss, SSIM loss and Smooth loss to suppress it. To verify its validity, we replaced the loss function of DFNet with ours and remain its other parts unchanged, then compared the results output by the mixed model (New Loss in Table \ref{tab:table1}) with the original one.
% All metrics are improved without extra parameters. Furthermore, we manually designed a feature to describe those pixels by computing the gradient of depth image and doing Gaussian blur to form an 'edge mask'. As their wights drop, the performance of the model is improved (Edge weight modified in Table \ref{tab:table2}), suggesting that it is necessary to treat pixels differently.
%and lower their weight during training. Specifically, we compute the gradient of depth image and do gaussian blur to form an 'edge mask'. Result (Edge weight modified in Table \ref{tab:table2}) supports our idea and shows it is necessary to treat pixels differently. 

\subsubsection{\bf Fusion Branch and Cross-layer Shortcuts}
In order to evaluate the impact of our proposed fusion branch and cross-layer shortcuts, we make changes to DFNet's architecture. First, we remove the redundant CDC blocks in DFNet from its skip connections, in line with our insight of preserving low-level features and the purpose of light weighting. Then, we added cross-layer shortcuts and a fusion branch to the modified network. It can be seen in Table \ref{tab:table1} that adopting this new architecture (referred to as Shortcut Fusion), almost all metrics show improvement with fewer parameters. 

\subsubsection{\bf Backbone}
We finally replace the denseblock in DFNet with our OSA module and utilized max pooling as the downsampling method. This final modification has transformed DFNet into our network. As shown in Table \ref{tab:table1}, our network outperforms the previous state-of-the-art (SOTA) by at least 16\% on difference-based metrics and improves ratio-based metrics by up to 4.42\%, resulting in a new SOTA performance. To make it practical for low-power robots, we created a slim version to balance speed and accuracy. 


% Qualitatively, figure \ref{fig:figure5} shows our method predicts clearer edges and is better handling crowded area.

% The fusion branch in our proposed network introduces a rich collection of low-level features, while the OSA module promotes feature reuse. Additionally, raw depth information is provided throughout the network, which enhances the representation of low-level features but may also hinder the learning of high-level semantic information. Our hypothesis is that the use of max pooling as a less aggressive downsampling method can mitigate these side effects while also reducing the number of parameters. The results in Table \ref{tab:table2} support our viewpoint.
% We fianlly relace the denseblock in DFNet by our used OSA module, and use max pooling as downsampling method. After this final modification, DFNet is tranformed to our proposed network. We thus show the performance by :Our"  in Table \ref{tab:table1}. It can be observed that ours outperforms previous SOTA by at least 16\% on difference-based metrics and improves ratio-based metrics by 0.1\% to 4.42\%, achieving the new state-of-the-art performance. In order to be capable in real applications, we also construct a slim version for speed/accuracy trade-off. 
% As we mentioned above, fusion branch introduces abundant low-level features and OSA encourages feature reuse. Furthermore, Raw depth is provided throughout the network. They enrich the representation of low-level features but may also harm to the learning of high-level semantic information. We suppose that using maxpooling to loosely downsampling may reduce their side effects as well as parameters saving. Result in Table \ref{tab:table2} proved our point of view.

% For summary, with our loss function, network tend to learn high-level features, with fusion branch, raw depth image and shortcuts, network can take advantage of low-level features. These components working together gives the network ability to take into account both local details and global structures. OSA module and max-pooling downsampling accelerate inference speed and reduce side effects.




% To intuitively show the impact of the proposed components, we visualize the predicted depth on TransCG and CleargGrasp dataset in Figure \ref{fig:figure5}. All networks are trained on TransCG dataset. Qualitatively, with our loss function, network is likely to focus on global structure rather than local detail. Red rectangle in row 3 and 4 show that with our loss function, boundaries become smoothy and even ambiguous, and outliers in the bottom right corner of the second column are suppressed. 



% FDCT performs domain adaption to the concatenation of raw depth and deep features and adopts maxpooling to lossly downsampling. It is supposed to reduce the disadvantage of the inaccuracy of raw depth. Our method predicts more accuracy and smooth edge as shown by the red circle on the left and the black square on the right. And even correct the ground truth as depicted in black circle on the right. The light spot reflected on the apple significantly affects the performance in row 2,3,5, but has little impact on row 4,6. Our methods successfully overcome the side effect of the raw depth information.

\subsection{Depth Completion Experiments} \label{sec:depth}

We compare our method with others on synthetic dataset ClearGrasp and real-world dataset TransCG. The quantitative results are respectively reported in Table \ref{tab:table2} and Table \ref{tab:table3}. Our proposed network surpasses others in almost every metric on these datasets which contain  synthetic and real-world scenes. Our method achieves a new state-of-the-art performance with a smaller model size and faster inference time, making it a highly competitive solution in this field.
%except on ClearGrasp synthetic validation set. It may be result of that the local implicit depth function which is environment-dependent, as well as the extra training data. 

% {\color{blue}
Specifically, our method outperforms the other methods by a larger margin in terms of REL and $\delta1.05$ metrics. This indicates its robustness to noise in the raw depth information, as these metrics are computed based on relative values and are sensitive to noise. Additionally, the gap between our method and others is larger in tests involving novel objects in ClearGrasp (CG Syn-novel in Table \ref{tab:table4} and the ClearGrasp column in Figure \ref{fig:figure5}), indicating that our method has a better ability to generalize to unseen objects. The qualitative results is reported in Figure \ref{fig:figure5}. The prediction of our method exhibits a clearer boundary and finer details than DFNet.
% }
% Specifically, our method has a bigger gap in REL and $\delta1.05$ to others most of the time. It demonstrates that our method is more stable to the noise in raw depth information of pixels, because these metrics are computed by relative value and significantly affected by noise. Noteworthy, the gap between our method and others getting bigger in the test of novel objects in most cases, indicates our method is able to generalize better to unseen objects.

\begin{table}[!t]
\caption{Depth Completion Result on TransCG dataset.}
\label{tab:table2}

\centering
\resizebox{\linewidth}{!}{%
\begin{tabular}{ccccccccc}
\toprule
Model & RMSE  & REL   & MAE   & $\delta1.05$ & $\delta1.10$ & $\delta1.25$ & Time ($\second$)   & Size ($\mega$B)    \\ \midrule
ClearGrasp\cite{sajjan2020clear}   & 0.054 & 0.083 & 0.037 & 50.48 & 68.68 & 95.28 & 2.281          & 934          \\
LIDF-Refine\cite{zhou2021pr}  & 0.019 & 0.034 & 0.015 & 78.22 & 94.26 & 99.80 & 0.018          & 251          \\
DFNet\cite{fang2022transcg}        & 0.018 & 0.027 & 0.012 & 83.76 & 95.67 & 99.71 & 0.024          & 4.8          \\
Ours (slim)   & 0.017 & 0.025 & 0.011 & 85.53 & 96.46 & 99.79 & \textbf{0.014} & \textbf{1.6} \\
Ours & \textbf{0.015} & \textbf{0.022} & \textbf{0.010} & \textbf{88.18} & \textbf{97.15} & \textbf{99.81} & 0.015 & 4.8 \\ \bottomrule
\end{tabular}}
% \vspace{-0.5cm}
\end{table}


\begin{table}[!t]
\renewcommand{\arraystretch}{0.9}
\caption{Depth Completion Results on ClearGrasp dataset\label{tab:table3}}
\centering
\resizebox{\linewidth}{!}{%
\begin{tabular}{ccccccc}
\toprule
\multicolumn{1}{c}{Model/Metric} &
  \multicolumn{1}{c}{RMSE} &
  \multicolumn{1}{c}{REL} &
  \multicolumn{1}{c}{MAE} &
  \multicolumn{1}{c}{$\delta$1.05} &
  \multicolumn{1}{c}{$\delta$1.10} &
  $\delta$1.25 \\ \midrule
\multicolumn{7}{c}{Train CG Test CG Syn-novel} \\ \midrule
\multicolumn{1}{c}{ClearGrasp} &
  \multicolumn{1}{c}{0.040} &
  \multicolumn{1}{c}{0.071} &
  \multicolumn{1}{c}{0.035} &
  \multicolumn{1}{c}{42.95} &
  \multicolumn{1}{c}{80.04} &
  98.10 \\ 
\multicolumn{1}{c}{Local Implicit} &
  \multicolumn{1}{c}{\underline{0.028}} &
  \multicolumn{1}{c}{\underline{0.045}} &
  \multicolumn{1}{c}{\underline{0.023}} &
  \multicolumn{1}{c}{\underline{68.62}} &
  \multicolumn{1}{c}{\underline{89.10}} &
  \underline{99.20} \\ 
\multicolumn{1}{c}{DFNet} &
  \multicolumn{1}{c}{0.032} &
  \multicolumn{1}{c}{0.051} &
  \multicolumn{1}{c}{0.027} &
  \multicolumn{1}{c}{62.59} &
  \multicolumn{1}{c}{84.37} &
  98.39 \\ 
\multicolumn{1}{c}{FDCT (Ours)} &
  \multicolumn{1}{c}{\textbf{0.025}} &
  \multicolumn{1}{c}{\textbf{0.040}} &
  \multicolumn{1}{c}{\textbf{0.021}} &
  \multicolumn{1}{c}{\textbf{71.66}} &
  \multicolumn{1}{c}{\textbf{92.95}} &
  \textbf{99.64} \\ \midrule
\multicolumn{7}{c}{Train CG Test CG Syn-known} \\ \midrule
\multicolumn{1}{c}{Local Implicit} &
  \multicolumn{1}{c}{\textbf{0.012}} &
  \multicolumn{1}{c}{\textbf{0.017}} &
  \multicolumn{1}{c}{\textbf{0.009}} &
  \multicolumn{1}{c}{\textbf{94.79}} &
  \multicolumn{1}{c}{\textbf{98.52}} &
  99.67 \\ 
\multicolumn{1}{c}{ClearGrasp} &
  \multicolumn{1}{c}{0.044} &
  \multicolumn{1}{c}{0.047} &
  \multicolumn{1}{c}{0.033} &
  \multicolumn{1}{c}{71.23} &
  \multicolumn{1}{c}{92.60} &
  98.24 \\ 
\multicolumn{1}{c}{DFNet} &
  \multicolumn{1}{c}{0.018} &
  \multicolumn{1}{c}{0.023} &
  \multicolumn{1}{c}{0.013} &
  \multicolumn{1}{c}{88.85} &
  \multicolumn{1}{c}{97.57} &
  \underline{99.92} \\ 
\multicolumn{1}{c}{FDCT (Ours)} &
  \multicolumn{1}{c}{\underline{0.015}} &
  \multicolumn{1}{c}{\underline{0.020}} &
  \multicolumn{1}{c}{\underline{0.012}} &
  \multicolumn{1}{c}{\underline{90.53}} &
  \multicolumn{1}{c}{\underline{98.21}} &
  \textbf{99.99} \\ \bottomrule

\end{tabular}%
% \tablen}
}
\end{table}



\subsection{Generalization Experiment} \label{sec:generalization}
% The generalization capability of a network is essential for practical applications. We evaluated the generalization ability of our proposed method from two perspectives: from synthetic images to real-world images and from one real-world dataset to another. The results of our experiments, shown in Table \ref{tab:table6}, indicate that our method (FDCT) has a comparable generalization capability to the state-of-the-art methods in cross-dataset evaluations, and it outperforms similar works in the synthetic-to-real test. However, it lags behind methods that focus solely on sim-to-real (noted as "local implicit*").
% The generalization ability of a network is critical for real-world application. The proposed method has a generalization ability that can be trained on synthetic data and aply to real world scene (syn-to-real) or trained on one real world dataset TransCG and adap to ClearGrasp (real-to-real). Comparison result is reported in Table \ref{tab:table4}. It shows that although there is still a certain gap compared with the method Local Implicit designed for syn-to-real; compared with the similar method DFNet, our method achieves a better result in the syn-to-real setting, and a competitive result in the syn-to-syn setting.
The generalization ability of a network is critical for real-world application. Our proposed method exhibits a high degree of generalization, being able to be trained on synthetic data and applied to real-world scenes (syn-to-real), or trained on one real-world dataset TransCG and adapted to the other real-world dataset (real-to-real), such as ClearGrasp. Comparison results are reported in Table \ref{tab:table4}, which show that while there is still a certain gap compared to the syn-to-real method (Local Implicit \cite{zhu2021rgb}), our method achieves better results in the syn-to-real setting when compared to the similar method DFNet, and competitive results in the real-to-real setting.

% We inspect the generalization ability of our proposed method from two aspects, from synthetic image to real-world image and from one real-world dataset to another. Experiment results in Table \ref{tab:table5} show that FDCT has a similar generalization ability to previous SOTA in cross-dataset and get better result in synthetic-to-real test compared to similar work, but is far below to methods focusing on sim-to-real.

% Since both datasets comprise real-world image, we train models on TransCG and test it on ClearGrasp real-world set for cross-dataset test. DFNet outperformed other method with a huge gap in generalization test and is chosen to be compared with ours. Comparison result is reported in Table \ref{tab:table5}. Our method outperforms the closest work in all metrics both for known and novel objects in synthetic-to-real test. There is a bigger gap between DFNet and ours in terms of novel objects. It might owe to a better utilization of RGB cues. Our method gets similar results to DFNet in cross dataset test, showing that our method has the ability to generalize from real-world dataset to another. With a series of real-world transparent objects datasets being proposed, we believe that the generalization ability in real-world is more important than sim-to-real.



% {\color{blue}
% Figure environment removed

\begin{table}[!t]
\caption{
% Result of Synthetic to Real and Cross Dataset Generalization Experiment
Generalization test on syn-to-real and real-to-real.}
\label{tab:table4}
\renewcommand{\arraystretch}{0.95}
\centering
\resizebox{\linewidth}{!}{%
% \begin{threeparttable}
\begin{tabular}{ccclclclclcl}
\toprule
\multicolumn{1}{c}{Model/Metric} &
  \multicolumn{1}{c}{RMSE} &
  \multicolumn{2}{c}{REL} &
  \multicolumn{2}{c}{MAE} &
  \multicolumn{2}{c}{$\delta$1.05} &
  \multicolumn{2}{c}{$\delta$1.10} &
  \multicolumn{2}{c}{$\delta$1.25} \\ \midrule
\multicolumn{12}{c}{Train CG Test CG Real-known (syn-to-real)} \\ \midrule
\multicolumn{1}{c}{Local Implicit\cite{zhu2021rgb}} &
  \multicolumn{1}{c}{\textbf{0.028}} &
  \multicolumn{2}{c}{\textbf{0.033}} &
  \multicolumn{2}{c}{\textbf{0.020}} &
  \multicolumn{2}{c}{\textbf{82.37}} &
  \multicolumn{2}{c}{\textbf{92.98}} &
  \multicolumn{2}{c}{\textbf{98.63}} \\ 
\multicolumn{1}{c}{DFNet} &
  \multicolumn{1}{c}{0.068} &
  \multicolumn{2}{c}{0.107} &
  \multicolumn{2}{c}{0.059} &
  \multicolumn{2}{c}{32.42} &
  \multicolumn{2}{c}{56.88} &
  \multicolumn{2}{c}{91.47} \\ 
\multicolumn{1}{c}{FDCT (Ours)} &
  \multicolumn{1}{c}{\underline{0.065}} &
  \multicolumn{2}{c}{\underline{0.103}} &
  \multicolumn{2}{c}{\underline{0.057}} &
  \multicolumn{2}{c}{\underline{33.08}} &
  \multicolumn{2}{c}{\underline{59.81}} &
  \multicolumn{2}{c}{\underline{91.70}} \\ \midrule
\multicolumn{12}{c}{Train CG Test CG Real-novel (syn-to-real)} \\ \midrule
\multicolumn{1}{c}{Local Implicit\cite{zhu2021rgb}} &
  \multicolumn{1}{c}{\textbf{0.025}} &
  \multicolumn{2}{c}{\textbf{0.036}} &
  \multicolumn{2}{c}{\textbf{0.020}} &
  \multicolumn{2}{c}{\textbf{76.21}} &
  \multicolumn{2}{c}{\textbf{94.01}} &
  \multicolumn{2}{c}{\textbf{99.35}} \\ 
\multicolumn{1}{c}{DFNet} &
  \multicolumn{1}{c}{0.051} &
  \multicolumn{2}{c}{0.088} &
  \multicolumn{2}{c}{0.046} &
  \multicolumn{2}{c}{31.23} &
  \multicolumn{2}{c}{64.66} &
  \multicolumn{2}{c}{97.77} \\ 
\multicolumn{1}{c}{FDCT (Ours)} &
  \multicolumn{1}{c}{\underline{0.043}} &
  \multicolumn{2}{c}{\underline{0.073}} &
  \multicolumn{2}{c}{\underline{0.038}} &
  \multicolumn{2}{c}{\underline{39.42}} &
  \multicolumn{2}{c}{\underline{75.54}} &
  \multicolumn{2}{c}{\underline{99.09}} \\ \midrule
\multicolumn{12}{c}{Train TCG Test CG Real-novel (real-to-real)} \\ \midrule
\multicolumn{1}{c}{Local Implicit\cite{zhu2021rgb}} &
  \multicolumn{1}{c}{0.152} &
  \multicolumn{2}{c}{0.225} &
  \multicolumn{2}{c}{0.139} &
  \multicolumn{2}{c}{9.86} &
  \multicolumn{2}{c}{20.63} &
  \multicolumn{2}{c}{46.02} \\ 
\multicolumn{1}{c}{DFNet} &
  \multicolumn{1}{c}{\textbf{0.041}} &
  \multicolumn{2}{c}{\textbf{0.054}} &
  \multicolumn{2}{c}{\textbf{0.031}} &
  \multicolumn{2}{c}{\textbf{62.74}} &
  \multicolumn{2}{c}{\textbf{83.31}} &
  \multicolumn{2}{c}{\textbf{97.33}} \\ 
\multicolumn{1}{c}{FDCT (Ours)} &
  \multicolumn{1}{c}{\textbf{0.041}} &
  \multicolumn{2}{c}{\underline{0.055}} &
  \multicolumn{2}{c}{\underline{0.032}} &
  \multicolumn{2}{c}{\underline{61.23}} &
  \multicolumn{2}{c}{\underline{82.84}} &
  \multicolumn{2}{c}{\underline{97.28}} \\ \bottomrule
\end{tabular}
%     \begin{tablenote}
%         \footnotesize
%         \item [*]Local Implicit is method aiming at sim-to-real.
%     \end{tablenote}
% \end{threeparttable}
}
%\vspace{-0.5cm}
\end{table}
% Figure environment removed
\subsection{Analysis} \label{sec:analysis}
In our proposed method, the loss function plays a crucial role in enabling the network to focus on structural information and alleviate the effects of unstable pixels. However, this focus on structural information may come at the expense of some details. On the other hand, the fusion branch and shortcuts draw attention to the details, which can introduce extra redundancy. Nonetheless, the use of maxpooling facilitates lossy and aggressive downsampling, which can reduce redundancy and improve robustness. The convolution based fusion method make better use of the raw depth image. All components work together and complement each other to achieve the best possible balance between structural information and details. In this section, we analyze the four critical components of our method and demonstrate their effectiveness.

\subsubsection{Influence of loss term}
% As we mentioned above, some unstable pixels can unwantedly make big penalty to the loss. By computing the gradient of the depth image and applying Gaussian blur, we manually created a feature to represent these pixels. As the weights of these pixels were reduced, the model's performance improved (as seen in Experiment of weight in Table \ref{tab:table5}), indicating the importance of treating pixels differently and pointing out the necessity of the so designed loss function. However, the side effect of such loss function is that the network pays too much attention to the structure and ignores some details. The highlighted area of the feature map changes from dotted to regional in the Loss column in Figure \ref{fig:figure6}.
As mentioned in \ref{section:Loss}, unstable pixels can have a significant negative influence on the calculation of the training loss. To illustrate this issue, we manually created a feature to represent these pixels by computing the gradient of the depth image and applying a Gaussian blur. By reducing the weights of these pixels, we observed an improvement in the model's performance (as seen in the Experiment of weight in Table \ref{tab:table5}), highlighting the importance of treating pixels differently and emphasizing the necessity of the used loss functions (especially the Huber Loss). Qualitatively, as shown in Figure \ref{fig:figure6}, the New Loss model places greater emphasis on the overall structure of transparent objects, as compared to DFNet, which primarily focuses on local information. The downside of such a loss function is that the network may ignore some details.
% Figure environment removed

\subsubsection{Low-level feature preservation}
% Fusion branch and cross-layer shortcuts alleviate the indistinct boundaries and perceptual details by taking more low-level cues into consideration. The highlighted area of the feature map changes from regional to scattered in the Fusion column in Figure \ref{fig:figure6}. Loss function and low-level feature awareness components together make a good trade-off between detail and structure information.
The fusion branch and cross-layer shortcuts help alleviate the issue of blurry boundaries and low perceptual details by incorporating more low-level cues. As a result, more low-level features such as object edges and holes are preserved in the feature map of Fusion model in Figure \ref{fig:figure6}. The combination of the loss function and low-level feature awareness components strikes a good balance between detail and structural information.

\subsubsection{Influence of downsampling}
Our hypothesis is that the use of max pooling as a lossy downsampling method can mitigate the side effects of the low-level awareness components while reducing the number of parameters. The results in Table \ref{tab:table5} that are noted as ``Experiment of downsampling'' support our viewpoint. It can be observed that the performance of using convolutional downsampling and average pooling is slightly worse than that of using max pooling.

% The loss function makes the network focus on structural information and alleviating the affects of unstable pixels, but may harming to the details. The fusion branch and shortcuts draws the attention to details, but may introduce extra redundancy. Maxpooling is used to lossy and aggressively downsampling. It can reduce redundancy and improve robustness. These components work together and complement each other.
% }

\subsubsection{Fusion method of depth image}
To demonstrate that fusing the raw depth image with feature map via convolution is better than directly concatenation. We removed the convolution layers used for fusion in the model Ours and named it Ours(concat). The result labeled Table ``Experiment on fusion method'' in Table \ref{tab:table5} support our viewpoint.

\begin{table}[!ht]
\centering
\caption{Experiment Result on Weight Modification, Downsampling Implementation and Fusion Method\label{tab:table5}}

\resizebox{\linewidth}{!}{%
\begin{tabular}{ccccccc}
\toprule
\multicolumn{1}{c}{Model/Metric} &
  \multicolumn{1}{c}{RMSE} &
  \multicolumn{1}{c}{REL} &
  \multicolumn{1}{c}{MAE} &
  \multicolumn{1}{c}{$\delta$1.05} &
  \multicolumn{1}{c}{$\delta$1.10} &
  $\delta$1.25 \\ \midrule
\multicolumn{7}{c}{Experiment on weight} \\ \midrule
\multicolumn{1}{c}{Baseline} &
  \multicolumn{1}{c}{0.018} &
  \multicolumn{1}{c}{0.027} &
  \multicolumn{1}{c}{0.012} &
  \multicolumn{1}{c}{83.76} &
  \multicolumn{1}{c}{95.67} &
  99.71 \\ 
\multicolumn{1}{c}{Edge Weight Modified} &
  \multicolumn{1}{c}{\textbf{0.017}} &
  \multicolumn{1}{c}{\textbf{0.025}} &
  \multicolumn{1}{c}{\textbf{0.011}} &
  \multicolumn{1}{c}{\textbf{85.34}} &
  \multicolumn{1}{c}{\textbf{96.26}} &
  \textbf{99.75} \\ \midrule
\multicolumn{7}{c}{Experiment on downsampling} \\ \midrule
\multicolumn{1}{c}{Conv Down} &
  \multicolumn{1}{c}{0.016} &
  \multicolumn{1}{c}{0.023} &
  \multicolumn{1}{c}{0.011} &
  \multicolumn{1}{c}{87.16} &
  \multicolumn{1}{c}{96.83} &
  99.80 \\ 
\multicolumn{1}{c}{AvgPooling Down} &
  \multicolumn{1}{c}{0.016} &
  \multicolumn{1}{c}{0.024} &
  \multicolumn{1}{c}{0.011} &
  \multicolumn{1}{c}{87.16} &
  \multicolumn{1}{c}{96.93} &
  99.80 \\ 
\multicolumn{1}{c}{MaxPooling Down} &
  \multicolumn{1}{c}{\textbf{0.015}} &
  \multicolumn{1}{c}{\textbf{0.022}} &
  \multicolumn{1}{c}{\textbf{0.010}} &
  \multicolumn{1}{c}{\textbf{88.18}} &
  \multicolumn{1}{c}{\textbf{97.15}} &
  \textbf{99.81} \\ \midrule
  \multicolumn{7}{c}{Experiment on fusion method} \\ \midrule
  \multicolumn{1}{c}{Ours(concat)} &
  \multicolumn{1}{c}{\textbf{0.015}} &
  \multicolumn{1}{c}{0.023} &
  \multicolumn{1}{c}{0.011} &
  \multicolumn{1}{c}{87.90} &
  \multicolumn{1}{c}{96.68} &
  99.80 \\ 
\multicolumn{1}{c}{Ours} &
  \multicolumn{1}{c}{\textbf{0.015}} &
  \multicolumn{1}{c}{\textbf{0.022}} &
  \multicolumn{1}{c}{\textbf{0.010}} &
  \multicolumn{1}{c}{\textbf{88.18}} &
  \multicolumn{1}{c}{\textbf{97.15}} &
  \textbf{99.81} \\ 
\bottomrule
\end{tabular}%
}
%\vspace{-0.5cm}
\end{table}
\vspace{-0.2cm}


\subsection{Pose Estimation Experiment}
In this experiment, we aim to demonstrate the applicability of our network for downstream tasks and to show that it can improve the accuracy of pose estimate.
To evaluate the performance of pose estimation, we use three evaluation metrics, i.e, the average closest point distance (ADD-S), the area under the ADD-S curve (AUC), and the percentage of ADD-S values that are smaller than 2 \centi\meter.
%\cite{xiang2017posecnn}
% The higher the metrics the stronger the performance.

% This experiment is carried out on the set1 of ClearPose, since Clearpose has an accurate pose annotation without sticker. We use typical network DenseFusion \cite{wang2019densefusion} as pose estimation network. Following the learning strategy of DenseFusion, we train the network on 12G NVIDIA TITAN Xp GPU for 5 epochs with batch size of 128. The margin of refinement is set to 0.03. For fair comparison, we evaluate others works using their released source codes and optimal hyper-parameters or statistics reported in their paper.
Both our method and DFNet are trained on the ClearPose Set 1 and are used to predict the depth of Set 1-Scene 5 for pose estimation purposes. The depth completion result is reported in Table \ref{tab:table6} and a screenshot of the live demonstration is reported in Figure \ref{fig:figure7}. In our experiments, we use DenseFusion \cite{wang2019densefusion}  as the pose estimation method. We trained DenseFusion with the restored depth and tested it on 3,000 randomly selected images. Ideally, a more accurate depth prediction can lead to improved performance in pose estimation. The results of our evaluations, presented in Table \ref{tab:table7}, indicate that the depth restored by our method outperforms DFNet in almost every object in the pose estimation task. This results validate that the depth map given by our method is more appropriate for addressing the downstream task, i.e., pose estimation.
% Depth completion models are trained on ClearPose set 1 and predict the depth of set 1-scene 5 for pose estimation. We train DenseFusion with the restored depth and test on 3k randomly chosen images. Metrics for each object are reported in Table \ref{tab:table7}. Result shows that the depth restored by FDCT outperforms DFNet's in almost every object in pose estimation task.
% \todo{format of tablehead!!}
\begin{table}[!t]
\caption{Depth Completion Results on ClearPose dataset.}
\label{tab:table6}
\centering
\begin{tabular}{ccccccc}
\toprule
Model & RMSE           & REL            & MAE            & $\delta$1.05          & $\delta$1.10          & $\delta$1.25          \\ \midrule
DFNet        & 0.048          & 0.038          & 0.033          & 76.36          & 94.22          & \textbf{99.40} \\
Ours         & \textbf{0.045} & \textbf{0.033} & \textbf{0.028} & \textbf{82.15} & \textbf{94.43} & 99.25          \\
\bottomrule
\end{tabular}%
\end{table}



\begin{table}[!t]
\caption{Pose Estimation Results on ClearPose dataset\label{tab:table7}}
\centering
\resizebox{\linewidth}{!}{%
\begin{tabular}{ccccccc}
\toprule
Models &
  \multicolumn{3}{c}{DFNet} &
  \multicolumn{3}{c}{Ours} \\ \midrule
Object/Metirc &
  \multicolumn{1}{c}{AUC} &
  \multicolumn{1}{c}{\textless{}2cm} &
  ADD-S(10\%) &
  \multicolumn{1}{c}{AUC} &
  \multicolumn{1}{c}{\textless{}2cm} &
  ADD-S(10\%) \\ 
beaker\_1 &
  \multicolumn{1}{c}{79.07} &
  \multicolumn{1}{c}{\textbf{0.00}} &
  0.68 &
  \multicolumn{1}{c}{\textbf{80.44}} &
  \multicolumn{1}{c}{\textbf{0.00}} &
  \textbf{7.53} \\ 
dropper\_1 &
  \multicolumn{1}{c}{\textbf{67.76}} &
  \multicolumn{1}{c}{61.00} &
  \textbf{48.00} &
  \multicolumn{1}{c}{31.70} &
  \multicolumn{1}{c}{\textbf{65.33}} &
  0.00 \\ 
dropper\_2 &
  \multicolumn{1}{c}{81.09} &
  \multicolumn{1}{c}{\textbf{33.10}} &
  1.78 &
  \multicolumn{1}{c}{\textbf{84.24}} &
  \multicolumn{1}{c}{0.00} &
  \textbf{9.61} \\ 
flask\_1 &
  \multicolumn{1}{c}{84.96} &
  \multicolumn{1}{c}{60.33} &
  42.33 &
  \multicolumn{1}{c}{\textbf{86.71}} &
  \multicolumn{1}{c}{\textbf{68.33}} &
  \textbf{68.00} \\ 
funnel\_1 &
  \multicolumn{1}{c}{78.85} &
  \multicolumn{1}{c}{91.33} &
  0.00 &
  \multicolumn{1}{c}{\textbf{82.91}} &
  \multicolumn{1}{c}{\textbf{98.33}} &
  \textbf{12.33} \\ 
cylinder\_1 &
  \multicolumn{1}{c}{78.77} &
  \multicolumn{1}{c}{48.33} &
  28.67 &
  \multicolumn{1}{c}{\textbf{79.83}} &
  \multicolumn{1}{c}{\textbf{77.00}} &
  \textbf{33.33} \\ 
cylinder\_2 &
  \multicolumn{1}{c}{62.75} &
  \multicolumn{1}{c}{54.67} &
  3.33 &
  \multicolumn{1}{c}{\textbf{75.68}} &
  \multicolumn{1}{c}{\textbf{58.67}} &
  \textbf{29.33} \\ 
pan\_1 &
  \multicolumn{1}{c}{86.76} &
  \multicolumn{1}{c}{13.67} &
  33.33 &
  \multicolumn{1}{c}{\textbf{89.37}} &
  \multicolumn{1}{c}{\textbf{53.67}} &
  \textbf{50.00} \\ 
pan\_2 &
  \multicolumn{1}{c}{88.71} &
  \multicolumn{1}{c}{84.67} &
  44.00 &
  \multicolumn{1}{c}{\textbf{89.73}} &
  \multicolumn{1}{c}{\textbf{90.33}} &
  \textbf{56.00} \\ 
pan\_3 &
  \multicolumn{1}{c}{\textbf{88.90}} &
  \multicolumn{1}{c}{87.67} &
  \textbf{53.33} &
  \multicolumn{1}{c}{88.10} &
  \multicolumn{1}{c}{\textbf{91.00}} &
  48.00 \\ 
bottle\_1 &
  \multicolumn{1}{c}{86.05} &
  \multicolumn{1}{c}{91.53} &
  24.41 &
  \multicolumn{1}{c}{\textbf{88.71}} &
  \multicolumn{1}{c}{\textbf{93.22}} &
  \textbf{31.53} \\ 
bottle\_2 &
  \multicolumn{1}{c}{71.81} &
  \multicolumn{1}{c}{83.16} &
  4.04 &
  \multicolumn{1}{c}{\textbf{77.01}} &
  \multicolumn{1}{c}{\textbf{88.22}} &
  \textbf{13.47} \\ 
stick\_1 &
  \multicolumn{1}{c}{69.53} &
  \multicolumn{1}{c}{32.32} &
  32.66 &
  \multicolumn{1}{c}{\textbf{79.60}} &
  \multicolumn{1}{c}{\textbf{57.58}} &
  \textbf{58.92} \\ 
syringe\_1 &
  \multicolumn{1}{c}{73.03} &
  \multicolumn{1}{c}{31.67} &
  25.67 &
  \multicolumn{1}{c}{\textbf{80.15}} &
  \multicolumn{1}{c}{\textbf{57.00}} &
  \textbf{47.00} \\ 
MEAN &
  \multicolumn{1}{c}{78.43} &
  \multicolumn{1}{c}{55.25} &
  24.45 &
  \multicolumn{1}{c}{\textbf{79.58}} &
  \multicolumn{1}{c}{\textbf{64.19}} &
  \textbf{33.22} \\


  \bottomrule
  \end{tabular}%
}
\vspace{-0.5cm}
\end{table}

\section{Conclusion}

In this paper, we establish a new convergence analysis for RSGDA under milder assumptions beyond strong concavity.
In particular, we show that RSGDA has the same convergence rate with SGDA in the NC-P{\L} condition, which is a better complexity compared with the previous analysis of RSGDA.
Moreover, we propose an intuitive method to choose the parameter $p$ for RSGDA and confirm its efficiency in experiments. 
There are quite a few works worthy to be considered in the future. 
On the one hand, it is still an open question whether there exists a tighter complexity bound for the minimax problems in the NC-P{\L} condition or even more general cases. 
On the other hand, a more efficient adaptive RSGDA for minimax optimization needs further investigation. 
% First, how to design adaptive algorithms for the minimax problems?
% Moreover, it is still an open question that what is the lower complexity bound for the minimax problems in the NC-P{\L} condition.
% Furthermore, does RSGDA still hold for the general minimax problems?

\bibliography{src/reference}

\onecolumn
\begin{appendices}
\addcontentsline{toc}{chapter}{Appendix}

% this is set for multiple appendices. Comment out and use the commented line below if only one Appendix.
\chapter*{Appendices\markboth{Appendices}{}}
%\chapter*{Appendix \markboth{Appendix \thesection}{}}



\renewcommand{\thesection}{\Alph{section}}

%\renewcommand{\thesubsection}{\Alph{section}.\arabic{subsection}}

\renewcommand{\theequation}{\Alph{section}.\arabic{equation}}

\renewcommand{\thefigure}{\Alph{section}.\arabic{figure}}



\appendix



%-------------------------------------------------------------------------------

\section{Title of Appendix 1}\label{appendix1}

Insert text here

\clearpage
\newpage
\section{Title of Appendix 2}\label{appendix2}

Insert text here


\end{appendices}

\end{document}
