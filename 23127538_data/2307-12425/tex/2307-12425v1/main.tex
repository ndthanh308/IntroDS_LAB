%%%%%%%% ICML 2023 EXAMPLE LATEX SUBMISSION FILE %%%%%%%%%%%%%%%%%

\documentclass[nohyperref]{article}

% Recommended, but optional, packages for figures and better typesetting:
\usepackage{microtype}
\usepackage{graphicx}
\usepackage{subfigure}
\usepackage{booktabs} % for professional tables

% hyperref makes hyperlinks in the resulting PDF.
% If your build breaks (sometimes temporarily if a hyperlink spans a page)
% please comment out the following usepackage line and replace
% \usepackage{icml2023} with \usepackage[nohyperref]{icml2023} above.
\usepackage[hidelinks]{hyperref}

% Attempt to make hyperref and algorithmic work together better:
\newcommand{\theHalgorithm}{\arabic{algorithm}}

% Use the following line for the initial blind version submitted for review:
% \usepackage{styles/icml2023}

% If accepted, instead use the following line for the camera-ready submission:
\usepackage[accepted]{icml2023}

% For theorems and such
\usepackage{amsmath}
\usepackage{amssymb}
\usepackage{mathtools}
\usepackage{amsthm}
\usepackage{bbm}

% additional packages
\usepackage{color,soul}
\usepackage{colortbl}
% \usepackage[table]{xcolor}% http://ctan.org/pkg/xcolor This is included in ICML style already
\usepackage{balance}
\usepackage{multirow}
\usepackage{multicol}
\newcommand{\specialcell}[2][c]{%
  \begin{tabular}[#1]{@{}c@{}}#2\end{tabular}}

\usepackage{enumitem}

% if you use cleveref..
\usepackage[capitalize,noabbrev]{cleveref}

%%%%%%%%%%%%%%%%%%%%%%%%%%%%%%%%
% THEOREMS
%%%%%%%%%%%%%%%%%%%%%%%%%%%%%%%%
\theoremstyle{plain}
\newtheorem{theorem}{Theorem}[section]
\newtheorem{proposition}[theorem]{Proposition}
\newtheorem{lemma}[theorem]{Lemma}
\newtheorem{corollary}[theorem]{Corollary}
\theoremstyle{definition}
\newtheorem{definition}[theorem]{Definition}
\newtheorem{assumption}[theorem]{Assumption}
\theoremstyle{remark}
\newtheorem{remark}[theorem]{Remark}

% \newtheoremstyle{indented}{3pt}{3pt}{\addtolength{\leftskip}{2.5em}}{}{\bfseries}{.}{.5em}{}
% \theoremstyle{indented}
% \theoremstyle{definition}

% \newtheorem{problem}{Problem}
% \newtheorem{observation}{O}
% \newtheorem{insight}{Insight}
% \newtheorem{hypothesis}{Hypothesis}

% Todonotes is useful during development; simply uncomment the next line
%    and comment out the line below the next line to turn off comments
%\usepackage[disable,textsize=tiny]{todonotes}
\usepackage[textsize=tiny]{todonotes}

%%%%%%%%%%%%%%%%%%%%%%%%%%%%%%%%
% Custom commands
%%%%%%%%%%%%%%%%%%%%%%%%%%%%%%%%
\graphicspath{{figures/}}
\newcommand{\etal}{\mbox{{et al.\ }}}
\newcommand{\figGap}[0]{\vspace{-1.1\baselineskip}}

\definecolor{lightgreen}{rgb}{0.8,1,0.8}
\definecolor{lightyellow}{rgb}{1,0.95,0.8}
\newcommand{\tbcolorg}{\cellcolor{lightgreen}}
\newcommand{\tbcolory}{\cellcolor{lightyellow}}

\newcommand{\argmin}{\operatorname*{argmin}}
\newcommand{\argmax}{\operatorname*{argmax}}
\newcommand{\argminx}{\underset{{x}}{\operatorname{argmin}}}

\newcommand{\softmin}{\operatorname*{softmin}}
\newcommand{\softmax}{\operatorname*{softmax}}
\newcommand{\norm}[1]{\left\lVert#1\right\rVert}
\newcommand{\kqw}[1]{{\color{blue}{#1}}}

\newcommand{\newcite}[1]{\citeauthor{#1} \citeyearpar{#1}}

\newcommand{\tbd}{\textcolor{red}{TBD}}

% The \icmltitle you define below is probably too long as a header.
% Therefore, a short form for the running title is supplied here:
\icmltitlerunning{On the Effectiveness of Offline RL for Dialogue Response Generation}

\begin{document}

\twocolumn[
\icmltitle{On the Effectiveness of Offline RL for Dialogue Response Generation}

% It is OKAY to include author information, even for blind
% submissions: the style file will automatically remove it for you
% unless you've provided the [accepted] option to the icml2023
% package.

% List of affiliations: The first argument should be a (short)
% identifier you will use later to specify author affiliations
% Academic affiliations should list Department, University, City, Region, Country
% Industry affiliations should list Company, City, Region, Country

% You can specify symbols, otherwise they are numbered in order.
% Ideally, you should not use this facility. Affiliations will be numbered
% in order of appearance and this is the preferred way.
% \icmlsetsymbol{equal}{*}

\begin{icmlauthorlist}
\icmlauthor{Paloma Sodhi}{comp}
\icmlauthor{Felix Wu}{comp}
\icmlauthor{Ethan R. Elenberg}{comp}
\icmlauthor{Kilian Q. Weinberger}{comp,sch}
\icmlauthor{Ryan McDonald}{comp}
% \icmlauthor{Firstname6 Lastname6}{sch,yyy,comp}
% \icmlauthor{Firstname7 Lastname7}{comp}
%\icmlauthor{}{sch}
% \icmlauthor{Firstname8 Lastname8}{sch}
% \icmlauthor{Firstname8 Lastname8}{yyy,comp}
%\icmlauthor{}{sch}
%\icmlauthor{}{sch}
\end{icmlauthorlist}

\icmlaffiliation{comp}{ASAPP, New York, United States}
\icmlaffiliation{sch}{Cornell University, New York, United States}
% \icmlaffiliation{sch}{School of ZZZ, Institute of WWW, Location, Country}

\icmlcorrespondingauthor{Paloma Sodhi}{psodhi@asapp.com}

% You may provide any keywords that you
% find helpful for describing your paper; these are used to populate
% the "keywords" metadata in the PDF but will not be shown in the document
\icmlkeywords{text generation, reinforcement learning, dialogue systems}

\vskip 0.3in
]

% this must go after the closing bracket ] following \twocolumn[ ...

% This command actually creates the footnote in the first column
% listing the affiliations and the copyright notice.
% The command takes one argument, which is text to display at the start of the footnote.
% The \icmlEqualContribution command is standard text for equal contribution.
% Remove it (just {}) if you do not need this facility.

\printAffiliationsAndNotice{Code available at {\url{https://github.com/asappresearch/dialogue-offline-rl}} \\ \\}  % leave blank if no need to mention equal contribution
% \printAffiliationsAndNotice{\icmlEqualContribution} % otherwise use the standard text.

\begin{abstract}

The Fast Reciprocal Square Root Algorithm is a well-established approximation technique consisting of two stages: first, a coarse approximation is obtained by manipulating the bit pattern of the floating point argument using integer instructions, and second, the coarse result is refined through one or more steps, traditionally using Newtonian iteration but alternatively using improved expressions with carefully chosen numerical constants found by other authors. The algorithm was widely used before microprocessors carried built-in hardware support for computing reciprocal square roots. At the time of writing, however, there is in general no hardware acceleration for computing other fixed fractional powers. This paper generalises the algorithm to cater to all rational powers, and to support any polynomial degree(s) in the refinement step(s), and under the assumption of unlimited floating point precision provides a procedure which automatically constructs provably optimal constants in all of these cases. It is also shown that, under certain assumptions, the use of monic refinement polynomials yields results which are much better placed with respect to the cost/accuracy tradeoff than those obtained using general polynomials. Further extensions are also analysed, and several new best approximations are given.

\end{abstract}


\section{Introduction}
Current quantum hardware is unable to carry out universal quantum computations due to the buildup of errors that occur during the computation. 
The magnitude of the individual error is currently above the value that the Threshold Theorem requires in order to kick-start quantum error correction and fault-tolerant quantum computation~\cite[Section 10.6]{nielsen_chuang_2010}. 
Although the experimentally achieved fidelity rates are promising and the error bounds are inching closer to the required threshold, we will have to work for the foreseeable future with quantum hardware with errors that build-up during the computation.  This implies that we can only do a limited number of steps before the output of the computation has become completely uncorrelated with the intended one.

For fault-tolerant quantum computing, we repeat four steps: 
1) We apply a number of single and two-qubit quantum gates, in parallel whenever possible; 
2) We perform a syndrome measurement on a subset of the qubits; 
3) We perform fast classical computations to determine which errors have occurred and how to correct them; 
and, 4) We apply correction terms based on the classical computations.
We then repeat these four steps with a next sequence of gates. 
These four steps are essential to fault-tolerant quantum computing. 


The starting point of this work is to use the four steps outlined above, not to carry out error correction and fault-tolerant computation, but to enhance short, constant-depth, {\em uncorrected} quantum circuits that perform single qubit gates and {\em nearest-neighbor} two qubit gates. 
Since in the long run we will have to implement error-correction and fault-tolerant computation anyhow, and this is done by such a four-step process, why not make other use of this architecture? Moreover, on some of the quantum hardware platforms, these operations are already in place.
Embracing this idea we naturally arrive at the question: what is the computational power of \textit{low-depth} quantum-classical circuits organized as in the four steps outlined above? 
We thus investigate circuits that execute a small, ideally constant, number of stages, where at each stage we may apply, in parallel, single qubit gates and {\em nearest-neighbor} two qubit gates, followed by measurements, followed by low-depth classical computations of which the outcome can control quantum gates in later stages. 
It is not clear, at first, whether such circuits, especially with constant depth, can do anything remotely useful. 
But we will see that this is indeed the case: many quantum computations can be done by such circuits in constant depth. 
By parallelizing quantum computations in this way, we improve the overall computational capabilities of these circuits, as we do not incur errors on qubits that are idle, simply because qubits are not idle for a very long time. 
Furthermore, reducing the depth of quantum circuits, at the cost of increasing width, allows the circuit to be run faster even if errors occur.

The first usage of such a four-step layout, not to do error correction, but to perform computations, can be found in the paradigm of measurement-based quantum computing~\cite{gottesman1999demonstrating,raussendorf2001one,jozsa2006introduction,clark2007generalised}: 
A universal form of quantum computing where a quantum state is prepared and operations are performed by measuring qubits in different bases, depending on previous measurements and intermediate measurements.

\citeauthor{PhamSvore2013} were the first to formalize the four-step protocol for performing computations~\cite{PhamSvore2013}. They included specific hardware topologies by considering two-dimensional graphs for imposing constraints on qubit interactions. In their model, they develop circuits for particularly useful multi-qubit gates, including specifying costs in the width, number of qubits, depth, number of concurrent time steps, size, and total number of non-Identity operations.
As a result, they find an algorithm that factors integers in polylogarithmic depth.
\citeauthor{Browne:2011} showed that the main tool in the work by \citeauthor{PhamSvore2013}, the fan-out gate, can also be replaced by additional log-depth classical computations in the measurement-based quantum computing setting~\cite{Browne:2011}.

More recently, \citeauthor{Cirac:2021} introduced a scheme to implement unitary operations involving quantum circuits combined with Local Operations and Classical Communication ($\mathsf{LOCC}$) channels: $\mathsf{LOCC}$-assisted quantum circuits~\cite{Cirac:2021}. Similarly to the four-step scheme we just described, they allow for a short depth circuit to be run on the qubits, followed by one round of $\mathsf{LOCC}$, in which ancilla qubits are measured and local unitaries are applied based on the measurement outcomes. They show that in this model any 1D transitionally invariant matrix-product state (MPS) with fixed bond dimension is in the same phase of matter as the trivial state. Similar ideas can be found in~\cite{TVV_NonAbelianTopologicalOrder_2022, tantivasadakarn2021long}.

In this work, we introduce a new model, called \textit{Local Alternating Quantum-Classical Computations} ($\LAQCC$). In this model we alternate between running quantum circuits (constrained by locality), ending in the measurement of a subset of qubits, and fast classical computations based on the measurement results. The outcome of the classical computations are then used to control future quantum circuits. We allow for flexibility in this model, by giving different constraints to the power of both the quantum circuits and the classical circuits as well as the number of alternations between them. 
Most attention will be given to $\LAQCC$ containing quantum circuits of constant depth, classical circuits of logarithmic depth and at most a constant number of alternations between them. 
Any circuit constructed in this model is considered to be of constant depth. 
We restrict ourselves to logarithmic depth classical computations, as this is the first natural and non-trivial extension beyond constant-depth classical computations. 
Constant-depth classical computations do however also have an equivalent constant-depth quantum implementation.

The definition of $\LAQCC$ sharpens the original definition of \citeauthor{PhamSvore2013} by adding constraints to the intermediate classical computations. This allows us to bound the power of $\LAQCC$ from above. 

The main result of \citeauthor{Cirac:2021}, that 1D translational invariant MPS with fixed bond dimension can be prepared by $\mathsf{LOCC}$-assisted circuits, relies on local symmetries of the MPS. These symmetries allow them to prepare local states (on a constant number of qubits) and glue them together by doing one round of the appropriate entangling measurement and corrections, after which they run a round of local unitaries to get the desired result. This general scheme for preparing states that exhibit an MPS description with the appropriate local symmetries requires only geometrically local unitaries and one round of measurement and corrections an therefore is accessible in $\LAQCC$. Studying different local symmetries, known as Symmetry Protected Topological (SPT) phases of matter, to find measurement-based constant depth circuits for states is a broad ongoing field of research~\cite{TVV_NonAbelianTopologicalOrder_2022, tantivasadakarn2021long, smith2023deterministic}. 
All these schemes have a $\LAQCC$ implementation.

%$\LAQCC$-circuits also exist for general schemes of preparing local states, based on the local tensors, and gluing them together using one round of entangled measurement and corrections, based on the local symmetry. 
%The main result of \citeauthor{Cirac:2021}, that 1D translational invariant MPS with fixed bond dimension can be prepared by $\mathsf{LOCC}$-assisted circuits, relies heavily on local symmetries of the MPS and as a result also has an equivalent $\LAQCC$ implementation. 
%The corrections applied after the measurement round are local unitaries depending on the local symmetries of the MPS. 

 

%This general scheme of preparing local states, based on the local tensors, and gluing it together by doing one round of entangled measurement and corrections, based on the local symmetry, is accessible in $\LAQCC$.
Note however that \citeauthor{Cirac:2021} also suggest a circuit for the $W$-state.
This circuit uses sequentially and dependent measurement-based corrections of the ancilla qubits. 
These dependent measurements translate to sequential alternations between the quantum and classical circuits and therefore increase the total depth to linear depth, exceeding the constant-depth constraints imposed by $\LAQCC$-circuits. 

We study the power of the $\LAQCC$ model with respect to state preparation, showing that even with only constant quantum-depth and logarithmic classical depth it remains possible to prepare states with long-range entanglement.
Another surprising result is that it is unlikely that $\LAQCC$ circuits are classically simulatable. We show that any instantaneous quantum polynomial-time (IQP) circuit~\cite{Bremner2010,Shepherd2009} has an $\LAQCC$ implementation.
Classical simulation of IQP circuits implies the collapse of the polynomial hierarchy to the third level, which is not believed to be true~\cite{Bremner2017}. Therefore, we expect that $\LAQCC$ circuits are unlikely to be classically simulatable. We bound the power of $\LAQCC$ by showing that it is contained in $\QNC^1$, the class of polynomial-size, log-depth circuits.

Next, we also study the power that intermediate classical calculations can add to quantum computations, by considering a new model that alternates between polynomially many polynomial-depth quantum circuits and unbounded classical computations
We study this model by doing a complexity theoretical analysis, where we draw inspiration from the notions of complexity given by \citeauthor{RosenthalYuen:2022}, \citeauthor{MetgerYuen:2023}, and \citeauthor{Aaronson:2004}.
All three complexity notions are based on the notion of state preparation, instead of more traditional definition of complexity such as the decidability of a computational problem. 
The first two consider classes based on sequences of quantum states preparable by a polynomial-sized quantum circuit, where the circuits are uniformly generated by a computational class, for instance, the class $\mathsf{PSPACE}$, which results in the complexity class $\mathsf{StatePSPACE}$~\cite{RosenthalYuen:2022,MetgerYuen:2023}.
The third notion considers a relative complexity, where the complexity is measured between two given states, and is measured by the number of gates, from a given gate-set, required to transform one state in another state~\cite{Aaronson:2004}. 
For our definition of state preparation complexity, we drop the uniformity constraint from~\cite{RosenthalYuen:2022,MetgerYuen:2023} and define a class as $\mathsf{StateX}$, which refers to states preparable by circuits of type $\mathsf{X}$. 
As an example, if $\mathsf{X} = \QNC^0$, this results in the class $\mathsf{StateQNC^0}$, which is the set of states preparable from the $\ket{0}^n$ state by poly-size constant-depth circuits. 
This notion is similar to the relative complexity from~\cite{Aaronson:2004}, where one state is the  $\ket{0}^n$ state and instead of counting the number of gates we consider the set of states preparable by a fixed number of gates. Using this notion of complexity we show that any state preparable by an $\LAQCC^*$ circuit is also preparable by a $\mathsf{PostQPoly}$ circuit, the class of circuits of polynomial depth with an additional post-selection gate. 

All Clifford circuits have a constant-depth $\LAQCC$ implementation, implying that any stabilizer state can be implemented by a constant-depth $\LAQCC$ circuit, see Section~\ref{sec:clifford_circuits} for a proof of this statement. 
Efficient circuits for stabilizer states have been known already through measurement-based quantum computing. Therefore this paper focuses on the preparation of non-stabilizer states, and as a surprising result we find novel constant-depth protocols for four very natural classes of non-stabilizer states.
Despite the extensive research into these four classes of non-stabilizer states and the many applications of them, no efficient constant- or low-depth state preparation protocols are known yet. We specifically consider these four classes as they are all often used as initial states in other algorithms.

The first state is a uniform superposition over an arbitrary number of states. 
This state finds applications in many quantum algorithms, as they often start with a uniform superposition over multiple states. 
This superposition is often achieved by applying Hadamard gates to every qubit due to its simplicity to prepare. 
Yet, the analysis of many algorithms, such as Shor's algorithm~\cite{Shor:1997}, would benefit from a different initial superposition. 
The circuit to prepare the uniform superposition over an arbitrary number of states uses an exact version of Grover search as a subroutine, that turns a probabilistic circuit, with a known constant probability of success, into a deterministic circuit. 
We use the circuit for preparing a uniform superposition over an arbitrary number of states as a subroutine in the next two quantum state preparation protocols. 

The second state is the $W$-state, the uniform superposition over all computational basis states of Hamming-weight~$1$, a natural long-ranged entangled state that displays a fundamentally nonequivalent type of entanglement from the Greenberger–Horne–Zeilinger state~\cite{WState:2000}, for which $\LAQCC$-type constant-depth circuits were previously known~\cite{PhamSvore2013, Cirac:2021}. 
The $W$-state is often used as benchmark for new quantum hardware~\cite{Haffner2005,Neeley2010,GarciaPerez:2021}. 
A novel way to prepare the $W$-state therefore gives a new way to benchmark different quantum devices with each other. 
A circuit for preparing the $W$-state was given in~\cite{Cirac:2021}, but this implementation requires sequentially alternating measurements followed by local unitaries, which in the $\LAQCC$ model is not considered to be of constant depth. 
We improve this protocol by giving an $\LAQCC$ implementation of the $W$-state, based on a compress-uncompress method that links the one-hot and binary encoding of integers.

The third state considered is the Dicke state, a generalization of the $W$-state, a superposition over all computational basis states with Hamming-weight $k$~\cite{Dicke:1954}. 
Dicke states have relevance in various practical settings.
For instance, for quantum game theory~\cite{zdemir2007}, quantum storage~\cite{Bacon_Compress:2006,Plesch:2010}, quantum error correction~\cite{ouyang2014permutation}, quantum metrology~\cite{toth2012multipartite}, and quantum networking~\cite{prevedel2009experimental}. 
Dicke states have been used as a starting state for variational optimization algorithms, most notably Quantum Alternating Operator Ansatz (QAOA)~\cite{Hadfield2019}, to find solutions to problems such as Maximum k-vertex Cover~\cite{Brandhofer2022,cook2020quantum}.
The ground states of physical Hamiltonians describing one-dimensional chains tend to show a resemblance to Dicke states such as states resulting from the Bethe ansatz, making them an ideal starting state when investigating the ground state behavior of these Hamiltonians~\cite{TDL_BetheAnsatzDerivation:2010,B_ExcitedStateQuantumPhaseTransitions:2013,DickeTransitions:2021}. 
For instance, the algorithm by \citeauthor{van2021preparing}, who give an algorithm to prepare the Bethe ansatz eigenstates of the spin-1/2 XXZ spin chain, starts by first preparing a Dicke state~\cite{van2021preparing}. 
A Dicke-state preparation protocol based on the compress-uncompress methodology used in the $W$-state furthermore finds applications in entanglement distillation, where the entanglement of a large state is concentrated on only a few qubits. 
Efficient deterministic circuits for preparing Dicke states have been proposed by \citeauthor{bartschi2019deterministic}~\cite{bartschi2019deterministic, bartschi2022deterministic_short_depth}. 
They provide a quantum circuit of depth $\mathO(k \log(\frac{n}{k}))$, allowing arbitrary connectivity, to prepare a Dicke state, which they conjecture to be optimal when $k$ is constant. 
In this work, we provide a constant-depth $\LAQCC$ circuit below their conjectured bound already for constant $k$. 
However, this does not directly disprove their conjecture, as we allow for intermediate measurements and classical computations. 
More significantly, we even construct constant-depth $\LAQCC$ circuits for $k = \mathO(\sqrt{n})$ greatly improving their bound.
This construction extends the compress-uncompress method for the $W$-state combined with additional subroutines. 

We continue with a log-depth state preparation protocol for the Dicke-state for arbitrary $k$. 
This protocol implements an efficient transformation between the factoradic number representation and the combinatorial number representation of a positive integer. 
The combinatorial number representation relates directly to the Dicke state. 
The provided efficient transformation between number representation systems might be of independent interest. 

We conclude by modifying our protocol for preparing a Dicke-state to a protocol that prepares quantum many-body scar states in constant-depth. 
These states have low entanglement and longer coherence times than states with similar energy density.
These characteristics make many-body scar states interesting to analyze and relevant within physics.
Many-body scar states appear for instance in the AKLT model~\cite{AKLT:1987,MRBAR:2018,MRB:2018} and different spin models~\cite{SI:2019,MOBFR:2020}.
Known methods for preparing these states have polynomial-depth~\cite{Gustafson:2023}, whereas our circuit has constant depth. 

% We conclude by studying the power that intermediate classical calculations can add to quantum computations. 
% In this study, we define a new model that relaxes constant-depth quantum circuits to polynomial depth quantum circuits, log-depth classical calculations to unbounded classical computations and a constant number of alternations to a polynomial number of alternations. 
% We call this model $\LAQCC^*$. 
% We study this model by doing a complexity theoretical analysis, where we draw inspiration from the notions of complexity given by \citeauthor{RosenthalYuen:2022}, \citeauthor{MetgerYuen:2023}, and \citeauthor{Aaronson:2004}.
% All three complexity notions are based on the notion of state preparation, instead of more traditional definition of complexity such as the decidability of a computational problem. 
% The first two consider classes based on sequences of quantum states preparable by a polynomial-sized quantum circuit, where the circuits are uniformly generated by a computational class, for instance, the class $\mathsf{PSPACE}$, which results in the complexity class $\mathsf{StatePSPACE}$~\cite{RosenthalYuen:2022,MetgerYuen:2023}.
% The third notion considers a relative complexity, where the complexity is measured between two given states, and is measured by the number of gates, from a given gate-set, required to transform one state in another state~\cite{Aaronson:2004}. 
% For our definition of state preparation complexity, we drop the uniformity constraint from~\cite{RosenthalYuen:2022,MetgerYuen:2023} and define a class as $\mathsf{StateX}$, which refers to states preparable by circuits of type $\mathsf{X}$. 
% As an example, if $\mathsf{X} = \QNC^0$, this results in the class $\mathsf{StateQNC^0}$, which is the set of states preparable from the $\ket{0}^n$ state by poly-size constant-depth circuits. 
% This notion is similar to the relative complexity from~\cite{Aaronson:2004}, where one state is the  $\ket{0}^n$ state and instead of counting the number of gates we consider the set of states preparable by a fixed number of gates. Using this notion of complexity we show that any state preparable by an $\LAQCC^*$ circuit is also preparable by a $\mathsf{PostQPoly}$ circuit, the class of circuits of polynomial depth with an additional post-selection gate. 

\paragraph{Summary of results}
\begin{itemize}
    \item We give a new definition of a computational model that captures the power of the four step process: applying a constant number of layers of one- and two-qubit gates; performing a syndrome measurement; perform a fast classical computation determining corrections; apply corrections. We call this model \emph{Local Alternating Quantum Classical Computations}, or $\LAQCC$ for short. In this model we bound the allowed quantum operations, intermediate classical calculations, and number of rounds separately. In Section~\ref{sec:LAQCC_model} we define this model and give a list of operations based on results from literature contained in this computational model. In some of these operations we explicitly use that we allow for multiple, but at most constant, rounds  of corrections.
    \item  We show show that there exist $\LAQCC$ circuits that can not be weakly simulated in Section~\ref{sec:IQP_in_LAQCC}. We further show that for every $\LAQCC$ circuit there exists a $\QNC^1$ circuit simulating it perfectly, in Section~\ref{sec:LAQCC_in_QNC1}.
    \item We introduce a new type computational complexity for preparing states and show that the extension of $\LAQCC$ where we allow a polynomial number of rounds and unbounded classical computation, is contained in $\mathsf{PostQPoly}$, the class of polynomial circuits with post-selection, in Section~\ref{sec:Complexity results}.
    \item We show a protocol to prepare the uniform superposition state of size $q$ in $\LAQCC$ using $\mathO(\ceil{\log_2(q)}^2)$ qubits in Section~\ref{sec:superposition_modulo_q}. 
    \item We show a protocol to prepare the $W_n$ state in $\LAQCC$ using $\mathO(n\log(n))$ qubits in Section~\ref{sec:W_state_in_LAQCC}.
    \item We show two ways of preparing the Dicke-$(n,k)$ state. The first method is in $\LAQCC$, works up to $k = \mathO(\sqrt{n})$, uses $\mathO(n^2\log(n))$ qubits, and is found in Section~\ref{sec:dicke:small_k}. The second method is in $\LAQCC\text{-}\mathsf{LOG}$ (an extension of $\LAQCC$ allowing for logarithmic number of alterations instead of constant), works for any $k$, uses $\mathO(\text{poly}(n))$ qubits, and is found in Section~\ref{sec:Dicke_in_LAQCC_LOG}. 
    \item We extend on our $\LAQCC$ method of generating Dicke-$(n,k)$ states for $k = \mathO(\sqrt{n})$ and show a protocol to generate many-body scar states for a particular Hamiltonian in $\LAQCC$ (Section~\ref{sec:many_body_scar}). 
\end{itemize}
Summarized in a table, we provide the following state generation protocols:
\begin{table}[htb]
\centering
\begin{tabular}{l|l|l|l}
\textbf{State description} & \textbf{Width} & \textbf{Depth} & \textbf{Implementation}\\
\hline 
Uniform superposition mod $q$: $\frac{1}{\sqrt{q}} \sum_{i = 0}^{q-1}\ket{i}$ & $\mathO(\ceil{\log^2 q})$ & $\mathO(1)$ & Section~\ref{sec:superposition_modulo_q}\\

$W$-state: $\frac{1}{\sqrt{n}}\sum_{i = 0}^{n-1}\ket{e_i}$ & $\mathO(n \log n)$ & $\mathO(1)$ & Section~\ref{sec:W_state_in_LAQCC}\\

Dicke-$(n,k)$, $k = \mathO(\sqrt{n})$: $\binom{n}{k}^{-1/2}\sum_{x \in \{0,1\}^n: |x| = k} \ket{x}$ &  $\mathO(n^2\log n)$ & $\mathO(1)$ 
&Section~\ref{sec:dicke:small_k}\\

Dicke-$(n,k)$: $\binom{n}{k}^{-1/2}\sum_{x \in \{0,1\}^n: |x| = k} \ket{x}$ & $\mathO(\text{poly}(n))$ & $\mathO(\log n)$ &Section~\ref{sec:Dicke_in_LAQCC_LOG}\\

QMBS: $\ket{S_k} = \frac{1}{k! \sqrt{\mathcal N(n,k)}}(Q^\dagger)^k \ket{\Omega}$ &  $\mathO(n^2\log n)$ & $\mathO(1)$  &  Section~\ref{sec:many_body_scar}
\end{tabular}
\caption{Summary of state preparation protocols given in this paper.}
\label{tab:sate_prep}
\end{table}
In the entry for the quantum many-body scar state $Q$ denotes the raising operator and $\mathcal N(n,k)=\binom{n-k-1}{k}$. 
Section~\ref{sec:many_body_scar} will provide more details on the variables and the implementation. 

\paragraph{Organization of the paper}
\noindent We first introduce relevant preliminaries in Section~\ref{sec:preliminaries}. 
In Section~\ref{sec:LAQCC_model} we formally define the class of Local Alternating Quantum-Classical Computations ($\LAQCC$). We also show that any Clifford circuit can be implemented in constant depth $\LAQCC$ (a result based on a result from measurement-based quantum computing~\cite{jozsa2006introduction}). 
This result allows us to give many useful multi-qubit gates and routines in Section~\ref{sec:gates_created_in_LAQCC}. 
Beyond that we show that constant depth $\LAQCC$ circuits are contained in $\QNC^1$ and that any $\mathsf{IQP}$ circuit has an $\LAQCC$ implementation.
We conclude this section with an analysis of a more powerful instantiation of $\LAQCC$ and show an inclusion with respect to the class $\mathsf{PostQPoly}$, which is the class of circuits of polynomial depth with one additional post-selection gate. 
In Section~\ref{sec:state_prep_in_LAQCC} we give $\LAQCC$ circuit implementations for preparing the uniform superposition over an arbitrary number of states, the $W$-state and the Dicke state up to $k = \mathO(\sqrt{n})$. We furthermore give a log-depth circuit implementation for preparing the Dicke state for any $k$. We conclude by showing a $\LAQCC$ circuit for generating many body scar states of a particular type of Hamiltonian.



\section{Related Work}
%\subsection{Cost Volume based Deep Stereo Matching}
%Stereo matching is a typical problem that has been studied for decades and a well-known four-step pipeline \cite{scharstein2002taxonomy} has been established, where cost volume construction is an indispensable step. Current state-of-the-art stereo matching methods are all cost volume based methods and they can be categorized into two types. Typically, a cost volume is a 4D tensor of height, width, disparity, and features. The first category just uses a full correlation to generate a single-feature cost volume. Such methods are usually efficient but lose much information because of the decimation of feature channels. Many previous work, including Dispnet \cite{dispnet}, MADNet \cite{madnet}, IResNet \cite{iresnet} and AANet \cite{aanet}, belong to this category. The second category usually uses concatenation \cite{gcnet} or group-wise correlation \cite{gwcnet} to generate a multi-feature 4D cost volume. Such a method can achieve better performance while requiring higher computational complexity and memory consumption. Actually, a majority of the top-performing networks in public leaderboards belong to this category, such as GANet \cite{ganet}, CSPN \cite{cspn} and ACFNet \cite{acfnet}. These methods generally employ multiple 3D convolution layers to constantly regularize the 4D cost volume and then apply softmax over the disparity dimension to produce a discrete disparity probability distribution. The final predicted disparity is obtained by softly weighting indices according to their probability, which is also called soft argmin in GCNet \cite{gcnet}. However, soft argmin leaves the output susceptible to multi-modal disparity probability distributions. ACFNet \cite{acfnet} observes this problem and proposes to directly supervise the cost volume with unimodal ground truth distributions. In contrast, we define an uncertainty estimation to quantify the degree to which the cost volume tends to be multi-modal distribution, higher implies the higher possibility of estimation error.

\subsection{Multi-scale Cost Volume based Stereo Matching}
Cost volume construction is an indispensable step in the well-known four-step pipeline for stereo matching \cite{scharstein2002taxonomy, pamisurvey1, pamisurvey2}. Typically, current state-of-the-art stereo matching methods can be categorized into two types of cost volume-based methods, where the cost volume is a 4D tensor of height, width, disparity, and features. The first category usually uses the single-feature 3D cost volume generated by full correlation, which is efficient while losing much information due to the decimation of feature channels. Many real-time methods, such as Dispnet \cite{dispnet}, MADNet \cite{madnet, madnet_pami} and AANet \cite{aanet}, belongs to the category. Moreover, two-stage refinement \cite{mcvmfc} and pyramidal towers \cite{madnet} are commonly applied in the single-feature cost volume based network to construct multi-scale cost volume. The second category usually uses the multi-feature 4D cost volume generated by concatenation \cite{gcnet} or group-wise correlation \cite{gwcnet}, which can achieve better performance with higher computational complexity and memory consumption. Most top-performing networks, including GANet \cite{ganet}, CSPN \cite{cspn} and ACFNet \cite{acfnet} belong to this category. 
% In these methods, the 4D cost volume is constantly regularized by multiple 3D convolution layers and then a discrete disparity probability distribution can be produced by softmax. Next, the final predicted disparity can be obtained by softly weighting indices according to their probability \cite{gcnet}. However, such output is susceptible to multimodal disparity probability distributions and ACFNet \cite{acfnet} gives a solution by directly supervising the cost volume with unimodal ground truth distributions to alleviate this problem. 
Recently, to alleviate the high computational complexity and memory consumption when employing multi-feature 4D cost volumes, \cite{cvpmvsnet, cascade, uscnet} propose to use cascade cost volume representation in multi-view stereo. These methods usually first predict an initial disparity at the coarsest resolution of the image and then gradually refine the disparity by narrowing down the disparity search space. More closely related to our approach is Casstereo \cite{cascade}, which first extended such representation to stereo matching. It selected to uniform sample a pre-defined range to generate the next stage’s disparity search range. Instead, we employ pixel-level uncertainty estimation to adaptively adjust the next stage disparity searching range and generate pseudo-labels for subsequent domain adaptation. Our method also shares similarities with UCSNet \cite{uscnet}, which constructs uncertainty-aware cost volume in multi-view stereo while it doesn’t employ uncertainty estimation to generate pseudo-labels.

%\subsection{Multi-scale Cost Volume based Deep Stereo Matching} 
% \subsection{Multi-scale Cost Volume based Stereo Matching} 
%Multi-scale cost volume firstly was applied in the single-feature cost volume based network with the form of two-stage refinement \cite{mcvmfc} and pyramidal towers \cite{madnet}. Recently, cascade cost volume representation \cite{cvpmvsnet, cascade, uscnet} was proposed in multi-view stereo to alleviate the high computational complexity and memory consumption when employing multi-feature 4D cost volumes. These methods generally predict an initial disparity at the coarsest resolution of the image. Then, they will narrow down the disparity search space and gradually refine the disparity. More closely related to our approach is Casstereo \cite{cascade}, which first extended such representation to stereo matching. It selected to uniform sample a pre-defined range to generate the next stage’s disparity search range. Instead, we employ uncertainty estimation to adaptively adjust the next stage pixel-level disparity searching range and push the next stage's cost volume to be predominantly unimodal.

% The single-feature cost volume based network with the form of two-stage refinement \cite{mcvmfc} and pyramidal towers \cite{madnet} first employ multi-scale cost volume for stereo matching. Recently, to alleviate the high computational complexity and memory consumption when employing multi-feature 4D cost volumes, \cite{cvpmvsnet, cascade, uscnet} propose to use cascade cost volume representation in multi-view stereo, which generally predict an initial disparity at the coarsest resolution of the image. Then, the disparity search space is narrowed down and the disparity is gradually refined. More closely related to our approach is Casstereo \cite{cascade}, which first extended such representation to stereo matching. It selected to uniform sample a pre-defined range to generate the next stage’s disparity search range. Instead, we employ uncertainty estimation to adaptively adjust the next stage pixel-level disparity searching range and push the next stage's cost volume to be predominantly unimodal.

% Figure environment removed

\subsection{Robust Stereo Matching} 
There exist three categories of generalization definitions for robust stereo matching. 1) Cross-domain Generalization: the network’s ability to perform well on unseen scenes (cannot see the image pairs of the target domain in advance). Towards this end, Jia et al \cite{sungeneralizaiton} propose to incorporate scene geometry priors into an end-to-end network. Zhang et al \cite{dsmnet} introduce a domain normalization and a trainable non-local graph-based filter to construct a domain-invariant stereo matching network. 2) Adapt Generalization: the network’s ability to adapt pre-trained models to the new domain with unlabeled target data. Previous work usually pre-trains the models on synthetic data and then adapts it to new target domains with Graph Laplacian regularization \cite{zoom}, non-adversarial progressive color transfer \cite{adastereo}, and Knowledge Reverse Distillation \cite{aohnet}. More closely related to our approach are \cite{aohnet, unsuperviseddomainadaptation} in stereo matching and Monoresmatch \cite{monoresmatch} in monocular depth estimation, which also proposes to generate a pseudo-label for domain adaptation. However, these methods all select to employ classical stereo matching methods \cite{sgm} alongside with confidence estimators, e.g., left-right consistency check to generate pseudo-labels. That is all these methods need an independent method to generate corresponding pseudo-labels. Instead, the proposed method is an end-to-end network that can generate the predicted disparity map, corresponding uncertainty map and pseudo-labels jointly, which is a more simple, yet efficient way. 
% Instead, our proposed method can employ pixel-level and area-level uncertainty estimation to self-distill the predicted disparity maps of our pre-training model and generate sparse while reliable pseudo-labels to align the domain gap, which is a more simple, yet efficient way. 
3) Joint Generalization: the network’s ability to perform well on a variety of datasets with the same model parameters. MCV-MFC \cite{mcvmfc} introduces a two-stage finetuning scheme to achieve a good trade-off between generalization and fitting capability on multiple datasets. However, it doesn’t touch the inner difference between diverse datasets, e.g, the unbalanced disparity distribution. To further address this problem, we propose a cascade cost volume to adaptively the next stage disparity searching space, where the pixel-level uncertainty estimation is at the core.

% \subsection{Monocular Depth Estimation}
% Monocular depth estimation aims to estimate depth values from a single image, instead of stereo images or multiple frames in a video. This problem is ill-posed because of the ambiguity of object sizes. However, humans could estimate the depth from a single image with prior knowledge of the scenes. Recently, learning based methods were explored to learn depth values by supervised or unsupervised learning. Eigen et al. first employed Convolutional Neural Networks (CNN) to predict depth in a coarse-to-fine manner and further improved its performance by multi-task learning. Liu et al. presented deep convolutional neural fields model by combining deep model with continuous CRF. Li et al. [22] refined deep CNN outputs with a hierarchical CRF. Multi-scale continuous CRF was formulated into a deep sequential network by Xu et al. [45] to refine depth estimation. Unsupervised methods tried to train monocular depth estimation with stereo
% image pairs or image sequences and test on single images. Garg et al. [9] used novel image view synthesis loss to train a depth estimation network in an unsupervised way. Godard et al. [11] introduced left-right consistency regularization to improve the performance of view synthesis loss. Recently, some work also propose to use the stereo matching network as a proxy to learn depth from synthetic data or directly employ traditional stereo matching methods to distill proxies labels from the target domain, which proves the feasibility of distilling stereo matching networks to learn monocular depth estimation.




\section{Problem Formulation}
\label{prob_form}

\subsection{Dialogue Response Generation as an MDP}
We look at the problem of dialogue text generation, \textit{i.e.}, generating response utterances in a dialogue setting. We begin with a supervised dataset of context response pairs $\{ x^i, y^i \}_{i=1}^{N}$, where context $x$ is the conversation history, and response $y = \{y_1, \dots, y_T \}$ is a target sequence of tokens. We map each data point $(x, y)$ to an episode of a Markov Decision Process (MDP), which we define below (Fig. \ref{fig:prob_form}):
\begin{itemize}[nosep]
\item \textbf{States, $s_t \in \mathcal{S}$} is the context $x$ and the partially generated sequence of tokens up to and including time step $t$, $\hat{y}_{<t}:=\{\hat{y}_1,\dots, \hat{y}_t\}$. 
\item \textbf{Actions, $a_t\in \mathcal{A}$} are the set of next tokens $\hat{y}_{t+1}$ available from the vocabulary $V$
\item \textbf{Transition function, $\mathcal{T}(s_{t+1} | s_t, a_t)$} is deterministic since every state-action pair $(\hat{y}_{<t}, \hat{y}_{t+1})$ leads to a unique state $\hat{y}_{<t+1}$ for the next step.
\item \textbf{Rewards, $r_t: \mathcal{S}\times\mathcal{A}\rightarrow [0, 1]$} is a terminal reward that computes similarity generated response $\hat{y}$ and 
target response $y$
\item \textbf{Horizon, $T$} is time horizon. Each episode ends when the current time step $t$ exceeds $T$ or an end-of-sentence (EOS) token is generated.
\end{itemize}
The goal is to learn a policy $\pi: s_t \rightarrow a_t$ maximizing \textit{return}, \emph{i.e.} the cumulative reward over an episode $\mathbb{E}_{\pi} \sum_{t=0}^T \gamma^t r_t$. We assume undiscounted cumulative rewards, \emph{i.e.} $\gamma=1$.

\subsection{Rewards for Dialogue Response Generation}
We define the reward to be a similarity metric between the generated text $\hat{y}$ and the speaker's ground truth utterance $y$. Such a metric should capture both what the speaker is trying to communicate and the relevance to the conversation. 
% Learnt / Human annotated vs automated?

One option is to collect human-in-the-loop annotations, \emph{i.e.} what the speaker would likely prefer to say. However, this requires costly human supervision. Automated metrics, such as BERTScore \cite{zhang2019bertscore}, BLEURT \cite{sellam2020bleurt}, offer a promising alternative. They are able to capture models of human preference and are cheap to evaluate. 

We use a terminal reward since the similarity can only be evaluated at the end of the utterance, and since the same content can be expressed in different styles, \emph{e.g.} \emph{The flight from New York to Boston has been confirmed.} vs. \emph{Your JFK to BOS flight has been booked.}

\subsection{Why Offline Reinforcement Learning?}
\label{sec:why_offline_rl}

In online reinforcement learning, an agent learns by interacting with an environment in real-time. This presents an explore-exploit trade-off, where the agent must balance the need to try out new actions to learn about the environment with the need to exploit its current knowledge to maximize reward. This can be particularly challenging in text generation, as action space (\emph{i.e.} vocabulary size) is often large, \emph{e.g.} of the order of 50,000 words for GPT-based models~\cite{radfordlanguage}. Another problem is that the reward landscape is sparse, hence policies during training can get stuck in local minima where reward is persistently zero. 

For text generation, we argue that an offline setting is reasonable. There exists good generation policies, e.g. policies from teacher forcing, that can generate a set of responses such that one of them is close enough to the human response. Also, once a token is generated, we \emph{deterministically} transition to next state with the additional token appended to the prefix, \emph{i.e.} no interaction is needed to learn the environment.
 
Offline RL provides a learning paradigm that combines both supervised learning’s ability to leverage existing data with the general utility optimization power of online reinforcement learning methods. We collect an offline dataset of state transitions $\mathcal{D} = \{(s_t^i, a_t^i, r_t^i, s_{t+1}^i)\}_{i=1}^{N}$\footnote{We suppress superscripts when considering a single transition.} using a behavior policy $\pi_{\beta}$, typically a policy trained via supervised learning. The goal is to learn a policy $\pi$ that maximizes performance on the dataset while staying close to the behavior policy:
% \begin{equation}
    \begin{align}
        \max_{\pi} J_{\mathcal{D}}(\pi) - \alpha D(\pi, \pi_{\beta}) \, ,
    \end{align}
% \end{equation}
where $J_{\mathcal{D}}(\cdot)$ is performance on dataset $\mathcal{D}$ and $D(\cdot, \pi_{\beta})$ is distributional regularization against behavior policy $\pi_{\beta}$.

In this section, we describe how to learn repair strategies from the  unsafe programs and edits collected in Section~\ref{sec:data}. We define a \dsl (Section~\ref{subsec:dsl}) to express repair strategies that take an \pdg of an unsafe program  as input and generate a safe program as output. The DSL is expressive and can even express bad strategies that don't generalize well to programs in the wild. We provide examples of such bad strategies and good strategies that generalize well  (Section~\ref{subsec:examples}). We learn good repair strategies  in a data-driven manner using an example-based synthesis algorithm (Section~\ref{subsec:synthesis}). %Finally, given a new unsafe program and a set of learned repair strategies, we apply these strategies and generate  candidate repairs (Section~\ref{subsec:applying}).


%Our goal is to use the collected data to learn high-level general repair strategies. We learn these repair strategies over a joint representation of the \astree with the annotations inferred from the \sa tool (the representation referred to as \pdg ahead).  These inferred \sa tool annotations allow us to take the advantage of rich semantic information while performing \unsure{repairs}. Figure ~\ref{fig:example1-pdg} shows an example \pdg corresponding to the unsafe code shown in Figure ~\ref{fig:unsafememberex}. We develop a powerful \dsl that can utilize the annotations in the \pdg structure and learns repair strategies using a deductive synthesis algorithm. More specifically, strategies in this \dsl operate over the \pdg structure of unseen code-snippets and suggest appropriate edits correspondingly. \aksays{The following sentence can be removed if space becomes a constraint.} Section~\ref{subsec:dsl} describes the \dsl, Section~\ref{subsec:synthesis} talks about the synthesis algorithm, and Section~\ref{subsec:applying} demonstrates strategies in this \dsl can be applied. 


\subsection{\dsl for repair strategies}
\label{subsec:dsl}
We introduce a novel \dsl to express repair strategies in Figure~\ref{fig:fixing-dsl}.
%that use the knowledge of program semantics annotated on \pdg instead of just using the syntactic program structure and in-turn are more expressive and generalize better. These strategies take the an unsafe-program as input and return candidate repair programs by performing tree-edit-operations.
At a high level, the strategies define a three-step process where  they provide a computation to identify the edit-location node \editloc, a computation to identify the child index $\editindex$ of \editloc where repair happens, and a computation to generate the AST that must be placed at  index $\editindex$ of \editloc for the repair. The main part of these computations involve traversing paths of the input unsafe program \prog.
%The edit-operation can either be inserting a syntactic-child at \editloc (\insertsc) or replacing a syntactic-child with another tree at \editloc (\replace). The index at the \astree-node $\editloc$ where the insertion or replacement occurs is called the edit-index (\editindex). The tree that is inserted or replaces another existing tree at the \editloc is materialized hierarchically for the given example by defining abstract program structure using a combination of constant structure and references to \astree-nodes in the existing program \prog. These \astree-nodes are called reference-locations (\refloc). To find these locations (\editloc, \refloc) in a given program, the strategies abstractly store \textit{traversals} which materialize into a \textit{concrete} \astree-node in the given programs.  

%\naman{todo - talk about traversal in the introduction, background etc.}
\input{dsl}

%We present our \dsl in Figure ~\ref{fig:fixing-dsl}. 
%The DSL is a list of definitions for various non-terminals in the grammar. For each non-terminal, we define a corresponding type and a set of production rules. Each production rule is either a fixed expression, or an operator applied to other non-terminals or fixed-expressions in the grammar.  
The top-level production rule of the DSL defines strategies, \strategy, with type \newtextsc{Strategy}. 
%A \node is either the source node (\prog.source) or an application of \traversal on another \node. 
\gettraversal, \getclauses, and \getindex are all functions that take a \node $n$ as input and return a \node, \bool, and \integer as output respectively. The edit-AST, \eastree, is similar to a syntactic variant of \astree (i.e. no semantic edges) which we define in Section~\ref{sec:data} with one addition. It has reference nodes that, when applying the strategy to the input \pdg of \prog, are materialized from sub-trees of this \pdg, where the root nodes of these sub-trees are identified by traversing paths in the input. 
%Finally, edge-type (denoted by \edgetype) is an enumeration describing the type of edge, i.e. syntactic or semantic, and parent or child, as defined in Section~\ref{sec:data}. 

% Given these types, we now define the operators used in our \dsl. 
The strategy \strategy is of two types, \insertsc or \replace. \DMethod{Insert}{\I{L}}{\I{I}}{\I{O}}\ declaratively expresses the computation that computes the edit-location \editloc by traversing the path supplied in \I{L}, then computes \editindex, the index of edit-location,  by evaluating \I{I}(\editloc), and inserts the materialization of \I{O} as a syntactic-child \astree at index \editindex of the edit-location \editloc. \DMethod{Replace}{\I{L}}{\I{I}}{\I{O}}\ is similar and performs a replacement instead of an insertion.
%computes the \editloc and \editindex, and replaces syntactic-child of \editloc at \editindex with \I{O}.
%The insertion and replacement operations modify the nodes $\mathcal{N}$ and edges $\mathcal{E}$ of the \astree (Figure~\ref{fig:astsyntax}) appropriately. 

\node (\I{L}) is either the node corresponding to the source of vulnerability (\prog.source) or the target of the path corresponding to the traversal \DMethod{ApplyTraversal}{L}{\I{F}$_k\ o\ $\I{F}$_{k-1}\ o\ \cdots$\I{F}$_0$}. 
Here, each \I{F}$_i$ 
is a function that takes a node 
$n$ as input, performs a traversal from $n$, and returns the traversal's target node $n'$. 
Thus, \T{ApplyTraversal} can be recursively defined as \DMethod{ApplyTraversal}{\I{F}$_0$(L)}{\I{F}$_k\ o\ $\I{F}$_{k-1}\ o\ \cdots$\I{F}$_1$}\ if $k>0$ and \I{F}$_0$(L) otherwise. 

\newtextsc{GetTraversal} (\I{F}) defines a function that takes a node $n$ and returns a node $n'$ reachable from $n$ and can be of two types. Given $n$, the \DMethod{GetEdge}{\I{ET}}{\I{I}}\ operator first finds the possible single-edge traversals of type \I{ET} and indexes it using \I{I}. Specifically, if edge type \I{ET} is a parent then it returns the parent of $n$. Otherwise, 
it finds a set of $N$ of nodes that are connected with $n$ via the edge type \I{ET}, i.e., $N = \mathcal{E}(n, \I{ET})$, and returns the node $N[I(\I{n})]$ at the index given by $I$. In contrast, $\DMethod{GetKleeneStar}{\I{ET}}{\I{C}}(n)$  performs a \newtextsc{KleeneStarTraversal} that iteratively traverses edges of type \I{ET}, staring from input node $n$, until it reaches an edge whose target  node $n^{i}$ satisfies the condition defined by the clause \I{C}. Formally, \newtextsc{KleeneStarTraversal} can be defined recursively as $KE(n_1,ET,C) = \I{C}(n_1)? n_1 : \left(let\ t\in\mathcal{E}(n_1,ET)\ in\ KE(t,ET,C)\right)$. Here, the node $t$, which is target of an edge with source $n_1$ and type $ET$,  is chosen non-deterministically and our implementation resolves this non-determinism through a breadth-first search.
%A \traversal is a relation between nodes $n_1$ and $n_2$ such that there is an edge or a sequence of edges between them. 

\newtextsc{GetIndex} (\I{I}) defines a function that takes a node $n$ and returns a \integer. It is either a constant function that returns a fixed integer $z$ or a \DMethod{GetOffsetIndex}{\I{L}, \I{z}}. \DMethod{GetOffsetIndex}{\I{L}, \I{z}}\ takes a node $n$ as input and returns an integer $DO(n,L)+z$, where $DO(n_1,n_2)$ returns the index of syntactic child of $n_2$ who is a syntactic ancestor of $n_1$. 

\eastree (\I{O}) defines the edit \astree with reference nodes which, given an input program \prog, are materialized to a concrete \astree. The \eastree can either be a \T{ConstantAST} or a \T{ReferenceAST}. Specifically, \DMethod{ConstantAST}{$\tau$}{\I{value}}{\I{O}$_1$}{\I{O}$_2$}{$\cdots$}{{\I{O}$_k$}}\ returns an \eastree that has a type $\tau$, string representation \I{value}, and is recursively constructed with sub-trees \I{O}$_1 \cdots$ \I{O}$_k$ as syntactic children, each of which can either be a \T{ConstantAST} or a \T{ReferenceAST}. The \DMethod{ReferenceAST}{\I{L}}, when applying the strategy, finds a node $n$ in \prog by traversing the path described in \I{L} and returns a copy of the (syntactic) sub-tree of \prog rooted at $n$. %Next, we show examples of strategies written in this DSL and how to learn them automatically.

% Finally, note that the traversals can be composed by applying multiple \T{ApplyTraversal} operators sequentially. We use this key insight into developing our learning from examples setup. 
\newcommand{\newwrapbox}[2]{\adjustbox{margin=1pt 1.3pt, bgcolor=white, frame=1pt, cframe=#1, color=#1}{#2}}
% Figure environment removed


\lstMakeShortInline[columns=fixed]@
\subsection{Example of strategies in our \dsl}
\label{subsec:examples}
Figure~\ref{fig:repair-strategy-ex1} describes   two possible repair strategies that are sufficient to repair the motivating example in Figure~\ref{fig:vulnerabilty-example1}. We first describe the good strategy in Figure~\ref{fig:strat1}, referred to as \strategyone,  and then compare it with the bad strategy \strategytwo in Figure~\ref{fig:strat2}. 

Given the program \prog in Figure~\ref{fig:vulnerabilty-example1}(a) as input, the strategy \strategyone
performs a replacement at index \I{I} of edit-location $L_e$ with the materialization of \I{O} (line 20 of \strategyone).
This process requires first finding the "semantic location" node \semloc. %The semantic location for \prog is shown in red in Figure~\ref{fig:example1-pdg}.
To this end, the strategy 
first  traverses a path from the node annotated as \T{source} by \sa  using \DMethod{GetKleeneStar}\ in Line~\ref{lst:line:semkleene} of \strategyone.  This \newtextsc{KleeneStarTraversal} starts from \T{source}, traverses semantic dataflow edges, and stops at a node 
corresponding to an identifier being used as the function name in a function call. 
 For the input program $P$, the traversal takes the semantic-child-edges 1-7 (Figure~\ref{fig:example1-pdg}) and stops at @foo@ in Line~\ref{lst:line:callerId-sink} of Figure~\ref{fig:vulnerabilty-example1}(a). Next, to reach the edit-location $L_e$, the strategy uses a \newtextsc{KleeneStarTraversal} that starts from \semloc, traverses syntactic parent edges,  and stops when it reaches a \blockstmt. For $P$, this traversal sets $L_e$  as the node corresponding to the \blockstmt between Lines~\ref{lst:line:handlers-run} and ~\ref{lst:line:handlers-run-end} of Figure~\ref{fig:vulnerabilty-example1}(a). Next, in Line~\ref{lst:line:offseteditindex} of \strategyone, the index \I{I} is set to the index corresponding to the  syntactic child of the edit-location $L_e$ who is an ancestor of the semantic location $L_s$ . For $P$, this index  materializes into $13$; the edge  outgoing from blue \blockstmt in Figure~\ref{fig:example1-pdg} to an ancestor of semantic location (shown in red) has label \T{ch:13}. Next, we materialize the \eastree defined in Line~\ref{lst:line:eastree} of \strategyone by  materializing the  reference-nodes. The \eastree \I{O} serializes into @if (REF1.hasOwnProperty(REF2)) { REF3 } @ where @REF1@, @REF2@, and @REF3@ correspond to \T{ReferenceAST} operators with locations as \reflocone, \refloctwo, and \reflocthree. \refloctwo traverses semantic-parent edges  from \semloc (Line~\ref{lst:line:goodref}) and materialize into @data.id@. Similarly, \reflocone and \reflocthree traverse syntactic children edges and materialize into @handlers@ and @foo(data);@ respectively. Thus, the \eastree \I{O} materializes  into @if (handlers.hasOwnProperty(data.id)) { foo(data); }@, which is the required repair. 

%When \strategyone is given the program \prog in Figure~\ref{fig:vulnerabilty-example1}(a) as input, then it first  traverses a path from the source node to the "semantic location"  \semloc using \DMethod{GetKleeneStar}{"SemChild"}{\DMethod{GetClause}{"Expr"}}\ in Line~\ref{lst:line:semkleene}. This leads to a \newtextsc{KleeneStarTraversal} with the stopping condition $\lambda n.\mathcal{T}[n] = \text{"CallExpr"}$. For the input program, the traversal skips through the semantic-child-edges 1-7 and reaches @foo@ in Line~\ref{lst:line:callerId-sink}. Next, it applies another \newtextsc{KleeneStarTraversal} starting from \semloc to reach \editloc in Line~\ref{lst:line:synkleene}. This traversal skips over syntactic-parent-edges and reaches the \blockstmt between Lines~\ref{lst:line:handlers-run} and ~\ref{lst:line:handlers-run-end}. Next, in Line~\ref{lst:line:offseteditindex}, the index \I{I} is computed as \DMethod{GetOffsetIndex}{Ls, 0}\ which means to pick the child-index of \editloc that has \semloc as its descendent. For our example, this index would materialize into the statement number in the block statement containing @foo@, which turns out to be $13$. Next, we instantiate the \eastree in Line~\ref{lst:line:eastree} which hierarchically defines the children-nodes or reference-nodes. The \eastree \I{O} deserializes into @if (REF1.hasOwnProperty(REF2)) { REF3 } @ where @REF1@, @REF2@, and @REF3@ correspond to \T{ReferenceAST} operators with locations as \reflocone, \refloctwo, and \reflocthree. \reflocone and \refloctwo use the semantic-parent edge traversals from \semloc (Line~\ref{lst:line:goodref}) and materialize into @handlers@ and @data.id@. \reflocthree performs a syntactic-child edge traversal from \editloc and materializes into @foo(data);@ thus materializing the entire \eastree \I{O} into @if (handlers.hasOwnProperty(data.id)) { foo(data); }@, i.e. the required repair. 

Now consider the repair strategy \strategytwo in Figure~\ref{fig:strat2}. This strategy shares a similar structure with the earlier strategy but differs in the way traversals and the index $\I{I}$ are computed. There are four key differences
\begin{enumerate}
    \item In order to reach \semloc from \prog.source, \strategytwo performs the \T{EdgeTraversal} using semantic-child edge seven times in Line~\ref{lst:line:nosemkleene}. The number of semantic edges varies widely across programs and prevents generalization to other scenarios. \T{KleeneStarTraversal} operator instead uses \newtextsc{Clauses} over nodes to find the edit-location.
    \item To reach $L_e$ from \semloc, \strategytwo performs the \T{EdgeTraversal} using syntactic-parent edge seven times in Line~\ref{lst:line:nosynkleene}. Consider a program that instead assigns output of the function-call @let out = foo(data)@. \strategytwo will find \assignexpr as the edit-location and fail to generalize whereas \strategyone will appropriately adjust and take four parent steps.
    \item In order to compute the index at which replacement needs to occur, \strategytwo uses a \DMethod{ConstantIndex}{13}\ in Line~\ref{lst:line:consteditindex} of Figure~\ref{fig:strat2}, which effectively assumes that replacement should always occur at 13$^{th}$ child of $L_e$ and again doesn't generalize. \strategyone on the other hand uses of \T{GetOffsetIndex} operator to instead compute index dynamically for a given input program
    \item In order to materialize reference nodes, \strategytwo uses syntactic edge traversals (Line~\ref{lst:line:badref} of Figure~\ref{fig:strat2}) which assume definite structure about the structure of the program (@GetConstant(7)@ used as syntactic child index to solve a long-ranged-dependency). \strategyone instead uses semantic-parent edges to capture the semantics here and produces a better generalizing repair.
\end{enumerate} 

\lstDeleteShortInline@

\noindent These programs highlight that our \dsl is expressive enough to perform complicated non-local repairs in a generic manner. At the same time, while many strategies can repair a given program, all applicable strategies are not equally good. A key realization is that we \emph{prefer shorter traversal functions} (\newtextsc{KleeneStarTraversal}\ over a long sequence of \newtextsc{EdgeTraversal}). Similarly, we \textit{prefer the traversals with none or small constants}. For example, we prefer \DMethod{GetOffsetIndex}{\semloc}{0}\  over \DMethod{GetConstant}{13}\ and semantic-parent traversal over syntactic-parent traversal with index \DMethod{GetConstant}{7}. %Finally, we also \emph{prefer strategies that share traversals across localizing \editloc and \refloc}.
We use these insights to guide the search in our synthesis algorithm.

% The strategy (\strategy) is defined by 
% performs this localization using an edit path (\editpath). We define a path (\genpath) in the strategy as a sequence of edges in the \prog. An edge is either a syntactic \astree edge or a semantic \taintpropedge in either direction (i.e. towards parent or child). 
% The localized node in the \prog is called edit location (\editloc). Next, the strategy either inserts or replaces a child of the edit location with a new \astree. This new \astree can either be a constant node or reference a node in the original \prog using a reference path (\rfpath). Figure ~\ref{fig:approach-notations} summarizes the notations 

% This \dsl was created so we can use the \sa annotations seamlessly and is guided by how humans fix such vulnerabilities. A line of previous works~\cite{} manually write repair patterns for fixing code. Our \dsl-based approach is strictly more general as it can perform various kinds of repairs and the exact repair strategies are learned from data. Moreover, we make effective use of high-level patterns and domain insights, and annotations. So while these other approaches tend to be simplistic and \textbf{either do not generalize well or over-generalize (generate a large number of false positives)}, concrete instantiations of strategies in our \dsl are better at capturing the high-level repair intent better. Following we describe the terminologies used in the repair \dsl.

% \lstMakeShortInline[columns=fixed]@
% The top level rule in our \dsl defines the Repair Strategy (denoted by \strategy). It is parameterized by the type of vulnerability the strategy fixes and the edit \edit. We consider two kinds of edits, either an insert operation or a replace operation. This means that the edit \edit either inserts an \astree child or replaces an \astree child with another \astree. Since we are fixing taint-flow vulnerabilities, we found these two operations to be sufficient. However, our \dsl can be expanded to also handle delete operations \aksays{Why can't we say that delete is replacement with an empty tree?}. In Figure ~\ref{fig:vulnerabilty-example1}, the fix applied in replaces the \astree corresponding to @handlers[callerId](data)@ (line ~\ref{lst:line:callerId-sink}, Figure ~\ref{fig:unsafememberex}) with the if statement in lines ~\ref{lst:line:fix-start}-\ref{lst:line:fix-end}, (Figure ~\ref{fig:safememberex}) and depicts a replace edit. 

% %\paragraph{Edit (\edit)} Since we are solving source-sink-sanitizer vulnerabilities, our \edit either inserts child \astree at edit-locations (denoted by \editloc) or replaces a child with another \astree (at \editloc). %This \editloc is a node in the \pdg which is reachable from the vulnerability source (as provided by the \astree) by traversing syntactic (\astree) or semantic edges in \pdg. %Once the \editloc is found, the new \astree (either replacing the existing child being inserted as a child) can be   

% Notice that in the \pdg, @handlers[callerId](data)@ is a child of the \blockstmt (parenthesis block between lines ~\ref{lst:line:handlers-run}-\ref{lst:line:handlers-run-end} and marked in blue in Figure ~\ref{fig:example1-pdg}). So while applying the fix, we replace the \textit{$k^{th}$} child of \blockstmt with the \ifstmt. We call the node in the \pdg where the edit operation applies as the edit location (\editloc). When a strategy applies, it has to determine this edit location based on the \pdg structure. Our \dsl defines an edit path (denoted by \editpath) to find edit location. In Figure ~\ref{fig:example1-pdg}, starting from the source-node @event@ (marked in orange), we take 7 semantic edges (reaching @callerId@) and then after hopping four synactic parent edges we reach the edit location \aksays{The notion of semantic edges should be defined and explained before this.}. This sequence of edge traversal defines our edit path. More generally, our \dsl considers the \editpath to be a set of semantic edges followed by a set of syntactic edges. This constraint on the paths allows expressivity to learn general strategies while also keeping the search space small. The semantic edges in \editpath allows navigating to ``somewhere close'' to sink location. Next once semantic edges are traversed, \editloc is reached by traversing a set of syntactic edges. Additionally, since the number of semantic edges might vary across examples, our \dsl allows a powerful  operator that navigates an indefinite number of semantic edges. This formulation helps our strategies to generalize well across widely different sets of programs. 

% %\paragraph{Edit Location (\editloc)} Edit location is the node in the \pdg where the edit operation (i.e. insertion or replacement of a child node) applies. \editloc is reachable from the vulnerability source found by traversing the edit-path (\editpath) in the \pdg.  So, in our running example, "$\dots$ function (data){$\dots$}" node (marked in blue) is the edit root and is reachable from the source via first traversing the semantic edges followed by traversing to "syntactic parent" four times. 
% Our kleene-star operator navigates indefinite semantic child edges until a ``stopping node'' (parameterized by a stopping condition) is reached. This stopping-condition is defined by a set of predicates applied on a \aksays{an} \astree node. We find that simplistic predicates about \newtextsc{ASTType} or \newtextsc{ASTValue} of \astree node and its neighbours suffice in locating this stopping node. For our running example, the stopping condition is the conjunction of the predicates @ASTType(node.parent) = IndexExpr@, @ASTType(node.parent.parent) = MethodCallExpr@. The stopping node lies on the taint-flow path from source to sink and therefore is quite relevant to the insert or replace operations (being a proxy for the semantic information of the vulnerability). Therefore, we call the stopping node as the semantic location (\semloc). In our running example, @calledId@, the sink-node is also the \semloc.

% As described above, our \dsl either inserts a new \astree, or it replaces an existing \astree with another \astree. An \astree is defined by three properties -- type, value and an array of children \astree. One can construct such an \astree by concretely initializing it using specific types and values for the tree and its descendants. However, a constant \astree cannot generalize because the fix depends on existing variables in the source code. Therefore, in addition to a constant \astree, our \dsl also allows referring to any existing node in the \astree. This referral is computed by traversing a path from the semantic location (\semloc defined above) to the node to reference \aksays{What is node to reference?}. The corresponding path is known as reference path. For e.g. the condition @handers.hasOwnProperty(callerId)@ which is used in the fix refers to @handlers@ and @callerId@ nodes in the tree and combines them in a constant \callexpr \astree. 

% We described edit paths and reference paths above. More generally a path is a sequence of edges in the \pdg where the edge can be one of syntactic or semantic or ancestral. When selecting a child edge, we also need to store \textit{which child} to select and it is determined by an index. Figure ~\ref{fig:repair-strategy-ex1} shows the entire strategy that fixes the example in Figure ~\ref{fig:vulnerabilty-example1}.
% \lstDeleteShortInline@

% \paragraph{Semantic Location (\semloc)} The node at which Kleene-Star traversal of semantic edges stops is called semantic location. This node is a key component in the repair since this node is a proxy for the semantically important values that would be necessary for making the edit. The sink node, callerId, is also the \semloc in our example. %Additionally, the \editloc is near this node and 

% \paragraph{Paths (\dslpath)} A path is described as a sequence of edges in the \pdg. The edges can be syntactic parent or child edges, semantic child edges, and ancestor edges in the \pdg.

% \paragraph{Index (\dslindex)}
% While selecting a child edge or while determining where the \concinsertcode needs to be inserted or replaced with, we need some index of which child to follow. Generally, this index is a constant value however it can be computed as an offset from the \semloc descendent direction as computed with the \newtextsc{OffsetFrom} operator. \naman{explain the requirement of offset with example}

%\paragraph{\astree} \astree is the tree-representation of the editcode that will replace some existing child of \editloc or will be inserted at some child indices of \editloc. One possible way to construct this \astree is to concretely initialize it using specific values and types. However, a constant \astree cannot generalize because the inserted code almost always depends on specific variables and the structures in the code. Therefore, in addition to a constant structure, the \astree can also refer to existing elements in the \pdg. This referral is again found using a path (\dslpath) traversal from the \semloc, the intuition being that it is a good \textit{proxy for the semantics of the vulnerability} and necessary variables to refer would be close to it.  


\subsection{Synthesizing \dsl strategies from examples}
\label{subsec:synthesis}
%\naman{The discussion about anti-unification would go in related work I presume?}

Given this high-level \dsl, we will now describe our example-based synthesis algorithm. 
We take as input a set of unsafe programs and edits generated as output at the end of data collection step (Section~\ref{sec:data}). 
Let $\{(\prog_{1},\edit_1),(\prog_{2},\edit_2),\dots$ $,(\prog_{n},\edit_n)\}$. 
Here, $\prog_{i}$ is the $i^{th}$ unsafe program and $\edit_i$ is the corresponding edit. Edit ($\edit$) contains the \astree-node of the edit-location ($\edit$.loc), the \textit{concrete} \astree of the edit-program ($\edit$.editprog), and the type of edit i.e. \insertsc or \replace ($\edit$.type). We use these to learn high-level repair strategies in our \dsl. 
\lstMakeShortInline[columns=fixed]@
Our goal is to combine specific paths, learned over examples that share similar repairs in different semantic and syntactic contexts, to obtain general strategies. Our repair strategies abstractly learn the following:
% \begin{enumerate}
%     \item the traversal for localizing edit-locations (\editloc) and reference-locations (\refloc). 
%     \item template-repair-program-representations using the reference-traversals . 
% \end{enumerate}
\begin{enumerate}
    \item Traversals for localizing edit-locations (\editloc) and reference-locations (\refloc). For example, @Ls@ in Line~\ref{lst:line:semkleene} (Strategy \strategyone) depicts a \T{KleeneTraversal} abstraction we can learn from examples having a variable number of semantic-edges. Similarly, @I@ in Line~\ref{lst:line:offseteditindex} (of \strategyone) depicts a generalized index expression we can learn from examples.
    \item \eastree which use reference-traversals. For example, @O@ in Line~\ref{lst:line:goodstratO} demonstrates templated-program-structure that we can learn from examples (say by generalizing from the witnessed guards @handlers.has(data)@  and @events.storage.has(event.name)@).
\end{enumerate}
\lstDeleteShortInline@


% In particular, we wish to abstract over examples that share similar repairs in different semantic or syntactic contexts. Consider the example abstractions below: 
% \begin{enumerate}
%     \item  
%     \item Line~\ref{lst:line:eastree} in Figure~\ref{fig:strat1} depicts 
%     the guard condition in Figure~\ref{fig:safememberex} @handlers.hasOwnProperty(data.id)@ can be abstracted with another guard @eventHandlers._storage.hasOwnProperty(event.name)@ into an abstract template @REF1.hasOwnProperty(REF2)@ where @REF1@ and @REF2@ are \T{ReferenceAST} have use traversals
% \end{enumerate}
% For example, . 
%Consider the example in Figures~\ref{fig:static-witnessing-1},~\ref{fig:static-witnessing-2}, and ~\ref{fig:static-witnessing-3}. They describe three different kinds of syntactic repairs and are not candidates to merge. Instead, 

We depict our synthesis algorithm in Figure~\ref{fig:strategy-learning}. At a high-level, our synthesis algorithm, first pre-processes the inputs, storing the required \textit{concrete} traversals. Next, it performs ranked pair-wise merging over the processed edits to synthesize strategies.
%We merge non-terminals recursively by deductively choosing production rules to enumerate and merging the non-terminals appearing in the productions. %During this recursion, it learns the \textit{traversals} and program templates abstractly. 

\noindent \textbf{Pre-processing.} In this step, given the programs and edits, we store the concrete traversals required for learning \editloc and \refloc (Line~\ref{algo:line:preprocess}). Naively computing all such traversals is very expensive and also leads to \textit{bad strategies}. Here, based on the insights from Section~\ref{subsec:examples}, we only compute the traversals that lead to shorter  
traversals
%\textit{abstract traversals}
which generalize better. In addition, we also share traversals between between \editloc and \refloc. Pre-processing has following three key steps:
\begin{enumerate}
    \item \textbf{Edit Traversals.} We compute the traversals between \prog.source and \editloc (Line~\ref{algo:line:conceditloc} of Figure~\ref{fig:strategy-learning}) that have the form of a sequence of semantic-edges followed by a sequence of syntactic-edges. This allows abstracting variable-length sequences of semantic-edge traversals as a \kleeneedge (corresponding to an abstract \newtextsc{KleeneTraversal}). We implement this using \newtextsc{BiDirecBFS} method at Line ~\ref{algo:line:bidirecbfs}. For every edit-traversal ($\I{T}_e$), we define {\em semantic-location} (\semloc for brevity) as the last-node on the semantic (dataflow) traversal before traversing the syntactic-edges.
    \item \textbf{Compressing Edit Traversals.} We compress these edit-traversals using the \newtextsc{Compress} method in Line~\ref{algo:line:compress}. It takes in a sequence of (syntactic or semantic) edges as input, greedily combines the consecutive edges with the same edge-type (\edgetype) into a \kleeneedge. The \kleeneedge is constructed using the edge type \edgetype, and a set of clauses $\clause_i$ that satisfy the target node of \kleeneedge. These clauses are either $\lambda n. \mathcal{T}(n) = \tau$ that check the type  or $\lambda n. \mathcal{T}(F_i(n)) = \nu$ that check the type of a neighbor. \newtextsc{Compress} returns a sequence of edges or \kleeneedges as output. 
    \item \textbf{Reference Traversals.} For every node of the edit-program, we locate nodes in the \pdg with the same \textit{value} using a \newtextsc{LevelOrderBFS} until a max-depth (Line~\ref{algo:line:maxlevel}). We perform this traversal from \semloc (defined in (1) above). We thus share parts of traversals between locating \editloc and \refloc which helps in learning \textit{better strategies}. The motivation behind using \semloc is that the expressions necessary for repair will be close to \semloc as it lies on the information-flow path. 
\end{enumerate}
%Specifically, for \editloc, we find the traversals between \prog.source and \editloc (Line~\ref{algo:line:conceditloc}) that first navigate a set of semantic-edges followed by a set of syntactic-edges. We implement this using the \newtextsc{BiDirecBFS} function at Line ~\ref{algo:line:bidirecbfs}. %It traverses semantic-edges from the source, syntactic-edges from the edit-location, and returns the intersecting traversals. 
%For every edit-traversal ($\I{T}_e$), we define semantic-location (\semloc for brevity) as the last-node on the semantic (dataflow) traversal before navigating a syntactic-edge. 
%Next, we compress the traversals greedy by combining consecutive edges of the same edge-type (\edgetype) into a \kleeneedge using the \newtextsc{Compress} method in Line~\ref{algo:line:compress}. Every \kleeneedge stores the \edgetype, and a set of clauses $\clause_i : i \in {1,\dots,n}$ that satisfy the end-node of \kleeneedge. These clauses are either $\lambda n. \mathcal{T}(n) = \nu$, i.e. a clause on the type of \semloc or $\lambda n. \mathcal{T}(F_i(n)) = \nu$, i.e. a clause on the type of a syntactic-neighbour of \semloc. We then compute traversals for finding reference locations. Here, instead of computing traversals from the source-node, we instead compute traversals from the semantic-locations. The expressions referenced in repairs are usually close to the \semloc (as it lies on the information-flow path and thus is affiliated with variables likely necessary for building the repair). This traversal-sharing optimizes the search and generalization of our strategies.

\noindent \textbf{Strategy Synthesis.} Given the edits and the associated traversal meta-data, we synthesize the strategy by pair-wise merging  (Line~\ref{algo:line:callmerge}). \newtextsc{MergeEdits}, the top-level synthesis method, takes a pair of edits as inputs and returns a list of strategies satisfying the example edits. We synthesize the strategies recursively using a deductive search over the non-terminals of the DSL (Figure~\ref{fig:fixing-dsl}). Specifically, to synthesize an expression corresponding to a non-terminal, we deduce which production to use and recursively synthesize the non-terminals given by its production-rule. This has the following key components: 
\begin{enumerate}
    \item \newtextsc{MergeEdits}: It takes pairs of edits as inputs and returns the strategy. It recursively synthesizes the traversal (for \editloc), index, and \eastree. It combines and returns them using the edit-type. 
    \item \newtextsc{MergeTraversal}: It takes two concrete traversals (sequence of edges or \kleeneedges) as inputs and returns the abstracted traversal. by merging elements in the sequence.
    \item \newtextsc{MergeEdge}: It takes two edges or \kleeneedges as inputs and returns a \T{GetKleeneTraversal} or \T{GetEdgeTraversal}. We combine two \kleeneedges using their edge-types and intersecting the clauses stored during pre-processing. We combine two edges using their edge-types, and recursively combining their indices.
    \item \newtextsc{MergeIndex}: It takes two integer indices as inputs and returns an abstracted index. If the two input indices are equal, we return a \T{GetConstant} operator with the input index value. Otherwise, we compute offset as the difference between input-index and index of child of $n$ which has \semloc as descendent (computed by $DO(n, \semloc)$). We return this offset if they are equal and an empty-list otherwise. 
    \item \newtextsc{MergeProg}: It takes two programs as input and returns a list of \eastree, where each list element can materialize into the input programs. If the top-level node in the programs have equal values and types, we combine them as a \T{ConstantAST}. Otherwise, we recursively combine their children. Finally, we merge the reference-traversals corresponding to the input programs and combine them into \T{ReferenceAST}.
\end{enumerate}

Our synthesis procedure is inspired by anti-unification~\cite{anti-unification} and we abstract the paths and edit-programs across different examples. Specifically, our \T{KleeneTraversal} and \T{OffsetIndex} functions allow generalization across paths having different number of edges and different indices where naive abstractions fail. Similarly, \eastree also resemble anti-unification over tree-edits. However, again we use traversals over syntactic and longer-context semantic-edges, for better generalizations  and repairs. 

Finally, note that while we perform pair-wise merges over the edits, the strategy synthesis algorithm can be extended to merge bigger cluster of edits together as well. However, from our experience, we find that the pair-wise merging performs well and is sufficient for our experiments. 
%by recursively synthesizing values corresponding to the non-terminals in our \dsl. 
%Thus, to synthesize a strategy, we synthesize traversals for the \editloc. To synthesize a traversal, we check if the two traversals contain an equal number of edges. Next, we try to merge the corresponding pairs of edges on each traversal. To merge an edge, if it is simply a syntactic or semantic edge, we instantiate a \T{GetEdge} operator appropriately. However, if the two edges are \kleeneedge, then we instantiate a \T{GetKleeneEdge} operator where the clauses are constructed by intersecting the clauses computed in \newtextsc{Compress} step. It additionally ensures that the number of clauses after the intersection is more than $1$ to prevent over-generalizing strategies. In order to merge strategies, we next merge the edit-programs (\edit.editprog). We first combine the programs as a \T{ConstantAST} by checking whether the type and value match and then recursively merge the children and finally take the cartesian product of children. Next, based on the concrete reference traversals, we merge them and construct \T{ReferenceAST} \eastree. 



%Note that in order to learn repair strategies, we need to synthesize paths corresponding to all locations (\location) used in the strategy. We use \location in three production rules in our \dsl (\locationone, \locationtwo, \locationthree in Rules~\ref{dslrule:strategy},~\ref{dslrule:index},~\ref{dslrule:node}). Naively, trying to synthesize these paths is very expensive and will lead to non-generalizing strategies.
%Here, based on common fix-patterns for these vulnerabilities, we reduce the search space by enforcing structure over the paths we learn. Additionally, we also share paths between the three locations. 

%We find that performing these 

%At a high-level, our synthesis algorithm takes these unsafe-programs and edits as inputs, preprocesses them to store relevant meta-data, clusters them and then recursively enumerates over the non-terminals in the \dsl. How


% \aksays{This section is very dense. It would be good to take a small example and illustrate the key steps visually.}
\input{synthesisalgo2}

% Given this high-level \dsl, we will now describe our synthesis algorithm. We build a top-down synthesis algorithm that learns strategies in this \dsl through a \pbe approach. We receive a set of unsafe codes and concrete edits \unsure{the terminology of concrete edits can be confusing for the paper. Basically, anologous to all things in \dsl, we have concrete edits, paths, etc.}. Let $\{(\prog_{1},\concedit_1),(\prog_{2},\concedit_2),\dots$ $,(\prog_{n},\concedit_n)\}$ be the data we collect from our data collection step where $\prog_{i}$ is the ith unsafe code snippets and $\concedit_i$ is the corresponding concrete edit. $\concedit_i$ contains the concrete edit-location $\conceditloc_i$, the \astree of the editcode $\concinsertcode_i$, and the type of edit i.e. \insertsc or \replace.

% Given this data, we instantiate our algorithm to learn repair strategies. Our algorithm takes in these set of examples and learns a set of ($\{\strategy_1,\strategy_1,\dots,\strategy_k\}$) that are supposed to cover the training examples. Later, these repair strategies, when given an unsafe program \pdg will generate the edit that needs to be applied. The sketch of our synthesis and learning algorithm is presented in Figure~\ref{fig:strategy-learning}.

% %%O := \{\(\strategy\sb{1},\strategy\sb{2},\dots,\strategy\sb{k}\)\}
\HUGE DONT USE THIS - USE synthesisalgo2
% Figure environment removed

% \unsure{\textbf{Terminology for reference!!:} Letters with overlines are concrete elements ($\concedit$ is a concrete edit, $\conceditpath$ is a concrete edit path) while the letters without lines are abstract elements that can generalize over examples ($\edit$ is an abstract edit, $\editpath$ is an abstract edit path). Following, we again define the various abbreviations used in the algorithm
% \begin{itemize}
%     \item \prog is the \pdg containing \sa annotations
%     \item \concedit is the concrete edit which itself contains editcode, edit-location, edit-type, and indices
%     \item \concinsertcode is the concrete editcode that is either inserted or replaces some existing region in the unsafe code. \concinsertcode itself can be represented as an \astree
%     \item \conceditloc is the concrete edit location i.e. where the edit takes place in a \prog
%     \item \concpath is a concrete path as a sequence of edges
%     \item \conceditpath is a concrete edit path from source to \conceditloc
%     \item semLoc or semantic location is the stopping node of \conceditpath
%     \item \concrefpath is a concrete reference path from semantic location to a particular node matching value of a \astree node in \concinsertcode
%     \item edit-type refers to whether edit is \insertsc or \replace
% \end{itemize}}

% \spsays{Consider breaking these into subsections and have a running example. for instance, first section can be running bidirectional bfs, other could be pairwise merging, and so on..} 

% Our top-level \newtextsc{Learn} method receives the programs and concrete edits as inputs (Line~\ref{algo:line:learn}). This method first invokes the \newtextsc{PreProcessConcEdit} method which computes edit-paths and reference-paths in the $\concedit$. Next, \newtextsc{Learn} method ranks edits based on the similarity of their $\concinsertcode$ and in that order tries to combine edits together in a pairwise manner using the \newtextsc{MergeEdit} method.

% \newtextsc{PreProcessConcEdit} method (in Line~\ref{algo:line:preprocess}) stores the relevant paths in the $\concedit$ structure that will be useful during pairwise-merging. It first computes a set of edit paths from source to sink using a bi-directional breadth-first search (\newtextsc{BiDirectBFS}) (Line~\ref{algo:line:conceditloc}) and stores it in the edit. Note that during this \newtextsc{BiDirectBFS}, it traverses only semantic edges from the source and only syntactic edges from the edit location. This naturally leads to paths that follow the required pattern of a set of semantic edges followed by a set of syntactic edges. Note that edit-paths also contain the semantic-locations in the semLoc field. Then for every edit-node in the edit code and every semantic location in the edit path, it stores reference paths from semantic locations to nodes in the \astree having the same value as edit-node (Line~\ref{algo:line:concrefpath}. These paths are computed using a \newtextsc{MaxLevelBFS} until a certain depth and storing the satisfying nodes. Note that during implementation, we memoize the path-finding steps to avoid repeating computations.

% \newtextsc{PairSimilar} method computes a score for every pair of edits in the edit set. To compute the similarity for a given pair of edits it performs \newtextsc{ASTSimilarity} on their editcodes. 

% \newtextsc{MergeEdit} is the top-level method of our deductive top-down synthesis algorithm. The merging procedure is intuitive. For every element in the edit, it recursively calls \newtextsc{Merge} operation on the elements and then assembles an edit using their outputs. Specifically, this method first ensures that edits are of the same type (\insertsc or \replace. Then it obtains a set of candidate edit-paths by calling the \newtextsc{MergeEditPaths} method on the concrete edit paths stored in the input concrete edits. Next for every candidate edit-path, it finds a candidate editcode using the \newtextsc{MergeEditCode} method. Finally, for every editpath and editcode pair, it assembles the final edit (Line~\ref{algo:line:assembleedit})
    
% \newtextsc{MergeEditPath} method tries to merge two concrete editpaths. It first computes a set of intersecting predicates over the semantic-locations of the two paths. Using the predicates, it builds the \semkleene edge. Finally, over the remaining set of edges in the edit-path, it calls the \newtextsc{MergePath} method which inturn merges all the edges successively (Line~\ref{algo:line:mergepath})


% \subsection{Applying the learned strategies}
% \label{subsec:applying}
% Given an unsafe-program and a set of repair strategies, we apply each strategy to the program to generate candidate repairs. 
% To apply a single repair strategy, we use the definitions of operators described in Section~\ref{subsec:dsl} to generate candidate repair programs. We find that, in practice, we obtain a few distinct repairs and we return them as the output of our system. 

%Once these high level repair strategies are learnt, applying them is natural. For an unsafe program, strategy \strategy ingests the \pdg of the program. Then it tries to build an edit, by first locating the edit location (using the edit path \editpath) and building the \astree recursively depending on whether it is a constant or reference tree. If edit location and \astree are generated, the edit operation is applied appropriately based on whether it is an \insertsc or \replace edit. Otherwise, if either of location or \astree is not generated then no edit is applied. 
% \subsection{Scribblings}
% Each node $n$ of the AST has an identifier $\mathit{id}\in\mathbb{N}$. The AST is characterized by a set $\mathcal{N}$ of node $\mathit{id}$s, i.e., $\mathcal{N}=\{\mathit{id}_0,\ldots\mathit{id}_k\}$. We have a map $\mathcal{T}$ from nodes to their types, i.e., $\mathcal{T}(n)=\tau$, where the primitive types $\tau$ include {\sc MethodCallExpr}, {\sc IndexExpr}, etc. We also have a set $\mathcal{E}$ of edges, where each edge is $(n_1,n_2,ET,z)$. Here, $n_1$ is a source node, $n_2$ is a target node, $ET$ is the type of edge (syntactic parent, syntactic child, semantic parent, or semantic child), and $z$ is a child's index (set to $-1$ if the edge is a parent edge). 

% The strategy $S$ is of two types, insert an AST $O$ at index $I$ of location $L$, $\mathit{Insert}(L(\mathit{source}),I,O)$, and replace the AST at index $I$ of location $L$ with $O$, $\mathit{Replace}(L(\mathit{source}),I,O)$. Each $L(n)$ takes a node $n$ and traverses a path to reach a location, i.e.,  $L(n)=\mathit{ApplyPath}(n,F_k\circ\ldots\circ F_0)$, where the output node is $F_k(F_{k-1}(\ldots F_0(n)\ldots)$. Each $F$ instantiates the edge traversal function $TE[I,ET,C]$ with an index $I$, an edge type $ET$, and an optional clause $C$ (relevant for KleeneEdge). We define $TE[I,ET](n)$ as $let\ i=I(n)\ in\ let\ N=\mathcal{E}(n,ET)\ in\ N[i]$, which gets an integer index $i$ of children of $n$, gets a set $N$ of nodes by dereferencing edges of type $ET$ from $n$ and returns the $i^{th}$ child of $n$.
% $F$ can also be $TE[ET,C](n)\equiv KE(n,ET,C)$.
%  The KleeneEdge $KE$ keeps dereferencing edges of type $ET$ till it hits a node where a clause $C$ holds, i.e., $KE(n,ET,C)$ is defined as $C(n)? n : \left(let\ t=\mathcal{E}(n,ET)\ in\ KE(t,ET,C)\right)$. The node $t$ which is target of an edge with type $ET$ and $n$ as a source here is chosen non-deterministically and our implementation resolves this non-determinism through a breadth-first search. A clause is a conjunction of predicates of the form $\lambda n. \mathcal{T}(L(n))=\tau$. The index $I$ is either an integer $z$ or of the form $\lambda n.DO(n,L(\mathit{source}))+z$, where $DO(n_1,n_2)$ returns the index of syntactic child of $n_2$ who is a syntactic ancestor of $n_1$. 
 
%  A strategy can fail to apply if in $DO(n_1,n_2)$ there is no path from $n_2$ to $n_1$,

\section{Experimental Results}\label{sec:results}
    \subsection{General Results}
        The basic ResSAN model is used to determine reference results which our expanded model can be compared to as it is structurally similar to ResLAN but does not possess the Lidar adaptive components of it. Further, we compare with the full-size PackNet-SAN and the unmodified NLSPN architecture. 
        As it can be seen from Tab.\,\ref{tab:sota-results}, our LiDAR-adaptive ResLAN achieves competitive performance compared to state-of-the-art standard depth completion methods, which are specialized to the unfiltered 64-beam-LiDAR. The performance differences are in the range of a few centimetres in terms of MAE, which is acceptable given the practical advantage that ResLAN can generalize to different beam patterns as will be shown below.

        Furthermore, we compared the architectures for a set of three different input types that contained 64, 32 or 16 LiDAR channels using both filter types on the metrics from the KITTI benchmark. The NLSPN model was trained for the standard depth completion task and then evaluated with different input data. As for the ResSAN models, we trained one model for each input type and tested it for the corresponding one which serve serve as the \emph{Baseline} in Tab.\,\ref{tab:overall-results}. Our ResLAN model was jointly trained for all three settings. As listed in Tab.\,\ref{tab:overall-results}, the ResLAN models outperform the challenging baseline in all metrics for FOV filtering and all but one for sparse filtering. This implies that our LiDAR adaptive model is able to outperform dedicated models in case of very sparse input depth. Fig.\,\ref{fig:comp-plot} shows this is indeed the case for 32 and even more for 16 channels. For FOV-filtered inputs with 16 channels, the ResLAN exhibits approx. $10\%$ smaller MAE than the baseline. As for the NLSPN, it becomes apparent that it is not capable of generalizing to other input types since it shows clearly worse results. The difference is especially pronounced for the FOV filtering where on average more than every fourth predicted pixel is more than $25 \%$ deviating from the ground truth\,($\delta_{1.25}$). Therefore, using a weight-adapting network in combination with differently filtered input depths allows us to train models that outperform their non-adaptive counterparts.

        \begin{table}[]
            \centering
    	    \small
            \vspace{0.4cm}
            \caption{\textbf{Depth estimation result for standard depth completion} when the ResSAN model was only trained for 64 channels and the ResLAN model for multiple tasks. The PackNet-SAN and NLSPN models were trained with the setup that was also used for our model architecture.}
            \footnotesize
            \setlength{\tabcolsep}{5pt}
            \begin{tabular}{@{}lrrrrl@{}}
            \toprule
            \multicolumn{6}{c}{\textbf{Standard LiDAR Depth Completion}}                                                                                                                         \\ \midrule
            \multicolumn{1}{l|}{Method}          & RMSE $\downarrow$            & MAE  $\downarrow$            & iRMSE $\downarrow$             & iMAE $\downarrow$ & $\delta_{1.25}$ $\uparrow$ \\
            \multicolumn{1}{l|}{}                & \multicolumn{1}{l}{{[}mm{]}} & \multicolumn{1}{l}{{[}mm{]}} & \multicolumn{1}{l}{{[}1/km{]}} & {[}1/km{]}        &                            \\ \midrule
            \multicolumn{1}{l|}{PackNet-SAN}     &  914                            &  298                            &  2.78                              &  1.4                 &  99.65 \%                          \\
            \multicolumn{1}{l|}{NLSPN}           &  \textbf{889}                            &   \textbf{263}                           &  \textbf{2.62}                              &   \textbf{1.3}                &   \textbf{99.61} \%                         \\ \midrule
            \multicolumn{1}{l|}{ResSAN (Ours)}   & 948                             &  275                            &  2.75                              &    1.4               &   99.58 \%                         \\
            \multicolumn{1}{l|}{ResLAN (Ours)} &   969                           &  283                            &   2.83                             &   1.4                &  99.56 \%                          \\ \bottomrule
            \end{tabular}
            \vspace{0.2cm}
            \label{tab:sota-results}
        \end{table}

        \begin{table}[]
    	    \centering
    	    \small
    	    \caption{\textbf{Depth estimation results of the two baseline setups and the explicit and implicit ResSAN} when evaluated on a combination of 16, 32 and 64 channel depth inputs. Please note that Specialist Methods need to train three specialized networks, one for each of the three types of inputs while our method only uses one network.}
            \footnotesize
            \setlength{\tabcolsep}{4.8pt}
            \begin{tabular}{@{}lrrrrl@{}}
                \toprule
                \multicolumn{6}{c}{\textbf{Sparse Channel Filter}}                                                                                                                                  \\ \midrule
                \multicolumn{1}{l|}{Method}        & RMSE $\downarrow$            & MAE  $\downarrow$            & iRMSE $\downarrow$             & iMAE $\downarrow$ & $\delta_{1.25}$ $\uparrow$  \\
                \multicolumn{1}{l|}{}              & \multicolumn{1}{l}{{[}mm{]}} & \multicolumn{1}{l}{{[}mm{]}} & \multicolumn{1}{l}{{[}1/km{]}} & {[}1/km{]}        &                             \\ \midrule
                \multicolumn{1}{l|}{NLSPN}         &  1396                            &  437                            & 5.54                               &  2.2                 &  98.82 \%                           \\
                \multicolumn{1}{l|}{Baseline}      & \textbf{1207}                             &  381                            & 4.41                               &  1.8                 &  \textbf{99.37} \%                           \\
                \multicolumn{1}{l|}{ResLAN (Ours)} &  1215                            &  \textbf{378}                            &  \textbf{4.27}                              &  \textbf{1.7}                 &  99.31 \%                           \\ \toprule
                \multicolumn{6}{c}{\textbf{Field-of-View Filter}}                                                                                                                                   \\ \midrule
                \multicolumn{1}{l|}{Method}        & RMSE $\downarrow$            & MAE  $\downarrow$            & iRMSE $\downarrow$             & iMAE $\downarrow$ & $\delta_{1.25}$ $\uparrow$ \\
                \multicolumn{1}{l|}{}              & \multicolumn{1}{l}{{[}mm{]}} & \multicolumn{1}{l}{{[}mm{]}} & \multicolumn{1}{l}{{[}1/km{]}} & {[}1/km{]}        &                             \\ \midrule
                \multicolumn{1}{l|}{NLSPN}         &  2738                            &  1702                            & 12.3                              &  4.3                 &  74.69 \%                           \\
                \multicolumn{1}{l|}{Baseline}      &  1556                            &  525                            &  6.8                              &  3.0                 & 98.14 \%                            \\
                \multicolumn{1}{l|}{ResLAN (Ours)} &  \textbf{1548}                            &  \textbf{519}                            &  \textbf{6.44}                              &  \textbf{2.8}                 & \textbf{98.52 \%}                            \\ \bottomrule
            \end{tabular}
            \label{tab:overall-results}
        \end{table}

        
        
        % Figure environment removed
        
        % Figure environment removed

    \subsection{Filter Effects}
        Comparing the effect of the two different types of depth input filters on the model performance, it becomes apparent that FOV filtering is the more challenging task. In that setting, reducing LiDAR channels is more detrimental to the performance than sparse filtering as it creates regions where no depth information is available. Effectively, the model is forced to perform depth prediction in these regions. These effects are highlighted in the depth images in Fig.\,\ref{fig:dense-maps} where the effect of a 16-channel sparse depth filter and a 16-channel FOV can be compared.

    \subsection{Generalization Capabilities}
        We trained three models for both filter types eaach, so the combinations and number of filtered depth inputs they receive are different. This serves the purpose of testing the generalization capabilities of the ResLAN architecture as well as the robustness to different filter settings. After training, the models were evaluated for the depth input settings they were trained for, as well as for ones they weren't exposed to. Overall, ResLAN shows good generalization capabilities. As one can gather from Fig.\,\ref{fig:explicit-comp} and Fig.\,\ref{fig:implicit-comp}, the consequences of slightly varying sets of input depth settings are limited. The most considerable deviations can be seen when the model is tasked to extrapolate. For instance, the model $\{64, 32, 16\}$ shows a noticeably higher MAE for eight-channel depth inputs than the model that was trained for it. Similar behaviour can be seen for the FOV filtering case as well for the model $\{64, 48, 32\}$ when tasked to generalize for a 16-channel input. There is no such pronounced effect for generalization tasks that lie between two filter settings the model was trained for. At most, it can be observed that models that were trained for a smaller range of filter values perform slightly better than ones that have to cover a wider range. The number of filter settings used in a fixed range does not relevantly influence the model performance, as can be seen, when comparing the two models in Fig.\,\ref{fig:implicit-comp}, which are both trained for a range of 64 to 32 channels but one with three filter settings and the other one with five.
    
    % Figure environment removed
    
    
    % Figure environment removed

\section{Discussion}
\label{sec: discussion}
\kmsdelete{In this work} We study \kmsreplace{Fairness-Aware PAC learning}{Fair-ERM} in the malicious noise model, and  in some cases allow 
the learner to maintain optimal overall accuracy despite the signal in Group $B$ being almost entirely washed out.
%when we allow learners to use the
%$\PQ$ randomized expansion of the hypothesis class $\mathcal{H}$
In particular we show that different fairness constraints have fundamentally different behavior in the presence of Malicious Noise, in terms of the amount of accuracy loss that a given level of Malicious Noise could cause a fairness-constrained learner to incur. 
The key to achieving our results, which are more optimistic than those in \cite{lampert}, is allowing for improper learners using the (P,Q)-randomized expansions of the given class $\mathcal{H}$.
%We \kmsreplace{present a picture of the}{prove upper and lower bounds on}
%accuracy loss for a range of fairness notions, given \kmsreplace{this simple randomization step.}{learning over $\PQ$.
%In general our results indicate Fair-ERM (given learning over $\PQ$) is more robust than claimed in \cite{lampert}.
The type of smoothness we create by using $\PQ$ seems to be a natural property that is likely shared by many natural hypothesis classes.

Fairness notions are motivated as a response to learned disparities when there is \kmsdelete{data corruption or} systemic error affecting \kmsdelete{the data for}
one group. 
Fairness notions are supposed to mitigate this by ruling out classifiers that have worse performance on a sub-group. 
This can peg both classifiers at a lower level of performance \kmsdelete{(e.g that the lower subgroup)} in order to \emph{motivate} \cite{hardt16} improving the data collection or labelling process to obtain more reliable performance. 
%So in \kmsreplace{some}{a} sense, sensitivity of the fairness notion to poor sub-group performance caused by malicious noise is the \textit{point} of fairness constraints! 
However, it also desirable that fairness constraints perform gracefully when subject to Malicious Noise because fairness constraints will be used in contexts where the data is unreliable and noisy and this might not be known to the learner.
This tension, exposed by our work, motivates 
%a revisiting of fairness notions from first principles approach and trying to axiomatize the 
%desired properties of a fairness intervention a la cryptography and privacy. \footnote{Work in multi-calibration \cite{multicalib} is a viable direction for this problem but it is unclear how 
%that and related notions behave with unreliable data. }
on going work studying the sensitivity level of fairness constraints. 
%If we we are to take a view, if a classifier is deployed 


% \section*{Accessibility}
% \section*{Software and Data}

% Acknowledgements should only appear in the accepted version.
\subsection*{Acknowledgements}

\noindent
USA {\textendash} U.S. National Science Foundation-Office of Polar Programs,
U.S. National Science Foundation-Physics Division,
U.S. National Science Foundation-EPSCoR,
Wisconsin Alumni Research Foundation,
Center for High Throughput Computing (CHTC) at the University of Wisconsin{\textendash}Madison,
Open Science Grid (OSG),
Extreme Science and Engineering Discovery Environment (XSEDE),
Frontera computing project at the Texas Advanced Computing Center,
U.S. Department of Energy-National Energy Research Scientific Computing Center,
Particle astrophysics research computing center at the University of Maryland,
Institute for Cyber-Enabled Research at Michigan State University,
and Astroparticle physics computational facility at Marquette University;
Belgium {\textendash} Funds for Scientific Research (FRS-FNRS and FWO),
FWO Odysseus and Big Science programmes,
and Belgian Federal Science Policy Office (Belspo);
Germany {\textendash} Bundesministerium f{\"u}r Bildung und Forschung (BMBF),
Deutsche Forschungsgemeinschaft (DFG),
Helmholtz Alliance for Astroparticle Physics (HAP),
Initiative and Networking Fund of the Helmholtz Association,
Deutsches Elektronen Synchrotron (DESY),
and High Performance Computing cluster of the RWTH Aachen;
Sweden {\textendash} Swedish Research Council,
Swedish Polar Research Secretariat,
Swedish National Infrastructure for Computing (SNIC),
and Knut and Alice Wallenberg Foundation;
Australia {\textendash} Australian Research Council;
Canada {\textendash} Natural Sciences and Engineering Research Council of Canada,
Calcul Qu{\'e}bec, Compute Ontario, Canada Foundation for Innovation, WestGrid, and Compute Canada;
Denmark {\textendash} Villum Fonden and Carlsberg Foundation;
New Zealand {\textendash} Marsden Fund;
Japan {\textendash} Japan Society for Promotion of Science (JSPS)
and Institute for Global Prominent Research (IGPR) of Chiba University;
Korea {\textendash} National Research Foundation of Korea (NRF);
Switzerland {\textendash} Swiss National Science Foundation (SNSF);
United Kingdom {\textendash} Department of Physics, University of Oxford.

% In the unusual situation where you want a paper to appear in the
% references without citing it in the main text, use \nocite
% \nocite{langley00}

\balance
\bibliography{references}
\bibliographystyle{icml2023}

%%%%%%%%%%%%%%%%%%%%%%%%%%%%%%%%%%%%%%%%%%%%%%%%%%%%%%%%%%%%%%%%%%%%%%%%%%%%%%%
%%%%%%%%%%%%%%%%%%%%%%%%%%%%%%%%%%%%%%%%%%%%%%%%%%%%%%%%%%%%%%%%%%%%%%%%%%%%%%%
% APPENDIX
%%%%%%%%%%%%%%%%%%%%%%%%%%%%%%%%%%%%%%%%%%%%%%%%%%%%%%%%%%%%%%%%%%%%%%%%%%%%%%%
%%%%%%%%%%%%%%%%%%%%%%%%%%%%%%%%%%%%%%%%%%%%%%%%%%%%%%%%%%%%%%%%%%%%%%%%%%%%%%%

\appendix
\onecolumn

\begin{comment}
\section{System Architecture}
\label{appendix:architecture}
\system has a novel modularized system architecture with three key components: 
\emph{StreamManager}, 
\emph{TxnManager} and \emph{TxnScheduler}. 
These components are instantiated in each thread locally.
The execution outline of \system is presented in Algorithm~\ref{alg:algo}.
Transactional stream processing is continuous and potentially never ends (Line 1$\sim$8).
The dependency resolution and execution of state transactions are separated into two non-overlapping phases by punctuations~\cite{Tucker:2003:EPS:776752.776780} (Line 2 and 5), which guarantees that no subsequent input event will have a smaller timestamp. 
Effectively, a batch of state transactions is collected during the first phase, and processed during the second phase.

In the first phase (i.e., stream processing phase), 
the \emph{StreamManager} conducts preprocessing for every input event ($e$). Similar to some prior works~\cite{tstream}, state transactions may be issued but not immediately processed during preprocessing (Line 3).
The \emph{pre\_processing} and \emph{post\_processing} functions are exposed as APIs to users.
The \emph{TxnManager} handles dependency resolution (Line 4) among state transactions and insert decomposed operations to construct a \tpg. We discuss the detailed two-phase \tpg construction process in Section~\ref{subsec:construction}.

In the second phase  (i.e., transaction processing phase), 
the \emph{TxnManager} is first involved again to refine (Line 6) the constructed \tpg with further dependency resolution.
The \emph{TxnScheduler} 
schedules operations for concurrent execution based on the constructed \tpg according to the three dimensions of scheduling decisions (Line 7). 
In particular, a scheduling decision model $M$ is instantiated based on the constructed \tpg (Line 14).
\textbf{\circled{1}} Guided by $M$, execution threads adopt an exploration strategy (Section~\ref{subsec:explore}) to explore the constructed \tpg for operations available to be scheduled constrained by dependencies. 
\textbf{\circled{2}} 
During exploration, one or multiple operations may be treated as the 
% basic 
unit of scheduling (Section~\ref{subsec:granularity}). 
Subsequently, \textbf{\circled{3}} every thread executes operation(s) in the unit of scheduling with various abort handling mechanisms (Section~\ref{subsec:abort_handling}).
Only when state transactions are processed (i.e., committed or aborted) can the associated input events be postprocessed (Line 8) by the \emph{StreamManager} based on transaction processing results.
\end{comment}

\begin{comment}
\begin{algorithm}
\footnotesize
    \KwData{$e$ \tcp{Input event}}
    \KwData{$txn_{ts}$ \tcp{State transaction}}
    \KwData{$G$ \tcp{The currently constructed TPG}}
    \While{!finish processing of input streams}{
        \eIf(\tcp*[h]{Phase 1}){\text{$e$ is not a $punctuation$}}{
                $txn_{ts}$ $\gets$ PRE\_Processing($e$)\;
                \textbf{TPG\_Construction}($G$, $txn_{ts}$)\; 
          }(\tcp*[h]{Phase 2}){
                \textbf{TPG\_Refinement}($G$)\; 
                \textbf{TXN\_Scheduling}($G$)\; 
                POST\_Processing()\;
          }
    }
    
    \SetKwFunction{FMain}{TPG\_Construction}
    \SetKwProg{Fn}{Function}{:}{}
    \Fn{\FMain{$G$, $txn_{ts}$}}{
        $O_{1..k}$ $\gets$ \textbf{Partition} $txn_{ts}$\;
        \ForEach{\text{operation $O_{i}$ $\in$ $O_{1..k}$}}{
            \textbf{Identify} its \ld\;
            $G$ $\gets$ $G$ + $O_{i}$ \;
        }
    }
    \SetKwFunction{FMain}{TPG\_Refinement}
    \SetKwProg{Fn}{Function}{:}{}
    \Fn{\FMain{$G$}}{
        \ForEach{\text{vertex $e_{i}$ $\in$ $G$}}{
            \textbf{Identify} its \td, \pd\;
        }
    }
    
    \SetKwFunction{FMain}{TXN\_Scheduling}
    \SetKwProg{Fn}{Function}{:}{}
    \Fn{\FMain{$G$}}{
        $M$ $\gets$ Instantiated with $G$;\tcp{A decision model}
        \While{!finish scheduling of $G$
        }{
          \textbf{\circled{2}} $Scheduling Unit$ $\gets$ \textbf{\circled{1}} \emph{Explore}($G$, $M$)\; 
            \textbf{\circled{3}} \emph{Execute with Abort Handling} ($Scheduling Unit$)\; 
        }
    }
  \caption{Execution Outline of \system}
  \label{alg:algo}
\end{algorithm}
\end{comment}

%%%%%%%%%%%%%%%%%%%%%%%%%%%%%%%%%%%%%%%%%%%%%%%%%%%%%%%%%%%%%%%%%%%%%%%%%%%%%%%
%%%%%%%%%%%%%%%%%%%%%%%%%%%%%%%%%%%%%%%%%%%%%%%%%%%%%%%%%%%%%%%%%%%%%%%%%%%%%%%

\end{document}

% This document was modified from the file originally made available by
% Pat Langley and Andrea Danyluk for ICML-2K. This version was created
% by Iain Murray in 2018, and modified by Alexandre Bouchard in
% 2019 and 2021 and by Csaba Szepesvari, Gang Niu and Sivan Sabato in 2022. 
% Previous contributors include Dan Roy, Lise Getoor and Tobias
% Scheffer, which was slightly modified from the 2010 version by
% Thorsten Joachims & Johannes Fuernkranz, slightly modified from the
% 2009 version by Kiri Wagstaff and Sam Roweis's 2008 version, which is
% slightly modified from Prasad Tadepalli's 2007 version which is a
% lightly changed version of the previous year's version by Andrew
% Moore, which was in turn edited from those of Kristian Kersting and
% Codrina Lauth. Alex Smola contributed to the algorithmic style files.
