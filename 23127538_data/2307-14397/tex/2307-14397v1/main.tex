%%
%% This is file `sample-manuscript.tex',
%% generated with the docstrip utility.
%%
%% The original source files were:
%%
%% samples.dtx  (with options: `manuscript')
%% 
%% IMPORTANT NOTICE:
%% 
%% For the copyright see the source file.
%% 
%% Any modified versions of this file must be renamed
%% with new filenames distinct from sample-manuscript.tex.
%% 
%% For distribution of the original source see the terms
%% for copying and modification in the file samples.dtx.
%% 
%% This generated file may be distributed as long as the
%% original source files, as listed above, are part of the
%% same distribution. (The sources need not necessarily be
%% in the same archive or directory.)
%%
%%
%% Commands for TeXCount
%TC:macro \cite [option:text,text]
%TC:macro \citep [option:text,text]
%TC:macro \citet [option:text,text]
%TC:envir table 0 1
%TC:envir table* 0 1
%TC:envir tabular [ignore] word
%TC:envir displaymath 0 word
%TC:envir math 0 word
%TC:envir comment 0 0
%%
%%
%% The first command in your LaTeX source must be the \documentclass command.
%\documentclass[manuscript,screen,review]{acmart}

\PassOptionsToPackage{prologue,dvipsnames}{xcolor} 
%\documentclass[manuscript,preprint]{acmart}
%\documentclass[manuscript,screen,review]{acmart}
\documentclass[acmsmall]{acmart}

%\usepackage[a4paper, total={6in, 7in}]{geometry}




% \usepackage{forest} %x
%%
%% \BibTeX command to typeset the BibTeX logo in the docs
\AtBeginDocument{%
  \providecommand\BibTeX{{%
    \normalfont B\kern-0.5em{\scshape i\kern-0.25em b}\kern-0.8em\TeX}}}

%% Rights management information.  This information is sent to you
%% when you complete the rights form.  These commands have SAMPLE
%% values in them; it is your responsibility as an author to replace
%% the commands and values with those provided to you when you
%% complete the rights form.


\setcopyright{acmcopyright}
\copyrightyear{2023}
\acmYear{2023}
\acmDOI{na}


%% These commands are for a PROCEEDINGS abstract or paper.
%\acmJournal{}
%\acmVolume{37}
%\acmNumber{4}
%\acmArticle{111}
%\acmMonth{8}
% \usepackage[dvipsnames]{xcolor}

\DeclareMathOperator*{\argmax}{arg\,max}
\DeclareMathOperator*{\argmin}{arg\,min}

\newcommand{\st}{\textit{s.t. }}
\newcommand{\wrt}{\textit{w.r.t.} }
\newcommand{\ie}{\textit{i.e. }}
\newcommand{\eg}{\textit{e.g. }}
\newcommand{\etal}{\textit{et al.}}

\usepackage{amsmath} %
\usepackage{makecell}
% \usepackage{graphicx} %
\usepackage{tabularx} %
\usepackage{longtable} %
% \usepackage{afterpage} %x


\usepackage{enumitem}
\setlist[itemize]{leftmargin=*}
\setlist[enumerate]{leftmargin=*}


\usepackage{amsthm}

\theoremstyle{definition}
\newtheorem{definition}{Definition}%[section]

\theoremstyle{remark}
\newtheorem*{remark}{Remark}


\usepackage{adjustbox} %x
\usepackage{tabularx} % Alternative package to adjustbox} %
% \usepackage{amssymb}% http://ctan.org/pkg/amssymb
% \usepackage[dvipsnames]{xcolor}

% \usepackage{pifont}% http://ctan.org/pkg/pifont
\usepackage{bbding} % Alternative package to pifont
\usepackage{multirow}
% \usepackage{xcolor,colortbl} 
\usepackage{colortbl}
\usepackage{CJK}
\usepackage{caption} %
\usepackage{subcaption}
\newcommand*\roth{\rotatebox{90}} %To rotate headers
\NewDocumentCommand{\rot}{O{45} O{1em} m}{\makebox[#2][l]{\rotatebox{#1}{#3}}}%
% \newcommand{\cmark}{\ding{108}}%
% \newcommand{\xmark}{\ding{109}}%
% ====using bbding to replace pifont package
\newcommand{\cmark}{\scalebox{0.75}{\CircleSolid}}%
\newcommand{\xmark}{\scalebox{0.75}{\CircleShadow}}%
\newcommand{\cell}[2]{\begin{tabular}{c} #1 \\ \midrule \textit{E.g. } #2\end{tabular}} %Figure 
\newcolumntype{R}[1]{>{\RaggedLeft\arraybackslash}p{#1}}
\usepackage{setspace} %Need for spacings

\newcommand{\D}{\mathcal{D}} %Arbituary domain (when not discussing x-domain)
\newcommand{\Ds}{\mathcal{D}_s} %Source Domain
\newcommand{\Dt}{\mathcal{D}_t} %Target Domain 
\newcommand{\Cs}{C_{seen}} %Seen Class
\newcommand{\Cu}{C_{unseen}} %Unseen Class
\newcommand{\bz}{\mathbf{z}} %bold z
\newcommand{\bx}{\mathbf{x}} %bold z
\newcommand{\bepsilon}{\mathbf{\epsilon}} %bold z


%%
%% Submission ID.
%% Use this when submitting an article to a sponsored event. You'll
%% receive a unique submission ID from the organizers
%% of the event and this ID should be used as the parameter to this command.
%%\acmSubmissionID{123-A56-BU3}

%%
%% The majority of ACM publications use numbered citations and
%% references.  The command \citestyle{authoryear} switches to the
%% "author year" style.
%%
%% If you are preparing content for an event
%% sponsored by ACM SIGGRAPH, you must use the "author year" style of
%% citations and references.
%% Uncommenting
%% the next command will enable that style.
%%\citestyle{acmauthoryear}


%%
%% end of the preamble, start of the body of the document source.
\begin{document}




%%
%% The "title" command has an optional parameter,
%% allowing the author to define a "short title" to be used in page headers.


\title{A Survey on Generative Modeling with Limited Data, Few Shots, and Zero Shot}



%%
%% The "author" command and its associated commands are used to define
%% the authors and their affiliations.
%% Of note is the shared affiliation of the first two authors and the
%% "authornote" and "authornotemark" commands
%% used to denote shared contribution to the research.
\author{Milad Abdollahzadeh}
%\authornote{Both authors contributed equally to this research.}
%\email{ { milad_abdollahzadeh, touba_malekzadeh, christopher_teo, keshigeyan, guimeng_liu, ngaiman_cheung} @sutd.edu.sg }
\orcid{0000-0003-4011-4670}
%\author{G.K.M. Tobin}
%\authornotemark[1]
\author{Touba Malekzadeh}
\authornote{Equal Contribution}
\email{ 
{milad_abdollahzadeh, 
touba_malekzadeh} 
@sutd.edu.sg }
\orcid{0000-0002-0348-9935}
\author{Christopher T.H. Teo}
\authornotemark[1]
\email{christopher_teo@mymail.sutd.edu.sg}
\orcid{0009-0006-3801-2658}
\author{\\Keshigeyan Chandrasegaran}
\authornotemark[1]
\orcid{0000-0002-6965-5377}
\author{Guimeng Liu}
\orcid{0009-0005-3609-1412}
\author{Ngai-Man Cheung}
\authornote{Corresponding Author}
\email{ 
{ 
keshigeyan, 
guimeng_liu, 
ngaiman_cheung} 
@sutd.edu.sg }
\orcid{0000-0003-0135-3791}
\affiliation{%
  \institution{\\Singapore University of Technology and Design (SUTD)}
  \streetaddress{8 Somapah Rd}
%  \city{}
%  \state{}
  \country{Singapore}
  \postcode{487372}
}






%%
%% By default, the full list of authors will be used on the page
%% headers. Often, this list is too long and will overlap
%% other information printed in the page headers. This command allows
%% the author to define a more concise list
%% of authors' names for this purpose.
\renewcommand{\shortauthors}{Abdollahzadeh, et al.}

%%
%% The abstract is a short summary of the work to be presented in the
%% article.
\begin{abstract}


In machine learning, generative modeling aims to learn to generate new data statistically similar to the training data distribution.
In this paper, we survey learning generative models under limited data, few shots and zero shot,
referred to as {\bf Generative Modeling under Data Constraint (GM-DC)}.
This is an important topic when data acquisition is challenging, \eg healthcare applications.
We discuss 
background, challenges, and propose two taxonomies:
one on GM-DC tasks and another on GM-DC approaches.
Importantly, we study interactions between different GM-DC tasks and approaches. 
Furthermore, we highlight  research gaps, research trends, and potential avenues for future exploration. Project website: \url{https://gmdc-survey.github.io}.


\end{abstract}


%%
%% The code below is generated by the tool at http://dl.acm.org/ccs.cfm.
%% Please copy and paste the code instead of the example below.
%%
\begin{CCSXML}
<ccs2012>
   <concept>
       <concept_id>10010147.10010257.10010293.10010294</concept_id>
       <concept_desc>Computing methodologies~Neural networks</concept_desc>
       <concept_significance>500</concept_significance>
       </concept>
   <concept>
       <concept_id>10010147.10010178.10010224</concept_id>
       <concept_desc>Computing methodologies~Computer vision</concept_desc>
       <concept_significance>500</concept_significance>
       </concept>
 </ccs2012>
\end{CCSXML}

\ccsdesc[500]{Computing methodologies~Neural networks}
\ccsdesc[500]{Computing methodologies~Computer vision}

%%
%% Keywords. The author(s) should pick words that accurately describe
%% the work being presented. Separate the keywords with commas.
\keywords{Generative Modeling, Generative AI, Generative Adversarial Networks, Diffusion Models, Variational Auto-Encoder, Few-Shot Learning, Zero-Shot Learning, Transfer Learning, Data Augmentation.}



%%
%% This command processes the author and affiliation and title
%% information and builds the first part of the formatted document.
\maketitle


\footnotetext[1]{This research is supported by the National Research Foundation, Singapore under its AI Singapore Programmes (AISG Award No.: AISG2-RP-2021-021; AISG Award No.: AISG-100E2018-005).}


\section{Introduction}


Generative modeling is a field of machine learning that focuses on learning the underlying distribution of the training samples, enabling the generation of new samples that exhibit similar statistical properties to the training data. Generative modeling has profound impacts in various fields including computer vision \cite{ramesh2022dalle2, karras2020analyzing, brock2019biggan}, natural language 
processing \cite{yu2017seqgan, gulrajani2017improved, van2017neural} and data engineering 
\cite{antoniou2017dagan, karras2020ada, tran2021dag}.
Over the years, significant advancements have been made in generative modeling.
Innovative approaches such as Generative Adversarial Networks (GANs) \cite{goodfellow2014GANs,karras2019style,brock2019biggan,arjovsky2017wasserstein,choi2020starganv2,park2019GauGAN,zhu2017cyclegGAN}, Variational Autoencoders (VAEs) \cite{kingma2013VAE,vahdat2020nvae,van2017neural}, and Diffusion Models (DMs) \cite{rombach2022latentdiffusion,song2020denoising,dhariwal2021diffusionvsGAN,nichol2021improveddenoisingDIM} have played a pivotal role in enhancing the quality and diversity of generated samples. 
The advancements in generative modeling have fueled the recent disruption in generative AI, unlocking new possibilities in various applications such as image synthesis \cite{reuters-2023, clarke-2022}, text generation \cite{hern-2023, hopkin-2023}, music composition \cite{easton-2023, xu-2022},
genomics \cite{nguyen2023hyenadna},
and more \cite{sargent2023vq3d, koh2023generating}. The ability to generate realistic and diverse samples has opened doors to creative applications and 
% innovative
novel
solutions \cite{roose-2023, mit-technology-review-2023}.



% Figure SanKey -----------------------------------------
% Figure environment removed
%-----------------------------------------

Research
on generative modeling has been mainly focusing on setups with sizeable training datasets.
StyleGAN \cite{karras2019style} learns to generate  realistic and diverse face images using
Flickr-Faces-HQ (FFHQ), a high-quality dataset of 70k human face images collected from the photo-sharing website Flickr.
The more recent text-to-image generative model is trained on millions of 
 image-text pairs, e.g.
Latent Diffusion Model \cite{rombach2022latentdiffusion} is trained on
 LAION-400M with 400 million samples \cite{schuhmann2021laion400m}.
However, in many domains (\eg, medical), the collection of data samples is challenging and expensive.


{\bf In this paper}, we survey 
Generative Modeling under Data Constraint (GM-DC). 
This research area is important for many domains/ applications where challenges in data collection exist. We conduct a thorough literature review on learning generative models under limited data, few shots, and zero shot.  
{\em Our survey is the first to provide a comprehensive overview and detailed analysis of
all types of generative models, tasks, and approaches studied in GM-DC, offering an accessible guide on the research landscape}
(Fig.~\ref{fig:sankey}).
We cover the essential backgrounds, provide detailed analysis of unique challenges of GM-DC, discuss current trends, and present the latest advancements in GM-DC. 

{\bf Our Contributions:}
i) Trends, technical evolution, and statistics of GM-DC (Fig.~\ref{fig:works_statistics};
Fig.~\ref{fig:timeline};
Sec.~\ref{ssec:landscape_analysis}); 
ii) New insights on GM-DC challenges (Sec.~\ref{ssec:challenges});
iii) Two novel and detailed taxonomies, one on GM-DC tasks (Sec.~\ref{ssec:tasks}) and another on GM-DC approaches (Sec.~\ref{sec:comprehensive_review});
iv) A novel Sankey diagram to visualize the research landscape and relationship between GM-DC tasks, approaches, and methods (Fig.~\ref{fig:sankey});
v) An organized summary of individual GM-DC works (Sec.~\ref{sec:comprehensive_review});
vi) Discussion of future directions (Sec.~\ref{ssec:future_direction}).
We further provide a \href{https://gmdc-survey.github.io/}{project website} with an interactive diagram to visualize GM-DC landscape.
Our survey aims to be an accessible guide 
to provide fresh perspectives on the current research landscape, organized pointers to comprehensive literature, and insightful trends on the latest advances of GM-DC.



% Figure TimeLine -----------------------------------------

% Figure environment removed


% Figure environment removed

%-----------------------------------------




{\em Survey on GM-DC is inadequate, and our work aims to fill this gap.}
We found only one survey in arXiv on the early work of GM-DC focusing on some aspects of GM-DC \cite{li2022degan}.
This previous survey has focused on a subset of GM-DC papers, studying only  works with  
GANs as the generative model and a subset of technical tasks/ approaches.
Our survey differentiates itself from \cite{li2022degan} in: 
i) Scope  - Our survey is the first to cover all types of generative models and all GM-DC tasks and approaches (Fig. \ref{fig:works_statistics});
ii) Scale -
Our study includes 
113 papers and covers broad GM-DC works, 
while 
previous survey \cite{li2022degan}
covers only $\approx$27\% of works discussed in our survey (Fig. \ref{fig:gm-dc-publications}); 
iii) Timeliness - Our survey collects and surveys the most up-to-date papers in GM-DC; 
iv) Detailedness - Our paper includes detailed visualizations (Sankey diagram, charts) and tables to 
highlight 
interactions and
important attributes of GM-DC literature;
v) Technical evolution analysis - Our paper analyzes the evolution of GM-DC tasks and approaches, providing new perspectives on recent advances;
vi) Horizon analysis - Our paper discusses 
distinctive obstacles encountered in GM-DC and identifies  avenues for future research. 

The rest of the paper is organized as follows.
In Sec.~\ref{sec:background} we provide the necessary background.
In Sec.~\ref{sec:taxonomy}, we discuss GM-DC tasks and unique challenges.
In Sec.~\ref{sec:comprehensive_review}, we analyze GM-DC approaches and methods.
In Sec.~\ref{sec:discussion}, we discuss open research problems and future directions.
Sec.~\ref{sec:conclusion} concludes the survey.





\section{Background}
\label{sec:background}

In this section, we first define `domain' and `generative modeling', then we  discuss common approaches of generative modeling and  data constraints studied in 
GM-DC.


\noindent
{\bf Domain.} In this survey, a {\em domain} consists of two components: i) a sample space $\mathcal{X}$, and ii) a marginal probability distribution $P_{data}$, which models the probability of samples from $\mathcal{X}$ \cite{pan2009yang-qiang-transfer}.
This is written as $\mathcal{D}=\{\mathcal{X},P_{data} \}$,
and $x\sim P_{data} \in \mathcal{X}$ denoting a sample in this space. 
An example of a domain is the domain of image of human faces: $\mathcal{D}^{h}=\{\mathcal{X},P_{data}^{h} \}$.
Here $\mathcal{X}$ is the sample space of images, and $P_{data}^{h}$ is the probability distribution of human faces.
\vspace{0.1cm}


\noindent
{\bf Generative Modeling.}
Given a set of training sample $x$ of 
a domain 
$\mathcal{D}=\{\mathcal{X},P_{data} \}$, i.e., 
with an underlying probability distribution $P_{data}$, 
generative modeling  aims to learn to  capture $P_{data}$ ---sometimes also denoted as $P(x)$ in literature.
The result of generative modeling is a 
{\em generative model} $G$ encoding 
a probability distribution $P_{model}$.
The learning objective is to have $P_{model}$ similar to  $P_{data}$ statistically.
After the training, $G$ can generate samples following 
$P_{model}$.
For example, generative modeling with a training set of  human face images 
aims to learn to capture $P_{data}^{h}$, thereby the resulting $G^{h}$ can generate human face images
that are statistically similar to samples from $P_{data}^{h}$.
We also refer to the domain of training samples as {\em target domain}.


\vspace{0.1cm}
\noindent
{\bf Conditional vs Unconditional Sample Generation.}
After learning the underlying distribution of data $P_{data}$, the generative model can generate new samples by sampling from the learned distribution $P_{model}$.
Typically, generation starts with sampling a random vector $z$ ---also called latent code--- as input. Then, this input is passed into the generative model $G$ to transform the latent code into a new sample $G(z) \sim P_{model}$.
Ideally, a good generator is able to capture  the characteristics, quality and diversity of the training dataset, \ie, $P_{model}$ is similar to  $P_{data}$ statistically.
If an additional condition $c$ (like a class label or attribute) is used alongside with the latent code to steer the sample generation towards $c$, the sample generation is called conditional generation:  $G(z,c)$.



% Figure environment removed
%\vspace{-0.3cm}

\subsection{Approaches for Generative Modeling}
\label{ssec:generative_models}


Earlier works on generative models  
study
Gaussian Mixture Models \cite{reynolds2009gaussian}, Hidden Markov Models 
\cite{phung2005topic}, 
Latent Dirichlet Allocation 
\cite{chauhan2021topic}
and Boltzmann Machines \cite{ackley1985learning}.
With the introduction of deep neural networks, recent works study  powerful generative models,
particularly those for image generation, which most GM-DC works focus on.

\vspace{0.1cm}
\noindent
{\bf Variational Auto Encoders (VAE)} 
\cite{pouyanfar2018survey}.
VAE is a variant of Auto-Encoder (AE) \cite{zhai2018autoencoder}, where both consist of the encoder and decoder networks. AE focuses on dimensional reduction.
The encoder in AE learns to map an input $x$ into a latent (compressed) representation, $z=E(x)$.
Then, the decoder aims to reconstruct the image from that latent representation, $\hat{x}=D(z)$. 
Model parameters are optimized with the following reconstruction loss:
\begin{equation}
    \mathcal{L}_{rec}=||x-D(z)||_2
\end{equation}
AEs are notorious for latent space irregularity making them improper for sample generation \cite{kingma2019introduction}. VAE aims to address this problem by enforcing $E$ to return a normal distribution over latent space.
Assuming a distribution $z\sim \mathcal{N}(\mu,\sigma^2)$ for latent space, this is done by adding the KL-divergence term to the loss function:
\begin{equation}
    \mathcal{L}=||x-D(z)||_2 + KL(\mathcal{N}(\mu,\sigma^2), \mathcal{N}(0,I))
\end{equation} 
Due to the challenges of direct maximization of the likelihood in pixel space, Vector-Quantized VAE (VQ-VAE) 
proposes {\em tokenization} where a codebook $\mathbf{e}_k$, $k \in 1, \dots, K$ is used to quantize the embeddings $E(x)$ into visual tokens (indices), acting like a lookup table.
In addition, a latent prior of the visual tokens is predicted (usually using a transformer), and the decoder is modified to map the visual tokens into the image space.

\begin{table}[t]
    \centering
    \fontsize{7pt}{7pt}
    \selectfont
    \vspace{-0.4cm}
    \caption{List of common datasets used in GM-DC works. 
    Number of samples (\# Samples) refers to the sample size of the entire dataset. In GM-DC experiments, usually, only a subset of the dataset is used.
    We remark that \xmark/\cmark denotes the absence/presence of the dataset under the data constraint settings: {\bf LD}: \underline{L}imited-\underline{D}ata, {\bf FS}: \underline{F}ew-\underline{S}hot and {\bf ZS}: \underline{Z}ero-\underline{S}hot, and Labels indicate if training labels are available (but not necessarily used).
    }
    \label{tab:datasets}
    \resizebox{\textwidth}{!}{
    \begin{tabular}{cc c@{\hspace{3pt}}c @{\hspace{3pt}}c@{\hspace{3pt}}c@{\hspace{3pt}}c@{\hspace{3pt}}c}
    \toprule
        {\bf Dataset} &  {\bf Description} & 
        {\bf \# Samples} &
        {\bf Resolution} &
        {\bf LD}  & {\bf FS} & {\bf ZS}
        & {\bf Labels} \\
       \midrule
        \arrayrulecolor{black!10}
        %1--------------------------------------------------------------
        \makecell[c]{Flickr-Faces-HQ\\ (FFHQ) \cite{karras2019style}} & 
        \parbox[l]{0.45\linewidth}
        {Images with human faces, containing variation in terms of age, ethnicity, and image background.}
        & 70K
        & 1024$\times$1024
        &
        \cmark & \cmark & \cmark &        
        \xmark
        \\ \midrule
         %2--------------------------------------------------------------
        \makecell[c]{Large-scale Scene\\ Understanding (LSUN) \cite{yu2015lsun}  }       &
        \parbox[l]{0.45\linewidth}
          {
        Images with large-scale scene containing 10 scene and 20 object categories.
        }
        &
        3M
        &
        256$\times$256 &
        \cmark & \cmark & \xmark
        &
        \cmark
        \\ \midrule
        %3--------------------------------------------------------------
        MetFace \cite{karras2020ada}            & 
        \parbox[l]{0.45\linewidth}
        {Images depicting paintings, drawings, and statues of human faces 
        } 
        &
        1336 & 1024$\times$1024 &
        \cmark & \cmark & \xmark
        &
        \xmark
        \\ \midrule
        %4--------------------------------------------------------------
        BreCaHAD \cite{aksac2019brecahad}           &
        \parbox[l]{0.45\linewidth}
        {Images of breast cancer histopathology.
        }
        &
        162 & 1360$\times$1024 &
        \cmark & \xmark & \xmark
        &
        \xmark
        \\ \midrule
        %5--------------------------------------------------------------
        \makecell[c]{Animal FacesHQ\\(AFHQ) \cite{choi2020starganv2}} &
        \parbox[l]{0.45\linewidth}
        {Images of animal faces in the domains of cat, dog, and wildlife.}
        &
        15K & 512$\times$512 &
        \cmark & \cmark & \xmark &
        \xmark
        \\ \midrule
        %6--------------------------------------------------------------
        CIFAR-10 \cite{krizhevsky2009cifar100}           &
        \parbox[l]{0.45\linewidth}
        {
        Images including objects and animals.
        }
        &
        60K & 32$\times$32 &
        \cmark & \xmark & \xmark &
        \cmark
        \\ \midrule
        %7--------------------------------------------
        CIFAR-100 \cite{krizhevsky2009cifar100}            & 
        \parbox[l]{0.45\linewidth}{
        A dataset similar to CIFAR-10, but with 100 classes}
        &
        60K & 32$\times$32 &
        \cmark & \xmark & \xmark 
        &
        \cmark
        \\ \midrule
        %8--------------------------------------------
        \makecell[c]{100-shot Obama/\\Gumpy Cat/Panda \cite{zhao2020diffaug}}      &
        \parbox[l]{0.45\linewidth}{Colored images of Obama/Gumpy Cat/Panda}
        &
        100 & 256$\times$256 &
        \cmark & \xmark & \xmark 
        &
        \xmark
        \\\midrule
        %11--------------------------------------------
        Sketches \cite{wang2008faceSketches}            & 
        \parbox[l]{0.45\linewidth}
        {Face sketches in frontal pose, normal lighting, and neutral expressions}
        &
        606 & 256$\times$256 &
        \xmark & \cmark & \xmark 
        &
        \xmark
        \\\midrule
        %12--------------------------------------------
        Sunglasses \cite{ojha2021cdc}         &
        \parbox[l]{0.45\linewidth}
        {Images of human faces wearing sunglasses.
        }
        & 2700 & 256$\times$256 &
        \xmark & \cmark & \xmark 
        &
        \xmark
        \\\midrule
        %13--------------------------------------------
        Babies \cite{ojha2021cdc}             &
        \parbox[l]{0.45\linewidth}
        {Images of baby faces.}
        &
        2500 & 256$\times$256 &
        \xmark & \cmark & \xmark 
        &
        \xmark
        \\\midrule
        %14--------------------------------------------
        \makecell[l]{Artistic-Faces \cite{yaniv2019face}} &
        \parbox[l]{0.45\linewidth}
        {Images containing 160 artistic portraits of 16 different artists.}
        &
        160 & 256$\times$256  &
        \xmark & \cmark & \xmark 
        &
        \xmark
        \\\midrule
        %15--------------------------------------------
        Haunted houses \cite{yu2015lsun}       &
        \parbox[l]{0.45\linewidth}
        {Images of haunted houses}
        &
        1K & 256$\times$256 &
        \xmark & \cmark & \xmark 
        &
        \xmark
        \\\midrule
        %16--------------------------------------------
        Wrecked cars \cite{yu2015lsun}       &
        \parbox[l]{0.45\linewidth}
        {Images of wrecked cars}
        &
        1K & 256$\times$256 &
        \xmark & \cmark & \xmark 
        &
        \xmark
        \\
        \arrayrulecolor{black!100}
        \bottomrule
    \end{tabular}
    }
    \vspace{-10pt}
\end{table}




\begin{table}[!htbp]
    \centering
%\small
\fontsize{7pt}{7pt}
\selectfont
\caption{{\bf Our proposed taxonomy for tasks in GM-DC}. For each task, we extract their key characteristics.  
[Attributes] {\bf C}: \underline{C}onditional generation, {\bf P}: \underline{P}re-trained generator given, {\bf I}: \underline{I}mages (as input), {\bf TP}: \underline{T}ext-\underline{P}rompt (as input), {\bf X}: \underline{X}(Cross)-domain adaptation; [Data Constraint] {\bf LD}: \underline{L}imited-\underline{D}ata, {\bf FS}: \underline{F}ew-\underline{S}hot, {\bf ZS}: \underline{Z}ero-\underline{S}hot.
\xmark/\cmark denotes the absence/presence, respectively.
Best viewed in color.
}
\label{tab:tasktaxonomy}
\resizebox{\linewidth}{!}{
\begin{tabular}{
c
c@{\hspace{15pt}}c@{\hspace{15pt}}c@{\hspace{15pt}}c@{\hspace{15pt}}c 
@{\hspace{25pt}}
c@{\hspace{15pt}}c@{\hspace{15pt}}c
c}
\toprule 
\makecell[c]{\bf Task} & \multicolumn{5}{c}{\bf Attributes} & \multicolumn{3}{c}{\bf Data Constraint} & \makecell[c]{\bf Task Illustration}\\
\cmidrule(lr){1-1}\cmidrule(lr){2-6}\cmidrule(lr){7-9}\cmidrule(lr){10-10}
%Headers
 %Attributes (Names)
     &{\bf C}
     &{\bf P}
     &{\bf I}
     &{\bf TP}
     &{\bf X}
 %Context
    &{\bf LD}
    &{\bf FS}
    &{\bf ZS}
 & \\
    \midrule
%Task 1
    %Row 1-----------------------------
        %Task Name
        \multirow{3}{*}{uGM-1} 
        &
        %Attributes (row 1)
        \xmark & \xmark & \cmark & \xmark & \xmark &
        \cmark &\xmark & \xmark
        &
        %Example (Fig)
        \multirow{3}{*}{
        \includegraphics[width=0.48\textwidth, clip, 
        trim={88pt 865pt 1099pt 60pt},]
        {figures/Tasks_v8_1.pdf}
        }
    \\
    \cmidrule(lr){2-9}
    \vspace{3pt}
    %Row 2-----------------------------
    & 
    %Description
    \multicolumn{8}{p{6cm}}{
    \textbf{Description:} Given $K$ samples from a domain $\D$,
    learn to generate diverse and  high-quality samples
    from $\D$} 
    \\
    %Row 3-----------------------------
    &
    %Examole
    \multicolumn{8}{p{6cm}}{
    \textbf{Example:} ADA \cite{karras2020ada} learns a StyleGAN2 using 1k
    images from AFHQ-Dog}  
    \vspace{3pt}
    \\
    \midrule
%Task 2
    %Row 1-----------------------------
        %Task Name
        \multirow{3}{*}{uGM-2}
        &
        %Attributes (row 1)
        \xmark & \cmark & \cmark & \xmark &\cmark &
        \cmark & \cmark & \xmark &
        %Example (Fig)]
    %Row 2-----------------------------
        \multirow{3}{*}{
        \includegraphics[width=0.48\textwidth, clip, 
        trim={88pt 535pt 1099pt 338pt},]
        {figures/Tasks_v8_1.pdf}
        }
    \vspace{1pt}
    \\
    \cmidrule(lr){2-9}
    \vspace{1pt}
    %Row 2-----------------------------
    & 
    %Description
    \multicolumn{8}{p{6cm}}{
    \textbf{Description:} Given a pre-trained generator on a 
    source domain $\Ds$ and $K$ samples from a target
    domain $\Dt$, learn to generate diverse and
    high-quality samples from $\Dt$
    }
    \\
    %\cmidrule(lr){2-9}
    %Row 3-----------------------------
    &
    %Example
    \multicolumn{8}{p{6cm}}{
    \textbf{Example:}
    CDC \cite{ojha2021cdc} adapts a pre-trained 
    GAN on FFHQ to Sketches using 10 samples}
    \vspace{2pt}
    \\
    \midrule
%Task 3
    %Row 1-----------------------------
        %Task Name
        \multirow{3}{*}{uGM-3} 
        &
        %Attributes (row 1)
        \xmark & \cmark & \xmark & \cmark & \cmark &
        \xmark & \xmark & \cmark &
        %Example (Fig)
        \multirow{3}{*}{
        \includegraphics[width=0.48\textwidth, clip, 
        trim={88pt 113pt 1099pt 698pt},]
        {figures/Tasks_v8_1.pdf}
        }
    \\
    \cmidrule(lr){2-9}
    \vspace{1pt}
    %Row 2-----------------------------
    & 
    %Description
    \multicolumn{8}{p{6cm}}{
    \textbf{Description:}  Given a pre-trained 
    generator on a 
    source domain $\Ds$
    and a text prompt 
    describing a target 
    domain $\Dt$, learn 
    to generate diverse 
    and high-quality 
    samples from $\Dt$
    }
    \vspace{2pt}
    \\
    %\cmidrule(lr){2-9}
    %Row 3-----------------------------
    &
    %Example
    \multicolumn{8}{p{6cm}}{
    %\textit{E.g.,} 
    \textbf{Example:}
    NADA \cite{gal2022stylegannada}
    adapts pre-trained GAN on 
    FFHQ to the 
    painting domain 
    using {\em `Fernando 
    Botero
    Painting'} as input}
    \\
    \midrule
%Task 4
    %Row 1-----------------------------
        %Task Name
        \multirow{3}{*}{cGM-1} 
        &
        %Attributes (row 1)
        \cmark & \xmark & \cmark & \xmark & \xmark
        & \cmark & \xmark & \xmark &
        %Example (Fig)
        \multirow{3}{*}{
        \includegraphics[width=0.48\textwidth, clip, 
        trim={88pt 834pt 1099pt 68pt},]
        {figures/Tasks_v8_2}
        }
    \\
    \cmidrule(lr){2-9}
    \vspace{1pt}
    %Row 2-----------------------------
    & 
    %Description
    \multicolumn{8}{p{6cm}}{
    \textbf{Description:} 
    Given $K$ samples 
    with class labels  
    from a domain $\D$,
    learn to generate 
    diverse and 
    high-quality 
    samples 
    conditioning 
    on the class 
    labels from $\D$
    }
    \\
    %Row 3-----------------------------
    &
    %Example
    \multicolumn{8}{p{6cm}}{
    \textbf{Example:} CbC \cite{shahbazi2022collapse} trains conditional generator on 20 classes of ImageNet Carnivores using 100 images per class
    }
    \vspace{2pt}
    \\
    \midrule
%Task 5
    %Row 1-----------------------------
        %Task Name
        \multirow{3}{*}{cGM-2} 
        &
        %Attributes (row 1)
        \cmark & \cmark & \cmark & \xmark & \xmark
      & \xmark & \cmark & \xmark &
        %Example (Fig)
        \multirow{3}{*}{
        \includegraphics[width=0.48\textwidth, clip, 
        trim={88pt 492pt 1099pt 342pt},]
        {figures/Tasks_v8_2}
        }
    \\
    \cmidrule(lr){2-9}
    \vspace{3pt}
    %Row 2-----------------------------
    & 
    %Description
    \multicolumn{8}{p{6cm}}{
    \textbf{Description:} 
    Given a pre-trained 
    generator on the
    seen classes $\Cs$
    of a domain $\D$ and
    $K$ samples with class
    labels from unseen
    classes $\Cu$ of
    $\D$, learn to generate
    diverse and
    high-quality samples
    conditioning on
    the class labels
    for $\Cu$ from $\D$
    }
    \\
    %\cmidrule(lr){2-9}
    %Row 3-----------------------------
    &
    %Example
    \multicolumn{8}{p{6cm}}{
    %\textit{E.g.,} 
    \textbf{Example:} LoftGAN \cite{gu2021lofgan} 
    learns from 85 classes of Flowers to generate 
    images for an unseen class with only 3 samples
    }
    \vspace{3pt}
    \\
    \midrule
%Task 6
    %Row 1-----------------------------
        %Task Name
        \multirow{3}{*}{cGM-3} 
        &
        %Attributes (row 1)
         \cmark & \cmark & \cmark & \xmark & \cmark &
    \cmark & \cmark & \xmark &
        %Example (Fig)
        \multirow{3}{*}{
        \includegraphics[width=0.48\textwidth, clip, 
        trim={88pt 113pt 1099pt 698pt},]
        {figures/Tasks_v8_2.pdf}
        }
    \vspace{2pt}
    \\
    \cmidrule(lr){2-9}
    \vspace{3pt}
    %Row 2-----------------------------
    & 
    %Description
    \multicolumn{8}{p{6cm}}{
    \textbf{Description:} 
    Given a pre-trained 
    generator on a 
    source domain $\Ds$
    and $K$ samples 
    with class labels 
    from a target
    domain $\Dt$ , learn 
    to generate diverse 
    and high-quality 
    samples conditioning
    on the class 
    labels from $\Dt$
    }
    \\
    %\cmidrule(lr){2-9}
    %Row 3-----------------------------
    &
    %Example
    \multicolumn{8}{p{6cm}}{
    %\textit{E.g.,} 
    \textbf{Example:}
     VPT \cite{sohn2023vpt} adapts
    a pre-trained 
    conditional 
    generator on
    ImageNet 
    to Places365 
    with 500 images per class
    }
    \vspace{3pt}
    \\
    \midrule
%Task 7
    %Row 1-----------------------------
        %Task Name
        \multirow{3}{*}{IGM} 
        &
        %Attributes (row 1)
         \xmark & \xmark & \cmark & \xmark & \xmark &
    \xmark & \cmark & \xmark &
        %Example (Fig)
        \multirow{3}{*}{
        \raisebox{-1.0cm}{
        \includegraphics[width=0.48\textwidth, clip, 
        trim={88pt 849pt 1099pt 20pt},]
        {figures/Tasks_v8_3.pdf}
        }
        }
    \\
    \cmidrule(lr){2-9}
    %Row 2-----------------------------
    & 
    %Description
    \multicolumn{8}{p{6cm}}{
    \textbf{Description:} 
     Given $K$ samples 
    (usually $K=1$) 
    and assuming rich internal
    distribution for
    patches within
    these samples,
    learn to generate
    diverse and
    high-quality
    samples with
    the same internal
    patch distribution
    }
    \\
    %\cmidrule(lr){2-9}
    %Row 3-----------------------------
    &
    %Example
    \multicolumn{8}{p{6cm}}{
    %\textit{E.g.,} 
    \textbf{Example:}
    SinDDM \cite{kulikov2023sinddm}
    trains a 
    generator using 
    a single image 
    of
    Marina Bay Sands, and
    generates
    %high-quality
    variants of it
    }
    \\
    \midrule
    %\pagebreak
%Task 8
    %Row 1-----------------------------
        %Task Name
        \multirow{3}{*}{SGM} 
        &
        %Attributes (row 1)
        \xmark & \cmark & \cmark & \cmark & \xmark
        & \xmark & \cmark & \xmark &
        %Example (Fig)
        \multirow{3}{*}{
        \includegraphics[width=0.48\textwidth, clip, 
        trim={90pt 440pt 1099pt 365pt},]
        {figures/Tasks_v8_3.pdf}
        }
        \vspace{3pt}
    \\
    \cmidrule(lr){2-9}
    %Row 2-----------------------------
    & 
    %Description
    \multicolumn{8}{p{6cm}}{
    \textbf{Description:} 
    Given a pre-trained generator, $K$ samples
    of a particular
    subject, and a 
    text prompt,
    learn to generate
    diverse and
    high-quality
    samples containing
    the same subject
    }
    \\
    %\cmidrule(lr){2-9}
    %Row 3-----------------------------
    &
    %Example
    \multicolumn{8}{p{6cm}}{
    %\textit{E.g.,} 
    \textbf{Example:}
    DreamBooth \cite{ruiz2023dreambooth}
    trains a  generator using 4 images of
    a particular
    backpack and
    adapts it with
    a text-prompt
    to be in the {\em `grand canyon'}
    }
    \\
    \midrule
\end{tabular}
}
\end{table}









% GANs
\vspace{0.1cm}
\noindent
{\bf Generative Adversarial Models (GAN)} \cite{saxena2021generative,jabbar2021survey}.
GAN applies an adversarial approach to learn the distribution of data $P_{data}$.
It consists of a generator $G$ and a discriminator $D$ playing a min-max game. 
Specifically, given the latent code $z$, the $G$ learns to generate the images $G(z)$, $z\sim P_z$, where $P_z$ is usually a Gaussian distribution.
Then, $D$ learns to distinguish the real images $x \sim P_{data}$ from the generated ones $G(z) \sim P_{model}$.
The $D$ and $G$ are optimized by respectively maximizing and minimizing the following value function:
\begin{equation}
    \mathcal{V}(D,G)= \mathbb{E}_{x\sim p_{data}} [ \log D(x)] + \mathbb{E}_{z\sim p_z} [\log (1-D(G(z)))] 
\end{equation}


% Normalizing Flow
\vspace{0.1cm}
\noindent
{\bf Flow-based Models} \cite{ho2019flow++}.
The flow-based model includes  a series of invertible yet differentiable functions $f$, between latent distribution $P_z$, and data distribution $P_{data}$.
The following log-likelihood function is maximized to train $f(.|\theta)$:
\begin{equation}
    \max_\theta \textstyle \sum_{i=1}^K \log P_{z} (f(x^{(i)}|\theta))+ \log |\det Df(x^{(i)}|\theta)|
\end{equation}
For ease of discussion, we simplify the model as a single flow and denote the training samples with $\{x^{(i)}\}^K_{i=1}$, and the Jacobian of $f(x)$ as $Df(x)$.
We remark that, unlike VAEs that estimate the lower bounds of the log-likelihood, flow-based models evaluate the exact log-likelihood in their loss function.

\vspace{0.1cm}
\noindent
{\bf Diffusion Models (DM)} \cite{kingma2021variational}. 
DM leverages the concept of the diffusion process from stochastic calculus and consists of forward diffusion and reverse diffusion processes.
In the forward diffusion process, based on the foundations of Markov chains, the noise $\epsilon \sim \mathcal{N}(0,I)$ is iteratively added to data samples until it approaches an isotropic Gaussian distribution.
Then, in the backward process, the DM learns to denoise the noisy vector $x_T$ and reconstruct the data samples $x_0$.
This is done by learning the noise estimation model $\epsilon_\theta$ with minimizing the following loss function \cite{ho2020denoising}:
\begin{equation}
    \mathcal{L}=\mathbb{E}_{t,x_0,\epsilon} [||\epsilon - \epsilon_{\theta}(\sqrt{\bar{\alpha}_t}x_0 + \sqrt{1-\bar{\alpha}_t}\epsilon, t)||_2]
    \label{eqn:DMLoss}
\end{equation}
Then, during the generation process, DM first samples a noise $x_T\sim \mathcal{N}(0,I)$, 
and utilizes the learned noise function $\epsilon_\theta$ to iteratively apply the following denoising process \cite{ho2020denoising}:
\begin{equation}
    x_{t-1}=\frac{1}{\sqrt{\alpha_t}}(x_t-\frac{1-\alpha_t}{\sqrt{1-\bar{\alpha}}}\epsilon_\theta(x_t,t)) +\sqrt{\beta_t}\epsilon, \quad t\in[0,T]
    \label{eqn:DMRec}
\end{equation}
Here, $x_t$ is the generated sample at step $T-t$, $\beta_t$ is variance scheduler, $\alpha_t=1-\beta_t$ and
$\bar{\alpha_t}=\prod^t_{s=1}\alpha_s$.

\vspace{0.1cm}
\noindent
{\bf Remark.} We remark that among discussed models, only GANs, DMs, and VAEs are adopted in the context of GM-DC.









\subsection{Data Constraints and Commonly Used Datasets}


In GM-DC, three data constraints have been considered in most works:
(i) \emph{Limited data (LD)}, 
when 50 to 5,000 training samples are given;
(ii) \emph{Few-Shot (FS)},
when 1 to 50 training samples are given;
(iii) \emph{Zero-Shot (ZS)},
when no training samples are given.
Training under these data constraints often results in various problems \eg over-fitting. We remark that these ranges are the typically used values as there are no fixed definitions in the literature.
Tab.~\ref{tab:datasets} lists the most common datasets used in GM-DC with related details.







% For aligning Sec 3.1 item (2)
\newenvironment{myquote}
  {\begin{list}{}{%
     \setlength{\leftmargin}{1.6em}%   adjust this to change the left margin
     \setlength{\rightmargin}{0pt}% adjust this for the right margin, 0 keeps it at default
     }
   \item\relax}
  {\end{list}}

\section{Generative Modeling under Data Constraint: Task Taxonomy, Challenges}
\label{sec:taxonomy}
In this section, first, we present our proposed taxonomy on different GM-DC tasks (Sec.~\ref{ssec:tasks}) highlighting their relationships and differences based on their attributes, e.g. unconditional or conditional  generation. 
Then, we present the unique challenges of GM-DC (Sec.~\ref{ssec:challenges}), including new insights such as domain proximity, and incompatible knowledge transfer.
Later, in Sec.~\ref{sec:comprehensive_review}, we present our proposed taxonomy on approaches for GM-DC, with a detailed review of individual work organized under our proposed taxonomy.



\subsection{Generative Modeling under Data Constraint: A Taxonomy on Tasks}
\label{ssec:tasks}


The goal of GM-DC is to learn to generate diverse and high-quality samples given only a small number of training samples. A number of GM-DC setups have been studied in different works
(Fig.~\ref{fig:timeline}). In this section, we propose a {\bf GM-DC task taxonomy} to categorize  setups in different works.  Tab. \ref{tab:tasktaxonomy} tabulates our GM-DC task taxonomy.

\begin{enumerate}

\item {\bf Unconditional generative modeling under data constraint (uGM-1).}
\vspace{-0.3em}
\begin{definition}[uGM-1]
{\em Given $K$ samples from domain $\D$, learn to generate diverse and high-quality samples from $\D$.}
\end{definition}
\vspace{-0.3em}
Without leveraging other side information, existing work has studied uGM-1 under limited samples ranging from 100 to several thousands. uGM-1 is an important task, especially for a domain that is distant from common domains, e.g. medical images which are distant from common personal photos in terms of content and characteristics.
In such scenarios, leveraging from common domains would not  provide any advantage. 

\item {\bf Unconditional generative modeling under data constraint with pre-trained generator  and cross-domain adaptation} (uGM-2).
\vspace{-0.3em}
\begin{definition}[uGM-2]
{\em Given a pre-trained generator on a source domain $\Ds$  (with numerous and diverse samples) and $K$ samples from a target domain $\Dt$, learn to generate diverse and high-quality samples from $\Dt$.}
\end{definition}
\vspace{-0.3em}
uGM-2 is similar to uGM-1, except that 
a pre-trained generator on another source domain $\Ds$ is additionally given.
uGM-2 is a major task in GM-DC and has been studied in many works.
In most works, close proximity in semantic 
between $\Ds$ and $\Dt$ is assumed, \eg $\Ds$ is photos of human faces, $\Dt$ is sketches of human faces.
For uGM-2, 
transfer learning  has been a popular approach to tackle this task driving GM-DC into the few-shot regime, \eg only 10 samples from $\Dt$ are given \cite{li2020ewc} (See Sec.~\ref{sec:comprehensive_review}
for the taxonomy of GM-DC approaches).
Recent work has started to look into the challenging setup when $\Ds$ and $\Dt$ are more semantically apart \cite{zhao2022adam}, \eg $\Ds$ is photos of human faces, $\Dt$ is photos of cat faces.  See Sec.~\ref{ssec:challenges} for further discussion on domain proximity in GM-DC.

\item {\bf Unconditional generative modeling under data constraint with pre-trained generator  and cross-domain adaptation, using text prompt (uGM-3).}
\vspace{-0.3em}
\begin{definition}[uGM-3]
{\em Given a pre-trained generator on a source domain $\Ds$  (with numerous and diverse samples) and a text prompt describing  a target domain $\Dt$, learn to generate diverse and high-quality samples from $\Dt$.}
\end{definition}
\vspace{-0.3em}
uGM-3 is similar to uGM-2, except that a text prompt is provided to describe $\Dt$ instead of samples from $\Dt$.
Particularly, this task requires generating samples  from $\Dt$ without seeing any sample from that domain, \ie 
 zero-shot domain adaptation.
Important work to tackle this task leverages recent large vision-language models to provide textual direction to guide the adaptation of the pre-trained generator to $\Dt$ \cite{gal2022stylegannada}.

\item {\bf Conditional generative modeling under data constraint (cGM-1).}
\vspace{-0.3em}
\begin{definition}[cGM-1]
{\em Given $K$ samples with class labels from a domain $\D$, learn to generate diverse and high-quality samples  conditioning on the class labels  from $\D$.}
\end{definition}
\vspace{-0.3em}
cGM-1 is similar to uGM-1 but focuses on conditional generation, \ie inputs to the generator include a random latent vector and a class label.
Conditional generative models such as BigGAN \cite{brock2019biggan} could achieve high-quality image generation 
when they are trained on large-scale datasets \eg ImageNet. However, under limited data, 
it is 
challenging to 
 achieve diverse and high-quality conditional sample generation.
As a natural extension of uGM-1, 
data augmentation has been studied for 
cGM-1 among other approaches, see Sec.~\ref{sec:comprehensive_review}.
 
\item {\bf Conditional generative modeling under data constraint with pre-trained generator (cGM-2).}
\vspace{-0.3em}
\begin{definition}[cGM-2]
{\em Given a pre-trained generator on the seen classes $\Cs$ of a domain $\D$, and
$K$ samples with class labels from unseen classes $\Cu$ of $\D$, learn to generate diverse and high-quality samples conditioning on the class labels for $\Cu$ from $\D$.}
\end{definition}
\vspace{-0.3em}
cGM-2 is similar to cGM-1, except that a pre-trained generator on the seen classes $\Cs$ is additionally given.
Note that in cGM-2, $\Cs$ and $\Cu$  contain disjoint classes, but both of them are from the same domain $\D$.
For example, \cite{shahbazi2021efficient} 
studies the setup when CIFAR100
\cite{krizhevsky2009cifar100}
is partitioned into 80 seen classes for the pre-trained generator and 20 unseen classes as the target, 
with 100 samples per unseen class given for training.
Meta-learning and transfer learning (regularizer-based fine-tuning, etc.) have been effective approaches for cGM-2, see Sec.~\ref{sec:comprehensive_review}.

\item {\bf Conditional Generative Modeling under data constraint with pre-trained generator and cross-domain adaptation (cGM-3).} 
\vspace{-0.3em}
\begin{definition}[cGM-3]
{\em Given a pre-trained generator on a source domain $\Ds$ (with numerous and diverse samples) and $K$ samples with class labels from a target domain $\Dt$, learn to generate diverse and high-quality samples conditioning on the class labels from $\Dt$.}
\end{definition}
\vspace{-0.3em}
\end{enumerate}

%--------------------------------------------------------------------
%                    Table 3: Our proposed taxonomy
%--------------------------------------------------------------------
\clearpage 
\begin{spacing}{0.88}
\fontsize{7pt}{7pt}
\selectfont
\setlength\tabcolsep{2pt}
\begin{longtable}{cc}
\caption{
{\bf Our proposed taxonomy for approaches in GM-DC.}
For each  approach,
the addressed GM-DC tasks (see Tab. \ref{tab:tasktaxonomy} for task definitions) and the data constraints 
are indicated. 
A detailed list of 
methods under 
each sub-category is also tabulated (some 
methods are under  multiple categories).
\xmark/\cmark ~denotes the absence/presence of the tasks commonly addressed by each approach, and  
the data constraints  
usually considered: {\bf LD}: \underline{L}imited-\underline{D}ata, {\bf FS}: \underline{F}ew-\underline{S}hot and {\bf ZS}: \underline{Z}ero-\underline{S}hot. 
}
\label{tab:approaches}
%
\\
\toprule
%\midrule
%==============Transfer Learning=====================
%Header----------------------
\rowcolor{gray!10}\multicolumn{2}{c}{{\bf \normalsize Transfer Learning} (Sec.~\ref{ssec:review_transferlearning})}
\\[2pt]
%Description----------------------
{\bf \small Description:}& \parbox[l]{0.90\linewidth}
{
Improve
GM-DC on target domain by knowledge of a generator pre-trained on source domain (with numerous and diverse samples).} 
\\[2pt]
{\bf \small Task:} & \parbox[l]{0.85\linewidth}
{
uGM-1 \xmark \hspace{0pt} %Task 1
uGM-2 \cmark \hspace{0pt} %Task 2
uGM-3 \cmark \hspace{0pt} %Task 3
cGM-1 \xmark \hspace{0pt} %Task 4
cGM-2 \cmark \hspace{0pt} %Task 5
cGM-3 \cmark \hspace{0pt}%Task 6
IGM \xmark \hspace{0pt} %Task 7
SGM \cmark %Task 8
\hspace{2pt} 
{\bf \small Data constraint:} \hspace{0pt}
LD \cmark \hspace{0pt}
FS \cmark \hspace{0pt}
ZS \cmark 
}
\\
\arrayrulecolor{black!50}
\midrule
%TF.1.1 (Definition)--------------------------------------------
\multicolumn{2}{l}{
    \cellcolor{gray!0}\parbox[l]{\linewidth}{
    \textbf{1) Regularizer-based Fine-Tuning:} 
    Explore regularizers to preserve source generators' knowledge.}
    }
 \\[2pt]
%TF.1.2 (Works)--------------------------------------------
\multicolumn{2}{l}{
    \parbox[l]{\linewidth}{
    \emph{Methods:} TGAN\cite{wang2018tgan}, BSA\cite{noguchi2019bsa}, FreezeD\cite{mo2020freezed}, EWC\cite{li2020ewc}, CDC\cite{ojha2021cdc}, cGANTransfer\cite{shahbazi2021efficient}, W\textsuperscript{3}\cite{grigoryev2022when}, C\textsuperscript{3}\cite{Lee2021C3}, DCL\cite{zhao2022dcl}, RSSA\cite{xiao2022rssa}, fairTL\cite{teo2023fairtl}, GenOS\cite{zhang2022generalizedoneshot}
           , SVD\cite{robb2020svd}, 
           D\textsuperscript{3}-TGAN\cite{wu2023d3tgan}, JoJoGAN\cite{chong2022jojogan}, 
           KDFSIG\cite{hou2022exploitingkd}, CtlGAN\cite{wang2022ctlgan}, 
           ICGAN\cite{casanova2021icgan}, 
           MaskD\cite{zhu2022few}, 
           F\textsuperscript{3}\cite{yuichi2023_fewshot},
           ICGAN \cite{casanova2021icgan},
           DDPM-PA \cite{zhu2022few_dm},
           DWSC \cite{hou_dynamic}, 
           CSR \cite{gou2023csr},
           ProSC \cite{moon2023prosc}
       }
       }
\\
\midrule
%TF.2,1 (Definition)--------------------------------------------
\multicolumn{2}{l}{
    \cellcolor{gray!0}\parbox[l]{\linewidth}{
    \textbf{2) Latent Space:} 
   Explore latent space of source generator to identify suitable knowledge for adaptation.}
    }
\\[2pt]
%TF.2.2 (Works)--------------------------------------------
\multicolumn{2}{l}{
    \parbox[l]{\linewidth}{
    \emph{Methods:} 
     MineGAN\cite{wang2020minegan}, MineGAN++\cite{wang2021minegan++}, LCL\cite{mondal2023lcl}, WeditGAN\cite{duan2023weditgan}, GenDA\cite{yang2021genda},
     %, AGE\cite{ding2022age}, SAGE\cite{ding2023sage}, LSO\cite{zheng2023lso}
     SiSTA \cite{thopalli2023targetaware}, MultiDiffusion \cite{bar2023multidiffusion}
       }
       }
 \\
\midrule
%TF.3.1 (Definition)--------------------------------------------
\multicolumn{2}{l}{
    \cellcolor{gray!0}\parbox[l]{\linewidth}{
    \textbf{3) Modulation:} 
   Leverage trainable modulation weights on top of frozen weights of the source generator.}
    }
 \\[2pt]
%TF.3.2 (Works)--------------------------------------------
\multicolumn{2}{l}{
    \parbox[l]{\linewidth}{
    \emph{Methods:} 
     AdaFMGAN\cite{zhao2020leveraging}, GAN-Memory \cite{cong2020ganmemory}, CAM-GAN\cite{varshney2021camgan}, AdAM\cite{zhao2022adam}, DynaGAN\cite{kim2022dynagan}, HyperDomainNet\cite{alanov2022hyperdomainnet}
       }
       }
\\
\midrule
%TF.4.1 (Definition)--------------------------------------------
\multicolumn{2}{l}{
    \cellcolor{gray!0}\parbox[l]{\linewidth}{
    \textbf{4) Natural Language-guided:} 
      Use the feedback of vision-language models to adapt the source generator with text prompts.}
    }
 \\[2pt]
%TF.4.2 (Works)--------------------------------------------
\multicolumn{2}{l}{
    \parbox[l]{\linewidth}{
    \emph{Methods:} 
       StyleGAN-NADA\cite{gal2022stylegannada}, MTG\cite{zhu2022mindthegap}, HyperDomainNet\cite{alanov2022hyperdomainnet}, DiFa\cite{zhang2022difa}, OneCLIP\cite{kwon2022oneclip}, IPL\cite{guo2023ipl},
       SINE\cite{zhang2023sine}, DreamBooth\cite{ruiz2023dreambooth},
       MCC\cite{kumari2023mcc}, 
       \\ Textual-Inversion\cite{gal2022textualinversion}, SpecialistDiffusion\cite{lu2023specialistdiffusion},
       BLIP-Diffusion\cite{li2023blipdiffusion}
       }
       }
\\
\midrule
%TF.5.1 (Definition)--------------------------------------------
\multicolumn{2}{l}{
    \cellcolor{gray!0}\parbox[l]{\linewidth}{
    \textbf{5) Adaptation-Aware:} 
      Preserve the source generator's knowledge that is important to the adaptation task.}
    }
 \\[2pt]
%TF.5.2 (Works)--------------------------------------------
\multicolumn{2}{l}{
    \parbox[l]{\linewidth}{
    \emph{Methods:} 
        AdAM\cite{zhao2022adam}, RICK\cite{zhao2023rick}
       }
       }
\\
\midrule
%TF.6.1 (Definition)--------------------------------------------
\multicolumn{2}{l}{
    \cellcolor{gray!0}\parbox[l]{\linewidth}{
    \textbf{6) Prompt Tuning:} 
      Freeze the source generator and add/ generate visual prompts to guide generation for the target domain.}
    }
 \\[2pt]
%TF.6.2 (Works)--------------------------------------------
\multicolumn{2}{l}{
    \parbox[l]{\linewidth}{
    \emph{Methods:} 
       VPT \cite{sohn2023vpt}
       }
       }
\\
\arrayrulecolor{black!100} \midrule %\bottomrule
%==============Data Augmentation-=====================
%Header----------------------
\rowcolor{gray!10}\multicolumn{2}{c}{{\bf \normalsize Data Augmentation}
(Sec.~\ref{ssec:review_dataaugmentation})}
\\[2pt]
%Description----------------------
{\bf \small Description:}& \parbox[l]{0.90\linewidth}{
Improve GM-DC by increasing coverage of the data distribution by applying various transformations on the given samples.}

\\[2pt]
%Buttons----------------------
{\bf \small Task:} & \parbox[l]{0.85\linewidth}
{
uGM-1 \cmark \hspace{0pt} %Task 1
uGM-2 \xmark \hspace{0pt} %Task 2
uGM-3 \xmark \hspace{0pt} %Task 3
cGM-1 \xmark \hspace{0pt} %Task 4
cGM-2 \xmark \hspace{0pt} %Task 5
cGM-3 \xmark \hspace{0pt} %Task 6
IGM \xmark \hspace{0pt} %Task 7
SGM \xmark %Task 8
\hspace{2pt}
{\bf \small Data constraint:} \hspace{0pt} 
LD \cmark \hspace{0pt}
FS \xmark \hspace{0pt}
ZS \xmark 
}
\\
\arrayrulecolor{black!50}
\midrule
%DA.1.1 (Definition)--------------------------------------------
\multicolumn{2}{l}{
    \cellcolor{gray!0}\parbox[l]{\linewidth}{
    \textbf{1) Image-Level Augmentation:} 
      Apply data transformations 
      %$\{ T_k \}$ 
      on image space.}
    }
 \\[2pt]
%DA.1.2 (Works)--------------------------------------------
\multicolumn{2}{l}{
    \parbox[l]{\linewidth}{
    \emph{Methods:} 
        ADA \cite{karras2020ada}, DiffAugment\cite{zhao2020diffaug}, IAG\cite{zhao2020imageaugmentation}, DiffusionGAN\cite{wang2023diffusiongan}, bCR\cite{zhao2021improved}, CR-GAN\cite{Zhang2020Consistency}, APA \cite{jiang2021deceive}, PatchDiffusion\cite{wang2023patchdiffusion}
       }
       }
\\
\midrule
%DA.2.1 (Definition)--------------------------------------------
\multicolumn{2}{l}{
    \cellcolor{gray!0}\parbox[l]{\linewidth}{
    \textbf{2) Feature-Level Augmentation:} 
     Apply data transformations 
     %$\{ T_k \}$ 
     on the feature space.
    }
    }
 \\[2pt]
%DA.2.2 (Works)--------------------------------------------
\multicolumn{2}{l}{
    \parbox[l]{\linewidth}{
    \emph{Methods:} 
        AdvAug\cite{chen2021advaug}, AFI\cite{dai2021implicit}
       }
       }
\\
\midrule
%DA.3.1 (Definition)--------------------------------------------
\multicolumn{2}{l}{
    \cellcolor{gray!0}\parbox[l]{\linewidth}{
    \textbf{3) Transformation-Driven Design:} 
    Leverage the information of 
    individual transformations
    to design an efficient learning mechanism.
    }
}
 \\[2pt]
%DA.3.2 (Works)--------------------------------------------
\multicolumn{2}{l}{
    \parbox[l]{\linewidth}{
    \emph{Methods:} 
        DAG\cite{tran2021dag}, SSGAN-LA\cite{hou2021labelaugmentation}
       }
       }
\\
\arrayrulecolor{black!100}
\midrule
%==============Network Architecture-=====================
%Header----------------------
\rowcolor{gray!10}\multicolumn{2}{c}{{\bf \normalsize Network Architectures} {(Sec.~\ref{ssec:review_networkarchitecture})}}
\\[2pt]
%Description----------------------
{\bf \small Description:}& \parbox[l]{0.90\linewidth}{Design specific architecture for the generator to improve its learning under data constraints.} 
\\[2pt]
%Challenges----------------------
%Buttons----------------------
{\bf \small Task:} & \parbox[l]{0.85\linewidth}
{
uGM-1 \cmark \hspace{0pt} %Task 1
uGM-2 \xmark \hspace{0pt} %Task 2
uGM-3 \xmark \hspace{0pt} %Task 3
cGM-1 \cmark \hspace{0pt} %Task 4
cGM-2 \xmark \hspace{0pt} %Task 5
cGM-3 \xmark \hspace{0pt} %Task 6
IGM \xmark \hspace{0pt} %Task 7
SGM \xmark %Task 8
\hspace{2pt} 
{\bf \small Data constraint:} \hspace{3pt} 
LD \cmark \hspace{1pt}
FS \xmark \hspace{1pt}
ZS \xmark 
}
\\
\arrayrulecolor{black!50}
\midrule
%NA.1.1 (Definition)--------------------------------------------
\multicolumn{2}{l}{
    \cellcolor{gray!0}\parbox[l]{\linewidth}{
    \textbf{1) Feature Enhancement:} 
      Design additional modules/ layers to enhance/ retain the feature maps of the generator for better generative modeling.}
    }
 \\[2pt]
%NA.1.2 (Works)--------------------------------------------
\multicolumn{2}{l}{
    \parbox[l]{\linewidth}{
    \emph{Methods:} 
    FastGAN\cite{liu2021fastgan}, MoCA\cite{li2022moca}, DFSGAN\cite{yang2023dfsgan},
    SCHA-VAE \cite{pmlr-v162-giannone22a}
       }
       }
\\
\midrule
%NA.2.1 (Definition)--------------------------------------------
\multicolumn{2}{l}{
    \cellcolor{gray!0}\parbox[l]{\linewidth}{
    \textbf{2) Ensemble Large Pre-trained Vision Models:} 
    Improve architecture by integrating pre-trained vision models to enable more accurate GM-DC.}
      %under limited data}
    }
 \\[2pt]
%NA.2.2 (Works)--------------------------------------------
\multicolumn{2}{l}{
    \parbox[l]{\linewidth}{
    \emph{Methods:} 
    Vision-aided GAN\cite{kumari2022ensembling}, ProjectedGAN \cite{sauer2021projectedgan}
       }
       }
\\
\midrule
%NA.3.1 (Definition)--------------------------------------------
\multicolumn{2}{l}{
    \cellcolor{gray!0}\parbox[l]{\linewidth}{
    \textbf{3) Dynamic Network Architecture:} 
     Improve generative learning with limited data by evolving the generator architecture during training.
    }
    }
 \\[2pt]
%NA.3.2 (Works)--------------------------------------------
\multicolumn{2}{l}{
    \parbox[l]{\linewidth}{
    \emph{Methods:} 
    CbC\cite{shahbazi2022collapse}, DynamicD\cite{yang2022dynamicd}, AdvAug\cite{chen2021advaug}, Re-GAN\cite{saxena2023regan},
    AutoInfoGAN \cite{shi2023autoinfogan}
       }
       }
\\
\arrayrulecolor{black!100}
\midrule
%==============Multi-Task Objectives-=====================
%Header----------------------
\rowcolor{gray!10}\multicolumn{2}{c}{{\bf \normalsize 
Multi-Task Objectives} 
{(Sec.~\ref{ssec:review_trainingtechniques})}}
\\[2pt]
%Description----------------------
{\bf \small Description:}& \parbox[l]{0.90\linewidth}{
Introduce additional task(s) to extract generalizable representations that are useful for all tasks, to reduce overfitting under data constraints.

} 
\\[2pt]
{\bf \small Task:} & \parbox[l]{0.85\linewidth}
{
uGM-1 \cmark \hspace{0pt} %Task 1
uGM-2 \cmark \hspace{0pt} %Task 2
uGM-3 \xmark \hspace{0pt} %Task 3
cGM-1 \cmark \hspace{0pt} %Task 4
cGM-2 \xmark \hspace{0pt} %Task 5
cGM-3 \xmark \hspace{0pt} %Task 6
IGM \xmark \hspace{0pt} %Task 7
SGM \xmark %Task 8
\hspace{2pt} 
{\bf \small Data constraint:} \hspace{3pt} 
LD \cmark \hspace{1pt}
FS \cmark \hspace{1pt}
ZS \xmark 
}
\\
\arrayrulecolor{black!50}
\midrule
%TT.1.1 (Definition)--------------------------------------------
\multicolumn{2}{l}{
    \cellcolor{gray!0}\parbox[l]{\linewidth}{
    \textbf{1) Regularizer:} 
    Add an additional task objective as a regularizer to prevent an undesirable behaviour during training generative model.
    }
    }
 \\[2pt]
%TT.1.1 (Works)--------------------------------------------
\multicolumn{2}{l}{
    \parbox[l]{\linewidth}{
    \emph{Methods:} 
         LeCam\cite{tseng2021lecam}, DigGAN\cite{fang2022diggan}, MDL\cite{kong2022mdl}, RegLA\cite{hou2023regularizing}
       }
       }
\\
\midrule

%TT.2.1 (Definition)--------------------------------------------
\multicolumn{2}{l}{
    \cellcolor{gray!0}\parbox[l]{\linewidth}{
    \textbf{2) Contrastive Learning:} 
    Introduce a pretext task to enhance the learning process of the generative model.
    }
    }
 \\[2pt]
%TT.2.1 (Works)--------------------------------------------
\multicolumn{2}{l}{
    \parbox[l]{\linewidth}{
    \emph{Methods:} 
     InsGen\cite{yang2021insgen}, FakeCLR\cite{li2022fakeclr}, DCL\cite{zhao2022dcl}, C\textsuperscript{3}\cite{Lee2021C3}, ctlGAN\cite{wang2022ctlgan}, IAG\cite{zhao2020imageaugmentation}, CML-GAN\cite{phaphuangwittayakul2022cmlgan}
       }
       }
       \\
\midrule
%TT.4.1 (Definition)--------------------------------------------
\multicolumn{2}{l}{
    \cellcolor{gray!0}\parbox[l]{\linewidth}{
    \textbf{3) Masking:} 
    Mask a part of the image/ information to increase the task hardness and prevent learning the trivial solutions.
    }
    }
 \\[2pt]
%TT.4.1 (Works)--------------------------------------------
\multicolumn{2}{l}{
    \parbox[l]{\linewidth}{
    \emph{Methods:} 
     MaskedGAN\cite{huang2022maskedgan}, MaskD\cite{zhu2022few},
     DMD \cite{zhang2023dmd}
       }
       }
\\
\midrule
\pagebreak
%TT.5.1 (Definition)--------------------------------------------
\multicolumn{2}{l}{
    \cellcolor{gray!0}\parbox[l]{\linewidth}{
    \textbf{4) Knowledge Distillation:} 
    Add a task objective that enforces the generator to follow a strong teacher.
    }
    }
\\[2pt]
%TT.5.1 (Works)--------------------------------------------
\multicolumn{2}{l}{
    \parbox[l]{\linewidth}{
    \emph{Methods:} 
    KD-DLGAN\cite{cui2023kddlgan}, KDFSIG\cite{hou2022exploitingkd}
       }
       }
\\
\midrule
%TT.6.1 (Definition)--------------------------------------------
\multicolumn{2}{l}{
    \cellcolor{gray!0}\parbox[l]{\linewidth}{
    \textbf{5) Prototype Learning:} 
    Emphasize learning prototypes for samples/ concepts within the distribution as an additional task objective.
    }
    }
 \\[2pt]
%TT.6.1 (Works)--------------------------------------------
\multicolumn{2}{l}{
    \parbox[l]{\linewidth}{
    \emph{Methods:} 
     ProtoGAN\cite{yang2023protogan}, MoCA\cite{li2022moca}
       }
       }
\\
\midrule

%TT.3.1 (Definition)--------------------------------------------
\multicolumn{2}{l}{
    \cellcolor{gray!0}\parbox[l]{\linewidth}{
    \textbf{6) Other Multi-Task Objectives:} 
    Apply other
    types of multi-task objectives
    including co-training, patch-level learning, and diffusion.
    }
    }
 \\[2pt]
%TT.3.1 (Works)--------------------------------------------
\multicolumn{2}{l}{    
    \parbox[l]{\linewidth}{
    \emph{Methods:} 
     GenCo\cite{cui2022genco}, PatchDiffusion\cite{wang2023patchdiffusion}, AnyRes-GAN\cite{chai2022anyresolution} , DiffusionGAN\cite{wang2023diffusiongan}, D2C\cite{sinha2021d2c},
     AdaptiveIMLE \cite{aghabozorgi2023adaptiveimle},
     FSDM \cite{giannone2022fsdm}
       }
       }
\\
\arrayrulecolor{black!100}
\midrule
%==============Exploiting Frequency Components=====================
%Header----------------------
\rowcolor{gray!10}\multicolumn{2}{c}{{\bf \normalsize Exploiting Frequency Components} {(Sec.~\ref{ssec:review_exploitingfrequency})}}
\\[2pt]
%Description----------------------
{\bf \small Description:}& \parbox[l]{0.90\linewidth}{
Exploit frequency components to improve learning the generative model by reducing frequency bias.} 
\\[2pt]
%Buttons----------------------
{\bf \small Task:} & \parbox[l]{0.85\linewidth}
{
uGM-1 \cmark \hspace{0pt} %Task 1
uGM-2 \xmark \hspace{0pt} %Task 2
uGM-3 \xmark \hspace{0pt} %Task 3
cGM-1 \xmark \hspace{0pt} %Task 4
cGM-2 \cmark \hspace{0pt} %Task 5
cGM-3 \xmark \hspace{0pt} %Task 6
IGM \xmark \hspace{0pt} %Task 7
SGM \xmark %Task 8
\hspace{2pt} 
{\bf \small Data constraint:} \hspace{3pt} 
LD \cmark \hspace{1pt}
FS \cmark \hspace{1pt}
ZS \xmark 
}
\\
\arrayrulecolor{black!50}
\midrule
%EFC.1.1 (Definition)--------------------------------------------
% No Subcategory
%EFC.1.2 (Works)--------------------------------------------
\multicolumn{2}{l}{
    \parbox[l]{\linewidth}{
    \emph{Methods:} 
     FreGAN\cite{yang2022fregan}, WaveGAN\cite{yang2022wavegan}, MaskedGAN\cite{huang2022maskedgan}, Gen-co\cite{cui2022genco}
       }
       }
\\
\arrayrulecolor{black!100}
\midrule
%==============Meta-Learning=====================
%Header----------------------
\rowcolor{gray!10}\multicolumn{2}{c}{{\bf \normalsize Meta-Learning} {(Sec.~\ref{ssec:review_metalearning})}}
\\[2pt]
%Description----------------------
{\bf \small Description:}& \parbox[l]{0.90\linewidth}{

Learn meta-knowledge from seen classes to improve generator learning for unseen classes.
} 
\\[2pt]

%Buttons----------------------
{\bf \small Task:} & \parbox[l]{0.85\linewidth}
{
uGM-1 \xmark \hspace{0pt} %Task 1
uGM-2 \xmark \hspace{0pt} %Task 2
uGM-3 \xmark \hspace{0pt} %Task 3
cGM-1 \xmark \hspace{0pt} %Task 4
cGM-2 \cmark \hspace{0pt} %Task 5
cGM-3 \xmark \hspace{0pt} %Task 6
IGM \xmark \hspace{0pt} %Task 7
SGM \xmark %Task 8
\hspace{2pt} 
{\bf \small Data constraint:} \hspace{3pt} 
LD \xmark \hspace{1pt}
FS \cmark \hspace{1pt}
ZS \xmark 
}
\\
\arrayrulecolor{black!50}
\midrule
%ML.1.1 (Definition)--------------------------------------------
\multicolumn{2}{l}{
    \cellcolor{gray!0}\parbox[l]{\linewidth}{
    \textbf{1) Optimization:}
    Learn initialization weights from the seen classes as meta-knowledge
    to enable quick adaptation to unseen classes.
    }
    }
 \\[2pt]
%ML.1.2 (Works)--------------------------------------------
\multicolumn{2}{l}{
    \parbox[l]{\linewidth}{
    \emph{Methods:} 
    GMN\cite{bartunov2018few}, FIGR\cite{clouatre2019figr}, Dawson\cite{liang2020dawson}, FAML\cite{phaphuangwittayakul2021faml}, CML-GAN\cite{phaphuangwittayakul2022cmlgan}
       }
       }
\\
\midrule
%ML.2.1 (Definition)--------------------------------------------
\multicolumn{2}{l}{
    \cellcolor{gray!0}\parbox[l]{\linewidth}{
    \textbf{2) Transformation:} 
    Learn sample transformations from the seen classes as meta-knowledge and use them for sample generation for unseen classes.
    }
    }
 \\[2pt]
%ML.2.2 (Works)--------------------------------------------
\multicolumn{2}{l}{
    \parbox[l]{\linewidth}{
    \emph{Methods:}
    DAGAN\cite{antoniou2017dagan}, DeltaGAN\cite{hong2022deltagan}, Disco\cite{hong2022disco}, AGE\cite{ding2022age}, SAGE\cite{ding2023sage}, HAE\cite{li2022hae}, LSO \cite{zheng2023lso}
       }
       }
\\
\midrule
%ML.3.1 (Definition)--------------------------------------------
\multicolumn{2}{l}{
    \cellcolor{gray!0}\parbox[l]{\linewidth}{
    \textbf{3) Fusion:} 
    Learn to fuse the samples of the seen classes as meta-knowledge, and apply learned meta-knowledge to generation for unseen classes.
    }
    }
 \\[2pt]
%ML.3.2 (Works)--------------------------------------------
\multicolumn{2}{l}{
    \parbox[l]{\linewidth}{
    \emph{Methods:} 
    MatchingGAN\cite{hong2020matchinggan}, F2GAN\cite{hong2020f2gan}, LofGAN\cite{gu2021lofgan}, WaveGAN\cite{yang2022wavegan}, AMMGAN\cite{li2023ammgan}
       }
       }
\\
\arrayrulecolor{black!100}
\midrule
%==============Modeling Internal Patch Distribution=====================
%Header----------------------
\rowcolor{gray!10}\multicolumn{2}{c}{{\bf \normalsize Modeling Internal Patch Distribution} {(Sec.~\ref{ssec:review_internalpatch})}}
\\[2pt]
%Description----------------------
{\bf \small Description:}& \parbox[l]{0.90\linewidth}{Learn the internal patch distribution within one image to generate diverse samples
%that carry the 
with the
same visual content (patch distribution).} 
\\[2pt]
%Buttons----------------------
{\bf \small Task:} & \parbox[l]{0.85\linewidth}
{
uGM-1 \xmark \hspace{0pt} %Task 1
uGM-2 \xmark \hspace{0pt} %Task 2
uGM-3 \xmark \hspace{0pt} %Task 3
cGM-1 \xmark \hspace{0pt} %Task 4
cGM-2 \xmark \hspace{0pt} %Task 5
cGM-3 \xmark \hspace{0pt} %Task 6
IGM \cmark \hspace{0pt} %Task 7
SGM \xmark %Task 8
\hspace{2pt} 
{\bf \small Data constraint:} \hspace{3pt} 
LD \xmark \hspace{1pt}
FS \cmark \hspace{1pt}
ZS \xmark 
}
\\
\arrayrulecolor{black!50}
\midrule
%MIPD.1.1 (Definition)--------------------------------------------
\multicolumn{2}{l}{
    \cellcolor{gray!0}\parbox[l]{\linewidth}{
    \textbf{1) Progressive Training:} 
    Train a generative model progressively %Progressively train a generative model
    to learn the patch distribution at different scales/ noise levels.
    }
    }
 \\[2pt]
%MIPD.1.2 (Works)--------------------------------------------
\multicolumn{2}{l}{
    \parbox[l]{\linewidth}{
    \emph{Methods:}
    SinDiffusion\cite{wang2022sindiffusion}, SinDDM\cite{kulikov2023sinddm}, Deff-GAN\cite{kumar2023deffGAN}, BlendGAN\cite{kligvasser2022blendgan}, SinGAN\cite{shaham2019singan}, ConSinGAN\cite{hinz2021consingan}
       }
       }
\\
\midrule
%MIPD.2.1 (Definition)--------------------------------------------
\multicolumn{2}{l}{
    \cellcolor{gray!0}\parbox[l]{\linewidth}{
    \textbf{2) Non-progressive Training:} 
    %Training takes place 
    Train a generative model
    on the same scale/ noise but with changes to the model’s architecture.
    }
    }
 \\[2pt]
%MIPD.2.2 (Works)--------------------------------------------
\multicolumn{2}{l}{
    \parbox[l]{\linewidth}{
    \emph{Methods:}
        SinFusion\cite{nikankin2022sinfusion}, One-Shot GAN\cite{sushko2021oneshotgan}
       }
       }
\\
\arrayrulecolor{black!100}
\bottomrule
\end{longtable}
\end{spacing}
% }

% continue \item 2


\begin{enumerate}\setcounter{enumi}{6}
\item[]{cGM-3 is similar to uGM-2 as cross-domain adaptation is required in both tasks, but cGM-3 focuses on conditional generation while uGM-2 focuses on unconditional generation.
Furthermore, cGM-3 is similar to cGM-2, but seen classes and unseen classes are from different domains in cGM-3.
For example, \cite{shahbazi2021efficient} has studied the setup when a pre-trained generator on ImageNet is adapted to generate samples for several classes from Places365 \cite{zhou2017places}.
Transfer learning is one of the effective approaches for cGM-3, see Sec.~\ref{sec:comprehensive_review}.}
\item {\bf Internal patch distribution Generative Modeling  (IGM).}
\vspace{-0.3em}
\begin{definition}[IGM]
{\em Given $K$ samples and assuming rich internal distribution for patches within these samples, learn to generate diverse and high-quality samples with the same internal patch distribution.}
\end{definition}
\vspace{-0.3em}
IGM aims to  capture the internal distribution of
patches within the samples. 
With the model capturing the samples' patch statistics, it is then possible  to generate high
quality, diverse samples 
with the same content as the given training samples.
In most works, $K = 1$, and IGM focuses on images \cite{shaham2019singan},
learning to generate new images with 
significant variability while maintaining
both the global structure and fine textures of the training image.



\item {\bf Subject-driven Generative Modeling  (SGM).}
\vspace{-0.3em}
\begin{definition}[SGM]
{\em Given $K$ samples of a particular subject and a text prompt, learn to generate diverse and high-quality samples containing the same subject.}
\end{definition}
\vspace{-0.3em}
SGM is a 
recent GM-DC task introduced in \cite{ruiz2023dreambooth}.
Given  a few images (3-5 in most cases) of a subject
and leveraging a large text-to-image generative model, \cite{ruiz2023dreambooth} learns to generate diverse images of the subject in different contexts with the guidance of text prompts. 
The goals are: i) to achieve  natural interactions between the subject and diverse new contexts, and ii)  to maintain high fidelity to the key visual features of the subject. 
In \cite{ruiz2023dreambooth}, a natural language-guided transfer learning approach and a new prior preservation loss have been proposed to achieve SGM.

\end{enumerate}



\def\checkmark{\tikz\fill[scale=0.4](0,.35) -- (.25,0) -- (1,.7) -- (.25,.15) -- cycle;} 

\subsection{Generative Modeling under Data Constraint: Challenges}
\label{ssec:challenges}

\subsubsection{Challenges for Training Generative Models under Data Constraint}~
Data constraints typically introduce additional challenges and 
amplify existing ones when training generative models. 
Here, we delve into the challenges of training GM-DC.
These limitations include pervasive issues of overfitting and frequency bias which are commonly observed across various approaches.
Additionally, knowledge transfer between domains brings forth specific problems including the proximity between source and target domains and the transfer of incompatible source knowledge.
As shown in Fig.~\ref{fig:works_statistics}, 
around $39\%$ of works directly rely on knowledge transfer as a mainstream method to tackle GM-DC, and more than $20\%$ of works propose methods based on other approaches that are compatible with transfer learning.

\vspace{0.1cm}
\textbf{Overfitting to Training Data.}
In machine learning, overfitting is a common issue
when powerful models start to 
 memorize the training data instead of learning the
generalizable semantics \cite{santos2022avoiding}.
In generative modeling, 
the overfitting problem exacerbates 
under data constraints due to the high capacity of current generative models \cite{noguchi2019bsa, liu2021fastgan, karras2020ada}. 
When limited training data is available,  generative models may simply remember the training data \cite{li2020ewc, ojha2021cdc} and learn to generate the exact training samples \cite{zhao2022adam} instead of capturing the data distribution.
Furthermore, under data constraints, generative modeling is more prone to mode collapse \cite{tran2021dag}, i.e., the generators learn only a limited set of modes and fail to capture other modes of the data distribution, resulting in limited diversity 
in generated samples \cite{yu2022understanding, nguyen2023rethinking}.

\vspace{0.1cm}
\textbf{Frequency Biases.}
Generative models are notorious for their spectral bias \cite{rahaman2019nnfrequncybias, khayatkhoei2022ganfrequencybias}, \ie tendency to prioritize fitting low-frequency components while disregarding high-frequency components within a data distribution \cite{durall2020watchupconvolution, tancik2020fourier, chandrasegaran2021closer}. 
The exclusion of these high-frequency components which encode intricate image details \cite{gonzales1987digital} can significantly impact the quality of generated samples, \ie, accurate modeling of high-frequency details is critical in various fields including medical imaging (X-rays, CT-scans, MRIs), satellite/ aerial imaging, astrophotography, and art restoration.
This issue becomes more severe under limited data \cite{yang2022fregan, yang2022wavegan} as even advanced network structures tailored for such scenarios \cite{liu2021fastgan} struggle to maintain the desired level of details in generated samples.




% T-SNE visualization of features
% Figure environment removed

\vspace{0.2cm}




\vspace{0.1cm}
\textbf{Modeling Distant/ Remote Target Domains under GM-DC Setups.}~
Substantial number of GM-DC tasks rely on the transfer learning principle (uGM-2, uGM-3, cGM-2, cGM-3, SGM),  
which aims to enhance the generative capabilities for a target domain by leveraging the knowledge of a generator pre-trained on a large and diverse source domain (See Fig. \ref{fig:sankey}).
A significant amount of research has been
focused on  target domains that are semantically/ perpetually similar to the source domain, \eg, learn to generate Baby faces using a pre-trained generator trained on Human faces. 
In particular, when dealing with GM-DC setups involving significant domain shifts between the source and target domains (Human Faces$\rightarrow$Animal Faces), many proposed methods fail to outperform a basic fine-tuning approach \cite{zhao2022adam}.  This is due to these methods prioritizing knowledge preservation from the source domain/ task, overlooking the adaptation step to the target domain \cite{zhao2022adam}.
Recently, adaptation-aware algorithms 
have characterized source$\rightarrow$target domain proximity \cite{zhao2022adam} and addressed GM-DC setups with pronounced domain shifts between the source and target domains (Human Faces$\rightarrow$Animal Faces) \cite{zhao2022adam, zhao2023rick}.
To understand the concept of distant/ remote target domains,
we additionally introduce two remote target domains that further exhibit a considerable degree of domain shifts: 
i) Human Faces (FFHQ) \cite{karras2019style}$\rightarrow$ Flowers \cite{nilsback2008oxford_flower}, 
ii) Human Faces (FFHQ) \cite{karras2019style}$\rightarrow$Church \cite{yu2015lsun}.
Domain proximity visualization 
is shown in Fig. \ref{fig:proximity-visualization}.
In particular, we conducted a GM-DC experiment (uGM-2) to adapt a pre-trained Human face (FFHQ) generator to Flowers under 10-shot setup using AdAM \cite{zhao2022adam}, 
obtaining a FID value of 124.46. 
Adaptation results are shown in Fig. \ref{fig:proximity-measurements-and-10-shot-flowers}.
As one can observe, multiple instances of {low quality synthesis} are observed in AdAM \cite{zhao2022adam}.
In summary, we remark that modeling distant/ remote target domains remains an important and challenging area for GM-DC.

\vspace{0.1cm}
\textbf{Identifying and Removing Incompatible Knowledge Transfer.}~
Another challenge with 
leveraging source domain's knowledge for GM-DC tasks is
incompatible knowledge transfer, which is discovered recently \cite{zhao2023rick}.
In particular, many methods may transfer knowledge that is  
incompatible with the target domain, \eg hat from source domain FFHQ to target domain flowers, significantly degrading the realisticness of the generated samples.
In Fig. \ref{fig:proximity-measurements-and-10-shot-flowers}, 
we show multiple examples of {incompatible knowledge transfer} using AdAM for 10-shot flower adaptation.
Although some recent effort has been invested in identifying and proactively truncating incompatible knowledge transfer \cite{zhao2023rick} in Human Faces $\rightarrow$ Animal Faces adaptation setups, it is worth noting that identifying and removing incompatible knowledge remains a critical and demanding area in GM-DC.



% Figure environment removed


\vspace{0.2cm}
\subsubsection{Challenges on Selecting Samples for GM-DC}~
Although considerable research effort has been invested in developing algorithms for GM-DC, the task of sample selection for GM-DC remains a challenging and relatively unexplored area.
It is essential that the samples selected for GM-DC should represent the target domain.
In particular, we observe significant variation in performance with different selection of target samples as the training datasets in  GM-DC.
We perform a 10-shot \textit{data-centric} GM-DC experiment using AdAM \cite{zhao2022adam} to emphasize the importance of sample selection in GM-DC. Following \cite{zhao2022adam, zhao2023rick}, we use AFHQ-Cat dataset \cite{choi2020starganv2} and select 3 random sets of 10-shot cat data for GM-DC. 
Data and 10-shot adaptation FID results are shown in Fig. \ref{fig:challenges-data-selection}.
We obtain FID values of 90.0, 71.6 and 49.9 for Sets 1, 2 and 3 respectively (iteration=2500). 
This study provides evidence that sample selection plays a vital role in determining the capabilities of GM-DC.
Specifically, due to cost/ privacy concerns, the role of sample selection is critical in applications including biomedical imaging, satellite/ aerial imaging and remote sensing. In summary, sample selection for GM-DC holds significant importance and remains an area with limited investigation thus far.


% Figure environment removed

\vspace{0.2cm}
\subsubsection{Challenges in Evaluating Generative Models under Data Constraint}
The assessment of generative modeling capabilities presents 
lots of challenges, encompassing both objective and subjective evaluation \cite{kynkaanniemi2023the}.
These issues are aggravated under low-data regimes resulting in the evaluation of GM-DC to be challenging and an active topic of research.
In contemporary GM-DC literature, sample quality and diversity are used as the main attributes for evaluating generation capability.
A summary of prominent metrics for GM-DC is included in Tab. \ref{tab:evaluation-metrics}.


Existing GM-DC evaluation metrics present multiple challenges:
i) Statistical measures including FID, KID, IS, FID\textsubscript{CLIP} lose their significance when dealing with 
setups
where there is an extreme scarcity (Few-shots) or complete absence (Zero-shot) of target domain data. For example, when the reference distribution contains only 10 real images, the mean and trace components of FID are not statistically significant.
ii) Although human judgment/ user feedback is used for the subjective evaluation of GM-DC, the absence of a unified framework/ protocol for such evaluation strategy results in inadequacy when comparing the generative capabilities of different GM-DC models.
iii) The over-reliance on objective GM-DC measures  on deep features extracted from pre-trained networks remains challenging and relatively unexplored. 
For example, FID, KID, and IS use features extracted from an Inception model trained on ImageNet-1K \cite{deng2009imagenet}; LPIPS, and Intra-LPIPS, 
use features extracted from models trained on BAPPS \cite{zhang2018lpips} dataset.
Although these pre-trained models effectively function as general-purpose feature extractors, their ability to capture properties/ attributes of out-of-domain data to objectively quantify the capabilities of GM-DC requires more investigation, \eg medical images.
In summary, the area of evaluation measures for GM-DC cannot be overstated, as it remains critical and challenging.


\begin{table}[!t]
%\fontsize{7pt}{7pt}
    \centering
    % \footnotesize
    \fontsize{7pt}{7pt}\selectfont
    \caption{List of common metrics used for evaluating GM-DC works.
    {\bf LD}: \underline{L}imited-\underline{D}ata, {\bf FS}: \underline{F}ew-\underline{S}hot, {\bf ZS}: \underline{Z}ero-\underline{S}hot. 
    \xmark/\cmark denotes the absence/presence, respectively.
    }
    \begin{adjustbox}{width=0.95\textwidth,center}{
    \begin{tabular}{lccccccc}
    \toprule
    \textbf{Metrics} &FID \cite{heusel2017twotimescale}/ FID\textsubscript{CLIP} \cite{kynkaanniemi2023the} &KID \cite{binkowski2018demystifying} &IS \cite{salimans2016improved} &Intra-LPIPS  \cite{ojha2021cdc} &SIFID \cite{shaham2019singan} &Image/ Text Similarity \cite{gal2022textualinversion} &User Feedback \\ \toprule
    \textbf{LD} &\cmark &\cmark &\cmark &\xmark &\xmark &\xmark &\cmark \\ \midrule
    \textbf{FS} &\cmark &\cmark &\cmark &\cmark &\cmark &\cmark &\cmark \\ \midrule
    \textbf{ZS} &\xmark &\xmark &\xmark &\cmark &\cmark &\cmark &\cmark \\
    \bottomrule
    \end{tabular}
    }
    \end{adjustbox}
    \label{tab:evaluation-metrics}
  % \vspace{-0.3cm}
\end{table}
%\vspace{-12pt}







\section{Comprehensive Review}
\label{sec:comprehensive_review}


In this section, 
first, we  will present our proposed taxonomy of approaches for GM-DC which systematically categorizes and organizes GM-DC methods under seven approaches (Tab.~\ref{tab:approaches}) based on the principal ideas of these methods.
Then, we will discuss individual GM-DC methods organized under our proposed taxonomy.


\noindent
{\bf Our Proposed Taxonomy of Approaches for GM-DC} categorizes  GM-DC methods into seven groups:
\begin{enumerate}
    \vspace{0.1cm}
    \item {\bf Transfer Learning:} 
    In GM-DC, transfer learning (TL) aims to improve the learning of the generator for the target domain using the knowledge of a generator pre-trained on a source domain (with numerous and diverse samples). For example, some methods under this category use the knowledge of a StyleGAN2 pre-trained on the large FFHQ \cite{karras2019style} to improve  the learning of generation for face paintings by an artist, given only a few images of the artist's paintings \cite{ojha2021cdc,zhao2022dcl,xiao2022rssa}. Major challenges for TL-based GM-DC are to identity, select and preserve suitable knowledge of the source generator for the target generator.
    Along this line, there are six subcategories: i) {\em Regularization-based Fine-tuning}, explores regularizers to preserve suitable source generator's knowledge to improve learning target generator;  
    ii) {\em Latent Space}, explores transformation/ manipulation of the source generator latent space;
    iii) {\em Modulation},  freezes and transfers weights of the source generator to the target generator and adds trainable modulation weights on top of frozen weights to increase the adaptation capability to the target domain; iv) {\em Natural Language-guided}, uses natural language prompt and supervision signal from language-vision models to adapt source generator to target domain; v) {\em Adaptation-Aware}, identifies and preserves the source generator's knowledge that is important to the adaptation task; vi) {\em Prompt Tuning}, is an emerging idea that freezes the weights of the source generator and learns to generate visual prompts (tokens) to guide generation for the target domain.
    \vspace{0.1cm}
    \item {\bf Data Augmentation:} Augmentation aims to improve GM-DC by increasing coverage of the  data distribution with  applying various transformations $\{ T_k \}_{k=1}^{K}$ to available data. 
    For example, within this category, some works augment the available limited data to train an unconditional StyleGAN2 \cite{karras2020analyzing} using the 100-shot Obama dataset or train a conditional BigGAN \cite{brock2019biggan} with only 10\% of the CIFAR-100 dataset.
    A major challenge of these approaches is augmentation leakage, 
    where
    the generator learns the augmented distribution, \eg, generating rotated/ noisy samples.
    There are three representative categories: 
    i) {\em Image-Level Augmentation}, applies the transformations on the image space; 
    ii) {\em Feature-Level Augmentation}, applies the transformations on the feature space; 
    iii) {\em Transformation-Driven Design}, leverages the information on each individual transformation $T_k$
    for
    an efficient learning mechanism.
    \vspace{0.1cm}
    \item {\bf Network Architectures:} These approaches design specific architectures for the generators to improve their learning under data constraints. 
    Some works in this category design shallow/ sparse generators to prevent overfitting to training data due to over-parameterization.
    The primary challenge is that when endeavoring to design a new architecture, the process of discovering the optimal hyperparameters can be laborious.
    There are three major types of architectural designs for GM-DC: 
    i) {\em Feature Enhancement}, introduces additional modules to enhance/ retain the knowledge within feature maps; 
    ii) {\em Ensemble Large Pre-trained Vision Models}, leverages large pre-trained vision models to aid more accurate generative modeling, iii) {\em Dynamic Network Architecture}, evolves the architecture of the generative model during training to compensate for data constraints.
    \vspace{0.1cm}
    \item {\bf 
    Multi-Task Objectives:} 
    These approaches modify the learning objective of the generative model by introducing additional task(s) to extract generalizable representations and reduce overfitting under data constraints.
    As an example, some works define a pretext task based on contrastive learning \cite{he2020momentum} to pull the positive samples and push away negative ones in addition to the original generative learning task to prevent overfitting under limited available data.
    The efficient integration of the new objective with the generative learning objective could be challenging under data constraints.
    These works can be categorized into several kinds of approaches: 
    i) {\em Regularizer}, adds an additional learning objective as a regularizer to prevent an undesirable behavior during training generative model under data constraint. Note that this category is different from regularizer-based fine-tuning as the latter aims to preserve source knowledge, but the former is for training without a source generator; 
    ii) {\em Contrastive Learning}, adds the learning objective related to a pretext task to enhance the learning process of the generative model using additional supervision signal from solving this pretext task; 
    iii) {\em Masking},
    introduces alternative learning objective by masking a part of the image/ information to improve generative modeling by increasing the task hardness and preventing learning the trivial solutions; 
    iv) {\em Knowledge Distillation}, introduces an additional learning objective that enforces the generator to follow a strong teacher; 
    v) {\em Prototype Learning}, emphasizes learning prototypes for samples/ concepts within the distribution as an additional objective; 
    vi) {\em Other Multi-Task Objectives}, include co-training, patch-level learning, and using diffusion to enhance 
    generation.
    \vspace{0.1cm}
    \item {\bf Exploiting Frequency Components:} Deep generative models exhibit frequency bias tending to ignore high-frequency signals as they are hard to generate \cite{schwarz2021frequency}. Data constraints can exacerbate this problem \cite{yang2022fregan}. 
    The approaches in this category aim to improve frequency awareness of the generative models by leveraging frequency components during training. 
    For instance, certain approaches employ Haar Wavelet transform to extract high-frequency components from the samples. 
    These frequency components are then fed into various layers using skip connections, aiming to alleviate the challenges associated with generating high-frequency details.
    Despite its effectiveness, utilizing frequency components for GM-DC has not been thoroughly investigated. The performance can be enhanced by incorporating more advanced techniques for extracting frequency components.
    \vspace{0.1cm}
    \item {\bf Meta-Learning:} These approaches create sample generation tasks with data constraints for the seen classes, and learn the meta-knowledge ---knowledge that is shared between all tasks--- 
    across these tasks during meta-training. This meta-knowledge is then used in improving generative modeling for the unseen classes with data constraints.
    For instance, some studies, as meta-knowledge, learn to fuse the samples from 
    the seen categories $C_{seen}$ of the Flowers  dataset \cite{nilsback2008oxford_flower} for sample generation.
    This meta-knowledge enables the model to generate new samples from unseen classes $C_{unseen}$ within the same dataset ($C_{seen} \cap C_{unseen} = \emptyset$) by fusing only 3 samples from each class.
    Note that as these works employ episodic learning within a generative framework, the training stability can be impacted.
    Approaches within this line can be classified into three categories: 
    i) {\em Optimization}, initializes the generative model with weights learned on the seen classes as meta-knowledge, to enable quick adaptation to unseen classes with limited steps of the optimization;
    ii) {\em Transformation}, learns cross-category transformations from the samples of the seen classes as meta-knowledge and applies them to available samples of the unseen classes to generate new samples; 
    iii) {\em Fusion}, learns to fuse the samples of the seen classes as meta-knowledge, and applies learned meta-knowledge to sample generation by fusing samples of the unseen classes.
    \vspace{0.1cm}
    \item {\bf Modeling Internal Patch Distribution:} These approaches aim to learn the internal patch distribution within one image (in some cases a few images), and then generate diverse samples that carry the same visual content (patch distribution) with an arbitrary size, and aspect ratio. 
    As an example, some works train a Diffusion Model using a single image of the ``Marina Bay Sands'', and after training, the Diffusion Model can generate similar images, but include additional towers topped by the similar ``Sands Skypark''.
    However, a major limitation of these methods lies in the fact that for every single image, usually a separate generative model is trained from scratch, 
    neglecting the potential for efficient training through knowledge transfer in this context.
    Approaches proposed along this line can be categorized into two major groups: i) {\em Progressive Training}, progressively trains a generative model to learn the patch-distribution at different scales or noise levels; ii) {\em Non-Progressive Training}, learns a generative model at a single scale by implementing additional sampling techniques or new model architectures.  
\end{enumerate}
In what follows we delve into detailed descriptions of the approaches within each category.
\vspace{-0.1cm}

\subsection{Transfer Learning}
\label{ssec:review_transferlearning}
Transfer Learning (TL) is a major approach for GM-DC. 
Given a source generator $G_s$ (for GANs both $G_s$ and $D_s$) pre-trained on a large and diverse source domain $\mathcal{D}_s$, these approaches aim to learn an adapted generator to the target domain $G_t$ by initializing the weights to that of the source generator. 



\subsubsection{Regularizer-based Fine-Tuning}

TGAN \cite{wang2018tgan} is the first systematic study to evaluate transfer learning in GANs. 
TGAN shows that transfer learning reduces the convergence time and improves generative modeling under limited data. 
The knowledge transfer is performed by using the source GAN for initializing the weights of the target GAN, followed by fine-tuning the weights on target data.
TGAN \cite{wang2018tgan} demonstrates that: i) transferring $D$ is more important than $G$, while transferring both $G$ and $D$ gives the best results; ii) transfer learning performance degrades by increasing the distance between source and target domains or decreasing the number of samples from target domain; iii) to select a pre-trained GAN for a target domain, in addition to a smaller distance, more dense source domains are preferable. 
As an example, for the Flower \cite{nilsback2008oxford_flower} target domain, surprisingly, a GAN pre-trained on semantically unrelated LSUN Bedrooms \cite{yu2015lsun} is shown to be among the best sources \cite{wang2018tgan}.

W\textsuperscript{3} \cite{grigoryev2022when} revisits the transfer learning in GANs with a modern structure ---StyleGAN2-ADA \cite{karras2020analyzing, karras2020ada} instead of WGAN-GP \cite{gulrajani2017improved} used in TGAN.
Results in \cite{grigoryev2022when} suggest that for SOTA GANs, it is beneficial to transfer the knowledge from sparse and diverse sources (pre-trained StyleGAN2 on ImageNet) rather than dense and less diverse ones.
One major limitation of TGAN is that under limited data simply fine-tuning the whole generator destroys a considerable portion of the general knowledge obtained on the source domain, and results in overfitting. Almost all of the following works aim to address this by different approaches to preserve the knowledge of the source generator.

%BSA
BSA \cite{noguchi2019bsa} limits scale and shift parameter updates during fine-tuning for batch normalization (BN) layers.
%FreezD
FreezeD \cite{mo2020freezed}, hypothesizes that as $D$ performs the discriminative task during training a GAN, based on common knowledge in discriminative learning \cite{yosinski2014transferable}, its early layers encode general knowledge which is shared between source and target domains. 
Therefore, this general knowledge is preserved during adaptation by freezing early layers of $D$.
% cGANTransfer
cGANTransfer \cite{shahbazi2021efficient} assumes that the pre-trained $G$ is conditioned on class labels using BN parameters, \ie each class has its own BN parameters \cite{brock2019biggan}. Then, explicit knowledge propagation from seen classes to unseen classes is enforced by defining the BN parameters of the unseen classes to be the weighted average of the BN parameters of seen classes.
%SVD
SVD \cite{robb2020svd} applies singular value decomposition \cite{van1976generalizing}, and only updates the singular values that are related to changing entanglement between different attributes within data to constrain the parameter update.

%EWC
EWC \cite{li2020ewc} aims to preserve the diversity of the source GAN during adapting to a target domain with only a few samples, \eg, 10-shot. The importance of each parameter in source GAN is measured by Fisher Information (FI), and the change on each parameter during adaption is penalized based on its importance, \eg, change over important parameters is penalized more.
% CDC
CDC \cite{ojha2021cdc} aims to keep the diversity of the generated samples using a cross-domain correspondence loss.
Specifically, first, a batch of $N+1$ latent codes are sampled for image generation: $\{ G(z_0), ..., G(z_N) \}$.
Then, using $G(z_0)$ as a reference, the similarity of generated samples to reference is measured for the generator before and after adaptation, resulting in two $N-way$ probability vectors. The diversity is preserved by adding the KL divergence between these two probability vectors to the standard loss as a regularizer.
% MaskD
MaskD~\cite{zhu2022few} applies random masks to extracted features of $D$, on top of CDC \cite{ojha2021cdc}, to prevent overfitting. 
% DDPM-PA
DDPM-PA \cite{zhu2022few_dm} uses a pairwise adaptation method similar to CDC for adapting diffusion models to the new domain.
% RSSA
RSSA \cite{xiao2022rssa} extends the cross-domain consistency idea of the CDC \cite{ojha2021cdc} to a more constrained form by preserving the structural similarity of the samples before and after adaptation.
% ProSC
ProSC \cite{moon2023prosc} extends RRSA by performing a progressive adaptation to the target domain in $N$ iterations instead of a single adaptation to reduce the gap between pair of domains.
% CSR
CSR \cite{gou2023csr} uses a similar idea to CDC but applies semantic loss directly to the spatial space, \ie, generated images with $G_s$ and $G_t$.

% C3 - DCL - CtlGAN
$C^3$ \cite{Lee2021C3}, DCL \cite{zhao2022dcl}, and CtlGAN \cite{wang2022ctlgan} aim to preserve the diversity by applying contrastive learning. 
Assuming $G_s(z_i)$ as an anchor point, the generated sample for the same latent code with the adapted generator ($G_t(z_i)$) is considered a positive pair, and the generated samples with the adapted generator for other latent code values ($G_t(z_j)$, $i \neq j$) are considered as negative pairs. 
Additionally, DCL applies similar contrastive learning to the $D$.

%JoJoGAN
JoJoGAN \cite{chong2022jojogan} addresses one-shot image generation using the style space of StyleGAN2.
First, GAN inversion is used to find the corresponding style code of the reference image. Then, style mixing is used to generate a set of style codes, and generated images with these styles are used for GAN adaptation.
%GenOS
GenOS \cite{zhang2022generalizedoneshot} includes entity transfer with some related entity mask using an auxiliary network.
%D3-TGAN
D\textsuperscript{3}-TGAN
\cite{wu2023d3tgan} first inverts each target sample into the latent code space of the source GAN.
Then, the maximum mean discrepancy between the features of the source $G$ for inverted code and features of the adapted GAN for a random latent code is used as a regularizer.


% FairTL
FairTL \cite{teo2023fairtl} adopts transfer learning in GANs to train a fair generative model \wrt a sensitive attribute (SA) using a limited fair dataset.
% IC-GAN
To model complex distributions like ImageNet, IC-GAN \cite{casanova2021icgan} learns data distribution as a mixture of conditional distributions.
This enables IC-GAN to generate images from unseen distributions, by just changing the conditioning instances on the target samples.
%KDFSIG
KDFSIG \cite{hou2022exploitingkd} exploits the knowledge distillation idea by treating the source model as a teacher and the target model as a student.
% F3
F\textsuperscript{3} \cite{yuichi2023_fewshot} proposes a faster method to generate face images with features of a specific group. First, a GAN Inversion of target images is applied and then PCA is leveraged as a feature extraction strategy to render features of target group.
% DWSC
DWSC \cite{hou_dynamic} proposes the dynamic weighted semantic correspondence between the source and target generator during adaption to preserve the diversity.



%-----------Latent Space---------------%
\subsubsection{Latent Space} 

% MineGAN
MineGAN \cite{wang2020minegan} trains a miner network $M$ during adaptation, to map the latent space $z$ of the source GAN to another space $u = M(z)$ more appropriate for the target domain. 
MineGAN++\cite{wang2021minegan++} extends MineGAN by only updating important parameters.
% GenDA
GenDA \cite{yang2021genda} proposes a lightweight attribute adaptor in the form of scaling and shifting latent codes
to adapt the latent space of the source GAN to the target GAN.
LCL \cite{mondal2023lcl} freezes $G$ and learns a network to map the latent codes from the $\mathcal{Z}$ space to the extended intermediate space $\mathcal{W}^+$ of a pre-trained StyleGAN2 during adapting GAN. 
%WeditGAN 
WeditGAN \cite{duan2023weditgan} proposes to learn a constant offset parameter ($\Delta w$) for the target domain in the intermediate latent space of StyleGAN2 to relocate source latent codes to the target domain.
%SiSTA
After fine-tuning the generator to a target domain, SiSTA \cite{thopalli2023targetaware} perturbs latent representations of the fine-tuned generator that falls below a threshold, either by replacing them with zero or reverting them back to the pre-trained generator's weights.
% MultiDiffusion
MultiDiffusion \cite{bar2023multidiffusion} freezes the whole parameters of the source diffusion model and optimizes the latent code as a post-processing method to generate the desired output based on a conditioned input.



%----------------------Modulation------------------%
\subsubsection{Modulation}
In signal processing literature, modulation varies some key attributes of a signal to add the desired information to it \cite{oppenheim1997signals}. 
Similarly, in deep neural networks, modulation is used to add some desired information to a base network by adding modulation parameters to the parameters/ features of the base network.
% AdaFM
AdaFMGAN \cite{zhao2020leveraging} shows that layers closer to the sample (earlier layers in $D$, and later layers in $G$) encoder general knowledge.
This general knowledge is conceptually shared between source and target domains and aimed to be preserved by Adaptive Filter Modulation which trains a scale and shift parameter for each $k \times k$ kernel.
GAN-Memory \cite{cong2020ganmemory}, and CAM-GAN \cite{varshney2021camgan} use similar modulation ideas to modulate a pre-trained GAN for generative continual learning.
AdAM \cite{zhao2022adam} uses kernel modulation \cite{milad2021revisit} for few-shot generative modeling by aiming to preserve the important wights of a pre-trained GAN during a few-shot adaptation to a target domain.
HyperDomainNet \cite{alanov2022hyperdomainnet} adds an additional modulation to StyleGAN2 \cite{karras2020analyzing} for adapting to a new domain, while optimizing only $6K$ parameters.

%------------------Adaptation-Aware-------------------%
\subsubsection{Adaptation-Aware}
Adaptation-aware transfer learning approaches propose that different parts of the knowledge encoded on a pre-trained generative model could be important based on the target domain. 
AdAM \cite{zhao2022adam} proposes a probing step before the main adaptation, where the importance of each kernel for adapting a source GAN to the target domain using a few samples is measured using FI.
Then, during the main adaptation, the important kernels are preserved using modulation, and the other kernels are simply fine-tuned.
RICK \cite{zhao2023rick} shows that incompatible knowledge from a source domain to a target domain is related to the kernels with the least importance to this adaptation, and this knowledge can not be removed by simple fine-tuning. Therefore, RICK proposes a dynamic probing schedule during adaptation where it gradually prunes the kernels with the least importance.

%--------------Natural Language-Guided----------------%
\subsubsection{Natural Language-Guided}
Vision-Language models like CLIP \cite{radford2021CLIP} are usually trained on large-scale image-text pairs and learn to encapsulate the generic information by combining image and text modalities.
This generic information is shown to be helpful in various downstream tasks including zero-shot and few-shot image generation.
StyleGAN-NADA (NADA) \cite{gal2022stylegannada} is the pioneering work on utilizing CLIP for zero-shot image generation.
NADA \cite{gal2022stylegannada} uses a text prompt $T_t$ --which describes the target domain-- as input and uses feedback to adapt a pre-trained StyleGAN2 to the target domain.
Specifically, assuming a text prompt $T_s$ that describes the source domain (\eg, "Photo" for a StyleGAN2 pre-trained on the FFHQ), and a given $T_t$ (\eg, "Fernando Botero Painting"), CLIP's text encoder $E_T$ is used to find the update direction in the embedding space: $\Delta T = E_T(T_t) - E_T(T_s)$. 
Similarly, the direction of the update/ change for the images can be computed using generated images with source and target generators: $\Delta I = E_I(G_t(z)) - E_I(G_s(z))$, where $E_I$ denotes CLIP's image encoder.
Since the image and corresponding texts are aligned in the CLIP space, NADA \cite{gal2022stylegannada} proposes to update the generator's parameters in a way to match $\Delta I$ and $\Delta T$ leading to the directional loss $\mathcal{L}_{directional}$:
\begin{equation}
    \label{eq:nada_loss}
    \mathcal{L}_{directional} = 1 - \frac{\Delta I(G_s(z),G_t(z)).\Delta T(T_s,T_t)}{|\Delta I(G_s(z),G_t(z))||\Delta T(T_s,T_t)|}
\end{equation}
This idea can be easily extended to one-shot image generation, by replacing $\Delta T$ with the direction obtained by target image $I_t$ and a batch of generated images by the source generator: $\Delta I' = E_I(I_t) - \mathbb{E}_i \{E_I(G_s(z_i))\}$, where $\mathbb{E}_i \{E_I(G_s(z_i))\}$ denotes the mean of the CLIP embedding for a batch of generated images.
% Mind The GAP (MTG)
MTG \cite{zhu2022mindthegap} extends the idea for one-shot image generation by replacing the mean embedding with the projection of the target image on the source generator denoted as $I^*_s$. Specifically, MTG uses GAN inversion to get the corresponding $z^{ref}$ for $I_t$, and uses it to generate the projected image: $I^*_s=G_s(z^{ref})$. 
% HyperDomainNet
HyperDomainNet improves the performance of the NADA and MTG by freezing the weights of the source generator and training modulation weights for the synthesis part inside the generator.
% DiFa
DiFa \cite{zhang2022difa} adds an attentive style loss to directional loss of NADA \cite {gal2022stylegannada} as a local-level adaptation which aligns the intermediate tokens of the generated image with source and pre-trained GAN.
% OneCLIP
OneCLIP \cite{kwon2022oneclip} exploits the CLIP embedding for three major modules in one-shot learning: i) inverting sample into latent space, ii) preserving the diversity of the GAN during adaptation, and iii) 
% during
a patch-wise contrastive learning approach for preserving local consistency.

% IPL
IPL \cite{guo2023ipl} mentions that NADA \cite{gal2022stylegannada} uses a fixed update direction for a target domain 
% because of using 
due to
manual text prompts being used for describing source and target domains.
To address this, \cite{guo2023ipl} learns a specific prompt for each generated image, \eg, "{\em Elder glass man} photo" $\rightarrow$ "{\em Elder glass man} Fernando Botero Painting".
A mapper function $F$ is used to learn prompts for the generated images $G_s(z_i)$:
\begin{equation}
\label{eq:ipl_prompt}
    F(z_i) = [V]_1^i [V]_2^i ... [V]_m^i
\end{equation}
where $[V]_j^i$ represents the $j^{th}$ prompt vector. Then, the domain embedding $[Y_s]$ is added to this prompt to represent both the prompt and the domain: $M_s^i =F(z_i)[Y_s]$.
The mapper function is trained by contrastive loss in the CLIP space.
After training $F$, the same adaptation process as NADA \cite{gal2022stylegannada} can be used but using $T_s(z_i)=[V]_1^i [V]_2^i ... [V]_m^i [Y_s]$. 

%DreamBooth
DreamBooth \cite{ruiz2023dreambooth} addresses subject-driven sample generation by fine-tuning a text-to-image diffusion model  e.g. Imagen \cite{saharia2022imagen}, DALL-E2 \cite{ramesh2022dalle2}. 
Input images are paired with a text prompt that contains a unique identifier and the subject class (e.g., ``A [V] dog''), and the pair is used to fine-tune the model. 
They further propose a class-specific prior preservation regularizer to encourage diversity and to mitigate {\em language drift}, i.e., the model progressively loses syntactic and semantic knowledge during fine-tuning.
% Textual Inversion
Textual-Inversion \cite{gal2022textualinversion} optimizes a word vector for the new subject given a few images and uses that word vector for the subject-driven generation. 
% MCC
MCC \cite{kumari2023mcc} extends DreamBooth by the capability of adding multiple subjects and improves convergence and performance by restricting fine-tuning to a subset of cross-attention layer parameters in DM.
%BLIP-DIffusion
BLIP-Diffusion \cite{li2023blipdiffusion} uses BLIP-2 \cite{li2023blip} multimodal encoder to extract a more text-aligned representation for each subject, and then a subject representation learning step is performed to enable DM to leverage such representation for fast and high-fidelity subject-driven sample generation. 
%SINE
SINE \cite{zhang2023sine} uses a similar fine-tuning of text-to-image diffusion model with a unique identifier, but at the patch level.
In addition, the mixing of the latent code is also used to edit the subject and put it in a new context. 
%SpecialistDiffusion
SpecialistDiffusion \cite{lu2023specialistdiffusion} addresses adapting the text-to-image models to an unseen style given a few samples from that style using augmenting both text and image, and sparse diffusion for computation efficiency.

%--------------Prompt Tuning---------------%

\subsubsection{Prompt Tuning}
VQ-VAEs (Sec.~\ref{ssec:generative_models}) can be broadly categorized into two types from the perspective of predicting the latent prior of visual tokens. AutoRegressive (AR) approaches like DALL$\cdot$E \cite{ramesh2021dalle} and VQ-GAN \cite{esser2021vqgan}, learn an AR predictor that follows a raster scan order and predicts the visual tokens from left to right, line-by-line. 
Non-AutoRegressive (NAR) approaches like DALL$\cdot$E2 \cite{ramesh2022dalle2}, MaskGIT \cite{chang2022maskgit}, Latent Diffusion \cite{rombach2022latentdiffusion}, or Imagen \cite{saharia2022imagen} usually resort to masking techniques \cite{devlin2019bert} to predict the visual tokens in a series of refinement or denoising steps.
VPT \cite{sohn2023vpt} is the first work that adopts the prompt tuning idea for image generation with generative knowledge transfer.
It uses a VQ-VAE framework where a MaskGIT\cite{chang2022maskgit}/ VQ-GAN\cite{esser2021vqgan} on the ImageNet dataset (as an example of NAR/ AR approach) is used as a pre-trained network.
Then, during adaptation, all the parameters of the VQ Encoder, VQ decoder, and transformer are frozen, and a generator is learned to minimize the adaptation loss by generating and appending a set of visual tokens to the predicted prior.
These visual tokens guide the generation process for the target domain by helping the transformer to predict proper tokens to the VQ decoder.


%%%%%%%%%%%%%%-----AUGMENTATION-----%%%%%%%%%%%%%%
\subsection{Data Augmentation}
\label{ssec:review_dataaugmentation}
Data augmentation increases the quantity and diversity of the training data  
which is shown to be beneficial for GM-DC.
However, if it is not deployed correctly, augmentations can leak into the generator resulting in generating samples with similar augmentations, \eg noisy or rotated images, which is undesirable.


%--------------Image-Level Augmentation--------------
\subsubsection{Image-Level Augmentation}
CR-GAN \cite{Zhang2020Consistency} and bCR \cite{zhao2021improved} apply various transformations on images and enforce the output of the generator to be the same for original and transformed images.
Even though not developed for GM-DC, experimental results in \cite{karras2020ada} show that CR-GAN and bCR are beneficial for limited data scenarios.
ADA proposes applying the transformations to real and fake images but with a probability $p < 1$.
The central design in ADA \cite{karras2020ada} is that the strength of the augmentation ($p$) is being adapted based on the training dynamics.
Specifically, the portion of the real images that get positive output from the discriminator, \ie, $r = \mathbb{E}[Sign(D)]$, is used as an indicator of the discriminator overfitting ($r=0$ no overfitting, and $r=1$ complete overfitting).
Then, during training, $p$ is adjusted to keep $r$ low.
DiffAugment \cite{zhao2020diffaug}, and IAG \cite{zhao2020imageaugmentation} use a similar idea to ADA, but without the adaptive component ($p=1$).
APA \cite{jiang2021deceive} uses the same adaptive augmentation mechanism in ADA, but instead of using transformations like rotation, it randomly labels generated images as pseudo-real ones to prevent an overconfident discriminator.

%DiffusionGAN
DiffusionGAN \cite{wang2023diffusiongan} applies the gradual diffusion process on real and generated images during training GAN. 
Training starts with real and generated images, and each diffusion step is applied after a certain number of training epochs making the bi-classification task harder for the discriminator to prevent its overfitting.
% PatchDiffusion
PatchDiffusion \cite{wang2023patchdiffusion} 
augments the data during training diffusion models by sampling patches with random locations and random sizes alongside the full image and conditioning the denoising score function \cite{karras2022elucidating} on the patch size and the location information. 


%--------------Feature-Level Augmentation--------------
\subsubsection{Feature-Level Augmentation}
AdvAug \cite{chen2021advaug} computes the adversarial perturbation $\delta$ for the feature maps of the discriminator and generator using the projected gradient descent \cite{madry2018towards}. Denoting the discriminator as $D = D_2 \circ D_1$, the adversarial augmentation is applied on the intermediate feature maps ($D_1$) of both real and generated images, resulting in $D_{1}(x) + {\delta}$, and $D_{1}(G(z)) + {\delta}$. The adversarial loss is then added to the loss function of $D$ during GAN training to maximize the score of the perturbed real image and minimize the score of the perturbed generated image:
\begin{equation}
    \mathcal{L}_{D}^{adv} \coloneqq \max_{\| \delta \|_{\infty} < \epsilon} \mathbb{E}_{x \sim p_{data}} [f_{D}(-D_2(D_1(x)+\delta))] 
    +
    \max_{\| \hat{\delta} \|_{\infty} < \epsilon} \mathbb{E}_{z \sim p_{z}} [f_{D}(D_2(D_1(G(z))+\hat{\delta}))] 
\end{equation}
As AdvAug is performed on the feature level, it is shown to be complementary to image-level augmentations like ADA \cite{karras2020ada} and DiffAug \cite{zhao2020diffaug}.
AFI \cite{dai2021implicit} observes a flattening effect in discriminators with multiple output neurons, and takes advantage of this observation by proposing feature interpolation as implicit data augmentation.

%--------------Transformation Information--------------
\subsubsection{Transformation-Driven Design}
DAG \cite{tran2021dag} uses a separate discriminator $D_k$ for discriminating real and fake images that are augmented by transformation $T_k$. 
A weight-sharing mechanism between all discriminators is used to prevent overfitting.
Additionally, DAG provides a theoretical ground for training convergence under augmentation. As mentioned in \cite{goodfellow2014GANs}, for an optimal discriminator $D^*$, optimizing $G$ is equivalent to minimizing the Jensen-Shannon (JS) divergence between the real data distribution $P_{data}$ and generated data distribution $P_{model}$, \ie, $JS(P_{data}, P_{model})$. 
Denoting $P_{data}^{T}$, and $P_{model}^{T}$ as the distribution of the real and generated data under augmentation $T$, \cite{tran2021dag} shows that JS divergence between two distributions is invariant under differentiable and invertible transformations:
\begin{equation}
    \label{eq:JS_invariant}
    JS(P_{data}, P_{g}) = JS(P_{data}^{T}, P_{g}^{T})
\end{equation}
This means that as long as the augmentation is differentiable and invertible, training convergence is guaranteed.
SSGAN-LA \cite{hou2021labelaugmentation} extends DAG by merging all discriminators to a single discriminator and augmenting the label space of the discriminator, \ie, asking $D$ to detect the type of augmentation in addition to conventional real/ fake detection.

%%%%%%%%%%%%%%-----Network Architecture-----%%%%%%%%%%%%%%
\subsection{Network Architectures}
\label{ssec:review_networkarchitecture}

\subsubsection{Feature Enhancement}
FastGAN \cite{liu2021fastgan} proposes a light-weight GAN structure ---shallower $G$ and $D$ compared to SOTA GANs like StyleGAN2--- to decrease the risk of overfitting.
Inspired by skip connections \cite{he2016resnet}, and squeeze-and-excitation module \cite{hu2018squeeze}, FastGAN fuses features with different resolutions in $G$ through proposed skip-layer excitation modules.
An additional reconstruction task is defined for $D$.
MoCA \cite{li2022moca} learns some prototypes for each semantic concept within a domain, \eg, railroad, or sky in a photo of a train.
Then, by attending to these prototypes during image generation, some residual feature maps are produced to improve image generation.
DFSGAN \cite{yang2023dfsgan} proposes to preserve the content and layout information in intermediate layers of $G$ by extracting channel-wise and pixel-wise information and using them to scale corresponding feature maps.
SCHA-VAE \cite{pmlr-v162-giannone22a} extend  latent variable models for sets to a fully hierarchical approach and propose Set-Context-Hierarchical-Aggregation VAE for few-shot generation. 


\subsubsection{Ensemble Pre-trained Vision Models}
ProjectedGAN \cite{sauer2021projectedgan} proposes to project real and generated images into the feature space of a pre-trained vision model to enhance $D$'s performance in discriminating real and fake images by adding two modules.
First, the output from multiple layers is used with separate discriminators.
Then a {\em random projection} is used to dilute the features and encourage the discriminator to focus on a subset of the features.
Vision-aided GAN \cite{kumari2022ensembling} uses an ensemble of the original discriminator $D$ and additional discriminators $\{ \hat{D}_n \}_{n=1}^{N}$ to perform the classification task.
The additional discriminators $\{ \hat{D}_n \}_{n=1}^{N}$ have a set of pre-trained feature extractors $\mathcal{F} = \{ F_n \}_{n=1}^{N}$ (extracted form pre-trained vision models) with a small trainable head $C_n$ added on top: $\hat{D}_n =  F_n \circ C_n$. 


\subsubsection{Dynamic Network Architecture}
CbC \cite{shahbazi2022collapse} shows that under data constraints, 
where an unconditional GAN can generate satisfactory performance, training the 
conditional GANs (cGANs) result in mode collapse. 
To mitigate this issue, CbC \cite{shahbazi2022collapse} starts training from an unconditional GAN and slowly transitions to a cGAN using a transition function $0 \leq \lambda_t \leq 1$.
Considering the conditioning variable as $c$, this transition is implemented in $G$ as $G(z,c,\lambda_t)=G(S(z)+\lambda_t.E(c))$, with $S$ and $E$ as neural networks that transform the latent code and the conditioning variable.
DynamicD \cite{yang2022dynamicd} dynamically reduces the capacity of $D$ by randomly sampling a subset of channels of $D$ during each training iteration to prevent overfitting.
Inspired by the lottery ticket hypothesis \cite{frankle2019lotteryticket}, AdvAug \cite{chen2021advaug} and Re-GAN \cite{saxena2023regan} have shown that a much sparse subnetwork of the original generator can be useful for GM-DC.
AutoInfoGAN \cite{shi2023autoinfogan} applies a reinforcement learning-based neural architecture search to find the best network architecture for the generator.


%%%%%%%%%%%%%%-----Multi-Task Objectives-----%%%%%%%%%%%%%%

\subsection{
Multi-Task Objectives
}
\label{ssec:review_trainingtechniques}

\subsubsection{Regularizer} 
LeCam \cite{tseng2021lecam} uses two moving average values to track $D$'s prediction for real and generated images, denoted by $\alpha_R$ and $\alpha_F$, respectively.
Then the distance between the $D$'s prediction for real (fake) images and $\alpha_F$ ($\alpha_R$) is decreased by adding a regularizer to prevent overfitting. 
Analysis in \cite{tseng2021lecam} shows that this regularizer enforces WGAN \cite{arjovsky2017wasserstein}/ BigGAN\cite{brock2019biggan} to minimize the LeCam-divergence which is beneficial for GM-DC.
Reg-LA \cite{hou2023regularizing} uses a similar idea to regularize the label-augmented GANs discussed in Sec. \ref{ssec:review_dataaugmentation}.
DIG \cite{fang2022diggan} shows that the discriminator gradient gap between real and generated images increases when training GANs with limited data, and adds this gap as a regularizer to prevent this behavior.
MDL \cite{kong2022mdl} addresses the pre-training free few-shot image generation by adding a regularizer that aims to keep the similarities between the latent codes in $\mathcal{Z}$ space and corresponding generated images in image space.

\subsubsection{Contrastive Learning}
InsGen \cite{yang2021insgen} uses contrastive learning to improve learning $D$ by introducing a pretext task.
The pretext task is defined as instance discrimination, meaning that each sample should be mapped to a separate class. This is done by constructing the query and key from the same sample as positive pair, and all remaining images as negative pair. 
FakeCLR \cite{li2022fakeclr}  
analyze three different contrastive learning strategies, namely instance-real,  instance-fake, and instance-perturbation. It is shown that instance-perturbation contributes the most improvement in quality and can effectively alleviate the issue of latent space discontinuity.
Note that contrastive learning is also combined with other approaches as discussed before (C\textsuperscript{3} \cite{Lee2021C3}, DCL \cite{zhao2022dcl}, CtlGAN \cite{wang2022ctlgan}, CML-GAN \cite{phaphuangwittayakul2022cmlgan}, and IAG \cite{zhao2020imageaugmentation}).

\subsubsection{Masking}
MaskedGAN \cite{huang2022maskedgan} utilizes a masking idea for training GANs under limited data by masking both spatial and spectral information.
For spatial masking, they use a patch-based mask to enable random masking of all spatial parts. For spectral masking, they mask each frequency channel (extracted by Fourier transform) based on the amount of information, \ie, channels with more information are more probable to be masked.
MaskD \cite{zhu2022few} randomly masking feature maps extracted by $D$ for a few-shot setup.
DMD \cite{zhang2023dmd} detects that the discriminator slows down learning and applies random masking to its features adaptively to balance its learning pace with the generator.


\subsubsection{Knowledge Distillation}
KD-DLGAN \cite{cui2023kddlgan} proposes a knowledge distillation (KD) \cite{hinton-distill,chandrasegaran2022revisiting} approach by leveraging CLIP \cite{radford2021CLIP} as the teacher model  to distill text-image knowledge  to the discriminator.
They propose two designs: aggregated generative knowledge designs a harder learning task, and correlated generative knowledge distillation improves the generation diversity by distilling and preserving the diverse
image-text correlation from CLIP.
As discussed before, KDFSIG \cite{hou2022exploitingkd} also uses KD in the context of transfer learning for few-shot image generation.

\subsubsection{Prototype Learning}
Inspired by the recent success of learning prototypes in few-shot classification, ProtoGAN \cite{yang2023protogan}, aims to improve the fidelity and diversity of the FastGAN under limited data\cite{snell2017prototypical}.
ProtoGAN has two main modules: prototype alignment for increasing the fidelity of the generated images, and diversity loss to improve the generation diversity.
MoCA \cite{li2022moca} also learns prototypes but for different semantic concepts through an attend and replace mechanism on the extracted feature maps of $G$.

\subsubsection{Other Multi-Task Objectives}
Gen-Co \cite{cui2022genco} uses multiple discriminators to extract diverse and complementary information from samples.
This {\em co-training} has two major modules: weight-discrepancy co-training which trains separate 
$D$s with different weights, and data-discrepancy co-training 
which in addition to training separate $D$s also uses different information as inputs, \ie, spatial or frequency information.
AdaptiveIMLE \cite{aghabozorgi2023adaptiveimle} proposes an adaptive version of implicit maximum likelihood estimation \cite{li2018imle} to improve the mode coverage by assigning different boundary radii for each sample. 
PathcDiffusion \cite{wang2023patchdiffusion} and AnyResGAN \cite{chai2022anyresolution} show the effectiveness of {\em Patch-Level} learning of the generators.
Diffusion-GAN \cite{wang2023diffusiongan} leverages the {\em diffusion process} to improve training GANs by gradually increasing the task hardness for $D$.  D2C \cite{sinha2021d2c} uses a DM to improve the sampling process of VAEs by denoising the latent codes and feeding VAE with a clean latent code for sample generation.
FSDM \cite{giannone2022fsdm} uses an attentive conditioning mechanism and aggregates image patch information using a vision transformer for image generation for unseen classes.


\subsection{Exploiting Frequency Components}
\label{ssec:review_exploitingfrequency}
Approaches in this category aim to improve frequency awareness to improve GM-DC.
FreGAN \cite{yang2022fregan} extracts high-frequency information 
($HF$)
of images (related to details in images) using Haar Wavelet transform \cite{porwik2004haar} and uses three different modules to emphasize learning high-frequency information: high-frequency discriminator uses $HF$ as an additional signal to perform real/fake classification, frequency skip connection feeds the $HF$ information of each feature map to the next one in $G$ to prevent frequency loss, and a frequency alignment loss is used to make sure $G$ and $D$ are learning frequency information in the same pace.
WaveGAN \cite{yang2022wavegan} uses similar idea, but in a different setup to address task cGM-2.
Gen-Co extracts some frequency information of the image and feeds it to a separate $D$ in addition to using original real and fake images.
MaskedGAN \cite{huang2022maskedgan} masks out some frequency bands of the input during training to enforce the generative model to focus more on under-represented frequency bands.



%%%%%%%%%%%%%%%%%%%%%%%%%%%%Meta-Learning%%%%%%%%%%%%%%%%%%%%%%%%
\subsection{Meta-Learning}
\label{ssec:review_metalearning}
Meta-learning shifts the learning paradigm from data level to task level to capture across-task knowledge as {\em meta-knowledge}, and then adapt this meta-knowledge to improve the learning process of unseen tasks in the future. 
An abundant of recent works adopt meta-learning to tackle few-shot classification \cite{finn2017maml, snell2017prototypical, vinyals2016matching, sung2018relationnet, milad2021revisit} and few-shot semantic segmentation \cite{wang2019panet}.
These works usually follow the {\em episodic learning} setup which matches the way that model is trained and tested.
Considering task distribution $P_{\mathcal{T}}$, a set of training tasks are constructed from seen classes $\mathcal{T}^{train} = \{ \mathcal{T}^{train}_i \}$, where $\mathcal{T}^{train}_i$ denotes $i^{th}$ training (meta-training) task. 
The model is trained on the meta-training tasks and later tested on the meta-test tasks $\mathcal{T}^{test} = \{ \mathcal{T}^{test}_j \}$ constructed from unseen classes. Usually meta-training and meta-testing tasks follow the same distribution $P_{\mathcal{T}}$.
Similarly, the approaches in this category use meta-learning to address image generation: train a generative model on a set of few-shot image generation tasks constructed from seen classes of a domain, then, test it on the few-shot image generation tasks from unseen classes of the same domain.

\subsubsection{Optimization} 
Optimization-based meta-learning algorithms are used in these approaches for learning meta-knowledge.  
Generative Matching Network (GMN) proposes a similar attention mechanism used in Matching Networks \cite{vinyals2016matching} for few-shot image generation with variational inference. 
FIGR \cite{clouatre2019figr} meta-trains a GAN using Reptile \cite{nichol2018reptile}.
Training has an inner loop that adapts the GAN weights based on a few-shot image generation task and an outer loop that updates the meta-knowledge using Reptile.
Dawson \cite{liang2020dawson} modifies the inner loop training to directly get the gradients for the generator from evaluation data.
FAML \cite{phaphuangwittayakul2021faml} uses a similar idea to FIGR, but instead of using the standard GAN structure, it uses an encoder-decoder architecture for the generator.
CML-GAN \cite{phaphuangwittayakul2022cmlgan} extends FAML \cite{phaphuangwittayakul2021faml} by leveraging contrastive learning to learn quality representations.


\subsubsection{Fusion} 
MatchingGAN \cite{hong2020matchinggan} learns to generate new images for a category by fusing the available images of that category.
A set of encoders are used to estimate the similarity between the embedding of the latent code and input images. Then, these similarities are used as interpolation coefficients by an auto-encoder to extract the embeddings of the training images and fuse them for generating new images.
F2GAN \cite{hong2020f2gan} uses random coefficients for general information, and attention module for details. 
The attention module takes the weighted average of the real image features and the corresponding features from the decoder to produce the image details.
LoFGAN \cite{gu2021lofgan} focuses on local features in the fusion process. 
Given a batch of images, one sample is selected as a base while the remaining are utilized as a reference set. This set acts as a feature bank for the fusing process.
WaveGAN \cite{yang2022wavegan} adds frequency awareness to LofGAN by extracting and feeding the frequency components of feature maps to later layers of the generator.
AMMGAN \cite{li2023ammgan} utilizes an adaptive fusion mechanism for learning pixel-wise metric coefficients during the fusion.

\subsubsection{Transformation}
DAGAN \cite{antoniou2017dagan} leverages the task of the learning augmentation manifold in the learning process of the GAN.
This is modeled as some transformations on the input, and these transformations are applied to the new sample from the unseen classes for sample generation. 
DeltaGAN \cite{hong2022deltagan} learns the difference between images (delta) in the feature space, and then uses this delta concept for diverse sample generation. 
Disco \cite{hong2022disco} learns a dictionary based on seen images to encode input images into visual tokens.
These tokens are then fed into the decoder with the style embedding of seen images to generate images from unseen classes.
AGE \cite{ding2022age} uses GAN inversion to invert the samples of a category to $\mathcal{W}^+$ space of StyleGAN2 \cite{karras2020analyzing}. The mean latent code for all samples of a category is used as a prototype and all differences are considered as general attributes.
These attributes are then used to diversify sample generation for unseen classes.
SAGE \cite{ding2023sage} addresses the class inconsistency in AGE by taking all given samples from unseen classes into account during inference.
HAE \cite{li2022hae} uses a similar idea to AGE \cite{ding2022age}, but in the Hyperbolic space instead of using Euclidian distance which allows more semantic diversity control.
LSO \cite{zheng2023lso} finds a prototype for each class similar to AGE \cite{ding2022age}. Then it adjusts GAN to produce similar images to target samples using latent samples from the neighborhood of the prototype, followed by updating the prototype in latent space using the adapted GAN.



%%%%%%%%%%%%%%%%%%%%INTERNAL PATCH DISTRIBUTION%%%%%%%%%%%%%%%%%%
\subsection{Modeling Internal Patch Distribution}
\label{ssec:review_internalpatch}

\subsubsection{Progressive Training}
SinGAN \cite{shaham2019singan} is the pioneering work that makes use of the internal distribution of the patches within an image to train a generative model.
It trains a pyramid of generators $\{G_0, \dots, G_N \}$ against a pyramid of real images $\{x_0, \dots, x_N \}$, where $x_n$ is a downsampled version of input image $x$ by a factor of $r^n$.
The generator at scale $n$ uses random noise $z_n$ and upsampled version of the generated image from the lower resolution $\tilde{x}_{n+1}$ as input:
$\tilde{x}_{n} = G_n(z_n, (\tilde{x}_{n+1})\uparrow^{r_{n}})$.
Similarly, a pyramid of discriminators is used where $D_n$ compares the $\tilde{x}_{n}$ and $x_{n}$ in patch-level for real-fake classification.
SinDDM \cite{kulikov2023sinddm} applies the same idea but uses diffusion models with a fully convolutional lightweight denoiser.
ConSinGAN \cite{hinz2021consingan} stacks the new layers for a bigger scale on top of the previous layers used for a smaller scale instead of using separate generators for each scale. 
BlendGAN \cite{kligvasser2022blendgan} and DEff-GAN \cite{kumar2023deffGAN} extend previous approaches for learning the internal distribution for $k$ images, thereby allowing for the potential mixing of different image semantics and improving diversity.
SinDiffusion \cite{wang2022sindiffusion} addresses artifacts in SinGAN due to progressive resolution growth by applying progressive denoising using a diffusion model architecture.


\subsubsection{Non-Progressive Training}
One-Shot GAN \cite{sushko2021oneshotgan} uses a standard generator (single-scale), but multiple paths for the discriminator to enforce learning objects' appearance and how to combine them.
Within the discriminator the low-level loss is defined on low-level features and two different losses are defined to learn the content and the layout in image patches.
SinFusion \cite{nikankin2022sinfusion}  explores learning the internal patch distribution from both a single image and video. 
SinFusion extends on DPPM \cite{ho2020denoising} and reduces the size of the receptive fields by first removing attention layers, then adopting ConvNext \cite{liu2022convnet} blocks in the U-Net \cite{ronneberger2015unet} architecture. 
To reconstruct videos, a series of images are fed into a series of 3 identical models. The first model predicts the next frame; the second model denoises and removes small artifacts from the generated images; the last model interpolates between the different frames.







\section{Discussion}
\label{sec:discussion}
Here, we present an analysis of  the literature and discuss the research gap and future directions in GM-DC. 

\subsection{Analysis of the Research Landscape}
\label{ssec:landscape_analysis}

In this work, we propose a {\bf taxonomy of eight different tasks for GM-DC} (
Fig.~\ref{fig:sankey},
Tab. \ref{tab:tasktaxonomy}) based on the problem setups of GM-DC publications.
Our investigation of the literature focusing on each task (Fig.~\ref{fig:works_statistics}) reveals that a significant portion of the works (up to 
80\%)
concentrate on unconditional generation, either through training from scratch or adapting from a pre-trained model. Additionally, zero-shot unconditional generation is beginning to attract more attention.
Similarly, adaptation for  in-domain classes has garnered considerable interest for conditional generation.
Meanwhile, conditional generation
for out-of-domain classes
via adaptation 
has not been 
explored adequately.
Furthermore, subject-driven generation, which enables more control over content generation, is an emerging task. 
We anticipate increasing interest on this task as recent text-to-image generative models become more accessible.

We further present a {\bf taxonomy of approaches for GM-DC} (Fig.~\ref{fig:sankey},
Tab.~\ref{tab:approaches}) as our another contribution. Our study reveals that transfer learning is a predominant solution for GM-DC, capable of tackling a large number of tasks (specifically, 5 out of 8 tasks, as indicated in Tab.~\ref{tab:approaches} and Fig.~\ref{fig:sankey}), while effectively handling all data constraints including limited data, few-shot, and zero-shot. 
Moreover, $\approx$39\% of the studies propose new methods based on  transfer learning (Fig.~\ref{fig:works_statistics}). More than 
20\% of the studies propose methods
based on other approaches 
that are compatible to transfer learning, \eg data augmentation.
These methods could be used with transfer learning-based methods to improve performance.
The primary challenges in transfer learning are
selection and 
preservation of  source knowledge 
useful for generating 
high-quality and diverse target domain samples. 
Adaptation-aware approach \cite{zhao2022adam, zhao2023rick}
could be a sound direction  in this aspect where they consider both source and target domains (the adaptation process) for knowledge preservation.
Language-guided approaches \cite{gal2022stylegannada, zhu2022mindthegap, alanov2022hyperdomainnet, guo2023ipl} are gaining increasing attention due to their ability to facilitate zero-shot generation through appropriate application of vision-language models during the transfer learning phase.
Visual prompt tuning \cite{sohn2023vpt} is a recent method, which guides the generation of target domain samples  by generating visual tokens. 



Data augmentation \cite{karras2020ada, zhao2020diffaug, tran2021dag} remains a potent technique in GM-DC where it boosts performance under limited data by increasing coverage of the data distribution through various transformations.
Multi-task objectives
\cite{yang2021insgen, tseng2021lecam, huang2022maskedgan} 
which incorporate additional learning objectives 
are usually complementary to data augmentation.
Various network architecture designs \cite{liu2021fastgan, li2022moca} that aim to prevent overfitting or preserve the feature maps are also shown to be 
significantly
effective for GM-DC.
Given that generative models tend to exhibit biases in capturing frequency components, enhancing the frequency awareness in these models is an emerging direction for GM-DC \cite{yang2022fregan}.
Meta-learning \cite{clouatre2019figr} enables generative models to learn
inter-task 
knowledge from seen classes, and then handle new generation tasks from unseen classes usually without fine-tuning \cite{gu2021lofgan, hong2022deltagan}.
Internal patch-distribution modeling  \cite{shaham2019singan, nikankin2022sinfusion} effectively trains a generative model from scratch using a single reference image (scene) to produce novel scene compositions.

Regarding the types of generating models, 
our study shows that around 86\% of the GM-DC works
focus on GANs
(Fig.~\ref{fig:works_statistics}).
This preference can be attributed to the extensive research in GANs.
Recently, there has been a growing interest in DMs (12\%) 
and VAEs (3\%), particularly VQ-VAE, driven by the success of DMs \cite{ramesh2022dalle2, saharia2022imagen} and transformer-based token prediction methods in generative modeling \cite{chang2022maskgit, esser2021vqgan}.
We anticipate increasing attention directed toward DMs and VQ-VAEs.
Furthermore, our survey reveals an interesting trend: around 
64\% of the works focus on addressing the challenging task of few-shot learning, while 
33\% concentrate on limited data scenarios. 
While only 3\% of works address zero-shot learning, we expect growing interest due to recent advancements in vision-language models \cite{kwon2022oneclip, li2023scaling}.


\subsection{Research Gap and Future Directions}
\label{ssec:future_direction}




\subsubsection{Harnessing the power of foundation models}
As previously discussed, transfer learning is a prominent and highly effective solution for GM-DC.
Nevertheless, the majority of existing literature uses pre-trained StyleGAN2 (FFHQ) or BigGAN (ImageNet) networks as source models.
A potential future direction for GM-DC is to explore the capabilities of foundation models \cite{bommasani2021opportunities},  \ie large models trained using massive amounts of data.
In particular, recent text-image generation models including DALL$\cdot$E-2 \cite{ramesh2022dalle2} ($\approx$3.5B parameters), Imagen \cite{saharia2022imagen} ($\approx$4.6B parameters) and Stable Diffusion \cite{rombach2022latentdiffusion} ($\approx$890M parameters) encode knowledge regarding a wide range of concepts for high-quality, diverse image generation.
Leveraging such foundation models for GM-DC is relatively under-explored.




\subsubsection{Grounding zero-shot image generative capabilities}
Recent studies have demonstrated the feasibility of zero-shot image generation for 
well-known concepts, \eg ``Tolkien Elf'' \cite{gal2022stylegannada}. 
However, grounding zero-shot image generation models to generate evolving/ new semantic concepts remains a relatively unexplored and challenging area. 
For instance, how to generate an image depicting ``The coronation of Charles III and Camilla as King and Queen of the United Kingdom,'' an event that occurred in May 2023, that related images may not be captured by existing models. 
This requires strategies that allow continual learning, semantic concept editing, and the incorporation of temporal contexts.


\subsubsection{
Knowledge transfer 
for distant/ remote target domains}
Knowledge transfer has received significant attention in GM-DC research. Many works concentrate on utilizing pre-trained knowledge
of a source domain
to enhance learning in the target domain, as evident from the statistics in Fig. \ref{fig:sankey} and Fig. \ref{fig:works_statistics}. However, we remark that exploring knowledge transfer for modeling
target domains which are distant/ remote from the source domains 
still remains largely unexplored.
This problem is challenging, as demonstrated in our experiment to transfer knowledge from Human Faces $\rightarrow$ Flowers 
(Fig. \ref{fig:proximity-measurements-and-10-shot-flowers}), which  clearly demonstrates the complexity of the task.
We urge more investigation in knowledge transfer for modeling distant/ remote target domains in GM-DC research.


\subsubsection{Exploring different types of generative models for GM-DC}
Our analysis reveals that around 
85\% 
of GM-DC works focus on GANs and there is less attention to other types of generative models.
Meanwhile, 
recent generative models such as diffusion models have made a lot of progress, achieving comparable performance to GANs in terms of quality and diversity of generated samples
\cite{dhariwal2021diffusionvsGAN}.
We remark that recent generative models are fundamentally different from GANs, \eg multiple iterations are required to generate samples in diffusion models, suggesting that current GM-DC methods developed for GANs could be  sub-optimal for other types of 
generative models.




\subsubsection{Holistic evaluation of GM-DC}
Evaluation of GM-DC presents multiple challenges including difficulties in estimating real data statistics under low-data regimes, lack of unified framework for human evaluation of GM-DC samples, and 
heavy reliance 
on
particular 
(pre-trained) feature extractors to quantify the capabilities of GM-DC. 
In particular, developing holistic evaluation frameworks integrating both objective measurements and subjective judgements tailored for different tasks is essential for understanding GM-DC capabilities.
Advancing holistic evaluation
is important for 
GM-DC methods to be applied in a 
variety of real-world scenarios.



\subsubsection{Data-centric approaches for GM-DC}
We remark that data-centric  approaches
\cite{whang2023datacollection}
for advancing  GM-DC  have been relatively overlooked in the literature.
Majority of GM-DC methods focus on advancing training procedures based on a given set of training samples, but little attention has been put on how GM-DC performance may be affected by characteristics of the given training samples.
Particularly, for 
GM-DC problems, where a domain is described using limited training samples, the characteristics of the  samples can have noticeable impact on performance of GM-DC methods, as hinted in our analysis (see Fig.
\ref{fig:challenges-data-selection}).
We suggest  
 greater emphasis on data collection, curation and pre-processing for GM-DC advancement. 


\subsection{Beyond Image Generation}

Existing GM-DC works focus on image generation primarily.
There are a few recent works to study other data types.
\cite{zhu2023fewshot_3d} studies {\em 3D shape generation} under few-shot target data (10-shot) utilizing pre-trained 3D generative models and optimization adaptation to retain the probability distributions of pairwise adapted samples. 
CLIP-Sculptor \cite{sanghi2023clipsculptor} leverages CLIP guidance for  zero-shot
{\em
shape generation}. 
\cite{wang2023cffont} studies few-shot {\em font generation} which aims to transfer the source domain style to the target domain. In particular, they introduce a content fusion module and a projected character loss to improve the quality of skeleton transfer for few-shot font generation.
\cite{careil2023few} explores the problem of few-shot {\em semantic image generation} where the objective is to generate realistic images based on semantic segmentation maps. Their approach employs transfer learning on both GANs and DMs for few-shot semantic image synthesis. 





\section{Conclusion}
\label{sec:conclusion}

Generative Modeling under Data Constraints (GM-DC) is a burgeoning research area. 
This survey delves into this field by meticulously examining
research papers in this area, 
encompassing 
different types of
generative models including VAEs, GANs, and Diffusion Models. 
Drawing from this analysis, we identify  several challenges encountered in GM-DC, including those related to training, data selection, and model evaluation. 
Moreover, we propose two taxonomies to categorize works related to  GM-DC: a task taxonomy that
identifies the variety  of 
generation 
tasks, and an approach taxonomy that categorizes 
the  extensive list of solutions for these tasks. 
We present a Sankey diagram to illuminate the interactions between different GM-DC tasks, approaches, and methods.
Additionally, we present
an organized review 
of existing GM-DC works and discuss  research gaps and  future research directions. 
Our aspiration is that this survey not only could offer valuable insights to researchers but also help spark further advancements in GM-DC.


\vspace{0.2cm}
\noindent
\textbf{Ethics Statement.}
Generative models could be mis-used to disseminate mis- and disinformation 
due to their ability to generate realistic content.
In particular, advanced generative models could be mis-used by malicious users to fabricate deepfake images,  
portraying individuals engaging in actions they never actually performed.
Advances in GM-DC could exacerbate the situation as it becomes possible to 
generate realistic content with less data.
We advocate for ethical and responsible usage of GM-DC methods and studying of  mitigation techniques \cite{mirsky2021creation, chandrasegaran2021closer, Chandrasegaran_2022_ECCV, zhao2023recipe, wen2023tree}.






\bibliographystyle{ACM-Reference-Format}
%\citestyle{acmnumeric}
% \bibliography{references}
% \bibliographystyle{ieeetr}
\bibliography{references}

                        




\end{document}
