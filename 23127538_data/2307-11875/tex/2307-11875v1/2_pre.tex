\section{Preliminaries} \label{sec:pre}



\subsection{Quantum state and visualization}

A qubit (short for a quantum bit) is the information carrier in the quantum computing/communication channel \cite{kilin2001quantum}. A qubit is defined as a two-dimensional Hilbert space with two orthonormal bases $|0\rangle$ and $|1\rangle$, which are known as computational bases in two-level quantum computing. These computational bases are usually represented as vectors $|0\rangle = [1, 0] ^\top$ and $|1\rangle = [0,1]^\top$. Due to the unique qubit characteristic of \textit{superposition}, the state of a qubit can be represented as the sum of two computational bases weighted by (complex) amplitudes:
\begin{align}
    |\psi \rangle = \alpha |0\rangle + \beta |1\rangle&= \begin{bmatrix}
            \alpha \\
           \beta 
         \end{bmatrix}
\end{align}
where $\alpha$ and $\beta  \in \mathbb{C}$, and $|\alpha| ^ 2 + |\beta|^2 = 1$. $|\alpha| ^ 2$ and $|\beta|^2$ are the probability of obtaining states $|0\rangle$ and $|1\rangle$ after multiple measurements, respectively. 
% Typically, a quantum system containing $n$ qubits is in a $2^n$-dimensional Hilbert space. Then a general form of a quantum system state is 
% \begin{align}
%     |\psi \rangle = \sum^{2^n}_{i=1} \alpha_i |x_i\rangle
% \end{align}
% where $\alpha_i \in \mathbb{C}$ and $\sum^{2^n}_{i=1} |\alpha_i|^2 = 1$. 

The Bloch sphere is a valuable tool for visualizing the state of a single qubit, as shown in Fig.~\ref{fig:block}. It encompasses all possible states of a qubit, making it an excellent representation of a two-dimensional Hilbert space. In this article, we will use the Bloch sphere to illustrate our proposed method clearly. Every pure state of a qubit can be mapped to a distinct point on the surface of the Bloch sphere, whereas mixed states correspond to points within the sphere. The state of a qubit on the Block sphere can be described with two real parameters, $\theta$ and $\phi$,
\begin{align}
    |\psi \rangle = \cos{\frac{\theta}{2}} |1\rangle + e^{i \phi} \sin{\frac{\theta}{2}} |1\rangle
\end{align}
where $\theta \in [0, \pi]$ and $\phi \in [0, 2\pi]$. I.e., $\theta=0$ for $|0\rangle$ and $\theta=\pi$ for $|1\rangle$, and global phase $\phi$ can be any value.




% \subsection{Bloch Sphere}

% The Bloch sphere, named after the physicist Felix Bloch \cite{bloch1946nuclear}, is a geometrical representation of a single qubit state. All possible states of a qubit form the Bloch sphere. Thus, the Bloch sphere offers a clear representation of a two-dimensional Hilbert space, and will be utilized to clarify the details of the proposed method in this article. Each pure state corresponds to a point on the surface of the Bloch sphere, while each mixed state corresponds to an interior point. On the Bloch sphere, the orthogonal states $|0\rangle$ and $|1\rangle$ are antiparallel, i.e. they are located in the north (+z) and south (-z) poles of the Bloch sphere, as shown in Fig.~\ref{fig:block}. The state of a qubit on the Block sphere can be described with two real parameters, $\theta$ and $\phi$,
% \begin{align}
%     |\psi \rangle = \cos{\frac{\theta}{2}} |1\rangle + e^{i \phi} \sin{\frac{\theta}{2}} |1\rangle
% \end{align}
% where $\theta \in [0, \pi]$ and $\phi \in [0, 2\pi]$. I.e., $\theta=0$ for $|0\rangle$ and $\theta=\pi$ for $|1\rangle$, and global phase $\phi$ can be any value. In addition, when $\theta =\pi /2$, the x- and y-axes include two groups of orthogonal states:
% \begin{equation}
%     \begin{split}
%         +x: \phi = 0,\ &|+\rangle = \frac{1}{\sqrt{2}}(|0\rangle + |1\rangle)\\
%         -x: \phi = \pi,\ &|+\rangle = \frac{1}{\sqrt{2}}(|0\rangle - |1\rangle)\\
%         +y: \phi = \frac{\pi}{2},\ &|i+\rangle = \frac{1}{\sqrt{2}}(|0\rangle+i|1\rangle)\\
%         -y: \phi = \frac{3\pi}{2},\ &|i-\rangle = \frac{1}{\sqrt{2}}(|0\rangle - i|1\rangle)
%     \end{split}
% \end{equation}




% Figure environment removed





% The axis of the Bloch sphere corresponding to the Pauli operators.





\subsection{Quantum measurement}
Quantum measurement is the retrieval of the numerical information stored in a qubit. A measurement result is +1 for state $|0\rangle$ and -1 for state $|1\rangle$ according to a specified probability distribution associated with the quantum state. Therefore, numerous measurements are required to determine the exact quantum state. The final result of the quantum measurement is the expected value of all outcomes.
% However, the state of the qubit is changed after once measurement, so multiple identical quantum states should be prepared and measured. 



Observables are used to understand the properties of a quantum system and can be measured. Mathematically, observables are formulated as Hermitian operators that map Hilbert space onto themselves. For a valid observable, its eigenvalues are real numbers and can be the outcomes of measurement. Moreover, observables can form an orthonormal basis of the target Hilbert space, which will be the state of the quantum system after measurement. The observables considered in this article are Pauli matrices:
\begin{equation}
    \sigma_x= \left[ \begin{array}{ccc}
    0 & 1\\
    1 & 0\end{array} \right],
    \sigma_y= \left[ \begin{array}{ccc}
    0 & -i\\
    i & 0\end{array} \right],
    \sigma_z = \left[ \begin{array}{ccc}
    1 & 0\\
    0 & -1\end{array} \right].
\end{equation}
These Pauli matrices span a complex two-dimensional Hilbert space (a qubit). The projection measurement is to extract quantum information by operating on the interested observable and the density matrix of the target quantum state
\begin{equation}
    \langle \sigma \rangle = Tr(\sigma |\psi\rangle \langle \psi|)
\end{equation}
where $|\psi\rangle \langle \psi|$ generates the density matrix.
For the general z-measurement, the state vector is projected onto the z-axis of the Bloch sphere, and the corresponding value on the z-axis is the expectation of the measurement results, as shown in Fig.~\ref{fig:block}.


\subsection{Variational quantum algorithm}

The variational quantum algorithm (VQA) is the standard approach to performing QNN. It processes prepared quantum information by applying a series of parametric quantum gates, ultimately producing an output through measurement. As an example, a binary quantum classifier is implemented using VQA in \cite{farhi2018classification}. It has an ($n$+1)-qubit circuit, where the first $n$ qubits are prepared using an encoding method (such as angle, basis, or amplitude encoding) to represent specific information. The final qubit acted as a readout, generating the output through measurements. These qubits then pass a sequence of quantum gates with trainable parameters $U(\theta) = \prod_{l=1}^N U_l(\theta_l)$, where $\theta$ is a set of parameters.  A measurement outcome of 1 corresponds to one class, while a result of -1 corresponds to the other class. 
VQA uses a hybrid quantum-classical procedure to iteratively optimize the trainable parameters.
% , aiming at minimizing the loss value associated with the objective function of interest. 
The popular optimization approach includes gradient descent \cite{sweke2020stochastic}, parameter shift \cite{wierichs2022general}, and gradient-free techniques, such as COBYLA. All of the methods take the training data as input and evaluate the model performance by comparing the generated and correct labels. Based on this evaluation, the methods update the model parameters for the next round, repeating the process until the model converges and achieves the desired performance. The hybrid method performs the evaluation and parameter selection on a classical machine, while the model inference is carried out on a quantum machine.








% \subsubsection{Projective measurement}

% Projective measurement is the primary measurement method adopted by applications of quantum computing. It operates on the interested observables and the density matrix of the target quantum state. Density matrix \cite{fano1957description} is a general way to represent the state of a quantum system. For a quantum system with state vector $|\psi\rangle$, its density matrix is defined as the outer product of $|\psi\rangle$ with itself
% \begin{equation}
%     \rho \coloneqq |\psi\rangle \langle \psi|
% \end{equation}

% The density matrix is a Hermitian matrix with trace one.

% - trace() = 1

% - $<\sigma >$ = Tr($o_{xyz}\rho$) ---- cartesian coordinate system

% - quantum state tomography (multiple measurement)

% - hardware implementation







