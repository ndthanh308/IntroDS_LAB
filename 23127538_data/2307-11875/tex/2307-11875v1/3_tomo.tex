\section{Quantum state tomography} \label{sec:tomo}

Quantum state tomography (QST) is a technique to reconstruct an unknown quantum state using its measurement results \cite{toninelli2019concepts}. The measured observables must form a basis in the Hilbert space so that all state information can be recorded and used to recover the state. In the present age of NISQ, when the number of qubits is limited, QST is an essential method for retrieving the complete information stored in a quantum system. Nevertheless, as the number of qubits grows, the number of measurements needed and the complexity of state reconstruction increase exponentially. Hence, for the sake of simplicity, we consider the reconstruction of a single qubit using three observables to restore its state in a two-dimensional Hilbert space in this work.
% For reasons of efficiency and intuition, we consider the reconstruction of a single qubit and restore its state on the Bloch sphere in this work.


% Figure environment removed

Any arbitrary density matrix of a 1-qubit state can be expressed as a linear combination of Pauli matrices (basis of two-dimensional Hilbert space) as
\begin{equation}
\begin{split}
    \rho &= \frac{1}{2}(I + r_x \sigma_x + r_y \sigma_y + r_z \sigma_z)\\
    &= \frac{1}{2}\left[ \begin{array}{ccc}
    r_z + 1 & r_x - r_y\\
    r_x + r_y & -r_z + 1\end{array} \right]
\end{split}
\end{equation}
where $r$ denotes real number and $r_x^2 + r_y^2 + r_z^2 = 1$. For the typical qubit measurement using observable $\sigma_z$, as used by most quantum applications, the expectation of the measurement result is 
\begin{equation} 
\label{eq:obz}
 \langle \sigma_z \rangle = Tr\bigg(\frac{1}{2}(I + r_x \sigma_x + r_y \sigma_y + r_z \sigma_z) \sigma_z \bigg) = r_z
\end{equation}
Eq.~\ref{eq:obz} demonstrates that the expectation of measurement results is directly related to the density matrix $\rho$ of interest. In this case, however, only diagonal entries of $\rho$ can be retrieved, while some useful information remains untouched. Hence, in order to rebuild the density matrix $\rho$ completely, the measurements on observables $\sigma_x$ and $\sigma_y$ are necessary to obtain $r_x$ and $r_y$. 
 
From the perspective of the geometric representation, the expectation values $\langle \sigma_x \rangle$, $\langle \sigma_y \rangle$ and $\langle \sigma_z \rangle$ are exactly the projections of the state vector on the x-, y- and z-axis of the Bloch sphere, respectively. Therefore, after getting sufficient measurement results, we can restore the quantum state in the Bloch sphere for an intuitive interpretation. In other words, the interested 1-qubit state is a state vector with an unknown direction in the Bloch sphere. By projecting the vector on the three axes, we can figure out the direction of the state vector, which contains all information about the state. 




