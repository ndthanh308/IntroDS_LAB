\documentclass[conference]{IEEEtran}
\IEEEoverridecommandlockouts
% The preceding line is only needed to identify funding in the first footnote. If that is unneeded, please comment it out.
\usepackage{cite}
\usepackage{amsmath,amssymb,amsfonts}
\newcommand{\overbar}[1]{\mkern 1.5mu\overline{\mkern-1.5mu#1\mkern-1.5mu}\mkern 1.5mu}

\usepackage{algorithmic}
\usepackage{booktabs}
\usepackage{graphicx}
\usepackage{textcomp}
\usepackage{xcolor}
\usepackage{mathtools}
\usepackage[htt]{hyphenat}
\usepackage[hyphens]{url}
\PassOptionsToPackage{hyphens}{url}\usepackage{hyperref}
% \documentclass{article}
\def\BibTeX{{\rm B\kern-.05em{\sc i\kern-.025em b}\kern-.08em
    T\kern-.1667em\lower.7ex\hbox{E}\kern-.125emX}}
\begin{document}

\title{MORE: Measurement and Correlation Based Variational Quantum Circuit for Multi-classification\\
% {\footnotesize \textsuperscript{*}Note: Sub-titles are not captured in Xplore and
% should not be used}
% \thanks{Identify applicable funding agency here. If none, delete this.}
}

\author{\IEEEauthorblockN{Jindi Wu}
\IEEEauthorblockA{\textit{Department of Computer Science} \\
\textit{William \& Mary}\\
Williamsburg, VA, USA \\
jwu21@wm.edu}
\and
\IEEEauthorblockN{Tianjie Hu}
\IEEEauthorblockA{\textit{Department of Computer Science} \\
\textit{William \& Mary}\\
Williamsburg, VA, USA \\
thu04@wm.edu}
\and
\IEEEauthorblockN{Qun Li}
\IEEEauthorblockA{\textit{Department of Computer Science} \\
\textit{William \& Mary}\\
Williamsburg, VA, USA \\
liqun@cs.wm.edu}

}

\maketitle

\begin{abstract}
Quantum computing has shown considerable promise for compute-intensive tasks in recent years. For instance, classification tasks based on quantum neural networks (QNN) have garnered significant interest from researchers and have been evaluated in various scenarios. However, the majority of quantum classifiers are currently limited to binary classification tasks due to either constrained quantum computing resources or the need for intensive classical post-processing. In this paper, we propose an efficient quantum multi-classifier called \textbf{MORE}, which stands for \underline{m}easurement and c\underline{or}r\underline{e}lation based variational quantum multi-classifier. MORE adopts the \textit{same} variational ansatz as binary classifiers while performing multi-classification by fully utilizing the quantum information of a \textit{single} readout qubit. To extract the complete information from the readout qubit, we select three observables that form the basis of a two-dimensional Hilbert space. We then use the quantum state tomography technique to reconstruct the readout state from the measurement results. Afterward, we explore the correlation between classes to determine the quantum labels for classes using the variational quantum clustering approach. Next, quantum label-based supervised learning is performed to identify the mapping between the input data and their corresponding quantum labels. Finally, the predicted label is determined by its closest quantum label when using the classifier. We implement this approach using the Qiskit Python library and evaluate it through extensive experiments on both noise-free and noisy quantum systems. Our evaluation results demonstrate that MORE, despite using a simple ansatz and limited quantum resources, achieves advanced performance.
\end{abstract}

\begin{IEEEkeywords}
quantum machine learning, quantum multi-classifier, variational quantum algorithm, hybrid quantum-classical method, observable-based method
\end{IEEEkeywords}




\section{Introduction}

% Figure environment removed

Reinforcement Learning from Human Feedback (RLHF) has recently been used to great effect to align pretrained large language models (LLMs) to human preferences, optimizing for desirable qualities like harmlessness and helpfulness~\citep{bai2022training} and achieving state-of-the-art results across a variety of natural language tasks~\citep{openai2023gpt4}. %RLHF approaches fundamentally rely on collecting pairs of LLM outputs $(o_1, o_2)$ from a shared prompt $p$, with a human indicating which output in each pair is better on a specified attribute.
% A fundamental component of RLHF is a preference model derived from human labels, typically formatted as pairs of LLM outputs $(o_1, o_2)$ generated from a shared prompt $p$.

A standard RLHF procedure fine-tunes an initial unaligned LLM using an RL algorithm such as PPO~\citep{schulman2017proximal}, optimizing the LLM to align with human preferences. %\violet{not sure whether we need to provide this detail in the intro, especially this has nothing to do with our contribution.} % i feel like this context is useful later when e.g. explaining that context distillation is SFT
RLHF is thus critically dependent on a reward model derived from human-labeled preferences, typically \textit{pairwise preferences} on LLM outputs $(o_1, o_2)$ generated from a shared prompt $p$. % and labeled by humans. 

However, collecting human pairwise preference data, especially high-quality data, may be expensive and time consuming at scale. To address this problem, approaches have been proposed to obtain labels without human annotation, such as Reinforcement Learning from AI Feedback (RLAIF) and context distillation. 

\iffalse
raising the question of whether we can generate high-quality data for RLHF without using human labeling. %accurately-labeled preference pairs $(o_1, o_2)$
%, motivating model alignment approaches that aim to generate accurately-labeled preference pairs $(o_1, o_2)$ without human involvement. 
Two major categories of such approaches are . 
\fi

RLAIF approaches (e.g.,~\citet{bai2022constitutional}) simulate human pairwise preferences by scoring $o_1$ and $o_2$ with an LLM (Figure \ref{fig:rlcd_differences} center); the scoring LLM is often the same as the one used to generate the original pairs $(o_1, o_2)$. Of course, the resulting LLM pairwise preferences will be somewhat noisier compared to human labels. However, this problem is exacerbated by using the same prompt $p$ to generate both $o_1$ and $o_2$, causing $o_1$ and $o_2$ to often be of very similar quality and thus hard to differentiate (e.g., Table~\ref{tab:rlaif_bad_example}). Consequently, training signal can be overwhelmed by label noise, yielding lower-quality preference data. 

% While it avoids human labeling efforts, it has weakness. First, LLM preference labels will naturally be somewhat noisier compared to human labels. Furthermore, since the same prompt $p$ is used to generate both $o_1$ and $o_2$, their quality is often very similar and hard to differentiate (See Table~\ref{tab:rlaif_bad_example}). As a result, training signals can be overwhelmed by label noise, yielding lower-quality preference data. 

Meanwhile, context distillation methods (e.g., \citet{sun2023principle}) create more training signal by modifying the initial prompt $p$. 
%to create more significant training signal. 
The modified prompt $p_+$ typically contains additional context encouraging a \textit{directional attribute change} in the output $o_+$ (Figure \ref{fig:rlcd_differences} right). However, context distillation methods only generate a single output $o_+$ per prompt $p_+$, which is then used for supervised fine-tuning, losing the pairwise preferences which help RLHF-style approaches to 
%rather than using a RLHF-style preference model to 
derive signal from the contrast between outputs. 
Multiple works have observed that RL approaches using preference models for pairwise preferences can substantially improve over supervised fine-tuning by itself when aligning LLMs~\citep{ouyang2022training,dubois2023alpacafarm}. 

% conduct alignment by running supervised fine-tuning on model outputs $o_+$ generated from a modified prompt $p_+$. $p_+$ typically contains additional context encouraging desirable attributes (Figure \ref{fig:rlcd_differences} right), such as in \citet{sun2023principle}. However, multiple works have observed that RLHF-style approaches can substantially improve over supervised fine-tuning by itself when aligning LLMs~\citep{ouyang2022training,dubois2023alpacafarm}. 

Therefore, while both RLAIF and context distillation approaches have already been successfully applied in practice to align language models, we posit that it may be even more effective to combine the key advantages of both. That is, we will use RL with \textit{pairwise preferences}, while also using modified prompts to encourage \textit{directional attribute change} in outputs. %In particular, we will adapt the RLAIF data generation process with two different prompts rather than a single $p$, modifying both prompts similarly to context distillation. %\violet{this motivation is a little unexciting. I think we can more specifically discuss the potential benefits of our approach, like the benefits from RL: exploration/data generation; benefits from contrast. I don't think we get too much benefits from context distillation since we switched to the RL framework.} 

Concretely, we propose \oursfull{} (\ours{}). 
\ours{} generates preference data as follows. Rather than producing two i.i.d.\ model outputs $(o_1, o_2)$ from the same prompt $p$ as in RLAIF, \ours{} creates two variations of $p$: a \textit{positive prompt} $p_+$ similar to context distillation which encourages directional change toward a desired attribute, and a \textit{negative prompt} $p_-$ which encourages directional change \textit{against} it (Figure \ref{fig:rlcd_differences} left). We then generate model outputs $(o_+, o_-)$ respectively, and automatically label $o_+$ as preferred---that is, \ours{} automatically ``generates'' pairwise preference labels by construction. %, without further post hoc labeling.\violet{should make it clearer that our approach `generates' labels by construction} 
We then follow the standard RL pipeline of training a preference model followed by PPO. 

Compared to RLAIF-generated preference pairs $(o_1, o_2)$ from the same input prompt $p$, there is typically a clearer difference in the quality of $o_+$ and $o_-$ generated using \ours{}'s directional prompts $p_+$ and $p_-$, which may result in less label noise. %which may result in better training signal for the preference model. 
That is, intuitively, \ours{} exchanges having examples be \textit{closer to the classification boundary} for much more \textit{accurate labels} on average. Compared to standard context distillation methods, on top of leveraging pairwise preferences for RL training, \ours{} can derive signal not only from the positive prompt $p_+$ which improves output quality, but also from the negative prompt $p_-$ which degrades it. %\ours{} is not learning to imitate $o_+$, but to distill the \textit{contrast} between $o_+$ and $o_-$. 
Positive outputs $o_+$ don't need to be perfect; they only need to contrast with $o_-$ on the desired attribute while otherwise following a similar style.

% \todo{discuss our method and why intuitively it may be better.}

We evaluate the practical effectiveness of \ours{} through both human and automatic evaluations on three tasks, aiming to improve the ability of LLaMA-7B~\citep{touvron2023llama} to generate harmless outputs, helpful outputs, and high-quality story outlines. %\ours{} outperforms both RLAIF and context distillation baselines in pairwise comparisons on 
As shown in Sec. \ref{sec:experiments}, \ours{} substantially outperforms both RLAIF and context distillation baselines in pairwise comparisons when simulating preference data with LLaMA-7B, while still performing equal or better when simulating with LLaMA-30B. 
%On all three tasks, \ours{} substantially outperforms both RLAIF and context distillation baselines in pairwise comparisons---by a margin of at least 9\% and often more than 30\%---validating our method's efficacy. 
We will release all code at a later date, although in any case \ours{} is fairly easy to implement by modifying any reference RLAIF codebase. %We release all code at \todo{github link}.
\section{Preliminaries} \label{sec:pre}



\subsection{Quantum state and visualization}

A qubit (short for a quantum bit) is the information carrier in the quantum computing/communication channel \cite{kilin2001quantum}. A qubit is defined as a two-dimensional Hilbert space with two orthonormal bases $|0\rangle$ and $|1\rangle$, which are known as computational bases in two-level quantum computing. These computational bases are usually represented as vectors $|0\rangle = [1, 0] ^\top$ and $|1\rangle = [0,1]^\top$. Due to the unique qubit characteristic of \textit{superposition}, the state of a qubit can be represented as the sum of two computational bases weighted by (complex) amplitudes:
\begin{align}
    |\psi \rangle = \alpha |0\rangle + \beta |1\rangle&= \begin{bmatrix}
            \alpha \\
           \beta 
         \end{bmatrix}
\end{align}
where $\alpha$ and $\beta  \in \mathbb{C}$, and $|\alpha| ^ 2 + |\beta|^2 = 1$. $|\alpha| ^ 2$ and $|\beta|^2$ are the probability of obtaining states $|0\rangle$ and $|1\rangle$ after multiple measurements, respectively. 
% Typically, a quantum system containing $n$ qubits is in a $2^n$-dimensional Hilbert space. Then a general form of a quantum system state is 
% \begin{align}
%     |\psi \rangle = \sum^{2^n}_{i=1} \alpha_i |x_i\rangle
% \end{align}
% where $\alpha_i \in \mathbb{C}$ and $\sum^{2^n}_{i=1} |\alpha_i|^2 = 1$. 

The Bloch sphere is a valuable tool for visualizing the state of a single qubit, as shown in Fig.~\ref{fig:block}. It encompasses all possible states of a qubit, making it an excellent representation of a two-dimensional Hilbert space. In this article, we will use the Bloch sphere to illustrate our proposed method clearly. Every pure state of a qubit can be mapped to a distinct point on the surface of the Bloch sphere, whereas mixed states correspond to points within the sphere. The state of a qubit on the Block sphere can be described with two real parameters, $\theta$ and $\phi$,
\begin{align}
    |\psi \rangle = \cos{\frac{\theta}{2}} |1\rangle + e^{i \phi} \sin{\frac{\theta}{2}} |1\rangle
\end{align}
where $\theta \in [0, \pi]$ and $\phi \in [0, 2\pi]$. I.e., $\theta=0$ for $|0\rangle$ and $\theta=\pi$ for $|1\rangle$, and global phase $\phi$ can be any value.




% \subsection{Bloch Sphere}

% The Bloch sphere, named after the physicist Felix Bloch \cite{bloch1946nuclear}, is a geometrical representation of a single qubit state. All possible states of a qubit form the Bloch sphere. Thus, the Bloch sphere offers a clear representation of a two-dimensional Hilbert space, and will be utilized to clarify the details of the proposed method in this article. Each pure state corresponds to a point on the surface of the Bloch sphere, while each mixed state corresponds to an interior point. On the Bloch sphere, the orthogonal states $|0\rangle$ and $|1\rangle$ are antiparallel, i.e. they are located in the north (+z) and south (-z) poles of the Bloch sphere, as shown in Fig.~\ref{fig:block}. The state of a qubit on the Block sphere can be described with two real parameters, $\theta$ and $\phi$,
% \begin{align}
%     |\psi \rangle = \cos{\frac{\theta}{2}} |1\rangle + e^{i \phi} \sin{\frac{\theta}{2}} |1\rangle
% \end{align}
% where $\theta \in [0, \pi]$ and $\phi \in [0, 2\pi]$. I.e., $\theta=0$ for $|0\rangle$ and $\theta=\pi$ for $|1\rangle$, and global phase $\phi$ can be any value. In addition, when $\theta =\pi /2$, the x- and y-axes include two groups of orthogonal states:
% \begin{equation}
%     \begin{split}
%         +x: \phi = 0,\ &|+\rangle = \frac{1}{\sqrt{2}}(|0\rangle + |1\rangle)\\
%         -x: \phi = \pi,\ &|+\rangle = \frac{1}{\sqrt{2}}(|0\rangle - |1\rangle)\\
%         +y: \phi = \frac{\pi}{2},\ &|i+\rangle = \frac{1}{\sqrt{2}}(|0\rangle+i|1\rangle)\\
%         -y: \phi = \frac{3\pi}{2},\ &|i-\rangle = \frac{1}{\sqrt{2}}(|0\rangle - i|1\rangle)
%     \end{split}
% \end{equation}




% Figure environment removed





% The axis of the Bloch sphere corresponding to the Pauli operators.





\subsection{Quantum measurement}
Quantum measurement is the retrieval of the numerical information stored in a qubit. A measurement result is +1 for state $|0\rangle$ and -1 for state $|1\rangle$ according to a specified probability distribution associated with the quantum state. Therefore, numerous measurements are required to determine the exact quantum state. The final result of the quantum measurement is the expected value of all outcomes.
% However, the state of the qubit is changed after once measurement, so multiple identical quantum states should be prepared and measured. 



Observables are used to understand the properties of a quantum system and can be measured. Mathematically, observables are formulated as Hermitian operators that map Hilbert space onto themselves. For a valid observable, its eigenvalues are real numbers and can be the outcomes of measurement. Moreover, observables can form an orthonormal basis of the target Hilbert space, which will be the state of the quantum system after measurement. The observables considered in this article are Pauli matrices:
\begin{equation}
    \sigma_x= \left[ \begin{array}{ccc}
    0 & 1\\
    1 & 0\end{array} \right],
    \sigma_y= \left[ \begin{array}{ccc}
    0 & -i\\
    i & 0\end{array} \right],
    \sigma_z = \left[ \begin{array}{ccc}
    1 & 0\\
    0 & -1\end{array} \right].
\end{equation}
These Pauli matrices span a complex two-dimensional Hilbert space (a qubit). The projection measurement is to extract quantum information by operating on the interested observable and the density matrix of the target quantum state
\begin{equation}
    \langle \sigma \rangle = Tr(\sigma |\psi\rangle \langle \psi|)
\end{equation}
where $|\psi\rangle \langle \psi|$ generates the density matrix.
For the general z-measurement, the state vector is projected onto the z-axis of the Bloch sphere, and the corresponding value on the z-axis is the expectation of the measurement results, as shown in Fig.~\ref{fig:block}.


\subsection{Variational quantum algorithm}

The variational quantum algorithm (VQA) is the standard approach to performing QNN. It processes prepared quantum information by applying a series of parametric quantum gates, ultimately producing an output through measurement. As an example, a binary quantum classifier is implemented using VQA in \cite{farhi2018classification}. It has an ($n$+1)-qubit circuit, where the first $n$ qubits are prepared using an encoding method (such as angle, basis, or amplitude encoding) to represent specific information. The final qubit acted as a readout, generating the output through measurements. These qubits then pass a sequence of quantum gates with trainable parameters $U(\theta) = \prod_{l=1}^N U_l(\theta_l)$, where $\theta$ is a set of parameters.  A measurement outcome of 1 corresponds to one class, while a result of -1 corresponds to the other class. 
VQA uses a hybrid quantum-classical procedure to iteratively optimize the trainable parameters.
% , aiming at minimizing the loss value associated with the objective function of interest. 
The popular optimization approach includes gradient descent \cite{sweke2020stochastic}, parameter shift \cite{wierichs2022general}, and gradient-free techniques, such as COBYLA. All of the methods take the training data as input and evaluate the model performance by comparing the generated and correct labels. Based on this evaluation, the methods update the model parameters for the next round, repeating the process until the model converges and achieves the desired performance. The hybrid method performs the evaluation and parameter selection on a classical machine, while the model inference is carried out on a quantum machine.








% \subsubsection{Projective measurement}

% Projective measurement is the primary measurement method adopted by applications of quantum computing. It operates on the interested observables and the density matrix of the target quantum state. Density matrix \cite{fano1957description} is a general way to represent the state of a quantum system. For a quantum system with state vector $|\psi\rangle$, its density matrix is defined as the outer product of $|\psi\rangle$ with itself
% \begin{equation}
%     \rho \coloneqq |\psi\rangle \langle \psi|
% \end{equation}

% The density matrix is a Hermitian matrix with trace one.

% - trace() = 1

% - $<\sigma >$ = Tr($o_{xyz}\rho$) ---- cartesian coordinate system

% - quantum state tomography (multiple measurement)

% - hardware implementation








\section{Quantum state tomography} \label{sec:tomo}

Quantum state tomography (QST) is a technique to reconstruct an unknown quantum state using its measurement results \cite{toninelli2019concepts}. The measured observables must form a basis in the Hilbert space so that all state information can be recorded and used to recover the state. In the present age of NISQ, when the number of qubits is limited, QST is an essential method for retrieving the complete information stored in a quantum system. Nevertheless, as the number of qubits grows, the number of measurements needed and the complexity of state reconstruction increase exponentially. Hence, for the sake of simplicity, we consider the reconstruction of a single qubit using three observables to restore its state in a two-dimensional Hilbert space in this work.
% For reasons of efficiency and intuition, we consider the reconstruction of a single qubit and restore its state on the Bloch sphere in this work.


% Figure environment removed

Any arbitrary density matrix of a 1-qubit state can be expressed as a linear combination of Pauli matrices (basis of two-dimensional Hilbert space) as
\begin{equation}
\begin{split}
    \rho &= \frac{1}{2}(I + r_x \sigma_x + r_y \sigma_y + r_z \sigma_z)\\
    &= \frac{1}{2}\left[ \begin{array}{ccc}
    r_z + 1 & r_x - r_y\\
    r_x + r_y & -r_z + 1\end{array} \right]
\end{split}
\end{equation}
where $r$ denotes real number and $r_x^2 + r_y^2 + r_z^2 = 1$. For the typical qubit measurement using observable $\sigma_z$, as used by most quantum applications, the expectation of the measurement result is 
\begin{equation} 
\label{eq:obz}
 \langle \sigma_z \rangle = Tr\bigg(\frac{1}{2}(I + r_x \sigma_x + r_y \sigma_y + r_z \sigma_z) \sigma_z \bigg) = r_z
\end{equation}
Eq.~\ref{eq:obz} demonstrates that the expectation of measurement results is directly related to the density matrix $\rho$ of interest. In this case, however, only diagonal entries of $\rho$ can be retrieved, while some useful information remains untouched. Hence, in order to rebuild the density matrix $\rho$ completely, the measurements on observables $\sigma_x$ and $\sigma_y$ are necessary to obtain $r_x$ and $r_y$. 
 
From the perspective of the geometric representation, the expectation values $\langle \sigma_x \rangle$, $\langle \sigma_y \rangle$ and $\langle \sigma_z \rangle$ are exactly the projections of the state vector on the x-, y- and z-axis of the Bloch sphere, respectively. Therefore, after getting sufficient measurement results, we can restore the quantum state in the Bloch sphere for an intuitive interpretation. In other words, the interested 1-qubit state is a state vector with an unknown direction in the Bloch sphere. By projecting the vector on the three axes, we can figure out the direction of the state vector, which contains all information about the state. 





\section{Methodology}
\label{sec:method}

\subsection{Overview}
\label{sec:method_fmwk}

As shown in~\cref{fig:method_fmwk}, the proposed unsupervised MOT framework is trained with the widely-used contrastive learning technique~\cite{chen2020simple,he2020momentum}. 
\lk{Specifically, for multi-object tracking}, objects within the tracklet ($\boldsymbol{k}_{+}$) should be pulled together and different tracklets ($\boldsymbol{k}_{-}$) should be separated. It can be mathematically formulated as:

\begin{equation}
% \begin{split}
    \mathcal{L}_{cl}( \boldsymbol{q}; \boldsymbol{k}_{+}; \boldsymbol{k}_{-} )= 
    - \log \frac{\exp(\boldsymbol{q} \cdot \boldsymbol{k}_{+} / \epsilon)}{\sum_{i}\exp(\boldsymbol{q} \cdot \boldsymbol{k}_{i} / \epsilon)}
  \label{eq:method_nce}
% \end{split}  
\end{equation}

\noindent where $\mathcal{L}_{cl}$ denotes the InfoNCE~\cite{oord2018representation} loss function, and $\epsilon$ is the temperature hyper-parameter~\cite{wu2018unsupervised}. 
Within a video, following the unsupervised tracking fashion~\cite{liu2022online,shuai2022id}, the positive and negative keys mainly come from two sources, \ie pseudo-labeled historical frame and self-augmented frame. 

\lk{However, two issues occur: (1) the uncertainty reduces the accuracy of pseudo-tracklets and (2) the randomly augmented samples fail to learn the inter-frame consistency.} 
We argue the above issues are not independent. 
\lk{By leveraging the uncertainty in turn,} the accurate pseudo-tracklets can guide the qualified positive and negative augmentations.

To address these two issues, we propose an uncertainty-aware pseudo-tracklet labeling strategy in \cref{sec:method_uoap}, which integrates a verification-and-rectification mechanism into the tracklet generation. Our method significantly improves the accuracy of pseudo-identities, especially in long-term interval. 
Then we propose a tracklet-guided augmentation strategy in \cref{sec:method_ada_aug}, which brings the temporary information into spatial augmentation. The augmented samples simulates the objects' motion. A hierarchical uncertainty-based sampling strategy is proposed for hard sample mining. More details are described in the following section.


\subsection{Uncertainty-aware Tracklet-Labeling}
\label{sec:method_uoap}

Accurate pseudo tracklet is critical in \liuk{learning feature consistency}. 
However, without manual annotation, \lk{the aggravated uncertainty makes} the tracklet-labeling a huge challenge due to various interference factors, including similar appearance among objects, frequent object cross and occlusions, \etc. 
\lk{In fact, the uncertainty can also be leveraged to improve the pseudo-accuracy in turn.}
In this section, we propose an \textbf{U}ncertainty-aware \textbf{T}racklet-\textbf{L}abeling (\textbf{UTL}) strategy for better pseudo-tracklets.

Given an input video sequence $V = \{I^{1}, I^{2}, \cdots, I^{N}\}$, each frame $I^{t}$ is annotated with the bounding boxes $B^{t} = \{b_{1}^{t}, b_{2}^{t}, \cdots, b_{M^{t}}^{t}\}$ of $M^{t}$ objects in $t_{th}$ frame, where $b_{i}^{t} = (cx_{i}^{t}, cy_{i}^{t}, w_{i}^{t}, h_{i}^{t})$ is the center coordinate and shape of the $i_{th}$ object $o_{i}^{t}$. As shown in~\cref{fig:method_fmwk}, \mywork~generates accurate pseudo-tracklets in four main steps:

1) \textbf{Association}. For a certain object $o_{i}^{t}$ in frame $I^{t}$, the $\ell_2$-normalized representation $\boldsymbol{f}_{i}^{t}$ can be expressed as $\boldsymbol{f}_{i}^{t} = {\phi}(I^{t}, b_{i}^{t})$, 
% \begin{equation}
%   \boldsymbol{f}_{i}^{t} = {\phi}(I^{t}, b_{i}^{t})
%   % / {\Vert {\phi}(I^{t}, b_{i}^{t}) \Vert}_{2}
%   \label{eq:method_feat}
% \end{equation}
where the embedding encoder is denoted as $\phi$.

To associate the objects in frame $I^{t}$ with the objects or trajectories in previous $I^{t \minus 1}$, a similarity matrix is constructed with their appearance embeddings:

\begin{equation}
  \boldsymbol{C} \in \mathbb{R}^{M^{t} \times M^{t \minus 1}}, \;
  c_{i,j} = {\boldsymbol{f}_{i}^{t}} \cdot  \boldsymbol{f}_{j}^{t \minus 1}
  \label{eq:method_matrix}
\end{equation}

\noindent where $c_{i,j}$ represents the cosine similarity between the $i_{th}$ object in frame $I^{t}$ and the $j_{th}$ object (or trajectory) in frame $I^{t \minus 1}$. Then the Hungarian algorithm~\cite{kuhn1955hungarian} is adopted to generate the identity association results.

2) \textbf{Verification}. However, the appearance representations are sometimes unreliable, especially in the unsupervised scenario. To solve this issue, an uncertainty metric is proposed to evaluate the association after the first stage.

% For an object $o_{i}^{t}$ in frame $I^{t}$, the similarities against the $M^{t \minus 1}$ objects in the previous frame can be expressed as:

% \begin{equation}
%   \boldsymbol{s}_{i} = \boldsymbol{C}_{i} = [c_{i,1}, c_{i,2}, \cdots, c_{i,M^{t \minus 1}}]^T
%   \label{eq:method_svec}
% \end{equation}

% Inspired by margin-based OOD detection~\cite{hendrycks2016baseline}, we assume that the assignment ($o_{i}^{t} \!\sim\! o_{j}^{t \minus 1}$) in the association stage is not convincing under the following circumstances:

% \begin{itemize}
%     \setlength{\itemsep}{0pt}
%     \item The assigned similarity between $o_{i}^{t}$ and $o_{j}^{t \minus 1}$ is relatively low (\ie, $c_{i,j} < m_1$).
%     \item The second-highest similarity with others ($c_{i,j_{2}}$) is close to the assigned $o_{j}^{t \minus 1}$ (\ie, $c_{i,j} - c_{i,j_{2}} < m_2$).
% \end{itemize}

% Based on these assumptions, we define an association-level uncertainty metric, which is formulated as:



Object association can be viewed as multi-category classification.
And confidence-score has been proved efficient and effective on detecting mis-classified examples~\cite{hendrycks2016baseline}.
Inspired by this, we propose to detect the mis-associated objects through the similarity-scores.


Given an object $o_{i}^{t}$ associated with $o_{j}^{t \minus 1}$ in the previous frame based on \cref{eq:method_matrix}, the association ($o_{i}^{t} \!\sim\! o_{j}^{t \minus 1}$) is unconvincing in two cases: 
1) the assigned similarity $c_{i,j}$ is relatively low (\eg, partial occlusion or motion blur) and 
2) there are other objects whose similarities are close to the assigned $c_{i,j}$ (\eg, similar appearance or indistinguishable embedding).
It can be formulated as:

\begin{equation}
  c_{i,j} < m_1; \quad c_{i,j_{2}} > c_{i,j} - m_2
  \label{eq:method_margin}
\end{equation}


\noindent 
where $m_1,m_2$ are constant margins. Only the second-highest similarity with others ($c_{i,j_{2}}$) is considered for simplicity.
In an ideal association, $c_{i,j}$ should be close to 1 and $c_{i,j_{2}}$ close to 0.
We thus proposed to estimate the association \lk{risk} as:

% \sigma_{i,j} = - \left( 
% \log c_{i,j} + \log \left( 1 - c_{i,j_{2}} \right)
% + \overline{\log \left( 1 - c_{i,l} \right) }
% \right)  
\begin{equation}
  \sigma_{i,j} = - \log c_{i,j} - \log \left( 1 - c_{i,j_{2}} \right)
  \label{eq:method_energy}
\end{equation}

Detailed derivation process refers to the supplementary materials.
Combining with \cref{eq:method_margin} and \cref{eq:method_energy} , an adaptive threshold is proposed:

\begin{equation}
  % \gamma_{i,j} = -\log \left( 1 + m_2 - c_{i,j} \right) -\log m_1 \left( 1 - m_3 \right)
  \gamma_{i,j} =  -\log m_1 - \log \left( 1 + m_2 - c_{i,j} \right)
  \label{eq:method_border}
\end{equation}

As shown in~\cref{fig:method_verify}, when the \lk{risk} $\sigma_{i,j}$ is higher than the threshold $\gamma_{i,j}$, the assignment ($o_{i}^{t} \!\sim\! o_{j}^{t \minus 1}$) should be re-considered. 
\lk{The \textbf{association uncertainty} is quantified as:}

\begin{equation}
  \delta_{i,j} = \sigma_{i,j} - \gamma_{i,j}
  \label{eq:method_uncertain}
\end{equation}

The results are not sensitive to the exact margins. We set $m_1 = 0.5$ and $m_2 = 0.05$ for a slightly better performance.
% More experimental details are shown in the supplementary materials.

The uncertain pairs after the verification stage and unmatched objects after the association stage are gathered as uncertain candidates for the rectification stage.


3) \textbf{Rectification}. 
The rectification stage is performed among the uncertain candidate. The similarities between two adjacent frames are no longer convincing.
% due to irregular motion, severe occlusion, and so on. 
More information should be taken into account, including motion \lk{estimation} and appearance \lk{variation} within a tracklet. 
% Specifically, intersection-over-union (IoU)~\cite{bewley2016simple} is the widely-used motion metric. At the same time, the tracklet embeddings can provide complementary appearance information.

For the uncertain candidates, \mywork~constructs another similarity matrix for the secondary rectification. 
First, \lk{the motion constraints should be relaxed}, so the association shares overlap \lk{higher than} $\beta$ 
% in adjacent frames 
\lk{are preserved}. 
Second, \lk{the appearance should not vary extremely fast}, so we adopt the averaged similarity between object $o_{i}^{t}$ and tracklet $trk_{j} = \{o_{j}^{t \minus K}, \cdots, o_{j}^{t \minus 1}\}$ within previous $K$ frames. 
In this stage, we solve the sub-problem of global identity assignments, which can be formulated as:

\begin{equation}
\begin{split}
  \boldsymbol{C}^\prime \in \mathbb{R}^{{M^{t}}^\prime \times {M^{t \minus 1}}^\prime} & \\
  c^\prime_{i,j} = \left( \frac{1}{K} \sum_{\hat{t} = t \minus K}^{t \minus 1} {\boldsymbol{f}_{i}^{t}} \cdot  \boldsymbol{f}_{j}^{\hat{t}} \right) 
            \times \mathbb{I} & \left( \text{IoU} \left( b_{i}^{t}, b_{j}^{t \minus 1} \right) > \beta \right) 
  \label{eq:method_recti}
\end{split}
\end{equation}

\noindent where $\mathbb{I}(*)$ is the indicator function. Then the match set is updated based on the Hungarian algorithm.

\lk{
\textit{Remark.} Our core contribution is the uncertainty-based verification mechanism, rather than the specific rectification, which shall be adjusted in practice. Empirically we set $\beta=0.1$ and $K=5$.
}

% Figure environment removed

4) \textbf{Propagation}. The pseudo-tracklets are propagated frame-by-frame. As shown in~\cref{fig:method_reidacc}, our strategy brings \lk{consistently} accurate pseudo-identities, \lk{\eg, reaching 97\% accuracy across 100 frames}.
% The pseudo-tracklets are progressively updated during the training process.
The long-term intra-tracklet consistency is successfully maintained.
% by the accurate pseudo-identities.

It is worth mentioning that the \lk{verification and rectification} stages can be naturally applied to the inference process to boost the performance, \lk{which does not conflict with existing association methods}.

\subsection{Tracklet-Guided Augmentation}
\label{sec:method_ada_aug}

The accurate pseudo-tracklets can guide the sample augmentation in the unsupervised MOT framework.
To learn the \liuk{inter-frame consistency}~\cite{chen2020simple,zhang2021fairmot}, good training samples should be diverse and \liuk{temporal-aware}. 
However, as illustrated in~\cref{fig:method_ada_aug}, existing methods usually treat augmentation and multi-object tracking as two isolated tasks, leading to ineffective augmentations. 
Instead, this paper utilizes the tracklet's spatial displacements to guide the augmentation process. 
According to a properly selected anchor pair, the proposed strategy makes the augmented frames aligned to the historical frames, simulating realistic tracklet movements. The proposed method concurrently focuses on the hard negative samples.
Details \lk{of the \textbf{T}racklet-\textbf{G}uided \textbf{A}ugmentation (TGA)} are given below.

% Figure environment removed

We introduce the temporal information into spatial transformation. 
First, given a current frame $I^{t}$ with $M^{t}$ objects, we select a source-anchor object $o_{a}^{t}$, whose bounding box is denoted as $b_{a}^{t} = (cx_{a}^{t}, cy_{a}^{t}, w_{a}^{t}, h_{a}^{t})$. Then, we choose a target-anchor $o_{a}^{t \minus \tau}$ in $(t \minus \tau)_{th}$  historical frame from the pseudo-tracklet $trk_{a} = \{o_{a}^{t_0}, o_{a}^{t_1}, \cdots, o_{a}^{t}\}$. 
Finally, to augment the current $I^{t}$ to align with historical $I^{t \minus \tau}$,  a tracklet-guided affine transformation can be expressed as:

\begin{equation}
  \begin{bmatrix}
      x^{t \minus \tau} \\ y^{t \minus \tau} \\ 1
  \end{bmatrix}
  =
  \boldsymbol{M}_{t}^{t \minus \tau}
  \begin{bmatrix}
      x^{t} \\ y^{t} \\ 1
  \end{bmatrix}
  =
  \begin{bmatrix}
      m_{11} & m_{12} & m_{13} \\
      m_{21} & m_{22} & m_{23} \\
      0      & 0      & 1
  \end{bmatrix}
  \begin{bmatrix}
      x^{t} \\ y^{t} \\ 1
  \end{bmatrix}
  \label{eq:method_affine}
\end{equation}

\noindent where $x^*,y^*$ are spatial coordinates, and $\boldsymbol{M}_{t}^{t \minus \tau}$ can be solved by direct linear transform (DLT) algorithm ~\cite{detone2016deep}. 
% with the corner locations of the anchor pair $(o_{a}^{t} \!\sim\! o_{a}^{t \minus \tau})$. 
Then an augmented frame $\tilde{I}^{t}$ is generated based on the tracklet-guided affine transformation with perspective jitter, which can be expressed as $\tilde{I}^{t} = \mathcal{T}\left(I^{t}, M_{t}^{t \minus \tau} \right)$.
% \begin{equation}
%   \tilde{I}^{t} = \mathcal{T}\left(I^{t}, M_{t}^{t \minus \tau} \right)
%   \label{eq:method_aug}
% \end{equation}

Intuitively, a proper anchor-selection is vitally important for our augmentation strategy. 

First, the identity accuracy of anchor pair $(o_{a}^{t} \!\sim\! o_{a}^{t \minus \tau})$ is important. In other words, the consistency of anchor tracklet $trk_{a}$ should be guaranteed. We thus design a tracklet-level uncertain metric based on the propagated association-level uncertainty defined in \cref{eq:method_uncertain}, which is formulated as:

\begin{equation}
  \Omega_{i} = \frac{1}{n} \sum_{s=t_0}^{t} \exp (\delta_{i}^{s})
  % \Omega_{i} = \sqrt[n]{\sigma_{i}^{t_0} \cdot \sigma_{i}^{t_1} \cdots \sigma_{i}^{t}}
  \label{eq:method_tenergy}
\end{equation}

\noindent where $\Omega_{i}$ represents the uncertainty of tracklet $trk_{i}$, \lk{and $n$ is the tracklet length}.
An uncertainty-based sampling strategy is designed to select the source anchor $o_{a}^{t}$ (along with the anchor $trk_{a}$) from the $M^{t}$ objects in frame $I^{t}$, which can be formulated as:

\begin{equation}
  p\left(a=i \mid t \right) 
  % = softmax\left(-\Omega_{i}\right)
  = \frac{\exp{\left(-\Omega_{i}\right)}}{\sum_{\hat{i}=1}^{M^{t}}\exp{\left(-\Omega_{\hat{i}}\right)}}
  \label{eq:method_sel_an_src}
\end{equation}

\noindent where $p\left(a=i \mid t \right)$ represents the probability to choose the $i_{th}$ tracklet $trk_{i}$ as the anchor $trk_{a}$.
The uncertain tracklet with high $\Omega$ is less likely to be selected, avoiding dramatic augmentations from erroneous pseudo-tracklets.

Second, hard negative samples matters in discriminablity learning. We tend to choose an indistinguishable (or, high uncertain) target anchor $o_{a}^{t \minus \tau}$ along the tracklet $trk_{i}$. The selection probability can be formulated as:

\begin{equation}
  p\left(\pi=t \minus \tau \mid a \right) 
  = \frac{\exp{\left(\delta_{a}^{t \minus \tau}\right)}}{\sum_{\hat{\tau}=t_0}^{t-1}\exp{\left(\delta_{a}^{t-\hat{\tau}}\right)}}
  \label{eq:method_sel_an_tgt}
\end{equation}

\lk{A visualization example are displayed in the supplementary material to illustrate the hierarchical sampling process.}

Compared with conventional random transformation, the proposed tracklet-guided augmentation is well-directed and tracking-related. 
\lk{Together with accurate pseudo-tracklets, \mywork~successfully improves the inter-frame consistency, as shown in \cref{fig:method_disc_vis}. }


% Figure environment removed

% \subsection{Momentum Memory Dictionary}
% \label{sec:method_md}


%To reuse the encoded samples from the intermediate mini-batches, we maintain a queue for each video in the memory dictionary by enqueueing the $M^{t}$ objects in the current frame and removing the oldest samples.
%In representation learning, high-quality negative samples play an essential role~\cite{chen2020simple,he2020momentum}. However, existing unsupervised trackers only take negative samples from adjacent frames, augmented frames, and the current frame itself. The lack of negative sample diversity prevents trackers from learning discriminative representations. \mywork~adopts a momentum dictionary mechanism to alleviate this problem.

%As shown in~\cref{fig:method_fmwk}, we build a memory dictionary for each \textit{independent} video input during training. Given an input image $I^{t}$ from video $V$, we randomly sample a number of negative object samples from other videos in the memory dictionary, so as to enrich the negative sample diversity. To reuse the encoded samples from the intermediate mini-batches, we maintain a queue for each video in the memory dictionary by enqueueing the $M^{t}$ objects in the current frame and removing the oldest samples.
\section{Evaluation} \label{sec:eval}
We conducted empirical studies in various scenarios to evaluate the performance of the proposed quantum multi-classifier MORE. To implement MORE and related baselines, we used Python 3.8 and the IBM Qiskit package \cite{qiskit} to simulate quantum systems both with and without noise. For simulating noisy systems, we employed several noisy backends, including \texttt{FakeAuckland}, \texttt{FakeAthensV2}, and \texttt{FakeBelemV2}. The source code used to generate the experiment results is available in \href{https://github.com/Jindi0/MORE.git}{github.com/Jindi0/MORE}. 

% Figure environment removed

% Figure environment removed

\textbf{Dataset:} (1)We utilize the \textbf{MNIST} dataset \cite{deng2012mnist} to conduct experiments in the noiseless quantum system. The MNIST is a widely used benchmark for image classification tasks, consisting of ten classes of hand-written digits ranging from 0 to 9. We randomly select 1,000 training and 200 test images from each class and reduce their dimensions to 8 using PCA. (2) Additionally, we evaluate MORE using the \textbf{Iris} dataset \cite{Iris} in simulated noisy quantum systems. The Iris dataset contains 150 instances classified into three distinct classes, each consisting of 50 instances with 4 pixels. We use 70\% of the instances (105 instances) for training, while the remaining 30\% (45 instances) are reserved for testing purposes.

\textbf{Model implementation:}
We develop the QNN classifiers based on Qiskit \texttt{NeuralNetworkClassifier} class, and use the optimizer \texttt{COBYLA} to update trainable parameters. MORE employs an 8-qubit QNN for the MNIST dataset and a 4-qubit QNN for the Iris dataset, with 91 and 39 trainable parameters, respectively. We build QNNs based on the design principles for quantum convolutional neural networks proposed in \cite{cong2019quantum}. For the readout, we choose the last active qubits at the end of the circuit and measure it with $\sigma_x$, $\sigma_y$, and $\sigma_z$ observables. To efficiently convert classical labels to quantum labels during the clustering step, we randomly select five instances from each class and form pairs from the resulting dataset. The clustering dataset consists of $\binom{5K}{2}$ pairs in total, where $K$ is the number of classes. We use MSE to calculate the interclass correlations. And to make the quantum labels spread out as much as possible, the MSE between different classes is normalized to the interval [0.5, 1]. During supervised learning, then, the entire training dataset is used.










% Furthermore, we compare MulQ's performance to that of related baselines, which are quantum classifiers with generic designs, as mentioned in Section 1, and the classical counterparts with the same number of parameters.





% Figure environment removed

\subsection{Accuracy}
We evaluate the accuracy of the proposed approach \textbf{MORE} using the MNIST dataset on noise-free quantum systems. We conduct nine classification problems ranging from binary to 10-class. 

Our evaluation starts by investigating the effect of using multiple observables of a readout qubit on binary classification tasks. To do so, we compare our results to \textbf{BaseBin}, which serves as the baseline in this experiment. The comparison is illustrated in Fig.~\ref{fig:eval-bin}. Both MORE and BaseBin use the same ansatz (QCNN with 91 trainable parameters) for binary classifications, but the readout qubit of BaseBin is measured only with observable $\sigma_z$. The expected value of its measurement results is associated with two distinct classes: a positive expected value represents one class, and a negative expected value represents the other. We conduct 45 binary classifications on MNIST using MORE and BaseBin, respectively. The class pairs for classification are listed on the x-axis of Fig.~\ref{fig:eval-bin} and are sorted in descending order of interclass correlations. E.g., the training data of classes `4' and `9' have the highest similarity, while those of classes `0' and `1' have the lowest. The results indicate that MORE outperforms BaseBin in 37 out of 45 tasks and achieves comparable accuracy in the remaining tasks. MORE improves accuracy by up to 22.28\% and by an average of 4.9\%. Furthermore, the performance of MORE demonstrates greater stability across tasks of varying difficulty than BaseBin. As observed, a roughly inverse relationship exists between accuracy and interclass correlation for both MORE and BaseBin, i.e., as class correlation increases, classification becomes more challenging. Consequently, both MORE and BaseBin exhibit the highest accuracy on (0, 1)-classification and the lowest accuracy on (4, 9)-classification. Nonetheless, the variance of MORE's accuracy over 45 tasks is only 34.01, while that of BaseBin is 59.41, indicating that MORE is more stable than BaseBin. It shows the advantage of MORE, which has three observables, in terms of accuracy and stability.


Next, we evaluate MORE on multi-classification tasks using the MNIST dataset and summarize the results in Fig.~\ref{fig:eval-mul}. We conduct eight multi-classification tasks, categorizing handwritten digits into three to ten classes. Note that the ansatzes used in multi-classifications are \textit{identical} to the binary classification ansatzes. The accuracy of MOREs is compared to that of \textbf{BaseAnc} and \textbf{BaseMea}, which serve as the baselines in this experiment. BaseAnc and BaseMea are both variational quantum multi-classifiers. BaseAnc utilizes eight qubits for data processing and $n$ ancilla qubits as the readout for $n$-class classifications. And BaseMea employs a total of eight qubits and measures a subset of qubits at the end of the circuit to produce a result. In a multi-classification task with $n$ classes, $n$ qubits are measured. They encode class labels as one-hot vectors and measure the readout qubits on the z-basis. So BaseAn and BaseMea can only support the classifications involving up to eight classes, due to the limitation in dimensions and the number of qubits, respectively. MORE, however, is capable of conducting 10-class classification regardless of the number of qubits, and we believe it still qualifies if the dataset contains more classes. Fig.~\ref{fig:eval-mul} indicates that MORE outperforms BaseAnc and BaseMea across the board. The reason is that both BaseAnc and BaseMea require simultaneous control of several qubits to represent the class label, which can be challenging, particularly for an unstable quantum system. In contrast, MORE only utilizes a single qubit to record the outcome, thus reducing the number of factors contributing to instability. 

Overall, our proposed MORE approach, which uses a simple circuit with only one readout qubit, can beat other general approaches and achieve the desired performance. In the following subsections, we will analyze the impact of the MORE's components and access MORE in noisy quantum systems.

% We consider two implementations of MORE: with and without the loss regulator R during quantum label-based training procedures. , with MORE w/ R achieving the highest accuracy.





\subsection{Quantum label}

We now examine the impact of selecting quantum labels for classes. To do so, we compare the following approaches:
\begin{itemize}
    \item \textbf{BaseRand}: A baseline method that randomly assigns quantum states to classical labels.
    \item \textbf{MORE$\backslash$cor}: A variant of the MORE approach that does not consider interclass correlation during the variational quantum clustering step, i.e., all diagonal entries in the scaler array $S$ are -1s and all other entries are 1s.
    \item \textbf{MORE}: The vanilla MORE method employs interclass correlation for determining quantum labels.
\end{itemize}

Then, based on the selected quantum labels, each of the three approaches is followed by quantum label-based supervised learning using objective function Eq.~\ref{eq:sup_loss1} without the loss regulator $\mathcal{R}$. The test accuracy of the above approaches is summarized in Fig.~\ref{fig:eval-labels}. Across eight multi-classification tasks, MORE outperforms other methods in all cases. BaseRand and MORE$\backslash$cor have comparable performance, all of which are inferior to MORE. Therefore, we conclude that it is beneficial to take interclass correlations into account when deciding the quantum labels for classical labels. A possible reason is that the data distribution in the Hilbert space is strongly related to that in the classical feature space. As an example, Fig~\ref{fig:overview} (d) shows the distribution of quantum labels for class `0', `1', and `2' of MNIST, according to their relationship as listed in Fig~\ref{fig:overview} (b). Classes `1' and `2' have the smallest MSE value (strongest correlation), so their quantum labels are closer to each other compared to other class pairs. Similarly, the readout states of the instance from classes `1' and `2' are supposed to be closer than those of other classes. As a result, assigning quantum labels based on class correlation can capture the training data pattern to some extent, which is advantageous for enhancing model quality, as shown in Fig.~\ref{fig:eval-labels}. Nonetheless, the figure reveals that the accuracy of quantum classifiers decreases as the number of classes increases. This trend is also observed in quantum classifiers that employ alternative implementation strategies. Thus, we introduce a loss adjuster to alleviate this issue, as shown in Eq.~\ref{eq:reg}, and analyze its impact in the next subsection. 






\subsection{Loss adjuster} \label{sec:reg}

\begin{table}[]
\centering
    \caption{Parameters and results for loss adjuster}
\begin{tabular}{cccccc} 
\toprule
Task & \begin{tabular}[c]{@{}c@{}}Min. label \\ distance\end{tabular} & $r$    & $w$   & More\textbackslash{}R acc. & MORE acc.\\ \midrule
0-2  & 1.302                                                          & 1.5  & 0.1 & 84.13\%                & 88.7\%  \\ \midrule
0-3  & 0.713                                                          & 1.0  & 0.2 & 63.45\%                  & 70.1\%  \\ \midrule
0-4  & 0.57                                                           & 0.8  & 0.1 & 53.5\%                   & 63.3\%  \\ \midrule
0-5  & 0.077                                                          & 0.2  & 0.4 & 40.48\%                  & 50.2\%  \\ \midrule
0-6  & 0.212                                                          & 0.4  & 0.5 & 29.06\%                  & 37.2\%  \\ \midrule
0-7  & 0.067                                                          & 0.2  & 0.5 & 44.8\%                   & 48.6\%  \\ \midrule
0-8  & 0.099                                                          & 0.15 & 0.2 & 29.4\%                   & 33\%    \\ \midrule
0-9  & 0.086                                                          & 0.15 & 0.2 & 22.6\%                   & 27.8\% \\
\bottomrule
\end{tabular}
\label{tab:lossreg}
\end{table}

% Furthermore, implementing MORE with the loss regulator R improves the accuracy by an average of 10\% than the implementation without R, indicating that fine-grained tuning is necessary when mapping multiple labels in a two-dimensional Hilbert space based on interclass correlations.

% We now analyze the impact of the loss regulator $\mathcal{R}$ in this subsection. 
Table~\ref{tab:lossreg} summarizes the parameters and results for the loss adjuster over the multi-classification tasks. The accuracy of \textbf{MORE} (with $\mathcal{R}$) and \textbf{MORE\textbackslash{}R} (without $\mathcal{R}$) are listed in the last two columns. The comparison indicates that MORE, which uses the loss function (Eq.~\ref{eq:loss_r}) with $\mathcal{R}$ during supervised learning, improves the accuracy across all tasks. This improvement is caused by the corrected misclassification between the classes whose quantum labels in the Hilbert space are too near together. In Eq.~\ref{eq:loss_r}, two hyper-parameters need to be determined by users: the threshold $r$ and the weight $w$. The threshold $r$ is empirically determined based on distances between quantum labels. In our experiments, the Cosine distance between quantum labels is calculated. Usually, $r$ is specified to be slightly larger than the shortest distance, as shown in the 2nd and 3rd columns of table~\ref{tab:lossreg}. E.g., the smallest Cosine distance between quantum labels in the 0-9 task, a 10-class classification, is 0.086 (about 24 degrees), so we set the $r$ to be 0.15 (about 32 degrees around the target quantum label). Moreover, we test MORE with a $w$ range from 0.1 to 1.0 and identify the optimal value of $w$ for each task. We report these values in the 4th column of the table, although we found accuracy improvements across all values of $w$ in each task. The $w$ values demonstrate that $\mathcal{R}$ has varying importance across different tasks, but its contribution is not greater than half in any task. Hence, we recommend that users search for an appropriate value of $w$ within the range of 0.1 to 0.5.


 % It is evident that the minimum distance between quantum labels decreases as the number of classes increases. 
 
 % Additionally, the parameter R becomes increasingly important in improving the quality of the model. 





\subsection{Noisy quantum system}

\begin{table}[]
    \centering
    \caption{Test accuracy of MORE on Iris dataset with noisy backends}
    \begin{tabular}{cccc}
\toprule
\# Shots & FakeAuckland & FakeAthensV2 & FakeBelemV2 \\ \midrule
1k    & 95.56        & 95.56        & 93.33       \\ \midrule
2k    & 97.78        & 93.33        & 95.56       \\ \midrule
3k    & 97.78        & 93.33        & 95.56       \\ \midrule
4k    & 97.78        & 93.33        & 95.56       \\ \midrule
5k    & 93.33        & 93.33        & 95.56       \\ \midrule
6k    & 95.56        & 95.56        & 95.56       \\ \bottomrule
\end{tabular} 
    
    \label{tab:iris}
\end{table}

Table \ref{tab:iris} summarizes the test accuracy of MORE using the noisy quantum backends, including FakeAuckland, FakeAthensV2, and FakeBelemV2 on the Iris dataset. The hybrid quantum-classical training method of QNN typically requires a long time to execute on a quantum machine, but the IBM cloud quantum computing platform imposes time limits for tasks. In this experiment, we therefore utilize simulators of noisy quantum computing systems. The noisy backends used in this experiment have the noise model collected from real quantum machines, which includes T1 and T2 time, 1-qubit and 2-qubit gate errors, and measurement errors. So these simulated backends provide us with a practical and reasonable way to evaluate the performance of MORE in noisy quantum systems. To process the Iris instances of size 4, we construct 4-qubit QNNs with 39 trainable parameters. We update the QNNs for 300 steps using the COBYLA optimizer. We varied the number of shots for the quantum program from 1,000 to 6,000 on the backends to examine the impact of quantum computation costs. 




% The Iris dataset comprises 150 instances divided into three classes, with each class containing 50 instances. We used 70\% of the instances (105 instances) for training, while the remaining 30\% (45 instances) were reserved for testing purposes. 

The MOREs reach their highest accuracies of 97.78 (44/45), 95.56 (43/45), and 95.56 (43/45) on the FakeAuckland, FakeAthensV2, and FakeBelemV2 backends, respectively. The noisy MOREs achieve comparable performance to the noise-free version, with an accuracy of 97.78\%. Surprisingly, we found that the accuracy was not directly proportional to the number of shots, suggesting that the quantum computation cost required for this task is not significant. However, we acknowledge that the number of shots required may increase with an increase in the number of classes. Nonetheless, the cost is still acceptable only if the distribution of learned quantum labels is scattered in the Hilbert space. Furthermore, while it is true that accurate measurement results are necessary for the MORE approach, we want to highlight that using a simple quantum circuit (with fewer qubits, gates, and shallow depth) can lead to less error compared to other methods. This makes the MORE approach feasible and promising to implement on NISQ machines.


















































\section{Related work}\label{sec:related}
We were unable to find any studies closely relating to our work. Looking at the wider area~\cite{ferrari2019we,ferrari2020methodological} examined multiple top-n recommender algorithms presented at prestigious conferences, and found that only 7 out of 18 works could be reproduced with reasonable effort. Furthermore, 6 of those 7 could be outperformed by simple baselines. The major problems they highlight relate to using weak baselines, arbitrary evaluation setups and suboptimal hyperparameters. The latter was also examined in detail in~\cite{rendle2022revisiting} through the example of the now 15 years old iALS algorithm. iALS can be a much stronger baseline -- capable of beating more modern algorithms -- with the appropriate parameterization.

\cite{ferrari2021troubling} reassures the findings of~\cite{ferrari2019we,ferrari2020methodological} for a wider variety of algorithms and baselines. They argue that while the majority of recent papers utilize evaluation setups, datasets, metrics, etc.~of previous work, these are reused without questioning their validity, thus popular but flawed evaluation setups can spread in the community.

A recent work~\cite{petrov2022systematic} reinforces the notion that claimed state-of-the-art performance is often skewed or invalid through the example of BERT4Rec, as the results of the original paper can not be reproduced.

Standardization of evaluation through frameworks like \cite{10.1145/3523227.3551472,argyriou2020microsoft} can help reproducibility, if those frameworks are mostly free of the issues discussed above. However, defining the proper evaluation setups with datasets, and correctly implementing a wide variety of algorithms is not an easy task. Our findings show that algorithms reimplemented in benchmarking frameworks can suffer from serious flaws.
\section{Conclusion} \label{sec:conclusion}


We propose MORE, a QNN-based multi-classifier that maximizes quantum resource efficiency. By fully leveraging quantum information, MORE employs a simple ansatz as the binary classifier with only one readout qubit to solve multi-classification problems. To achieve this, MORE first converts classical labels into corresponding quantum labels using variational quantum clustering. This process considers interclass correlations, allowing the learned quantum labels to capture intrinsic patterns in the classical feature space of the training data. Quantum supervised learning is then performed based on the quantum labels to efficiently train the model. Furthermore, we introduce a loss adjuster to enhance the model's quality by making it more sensitive to the labels that can result in misclassifications during training. Our comprehensive evaluations demonstrate that MORE outperforms general quantum multi-classifiers and is capable of effective operation in noisy quantum systems due to its reduced error sources. This suggests that fully utilizing quantum information offers a promising approach for scaling up the problem sizes that can be addressed during the current NISQ era.

\section*{Acknowledgment}
The authors would like to thank all the reviewers for their helpful comments. 
This project was supported in part by US National Science Foundation grant CNS-1816399. This work was also supported in part by the Commonwealth Cyber Initiative, an investment in the advancement of cyber R\&D, innovation and workforce development. For more information about CCI, visit cyberinitiative.org.





\bibliographystyle{IEEEtran}
\bibliography{ref}

\end{document}
