\section{Semantic Approach and Commutativity}\label{sec:new_logic}
    In this section, we discuss in more detail the semantic approach and we show how it can be used to obtain decidability results. Then, we introduce and analyze the commutativity axiom in the context of epistemic planning.

    With the semantic approach, we aim at devising a new way to approach decidability in epistemic planning, which deviates from the common line of research in the literature focused on limiting the action theory syntactically (\eg by imposing a limit on the maximum modal depth of formulae). Our approach is motivated by the fact that to obtain decidable fragments of the general problem, one must appeal to strong syntactical constraints. To substantiate this claim, recall that the problem is still undecidable when the maximum modal depth allowed is set to $d{=}2$ (see Table \ref{tab:complexity1} for more details, where $\mathcal{T}(\ell, m)$ denotes the class of epistemic planning tasks where preconditions and postconditions have modal depth at most $ \ell$ and $ m$, respectively). This is clearly a strong limitation of syntactic approaches, as the isolated classes are too strong in many practical cases, where reasoning about the knowledge of others is required. Instead, the semantic approach does not limit the structure of formulae of the action theory, but rather relies on devising a suitable set of axioms that guarantee desirable properties on the structure of epistemic states (\eg bounded number of possible worlds). Since, in principle, there are many ways one can obtain such desirable properties by means of modal axioms, we argue that the semantic approach constitutes a fruitful avenue of research that can be further explored in many different ways.

    Towards this goal, we analyze in detail the commutativity axiom. The key insight behind the definition of such axiom is that, in the logic S5$_n$, there is no rule or principle that describes how the knowledge of one agent should interact with the knowledge of another agent. Hence there is no restriction on the ability of agents to reason about the higher-order knowledge they possess about the knowledge of others. 
    This is clear in Example \ref{ex:update}, where at each step $k$ we obtain an epistemic state $s_k$ that contains a chain of worlds of the form $ w_1 R_{i} w_2 R_{j} w_3 R_{i} \dots w_k $ (for $i,j \in \{\mathsf{a}, \mathsf{b}\}, i \neq j $), that intuitively represents $i$'s perspective about $j$'s perspective about $i$'s perspective, and so forth. This idea has been exploited for building undecidability proofs of the plan existence problem in the logic S5$_n$, by showing a reduction from the halting problem of Turing machines \cite{journals/jancl/Bolander2011} and Minsky two-counter machines \cite{conf/ijcai/Aucher2013}.

	To weaken this reasoning power, we introduce a principle that governs the capability of agents to reason about the knowledge of others, which is captured by the following interaction axiom (where $i{\neq}j$):
    \begin{equation*}
        \begin{array}{lll}
            \axiom{C} & \B{i} \B{j} \varphi \rightarrow \B{j} \B{i} \varphi & \textnormal{(Commutativity)}
        \end{array}
    \end{equation*}

    \noindent The commutativity axiom is well-known in many-dimensional modal logics where it is part of the axiomatisation of the product of two modal logics \cite{book/nhpc/Gabbay2003}.
    %
    Here, we adopt it with a novel epistemic connotation. Namely, we can read \axiom{C} as follows: whenever an agent $i$ knows that another agent $j$ knows that $\varphi$, then $j$ knows that $i$ knows \emph{too} that $\varphi$.
    %
    Thus, intuitively, axiom \axiom{C} defines a principle of \emph{commutativity} in the knowledge that agents have about the knowledge of others. 
    %
    This intuition is formalized and proved in the next section (see Lemma~\ref{lem:ck-n} and Theorem~\ref{th:ck}).
    
    This axiom is instrumental in proving decidability of the plan existence problem. Moreover, it provides a useful principle for two main reasons. 
    %
    First, as we mentioned above, this axiom allows to govern the reasoning power of agents. As it turns out, in this way we obtain a \emph{finitary non-fixpoint} characterization of common knowledge (see Theorem \ref{th:ck}), which concretely shows the power of knowledge commutativity.
    %
    Second, this principle constitutes a reasonable assumption in several \emph{cooperative multi-agent planning tasks} \cite{journals/csur/Torreno2017} where agents are able to communicate or monitor each other. In fact, when autonomous agents cooperate to reach a shared goal, then they are expected to behave in such a way that the effects of their actions are observable by others. In other words, acting in a cooperative context results into a transparent behaviour of agents, which in turn well fits with the concept of knowledge commutativity.

    We call C-S5$_n$ the logic S5$_n$ augmented with axiom \axiom{C}.
    As a final remark, notice that axiom \axiom{C} is a Sahlqvist formula and it corresponds to the following frame property:
    \begin{equation}\label{eq:A-frame}
        \forall u, v, w (u R_j v \wedge v R_i w \rightarrow \exists x (u R_i x \wedge x R_j w))
    \end{equation}
    
    \begin{proposition}
        The logic C-S5$_n$ is sound and complete with the class of reflexive, symmetric and transitive epistemic models that enjoy property (\ref{eq:A-frame}).
    \end{proposition}
