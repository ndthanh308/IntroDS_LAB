\section{Dynamic Epistemic Logic}\label{sec:del}
    This section is organized as follows. The syntax and semantics of DEL~\cite{book/springer/vanDitmarsch2007} are introduced in Section~\ref{sec:epistemic_models}, event models and the product update in Section~\ref{sec:event_models}.
    In Section \ref{sec:s5}, we recall the axioms of the logic S5$_n$.
    In Section~\ref{sec:plan_ex} we define the plan existence problem.

\subsection{Epistemic Models}\label{sec:epistemic_models}
    Let $\atomSet$ be a finite set of atomic propositions and $\agentSet = \{1, \dots, n\} $ a finite set of agents. The language $\Lang{C}$ of \emph{multi-agent epistemic logic on $ \atomSet $ and $ \agentSet $ with common knowledge} is defined by the following BNF:
    \begin{equation*}
        \varphi ::= \atom{p} \mid \neg \varphi \mid \varphi \wedge \varphi \mid \B{i} \varphi \mid \CK{G} \varphi,
    \end{equation*}

    \noindent where $ \atom{p} \in \atomSet $, $ i \in \agentSet $, and $ \varnothing \neq G \subseteq \agentSet $. Formulae $ \B{i} \varphi $ and $ \CK{G} \varphi $ are respectively read as ``agent $ i $ knows that $ \varphi $'' and ``group $G$ has common knowledge that $ \varphi $''.
    We define $\top$, $\bot$, $\vee$, $\rightarrow$ and $\D{i}$ as usual.

    \begin{definition}[Epistemic Model and State]
        An \emph{epistemic model} of $ \Lang{C} $ is a triple $ M = (W, R, V) $ where:
        \begin{compactitem}
            \item $ W \neq \varnothing $ is a finite set of possible worlds; 
            \item $ R: \agentSet \rightarrow 2^{W \times W} $ assigns to each agent $ i $ an accessibility relation $R(i)$ (abbreviated as $R_i$);
            \item $ V: \atomSet \rightarrow 2^W $ assigns to each atom a set of worlds.
        \end{compactitem}
        \noindent An \emph{epistemic state} is a pair $ (M, W_d) $, where $ W_d \subseteq W $ is a non-empty set of designated worlds.
    \end{definition}

    \noindent Intuitively, a designated world in $W_d$ is considered the current ``real" world from the perspective of an external observer (the planner) rather than of agents in $\agentSet$. Thus, $|W_d|>1$ represents the uncertainty of the observer about the real world. 
    
    The pair $(W, R)$ is called the \emph{frame} of $M$. We use the infix notation $ w R_i v $ in place of $ (w, v) \in R_i $.
    We also define $ R_G \doteq \cup_{i \in G} R_i $, where $ G \subseteq \agentSet $. The reflexive and transitive closure of $ R $ is denoted by $ R^* $. 
    Relations $R_i$ capture what agents consider to be possible: $ w R_i v $ denotes the fact that, in $w$, agent $i$ considers $v$ to be possible.
    %Formulae in $\Lang{C}$ are evaluated in worlds of a model. 		
    Throughout the paper, to support our exposition, we consider the example of the \emph{coordinated attack problem} \cite{book/springer/Gray1978,journals/jacm/Halpern1990}. It is a well-known problem that is often analyzed in the distributed systems literature. In what follows, we appeal to the DEL representation of this problem provided in \cite{journals/ai/Bolander2020}.

    \begin{example}[The Coordinated Attack Problem]\label{ex:k_model}
        Two generals, $\mathbf{a}$ and $\mathbf{b}$, are camped with their armies on two hilltops overlooking a common valley, where the enemy is stationed. The only way for them to defeat the enemy is to attack simultaneously. They can only communicate by means of a messenger, who may be captured at any time when crossing the valley. Neither general will attack until he is sure that the other will attack as well.
        
        General $\mathbf{a}$ and the messenger are initially together, and general $\mathbf{a}$ decides to attack at dawn. We use the atomic propositions $d$ to denote that `\emph{general $\mathbf{a}$ will attack at dawn}' and $m_i$, for $i = \mathbf{a}, \mathbf{b}$, to denote that the messenger is currently at the camp of general $i$. In this way, $\neg m_a \land \neg m_b$ expresses the fact that the messenger has been captured.
        
        The initial situation can be described by the epistemic state $s_0$ shown below\footnote{In all figures, the reflexive, symmetric and transitive closures of the relations are left implicit.}. Each bullet represents a world and the designated world is denoted by a circled bullet. There are two possible worlds, denoting the possibility that general $\mathbf{a}$ will attack at dawn ($w_1$), or not ($w_2$). Both generals know that the messenger is camped with general $\mathbf{a}$. The fact that general $\mathbf{b}$ does not know whether general $\mathbf{a}$ has decided to attack is represented by the indistinguishability relations between worlds $w_1$ and $w_2$. In fact, initially, general $\mathbf{b}$ has not enough information to know whether his ally has decided to attack.

        {\centering
            \begin{tikzpicture}[-,>=stealth',shorten >=1pt,auto,semithick]
    \node (A0) []                 {$s_0$} ;
    \node (A1) [right=.1cm of A0] {$=$} ;
    \node (W1) [pointedworld, right=.3cm of A1, label=below:{$w_1 : d, m_a$}] {} ;
    \node (W2) [world,        right=2cm of W1,  label=below:{$w_2 : m_a$}] {} ;

    \path (W1)
        edge node [above] {$b$} (W2);
\end{tikzpicture}

        \par}
    \end{example}

    \begin{definition}[Truth in epistemic states]
        Let $ M = (W, R, V) $ be an epistemic model, $ w \in W $, $i \in \agentSet$, $\varnothing \neq G \subseteq \agentSet$, $p \in \atomSet$ and $\varphi,\psi \in \Lang{C}$ be two formulae. Then,
        
        {\centering
        $
            \begin{array}{@{}lll}                    
                (M, w) \models \atom{p}            & \text{ iff } & w \in V(\atom{p}) \\
                (M, w) \models \neg \varphi        & \text{ iff } & (M, w) \not\models \varphi \\
                (M, w) \models \varphi \wedge \psi & \text{ iff } & (M, w) \models \varphi \text{ and } (M, w) \models \psi \\
                (M, w) \models \B{i} \varphi       & \text{ iff } & \forall v \text{ if } w R_i v \text{ then } (M, v) \models \varphi \\
                (M, w) \models \CK{G} \varphi       & \text{ iff } & \forall v \text{ if } w R_G^* v \text{ then } (M, v) \models \varphi
            \end{array}
        $
        \par}

        \noindent Let $(M, W_d)$ be an epistemic state. Then,

        {\centering
        $
            \begin{array}{@{}lll}                    
                (M, W_d) \models \varphi           & \text{ iff } & (M, w) \models \varphi \textnormal{ for all } w \in W_d
            \end{array}
        $
        \par}
    \end{definition}
	
	\noindent For instance, $(M,w)\models p$ means that $p$ is true in $w$; $(M,w)\models \D{i} p$ means that in $w$ the agent $i$ admits the possibility of $\varphi$ being true, i.e., there exists a world $v$ that $i$ considers possible (i.e., with $w R_i v$) such that $(M,v)\models \varphi$; $(M,w)\models \B{i} \varphi$ means that in $w$ the agent $i$ knows $\varphi$, as $\varphi$ holds in all worlds that $i$ considers possible.
	
    We recall the notion of bisimulation for epistemic states \cite{conf/kr/Bolander2021}. 
                
    \begin{definition}[Bisimulation]
        Let $s=((W, R, V), W_d)$ and $s'=((W', R', V'), W'_d)$ be two epistemic states. We say that $s$ and $s'$ are \emph{bisimilar}, denoted by $s \bisim s'$, if there exists non-empty binary relation $ Z \subseteq W \times W' $ satisfying:

        \begin{compactitem}
            \item \emph{Atoms}: if $(w, w') \in Z$, then for all $ \atom{p} \in \atomSet $, $ w \in V(\atom{p}) $ iff $ w' \in V'(\atom{p}) $.
            \item \emph{Forth}: if $(w, w') \in Z$ and $ w R_i v $, then there exists $ v' \in W' $ such that $ w' R'_i v' $ and $ (v, v') \in Z $.
            \item \emph{Back}: if $(w, w') \in Z$ and $ w' R'_i v' $, then there exists $ v \in W $ such that $ w R_i v $ and $ (v, v') \in Z $.
            \item \emph{Designated}: if $ w \in W_d $, then there exists $w' \in W'_d$ such that $(w, w') \in Z$, and vice versa.
        \end{compactitem}

        \noindent We say that $Z$ is a \emph{bisimulation} between $s$ and $s'$.
    \end{definition}

    \noindent Throughout the rest of the paper, we assume that \emph{each epistemic state is minimal w.r.t. bisimulation}.
    We denote the fact that $(w, w') \in Z$ by $w \bisim w'$.
    Finally, we introduce a notion of $k$-bisimulation for epistemic states. The following definition follows the one in \cite{books/cup/Blackburn2001} and generalizes that of \cite{conf/ijcai/Yu2013} by considering epistemic states with (possibly) multiple designated worlds.

    \begin{definition}[$k$-bisimulation]\label{def:k-bisim}
        Let $k \geq 0$ and let $s=((W, R, V),$ $W_d)$ and $s'=((W', R', V'), W'_d)$ be two epistemic states. We say that $s$ and $s'$ are $k$-bisimilar, denoted by $s \bisim_k s'$, if there exists a sequence of non-empty binary relations $Z_k \subseteq \ldots \subseteq Z_0$ (with $Z_0 \subseteq W \times W'$) satisfying (for any $i < k$):
        \begin{compactitem}
            \item \emph{Atoms}: if $(v, v') \in Z_0$, then for all $ \atom{p} \in \atomSet $, $ v {\in} V(\atom{p}) $ iff $ v' {\in} V'(\atom{p}) $.
            \item \emph{Forth}: if $(v, v') \in Z_{i+1}$ and $ v R_i u $, then there exists $ u' \in W' $ such that $ v' R'_i u' $ and $ (u, u') \in Z_i $.
            \item \emph{Back}: if $(v, v') \in Z_{i+1}$ and $ v' R'_i u' $, then there exists $ u \in W $ such that $ v R_i u $ and $ (u, u') \in Z_i $.
            \item \emph{Designated}: if $ v \in W_d $, then there exists $v' \in W'_d$ such that $(v, v') \in Z_k$, and vice versa.
        \end{compactitem}

        \noindent We say that $Z_k$ is a \emph{$k$-bisimulation} between $s$ and $s'$.
    \end{definition}

    \subsection{Event Models and Product Update}\label{sec:event_models}
        Information change is captured by \emph{product updates} of the current epistemic state with the \emph{event model} of actions. 
        
        \begin{definition}[Event Model and Action]
            An \emph{event model} for $ \Lang{C} $ is a tuple $ \E = (E, Q, pre, post) $ where:
            \begin{compactitem}
                \item $ E \neq \varnothing $ is a finite set of events; %, called \emph{domain}; 
                \item $ Q: \agentSet \rightarrow 2^{E \times E} $ assigns to each agent $ i $ an accessibility relation $ Q(i) $ (abbreviated as $Q_i$); 
                \item $ pre: E \rightarrow \Lang{C} $ assigns to each event a \emph{precondition}; 
                \item $ post: E \rightarrow (\atomSet \rightarrow \Lang{C}) $ assigns to each event and atom a \emph{postcondition}.
            \end{compactitem}
            \noindent An \emph{action} is a pair $ (\E, E_d) $ where $ E_d \subseteq E $ is a non-empty set of designated events.
        \end{definition}

        \noindent Similarly to epistemic states, the designated events in $E_d$ represent the ``real'' events that are taking place from the perspective of an external observer.

        The pair $(E, Q)$ is called the \emph{frame} of $\E$.
 		We use the infix notation $ e Q_i f $ in place of $ (e, f) \in Q_i$. 
 		These relations are analogous to the accessibility relations of epistemic models: they are used to specify how the knowledge of each agent is affected by an action, depending on which events each agent considers possible. 
		Intuitively, the precondition of an event $e$ specify whether $e$ \emph{could happen} in a certain world $w$, whereas the postconditions of $e$ describe how such event might change the factual properties of $w$ (see Definition \ref{def:update_em}). Formally, we say that an event $e$ is \emph{applicable} in a world $w$ of $M$ if $(M,w)\models pre(e)$.
        
        \begin{example}\label{ex:e_model}
            Imagine that general $\mathbf{a}$ decides to send the messenger to general $\mathbf{b}$ (action send$_{ab}$). While doing  so, the general considers two possible outcomes:
            \begin{enumerate*}[label=\arabic*)]
                \item the messenger safely arrives to the other side of the valley, or
                \item the messenger is captured by the enemy. 
            \end{enumerate*}
            %
            In the figure below, these eventualities are represented by events $e^a_1$ and $e^a_2$, respectively. The precondition of $e^a_1$ is $pre(e^a_1) = d \land m_a$, namely the message can only arrive to general $\mathbf{b}$ if general $\mathbf{a}$ has indeed decided to attack at dawn and if the messenger is at $\mathbf{a}$'s camp. The precondition of $e^a_2$ is simply $pre(e^a_2) = \top$, since the messenger could always be captured. We represent the fact that the messenger travels from one hilltop to the other\footnote{For simplicity, we assume that the truth value of each atomic proposition remains unchanged unless explicitly specified.} by having $post(e^a_1)(m_a) = \bot$ and $post(e^a_1)(m_b) = \top$. Finally, we denote the fact that the messenger is captured by having $post(e^a_2)(m_a) = \bot$ and $post(e^a_2)(m_b) = \bot$.

            {\centering
                \begin{tikzpicture}[-,>=stealth',shorten >=1pt,auto,semithick]
    \node (A0) []                 {send$_{ab}$} ;
    \node (A1) [right=.1cm of A0] {$=$} ;
    \node (E1) [pointedevent, right=.3cm of A1, label=below:{$e^a_1$}] {} ;
    \node (E2) [event,        right=2cm of E1, label=below:{$e^a_2$}] {} ;

    \path (E1)
        edge node [above] {$a$} (E2);
\end{tikzpicture}

            \par}

            Action send$_{ba}$ is defined similarly, by having $pre(e^b_1) = d \land m_b$, $pre(e^b_2) = \top$, $post(e^b_1)(m_a) = \top$, $post(e^b_1)(m_b) = \bot$, $post(e^b_2)(m_a) = \bot$ and $post(e^b_2)(m_b) = \bot$.

            {\centering
                \begin{tikzpicture}[-,>=stealth',shorten >=1pt,auto,semithick]
    \node (A0) []                 {send$_{ba}$} ;
    \node (A1) [right=.1cm of A0] {$=$} ;
    \node (E1) [pointedevent, right=.3cm of A1, label=below:{$e^b_1$}] {} ;
    \node (E2) [event,        right=2cm of E1, label=below:{$e^b_2$}] {} ;

    \path (E1)
        edge node [above] {$b$} (E2);
\end{tikzpicture}

            \par}
        \end{example}
        
        The \emph{product update} formalizes the execution of an action $(\E, E_d)$ on the current epistemic state $(M, W_d)$. Intuitively, the resulting epistemic state $(M', W_d')$ is computed by a cross product between the worlds in $M$ and the events in $\E$. A pair $(w,e)$ represents the world of $M'$ that results from applying the event $e$ on the world $w$.
        %
        We say that $ (\E, E_d) $ is \emph{applicable} in $ (M, W_d) $ iff for each world $ w_d \in W_d $ there exists an event $e_d\in E_d$ that is applicable in $w_d$. 
        
        \begin{definition}[Product Update]\label{def:update_em}
            Let $ (\E, E_d) $ be an action applicable in an epistemic state $ (M, W_d) $, where $ M = (W, R, V) $ and $ \E = (E, Q, pre, post) $. The \emph{product update} of $ (M, W_d) $ with $ (\E, E_d) $ is the epistemic state $ (M, W_d) \otimes (\E, E_d) = ((W', R', V'), W'_d) $, where:

            {\centering
            $
                \begin{array}{l@{\;}ll}
                    W'	         & = \{(w, e) \in W {\times} E \mid (M, w) \models pre(e)\} \\
                    R'_i	     & = \{((w, e), (v, f)) \in W' {\times} W' \mid w R_i v \text{ and } e Q_i f\} \\
                    V'(\atom{p}) & = \{(w, e) \in W' \mid (M, w) \models post(e)(\atom{p})\} \\
                    W'_d	     & = \{(w, e) \in W' \mid w \in W_d \text{ and } e \in E_d\}
                \end{array}
            $
            \par}
        \end{definition}

        
        \begin{example}\label{ex:update}
            Suppose that general $\mathbf{a}$ sends the messenger to general $\mathbf{b}$ (action send$_{ab}$) and that the message is successfully delivered. The situation is represented by epistemic state $s_1$, where $w'_1 = (w_1, e^a_1)$, $w'_2 = (w_1, e^a_2)$ and $w'_3 = (w_2, e^a_2)$ (recall that the reflexive, symmetric and transitive closures of the relations are left implicit).

            {\centering
                \section{Introduction}
The fundamental dichotomy in contact topology separates manifolds into the collection of overtwisted contact manifolds, which are flexible in the sense that an $h$-principle holds by the seminal work of Eliashberg \cite{zbMATH04121041} and Borman-Eliashberg-Murphy \cite{zbMATH06567662}, and the collection of tight contact manifolds, where some forms of symplectic rigidity are expected. Understanding the boundary between these two phenomena in various forms is at the heart of contact topology. 
 
One way to study contact manifolds is from a surgical perspective. Weinstein \cite{zbMATH00011093} showed that one can modify a contact manifold by attaching a symplectic handle along a neighborhood of an isotropic sphere, which is now referred to as a Weinstein handle \cite{zbMATH06054083}. Such a procedure is called an isotropic surgery by Conway and Etnyre \cite{zbMATH07206659}. One can reverse the procedure by attaching a symplectic handle along a neighborhood of a coisotropic sphere,  this leads to the concept of coisotropic surgeries \cite{zbMATH07206659}. Among them, arguably, the most interesting case is when the sphere is both isotropic and coisotropic, i.e.\ Legendrian. An isotropic surgery along a Legendrian sphere is often called a contact $(-1)$ surgery, while the coisotropic surgery along the Legendrian sphere is called a contact $(+1)$-surgery. Ding and Geiges \cite{zbMATH02103046} showed that every closed\footnote{All contact manifolds are assumed to be closed in this paper.} contact $3$-manifold can be obtained by contact $(\pm 1)$-surgery along a Legendrian link in the standard contact $3$-sphere. In higher dimensions,  Conway and Etnyre \cite{zbMATH07206659} showed that any contact manifold can be obtained from the standard contact sphere from a sequence of isotropic and coisotropic surgeries. Therefore, to determine whether a contact manifold is overtwisted or tight, one needs to understand if tightness is preserved in surgeries. In dimension $3$, by the work of Colin \cite{MR1447038} and Wand \cite{zbMATH06487151}, isotropic surgeries preserve tightness. On the other hand, the contact $(+1)$ surgery along the standard unknot in the standard contact $3$-sphere yields a tight manifold, while we have an overtwisted manifold if we stabilize the unknot and apply the surgery. Hence the devil in the question is coisotropic surgeries, in particular, contact $(+1)$-surgeries. 

Invariants from pseudo-holomorphic curves, e.g.\ symplectic field theory (SFT) by Eliashberg, Givental, and Hofer \cite{zbMATH01643843} and Heegaard Floer theory by Ozsv\'ath and Szab\'o  \cite{zbMATH02144173}, provide necessary conditions for a contact manifold to be overtwisted, namely the contact homology must vanish \cite{zbMATH05709738} from the SFT side and the contact Ozsv\'ath-Szab\'o invariant must vanish \cite{zbMATH02207895} from the Heegaard Floer theory side. Bourgeois and Niederkr{\"u}ger \cite{zbMATH05658836} introduced the notion of algebraically overtwisted manifolds for those with vanishing contact homology. However, neither conditions are sufficient by Avdek \cite{avdek2020combinatorial} and Ghiggini, Honda, and Van Horn-Morris \cite{ghiggini2007vanishing}, hence the combination of the two vanishing properties does not imply overtwistedness either by a contact connected sum. From the surgical perspective, the non-vanishing of contact homology and the non-vanishing of the contact Ozsv\'ath-Szab\'o invariant (both hold for the standard contact sphere when they can apply) are preserved in isotropic surgeries by the functoriality of those invariants. While their behaviors under coisotropic surgeries are more complicated as illustrated by the same example above. Even though both conditions are not sufficient to determine overtwistedness, understanding them in $(+1)$-surgeries can be viewed as the first step towards the geometric question of overtwistedness through coisotropic surgeries. On the Heegaard Floer theory side, a complete answer for the vanishing of the contact Ozsv\'ath-Szab\'o invariant in $(+1)$-surgery along a Legendrian knot was obtained by Golla \cite{zbMATH06413573}. In \cite{DLW}, Ding, Li, and Wu studied the vanishing of the contact Ozsv\'ath-Szab\'o invariant for $(+1)$-surgeries on two-component links. On the SFT side, the vanishing of contact homology through $(+1)$-surgeries was first studied by Avdek \cite{avdek2020combinatorial} in the standard contact $3$-sphere. In this paper, we study the same question but for general dimensions. In particular, our main theorem below can be viewed as an SFT analog of Ding-Li-Wu's result. 
\begin{theorem}\label{thm:main}
    Let $Y^{2n-1}$ be the contact boundary of a Liouville domain $W$, where $W$ is one of the following:
    \begin{enumerate}
        \item $W=V\times \D$ for a Liouville domain $V$ and $\D\subset \C$ is the unit disk, in particular, any subcritical Weinstein domain.
        \item $W$ is a flexible Weinstein domain \cite{zbMATH06054083} with $c_1(W)\in H^2(W;\Z)$ torsion.
    \end{enumerate}
    Let  $\Lambda$ be a Legendrian sphere in $Y$, such that $[\Lambda]\in H_{n-1}(\partial W;\Q)$ is nontrivial in $H_{n-1}(W;\Q)$.   Then the contact manifold $Y_{\Lambda}$ from a $(+1)$-surgery\footnote{The $+1$ surgery depends on a parametrization $L\simeq S^n$.} along $\Lambda$ is algebraically overtwisted, i.e.\ the contact homology vanishes. 
\end{theorem}
\begin{remark}\label{rmk:twist}
    Moreover, the contact homology over the twisted coefficient $\Q[H_2(Y_{\Lambda};\R)]$\footnote{It corresponds to $\cR=0$, i.e.\ the fully twisted theory in \cite{zbMATH06000009}.}  also vanishes for all contact manifolds from $(+1)$-surgeries in this paper. This implies all such contact manifolds have no weak fillings by \cite[Theorem 5]{zbMATH06000009}.
\end{remark}

An instant corollary of \Cref{thm:main} is the following.
\begin{corollary}\label{cor:OT}
    Let $V$ be a Liouville domain and $L\subset V$ be a Lagrangian sphere such that $[L]\ne 0 \in H_*(V;\Q)$, then for any Dehn-Seidel twist $\tau_L$\footnote{As a Dehn-Seidel twist also depends on a parametrization $L\simeq S^n$.}, the open book $\mathrm{OB}(V,\tau^{-1}_{L})$ with page $V$ and monodromy $\tau^{-1}_L$ has vanishing contact homology.
\end{corollary}
\begin{proof}
    The open book $\mathrm{OB}(V,\tau^{-1}_{L})$ is obtained from $(+1)$-surgery on the Legendrian lift of $L$ in the open book $\mathrm{OB}(V,\Id)=\partial(V\times \D)$, then \Cref{thm:main} applies as $H_*(V;\Q)\to H_*(\partial(V\times \D);\Q) \to H_*(V\times \D;\Q)$ is injective. 
\end{proof}

In particular, homotopically standard overtwisted $S^{2n+1}= \mathrm{OB}(T^*S^n,\tau^{-1})$ has vanishing contact homology, this was established by Bourgeois and van Koert \cite{zbMATH05709738} by direct computation. The assumption on the fundamental class of $L$ is likely to be redundant in view of the regular Lagrangian conjecture of Eliashberg, Ganatra, and Lazarev \cite[Problem 2.5]{zbMATH07195660}. Although many of the open books in \Cref{cor:OT} are negative stabilization, hence overtwisted \cite{zbMATH07010365}, it is unclear whether \Cref{cor:OT} always yield overtwisted manifolds.

\begin{corollary}\label{cor:3D}
    Let $\Lambda \cup U$ be a two-component link in $(S^3,\xi_{\std})$ with a nontrivial linking number and $U$ is the standard unknot, then the $(+1)$ surgeries along $\Lambda \cup U$ yield a contact manifold with vanishing contact homology.
\end{corollary}
\begin{proof}
    We first apply $(+1)$ surgery along $U$ to get $Y=\partial(T^*S^1\times \D)=S^1\times S^2$, then $\Lambda$ becomes a Legendrian knot $\Lambda'$ on $Y$ representing a nontrivial homology class in the $S^1$ factor as the linking number is nontrivial. Then we apply \Cref{thm:main} to $\Lambda'$. 
\end{proof}

Ding, Li, and Wu \cite[Theorem 1.1]{DLW} showed that the contact Ozsv\'ath-Szab\'o invariant also vanishes for contact manifolds in \Cref{cor:3D}. In fact, they established the vanishing result for other types of $U$, which are ``unknots" in the Heegaard Floer theory sense. On the other hand, the nontrivial linking number is a crucial requirement, and so is the homology requirement in our formulation. Moreover, our construction enjoys a local property as follows.

\begin{theorem}\label{thm:main'}
    In the following two cases:
    \begin{enumerate}
        \item\label{thm1} $Y_1$ is flexibly fillable or $Y_1=\partial(V\times \D)$ such that $c_1(Y)$ is torsion, $Y_2$ is a contact manifold of the same dimension with $c_1(Y_2)$ torsion.
        \item\label{thm2} $Y_1=\partial(V\times \D)$ for a Weinstein domain $V$, $Y_2$ is a contact manifold of the same dimension.
    \end{enumerate}
    If $\Lambda$ is a Legendrian sphere in $Y=Y_1\# Y_2$, such that $[\Lambda]$ has non-trivial image through $H_{n-1}(Y;\Q)\simeq H_{n-1}(Y_1;\Q)\oplus H_{n-1}(Y_2;\Q)\to H_{n-1}(Y_1;\Q) \to H_{n-1}(W;\Q)$, where $W$ is the natural filling in \Cref{thm:main}, then $Y_{\Lambda}$ is algebraically overtwisted.
\end{theorem}
Then we can upgrade \Cref{cor:3D} to the following for the special case of $Y_1=\partial(T^*S^{n-1}\times \D)$.
\begin{corollary}
    Let $\Lambda,U$ be two Legendrian spheres in $Y$, where $Y$ is a $2n-1$ dimensional contact manifold, and $U$ is a standard unknot in a Darboux chart. If the linking number is nontrivial\footnote{Here the linking number is defined to the intersection number of $\Lambda$ with a bounding ball of $U$ in the Darboux chart.},  the $(+1)$ surgeries along $\Lambda \cup U$ yield a contact manifold with vanishing contact homology.
\end{corollary}
As any overtwisted contact manifold $Y\# (S^{2n-1},\xi_{\mathrm{ot}})$ can be written as $(+1)$ surgeries from such links, this yields another proof of overtwisted contact manifold having vanishing contact homology, which was first proved by Bourgeois and van Koert \cite{zbMATH05709738}. Combined with \Cref{rmk:twist}, this gives another proof of the following:
\begin{corollary}[\cite{zbMATH06182635,schmaltz2020non}]
    Overtwisted contact manifolds have no weak filling.
\end{corollary}

A $(+1)$-surgery gives rise to a Weinstein cobordism whose concave boundary is $Y_{\Lambda}$, while the convex boundary is $Y$ and $\Lambda$ is filled by a Lagrangian disk in the cobordism. On the other hand, contact manifold $Y$ in \Cref{thm:main} enjoys strong uniqueness property for symplectic fillings by \cite{zbMATH07367119,zbMATH07673358}, in particular, the homology class $[\Lambda]$ should survive in the filling. Indeed, one can apply \cite[Theorem 4.4]{bowden2022making} to prove that the contact manifold from $(+1)$-surgery has no strong fillings. The proof of \Cref{thm:main,thm:main'} follows from singling out the pseudo-holomorphic curves obstructing fillings, whose degeneration in the surgery cobordism then yields the vanishing of contact homology of the concave boundary. Contact homology of $(+1)$ surgeries was studied by Avdek \cite{Av,avdek2020combinatorial}, where a much deeper picture between the relative SFT of the convex boundary and the absolute SFT of the concave boundary was studied. We point out here that our results remain in the realm of absolute SFT, i.e.\ we only use the topology of the surgery cobordism but not holomorphic curves with Lagrangian boundary conditions. 

Our proof has a functorial explanation as follows. Let $Y$ be a contact manifold, one tries to define the positive symplectic cohomology, where the underlying cochain complex $C_+(Y)$ is generated by two generators from each Reeb orbit. $C_+(Y)$ does not always form a cochain complex, but $C_+(Y)\otimes \CC(Y)$ is a $\CC(Y)$ DGA-module, where $\CC(Y)$ is the contact homology algebra (chain level) of $Y$ and the differential counts Floer cylinders with negative punctures asymptotic to Reeb orbits. The cochain complex for positive symplectic cohomology of an exact filling $W$ is then the tensor product with the ground field using the augmentation from $W$. Now let $W$ be an exact cobordism (e.g.\ the surgery cobordism) from concave boundary $Y_-$ to convex boundary $Y_+$, then we have the following diagram, which is commutative on homology,
$$
\xymatrix{
C_+(Y_+)\otimes \CC(Y_+)\ar[d] \ar[r] & C(Y_+)\otimes \CC(Y_+)\ar[d]\\
C(W,Y_-) \otimes \CC(Y_-)\ar[r] & C(Y_+)\otimes \CC(Y_-)}
$$
where $C(Y_{\pm}),C(W,Y_-)$ are Morse cochain complexes. When phrased using such a structure, the core of the proofs is finding a closed class in $C_+(Y_+)\otimes \CC(Y_+)$ that is mapped to $\alpha \otimes 1 \in H^*(Y_+)\otimes \CH(Y_+)$ through the top map, such that $\alpha$ is not in the image of $H^*(W,Y_-)\to H^*(Y_+)$. Then we must have $1=0$ in $\CH(Y_-)$. However, such an element is easy to find for $Y_+=Y$ in \Cref{thm:main,thm:main'}.

It is quite a challenge to determine whether contact manifolds in \Cref{thm:main,thm:main'} are overtwisted. In dimension $3$, there are sufficient conditions for the $(+1)$-surgeries to yield overtwisted manifolds by Ozbagci \cite{zbMATH02147034} and Lisca-Stipsicz \cite{zbMATH05190395} for knots and Ding-Li-Wu \cite{DLW} for links. In higher dimensions, Casals, Murphy, and Presas \cite{zbMATH07010365} showed that $(+1)$-surgeries along loose Legendrian spheres give overtwisted manifolds. Indeed, some cases of \Cref{thm:main} give overtwisted manifolds, for example, $W=T^*{S^{n-1}}\times \D$ and $\Lambda$ is the Legendrian lift of the Lagrangian zero section in $T^*S^{n-1}$, as this Legendrian is loose/stabilized. On the other hand, there are Legendrian knots with the homology property in \Cref{thm:main} that are not stabilized found by Ekholm and Ng \cite[Corollary 2.22, Proposition 3.9]{zbMATH06471194}. In higher dimensions, we have many such Legendrians from exotic Weinstein balls constructed in \cite{abouzaid2010altering,zbMATH05553983,zbMATH02242665,zbMATH07367119} using the work of Lazarev \cite{zbMATH07305775}.
\begin{proposition}\label{prop:exotic}
    For $Y=\partial(T^*S^{n-1}\times \D)\simeq S^{n-1}\times S^n$ with $n\ge 3$, there are infinitely many different non-loose Legendrian spheres in $Y$ that are smoothly isotopic to the standard loose $S^{n-1}$ above. When $n$ is odd\footnote{We expect this condition to be redundant.}, those Legendrians are formally Legendrian isotopic to the standard loose $S^{n-1}$.
\end{proposition}

One of the motivations of this project is to study the differences between overtwisted manifolds and algebraically overtwisted manifolds. 
\begin{question}[Folklore]\label{question:AO}
For $n\ge 2$, are there algebraically overtwisted but tight $2n-1$ dimensional contact manifolds?
\end{question}
To put it in a broader perspective, this question is one of the fundamental questions to understand the boundary between flexibility and rigidity phenomena in symplectic and contact topology. The first example of an algebraically overtwisted tight manifold was found by Avdek \cite{avdek2020combinatorial} in dimension $3$. The example follows from a $(+1)$-surgery on $(S^3,\xi_{\std})$ along a trefoil knot, and the tightness follows from the non-vanishing of the contact Ozsv\'ath-Szab\'o invariant. In view of \cite[Theorem 1.1]{DLW}, although \Cref{thm:main} may give new examples in dimension $3$, the tight criterion from contact Ozsv\'ath-Szab\'o invariant can not help, i.e.\ we need other criteria of tightness. In fact, the lack of tight criteria beyond contact homology is the major difficulty in answering \Cref{question:AO} in higher dimensions. Indeed, the existence of fillings, hypertight property, and properties on Conley-Zehnder indices, used as tight criteria in general dimensions, are all manifestations of the non-vanishing of contact homology. Nevertheless, \Cref{thm:main} potentially solves the easy half of \Cref{question:AO} by providing a flexible enough list of algebraically overtwisted manifolds. More precisely, we ask the following question. 

\begin{question}
    If we apply a $(+1)$-surgery along Legendrian spheres in \Cref{prop:exotic}, do we get (different) tight contact manifolds? 
\end{question}
We suspect the answer to be affirmative, for otherwise, the $(+1)$-surgeries would yield the homotopically standard overtwisted sphere, i.e.\ we get infinitely many different ways to get the homotopically standard overtwisted sphere but with the same formal data (at least for $n$ odd). 
\subsection*{Acknowledgments}
The author is supported by the National Natural Science Foundation of China under Grant No.\ 12288201 and 12231010. The author is grateful to Russell Avdek for enlightening discussions which lead to the functorial perspective in \S \ref{ss:43}, and Otto van Koert for pointing out \cite{zbMATH06562001} which leads to the proof of \Cref{prop:corb'} and their feedback on a preliminary version of the paper. The author would like to thank Youlin Li and Zhongtao Wu for helpful discussions and interest in the project. 
            \par}
            
            Now general $\mathbf{b}$ knows about the intentions of his ally ($\B{b}d$), but general $\mathbf{a}$ does not know that general $\mathbf{b}$ knows. So, $\mathbf{b}$ decides to send the messenger back to acknowledge that the message was received (action send$_{ba}$). Here, $d$ assumes the meaning of ``general $\mathbf{b}$ has received the message''. Assume again that the messenger succeeds. We obtain the epistemic state $s_2$, where $w''_1 = (w'_1, e^b_1)$, $w''_2 = (w'_1, e^b_2)$, $w''_3 = (w'_2, e^b_2)$ and $w''_4 = (w'_3, e^b_2)$.
            
            {\centering
                \begin{tikzpicture}[-,>=stealth',shorten >=1pt,auto,semithick]
    \node (A0) []                 {$s_2$} ;
    \node (A1) [right=.1cm of A0] {$=$} ;
    \node (W1) [pointedworld, right=.3cm of A1,  label=below:{$w''_1 : d, m_a$}] {} ;
    \node (W2) [world,        right=1.5cm of W1, label=below:{$w''_2 : d$}] {} ;
    \node (W3) [world,        right=1.5cm of W2, label=below:{$w''_3 : d$}] {} ;
    \node (W4) [world,        right=1.5cm of W3, label=below:{$w''_4 $}] {} ;

    \path (W1)
        edge node [above] {$b$} (W2) ;
    
    \path (W2)
        edge node [above] {$a$} (W3) ;

    \path (W3)
        edge node [above] {$b$} (W4) ;
    
\end{tikzpicture}

            \par}

            Now it holds that $\B{a}\B{b}d$, but it does not hold that $\B{b}\B{a}\B{b}d$. So, general $\mathbf{a}$ would need to send the messenger once again to general $\mathbf{b}$. However, it can be intuitively seen that, regardless of how many messages the generals exchange, they will never be sure that the other will attack at dawn. This will be stated formally in Section \ref{sec:plan_ex}.
        \end{example}

    \subsection{The logic S5$_n$}\label{sec:s5}
        In the epistemic logic literature there exist many different axiomatizations of the concept of \emph{knowledge}. In this paper, we adopt the multimodal logic S5$_n$. Its axioms are\footnote{Even though \axiom{K}, \axiom{T} and \axiom{5} are sufficient to characterize S5$_n$, we include axiom \axiom{4} as it constitutes an important epistemic principle.}:
        
        {\centering
        $
            \begin{array}{l@{}l@{}r@{}}
                \axiom{K}\;\; & \B{i} (\varphi \rightarrow \psi) \rightarrow
                                (\B{i} \varphi \rightarrow \B{i} \psi)  			& \textnormal{(Distribution)}           \\
                \axiom{T} & \B{i} \varphi \rightarrow \varphi                       & \textnormal{(Knowledge)}              \\
                \axiom{4} & \B{i} \varphi \rightarrow \B{i} \B{i} \varphi           & \textnormal{(Positive introspection)} \\
                \axiom{5} & \neg \B{i} \varphi \rightarrow \B{i} \neg \B{i} \varphi & \textnormal{(Negative introspection)}
            \end{array}
        $
        \par}

        \noindent Axioms \axiom{T}, \axiom{4} and \axiom{5} correspond, to the following \emph{frame properties}: reflexivity ($ \forall u (u R_i u) $), transitivity ($ \forall u, v, w (u R_i v \wedge v R_i w \rightarrow u R_i w) $) and Euclidicity ($ \forall u, v, w (u R_i v \wedge u R_i w \rightarrow v R_i w) $). 
        Moreover, axioms \axiom{T} and \axiom{5} together entail symmetry ($ \forall u, v (u R_i v \rightarrow v R_i u) $). 
        Thus, accessibility relations in S5$_n$ are equivalence relations.
        We refer to epistemic states (resp., epistemic models, actions, frames) satisfying the axioms of a logic $L$ as $L$-states (resp., $L$-models, $L$-actions, $L$-frames). In the rest of the paper, we assume that the accessibility relations of epistemic states and actions are equivalence relations.

\subsection{Plan Existence Problem}\label{sec:plan_ex}
	We now define our problem, adapting the formulation in~\cite{conf/ijcai/Aucher2013}. 

    \begin{definition}[Planning Task]
    \label{def:planning_task}
        An \emph{(epistemic) planning task} is a triple $ T = (s_0, \actionSet,$ $ \varphi_g) $, where $ s_0 $ is an initial epistemic state; $ \actionSet $ is a finite set of actions; $ \varphi_g \in \Lang{C} $ is a \emph{goal formula}.
    \end{definition}

    Given a logic $ L $, an \emph{$L$-planning task} $ (s_0, \actionSet, \varphi_g) $ is a planning task where $s_0$ is an $L$-state and each action in $\actionSet$ is an $L$-action. We denote the class of $L$-planning tasks with $\mathcal{T}_L$.
    We remark that, given a generic logic $L$, the product update of an L-state with an L-action, in general, is not necessarily an L-state. This is not a desired outcome in general, since axioms model some principles of knowledge/belief that always need to be satisfied.
    For instance, this is the case of the logic KD45$_n$, that captures the concept of \emph{belief}. In the literature, there exist different approaches to guarantee the preservation of the KD45$_n$ frame properties after the product update. For instance, some techniques involve belief revision techniques \cite{workshop/nrac/Herzig2005}, whereas others focus on defining some additional conditions to impose to both states and actions \cite{conf/aaai/Son2015}.

    The case of our logic C-S5$_n$ is similar to that of KD45$_n$. In fact, the frame property corresponding to axiom \axiom{C} (see Equation \ref{eq:A-frame} in Section \ref{sec:new_logic}) is not guaranteed to hold after the application of an action. In this paper, rather than devising some technique to guarantee the preservation of frame property \ref{eq:A-frame}, we instead opt for a rollback-style approach: an action is not to be applied in a state if it would lead to violate the axioms of C-S5$_n$.
    This leads to the following definition.

    \begin{definition}[Solution]\label{def:solution}
        A \emph{solution} to an $L$-planning task $(s_0, \actionSet,$ $ \varphi_g)$ is a finite sequence $ \alpha_1, \dots, \alpha_m $ of actions of $\actionSet$ such that:
        \begin{compactenum}
            \item $ s_0 \otimes \alpha_1 \otimes \dots \otimes \alpha_m \models \varphi_g $, and
            \item For each $ 1 \leq k \leq m $, $ \alpha_k $ is applicable in $ s_0 \otimes \alpha_1 \otimes \dots \otimes \alpha_{k-1} $ and $s_0 \otimes \alpha_1 \otimes \dots \otimes \alpha_k$ is an $L$-state.
        \end{compactenum}
    \end{definition}
    
    \begin{definition}[Plan Existence Problem]\label{def:plan_ex_problem}
        Let $n \geq 1$ and $\mathcal{T}_L$ be a class of epistemic planning tasks for a logic $L$. \planex{$\mathcal{T}_L$}{$n$} is the following decision problem: ``Given an $L$-planning task $ T = (s_0, \actionSet, \varphi_g) \in \mathcal{T}_L $, where $|\agentSet|=n$, does $ T $ have a solution?''
    \end{definition}

    
    \begin{example}\label{ex:task}
        In Example~\ref{ex:update} we have seen that, intuitively speaking, the two generals can not coordinate a winning attack. We now state this formally. Let $T_\textnormal{coord} = (s_0, \actionSet, \varphi_g) $ be an S5$_n$-planning task, where $\actionSet = \{\textnormal{send}_{ab}, \textnormal{send}_{ba}\}$ and $\varphi = \CK{\{a,b\}}d$. Then, $T_\textnormal{coord}$ has no solution. In fact, for any number $k \geq 0$ of delivered messages, one can show by induction the following (where $h\geq 0$):
        \begin{compactitem}
            \item $k{=}2h$: $(\B{a}\B{b})^h\B{a} d $ holds in $s_k$, but not $(\B{b}\B{a})^{h+1}d$;
            \item $k{=}2h{+}1$: $(\B{b}\B{a})^{h+1}d $ holds in $s_k$, but not $(\B{a}\B{b})^{h+1}\B{a}d$.
        \end{compactitem}
        \noindent Thus, common knowledge can not be achieved between the two generals in a finite number of steps. However, any search algorithm would never terminate, since at each step there is exactly one applicable action that, when applied, results in a new S5$_n$-state.
    \end{example}
