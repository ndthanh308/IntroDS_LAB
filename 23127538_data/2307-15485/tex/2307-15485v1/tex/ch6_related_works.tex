\section{Related Works}\label{sec:rel_works}
    The DEL semantics has been widely exploited to obtain complexity and decidability results for the plan existence problem \cite{conf/ijcai/Aucher2013,journals/corr/Aucher2014,journals/jancl/Bolander2011,journals/ai/Bolander2020,conf/ijcai/Bolander2015,conf/ijcai/Charrier2016,conf/lori/Lowe2011,conf/ijcai/Yu2013}.
    We identify three main lines of research.
    %
    The first one restricts the class of actions that are allowed in planning tasks by means of syntactical conditions. In particular, some approaches constrain actions to be \emph{purely epistemic} (\ie without postconditions), while others introduce limitations to the modal depth preconditions and/or postconditions. Following the notation of \cite{journals/ai/Bolander2020}, we denote with $\mathcal{T}(\ell, m)$ the class of epistemic planning tasks in which the modal depth of the preconditions are $\leq \ell$, and that of the postconditions are $\leq m$. Similarly, planning tasks with purely epistemic actions are denoted with $\mathcal{T}(\ell,-1)$. We report the main results in Table \ref{tab:complexity1}.

    \begin{table}[t]
        \centering
        \begin{tabular}{|l|l|}
            \hline
            \planex{$\mathcal{T}(0,-1)$}{$m$} & \textsc{pspace}-complete \cite{conf/ijcai/Charrier2016} \\ \hline
            \planex{$\mathcal{T}(1,-1)$}{$n$} & \textsc{unknown}         \cite{conf/ijcai/Charrier2016} \\ \hline
            \planex{$\mathcal{T}(2,-1)$}{$n$} & \textsc{undecidable}     \cite{conf/ijcai/Charrier2016} \\ \hline
            \planex{$\mathcal{T}(0,0)$}{$n$}  & \textsc{decidable}       \cite{conf/ijcai/Yu2013,journals/corr/Aucher2014} \\ \hline
            \planex{$\mathcal{T}(1,0)$}{$n$}  & \textsc{decidable}       \cite{journals/ai/Bolander2020} \\ \hline
        \end{tabular}
        \caption{Decidability and complexity results of plan existence problem based on the \emph{syntactical} approach.}
        \label{tab:complexity1}
    \end{table}

    The second line of research pivots both on syntactical limitations to formulae and on constraining the frames of event models (\eg singletons, chains, trees). In particular, \planex{$\mathcal{T}(0,-1)$}{$n$} is \textsc{np}-complete on singletons and chains and it is \textsc{pspace}-complete on trees \cite{conf/ijcai/Bolander2015}, whereas \planex{$\mathcal{T}(\ell, m)}{n}$ (for any $\ell, m \geq 0$) on singletons is \textsc{pspace}-hard \cite{phd/dtu/Jensen2014}.

    Finally, the third line of research revolves around the choice of the logic for epistemic models and actions \cite{conf/ijcai/Aucher2013}. This approach is more similar to the one we adopted in this paper, with the difference that the logics considered in previous works are a combination of standard and well-known axioms of epistemic logic (see Section \ref{sec:new_logic}). We report the main results in Table \ref{tab:complexity2}. For a more detailed analysis of complexity and decidability results in epistemic planning we refer the interested reader to \cite{journals/ai/Bolander2020}.

    \begin{table}[t]
        \centering
        \begin{tabular}{|c|c|}
            \hline
            Logic                                                 & Decidability                                             \\ \hline
            K$_n$, K$_n$, KT$_n$, K4$_n$, K45$_n$, S4$_n$, S5$_n$ & \textsc{undecidable} \cite{conf/ijcai/Aucher2013}        \\ \cline{1-1}
            C$^b$-S5$_n$ ($n{>}2$)                                & \cellcolor{gray!35} \textsc{undecidable}                 \\
            C$^b$-S5$_2$                                          & \cellcolor{gray!15}                                      \\ \cline{1-1}
            wC$_\ell$-S5$_n$                                      & \cellcolor{gray!15}                                      \\ \cline{1-1}
            C-S5$_n$                                              & \cellcolor{gray!15} \multirow{-3}{*}{\textsc{decidable}} \\ \hline
        \end{tabular}
        \caption{Decidability results of plan existence problem based on the semantic approach, compared to our results (in gray).}
        \label{tab:complexity2}
    \end{table}
