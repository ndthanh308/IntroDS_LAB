\usepackage[table]{colortbl}
\usepackage{xifthen}
\usepackage{mathtools}
\usepackage{tikz-cd}
\usetikzlibrary{decorations.pathmorphing}

\newcommand{\B}[1]{\Box_{#1}}
\newcommand{\D}[1]{\Diamond_{#1}}
\newcommand{\CK}[1]{C_{#1}}
\newcommand{\BB}[2]{\blacksquare_{#1|#2}}

\newcommand{\atom}[1]{#1}

\newcommand{\atomSet}{\mathcal{P}}
\newcommand{\statesSet}{\mathcal{S}}
\newcommand{\agentSet}{\mathcal{AG}}
\newcommand{\actionSet}{\mathcal{A}}

\newcommand{\E}{\mathcal{E}}
\newcommand{\Lang}[1]{\mathcal{L}_{\atomSet, \agentSet}^{#1}}

\newcommand{\axiom}[1]{\textbf{#1}}
\newcommand{\planex}[2]{\textnormal{\textsc{PlanEx}}(#1, #2)}


%%%%%%%%%%%%%%%%%%%%%%%%%%%%%% SUPPLEMENTARY DOCUMENT STUFF %%%%%%%%%%%%%%%%%%%%%%%%%%%%%%
\newcommand\boldparagraph[1]{\textbf{\textit{#1}}~}

\newcommand{\METAWORLD}{\textsf{META-WORLD}}
\newcommand{\METACHAIN}{\textsf{META-CHAIN}}
\newcommand{\METAINC}{\textsf{META-INC}}
\newcommand{\METADEC}{\textsf{META-DEC}}
\newcommand{\METAREPL}{\textsf{META-REPL}}
\newcommand{\METASTATE}{\textsf{META-S}}
\newcommand{\gsm}[2]{\textsc{GSM}(#1,#2)}

\tikzstyle{world}       =[circle,   thick,draw=black,       fill=black,minimum size=5pt,inner sep=0pt]
\tikzstyle{pointedworld}=[circle,   thick,draw=black,double,fill=black,minimum size=5pt,inner sep=0pt]
\tikzstyle{event}       =[rectangle,thick,draw=black,       fill=black,minimum size=5pt,inner sep=0pt]
\tikzstyle{pointedevent}=[rectangle,thick,draw=black,double,fill=black,minimum size=5pt,inner sep=0pt]
\tikzstyle{itria}       =[draw,isosceles triangle,shape border rotate=90,inner sep=2pt,outer sep=-0.5pt,anchor=north]

%\newcommand{\prepost}[2]{\langle #1, #2 \rangle}
\newcommand{\precond}[1]{\langle #1, \mathtt{id}\rangle}

%%%%%%%%%%%%%%%%%%%%%%%%%%%%%% END SUPPLEMENTARY DOCUMENT STUFF %%%%%%%%%%%%%%%%%%%%%%%%%%%%%%

% bisimulation
\newcommand\bisim{\underline{\leftrightarrow}}

% graphics: worlds
\newcommand*\world{
    \tikz{
        \node[shape=circle,fill=black,inner sep=2.2pt] {};
        % \node[shape=circle,draw=white,inner sep=2.7pt] {};
    }
}

\newcommand*\pointedworld{
    \tikz{
        \node[shape=circle,fill=black,inner sep=1.25pt] {};
        \node[shape=circle,draw=black,inner sep=2.5pt] {};
    }
}

% graphics: worlds
\newcommand*\event{
    \tikz{
        \node[shape=rectangle,fill=black,inner sep=2.4pt] {};
        \node[shape=rectangle,draw=white,inner sep=2.7pt] {};
    }
}

\newcommand*\pointedevent{
    \tikz{
        \node[shape=rectangle,fill=black,inner sep=1.6pt] {};
        \node[shape=rectangle,draw=black,inner sep=2.7pt] {};
    }
}

% Squig arrows with superscript/subscript
\newcommand\xrsquigarrow[1]{%
    \mathrel{%
        \begin{tikzpicture}[%
            baseline={(current bounding box.south)}
            ]
        \node[%
            ,inner sep=.44ex
            ,align=center
            ] (tmp) {\ $\scriptstyle #1$\ };
        \path[%
            ,draw,<-
            ,decorate,decoration={%
                ,zigzag
                ,amplitude=0.7pt
                ,segment length=1.2mm,pre length=3.5pt
                }
            ] 
        (tmp.south east) -- (tmp.south west);
        \end{tikzpicture}
    }
}

\newcommand\xlsquigarrow[1]{%
    \mathrel{%
        \begin{tikzpicture}[%
            ,baseline={(current bounding box.south)}
            ]
        \node[%
            ,inner sep=.44ex
            ,align=center
            ] (tmp) {\ $\scriptstyle #1$\ };
        \path[%
            ,draw,<-
            ,decorate,decoration={%
                ,zigzag
                ,amplitude=0.7pt
                ,segment length=1.2mm,pre length=3.5pt
                }
            ]
        (tmp.south west) -- (tmp.south east);
        \end{tikzpicture}
    }
}
