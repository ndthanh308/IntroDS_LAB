\section{Epistemic Planning Systems}\label{sec:systems}
    In this section, we look at two well-known epistemic planning systems that adopt the DEL semantics and we show their decidability by applying our previous results. These are $m\mathcal{A}^*$ \cite{journals/corr/Baral2015} and the framework by Kominis and Geffner \cite{conf/aips/Kominis2015}.
    %
    While decidability is already known for the latter, the decidability of the former is still an open problem. In what follows, we show that both these systems are captured by our setting when knowledge is considered, namely under S5$_n$ axioms. We begin by briefly introducing the two systems.
        
    % Figure environment removed

    \noindent
    \textbf{1.} The epistemic planning framework $m\mathcal{A}^*$ \cite{journals/corr/Baral2015} features three action types: 
    \emph{ontic}, \emph{sensing} and \emph{announcement} actions. 
    In $m\mathcal{A}^*$, agents are partitioned in three sets: \emph{fully observant} agents ($F$), that are able to observe the action corresponding to an event, \emph{partially observant} agents ($P$) that only know about the execution of an action, but not the effects, and \emph{oblivious} agents ($O$), that are ignorant about the fact that the action is taking place. 
    When oblivious agents are considered, however, event models fall beyond C-S5$_n$ (they have KD45$_n$-frames), as they are not \emph{symmetric} (see Figure \ref{fig:ma_star}).
    Consequently, we restrict ourselves to a fragment of $m\mathcal{A}^*$ that includes \emph{public} ontic actions and \emph{semi-private} sensing and announcement actions. This is achieved by removing from the event models of Figure \ref{fig:ma_star} all events considered possible by oblivious agents. It is easy to see that the frames of the resulting event models are indeed S5$_n$-frames.
    %
    We call the resulting system the \emph{S5$_n$-fragment} of $m\mathcal{A}^*$, and we denote with $\mathcal{T}_{\textnormal{S5$_n$-}m\mathcal{A}^*}$ the class of planning tasks of such system.
    
    \noindent
    \textbf{2.} 
    Kominis and Geffner \cite{conf/aips/Kominis2015} describe a system for handling beliefs in multi-agent scenarios. They describe three types of actions: \emph{do}, \emph{update}, and \emph{sense}. Although their formulation is not given in terms of DEL semantics, the authors briefly describe the event models corresponding to each action type (Figure \ref{fig:kom15}). We denote with $\mathcal{T}_{\textnormal{KG}}$ the class of planning tasks of such system. Differently from $m\mathcal{A}^*$, all the described event models already have S5$_n$-frames.

    % Figure environment removed

    We show that the two described systems fall within our logic, thus proving their decidability (see arXiv Appendix for full proofs):
    \begin{lemma}\label{lem:systems}
        $\mathcal{T}_{\textnormal{S5$_n$-}m\mathcal{A}^*} \subseteq \mathcal{T}_{\textnormal{C-S5}}$ and $\mathcal{T}_{\textnormal{KG}} \subseteq \mathcal{T}_{\textnormal{C-S5}}$.
    \end{lemma}

    \begin{corollary}
        For any $n>1$, \planex{$\mathcal{T}_{\textnormal{S5$_n$-}m\mathcal{A}^*}$}{$n$} and \planex{$\mathcal{T}_{\textnormal{\textbf{KG}}}$}{$n$} are \emph{decidable}.
    \end{corollary}
