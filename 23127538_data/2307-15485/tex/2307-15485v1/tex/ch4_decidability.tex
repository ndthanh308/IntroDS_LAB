\section{Epistemic Planning with Commutativity}\label{sec:decidability}
    This section is organized as follows. First, we prove the decidability of the plan existence problem under the logic C-S5$_n$. Then, in Section \ref{sec:general-comm}, we consider two generalizations of the commutativity axiom and we analyze their impact in the decidability of the plan existence problem. We first recall the following result:

    \begin{thm}[\cite{conf/ijcai/Aucher2013}]\label{th:aucher-bolander}
        For any $n > 1$, \planex{$\mathcal{T}_{\textnormal{S5}}$}{$n$} is undecidable.
    \end{thm}

    We now focus on the logic C-S5$_n$, assuming that $n>1$. We begin by giving some preliminary results.

    \begin{lemma}\label{lem:slide-box}
        Let $ G = \{i_1, \dots, i_m\} \subseteq \agentSet $, with $ m \geq 2 $ and let $ \vec{v} \in G^* $ ($ |\vec{v}| = \lambda  \geq 2$). Let $ \pi $ and $ \rho $ be two permutations of elements of $ \vec{v} $. Then, for any $\varphi$, in the logic C-S5$_n$ the following is a theorem:
        \begin{equation*}
            \B{\pi_1} \dots \B{\pi_\lambda} \varphi \leftrightarrow \B{\rho_1} \dots \B{\rho_\lambda} \varphi
        \end{equation*}
    \end{lemma}

    For a group $G$ of agents, the knowledge of agent $i_1$ about what agent $i_2$ knows about what agent $i_3$ knows, and so forth up to agent $i_m$, \emph{coincides} exactly with the knowledge of any of the agents of $G$ about what some other agent knows about some third agent, and so forth up to the $m$-th agent. That is, higher-order knowledge involving a group of agents is independent from the order in which we consider said agents (\ie it commutes).
    In other words, Lemma \ref{lem:slide-box} shows us that we can \emph{rearrange} the order of any sequence of boxes in a formula and obtain an equivalent one.

    \begin{lemma}\label{lem:ck-n}
        Let $ G = \{i_1, \dots, i_m\} \subseteq \agentSet $, with $ m \geq 2 $. In the logic C-S5$_n$, for any $\varphi$ and $ \vec{v} \in G^* $ we have that $ \B{i_1} \dots \B{i_m} \varphi \rightarrow \B{v_1} \cdots \B{v_{|\vec{v}|}} \varphi $ is a theorem.
    \end{lemma}

	Lemma~\ref{lem:ck-n} provides the basis to show a first important result related to common knowledge in C-S5$_n$. Namely, we obtain a \emph{finitary non-fixpoint} characterization of common knowledge.

    \begin{thm}\label{th:ck}
        Let $ G = \{i_1, \dots, i_m\} \subseteq \agentSet $, with $ m \geq 2 $. In the logic C-S5$_n$, for any $\varphi$, the formula $ \B{i_1} \dots \B{i_m} \varphi \leftrightarrow \CK{G} \varphi $ is a theorem.
        \begin{proof} 
            ($\Leftarrow$) This follows by definition of common knowledge; ($\Rightarrow$) this immediately follows by Lemma \ref{lem:ck-n}.
        \end{proof}
    \end{thm}

    \begin{corollary}\label{cor:diameter}
        Let $ G = \{i_1, \dots, i_m\} \subseteq \agentSet $, with $ m \geq 2 $. %such that $ i_x \neq i_y $ for each $ x \neq y $. 
        In an C-S5$_n$-model, for any $\vec{v} \in G^*$, we have that if $w R_{v_1} \circ \ldots \circ R_{v_{|\vec{v}|}} w'$, then $w R_{i_1} \circ \dots \circ R_{i_m} w'$.
    \end{corollary}
    
    The statement above directly follows from the contrapositive of the implication in Lemma \ref{lem:ck-n}, under the assumption of minimality of states (w.r.t. bisimulation).
    Intuitively, this states that in a C-S5$_n$-model, given any subset of $m\geq 2$ agents, if a world is reachable in an arbitrary number of steps, then it is also reachable in exactly $m$ steps. Thus, in general, any pair of worlds of a C-S5$_n$-model that are reachable from one another are connected by a path of length \emph{at most $n$}. This property suggests the existence of some boundedness property on the size of C-S5$_n$-states. Indeed, we exploit Corollary \ref{cor:diameter} to prove the following lemma.

    \begin{lemma}\label{lem:bounded-bisim}
        Let $(M, W_d)$ be an C-S5$_n$-state, with $M=(W,R,V)$. For any $w,v \in W$, we have that $ w \bisim_{n+1} v \Leftrightarrow w \bisim v$.
    \end{lemma}

    Lemma \ref{lem:bounded-bisim} shows that, in the logic C-S5$_n$, to verify whether two worlds are bisimilar, we only need to check their neighborhoods up to distance $n+1$. We exploit this intuition, under the assumption of bisimulation minimality, to prove the following lemma.

    \begin{lemma}\label{lem:char-formulae}
        Let $(M,W_d)$ be a bisimulation-contracted C-S5$_n$-state, with $M=(W,R,V)$. Then, $|W|$ is bounded in $n$ and $|\atomSet|$.
    \end{lemma}

    \noindent Having a bound in the number of worlds of a C-S5$_n$-state immediately provides us with the following decidability result.

    \begin{thm}\label{th:dec}
    	For any $n{>}1$, \planex{$\mathcal{T}_{\textnormal{C-S5}}$}{$n$} is \emph{decidable}.
        \begin{proof}
            Let $T \in \mathcal{T}_{\textnormal{C-S5}_n}$ be an epistemic planning task. By Lemma \ref{lem:char-formulae}, it follows that we can perform a breadth-first search on the search space that would only visit a finite number of epistemic states (up to bisimulation contraction) to find a solution for $T$. Thus, we obtain the claim.
        \end{proof}
    \end{thm}

    The following example shows that in C-S5$_n$ we can effectively obtain an answer to the Coordinated Attack Problem, hence showing that common knowledge can not be achieved by the two generals.

    \begin{example}\label{rem:1}
        As shown in Example \ref{ex:task}, the S5$_n$-planning task $T_\textnormal{coord}$ has no solution, but any search algorithm would never terminate. We now consider the C-S5$_n$-planning task $T^A_\textnormal{coord} = (s_0, \actionSet, \varphi_g)$, with $s_0$, $\actionSet$ and $\varphi_g$ defined as in Example \ref{ex:task} (notice that $s_0$ is an C-S5$_n$-state and that send$_{ab}$ and send$_{ba}$ are both C-S5$_n$-actions). We immediately note that the epistemic state $s_1$ of Example \ref{ex:update} is \emph{not} an C-S5$_n$-state. Thus, by Definition \ref{def:solution} and since send$_{ab}$ is the only applicable action in $s_0$, any search algorithm would immediately stop returning the answer ``\emph{no}''. Hence, in the logic C-S5$_n$, one can conclude that it is impossible for the two generals to coordinate an attack in a finite number of steps.
    \end{example}
    