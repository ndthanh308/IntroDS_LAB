\section{Introduction} 
    Multi-agent systems find applications in a wide range of settings where the agents need to be able to reason about both the physical world and the \emph{knowledge} that other agents possess---that is, their \emph{epistemic state}.  
    %
    \emph{Epistemic planning} \cite{journals/jancl/Bolander2011} employs the theoretical framework of Dynamic Epistemic Logic (DEL) \cite{book/springer/vanDitmarsch2007} in the context of automated planning. The resulting formalism is able to represent nondeterminism, partial observability and arbitrary knowledge nesting. That is, agents have the power to reason about higher-order knowledge of other agents with no limitations.
    
    Due to the high expressive power of the DEL framework, the \emph{plan existence problem} (see Definition \ref{def:plan_ex_problem}), that asks whether there exists a plan to achieve a goal of interest, is undecidable in general \cite{journals/jancl/Bolander2011}. 
    As a consequence, in the past decade, DEL has been widely studied to obtain (un)decidability and complexity results for fragments of the planning problem.
    A common approach (see~Section~\ref{sec:rel_works}) consists in syntactically restricting the action theory, for instance by limiting the modal depth of the preconditions and postconditions of actions to a certain bound $d$ \cite{conf/ijcai/Bolander2015,conf/ijcai/Charrier2016,journals/ai/Bolander2020}. Nonetheless, the problem remains undecidable even with $d{=}2$ when only \emph{purely epistemic actions} are allowed, and with $d{=}1$ when factual change is involved. 
    This suggests that such syntactic restrictions are too strong in many practical cases, where reasoning about the knowledge of others is required.
    
    For this reason, in this paper we pursue a different strategy that we call \emph{semantic approach}. Namely, rather than imposing syntactical constraints, the semantic approach focuses on the axioms of the logic for epistemic planning. Specifically, we consider the multi-agent logic for knowledge S5$_n$ (where $n$ denotes the number of agents) and we augment it with an interaction axiom, called the \emph{(knowledge) commutativity} axiom (where, as customary, $\B{i} \varphi$ indicates that agent $i$ \emph{knows} that $\varphi$ holds):

    \begin{equation*}
        \begin{array}{lll}
            \axiom{C} & \B{i} \B{j} \varphi \rightarrow \B{j} \B{i} \varphi & \textnormal{(Commutativity)}
        \end{array}
    \end{equation*}
    
    \noindent This axiom imposes a principle of commutativity in the higher-order knowledge across agents. In the resulting logic, which we call C-S5$_n$, while agents have their own distinct individual knowledge, higher-order levels of perspectives of agents \emph{commute}. This assumption is well suited in cooperative planning domains \cite{journals/csur/Torreno2017}, where it is required that agents act and communicate in an observable way, thus making knowledge of agents accessible to others.
    
    We provide a threefold contribution. First, we show that the epistemic plan existence problem in the resulting framework becomes decidable. We do so by proving that the commutativity axiom ensures that the states in the logic C-S5$_n$ are bounded in size, which entails that the search space of the plan existence problem is finite. In doing so, we show that the logic C-S5$_n$ admits a \emph{finitary non-fixpoint} characterization of common knowledge, which is often regarded as a possible solution to paradoxes involving common knowledge (see \cite{journals/synthese/Paternotte11} for an overview).
    
    Second, we investigate the plan existence problem with different generalized principles of commutativity.
    Indeed, although the commutativity axiom is better fitting for tight-knit groups of agents, it may be less suited for representing more loosely organized groups.
    We define suitable generalizations parametrized by fixed integer constants $b{>}1$ and $1 {<} \ell {\leq} n$. The resulting axioms are the following (where $\pi$ is a permutation of the sequence $\langle i_1, \dots i_\ell\rangle$ of agents, as explained in more detail in Section \ref{sec:general-comm}):
    
    \begin{equation*}
        \begin{array}{lll}
            \axiom{C$^b$} & (\B{i} \B{j})^b \varphi \rightarrow (\B{j} \B{i})^b \varphi & \textnormal{($b$-Comm.)} \\
            \axiom{wC$_\ell$} & \B{i_1} \dots \B{i_\ell}\varphi \rightarrow \B{\pi_{i_1}} \dots \B{\pi_{i_\ell}}\varphi & \textnormal{(Weak comm.)}
        \end{array}
    \end{equation*}

    \noindent Concerning axiom \axiom{wC$_\ell$}, we show that the plan existence problem remains decidable for any $1 {<} \ell {\leq} n$. Relating axiom \axiom{C$^b$}, we show that the plan existence problem remains decidable in the presence of two agents ($n{=}2$), for any $b{>}1$. We also show that for any $n{>}2$ and any $b{>}1$ the problem becomes undecidable.

    Finally, we show that the knowledge (\ie S5$_n$) fragment of the well known planning system $m\mathcal{A}^*$ \cite{journals/corr/Baral2015} and the system by Kominis and Geffner \cite{conf/aips/Kominis2015} are captured by our formalism. Thus, we prove the decidability of a fragment $m\mathcal{A}^*$, which was still an open problem, and of the action formalism in \cite{conf/aips/Kominis2015}, confirming their previous results.

    Since the axioms of the logic for epistemic planning lie at the core of the semantic approach, we consider such axioms to define \emph{meaningful states}. In other words, when a certain principle is introduced to be an \emph{axiom} of the logic, an epistemic state is considered to be \emph{meaningful} if and only if such principle is satisfied. Thus, when planning under a logic $L$, we consider a plan to be meaningful and, in turn, valid only if all the states that it visits satisfy the axioms of $L$.
    At the same time, the semantics of the product update of DEL (see Definition \ref{def:update_em}) does not guarantee that the application of an action of a generic logic $L$ to an epistemic state of the same logic necessarily results in an epistemic state that satisfies all axioms of $L$. A well-known example of this phenomenon is found in the widely studied doxastic logic KD45$_n$ \cite{books/mit/Fagin2004}, where the \emph{consistency axiom} \axiom{D} is not guaranteed to be preserved by the product update. Addressing the problem of preservation of axioms after action updates is not trivial. Indeed, in the literature, considerable effort has been spent in developing different techniques to handle the preservation of axiom \axiom{D} \cite{workshop/nrac/Herzig2005,conf/aaai/Son2015}. Analogously to the case of \axiom{D} in KD45$_n$, in our framework axioms \axiom{C}, \axiom{C$^b$} and \axiom{wC$_\ell$} are not guaranteed to be preserved by the product update. As a result, as explained above, we consider a plan to be valid if it only visits meaningful epistemic states, \ie those satisfying the axioms of the considered logic. Importantly, our decidability results continue to hold even when one adopts more sophisticated revision techniques that handle the preservation problem by accepting and suitably curating non-preserving states. The development of such non-trivial techniques for our logics is independent of the analysis of decidability of the plan existence problem under the same logics, and is left as an important, follow-up work.
    
    The paper is organised as follows.
    In Section~\ref{sec:del}, we recall some preliminaries and define epistemic planning tasks. In Section~\ref{sec:new_logic}, we discuss in more detail the semantic approach, we introduce our new logic for epistemic planning and we discuss the commutativity axiom. In Section~\ref{sec:decidability}, we analyze decidability of epistemic planning under commutativity and its generalizations. In Section~\ref{sec:systems}, we apply our decidability results to existing epistemic planning systems.
    Finally, in Section~\ref{sec:rel_works}, we discuss related work.
