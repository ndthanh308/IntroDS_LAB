\section{Conclusions}
    The paper presents novel decidability results in epistemic planning. The approach adopted in this work deviates from previous ones, where syntactical conditions are imposed to actions. In particular, we pursue a novel semantic approach by introducing a principle of knowledge commutativity that is well suited for cooperative multi-agent planning contexts. In this way, we govern the extent to which agents can reason about the knowledge of their peers. This results in a boundedness property of the size of epistemic states, which in turn guarantees that the search space is finite.
    % As a result, we obtain a decidable formalism that can be directly analyzed in comparison with existing frameworks from the literature, thus bringing together two lines of research that were previously left isolated.
    Starting from this key result, we studied decidability of the plan existence problem under different generalizations of the commutativity axiom, showing both positive and negative results.

    Notably, our decidability results are orthogonal to the problem of preservation of commutativity during planning. In this paper we adopt the baseline strategy of rejecting action sequences that visit invalid states. A natural follow up is then to incorporate more sophisticated techniques to revise/repair such invalid states, in the style of belief revision for KD45$_n$ \cite{workshop/nrac/Herzig2005} and to single out fragments of C-S5$_n$ where preservation is guaranteed by design \cite{conf/aaai/Son2015}.

    In the future, we plan on further investigating our semantic approach by formulating other properties to add to the logic of knowledge. In particular, we are interested into defining new generalizations of the commutativity axiom to obtain broader fragments that maintain decidability of the plan existence problem. Moreover, we are interested into verifying if and how this approach can be suitably recast in a doxastic setting, where knowledge is replaced by \emph{belief}. This is not a trivial task, as the results of this paper do not readily apply to the logic KD45$_n$. We also intend to delve into a fine-grained analysis of the computational complexity of $\mathcal{T}_{\textnormal{C-S5}_n}$ and to compare it with the current results in the literature.

    % Also, we plan on determining sufficient conditions for preserving axiom \axiom{C} during the planning process. This is similar to what is done in \cite{conf/aaai/Son2015} for the preservation of axioms of the logic KD45$_n$. Finally, we intend to investigate the decidability of planning systems \cite{journals/corr/Baral2015,conf/icaps/Fabiano2020,conf/aips/Kominis2015} by comparing them with our framework.
