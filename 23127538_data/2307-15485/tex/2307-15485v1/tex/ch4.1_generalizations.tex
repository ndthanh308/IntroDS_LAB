\subsection{Generalizing the Principle of Commutativity}\label{sec:general-comm}
    Although the commutativity axiom is better fitting for tight-knit groups of agents, it may be less suited for representing more loosely organized groups.
    Thus, having established that adding axiom \axiom{C} to S5$_n$ leads to decidability of the plan existence problem, we investigate two generalized principles of commutativity, namely \emph{$b$-commutativity} and \emph{weak commutativity}. In what follows, we consider such generalizations and we provide (un)decidability results of their corresponding plan existence problems.
        
        \paragraph{$b$-Commutativity}\label{par:b-comm}
        Let $b{>}1$ be a fixed constant. Then, we define the following axiom:
        \begin{equation*}
            \begin{array}{lll}
                \axiom{C$^b$} & (\B{i} \B{j})^b \varphi \rightarrow (\B{j} \B{i})^b \varphi & \textnormal{($b$-Commutativity)}
            \end{array}
        \end{equation*}

        \noindent We call C$^b$-S5$_n$ the logic S5$_n$ augmented with axiom \axiom{C$^b$}. Axiom \axiom{C$^b$} generalizes commutativity by considering an arbitrary fixed amount of repetitions of box operators. Indeed, notice that \axiom{C$^1$} $=$ \axiom{C}.
        Moreover, since every $\Box_i$ is a monotone modality (\ie from $\varphi \rightarrow \psi$ we can infer $\B{i}\varphi \rightarrow \B{i}\psi$), it is easy to see that each axiom \axiom{C$^{b+1}$} leads to a weaker logic than axiom \axiom{C$^b$}, and that every logic C$^b$-S5$_n$ is weaker than C-S5$_n$. 
        %
        One could hope that the plan existence problem remains decidable when replacing axiom \axiom{C} with \axiom{C$^b$}. But this is not true in general. In fact, we prove that it remains decidable for $n{=}2$ and any $b{>}1$ (Theorem \ref{th:dec-b-2}) and that it becomes undecidable for any $n{>}2$ and $b{>}1$ (Theorem \ref{th:undec-b-n}).
        %
        Due to space constraints, we only provide the proof sketches (full proofs are available in the arXiv Appendix).

        \begin{thm}\label{th:dec-b-2}
            For any $b{>}1$, \planex{$\mathcal{T}_{\textnormal{C$^b$-S5}}$}{$2$} is \emph{decidable}.
            
            \begin{proof}
                \emph{(Sketch)} Analogous to the case of the logic C-S5$_n$. Namely, we can prove the correspondent versions of Lemma %ta \ref{lem:slide-2-box} and
                \ref{lem:ck-n}, Theorem \ref{th:ck}, Corollary \ref{cor:diameter} and Lemmata \ref{lem:bounded-bisim} and \ref{lem:char-formulae}. The claim follows by combining these results as in Theorem \ref{th:dec}.
            \end{proof}
        \end{thm}

        \begin{example}
            As the Coordinated Attack Problem involves exactly two agents, for any $b{>}1$, we can define the C$^b$-S5$_2$-planning task $T^{\textnormal{C}^b}_\textnormal{coord} = (s_0, \actionSet, \varphi_g)$, with $s_0$, $\actionSet$ and $\varphi_g$ defined as in Example \ref{rem:1}. Then, as above, we note that the epistemic state $s_{1+2(b-1)}$ of Example \ref{ex:task} is \emph{not} a C$^b$-S5$_2$-state. Thus, by Definition \ref{def:solution} and since send$_{ab}$ is the only applicable action in $s_{2(b-1)}$, a search algorithm would return the answer ``\emph{no}'' in exactly $2(b{-}1)$ steps.
        \end{example}

        \begin{thm}\label{th:undec-b-n}
            For any $n{>}2$, $b{>}1$, \planex{$\mathcal{T}_{\textnormal{C$^b$-S5}}$}{$n$} is \emph{undecidable}.

            \begin{proof}
                \emph{(Sketch)} We adapt the proof in \cite[Section~6]{conf/ijcai/Aucher2013}, of the undecidability of epistemic planning in the logic $S5_n$ ($n > 1$). It is an elegant reduction from the halting problem of Minsky two-counter machines \cite{book/ph/Minsky1967} to the plan existence problem.
                
                We prove our result for the logic C$^2$-S5$_3$ (\ie having $b=2$ and $n=3$). Since C$^2$-S5$_3$-models are also C$^b$-S5$_n$-models for any $n > 3$ and $b > 2$, our results hold for any combination of the values of $n \geq 3$ and $b \geq 2$. Given a two-counter machine $M$, the procedure follows three steps:
                \begin{compactenum}
                    \item We define an encoding for integers and configurations;
                    \item We build a finite set of actions for encoding the computation function $f_M$; and
                    \item We combine the previous steps and we encode the halting problem as an C$^2$-S5$_3$-planning task.
                \end{compactenum}
                %
                Finally, the claim follows by showing that the resulting planning task has a solution iff $M$ halts.
            \end{proof}
        \end{thm}

        These results show that it is not straightforward to generalize knowledge commutativity and to maintain the decidability of the plan existence problem. However, Theorem \ref{th:dec-b-2} suggests that the logic C$^b$-S5$_n$ could result into interesting developments in contexts where only two agents are involved (\eg epistemic games).

        \paragraph{Weak commutativity}\label{par:weak-comm}
        Let $1{<}\ell{\leq} n$ be a fixed constant. Let $\langle i_1, \dots, i_\ell \rangle$ be a sequence of agents with no repetitions, and let $\pi$ be any permutation of this sequence. Then, we define the following axiom (for any such $\pi$):
        \begin{equation*}
            \begin{array}{lll}
                \axiom{wC$_\ell$} & \B{i_1} \dots \B{i_\ell}\varphi \rightarrow \B{\pi_{i_1}} \dots \B{\pi_{i_\ell}}\varphi & \textnormal{(Weak comm.)}
            \end{array}
        \end{equation*}

        \noindent We call wC$_\ell$-S5$_n$ the logic S5$_n$ augmented with axiom \axiom{wC$_\ell$}. Axiom \axiom{wC$_\ell$} generalizes commutativity by extending it to more than two agents, whereas \axiom{C} corresponds to taking $\ell=2$. 
        Indeed, notice that \axiom{wC$_2$} $=$ \axiom{C}. Moreover, since every $\Box_i$ is a monotone modality, it is easy to see that each axiom \axiom{wC$_{\ell+1}$} leads to a weaker logic than axiom \axiom{wC$_\ell$}, and that every logic wC$_\ell$-S5$_n$ is weaker than wC-S5$_n$.

        By considering this form of generalization of axiom \axiom{C}, we are able to provide a decidability result that holds for any $1 < \ell \leq n$. The arguments adopted by the proof are similar to those of Theorem \ref{th:dec}.

        \begin{thm}\label{th:dec-l}
            For any $n{>}1$ and $1 {<} \ell {\leq} n$, \planex{$\mathcal{T}_{\textnormal{wC$_\ell$-S5}}$}{$n$} is \emph{decidable}.
            \begin{proof}
                \emph{(Sketch)} As in the proof of Theorem \ref{th:dec}, we can prove the correspondent versions of Lemmata \ref{lem:slide-box}, \ref{lem:ck-n}, Theorem \ref{th:ck} and Corollary \ref{cor:diameter}. From these results, we obtain that any pair of worlds of a wC$_\ell$-S5$_n$-model that are reachable from one another are connected by a path of length \emph{at most $n$}. Hence, we show that Lemmata \ref{lem:bounded-bisim} and \ref{lem:char-formulae} hold also in the logic wC$_\ell$-S5$_n$ (for any $\ell > 1$). Thus, to obtain the claim, we use Lemmata \ref{lem:bounded-bisim} and \ref{lem:char-formulae} by combining them as in Theorem \ref{th:dec}.
            \end{proof}
        \end{thm}

        To summarize, $b$-commutativity and weak commutativity constitute two generalizations of axiom \axiom{C}. All of the above decidability results are outlined in Table \ref{tab:complexity2}.
