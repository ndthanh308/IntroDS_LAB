\newcommand{\inc}[1]{\textnormal{inc}(#1)}
\newcommand{\jump}[1]{\textnormal{jump}(#1)}
\newcommand{\jzdec}[2]{\textnormal{jzdec}(#1,#2)}
\newcommand{\halt}{\textnormal{halt}}

\subsubsection{Proof of Theorem \ref{th:undec-b-n}}
    To prove Theorem \ref{th:undec-b-n}, we first show some propaedeutical results (Lemmata \ref{lem:index-meta-chain}, \ref{lem:index-operations}, \ref{lem:comp-function}, \ref{lem:halting}).

    In what follows, we consider the case with $\agentSet = \{0,1,2\}$ and $b=2$, \ie we focus on the logic C$^2$-S5$_3$. Since C$^2$-S5$_3$-models are also C$^b$-S5$_n$-models for any $n > 3$ and $b > 2$, our results hold for any combination of the values of $n \geq 3$ and $b \geq 2$. Moreover, we fix $\atomSet = \{p_1,p_2,p_3,r\}$.
    
    We adapt the proof in \cite[Section~6]{conf/ijcai/Aucher2013}, of the undecidability of epistemic planning in the logic $S5_n$ ($n > 1$). It is an elegant reduction from the problem of reachability in Minsky two-counter machines to the problem of epistemic planning. We first, recall the definition.

    \begin{appdefinition}[Two-counter machine]
        A \emph{two-counter machine} $M$ is a finite sequence of instructions $(I_0, \dots, I_T$), where each instruction $I_t$, with $t<T$, is from the set:
        \begin{equation*}
            \{\inc{i}, \jump{j}, \jzdec{i}{j} \mid i = 0,1, j \leq T\},
        \end{equation*}
        %
        and $I_T=\halt$. A \emph{configuration} of $M$ is a triple $(k,l,m) \in \mathbb{N}^3$, where $k$ is the index of the current instruction, and $l$ and $m$ are the current contents of counters 0 and 1, respectively. The \emph{computation function} $f_M : \mathbb{N} \rightarrow \mathbb{N}^3$ of $M$ maps time steps into configurations, ad is given by $f_M(0) = (0,0,0)$ and if $f_M(n) = (k,l,m)$, then:
        \begin{equation*}
            f_M(n{+}1){=}
            \begin{cases}
                (k{+}1,l{+}1,m    ) & \textnormal{if } I_k {=} \inc{0}                                 \\
                (k{+}1,l    ,m{+}1) & \textnormal{if } I_k {=} \inc{1}                                 \\
                (j    ,l    ,m    ) & \textnormal{if } I_k {=} \jump{j}                                \\
                (j    ,l    ,m    ) & \textnormal{if } I_k {=} \jzdec{0}{j} \textnormal{ and } l {=} 0 \\
                (j    ,l    ,m    ) & \textnormal{if } I_k {=} \jzdec{1}{j} \textnormal{ and } m {=} 0 \\
                (k{+}1,l{-}1,m    ) & \textnormal{if } I_k {=} \jzdec{0}{j} \textnormal{ and } l {>} 0 \\
                (k{+}1,l    ,m{-}1) & \textnormal{if } I_k {=} \jzdec{1}{j} \textnormal{ and } m {>} 0 \\
                (k    ,l    ,m    ) & \textnormal{if } I_k {=} \halt
            \end{cases}
        \end{equation*}
        %
        We say that $M$ \emph{halts} if $f_M(n) {=} (T,l,m)$ for some $n,l,m {\in} \mathbb{N}$.
    \end{appdefinition}

    \begin{apptheorem}[\cite{book/ph/Minsky1967}]\label{th:minsky}
        The halting problem for two-counter machines is undecidable.
    \end{apptheorem}

    We follow the approach of \cite{conf/ijcai/Aucher2013} step by step by encoding the halting problem of a Minsky machine $M$ as an epistemic planning task. The procedure follows three steps:
    \begin{enumerate}
        \item We define an encoding for integers and configurations;
        \item We build a finite set of actions for encoding the computation function $f_M$; and
        \item We combine the previous steps and we encode the halting problem as an epistemic planning task.
    \end{enumerate}

    In all figures, reflexive, transitive (and symmetric) edges are implicit. In the models, the worlds are labelled with the name of the world and the propositions true in it. In the event models, the events are labelled with the name of the event and the precondition; there are no postconditions.

    \boldparagraph{Integers and configurations.}
    %
    For each $p \in \atomSet$ and each $n \in \mathbb{N}$, we define an epistemic model \METACHAIN$(p,n)$, represented in Figure~\ref{fig:metachain}, which contains $n+1$ meta-worlds (models themselves, Figure~\ref{fig:metaworld}). Thus, the integer $0$ is represented by the meta-chain made of only the meta-world model of Figure~\ref{fig:metachain-p-0}.
    Finally, for each configuration $(k,l,m) \in \mathbb{N}^3$ of two-counter machines, we define the epistemic model \METASTATE$_{(k,l,m)}$ as in Figure~\ref{fig:meta-state}.

    In \cite{conf/ijcai/Aucher2013}, the meta-worlds that compose a meta-chain are always linked together with the same accessibility relation. As a consequence, a meta-chain can have an arbitrary long series of alternating distinct worlds $u_1 \overset{i}{\rightarrow} u_2 \overset{j}{\rightarrow} u_3 \overset{i}{\rightarrow} u_4 \overset{j}{\rightarrow} u_5 \cdots$ with $i$ and $j$ distinct and all $u_k$ distinct.
    This is not possible in the logic C$^2$-S5$_n$, due to axiom \axiom{C$^2$}.
    % Since our logic contains axiom \axiom{C$^2$}, such models would also have $u_{k'} \overset{i}{\rightarrow} u_{k''}$ and $u_{k'} \overset{j}{\rightarrow} u_{k''}$, for some $k$, and for all $k' < k$ and $k'' > k$. Meta-chains constructed this way would uniquely represent integers, but not up to bisimulation (this is of course inadequate in a reduction to epistemic planning).\todo{controllare da ``since our logic\dots''}
    Thus, we need to `bypass' axiom \axiom{C$^2$} in the meta-chains. To do so, we devised meta-chains so that meta-worlds are alternatingly linked together with two different relations. This difference also forces us to use three agents instead of two (like in \cite{conf/ijcai/Aucher2013}) and, in fact, the plan existence problem in the two agents case is decidable (see Theorem \ref{th:dec-b-2}).% It also makes the gadgets that are necessary for the reduction more complex.

    Notice how in a meta-state, for $i \not = j$, the longest series of alternating distinct worlds $u_1 \overset{i}{\rightarrow} u_2 \overset{j}{\rightarrow} u_3 \overset{i}{\rightarrow} u_4 \overset{j}{\rightarrow} u_5 \cdots$ with all $u_k$ distinct, is bounded, and is 4. Thus, the models are thus vacuously models of C$^2$-S5$_3$.

    % Figure environment removed
    
    % Figure environment removed

    % Figure environment removed

    % Figure environment removed

    %%%%%%%%%%%%%%%%%%%%%%%%%%%%%%%%%%%%%%%%%%%%%%%%%%%%%%%%%%%%%%%%%%%%%%%%%%%%%%%%%%%%%%%%%%%%%%%%%%%%%%%%%%%%%%%%%%%%%%%%%%

    \boldparagraph{Computation function.}
    First, we define path formulae.

    \begin{appdefinition}[Path formulae]
        For all $p \in \atomSet$ and $n \in \mathbb{N}$, we inductively define the formulae $\lambda_0(p), \mu_0(p), \tau_0(p)$ as follows:
        \begin{itemize}
            \item $\lambda_0(p) = p \land \B{0} \lnot r \land \B{1} \lnot r$
            \item $\mu_0(p) = p \land \D{2} \lambda_0(p) \land \neg \lambda_0(p)$
            \item $\tau_0(p) = p \land r \land (\D{0} \mu_0(p) \lor \D{1} \mu_0(p))$

            \item $\lambda_{n+1}(p) = p \land \lnot r \land \lnot \mu_n(p) \land (\Diamond_0 \mu_n(p) \lor \Diamond_1 \mu_n(p))$
            \item $\mu_{n+1}(p) = p \land \D{2} \lambda_{n+1}(p) \land \neg \lambda_{n+1}(p)$
            \item $\tau_{n+1}(p) = p \land r \land (\D{0} \mu_{n+1}(p) \lor \D{1} \mu_{n+1}(p))$
            %
            % \item $\gamma_0(p) = p \land s \land \Diamond_1 \mu_0(p) \land \lnot \mu_0(p)$
            % \item $\gamma_{n+1}(p) = p \land s \land \Diamond_1 \mu_{n+1}(p) \land \lnot \mu_{n+1}(p)$
        \end{itemize}
    \end{appdefinition}

    \begin{applemma}\label{lem:index-meta-chain}
        For all $p {\in} \atomSet$, $n {\in} \mathbb{N}$, $0 \leq i \leq n$, $1 \leq j \leq 3n{+}3$:
        \begin{equation*}
            \begin{array}{lll}
                (\METACHAIN(p,n), w_j) \models \lambda_i(p) & {\Leftrightarrow} & j {=} 3(n{-}i) {+} 3 \\
                (\METACHAIN(p,n), w_j) \models \mu_i(p)     & {\Leftrightarrow} & j {=} 3(n{-}i) {+} 2 \\
                (\METACHAIN(p,n), w_j) \models \tau_i(p)    & {\Leftrightarrow} & j {=} 3(n{-}i) {+} 1
            \end{array}
        \end{equation*}
        % \begin{enumerate}
        %     \item $\METACHAIN(p,n), w_j \models \lambda_i(p)$ iff $j = 3(n-i) + 3$
        %     \item $\METACHAIN(p,n), w_j \models \mu_i(p)$     iff $j = 3(n-i) + 2$
        %     \item $\METACHAIN(p,n), w_j \models \tau_i(p)$    iff $j = 3(n-i) + 1$
        % \end{enumerate}
    \end{applemma}
    That is, the path formulas allow one to uniquely identify worlds in a meta-chain.
    In the $(i + 1)th$ to last meta-world in META-CHAIN$(p,n)$, 
    $\lambda_i(p)$ holds in the bottom world,
    $\mu_i(p)$ in the top-right world, 
    $\tau_i(p)$ in the top-left world.
    Figure~\ref{fig:metachain-p-0} illustrates the base cases.

    The instructions of a two-counter machine can be decomposed in simple operations on integers: \emph{increment}, \emph{decrement} and \emph{replacement}. We encode each operation with an event model, represented on Figures \ref{fig:metainc}, \ref{fig:metadec} and \ref{fig:metarepl}, respectively. Due to the structure of meta-chains, that comprise alternating accessibility relations, we need to define two different event models for increment. Namely, \METAINC$_0(p)$ is used to increment odd numbers and \METAINC$_1(p)$ handles even numbers. Thus, given an integer $n$, to increment \METACHAIN$(p,n)$, we use \METAINC$_i(p)$, where $i = 1-(n~mod~2)$.

    The following Lemma makes sure that the operations on integers are correctly captured by the product update of meta-chains with the event models for increment, decrement and replacement.

    \begin{applemma}\label{lem:index-operations}
        For all $m,n \in \mathbb{N}$ and for all $p \in \atomSet$, we have:
        \begin{enumerate}
            \item\label{item:inc}  $\METACHAIN(p, n) \otimes \METAINC_i(p) =\\ \METACHAIN(p, n+1)$, where $i = 1-(n~mod~2)$;
            \item\label{item:dec}  If $n > 0$, $\METACHAIN(p, n) \otimes \METADEC(p) =\\ \METACHAIN(p, n - 1)$;
            \item\label{item:repl} $\METACHAIN(p, n) \otimes \METAREPL(p, n, m) =\\ \METACHAIN(p, m)$.
        \end{enumerate}

        \begin{proof}
            First, we consider item \ref{item:inc}, \ie event model of Figure \ref{fig:metainc}. Let $n$ be even, \ie $i=1$ (the case with $i=0$ is identical). The top event of Figure \ref{fig:metainc} is paired with all worlds in $\METACHAIN(p, n)$, except for the one where $\lambda_0(p)$ holds. From Lemma \ref{lem:index-meta-chain}, this world in unique and it is the bottom world of $\METACHAIN(p, n)$. Thus, after the product of $\METACHAIN(p, n)$ with the top event, we obtain a copy of the chain, except for its bottom world. Such world is paired with the second-to-top event of Figure \ref{fig:metainc}. At this point, we obtain an exact copy of $\METACHAIN(p, n)$. Finally, the last three events of Figure \ref{fig:metainc} create an additional meta-world. Since $n$ is even, the last relation linking meta-worlds in the chain is that of agent $0$. Then, the event model $\METAINC_1(p)$ links the additional meta-world to the bottom of chain with the accessibility relation of agent $1$. Thus, we obtain $\METACHAIN(p, n+1)$.

            We now focus on item \ref{item:dec}, \ie event model of Figure \ref{fig:metadec}. By Lemma \ref{lem:index-meta-chain}, its only event is paired with all worlds of $\METACHAIN(p, n)$, except for those in the bottom meta-world. Since $n>0$, we obtain $\METACHAIN(p, n-1)$.

            Finally, we consider item \ref{item:repl}, \ie event model of Figure \ref{fig:metarepl}. By Lemma \ref{lem:index-meta-chain}, the $2 \cdot (m+1)$ event models on the right hand side of Figure \ref{fig:metarepl} all pair with the top-right world of the top meta-world of $\METACHAIN(p, n)$ and the $m+1$ events on the left hand side pair with the top left world of the same meta-world. Thus, we create $m+1$ copies of the top meta-world of $\METACHAIN(p, n)$. Then, these copies are alternatingly linked together with the accessibility relations of agents $0$ and $1$. Thus, we obtain $\METACHAIN(p, m)$.
        \end{proof}
    \end{applemma}

    % Figure environment removed 

    % Figure environment removed

    % Figure environment removed

    For all $k \in \mathbb{N}$, we define $\phi_k = \D{0}\mu_k(p_1)$. By Lemma \ref{lem:index-meta-chain} and the definition of \METASTATE$_{(k,l,m)}$, we immediately obtain that for all $k,l,m,k' \in \mathbb{N}$ the following holds:
    \begin{equation}\label{eq:meta-s}
        \METASTATE_{(k,l,m)} \models \phi_{k'} \textnormal{ iff } k' = k.
    \end{equation}

    Let now $M = (I_0, \dots, I_T)$ be a two-counter machine. For all $k<T$ and $l,m \in \mathbb{N}$, we define an epistemic action $a_M(k,l,m)$ as in Figures \ref{fig:action-inc}-\ref{fig:action-jzdec}.

    % Figure environment removed

    % Figure environment removed

    % Figure environment removed

    % Figure environment removed

    We now define a notion of \emph{equivalence} between configurations. Two configurations $(k,l,m), (k',l',m') \in \mathbb{N}^3$ are equivalent, denoted by $(k,l,m) \approx (k',l',m')$ if the following holds:
    \begin{equation*}
        k=k' \textnormal{ and }
        \begin{cases}
            l=0 \leftrightarrow l'=0 & \textnormal{if } I_k = \jzdec{0}{j} \\
            m=0 \leftrightarrow m'=0 & \textnormal{if } I_k = \jzdec{1}{j}
        \end{cases}.
    \end{equation*}
    %
    Notice that if $(k,l,m) \approx (k',l',m')$, then $a_M(k,l,m) = a_M(k',l',m')$. Thus, the following set is \emph{finite}:
    \begin{equation*}
        \mathcal{F}_M := \{a_M(k,l,m) \mid 0 \leq k < T \textnormal{ and } l,m \in \mathbb{N}\}.
    \end{equation*}

    The following Lemma shows that $\mathcal{F}_M$ correctly encodes the computation function of the two-counter machine $M$.
    \begin{applemma}\label{lem:comp-function}
        Let $M = (I_0, \dots, I_T)$ be a two-counter machine, $l,m,n \in \mathbb{N}$ and $k<T$. Then, the following holds:
        \begin{enumerate}
            \item $a_M(k,l,m)$ is applicable in $\METASTATE_{f_M(n)}$ iff $(k,l,m) \approx f_M(n)$;
            \item $\METASTATE_{f_M(n)} \otimes a_M(f_M(n)) = \METASTATE_{f_M(n+1)}$.
        \end{enumerate}

        \begin{proof}
            Let $f_M(n) = (k',l',m')$. The first item by case of $I_k$.
            \begin{itemize}
                \item $I_k = \inc{0}$, $\inc{1}$, or $\jump{j}$: $a_M(k,l,m)$ is an action of the form $(\E, \{e\})$ with $pre(e) = \D{0}\mu_k(p_1) = \phi_k$. Thus, by equation \ref{eq:meta-s}, we have: $a_M(k,l,m)$ is applicable in $\METASTATE_{f_M(n)} \Leftrightarrow \METASTATE_{(k',l',m')} \models \phi_k \Leftrightarrow k=k' \Leftrightarrow (k,l,m) \approx (k',l',m')$.
                \item $I_k = \jzdec{0}{j}$ and $l = 0$: $a_M(k,l,m)$ is an action of the form $(\E, \{e\})$ with $pre(e) = \D{0} \mu_k(p_1) \land \D{0} \mu_0(p_2) = \phi_k \land \D{0} \mu_0(p_2)$. Thus, by equation \ref{eq:meta-s}, we have: $a_M(k,l,m)$ is applicable in $\METASTATE_{f_M(n)} \Leftrightarrow \METASTATE_{(k',l',m')} \models \phi_k \land \D{0} \mu_0(p_2)$ $\Leftrightarrow k=k' \land l'=0 \Leftrightarrow (k,l,m) \approx (k',l',m')$.
                \item $I_k = \jzdec{1}{j}$ and $m = 0$: analogous to the previous case.
                \item $I_k = \jzdec{0}{j}$ and $l > 0$: $a_M(k,l,m)$ is an action of the form $(\E, \{e\})$ with $pre(e) = \D{0} \mu_k(p_1) \land \lnot\D{0} \mu_0(p_2) = \phi_k \land \lnot\D{0} \mu_0(p_2)$. Thus, by equation \ref{eq:meta-s}, we have: $a_M(k,l,m)$ is applicable in $\METASTATE_{f_M(n)} \Leftrightarrow \METASTATE_{(k',l',m')} \models \phi_k \land \lnot\D{0} \mu_0(p_2)$ $\Leftrightarrow k=k' \land l'\not=0 \Leftrightarrow (k,l,m) \approx (k',l',m')$.
                \item $I_k = \jzdec{1}{j}$ and $m > 0$: analogous to the previous case.
            \end{itemize}

            The second item is by case of $I_{k'}$:
            \begin{itemize}
                \item $I_{k'} = \inc{0}$: $a_M(k',l',m')$ is the action of Figure \ref{fig:action-inc}. Thus, by Lemma \ref{lem:index-operations}, we have that: $\METASTATE_{f_M(n)} \otimes a_M(f_M(n)) =$ $\METASTATE_{(k',l',m')} \otimes a_M(k',l',m') = \METASTATE_{(k'+1,l'+1,m')} = \METASTATE_{f_M(n+1)}$.
                \item $I_{k'} = \inc{1}$: analogous to the previous case.
                \item $I_{k'} = \jump{j}$: $a_M(k',l',m')$ is the action of Figure \ref{fig:action-jump}. Thus, by Lemma \ref{lem:index-operations}, we have that: $\METASTATE_{f_M(n)} \otimes a_M(f_M(n)) = \METASTATE_{(k',l',m')} \otimes a_M(k',l',m') = \METASTATE_{(j,l',m')} = \METASTATE_{f_M(n+1)}$.
                \item $I_{k'} = \jzdec{0}{j}$ and $l = 0$: $a_M(k',l',m')$ is the action of Figure \ref{fig:action-jzdec-z}. Thus, by Lemma \ref{lem:index-operations}, we have that: $\METASTATE_{f_M(n)} \otimes a_M(f_M(n)) = \METASTATE_{(k',l',m')} \otimes a_M(k',l',m') = \METASTATE_{(j,l',m')} = \METASTATE_{f_M(n+1)}$.
                \item $I_{k'} = \jzdec{1}{j}$ and $m = 0$: analogous to the previous case.
                \item $I_{k'} = \jzdec{0}{j}$ and $l > 0$: $a_M(k',l',m')$ is the action of Figure \ref{fig:action-jzdec}. Thus, by Lemma \ref{lem:index-operations}, we have that: $\METASTATE_{f_M(n)} \otimes a_M(f_M(n)) =$ $\METASTATE_{(k',l',m')} \otimes a_M(k',l',m') = \METASTATE_{(k'+1,l'-1,m')} = \METASTATE_{f_M(n+1)}$.
                \item $I_{k'} = \jzdec{1}{j}$ and $m > 0$: analogous to the previous case.
            \end{itemize}
        \end{proof}
    \end{applemma}

    % Notice in item~\ref{item:metchain-metainc} the equality to the point-generated sub-model of the product update. This is because the product update of a meta-chain with \METAINC\ will always yield a model with two disconnected components. We are only interested with the largest connected sub-model containing the designated world $(w_2, e_1)$.\footnote{If the second- and first-from-last meta-worlds in \METACHAIN$(p,n)$ are linked together with a relation $2$ (resp.\ $1$), we must append a meta-world with a relation $1$ (resp.\ $2$). A disconnected meta-chain of two meta-worlds linked together with a relation $2$ (resp.\ $1$) will also result from the product update.}
    % \todo{Missing: proof of finiteness of action set.}

    \boldparagraph{Halting problem.}
    From Lemma \ref{lem:comp-function}, we obtain the following result:

    \begin{applemma}\label{lem:halting}
        Let $M {=} (I_0, \ldots, I_T)$ be a two-counter machine. We define the epistemic planning task $T_M = (\METASTATE_{(0,0,0)}, \mathcal{F}_M,$ $\phi_T)$. Then, $T_M$ has a solution iff $M$ halts.
        % \begin{itemize}
        %     \item $s_0$ is $\METASTATE_{(0,0,0)}$;
        %     \item $\mathcal{A}$ is an adequate finite set of event models as in Figures~\ref{fig:action-inc}, \ref{fig:action-jump}, \ref{fig:action-jzdec-z}, and \ref{fig:action-jzdec} that represents each instruction $I_i$, with $i < T$:
        %     \begin{itemize}
        %         \item if $I_i = jzdec(0,j)$, and counter 1 is 0, or if $I_i = jzdec(1,j)$ and counter 2 is 0, use action type of Figure~\ref{fig:action-jzdec-z};
        %         \item if $I_i = jzdec(0,j)$, and counter 1 is greater than 0, or if $I_i = jzdec(1,j)$ and counter 2 is greater than 0, use action type of Figure~\ref{fig:action-jzdec};
        %         \item if $I_i = inc(0)$ or if $I_i = inc(1)$ use action type of Figure~\ref{fig:action-inc};
        %         \item if $I_i = jump(j)$ use action type of Figure~\ref{fig:action-jump}.
        %     \end{itemize}
        %     \item $\varphi_g = \Diamond_0 \mu_T(p_1)$.
        % \end{itemize}

        % Then, $T_M$ has a solution iff $M$ halts.
    \end{applemma}

    Thus, from Lemma \ref{lem:halting} and Theorem \ref{th:minsky} and from the fact that C$^2$-S5$_3$-models are also C$^b$-S5$_n$-models for any $n > 3$ and $b > 2$, we obtain:
    
    \settheoremcountertoref{th:undec-b-n}
    \begin{thm}\label{th:undec-b-n-proof}
        For any $n{>}2$ and $b{>}1$, \planex{$\mathcal{T}_{\textnormal{C$^b$-S5}}$}{$n$} is \emph{undecidable}.
    \end{thm}
    
    We summarize the results of this section in Table~\ref{tab:summary}.

    \begin{table}[h]
        \centering
        \begin{tabular}{l|cccc}
            \toprule
            $\downarrow n$ / $b \rightarrow$ & 1 & 2 & 3 & \ldots \\
            \midrule
            1 & D & D & D & D\\
            2 & D & D & D & D \\
            3 & D & UD & UD & UD \\
            4 & D & UD & UD & UD \\
            \ldots & D & UD & UD & UD \\
            \bottomrule
        \end{tabular}
    \caption{\label{tab:summary} Summary of complexity results for the \planex{$\mathcal{T}_{\textnormal{C$^b$-S5}}$}{$n$}. D: decidable; UD: undecidable.}
    \end{table}
