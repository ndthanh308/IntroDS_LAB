\subsubsection{Proof of Theorem \ref{th:dec-b-2}}
    To prove Theorem \ref{th:dec-b-2}, we first show some propaedeutical results (Lemma \ref{lem:ck-2-b}, Theorem \ref{th:ck-b}, Corollary \ref{cor:diameter-b} and Lemmata \ref{lem:bounded-bisim-b}, \ref{lem:char-formulae-b}).
    %
    Notice that we follow step by step the proof of Theorem \ref{th:dec} and we give the corresponding results in the logic C$^b$-S5$_2$, for any $b>1$. Since we consider the specific case involving 2 agents, we fix $\agentSet = \{0,1\}$.

    The following is the corresponding version of Lemma \ref{lem:ck-n} of Section \ref{sec:decidability}.

    \begin{applemma}\label{lem:ck-2-b}
        Let $i,j \in \agentSet$ with $i \not= j$, let $ \vec{v} \in \agentSet^* $ and let $\varphi$ be any formula. Then, for any $b{>}1$, in the logic C$^b$-S5$_2$ the formula $ (\B{i} \B{j})^b \varphi \rightarrow \B{v_1} \cdots \B{v_{|\vec{v}|}} \varphi $ is a theorem.

        \begin{proof}
            The proof is by induction on $|\vec{v}|$.
            For the base case ($|\vec{v}|=0$) we have that the formulae $(\B{i}\B{j})^a \varphi \rightarrow \B{j}(\B{i}\B{j})^{a-1} \varphi$ and $\B{j}(\B{i}\B{j})^{a-1} \varphi \rightarrow (\B{i}\B{j})^{a-1} \varphi$ (for each $1 \leq a \leq b$) are instances of \axiom{T}. Together with propositional reasoning, we get that $ (\B{i} \B{j})^b \varphi \rightarrow \varphi $ is a theorem.

            Let now $|\vec{v}| = \lambda$ and suppose by induction that $(\B{i} \B{j})^b \varphi \rightarrow \B{v_1} \cdots \B{v_{\lambda}} \varphi$ is a theorem (for any formula $\varphi$). We now show that, for each $ k \in \agentSet $, the formula $(\B{i} \B{j})^b \varphi \rightarrow \B{v_1} \cdots \B{v_{\lambda}} \B{k} \varphi$ is also a theorem. By inductive hypothesis, substituting $\varphi$ with $\B{k} \varphi$, the following is a theorem:
            \begin{equation*}
                (\B{i} \B{j})^b \B{k} \varphi \rightarrow \B{v_1} \cdots \B{v_{\lambda}} \B{k} \varphi
            \end{equation*}

            \noindent There are now two cases: either $k=j$, or $k=i$. In the former case, we use Equation \ref{eq:box-absoption} as in the proof of Lemma \ref{lem:ck-n-proof} to rewrite the antecedent of the above implication as follows: $ (\B{i} \B{j})^b \B{j} \varphi \equiv (\B{i} \B{j})^{b-1} \B{i} \B{j} \B{j} \varphi \equiv (\B{i} \B{j})^{b-1} \B{i} \B{j} \varphi \equiv (\B{i} \B{j})^b \varphi $.

            In the latter case, we notice that in the logic C$^b$-S5$_2$, for any formula $\varphi$, the following formula is a theorem:
            \begin{equation}\label{eq:comm-2-b}
                (\B{i} \B{j})^b \varphi \leftrightarrow (\B{j} \B{i})^b \varphi
            \end{equation}

            \noindent This immediately follows from axiom \axiom{C$^b$}.

            By using Equation \ref{eq:comm-2-b} we obtain: $ (\B{i} \B{j})^b \B{i} \varphi \equiv (\B{j} \B{i})^b \B{i} \varphi $. By repeating the manipulation of the former case and subsequently reapplying Equation \ref{eq:comm-2-b}, we get: $(\B{j} \B{i})^b \B{i} \varphi \equiv (\B{j} \B{i})^b \varphi \equiv (\B{i} \B{j})^b \varphi$.
            %
            Thus, for each $ k \in \agentSet $, we obtain that the following is a theorem:
            \begin{equation*}
                (\B{i} \B{j})^b \varphi \rightarrow \B{v_1} \cdots \B{v_{\lambda}} \B{k} \varphi.
            \end{equation*}
            %
            This is the required result.
        \end{proof}
    \end{applemma}
    
    The following is the corresponding version of Theorem \ref{th:ck} of Section \ref{sec:decidability}.

    \begin{apptheorem}\label{th:ck-b}
        Let $i,j \in \agentSet$ with $i \not= j$ and let $\varphi$ be any formula. Then, for any $b{>}1$, in the logic C$^b$-S5$_2$ the formula $ (\B{i} \B{j})^b \varphi \leftrightarrow \CK{\agentSet} \varphi $ is a theorem.
        \begin{proof} 
            ($\Leftarrow$) This follows by definition of common knowledge; ($\Rightarrow$) this immediately follows by Lemma \ref{lem:ck-2-b}.
        \end{proof}
    \end{apptheorem}

    The following is the corresponding version of Corollary \ref{cor:diameter} of Section \ref{sec:decidability}.

    \begin{appcorollary}\label{cor:diameter-b}
        Let $i,j \in \agentSet$ with $i \not= j$, let $ \vec{v} \in \agentSet^* $ and let $\varphi$ be any formula. Then, for any $b{>}1$, in an C$^b$-S5$_2$-model we have that if $w R_{v_1} \circ \ldots \circ R_{v_{|\vec{v}|}} w'$, then $w (R_{i} \circ R_{j})^b w'$.
    \end{appcorollary}

    The statement above directly follows from the contrapositive of the implication in Lemma \ref{lem:ck-2-b}, under the assumption of minimality of models (w.r.t. bisimulation).
    Intuitively, this states that if a world of a C$^b$-S5$_2$-model is reachable in an arbitrary number of steps, then it is also reachable in exactly $2b$ steps.

    The following is the corresponding version of Lemma \ref{lem:bounded-bisim} of Section \ref{sec:decidability}.

    \begin{applemma}\label{lem:bounded-bisim-b}
        Let $(M, W_d)$ be an C$^b$-S5$_2$-state, with $M=(W,R,V)$. For any $w,v \in W$, we have that $ w \bisim_{2b+1} v \iff w \bisim v$.

        \begin{proof}
            The proof is identical to that of Lemma \ref{lem:bounded-bisim}, by using Corollary \ref{cor:diameter-b} instead of Corollary \ref{cor:diameter}.
        \end{proof}
    \end{applemma}

    The following is the corresponding version of Lemma \ref{lem:char-formulae} of Section \ref{sec:decidability}.

    \begin{applemma}\label{lem:char-formulae-b}
        Let $(M,W_d)$ be an C$^b$-S5$_2$-state, with $M=(W,R,V)$. Then, $|W|$ is bounded in $2b$ and $|\atomSet|$.

        \begin{proof}
            The proof is identical to that of Lemma \ref{lem:char-formulae}.
        \end{proof}
    \end{applemma}

    \settheoremcountertoref{th:dec-b-2}
    \begin{thm}\label{th:dec-b-2-proof}
        For any $b{>}1$, \planex{$\mathcal{T}_{\textnormal{C$^b$-S5}}$}{$2$} is \emph{decidable}.

        \begin{proof}
            Let $T \in \mathcal{T}_{\textnormal{C}^b-\textnormal{S5}_2}$ be an epistemic planning task. By Lemma \ref{lem:char-formulae-b}, it follows that we can perform a breadth-first search on the search space that would only visit a finite number of epistemic states (up to bisimulation contraction) to find a solution for $T$. Thus, we obtain the claim.
        \end{proof}
    \end{thm}
    