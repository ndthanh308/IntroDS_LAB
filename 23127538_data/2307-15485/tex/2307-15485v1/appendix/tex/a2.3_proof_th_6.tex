\subsubsection{Proof of Theorem \ref{th:dec-l}}
    To prove Theorem \ref{th:dec-l}, we first show some propaedeutical results (Lemmata \ref{lem:slide-box-l}, \ref{lem:ck-n-l}, Theorem \ref{th:ck-l} and Corollary \ref{cor:diameter-l}).

    The following is the corresponding version of Lemma \ref{lem:slide-box} of Section \ref{sec:decidability}.
    \begin{applemma}\label{lem:slide-box-l}
        Let $ G \subseteq \agentSet $, with $ |G| \geq \ell $ and let $ \vec{v} \in G^* $ ($ |\vec{v}| = \lambda \geq \ell $) such that each agent in $G$ appears in $\vec{v}$ at least once. Let $ \rho $ and $ \tau $ be two permutations of elements of $ \vec{v} $. Then, for any $\varphi$, in the logic wC$_\ell$-S5$_n$ the following is a theorem:
        \begin{equation*}
            \B{\rho_1} \dots \B{\rho_\lambda} \varphi \leftrightarrow \B{\tau_1} \dots \B{\tau_\lambda} \varphi
        \end{equation*}

        \begin{proof}
            First, we notice that in the logic wC$_\ell$-S5$_n$, for any formula $\varphi$, the following formula is a theorem (recall that $\langle i_1, \dots, i_\ell \rangle$ is a sequence of agents with no repetitions, and that $\pi$ is a permutation of this sequence):
            \begin{equation}\label{eq:comm-l}
                \B{i_1} \dots \B{i_\ell}\varphi \leftrightarrow \B{\pi_{i_1}} \dots \B{\pi_{i_\ell}}\varphi
            \end{equation}

            \noindent This immediately follows from axiom \axiom{wC$_\ell$}.

            Second, by construction, we have that for each $ \rho_i $ there exists $ \tau_{k_i} $ such that $ \tau_{k_i} = \rho_i $. Consider $ \tau_{k_1} = \rho_1 $. Then, by iterating Equation \ref{eq:comm-l}, we obtain:
            \begin{align*}
                                &~ \B{\tau_1} \dots \overbrace{\left(\B{\tau_{{k_1}-1}} \B{\tau_{k_1}} \dots \B{\tau_{{k_1}+\ell-1}}\right)}^\ell \dots \B{\tau_\lambda} \varphi \\
                \leftrightarrow &~ \B{\tau_1} \dots            \left(\B{\tau_{k_1}} \B{\tau_{{k_1}-1}} \dots \B{\tau_{{k_1}+\ell-1}}\right)       \dots \B{\tau_\lambda} \varphi \\
                \dots           &~                                                                                                \\
                \leftrightarrow &~ \overbrace{\left(\B{\tau_1} \B{\tau_{k_1}} \dots \B{\tau_j}\right)}^\ell \dots \B{\tau_{{k_1}-1}} \dots \B{\tau_{{k_1}+\ell-1}} \dots \B{\tau_\lambda} \varphi \\
                \leftrightarrow &~ \B{\tau_{k_1}} \B{\tau_1} \dots \B{\tau_j} \dots \B{\tau_{{k_1}-1}} \dots \B{\tau_{{k_1}+\ell-1}} \dots \B{\tau_\lambda} \varphi
            \end{align*}
            %
            By repeating this manipulation for $ \rho_2, \dots \rho_m $, we obtain the conclusion.
        \end{proof}
    \end{applemma}

    The following is the corresponding version of Lemma \ref{lem:ck-n} of Section \ref{sec:decidability}.
    \begin{applemma}\label{lem:ck-n-l}
        Let $ G = \{i_1, \dots, i_m\} \subseteq \agentSet $, with $ m \geq \ell $. In the logic wC$_\ell$-S5$_n$, for any $\varphi$ and $ \vec{v} \in G^* $ we have that $ \B{i_1} \dots \B{i_m} \varphi \rightarrow \B{v_1} \cdots \B{v_{|\vec{v}|}} \varphi $ is a theorem.

        \begin{proof}
            The proof is by induction on $|\vec{v}|$.
            For the base case ($|\vec{v}|=0$) we have that the formulae $ \B{i_h} \B{i_{h+1}} \dots \B{i_m} \varphi \rightarrow \B{i_{h+1}} \dots \B{i_m} \varphi $ ($ 1 \leq h < m $) and $ \B{i_m} \varphi \rightarrow \varphi $ are instances of \axiom{T}. Together with propositional reasoning, we get that $ \B{i_1} \dots \B{i_m} \varphi \rightarrow \varphi $ is a theorem.

            Let now $|\vec{v}| = \lambda$ and suppose, by inductive hypothesis, that $\B{i_1} \dots \B{i_m} \varphi \rightarrow \B{v_1} \cdots \B{v_{\lambda}} \varphi$ is a theorem (for any formula $\varphi$). We now show that, for each $ j \in G $, the formula $\B{i_1} \dots \B{i_m} \varphi \rightarrow \B{v_1} \cdots \B{v_{\lambda}} \B{j} \varphi$ is also a theorem. By inductive hypothesis, substituting $\varphi$ with $\B{j} \varphi$, the following is a theorem:
            \begin{equation*}
                \B{i_1} \dots \B{i_m} \B{j} \varphi \rightarrow \B{v_1} \cdots \B{v_{\lambda}} \B{j} \varphi.
            \end{equation*}

            \noindent Let $\rho$ be any permutation of $G = \{i_1, \dots, i_m\}$, such that $\rho_m = j$. By Lemma B.8, we can now rewrite the formula $\Box_{i_1} \dots \Box_{i_m}\Box_{j}\varphi$ as $\Box_{\rho_1} \dots \Box_{\rho_m}\Box_{j}\varphi$, which is $\Box_{\rho_1} \dots \Box_{\rho_m}\Box_{\rho_m}\varphi$. Then, we use Equation 2 as in the proof of Lemma 2 to rewrite the above formula as $\Box_{\rho_1} \dots \Box_{\rho_m}\varphi$. By using Lemma B.8, we can rewrite this formula as $\Box_{i_1} \dots \Box_{i_m}\varphi$.
            % Since $ j {\in} G $, there exists $ h {\in} \{1, \dots m\} $ such that $ j = i_h $.
            % From this and Lemma \ref{lem:slide-box-l}, we can rewrite the antecedent of the above implication, $ \B{i_1} \dots \B{i_m} \B{j} \varphi $, as $ \B{i_1} \dots \B{j} \B{j} \dots \B{i_m} \varphi $. Then, we use Equation \ref{eq:box-absoption} as in the proof of Lemma \ref{lem:ck-n-proof} to rewrite the above formula as $ \B{i_1} \dots \B{j} \dots \B{i_m} \varphi $, which is simply $ \B{i_1} \dots \B{i_m} \varphi $.
            %
            Finally, we obtain that the following is a theorem:
            \begin{equation*}
                \B{i_1} \dots \B{i_m} \varphi \rightarrow \B{v_1} \cdots \B{v_{\lambda}} \B{j} \varphi.
            \end{equation*}
            %
            This is the required result.
        \end{proof}
    \end{applemma}

    The following is the corresponding version of Theorem \ref{th:ck} of Section \ref{sec:decidability}.
    \begin{apptheorem}\label{th:ck-l}
        Let $ G = \{i_1, \dots, i_m\} \subseteq \agentSet $, with $ m \geq \ell $. In the logic wC$_\ell$-S5$_n$, for any $\varphi$, the formula $ \B{i_1} \dots \B{i_m} \varphi \leftrightarrow \CK{G} \varphi $ is a theorem.
        \begin{proof} 
            ($\Leftarrow$) This follows by definition of common knowledge; ($\Rightarrow$) this immediately follows by Lemma \ref{lem:ck-n-l}.
        \end{proof}
    \end{apptheorem}

    The following is the corresponding version of Corollary \ref{cor:diameter} of Section \ref{sec:decidability}.
    \begin{appcorollary}\label{cor:diameter-l}
        Let $ G {=} \{i_1, \dots, i_m\} \subseteq \agentSet $, with $ m \geq \ell $. %such that $ i_x \neq i_y $ for each $ x \neq y $. 
        In an wC$_\ell$-S5$_n$-model, for any $\vec{v} \in G^*$, we have that if $w R_{v_1} \circ \ldots \circ R_{v_{|\vec{v}|}} w'$, then $w R_{i_1} \circ \dots \circ R_{i_m} w'$.
    \end{appcorollary}

    The statement above directly follows from the contrapositive of the implication in Lemma \ref{lem:ck-n-l}, under the assumption of minimality of states (w.r.t. bisimulation).
    Intuitively, this states that in a wC$_\ell$-S5$_n$-model, given any subset of $m \geq \ell$ agents, if a world is reachable in an arbitrary number of steps, then it is also reachable in exactly $m$ steps. Thus, in general, any pair of worlds of a wC$_\ell$-S5$_n$-model that are reachable one another are connected by a path of length \emph{at most $n$}.

    \settheoremcountertoref{th:dec-l}
    \begin{thm}\label{th:dec-l-proof}
        For any $n{>}1$ and $1 {<} \ell {\leq} n$, \planex{$\mathcal{T}_{\textnormal{wC$_\ell$-S5}}$}{$n$} is \emph{decidable}.
        \begin{proof}
            As a result of Corollary \ref{cor:diameter-l}, Lemmata \ref{lem:bounded-bisim} and \ref{lem:char-formulae} hold also in the logic wC$_\ell$-S5$_n$ (for any $\ell > 1$). Thus, as in the proof of Theorem \ref{th:dec}, let $T \in \mathcal{T}_{\textnormal{wC$_\ell$-S5}_n}$ be an epistemic planning task (for any such $\ell$). By Lemma \ref{lem:char-formulae}, it follows that we can perform a breadth-first search on the search space that would only visit a finite number of epistemic states (up to bisimulation contraction) to find a solution for $T$. Thus, we obtain the claim.
        \end{proof}
    \end{thm}
