\section{\nameref*{sec:decidability}}

    \setlemmacountertoref{lem:slide-box}
    \begin{lemma}\label{lem:slide-box-proof}
        Let $ G = \{i_1, \dots, i_m\} \subseteq \agentSet $, with $ m \geq 2 $ and let $ \vec{v} \in G^* $ ($ |\vec{v}| = \lambda \geq 2$). Let $ \pi $ and $ \rho $ be two permutations of elements of $ \vec{v} $. Then, for any $\varphi$, in the logic C-S5$_n$ the following is a theorem:
        \begin{equation*}
            \B{\pi_1} \dots \B{\pi_\lambda} \varphi \leftrightarrow \B{\rho_1} \dots \B{\rho_\lambda} \varphi
        \end{equation*}

        \begin{proof}
            First, we notice that in the logic C-S5$_n$, for any formula $\varphi$, the following formula is a theorem:
            \begin{equation}\label{eq:comm-2}
                \B{i} \B{j} \varphi \leftrightarrow \B{j} \B{i} \varphi
            \end{equation}

            \noindent This immediately follows from axiom \axiom{C}.

            Second, by construction, we have that for each $ \pi_i $ there exists $ \rho_{k_i} $ such that $ \rho_{k_i} = \pi_i $. Consider $ \rho_{k_1} = \pi_1 $. Then, by iterating Equation \ref{eq:comm-2}, we obtain:
            \begin{align*}
                                &~ \B{\rho_1} \dots \B{\rho_{{k_1}-1}} \B{\rho_{k_1}} \B{\rho_{{k_1}+1}} \dots \B{\rho_\lambda} \varphi \\
                \leftrightarrow &~ \B{\rho_1} \dots \B{\rho_{k_1}} \B{\rho_{{k_1}-1}} \B{\rho_{{k_1}+1}} \dots \B{\rho_\lambda} \varphi \\
                \dots           &~                                                                                                \\
                \leftrightarrow &~ \B{\rho_1} \B{\rho_{k_1}} \dots \B{\rho_{{k_1}-1}} \B{\rho_{{k_1}+1}} \dots \B{\rho_\lambda} \varphi \\
                \leftrightarrow &~ \B{\rho_{k_1}} \B{\rho_1} \dots \B{\rho_{{k_1}-1}} \B{\rho_{{k_1}+1}} \dots \B{\rho_\lambda} \varphi
            \end{align*}
            %
            By repeating this manipulation for $ \pi_2, \dots \pi_m $, we obtain the conclusion.
        \end{proof}
    \end{lemma}

    \setlemmacountertoref{lem:ck-n}
    \begin{lemma}\label{lem:ck-n-proof}
        Let $ G = \{i_1, \dots, i_m\} \subseteq \agentSet $, with $ m \geq 2 $. In the logic C-S5$_n$, for any $\varphi$ and $ \vec{v} \in G^* $ we have that $ \B{i_1} \dots \B{i_m} \varphi \rightarrow \B{v_1} \cdots \B{v_{|\vec{v}|}} \varphi $ is a theorem.

        \begin{proof}
            The proof is by induction on $|\vec{v}|$.
            For the base case ($|\vec{v}|=0$) we have that the formulae $ \B{i_h} \B{i_{h+1}} \dots \B{i_m} \varphi \rightarrow \B{i_{h+1}} \dots \B{i_m} \varphi $ ($ 1 \leq h < m $) and $ \B{i_m} \varphi \rightarrow \varphi $ are instances of \axiom{T}. Together with propositional reasoning, we get that $ \B{i_1} \dots \B{i_m} \varphi \rightarrow \varphi $ is a theorem.

            Let now $|\vec{v}| = \lambda$ and suppose, by inductive hypothesis, that $\B{i_1} \dots \B{i_m} \varphi \rightarrow \B{v_1} \cdots \B{v_{\lambda}} \varphi$ is a theorem (for any formula $\varphi$). We now show that, for each $ j \in G $, the formula $\B{i_1} \dots \B{i_m} \varphi \rightarrow \B{v_1} \cdots \B{v_{\lambda}} \B{j} \varphi$ is also a theorem. By inductive hypothesis, substituting $\varphi$ with $\B{j} \varphi$, the following is a theorem:
            \begin{equation*}
                \B{i_1} \dots \B{i_m} \B{j} \varphi \rightarrow \B{v_1} \cdots \B{v_{\lambda}} \B{j} \varphi.
            \end{equation*}

            \noindent Since $ j {\in} G $, there exists $ h {\in} \{1, \dots m\} $ such that $ j = i_h $. From this and Lemma \ref{lem:slide-box-proof}, we can rewrite the antecedent of the above implication, $ \B{i_1} \dots \B{i_m} \B{j} \varphi $, as $ \B{i_1} \dots \B{j} \B{j} \dots \B{i_m} \varphi $. Moreover, it is easy to prove that from axioms \axiom{T} and \axiom{4} the following formula is a theorem (for any formula $\varphi$):
            \begin{equation}\label{eq:box-absoption}
                \B{j} \varphi \leftrightarrow \B{j} \B{j} \varphi.
            \end{equation}
            %
            Thus, we can rewrite the above formula as $ \B{i_1} \dots \B{j} \dots \B{i_m} \varphi $, which is simply $ \B{i_1} \dots \B{i_m} \varphi $. Finally, we obtain that the following is a theorem:
            \begin{equation*}
                \B{i_1} \dots \B{i_m} \varphi \rightarrow \B{v_1} \cdots \B{v_{\lambda}} \B{j} \varphi.
            \end{equation*}
            %
            This is the required result.
        \end{proof}
    \end{lemma}

    \setlemmacountertoref{lem:bounded-bisim}
    \begin{lemma}\label{lem:bounded-bisim-proof}
        Let $(M, W_d)$ be an C-S5$_n$-state, with $M=(W,R,V)$. For any $w,v \in W$, we have that $ w \bisim_{n+1} v \Leftrightarrow w \bisim v$.

        \begin{proof}
            % Without loss of generality, assume that $M$ has exactly one (strongly) connected component. We prove that for any pair of worlds $w,v\in W$, it holds that $ w \bisim_{n+1} v $ iff $ w \bisim v $. 
                        
            Clearly, if $ w \bisim v$, then $w \bisim_{n+1} v$. 
            %
            For the other direction, assume $w \bisim_{n+1} v$. First recall that there exists a path between any two worlds $w$ and $v$ of length at most $n$ (Corollary \ref{cor:diameter}). We refer to this property as ($\dagger$). By contradiction, assume that that it is not the case that $ w \bisim v $, namely there exist two worlds $w',v' \in W$ such that:
            \begin{itemize}
                \item $w' \bisim_0 v'$;
                \item $w'$ is reached by a path $\pi_w$ starting from $w$ (with $|\pi_w| {=} \ell$);
                \item $v'$ is reached by a path $\pi_v$ starting from $v$ (with $|\pi_v| {=} \ell$);
                \item There exists a world $w'' \in W$ such that $w' R_{i_{\ell+1}} w''$ (with ${i_{\ell+1}} \in \agentSet$), such that for all worlds $v'' \in W$ such that $v' R_{i_{\ell+1}} v''$, it is not the case that $w''\bisim_0 v''$ and, thus, that $w'\bisim_1 v'$ (or vice-versa, swapping $w'$ with $v'$ and $w''$ with $v''$).
                % \begin{enumerate*}[label=\roman*.]
                %     \item either there exists no world $v'' \in W$ with $v' R_i v''$, or
                %     \item for all worlds $v'' \in W$, it is not the case that $w''\bisim_0 v''$ and, thus, that $w'\bisim_1 v'$ (or vice-versa, swapping $w$ and $v$).
                % \end{enumerate*}
            \end{itemize}
            % $w' \bisim_0 v'$ reached from $w$ and $v$ with paths $\pi_w$ and $\pi_v$ of length $\ell\geq 0$, respectively, so that $w' R_i w''$ for some $i$ but either no $v''$ so that $v' R_i v''$ exists, or for all such $v''$ it is not the case that $w''\bisim_0 v''$ and thus it is not true that $w'\bisim_1 v'$ (or vice-versa, swapping $w$ and $v$).
            %
            Let us denote these two paths as:
            $\pi_w = w R_{i_1} \circ \dots \circ R_{i_\ell} w' R_{i_{\ell+1}} w''$ and
            $\pi_v = v R_{i_1} \circ \dots \circ R_{i_\ell} v' R_{i_{\ell+1}} v''$, with each $i_x\in \agentSet$ for $1 \leq x \leq \ell+1$.
            % \xrightarrow{i_1 \ldots i_\ell } w' \xrightarrow{i_{\ell+1}}
            % v \xrightarrow{i_1 \ldots i_\ell } v' \xrightarrow{i_{\ell+1}} v''

            Clearly, $\ell \geq n+1$ otherwise this would contradict the hypothesis that $ w \bisim_{n+1} v $. 
            Assume $\ell=n+1$. Thus, $|\pi_w| = |\pi_v| = n+2$. 
            %
            We now show that $w'\bisim_1 v'$. From ($\dagger$) it follows that there exist two shorter paths
            $\pi'_w = w R_{j_1} \circ \dots \circ R_{j_n} w' R_{j_{n+1}} w''$ and
            $\pi'_v = v R_{j_1} \circ \dots \circ R_{j_n} v' R_{j_{n+1}} v''$, with each $j_x\in \agentSet$ for $1 \leq x \leq n+1$.

            % \xrightarrow{i_1 \ldots i_n } \xrightarrow{i_{n+1}}
            % \xrightarrow{i_1 \ldots i_n } \xrightarrow{i_{n+1}}
            Since by hypothesis $w \bisim_{n+1} v$, this means that $v''$ as above exists and also that $w'' \bisim_0 v''$ for any such $w''$ and $v''$.
            %
            This implies $w'\bisim_1 v'$ and thus $w\bisim_{n+2} v$.
            %
            Since the same argument applies for any $\ell > n+1$, we obtain that $w \bisim v$. 
            
            % The argument can be repeated for each connected component.
        \end{proof}
    \end{lemma}

    \setlemmacountertoref{lem:char-formulae}
    \begin{lemma}\label{lem:char-formulae-proof}
        Let $(M,W_d)$ be a bisimulation-contracted C-S5$_n$-state, with $M=(W,R,V)$. Then, $|W|$ is bounded in $n$ and $|\atomSet|$.

        \begin{proof}
            % First, assume that $M$ has exactly one SCC, call it $S$.
            Given $k \geq 0$ and a world $w \in W$, we define its \emph{$k$-characteristic formula} $\chi_w^k$ as in \cite{books/el/Goranko2007}:
            \begin{equation*}
                \chi_w^k =
                \begin{cases}
                    L_w,                                                  & \textnormal{if } k=0 \\
                    L_w \wedge
                        \bigwedge\limits_{i \in \agentSet}\left(
                            \textnormal{forth}_{w,i}^k \wedge
                            \textnormal{back}_{w,i}^k
                        \right)                                           & \textnormal{otherwise}
                \end{cases},
            \end{equation*}
            %
            where:
            \begin{align*}
                L_w                        & = \bigwedge_{p \mid w \in V(p)} p \wedge \bigwedge_{p \mid w \not\in V(p)} \neg p \\
                \textnormal{forth}_{w,i}^k & = \bigwedge_{w' \mid w R_i w'} \D{i} \chi_{w'}^{k-1} \\
                \textnormal{back}_{w,i}^k  & = \B{i} \bigvee_{w' \mid w R_i w'} \chi_{w'}^{k-1}
            \end{align*}
            
            We recall the following well-known result from the literature \cite[Theorem 32]{books/el/Goranko2007}:
            
            \begin{appclaim*}[$\star$]
                The following statements are equivalent
                \begin{enumerate}
                    \item $(M, w) \models \chi_v^k$;
                    \item $w \bisim_k v$.
                \end{enumerate}
            \end{appclaim*}
            
            By using Claim ($\star$), Lemma \ref{lem:bounded-bisim-proof} and minimality of models w.r.t. bisimulation, we obtain that for any $w,v \in W$, it holds:
            \begin{equation*}
                (M, w) \models \chi_v^{n+1} \Leftrightarrow w \bisim_{n+1} v \Leftrightarrow w = v.
            \end{equation*}
            
            \noindent Clearly, the size of the set $\{\chi_w^{n+1} \mid w \in W\}$ is bounded in $n$ and $|\atomSet|$, and, hence, the number of worlds of $W$ is also bounded.% Finally, the claim follows by observing that there can only exist a bounded number (in $n$ and $|\atomSet|$) of non-bisimilar SCCs of $M$.
            % Notice that, for any $w,v \in W$, it holds that $w=v \Leftrightarrow \tau(w) = \tau(v)$.
        \end{proof}
    \end{lemma}
