\section{SUPPLEMENTARY MATERIAL: IMPLEMENTATION OF THE SUMMARY STATISTICS }\label{sect:summary_stats}

We provide here additional details concerning the implementation of the summary statistics used in this work. 

\minorsection{Second and Third Map Moments}
The implementation of second and third map moments follows \cite{moments2021}. We first smooth the maps using a top-hat filter with different smoothing scales. %\footnote{This is achieved by multiplying the coefficients of the harmonic decompositions of the weak lensing mass maps by:}
%
%\begin{equation}
%\label{eq:filter}
%W_{\ell}(\theta_0) = \frac{P_{\ell-1}({\rm %cos}(\theta_0))-P_{\ell+1}({\rm cos}(\theta_0))}{(2\ell+1)(1-{\rm %cos}(\theta_0))},
%\end{equation}
%where $P_{\ell}$ are Legendre polynomials of order $\ell$, and $\theta_0$ indicates the smoothing length.
We consider ten equally (logarithmic) spaced smoothing scales $\theta_0$ between 3.2 and 200 arcmin.  The second and third moments of the smoothed maps are computed as follows:
\begin{equation}
\label{eq:second_moments}
\avg{\kappa^2_{\theta_0}}^{i,j} = \frac{1}{N_{\mathrm{tot}}}\sum_{\mathrm{pix}}^{N_{\rm tot}} \kappa_{\theta_0,\mathrm{pix}}^i \kappa_{\theta_0,\mathrm{pix}}^j,
\end{equation}
\begin{equation}
\avg{\kappa^3_{\theta_0}}^{i,j,k} = \frac{1}{N_{\mathrm{tot}}}\sum_{\mathrm{pix}}^{N_{\rm tot}} \kappa_{\theta_0,\mathrm{pix}}^i \kappa_{\theta_0,\mathrm{pix}}^j \kappa_{\theta_0, \mathrm{pix}}^k.
\end{equation}
We can only estimate \textit{noisy} realisations of the weak lensing mass maps: $\kappa_{{\rm obs}} = \kappa + \kappa_{{\rm N}}$. Any statistic measured with data will include noise contributions  \citep{VanWaerbeke2013}, for example:
\begin{multline}
\label{eq:deno3}
\langle{\kappa_{\rm obs}}^3\rangle^{i,j,k} =  \langle \kappa^3\rangle^{i,j,k} +  \langle \kappa^3_{{\rm N}}\rangle^{i,j,k} + \\ \left[\langle  \kappa \kappa_{{\rm N}} \kappa_{{\rm N}} \rangle^{i,j,k} + \langle  \kappa \kappa \kappa_{{\rm N}}  \rangle^{i,j,k} + {\rm cycl.} \right],
\end{multline}
where `$\rm{cycl.}$' refers to the cyclic permutation of the tomographic bin indexes $i$, $j$, $k$ for the terms in parenthesis. Many of these terms are expected to be affected by source clustering. Most strikingly, certain combinations of terms that would otherwise be expected to be zero can become non-zero due to source clustering (e.g. $\avg{\kappa \kappa_{\rm N}^2}$ or $\avg{\kappa_{\rm obs} \kappa_{\rm N}^2} = \avg{\kappa \kappa_{\rm N}^2} + \avg{\kappa_{\rm N}^3}$). Note we can estimate $\kappa_{\rm N}$ from noise-only shear maps.

% These moments are estimated from data using noise-only maps. Under the hypothesis of Gaussian noise and no correlation with the convergence field, the only non-vanishing term is  $\avg{\kappa_{\rm N}\kappa_{\rm N}}$.

%When we measure the third moments of our maps, we actually measure $\langle{\kappa}^3_{\rm obs}\rangle^{i,j,k} =\langle \kappa^3\rangle^{i,j,k} + \langle \kappa^3_{\rm N}\rangle^{i,j,k} + \\ \left[\langle  \kappa^2_{\rm N}\kappa  \rangle^{i,j,k} + \langle  \kappa_{\rm N}\kappa^2  \rangle^{i,j,k} + {\rm cycl.} \right]$.  For Gaussian noise, and in absence of correlation between the convergence field and the noise field, all the moments involving the noise should vanish. If they do not, they can be subtracted off the measurements \citep{VanWaerbeke2013}. This is done when comparing the measurements to theoretical predictions, as theory models usually predict only the ``noiseless'' part of the moments.}
%where $\kappa_{\theta_0,\mathrm{pix}}^i$ is the map of the $i$-th tomographic bin,smoothed by $\theta_0$.  The sum runs over all the pixels of the maps.  As a standard practice, second moments of noise-only maps are subtracted off the second moments measurements. }

\minorsection{Peaks}
We follow the implementation of peak counts in \cite{Zuercher2021b}. We smooth the maps with Gaussian filters of different scales (12 scales with full-width-half-maximum between 2.6 and 31.6 arcmin), and we use $\sim$ 15 equally spaced thresholds in the value of $\kappa$. 

Peaks are detected in the maps corresponding to the four tomographic bins (`auto' peaks). Peaks are also detected in  maps obtained by combining two convergence maps from different tomographic bins  (`cross' peaks), following the procedure outlined in \cite{Zuercher2021}. As standard practice, peak counts obtained from noise-only maps are subtracted off the measurements. This subtraction is not guaranteed to completely remove the noise-only contribution from the measurement, due the non-linearity of the peak function. 

%We detect peaks separately on each one of the tomographic convergence maps $\kappa_i(\theta, \phi)$, where the index $i$ indicates the tomographic bin number. These maps
%can be written in the basis of the spin-0 spherical harmonics $Y_{\ell m}(\theta, \phi)$ as
%\begin{equation}
%    \kappa_i(\theta, \phi) = \sum_{\ell = 0}^{\ell_{\mathrm{max}}}\sum_{m = 0}^{\ell} \hat{\kappa}_{i, \ell m} Y_{\ell m}(\theta, \phi),
%\end{equation}
%where the upper limit $\ell_{\mathrm{max}}$ is dictated by the pixel resolution of the maps ($\ell_{\mathrm{max}} = 3072$ in our case) 

%where the indices $i$ and $j$ indicate the two different tomographic bins.
%We apply the same set of Gaussian filters to the convolved convergence maps before peak detection, as we do for the auto-peaks.

\minorsection{WPH}
Our implementation of the WPH follows \cite{Allys2020}. %Wavelet Phase Harmonics efficiently characterise the coherent structures in non-Gaussian random fields, by quantifying the phase alignment at different spatial scales~\citep{Mallat2020, Zhang2019}. They are a powerful new tool for astrophysical analysis of non-Gaussian fields for the cosmic web \citep{Allys2020} and for galactic foregrounds ~\citep{jeffrey_lfi_wph, bruno_multi_channel}.
%The WPH statistics are part of a broader set of methods, sometimes described as \textit{wavelet moments}, that were designed to emulate information capture in the manner of a convolutional neural network~\citep{Mallat_2016}. Therefore, these statistics (which include \textit{wavelet scattering transforms}, e.g. ~\citealt{Cheng2020}) can be used as highly informative summary statistics without the need for training data, and can provide useful insights as a direct analogy with deep learning.
We use the package \href{https://github.com/bregaldo/pywph}{pyWPH}~\citep{bruno_denoising} to measure the WPH statistics from flat-sky projections of the weak lensing mass maps. We first cut multiple square patches out of the DES footprint; the patches are made of 64$\times$64 pixels, with a pixel scale of 6.8 arcmin. Each patch is smoothed by a \textit{bump steerable wavelet} ${\psi}_{j,\ell}$, where $j$ specifies the spatial frequency of the order of $2^{j+1}$ pixels, and $\ell$ a rotation angle $2\pi \ell / L$ (see \citealt{Allys2020} for detailed definitions). We consider $j = {0,1,2,3}$ (with $j=3$ corresponding to a frequency of $\approx109$ arcmin), and  $\ell = {0,1,2}$ with $L = 3$.

%defined in Fourier space:
%%
%\begin{multline}
%\label{eq:mother}
%    \hat{\psi} (\overrightarrow{k}) = 0.59 %\,{\rm exp}\left( %\frac{-(|\overrightarrow{k}|-1.7\pi)^2%}{(1.7\pi)^2-(|\overrightarrow{k}|-1.7%\pi)^2}\right) {\rm %cos^7(arg(\overrightarrow{k}))} %\times\\ %\mathds{1}(0<|\overrightarrow{k}|<1.18%)\mathds{1} (0<|{\rm %arg}(\overrightarrow{k})|<\pi/2).
%\end{multline}
%
%In the above equations, $\mathds{1}$ is %the indicator function (which is either 1 %if the condition in parenthesis is passed, %or 0), while $\overrightarrow{k}$ %indicates the two dimensional Fourier mode. Eq. %\ref{eq:mother} represents the ``mother'' %wavelet; other wavelets can be obtained by %dilating and rotating the mother wavelet %in real space:
%%
%\begin{equation}
%    {\psi}_{j,\ell}\left(\overrightarrow{x%} \right)= 2^{-j}{\psi}\left( %2^{-j}r_{-\ell} %\overrightarrow{x}\right),
%\end{equation}
%
%%
%where $r_{\ell}$ is a rotation angle of %$2\pi \ell L$. The number $j$ specifies an %oscillation of the order of $2^{j+1}$ %pixels; given the fact we are using 64x64 %pixels patches, the index $j$ runs from 0 %to 4. We also consider $L=3$ (such that %$\ell$ can either be 0,1,2), which %corresponds to 3 possible orientations of %the steerable wavelet.

We apply a non-linear operation to the smoothed field that allows the capturing of interactions between scales and that provides access to non-Gaussian features of the field using second moments.\footnote{While the second moments of a wavelet transformed field depend only on the power spectrum, the second moments of a  wavelet transformed field that has undergone a non-linear transformation depend on the field's higher order moments.} Define the \textit{phase harmonic of order $p$} as \citep{Mallat2020}:
%\begin{equation}
%    {\rm PH} (z,p) \equiv |z| \exp \left[ ip \ {\rm arg} (z)\right] = z^p |z|^{1-p}.
%\end{equation}
\begin{equation}
    {\rm PH} (r e^{\textrm{i} \theta},p) \equiv r e^{\textrm{i} p \theta}.
\end{equation}
Many summary statistics can be constructed computing the second moments of two transformed and smoothed fields \citep{Allys2020}. In this work, since we are not interested in constraining cosmological parameters but rather in showcasing the impact of SC, we limit ourselves to one summary statistic: \begin{multline}
    C01^{i,k}_{j_1,j_2\neq j_1} \equiv \frac{1}{N_{\mathrm{tot}}} \sum_{\mathrm{pix}}^{N_{\rm tot}} \sum_{\ell}^{L} {\rm PH}(\kappa_{{j_1,\ell}}^i,0) {\rm PH} (\kappa_{{j_2,\ell}}^k,1) = \\ \frac{1}{N_{\mathrm{tot}}} \sum_{\mathrm{pix}}^{N_{\rm tot}} \sum_{\ell}^{L}|\kappa_{{j_1,\ell}}^i| \kappa_{{j_2,\ell}}^k,
\end{multline}
%
\noindent where $\kappa_{{j,\ell}}^i$ is the map of the $i$-th tomographic bin smoothed by the filter ${\psi}_{j,\ell}$ and where we have considered the case $j_1 \neq j_2$.

% Since $\sim |\kappa|\kappa$ is a proxy of the non-Gaussian features of the field, we chose to consider only the case $j_1 \neq j_2$, as the the skeweness of multiple fields smoothed by filters with the same size is already tested by the third moments.

% We note that the concept of wavelet phase harmonics is similar to the one of the scattering transform, introduced to weak lensing by \cite{Cheng2020}. In that case, the fields are also smoothed by directional wavelets; however, before computing the moments of the field, a different non-linear operation is considered\footnote{In particular, the authors first take the modulus of the smoothed field, then they apply another smoothing using other wavelets}.

%(with restrictions in the case of $C{\rm phase}$, $C01{\delta \ell 0}$, and $C01{\delta \ell 1}$. With the sole exceptions of $S11$, which is a Gaussian statistic, all the others probe the non Gaussian features of the field. We note that more combination of statistics could have been considered, following \cite{Allys2020}; we decided, however, to restrict ourselves to a reasonable and limited number of summary statistics, for both readability and computational reasons. 
%% Figure environment removed


\section{SUPPLEMENTARY MATERIAL: SOURCE CLUSTERING, SHAPE NOISE and GALAXY-MATTER BIAS}\label{sect:summary_stats_2}


% Figure environment removed

We show in Fig. \ref{fig:sc_2} the effect of source clustering on third moments for a mock sample with no shape noise. The effect is significantly smaller then in the case with shape noise (but does not vanish completely). In the same Figure we also show that the magnitude of the source clustering effect depends on the clustering properties of the source sample (e.g. the sources galaxy-matter bias), %, Fig \ref{fig:sc_2}).
which should be marginalised over when analysing map-based weak lensing higher-order statistics.






