% mnras_template.tex 
%
% LaTeX template for creating an MNRAS paper
%
% v3.0 released 14 May 2015
% (version numbers match those of mnras.cls)
%
% Copyright (C) Royal Astronomical Society 2015
% Authors:
% Keith T. Smith (Royal Astronomical Society)

% Change log
%
% v3.0 May 2015https://www.overleaf.com/project/5c34d2f29e67ce4bc543743f
%    Renamed to match the new package name
%    Version number matches mnras.cls
%    A few minor tweaks to wording
% v1.0 September 2013
%    Beta testing only - never publicly released
%    First version: a simple (ish) template for creating an MNRAS paper

%%%%%%%%%%%%%%%%%%%%%%%%%%%%%%%%%%%%%%%%%%%%%%%%%%
% Basic setup. Most papers should leave these options alone.
\documentclass[fleqn,usenatbib]{mnras}
%\documentclass[aps, prd,twocolumn,superscriptaddress,nofootinbib,preprintnumbers]{revtex4-1}

\usepackage{eso-pic}% http://ctan.org/pkg/eso-pic

\AddToShipoutPictureBG*{%
  \AtPageUpperLeft{%
    \hspace{0.75\paperwidth}%
    \raisebox{-1.\baselineskip}{%
      \makebox[0pt][l]{\textnormal{DES-2015-0048}}
}}}%

\AddToShipoutPictureBG*{%
  \AtPageUpperLeft{%
    \hspace{0.75\paperwidth}%
    \raisebox{-2.\baselineskip}{%
      \makebox[0pt][l]{\textnormal{FERMILAB-PUB-23-369}}
}}}%


% MNRAS is set in Times font. If you don't have this installed (most LaTeX
% installations will be fine) or prefer the old Computer Modern fonts, comment
% out the following line
\usepackage{newtxtext,newtxmath}
% Depending on your LaTeX fonts installation, you might get better results with one of these:
%\usepackage{mathptmx}
%\usepackage{txfonts}
\usepackage{mdframed}
\usepackage{graphicx}
\usepackage{subcaption}
\usepackage{lineno}
\usepackage{dsfont}
%\linenumbers
% Allow line numbering for paragraphs that include equations.
% See https://tex.stackexchange.com/questions/25784
\let\oldequation\equation
\let\oldendequation\endequation
\renewenvironment{equation}
  {\linenomathNonumbers\oldequation}
  {\oldendequation\endlinenomath}



\captionsetup{compatibility=false}
\usepackage{graphicx}
\usepackage{float}
\usepackage{hyperref}
% Use vector fonts, so it zooms properly in on-screen viewing software
% Don't change these lines unless you know what you are doing
\usepackage[T1]{fontenc}
\usepackage{ae,aecompl}
%\setlength {\marginparwidth }{2cm}
\usepackage{todonotes}


%%%%% AUTHORS - PLACE YOUR OWN PACKAGES HERE %%%%%

% Only include extra packages if you really need them. Common packages are:
\usepackage{float}
\usepackage{graphicx}	% Including figure files
\usepackage{amsmath}	% Advanced maths commands
%\usepackage{amssymb}	% Extra maths symbols
\usepackage{adjustbox}
\usepackage{multirow}
%%%%%%%%%%%%%%%%%%%%%%%%%%%%%%%%%%%%%%%%%%%%%%%%%%

%%%%% AUTHORS - PLACE YOUR OWN COMMANDS HERE %%%%%

% Please keep new commands to a minimum, and use \newcommand not \def to avoid
% overwriting existing commands. Example:
%\newcommand{\pcm}{\,cm$^{-2}$}	% per cm-squared

% names
\newcommand{\redmapper}{\textit{redMaPPer}}
\newcommand{\redmagic}{\textit{redMaGiC}}
\newcommand{\sdss}{\textit{SDSS}}
\newcommand{\des}{\textit{DES}}
\newcommand{\lsst}{\textit{LSST}}
\newcommand{\photoz}{photo-$z$}
\newcommand{\Photoz}{Photo-$z$}

\newcommand{\bk}{\textbf{k}}

% Title of the paper, and the short title which is used in the headers.
\newcommand{\Photozs}{Photo-$z$'s}
\newcommand{\PHOTOZS}{PHOTO-$z$'s}
\newcommand{\photozs}{photo-$z$'s}

\newcommand{\ccz}{cross-correlation}
\newcommand{\Ccz}{Cross-correlation}
\newcommand{\cczs}{cross-correlations}
\newcommand{\Cczs}{Cross-correlations}

\newcommand{\tj}[6]{ \begin{pmatrix}
       #1 & #2 & #3 \\
       #4 & #5 & #6 
    \end{pmatrix}}
    

%%%%%%%%%%%%%%%%%%%%%%%%%%%%%%%%%%%%%%%

% old
%\newcommand{\Nz}{$\phi(z)$}
%\newcommand{\Nuz}{$\phi_{{\rm u}}(z)$}
%\newcommand{\Nrz}{$\phi_{{\rm r}}(z)$}
%\newcommand{\phiu}{\phi_{{\rm u}}}
%\newcommand{\phir}{\phi_{{\rm r}}}

%\newcommand{\phipz}{\phi_{{\rm pz}}}
%\newcommand{\phiwz}{\phi_{{\rm wz}}}
%\newcommand{\phidel}{\phi_{\Delta}}




\newcommand{\Nz}{$n(z)$}
\newcommand{\Nuz}{$n_{{\rm u}}(z)$}
\newcommand{\Nrz}{$n_{{\rm r}}(z)$}
\newcommand{\phiu}{n_{{\rm u}}}
\newcommand{\phir}{n_{{\rm r}}}

\newcommand{\phipz}{n_{{\rm pz}}}
\newcommand{\phiwz}{n_{{\rm wz}}}
\newcommand{\phidel}{n_{\Delta}}

%%%%%%%%%%%%%%%%%%%%%%%%%%

\newcommand{\Delu}{\Delta_{\rm{u}}}
\newcommand{\Delref}{\Delta_{\rm{ref}}}
\newcommand{\delu}{\delta_{\rm{u}}}
\newcommand{\delref}{\delta_{\rm{ref}}}
\newcommand{\pu}{\phi_{\rm{u}}}
\newcommand{\pref}{\phi_{\rm{ref}}}

\newcommand{\rmin}{r_{{\rm min}}}
\newcommand{\rmax}{r_{{\rm max}}}

\newcommand{\bu}{b_{\rm{u}}}
\newcommand{\bref}{b_{\rm{ref}}}

\newcommand{\zref}{z_{\rm{ref}}}

\newcommand{\wm}{w_{\rm{mm}}}
\newcommand{\wuu}{\bar w_{\rm{uu}}}
\newcommand{\wrr}{\bar w_{\rm{rr}}}
\newcommand{\wur}{\bar w_{\rm{ur}}}


\newcommand{\Rmin}{R_{\rm{min}}}
\newcommand{\Rmax}{R_{\rm{max}}}

\newcommand{\zmin}{z_{\rm{min}}}
\newcommand{\zmax}{z_{\rm{max}}}

\newcommand{\nun}{n_{\rm{u}}}
\newcommand{\barnun}{\bar{n}_{\rm{u}}}
\newcommand{\nref}{n_{\rm{ref}}}
\newcommand{\barnref}{\bar{n}_{\rm{ref}}}

\newcommand{\Mpc}{\rm{Mpc}}
\newcommand{\avg}[1]{\langle #1 \rangle}



\newcommand{\snr}{S/N}
\newcommand{\mcal}{\textsc{metacalibration}}
\newcommand{\Mcal}{\textsc{Metacalibration}}
\newcommand{\MCAL}{\textsc{METACALIBRATION}}
\newcommand{\sx}{\textsc{SExtractor}}
\newcommand{\psfex}{\textsc{PSFEx}}
\newcommand{\ngmix}{\textsc{ngmix}}
\newcommand{\galsim}{\textsc{Galsim}}
\newcommand{\est}{e}
\newcommand{\pp}{\mbox{\boldmath $p$}}
\newcommand{\vesta}{\mbox{\boldmath $e_{\alpha}$}}
\newcommand{\vestb}{\mbox{\boldmath $e_{\beta}$}}
\newcommand{\vest}{\mbox{\boldmath $e$}}
\newcommand{\vwst}{\mbox{\boldmath $w$}}
\newcommand{\vqst}{\mbox{\boldmath $q$}}
\newcommand{\vecg}{\mbox{\boldmath $\gamma$}}
\newcommand{\vecc}{\mbox{\boldmath $c$}}
\newcommand{\vecgest}{\mbox{\boldmath $\gamma^{\mathrm{est}}$}}
\newcommand{\mcalR}{\mbox{\boldmath $R$}}
\newcommand{\mcalRtwo}{\mbox{\boldmath $R^{\rm 2pt}$}}
\newcommand{\mcalRg}{\mbox{\boldmath $R_\gamma$}}
\newcommand{\mcalRs}{\mbox{\boldmath $R_s$}}
\newcommand{\minorsection}[1]{\vspace{0.1cm} \noindent \textbf{{#1}}:}



% commentary
\definecolor{purple}{RGB}{150,0,200}
\newcommand{\question}[2]{\textcolor{purple}{\bf \{Question from #1: #2\}}}
\newcommand{\flag}[2]{\textcolor{blue}{\bf \{Comment from #1: #2\}}}
\newcommand{\new}[1]{{\bf #1}}

\newcommand{\red}[1]{\textcolor{red}{#1}}
\newcommand{\blue}[1]{\textcolor{blue}{#1}}
\newcommand{\green}[1]{\textcolor{green}{#1}}
\newcommand{\purple}[1]{\textcolor{purple}{#1}}
\newcommand{\healpix}[0]{\textsc{HEALPix}}
\newcommand{\glimpse}[0]{\textsc{Glimpse}}
\newcommand{\nside}[0]{\textsc{NSIDE}}
\newcommand{\cc}[1]{\textcolor{orange}{#1}}


\newcommand{\NJ}[1]{{\color{red}[NJ: #1]}} % Niall
\newcommand{\LW}[1]{{\color{purple}[LW: #1]}} % Lorne
\newcommand{\MG}[1]{{\color{green}[MG: #1]}} % Lorne
\newcommand{\VA}[1]{{\color{magenta}[VA: #1]}} % Virginia
\newcommand{\TK}[1]{{\color{orange}[TK: #1]}} % Virginia


\usepackage{lineno}
%\linenumbers
%Virginia's text 
\newcommand{\vac}[1]{\textcolor{orange}{#1}}



%\linenumbers

% COMMENT WHEN SWITCH TO MNRAS
%\newcommand\mnras{MNRAS}             % Monthly Notices of the Royal Astronomical Society
%\newcommand\apjl{ApJ}                % Astrophysical Journal, Letters
%\let\apjlett=\apjl                       % alternative shortcut
%\newcommand\apjs{ApJS}               % Astrophysical Journal, Supplement
%\let\apjsupp=\apjs                       % alternative shortcut

             
%\newcommand\aj{AJ}              
%\newcommand\physrep{Phys.Rep.}               
%\newcommand\aap{A\&A}               

%\newcommand\pasp{PASP}     

%%%%%%%%%%%%%%%%%%%%%%%%%%%%%%%%%%%%%%%%%%%%%%%%%%



%%%%%%%%%%%%%%%%%%% TITLE PAGE %%%%%%%%%%%%%%%%%%% 


%\begin{document}		%SWITCH TO PRD

\title[Source clustering in DES Y3]{Detection of the significant impact of source clustering on higher-order statistics with DES Year 3 weak gravitational lensing data}

%\title[Cosmology with mass map moments]{Cosmology with moments of weak lensing mass maps: methodology and validation on simulations}




\makeatletter
\def \blfootnote{\xdef\@thefnmark{}\@footnotetext}
\makeatother


%\author[IFAE]{M.~Gatti}%\thanks{E-mail: mgatti@ifae.es}
%MNRAS



\author[M. Gatti et al.]{
\parbox{\textwidth}{
\large{M.~Gatti$^{1\star}$,  
N.~Jeffrey$^{2}$  
L.~Whiteway$^{2}$
V.~Ajani$^{3}$,
T.~Kacprzak$^{3}$,
D.~Zürcher$^{3}$,
C.~Chang$^{4,5}$, 
B.~Jain$^{1}$, 
J.~Blazek$^{6}$,
E.~Krause$^{7}$, 
A.~Alarcon$^{8}$,
A.~Amon$^{9,10}$,
K.~Bechtol$^{11}$,
M.~Becker$^{8}$,
G.~Bernstein$^{1}$,
A.~Campos$^{12}$,
R.~Chen$^{13}$,
A.~Choi$^{14}$,
C.~Davis$^{15}$,
J.~Derose$^{16}$,
H.~T.~Diehl$^{17}$,
S.~Dodelson$^{12,18}$,
C.~Doux$^{19}$,
K.~Eckert$^{1}$,
J.~Elvin-Poole$^{20}$,
S.~Everett$^{21}$,
A.~Ferte$^{22}$,
D.~Gruen$^{23}$,
R.~Gruendl$^{24,25}$,
I.~Harrison$^{26}$,
W.~G.~Hartley$^{27}$,
K.~Herner$^{17}$,
E.~M.~Huff$^{21}$,
M.~Jarvis$^{1}$,
N.~Kuropatkin$^{17}$,
P.~F.~Leget$^{15}$,
N.~MacCrann$^{28}$,
J.~McCullough$^{15}$,
J.~Myles$^{29,15,22}$,
A.~Navarro-Alsina$^{30}$,
S.~Pandey$^{1}$,
J.~Prat$^{4,5}$,
M.~Raveri$^{31}$,
R.~P.~Rollins$^{32}$,
A.~Roodman$^{15,22}$,
C.~Sanchez$^{1}$,
L.~F.~Secco$^{5}$,
I.~Sevilla-Noarbe$^{33}$,
E.~Sheldon$^{34}$,
T.~Shin$^{35}$,
M.~Troxel$^{36}$,
I.~Tutusaus$^{37,38,39}$,
T.~N.~Varga$^{40,41,42}$,
B.~Yanny$^{17}$,
B.~Yin$^{12}$,
Y.~Zhang$^{43,44}$,
J.~Zuntz$^{45}$,
S.~S.~Allam$^{17}$,
O.~Alves$^{55}$,
M.~Aguena$^{47}$,
D.~Bacon$^{49}$,
E.~Bertin$^{52,53}$,
D.~Brooks$^{2}$,
D.~L.~Burke$^{15,22}$,
A.~Carnero~Rosell$^{46,47,48}$,
J.~Carretero$^{58}$,
R.~Cawthon$^{68}$,
L.~N.~da Costa$^{47}$,
T.~M.~Davis$^{71}$,
J.~De~Vicente$^{33}$,
S.~Desai$^{70}$,
P.~Doel$^{2}$,
J.~Garc\'ia-Bellido$^{59}$,
G.~Giannini$^{4}$,
G.~Gutierrez$^{17}$,
I.~Ferrero$^{56}$,
J.~Frieman$^{17,5}$,
S.~R.~Hinton$^{69}$,
D.~L.~Hollowood$^{51}$,
K.~Honscheid$^{60,61}$,
D.~J.~James$^{50}$,
K.~Kuehn$^{62,63}$,
O.~Lahav$^{2}$,
J.~L.~Marshall$^{57}$,
J. Mena-Fern{\'a}ndez$^{33}$,
R.~Miquel$^{66,58}$,
R.~L.~C.~Ogando$^{67}$,
A.~Palmese$^{12}$,
M.~E.~S.~Pereira$^{64}$,
A.~A.~Plazas~Malag\'on$^{15,22}$,
M.~Rodriguez-Monroy$^{33}$,
S.~Samuroff$^{6}$,
E.~Sanchez$^{33}$,
M.~Schubnell$^{55}$,
M.~Smith$^{65}$,
F.~Sobreira$^{30,47}$,
E.~Suchyta$^{54}$,
M.~E.~C.~Swanson$^{2}$,
G.~Tarle$^{55}$,
N.~Weaverdyck$^{55,16}$,
and P.~Wiseman$^{65}$
\begin{center} (DES Collaboration) \end{center}
}
%\parbox{\textwidth}{ \small
%\textit{The authors' affiliations are shown in Appendix~\ref{sec:affiliations}.
%}}
}}
% These dates will be filled out by the publisher
% \date{Accepted 2021. Received 2021; in original form ZZZ}
%\date{Accepted 2021 May 11. Received 2021 May 07; in original form 2021 March 26 
%}







		%SWITCH TO MNRAS
%\input{authors_PRD.tex}		%SWITCH TO PRD
\date{\today}



\begin{document}     %SWITCH TO MNRAS
\label{firstpage} % SWITCH TO MNRAS
\pagerange{\pageref{firstpage}--\pageref{lastpage}}% 
\maketitle        	 %SWITCH TO MNRAS



% Abstract of the paper
\begin{abstract}
We demonstrate and measure the impact of source galaxy clustering on higher-order summary statistics of weak gravitational lensing data. By comparing simulated data with galaxies that either trace or do not trace the underlying density field, we show this effect can exceed measurement uncertainties for common higher-order statistics for certain analysis choices. Source clustering effects are larger at small scales and for statistics applied to combinations of low and high redshift samples, and diminish at high redshift. We evaluate the impact on different weak lensing observables, finding that third moments and wavelet phase harmonics are more affected than peak count statistics. Using Dark Energy Survey Year 3 data we construct null tests for the source-clustering-free case, finding a $p$-value of $p=4\times10^{-3}$ (2.6 $\sigma$) using third-order map moments and $p=3\times10^{-11}$ (6.5 $\sigma$) using wavelet phase harmonics. The impact of source clustering on cosmological inference can be either be included in the model or minimized through \textit{ad-hoc} procedures (e.g. scale cuts). We verify that the procedures adopted in existing DES Y3 cosmological analyses (using map moments and peaks) were sufficient to render this effect negligible.  Failing to account for source clustering can significantly impact cosmological inference from higher-order gravitational lensing statistics, e.g. higher-order N-point functions, wavelet-moment observables (including phase harmonics and scattering transforms), and deep learning or field level summary statistics of weak lensing maps. We provide recipes both to minimise the impact of source clustering and to incorporate source clustering effects into forward-modelled mock data. 
%If left unaccounted for, this could significantly impact cosmological inference from higher-order gravitational lensing statistics, e.g. higher-order N-point functions, wavelet-moment observables (including phase harmonics and scattering transforms), and deep learning or field level summary statistics of weak lensing maps. We provide an efficient recipe for incorporating source clustering effects into forward-modelled mock data.
% \TK{it would be good to add info like, "for stage 3 surveys that use large scales we find this effect to be sub-dominant, whereas for stage 4 can dominate the error budget if unaccounted for..}
\end{abstract}

\begin{keywords}
cosmology: observations 
\end{keywords}


%\blfootnote{$^{\star}$ E-mail: marcogatti29@gmail.com}
%\blfootnote{Affiliations are listed at the end of the paper.}

%\maketitle  %SWITCH TO PRD

%%%%%%%%%%%%%%%%%%%%%%%%%%%%%%%%%%%%%%%%%%%%%%%%%%

%%%%%%%%%%%%%%%%% BODY OF PAPER %%%%%%%%%%%%%%%%%%



%%%%%%%%%%%%%%%%% INTRODUCTION %%%%%%%%%%%%%%%%%%%%%
\section{Introduction}
% \vspace{-0.1cm}
% \enlargethispage*{0.4cm}

Weak gravitational lensing from large-scale structure in the Universe induces small distortions in the observed shape of background source galaxies. The weak lensing signal can be measured using large samples of galaxies to observe correlated distortions in observed galaxy ellipticities \citep[see][]{Bartelmann2001}. 
The angular distribution of source galaxies is not uniform; it is modulated by observational and selection effects (such as varying observing depth) and by clustering due to galaxies tracing the underlying density field. The latter effect, called \textit{source clustering} \citep{Schneider2002,Schmidt2009,Valageas2014,Krause2021}, causes the galaxy number density to be correlated with the target lensing signal: since we expect a larger lensing signal along overdense lines-of-sight, we preferentially sample the shear field where its value is larger. 
For pixelized shear maps, this results in two distinct effects: (1) the average noise-free lensing signal is modulated by a different effective redshift distribution, and (2) the \textit{shape noise} (due to the intrinsic ellipticities of galaxies) is correlated with the lensing signal.

Higher-order statistics have recently been growing in popularity as powerful tools for efficiently extracting cosmological information from current weak lensing data (e.g. \citealt{ Vicinanza2016,Martinet2018, Fluri2019,Cheng2020,moments2021,jeffrey_lfi,Zuercher2022}).  Their use can improve constraints on cosmological parameters (relative to standard two-point statistics), can help discriminate between general relativity and modified gravity theories \citep{Cardone2013,Peel2018}, and can help self-calibrate astrophysical and observational nuisance parameters \citep{Pyne2021}. Given the increasing precision of these measurements, the impact of systematic errors on higher-order statistics is a subject of careful consideration.
 
 The impact of source clustering has generally been neglected in the forward model, as it has often been considered a small, higher-order contribution to weak lensing observables. The efficiency of lensing peaks roughly halfway between the source and the observer, and vanishes at the source location; any correlation between the shear field `seen' by the source galaxy and the density field it lives on is suppressed. Source clustering has been studied in the context of two-point correlation functions, and theoretical calculations by \cite{Krause2021} have shown it to be negligible for Stage III surveys for catalogue-based Gaussian statistics. Whether its impact on weak lensing higher-order statistics is also negligible is less clear, although some early estimates suggested a stronger impact on three-point correlation functions \citep{Valageas2014}. The effect of source clustering has not to date been explicitly included in the suites of simulations used for simulation-based cosmological analyses \citep[e.g.][]{Martinet2018,Zuercher2021}, although peak statistics analyses by \cite{Kacprzak2016} and by \cite{Zuercher2021} performed initial tests of this effect (under some simplifying assumptions), showing no significant effect on their cosmological constraints.

This work develops a forward-modelling procedure to introduce source clustering effects into the simulated maps.  
We consider the impact of source clustering on several non-Gaussian observables, looking primarily at map-based estimators.
We show that source clustering generates a clear signature on higher-order summary statistics for specific analysis choices, we demonstrate this effect in the Dark Energy Survey (DES) Year 3 (Y3) data, and we discuss the impact of this effect on previously published DES measurements.
%  $$\kappa = \frac{3 H_0^2 \Omega_m}{2} \sum_{i=0}^{N-1} \big[ \frac{1}{2} (\chi_{i+1} +  \chi_{i}) \big]   \frac{ (\chi_N - \frac{1}{2} (\chi_{i+1} +  \chi_{i}))}{\chi_N} $$
%  $$\frac{\delta_i)}{a(\frac{1}{2} (\chi_{i+1} +  \chi_{i}))} \ (\chi_{i+1} -  \chi_{i})$$
%  These features in the data would be otherwise left unnaccounted.
 %and that are able to explain features in the measurements on data that would be otherwise left unaccounted.
%  The paper is organised as follows.
%and are able to explain a number of features in our measurements that would be otherwise left 
%We demonstrate the impact of galaxy source clustering on summary statistics of weak gravitational lensing observables. Using Dark Energy Survey (DES) Year 3 (Y3) data, we show a clear detection of {\color{red}... using (peaks, wavelet moments, 3rd moments...).} Using simulations with random source galaxy positions and clustered source galaxy positions, we show how this effect can be modelled in mock data. If left unaccounted, this effect could significantly impact cosmological inference made with higher-order statistics using weak gravitational lensing. Affected observables include, but are not limited to: higher-order N-point functions, wavelet-moment observables (including phase harmonics and scattering transforms), and deep 
%has been shown to be negligible for two-point statistics for current stage III surveys \citep{Krause2021}. Its effects on higher-order statistics have not been studied in depth; one one hand, it is hard to model
%is the effect of source clustering, which refers to the fact that galaxies selected by weak lensing surveys are not randomly distributed across the sky, but are mostly located on top of matter overdensities. 
%This effect modulates the source density distribution in a way that correlates with the lensing signal itself, since we expect a larger lensing signal along overdense line-of-sights. source clustering has been shown to be negligible for two-point statistics for current stage III surveys \citep{Krause2021}; here we study its impact on higher-order summary statistics.
%%%%%%%%%%%%%%%%% DATA AND SIMULATIONS %
\section{Data and Simulations}
% \vspace{-0.1cm}
% \enlargethispage*{0.4cm}
\subsection{DES Y3 weak lensing catalogue}
We use the DES Y3 weak lensing catalogue \citep*{y3-shapecatalog}; this consists of 100,204,026 galaxies, with a weighted $n_{\rm eff}=5.59$~galaxies~arcmin$^{-2}$, over an effective area of 4139  deg$^2$. It was created using the \textsc{METACALIBRATION} algorithm \citep{HuffMcal2017, SheldonMcal2017}, which provides self-calibrated shear estimates starting from (multi-band) noisy images of the detected objects. A residual small calibration (via a multiplicative shear bias) is provided; based on sophisticated image simulations \citep{y3-imagesims}, it accounts for blending-related detection effects. An inverse variance weight is further assigned to each galaxy in the catalogue to enhance the overall signal-to-noise. The sample is divided into four tomographic bins of roughly equal number density \citep*{y3-sompz}. Redshift distributions are provided by the SOMPZ method, in combination with  clustering redshift constraints \citep*{y3-sompz}. % in combination with clustering redshift constraints \citep*{y3-sourcewz} and corrections due to the redshift-dependent effects of blending \citep{y3-imagesims}.%, and redshift distributions are provided in combination with the clustering redshift method \citep*{y3-sourcewz}. The $n(z)$'s are further tweaked to take into account the redshift-dependent effects of blending \citep{y3-imagesims}. 

%The DES Y3 weak lensing catalogue \citep*{y3-shapecatalog} consists of 100,204,026 objects, with a weighted $n_{\rm eff}=5.59$~galaxies~arcmin$^{-2}$, over an effective area of 4139 square degrees. It has been created using the \textsc{METACALIBRATION} algorithm \citep{HuffMcal2017, SheldonMcal2017}, which provides shear estimates starting from (multi-band) noisy images of the detected objects. For the DES Y3 catalogue, the \textit{r, i, z} bands have been used. The \textsc{METACALIBRATION} algorithm relies on an approximate per-galaxy shear field estimator, and subsequently self-calibrates the shear estimates, also accounting for selection effects. Selection cuts are provided with the catalogue to minimise the impact of systematic effects. A residual small calibration, in the form of a multiplicative shear bias, is provided based on sophisticated image simulations \citep{y3-imagesims}, which accounts for blending-related detection effects. An inverse variance weight is further assigned to each galaxy in the catalogue which enhances the overall signal-to-noise. The sample is divided into four tomographic bins of roughly equal number density using the SOMPZ method \citep*{y3-sompz}, and redshift distributions are provided in combination with the clustering redshift method \citep*{y3-sourcewz}. The $n(z)$'s are further tweaked to take into account the redshift-dependent effects of blending \citep{y3-imagesims}. 



%DES \citep{DES2016} is a six-year survey that spans $\sim 5000~\mathrm{\deg}^2$ of the southern hemisphere. Images have been taken in $grizY$ filters by the $570$~megapixel Dark Energy Camera \citep[DECam,][]{Flaugher2015}, mounted on the Cerro Tololo Inter-American Observatory (CTIO) four-meter Blanco telescope in Chile.  The raw images were processed by the DES Data Management (DESDM) team  \citep{Sevilla2011,Morganson2018,DES_DR1}. Full details about the image processing are provided in \cite{,Morganson2018,DES_DR1}.



%The weak lensing sample is divided into four tomographic bins of roughly equal number density using the SOMPZ method \citep*{y3-sompz}; SOMPZ, in combination with constraints from clustering redshifts \citep*{y3-sourcewz}, also provides redshift distribution estimates. The $n(z)$'s are further tweaked to take into account the redshift-dependent effects of blending \citep{y3-imagesims}. 

\vspace{-0.1cm}
\subsection{Simulations}\vspace{-0.1cm}
We rely on simulations produced using the \textsc{PKDGRAV3} code \citep{potter2017pkdgrav3}. We use 50 independent realisations at the fixed cosmology $\Omega_{\rm m} = 0.26$, $\sigma_8 = 0.84$, $\Omega_{\rm b} = 0.0493$, $n_{\rm s} = 0.9649 $, $h = 0.673$ from the DarkGridV1 suite, described in detail in \cite{Zuercher2021,Zuercher2021b}. All simulations include three massive neutrino species with a mass of $m_{\nu}=0.02$ eV per species. The simulations were obtained using 14 replicated boxes in each direction ($14^3$ replicas in total) so as to span the redshift interval from $z = 0$ to $z = 3$. Each individual box contains $768^3$ particles and has a side-length of 900 $h^{ - 1}$ Mpc.  For each simulation, lens planes $\delta_{\rm shell}(\hat{\boldsymbol{\rm n}}, \chi)$ are provided at $\sim87$ redshifts from $z=3$ to $z=0$. The lens planes are provided as \textsc{HEALPIX} \citep{GORSKI2005} maps and are obtained as the overdensity of raw number particle counts; for this work, we downsample the orginal resolution of \textsc{NSIDE} = 2048 to \textsc{NSIDE} = 1024 (with pixel size $\approx$ 3.4 arcmin).  The lens planes are converted into convergence planes $\kappa_{\rm shell}(\hat{\boldsymbol{\rm n}}, \chi)$ under the Born approximation (e.g. Eq. 2 from \citealt{Fosalba2015}). Lastly, shear planes $\gamma_{\rm shell}(\hat{\boldsymbol{\rm n}}, \chi)$ are obtained from the convergence maps using a full-sky generalisation of the \cite{KaiserSquires} algorithm \citep*{y3-massmapping}.

%Besides the 50 realisations at the fiducial cosmology, we also use for testing purposes 5 additional realisations with different $\Omega_{\rm m} = X $, $\sigma_8 = Y$. \textbf{Mention high resolution simulation}
%Lensing quantities (shear and convergence) were obtained under the Born approximation. For each simulation, we cut out four independent DES Y3 footprints and thereby created 200 independent catalogues in a fashion similar to the T17 simulations.

% \vspace{-0.1cm}
%%%%%%%%%%%%%%%%% SOURCE CLUSTERING IMPLEMENTATION %%%%%%%%%%%%%%%%%%%%%
\section{Source Clustering Implementation}\label{sec:implementation}
% \enlargethispage{0.4cm}\vspace{-0.1cm}
% \subsection{Theory}
In the limit of high source galaxy density, the observed projected shear in direction $\vv{\theta}$ will be
%\begin{equation}
%\gamma(\vv{\theta}) = \frac{\int \ n(\vv{\theta}, z) \gamma(\vv{\theta}, z) \mathrm{d} z}{\int \  n(\vv{\theta}, z) \mathrm{d} z} = \frac{1}{n_g (\theta)} \int \ n(\vv{\theta}, z) \gamma(\vv{\theta}, z) \mathrm{d} z
%\end{equation}
\begin{equation}
\gamma(\vv{\theta}) = \frac{\int n(\vv{\theta}, z) \, \gamma(\vv{\theta}, z) \, \mathrm{d} z}{\int n(\vv{\theta}, z) \, \mathrm{d} z},
\end{equation}
\noindent where $n(\vv{\theta}, z)$ is the unnormalised galaxy density (i.e. $\int_{V} n(\vv{\theta}, z) \, \mathrm{d} \! \vv{\theta} \mathrm{d} z$ is the number of source galaxies in the volume $V$). The observed shear $\gamma$ is the sum of signal $\gamma_s$ and noise $\epsilon_n$:
\begin{multline}
\gamma(\vv{\theta}) =
\frac{\int n(\vv{\theta}, z) \Big( \gamma_s(\vv{\theta}, z) + \epsilon_n(\vv{\theta}, z) \Big) \, \mathrm{d}z}{\int n(\vv{\theta}, z) \, \mathrm{d} z} = \gamma_{s}(\vv{\theta}) + \gamma_{n}(\vv{\theta}) .
\end{multline}
It has been standard in many previous analyses to use the spatial average
\begin{equation}
\bar{n}(z) = \frac{\int n(\vv{\theta}, z) \, \mathrm{d} \! \vv{\theta}}{\int \mathrm{d} \! \vv{\theta}}
\end{equation}
as an approximation to $n(\vv{\theta}, z)$; however, this approximation cannot include the effect of source clustering. 
%This approximation will not change the expected value of $\gamma_{\textsc{p}}$ (but will add variance) if we are able to assume $n(\vv{\theta}, z)$ varies randomly with $\vv{\theta}$,  %\footnote{This is not expected to significantly impact cosmological inference ({\color{red}  NJ: CITE THIS}).}.
%{\color{red} NJ: {check the following is all consistent!}}
We instead model the directional variation of the source galaxy distribution arising from its dependence on the overdensity field $\delta(\vv{\theta}, z)$, i.e. $n(\vv{\theta}, z) = \bar{n}(z) \left[ 1 + f(\delta(\vv{\theta}, z)) \right]$ for some function $f$. This leads to a relation between $n(\vv{\theta}, z)$ and the observed shear $\gamma(\vv{\theta}, z)$, as they both depend on $\delta$.
This relation has a direct impact on the expected value $\gamma_{s}$ (i.e. the signal is modulated). 
Additionally, as the variance of the noise term $\gamma_{n}$ depends on $n$ (more source galaxies leads to reduced noise), this relation will have an impact on the expected value of terms such as $\gamma_{s}\gamma_{n}^2$.
A simulation that does not include source clustering effects is in danger of incorrectly modelling these expected values.

% \subsection{Implementation}\label{sec:implementation} \enlargethispage{0.4cm}\vspace{-0.1cm}

Below we describe how to create pixelized shear maps both without and with source clustering effects. We consider one fixed tomographic bin. We assume as inputs a noiseless pixelized simulated shear map and a separate galaxy shape catalogue. The latter is needed to supply shape noise information (as the simulated shear map is not assumed to have an associated simulated galaxy catalogue); in our case the DES Y3 shape catalogue serves this purpose. We then add a source clustering effect by amending both signal and noise terms using factors related to the matter overdensity in the shear simulation.

An alternative method for creating shear simulations with source clustering would be to use the results of the n-body simulation (i.e. the simulation used to create the simulated shear field) to directly create a galaxy catalogue (using some HOD prescription, for example), to assign shape noise to these galaxies, and to use this information to add noise to the shear simulation. However this task is complex, and therefore we opt for the simpler approach implemented in this work.

%From the shear simulation, let $s$ denote a redshift shell and $\gamma(p, s)$ the noiseless shear.
%From the galaxy catalogue, let $g$ denote a galaxy; it has weight $w_g$ and ellipticity $e_g$.
%Let $\bar{n}(s)$ denote the galaxy count in shell $s$ \citep{y3-sompz}.
%this is an average over the galaxy survey footprint and hence is only weakly dependent on the details of this survey.

%We stress that our shear simulations do not have an associated simulated galaxy catalogue (such a catalogue is not needed to model noisy pixelized maps). The effect of source clustering can be added later at the map level, modulating the signal per pixel using the simulated density field. We comment more on this later in the paper.

%We generate (based on the DES galaxy catalogue) a set of \textit{simulated galaxies} each with a weight and an ellipticity, 
%\textbf{but without the effect of source clustering. The effect of source clustering can then be added later at the map level, modulating the signal per pixel.}  
%Let $D$ be the two dimensional distribution of galaxy weights $w_g$ versus magnitudes of galaxy ellipticities $|e_g|$.

%The number of simulated galaxies in a pixel is a Poisson draw with expected value $\langle n_{\rm gal} \rangle$, which is a function of with the pixel's $r$-band galaxy catalogue depth ${\rm M}_r(p)$ (i.e. mean magnitude limit).
%Each simulated galaxy $g$ gets a weight $w_g$ (equal to the weight of one of the catalogue galaxies in the same pixel -- in this correspondence, catalogue galaxies may need to be used multiple times) and gets an ellipticity $e_g$ (set to have a uniformly random phase angle and with magnitude randomly selected conditional on $w_g$).
% This process has the advantage (over the naive approach of using the weights and random rotations of the ellipticities of the catalogue galaxies) of nullifying not only the shear signal but also any source clustering signal present in the catalogue, while retaining the observational selection effects on the galaxy distribution.
%Compared to the naive approach of using the existing galaxies (with random ellipticity rotations) from the observed catalogue, our process has the advantage of nullifying not only the shear signal but also any source clustering signal present in the catalogue, while retaining the observational selection effects on the galaxy distribution.

 In what follows let $p$ be a pixel, $s$ a thin redshift shell, $\gamma(p, s)$ the noiseless shear from the shear simulation, and $\bar{n}(s)$ the galaxy count across the whole footprint \citep{y3-sompz}. From the galaxy catalogue, let $g$ denote a galaxy, $w_g$ its weight, and $e_g$ its ellipticity after the application of a random rotation to erase the shear signal.

\minorsection{Mock shear maps with no source clustering}
The output simulated shear for a given pixel $p$ is the sum of signal and shape noise contributions:
\begin{equation}
\gamma(p) = \frac{\sum_s \bar{n}(s) \gamma(p, s)}{\sum_s \bar{n}(s)} + \frac{\sum_g w_g e_g}{\sum_g w_g}.
\end{equation}
In the signal term the sum is over all shells $s$, and in the noise term the sum is over all the shape catalogue (i.e. DES Y3) galaxies $g$ in $p$.
%Note that (except for the minor influence of $\bar{n}$) the signal comes from the shear simulation and the noise from the catalogue; signal and noise are therefore unconnected (contrary to physical expectations).

\minorsection{Mock shear maps with source clustering}
Let $\delta(p, s)$ be the matter overdensity in the shear simulation. Let $b_g$ be the galaxy-matter bias; for simplicity we assume linear biasing to hold and moreover for our main tests we assume $b_g=1$ (a reasonable choice for the blue field galaxies that constitute most of the galaxies in the shear catalogue). The factor $\bar{n}(s) \left[1 + b_g \delta(p, s)\right]$ is then the relative galaxy count in pixel $p$ and shell $s$; it is generated from the shear simulation and is therefore consistent with the shear signal. In the output simulated shear both the signal and the shape noise contributions have been amended to account for source clustering as follows:
\begin{multline}
\label{eq:sc_pixel}
\gamma_{\textrm{SC}}(p) = \frac{\sum_s \bar{n}(s) \left[1 + b_g \delta(p, s) \right] \gamma(p, s)} {\sum_s \bar{n}(s) \left[1 + b_g \delta(p, s)\right]} \  + \\
\left(\frac{\sum_s \bar{n}(s)}{\sum_s \bar{n}(s) \left[1 + b_g \delta(p, s)\right]}\right)^{1/2} F(p) \, \frac{\sum_g w_g e_g}{\sum_g w_g}.
\end{multline}
The signal term is a weighted average over shells; here the average has been amended to include a shear-correlated source galaxy count. In the shape noise term there are two additional factors. The first, a source clustering factor, results in the shape noise variance scaling as the inverse of the relative galaxy count, as desired; this gives a correlation between the shear signal in a pixel and the square of the shape noise that was not present before. The second, $F(p)$, is a near-unity scale factor introduced to avoid double-counting source clustering effects. The DES Y3 catalogue used to model the shape noise of the pixels is already affected by source clustering. In practice this means that the noise of the catalogue is already modulated by $1/\sqrt{\sum_s \bar{n}(s) \left[1 + b_g \delta_{\rm data}(p, s)\right]}$. This modulation is not correlated with the large scale structure of the simulations. However, since Eq. \ref{eq:sc_pixel} introduces a similar modulation, the net effect is that the even moments of the pixel's simulated noise (variance,  kurtosis, etc.) are slightly enhanced with respect to data, mostly at small scales and low redshifts. The function $F(p)$ corrects this enhancement. We opted for the following expression:
%
\begin{equation}
    F(p) = A\sqrt{1-B \sigma_{e}^2(p)},
\end{equation}
%
where the coefficients $A$ and $B$ are per-bin constants, and  
$\sigma_{e}^2(p)$ is the variance of the pixel noise. This correction is (only mildly) cosmology dependent; we used our simulations at fixed cosmology to estimate the two sets of constants for the four bins: $A = [0.97,0.985,0.990,0.995]$, and $B = [0.1,0.05,0.035,0.035]$.


%\footnote{The variance of the noise is increased at most by 5 (resp. 1) per cent at small scales for the first (resp. fourth) DES tomographic bin.} This effect is small enough to not affect any of our conclusions. One could avoid this bias by using realistic galaxy models for the simulations and fully modelling the shape catalogue selection function, but as noted above this is a complex task.

\minorsection{Remarks concerning our implementation}
We generate shear maps for each tomographic bin. The 50 independent simulations at fixed cosmology for our main tests yield 200 independent simulated DES Y3 shear catalogues (as we can cut four independent DES Y3 footprints from each full-sky map). The simulations have not been run at the best-fitting cosmology for the data. However, based on the results presented in \cite{moments2021}, the cosmology chosen for the simulations should still provide a reasonable fit to the data. Moreover, for simplicity we did not include any intrinsic alignments and we assumed zero shear and redshift biases; we do not expect this to affect any of our conclusions.

%The DES Y3 catalogue used to model the shape noise of the pixels is already affected by source clustering. In practice this means that the noise of the catalogue is already modulated by $1/\sqrt{\sum_s \bar{n}(s) (1 + b_g \delta_{\rm data}(p, s))}$. This modulation is not correlated with the large scale structure of the simulations. Since Eq. \ref{eq:sc_pixel} introduces a similar modulation, the net effect is that the pixel even moments of the simulated noise (variance,  kurtosis, etc.) are slightly enhanced with respect to data, mostly at small scales and low redshift.\footnote{The variance of the noise is increased at most by 5 (resp. 1) per cent at small scales for the first (resp. fourth) DES tomographic bin.} This effect is small enough to not affect any of our conclusions. One could avoid this bias by using realistic galaxy models for the simulations and fully modelling the shape catalogue selection function, but as noted above this is a complex task.



% \minorsection{Mock shear catalogues with source clustering}
% \textbf{Alternatively, to create shear \textit{catalogues} with source clustering effects (rather than shear maps), it suffices to create  `simulated' galaxies with their number being a Poisson draw with expected value $\langle n_{\rm gal} \rangle ({\rm M}_r,p) {\left[\bar{n}(s) (1 + b_g \delta(p, s))\right]}$), and then shear individual galaxies by $\gamma(p, s)$}. No further modulation of the shape noise is required. This technique of modulating the \textit{number} of simulated galaxies could also have been used when creating shear maps (as an alternative to scaling the shape noise term in Eq. \ref{eq:sc_pixel}); the two methods are equivalent, but we follow the latter as it makes the interpretation of the results easier.
%%\noindent As we look primarily at map-based estimators, we provided a prescription to create pixelized shear maps. Catalogues with source clustering effects could be created by generating a sample of simulated galaxies through a Poisson draw with expected value $\langle n_{\rm gal} \rangle ({\rm M}_r,p) {\left[\sum_s \bar{n}(s) (1 + b_g \delta(p, s))\right]}$, and shearing galaxies by $\frac{\sum_s \bar{n}(s) (1 + b_g \delta(p, s)) \gamma(p, s)}{\sum_s \bar{n}(s) (1 + b_g \delta(p, s))}$
%%%%%%%%%%%%%%%%% SUMMARY STATISTICS %%%%%%%%%%%%%%%%%%%%%



%%%%%%%%%%%%%%%%% RESULTS %%%%%%%%%%%%%%%%%%%%%
\section{Results}

% Figure environment removed

%% Figure environment removed



In this work, we consider the following summary statistics:

\begin{itemize}
    \item \textbf{Second and Third Map Moments}: second moments are a Gaussian statistic (i.e. a function only of the power spectrum), whereas third moments probe additional non-Gaussian features of the field \citep{VanWaerbeke2013,Petri2015,Chang2018,Vicinanza2018,Peel2018, G20,moments2021}. Second and third moments of the DES Y3 weak lensing mass maps were used in \cite{moments2021} to infer cosmology; we use the same implementation of the moments estimator.
    \item \textbf{Peaks}: the peaks statistic counts the number of peaks of the smoothed map above a certain threshold. We follow the implementation of peak counts in \cite{Zuercher2021b}. 
    \item \textbf{Wavelet Phase Harmonics (WPH)}: these statistics are part of a broader set of methods (which include \textit{wavelet scattering transforms}, e.g. ~\citealt{Cheng2020}) that were designed to emulate information capture in the manner of a convolutional neural network~\citep{Mallat_2016} without the need for training data. WPH statistics characterise the coherent structures in non-Gaussian random fields, by quantifying the phase alignment at different spatial scales~\citep{Mallat2020, Zhang2019}, and they can provide useful insights as a direct analogy with deep learning. 
    We follow the implementation of WPH in \cite{Allys2020}, which has already found success with astrophysical applications~\citep{bruno_denoising, jeffrey_lfi_wph}.
\end{itemize}

These map-based statistics are applied to reconstructed weak lensing mass maps, using a full-sky generalisation of the \cite{KaiserSquires} algorithm that recovers a noisy estimate of the lensing convergence field $\kappa$ from pixelized shear maps \citep*[see][]{y3-massmapping}. The statistics are applied to `smoothed' versions of the maps. More details about the specific implementation of each statistic is provided in Appendix \ref{sect:summary_stats}. 

{For each statistic, we assess in Fig. \ref{fig:sc} the impact of source clustering by comparing the measurements from the simulations with and without source clustering (solid and dashed lines); these measurements are then compared to data (red points). When possible, we highlight the part of the measurements not included in the DES Y3 cosmological analyses (grey regions in Fig. \ref{fig:sc}).}

\minorsection{Second and Third Map Moments}
Given current measurement uncertainties, the impact of source clustering on second moments is negligible (first row of Fig. \ref{fig:sc}), in line with the findings of \cite{Krause2021}. It only slightly dampens the signal at small scales and in moments that include a low redshift bin, for both `auto' and `cross' moments. %\MG{Should I keep the following sentences?} Small scales are often removed from the cosmological analyses to minimise the impact of potential baryonic contamination (.g., \citealt{y3-cosmicshear1,y3-cosmicshear2,moments2021}, which also helps reducing the impact of source clustering on Gaussian statistics.  Analyses that incorporate baryonic feedback treatments in order to take advantage of the small scale regime might want to assess the impact of source clustering on their results, as it might erroneously be interpreted as an excess of baryonic feedback at low redshift.
For third moments the impact is more dramatic (second row of Fig. \ref{fig:sc}), particularly for moments that include low redshift bins. The data clearly follow better the simulations with source clustering, and the difference between the two sets of simulations is often significantly larger than measurement uncertainties. 

Most of the effect induced by source clustering is due to a non-zero correlation between the convergence field and the noise. The effect of source clustering for a mock sample with no shape noise is significantly smaller (but does not vanish completely, see Fig. \ref{fig:sc_2}). The non-zero noise-signal correlation follows from the noise modulation introduced in Eq. \ref{eq:sc_pixel}, and it is a consequence of the map-making procedure. This can also be tested in data by looking at third moments that combine the noisy convergence maps and `noise-only' maps created by randomly rotating the galaxy ellipticities of the shape catalogue. The rotation erases the shear signal but preserves the source clustering modulation of the noise. In simulations, we find that while moments of the form $\avg{{\kappa}^2{\kappa}_{{\rm N}}}$ or $\avg{{\kappa}_{{\rm N}}^3}$ are consistent with zero within uncertainties, $\avg{{\kappa}{\kappa}^2_{{\rm N}}}$ are not (in the presence of source clustering).  This is shown in the third row of Fig. \ref{fig:sc}, where simulations with source clustering provide a good match to the data. That $\avg{{\kappa}{\kappa}^2_{{\rm N}}}$ is non-zero was already noted in \cite{G20,moments2021}, although the nature of the effect was not then understood. To compare the measurements to theory predictions, the authors of those papers subtracted $\avg{{\kappa}{\kappa}^2_{{\rm N}}}$ from the estimated third moments $\langle{\kappa_{\rm obs}}^3\rangle$. The result of this procedure is shown in the fourth row of Fig. \ref{fig:sc}; the impact of source clustering is greatly minimised, {although the measurement errors are now larger}. This procedure completely removes the contribution due to the non-zero correlation between the convergence field and the noise, and leaves the part of the effect associated with the modification of the average shear signal in the pixels, which is sub-dominant. Using the simulations produced in this work, we verified that the scale cut adopted in the \cite{G20,moments2021} analysis, in combination with the subtraction of $\avg{{\kappa}{\kappa}^2_{{\rm N}}}$ terms, makes the analysis robust against source clustering effects (neglecting source clustering effects produces only a $0.08 \sigma$ shift in the marginalised two dimensional posterior of $\Omega_{\rm m}$ and $S_8$). 


\minorsection{Peaks}
The fifth row of Fig. \ref{fig:sc} shows the impact of source clustering on the peak count function. We show the measurements only for the smoothing scale $\theta_0 = 13.2$ arcmin, intermediate among the several smoothing scales included in the DES Y3 peaks analysis in \cite{Zuercher2021b}; the trend with scales (not shown here) and redshift is similar to the moments case, i.e. the difference between the two sets of simulations increases with smaller smoothing scales and when low redshift bins are considered. Noise and signal are non-trivially mixed together due to the strong non-linearity of the peak function, and so, unlike the moments case, we did not try to create a procedure to minimise the impact of source clustering, nor did we try to single out the effects due to the extra noise-signal correlations. We found that for peaks statistic the effect is less striking than the moments case. We verified that for the scales considered in the analysis by \cite{Zuercher2021}, i.e. $[7.9, 31.6]$ arcmin, the difference between two simulated data vectors with and without source clustering is small enough to not bias the cosmological inference (neglecting source clustering effects produces only a $0.18 \sigma$ shift in the marginalised two dimensional posterior of $\Omega_{\rm m}$ and $S_8$).
%\footnote{The difference between the two data vectors } 
%not statistically significant (reduced $\chi^2 = 0.18$, corresponding to a $p$-value = 1).

%\VA{We compute the $\chi^2$ for a data vector given by the difference between the simulations including source clustering and the simulations without. We find a negligible impact of this effect for the scales considered in \cite{Zuercher2021}, in the range $[10.5, 31.6]$ arcmin for all tomographic bins. Specifically, we find a reduced $\chi^2 = 0.18$, corresponding to a p-value = 1 and no significant detection. If considered for separated smoothing scales we notice that the reduced $\chi^2$ increases for decreasing smoothing with a value of $\sim 0.24$ for the smallest scale included in the analysis.}

\minorsection{WPH}
The last row of Fig. \ref{fig:sc} shows the WPH statistics obtained using one of the noisy convergence maps and one of the noise-only maps. We do not show the harmonics obtained using only the noisy convergence maps because a cosmological analysis using the measurements is currently underway (Gatti et al., in prep.); {since the measurements are blinded, we cannot compare them to simulations.} These statistics are consistent with zero in the absence of source clustering; however, {we detect a clear signal in data due to noise-signal correlations}, and this is well reproduced by the simulations with source clustering.

\minorsection{Significance}
Using the moments and the WPH coefficients, we can construct two null-tests for source clustering. {The $C01$ coefficients of the WPH statistics of noisy convergence maps and noise only maps are expected to be zero in the absence of source clustering (consistent with the simulation without source clustering). Using this null-test for the bins combination (3,1), we find a $p$-value for our observed $\chi^2$ of {$p=3\times10^{-11}$, which corresponds to 6.5 $\sigma$ significance.}  This result assumes a mean-zero Gaussian likelihood with covariance matrix $\Sigma$ estimated from simulations with no source clustering}, where $\chi^2 = d^\textrm{T} \Sigma^{-1} d$ with measured observable vector $d$. The same null-test for the third moment $\avg{{\kappa}{\kappa}^2_{{\rm N}}}$ for the bins combination (3,1,1) yields $p=4\times10^{-3}$ (2.6 $\sigma$). No trivial null-test can be constructed with the peaks statistics. %\VA{is this true in the end? we did all those $\chi^2$ tests at some points but at the time we were using the depth map SC simulations, so we were finding a huge shift probably because of those simulations. Maybe we should remove this sentence and say that for the peaks we tested the significance at the Fisher contour level instead?}.

Finally, we note that the magnitude of the source clustering effect also depends on the clustering properties of the source sample (e.g. the source galaxy-matter bias, Fig \ref{fig:sc_2}), which should be marginalised over when analysing map-based weak lensing higher-order statistics.
\section{Discussion and Conclusion}

We have demonstrated the impact of source galaxy clustering on map-based higher-order summary statistics of weak gravitational lensing observables. Source clustering affects the mean shear field estimated from galaxy catalogues, as the noise-free lensing signal is modulated by a different effective redshift distribution; moreover, it induces a strong correlation between a pixel's shear signal and its noise properties. The latter effect is the dominant one in map-based higher-order statistics. Using simulations with galaxies that either trace or do not trace the underlying density field, we show that the effect induced in the signals of common higher-order statistics can exceed the current measurement uncertainties, depending on the choice of scale cut and choice of summary statistic redshift range. In particular, we find that third moments and wavelet phase harmonic coefficients are the most affected ones, whereas peak count statistics are less affected. Generally, source clustering effects are larger at small scales and for statistics applied to combinations of low and high redshift samples, and diminish at high redshift.

Further, we have shown a clear source clustering feature using Dark Energy Survey Year 3 data.  Due to the induced correlation between the shear signal and the noise properties of the maps, third moments combining the noisy convergence maps and `noise-only' maps no longer vanish. We detected a similar feature at high statistical significance for wavelet phase harmonics. Mocks with source clustering were well able to reproduce these features; mocks without source clustering provided a poor fit to the data ($p$-values of 4e-3 for third moments and 3e-11 for wavelet phase harmonics).

{Cosmological analyses using map-based higher-order statistics have two strategies for dealing with source clustering: either minimise its effect by introducing \textit{ad-hoc} scale cuts and/or de-noising procedures, or fully forward model it, incorporating it into simulations. This work presents a recipe for efficiently incorporating source clustering effects into simulations, and also shows how to minimise the impact of source clustering for third moments using a de-noising procedure. If left unaccounted for, or if not tested, this effect could impact cosmological inference made with statistics using weak gravitational lensing observables, especially map-based higher-order statistics (including ones not considered here, e.g. scattering transforms, deep learning summary statistics, Minkowski functionals, etc.). In the case of the DES Y3 higher-order statistics analyses -- moments \citep{moments2021} and peaks \citep{Zuercher2022} -- we verified that the scale cuts and de-noising procedures adopted were sufficient to render this effect negligible. }

{Other effects could cause noise-signal correlations in map-based estimators, e.g. any selection effect depending on the local value of the matter and shear fields modulating the source number density. Source magnification induces an extra modulation proportional to $1 + \kappa(p,s)$, however our tests shows this to be negligible (owing to a lower signal amplitude compared to the density field). Blending effects are also likely negligible, as they are expected to affect only a small fraction of the sample. In general, any deviation from the simple $1+b_g \delta(p,s)$ modulation considered here would lead to a specific redshift evolution and/or amplitude signature in the measurements, and we do not see this. Other astrophysical effects such as intrinsic alignment and baryonic feedback can impact $\gamma(s,p)$ and $\delta(s,p)$, but they do not directly modulate the number of galaxies. They could, however, enhance the source clustering effects: intrinsic alignment, in particular, is a local effect modulated by the same density fluctuations that modulate the source clustering \citep{Blazek2019}, and hence it could boost the amplitude of the noise-signal correlations.}


{This work looked primarily at map-based statistics. Source clustering is expected to affect catalogue-based statistics (such as three-point correlation functions) differently: there should be no noise-signal contributions (as these are due to averaging the shear in pixels before estimating the summary statistics), but sources would still be preferentially sampled in regions with high shear/convergence. The total effect is thus expected to be smaller; we leave this investigation to future works.}



%\textbf{mg: mention that the effect can be mitigated with scale cuts/ mixed noise subtractions}
%This work presents a recipe for efficiently incorporating source clustering effects into simulations. If left unaccounted for, this effect could significantly impact cosmological inference made with higher-order statistics using weak gravitational lensing. Affected observables are not limited to the ones considered in this work; any other higher-order statistic relying on weak lensing mass maps (e.g. scattering transforms, deep learning summary statistics, Minkowski functionals, etc.) are potentially affected.




%%%%%%%%%%%%%%%%%%%% REFERENCES %%%%%%%%%%%%%%%%%%



% The best way to enter references is to use BibTeX:

%\bibliographystyle{mnras}		  %SWITCH TO MNRAS
\bibliography{bibliography,des_y3kp}
\bibliographystyle{mn2e_2author_arxiv_amp.bst}
%\bibliographystyle{mn2e_2author_amp.bst} % LW: with mn2e_2author_amp.bst we don't get the a:rXiv problem.


% \bsp
%%%%%%%%%%%%%%%%%%%%%%%%%%%%%%%%%%%%%%%%%%%%%%%%%%

%%%%%%%%%%%%%%%%% APPENDICES %%%%%%%%%%%%%%%%%%%%%
\newpage
\appendix
\appendices
\section{The Proof of Proposition \ref{prop2}}
\label{appa}
For the jointly Gaussian random vectors $\bm{x}$ and $\bm{y}$, we have
\begin{equation}
\begin{aligned}
&    \left[\begin{matrix}\bm{x}\\\bm{y}\\\end{matrix}\right] \sim \mathcal{N}\left(\left[\begin{matrix}\bm{\mu}_x\\\bm{\mu}_y\\\end{matrix}\right],\left[\begin{matrix}A&C\\C^T&B\\\end{matrix}\right]\right) \\
& = \mathcal{N}\left(\left[\begin{matrix}\bm{\mu}_x\\\bm{\mu}_y\\\end{matrix}\right],\left[\begin{matrix}\widetilde{A}&\widetilde{C}\\{\widetilde{C}}^T&B\\\end{matrix}\right]^{-1}\right)
\end{aligned}
\end{equation}
then the marginal and conditional distribution of $\bm{x}$ are shown as follows according to \cite{williams2006gaussian}.
\begin{equation}
    \bm{x} \sim \mathcal{N}\left(\bm{\mu}_x,A\right)
\end{equation}
% and
\begin{equation}
\label{app2-1}
    \bm{x}|\bm{y} \sim \mathcal{N}\left(\bm{\mu}_x+CB^{-1}\left(\bm{y}-\bm{\mu}_y\right),A-CB^{-1}C^T\right)
\end{equation}
% or
\begin{equation}
\label{app2-2}
    \bm{x}|\bm{y} \sim \mathcal{N}\left(\bm{\mu}_x-{\widetilde{A}}^{-1}\widetilde{C}\left(\bm{y}-\bm{\mu}_y\right),{\widetilde{A}}^{-1}\right)
\end{equation}

Thus, \textbf{Proposition \ref{prop2}} is proved.










\section{The Proof of Proposition \ref{prop3}}
\label{appb}
The product of two Gaussian distributions is represented as
\begin{equation}
\mathcal{N}\left(\bm{x}\middle|\bm{a},A\right)\mathcal{N}\left(\bm{x}\middle|\bm{b},B\right)=Z^{-1}\mathcal{N}\left(\bm{x}\middle|\bm{c},C\right)
\end{equation}
where
\begin{equation}
\label{app4}
    \bm{c}=C\left(A^{-1}\bm{a}+B^{-1}\bm{b}\right)
\end{equation}
\begin{equation}
\label{app5}
    C=\left(A^{-1}+B^{-1}\right)^{-1}
\end{equation}
\begin{equation}
\label{app6}
    Z^{-1}=\left(2\pi\right)^{-\frac{D}{2}}\left|A+B\right|^{-\frac{1}{2}}\exp{\left(-\frac{\left(\bm{a}-\bm{b}\right)^T\left(\bm{a}-\bm{b}\right)}{2\left(A+B\right)}\right)}
\end{equation}

Thus, through multiplying the cavity distribution by $t_i$ from (\ref{11}), \textbf{Proposition \ref{prop3}} is proved.


\section{The Proof of Proposition \ref{prop4}}
\label{appc}
Consider
\begin{equation}
\label{app7}
Z=\int_{-\infty}^{\infty}{\Phi\left(\frac{x-m}{v}\right)\mathcal{N}(x|\mu,\sigma^2)dx}
\end{equation}
% where
% \begin{equation}
%     \Phi\left(x\right)=\int_{-\infty}^{x}{\mathcal{N}\left(y\right)dy}
% \end{equation}
When $v>0$, by combining$ z=y-x+\mu-m$ and $w=x-\mu$ we can get
\begin{equation}
\begin{aligned}
& Z_{v>0}=\frac{\int_{-\infty}^{\infty}\int_{-\infty}^{x}\exp{\left(-\frac{\left(y-m\right)^2}{2v^2}-\frac{\left(x-\mu\right)^2}{2\sigma^2}\right)}}{2\pi\sigma v}dydx \\
& =\frac{\int_{-\infty}^{\mu-m}\int_{-\infty}^{\infty}\exp{\left(-\frac{\left(z+w\right)^2}{2v^2}-\frac{w^2}{2\sigma^2}\right)}}{2\pi\sigma v}dwdz
\end{aligned}
\end{equation}
% and
\begin{equation}
\begin{aligned}
& Z_{v>0} \\
& =\frac{\int_{-\infty}^{\mu-m}\int_{-\infty}^{\infty}\exp{\left(-\frac{1}{2}\left[\begin{matrix}w\\z\\\end{matrix}\right]^T\left[\begin{matrix}\frac{1}{v^2}+\frac{1}{\sigma^2}&\frac{1}{v^2}\\\frac{1}{v^2}&\frac{1}{v^2}\\\end{matrix}\right]\left[\begin{matrix}w\\z\\\end{matrix}\right]\right)}}{2\pi\sigma v}dwdz \\
& =\int_{-\infty}^{\mu-m}\int_{-\infty}^{\infty}\mathcal{N}\left(\left[\begin{matrix}w\\z\\\end{matrix}\right]|\mathbf{0},\left[\begin{matrix}\sigma^2&-\sigma^2\\-\sigma^2&v^2+\sigma^2\\\end{matrix}\right]\right)dwdz
\end{aligned}
\end{equation}
According to (\ref{app2-1}) and (\ref{app2-2}), we can get
\begin{equation}
\label{app11}
    Z_{v>0}=\frac{\int_{-\infty}^{\mu-m}\exp{\left(-\frac{z^2}{2\left(v^2+\sigma^2\right)}\right)}dz}{\sqrt{2\pi(v^2+\sigma^2)}}=\Phi\left(\frac{\mu-m}{\sqrt{v^2+\sigma^2}}\right)
\end{equation}
When $v<0$, by combining $\Phi\left(-z\right)=1-\Phi\left(z\right)$ and (\ref{app7}),
% we can obtain
\begin{equation}
\label{app12}
Z_{v<0}=1-\Phi\left(\frac{\mu-m}{\sqrt{v^2+\sigma^2}}\right)=\Phi\left(-\frac{\mu-m}{\sqrt{v^2+\sigma^2}}\right)
\end{equation}

By collecting (\ref{app11}) and (\ref{app12}), we can get
\begin{equation}
\label{app13}
Z=\int\Phi\left(\frac{x-m}{v}\right)\mathcal{N}\left(x\middle|\mu,\sigma^2\right)dx=\Phi\left(z\right)
\end{equation}
where $z=\frac{\mu-m}{v\sqrt{1+\sigma^2/v^2}} (v\neq0)$. 
% We aim to get the moments of
% \begin{equation}
% q\left(x\right)=Z^{-1}\Phi\left(\frac{x-m}{v}\right)\mathcal{N}\left(x\middle|\mu,\sigma^2\right)
% \end{equation}
By differentiating with respect to $\mu$ on (\ref{app13}), we can obtain
\begin{equation}
\begin{aligned}
& \frac{\partial Z}{\partial\mu}=\int{\frac{x-\mu}{\sigma^2}\Phi\left(\frac{x-m}{v}\right)}\mathcal{N}\left(x\middle|\mu,\sigma^2\right)dx =\frac{\partial}{\partial\mu}\Phi\left(z\right) \\
& \Longleftrightarrow \frac{1}{\sigma^2}\int x\Phi\left(\frac{x-m}{v}\right)\mathcal{N}\left(x\middle|\mu,\sigma^2\right)dx-\frac{\mu Z}{\sigma^2} \\
& =\frac{\mathcal{N}(z)}{v\sqrt{1+\sigma^2/v^2}}
\end{aligned}
\end{equation}
where $\partial\Phi\left(z\right)/\partial\mu=\mathcal{N}(z)\partial z/\partial\mu$ is utilized. Multiplying through by $\sigma^2/Z$, (\ref{app16}) is obtained.
\begin{equation}
\label{app16}
\mathbb{E}_q\left[x\right]=\mu+\frac{\sigma^2\mathcal{N}\left(z\right)}{\Phi\left(z\right)v\sqrt{1+\frac{\sigma^2}{v^2}}}
\end{equation}
Similarly, we can obtain the second moment as
\begin{equation}
\label{app17}
\begin{aligned}
 & \frac{\partial^2Z}{\partial\mu^2} \\
 & =\int{[\frac{x^2}{\sigma^4}-\frac{2\mu x}{\sigma^4}+\frac{\mu^2}{\sigma^4}-\frac{1}{\sigma^2}] \Phi\left(\frac{x-m}{v}\right)\mathcal{N}\left(x\middle|\mu,\sigma^2\right)} dx  \\
 & =-\frac{z\mathcal{N}(z)}{v^2+\sigma^2} \Longleftrightarrow \\
 & \mathbb{E}_q\left[x^2\right]=2\mu\mathbb{E}_q\left[x\right]-\mu^2+\sigma^2-\frac{\sigma^4z\mathcal{N}\left(z\right)}{\Phi\left(z\right)\left(v^2+\sigma^2\right)}
\end{aligned}
\end{equation}
By combining (\ref{app16}) and (\ref{app17}), we can get
\begin{equation}
\begin{aligned}
& \mathbb{E}_q\left[{(x-\mathbb{E}_q\left[x\right])}^2\right]=\mathbb{E}_q\left[x^2\right]-\mathbb{E}_q[x]^2 \\
& =\sigma^2-\frac{\sigma^4\mathcal{N}\left(z\right)}{\left(v^2+\sigma^2\right)\Phi\left(z\right)}\left(z+\frac{\mathcal{N}\left(z\right)}{\Phi\left(z\right)}\right)
\end{aligned}
\end{equation}

Thus, \textbf{Proposition \ref{prop4}} is proved.

\section{The Proof of Proposition \ref{prop5}}
\label{appd}
We can obtain (\ref{19-1}), (\ref{19-2}), and (\ref{19-3}) according to (\ref{app4}), (\ref{app5}), and (\ref{app6}). Hence, \textbf{Proposition \ref{prop5}} is proved.



\section{The Proof of Proposition \ref{prop6}}
\label{appe}
The approximated mean for $f_\ast$ can be denoted as
\begin{equation}
\begin{aligned}
& \mathbb{E}_q\left[f_\ast|X,\bm{y},\bm{x}_\ast\right]=\bm{k}_\ast^TK^{-1}\bm{\mu} \\
& =\bm{k}_\ast^TK^{-1}\left(K^{-1}+{\widetilde{\Sigma}}^{-1}\right)^{-1}{\widetilde{\Sigma}}^{-1}\widetilde{\bm{\mu}} \\
& =\bm{k}_\ast^T\left(K+\widetilde{\Sigma}\right)^{-1}\widetilde{\bm{\mu}}
\end{aligned}
\end{equation}

The variance of $f_\ast|(X,\bm{y})$ under the Gaussian approximation can be denoted as
\begin{equation}
\begin{aligned}
& \mathbb{V}_q\left[f_\ast\middle| X,\bm{y},\bm{x}_\ast\right] = \mathbb{E}_{p(f_\ast|X,\bm{x}_\ast,\bm{f})} {f_\ast-\mathbb{E}[f_\ast|X,\bm{x}_\ast,\bm{f}]}^2 \\
& =k\left(\bm{x}_\ast,\bm{x}_\ast\right)-\bm{k}_\ast^TK^{-1}\bm{k}_\ast+\bm{k}_\ast^TK^{-1}\left(K^{-1}+\widetilde{\Sigma}\right)^{-1}K^{-1}\bm{k}_\ast \\
& =k\left(\bm{x}_\ast,\bm{x}_\ast\right)-\bm{k}_\ast^T\left(K^{-1}+\widetilde{\Sigma}\right)^{-1}\bm{k}_\ast
\end{aligned}
\end{equation}

Then, we can obtain
\begin{equation}
\begin{aligned}
& q\left(y_\ast\middle| X,\bm{y},\bm{x}_\ast\right)=\mathbb{E}_q\left[\pi_\ast|X,\bm{y},\bm{x}_\ast\right] \\
& =\int\Phi\left(f_\ast\right)q\left(f_\ast\middle| X,\bm{y},\bm{x}_\ast\right)df_\ast
\end{aligned}
\end{equation}

According to (\ref{app11}), we can obtain
\begin{equation}
\label{app22}
\begin{aligned}
& q\left(y_\ast\middle| X,\bm{y},\bm{x}_\ast\right) \\
& =\Phi\left(\frac{\bm{k}_\ast^T\left(K+\widetilde{\Sigma}\right)^{-1}\widetilde{\bm{\mu}}}{\sqrt{1+k\left(\bm{x}_\ast,\bm{x}_\ast\right)-\bm{k}_\ast^T\left(K+\widetilde{\Sigma}\right)^{-1}\bm{k}_\ast}}\right)
\end{aligned}
\end{equation}

By combining (\ref{13}) and (\ref{app22}), \textbf{Proposition \ref{prop6}} is proved.




\section{The Proof of Proposition \ref{prop7}}
\label{appf}
Given $f_s$ and $f_\ast$, $y_s$ and $y_\ast$ are conditionally independent. Hence, $p\left(y_s,y_\ast\middle|\bm{x}_s,\bm{x}_\ast\right)$ can be represented as
\begin{equation}
\begin{aligned}
& p\left(y_s=1,y_\ast=1\middle|\bm{x}_s,\bm{x}_\ast\right) \\
& =\iint{\Phi\left(f_s\right)\Phi\left(f_\ast\right)\phi\left(f_s,f_\ast\middle|\mu_{s\ast},\Sigma_{s\ast}\right)}df_sdf_\ast \\
& =\iint{\Phi\left(f_\ast\right)\phi\left(f_\ast\middle|{\widetilde{\mu}}_\ast\left(f_s\right),{\widetilde{\sigma}}_{\ast\ast}\right)df_\ast\Phi\left(f_s\right)}\phi\left(f_s\middle|\mu_s,\sigma_{ss}\right)df_s \\
& =\int\Phi\left(\frac{{\widetilde{\mu}}_\ast\left(f_s\right)}{\sqrt{{\widetilde{\sigma}}_{\ast\ast}+1}}\right)\Phi\left(f_s\right)\phi\left(f_s\middle|\mu_s,\sigma_{ss}\right)df_s
\end{aligned}
\end{equation}

Hence, \textbf{Proposition \ref{prop7}} is proved.

% \section{The Proof of Lemma \ref{lem}}
% \label{appg}
% \begin{equation}
% \begin{aligned}
% & R_e=\frac{1}{N_a}\sum_{n=1}^{N_a}\mathbb{I}\left(\bm{L}_n \neq \bm{Y}_n\right) \\
% & =\displaystyle\frac{FA+FL}{TL+TA+FL+FA} \\
% & =\displaystyle\frac{1}{\displaystyle\frac{TL+TA+FL+FA}{FA+FL}} \\
% & =\displaystyle\frac{1}{1+\displaystyle\frac{TL+TA}{FA+FL}} \\
% & =\displaystyle\frac{1}{1+\displaystyle\frac{\displaystyle\frac{TL}{TA}+1}{\displaystyle\frac{FA}{TA}+\displaystyle\frac{FL}{TA}}} \\
% & =\frac{1}{1+\displaystyle\frac{\displaystyle\frac{TL}{TA}+1}{\displaystyle\frac{1}{P_{md}-1}+\displaystyle\frac{1}{P_{fa}-1}}}
% \end{aligned}
% \end{equation}

% Hence, \textbf{Lemma \ref{lem}} is proved.
%\section*{Affiliations}
\section{Affiliations}\label{sec:affiliations}
{$^{\star}$ E-mail: marcogatti29@gmail.com}\\
$^{1}$ Department of Physics and Astronomy, University of Pennsylvania, Philadelphia, PA 19104, USA\\
$^{2}$ Department of Physics \& Astronomy, University College London, Gower Street, London, WC1E 6BT, UK\\
$^{3}$ Department of Physics, ETH Zurich, Wolfgang-Pauli-Strasse 16, CH-8093 Zurich, Switzerland \\
$^{4}$ Department of Astronomy and Astrophysics, University of Chicago, Chicago, IL 60637, USA \\
$^{5}$ Kavli Institute for Cosmological Physics, University of Chicago, Chicago, IL 60637, USA \\
$^{6}$ Department of Physics, Northeastern University, Boston, MA 02115, USA \\
$^{7}$  Department of Astronomy/Steward Observatory, University of Arizona, 933 North Cherry Avenue, Tucson, AZ 85721-0065, USA \\
$^{8}$  Argonne National Laboratory, 9700 South Cass Avenue, Lemont, IL 60439, USA\\
$^{9}$  Institute of Astronomy, University of Cambridge, Madingley Road, Cambridge CB3 0HA, UK\\
$^{10}$ Kavli Institute for Cosmology, University of Cambridge, Madingley Road, Cambridge CB3 0HA, UK\\
$^{11}$ Physics Department, 2320 Chamberlin Hall, University of Wisconsin-Madison, 1150 University Avenue Madison, WI  53706-1390\\
$^{12}$ Department of Physics, Carnegie Mellon University, Pittsburgh, Pennsylvania 15312, USA \\
$^{13}$ Department of Physics, Duke University Durham, NC 27708, USA\\
$^{14}$  NASA Goddard Space Flight Center, 8800 Greenbelt Rd, Greenbelt, MD 20771, USA\\
$^{15}$ Kavli Institute for Particle Astrophysics \& Cosmology, P. O. Box 2450, Stanford University, Stanford, CA 94305, USA\\
$^{16}$ Lawrence Berkeley National Laboratory, 1 Cyclotron Road, Berkeley, CA 94720, USA\\
$^{17}$ Fermi National Accelerator Laboratory, P. O. Box 500, Batavia, IL 60510, USA \\
$^{18}$ NSF AI Planning Institute for Physics of the Future, Carnegie Mellon University, Pittsburgh, PA 15213, USA\\
$^{19}$ Universit\'e Grenoble Alpes, CNRS, LPSC-IN2P3, 38000 Grenoble, France\\
$^{20}$ Department of Physics and Astronomy, University of Waterloo, 200 University Ave W, Waterloo, ON N2L 3G1, Canada\\
$^{21}$ Jet Propulsion Laboratory, California Institute of Technology, 4800 Oak Grove Dr., Pasadena, CA 91109, USA\\
$^{22}$ SLAC National Accelerator Laboratory, Menlo Park, CA 94025, USA\\
$^{23}$ University Observatory, Faculty of Physics, Ludwig-Maximilians-Universit\"at, Scheinerstr. 1, 81679 Munich, Germany\\
$^{24}$ Center for Astrophysical Surveys, National Center for Supercomputing Applications, 1205 West Clark St., Urbana, IL 61801, USA\\
$^{25}$ Department of Astronomy, University of Illinois at Urbana-Champaign, 1002 W. Green Street, Urbana, IL 61801, USA\\
$^{26}$ School of Physics and Astronomy, Cardiff University, CF24 3AA, UK\\
$^{27}$ Department of Astronomy, University of Geneva, ch. d'\'Ecogia 16, CH-1290 Versoix, Switzerland \\
$^{28}$ Department of Applied Mathematics and Theoretical Physics, University of Cambridge, Cambridge CB3 0WA, UK \\
$^{29}$ Department of Physics, Stanford University, 382 Via Pueblo Mall, Stanford, CA 94305, USA \\
$^{30}$ Instituto de F\'isica Gleb Wataghin, Universidade Estadual de Campinas, 13083-859, Campinas, SP, Brazil\\
$^{31}$ Department of Physics, University of Genova and INFN, Via Dodecaneso 33, 16146, Genova, Italy\\
$^{32}$ Jodrell Bank Center for Astrophysics, School of Physics and Astronomy, University of Manchester, Oxford Road, Manchester, M13 9PL, UK\\
$^{33}$ Centro de Investigaciones Energ\'eticas, Medioambientales y Tecnol\'ogicas (CIEMAT), Madrid, Spain\\
$^{34}$ Brookhaven National Laboratory, Bldg 510, Upton, NY 11973, USA\\
$^{35}$  Department of Physics and Astronomy, Stony Brook University, Stony Brook, NY 11794, USA\\
$^{36}$ Department of Physics, Duke University Durham, NC 27708, USA\\
$^{37}$  Institut de Recherche en Astrophysique et Plan\'etologie (IRAP), Universit\'e de Toulouse, CNRS, UPS, CNES, 14 Av. Edouard Belin, 31400 Toulouse, France\\
$^{38}$ Institut d'Estudis Espacials de Catalunya (IEEC), 08034 Barcelona, Spain\\
$^{39}$ Institute of Space Sciences (ICE, CSIC),  Campus UAB, Carrer de Can Magrans, s/n,  08193 Barcelona, Spain\\
$^{40}$ Excellence Cluster Origins, Boltzmannstr.\ 2, 85748 Garching, Germany\\
$^{41}$ Max Planck Institute for Extraterrestrial Physics, Giessenbachstrasse, 85748 Garching, Germany\\
$^{42}$ Universit\"ats-Sternwarte, Fakult\"at f\"ur Physik, Ludwig-Maximilians Universit\"at M\"unchen, Scheinerstr. 1, 81679 M\"unchen, Germany\\
$^{43}$ Cerro Tololo Inter-American Observatory, NSF's National Optical-Infrared Astronomy Research Laboratory, Casilla 603, La Serena, Chile\\
$^{44}$ Department of Astronomy, University of Michigan, Ann Arbor, MI 48109, USA\\
$^{45}$ Institute for Astronomy, University of Edinburgh, Edinburgh EH9 3HJ, UK\\
$^{46}$ Instituto de Astrofisica de Canarias, E-38205 La Laguna, Tenerife, Spain\\
$^{47}$ Laborat\'orio Interinstitucional de e-Astronomia - LIneA, Rua Gal. Jos\'e Cristino 77, Rio de Janeiro, RJ - 20921-400, Brazil\\
$^{48}$ Universidad de La Laguna, Dpto. Astrofísica, E-38206 La Laguna, Tenerife, Spain\\
$^{49}$ Institute of Cosmology and Gravitation, University of Portsmouth, Portsmouth, PO1 3FX, UK\\
$^{50}$ Center for Astrophysics $\vert$ Harvard \& Smithsonian, 60 Garden Street, Cambridge, MA 02138, USA\\
$^{51}$ Santa Cruz Institute for Particle Physics, Santa Cruz, CA 95064, USA\\
$^{52}$ CNRS, UMR 7095, Institut d'Astrophysique de Paris, F-75014, Paris, France\\
$^{53}$ Sorbonne Universit\'es, UPMC Univ Paris 06, UMR 7095, Institut d'Astrophysique de Paris, F-75014, Paris, France\\
$^{54}$ Computer Science and Mathematics Division, Oak Ridge National Laboratory, Oak Ridge, TN 37831\\
$^{55}$ Department of Physics, University of Michigan, Ann Arbor, MI 48109, USA\\
$^{56}$ Institute of Theoretical Astrophysics, University of Oslo. P.O. Box 1029 Blindern, NO-0315 Oslo, Norway\\
$^{57}$ George P. and Cynthia Woods Mitchell Institute for Fundamental Physics and Astronomy, and Department of Physics and Astronomy, Texas A\&M University, College Station, TX 77843,  USA\\
$^{58}$ Institut de F\'{\i}sica d'Altes Energies (IFAE), The Barcelona Institute of Science and Technology, Campus UAB, 08193 Bellaterra (Barcelona) Spain\\
$^{59}$ Instituto de Fisica Teorica UAM/CSIC, Universidad Autonoma de Madrid, 28049 Madrid, Spain\\
$^{60}$ Center for Cosmology and Astro-Particle Physics, The Ohio State University, Columbus, OH 43210, USA\\
$^{61}$ Department of Physics, The Ohio State University, Columbus, OH 43210, USA\\
$^{62}$ Australian Astronomical Optics, Macquarie University, North Ryde, NSW 2113, Australia\\
$^{63}$ Lowell Observatory, 1400 Mars Hill Rd, Flagstaff, AZ 86001, USA\\
$^{64}$ Hamburger Sternwarte, Universit\"{a}t Hamburg, Gojenbergsweg 112, 21029 Hamburg, Germany\\
$^{65}$ School of Physics and Astronomy, University of Southampton,  Southampton, SO17 1BJ, UK\\
$^{66}$ Instituci\'o Catalana de Recerca i Estudis Avan\c{c}ats, E-08010 Barcelona, Spain\\
$^{67}$ Observat\'orio Nacional, Rua Gal. Jos\'e Cristino 77, Rio de Janeiro, RJ - 20921-400, Brazil\\
$^{68}$ Physics Department, William Jewell College, Liberty, MO, 64068\\
$^{69}$ School of Mathematics and Physics, University of Queensland,  Brisbane, QLD 4072, Australia\\
$^{70}$ Department of Physics, IIT Hyderabad, Kandi, Telangana 502285, India\\
$^{71}$ School of Mathematics and Physics, University of Queensland,  Brisbane, QLD 4072, Australia\\


%%%%%%%%%%%%%%%%%%%%%%%%%%%%%%%%%%%%%%%%%%%%%%%%%%


% Don't change these lines
% \bsp	% typesetting comment
\label{lastpage}
\end{document}

% Don't change these lines
\end{document}

