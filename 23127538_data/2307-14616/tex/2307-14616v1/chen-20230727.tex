\documentclass[aps,prx,twocolumn,superscriptaddress,nofootinbib,showpacs]{revtex4-2}
%\documentclass[
%    aps,
%    twocolumn,
%    superscriptaddress
%]{revtex4-2}
\usepackage{amsmath}
\usepackage{amssymb}
\usepackage{graphicx}
%\usepackage{epsfig}
\usepackage{xcolor,color}
\usepackage{url}
\usepackage{bm}
\usepackage{mathrsfs}
\usepackage[utf8]{inputenc}
\usepackage{hyperref}
\usepackage{enumerate}
\usepackage{amsthm}
\usepackage{bbm}
\usepackage[normalem]{ulem}
\usepackage{upgreek}
\usepackage{tensor}
\usepackage{siunitx}
\usepackage{orcidlink}
\usepackage[caption=false]{subfig}
\usepackage{float}
\usepackage{colortbl}

%\usepackage{ctex}
%\usepackage{media9}

%------------------------------------------------
%	Colour Definitions
%------------------------------------------------

\definecolor{colour1}{HTML}{0571b0} %--- Blue
\definecolor{colour2}{HTML}{92c5de} %--- Cyan
\definecolor{colour3}{HTML}{f4a582} %--- Orange
\definecolor{colour4}{HTML}{ca0020} %--- Maroon
\definecolor{colour5}{HTML}{fe4a49} %--- Red
\definecolor{colour6}{HTML}{2d3092} %--- Red
% \definecolor{ColorName}{RGB}{R,G,B}

%------------------------------------------------
%	Further Preamble
%------------------------------------------------

\hypersetup{colorlinks=true, linkcolor=colour6, citecolor=colour6,
filecolor=colour6, urlcolor=colour6}

% Scri
%\def\scri{\mathscr{I}}
%\makeatletter
%\newcommand{\subalign}[1]{%
%  \vcenter{%
%    \Let@ \restore@math@cr \default@tag
%    \baselineskip\fontdimen10 \scriptfont\tw@
%    \advance\baselineskip\fontdimen12 \scriptfont\tw@
%    \lineskip\thr@@\fontdimen8 \scriptfont\thr@@
%    \lineskiplimit\lineskip
%    \ialign{\hfil$\m@th\scriptstyle##$&$\m@th\scriptstyle{}##$\hfil\crcr
%      #1\crcr
%    }%
%  }%
%}
%\makeatother


\theoremstyle{plain}
\newtheorem{proposition}{Proposition}
\newtheorem{lemma}{Lemma}
\newtheorem{theorem}{Theorem}
\newtheorem{assumption}{Assumption}
\newtheorem*{conjecture}{Conjecture}
\newtheorem*{subconjecture}{Subconjecture}
\newtheorem{corollary}{Corollary}
\newtheorem{main}{Main Result}
\newtheorem*{definition}{Definition}
\newtheorem{remark}{Remark}


%
\usepackage{color,xcolor}
\usepackage{ulem}
\usepackage{float}
\usepackage{subfig}
\usepackage{cleveref}
\usepackage{multirow}
\usepackage{booktabs}
\usepackage{booktabs}
\usepackage{threeparttable}
% \geometry{a4paper,scale=0.8}

\usepackage[export]{adjustbox}
\usepackage{times}
\usepackage{commath}



\crefname{equation}{Eq.}{Eq.}
\Crefname{equation}{Eqs.}{Eqs.}

\Crefname{section}{Sec.}{Sec.}

\Crefname{figure}{Fig.}{Fig.}


%%
%\newcolumntype{C}{>{$}c<{$}}
%
%\AtBeginDocument{%
%  \heavyrulewidth=.08em
%  \lightrulewidth=.05em
%  \cmidrulewidth=.03em
%  \belowrulesep=.65ex
%  \belowbottomsep=0pt
%  \aboverulesep=.4ex
%  \abovetopsep=0pt
%  \cmidrulesep=\doublerulesep
%  \cmidrulekern=.5em
%  \defaultaddspace=.5em
%}
%%
\usepackage{threeparttable}



%%document
\begin{document}
%\pagewiselinenumbers% 按页重新编号
%\switchlinenumbers

\title{Exact solutions of general Teukolsky equations with source and their applications}

% \title{New Analytic Black Hole Perturbation Approach}

%\title{New Analytic Black Hole Perturbation Approach: Exact solutions of general Teukolsky equations with source}

\author{Changkai Chen\,\orcidlink{0000-0002-4023-0682},}
\affiliation{Department of Physics, Key Laboratory of Low Dimensional Quantum Structures and Quantum Control of Ministry of Education, Synergetic Innovation Center for Quantum Effects and Applications, Hunan Normal University, Changsha, 410081, Hunan, China}


\author{{Jiliang} {Jing}\,\orcidlink{0000-0002-2803-7900}, } %\href{https://scholar.google.co.kr/citations?hl=zh-CN&user=LKZTN0AAAAAJ}{\aiGoogleScholar},\href{https://www.scopus.com/authid/detail.uri?authorId=7101759560}{\aiScopus},}
\email[Corresponding author: ]{jljing@hunnu.edu.cn}
\affiliation{Department of Physics, Key Laboratory of Low Dimensional Quantum Structures and Quantum Control of Ministry of Education, Synergetic Innovation Center for Quantum Effects and Applications, Hunan Normal University, Changsha, 410081, Hunan, China}


\date{\today}
\begin{abstract}
Teukolsky equations can describe many physical processes in black-hole spacetime. This paper presents a novel method for solving general Teukolsky equations (GTEs) which contains six kinds (Schwarzschild, Reissner-N{\"{o}}rdstrom, Kerr, Kerr-Newman, Kerr-Sen, and high-dimensional Schwarzschild) of black holes and nine kinds of Teukolsky equations. Firstly, the exact general solution of homogeneous GTEs is constituted by the linear combination of two linearly independent solutions which are expressed by confluent Heun functions. The exact solution, benefiting from the analytical asymptotic expression of the confluent Heun function at infinity which is first obtained in this paper, can be applied to the physical models corresponding to various boundary conditions. Meanwhile, ingoing and outgoing wave solutions can directly be obtained from the exact general solution according to boundary conditions. Consequently, the exact solution of inhomogeneous GTEs is found utilizing Green's function.  It is interesting to note that these exact solutions are not subject to any constraints (such as low-frequency and weak-field). To illustrate the high accuracy and excellent efficiency, we apply these exact solutions to calculate the gravitational radiation of the Schwarzschild black hole. The numerical results demonstrate the homogeneous solutions at low floating numbers $\bf N$ are comparable to those obtained using the Mano-Suzuki-Takasugi (MST) method and numerical integration methods with high floating-point numbers $\bf 2N$. And the inhomogeneous solutions can accurately calculate the arbitrary mode $(\ell,m)$ and orbital radius $r_0$, and its accuracy (relative error of energy flux (REEF) $\sim {10^{ - 99}}$) is much higher than the results obtained by 27.5PN expansion (REEF$\sim 10^{-16}$) and the MST method (REEF$\sim 10^{-53}$) at $r_0=15M$  with ${\bf N}=100$ and mode $(2,2)$.


\end{abstract}
\maketitle
%%%%%%%%%%%%%%%%%%%%%%%%%%%%%%%%%%%%%%%%%%%%%%%%%%%%%%%%%%%%%%%%
%
%\newpage
%\tableofcontents


\section{Introduction}
Black hole (BH) perturbation theory  \cite{10.1143/PTPS.128.1,Sasaki_2003} is a method used to study various real relativistic objects, such as massive compact objects, jets, supernova explosions, binary systems, etc. Originally developed as a metric perturbation theory, for Schwarzschild BHs, Regge \cite{Regge_1957} and Zerilli \cite{Zerilli_1970} decoupled and separated a single master equation for the metric perturbation into odd and even parity parts, respectively.
However, no such equation has been established for rotating Kerr BHs.
This led Bardeen and Press \cite{Bardeen1973RadiationFI} to derive a master equation for the curvature perturbation of a Schwarzschild BH without source ($T_{\ell m \omega}=0$) using the Newman-Penrose null-tetrad formalism, where the tetrad components of the curvature tensor serve as fundamental variables.
Expanding to a Kerr BH with the source ($T_{\ell m \omega}\neq 0$), Teukolsky \cite{Teukolsky1973} derived the curvature perturbation equation. This resulting equation, known as the Teukolsky equation, represents a wave equation for the null-tetrad component of the Weyl tensors $\psi_0$ and $\psi_4$.
The Teukolsky equation describes the dynamics of various fields of  different spins as perturbations (scalar, neutrino, electromagnetic, and gravitational perturbations) to the Kerr metric. Therefore, the Teukolsky equation can be used as a mathematical model of gravitational waves (GWs) to construct GW waveform templates.
In recent years  there is an increased interest in GW detection, and we anticipate gaining a deeper understanding of BHs' demographics and properties in the coming years \cite{Abbott2016,Abbott2016a,Abbott2019,Abbott2021,Abbott2023}.
Future space-based GW detectors, such as the Laser Interferometer Space Antenna (LISA) \cite{Audley:2017drz}, TianQin \cite{Luo_2016,Mei_2020}, and Taiji \cite{Gong_2021,Ruan_2020}, will be built with the specific purpose of detecting GW signals from sources that radiate in the millihertz ($mHz$) bandwidth, namely extreme mass ratio inspirals (EMRIs) \cite{Amaro_Seoane_2018,AmaroSeoane2020,Isoyama_2022}.
BH perturbation theory has been utilized in modeling EMRI systems and their associated GWs \cite{Poisson:2011,Pound:2021}.
Under the influence of gravitational radiation reaction, the compact secondary body undergoes a slow inspiral into the primary supermassive black hole, emitting GWs that propagate to infinity.
Due to potential interference from other sources concurrently emitting GWs in the $mHz$ bandwidth \cite{Audley:2017drz}, detection and parameter estimation of EMRI signals will rely on matched filtering techniques.
Accurate calculation of GW waveform templates is thus of paramount importance, requiring precision on the order of fractions of radians in GW phases \cite{Babak_2017,Barack_2019}.
Upon precise modeling and interpretation, the obtained data presents an opportunity for unparalleled experiments examining general relativity and detecting hitherto unknown astrophysical phenomena \cite{Amaro-Seoane_2007,Fan:2020zhy,Zi:2021pdp}.

The Teukolsky equation without source simplifies the perturbation problem significantly because it does not need to construct Green's function. This homogenous equation has been widely adopted to study the quasinormal modes of black holes and other relativistic celestial bodies, including massive dense stars, jets, supernova explosions, and more.
By exploiting infinite series of special  functions, Leaver derived the analytical expressions for the solutions of the Regge-Wheeler (RW) and Teukolsky equations \cite{Leaver1986}, which enabled precise computation of the quasinormal modes (QNMs) spectrum of black holes \cite{leaver1985a} and discussed their excitation using Green's function method \cite{Leaver_1986a}. But Green's function constructed by Leaver cannot be applied to calculate gravitational radiation.

It is known, but  historically somewhat under-appreciated in the physics literature, that both the RW and Teukolsky equations without source are examples of the confluent Heun equation \cite{Ronveaux:1995,slavyanov2000special}.
One possible reason for the historical lack of attention given to the confluent Heun equation in physics literature is the difficulty in numerically calculating its associated Heun class functions in the past, leading to a prevalence of Leaver's solution.
However, recent advances in numerical algorithms have enabled the computation of Heun class functions using various mathematical software packages. Maple software version 7 introduced numerical calculations of Heun class functions in 2001.
And Motygin provided MATLAB code for general and confluent Heun functions in 2015 \cite{Motygin_2015} and 2018 \cite{Motygin2018}, respectively.
The 12.1 version of Mathematica software released in 2020 also includes numerical calculation capabilities for Heun class functions. It should be noted, however, that the performance and accuracy of Heun class functions can differ across software packages.
Fiziev provided the analytical solutions \cite{Fiziev_2006,Fiziev_2007,Fiziev_2010,fiziev2009classes} to the source-free perturbation equations (RW or Teukolsky equations) in terms of the confluent Heun function.
Furthermore, these solutions have been utilized to calculate QNMs of the Teukolsky equation describing the Schwarzschild black hole\cite{Fiziev_2011a}, including its continuous spectrum\cite{borissov2010exact}, as well as for the central engine of Gamma-ray bursts (GRB) and cosmic jet of the Kerr black hole\cite{fiziev2009new,Staicova_2010}.
Following Fiziev's work, numerous researchers have employed Heun class functions to study source-free perturbation equations in various spacetimes, such as QNMs \cite{Semra2019} of Dirac field in $2+1$ dimensional gravitational wave background \cite{Zhang2014BlackHA}, QNMs and the reflection coefficient\cite{Horta_su_2021a} of massless fields in Kerr-Newman–de Sitter BH \cite{Hatsuda_2021}, and QNMs of the massive scalar field in Kerr-AdS$_5$ BH \cite{Noda2022}.
Moreover, Cook et al. \cite{Cook_2014} converted the modes equations of Kerr BH into the confluent Heun equation and then solved the radial equation using the continued fraction method, and the angular equation using the spectral method, resulting in improved accuracy for the QNMs.
% Figure environment removed



In previous research, the construction of Green's function for the Teukolsky equation with the source was deemed analytically infeasible. As a result, numerical integration methods were widely employed by most scholars initially to investigate the radiation phenomenon associated with black hole perturbation.
These numerical methods involve simulating the propagation of gravitational waves in a background spacetime distorted by a rotating black hole with an external particle or other perturbation included.
Such simulations furnish invaluable insight into the behavior of gravitational waves and their interactions with black holes\cite{Press1973,Chrzanowski_1975,Tashiro_1981,chandrasekhar1998mathematical,Nakamura_1987}.
Subsequently, using the post-Newtonian (PN) expansion, Sasaki derived a part of Green's function \cite{Sasaki_1994} for the RW equation without source.
This derivation involved only the ingoing wave solution $X_{\ell m\omega }^{{\rm{in}}}$ and its asymptotic amplitudes at infinity.
Utilizing the Sasaki-Nakamura (SN) transformation, Sasaki transformed $X_{\ell m\omega }^{{\rm{in}}}$ into the ingoing wave solution $R_{\ell m\omega }^{{\rm{in}}}$ for the Teukolsky equation with source in the Schwarzschild spacetime \cite{Tagoshi_1994} and Kerr spacetime \cite{Shibata_1995}.
This approach is applied to determining $R^{\rm in}_{\ell m\omega}$ of Schwarzschild BHs up to 5.5PN order \cite{Tanaka_1996}, as well as $R^{\rm in}_{\ell m\omega}$ of Kerr BHs up to 4PN order \cite{10.1143/PTPS.128.1,Tagoshi:1996gh}.
For the black hole absorption and tail correction, it is necessary to construct another part of Green's function containing $R_{\ell m\omega }^{{\rm{up}}}$, which represents the homogeneous solution of the Teukolsky equation of pure outgoing waves.
Therefore, Poisson and Sasaki \cite{Poisson_1995} utilized the spherical Hankel function to construct only 1PN $X_{\ell m\omega }^{{\rm{up}}}$, but this solution is controversial and does not satisfy the conservation of the Wronskian \footnote{It follows from the conservation of the Wronskian that $X_{\ell m\omega }^{{\rm{up}}}(r \to 2M)\sim - \bar A_{\ell m\omega }^{{\rm{up }}}{e^{ - i\omega {r^*}}} + A_{\ell m\omega }^{{\rm{in }}}{e^{i\omega {r^*}}},A_{\ell m\omega }^{{\rm{in }}} \ne 0$. But $X_{\ell m\omega }^{{\rm{up}}}$ constructed using spherical Hankel functions has the amplitude $A_{\ell m\omega }^{{\rm{in }}} = 0$, which fails to satisfy this conservation relation.}.



The Mano-Suzuki-Takasugi (MST) method is a second analytical approach for constructing Green's function of the Teukolsky equation with source.
Japanese researchers `revamped' Leaver's solutions and developed new series solutions for the radial solutions \cite{Mano1996RWE,Mano_1996,Sasaki_2003}.
The primary distinction between the MST solution and Leaver's solutions is that the former can obtain analytical expressions for asymptotic amplitudes, which the five solutions produced by Leaver do not satisfy.
The series solutions employed by the MST method, known as the MST expansions, are naturally adapted to carrying out low-frequency expansions. Casals believed that the MST series converge theoretically for any frequency value, albeit their convergence speed diminishes as the frequency magnitude increases \cite{Casals_2015}.
Initially, Mano et al. \cite{Mano1996RWE,Mano_1996} only presented the MST method as the second-order post-Minkowskian expansion result. Subsequently, Fujita numerically computed the renormalized angular momentum of the MST method and achieved remarkably precise solutions \cite{Fujita_2004,Fujita_2005}. Additionally, Fujita utilized the MST method for computing gravitational radiation at arbitrary high PN order.
For instance, Fujita provided 22PN results in a Schwarzschild BH \cite{Fujita_2012} and 11PN results in a Kerr BH\cite{Fujita_2015a}. Other than that, Fujita presented 5.5PN GW polarizations and associated factorized resumed waveforms\cite{Fujita_2010}. The MST method can also be employed to compute the self-force (SF) acting on point particles \cite{Poisson:2011,Fujita_2015GSF,Hikida_2004,Hikida_2005,Casals_2013}.



Apart from the MST method, there exist alternative approaches for computing the gravitational wave energy flux.
Fully relativistic GW fluxes from orbits of non-spinning particles were initially computed for eccentric orbits around a Schwarzschild BH \cite{Cutler:1994} and circular equatorial orbits around a Kerr BH \cite{Finn:2000}. Glampedakis et al. \cite{Glampedakis_2002} calculated Fluxes from eccentric orbits in the Kerr spacetime.
%
Fully generic GW fluxes from non-spinning particles were derived by Drasco\cite{Drasco_2006}, and Hughes et al. computed adiabatic waveforms of EMRIs \cite{Hughes:2021}.
%
Circular orbits in a BH spacetime with the spin of the secondary taken into account were investigated via GW fluxes in
Refs.~\cite{Han:2010,Harms:2016a,Harms:2016b,Lukes-Gerakopoulos:2017,Akcay:2020}, while quasi-circular adiabatic evolution of such orbits that includes spin was presented in Refs.~\cite{Piovano:2020,Piovano:2021,Skoupy:2021a,Rahman:2023}.
%
The first-order SF for circular orbits in the Schwarzschild spacetime was calculated in Ref.~\cite{Mathews:2022}.
%
Moreover, Skoupy et al. \cite{Skoupy:2021b} computed the fluxes from spinning bodies on eccentric equatorial orbits around a Kerr BH, and provided the linear in spin approximation for the adiabatic evolution \cite{Skoupy:2022}.
%
Referring to the frequency-domain method developed in Refs.~\cite{Drummond:2022a, Drummond:2022b} for generic orbits of spinning bodies around a Kerr black hole, Hughes et al. \cite{skoup2023asymptotic} computed the asymptotic GW fluxes from a spinning body moving on such orbits up to linear order in the secondary spin in a Kerr spacetime.



In this work, \textsl{we extend Fiziev's solutions, which solely include the Teukolsky equation without the source, to encompass the Teukolsky equation with the source}. We aim to obtain the complete Green's functions including $R_{\ell m\omega }^{{\rm{in,up}}}$.
Different from the PN expansion and MST method in the black hole perturbation approach, our strategy involves expressing the general solution of the homogeneous Teukolsky equation as a linear combination of two linearly independent particular solutions (constructed in the form of confluent Heun functions), and then using the boundary conditions to obtain $R_{\ell m\omega }^{{\rm{in,up}}}$.
%Our method offers benefits over other techniques:
%The constructions of $R_{\ell m\omega }^{{\rm{in,up}}}$  of PN expansion are independent of each other, but our $R_{\ell m\omega }^{{\rm{in,up}}}$ are derived from the same general solution. This causes $R_{\ell m\omega }^{{\rm{up}}}$ of PN expansion to be incorrect, but our $R_{\ell m\omega }^{{\rm{up}}}$ satisfies the conservation of the Wronskian.
The construction difference between our method and other methods is shown in \Cref{fig:introduction}.


The remainder of the paper is arranged as follows.
In \Cref{sec:GTF}, the Teukolsky equation is rewritten into a more general form, which can cover many curvature perturbation equations of Type-D spacetimes.
In \Cref{sec:GsolGFRTE}, we proposed a general solution based on the confluent Heun function to construct Green's function including $R_{\ell m\omega }^{{\rm{in,up}}}$. This method is not limited to the Teukolsky equation of the Kerr black hole, but also applies to other Type-D spacetime cases.
We present the analytical expression for the confluent Heun function at infinity for the first time in \Cref{sec:AsympForHeunC}.
To verify the correctness of our method, our method is applied to calculate the energy flux and waveform of the Schwarzschild BH in \Cref{sec:Application}.
Moreover, we compare our results with those of PN expansion, MST, and numerical integration methods, to validate the accuracy and effectiveness of our method.
The conclusions are presented in \Cref{sec:Conclusion}.
In this paper, we use geometrized units: $c=G=1$.



%%%%%%%%%%%%%%%%%%%%%%%%%%%%%%%%%%%%%%%%%%%%%%%%%%%%%%%%%%%%%%%%


\section{General Form of Teukolsky Equations}\label{sec:GTF}
We now reformulate radial Teukolsky equations (RTE) into a more comprehensive form that can encompass all previously identified Teukolsky equations of Type-D spacetimes, which, names the general form of radial Teukolsky equation (GFRTE), is expressed as
%

\begin{equation}\label{eq:GFoTRE}
 \left[ {\Delta _n^{ - s + 1}\frac{d}{{dr}}\Delta _n^{s + 1}\frac{d}{{dr}} + V(r)} \right]{R_{\ell m\omega }} = {\Delta _n}{T_{\ell m\omega }},
\end{equation}
where
\begin{eqnarray}\label{eq:Delta0}
 && \Delta_n  = \sum\limits_{i = 0}^n  {{b_i}{r^{2 - i}}}  = \prod\limits_{i = 1}^n  {{{\left( {r - {r_i}} \right)}}},
\\ \label{eq:VV}
 && V(r) = \sum\limits_{i = 0}^\infty  {{{\bf v}_i}{r^i}}.
\end{eqnarray}
%And the general solution of \cref{eq:GFoTRE} can be represented as
%\begin{align}\label{eq:Gsol-GFoTRE}
%{R_{\ell m\omega }} &= {C_1}{R_1} + {C_2}{R_2}  \\
% &+ {R_2}\int {{R_1}\frac{{{T_{\ell m\omega }}}}{{{\Delta _n}}}\frac{{dx}}{{{W_{\rm{T}}}}}}  - {R_1}\int {{R_2}\frac{{{T_{\ell m\omega }}}}{{{\Delta _n}}}\frac{{dx}}{{{W_{\rm{T}}}}}} \nonumber,
%\end{align}
%where $R_1$ and $R_2$ are  the homogenous (free-source) solutions of \cref{eq:GFoTRE}, and ${{{W}}_{\rm T}}$ is  the Wronskian  of Teukolsky Equation.

The Newman-Penrose formalism allows for the determination of the explicit form \eqref{eq:VV} of $V(r)$, and it can be observed that the simplest expansion of $V(r)$ does not contain any terms with fractional powers of $r$.
The common approach is to approximate $\Delta_n$ by $\Delta_2$ and $\Delta_4$ \footnote{The Teukolsky equation of Kerr-Newman (anti-)de Sitter BHs \cite{Khanal_1983,Suzuki_1998,Suzuki_1999} without the source corresponds to $\Delta_4$-type.}, resulting in various types of the GFRTEs with different forms in the potential term $V(r)$. This paper specifically focuses on the Teukolsky equation of $\Delta_2$-type, that is $\Delta_2  = \left( {r - {r_-}} \right)\left( {r - {r_+}} \right)$. Here,  $r_- $ is the inner horizon, and $r_+ $ is the outer (event) horizon.

For different black holes, the values of $\Delta_2$ and $V(r)$ in the Teukolsky equation will change. Now, we collect information on six black holes, and the corresponding $\Delta_2$ and $V(r)$ are organized as follows:

 \textcolor{colour6}{\uppercase\expandafter{\romannumeral1}. Schwarzschild BHs}

The potential $V(r)$ of Schwarzschild BH for all spin fields \cite{Bardeen1973RadiationFI,Tagoshi_1994} is
\begin{equation}\label{eq:ScharzchildV}
{V_{{\rm{Sch}}}} = {\omega ^2}{r^4} + 2is\omega {r^2}(r - 3M) - {\Delta _2}{\lambda},
\end{equation}
where ${r_-} = 0$, ${r_+}=r_{\rm H}=2M$ and  ${\lambda }$ is given in Eq. (112) of Ref. \cite{Sasaki_2003}.

  \textcolor{colour6}{ \uppercase\expandafter{\romannumeral2}. Reissner-Nordstr{\"{o}}m BHs}

The potential $V(r)$ of Reissner-Nordstr{\"{o}}m BHs for the massive charged scalar field \cite{Garc_a_2021} is
 \begin{equation}
   {V_{{\rm{RN}}}} = {{(\omega r^2-e Q r)}^2} -\Delta_2 \left( {{\lambda} + {\mu ^2}{r^2}} \right),
 \end{equation}
 with $r_{\pm}=M \pm \sqrt {{M^2} - {Q^2}}$. Here, $Q$ is the charge of the black hole and $e$ is the elementary charge.

 \textcolor{colour6}{ \uppercase\expandafter{\romannumeral3}. high-dimensional Schwarzschild BHs}

The potential $V(r)$ of the background of a ($4 + 1$)-dimensional, non-rotating, neutral BH projected onto a 3-brane with the spin fields of $s=0, \frac{1}{2}, 1$ \cite{Harris_2003}, that is
\begin{equation}
\begin{array}{*{20}{l}}
{{V_{(4 + 1)}} = {\omega ^2}{r^4} - is\omega \hat r_{\rm{H}}^2r}\\
{\quad \quad  + {\Delta _2}\left( {2i\omega sr + s{\Delta ^{\prime \prime }} - 2s - \lambda } \right)},
\end{array}
\end{equation}
with ${r_\pm} =\pm  {\hat r}_{\rm H}$. The horizon radius $ {\hat r}_{\rm H}$ is shown in Eq. (2.4) of Ref. \cite{Harris_2003}.

 \textcolor{colour6}{ \uppercase\expandafter{\romannumeral4}. Kerr BHs}

The potential  $V(r)$ of  Kerr BH for all spin fields \cite{Teukolsky1973}:
\begin{equation}
  V_{\rm K} = {K^2} - isK\Delta_2  ' + \Delta_2  \left( {2isK' - \lambda } \right),
\end{equation}
where ${r_\pm} = M \pm \sqrt {{M^2} - {a^2}} $,  $K = \left( {{r^2} + {a^2}} \right)\omega  - ma$.

  \textcolor{colour6}{ \uppercase\expandafter{\romannumeral5}. Kerr-Sen BHs}

The potential $V(r)$ of  Kerr-Sen BH \cite{Sen_1992} for the massive charged scalar field \cite{Wu_2003,Siahaan_2015,Bernard_2016} is
\begin{equation}\label{eq:Vks}
\begin{array}{l} V_{\rm KS} = {(\omega (\Delta_2  + 2Mr) - eQr - am)^2}\\
\quad \quad - {\Delta _2}\left( {{\mu ^2}(\Delta_2  + 2Mr) + {\lambda } } \right),
\end{array}
\end{equation}
where ${r_\pm} = M-b \pm \sqrt {{(M-b)^2} - {a^2}} $ and $b = Q^2/2M$.
\begin{table*}[htbp]
	\centering
%\scriptsize
\caption{ General Form of Teukolsky Equations }\label{table:GFTE}
      \begin{tabular}{lccccc}
  \toprule
Potential Term $V(r)$     &   $ \{{r_\pm}\}$      & Spin Fields          \\   \midrule
${V_{{\rm{Sch}}}} = {\omega ^2}{r^4} + 2is\omega {r^2}(r - 3M) - {\Delta _2}{\lambda }$ & $ \left\{M \pm M\right\}$  &  all spins      \\  \midrule
$ {V_{{\rm{RN}}}} = {{(\omega r^2-e Q r)}^2} -\Delta_2 \left( {{\lambda} + {\mu ^2}{r^2}} \right)$ &  $ \left\{M \pm \sqrt {{M^2} - {Q^2}}\right\}$ &  $s=0$ \\  \midrule
 $\begin{array}{*{20}{l}}
{{V_{(4 + 1)}} = {\omega ^2}{r^4} - is\omega \hat r_{\rm{H}}^2r}\\
{\quad \quad  + {\Delta _2}\left( {2i\omega sr + s{\Delta ^{\prime \prime }} - 2s - \lambda } \right)}
\end{array}$   & $\{ {\pm {\hat r}_{\rm H}}\} $      &  $|s|=0, \frac{1}{2}, 1 \quad $    \\  \midrule
$ V_{\rm K} = {K^2} - isK\Delta_2  ' + \Delta_2  \left( {2isK' -\lambda } \right)$  &   $ \left\{ {M \pm \sqrt {{M^2} - {a^2}} } \right\}$   &  all spins \\  \midrule
 $\begin{array}{l} V_{\rm KS} = {(\omega (\Delta_2  + 2Mr) - eQr - am)^2}\\
\quad \quad - {\Delta _2}\left( {{\mu ^2}(\Delta_2  + 2Mr) + {\lambda } } \right)
\end{array}$   &   $ \left\{ {M - b \pm \sqrt {{{(M - b)}^2} - {a^2}} } \right\}$     &  $s=0$  \\\midrule
 ${V_{{\rm{KN1}}}} = {K^2} - {\rm{i}}sK\Delta_2 ' + \Delta_2 \left( {2{\rm{i}}sK' - \lambda } \right)$    & {$ \left\{ {M \pm \sqrt {{M^2} - {a^2} - {Q^2}} } \right\}$}   &  all spins \\  \midrule
${V_{{\rm{KN2}}}} =K^2-\lambda{\Delta_{2}}$  &  $ \left\{ {M \pm \sqrt {{M^2}-{a^2}-{Q^2}} } \right\}$   &   all spins \\  \midrule
 $V_{\rm KN3}= {\left( {K - eQr} \right)^2} - \Delta_2  \left( {{\mu ^2}\left( {{r^2} + {a^2}} \right) + \lambda } \right)\quad $    & $ \left\{ {M \pm \sqrt {{M^2} - {a^2} - {Q^2}} } \right\}$ &  $s=0$ \\  \midrule
 $\begin{array}{l}
{V_{{\rm{KN4}}}} = {\left( {K - eQr} \right)^2} - is\left( {K - eQr} \right){\Delta'_{2}}\\
\quad \quad \quad + {\Delta _2}\left( {2{\rm{i}}sK' + \lambda } \right)\end{array}$ &$ \left\{ {M \pm \sqrt {{M^2} - {a^2} - {Q^2}} } \right\}$ &    $|s|=1,2$\\ \bottomrule
\end{tabular}
\end{table*}
 \textcolor{colour6}{ \uppercase\expandafter{\romannumeral6}. Kerr-Newman BHs}

For the curvature perturbation equation, due to the coupling between different types of perturbation fields, it seems impossible to transform the general perturbation of KN BHs into a single equation except for some limit cases or the scalar field. We collected four examples with ${r_ \pm } = M \pm \sqrt {{M^2} - {a^2} - {Q^2}}$ as follow:

\textcolor[rgb]{0.50,0.50,1.00}{ \uppercase\expandafter{\romannumeral6}-1.} Using weakly charged approximation, Dudley and Finley \cite{DF1979JMP,Dudley_1977PRL} derived a Teukolsky-like equation (named as Dudley-Finley (DF) equation) for all spin fields. And $V(r)$ of DF equation for all spin fields is


\begin{equation}\label{eq:Vkn1}
  {V_{{\rm{KN1}}}} = {K^2} - {\rm{i}}sK\Delta_2 ' + \Delta_2 \left( {2{\rm{i}}sK' - \lambda } \right),
\end{equation}
While the QNMs of DF equation yield exact solutions for scalar perturbations, they are considered a conceptually questionable approximation for gravitational and electromagnetic modes\cite{Berti_2005,Mark_2015cyb}.



\textcolor[rgb]{0.50,0.50,1.00}{ \uppercase\expandafter{\romannumeral6}-2.} $V(r)$ of the eikonal limit $\ell \gg1$ for all spin fields is \cite{Li_2021}
\begin{equation}\label{eq:Vkn2}
  {V_{{\rm{KN2}}}} =K^2-\lambda{\Delta_{2}},
\end{equation}

\textcolor[rgb]{0.50,0.50,1.00}{ \uppercase\expandafter{\romannumeral6}-3.} $V(r)$ of  Kerr-Newman BH for charged massive scalar field \cite{Hod_2014,Hod_2015}:
\begin{equation}\label{eq:Vkn3}
V_{\rm KN3}= {\left( {K - eQr} \right)^2} - \Delta_2  \left( {{\mu ^2}\left( {{r^2} + {a^2}} \right) + \lambda } \right),
\end{equation}


\textcolor[rgb]{0.50,0.50,1.00}{ \uppercase\expandafter{\romannumeral6}-4.} $V(r)$ of  Kerr-Newman BH for charged photons and gravitons fields \cite{Hartman_2010} is
\begin{equation}\label{eq:Vkn4}
\begin{array}{l}
{V_{{\rm{KN4}}}} = {\left( {K - eQr} \right)^2} - is\left( {K - eQr} \right){\Delta'_{2}}\\
\quad \quad \quad + {\Delta _2}\left( {2{\rm{i}}sK' + \lambda } \right),
\end{array}
\end{equation}
The potential terms of the above nine Teukolsky equations are summarized in \Cref{table:GFTE}.



\section{General Solution of Teukolsky Equations}\label{sec:GsolGFRTE}

The general Teukolsky equation (\ref{eq:GFoTRE}) can be classified into two kinds: one without the source (homogenous Teukolsky equation) which has been extensively utilized to study various physical phenomena related to black holes, such as quasinormal modes \cite{leaver1985a,Cook_2014,Berti_2005,Mark_2015cyb,Li_2021}, Hawking radiation \cite{Harris_2003}, near-superradiant scattering \cite{Hartman_2010}, scalar clouds \cite{Siahaan_2015,Bernard_2016,Hod_2014,Hod_2015}, the central engine \cite{fiziev2009new} of Gamma-ray bursts, and cosmic jets \cite{Staicova_2010}.
And the other with the source (inhomogeneous Teukolsky equation) which has found widespread application in the study of energy fluxes and waveforms \cite{Sasaki_2003,Poisson:2011,Pound:2021,Amaro_Seoane_2018,AmaroSeoane2020,Isoyama_2022,Fujita:2020}, absorption \cite{Teukolsky1974Perturbations} of gravitational waves and tail correction \cite{Casals_2015} of linear field perturbations, as well as scalar self-force \cite{Warburton_2010} and electromagnetic self-force \cite{Torres_2022} acting on a charged particle in Kerr spacetime. To obtain general solutions for these equations, it is advisable to solve them separately. In this section, we should first present the solution without the source, followed by the solution with the source.


\subsection{General Solution of Teukolsky Equations without source}


The HRTE can be expressed as
\begin{equation}\label{eq:HRTE}
 \left[ {\Delta^{ - s + 1}\frac{d}{{dr}}\Delta^{s + 1}\frac{d}{{dr}} + V(r)} \right]{R_{\ell m\omega }} = 0,
\end{equation}
where $\Delta_2$ is abbreviated as $\Delta$.

\cref{eq:HRTE} is a second-order ordinary differential equation (ODE) whose general solution can be expressed as:
\begin{equation}
{R_{\ell m\omega }}(r) = {C_1}R_0^\beta (r) + {C_2}R_0^{ - \beta }(r), \label{eq:D2Gsol}
\end{equation}
where $R_0^{\pm\beta }(r)$ are two linear independent particular solutions of \cref{eq:HRTE}, $C_1$ and $C_2$ are constants that should be determined based on different boundary conditions.
%


The HRTE \eqref{eq:HRTE} is not the standard ODE that matches the specific function.
By utilizing the so-called S-homotopic transformation \cite{slavyanov2000special}, the HRTE \eqref{eq:HRTE} can be transformed into an ODE that corresponds to a special  function known as the Heun class equation.
Thus, two linear independent particular solutions of \cref{eq:HRTE} over the entire range $r\in {\cal{R}}= [r_H,\infty)$ can be obtained as
\begin{equation}\label{eq:D2Psol}
R_0^{ \pm \beta }(r) = S_{\rm{0}}^{\pm\beta} (x)\mathbb{H}^{ \pm \beta }(x),
\end{equation}
where $x$ is defined as a new coordinate that is obtained by applying a M\"{o}bius (isomorphic) transformation, which is
\begin{equation}\label{eq:mtran}
x =- \frac{{  r - r_+}}{{r_ + } - {r_ - }}.
\end{equation}
Introducing the unnormalized S-homotopic transformation
\begin{equation}\label{eq:D2SHT}
  S_{\rm{0}}^{\pm \beta}(x) = {\left( { - x} \right)^{ \frac{1}{2}(\pm \beta-s) }}{\left( {1 - x} \right)^{\frac{1}{2}(\gamma-s) }}{{\rm{e}}^{\frac{1}{2}\alpha x}},
\end{equation}
which can be regarded as the  asymptotic behaviors of the function $R_0^{ \pm \beta }(r)$  at regular singularities, then substituting \Cref{eq:D2Psol,eq:mtran,eq:D2SHT} into \cref{eq:HRTE}, we can obtain standard confluent Heun equation
\begin{align}\label{eq:HeunC}
&\mathbb{H} '' - \frac{{\left( { - {x^2}\alpha  + \left( { - 2 - \beta  - \gamma  + \alpha } \right)x + 1 + \beta } \right)}}{{x\left( {x - 1} \right)}}\mathbb{H}'\nonumber\\
 &  - \left( \begin{array}{l}
\left( {\left( { - 2 - \beta  - \gamma } \right)\alpha  - 2\delta } \right)x \\
 + \left( {\beta  + 1} \right)\alpha  + \left( { - \gamma  - 1} \right)\beta  - \gamma  - 2\eta
\end{array} \right)\frac{\mathbb{H} }{{2x\left( {x - 1} \right)}} = 0.
\end{align}
Two linear independent particular solutions $\mathbb{H}_{\rm{0}}^{\pm \beta } (x)$ of \cref{eq:HeunC} can be expressed as
\begin{subequations}\label{eq:HeunC-parsol}
\begin{align}
&\mathbb{H}_{\rm{0}}^{\beta } (x) = {\rm{HeunC}}(\alpha , \beta ,\gamma ,\delta ,\eta ;x) ,\\
&\mathbb{H}_{\rm{0}}^{-\beta} (x) = {( - x)^{ - \beta }}{\rm{HeunC}}(\alpha , - \beta ,\gamma ,\delta ,\eta ;x),
\end{align}
\end{subequations}
where  $\text{HeunC}$ is the confluent Heun\footnote{In this paper, the HeunC function are implemented using the corresponding functions and symbolic notations provided by the computational software \textit{Maple}.}  function \cite{Ronveaux:1995,slavyanov2000special,olver2010nist}, and $\alpha,\beta,\gamma,\delta$ and $\eta$ are parameters that should be determined for given black holes.
In \Cref{table:parameter}, we calculate the parameters corresponding to the nine Teukolsky equations proposed in \Cref{sec:GTF} .


% Table generated by Excel2LaTeX from sheet 'Sheet1'

\begin{table*}[htbp]
%\footnotesize
  \centering
\caption{ parameters that should be determined for given black holes}\label{table:parameter}
\begin{threeparttable}
    \begin{tabular}{c|c|c|c|c|c} \toprule
 $V(r)$   &   $\alpha$&   $\beta$& $\gamma$& $\delta$ & $\eta$ \\ \midrule
%% 1
${V_{{\rm{Sch}}}} $ &   $ 2i\omega {r_+}$  &  $ - s - 2i\omega {r_+}$  & $ s$  &  $ -2i s\omega {r_ + } - 2{\omega ^2}r_ + ^2$   &  $\begin{array}{l}
2is\omega {r_ + } + 2{\omega ^2}r_ + ^2\\
 - \frac{1}{2}{s^2} - s - \lambda
\end{array}$     \\ \midrule
%% 2
$V_{{\rm{RN}}} $ &   $ - 2\sqrt {{\mu ^2} - {\omega ^2}} {r_{\rm{x}}}$
&  $\frac{{2i{r_ + }}}{{{r_{\rm{x}}}}}(\omega {r_ + } - eQ)$
& $ \frac{{2i{r_ - }}}{{{r_{\rm{x}}}}}(eQ - \omega {r_ - })$
&  $\begin{array}{l}2r_{\rm{x}}^2{\omega ^2} - 2eQ\omega {r_{\rm{x}}}\\- {\mu ^2}(r_ - ^2 - r_ + ^2){r_{\rm{x}}}\end{array}$
&   $\begin{array}{l}
 - \lambda  - \frac{{{r_ + }}}{{r_{\rm{x}}^2}}(2{Q^2}{e^2}{r_ - } + \\
2Qe\omega r_ + ^2 - 2{\omega ^2}r_ + ^3\\
 + {\mu ^2}r_ - ^2{r_ + } - 2{\mu ^2}{r_ - }r_ + ^2\\
 + {\mu ^2}r_ + ^3 + 4{\omega ^2}{r_ - }r_ + ^2\\
 - 6Qe\omega {r_ - }{r_ + })
\end{array}$     \\ \midrule
%% 3
$V_{(4 + 1)}$ &   $- 2i\omega {r_{\rm{x}}}$
&  $\begin{array}{l}
 - \frac{1}{{{r_{\rm{x}}}}}(r_{\rm{x}}^2{s^2} + \\
4is\omega r_ + ^3 - 4{\omega ^2}r_ + ^4{)^{\frac{1}{2}}}
\end{array}$
& $\begin{array}{l}
\frac{1}{{{r_{\rm{x}}}}}(r_{\rm{x}}^2{s^2} + \\
4is\omega {r_ - }r_ + ^2 - 4{\omega ^2}r_ - ^4{)^{\frac{1}{2}}}
\end{array} $
&  $2\omega {r_{\rm{x}}}\left( {\omega {r_{\rm{x}}} + is} \right)$
&   $ \begin{array}{l}
 - \frac{{{s^2}}}{2} + 3s - \lambda  + \\
\frac{{\omega {r_ + }}}{{r_{\rm{x}}^2}}(2isr_ - ^2 - 3is{r_ - }{r_ + } + \\
3isr_ + ^2 - 4\omega {r_ - }r_ + ^2 + 2\omega r_ + ^3)
\end{array}$     \\ \midrule

%% 4
$V_{\rm K} $ &   $- 2i\omega r_{\rm x}$
&  $-s+\frac{{2i\omega (r_ + ^2 + {a^2}) - 2iam}}{{{r_{\rm{x}}}}}$
& $ s+\frac{{2i\omega (r_ - ^2 + {a^2}) - 2iam}}{{{r_{\rm{x}}}}}$
&  $2\omega  r_{\rm x}( {\omega ( r_{-}+r_{+})+ is})$
&   $\begin{array}{l} 2is\omega {r_ + }- \frac{1}{2}{s^2} - s - \lambda \\ - \frac{2}{{r_{\rm{x}}^2}}( {\omega r_{\rm m}  - am} ) ({\omega r_{\rm o}  - am})  \end{array}$     \\ \midrule

%% 5
$V_{\rm KS} $ &   $ - 2\sqrt {{\mu ^2} - {\omega ^2}} {r_{\rm{x}}}$
&  $  \begin{array}{l}
 - \frac{{2i}}{{{r_{\rm{x}}}}}(2M\omega{r_ + } \\
 - ma - eQ{r_ + })
\end{array}$
& $\begin{array}{l}
\frac{{2i}}{{{r_{\rm{x}}}}}(2M\omega {r_ - }\\
 - ma - eQ{r_ - })
\end{array}$
&  $ \begin{array}{l}
 - 2{r_{\rm{x}}}(eQ\omega \\
 + ({\mu ^2} - 2{\omega ^2})M)
\end{array}$
&   $ \begin{array}{l}
 - 2{r_ + }({\mu ^2}M + 2{\omega ^2}M + eQ\omega )\\
 - \lambda  - 2am\omega  + \frac{1}{{r_{\rm{x}}^2}}\left\{ {[ - 2{a^2}{m^2}} \right.\\
 + 2{r_ - }{r_ + }(4M\omega  - eQ)\\
 - 2ma({r_ - } + {r_ + })]eQ\\
 - 8M\omega ( - \omega r_ + ^3 + 2\omega {r_ - } r_ + ^2\\
 + M\omega {r_ - }{r_ + } - {r_ + } r_ - ^2\omega \\
\left. { - \frac{{ma}}{2}({r_ - } + {r_ + }))} \right\}
\end{array}$     \\ \midrule

%% 6
$V_{\rm{KN1}} $ &   $- 2i\omega r_{\rm x}$
&  $-s+\frac{{2i\omega (r_ + ^2 + {a^2}) - 2iam}}{{{r_{\rm{x}}}}}$
& $ s+\frac{{2i\omega (r_ - ^2 + {a^2}) - 2iam}}{{{r_{\rm{x}}}}}$
&  $2\omega  r_{\rm x}( {\omega ( r_{-}+r_{+})+ is})$
&   $\begin{array}{l} 2is\omega {r_ + }- \frac{1}{2}{s^2} - s - \lambda \\ - \frac{2}{{r_{\rm{x}}^2}}\left[ {\omega r_{\rm m}  - am} \right] \times \\
\left[ {\omega r_{\rm o}  - am} \right] \end{array}$     \\ \midrule

%% 7
$V_{\rm{KN2}} $ &   $- 2i\omega r_{\rm x}$
&  $\begin{array}{l}
 - \frac{2}{{{r_{\rm{x}}}}}[r_{\rm{x}}^2{s^2} - \\
{( - ma + {r_{\rm{m}}}\omega )^2}{]^{\frac{1}{2}}}
\end{array}$
& $ \begin{array}{l}
\frac{2}{{{r_{\rm{x}}}}}[r_{\rm{x}}^2{s^2} - \\
{( - ma + {r_{\rm{m}}}\omega )^2}{]^{\frac{1}{2}}}
\end{array}$
&  $2{\omega ^2}r_ - ^2 - 2{\omega ^2}r_ + ^2$
&   $\begin{array}{l}- \frac{1}{2}{s^2} - s - \lambda \\ - \frac{2}{{r_{\rm{x}}^2}}\left[ {\omega r_{\rm m}  - am} \right] \times \\
\left[ {\omega r_{\rm o}  - am} \right] \end{array}$     \\ \midrule

%% 8
$V_{\rm{KN3}} $ &   $ - 2\sqrt {{\mu ^2} - {\omega ^2}} {r_{\rm{x}}}$
&  $ \begin{array}{l}- \frac{{2{\kern 1pt} i}}{{{r_{\rm{x}}}}}({a^2}\omega  + \omega r_ + ^2\\- ma - eQ{r_ + })\end{array}$
& $\begin{array}{l}\frac{{2{\kern 1pt} i}}{{{r_{\rm{x}}}}}({a^2}\omega  + \omega r_ - ^2\\- ma - eQ{r_ - })\end{array}$
&  $ \begin{array}{l} - {r_{\rm{x}}}[2eQ\omega  + \\({r_ - } + {r_ + })({\mu ^2} - 2{\omega ^2})]\end{array}$
&   $ \begin{array}{l}
 - {\mu ^2}(r_ + ^2 + {a^2}) - \lambda \\
 - \frac{2}{{r_{\rm{x}}^2}}(\omega r_{\rm m}  - ma - {r_ +}eQ) \\
\times (  \omega r_{\rm o} - ma - eQ{r_-})
\end{array}$     \\ \midrule

%% 9
$V_{\rm{KN4}} $ &   $- 2i\omega r_{\rm x}$
&  $\begin{array}{l}
 - \frac{1}{{r_{\rm{x}}^2}}[ - 4{a^4}{\omega ^2}\\
 + 8{a^3}m\omega \\
 + 4{a^2}(2eQ\omega {r_ + }\\
 - is\omega {r_{\rm{x}}}\\
 - 2{\omega ^2}r_ + ^2 - {m^2})\\
 + 4ma(2i\omega r_ + ^2\\
 - 2eQ{r_ + } + s{r_{\rm{x}}})\\
 + r_{\rm{x}}^2{s^2} - 4is\omega r_ + ^2r_{\rm{x}}^2\\
 - 4r_ + ^2{\left( {eQ - \omega {r_ + }} \right)^2}{]^{\frac{1}{2}}}
\end{array}$
& $\begin{array}{l}
\frac{1}{{{r_{\rm{x}}}}}[ - 4{a^4}{\omega ^2}\\
 + 8{a^3}m\omega \\
 + 4{a^2}(2eQ\omega {r_ - }\\
 + is\omega {r_{\rm{x}}}\\
 - 2{\omega ^2}r_ - ^2 - {m^2})\\
 - 4ma(is{r_{\rm{x}}}\\
 + 2eQ{r_ - } - 2\omega r_ - ^2)\\
 + r_{\rm{x}}^2{s^2} + 4is\omega r_ - ^2{r_{\rm{x}}}\\
 - 4r_ - ^2{\left( {eQ - \omega {r_ - }} \right)^2}{]^{\frac{1}{2}}}
\end{array}$
&  $\begin{array}{l}
2\omega {r_{\rm{x}}}[({r_ - } + {r_ + })\omega \\
 + is - eQ]
\end{array}$
&   $\begin{array}{*{20}{l}}
{2is\omega {r_ + } - \frac{1}{2}{s^2} - s - \lambda }\\
{ - \frac{2}{{r_{\rm{x}}^2}}(\omega {r_{\rm{m}}} - ma - {r_ + }eQ)}\\
{ \times (\omega {r_{\rm{o}}} - ma - eQ{r_ - })}
\end{array}$     \\  \bottomrule

    \end{tabular}%
\begin{tablenotes}
        \footnotesize
        \item[$\clubsuit$] In \Cref{table:parameter}, some symbols are defined as: ${r_{\rm{x}}} = {r_ - } - {r_ + }$, $r_{\rm o}={a^2} + 2{r_-} {r_+} - r_{+}^2$, $r_{\rm m} = r_+ ^2+{a^2}$.
\end{tablenotes}
\end{threeparttable}


\end{table*}

Using the $\rm HeunC$ function (\ref{eq:HeunC-parsol}) and the S-homotopic transformation (\ref{eq:D2SHT}), the general solution of \cref{eq:HRTE} without source for $\Delta_2$-type can be expressed as:
\begin{align}\label{eq:D2GSol1}
{R_{\ell m\omega }} &= {C_1}S_0^\beta (x){\rm{HeunC}}(\alpha ,\beta ,\gamma ,\delta ,\eta ;x) \nonumber\\
&+ {C_2}S_0^{ - \beta }(x){\rm{HeunC}}(\alpha , - \beta ,\gamma ,\delta ,\eta ;x).
\end{align}

The general solution \eqref{eq:D2GSol1} can be effectively applied to a specific physical model by fixing the parameters ($\alpha,\beta,\gamma,\delta$, $\eta$) for a given black hole. This allows us to tailor the solution to accurately describe the characteristics and properties of the particular black hole under consideration.
$C_1$ and $C_2$  are the combination coefficients that are explicitly solved by three significant boundary conditions (Dirichlet, Neumann, and Robbin) in this paper. The use of these boundary conditions requires the asymptotic behavior of the confluent Heun function at the outer horizon and infinity, respectively.
Expanding the confluent Heun function in power series for the independent variable $x$ around the regular singular point $x=0$ \cite{Ronveaux:1995}, yields the following asymptotic behavior at the outer horizon:
\begin{equation}\label{eq:hc-hor}
  \mathop {\lim }\limits_{x \to 0} {\rm{ HeunC}}(\alpha ,\beta ,\gamma ,\delta ,\eta ;x) = 1,\quad r\rightarrow r_+,
\end{equation}
Expanding the confluent Heun function in a sector around the irregular singular point at infinity \cite{Ronveaux:1995}, the asymptotic behavior at infinity can be expressed as:
\begin{align}\label{eq:hc-inf}
&\mathop {\lim }\limits_{\left| x \right| \to \infty } {\rm{HeunC}}(\alpha ,\beta ,\gamma ,\delta ,\eta ;x) \to \nonumber \\
&D_ \odot^\beta \;{x^{ - \frac{{\beta  + \gamma  + 2}}{2} - \frac{\delta }{\alpha }}} + D_ \otimes ^\beta {{\rm{e}}^{ - \alpha x}}{x^{ - \frac{{\beta  + \gamma  + 2}}{2} + \frac{\delta }{\alpha }}},\ \ \ \ r \to \infty,
\end{align}
where ${{  D}_{\otimes}^\beta }$ and ${{  D}_{\odot }^\beta }$ are undetermined constants. Only when the constants ${{  D}_{\otimes}^\beta }$ and ${{  D}_{\odot }^\beta }$ are known, the value of the combination coefficient $C_1$ and $C_2$  can be determined according to the boundary conditions.
Because the calculation process of the constants ${{D}_{\otimes}^\beta }$ and ${{D}_{\odot }^\beta }$ is very complicated and lengthy,
we will introduce their calculation process in detail in \Cref{sec:AsympForHeunC}.


While Fiziev proposed a general solutions similar to \cref{eq:D2GSol1} for solving QNMs \cite{Fiziev_2011a}, he was not acquainted with the constants ${{  D}_{\otimes}^\beta }$ and ${{  D}_{\odot }^\beta }$.
Therefore,  he believes that providing explicit analytical expressions for $C_1$ and $C_2$ is an unsolved difficult problem in mathematics\cite{Fiziev_2006,fiziev2009newa}.
In the past decade, Bezerra and Vieira have dedicated their efforts to the exploration of Hawking radiation in scalar fields\cite{VIEIRA201414,Bezerra_2014,VIEIRA2015576,Vieira_2020,VIEIRA2020168197}. They also constructed a general solution similar to \cref{eq:D2GSol1} for the Klein-Gordon equation. Since the decay rate $\Gamma_+$ does not involve the calculation of ${{  D}_{\otimes}^\beta }$ and ${{  D}_{\odot }^\beta }$, they can simply use asymptotic behavior \eqref{eq:hc-hor} to obtain the Hawking radiation spectrum.


Solving ${{  D}_{\otimes}^\beta }$ and ${{  D}_{\odot }^\beta }$ analytically is one of the key advantages of our general solutions, which is different from the general solutions of these literatures \cite{Fiziev_2006,fiziev2009classes,Fiziev_2011a,VIEIRA201414,Bezerra_2014,VIEIRA2015576,Vieira_2020,VIEIRA2020168197}.
It is also beneficial to extend the general solutions of homogeneous equations to inhomogeneous equations \eqref{eq:GFoTRE}.
In contrast, the general solutions presented by Fiziev and Vieira are specific cases within our broader framework. \cref{eq:D2GSol1} covers the general solution of QNM, Hawking radiation and other physical problems.
Furthermore, their general solution is unable to construct the Green's function required by the inhomogeneous equation and the outgoing wave solution $R_{\ell m\omega }^{{\rm{up}}}$. Consequently, we apply the general solution to construct  the solution of inhomogeneous equations and $R_{\ell m\omega }^{{\rm{up}}}$, thereby broadening its application scope.





\subsection{General solution of Teukolsky equation with source}
To get the general solution of the Teukolsky equation with source under given boundary conditions, we should first find the ingoing wave and outgoing wave solutions $R_{\ell m\omega}^{\rm in,up}$ of the homogenous Teukolsky equation, then obtain the general solution of the inhomogeneous Teukolsky equation utilizing the Green's function method \cite{Detweiler:1978ge}.
The most common methods for solving the homogenous solutions $R^{\rm in,up}_{\ell m\omega}$  are the PN expansion \cite{Tagoshi_1994} of the SN equation and the MST method\cite{Sasaki_2003,Mano1996RWE,Mano_1996}. The former method involves utilizing the Chandrasekhar-Sasaki-Nakamura transformation  \cite{Chandrasekhar1975,Sasaki_1982a,Sasaki_1982b,Tagoshi_1994} to convert the HRTE \eqref{eq:HRTE} into the SN equation, and then derives the PN expansion of $R^{\rm in}_{\ell m\omega}$.
On the other hand, in the MST method, a series solution of the hypergeometric function that converges within a finite region is first constructed, and then a series solution of the Coulomb wave function that converges at infinity is generated. The two solutions are subsequently matched to yield a convergent solution that extends from the horizon to infinity.




\subsubsection{Ingoing and Outgoing Wave Solutions  of homogenous Teukolsky equation}\label{sec:in-outgongsol}

Now, we commence with the construction of the ingoing and outgoing wave solutions from the solution \eqref{eq:D2GSol1}.
Using the asymptotic properties \eqref{eq:hc-hor} and \eqref{eq:hc-inf} of the confluent Heun function and the boundary condition of the ingoing wave, we can construct the ingoing wave solution. Noting the ingoing wave solution at the horizon is purely ingoing wave, we have $C_2^{\rm in}=0$. Thus, the general ingoing wave solution $R_{\ell m\omega }^{{\rm{in}}}$ is given by
\begin{align}\label{eq:uHor}
R_{\ell m\omega }^{{\rm{in}}} &= C_1^{{\rm{in}}}R_{\rm{0}}^\beta  + C_2^{{\rm{in}}}R_{\rm{0}}^{ - \beta } = C_1^{{\rm{in}}}R_{\rm{0}}^\beta \nonumber \\
 &=C_1^{{\rm{in}}} S_0^\beta (x) {\rm{HeunC}}(\alpha ,\beta ,\gamma ,\delta ,\eta ;x).
\end{align}
Similarly, we can construct the general outgoing wave solution as
\begin{align}\label{eq:uOut}
R_{\ell m\omega }^{{\rm{up}}}&= C_1^{{\rm{up}}}R_{\rm{0}}^\beta  + C_2^{{\rm{up}}}R_{\rm{0}}^{ - \beta } \nonumber \\
 &= C_1^{{\rm{up}}} S_0^\beta (x)  {\rm{HeunC}}(\alpha ,\beta ,\gamma ,\delta ,\eta ;x) \nonumber  \\
 &+ C_2^{{\rm{up}}} S_0^{-\beta} (x) {\rm{HeunC}}(\alpha ,-\beta ,\gamma ,\delta ,\eta ;x).
\end{align}
According to the asymptotic properties \eqref{eq:hc-hor} and \eqref{eq:hc-inf} of the HeunC function, and noting that the outgoing wave solution at infinity is purely outgoing wave, we have
\begin{align}
& C_1^{{\rm{up}}} = {\left( { - 1} \right)^{\beta+1} }\Big( {  \frac{{D_ \odot ^{ - \beta }}}{{D_ \odot ^\beta }}} \Big)C_2^{{\rm{up}}}.
\end{align}
Thus, the general  outgoing wave is described by
\begin{align}\label{eq:uOut1}
R_{\ell m\omega }^{{\rm{up}}}&= C_2^{{\rm{up}}}\Big[ {\left( { - 1} \right)^{\beta+1} }\Big( {  \frac{{D_ \odot ^{ - \beta }}}{{D_ \odot ^\beta }}} \Big)S_0^\beta (x)  {\rm{HeunC}}(\alpha ,\beta ,\gamma ,\delta ,\eta ;x) \nonumber  \\
 &+ S_0^{-\beta} (x) {\rm{HeunC}}(\alpha ,-\beta ,\gamma ,\delta ,\eta ;x)\Big].
\end{align}


Compared with PN expansion results, our ingoing wave solution \eqref{eq:uHor} is considered a complete solution without the need for series expansion. And our outgoing wave solution \eqref{eq:uOut1} is undisputedly accurate, satisfying the conservation of the Wronskian determinant.
Additionally, the solutions \eqref{eq:uHor} and  \eqref{eq:uOut1} are not constrained by limitations stemming from slow motion and weak-field approximations, rendering its findings superior to those of the PN expansion results near the horizon.
Compared with the MST method, our method employs a special function to construct the solutions $R_{\ell m\omega }^{{\rm{in,up}}}$ which  does not involve computing two-sided infinite series or renormalized angular momentum solutions $\nu$ ( MST method faces solving the transcendental equation of renormalized angular momentum $\nu$ ). Therefore, the solutions impose no limitations, while the MST method is constrained by low-frequency approximations.







\subsubsection{Analytical  solution of Teukolsky equations with source}

With the two homogeneous solutions $R_{\ell m\omega }^{{\rm{in,up}}}$ shown by Eqs. \eqref{eq:uHor} and  \eqref{eq:uOut1} at hand, we can easily obtain a general solution of the inhomogeneous Teukolsky equation  \eqref{eq:GFoTRE} with purely ingoing behavior at the horizon and purely outgoing behavior at infinity, which is described by

\begin{align}\label{eq:Rlmw-source}
{R_{\ell m\omega }}(r)
 &= R_{\ell m\omega }^{{\rm{up}}}(r)\int_{{r_ + }}^r d r'\frac{{{T_{\ell m\omega }}(r')R_{\ell m\omega }^{{\rm{in}}}\left( {r'} \right)}}{{\Delta (r'){W_T}\left( {r'} \right)}}\nonumber\\
 &+ R_{\ell m\omega }^{{\rm{in}}}(r)\int_r^\infty  d r'\frac{{{T_{\ell m\omega }}\left( {r'} \right)R_{\ell m\omega }^{{\rm{up}}}\left( {r'} \right)}}{{\Delta (r'){W_T}\left( {r'} \right)}},
\end{align}
where the Wronskian ${{\rm{W}}_T}$\footnote{ Another form of the Wronskian is often used in the literature, namely
the conserved Wronskian ${{\rm{W}}_C}$ can also can be defined as ${W_{\rm C}} = R_{\ell m\omega }^{{\rm{up}}}\frac{d}{{d{r^*}}}R_{\ell m\omega }^{{\rm{in}}} - R_{\ell m\omega }^{{\rm{in}}}\frac{d}{{d{r^*}}}R_{\ell m\omega }^{{\rm{up}}} = 2i\omega B_{\ell m\omega }^{{\rm{inc}}}C_{\ell m\omega }^{{\rm{trans}}}$, so ${W_{\rm T}}$ can also be written as ${W_{\rm T}}={{\rm{W}}_{\rm C}}{\Delta} $. } of Teukolsky Equation is given by
\begin{align}
    {W_{\rm T}} &= R_{\ell m\omega }^{{\rm{up}}}\frac{d}{{dr}}R_{\ell m\omega }^{{\rm{in}}} - R_{\ell m\omega }^{{\rm{in}}}\frac{d}{{dr}}R_{\ell m\omega }^{{\rm{up}}} \nonumber \\
    &\equiv 2i\omega B_{\ell m\omega }^{{\rm{inc}}}C_{\ell m\omega }^{{\rm{trans}}}{\Delta}.
\end{align}
The asymptotic solution at the horizon $(r \to {r_ + })$ is given by
\begin{align}\label{eq:Hsol}
R_{\ell m\omega }^{\rm{H}} &= \frac{{B_{\ell m\omega }^{{\rm{trans}}}{\Delta ^2}{e^{ - ik{r^*}}}}}{{2i\omega C_{\ell m\omega }^{{\rm{trans}}}B_{\ell m\omega }^{{\rm{inc}}}}}\int_{{r_ + }}^\infty  d r'\frac{{{T_{\ell m\omega }}(r')R_{\ell m\omega }^{{\rm{up}}}(r')}}{{{\Delta ^2}(r')}} \nonumber \\
 &\equiv \tilde Z_{\ell m\omega }^{\rm{H}}{\Delta ^2}{e^{ - ik{r^*}}}.
\end{align}
And the asymptotic solution at infinity $(r \to \infty ) $ is shown by
\begin{align}\label{eq:InfSol}
R_{\ell m\omega }^\infty  &= \frac{{{r^3}{e^{i\omega {r^*}}}}}{{2i\omega B_{\ell m\omega }^{{\rm{inc}}}}}\int_{{r_ + }}^\infty  d r'\frac{{{T_{\ell m\omega }}(r')R_{\ell m\omega }^{{\rm{in}}}(r')}}{{{\Delta ^2}(r')}}\nonumber\\
 &\equiv \tilde Z_{\ell m\omega }^\infty {r^3}{e^{i\omega {r^*}}}.
\end{align}



Only this and nothing more, we have obtained the analytical solution \eqref{eq:Rlmw-source} and asymptotic solutions $R_{\ell m\omega }^{{\rm H},\infty}$ shown by Eqs.  (\ref{eq:Hsol}) and  (\ref{eq:InfSol}) for the Teukolsky equation with the source.

Moreover, the paper presents several functions with symmetry, which we summarize as follows:
\begin{subequations}
  \begin{align}
&Z_{\ell , - m\omega }^{\infty ,{\rm{H}}} = {( - 1)^\ell }\bar Z_{\ell m\omega }^{\infty ,{\rm{H}}},\\
& {{} _s}{S_{\ell m}}(\theta ) = {( - 1)^{(s + \ell )}}{{\kern 1pt} _s}{S_{\ell , - m}}(\pi  - \theta ),\\
& {{\bar R}_{\ell , - m, - \omega }} = {R_{\ell m\omega }},
\end{align}
\end{subequations}
where the bar denotes complex conjugation.

\section{Asymptotic Formula of HeunC Function at Infinity}\label{sec:AsympForHeunC}
Up to now, an analytic asymptotic expression for the confluent Heun function $\mathbb{H}(x)$ at infinity  $|x|$ has not been reported in the literature. While most literature provides a linear combination of the two asymptotic solutions of the confluent Heun function at infinity, as in \cref{eq:hc-inf}, the coefficients ${{D}_{\otimes}^\beta }$ and ${{D}_{\odot}^\beta}$ remain undetermined.
The solution $Y(x)$ to the generalized spherical wave equation (GSWE\footnote{Some researchers\cite{Figueiredo2002,El_Jaick_2013} have named GSWE as confluent Heun equations, which is not used in this paper.} ) is related to $\mathbb{H}(x)$ by $Y(x)={{\rm{e}}^{i\omega x}}\mathbb{H}(x)$.
Additionally,  $Y(x)$ can be expressed as a series in terms of Coulomb wave functions ${F_{n + \nu }}(x)$ \cite{Leaver1986} that converges for $x>0$.
Therefore, an analytic asymptotic expression for $\mathbb{H}(x)$ at infinity can also be constructed in terms of a series solution in terms of ${F_{n + \nu }}(x)$.
However, the series in terms of ${F_{n + \nu }}(x)$ does not converge at $x=0$, which prohibits normalization with the asymptotic expression of $\mathbb{H}(x)$ at $x=0$.
Instead, a proportionality relation between the two can be established, but the proportionality coefficient $\Xi$ is undetermined.
To determine this coefficient $\Xi$, we can represent $\mathbb{H}(x)$ as a series solution in terms of hypergeometric functions ${}_2{F_1}(x)$ \cite{olver2010nist}, which converge at $x=0$.
Therefore, $\mathbb{H}(x)$ can be expressed as a series in terms of ${}_2{F_1}(x)$ and as a proportionality coefficient multiplied by a series solution in terms of ${F_{n + \nu }}(x)$.
By expanding both series in the interval $0< x < \infty $, the proportionality coefficient $\Xi$ can be determined.
This mathematical technique is similar to the approach used in Refs. \cite{Mano1996RWE,Mano_1996} to determine the analytic asymptotic amplitudes of $ R_{\ell m\omega }^{{\rm{in}}}$ at infinity.



\subsection{Expansion in Series of Hypergeometric Function}
The series expansion of the hypergeometric function can be utilized to represent the confluent Heun function\footnote{This paper assumes ${\rm{Im}}(\alpha ) > 0$. }.

\begin{align}\label{eq:HeunC-2F1}
 \mathbb{H} (x)&=  {\rm{HeunC}}(\alpha ,\beta ,\gamma ,\delta ,\eta ;x)  \nonumber \\
              &= \mathbb{F} \sum\limits_{n =  - \infty }^{+\infty}  {f{{_n^{\nu} }}\,{}_2{F_1}\left( {a, { b}, { c};x} \right)}
\end{align}
with $\mathbb{F}  ={\left(\sum\nolimits_{n =  - \infty }^\infty  {f_n^\nu } \right)^{ - 1}}$ is the normalized function at $x=0$. Here, $a = n + \nu  + 1 + \frac{{\beta  + \gamma }}{2}, { b} =  - n - \nu  + \frac{{\beta  + \gamma }}{2}$ and ${ c} = \beta  + 1$.
Substituting the series solution, as given by \cref{eq:HeunC-2F1}, into the confluent Heun equation \eqref{eq:HeunC}, we can derive the following three-term recurrence relation for the expansion coefficients $f_n^\nu$.
\begin{equation}\label{recurrence-fnv}
 {{\hat \alpha }_n}f_{n + 1}^\nu  + {{\hat \beta }_n}f_n^\nu  + {{\hat \gamma }_n}f_{n - 1}^\nu  = 0
\end{equation}
where
\begin{subequations}
\begin{align}
{{\hat \alpha }_n} &=   \frac{{\left( {2n + 2\nu  + 2 - \beta  + \gamma } \right)}}{{8\left( {n + \nu  + 1} \right)\left( {2n + 2\nu  + 3} \right)}}\nonumber\\
  &\times \left( {\alpha n + \alpha \nu  + \alpha  -\delta } \right)\left( {2n + 2\nu  + 2 - \gamma  - \beta } \right),\\
{{\hat \beta }_n} &= \eta  + \frac{\delta }{2} - \frac{{{\beta ^2}}}{4} - \frac{{{\gamma ^2}}}{4} + \left( {n + \nu } \right)\left( {n + \nu  + 1} \right) \nonumber\\
 &+ \frac{{\delta \left( {\gamma  + \beta } \right)\left( {\beta  - \gamma } \right)}}{{8\left( {n + \nu } \right)\left( {n + \nu  + 1} \right)}},\\
{{\hat \gamma }_n} &= -\frac{{\left( {2n + 2\nu  + \beta  - \gamma } \right)}}{{8\left( {n + \nu } \right)\left( {2n + 2\nu  - 1} \right)}} \nonumber\\
 &\times \left( {\alpha n + \alpha \nu  + \delta } \right)\left( {2n + 2\nu  + \gamma  + \beta } \right).
\end{align}
\end{subequations}
The phase parameter $\nu$, also known as the renormalized angular momentum, may be obtained by solving a characteristic equation expressed as the sum of two infinite continued fractions \cite{Leaver1986,Figueiredo1993,Figueiredo2002,El_Jaick_2013}.
\begin{align}\label{eq:eig}
{{\hat \beta }_0} &= \frac{{{{\hat \alpha }_{ - 1}}{{\hat \gamma }_0}}}{{{{\hat \beta }_{ - 1}} - }}\frac{{{{\hat \alpha }_{ - 2}}{{\hat \gamma }_{ - 1}}}}{{{{\hat \beta }_{ - 2}} - }}\frac{{{{\hat \alpha }_{ - 3}}{{\hat \gamma }_{ - 2}}}}{{{{\hat \beta }_{ - 3}} - }} +  \cdots\nonumber  \\
 &+ \frac{{{{\hat \alpha }_0}{{\hat \gamma }_1}}}{{{{\hat \beta }_1} - }}\frac{{{{\hat \alpha }_1}{{\hat \gamma }_2}}}{{{{\hat \beta }_2} - }}\frac{{{{\hat \alpha }_2}{{\hat \gamma }_3}}}{{{{\hat \beta }_3} - }} \cdots .
\end{align}



Solving the solution $\nu$ of \cref{eq:eig} is a complex task, because \cref{eq:eig} is a transcendental equation. Nevertheless, two approaches have been developed to determine $\nu$. The first method involves presenting a series expansion of $\nu$, without having to solve the transcendental equation directly \cite{Mano1996RWE,Mano_1996,Casals_2015}. However, this approach requires enforcing low-frequency constraints. The second method, originally introduced by Fujita and Tagoshi\cite{Fujita_2004,Fujita_2005}, utilizes the Steed algorithm for continued fractions to numerically solve the transcendental equation \eqref{eq:eig} and obtain $\nu$, without requiring any constraints. This paper selects the unconstrained second method over the first due to the limitations of the series expansion approach.

The series representation of the hypergeometric function is given by \cite{olver2010nist}
\begin{equation}\label{eq:2F1series}
{}_2{F_1}({{a}},{{b}},{{c}},\tilde x) = \sum\limits_{j = 0}^\infty  {\frac{{{{( a)}_j}{{(b)}_j}}}{{{{(c)}_j}}}{{\frac{{\tilde x}}{{j!}}}^j}} ,\quad 0 < \left| {\tilde x} \right| < 1
\end{equation}
where $({{a}})_n$ denotes the Pochhammer symbol defined as $({{a}})_j={{a}}({{a}}+1)({{a}}+2)\cdots({{a}}+j-1)$ with $({{a}})_0=1$. And $\tilde x$ is new variable, defined as $\tilde x = 1/(1-x)$.

Applying the linear transformation of hypergeometric functions ( Eq. (15.3.8) in Ref. \cite{abramowitz1948handbook} ), we can derive a relation between ${}_2{F_1}(x)$ and ${}_2{F_1}( \tilde x) $.
\begin{align}\label{eq:2F1Transformation}
{}_2{F_1}&\left( {{{a}},{{b}},{{c}};x} \right) =\nonumber  \\
 &{{\tilde x}^{{a}}}\frac{{\Gamma ({{c}})\Gamma ({{b}} - {{a}})}}{{\Gamma ({{b}})\Gamma ({{c}} - {{a}})}}{{\kern 1pt} _2}{F_1}\left( {{{a}},{{c}} - {{b}},{{a}} - {{b}} + 1;\tilde x} \right) \nonumber  \\
  +& {{\tilde x}^{{b}}}\frac{{\Gamma ({{c}})\Gamma ({{a}} - {{b}})}}{{\Gamma ({{a}})\Gamma ({{c}} - {{b}})}}{{\kern 1pt} _2}{F_1}\left( {{{b}},{{c}} - {{a}},{{b}} - {{a}} + 1;\tilde x} \right).
\end{align}


The confluent Heun function can be expressed in an alternate form by utilizing \Cref{eq:HeunC-2F1,eq:2F1Transformation}, as shown below:
\begin{equation}\label{eq:HeunC-hyper}
\mathbb{H}(x)= {\mathbb{H} _{n,\nu }}\left( {\tilde x} \right) + {\mathbb{H} _{ - n, - \nu  - 1}}\left( {\tilde x} \right),
\end{equation}
where
\begin{align}
& \begin{array}{l}\label{eq:Hnv}
{\mathbb{H} _{n,\nu }}\left( {\tilde x} \right) = \mathbb{F} {{\tilde x}^{ - \nu  + \frac{{\beta  + \gamma }}{2}}}\sum\limits_{n =  - \infty }^\infty  {\frac{{\Gamma ({ {c}})\Gamma ({ {b}} - { {a}})}}{{\Gamma ({ {b}})\Gamma ({ {c}} - { {a}})}}} \\
\quad \quad \times f_n^\nu {{\kern 1pt} _2}{F_1}\left( {{ {a}},{ {c}} - { {b}};{ {a}} - { {b}} + 1;\tilde x} \right),
\end{array} \\
 & \begin{array}{l}
{\mathbb{H} _{ - n, - \nu  - 1}}\left( {\tilde x} \right) = \mathbb{F} {{\tilde x}^{\nu  + 1 + \frac{{\beta  + \gamma }}{2}}}\sum\limits_{n =  - \infty }^\infty  {\frac{{\Gamma ({ {c}})\Gamma ({ {a}} - { {b}})}}{{\Gamma ({ {a}})\Gamma ({ {c}} - { {b}})}}} \\
\quad \quad  \times f_n^{ - \nu  - 1}{{\kern 1pt} _2}{F_1}\left( {{ {b}},{ {c}} - { {a}};{ {b}} - { {a}} + 1;\tilde x} \right).
\end{array}
\end{align}


%Then, by using \cref{eq:2F1series}, we can expand ${\mathbb{H}_{n,\nu}}(\tilde{x})$ to series of $\hat{z}$
%\begin{equation}\label{hyper-expand-1}
%{\mathbb{H} _{n,\nu }}\left( {\tilde x} \right) = \mathbb{F} {{\tilde x}^{ - \nu  + \frac{{\beta  + \gamma }}{2}}}\sum\limits_{n =  - \infty }^\infty  {\sum\limits_{j = 0}^\infty  {\frac{{\Gamma \left( {\beta  + 1} \right)\Gamma \left( {2n + 2\nu  + 1} \right){{\left( { - n - \nu  + \frac{{\beta  + \gamma }}{2}} \right)}_j}{{\left( { - n - \nu  - \frac{\delta }{\alpha }} \right)}_j}}}{{\Gamma \left( {n + \nu  + 1 + \frac{{\beta  + \gamma }}{2}} \right)\Gamma \left( {n + \nu  + 1 + \frac{{\beta  - \gamma }}{2}} \right){{\left( { - 2n - 2\nu } \right)}_j}\left( {j!} \right)}}f_n^\nu {{\left( {\tilde x} \right)}^{ - n + j}}} }
%\end{equation}
%Similarly, ${\mathbb{H}_{-n,-\nu-1}}(\tilde{x})$ can be expanded in a comparable series. Here, $\hat z $ is new coordinate variable $\hat z =   \frac{i}{2}\alpha \left( {1 - x} \right)$. Expanding Eq. \eqref{hyper-expand-1} to series of $\hat{z}$,
%%$\tilde x = {\left(- {\frac{{2i\hat z}}{\alpha }} \right)^{ - 1}}$
Accordingly, the recurrence relation \eqref{recurrence-fnv} possesses a structure such that ${f_n^{-\nu-1}}$ satisfies an identical recurrence relation to that of ${f_n^\nu}$.
Introducing a new coordinate variable $\hat z = 2i/(\alpha \tilde x)$ and applying \cref{eq:2F1series} into \cref{eq:Hnv} , we can expand ${\mathbb{H}_{n,\nu}}(\tilde{x})$ as a series in terms of $z$.
\begin{equation}\label{eq:hyper-expand-2}
  {\mathbb{H} _{n,\nu }}= \mathbb{F} {\left({\frac{{2i\hat z}}{\alpha }} \right)^{\nu  - \frac{{ \beta+ \gamma  }}{2}}}\sum\limits_{k =  - \infty }^\infty  {\sum\limits_{n = k}^\infty  {{C_{n,n - k}}{{\hat z}^k}} },
\end{equation}
where
\begin{align}
{C_{n,n - k}} &= \frac{{{{\left( { - n - \nu  + \frac{{\beta  + \gamma }}{2}} \right)}_{n - k}}{{\left( { - n - \nu  + \frac{\delta }{\alpha }} \right)}_{n - k}}}}{{\Gamma \left( {n + \nu  + 1 + \frac{{\beta  + \gamma }}{2}} \right)\Gamma \left( {n + \nu  + 1 + \frac{{\beta  - \gamma }}{2}} \right)}} \nonumber \\
& \times\frac{{\Gamma \left( {\beta  + 1} \right)\Gamma \left( {2n + 2\nu  + 1} \right)}}{{{{\left( { - 2n - 2\nu } \right)}_{n - k}}\left( {n - k} \right)!}}{\left( { - \frac{{i\alpha }}{2}} \right)^{ - k}}f_n^\nu,
\end{align}






\subsection{Expansion in Series of Coulomb Wave Function}\label{sec:Expan-S-CWF}

Assuming that ${\cal F}_{n,\nu }^C$ is a non-trivial solution of the confluent Heun equation \eqref{eq:HeunC}, which can be written as a series expansion of Coulomb wave functions. Here, $\hat z =   \frac{i}{2}\alpha \left( {x-1} \right)$.
And ${\cal F}_{n,\nu }^C$ is proportional to $\mathbb{H} ={\rm{HeunC}}(\alpha ,\beta ,\gamma ,\delta ,\eta ;x)$, such that $ \mathbb{H} \sim{\cal F}_{n,\nu }^C$. The expression of ${\cal F}_{n,\nu }^C$ can written as
\begin{equation}\label{eq:HeunC-FC}
{\cal F}_{n,\nu }^C(\hat z) = {\left( {\frac{{ \alpha }}{2}} \right)^\tau }{{\rm{e}}^{ - \frac{{i\pi \tau }}{2} - \frac{\alpha }{2} + i\hat z}}{{\hat z}^{ - \frac{{\gamma  + \beta  + 2}}{2}}}f_{n,\nu }^C(\hat z),
\end{equation}
where $\tau  = \frac{1}{4}\left( {3\beta  + \gamma  + \frac{{2\delta }}{\alpha }} \right)$, and $f_{n,\nu }^C$ can be defined as
\begin{equation}\label{fnnc}
f_{n,\nu }^C = \mathbb{F} \sum\limits_{n =  - \infty }^\infty  {{{( - i)}^n}\frac{{{{(\nu  + 1 + i\hat \eta )}_n}}}{{{{(\nu  + 1 - i\hat \eta )}_n}}}f_n^\nu } {F_{n + \nu }}(\hat \eta ,\hat z),
\end{equation}
in which the Coulomb wave function ${F_{n + \nu }}(\hat \eta ,\hat z)$ can be defined as
\begin{equation}
{F_{n + \nu }}(\hat \eta ,\hat z) = {2^{n + \nu }}{{\hat z}^{n + \nu  + 1}}{e^{ - i\hat z}}\frac{{\Gamma (\hat a)}}{{\Gamma (\hat c)}}\Phi \left( {\hat a,\hat c;2i\hat z} \right),
\end{equation}
with $ \hat a = n + \nu  + 1 + \frac{{\delta }}{\alpha }$ , $\hat c = 2n + 2\nu  + 2,$  and $ i{\hat \eta}  =-\frac{{\delta }}{\alpha }$, $\Phi$ is the regular confluent hypergeometric function which can be represented by the following series expansion \cite{olver2010nist}:
\begin{equation}
  \Phi \left( {\hat a,\hat c;2i\hat z} \right) = \sum\limits_{j = 0}^\infty  {\frac{{{{(\hat a)}_j}}}{{{{(\hat c)}_j}}}{{\frac{{\left( {2i\hat z} \right)}}{{j!}}}^j}}.
\end{equation}
Expanding $f_{n,\nu }^C$ to series of $\hat{z}$
\begin{equation}\label{eq:fc-series}
f_{n,\nu }^C = \mathbb{F} {e^{ - i\hat z}}{2^\nu }{{\hat z}^{\nu  + 1}}\sum\limits_{k =  - \infty }^\infty  {\sum\limits_{n =  - \infty }^k {{D_{n,n - k}}} } {{\hat z}^k},
\end{equation}
where
\begin{align}
{D_{n,k - n}} &= \frac{{\Gamma \left( {n + \nu + 1 + \frac{\delta }{\alpha }} \right){{\left( {\nu  + 1 - \frac{\delta }{\alpha }} \right)}_n}}}{{\Gamma \left( {2n + 2\nu  + 2} \right){{\left( {\nu  + 1 + \frac{\delta }{\alpha }} \right)}_n}}}\nonumber\\
& \times \frac{{{{\left( {n + \nu  + 1 + \frac{\delta }{\alpha }} \right)}_{k - n}}}}{{{{(2n + 2\nu  + 2)}_{k - n}}(k - n)!}}{( - 1)^n}{(2i)^k}f_n^\nu .
\end{align}

Because $f_{n,\nu }^C ({\hat z})$ is convergent at infinity, then we discuss its analytic asymptotic formula at infinity.
There is the analytic property \cite{Figueiredo2002,El_Jaick_2013,bateman1953higher} of the confluent hypergeometric function
\begin{align}
\Phi (\hat a,\hat c;2i\hat z) &= \frac{{\Gamma (\hat c)}}{{\Gamma (\hat c - \hat a)}}{e^{i\epsilon \hat a\pi }}\Psi (\hat a,\hat c;2i\hat z)\\
 &+ \frac{{\Gamma (\hat c)}}{{\Gamma (\hat a)}}{e^{i\pi (\hat a - \hat c)\epsilon }}{e^{2i\hat z}}\Psi (\hat c - \hat a,\hat c; - 2i\hat z), \nonumber
\end{align}
with
\begin{equation}
  {\epsilon  = {\rm{sgn}}\left( {{\rm{Im}}(2i\hat z)} \right) = \left\{ {\begin{array}{*{20}{l}}
{1,\quad \quad {\rm{if}}\quad {\rm{Im}}(2i\hat z) > 0,}\\
{ - 1,\quad {\rm{if}}\quad {\rm{Im}}(2i\hat z) < 0,}
\end{array}} \right.}
\end{equation}
where  $\Psi$ is the irregular confluent hypergeometric function. Then, Eq. (\ref{fnnc}) can be expressed as
\begin{subequations}
  \begin{align}
&f_{n,\nu }^C =  f_{n,\nu }^ \odot +f_{n,\nu }^ \otimes \label{eq:fnC2} \\
&\begin{array}{l}
f_{n,\nu }^ \odot  ={2^\nu } \mathbb{F} {e^{i\pi \left( {\nu  + 1+ \frac{\delta }{\alpha }} \right)}}{{\hat z}^{\nu  + 1}}{e^{ - i\hat z}}\frac{{\Gamma \left( {\nu  + 1 + \frac{\delta }{\alpha }} \right)}}{{\Gamma \left( {\nu  + 1 - \frac{\delta }{\alpha }} \right)}}\\
\quad \quad  \times \sum\limits_{n =  - \infty }^\infty  {f_n^\nu {{(2i\hat z)}^n}} \Psi \left( {\hat a,\hat c,2i\hat z} \right),
\end{array}\\
&\begin{array}{l}
f_{n,\nu }^ \otimes  = {2^\nu }\mathbb{F}{e^{ - i\pi \left( {\nu  + 1 - \frac{\delta }{\alpha }} \right)}}{{\hat z}^{\nu  + 1}}{e^{i\hat z}}\sum\limits_{n =  - \infty }^\infty  {\frac{{{{\left( {\nu  + 1 - \frac{\delta }{\alpha }} \right)}_n}}}{{{{\left( {\nu  + 1 + \frac{\delta }{\alpha }} \right)}_n}}}} \\
\quad \quad  \times {( - 1)^n}f_n^\nu {( - 2i\hat z)^n}\Psi (\hat c - \hat a,\hat c; - 2i\hat z).
\end{array}
\end{align}
\end{subequations}

%\begin{align}
%f_{n,\nu }^C =  f_{n,\nu }^ \odot +f_{n,\nu }^ \otimes \label{fnC2}
%\\
%  f_{n,\nu }^ \odot = \mathbb{F} {2^\nu }{e^{i\pi \left( {\nu  + 1 + \frac{\delta }{\alpha }} \right)}}{{\hat z}^{\nu  + 1}}{e^{ - i\hat z}}\frac{{\Gamma \left( {\nu  + 1 + \frac{\delta }{\alpha }} \right)}}{{\Gamma \left( {\nu  + 1 - \frac{\delta }{\alpha }} \right)}}\sum\limits_{n =  - \infty }^\infty  {f_n^\nu {{(2i\hat z)}^n}} &\Psi \left( {n + \nu  + 1 + \frac{\delta }{\alpha },2n + 2\nu  + 2,2i\hat z} \right) \\
%    f_{n,\nu }^ \otimes  = \mathbb{F} {2^\nu }{e^{ - i\pi \left( {\nu  + 1 - \frac{\delta }{\alpha }} \right)}}{{\hat z}^{\nu  + 1}}{e^{i\hat z}}\sum\limits_{n =  - \infty }^\infty  {{{( - 1)}^n}\frac{{{{\left( {\nu  + 1 - \frac{\delta }{\alpha }} \right)}_n}}}{{{{\left( {\nu  + 1 + \frac{\delta }{\alpha }} \right)}_n}}}f_n^\nu {{( - 2i\hat z)}^n}}& \Psi (n + \nu  + 1 - \frac{\delta }{\alpha },2n + 2\nu  + 2; - 2i\hat z).
%\end{align}
By taking into account the asymptotic behavior \cite{olver2010nist} of $\Psi\left( {\hat a,\hat c;2i\hat z} \right)$ at large $|x|$:
\begin{equation}\label{irrch}
  \mathop {\lim }\limits_{x \to \infty } \Psi\left( {\hat a,\hat c;2i\hat z} \right)\rightarrow{(2i\hat z)^{ - \hat{a}}},
\end{equation}
we can obtain the asymptotic analytic expression of $f_{n,\nu }^C $ at infinity:
\begin{subequations}
  \begin{align}
&f_{n,\nu }^ \odot  = A_{n,\nu }^ \odot {x^{ - \frac{\delta }{\alpha }}}{{\rm{e}}^{\frac{\alpha }{2}x}},\quad f_{n,\nu }^ \otimes  = A_{n,\nu }^ \otimes {x^{\frac{\delta }{\alpha }}}{{\rm{e}}^{ - \frac{\alpha }{2}x}},\\
& \begin{array}{l}
A_{n,\nu }^ \odot  = {{\rm{e}}^{ - \frac{\alpha }{2}}}{\left( {\frac{{i\alpha }}{2}} \right)^{ - \frac{\delta }{\alpha }}}\tilde A_{n,\nu }^ \odot ,{\kern 1pt} \\
A_{n,\nu }^ \otimes  = {{\rm{e}}^{\frac{\alpha }{2}}}{\left( {\frac{{i\alpha }}{2}} \right)^{\frac{\delta }{\alpha }}}\tilde A_{n,\nu }^ \otimes .
\end{array}
\end{align}
\end{subequations}
where
\begin{subequations}
  \begin{align}
&\tilde A_{n,\nu }^ \odot  = {2^{ - 1 - \frac{\delta }{\alpha }}}{{\rm{e}}^{\frac{{i\pi }}{2}\left( {\nu  + 1 + \frac{\delta }{\alpha }} \right)}}\frac{{\Gamma \left( {\nu  + 1 + \frac{\delta }{\alpha }} \right)}}{{\Gamma \left( {\nu  + 1 - \frac{\delta }{\alpha }} \right)}},\\
&\begin{array}{l}
\tilde A_{n,\nu }^ \otimes  = {2^{ - 1 + \frac{\delta }{\alpha }}}{{\rm{e}}^{ - \frac{{i\pi }}{2}\left( {\nu  + 1 - \frac{\delta }{\alpha }} \right)}}{\left( {\sum\limits_{n =  - \infty }^{ + \infty } {f_n^\nu } } \right)^{ - 1}}\\
\quad \quad \times
\left( {\sum\limits_{n =  - \infty }^{ + \infty } {{{( - 1)}^n}} \frac{{{{\left( {\nu  + 1 - \frac{\delta }{\alpha }} \right)}_n}}}{{{{\left( {\nu  + 1 + \frac{\delta }{\alpha }} \right)}_n}}}f_n^\nu } \right).
\end{array}
\end{align}
\end{subequations}

Analogous to \cref{eq:fnC2}, we can derive the asymptotic analytical expression of $f_{-n,-\nu-1}^C$ as $|x| \rightarrow \infty$,
\begin{equation}
\begin{array}{*{20}{l}}
{f_{ - n, - \nu  - 1}^C = {{\rm{e}}^{i\pi \left( {\nu  + \frac{1}{2}} \right)}}f_{n,\nu }^ \otimes }\\
{\quad \quad \quad \quad \quad  + \frac{{\sin \pi \left( {\nu  + \frac{\delta }{\alpha }} \right)}}{{\sin \pi \left( {\nu  - \frac{\delta }{\alpha }} \right)}}{{\rm{e}}^{ - i\pi \left( {\nu  + \frac{1}{2}} \right)}}f_{n,\nu }^ \odot  }.
\end{array}
\end{equation}


\subsection{Proportionality Coefficient}\label{sec:PC}
Both solutions, ${\mathbb{H} _{n,\nu }}$ and ${{\cal F}_{n,\nu }^C}$, converge within an extremely wide region of $\hat z$.
As $k$ is an arbitrary integer, we set $k=0$, and find that the series representations ( \Cref{eq:hyper-expand-2,eq:HeunC-FC,eq:fc-series} ) of ${\mathbb{H} _{n,\nu }}$ and ${{\cal F}_{n,\nu }^C}$ are proportional to the same single-valued function of $\hat z$.
Therefore, the analytical properties of ${\mathbb{H} _{n,\nu }}$ and ${{\cal F}_{n,\nu }^C}$ are identical, indicating that these two solutions are equivalent up to a multiplicative constant.
The proportional coefficient between ${{\mathbb{H} _{n,\nu }}}$ and ${{{\cal F}_{n,\nu }^C }}$ can now be determined
\begin{equation}\label{eq:Xi1}
\Xi _{n,\nu }^\beta  = \frac{{{\mathbb{H} _{n,\nu }}}}{{{\cal F}_{n,\nu }^C}} = {2^{ - \nu }}{\left( {\frac{\alpha }{2}} \right)^{ - \hat \tau }}{{\rm{e}}^{\frac{{i\pi \hat \tau  + \alpha }}{2}}}\frac{{\sum\limits_{j = 0}^\infty  {{C_{n,n - k}}} }}{{\sum\limits_{n =  - \infty }^k {{D_{n,n - k}}} }},
\end{equation}
where $\hat \tau  = \frac{{\beta  - \gamma }}{4} + \nu  + \frac{\delta }{{2\alpha }}.$

For the convenience of calculation, we set $k = 0$, so Eq. \eqref{eq:Xi1} is simplified as
\begin{align}\label{eq:Xinv}
&\begin{array}{*{20}{l}}
{\Xi _{n,\nu }^\beta  = \frac{{{2^{ - \nu }}{{\left( {\frac{\alpha }{2}} \right)}^{ - \hat \tau }}{{\rm{e}}^{\frac{{i\pi \hat \tau  + \alpha }}{2}}}\Gamma \left( {\beta  + 1} \right)\Gamma \left( {2\nu  + 2} \right)}}{{\Gamma \left( {\nu  + 1 + \frac{\delta }{\alpha }} \right)\Gamma \left( {\nu  + 1 - \frac{{\beta  + \gamma }}{2}} \right)\Gamma \left( {\nu  + 1 + \frac{{\gamma  - \beta }}{2}} \right)}}}\\
{ \times {{\Big( {\sum\limits_{n =  - \infty }^0 {\frac{{{{( - 1)}^n}{{\left( {\nu  + 1 - \frac{\delta }{\alpha }} \right)}_n}}}{{( - n)!{{\left( {2\nu  + 2} \right)}_n}{{\left( {\nu  + 1 + \frac{\delta }{\alpha }} \right)}_n}}}f_n^\nu } } \Big)}^{ - 1}}}
\end{array}\\
&\begin{array}{l}
{ \times \Big( {\sum\limits_{n = 0}^\infty  {{{\left( { - 1} \right)}^n}\frac{{\Gamma \left( {n + 2\nu  + 1} \right)\Gamma \left( {n + \nu  + 1 + \frac{{\gamma  - \beta }}{2}} \right)\Gamma \left( {n + \nu  + 1 - \frac{{\beta  + \gamma }}{2}} \right)}}{{( n!)\Gamma \left( {n + \nu  + 1 + \frac{{\beta  - \gamma }}{2}} \right)\Gamma \left( {n + \nu  + 1 + \frac{{\beta  + \gamma }}{2}} \right)}}f_n^\nu } } \Big)}.
\end{array}\nonumber
\end{align}
To obtain the proportional coefficients $\Xi _{ - n, - \nu  - 1}^{\beta}$ for ${\mathbb{H} _{-n,-\nu-1 }}$ and ${{\cal F}_{-n,-\nu-1 }^C}$, we can simply substitute $n\Rightarrow-n$ and $\nu\Rightarrow-\nu-1$  into \cref{eq:Xinv}.


\subsection{Infinite Asymptotic Behavior of HeunC Function}
Based on the previous sections, we are now able to derive the analytical asymptotic expression for the confluence Heun function at infinity. Specifically, we can rewrite \cref{eq:hc-inf} as follows.

\begin{align}\label{eq:87}
\mathop {\lim }\limits_{|x| \to \infty } \mathbb{H} (x) &= \Xi _{n,\nu }^\beta {\cal F}_{n,\nu }^C + \Xi _{ - n, - \nu  - 1}^\beta {\cal F}_{ - n, - \nu  - 1}^C\nonumber\\
 &= D_ \odot^\beta \;{x^{ - \frac{{\beta  + \gamma  + 2}}{2} - \frac{\delta }{\alpha }}} + D_\otimes ^\beta {{\rm{e}}^{ - \alpha x}}{x^{ - \frac{{\beta  + \gamma  + 2}}{2} + \frac{\delta }{\alpha }}} .
\end{align}

Substituting the results from \cref{sec:Expan-S-CWF,sec:PC} into \cref{eq:87}, the constants $D_ \odot ^\beta$ and $D_ \otimes ^\beta$ in the asymptotic behavior at infinity (\ref{eq:hc-inf}) are given by
\begin{align}
{}&\begin{array}{l}\label{eq:Dinc}
D_ \odot ^\beta  = \Xi _{n,\nu }^\beta D_{ \odot ,n,\nu }^\beta \\
 \quad \quad+ {{\rm{e}}^{ - i\pi \left( {\nu  + \frac{1}{2}} \right)}}\frac{{\sin \pi \left( {\nu  + \frac{\delta }{\alpha }} \right)}}{{\sin \pi \left( {\nu  - \frac{\delta }{\alpha }} \right)}}\Xi _{ - n, - \nu  - 1}^\beta D_{ \odot , - n, - \nu  - 1}^\beta ,
\end{array}\\
{}&{D_ \otimes ^\beta  = \Xi _{n,\nu }^\beta D_{ \otimes ,n,\nu }^\beta  + {{\rm{e}}^{i\pi \left( {\nu  + \frac{1}{2}} \right)}}\Xi _{ - n, - \nu  - 1}^\beta D_{ \otimes , - n, - \nu  - 1}^\beta ,}\label{eq:Dout}
\end{align}
with
\begin{align}
D_{ \odot ,n,\nu }^\beta  &= {\left( { - 1} \right)^{\frac{{\gamma  + \beta  + 2}}{2} + \frac{\delta }{\alpha }}}{\big( {\frac{\alpha }{2}} \big)^\tau } {\big( -{\frac{{i\alpha }}{2}} \big)^{ - \frac{{\gamma  + \beta  + 2}}{2} - \frac{\delta }{\alpha }}}{{\rm{e}}^{ - \frac{{i\pi \tau +\alpha}}{2}}}
\nonumber \\
&\times
{2^{ - 1 - \frac{\delta }{\alpha }}}{{\rm{e}}^{\frac{{i\pi }}{2}\big( {\nu  + 1 + \frac{\delta }{\alpha }} \big)}} \Xi _{n,\nu }^\beta\frac{{\Gamma \big( {\nu  + 1 + \frac{\delta }{\alpha }} \big)}}{{\Gamma \big( {\nu  + 1 - \frac{\delta }{\alpha }} \big)}}, \\
 D_{  \otimes,n,\nu }^\beta  &= {\big( { - 1} \big)^{\frac{{\gamma  + \beta  + 2}}{2} - \frac{\delta }{\alpha }}}{\left( {\frac{\alpha }{2}} \right)^\tau } {\left(- {\frac{{i\alpha }}{2}} \right)^{ - \frac{{\gamma  + \beta  + 2}}{2} + \frac{\delta }{\alpha }}}\nonumber \\
&\times  {{\rm{e}}^{ - \frac{{i\pi \tau - \alpha}}{2}}}\Xi _{n,\nu }^\beta  \frac{ {2^{ - 1 + \frac{\delta }{\alpha }}}{{\rm{e}}^{ - \frac{{i\pi }}{2}\big( {\nu  + 1 - \frac{\delta }{\alpha }} \big)}}} {\sum\limits_{n =  - \infty }^{ + \infty } {f_n^\nu } }   \nonumber \\
&\times\Big( {\sum\limits_{n =  - \infty }^{ + \infty } {{{( - 1)}^n}} \frac{{{{\big( {\nu  + 1 - \frac{\delta }{\alpha }} \big)}_n}}}{{{{\big( {\nu  + 1 + \frac{\delta }{\alpha }} \big)}_n}}}f_n^\nu } \Big).
\end{align}





\section{Application and Correctness of Analytical Solutions}\label{sec:Application}




Using the proposed analytical solution in this paper to calculate the gravitational radiation of a specific black hole is an excellent way to test its correctness. By comparing the results obtained from our method with various methods described in the literature, we can assess the accuracy and reliability of our method. This comparative analysis will help validate the effectiveness of our proposed method and provide further confidence in its correctness.




\subsection{Application to Schwarzschild BHs}

For simplicity but without loss of generality, we shall consider the gravitational radiation of the Schwarzschild black hole as an illustrative example, and we shall provide the complete solution for the purely ingoing wave at the horizon and the purely outgoing wave at infinity, along with their respective amplitudes.
In the end, the method proposed in this paper is put into practical use for the computation of the energy fluxes and waveforms of the gravitational radiation from a particle in circular orbit around a Schwarzschild BH.

\subsubsection{Analytical Solutions of Schwarzschild BHs}
We consider the case when a test particle of mass $\mu$ travels a circular orbit around a Schwarzschild black hole of mass $M\gg \mu$.
To calculate the gravitational fluxes and waveforms, we consider the radial Teukolsky equation of the Schwarzschild BH with the point source, as given by \cref{eq:HRTE}, reduces to:

\begin{align}\label{eq:Steuk}
&\Delta R_{\ell m\omega }^{\prime \prime } + 2(r - M)(s + 1)R_{\ell m\omega }^\prime \\
 &+ \left[ {r^2}\left( {{\omega ^2}{r^2} - 4i\omega (r - 3M)} \right) - \Delta  {\lambda} \right]{R_{\ell m\omega }} = T_{\ell m\omega}, \nonumber
\end{align}
where $\Delta  = r( r - {r_{\rm H}} )$, and ${r_{\rm H}}=2M$. The source term $T_{\ell m\omega}$ for the gravitational perturbation is explicitly given by Eq. (18) of Ref. \cite{Sasaki_2003}.

Using Green's function method, it is necessary to construct  two linear independent solutions of \cref{eq:Steuk}, denoted as $R^{\rm in}_{\ell\omega}(r)$ and $R^{\rm up}_{\ell\omega}(r)$, that satisfy the following boundary conditions:
\begin{align}
    & R_{\ell m\omega }^{{\rm{in}}}\to \left\{ {
    \begin{array}{*{20}{l}}
{B_{\ell m\omega }^{{\rm{trans}}}{\Delta ^{ - s}}{{\rm{e}}^{ - i\omega{r^*}}},}&{r \to r_{\rm H}},\\
{\frac{{B_{\ell m\omega }^{{\rm{ref}}}}}{{{r^{2s + 1}}}}{{\rm{e}}^{i\omega {r^*}}} + \frac{{B_{\ell m\omega }^{{\rm{inc}}}}}{r}{{\rm{e}}^{ - i\omega {r^*}}},}&{r \to  + \infty ,}
\end{array}
    } \right.\label{eq:boundary1}\\
 & R_{\ell m\omega }^{{\rm{up}}} \to \left\{ {\begin{array}{*{20}{l}}
{C_{\ell m\omega }^{{\rm{up}}}{{\rm{e}}^{i\omega{r^*}}} + \frac{{C_{\ell m\omega }^{{\rm{ref}}}}}{{{\Delta ^s}}}{{\rm{e}}^{ - i\omega{r^*}}},}&{r \to {r_{\rm H} },}\\
{\frac{{C_{\ell m\omega }^{{\rm{trans}}}}}{{{r^{2s + 1}}}}{{\rm{e}}^{i\omega {r^*}}},}&{r \to  + \infty ,}
\end{array}} \right.\label{eq:boundary2}
\end{align}
where
$ r^{*}=r+r_{\rm H} \ln(r/r_{\rm H}-1)$
and $s=-2$.

In the case of a circular orbit, the specific energy $\tilde E$ and angular momentum $\tilde L_z$ of the particle are given by
\begin{align}
    &\tilde E =(r_0-2M)/\sqrt{r_0(r_0-3M)}, \label{eq:ene}\\
 & \tilde L_z =\sqrt{Mr_0}/\sqrt{1-3M/r_0},
\end{align}
where $r_0$ is the orbital radius. The angular frequency is given by $\Omega=(M/r_0^3)^{1/2}$ , so orbital frequency is $\omega = m \Omega$.



The solutions $R_{\ell m\omega }^{{\rm{in,up}}}$ proposed in \Cref{sec:in-outgongsol} are the analytical solutions satisfying the boundary conditions \eqref{eq:boundary1} and \eqref{eq:boundary2} of the Schwarzschild BH.
To facilitate comparisons with other methods, we need normalize $R_{\ell m\omega }^{{\rm{in}}}$ at the event horizon and $R_{\ell m\omega }^{{\rm{up}}}$ at infinity, obtaining normalized asymptotic amplitudes of homogeneous equations \eqref{eq:HRTE}.
Using the normalized condition $B_{\ell m\omega }^{{\rm{trans}}} = 1$ at event horizon, the the ingoing wave solution \eqref{eq:uHor} is normalized to obtain
\begin{equation}
\tilde R_{\ell m\omega }^{{\rm{in}}} =  S_{\rm H}^\beta \left( x \right){\rm{HeunC}}\left( {\alpha ,\beta ,\gamma ,\delta ,\eta ;x} \right),
\end{equation}
where the normalized S-homotopic transformation at horizon is
\begin{equation}
 S_{\rm H}^\beta \left( x \right) =r^{-2s}_{\rm{H}} {{\rm{e}}^{\frac{1}{2}\alpha \left( {x - 1} \right)}}{{{\left( { - x} \right)}^{\frac{1}{2}\left( {\beta  - s} \right)}}{{\left( {1 - x} \right)}^{\frac{1}{2}\left( {\gamma  - s} \right)}}}.
\end{equation}
By utilizing the asymptotic behavior \eqref{eq:boundary1} of the solution $\tilde R_{\ell m\omega }^{{\rm{in }}}$ as $r\rightarrow \infty$, we derive analytic expressions for the asymptotic amplitudes $B_{\ell m\omega }^{\rm inc}$ and $B_{\ell m\omega }^{\rm ref}$.
\begin{subequations}
  \begin{align}
  B_{\ell m\omega }^{{\rm{inc}}} &=  - r_{\rm{H}}^5D_ \odot ^\beta,\\
   B_{\ell m\omega }^{{\rm{ref}}} &= {\left( { - 1} \right)^{2i\omega {r_{\rm{H}}} - 1}}{r_{\rm{H}}}{{\rm{e}}^{ - 2i\omega {r_{\rm{H}}}}}D_ \otimes ^\beta .
\end{align}
\end{subequations}


Meanwhile, using the normalized condition $C_{\ell m\omega }^{{\rm{trans}}} = 1$ at infinity, the solution of the outgoing wave is normalized to obtain


\begin{align}
\tilde R_{\ell m\omega }^{{\rm{up}}} = & - \frac{{D_ \odot ^{ - \beta }}}{{D_ \odot ^\beta }}\tilde D S_\infty ^\beta (x){\rm{HeunC}}(\alpha ,\beta ,\gamma ,\delta ,\eta ;x) \nonumber \\
&+ \tilde D S_\infty ^{ - \beta }(x){\rm{HeunC}}(\alpha , - \beta ,\gamma ,\delta ,\eta ;x),
\end{align}
where the normalized S-homotopic transformation at infinity is
\begin{equation}
S_\infty ^\beta (x) =  - r_{\rm{H}}^{ - 3}{{\rm{e}}^{\frac{1}{2}\alpha \left( {x - 1} \right)}}{\left( { - x} \right)^{\frac{1}{2}(\beta  - s)}}{\left( {1 - x} \right)^{\frac{1}{2}(\gamma  - s)}},
\end{equation}
and
\[\tilde D = {\left( {D_ \otimes ^{ - \beta } - \frac{{D_ \odot ^{ - \beta }}}{{D_ \odot ^\beta }}D_ \otimes ^\beta } \right)^{ - 1}}.\]
By utilizing the asymptotic behavior \eqref{eq:boundary2} of the solution $\tilde R_{\ell m\omega }^{{\rm{up }}}$ as $r\rightarrow r_{\rm{H}}$, we derive analytic expressions for the asymptotic amplitudes $C_{\ell m\omega }^{{\rm{ref}}}$ and $C_{\ell m\omega }^{{\rm{up}}}$.
\begin{subequations}
  \begin{align}
   C_{\ell m\omega }^{{\rm{up}}} &=  - {r}^3_{\rm{H}}\tilde D,\\
   C_{\ell m\omega }^{{\rm{ref}}} &= \frac{{{{\left( { - 1} \right)}^{ - 2i\omega {r_{\rm{H}}}}}}}{{{r_{\rm{H}}}}}\frac{{D_ \odot ^{ - \beta }}}{{\;D_ \odot ^\beta }}{{\rm{e}}^{2i\omega {r_{\rm{H}}}}}\tilde D.
\end{align}
\end{subequations}

For the ingoing wave solution $\tilde R_{\ell m\omega }^{{\rm{in }}}$ and outgoing wave solution $\tilde R_{\ell m\omega }^{{\rm{up }}}$, the parameters ($\alpha,\beta,\gamma,\delta$, $\eta$) of these two solutions can be seen in \Cref{table:parameter}.












\subsubsection{Flux and Waveform}
The asymptotic amplitudes $\tilde Z_{\ell m\omega}^{{\rm H},\infty}$ of inhomogeneous equations \eqref{eq:GFoTRE} can be employed to precisely obtain the radiation energy flux and waveform of gravitational waves.
Substituting the point sources $T_{\ell m\omega}$ into \Cref{eq:Hsol,eq:InfSol} respectively, we can obtain the asymptotic amplitudes $\tilde Z_{\ell m\omega}^{{\rm H},\infty}$.
\begin{align}
\tilde Z_{\ell m\omega }^{\rm{H}} &= \frac{{\mu B_{lm\omega }^{{\rm{trans}}}}}{{2i\omega C_{lm\omega }^{{\rm{trans}}}B_{lm\omega }^{{\rm{inc}}}}}\int_{ - \infty }^\infty  d t{e^{i\omega t - im\varphi (t)}}{\cal I}_{\ell m\omega }^H\;,\label{eq:Zh}\\
\tilde Z_{\ell m\omega }^\infty  &= \frac{\mu }{{2i\omega B_{lm\omega }^{{\rm{inc}}}}}\int_{ - \infty }^\infty  d t{e^{i\omega t - im\varphi (t)}}{\cal I}_{\ell m\omega }^\infty \;,\label{eq:Zinf}
\end{align}
where
\begin{subequations}
\begin{align}
{\cal I}_{\ell m\omega }^{\rm{H}} &= \left[ {\left( {{A_{nn0}} + {A_{\bar mn0}}+{A_{\bar m\bar m0}}}\right)R_{\ell m\omega }^{{\rm{up}}}} \right. \nonumber \\
 &- \left( {{A_{\bar mn1}} + {A_{\bar m\bar m1}}} \right){\left( {R_{\ell m\omega }^{{\rm{up}}}} \right)^\prime }\\
 &{\left. + {A_{\bar m\bar m2}}{{\left( {R_{\ell m\omega }^{{\rm{up}}}} \right)}^{\prime \prime }} \right]_{r = r(t),\theta  = \theta (t)}},  \nonumber \\
{\cal I}_{\ell m\omega }^\infty  &= \left[ {\left( {{A_{nn0}} + {A_{\bar mn0}} + {A_{\bar m\bar m0}}}\right)R_{\ell m\omega }^{{\rm{in}}}}\right.  \nonumber\\
 &- \left( {{A_{\bar mn1}} + {A_{\bar m\bar m1}}} \right){\left( {R_{\ell m\omega }^{{\rm{in}}}} \right)^\prime }\\
&{\left. { + {A_{\bar m\bar m2}}{{\left( {R_{\ell m\omega }^{{\rm{in}}}} \right)}^{\prime \prime }}} \right]_{r = r(t),\theta  = \theta (t)}}. \nonumber
\end{align}
\end{subequations}
where the explicit form of $A_{nn0}$ and other terms can be found in Eqs. (37) to (42) of Ref. \cite{Sasaki_2003}.




If a particle follows the bound geodesics of the Schwarzschild spacetime, leading to a discrete frequency spectrum of $T_{\ell m\omega}$. In the case of circular orbits, $Z_{\ell m\omega}^{\infty,{\rm H}}$ in \Cref{eq:Zh,eq:Zinf} takes the form
\begin{equation}
  {\tilde Z}^{\infty,{\rm H}}_{\ell m\omega}= Z^{\infty,{\rm H}}_{\ell m\omega}\,\delta(\omega-m\,\Omega),
\label{eq:tildeZ}
\end{equation}
where $ Z_{\ell m\omega}^{\infty,{\rm H}}$ is essential to the accurate computation of gravitational wave energy flux and waveform.
The time-averaged gravitational wave luminosity (energy flux) at infinity is provided by \cite{Teukolsky1974Perturbations}
\begin{equation}\label{eq:dEdtInf}
 \left<{dE\over dt}\right>_\infty =\sum_{\ell=2}^{\infty} \sum_{m=-\ell}^{\ell}
   \frac{\vert Z^\infty_{\ell m\omega}\vert^2}{4\pi \omega^2},
\end{equation}
where $\left<\cdots\right>$ represents the time average and  $\omega =m\Omega$.

Similarly, the time-averaged gravitational wave luminosity at the horizon becomes\cite{Teukolsky1974Perturbations,Fujita_2015}
\begin{equation}\label{eq:dEdtH}
 \left<{dE\over dt}\right>_{\rm H} =\sum_{\ell=2}^{\infty} \sum_{m=-\ell}^{\ell}
  \alpha_{\ell m\omega}\frac{\vert  Z^{\rm H}_{\ell m\omega}\vert^2}{4\pi \omega^2},
\end{equation}
where
\begin{equation}
{\alpha _{\ell m\omega }} = \frac{{256{{(2M{r_{\rm{H}}})}^5}\omega ({\omega ^2} + 4{{\tilde \epsilon }^2})({\omega ^2} + 16{{\tilde \epsilon }^2}){\omega ^3}}}{{|{\bf{C}}{|^2}}},
\label{eq:alphaH}
\end{equation}
with $\tilde \epsilon  = {(4{r_{\rm{H}}})^{ - 1}}$ and $|{\bf{C}}{|^2} = {\lambda ^2}{(\lambda  + 2)^2}{\kern 1pt}  + 144{\kern 1pt} {\omega ^2}{\kern 1pt} {M^2}.$


Finally, the gravitational waveforms are given in terms of
$\tilde Z_{\ell m\omega}^{\infty}$ as

\begin{align}\label{eq:hpm_slm}
h_{+}-i\,h_{\times }&=-\frac{2}{r}\,\sum _{\ell,m} \frac{\tilde Z^\infty_{\ell m\omega}}{\omega^2}\frac{e^{i m \varphi}}{\sqrt{2\pi}}\,_{-2}S_{\ell m}^{a\omega}(\theta)\,e^{i \omega (r^{*}-t)} \nonumber\\
  &\equiv\sum_{\ell,m}(h_{lm}^+-i h_{\ell m}^\times).
\end{align}

The significant application of our method is to calculate gravitational radiation in which both $R_{\ell m\omega }^{{\rm{in}}}$  and $R_{\ell m\omega }^{{\rm{up}}}$ can be obtained using Eq. (\ref{eq:D2GSol1}) directly with the boundary conditions \eqref{eq:boundary1} and \eqref{eq:boundary2}.
Our method differs from traditional results obtained through post-Newtonian expansion and post-Minkowskian expansion, as it provides complete results without requiring series expansion. Theoretically, our method should be more efficient and accurate than the expansion method, and it has a broader scope of application as it is not limited by physical constraints such as low frequency, slow-motion or weak-field limits.







% Table generated by Excel2LaTeX from sheet 'Sheet1'
%
\begin{table*}[htbp]
\footnotesize
  \centering
  \caption{The comparison of three methods for $\tilde R_{\ell m\omega }^{{\rm{in,up}}}$  with the  different floating-point numbers $\bf N$ at $r_0=5M, 10M, 15M$, respectively. } \label{tab:RinRup-Err}%
  \begin{threeparttable}
    \begin{tabular}{c|c|c|cccccccccc|c}
    \toprule
    $r_0$     & $\tilde  R_{\ell m\omega }^{{\rm{in,up}}}$  & {Method} & \textbf{N} & 20    & 30    & 40    & 50    & 60    & 70    & 80    & 90    & 100   & Mean \\
    \midrule
\rowcolor{blue!30} \cellcolor{white}\multirow{10}[4]{*}{$5M$} & \cellcolor{white}\multirow{6}[2]{*}{$\tilde R_{\ell m\omega }^{\rm{in}}$} &\cellcolor{white}\multirow{1}[1]{*}{HeunC} & \cellcolor{white}$\bf RE$  & 1.674E-20\tnote{*}\quad &\, 5.580E-31 & 5.580E-41 & 5.580E-51 & 2.232E-60 & 5.580E-71 & 9.317E-81 & 2.381E-91 & 1.116E-100 & \cellcolor{white}- \\
\rowcolor{red!40} \cellcolor{white}&\cellcolor{white}& \cellcolor{white}&\cellcolor{white} ROC   &\cellcolor{white} -       & 10.477  & 10.000  & 10.000  & 9.398  & 10.602  & 9.777  & 10.593  & 9.329  & 10.022  \\
          &       & \multirow{2}[0]{*}{MST\tnote{**}} & RE    & 3.331E-12 & 9.611E-17 & 7.055E-23 & 4.337E-28 & 3.769E-32 & 9.537E-38 & 4.551E-42 & 5.568E-48 & 1.617E-52 &-\\
          &       &       & ROC   & -     & 4.540  & 6.134  & 5.211  & 4.061  & 5.597  & 4.321  & 5.912  & 4.537  & 5.039  \\
          &       & \multirow{2}[1]{*}{NI} & RE    & 6.297E-12 & 7.639E-17 & 1.359E-21 & 1.518E-26 & 1.487E-31 & 1.501E-36 & 1.451E-41 & 1.419E-46 & 1.272E-51 & - \\
          &       &       & ROC   & -     & 4.916  & 4.750  & 4.952  & 5.009  & 4.996  & 5.015  & 5.010  & 5.047  & 4.962  \\
\cmidrule{2-14}
\rowcolor{blue!30} \cellcolor{white}         & \cellcolor{white}  \multirow{6}[2]{*}{$\tilde R_{\ell m\omega }^{\rm{up}}$} & \cellcolor{white}  \multirow{1}[1]{*}{HeunC} & \cellcolor{white}  $\bf RE$     & 5.370E-12 & 1.702E-21 & 9.107E-32 & 9.355E-40 & 1.144E-50 & 8.525E-62 & 9.280E-71 & 1.673E-80 & 4.741E-90 & \cellcolor{white}- \\
\rowcolor{red!40} \cellcolor{white}&\cellcolor{white}& \cellcolor{white}&\cellcolor{white} ROC   &\cellcolor{white} -   & 9.499  & 10.272  & 7.988  & 10.913  & 11.128  & 8.963  & 9.744  & 9.548  & 9.757  \\
          &       & \multirow{2}[0]{*}{MST} & RE    & 9.441E-11 & 9.356E-16 & 8.241E-21 & 6.849E-26 & 5.482E-31 & 1.057E-35 & 8.090E-41 & 6.098E-46 & 1.127E-50 & - \\
          &       &       & ROC   & -     & 5.004  & 5.055  & 5.080  & 5.097  & 4.715  & 5.116  & 5.123  & 4.733  & 4.990  \\
          &       & \multirow{2}[1]{*}{NI} & RE    & 9.111E-07 & 4.811E-12 & 3.418E-17 & 6.648E-22 & 1.487E-27 & 2.631E-32 & 1.946E-36 & 1.302E-41 & 1.167E-46 & - \\
          &       &       & ROC   & -     & 5.277  & 5.148  & 4.711  & 5.650  & 4.752  & 4.131  & 5.175  & 5.048  & 4.987  \\
    \midrule
\rowcolor{blue!30} \cellcolor{white}\multirow{10}[4]{*}{$10M$} & \cellcolor{white}\multirow{6}[2]{*}{$\tilde R_{\ell m\omega }^{\rm{in}}$} &\cellcolor{white}\multirow{1}[1]{*}{HeunC} & \cellcolor{white}$\bf RE$  & 1.659E-20 & 2.347E-30 & 7.437E-41 & 1.662E-50 & 6.033E-61 & 9.147E-71 & 9.416E-81 & 8.754E-91 & 4.244E-101 &\cellcolor{white} - \\
\rowcolor{red!40} \cellcolor{white}&\cellcolor{white}& \cellcolor{white}&\cellcolor{white} ROC   &\cellcolor{white} -     & 9.849  & 10.499  & 9.651  & 10.440  & 9.819  & 9.987  & 10.032  & 10.314  & 10.074  \\
          &       & \multirow{2}[0]{*}{MST} & $\bf{RE}$ & 7.192E-12 & 1.997E-17 & 1.339E-22 & 4.620E-28 & 2.537E-32 & 3.397E-38 & 9.899E-43 & 2.280E-47 & 4.222E-52 & - \\
          &       &       & ROC   & -     & 5.556  & 5.174  & 5.462  & 4.260  & 5.873  & 4.536  & 4.638  & 4.732  & 5.029  \\
          &       & \multirow{2}[1]{*}{NI} & $\bf{RE}$ & 6.862E-12 & 9.780E-17 & 1.931E-21 & 1.823E-26 & 1.757E-31 & 1.479E-36 & 1.399E-41 & 1.229E-46 & 1.270E-51 & - \\
          &       &       & ROC   & -     & 4.846  & 4.705  & 5.025  & 5.016  & 5.075  & 5.024  & 5.056  & 4.986  & 4.967  \\
\cmidrule{2-14}
\rowcolor{blue!30} \cellcolor{white}         & \cellcolor{white}  \multirow{6}[2]{*}{$\Tilde R_{\ell m\omega }^{\rm{up}}$} & \cellcolor{white}  \multirow{1}[1]{*}{HeunC} & \cellcolor{white}  $\bf RE$    & 1.486E-10 & 1.817E-21 & 4.842E-31 & 5.754E-41 & 3.391E-51 & 2.871E-61 & 4.839E-72 & 1.118E-80 & 8.564E-92 & \cellcolor{white}- \\
\rowcolor{red!40} \cellcolor{white}&\cellcolor{white}& \cellcolor{white}&\cellcolor{white} ROC   &\cellcolor{white} -    & 10.913  & 9.574  & 9.925  & 10.230  & 10.072  & 10.773  & 8.636  & 11.116  & 10.155  \\
          &       & \multirow{2}[0]{*}{MST} & $\bf{RE}$ & 3.697E-11 & 1.413E-16 & 2.335E-21 & 3.665E-26 & 1.137E-31 & 1.677E-36 & 2.431E-41 & 3.477E-46 & 9.957E-52 & - \\
          &       &       & ROC   & -     & 5.418  & 4.782  & 4.804  & 5.508  & 4.831  & 4.839  & 4.845  & 5.543  & 5.071  \\
          &       & \multirow{2}[1]{*}{NI} & $\bf{RE}$ & 5.125E-08 & 3.074E-13 & 1.164E-18 & 7.905E-23 & 9.104E-28 & 5.860E-33 & 2.228E-37 & 8.457E-43 & 1.701E-47 & - \\
          &       &       & ROC   & -     & 5.222  & 5.422  & 4.168  & 4.939  & 5.191  & 4.420  & 5.421  & 4.697  & 4.935  \\
    \midrule
\rowcolor{blue!30} \cellcolor{white}\multirow{10}[4]{*}{$15M$} & \cellcolor{white}\multirow{6}[2]{*}{$\tilde R_{\ell m\omega }^{\rm{in}}$} &\cellcolor{white}\multirow{1}[1]{*}{HeunC} & \cellcolor{white}$\bf RE$  & 1.222E-20 & 2.926E-30 & 2.410E-40 & 1.737E-50 & 2.239E-60 & 2.655E-70 & 2.536E-80 & 4.398E-90 & 2.290E-100 & \cellcolor{white}- \\
\rowcolor{red!40} \cellcolor{white}&\cellcolor{white}& \cellcolor{white}&\cellcolor{white} ROC   &\cellcolor{white} -     & 9.621  & 10.084  & 10.142  & 9.890  & 9.926  & 10.020  & 9.761  & 10.283  & 9.966  \\
          &       & \multirow{2}[0]{*}{MST} & $\bf{RE}$ & 1.434E-12 & 1.756E-17 & 7.326E-23 & 4.265E-27 & 5.674E-33 & 1.564E-37 & 3.287E-42 & 1.301E-48 & 1.578E-53 & - \\
          &       &       & ROC   & -     & 4.912  & 5.380  & 4.235  & 5.876  & 4.560  & 4.677  & 6.403  & 4.916  & 5.120  \\
          &       & \multirow{2}[1]{*}{NI} & $\bf{RE}$ & 8.509E-12 & 7.879E-17 & 1.527E-21 & 1.846E-26 & 1.817E-31 & 1.640E-36 & 1.446E-41 & 1.353E-46 & 1.288E-51 & - \\
          &       &       & ROC   & -     & 5.033  & 4.713  & 4.918  & 5.007  & 5.045  & 5.055  & 5.029  & 5.021  & 4.977  \\
\cmidrule{2-14} \rowcolor{blue!30} \cellcolor{white}         & \cellcolor{white}  \multirow{6}[2]{*}{$\tilde R_{\ell m\omega }^{\rm{up}}$ } & \cellcolor{white}  \multirow{1}[1]{*}{HeunC} & \cellcolor{white}  $\bf RE$     & 9.656E-12 & 1.680E-21 & 2.885E-31 & 5.408E-40 & 3.757E-51 & 1.355E-61 & 2.637E-71 & 3.447E-80 & 4.157E-92 & \cellcolor{white}- \\
\rowcolor{red!40} \cellcolor{white}&\cellcolor{white}& \cellcolor{white}&\cellcolor{white} ROC   &\cellcolor{white} -      & 9.759  & 9.765  & 8.727  & 11.158  & 10.443  & 9.711  & 8.884  & 11.919  & 10.046  \\
          &       & \multirow{2}[0]{*}{MST} & $\bf{RE}$ & 1.444E-11 & 1.181E-16 & 8.730E-22 & 6.088E-27 & 4.087E-32 & 1.957E-36 & 1.258E-41 & 7.956E-47 & 4.971E-52 & - \\
          &       &       & ROC   & -     & 5.087  & 5.131  & 5.157  & 5.173  & 4.320  & 5.192  & 5.199  & 5.204  & 5.058  \\
          &       & \multirow{2}[1]{*}{NI} & $\bf{RE}$ & 1.402E-08 & 8.764E-14 & 6.418E-19 & 4.549E-24 & 8.986E-29 & 1.060E-33 & 3.887E-38 & 1.596E-42 & 2.745E-48 & - \\
          &       &       & ROC   & -     & 5.204  & 5.135  & 5.149  & 4.704  & 4.928  & 4.436  & 4.387  & 5.764  & 4.964  \\
    \bottomrule
    \end{tabular}%


  \begin{tablenotes}
        \footnotesize
        \item[*] The table data in this paper are abbreviated using scientific notation. For instance, $ 1.674\times10^{-20}$ is represented as  1.674E-20.
        \item[**] The MST method in \Cref{tab:RinRup-Err} is the MST-RTE method.
      \end{tablenotes}

  \end{threeparttable}
\end{table*}
%% Figure environment removed

% Figure environment removed

\subsection{Comparisons with other methods}\label{sec:results}
% Figure environment removed



% Figure environment removed
%

%
% Figure environment removed



\begin{table*}[htbp]
  \centering
  \footnotesize
  \caption{The comparison of four methods for energy fluxes $\left<{dE_{2,2}/dt}\right>_{\infty,{\rm H}}$ with the  different floating-point numbers $\bf N$ at $r_0=5M, 10M, 15M$, respectively. }\label{tab:flux-inf-hor}%
    \begin{tabular}{c|c|c|cccccccccc|c}
    \toprule
 $r_0$     & Flux  & Method &   $\bf{N}$    & 20    & 30    & 40    & 50    & 60    & 70    & 80    & 90    & 100   & Mean \\
    \midrule
\rowcolor{blue!30} \cellcolor{white}\multirow{14}[4]{*}{$5M$} & \cellcolor{white}\multirow{8}[2]{*}{${{\dot E}_\infty }$} &\cellcolor{white}\multirow{1}[1]{*}{HeunC} & \cellcolor{white}$\bf RE$    & 1.302E{-15} &  {2.050E{-23}} & {9.694E{-35}} & 4.549E-43 & 7.961E-56 &  4.649E-63 & 3.475E-72 &  7.711E-82 & 1.597E-93 & \cellcolor{white} {-} \\
\rowcolor{red!40} \cellcolor{white}&\cellcolor{white}& \cellcolor{white}&\cellcolor{white} ROC   &\cellcolor{white} -     & 7.803  & 11.325  & 8.329  & 12.757  & 7.234 & 9.126  &9.654 & 11.684& 9.739  \\
          &          & \multirow{1}[0]{*}{MST-RTE} & $\bf RE$    & 6.222E-12 &  {1.669E-16} &  {1.188E-21} &  {8.259E-27} &  {5.668E-32} & {1.370E-37} & {7.886E-42} &  {9.898E-48} & \multicolumn{1}{l|}{2.106E-52} &  {-} \\
          &          &       & ROC   & -     & 4.572  & 5.148  & 5.158  & 5.163  & 5.616  & 4.240  & 5.901  & 4.672  & 5.059  \\
          &          & \multirow{1}[0]{*}{MST-RWE} & $\bf RE$    & 1.093E-10 &  {2.018E-15} &  {1.339E-20} &  {2.186E-25} &  {2.140E-30} & {2.297E-35} & {2.113E-40} &  {1.271E-45} & \multicolumn{1}{l|}{2.123E-50} &  {-} \\
          &          &       & ROC   & -     & 4.733  & 5.178  & 4.787  & 5.009  & 4.969  & 5.036  & 5.221  & 4.777  & 4.964  \\
          &          & \multirow{1}[1]{*}{27.5PN} & $\bf RE$    & 2.664E-05 &  {2.664E-05} &  {2.664E-05} &  {2.664E-05} &  {2.664E-05} & {2.664E-05} & {2.664E-05} &  {2.664E-05} & \multicolumn{1}{l|}{2.664E-05} &  {-} \\
          &          &       & ROC   & -     & 0.000  & 0.000  & 0.000  & 0.000  & 0.000  & 0.000  & 0.000  & 0.000  & 0.000  \\
\cmidrule{2-14}
\rowcolor{blue!30} \cellcolor{white}   & \cellcolor{white}  \multirow{6}[2]{*}{${{\dot E}_{\rm H} }$} & \cellcolor{white}  \multirow{1}[1]{*}{HeunC} & \cellcolor{white}  $\bf RE$    & 3.672E-12 &  3.738E-21 &  4.314E-32 & 5.570E-40 & 1.855E-50 &  3.377E-62 &  1.894E-70 &  3.589E-80 & \multicolumn{1}{c|}{1.029E-89} & \cellcolor{white} {-} \\
\rowcolor{red!40} \cellcolor{white}& \cellcolor{white} & \cellcolor{white} & \cellcolor{white}  ROC   & -  \cellcolor{white} & 8.992& 10.938 & 7.889 & 10.478 & 11.740 & 8.251 & 9.722 & 9.543 & 9.694\\
          &          & \multirow{1}[0]{*}{MST-RTE} & $\bf RE$    & 1.126E-10 &  {8.712E-16} &  {8.532E-23} &  {1.632E-26} &  {2.587E-31} & {6.958E-36} & {7.091E-41} &  {6.816E-46} & \multicolumn{1}{c|}{1.391E-50} &  {-} \\
          &          &       & ROC   & -     & 5.110 & 7.009 & 3.718 & 4.800 & 4.570 & 4.992 & 5.017 & 4.69 & 4.988  \\
          &          & \multirow{1}[1]{*}{MST-RWE} & $\bf RE$    & 3.436E-11 &  {1.305E-15} &  {1.319E-20} &  {2.729E-25} &  {7.259E-32} & {3.973E-35} & {7.487E-41} &  {9.097E-46} & \multicolumn{1}{c|}{1.307E-50} &  {-} \\
          &          &       & ROC   & -     & 4.420 & 4.996 & 4.684 & 6.575 & 3.262 & 5.725 & 4.915 & 4.843 & 4.927  \\
    \midrule
\rowcolor{blue!30} \cellcolor{white}\multirow{14}[4]{*}{$10M$} & \cellcolor{white}\multirow{8}[2]{*}{${{\dot E}_\infty }$} &\cellcolor{white}\multirow{1}[1]{*}{HeunC} & \cellcolor{white}$\bf RE$  & 2.928E-17 & 1.896E-27 & 2.204E-36 & 9.022E-46 & 3.393E-56 & 6.502E-66 & 5.737E-78 & 2.182E-86 & 1.792E-97 & \cellcolor{white}- \\
\rowcolor{red!40} \cellcolor{white}& \cellcolor{white} & \cellcolor{white} & \cellcolor{white}  ROC   & -  \cellcolor{white} & 10.189  & 8.935  & 9.388  & 10.425  & 9.718  & 12.054  & 8.420  & 11.086  & 10.027  \\
          &       & \multirow{1}[0]{*}{MST-RTE} & $\bf{RE}$ & 1.033E-11 & 6.568E-18 & 1.495E-21 & 5.964E-27 & 5.681E-32 & 4.055E-38 & 4.394E-43 & 5.054E-47 & 2.002E-52 & - \\
          &       &       & ROC   & -     & 6.197  & 3.643  & 5.399  & 5.021  & 6.146  & 4.965  & 3.939  & 5.402  & 5.089  \\
          &       & \multirow{1}[0]{*}{MST-RWE} & $\bf{RE}$ & 7.240E-11 & 3.117E-16 & 2.933E-21 & 2.906E-26 & 4.987E-31 & 5.562E-36 & 3.153E-41 & 4.736E-47 & 5.023E-52 & - \\
          &       &       & ROC   & -     & 5.366  & 5.026  & 5.004  & 4.765  & 4.953  & 5.247  & 5.823  & 4.974  & 5.145  \\
          &       & \multirow{1}[1]{*}{27.5PN} & $\bf{RE}$ & 7.473E-12 & 7.473E-12 & 7.473E-12 & 7.473E-12 & 7.473E-12 & 7.473E-12 & 7.473E-12 & 7.473E-12 & 7.473E-12 & - \\
          &       &       & ROC   & -     & 0.000  & 0.000  & 0.000  & 0.000  & 0.000  & 0.000  & 0.000  & 0.000  & 0.000  \\
          \cmidrule{2-14}
\rowcolor{blue!30} \cellcolor{white}& \cellcolor{white}\multirow{6}[2]{*}{${{\dot E}_{\rm H} }$} &\cellcolor{white}\multirow{1}[1]{*}{HeunC} & \cellcolor{white}$\bf RE$  & 3.094E-10 & 1.016E-21 & 5.599E-31 & 7.962E-41 & 4.972E-51 & 3.013E-62 & 7.399E-72 & 1.533E-80 & 1.122E-91 & \cellcolor{white}- \\
\rowcolor{red!40} \cellcolor{white}& \cellcolor{white} & \cellcolor{white} & \cellcolor{white}  ROC   & -  \cellcolor{white} & 11.484  & 9.259  & 9.847  & 10.204  & 11.218  & 9.610  & 8.684  & 11.136  & 10.180  \\
          &       & \multirow{1}[0]{*}{MST-RTE} & $\bf{RE}$ & 7.824E-11 & 3.102E-16 & 5.299E-21 & 8.606E-26 & 2.776E-31 & 4.241E-36 & 6.368E-41 & 6.556E-46 & 2.807E-51 & - \\
          &       &       & ROC   & -     & 5.402  & 4.767  & 4.789  & 5.491  & 4.816  & 4.823  & 4.987  & 5.368  & 5.056  \\
          &       & \multirow{1}[1]{*}{MST-RWE} & $\bf{RE}$ & 4.694E-11 & 1.261E-15 & 1.226E-20 & 1.239E-25 & 3.294E-31 & 1.949E-36 & 1.069E-40 & 3.133E-46 & 3.793E-51 & - \\
          &       &       & ROC   & -     & 4.571  & 5.012  & 4.995  & 5.575  & 5.228  & 4.261  & 5.533  & 4.917  & 5.012  \\
\cmidrule{1-14}
 \rowcolor{blue!30} \cellcolor{white}\multirow{14}[4]{*}{$15M$} & \cellcolor{white}\multirow{8}[2]{*}{${{\dot E}_\infty }$} &\cellcolor{white}\multirow{1}[1]{*}{HeunC} & \cellcolor{white}$\bf RE$   & 1.009E-17 & 6.765E-28 & 9.174E-37 & 2.107E-48 & 2.193E-57 & 5.545E-69 & 9.870E-78 & 1.109E-89 & 9.704E-99 & \cellcolor{white}- \\
\rowcolor{red!40} \cellcolor{white}& \cellcolor{white} & \cellcolor{white} & \cellcolor{white}  ROC   & -  \cellcolor{white}  & 10.174  & 8.868  & 11.639  & 8.983  & 11.597  & 8.750  & 11.949  & 9.058  & 10.127  \\
          &       & \multirow{1}[0]{*}{MST-RTE} & $\bf{RE}$ & 6.971E-14 & 5.800E-18 & 4.959E-23 & 3.717E-28 & 6.282E-32 & 3.453E-37 & 3.749E-43 & 1.889E-47 & 3.470E-53 & - \\
          &       &       & ROC   & -     & 4.080  & 5.068  & 5.125  & 3.772  & 5.260  & 5.964  & 4.298  & 5.736  & 4.913  \\
          &       & \multirow{1}[0]{*}{MST-RWE} & $\bf{RE}$ & 4.974E-12 & 5.241E-16 & 2.397E-21 & 1.059E-26 & 1.798E-31 & 2.677E-36 & 1.281E-41 & 2.528E-46 &  3.079E-51 & - \\
          &       &       & ROC   & -     & 3.977  & 5.340  & 5.355  & 4.770  & 4.827  & 5.320  & 4.705  & 4.914  & 4.901  \\
          &       & \multirow{1}[1]{*}{27.5PN} & $\bf{RE}$ & 9.123E-16 & 9.123E-16 & 9.123E-16 & 9.123E-16 & 9.123E-16 & 9.123E-16 & 9.123E-16 & 9.123E-16 &  9.123E-16 & - \\
          &       &       & ROC   & -     & 0.000  & 0.000  & 0.000  & 0.000  & 0.000  & 0.000  & 0.000  & 0.000  & 0.000  \\
\cmidrule{2-14}
\rowcolor{blue!30} \cellcolor{white} & \cellcolor{white}\multirow{6}[2]{*}{${{\dot E}_{\rm H} }$}&\cellcolor{white}\multirow{1}[1]{*}{HeunC} & \cellcolor{white}$\bf RE$   & 2.022E-11 & 3.535E-21 & 2.091E-31 & 4.269E-40 & 3.446E-51 & 1.422E-61 & 5.040E-71 & 3.370E-80 & 2.647E-92 & \cellcolor{white}- \\
\rowcolor{red!40} \cellcolor{white}& \cellcolor{white} & \cellcolor{white} & \cellcolor{white}  ROC   & -  \cellcolor{white}  & 9.757  & 10.228  & 8.690  & 11.093  & 10.384  & 9.450  & 9.175  & 12.105  & 10.110  \\
          &       & \multirow{1}[0]{*}{MST-RTE} & $\bf{RE}$ & 3.306E-11 & 2.856E-16 & 2.232E-21 & 1.644E-26 & 1.165E-31 & 4.127E-36 & 2.675E-41 & 1.707E-46 &  1.076E-51 & - \\
          &       &       & ROC   & -     & 5.064  & 5.107  & 5.133  & 5.150  & 4.451  & 5.188  & 5.195  & 5.200  & 5.061  \\
          &       & \multirow{1}[1]{*}{MST-RWE} & $\bf{RE}$ & 5.920E-11 & 1.693E-17 & 2.088E-21 & 2.175E-26 & 1.107E-30 & 4.453E-36 & 3.261E-41 & 4.859E-46 &  7.504E-51 & - \\
          &       &       & ROC   & -     & 6.544  & 3.909  & 4.982  & 4.293  & 5.395  & 5.135  & 4.827  & 4.811  & 4.987  \\
    \bottomrule
    \end{tabular}%

\end{table*}


% Table generated by Excel2LaTeX from sheet 'Sheet1'
%
\begin{table*}[htbp!]
	\centering
%\scriptsize
\centering
{
\caption{Relative errors  between energy fluxes $\left<{dE_{2,2}/dt}\right>_{\infty,{\rm H}}$ of  the four methods ( ${\bf N} =100$ ) and the exact solution in different orbital radius $r_0$.}\label{table:Flux22}
      \begin{tabular}{cccccccc}
  \toprule
Relative errors   &   $r_0$   &   $5M$       & $10M$       & $20M$                   & $50M$                   & $100M$ & $200M$ \\   \midrule
\rowcolor{red!40} \cellcolor{white}\multirow{4}{*}{${\bf{RE}}\left( {{{\dot E}_{\infty }}} \right)$ }&{ HeunC}\cellcolor{white}
& {1.597E-93}&{1.792E-97} & {7.120E-96} & { 1.361E-99} & {9.677E-101}& {1.02E-100}\\
 &MST-RTE       &   {2.106E-52}              &  2.002E-52                  & 4.321E-52 & 4.066E-52 & 9.754E-54& 1.02E-53 \\
 &MST-RWE       &   {2.123E-50}              &  5.023E-52                 & 2.120E-51 & 7.431E-52 & 1.593E-52  & 1.93E-52 \\
 &27.5PN        &   2.664E{-5}               &  7.473E-12                & 1.467E-18 & 1.512E-27 & 1.673E-34 & 1.63E-42 \\
\midrule
\rowcolor{red!40} \cellcolor{white}\multirow{3}{*}{${\bf{RE}}\left( {{{\dot E}_{{\rm{H}}}}} \right)$ } &{HeunC }\cellcolor{white}
&{{1.029E-89}}& {1.122E-91} & {5.333E-92} & {3.702E-89} & {3.243E-87}& {6.68E-86} \\
&MST-RTE       &    {1.391E-50}                  &   2.807E-51                   & 2.242E-51 & 1.287E-51 & 6.314E-53 & 1.25E-52 \\
&MST-RWE       &    {1.307E-50}                  &   3.793E-51                & 2.724E-51 & 2.342E-51 & 8.606E-52 & 9.63E-52 \\
 \bottomrule
\end{tabular}
}
\end{table*}
% Table generated by Excel2LaTeX from sheet 'Sheet1'
%%



\begin{table*}[htbp]
	\centering
%\scriptsize
\caption{Relative errors between energy fluxes  $\left<{dE_{\ell m}/dt}\right>_{\infty,{\rm H}}$  of three methods (  ${\bf N} =100$  ) and the exact solution with different modes at $r_0=5M, 10M, 15M$, respectively ( NA =  not available). }\label{table:Flux7MHorizon}
  \begin{threeparttable}
      \begin{tabular}{cccccccccc}
\toprule
 &  &\multicolumn{4}{c}{Relative Errors of $\left<{dE_{\ell m}/dt}\right>_\infty$ }& \multicolumn{3}{c}{Relative Errors of $\left<{dE_{\ell m}/dt}\right>_{\rm H}$}  \\   \cmidrule(r){3-6} \cmidrule(r){7-9}
{$r_0$}  &   $\left({\ell ,|m|}\right)$   \quad & \quad   \textcolor{colour6}{ HeunC } \quad & \quad  { MST-RTE  }  \quad & \quad  { MST-RWE} \quad & \quad {27.5PN  } \quad & \quad   \textcolor{colour6}{ HeunC } \quad & \quad  { MST-RTE } \quad & \quad { MST-RWE }  \\   \midrule
%  r = 5M
\multirow{9}{*}{$5M$} &  $ \left( {2 ,2} \right)$ \quad   &   \quad \textcolor{colour6}{1.597E-93} \quad & \quad    2.106E{-52}    \quad & \quad   2.123E{-50}   \quad & \quad   2.664E{-5}
\quad & \quad   \textcolor{colour6}{1.029E-89}     \quad & \quad     1.391E{-51}    \quad & \quad       1.307E{-51}     \\
  &  $ \left( {3,3} \right) $  \quad & \quad    \textcolor{colour6}{  2.552E-90 } \quad & \quad    7.251E{-52}    \quad & \quad   1.021E{-50}   \quad & \quad   2.325E{-4}
\quad & \quad   \textcolor{colour6}{ 4.736E-86 }     \quad & \quad     2.145E{-51}    \quad & \quad       1.404E{-50}     \\
  &  $ \left( {4,4} \right)$    \quad & \quad    \textcolor{colour6}{1.302E-88 } \quad & \quad    9.186E{-52}    \quad & \quad  1.148E{-51}    \quad & \quad   2.672E{-3}
\quad & \quad   \textcolor{colour6}{1.137E-83 }     \quad & \quad     1.266E{-50}    \quad & \quad       1.892E{-52}     \\
  &  $ \left( {5,5} \right) $   \quad & \quad    \textcolor{colour6}{ 1.391E-85 } \quad & \quad    1.294E{-51}    \quad & \quad   1.259E{-50}   \quad & \quad   1.580E{-1}
\quad & \quad   \textcolor{colour6}{2.777E-78 }     \quad & \quad     7.337E{-51}    \quad & \quad       4.423E{-51}     \\
  &  $\left( {6,6} \right) $   \quad & \quad    \textcolor{colour6}{  8.246E-84 } \quad & \quad    2.968E{-52}    \quad & \quad   1.979E{-50}   \quad & \quad    NA
\quad & \quad   \textcolor{colour6}{ 6.506E-76 }     \quad & \quad     1.253E{-50}    \quad & \quad       3.686E{-51}     \\
  &   $\left( {7,7} \right)$    \quad & \quad    \textcolor{colour6}{ 2.491E-81 } \quad & \quad    6.137E{-52}    \quad & \quad   1.775E{-50}   \quad & \quad    NA
\quad & \quad   \textcolor{colour6}{8.890E-73 }     \quad & \quad     1.694E{-50}    \quad & \quad       1.694E{-51}     \\
  &  $ \left( {8,8} \right) $  \quad & \quad    \textcolor{colour6}{1.192E-79 } \quad & \quad    6.515E{-52}    \quad & \quad   8.741E{-51}    \quad & \quad    NA
\quad & \quad   \textcolor{colour6}{1.721E-68 }     \quad & \quad     2.387E{-50}    \quad & \quad       1.298E{-52}     \\
  &  $ \left( {9,9} \right)$    \quad & \quad   \textcolor{colour6}{ 1.663E-76 } \quad & \quad    3.532E{-52}    \quad & \quad    3.820E{-51}   \quad & \quad    NA
\quad & \quad  \textcolor{colour6}{ 2.080E-66 }     \quad & \quad     1.330E{-50}    \quad & \quad       3.441E{-51}     \\
  &  $ \left({10,10}\right) $    \quad & \quad   \textcolor{colour6}{  1.321E-75 } \quad & \quad    2.873E{-51}    \quad & \quad   1.285E{-50}   \quad & \quad    NA
\quad & \quad \textcolor{colour6}{  1.922E-62 }     \quad & \quad {   1.647E{-51} }   \quad & \quad       7.344E{-51}     \\
 \rowcolor{red!40} \cellcolor{white}&  \cellcolor{white} Summation\tnote{+}  & \quad  { 8.196E-80 } \quad & \quad  4.517E-53 \quad & \quad    1.711E-50  \quad & \quad    \cellcolor{white}-
\quad & \quad{  2.551E-72 }     \quad & \quad {  1.421E-50  }   \quad & \quad    1.244E-50   \\
%  r = 10M
\bottomrule
\multirow{9}{*}{$10M$} &  $ \left( {2 ,2} \right)$  \quad  &   \quad \textcolor{colour6}{1.792E-97 } \quad & \quad  2.002E-52\quad & \quad  5.023E-52 \quad & \quad   7.473E-12
\quad & \quad   \textcolor{colour6}{ 1.685E{-87} }     \quad & \quad  2.807E-51 \quad & \quad  3.793E-51  \\
  &  $ \left( {3,3} \right) $  \quad & \quad    \textcolor{colour6}{ 3.579E-94 } \quad & \quad  3.453E-52 \quad & \quad 1.045E-51 \quad & \quad 1.768E-10
\quad & \quad   \textcolor{colour6}{ 8.661E-87 }     \quad & \quad   4.743E-51  \quad & \quad    6.225E-52  \\
  &  $ \left( {4,4} \right)$    \quad & \quad    \textcolor{colour6}{8.370E-95} \quad & \quad  2.334E-52  \quad & \quad  4.653E-52 \quad & \quad  7.812E-9
\quad & \quad   \textcolor{colour6}{6.160E-84 }     \quad & \quad  1.556E-51   \quad & \quad    6.750E-52  \\
  &  $ \left( {5,5} \right) $   \quad & \quad    \textcolor{colour6}{ 1.300E-91 } \quad & \quad  3.171E-52  \quad & \quad  8.646E-52  \quad & \quad 4.014E-7
\quad & \quad   \textcolor{colour6}{ 4.283E-82 }     \quad & \quad    2.215E-51  \quad & \quad     1.315E-51   \\
  &  $\left( {6,6} \right) $   \quad & \quad    \textcolor{colour6}{ 2.093E-91 } \quad & \quad    7.037E-52  \quad & \quad 1.810E-51 \quad & \quad    NA
\quad & \quad   \textcolor{colour6}{9.074E-77 }     \quad & \quad 3.378E-51   \quad & \quad   1.751E-51   \\
  &   $\left( {7,7} \right)$    \quad & \quad    \textcolor{colour6}{2.158E-91} \quad & \quad 3.318E-52  \quad & \quad  1.617E-51 \quad & \quad    NA
\quad & \quad   \textcolor{colour6}{1.082E-73}     \quad & \quad  5.287E-52 \quad & \quad   8.500E-52   \\
  &  $ \left( {8,8} \right) $  \quad & \quad    \textcolor{colour6}{1.549E-89} \quad & \quad  1.939E-51  \quad & \quad   8.741E{-51}    \quad & \quad    NA
\quad & \quad   \textcolor{colour6}{ 3.386E-70}     \quad & \quad 1.395E-51 \quad & \quad    2.163E-51   \\
  &  $ \left( {9,9} \right)$    \quad & \quad   \textcolor{colour6}{2.935E-89} \quad & \quad  1.955E-52  \quad & \quad   2.484E-51   \quad & \quad    NA
\quad & \quad  \textcolor{colour6}{9.742E-68  }     \quad & \quad  3.668E-51\quad & \quad   4.278E-52  \\
  &  $ \left({10,10}\right) $    \quad & \quad   \textcolor{colour6}{3.010E-88 } \quad & \quad  2.175E-52 \quad & \quad 5.033E-51  \quad & \quad    NA
\quad & \quad \textcolor{colour6}{ 2.543E-65  }     \quad & \quad {9.362E-52  }   \quad & \quad 1.602E-51   \\
 \rowcolor{red!40}\cellcolor{white} &   Summation \cellcolor{white} & \quad { 4.928E-94} \quad & \quad    2.143E-52    \quad & \quad   5.542E-52   \quad & \quad    -\cellcolor{white}
\quad & \quad { 1.913E-83  }     \quad & \quad {  2.815E-51 }   \quad & \quad  3.780E-51 \\
% %  r = 15M
\bottomrule
\multirow{9}{*}{$15M$} &  $ \left( {2 ,2} \right)$  \quad  &   \quad \textcolor{colour6}{ 9.704E-99 } \quad & \quad  3.470E-53  \quad & \quad   3.079E-51 \quad & \quad  9.123E-16
\quad & \quad   \textcolor{colour6}{ 2.647E-92 }     \quad & \quad  1.076E-51 \quad & \quad   7.504E-51   \\
  &  $ \left( {3,3} \right) $  \quad & \quad    \textcolor{colour6}{1.779E-94 } \quad & \quad  1.374E-52  \quad & \quad  2.175E-51   \quad & \quad   3.657E-14
\quad & \quad   \textcolor{colour6}{ 3.541E-84  }     \quad & \quad   2.266E-51   \quad & \quad   3.589E-51   \\
  &  $ \left( {4,4} \right)$    \quad & \quad    \textcolor{colour6}{ 1.729E-95} \quad & \quad   3.506E-52   \quad & \quad 1.144E-51   \quad & \quad  3.026E-12
\quad & \quad   \textcolor{colour6}{9.717E-85}     \quad & \quad   4.964E-51  \quad & \quad   1.013E-51   \\
  &  $ \left( {5,5} \right) $   \quad & \quad    \textcolor{colour6}{ 6.911E-95 } \quad & \quad   9.337E-52  \quad & \quad 1.991E-51   \quad & \quad  1.666E-10
\quad & \quad   \textcolor{colour6}{ 7.121E-81  }     \quad & \quad   2.553E-51  \quad & \quad  2.193E-51  \\
  &  $\left( {6,6} \right) $   \quad & \quad    \textcolor{colour6}{3.349E-94 } \quad & \quad   3.675E-52   \quad & \quad   1.313E-51   \quad & \quad    NA
\quad & \quad   \textcolor{colour6}{ 2.723E-76 }     \quad & \quad  1.813E-53\quad & \quad    3.312E-51  \\
  &   $\left( {7,7} \right)$    \quad & \quad    \textcolor{colour6}{2.719E-93 } \quad & \quad  4.107E-52   \quad & \quad 5.341E-52   \quad & \quad    NA
\quad & \quad   \textcolor{colour6}{ 4.410E-73 }     \quad & \quad    1.816E-51 \quad & \quad   3.799E-52   \\
  &  $ \left( {8,8} \right) $  \quad & \quad    \textcolor{colour6}{1.519E-93 } \quad & \quad  1.307E-52  \quad & \quad   1.341E-51   \quad & \quad    NA
\quad & \quad   \textcolor{colour6}{ 7.063E-66 }     \quad & \quad   4.122E-51  \quad & \quad    5.032E-52 \\
  &  $ \left( {9,9} \right)$    \quad & \quad   \textcolor{colour6}{8.643E-92} \quad & \quad  1.156E-52  \quad & \quad  8.589E-52  \quad & \quad    NA
\quad & \quad  \textcolor{colour6}{5.078E-68 }     \quad & \quad  1.261E-51  \quad & \quad  7.931E-52  \\
  &  $ \left({10,10}\right) $    \quad & \quad   \textcolor{colour6}{2.667E-90} \quad & \quad  6.047E-54  \quad & \quad 5.829E-52 \quad & \quad    NA
\quad & \quad \textcolor{colour6}{  1.155E-62    }     \quad & \quad { 2.775E-51}   \quad & \quad    1.006E-51           \\
 \rowcolor{red!40}\cellcolor{white} &   \cellcolor{white}Summation  & \quad 1.305E-95\quad & \quad  2.491E-53   \quad & \quad  2.996E-51   \quad & \quad   -\cellcolor{white}
\quad & \quad 1.587E-82 \quad & \quad   1.078E-51 \quad & \quad  7.498E-51  \\
  \bottomrule
\end{tabular}
\begin{tablenotes}
        \footnotesize
        \item[+] The Summation indicates that the relative error is calculated after summing the nine modes in \Cref{table:Flux7MHorizon}.
\end{tablenotes}
\end{threeparttable}

\end{table*}


This subsection presents several numerical results to validate the adaptability of our method, and compare their accuracy with those which are already available in the literature for calculating gravitational radiation.
These methods in the literature include numerical integration (NI) method \cite{BHPToolkit}, high-order post-Newtonian expansion \cite{Fujita_2015} , MST-RTE method \cite{Mano_1996,Sasaki_2003} and MST-RWE method \cite{Mano1996RWE,Casals_2015}, whose codes are provided by the black hole perturbation toolkit (BHPT) \cite{BHPToolkit}.
In this subsection, there are abbreviations of some methods, MST-RTE and MST-RWE \footnote{MST-RWE first utilizes the MST method to solve the Regge-Wheeler equation, and then uses the Chandrasekhar-Sasaki-Nakamura transformation to convert it into the solution of the homogeneous Teukolsky equation.}  are the MST methods for solving the radial Teukolsky equation (RTE) and Regge-Wheeler equation (RWE), respectively.
Meanwhile, 27.5PN represents 27.5th post-Newtonian order expansion\footnote{ In BHPToolkit 's package \cite{BHPToolkit}, Fujita has been updating the energy flux of the Schwarzschild black hole at the infinity of its circular orbit, which now reaches 27.5PN order. }, and the HeunC method is the method proposed in this paper.

The BHPT is a collection of multiple scattered black hole perturbation theory codes, which have been developed by various individuals or groups over several decades.
Although the BHPToolkit yields a formal solution of the confluent Heun function that solves the homogenous Teukolsky equation, it solely provides  the ingoing wave solution $R_{\ell m\omega }^{{\rm{in }}}$, but the outgoing wave solution $R_{\ell m\omega }^{{\rm{up}}}$has not been given so far. Therefore, this formal solution
(This solution is composed of the confluent Heun function, but it is different from our solution.) of the BHPToolkit is an imperfect method and cannot calculate gravitational radiation.



In numerical calculations for partial differential equations, the rate of convergence (ROC) is a measure of how well a numerical method approximates the exact solution as the spatial or temporal resolution is increased. It provides insight into the accuracy and reliability of the method.
Therefore, we introduce the ROC of the floating-point numbers\footnote{$\bf N$ is the software floating-point numbers, which can affect the computational efficiency. The larger $\bf N$, the more calculation time is required.} $\bf N$ to evaluate the speed of convergence for the computational methods.
And the ROC of floating-point numbers is defined as \cite{Li2008,CHEN2020125009,Chen2020b}
\begin{equation}
  {\rm{ROC}} = {\log _{10}}\left( \frac{{\bf{RE}}_{\bf N}}{{\bf{RE}}_{{\bf N}+10}} \right),
\end{equation}
where ${{\bf{RE}}_{{\bf N}}}$ and ${{\bf{RE}}_{{\bf N}+10}}$ are the relative error (RE) of ${\bf N}$ and ${\bf N}+10$, respectively.
The exact solution in this paper is the results of the MST-RTE method with ${\bf N} =200$.

This subsection comprises two distinct parts.
The first evaluates the validity of the solution to the Teukolsky equation without the source.
The second part considers the Teukolsky equation with the source and calculates the corresponding energy flux.

%
%E (for exponent) means to multiply the previous number by 10 to the NTH power.
\subsubsection{Test of Homogenous Solutions}

To verify the accuracy of homogenous solutions, we test two solutions $\tilde R_{\ell m\omega }^{{\rm{in,up}}}$ of the HRTE \eqref{eq:HRTE} with the mode $(\ell ,|m|)=(2,2)$.
In \Cref{fig:RinRupError}, we present the logarithmic relative errors of $\tilde R_{\ell m\omega}^{{\rm{in, up}}}$ for three distinct methods (HeunC, MST-RTE, and NI) compared to the exact solution. The analysis is conducted using different sets of floating-point numbers, with the resulting errors plotted at three specific locations: $r_0=5M, 10M, 15M$. And  \Cref{tab:RinRup-Err} shows numerical comparisons of three methods for ingoing wave and outgoing wave solutions $\tilde R_{\ell m\omega }^{{\rm{in,up}}}$.


\Cref{tab:RinRup-Err} highlights the clear trend where the relative error of the HeunC method diminishes considerably with increasing floating-point numbers, surpassing both the MST and NI methods. Notably, the HeunC method exhibits the ROC approach 10, while the MST method and the NI method approach 5.
This means that for each additional floating-point numbers $\bf N$, the relative error of the HeunC method will be multiplied by $10^{-1}$, whereas the relative error of the MST and NI methods will be multiplied by $5\times 10^{-1}$.
Thus, it can be inferred that the HeunC method is a 10th-order method, while the MST method and the NI method are 5th-order methods.
In other words, the convergence speed of the HeunC method is faster than that of MST and NI methods. This phenomenon can be observed in \Cref{fig:RinRupError}.
These results indicate that the HeunC method exhibits significantly superior computational accuracy and efficiency compared to both the MST method and the NI method.






\subsubsection{Test of Energy Flux}


To validate the applicability and effectiveness of the HeunC method, we test two gravitational luminosities $\left<{dE_{\ell m}/dt}\right>_{\infty, {\rm H}}$ of the GFRTE \eqref{eq:GFoTRE}.
\Cref{fig:FluxInf,fig:FluxHor} shows logarithmic values of $\left<{dE_{\ell ,m}/dt}\right>_{\infty, {\rm H}}$ of the three methods (HeunC, MST-RTE and MST-RWE) with different modes, over the domain $[5M,15M] $. It can be seen from \Cref{fig:FluxInf,fig:FluxHor}  that when N = 40,  the energy fluxes obtained by the HeunC method exhibit a commendable agreement with the results obtained by the MST-RTE and MST-RWE methods.
This numerical simulation confirms the physical phenomenon that the closer a point particle approaches a black hole, the greater the amount of radiation energy flux it generates. Furthermore, when $r_0$ is fixed, the maximum energy flux across all modes are $\left<{dE_{2,2}/dt}\right>_{\infty,{\rm H} }$.



In Table \ref{tab:flux-inf-hor}, we provide numerical comparisons of energy fluxes $\left<{dE_{2,2}/dt}\right>_{\infty,{\rm H}}$ obtained from four methods (HeunC, MST-RTE, MST-RWE, and 27.5PN) for different floating-point numbers ${\bf N}$ at $r_0=5M, 10M, 15M$.
Analyzing the ROC presented in \ref{tab:flux-inf-hor}, it can be concluded that the HeunC method is a 10th-order method for the calculation of energy flux, and the MST-RTE and MST-RWE methods are 5th-order methods.
Notably, the value $\left<{dE_{2,2}/dt}\right>_{\infty,{\rm H}}$ of the 27.5PN method does not exhibit convergence with increasing floating-point numbers. Thus, the 27.5PN method is not convergent about the floating-point numbers.
On other hand, the convergence speed of HeunC is faster than that of the MST-RTE and MST-RWE methods.
This behavior is evident from the observations presented in \Cref{fig:FluxLogError}. As $\bf N$ increases, the HeunC method exhibits a notably larger decrease in relative errors in contrast to both the MST-RTE and MST-RWE methods.
Moreover, \Cref{fig:FluxLogError} and ROC of 27.5PN method presented in table \ref{tab:flux-inf-hor}, reveal that  the relative error of the 27.5PN method decreases as $r_0$ increases, while remaining unaffected by the floating-point numbers.
This also exposes a limitation of the 27.5 PN method: it exhibits inaccuracies in the vicinity of the horizon,  but when $r_0\rightarrow \infty $, the results obtained from this method become increasingly accurate.


\Cref{table:Flux22} exhibits the relative errors of energy fluxes within a region $r_0\in [5M,200M] $, aiming to assess the precision of four distinct methods (HeunC, MST-RTE, MST-RWE and 27.5PN) as $r_0$ is close to the horizon or infinity. Notably, in both near the horizon and at infinity, the HeunC method showcases exceptional accuracy in calculating the energy fluxes, surpassing the other three methods by a significant margin.
To explore the precision of energy flux across various modes, \Cref{table:Flux7MHorizon} presents the relative errors of energy flux within nine distinct modes. Determining the complete energy flux entails summing the energy fluxes of all modes. For convenience, we focus on calculating the energy flux of the nine main modes.
It is observed that when $r_0 < r_{\text{ISCO}}$\footnote{The innermost stable circular orbit(ISCO) is only defined in the equatorial plane, that is $r_{\rm ISCO}=6M$.}, the accuracy of the HeunC method seems to exhibit a slight attenuation; nevertheless, its accuracy remains significantly superior to the other three methods.


All numerical results of this test show that the computational accuracy and efficiency of the HeunC method surpasses that of the other three methods to a significant extent.
This is advantageous for waveform construction.










\section{CONCLUSION}\label{sec:Conclusion}

We first reformulate the radial Teukolsky equations into a general form, which is classified into multiple types based on $\Delta_n$. Then, we focus our attention on solving the $\Delta_2$-type Teukolsky equation which includes nine kinds of Teukolsky equations shown in \Cref{table:GFTE}.
For homogenous form \eqref{eq:HRTE} of  the $\Delta_2$-type Teukolsky equation, by the linear combination of two independent particular solutions, we provide a general exact solution:
\begin{equation}\label{eq:GeneralSolution}
{R_{\ell m\omega }} = {C_1}S_{\rm{0}}^\beta (x){\mathbb{H} ^\beta }(x) + {C_2}S_{\rm{0}}^{ - \beta }(x){\mathbb{H} ^{ - \beta }}(x).
\end{equation}
Starting from the general solution \eqref{eq:GeneralSolution}, we can derive the ingoing wave and outgoing wave  $R_{\ell m\omega }^{{\rm{in,up}}}$ solutions via the boundary conditions.
The amplitudes of $R_{\ell m\omega }^{{\rm{in,up}}}$ can be obtained by exploiting the analytical asymptotic form of ${\mathbb{H} ^{\pm \beta} }$.
Next, Green's function method is employed to evaluate $Z^{\infty,{\rm H}}_{\ell m\omega}$, which is used in \Cref{eq:dEdtInf,eq:dEdtH} to compute the energy flux of the gravitational waves. Finally,
the HeunC Method is used for numerical simulations of the gravitational radiation of the Schwarzschild black hole, and the results are entirely satisfactory in comparison with the analytical (27.5th post-Newtonian order expansion) and numerical (MST and numerical integration) methods. The findings are summarized in detail as follows:

1.  For the Teukolsky equation without the source, our method is superior to both the MST-RTE and NI methods in terms of its exceptional precision and efficiency.
Our computation of $\tilde R_{\ell m\omega }^{{\rm{in}}}$ does not require renormalized angular momentum $\nu$ and two-sided infinite summation series which are used in the other methods, so our method produces results with low floating-point numbers $\bf N$ that are approximately equal to the high floating-point numbers $\bf 2N$ outputs obtained by other methods. For instance, the result of calculating $\tilde R_{\ell m\omega }^{{\rm{in}}}$ using the HeunC method with ${\bf N}=50$ is approximately equal to the outcome of the MST-RTE method with ${\bf N}=100$, or the NI method with ${\bf N}=100$.
In the proposed outgoing solution $\tilde R_{\ell m\omega }^{{\rm{up}}}$, the accuracy of $\tilde R_{\ell m\omega }^{{\rm{up}}}$ may experience a slight attenuation compared to $\tilde R_{\ell m\omega }^{{\rm{in}}}$. However, it is important to note that the accuracy of $\tilde R_{\ell m\omega }^{{\rm{up}}}$ still remains significantly better than that of MST-RTE and NI methods.
Ingoing and outgoing wave solutions $\tilde R_{\ell m\omega }^{{\rm{in, up}}}$ serve as exact solutions in the context of our study.
Notably, the disparity in floating-point accuracy observed between $\tilde R_{\ell m\omega }^{{\rm{in}}}$ and $\tilde R_{\ell m\omega }^{{\rm{up}}}$ can be attributed to the dissimilar numerical errors inherent within their respective formal solutions.
Because $\tilde R_{\ell m\omega }^{{\rm{up}}}$ involves the iterative calculation of \Cref{eq:Dinc,eq:Dout}, it causes a marginal loss in accuracy.



2. This paper introduces, for the first time, the asymptotic analytic expression of the confluent Heun function at infinity, resolving a long-standing unresolved problem in mathematics.
The derivation of this analytical formula extends the general solution  \eqref{eq:GeneralSolution} to the Teukolsky equation with source by means of the Green's function method.
For the Teukolsky equation with the source, our method can accurately calculate any mode of the entire spatial region $r_0\in[r_+,\infty)$, and its accuracy is much higher than the results obtained by other methods. Additionally, the PN expansion suffers from slow convergence and limitations in the weak-field approximation, making its results near the event horizon less precise. Even in the far distance, our approach produces superior results to those obtained using the PN expansion, MST-RTE, and MST-RWE methods.


3. Fujita and Tagoshi have demonstrated that the homogeneous solution of the MST method converges very fast \cite{Fujita_2004,Fujita_2005}. However, they did not provide a mathematical evaluation of the convergence specifically for floating-point numbers. The increase in the floating-point numbers is accompanied by improved accuracy but also increased computation time. To this end, we introduce ROC for floating-point numbers.
By comparing the ROC of the five methods, it has been determined that HeunC is a 10th-order method, while MST-RTE, MST-RTE, and NI are all 5th-order methods.
But, the 27.5PN method is not convergent about the floating-point numbers.
These comparisons mean that for each additional floating-point number $\bf N$, the relative error of the HeunC method will be multiplied by $10^{-1}$, whereas the relative error of the MST and NI methods will be multiplied by $5\times 10^{-1}$.
This indicates that the HeunC method exhibits a higher reduction in relative errors compared to the MST and NI methods when  $\bf N$  increases.

All in all, the method presented in this paper can be employed to accurately compute gravitational radiation to any desired floating-point precision or high post-Newtonian order. Furthermore, our general solution is beneficial for solving the Teukolsky equation in complicated spacetimes ( Kerr-Newman anti-de Sitter BHs \cite{Khanal_1983,Suzuki_1998,Suzuki_1999} ), modified gravity theories \cite{Li_2023}, or effective-one-body theories based on post-Minkowskian approximation \cite{Jing_2022,Jing_2023}.


\section*{Acknowledgement}
This work was supported by the Grant of NSFC No. 12035005,
and National Key Research and Development Program of China No. 2020YFC2201400.
%%%%%%%%%%%%%%%%%%%%%%%%%%%%%%%%%%%%%%%%%%%%%%%%%%%%%%%%%%%%%%%%
%%%%%%%%%%%%%%%%%%%%%%%%%%%%%%%%%%%%%%%%%%%%%%%%%%%%%%%%%%%%%%%%

\bibliography{mybibfile}

\end{document}



