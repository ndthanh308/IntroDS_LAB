% This is samplepaper.tex, a sample chapter demonstrating the
% LLNCS macro package for Springer Computer Science proceedings;
% Version 2.20 of 2017/10/04
%
\documentclass[runningheads]{llncs}
%
\usepackage{graphicx}
\usepackage{amsmath, calc}
\usepackage[margin=1.5in]{geometry}

\usepackage[misc,geometry]{ifsym}

\usepackage{xcolor}


% Used for displaying a sample figure. If possible, figure files should
% be included in EPS format.
%
% If you use the hyperref package, please uncomment the following line
% to display URLs in blue roman font according to Springer's eBook style:
% \renewcommand\UrlFont{\color{blue}\rmfamily}

\begin{document}
%

\title{Our Nudges, Our Selves: Tailoring Mobile User Engagement Using Personality}
\titlerunning{Our Nudges, Our Selves: Tailoring Mobile User Engagement Using Personality}
% If the paper title is too long for the running head, you can set
% an abbreviated paper title here
%

\author{Nima Jamalian\inst{1}\and
Marios Constantinides\textsuperscript{(\Letter)}\inst{2} \and
Sagar Joglekar\inst{2} \and 
Xueni Pan\inst{1} \and 
Daniele Quercia\inst{2}}
%

\authorrunning{Jamalian et al.}
% First names are abbreviated in the running head.
% If there are more than two authors, 'et al.' is used.
%

\institute{Goldsmiths, University of London, London, UK \email{\{n.jamalian, x.pan\}@gold.ac.uk}\\ \and
Nokia Bell Labs, Cambridge, UK
\email{\{marios.constantinides, sagar.joglekar, daniele.quercia\}@nokia-bell-labs.com}\\}


\maketitle              % typeset the header of the contribution
%
\begin{abstract}
To increase mobile user engagement, current apps employ a variety of behavioral nudges, but these engagement techniques are applied in a one-size-fits-all approach. Yet the very same techniques may be perceived differently by different individuals. To test this, we developed HarrySpotter, a location-based AR app that embedded six engagement techniques. We deployed it in a 2-week study involving 29 users who also took the Big-Five personality test. Preferences for specific engagement techniques are not only descriptive but also predictive of personality traits. The Adj. $R^2$  ranges from 0.16 for conscientious users (encouraged by competition) to 0.32 for neurotic users (self-centered and focused on their own achievements), and even up to 0.61 for extroverts (motivated by both exploration of objects and places). These findings suggest that these techniques need to be personalized in the future.

\keywords{ mobile engagement  \and gamification \and personality traits \and personalization.}
\end{abstract}
%
%
%

%%%%%%%%%%%%%%%%%%%%%%%%%%%%%%%%%%%%%%%%%%%%%%%%%%%%%%%%%%%%%%%%%%%%%%%%%%%%%%%%
\section{Introduction}

Autonomous driving (AD) %with deep learning networks 
has shown promising achievements and is considered an important technological breakthrough that could revolutionize the future of transportation. Currently, ensuring the safety of autonomous driving systems has become a topic of extensive development.
% There has been much discussion on how to verify the safety of autonomous driving systems.
One traditional solution for safety tests is to exhaustively enumerate real scenarios for validation. Nevertheless, this process is not only labor-intensive and costly but also dangerous. Simulation has emerged as a robust, safe, and efficient alternative for training and evaluating AD software and algorithms~\cite{li2019aads, amini2020learning, amini2022vista}.

% Figure environment removed

Recently, neural radiance field (NeRF)~\cite{mildenhall2020nerf} has gained significant attention in AD simulation~\cite{drivesim}. This approach leverages multi-view images to construct a 3D scene and enable novel view synthesis for both indoor and outdoor applications. When it comes to constructing NeRF models in AD simulation, there are two options available: 1) collecting a large amount of data to cover as many viewpoints as possible, and constructing a fine-grained scene offline; 2) directly using log data from road tests to quickly create an environment and dynamically simulate driving scenarios. The first choice can deliver high-quality simulation~\cite{tancik2022block} by transforming the problem of view extrapolation into view interpolation through the use of large amounts of data. However, it is time- and cost-intensive, which makes it challenging to generalize. As for the second choice, the collected images from log data are usually similar to each other along the running trajectory, which may result in unsatisfactory outcomes, particularly when the camera pose is placed out-of-trajectory (see \figref{figSupportComp} as an example), semantic consistency cannot be guaranteed when synthesizing images from deviated views. We observe this problem under this data condition in all neural radiance approaches, and to the best of our knowledge, none of the existing work has solved this issue.
In our opinion, semantic consistency is crucial for AD simulation, and synthesizing on deviated views is unavoidable for scalability.

AD simulation usually involves map data for planning and control, which can be obtained from a prebuilt High-Definition Map (HD Map) or an online mapping module. While the map data may not be pixel-perfect, it can provide semantic-level information that is useful for enhancing the semantic consistency of the trained neural radiance field.
In this paper, we propose incorporating map priors into neural radiance fields to enhance the semantic consistency and rendering quality of deviated driving view synthesis. Firstly, we employ ground information from maps to supervise the density field of NeRF, providing a more reliable road base for semantic entities. Next, we propose sampling rays to simulate unseen views. Unlike most NeRF augmentation methods~\cite{zhang2022ray, chen2022geoaug}, we utilize ground and lane information in sampling computations to guide the radiance field. More importantly, we model the above two supervision methods as weak supervision by using an uncertainty parameter and propose an uncertainty tempering scheme to increase the uncertainty. This ensures that map priors only guide the training process rather than enforce it towards their absolute values. As a result, our proposed method not only improves the rendering quality of interpolated novel view synthesis quantitatively but also enhances the semantic consistency of deviated novel view synthesis. 
Our contributions can be summarized as follows:
% We summarize the contributions of this paper as follows.



% To overcome the limitations of the collected data, this paper proposes a novel approach that leverages map information to enhance the semantic consistency of the synthesized driving views. 

% Autonomous driving (AD) vehicles are being trained with the help of deep learning networks and have shown promising achievements. This technology is considered to be a breakthrough that could change the way of transportation in the near future. However, there are many discussions on how to verify or judge the safety of autonomous driving systems.
% A straightforward solution towards the safety tests is to exhaustively enumerate real scenarios for validation as many as possible. However, the process of implementing different real scenarios is not only labor-intensive and costly, but also dangerous. Simulation has been proved to be an alternative, which is robust, safe, efficient in training, and evaluating AD software and algorithms.
% Now, the emerging technology of neural radiance field (NeRF)~\cite{} leverages multi-view images to construct a 3D scene and enable novel view synthesis for many indoor and outdoor applications. For AD simulation, there are two choices for constructing NeRF models: 1) collect a large amount of data, such as LiDAR and camera data, similar to mapping, to construct a fine-grained scene offline; or 2) directly use the log file (typically in the format of ROS bag) to rapidly create an environment and then dynamically simulate the driving scenarios.
% The first choice can achieve high-quality simulation, but it is time-consuming and expensive, making it difficult to generalize to very large scales. On the other hand, the second option is fast but can lead to low-quality simulation due to the data being sparse and similar to each other in log data. This paper tackles the problem raised by choosing the latter option and attempts to improve the quality of out-of-trajectory driving view synthesis by incorporating map information. This approach is practical for many autonomous driving tests.
% In conclusion, the use of NeRF technology for AD simulation is a promising avenue for training and evaluating AD software and algorithms. While both options for constructing NeRF models have their pros and cons, this paper addresses the challenges of the second option and proposes a potential solution to improve the quality of simulation.

%There exist a few attempts to facilitate training a NeRF model for synthesizing out-of-trajectory (or called as extrapo trajectory) views.


\begin{itemize}
    \item We propose a novel method to incorporate commonly used map priors in AD scenes into neural radiance fields to improve the out-of-trajectory driving view synthesis.
    \item We explicitly model the uncertainty in map priors as a parameter and propose an uncertainty tempering scheme to guide the training process of the neural radiance field.
    \item Experiments demonstrated that the proposed method can improve the semantic consistency of out-of-trajectory views and the rendering quality of novel view trajectory interpolation.
\end{itemize}

Our proposed method is easy to implement, can be easily plugged into existing NeRF algorithms, and has the capability of extending to other formats of priors.
\section{Related Work}
\label{sec:related}
We surveyed various lines of research that our work draws upon, and grouped them into two main areas: \emph{i)} AI regulation and governance, and \emph{ii)} responsible AI practices and toolkits. 

\subsection{AI Regulation and Governance}
The landscape of AI regulation and governance is constantly evolving~\cite{jobin2019global, mittelstadt2016ethics}. At the time of writing, the European Union (EU) has endorsed new transparency and risk-management rules for AI systems known as the EU AI Act~\cite{eu_ai_act}, which is expected to become law in 2023. Similarly, the United States (US) has recently passed a blueprint of the AI Bill of Rights in late 2022~\cite{us_ai_bill}. This bill comprises \emph{``five principles and associated practices to help guide the design, use, and deployment of automated systems to protect the rights of the American public in the age of AI.''} While both the EU and US share a conceptual alignment on key principles of responsible AI, such as fairness and explainability, as well as the importance of international standards (e.g., ISO), the specific AI risk management regimes they are developing are potentially diverging, creating an ``artificial divide''~\cite{ecfr}. The EU aims to become the leading regulator for AI globally, while the US takes the view that excessive regulation may impede innovation.

Notable predecessors to AI regulations include the EU GDPR law on data protection and privacy~\cite{eu_gdpr}, the US Anti-discrimination Act~\cite{us_anti_discrimination}, and the UK Equality Act 2010~\cite{uk_equality}. GDPR's Article 25 mandates that data controllers must implement appropriate technical and organizational measures during the design and implementation stages of data processing to safeguard the rights of data subjects. The Anti-discrimination Act prohibits employment decisions based on an individual's race, color, religion, sex (including gender identity, sexual orientation, and pregnancy), national origin, age (40 or older), disability, or genetic information. This legislation ensures fairness in AI-assisted hiring systems. Similarly, the UK Equality Act provides legal protection against discrimination in the workplace and wider society.

The National Institute of Standards and Technology (NIST), a renowned organization for developing frameworks and standards, recently published an AI risk management framework~\cite{nist2023aiRisk}. According to the NIST framework, an AI system is defined as \emph{``an engineered or machine-based system capable of generating outputs such as predictions, recommendations, or decisions that influence real or virtual environments, based on a given set of objectives. These systems are designed to operate with varying levels of autonomy.''} Similarly, the Principled Artificial Intelligence white paper from the Berkman Klein Center~\cite{fjeld2020principled} highlights eight key thematic trends that represent a growing consensus on responsible AI. These themes include privacy, accountability, safety and security, transparency and explainability, fairness and non-discrimination, human control of technology, professional responsibility, and the promotion of human values.

As AI regulation and governance continue to evolve, AI practitioners are faced with the challenge of staying updated with the changing guidelines and regulations, requiring significant time and effort. Therefore, the focus of this work is to develop an adaptable methodology for generating responsible AI guidelines.

\subsection{Responsible AI Practices and Toolkits}
\subsubsection{Responsible AI Practices.} 
\label{sec:sub-raipractices}
A growing body of research, typically discussed in conferences with a long-standing commitment to human-centered design, such as the Conference on Human Factors in Computing Systems (CHI) and the Conference on Computer-Supported Cooperative Work and Social Computing (CSCW), as well as in newer conferences like the Conference on AI, Ethics, and Society (AIES) and the Conference on Fairness, Accountability, and Transparency (FAccT), focuses on the work practices of AI practitioners in addressing responsible AI issues. This strand of research encompasses various aspects of responsible AI, including fairness, explainability, sustainability, and best practices for data and model documentation and evaluation.

Fairness is a fundamental value in responsible AI, but its definition is complex and multifaceted~\cite{narayanan21fairness}. To assess bias in classification outputs, various research efforts have introduced quantitative metrics such as disparate impact and equalized odds, as discussed by Dixon et al.~\cite{dixon2018measuring}. Another concept explored in the literature is ``equality of opportunity,'' advocated by Hardt et al.~\cite{hardt2016equality}, which ensures that predictive models are equally accurate across different groups defined by protected attributes like race or gender.

Explainable AI (XAI) is another aspect of responsible AI. XAI involves tools and frameworks that assist end users and stakeholders in understanding and interpreting predictions made by machine learning models~\cite{arrieta2020explainable, kulesza2015principles, gunning2019xai,liao2021human, ehsan2020human, ibm2019fairness}. Furthermore, the environmental impact of training AI models should also be considered. Numerous reports have highlighted the significant carbon footprint associated with deep learning and large language models~\cite{sharir2020cost, hao2019training, strubell2019energy}. 

Best practices for data documentation and model evaluation have also been developed to promote fairness in AI systems. Gebru et al.~\cite{gebru2021datasheets} proposed ``Datasheets for Datasets'' as a comprehensive means of providing information about a dataset, including data provenance, key characteristics, relevant regulations, test results, and potential biases. Similarly, Bender et al.\cite{bender2018data} introduced ``data statements'' as qualitative summaries that offer crucial context about a dataset's population, aiding in identifying biases and understanding generalizability. For model evaluation, Mitchell et al.~\cite{mitchell2019model} suggested the use of model cards, which provide standardized information about machine learning models, including their intended use, performance metrics, potential biases, and data limitations. Transparent reporting practices, such as the TRIPOD statement by Collins et al.~\cite{collins2015transparent} in the medical domain, emphasize standardized and comprehensive reporting to enhance credibility and reproducibility of AI prediction models.\\

\subsubsection{Responsible AI Toolkits} Translating these practices into practical responsible AI is another area of growing research. New tools and frameworks are being proposed to assist developers in mitigating biases~\cite{bird2020fairlearn, gebru2021datasheets}, explaining algorithmic decisions~\cite{arya2019one}, and ensuring privacy-preserving AI systems~\cite{fjeld2020principled}.

Fairness auditing tools typically offer a set of metrics to test for potential biases, and algorithms to mitigate biases that may arise in AI models~\cite{saleiro2018aequitas, baeza2018bias}. For instance, Google's fairness-indicators toolkit~\cite{google2022fairness} enables developers to evaluate the distribution of datasets, performance of models across user-defined groups, and delve into individual slices to identify root causes and areas for improvement. IBM's AI Fairness 360~\cite{ibm2022ai} implements metrics for comparing subgroups of datasets (e.g., differential fairness and bias amplification~\cite{foulds2020intersectional}) and algorithms for mitigating biases (e.g., learning fair representations~\cite{zemel2013learning}, adversarial debiasing~\cite{zhang2018mitigating}). Microsoft's Fairlearn provides metrics to assess the negative impact on specific groups by a model and compare multiple models in terms of fairness and accuracy metrics. It also offers algorithms to mitigate unfairness across various AI tasks and definitions of fairness~\cite{fairlearn2022}.

Explainable AI systems are typically achieved through interpretable models or model-agnostic methods. Interpretable models employ simpler models like linear or logistic regression to explain the outputs of black-box models. On the other hand, model-agnostic methods (e.g., LIME~\cite{ribeiro2016should} or SHAP~\cite{lundberg2017unified}) have shown effectiveness with any model. IBM's AI Explainability 360 provides metrics that serve as quantitative proxies for the quality of explanations and offers guidance to developers and practitioners on ensuring AI explainability~\cite{ibm2022ai}. Another research direction introduced new genres of AI-related visualizations for explainability, drawing inspiration from domains such as visual storytelling, uncertainty visualizations, and visual analytics. Examples include Google's explorables, which are interactive visual explanations of the internal workings of AI techniques~\cite{google2022pair}; model and data cards that support model transparency and accountability (e.g., NVIDIA's Model Card++)\cite{nvidia2022}; computational notebook additions for data validations like AIF360\cite{ibm2022ai}, Fairlearn~\cite{fairlearn2022}, and Aequitas~\cite{saleiro2018aequitas}; and data exploration dashboards such as Google's Know Your Data~\cite{google2022know} and Microsoft's Responsible AI dashboard~\cite{microsoft2022aiLab}.

Ensuring privacy-preserving AI systems is commonly attributed to the practice of ``Privacy by Design''~\cite{cavoukian2009privacy, cavoukian2010privacy}, which involves integrating data privacy considerations throughout the AI lifecycle, particularly during the design stage to ensure compliance with laws, regulations, and standards~\cite{fjeld2020principled} such as the European General Data Protection Regulation (GDPR)~\cite{eu_gdpr}. IBM's AI Privacy 360 is an example of a toolkit that assesses privacy risks and helps mitigate potential privacy concerns. It includes modules for data anonymization (training a model on anonymized data) and data minimization (collecting only relevant and necessary data for model training) to evaluate privacy risks and ensure compliance with privacy regulations.

While many toolkits and frameworks emphasize the importance of involving stakeholders from diverse roles and backgrounds, they often lack sufficient support for collaborative action. Wong et al.~\cite{wong2023seeing} have also highlighted the ``mismatch between the promise of toolkits and their current design'' in terms of supporting collaboration. Collaboration is key to enhance creativity by allowing AI practitioners to share knowledge with other stakeholders. To address this gap, we aim to develop a set of actionable guidelines that will facilitate the engagement of a diverse range of stakeholders in AI ethics. By doing so, we hope to take a significant step forward in fostering collaboration and inclusivity within the field.
\section{The Big-Five Personality}
\label{sec:hypotheses}
The Big-Five personality model assigns individuals scores~\cite{gosling2003very}, representing the main personality traits of Openness, Conscientiousness, Extroversion, Agreeableness, and Neuroticism. We hypothesized the relationship between these traits and our six engagement strategies, and summarized the positive and negative relationships of these hypotheses in Table~\ref{tab:questions}. \\

\noindent\textbf{\emph{Openness}} is associated with descriptive terms such as imaginative, spontaneous, and adventurous. Individuals high in Openness are more likely to try new methods of communication, including social networking sites or mobile apps. For example, studies have reported that individuals high in Openness tend to utilize a greater number of features that facilitate exploration in such technologies~\cite{ross2009personality}.

\noindent\textbf{\emph{Conscientiousness}} is associated with traits like ambition, resourcefulness, and persistence. Individuals high in Conscientiousness are less likely to engage in mobile content generation. They often view computer-mediated communication as a distraction from their daily tasks~\cite{amichai2010social}. However, when they do engage in such communication, they tend to approach it in a highly methodical and competitive manner. Their motivation is often driven by a desire for positive competition~\cite{halko2010personality}.

\noindent\textbf{\emph{Extraversion}} is associated with descriptive terms such as sociability, activity, and excitement seeking.  Individuals high in Extraversion typically prefer face-to-face interactions and are less inclined to utilize social networking sites or mobile apps. However, if they do join such platforms, they often participate in multiple groups, contribute content, and are motivated by the positive aspect of exploration as a means of social stimulation~\cite{phillips2006personality}.

\noindent\textbf{\emph{Agreeableness}} is associated with descriptive terms such as trusting, altruistic and tender-minded. Individuals high in Agreeableness, who are less competitive~\cite{halko2010personality} and less likely to share content~\cite{amichai2010social}, are more likely to be negatively motivated by rewards or competition.

\noindent\textbf{\emph{Neuroticism}} is associated with descriptive terms such as emotional liability and impulsiveness. Individuals with high levels of Neuroticism exhibit diverse behaviors across different media platforms. They tend to use the Internet and mobile apps as a means to alleviate loneliness, share accurate personal information in anonymous online forums (e.g., chat rooms), exercise control over their shared information on mobile devices~\cite{butt2008personality}, and focus on their own achievements in positive ways~\cite{lane2012influence}.

\begin{table*}[t]
\centering
\caption{Question statements assessing HarrySpotter's six engagement strategies. Positive and negative signs indicate the association of these engagement strategies with each personality trait as found in prior literature, and empty cells indicate that no reference has been found. O: Openness; C: Conscientiousness; E: Extraversion; A: Agreeableness; and N: Neuroticism. }
\resizebox{\textwidth}{!}
{\begin{tabular}{l l l l l l l} 
 \hline
 \textbf{Strategy} & \textbf{Question Statement} & \textbf{O} & \textbf{C} & \textbf{E} & \textbf{A} & \textbf{N} \\  \hline
 Q1(Point Rewards) & I pay attention to others' spell energy scores. &  &  &  & - & \\
 Q2(Place Rewards) & I am proud of my mayorships. &  &  &  & - & +\\ 
 Q3(Game with Yourself) & When I play the game, I feel I am representing my house. &  &  &  &  & + \\
 Q4(Social Connection) & With HarrySpotter, I track the competition among the four houses. &  & + &  & -  &\\
 Q5(Object Discovery) & HarrySpotter motivated me to discover new objects. & + &  & + &  &\\
 Q6(Place Discovery) & HarrySpotter motivated me to visit new places.  & + &  & + &  &\\
    \hline
\end{tabular}
}
\label{tab:questions}
\end{table*}
\section{HarrySpotter}
\label{sec:mobilegame}

% Figure environment removed

We developed a mobile app called HarrySpotter, which incorporates gameplay elements inspired by the popular Harry Potter series. Authored by J.K. Rowling, the Harry Potter series revolves around the adventures of a young wizard named Harry Potter, his friends, and their quest to defeat the dark wizard Lord Voldemort. Our app draws inspiration from the series' concept of affiliation through four houses, namely Gryffindor (known for courage and bravery), Hufflepuff (emphasizing hard work and patience), Ravenclaw (highlighting intelligence and learning), and Slytherin (representing ambitions and cunning). HarrySpotter was developed using the Unity game engine for both Android and iOS platforms. The app uses the Mapbox SDK for location-based features, and Vuforia SDK to deliver an augmented reality experience, particularly during the process of claiming a mayorship.

HarrySpotter employs six strategies to engage users in the task of annotating objects: Point Rewards, Places Rewards, Game with Yourself, Social Connection, Object Discovery, and Place Discovery. These strategies were initially derived from the work of Lindqvist et al.~\cite{lindqvist2011m}, but were modified to align with our gameplay's requirements. For example, the effectiveness of badges in Lindqvist's study was evaluated using the question statement: ``I pay attention to the badges that others earn.'' In our case, to assess the effectiveness of point rewards (which function as a type of badge), we adapted the statement to: ``I pay attention to others' spell energy scores'' (Table\ref{tab:questions}). Additionally, as our gameplay includes two types of rewards---points and place rewards---we categorized them separately, resulting in a total of six strategies, contrasting with the five strategies described in~\cite{lindqvist2011m}. \\ 

\noindent 
\textbf{Point and Place Rewards}: Previous research has demonstrated the effectiveness of reward systems, such as points, in engaging users with mobile apps~\cite{brauer2019badges}. In HarrySpotter, users are rewarded with spell energy for annotating new objects (Figure~\ref{fig:strategies}d), which reflects their ability to claim mayorships of places. When a user annotates an object, the app compares the user-generated label (object name) with the label automatically detected by an image classifier running on our server. The classifier used is a deep-learning ResNet-162 model with a top 5\% accuracy of 94.2\% on ImageNet classes. The semantic distance between the user-generated label and the automatically detected label is computed using WordNet~\cite{miller1998wordnet}. If they match, the user receives extra spell energy. Additionally, the app tracks previously scanned object types and rewards the user when they scan a new type for the first time. However, if the user scans the same object type repeatedly, the reward amount decreases until it reaches the minimum of 10 points. This design choice ensures that the user's score does not reach zero and maintains a balance between engagement and avoiding penalization, such as providing incorrect labels or repetitive images.

\noindent \textbf{Game With Yourself}: Users have the option to play the game alone, engaging in various single-player elements such as object annotation, places, and challenging mayorships (Figure~\ref{fig:strategies}b-d). When it comes to mayorships, a user can become the mayor of a place. Subsequently, other users have the opportunity to visit that place and challenge the current mayor. This feature adds a competitive aspect to the game, even when playing individually.

\noindent\textbf{Social Connection}: Previous research has demonstrated that leaderboards are effective in enhancing user performance in various tasks~\cite{brauer2019badges}. In HarrySpotter, during the onboarding process, users respond to a series of questions inspired by the Harry Potter sorting hat quiz (Figure~\ref{fig:strategies}a) and are sorted into one of the Harry Potter houses~\cite{jakob2019science}. Through the leaderboard, we encourage users to actively participate in the game and contribute to their respective houses' efforts in claiming mayorships of different places.

\noindent \textbf{Object Discovery}: Enabling users to explore and discover new places or objects is a crucial aspect of location-based apps. Previous research has demonstrated that incorporating points of interest (similar to Pok\'emon GO) encourages users to engage with the app while on the move and at various locations~\cite{althoff2016influence}. In HarrySpotter, we motivate users to explore different locations by allowing them to become mayors of real places (Figure~\ref{fig:strategies}c). When a user is within the mayorship location range, an AR mage appears and challenges the user for mayorship. The user's spell energy (points) plays a significant role in their chances of claiming the mayorship. To strike the right balance, we set the mayorship range to an 80-meter radius based on empirical evidence. Lower ranges limited accessibility, while higher ranges diminished proximity and overall engagement. Through this strategy, we encourage users to discover and scan objects they may have overlooked in new locations, fostering exploration and engagement.

\noindent \textbf{Place Discovery.} Previous research on the motivations behind using location-based apps has revealed that users are driven by their curiosity to obtain information about specific points of interest~\cite{lindqvist2011m}. This curiosity acts as an incentive for users to actively pursue becoming the mayor of those places. In HarrySpotter, when a user successfully claims mayorships of places, their map visually represents a sense of territorial ownership (Figure~\ref{fig:strategies}b). For example, if a user belongs to Hufflepuff house and becomes a mayor, their map pins will be displayed in yellow, symbolizing their affiliation with Hufflepuff.
\section{User Study}
\label{sec:userstudy}

\noindent
\textbf{Participants and Ethical Considerations.} We deployed HarrySpotter in a 2-week study with 29 users (13 female), aged between 18-49 years (median = 34). To be eligible for the study, participants were required to own an Android or iOS smartphone and be located in London, UK. In compliance with GDPR and the Data Protection Act, all individual user data were anonymized to ensure the privacy and confidentiality of the participants. The study was approved by the Ethics Committee of Goldsmiths, University of London.

\noindent \textbf{Procedure.} All participants underwent a pre-screening process where we collected demographic information and obtained the unique identifier of their device for generating the app download link. After installing the app, participants were prompted to grant access to the camera and location. Basic instructions were provided on how to use the app, such as annotating objects, with no specific guidelines on what to annotate or how frequently. To maintain study integrity, no information regarding the relationship between personality and engagement techniques was revealed to the participants.

\noindent \textbf{Materials and Apparatus.} At the end of the study, we administered a 6-item questionnaire (Table~\ref{tab:questions}) and the 10-item TIPI personality questionnaire~\cite{gosling2003very}. The 6-item questionnaire included statements derived from~\cite{lindqvist2011m} and had previously been validated in the context of the Foursquare app to assess users' motivations for engagement. Participants rated both questionnaires using a 7-point Likert scale (1: Strongly Disagree; 7: Strongly Agree).

\noindent
\textbf{Self-reports and Big-Five personality traits.} We coded the Likert-scale answers to the 6-item questionnaire and the TIPI~\cite{gosling2003very}. On average, our participants scored as follows on a 1-7 scale: average in Openness ($\mu\textrm{=5.14}$, $\sigma\textrm{=0.8}$), high in Conscientiousness ($\mu\textrm{=5.12}$, $\sigma\textrm{=0.85}$), average in Neuroticism ($\mu\textrm{=4.59}$, $\sigma\textrm{=0.85}$), average in Agreeableness ($\mu\textrm{=4.5}$, $\sigma\textrm{=0.71}$), and low in Extraversion ($\mu\textrm{=4.43}$, $\sigma\textrm{=0.94}$). These trait distributions aligned with the normative personality values derived from a large sample of the U.S. population~\cite{soto2011age}.

\noindent\textbf{Annotations.} Each annotation in our study involved storing the raw image and its corresponding label in a database. To ensure data quality, we implemented checks for image duplication and semantic correctness. To prevent duplication, we utilized FAISS~\cite{johnson2019billion}, a framework for indexing images based on visual similarity. This allowed us to retrieve the most visually similar images for comparison. We penalized scores for each annotation based on visual similarity to the user's previously uploaded images. For instance, if an image closely resembled a previously captured one, the user would not receive a reward in the form of spell energy. To assess semantic correctness, we first subjected the uploaded image to an off-the-shelf object detector~\cite{cheng2019panoptic}. We then calculated the WordNet semantic distance~\cite{miller1998wordnet} between the detected label and the user-generated label. The awarded spell energy was proportional to the semantic similarity, discouraging grossly inaccurate or garbled labels. Regarding annotation quantity, we recorded the total number of annotations $n_k$ uploaded by each user $k$ along with their respective images. For annotation quality, three independent annotators rated each annotation on a 1-5 Likert scale, with 5 indicating a perfect match between the image and the user-generated label. For example, if an image depicted a ``computer mouse'' and the user's label was ``mouse,'' the annotator would assign a score of 5. To ensure reliable results, we calculated a Fleiss kappa score of 0.57, indicating moderate to good agreement among the three annotators. We compiled a set of $n$ images $I$ annotated by each user $k$ as ${I_1, I_2, ..., I_n}$. The quality score for user $k$ was determined by the median of the quality scores assigned to their $n$ annotated images.

Before using the six self-reports and the quantity and quality of annotation metrics into our regression models, we conducted a Shapiro-Wilk test for normality. As the eight variables exhibited skewed distributions, we applied a log transformation to them. Among the five personality traits, only Extraversion showed a slight skewness, so we also applied a log transformation to it.



\section{Results}
\label{sec:results}
To ease the interpretation of our results, we applied a min-max transformation to scale our variables within the range of [0-100]. We first examined the pairwise correlation among personality traits, the six self-reports, and the quantity and quality of annotations metrics. We found that neurotic users discovered fewer objects ($\textrm{r=-0.37, p\textless0.1}$). Additionally, these users tended to take pride in their mayorships ($\textrm{r=0.44, p\textless0.05}$) and their affiliation with their respective Harry Potter house ($\textrm{r=0.47, p\textless0.05}$). Conversely, Extroverts and those high in Openness liked to discover new objects ($\textrm{r=0.44, p\textless0.05}$ and $\textrm{r=0.39, p\textless0.05}$, respectively) and new places ($\textrm{r=0.38, p\textless0.05}$ and $\textrm{r=0.35, p\textless0.1}$, respectively). 

\begin{table*}[ht!]
\centering
\caption{Linear regressions that predict the Big-Five personality traits from the six self-reports and the quantity and quality of annotations. Significant predictors with $p$ values $<$ .05 are marked in bold. The most predictable personality trait was Extraversion ($M_{E}$), while the least predictable was Conscientiousness ($M_{C}$).}
   % \resizebox{\textwidth}{!}{%
      \begin{tabular}{lccc}
        \hline
        $\boldsymbol{M_{O}}$: $\boldsymbol{\emph{Adj $R^2$} = 0.28}$, Durbin-Watson = 1.96, AIC = 0.47 & & &  \\
        \hline
        Predictor & $\beta$ & std. error & $p$-value \\
        \hline
        Intercept & 0.42 & 0.13 & 0.005\\
        Q1(Point Rewards) & -1.08 & 0.36 & \textbf{0.01}\\ 
        Q2(Place Rewards) & 0.42 & 0.35 & 0.23\\ 
        Q5(Object Discover) & 0.84 & 0.27 & \textbf{0.01}\\ 
        \hline
        $\boldsymbol{M_{C}}$: $\boldsymbol{\emph{Adj $R^2$} = 0.16}$, Durbin-Watson = 1.15, AIC =  -7.48 & & &  \\
        \hline
        Predictor & $\beta$ & std. error & $p$-value \\
        \hline
        Intercept & 0.57 & 0.12 & 0.00\\
        Q3(Game with Yourself) & 0.59 & 0.22 & \textbf{0.01}\\ 
        Q6(Place Discovery) & -0.48 & 0.23 & \textbf{0.04}\\
        \hline
        $\boldsymbol{M_{E}}$: $\boldsymbol{\emph{Adj $R^2$} = 0.61}$, Durbin-Watson = 1.88, AIC = -14 & & &  \\
        \hline
        Predictor & $\beta$ & std. error & $p$-value \\
        \hline
        Intercept & 0.48 & 0.19 & 0.02\\
        Q2(Place Rewards) & -0.60 & 0.22 & \textbf{0.01}\\ 
        Q4(Social Connection) & -0.72 & 0.22 & \textbf{0.00}\\ 
        Q5(Object Discovery) & 0.53 & 0.22 & \textbf{0.02}\\ 
        Q6(Place Discovery) & 0.99 & 0.28 & \textbf{0.00}\\
        Quantity & -0.37 & 0.12 & \textbf{0.00}\\
        Quality & 0.18 & 0.16 & 0.27\\
        \hline
        $\boldsymbol{M_{A}}$: $\boldsymbol{\emph{Adj $R^2$} = 0.21}$, Durbin-Watson = 1.84, AIC = -12.86 & & &  \\
        \hline
        Predictor & $\beta$ & std. error & $p$-value \\
        \hline
        Intercept & 0.57 & 0.11 & 0.00\\
        Q1(Point Rewards) & -0.45 & 0.24 & 0.07\\  
        Q3(Game with Yourself) & 0.21 & 0.21 & 0.31\\
        Q5(Object Discovery) & -0.37 & 0.24 & 0.13\\
        Q6(Place Discovery) & 0.71 & 0.29 & \textbf{0.02}\\  
        \hline
        $\boldsymbol{M_{N}}$: $\boldsymbol{\emph{Adj $R^2$} = 0.32}$, Durbin-Watson = 2.79, AIC = -3.68 & & &  \\
        \hline
        Predictor & $\beta$ & std. error & $p$-value \\
        \hline
        Intercept & 0.36 & 0.14 & 0.02\\
        Q2(Place Rewards) & 0.37 & 0.26 & 0.17\\  
        Q4(Social Connection) & 0.58 & 0.27 & \textbf{0.04}\\
        Q6(Place Discovery) & -0.58 & 0.29 & 0.06\\
        Quantity & -0.32 & 0.14 & \textbf{0.03} \\
        \hline
      \end{tabular}
      % }
      \label{tab:regression}
\end{table*}

Considering these significant correlations, one might wonder whether it is possible to predict users' personality traits based on their self-reports and the quantity and quality annotation metrics. Using these metrics as predictors, we fitted five linear regression models (Table~\ref{tab:regression}) to predict the Big-Five personality dimensions and determined the best set of predictors using the stepAIC function~\cite{zhang2016variable}. Overall, our findings indicate that predicting certain personality dimensions, such as Extraversion (Adj. $R^2$ = 0.61), was relatively easier compared to others like Conscientiousness (Adj. $R^2$ = 0.16).

As expected, users with non-competitive traits (high in Agreeableness) demonstrated a lack of motivation for competing with others ($\beta_\textrm{(Q1(Point Rewards))}\textrm{=-0.45}$). On the other hand, conscientious users, known for their organizational skills, exhibited motivation for competition ($\beta_\textrm{(Q1(Point Rewards))}\textrm{=0.59}$) among the four houses but moderately predicted their personality trait (Adj. $R^2$\textrm{=0.16}). Individuals open to new experiences and extroverts were primarily motivated by the discovery of new objects ($\beta_\textrm{(Q5(Object Discovery))}\textrm{=0.84}$ and $\beta_\textrm{(Q5(Object Discovery))}\textrm{=0.53}$, respectively). Moreover, individuals open to new experiences did not find motivation in competing with others ($\beta_\textrm{(Q1)}\textrm{=-1.08}$), while extroverts were not motivated by mayorships ($\beta_\textrm{(Q2(Place Rewards))}\textrm{=-0.60}$) or representing their own house ($\beta_\textrm{(Q1(Point Rewards))}\textrm{=-0.72}$). Emotionally unstable users (neurotics) found motivation in representing their own house ($\beta_\textrm{(Q4(Social Connection))}\textrm{=0.58}$) but not in the discovery of new places ($\beta_\textrm{(Q6(Place Discovery))}\textrm{=-0.58}$), leading them to discover fewer objects ($\beta_\textrm{(Quantity)}\textrm{=-0.32}$).
\section{Discussion and Conclusion}
\label{sec:discussion}
Mobile user engagement is commonly pursued through a range of techniques; however, there is a tendency to apply these techniques uniformly to all users, adopting a one-size-fits-all approach. To investigate the possibility that individuals may perceive engagement techniques differently, we created HarrySpotter, a location-based augmented reality (AR) app that enables users to annotate real-world objects using six distinct engagement techniques. By deploying HarrySpotter and analyzing data from 29 users, we found that agreeable users were not motivated by competition, while conscientious users were motivated by it to a greater extent. As expected, individuals open to new experiences and extroverts were motivated by exploration, while neurotics exhibited a stronger drive towards personal achievements. These preferences for specific engagement techniques also predicted personality traits to different extents (e.g., Extraversion with an Adj. $R^2$ of 0.61, while Conscientiousness with an Adj. $R^2$ of 0.16), suggesting that engagement strategies should be tailored to one's personality.

From a theoretical perspective, our work is situated within the domain of adaptive user interfaces (AUI). The effectiveness of an AUI hinges on the ability to construct and utilize individual user profiles, allowing for the delivery of personalized versions of the user interface~\cite{jameson2007adaptive,constantinides2015exploring,constantinides2018framework,constantinides2016user}. Building upon this foundation, we envision that our methodology could be employed to enhance user models with specific characteristics, such as personality traits. This, in turn, could facilitate the personalization of user interfaces in various contexts, such as advancing levels in a gamified app or completing tasks. From a practical standpoint, our findings can inform the design of personalized engagement strategies. To illustrate, let us consider the scenario of mobile crowdsourcing systems, where a one-size-fits-all approach has proven ineffective in engaging users for specific tasks, such as object annotation~\cite{rula2014no}. By incorporating brief personality questionnaires, for example during account setup (e.g., the TIPI~\cite{gosling2003very} questionnaire, which can be completed in a minute), mobile developers can implement in-app mechanisms to dynamically infer personality traits, thereby adapting engagement strategies based on users' interactions.

Our work has three limitations that warrant further research efforts. Firstly, our findings are specific to the HarrySpotter game and this particular cohort. Future studies could extend our methodology to different types of mobile apps. For example, developing tailored strategies for Conscientiousness could enhance prediction accuracy by incorporating logging mechanisms for organized individuals. Secondly, the slight skew in Extraversion may be due to self-selection bias, as introverted individuals are less likely to engage with such apps. Future studies should replicate our methodology with larger and culturally diverse populations. Lastly, while our two-week study provided ample data, longer deployments can explore user retention and preferences more comprehensively, thereby enhancing our understanding of personalized engagement strategies.





%
% ---- Bibliography ----
%
% BibTeX users should specify bibliography style 'splncs04'.
% References will then be sorted and formatted in the correct style.
%
\bibliographystyle{splncs04}
\bibliography{main}
%


\end{document}
