\section{HarrySpotter}
\label{sec:mobilegame}

% Figure environment removed

We developed a mobile app called HarrySpotter, which incorporates gameplay elements inspired by the popular Harry Potter series. Authored by J.K. Rowling, the Harry Potter series revolves around the adventures of a young wizard named Harry Potter, his friends, and their quest to defeat the dark wizard Lord Voldemort. Our app draws inspiration from the series' concept of affiliation through four houses, namely Gryffindor (known for courage and bravery), Hufflepuff (emphasizing hard work and patience), Ravenclaw (highlighting intelligence and learning), and Slytherin (representing ambitions and cunning). HarrySpotter was developed using the Unity game engine for both Android and iOS platforms. The app uses the Mapbox SDK for location-based features, and Vuforia SDK to deliver an augmented reality experience, particularly during the process of claiming a mayorship.

HarrySpotter employs six strategies to engage users in the task of annotating objects: Point Rewards, Places Rewards, Game with Yourself, Social Connection, Object Discovery, and Place Discovery. These strategies were initially derived from the work of Lindqvist et al.~\cite{lindqvist2011m}, but were modified to align with our gameplay's requirements. For example, the effectiveness of badges in Lindqvist's study was evaluated using the question statement: ``I pay attention to the badges that others earn.'' In our case, to assess the effectiveness of point rewards (which function as a type of badge), we adapted the statement to: ``I pay attention to others' spell energy scores'' (Table\ref{tab:questions}). Additionally, as our gameplay includes two types of rewards---points and place rewards---we categorized them separately, resulting in a total of six strategies, contrasting with the five strategies described in~\cite{lindqvist2011m}. \\ 

\noindent 
\textbf{Point and Place Rewards}: Previous research has demonstrated the effectiveness of reward systems, such as points, in engaging users with mobile apps~\cite{brauer2019badges}. In HarrySpotter, users are rewarded with spell energy for annotating new objects (Figure~\ref{fig:strategies}d), which reflects their ability to claim mayorships of places. When a user annotates an object, the app compares the user-generated label (object name) with the label automatically detected by an image classifier running on our server. The classifier used is a deep-learning ResNet-162 model with a top 5\% accuracy of 94.2\% on ImageNet classes. The semantic distance between the user-generated label and the automatically detected label is computed using WordNet~\cite{miller1998wordnet}. If they match, the user receives extra spell energy. Additionally, the app tracks previously scanned object types and rewards the user when they scan a new type for the first time. However, if the user scans the same object type repeatedly, the reward amount decreases until it reaches the minimum of 10 points. This design choice ensures that the user's score does not reach zero and maintains a balance between engagement and avoiding penalization, such as providing incorrect labels or repetitive images.

\noindent \textbf{Game With Yourself}: Users have the option to play the game alone, engaging in various single-player elements such as object annotation, places, and challenging mayorships (Figure~\ref{fig:strategies}b-d). When it comes to mayorships, a user can become the mayor of a place. Subsequently, other users have the opportunity to visit that place and challenge the current mayor. This feature adds a competitive aspect to the game, even when playing individually.

\noindent\textbf{Social Connection}: Previous research has demonstrated that leaderboards are effective in enhancing user performance in various tasks~\cite{brauer2019badges}. In HarrySpotter, during the onboarding process, users respond to a series of questions inspired by the Harry Potter sorting hat quiz (Figure~\ref{fig:strategies}a) and are sorted into one of the Harry Potter houses~\cite{jakob2019science}. Through the leaderboard, we encourage users to actively participate in the game and contribute to their respective houses' efforts in claiming mayorships of different places.

\noindent \textbf{Object Discovery}: Enabling users to explore and discover new places or objects is a crucial aspect of location-based apps. Previous research has demonstrated that incorporating points of interest (similar to Pok\'emon GO) encourages users to engage with the app while on the move and at various locations~\cite{althoff2016influence}. In HarrySpotter, we motivate users to explore different locations by allowing them to become mayors of real places (Figure~\ref{fig:strategies}c). When a user is within the mayorship location range, an AR mage appears and challenges the user for mayorship. The user's spell energy (points) plays a significant role in their chances of claiming the mayorship. To strike the right balance, we set the mayorship range to an 80-meter radius based on empirical evidence. Lower ranges limited accessibility, while higher ranges diminished proximity and overall engagement. Through this strategy, we encourage users to discover and scan objects they may have overlooked in new locations, fostering exploration and engagement.

\noindent \textbf{Place Discovery.} Previous research on the motivations behind using location-based apps has revealed that users are driven by their curiosity to obtain information about specific points of interest~\cite{lindqvist2011m}. This curiosity acts as an incentive for users to actively pursue becoming the mayor of those places. In HarrySpotter, when a user successfully claims mayorships of places, their map visually represents a sense of territorial ownership (Figure~\ref{fig:strategies}b). For example, if a user belongs to Hufflepuff house and becomes a mayor, their map pins will be displayed in yellow, symbolizing their affiliation with Hufflepuff.