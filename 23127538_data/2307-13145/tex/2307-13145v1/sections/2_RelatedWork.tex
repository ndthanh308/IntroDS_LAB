\section{Related Work}
\label{sec:related}
User engagement is a critical factor in the success of various digital experiences, including websites, mobile apps, and online platforms~\cite{lalmas2014measuring}. It refers to the level of involvement, interaction, and interest that users have with a product or service~\cite{o2008user}, resulting in increased engagement. Similarly, mobile user engagement is described by the level of engagement users 
have with mobile apps on their smartphones or tablets. Factors such as intuitive user interface design, personalized content delivery, and interactive features play a significant role in fostering mobile user engagement~\cite{sutcliffe2016designing,leiras2017mobile}. Push notifications, in-app messaging, and social sharing features also contribute to enhancing mobile user engagement~\cite{kim2018examining}.

Gamification techniques also play a crucial role in fostering engagement~\cite{deterding2012gamification,hakulinen2015effect}. Gamification involves applying game elements and mechanics to non-game contexts to enhance engagement, motivation, and participation~\cite{deterding2012gamification}. It taps into people's natural inclination for competition, achievement, and recognition, making it a powerful tool for motivating and incentivizing users~\cite{hamari2014does}. By incorporating game-like features such as points, badges, leaderboards, challenges, and rewards, mobile app developers can transform mundane tasks or activities into more enjoyable and immersive experiences. Rewards can take various forms, such as adding points or levels, to entice users to engage with an app to earn these rewards~\cite{nicholson2015recipe,lindqvist2011m}. Badges and leaderboards are also popular gamification elements, which were shown to boost motivation~\cite{brauer2019badges}.
Additionally, gamification strategies have been used to increase users' physical activity. For example, Althoff et al.~\cite{althoff2016influence} conducted a study on the impact of Pok\'emon GO~\cite{pokemongo}, an augmented reality (AR) location-based game, and found that the game led to a more than 25\% increase in users' physical activity.

While previous studies have explored the use of gamification strategies to engage mobile app users in a variety of tasks or games, the relationship between these strategies and a user's personality remains relatively unexplored. The purpose of this study is to investigate whether different individuals perceive the same gamification strategies differently.