\section{The Big-Five Personality}
\label{sec:hypotheses}
The Big-Five personality model assigns individuals scores~\cite{gosling2003very}, representing the main personality traits of Openness, Conscientiousness, Extroversion, Agreeableness, and Neuroticism. We hypothesized the relationship between these traits and our six engagement strategies, and summarized the positive and negative relationships of these hypotheses in Table~\ref{tab:questions}. \\

\noindent\textbf{\emph{Openness}} is associated with descriptive terms such as imaginative, spontaneous, and adventurous. Individuals high in Openness are more likely to try new methods of communication, including social networking sites or mobile apps. For example, studies have reported that individuals high in Openness tend to utilize a greater number of features that facilitate exploration in such technologies~\cite{ross2009personality}.

\noindent\textbf{\emph{Conscientiousness}} is associated with traits like ambition, resourcefulness, and persistence. Individuals high in Conscientiousness are less likely to engage in mobile content generation. They often view computer-mediated communication as a distraction from their daily tasks~\cite{amichai2010social}. However, when they do engage in such communication, they tend to approach it in a highly methodical and competitive manner. Their motivation is often driven by a desire for positive competition~\cite{halko2010personality}.

\noindent\textbf{\emph{Extraversion}} is associated with descriptive terms such as sociability, activity, and excitement seeking.  Individuals high in Extraversion typically prefer face-to-face interactions and are less inclined to utilize social networking sites or mobile apps. However, if they do join such platforms, they often participate in multiple groups, contribute content, and are motivated by the positive aspect of exploration as a means of social stimulation~\cite{phillips2006personality}.

\noindent\textbf{\emph{Agreeableness}} is associated with descriptive terms such as trusting, altruistic and tender-minded. Individuals high in Agreeableness, who are less competitive~\cite{halko2010personality} and less likely to share content~\cite{amichai2010social}, are more likely to be negatively motivated by rewards or competition.

\noindent\textbf{\emph{Neuroticism}} is associated with descriptive terms such as emotional liability and impulsiveness. Individuals with high levels of Neuroticism exhibit diverse behaviors across different media platforms. They tend to use the Internet and mobile apps as a means to alleviate loneliness, share accurate personal information in anonymous online forums (e.g., chat rooms), exercise control over their shared information on mobile devices~\cite{butt2008personality}, and focus on their own achievements in positive ways~\cite{lane2012influence}.

\begin{table*}[t]
\centering
\caption{Question statements assessing HarrySpotter's six engagement strategies. Positive and negative signs indicate the association of these engagement strategies with each personality trait as found in prior literature, and empty cells indicate that no reference has been found. O: Openness; C: Conscientiousness; E: Extraversion; A: Agreeableness; and N: Neuroticism. }
\resizebox{\textwidth}{!}
{\begin{tabular}{l l l l l l l} 
 \hline
 \textbf{Strategy} & \textbf{Question Statement} & \textbf{O} & \textbf{C} & \textbf{E} & \textbf{A} & \textbf{N} \\  \hline
 Q1(Point Rewards) & I pay attention to others' spell energy scores. &  &  &  & - & \\
 Q2(Place Rewards) & I am proud of my mayorships. &  &  &  & - & +\\ 
 Q3(Game with Yourself) & When I play the game, I feel I am representing my house. &  &  &  &  & + \\
 Q4(Social Connection) & With HarrySpotter, I track the competition among the four houses. &  & + &  & -  &\\
 Q5(Object Discovery) & HarrySpotter motivated me to discover new objects. & + &  & + &  &\\
 Q6(Place Discovery) & HarrySpotter motivated me to visit new places.  & + &  & + &  &\\
    \hline
\end{tabular}
}
\label{tab:questions}
\end{table*}