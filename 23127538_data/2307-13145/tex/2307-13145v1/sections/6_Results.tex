\section{Results}
\label{sec:results}
To ease the interpretation of our results, we applied a min-max transformation to scale our variables within the range of [0-100]. We first examined the pairwise correlation among personality traits, the six self-reports, and the quantity and quality of annotations metrics. We found that neurotic users discovered fewer objects ($\textrm{r=-0.37, p\textless0.1}$). Additionally, these users tended to take pride in their mayorships ($\textrm{r=0.44, p\textless0.05}$) and their affiliation with their respective Harry Potter house ($\textrm{r=0.47, p\textless0.05}$). Conversely, Extroverts and those high in Openness liked to discover new objects ($\textrm{r=0.44, p\textless0.05}$ and $\textrm{r=0.39, p\textless0.05}$, respectively) and new places ($\textrm{r=0.38, p\textless0.05}$ and $\textrm{r=0.35, p\textless0.1}$, respectively). 

\begin{table*}[ht!]
\centering
\caption{Linear regressions that predict the Big-Five personality traits from the six self-reports and the quantity and quality of annotations. Significant predictors with $p$ values $<$ .05 are marked in bold. The most predictable personality trait was Extraversion ($M_{E}$), while the least predictable was Conscientiousness ($M_{C}$).}
   % \resizebox{\textwidth}{!}{%
      \begin{tabular}{lccc}
        \hline
        $\boldsymbol{M_{O}}$: $\boldsymbol{\emph{Adj $R^2$} = 0.28}$, Durbin-Watson = 1.96, AIC = 0.47 & & &  \\
        \hline
        Predictor & $\beta$ & std. error & $p$-value \\
        \hline
        Intercept & 0.42 & 0.13 & 0.005\\
        Q1(Point Rewards) & -1.08 & 0.36 & \textbf{0.01}\\ 
        Q2(Place Rewards) & 0.42 & 0.35 & 0.23\\ 
        Q5(Object Discover) & 0.84 & 0.27 & \textbf{0.01}\\ 
        \hline
        $\boldsymbol{M_{C}}$: $\boldsymbol{\emph{Adj $R^2$} = 0.16}$, Durbin-Watson = 1.15, AIC =  -7.48 & & &  \\
        \hline
        Predictor & $\beta$ & std. error & $p$-value \\
        \hline
        Intercept & 0.57 & 0.12 & 0.00\\
        Q3(Game with Yourself) & 0.59 & 0.22 & \textbf{0.01}\\ 
        Q6(Place Discovery) & -0.48 & 0.23 & \textbf{0.04}\\
        \hline
        $\boldsymbol{M_{E}}$: $\boldsymbol{\emph{Adj $R^2$} = 0.61}$, Durbin-Watson = 1.88, AIC = -14 & & &  \\
        \hline
        Predictor & $\beta$ & std. error & $p$-value \\
        \hline
        Intercept & 0.48 & 0.19 & 0.02\\
        Q2(Place Rewards) & -0.60 & 0.22 & \textbf{0.01}\\ 
        Q4(Social Connection) & -0.72 & 0.22 & \textbf{0.00}\\ 
        Q5(Object Discovery) & 0.53 & 0.22 & \textbf{0.02}\\ 
        Q6(Place Discovery) & 0.99 & 0.28 & \textbf{0.00}\\
        Quantity & -0.37 & 0.12 & \textbf{0.00}\\
        Quality & 0.18 & 0.16 & 0.27\\
        \hline
        $\boldsymbol{M_{A}}$: $\boldsymbol{\emph{Adj $R^2$} = 0.21}$, Durbin-Watson = 1.84, AIC = -12.86 & & &  \\
        \hline
        Predictor & $\beta$ & std. error & $p$-value \\
        \hline
        Intercept & 0.57 & 0.11 & 0.00\\
        Q1(Point Rewards) & -0.45 & 0.24 & 0.07\\  
        Q3(Game with Yourself) & 0.21 & 0.21 & 0.31\\
        Q5(Object Discovery) & -0.37 & 0.24 & 0.13\\
        Q6(Place Discovery) & 0.71 & 0.29 & \textbf{0.02}\\  
        \hline
        $\boldsymbol{M_{N}}$: $\boldsymbol{\emph{Adj $R^2$} = 0.32}$, Durbin-Watson = 2.79, AIC = -3.68 & & &  \\
        \hline
        Predictor & $\beta$ & std. error & $p$-value \\
        \hline
        Intercept & 0.36 & 0.14 & 0.02\\
        Q2(Place Rewards) & 0.37 & 0.26 & 0.17\\  
        Q4(Social Connection) & 0.58 & 0.27 & \textbf{0.04}\\
        Q6(Place Discovery) & -0.58 & 0.29 & 0.06\\
        Quantity & -0.32 & 0.14 & \textbf{0.03} \\
        \hline
      \end{tabular}
      % }
      \label{tab:regression}
\end{table*}

Considering these significant correlations, one might wonder whether it is possible to predict users' personality traits based on their self-reports and the quantity and quality annotation metrics. Using these metrics as predictors, we fitted five linear regression models (Table~\ref{tab:regression}) to predict the Big-Five personality dimensions and determined the best set of predictors using the stepAIC function~\cite{zhang2016variable}. Overall, our findings indicate that predicting certain personality dimensions, such as Extraversion (Adj. $R^2$ = 0.61), was relatively easier compared to others like Conscientiousness (Adj. $R^2$ = 0.16).

As expected, users with non-competitive traits (high in Agreeableness) demonstrated a lack of motivation for competing with others ($\beta_\textrm{(Q1(Point Rewards))}\textrm{=-0.45}$). On the other hand, conscientious users, known for their organizational skills, exhibited motivation for competition ($\beta_\textrm{(Q1(Point Rewards))}\textrm{=0.59}$) among the four houses but moderately predicted their personality trait (Adj. $R^2$\textrm{=0.16}). Individuals open to new experiences and extroverts were primarily motivated by the discovery of new objects ($\beta_\textrm{(Q5(Object Discovery))}\textrm{=0.84}$ and $\beta_\textrm{(Q5(Object Discovery))}\textrm{=0.53}$, respectively). Moreover, individuals open to new experiences did not find motivation in competing with others ($\beta_\textrm{(Q1)}\textrm{=-1.08}$), while extroverts were not motivated by mayorships ($\beta_\textrm{(Q2(Place Rewards))}\textrm{=-0.60}$) or representing their own house ($\beta_\textrm{(Q1(Point Rewards))}\textrm{=-0.72}$). Emotionally unstable users (neurotics) found motivation in representing their own house ($\beta_\textrm{(Q4(Social Connection))}\textrm{=0.58}$) but not in the discovery of new places ($\beta_\textrm{(Q6(Place Discovery))}\textrm{=-0.58}$), leading them to discover fewer objects ($\beta_\textrm{(Quantity)}\textrm{=-0.32}$).