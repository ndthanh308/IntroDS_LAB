\section{Discussion and Conclusion}
\label{sec:discussion}
Mobile user engagement is commonly pursued through a range of techniques; however, there is a tendency to apply these techniques uniformly to all users, adopting a one-size-fits-all approach. To investigate the possibility that individuals may perceive engagement techniques differently, we created HarrySpotter, a location-based augmented reality (AR) app that enables users to annotate real-world objects using six distinct engagement techniques. By deploying HarrySpotter and analyzing data from 29 users, we found that agreeable users were not motivated by competition, while conscientious users were motivated by it to a greater extent. As expected, individuals open to new experiences and extroverts were motivated by exploration, while neurotics exhibited a stronger drive towards personal achievements. These preferences for specific engagement techniques also predicted personality traits to different extents (e.g., Extraversion with an Adj. $R^2$ of 0.61, while Conscientiousness with an Adj. $R^2$ of 0.16), suggesting that engagement strategies should be tailored to one's personality.

From a theoretical perspective, our work is situated within the domain of adaptive user interfaces (AUI). The effectiveness of an AUI hinges on the ability to construct and utilize individual user profiles, allowing for the delivery of personalized versions of the user interface~\cite{jameson2007adaptive,constantinides2015exploring,constantinides2018framework,constantinides2016user}. Building upon this foundation, we envision that our methodology could be employed to enhance user models with specific characteristics, such as personality traits. This, in turn, could facilitate the personalization of user interfaces in various contexts, such as advancing levels in a gamified app or completing tasks. From a practical standpoint, our findings can inform the design of personalized engagement strategies. To illustrate, let us consider the scenario of mobile crowdsourcing systems, where a one-size-fits-all approach has proven ineffective in engaging users for specific tasks, such as object annotation~\cite{rula2014no}. By incorporating brief personality questionnaires, for example during account setup (e.g., the TIPI~\cite{gosling2003very} questionnaire, which can be completed in a minute), mobile developers can implement in-app mechanisms to dynamically infer personality traits, thereby adapting engagement strategies based on users' interactions.

Our work has three limitations that warrant further research efforts. Firstly, our findings are specific to the HarrySpotter game and this particular cohort. Future studies could extend our methodology to different types of mobile apps. For example, developing tailored strategies for Conscientiousness could enhance prediction accuracy by incorporating logging mechanisms for organized individuals. Secondly, the slight skew in Extraversion may be due to self-selection bias, as introverted individuals are less likely to engage with such apps. Future studies should replicate our methodology with larger and culturally diverse populations. Lastly, while our two-week study provided ample data, longer deployments can explore user retention and preferences more comprehensively, thereby enhancing our understanding of personalized engagement strategies.