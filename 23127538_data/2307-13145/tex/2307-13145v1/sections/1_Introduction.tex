\section{Introduction} 
\label{sec:introduction}
User engagement is crucial for the success of mobile apps, especially in modern Internet companies~\cite{lalmas2014measuring}. Mobile apps employ various techniques to capture users' attention and increase their engagement. For example, Foursquare introduced game mechanics to enhance engagement~\cite{lindqvist2011m}. Users could check-in at venues and inform their friends about their location. However, since not all friends may use the app, incentivizing early adopters became vital for the app's success. To motivate early adopters, Foursquare introduced badges, appealing to their desire for status. By default, the app shared this activity on social media platforms like Twitter, creating a sense of accomplishment and effectively engaging users. In general, engagement strategies, such as badges and rewards~\cite{brauer2019badges,van2019collecting}), encompass various mechanisms to increase user engagement~\cite{deterding2012gamification}. However, most current mobile apps follow a one-size-fits-all approach~\cite{adexchanger,rula2014no}, where all users are exposed to the same engagement techniques.

While users' personality has been extensively studied in various domains and linked to diverse aspects including online browsing behavior~\cite{bachrach2012personality,kosinski2014manifestations,quercia2011our} and patterns of behavior collected with smartphones~\cite{stachl2020predicting}, limited research has investigated its influence on mobile user engagement. To explore the relationship between personality and engagement strategies, we developed and deployed a location-based Augmented Reality (AR) mobile app called HarrySpotter, which incorporates six engagement techniques.

In this study, we made three sets of contributions. First, we developed HarrySpotter, a location-based app that enables users to annotate real-world objects. We conducted a two-week in-the-wild study involving 29 participants, resulting in a collection of 503 annotated objects. Second, we analyzed the engagement techniques chosen by users when capturing these objects and examined their correlation with their Big-Five personality traits. Our findings revealed that competition-based techniques discouraged agreeable users but encouraged conscientious users. Techniques promoting exploration of objects and places were particularly appealing to extroverts and individuals open to new experiences. Additionally, techniques focusing on personal achievements were found to motivate neurotic users. Lastly, we found that these preferences for specific engagement techniques not only correlated with personality traits but also had predictive value. The Adj. $R^2$ values ranged from 0.16 for conscientious users to 0.32 for neurotic users and as high as 0.61 for extroverts.