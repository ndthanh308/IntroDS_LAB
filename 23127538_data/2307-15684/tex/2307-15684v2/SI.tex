\section*{Supplementary information}

\section{Pseudopotentials}
The pseudopotentials for DFT calculations are tested against all-electron calculations, performed with the same functional. Results are shown in Fig.~\ref{fig:pp_quality} where two different functionals (PBE\cite{Perdew1996} on the left and BLYP\cite{Becke_1988,Lee_1988} on the right) have been analysed. In both cases, the bare Coulomb potential for hydrogen and the ccECP\cite{Bennett_2017} for sulfur (green points) give the best agreement with all-electron results (black squares).

% Figure environment removed

\section{Impact of the fixed node approximation in diffusion Monte Carlo calculations}

As reported in the Methods section of the main paper, the Slater orbitals of the Jastrow-Slater quantum Monte Carlo wave function, defining the nodal surface, are generated by the LDA functional\cite{Kohn1965}. As we do not relax the nodal surface at the variational Monte Carlo (VMC) level, the nodes we use in the diffusion Monte Carlo (DMC) are those directly provided by the LDA Khon-Sham orbitals. 

Here, we would like to assess the quality of the LDA nodes for the H$_3$S Hamiltonian. To this aim,
we carried out selected QMC calculations, where we first relaxed the LDA nodes at the VMC level in the 2$\times$2$\times$2 cubic supercell at the Baldereschi k-point, then we performed DMC calculations with the optimized nodal surface to see the impact of the nodal surface optimization, by comparing these results  with those obtained with the frozen LDA nodes calculations. We checked the quality of the LDA nodes for both the barrier height at the volume of 110 a$_0^3$, and the equation of state (EOS) in the symmetric Im$\bar{3}$m configuration. 

As for the barrier height, we report in Tab.~\ref{tab:relaxed_det} 
the energy differences between the symmetric configuration, i.e. with the proton in the middle of the S-S segment, and a displaced one with the proton sitting at the position shifted by 0.06 $d_\textrm{SS}$ from the S-S center, quite close to the actual minimum of the double well potential at that volume.

\begin{table}[h!]
\centering
\begin{tabular}{|c|c|c|}
\hline
\textbf{Wave function} & \textbf{VMC} & \textbf{DMC} \\ 
\hline
\hline
\textbf{LDA nodes}    &  0.228(6)  &  0.261(10) \\ \hline
\textbf{relaxed determinant}  & 0.243(5) & 0.256(10) \\ \hline
\textbf{Z-relaxed determinant} & 0.245(6) & 0.262(8) \\ \hline \hline
\end{tabular}
\caption{
DMC energy differences in meV/H$_3$S between the symmetric proton configuration and the one displaced by 0.6 $d_\textrm{SS}$, with  $d_\textrm{SS} = 6.0368 a_0$, corresponding to a volume per H$_3$S of 110 a$_0^3$. The calculations are performed in a 2$\times$2$\times$2 cubic supercell at the Balderschi k-point, determined at the DFT-PBE level. The DMC nodes are either the LDA ones (first row), or the ones obtained by optimizing the linear coefficients of the one-body Slater orbitals at the VMC level (second row), or the ones yielded by the VMC optimization of both linear coefficients and Gaussian exponents (third row). The optimization is done by VMC energy minimization.
  }
\label{tab:relaxed_det}
\end{table}

As one can see from Tab.~\ref{tab:relaxed_det}, the energy differences, giving an estimate of the barrier height at the DMC level with various nodal surfaces, are all compatible within the statistical error bar of 10 meV/H$_3$S. This represents a relative error smaller than 4\%, given the value of the barrier. Therefore, the fixed node approximation based on the LDA nodes, used throughout the paper, is fully satisfactory, at least for estimating the barriers of the potential energy surface (PES) at a fixed volume and with an accuracy as high as 8 meV/H$_3$S.

Let us study now the impact of the fixed node approximation with LDA nodes to the equation of state (EOS), as determined in the main paper and shown in Fig.~9. As for the barrier height, we carried out a full determinant relaxation, i.e. the optimization of both linear coefficients and the Gaussian exponents of the Slater one-body orbitals, for several volumes. This is the most severe test, because errors are more hardly compensated across different volumes. The optimizations, i.e. VMC energy minimizations, are always performed at the Baldereschi k-point and in the 2$\times$2$\times$2 cubic supercell. Once the nodes optimized, we carried out DMC simulations for different volumes, and then fitted our DMC total energies with a Vinet EOS. The resulting pressure calibration relation, $\Delta p=\Delta p(V)$, is plotted in Fig.~\ref{fig:pofV_poly_diff}, by using the $p=p(V)$ yielded by DMC energies with LDA nodes as reference.

% Figure environment removed

Fig.~\ref{fig:pofV_poly_diff} shows a pressure calibration that, for the best correction, namely for the ``Z-relaxed determinant nodes'', does not exceed a 3.5 GPa difference with respect to the original LDA-nodes EOS in a wide volume range, going from 85 a$_0^3$ to 120 a$_0^3$, thus covering all the most interesting regions of the phase diagram reported in Figs.~8 and 9 of the main paper. Therefore, for the EOS as well as for the PES barriers, the bias due to the DMC fixed node approximation with LDA nodes is negligible. The LDA nodes are accurate enough for high-pressure hydrogen-based systems, as also verified in another system, the high-pressure pristine hydrogen in Ref.~\cite{monacelli2023quantum}.


\section{Convergence of PIMD simulations with respect to the number of beads}

In this Section, we present the convergence analysis of the PIMD simulations with respect to the number of beads, both for the \emph{ab initio} and the 3D model simulations using the DFT-BLYP parametrization of the potential energy surface. In Fig. \ref{fig:convergence_abinitio}, we illustrate the convergence of kinetic energy using the virial estimator \cite{Mouhat_2017} from \emph{ab initio} simulations for the volume $V = 110$ a$_0^3$. Notably, with 20 beads, the kinetic energy converges to its asymptotic value (indicated by the red line).
% Figure environment removed

In Fig.~\ref{fig:convergence_model}, we show instead the behaviour of the kinetic energy (top panel of \ref{fig:convergence_model}) and the shuttling mode (bottom panel of \ref{fig:convergence_model}) using the 3D model with the DFT-BLYP parametrization of the PES for V=110 a$_0^3$.
Based on this analysis, we chose
40 beads for our 3D-model PIMD simulations.

% Figure environment removed

\newpage

\section{SSCHA simulations}

In this Section, we present the estimation of the ferroelectric phase transition within the SSCHA description of quantum nuclei for the 3D model. In Fig.~\ref{fig:susceptibility_SSCHA_SI}, we illustrate the volume dependence of the centroid positions with the DFT-BLYP and QMC 3D-PES for both $\ce{H_3S}$ and $\ce{D_3S}$. The transition occurs at the volume where the SSCHA centroid position for hydrogen (deuterium) atoms leaves the S-S midpoint. A direct assessment of the isotope effect within the SSCHA framework is obtained by comparing the left and right panels of Fig.~\ref{fig:susceptibility_SSCHA_SI}.

% Figure environment removed


\section{Classical \emph{ab initio} BLYP simulations at 200K}
\label{SI:classic}

% Figure environment removed

The ferroelectric transition can be detected by examining the displacements of all $N$ hydrogen atoms with respect to their relative S-S midpoint at a given configurations, and then averaging over a statistical sample of $N$-atom configurations. The corresponding order parameter is defined as $\Delta = \langle \sum_{i=1}^N \delta x_i \rangle$, where the brackets indicate the average over the classical Boltzmann distribution, generated by Langevin molecular dynamics. An analogous order parameter is $\Delta_\textrm{abs} = \langle \lvert \sum_{i=1}^N \delta x_i \rvert \rangle$. In Fig.~\ref{fig:chi_abinitio_classic}, we present the volume dependence of these order parameters, averaged over classical \emph{ab initio} molecular dynamics (MD) trajectories at 200K. We observe a strong correlation between the volume at which hydrogen atoms deviate from the S-S midpoint, the peak of the variance of the order parameters, and the jump in the shuttling mode frequencies.
The remarkable consistency between the values for the critical pressure obtained through different probes underscores the reliability of our results.
