\documentclass[prx,reprint,amsmath,amssymb,aps,floatfix,dvipsnames,longbibliography]{revtex4-2}

\usepackage[normalem]{ulem}
\usepackage{cancel}
\usepackage{subfiles}
\usepackage{natbib}
\usepackage{graphicx}% Include figure files
\usepackage{dcolumn}% Align table columns on decimal point
\usepackage{bm}% bold math
%\usepackage{hyperref}% add hypertext capabilities
\usepackage{xcolor}
%\usepackage[dvipsnames,usenames]{xcolor}

\newcommand{\change}[1]{\textcolor{black}{\textrm{#1}}}
\newcommand{\michele}[1]{\textcolor{red}{\textrm{#1}}}
\newcommand{\marco}[1]{\textcolor{teal}{\textrm{#1}}}
\newcommand{\tommaso}[1]{\textcolor{brown}{\textrm{#1}}}
\newcommand{\romain}[1]{\textcolor{dandelion}{\textrm{#1}}}
%\newcommand{\sout}[1]{\sout{\textrm{#1}}}
\usepackage[version=4]{mhchem}


\begin{document}

\title{Quantum symmetrization transition in superconducting sulfur hydride from quantum Monte Carlo and path integral molecular dynamics}

\author{Romain Taureau\textsuperscript{1}}
\author{Marco Cherubini\textsuperscript{1}}
\author{Tommaso Morres\textsuperscript{2}}
\author{Michele Casula\textsuperscript{1}}
\email{michele.casula@sorbonne-universite.fr}
\affiliation{
\textsuperscript{1}Sorbonne Universit\'e, Institut de min\'eralogie, de physique des mat\'eriaux et de cosmochimie (IMPMC),  CNRS UMR 7590, MNHM, 4 Place Jussieu, 75005 Paris.\\
\textsuperscript{2}European Center for Theoretical Studies in Nuclear
Physics and Related Areas, Fondazione Bruno Kessler, Strada delle
Tabarelle 286, Trento, 38123, Italy.
}

\date{\today}

\begin{abstract}
  We study the structural phase transition,  originally associated
  with the highest superconducting critical temperature $T_c$ measured
  in high-pressure sulfur hydride. A quantitative description of its
  pressure dependence has been elusive for any \emph{ab initio} theory
  attempted so far, raising questions on the actual mechanism leading to the maximum of $T_c$.
Here, we estimate the critical pressure of the hydrogen bond symmetrization in the Im$\bar{3}$m structure, 
by combining density functional theory and quantum Monte Carlo simulations for
electrons with path integral molecular dynamics for quantum nuclei.
We find that the $T_c$ maximum corresponds to pressures where local dipole moments dynamically form
  on the hydrogen sites, as precursors of the ferroelectric
  Im$\bar{3}$m-R3m transition, happening at lower pressures.
For comparison, we also apply the self-consistent harmonic approximation,
whose ferroelectric critical pressure lies in between the ferroelectric transition estimated by path integral
  molecular dynamics and the local dipole formation. Nuclear quantum effects 
play a major role in a significant reduction ($\approx$ 50 GPa) of the classical
ferroelectric transition pressure at 200K and in a
large isotope shift ($\approx$ 25 GPa) upon hydrogen-to-deuterium
substitution of the local dipole formation pressure, in agreement
with the corresponding change in the $T_c$ maximum location.
\end{abstract}

\maketitle

\section{Introduction}\label{sec1}

Since its discovery in 1911 \cite{Onnes_supra}, superconductivity has been 
one of the most investigated topics  
in both theoretical and experimental physics.
While it was discovered that almost every conductor could reach 
zero resistance
at low-enough temperatures (T $<$ 10 K) \cite{Tresca2022}, the quest
for higher critical temperature ($T_c$) superconductors became the new
challenge. Until recently, cuprates were leading the race with a $T_c$
as large as 133 K for Hg-Ba-Ca-Cu-O systems \cite{Schilling_1993},
although the pairing mechanism in these materials is considered
unconventional and it is not explained by the standard
Bardeen–Cooper–Schrieffer (BCS) theory \cite{Bardeen_1957}.

In 2015, the discovery of conventional superconductivity in $\ce{H_3S}$ with a maximum $T_c$ of 203 K reached at a pressure $P_c$ as high as 150 GPa \cite{Drozdov_2015} paved the way to a new era of high-$T_c$ materials. Indeed, hydrogen (H) -based systems are nowadays the most promising candidates to achieve room-temperature superconductivity.
As a matter of fact, in 2019, the same team that discovered $\ce{H_3S}$ claimed to have measured an even higher $T_c$ in $\ce{LaH_{10}}$, 
superconducting
already at 250 K \cite{Drozdov_2019}, later followed by a similar
discovery in the yttrium hydride \cite{Kong2021}. In a rush towards
room-temperature superconductivity, more recent claims of $T_c$ larger
than the one found in $\ce{LaH_{10}}$ did not meet the consensus of
the whole community
\cite{Service_2020,Boeri_2023}.
The main issue of these materials is the extreme pressure conditions, 
usually
larger than 150 GPa, needed to obtain the high-$T_c$ superconducting phase. 
Indeed,
while all the binary candidates involving hydrogen were theoretically investigated, none of them seems to sufficiently decrease the pressure of the superconducting state. Eyes are now turned towards ternary materials \cite{DiCataldo2022}.

In this work, we focus on the prototypical case of $\ce{H_3S}$ and we study its structural phase transition generally associated with the maximum of 
the  
superconducting critical temperature, located at around 150 GPa \cite{Einaga_2016,mozaffari_2019,Minkov_2020,Osmond_2022}. 
According to x-ray diffraction data \cite{goncharov_2017},   
at lower pressures the sulfur (S) sites are arranged in a geometry that is compatible with
the trigonal R3m 
symmetry
(Fig.~\ref{fig:geometry}(b)) and, upon compression, 
the system
undergoes a phase transition towards a body-centered-cubic (bcc) Im$\bar{3}$m 
structure
(Fig.~\ref{fig:geometry}(a)).

% Figure environment removed

After the first theoretical prediction of high-$T_c$ superconductivity in H$_3$S \cite{duan2014pressure}, several works
tried to explain 
the origin of the maximum of $T_c$ found in experiments as a function of pressure. 
Even if
the magnitude of the calculated $T_c$ is right, confirming the BCS origin of the superconducting state, a quantitative disagreement between various theoretical approaches was found, with estimated $T_c$ values fluctuating over a 50 K range for the high-pressure phase \cite{Sano_2016}. Moreover, theoretical studies more oriented to understand the underlying structural properties of H$_3$S, revealed a significant disagreement in the transition pressures between the predicted phases. In those works \cite{Errea_2016,Bianco_2018}, the structural phase transition is explained by a quantum proton symmetrization from the R3m phase, with displaced protons, to the Im$\bar{3}$m one, where every hydrogen 
lies in the midpoint of the two neighboring sulfur 
atoms (S-S midpoint).
\change{This is also called ferroelectric transition, because
  the hydrogen atoms displaced from the S-S midpoint
  lead in the R3m phase to a long-range order of local
  dipole moments, created by the H-S bond asymmetry.}
In 
that
context, the shuttling mode of 
hydrogen atoms, namely their 
vibrational mode along the direction linking 
two neighboring sulfur atoms, 
was
thoroughly investigated. 
The phase transition 
was then
identified by looking at the dynamical instability of the symmetric Im$\bar{3}$m phase when the pressure is lowered and the shuttling mode softens.
On general grounds,
this
reflects the sudden transformation of the free energy profile, leading to a sign change of its curvature across the transition between two different crystal structures, one with lower symmetry than the other.


These findings were obtained by solving the nuclear Hamiltonian within the Stochastic Self Consistent Harmonic Approximation (SSCHA) \cite{Monacelli2021,Errea_2013,Bianco_2017}, which 
has proven
to be one of the best approximated theories to deal with nuclear quantum effects (NQE). Within this framework, 
the electronic part 
was
solved 
by
Density Functional Theory (DFT) using different parametrizations for the exchange-correlation functional, like the Perdew-Burke-Ernzerhof (PBE) \cite{Perdew1996} and the Becke-Lee-Yang-Parr (BLYP) \cite{Becke_1988,Lee_1988} ones. Independently of the DFT functional used, 
a sizable underestimation of the experimental critical pressure $P_c$ by $\approx$ 40 GPa 
was always
observed, leaving open the question about the origin of 
this
mismatch, and whether this should be attributed to the electronic or to the nuclear 
components.


Here, we go beyond the previous state-of-the-art calculations by treating the electronic 
problem
not only at the DFT-BLYP, but also at Quantum Monte Carlo (QMC) level,
which provides a
benchmark for the 
DFT methods.
QMC is known to 
yield
very accurate total energies in both molecules and solids
\cite{lester_2009,saritas2017investigation,raghav2023toward}, thanks
to its stochastic Green's function algorithms
\cite{Foulkes2001,Wagner2016}, such as the lattice regularized
diffusion Monte Carlo \cite{Casula_lrdmc}, projecting any initial
trial wavefunction towards the ground state of the system within the
fixed node approximation. Moreover, we solve the nuclear Hamiltonian
by using Path Integral Molecular Dynamics (PIMD), which is in
principle exact, outperforming any other approximation for the nuclear
degrees of freedom. Then,
\change{we analyze the resulting phase diagram}
by looking \change{at the
  ferroelectric order parameter,} 
  at the hydrogen/deuterium 
density,
focusing on its 
transformation 
from the unimodal to bimodal distribution, and \change{finally} at its quantum fluctuations, 
detecting
when 
\change{the associated local polarization}
freezes
in a displaced geometry.


In this work, 
we 
have been
able to track the evolution of the mode distribution 
with a high resolution in volume (and pressure),
thanks to a three-dimensional (3D) model of the shuttling mode.
The reliability of 
our
model
has been
benchmarked using \emph{ab initio} PIMD simulations 
with BLYP electrons,
across
the 
\change{local moment formation}. 
The advantage of the 3D model is that its potential energy surface (PES) can still be derived by much more
expensive,
although more 
accurate,
QMC calculations, allowing us to check the impact of the electronic
description on the
\change{occurrence of a local polarization}.


In the model we developed, all hydrogen atoms in the system are 
allowed
to move in the same way.
\change{However, only the spatial degrees of freedom of a single H site are retained.}
This feature induces some limitations,
such as the lack of spatially disordered H configurations, \change{and
  of correlations beyond a single-site description}. 
In spite of this,
we can 
accurately
describe 
the \change{local} path from the symmetric
\change{proton arrangement}
to the 
asymmetric
\change{one, by detecting the local 
moment
  formation in the system, related to the shuttling mode softening.}
We have finally
performed both 
SSCHA and PIMD simulations of the 3D model to 
investigate
how NQE treated at different levels of approximation 
affect the final outcome.


\section{Results}\label{sec2}

\subsection{Harmonic and anharmonic phonons}

At high pressure, above 150 GPa, the $\ce{H_3S}$ crystal is expected to be in the cubic Im$\bar{3}$m symmetric phase (Fig.~\ref{fig:geometry}(a)), where every hydrogen atom 
sits on
the midpoint of 
two neighboring sulfur atoms. Upon 
pressure release,
the lattice 
undergoes
a trigonal distortion and the hydrogen atoms 
leave
the aforementioned midpoint to 
move
closer to one of the two flanking sulfur atoms, leading to the R3m asymmetric phase, depicted in Fig.~\ref{fig:geometry}(b). In our 
description,
we introduce a 
simplification by neglecting the trigonal distortion, which is however very weak ($<$0.6°) \cite{Bianco_2018}. Thus, the R3m phase considered here 
differs from
the Im$\bar{3}$m one just by the 
hydrogen
positions.

% Figure environment removed

In Fig.~\ref{fig:phonons}, we report the analysis of the phonon dispersion for different volumes of the cubic Im$\bar{3}$m unit cell, obtained at the \emph{ab initio} level using the BLYP functional, either through the harmonic approximation via Density Functional Perturbation Theory (DFPT), or with the inclusion of quantum anharmonicity via PIMD simulations. Hereafter, volumes and energies will be expressed per H$_3$S unit, while the unit cell will be taken as cubic with S atoms 
arranged
in a bcc lattice.


At this point, it is important to underline that the DFPT and the PIMD phonons 
bear different information (see also Sec.~\ref{Sec:Phonons}).
The PIMD phonons are computed through the quantum displacement-displacement correlator recently developed in Ref.~\cite{Morresi_2021}. They describe the lowest vibrational excitations \cite{Morresi2022}, that is the energy difference between the first excited state and the ground state of the nuclear Hamiltonian.
This is the quantity normally measured by experimental probes, such as infrared or Raman spectroscopies.
Consequently, 
phonons computed in this way fully include anharmonic effects and are always positive definite, meaning that they cannot describe dynamical instabilities via the appearance of imaginary phonons.
This is at variance with
the harmonic case or with
approximated theories devised to deal with NQE, 
such as the SSCHA \cite{Monacelli2021}, which instead provide information about the sign of the free energy curvature at the reference geometry.

While for V=83.6 a$^3_0$ 
only 
the harmonic dispersion 
is reported in Fig.~\ref{fig:phonons}, 
for 
larger volumes we compare the PIMD phonons (green lines) obtained in a 2x2x2 supercell with the harmonic ones (light-blue lines). %\t
We notice that for PIMD phonons, the spatial range of the force constant matrix is such that the 2x2x2 supercell is large enough to allow for a $\mathbf{q}$-interpolation of the phonon branches $\omega_m=\omega_m(\mathbf{q})$. The comparison between PIMD and harmonic phonons of Fig.~\ref{fig:phonons} clearly shows how strong NQE are and how sizable is the softening of the most energetic phonons due to quantum anharmonicity, particularly at the largest volumes.
In the harmonic framework, for V$>$85 a$_0^3$ (see Figs.~\ref{fig:phonons} and \ref{fig:transition_shuttle_phon}), 
% Figure environment removed
the appearance of imaginary frequencies indicates the dynamical instability of the Im$\bar{3}$m 
structure.
More specifically, the softening of the shuttling hydrogen mode at $\mathbf{q}=\Gamma$ signals the transition towards the 
asymmetric R3m phase \cite{Errea_2016}.
From Fig.~\ref{fig:phonons}, one can see that
imaginary frequencies disappear in PIMD phonons and their evolution 
as a function of volume is much smoother than in the harmonic case.  
As expected
from the definition of PIMD phonons,
PIMD simulations
never yield 
imaginary frequencies for the shuttling mode. 
In this regard, see also Fig.~\ref{fig:transition_shuttle_phon}, where we report the shuttling mode frequency we obtained
as a function of volume at different levels of theory. This analysis
\change{shows}
that 
\change{the putative transition implied by the maximum in T$_c$}
cannot be 
determined
using solely the shuttling mode frequency as a proxy,
\change{because in the volume range corresponding to the experimental
  T$_c$ maximum\cite{Drozdov_2015},
  i.e. between 98 $a_0^3$ and 100 $a_0^3$,\footnote{See the equations of
    state reported in Fig.~\ref{fig:drozdov_mod}(a) for the volume estimate,
    corresponding to the experimental pressure of $\approx$ 150 GPa.} there is no
  anomalous behavior of the proton shuttling mode frequency.}
We need to rely upon other observables in a framework describing nuclei as quantum particles.


So far, we have reported the structural behavior as a function of volume, fixed in our simulations. However, we can easily deduce the corresponding pressure by deriving the 
equation of state (EOS) $P=P(V)$ using the Vinet 
relation \cite{Vinet_EOS} 
computed with the same functionals 
employed
to calculate the phonon dispersions. Nevertheless, our goal is to go beyond DFT and reach a more accurate electronic description of the system using QMC methods (the details of our QMC calculations are reported in Sec.~\ref{Sec:elstructurePES}). A simple comparison of the EOS produced by the two approaches, shown in Fig.~\ref{fig:drozdov_mod}(a), reveals visible differences, suggesting that a description of the electronic structure at the QMC level is crucial to estimate correctly the critical pressure. Unfortunately, QMC calculations are much more expensive than DFT, and 
coupling them
with \emph{ab initio} PIMD simulations to study 
the real crystalline system is out of reach. Therefore, we need a simplified 
PES
describing the hydrogen shuttling mode that can be derived, after a fitting procedure, from QMC total energy calculations performed on a coarse grid of nuclear configurations. This model PES can then be used to compute the %phonons 
shuttling mode frequencies 
and to study the
\change{local polarization properties induced by the proton displacement}
at the PIMD level. 

\subsection{Classical 3D model}


The model PES is derived by considering the collective and coherent motion of all the hydrogen atoms along the direction connecting the two 
S atoms flanking each H (S-S direction), by allowing also 
hydrogen
out-of-axis mobility, while the S atoms are pinned in their bcc
positions. In this way, we aim at reproducing the shuttling mode
dynamics that takes place at $\mathbf{q}=\Gamma$, thus having the same
modulation for all hydrogen atoms in the crystal. Therefore, we reduce
the 3$N$ dimensions of the \emph{ab initio} potential (with $N$ the
number of atoms in the supercell) to only 3 dimensions.
The PES is 
fitted over total energies generated either by DFT-BLYP or by QMC for nuclear configurations defined on a cylindrical grid. Further details about the model description can be found in Sec.~\ref{Sec:Pes}.

In Fig.~\ref{fig:pes_qmc_dft}, we report the PES profiles obtained by
% Figure environment removed
solving the electronic problem
within the DFT-BLYP (first row) and QMC (third row) methods.
At the volumes taken into account here, both DFT-BLYP PES and QMC PES have two minima connected through the inversion symmetry with respect to the S-S midpoint ((0.5,0,0) in fractional units).
The second row shows a comparison of both energy profiles cut along the line connecting these two PES minima, 
going
through the S-S midpoint. 
For the smallest volume analyzed, V=90.9 a$_0^3$, we found a good agreement between the DFT-BLYP PES and QMC PES, suggesting that electron correlation effects are reasonably well described at the DFT level at high-enough pressures. 
However,
the discrepancy between the two approaches 
appears
when we increase the volume and it grows continuously upon 
pressure release.
For the largest volume considered, V=110.0 a$_0^3$, the height of the double well barrier for QMC is $\sim$270 meV/$\ce{H_3S}$, 80 \% larger than the DFT-BLYP one.

The \change{ferroelectric} transition volume for classical nuclei can be estimated based on the PES by using the Landau theory for continuous phase transitions \cite{Landau_1937}. This method relies on the sign change of the free energy curvature (the total energy curvature at $T$=0 K) at the volume when the two displaced minima merge into a single one, in the symmetric configuration corresponding 
to the point (0.5,0,0) in Fig.~\ref{fig:pes_qmc_dft}.
For DFT-BLYP, we found a critical volume \change{$V_\textrm{ferro}$} around 85 $a_0^3$ corresponding to a pressure of 263 GPa, while for QMC we found the same volume ($\approx$ 85 $a_0^3$) which corresponds to 238 GPa in this case (see Fig.~\ref{fig:drozdov_mod}(a)). We note that the 
\change{$V_\textrm{ferro}$} yielded by the BLYP 3D-PES is in nice agreement with the value at which the shuttling mode frequency vanishes, computed 
\emph{ab initio} 
in the harmonic approximation (see Fig.~\ref{fig:transition_shuttle_phon}). This is a signature of our model PES quality.


\subsection{Quantum 3D model}

In order to have a reliable description of the structural phase transition based on our 3D-PES, we need
to include nuclear quantum effects. We add them by performing PIMD calculations
as implemented in Ref.~\cite{Mouhat_2017}. Numerical details of these
simulations can be found in Sec.~\ref{Sec:PIMD}.
\change{Here, we mention only that in the PIMD simulations of our 3D model the hydrogen atom has an
 effective mass equal to three times the physical hydrogen mass, owing
 to the fact that the PES is expressed per H$_3$S unit and the 3D motion of all hydrogen atoms in the H$_3$S
molecule is concerted by construction.}

In Fig.~\ref{fig:distributions}, we report the projections of the resulting 3D proton density, 
% Figure environment removed
which takes two distinct 
shapes
depending on the volume.
The density
exhibits only one peak centered in the middle of the S-S axis
\change{for small volumes} and
the central peak \change{splits} into two lobes for the largest volumes. 
In the contour plot of Fig.~\ref{fig:distributions}(a), one can clearly see that the doubling of the peak happens in QMC at smaller volumes (higher pressures) than in DFT-BLYP, as expected from the analysis of the classical PES, which shows deeper minima in the QMC PES at fixed volume. In Fig.~\ref{fig:distributions}(b), we plot the distribution of the hydrogen position projected along the shuttling mode direction, by including also the data coming from the \emph{ab initio} PIMD simulations driven by DFT-BLYP forces. Our 3D model has a
similar behavior in comparison with the 
full 
3$N$ dimensional system.
The mode distribution assumes a double peak shape at approximately the same volume for the model (blue lines) and the \emph{ab initio} system (green lines), evaluated for the same BLYP functional. The main differences are the broadness of the distribution, underestimated by the model, and the position of the peak, which lies closer to the S atoms in the 
\emph{ab initio} simulations. 
These differences can
be understood based on the enhanced quantum-thermal fluctuations of
the \emph{ab initio} system compared to the one with a reduced number
of degrees of freedom. Nevertheless, as far as the
\change{the peak splitting} is concerned, the \emph{ab initio} and the model PIMD calculations are in agreement. This 
validates the accuracy of our 3D-PES model, which then allows one to compare directly 
BLYP and QMC results. The projected 1D distribution in
Fig.~\ref{fig:distributions}(b) reveals that the QMC PES leads to a
smaller
volume \change{for the peak splitting}, as shown already in the contour plot of Fig.~\ref{fig:distributions}(a).

By fitting the 
distribution in Fig.~\ref{fig:distributions}(b) and interpolating the parameters obtained for several volumes, it is possible to determine precisely the position of its maximum as a function of volume, and thus the occurrence of 
\change{the bimodal distribution}. 
Moreover, in PIMD we can easily quantify isotope effects
by replacing the hydrogen with the deuterium mass.

We estimate the
\change{peak splitting} to take place in $\ce{H_3S}$ at a volume of
99.6 $a_0^3$ for DFT-BLYP, and of 96.3 $a_0^3$ for QMC.
According to the EOS of
Fig.~\ref{fig:drozdov_mod}(a), these volumes correspond to
pressures of 153 GPa for DFT-BLYP and of 152 GPa for
QMC. These values, reported in
Tab.~\ref{tab:comparison_fluc_dens_d3s_h3s_qmc_blyp}, are in good
agreement with the position of the maximum $T_c$ measured in
experiments \cite{Drozdov_2015}, \change{and they are strongly
  affected by NQE.}
Nevertheless, it is important to underline that the two similar
pressures obtained by BLYP PES and QMC PES after
inclusion of NQE originate from a compensation of errors in the BLYP
values, if we take QMC as reference. Indeed, a
volume
overestimation found in the BLYP PES compensates with a pressure
overestimation in the BLYP EOS to yield approximately the same
pressure \change{for the peak splitting} as the one found in QMC (see Fig.~\ref{fig:drozdov_mod}(a)).

% Figure environment removed

\change{The occurrance of the bimodal distribution signals the
  proximity of a critical region where the quantum proton is  more localized, either
  dynamically
  or statically, in one of the two
  wells. However, from a more rigourous point of view the
  instantaneous localization of the proton, namely the local
  dipole
  formation, is better characterized by}
the ``local moment'' susceptibility, \change{as defined
  below}. Indeed, quantum fluctuations are at work across the
transition to make the hydrogen shuttle between the two PES minima. As
the volume increases and the minima deepen, the fluctuations will
start freezing, leading to the creation of a local
\change{electric dipole} moment,
generated by the \change{statically} displaced proton in the R3m phase, \change{or generated dynamically, by instantaneous configurations where the whole path representing the quantum proton is fully localized in one of the two wells.} PIMD fully accounts for quantum fluctuations, thanks to its imaginary time resolution. We can measure them by computing the imaginary time correlator $ g(\beta/2) = \langle\delta x (0) \delta x(\beta/2)\rangle$, with $\beta = 1/(k_B T)$ the inverse temperature used in the PIMD simulations, and $\delta x (\tau) = x(\tau) - \left\langle x \right\rangle$, where $\left\langle x \right\rangle$ is the thermal quantum average of the $x$ coordinate\change{, corresponding to the symmetric position in the 3D model}. Quantum fluctuations reduce the value of $g(\beta/2)$. A non-zero value of $g(\beta/2)$ can be interpreted by the presence of a finite moment in the 
distribution. In our 3D-PES, this moment is by definition local,
because by model construction the hydrogen dynamics is condensed in a
single 3D site. Therefore, the local moment susceptibility $\chi_g$ is
the normalized variance of $g(\beta/2)$, namely
$\chi_g=\textrm{Var}[g(\beta/2)] / \textrm{Var}[g(0)]$. Within the
local moment fluctuation picture, the
\change{occurrence of local polarization}
can then be estimated by evaluating the volume at which $\chi_g$ is
maximum, as shown in
\change{Fig.~\ref{fig:susceptibility}(a-b)}. This quantity has already been
used in a previous work \cite{Miha} to identify the transition from a
\change{paraelectric}
phase to a disordered regime in an anharmonic oscillator chain,
characterized by a tunable double well potential, where the symmetry
is locally \change{and instantaneously} broken in favor of displaced configurations.
If we describe the
\change{change of regime} based on local moment fluctuations,  we obtain $V_c$ = 104.2 $a_0^3$ for DFT-BLYP, corresponding to $P_c$ = 131 GPa, and $V_c$ = 100.9 $a_0^3$ for QMC, corresponding to $P_c$= 126 GPa (Tab.~\ref{tab:comparison_fluc_dens_d3s_h3s_qmc_blyp}). Also in this case, like for the density probe, cancellation of errors is at play and, by consequence, the two electronic descriptions provide 
almost
the same critical pressure. 

\change{We evaluated $g(\beta/2)$ also in our \emph{ab initio} PIMD simulations, and in Fig.~\ref{fig:susceptibility}(c) we compare it against the values of $g(\beta/2)$ coming from the PIMD solution of the 3D model. The \emph{ab initio} and model results are in statistical agreement for this local quantity, confirming that the 3D model correctly captures the volume evolution of the local polarization.}


\begin{table}[b!]
\centering
\begin{tabular}{|c|cccc|cccc|}
\hline
\textbf{Theory}     & \multicolumn{4}{c|}{\textbf{DFT-BLYP}}                                                       & \multicolumn{4}{c|}{\textbf{QMC}}                                                            \\ \hline
\textbf{Isotope}    & \multicolumn{2}{c|}{$\ce{H_3S}$}                             & \multicolumn{2}{c|}{$\ce{D_3S}$}        & \multicolumn{2}{c|}{$\ce{H_3S}$}                             & \multicolumn{2}{c|}{$\ce{D_3S}$}        \\ \hline
\textbf{Approach}   & \multicolumn{1}{c|}{Fluc.} & \multicolumn{1}{c|}{Dens.} & \multicolumn{1}{c|}{Fluc.} & Dens. & \multicolumn{1}{c|}{Fluc.} & \multicolumn{1}{c|}{Dens.} & \multicolumn{1}{c|}{Fluc.} & Dens. \\ \hline
$V_c$ {[}$a_0^3${]} & \multicolumn{1}{c|}{104.2} & \multicolumn{1}{c|}{99.6}  & \multicolumn{1}{c|}{97.6}  & 96.6  & \multicolumn{1}{c|}{100.9} & \multicolumn{1}{c|}{96.3}  & \multicolumn{1}{c|}{95.6}  & 93.6  \\ \hline
$P_c$ {[}GPa{]}     & \multicolumn{1}{c|}{130}   & \multicolumn{1}{c|}{153}   & \multicolumn{1}{c|}{165}   & 171   & \multicolumn{1}{c|}{126}   & \multicolumn{1}{c|}{152}   & \multicolumn{1}{c|}{156}   & 169   \\ \hline
\end{tabular}
\caption{\change{Transition} pressures and volumes for
  the \change{local moment formation} yielded by PIMD according to the electronic description (DFT-BLYP or QMC), the probe used (density or local moment fluctuations), and the isotope considered ($\ce{H_3S}$ or $\ce{D_3S}$).}
\label{tab:comparison_fluc_dens_d3s_h3s_qmc_blyp}
\end{table}




The two probes we used in this work, \change{the local moment
  susceptibility and the peak splitting}, allow us to determine a lower and
an upper bound for the pressure where the fluctuating local
  dipoles disappear
  in favour of a paraelectric phase, by
  squeezing the compound. Notice that this does not correspond to the ferroelectric
  transition pressure, associated instead with the global
  Im$\bar{3}$m-R3m symmetry breaking, and long-range dipole order,
  which happens at lower values.
The same analysis is carried out for both the $\ce{H_3S}$ and $\ce{D_3S}$ crystals, to estimate the magnitude of isotope effects. We summarize the results in Tab.~\ref{tab:comparison_fluc_dens_d3s_h3s_qmc_blyp}, 
where we show that the hydrogen-to-deuterium substitution brings about
an increase of the
\change{local polarization formation} pressure that falls into the [17-35] GPa range. 



\change{\subsection{Full BLYP-PIMD solution of the H$_3$S phase diagram at 200
    K and comparison with SSCHA}}


\change{
  After having analyzed the local moment formation with the help of
  the 3D model, we turn now the
  attention to the ferroelectric transition, associated with the
  global R3m
  $\rightarrow$ Im$\bar 3$m transformation.
  The suitable order parameter to identify this transition is  $\Delta
  =  \langle \frac{1}{N} \sum_{i=1}^N \delta x_i \rangle$, where the sum runs over all
  the $N$ hydrogen atoms in the supercell, and $\delta x_i$ is the
  distance of the $i$-th proton from the S-S midpoint at a given
  snapshot. An equivalent
  order parameter, showing usually less statistical fluctuations, is
  $\Delta_\text{abs} =  \langle \lvert \frac{1}{N} \sum_{i=1}^N \delta
  x_i \rvert \rangle$. The brackets indicate
  the average over the classical or quantum nuclear distribution. In
 PIMD simulations an additional average is then
 done over the beads positions. From these definitions, it is clear
 that this order parameter can only be computed in our \emph{ab
   initio} simulations, being the 3D model local.
 }

\change{ 
In Fig.~\ref{fig:chi_abinitio}, we plot the volume dependence of the
order parameter $\Delta$ and $\Delta_\text{abs}$ and their
susceptibilities. Their peak is located at the
ferroelectric transition, occurring at $V_\textrm{ferro} = 117 a_0^3$,
$P_\textrm{ferro} \simeq$ 82 GPa, as found in our BLYP-PIMD simulations in the
2$\times$2$\times$2 supercell.
We notice that the peak location is correlated
with the jump in the shuttling mode frequency, reported in both
Fig.~\ref{fig:transition_shuttle_phon} and \ref{fig:chi_abinitio}.
The agreement between the  $\Delta$ and $\Delta_\text{abs}$
susceptibilities and the shuttling mode frequency jump strengthens the
reliability of our estimate. 
Interestingly, at 200K the ferroelectric transition takes place at a pressure
much lower than the one where the local moments are suppressed.
For quantum nuclei,
between these two pressures the
system is in a
regime characterized by disordered local
moments and Im$\bar{3}$m symmetry (see Fig.~\ref{fig:phasediagram} for
the resulting phase diagram).
Accounting for thermal and quantum effects leads to a strong reduction
of the critical ferroelectric pressure observed in the classical
framework, which is as large as 263 GPa at zero temperature.
Furthermore, in order to distinguish between anharmonicity coming from
thermal and NQE, we also performed classical \emph{ab initio} MD
simulations at 200K (see Supplementary Note V of the Supplementary Information (SI)),
which yield a ferroelectric transition at $\approx$
133 GPa (see Fig.~\ref{fig:phasediagram}). Thus, classical anharmonicity accounts for
about 70$\%$ of the total pressure reduction of the ferroelectric
transition at 200K. The remaining 30 $\%$ is due to NQE at the same temperature.
}

% Figure environment removed

\change{
  The ferroelectric transition
cannot be
estimated
at the
QMC level, due to its computational cost when applied to the dynamics of a real
extended system. However, as we have seen, the 3D model,
derived at both the DFT-BLYP and QMC levels, is
enough to determine the formation of local electric dipole moments,
whose pressure $P_c$ matches well the position of the experimental $T_c$ maximum.
}
At variance with previous state-of-the-art calculations based on a
combination of DFT-BLYP and SSCHA frameworks, our PIMD results yield
$P_c$ in a substantial agreement with the experimental finding, and
this irrespective of the electronic theory used to generate the PES.
\change{One should notice here  that the original SSCHA approximation is not able to
  capture the disordered $Im\bar{3}m$ phase, being a mean-field theory
  with no configurational entropy and no direct information of imaginary time correlations, key to detect
  the 
  dynamical
  local moment formation\footnote{Note, however, that time resolved
    extensions of SSCHA have been recently proposed, able in principle to access also retardation effects\cite{Monacelli_2021_PRB}.}. Thus, the critical pressure SSCHA can normally compute
  is the ferroelectric one, $P_\textrm{ferro}$, and not $P_c$.
  }
  To investigate more deeply
  \change{this mismatch,}
  we carry out SSCHA calculations with our model PES (see
  Sec.~\ref{Sec:SSCHA} for details).
  In SSCHA, the occurrence of the asymmetric R3m phase is signalled by
  a centroid displaced with respect to the S-S midpoint.
  The SSCHA 
critical values are $V_\textrm{ferro} = 107.8$ $a_0^3$, $P_\textrm{ferro} = 114$ GPa for the DFT-BLYP PES, and $V_\textrm{ferro} = 102.4$ $a_0^3$, $P_\textrm{ferro} = 118$ GPa for the QMC PES. As in PIMD, there is no significant difference between the electronic structure method used to generate the PES.
Our SSCHA results for 
the DFT-BLYP PES
are in a very good agreement with the outcome of previous SSCHA simulations for the full \emph{ab initio} system \cite{Bianco_2018}, calculated with the same DFT-BLYP functional. 
We find that the SSCHA
\change{overestimates} $P_\textrm{ferro}$ with respect to the one
obtained in PIMD for the same
\change{BLYP functional}, \change{as expected from a
  mean-field theory, and underestimates $P_c$, which is however out
of reach by the SSCHA,} suggesting that the approximated description
of the nuclear Hamiltonian
\change{is the source of}
disagreement with both PIMD and experimental results.

% Figure environment removed

Let us look 
now
at the predictions for the shuttling mode frequencies, plotted in
Fig.~\ref{fig:transition_shuttle_phon} for various methods. It has to
be noted that, within the SSCHA, the phonon frequency of the shuttling
mode shows a jump at \change{$V_\textrm{ferro}$}.
This is due to the 
hop of
the SSCHA centroid 
from
its symmetric position
to a different minimum of the free energy,
already
``preformed'', 
which breaks the symmetry and 
becomes energetically more favorable at $V_\textrm{ferro}$.
Moreover, we also observe an increase of the SSCHA phonon line-width across the transition of the order of 10 cm$^{-1}$. 
\change{A similar jump in the shuttling phonon frequencies is
  detected by our \emph{ab initio} BLYP simulations in correspondence
  with the ferroelectric transition (see Figs.~\ref{fig:transition_shuttle_phon} and \ref{fig:chi_abinitio}).
  }
\change{Nevertheless,} our PIMD phonon determination, 
shows
a progressive phonon softening
without jumps across the
\change{volume region of local
  moment formation}. 
This is not only true within our 3D model PES, but also for our PIMD
calculations driven by \emph{ab initio} forces computed at the
DFT-BLYP level, as shown in
Fig.~\ref{fig:transition_shuttle_phon}. The agreement between
shuttling mode frequencies yielded by the 3D model and the ones given
by \emph{ab initio} calculations \change{in this volume region} highlights once again the quality of our model PES.
\change{This supports the hypothesis of two different
  transitions. The first one is a smooth transition, or crossover, from the paraelectric
  Im$\bar{3}$m to a phase sharing the same  Im$\bar{3}$m
  symmetry and characterized by the formation
  of local and spatially disordered local moments. This phase 
  cannot be detected by looking at the phonon
  frequencies, and it is not accessible within
  the SSCHA formulation. The second one is the ferroelectric transition from the
  disordered Im$\bar{3}$m to the asymmetric R3m phase, 
  which happens at significantly lower pressure than the
  first one, where the shuttling phonon frequency shows a jump.
  The phase diagram deducible from our combined \emph{ab initio} and
  3D model results is drawn in Fig.~\ref{fig:phasediagram}. 
}

\section{Conclusions}
\label{sec12}

% Figure environment removed

In this work, starting from \emph{ab initio} electronic structure calculations, we generated a model PES to describe the shuttling mode of hydrogen in $\ce{H_3S}$, responsible for the R3m $\rightarrow$ Im$\bar 3$m transition, which was originally associated with the 
$T_c$ maximum
as a function of pressure. Despite the fact that such a hydrogen symmetrization 
is expected to happen in $\ce{H_3S}$ upon compression, so far no theoretical method has been able to spot it 
at pressures near the one that maximizes $T_c$ in experiments. This raised doubts on the original association between superconductivity and structural transition \cite{Akashi_2016,Azadi_2017}, worsened by the fact that other competing symmetries could be stable in the same pressure range \cite{Goncharov_2016,Li_2016,Guigue_2017,cui2019favored}.
The mismatch found between previous theoretical estimates of the
critical pressure $P_c$
and
the experimental values for the $T_c$ maximum is solved 
by applying
state-of-the-art computational methods in both the electronic and nuclear Hamiltonians, namely using QMC calculations for electrons, and the PIMD approach for nuclei. 
Within our QMC+PIMD approach, the experimental pressure where $T_c$ is maximum is bracketed by the $P_c$ value estimated from the local fluctuations probe and the one determined by the 
transformation
of the bimodal hydrogen distribution into a 
unimodal one.
Consequently, these two probes provide a lower and an upper bound for the critical pressure, with a range between the two of $\approx$ 20 GPa. 
The range of transition pressures identified is consistent with the available experimental data for the $T_c$ maximum \cite{Drozdov_2015,Einaga_2016,Minkov_2020} 
for both $\ce{H_3S}$ and $\ce{D_3S}$, as we can see in
Fig.~\ref{fig:drozdov_mod}(b).


We have thus shown that the occurrence of the $T_c$ maximum 
should be linked with the formation of the phase characterized by disordered local moments\cite{Miha}, and it
  cannot be associated with the ferroelectric
  R3m $\longrightarrow$ Im$\bar{3}$m transformation, which takes place at a lower pressure $P_\textrm{ferro}$ compared with $P_c$. According to our outcome, $T_c$ reaches its maximum when the local dipole moments melt upon compression, and protons become fully delocalized across the PES barrier.

Furthermore, we notice that the \emph{ab initio} electronic structure computed at the DFT-BLYP level predicts very good results for the critical pressure, similar to those obtained by QMC.
However, it is important to stress that the DFT-BLYP pressures are affected by error compensation, 
the overestimation of the critical volume 
being
balanced by a different EOS if compared against QMC calculations. This aspect underlines the importance of using an accurate electronic description, beyond the DFT level. The generation of our model PES, built to describe the hydrogen shuttling mode, allowed us to exploit the QMC energies in a PIMD framework, otherwise unfeasible in the full 3$N$ dimensional system.

We conclude by noting that the the R3m $\rightarrow$ Im$\bar 3$m structural phase transition in sulfur hydride has strong analogies with the hydrogen bond symmetrization in other compounds such as high-pressure ice, where, upon compression, 
phase VII and VIII
hosting displaced protons, 
stable at lower pressure, are expected to transform into the symmetric phase X \cite{Pruzan2003,Benoit1998}.
However, it is still a matter of debate whether the transformation is direct or whether another intermediate disordered 
structure
appears, with protons only partially symmetrized. 
In
this respect, 
further work is needed to extend our model beyond the collective path dynamics to treat non-local spatial correlations and disordered patterns. Machine learning schemes could then be useful to generate more extended PES from QMC data \cite{tirelli_2022,Huang2022,Ceperley2023} 
with the aim at including a larger variety of hydrogen configurations in PIMD calculations by keeping the same QMC accuracy. 


\section{Methods}
\label{sec11}

\subsection{Electronic structure calculations for the PES model}
\label{Sec:elstructurePES}
For the DFT electronic structure calculations, we used the Quantum Espresso (QE) suite of codes \cite{qe1,qe2}, while for the QMC calculations, we employed the TurboRVB package \cite{Nakano_2020}. For sake of consistency, in both DFT and QMC calculations, we used the same set of pseudopotentials. Namely, 
we treated the sulfur atom with the ccECP neon-core pseudopotential \cite{Bennett_2017} particularly suited for correlated calculations, available in both the QE-compatible Unified Pseudopotential Format (UPF) and in the TurboRVB-compatible Gaussian expansion format. For hydrogen, we used the bare Coulomb potential, with a very short-range cutoff for a QE usage within the plane-wave framework. In the QMC calculations instead, no short-range cutoff is needed for the bare Coulomb potential, because the nuclear cusp conditions are automatically fulfilled by our QMC wave function (see below). These pseudopotentials have been chosen after performing preliminary calculations at the DFT level to test their accuracy. We also tested other pseudopotentials (ultrasoft (US), projector augmented wave (PAW), and a combination of the above), by comparing the total energy profile obtained by moving the hydrogen atom away from the S-S midpoint, and constrained to stay on
the S-S axis. This leads to a very crude one-dimensional (1D) PES, which is however useful for testing purposes, with the advantage that it is easily computable for its simplicity.
We took as reference the total DFT energy computed with the all-electron LAPW approach, as implemented in Elk \cite{elk}. The ccECP pseudopotential for the sulfur atom and the bare Coulomb potential with short-range cutoff for the hydrogen atom turned out to be the most accurate choice \change{(see Fig.~S.1 of the SI)}. 

For 
single-point calculations at selected nuclear configurations,
we carried out DFT calculations with the Becke-Lee-Yang-Parr (BLYP) functional \cite{Becke_1988,Lee_1988}. The cutoff energy for plane waves is set to 200 Ry (due to the hardness of the H Coulomb pseudopotential), with the smearing parameter equal to 0.002 Ry and a $\mathbf{k}$-points grid of 32x32x32.

For the QMC calculations, we used a Slater-Jastrow wavefunction $\Psi$, which reads as:
\begin{equation}
\label{Eq:QMC_wavefunction}
    \Psi = \Phi_{S} \cdot \exp(J),
\end{equation}
where the term $\exp\left( J \right)$ is the Jastrow factor, symmetric under electron exchange, while $\Phi_{S}$ is the antisymmetric Slater determinant. 
The Slater orbitals in $\Phi_{S}$ are generated by DFT calculations within the Local Density Approximation (LDA) 
\cite{Kohn1965}, 
performed in a Gaussian basis set by means of the DFT 
code
built in TurboRVB. For the sulfur atom, we employed a modified cc-pVTZ primitive basis set with $6s6p2d1f$ components, contracted into 11 hybrid orbitals through the Geminal Embedded Orbitals (GEO) procedure \cite{Sorella_GEO}. For hydrogen, we used a modified cc-pVTZ primitive basis set with $4s2p1d$ components contracted into 6 GEO hybrid orbitals. 

The Jastrow exponent $J$ 
introduces explicitly electronic 
correlation in the wavefunction, and it can be decomposed into three terms, such that $J  = J_1 + J_2 + J_{3}$. 

$J_1$ is the so-called one-body term, which takes into account the interaction effects between the electrons $i$ and a nucleus $I$, and it depends on the relative electron-nucleus distances $r_{iI}$. $J_2$ is the so-called two-body term, treating the correlations between
electrons $i$ and $j$, and depending on their relative distance $r_{ij}$. Both $J_1$ and $J_2$ are designed to fulfill the electron-nucleus and electron-electron cusp conditions, respectively. They read as $J_1=\sum_{i=1}^{N_e} \sum_{I=1}^{N} u_I(r_{iI})$, and $J_2=\sum_{i<j=1}^{N_e} v(r_{ij})$, where $N$ ($N_e$) is the number of nuclei (electrons) in the supercell, and the functions $u$ and $v$ are defined as follows:
\begin{eqnarray}
    u_I(r) &=& \frac{Z_I}{a} (1-e^{-ar }) \label{Eq:u_func}\\
    v(r) &=& \frac{ r }{2(1+b r)} \label{Eq:v_func},
\end{eqnarray}
with $a$ and $b$ variational parameters, and $Z_I$ the charge of the $I$-th pseudoatom.  
The coefficients in Eqs.~\ref{Eq:u_func} and \ref{Eq:v_func} are set to fulfill the Kato cusp conditions for electron-nucleus and electron-electron coalescence, respectively \cite{Kato1957}. 

$J_{3}$ is the three-body term that accounts for the electron-electron-nucleus interactions. As defined in TurboRVB, it is also intrinsically non-homogeneous, because it depends on the individual electron positions and not only on the relative distances, which is less accurate.
Being non-homogeneous, it is expanded on a modified atomic Gaussian basis set of $2s2p1d$ atomic orbitals, for both sulfur and hydrogen atoms. 

The $J_3$ parameters, together with $a$ and $b$, are optimized by minimizing the variational energy of the many-body wavefunction in Eq.~\ref{Eq:QMC_wavefunction}.
The Slater part is instead kept frozen as determined by DFT-LDA.
As stochastic minimization algorithm, we employed the linear method \cite{Umrigar_opt}.
We then carried out lattice regularized diffusion Monte Carlo (LRDMC) calculations \cite{Casula_lrdmc}, to stochastically project the initial wavefunction towards the ground state of the system, within the fixed node approximation. 
\change{Within this approximation, the LDA nodes provide accurate results for this system, as verified in Supplementary Note II of the SI.}
In LRDMC, we used a lattice space of 0.25 $a_0$, which is known to produce converged energy differences. We started the projection from the best variational state optimized in the previous step, taken as trial wavefunction. Finite-size scaling has been performed on the 2x2x1, 2x2x2, 3x2x2 and 3x3x2 real-space supercells in order to extrapolate the LRDMC total energy to the thermodynamic limit, by also using Kwee-Zhang-Krakauer (KZK) \cite{Kwee2008}  corrections to make its size dependence milder.

This workflow has been repeated for every point in the real-space grid used to interpolate the PES model from \emph{ab initio} data (see Sec.~\ref{Sec:Pes}).

\subsection{Potential energy surface parametrization}
\label{Sec:Pes}

To derive an effective low-dimensional PES, we considered the collective and concerted motion of all the hydrogen atoms of the cubic unit cell, with sulfur atoms forming a bcc sublattice. The position of a hydrogen atom is described by the cylindrical coordinates $x$, $r$ and $\phi$, defined along the axis connecting the two flanking sulfur atoms (S-S axis): $x$ is the position of the hydrogen atom along the S-S axis, $r$ is the radial distance from the S-S axis, and $\phi$ is the azimuthal angle, wrapping around the same axis. We use fractional coordinates, where the lengths are expressed in $d_\textrm{SS}$ units, $d_\textrm{SS}$ being the lattice parameter of the bcc unit cell. Within this reference system, the S-S midpoint has coordinates $(x, r, \phi) \equiv (0.5,0,0)$. We assume that all hydrogen atoms in the unit cell move in the same way. This fixes the choice of a collective path connecting the Im$\bar 3$m symmetry (with all hydrogen atoms sitting at the S-S midpoints) to the R3m one (with all hydrogen atoms coherently displaced from the midpoint). In this way, we apply a dimensionality reduction of the full potential, depending on 3$N$ dimensional coordinates, where $N$ is the number of atoms in the cell, to a much simpler 3D PES: $E=E(x,r,\phi)$.

The functional form of our 3D PES is constructed as follows:
\begin{equation}
    E(x,r,\phi) = A(x,r) + B(x,r)\sin(\phi + 5\pi/4),
    \label{Eq:E_def}
\end{equation}
with:
\begin{eqnarray}
    A(x,r) & = &
                 \frac{f_{\textrm{max}}(x,r)+f_{\textrm{min}}(x,r)}{2}, \nonumber
  \\
  B(x,r) & = & \frac{f_{\textrm{max}}(x,r)-f_{\textrm{min}}(x,r)}{2},
    \label{Eq:AB_def}
\end{eqnarray}
and where $f_{\textrm{min}}$ and $f_{\textrm{max}}$ are defined as:
\begin{equation}
\begin{split}
f_{\textrm{min},\textrm{max}}(x,r) = 
    & \ a +  \frac{1}{2}b (x-0.5)^2 + \frac{1}{4} c (x-0.5)^4\\
    &+ d r + \frac{1}{2}e  r^2\\
    &\pm f  (x-0.5) r \pm  g  (x-0.5) r^2 \\
    &\pm h_1 (x-0.5)^3  r \pm h_2 (x-0.5)^3  r^2\\
    &+ h_3  (x-0.5)^2  r + h_4  (x-0.5)^2  r^2 \\
    &+ h_5  (x-0.5)^4  r + h_6  (x-0.5)^4  r^2
\end{split}
\label{Eq:fmaxmin_def}
\end{equation}

The choice of this functional form 
is motivated by
%comes from 
the symmetries of the system. For a fixed $\lbrace x,r \rbrace$, the %energy 
potential
$E$ has an angular dependence that varies following a sine curve with $2\pi$-periodicity. In particular, for $x < 0.5$, $E$ has a minimum given by $f_{\textrm{min}}$ at $\phi=\pi/4$ and the maximum $f_{\textrm{max}}$ at $\phi=5\pi/4$. This dependence is built in Eqs.~\ref{Eq:E_def} and \ref{Eq:AB_def}. The $\{f_i(x,r)\}_{i=\textrm{min},\textrm{max}}$ functions in Eq.~\ref{Eq:fmaxmin_def} are a composition of the following terms: a Landau-type potential that well describes the energy profile for $r=0$, a second-order polynomial function in $r$ for $x=0.5$, and mixed terms made of cross products of factors up to the fourth order in $(x-0.5)$ and up to the second order in $r$, which give enough flexibility in order to well reproduce the total PES. The signs in $\{f_i(x,r)\}_{i=\textrm{min},\textrm{max}}$ ensure the symmetry: $E(1-x,r,\phi+\pi)=E(x,r,\phi)$, fulfilled by the system.

We sampled the PES by discretizing the 3D space according to the following grid defined in cylindrical coordinates: $x = \left[0.42,0.44,0.46,0.48,0.5\right]$ (in $d_{SS}$ units), $r=[0.00,0.02,0.05,0.08]$ (in $d_{SS}$ units) and $\phi=[\pi/4,5\pi /4]$. For these points we computed the \emph{ab initio} total energies, given either by DFT-BLYP or by QMC calculations. We finally used the generated datasets to best fit the PES, parametrized according to Eqs.~\ref{Eq:E_def}, \ref{Eq:AB_def} and \ref{Eq:fmaxmin_def}. The root mean square error of these fits amounts to $\approx$ 1 meV/H$_3$S.

\subsection{Equation of state}
\label{Sec:EOS}

In order to get the pressure associated to each volume, $P=P(V)$, we use the Vinet EOS:
\begin{equation}
         P(V) = 3B_0 \frac{1-\eta}{\eta^2} \exp\left(-\frac{3}{2}(B_0'-1)(1-\eta)\right).
        \label{eq:vinet_P}
\end{equation}
with $\eta=(V/V_0)^{1/3}$. In Eq.~\ref{eq:vinet_P}, the parameters $V_0$, $B_0$ and $B_0'$ are the equilibrium volume, the isothermal bulk modulus, and the derivative of bulk modulus with respect to pressure, respectively. 
The Vinet EOS \cite{Vinet_EOS} is empirical and, despite having only a few parameters, it is very accurate 
to describe solids under extreme conditions.
We obtained
$V_0$, $B_0$ and $B_0'$
by fitting the $E=E(V)$ relation for the Im$\bar{3}$m phase, where the total energy is computed from first principles, either by DFT-BLYP or by QMC, on a grid of volumes (see Fig.~\ref{fig:drozdov_mod}(a)). In the fit, we disregarded the Zero-Point Energy (ZPE) contribution, because we verified that the ZPE variation is very small ($<$ 1 mHa/H$_3$S) in the range of pressures analyzed here within the same Im$\bar{3}$m phase. 

\subsection{PIMD simulations}
\label{Sec:PIMD}
The 
PIMD simulations are carried out at 200 K using 20 beads 
\change{with \emph{ab initio} DFT forces, while using 40 beads with 3D-PES forces,}
to take into account quantum effects. 
\change{A convergence study of the PIMD results with respect to the number of beads is reported in Supplementary Note III of the SI.}
Nuclei are evolved in time using the PIOUD integrator~\cite{Mouhat_2017} with a time step equal to 0.75 fs and a friction parameter of the Langevin thermostat equal to 1.46$\cdot$10$^{-3}$ atomic units. The latter value is the same as in Ref.~\cite{Mouhat_2017}, where it is found to be optimal for both stochastic and deterministic forces. Simulations lasted around 6 ps, until the convergence of the vibrational modes at $\Gamma$ is reached. Forces are computed from the Born-Oppenheimer PES evaluated at DFT level within the QE package, or from the model PES defined in Sec.~\ref{Sec:Pes}. In case of \emph{ab initio} PIMD, we used a BLYP 
functional for computing the PES. The wavefunction cut-off for the PES is set to 90 Ry (420 Ry for the charge density), while the Fermi smearing is Gaussian and set equal to 0.03 Ry. PIMD simulations are performed using 2x2x2 real-space supercells, containing in each case 32 atoms, and the corresponding reciprocal-space mesh is always equal to 9x9x9. We used a smaller plane-wave cutoff than the one used in single-point DFT calculations, because in PIMD we replaced the hard H Coulomb pseudopotential with a smoother PAW one. This has been necessary to speed up the PIMD calculations, which would otherwise have been too time consuming.

\subsection{SSCHA simulations}
\label{Sec:SSCHA}

Besides the exact description of quantum nuclear motion provided by
PIMD, one can also rely on approximated theories like the SSCHA
\cite{Monacelli2021}, based on a variational principle on the free
energy, which allows one to include quantum nuclear anharmonicity in a
non-perturbative way. Here, we performed SSCHA simulations on the 3D
$\ce{H_3S}$ ($\ce{D_3S}$) model using up to 30000 configurations. The
average proton position (centroid) reported in
\change{Supplementary Fig.~S.5 of the SI} are directly
accessible through the SSCHA free energy minimization.


\subsection{Phonons}
\label{Sec:Phonons}

Harmonic phonons are obtained through 
DFPT
simulations \cite{Baroni_2001} as implemented within QE \cite{qe2}. The same set of DFT parameters and pseudopotentials employed for PIMD simulations were used to compute harmonic phonon dispersions, except for the \textbf{k}-space grid that was chosen equal to 18x18x18, as in this case 
it is referred to
the unit cell. The results of these calculations are shown in Fig.~\ref{fig:phonons}. We specify that the DFPT shuttling mode frequency at $\mathbf{q}=\Gamma$, reported in Fig.~\ref{fig:transition_shuttle_phon} has been computed with higher precision by employing the more accurate H Coulomb pseudopotential, requiring a plane-wave cutoff of 200 Ryd.

Anharmonic phonon frequencies at PIMD level are evaluated by computing the
zero frequency component of the phonon Matsubara Green's function from PIMD simulations. This method has been recently implemented in Ref.~\cite{Morresi_2021} and it has been shown to describe accurately the vibron frequencies of solid phases of hydrogen. Conversely, within the SSCHA, auxiliary phonons are a byproduct of the free energy minimization. However, to get the physical phonons of Fig.~\ref{fig:transition_shuttle_phon}, probed by spectroscopies, we apply the full self-energy dynamical corrections to the auxiliary dynamical matrix, described in detail in Ref.~\cite{Monacelli2021}, including both the third and fourth-order terms.

\begin{acknowledgments}

The authors thank M. Calandra, I. Errea, F. Mauri and L. Monacelli for useful discussions. They acknowledge computational resources provided by GENCI under the allocation number 0906493, which granted access to the HPC resources of IDRIS and TGCC. They also thank RIKEN for providing computational resources of the supercomputer Fugaku through the HPCI System Research Project ID hp220060.
The authors are grateful to the European Centre of Excellence in Exascale Computing TREX-Targeting Real Chemical Accuracy at the Exascale, which partially supported this work.
This project has received funding from the European Unions Horizon
2020 Research and Innovation program under Grant Agreement No. 952165.

\end{acknowledgments}


\clearpage

\onecolumngrid
%\hypersetup{pageanchor=false}

\setcounter{figure}{0}
\setcounter{page}{1}
\setcounter{section}{0}
\setcounter{table}{0}

\renewcommand{\thepage}{S\arabic{page}} 

%\renewcommand{\thesection}{S\arabic{section}}  
%\renewcommand{\thetable}{S\arabic{table}}  
%\renewcommand{\thefigure}{S\arabic{figure}}
%\renewcommand{\thetable}{{\bf \arabic{table}}}   
%\renewcommand{\thefigure}{{\bf \arabic{figure}}}

\renewcommand{\figurename}{{\bf Supplementary Figure}}
\renewcommand{\tablename}{{\bf Supplementary Table}}
\renewcommand\thesection{Supplementary Note \Roman{section}}
\renewcommand\thesubsection{Supplementary Note \Roman{section}.\arabic{subsection}}
\renewcommand\thesubsubsection{Supplementary Note \Roman{section}.\arabic{subsection}.\alph{subsubsection}}
\renewcommand{\theequation}{S.\arabic{equation}}
\renewcommand{\thefigure}{S.\arabic{figure}}
\renewcommand{\thetable}{S.\arabic{table}}
%\renewcommand\refname{Supplementary References}
%\renewcommand*{\citenumfont}[1]{S#1}
%\renewcommand*{\bibnumfmt}[1]{[S#1]}

% ****** Start of file aipsamp.tex ******
%
%   This file is part of the AIP files in the AIP distribution for REVTeX 4.
%   Version 4.1 of REVTeX, October 2009
%
%   Copyright (c) 2009 American Institute of Physics.
%
%   See the AIP README file for restrictions and more information.
%
% TeX'ing this file requires that you have AMS-LaTeX 2.0 installed
% as well as the rest of the prerequisites for REVTeX 4.1
% 
% It also requires running BibTeX. The commands are as follows:
%
%  1)  latex  aipsamp
%  2)  bibtex aipsamp
%  3)  latex  aipsamp
%  4)  latex  aipsamp
%
% Use this file as a source of example code for your aip document.
% Use the file aiptemplate.tex as a template for your document.
\documentclass[%
aip,
% jmp,
% bmf,
% sd,
% rsi,
amsmath,amssymb,
preprint,%
%reprint,%
%author-year,%
%author-numerical,%
% Conference Proceedings
]{revtex4-1}
\usepackage[dvipsnames]{xcolor}
\usepackage{graphicx}% Include figure files
\usepackage{dcolumn}% Align table columns on decimal point
\usepackage{bm}% bold math
%\usepackage[mathlines]{lineno}% Enable numbering of text and display math
%\linenumbers\relax % Commence numbering lines
\usepackage{xr}
\usepackage{mathrsfs}
\usepackage[pdfborder={0 0 0},colorlinks=true,citecolor=blue,linkcolor=blue,CJKbookmarks=true,citecolor=blue]{hyperref}
\usepackage[utf8]{inputenc}
\usepackage[T1]{fontenc}
\usepackage{mathptmx}
\usepackage{etoolbox}
\usepackage{subfigure}
\usepackage[ruled,lined]{algorithm2e}


\newtheorem{theorem}{Theorem}
\newtheorem{lemma}[theorem]{Lemma}
\newtheorem{corollary}[theorem]{Corollary}
\newtheorem{remark}[theorem]{Remark}
%\usepackage{algpseudocode}
%% Apr 2021: AIP requests that the corresponding 
%% email to be moved after the affiliations
\makeatletter
\def\@email#1#2{%
	\endgroup
	\patchcmd{\titleblock@produce}
	{\frontmatter@RRAPformat}
	{\frontmatter@RRAPformat{\produce@RRAP{*#1\href{mailto:#2}{#2}}}\frontmatter@RRAPformat}
	{}{}
}%
\makeatother

%\def\theequation{S\arabic{section}.\arabic{equation}}
%\def\thesection{S\arabic{section}}
%\def\thefigure{S\arabic{figure}}
\externaldocument{aipsamp4}



\begin{document}
	
		
\title{ A new approach for stability analysis of $1-$D wave equation with time delay}
\author{Shijie Zhou}	
%\affiliation{School of Mathematical Sciences, Shandong University, Shandong 250100 ,China}
\affiliation{Research Institute of Intelligent Complex Systems, Fudan University, Shanghai 200433, China}
\author{Hongyinping Feng}
\affiliation{School of Mathematical Sciences, Shanxi University}	
\author{Zhiqiang Wang} \email{zqw@fudan.edu.cn}
\affiliation{School of Mathematical Sciences, Fudan University}
	
	
	
	\date{\today}
	
		\maketitle


	
	
	\section{Introduction}\label{section1}
The time delay  is  ubiquitous   in many engineering control systems. Both
  actuator delay
and sensor delay  may change not only the performance
but also cause damages to the stability of the systems.
Since  the time delay is an infinite-dimensional dynamics itself,  the
  time delay in observation and control presents great
challenge in distributed parameter control systems \cite{Logemann}.
It is well known that a small
time delay in a stabilizing boundary output feedback could destabilize the system
\cite{Datko86, Datko88,Datko93}. This  implies that
some       controllers      of   PDEs  may become     practically  not implementable in the
presence of the time delay.
It is therefore  very important to consider the time delay
in  the process of the controller designs.

\textcolor{black}{In the past a few decades, many efforts have been made   for the  stabilization and \textcolor{black}{stability analysis} of PDEs
 with time delay. The seminal result on feedback stabilization for one-dimensional wave equation was considered in \cite{Datko86}}
%In this paper, we consider  output feedback stabilization for the following
 %one-dimensional wave equation which was also  considered in \cite{Datko86}:
\begin{equation} \label{20199111658}
\left\{\begin{array}{l}
  z_{tt}(x,t)=z_{xx}(x,t), x\in(0,1), t>0, \\
   z(0,t)=0,\\
 z_x(1,t)=u(t), t>0, \\
y(t)=z_t(1,t), t>0,
\end{array}\right.
\end{equation}
 where $u(t)$ is the control input and  $y(t)$ the output. %We shall \textcolor{black}{investigate} an
%analytic form of  output feedback to  stabilize system \eqref{20199111658}
%with time delay appearing in the input or in the output.
  When the time delay is absent, the stabilization of \eqref{20199111658} is trivial
that a propositional  feedback $u(t)=-kz_t(1,t)$ with $k>0$ can stabilize exponentially the system.
However, this feedback is not robust to the input delay. In other word,
 system
\begin{equation} \label{20199111503}
\left\{\begin{array}{l}
    z_{tt}(x,t)=z_{xx}(x,t) , x\in(0,1), t>0,
 \\z(0,t)=0,
 \\z_x(1,t)=-kz_t(1,t-\tau), k>0, t\geq0
\end{array}\right.
\end{equation}
is always unstable as indicated in \cite{Datko86}, no matter how small the time delay  $\tau$ is. %\sout{In the past a few decades, many efforts have been made   for the  stabilization and \textcolor{black}{stability analysis} of PDEs
% with time delay.} \textcolor{black}{Since then, numerous researchers have made significant contributions to stabilization and stability analysis for different structures of PDEs with time delay.}
 A typical study was first made  in  \cite{XuGQ2006delay} where stabilization for
 system \eqref{20199111658} with input delay was considered by
 regarding the time delay as
   a dynamics represented by  one-dimensional   transport equation.
 The controller in \cite{XuGQ2006delay} is
split   into two parts: $u(t)=-k\mu z_t(1, t) - k(1 - \mu)z_t(1, t-\tau)$ and
the time delay free part cannot be zero. The same approach was also used in \cite{Sage1,Sage2}.
The  backstepping method is an another powerful tool to compensate the
 time delay, which has been applied  to finite-dimensional systems with actuator and sensor delays
 in \cite{Krstic2008scl}.   In monograph
 \cite[Chapter 19]{Krsticdelaybook}, stabilization for  an anti-stable wave equation
 with input delay was  realized by the backstepping method.
 Although the time delay can be  compensated in the infinite-dimensional   actuator,
  the controller,  as a full state feedback, was  very complicated in  \cite[Chapter 19]{Krsticdelaybook}.
  When the output is suffered from a time delay, an  observer/predictor
  was proposed in  \cite{GuoXuCZDelay2012}, which is  systematical
  and has been extended recently to the abstract systems  in
   \cite{MeiZD}  and \cite{GuoMEiZDTAC}.  But in general, the
   convergence in \cite{MeiZD, GuoMEiZDTAC} is valid only for smooth initial states.
   Although the convergence of \cite{GuoXuCZDelay2012} is true for general initial states,
   the controller seems still not straightforward. When
the input delay equals even multiples of the wave propagation time, the system can be
stabilized exponentially by  direct feedback discussed in \cite{Wangdelay2011}.
In \cite{Gugat}, a specific time delay in boundary observation can be used to stabilize the wave equation. \textcolor{black}{In \cite{Gugat2}, a switching delay feedback can be used to stabilize a vibrating string.} In \cite{FSIAM2} and \cite{FengTAC},  a  non-collocated feedback was
proposed to cope with the output delay being equal to one.
	
	
	
	
In this paper,  we \textcolor{black}{consider} a new controller to stabilize system \eqref{20199111658}
with input delay.  %{\color{red} or   output delay}.
 By considering  the   time delay  dynamic as a first order transport equation,
 the problem is converted into boundary control of a cascaded PDE system. By virtue of
boundary stabilization for first order hyperbolic systems,  we are able to design the
feedback control.
  The idea was inspired by \cite[Lemma 3.5]{WangZQ2013}  where
   the following system of transport equation
\begin{equation} \label{20189202001}
\left\{\begin{array}{l}
  \tau w_t(x,t)+w_x(x,t)=0,\\
 w(0,t)=k w(1,t) +  {(k-1) \mu}  \int_0^1w(x,t){\rm d}x
\end{array}\right.
\end{equation}
is exponentially stable
in the state space $L^2(0,1)$ if and only if $ |k |<1$ and $\mu>-1$.
 This     inspires us  that the boundary feedback   $u(t)= -c_1w(1,t) -c_2\int_0^1w(x,t){\rm d}x$ with $c_1,c_2 \in\mathbb{R}$
  may compensate   the time delay.

	
	
{\color{black}
Enlightened by \cite{Wangdelay2011}, we use the semigroup approach to explain the well-posedness of the system and the Riesz basis approach to get the dynamical behavior of the system in terms of vibrating frequencies. 
%We get the spectrum-determined growth condition for our system. 
The major contribution of our paper is to develop a new method for spectral analysis.
% (see Section \ref{section3}).
 We derive sufficient and necessary conditions for the feedback gain and time delay which guarantee the exponential stability of the closed-loop system.  Comparing with similar conditions developed in \cite{Wangdelay2011},
  %we not only discuss about the situation in which $\tau$ is rational, but also talk about the situation in which $\tau$ is irrational.  More importantly,
 %We prove that the exponential stability can be achieved if and only if the time delay $\tau>0$ is an even number.  
 we   get 
 the \textcolor{black}{explicit term} of the stability region of $c$ for different values of $\tau$ \textcolor{black}{and from this}
  we \textcolor{black}{easily} obtain the shrink of the stability region as $\tau$ increases.
    %As a by-product, we design an observer based output feedback to stabilize exponentially the system with the time delay appearing in the boundary output (see Remark \ref{remarkoutputdelay}).
 %Our methods can also be used to solve the spectral analysis problem in \cite{Wangdelay2011}~(see Remark \ref{remarkwangdelay2011}). \textcolor{black}{Furthermore, we investigate the robustness to a small perturbation in time delay in low frequencies (See Section \ref{section4.5}).  Indifferent from \cite[Section 7]{Wangdelay2011}, \cite{Datko86} in which they investigate the lack of robustness by giving exact expressions for eigenvalues for a particular sequence of delay perturbations (See \cite[Theorem 7.2]{Wangdelay2011}), we prove that any small perturbation of $\epsilon\in\mathbb{R}$ in time delay will excite a high frequency mode (i.e., a mode with frequency on the order of $\mathcal{O}(\dfrac{1}{|\epsilon|})$).}
 Another main advantage of the proposed method lies in the 
  investigation of   the robustness to a small perturbation in time delay in high frequencies. 
Actually, we prove that any small perturbation of $\epsilon\in\mathbb{R}$ in time delay will excite a high frequency mode (i.e., a mode with frequency on the order of $\mathcal{O}(\dfrac{1}{|\epsilon|})$ as $\epsilon\to 0$).
  In this way, we
give an intrisic mathematical  interpretation of the destabilizing effect of arbitrarily small time delays and hence verify the judgement (or conjecture) given in \cite[Page 5, remark]{Datko86}.  This gives a mathematical explanation why numerical experiments usually do not demonstrate the non-robustness when a small perturbation is added to the time delay (See Remark \ref{robustremark}).}

We proceed as follows. %In Section \ref{section2}, we present  a  simple feedback  which stabilizes exponentially the system in the presence of the input delay.
  In Section \ref{section2}, we present the model formulation, well-posedness of the closed-loop system and fundamental spectral analysis.  In Section \ref{section3}, we derive sufficient and necessary conditions for the feedback gain and time delay which guarantee the exponential stability of the closed-loop system.  In Section \ref{section4.5}, we investigate the robustness to a small perturbation in time delay in high frequencies.  In Section \ref{section5}, we present some numerical simulations for illustration.
%For the sake of readability, some mathematical proofs that are less relevant to the control design
  %are put in the Appendix.

	
	\textbf{Notations}. In this paper, $\mathbb{R}\ ( \mathbb{R}_{+}, \mathbb{R}_{-})$ denote the set of all real (positive, negative) numbers, respectively. $\mathbb{C}_+\ (\mathbb{C}_-,\mathbb{C}_0)$ denote the set of complex numbers with positive (negative, zero) real parts, respectively. $\mathbb{Z}\ (\mathbb{N}^{*},\mathbb{N})$ denote the set of integers (positive integers, non-negative integers), respectively. The imaginary unit is denoted by ${\rm i}$, where $\rm i=\sqrt{-1}$. ${\rm det}(\cdot)$ denotes the determinant of a matrix. For $\lambda\in\mathbb{C}$, ${\rm Re} \lambda$, ${\rm Im} \lambda$, ${\rm arg}\lambda$ and $|\lambda|$ denote the real part, imaginary part, principle value of argument and the norm of $\lambda$, respectively. For $x\in\mathbb{R}$, ${\rm Sgn}(x)$ denotes the sign of $x$, which indicates ${\rm Sgn}(x)=1,0$ and $-1$ for $x>0$, $x=0$ and $x<0$, respectively.  For an operator $\mathscr{A}$, $D(\mathscr{A})$, $\rho(\mathscr{A})$, $\sigma(\mathscr{A})$ and $\sigma_p(\mathscr{A})$ denote the domain, regular point set, spectral point set and eigenvalue set of the operator $\mathscr{A}$, respectively. $\mathbb{D}$ denotes the unit circle $\big\{z\big||z|< 1\big\}$ in the complex plane, while $\overline{\mathbb{D}}$ denotes its closure $\big\{z\big||z|\leq 1\big\}$.

	
	
\section{Preliminaries} \label{section2}
\subsection{Model formulation}
We consider the
 stabilization of  the following wave equation with time delay in the control:
\begin{equation} \label{2018923937}
\left\{\begin{array}{l}
  z_{tt}(x,t)=z_{xx}(x,t),\\
 z(0,t)=0,\\
 z_x(1,t)=u(t-\tau),\\
y(t)=z_t(1,t),{t\geq0,}
\end{array}\right.
\end{equation}
 where $u(t)$ is the control input, $y(t)$ is the output,  and $\tau$ is the
time delay. For notational simplicity, we omit in equations hereafter the
obvious domains for both time $t$ and spatial variable $x$ when
there is no confusion.
Set $w(x,t)=u(t-\tau x)$. Then, the time delay system
\eqref{2018923937} is written as
 \begin{equation} \label{2018923934}
\left\{\begin{array}{l}
  z_{tt}(x,t)=z_{xx}(x,t),\\
 z(0,t)=0,\\
 z_x(1,t)=w(1,t),\\
 \tau w_t(x,t)+w_x(x,t)=0,\\
 w(0,t)=u(t),\\
y(t)=z_t(1,t),
\end{array}\right.
\end{equation}
which is a  {cascaded}  PDE system without explicitly the time delay.
 Inspired from \eqref{20189202001}, a  feedback control  is  designed as
\begin{equation} \label{2018923949}
\left.\begin{array}{l}
  u(t)=- c_1 w(1,t)-c_2   z_t(1,t) ,
\end{array}\right.
\end{equation}
where $c_1, c_2\in\mathbb{R}$ are   tuning parameters.  Since we only consider the
output feedback, the integral term $\int_0^1w(x,t){\rm d}x$  in  \eqref{20189202001}  is ignored.
The first term of \eqref{2018923949} is damper for the transport equation
and the second term is a direct proportional feedback for wave equation without time delay. When $c_1=0$, the system becomes \eqref{20199111503}.
Under the feedback  \eqref{2018923949}, the closed-loop of system \eqref{2018923934} reads
\begin{equation} \label{20189202158}
\left\{\begin{array}{l}
  z_{tt}(x,t)=z_{xx}(x,t),\\
 z(0,t)=0,\\
 z_x(1,t)=w(1,t),\\
 \tau w_t(x,t)+w_x(x,t)=0,\\
 w(0,t)= kz_t(1,t) .
\end{array}\right.
\end{equation}

	\subsection{Well-posedness of system \eqref{20189202158}}








We consider system \eqref{20189202158} in the state space
\begin{equation} \label{20199121519}
 \mathcal{X}=H^1_L(0,1)\times (L^2(0,1))^2,
 \end{equation}
  where
$H^1_L(0,1)=\{f\in H^1 (0,1)\ |\ f(0)=0\}$. The inner product in $ \mathcal{X}$ is defined  by
\begin{equation} \label{20189241648}
\langle (f_1,g_1,h_1),(f_2,g_2,h_2)\rangle_{ \mathcal{X}}=\int_0^1[ f_1'(x)\overline{ f_2'(x)}
+ g_1(x)\overline{g_2(x)}+  h_1(x)\overline{h_2(x)}]{\rm d}x
\end{equation}
for $   (f_i,g_i,h_i)\in \mathcal{X},\ i=1,2 $.
System \eqref{20189202158} can be written as an evolutionary equation in $ \mathcal{X}$:
\begin{equation} \label{20189241451}
\dfrac{\rm d}{{\rm d}t}(z(\cdot,t),z_t(\cdot,t),w (\cdot,t))= \mathscr{A} (z(\cdot,t),z_t(\cdot,t),w (\cdot,t)),
\end{equation}
where the operator $ \mathscr{A}: D( \mathscr{A})\subset  \mathcal{X}\to \mathcal{X}$ is defined by
\begin{equation} \label{20189241453}
\left\{\begin{array}{l}
 \mathscr{A}(f,g,h)=(g,f'',-\tau^{-1}h' ),\forall (f,g,h)\in D(\mathscr{A}),
\\D(\mathscr{A})=\left\{( f,g,h) \ |\ f\in H^2(0,1),g,h\in H^1(0,1),f(0)=g(0)=0,\right.
\\ \hspace{2cm}\left. f'(1)=h(1),\ h(0)=-c_1 h(1)-c_2 g(1)\right\}.
\end{array}\right.
\end{equation}


\begin{lemma}\label{Lemma20232101235}
Let  the operator  $\mathscr{A} $ be defined by  \eqref{20189241453}. Then $\mathscr{A}^{-1}$ exists and is compact. Hence, $\sigma(\mathscr{A})$ consists of isolated eigenvalues multiplicity only.
\end{lemma}
Proof
For any $(\hat{f},\hat{g},\hat{h})\in\mathcal{X}$, we solve the equation $\mathscr{A}(f,g,h)=(\hat{f},\hat{g},\hat{h})$ to get
 \begin{equation} \label{20189261005-2}
\left\{\begin{array}{l}
  g=\hat{f},
\\ f(x)=f'(1)x-\int_0^x\int_{\alpha}^1\hat{g}(s){\rm d}s,\ f'(1)=-\frac{1}{1+c}
\left( c \hat{f}(1)+ \tau\int_0^1\hat{h}(s){\rm d}s\right),
\\ h(x)=h(0)-\tau\int_0^x\hat{h}(s){\rm d}s,\ h(0)=\frac{c}{1+c }
\left(  \tau\int_0^1\hat{h}(s){\rm d}s-  \hat{f}(1)\right),
\end{array}\right.
\end{equation}
which implies, by the Sobolev trace-embedding,  that $\mathscr{A}^{-1}$
exists and is compact in $\mathcal{X}$. So
$\sigma(\mathscr{A})$ consists of isolated eigenvalues of finite algebraic multiplicity.
~~


\begin{theorem}\label{wellposedness}
Suppose that  $c_1, c_2\in \mathbb{R} $ and   $  \tau  >0$.
  Then,  the operator  $\mathscr{A} $  defined by  \eqref{20189241453} generates a
$C_0$-group ${\rm e}^{\mathscr{A}t}$ on $\mathcal{X}$.
\end{theorem}
Proof
Inspired by  \cite[Theorem 2.2]{Wangdelay2011},
we first introduce  a new inner product
 \begin{small}\begin{equation*} \label{20189241453*0}
\left.\begin{array}{l}
     \langle(f_1,g_1,h_1),(f_2,g_2,h_2)\rangle_1
\triangleq \int_0^1 {\rm e}^{\alpha x}(f_1'-g_1)\overline{ (f_2'-g_2)}{\rm d}x
+ \int_0^1 {\rm e}^{\beta x}(f_1'+g_1)\overline{( f_2'+g_2)}{\rm d}x \\ %\\
     \hspace{46mm} +\tau \int_0^1 {\rm e}^{\gamma x} h_1(x)\overline{h_2(x)}{\rm d}x ,\quad \
\forall\ (f_i,g_i,h_i)\in\mathcal{X},\ i=1,2 ,
\end{array}\right.
\end{equation*}\end{small}
%
where
\begin{equation} \label{201998920}
  \beta\leq 0,\ \ {\rm e}^{\alpha}>1+c_2^2,\ \ {\rm e}^{\gamma}>\frac{\left({\rm e}^{\alpha}+{\rm e}^{\beta}+|c_1c_2|\right)^2}{{\rm e}^{\alpha}-1-c_2^2}.
  \end{equation}
For $( {f}, {g}, {h})\in D(\mathscr{A})$, a simple computation shows that
 \begin{small}
 \begin{align}
 &{\rm Re}\langle \mathscr{A}(f,g,h),  (f,g,h) \rangle_{1}\\ \notag
 =&{\rm Re} \langle (g,f'',-\tau^{-1}h' ),  (f,g,h) \rangle_{1}
  \\=& -\frac12{\rm e}^{\alpha x}|g-f'|^2\Big{|}_0^1+\frac{\alpha}{2}\int_0^1
 {\rm e}^{\alpha x}|g-f'|^2{\rm d}x +
   \frac12{\rm e}^{\beta x}|g+f'|^2\Big{|}_0^1-\frac{\beta}{2}\int_0^1
 {\rm e}^{\beta x}|g+f'|^2{\rm d}x \notag
 \\& -\frac12{\rm e}^{\gamma x}|h|^2\Big{|}_0^1+\frac{\gamma}{2}\int_0^1
 {\rm e}^{\gamma x}|h |^2{\rm d}x \notag
  \\=&-\frac12{\rm e}^{\alpha}|g(1)-h(1)|^2+ \frac12{\rm e}^{\beta}|g(1)+h(1)|^2
 -\frac12{\rm e}^{\gamma}|h(1)|^2+ \frac12 | h(0)|^2 \notag
   \\ &+\frac{\alpha}{2}\int_0^1
 {\rm e}^{\alpha x}|g-f'|^2{\rm d}x  -\frac{\beta}{2}\int_0^1
 {\rm e}^{\beta x}|g+f'|^2{\rm d}x+\frac{\gamma}{2}\int_0^1
 {\rm e}^{\gamma x}|h |^2{\rm d}x \notag
   \\=&-\frac12{\rm e}^{\alpha}|g(1)-h(1)|^2+ \frac12{\rm e}^{\beta}|g(1)+h(1)|^2+\frac12|c_1h(1)+c_2g(1)|^2
 -\frac12{\rm e}^{\gamma}|h(1)|^2 \notag
   \\& +\frac{\alpha}{2}\int_0^1
 {\rm e}^{\alpha x}|g-f'|^2{\rm d}x  -\frac{\beta}{2}\int_0^1
 {\rm e}^{\beta x}|g+f'|^2{\rm d}x+\frac{\gamma}{2}\int_0^1
 {\rm e}^{\gamma x}|h |^2{\rm d}x. \notag
      \end{align}
\end{small}
  Thus
  \begin{small}
 \begin{align} \label{20189261005-1}
 & {\rm Re}\langle \mathscr{A}(f,g,h),  (f,g,h) \rangle_{1}
  \\ \leq & -\frac12\left({\rm e}^{\alpha}-{\rm e}^{\beta}-c_2^2\right)|g(1)|^2-
 \frac12\left({\rm e}^{\alpha}+{\rm e}^{\gamma}-{\rm e}^{\beta}-c_1^2\right)|h(1)|^2 \notag
   \\&+\left({\rm e}^{\alpha}
 +{\rm e}^{\beta}+|c_1c_2|\right)|h(1)g(1)|
    +\frac{\alpha}{2}\int_0^1{\rm e}^{\alpha x}|g-f'|^2{\rm d}x-\frac{\beta}{2}\int_0^1
 {\rm e}^{\beta x}|g+f'|^2{\rm d}x \notag
  \\&+\frac{\gamma}{2}\int_0^1
 {\rm e}^{\gamma x}|h |^2{\rm d}x. \notag
    \end{align}
\end{small}
By Young's inequality, for any $\delta>0$,
 \begin{equation} \label{2019911824}
|h(1)g(1)|\leq \frac{\delta}{2}|g(1)|^2+\frac{1}{2\delta}|h(1)|^2.
\end{equation}
Combine \eqref{20189261005-1} and \eqref{2019911824} to obtain
\begin{equation} \label{20189261005}
 \begin{array}{l}
 {\rm Re}\langle \mathscr{A}(f,g,h),  (f,g,h) \rangle_{1}
 \leq -a_1|g(1)|^2-a_2|h(1)|^2
      +\frac{\alpha}{2}\int_0^1\\
 {\rm e}^{\alpha x}|g-f'|^2{\rm d}x  -\frac{\beta}{2}\int_0^1
 {\rm e}^{\beta x}|g+f'|^2{\rm d}x+\frac{\gamma}{2}\int_0^1
 {\rm e}^{\gamma x}|h |^2{\rm d}x,
    \end{array}
\end{equation}
where
\begin{equation} \label{20189911941}
\left\{\begin{array}{l}
a_1\triangleq \frac12\left[{\rm e}^{\alpha}-{\rm e}^{\beta}-c_2^2-\delta\left({\rm e}^{\alpha}+{\rm e}^{\beta}+|c_1c_2|\right)\right],\\
a_2\triangleq  \frac12\left[{\rm e}^{\alpha}+{\rm e}^{\gamma}-{\rm e}^{\beta}-c_1^2-\frac{{\rm e}^{\alpha}+{\rm e}^{\beta}+|c_1c_2|}{\delta}\right].
\end{array}\right.
\end{equation}
Owing to  \eqref{201998920},  we  can choose  $\delta$ small enough such that
\begin{equation} \label{2019926811}
0<\delta \left({\rm e}^{\alpha}+{\rm e}^{\beta}+|c_1c_2|\right) <{\rm e}^{\alpha}-1-c_2^2.
\end{equation}
By \eqref{201998920}, it follows that
\begin{equation} \label{2019926812}
 \frac{{\rm e}^{\alpha}+{\rm e}^{\beta}+|c_1c_2|}{\delta}<{\rm e}^{\gamma}+{\rm e}^{\alpha}-1-c_1^2
 \leq {\rm e}^{\gamma}+{\rm e}^{\alpha}-{\rm e}^{\beta}-c_1^2.
\end{equation}
By \eqref{20189911941}, \eqref{2019926811} and \eqref{2019926812},
\begin{equation} \label{20189911957}
a_1>0\ \ \mbox{ and}\ \  a_2>0.
\end{equation}
Hence, it follows from \eqref{20189261005}  and   \eqref{20189911957}  that
  there is an $M  > 0 $ such that
   $\forall\  (f,g,h) \in D(\mathscr{A})$,
  \begin{equation} \label{20199111019}
 {\rm Re}\langle \mathscr{A}(f,g,h),  (f,g,h) \rangle_{1}
 \leq  -a_1|g(1)|^2-a_2|h(1)|^2 +
 M\langle (f,g,h),  (f,g,h) \rangle_{1},
\end{equation}
This implies that $\mathscr{A} - M $ is dissipative. By \eqref{20189261005}, $\mathscr{A}$ is a discrete operator
 (i.e., $\mathscr{A}^{-1}$ is
compact), so there is a sequence $M_n\to\infty$ such that $M_n\in \rho(\mathscr{A})$, the resolvent set of
$\mathscr{A}$. We may assume without loss of generality that $M \in \rho(\mathscr{A})$. By the Lumer-Phillips theorem, $\mathscr{A}- M$ generates a $C_0$-semigroup of contractions
${\rm e}^{(\mathscr{A}-M)t}$ on  $\mathcal{X}$ (\cite[Theorem 4.3, p.14]{Pazy}). The bounded perturbation theorem of $C_0$-semigroups
ensures that $\mathscr{A}$ generates a $C_0$-semigroup ${\rm e}^{\mathscr{A}t}$ on $\mathcal{X}$
(\cite[Theorem 1.1, p.76]{Pazy}).
Similarly, we apply the same argument  to get that
$-\mathscr{A}$ also generates a $C_0$-semigroup in $\mathcal{X}$ (see, e.g., \cite{Wangdelay2011}).
Therefore, $\mathscr{A}$  actually generates a $C_0$-group
on $\mathcal{X}$. This completes the proof.
~~


\subsection{Spectral analysis}
Let us now consider the eigenvalue problem of $\mathscr{A}$.





Let us now consider the eigenvalue problem of $\mathscr{A}$, $\mathscr{A}(f,g,h)=\lambda(f,g,h)$, then
  $g=\lambda f$ and
\begin{equation} \label{20189231117}
\left\{\begin{array}{l}
  f''=\lambda^2f, h'=-\tau\lambda h,\\
f(0)=0, f'(1)=h(1),\\
 h(0)=-c_1 h(1)-   c_2 \lambda f(1)  .
\end{array}\right.
\end{equation}

When $\lambda=0$, the solution of \eqref{20189231117} is found to be
\begin{equation}
\left\{\begin{array}{l}
f(x)=bx,
\\h(x)=b,
 \end{array}\right.
 \end{equation}
where the constant $b$ satisfies that $b=-c_1b$. Thus, the equation has a nontrivial solution if and only if $c_1=-1$. When $c_1=-1$, the corresponding eigenfunction
$(f_0, g_0, h_0)$ is given by
$$
f_0(x)=x,~~  g_0(x)=1, ~~  h_0(x)=1.
$$






When $\lambda\neq 0$, the solution of \eqref{20189231117} is found to be
\begin{equation} \label{20189231123}
\left\{\begin{array}{l}
  f(x)=a{\rm e}^{\lambda x}-a{\rm e}^{-\lambda x}=2a\sinh \lambda x,\\
  h(x)=b{\rm e}^{-\tau\lambda x},
 \end{array}\right.
\end{equation}
where $a$ and $b$ are constants that satisfy
\begin{equation} \label{20189231125}
\left\{\begin{array}{l}
  c_2\lambda\left({\rm e}^{\lambda}-{\rm e}^{-\lambda}\right) a
+\left(1+c_1 {\rm e}^{-\tau\lambda} \right) b=0,\\
     \lambda\left({\rm e}^{\lambda}+{\rm e}^{-\lambda}\right) a
  -{\rm e}^{-\tau\lambda} b=0.
 \end{array}\right.
\end{equation}
 The characteristic determinant of \eqref{20189231125}
is
\begin{equation} \label{20189231135}
 \begin{array}{l}
 \Delta(\lambda) =\left|\begin{array}{cc}
    c_2 \lambda\left({\rm e}^{\lambda}-{\rm e}^{-\lambda}\right)&
  1+c_1{\rm e}^{-\tau\lambda} \\
    \lambda\left({\rm e}^{\lambda}+{\rm e}^{-\lambda}\right) &
  -{\rm e}^{-\tau\lambda}
 \end{array}\right|
  =-\lambda\cosh \lambda(1+c_1{\rm e}^{-\lambda\tau})-c_2\lambda\sinh \lambda {\rm e}^{-\lambda\tau}. \end{array}
\end{equation}
Thus, the equation has a nontrival solution  if and only if
 \begin{equation} \label{20189231232}
\tilde{\Delta}(\lambda)\triangleq-\cosh \lambda(1+c_1{\rm e}^{-\lambda\tau})-c_2\sinh \lambda {\rm e}^{-\lambda\tau}=0.
\end{equation}
Furthermore, we observe that when $c_1=-1$, $\lambda=0$ is also a root for $\tilde{\Delta}(\lambda)=0$, then we conclude that
$$
\sigma({\mathscr{A}})=\sigma_p(\mathscr{A})=\{\lambda\in\mathbb{C}|\tilde{\Delta}(\lambda)=0\}.
$$
Therefore, each $\lambda\in\sigma(\mathscr{A})\backslash\{0\}$ is geometrically simple, and the corresponding eigenfunction
$(f_\lambda, g_\lambda, h_\lambda)$ is given by
$$
f_\lambda(x)={\rm e}^{-\tau\lambda}\sinh \lambda x,~~  g_\lambda(x)=\lambda{\rm e}^{-\tau\lambda}\sinh \lambda x, ~~  h_\lambda(x)=\lambda\cosh \lambda {\rm e}^{-\tau\lambda x}.
$$
For $\lambda=0, c_1=-1$, the corresponding eigenfunction
$(f_0, g_0, h_0)$ is given by
$$
f_0(x)=x,~~  g_0(x)=1, ~~  h_0(x)=1.
$$

	
For any $\lambda\in\rho(\mathscr{A})$, we have following lemma on the expression of the resolvent operator.
\begin{lemma}\label{Lemma20199101635}
Let  the operator  $\mathscr{A} $ be defined by  \eqref{20189241453}.
 %and let $\det\Delta(\lambda)$ be given by \eqref{20189231135}.
 Then,
for any $\lambda\in \rho(\mathscr{A})$ and
$Y=(f_1,g_1,h_1)\in {\mathcal X}$, $X=R(\lambda,\mathscr{A})Y$,
where $X=(f,g,h)\in D(\mathscr{A})$ is given by
\begin{equation}\label{wxh2019992018}\left\{\begin{array}{l}
f(x,\lambda)=laystyle\frac{(1+c_1{\rm e}^{-\tau\lambda})F_1+F_2{\rm e}^{-\tau\lambda}}
{ \Delta(\lambda)}\sinh\lambda x+F_0(x,\lambda),\\
g(x,\lambda)=\lambda f(x,\lambda)-f_1(x),\\
h(x,\lambda)=laystyle\frac{F_2\lambda \cosh\lambda-F_1c_2\lambda\sinh\lambda}{ \Delta(\lambda)}{\rm e}^{-\tau\lambda x}+H_0(x,\lambda),
\end{array}\right.\end{equation}
and
\begin{equation}\label{wxh2019992059}\left\{\begin{array}{l}
  \Delta(\lambda)=\lambda\cosh \lambda(1+c_1{\rm e}^{-\lambda\tau})+c_2\lambda\sinh \lambda {\rm e}^{-\lambda\tau},\\
F_0(x,\lambda)=laystyle
\int_{0}^{x}\lambda^{-1}\sinh\lambda(x-s)[-\lambda f_1(s)-g_1(s)]{\rm d}s,\\
H_0(x,\lambda)=laystyle\int_{0}^{x}{\rm e}^{-\tau\lambda(x-s)}\tau h_1(s){\rm d}s,\\
F_1=laystyle\int_{0}^{1}\cosh\lambda(1-s)[\lambda f_1(s)+g_1(s)]{\rm d}s+
\int_{0}^{1}{\rm e}^{-\tau\lambda(1-s)}\tau h_1(s){\rm d}s,\\
F_2=laystyle\int_{0}^{1} c\sinh\lambda(1-s)[\lambda f_1(s)+g_1(s)]{\rm d}s-
\int_{0}^{1}c{\rm e}^{-\tau\lambda(1-s)}\tau h_1(s){\rm d}s.
\end{array}\right.\end{equation}
\end{lemma}
Proof:
For any $Y=(f_1,g_1,h_1)\in{\mathcal X}$ and  $\lambda\in\rho(\mathscr{A})$,
let
\begin{equation}\label{wxh2019992020}
X=R(\lambda,\mathscr{A})Y, \ X=(f,g,h)\in D(\mathscr{A}).
\end{equation}
Then,
\begin{equation}\label{wxh2019992022}
(\lambda I-\mathscr{A})X=(\lambda f-g,\lambda g-f^{\prime\prime},\lambda h+\tau^{-1}h')=(f_1,g_1,h_1).
\end{equation}
Hence,  $g=\lambda f-f_1$ with $f,h$ satisfying
\begin{equation}\label{wxh2019992023}
\left\{\begin{array}{l}
f^{\prime\prime}(x)-\lambda^2 f(x)=-g_1-\lambda f_1,\\
f'(1)=h(1),\\
h^{\prime}(x)+\tau\lambda h=\tau h_1,\\
h(0)=-c_1h(1)-c_2\lambda f(1),
\end{array}\right.
\end{equation}
which gives
\begin{equation}\label{wxh2019002045}
\left\{\begin{array}{l}
f(x,\lambda)=a\sinh\lambda x+F_0(x,\lambda),\\
h(x,\lambda)=b {\rm e}^{-\tau\lambda x}+H_0(x,\lambda),
\end{array}\right.
\end{equation}
where $F_0(x,\lambda)$ and $H_0(x,\lambda)$ are given by
\eqref{wxh2019992059}.
By the boundary conditions of
\eqref{wxh2019992023}, we have
%$a,b$ satisfy
%the following algebraic equations:
\begin{equation}\label{wxh2019992208}\left\{\begin{array}{l}
a\lambda\cosh\lambda-b{\rm e}^{-\tau\lambda}=F_1,\\
ac_2\lambda\sinh\lambda+b(1+c_1{\rm e}^{-\tau\lambda})=F_2,
\end{array}\right.a,b\in\mathbb{R},
\end{equation}
where $F_1$ and $F_2$ are given by
\eqref{wxh2019992059}. Now, we determine the constants $a$ and $b$.
Since   $\lambda\in\rho(\mathscr{A})$ and $\Delta(\lambda) $ happens to be the characteristic determinant
of \eqref{wxh2019992208}, it follows that $ \Delta(\lambda)\neq 0$. Therefore,
 $a$ and $b$ can be determined by solving
 equation \eqref{wxh2019992208}.
Moreover, the solution $X$ of \eqref{wxh2019992020} can
be written in \eqref{wxh2019992018}.



$$
\Delta(\lambda)={\rm e}^{-\lambda}+k{\rm e}^{-(1+\tau)\lambda}+{\rm e}^{\lambda}-k{\rm e}^{(1-\tau)\lambda}
$$



Enlightened by \cite[Proposition 3.3]{Wangdelay2011}, we characterize the spectrum of $\mathscr{A}$ as follows.
\begin{theorem}\label{20232101439}
Let $\mathscr{A} $  defined by  \eqref{20189241453} and $\tilde{\Delta}(\lambda)$ defined by \eqref{20189231232}. The following assertions hold for the spectrum of $\mathscr{A}$:
\\(i) There is an $M>0$ such that for all $\lambda\in\rho(\mathscr{A})$, $|{\rm Re}\lambda|<M$; that is, all the eigenvalues of $\mathscr{A}$ lies in some vertical strip paraller to the imaginary axis in the complex plane.
\\(ii)The multiplicity of each root of $\tilde{\Delta}(\lambda)=0$ is at most two.
\\(iii) If $\tau$ is rational, then the eigenvalue of $\mathscr{A}$ are located on finitely many lines parallel to the imaginary axis.
\\(iv)If $\tau$ is irrational, then all roots of $\tilde{\Delta}(\lambda)=0$ are simple.
\\(v)The eigenvalues of $\mathscr{A}$ are separated, that is
$$
\inf_{\lambda_m,\lambda_n\in\sigma(\mathscr{A}),\lambda_m\neq\lambda_n}|\lambda_m-\lambda_n|>0.
$$
\\(vi) The algebric multiplicity of each eigenvalue of $\mathscr{A}$ is at most two.
\end{theorem}
The proof of Theorem \ref{20232101439} is similar to  that of \cite[Proposition 3.3]{Wangdelay2011}, so we omit it.







\begin{lemma}\label{Lm20199101733}
 Let  $\mathscr{A} $       be given by \eqref{20189241453}. Then, the root subspace of $\mathscr{A} $  is complete
in $\mathcal{X}$, that is, ${\rm Sp}( \mathscr{A}   ) = \mathcal{X}$, where ${\rm Sp}( \mathscr{A}   ) $ denotes the root subspace of  $ \mathscr{A}  $  spanned by the
generalized eigenfunctions of $ \mathscr{A}  $.
 \end{lemma}
Proof:
From Lemma \ref{Lemma20199101635}, $X = R(\lambda,\mathscr{A} )Y$ can be further represented as
$$
 X= R(\lambda,\mathscr{A} )Y=\frac{G(\lambda,Y )}{\Delta(\lambda)},
$$
where $G(\lambda,Y )$ is an  $\mathcal{X}$-valued entire function with order less than or equal to 1, and
by \eqref{20189231135}, $\Delta(\lambda)$ is a scalar entire function of order 1. Since  from  Theorem \ref{wellposedness}, $\mathscr{A}$ generates a $C_0$-group on $\mathcal{X}$,
$|R(\lambda,\mathscr{A})|$ is uniformly bounded as ${\rm Re}\lambda\to \pm\infty$.
 By \cite[Theorem 4.1]{Wangdelay2011} or \cite[Theorem 4]{XuGQ2003}, ${\rm Sp}( \mathscr{A}   ) = \mathcal{X}$. This completes the
  proof of the lemma.
 %with ρ = 1, n = 2, γ1 = {λ| arg λ = π}.

	

\begin{lemma}\label{Lm20199101844}
Let  the operator  $\mathscr{A} $ be defined by  \eqref{20189241453}.
Then,  the  spectrum determined growth condition holds for $\mathscr{A} $: $s(\mathscr{A} ) = \omega(\mathscr{A} )$,
where $s(\mathscr{A} )$ and  $\omega(\mathscr{A} )$ are the spectral bound of $\mathscr{A}$ and the
the growth order of ${\rm e}^{\mathscr{A}t}$, respectively.
\end{lemma}
Proof
  By \eqref{20189261005}, $\mathscr{A}$ is a discrete operator. Suppose that
  $\{\lambda_n\}_{n=1}^{\infty}$ is the set of the eigenvalues of $\mathscr{A}$.
 Then,  $\tilde{\Delta}(\lambda_n)=0$, where the function $\tilde{\Delta}$ is defined by \eqref{20189231232}.
  It is evident that $\tilde{\Delta}(\lambda) $ is   an entire function of
exponential type. Moreover, by Theorem \ref{20232101439}   and the fact that
$|\tilde{\Delta}(\lambda)| \to\infty$ as
${\rm Re}\lambda \to \pm\infty$, it follows that $\tilde{\Delta}(\lambda) $ is a sine-type function.
  By \cite[Theorem 1]{XuGQ2003} or \cite{PSulian1979},
   $\{{\rm e}^{\lambda_nt}\}_{n=1}^{\infty}$ forms a Riesz basis for $L^2(0, T)$ for some $T > 0$.
Combining Theorem \ref{20232101439}-(v), Lemma \ref{Lm20199101733}  and
\cite[Theorem 4.3]{Wangdelay2011}, the spectrum-determined growth condition holds for $\mathscr{A} $.


A necessary and sufficient condition that the polynomial 
$$
F(\lambda)=a_m\lambda^m+a_{m-1}\lambda^{m-1}+\cdots+a_1\lambda+a_0, a_m>0,
$$
with real coefficient have all of its roots inside the unit circle is given by
$$
F(1)>0,~~~~~ (-1)^{m}F(-1)>0,
$$
and the $(m-1)\times (m-1)$ Jury matrices
$$
\Delta^{\pm}_{m-1}=\left(\begin{array}{lllll}
   a_m &0&0&\cdots&0
      \\  a_{m-1} & a_m & 0 &\cdots &0
      \\ a_{m-2} & a_{m-1} & a_m &\cdots & 0
      \\ \vdots & \vdots & \vdots & \vdots & \vdots
      \\ a_2 & a_3 & a_4 &\cdots &a_m
\end{array}\right)\pm \left(\begin{array}{lllll}
   0 &0&\cdots&0&a_0
      \\  0 &0 & \cdots &a_0 & a_1
      \\ \vdots & \vdots& \vdots &\vdots& \vdots
      \\ 0& a_0 & \cdots & a_{m-4} & a_{m-3}
      \\ a_0 & a_1 & \cdots &a_{m-3} &a_{m-2}
\end{array}\right)$$
are both positive innerwise; that is, the determinants of all the inners of $\Delta^{\pm}_{m-1}$ are positive. Here, the inners of a square matrix are the matrix itself and all the
matrices obtained by omitting successively the first and last rows and the first and last
columns.






\section{Exponential stability of system \eqref{20189202158}}\label{section3}
In this section, we discuss the exponential stability of the system \eqref{20189202158}. By Lemma \ref{Lm20199101844}, we need only to check whether all the eigenvalues of $\mathscr{A}$ is located in $\mathbb{C}_{-}$.  By $\tilde{\Delta}(\lambda)=0$, we obtain
\begin{equation}\label{cha}
\tilde{\Delta}(\lambda)\triangleq-\cosh \lambda(1+c_1{\rm e}^{-\lambda\tau})-c_2\sinh \lambda {\rm e}^{-\lambda\tau}=0.
\end{equation}
For the sake of simplicity, we consider the case $c_1=c_2=c\in\mathbb{R}$ in this section. The stability region on the whole $c_1-c_2$ plane will be investigated in our future work.
Thus, Eq. \eqref{cha} becomes
\begin{equation} \label{202302042155}
{\rm e}^{2\lambda}+2c {\rm e}^{(2-\tau)\lambda}+ 1=0.
\end{equation}

\subsection{$\tau>0$ is rational}
First, we talk about the situation that $\tau>0$ is rational.
Eq. \eqref{202302042155} can be written as
$
c=-\dfrac{1}{2}[{\rm e}^{\tau\lambda}+{\rm e}^{(\tau-2)\lambda}].
$

If $\tau=1$, suppose that $z={\rm e}^{{\lambda}}$, the Eq. \eqref{202302042155} becomes $z^2+2cz+1=0$.  No matter what the value of $c$ is, this equation
 has at least one root satisfying  $|z|\geq 1$, which indicates that Eq. \eqref{202302042155} has at least one root located in $\mathbb{C}_0\cup\mathbb{C}_{+}$.
$$
\tau=\dfrac{m}{n}$$

$$z={\rm e}^{-{\lambda}/{n}}$$
 
$$
z^n+kz^{m+n}+z^{-n}-kz^{m-n}=0
$$
no roots are inside the unit circle.

The characteristic equation \eqref{202302042155} has no complex root located in $\mathbb{C}_0\cup\mathbb{C}_{+}$ is equivalent to that the equation $f(z)=c$ has no root which is located in the unit circle $\overline{\mathbb{D}}$.

Denote by the set
$$
\mathscr{E}=\left\{k\in\mathbb{R} \Bigg| \mbox{there exists}~z\in\mathbb{C}, |z|=1~~\mbox{such that}~ f(z,k)=0 \right\}.
$$
For $k\notin\mathscr{E}_{m,n}$, we denote by a function $N(k)$, which is the number of root, counted by multiplicity, of the equation $f(z,k)=0$ located in the unit circle $\mathbb{D}$.

In order to study the stability region, we  find the region for $c$ in which $N(c)=0$. We  give an outline of our following dissertation.  $N(c)$ is the number of root branches for different values of $c$.  Each time when $c$ moves across some critical value $c\in\mathscr{E}$, a root branch disappear or emerge at the boundary of unit circle. We use implicit theorem to investigate whether a root branch disappears or emerges. When $\tau<1$, we move $c$ from $\pm\infty$ to $0$. When $c$ is at $\pm\infty$, we prove that there are $m$ branches of root by argument principle (Lemma \ref{lemma6}). Each time when $c$ comes arcoss a critical value $c$, we prove that a new branch emerges  from the boundary of the unit circle. As a result, we prove that there is no stability region. When $\tau>1$, we move $c$ from $0$ to $\pm\infty$. We prove that each time when $c$ comes arcoss a critical value $c\neq 0$, a new branch emeges from the boundary of the unit circle. Thus, we only need to investigate the root branch near $c=0$. When $c=0$, there are several roots on the unit circle. We prove that only when $\tau$ is an even number, there is a region for $c$ such that all the roots leave the unit circle and therefore there is a stability region for $c$.






According to Lemma \ref{lemma6}, we get that for an interval $[c_1,c_2]$, if $[c_1,c_2]\cap \mathscr{E}=\emptyset$, $N(c)$ is a constant. And when $|c|>1$, $N(c)\equiv m$.


Now we turn to investigate  the relationship between $\lim_{c\to c_*-}N(c)$ and  $\lim_{c\to c_*+}N(c)$ for any $c_*\in\mathscr{E}$. We will discuss into following categories. Case A: $\tau<1$. Case B: $\tau>1$.

Case A: $\tau<1$. According to Lemma \ref{lemma2}, for each $c_*=f(z_*)\in\mathscr{E}\backslash\{0\}, |z_*|=1$, there exists an implicit function $z(c)$ such that $z(c_*)=z_*$ and $c=f(z(c))$ for each $c\in(c_*-\epsilon, c_*+\epsilon)$. Here, $\epsilon>0$ is efficiently small. Suppose that $z(c)=r(c){\rm e}^{{\rm i}\theta(c)}$, where $r(c)$ and $\theta(c)$ represent the absolute value and argument value of the function $z(c)$, respectively. Then we have
$$
{\rm Sgn}[r'(c_*)]={\rm Sgn}(n-m){\rm Sgn}(c_*)={\rm Sgn}(c_*).
$$
When $c_*>0, r'(c_*)>0$. We obtain that $|z(c)|>1$ for $c>c_*$ and $|z(c)|<1$ for $c<c_*$. This indicates that when $c$ goes from $c_*-$ to $c_*+$, a root for the equation $f(z)=c$ enters the unit circle from the outside. Thus we obtain that $\lim_{c\to c_*-}N(c)>\lim_{c\to c_*+}N(c)$ for $c>0$. Similarly, we obtain that when $c_*<0$, $\lim_{c\to c_*-}N(c)<\lim_{c\to c_*+}N(c)$. Therefore, $N(c)$ is monotonically increasing on $\mathbb{R}_{-}\backslash\mathscr{E}$ and monotonically decreasing on $\mathbb{R}_+\backslash\mathscr{E}$. Since $N(c)=m$ when $|c|>1$, $N(c)$ is not zero on $\mathbb{R}\backslash\mathscr{E}$.


Case B: $\tau>1$. Similarly, we get that $N(c)$ is monotonically decreasing on $\mathbb{R}_{-}\backslash\mathscr{E}$ and monotonically increasing on $\mathbb{R}_+\backslash\mathscr{E}$. We try to find the region for $c$ in which $N(c)=0$. Thus, we need to investigate $N(c)$ near $c=0$.

When $c=0$, $f(z)=0$ has $2n$ roots which read
$$
z_k={\rm e}^{\rm i\theta_k}, \theta_k=\dfrac{(2k+1)\pi}{2n} ,k=0,1,2,\cdots, 2n-1.
$$
There exist $2n$ different implicit functions $z_k(c)$ such that $z_k(0)=z_k={\rm e}^{\rm i \theta_k}$ with $\theta_k=\dfrac{(2k+1)\pi}{2n}$, $k=0,1,2,\cdots, 2n-1$. According to Lemma \ref{lemma2}, we have ${\rm Sgn}[r_k'(0)]={\rm Sgn}\cos[\dfrac{m(2k+1)\pi}{2n}]$. We discuss into following categories. Case I: $n>1$. Case  II: $n=1, m\geq 3$ is an odd number. Case  III: $n=1, m=4s-2, s\in\mathbb{N}^*$. Case IV: $n=1, m=4s, s\in\mathbb{N}^*$.

Case  I: $n>1$.  Lemma \ref{lemma3} shows that there exist at least two integer numbers $k,j$ such that $r_k'(0)>0$ and $r_j'(0)<0$. This implies that there exists small efficiently $\delta>0$, $|z_k(c)|<1$ for $c\in(-\delta,0)$ and  $|z_j(c)|<1$  for   $c\in(0,\delta)$. Thus $\lim_{c\to 0+}N(c)$ and $\lim_{c\to 0-}N(c)$ are both nonzero.  As a result, $N(c)$ is nonzero on $\mathbb{R}\backslash\mathscr{E}$.


Case II: $n=1$, $m\geq 3$ is an odd number. According to Lemma \ref{lemma5}, we obtain that $r_k'(0)=0$ and $r_k''(0)<0$. By Taylor expression
$
r_k(c)=r_k(0)+r_k'(0)c+\dfrac{1}{2}r_k''(0)c^2+o(c^2),
$
we obtain that $|z_k(c)|<1$ for $c$ in a neighborhood of $0$. This implies that $N(c)$ is nonzero in the neighborhood of $0$. Thus, $N(c)$ is nonzero on $\mathbb{R}\backslash\mathscr{E}$.


Case III: $n=1$, $m=4s-2, s\in\mathbb{N}^*$. A direct computation leads to that $r_k'(0)<0$ for both $k=0,1$. This indicates that $|z_k(c)|<1$ when $c\to 0+$ and $|z_k(c)|>1$ when $c\to 0-$. This further indicates that $N(c)$ is nonzero for $c\to 0+$ and $N(c)$ is zero when $c\to 0-$.  Lemma \ref{lemma4} tells us the nearest element    in $\mathscr{E}\cap\mathbb{R}_{-}$      to $0$  is  $-\sin [\dfrac{\pi}{2(m-1)}]$. Therefore, we obtain that $N(c)=0$ on the interval $(-\sin [\dfrac{\pi}{2(m-1)}],0).$

Case IV: $n=1$, $m=4s, s\in\mathbb{N}^*$. Using the same argument, we get that $N(c)=0$ on the interval $(0,\sin [\dfrac{\pi}{2(m-1)}])$.

	
	
	
\subsection{$\tau>0$ is irrational}\label{subsection32}
In this subsection, we prove that for any irrational $\tau>0$ and $c\in\mathbb{R}$, Eq. \eqref{202302042155} has at least one root located in $\mathbb{C}_0\cup\mathbb{C}_+$. The idea of the proof is similar to that in the last subsection. In this case, Eq. \eqref{202302042155} with respect to $\lambda$ has no periodicity. Therefore, we consider the equation in the complex plane directly. Denote by
$$g(\lambda)\triangleq-\dfrac{1}{2}[{\rm e}^{\tau\lambda}+{\rm e}^{(\tau-2)\lambda}].$$
For $a,b\in\mathbb{Z}\backslash\{0\}, a<b$, denote by the set
$$
\mathscr{C}_{a,b}\triangleq\left\{g(\lambda) \Bigg| \lambda\in\mathbb{C}_0, a\pi\leq{\rm Im}\lambda\leq b \pi \right\} \cap \mathbb{R}.
$$


For $c\notin\mathscr{C}_{a,b}$, we denote by a function $M_{a,b}(c)$, which is the number of root, counted by multiplicity, of the equation $g(\lambda)=c$ located in the vertical strip $\left\{\lambda \Bigg| \lambda\in\mathbb{C}_+, a\pi\leq{\rm Im}\lambda\leq b \pi \right\}$ (See Fig. \ref{fig16}).

% Figure environment removed

We only need to prove that there exist a pair of $a,b\in\mathbb{Z}\backslash\{0\}, a<b$ such that $M_{a,b}(c)$ is not zero for all $c\in\mathbb{R}$. The idea of the proof is quite similar to that in the last subsection. $M_{a,b}(c)$ is the number of root branches for different values of $c$. We will prove that $g(\lambda)\notin\mathbb{R}$ when ${\rm Im}\lambda=a\pi, b\pi$ for $a,b\in\mathbb{Z}\backslash{0}$. This indicates there is no root branch disappear or emerge at the boundary $\Big\{\lambda \Big|{\rm Im}\lambda=a\pi, b\pi\Big\}$. Each time when $c$ moves across some critical value $c\in\mathscr{C}_{a,b}$, a root branch disappear or emerge at the imaginary axis. We use implicit theorem to investigate whether a root branch disappears of emerges. When $\tau<1$, we move $c$ from $\pm\infty$ to $0$. When $c$ is at $\pm\infty$, we prove that there are at least one root branch by Lemma \ref{lemma8}. Each time when $c$ comes arcoss a critical value $c$, we prove that a new branch emerges  from the imaginary axis. Then we prove that there is no stability region. When $\tau>1$, we move $c$ from $0$ to $\pm\infty$. We prove that each time when $c$ comes arcoss a critical value $c\neq 0$, a new branch emeges from the imaginary axis. Thus, we only need to investigate the root branch near $c=0$. We prove that no matter $c$ moves from $0$ to $0+$ or $0-$, there is at least one root branch emerging and thus prove that there is no stability region for $c$.





Case A: $\tau<1$. According to Lemma \ref{lemma11},
we get that
$
{\rm Sgn}[{\rm Re}\lambda'(c_*)]={\rm Sgn}(\tau-1){\rm Sgn}(c_*)=-{\rm Sgn}(c_*).
$
Similar to the argument in last subsection, we obtain that $M_{a,b}(c)$ is monotonically increasing on $\mathbb{R}_{-}\backslash\mathscr{C}_{a,b}$ and monotonically decreasing on $\mathbb{R}_+\backslash\mathscr{C}_{a,b}$.  Lemma \ref{lemma8} shows that if we take $(b-a)\tau>2$, $M_{a,b}(c)$ is a nonzero constant when $|c|>1$. Therefore, $M_{a,b}(c)$ is not zero on $\mathbb{R}\backslash\mathscr{C}_{a,b}$.


Case B: $\tau>1$. We obtain that  $M_{a,b}(c)$ is monotonically increasing on $\mathbb{R}_{+}\backslash\mathscr{C}_{a,b}$ and monotonically decreasing on $\mathbb{R}_{-}\backslash\mathscr{C}_{a,b}$. We need to investigate $M_{a,b}(c)$ near $c=0$.  According to Lemma \ref{lemma11}, there exist $b-a-1$ different implicit functions $\lambda_k(c)$ such that $c=g(\lambda_k(c)), \lambda_k(0)=\lambda_k={\rm i}(k+\dfrac{1}{2})\pi$ , $a+1\leq k\leq b-1, k\in\mathbb{Z}$.  Furthermore,
$
{\rm Sgn}[{\rm Re}\lambda_k'(c_*)]=-{\rm Sgn}\cos[\tau(k+\dfrac{1}{2})\pi].
$
According to Lemma \ref{lemma12}, there exists $k,j\in\mathbb{N}^*$ such that
$\cos[\tau(j+\dfrac{1}{2})\pi]>0, \cos[\tau(l+\dfrac{1}{2})\pi]<0.$
Therefore, we take $a,b\in\mathbb{Z}\backslash\{0\}$ such that $a\leq\min\{l,j\}-1, b\geq \max\{l,j\}+1$. Then we get that
$
{\rm Sgn}[{\rm Re}\lambda_j'(c_*)]<0, {\rm Sgn}[{\rm Re}\lambda_l'(c_*)]>0.
$
Therefore, we get that $\lambda_l(c)\in\mathbb{C}_+$ for $c\in(0,\epsilon)$ and  $\lambda_j(c)\in\mathbb{C}_+$ for $c\in(-\epsilon,0)$. This leads to that $M_{a,b}(c)$ is not zero for $c\to 0+$ and $c\to 0-$. Thus, $M_{a,b}(c)$ is nonzero on $\mathbb{R}\backslash\mathscr{C}$.




From above all, the sufficient and necessary condition for the stability region for the parameter $c, \tau$ can be summarized as follows.
%\begin{small}\begin{equation}\label{condition}
%\mbox{For}~~\tau=4s-2, s\in \mathbb{N}^*,  c\in(-\sin [\dfrac{\pi}{2(\tau-1)}],0).~~\mbox{For}~~\tau=4s, s\in\mathbb{N}^*,  c\in(0,\sin [\dfrac{\pi}{2(\tau-1)}]).
%\end{equation}
%\end{small}
\begin{equation} \label{condition}
\left\{\begin{array}{l}
   c\in(-\sin [\dfrac{\pi}{2(\tau-1)}],0), \quad ~\mbox{for}~~\tau=4l-2, l\in \mathbb{N}^*,
      \\  c\in(0,\sin [\dfrac{\pi}{2(\tau-1)}]), \quad ~\mbox{for}~~\tau=4l, l\in\mathbb{N}^*.
\end{array}\right.
\end{equation}







It follows from Lemma \ref{Lm20199101844} that we get the following theorem.
\begin{theorem}\label{Th20189231016}
Suppose that $c $ and $\tau$  satisfy condition \eqref{condition}. Then, for any  initial state $(z( \cdot, 0),z_t(\cdot,0), w(\cdot, 0) )\in  \mathcal{X}$,
the closed-loop  system  \eqref{20189202158} with $c_1=c_2=c$
 admits a unique solution
$(z( \cdot, t),z_t(\cdot,t), w(\cdot, t) )\in C([0,\infty);\mathcal{X})$ which satisfies
 \begin{equation}\label{20189231017}
 \begin{array}{l}
 \|(z( \cdot, t),z_t(\cdot,t), w(\cdot, t) )\|_{ \mathcal{X} }\leq L_1{\rm e}^{-\omega_1  t}  \|(z( \cdot, 0),z_t(\cdot,0), w(\cdot, 0) )\|_{ \mathcal{X} } ,
\end{array}
\end{equation}
where    $L_1 $ and $\omega_1 $ are positive constants  independent of time and initial state.
\end{theorem}

\begin{remark}\label{remarkwangdelay2011}
We talk about the situation when $c_1=c_2=c$. If we take $c_1=0$, then system  \eqref{2018923949}  becomes \eqref{20199111503}, which has been fully investigated in \cite{Wangdelay2011}. In \cite{Wangdelay2011}, authors proved that when $\tau$ is an even number, there is not empty stability region for $c_2\in\mathbb{R}$. However, they did not prove the necessity of it and  could not provide the general formula for the stability region of $c_2$ for different values of $\tau$. They did not talk about the situation in which $\tau$ is irrational. If we use the same method employed in this section, we could prove that if and only if $\tau>0$ is an even number, there is  a stability region for $c_2$. Moreover, when $\tau$ is even number, the stability region can be summarized as follows.
\begin{equation} \label{condition2}
\left\{\begin{array}{l}
   c_2\in(-\tan (\frac{\pi}{2\tau}),0), \quad ~\mbox{for}~~\tau=4l-2, l\in \mathbb{N}^*,
      \\  c_2\in(0,\tan (\frac{\pi}{2\tau})), \quad ~\mbox{for}~~\tau=4l, l\in\mathbb{N}^*.
\end{array}\right.
\end{equation}
Obviously, the shrink of the stability region as $\tau$ is increasing can be explicitly obtained by $$\lim_{\tau\to+\infty}\tan(\dfrac{\pi}{2\tau})=0,$$
{which improves the results in \cite[Section 6]{Wangdelay2011}.  }
\end{remark}


\begin{remark}
For the situation $\tau>0$ is irrational, we can get the conclusion that \eqref{cha} has unstable roots from  \cite[Page 287, Page 288, Eq. (6.11)]{Hale}, which can be described as follows.

If $r_1,r_2>0$ are rationally independent, the sufficient and necessary condition for all the roots of the characteristic equation $1=a_1{\rm e}^{-\lambda r_1}+a_2{\rm e}^{-\lambda r_2}+a_3{\rm e}^{-\lambda (r_1+r_2)}$ lie in $\mathbb{C}_{-}$ is $1+a_1>|a_2+a_3|,1-a_1>|a_2-a_3|$.

Note that the characteristic equation \eqref{cha} can be written as $$1=-{\rm e}^{-2\lambda}-(c_1+c_2){\rm e}^{-\tau\lambda}-(c_1-c_2){\rm e}^{-(2+\tau)\lambda}.$$ 
By taking $a_1=-1, a_2=-(c_1+c_2), a_3=-(c_1-c_2)$ into  $1+a_1>|a_2+a_3|,1-a_1>|a_2-a_3|$, we easily get that the stable region for $c_1,c_2$ is empty.

\end{remark}




\section{Robustness to a small perturbation in time delay in low frequencies}\label{section4.5}



Enlightened by \cite{Datko86,Datko88,Datko93,Wangdelay2011}, we know that the feedback loop is not robust to a small perturbation in time delay. Authors in those literatures demonstrate the lack of robustness by giving exact expressions for eigenvalues for a special sequence of delay perturbations (See \cite[Theorem 7.2]{Wangdelay2011}, \cite[Lemma 2]{Datko86}) and thus in \cite[Page 5, Remark]{Datko86}, the author guessed that a small perturbation of $\epsilon$ in time delay will excite a high frequency mode~(i.e., a mode with frequency $\approx\mathcal{O}(\dfrac{1}{ |\epsilon| })$ as $\epsilon\to0$).  In this section, we will verify this judgement by spectral analysis for system \eqref{20189202158}. For the sake of simplicity, we only consider the situation $c_1=c_2=c$.


Firstly, we discuss about the robustness for $\tau=0$. When $\tau=0$, the stability region for $c$ is $(-\infty,-1)\cup(0,+\infty)$. For the sake of simplicity, we consider the robustness when $c>0$.

\begin{theorem}\label{robust4}
 Consider Eq.~\eqref{202302042155} when  $\tau_\epsilon=\epsilon~(\epsilon>0)$ and $c>0$. Denote by $\lambda_\epsilon\triangleq\inf\{|{\rm Im}\lambda|\Big|\lambda~\mbox{is a root of }~Eq.~\eqref{202302042155}~\mbox{located in}~\mathbb{C}_+\cup\mathbb{C}_0\}$, then $\lambda_\epsilon=\mathcal{O}(\dfrac{1}{\epsilon})$ as $\epsilon\to 0+$. This implies that there exists a positive constant $C_1$ independent of $\epsilon$, such that Eq. \eqref{202302042155} has no roots located in $\Big\{\lambda\in\mathbb{C}_0\cup\mathbb{C}_+\Big||{\rm Im}\lambda|<\dfrac{C_1}{\epsilon}\Big\}$ provided $\epsilon$ is sufficiently small.
 \end{theorem}
Proof We will prove that there exists two positive constants $C_1,C_2$~(independent of $\epsilon$) such that Eq. \eqref{202302042155} has no roots located in $\Big\{\lambda\in\mathbb{C}_0\cup\mathbb{C}_+\Big||{\rm Im}\lambda|<\dfrac{C_1}{\epsilon}\Big\}$ while Eq. \eqref{202302042155} has at least one root located in $\Big\{\lambda\in\mathbb{C}_0\cup\mathbb{C}_+\Big||{\rm Im}\lambda|<\dfrac{C_2}{\epsilon}\Big\}$. The proof will be divided into two parts. For the first part, we prove that Eq. \eqref{202302042155} has no roots located in $\Big\{\lambda\in\mathbb{C}_0\cup\mathbb{C}_+\Big||{\rm Im}\lambda|<\dfrac{C_1}{\epsilon}\Big\}$. For the second part, we prove that Eq. \eqref{202302042155} has at least one root located in $\Big\{\lambda\in\mathbb{C}_0\cup\mathbb{C}_+\Big||{\rm Im}\lambda|<\dfrac{C_2}{\epsilon}\Big\}$.

For the first part. We choose $C_1=\dfrac{\pi}{2}$.  Suppose that $\hat{\lambda}_\epsilon=p+{\rm i}q, p\geq 0, |q|<\dfrac{\pi}{2\epsilon}$ is a root of \eqref{202302042155} when $\tau=\epsilon, c>0$. We write \eqref{202302042155} as
$${\rm e}^{2\hat{\lambda}_\epsilon}(1+2c{\rm e}^{-\epsilon\hat{\lambda}_\epsilon})=-1,$$which, by taking absolute value of both sides, leads to
\begin{equation}\label{202302151106}
{\rm e}^{4p}(1+4c^2{\rm e}^{-2\epsilon p}+4c{\rm e}^{-\epsilon p}\cos {\epsilon q})=1.
\end{equation}
Since $|q|<\dfrac{\pi}{2\epsilon}$, $\cos{\epsilon q}>0$. Then we have
$1+4c^2{\rm e}^{-2\tau p}+4c{\rm e}^{-\tau p}\cos {\epsilon q}>1$, and ${\rm e}^{4b}\geq 1$. This contradicts \eqref{202302151106} and completes the proof of the first part.

For the second part, we choose $C_2>\pi$. Denote by $S_\epsilon$ as the smallest integer number such that $S_\epsilon>\dfrac{1}{\epsilon}$. Since $\tau=\epsilon<1$, by the proof in subsection \ref{subsection32}~(when $(b-a)\tau>2$, $M_{a,b}(c)$ is nonzero), we know that $M_{-S_\epsilon,S_\epsilon}(c)$~(defined in \ref{subsection32}) is nonzero when $\tau$ is irrational. This implies that Eq. $\eqref{202302042155}$ has at least one root located in $\Big\{\lambda\in\mathbb{C}_0\cup\mathbb{C}_+\Big||{\rm Im}\lambda|<S_\epsilon\pi\Big\}\subseteq \Big\{\lambda\in\mathbb{C}_0\cup\mathbb{C}_+\Big||{\rm Im}\lambda|<\dfrac{C_2}{\epsilon}\Big\}$. For rational $\tau=\epsilon>0$ efficiently small, we can use the similar idea of Lemma \ref{lemma7},\ref{lemma8} and \ref{lemma11} to prove that Eq. $\eqref{202302042155}$ has at least one root located in $\Big\{\lambda\in\mathbb{C}_0\cup\mathbb{C}_+\Big||{\rm Im}\lambda|<S_\epsilon\pi\Big\}$. We put the details of the proof in Appendix, Lemma \ref{lemma16}. This completes the proof.



Secondly, we discuss about the robustness for $\tau=2l, l\in\mathbb{N}^*$.

\begin{theorem}\label{robust3}
%Suppose that $\tau^*=2n,n\in\mathbb{N}^*,c^*\in\mathbb{R}$ satisfy Condition \eqref{condition}.
Consider Eq.~\eqref{202302042155} when $\tau_{\epsilon}=2l+\epsilon~(\epsilon\in\mathbb{R},l\in\mathbb{N}^*)$ and $c$ for which $2l,c$ satisfy Condition \eqref{condition}. Denote by $\lambda_\epsilon\triangleq\inf\{|{\rm Im}\lambda|\Big|\lambda~\mbox{is a root of }~Eq.~\eqref{202302042155}~\mbox{located in}~\mathbb{C}_+\cup\mathbb{C}_0\}$, then $\lambda_\epsilon=\mathcal{O}(\dfrac{1}{|\epsilon|})$ as $|\epsilon|\to 0$. This implies that there exsits a positive constant $C_1$ independent of $\epsilon$, such that Eq. \eqref{202302042155} has no roots located in $\Big\{\lambda\in\mathbb{C}_0\cup\mathbb{C}_+\Big||{\rm Im}\lambda|<\dfrac{C_1}{|\epsilon|}\Big\}$ provided $|\epsilon|$ is sufficiently small.

% Then there exist two positive constants  $C_1=C_1(\tau^*,c^*), C_2=C_2(\tau^*,c^*)$ independent of $\epsilon$, such that
%Eq. \eqref{202302042155} has no roots located in $\Big\{\lambda\in\mathbb{C}_0\cup\mathbb{C}_+\Big||{\rm Im}\lambda|<\dfrac{C_1}{|\epsilon|}\Big\}$, %but has at least one root located in $\Big\{\lambda\in\mathbb{C}_0\cup\mathbb{C}_+\Big||{\rm Im}\lambda|<\dfrac{C_2}{|\epsilon|}\Big\}$.
\end{theorem}

Proof
We will prove that there exists two positive constants $C_1,C_2$~(independent of $\epsilon$) such that Eq. \eqref{202302042155} has no roots located in $\Big\{\lambda\in\mathbb{C}_0\cup\mathbb{C}_+\Big||{\rm Im}\lambda|<\dfrac{C_1}{|\epsilon|}\Big\}$ while Eq. \eqref{202302042155} has at least one root located in $\Big\{\lambda\in\mathbb{C}_0\cup\mathbb{C}_+\Big||{\rm Im}\lambda|<\dfrac{C_2}{|\epsilon|}\Big\}$. The proof will be divide into two parts. For the first part, we prove that Eq. \eqref{202302042155} has no roots located in $\Big\{\lambda\in\mathbb{C}_0\cup\mathbb{C}_+\Big||{\rm Im}\lambda|<\dfrac{C_1}{|\epsilon|}\Big\}$. For the second part, we prove that Eq. \eqref{202302042155} has at least one root located in $\Big\{\lambda\in\mathbb{C}_0\cup\mathbb{C}_+\Big||{\rm Im}\lambda|<\dfrac{C_2}{|\epsilon|}\Big\}$.


Taking $\tau_{\epsilon}=2l+\epsilon$ and $c$ in Eq. \eqref{202302042155} leads to the equation $h_\epsilon(\lambda)=c$, where
$$
h_\epsilon(\lambda)\triangleq -\dfrac{1}{2}{\rm e}^{\epsilon\lambda}{\rm e}^{2l\lambda}(1+{\rm e}^{-2\lambda}).
$$
Since $0<|c|<\sin[\dfrac{\pi}{2(2l-1)}]$, there exists $\tilde{c}\in(0,1)$ such that $|c|=\sin[\dfrac{\tilde{c}\pi}{2(2l-1)}]$. We choose $C_1\triangleq\dfrac{(1-\tilde{c})\pi}{2}>0, C_2\triangleq\dfrac{\pi}{2}$ and $s_\epsilon$ as the smallest integer number such that $s_\epsilon\pi>\dfrac{C_1}{|\epsilon|}$, while $S_\epsilon$ as the largest integer number such that  $S_\epsilon\pi<\dfrac{C_2}{|\epsilon|}$.

 Firstly, we prove that  for sufficiently small $|\epsilon|$, the equation $h_{\epsilon}(\lambda)=c$ has no roots lie in the vertical stripe $\left\{\lambda \Bigg| {\rm Re\lambda}\geq 0,|{\rm Im}\lambda|\leq s_\epsilon\pi \right\}$.
 For a fixed $\epsilon\neq 0$ and $\delta\in[-|\epsilon|,|\epsilon|]$, denote by $Q_\epsilon(\delta)$ as the number of roots  located in the vertial stripe $\left\{\lambda \Bigg| {\rm Re\lambda}\geq 0,|{\rm Im}\lambda|\leq s_\epsilon\pi \right\}$ for the equation $h_\delta(\lambda)=c$. By argument principle, we obtain that
$$
Q_\epsilon(\delta)=\dfrac{1}{2\pi{\rm i}}\int_{P_{\epsilon,R}}\dfrac{h'_{\delta}(\lambda)}{h_{\delta}(\lambda)-c}{\rm d}\lambda.
$$



Here, $P_{\epsilon,R}$ denotes the rectangle contour $p_1\cup p_2\cup p_3\cup p_4$, where
$$
p_1\triangleq\{\lambda={\rm i}\beta, \beta:s_\epsilon\pi\to -s_\epsilon\pi\}, p_2\triangleq\{\lambda=u-{\rm i}s_\epsilon\pi, u:0\to R \},
$$
$$
p_3\triangleq\{\lambda=R+{\rm i}\beta, \beta:-s_\epsilon\pi\to s_\epsilon\pi\}, p_4\triangleq\{\lambda=u+{\rm i}s_\epsilon\pi, u:R\to 0 \},
$$
and $R>0$ is a sufficiently large number such that $|h_\delta(\lambda)|>1>|c|$ for ${\rm Re}\lambda\geq R$ and $\delta\in[-|\epsilon|,|\epsilon|]$~(See Fig. \ref{fig17a}). The existence of $R>0$ is guaranteed by  $\lim_{{\rm Re}\lambda\to+\infty}|h_\delta(\lambda)|=+\infty$, where the limit is taken uniformly with respect to $\delta\in[-|\epsilon|,|\epsilon|]$.
% Figure environment removed

Condition \eqref{condition} ensures that $h_0(\lambda)=c$ has no roots in $\mathbb{C}_0\cup\mathbb{C}_+$, which indicates that $Q_\epsilon(0)=0$. Similar to the proof in Section \ref{section3}, we hope to prove that $Q_\epsilon(\delta)$ is continuous with respect to $\delta\in[-|\epsilon|,|\epsilon|]$. For this, we just need to prove that $h_\delta(\lambda)\neq c$ on the rectangle $P_{\epsilon,R}$. For $\lambda\in p_3$, $|h_\delta(\lambda)|>1>|c|$, which indicates $h_\delta(\lambda)\neq c$ on the line $h_3$. For $\lambda\in p_2\cup p_4$, $|{\rm Im}\lambda|=s_\epsilon\pi$, which indicates that ${\rm e}^{2l\lambda}(1+{\rm e}^{-2\lambda})\in\mathbb{R}\backslash\{0\}$. Furthermore, when $\delta\neq 0$, the argument of ${\rm e}^{\delta\lambda}$  for $\lambda\in p_2\cup p_4$ can be estimated as
$$
0<|{\rm arg}~({\rm e}^{\delta\lambda})|=|\delta{\rm Im}\lambda|\leq |\epsilon|s_\epsilon\pi\leq C_1+|\epsilon|\pi<(\dfrac{1-\tilde{c}}{2}+|\epsilon|)\pi<\pi.
$$
Therefore, we obtain ${\rm e}^{\delta\lambda}\notin\mathbb{R}$. This implies that $h_\delta(\lambda)\notin\mathbb{R}$ and thus $h_\delta(\lambda)\neq c$ for $\lambda\in p_2\cup p_4$. For $\delta=0$, we use Condition \eqref{condition} to ensure that $h_{0}(\lambda)\neq c$ on the line $p_2\cup p_4$. Therefore, we have proved that $h_\delta(\lambda)\neq c$ for $\lambda\in p_2\cup p_4,\delta\in[-|\epsilon|,|\epsilon|]$.

The most complicated part is to prove $h_{\delta}(\lambda)\neq c$ for $\lambda\in p_1$. We put this in the Appendix, Lemma \ref{lemma13} and completes the first part of proof.





Secondly, we prove that  for sufficiently small $|\epsilon|$, the equation $h_{\epsilon}(\lambda)=c$ has at least one root lie in the vertical stripe $\left\{\lambda \Bigg| {\rm Re\lambda}\geq 0,|{\rm Im}\lambda|\leq S_\epsilon\pi \right\}$. We only prove the case $\epsilon>0$. The case $\epsilon<0$ is similar so we omit it.

For a fixed $\epsilon>0$ and $\delta\in[-\epsilon,\epsilon]$, denote by $H_\epsilon(\delta)$ as the number of roots  located in the vertial stripe $\left\{\lambda \Bigg| {\rm Re\lambda}\geq 0,|{\rm Im}\lambda|\leq S_\epsilon\pi \right\}$ for the equation $h_\delta(\lambda)=c$. By argument principle, we obtain that
$$
H_\epsilon(\delta)=\dfrac{1}{2\pi{\rm i}}\int_{T_{\epsilon,R}}\dfrac{h'_{\delta}(\lambda)}{h_{\delta}(\lambda)-c}{\rm d}\lambda.
$$
Here,  $T_{\epsilon,R}$ denotes the rectangle contour $q_1\cup q_2\cup q_3\cup q_4$, where
$$
q_1\triangleq\{\lambda={\rm i}\beta, \beta:S_\epsilon\pi\to -S_\epsilon\pi\}, q_2\triangleq\{\lambda=u-{\rm i}S_\epsilon\pi, u:0\to R \},
$$
$$
q_3\triangleq\{\lambda=R+{\rm i}\beta, \beta:-S_\epsilon\pi\to S_\epsilon\pi\}, q_4\triangleq\{\lambda=u+{\rm i}S_\epsilon\pi, u:R\to 0 \},
$$
while $R>0$ is sufficiently large~(See Fig. \ref{fig17b}). Similarly to the first part, we get that $h_\delta(\lambda)\neq c$ on the line $q_2\cup q_3\cup q_4$. We only need to investigate $h_\delta(\lambda)$ on $q_1$. Akin to the proof in Section \ref{section3}, denote by a set
$$
\mathscr{F}_{\epsilon}\triangleq\Big\{ \delta\in[0,\epsilon]\Big |   \mbox{there exists}~~\beta\in\mathbb{R}~~\mbox{such that}~~h_\delta({\rm i}\beta)=c, |\beta|\leq S_\epsilon\pi   \Big\}.
$$
Thus, $\mathscr{F}_{\epsilon}$ is the set of the critical value for $H_\epsilon(\delta)$.  We also need to investigate whether $H_\epsilon(\delta+)-H_\epsilon(\delta-)$ is positive or negative by the Implicit Theorem for each $\delta\in\mathscr{F}_{\epsilon}$. We prove that $\mathscr{F}_{\epsilon}$ is nonempty~(Lemma \ref{lemma14}) and furthermore for each $\delta\in\mathscr{F}_{\epsilon}$,  $H_\epsilon(\delta+)-H_\epsilon(\delta-)>0$~(Lemma \ref{lemma15}). We put all the details in the Appendix, Lemma \ref{lemma14} and Lemma \ref{lemma15} and complete the proof of the second part.








%For the case $c=c^*<-1$, we show that the equation has at least one unstable root with zero frequency (real positive root).

%\begin{theorem}
%Suppose that $c=c^*<-1, \tau\in(0,1)$. Then Eq. \eqref{202302042155} has at least one root lies in $\mathbb{R}_+$.
%\end{theorem}
%Proof Eq.\eqref{202302042155} can be written as
%$$c^*=g_\tau(\lambda)\triangleq-\dfrac{1}{2}[{\rm e}^{\tau\lambda}+{\rm e}^{(\tau-2)\lambda}].$$
%We only consider the root $\lambda\in\mathbb{R}_+$. By $g_\tau(0)=-1$ and $\lim_{\lambda\to+\infty}g_\tau(\lambda)=-\infty$, we know that there exists at least one root located in $\mathbb{R}_+$ for equation $g_\tau(\lambda)=c^*$. This completes the proof.
%


\begin{remark}\label{robustremark}
As noted by \cite{Wangdelay2011, Datko86, Logemann}, {\color{black} the feedback stabilizer of wave equation    usually shows lack of robustness to a small  delay perturbations. %However,
%this fact can not be well embodied on the numerical experiments.
}Despite this non-robustness, numerical experiments often demonstrate an absence of the destabilizing effect when a small perturbation is added to the time delay. %This section provides a mathematical explanation for this phenomenon.
It is important to note that numerical experiments often neglect high frequency modes. Theorem \ref{robust4} and \ref{robust3} confirm that when a small perturbation is added to the time delay, no roots are found in low frequencies.\end{remark}






\section{Appendix}



\begin{lemma}\label{lemma1}
The set $\mathscr{E}$ can be characterized as


$$
\mathscr{E}_{m,n}=\Big\{(-1)^{l}\cot{\dfrac{n(2l+1)\pi}{2m}} \Big|   
l=0,1,2,\cdots, 2m-1          \Big\}\cup \Big\{0 \Big\}.
$$


\end{lemma}
Proof For an element $k\in\mathscr{E}$, there exists a complex number $z={\rm e}^{{\rm i}\theta}, \theta\in [0,2\pi)$, such that
$c=f(z)$. Thus,

By taking $z={\rm e}^{{\rm i}\theta}, \theta\in[0,2\pi)$ into $f(z,k)=0$, we obtain that
Case A: $ k=0,   z^{2n}+1=0$

Case B: $\cos(n\theta)=-k{\rm e}^{{\rm i}m\theta}{\rm i}\sin(n\theta)$




$$
c=-\dfrac{1}{2}\bigg\{\cos[(2n-m)\theta]+\cos (m\theta)\bigg\}
-\dfrac{\rm i}{2}\bigg\{\sin[(2n-m)\theta]-\sin (m\theta)\bigg\}.
$$
Since $\sin[(2n-m)\theta]-\sin (m\theta)=0$, we get that $\theta=\dfrac{k\pi}{|m-n|}$, $k=0,1,2,\cdots, 2|m-n|-1$ and in this case, $s=-\cos(m\theta)$
or $\theta=\dfrac{(2k+1)\pi}{2n}, k=0,1,2,\cdots, 2n-1$ and in this case, $c=0$. This completes the proof.

\begin{lemma}\label{lemma2}
For each $c_*=f(z_*)\in\mathscr{E}\backslash\{0\}, |z_*|=1$, there exists an implicit function $z(c)$ such that $z(c_*)=z_*$ and $c=f(z(c))$ for each $c\in(c_*-\epsilon, c_*+\epsilon)$. Here, $\epsilon>0$ is sufficiently small. Suppose that $z(c)=r(c){\rm e}^{{\rm i}\theta(c)}$, where $r(c)$ and $\theta(c)$ represent the absolute value and argument value of the function $z(c)$, respectively. Then we have
$$
{\rm Sgn}[r'(c_*)]={\rm Sgn}(n-m){\rm Sgn}(c_*).
$$
For $c_*=0$, there exist $2n$ different implicit functions $z_k(c)$ such that $z_k(0)=z_k={\rm e}^{\rm i \theta_k}$ with $\theta_k=\dfrac{(2k+1)\pi}{2n}$, $k=0,1,2,\cdots, 2n-1$.  And we have
$$
{\rm Sgn}[r_k'(0)]={\rm Sgn}\cos[\dfrac{m(2k+1)\pi}{2n}].
$$
\end{lemma}
Proof Since $0=f(z(k),k)$, Taking derivatives with respect to $k$ of both sides of   $f(z(k),k)=0$   leads to
$z'(k)=-\dfrac{f_k}{f_z}$
Taking derivatives with respect to variable $k$ of both sides  of   $z=r{\rm e}^{\rm i\theta}$    leads to that
$\dfrac{z'}{z}=\dfrac{r'}{r}+{\rm i}\theta'$,
which further indicates that ${\rm Sgn}[r'(c)]={\rm Sgn}\{{\rm Re}[\dfrac{1}{f'(z)z}]\}$
For $c_*=f(z_*)\in\mathscr{E}\backslash\{0\}$, then $z_*=-\cos(m\theta)$ with $\theta=\dfrac{k\pi}{|m-n|}$, $k=0,1,2,\cdots, 2|m-n|-1$. We compute that ${\rm Re} [f'(z_*)z_*]=-(n-m)\cos(m\theta)=(n-m)c_*\neq 0$, which ensures the existence of the implicit function $z(c)$ near each nonzero $c_*\in\mathscr{E}$.
Furthermore, we get ${\rm Sgn}[r'(c_*)]={\rm Sgn}(n-m){\rm Sgn}(c_*)$.

 For $k_*=0$, the equation $f(z,k)=0$ has $2n$ different roots $z_l={\rm e}^{\rm i \theta_l}$ with $\theta_l=\dfrac{(2l+1)\pi}{2n}$, $k=0,1,2,\cdots, 2n-1$. 
 
 
 
 Since $z_l^{2n}=-1$, we compute that$f'(z_k)z_k=nz_k^{-m}\neq 0$, which ensures the existence of each implicit function $z_k(c)$ and further implies that
${\rm Sgn}[r_k'(0)]={\rm Sgn}\cos[\dfrac{m(2k+1)\pi}{2n}]$. This completes the proof.

\begin{lemma}\label{lemma3}
Denote by the set
$$
\mathscr{B}=\left\{  \cos[\dfrac{m(2k+1)\pi}{2n}]  \Bigg|k=0,1,2,\cdots, 2n-1 \right\}.
$$
If $m,n$ are coprime positive integers and $m>n\geq 2$, then $\mathscr{B}\cap \mathbb{R}_+\neq \emptyset$ and $\mathscr{B}\cap \mathbb{R}_-\neq \emptyset$.
\end{lemma}
Proof Case I: $m$ is an odd number. Since $m$ and $2n$ are coprime, for any integer number $x$, there exists integer number $p\in\{0,1,2,\cdots, 2n-1\}$ and $q$ such that $mp-2nq=x$. Thus, we can find integer number $p_1,q_1,p_2,q_2$ with $p_1,p_2\in\{0,1,2,\cdots, 2n-1\}$ such that
$mp_1-2nq_1=\dfrac{1-m}{2}$, $mp_2-2nq_2=n-\dfrac{1+m}{2}$,
which implies that
%$\dfrac{m(2p_1+1)}{2n}\pi-2nq_1\pi=\dfrac{\pi}{2n}>0$,
%$\dfrac{m(2p_2+1)}{2n}\pi-2nq_2\pi=(1-\dfrac{1}{2n})\pi<0$
 $\cos[\dfrac{m(2p_1+1)\pi}{2n}]=\cos(\dfrac{\pi}{2n})>0$ and $\cos[\dfrac{m(2p_2+1)\pi}{2n}]=-\cos(\dfrac{\pi}{2n})<0$.

Case II: $m$ is an even number and $n\geq 3$ is an odd number. In this case, $\cos(\dfrac{s\pi}{2n})>0$ and $\cos[\dfrac{(2n-s)\pi}{2n}]<0$ for $s=0,2.$
We find $s_1,s_2\in\{0,2\}$ such that $(s_1-m)/4$ and $(2n-m-s_2)/4$ are both integers. Since $\dfrac{m}{2}$ and $n$ are coprime, there exists integer number $p_1,q_1,p_2,q_2$ with $p_1,p_2\in\{0,1,2,\cdots, n-1\}$ such that
$\dfrac{m}{2}p_1-nq_1=\dfrac{s_1-m}{4}$,
$
\dfrac{m}{2}p_2-nq_2=\dfrac{2n-s_2-m}{4},
$
which further implies that
$
\cos[\dfrac{m(2p_1+1)}{2n}\pi]=\cos(\dfrac{s_1\pi}{2n})>0,
$
$
\cos[\dfrac{m(2p_2+1)}{2n}\pi]=\cos(\dfrac{2n\pi-s_2\pi}{2n})<0.
$
%Thus, $\cos[\dfrac{m(2p_1+1)\pi}{2n}]=\cos(\dfrac{s_1\pi}{2n})>0$ and $\cos[\dfrac{m(2p_2+1)\pi}{2n}]=-\cos(\dfrac{s_2\pi}{2n})<0$.


\begin{lemma}\label{lemma4}
If $n=1,m=4s-2, s\in\mathbb{N}^*$ , the largest element in the set $\mathscr{E}\cap \mathbb{R}_{-}$ is $-\sin[\dfrac{\pi}{2(m-1)}]$. If $n=1, m=4s, s\in\mathbb{N}^*$, the smallest element in the set $\mathscr{E}\cap \mathbb{R}_+$ is $\sin[\dfrac{\pi}{2(m-1)}]$.
\end{lemma}
Proof We only talk about the situation $n=1, m=4s-2$, the another is similar. Since $m$ and $m-1$ are coprime, there exists integer number $p,q$ with $p\in\{0,1,2,\cdots,m-2\}$ such that
$
\dfrac{m}{2}p-(m-1)q=\dfrac{m-2}{4},
$
which leads to
$
\dfrac{pm\pi}{m-1}-2q\pi=\dfrac{\pi}{2}-\dfrac{\pi}{2(m-1)}.
$
Therefore, we get that
$
-\cos [\dfrac{pm\pi}{m-1}]=-\sin [\dfrac{\pi}{2(m-1)}].$
On the other hand, for any integer number $k$ and $l$, if $\dfrac{km\pi}{m-1}\neq \dfrac{2l+1}{2}\pi$, then
$$
|\dfrac{km\pi}{m-1}-\dfrac{2l+1}{2}\pi|=|\dfrac{2km-(l+1)(m-1)}{2(m-1)}\pi|\geq \dfrac{\pi}{2(m-1)},$$ which implies that the absolute value of any nonzero element in $\mathscr{E}$ is larger than $\sin [\dfrac{\pi}{2(m-1)}].$ This completes the proof.


\begin{lemma}\label{lemma5}
If $n=1$ and $m\geq 3$ is an odd number, then $r_k'(0)=0, r_k''(0)<0$ for $k=0,1$, where the function $r_k$ is defined in Lemma \ref{lemma2}.
\end{lemma}
Proof When $n=1$ and $m\geq 3$ is an odd number, we get that $\cos[\dfrac{m(2k+1)\pi}{2n}]=0$ for $k=0,1$. According to Lemma \ref{lemma2}, we get that  $r_k'(0)=0$ for $k=0,1$.

Now we turn to compute $r_k''(0)$.
Taking derivatives of both sides of $1=f'(z)z'(c)$ leads to that
$0=f''(z)z'(c)^2+f'(z)z''(c)$,
and we obtain that $z''(c)=-\dfrac{f''(z)z'(c)^2}{f'(z)}=-\dfrac{f''(z)}{f'(z)^3}$.
Taking second derivatives of both sides of $z=r{\rm e}^{\rm i\theta}$ leads to that
$
z''=r''{\rm e}^{\rm i\theta}-r{\rm e}^{\rm i\theta}\theta'^2+{\rm i}[2r'\theta'{\rm e}^{\rm i\theta}+r{\rm e}^{\rm i\theta}\theta''],
$
which divided by $z$ leads to that
$
-\dfrac{f''(z)}{zf'(z)^3}=\dfrac{z''}{z}=\dfrac{r''}{r}-(\theta')^2+{\rm i}[\dfrac{2r'\theta'+\theta''}{r}].
$
We take the implicit function $z_k$ at point $c=0$. In this case $r_k(0)=1$,we compute that
$r_k''(0)=\theta_k'(0)^2+{\rm Re}[-\dfrac{f''(z)}{zf'(z)^3}]|_{z=z_k}= \theta_k'(0)^2-(2m-1)$.
From $\dfrac{z'}{z}=\dfrac{r'}{r}+{\rm i}\theta'$, we obtain that
$
\theta'={\rm Im}[\dfrac{z'}{z}]={\rm Im}[\dfrac{1}{zf'(z)}]={\rm Im} z^m.
$
The last equality is taken at $c=0, z=z_k$. This indicates that $|\theta_k'(0)|\leq 1$ and we obtain that $r_k''(0)<0$. This completes the proof.

\begin{lemma}\label{lemma6}
For an interval $[k_1,k_2]$, if $[k_1,k_2]\cap \mathscr{E}_{m,n}=\emptyset$, $N(k)$ is a constant. 
\end{lemma}
Proof
By argument principle , for $k\notin\mathscr{E}_{m,n}$, we have
\begin{equation}\label{argument}
N(k)-n=\dfrac{1}{2\pi{\rm i}}\int_{|z|=1}\dfrac{ f_z(z,k)}{f(z,k)}{\rm d}z.
\end{equation}
For an interval $[k_1,k_2]$, if $[k_1,k_2]\cap \mathscr{E}_{m,n}=\emptyset$, the right side is continuous with respect to variable $k\in[k_1,k_2]$. Since $N(k)$ is integer, we obtain that $N(k)$ is constant on the interval $[k_1,k_2]$.

Furthermore, we have
$
\lim_{|c|\to+\infty}\dfrac{1}{2\pi{\rm i}}\int_{|z|=1}\dfrac{f'(z)}{f(z)-c}{\rm d}z=0
$
and $\mathscr{E}\cap \big\{c\big||c|>1\big\}=\emptyset$, thus we get that
$N(c)\equiv m$ when $|c|>1$.



\begin{lemma}\label{lemma7}
For an interval $[c_1,c_2]$, if $[c_1,c_2]\cap \mathscr{C}_{a,b}=\emptyset$, $M_{a,b}(c)$ is a constant.
\end{lemma}
Proof
Since $\lim_{{\rm Re}\lambda\to+\infty}|g(\lambda)|=+\infty$, for any sufficiently large $R>0$, we find a sufficiently large $N>0$ such that $|g(\lambda)|>R$ for ${\rm Re} \lambda>N$. By argument principle \cite{complexanalysis}, we obtain that for  $c\in (-R,R)\backslash\mathscr{C}_{a,b}$,
\begin{equation}\label{argument2}
M_{a,b}(c)=\dfrac{1}{2\pi{\rm i}}\int_{P_{a,b,N}}\dfrac{g'(\lambda)}{g(\lambda)-c}{\rm d}\lambda.
\end{equation}
Here, $P_{a,b,N}$ denotes the rectangle contour $h_1\cup h_2\cup h_3\cup h_4$, where
$$
h_1\triangleq\{\lambda={\rm i}\beta, \beta:b\pi\to a\pi \}, h_2\triangleq\{\lambda=s+a{\rm i}\pi, s:0\to N \},
$$
$$
h_3\triangleq\{\lambda=N+{\rm i}\beta, \beta:a\pi\to b \pi\}, h_4\triangleq\{\lambda=s+b{\rm i}\pi, s:N\to 0 \}.
$$

Since $a,b\in\mathbb{Z}\backslash\{0\}$, we get $1+{\rm e}^{-2\lambda}\in\mathbb{R}_{+}$ on the line $h_2, h_4$, which further indicates that $g(\lambda)=-\dfrac{1}{2}(1+{\rm e}^{-2\lambda}){\rm e}^{\lambda\tau}\notin\mathbb{R}$ on the line $h_2, h_4$. Note that $g(\lambda)\neq c$ on $h_1, h_3$. Therefore, for an interval $[c_1,c_2]\subseteq (-R, R)$, if $[c_1,c_2]\cap \mathscr{C}_{a,b}=\emptyset$, the right side of Eq. \eqref{argument2} is continuous with respect to variable $c\in[c_1,c_2]$. Since $M_{a,b}(c)$ is integer number, $M_{a,b}(c)$ is a constant on interval $[c_1,c_2]$.


\begin{lemma}\label{lemma8}
If $(b-a)\tau>2$,$\tau$ is irrational, $M_{a,b}(c)$ is a nonzero constant  on the interval $(-\infty,-1)$ and $(1,+\infty)$, respectively.
\end{lemma}
Proof
Obviously, $|c|\leq 1$ for $c\in\mathscr{C}_{a,b}$. By Lemma \ref{lemma7}, we obtain that $M_{a,b}(c)$ is a constant on the interval $(1,+\infty)$ and $(-\infty,-1)$, respectively. We only need to prove that there exist  $\lambda_{1,2}\in\mathscr{C}_{a,b}$ such that $g(\lambda_1)\in(1,+\infty)$ and $g(\lambda_2)\in(-\infty,-1)$.


Suppose that $\lambda=p+{\rm i}q, p>0,a\pi<q<b\pi$. Thus, we get that
$$
g(\lambda)=-\dfrac{1}{2}[{\rm e}^{\tau p}\cos(\tau q)+{\rm e}^{\tau p-2p}\cos(\tau q-2q)]-\dfrac{{\rm i}}{2}[{\rm e}^{\tau p}\sin(\tau q)+{\rm e}^{\tau p-2p}\sin(\tau q-2q)].
$$
When ${\rm e}^{\tau p}\sin(\tau q)+{\rm e}^{\tau p-2p}\sin(\tau q-2q)=0,$ we compute that
$
{\rm e}^{2p}=-\dfrac{\sin(\tau q-2q)}{\sin(\tau q)},g(\lambda)=\dfrac{{\rm e}^{\tau p}\sin(2q)}{2\sin(\tau q-2q)}.
$





Since $b\tau-a\tau>2$,  there exists  odd number $k_1$ and even number $k_2$ such that $\tau a< k_i<\tau b$ for $i=1,2$. %Denote by a function
%$
%p(q)=\dfrac{1}{2}\log[-\dfrac{\sin(\tau q-2q)}{\sin(\tau q)}]>0.
%$


 Consider $q^*_i=\dfrac{k_i\pi}{\tau}, i=1,2$, since $\tau$ is irrational, we obtain that $\sin[(\tau-2)q^*_i]\neq 0$. Thus, we can take two sequences of $q_{i}^n,p_{i}^n, i=1,2, n=1,2,\cdots$, such that $\lim_{n\to+\infty}q_{i}^n=q^*_i$ and $\lim_{n\to+\infty}p_{i}^n=+\infty$  while  ${\rm e}^{2p_{i}^n}=-\dfrac{\sin(\tau q_{i}^n-2q_{i}^n)}{\sin(\tau q_{i}^n)} $   for $i=1,2$.







 For these two sequences, we obtain that
 $\dfrac{\sin(2q_i^n)}{\sin(\tau q_i^n-2q_i^n)}=-\lim_{n\to+\infty}\dfrac{\sin(\tau q_i^n-2q_i^n-\tau q_i^n)}{\sin(\tau q_i^n-2q_i^n)}=(-1)^{k_i+1}.$
If we take $\lambda_i^n=p_i^n+{\rm i}q_i^n$, we obtain that $g(\lambda_1^n)>0$, $\lim_{n\to+\infty}g(\lambda_1^n)=+\infty$ and  $g(\lambda_2^n)<0$, $\lim_{n\to+\infty}g(\lambda_2^n)=-\infty$. This completes the proof.








\begin{lemma}\label{lemma11}
For each nonzero $c_*=g(\lambda_*)\in\mathscr{C}_{a,b}, \lambda_*\in\mathbb{C}_0, a\pi\leq{\rm Im}\lambda_*\leq b\pi$, there exists an implicit function $\lambda(c)$ such that $\lambda(c_*)=\lambda_*$ and $c=g(\lambda(c))$ for each $c\in(c_*-\epsilon, c_*+\epsilon)$. Here, $\epsilon>0$ is sufficiently small. Furthermore, we have
$$
{\rm Sgn}[{\rm Re}\lambda'(c_*)]={\rm Sgn}(\tau-1){\rm Sgn}(c_*).
$$
For $c_*=0$, there exist $b-a-1$ different implicit functions $\lambda_k(c)$ such that $c=g(\lambda_k(c)), \lambda_k(0)=\lambda_k={\rm i}(k+\dfrac{1}{2})\pi$ , $a+1\leq k\leq b-1, k\in\mathbb{Z}$.  And we have
$$
{\rm Sgn}[{\rm Re}\lambda'(c_*)]=-{\rm Sgn}\cos[\tau(k+\dfrac{1}{2})\pi].
$$
\end{lemma}
ProofFor $c_*=g(\lambda_*)\in\mathscr{C}_{a,b}\backslash\{0\}$, since
$
c_*=g(\lambda_*)=-\dfrac{1}{2}[{\rm e}^{\tau\lambda_*}+{\rm e}^{(\tau-2)\lambda_*}]$,
we get that
$
{\rm Im} ({\rm e}^{\tau\lambda_*})=-{\rm Im}({\rm e}^{(\tau-2)\lambda_*}).$
By
$
|{\rm e}^{\tau\lambda_*}|=|{\rm e}^{(\tau-2)\lambda_*}|=1,
$
we obtain that $ |{\rm Re} ({\rm e}^{\tau\lambda_*})|=|{\rm Re}({\rm e}^{(\tau-2)\lambda_*})|$. Since $c_*\neq 0$, we have $ {\rm Re} ({\rm e}^{\tau\lambda_*})={\rm Re}({\rm e}^{(\tau-2)\lambda_*})=-\dfrac{c_*}{2}.$
%Using Implicit Theorem, we get that
%$$
%\lambda'(c^*)=\dfrac{1}{g'(\lambda_*)},
%$$
Thus we have~
${\rm Sgn}[{\rm Re}\lambda'(c_*)]={\rm Sgn}[{\rm Re}\dfrac{1}{g'(\lambda_*)}]={\rm Sgn}[{\rm Re}g'(\lambda_*)]
=-{\rm Sgn}\{{\rm Re}[\tau {\rm e}^{\tau\lambda^*}+(\tau-2){\rm e}^{(\tau-2)\lambda^*}]\}={\rm Sgn}(\tau-1){\rm Sgn}(c_*).
$

For $c_*=0$, $g(\lambda)=0$ has $b-a-1$ different roots $\lambda_k={\rm i}(k+\dfrac{1}{2})\pi$ , $a+1\leq k\leq b-1, k\in\mathbb{Z}$. By ${\rm e}^{2\lambda_k}=-1$,  we compute that
$
g'(\lambda_k)=-\dfrac{1}{2}[\tau{\rm e}^{\tau\lambda_k}+(\tau-2){\rm e}^{(\tau-2)\lambda_k}]
=-\dfrac{1}{2}[\tau{\rm e}^{\tau\lambda_k}-(\tau-2){\rm e}^{\tau\lambda_k}]=-{\rm e}^{\tau\lambda_k}.
$
Then we have
$
{\rm Sgn}[{\rm Re}\lambda'(c_*)]={\rm Sgn}[{\rm Re}\dfrac{1}{g'(\lambda_*)}]={\rm Sgn}[{\rm Re}g'(\lambda_*)]=-{\rm Sgn}\cos[\tau(k+\dfrac{1}{2})\pi].
$
This completes the proof.


\begin{lemma}\label{lemma12}
There exist  $j,l\in\mathbb{Z}$ such that $$\cos[\tau(j+\dfrac{1}{2})\pi]>0, \cos[\tau(l+\dfrac{1}{2})\pi]<0.$$
\end{lemma}
Proof
We only prove there exists a suitable integer number $j\in\mathbb{Z}$ such that $\cos[\tau(j+\dfrac{1}{2})\pi]>0$. The existence of $l\in\mathbb{Z}$ is similar.
Take a nonnegative continuous periodic function $f$ with periodic $2$ with its support on interval $(2m-\dfrac{\tau}{2},2m+\dfrac{1-\tau}{2})$ for integer $m$.

By ergodic theorem \cite{probability}, we obtain that
$
\lim_{N\to+\infty}\dfrac{1}{N}\sum_{k=1}^N f(k\tau)=\dfrac{1}{2}\int_0^{2} f(x){\rm d}x>0.
$
This further indicates that there exist $j, m\in\mathbb{N}^*$ such that
$
2m-\dfrac{\tau}{2}<j\tau<2m+\dfrac{1-\tau}{2},
$
which implies that
$
2m\pi<\tau(j+\dfrac{1}{2})\pi<(2m+\dfrac{1}{2})\pi,
$
and therefore leads to that $\cos[\tau(j+\dfrac{1}{2})\pi]>0.$



\begin{lemma}\label{lemma13}
For $\lambda={\rm i}\beta$, $|\beta|\leq s_\epsilon\pi$, $\delta\in[-|\epsilon|,|\epsilon|]$, $|\epsilon|<\dfrac{1-\tilde{c}}{2}$, we have $h_\delta(\lambda)\neq c.$
\end{lemma}
Proof
By taking $\lambda={\rm i}\beta$, we compute that
\begin{small}$$
h_\delta(\lambda)=-\dfrac{1}{2}\left\{\cos [(2n+\delta)\beta]+\cos[(2n-2+\delta)\beta]\right\}
-\dfrac{\rm i}{2}\left\{\sin [(2n+\delta)\beta]+\sin[(2n-2+\delta)\beta]\right\}.
$$
\end{small}
If $h_\delta(\lambda)\in\mathbb{R}$, then we obtain that $\sin (2n+\delta)\beta+\sin(2n-2+\delta)\beta=0$, which, by Trigonometric Identities Equations, implies that $\cos\beta=0$ or $\sin[(2n-1+\delta)\beta]=0$.

If $\cos\beta=0$, we compute that $h_\delta(\lambda)=0\neq c$.
Now we turn to investigate the situation  $\sin[(2n-1+\delta)\beta]=0$. In this case, we get  that $(2n-1+\delta)\beta=k\pi, k\in\mathbb{Z}$. Furthermore, we compute that $h_\delta(\lambda)=(-1)^{k+1}\cos(\dfrac{k\pi}{2n-1+\delta})$. Now we prove that $h_\delta(\lambda)\neq c$. If not, by $|h_\delta(\lambda)|=|c|$, we get
$
|\cos(\dfrac{k\pi}{2n-1+\delta})|=\sin [\dfrac{\tilde{c}\pi}{2(2n-1)}].
$
By  Trigonometric Identities Equations, we obtain that
$\dfrac{k\pi}{2n-1+\delta}+ \dfrac{\tilde{c}\pi}{2(2n-1)}=(l+\dfrac{1}{2})\pi$,or~$\dfrac{k\pi}{2n-1+\delta}-\dfrac{\tilde{c}\pi}{2(2n-1)}=(l+\dfrac{1}{2})\pi$,
for some $l\in\mathbb{Z}$. Multiplying $\dfrac{2(2n-1)}{\pi}$ of both sides leads to that
$-\dfrac{2\delta k}{2n-1+\delta}+\tilde{c}=(2l+1)(2n-1)-2k$,~or $-\dfrac{2\delta k}{2n-1+\delta}-\tilde{c}=(2l+1)(2n-1)-2k.$
By using $(2n-1+\delta)\beta=k\pi$, we obtain $-\dfrac{2\delta\beta}{\pi}\pm\tilde{c}=(2l+1)(2n-1)-2k.$
However, by using $|\delta|\leq |\epsilon|$ and $|\beta|\leq s_\epsilon\pi$, we estimate that
$$|-\dfrac{2\delta\beta}{\pi}\pm\tilde{c}|\leq |\dfrac{2\delta\beta}{\pi}|+|\tilde{c}|\leq 2|\epsilon| s_\epsilon+\tilde{c}\leq |\epsilon|+\dfrac{C_1}{\pi}+\tilde{c}\leq  |\epsilon|+\dfrac{1-\tilde{c}}{2}+\tilde{c}<1.
$$
This is a contradiction because $(2l+1)(2n-1)-2k$ must be an odd number. This completes the proof.




\begin{lemma}\label{lemma14}
$\mathscr{F}_{\epsilon}$ is nonempty.
\end{lemma}
ProofSimilar to the proof of Lemma \ref{lemma13}, by taking $\lambda={\rm i}\beta$  and $h_\delta(\lambda)$ is a nonzero real number, we get that
$(2n-1+\delta)\beta=k\pi, k\in\mathbb{Z}$, $h_\delta(\lambda)=(-1)^{k+1}\cos(\dfrac{k\pi}{2n-1+\delta})$.

Condition \ref{condition} implies that $c=(-1)^n\sin[\dfrac{\tilde{c}\pi}{2(2n-1)}]$. By taking $h_\delta(\lambda)=c$, we obtain that
$\cos(\dfrac{k\pi}{2n-1+\delta})=(-1)^{n+k+1}\sin[\dfrac{\tilde{c}\pi}{2(2n-1)}].$
By  Trigonometric Identities Equations, we obtain that
$\dfrac{\pi}{2}-\dfrac{k\pi}{2n-1+\delta}=(n+k+1)\pi+\dfrac{\tilde{c}\pi}{2(2n-1)}+2l\pi$,
or
$
\dfrac{\pi}{2}-\dfrac{k\pi}{2n-1+\delta}+(n+k+1)\pi+\dfrac{\tilde{c}\pi}{2(2n-1)}=2l\pi+\pi,
$
for some $l\in\mathbb{Z}$.
Multiplying $\dfrac{2(2n-1)}{\pi}$ of both sides leads to that
$\dfrac{2\delta k}{2n-1+\delta}-\tilde{c}=4n^2+4kn+8ln-4l-1,$
$\dfrac{2\delta k}{2n-1+\delta}+\tilde{c}=-4n^2-4kn+8ln+4k-4l+1$.
By using $(2n-1+\delta)\beta=k\pi$, we obtain that
$\dfrac{2\delta\beta}{\pi}-\tilde{c}=4n^2+4kn+8ln-4l-1$, or
$\dfrac{2\delta\beta}{\pi}+\tilde{c}=-4n^2-4kn+8ln+4k-4l+1$.
By using $|\delta|\leq |\epsilon|$ and $|\beta|\leq S_\epsilon\pi$, we estimate that
$|\dfrac{2\delta\beta}{\pi}\pm\tilde{c}|\leq |\dfrac{2\delta\beta}{\pi}|+\tilde{c}\leq 2\epsilon S_\epsilon+\tilde{c}\leq 1+\tilde{c}$.
Note that $4n^2+4kn+8ln-4l-1$  is an integer number with the formation $4s-1$  for some  $s\in\mathbb{Z}$, thus we obtain that $\dfrac{2\delta\beta}{\pi}-\tilde{c}=-1=4n^2+4kn+8ln-4l-1$. Since $0<\tilde{c}<1$, we obtain that $\delta<0$ and $k<0$.
Similarly for the case $\dfrac{2\delta\beta}{\pi}+\tilde{c}=-4n^2-4kn+8ln+4k-4l+1$,
we obtain $\dfrac{2\delta\beta}{\pi}+\tilde{c}=1=-4n^2-4kn+8ln+4k-4l+1, \delta>0, k>0.$
Now we prove this Lemma by finding suitable $k,l\in\mathbb{Z}, \delta>0,\beta$. We consider the situation $k>0$. $-1=4n^2+4kn+8ln-4l-1$ is equivalent to
$k(1-n)+l(2n-1)=-n^2.$
Denote by $q$ as the largest integer number smaller than $(2n-1)S_\epsilon$. Since $1-n$ and $2n-1$ are coprime, there exist two integers $k^*,l^*$ with $k^*\in\{q-2n+2,q-2n+1,\cdots, q\}$ such that
$k^*(1-n)+l^*(2n-1)=-n^2$. We know that $(2n-1)S_\epsilon-k^*<2n-1$. Then by considering $\dfrac{2\delta^* k^*}{2n-1+\delta^*}=2\tilde{c}$, we choose $\delta^*=\dfrac{(2n-1)(1-\tilde{c})}{2k^*-1+\tilde{c}}$, and
$\beta^*=\dfrac{k^*\pi}{2n-1+\delta^*}$. Finally, we need to verify that $\delta^*\leq \epsilon$ and $\beta^*\leq S_\epsilon\pi$. By using $k^*>(2n-1)(S_\epsilon-1)$, we obtain that
$
\delta^*<\dfrac{(2n-1)(1-\tilde{c})}{2(2n-1)(S_\epsilon-1)-1+\tilde{c}}<\epsilon,
$
$
\beta^*<\dfrac{k^*\pi}{2n-1}<S_\epsilon\pi.
$
This completes the proof.



\begin{lemma}\label{lemma15}
$H_\epsilon({\delta})$ is increasing with respect to $\delta\in(0,\epsilon]$.
\end{lemma}

ProofSimilar to the proof in Section \ref{section3}, we prove that for each element $\delta_*\in\mathscr{F}_{\epsilon}$, we have $h_{\delta^*}({\rm i}\beta_*)=c$. Then there exists an implicit function $\lambda(\delta)$ such that $h_{\delta}(\lambda(\delta))=c$, $\lambda(\delta_*)={\rm i}\beta_*$. Furthermore, we have
${\rm Sgn}[{\rm Re}\lambda'(\delta_*)]>0$.

By taking derivatives of $\delta$ of $h_{\delta^*}({\rm i}\beta_*)=c$, we obtain that
$\lambda'(\delta)=-\dfrac{\dfrac{\partial h_\delta}{\partial \delta}}{h'_{\delta}(\lambda)}=-\dfrac{\lambda(1+{\rm e}^{-2\lambda})}{(2n+\delta)(1+{\rm e}^{-2\lambda})-2{\rm e}^{-2\lambda}}.$
%Then we have
%$
%{\rm Sgn}[{\rm Re}\lambda'(\delta)]&={\rm Sgn}[{\rm Re}\dfrac{1}{\lambda'(\delta)}]
%={\rm Sgn}{\rm Re}[-\dfrac{2n+\delta}{\lambda}+\dfrac{2{\rm e}^{-2\lambda}}{\lambda(1+{\rm e}^{-2\lambda})}].
%$
By taking $\lambda={\rm i}\beta_*$, we compute
$
{\rm Sgn}[{\rm Re}\lambda'(\delta_*)]=-{\rm Sgn}[\beta_*\sin(2\beta_*)].
$
The first case is $\beta_*>0$, then there exists integer number $k_*>0,n,l_*$ such that
$\dfrac{\pi}{2}-\dfrac{k_*\pi}{2n_*-1+\delta}+(n_*+k_*+1)\pi+\dfrac{\tilde{c}\pi}{2(2n_*-1)}=2l_*\pi+\pi$. Multiplying $2$ on both sides leads to that
$
2\beta_*=\dfrac{\tilde{c}\pi}{2n-1}+2(n+k_*+1)\pi-4l_*\pi-2\pi+\pi,
$
which implies that
$\sin(2\beta_*)=-\sin [\dfrac{\tilde{c}\pi}{2n-1} ]<0$. Thus we have  ${\rm Sgn}[{\rm Re}\lambda'(\delta_*)]>0$. For the case $\beta^*<0$, the proof is similar. This completes the proof.



\begin{lemma}\label{lemma16}
For $\epsilon>0$ small efficiently,  Eq. $\eqref{202302042155}$ has at least one root located in $\Big\{\lambda\in\mathbb{C}_0\cup\mathbb{C}_+\Big||{\rm Im}\lambda|<S_\epsilon\pi\Big\}$ when we take $c=c>0$, $\tau=\epsilon$. Here $S_\epsilon$ is the smallest integer number such that $S_\epsilon>\dfrac{1}{\epsilon}$.\end{lemma}
Proof
The idea of the proof is quite similar to Lemma \ref{lemma7}, \ref{lemma8} and \ref{lemma11}. We use the notation $M_{-S_{\epsilon},S_{\epsilon}}(c)$ to denote the number of root of  Eq. $\eqref{202302042155}$ located in $\Big\{\lambda\in\mathbb{C}_0\cup\mathbb{C}_+\Big||{\rm Im}\lambda|<S_\epsilon\pi\Big\}$. In this notation, we fix $\tau=\epsilon$ and allow $c$ to vary on the whole interval $(0,+\infty)$. If we denote by a set
$$
\mathscr{H}\triangleq\left\{g(\lambda) \Bigg| \lambda\in\mathbb{C}_0, |{\rm Im}\lambda|\leq S_\epsilon \pi \right\} \cap \mathbb{R},
$$
where $$g(\lambda)\triangleq-\dfrac{1}{2}[{\rm e}^{\epsilon\lambda}+{\rm e}^{(\epsilon-2)\lambda}].$$
Using the same method as in the poof of Lemma \ref{lemma7} and \ref{lemma11}, we obtain that $M_{-S_{\epsilon},S_{\epsilon}}(c)$ is a constant on $(1,+\infty)$ and $M_{-S_{\epsilon},S_{\epsilon}}(c)$ is monotonically decreasing on $\mathbb{R}_+\backslash\mathscr{H}$. Finally, we only to validate that $M_{-S_{\epsilon},S_{\epsilon}}(c)$ is nonzero on  $(1,+\infty)$. We use the same method as in the proof of Lemma \ref{lemma8}. We prove that there exists $\lambda\in\Big\{\lambda\in\mathbb{C}_0\cup\mathbb{C}_+\Big||{\rm Im}\lambda|<S_\epsilon\pi\Big\}$ such that $g(\lambda)\in(1,+\infty)$.

Suppose that $\lambda=p+{\rm i}q, p>0,|q|<S_\epsilon\pi$. Thus, we get that
$$
g(\lambda)=-\dfrac{1}{2}[{\rm e}^{\epsilon p}\cos(\epsilon q)+{\rm e}^{\epsilon p-2p}\cos(\epsilon q-2q)]-\dfrac{{\rm i}}{2}[{\rm e}^{\epsilon p}\sin(\epsilon q)+{\rm e}^{\epsilon p-2p}\sin(\epsilon q-2q)].
$$
%Considering ${\rm e}^{\epsilon p}\sin(\epsilon q)+{\rm e}^{\epsilon p-2p}\sin(\epsilon q-2q)=0$.

Denote by $q^*\triangleq\dfrac{\pi}{\epsilon}<S_{\epsilon}\pi$, there are two different situations. Case I: $\sin(\epsilon q^*-2q^*)=0$. Case II: $\sin(\epsilon q^*-2q^*)\neq 0$.

For Case I, when $q=q^*$, we have ${\rm Im}~g(\lambda)=0$, $\cos(\epsilon q^*)=-1$, $\cos(\epsilon q^*-2q^*)=\pm 1$. Then $g(\lambda)=\dfrac{1}{2}{\rm e}^{\epsilon p}(1\pm {\rm e}^{-2p})$. For sufficiently large $p>0$, we have $g(\lambda)\in(1,+\infty)$.

For Case II, considering that ${\rm e}^{\epsilon p}\sin(\epsilon q)+{\rm e}^{\epsilon p-2p}\sin(\epsilon q-2q)=0$ yields that
${\rm e}^{2p}=-\dfrac{\sin(\epsilon q-2q)}{\sin(\epsilon q)},g(\lambda)=\dfrac{{\rm e}^{\epsilon p}\sin(2q)}{2\sin(\epsilon q-2q)}$. Since $\sin(\epsilon q^*)=0, \sin(\epsilon q^*-2q^*)\neq 0$,
we can take a sequence of $\{p_n\},\{q_n\}$ such that $\lim_{n\to+\infty}q_n=q^*,\lim_{n\to+\infty}p_n=+\infty$ while   ${\rm e}^{2p_n}=-\dfrac{\sin(\epsilon q_n-2q_n)}{\sin(\epsilon q_n)}.$ Thus, we have $\lim_{n\to+\infty}\dfrac{\sin(2q_n)}{\sin(\epsilon q_n-2q_n)}=1$. If we take $\lambda_n=p_n+{\rm i}q_n$, we obtain that $\lim_{n\to+\infty}g(\lambda_n)=+\infty$ and thus complete the proof.







%






\begin{thebibliography}{0}

\bibliographystyle{siamplain}

\bibitem{WangZQ2013}  J.-M. Coron  and  Z. Wang,
Output feedback stabilization for a scalar
conservation law with a nonlocal velocity,
 {\it SIAM J. Math. Anal. },    45(2013),   2646-2665.



\bibitem{Datko86}  R. Datko, J. Lagnese, and M.P. Polis, An example on the effect of time delays in boundary feedback
stabilization of wave equation, {\it SIAM J Control Optim.}, 24(1986), 152-156.
\bibitem{Datko88}  R. Datko, Not all feedback stabilized hyperbolic systems are robust with respect to small time
delays in their feedbacks, {\it SIAM J. Control Optim.}, 26(1988), 697-713.

%\bibitem{Datko91}  R. Datko, Two questions concerning the boundary control of certain elastic systems, {\it J. Differential Equations}, 92(1991), 27-44.

\bibitem{Datko93}  R. Datko, Two examples of ill-posedness with respect to small time delays in stabilized elastic
systems, {\it IEEE Trans Autom. Control}, 38(1993), 163-166.







\bibitem{FSIAM2} H. Feng,  Stabilization of     one-dimensional wave equation with   Van Der
Pol type   boundary condition,  {\it SIAM J. Control Optim.},
54(2016), 2436-2449.

\bibitem{FengTAC} H. Feng and B.Z. Guo,       Observer design and exponential stabilization of
wave equation in energy state space by boundary displacement
measurement only,  {\it  IEEE Trans. Autom. Control}, 62(2017), 1438-1444.



\bibitem{GuoMEiZDTAC} B.Z. Guo and Z.D. Mei, Output feedback stabilization for a class  of first-order equation setting of  collocated  well-posed linear systems with time delay in observation,
     {\it IEEE Trans. Autom. Control},   65(2020), 2612-2618.

%\bibitem{GuoXuGQ2006}   B.Z. Guo and G.Q. Xu, Expansion of solution in terms of generalized eigenfunctions for a hyperbolic system with static boundary condition, {\it Journal of Functional Analysis}, 231(2006), 245-268.

 \bibitem{GuoXuCZDelay2012}    B.Z. Guo, C.Z. Xu and H. Hammouri, Output feedback stabilization of a one-dimensional wave equation with an arbitrary time delay in boundary observation, {\it  ESAIM Control Optim. Calc. Var.}, 18(2012), 22-35.



\bibitem{Gugat} M. Gugat, Boundary feedback stabilization by time delay for one-dimensional wave equations,
{\it IMA J. Math. Control Inform.},  27.2 (2010), 189-203.

\bibitem{Gugat2} M. Gugat, M. Tucsnak. An example for the switching delay feedback stabilization of an infinite dimensional system: The boundary stabilization of a string, {\it Systems  Control Lett.}, 2011, 60(4): 226-233.



\bibitem{Hale} Hale J K, Lunel S M V. Introduction to functional differential equations, Springer Science \& Business Media, 2013.

\bibitem{Krstic2008scl} M. Krstic, and A. Smyshlyaev, Backstepping boundary control for first-order hyperbolic PDEs
and application to systems with actuator and sensor delays, {\it Systems Control Lett.}, 57(2008),
750-758.


\bibitem{Krsticdelaybook} M. Krstic, {\it Delay Compensation for Nonlinear, Adaptive, and PDE systems},  Birkh\"{a}user, Boston, 2009.


\bibitem{LuoGuo}   H. Luo, B.Z. Guo, and O. Morgul, {\it Stability and Stabilization of Infinite-Dimensional Systems with Applications}, Springer-Verlag, London, 1999.


 \bibitem{Logemann}  H. Logemann, R. Rebarber, and G. Weiss, Conditions for robustness and nonrobustness of
the stability of feedback systems with respect to small delays in the feedback loop, {\it SIAM J.
Control Optim.}, 34(1996),  572-600.

\bibitem{probability} M. Loève, Probability theory, {\it Courier Dover Publications}, 2017.




  \bibitem{MeiZD} Z.D. Mei and B.Z. Guo,  Stabilization for infinite-dimensional linear systems with bounded control and time delayed observation,  {\it Systems Control Lett.},    134(2019), 104532, 9 pp.

\bibitem{Sage1} S. Nicaise and  C. Pignotti, Stability and instability results of the wave equation
with a delay term in the boundary or internal feedbacks, {\it SIAM J. Control
Optim.},  45(2006), 1561-1585.

\bibitem{Sage2} S. Nicaise and C. Pignotti, Exponential stability of abstract evolution equations
with time delay, {\it J. Evol. Equ.},  15(2015), 107-129.



  \bibitem{Pazy} A. Pazy,  {\it Semigroups of Linear Operators and Applications
  to Partial Differential Equations},
  Springer-Verlag, New York, 1983.


  \bibitem{PSulian1979} B.S. Pavlov,   Basicity of an exponential systems and Muckenhoupt's condition, {\it Soviet Math.
Dokl.}, 20(1979),   655-659.


\bibitem{complexanalysis} E. M. Stein, R.Shakarchi, Complex analysis, {\it Princeton University Press}, 2010.


\bibitem{Wangdelay2011} J.M. Wang, B.Z. Guo,  and  M. Krstic, Wave equation stabilization by delays equal to even multiples of the wave propagation time, {\it SIAM J. Control
Optim.},  49(2011), 517-554.

%\bibitem{WeissG}G. Weiss, R. F. Curtain, Dynamic stabilization of regular linear systems, {\it IEEE Trans. Autom. Control}, 1997, 42(1): 4-21.





\bibitem{XuGQ2003} G.Q. Xu and B.Z. Guo, Riesz basis property of evolution equations in Hilbert spaces and application to a coupled string equation, {\it SIAM J. Control
Optim.} , 42(2003), 966-984.


\bibitem{XuGQ2006delay} G.Q. Xu, S.P. Yung,  and L.K. Li, Stabilization of wave
systems with input delay in the boundary control,
{\it  ESAIM Control Optim. Calc. Var.}, 12(2006),  770-785.







\end{thebibliography}


















	
\end{document}
%
% ****** End of file aipsamp.tex ******

\twocolumngrid

\setcounter{page}{1}
\renewcommand{\thepage}{R\arabic{page}} 

\bibliography{sn-bibliography}

\end{document}
