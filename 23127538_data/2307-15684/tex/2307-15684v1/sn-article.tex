\documentclass[pdflatex,sn-mathphys]{sn-jnl}
\usepackage[normalem]{ulem}

\newcommand{\michele}[1]{\textcolor{red}{\textrm{#1}}}
\newcommand{\marco}[1]{\textcolor{teal}{\textrm{#1}}}
\newcommand{\tommaso}[1]{\textcolor{brown}{\textrm{#1}}}
\newcommand{\romain}[1]{\textcolor{dandelion}{\textrm{#1}}}
\usepackage[version=4]{mhchem}

\jyear{2022}


\newtheorem{theorem}{Theorem}
\newtheorem{proposition}[theorem]{Proposition}

\theoremstyle{thmstyletwo}%
\newtheorem{example}{Example}%
\newtheorem{remark}{Remark}%

\theoremstyle{thmstylethree}%
\newtheorem{definition}{Definition}%

\raggedbottom



\begin{document}


\title[Quantum symmetrization transition in H$_3$S from QMC+PIMD]{Quantum symmetrization transition in superconducting sulfur hydride from quantum Monte Carlo and path integral molecular dynamics}

\author*[1]{\fnm{Romain} \sur{Taureau}}\email{romain.taureau@sorbonne-universite.fr}

\author[1]{\fnm{Marco} \sur{Cherubini}}\email{marco.cherubini@sorbonne-universite.fr}


\author[2]{\fnm{Tommaso} \sur{Morresi}}\email{tommaso.morresi1@gmail.com}

\author[1]{\fnm{Michele} \sur{Casula}}\email{michele.casula@sorbonne-universite.fr}

\affil*[1]{\orgdiv{Institut de Min\'eralogie, de Physique de Mat\'eriaux et de Cosmochimie}, \orgname{CNRS, MNHN, Sorbonne Universit\'e}, \orgaddress{\street{4 Place Jussieu}, \city{Paris}, \postcode{75005}, 

\country{France}}}

\affil[2]{\orgdiv{European Center for Theoretical Studies in Nuclear Physics and Related Areas}, \orgname{Fondazione Bruno Kessler}, \orgaddress{\street{Strada delle Tabarelle 286}, \city{Trento}, \postcode{38123}, \country{Italy}}}

\abstract{
We study the structural phase transition associated with 
the highest superconducting critical temperature measured in 
high-pressure sulfur hydride. A quantitative description 
of its pressure dependence has been elusive for 
any \emph{ab initio} theory attempted so far, raising 
questions on the actual mechanism driving the
transition.
Here, we reproduce the critical pressure of the hydrogen bond symmetrization in the Im$\bar{3}$m structure, in agreement with experimental data, by combining quantum Monte Carlo simulations for electrons 
with path integral molecular dynamics for quantum nuclei.
For comparison, we also apply the self-consistent harmonic approximation, which underestimates the critical pressure by about 40 GPa even when the most accurate potential energy surface is used, pinpointing the importance of an exact treatment of nuclear quantum effects. They indeed
play a major role in a significant reduction ($\approx$ 100 GPa) of the classical transition pressure and in a large isotope shift ($\approx$ 25 GPa) upon hydrogen-to-deuterium substitution.
}

\maketitle

\section{Introduction}\label{sec1}


Since its discovery in 1911 \cite{Onnes_supra}, superconductivity has been one of the most investigated topics in both theoretical and experimental physics. While it was discovered that almost every conductor could reach zero resistance at low-enough temperatures (T $<$ 10 K) \cite{Tresca2022}, the quest for higher critical temperature ($T_c$) superconductors became the new challenge. Until recently, cuprates were leading the race with a $T_c$ as large as 133 K for Hg-Ba-Ca-Cu-O systems \cite{Schilling_1993}, although the pairing mechanism in these materials is considered unconventional and it is not explained by the standard Bardeen–Cooper–Schrieffer (BCS) theory \cite{Bardeen_1957}. In 2015, the discovery of conventional superconductivity in $\ce{H_3S}$ with a maximum $T_c$ of 203 K reached at a pressure $P_c$ as high as 150 GPa \cite{Drozdov_2015} paved the way to a new era of high-$T_c$ materials. Indeed, hydrogen (H) -based systems are nowadays the most promising candidates to achieve room-temperature superconductivity.
As a matter of fact, in 2019, the same team that discovered $\ce{H_3S}$ claimed to have measured an even higher $T_c$ in $\ce{LaH_{10}}$, superconducting already at 250 K \cite{Drozdov_2019}, later followed by a similar discovery in the yttrium hydride \cite{Kong2021}. In a rush towards room-temperature superconductivity, more recent claims of $T_c$ larger than the one found in $\ce{LaH_{10}}$ did not meet the consensus of the whole community \cite{Service_2020,Dias_2023,Boeri_2023}. The main issue of these materials is the extreme pressure conditions, usually
larger than 150 GPa, needed to obtain the high-$T_c$ superconducting phase. Indeed,
while all the binary candidates involving hydrogen were theoretically investigated, none of them seems to sufficiently decrease the pressure of the superconducting state. Eyes are now turned towards ternary materials \cite{DiCataldo2022}.

In this work, we focus on the prototypical case of $\ce{H_3S}$ and we study its structural phase transition generally associated with the maximum of 
the  
%its
superconducting critical temperature, located at around 150 GPa \cite{Einaga_2016,mozaffari_2019,Minkov_2020,Osmond_2022}. According to x-ray diffraction data \cite{goncharov_2017},   
at lower pressures the sulfur (S) sites are arranged in a geometry that is compatible with the trigonal R3m symmetry (Fig.~\ref{fig:geometry}(b)) and, upon compression, the system undergoes a phase transition towards a body-centered-cubic (bcc) Im$\bar{3}$m structure
(Fig.~\ref{fig:geometry}(a)).

% Figure environment removed

After the first theoretical prediction of high-$T_c$ superconductivity in H$_3$S \cite{duan2014pressure}, several works
tried to explain 
the origin of the peak of $T_c$ found in experiments as a function of pressure. 
Even if
the magnitude of the calculated $T_c$ is right, confirming the BCS origin of the superconducting state, a quantitative disagreement between various theoretical approaches was found, with estimated $T_c$ values fluctuating over a 50 K range for the high-pressure phase \cite{Sano_2016}. Moreover, theoretical studies more oriented to understand the underlying structural properties of H$_3$S, revealed a significant disagreement in the transition pressures between the predicted phases. In those works \cite{Errea_2016,Bianco_2018}, the structural phase transition is explained by a quantum proton symmetrization from the R3m phase, with displaced protons, to the Im$\bar{3}$m one, where every hydrogen
lies in the midpoint of the two neighboring sulfur 
atoms (S-S midpoint). In that context, the shuttling mode of 
hydrogen atoms, namely their vibrational mode along the direction linking 
two neighboring sulfur atoms, was thoroughly investigated. The phase transition was then identified by looking at the dynamical instability of the symmetric Im$\bar{3}$m phase when the pressure is lowered and the shuttling mode softens.
On general grounds,
this
reflects the sudden transformation of the free energy profile, leading to a sign change of its curvature across the transition between two different crystal structures, one with lower symmetry than the other.


These findings were obtained by solving the nuclear Hamiltonian within the Stochastic Self Consistent Harmonic Approximation (SSCHA) \cite{Monacelli2021,Errea_2013,Bianco_2017}, which has proven
to be one of the best approximated theories to deal with nuclear quantum effects (NQE). Within this framework, the electronic part was solved by
Density Functional Theory (DFT) using different parametrizations for the exchange-correlation functional, like the Perdew-Burke-Ernzerhof (PBE) \cite{Perdew1996} and the Becke-Lee-Yang-Parr (BLYP) \cite{Becke_1988,Lee_1988} ones. Independently of the DFT functional used, a sizable underestimation of the experimental critical pressure $P_c$ by $\approx$ 40 GPa was always
observed, leaving open the question about the origin of this
mismatch, and whether this should be attributed to the electronic or to the nuclear 
components.

Here, we go beyond the previous state-of-the-art calculations by treating the electronic 
problem
not only at the DFT-BLYP, but also at Quantum Monte Carlo (QMC) level,
which provides a 
benchmark for the 
DFT methods.
QMC is known to 
yield
very accurate total energies in both molecules and solids \cite{lester_2009,saritas2017investigation,raghav2023toward}, thanks to its stochastic Green's function algorithms \cite{Foulkes2001,Wagner2016}, such as the lattice regularized diffusion Monte Carlo \cite{Casula_lrdmc}, projecting any initial trial wavefunction towards the ground state of the system within the fixed node approximation. Moreover, we solve the nuclear Hamiltonian by using Path Integral Molecular Dynamics (PIMD), which is in principle exact, outperforming any other approximation for the nuclear degrees of freedom. Then, we estimate the phase transition by looking at the hydrogen/deuterium density, focusing on its 
transformation from the unimodal to bimodal distribution, and at its quantum fluctuations, detecting when they freeze in a displaced geometry.

In this work, we 
have been able to track the evolution of the mode distribution 
with a high resolution in volume (and pressure), thanks to a three-dimensional (3D) model of the shuttling mode.
The reliability of our model has been
benchmarked using \emph{ab initio} PIMD simulations 
with BLYP electrons, across
the Im$\bar{3}m \rightarrow$ R3m transition. The advantage of the 3D model is that its potential energy surface (PES) can still be derived by much more expensive, although more accurate, QMC calculations, allowing us to check the impact of the electronic description on the phase transition.

In the model we developed, all hydrogen atoms in the system are allowed
to move in the same way. This feature induces some limitations, such as the lack of spatially disordered H configurations. In spite of this, we can 
accurately describe the path from the symmetric Im$\bar{3}$m phase to the asymmetric R3m geometry. We have finally performed both SSCHA and PIMD simulations of the 3D model to investigate how NQE treated at different levels of approximation affect the final outcome.

\section{Results}\label{sec2}

\subsection{Harmonic and anharmonic phonons}

At high pressure, above 150 GPa, the $\ce{H_3S}$ crystal is expected to be in the cubic Im$\bar{3}$m symmetric phase (Fig.~\ref{fig:geometry}(a)), where every hydrogen atom 
sits on
the midpoint of 
two neighboring sulfur atoms. Upon 
pressure release,
the lattice 
undergoes
a trigonal distortion and the hydrogen atoms 
leave
the aforementioned midpoint to 
move
closer to one of the two flanking sulfur atoms, leading to the R3m asymmetric phase, depicted in Fig.~\ref{fig:geometry}(b). In our 
description,
we introduce a 
simplification by neglecting the trigonal distortion, which is however very weak ($<$0.6°) \cite{Bianco_2018}. Thus, the R3m phase considered here 
differs from
the Im$\bar{3}$m one just by the 
hydrogen
positions.

% Figure environment removed

In Fig.~\ref{fig:phonons}, we report the analysis of the phonon dispersion for different volumes of the cubic Im$\bar{3}$m unit cell, obtained at the \emph{ab initio} level using the BLYP functional, either through the harmonic approximation via Density Functional Perturbation Theory (DFPT), or with the inclusion of quantum anharmonicity via PIMD simulations. Hereafter, volumes and energies will be expressed per H$_3$S unit, while the unit cell will be taken as cubic with S atoms 
arranged
in a bcc lattice.

At this point, it is important to underline that the DFPT and the PIMD phonons 
bear different information (see also Sec.~\ref{Sec:Phonons}).
The PIMD phonons are computed through the quantum displacement-displacement correlator recently developed in Ref.~\cite{Morresi_2021}. They describe the lowest vibrational excitations \cite{Morresi2022}, that is the energy difference between the first excited state and the ground state of the nuclear Hamiltonian.
This is the quantity normally measured by experimental probes, such as infrared or Raman spectroscopies.
Consequently, 
phonons computed in this way fully include anharmonic effects and are always positive definite, meaning that they cannot describe dynamical instabilities via the appearance of imaginary phonons.
This is at variance with
the harmonic case or with
approximated theories devised to deal with NQE, 
such as the SSCHA \cite{Monacelli2021}, which instead provide information about the sign of the free energy curvature at the reference geometry.

While for V=83.6 a$^3_0$ 
only 
the harmonic dispersion 
is reported in Fig.~\ref{fig:phonons}, 
for 
larger volumes we compare the PIMD phonons (green lines) obtained in a 2x2x2 supercell with the harmonic ones (light-blue lines). %\textit{
We notice that for PIMD phonons, the spatial range of the force constant matrix is such that the 2x2x2 supercell is large enough to allow for a $\mathbf{q}$-interpolation of the phonon branches $\omega_m=\omega_m(\mathbf{q})$. The comparison between PIMD and harmonic phonons of Fig.~\ref{fig:phonons} clearly shows how strong NQE are and how sizable is the softening of the most energetic phonons due to quantum anharmonicity, particularly at the largest volumes.
In the harmonic framework, for V$>$85 a$_0^3$ (see Figs.~\ref{fig:phonons} and \ref{fig:transition_shuttle_phon}), 
% Figure environment removed
the appearance of imaginary frequencies indicates the dynamical instability of the Im$\bar{3}$m 
structure.
More specifically, the softening of the shuttling hydrogen mode at $\mathbf{q}=\Gamma$ signals the transition towards the 
asymmetric R3m phase \cite{Errea_2016}. 
From Fig.~\ref{fig:phonons}, one can see that
imaginary frequencies disappear in PIMD phonons and their evolution 
as a function of volume is much smoother than in the harmonic case.  
As expected
from the definition of PIMD phonons,
PIMD simulations
never yield 
imaginary frequencies for the shuttling mode. 
In this regard, see also Fig.~\ref{fig:transition_shuttle_phon}, where we report the shuttling mode frequency we obtained
as a function of volume at different levels of theory. This analysis proves that
the phase transition cannot be 
determined
using solely the shuttling mode frequency as a proxy. We need to rely upon other observables in a framework describing nuclei as quantum particles.

So far, we have reported the structural behavior as a function of volume, fixed in our simulations. However, we can easily deduce the corresponding pressure by deriving the 
equation of state (EOS) $P=P(V)$ using the Vinet 
relation \cite{Vinet_EOS} 
computed with the same functionals 
employed
to calculate the phonon dispersions. Nevertheless, our goal is to go beyond DFT and reach a more accurate electronic description of the system using QMC methods (the details of our QMC calculations are reported in Sec.~\ref{Sec:elstructurePES}). A simple comparison of the EOS produced by the two approaches, shown in Fig.~\ref{fig:drozdov_mod}(a), reveals visible differences, suggesting that a description of the electronic structure at the QMC level is crucial to estimate correctly the critical pressure. Unfortunately, QMC calculations are much more expensive than DFT, and 
coupling them
with \emph{ab initio} PIMD simulations to study 
the real crystalline system is out of reach. Therefore, we need a simplified 
PES
describing the hydrogen shuttling mode that can be derived, after a fitting procedure, from QMC total energy calculations performed on a coarse grid of nuclear configurations. This model PES can then be used to compute the
shuttling mode frequencies 
and to study the phase transition at the PIMD level. 

\subsection{Classical 3D model}

The model PES is derived by considering the collective and coherent motion of all the hydrogen atoms along the direction connecting the two 
S atoms flanking each H (S-S direction), by allowing also 
hydrogen
out-of-axis mobility, while the S atoms are pinned in their bcc positions. In this way, we aim at reproducing the shuttling mode dynamics that takes place at $\mathbf{q}=\Gamma$, thus having the same modulation for all hydrogen atoms in the crystal. Therefore, we reduce the 3$N$ dimensions of the \emph{ab initio} potential (with $N$ the number of atoms in the supercell) to only 3 dimensions. 
The 3D-PES functional form is best represented 
through cylindrical coordinates, with the reference $x$-axis associated with the H position along the S-S axis, and its out-of-axis 
components (i.e. with $y$,$z$ $\ne 0$) described by the radial distance $r$ from the reference axis and the relative azimuthal angle $\phi$. The PES is then fitted over total energies generated either by DFT-BLYP or by QMC for nuclear configurations defined on a cylindrical grid. Further details about the model description can be found in Sec.~\ref{Sec:Pes}.

In Fig.~\ref{fig:pes_qmc_dft}, we report the PES profiles obtained by
% Figure environment removed
solving the electronic problem
within the DFT-BLYP (first row) and QMC (third row) methods.
At the volumes taken into account here, both DFT-BLYP PES and QMC PES have two minima connected through the inversion symmetry with respect to the S-S midpoint ((0.5,0,0) in fractional units).
The second row shows a comparison of both energy profiles cut along the line connecting these two PES minima, 
going
through the S-S midpoint. 
For the smallest volume analyzed, V=90.9 a$_0^3$, we found a good agreement between the DFT-BLYP PES and QMC PES, suggesting that electron correlation effects are reasonably well described at the DFT level at high-enough pressures. 
However,
the discrepancy between the two approaches 
appears
when we increase the volume and it grows continuously upon 
pressure release.
For the largest volume considered, V=110.0 a$_0^3$, the height of the double well barrier for QMC is $\sim$270 meV/$\ce{H_3S}$, 80 \% larger than the DFT-BLYP one.

The transition volume for classical nuclei can be estimated based on the PES by using the Landau theory for continuous phase transitions \cite{Landau_1937}. This method relies on the sign change of the free energy curvature (the total energy curvature at $T$=0 K) at the volume when the two displaced minima merge into a single one, in the symmetric configuration corresponding 
to the point (0.5,0,0) in Fig.~\ref{fig:pes_qmc_dft}.
For DFT-BLYP, we found a critical volume $V_c$ around 85 $a_0^3$ corresponding to a pressure of 263 GPa, while for QMC we found the same volume ($\approx$ 85 $a_0^3$) which corresponds to 238 GPa in this case (see Fig.~\ref{fig:drozdov_mod}(a)). We note that the 
$V_c$ yielded by the BLYP 3D-PES is in nice agreement with the value at which the shuttling mode frequency vanishes, computed 
\emph{ab initio} 
in the harmonic approximation (see Fig.~\ref{fig:transition_shuttle_phon}). This is a signature of our model PES quality.


\subsection{Quantum 3D model}

In order to have a reliable description of the structural phase transition based on our 3D-PES, we need 
to include nuclear quantum effects. We add them by performing PIMD calculations 
as implemented in Ref.~\cite{Mouhat_2017}. Numerical details of these simulations can be found in Sec.~\ref{Sec:PIMD}.

In Fig.~\ref{fig:distributions}, we report the projections of the resulting 3D proton density, 
% Figure environment removed
which takes two distinct 
shapes
depending on the volume of the corresponding phase (symmetric or asymmetric) of the Im$\bar3$m $\rightarrow$ R3m transition. The symmetric phase is associated with the regime where the density exhibits only one peak centered in the middle of the S-S axis and the asymmetric phase with 
the splitting of the central peak into two lobes for the largest volumes. 

In the contour plot of Fig.~\ref{fig:distributions}(a), one can clearly see that the doubling of the peak happens in QMC at smaller volumes (higher pressures) than in DFT-BLYP, as expected from the analysis of the classical PES, which shows deeper minima in the QMC PES at fixed volume. In Fig.~\ref{fig:distributions}(b), we plot the distribution of the hydrogen position projected along the shuttling mode direction, by including also the data coming from the \emph{ab initio} PIMD simulations driven by DFT-BLYP forces. Our 3D model has a
similar behavior in comparison with the 
full 
3$N$ dimensional system.
The mode distribution assumes a double peak shape at approximately the same volume for the model (blue lines) and the \emph{ab initio} system (green lines), evaluated for the same BLYP functional. The main differences are the broadness of the distribution, underestimated by the model, and the position of the peak, which lies closer to the S atoms in the 
\emph{ab initio} simulations of the R3m phase. These differences can be understood based on the enhanced quantum-thermal fluctuations of the \emph{ab initio} system compared to the one with a reduced number of degrees of freedom. Nevertheless, as far as the critical volume is concerned, the \emph{ab initio} and the model PIMD calculations are in agreement. This 
validates the accuracy of our 3D-PES model, which then allows one to compare directly 
BLYP and QMC results. The projected 1D distribution in Fig.~\ref{fig:distributions}(b) reveals that the QMC PES leads to a smaller symmetrization volume, as shown already in the contour plot of Fig.~\ref{fig:distributions}(a).

By fitting the 
distribution in Fig.~\ref{fig:distributions}(b) and interpolating the parameters obtained for several volumes, it is possible to determine precisely the position of its maximum as a function of volume, and thus the occurrence of 
quantum
symmetrization. Moreover, in PIMD we can easily quantify isotope effects on the symmetrization transition by replacing the hydrogen with the deuterium mass.
The shift of the hydrogen or deuterium (D) distribution peak with respect to the S-S axis midpoint
is presented in the left panels of Figs.~\ref{fig:probes_BLYP} and \ref{fig:probes_QMC}, for BLYP and QMC-derived PES, respectively. 
% Figure environment removed
% Figure environment removed
As mentioned before, the critical volume of the symmetrization transition can be defined as the value where the maximum of hydrogen, or deuterium, distribution starts merging with the S-S midpoint.
In this way, we estimate the symmetrization transition to take place in $\ce{H_3S}$ at a volume of 99.6 $a_0^3$ for DFT-BLYP, and of 96.3 $a_0^3$ for QMC. According to the EOS of %Fig.~\ref{fig:transition_shuttle_phon}
Fig.~\ref{fig:drozdov_mod}(a), these volumes correspond to critical pressures of 153 GPa for DFT-BLYP and of 152 GPa for QMC. These values, reported in Tab.~\ref{tab:comparison_fluc_dens_d3s_h3s_qmc_blyp}, are in good agreement with the position of the maximum $T_c$ measured in experiments \cite{Drozdov_2015}. Accounting for NQE leads to a strong reduction of the critical pressures observed in the classical framework, irrespective of the electronic theory used to generate the PES. Within the BLYP PES, the critical pressure decreases from the classical value of 263 GPa to $\simeq$ 150 GPa. Within the QMC PES, NQE 
shift
the pressure down from 238 GPa to approximately the same pressure as the one found for the BLYP PES ($\simeq$ 150 GPa). Nevertheless, it is important to underline that the two similar transition pressures obtained by BLYP PES and QMC PES after inclusion of NQE originate from a compensation of errors in the BLYP values, if we take QMC as reference. Indeed, a critical volume overestimation found in the BLYP PES compensates with a pressure overestimation in the BLYP EOS to yield approximately the same transition pressure as the one found in QMC (see Fig.~\ref{fig:drozdov_mod}(a)).

To better characterize the transition, we introduce a second probe, based on the ``local moment'' susceptibility. Indeed, quantum fluctuations are at work across the transition to make the hydrogen shuttle between the two PES minima. As the volume increases and the minima deepen, the fluctuations will start freezing, leading to the creation of a local moment, generated by the displaced proton in the R3m phase. PIMD fully accounts for quantum fluctuations, thanks to its imaginary time resolution. We can measure them by computing the imaginary time correlator $ g(\beta/2) = \langle\delta x (0) \delta x(\beta/2)\rangle$, with $\beta = 1/(k_B T)$ the inverse temperature used in the PIMD simulations, and $\delta x (\tau) = x(\tau) - \left\langle x \right\rangle$, where $\left\langle x \right\rangle$ is the thermal quantum average of the $x$ coordinate. Quantum fluctuations reduce the value of $g(\beta/2)$. A non-zero value of $g(\beta/2)$ can be interpreted by the presence of a finite moment in the  
distribution. In our 3D-PES, this moment is by definition local, because by model construction the hydrogen dynamics is condensed in a single 3D site. Therefore, the local moment susceptibility $\chi_g$ is the normalized variance of $g(\beta/2)$, namely $\chi_g=\textit{Var}[g(\beta/2)] / \textit{Var}[g(0)]$. Within the local moment fluctuation picture, the symmetrization transition can then be estimated by evaluating the volume at which $\chi_g$ is maximum, as shown in the right panels of Figs.~\ref{fig:probes_BLYP} and \ref{fig:probes_QMC}, for the BLYP PES and QMC PES, respectively. This quantity has already been used in a previous work\cite{Miha} to identify the transition from a symmetric phase to a disordered regime in an anharmonic oscillator chain, characterized by a tunable double well potential, where the symmetry is locally broken in favor of displaced configurations.
If we describe the symmetrization transition based on local moment fluctuations,  we obtain $V_c$ = 104.2 $a_0^3$ for DFT-BLYP, corresponding to $P_c$ = 130 GPa, and $V_c$ = 100.9 $a_0^3$ for QMC, corresponding to $P_c$= 126 GPa (Tab.~\ref{tab:comparison_fluc_dens_d3s_h3s_qmc_blyp}). Also in this case, like for the density probe, cancellation of errors is at play and, by consequence, the two electronic descriptions provide 
almost
the same critical pressure. 

\begin{table}[t!]
\centering
\begin{tabular}{|c|cccc|cccc|}
\hline
\textbf{Theory}     & \multicolumn{4}{c|}{\textbf{DFT-BLYP}}                                                       & \multicolumn{4}{c|}{\textbf{QMC}}                                                            \\ \hline
\textbf{Isotope}    & \multicolumn{2}{c|}{$\ce{H_3S}$}                             & \multicolumn{2}{c|}{$\ce{D_3S}$}        & \multicolumn{2}{c|}{$\ce{H_3S}$}                             & \multicolumn{2}{c|}{$\ce{D_3S}$}        \\ \hline
\textbf{Approach}   & \multicolumn{1}{c|}{Fluc.} & \multicolumn{1}{c|}{Dens.} & \multicolumn{1}{c|}{Fluc.} & Dens. & \multicolumn{1}{c|}{Fluc.} & \multicolumn{1}{c|}{Dens.} & \multicolumn{1}{c|}{Fluc.} & Dens. \\ \hline
$V_c$ {[}$a_0^3${]} & \multicolumn{1}{c|}{104.2} & \multicolumn{1}{c|}{99.6}  & \multicolumn{1}{c|}{97.6}  & 96.6  & \multicolumn{1}{c|}{100.9} & \multicolumn{1}{c|}{96.3}  & \multicolumn{1}{c|}{95.6}  & 93.6  \\ \hline
$P_c$ {[}GPa{]}     & \multicolumn{1}{c|}{130}   & \multicolumn{1}{c|}{153}   & \multicolumn{1}{c|}{165}   & 171   & \multicolumn{1}{c|}{126}   & \multicolumn{1}{c|}{152}   & \multicolumn{1}{c|}{156}   & 169   \\ \hline
\end{tabular}
\caption{Critical pressures and volumes for the symmetrization transition yielded by PIMD according to the electronic description (DFT-BLYP or QMC), the probe used (density or local moment fluctuations), and the isotope considered ($\ce{H_3S}$ or $\ce{D_3S}$).}
\label{tab:comparison_fluc_dens_d3s_h3s_qmc_blyp}
\end{table}

The two probes we used in this work allow us to determine a lower and an upper bound for the critical pressure, where the lower one is associated with the local moments formation due to the freezing of quantum fluctuations, and the upper one is related to the displacement of the maximum of the quantum distribution. 
The same analysis is carried out for both the $\ce{H_3S}$ and $\ce{D_3S}$ crystals, to estimate the magnitude of isotope effects. We summarize the results in Tab.~\ref{tab:comparison_fluc_dens_d3s_h3s_qmc_blyp}, 
where we show that the hydrogen-to-deuterium substitution brings about an increase of the critical pressure that falls into the [17-35] GPa range. 



\subsubsection{Comparing PIMD and SSCHA solutions of the quantum 3D model}

At variance with previous state-of-the-art calculations based on a combination of DFT-BLYP and SSCHA frameworks, our PIMD results yield  $P_c$ in a substantial agreement with the experimental finding, and this irrespective of the electronic theory used to generate the PES. To investigate more deeply the reasons for this contrasted outcome, we carry out SSCHA calculations with our model PES (see Sec.~\ref{Sec:SSCHA} for details), and we compare the relative accuracy of PIMD and SSCHA in $\ce{H_3S}$ by solving the same 3D PES with the two methods. The SSCHA results are shown in the left panels of Figs.~\ref{fig:probes_BLYP} and \ref{fig:probes_QMC}, where we plot the SSCHA centroid positions as a function of volume. In SSCHA, the occurrence of the asymmetric R3m phase is signalled by a centroid displaced with respect to the S-S midpoint.
The SSCHA 
critical values are $V_c = 107.8$ $a_0^3$, $P_c = 114$ GPa for the DFT-BLYP PES, and $V_c = 102.4$ $a_0^3$, $P_c = 118$ GPa for the QMC PES. As in PIMD, there is no significant difference between the electronic structure method used to generate the PES.
Our SSCHA results for 
the DFT-BLYP PES
are in a very good agreement with the outcome of previous SSCHA simulations for the full \emph{ab initio} system \cite{Bianco_2018}, calculated with the same DFT-BLYP functional. 
However, we find that the SSCHA strongly underestimates $P_c$ with respect to the one obtained in PIMD for the same PES, suggesting that the approximated description of the nuclear Hamiltonian has to be blamed 
for the disagreement with both PIMD and experimental results. As one can evince from Figs.~\ref{fig:probes_BLYP} and \ref{fig:probes_QMC}, SSCHA overestimates $V_c$ (underestimates $P_c$)  also for $\ce{D_3S}$. While the absolute values of $P_c$ are systematically underestimated for this hydride, the size of the isotope effect is correctly accounted for by the SSCHA theory, thanks to its variational nature. 

Let us look 
now
at the predictions for the shuttling mode frequencies, plotted in Fig.~\ref{fig:transition_shuttle_phon} for various methods. It has to be noted that, within the SSCHA, the phonon frequency of the shuttling mode shows a jump at $V_c$, at variance with the PIMD frequencies, which are instead smooth across the transition. 
This is due to the 
hop of
the SSCHA centroid 
from
its symmetric position
to a different minimum of the free energy,
already
``preformed'', 
which breaks the symmetry and 
becomes energetically more favorable at $V_c$.
Thus, the development of the SSCHA bimodal distribution is much sharper 
than in PIMD, as also seen in Figs.~\ref{fig:probes_BLYP} and \ref{fig:probes_QMC}.
Moreover, we also observe an increase of the SSCHA phonon line-width across the transition of the order of 10 cm$^{-1}$. The PIMD phonons produce 
an abrupt variation
in the phonon frequencies at the transition, only if the reference position of the displacement-displacement correlators is 
constrained to follow the position of
one of the two density peaks, instead of being kept at the center of the bimodal distribution. However, 
the true displacement $\delta x$ in the PIMD phonon correlator is defined with respect to the quantum average position $\langle x \rangle$, which stays
at the center of the double well. Thus, we believe that the jump seen in the SSCHA phonon frequencies is fictitious and due to the underlying Gaussian mean-field approximation used to compute 
anharmonic phonons in a low-dimensional potential. Such a jump is not seen, for instance, in the \emph{ab initio} SSCHA calculations for the same system \cite{Bianco_2018} using the full 3$N$-dimensional BLYP potential. 
Our most reliable phonon determination, based on PIMD, shows a progressive phonon softening. 
This is not only true within our 3D model PES, but also for our PIMD calculations driven by \emph{ab initio} forces computed at the DFT-BLYP level, as shown in Fig.~\ref{fig:transition_shuttle_phon}. The agreement between shuttling mode frequencies yielded by the 3D model and the ones given by \emph{ab initio} calculations highlights once again the quality of our model PES.

\section{Discussion}
\label{sec12}

In this work, starting from \emph{ab initio} electronic structure calculations, we generated a model PES to describe the shuttling mode of hydrogen in $\ce{H_3S}$, responsible for the R3m $\rightarrow$ Im$\bar 3$m transition, which was originally associated with the 
$T_c$ maximum
as a function of pressure. Despite the fact that such a hydrogen symmetrization 
is expected to happen in $\ce{H_3S}$ upon compression, so far no theoretical method has been able to spot it 
at pressures near the one that maximizes $T_c$ in experiments. This raised doubts on the original association between superconductivity and structural transition \cite{Akashi_2016,Azadi_2017}, worsened by the fact that other competing symmetries could be stable in the same pressure range \cite{Goncharov_2016,Li_2016,Guigue_2017,cui2019favored}.
The mismatch found between previous theoretical estimates of the
critical pressure $P_c$
and
the experimental values for the $T_c$ maximum is solved 
by applying
state-of-the-art computational methods in both the electronic and nuclear Hamiltonians, namely using QMC calculations for electrons, and the PIMD approach for nuclei. 
The quantum description of nuclei via PIMD on the top of an accurate PES generated by QMC
allowed us to study precisely
the impact of quantum effects on the evaluation of the transition pressure as well as on the vibrational 
properties
of the system. 
By comparing the phonon dispersion obtained from PIMD calculations with the one yielded by the harmonic approximation (Fig.~\ref{fig:phonons}), we have shown that this system exhibits strong anharmonicity that needs to be taken into account. 
The PIMD phonons, related to the lowest-energy phonon excitations \cite{Morresi_2021}, undergo a progressive softening, by keeping a smooth behavior across the hydrogen symmetrization transition, also for the shuttling mode driving the R3m $\rightarrow$ Im$\bar 3$m structural change (Fig.~\ref{fig:transition_shuttle_phon}).
To determine 
the transition at the PIMD level,
we have 
proposed
two distinct probes: the splitting of the hydrogen density maximum into a bimodal distribution, and the local moment 
susceptibility
related to the freezing of local quantum fluctuations.
Within our QMC+PIMD approach, the experimental pressure where $T_c$ is maximum is bracketed by the $P_c$ value estimated from the local fluctuations probe and the one determined by the 
transformation
of the bimodal hydrogen distribution into a 
unimodal one.
Consequently, these two probes provide a lower and an upper bound for the critical pressure, with a range between the two of $\approx$ 20 GPa. 
The range of critical pressures identified is consistent with the available experimental data for the $T_c$ peak \cite{Drozdov_2015,Einaga_2016,Minkov_2020} %observed 
for both $\ce{H_3S}$ and $\ce{D_3S}$, as we can see in Fig.~\ref{fig:drozdov_mod}(b). This result is remarkable, as we are able to predict accurately the structural phase transition in $\ce{H_3S}$ and $\ce{D_3S}$ using a simple model, where the freezing of local fluctuations 
and
the onset of a bimodal distribution in the proton density correspond to the stabilization of a long-range order in the full system. 
This can be understood by the fact that in the ordered R3m phase all the hydrogen atoms break the Im$\bar 3$m symmetry in the same way, which is exactly the collective path captured through the model PES by construction.


% Figure environment removed

Furthermore, we notice that the \emph{ab initio} electronic structure computed at the DFT-BLYP level predicts very good results for the critical pressure, similar to those obtained by QMC.
However, it is important to stress that the DFT-BLYP pressures are affected by error compensation, 
the overestimation of the critical volume 
being
balanced by a different EOS if compared against QMC calculations. This aspect underlines the importance of using an accurate electronic description, beyond the DFT level. The generation of our model PES, built to describe the hydrogen shuttling mode, allowed us to exploit the QMC energies in a PIMD framework, otherwise unfeasible in the full 3$N$ dimensional system.

To shed light on the reasons why previous theoretical works, mainly performed within the SSCHA framework, did not succeed 
in reproducing
the experimental $P_c$ in $\ce{H_3S}$, we carried out SSCHA calculations to compare their solution against PIMD for the same PES. We found 
that SSCHA underestimates the critical pressure by about 35 GPa, in accordance with previous estimates. This shows how in this system, where the transition can be modelled by a 3D double-well potential, the self-consistent harmonic approximation is particularly fragile.

We conclude by noting that the the R3m $\rightarrow$ Im$\bar 3$m structural phase transition in sulfur hydride has strong analogies with the hydrogen bond symmetrization in other compounds such as high-pressure ice, where, upon compression, 
phase VII and VIII
hosting displaced protons, 
stable at lower pressure, are expected to transform into the symmetric phase X \cite{Pruzan2003,Benoit1998}.
However, it is still a matter of debate whether the transformation is direct or whether another intermediate disordered 
structure %(phase VII) 
appears, with protons only partially symmetrized. 
In
this respect, 
further work is needed to extend our model beyond the collective path dynamics to treat non-local spatial correlations and disordered patterns. Machine learning schemes could then be useful to generate more extended PES from QMC data \cite{tirelli_2022,Huang2022,Ceperley2023} 
with the aim at including a larger variety of hydrogen configurations in PIMD calculations by keeping the same QMC accuracy. 


\section{Methods}
\label{sec11}

\subsection{Electronic structure calculations for the PES model}
\label{Sec:elstructurePES}
For the DFT electronic structure calculations, we used the Quantum Espresso (QE) suite of codes \cite{qe1,qe2}, while for the QMC calculations, we employed the TurboRVB package \cite{Nakano_2020}. For sake of consistency, in both DFT and QMC calculations, we used the same set of pseudopotentials. Namely, 
we treated the sulfur atom with the ccECP neon-core pseudopotential \cite{Bennett_2017} particularly suited for correlated calculations, available in both the QE-compatible Unified Pseudopotential Format (UPF) and in the TurboRVB-compatible Gaussian expansion format. For hydrogen, we used the bare Coulomb potential, with a very short-range cutoff for a QE usage within the plane-wave framework. In the QMC calculations instead, no short-range cutoff is needed for the bare Coulomb potential, because the nuclear cusp conditions are automatically fulfilled by our QMC wave function (see below). These pseudopotentials have been chosen after performing preliminary calculations at the DFT level to test their accuracy. We also tested other pseudopotentials (ultrasoft (US), projector augmented wave (PAW), and a combination of the above), by comparing the total energy profile obtained by moving the hydrogen atom away from the S-S midpoint, and constrained to stay on
the S-S axis. This leads to a very crude one-dimensional (1D) PES, which is however useful for testing purposes, with the advantage that it is easily computable for its simplicity.
We took as reference the total DFT energy computed with the all-electron LAPW approach, as implemented in Elk \cite{elk}. The ccECP pseudopotential for the sulfur atom and the bare Coulomb potential with short-range cutoff for the hydrogen atom turned out to be the most accurate choice (Fig.~\ref{fig:pp_quality}). 

% Figure environment removed

For 
single-point calculations at selected nuclear configurations,
we carried out DFT calculations with the Becke-Lee-Yang-Parr (BLYP) functional \cite{Becke_1988,Lee_1988}. The cutoff energy for plane waves is set to 200 Ry (due to the hardness of the H Coulomb pseudopotential), with the smearing parameter equal to 0.002 Ry and a $\mathbf{k}$-points grid of 32x32x32.

For the QMC calculations, we used a Slater-Jastrow wavefunction $\Psi$, which reads as:
\begin{equation}
\label{Eq:QMC_wavefunction}
    \Psi = \Phi_{S} \cdot \exp(J),
\end{equation}
where the term $\exp\left( J \right)$ is the Jastrow factor, symmetric under electron exchange, while $\Phi_{S}$ is the antisymmetric Slater determinant. 
The Slater orbitals in $\Phi_{S}$ are generated by DFT calculations within the Local Density Approximation (LDA) 
\cite{Kohn1965}, 
performed in a Gaussian basis set by means of the DFT 
code
built in TurboRVB. For the sulfur atom, we employed a modified cc-pVTZ primitive basis set with $6s6p2d1f$ components, contracted into 11 hybrid orbitals through the Geminal Embedded Orbitals (GEO) procedure \cite{Sorella_GEO}. For hydrogen, we used a modified cc-pVTZ primitive basis set with $4s2p1d$ components contracted into 6 GEO hybrid orbitals. 

The Jastrow exponent $J$ 
introduces explicitly electronic 
correlation in the wavefunction, and it can be decomposed into three terms, such that $J  = J_1 + J_2 + J_{3}$. 

$J_1$ is the so-called one-body term, which takes into account the interaction effects between the electrons $i$ and a nucleus $I$, and it depends on the relative electron-nucleus distances $r_{iI}$. $J_2$ is the so-called two-body term, treating the correlations between
electrons $i$ and $j$, and depending on their relative distance $r_{ij}$. Both $J_1$ and $J_2$ are designed to fulfill the electron-nucleus and electron-electron cusp conditions, respectively. They read as $J_1=\sum_{i=1}^{N_e} \sum_{I=1}^{N} u_I(r_{iI})$, and $J_2=\sum_{i<j=1}^{N_e} v(r_{ij})$, where $N$ ($N_e$) is the number of nuclei (electrons) in the supercell, and the functions $u$ and $v$ are defined as follows:
\begin{eqnarray}
    u_I(r) &=& \frac{Z_I}{a} (1-e^{-ar }) \label{Eq:u_func}\\
    v(r) &=& \frac{ r }{2(1+b r)} \label{Eq:v_func},
\end{eqnarray}
with $a$ and $b$ variational parameters, and $Z_I$ the charge of the $I$-th pseudoatom.  
The coefficients in Eqs.~\ref{Eq:u_func} and \ref{Eq:v_func} are set to fulfill the Kato cusp conditions for electron-nucleus and electron-electron coalescence, respectively \cite{Kato1957}. 

$J_{3}$ is the three-body term that accounts for the electron-electron-nucleus interactions. As defined in TurboRVB, it is also intrinsically non-homogeneous, because it depends on the individual electron positions and not only on the relative distances, which is less accurate.
Being non-homogeneous, it is expanded on a modified atomic Gaussian basis set of $2s2p1d$ atomic orbitals, for both sulfur and hydrogen atoms. 

The $J_3$ parameters, together with $a$ and $b$, are optimized by minimizing the variational energy of the many-body wavefunction in Eq.~\ref{Eq:QMC_wavefunction}.
The Slater part is instead kept frozen as determined by DFT-LDA.
As stochastic minimization algorithm, we employed the linear method \cite{Umrigar_opt}
We then carried out lattice regularized diffusion Monte Carlo (LRDMC) calculations \cite{Casula_lrdmc}, to stochastically project the initial wavefunction towards the ground state of the system, within the fixed node approximation. In LRDMC, we used a lattice space of 0.25 $a_0$, which is known to produce converged energy differences. We started the projection from the best variational state optimized in the previous step, taken as trial wavefunction. Finite-size scaling has been performed on the 2x2x1, 2x2x2, 3x2x2 and 3x3x2 real-space supercells in order to extrapolate the LRDMC total energy to the thermodynamic limit, by also using Kwee-Zhang-Krakauer (KZK) \cite{Kwee2008}  corrections to make its size dependence milder.

This workflow has been repeated for every point in the real-space grid used to interpolate the PES model from \emph{ab initio} data (see Sec.~\ref{Sec:Pes}).

\subsection{Potential energy surface parametrization}
\label{Sec:Pes}

To derive an effective low-dimensional PES, we considered the collective and concerted motion of all the hydrogen atoms of the cubic unit cell, with sulfur atoms forming a bcc sublattice. The position of a hydrogen atom is described by the cylindrical coordinates $x$, $r$ and $\phi$, defined along the axis connecting the two flanking sulfur atoms (S-S axis): $x$ is the position of the hydrogen atom along the S-S axis, $r$ is the radial distance from the S-S axis, and $\phi$ is the azimuthal angle, wrapping around the same axis. We use fractional coordinates, where the lengths are expressed in $d_\textrm{SS}$ units, $d_\textrm{SS}$ being the lattice parameter of the bcc unit cell. Within this reference system, the S-S midpoint has coordinates $(x, r, \phi) \equiv (0.5,0,0)$. We assume that all hydrogen atoms in the unit cell move in the same way. This fixes the choice of a collective path connecting the Im$\bar 3$m symmetry (with all hydrogen atoms sitting at the S-S midpoints) to the R3m one (with all hydrogen atoms coherently displaced from the midpoint). In this way, we apply a dimensionality reduction of the full potential, depending on 3$N$ dimensional coordinates, where $N$ is the number of atoms in the cell, to a much simpler 3D PES: $E=E(x,r,\phi)$.

The functional form of our 3D PES is constructed as follows:
\begin{equation}
    E(x,r,\phi) = A(x,r) + B(x,r)\sin(\phi + 5\pi/4),
    \label{Eq:E_def}
\end{equation}
with:
\begin{equation}
    A(x,r)=\frac{f_{\textrm{max}}(x,r)+f_{\textrm{min}}(x,r)}{2},\quad B(x,r)=\frac{f_{\textrm{max}}(x,r)-f_{\textrm{min}}(x,r)}{2},
    \label{Eq:AB_def}
\end{equation}
and where $f_{\textrm{min}}$ and $f_{\textrm{max}}$ are defined as:
\begin{equation}
\begin{split}
f_{\textrm{min},\textrm{max}}(x,r) = 
    & \ a +  \frac{1}{2}b (x-0.5)^2 + \frac{1}{2} c (x-0.5)^4\\
    &+ d r + \frac{1}{2}e  r^2\\
    &\pm f  (x-0.5) r \pm  g  (x-0.5) r^2 \\
    &\pm h_1 (x-0.5)^3  r \pm h_2 (x-0.5)^3  r^2\\
    &+ h_3  (x-0.5)^2  r + h_4  (x-0.5)^2  r^2 \\
    &+ h_5  (x-0.5)^4  r + h_6  (x-0.5)^4  r^2
\end{split}
\label{Eq:fmaxmin_def}
\end{equation}

The choice of this functional form 
is motivated by
the symmetries of the system. For a fixed $\lbrace x,r \rbrace$, the %energy 
potential
$E$ has an angular dependence that varies following a sine curve with $2\pi$-periodicity. In particular, for $x < 0.5$, $E$ has a minimum given by $f_{\textrm{min}}$ at $\phi=\pi/4$ and the maximum $f_{\textrm{max}}$ at $\phi=5\pi/4$. This dependence is built in Eqs.~\ref{Eq:E_def} and \ref{Eq:AB_def}. The $\{f_i(x,r)\}_{i=\textrm{min},\textrm{max}}$ functions in Eq.~\ref{Eq:fmaxmin_def} are a composition of the following terms: a Landau-type potential that well describes the energy profile for $r=0$, a second-order polynomial function in $r$ for $x=0.5$, and mixed terms made of cross products of factors up to the fourth order in $(x-0.5)$ and up to the second order in $r$, which give enough flexibility in order to well reproduce the total PES. The signs in $\{f_i(x,r)\}_{i=\textrm{min},\textrm{max}}$ ensure the symmetry: $E(1-x,r,\phi+\pi)=E(x,r,\phi)$, fulfilled by the system.

We sampled the PES by discretizing the 3D space according to the following grid defined in cylindrical coordinates: $x = \left[0.42,0.44,0.46,0.48,0.5\right]$ (in $d_{SS}$ units), $r=[0.00,0.02,0.05,0.08]$ (in $d_{SS}$ units) and $\phi=[\pi/4,5\pi /4]$. For these points we computed the \emph{ab initio} total energies, given either by DFT-BLYP or by QMC calculations. We finally used the generated datasets to best fit the PES, parametrized according to Eqs.~\ref{Eq:E_def}, \ref{Eq:AB_def} and \ref{Eq:fmaxmin_def}. The root mean square error of these fits amounts to $\approx$ 1 meV/H$_3$S.

\subsection{Equation of state}
\label{Sec:EOS}

In order to get the pressure associated to each volume, $P=P(V)$, we use the Vinet EOS:
\begin{equation}
         P(V) = 3B_0 \frac{1-\eta}{\eta^2} \exp\left(-\frac{3}{2}(B_0'-1)(1-\eta)\right).
        \label{eq:vinet_P}
\end{equation}
with $\eta=(V/V_0)^{1/3}$. In Eq.~\ref{eq:vinet_P}, the parameters $V_0$, $B_0$ and $B_0'$ are the equilibrium volume, the isothermal bulk modulus, and the derivative of bulk modulus with respect to pressure, respectively. 
The Vinet EOS \cite{Vinet_EOS} is empirical and, despite having only a few parameters, it is very accurate 
to describe solids under extreme conditions.
We obtained
$V_0$, $B_0$ and $B_0'$
by fitting the $E=E(V)$ relation for the Im$\bar{3}$m phase, where the total energy is computed from first principles, either by DFT-BLYP or by QMC, on a grid of volumes (see Fig.~\ref{fig:drozdov_mod}(a)). In the fit, we disregarded the Zero-Point Energy (ZPE) contribution, because we verified that the ZPE variation is very small ($<$ 1 mHa/H$_3$S) in the range of pressures analyzed here within the same Im$\bar{3}$m phase. 

\subsection{PIMD simulations}
\label{Sec:PIMD}
The 
PIMD simulations are carried out at 200 K using 20 beads to take into account quantum effects. Nuclei are evolved in time using the PIOUD integrator~\cite{Mouhat_2017} with a time step equal to 0.75 fs and a friction parameter of the Langevin thermostat equal to 1.46$\cdot$10$^{-3}$ atomic units. The latter value is the same as in Ref.~\cite{Mouhat_2017}, where it is found to be optimal for both stochastic and deterministic forces. Simulations lasted around 6 ps, until the convergence of the vibrational modes at $\Gamma$ is reached. Forces are computed from the Born-Oppenheimer PES evaluated at DFT level within the QE package, or from the model PES defined in Sec.~\ref{Sec:Pes}. In case of \emph{ab initio} PIMD, we used a BLYP 
functional for computing the PES. The wavefunction cut-off for the PES is set to 90 Ry (420 Ry for the charge density), while the Fermi smearing is Gaussian and set equal to 0.03 Ry. PIMD simulations are performed using 2x2x2 real-space supercells, containing in each case 32 atoms, and the corresponding reciprocal-space mesh is always equal to 9x9x9. We used a smaller plane-wave cutoff than the one used in single-point DFT calculations, because in PIMD we replaced the hard H Coulomb pseudopotential with a smoother PAW one. This has been necessary to speed up the PIMD calculations, which would otherwise have been too time consuming.

\subsection{SSCHA simulations}
\label{Sec:SSCHA}

Besides the exact description of quantum nuclear motion provided by PIMD, one can also rely on approximated theories like the SSCHA \cite{Monacelli2021}, based on a variational principle on the free energy, which allows one to include quantum nuclear anharmonicity in a non-perturbative way. Here, we performed SSCHA simulations on the 3D $\ce{H_3S}$ ($\ce{D_3S}$) model using up to 30000 configurations. The average proton position (centroid) reported in Figs.~\ref{fig:probes_BLYP} and \ref{fig:probes_QMC} are directly accessible through the SSCHA free energy imization. 

\subsection{Phonons}
\label{Sec:Phonons}

Harmonic phonons are obtained through 
DFPT
simulations \cite{Baroni_2001} as implemented within QE \cite{qe2}. The same set of DFT parameters and pseudopotentials employed for PIMD simulations were used to compute harmonic phonon dispersions, except for the \textbf{k}-space grid that was chosen equal to 18x18x18, as in this case 
it is referred to
the unit cell. The results of these calculations are shown in Fig.~\ref{fig:phonons}. We specify that the DFPT shuttling mode frequency at $\mathbf{q}=\Gamma$, reported in Fig.~\ref{fig:transition_shuttle_phon} has been computed with higher precision by employing the more accurate H Coulomb pseudopotential, requiring a plane-wave cutoff of 200 Ryd.

Anharmonic phonon frequencies at PIMD level are evaluated by computing the 
zero frequency component of the phonon Matsubara Green's function from PIMD simulations. This method has been recently implemented in Ref.~\cite{Morresi_2021} and it has been shown to describe accurately the vibron frequencies of solid phases of hydrogen. Conversely, within the SSCHA, auxiliary phonons are a byproduct of the free energy minimization. However, to get the physical phonons of Fig.~\ref{fig:transition_shuttle_phon}, probed by spectroscopies, we apply the full self-energy dynamical corrections to the auxiliary dynamical matrix, described in detail in Ref.~\cite{Monacelli2021}, including both the third and fourth-order terms.


\bmhead{Acknowledgments}

The authors thank M. Calandra, I. Errea, F. Mauri and L. Monacelli for useful discussions. They acknowledge computational resources provided by GENCI under the allocation number 0906493, which granted access to the HPC resources of IDRIS and TGCC. They also thank RIKEN for providing computational resources of the supercomputer Fugaku through the HPCI System Research Project ID hp220060.
The authors are grateful to the European Centre of Excellence in Exascale Computing TREX-Targeting Real Chemical Accuracy at the Exascale, which partially supported this work.
This project has received funding from the European Unions Horizon 2020 Research and Innovation program under Grant Agreement No. 952165.

%Version 2.1 April 2023
% See section 11 of the User Manual for version history
%
%%%%%%%%%%%%%%%%%%%%%%%%%%%%%%%%%%%%%%%%%%%%%%%%%%%%%%%%%%%%%%%%%%%%%%
%%                                                                 %%
%% Please do not use \input{...} to include other tex files.       %%
%% Submit your LaTeX manuscript as one .tex document.              %%
%%                                                                 %%
%% All additional figures and files should be attached             %%
%% separately and not embedded in the \TeX\ document itself.       %%
%%                                                                 %%
%%%%%%%%%%%%%%%%%%%%%%%%%%%%%%%%%%%%%%%%%%%%%%%%%%%%%%%%%%%%%%%%%%%%%

%%\documentclass[referee,sn-basic]{sn-jnl}% referee option is meant for double line spacing

%%=======================================================%%
%% to print line numbers in the margin use lineno option %%
%%=======================================================%%

%%\documentclass[lineno,sn-basic]{sn-jnl}% Basic Springer Nature Reference Style/Chemistry Reference Style

%%======================================================%%
%% to compile with pdflatex/xelatex use pdflatex option %%
%%======================================================%%

%%\documentclass[pdflatex,sn-basic]{sn-jnl}% Basic Springer Nature Reference Style/Chemistry Reference Style


%%Note: the following reference styles support Namedate and Numbered referencing. By default the style follows the most common style. To switch between the options you can add or remove “Numbered” in the optional parenthesis. 
%%The option is available for: sn-basic.bst, sn-vancouver.bst, sn-chicago.bst, sn-mathphys.bst. %  
 
%%\documentclass[sn-nature]{sn-jnl}% Style for submissions to Nature Portfolio journals
%%\documentclass[sn-basic]{sn-jnl}% Basic Springer Nature Reference Style/Chemistry Reference Style
\documentclass[sn-mathphys,Numbered]{sn-jnl}% Math and Physical Sciences Reference Style
%%\documentclass[sn-aps]{sn-jnl}% American Physical Society (APS) Reference Style
%%\documentclass[sn-vancouver,Numbered]{sn-jnl}% Vancouver Reference Style
%%\documentclass[sn-apa]{sn-jnl}% APA Reference Style 
%%\documentclass[sn-chicago]{sn-jnl}% Chicago-based Humanities Reference Style
%%\documentclass[default]{sn-jnl}% Default
%%\documentclass[default,iicol]{sn-jnl}% Default with double column layout

%%%% Standard Packages
%%<additional latex packages if required can be included here>

\usepackage{graphicx}%
\usepackage{multirow}%
\usepackage{amsmath,amssymb,amsfonts}%
\usepackage{amsthm}%
\usepackage{mathrsfs}%
\usepackage[title]{appendix}%
\usepackage{xcolor}%
\usepackage{textcomp}%
\usepackage{manyfoot}%
\usepackage{tabularx}
\usepackage{multirow}
\usepackage{booktabs}%
\usepackage{algorithm}%
\usepackage{algorithmicx}%
\usepackage{algpseudocode}%
\usepackage{listings}%
\usepackage{tabularx}
\usepackage{fullwidth}
\renewcommand\tabularxcolumn[1]{m{#1}}% for vertical centering text in X column
%%%%

%%%%%=============================================================================%%%%
%%%%  Remarks: This template is provided to aid authors with the preparation
%%%%  of original research articles intended for submission to journals published 
%%%%  by Springer Nature. The guidance has been prepared in partnership with 
%%%%  production teams to conform to Springer Nature technical requirements. 
%%%%  Editorial and presentation requirements differ among journal portfolios and 
%%%%  research disciplines. You may find sections in this template are irrelevant 
%%%%  to your work and are empowered to omit any such section if allowed by the 
%%%%  journal you intend to submit to. The submission guidelines and policies 
%%%%  of the journal take precedence. A detailed User Manual is available in the 
%%%%  template package for technical guidance.
%%%%%=============================================================================%%%%

%\jyear{2021}%

%% as per the requirement new theorem styles can be included as shown below
\theoremstyle{thmstyleone}%
\newtheorem{theorem}{Theorem}%  meant for continuous numbers
%%\newtheorem{theorem}{Theorem}[section]% meant for sectionwise numbers
%% optional argument [theorem] produces theorem numbering sequence instead of independent numbers for Proposition
\newtheorem{proposition}[theorem]{Proposition}% 
%%\newtheorem{proposition}{Proposition}% to get separate numbers for theorem and proposition etc.

\theoremstyle{thmstyletwo}%
\newtheorem{example}{Example}%
\newtheorem{remark}{Remark}%
\newtheorem{lemma}{Lemma}
\newtheorem{corollary}{Corollary}

\theoremstyle{thmstylethree}%
\newtheorem{definition}{Definition}%

\usepackage{csquotes}
\usepackage{float}
\usepackage{mathtools}

\usepackage{tabularx}
\usepackage{multirow}
%\def\checkmark{\tikz\fill[scale=0.4](0,.35) -- (.25,0) -- (1,.7) -- (.25,.15) -- cycle;}
\usepackage{fullwidth}
\renewcommand\tabularxcolumn[1]{m{#1}}% for vertical centering text in X column

\usepackage{csquotes}
\usepackage{float}
\usepackage{mathtools}

\usepackage{caption}
\captionsetup{labelsep=period}
\captionsetup[table]{labelfont=bf}

\raggedbottom
%%\unnumbered% uncomment this for unnumbered level heads

\begin{document}

\title[Article Title]{Statistical complexity as a criterion for the useful signal detection problem}

%%=============================================================%%
%% Prefix	-> \pfx{Dr}
%% GivenName	-> \fnm{Joergen W.}
%% Particle	-> \spfx{van der} -> surname prefix
%% FamilyName	-> \sur{Ploeg}
%% Suffix	-> \sfx{IV}
%% NatureName	-> \tanm{Poet Laureate} -> Title after name
%% Degrees	-> \dgr{MSc, PhD}
%% \author*[1,2]{\pfx{Dr} \fnm{Joergen W.} \spfx{van der} \sur{Ploeg} \sfx{IV} \tanm{Poet Laureate} 
%%                 \dgr{MSc, PhD}}\email{iauthor@gmail.com}
%%=============================================================%%

\author*[1]{\fnm{Leonid} \sur{Berlin}}\email{berlin.lm@phystech.edu}
\equalcont{These authors contributed equally to this work.}

\author[1]{\fnm{Andrey} \sur{Galyaev}}\email{galaev@ipu.ru}
\equalcont{These authors contributed equally to this work.}

\author[1]{\fnm{Pavel} \sur{Lysenko}}\email{pavellysen@ipu.ru}
\equalcont{These authors contributed equally to this work.}

\affil[1]{\orgdiv{Laboratory 38}, \orgname{Institute of Control Sciences of RAS}, \orgaddress{\city{Moscow}, \country{Russia}}}

%%==================================%%
%% sample for unstructured abstract %%
%%==================================%%

\abstract{Three variants of the statistical complexity function, which is used as a criterion in the problem of detection of a useful signal in the signal-noise mixture, are considered. The probability distributions maximizing the considered variants of statistical complexity are obtained analytically and conclusions about the efficiency of using one or another variant for detection problem are made.
The comparison of considered information characteristics is shown and analytical results are illustrated on an example of synthesized signals. A method is proposed for selecting the threshold of the information criterion, which can be used in decision rule for useful signal detection in the signal-noise mixture. The choice of the threshold depends a priori on the analytically obtained maximum values. As a result, the complexity based on the total variation demonstrates the best ability of useful signal detection.}
%%================================%%
%% Sample for structured abstract %%
%%================================%%

% \abstract{\textbf{Purpose:} The abstract serves both as a general introduction to the topic and as a brief, non-technical summary of the main results and their implications. The abstract must not include subheadings (unless expressly permitted in the journal's Instructions to Authors), equations or citations. As a guide the abstract should not exceed 200 words. Most journals do not set a hard limit however authors are advised to check the author instructions for the journal they are submitting to.
% 
% \textbf{Methods:} The abstract serves both as a general introduction to the topic and as a brief, non-technical summary of the main results and their implications. The abstract must not include subheadings (unless expressly permitted in the journal's Instructions to Authors), equations or citations. As a guide the abstract should not exceed 200 words. Most journals do not set a hard limit however authors are advised to check the author instructions for the journal they are submitting to.
% 
% \textbf{Results:} The abstract serves both as a general introduction to the topic and as a brief, non-technical summary of the main results and their implications. The abstract must not include subheadings (unless expressly permitted in the journal's Instructions to Authors), equations or citations. As a guide the abstract should not exceed 200 words. Most journals do not set a hard limit however authors are advised to check the author instructions for the journal they are submitting to.
% 
% \textbf{Conclusion:} The abstract serves both as a general introduction to the topic and as a brief, non-technical summary of the main results and their implications. The abstract must not include subheadings (unless expressly permitted in the journal's Instructions to Authors), equations or citations. As a guide the abstract should not exceed 200 words. Most journals do not set a hard limit however authors are advised to check the author instructions for the journal they are submitting to.}

\keywords{statistical complexity, signal detection, information divergence}

%%\pacs[JEL Classification]{D8, H51}

%%\pacs[MSC Classification]{35A01, 65L10, 65L12, 65L20, 65L70}

\maketitle

\section{Introduction}\label{Intro}

The concept of information entropy was firstly introduced in Claude Shannon's article \cite{Shannon} in 1948. This work marked the beginning of a new field of science called information theory \cite{EntInf}. The development of information theory made possible an analytical and practical research in many applied fields of science and technology. Such terms as Gibbs and von Neumann entropies, Kullback-Leibler distance, Jensen-Shannon divergence, information divergences and some others were introduced and interpreted and later began to serve as criteria for various optimization problems of recognition \cite{recognition}, classification \cite{classification} and filtering.

By the end of the last century various information criteria, mainly Shannon information entropy, had began to be actively applied in the tasks of digital signal processing, in particular in the problem of detection of a useful signal in a noise environment \cite{RobustDetection}. The concept of spectral entropy \cite{EntUni}, associated with the Fourier spectrum of the considered signal, has appeared and proved to be especially relevant in the analysis of acoustic signals \cite{VAD}. In addition, the entropic approach has been successfully applied in the analysis of time series in the medical field, such as ECG or EEG \cite{ECG}. Later, a statistical complexity function was proposed as a development of the entropy concept \cite{ComplexIntro, Complexity, Complex_???}. However, the articles mostly do not provide an analytical study of the properties of these functions, which turns out to be especially important when solving the problem of hypothesis testing. 

It should be noted that there are several classical ways of solving the detection problem. The first of them is based on solving the problem of optimal filtering and requires knowledge of the properties of the signal: periodicity, bandwidth, etc. \cite{SignalAnalysis}. The second way is based on the Neumann-Pearson Lemma, solves the problem of hypothesis testing, and determines the fact of exceeding the optimal threshold at a given false alarm probability and requires estimation of statistical properties of sample distributions of noise and mixture of signal and noise \cite{Shiryaev}. The third way is equivalent to solving the changepoint detection problem when the unknown statistical characteristics of the signal distributions change. The anomaly detection problem \cite{Anomaly} has a similar formulation. All these methods demonstrate qualitative and reliable performance when the signal exceeds the noise, but for small signal-to-noise ratios often give the wrong answer.

The article is devoted to the problem of detection of useful signal in the signal-noise mixture and combines all three previously listed ways of solving the detection problem. We propose to use a variant of the Neumann-Pearson Lemma for the problem of hypothesis testing \cite{Shiryaev}, which is indeed valid when the error probability is close to one and depends on the total variation of the measure of two distributions of the null and alternative hypotheses. Based on the analytical expression of this error function, the criterion of the signal detection problem is formalized as one of the variants of the statistical complexity \cite{Sensors}, which takes into account the deterministic nature of the signal mixed in with the noise. The peculiarity of the statistical complexity is that it is multiplicative and consists of two multipliers, one of which is zero on deterministic sinusoidal signals of the same frequency (in physics these are objects of a given structure, such as crystals \cite{ComplexIntro}) and the other is zero on uniform distribution functions \cite{Complexity}, corresponding, for example, to white noise. Then the introduced criterion is compared with the already known two variants of statistical complexity based on Euclidean distance square and Jensen-Shannon divergence, their properties are established, and optimization as a function of many variables on a set of discrete distributions is performed. As a result, families of optimal distributions are identified and maxima of statistical complexity functions are calculated.

The article has the following structure. Section \ref{Intro} provides a literature review and highlights the current state of research on the topic of the article. Section \ref{hypothesis} is devoted to the connection of the considered information criteria  with the classical criterion of the signal detection problem. Section \ref{statoptimize} investigates the properties of the three types of statistical complexity. In \ref{Modelling} the analytical results of the previous section are supported by numerical simulations for synthesized signals. Section \ref{Conclusion} summarizes the results obtained in the paper and lists plans for the future.

\section{Neyman-Pearson Lemma and statistical complexity}\label{hypothesis}

The problem of signal detection $s(n)$ is traditionally reduced to the problem of hypothesis testing
\begin{equation*}
    \left\{
    \begin{array}{lll}
    \Gamma_0: x(n) = w(n),\\
    \Gamma_1: x(n) = s(n) + w(n),~n = 1, \dots, N.
    \end{array}
    \right.
\end{equation*}
Hypothesis $\Gamma_0$ corresponds to the decision of receiving only noise, and hypothesis $\Gamma_1$ -- of receiving a mixture of useful signal and noise, where the sequences $\{x(n)\},~n = 1, \dots, N$ are time series of the received data, $\{s(n)\}$ -- useful signal, $\{w(n)\}$ --additive white Gaussian noise, $N$ -- the length of the time series of data.

The random variables of the time series $(x(1),\dots,x(n),\dots,x(N))$ take values $(x_1,\dots,x_n,\dots,x_N) \in \mathbb{R}^N$.
 
In order to obtain an analytical expression for estimating the error probability in hypothesis testing, we can apply a variant of the Neyman-Pearson Lemma \cite{Shiryaev, DetectingDook}.

\begin{lemma} [Neyman-Pearson] \label{NeymanPearson}
Let there be an arbitrary, called a decision rule or test, measurable function of many variables \mbox{$(x_1, \dots, x_N) \in \mathbb{R}^N$} such that
\begin{equation*} 
d(x_1, ..., x_N)= \left\{{}
\begin{array}{l}
    1, ~\textrm {hypothesis}~\Gamma_0 ~\textrm {is true},\\\\
    0, ~\textrm {hypothesis}~\Gamma_1 ~\textrm {is true}, 
\end{array}
\right.
\end{equation*}
by which the following probabilities can be determined:

\begin{equation*}
    \begin{array}{cc}
    \alpha(d)=\textrm {Probability~(accept~} \Gamma_0 | \Gamma_1~\textrm {is true}),\\\\
    \beta(d)=\textrm {Probability~(accept~} \Gamma_1 | \Gamma_0~\textrm {is true}).
    \end{array}
\end{equation*}

Then the decision rule $d^*$ is optimal if

\begin{equation}
\alpha(d^*)+\beta(d^*)=\inf_d[\alpha(d)+\beta(d)]=\mathcal{E}r(N;\Gamma_0,\Gamma_1)~\text{-- error function},
\end{equation}
where the infinum is taken for all tests.
\end{lemma}
Here $\alpha(\cdot)$ is the probability of a false alarm, and $\beta(\cdot)$ is the probability of a useful signal missing.

The exact formula for the error function is as follows:
\begin{equation}\label{ErrTVR}
\displaystyle \mathcal{E}r(N;\Gamma_0,\Gamma_1)=1-\frac{1}{2}\lVert P_0^{(N)}-P_1^{(N)}\rVert = 1 - TV(P_0, P_1),
\end{equation}
where $P_0^{(N)}$ is the multivariate distribution function of the observation statistics by hypothesis $\Gamma_0$, $P_1^{(N)}$ is the multivariate distribution function of the observation statistics by hypothesis $\Gamma_1$, and $TV(P_0, P_1)$ is the total variation of the signed measure, \mbox{$\|Q\|=2\sup_A|Q(A)|$}. Thus, if the supports of measures $P_0$, $P_1$ do not overlap, then error-free distinguishing of hypotheses is possible. If the measures $P_0^{(N)}$ and $P_1^{(N)}$ are close, then $\lVert P_0^{(N)}-P_1^{(N)}\rVert \approx0$, leading to $\mathcal{E}r(N;\Gamma_0,\Gamma_1)\approx 1$.

For the problem of detecting a deterministic useful signal, for example, at a small signal-to-noise ratio, the case $\lVert P_0^{(N)}-P_1^{(N)}\rVert = 2 TV(P_0, P_1) \approx 0$ is of interest and the possibility of reasonable estimation of this value. Therefore, when the probability of the total error of distinguishing two hypotheses is close to one, it becomes possible to use the analytical expression $TV(P_0, P_1)$ to design a criterion in the problem of detecting a useful signal in a mixture. But first let us turn to already known criteria and establish their properties.

Most often, for the convenience of mathematical investigation, both of the useful signal and noise are modeled by Gaussian random processes with different parameters. In that case the problem of finding the moment of appearance of the signal $s(n)$ in the received sequence of samples is called the the problem of changepoint detection \cite{Shiryaev}.

Here and below, we consider discrete probability distributions $p = (p_1,\dots,~p_i,\dots,p_N)$, that by definition have the following properties:
\begin{equation}\label{discdefinit}
    \forall~ p_i \in [0, 1], \quad \sum_{i=1}^N p_i = 1.
\end{equation}

To formalize criterion that takes into account the deterministic component of the signal as well as the random one, let us explore the concepts of disequilibrium function $D$ and statistical complexity $C$ of the distribution. The simplest example of the disequilibrium function  is the square of Euclidean distance in the space of discrete probability distributions\cite{Complexity}.
\begin{definition}\label{Def1}
The disequilibrium $D_{SQ}$ has the meaning of the variance of a distribution relative to a uniform distribution
\begin{equation}\label{disequ_1}
    \displaystyle D_{SQ}(p)= \sum_{i=1}^N\left(p_i - \frac{1}{N} \right)^2=\sum_{i=1}^N p_i^2 - \frac{1}{N} .
\end{equation}
\end{definition}

\begin{definition}\label{Def2}
The statistical complexity, defined through the expression of disequilibrium by the Definition \ref{Def1}, is equal to
\begin{equation}\label{compSQ}
   C_{SQ}(p)= H(p)\cdot D_{SQ}(p),
\end{equation}
where
\begin{equation}\label{shannon_entropy}
    H(p) = \frac{1}{\log N}\left(-\sum_i^N p_i\log p_i \right)
\end{equation}
-- Shannon's normalized entropy \cite{Shannon}.
\end{definition}

In evaluating the sum \eqref{shannon_entropy}, it is assumed that $\displaystyle \frac{0}{\log 0} = 0$ by continuity, and this assumption holds for all subsequent equations.

It follows from the Definition \ref{Def1} that disequilibrium of the form (\ref{disequ_1}) and complexity of the form (\ref{compSQ}) are convenient to apply in estimation and comparison of signals having spectral distribution close to uniform. In general, instead of a uniform distribution $q_i=1/N$ at $i=1,...,N$, the formula (\ref{disequ_1}) may include an arbitrary discrete distribution. 

The formula \eqref{disequ_1} is proposed in \cite{Complexity} for computing the disequilibrium with respect to a uniform distribution, but most studies use the Jensen-Shannon divergence $JSD(p || q)$ \cite{Features} instead.

\begin{definition} \label{Def3}
The Jensen-Shannon disequilibrium equals
\begin{equation}\label{DJSD}
D_{JSD}(p)=JSD(p||q),
\end{equation}
where $q=(1/N, \dots, 1/N)$ is the uniform distribution.
\end{definition}
\begin{definition}\label{Def4}
Statistical complexity defined through the expression of disequilibrium from Definition \ref{Def3}, is expressed as
\begin{equation}\label{compJSD}
    C_{JSD}(p) = H(p)\cdot D_{JSD}(p).
\end{equation}
\end{definition}

\begin{remark}
It was noted above that $\displaystyle\sqrt{D_{SQ}}$ is a Euclidean metric on the space of discrete distributions. At the same time $\displaystyle\sqrt{D_{JSD}}$ is also a metric which is proportional to the Fisher metric.
\end{remark}

Since the error function of distinguishing between two hypotheses depends on the total variation $TV(p,q)$, which is obtained in the Neyman-Pearson Lemma \ref{NeymanPearson}, we introduce another notion of disequilibrium.

\begin{definition}\label{Def5}
The disequilibrium based on the total variation of signed measure is equal to
\begin{equation}\label{DTV}
D_{TV}(p)=TV^2(p,q),
\end{equation}
where $q=(1/N, \dots, 1/N)$.
\end{definition}
\begin{definition}\label{Def6}
The statistical complexity, defined through the disequilibrium expression according to the Definition \ref{Def5}, is equal to
\begin{equation}\label{compTV}
    C_{TV}(p) = H(p)\cdot D_{TV}(p).
\end{equation}
\end{definition}

The information divergence functions presented above, which define different variants of the disequilibrium function, can be unified by the general concept of \textit{f--divergence} \cite{fdivergence}:
\begin{equation}\label{generalfdiv}
    D_f(p||q) = \sum_{x \in \mathbb{R}^N} q(x) f\left(\frac{p(x)}{q(x)}\right).
\end{equation}

The choice of function $f$ gives rise to a whole family of different divergences:

\begin{itemize}
    \item The Kulback-Leibler divergence $D_{KL}(p, q)$ is obtained from \eqref{generalfdiv} by choosing $f(x) = x\log(x),~x > 0$.
    \item The Jensen-Shannon divergence is obtained from \eqref{generalfdiv} by choosing 
    \begin{equation}
        \displaystyle f(x) = x\log\frac{2x}{x+1} + \log \frac{2}{x+1}, ~x > 0.
    \end{equation}
     \item The total variation is obtained when $\displaystyle f(x) = \frac{1}{2}|1 - x|$:
    \begin{equation}\label{TVequ}
        \displaystyle TV(p, q)={\frac {1}{2}}\sum _{x \in \mathbb{R}^N }|p({x)-q(x)|};
    \end{equation}
 $TV(p,q)$ is also a metric on the space of probability distributions.
    The total variation is related to the Jensen-Shannon divergence by the following relation:
    \begin{equation}\label{TVvsJSD}
        JSD(p || q) \le TV(p,q).
    \end{equation}
\end{itemize}

It follows from the inequality (\ref{TVvsJSD}) that the total variation is the upper bound of Jensen-Shannon divergence.

Next, let us investigate the possibility of using each variant of statistical complexity as a criterion for indicating the appearance of a signal, but at first their properties must be established.

\section{Statistical complexity optimization}\label{statoptimize}
\subsection{Optimization of $C_{SQ}$}
Let us formulate the problem of maximizing the statistical complexity function on the set of discrete distributions $p=(p_1,...,p_N)$
\begin{equation}\label{comp}
   C_{SQ}(p)= \frac{1}{\log N}\left(-\sum_{i=1}^N p_i\log p_i \right)\cdot \left(\sum_{i=1}^N \left(p_i - \frac{1}{N} \right)^2\right)\longrightarrow \max_{p}
\end{equation}
with the condition
\begin{equation}\label{norm}
    \sum_{i=1}^N p_i = 1.
\end{equation}


An auxiliary result will be needed to formulate the Lemma about the maximum value of statistical complexity.

\begin{lemma}\label{lemma3}
Let $ 0< x\leq y \leq z\leq 1$, then $f(x,y,z)=x^y y^{-x}z^x x^{-z}y^z z^{-y}\geq 1$, with equality possible only when either $x=y$ or $y=z$.
\end{lemma}
\begin{proof}
Let us introduce a new function $g(x,y,z)=\ln f(x,y,z)$, 
$$
g(x,y,z)=y\ln x-x\ln y+x\ln z-z\ln x+z\ln y-y\ln z.
$$
Then it is required to prove that $g(x,y,z)\geq 0$ for $ 0 <x\leq y \leq z\leq 1$.

By the Kuhn-Tucker theorem, the solution of the conditional optimization problem of a function of three variables is either at the interior point of the constraint manifold or at its boundary.
The necessary conditions for the unconditional extremum of the function $g(x,y,z)$ take the following form
\begin{equation}\label{ENC}
\begin{array}{l}
\displaystyle \frac{\partial g}{\partial x}=\ln z-\ln y+\frac{y-z}{x}=0,\\
\displaystyle \frac{\partial g}{\partial y}=\ln x-\ln z+\frac{z-x}{y}=0,\\
\displaystyle \frac{\partial g}{\partial z}=\ln y-\ln x+\frac{x-y}{z}=0.
\end{array}
\end{equation}
Let us summarize all the equations of the last system:
\begin{equation*}
\frac{y-z}{x}+\frac{z-x}{y}+\frac{x-y}{z}=0,
\end{equation*}
which can be rewritten as
\begin{equation*}
\frac{(y-z)(x-y)(z-x)}{xyz}=0.
\end{equation*}
This means that when one of the equalities either $x=y$ or $y=z$ is satisfied, the function $g(x,y,z)$ possibly has a minimum. Let $x=y$, then the third equation from (\ref{ENC}) is fulfilled identically, and the first and second equations are identical and can be written as
\begin{equation*}
\displaystyle \ln \eta=\eta-1,~~\eta=\frac{z}{y}.
\end{equation*}
The last equation has only one root $\eta=1$, i.e. $y=z$.

Let us calculate the second derivatives and write the Hesse matrix:
\begin{equation}
G(x,y,z)=\left(\begin{array}{lll}
\displaystyle \frac{z-y}{x^2} & \displaystyle\frac{1}{x}-\frac{1}{y} & \displaystyle\frac{1}{z}-\frac{1}{x}\\
\displaystyle \frac{1}{x}-\frac{1}{y} & \displaystyle \frac{x-z}{y^2} & \displaystyle\frac{1}{y}-\frac{1}{z}\\
\displaystyle \frac{1}{z}-\frac{1}{x} & \displaystyle\frac{1}{y}-\frac{1}{z} & \displaystyle \frac{y-x}{z^2}
\end{array}
\right).
\end{equation}
Minors of the Hesse matrix are equal to
\begin{equation}
\begin{array}{c}
\displaystyle M_1(x,y,z)=\frac{z-y}{x^2},~M_2(x,y,z)=-\frac{(x-y)^2+(z-x)^2+(y-z)^2}{2x^2y^2},\\
\displaystyle M_3(x,y,z)=0.
\end{array}
\end{equation}

The Hesse matrix is not sign-defined, moreover, its determinant equals zero. Therefore, let us consider a small vicinity of the extremum point.

In a small vicinity $x=y=z$, provided that $\delta x\leq \delta y\leq \delta z$, the variation $\delta g$ of the function $g(x,y,z)$ is written in the form of
\begin{equation*}
\begin{array}{lll}
\delta g=(x+\delta y)\ln (x+\delta x)-(x+\delta x)\ln (x+\delta y)+(x+\delta x)\ln (x+\delta z)-\\
-(x+\delta z)\ln (x+\delta x)+(x+\delta z)\ln (x+\delta y)-(x+\delta y)\ln (x+\delta z)=\\
=(\delta z-\delta x)(\delta y-\delta x)(\delta z-\delta y)+o(((\delta x)^2+ (\delta y)^2+(\delta z)^2)^{3/2})\geq 0,
\end{array}
\end{equation*}
where values in the cubes of variations of the independent variables are nonzero, and the variation of the function $g(x,y,z)$ itself is positive by virtue of the Lemma conditions.  In the case when, for example, $\delta y=\delta x$, we have $g(x,y,z)\equiv 0$, and $f(x,y,z)\equiv 1$. Therefore, the extremum of the function is its non-strict minimum.
\end{proof}


\begin{lemma}\label{lemmSQ}
The maximum statistical complexity (\ref{comp}) is achieved on the distribution of the form 
\begin{equation}\label{maximal}
    \left\{
\begin{array}{lll}
         \displaystyle p_i = \frac{1-p_{\max}}{N-1}, \quad i = \overline{1,N}~\backslash ~k,\\
         \displaystyle p_k = p_{\max},         
    \end{array}
    \right.
\end{equation}
where $p_{\max} = const$, i.e., at the appearance of a single component of an arbitrary index $k$ over the uniform distribution.
\end{lemma}
\begin{proof}
Without loss of generality, let us assume $k=N$. From equation (\ref{norm}) one variable $p_N$ from the set ${p_i}$ can be expressed through all the others:
\begin{equation}\label{p_n_equ}
    p_N = 1 - \sum_{i=1}^{N-1} p_i.
\end{equation}
Let us rewrite the equation \eqref{comp} in the form
\begin{equation}\label{compnew}
   C_{SQ}= -\frac{1}{\log N}\left(\sum_{i=1}^{N-1} p_i\log p_i + p_N \log p_N\right)\cdot \left(\sum_{i=1}^{N-1} \left(p_i - \frac{1}{N} \right)^2 + \left(p_N - \frac{1}{N}\right)^2\right).
\end{equation}
A necessary condition for the extremum of a function at an interior point of the domain (simplex \ref{discdefinit}) is that all partial derivatives of $p_i$ are equal to zero:
\begin{equation}\label{conditi}
    \frac{\partial C_{SQ}}{\partial p_i} = 0, \quad i  = 1, \dots, N - 1.
\end{equation}
Substituting the function \eqref{compnew} into \eqref{conditi} gives (provided that $\displaystyle \frac{\partial p_N}{\partial p_i} = -1$):
\begin{equation}
\begin{array}{ccc}
    \displaystyle \frac{\partial C_{SQ}}{\partial p_i} = -\frac{1}{\log N}\left( \log p_i -\log p_N  \right)\cdot \left(\sum_{i=1}^{N-1} \left(p_i - \frac{1}{N} \right)^2 + \left(p_N - \frac{1}{N}\right)^2\right) - \\
    \displaystyle - \frac{2}{\log N}\left(\sum_{i=1}^{N-1} p_i\log p_i + p_N \log p_N\right)\cdot  \left(p_i - p_{N} \right) = 0, \quad i  = 1, \dots, N - 1.
\end{array}    
\end{equation}
In a more convenient form the equations can be rewritten as
\begin{equation}\label{eq_proof_1}
    \displaystyle \frac{\partial C_{SQ}}{\partial p_i} = \frac{1}{\log N}\left( -\log p_i + \log p_N \right)\cdot D + 2H \cdot \left(p_i - p_N \right)= 0, \quad i  = 1, \dots, N - 1.
\end{equation}
Let us write the difference of any two equations from the system above for indices $i$ and $j$:
\begin{equation}\label{eq_proof_2}
\begin{array}{ccc}
    \displaystyle \frac{\partial C_{SQ}}{\partial p_i} - \frac{\partial C_{SQ}}{\partial p_j} = \frac{1}{\log N}\left( -\log p_i + \log p_j \right)\cdot D + 2H \cdot \left(p_i - p_j \right)= 0.
\end{array} 
\end{equation}
Given that the values of $D$ and $H$ are positive, the following equations can be constructed from the equations \eqref{eq_proof_1} and \eqref{eq_proof_2}, provided that the considered probabilities $p_j,~j=1,\ldots,N-1$ are not equal to $p_N$:
\begin{equation}\label{eq_proof_3}
\begin{array}{ccc}
    \displaystyle \frac{ \log p_i - \log p_j }{ \log p_N - \log p_j }-\frac{p_i - p_j }{p_N - p_j }=0,
\end{array} 
\end{equation}
\begin{equation}\label{eq_proof_4}
    \displaystyle (p_N-p_j)\log p_i+(p_i-p_N)\log p_j+(p_j-p_i)\log p_N=0,
\end{equation}
\begin{equation}\label{eq_proof_5}
    \displaystyle  p_i^{p_N-p_j} \cdot p_j^{p_i-p_N} \cdot p_N^{p_j-p_i}=1.
\end{equation}

After applying the Lemma \ref{lemma3} we conclude that the last equation can be satisfied when $p_i = p_j$.

Thus, it is obtained that each of the probabilities $p_i$ can take one of two different values, which define a distribution of the form
\begin{equation}\label{distrK}
\left\{
\begin{array}{l}
    \displaystyle p_i = \frac{1-p_{\max}}{K}, \quad \forall ~  i = 1, \dots, K,\\ 
 \displaystyle p_i = p_N = \frac{p_{\max}}{N-K}, \quad \forall ~  i = K+1, \dots, N.
\end{array}
\right.
\end{equation} 
Now we need to show that the maximum complexity corresponds to values $K=1$ and $K=N-1$.  For this purpose, let us calculate the value of the disequilibrium \eqref{disequ_1} on the distribution (\ref{distrK}), which we denote by $D^{(K)} (\omega,p_{\max})$:
\begin{equation}\label{disbKSQ}
D^{(K)}(\omega,p_{\max})=   \frac{1}{N}\frac{(p_{\max}+\omega-1)^2}{\omega(1-\omega)}, ~~~\omega=\frac{K}{N}.
\end{equation}
In turn, entropy is equal to
\begin{equation}\label{entrKSQ}
 \displaystyle H^{(K)}(\omega,p_{\max})= 1-\frac{1}{\log N}\left((1-p_{\max})\log\frac{1-p_{\max}}{\omega} +p_{\max}\log\frac{p_{\max}}{1-\omega}\right).
\end{equation}

The maximum of $C_{SQ}(\omega,p_{\max})$ at $N\leq 100$ was investigated numerically, and it was reached at $K=1$. From the expression for $D^{(K)}(\omega,p_{\max})$ (\ref{disbKSQ}), it can be seen that at $N\geq 101$ and when changing from $K=1$ to $K=2$ or from $K=N-1$ to $K=N-2$, its value changes by almost a factor of two, while the entropy (\ref{entrKSQ}) changes only slightly.
Thus, the probability distribution (\ref{distrK}) that delivers the complexity function to the extremum value at $K=1$ or $K=N-1$ is of the form (\ref{maximal}).
\end{proof}
For clarity Fig. \ref{Pic101} shows the graph $C_{SQ}=C_{SQ}(\omega,p_{\max})$ at $N=1024$, where $\omega$ is changing continuously (although $K$ is changing discretely).


% Figure environment removed


\begin{corollary}
Let us substitute the values of $p_i$ and $p_N=p_{\max}$ from (\ref{maximal}) into (\ref{compnew}) and consider the complexity $C_{SQ}$ as a function of $p_{\max}$. For sufficiently large values of $N$, it will take the following form
$$
C_{SQ} \approx (1-p_{\max})\cdot p_{\max}^2.
$$    
Whence it follows that this function takes the maximum value $C_{SQ}^* \approx 4/27$ when $p_{\max}=2/3$.
\end{corollary}

\begin{corollary}\label{corSQmin}
The minimum value of $C_{SQ}=0$ is achieved on a uniform distribution $\displaystyle p = (1/N, \dots, 1/N)$.
\end{corollary}

The validity of the Lemma \ref{lemmSQ} for the case when the discrete distribution $p=\{p_1,p_2,p_3\}$ consists of three samples is demonstrated in Fig. \ref{Pic10}. The complexity depends on two variables, since one of the probabilities can be expressed through the others. Here $C_{SQ}$ has three identical pronounced maxima and three identical local minima pertaining to the cases $p_1=p_2$, $p_2=p_3$, $p_1=p_3$ when the necessary extremum conditions are met, and a global minimum when $p_1=p_2=p_3$.


% Figure environment removed

Table \ref{tableSQ} shows the change of optimal parameters $C_{SQ}(w,p_{\max})$ with increasing $N$. 
\def\arraystretch{1.8}
\begin{table}[h]
\caption{Optimal parameters $C_{SQ}(\omega, p_{max})$ for different values of $N$.}\label{tableSQ}
\centering
\begin{tabularx}{0.88\textwidth} { 
  | >{\centering\arraybackslash}X 
  | >{\centering\arraybackslash}X 
  | >{\centering\arraybackslash}X 
  | >{\centering\arraybackslash}X 
  | >{\centering\arraybackslash}X | }
 \hline
{ $N$ } & $C_{SQ}(\omega^*,p_{\max}^*)$ & $p_{\max}^*$  & $\omega^*$ & $N-K^*$ \\ 

 \hline
$3$  & $0,1932$ & $0,8315$  & $0,6666$  & $1$ \\
 \hline
$256$  & $0,1994$ & $0,7044$  & $0,9960$  & $1$ \\

 \hline
$512$  & $0,1942$ & $0,7008$  & $0,9980$  & $1$ \\
 \hline
 $1024$  & $0,1898$ & $0,6979$  & $0,9990$  & $1$ \\
 \hline
  $2048$  & $0,1861$ & $0,6955$  & $0,9995$  & $1$ \\
 \hline
\end{tabularx}
\end{table}

The necessary extremum conditions $C_{SQ}$ for the discrete distribution $p=\{p_1=x,~p_2=y,~p_3=1-x-y\}$ are written out according to \eqref{eq_proof_1} as follows:
\begin{equation}\label{necess_cond}
\begin{cases}
    \displaystyle \left( -\log x + \log (1-x-y) \right) \left(\left(x-\frac{1}{3}\right)^2+\left(y-\frac{1}{3}\right)^2+\left(1-x-y-\frac{1}{3}\right)^2\right) -\\- 2(x\log x+y\log y+(1-x-y)\log (1-x-y))  \left( -1+y) \right)= 0,\\
    \displaystyle\left( -\log y + \log (1-x-y) \right) \left(\left(x-\frac{1}{3}\right)^2+\left(y-\frac{1}{3}\right)^2+\left(1-x-y-\frac{1}{3}\right)^2\right) -\\- 2(x\log x+y\log y+(1-x-y)\log (1-x-y))  \left(-1+x \right)= 0.\\
    \end{cases}
\end{equation}

The implicit equations of the system \eqref{necess_cond} describe the curves shown in Fig. \ref{extr_cond_3}.
% Figure environment removed


The black and green curves correspond to the first and second implicit equations of the \eqref{necess_cond} system, respectively. In Fig. \ref{extr_cond_3} seven extremum points are marked, for which the value of statistical complexity is calculated. All the obtained data is summarized in Table \ref{tableextrem}.

\def\arraystretch{1.8}%  1 is the default, change whatever you need
\begin{table}[h]
\caption{Extremum points of statistical complexity at $N=3$.}\label{tableextrem}
\centering
\begin{tabularx}{0.9\textwidth} { 
  | >{\centering\arraybackslash}X 
  | >{\centering\arraybackslash}X 
  | >{\centering\arraybackslash}X 
  | >{\centering\arraybackslash}X 
  | >{\centering\arraybackslash}X 
  | >{\centering\arraybackslash}X 
  | >{\centering\arraybackslash}X 
  | >{\centering\arraybackslash}X | }
 \hline
{ $p$ } & 1 & 2  & 3 & 4 & 5 & 6 & 7\\ 

 \hline
$p_1$&   $0,08425$ & $0,006$  & $0,08425$  & $0,497$ & $0,8315$ & $0,497$ & $0,(3)$\\
 \hline
$p_2$ & $0,08425$ & $0,497$  & $0,8315$  & $0,497$ & $0,08425$ & $0,006$ & $0,(3)$\\
 \hline
$p_3$ & $0,8315$ & $0,497$  & $0,08425$  & $0,006$ & $0,08425$ & $0,497$&$0.(3)$\\
\hline
$C_{SQ}$ & $0,1932$ & $0,1062$  & $0,1932$  & $0,1062$ & $0,1932$ & $0,1062$&$0$\\
 \hline
\end{tabularx}
\end{table}


The first, third and fifth points of maximum correspond to the same value of the maximum of the function. The second, fourth and sixth minimum points correspond to the same value of the minimum of the function. It is worth noting that the extremum points correspond to the case $K=N-1=2$ except for the global minimum, where all probabilities are equal to each other, and thus describe three local minima, one global minimum, and three equal maxima of statistical complexity in Fig. \ref{Pic10}. We can separately plot the statistical complexity in the case $p=\{p_1=x,p_2=x,p_3=1-2x\}$.



% Figure environment removed
In Fig. \ref{csq_2x} the extremum points are marked according to Table {\ref{tableextrem}, which cover all cases $p=\{p_1=x,~p_2=y,~p_3=1-x-y\}$.

\subsection{Optimization of $C_{JSD}$}\label{JSDoptimize}

Let us apply a similar approach using the Jensen-Shannon divergence as the disequilibrium to the statistical complexity $C_{JSD}$
\begin{equation}
    C_{JSD}(p) = H(p)\cdot JSD(p||q), \quad q_j=1/N,~j=1,\ldots,N,
\end{equation}
which can be written by considering the expression $JSD(p||q)$ through entropy
\begin{equation}\label{CJSD}
    C_{JSD}(p) = H(p)\cdot \left(H(m) - \frac{1}{2}(H(p)+H(q))\right)\cdot\log N, \quad m=\frac{p+q}{2}.
\end{equation}

It is possible to write out the necessary conditions of extremum for the statistical complexity of the form (\ref{CJSD}), but at the same time a Lemma similar to the Lemma \ref{lemmSQ} cannot be proved.

The equation (\ref{CJSD}) is written in variables $p_i$ and $p_N$ in accordance with the approach of the Lemma \ref{lemmSQ}
\begin{equation}
\hspace{-1cm}
    C_{JSD}(p) = -\left(\sum_{i=1}^{N-1} p_i\log p_i + p_N \log p_N\right)\cdot \left(H(m) - \frac{1}{2}\left(-\frac{1}{\log N}\left(\sum_{i=1}^{N-1} p_i\log p_i + p_N \log p_N\right)+1\right)\right),
\end{equation}
where
\begin{equation}
    \displaystyle H(m) = -\frac{1}{\log N}\left(\sum_{i=1}^{N-1} \frac{p_i + \frac{1}{N}}{2}\log \frac{p_i + \frac{1}{N}}{2} + \frac{p_N + \frac{1}{N}}{2}\log \frac{p_N + \frac{1}{N}}{2}\right).
\end{equation}
Then taking into account \eqref{p_n_equ}
\begin{equation}
    \displaystyle \frac{\partial H(m)}{\partial p_i} = -\frac{1}{\log N}\left(\frac{1}{2}\log \frac{p_i + \frac{1}{N}}{2} - \frac{1}{2}\log \frac{p_N + \frac{1}{N}}{2}\right),~i=1,\ldots,N-1.
\end{equation}
The necessary conditions of extremum are obtained in the following form after combining all partial derivatives:
\begin{equation}
\begin{array}{cc}
    \displaystyle \frac{\partial C_{JSD}(p)}{\partial p_i} =\displaystyle \frac{H(p)}{2}\cdot \left(-\left(\log \frac{p_i + \frac{1}{N}}{2} - \log \frac{p_N + \frac{1}{N}}{2}\right) +\left( \log p_i -\log p_N  \right)\right)- \\
    \displaystyle-\left( \log p_i -\log p_N  \right)\cdot JSD(p||q)= 0,~i=1,\ldots,N-1.
\end{array}
\end{equation}
The equations after simplification are given:
\begin{equation}\label{NECCJSD}
\begin{array}{cc}
    \displaystyle \displaystyle H(p)\cdot \left(\log \frac{p_i + \frac{1}{N}}{2} - \log \frac{p_N + \frac{1}{N}}{2}  \right)
    +\left( \log p_i -\log p_N  \right)\cdot \left(2JSD(p||q)-H(p)\right) = 0,~i=1,\ldots,N-1.
\end{array}
\end{equation}
Then the difference of equations (\ref{NECCJSD}) for indices $i,j$ takes the following form
\begin{equation}\label{eq75}
\begin{array}{cc}
    \displaystyle \left( \log p_i -\log p_j  \right)\cdot \left(2JSD(p||q)-H(p)\right) + H(p)\cdot \left(\log \frac{p_i + \frac{1}{N}}{2} - \log \frac{p_j + \frac{1}{N}}{2}  \right) = 0.
\end{array}
\end{equation}

\begin{remark}\label{remjsd}
    It follows from the form of the system of equations (\ref{eq75}) that the system is satisfied if $p_i=p_j$, which is one of the necessary conditions for the extremum of the function (\ref{CJSD}).
    Due to the nonlinearity of the system consisting of equations (\ref{eq75}), it may have other roots.
\end{remark}


Fig. \ref{Pic11} shows a surface plot of the statistical complexity level of the form (\ref{CJSD}) when the discrete distribution $p=\{p_1,~p_2,~p_3\}$ consists of three samples to illustrate the Remark \ref{remjsd}.

% Figure environment removed


It can be seen from Fig. \ref{Pic11} that the points satisfying $p_1=p_2$, $p_2=p_3$ and $p_2=p_3$ are the saddle points of the surface if the necessary extremum conditions are satisfied.

It was previously established that the distribution \eqref{distrK} delivers an extremum to $C_{SQ}$ at $K=N-1$. Next, it will be shown that it also delivers the extremum to the complexity based on the total variation of the measure $TV(p,q)$. Therefore, we propose to find the maximum of $C_{JSD}$ on this distribution and compare the obtained optimal distribution parameters at fixed $N$. Let us write out the complexity explicitly and obtain
\begin{equation}\label{CJSD_k_p}
    C^{(K)}_{JSD} = H^{(K)}\cdot \left(H^{(K)}(m) - \frac{1}{2}(H^{(K)}+1)\right)\cdot \log N,
\end{equation}
where $H^{(K)}$ is corresponding to \eqref{entrKSQ}, and $H^{(K)}(m)$ is given by the following formula:
\begin{equation}
 \displaystyle H^{(K)}(m)= 1-\frac{1}{\log N}\left(\frac{(1-p_{\max}+\omega)}{2}\log\frac{(1-p_{\max}+\omega)}{2\omega} +\frac{(1+p_{\max}-\omega)}{2}\log\frac{(1+p_{\max}-\omega)}{2-2\omega}\right).
\end{equation}

Table \ref{tableJSD} shows the change of optimal parameters $C_{JSD}$ with the growth of $N$. 

\def\arraystretch{1.8}%  1 is the default, change whatever you need
\begin{table}[h]
\caption{Optimal parameters $C_{JSD}(\omega,p_{\max})$ for different values of $N$.}\label{tableJSD}
\centering
\begin{tabularx}{0.99\textwidth} { 
  | >{\centering\arraybackslash}X 
  | >{\centering\arraybackslash}X 
  | >{\centering\arraybackslash}X 
  | >{\centering\arraybackslash}X 
  | >{\centering\arraybackslash}X | }
 \hline
{ $N$ } & $C_{JSD}(\omega^*,p_{\max}^*)$ & $p_{\max}^*$  & $\omega^*$ & $N-K^*$ \\ 
 \hline
$3$  & $0,1266$ & $1$  & $0,4083$  & $1\text{ or }2$ \\
 \hline
$256$  & $0,4482$ & $1$  & $0,8703$  & $33$ \\
 \hline
$512$  & $0,4790$ & $1$  & $0,8897$  & $56$ \\
 \hline
 $1024$  & $0,5065$ & $1$  & $0,9051$  & $97$ \\
 \hline
  $2048$  & $0,5312$ & $1$  & $0,9171$  & $170$ \\
 \hline
\end{tabularx}
\end{table}

For clarity, Fig. \ref{PicJSD_FIG} shows the graph
$C_{JSD}=C_{JSD}(\omega,p_{\max})$ at $N=1024$, where $\omega$ is changing continuously (although $K$ is changing discretely). 

% Figure environment removed


The results shown in Table \ref{tableJSD} demonstrate that for the chosen class of distributions \eqref{distrK} the set of $N$ components, where $K$ is equal to each other and the rest are zero, is optimal. It is worth noting that for the resulting distribution $C_{JSD}$ is not zero, as well due to the summand $H^{(K)}(m)$, which corresponds to the already "shifted" $~$ distribution consisting of $K$ elements equal to $\displaystyle\frac{\frac{1}{K}+\frac{1}{N}}{2}$, and $N-K$ samples of $\displaystyle\frac{1}{2N}$ each.

\subsection{Optimization $C_{TV}$}
Let us proceed to analyze statistical complexity based on total variation 
  \begin{equation}\label{CTV}
        \displaystyle C_{TV}(p)=-\frac{1}{4\log N}\left(\sum_{i=1}^N p_i\log p_i \right)\cdot\left(\sum _{i=1}^N\left|p_i-\frac{1}{N}\right|\right)^2.
    \end{equation}
\begin{proposition}
According to the expression for the error function \eqref{ErrTVR} from the Neyman-Pearson Lemma \ref{NeymanPearson} and the definition of \eqref{CTV}, we propose to use $C_{TV}$ as a criterion to solve the problem of hypothesis testing and indicating the appearance of a deterministic component of a useful signal in noise.
\end{proposition}
The following Lemma is valid.
 \begin{lemma}
% Максимум статистической сложности (\ref{CTV}) достигается на семействе распределений (\ref{distrK}).
The maximum statistical complexity (\ref{CTV}) is achieved on the family of distributions (\ref{distrK}).
 \end{lemma}   
\begin{proof}    

Given the symmetry of the function \eqref{CTV} and the simplex \eqref{discdefinit}, without restriction of generality, we find an integer $K \in\{1,...,N-1\}$ for which the maximum of this Lemma is achieved on the part of the simplex \eqref{discdefinit} defined by the constraints $p_i\leq 1/N$ for $i=1,...,K$ and $p_i\geq 1/N$ for $i=K+1,...,N$. Let us rewrite the equation (\ref{CTV}) in the form of
\begin{equation}
   C_{TV}= -\frac{1}{4\log N}\left(\sum_{i=1}^{N-1} p_i\log p_i + p_N \log p_N\right)\cdot \left(\sum_{i=1}^{K} \left(-p_i + \frac{1}{N} \right) + \sum_{i=K+1}^{N}\left(p_i - \frac{1}{N}\right)\right)^2.
\end{equation}
%Тогда для $i=1,\ldots,K$ необходимые условия экстремума принимают вид
Then for $i=1,\ldots,K$ the necessary conditions of extremum take the form
\begin{equation}\label{NECCTV1}
\begin{array}{ccc}
    \displaystyle \frac{\partial C_{TV}}{\partial p_i} = -\frac{1}{\log N}\left( \log p_i -\log p_N  \right)\cdot D_{TV} - 2H(p)\sqrt{D_{TV}}=0,
     \quad i  = 1, \dots, K,
\end{array}    
\end{equation}
%а для $i=K+1,\ldots,N$ справедливо 
and for $i=K+1,\ldots,N$ the following is true 
\begin{equation}\label{NECCTV2}
\begin{array}{ccc}
    \displaystyle \frac{\partial C_{TV}}{\partial p_i} = -\frac{1}{\log N}\left( \log p_i -\log p_N  \right)\cdot D_{TV} =0,
     \quad i  = K+1, \dots, N.
\end{array}    
\end{equation}
Let us compose the difference of two equations from (\ref{NECCTV1}) for indices $i$ and $j$. Whence it follows that if $D_{TV}\not=0$, then $p_i=p_j$ at $i=1,\ldots,K$. Whereas it follows from (\ref{NECCTV2}) that $p_i=p_N$ at $i=K+1,\ldots,N$. Again we obtain that the family of distributions (\ref{distrK}) delivers the maximum of the complexity function, now $C_{TV}$.
\end{proof}

Next, we need to determine the optimal values of $K$ and $p_{\max}$. 
For this purpose, let us calculate the disequilibrium value $D^{(K)}_{TV}$ on the distribution (\ref{distrK}): 
\begin{equation}\label{DKTV}
D^{(K)}_{TV}=  (p_{\max}+\omega-1)^2, ~~~\omega=\frac{K}{N}.
\end{equation}
In turn, entropy is equal to
\begin{equation}\label{HKTV}
 \displaystyle H^{(K)}= 1-\frac{1}{\log N}\left((1-p_{\max})\log\frac{1-p_{\max}}{\omega} +p_{\max}\log\frac{p_{\max}}{1-\omega}\right).
\end{equation}


For clarity, Fig. \ref{Pic13} shows the graph $C_{TV}=C_{TV}(\omega,~p_{\max})$ at $N=1024$, where $\omega$ is changing continuously (although $K$ is changing discretely).


% Figure environment removed

Let us compose the necessary conditions \eqref{NECCTV1} and \eqref{NECCTV2} of the extremum of the statistical complexity $C_{TV}$ written out through the variables $p_{\max},~\omega$.
%\footnotesize
\small
\begin{equation}\label{nece_cond_ctv}
    \begin{cases}
    \displaystyle f_{1}^N(p_{\max},\omega) \coloneqq 2(p_{\max}+\omega-1)\left(\left( 1-\frac{(1-p_{\max})\log \frac{1-p_{\max}}{\omega}+p_{\max}\log\frac{p_{\max}}{1-\omega} }{\log N}  \right)\right.-\\
    \displaystyle\left.-\frac{(p_{\max}+\omega-1)}{2\log N}\left(\log\frac{p_{\max}}{1-\omega}-\log\frac{1-p_{\max}}{\omega}\right)\right)=0,\\\\
    
    \displaystyle f_{2}^N(p_{\max},\omega) \coloneqq 2(p_{\max}+\omega-1)\left(\left( 1-\frac{(1-p_{\max})\log\frac{1-p_{\max}}{\omega}+p_{\max}\log\frac{p_{\max}}{1-\omega}}{\log N}  \right)-\right.\\
\displaystyle\left.-\frac{(p_{\max}+\omega-1)}{2\log N}\left(\frac{p_{\max}}{1-\omega}-\frac{1-p_{\max}}{\omega}\right)\right)=0.
    \end{cases}
\end{equation}
\normalsize
The intersections of the curves corresponding to the implicit equations \eqref{nece_cond_ctv} are related to to the extremum points of $C_{TV}$. Let us compose the difference of two necessary conditions of the extremum
\begin{equation}\label{f1-f2}
\begin{array}{c}
    \displaystyle f_{3}^N(p_{\max},\omega) \coloneqq \displaystyle f_{1}^N(p_{\max},\omega)-\displaystyle f_{2}^N(p_{\max},\omega) = \\
    \displaystyle=\frac{(p_{\max}+\omega-1)^2}{\log N}\left(-\log\frac{p_{\max}}{1-\omega}+\log\frac{1-p_{\max}}{\omega}+\frac{p_{\max}}{1-\omega}-\frac{1-p_{\max}}{\omega}\right)=0.
    \end{array}
\end{equation}

Let us construct the implicit curves of equations $f_{1}^N(p_{\max},\omega),~f_{2}^N(p_{\max},\omega),~f_{3}^N(p_{\max},\omega)$ for some values of $N$. For convenience, the index $N$ of $f_{3}^N(p_{\max},\omega)$ can be omitted since the implicit curve of the equation \eqref{f1-f2} is independent of $N$.

% Figure environment removed


According to Fig. \ref{CTV_pic} the statistical complexity has, in addition to the minimum points $(p_{\max}+\omega-1=0=0)$, where $C_{TV}=0$, also two maximum points for each value of $N$: $(p_{\max}^*,\omega^*)$ and $(1-p_{\max}^*,1-\omega^*)$, which lie on the curve $f_{3}(p_{\max},\omega)$. 


In Table \ref{tableTV} the optimal values of the parameters of the formulas (\ref{DKTV}) and (\ref{HKTV}) that maximize the statistical complexity of $C_{TV}$ are given.

\def\arraystretch{1.8}%  1 is the default, change whatever you need
\begin{table}[h]
\caption{Optimal parameters $C_{TV}(\omega,p_{\max})$ for different values of $N$.}\label{tableTV}
\centering
\begin{tabularx}{0.9\textwidth} { 
  | >{\centering\arraybackslash}X 
  | >{\centering\arraybackslash}X 
  | >{\centering\arraybackslash}X 
  | >{\centering\arraybackslash}X 
  | >{\centering\arraybackslash}X | }
 \hline
{ $N$ } & $C_{TV}(\omega^*,p_{\max}^*)$ & $p_{\max}^*$  & $\omega^*$ & $N-K^*$ \\ 

 \hline
$3$  & $0,1289$ & $0,8241$  & $0,6751$  & $1$ or $2$\\
 \hline
$256$  & $0,4789$ & $0,9976$  & $0,8752$  & $32$ \\
 \hline
$512$  & $0,5120$ & $0,9991$  & $0,8901$  & $56$ \\
 \hline
 $1024$  & $0,5410$ & $0,9997$  & $0,9022$  & $100$ \\
 \hline
 $2048$  & $0,5667$ & $0,9999$  & $0,9122$  & $180$ \\
  \hline
\end{tabularx}
\end{table}


Additionally, the case $N=3$ is shown in Fig. \ref{Pic14}, which shows a graph of the level surface $C_{TV}$ when the discrete distribution $p=\{p_1,p_2,p_3\}$ consists of three samples.

% Figure environment removed

\section{Statistical Complexity Modeling and Comparison}\label{Modelling}
We analyze the optimal parameters that maximize different types of statistical complexity and compare the values in Tables. \ref{tableSQ}, \ref{tableJSD}, and \ref{tableTV}. Of main interest are the maximum complexity values and the optimal values of $K$. The maximum values of $C_{TV}(\omega^*,p_{\max}^*)\in[0,1]$, $C_{JSD}(\omega^*,p_{\max}^*)\in[0,1]$ are close to each other and grow with increasing $N$. The optimal values of $K$ for these two types of complexity are also close.


To demonstrate the analytical results obtained in the previous sections, the application of three variants of statistical complexity in the problem of useful signal indication in a noise mixture for synthesized signals is shown. An algorithm from \cite{Sensors} based on the computation of discrete distributions $p$ from the spectral representation of time series is applied. 



The synthesized 10-second signal is the sum of a finite number of cosine oscillations mixed with white noise:
\begin{equation}\label{signal_mod1}
\displaystyle x(t)=I(t)\sum_{i=1}^K A_i\cos(2\pi f_i t + \Delta \phi_i) + w(t), \quad t \in [0, 10],\\
\end{equation}

where $A_i, ~f_i,~\Delta \phi_i$ are the amplitudes, frequencies, and random phases of the harmonic oscillations, respectively, $w(t)$ is the white noise, and $I(t)$ is the indicator function for the presence of the useful signal in the signal-noise mixture.



$I(t)$ is chosen so that the harmonic signals are present in the middle of the final sequence $x(t)$.

\begin{equation}
    I(t) = \left\{
    \begin{array}{lll}
        0, & t \in [0, 3), \\
        1, & t \in [3, 7], \\
        0, & t \in (7, 10].
    \end{array}
    \right.
\end{equation}


The algorithm has the following structure:
\begin{enumerate}
    \item The signal synthesized with sampling frequency $f_s$ is divided into short windows containing $N = 2048$ samples each.
    \item Next, the spectrum for each window is calculated using the FFT algorithm.
    \item Based on the spectrum, the discrete densities $p_i,~i = 1, \dots, N$ are calculated by normalizing it.
    \item The information characteristics $C_{SQ}(p),~C_{JSD}(p),~C_{TV}(p)$ are calculated for the obtained set $p_i$.
    \item The obtained sequence of information characteristic values is shown along with the signal on the time axis.
\end{enumerate}


It should be noted that the parameters $f_s$ and $N$ were chosen to exclude the effect of spectrum spreading, i.e., to obtain clear spectral components corresponding to $K$ harmonic functions from the formula \eqref{signal_mod1}. The signal-to-noise ration is chosen to be close to one.


The threshold $\gamma$ for the decision rule is proposed to be chosen as $25\%$ of the maximum criterion value for selected $N$ from the tables \ref{tableSQ}, \ref{tableJSD}, \ref{tableTV}:
\begin{equation}
    \begin{array}{ccc}
        \gamma_{CQ} = 0,25 \cdot 0,1861 = 0,0465; \\
        \gamma_{JSD} = 0,25 \cdot 0,5312 = 0,1328; \\
        \gamma_{TV} = 0,25 \cdot 0,5667 = 0,1417.
    \end{array}
\end{equation}


The convenience of choosing such a threshold is that it does not depend on a particular noise realization and is based on analytically derived maximum values of statistical complexity functions. 


In all plots, the blue color indicates the amplitude of the original signal, and the red color indicates the statistical complexity, which is calculated using the algorithm described above. The horizontal axis represents time in seconds, and the vertical axes represent the magnitude of the signal amplitude (left) and the criterion (right). The black dashed line shows the value of the selected threshold $\gamma$.

In the first experiment, the number of sinusoidal signals and respectively spectral components is equal to $K = 3$ at $N=2048$. Fig. \ref{firstexp} shows the dependencies of statistical complexities on time for the synthesized signal. 


As can be seen, the values of $C_{SQ}$ and $C_{TV}$ exceed the selected threshold for the interval of signal presence, which allows us to confidently conclude that the signal has occurred. As for $C_{JSD}$, the a priori threshold selection was unsuccessful because the true value of its maximum is unknown, as shown in subsection \ref{JSDoptimize}. If we change the threshold value upwards by $20\%$, the detection based on $C_{JSD}$ will be as successful as that based on $C_{TV}$.


% Figure environment removed


In the second experiment, the number of spectral components $K = 30$. In this case, $C_{SQ}$ ceases to show a satisfactory result in the sense of exceeding the chosen threshold, since the function $C_{SQ}$ degrades strongly with increasing of $K$, but still allows a signal indication, as can be seen in Fig. \ref{secondexp}. The complexity function $C_{TV}$ still confidently exceeds the threshold, as in the first experiment, and $C_{JSD}$ exceeds the threshold on the whole signal, as in the first experiment. 

Thus, we can conclude that $C_{TV}$ is the most convenient in a practical sense, since it works well on signals with a large number of spectral components and allows one to decide on the appearance of an useful signal, using a fairly simple rule, associated with the choice of threshold based on the theoretical maximum value for the statistical complexity.

% Figure environment removed
\section{Conclusion}\label{Conclusion}

The article provides a theoretical justification for the usage of statistical complexity as a criterion for solving the problem of hypothesis testing when its error probability is close to one. Three variants of statistical complexity for different disequilibrium functions are considered. New notions of disequilibrium and statistical complexity based on the total variation measure are introduced. Information criteria are compared and the classes of discrete distributions which provide maximum for different types of statistical complexity are discovered. The values of maxima for fixed numbers of distribution samples are found. It is shown that the statistical complexity $C_{TV}$ based on total variation is directly related to the problem of hypothesis testing, while the statistical complexity $C_{JSD}$ based on Jensen-Shannon entropy gives a close estimate of $C_{TV}$ on sample distributions. In turn, $C_{SQ}$ is most promising for detecting an individual component over a uniform distribution. We propose a method for selecting the threshold for the decisive rule for the detection of a useful signal, taking into account the maximum values of the criteria obtained and show the effectiveness of this approach on the synthesized signals. 



Future work will be devoted to the study of information criteria based on bivariate and multivariate distributions, as well as investigation of typical acoustic signals with realistic background noise.

\bmhead{Funding}

The work was supported by the Russian Science Foundation under grant \mbox{no 23-19-00134}.

\bibliography{sn-bibliography}

%%===========================================================================================%%
%% If you are submitting to one of the Nature Portfolio journals, using the eJP submission   %%
%% system, please include the references within the manuscript file itself. You may do this  %%
%% by copying the reference list from your .bbl file, paste it into the main manuscript .tex %%
%% file, and delete the associated \verb+\bibliography+ commands.                            %%
%%===========================================================================================%%

% \bibliography{sn-bibliography}% common bib file
%% if required, the content of .bbl file can be included here once bbl is generated
%\input sn-bibliography.bbl


\end{document}


\end{document}
