%\documentclass[manuscript, screen, dvipsnames, anonymous]{acmart}
%\documentclass[manuscript, screen, dvipsnames, authordraft]{acmart}
\documentclass[sigconf, screen, dvipsnames, authorversion]{acmart}
%\documentclass[sigconf, screen, dvipsnames, anonymous]{acmart}
%\documentclass[sigconf, screen, dvipsnames, anonymous, nonacm]{acmart}

\usepackage{multirow}
\usepackage{amsmath}
\usepackage{amsthm}
\usepackage[bb=dsserif,bbscaled=1.2]{mathalpha}
\usepackage{bm}
\usepackage{mathtools}
\usepackage[inline]{enumitem} % Inline lists
\usepackage{subcaption}
\usepackage{xcolor}

\usepackage{wrapfig}

% Dashed lines in tables
\usepackage{arydshln}
\makeatletter
\def\adl@drawiv#1#2#3{
        \hskip.5\tabcolsep
        \xleaders#3{#2.5\@tempdimb #1{1}#2.5\@tempdimb}%
                #2\z@ plus1fil minus1fil\relax
        \hskip.5\tabcolsep}
\newcommand{\cdashlinelr}[1]{%
  \noalign{\vskip\aboverulesep
          \global\let\@dashdrawstore\adl@draw
          \global\let\adl@draw\adl@drawiv}
  \cdashline{#1}
  \noalign{\global\let\adl@draw\@dashdrawstore
          \vskip\belowrulesep}}
\makeatother

\newtheorem{assumption}{Assumption}

\usepackage{colortbl}
\newcolumntype{p}{>{\columncolor{gray!10}}r}
\newcolumntype{R}{S[table-format=1.2]}

\usepackage{siunitx}

% PGM in Tikz
\usepackage{tikz}
\usetikzlibrary{automata, arrows, bayesnet, bending}

% % Reduce space after floats (such as tables and figures)
\setlength{\textfloatsep}{1pt plus 0pt minus .5pt}
\setlength{\intextsep}{1pt plus 0pt minus 1.0pt}
\setlength{\abovecaptionskip}{1pt plus 0pt minus 1pt}
\setlength{\belowcaptionskip}{1pt plus 0pt minus 1pt}
% Reduce space around equations
\setlength{\abovedisplayskip}{1pt plus 1pt minus 1pt}
\setlength{\belowdisplayskip}{1pt plus 1pt minus 1pt}

% Argmax as mathematical operator
\DeclareMathOperator{\argmax}{arg\,max}
\DeclareMathOperator{\argmin}{arg\,min}
\DeclareMathOperator{\argsort}{arg\,sort}
\DeclareMathOperator{\vect}{vec}

%Conference
\copyrightyear{2024}
\acmYear{2024}
\setcopyright{acmlicensed}\acmConference[KDD '24]{Proceedings of the 30th ACM SIGKDD Conference on Knowledge Discovery and Data Mining}{August 25--29, 2024}{Barcelona, Spain}
\acmBooktitle{Proceedings of the 30th ACM SIGKDD Conference on Knowledge Discovery and Data Mining (KDD '24), August 25--29, 2024, Barcelona, Spain}
\acmDOI{10.1145/3637528.3671687}
\acmISBN{979-8-4007-0490-1/24/08}


\begin{document}
\title{On (Normalised) Discounted Cumulative Gain as an Off-Policy Evaluation Metric for Top-$n$ Recommendation}

\author{Olivier Jeunen}
\affiliation{
  \institution{ShareChat}
  \city{Edinburgh}
  \country{United Kingdom}
}
\author{Ivan Potapov}
\affiliation{
  \institution{ShareChat}
  \city{London}
  \country{United Kingdom}
}
\author{Aleksei Ustimenko}
\affiliation{
  \institution{ShareChat}
  \city{London}
  \country{United Kingdom}
}

\begin{abstract}
Approaches to recommendation are typically evaluated in one of two ways:
\begin{enumerate*}
    \item via a (simulated) online experiment, often seen as the \emph{gold standard}, or
    \item via some offline evaluation procedure, where the goal is to approximate the outcome of an online experiment.
\end{enumerate*}
Several offline evaluation metrics have been adopted in the literature, inspired by ranking metrics prevalent in the field of Information Retrieval.
(Normalised) Discounted Cumulative Gain (nDCG) is one such metric that has seen widespread adoption in empirical studies, and higher (n)DCG values have been used to present new methods as  the \emph{state-of-the-art} in top-$n$ recommendation for many years.\looseness=-1

Our work takes a critical look at this approach, and investigates \emph{when} we can expect such metrics to approximate the gold standard outcome of an online experiment.
We formally present the assumptions that are necessary to consider DCG an \emph{unbiased} estimator of online reward and provide a derivation for this metric from first principles, highlighting where we deviate from its traditional uses in IR.
Importantly, we show that normalising the metric renders it \emph{inconsistent}, in that even when DCG is unbiased, ranking competing methods by their normalised DCG can invert their relative order.
Through a correlation analysis between off- and on-line experiments conducted on a large-scale recommendation platform, we show that our \emph{unbiased} DCG estimates strongly correlate with online reward, even when some of the metric's inherent assumptions are violated.
This statement no longer holds for its normalised variant, suggesting that nDCG's practical utility may be limited.
\end{abstract}

\begin{CCSXML}
<ccs2012>
   <concept>
       <concept_id>10002951.10003317.10003359</concept_id>
       <concept_desc>Information systems~Evaluation of retrieval results</concept_desc>
       <concept_significance>300</concept_significance>
       </concept>
   <concept>
       <concept_id>10002951.10003317.10003347.10003350</concept_id>
       <concept_desc>Information systems~Recommender systems</concept_desc>
       <concept_significance>500</concept_significance>
       </concept>
   <concept>
       <concept_id>10002950.10003648.10003662</concept_id>
       <concept_desc>Mathematics of computing~Probabilistic inference problems</concept_desc>
       <concept_significance>300</concept_significance>
       </concept>
 </ccs2012>
\end{CCSXML}

\ccsdesc[500]{Information systems~Recommender systems}
\ccsdesc[300]{Information systems~Evaluation of retrieval results}
\ccsdesc[300]{Mathematics of computing~Probabilistic inference problems}

\keywords{Offline Evaluation; Off-Policy Evaluation; Counterfactual Inference}

\maketitle

\section{Introduction}
%the Introduction section need to be improved (writting & arguments, and reduce non-sense words)
Air quality forecasting using data-driven models has gained significant attention in recent years, thanks to the proliferation of data collection infrastructures such as sensor stations and advancements of telecommunication technologies. These infrastructures are typically managed by national institutes (e.g., AirParif\footnote{https://www.airparif.asso.fr/}, EPA\footnote{https://www.epa.gov/air-quality}) or large companies (e.g., PurpleAir\footnote{https://www2.purpleair.com/}) that specialize in air quality monitoring or forecasting services and products. Leveraging existing data collection infrastructures proves beneficial for initial research exploration or validating product prototypes.
However, reliance on fixed infrastructures presents practical constraints when customization is required for specific tasks. For instance, certain monitoring areas may be inadequately covered or completely absent from the existing infrastructures, or the density of coverage may not be sufficient. This issue particularly affects small or mid-sized industrial and academic players who face budget limitations that prevent them from investing in their own infrastructure from scratch, but have specific customization needs.

% give the motivation from another aspect: incrementally built infrastructure, no need to re-train the model
In addition to data collection, air quality forecasting models trained solely with data from public fixed infrastructures may not perform well for users' specific scenarios, such as forecasting at a higher spatial resolution. Deploying additional sensors as a cost-effective solution can enrich the data and improve forecasting performance without the need to build infrastructures from scratch. 
Subsequently, this targeted solution leads us to consider the practical question: \textit{how we can make use of the data collected from existing infrastructures, when integrating new sensor infrastructures?} 
%which can be equipped on fixed sensor stations or moving objects (e.g., drones) with a higher flexibility.

% Figure environment removed

As depicted in Figure \ref{fig:research_background}, the topological sensor network may change as the urban infrastructure evolves, resulting in varying network structures of air quality sensors. The data collected from the network $G_{\tau}$ needs to be augmented with enriched data from newly installed sensors $\Delta G_{\tau'}$ and $\Delta G_{\tau''}$. Training a model solely on recent data with $G_{\tau''}$ would overlook valuable information contained in the historical data with $G_{\tau}$ and $G_{\tau'}$.

In this paper, we propose an expandable graph attention network (EGAT) that effectively integrates data with various graph structures. This approach is versatile and can be seamlessly embedded into any existing air quality forecasting model. Furthermore, it applies to scenarios where sensors are not installed, enabling accurate forecasting in such areas.
We summarize our approach's main advantages as follows:
\begin{itemize}
    %\item \textbf{Air quality forecasting in real scenarios:} we consider the complex data quality issues, e.g., missing values

    \item \textbf{Less is more:} With fewer installed sensors, we can directly predict the air quality of other unknown area where sensors are not installed and achieve comparable performance to models relying on extensive data collection infrastructures with more sensors.
    \item \textbf{Continual learning with self-adaptation:} The proposed model enables continuous learning from newly collected data with expanded sensor networks, demonstrating self-adaptability to different topological sensor networks.
    \item \textbf{Embeddable module with scalability:} The proposed module can be seamlessly integrated into any air quality forecasting model, enhancing its ability to forecast in real-world scenarios.

\end{itemize}

The rest of this paper starts with a review of the most related work. Then, we formulate the problems of the paper. Later, we present in detail our proposal, which is followed by the experiments on real-life datasets and the conclusion.


\section{Related Work}
\subsection{Air Quality Forecasting}
% Air quality monitoring and forecasting: data-driven and model-driven approaches
Data-driven models for air quality forecasting has gained a huge popularity recently. Recent work \cite{zuo2023graph,liang2022airformer} studies graph-based representations of the air quality data by considering the sensor network as a graph structure, which extracts decent structural features between sensor data from a topological view. The air quality forecasting can be then formulated as a spatio-temporal forecasting problem.

% work on general spatio-temporal forecasting tasks, and what make Air Quality forecasting different? 
Works like DCRNN~\cite{li2018diffusion}, STGCN~\cite{yu2018spatio} and Graph WaveNet~\cite{wu2019graph}, have shown promising results in traffic forecasting tasks. These models can be adapted to air quality forecasting tasks owing to the shared spatio-temporal features present in the data. 
%The key differences lie in two aspects: i) Air quality exhibits continuous state changes between areas, while traffics change abruptly between nodes; ii) External factors impacting air quality are more complex, which can be from human activities, or cased by natural environment. 
% the typical research problems considered in air quality forecasting: multiple data sources -> data fusion -> forecasting target 
%DeepAir~\cite{yi2018deep}: spatial transformation for various sensor readings, a fusion network to model the relationships between different factors and AQIs.
%AirFormer~\cite{liang2022airformer}: self-attention for learning spatio-temporal representations; capture the intrinsic uncertainty of air quality data.
% Challenges/constrains of previous air quality forecasting models
However, in practice, the above-mentioned models often overlook the evolving nature of sensor networks as more data collection infrastructures are incrementally built. Consequently, these models require re-training from scratch on the most recent data that reflects the evolved sensor network. It may result in the loss of valuable information contained in outdated data collected from different network configurations.
%or using the previous model to do approximate inference of unlearned areas relying on neighbor predictions. 

% Learning models
%Some work models the air pollution in the whole city with an image-based representation. However, this representation may not be ideal, as air pollution and other impact factors have natural graph structures. 
% Challenges: huge amount of sensors are required

\subsection{Expandable Graph Neural Networks}
%Dynamic graph structures, meaning at different time stamps, the graph structure can be different. Although various works have studied the dynamic graphs with a focus on the dynamic edge weight but over a fixed set of graph nodes, they barely consider a dynamic graph with evolving number of nodes. 

In the field of graph learning, several works, such as ContinualGNN \cite{wang2020streaming} and ER-GNN \cite{zhou2021overcoming}, have incorporated the concept of Continual Learning to capture the evolving patterns within graph nodes.
While these approaches are valuable, it is important to consider spatio-temporal features in air quality forecasting tasks.
Designed for traffic forecasting, TrafficStream~\cite{chen2021trafficstream} considers evolving patterns on both temporal and spatial axes; ST-GFSL~\cite{lu2022spatio} introduces a meta-learning model for cross-city spatio-temporal knowledge transfer. However, these works primarily focus on shared (meta-)knowledge between nodes, and give less attention to expandable graph structures. 
Basically, spectral-based graph neural networks (GNNs) face challenges when scaling to graphs with different structures due to the complexity of reconstructing the Laplacian matrix. To address this issue, our paper explores the use of spatial-based GNNs, such as Graph Attention Networks (GAT) \cite{velivckovic2018graph}, for expandable graph learning in air quality forecasting tasks.
%ContinualGNN~\cite{wang2020streaming}: streaming GNN considering continual learning of new patterns and preservation of existing patterns.
%ER-GNN~\cite{zhou2021overcoming}: consider catastrophic forgetting problems using Memory Replay.

%ST-GFSL~\cite{lu2022spatio}: meta-learning for cross-city knowledge transfer, the graphs of different cities can be different. 

%GAT~\cite{velivckovic2018graph}; 

%TrafficStream~\cite{chen2021trafficstream}: evloving patterns on both temporal axis and spatial axis (expanded graphs) 



\section{Formalising the problem setting}\label{sec:problemsetting}
Throughout, we represent the domain for a random variable $X$ as $\mathcal{X}$ and a specific instantiation as $x$, unless explicitly mentioned otherwise.
We deal with a session-based feed recommendation setup, describing a user's journey on the platform as a trajectory $\tau$.
Contextual features describing a trajectory are encoded in $x \in \mathcal{X}$, which includes features describing the user $u \in \mathcal{U}$ and possible historical interactions they have had with items on the platform.
In line with common notation in the decision-making literature, we will refer to these items as \emph{actions} $a \in \mathcal{A}$.
As is common in real-world systems, the size of the item catalogue (i.e. the action space $|\mathcal{A}|$) can easily grow to be in the order of  hundreds of millions, prohibiting us to score and rank the entire catalogue directly.
This is typically dealt with through a two-stage ranking setup, where a more lightweight \emph{candidate generator} stage is followed by a \emph{ranking} stage that decides the final personalised order in which we present items to the user~\cite{VanDang2013, Covington2016, Ma2020}.

We adopt generalised probabilistic notation for two-stage rankers in this work, but stress that our insights are model-agnostic and directly applicable to single-stage rankers as well. % !!!

Let $\mathcal{A}^{k}$ denote all subsets of $k$ actions: $\mathcal{A}^{k} \subseteq 2^{\mathcal{A}}: \forall a^{k} \in \mathcal{A}^{k}, |a^{k}|=k$.
A candidate generation policy $\mathcal{G}$ defines a conditional probability distribution over such sets of \emph{candidate} actions, given a context:
%\begin{equation}
$
    \mathcal{G}(A^{k}|X)\coloneqq\mathsf{P}(A^{k}|X,\mathcal{G}).
$
%\end{equation}
We will use the shorthand notation $\mathcal{G}(X)$ when context allows it.
Note that this general notation subsumes other common scenarios, such as candidate generation policies that are deterministic, consist of ensembles, or simply yield the entire item catalogue $\mathcal{A}$ (i.e. \emph{single-stage} ranking systems).

After obtaining a set of candidate items for context $x$ by sampling $a^{k} \sim \mathcal{G}(x)$, we pass them on to the ranking stage.

In line with our probabilistic framework, we have a ranking policy $\mathcal{R}$ that defines a conditional probability distribution over rankings (i.e. permutations of $a^{k}$).
Define $\bm{\sigma}$ as a possible permutation over $A^{k}$.
Formally, we have:
%\begin{equation}
$
%    \mathcal{R}(S(A^{k})|A^{k}, X)\coloneqq\mathsf{P}(S(A^{k})|A^{k}, X,\mathcal{R}).
    \mathcal{R}(\bm{\sigma}|A^{k}, X)\coloneqq\mathsf{P}(\bm{\sigma}|A^{k}, X,\mathcal{R}).
$
%\end{equation}
When context allows it, we will use shorthand notation $\mathcal{R}(X)$ to absorb the candidate generation step. Then, the ranker is given by: 
\begin{equation}
%    \mathcal{R}(X) \coloneqq \sum_{a^{k} \in \mathcal{A}^{k}}\mathsf{P}(S(a^{k})|A^{k}=a^{k}, X,\mathcal{R})\mathsf{P}(A^{k}=a^{k}|X,\mathcal{G}).
    \mathcal{R}(X) \coloneqq \sum_{a^{k} \in \mathcal{A}^{k}}\mathsf{P}(\bm{\sigma}|A^{k}=a^{k}, X,\mathcal{R})\mathsf{P}(A^{k}=a^{k}|X,\mathcal{G}).
\end{equation}

Now, for a given context $x$, we can obtain rankings over actions by sampling $\sigma=(a_{1},\ldots,a_{k}) \sim \mathcal{R}(x)$.
These rankings are then presented to users, who scroll through the ordered list of items and \emph{view} them along the way.
We will not \emph{yet} restrict our setup to a specific user model that describes \emph{how} users interact with rankings, but introduce a binary random variable $V$ to indicate whether a user has viewed a given item.\footnote{Note that this problem setting deviates from traditional work dealing with web search in IR, where multiple items are shown on screen and item-specific view events cannot be disentangled trivially.
In contrast, our items take up most of the user's mobile screen when presented, and we are able to deduce accurate item-level view labels from logged scrolling behaviour. See e.g.~\citet{Jeunen2023_C3PO} for work dealing with this setting.}
As such, we will assume that logged trajectories only contain items that were viewed by the user (i.e. $V=1$).
That is, if the user abandons the feed after action $a_{i}$ shown at rank $R=i$, we do not log samples for actions $a_{j}, \forall j>i$.

Users can not only view items, but they can interact with them in several ways.
These interaction signals could be seen as the \emph{reward} or \emph{(relevance) label}.
Following traditional notation where rewards are \emph{clicks} we will denote them by random variable $C$.
Nevertheless, these signals are general and can be binary (e.g. likes), real-valued (e.g. revenue), or higher-dimensional to support multiple objectives (e.g. diversity, satisfaction, and fairness~\cite{Mehrotra2018,Mehrotra2020}).

As users can interact with every item separately, we define $C^{k} = (c_1, \ldots, c_{k})$ as the logged rewards over all ranks.
We do not place any restrictions on the reward distribution yet, so the reward at any given rank can be dependent on actions at other ranks: $(c_{1}, \ldots, c_{k})\sim\mathsf{P}(C^{k}|X,A_{1}, \ldots, A_{k})$.
Now, a trajectory consists of contextual information, as well as a sequence of user-item interactions with observed reward labels: $\tau = \{x_{\tau}, (a_1,c_{1}),\ldots(a_{t},c_{t})\}$, where $|\tau|=t$.
The true metric of interest that we care about is the expectation of reward, over contexts sampled from an unknown marginal distribution $x\sim\mathsf{P}(X)$, candidates sampled from our candidate generator $a^{k} \sim \mathcal{G}(x)$, rankings sampled from our ranker $(a_{1},\ldots,a_{k})\sim\mathcal{R}(a^{k},x)$, and rewards sampled from the unknown reward distribution $(c_{1}, \ldots, c_{k})\sim\mathsf{P}(C^{k}|X=x,A_{1}=a_{1}, \ldots, A_{k}=a_{k})$.
We will denote with $C$ the sum of rewards over all observed ranks: $C = \sum_{i=1}^{|\tau|}c_{i}$.
Note that this notation generalises typical evaluation metrics that are used in real-world recommender systems, such as per-item dwell-time or counters of engagement signals.

In order to obtain an estimate for our true metric of interest $\mathbb{E}[C]$, we can perform an online experiment.
Indeed, in doing so, we effectively sample from the above-mentioned distributions and obtain an empirical estimate of our metric by averaging observed samples.
For a dataset $\mathcal{D}_{0}$ containing logged interactions under policies $\mathcal{G}_{0}$ and $\mathcal{R}_{0}$, Eq.~\ref{eq:online_exp} shows how to obtain this empirical estimate:
\begin{equation}\label{eq:online_exp}
    \mathop{\mathbb{E}}\limits_{(a_{1},\ldots,a_{k}) \sim \mathcal{R}_{0}(x)}[C] \approx \frac{1}{|\mathcal{D}_{0}|\cdot|\tau|}\sum_{\tau \in \mathcal{D}_{0}}\sum_{i=1}^{|\tau|}c_{i}.
\end{equation}
As we directly measure the quantity of interest that depends on the deployed policies $\mathcal{G}_{0}$ and $\mathcal{R}_{0}$, this online estimator is often seen as the gold standard.
Nevertheless, it is costly to obtain.
Indeed, as has been widely reported~\cite{Jeunen2019DS,Gilotte2018,Larsen2023}, online experiments require us to:
\begin{enumerate*}
    \item bring hypotheses for new policies up to production standards for an initial test,
    \item wait several days or weeks to deduce statistical significant improvements, and
    \item possibly harm user experience when the policies are performing subpar.
\end{enumerate*}
Furthermore, several pitfalls arise when setting up or interpreting results from online A/B experiments~\cite{Kohavi2022,Jeunen2023_misassumption}.

For these reasons, we want to be able to perform accurate offline experiments.
That is, given a dataset of interactions $\mathcal{D}_{0}$ that were logged under the production policies $\mathcal{G}_{0}$ and $\mathcal{R}_{0}$ (often referred to as \emph{logging} policies), we want to estimate what the reward \emph{would have been} if we had deployed a new policy $\mathcal{R}$ instead (assume $\mathcal{R}$ includes sampling candidates from $\mathcal{G}$).
The goal at hand is thus to devise some function $f$ that takes in a dataset of interactions collected under the logging policy, and is able to approximate the ground truth metric for a target policy $\mathcal{R}$, as shown in Eq.~\ref{eq:offline_exp}:
\begin{equation}\label{eq:offline_exp}
    \mathop{\mathbb{E}}\limits_{(a_{1},\ldots,a_{k}) \sim \mathcal{R}(x)}[C] \stackrel{?}{\approx} f(\mathcal{D}_{0},\mathcal{R}).
\end{equation}
\section{Discounted Cumulative Gain as an unbiased offline evaluation metric}\label{sec:dcg_unbiased}
In reality, this problem can be very complex.
Indeed, the reward that we obtain from presenting a certain ranking to a user can depend on the entire slate at once (of which there are $n!$ versions in a top-$n$ setting), and will be non-i.i.d. over trajectories (i.e. the reward distribution can depend on a user's state, influenced by actions we have taken in the past).

As is typical in machine learning research, we require assumptions that make the problem more tractable.
These assumptions are flawed, but they give us a starting point and a strong foundation to build upon for future iterations.
Note that we will lay out the \emph{specific} assumptions that are \emph{necessary} to motivate the use of Discounted Cumulative Gain as an offline evaluation metric---we discuss ways of relaxing these assumptions in Section~\ref{sec:beyond}.

\begin{assumption}[reward independence across trajectories]\label{ass:no_RL}
The reward for a context-action pair $(X,A_{i})$ in trajectory $\tau$ is independent of the rankings presented in other trajectories $\tau^{\prime} \in \mathcal{D}_{\setminus \tau}$.
\end{assumption}

\begin{assumption}[position-based model~\cite{Craswell2008}]\label{ass:pbm}
We follow the position-based model (PBM) to describe user scrolling behaviour, implying that the probability of a user viewing an item is only dependent on its rank and described by $\mathsf{P}(V|R)$.
\end{assumption}

\begin{assumption}[reward independence across ranks]\label{ass:no_slate}
The reward for a context-action pair $(X,A_{i})$ is independent of other actions in the user trajectory.
Formally, $C_{i} \perp \!\!\! \perp A_{j} | X, A_{i} \forall j \neq i$.
We describe the resulting reward distribution as $\mathsf{P}(C_{i}|X,A_{i},R)$.
\end{assumption}

\begin{assumption}[examination hypothesis~\cite{Craswell2008}]\label{ass:exam}
The reward for an action $A$ shown at rank $R$ is dependent on its inherent quality (unobserved random variable $Q$), and whether it was viewed $V$.
These quantities relate as:
\begin{equation}\label{eq:quality}
\begin{gathered}
    \mathsf{P}(C|X,A,R) = \mathsf{P}(Q|X,A) \cdot \mathsf{P}(V|R),\\
    {\text{ which implies }} \mathsf{P}(Q|X,A)=\frac{\mathsf{P}(C|X,A,R)}{\mathsf{P}(V|R)}.
\end{gathered}
\end{equation}
\end{assumption}
Asm.~\ref{ass:no_RL} allows us to avoid reinforcement learning scenarios, and Asm.~\ref{ass:pbm} prohibits cascading behaviour (that would give rise to other metrics, such as ERR~\cite{Chapelle2009}).
Asm.~\ref{ass:no_slate} allows us to avoid modelling entire slates as individual actions (leading to a combinatorial explosion of the action space), and through Asm.~\ref{ass:exam}, Eq.~\ref{eq:quality} estimates the unobserved context-dependent \emph{quality} of a given item from observable quantities alone.
From Eq.~\ref{eq:online_exp}, we can now rewrite the expected reward we obtain under a ranking policy $\mathcal{R}$ as:
%\begin{equation}\label{eq:dcg_subtle}
%\begin{gathered}
\begin{flalign}\label{eq:dcg_subtle}
  \begin{aligned}
    \mathop{\mathbb{E}}\limits_{(a_{1},\ldots,a_{k}) \sim \mathcal{R}(x)}[C] &\approx \\
    \frac{1}{|\mathcal{D}_{0}|\cdot|\tau|}\sum_{\tau \in \mathcal{D}_{0}}\sum_{i=1}^{|\tau|}&\mathsf{P}(Q=1|X=x_{\tau},A=a_{i})\cdot\mathsf{P}(V=1|R=i).    
\end{aligned}
\end{flalign}
%\end{gathered}
%\end{equation}
Through the position-based model and the examination hypothesis, items shown at lower ranks are \emph{discounted}.
Because we assumed independence of rewards across ranks, rewards observed at different ranks are \emph{cumulative}.
A key insight here is that the ranking policy $\mathcal{R}$ only affects the rank $i$ at which an item is shown, and as such, the exposure probability that is allocated to the item $a_{i}$.
We formalise \emph{exposure} as the expected number of views a target item $a^{\prime}$ will obtain under a given context $x$, for candidate generation and ranking policies $\mathcal{G},\mathcal{R}$:
\vspace{-3ex}
%\begin{equation}
\begin{flalign}\label{eq:exposure}
\begin{gathered}
\mathop{\mathbb{E}}\limits_{\mathcal{G},\mathcal{R}}[V|X=x,A=a^{\prime}] = \sum\limits_{A^{k} \in \mathcal{A}^{k}} \Bigg( \mathcal{G}\left(A^{k}|X=x\right)\cdot\Bigg.\\
\Bigg.\hspace{-5ex}\sum\limits_{(a_{1},\ldots,a_{k}) \in S(A^{k})} \hspace{-5ex}\mathcal{R}\left((a_{1},\ldots,a_{k})|A^{k}, X=x\right) \sum\limits_{i=1}^{k} \mathsf{P}(V=1|R=i) \cdot\mathbb{1}_{\{a_i=a^{\prime}\}}\Bigg).
\vspace{-3ex}
\end{gathered}
\end{flalign}
%\end{equation}
Note that this general notation accommodates recommendation scenarios where we retrieve and rank $k$ candidates but only show the top-$n$ to the user, where $k>n$, by simply defining $\mathsf{P}(V=1|R=j)=0 \forall j=n+1,\ldots,k$.
By encoding cut-offs directly in the position bias model, we forgo the need to consider metrics like DCG@$n$~\cite{Valcarce2020}.

Recent work in unbiased learning-to-rank leverages similar exposure definitions to jointly combat selection and position bias through Inverse Propensity Score (IPS) weighting~\cite{Oosterhuis2020,Gupta2023}.
\begin{assumption}[full support of the logging policy~\cite{Owen2013}]\label{ass:full_support}
Given context $x$, any item that would be assigned non-zero exposure under the target policies $(\mathcal{G}, \mathcal{R})$ has non-zero exposure under the logging policies $(\mathcal{G}_{0}, \mathcal{R}_{0})$:
\begin{equation}
   \forall a \in \mathcal{A}: \mathop{\mathbb{E}}\limits_{\mathcal{G},\mathcal{R}}[V|X=x,A=a] > 0\Rightarrow \mathop{\mathbb{E}}\limits_{\mathcal{G}_{0},\mathcal{R}_{0}}[V|X=x,A=a] > 0.
\end{equation}
\end{assumption}
Through Asm.~\ref{ass:full_support} and Eq.~\ref{eq:exposure}, we can now formulate an \emph{importance sampling} estimator for the reward under $(\mathcal{G},\mathcal{R})$ given data collected under $(\mathcal{G}_{0}, \mathcal{R}_{0})$.
Let $\varepsilon(x,a)\coloneqq\mathbb{E}_{\mathcal{G},\mathcal{R}}[V|X=x,A=a]$ and $\varepsilon_{0}(x,a)\coloneqq\mathbb{E}_{\mathcal{G}_{0},\mathcal{R}_{0}}[V|X=x,A=a]$.
\begin{equation}\label{eq:unbiased_reward}
\mathop{\mathbb{E}}\limits_{\mathcal{G},\mathcal{R}}[C] = \mathop{\mathbb{E}}\limits_{\mathcal{G}_{0},\mathcal{R}_{0}}\left[C \cdot \frac{\varepsilon}{\varepsilon_{0}}\right] \approx \frac{1}{|\mathcal{D}_{0}|\cdot|\tau|}\sum_{\tau \in \mathcal{D}_{0}}\sum_{i=1}^{|\tau|} c_{i} \cdot \frac{\varepsilon(x_{\tau},a_{i})}{\varepsilon_{0}(x_{\tau},a_{i})}.
\end{equation}
Eq.~\ref{eq:unbiased_reward} provides a general unbiased estimator for the reward under $(\mathcal{G}, \mathcal{R})$, computed from data collected under $(\mathcal{G}_{0}, \mathcal{R}_{0})$.\footnote{Note that, for general two-stage ranking scenarios, this is a novel contribution to the research literature in and of itself. Existing work on off-policy corrections in two-stage recommender systems only considers a top-$1$ scenario, instead of a ranking policy~\cite{Ma2020}.}
For simplicity of notation, but without loss of generality, we now restrict ourselves to a deterministic ranking policy $\mathcal{R}$ and assume $\mathcal{G}$ is constant (i.e. $\mathcal{G}\equiv\mathcal{G}_{0}$).
With a slight abuse of notation, we briefly denote with $\mathcal{R}(x,a)$ the \emph{rank} at which item $a$ is placed when policy $\mathcal{R}$ is presented with context $x$.
In doing so, we observe that the importance weights in Eq.~\ref{eq:unbiased_reward} can be simplified to:
%\begin{equation}
$ 
    \frac{\varepsilon(x,a)}{\varepsilon_{0}(x,a)}=\frac{\mathsf{P}(V=1|R=\mathcal{R}(x,a))}{\mathsf{P}(V=1|R=\mathcal{R}_{0}(x,a))}.$  
%\end{equation}
We stress again that we simply adopt this view for simplified notation, but that the derivation holds for general \emph{stochastic} two- or single-stage ranking systems alike.

Recall from Eq.~\ref{eq:quality} that $\mathsf{P}(Q|X,A)=\frac{\mathsf{P}(C|X,A,R)}{\mathsf{P}(V|R)}$.

Now, we can formally describe the \emph{discounted cumulative gain} (DCG) metric as an importance sampling estimator:
\begin{flalign}\label{eq:formal_DCG}
\begin{aligned}
    &\mathop{\mathbb{E}}\limits_{(a_{1},\ldots,a_{k}) \sim \mathcal{R}(x)}[C] \approx f_{\rm DCG}(\mathcal{D}_{0},\mathcal{R})\\
     &= \frac{1}{|\mathcal{D}_{0}|\cdot|\tau|}\sum_{\tau \in \mathcal{D}_{0}}\sum_{i=1}^{|\tau|} c_{i}\cdot\frac{\mathsf{P}\left(V=1|R=\mathcal{R}(x_{i},a_{i})\right)}{\mathsf{P}(V=1|R=i)} \\
     &\approx \frac{1}{|\mathcal{D}_{0}|\cdot|\tau|}\sum_{\tau \in \mathcal{D}_{0}}\sum_{i=1}^{|\tau|}  \mathsf{P}(Q|X=x_{\tau},A=a_{i})\cdot\mathsf{P}(V=1|R=\mathcal{R}(x_{i},a_{i})).    
\end{aligned}
\end{flalign}
Through this derivation, two different views of the DCG metric arise.
That is, we either
\begin{enumerate*}
\item view it as a pure importance sampling estimator that reweights the exposure that is allocated to a certain item in a certain context, or
\item we view it as a way to de-bias observed interactions (i.e. estimate $Q$ from $C$ and $V$), and use the position-based model (Asm.~\ref{ass:pbm}) and the examination hypothesis (Asm.~\ref{ass:exam}) to obtain a final estimate of the cumulative reward.
\end{enumerate*}

If the assumptions laid out above hold, Eq.~\ref{eq:formal_DCG} provides an unbiased estimate of the online reward policy $\mathcal{R}$ will incur, based on data collected under $\mathcal{R}_{0}$; providing a strong motivation for DCG.
Even though unbiasedness is an attractive theoretical property, this estimator's variance can become problematic in cases where the logging and target policies $(\mathcal{R}_{0},\mathcal{R})$ diverge. 
We can adopt methods that were originally proposed to strike a balance between bias and variance for general IPS-based estimators, such as clipping the weights~\cite{Ionides2008,Gilotte2018}, self-normalising them~\cite{Swaminathan2015snips}, adapting the logging policy~\cite{Tucker2023}, or extending the estimator with a reward model to enable doubly robust estimation~\cite{Dudik2011,Kiyohara2022,Oosterhuis2023}.
Similarly, when the logged data that is used for offline evaluation was collected by multiple logging policies, ideas from ``\emph{multiple importance sampling}''~\cite{Elvira2019} are effective at reducing the variance of the final estimator~\cite{Agarwal2017,Kallus2021Optimal}.
\citeauthor{Saito2023_ICML} describe extensions for large action spaces~\cite{Saito2022_ICML,Saito2023_ICML}.

In the traditional IR use-case of web search, it is often assumed that we have access to human-annotated relevance labels ${\rm rel}(q,d)$ for query-document pairs $(q,d)$.
Such crowdsourced labels are seen as a proxy to $\mathsf{P}(Q|X,A)$, which makes them understandably attractive.
Nevertheless, for an offline evaluation metric to be useful in real-world recommendation systems, access to direct relevance labels is seldom a realistic requirement.
The \emph{discount} function for DCG that is most often used in practice, makes the assumption that $\mathsf{P}(V=1|R=i)=\frac{1}{\log_{2}(i+1)}$ is a good approximation for empirical exposure (see, e.g.,~\cite{Chapelle2009}).
This gives rise to the more widely recognisable form of DCG, as $\sum_{i=1}^{k}\frac{{\rm rel}(q,d_{i})}{\log_{2}(i+1)}$.
We note that neither one of these additional assumptions is likely to hold in real-world applications, which imply that even if Assumptions~\ref{ass:no_RL}--\ref{ass:full_support} hold, this estimator is \emph{biased}.
The more general form presented in Eq.~\ref{eq:formal_DCG}, however, is supported by a theoretical framework that allows for counterfactual evaluation, formally describing settings for its appropriate use.
%\vspace{-3ex}
\begin{table*}[!t]
%\vspace{-4ex}
\setlength{\fboxrule}{1pt}
\begin{minipage}{0.65\textwidth}
\begin{flushleft}
    \begin{tabular}{cccccccc}
    \toprule
    \textbf{Top-1 Model} & \textbf{DCG}($x_{1}$) & \textbf{DCG}($x_{2}$) & \textbf{nDCG}($x_{1}$) & \textbf{nDCG}($x_{2}$) &~& \textbf{DCG}($\mathbf{X}$) & \textbf{nDCG}($\mathbf{X}$) \\
    \midrule
    $\mathcal{R}(x) = a_{1}$ & 1.00 & 1.00 & 1.00 & 0.29 &~& 1.00 & \fcolorbox{Maroon}{white}{0.64}\\
    $\mathcal{R}^{\prime}(x) = a_{2}$ & 0.00 & 2.50 & 0.00 & 0.71 &~& \fcolorbox{PineGreen}{white}{1.25} & 0.36\\
    \bottomrule
    \end{tabular}
\end{flushleft}
\end{minipage}
\hspace{2ex}
\begin{minipage}{0.32\textwidth}
\begin{flushright}
    ~\\%~\\
where $\quad$ $\mathbb{E}[Q|X=x_{1},A=a_{1}] = 1.0$,\\
    $\mathbb{E}[Q|X=x_{1},A=a_{2}] = 0.0$,\\
    $\mathbb{E}[Q|X=x_{2},A=a_{1}] = 1.0$,\\
    $\mathbb{E}[Q|X=x_{2},A=a_{2}] = 2.5$,\\
    ~\hfill~$\textbf{X} = \{x_{1}, x_{2}\}$.
\end{flushright}
\end{minipage}
%\vspace{-2ex}
\caption{A proof by example that, while rankings inferred from the DCG and nDCG metrics are consistent for a \emph{single sample}, they can be \emph{inconsistent} when aggregated over multiple samples (i.e. ${\rm DCG}(\bm{X}, \mathcal{R}^{\prime}) > {\rm DCG}(\bm{X},\mathcal{R}) \nRightarrow {\rm nDCG}(\bm{X},\mathcal{R}^{\prime}) > {\rm nDCG}(\bm{X},\mathcal{R}) $).}\label{tab:proof}
%\vspace{-4ex}
\end{table*}

\section{Normalising DCG is Inconsistent}\label{sec:ndcg_inconsistent}
We have introduced the DCG metric from first principles, and have shown that under several assumptions, it can be seen as an unbiased estimator for the reward that a new ranking policy $\mathcal{R}$ will obtain. 
Nevertheless, this metric seldom appears in the research literature in its unadulterated form.
Much more prevalent is \emph{normalised} DCG (nDCG), which rescales the DCG metric to be at most $1$ for an ``ideal'' ranking.
For notational simplicity, we define ground truth quality labels as $\rho(x) = \left[ \mathbb{E}[Q|X=x,A=a] \forall a \in \mathcal{A}\right]$.
Then, we can compute \emph{ideal} DCG as the DCG obtained under an oracle ranker that yields $\mathcal{R}^{\star}\coloneqq{\texttt{sort}}(-\rho(x))$.
In what follows, we first show that this is reasonable practice under a single sample (context), in that it retains a \emph{consistent} ordering among ranking policies.
We then go on to show that:
\begin{enumerate*}
    \item nDCG yields \emph{inconsistent} orderings over competing policies in expectation when compared to DCG, and
    \item defining iDCG is problematic in the realistic setting of partial information (i.e. $Q$ and thus $\rho(x)$ are unobservable).
\end{enumerate*}



\begin{lemma}\label{lem:consistent_single}
The Discounted Cumulative Gain (DCG) and Normalised Discounted Cumulative Gain (nDCG) metrics yield consistent relative orders over a competing set of policies $\Omega$ that are being evaluated \textbf{for a single sample $\bm{x}$}.
That is, $$\mathop{\argsort}\limits_{\mathcal{R} \in \Omega} f_{\rm DCG}(x,\mathcal{R}) \equiv \mathop{\argsort}\limits_{\mathcal{R} \in \Omega} f_{\rm nDCG}(x,\mathcal{R}), \forall x \in \mathcal{X}.$$
\end{lemma}
\begin{proof}
For any given context $x \in \mathcal{X}$, assume a relative order exists between two methods $\mathcal{R}$ and $\mathcal{R}^{\prime}$ for a given non-negative metric $f$: i.e. $f(x,\mathcal{R}) \geq f(x,\mathcal{R}^{\prime})$.
Define $f^{\star}(x)$ as the \emph{ideal} metric value, i.e. the metric value that is obtained by the optimal ranking:
$f^{\star}(x) \coloneqq f(x, \mathcal{R}^{\star})$, where $\mathcal{R}^{\star} = \mathop{\argmax}\limits_{\mathcal{R}} f(x, \mathcal{R}) = {\texttt{sort}}(-\rho(x))$.

Because $f$ is a non-negative metric, $f^{\star}$ is non-negative, and we have that:
$$ f(x,\mathcal{R}) \geq f(x,\mathcal{R}^{\prime}) \Rightarrow  \frac{f(x,\mathcal{R})}{f^{\star}(x)} \geq \frac{f(x,\mathcal{R}^{\prime})}{f^{\star}(x)}. $$

DCG is, in general, not restricted to be non-negative.
However, if we assume that the \emph{ideal} discounted cumulative gain is non-negative, we have that the above inequality applies for $f \coloneqq f_{\rm DCG}$ and $f^{\star} \coloneqq f_{\rm iDCG}$.
\end{proof}

We believe that this (seemingly trivial) insight has led to the widespread adoption of nDCG as an offline evaluation metric in the recommender systems and Learning-to-Rank research fields, as normalised metric values where $1$ indicates a perfect model facilitate comparisons of methods over different datasets.
Indeed, when describing its prevalence in the literature, \citeauthor{Ferrante2021} argue that ``\emph{usually nDCG is preferred over DCG because it is bounded and normalised}''~\cite{Ferrante2021}.
Nevertheless, nDCG does \emph{not} retain consistent orderings with respect to DCG when the metrics are calculated over multiple samples and aggregated:

\begin{lemma}
The Discounted Cumulative Gain (DCG) and Normalised Discounted Cumulative Gain (nDCG) metrics yield \textbf{inconsistent} relative orders over a competing set of policies $\Omega$ that are being evaluated \textbf{over a set of samples $\bm{X}$}.
That is, $$\mathop{\argsort}\limits_{\mathcal{R} \in \Omega} f_{\rm DCG}(\bm{X},\mathcal{R}) \nequiv \mathop{\argsort}\limits_{\mathcal{R} \in \Omega} f_{\rm nDCG}(\bm{X},\mathcal{R}) \text{ in the general case}.$$
\end{lemma}

\begin{proof}
    We provide a proof by counterexample, for which the details are presented in Table~\ref{tab:proof}.
    Indeed, even though the $f_{\rm DCG}$ and $f_{\rm nDCG}$ metrics align for every sample in \emph{isolation} (see columns for $x_1, x_2$), they are inconsistent in \emph{aggregate} (see columns for $\bm{X}$).
\end{proof}

Discrepancies between (n)DCG have been touched upon in the IR literature, focused on search engine evaluation and blaming ``\emph{a limited number of relevance judgments}''~\cite{AlMaskari2007}.
Table~\ref{tab:proof} shows that the issue has deeper roots than this: the normalisation procedure is \emph{inconsistent}.
This insight is problematic, as virtually all offline evaluation protocols consist of first aggregating evaluation metrics over sets of samples, and then inferring preferences over competing policies based on these metric averages.
Our work shows that, when the assumptions laid out in Section~\ref{sec:dcg_unbiased} are met and DCG provides an unbiased estimate of reward, this gives rise to a theoretically sound model selection protocol.
The same statement does \emph{not} hold for nDCG as it widely appears in the research literature (see e.g. \cite[Eq. 8.9]{schutze2008introduction} and \cite[Eq. 5]{Valcarce2020}).
We hypothesise that the normalisation formula has been widely adopted for its ease-of-use, rather than for its theoretical properties.
This suggests that nDCG is of limited practical use as an offline evaluation metric, and that it should be avoided by researchers and practitioners who wish to use DCG for offline evaluation and model selection purposes.

We provide empirical evidence of discrepancies between (n)DCG on common top-$n$ recommendation evaluation tasks on publicly available data in Appendix~\ref{sec:appx}.

In the odd case where metric values that maximise at $1$ are required, we propose the use of a \emph{post}-normalisation procedure, where $f_{\rm pnDCG}(\mathcal{D},\mathcal{R}) = \frac{f_{\rm DCG}(\mathcal{D},\mathcal{R})}{f_{\rm iDCG}(\mathcal{D})}$.
Recent work on LTR in IR leverages this nDCG formulation~\cite{Oosterhuis2022}.
Indeed, through Lemma~\ref{lem:consistent_single}, one can trivially show that this metric \emph{is} consistent with respect to $f_{\rm DCG}$.

Nevertheless, we wish to advise against this practice altogether, as computing the \emph{ideal} DCG metric implies that we must construct the \emph{ideal} ranking policy $\mathcal{R}^{\star}$.
To do so, we require full knowledge of $\rho(x)$, which is hardly realistic in real-world scenarios.
In traditional IR use-cases where human-annotated relevance labels are available as \emph{ground truth}, these labels can be used to inform $\rho(x)$.
In academic recommendation datasets where we have \emph{explicit} feedback and \emph{full observability} of the user-item matrix, this can inform $\rho(x)$ similarly.
In practical applications, however, we typically estimate $\widehat{\rho}(x) \approx \rho(x)$ from logged implicit feedback.
Aside from the problems that occur when accurately interpreting this feedback~\cite{Diaz2021}, such logged datasets are known to be riddled with biases that complicate estimating $\widehat{\rho}(x)$~\cite{Jeunen2021A}, resulting in only partial observability and noisy estimates that should \emph{not} be taken at face value to inform an ``\emph{optimal}'' ranker.\looseness=-1

One final argument in favour of nDCG, is that it partially alleviates the impact of outliers.
Indeed, the normalisation procedure rescales the contribution of every sample, which can be preferable in cases where strong outliers are present.
Nevertheless, in such scenarios, we would propose to \emph{first} devise a more appropriate online metric than the average cumulative reward, and \emph{then} derive an offline estimator for this quantity, rather than trying to repurpose the existing DCG estimator.
Exactly what such metrics and estimators would look like, is an interesting area for future work.


\section{Empirical Evaluation}\label{sec:experiments}


In this section, we empirically evaluate the performance of our proposed solution by a comparative study on real benchmark datasets. It is organized as follows: Subsection~\ref{sec:setup} describes the experimental setup. Subsection~\ref{sec:comparison} presents the comparative evaluation results. Subsection~\ref{sec:ablation} presents the evaluation results of ablation study. Finally, Subsection~\ref{sec:sensitivity} evaluates the performance sensitivity of GML w.r.t key parameters. 


\subsection{Experimental Setup} \label{sec:setup}

  We use four widely used benchmark datasets in our empirical study: 
\begin{itemize}
\item \textbf{MiniImageNet\cite{ravi2017optimization}}: it contains totally 100 classes, each of which contains 600 images with a size of 84$\times$84. The classes are split among training, validation, and test sets by the ratio of (64:16:20);

\item \textbf{TieredImageNet\cite{ren2018meta}}:it was created by selecting 34 categories from the ILSVRC-2012 imagenet, with each superclass containing 10-30 subclasses. There are totally 20 superclasses (351 subclasses) on the training set, 6 superclasses (97 subclasses) in the validation set and 8 superclasses (160 subclasses) in the test set. All the images are of the size of 84$\times$84; 

\item \textbf{Cifar-FS\cite{bertinetto2018meta}}: it contains totally 100 classes, each of which contains 600 images with the size of 32$\times$32. The classes are split among training, validation and test sets by the ratio of (64:16:20);

\item \textbf{CUB-200-2011\cite{wah2011caltech}}: also known as Caltech-UCSD Birds-200-2011, it contains totally 200 bird species, which are split among training, validation, and test sets by the ratio of (100:50:50).
\end{itemize}	

%methods involve utilizing common features or patterns in the samples for classification. Examples of inductive methods in the context of few-shot classification include 

 We compare the proposed GML solution with the SOTA methods of both inductive learning and transductive learning: 1) the methods of inductive learning include recently proposed ProtoNet~\cite{2017Prototypical}, MetaQDA~\cite{zhang2021shallow}, SetFeat-12~\cite{afrasiyabi2022matching} and PFENet~\cite{zhao2022self}. Among them, PFENet reported the overall best performance; 2) the methods of transductive learning include recently proposed TPN~\cite{sung2018learning}, LaplacianShot~\cite{ziko2020laplacian}, COM-FSC~\cite{liu2023cycle}, DFMN-MCT~\cite{kye2020meta}, PT-MAP~\cite{hu2021leveraging}, EASE+SIAMESE~\cite{zhu2022ease}, EASY~\cite{bendou2022easy} and PEM$_n$E-BMS*~\cite{hu2022squeezing}. Among them, the most recently proposed models, e.g., EASY, EASE+SIAMESE and PEM$_n$E-BMS*, have reported highly competitive performance. 

  Due to the large number of the compared methods, we directly compare the results of GML in term of accuracy with the results that have been reported in these methods' original papers. For fair comparison, on each workload, as usual, we report the average and standard variance over 10000 rounds of testing. Since most of the existing methods only report results on 2-3 test datasets of the 4 datasets we have used, we mark a compared method's result on a dataset as null (-) if it was not reported in the original paper.       

  For comparative evaluation, we have compared performance in both scenarios of intro-domain classification, where training and test sets come from the same data source, and cross-domain classification, where training and test sets come from different sources, e.g., a model is trained on MiniImageNet but tested on CUB-200-2011. Obviously, cross-domain classification is more challenging than intro-domain classification. In practical scenarios, it is usually expensive to manually label samples, but much easier to retrieve unlabeled samples. Therefore, we have also evaluated the robustness of the proposed GML solution by increasing the size of query set, or the samples to be labeled in the query set, and compared its performance with the existing SOTA alternatives.  

\iffalse
 models for few-shot image classification, which include: {\color{red}
\begin{itemize}
  \item Inductive Few-shot Learning: ProtoNet\cite{2017Prototypical}, MetaQDA\cite{zhang2021shallow}, SetFeat-12\cite{afrasiyabi2022matching}, and PFENet\cite{zhao2022self}. Among them, PFENet uses an improved convolutional structure, known as Self-Guided Information Convolution to obtains effective feature embeddings. PFENet is widely considered as the optimal inductive model.
  
  \item Transductive Few-shot Learning: {\color{blue}TPN\cite{sung2018learning}, LaplacianShot\cite{ziko2020laplacian}, COM-FSC\cite{liu2023cycle}, DFMN-MCT\cite{kye2020meta}, PT-MAP\cite{hu2021leveraging}, EASE+SIAMESE\cite{zhu2022ease}, EASY\cite{bendou2022easy}, and PEM$_n$E-BMS*\cite{hu2022squeezing}. The reported performance of these models, especially recently proposed ones, EASY, EASE+SIAMESE and PEM$_n$E-BMS* are highly competitive. }
\end{itemize}
}
\fi
%Among them, PEM$_n$E-BMS*\cite{hu2022squeezing} aims to make the distribution of feature vectors closer to a Gaussian distribution. It utilizes the idea of optimal transport algorithm to iteratively adjust the feature relationships between the support set and query set, aiming to obtain optimal classification performance.

% 归纳式方法:利用样本中的共同特征或规律来进行分类,例如,ProtoNet,MetaQDA,SetFeat12,FT+SGI-Conv+GCBNet等。其中,FT+SGI-Conv+GCBNet改进卷积结构the Self-Guided Information Convolution,通过获取有效的特征嵌入,将度量网络划分成多个块,以共享相邻矩阵构建多层图卷积网络,该方法是目前归纳式最优方法。
% 直推式学习:利用先验知识或已有的分类模型进行推理。例如,PT-MAP,EASE+SIAMESE,EASY,PEMnE_BMS等。其中,PEMnE_BMS将特征向量的分布更接近于高斯分布,利用最优传输算法的思想,通过迭代调整支持集和查询集间的特征关系,获取最优分类性能。
% 


We have implemented the proposed solution based on the open-sourced GML engine\footnote{https://github.com/gml-explore/gml}. In the GML implementation, we leverage both ResNet-12 and WRN-28-10 to extract deep vector representations. In the generation of \emph{KNN} features, by default, we set $k=6$. In the implementation of gradual inference, by default, we set the number of candidates with the highest evidential support at 50, and the number of candidates with the smallest approximate entropy at 10.  At each iteration of gradual inference, GML labels the 10 samples with the smallest approximate entropy by factor inference. After each iteration, given the newly labeled images, the algorithm correspondingly updates evidential support and approximate entropy estimation. In our sensitivity evaluation, we will show that the performance of GML is very stable w.r.t these parameters provided that their values are set within reasonable ranges. We have open-sourced the GML implementation\footnote{https://github.com/chn05/FSIC\_GML}.

%As usual, we measure performance by the metric of accuracy. {\color{blue} On each workload, as usual, we report the average and standard variance over 10000 rounds of testing. 

% We also set a distance threshold of 0.01 to filter out not-close-enough nearest neighbors.

\subsection{Comparative Evaluation} \label{sec:comparison}

%We have compared performance in two scenarios: 1) intro-domain classification, where training and test sets come from the same data source; 2) cross-domain classification, where training and test datasets come from different sources, e.g., a model is trained on MiniImageNet but tested on CUB-200-2011. It is obvious that compared with intro-domain classification, cross-domain classification is more challenging. 


% \hspace{-0.2in}
\textbf{Intro-domain classification:}
the comparative evaluation results have been presented in Table~\ref{table:minitiered} and~\ref{table:cifar-fs_cub}. It can be observed that the transductive approaches consistently perform considerably better than their inductive alternatives on all the workloads. It is worth pointing out that GML consistently outperforms the best transductive alternative by considerable margins on all the workloads. For instance, on MiniImageNet and TierImageNet, in the case of 5-way 1-shot learning, GML beats EASY+SIMESE, which is the best transductive approach based on the reported results, by the margins of 2.79\% and 2.41\% respectively. In the case of 5-way 5-shot learning, the improvement margins are 4.65\% and 1.5\% respectively. On Cifar-FS and CUB-100-2011, PEM$_n$E-BMS* is instead the best transductive approach. In the case of 5-way 1-shot learning, GML outperforms PEM$_n$E-BMS* by the margins of 4.58\% and 3.72\% respectively. In the case of 5-way 5-shot learning, the improvement margins are 3.89\% and 2.93\% respectively. 

\begin{table*}[htbp]
    %\setlength\tabcolsep{1.5pt}%设置表格列间距
    \centering
    \caption{Comparative results on the MiniImageNet and TieredImageNet datasets: GML clearly achieves the SOTA performance on both datasets, and the margins are considerable.}\label{table:minitiered}
    % \begin{tabular}{c,p{4cm},c,p{7cm}|}
    \small
    \begin{tabular}{c|c|c|c|c|c}
    \toprule
    \multicolumn{2}{c|}{\textbf{DataSets}} & \multicolumn{2}{c|}{\cellcolor[HTML]{9698ED}{\textbf{Mini-ImageNet}}} & \multicolumn{2}{c}{\cellcolor[HTML]{FFCE93}{\textbf{Tiered-ImageNet}}} \\ 
    \midrule
    Setting & Methods & 5-way 1-shot(\%) &5-way 5-shot(\%) & 5-way 1-shot(\%) & 5-way 5-shot(\%)\\ 
    \midrule
    \multirow{10}{*}{\textbf{Inductive}} &  Relation\cite{sung2018learning}  &$52.48\pm0.86$ & $69.83\pm0.68$  & $-$ & $-$  \\
    & Baseline++\cite{chen2019closer}  &  $53.97\pm0.79$ & $75.90\pm0.61$  & $-$ & $-$\\
    &MatchingNet\cite{2016Matching} &   $52.91\pm0.88$ & $68.88\pm0.69$ & $-$ & $-$ \\
    &ProtoNet\cite{2017Prototypical} &  $54.16\pm0.82$ & $73.68\pm0.65$ &  $65.65\pm0.92$ & $83.40\pm0.65$ \\
    &$S2M2_R$~\cite{mangla2020charting} &  $64.93\pm0.18$ & $83.18\pm0.11$ & $73.71\pm0.22$ & $88.59\pm0.14$\\
    &DeepEMD\cite{zhang2020deepemd} &  $65.91\pm0.82$ & $82.41\pm0.56$ & $71.16\pm0.87$ & $86.03\pm0.58$ \\
    &FRN\cite{wertheimer2021few} &   $66.45\pm0.19$ & $82.83\pm0.13$ & $71.16\pm0.22$ & $86.01\pm0.15$\\
    &MetaQDA\cite{zhang2021shallow} &  $67.83\pm0.64$ & $84.28\pm0.69$ & $74.33\pm0.65$ & $89.56\pm0.79$\\
    &SetFeat12\cite{afrasiyabi2022matching} &  $68.32\pm0.62$ & $82.71\pm0.46$ & $73.63\pm0.88$ & $87.59\pm0.57$ \\
    &PFENet\cite{zhao2022self} &  $68.76\pm0.75$ & $84.67\pm0.52$ & $74.93\pm0.84$ & $89.62\pm0.50$ \\
    \midrule
    \multirow{10}{*}{\textbf{Transductive}} &TPN\cite{sung2018learning} & $55.51\pm0.86$ & $69.86\pm0.65$ &  $59.91\pm0.94$ & $73.30\pm0.75$\\
    &COM-FSC\cite{liu2023cycle} &   $68.92\pm0.72$ & $85.37\pm0.49$ &  $79.69\pm0.74$ & $90.57\pm0.45$\\
    &LaplacianShot\cite{ziko2020laplacian} & $75.57\pm0.19$ & $84.72\pm0.13$ &  $80.30\pm0.22$ & $87.93\pm0.15$\\
    &DFMN-MCT\cite{kye2020meta} &  $78.55\pm0.86$ & $86.03\pm0.42$ &  $80.89\pm0.84$ & $87.30\pm0.49$\\
    &Transd-CNAPS+FETI\cite{bateni2022enhancing} &   $79.90\pm0.80$ & $91.50\pm0.40$ & $73.80\pm0.10$ & $87.70\pm0.60$ \\
    &PT-MAP\cite{hu2021leveraging} &   $82.92\pm0.26$ & $88.82\pm0.13$ & $85.67\pm0.26$ & $90.45\pm0.14$\\
    &EASE+SIAMESE\cite{zhu2022ease} &$83.00\pm0.21$ & $88.92\pm0.13$ & $88.96\pm0.23$ & $92.63\pm0.13$\\
    &EASY\cite{bendou2022easy} & $84.04\pm0.23$ & $89.14\pm0.11$ &  $84.29\pm0.24$ & $89.76\pm0.14$\\
    &PEM$_n$E-BMS*\cite{hu2022squeezing} &  $83.35\pm0.25$ & $89.53\pm0.13$ & $86.07\pm0.75$ & $91.09\pm0.14$\\
    % GiFeic & Transductive & WRN & \textcolor{red}{\textbf{0$\pm$0.32}} & \textbf{93.14$\pm$0.25} & \makecell{ResNet-12\\+WRN} & \textcolor{red}{\textbf{0$\pm$0.42}} & \textcolor{red}{\textbf{0$\pm$0.43}}\\
    \rowcolor[HTML]{FFFFFF}&\cellcolor[HTML]{96FFFB} \textbf{GML}  & \cellcolor[HTML]{96FFFB} \textbf{$85.79\pm0.32$} & \cellcolor[HTML]{96FFFB} \textbf{$93.57\pm0.25$} & \cellcolor[HTML]{96FFFB}  \textbf{$91.37\pm0.42$} & \cellcolor[HTML]{96FFFB} \textbf{$94.13\pm0.13$}\\
    \bottomrule
    \end{tabular}
    \end{table*}



\begin{table*}[htbp]
    %\setlength\tabcolsep{1.5pt}%设置表格列间距
    \centering
    \caption{Comparative results on the Cifar-FS and CUB-100-2011 datasets: GML clearly achieves the SOTA performance on both datasets, and the margins are considerable.}\label{table:cifar-fs_cub}
    % \begin{tabular}{c,p{4cm},c,p{7cm}|}
    \small
    \begin{tabular}{c|c|c|c|c|c}
    \toprule
    \multicolumn{2}{c|}{\textbf{DataSets}} & \multicolumn{2}{c|}{\cellcolor[HTML]{C0C0C0}{\textbf{Cifar-FS}}}  &  \multicolumn{2}{c|}{\cellcolor[HTML]{FE996B}{\textbf{CUB-100-2011}}} \\ 
    \midrule
    \textbf{Setting} &\textbf{Method} & \textbf{5-way 1-shot(\%)} & \textbf{5-way 5-shot(\%)} & \textbf{5-way 1-shot(\%)} & \textbf{5-way 5-shot(\%)} \\ 
    \midrule
    \multirow{9}{*}{\textbf{Inductive}} & MatchingNet\cite{2016Matching} & $43.88\pm0.75$ & $57.05\pm0.76$ & $-$ & $-$\\
    &ProtoNet\cite{2017Prototypical} & $41.54\pm0.76$ & $57.08\pm0.76$ & $-$ & $-$\\
    &DeepEMD\cite{zhang2020deepemd} &$46.47\pm0.78$ & $63.22\pm0.71$ &$-$ & $-$\\
    &SetFeat12\cite{afrasiyabi2022matching} &  $-$ & $-$ &  $79.60\pm0.80$ & $90.48\pm0.44$ \\
    &$S2M2_R$\cite{mangla2020charting} & $74.81\pm0.19$ & $87.47\pm0.13$ & $80.68\pm0.81$ & $90.85\pm0.44$\\
    &RENet\cite{2021Relational} &   $74.51\pm0.46$ & $86.60\pm0.32$ &  $79.49\pm0.44$ & $91.11\pm0.24$\\
    &FRN\cite{wertheimer2021few} & $-$ & $-$ &  $83.55\pm0.19$ & $92.92\pm0.10$ \\
    &PFENet\cite{zhao2022self} &  $-$ & $-$ & $86.09\pm0.19$ & $93.15\pm0.10$ \\
    &MetaQDA\cite{zhang2021shallow} &  $75.83\pm0.88$ & $88.79\pm0.75$ & $-$ & $-$\\
    \midrule
    \multirow{7}{*}{\textbf{Transductive}} & COM-FSC\cite{liu2023cycle} & $-$ & $-$ &  $83.93\pm0.66$ & $93.95\pm0.30$\\
    &EASY\cite{bendou2022easy} & $87.16\pm0.21$ & $90.47\pm0.15$ &  $90.56\pm0.19$ & $93.79\pm0.10$\\
    &iLPC\cite{Lazarou_2021_ICCV}& $86.51\pm0.23$ & $90.60\pm0.48$ &  $91.03\pm0.63$ & $94.11\pm0.30$\\
    &EASE+SIAMESE\cite{zhu2022ease} &  $87.60\pm0.23$ & $90.60\pm0.16$ & $91.68\pm0.19$ & $94.12\pm0.09$\\
    &PT-MAP\cite{hu2021leveraging} &  $87.69\pm0.23$ & $90.68\pm0.15$ & $91.55\pm0.19$ & $93.99\pm0.10$\\
    &PEM$_n$E-BMS*\cite{hu2022squeezing} & $87.83\pm0.22$ & $91.20\pm0.15$ & $91.91\pm0.18$ & $94.62\pm0.09$\\
    \rowcolor[HTML]{FFFFFF}  & \cellcolor[HTML]{96FFFB} \textbf{GML} & \cellcolor[HTML]{96FFFB} \textbf{$92.41\pm0.32$} & \cellcolor[HTML]{96FFFB} \textbf{$95.09\pm0.18$} &  \cellcolor[HTML]{96FFFB} \textbf{$95.63\pm0.06$} & \cellcolor[HTML]{96FFFB} \textbf{$97.55\pm0.43$}\\
    \bottomrule
    \end{tabular}
    \end{table*}
    
	Even with the existing inductive and transductive solutions being considered as a whole, GML consistently improves the reported SOTA performance on the four workloads by considerable margins. In the case of 5-way 1-shot learning, the improvement margins over the SOTA results are 1.75\%, 2.41\%, 4.58\%, and 3.95\% respectively. In the case of 5-way 5-shot learning, the improvement margins are 2.07\%, 1.5\%, 3.89\%, and 2.93\% respectively. Due to the widely recognized challenge of few-shot learning, these margins are truly considerable. Since our GML solution extracts discriminative features by the same deep neural models leveraged by the existing solutions, these evaluation results clearly demonstrate that compared with the existing transductive alternatives, gradual inference is a more effective mechanism for few-shot learning.       

%Our evaluation results clearly demonstrate the performance advantage of GML over the existing approaches in the scenario of few-shot learning. 


\begin{table}[htbp]
    %\setlength\tabcolsep{1.5pt}%设置表格列间距
    \centering
    \caption{Comparative results on cross-domain classification: the models are trained on MiniImageNet but tested on CUB-100-2011.}\label{table:cross}
    % \begin{tabular}{c,p{4cm},c,p{7cm}|}
    \small
    \begin{tabular}{c|c|c|c}
    \toprule
    \textbf{DataSets} & \multicolumn{3}{c}{\textbf{MiniImageNet$\rightarrow$CUB-100-2011 (5-way)}} \\
    \toprule
    \textbf{Setting} & \textbf{Methods}  &\textbf{1-shot(\%)} &\textbf{5-shot(\%)} \\ 
    \midrule
    \multirow{4}{*}{\textbf{Inductive}} &PFENet\cite{zhao2022self}   &$48.27$ & $69.51$ \\
    &$S2M2_R$\cite{mangla2020charting} & $48.24$ & $70.44$ \\
    &MetaQDA\cite{zhang2021shallow}  &  $53.75$ & $71.84$\\
    &FRN\cite{wertheimer2021few} & $54.11$ & $77.09$ \\
    \midrule
    \multirow{4}{*}{\textbf{Transductive}} &LaplacianShot\cite{ziko2020laplacian} & $55.46$ & $66.33$ \\
    &COM-FSC\cite{liu2023cycle} &  $53.14$ & $73.02$ \\
    &PEM$_n$E-BMS*\cite{hu2022squeezing}& $63.00$ & $79.15$ \\
    % \midrule
    \rowcolor[HTML]{FFFFFF}  & \cellcolor[HTML]{96FFFB} \textbf{GML} &\cellcolor[HTML]{96FFFB} $\textbf{67.29}$ & \cellcolor[HTML]{96FFFB} $\textbf{82.81}$ \\
    \bottomrule
    \end{tabular}
\end{table}

% \vspace{0.05in}
% \hspace{-0.2in}
\textbf{Cross-domain classification:} 
 cross-domain classification is usually performed on two datasets containing the same type of objects. Since both MiniImageNet and CUB-100-2011 contain images of bird species, as in previous work~\cite{mangla2020charting,zhang2021shallow,zhao2022self,ziko2020laplacian}, we train models on the MiniImageNet dataset and test its performance on another dataset of CUB-100-2011. 
	
  The comparative evaluation results have been presented in Table~\ref{table:cross}. It can be observed that on both cases of 1-shot and 5-shot learning, GML outperforms the existing alternatives by considerable margins. Specifically, on 1-shot learning, GML beats PEM$_n$E-BMS*, which is the best approach among the existing alternatives, by 4.29\% in terms of accuracy. On 5-shot learning, GML's improvement margin over PEM$_n$E-BMS* is similarly large at 3.66\%. Our experimental results clearly demonstrate that by gradual learning, the features learned in training classes can be better generalized to unseen classes. 

% It is interesting to point out that these observed margins are even more considerable than what have been observed on intro-domain classification. 


% Figure environment removed


\begin{table*}[htbp]
    %\setlength\tabcolsep{1.5pt}%设置表格列间距
    \centering
    \caption{The evaluation result of GML ablation study on MiniImagenet and Cifar-FS: using both ResNet-12 and WRN-28-10 vs using either of them.}\label{table:backbone}
    % \begin{tabular}{c,p{4cm},c,p{7cm}|}
    \small
    \begin{tabular}{c|c|c|c|c|c}
    \toprule
    \multicolumn{2}{c|}{\textbf{DataSets}}  &  \multicolumn{2}{c|}{\textbf{MiniImageNet}} &\multicolumn{2}{c}{\textbf{Cifar-FS}} \\
    \midrule
    \textbf{Methods} & \textbf{Network} & \textbf{5-way 1-shot(\%)} & \textbf{5-way 5-shot(\%)}  & \textbf{5-way 1-shot(\%)} & \textbf{5-way 5-shot(\%)} \\ 
    \midrule
    \multirow{3}{*}{\textbf{GML}} &  ResNet-12 & $84.74$ & $89.50$ & $85.69$ & $89.43$\\
    &  WRN-28-10 & $83.28$ & $88.80$ & $84.06$ & $88.35$\\
    \rowcolor[HTML]{FFFFFF}  &  \cellcolor[HTML]{96FFFB} ResNet-12+WRN-28-10 &\cellcolor[HTML]{96FFFB}$\textbf{85.79}$ &\cellcolor[HTML]{96FFFB}$\textbf{93.57}$ & \cellcolor[HTML]{96FFFB}$\textbf{92.41}$ &\cellcolor[HTML]{96FFFB}$\textbf{95.09}$\\
  
    \bottomrule
    \end{tabular}
    \end{table*}

\begin{table*}[htbp]
        %\setlength\tabcolsep{1.5pt}%设置表格列间距
        \centering
        \caption{ Parameter sensitivity evaluation results.}\label{table:sensitivity}
        % \begin{tabular}{c,p{4cm},c,p{7cm}|}
        \small
        \begin{tabular}{c|c|c|c|c|c}
        \toprule
        \multicolumn{2}{c|}{\textbf{DataSets}} & \multicolumn{2}{c|}{\textbf{MiniImageNet}} & \multicolumn{2}{c}{\textbf{Cifar-FS}} \\
        \midrule
         & \textbf{top-m/n/k}  &\textbf{5-way 1-shot(\%)} & \textbf{5-way 5-shot(\%)}  &\textbf{5-way 1-shot(\%)} & \textbf{5-way 5-shot(\%)} \\ 
        \midrule
        \rowcolor[HTML]{FFFFFF} \multirow{2}{*}{\textbf{$w.r.t$ m (n=10,k=6)}}  & m=40 &  $85.61$ & $ 93.28$  &  $92.16$ & $95.12$ \\
         & m=60 &  $ 85.56 $ & $ 93.23 $ &  $92.31$ & $ 94.59$  \\
         \midrule
         \multirow{2}{*}{\textbf{$w.r.t$ n (m=50,k=6)}} & n=8 &  $ 85.39$ & $ 93.48$  &  $92.09$ & $94.95$ \\
         & n=12 & $85.69$ & $ 93.56$  &  $92.24$ & $95.10$ \\
         \midrule
        \multirow{2}{*}{\textbf{$w.r.t$ k (m=50,n=10)}}  & k=5 & $ 85.76 $ & $ 93.33 $  &  $ 92.40 $ & $ 94.98 $   \\
        % & k=3 &  $85.59$ & $ 92.88$  &  $92.18$ & $94.57$ \\
        & k=7 & $ 85.15 $ & $ 93.04 $  &  $ 91.83 $ & $ 94.66 $   \\
        \midrule
        \rowcolor[HTML]{96FFFB}\textbf{GML}  & m=50, n=10, k=6 &$ 85.79 $ &$ 93.57 $  & $ 92.41 $ & $ 95.09 $   \\
        \bottomrule
        \end{tabular}
\end{table*}
%In practical scenarios, it is usually expensive to manually label many samples, but much easier to retrieve many unlabeled samples. Therefore, we evaluate the robustness of the proposed GML solution by increasing the size of query set, or the samples to be labeled in the query set. I


\textbf{Comparative evaluation with increasing size of query set:}
in the classical setting of few-shot learning, the number of queries per class is set at 15. Therefore, we increase the number of queries from 15 to 30, 50, 100, and finally up to 200, and compare the performance of GML with two recently proposed approaches, EASY and EASE, which can be considered as the SOTA representatives of the existing transductive approaches. 

%{\color{red} We report the evaluation results on MiniImageNet and Cifar-FS; the evaluation results on the other two datasets are similar, but omitted here.} 

The comparative evaluation results on MiniImageNet and Cifar-FS have been presented in Figure~\ref{fig:query structure}. The evaluation results on the other two datasets are similar, thus omitted due to space limit. It can be observed that on both 1-shot and 5-shot learning, the performance of GML consistently improves as the number of queries increases, even though by different margins on different workloads. The common pattern is that the performance of GML initially improves considerably as the number of queries increases from 15 to 30, but then gradually flattens out as it continues to increase. For instance, in the case of 1-shot learning on MiniImageNet, the accuracy improves by the margin of around 6\%. from 85.79\% to 91.88\%, when the number of queries increases from 15 to 30, and then continues to improve, even though by smaller margins, up to 94.3\% when the number of queries reaches 200. In comparison, the performance of EASE and EASY fluctuates only marginally when the number of queries increases. For instance, in the case of 1-shot learning on MiniImageNet, the performance of EASY even deteriorates marginally from 83.84\% to 82.98\%; but on 5-shot learning, its performance instead improves slightly, from 88.37\% to 89.06\%. These observations clearly demonstrate that GML is more robust than the existing transductive alternatives. They bode well for its application in real scenarios.  




\subsection{Ablation Study} \label{sec:ablation}



To verify the efficacy of leveraging two distinct backbones, i.e., ResNet-12 and WRN-28-10, for gradual learning, we have conducted an ablation study on the GML approach, which compares the solution using both models with the alternatives using either of them. The evaluation results on MiniImagenet and Cifar-FS have been presented in Table~\ref{table:backbone}. The evaluation results on the two other datasets are similar, thus omitted here. It can be observed that the GML using both of them performs considerably better that the alternatives using either of them. These results clearly demonstrate that even though both ResNet-12 and WRN-28-10 have been constructed based on the ResNet network, they are some extent complementary in feature extraction, and integrating them for knowledge conveyance can effectively improve the performance of gradual learning. 



%{\color{blue}
\textbf{An Illustrative Example:} in a run on MiniImageNet, inference accuracy based on ResNet-12 or WRN-28-10 is 85.33\% and 68\% respectively, but the accuracy is better at 94.67\% if gradual inference uses both of them. As shown in Figure~\ref{fig:example}, we take the sample with the id of 64 as an example. In the factor graph constructed based on ResNet, both unary and binary factors point to the class of $c_3$ for the sample. However, based on WRN-28-10, the factors point to its ground-truth class of $c_2$. It can be observed that the fused factor graph constructed based on both ResNet-12 and WRN-28-10 correctly point to the class of $c_2$ while labeling the sample of 64. It is noteworthy that the inference order in different factors may vary. This example clearly demonstrates that the framework of GML can effectively fuse diverse and noisy features to improve gradual knowledge conveyance.
%}     

% Figure environment removed

%a task is randomly extracted, and the feature vectors extracted by ResNet-12, WRN-28-10, ResNet-12+WRN-28-10 are used respectively. The prediction accuracy in GML are: 85.33\%, 68\% and 94.67\%. Take sample 64 as an example to illustrate the effectiveness of this method. (a) is the unary and binary factor constructed by the eigenvector of ResNet-12. When marking sample 64, the weight of the unary and binary factor of the evidence variable points to the $c_3$. (b) is the unary and binary factor constructed by obtaining the feature vector through WRN-28-10, and the weight of the unary and binary factor points the 64 sample to the $c_2$. (c) Constructing two types of unary and binary factors, due to the increase of features, the fusion of DS theory changes the order of reasoning, so sample 64 is indicated as the $c_2$ by more evidence variables. It is obvious that the method proposed in this paper adopts gradual inference to fuse different factors for feature complementation, so as to obtain higher accuracy.



%The accuracy on ResNet-12 and WRN-28-10 models is lower, while the accuracy on integrated feature models ResNet-12+WRN-28-10 is higher. Due to the issue of feature redundancy when adding ResNet-12 and WRN-28-10 features, its accuracy is lower than ResNet-12+WRN-28-10. gradual inference enhances detection accuracy by obtaining image feature vectors with different expressive capabilities and applying different features to factor graph modeling to improve evidence support for factors.

\subsection{Parameter Sensitivity Study} \label{sec:sensitivity}

In this subsection, we evaluate the performance sensitivity of the proposed GML solution w.r.t one key parameter of feature extraction, the $k$ value of k-nearest neighbors (\emph{KNN}) for binary feature extraction, and two key parameters of scalable gradual inference, the number of candidates with the most evidential support and the number of candidates with the smallest approximate entropy, or $m$ and $n$ as shown in Algorithm~\ref{alg:gradualinference}. By default, we set $m$ = 50, $n$ = 10 and $k$ = 6. In the sensitivity study, we vary the value of a parameter, but fix the values of the other two parameters. We set the values of parameters within reasonable ranges. Specifically, we vary the value of $k$ from 5 to 7, the value of $m$ from 40 to 60, and the value of $n$ from 8 to 12. We report the evaluation results on the MiniImageNet and Cifar-FS workloads; the results on other workloads are similar, thus omitted here. 
	
The detailed evaluation results have been presented in Table~\ref{table:sensitivity}. It can be observed that the performance of GML only fluctuates marginally ($\leq 0.5\%$ in most cases) as the values of $m$, $n$ and $k$ change. These observations clearly indicate that the performance of GML is very robust w.r.t these parameters. They bode well for their application in real scenarios. 

%\section{Beyond current assumptions: perspectives going forward}\label{sec:beyond}
\vspace{-2ex}\section{Perspectives going forward}\label{sec:beyond}
%In what follows, we revisit the assumptions that we have made in Section~\ref{sec:dcg_unbiased} to formally state \emph{when} the DCG metric can be considered an unbiased estimator of online reward, and hence, a suitable offline evaluation metric.
In what follows, we revisit the assumptions that are \emph{necessary} to consider the DCG metric an unbiased estimator of online reward.

%\paragraph{1. Reward independence across trajectories.}
\textit{1. Reward independence across trajectories} is necessary to avoid having to model any internal user \emph{state} that is influenced by actions taken by a ranking policy.
Indeed, if we do allow this to happen, we must resort to Reinforcement Learning (RL) formulations of our problem, which inhibits the simple form that DCG allows.
Nevertheless, unbiased evaluation of RL policies is an active research area, which has found applications in recommendation research~\cite{Chen2019}, also for two-stage policies (without considering rankings)~\cite{Ma2020}.
\citeauthor{Ie2019SlateQ} can provide inspiration for learnt RL policies in top-$n$ recommendation domains with DCG-like reward structures~\cite{Ie2019SlateQ}.

\textit{2. Position-based model (PBM).}
%\paragraph{2. Position-based model (PBM)}
The classical PBM allows for general formulations of $\mathsf{P}(V|R)$, including the widely adopted functional form $\frac{1}{\log_{2}(i+1)}$.
The recently proposed Contextual PBM~\cite{Fang2019} can be plugged into Eq.~\ref{eq:exposure} to directly provide an unbiased DCG formulation with a context-dependent discount function, enjoying the same theoretical guarantees we have derived for DCG under the PBM.
A variety of other click models~\cite{chuklin2015click} have been proposed in the research literature~\cite{Borisov2016,Chen2020_CACM}, as well as ways to evaluate them~\cite{Deffayet2022}.
We expect that our work provides a basis for further connections to be drawn between click models and unbiased evaluation metrics.

\textit{3. Reward independence across ranks.}
%\paragraph{3. Reward independence across ranks.}
When we do not assume any structure between the actions taken by the ranking policy and the observed rewards, the problem quickly becomes intractable, as we suffer from a combinatorial explosion of the action space.
This is a well-known problem, and the independence assumption has been adopted (either explicitly or implicitly) by a wide array of related work~\cite{Swaminathan2017,Ie2019SlateQ, Bendada2020,Jeunen2021B}.
Note that this assumption does not simply relate to \emph{observing} rewards, but to the underlying distribution of $Q$.
Indeed, related work that adopts a cascading user behaviour model also relies on this assumption, as the cascade relates to the distribution of $V$ (and thus $C$) rather than that of $Q$~\cite{McInerney2020, Kiyohara2022}.
Evaluation metrics have been proposed to encode concepts of listwise novelty and diversity into DCG-like formulations for top-$n$ recommendations~\cite{Clarke2008,Parapar2021}; we conjecture they can be extended to our unbiased setup as well.

%\paragraph{4. Examination hypothesis.}
\textit{4. The examination hypothesis} implies that \emph{exposure} bias (through $V$) is the main culprit that makes $C$ a noisy indicator of $Q$.
Differences in exposure can then purely come from \emph{position} bias (as in the PBM), but they can also be perpetuated by \emph{selection} bias (as made evident by Eq.~\ref{eq:exposure})~\cite{Diaz2020,Jeunen2021B}.
Other sources of bias have been raised in the literature, such as presentation or trust bias~\cite{Agarwal2019TrustBias,Vardasbi2020}.
We expect that these types of biases can be incorporated into the theoretical derivation of Sec.~\ref{sec:dcg_unbiased} to devise DCG-like formulations that remain unbiased estimators of online reward, even when additional biases are present.
Naturally, such biases are use-case-specific.

\textit{5. Full support of the logging policy.}
%\paragraph{5. Full support of the logging policy.}
The main assumption that makes IPS work, is that no actions with non-zero probability under the target policy can have zero probability under the logging policy.
Indeed, if a context-action pair is known not to be present in the data, we cannot make any inferences about its reward (with guarantees).
This is at the heart of policy-based estimation, but especially problematic in real-world systems where the action space is large and the cost of such \emph{full} randomisation, even with small probabilities, can be high.
Recent work in \emph{learning} from bandit feedback deals with such cases empirically~\cite{JeunenKDD2020,Jeunen2021A} and theoretically~\cite{Sachdeva2020, Lopez2021}, providing a source of inspiration to (partially) alleviate these issues.
%\vspace{-4ex}
\vspace{-4ex}\section{Conclusions \& Outlook}\label{sec:conclusion}
Offline evaluation of recommender systems is a common task, and known to be problematic.
This work investigates the commonly used (normalised) discounted cumulative gain metric and its uses in the research literature.
Specifically, we have investigated \emph{when} we can expect such metrics to approximate the gold standard outcome of an online experiment.
In a counterfactual estimation framework, we formally derived the necessary assumptions to consider DCG an unbiased estimator of online reward.
Whilst it is reassuring that such assumptions exist and we can directly map DCG to online metrics --- we also highlighted how this \emph{ideal} use deviates from the traditional uses of the metric in IR, and how it often appears in the research literature.
We then shifted our focus to \emph{normalised} DCG, and demonstrated its \emph{inconsistency}, both theoretically and empirically with reproducible experiments.
Indeed, even when all neccesary assumptions hold and DCG provides unbiased estimates of online reward, nDCG \emph{cannot} be used to rank competing models, as it does does \emph{not} preserve the rankings we would obtain from DCG.\looseness=-1

Through a correlation analysis between results obtained from off- and on-line experiments on a large-scale recommendation platform, we show that our \emph{unbiased} DCG estimates strongly correlate with online metrics in a real-world use-case. %even when some of the metric's underlying assumptions are violated.
Additionally, we show how the offline metric can be used to detect statistically significant online improvements with high sensitivity, further highlighting its promise for offline evaluation in both academia and industry.
Normalised DCG, on the other hand, suffers from a weaker correlation with online results, and lower sensitivity than DCG.
These results suggest that nDCG's practical utility may be limited.

We believe our work opens up interesting areas for future research, where our theoretical framework can be extended to formally assess the assumptions required by other commonly used evaluation metrics in the field.
Furthermore, theoretical and empirical connections between other types of commonly used online evaluation metrics (e.g. user retention) would be fruitful.


\begin{acks}
We are grateful to Lien Michiels, co-author of RecPack~\cite{Michiels2022}, for early feedback and help setting up the experiment in Appendix~\ref{sec:appx}.
\end{acks}

\bibliographystyle{ACM-Reference-Format}
\bibliography{bibliography}

\appendix
% Figure environment removed

\section{Empirical Evidence of (n)DCG Inconsistency on Public Data}\label{sec:appx}
%\paragraph{Empirical Evidence on Public Data}
Table~\ref{tab:proof} provides a formal proof that an ordering over competing recommendation (or IR) models obtained through a normalised metric is \emph{not} guaranteed to be consistent with the original metric.
Nevertheless, one might wonder whether this single example represents a misguided pathological case, or whether metric disagreement occurs in practice.
This gives rise to the research question:
\begin{description}    
    \item[\textbf{RQ5}] \textit{Do DCG and normalised DCG disagree when ranking recommendation models in typical offline evaluation setups?}
\end{description}
To answer this question, we make use of the RecPack Python package~\cite{Michiels2022} and the MovieLens-1M dataset~\cite{Harper2015}.
We consider two types of models, \textsc{ease}\textsuperscript{r}~\cite{Steck2019} and \textsc{kunn}~\cite{Verstrepen2014}, varying their hyperparameters to train 192 models on a fixed 50\% of the available user-item interactions, and assess their performance on the held-out 50\%.
This style of evaluation setup is prevalent in the recommendation field~\cite{Steck2013,Jeunen2019DS,Zangerle2023}.
We adopt this package, dataset and methods to provide a reproducible setup that runs in under 20 minutes on a 2021 MacBook Pro.
All source code, including hyperparameter ranges, is available at \href{https://github.com/olivierjeunen/nDCG-disagreement/}{github.com/olivierjeunen/nDCG-disagreement}.

We do not de-bias the interactions (as MovieLens does not provide information about exposure), and adopt the traditional logarithmic discount for DCG with a cut-off at rank 100.
Results are visualised in Figure~\ref{fig:disagreement} with DCG@100 on the $x$-axis and nDCG@100 on the $y$-axis.
The two metrics exhibit a linear correlation of $\approx 0.6$ (Pearson), and a rank correlation of $\approx 0.5$ (Kendall).
Whilst they are clearly correlated, practitioners should \emph{not} blindly adopt nDCG when DCG estimates their online metric.
Indeed, DCG can be formulated as an unbiased estimator of the average reward per trajectory, but nDCG cannot.
As can be seen from the plot, significant \emph{disagreement} occurs between the two metrics: when randomly choosing two observations, the empirical probability of nDCG inverting the ordering implied by DCG is roughly 25\% on this example.
Naturally, one would expect this type of disagreement to occur even more frequently when considering Learning-to-Rank algorithms that directly optimise listwise objectives such as (n)DCG~\cite{Jagerman2022,Ustimenko2020,Lyzhin2023}.

Note that this discrepancy would not occur if we would sample the exact same number of held-out items for every user (as in Leave-One-Out Cross-Validation).
Indeed, in such cases $f^{\star}(x)$ is constant $\forall x \in \mathcal{X}$, simply rescaling the metric.
Whilst this practice can be common in academic scenarios, real-world use-cases typically imply varying numbers of ``relevant'' items per user or context.

We include these results to aid in the reproducibility of the empirical phenomena we report in this work.


\end{document}