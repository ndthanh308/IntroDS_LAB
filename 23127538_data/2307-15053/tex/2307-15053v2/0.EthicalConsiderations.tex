\section*{Ethical Considerations}
Our work focuses on offline evaluation practices that are prevalent in the recommender systems field, used both by researchers and practitioners to determine whether one method outperforms another.
In industry, effective and efficient offline evaluation methods lead to shortened experimental cycles and faster deployment of new and improved models.
If these models are optimised for harmful objectives, or do not properly take the into account the interests of all stakeholders impacted by the model, this can be problematic~\cite{Abdollahpouri2022}.
Even though this is not a direct consequence of our work, we encourage researchers and practitioners to consider competing objectives appropriately~\cite{Mehrotra2018}.
In academia, offline evaluation methods that more accurately align with online experiment outcomes can lead to faster adoption in real-world applications, where similar considerations are apparent.
We do not foresee any further ethical concerns stemming from our work.