\documentclass[conference]{IEEEtran}
\IEEEoverridecommandlockouts

\usepackage{tikz}
% \usepackage{appendix}
\usepackage{amsmath}
\usepackage{amsthm}
\usepackage{amssymb}
\usepackage[]{graphicx}
\usepackage{subfig}
\usepackage{algorithm}
\usepackage[noend]{algpseudocode}
\renewcommand{\algorithmicrequire}{\textbf{Input:}} 
\renewcommand{\algorithmicensure}{\textbf{Output:}} 
\usepackage{booktabs}
\usepackage{bm}
\usepackage{multirow} 
\usepackage{multicol}
\usepackage{flushend}
\usepackage{makecell}
\usepackage{threeparttable}
\usepackage{diagbox}
\usepackage{footnote}
\usepackage{lipsum}
\usepackage{xcolor}
\usepackage{enumitem}
\usepackage{fancyhdr}
\usepackage{listings}
\definecolor{difftitle}{HTML}{000099}
\definecolor{diffstart}{HTML}{660099}
\definecolor{diffincl}{HTML}{006600}
\definecolor{diffrem}{HTML}{AA3300}
\lstdefinelanguage{diff}{
    backgroundcolor=\color{white},  % choose the background color
    basicstyle=\ttfamily\small,
    morecomment=[l][\color{difftitle}]{diff},
    % morecomment=[l][\color{difftitle}]{index},
    morecomment=[l][\color{diffstart}]{@@},
    morecomment=[f][\color{diffincl}]{+},
    morecomment=[f][\color{diffrem}]{-},
    columns=fullflexible,
    tabsize=4,
    breaklines=true,% automatic line breaking only at whitespace
    captionpos=b, % sets the caption-position to bottom
    frame=none,
}    

\usepackage{hyperref}
\hypersetup{
    colorlinks=true,
    linkcolor=blue,
    filecolor=magenta,      
    urlcolor=blue,
    }
% The preceding line is only needed to identify funding in the first footnote. If that is unneeded, please comment it out.
\usepackage[utf8x]{inputenc}
\usepackage{amscd}
\usepackage{amsmath}
%\usepackage{amssymb}
\usepackage{amsthm}

\usepackage[colorinlistoftodos]{todonotes}

\usepackage{thmtools}
\usepackage{thm-restate}
\usepackage{mathtools}
\usepackage[full]{complexity}
\usepackage{longtable}

%\usepackage[usenames,dvipsnames]{xcolor}
\usepackage{xcolor}
% % tables
 \usepackage{array}

\usepackage{bbm}
\usepackage{comment}
\usepackage{enumerate}
\usepackage{floatrow}


\usepackage{parallel,enumitem}

\usepackage{xspace}
\usepackage{paralist}
\usepackage{xifthen}
\usepackage{url}
\usepackage{csquotes}
% \usepackage{graphicx}
\usepackage{wrapfig}
\usepackage{multirow}
\usepackage[binary-units=true]{siunitx}

\usepackage{tikz}
\usetikzlibrary{trees,decorations,arrows,automata,shadows,positioning,plotmarks,backgrounds,shapes}
\usetikzlibrary{calc,matrix,fit,petri,decorations.markings,decorations.pathmorphing,patterns,intersections,decorations.text}
\usepackage{pgfplots}
\usepackage{pgfplotstable}

\tikzstyle{mystate}=[state,inner sep=3pt,minimum size=20pt,line width=0.2mm]
\tikzstyle{fstate}=[state,accepting,inner sep=2pt,minimum size=3pt]
\tikzstyle{istate}=[state,initial,inner sep=2pt,minimum size=3pt]
\tikzstyle{mysquare}=[inner sep=3pt,minimum size=15pt,line width=0.2mm]
\tikzstyle{fmysquare}=[inner sep=3pt,minimum size=15pt,line width=0.5mm,accepting]
\newcommand{\SFSAutomatEdge}[5]{\path[->](#1) edge[#4,line width=0.2mm] node[#5] {\ensuremath{#2}} (#3);}
\usepackage{subcaption}
\usepackage{tabularx}
\usepackage{booktabs}
\usepackage{xfrac}

\usepackage{etoc}
\etocsettocdepth{3}

% \usepackage{minitoc}

% \usepackage{titletoc}
% 
% \newcommand\DoToC{%
%   \startcontents
%   \printcontents{}{2}{\textbf{Contents}\vskip3pt\hrule\vskip5pt}
%   \vskip3pt\hrule\vskip5pt
% }


\renewcommand{\headrulewidth}{0pt}
\def\footnoterule{\relax%
  \kern-0pt
  \hbox to \columnwidth{\hfill\vrule width \columnwidth height 0.5pt\hfill}
  \kern3pt}
\makeatother

\newcommand\blfootnote[1]{%
  \begingroup
  \renewcommand\thefootnote{}\footnote{#1}%
  \addtocounter{footnote}{-1}%
  \endgroup
}
\newcommand{\yunlong}[1]{\textcolor{blue}{\{\textbf{Yunlong:} #1\}}}
\newcommand{\XD}[1]{\textcolor{purple}{\{\textbf{XD:} #1\}}}
\newcommand{\sw}[1]{\textcolor{orange}{{}#1}}

\begin{document}

\newcommand{\db}{PySecDB}
\newcommand{\dbone}{base}
\newcommand{\dbtwo}{pilot}
\newcommand{\dbthree}{augmented}
%\newcommand{\gnn}{CPGCN}
\newcommand{\gnn}{SCOPY}%Secure COmmit PYthon
\newcommand{\cpg}{CommitCPG}
\newcommand{\emb}{CommitEmbedding}
%\newcommand{\TN}{ABCDEFG}

\title{Exploring Security Commits in Python}

% \title{Discovering Security Commits in Python via Learning Code Property Graphs}

\author{Shiyu Sun$^{*}$,
Shu Wang$^{*}$,
Xinda Wang$^{*}$,
Yunlong Xing$^{*}$,
Elisa Zhang$^{\dagger}$,
Kun Sun$^{*}$\\
$^{*}$George Mason University,
$^{\dagger}$Dougherty Valley High School\\
\{ssun20, swang47, xwang44, yxing4, ksun3\}@gmu.edu, elisaz.ca@gmail.com}

\maketitle
\thispagestyle{fancy}
\pagestyle{fancy}

\begin{abstract}
Graph Neural Networks (GNNs) have proven to be effective in processing and learning from graph-structured data.
However, previous works mainly focused on understanding single graph inputs while many real-world applications require pair-wise analysis for graph-structured data (e.g., scene graph matching, code searching, and drug-drug interaction prediction).
To this end, recent works have shifted their focus to learning the interaction between pairs of graphs.
Despite their improved performance, these works were still limited in that the interactions were considered at the node-level, resulting in high computational costs and suboptimal performance.
To address this issue, we propose a novel and efficient graph-level approach for extracting interaction representations using co-attention in graph pooling. 
Our method, Co-Attention Graph Pooling (CAGPool), exhibits competitive performance relative to existing methods in both classification and regression tasks using real-world datasets, while maintaining lower computational complexity.

\end{abstract}

\section{Introduction}
Deep learning models have been widely used in many applications.
For example, BERT~\citep{devlin_bert_2019}, GPT-3~\citep{brown_language_2020}, and T5~\citep{raffel_exploring_2020} achieved state-of-the-art~(SOTA) results on different natural language processing~(NLP) tasks. 
For computer vision~(CV), Transformer-like models such as ViT~\citep{dosovitskiy_image_2021} and Swin Transformer~\citep{liu_swin_2021} deliver excellent accuracy performance upon multiple tasks. 


At the same time, training deep learning models has been a critical problem troubling the community due to the long training time, especially for those large models with billions of parameters~\citep{brown_language_2020}. 
In order to enhance the training efficiency, researchers propose some manually designed parallel training strategies~\citep{narayanan_efficient_2021,shazeer_mesh-tensorflow_2018,xu_gspmd_2021}. 
However, selecting, tuning, and combining these strategies require extensive domain knowledge in deep learning models and hardware environments. With the increasing diversity of modern hardware architectures~\cite{flynn_very_1966,flynn_computer_1972} and the rapid development of deep learning models, these manually designed approaches are bringing heavier burdens to developers. 
Hence, \emph{automatic parallelism} is introduced to automate the parallel strategy searching for training models.


There are two main categories of parallelism in deep learning models: inter-layer parallelism~\citep{huang_gpipe_2019,narayanan_pipedream_2019,narayanan_memory-efficient_2021,fan_dapple_2021,li_chimera_2021,lepikhin_gshard_2021,du_glam_2022,fedus_switch_2022} and intra-layer parallelism~\citep{li_pytorch_2020,narayanan_efficient_2021,rasley_deepspeed_2020,fairscale_authors_fairscale_2021}. 
Inter-layer parallelism partitions the model into disjoint sets on different devices without slicing tensors. 
Alternatively, intra-layer parallelism partitions tensors in a layer along one or more axes and distributes them across different devices.


Current automatic parallelism techniques focus on optimizing strategies within these two categories. However, they treat these two categories separately. 
Some methods~\citep{zhao_vpipe_2022,jia_exploring_2018,cai_tensoropt_2022,wang_supporting_2019,jia_beyond_2019,schaarschmidt_automap_2021,liu_colossal-auto_2023} overlook potential opportunities for inter- or intra-layer parallelism, the others optimize inter- and intra-layer parallelism hierarchically and sequentially~\citep{narayanan_pipedream_2019,fan_dapple_2021,he_pipetransformer_2021,tarnawski_efficient_2020,tarnawski_piper_2021,zheng_alpa_2022}. 
As a result, current automatic parallelism techniques often fail to achieve the global optima and instead become trapped in local optima. 
Therefore, a unified inter- and intra-layer approach is needed to enhance the effectiveness of automatic parallelism.


This paper aims to find the optimal parallelism strategy while simultaneously considering inter- and intra-layer parallelism. 
It enables us to search in a more extensive strategy space where the globally optimal solution lurk. 
However, unifying inter- and intra-layer parallelism in automatic parallelism brings us two challenges. 
Firstly, to adopt a unified perspective on the inter- and intra-layer automatic parallelism, we should not formalize them with separate formulations as prior works. Therefore, how can we express these parallelism strategies in a unified formulation? 
Secondly, previous methods take a long time to obtain the solution with a limited strategy space. Therefore, how can we ensure that the best solution can be obtained in a reasonable time while expanding the strategy space?


To solve the above challenges, we propose UniAP. For the first challenge, UniAP adopts the mixed integer quadratic programming~(MIQP)~\citep{lazimy_mixed_1982} to search for the globally optimal parallel strategy automatically. 
It unifies the inter- and intra-layer automatic parallelism in a single MIQP formulation. 
For the second challenge, our complexity analysis and experimental results show that UniAP can obtain the globally optimal solution in a significantly shorter time.


The contributions of this paper are summarized as follows: 
\begin{itemize}
    \item We propose UniAP, the first framework to unify inter- and intra-layer automatic parallelism in model training.
    \item The optimal parallel strategies discovered by UniAP exhibit scalability on training throughput and strategy searching time.
    \item The experimental results show that UniAP speeds up model training on four Transformer-like models by up to 1.70$\times$ and reduces the strategy searching time by up to 16$\times$, compared with the SOTA method.
\end{itemize}

\section{Background}
\subsection{Parallel Strategy}
\label{subsec:background:parallel-strtegy}
\paragraph{Pipeline parallelism~(PP)} In PP, each worker~(machine or GPU) holds a subset of model layers. Adjacent layers on different workers need to transfer activations in the forward propagation~(FP) step and gradients in the backward propagation~(BP) step. 
\paragraph{Data parallelism~(DP)} In DP, each worker holds a replica of the whole model and partitions training samples. In each iteration, each worker computes gradients and synchronizes them with the other workers using all-reduce collective communication~(CC). All workers will have the same model parameters after the synchronization step.
\paragraph{Tensor parallelism~(TP)} In TP, each worker holds a replica of training samples and partitions within model layers. In each iteration, each worker computes its local outputs in FP and its local gradients in BP. To synchronize outputs and gradients, all workers will perform all-reduce CC in FP and BP steps according to the partition scheme.
\paragraph{Fully sharded data parallelism~(FSDP)} FSDP partitions optimizer states, parameters and gradients of the model into separate workers. During the FP and BP step of each iteration, FSDP performs an all-gather CC to obtain the complete parameters for the relevant layer, respectively. After computing the gradients, FSDP conducts a reduce-scatter CC to distribute the global gradients among the workers.

\subsection{Manual Parallelism}
MP refers to the parallel methods in which human experts design and optimize the parallel strategies. Representative MP methods include Megatron-LM~\citep{narayanan_efficient_2021}, Mesh-TensorFlow~\citep{shazeer_mesh-tensorflow_2018}, and GSPMD~\citep{xu_gspmd_2021}. Megatron-LM manually designs TP and PP strategies for training Transformer-based models and exhibits superior efficiency. Mesh-TensorFlow and GSPMD require human effort to designate and tune the intra-layer parallel strategy. These methods rely on expert design and have little flexibility, challenging their automatic application to other models.

\subsection{Automatic Parallelism}
\paragraph{Inter-layer-only AP or intra-layer-only AP} For inter-layer-only AP, GPipe~\citep{huang_gpipe_2019} and vPipe~\citep{zhao_vpipe_2022} employ a balanced partition algorithm and a dynamic layer partitioning middleware to partition pipelines, respectively. For intra-layer-only AP, OptCNN~\citep{jia_exploring_2018}, TensorOpt~\citep{cai_tensoropt_2022}, and Tofu~\citep{wang_supporting_2019} employ dynamic programming methods to optimize DP and TP strategies together. FlexFlow~\citep{jia_beyond_2019} and Automap~\citep{schaarschmidt_automap_2021} use the Monte Carlo method to find the optimal DP and TP strategy. Colossal-Auto~\citep{liu_colossal-auto_2023} utilizes integer programming techniques to generate intra-layer parallelism and activation checkpointing strategies without optimizing inter-layer parallelism. All these methods optimize only one category of parallel strategies.


\paragraph{Inter- and intra-layer AP} PipeDream~\citep{narayanan_pipedream_2019}, DAPPLE~\citep{fan_dapple_2021}, and PipeTransformer~\citep{he_pipetransformer_2021} use dynamic programming to determine optimal strategies for both DP and PP. DNN-partitioning~\citep{tarnawski_efficient_2020} adopts integer and dynamic programming to explore DP and PP strategies. Piper~\citep{tarnawski_piper_2021} and Alpa~\citep{zheng_alpa_2022} adopt a parallel method considering DP, TP, and PP.
Galvatron~\citep{miao_galvatron_2022} uses dynamic programming to determine DP, TP, and FSDP strategies in a single pipeline stage. As for PP, it partitions stages and determines micro-batch size using naive greedy algorithms. All these methods are hierarchical, which will result in sub-optimal solutions.


% \vspace{0.05in}
\section{Data Collection}
\label{sec:design}



%\subsection{Overview}

%\noindent \textbf{Overview.} 
Our dataset consists of three sections: \textit{a) base dataset, b) pilot dataset}, and \textit{c) augmented dataset}. Figure~\ref{fig:system} illustrates the composition and construction procedure of~\db{}. 
We first form the base dataset by collecting the commits associated with CVE records indexed by MITRE. Yet, less than 50\% of CVE records have published their security commits. %, which only provides partial semantics for further security commit mining.
To introduce more code semantics, we further build the pilot dataset by filtering the wild GitHub commits that have pre-defined security keywords in their commit messages. 
However, not all commits contain well-maintained commit messages that precisely describe the rationales of changed code.
Thus, we consider directly mining the most critical part of commits, i.e., the source code changes. 
% While well-maintained commit messages that precisely describe the rationales of changed code are critically lacking in practice, we consider directly mining the source code changes, the most necessary part of a commit.
To the end, we propose an intermediate commit representation (i.e., \cpg{}) and design a dependency-aggregation graph neural network (i.e., \gnn{}) to capture the inherent sequential and structural semantics of code changes.
%and expand the current dataset with an augmented dataset that includes the commit pinpointed by \gnn{}. 
Trained with the base and pilot datasets, \gnn{} is able to further build the augmented dataset by pinpointing the silent security commits from the wild.

% Figure environment removed

% In this section, we elaborate on the design of~\TN{}. 
% It collects an extensive security commits dataset from version control repositories for Python projects, represents code control/data dependency and syntax using Code Property Graphs, and learns the nodes embedding with CodeBERT~\cite{feng2020codebert}, the edge embedding with multiple attributes. 
% We then build a commit graph classifier expediently leveraging a graph-based deep neural network. 
% Figure~\ref{} illustrates the overall workflow of PySP with phase (1) for Commit Dataset Collection, phase (2) for Graph Generation for Commits and phase (3) for Learning from Commit Graphs.
\subsection{Base Dataset Collection}
We build the base dataset % from the CVE records indexed with MITRE~\cite{cve}.
according to the CVE records~\cite{cve}.
The first step is to retrieve the vulnerabilities that have already been indexed with CVE IDs. Then, we parse the vulnerability reports and crawl the corresponding commits via the provided reference hyperlinks. It comes to our attention that the collected commits may contain some noise, e.g., changelog, test case, refactoring, and renaming. %After excluding the changed file other than Python, we remove the commits that only contain changes of comments, refactoring, and renaming. In this way, we obtained 730 commits to form the base dataset.
After excluding these unrelated documents, we obtain 729 security commits to form the base dataset; meanwhile, we collect the excluded commits as the non-security subset in the base dataset and expand it by manually identifying the commits that add new features or perform refactoring, linting, and version updates.

\subsection{Pilot Dataset Collection}
\label{db:pilot}

%We could only find 46\% fixes of the indexed CVE records, which implies that there are still many vulnerabilities that have not been fixed or the fix commit is still in the wild. Also, the limited patches only provide partial semantics, which will impede the security commits understanding from capturing a wide variety of code change features.

After examining all the indexed CVE records (as of 01/27/2023), only 46\% of them contain the corresponding security fixes. Therefore, the limited samples in the base dataset may not provide adequate syntactic and semantic information.
That means, only with the base dataset, we are unable to train a robust model for capturing a wide variety of security commits in the real world.
% simply using the base dataset may be unable 
% with complete understanding of security commits for capturing a wide variety of code change features in the real world.
Given the fact that a majority number of security patches are silently committed without reporting to the MITRE~\cite{wang2019detecting,zhou2021spi}, we propose to enrich the security commits with the pilot dataset collected from GitHub, i.e., the most common OSS hosting platform. 

The pilot dataset is constructed by keyword filtering with humans in the loop. %We check the existence of security keywords in the wild commit's subject. 
%We look for security-related keywords in the subject of the GitHub commit. 
The list of security-related keywords is built automatically by analyzing the CVE descriptions, CWE types (if exist), and commit messages.
We determine the final keywords by calculating the word frequency and evaluating the correlation between the keywords and security commits. 
To obtain the security subset of the pilot dataset, we locate and manually verify the security commit candidates that contain the pre-defined security keywords in the commit messages; the excluded commits are collected as the non-security subset.
% After locating the commits with security-related keywords in their commit messages, we manually verify these security commit candidates to finalize the security subset of the pilot dataset; the excluded commits have been grouped to form the non-security part.

% Next, we cloned popular repositories for Python projects and retrieved all commit history from it till Jan 01, 2023, and filter out the candidate security commits by checking the existence of any keyword in the security-related keywords. 

\noindent{\bf Security Keyword Extraction. }%Modeling.} 
%Table~\ref{tab:keywords} includes the essential part of keywords list. 
For each security commit collected from the CVE records, we generate its security impact summary by combining the commit message, the CWE information (if exists), and the CVE report.
% we combine its commit message, its CWE information (if exists), and its description in its CVE report as the summary of the security impact. 
After generating the summary, we conduct 1-gram, 2-gram, and 3-gram tokenization. 
Then, we consider the frequency of each token and the correlation of each token with security and non-security commits.
% By considering the frequency of each token as well as the correlation of the tokens with security commits compared to non-security commits, 
We set the frequency threshold and derive the list of security-related keywords, as shown in Table~\ref{tab:keywords}. 
%formed the selection of the list of keywords. 
%\noindent{\bf Keyword Filtering.} We check the existence of security keywords in the wild commit's subject, and select the matched commits as candidates, which will be verified manually later.
Then, we determine the security commit candidates by checking if the wild GitHub commits contain any security keywords in their commit messages.
These candidates will be manually verified.


% Then, we check the existence of security keywords in the commit messages of wild GitHub commits and select ones containing at least one keyword as candidates, which will be manually verified later.

\begin{table}[t]
\begin{center}
\caption{Security-related keywords for commit filtering.}
\label{tab:keywords}
\resizebox{\linewidth}{!}{
\begin{tabular}{c|c}
\toprule
{\bf \#Tokens} & \textbf{Keywords} \\ 
\midrule
{1-gram} & {\begin{tabular}[c]{@{}c@{}}
attack, bypass, CVE, DoS, exploit, injection, \\ leakage, malicious, overflow, smuggling, \\ spoofing, unauthorized, underflow, vulnerability
% vulnerability, malicious, exploit, \\ attack, cve, unauthorized, bypass, injection
\end{tabular}} \\ 
\midrule
{2-gram} & {access control, open redirect, race condition} \\ 
\midrule
{3-gram} & {denial of service, out of bound,  dot dot slash} \\ 
\bottomrule
\end{tabular}
}
\end{center}
\vspace{-0.2in}
\end{table}



\noindent{\bf Manual Verification.} 
We hire three security experts to manually verify the security commit candidates. 
To guarantee the data quality and minimize false positives, the experts are required to follow our two-step labeling procedure strictly. 
First, each expert tags the commits independently with the labels: security, non-security, or unsure. 
Then, they gather together to discuss each disagreement and reach a consensus on each uncertain candidate. 
Only the security commits with 100\% agreement will be included in the pilot dataset. In total, it takes 48 man-hours to finish the labeling work.



The proposed keyword filtering mechanism reduces the workload and time for manual verification; meanwhile, the human-in-the-loop ensures the quality of the pilot dataset. More details on labeling efficiency are shown in Section~\ref{results:efficiency}.

\subsection{Augmented Dataset Construction}

The pilot dataset overlooks the commits that lack security keywords in the commit messages, while these commits may provide additional variants in syntax and semantics. 
Therefore, we propose to further augment our dataset. 
While the pilot dataset is collected based on commit messages, we build the augmented dataset by only analyzing the source code changes.
%while the state-of-the-art code representation models (e.g., PatchRNN~\cite{wang2021patchrnn}) neglect the inherent structure semantics. Thus, 
Different from existing works that simply regard source code as sequential data~\cite{wang2021patchrnn,zhou2021spi}, we present a commit graph representation named CommitCPG and a graph learning-based model \gnn{} to capture the inherent structural information.% \gnn{} is mainly composed of two parts: graph representation for commits and corresponding graph understanding.

%\subsubsection{Graph Representation for Commits}
\vspace{0.05in}
\noindent \textit{1) CommitCPG: Graph Representation for Commits}
\vspace{0.02in}

% Figure environment removed

To preserve the inherent structure of source code and the modified content between two versions, we propose a graph-based commit representation called \cpg{}, which offers essential syntactic and semantic information for comprehensive commit understanding. 
Code property graph (CPG)~\cite{y2014cpg} is a program representation that contains abstract syntax trees (AST), control-flow graphs (CFG), and program dependence graphs (PDG), providing a more comprehensive view for code static analysis, compared with traditional sequential structure adopted by NLP-based works~\cite{guo2020graphcodebert}.
% With various structural information (i.e., control flow/dependency, intra-procedural data dependency, and program syntax), CPG 
%With this goal in mind, 
In Figure \ref{fig:commitCPG}, we first preprocess the raw commits by excluding the irrelevant functions. Inspired by \cite{wang2022graphspd}, we then merge the CPGs \cite{y2014cpg} constructed from the previous and current versions by aligning the unchanged statements. Next, we adopt a code slicing method to retain the crucial context-related code snippets, which are not changed directly by commits but can assist us to understand the reason and the effects of  code changes.%and apply the intermediate representation to assist us to understand the security-related code change semantics.

\noindent{\bf Commit Preprocessing.} 
To generate the CPG for each code version, we need to retrieve the source code of the previous version and the current version, respectively.
% A CPG reveals the property of a program under a specific version, while a commit is a set of code changes between two versions. Thus, we need to retrieve the source code of the previous version and the current version so as to process them respectively. 
% \XD{Not Pre/Cur-Commit, should be version, commit refers to changes between two versions.} 
To reduce the overhead of CPG generation, we only focus on the modified files instead of the whole project. 
Then, we extract the functions with code revisions as well as the modified global statements. 
To achieve this goal, Joern parser \cite{y2014cpg} is applied to detect all relevant functions and their corresponding scopes. 
% (i.e., the range of line numbers between function start and end point). Among them, 
We only retain the functions whose scope overlaps with the modified lines in the commits.
% , i.e., functions modified by the commit. 
For example, in List~\ref{lst:security commit}, we will only keep the content of function \texttt{\small \_load\_yamlconfig()}.

\noindent{\bf CPG Generation for Previous and Current Versions.} 
With the extracted source code of two versions, we employ Joern~\cite{y2014cpg} to generate the CPGs for both versions. 
% Taking the source code of the previous and current versions as well as the modified function as input, we employ Joern~\cite{y2014cpg} to generate the CPGs for both versions. 
A CPG can be described as $(V, E)$, where $V$ is the node set and $E$ is the edge set.
$V$ is comprised of multiple 5-tuples $(id, func\_name, file\_name, version, code)$, which contain the information of each node.  
% For a CPG described as $(V, E)$, the set of nodes, $V$, is comprised of 5-tuples $(id, function\ name, file\ name, version, code)$, where $id$ is a unique identifier for the node, $function\ name$ and $file\ name$ records the function and file. 
The node version, represented by $version \in \{previous, current\}$, reflects if the code line belongs to the previous version or current version.
% and the $code$ represents a statement (usually a line of source code). 
The directed edge set $E$ is made up of 4-tuples $(id_1, id_2, type, version)$, where $id_1$ and $id_2$ denote the start and end node IDs, respectively.
The edge type, represented by $type \in \{AST, CDG, DDG\}$, specifies if the edge belongs to the AST or control/data dependency graphs. 
The edge $version$ is consistent with the node's version.

\noindent{\bf CPG Merging and CommitCPG Slicing.}
We first generate a unified commit graph by fusing the CPGs of two versions according to each node pair.
% Then, we conduct a bi-directional program slicing to generate the \cpg{}, which preserves the most related security semantics.
% We fuse the CPGs for each pair of previous and current versions to create a unified graph structure %, which we refer to as MergedCPG, 
% and further conduct two-direction program slicing %on MergedCPG 
% to generate \cpg{}, which preserves most related security semantics.
% Function names are used to match the functions in the previous version with their corresponding ones in the current version. 
Then, we update the representation of the merged CPG by two sets: $(V', E')$. 
The node set $V '$ is comprised of 5-tuples, which are denoted as $(id, func\_name, file\_name, version, code)$. 
$id$ is updated so that each node has a unique identifier and the node version is changed to $version \in \{current, previous, unchanged\}$, representing if the code in this node belongs to the current, previous, or both commits. 
% The $function\ name$, $file\ name$, and $code$ remain consistent with $V$. 
The directed edge set $E'$ is made up of 4-tuples $(id_1, id_2, type, version)$, where $id_1$, $id_2$, and $type$ stay the same as $E$. The edge $version$ will be updated as $unchanged$ if both connected nodes are unchanged nodes.

%\noindent{\bf Slicing the \cpg{}.} 
To reduce the noise introduced by irrelevant nodes and emphasize the semantics of code changes, we generate the \cpg{} by a bi-directional program slicing \cite{weiser1984program}, i.e., forward and backward slicing. 
Backward slicing is to reason the code changes, while forward slicing is to locate the statements affected by the commit. 
For example, in List~\ref{lst:security commit}, if we set the deleted statement (Line 10) as a backward slicing criterion, the slicing results include Lines 5, 7, and 8; if we set the added statement (Line 11) as a forward slicing criterion, the slicing results contain Line 14.
% that are influenced by the code revision. 
After we conduct code slicing over control/data dependency, we can obtain \cpg{} by only retaining all the nodes of modified and sliced statements (i.e., Line 7, 8, 10, 11, and 14) along with the traced edges.

%when we set the deleted statement (Line 6) as a backward slicing criterion, there are no previous statements affected by it, which reveals Line 6 is the vulnerability location; when we set the added statement (Line 7) as a forward slicing criterion, the slicing results contain Line 8 that has been influenced by the modification. Once we conduct backward and forward slicing using control and data dependency, we keep all the nodes of the modified and sliced statements, along with the traced edges, denoted as \cpg{}.

\vspace{0.05in}
\noindent \textit{2) \gnn{}: Graph Learning for Commits}
\vspace{0.02in}

% Figure environment removed

Figure~\ref{fig:model} illustrates the workflow of \gnn{}, our proposed graph-based network to identify security commits in Python. \gnn{} contains two steps: (i) node embedding with CodeBERT and edge embedding with dependency-aggregation mechanism, (ii) graph convolution with multi-head attention.
% , and (iii) \cpg{} classification. 

\noindent{\bf \cpg{} Embedding.} To feed the \cpg{} to our \gnn{}, we encode the node and edge attributes into numeric vectors. 
Each node represents a code statement; hence, the node embedding should capture the semantics within the statement.
Thus, we first utilize CodeBERT~\cite{feng2020codebert} to generate token embeddings and grasp the sequential-based semantics.
Then, we obtain the node embedding by aggregating the token embeddings.
% We formulate the embedding for each node in \cpg{} using , assisting our graph-based neural network to grasp the security-related semantics.
% Recall that each node represents a code statement and each edge represents the dependency between two nodes.
% Node embedding helps us capture the semantic information within the code statement, and 
In addition, each edge presents the dependency between two nodes; thus edge embedding preserves crucial structural and attribute information.
We generate edge embeddings with 5-dimensional one-hot vectors. 
The first two dimensions are used to embed the structural information and present which code version the edge belongs to.
The last three dimensions are used to embed the attribute information, which indicates if the edge presents control dependency, data dependency, or syntax relationship.


% To extract the dependencies between two nodes, which also reveal the inherent structure of code, we take both syntactic-level structures (i.e., AST) and semantic-level structures (i.e., CDG and DDG) into consideration.

% We merge the syntactic elements from the same statement as one node, thus, the edges in AST will also be merged into the statement level and the attribute from each edge will be concatenated. 

% The semantic-level structure of code encodes the dependency of “where-the-value-comes-from” between variables and the dependency of "what-the-next-statement-to-be-executed" among each line of code. 

% The semantic-level edge attributes have been appended to the syntactic-level edge attributes. 

% Therefore, edge embedding has four dimensions. The first dimension is used to indicate the version information, the second dimension demonstrates the AST attributes and the last two bits show if there are any control dependency and data dependency, respectively.




\noindent{\bf Graph Convolution with Multi-Head Attention.} 
After embedding the \cpg{} with a sequential model, we adopt a graph convolutional network with multi-head attention to learn the structural representation of commits. 
% To learn commit representation from code sequences and code structures, we adopt a graph convolution network with multi-head attention. 
To avoid over-smoothing, the number of convolutional layers is limited to 3. 
We feed the embedded \cpg{} into 3 multi-attributed graph convolutional layers. 
The node embeddings of \cpg{} are updated with the neighborhood information from different subgraphs. 
Then, the graph embeddings, i.e., a unified vector representation transformed from all the nodes, edges, and features, can be obtained through graph pooling and vector concatenation.
% Graph embedding is a unified representation that transforms all the nodes, edges, and features into a single vector. 
The graph embeddings learned by the \gnn{} are finally fed into a multi-layer perceptron to determine the likelihood that a commit fixes a security vulnerability, i.e., whether the given commit is a security commit.



To demonstrate the generalization ability of \gnn{}, we utilize the trained \gnn{} to discover more security commits in the wild. 
We directly send the \cpg{} of wild commits into \gnn{}. For the commits labeled as security commits, we adopt a similar process (as described in~\ref{db:pilot}) to manually verify if they are real security commits. The excluded commits composite the non-security subset of the augmented dataset.

%the commits that have been labeled as security commits. The labor process keeps consistent with the manual verification described in~\ref{db:pilot}.
\section{Implementation}
\label{sec:implementation}

% We demonstrate the implementation of constructing each dataset as follows. 

\subsection{Base Dataset Construction}

The base dataset consists of the commits linked with CVE records, which have been indexed by MITRE~\cite{cve}. MITRE provides hyperlinks of vulnerability fixes for 46\% of CVE entries. We focus on the hyperlinks from GitHub, where each commit is identified with a unique hash value and the hyperlink is in the form: \emph{https://github.com/\{owner\}/\{repo\}/commit/\{hash\}}. {We build the base dataset by downloading the vulnerability fixing commits and removing the commits that are not written in Python or only focus on security-unrelated modifications (e.g., renaming and refactoring).}

\subsection{Pilot Dataset Construction}
%We use topic modeling methods to extract essential security-related tokens from the commit messages. We adopt Latent Dirichlet Allocation (LDA)~\cite{blei2003latent} to collect the security-related tokens in the commit message. 
We adopt Latent Dirichlet Allocation (LDA)~\cite{blei2003latent}, a topic modeling method, to extract the essential security-related tokens from the commit messages.
Then, a keyword filtering algorithm is applied to exclude non-security commits. %Given a commit with the security key token set, we assign the security-related commit candidate to it if it has at least one keyword in the token set. 
We download popular open-source repositories in Python and retrieve their commit histories till Jan 27, 2023. For each commit, if it contains at least one proposed keyword, we regard it as a security commit candidate. Later, we manually check these security commit candidates and finalize the dataset.

To facilitate the verification process, we use {PyQt}~\cite{pyqt} to develop a graphical user interface (GUI) that %loads the commit candidates sequentially, records the verification results, and groups the security commits after verification.
visualizes the code changes of each individual commit and stores the verified security commits according to verification results.

\subsection{Augmented Dataset Construction}

% To build the augmented dataset, we implement each part of \gnn{}:  \cpg{} Generator, \cpg{} Learner, and Commit Classifier respectively

%\noindent{\bf \cpg{} Generator.} 
To build \gnn{} for further dataset augmentation, we first generate the corresponding \cpg{} for each commit in the base and pilot datasets. Specifically,
%While we cannot access the code property graph directly from the code change itself, we need to retrieve the source code of the previous commit and the current commit and merge them at the unchanged lines. Thus, 
we adopt {Joern}~\cite{y2014cpg} to generate CPGs for the code versions before and after applying the commit, respectively. Then, we parse the generated graph files and merge the graphs to build \cpg{}. To achieve the program slicing, we develop a Python script to analyze the control/data dependency and AST information and output a sliced CommitCPG ready to be embedded.

%\noindent{\bf \cpg{} Learner.} 
To prepare an embedded graph for \gnn{}, we embed the nodes and edges respectively. We fine-tune CodeBERT~\cite{feng2020codebert} to generate the node embedding dedicated to Python. For the edge embeddings, we apply the one-hot encoding to represent the attributes on each edge. We build the \gnn{} on the deep learning library PyTorch 1.6, which is optimized for tensor computing. We develop and optimize our graph model based on the PyTorch-geometric 1.6 library, which supports deep learning on graphs and other structured data.
%\noindent{\bf Commit Classifier.} 
Finally, a multiple-layer perceptron (MLP) is used as a binary predictor, which converts the graph embeddings into predicted labels. 
We train \cpg{} with the base and pilot datasets. 
Then, we feed the wild unlabeled commits into the trained \gnn{} and apply manual verification to generate our augmented dataset.

\section{Analysis Results}
\label{sec:experiment}
%To develop a more comprehensive understanding of Python security commits, we conduct a series of analysis: (1) Dataset Characteristics, (2) Security Commits Categories to reveal what types of vulnerabilities have been fixed, (3) Security Commits Distribution over Repositories, (4) Security Commits Complexity, (5) Security Commits Locality and (6) Fix Patterns Summarization to assist in automated program repair projects. 

After constructing our datasets, we frame our evaluation into four research questions, as outlined below. 
\begin{itemize}[leftmargin=*]

\item \textbf{RQ1:} Can the graph learning-based method help improve the data collection efficiency?
\item \textbf{RQ2:} How various and representative are the collected security commits? 
\item \textbf{RQ3:} What are the unique patterns of security commits in Python? 
\item \textbf{RQ4:} How do the wild commit samples help improve \gnn{} model for downstream security commit detection? 
\end{itemize}

\subsection{Dataset Construction (RQ1)}\label{results:efficiency}

After keyword filtering and graph-based identification with humans in the loop, we collect 1,258 security commits in total. Specifically, as shown in Table~\ref{tab: dataset}, there are 729, 400, and 129 security commits in the base, pilot, and augmented datasets, respectively. Also, 2,791 non-security commits are manually labeled during the collection process.

% \begin{table}[h]
% \centering
%     \caption{The statistical information of \db{}.}
%     \setlength{\tabcolsep}{3.4mm}{
%     \begin{tabular}{c|c|c|c|c}
%     \toprule
%     {} & \multirow{2}{*}{\shortstack{\bf Base\\\bf Dataset}} & \multirow{2}{*}{\shortstack{\bf Pilot\\\bf Dataset}} & \multirow{2}{*}{\shortstack{\bf Augmented\\\bf Dataset}} & \multirow{2}{*}{\bf Total} \\
%     {} & {} & {} & {} & {} \\
%      % & \textbf{Base} & \textbf{Pilot} & \textbf{Augmented} & \multirow{2}{*}{\textbf{Total}} \\
%      % & \textbf{Dataset} & \textbf{Dataset} & \textbf{Dataset} & \\
%      % & Base Dataset & Pilot Dataset & Augmented Dataset & Total \\
%                          % & \begin{tabular}[c]{@{}c@{}}Base \\ Dataset\end{tabular} & \begin{tabular}[c]{@{}c@{}}Pilot \\ Dataset\end{tabular} & \begin{tabular}[c]{@{}c@{}}Augmented \\ Dataset\end{tabular}  & Total \\ \hline
%     \midrule
%     \multirow{2}{*}{\shortstack{\bf Security\\\bf Commits}} & \multirow{2}{*}{729} & \multirow{2}{*}{400} & \multirow{2}{*}{129} & \multirow{2}{*}{1258} \\
%     {} & {} & {} & {} & {} \\
%     \midrule
%     \multirow{2}{*}{\shortstack{\bf Non-security\\\bf Commits}} & \multirow{2}{*}{2134} & \multirow{2}{*}{535} & \multirow{2}{*}{122} & \multirow{2}{*}{2791} \\
%     {} & {} & {} & {} & {} \\
%     \bottomrule
%     \end{tabular}
%     }
%     \label{tab: dataset}
% \end{table}

\begin{table}[h]
\vspace{-0.05in}
\centering
    \caption{The composition of \db{}.}
    \setlength{\tabcolsep}{1.7mm}{
    \begin{tabular}{c|p{1.0cm}<{\centering}|p{1.0cm}<{\centering}|p{1.4cm}<{\centering}|p{1.0cm}<{\centering}}
    \toprule
    {\diagbox{\bf Commit}{\bf Dataset}} & {\bf Base} & {\bf Pilot} & {\bf Augmented} & {\bf Total} \\
     % & \textbf{Base} & \textbf{Pilot} & \textbf{Augmented} & \multirow{2}{*}{\textbf{Total}} \\
     % & \textbf{Dataset} & \textbf{Dataset} & \textbf{Dataset} & \\
     % & Base Dataset & Pilot Dataset & Augmented Dataset & Total \\
                         % & \begin{tabular}[c]{@{}c@{}}Base \\ Dataset\end{tabular} & \begin{tabular}[c]{@{}c@{}}Pilot \\ Dataset\end{tabular} & \begin{tabular}[c]{@{}c@{}}Augmented \\ Dataset\end{tabular}  & Total \\ \hline
    \midrule
    {\bf Security} & {729} &  {400} &  {129} & {1258} \\
    \midrule
    {\bf Non-Security} & {2134} & {535} & {122} &{2791} \\
    \bottomrule
    \end{tabular}
    }
    \label{tab: dataset}
\vspace{-0.05in}
\end{table}

Table~\ref{tab:spr} lists the augmentation efficiency of random selection, keyword filtering, and \gnn{}.
Compared with identifying security commits from scratch, the keyword filtering mechanism improves the collecting efficiency by over 30 percentage points and \gnn{} improves the efficiency by 40 percentage points. %It is to be noted that we only test our data augmentation methods on a small portion of commits, which has already shown effectiveness.



\begin{table}[ht]
\vspace{-0.05in}
\centering
\caption{Efficiency of keyword filtering and \gnn{}.}
\setlength{\tabcolsep}{4mm}{
\begin{tabular}{c|c|c|c}
\toprule
% \multirow{2}{*}{\textbf{Methods}} & \multirow{2}{*}{\textbf{Candidates}} & \multirow{2}{*}{\shortstack{\bf Verified\\ \bf Security Commits}} & \multirow{2}{*}{\textbf{Ratio}} \\
% {} & {} & {} & {} \\
%  \midrule
\textbf{Method}      & \textbf{\# Candidates} & \textbf{\# Verified SC$^{*}$} & \textbf{Ratio} \\
 \midrule
% Methods      & Candidates & \begin{tabular}[c]{@{}c@{}}Verified \\ Security Commits\end{tabular} & Ratio   \\ \hline
{Random~\cite{wang2021patchdb}} & {-} & {-} & {6-10\%}    \\ 
\midrule
{Keywords} & {935} & {400} & {42.70\%} \\ 
\midrule
{\gnn{}} & {251} & {129} & {51.39\%} \\ 
\bottomrule

\end{tabular}
}
\begin{tablenotes}[flushleft]
    \footnotesize
    \item $^{*}$ SC = Security Commits.
\end{tablenotes}
% \vspace{-0.1in}
\label{tab:spr}
\vspace{-0.15in}
\end{table}

\begin{table}[]
\centering
\caption{Top 5 repositories by number of security commits.}
\label{tab:repo}
\setlength{\tabcolsep}{5.4mm}{
\begin{tabular}{c|c|c}
\toprule
\textbf{Repository} & \textbf{\#SecurityCommits} & \textbf{\textbf{Proportion}} \\
\midrule
django      & 166  & 13.20\%   \\ \midrule
twisted     & 87   & 6.91\%   \\ \midrule
glance      & 54   & 4.29\%     \\ \midrule
pillow     & 41   & 3.26\%     \\ \midrule
numpy       & 39   & 3.10\%        \\ \midrule
\rowcolor{gray!10}\textbf{Total of Top 5}                   & \textbf{387}   &   \textbf{30.76\%} \\
\bottomrule
\end{tabular}
}
\vspace{-0.1in}
\end{table}


\subsection{Security Commits Categorization and Distribution (RQ2)}
% \XD{should show broad coverage and variety of DB}}

%The Common Weakness Enumeration (CWE)
NVD CWE slice~\cite{CWE_slice} associated classification taxonomy serves to identify and describe security vulnerabilities.
% in terms of CWEs. 
To understand the purpose of these commits, we investigate the CWE types associated with the CVE reports and plot the distribution of the CWE types that have been explicitly documented. Among the 729 security commits linked to 556 CVEs, due to the limited number of MITRE human analysts, only 312 (56.1\%) CVEs have been assigned CWEs. %Since the CWE taxonomy is a hierarchical structure, a CVE can be assigned with more than one CWE. 
Even so, there are already 119 distinct CWEs associated with our security commits in the base dataset, which means our \db{} contains at least 119 types of security commits in terms of corresponding vulnerabilities.
Figure~\ref{fig:cwe} enumerates the most common CWEs, including frequent security problems such as cross-site scripting (CWE-79), path traversal (CWE-22), etc. Note that we do not directly assign CWE type to security samples in the remaining base, pilot, and augmented dataset since the MITRE CWE team has its own internal process. However, based on our observation and our data collection approaches that are able to introduce wild security commits with more variance (as discussed in~\ref{exp:variance}), \db{} can encompass a broad range of security concerns with various kinds of security commits, including but not limited to above 119 CWEs. 

% % Figure environment removed

% Figure environment removed


Our collected security commits distribute among 351 popular GitHub repositories unevenly. Among them, 69 repositories provide more than two security commits, bringing a certain amount of variety. In Table~\ref{tab:repo}, the top five repositories that have the most occurrence in our dataset are django~\cite{django_2023}, twisted~\cite{twisted_2023}, glance~\cite{openstack_2023}, pillow~\cite{python-pillow_2023}, and numpy~\cite{numpy_2023}, implying that the samples in \db{} align with the popularity trend of security issue in practice.
%react to security issues on time.




% \subsection{Security Commits Complexity}

% \subsection{Security Commits Locality}

\subsection{Patch Patterns (RQ3)}
\label{rq3}

We manually go through the whole \db{} dataset, 
% we manually explore the full security commit dataset. 
% samples that have less than 200 code lines.
% We find 1,027 samples in total, which take up 81.7\% of the security commits.
% After the comprehensive analysis, 
and discover four common security fix patterns (taking up 85.85\% of all security commit samples) that may benefit software maintenance, i.e., adding or updating sanity checks,  updating APIs, updating regular expressions, and updating security properties, as listed in Table~\ref{tab:pattern}.


\begin{table}[]
\centering
%\scriptsize
\caption{The pattern types of security commits in \db{}.}
\label{tab:pattern}
\setlength{\tabcolsep}{4.0mm}{
\begin{tabular}{l|c|c}
\toprule
    \textbf{Pattern} & \textbf{\#Commits} & \textbf{Proportion} \\ 
    \midrule
    {1) Add or Update Sanity Checks} & {416} & {37.12\%} \\ 
    \midrule
    {2) Update API Usage} & {241} & {19.16\%} \\ 
    \midrule
    {3) Update Regular Expressions} & {189} & {15.02\%} \\ 
    \midrule
    {4) Restrict Security Properties} & {183} & {14.55\%} \\ 
    \midrule
    {5) Others} & {178} & {14.15\%} \\ 
    \midrule 
    \rowcolor{gray!10}{\bf Total} & {\bf 1258} & {\bf 100\%} \\
    \bottomrule
\end{tabular}
}
\vspace{-0.1in}
\end{table}


%re 168 + 21
%api 214 +27
%if 420 + 47
%property 167 +16
%others 159 + 19

%re 151
%api 190
%if 369
%property 146
%others 43


\subsubsection{Add or Update Sanity Checks}
A sanity check is a basic method to quickly evaluate if a claim or a calculation result can be true, which has been extensively applied to multiple scenarios, e.g., authentication property verification, access control, HTTP request checking~\cite{wang2020machine}. 
We summarize three representative patterns that fix the vulnerabilities via adding or updating sanity checks, which are presented by 37.12\% of security commits in \db{}.

\noindent{\bf Authentication.} Authentication is the act of proving an assertion, e.g., we need to compare the identity with the system data to verify a system user. The authentication-related vulnerabilities
% occurs when the authentication is performed improperly, which 
provide attackers the opportunities to masquerade as legitimate users. To defend them, an effective solution is to perform the additional authentication by adding more check requirements or making existing conditions more restrictive. List~\ref{lst:auth} presents an example of fixing an authentication vulnerability by narrowing down an existing restriction from \texttt{\small True} (i.e., all possible return values except \texttt{\small False}) to \texttt{\small "on"} only.


\lstdefinestyle{lst}{
    float=th,
    floatplacement=tbp,
    % abovecaptionskip=0.01in,
    numbers=left, 
    numberstyle=\scriptsize, 
    numbersep = 5pt,
    framexleftmargin = 0in,
    framexrightmargin = 0in,
    breaklines = true,
    xleftmargin = 0.18in,
    xrightmargin = 0.1in,
    basicstyle=\ttfamily\scriptsize, 
    frame=lines,
    showtabs=true,
    showspaces=true,
    showstringspaces=false,
    literate={\ }{{\ }}1,
    aboveskip=-0.00in,
    belowskip=-0.15in,
}

\begin{lstlisting}[
language=diff, 
style=lst,
caption=An example of security commit to fix authentication vulnerability (CVE-2022-0273).,
label={lst:auth},
mathescape=true
]
 $\textbf{commit 0c0313f375bed7b035c8c0482bbb09599e16bfcf}$ 
 diff --git a/cps/shelf.py b/cps/shelf.py
 @@ -248,7 +248,7 @@ def create_edit_shelf(shelf,
 ...
         $\textbf{return}$ redirect(url_for('web.index'))
-    is_public = 1 if to_save.get("is_public") else 0
+    is_public = 1 if to_save.get("is_public") == "on" else 0
     $\textbf{if}$ config.config_kobo_sync:
 ...
\end{lstlisting}

\noindent{\bf Authorization.} Authorization refers to the process of granting or denying access to certain data or actions within a system.
Authorization comes after authentication and is achieved by an access control list (ACL).
The ACL is used to check the user identity with a list of authorized operations and determine which actions a user is allowed to take, e.g., file and data permission.
% determining  to perform and which are restricted, including but not limited to file permission and data permission. 
Unrestricted authorization may lead to improper resource consumption since attackers could bypass the system to access high-security level data. List~\ref{lst:access control1} is an example that fixes an authorization bypass exploit by requiring the value of \texttt{\small os.environ.get('GITHUB\_ACTIONS')} to be \texttt{\small true}.

\lstdefinestyle{lst}{
    float=th,
    floatplacement=tbp,
    % abovecaptionskip=0.01in,
    numbers=left, 
    numberstyle=\scriptsize, 
    numbersep = 5pt,
    framexleftmargin = 0in,
    framexrightmargin = 0in,
    breaklines = true,
    xleftmargin = 0.18in,
    xrightmargin = 0.1in,
    basicstyle=\ttfamily\scriptsize, 
    frame=lines,
    showtabs=true,
    showspaces=true,
    showstringspaces=false,
    literate={\ }{{\ }}1,
    aboveskip=+0.10in,
    belowskip=-0.30in,
}

\begin{lstlisting}[
language=diff, 
style=lst,
caption=An example of security commit that fixes an authorization bypass exploit vulnerability (CVE-2022-46179).,
label={lst:access control1},
mathescape=true
]
 $\textbf{commit c658b4f3e57258acf5f6207a90c2f2169698ae22}$  
 diff --git a/core.py b/core.py
 @@ -112,7 +112,7 @@ def actualsys() :
     $\textbf{if}$ attemps == 6:
         ## Brute force protection
         $\textbf{raise}$ Exception("Too many password attempts.")
-    if os.environ.get('GITHUB_ACTIONS') != "":
+    if os.environ.get('GITHUB_ACTIONS') == "true":
         logging.warning("Running on Github Actions")
         actualsys()
     $\textbf{elif}$ uname == cred.name and pwdhash == cred.pass:
\end{lstlisting}


\noindent{\bf HTTP Request.} If the interpretation of Content-Length and/or Transfer-Encoding headers between HTTP servers are inconsistent, the attackers may take advantage of this issue and send malicious requests to the servers, i.e., HTTP request smuggling. 
A good solution is to maintain the same interpretation methods in both front-end and back-end servers. 
In this way, an effective coding practice is to add consistent sanity checks on request interpretation for both servers. 
List~\ref{lst:http} adds such a sanity check on \texttt{\small data} to determine if all characters are digits.% to avoid such exploits.

\lstdefinestyle{lst}{
    float=th,
    floatplacement=tbp,
    % abovecaptionskip=0.01in,
    numbers=left, 
    numberstyle=\scriptsize, 
    numbersep = 5pt,
    framexleftmargin = 0in,
    framexrightmargin = 0in,
    breaklines = true,
    xleftmargin = 0.18in,
    xrightmargin = 0.1in,
    basicstyle=\ttfamily\scriptsize, 
    frame=lines,
    showtabs=true,
    showspaces=true,
    showstringspaces=false,
    literate={\ }{{\ }}1,
    aboveskip=-0.00in,
    belowskip=-0.15in,
}

\begin{lstlisting}[
language=diff, 
style=lst,
caption=An example of security commit that fixes an HTTP request smuggling vulnerability (CVE-2022-24801).,
label={lst:http},
mathescape=true
]
 $\textbf{commit 8ebfa8f6577431226e109ff98ba48f5152a2c416}$ 
 diff --git a/src/twisted/web/http.py b/src/twisted/web/http.py
 @@ -2274,6 +2274,8 @@ def fail():
     $\textbf{if}$ header == b"content-length":
+        if not data.isdigit():
+            return fail()
         $\textbf{try}$:
             length = int(data)
         $\textbf{except}$ ValueError:
\end{lstlisting}


\subsubsection{Update API Usage}% Packages}
Compared with implementing the fixes from scratch, there are abundant well-formulated packages that can be adopted to realize the intended functionalities and help enforce security restrictions. 
We notice that a large number (19.16\%) of Python security commits fix vulnerabilities by imposing or substituting APIs. %packages, which is different from other languages like C/C++.
%This pattern differs from the fix patterns in other languages, e.g., C/C++.
We further categorize such security fixes % into different types 
according to their application scenarios. %scopes.

\noindent{\bf General Purpose.} There is a set of security-related modifications on built-in packages shared by applications for various purposes. For instance, \texttt{\small re.escape} is an API to escape non-alphanumerics that are not part of regular expression syntax, to avoid OS command injection, code injection, and regular expression injection. List~\ref{lst:re} is a commit example to fix regular expression injection vulnerability, which demonstrates the application of \texttt{\small re.escape} on \texttt{\small user} and \texttt{\small collection\_url}.

\lstdefinestyle{lst}{
    float=th,
    floatplacement=tbp,
    % abovecaptionskip=0.01in,
    numbers=left, 
    numberstyle=\scriptsize, 
    numbersep = 5pt,
    framexleftmargin = 0in,
    framexrightmargin = 0in,
    breaklines = true,
    xleftmargin = 0.18in,
    xrightmargin = 0.1in,
    basicstyle=\ttfamily\scriptsize, 
    frame=lines,
    showtabs=true,
    showspaces=true,
    showstringspaces=false,
    literate={\ }{{\ }}1,
    aboveskip=-0.00in,
    belowskip=-0.15in,
}

\begin{lstlisting}[
language=diff, 
style=lst,
caption=An example of security commit that fixes a regular expression injection vulnerability (CVE-2015-8748).,
label={lst:re},
mathescape=true
]
 $\textbf{commit 4bfe7c9f7991d534c8b9fbe153af9d341f925f98}$ 
 diff --git a/radicale/rights/regex.py b/radicale/rights/regex.py
 @@ -65,7 +65,10 @@ def _read_from_sections(user, collection_url, permission):
 ...
-    regex = ConfigParser({"login": user, "path": collection_url})
+    # Prevent "regex injection"
+    user_escaped = re.escape(user)
+    collection_url_escaped = re.escape(collection_url)
+    regex = ConfigParser({"login": user_escaped, "path": collection_url_escaped})
 ...
\end{lstlisting}

\noindent{\bf Web Applications.} To properly process the inputs of web applications, security commits can adopt %the existing APIs %, e.g., \texttt{\small escape\_html}, 
APIs in third-party packages for Python
(e.g., \texttt{\small parser.quote}, \texttt{\small request.server.escape}, \texttt{\small django.utils.html.escape}, and \texttt{\small html.unescape}) to escape ampersands, brackets, and quotes to the HTML/XML entities or HTTP requests for defeating cross-site scripting (XSS) and HTTP Smuggling. 
List~\ref{lst:xss2} is an example that fixes an XSS vulnerability by using the API \texttt{\small django.utils.html.escape}.

\lstdefinestyle{lst}{
    float=th,
    floatplacement=tbp,
    % abovecaptionskip=0.01in,
    numbers=left, 
    numberstyle=\scriptsize, 
    numbersep = 5pt,
    framexleftmargin = 0in,
    framexrightmargin = 0in,
    breaklines = true,
    xleftmargin = 0.18in,
    xrightmargin = 0.1in,
    basicstyle=\ttfamily\scriptsize, 
    frame=lines,
    showtabs=true,
    showspaces=true,
    showstringspaces=false,
    literate={\ }{{\ }}1,
    aboveskip=+0.00in,
    belowskip=-0.15in,
}

\begin{lstlisting}[
language=diff, 
style=lst,
caption=An example of security commit that fixes an XSS vulnerability (CVE-2022-24710).,
label={lst:xss2},
mathescape=true
]
 $\textbf{commit f6753a1a1c63fade6ad418fbda827c6750ab0bda }$
 diff --git a/weblate/trans/forms.py b/weblate/trans/forms.py
 @@ -37,6 +37,7 @@
 ...
+from django.utils.html import escape
 ...
-    label = str(unit.translation.language)
+    label = escape(unit.translation.language)
 ...
\end{lstlisting}


\noindent{\bf Shell Commands.} To handle the shell commands securely, security fixes can adopt \texttt{\small shlex.quote} and \texttt{\small subprocess} to load or execute the commands. 
With the \texttt{\small shlex.quote} API, we can have an escaped version of shell inputs, which can be safely used as tokens in a command line to avoid shell command injection.
List~\ref{lst:shell} is an example that shows the usage of \texttt{\small shlex.quote} to fix a shell injection vulnerability. 

\lstdefinestyle{lst}{
    float=th,
    floatplacement=tbp,
    % abovecaptionskip=0.01in,
    numbers=left, 
    numberstyle=\scriptsize, 
    numbersep = 5pt,
    framexleftmargin = 0in,
    framexrightmargin = 0in,
    breaklines = true,
    xleftmargin = 0.18in,
    xrightmargin = 0.1in,
    basicstyle=\ttfamily\scriptsize, 
    frame=lines,
    showtabs=true,
    showspaces=true,
    showstringspaces=false,
    literate={\ }{{\ }}1,
    aboveskip=-0.00in,
    belowskip=-0.15in,
}

\begin{lstlisting}[
language=diff, 
style=lst,
caption=An example of security commit that fixes a shell injection vulnerability (CVE-2013-7416).,
label={lst:shell},
mathescape=true
]
 $\textbf{commit 2817869f98c54975f31e2dd674c1aefa70749cca }$
 diff --git a/canto_curses/guibase.py b/canto_curses/guibase.py
 @@ -156,6 +156,11 @@ def _fork(self, path, href, text, fetch=False):
 ...
+    href = shlex.quote(href)
 ...
\end{lstlisting}


\noindent{\bf Path Name.} 
If a path name is improperly neutralized, attackers may access the files and directories outside of the restricted location. 
This vulnerability can occur by using absolute file paths or manipulating the path variables where the reference files contain ``dot-dot-slash (../)" sequences or variations.
To effectively escape such unsafe sequences, Python security commits usually adopt the secure APIs, e.g., \texttt{\small werkzeug.utils.safe\_join}, \texttt{\small yaml.safe\_load}, and \texttt{\small werkzeug.utils.secure\_filename}, to prevent the files or directories from being accessed by malicious users. 
List~\ref{lst:path traversal} is a commit example that fixes a path traversal via using the API \texttt{\small werkzeug.utils.secure\_filename}.

\lstdefinestyle{lst}{
    float=th,
    floatplacement=tbp,
    % abovecaptionskip=0.01in,
    numbers=left, 
    numberstyle=\scriptsize, 
    numbersep = 5pt,
    framexleftmargin = 0in,
    framexrightmargin = 0in,
    breaklines = true,
    xleftmargin = 0.18in,
    xrightmargin = 0.1in,
    basicstyle=\ttfamily\scriptsize, 
    frame=lines,
    showtabs=true,
    showspaces=true,
    showstringspaces=false,
    literate={\ }{{\ }}1,
    aboveskip=+0.10in,
    belowskip=-0.25in,
}

\begin{lstlisting}[
language=diff, 
style=lst,
caption=An example of security commit that fixes a path traversal vulnerability (CVE-2022-23609).,
label={lst:path traversal},
mathescape=true
]
 $\textbf{commit 1eb1e5428f0926b2829a0bbbb65b0d946e608593}$ 
 diff --git a/upload/server.py b/upload/server.py
 @@ -5,7 +5,7 @@
-
+import werkzeug.utils
 @@ -189,7 +189,7 @@ def uploadimage():
     filename = all_files[0][1] + all_files[0][2]
-    remove(filename)
+    remove(werkzeug.utils.secure_filename(filename))
     $\textbf{del}$ all_files[0]
     length = len(all_files)
\end{lstlisting}


\subsubsection{Update Regular Expressions}
Python has become a popular choice for back-end web development, and it is usually combined with some other front-end languages~\cite{python_app}. For this reason, we observe there are 15.02\% fixes that modify the regular expressions to avoid XSS, SQL injection, and open redirect vulnerabilities. 
% Python inserts itself in web development as a back-end language, and it is usually combined with some other front-end language (e.g., javascript) to build a whole website.
% We observe 15.02\% 
The regular expression patterns are tailored to match specific strings within the given text, including SQL commands, URLs, and other scripts.

\noindent{\bf SQL Commands.} The improper neutralization of SQL commands may lead to SQL injection vulnerabilities, which allow attackers to manipulate the backend database and access the information not intended to be displayed.
The corresponding fixes need to escape the unsafe characters. 
List~\ref{lst:sql} is a fixed example of SQL injection vulnerability, which substitutes the matched single and double quote characters (i.e., \texttt{\small '} and \texttt{\small "}) in the string \texttt{\small self.queueid}.

\lstdefinestyle{lst}{
    float=th,
    floatplacement=tbp,
    % abovecaptionskip=0.01in,
    numbers=left, 
    numberstyle=\scriptsize, 
    numbersep = 5pt,
    framexleftmargin = 0in,
    framexrightmargin = 0in,
    breaklines = true,
    xleftmargin = 0.18in,
    xrightmargin = 0.1in,
    basicstyle=\ttfamily\scriptsize, 
    frame=lines,
    showtabs=true,
    showspaces=true,
    showstringspaces=false,
    literate={\ }{{\ }}1,
    aboveskip=+0.0in,
    belowskip=-0.15in,
}

\begin{lstlisting}[
language=diff, 
style=lst,
caption=An example of security commit that fixes a SQL injection vulnerability (CVE-2014-125082).,
label={lst:sql},
mathescape=true
]
 $\textbf{commit fc2c1ea1b8d795094abb15ac73cab90830534e04}$
 diff --git a/.../model.py b/.../model.py
 @@ -772,13 +772,13 @@ def _get_filter(self):
 $\textbf{if}$ self.queueid:
-    ... = '%s'" % (self.queueid)
+    ... = '%s'" % (re.sub("[\"']", "", self.queueid))
\end{lstlisting}


\noindent{\bf URLs.} The improper neutralization of URLs may lead to open redirect vulnerability, which redirects an unsuspecting victim from a legitimate domain to an attacker’s phishing site. 
Effective mitigation is to replace the dangerous special characters with trusted symbols. List~\ref{lst:redirect} is an example of an open redirect vulnerability, which replaces the explicit backslash with an encoded backslash to circumvent the dangerous redirect.

\lstdefinestyle{lst}{
    float=th,
    floatplacement=tbp,
    % abovecaptionskip=0.01in,
    numbers=left, 
    numberstyle=\scriptsize, 
    numbersep = 5pt,
    framexleftmargin = 0in,
    framexrightmargin = 0in,
    breaklines = true,
    xleftmargin = 0.18in,
    xrightmargin = 0.1in,
    basicstyle=\ttfamily\scriptsize, 
    frame=lines,
    showtabs=true,
    showspaces=true,
    showstringspaces=false,
    literate={\ }{{\ }}1,
    aboveskip=-0.00in,
    belowskip=-0.15in,
}

\begin{lstlisting}[
language=diff, 
style=lst,
caption=An example of security commit that fixes an open redirect vulnerability (CVE-2019-10255).,
label={lst:redirect},
mathescape=true
]
 $\textbf{commit 08c4c898182edbe97aadef1815cce50448f975cb}$ 
 diff --git a/auth/login.py b/auth/login.py
 @@ -39,6 +39,10 @@ def _redirect_safe(self, url, ...):
+    url = url.replace("\\", "%5C")
     parsed = urlparse(url)
     $\textbf{if}$ parsed.netloc $\textbf{or not}$ (parsed.path + '/').startswith(self.base_url):
\end{lstlisting}

\noindent{\bf Scripts.} The improper input validation and encoding during web page generation may lead to XSS, which is able to reveal the cookies, session tokens, or other sensitive information retained by the browser to the attackers. A straightforward solution is to validate the matched characters of a pre-defined pattern. List~\ref{lst:xss} is an example to fix the XSS vulnerability by re-matching the characters between parentheses instead of the characters between square brackets and validating the matched pattern one by one.

\lstdefinestyle{lst}{
    float=th,
    floatplacement=tbp,
    % abovecaptionskip=0.01in,
    numbers=left, 
    numberstyle=\scriptsize, 
    numbersep = 5pt,
    framexleftmargin = 0in,
    framexrightmargin = 0in,
    breaklines = true,
    xleftmargin = 0.18in,
    xrightmargin = 0.1in,
    basicstyle=\ttfamily\scriptsize, 
    frame=lines,
    showtabs=true,
    showspaces=true,
    showstringspaces=false,
    literate={\ }{{\ }}1,
    aboveskip=+0.10in,
    belowskip=-0.25in,
}

\begin{lstlisting}[
language=diff, 
style=lst,
caption=An example of security commit that fixes an XSS vulnerability (CVE-2021-3994).,
label={lst:xss},
mathescape=true
]
 $\textbf{commit a22eb0673fe0b7784f99c6b5fd343b64a6700f06}$ 
 diff --git a/helpdesk/models.py b/helpdesk/models.py
 @@ -238 +238 @@ def cvesForCPE(cpe,
     $\textbf{if not}$ text:
         $\textbf{return}$ ""
-    pattern = fr'([\[\s\S\]]*?)\(([\s\S]*?):([\[\s\S\]]*?)\)'
+    pattern = fr'([\[\s\S\]]*?)\(([\s\S]*?):([\s\S]*?)\)'
     # Regex check
     $\textbf{if}$ re.match(pattern, text):
         # get get value of group regex
\end{lstlisting}



\subsubsection{Restrict Security Properties} 
The exploits often result from improper settings of security properties. 
14.55\% security commits in \db{} fix improper settings by updating boolean flags from \texttt{\small True} to \texttt{\small False} or vice versa, adding more arguments to methods, or adding security decorators.


\noindent{\bf Update Security Flags.} 
Security flags perform restrictions on the methods that may have access to sensitive objects. 
Improper restrictions on such flags may expose users to a risky environment and/or lead to sensitive information leakage. 
List~\ref{lst:flag} changes the flag from \texttt{\small False} to \texttt{\small True} to fix a vulnerability, where a sensitive cookie does not have a `HttpOnly' flag.


\lstdefinestyle{lst}{
    float=th,
    floatplacement=tbp,
    % abovecaptionskip=0.01in,
    numbers=left, 
    numberstyle=\scriptsize, 
    numbersep = 5pt,
    framexleftmargin = 0in,
    framexrightmargin = 0in,
    breaklines = true,
    xleftmargin = 0.18in,
    xrightmargin = 0.1in,
    basicstyle=\ttfamily\scriptsize, 
    frame=lines,
    showtabs=true,
    showspaces=true,
    showstringspaces=false,
    literate={\ }{{\ }}1,
    aboveskip=-0.00in,
    belowskip=-0.15in,
}

\begin{lstlisting}[
language=diff, 
style=lst,
caption=An example of security commit that fixes a vulnerability where the sensitive cookie does not have a `HttpOnly' flag (CVE-2019-25091).,
label={lst:flag},
mathescape=true
]
 $\textbf{commit 60a3fe559c453bc36b0ec3e5dd39c1303640a59a}$ 
 diff --git a/src/nsupdate/settings/base.py b/src/nsupdate/settings/base.py
 @@ -283,7 +283,7 @@
 ...
-CSRF_COOKIE_HTTPONLY = False
+CSRF_COOKIE_HTTPONLY = True
 ...
\end{lstlisting}

\noindent{\bf Add Restriction Arguments.} Some restriction arguments will be passed to the functions during execution. Improper argument settings may lead to a variety of mishandling. As shown in List~\ref{lst:arg}, the \texttt{\small formaction} is added to restrict the attributes of a variable to avoid XSS vulnerability.


\lstdefinestyle{lst}{
    float=th,
    floatplacement=tbp,
    % abovecaptionskip=0.01in,
    numbers=left, 
    numberstyle=\scriptsize, 
    numbersep = 5pt,
    framexleftmargin = 0in,
    framexrightmargin = 0in,
    breaklines = true,
    xleftmargin = 0.18in,
    xrightmargin = 0.1in,
    basicstyle=\ttfamily\scriptsize, 
    frame=lines,
    showtabs=true,
    showspaces=true,
    showstringspaces=false,
    literate={\ }{{\ }}1,
    aboveskip=-0.00in,
    belowskip=-0.15in,
}

\begin{lstlisting}[
language=diff, 
style=lst,
caption=An example of security commit that fixes a cross-site-scripting (XSS) vulnerability (CVE-2021-28957).,
label={lst:arg},
mathescape=true
]
 $\textbf{commit 10ec1b4e9f93713513a3264ed6158af22492f270}$ 
 diff --git a/src/lxml/html/defs.py b/src/lxml/html/defs.py
 @@ -23,6 +23,8 @@
 ...
+    # HTML5 formaction
+    'formaction'
     ])
 ...
\end{lstlisting}

\noindent{\bf Add Security Decorators.} A decorator is a function that takes another function and extends the behavior of the function without explicit modification. This mechanism has been widely adopted by security commits to add more detailed security restrictions on existing methods. List~\ref{lst:access control2} shows a security commit that fixes an access control vulnerability by adding decorator \texttt{\small security.private} to function \texttt{\small enumerateRoles}.

\lstdefinestyle{lst}{
    float=th,
    floatplacement=tbp,
    % abovecaptionskip=0.01in,
    numbers=left, 
    numberstyle=\scriptsize, 
    numbersep = 5pt,
    framexleftmargin = 0in,
    framexrightmargin = 0in,
    breaklines = true,
    xleftmargin = 0.18in,
    xrightmargin = 0.1in,
    basicstyle=\ttfamily\scriptsize, 
    frame=lines,
    showtabs=true,
    showspaces=true,
    showstringspaces=false,
    literate={\ }{{\ }}1,
    aboveskip=+0.10in,
    belowskip=-0.25in,
}

\begin{lstlisting}[
language=diff, 
style=lst,
caption=An example of security commit that fixes an access control vulnerability (CVE-2021-21336).,
label={lst:access control2},
mathescape=true
]
 $\textbf{commit 2dad81128250cb2e5d950cddc9d3c0314a80b4bb}$ 
 diff --git a/src/Products/plugins/ZODBRoleManager.py b/src/Products/plugins/ZODBRoleManager.py
 @@ -112,6 +112,7 @@ def getRolesForPrincipal(self, principal, request=None):
     #   IRoleEnumerationPlugin implementation
+    @security.private
     $\textbf{def}$ enumerateRoles(self, id=None, exact_match=False, sort_by=None, max_results=None, **kw):
         """ See IRoleEnumerationPlugin.
\end{lstlisting}



\subsection{Unique Patterns Captured from the Wild (RQ4)}\label{exp:variance}

Recall that we construct pilot and augmented datasets because the base dataset provides a limited number of security commits samples. Here, we further show the examples captured by our security commit collection approaches that introduce more variety in syntax and semantics of security-related code changes, enabling wider applications of \db{} in solving real-world Python-related security issues.

\subsubsection{Data Variety Introduced by Pilot Dataset}
We study the contribution of involving the pilot dataset for \gnn{} by comparing the model trained only on the base dataset and the model trained on the combination of the base and pilot datasets.
% first training on the base dataset and then training on the combination of the base dataset and the pilot dataset.
%Then, we analyze the samples that have not been identified by the first model but have been identified by the second model. 
We find that the pilot dataset helps the latter model to be able to identify more wild security commits. For instance, the latter \gnn{} can detect more subtle changes. % after expanding the training set with the pilot dataset. 
In List~\ref{lst:pilot}, the \texttt{\small '\%s'} has been changed to \texttt{\small ?} in a SQL query, protecting the database from being injected. 
% the condition refines the value of \texttt{\small GITHUB\_ACTIONS} from not null to true, protecting the authentication from being bypassed. 
The capability of detecting such minor changes is enabled by similar samples in the pilot dataset but not existed in the base dataset.

\lstdefinestyle{lst}{
    float=th,
    floatplacement=tbp,
    % abovecaptionskip=0.01in,
    numbers=left, 
    numberstyle=\scriptsize, 
    numbersep = 5pt,
    framexleftmargin = 0in,
    framexrightmargin = 0in,
    breaklines = true,
    xleftmargin = 0.18in,
    xrightmargin = 0.1in,
    basicstyle=\ttfamily\scriptsize, 
    frame=lines,
    showtabs=true,
    showspaces=true,
    showstringspaces=false,
    literate={\ }{{\ }}1,
    aboveskip=-0.00in,
    belowskip=-0.15in,
}

\begin{lstlisting}[
language=diff, 
style=lst,
caption=An example of security commit detected by \gnn{} trained on the base and pilot datasets.,
label={lst:pilot},
mathescape=true
]
 $\textbf{commit 9d8adbc07c384ba51c2583ce0819c9abb77dc648}$ 
 diff --git .../__init__.py .../__init__.py
 @@ -71,7 +71,7 @@ def klauen(self,
-    a = u"name == '%s' AND item =='%s'" % (name, item)
+    a = u"name == ? AND item ==?", (name, item)
\end{lstlisting}
%  $\textbf{commit c658b4f3e57258acf5f6207a90c2f2169698ae22}$ 
%  diff --git a/core.py b/core.py
%  @@ -112,7 +112,7 @@ def actualsys() :
% -    if os.environ.get('GITHUB_ACTIONS') != "":
% +    if os.environ.get('GITHUB_ACTIONS') == "true":
%          logging.warning("Running on Github Actions")
%          actualsys()

\subsubsection{Variance Introduced by Augmented Dataset}
We further evaluate to show that our augmented dataset can help train a model that is able to identify more various security commits from the wild. For example, after introducing augmented dataset into the training phase, the model detects a new escape pattern. As shown in List~\ref{lst:augmented}, the characters \texttt{\small <}, \texttt{\small >}, and \texttt{\small \&} have been escaped by being translated into Unicode, which prevents cross-site-scripting crafted with a partial JSON-serializable object. Compared with the escape expressions in Section~\ref{rq3} that only include ASCII characters, the augmented dataset help \gnn{} generalize the escapes to Unicode.


%46e95f5

\lstdefinestyle{lst}{
    float=th,
    floatplacement=tbp,
    % abovecaptionskip=0.01in,
    numbers=left, 
    numberstyle=\scriptsize, 
    numbersep = 5pt,
    framexleftmargin = 0in,
    framexrightmargin = 0in,
    breaklines = true,
    xleftmargin = 0.18in,
    xrightmargin = 0.1in,
    basicstyle=\ttfamily\scriptsize, 
    frame=lines,
    showtabs=true,
    showspaces=true,
    showstringspaces=false,
    literate={\ }{{\ }}1,
    aboveskip=-0.00in,
    belowskip=-0.15in,
}

\begin{lstlisting}[
language=diff, 
style=lst,
caption=A security commit example detected by the \gnn{} trained on the base{,} pilot{,} and augmented datasets.,
label={lst:augmented},
mathescape=true
]
 $\textbf{commit d3e428a6f7bc4c04d100b06e663c071fdc1717d9}$ 
 diff --git a/.../djblets_js.py b/.../djblets_js.py 
 @@ -28,11 +28,18 @@
+_safe_js_escapes = {
+    ord('&'): u'\\u0026',
+    ord('<'): u'\\u003C',
+    ord('>'): u'\\u003E',
+}
\end{lstlisting}
\vspace{-0.03in}
\section{Discussion}
\label{sec:discussion}

\noindent{\bf Usability.} The \gnn{} is versatile and not tied to any specific platform or commit format, making it compatible with a wide range of version control systems like GitHub and GitLab. By integrating the \gnn{} as an extension, contributors can easily add vulnerability fix labels to their commits, streamlining the code auditing process and reducing the workload on developers. In addition, downstream developers and users who use third-party libraries can benefit from the \gnn{} by being reminded to make necessary security fixes on time. %Users will also be more aware of the importance of patching vulnerabilities, which can reduce the risk of exploitation. 
Finally, researchers can obtain labeled commits without requiring extensive manual labor for future data-driven vulnerability and patch-related research.

%\gnn{} is platform-independent and commit format independent. Thus, it can be integrated into any version control platform, such as GitHub and GitLab. With the \gnn{} as an extension, the contributor will be assisted in attaching a vulnerability fix label to the commit to be pushed, which will release pressure on developers and speed up the code auditing process. Moreover, \gnn{} will also benefit downstream developers who utilize third-party libraries to pinpoint security commits in the upstream dependencies by reminding them to make necessary security fixes in time. Besides, users can be aware of the importance and urgency of patching the vulnerabilities to reduce the time window of being exploited. Meanwhile, researchers could acquire such labeled commits without exhaustive labor work. 

\noindent{\bf Ethical Consideration.} The \gnn{} has the potential to identify undisclosed vulnerability fixes, but this presents a mixed blessing as attackers could exploit this information to target unpatched systems. Our objective in this paper is to prioritize the security of the users' systems; that is why we only share detailed information on the security fixes, rather than the vulnerabilities. By taking this approach, attackers cannot leverage the \gnn{} to gain additional details on the vulnerabilities. However, with the \gnn{}, open-source software maintainers can quickly reveal vulnerabilities as soon as security fixes become public, improving the overall security of their software systems.

%\gnn{} could help pinpoint silent vulnerability fixing before they are publicly disclosed. However, this is a mixed blessing, since attackers may leverage the silent information to exploit the delay-patched vulnerability. In this paper, our primary goal is to enhance the security of the user system, thus we only release detailed information about the silent patches, not the vulnerability itself. Under such conditions, attackers cannot be beneficial from \gnn{} to obtain additional vulnerability details. On the other side, with the assistance of \gnn{}, OSS maintainers can promptly disclose vulnerabilities once the corresponding fixes are publicly available. 

%\gnn{} could help pinpoint silent vulnerability fixes before these fixes are disclosed. Our primary goal is to protect users' systems from being exploited, and we will only release the details of its patch without publishing the vulnerabilities detail. Thus, the attackers cannot benefit from \gnn{} to exploit the target user's system. Meanwhile, we recommend OSS maintainers promptly disclose vulnerabilities once the corresponding fixes are publicly available. 

% \subsection{Qualitative Analysis}

% % ignore code features in dataset generation

% for example, the security commits are smaller than non-security ones due to ..



% In which cases the samples will have a higher probability to be security commits
\vspace{-0.03in}
\section{Threats to Validity}
\label{sec:threats}

%真的彻底解决这个问题了吗,有没有引入新的问题,我们不予考虑


\noindent{\bf Threats to internal validity.} 
Internal bias and errors pose a significant risk in the dataset construction phase. %which may lead to several threats. 
The most essential threat is the exclusion of critical keywords or tokens that are important for identifying security-related commits. 
To address this issue, we propose an automated approach for learning security-related keywords. 
However, the current approach of topic models prioritizes the occurrence probabilities of words, which may lead to ignoring infrequent but essential words or tokens. Additionally, commits lacking proper documentation or commit messages are often overlooked, leading to further bias in the dataset. 


%There are \textcolor{red}{xxx} threats related to internal bias and errors. One threat is from the dataset construction phase. We propose to learn the security-related keywords automatically. Since the topic models prioritize the probabilities of occurrence, the word with less occurrence will be excluded, even if the word/token is important, which will lead to the ignorance of the commit that shares that keyword/token. Meanwhile, we neglect the commits without a commit message or have not been well-documented. To reduce this bias, we randomly selected 10\% commits from the non-security set to make sure we will not assign such commits as non-security. 

%Besides, we assume that all information from MITRE CVE is correct, which implies that the commit from the URLs provided by the MITRE CVE is its corresponding vulnerability fixes.

%Another threat may introduce bias caused by manually verifying security commits candidates. To mitigate such bias, we hire three experts and only take the commits that have a unanimous opinion.

\noindent{\bf Threats to external validity.} 
Our experiment and dataset are focused on Python commits, which may limit the generalizability of the \gnn{} to other programming languages. Also, since our dataset derives from open-source software, the data may not be applicable for identifying security-related commits in closed-source systems. However, we aim to expand the scope of our research in the future by incorporating more programming languages and diversifying our dataset to enhance the applicability of the \gnn{}.

%We build the dataset and conduct our experiment on Python commits, thus we may not generalize \gnn{} to other programming languages. In addition, our dataset is from open-source software. It might be not extensible to identify security commits from the close source system. In the future, we plan to include more programming languages and diversify our dataset.

\noindent{\bf Threats to construct validity.} 
\gnn{} is built on top of Joern~\cite{joern} to construct the code property graphs for the programs of previous and current versions; thus, \gnn{} inherits the limitations of Joern. Since Joern disregards the calling relations among multiple functions, \gnn{} cannot handle the commits that only change the function calls. Besides, Joern cannot identify the import and use operations of Python packages; thus, \gnn{} discards these edges in \cpg{} when the commits only operate packages.

%We rely on Joern~\cite{joern} to construct the code property graph of the previous commit and current commit, which imply that \gnn{} inherit the limitations from Joern. 
%One threat is that the calling relations among multiple functions are disregarded, so we cannot handle the commits only by changing the function calls. 
%Furthermore, we do not consider the relationships between the package import and the package usage. Thus, there will be no edge when commits only change the package usage, which will be discarded.

% \noindent{\bf Threats to conclusion validity.} 

% data not in the training set may lead to bias in the conclusion.
\vspace{-0.03in}
\section{Conclusion and Future Work}
\label{sec:conclusion}
By fully leveraging the commit message and the code change semantics, we construct a large-scale Python security commit dataset named \db{} that consists of three parts: base, pilot, and augmented datasets. 
To enrich the base dataset extracted from CVE reports, the pilot dataset collects the commits that contain the pre-defined security keywords in their commit messages. 
Given the diversified data samples, we further train \gnn{} to learn the security semantics in the code changes to compensate for the poorly documented commits. 
We conduct a large-scale empirical study of security commits by analyzing \db{} of 119 CWE categories across 351 repositories. 
The summarized patterns can assist further software maintenance, e.g., auto program repair.
% Our summarized pattern can be well-assisted in the auto program repair designs.
In the future, we will adopt large language models to expand the dataset and apply the summarized patterns to fix the vulnerabilities automatically.

\section*{Acknowledgments}
% \vspace{-0.03in}
We would like to thank our anonymous reviewers for their valuable comments and suggestions. This work was partially supported by the US Office of Naval Research grants N00014-23-1-2122.

\bibliographystyle{ieeetr}
\bibliography{reference}

\end{document}
