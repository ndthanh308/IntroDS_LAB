% \section{Case Study}
% \label{sec:casestudy}

\lstdefinestyle{lst}{
    float=th,
    floatplacement=tbp,
    % abovecaptionskip=0.01in,
    numbers=left, 
    numberstyle=\scriptsize, 
    numbersep = 5pt,
    framexleftmargin = 0in,
    framexrightmargin = 0in,
    breaklines = true,
    xleftmargin = 0.18in,
    xrightmargin = 0.1in,
    basicstyle=\ttfamily\scriptsize, 
    frame=lines,
    showtabs=true,
    showspaces=true,
    showstringspaces=false,
    literate={\ }{{\ }}1,
    aboveskip=-0.00in,
    belowskip=-0.15in,
}



To develop a more comprehensive understanding of Python security commits and better assist in automated program repair, 
%we sample xxx\% from the 
manually go through the final dataset and analyze their code change patterns. %We find that the summarized patterns 
Our summarized patterns can cover xxx\% of samples in the dataset, which can be further grouped into the following five distinct categories: update regular expression, add or update sanity check, update API package, and update security properties. We also visualize and interpret the latent representations of these commits to explain the \TN{}.

\subsection{Pattern Summarization}

\subsubsection{Update Regular Expression}
%Among the xxxx sampled commits, 
In our dataset, around xxx\% fix the vulnerability by updating the regular expression. The regular expression patterns are tailored to the specific requirements of each distinct application dealing with divergent kinds of strings, including URLs, SQL commands, Code, and other scripts.

\noindent{\bf SQL Command.} The improper neutralization of SQL command may lead to SQL injection vulnerability, which makes the information from the backend database that was not intended to be displayed accessible. The corresponding fixes need to escape the unsafe character. List~\ref{lst:sql} is a fixed example of SQL injection vulnerability which substitutes the matched single or double quote character (\texttt{"} or \texttt{'}) in $self.queueid$.


\begin{lstlisting}[
language=diff, 
style=lst,
caption=An example of a security commit that fixes a SQL injection vulnerability (CVE-2014-125082).,
label={lst:sql},
mathescape=true
]
 $\textbf{commit fc2c1ea1b8d795094abb15ac73cab90830534e04}$
 diff --git a/.../model.py b/.../model.py
 @@ -772,13 +772,13 @@ def _get_filter(self):
 $\textbf{if}$ self.queueid:
-    ... = '%s'" % (self.queueid)
+    ... = '%s'" % (re.sub("[\"']", "", self.queueid))
\end{lstlisting}


\noindent{\bf URL.} The improper neutralization of URLs may lead to open redirect vulnerability, which redirects an unsuspecting victim from a legitimate domain to an attacker’s phishing site. Effective mitigation is to replace the dangerous special characters with trusted symbols. List~\ref{lst:redirect} is an example of an open redirect vulnerability that replace the backslash with an encoded backslash to circumvent the dangerous redirect.

\begin{lstlisting}[
language=diff, 
style=lst,
caption=An example of a security commit that fixes an open redirect vulnerability (CVE-2019-10255).,
label={lst:redirect},
mathescape=true
]
 $\textbf{commit 08c4c898182edbe97aadef1815cce50448f975cb}$ 
 diff --git a/notebook/auth/login.py b/notebook/auth/login.py
 @@ -39,6 +39,10 @@ def _redirect_safe(self, url, default=None):

+    url = url.replace("\\", "%5C")
     parsed = urlparse(url)
     $\textbf{if}$ parsed.netloc $\textbf{or not}$ (parsed.path + '/').startswith(self.base_url):
\end{lstlisting}

\noindent{\bf Scripts.} The improper validation and encoding of input during web page generation may lead to Cross-site Scripting (XSS), which will reveal the cookies, session tokens, or other sensitive information retained by the browser to the attacker. A straightforward solution is to validate the matched characters of a predefined pattern. List~\ref{lst:xss} is an example to fix XSS by re-matching the characters between parentheses instead of characters between square brackets and validate the matched pattern one by one.

\begin{lstlisting}[
language=diff, 
style=lst,
caption=An example of a security commit that fixes an XSS (CVE-2021-3994).,
label={lst:xss},
mathescape=true
]
 $\textbf{commit a22eb0673fe0b7784f99c6b5fd343b64a6700f06}$ 
 diff --git a/helpdesk/models.py b/helpdesk/models.py
 @@ -238 +238 @@ def cvesForCPE(cpe,
     $\textbf{if not}$ text:
         $\textbf{return}$ ""
-    pattern = fr'([\[\s\S\]]*?)\(([\s\S]*?):([\[\s\S\]]*?)\)'
+    pattern = fr'([\[\s\S\]]*?)\(([\s\S]*?):([\s\S]*?)\)'
     # Regex check
     $\textbf{if}$ re.match(pattern, text):
         # get get value of group regex
\end{lstlisting}

\subsubsection{Add or Update Sanity Checks}
A sanity check is a basic method to quickly evaluate whether a claim or the result of a calculation can possibly be true, which has been extensively applied to property checking for authentication, access control and etc. We summarize three representative patterns that fix the vulnerability via adding or updating sanity checks, which are presented by xxx\% of samples.

\noindent{\bf Authentication.} Authentication is the act of proving an assertion, such as the identity of a system user. To verify the identity of a user, we need to their identity with system data. The authentication-related vulnerability occurs when the authentication is performed improperly, which gives the attacker an opportunity to masquerade as a legitimate user. An effective solution to perform the restricted authentication is to add more check requirements or narrow the requirement's scope. List~\ref{lst:auth} presents an example of fixing the authentication vulnerability by restricting the condition to $on$ instead of $true$.

\begin{lstlisting}[
language=diff, 
style=lst,
caption=An example of security commit of authentication (CVE-2022-0273).,
label={lst:auth},
mathescape=true
]
 $\textbf{commit 0c0313f375bed7b035c8c0482bbb09599e16bfcf}$ 
 diff --git a/cps/shelf.py b/cps/shelf.py
 @@ -248,7 +248,7 @@ def create_edit_shelf(shelf,
 ...
              return redirect(url_for('web.index'))
-        is_public = 1 if to_save.get("is_public") else 0
+        is_public = 1 if to_save.get("is_public") == "on" else 0
         if config.config_kobo_sync:
 ...
\end{lstlisting}

\noindent{\bf Authorization.} Authorization refers to the process of granting or denying access to certain data or actions within a system, which is followed after the authentication and achieved by an access control list (ACL) to compare the user's identity with a list of authorized operations, determining what actions the user is allowed to perform and which are restricted, including but not limited to file permission and data permission. Unrestricted authorization may lead to improper resource consumption. Attackers could bypass 

\begin{lstlisting}[
language=diff, 
style=lst,
caption=An example of a security commit that fixes an authorization bypass exploit vulnerability (CVE-2022-46179).,
label={lst:access control1},
mathescape=true
]
 $\textbf{commit c658b4f3e57258acf5f6207a90c2f2169698ae22}$  
 diff --git a/core.py b/core.py
 @@ -112,7 +112,7 @@ def actualsys() :
             raise Exception("Too many password attempts. Because of the risk of a brute force attack, after 6 attempts, you will need to rerun LiuOS to try 6 more times.")
-        if os.environ.get('GITHUB_ACTIONS') != "":
+        if os.environ.get('GITHUB_ACTIONS') == "true":
              logging.warning("Running on Github Actions")
              actualsys()
          elif username == cred.loginname and pwdreshash == cred.loginpass:
\end{lstlisting}


\noindent{\bf HTTP Request.} If the interpretation of Content-Length and/or Transfer-Encoding headers between HTTP servers are inconsistent, the attackers may take advantage of it and send malicious requests to the server, which is known as HTTP request smuggling. The best prevention to it would clearly be if front-end and back-end servers interpreted HTTP requests the same way. An effective way is to add consistent sanity checks on the request interpretation of the front-end and back-end servers. List~\ref{lst:http} added sanity on data to determine the content length.

\begin{lstlisting}[
language=diff, 
style=lst,
caption=An example of a security commit that fixes an HTTP smuggling vulnerability (CVE-2022-24801).,
label={lst:http},
mathescape=true
]
$\textbf{commit 8ebfa8f6577431226e109ff98ba48f5152a2c416}$ 
diff --git a/src/twisted/web/http.py b/src/twisted/web/http.py
@@ -2274,6 +2274,8 @@ def fail():
 
         if header == b"content-length":
+            if not data.isdigit():
+                return fail()
             try:
                 length = int(data)
             except ValueError:
\end{lstlisting}


\subsubsection{Update API Packages}
Compared with implementing the fixes from scratch, there are abundant well-formulated packages that can be adopted to realize the intended functionality. We notice that a large number (xxx\%) of Python security commits to fix the vulnerability by substituting the used API packages. We further classify the fixes into different groups according to their application scope.

\noindent{\bf General Purpose.} $re.escape$ is an API that could be utilized to escape non-alphanumerics that are not part of regular expression syntax to avoid OS command injection, code injection, regular expression injection and etc. List~\ref{lst:re} is a patch example of regular expression injection vulnerability, which demonstrates the application of $re.escape$ on $user$ and $collection_url$.

\begin{lstlisting}[
language=diff, 
style=lst,
caption=An example of a security commit that fixes a regular expression injection vulnerability (CVE-2015-8748).,
label={lst:re},
mathescape=true
]
 $\textbf{commit 4bfe7c9f7991d534c8b9fbe153af9d341f925f98}$ 
 diff --git a/radicale/rights/regex.py b/radicale/rights/regex.py
 @@ -65,7 +65,10 @@ def _read_from_sections(user, collection_url, permission):
...
-    regex = ConfigParser({"login": user, "path": collection_url})
+    # Prevent "regex injection"
+    user_escaped = re.escape(user)
+    collection_url_escaped = re.escape(collection_url)
+    regex = ConfigParser({"login": user_escaped, "path": collection_url_escaped})
...
\end{lstlisting}

\noindent{\bf Web Application.} In order to process the input of web applications properly, security fixes adopt APIs, such as $escape\_html$, $request.server.escape$, $django.utils.html.escape$ and etc., to escape ampersands, brackets, and quotes to the HTML/XML entities or HTTP requests, against XSS and HTTP Smuggling. List~\ref{lst:xss2} is an example that shows the usage of $django.utils.html.escape$ fixing XSS vulnerability.

\begin{lstlisting}[
language=diff, 
style=lst,
caption=An example of a security commit that fixes an XSS vulnerability (CVE-2022-24710).,
label={lst:xss2},
mathescape=true
]
 $\textbf{commit f6753a1a1c63fade6ad418fbda827c6750ab0bda }$
diff --git a/weblate/trans/forms.py b/weblate/trans/forms.py
@@ -37,6 +37,7 @@
...
+from django.utils.html import escape
...
-            label = str(unit.translation.language)
+            label = escape(unit.translation.language)
...
\end{lstlisting}


\noindent{\bf Shell Command.} With $shlex.quote$, we can have a shell-escaped version of the input, which can safely be used as one token in a shell command line to avoid shell command injection. List~\ref{lst:shell} is an example that shows the usage of $shlex.quote$.

\begin{lstlisting}[
language=diff, 
style=lst,
caption=An example of a security commit that fixes a shell injection vulnerability (CVE-2013-7416).,
label={lst:shell},
mathescape=true
]
 $\textbf{commit 2817869f98c54975f31e2dd674c1aefa70749cca }$
 diff --git a/canto_curses/guibase.py b/canto_curses/guibase.py
 @@ -156,6 +156,11 @@ def _fork(self, path, href, text, fetch=False):
...
+        href = shlex.quote(href)
...
\end{lstlisting}


\noindent{\bf Path Name.} If the path name has not been properly neutralized, it will attack the chance to access files and directories that are stored outside the web root folder. It can occur by manipulating variables that reference files with “dot-dot-slash (../)” sequences and their variations or by using absolute file paths. To effectively escape such unsafe sequences, Python security commits usually adopt $werkzeug.utils.safe\_join$ and $werkzeug.utils.secure_filename$ to prevent the file from being accessed by malicious users. List~\ref{lst:path traversal} is an example that shows the usage of $werkzeug.utils.secure_filename$.

\begin{lstlisting}[
language=diff, 
style=lst,
caption=An example of a security commit that fixes a path traversal vulnerability (CVE-2022-23609).,
label={lst:path traversal},
mathescape=true
]
 $\textbf{commit 1eb1e5428f0926b2829a0bbbb65b0d946e608593}$ 
 diff --git a/upload/server.py b/upload/server.py
 @@ -5,7 +5,7 @@
-
+import werkzeug.utils
 @@ -189,7 +189,7 @@ def uploadimage():
         filename = all_files[0][1] + all_files[0][2]
-        remove(filename)
+        remove(werkzeug.utils.secure_filename(filename))
         del all_files[0]
         length = len(all_files)
\end{lstlisting}


\subsubsection{Add Security Decorator.}
A decorator is a function that takes another function and extends the behavior of the latter function without explicitly modifying it. This mechanism has been widely adopted by security commits to add more detailed security restrictions on existing methods. List~\ref{lst:access control2} is an example of the commit that fixed an access control vulnerability by adding the $security.private$ to the $enumerateRoles$ function.

\begin{lstlisting}[
language=diff, 
style=lst,
caption=An example of a security commit that fixes an access control vulnerability (CVE-2021-21336).,
label={lst:access control2},
mathescape=true
]
$\textbf{commit 2dad81128250cb2e5d950cddc9d3c0314a80b4bb}$ 
diff --git a/src/Products/PluggableAuthService/plugins/ZODBRoleManager.py b/src/Products/PluggableAuthService/plugins/ZODBRoleManager.py
@@ -112,6 +112,7 @@ def getRolesForPrincipal(self, principal, request=None):
     #   IRoleEnumerationPlugin implementation
+    @security.private
     def enumerateRoles(self, id=None, exact_match=False, sort_by=None,
                        max_results=None, **kw):
         """ See IRoleEnumerationPlugin.
\end{lstlisting}



\subsection{\TN{} Interpretation}
