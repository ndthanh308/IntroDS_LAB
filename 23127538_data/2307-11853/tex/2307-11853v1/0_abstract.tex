\begin{abstract}

Python has become the most popular programming language as it is friendly to work with for beginners. However, a recent study has found that most security issues in Python have not been indexed by CVE and may only be fixed by ``silent" security commits, which pose a threat to software security and hinder the security fixes to downstream software. It is critical to identify the hidden security commits; however, the existing datasets and methods are insufficient for security commit detection in Python, due to the limited data variety, non-comprehensive code semantics, and uninterpretable learned features.
%
In this paper, we construct the first security commit dataset in Python, namely \db{}, which consists of three subsets including a base dataset, a pilot dataset, and an augmented dataset. The base dataset contains the security commits associated with CVE records provided by MITRE.
To increase the variety of security commits, we build the pilot dataset from GitHub by filtering keywords within the commit messages.
Since not all commits provide commit messages, we further construct the augmented dataset by understanding the semantics of code changes.
To build the augmented dataset, we propose a new graph representation named \cpg{} and a multi-attributed graph learning model named \gnn{} to identify the security commit candidates through both sequential and structural code semantics. The evaluation shows our proposed algorithms can improve the data collection efficiency by up to 40 percentage points.
After manual verification by three security experts, \db{} consists of 1,258 security commits and 2,791 non-security commits.
Furthermore, we conduct an extensive case study on \db{} and discover four common security fix patterns that cover over 85\% of security commits in Python, providing insight into secure software maintenance, vulnerability detection, and automated program repair.


% In GitHub, a large number of security patches are silently committed without reporting to NVD or indicating any potential security impacts in its commit messages. 
% Since armored attackers are still able to examine the vulnerabilities from source code changes, OSS users need to timely realize these security commits before being exploited. 
% On the other hand, with the popularity of Python, a non-negligible number of exploits emerge. 
% However, a dataset dedicated to studying security vulnerabilities and patches in Python is critically lacking. 
% While discovering security patches from GitHub commits is promising, existing works either require well-maintained documentation or fail to extract structural information from source code. 
% To facilitate silent security commit detection, we construct a large-scale real-world Python security commit dataset named \db{} that consists of three parts: base, pilot, and augmented datasets. 
% The base dataset is extracted from the hyperlinks containing security commits associated with CVE records provided by the MITRE. 
% The pilot dataset is composed of security commits from GitHub. 
% To collect more patches with variances of code change semantics, we design a keyword filtering mechanism with humans in the loop to efficiently collect the commits that illustrate their security impacts in the commit message. 
% With datasets including diversified code semantics, we develop a graph learning-based model to automatically discover security commits from the wild. 
% A novel graph representation named \cpg{} and a multi-attributed graph neural network named \gnn{} are employed to effectively capture inherent sequential and structural semantics from patch code changes. In this way, we further provide an augmented dataset of disclosed security commits. 
% Extensive evaluations are conducted to examine the effectiveness of the proposed algorithms, and the results show that they can reduce the extensive labor work to a large extent. 
% Leveraging the diversified security commits, we investigate what and how the commit fixes the corresponding vulnerability and discover four common patterns. Such patterns can be applied to facilitate software maintenance, especially security patch generation.

\end{abstract}

\begin{IEEEkeywords}
Security Commit, Python, Dataset Construction, Code Property Graph, Graph Learning, Vulnerability Fixes
\end{IEEEkeywords}



% Python overtakes Java and C as the most popular programming language, and a non-negligible number of exploits emerge. However, more than 50\% of the vulnerabilities do not have a public fix. Such silent fixes pose a threat to the security and privacy of users, since attackers may exploit the undisclosed vulnerabilities to comprise the unpatched software systems. 


