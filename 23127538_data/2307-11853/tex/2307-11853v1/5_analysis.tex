\section{Analysis Results}
\label{sec:experiment}
%To develop a more comprehensive understanding of Python security commits, we conduct a series of analysis: (1) Dataset Characteristics, (2) Security Commits Categories to reveal what types of vulnerabilities have been fixed, (3) Security Commits Distribution over Repositories, (4) Security Commits Complexity, (5) Security Commits Locality and (6) Fix Patterns Summarization to assist in automated program repair projects. 

After constructing our datasets, we frame our evaluation into four research questions, as outlined below. 
\begin{itemize}[leftmargin=*]

\item \textbf{RQ1:} Can the graph learning-based method help improve the data collection efficiency?
\item \textbf{RQ2:} How various and representative are the collected security commits? 
\item \textbf{RQ3:} What are the unique patterns of security commits in Python? 
\item \textbf{RQ4:} How do the wild commit samples help improve \gnn{} model for downstream security commit detection? 
\end{itemize}

\subsection{Dataset Construction (RQ1)}\label{results:efficiency}

After keyword filtering and graph-based identification with humans in the loop, we collect 1,258 security commits in total. Specifically, as shown in Table~\ref{tab: dataset}, there are 729, 400, and 129 security commits in the base, pilot, and augmented datasets, respectively. Also, 2,791 non-security commits are manually labeled during the collection process.

% \begin{table}[h]
% \centering
%     \caption{The statistical information of \db{}.}
%     \setlength{\tabcolsep}{3.4mm}{
%     \begin{tabular}{c|c|c|c|c}
%     \toprule
%     {} & \multirow{2}{*}{\shortstack{\bf Base\\\bf Dataset}} & \multirow{2}{*}{\shortstack{\bf Pilot\\\bf Dataset}} & \multirow{2}{*}{\shortstack{\bf Augmented\\\bf Dataset}} & \multirow{2}{*}{\bf Total} \\
%     {} & {} & {} & {} & {} \\
%      % & \textbf{Base} & \textbf{Pilot} & \textbf{Augmented} & \multirow{2}{*}{\textbf{Total}} \\
%      % & \textbf{Dataset} & \textbf{Dataset} & \textbf{Dataset} & \\
%      % & Base Dataset & Pilot Dataset & Augmented Dataset & Total \\
%                          % & \begin{tabular}[c]{@{}c@{}}Base \\ Dataset\end{tabular} & \begin{tabular}[c]{@{}c@{}}Pilot \\ Dataset\end{tabular} & \begin{tabular}[c]{@{}c@{}}Augmented \\ Dataset\end{tabular}  & Total \\ \hline
%     \midrule
%     \multirow{2}{*}{\shortstack{\bf Security\\\bf Commits}} & \multirow{2}{*}{729} & \multirow{2}{*}{400} & \multirow{2}{*}{129} & \multirow{2}{*}{1258} \\
%     {} & {} & {} & {} & {} \\
%     \midrule
%     \multirow{2}{*}{\shortstack{\bf Non-security\\\bf Commits}} & \multirow{2}{*}{2134} & \multirow{2}{*}{535} & \multirow{2}{*}{122} & \multirow{2}{*}{2791} \\
%     {} & {} & {} & {} & {} \\
%     \bottomrule
%     \end{tabular}
%     }
%     \label{tab: dataset}
% \end{table}

\begin{table}[h]
\vspace{-0.05in}
\centering
    \caption{The composition of \db{}.}
    \setlength{\tabcolsep}{1.7mm}{
    \begin{tabular}{c|p{1.0cm}<{\centering}|p{1.0cm}<{\centering}|p{1.4cm}<{\centering}|p{1.0cm}<{\centering}}
    \toprule
    {\diagbox{\bf Commit}{\bf Dataset}} & {\bf Base} & {\bf Pilot} & {\bf Augmented} & {\bf Total} \\
     % & \textbf{Base} & \textbf{Pilot} & \textbf{Augmented} & \multirow{2}{*}{\textbf{Total}} \\
     % & \textbf{Dataset} & \textbf{Dataset} & \textbf{Dataset} & \\
     % & Base Dataset & Pilot Dataset & Augmented Dataset & Total \\
                         % & \begin{tabular}[c]{@{}c@{}}Base \\ Dataset\end{tabular} & \begin{tabular}[c]{@{}c@{}}Pilot \\ Dataset\end{tabular} & \begin{tabular}[c]{@{}c@{}}Augmented \\ Dataset\end{tabular}  & Total \\ \hline
    \midrule
    {\bf Security} & {729} &  {400} &  {129} & {1258} \\
    \midrule
    {\bf Non-Security} & {2134} & {535} & {122} &{2791} \\
    \bottomrule
    \end{tabular}
    }
    \label{tab: dataset}
\vspace{-0.05in}
\end{table}

Table~\ref{tab:spr} lists the augmentation efficiency of random selection, keyword filtering, and \gnn{}.
Compared with identifying security commits from scratch, the keyword filtering mechanism improves the collecting efficiency by over 30 percentage points and \gnn{} improves the efficiency by 40 percentage points. %It is to be noted that we only test our data augmentation methods on a small portion of commits, which has already shown effectiveness.



\begin{table}[ht]
\vspace{-0.05in}
\centering
\caption{Efficiency of keyword filtering and \gnn{}.}
\setlength{\tabcolsep}{4mm}{
\begin{tabular}{c|c|c|c}
\toprule
% \multirow{2}{*}{\textbf{Methods}} & \multirow{2}{*}{\textbf{Candidates}} & \multirow{2}{*}{\shortstack{\bf Verified\\ \bf Security Commits}} & \multirow{2}{*}{\textbf{Ratio}} \\
% {} & {} & {} & {} \\
%  \midrule
\textbf{Method}      & \textbf{\# Candidates} & \textbf{\# Verified SC$^{*}$} & \textbf{Ratio} \\
 \midrule
% Methods      & Candidates & \begin{tabular}[c]{@{}c@{}}Verified \\ Security Commits\end{tabular} & Ratio   \\ \hline
{Random~\cite{wang2021patchdb}} & {-} & {-} & {6-10\%}    \\ 
\midrule
{Keywords} & {935} & {400} & {42.70\%} \\ 
\midrule
{\gnn{}} & {251} & {129} & {51.39\%} \\ 
\bottomrule

\end{tabular}
}
\begin{tablenotes}[flushleft]
    \footnotesize
    \item $^{*}$ SC = Security Commits.
\end{tablenotes}
% \vspace{-0.1in}
\label{tab:spr}
\vspace{-0.15in}
\end{table}

\begin{table}[]
\centering
\caption{Top 5 repositories by number of security commits.}
\label{tab:repo}
\setlength{\tabcolsep}{5.4mm}{
\begin{tabular}{c|c|c}
\toprule
\textbf{Repository} & \textbf{\#SecurityCommits} & \textbf{\textbf{Proportion}} \\
\midrule
django      & 166  & 13.20\%   \\ \midrule
twisted     & 87   & 6.91\%   \\ \midrule
glance      & 54   & 4.29\%     \\ \midrule
pillow     & 41   & 3.26\%     \\ \midrule
numpy       & 39   & 3.10\%        \\ \midrule
\rowcolor{gray!10}\textbf{Total of Top 5}                   & \textbf{387}   &   \textbf{30.76\%} \\
\bottomrule
\end{tabular}
}
\vspace{-0.1in}
\end{table}


\subsection{Security Commits Categorization and Distribution (RQ2)}
% \XD{should show broad coverage and variety of DB}}

%The Common Weakness Enumeration (CWE)
NVD CWE slice~\cite{CWE_slice} associated classification taxonomy serves to identify and describe security vulnerabilities.
% in terms of CWEs. 
To understand the purpose of these commits, we investigate the CWE types associated with the CVE reports and plot the distribution of the CWE types that have been explicitly documented. Among the 729 security commits linked to 556 CVEs, due to the limited number of MITRE human analysts, only 312 (56.1\%) CVEs have been assigned CWEs. %Since the CWE taxonomy is a hierarchical structure, a CVE can be assigned with more than one CWE. 
Even so, there are already 119 distinct CWEs associated with our security commits in the base dataset, which means our \db{} contains at least 119 types of security commits in terms of corresponding vulnerabilities.
Figure~\ref{fig:cwe} enumerates the most common CWEs, including frequent security problems such as cross-site scripting (CWE-79), path traversal (CWE-22), etc. Note that we do not directly assign CWE type to security samples in the remaining base, pilot, and augmented dataset since the MITRE CWE team has its own internal process. However, based on our observation and our data collection approaches that are able to introduce wild security commits with more variance (as discussed in~\ref{exp:variance}), \db{} can encompass a broad range of security concerns with various kinds of security commits, including but not limited to above 119 CWEs. 

% % Figure environment removed

% Figure environment removed


Our collected security commits distribute among 351 popular GitHub repositories unevenly. Among them, 69 repositories provide more than two security commits, bringing a certain amount of variety. In Table~\ref{tab:repo}, the top five repositories that have the most occurrence in our dataset are django~\cite{django_2023}, twisted~\cite{twisted_2023}, glance~\cite{openstack_2023}, pillow~\cite{python-pillow_2023}, and numpy~\cite{numpy_2023}, implying that the samples in \db{} align with the popularity trend of security issue in practice.
%react to security issues on time.




% \subsection{Security Commits Complexity}

% \subsection{Security Commits Locality}

\subsection{Patch Patterns (RQ3)}
\label{rq3}

We manually go through the whole \db{} dataset, 
% we manually explore the full security commit dataset. 
% samples that have less than 200 code lines.
% We find 1,027 samples in total, which take up 81.7\% of the security commits.
% After the comprehensive analysis, 
and discover four common security fix patterns (taking up 85.85\% of all security commit samples) that may benefit software maintenance, i.e., adding or updating sanity checks,  updating APIs, updating regular expressions, and updating security properties, as listed in Table~\ref{tab:pattern}.


\begin{table}[]
\centering
%\scriptsize
\caption{The pattern types of security commits in \db{}.}
\label{tab:pattern}
\setlength{\tabcolsep}{4.0mm}{
\begin{tabular}{l|c|c}
\toprule
    \textbf{Pattern} & \textbf{\#Commits} & \textbf{Proportion} \\ 
    \midrule
    {1) Add or Update Sanity Checks} & {416} & {37.12\%} \\ 
    \midrule
    {2) Update API Usage} & {241} & {19.16\%} \\ 
    \midrule
    {3) Update Regular Expressions} & {189} & {15.02\%} \\ 
    \midrule
    {4) Restrict Security Properties} & {183} & {14.55\%} \\ 
    \midrule
    {5) Others} & {178} & {14.15\%} \\ 
    \midrule 
    \rowcolor{gray!10}{\bf Total} & {\bf 1258} & {\bf 100\%} \\
    \bottomrule
\end{tabular}
}
\vspace{-0.1in}
\end{table}


%re 168 + 21
%api 214 +27
%if 420 + 47
%property 167 +16
%others 159 + 19

%re 151
%api 190
%if 369
%property 146
%others 43


\subsubsection{Add or Update Sanity Checks}
A sanity check is a basic method to quickly evaluate if a claim or a calculation result can be true, which has been extensively applied to multiple scenarios, e.g., authentication property verification, access control, HTTP request checking~\cite{wang2020machine}. 
We summarize three representative patterns that fix the vulnerabilities via adding or updating sanity checks, which are presented by 37.12\% of security commits in \db{}.

\noindent{\bf Authentication.} Authentication is the act of proving an assertion, e.g., we need to compare the identity with the system data to verify a system user. The authentication-related vulnerabilities
% occurs when the authentication is performed improperly, which 
provide attackers the opportunities to masquerade as legitimate users. To defend them, an effective solution is to perform the additional authentication by adding more check requirements or making existing conditions more restrictive. List~\ref{lst:auth} presents an example of fixing an authentication vulnerability by narrowing down an existing restriction from \texttt{\small True} (i.e., all possible return values except \texttt{\small False}) to \texttt{\small "on"} only.


\lstdefinestyle{lst}{
    float=th,
    floatplacement=tbp,
    % abovecaptionskip=0.01in,
    numbers=left, 
    numberstyle=\scriptsize, 
    numbersep = 5pt,
    framexleftmargin = 0in,
    framexrightmargin = 0in,
    breaklines = true,
    xleftmargin = 0.18in,
    xrightmargin = 0.1in,
    basicstyle=\ttfamily\scriptsize, 
    frame=lines,
    showtabs=true,
    showspaces=true,
    showstringspaces=false,
    literate={\ }{{\ }}1,
    aboveskip=-0.00in,
    belowskip=-0.15in,
}

\begin{lstlisting}[
language=diff, 
style=lst,
caption=An example of security commit to fix authentication vulnerability (CVE-2022-0273).,
label={lst:auth},
mathescape=true
]
 $\textbf{commit 0c0313f375bed7b035c8c0482bbb09599e16bfcf}$ 
 diff --git a/cps/shelf.py b/cps/shelf.py
 @@ -248,7 +248,7 @@ def create_edit_shelf(shelf,
 ...
         $\textbf{return}$ redirect(url_for('web.index'))
-    is_public = 1 if to_save.get("is_public") else 0
+    is_public = 1 if to_save.get("is_public") == "on" else 0
     $\textbf{if}$ config.config_kobo_sync:
 ...
\end{lstlisting}

\noindent{\bf Authorization.} Authorization refers to the process of granting or denying access to certain data or actions within a system.
Authorization comes after authentication and is achieved by an access control list (ACL).
The ACL is used to check the user identity with a list of authorized operations and determine which actions a user is allowed to take, e.g., file and data permission.
% determining  to perform and which are restricted, including but not limited to file permission and data permission. 
Unrestricted authorization may lead to improper resource consumption since attackers could bypass the system to access high-security level data. List~\ref{lst:access control1} is an example that fixes an authorization bypass exploit by requiring the value of \texttt{\small os.environ.get('GITHUB\_ACTIONS')} to be \texttt{\small true}.

\lstdefinestyle{lst}{
    float=th,
    floatplacement=tbp,
    % abovecaptionskip=0.01in,
    numbers=left, 
    numberstyle=\scriptsize, 
    numbersep = 5pt,
    framexleftmargin = 0in,
    framexrightmargin = 0in,
    breaklines = true,
    xleftmargin = 0.18in,
    xrightmargin = 0.1in,
    basicstyle=\ttfamily\scriptsize, 
    frame=lines,
    showtabs=true,
    showspaces=true,
    showstringspaces=false,
    literate={\ }{{\ }}1,
    aboveskip=+0.10in,
    belowskip=-0.30in,
}

\begin{lstlisting}[
language=diff, 
style=lst,
caption=An example of security commit that fixes an authorization bypass exploit vulnerability (CVE-2022-46179).,
label={lst:access control1},
mathescape=true
]
 $\textbf{commit c658b4f3e57258acf5f6207a90c2f2169698ae22}$  
 diff --git a/core.py b/core.py
 @@ -112,7 +112,7 @@ def actualsys() :
     $\textbf{if}$ attemps == 6:
         ## Brute force protection
         $\textbf{raise}$ Exception("Too many password attempts.")
-    if os.environ.get('GITHUB_ACTIONS') != "":
+    if os.environ.get('GITHUB_ACTIONS') == "true":
         logging.warning("Running on Github Actions")
         actualsys()
     $\textbf{elif}$ uname == cred.name and pwdhash == cred.pass:
\end{lstlisting}


\noindent{\bf HTTP Request.} If the interpretation of Content-Length and/or Transfer-Encoding headers between HTTP servers are inconsistent, the attackers may take advantage of this issue and send malicious requests to the servers, i.e., HTTP request smuggling. 
A good solution is to maintain the same interpretation methods in both front-end and back-end servers. 
In this way, an effective coding practice is to add consistent sanity checks on request interpretation for both servers. 
List~\ref{lst:http} adds such a sanity check on \texttt{\small data} to determine if all characters are digits.% to avoid such exploits.

\lstdefinestyle{lst}{
    float=th,
    floatplacement=tbp,
    % abovecaptionskip=0.01in,
    numbers=left, 
    numberstyle=\scriptsize, 
    numbersep = 5pt,
    framexleftmargin = 0in,
    framexrightmargin = 0in,
    breaklines = true,
    xleftmargin = 0.18in,
    xrightmargin = 0.1in,
    basicstyle=\ttfamily\scriptsize, 
    frame=lines,
    showtabs=true,
    showspaces=true,
    showstringspaces=false,
    literate={\ }{{\ }}1,
    aboveskip=-0.00in,
    belowskip=-0.15in,
}

\begin{lstlisting}[
language=diff, 
style=lst,
caption=An example of security commit that fixes an HTTP request smuggling vulnerability (CVE-2022-24801).,
label={lst:http},
mathescape=true
]
 $\textbf{commit 8ebfa8f6577431226e109ff98ba48f5152a2c416}$ 
 diff --git a/src/twisted/web/http.py b/src/twisted/web/http.py
 @@ -2274,6 +2274,8 @@ def fail():
     $\textbf{if}$ header == b"content-length":
+        if not data.isdigit():
+            return fail()
         $\textbf{try}$:
             length = int(data)
         $\textbf{except}$ ValueError:
\end{lstlisting}


\subsubsection{Update API Usage}% Packages}
Compared with implementing the fixes from scratch, there are abundant well-formulated packages that can be adopted to realize the intended functionalities and help enforce security restrictions. 
We notice that a large number (19.16\%) of Python security commits fix vulnerabilities by imposing or substituting APIs. %packages, which is different from other languages like C/C++.
%This pattern differs from the fix patterns in other languages, e.g., C/C++.
We further categorize such security fixes % into different types 
according to their application scenarios. %scopes.

\noindent{\bf General Purpose.} There is a set of security-related modifications on built-in packages shared by applications for various purposes. For instance, \texttt{\small re.escape} is an API to escape non-alphanumerics that are not part of regular expression syntax, to avoid OS command injection, code injection, and regular expression injection. List~\ref{lst:re} is a commit example to fix regular expression injection vulnerability, which demonstrates the application of \texttt{\small re.escape} on \texttt{\small user} and \texttt{\small collection\_url}.

\lstdefinestyle{lst}{
    float=th,
    floatplacement=tbp,
    % abovecaptionskip=0.01in,
    numbers=left, 
    numberstyle=\scriptsize, 
    numbersep = 5pt,
    framexleftmargin = 0in,
    framexrightmargin = 0in,
    breaklines = true,
    xleftmargin = 0.18in,
    xrightmargin = 0.1in,
    basicstyle=\ttfamily\scriptsize, 
    frame=lines,
    showtabs=true,
    showspaces=true,
    showstringspaces=false,
    literate={\ }{{\ }}1,
    aboveskip=-0.00in,
    belowskip=-0.15in,
}

\begin{lstlisting}[
language=diff, 
style=lst,
caption=An example of security commit that fixes a regular expression injection vulnerability (CVE-2015-8748).,
label={lst:re},
mathescape=true
]
 $\textbf{commit 4bfe7c9f7991d534c8b9fbe153af9d341f925f98}$ 
 diff --git a/radicale/rights/regex.py b/radicale/rights/regex.py
 @@ -65,7 +65,10 @@ def _read_from_sections(user, collection_url, permission):
 ...
-    regex = ConfigParser({"login": user, "path": collection_url})
+    # Prevent "regex injection"
+    user_escaped = re.escape(user)
+    collection_url_escaped = re.escape(collection_url)
+    regex = ConfigParser({"login": user_escaped, "path": collection_url_escaped})
 ...
\end{lstlisting}

\noindent{\bf Web Applications.} To properly process the inputs of web applications, security commits can adopt %the existing APIs %, e.g., \texttt{\small escape\_html}, 
APIs in third-party packages for Python
(e.g., \texttt{\small parser.quote}, \texttt{\small request.server.escape}, \texttt{\small django.utils.html.escape}, and \texttt{\small html.unescape}) to escape ampersands, brackets, and quotes to the HTML/XML entities or HTTP requests for defeating cross-site scripting (XSS) and HTTP Smuggling. 
List~\ref{lst:xss2} is an example that fixes an XSS vulnerability by using the API \texttt{\small django.utils.html.escape}.

\lstdefinestyle{lst}{
    float=th,
    floatplacement=tbp,
    % abovecaptionskip=0.01in,
    numbers=left, 
    numberstyle=\scriptsize, 
    numbersep = 5pt,
    framexleftmargin = 0in,
    framexrightmargin = 0in,
    breaklines = true,
    xleftmargin = 0.18in,
    xrightmargin = 0.1in,
    basicstyle=\ttfamily\scriptsize, 
    frame=lines,
    showtabs=true,
    showspaces=true,
    showstringspaces=false,
    literate={\ }{{\ }}1,
    aboveskip=+0.00in,
    belowskip=-0.15in,
}

\begin{lstlisting}[
language=diff, 
style=lst,
caption=An example of security commit that fixes an XSS vulnerability (CVE-2022-24710).,
label={lst:xss2},
mathescape=true
]
 $\textbf{commit f6753a1a1c63fade6ad418fbda827c6750ab0bda }$
 diff --git a/weblate/trans/forms.py b/weblate/trans/forms.py
 @@ -37,6 +37,7 @@
 ...
+from django.utils.html import escape
 ...
-    label = str(unit.translation.language)
+    label = escape(unit.translation.language)
 ...
\end{lstlisting}


\noindent{\bf Shell Commands.} To handle the shell commands securely, security fixes can adopt \texttt{\small shlex.quote} and \texttt{\small subprocess} to load or execute the commands. 
With the \texttt{\small shlex.quote} API, we can have an escaped version of shell inputs, which can be safely used as tokens in a command line to avoid shell command injection.
List~\ref{lst:shell} is an example that shows the usage of \texttt{\small shlex.quote} to fix a shell injection vulnerability. 

\lstdefinestyle{lst}{
    float=th,
    floatplacement=tbp,
    % abovecaptionskip=0.01in,
    numbers=left, 
    numberstyle=\scriptsize, 
    numbersep = 5pt,
    framexleftmargin = 0in,
    framexrightmargin = 0in,
    breaklines = true,
    xleftmargin = 0.18in,
    xrightmargin = 0.1in,
    basicstyle=\ttfamily\scriptsize, 
    frame=lines,
    showtabs=true,
    showspaces=true,
    showstringspaces=false,
    literate={\ }{{\ }}1,
    aboveskip=-0.00in,
    belowskip=-0.15in,
}

\begin{lstlisting}[
language=diff, 
style=lst,
caption=An example of security commit that fixes a shell injection vulnerability (CVE-2013-7416).,
label={lst:shell},
mathescape=true
]
 $\textbf{commit 2817869f98c54975f31e2dd674c1aefa70749cca }$
 diff --git a/canto_curses/guibase.py b/canto_curses/guibase.py
 @@ -156,6 +156,11 @@ def _fork(self, path, href, text, fetch=False):
 ...
+    href = shlex.quote(href)
 ...
\end{lstlisting}


\noindent{\bf Path Name.} 
If a path name is improperly neutralized, attackers may access the files and directories outside of the restricted location. 
This vulnerability can occur by using absolute file paths or manipulating the path variables where the reference files contain ``dot-dot-slash (../)" sequences or variations.
To effectively escape such unsafe sequences, Python security commits usually adopt the secure APIs, e.g., \texttt{\small werkzeug.utils.safe\_join}, \texttt{\small yaml.safe\_load}, and \texttt{\small werkzeug.utils.secure\_filename}, to prevent the files or directories from being accessed by malicious users. 
List~\ref{lst:path traversal} is a commit example that fixes a path traversal via using the API \texttt{\small werkzeug.utils.secure\_filename}.

\lstdefinestyle{lst}{
    float=th,
    floatplacement=tbp,
    % abovecaptionskip=0.01in,
    numbers=left, 
    numberstyle=\scriptsize, 
    numbersep = 5pt,
    framexleftmargin = 0in,
    framexrightmargin = 0in,
    breaklines = true,
    xleftmargin = 0.18in,
    xrightmargin = 0.1in,
    basicstyle=\ttfamily\scriptsize, 
    frame=lines,
    showtabs=true,
    showspaces=true,
    showstringspaces=false,
    literate={\ }{{\ }}1,
    aboveskip=+0.10in,
    belowskip=-0.25in,
}

\begin{lstlisting}[
language=diff, 
style=lst,
caption=An example of security commit that fixes a path traversal vulnerability (CVE-2022-23609).,
label={lst:path traversal},
mathescape=true
]
 $\textbf{commit 1eb1e5428f0926b2829a0bbbb65b0d946e608593}$ 
 diff --git a/upload/server.py b/upload/server.py
 @@ -5,7 +5,7 @@
-
+import werkzeug.utils
 @@ -189,7 +189,7 @@ def uploadimage():
     filename = all_files[0][1] + all_files[0][2]
-    remove(filename)
+    remove(werkzeug.utils.secure_filename(filename))
     $\textbf{del}$ all_files[0]
     length = len(all_files)
\end{lstlisting}


\subsubsection{Update Regular Expressions}
Python has become a popular choice for back-end web development, and it is usually combined with some other front-end languages~\cite{python_app}. For this reason, we observe there are 15.02\% fixes that modify the regular expressions to avoid XSS, SQL injection, and open redirect vulnerabilities. 
% Python inserts itself in web development as a back-end language, and it is usually combined with some other front-end language (e.g., javascript) to build a whole website.
% We observe 15.02\% 
The regular expression patterns are tailored to match specific strings within the given text, including SQL commands, URLs, and other scripts.

\noindent{\bf SQL Commands.} The improper neutralization of SQL commands may lead to SQL injection vulnerabilities, which allow attackers to manipulate the backend database and access the information not intended to be displayed.
The corresponding fixes need to escape the unsafe characters. 
List~\ref{lst:sql} is a fixed example of SQL injection vulnerability, which substitutes the matched single and double quote characters (i.e., \texttt{\small '} and \texttt{\small "}) in the string \texttt{\small self.queueid}.

\lstdefinestyle{lst}{
    float=th,
    floatplacement=tbp,
    % abovecaptionskip=0.01in,
    numbers=left, 
    numberstyle=\scriptsize, 
    numbersep = 5pt,
    framexleftmargin = 0in,
    framexrightmargin = 0in,
    breaklines = true,
    xleftmargin = 0.18in,
    xrightmargin = 0.1in,
    basicstyle=\ttfamily\scriptsize, 
    frame=lines,
    showtabs=true,
    showspaces=true,
    showstringspaces=false,
    literate={\ }{{\ }}1,
    aboveskip=+0.0in,
    belowskip=-0.15in,
}

\begin{lstlisting}[
language=diff, 
style=lst,
caption=An example of security commit that fixes a SQL injection vulnerability (CVE-2014-125082).,
label={lst:sql},
mathescape=true
]
 $\textbf{commit fc2c1ea1b8d795094abb15ac73cab90830534e04}$
 diff --git a/.../model.py b/.../model.py
 @@ -772,13 +772,13 @@ def _get_filter(self):
 $\textbf{if}$ self.queueid:
-    ... = '%s'" % (self.queueid)
+    ... = '%s'" % (re.sub("[\"']", "", self.queueid))
\end{lstlisting}


\noindent{\bf URLs.} The improper neutralization of URLs may lead to open redirect vulnerability, which redirects an unsuspecting victim from a legitimate domain to an attacker’s phishing site. 
Effective mitigation is to replace the dangerous special characters with trusted symbols. List~\ref{lst:redirect} is an example of an open redirect vulnerability, which replaces the explicit backslash with an encoded backslash to circumvent the dangerous redirect.

\lstdefinestyle{lst}{
    float=th,
    floatplacement=tbp,
    % abovecaptionskip=0.01in,
    numbers=left, 
    numberstyle=\scriptsize, 
    numbersep = 5pt,
    framexleftmargin = 0in,
    framexrightmargin = 0in,
    breaklines = true,
    xleftmargin = 0.18in,
    xrightmargin = 0.1in,
    basicstyle=\ttfamily\scriptsize, 
    frame=lines,
    showtabs=true,
    showspaces=true,
    showstringspaces=false,
    literate={\ }{{\ }}1,
    aboveskip=-0.00in,
    belowskip=-0.15in,
}

\begin{lstlisting}[
language=diff, 
style=lst,
caption=An example of security commit that fixes an open redirect vulnerability (CVE-2019-10255).,
label={lst:redirect},
mathescape=true
]
 $\textbf{commit 08c4c898182edbe97aadef1815cce50448f975cb}$ 
 diff --git a/auth/login.py b/auth/login.py
 @@ -39,6 +39,10 @@ def _redirect_safe(self, url, ...):
+    url = url.replace("\\", "%5C")
     parsed = urlparse(url)
     $\textbf{if}$ parsed.netloc $\textbf{or not}$ (parsed.path + '/').startswith(self.base_url):
\end{lstlisting}

\noindent{\bf Scripts.} The improper input validation and encoding during web page generation may lead to XSS, which is able to reveal the cookies, session tokens, or other sensitive information retained by the browser to the attackers. A straightforward solution is to validate the matched characters of a pre-defined pattern. List~\ref{lst:xss} is an example to fix the XSS vulnerability by re-matching the characters between parentheses instead of the characters between square brackets and validating the matched pattern one by one.

\lstdefinestyle{lst}{
    float=th,
    floatplacement=tbp,
    % abovecaptionskip=0.01in,
    numbers=left, 
    numberstyle=\scriptsize, 
    numbersep = 5pt,
    framexleftmargin = 0in,
    framexrightmargin = 0in,
    breaklines = true,
    xleftmargin = 0.18in,
    xrightmargin = 0.1in,
    basicstyle=\ttfamily\scriptsize, 
    frame=lines,
    showtabs=true,
    showspaces=true,
    showstringspaces=false,
    literate={\ }{{\ }}1,
    aboveskip=+0.10in,
    belowskip=-0.25in,
}

\begin{lstlisting}[
language=diff, 
style=lst,
caption=An example of security commit that fixes an XSS vulnerability (CVE-2021-3994).,
label={lst:xss},
mathescape=true
]
 $\textbf{commit a22eb0673fe0b7784f99c6b5fd343b64a6700f06}$ 
 diff --git a/helpdesk/models.py b/helpdesk/models.py
 @@ -238 +238 @@ def cvesForCPE(cpe,
     $\textbf{if not}$ text:
         $\textbf{return}$ ""
-    pattern = fr'([\[\s\S\]]*?)\(([\s\S]*?):([\[\s\S\]]*?)\)'
+    pattern = fr'([\[\s\S\]]*?)\(([\s\S]*?):([\s\S]*?)\)'
     # Regex check
     $\textbf{if}$ re.match(pattern, text):
         # get get value of group regex
\end{lstlisting}



\subsubsection{Restrict Security Properties} 
The exploits often result from improper settings of security properties. 
14.55\% security commits in \db{} fix improper settings by updating boolean flags from \texttt{\small True} to \texttt{\small False} or vice versa, adding more arguments to methods, or adding security decorators.


\noindent{\bf Update Security Flags.} 
Security flags perform restrictions on the methods that may have access to sensitive objects. 
Improper restrictions on such flags may expose users to a risky environment and/or lead to sensitive information leakage. 
List~\ref{lst:flag} changes the flag from \texttt{\small False} to \texttt{\small True} to fix a vulnerability, where a sensitive cookie does not have a `HttpOnly' flag.


\lstdefinestyle{lst}{
    float=th,
    floatplacement=tbp,
    % abovecaptionskip=0.01in,
    numbers=left, 
    numberstyle=\scriptsize, 
    numbersep = 5pt,
    framexleftmargin = 0in,
    framexrightmargin = 0in,
    breaklines = true,
    xleftmargin = 0.18in,
    xrightmargin = 0.1in,
    basicstyle=\ttfamily\scriptsize, 
    frame=lines,
    showtabs=true,
    showspaces=true,
    showstringspaces=false,
    literate={\ }{{\ }}1,
    aboveskip=-0.00in,
    belowskip=-0.15in,
}

\begin{lstlisting}[
language=diff, 
style=lst,
caption=An example of security commit that fixes a vulnerability where the sensitive cookie does not have a `HttpOnly' flag (CVE-2019-25091).,
label={lst:flag},
mathescape=true
]
 $\textbf{commit 60a3fe559c453bc36b0ec3e5dd39c1303640a59a}$ 
 diff --git a/src/nsupdate/settings/base.py b/src/nsupdate/settings/base.py
 @@ -283,7 +283,7 @@
 ...
-CSRF_COOKIE_HTTPONLY = False
+CSRF_COOKIE_HTTPONLY = True
 ...
\end{lstlisting}

\noindent{\bf Add Restriction Arguments.} Some restriction arguments will be passed to the functions during execution. Improper argument settings may lead to a variety of mishandling. As shown in List~\ref{lst:arg}, the \texttt{\small formaction} is added to restrict the attributes of a variable to avoid XSS vulnerability.


\lstdefinestyle{lst}{
    float=th,
    floatplacement=tbp,
    % abovecaptionskip=0.01in,
    numbers=left, 
    numberstyle=\scriptsize, 
    numbersep = 5pt,
    framexleftmargin = 0in,
    framexrightmargin = 0in,
    breaklines = true,
    xleftmargin = 0.18in,
    xrightmargin = 0.1in,
    basicstyle=\ttfamily\scriptsize, 
    frame=lines,
    showtabs=true,
    showspaces=true,
    showstringspaces=false,
    literate={\ }{{\ }}1,
    aboveskip=-0.00in,
    belowskip=-0.15in,
}

\begin{lstlisting}[
language=diff, 
style=lst,
caption=An example of security commit that fixes a cross-site-scripting (XSS) vulnerability (CVE-2021-28957).,
label={lst:arg},
mathescape=true
]
 $\textbf{commit 10ec1b4e9f93713513a3264ed6158af22492f270}$ 
 diff --git a/src/lxml/html/defs.py b/src/lxml/html/defs.py
 @@ -23,6 +23,8 @@
 ...
+    # HTML5 formaction
+    'formaction'
     ])
 ...
\end{lstlisting}

\noindent{\bf Add Security Decorators.} A decorator is a function that takes another function and extends the behavior of the function without explicit modification. This mechanism has been widely adopted by security commits to add more detailed security restrictions on existing methods. List~\ref{lst:access control2} shows a security commit that fixes an access control vulnerability by adding decorator \texttt{\small security.private} to function \texttt{\small enumerateRoles}.

\lstdefinestyle{lst}{
    float=th,
    floatplacement=tbp,
    % abovecaptionskip=0.01in,
    numbers=left, 
    numberstyle=\scriptsize, 
    numbersep = 5pt,
    framexleftmargin = 0in,
    framexrightmargin = 0in,
    breaklines = true,
    xleftmargin = 0.18in,
    xrightmargin = 0.1in,
    basicstyle=\ttfamily\scriptsize, 
    frame=lines,
    showtabs=true,
    showspaces=true,
    showstringspaces=false,
    literate={\ }{{\ }}1,
    aboveskip=+0.10in,
    belowskip=-0.25in,
}

\begin{lstlisting}[
language=diff, 
style=lst,
caption=An example of security commit that fixes an access control vulnerability (CVE-2021-21336).,
label={lst:access control2},
mathescape=true
]
 $\textbf{commit 2dad81128250cb2e5d950cddc9d3c0314a80b4bb}$ 
 diff --git a/src/Products/plugins/ZODBRoleManager.py b/src/Products/plugins/ZODBRoleManager.py
 @@ -112,6 +112,7 @@ def getRolesForPrincipal(self, principal, request=None):
     #   IRoleEnumerationPlugin implementation
+    @security.private
     $\textbf{def}$ enumerateRoles(self, id=None, exact_match=False, sort_by=None, max_results=None, **kw):
         """ See IRoleEnumerationPlugin.
\end{lstlisting}



\subsection{Unique Patterns Captured from the Wild (RQ4)}\label{exp:variance}

Recall that we construct pilot and augmented datasets because the base dataset provides a limited number of security commits samples. Here, we further show the examples captured by our security commit collection approaches that introduce more variety in syntax and semantics of security-related code changes, enabling wider applications of \db{} in solving real-world Python-related security issues.

\subsubsection{Data Variety Introduced by Pilot Dataset}
We study the contribution of involving the pilot dataset for \gnn{} by comparing the model trained only on the base dataset and the model trained on the combination of the base and pilot datasets.
% first training on the base dataset and then training on the combination of the base dataset and the pilot dataset.
%Then, we analyze the samples that have not been identified by the first model but have been identified by the second model. 
We find that the pilot dataset helps the latter model to be able to identify more wild security commits. For instance, the latter \gnn{} can detect more subtle changes. % after expanding the training set with the pilot dataset. 
In List~\ref{lst:pilot}, the \texttt{\small '\%s'} has been changed to \texttt{\small ?} in a SQL query, protecting the database from being injected. 
% the condition refines the value of \texttt{\small GITHUB\_ACTIONS} from not null to true, protecting the authentication from being bypassed. 
The capability of detecting such minor changes is enabled by similar samples in the pilot dataset but not existed in the base dataset.

\lstdefinestyle{lst}{
    float=th,
    floatplacement=tbp,
    % abovecaptionskip=0.01in,
    numbers=left, 
    numberstyle=\scriptsize, 
    numbersep = 5pt,
    framexleftmargin = 0in,
    framexrightmargin = 0in,
    breaklines = true,
    xleftmargin = 0.18in,
    xrightmargin = 0.1in,
    basicstyle=\ttfamily\scriptsize, 
    frame=lines,
    showtabs=true,
    showspaces=true,
    showstringspaces=false,
    literate={\ }{{\ }}1,
    aboveskip=-0.00in,
    belowskip=-0.15in,
}

\begin{lstlisting}[
language=diff, 
style=lst,
caption=An example of security commit detected by \gnn{} trained on the base and pilot datasets.,
label={lst:pilot},
mathescape=true
]
 $\textbf{commit 9d8adbc07c384ba51c2583ce0819c9abb77dc648}$ 
 diff --git .../__init__.py .../__init__.py
 @@ -71,7 +71,7 @@ def klauen(self,
-    a = u"name == '%s' AND item =='%s'" % (name, item)
+    a = u"name == ? AND item ==?", (name, item)
\end{lstlisting}
%  $\textbf{commit c658b4f3e57258acf5f6207a90c2f2169698ae22}$ 
%  diff --git a/core.py b/core.py
%  @@ -112,7 +112,7 @@ def actualsys() :
% -    if os.environ.get('GITHUB_ACTIONS') != "":
% +    if os.environ.get('GITHUB_ACTIONS') == "true":
%          logging.warning("Running on Github Actions")
%          actualsys()

\subsubsection{Variance Introduced by Augmented Dataset}
We further evaluate to show that our augmented dataset can help train a model that is able to identify more various security commits from the wild. For example, after introducing augmented dataset into the training phase, the model detects a new escape pattern. As shown in List~\ref{lst:augmented}, the characters \texttt{\small <}, \texttt{\small >}, and \texttt{\small \&} have been escaped by being translated into Unicode, which prevents cross-site-scripting crafted with a partial JSON-serializable object. Compared with the escape expressions in Section~\ref{rq3} that only include ASCII characters, the augmented dataset help \gnn{} generalize the escapes to Unicode.


%46e95f5

\lstdefinestyle{lst}{
    float=th,
    floatplacement=tbp,
    % abovecaptionskip=0.01in,
    numbers=left, 
    numberstyle=\scriptsize, 
    numbersep = 5pt,
    framexleftmargin = 0in,
    framexrightmargin = 0in,
    breaklines = true,
    xleftmargin = 0.18in,
    xrightmargin = 0.1in,
    basicstyle=\ttfamily\scriptsize, 
    frame=lines,
    showtabs=true,
    showspaces=true,
    showstringspaces=false,
    literate={\ }{{\ }}1,
    aboveskip=-0.00in,
    belowskip=-0.15in,
}

\begin{lstlisting}[
language=diff, 
style=lst,
caption=A security commit example detected by the \gnn{} trained on the base{,} pilot{,} and augmented datasets.,
label={lst:augmented},
mathescape=true
]
 $\textbf{commit d3e428a6f7bc4c04d100b06e663c071fdc1717d9}$ 
 diff --git a/.../djblets_js.py b/.../djblets_js.py 
 @@ -28,11 +28,18 @@
+_safe_js_escapes = {
+    ord('&'): u'\\u0026',
+    ord('<'): u'\\u003C',
+    ord('>'): u'\\u003E',
+}
\end{lstlisting}