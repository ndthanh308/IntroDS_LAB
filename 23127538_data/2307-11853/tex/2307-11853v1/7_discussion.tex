\vspace{-0.03in}
\section{Discussion}
\label{sec:discussion}

\noindent{\bf Usability.} The \gnn{} is versatile and not tied to any specific platform or commit format, making it compatible with a wide range of version control systems like GitHub and GitLab. By integrating the \gnn{} as an extension, contributors can easily add vulnerability fix labels to their commits, streamlining the code auditing process and reducing the workload on developers. In addition, downstream developers and users who use third-party libraries can benefit from the \gnn{} by being reminded to make necessary security fixes on time. %Users will also be more aware of the importance of patching vulnerabilities, which can reduce the risk of exploitation. 
Finally, researchers can obtain labeled commits without requiring extensive manual labor for future data-driven vulnerability and patch-related research.

%\gnn{} is platform-independent and commit format independent. Thus, it can be integrated into any version control platform, such as GitHub and GitLab. With the \gnn{} as an extension, the contributor will be assisted in attaching a vulnerability fix label to the commit to be pushed, which will release pressure on developers and speed up the code auditing process. Moreover, \gnn{} will also benefit downstream developers who utilize third-party libraries to pinpoint security commits in the upstream dependencies by reminding them to make necessary security fixes in time. Besides, users can be aware of the importance and urgency of patching the vulnerabilities to reduce the time window of being exploited. Meanwhile, researchers could acquire such labeled commits without exhaustive labor work. 

\noindent{\bf Ethical Consideration.} The \gnn{} has the potential to identify undisclosed vulnerability fixes, but this presents a mixed blessing as attackers could exploit this information to target unpatched systems. Our objective in this paper is to prioritize the security of the users' systems; that is why we only share detailed information on the security fixes, rather than the vulnerabilities. By taking this approach, attackers cannot leverage the \gnn{} to gain additional details on the vulnerabilities. However, with the \gnn{}, open-source software maintainers can quickly reveal vulnerabilities as soon as security fixes become public, improving the overall security of their software systems.

%\gnn{} could help pinpoint silent vulnerability fixing before they are publicly disclosed. However, this is a mixed blessing, since attackers may leverage the silent information to exploit the delay-patched vulnerability. In this paper, our primary goal is to enhance the security of the user system, thus we only release detailed information about the silent patches, not the vulnerability itself. Under such conditions, attackers cannot be beneficial from \gnn{} to obtain additional vulnerability details. On the other side, with the assistance of \gnn{}, OSS maintainers can promptly disclose vulnerabilities once the corresponding fixes are publicly available. 

%\gnn{} could help pinpoint silent vulnerability fixes before these fixes are disclosed. Our primary goal is to protect users' systems from being exploited, and we will only release the details of its patch without publishing the vulnerabilities detail. Thus, the attackers cannot benefit from \gnn{} to exploit the target user's system. Meanwhile, we recommend OSS maintainers promptly disclose vulnerabilities once the corresponding fixes are publicly available. 

% \subsection{Qualitative Analysis}

% % ignore code features in dataset generation

% for example, the security commits are smaller than non-security ones due to ..



% In which cases the samples will have a higher probability to be security commits