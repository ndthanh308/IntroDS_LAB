\documentclass[a4paper,11pt]{ltdartcl_dfg}
\makeatletter

\usepackage{amsmath}
\usepackage{mathtools}
\usepackage{amsthm}
\usepackage{amsfonts}  
\usepackage{amssymb}
\DeclareOldFontCommand{\rm}{\normalfont\rmfamily}{\mathrm}
\DeclareOldFontCommand{\sf}{\normalfont\sffamily}{\mathsf}
\DeclareOldFontCommand{\tt}{\normalfont\ttfamily}{\mathtt}
%\DeclareOldFontCommand{\bf}{\normalfont\bfhttps://www.overleaf.com/project/62022d046a9b74af4550fc2aseries}{\mathbf}
\DeclareOldFontCommand{\it}{\normalfont\itshape}{\mathit}
\DeclareOldFontCommand{\sl}{\normalfont\slshape}{\@nomath\sl}
\DeclareOldFontCommand{\sc}{\normalfont\scshape}{\@nomath\sc}
\makeatother
\usepackage{leftidx}
\usepackage{psfrag}  
\usepackage[utf8]{inputenc}
\usepackage{bm}
\usepackage{ulem}
\usepackage{graphicx}
\usepackage{etaremune}
\usepackage{graphicx}
\usepackage{booktabs}
\usepackage{adjustbox}
\usepackage[numbib,notlof,notlot,nottoc]{tocbibind}
\usepackage{hyperref}
\usepackage{todonotes}
\usepackage{filecontents}
%\usepackage[pagewise,mathlines]{lineno}\linenumbers

\allowdisplaybreaks

\let\textquotedbl="
%%%%%%%%%%%%%%%%%%%%
\usepackage{url, eurosym}

\usepackage{epstopdf}

\usepackage{tikz} % for graphics
\tikzstyle{every pin}=[%fill=white,
					   %draw=black,
					   pin edge = black,					   
					   font=\footnotesize]
\tikzstyle{block} = [draw, fill=blue!20, rectangle, minimum height=3em, minimum width=6em]
\tikzstyle{sum} = [draw, fill=blue!20, circle, node distance=1cm]
\tikzstyle{input} = [coordinate]
\tikzstyle{output} = [coordinate]
\tikzstyle{pinstyle} = [pin edge={to-,thin,black}]

%\usepackage{etoolbox} %% <- for \cspreto, \csappto, \patchcmd, \pretocmd, \apptocmd

%% Patch 'normal' math environments:
%\newcommand*\linenomathpatch[1]{%
%  \cspreto{#1}{\linenomath}%
%  \cspreto{#1*}{\linenomath}%
%  \csappto{end#1}{\endlinenomath}%
%  \csappto{end#1*}{\endlinenomath}%
%}
%% Patch AMS math environments:
%\newcommand*\linenomathpatchAMS[1]{%
%  \cspreto{#1}{\linenomathAMS}%
%  \cspreto{#1*}{\linenomathAMS}%
%  \csappto{end#1}{\endlinenomath}%
%  \csappto{end#1*}{\endlinenomath}%
%}

%% Definition of \linenomathAMS depends on whether the mathlines option %is provided
%\expandafter\ifx\linenomath\linenomathWithnumbers
%  \let\linenomathAMS\linenomathWithnumbers
%  %% The following line gets rid of an extra line numbers at the bottom:
%  \patchcmd\linenomathAMS{\advance\postdisplaypenalty\linenopenalty}{}{}{}
%\else
%  \let\linenomathAMS\linenomathNonumbers
%\fi

%\linenomathpatch{equation}
%\linenomathpatchAMS{gather}
%\linenomathpatchAMS{multline}
%\linenomathpatchAMS{align}
%\linenomathpatchAMS{alignat}
%\linenomathpatchAMS{flalign}

\usepackage{pgffor} 
	\usetikzlibrary{arrows.meta}% Pfeile in TikZ
	\usetikzlibrary{positioning}% Pfeile in TikZ
	\usetikzlibrary{bending}% Pfeile in TikZ
	\usetikzlibrary{shadows}% Schatten in TikZ
	\usetikzlibrary{shadows.blur}% Schatten in TikZ
	\usetikzlibrary{scopes} %for TikZ Scopes
	\usetikzlibrary{patterns} % fill patterns 
	\usetikzlibrary{fadings} % fading effects
	\usetikzlibrary{decorations.pathmorphing} %Freihandlinien 
	\usetikzlibrary{calc} %Vektorrechnung
	\usetikzlibrary{datavisualization} %Data Visualization
	\usetikzlibrary{datavisualization.formats.functions} %Function evaluation within Visualizations
	\usetikzlibrary{intersections} %calc intersections of paths	
	\usetikzlibrary{plotmarks}
	\usetikzlibrary{matrix} % for block alignment
	\usetikzlibrary{arrows} % for arrow heads
	\usetikzlibrary{calc} % for manimulation of coordinates
	\usetikzlibrary{shapes,arrows}
	\usetikzlibrary{angles,quotes}
	\usetikzlibrary{cd}


\usepackage{pgfplots}
\pgfplotsset{compat=newest}
%% the following commands are needed for some matlab2tikz features
\usetikzlibrary{plotmarks}
\usetikzlibrary{arrows.meta}
\usepgfplotslibrary{patchplots}
\usepackage{grffile}
\usepackage{setspace}

%% you may also want the following commands
\pgfplotsset{plot coordinates/math parser=false}
\newlength\figureheight
\newlength\figurewidth

\newcommand{\tff }{\bfseries \sffamily {\upshape }}
\ltdsetup{\today}{}{}


\fancyhead[L]{\footnotesize {\sffamily {\upshape \caltechgray{Symplectic discretization for optimal control problems in mechanics}}}}
\fancyfoot[R]{\footnotesize {\sffamily {\upshape \caltechgray{\thepage}}}}
% initial header and footer
\fancyhead[C]{}
\fancyhead[R]{}

\newcommand\R{\mathbb{R}}
%%%%%%%%%%%%%%%%%%%%%%%%%%%%%%%%%%%%%%%%%%%%%%
%%%%%%%%%%%%%%%%%%%%%%%%%%%%%%%%%%%%%%%%%%%%%%
\usepackage{caption} % needed for subcaption
\usepackage{subcaption} % side-by-side figures
\usepackage{tabularx} % whole page width spanning tables
\usepackage[font=small]{caption}
\newcommand{\Cpp}{C{\ttfamily ++}} % nice C++ symbol
\captionsetup{subrefformat=parens}

\usepackage[utf8]{inputenc}
\usepackage{paralist}
%\usepackage{refcheck}
\usepackage{sectsty}

\sectionfont{\fontsize{12}{15}\selectfont}

\renewcommand\refname{Bibliography concerning the state of the art, the research objectives, and the work program}
%%%%%%%%%%%%%%%%%%%%%%%%%%%%%%%%%%%%%%%%%%%%%%%%%%%%%%%%%%%
%%%%%%%%%%%%%%%%%%%%%%%%%%%%%%%%%%%%%%%%%%%%%%%%%%%%%%%%%%%

\newtheorem{theorem}{Theorem}[section]
\newtheorem{definition}[theorem]{Definition}
\newtheorem{lemma}[theorem]{Lemma}
\newtheorem{proposition}[theorem]{Proposition}
\newtheorem{example}[theorem]{Example}
\newtheorem{remark}[theorem]{Remark}
\newtheorem{corollary}[theorem]{Corollary}


\title{A new Lagrangian approach to control affine systems with a quadratic
Lagrange term}
\date{\today}

\author
{Sigrid Leyendecker\footnote{Friedrich-Alexander-Universität Erlangen-Nürnberg (FAU), Institute of Applied Dynamics (LTD), Immerwahrstrasse 1, 91058 Erlangen, Germany. Email: \href{mailto:sigrid.leyendecker@fau.de}{sigrid.leyendecker@fau.de}}\ \thanks{The work of this author has been supported by Deutsche Forschungsgemeinschaft (DFG), Grant No. LE 1841/12-1, AOBJ: 692092.}
\qquad 
 Sofya Maslovskaya\footnote{\textit{First author}. Universität Paderborn (UPB), Numerical Mathematics and Control (NMC), Warburger Straße 100, 33098 Paderborn, Germany. Email: \href{mailto:sofya.maslovskaya@upb.de}{sofya.maslovskaya@upb.de}}\
%\thanks{The work of this author has been supported ....?}
\\
Sina Ober-Bl\"obaum\footnote{  Universität Paderborn (UPB), Numerical Mathematics and Control (NMC), Warburger Straße 100, 33098 Paderborn, Germany. Email: \href{mailto:sinaober@math.uni-paderborn.de}{sinaober@math.uni-paderborn.de}}\
\thanks{The work of this author has been supported by Deutsche Forschungsgemeinschaft (DFG), Grant No. OB 368/5-1, AOBJ: 692093}
\qquad
Rodrigo T. Sato Mart{\'\i}n de Almagro\footnote{Friedrich-Alexander-Universität Erlangen-Nürnberg (FAU), Institute of Applied Dynamics (LTD), Immerwahrstrasse 1, 91058 Erlangen, Germany. Email: \href{mailto:rodrigo.t.sato@fau.de}{rodrigo.t.sato@fau.de}}\
%\thanks{The work of this author has been supported ....?}
 \\
Flóra Orsolya Szemenyei\footnote{\textit{First author}, \textit{corresponding author}. Friedrich-Alexander-Universität Erlangen-Nürnberg (FAU), Institute of Applied Dynamics (LTD), Immerwahrstrasse 1, 91058 Erlangen, Germany. Email: \href{mailto:flora.szemenyei@fau.de}{flora.szemenyei@fau.de}}\
%\thanks{The work of this author has been supported ....?}
}
 


\begin{document}
\maketitle

\footnotetext{{\textit{ Math Subject Classifications. Primary:}} 65K10, 49M25.   {\textit{ Secondary:}} 65K15. }

\footnotetext{\textit{Keywords and Phrases.} Optimal control problem, Lagrangian system, Hamiltonian system, Variations, Pontryagin's maximum principle.}



\section*{Abstract}
In this work, we consider optimal control problems for mechanical systems {on vector spaces} with fixed initial and free final
state and a quadratic Lagrange term. Specifically, the dynamics is described by a second order ODE
containing an affine control term {and we allow linear coordinate changes in the configuration space}. Classically, Pontryagin’s maximum principle gives necessary optimality
conditions for the optimal control problem. For smooth problems, alternatively, a variational approach
based on an augmented objective can be followed. Here, we propose a new Lagrangian approach leading to
equivalent necessary optimality conditions in the form of Euler-Lagrange equations. Thus, the differential
geometric structure (similar to classical Lagrangian dynamics) can be exploited in the framework of
optimal control problems. In particular, the formulation enables the symplectic discretisation of the
optimal control problem via variational integrators in a straightforward way.



\section{Introduction}
The optimal control of mechanical problems is omnipresent in our technically affected daily living as well as in many scientific questions. {These problems have a rather rich geometric structure. The underlying uncontrolled system frequently lives on a manifold $\mathcal{M}$ that admits a natural symplectic structure, such is the case of Hamiltonian or regular Lagrangian mechanical systems. Moreover, the associated optimal control problem evolves on $T^* \mathcal{M}$, which always admits a symplectic structure. This hierarchy of structures is even more critical in the fully-actuated problem, where one naturally arrives at higher-order mechanical problems \cite{deLeonRodrigues85}, \cite{Colombo2016}}\textcolor{black}{, \cite{treanta14}}.\\
{The symplectic structure of optimal control problems also plays a major role in analysing numerical methods for the approximation of solutions.}
%As analytical solutions of optimal control problems are in general not available,  applications rely on numerical simulations that are robust and accurate, and directly utilizable, \textcolor{magenta}{e.g. \cite{Mehrmann2000, grune2008, deleon2007}}.
{In principle, numerical solution methods for optimal control problems can be classified into direct and indirect methods (see \cite{formalskii10}, \textcolor{black}{\cite{betts2010}}).}
%\it{I do not know this reference, but please also cite Betts: J.T. Betts, Practical methods for optimal control and estimation using nonlinear programming,
%Second edition, Advances in Design and Control, 19, SIAM, Philadelphia, PA, 2010.})}
%
%One can classify direct and indirect numerical methods which can be used to obtain an approximate solution, \textcolor{magenta}{see \cite{formalskii10}}. 
The main difference between the two approaches is the order in which the discretization and the optimisation steps take place. 
{The indirect approach (first optimise, then discretise) provides necessary optimality conditions given by the adjoint differential equation whereas the direct approach (first discretise, then optimise) yields a discrete version of the adjoint differential equation through the derivation of Karush-Kuhn-Tucker equations (\textcolor{black}{\cite{betts2010,gerdts2003}}). The relation between direct and indirect approaches is given by symplectic methods, i.e.~the discrete state and adjoint systems derived by the direct approach is a symplectic discretisation of the continuous state and adjoint system derived in the indirect approach. First works analysing the relationship of direct and indirect approaches for Runge-Kutta methods are e.g.~\cite{hager00}, \cite{bonnans04} and \cite{sanz-serna2015} (see also references therein). Starting directly with a symplectic method in the direct approach for the optimal control of mechanical systems provides a double symplectic scheme (symplectic in the state and symplectic in the state-adjoint equations). This was proven for a particular class of symplectic methods (\cite{ober-blobaum2008, campos15}) by exploiting the hierachy of symplectic structures mentioned above.}   

{The symplectic nature of optimal control problems has motivated many different works over the last years not only with respect to the relation between direct and indirect approaches. Further topics of investigation are e.g.~the geometric interpretation of adjoint systems, the formulation of concise Lagrangians and corresponding variational principles for optimal control problems based on e.g.~higher order Lagrangians or generating functions and associated consistent symplectic discretisation schemes for optimal control problems (see \cite{deleon2007, leok_tran22}).}

  
%With indirect methods, where one first optimises then discretises, often more accurate approximations \sofya{of the control} can be reached than with direct methods, where one first discretises then optimises. However, a priori knowledge on the structure of the optimal solution is required \sofya{in the indirect method in case of state and control constrains}. 
%On the other hand, direct methods are generally more intuitive and robust in their implementation, which make them preferable in many engineering applications, \textcolor{magenta}{see \cite{m.campos2015, ober-blobaum2008, leyendecker2010}}.
%For general \sina{direct} discretization methods, the resulting approximation to the state and adjoint equations are different.
%\sina{Results to date suggest that Runge-Kutta discretizations of the state equations lead to symplectic methods for the approximation of the combined state and adjoint equations, see \cite{hager00}, \cite{bonnans04} and \cite{sanz-serna2015}.}
%certain symplectic methods can produce the same approqimation for both equations and 
%and thus, lead to commutation between the discretization and the optimization step. 
%\sina{(cite Hager, Bonnans and Sanz Serna)}
%Thus, symplectic methods provide a link between direct and indirect approaches. %Furthermore, considering optimal control problems for mechanical systems which can be described by Lagrangian or Hamiltonian dynamics, symplectic discretizations can be applied on two levels -- first on the level of Lagrangian or Hamiltonian state dynamics and second, on the level of optimal control problem including the adjoint dynamics.
%\sina{For specific symplectic methods it was shown that this leads to the same approximations for state and adjoint dynamics, see \cite{ober-blobaum2008, campos15}.}\\
%Many works have been done in the direction \textcolor{magenta}{of variational principle and calculus of variation regarding optimal control problems}, e.g. \cite{deleon2007, leok_tran22}, where the authors investigate geometric properties and methods for adjoint systems.
%\textcolor{magenta}{However, our new contribution to this research field is that we provide a systematic approach based on a new Lagrangian
%formulation and a variational principle involving both, state and adjoint variables of the optimal control problem.}
{As a new contribution to this research field, we provide a systematic approach based on a new Lagrangian
formulation and a variational principle involving both, state and adjoint variables of the optimal control problem.} {More concretely, for a specific class of mechanical optimal control problems (namely control affine systems with a quadratic Lagrange term on vector spaces),} {we show that the Euler-Lagrange equations of the new Lagrangian provide the classical necessary optimality conditions. Since the Lagrangian is regular, a Hamiltonian can be easily derived based on the Legendre transformation. Furthermore, we investigate invariances of the Lagrangian and associated symmetries in the optimal control problem leading to conserved quantities by Noether's theorem.}



\section{Preliminaries, notations}

{In this chapter we introduce basic concepts of Hamiltonian and Lagrangian mechanics, optimal control problems and different approaches to derive necessary optimality conditions.}

\subsection{Lagrangian and Hamiltonian mechanics}

%In Lagrangian mechanics, a configuration manifold $\mathcal{Q}$ with time-dependent coordinates $q(t)$ and its tangent bundle $\mathcal{TQ}$ with coordinates $(q(t),\dot{q}(t))$ are considered. The Lagrangian is a map $L: \mathcal{TQ} \rightarrow \mathbb{R}$.
{A Lagrangian mechanical system is defined by a pair $(\mathcal{Q}, L)$, where $\mathcal{Q}$ is the configuration manifold of  $\dim \mathcal{Q} = d$ and $L: T \mathcal{Q} \to \mathbb{R}$ is the Lagrangian function of the system. Here, $T \mathcal{Q}$ denotes the tangent bundle of $\mathcal{Q}$, also known as velocity phase space. Throughout we assume local coordinates $(q^1,...,q^d) = q$ on $\mathcal{Q}$ and adapted coordinates on $T\mathcal{Q}$, $(q^1,...,q^d, \dot{q}^1,...,\dot{q}^d) = (q,\dot{q})$.}
In mechanics, we usually restrict our attention to Lagrangians of the form kinetic energy $T(q, \dot{q})$ minus potential energy $V(q)$. Hamilton's principle requires the action integral
%over the Lagrangian
{\begin{equation*}
\int_{0}^T L(q(t),\dot{q}(t)) \, \mathrm{d} t
\end{equation*}
to be stationary over physical trajectories $q \in C^k([0,T],\mathcal{Q})$, $k \geq 2$, subject to fixed boundary conditions} resulting in the Euler-Lagrange equations of motion. The connection between the Lagrangian and Hamiltonian settings can be achieved by the Legendre transformation. The dynamics in the Hamiltonian setting is defined on the cotangent bundle $\mathcal{T}^* \mathcal{Q}$ with coordinates %$(q(t), p(t))$
{$(q^1,...,q^d, p_1,...,p_d) = (q,p)$}
and the Hamiltonian $H:\mathcal{T}^*\mathcal{Q} \rightarrow \mathbb{R}$ representing the system's energy. Figure \ref{fig:LHmech} depicts the connections. \textcolor{black}{For further details we refer to \cite{arnold78, marsden94}.}\\
\textcolor{black}{In this paper we denote the Lagrangian and Hamiltonian of mechanical systems with $L$ and $H$, respectively. Next, we will introduce the Hamiltonian of Pontryagin’s maximum principle and the augmented objective and Lagrangian, which are all denoted by calligraphic letters, namely, $\mathcal{H}$ for the Hamiltonian and $\mathcal{J}$ and $\mathcal{L}$ for the objective and Lagrangian respectively.}

%%%%%%%%%%%%%%%%%%%%%%%

% Figure environment removed


 \subsection{Optimal control problem}
 
An optimal control problem (OCP) on a smooth manifold $\mathcal{M}$ can be stated in a general form as follows
\begin{equation} \label{eq:OCP}
	\begin{aligned}
		& \underset{u}{\text{min}} & & J\big(x,u\big) =  \phi\big( x(T)\big)+ \int_0^T \ell\big(x(t), u(t) \big)~dt& \\
		& \text{subject to} & & x(0) \in \mathcal{M}_0,&\\
		& & &x(T) \in \mathcal{M}_T,&\\
		& & & \dot{x}(t)=f(x(t),u(t)),&\\
	\end{aligned}
\end{equation}
where $x \in \mathcal{M}$ is the state, $u$ is the control{, which for this work is assumed to take values on a smooth manifold $\mathcal{N}$,} and $t \in [0,T]$ is the time. The final time $T$ is assumed to be finite, i.e.\ $0 < T < +\infty$. %The control $u$ takes its values in %a subset $U$ of  a smooth manifold $\mathcal{N}$.
Initial and final constraints on the state are defined by smooth manifolds $\mathcal{M}_0$ and $\mathcal{M}_T$. The dynamical constraint is defined by a {continuously differentiable} function $f: \mathcal{M} \times \mathcal{N} \rightarrow T\mathcal{M}$. The cost functional $J$ consists of an integral over the Lagrange term $\ell$, also called running cost, and the Mayer term $\phi$ for the end condition on the state. Both maps $\ell: \mathcal{M} \times \mathcal{N} \rightarrow \mathbb{R}$ and $\phi: \mathbb{R} \times \mathcal{M} \rightarrow \mathbb{R}$ are assumed to be {continuously differentiable.} %smooth. 
In general, the control $u(\cdot)$ is assumed to be a bounded measurable function.
More precisely, we have the following most general function spaces,
\begin{align*}
x\in W^{1,\infty} ([0,T],\mathcal{M}), \,%
 				u\in L^{\infty} ([0,T],\mathcal{N}),\,%
 				\ell \in C^1 (\mathcal{M}\times \mathcal{N}, \mathbb{R}),\\
 				\phi \in C^1(\mathcal{M},\mathbb{R}),\,%
 				f \in C^1(\mathcal{M}\times \mathcal{N}, T\mathcal{M})% 
 				, \, J \in C^1 (\mathcal{M} \times \mathcal{N}, \mathbb{R}).
\end{align*}
 However, %in this paper we intend to use variational principles, thus
 later we will see that we need stronger assumptions for our purposes.

 \subsection{Necessary optimality conditions}

 
Pontryagin's maximum principle (PMP), \textcolor{black}{see \cite{pontryagin1964}}, yields necessary optimality conditions for the OCP in \eqref{eq:OCP} in form of a generalized Hamiltonian system and can be stated as follows. If %$u(\cdot)$
{$u$ denotes an optimal solution curve}
%is an optimal solution
of \eqref{eq:OCP} and %$x(\cdot)$ 
{$x$ denotes}
the associated optimal trajectory, then there exist %s a nontrivial adjoint state $(\lambda(\cdot), \lambda_0) \in T^*\mathcal{M}\times \R_-$ such that $(x, \lambda)$ satisfy
{an adjoint curve $\lambda$ and a multiplier $\lambda_0 \in \mathbb{R}_{-}$ such that $(\lambda,\lambda_0) \neq 0$, and $(x,\lambda,\lambda_0,u)$ satisfies}
a generalized Hamiltonian system associated with the control Hamiltonian %\mathcal{H}(x, \lambda, \lambda_0, u)$
{$\mathcal{H} : T^* \mathcal{M} \times \mathbb{R} \times \mathcal{N} \to \mathbb{R}$,}
defined by 
\begin{align*}
    \mathcal{H}(x, \lambda, \lambda_0, u) = \langle \lambda, f(x,u) \rangle + \lambda_0 \, \ell(x,u),
\end{align*}
 where {$\langle \cdot, \cdot \rangle: T^* \mathcal{M} \times T \mathcal{M} \to \mathbb{R}$} is the {canonical} pairing %between $T^*\mathcal{M}$ and $T\mathcal{M}$.
 {of covectors and vectors.}
 In local coordinates, it takes the form of {the} standard Euclidean product between vectors. The generalized Hamiltonian system associated to $\mathcal{H}$ provides us with 
differential state equation,
differential adjoint equation and optimality condition
in the following form \begin{equation} \label{eq:PMP}
\begin{cases}
    \dot{x}= \partial_{\lambda}\mathcal{H}(x, \lambda, \lambda_0, u), \\ % = f(x,u), \\
    \dot{\lambda}= -\partial_x\mathcal{H}(x, \lambda, \lambda_0, u), \\ %= -\lambda^T \frac{\partial}{\partial q}f(q,u) - \lambda_0 \ell_q(q,u), \\
     %\teqtcolor{black}{
     %\mathcal{H}(q, \lambda_0, \lambda, u) = \maq_{v \in \mathcal{N}} \mathcal{H}(q, \lambda_0, \lambda, v) 
     0 = \partial_u\mathcal{H}(x,\lambda,\lambda_0,u), %\,\,\,\,=\lambda^T f_u(q,u)+\lambda_0l_u(q,u)},
\end{cases}
\end{equation}
where $ (x,\lambda) \in W^{1,\infty}([0,T],T^*\mathcal{M}), \,%
		%q\in W^{1,\infty} ([0,T],\mathcal{M}), \,%
		u\in L^{\infty} ([0,T],\mathcal{N}),\,%
		f \in C^1(\mathcal{M}\times \mathcal{N}, T\mathcal{M}),\,%
		\ell \in C^1 (\mathcal{M}\times \mathcal{N}, \mathbb{R})%
	$ and hence $\mathcal{H}\in C^1(T^* \mathcal{M}\times \mathbb{R} \times \mathcal{N}, \mathbb{R})$.
In addition, the transversality conditions define the relation between state and adjoint variables at the initial and final time as follows
\begin{equation} \label{eq:trasvers}
\lambda(0) \perp T_{x(0)}\textcolor{black}{\mathcal{M}}_{\textcolor{black}{0}},  \qquad \lambda(T) - \lambda_0 \, \partial_x\phi\big( x(T)\big)  \perp T_{x(T)}\textcolor{black}{\mathcal{M}}_{\textcolor{black}{T}}.
\end{equation}
A map $t\mapsto(x(t), \lambda(t))$ satisfying the conditions of the PMP is called an extremal. An extremal is called normal if the associated $\lambda_0$ satisfies $\lambda_0 < 0$ and abnormal if $\lambda_0 = 0$. Notice that \eqref{eq:PMP} is invariant under the rescaling of $(\lambda_0, \lambda(t))$ by any positive constant and in {the} case of a normal extremal it is usual to fix $\lambda_0 = -1$. 
Two special cases of interest are the case {where} the initial and final conditions in \eqref{eq:OCP} are fixed, i.e., $x(0) = x_0, \ x(T) = x_T$, and the case {where} the initial condition is fixed, $x(0) = x_0$, and the final state is free, $x(T) \in \mathcal{M}$. 
In the first case, the transversality conditions \eqref{eq:trasvers} are empty because $T_{x(0)}\mathcal{M}_0 = T_{x(T)}\mathcal{M}_T = \{0\}$. In the second case, \eqref{eq:trasvers} implies $\lambda(T) = \lambda_0 \, {\partial_x}\phi\big(x(T)\big)$ and  only $\lambda_0 = -1$ is possible to ensure {$(\lambda,\lambda_0) \neq 0$. Therefore, there are no abnormal extremals in this case.} \\%From now on we only address normal cases with $\lambda_0 = -1$.\\
Let us now consider in more detail the variational augmented objective approach as an alternative technique to derive optimality conditions. Compared to PMP, the variational approach %requires more regularity on the control
{assumes more regularity}
\cite{clarke1990}.  %The variational approach
{It} relies on the introduction of an augmented objective $\mathcal{J}$, %. The augmented objective $\mathcal{J}$ is
constructed in such a way that it appends the equality and inequality constraints from~\eqref{eq:OCP} to the objective functional via Lagrange multipliers. The resulting cost can be written in the following form.
\begin{align}
	\mathcal{J}\big(x, \lambda, u\big) &=  \phi\big(x(T)\big)+ \int_0^T \mathcal{L}(x(t), \dot{x}(t),\lambda(t),u(t)) \, dt\nonumber\\
 &= \phi\big(x(T)\big)+ \int_0^T \Big[ \ell\big(x, u \big) + \lambda^\top \big(\dot{x} - f(x,u) \big) \Big] ~dt,\label{eqn:augObj}
\end{align}
where the integral term
\begin{equation*}
    \mathcal{L}(x,\dot{x},\lambda,u)=\ell(x,u)+\lambda^\top(\dot{x}-f(x,u) )
\end{equation*}
 of the augmented objective $\mathcal{J}$ in (\ref{eqn:augObj}) is also called augmented Lagrangian of~\eqref{eq:OCP} (not to confuse with the Lagrangian term $\ell$).
 Necessary optimality conditions are derived via the calculus of variations requiring the stationarity of the augmented objective, i.e.,
\begin{align}
\label{stationarityJ}
    \delta \mathcal{J}(x,\lambda, u)=0.
\end{align}
 This variational approach gives equivalent necessary optimality conditions as PMP {for normal cases}, see e.g. \cite[Sec.~3.4,~Sec.~4.1]{liberzon12}, and thus equation \eqref{stationarityJ} 
 again leads to \eqref{eq:PMP} and \eqref{eq:trasvers}.
 However, this approach requires stronger smoothness assumptions on the state and adjoint variables compared to PMP. More precisely, we have to assume 
 \begin{align*}
%q\in C^{2 }([0,T],\mathcal{M}),  \, 
     u \in C^1([0,T], \mathcal{N}), \,
			(x,\lambda) \in C^2([0,T],T^*\mathcal{M}), \\
			f \in C^1(\mathcal{M}\times \mathcal{N}, T\mathcal{M}),\,%
			\ell \in C^1 (\mathcal{M}\times \mathcal{N}, \mathbb{R}),\,%
			\phi \in C^1(\mathcal{M}, \mathbb{R}), 
 \end{align*}
 and hence
 \begin{align*}
     \mathcal{L} \in C^1 (T\mathcal{M} \oplus T^*\mathcal{M}\times \mathcal{N}, \mathbb{R}) \quad \text{and} \quad \mathcal{J} \in C^1( T^*\mathcal{M}\times \mathcal{N}, \mathbb{R}),
 \end{align*}
 {where $\oplus$ denotes the Whitney sum of vector bundles \cite{CrampinPirani86, Saunders89}.}
 For a general optimal control problem such as \eqref{eq:OCP}, the Lagrangian $\mathcal{L}$ is non-regular as no time derivative $\dot{\lambda}$ appears. Therefore, a Legendre transformation to the Hamiltonian setting is not well defined. There exist some approaches where a generalized Legendre transformation is defined to overcome this issue. For example, Gotay and Nester's version of Dirac's constraint algorithm \cite{gotay1979} provides the correct setting for non-regular Lagrangians due to constraints. However, the algorithm is difficult to use in practice. \\
 The goal of this work is
to derive a new uniform Lagrangian on a proper tangent space
which then directly yields a symplectic integration scheme for
state and adjoint variable if we apply a variational integration scheme {\cite{marsden01}} and from which easily a Hamiltonian can be derived via Legendre
transformation. A key ingredient of the following is the introduction of a new Hamiltonian $\tilde{\mathcal{H}}$ and Lagrangian $\tilde{\mathcal{L}}$ for optimal control problems. Explicit dependence on the control $u$ is denoted by a superscript, e.g.~$\tilde{\mathcal{H}}^u$.


\section{Construction of a new Lagrangian for a control affine system with a specific objective}

In the present work we consider a special case of the OCP in \eqref{eq:OCP}, where the state equation is described by a second order ordinary differential equation (ODE) as a dynamical constraint with fixed initial and free final state. In addition, the configuration space $\mathcal{Q}$ is given by a vector space. Notice that by fixing coordinates in $\mathcal{Q}$ we identify it with $\R^n$ and only linear coordinate transformations are allowed to preserve {the} linear structure of $\mathcal{Q}$. From now on, we work in coordinates on $\mathcal{Q}$. A running cost term that is quadratic in the control $u$ is considered. We assume that the objective, the differential equation, as well as the control function are sufficiently smooth so that we can take all necessary derivatives. As dynamical constraints, we focus on differential equations representing the motion of a mechanical system.
Thus, $f(q)$ can be interpreted as a force due to a physical potential and the control $u$ as an external force. Precisely, we consider the following problem:
\begin{equation} \label{eq:eqample.OCP.}
	\begin{aligned}
		& \underset{u}{\text{min}} & & J\big(q,u\big) =  \phi\big(q(T), \dot{q}(T)\big) + \int_0^T \frac{1}{2} u^2(t)~dt& \\
		& \text{subject to} & & q(0) = q^0,&\\
		& & &\dot{q}(0) = \dot{q}^0,&\\
		& & & \ddot{q}=f(q)+u,&\\
	\end{aligned}
\end{equation}
{where $q\in C^3 ([0,T],\mathcal{Q}),\,%
	u \in C^1([0,T],\mathcal{N}), \,%
	f \in C^1(\mathcal{Q}, T(T\mathcal{Q})), \, \phi \in C^1( T\mathcal{Q}, \mathbb{R})$}, and thus, $J \in C^1(\mathcal{Q}\times \mathcal{N}, \mathbb{R})$. 
The second order ODE can be rewritten as an explicit system of two first-order ODEs: 
\begin{equation}\label{eq:SODE}
	\begin{bmatrix*}\dot{q}\\\dot{v}\end{bmatrix*} = \begin{bmatrix*}v\\f(q)+u\end{bmatrix*}.
\end{equation}

\subsection{Pontryagin's maximum principle for the control affine system}

We first apply Pontryagin's maximum principle to \eqref{eq:eqample.OCP.}. Since the final condition in \eqref{eq:eqample.OCP.} is free, we have that $\lambda_0=-1$ and the normal Hamiltonian from PMP is defined as follows
\begin{align}
\label{normalhamiltonian}
    \mathcal{H}(q,v, \lambda_q, \lambda_v, u)= \lambda_q^\top v+ \lambda_v ^\top f(q)+\lambda_v^\top u-\frac{1}{2} u^2,
\end{align}
\textcolor{black}{where $\lambda_q$ and $\lambda_v$ are the adjoints related to $q$ and $v$, respectively.}
%for which it holds
%\[ \mathcal{H}\in C^1 ( T^*(T\mathcal{M})\times \mathcal{N}, \mathbb{R}).\]
For \eqref{normalhamiltonian} we obtain the following theorem applying PMP.
\begin{theorem}
   Considering the optimal control problem \eqref{eq:eqample.OCP.} the following equations hold
\begin{align*}
%\label{PMPto5b}
\dot{\lambda}_q=-\frac{\partial}{\partial q}f(q)^\top\lambda_v,\qquad
 \dot{\lambda}_v =-\lambda_q,\qquad
 0=u-\lambda_v.
\end{align*}
Moreover, for the final time one has 
\begin{align}
\label{finalconstraint}
     \lambda_q(T) =-\frac{\partial}{\partial q}\phi(q(T), v(T)),\qquad \lambda_v(T) =-\frac{\partial}{\partial v}\phi(q(T), v(T)). 
\end{align}
\end{theorem}


\begin{proof}
Using Pontryagin's maximum principle for \eqref{eq:eqample.OCP.}, we obtain the following equations
\begin{align*}
%\label{PMPto5a}
    \begin{bmatrix}
			\dot{q}\\ \dot{v}
		\end{bmatrix}= %\nabla_{(\lambda_q,\lambda_v)}\mathcal{H}=
   \begin{bmatrix}
			\frac{\partial \mathcal{H}}{\partial \lambda_q}\\[4pt] \frac{\partial \mathcal{H}}{\partial \lambda_v}
		\end{bmatrix} = 
  \begin{bmatrix}
			v\\ f(q)+u
		\end{bmatrix},\qquad 
    \begin{bmatrix}
    			\dot{\lambda}_q\\ \dot{\lambda}_v
		\end{bmatrix}=- %\nabla_{(q,v)}\mathcal{H}=
  \begin{bmatrix}
			\frac{\partial \mathcal{H}}{\partial q}\\[4pt] \frac{\partial \mathcal{H}}{\partial v}
		\end{bmatrix} = 
  \begin{bmatrix}
			-\frac{\partial}{\partial q}f(q)^\top\lambda_v\\ -\lambda_q
		\end{bmatrix},
\end{align*}
and 
$$ 0=%\nabla_u\mathcal{H}=u-\lambda_v.$$
\frac{\partial}{\partial u}\mathcal{H}=u-\lambda_v.$$


From Pontryagin's maximum principle one can also derive condition \eqref{finalconstraint} directly, which finishes the proof.
\end{proof}

\subsection{Augmented objective approach} We now consider the variational approach for problem \eqref{eq:eqample.OCP.}.
Note that the classical approaches work with the system of first order ODEs, {which is \eqref{eq:SODE} in our case. The corresponding Lagrange multipliers are $\lambda_q,\: \lambda_v$.} %such that corresponding adjoint variables, $\lambda_q,\: \lambda_v$ need to be introduced. 
With that, we can
define the augmented Lagrangian for \eqref{eq:eqample.OCP.} as follows
\begin{align}
\label{lagr}
\mathcal{L}(q,v, \dot{q}, \dot{v}, \lambda_q, \lambda_v, u) &= \frac{1}{2} u^2+ [\lambda_q^\top,\lambda_v^\top] \begin{bmatrix*}\dot{q}-v\\\dot{v}-f(q)-u\end{bmatrix*}\notag\\
&=\frac{1}{2} u^2 + \lambda_q^\top\big(\dot{q} - v \big)+ \lambda_v^\top \big( \dot{v} -f(q)\big) - \lambda_v^\top u,
\end{align}
where
\begin{align*}
   & (q,\lambda_q) \in C^3([0,T], T^*\mathcal{Q}),\quad (q,v,\lambda_q,\lambda_v) \in C^2([0,T], T^*(T\mathcal{Q})),\\
    &\mathcal{L}\in C^1 (T(T\mathcal{Q}) \oplus T^*(T\mathcal{Q}) \times \mathcal{N}, \mathbb{R}).
\end{align*}

With this Lagrangian, we can derive the following theorem considering the augmented objective approach.

\begin{theorem}
   The augmented objective 
   \begin{align*}
\mathcal{J}(q, v,\lambda_q, \lambda_v, u)&=
   \phi\big(q(T),v(T)\big)+ \int_0^T \mathcal{L}(q,v, \dot{q}, \dot{v}, \lambda_q, \lambda_v, u) \, dt \\
   &=\phi\big(q(T),v(T)\big)+ \int_0^T \frac{1}{2} u^2 + \lambda_q^\top\big(\dot{q} - v \big)+ \lambda_v^\top \big( \dot{v} -f(q)\big) - \lambda_v^\top u \, dt
   \end{align*}
   is stationary for the respective variations, if the constraints in \eqref{finalconstraint} and the following equations hold:
   \begin{align*}
    -\dot{\lambda}_q  -\frac{\partial}{\partial q}f(q)^\top\lambda_v =0,\qquad
      -\dot{\lambda}_v - \lambda_q = 0,\qquad
        u - \lambda_v =0.
   \end{align*}
   \end{theorem}

\begin{proof}
After partial integration of the term $\lambda_q^\top v $ and applying the variational approach, we require stationarity of $\mathcal{J}$, i.e.,
\begin{align*}
    0=\delta\mathcal{J} &=\left(\lambda_q(T) + \frac{\partial}{\partial q}\phi(q(T), v(T))
  \right)\delta q(T) + \left(  \lambda_v(T) +\frac{\partial}{\partial v}\phi(q(T), v(T))\right) \delta \dot{q}(T)\\
    &\qquad + \int_0^T (\dot{q} - v) \delta \lambda_q+ \left( \dot{v} -( f(q) + u)\right)\delta \lambda_v+ (-\dot{\lambda}_q -\frac{\partial}{\partial q}f(q)^\top\lambda_v) \delta q+ (-\dot{\lambda}_v - \lambda_q)\delta v+ (u - \lambda_v) \delta u.
\end{align*}
    Since the variations are independent, using the fundamental lemma of the calculus of variation, we deduce that the augmented objective is stationary, if the following conditions hold:
\begin{alignat}{2}
		&   \dot{q} = v, &&\dot{v} = f(q) + u, \nonumber \\
		 &  -\dot{\lambda}_q  -\frac{\partial}{\partial q}f(q)^\top\lambda_v =0, && -\dot{\lambda}_v - \lambda_q = 0, \qquad  u - \lambda_v = 0, \label{eq:eqample.constraint.} \\
		& \lambda_q(T) + \frac{\partial}{\partial q}\phi(q(T), v(T))=0, \qquad&&  \lambda_v(T) + \frac{\partial}{\partial v}\phi(q(T), v(T)) =0, \nonumber
	\end{alignat}
 which proves the theorem.
 \end{proof}
 % Written by Sofya
 \begin{remark}
     Notice that by substituting $u = \lambda_v$ and $\lambda_q=-\dot{\lambda}_v$, \eqref{eq:eqample.constraint.} can be equivalently written as follows
     \begin{equation} \label{eq:eqample.constraint.2order}
         \ddot{q} = f(q) + \lambda_v, \qquad \ddot{\lambda}_v = \frac{\partial}{\partial q}f(q)^\top\lambda_v.
     \end{equation}
 \end{remark}
\begin{remark}
    \textcolor{black}{Notice also that the derivatives $\dot{\lambda}_q$ and $\dot{\lambda}_v$ do not appear in $\mathcal{L}$ in \eqref{lagr}, and thus, the derivatives of $\mathcal{L}$ with respect to $\dot{\lambda}_q,\: \dot{\lambda}_v$ vanish, and therefore the Jacobian of $\mathcal{L}$ is singular.}
\end{remark}

 
\subsection{New Lagrangian and Euler-Lagrange equations}\label{ssec:new_lagrangian}

 %The variation with respect to the control $u$ can be used to derive the optimal control $u^* = \lambda_v$ and thus to eliminate the control in every equation. The relation between the Lagrange multipliers, i.e. $\lambda_q=-\dot{\lambda}_v$, can be used to write the condition for the variation w.r.t. the state variable $q$ as a second order ODE 
%\begin{align}
%\label{equivODE}
%    \ddot{\lambda}_v-\lambda_v \frac{\partial}{\partial q}f(q) = 0.
%\end{align}
%This equation is the direct result of the augmented objective approach for the second order ODE. \\
%Using the relation between the Lagrange multipliers obtained via the variation of the velocity $v$ in \eqref{eq:eqample.constraint.}, we replace $\lambda_q$ by $-\dot{\lambda}_v$ in $\mathcal{L}$ and perform partial integration for the term $\lambda_v\dot{v}$ to form a new regular control Lagrangian $\tilde{\mathcal{L}}^u$ and boundary terms. Denoting $\lambda_v$ by $\lambda$ to simplify notations, results in the new Lagrangian of the following form
%\begin{align}
%\label{eq:new.lagrange.}
%\mathcal{L}(q,v, \dot{q}, \dot{v}, \lambda_q, %\lambda_v, u) 
%&=u^2 / 2 + \lambda_q\big(\dot{q} - v \big)+ \lambda_v \big( \dot{v} -f(q)\big) - \lambda_v u\notag\\
%&= u^2 / 2 - \dot{\lambda}_v\big(\dot{q} - v \big)+ \lambda_v \big( \dot{v} -f(q)\big) - \lambda_v u\notag
%\\
%&\equiv u^2 / 2 - \dot{\lambda}_v\big(\dot{q} - v \big)-\dot{\lambda}_v v-\lambda_v f(q)  - \lambda_v u\notag\\
%	& = \frac{1}{2} u^2 -\dot{\lambda} \dot{q} - \lambda f(q) -\lambda u=:\tilde{\mathcal{L}}^u(q, \lambda, \dot{q}, \dot{\lambda},u).
%\end{align} 

We derive the new Lagrangian $\tilde{\mathcal{L}}^u$ appending the second order ODE $\ddot{q}=f(q)+u$ to the cost and doing integration by part of the term $\lambda^\top \ddot{q}$ as follows:
\begin{align}
\label{eq:new.lagrange.}
   & \int_0^T \frac{1}{2}u^2+\lambda(t)^\top (\ddot{q}(t)-f(q(t))-u(t))\: dt\notag\\
    &\quad = \int_0^T \frac{1}{2}u^2- \dot{\lambda}(t)^\top \dot{q}(t)-{\lambda}(t)^\top f(q(t))-{\lambda}(t)^\top u(t) \: dt+ [\lambda(t)^\top \dot{q}(t)]_0^T\notag\\
    &\quad= \int_0^T {\hat{\mathcal{L}}}^u(q, \dot{\lambda}, \dot{q}, {\lambda},u) \: dt+[\lambda(t)^\top \dot{q}(t)]_0^T,
\end{align}
where
\begin{align*}
(q,\dot{\lambda})\in C^3 ([0,T],T^*\mathcal{Q}),%,\,%
%	u \in C^1([0,T],\mathcal{N}), \,%
%f \in C^1(\mathcal{Q},T(T\mathcal{Q})),\,%
%    \Phi \in C^1(\mathcal{Q} \times T\mathcal{Q}, \mathbb{R}),
\end{align*}
and thus 
\begin{align*}
{\hat{\mathcal{L}}}^u\in C^1(T(T^*\mathcal{Q})\times \mathcal{N}, \mathbb{R}).
\end{align*}
%\rodrigo{The v}ariational principle applied to %the new Lagrangian
%\rodrigo{Eq.~\eqref{eq:new.lagrange.}}
%leads to the correct boundary conditions, i.e., %new derivation of our new Lagrangian also directly gives us the correct boundary conditions, i.e.,
%$$\frac{\partial\phi}{\partial  q}\big(q(T), \dot{q}(T)\big)-\dot{\lambda}(T) = 0,\qquad  \frac{\partial\phi}{\partial  \dot{q}}\big(q(T), \dot{q}(T)\big)+{\lambda}(T)
%  = 0. $$
%as follows: \\
Consider variations of the new augmented objective 
$$\tilde{\mathcal{J}}^u(q,\lambda,u) =  \phi\big(q(T), \dot{q}(T)\big) +\lambda(T)\dot{q}(T)-\lambda(0)\dot{q}(0)+\int_0^T   
    \frac{1}{2} u^2 -\dot{\lambda}^\top \dot{q} - \lambda^\top f(q) -\lambda^\top u
    ~dt,  $$
    with $ \tilde{\mathcal{J}}^u \in C^1(T^*(T\mathcal{Q})  \times \mathcal{N}, 
 		\mathbb{R})$,
the application of partial integration for the term $\delta(-\dot{\lambda}^\top \dot{q})$ leads to \newpage
\begin{align*}
0=\delta \tilde{\mathcal{J}}^u&=
    \frac{\partial\phi}{\partial  q} \delta q(T)+\frac{\partial\phi}{\partial  \dot{q}} \delta  \dot{q}(T)+\lambda(T)\delta  \dot{q}(T)+\dot{q}(T)\delta\lambda(T)-
    \lambda(0)\delta  \dot{q}(0)\\ &\qquad -
    \dot{q}(0)\delta \lambda(0)-\dot{\lambda}(T)\delta  q(T)+
   \dot{\lambda}(0)\delta  q(0)+  \dot{q}(T)\delta\lambda(T)+\dot{q}(0)\delta\lambda(0)\\
   &\qquad +\int_0^T (u-\lambda)\delta u+ (\ddot{q}-f(q)-u)\delta\lambda + (\ddot{\lambda}-\frac{\partial}{\partial q}f(q)^\top \lambda )\delta q \, dt,
\end{align*}
from which focusing only on the boundary terms and using that the variations $\delta q(0)=\delta \dot{q}(0)=0$ we deduce 
\begin{alignat}{2}
\label{endcond}
		&  \frac{\partial\phi}{\partial  q}\big(q(T), \dot{q}(T)\big)-\dot{\lambda}(T) = 0, \quad  &&   
  \frac{\partial\phi}{\partial  \dot{q}}\big(q(T), \dot{q}(T)\big)+{\lambda}(T)
  = 0,  \\
		& -\dot{q}(0) + \dot{q}(0) =0, \quad &&  \dot{q}(T) - \dot{q}(T) = 0. \nonumber
	\end{alignat}
In order to define a new, regular Lagrangian, let us introduce a new variable %$y \in \mathcal{Y}=\mathcal{Q}\times T^*(T\mathcal{Q})$ defined by $y=(q,\lambda)$ describing the combined state and adjoint variable living in this
{$y \in T^*\mathcal{Q}$, with $y = (q, \lambda)$} describing the combined state and adjoint variable living in this combined configuration manifold. We assume that %$y\in C^3([0,T]\times [0,T], \mathcal{Q}\times T^*(T\mathcal{Q}))$
{$y\in C^3([0,T], T^*Q)$}. Let us also define a new Lagrangian (see Section~\ref{ssec:geometric_setting})
\begin{equation}
    \tilde{\mathcal{L}}^u(q,\lambda,\dot{q},\dot{\lambda},u) := \hat{\mathcal{L}}^u(q,\dot{\lambda},\dot{q},\lambda,u).\label{eq:actual.new.lagrange.}
\end{equation}
The OCP with the new Lagrangian term $\tilde{\mathcal{L}}^u$ %(q, \dot{\lambda}, \dot{q}, {\lambda},u)$
and the corresponding differential equations as in \eqref{eq:eqample.constraint.} can be stated in terms of the new state variable $y$. \textcolor{black}{The optimal control $u^*=\lambda$ associated with the new Lagrangian \eqref{eq:actual.new.lagrange.} can be derived in the same way as for the augmented objective, i.e. via a variational principle. Substituting the expression of $u^*=\lambda$ in $\tilde{\mathcal{L}}^u$ as a function of $\lambda$, we define $\tilde{\mathcal{L}}$, the new Lagrangian without dependence on the control}
\begin{equation}\label{eq:new.lagrange.no.input}
\tilde{\mathcal{L}}:T\mathcal{Y}\rightarrow \mathbb{R},\qquad 
	\tilde{\mathcal{L}}(y, \dot{y})=\tilde{\mathcal{L}}(q, \lambda, \dot{q}, \dot{\lambda}) =  -\dot{\lambda}^\top \dot{q} - \lambda^\top f(q) - \frac{1}{2} \lambda^\top \lambda,
\end{equation}
where 
\begin{align*}
   {(q,\lambda)\in C^3 ([0,T],T^*\mathcal{Q})},\,%
	f \in C^1(\mathcal{Q}, T(T\mathcal{Q})).
\end{align*}
The new Lagrangian $\tilde{\mathcal{L}}$ is regular and is defined on the tangent bundle $\mathcal{TY}$ of the manifold $\mathcal{Y}$. Structurally, it is of the same form as a mechanical Lagrangian $L$, thus it is possible to apply the same techniques such as Euler-Lagrange equations, Legendre transformation and formulation of the Hamiltonian counterpart as depicted in Figure~\ref{fig:LHmech}. 
Let us first consider the following theorem.
\begin{theorem} 
The Euler-Lagrange equations for the Lagrangian $ \tilde{\mathcal{L}}$ provide the same necessary conditions and boundary constraints as the ones in \eqref{eq:eqample.constraint.} of the original minimization problem \eqref{eq:eqample.OCP.}, i.e.,
\begin{align}
\label{eq:EL_new.}
\ddot{\lambda}=  \frac{\partial}{\partial q}f(q)^\top \lambda,\qquad \ddot{q} = f(q) + \lambda,    
\end{align}
and 

  $$  \frac{\partial\phi}{\partial  q}\big(q(T), \dot{q}(T)\big)-\dot{\lambda}(T) = 0,\qquad  \frac{\partial\phi}{\partial  \dot{q}}\big(q(T), \dot{q}(T)\big)+{\lambda}(T)
  = 0.$$
\end{theorem}
\begin{proof}
The Euler-Lagrange equations for $y=(q,\lambda)$ are given as
\begin{align*}
    \frac{\partial \tilde{\mathcal{L}}}{\partial y} (y,\dot{y})-\frac{d}{dt}\left( \frac{\partial \tilde{\mathcal{L}}}{\partial \dot{y}} (y,\dot{y}) \right)=0,
\end{align*}
from which we derive \eqref{eq:EL_new.},
which corresponds to the state and adjoint equations derived in \eqref{eq:eqample.constraint.2order}. {The boundary conditions} {in \eqref{eq:eqample.constraint.}
are obtained in \eqref{endcond}.}
%Moreover, from the variational principle substituting $u$ by $\lambda$ one obtains the boundary conditions \eqref{endcond}.
\end{proof}

\subsection{New Hamiltonian system}\label{ssec:new_hamiltonian}

Let us now consider the Hamiltonian system of the new, regular Lagrangian $\tilde{\mathcal{L}}$. The Legendre transformation,
$\mathbb{F}L:(y,\dot{y}) \mapsto \left(y,{p_y} = \frac{\partial \tilde{\mathcal{L}}}{\partial \dot{y}}(y,\dot{y})\right)$,
for the new Lagrangian provides the conjugate momenta  

\begin{align*}
		p_y=
			 \frac{\partial \tilde{\mathcal{L}}}{\partial \dot{y}}, \qquad \quad\begin{bmatrix}
			p_q\\p_{\lambda}
		\end{bmatrix} =
		\begin{bmatrix}
			 -\dot{\lambda}\\-\dot{q}
		\end{bmatrix},	
\end{align*}
and therefore invertible. 
The new Hamiltonian can be computed as
\begin{align*}
\tilde{\mathcal{H}}: T^* \mathcal{Y} \rightarrow \mathbb{R}, \quad 
\tilde{\mathcal{H}}(y, p_y)&=\tilde{\mathcal{H}}\big(q, \lambda, p_q, p_{\lambda}\big)  = \left\langle\begin{bmatrix}
		p_q \\ p_{\lambda}
	\end{bmatrix},
 \begin{bmatrix}
		\dot{q} \\ \dot{\lambda}
	\end{bmatrix}\right\rangle
 -\tilde{\mathcal{L}}(q, \lambda, \dot{q}, \dot{\lambda}) \notag\\
 &=p_q^\top \dot{q}+p_{\lambda}^\top \dot{\lambda}+ \dot{\lambda}^\top\dot{q}+\lambda^\top f(q)+\frac{1}{2}\lambda^2=-\dot{q}^\top\dot{\lambda}+\lambda^\top f(q)+\frac{1}{2}\lambda^2\notag\\
 &= -p_{\lambda}^\top p_q+\lambda^\top f(q)+\frac{1}{2}\lambda^2,
\end{align*}
from which we deduce the following.
\begin{theorem} \label{Hamiltonsystem}
   Hamilton's equations for $\tilde{\mathcal{H}}$ give the same necessary conditions as the ones in \eqref{eq:eqample.constraint.2order}, i.e.,
$$\ddot{\lambda}=\frac{\partial}{\partial q}f(q)^\top  \lambda ,\qquad
\ddot{q} = f(q) + \lambda.$$   
   Moreover, one has the following boundary conditions
   \begin{align}
    \label{Hambc}
       \dot{q}(0)&=-p_{\lambda}(0), \quad  \dot{\lambda}(0)=-p_{q}(0),\notag\\
    \dot{q}(T)&=-p_{\lambda}(T), \quad  \dot{\lambda}(T)=-p_{q}(T) ,
   \end{align}
 and
 \begin{align}
     \label{Hambc2}
\frac{\partial\phi}{\partial  q}\big(q(T), \dot{q}(T)\big)+p_q(T) = 0.  
 \end{align}
\end{theorem}

\begin{proof}
 With $p_y = (p_q, p_{\lambda})$, the associated Hamilton system is 

\begin{equation} \label{eq:Ham.sys.}
	\begin{aligned}
		&\dot{y} = \begin{bmatrix}
			\dot{q}\\ \dot{\lambda}
		\end{bmatrix}=
 %\nabla_{p_y}\tilde{\mathcal{H}}=
 \frac{\partial}{\partial p_y}\tilde{\mathcal{H}}=
  \begin{bmatrix}
			-p_{\lambda}\\ -p_q
		\end{bmatrix},\qquad 
		&\dot{p}_y = 
  \begin{bmatrix}
			\dot{p}_q\\ \dot{p}_{\lambda} 
		\end{bmatrix}
  %= -\nabla_y \tilde{\mathcal{H}}=
  = - \frac{\partial}{\partial y} \tilde{\mathcal{H}}=
  \begin{bmatrix}
			- \frac{\partial}{\partial q}f(q)^\top \lambda\\ -f(q)-\lambda
		\end{bmatrix},
	\end{aligned}
\end{equation}
which is equivalent to the Euler-Lagrange equations in~\eqref{eq:EL_new.}, since
\begin{align*}
    \ddot{q} =-\dot{p}_{\lambda}= f(q) + \lambda \qquad \text{and} \qquad
     \ddot{\lambda} =-\dot{p}_{q}=\frac{\partial}{\partial q}f(q)^\top \lambda .
\end{align*}
Therefore, they are also equivalent to the conditions in \eqref{eq:eqample.constraint.2order} of the original OCP problem. Using the left-hand side of \eqref{eq:Ham.sys.} one obtains the desired conditions in \eqref{Hambc},
  which together with \eqref{eq:eqample.constraint.} yield \eqref{Hambc2}.
\end{proof}

As it was shown in Theorem \ref{Hamiltonsystem} we have commutation in the following diagram.

\begin{center}
	% Figure removed
\end{center}


In summary, for the optimal control problem \eqref{eq:eqample.OCP.} we derived a purely Lagrangian system on the new tangent bundle $\mathcal{TY}$ which involves both, state and adjoint variables of the optimal control problem. Thus we can apply standard techniques as for mechanical Lagrangians. The Euler-Lagrange equations provide the correct state and adjoint dynamics. Furthermore, a purely Hamiltonian description is obtained by applying a Legendre transformation.

\subsection{The broader geometric setting}\label{ssec:geometric_setting}

\textcolor{black}{For the background in geometric mechanics required in the following section, see e.g. \cite{marsden94, treanta14}}.\\
The construction outlined in Sections \ref{ssec:new_lagrangian} and \ref{ssec:new_hamiltonian} is not generally applicable when working outside of the restriction to vector spaces and linear coordinate transformations.\\

First, notice that the integrand defined after integrating by parts in Eq.~\eqref{eq:new.lagrange.} can be regarded as a function on $T(T^*\mathcal{Q}) \times \mathcal{N}$. The terms containing the adjoints can be seen as the natural pairing of a 1-form $\alpha$ and a vector field $X$ on $T \mathcal{Q}$ as follows. If we assume adapted local coordinates $(q, v)$ on $T \mathcal{Q}$, then we can write these as
\begin{align*}
X &= X_q^i \, \partial_{q^i} + X_v^i \,\partial_{v^i}\,, \quad i = 1,..., d\\
\alpha &= \alpha_{q\,i} \,\mathrm{d}{q^i} + \alpha_{v\,i} \,\mathrm{d}{v^i}\,,
\end{align*}
and the pairing gives us
\begin{equation*}
\left\langle \alpha, X\right\rangle = \alpha_{q\,i} X_q^i + \alpha_{v\,i} X_v^i = \alpha_{q} X_q + \alpha_{v} X_v\,.
\end{equation*}

In our case, we can identify $X$ as the local section from $T \mathcal{Q} \times \mathcal{N}$ to $T(T\mathcal{Q}) \times \mathcal{N}$ that defines the right-hand side of the ODE in \eqref{eq:SODE}, i.e. $X_q = v$ and $X_v = f(q) + u$. Thus, assuming adapted local coordinates $(q, \mu, v, \lambda, u)$ in $T(T^*\mathcal{Q}) \times \mathcal{N}$, {this implies the identification $\alpha_q = \mu$, i.e. $\dot{\lambda}$ in Eq.~\eqref{eq:new.lagrange.}, and $\alpha_v = \lambda$, and} we can write
\begin{equation*}
{\hat{\mathcal{L}}}^u(q, \mu, v, \lambda, u) = \frac{1}{2} u^2 - \textcolor{black}{\mu^\top }v - \lambda\textcolor{black}{^\top} [ f(q) + u ]\,.
\end{equation*}

Second, notice that the way this ${\hat{\mathcal{L}}}^u$ appears in the integrand does not correspond to an evaluation over the canonical lift of a curve \cite{CrampinPirani86} on $T^*\mathcal{Q} \times \mathcal{N}$ to $T(T^*\mathcal{Q}) \times \mathcal{N}$. More explicitly, assume a curve $(q,\lambda) \in C^k([0,T], T^*\mathcal{Q})$ with $k > 0$, is lifted to $T(T^*\mathcal{Q})$, with which we obtain $(q, \lambda, \dot{q}, \dot{\lambda}) \in C^{k-1}([0,T], T(T^*\mathcal{Q}))$. Therefore, if we insert the lift into ${\hat{\mathcal{L}}}^u$ we obtain
\begin{equation*}
{\hat{\mathcal{L}}}^u(q(t), \lambda(t), \dot{q}(t), \dot{\lambda}(t), u) = \frac{1}{2} u^2 - \lambda(t)\textcolor{black}{^\top} \dot{q}(t) - \dot{\lambda}(t)\textcolor{black}{^\top} [ f(q(t)) + u ]\,,
\end{equation*}
which does not correspond to the integrand in \eqref{eq:new.lagrange.}. Moreover, the Euler-Lagrange equations induced by this ${\hat{\mathcal{L}}}^u$ are inequivalent to the ones we need.\\

The issue that restricts the definition of a new Lagrangian as in Eqs.~{\eqref{eq:actual.new.lagrange.} and}~\eqref{eq:new.lagrange.no.input} in the general case is that we need a canonical automorphism on $T(T^*\mathcal{Q})$ that would allow us to perform the flip $(q, \mu, v, \lambda) \leftrightarrow (q, \lambda, v, \mu)$. {This however does not generally exist, as a general coordinate transformation on $\mathcal{Q}$ immediately shows:}

{Consider the transformation $(q^i) \mapsto (Q^i(q))$, which induces the change of adapted coordinates $(q^i, \mu_i) \mapsto (Q^i(q),M_{i}(q,\mu))$. Particularly,
\begin{equation*}
    M_i = \mu_j \frac{\partial q^j}{\partial Q^i}(q) = \mu_j \left[\left(\frac{\partial Q}{\partial q}\right)^{-1}\right]^j_i(q)\,.
\end{equation*}
With this, a vector in $T^*Q$ transforms as,
\begin{align*}
    Y &= Y_q^i \, \partial_{q^i} + Y_{\mu\, i} \,\partial_{\mu_i} \\
    &= Y_Q^j \, \partial_{Q^j} + Y_{M\, j} \,\partial_{M_j} = Y_q^i \frac{\partial Q^j}{\partial q^i}(q) \partial_{Q^j} + \left[ Y_{q}^i \frac{\partial M_j}{\partial q^i}(q,\mu) + Y_{\mu \, i} \frac{\partial M_j}{\partial \mu_i}(q,\mu)\right]\,\partial_{M_j}.
\end{align*}}

{Nevertheless, in our restricted setting, transformations are assumed linear, and thus $\frac{\partial M_j}{\partial q^i} = 0$ and $\frac{\partial M_j}{\partial \mu_i} = \frac{\partial q^i}{\partial Q^j}$, which means that $Y_{\mu}$ transforms as $\mu$.\\}

{Since when $\mathcal{Q}$ is a vector space $T \mathcal{Q} \cong \mathcal{Q} \times \mathcal{Q}$ and $T^* \mathcal{Q} \cong \mathcal{Q} \times \mathcal{Q}^*$, and thus $T (T^* \mathcal{Q}) \cong \mathcal{Q} \times \mathcal{Q}^* \times \mathcal{Q} \times \mathcal{Q}^*$, and together with the previous considerations, we can state the following
\begin{proposition}
    Let $\mathcal{Q}$ be a vector space. If only linear transformations on $\mathcal{Q}$ are admissible, then there is an involution $\varphi$ on $T(T^*\mathcal{Q})$ such that, in adapted coordinates, it can be expressed as
    \begin{equation*}
        \varphi(q, \mu, v, \lambda) = (q, \lambda, v, \mu).
    \end{equation*}
\end{proposition}
}



\subsection{Noether's theorem, conserved quantities}

Having the setting as in classical mechanics, in this section we investigate what are
the conserved quantities of a Lagrangian flow corresponding to symmetries (also called invariance)
of the new Lagrangian $\tilde{\mathcal{L}}$ and what is their meaning and their relation to the dynamics and the
OCP \eqref{eq:eqample.OCP.}.

Let a Lie group $G$ act on $\mathcal{Q}$ by a smooth action $\Phi_g :\mathcal{Q} \rightarrow \mathcal{Q}$  and on $\mathcal{N}$ by a smooth action $\Psi_g : \mathcal{N} \rightarrow \mathcal{N}$ for any $g \in G$. The action has a natural lift to the space of definition of the new Lagrangian $\tilde{\mathcal{L}}^u(q, {\lambda}, \dot{q}, \dot{\lambda},u)$, namely $T(T^*\mathcal{Q})\times \mathcal{N}$, and is given by $T(T^*\Phi_{g^{-1}})\times\Psi_g$. In a local trivialization $(z,\alpha)$ of $T^*\mathcal{M}$, the action is defined by $T^*\Phi_{g^{-1}}(z, \alpha) = (\Phi_g(z), T^*_{\Phi_g(z)}\Phi_{g^{-1}}(\alpha))$. Then, it can be further lifted to $T(T^*\mathcal{Q})$ by
\begin{equation*}
    (z, \alpha, w, w_{\alpha}) \mapsto (\Phi_g(z), T^*_{\Phi_g(z)}\Phi_{g^{-1}}(\alpha), T_z\Phi_g(w), T_{\alpha}T^*_{\Phi_g(z)}\Phi_{g^{-1}}(w_{\alpha})).
\end{equation*}
In our case $\mathcal{Q}$ is a vector space, therefore, we restrict our attention to linear Lie groups $G$ in $GL(\mathcal{Q})$. The corresponding action in fixed coordinates on $\mathcal{Q}$ is given by matrix multiplication $\Phi_g(q) = gq$. We also assume that the group acts by the same action on $\mathcal{N}$, i.e. $\Psi_g(u) = gu$. The lifted group action is then defined on $(q, {\lambda}, \dot{q}, \dot{\lambda},u) $ by 
\begin{equation*}
    (q, {\lambda}, \dot{q}, \dot{\lambda},u) \mapsto (gq, (g^{-1})^\top{\lambda}, g\dot{q}, (g^{-1})^\top\dot{\lambda},gu).
\end{equation*}
%For that we first define matriq group element $g\in G$ acts on state $$\Phi_g^q:\mathcal{M}\rightarrow \mathcal{M},\quad\Phi_{g}^q(q)=gq$$ and control $$\Phi_g^u: \mathcal{N} \rightarrow \mathcal{N},\quad\Phi_{g}^u(u)=gu.$$
We shall consider symmetric optimal control problems, in the sense that we assume to have an equivariant right-hand side of state equations %under $\Phi_g^q$, i.e.,
\begin{align}
\label{equivariance_f}
    f(gq) = gf(q),
\end{align}
and invariance of the running cost %under $\Phi_g^u$, i.e.,
\begin{align*}
    \ell(gq, gu)=\ell(q,u).
\end{align*}

Moreover, we define the one parameter group of matrix transformations 
%\begin{align*}
%   \{\Phi_{g_s}:\mathcal{Y}\rightarrow \mathcal{Y}, \quad s \in \mathbb{R}\} \qquad \teqt{with}\qquad \Phi_{g_s}(y) = \big(\Phi_{g_s}^q(q),\, T_q^*\Phi_{g_s^{-1}}^q(\lambda)\big),
%\end{align*}
\begin{align*}
   \{{\tilde{\Phi}}_{g_s}:\mathcal{Y}\rightarrow \mathcal{Y}, \quad s \in \mathbb{R}\} \qquad \text{with}\qquad {\tilde{\Phi}}_{g_s}(y) = \big(g_sq, (g_s^{-1})^T \lambda\big),%\big(\Phi_{g_s}^q(q),\, T_q^*\Phi_{g_s^{-1}}^q(\lambda)\big),
\end{align*}
and the lift of {the} action 
\[
\begin{aligned}
	T{\tilde{\Phi}}_{g_s}(y, \dot{y}) %&= \Big(\big( \Phi_{g_s}^q(q),T_q^*\Phi^q_{g_s^{-1}}(\lambda)\big),\, \big( T_q\Phi^q_{g_s}(\dot{q}), T_yT_q^*\Phi^q_{g_s^{-1}}(\dot{\lambda})\big) \Big)\\
	&= \big( g_sq, (g_s^{-1})^T \lambda, g_s \dot{q}, (g_s^{-1})^T\dot{\lambda}\big).
\end{aligned}
\]
%where $$ T\Phi^q_g:T\mathcal{M} \rightarrow T\mathcal{M},\qquad
%			T^*\Phi^q_{g^{-1}}: T^*\mathcal{M} \rightarrow T^*\mathcal{M}.$$

Applying Noether's theorem, \textcolor{black}{see e.g. \cite{marsden94}}, to the new Lagrangian $\tilde{\mathcal{L}}$, one obtains immediately the following statement.
\begin{proposition}
\label{noetherforL}
 Let the set $G := \{ {\tilde{\Phi}}_{g_s} : \mathcal{Y} \rightarrow\mathcal{Y} ,\: s \in \mathbb{R}\}$ be a one-parameter group of transformations such that the following conditions hold:
 \begin{itemize}
     \item $ {\tilde{\Phi}}_{g_s} : \mathcal{Y} \mapsto\mathcal{Y}$ is a diffeomorphism for all $s \in \mathbb{R}$,
     \item ${\tilde{\Phi}}_{g_s}(y)$ as a function of $s$ is differentiable for all $y\in \mathcal{Y}$,
     \item ${\tilde{\Phi}}_{g_t}\circ {\tilde{\Phi}}_{g_s}={\tilde{\Phi}}_{g_{t+s}}$ for all $t,\:s \in \mathbb{R} $.
 \end{itemize}
    Moreover, assume that the Lagrangian system $(\mathcal{Y},\tilde{\mathcal{L}})$ is invariant under the action of $G$, i.e.,
    \begin{align*}
       \tilde{\mathcal{L}} \big(  {\tilde{\Phi}}_{g_s}(y), \, T_y{\tilde{\Phi}}_{g_s}(\dot{y}) \big)= \tilde{\mathcal{L}}(y,\dot{y})\quad \text{for all}\quad s\in \mathbb{R}, \: (y,\dot{y})\in T\mathcal{Y}.
    \end{align*}
  Then the momentum map
\begin{align*}
  I(y,\dot{y})= \frac{\partial \tilde{\mathcal{L}}}{\partial \dot{y}}(y,\dot{y}) \cdot \big( 
  \frac{d}{ds}g_s(y)
  \big)|_{s=0}
\end{align*}
is a conserved quantity of the motion. 
\end{proposition}
Our intention is now to find such one-parameter groups $G$ which satisfy the conditions of the statement. Thus, we need that the specific new {Lagrangian} is invariant under $G$, i.e.,
\begin{align*}
    \tilde{\mathcal{L}}\big(g_sq,(g_s^{-1})^T \lambda, g_s\dot{q}, (g_s^{-1})^T\dot{\lambda}\big) = \tilde{\mathcal{L}}\big(q, \lambda, \dot{q}, \dot{\lambda}\big).
\end{align*}
Using the equivariance of $f$ in \eqref{equivariance_f} we obtain 
\begin{align*}
    \tilde{\mathcal{L}}\big((g_sq,(g_s^{-1})^T \lambda), (g_s\dot{q}, (g_s^{-1})^T\dot{\lambda})\big)  &= -\langle(g_s^{-1})^T\dot{\lambda} ,g_s\dot{q}\rangle - \langle(g_s^{-1})^T\lambda, f(g_sq)\rangle - \frac{1}{2} \lambda^T(g_s^{-1})(g_s^{-1})^T\lambda\\
     &=
      -\langle\dot{\lambda} ,g_s^{-1}g_s\dot{q}\rangle 
      - \langle\lambda, g_s^{-1}g_sf(q)\rangle - \frac{1}{2} \lambda^T g_s^{-1} (g_s^{-1})^T\lambda\\
      &=  \tilde{\mathcal{L}}(q, \lambda, \dot{q}, \dot{\lambda}),
\end{align*}
if and only if
\begin{align*}
    g_s^{-1} (g_s^{-1})^T=Id, 
\end{align*}
thus, 
$$ g_s^T=g_s^{-1},$$
and therefore, $g_s\in O(n)$. Thus, we obtained that $\tilde{\mathcal{L}} $ is invariant under $G$ if and only if $G\subseteq O(n)$.\\

Let us consider now the rotation matrix as a special case of orthogonal matrices for our problem.
\begin{example}
    Let $\mathcal{Q}=\mathbb{R}^3$ and
$G=SO(3)$ be the one-parameter group of rotations around the axis $n\in\mathbb{R}^3$, $\| n\|_2=1$. Its action on $T\mathcal{Y}$ can be written as 
\begin{equation*}
    {\tilde{\Phi}}_{g_s}(y)={\tilde{\Phi}}_{g_s}(q,\lambda)=(\exp(s\hat{n})q,\exp(s\hat{n})\lambda ),
\end{equation*}
where  
\begin{equation*}
    \hat{n}=
\begin{bmatrix}
    0 & -n_3& n_2\\
    n_3 &0& -n_1\\
    -n_2& n_1& 0
\end{bmatrix}.
\end{equation*}
Moreover we assume the equivariance of $f$ w.r.t. $G$, i.e.,
\begin{align}
    \label{equivariancef}
    f(\exp(s\hat{n})q)=\exp(s\hat{n})f(q), \quad s\in \mathbb{R},\: q\in \mathbb{R}^3.
\end{align}
In this case, $\tilde{\mathcal{L}} $ is invariant under the action of $G$, since
\begin{align*}
   \tilde{\mathcal{L}}\big(g_sq,(g_s^{-1})^T \lambda, g_s\dot{q}, (g_s^{-1})^T\dot{\lambda}\big) = -\left(\exp(s\hat{n})\dot{\lambda}\right)\textcolor{black}{^\top}\exp(s\hat{n}) \dot{q} - \left(\exp(s\hat{n})\lambda\right)\textcolor{black}{^\top} f(\exp(s\hat{n})q) - \frac{1}{2} (\exp(s\hat{n})\lambda)^2=\tilde{\mathcal{L}}(y,\dot{y}) ,
\end{align*}  
where we used \eqref{equivariancef} and the fact that
\begin{equation*}
    \left(\exp(s\hat{n})a\right)\textcolor{black}{^\top}\exp(s\hat{n})b=a\textcolor{black}{^\top}b \quad \text{for all}\quad  a,\:b\in \mathbb{R}^3,
\end{equation*}
which can be shown e.g. by using Rodrigues' rotation formula
\begin{equation*}
    g_s=\exp(s\hat{n})=I_3+ \sin(s) \hat{n}+(1-\cos(s))\hat{n}^2.
\end{equation*}

Thus, applying Proposition \ref{noetherforL} we obtain the conserved quantity
\begin{align*}
     I(y,\dot{y})= \frac{\partial \tilde{\mathcal{L}}}{\partial \dot{y}}(y,\dot{y}) \cdot  
  \left.\frac{d}{ds}g_s(y)\right\vert_{s=0}= 
  \begin{bmatrix}
      -\dot{\lambda}\\
       -\dot{q}
\end{bmatrix}\textcolor{black}{^\top}
    \begin{bmatrix}
      \hat{n}q\\
       \hat{n}\lambda
  \end{bmatrix}= -\dot{\lambda}\textcolor{black}{^\top}(n\times q)-\dot{q}\textcolor{black}{^\top}(n\times \lambda)= n\textcolor{black}{^\top}(\dot{q}\times \lambda + \dot{\lambda}\times q).
\end{align*}
Note that in our case $\dot{q}=-p_{\lambda}$ and $\dot{\lambda}=-p_q$, see \eqref{eq:Ham.sys.}, and therefore 
\begin{align*}
n\textcolor{black}{^\top}(\dot{q}\times \lambda + \dot{\lambda}\times q)= n\textcolor{black}{^\top}(q\times p_q+\lambda\times p_{\lambda})=n\textcolor{black}{^\top}(y\times p_y)
\end{align*}
is conserved quantity, where $ (y\times p_y)$ is meant to be componentwisely evaluated.
\end{example}


\paragraph{Outlook on future work}
The method considered in the paper is based on the introduction of the new Lagrangian formulation. {It was applied to optimal control problems on vector spaces. In our future work we will extend the formulation to more general manifolds and identify the minimal requirements on the structure of the manifolds.} The approach can be {further} generalized %in future works 
to optimal control problems with general second order dynamical constraints. \textcolor{black}{Moreover, different regularity requirements are also an interesting question in future works.} It would be of particular {importance} %interest 
to consider %can be further generalized by considering 
larger classes of control, such as piecewise continuous or essentially bounded measurable functions. This can be {crucial} %important 
for applications where it is not possible to have smooth controls. In addition, 
the numerical methods can be extended with tools available in the variational integrator setting. From the latter side, higher-order variational integrators as well as multirate integrators could be applied to the optimal control problem. An extension for multisymplectic dynamics could also provide an interesting research area. Optimal time problems could be handled via an extended Lagrangian viewpoint. 

\section*{Acknowledgements}
\textcolor{black}{The authors acknowledge the support of Deutsche Forschungsgemeinschaft (DFG) with the projects: LE 1841/12-1, AOBJ: 692092 and OB 368/5-1, AOBJ: 692093.}

%\textcolor{blue}{ Use more references: e.g. \cite{a.m.bloch14} \cite{abraham78} \cite{abraham88}  \cite{ackermann06} \cite{ackermann07} \cite{ackermann10} \cite{adams12}   \cite{agricola01} \cite{alder59} \cite{algwaiz92} \cite{allabar14} \cite{allgoewer99} \cite{ambrosio07} \cite{andersen83} \cite{anderson01} \cite{angeles86} \cite{angeles88} \cite{leok_tran22}..... see the list in bibOCP.bib }


{\footnotesize{
		\makeatletter
		\renewenvironment{thebibliography}[1]
		{%
			\@mkboth{\MakeUppercase\refname}{\MakeUppercase\refname}%
			\list{\@biblabel{\@arabic\c@enumiv}}%
			{\settowidth\labelwidth{\@biblabel{#1}}%
				\leftmargin\labelwidth
				\advance\leftmargin\labelsep
				\@openbib@code
				\usecounter{enumiv}%
				\let\p@enumiv\@empty
				\renewcommand\theenumiv{\@arabic\c@enumiv}}%
			\sloppy
			\clubpenalty4000
			\@clubpenalty \clubpenalty
			\widowpenalty4000%
			\sfcode`\.\@m}
		{\def\@noitemerr
			{\@latex@warning{Empty `thebibliography' environment}}%
			\endlist}
		\makeatother

		\bibliographystyle{wmaainf}
		
		\begin{thebibliography}{aaaaaaaaaa}




   \end{thebibliography}}}

\begingroup\scriptsize

\makeatletter
\renewcommand\@openbib@code{\itemsep\z@}


%\sina{we need more references}
\bibliography{bib_OCP}
\endgroup




\end{document}