\documentclass[twocolumn, pre,floats,floatfix,aps,amsmath,amssymb, superscriptaddress]{revtex4-2}


\usepackage[colorlinks = true,linkcolor = red,citecolor = magenta]{hyperref}
\usepackage[sort&compress]{natbib}
\usepackage{graphicx}
\usepackage{epstopdf}
\usepackage{enumitem}
\usepackage{xcolor}
\usepackage{physics}


\DeclareGraphicsExtensions{.pdf,.eps,.png,.jpg,.mps}

\newcommand{\dcadd}{Department of Physics and Astronomy, Dickinson College, Carlisle, Pennsylvania, 17013, USA}
\newcommand{\pcsadd}{Center for Theoretical Physics of Complex Systems, Institute for Basic Science (IBS), Daejeon, Korea, 34126}
\newcommand{\ustadd}{Basic Science Program, Korea University of Science and Technology (UST), Daejeon 34113, Republic of Korea}

\newcommand{\yeongjun}[1]{\textcolor{orange}{#1}}
\newcommand{\shlee}[1]{\textcolor{blue}{SL : #1}}
\newcommand{\sergej}[1]{\textcolor{green}{#1}}
\newcommand{\alexei}[1]{\textcolor{purple}{AA:#1}}
\newcommand{\lars}[1]{\textcolor{brown}{#1}}
\newcommand{\panos}[1]{\textcolor{red}{#1}}
\newcommand{\carys}[1]{\textcolor{cyan}{#1}}

\DeclareMathOperator{\FB}{FB}
\DeclareMathOperator{\DB}{DB}
\DeclareMathOperator{\CLS}{CLS}

\newcommand{\appropto}{\mathrel{\vcenter{
  \offinterlineskip\halign{\hfil$##$\cr
    \propto\cr\noalign{\kern2pt}\sim\cr\noalign{\kern-2pt}}}}}


\begin{document}

\title{Supplementary Material for ``Compact Localized States in Electric Circuit Flatband Lattices''}

\author{Carys Chase-Mayoral}
    \affiliation{\dcadd}
\author{L.Q. English}
    \affiliation{\dcadd}
\author{Yeongjun Kim}
    \affiliation{\pcsadd}
    \affiliation{\ustadd}
\author{Sanghoon Lee}
    \affiliation{\pcsadd}
    \affiliation{\ustadd}
\author{Noah Lape}
    \affiliation{\dcadd}
\author{Alexei Andreanov}
    \affiliation{\pcsadd}
    \affiliation{\ustadd}
\author{P.G. Kevrekidis}
    \affiliation{Department of Mathematics and Statistics, University of Massachusetts, Amherst, Massachusetts 01003, USA}
\author{Sergej Flach}
    \affiliation{\pcsadd}
    \affiliation{\ustadd}
    
\maketitle

In this supplementary material we provide further technical details related to the main text ``Compact Localized States in Electric Circuit Flatband Lattices''.


\section{Models}

In the electrical circuit lattices we examine, each node with capacitor to ground is considered as a site of a tight-binding lattice, having voltages with respect to the ground considered as the corresponding amplitude. 
As written in Eq.~(1) of the main text, applying the Krichhoff's current law, the governing equations of motions for the diamond lattice is given by:
\begin{align}
    \ddot{T}_n +\beta \dot{T}_n &= -\omega^2_b\left[4T_n - U_n -U_{n+1} - V_n -V_{n+1}\right] \notag \\
    \ddot{U}_n \!+\beta \dot{U}_n &= -\omega^2_b\left[\left(2 + \alpha\right) U_n -\alpha V_n - T_n-T_{n-1}\right] \label{eq:diamond_undriven}
 \\
    \ddot{V}_n +\beta \dot{V}_n &= -\omega^2_b\left[\left(2 + \alpha\right) V_n -\alpha U_n - T_n-T_{n-1}\right]. \notag
\end{align}
At the driven site \(U_m\), the equation is modified as:
\begin{align}
    \label{eq:diamond_driven}
    \ddot{U}_m +\beta \dot{U}_m &= -\frac{\omega^2_b}{1+\gamma}\left(\left(2 + \alpha\right) U_m -\alpha V_m - T_m-T_{m-1} \vphantom{\sum}\right) \notag \\
    &+ A\sin(\omega_d t). 
\end{align}
As in the main text, here we have \(\omega_b^2 \!\!=\!\! 1/(L_bC)\), \(\alpha \!\!=\!\! L_b/L_r$, $\beta \!\!=\!\! R/L\), \(\gamma = C_d/C\), and \(A= \gamma(1+\gamma)^{-1}v_d\omega_d^2\). 
Here, \(\alpha\) is a tunable parameter that is able to shift the flat band, \(\beta\) represents dissipation, and \(\gamma\) is an impurity artifact that appears as a result of driving which can be minimized to within the experimental tolerance of \(\omega_b^2\).
The equation of motion is written succinctly as:
\begin{align}
    \label{eq:diamond_schrodinger_eq}
    \left(\frac{d^2}{dt^2} + \beta \frac{d}{dt} \right) \vec{\psi} = \hat{H} \vec{\psi} + \vec{F}(t),
\end{align}
where \(\vec{\psi} = (T_0, U_0, V_0, \ldots)^T\), and the \(\hat{H}\) is the matrix of the diamond lattice giving rise to the right side of Eq.~\eqref{eq:diamond_undriven} when acting on \(\vec{\psi}\).
\(\vec{F}(t)\) is the driving yielding the last term of Eq.~\eqref{eq:diamond_driven}.
Unlike the hoppings in tight-binding diamond lattices, the inductors also add to the ``onsite potential'' (diagonal elements of \(\hat{H}\)), which in turn guarantees that \(\omega^2\) is positive. 
To obtain the eigenfrequencies, the driving term is ignored, and we assume the Bloch waveform, \(U_n = U(k)e^{i(\omega t - kn)}\) (and similarly for \(V_n, T_n\)) in Eq.~\eqref{eq:diamond_undriven}, to arrive at the following eigenvalue problem:
\begin{gather}
    \frac{\left(\omega^2 - i\beta \omega\right)}{\omega^2_b}
    \begin{bmatrix}
        T(k) \\ U(k) \\ V(k)
    \end{bmatrix} \!\!
    = \!\! \begin{bmatrix}
        4 & Q & Q \\
        Q^* & (2+\alpha) & -\alpha \\
        Q^* & -\alpha & (2+\alpha)
    \end{bmatrix} \!\!\!
    \begin{bmatrix}
        T(k) \\ U(k) \\ V(k)
    \end{bmatrix}
\end{gather}
where \(Q = -1 - \exp(ik)\).
In the \(\beta = 0\) limit, we then get:
\begin{gather}
    \label{eq:diamond_eigenvalues}
    \omega_{\FB,0}^{2} = 2\omega^{2}_b(\alpha + 1),
    \hspace{0.45em} \omega_{\DB,0}^2 = \omega^2_b(3\pm \sqrt{4\cos(k) + 5}).
\end{gather}

For the diamond lattice, the translated copies of a normalized CLS is given by:
\begin{align}
    \label{eq:diamond_CLS}
    U_n = \frac{\delta_{n, n'}}{\sqrt{2}},\quad V_n = -U_n, \quad T_n = 0,
\end{align}
and these states form an orthogonal basis of the FB subspace.

A similar analysis can be performed on the stub lattice flat band. 
The equations of motion for the stub lattice are given by:
\begin{align}
    \label{eq:stub_undriven}
    \ddot{A}_n +\beta \dot{A}_n &= -\omega^2_b\left[2A_n -C_{n-1} - C_n \right] \notag \\
    \ddot{B}_n +\beta \dot{B}_n &= -\omega^2_b\left[2B_n - C_{n}\right] \\
    \ddot{C}_n +\beta \dot{C}_n &= -\omega^2_b\left[3C_n - B_n - A_{n} - A_{n+1}\right] \notag .
\end{align}
The driven site is \(B_m\) and is modified similarly to Eq.~\eqref{eq:diamond_driven}.
Using similar analysis, the 
eigenfrequencies for the stub lattice are computed as:
\begin{align}
    \label{eq:stubdisp}
    \omega^2_{\FB,0} = 2\omega^2_b,\quad \omega^2_{\DB,0} = \omega^2_b\left(\frac{5}{2} \pm \frac{\sqrt{8\cos(k) + 13}}{2}\right).
\end{align}
The blue and red traces in Fig.~\ref{fig:stubdisp} plot Eq.~\eqref{eq:stubdisp}; the black trace along the right vertical axis depicts the experimental spectrum obtained by sweeping the frequency of the driver at a CLS site.
Note that the largest experimental response at this CLS site is registered at the predicted flat-band frequency.
In contrast to the diamond lattice, here the zone-center acoustic mode occurs at a nonzero frequency and is registered in the spectrum.

Furthermore, at \(\omega_{\FB}\) the CLS in this lattice is predicted to have the form:
\begin{align}
    \label{eq:stub_cls}
    B_n = B_{n+1} = \frac{\delta_{n,n'}}{\sqrt{3}}, \quad A_{n+1} = -B_n.
\end{align}

% Figure environment removed


\section{Flat-band response to local driving}

In this section we first consider the flat-band response to local driving in general before applying it to the diamond and stub cases.

In general, the linear equation of motion Eq.~\eqref{eq:diamond_schrodinger_eq} can be solved using the impulse response method~\cite{economou2006green}:
\begin{align}
    \label{eq:green_schrodinger_eq}
    \left(\frac{d^2}{dt^2} + \beta \frac{d}{dt} 
    -\hat{H}\right) \hat{G}(t-t') = \hat{I}\delta(t - t').
\end{align}
We set \(t' = 0\).
In the frequency domain, we obtain:
\begin{align}
    \left(-\omega^2 + i\beta\omega
    -\hat{H}\right) \hat{G}(\omega) = \hat{I},
\end{align}
where \(\hat{G}(\omega)\) is the transfer function operator into which all information on the response to local sinusoidal driving is encoded.
With zero initial condition, the response to the ideal sinusoidal driving is then given as:
\begin{align}
    \label{eq:driven_psi_t}
    \vec{\psi}(t) = \hat{G}(t) * \vec{F}(t)\\
    \label{eq:driven_psi_omega}
    \vec{\psi}(\omega) = \hat{G}(\omega)\vec{F}(\omega),
\end{align}
where, \(*\) denotes convolution and where multiplications in matrix-vector multiplication are replaced by convolution, according to the convolution theorem.
Here, the transfer function operator \(\hat{G}(\omega)\) is given by:
\begin{align}
    \label{eq:green_diagonal_form}
    \begin{split}
        \hat{G}(\omega) &= (-\omega^2 + i\beta\omega - \hat{H})^{-1}\\
        & = \sum_i \frac{\ket{\psi_i}\bra{\psi_i}}{-\omega^2 + i\beta\omega + \omega^2_i}
    \end{split}.
 \end{align}
Here, we have employed the bra-ket notation for convenience, and \(i\) is the eigenvalue index having eigenvalue \(-\omega^2_i\) and eigenvector \(\psi_i\).
 In our case,
\begin{align}
    \notag G(\omega) &= \frac{\hat{P}_{\FB}}{-\omega^2 + i\beta\omega + \omega^2_{\FB}} + \sum_{j \in \DB, k}\frac{\ketbra{\psi_j(k)}{\psi_j(k)}}{-\omega^2 + i\beta\omega + \omega_j(k)^2} \\
    &= \hat{G}_{\FB}(\omega) + \hat{G}_{\DB}(\omega)
\end{align}
where \(j \in \DB = \{1, 2\}\) is the band index for the dispersive bands.
Here, we have separated the frequency responses of the flat band and dispersive bands, the left-hand side and right-hand side of the last equation being the response of flat band / dispersive band, respectively.
If we assume ideal local sinusoidal driving at \(U_n\), we can write:
\begin{align}
    \label{eq:flatband_response}
    & \vec{F}(t) = A\cos(\omega_d t)\ket{U_n} \\
    & \vec{F}(\omega) = \frac{A}{2}(\delta(\omega - \omega_d) + \delta(\omega + \omega_d))\ket{U_n}.
\end{align}
If we ignore the dispersive term \(\hat{G}_{\DB}\), which is reasonable when \(\omega_d \approx \omega_{\FB}\) and the dispersive bands are sufficiently far from the flat bands compared to the width of resonance peaks, and assuming local driving at \(Y_n\), where \(Y\) is a sublattice index, then the spatial profile is given by:
\begin{align}
    \label{eq:flatband_response_to_local_driving}
    \braket{X_m}{\vec{\psi}(\omega)} \approx \mel{X_m}{\hat{G}_{\FB}}{Y_n} \propto \mel{X_m}{\hat{P}_{\FB}}{Y_n},
\end{align}
where
\begin{align}
    \label{eq:projector_nonorthogonal}
    \hat{P}_{\FB} = \sum_{i,j} S^{-1}_{ij} \ketbra{\CLS_i}{\CLS_j}.
\end{align}
Here, \(S_{ij}\) is the overlap matrix defined as:
\begin{align}
    S_{ij} = \braket{\CLS_i}{\CLS_j}.
\end{align}

Now we consider two cases, depending on whether the CLSs are orthogonal (as in the diamond lattice) or not (as in the stub lattice).


\subsection{Orthogonal CLS -- Diamond}

For flat bands hosting orthogonal CLSs, we set (see Eq.~\eqref{eq:diamond_CLS}, for example) the overlap \(S_{ij} = \delta_{ij}\) in Eq.~\eqref{eq:projector_nonorthogonal}.
Therefore, we obtain:
\begin{align}
    \label{eq:orthogonal_cls_projector}
    \hat{P}_{\FB} = \sum_{i} \ketbra{\CLS_i}{\CLS_i},
\end{align}
giving a \emph{compact} projector~\cite{sathe2021compactly}. 
Here, \emph{compact} means that there exists an integer \(l > |n-m|\), such that
\begin{align}
    \mel{X_m}{\hat{P}_{\FB}}{Y_n} = 0
\end{align}
where \(X, Y \in \{T, U, V\}\).
This is true for the orthogonal CLSs having the projector of the form Eq.~\eqref{eq:orthogonal_cls_projector}. 
For example, for the diamond lattice, since we have from Eq.~\eqref{eq:diamond_CLS}:
\begin{align}
    \ket{\CLS_i} = \frac{1}{\sqrt{2}}\left(\ket{U_i} - \ket{V_i}\right),
\end{align}
we now get:
\begin{align}
    \label{eq:diamond_projector_realspace}
    \notag \mel{X_m}{\hat{P}_{FB}}{Y_n} = \sum_i \braket{X_m}{\CLS_i}\braket{\CLS_i}{Y_n}\\
    = \sum_i \frac{1}{2}\delta_{i,m}\delta_{i,n}(\delta_{X,U} - \delta_{X,V})(\delta_{Y,U} - \delta_{Y,V}),
\end{align}
which is zero except when \(m = n\).
One obtains the CLS response of the diamond lattice from Eq.~\eqref{eq:flatband_response}.
The response amplitude of the CLS is given by:
\begin{align}
    \label{eq:diamond_CLS_response}
    |\braket{U_n}{\psi(\omega)}| = \frac{\gamma(1+\gamma)^{-1}v_d \omega^2_d}
    {2\sqrt{(\omega_{FB}^2 - \omega_d^2)^2 + \beta^2\omega_d^2}},
\end{align}
where the factor of 2 in the denominator comes from Eq.~\eqref{eq:diamond_projector_realspace}.
For the parameters in the experimental setup, we have \(\gamma = 0.015\), \(v_d = 1 \textrm{ V}\), \(\omega_d = \omega_{\FB} = 2\pi \times 429 \textrm{ kHz}\), \(\beta = R/L_b = 49356\textrm{/sec}\), the maximum CLS response is given by \(0.4\textrm{ V}\), which is in excellent agreement with the experiment and simulation within 10~\%.


\subsection{Non-orthogonal CLS -- Stub}

When we consider the overlap between nearest neighboring CLSs, \(\sigma\) is nonzero (see Eq.~\eqref{eq:stub_cls}, for example), and hence we can write for \(S_{ij}\) of Eq.~\eqref{eq:projector_nonorthogonal} the tridiagonal matrix:
\begin{align}
    \label{eq:overlap_matrix_nearest_neighbor}
    S_{ij} = \braket{\CLS_i}{\CLS_j} = \delta_{ij} + \sigma \delta_{i \pm 1, j}.
\end{align}
We can conclude that the spatial profile is determined by the overlap of CLSs.
If the CLSs are orthogonal (\(\sigma = 0\)), \(S\) is a diagonal matrix; hence the projector is compact in real space as was the case of the diamond lattice. 
On the other hand, when there is overlap, one can work out the details to obtain the inverse of \(S\).
In fact, \(S\) is a tridiagonal matrix with translational invariance. 
The inverse can be obtained easily in the Bloch basis:
\begin{align}
    S^{-1}_{ij} = \int^{\pi}_{-\pi} \frac{dk}{2\pi}\frac{\exp(ik|i-j|)}{-1 - 2\sigma \cos (k)}.
\end{align}
Here, we assume that the system size is infinite.
The integration can be solved in the complex plane using Cauchy's integral formula with substitution~\cite{sheng2007introduction}:
\begin{align}
    \notag \omega &= \exp(ik)=\cos(k) + i\sin(k) \\
    d\omega &= i\omega dk
\end{align}
The integration range becomes a unit circle (C) in the complex plane:
\begin{align}
    S^{-1}_{ij} = -\oint_C \left(\frac{d\omega}{2\pi i}\right)\frac{\omega^{|i-j|}}{\sigma\omega^2+ \omega + \sigma}
\end{align}
This gives us:
\begin{align}
    S^{-1}_{ij} \propto e^{|i-j|/\xi},
\end{align}
where
\begin{align}
    \xi^{-1} = \ln \abs{\frac{2\sigma}{-1 + \sqrt{1 - 4\sigma^2}}} 
\end{align}
The conclusion is that for flat bands hosting non-orthogonal CLSs, the projector is exponentially localized.
For the stub lattice, from Eq.~\eqref{eq:stub_cls} we have \(\sigma = 1/3\) and thus \(\xi \approx 1.03\).


\section{CLS in nonlinear diamond lattice}

% Figure environment removed

We now provide more detail on how the symmetric nonlinearity arising from the capacitance-voltage relationship of the diode-pair setup is theoretically modelled.
As explained in the main text, the capacitors were replaced with varactor-diode pairs exhibiting a symmetric nonlinear capacitance. 
In the equations of motion for the undriven diamond lattice, Eq.~\eqref{eq:diamond_undriven}, for example at \(U_n\), the following parameter must then be modified:
\begin{align}
    \omega^2_b \approx \sum^{\infty}_{i} g_i U_n^{2i} \approx g_0 + g_1 U_n^2
\end{align}
Our CLS ansatz is \(U_n = \delta_{n,n'}f(t)\), \(V_n = -U_n\) and \(T_n = 0\), where \(f(t)\) is some periodic function of time arising as a result of nonlinearity.
If we insert this ansatz into the original equation of motion, we get:
\begin{align}
    \label{eq:diamond_duffing}
    \ddot{U}_n + \beta \dot{U}_n = -2g_0U_n - g_1U^3_n.
\end{align}
This is a damped version of the well-known Duffing oscillator.
Note that once we find a solution to Eq.~\eqref{eq:diamond_duffing}, the negative of it is also a solution.
This is because Eq.~\eqref{eq:diamond_duffing} only contains odd powers of \(U_n\) due to the symmetric nature of the nonlinearity.
Therefore, \(U_n\) and \(V_n = -U_n\) both satisfy Eq.~\eqref{eq:diamond_duffing} and exactly cancel out at the bottleneck sites of \(T_n\) and \(T_{n+1}\).
Therefore, this nonlinear CLS is a solution to the undriven nonlinear diamond chain.

Fig.~\ref{fig:nonlineardiamond} shows the simulation results of the undriven nonlinear diamond chain.
We assumed \(g_0 = \omega_b^2\), as in linear case, Eq.~\eqref{eq:diamond_undriven} and the strength of nonlinearity \(g_1 = \omega_b^2\).
The initial condition is the CLS eigenstate of the linear case, \(U_1 = -V_1 = 10\)~V, and zero for the rest of the sites. 
The dissipation is ignored, by setting \(\beta = 0\).
The observed shift of the resonance frequency is from 429~kHz to 578~kHz, with only odd numbers of higher harmonics as in the inset of Fig.~\ref{fig:nonlineardiamond} (a).
The out-of-phase relationship of the CLS, \(U_1(t) = -V_1(t)\) is clear from Fig.~\ref{fig:nonlineardiamond}(b).


\bibliography{general, flatband, local}

\end{document}
