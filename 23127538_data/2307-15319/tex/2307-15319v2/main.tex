\documentclass[twocolumn,pre,floats,floatfix,aps,amsmath,amssymb, superscriptaddress]{revtex4-2}

\usepackage[colorlinks = true,linkcolor = red,citecolor = magenta]{hyperref}
\usepackage[sort&compress]{natbib}
\usepackage{graphicx}
\usepackage{epstopdf}
\usepackage{enumitem}
\usepackage{xcolor}
\usepackage{physics}


\DeclareGraphicsExtensions{.pdf,.eps,.png,.jpg,.mps}

\newcommand{\pcsadd}{Center for Theoretical Physics of Complex Systems, Institute for Basic Science (IBS), Daejeon, Korea, 34126}
\newcommand{\ustadd}{Basic Science Program, Korea University of Science and Technology (UST), Daejeon 34113, Republic of Korea}

\newcommand{\yeongjun}[1]{\textcolor{orange}{#1}}
\newcommand{\shlee}[1]{\textcolor{blue}{SL : #1}}
\newcommand{\sergej}[1]{\textcolor{green}{#1}}
\newcommand{\alexei}[1]{\textcolor{purple}{AA:#1}}
\newcommand{\lars}[1]{\textcolor{brown}{#1}}
\newcommand{\panos}[1]{\textcolor{red}{#1}}
\newcommand{\carys}[1]{\textcolor{cyan}{#1}}

\DeclareMathOperator{\FB}{FB}
\DeclareMathOperator{\DB}{DB}

\newcommand{\appropto}{\mathrel{\vcenter{
  \offinterlineskip\halign{\hfil$##$\cr
    \propto\cr\noalign{\kern2pt}\sim\cr\noalign{\kern-2pt}}}}}

\newcommand{\sect}[1]{\textit{#1} --}


\begin{document}

\title{Compact Localized States in Electric Circuit Flatband Lattices}

\author{Carys Chase-Mayoral}
    \email{chasemac@dickinson.edu}
    \affiliation{Department of Physics and Astronomy, Dickinson College, Carlisle, Pennsylvania, 17013, USA}
\author{L.Q. English}
    \email{englishl@dickinson.edu}
    \affiliation{Department of Physics and Astronomy, Dickinson College, Carlisle, Pennsylvania, 17013, USA}
\author{Yeongjun Kim}
    \email{yeongjun.kim.04@gmail.com}
    \affiliation{\pcsadd}
    \affiliation{\ustadd}
\author{Sanghoon Lee}
    \email{scott430@naver.com}
    \affiliation{\pcsadd}
    \affiliation{\ustadd}
\author{Noah Lape}
    \email{lapenoah@dickinson.edu}
    \affiliation{Department of Physics and Astronomy, Dickinson College, Carlisle, Pennsylvania, 17013, USA}
\author{Alexei Andreanov}
    \email{aalexei@ibs.re.kr}
    \affiliation{\pcsadd}
    \affiliation{\ustadd}
\author{P.G. Kevrekidis}
    \email{kevrekid@umass.edu}
    \affiliation{Department of Mathematics and Statistics, University of Massachusetts, Amherst, Massachusetts 01003, USA}
\author{Sergej Flach}
    \email{sflach@ibs.re.kr}
    \affiliation{\pcsadd}
    \affiliation{\ustadd}

\begin{abstract}
    We generate compact localized states in an electrical diamond lattice, comprised of only capacitors and inductors, via local driving near its flatband frequency.
    We compare experimental results to numerical simulations and find very good agreement. 
    We also examine the stub lattice, which features a flatband of a different class where neighboring compact localized states share lattice sites. 
    We find that local driving, while exciting the lattice at that flatband frequency, is unable to isolate a single compact localized state due to their non-orthogonality.
    Finally, we introduce lattice nonlinearity and showcase the realization of nonlinear compact localized states in the diamond lattice. 
    Our findings pave the way of applying flatband physics to complex electric circuit dynamics.
\end{abstract}

\maketitle

\sect{Introduction} 
Flatbands (FBs) arise in certain tight-binding lattices in the form of completely degenerate energy bands ~\cite{leykam2013flat, leykam2018artificial}.
In the context of band theory, a FB signals zero group-velocity, infinite effective mass, and no transmission or transport.
Indeed, even though we expect plane wave eigenstates owing to the translational invariance, macroscopic degeneracy of flatbands allows for the existence of compactly localized eigenstates (CLS) at the flatband energy.
The flatness of the band also means infinite sensitivity to perturbations. 
A wide range of nontrivial phenomena related to FB physics has been reported, including ferromagnetism~\cite{derzhko2015strongly, lieb1989two, mielke1991ferromagnetism, tasaki1992ferromagnetism, mielke1999stability}, superfluidity and superconductivity~\cite{peotta2015superfluidity, julku2016geometric, tovmasyan2018preformed, mondaini2018pairing, aoki2020theoretical}, localization-delocalization transition at weak disorder~\cite{goda2006inverse,cadez2021metal, kim2022flat, lee2023critical, lee2023critical2}, 
many-body flat-band localization~\cite{kuno2020flat_qs, danieli2020many, vakulchyk2021heat, danieli2022many}, symmetry-breaking transitions~\cite{vicencio2013discrete,nguyen2018symmetry}, and compact discrete breathers~\cite{danieli2021nonlinear, danieli2018compact}, among others, as discussed, e.g., in recent reviews such as~\cite{derzhko2015strongly,leykam2018artificial,leykam2018perspective}.

Given the current multi-pronged focus on FB physics, the experimental realization of artificial FB lattices becomes a priority. 
The relevant construction is challenging because FB lattices need to be carefully fine-tuned in order to preserve the CLS over long times.
There were several attempts to realize them in various experimental contexts, typically over short times, without crucial relative phase information needed for the CLS, and sometimes even lacking sufficient spatial resolution.
Examples are photonic lattices~\cite{nakata2012observation, mukherjee2015observation,kajiwara2016observation, vicencio2015observation, nguyen2018symmetry, ma2020direct}, cold atoms~\cite{taie2015coherent, ozawa2017interaction}, polariton condensates~\cite{baboux2016bosonic, masumoto2012exciton}, 
electrical circuits~\cite{zhang2023non, wang2022observationof, wang2019highly} and topological material~\cite{kang2020topological}, as well as magnonic~\cite{tacchi2023experimental} crystal lattices.
Electrical circuits provide a particularly promising experimental avenue for studying flatbands and CLS in detail, due to the relative ease of constructing lattices of various topologies, the feasibility of fine-tuning lattice parameters, as well as the precise experimental control and measurement achievable there.

In this Letter, we experimentally construct and characterize one-dimensional FB electrical lattices comprised of discrete capacitive and inductive circuit elements and excite FB eigenstates via local, sinusoidal driving at the FB frequency.
We report results on two lattice structures - the diamond and stub lattices - which belong to different FB classes.
The diamond lattice contains orthogonal CLSs and exhibits CLS resonant modes when locally driven at the FB.
The stub lattice, however, features non-orthogonal CLSs and shows exponentially localized resonant modes due to overlapping CLSs.
Finally, we impart nonlinearity to the studied lattices by replacing the capacitors with varactor diodes characterized by a voltage-dependent capacitance.
Here we demonstrate that CLSs can be continued in the diamond lattice into the highly nonlinear regime and therefore showcase the robustness of the relevant linear mechanism over a wide range of excitation amplitudes.

% Figure environment removed


\sect{Models}
In the electrical-lattice context, vertices and edges of a tight-binding lattice represent capacitors and inductors, respectively.
Thus, the lattices can be thought of as discrete examples of electrical transmission lines, albeit bearing nontrivial geometry.
This is illustrated in Fig.~\ref{fig:setup}(b), showing a diamond lattice with two different hopping values, depicted as black and red lines;
Fig.~\ref{fig:setup}(a) shows the corresponding electrical circuit.
The capacitance of each node is \(C\), and the lattice also incorporates inductors of two different inductance values, \(L_b\) and \(L_r\), shown in the figure as black and red, respectively.
The main source of dissipation in the lattice is the ferrite core inductors.
Therefore, the inductors are assumed to have an effective serial resistance (ESR) \(R\), and the capacitors are considered ideal.

The voltages at the three nodes of the \(n\)-th unit cell are denoted as \(T_n, U_n, V_n\).
The lattice is driven at one site (\(U_m\)) participating in the two-site CLS located at the \(m\)-th unit cell, via a small driving capacitor of capacitance \(C_d \ll C\).
Using Kirchhoff's current law at each node, the equations of motion for the voltages at each nodes of the \(n\)-th unit cell at the linear level take the form:
\begin{align}
    \label{eq:diamond_undriven}
    \ddot{T}_n +\beta \dot{T}_n &= -\omega^2_b\left[4T_n - U_n -U_{n+1} - V_n -V_{n+1}\right] \notag \\
    \ddot{U}_n \!+\beta \dot{U}_n &= -\omega^2_b\left[\left(2 + \alpha\right) U_n -\alpha V_n - T_n-T_{n-1}\right] \\
    \ddot{V}_n +\beta \dot{V}_n &= -\omega^2_b\left[\left(2 + \alpha\right) V_n -\alpha U_n - T_n-T_{n-1}\right] \notag.
\end{align}
At the driven site \(U_m\), the rhs term in (\ref{eq:diamond_undriven}) gets a correction factor \(1/(1+\gamma)\) and an additive driving force \(A\sin(\omega_d t)\) (see~\cite{supp} for details).
Note that \(\omega_b^2 \!\!=\!\! 1/(L_bC)\), \(\gamma = C_d/C\), and \(A= \gamma(1+\gamma)^{-1}v_d\omega_d^2\).
%Note that \(\omega_b^2 \!\!=\!\! 1/(L_bC)\), \(\alpha \!\!=\!\! L_b/L_r\), \(\beta \!\!=\!\! R/L\), \(\gamma = C_d/C\), and \(A= \gamma(1+\gamma)^{-1}v_d\omega_d^2\).
Here, \(\alpha=L_b/L_r\) is a tunable parameter that is able to shift the flatband, \(\beta = R/L\) represents dissipation.
Note that unlike the hoppings in tight-binding diamond lattices, the inductors also add to the ``onsite potential'' (diagonal elements of \(\hat{H}\)), which in turn guarantees that \(\omega^2\) is positive.
To obtain the eigenfrequencies, the driving term is ignored, and we assume the Bloch waveform, \(U_n = U(k)e^{i(\omega t - kn)}\) (and similarly for \(V_n, T_n\)).
For \(\beta = 0\) we find:
\begin{align}
    \label{eq:diamond_eigenvalues}
    \omega_{\FB,0}^{2} = 2\omega^{2}_b(\alpha + 1),
    \hspace{0.45em} \omega_{\DB,0}^2 = \omega^2_b(3\pm \sqrt{4\cos(k) + 5}).
\end{align}
For the dissipative (\(\beta \neq 0\)) case, one solves \(\omega^2 - i\beta\omega = \omega^2_{\FB/\DB,0}\):
\begin{align}
    \label{eq:diamond_eigenvalues_complex}
    \omega_{\FB/\DB} = i\frac{\beta}{2} \pm \sqrt{-\frac{\beta^2}{4} + \omega^2_{\FB/\DB,0}},
\end{align}
giving the dissipation time \(\tau = 2/\beta\), and the real part which is shifted (\(< 1\%\) in our experiment) from Eq.~\eqref{eq:diamond_eigenvalues}.
Throughout the paper \(\omega_{\FB/\DB}\) is assumed to be \(\omega_{\FB/\DB,0}\)~\footnote{We have assumed only underdamped frequencies making the square root part always real, i.e. frequencies that are not too close to DC.}.

While the above FB eigenvectors correspond to spatially extended Bloch waves,
the degeneracy of the flatband allows for CLS eigenstates.
For the diamond lattice, the CLS is given by \(U_n =\delta_{n, n'}\), \(V_n = -U_n\) and \(T_n = 0\) for any unit cell \(n'\), as we will see below in further detail in Fig.~\ref{fig:diamond_result}.

A similar analysis can be performed on the stub lattice flatband, which is shown schematically in Figs.~\ref{fig:stub}(a) and (b).
The electrical circuit implementation is shown in Fig.~\ref{fig:stub}(b).
Note the addition of the inductors shown there in red.
In fact, a close examination of the governing equation reveals that all sites of the CLS (shown in Fig.~\ref{fig:stub}(a)) must be connected to the same number of inductors and must therefore have the same ``onsite potentials''.
In the stub lattice, there are three CLS sites, and an inductor has to be added to the \(B_n\) sites, so that all CLS-sites have the same network-degree of \(2\).

% Figure environment removed


\sect{Experimental and Numerical Results: diamond}
We construct an electrical-circuit diamond lattice consisting of \(N = 5\) unit cells, with periodic boundary conditions.
The lattice thus incorporates \(3N = 15\) capacitors, with the value of the capacitance \(C=1 \pm 0.01\) nF; the driving capacitor of \(C_d=15\) pF then yields \(\gamma = 0.015\).
The two different inductance values are given by \(L_b=466\) \(\mu\)H and \(L_r=674\) \(\mu\)H, within a 1\% tolerance. 

The main sources of dissipation of the inductors are the ferrite cores and coil-wire resistance, diminishing the quality factor, defined as \(Q=\omega L/R_\mathrm{eff}\).
The \(Q\) factor of an inductor remains essentially constant while the effective series resistance (ESR) varies according to the resonant frequency.
Measured with a \(Q\) of roughly \(55\) at \(232\) kHz, this translates to an effective series resistance (ESR) \(R_\mathrm{eff}\) of about \(23\) \(\Omega\) for the \(L_b\) inductor~\cite{bourns}.
Additionally, a 10 k\(\Omega\) resistor is placed in parallel with the lattice in order to suppress a DC voltage component (and charge buildup on the capacitors).
With these values, we obtain the band structure using Eq.~\eqref{eq:diamond_eigenvalues}, which is shown as red (dispersive)  and blue (flatband) lines in Fig.~\ref{fig:diamond_result}(a).
Specifically, the flatband frequency is calculated as \(f_{\FB} = 429\) kHz, which falls within the spectral gap between the two dispersive bands.

In order to experimentally probe the existence of a flatband and excite the flatband CLS, we locally supply energy in the form of a sinusoidal voltage input from a signal generator (Agilent 33220A function/sweep generator), as shown in Fig.~\ref{fig:setup}.
The lattice excitation and measurements are schematically displayed in Fig.~\ref{fig:diamond_result}. 
We also monitor the response voltage at each lattice site from \(0\) to \(14\), which corresponds to \(U_0, V_0, T_0, \ldots, U_4, V_4, T_4 \), simultaneously with a 16-channel data acquisition system (NI PXI-1033 with NI 6133 cards) at a \(2.5\) MHz sampling rate. 
The driving voltage is injected into the lattice at the fourth site (this site index corresponds to \(U_1\)).
In this way, the flatband response shows up prominently, but other extended-wave modes can also be excited.
Note that while we draw a continuous band structure, there are only \(N=5\) resonance peaks per band (with two-fold degeneracy of peaks at \(k = 2\pi/5, 4\pi/5\) for the dispersive bands), which satisfies the periodic boundary conditions, so we expect to observe \(\lceil N/2 \rceil=3\) peaks per band.

Let us now turn to the impact of the frequency of local driving \(f_d\).
We used the function generator in sweep mode, sweeping from \(200\) to \(600\) kHz within \(25\) ms, and obtained the steady state amplitude responses of site \(4\), which was obtained with an oscilloscope (no DAQ card), shown by the black trace along the right vertical axis of Fig.~\ref{fig:diamond_result}(a).
The flatband frequency prediction is accurately matched.
This resonance peak strength depends on a number of parameters: dissipation, driving voltage, and amplitude of resonant eigenvector at \(U_m\).
We observe the largest peak reaching \(0.4\) V at \(429\) kHz.
Two other prominent modes in the acoustic branch are also visible at \(k=2\pi/5, 4\pi/5\), while the resonance modes at the upper dispersive band are not as easily visible.

We now tune the function generator to the frequencies of the observed resonance peaks.
The data shown in the color-image panels depict the spatial patterns observed at various drive frequencies after reaching steady state.
Here, the voltage is represented by color, with red indicating positive, blue negative, and black zero voltage, as depicted in the panels (1)-(5) of Fig.~\ref{fig:diamond_result}.
The color map is re-scaled to the peak amplitude.
For panels (4) and (5) the inset gives the spatial voltage profile at a moment in time and elucidates the corresponding \(k\)-value of the Bloch eigenfunction.

At \(f_d = 429\) kHz, which corresponds to the largest resonance peak located at the flatband, the associated CLS is predicted to sit at the two vertical sites \(U_n, V_n\) of a single unit cell, with their respective excitations being out of phase. Figs.~\ref{fig:diamond_result}(b), (c) compare experiment and numerics, respectively.
In Fig.~\ref{fig:diamond_result}(b), the red trace depicts the response of the driven site, whereas the blue trace is the other CLS site.
Here, the traces depict the voltage-time profiles of all \(14\) sites with time folded back into the interval of \((0, 2/f_d)\).
Notice that the two are exactly out-of-phase, so that  destructive interference will occur at the neighboring bottleneck sites (\(T_n, T_{n-1}\)).
However, we do observe a small amount of leakage out to the rest of the lattice (black traces).
This is due to disorder that is introduced through experimental imperfections, and the dissipation which broadens the dispersive resonance peaks.
Nevertheless, a very high degree of localization of energy is observed.
Note also that in order to generate the CLS it is essential to drive at a site that participates in the CLS. 
For instance, if we drive at \(T_n\), the CLS cannot be obtained at any driver frequency.

The corresponding numerical simulation result is shown in Fig.~\ref{fig:diamond_result}(c).
We see excellent agreement between experiment and simulation. 
In the simulation, in the limit of \(\beta = 0\) all sites other than the CLS sites approach identically zero voltages, generating a true CLS where the energy is perfectly localized on the two CLS sites, oscillating at the flatband frequency of \(429\) kHz. 
Note that the experimental frequency is shifted down slightly to \(401\) kHz, due to small parasitic capacitances associated with the measurement apparatus (ribbon cables and DAQ board).

% Figure environment removed


\sect{Experimental and Numerical Results: stub}
Figure~\ref{fig:stub}(a) shows the stub lattice in circuit form and schematically, respectively. 
For the stub case, only one type of inductor of inductance \(L_b\) is required.
As mentioned before, the additional inductor for the `\(C\)' sublattices to ground is necessary to have a flat band.
With their inclusion, we can verify pictorially that the three capacitors involved in the CLS all have two inductor connections, indicating the same ``onsite potentials'' for these sites (whereas the connecting nodes have three inductors), and a flat band appears at \(330\) kHz. 

The CLS involves three lattice sites, as depicted in Fig.~\ref{fig:stub}(b).
However, a local drive is not able to target one unique CLS in the lattice. 
Two neighboring CLSs share one site in common and are thus not orthogonal in the stub chain.
A driver injected at site \(6\) (red) shown in Fig.~\ref{fig:stub}(a) is expected to partially excite the CLSs to the right and left.
This is illustrated in Fig.~\ref{fig:stub}(c), where a driver frequency and amplitude of \(312\)~kHz and \(1\)~V were used, respectively.
Again the traces depict the voltage-time profiles of all \(14\) sites for two periods.
As is apparent in the upper panel, when driving site 6 (red), neighboring CLS are also excited (here shown in yellow and blue).
This happens because sites 5 and 8 are shared with neighbouring CLSs, and despite the driving site 6 being a unique site of the original CLS.
Note that due to slight inhomogeneities, the two neighboring CLS are not excited to the same amplitude.
We computed the Green's function projected onto the flat band, and confirm that local driving excites all CLS states with exponential decay of their amplitudes~\cite{supp}.
This localized resonance mode is qualitatively different from what is expected to be observed in resonance modes of dispersive bands with high dissipation, since the flat-band localization exists even when the dissipation is absent. 

% Figure environment removed


\sect{Nonlinear CLS}
We now return to the diamond lattice to explore the effect of nonlinearity on the CLS.
In the literature, it has been shown~\cite{danieli2018compact} that linear CLSs can be continued into the nonlinear regime for FBs gapped away from the dispersive spectrum.
The lattice shown in Fig.~\ref{fig:setup}(a) can be made nonlinear by replacing the capacitors with varactor diodes. 
For our purposes here, it is advantageous to replace each capacitor with two diodes facing each other, as shown in the inset of Fig.~\ref{fig:nonlinear}(a). 
A resistor from the junction between the diodes to ground is necessary to prevent a DC charge buildup. 
Such an arrangement is known to produce a hard-type nonlinearity~\cite{fukushima1980envelope} because the effective capacitance decreases symmetrically as the voltage becomes either more positive or negative. 
Assuming nonlinearity strength is not too strong, this replaces \(\omega^2_b\) with \(\omega^2_{b, 0}(1 + gU^2_n + \ldots)\) in Eq.~\eqref{eq:diamond_undriven} (similarly for \(V_n, T_n\)), where \(g > 0\) for hard-type nonlinearity.

This effect is experimentally demonstrated here by using these diodes together with a \(680\) \(\mu\)H inductor to build an RF-resonator, driving via a sweep generator and a linear capacitor, and then recording the resulting resonance curves. 
As illustrated in Fig.~\ref{fig:nonlinear}(a), at low driving amplitude, the sweep produces a symmetric peak (black). 
Note that the frequency is shifted up because the effective capacitance of the diodes in series is reduced to around \(400\)~pF. 
As the driving amplitude is increased (red), a shift in the curve towards higher frequencies is observed. 
At the largest driving amplitude (\(10\)~V, blue and yellow traces, now depicting only amplitude for visual clarity), a significant bistability window opens up (\(320\) -- \(380\) kHz), with hysteresis evident in the up and down sweeps (blue and yellow, respectively).

We now demonstrate that the CLS in the diamond lattice can be continued into the nonlinear regime. 
Panels (b) of Fig.~\ref{fig:nonlinear} capture the CLS at small (\(v_d=1\)~V, left panel) and large (\(v_d=7\)~V, right panel) driving amplitudes, respectively.
In the latter case, the driving frequency has to be increased by about 10\% (from \(560\) to \(617\)~kHz) due to the nonlinear frequency shift, but the spatial structure of the CLS does not appear to be altered. 
The spectral composition of the CLS does, however, show the emergence of harmonics with larger amplitude (see insets), a tell-tale sign  of the impact of the nonlinearity on this structure. 
At \(v_d=7\) V, the third harmonic \(3f_d\) is prominent, and at higher driving strengths still, more harmonics appear in the spectrum, confirming the departure from the linear sinusoidal response. 
Note that lowering the amplitude while remaining at \(617\) kHz destroys the CLS.
In the undriven case, it can be shown theoretically that the form of the CLS is still an exact solution in the presence of this type of symmetric nonlinearity.
This is because the symmetric nonlinearity still allows for the out of phase solution \(U_{n'} = -V_{n'}\), causing perfect destructive interference at the neighbouring bottleneck sites \(T_{n'}\) and \(T_{n'+1}\).


\sect{Conclusions and future directions}
We constructed 1D flatband lattices using complex electrical circuits and observed resonant modes through local sinusoidal driving.
The results agree very well with theoretical predictions, and numerical simulations.
In a diamond lattice, we confirmed that the driving at the flatband frequency excites a CLS.
In the stub lattice the lack of orthogonality (due to spatial overlap) of neighbouring CLSs precludes the observation of individual CLSs, leading instead to resonance modes with exponentially localized spatial profiles.
Finally, we found that CLSs persist in the diamond lattice when nonlinearity is introduced by replacing the capacitors with varactor diodes exhibiting symmetrical nonlinearity of capacitance.

The clarity, as well as the qualitative and quantitative correspondence between theory, numerics and computations affirms the 
particular relevance of such linear, and {\it especially through our work} nonlinear electrical lattices as a fruitful platform for exploring flat bands and CLSs.
Naturally, numerous open questions emerge from the present work, including, e.g., whether a CLS can be distilled suitably in stub lattices, and whether more complex quasi-one-dimensional~\cite{morales2016simple}, but also two-dimensional structures (such as the Lieb lattice)~\cite{vicencio2015observation} can also be engineered.
In such settings the fate of linear and, perhaps especially, nonlinear flatband states remains an appealing and challenging task for future considerations.


\sect{Acknowledgement}
The authors acknowledge the financial support from the Institute for Basic Science (IBS) in the Republic of Korea through the project IBS-R024-D1.
This material is based upon work supported by the US National Science Foundation under Grants DMS-2204702 and PHY-2110030 (P.G.K.).


\bibliography{flatband, mbl, general, local}

\end{document}
