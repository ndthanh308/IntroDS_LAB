\documentclass[aps,prl,onecolumn]{revtex4-2}

\usepackage{xcolor}
%\usepackage{amsmath, amssymb, graphicx, xcolor, float}

\renewcommand{\r}{\right}
\renewcommand{\l}{\left}

\newcommand{\gml}[1]{\textcolor{blue}{#1}}
\newcommand{\sergej}[1]{\textcolor{brown}{#1}}

\pagenumbering{gobble}


\begin{document}

\textcolor{white}{ }
\vspace{1cm}

\noindent 
Dear Editors,

\bigskip 
\bigskip

\noindent

We are pleased to submit our paper, entitled ``Compact Localized States in Electric Circuit Flatband Lattices'' to Physical Review Letters.
Our work presents the first systematic and detailed experimental observation of tunable compact localized states (CLS) in flatband electric circuit networks, which is in excellent agreement with our modeling approach. 
This is an important milestone in the young and active field of flat band physics. It is also of current interest for the general design of tailored circuit networks for potential novel electronic applications.


Flatbands arise in a wide range of suitably tuned tight-binding lattices of relevance to a broad number of applications.
The existence of flat bands usually guarantees the existence of compact localized eigenstates (CLS) which are of particular interest due to their extreme spatial localization.
Flatband systems have attracted growing attention in recent years in the condensed matter, optics and photonics communities as building blocks for various exotic phenomena.
This has led to attempts to create artificial flatband systems in various experimental setups.
The systematic experimental realization of physical settings possessing flatbands holds paramount importance in this vein of research, as it enables to both address the practical implementation of such concepts and its correspondence to the theoretical propositions, 
but also introduces additional important features worth exploring such as, e.g., the role of nonlinearity to the original CLS.


Our manuscript provides the first systematic and detailed experimental demonstration of compact localized states (CLS) in one-dimensional flatband electric circuits.
The key importance of realizing a CLS in the experiment is to provide a strong evidence that our experimental setup truly represents a flatband system. 
Another key feature of our work concerns the examination of how flatbands respond to the (local in space) sinusoidal driving.
Importantly, we demonstrate the robustness of CLS states to adding nonlinear electronic varactor elements.
Our experimental results are thoroughly analyzed by means of analytical studies and numerical simulations, which provide excellent agreement with the experiment.
Further supporting analysis corroborating the presented ideas is provided in detail in the Supplementary Material. 

The complexity of our experiments and simulations, with stringent tests that we include in a comprehensive supplemental material, is nevertheless not reflected in the simplicity and generality of our results, which are valid for an even broader family of flatband systems. 
They should be of interest to most physicists working on related topics of macroscopic degeneracies, flatbands, and potentially many-body interactions therein. 
We are, therefore, convinced that our manuscript fulfills all the requirements demanded of a Letter and should be warranted publication in your prestigious journal. We will be accordingly looking
forward to your editorial decision on this manuscript.
\\
\\
Sincerely, 

\bigskip

The authors.


\end{document}