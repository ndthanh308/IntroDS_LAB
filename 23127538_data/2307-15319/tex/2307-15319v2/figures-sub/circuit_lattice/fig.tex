%---------------------------------------------------------------------------%
% LaTeX Standalone Template for TikZ/PGFPlots (Version 1.1)
% Author : Sanghoon (http://stelladuck.tistory.com)
% License: CC BY-NC-SA 3.0 (http://creativecommons.org/licenses/by-nc-sa/3.0/)
%---------------------------------------------------------------------------%
%
% compile command : lualatex -shell-escape -synctex=1 -interaction=nonstopmode src.tex
%
%---------------------------------------------------------------------------%
% NECESSARY PACKAGES                                                        %
%---------------------------------------------------------------------------%
\documentclass{standalone}
%---------------------------------------------------------------------------% ::: FUNDAMENTAL PREAMBLE :::
\usepackage{amsmath}                      % Required for some math elements: I
\usepackage{amssymb}                      % Required for some math elements: II
\usepackage{mathtools}                    % Required for some math elements: III
\usepackage{pgfplots}
\usepackage{graphicx}
\usetikzlibrary{pgfplots.groupplots}
\pgfplotsset{compat = newest, width = 300pt}
\usepackage{pgfplotstable}
\usepackage{circuitikz}
%---------------------------------------------------------------------------% :::PRACTICAL PREAMBLE :::
\usepackage{xcolor}															% Required for applying different colors
%---------------------------------------------------------------------------% ::: USER SETTINGS :::
%\usepgfplotslibrary{external} 
%\tikzexternalize
%
%
%---------------------------------------------------------------------------%
% DOCUMENT CONFIGURATION                                                    %
%---------------------------------------------------------------------------%
%
%
%
%---------------------------------------------------------------------------% ::::: START DOCUMENT :::::
\begin{document}
	\begin{circuitikz}
            \draw[line width=0.3mm] (0,3) -- (-1,3);
            \draw[line width=0.3mm] (0,0) -- (0,0) node[ground]{};
            \draw[line width=0.3mm] (0,0) to[capacitor] (0,3);
            \draw[line width=0.3mm] (0,3) to[inductor] (3,3);

            \draw[line width=0.3mm] (3,0) -- (3,-0.01) node[ground]{};
            \draw[line width=0.3mm] (3,0) to[capacitor] (3,3);
            \draw[line width=0.3mm] (3,3) to[inductor] (6,3);

            \draw[line width=0.3mm] (6,0) -- (6,-0.01) node[ground]{};
            \draw[line width=0.3mm] (6,0) to[capacitor] (6,3);
            \draw[line width=0.3mm] (6,3) -- (7,3);
            
            % \draw[line width=0.3mm] (9,0) -- (9,-0.01) node[ground]{};
            % \draw[line width=0.3mm] (9,0) to[capacitor] (9,3);
            % \draw[line width=0.3mm] (9,3) -- (10,3);
 \end{circuitikz}

		
\end{document}