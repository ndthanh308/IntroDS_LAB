% This must be in the first 5 lines to tell arXiv to use pdfLaTeX, which is strongly recommended.
\pdfoutput=1
% In particular, the hyperref package requires pdfLaTeX in order to break URLs across lines.

\documentclass[11pt]{article}

% Remove the "review" option to generate the final version.
% \usepackage[review]{acl}
\usepackage[]{acl}
% \usepackage{EMNLP2022}

% Standard package includes
\usepackage{times}
\usepackage{latexsym}

% For proper rendering and hyphenation of words containing Latin characters (including in bib files)
\usepackage[T1]{fontenc}
% For Vietnamese characters
% \usepackage[T5]{fontenc}
% See https://www.latex-project.org/help/documentation/encguide.pdf for other character sets

% This assumes your files are encoded as UTF8
\usepackage[utf8]{inputenc}

% This is not strictly necessary, and may be commented out.
% However, it will improve the layout of the manuscript,
% and will typically save some space.
\usepackage{microtype}

% This is also not strictly necessary, and may be commented out.
% However, it will improve the aesthetics of text in
% the typewriter font.
\usepackage{inconsolata}

\usepackage{hyperref}
\usepackage{url}


\usepackage{subcaption}
\usepackage{multirow}
\usepackage[shortlabels]{enumitem}
\usepackage{microtype}
\usepackage{graphicx}
\usepackage{booktabs}
\usepackage{hyperref}
\usepackage{ifthen}
\usepackage{cuted}
\usepackage{makecell}
\usepackage{tabularx}
\usepackage{multicol}
\usepackage{wrapfig}

\usepackage{amsmath}
\usepackage{amssymb}
\usepackage{amsthm}
\usepackage{dsfont}
\usepackage{multirow}
\usepackage[shortlabels]{enumitem}
\usepackage{setspace}

\usepackage{bbm}
\usepackage{hyperref}
\usepackage{pifont}
\usepackage{afterpage}


% # PROBABILITY	

\newcommand{\g}{\,\vert\,}	
\newcommand{\s}{\,;\,}	
\newcommand{\indpt}{\protect\mathpalette{\protect\independenT}{\perp}}	
\def\independenT#1#2{\mathrel{\rlap{$#1#2$}\mkern2mu{#1#2}}}	
\newcommand{\E}[1]{\mathbb{E}\left[#1\right]}	
\newcommand{\EE}[2]{\mathbb{E}_{#1}\left[#2\right]}	
\newcommand{\EEspecial}[2]{\mathbb{E}_{#1}\big[#2\big]}	
\newcommand{\var}[1]{\textrm{Var}\left[#1\right]}	
\newcommand{\h}[1]{\mathbb{H}\left( #1 \right)}	
\newcommand{\kl}[1]{\mathrm{KL}\left(#1\right)}	

% # DISTRIBUTIONS	

\newcommand{\mult}{\mathrm{Mult}}	
\newcommand{\dir}{\mathrm{Dir}}	
\newcommand{\discrete}{\mathrm{Discrete}}	
\newcommand{\Bern}{\mathrm{Bernoulli}}	
\newcommand{\DP}{\mathrm{DP}}	
\newcommand{\GP}{\mathrm{GP}}	
\newcommand{\Bet}{\mathrm{Beta}}	
\newcommand{\pois}{\mathrm{Pois}}	

% # CALLIGRAPHY	

\newcommand{\cL}{\mathcal{L}}	
\newcommand{\cN}{\mathcal{N}}	
\newcommand{\cD}{\mathcal{D}}	
\newcommand{\cE}{\mathcal{E}}	

% # USEFUL	

\newcommand{\RR}{\mathbb{R}}	
\newcommand{\half}{\frac{1}{2}}	
\newcommand{\new}{\mathrm{new}}	
\newcommand{\const}{\mathrm{const.}}	
\newcommand{\dd}{\mathrm{d}}	
\newcommand{\ddx}[1]{\frac{\partial}{\partial #1}}	
\newcommand{\dfdx}[2]{\textstyle \frac{\partial #1}{\partial #2}}	
\newcommand\dif{\mathop{}\!\mathrm{d}}	
\newcommand{\ind}{\mathbb{1}}	
\newcommand{\rmp}{\mathrm{p}}	
\newcommand{\prob}{\mathrm{P}}	

\newcommand{\epehe}{\mathbf{\epsilon_{\text{PEHE}}}}	

\newcommand{\eate}{\mathbf{\epsilon_{\text{ate}}}}	
% # BOLD MATHEMATICS	

\DeclareRobustCommand{\mb}[1]{\ensuremath{\boldsymbol{\mathbf{#1}}}}	

\newcommand{\mba}{\mathbf{a}}	
\newcommand{\mbb}{\mathbf{b}}	
\newcommand{\mbc}{\mathbf{c}}	
\newcommand{\mbd}{\mathbf{d}}	
\newcommand{\mbe}{\mathbf{e}}	
%\newcommand{\mbf}{\mathbf{f}}	
\newcommand{\mbg}{\mathbf{g}}	
\newcommand{\mbh}{\mathbf{h}}	
\newcommand{\mbi}{\mathbf{i}}	
\newcommand{\mbj}{\mathbf{j}}	
\newcommand{\mbk}{\mathbf{k}}	
\newcommand{\mbl}{\mathbf{l}}	
\newcommand{\mbm}{\mathbf{m}}	
\newcommand{\mbn}{\mathbf{n}}	
\newcommand{\mbo}{\mathbf{o}}	
\newcommand{\mbp}{\mathbf{p}}	
\newcommand{\mbq}{\mathbf{q}}	
\newcommand{\mbr}{\mathbf{r}}	
\newcommand{\mbs}{\mathbf{s}}	
\newcommand{\mbt}{\mathbf{t}}	
\newcommand{\mbu}{\mathbf{u}}	
\newcommand{\mbv}{\mathbf{v}}	
\newcommand{\mbw}{\mathbf{w}}	
\newcommand{\mbx}{\mathbf{x}}	
\newcommand{\mby}{\mathbf{y}}	
\newcommand{\mbz}{\mathbf{z}}	

\newcommand{\mbA}{\mathbf{A}}	
\newcommand{\mbB}{\mathbf{B}}	
\newcommand{\mbC}{\mathbf{C}}	
\newcommand{\mbD}{\mathbf{D}}	
\newcommand{\mbE}{\mathbf{E}}	
\newcommand{\mbF}{\mathbf{F}}	
\newcommand{\mbG}{\mathbf{G}}	
\newcommand{\mbH}{\mathbf{H}}	
\newcommand{\mbI}{\mathbf{I}}	
\newcommand{\mbJ}{\mathbf{J}}	
\newcommand{\mbK}{\mathbf{K}}	
\newcommand{\mbL}{\mathbf{L}}	
\newcommand{\mbM}{\mathbf{M}}	
\newcommand{\mbN}{\mathbf{N}}	
\newcommand{\mbO}{\mathbf{O}}	
\newcommand{\mbP}{\mathbf{P}}	
\newcommand{\mbQ}{\mathbf{Q}}	
\newcommand{\mbR}{\mathbf{R}}	
\newcommand{\mbS}{\mathbf{S}}	
\newcommand{\mbT}{\mathbf{T}}	
\newcommand{\mbU}{\mathbf{U}}	
\newcommand{\mbV}{\mathbf{V}}	
\newcommand{\mbW}{\mathbf{W}}	
\newcommand{\mbX}{\mathbf{X}}	
\newcommand{\mbY}{\mathbf{Y}}	
\newcommand{\mbZ}{\mathbf{Z}}	
\newcommand{\R}{\mathds{R}}

\newcommand{\mbalpha}{\mathbold{\alpha}}	
\newcommand{\mbbeta}{\mathbold{\beta}}	
\newcommand{\mbdelta}{\mathbold{\delta}}	

\newcommand{\mbepsilon}{\mathbold{\epsilon}}	
\newcommand{\mbchi}{\mathbold{\chi}}	
\newcommand{\mbeta}{\mathbold{\eta}}	
\newcommand{\mbgamma}{\mathbold{\gamma}}	
\newcommand{\mbiota}{\mathbold{\iota}}	
\newcommand{\mbkappa}{\mathbold{\kappa}}	
\newcommand{\mblambda}{\mathbold{\lambda}}	
\newcommand{\mbmu}{\mathbold{\mu}}	
\newcommand{\mbnu}{\mathbold{\nu}}	
\newcommand{\mbomega}{\mathbold{\omega}}	
\newcommand{\mbphi}{\mathbold{\phi}}	
\newcommand{\mbpi}{\mathbold{\pi}}	
\newcommand{\mbpsi}{\mathbold{\psi}}	
\newcommand{\mbrho}{\mathbold{\rho}}	
\newcommand{\mbsigma}{\mathbold{\sigma}}	
\newcommand{\mbtau}{\mathbold{\tau}}	
\newcommand{\mbtheta}{\mathbold{\theta}}	
\newcommand{\mbupsilon}{\mathbold{\upsilon}}	
\newcommand{\mbvarepsilon}{\mathbold{\varepsilon}}	
\newcommand{\mbvarphi}{\mathbold{\varphi}}	
\newcommand{\mbvartheta}{\mathbold{\vartheta}}	
\newcommand{\mbvarrho}{\mathbold{\varrho}}	
\newcommand{\mbxi}{\mathbold{\xi}}	
\newcommand{\mbzeta}{\mathbold{\zeta}}	

\newcommand{\mbDelta}{\mathbold{\Delta}}	
\newcommand{\mbGamma}{\mathbold{\Gamma}}	
\newcommand{\mbLambda}{\mathbold{\Lambda}}	
\newcommand{\mbOmega}{\mathbold{\Omega}}	
\newcommand{\mbPhi}{\mathbold{\Phi}}	
\newcommand{\mbPi}{\mathbold{\Pi}}	
\newcommand{\mbPsi}{\mathbold{\Psi}}	
\newcommand{\mbSigma}{\mathbold{\Sigma}}	
\newcommand{\mbTheta}{\mathbold{\Theta}}	
\newcommand{\mbUpsilon}{\mathbold{\Upsilon}}	
\newcommand{\mbXi}{\mathbold{\Xi}}	

\newcommand{\rmdo}{\mathrm{do}}	
\newcommand{\rmP}{\mathrm{P}}
	
\newcommand{\indep}{\perp \!\!\! \perp}

\newtheorem{assumption}{Assumption}
\newtheorem{theorem}{Theorem}
\newtheorem{corollary}[theorem]{Corollary}
\newtheorem{definition}{Definition}
\newtheorem{lemma}{Lemma}
\newtheorem{exercise}{Exercise}
\newtheorem{remark}{Remark}
\newtheorem{example}{Example}
\newtheorem{proposition}{Proposition}
% \newtheorem{warning}{Warning}
\newtheorem{claim}{Claim}
\newtheorem{note}[remark]{Note}
\newtheorem{fact}[remark]{Fact}


\DeclareMathOperator*{\argmax}{arg\,max}
\DeclareMathOperator*{\argmin}{arg\,min}
\DeclareMathOperator\supp{supp}



\usepackage[ruled,boxed,lined]{algorithm2e} % linesnumbered
\let\oldnl\nl% Store \nl in \oldnl
\newcommand{\nonl}{\renewcommand{\nl}{\let\nl\oldnl}}% Remove line number for one line


\newcommand{\pang}[1]{\textcolor{blue}{\bf \small [ #1 --RP]}}

\newcommand{\hh}[1]{\textcolor{red}{[HH: #1]}}
\newcommand{\ap}[1]{\textcolor{green}{[AP: #1]}}



% If the title and author information does not fit in the area allocated, uncomment the following
%
%\setlength\titlebox{<dim>}
%
% and set <dim> to something 5cm or larger.

\title{Leveraging Implicit Feedback from Deployment Data in Dialogue} % \\ DRAFT 7/5; DO NOT DISTRIBUTE}

\author{Richard Yuanzhe Pang$^{12}$~~~~~Stephen Roller\thanks{~~All work done at Meta AI.}~~~~~Kyunghyun Cho$^{1}$\\
\textbf{He He$^1$~~~~~Jason Weston$^{12}$}\\
$^1$New York University~~~$^2$Meta AI\\
{\tt yzpang@nyu.edu}}

\begin{document}
\maketitle
\begin{abstract}
We study improving social conversational agents by learning from natural dialogue between users and a deployed model, without extra annotations. 
 To implicitly measure the quality of a machine-generated utterance, we leverage signals like user response length, sentiment and reaction of the future human utterances in the collected dialogue episodes. Our experiments use the publicly released deployment data from BlenderBot \cite{xu2023improving}. Human evaluation 
 indicates improvements in our new models over baseline responses; however, we find that some proxy signals can lead to more generations with undesirable properties as well. 
For example, optimizing for conversation length can lead to more controversial or unfriendly generations compared to the baseline, whereas optimizing for positive sentiment or reaction can decrease these behaviors. 
\end{abstract}



\input{sec_1-6}





\section*{Acknowledgements}

We thank Jing Xu, Da Ju, Vishakh Padmakumar, Nitish Joshi, and Leshem Choshen for valuable discussion. The work is undertaken as part of the Meta--NYU mentorship program. 




% % Entries for the entire Anthology, followed by custom entries
\bibliography{anthology,custom}
% \bibliographystyle{acl_natbib}

% \clearpage

\appendix
\begin{comment}
\section{System Architecture}
\label{appendix:architecture}
\system has a novel modularized system architecture with three key components: 
\emph{StreamManager}, 
\emph{TxnManager} and \emph{TxnScheduler}. 
These components are instantiated in each thread locally.
The execution outline of \system is presented in Algorithm~\ref{alg:algo}.
Transactional stream processing is continuous and potentially never ends (Line 1$\sim$8).
The dependency resolution and execution of state transactions are separated into two non-overlapping phases by punctuations~\cite{Tucker:2003:EPS:776752.776780} (Line 2 and 5), which guarantees that no subsequent input event will have a smaller timestamp. 
Effectively, a batch of state transactions is collected during the first phase, and processed during the second phase.

In the first phase (i.e., stream processing phase), 
the \emph{StreamManager} conducts preprocessing for every input event ($e$). Similar to some prior works~\cite{tstream}, state transactions may be issued but not immediately processed during preprocessing (Line 3).
The \emph{pre\_processing} and \emph{post\_processing} functions are exposed as APIs to users.
The \emph{TxnManager} handles dependency resolution (Line 4) among state transactions and insert decomposed operations to construct a \tpg. We discuss the detailed two-phase \tpg construction process in Section~\ref{subsec:construction}.

In the second phase  (i.e., transaction processing phase), 
the \emph{TxnManager} is first involved again to refine (Line 6) the constructed \tpg with further dependency resolution.
The \emph{TxnScheduler} 
schedules operations for concurrent execution based on the constructed \tpg according to the three dimensions of scheduling decisions (Line 7). 
In particular, a scheduling decision model $M$ is instantiated based on the constructed \tpg (Line 14).
\textbf{\circled{1}} Guided by $M$, execution threads adopt an exploration strategy (Section~\ref{subsec:explore}) to explore the constructed \tpg for operations available to be scheduled constrained by dependencies. 
\textbf{\circled{2}} 
During exploration, one or multiple operations may be treated as the 
% basic 
unit of scheduling (Section~\ref{subsec:granularity}). 
Subsequently, \textbf{\circled{3}} every thread executes operation(s) in the unit of scheduling with various abort handling mechanisms (Section~\ref{subsec:abort_handling}).
Only when state transactions are processed (i.e., committed or aborted) can the associated input events be postprocessed (Line 8) by the \emph{StreamManager} based on transaction processing results.
\end{comment}

\begin{comment}
\begin{algorithm}
\footnotesize
    \KwData{$e$ \tcp{Input event}}
    \KwData{$txn_{ts}$ \tcp{State transaction}}
    \KwData{$G$ \tcp{The currently constructed TPG}}
    \While{!finish processing of input streams}{
        \eIf(\tcp*[h]{Phase 1}){\text{$e$ is not a $punctuation$}}{
                $txn_{ts}$ $\gets$ PRE\_Processing($e$)\;
                \textbf{TPG\_Construction}($G$, $txn_{ts}$)\; 
          }(\tcp*[h]{Phase 2}){
                \textbf{TPG\_Refinement}($G$)\; 
                \textbf{TXN\_Scheduling}($G$)\; 
                POST\_Processing()\;
          }
    }
    
    \SetKwFunction{FMain}{TPG\_Construction}
    \SetKwProg{Fn}{Function}{:}{}
    \Fn{\FMain{$G$, $txn_{ts}$}}{
        $O_{1..k}$ $\gets$ \textbf{Partition} $txn_{ts}$\;
        \ForEach{\text{operation $O_{i}$ $\in$ $O_{1..k}$}}{
            \textbf{Identify} its \ld\;
            $G$ $\gets$ $G$ + $O_{i}$ \;
        }
    }
    \SetKwFunction{FMain}{TPG\_Refinement}
    \SetKwProg{Fn}{Function}{:}{}
    \Fn{\FMain{$G$}}{
        \ForEach{\text{vertex $e_{i}$ $\in$ $G$}}{
            \textbf{Identify} its \td, \pd\;
        }
    }
    
    \SetKwFunction{FMain}{TXN\_Scheduling}
    \SetKwProg{Fn}{Function}{:}{}
    \Fn{\FMain{$G$}}{
        $M$ $\gets$ Instantiated with $G$;\tcp{A decision model}
        \While{!finish scheduling of $G$
        }{
          \textbf{\circled{2}} $Scheduling Unit$ $\gets$ \textbf{\circled{1}} \emph{Explore}($G$, $M$)\; 
            \textbf{\circled{3}} \emph{Execute with Abort Handling} ($Scheduling Unit$)\; 
        }
    }
  \caption{Execution Outline of \system}
  \label{alg:algo}
\end{algorithm}
\end{comment}

\end{document}