%% Show in pic label that we used Monte-Carlo fitting. PURPOSES??????????????
\begin{comment}


Relativistic effects play crucial role in AGN jets observations. Due to anisotropic emission distribution of high Lorenz factors jet matter, observed flux depend on the jet angle to the line of sight. Viewing angle is one of the most important parameter needed to interpret obtained results. This angle is hard to measure because there is no one sustained method that can give robust result. The method based on jet proper motion measurement can be used to constrain the viewing angle. The highest observed superluminal speed requires $\theta < 18\fdg 6^{+2.0}_{-1.6}$ \evk{Whose measures are these? Give reference.}. There are model-dependent estimations like $\theta = 14^{\circ}$ \citep{Nakamura2014} and $\theta = 15^{\circ}$ \citep{Wang2009}. Less model-dependent method is based on jet to counter-jet flux ratio estimation. Viewing angle of $\theta = 17.2^{\circ} \pm 3.3^{\circ}$ is obtained using this method \citep{MLWH} \evk{There are also estimated by Walker et al. 2018}. 

As jet to counter-jet flux ratio estimation, jet geometry analysis is important to restore and estimate several intrinsic parameters. To derive geometry parameters one usually make jet width to distance from the VLBI core dependence and fit with power law functions like $w(r) \propto r^{k}$, where $w$ is width, $r$ is distance and $k$ is form parameter. Recent work in M87 jet geometry \citep{Asada2012} show transition from parabolic ($k \approx 0.5$) to conical ($k \approx 1$) near HST-1 feature. This information may lead to conclusion of changing boundary or internal jet conditions. 

The main emission mechanism of radio jet is synchrotron i.e. charged particles in magnetic field. In great distances from the brightest feature AGN outflow is optically thin. Getting closer to the origin jet matter density is increasing, so emitted by jet particles photons absorbed by same matter. This effect called as synchrotron self-absorption that depend on frequency \citep{Konigl1981}. Thus radio observations especially in low frequencies show studied outflows with bright and compact feature that is not the jet origin, but the $\tau \approx 1$ region. In radioastronomy it is called as VLBI core. As self-absorption depends on frequency, VLBI core location depends on it too. So an effect of changing of core location with frequency called core shift. Recent phase-referencing multi-frequency observations of M87 in VLBA provided core shift measurements \citep{Hada2011}. 

The 1.3~mm image of M87 SMBH region \citep{EHT2019} was obtained by the Event horizon telescope. Due to lack of short baselines, interferometer has poor sensitivity of faint extended structures, so why one can not see the jet in the image \evk{Well, they see a flux, but excluded it}. In addition to this, during data calibration process an information of absolute coordinates is lost. Thus it is impossible to obtain the distance from the SMBH to VLBI core. \evk{There is no VLBI core on the EHT image. And I believe they actually can measure the real distance from the BH to the place, where the jet is produced, in the future observations.}

In numbers of simulations the structure of magnetic field is crucial about jet formation, its collimation and propagation. \evk{There are various papers discussing optical/uv/radio observations which strongly favours the helical magnetic field in the jet of M87. Would be worth to mention it.}
Unfortunately, parsec-scale jet of M87 has strong depolarization. \evk{Why? Would be correct first to say, that Park et al. estimated very high RM toward the M87 jet, and then talk about depolarization. And actually in the innermost region, the M87 jet might be unpolarized.} So there is not to much linear polarization results. \evk{Walker et al. 2018 shows the core polarization.} One of this provided linear polarization and Faraday rotation study in 2, 5 and 8 GHz \citep{Park2019}. Nevertheless full-track observations can provide enough polarized signal to analyse. Observations reduced in this paper carried out full Stokes, so there are sensitive polarimetry needed to detect extended polarized flux. 

\end{comment}

\begin{comment}

\evk{Have you try to estimate the SNR of a polarization on the core, and about 4~mas from the core? May be you can mention the value of P there, and considering estimated RM vs. distance say, if this level corresponds to depolarization?}

\begin{table}
    \centering
    \begin{tabular}{|l|l|c|c|c|c|}
    \hline
        \multicolumn{2}{|c|}{One gaussian}&\multicolumn{2}{|c|}{VLBA+Y1}&\multicolumn{2}{|c|}{VLBA+Y1+Ef} \\
        \hline
        $\nu$ & Beam &k & $r_{0}$, mas &k & $r_{0}$, mas\\
        \hline
        15 & Ellipse & 0.55 $\pm$ 0.04 & 0.2 $\pm$ 0.3 & 0.58 $\pm$ 0.04 & 0.6 $\pm$ 0.3 \\
        GHz & Circle & 0.53 $\pm$ 0.03 & -0.3 $\pm$ 0.1 & 0.58 $\pm$ 0.03 & 0.1 $\pm$ 0.1\\
        8 &Ellipse & 0.35 $\pm$ 0.03 & -0.6 $\pm$ 0.3 & 0.43 $\pm$ 0.03 & 0.8 $\pm$ 0.4\\
        GHz & Circle & 0.48 $\pm$ 0.02 & -0.8 $\pm$ 2 & 0.5 $\pm$ 0.02& -0.6 $\pm$ 2\\
    \hline
        \multicolumn{2}{|c|}{Two gaussians}&\multicolumn{2}{|c|}{VLBA+Y1}&\multicolumn{2}{|c|}{VLBA+Y1+Ef} \\
        \hline
        $\nu$& Beam &k & $r_{0}$, mas &k & $r_{0}$, mas\\
        \hline
        15 & Ellipse & 0.55 $\pm$ 0.04 & 0.2 $\pm$ 0.3 & 0.59 $\pm$ 0.04 & 0.7 $\pm$ 0.3 \\
        GHz & Circle & 0.53 $\pm$ 0.02 & 0 $\pm$ 0.2 & 0.67 $\pm$ 0.03 & 0.6 $\pm$ 0.2\\
        8 &Ellipse & 0.4 $\pm$ 0.03 & -0.6 $\pm$ 0.2 & 0.4 $\pm$ 0.03 & 0.4 $\pm$ 0.4\\
        GHz & Circle & 0.40 $\pm$ 0.02 & -1.3 $\pm$ 2 & 0.43 $\pm$ 0.01& -1.1 $\pm$ 0.4\\
    \hline
    \end{tabular}
    \caption{Approximation parameters of jet width vs core distance dependence obtained using Monte-Carlo fitting method. }
    \label{tab:geometry_fit}
\end{table}
\end{comment}


\begin{comment}


\section{Core shift}
\label{sec:coreshift}
In VLBI images of AGN the ``core'' is identified as the most closest and compact component to the jet base. Commonly this region has optical depth $ \tau \approx 1$ and the position of the core depends on observing frequency $\nu$ \citep{Blandford1979}. 

Dual-frequency observations give an opportunity to measure the coreshift value and estimate the distance from core to the jet base. 

Due to using of self-calibration in image reconstruction process the information about absolute celestial coordinates is lost. So in order to obtain the value of the coreshift between 8 and 15 GHz one should identify core region and cross-reference positions at different frequencies. 

Core region can be found as the brightest and the most compact component of the jet. This method of identification may be affected by blending of core emission with the emission downstream the jet \citep{Pushkarev2012} or can identify stationary shock as the core \citep{Gomez2016}. Nevertheless these uncertainties can be solved by constructing the spectral index map which will show optically thick regions. 

Cross-reference core positions can be done measuring the shift between radio jet images in different frequencies (image align). 

\subsection{Image alignment}
\label{sec:im_align}
To measure the shift, the core must be identified correctly. This problem is non-trivial and depend on complexity of jet structure. In some cases core may not be the brightest region e.g. 0932+326. This object have a peak in intensity image that is not corresponding the opaque core region found by analysing spectral index map \citep{0932+spind}. 


The case of M87 is complicated because there is the region of small intensity level upstream the jet. It was argued that this region can be either upstream jet either counter-jet. Spectral index map obtained for M87 jet in this paper reveals that the faint region south-east to the brightest component is counter-jet. Thus one can conclude that the brightest region is the region of $\tau \approx 1$ and can be called VLBI core.

To obtain spectral index map images in both frequencies should be aligned to correspond to each other. In this paper two methods are used.

First method is normalized 2D cross-correlation \citep{Lewis1995}. The main idea of measuring shift is to use optically thin parts of jet as reference. Due to their transparency these parts can be assumed to be at the same location at both frequencies. So choosing jet image regions far away from the core and applying normalized 2D cross-correlation one can obtain the shift between images in different frequencies. 

Second method of image alignment is based on spectral properties of jet \citep{Plavin2019}. The main idea is to choose manually correct alignment of images from visual inspection of spectral index maps. The decision of choosing the alignment is based on a priori information about spectral properties of radio jet for example the symmetry of spectral index distribution relative to jet axis.

Both methods give comparable results giving alignment shift between images $r_{8 \xrightarrow{} 15}$. 
% Нужно сравнение?

\subsection{Identification of the core region}
%As well as we become sure that the brightest jet region is optically thick core, 
Analysis of obtained spectral index maps that will be discussed later in section \ref{sec:sp_ind} gave an identification of the brightest region of jet image as core. 
To measure core location MODELFIT function in DIFMAP \citep{Shepherd1994} has been used. M87 image was fitted with model consisting of 2D Gaussian components. Coordinates of model components which have the highest amplitude were used as core position ($\vec{r}_{8}$ for 8 GHz and $\vec{r}_{15}$ for 15 GHz) relative to phase center. 

\subsection{Core shift vector}
\label{sec:cs_vector}
Having measured core positions at both frequencies ($\vec{r}_{8}$, $\vec{r}_{15}$) and the alignment shift between images $\vec{r}_{8 \xrightarrow{} 15}$ we calculate core shift vector as $\Delta \vec{r}_{8 \xrightarrow{} 15} = \vec{r}_{8} - \vec{r}_{15} - \vec{r}_{8 \xrightarrow{} 15}$.

Assuming $\vec{r}_{c}(\nu) \propto \nu^{-1}$ one can estimate the distance from the core to the jet origin at frequency $\nu_{15}$. Thus 
\begin{equation}
    \vec{r_{c}}(\nu_{15}) = \frac{\Delta \vec{r}}{\frac{\nu_{15}}{\nu_{8}} - 1}.
\end{equation}

Averaging the results of measurements for all restored images with different beams and antennae configuration we estimated $|\vec{r}_{c}(\nu_{15})| = 0.2 \pm 0.1$ mas.

\end{comment}

\begin{comment}


\subsection{The M87 black hole location}
\label{sec:bh_locat}
\evk{You already estimate $r_0$ above and wrote 'is the distance from the core to the jet base'. Thus, I don't understand the meaning of this section. I will move some text from this section above (I will give it in magenta), and then this section can be removed.}

Relativistic jets of AGN are directly connected to the central engine - supermassive black hole. Recently, the image of SMBH vicinities of M87 was presented by \citet{EHT2019} which gave an opportunity to estimate black hole parameters with direct independent method. Due to self-calibration process, the information of absolute black hole position is lost. Furthermore the lack of short baselines caused poor sensitivity to extended and faint structures such as jet itself. So there is no reference regions to measure relative distance. In this paper we try to give an estimate of supermassive black hole location relative to VLBI core. The main assumption to this estimation is close proximity of jet origin to supermassive black hole. Thus $r_{0}$ is the distance from the M87 supermassive black hole to 15~GHz VLBI core. 

It was difficult to measure $r_{0}$ investigating jet geometry with one-Gaussian method, because an uncertainties of widths measurements lead to great errors in a jet origin location estimation. With the help of multi-Gaussian modelling approach, we are able to obtain $r_{0}$ more accurate. This method gives the opportunity to estimate the distance to a black hole using only a single jet image. 
\evk{So, what is the result here?}

\end{comment}

\begin{comment}


\subsection{Identification of the core region}
\label{sec:identification_of_the_core}

Core region can be found as the brightest and the most compact component of the jet. This method of identification may be affected by blending of core emission with the emission downstream the jet \citep{Pushkarev2012} or can identify stationary shock as the core \citep{Gomez2016}. Nevertheless these uncertainties can be solved by constructing the spectral index map which will show optically thick regions. The jet from the base to the core due to synchrotron self-absorption should have flat spectrum. This can be seen in spectral index map in Fig. \ref{fig:sp_map_noEB_max}. Conversely, the region south-east to the core have steep spectrum. This feature shows that this region is transparent counter-jet and can not be the jet base. Thus we conclude that the brightest feature in intensity maps is the VLBI core.

\end{comment}




