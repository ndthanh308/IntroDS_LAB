%%%%%%%% ICML 2025 EXAMPLE LATEX SUBMISSION FILE %%%%%%%%%%%%%%%%%

\documentclass{article}

% Recommended, but optional, packages for figures and better typesetting:
\usepackage{microtype}
\usepackage{graphicx}
\usepackage{subfigure}
\usepackage{booktabs} % for professional tables

% hyperref makes hyperlinks in the resulting PDF.
% If your build breaks (sometimes temporarily if a hyperlink spans a page)
% please comment out the following usepackage line and replace
% \usepackage{icml2025} with \usepackage[nohyperref]{icml2025} above.
\usepackage{hyperref}
\usepackage{dot2texi}
\usepackage{tikz}

% Attempt to make hyperref and algorithmic work together better:
\newcommand{\theHalgorithm}{\arabic{algorithm}}

% Use the following line for the initial blind version submitted for review:
%\usepackage{icml2025}

% If accepted, instead use the following line for the camera-ready submission:
\usepackage[accepted]{icml2025}

% For theorems and such
\usepackage{amsmath}
\usepackage{amssymb}
\usepackage{mathtools}
\usepackage{amsthm}

% if you use cleveref..
\usepackage[capitalize,noabbrev]{cleveref}
\usepackage{xspace}
\usepackage{colortbl}
\definecolor{Gray}{gray}{0.9}

\usepackage{xurl}

%%%%%%%%%%%%%%%%%%%%%%%%%%%%%%%%
% THEOREMS
%%%%%%%%%%%%%%%%%%%%%%%%%%%%%%%%
\theoremstyle{plain}
\newtheorem{theorem}{Theorem}[section]
\newtheorem{proposition}[theorem]{Proposition}
\newtheorem{lemma}[theorem]{Lemma}
\newtheorem{corollary}[theorem]{Corollary}
\theoremstyle{definition}
\newtheorem{definition}[theorem]{Definition}
\newtheorem{assumption}[theorem]{Assumption}
\theoremstyle{remark}
\newtheorem{remark}[theorem]{Remark}

% Todonotes is useful during development; simply uncomment the next line
%    and comment out the line below the next line to turn off comments
%\usepackage[disable,textsize=tiny]{todonotes}
\usepackage[textsize=tiny]{todonotes}


% The \icmltitle you define below is probably too long as a header.
% Therefore, a short form for the running title is supplied here:
\icmltitlerunning{Under review in ICML 2025}

\begin{document}

\twocolumn[
\icmltitle{Pay Attention to What You Need}

% It is OKAY to include author information, even for blind
% submissions: the style file will automatically remove it for you
% unless you've provided the [accepted] option to the icml2025
% package.

% List of affiliations: The first argument should be a (short)
% identifier you will use later to specify author affiliations
% Academic affiliations should list Department, University, City, Region, Country
% Industry affiliations should list Company, City, Region, Country

% You can specify symbols, otherwise they are numbered in order.
% Ideally, you should not use this facility. Affiliations will be numbered
% in order of appearance and this is the preferred way.
\icmlsetsymbol{equal}{*}

\begin{icmlauthorlist}
\icmlauthor{Yifei Gao}{equal}
\icmlauthor{Shaohong Chen}{equal}
\icmlauthor{Lei Wang$\dag$}{}
\icmlauthor{Ruiting Dai}{}
\icmlauthor{Ziyun Zhang}{}
\icmlauthor{Kerui Ren}{}
\icmlauthor{Jiaji Wu}{}
\icmlauthor{Jun Cheng}{}
%\icmlauthor{}{sch}
%\icmlauthor{}{sch}
%\icmlauthor{}{sch}
\end{icmlauthorlist}

%\icmlaffiliation{yyy}{Department of XXX, University of YYY, Location, Country}
%\icmlaffiliation{comp}{Company Name, Location, Country}
%\icmlaffiliation{sch}{School of ZZZ, Institute of WWW, Location, Country}

%\icmlcorrespondingauthor{Firstname1 Lastname1}{first1.last1@xxx.edu}
%\icmlcorrespondingauthor{Firstname2 Lastname2}{first2.last2@www.uk}

% You may provide any keywords that you
% find helpful for describing your paper; these are used to populate
% the "keywords" metadata in the PDF but will not be shown in the document
\icmlkeywords{Machine Learning, ICML}

\vskip 0.3in
]

% this must go after the closing bracket ] following \twocolumn[ ...

% This command actually creates the footnote in the first column
% listing the affiliations and the copyright notice.
% The command takes one argument, which is text to display at the start of the footnote.
% The \icmlEqualContribution command is standard text for equal contribution.
% Remove it (just {}) if you do not need this facility.

%\printAffiliationsAndNotice{}  % leave blank if no need to mention equal contribution
%\printAffiliationsAndNotice{\icmlEqualContribution} % otherwise use the standard text.

\begin{abstract}
Although large language models (LLMs) have achieved significant success in natural language processing, they still struggle with long-context comprehension. Traditional approaches to mitigating this issue typically rely on fine-tuning or retraining, which is both resource-intensive and challenging to deploy in lightweight industrial settings. In this paper, we investigate the potential to accomplish this without any additional resources. Through an in-depth study of the attention mechanism in LLMs, we propose a method called \textbf{S}caled \textbf{R}e\textbf{A}ttention (SRA) to strengthen LLMs' ability to interpret and retrieve information by strategically manipulating their attention scores during inference. Through extensive experiments, we demonstrate that integrating SRA significantly boosts LLMs’ performance on a variety of downstream tasks, highlighting its practical potential for enhancing language understanding without incurring the overhead of traditional training.
\end{abstract}

\section{Introduction}
Large language models (LLMs) with attention mechanisms~\cite{openaiannouncement} have achieved tremendous success across a wide range of downstream tasks in recent years. Their success can largely be attributed to the superiority of the attention architecture~\cite{vaswani2017attention}. However, as tasks become more complex and the required contextual understanding increases, LLMs often fall short. 

When the input length exceeds a certain limit, LLMs often ``forget" previously mentioned content or experience ``memory confusion," leading to incorrect outputs. Even with prompt engineering techniques like Chain of Thought (CoT)~\cite{nye2021show,wei2022chain}, the models still struggle with complex problems. This limitation originates inherently from the model itself, making it unavoidable through fine-tuning or retraining—both of which demand substantial resources. This inspired the motivation for this paper: \textit{enhancing the model's comprehension and retrieval capabilities without additional training.}


% Figure environment removed

We began by identifying the attention mechanism as the critical component for retrieving and interpreting context within LLMs. Building on our empirical findings and existing research \cite{wang2020linformer,zandieh2023kdeformer}, we noted that most tokens—and their corresponding attention scores—have a negligible effect on the model’s reasoning. Even after eliminating the majority of these scores, the model’s performance remained nearly unchanged, as illustrated in Figure~\ref{fig:attn_sparse_elimi}. Intuitively, if we can better utilize the ``wasted" attention scores, the model should achieve improved performance. By manually adjusting attention scores during inference and accepting a slight trade-off in model stability, we achieved a significant improvement in comprehension and retrieval capabilities, all without any fine-tuning, retraining, or auxiliary resources. To the best of our knowledge, \textbf{this represents the first effort to address these challenges from such a perspective}.

In this paper, we introduce \textbf{Scaled ReAttention} (SRA), a technique that first discards unimportant attention scores and then redirects them toward more informative tokens. During this process, SRA strategically relaxes the model’s inherent stability, leveraging the elimination results to further enhance its comprehension. Our technique is plug-and-play and could be integrated into a wide range of existing LLMs. With SRA, we successfully improved the performance of LongChat-7B-16K and LLaMA-3-8B on the LongChat retrieval task by over 10\% compared to the original models. Additionally, we significantly outperformed the original models with LLaMA-3-8B-Instruct and LLaMA-2-13B-Chat on the XSUM summarization task. Furthermore, on the public datasets such as LongBench v1 (v2), we improved the performance of a series of LLMs by above 1.5\%.

Our contribution can be concluded as follows:
\vspace{-10pt}
\begin{itemize}
    \setlength{\itemsep}{-3pt}
    \item A comprehensive analysis of the attention mechanism and attention scores in LLMs, offering foundational insights into the SRA technique.
    \item A novel plug-and-play method that enhances the comprehension and retrieval capabilities of LLMs without the need for fine-tuning or retraining.
    \item Empirical evidence from extensive experiments showcasing SRA's ability to significantly improve performance in a variety of tasks.
\end{itemize}

\section{Related Work}

\subsection{Strengthen Long-Context Comprehension}
Prior work has primarily shown how better training methods~\cite{zhang2021deep,wang2022self} or larger datasets~\cite{hoffmann2022training,gpt4} can be used to improve model performance. Despite promising results, their excessive reliance on human and computational resources imposes significant limitations on their industrial applications.

On the other hand, solving relevant issues by retrieval~\cite{izacard2023atlas,jiang2022retrieval} to locate the main content while discarding irrelevant information can be equally effective. However, these approaches often require additional training of a ``retriever"~\cite{karpukhin2020dense} to assist with retrieval and are powerless when addressing problems that demand improved model understanding.

\subsection{Extend Context Window}
Previous research has highlighted the critical role of positional encoding (PE) in model performance~\cite{vaswani2017attention,su2023roformer,ni2021t5}, as PE conveys essential information about the relationships between tokens. However, this adaptability can introduce substantial disruption when handling text that exceeds the model’s pretraining length~\cite{press2021alibi}. To address this, methods such as Position Interpolation (PI)~\cite{chen2023pi,emozillareddit} have been proposed to extend RoPE by creating intermediate angles. Meanwhile, LandMark Attention\cite{mohtashami2023landmark} incorporates an additional “Landmark” token for block-wise information representation, which slightly modifies the underlying model structure.

Although these approaches effectively broaden the context window of LLMs without introducing extensive additional resources, their achievements are at the expense of the model's performance on downstream tasks, which severely limits their practical applications. However, with the method proposed in this paper, their performance can be substantially improved.

% Figure environment removed

\section{Preliminaries}
\def\mX{\mathbf{X}}
\def\mA{\mathbf{A}}
\def\mD{\mathbf{D}}
\def\mQ{\mathbf{Q}}
\def\mK{\mathbf{K}}
\def\mV{\mathbf{V}}
\def\Wq{\mathbf{W}_q}
\def\Wk{\mathbf{W}_k}
\def\Wv{\mathbf{W}_v}
\def\R{\mathbb{R}}
\def\mP{\mathbf{P}}
\def\vf{\mathbf{f}}
\def\vx{\mathbf{x}}
\def\vk{\mathbf{k}}
\def\vq{\mathbf{q}}
\def\vu{\mathbf{u}}
\def\di{\mathrm{i}}

\textbf{Attention Mechanism} Given the input token embeddings as $ \mX \in \R^{n\times d} $, the attention mechanism in transformers can be computed as: 
\begin{equation}
\label{eq:attn}
    \mathrm{Softmax} \left( \mQ \mK^{\top}/{\sqrt{\mathrm{C}}} \right) \mV = \mD \mA \mV
\end{equation}
where $\mQ=\mX\Wq, \mK=\mX\Wk, \mV=\mX\Wv$ are Query, Key, Value matrices, $\mathrm{C}$ is a scaling factor, and $ \Wq, \Wk, \Wv \in \R^{d \times d} $ are projection matrices. Since $\mathrm{Softmax}$ can be regarded as a dynamic nonlinear scaling of KV similarity $\mA$, we can use $\mD \in \R^{d \times d}$ to integrate $\mathrm{C}$ and $\mathrm{Softmax}$ for a direct representation, where $\mD$ is dependent on $\mA$.

\textbf{Rotary Position Embedding} Transformer models require explicit positional information to be injected. We only consider RoPE~\citep{su2023roformer} here, which is frequently used in many LLMs \citep{touvron2023llama,jiang2023mistral}. Given a position index $m \in [0, c)$ and $\mX := [x_0, x_1, \ldots, x_{d}]^\top$, RoPE defines a vector-valued complex function $\vf(\mX, m)$ as follows:
\begin{eqnarray}
\label{eq:rope}
    \vf(\mX,m) &= &[(x_0 + \di x_1) e^{\di m \theta_0}, \nonumber \\
    &\ldots&, (x_{d-2} + \di x_{d-1})e^{\di m \theta_{d/2-1}}]^\top
\end{eqnarray}
where $\di := \sqrt{-1}$ is the imaginary unit and $\theta_j = 10000^{-2j/d}$. In conjunction with Eq.~\ref{eq:attn}, we can also integrate RoPE into a changing coefficient matrix $\mP \in \R^{d \times d} $ to achieve scaling determined by relative positions:
\begin{equation}
\label{eq:attn_rope}
   \mathrm{Softmax}(\mathrm{Re}\langle\vf(\mQ, m), \vf(\mV, n)\rangle) = \mD \mP \mA
\end{equation}
After this change, $\mD$ is dependent on both $\mP$ and $\mA$. 
%This form of rewriting facilitates subsequent reasoning, as we can use methods like block decomposition and reciprocal estimation to further analyze the effects of RoPE and softmax on the distribution of attention weights.


\iffalse

Using RoPE, the self-attention score $a$ is:
\begin{align}
&a(m,n) = \mathrm{Re}\langle\vf(\mQ, m), \vf(\mV, n)\rangle =:a(m-n) \nonumber \\
&=\mathrm{Re}\left[\sum_{j=0}^{d/2-1} (q_{2j} +\di q_{2j+1})(k_{2j} - \di k_{2j+1}) e^{\di (m-n)\theta_j}\right] \nonumber \\
\label{eq:rope}
\end{align}
is only dependent on relative position $m-n$ through trigonometric functions.

\begin{equation}
	\vf(\vx, m) = 
	\begin{pmatrix}
		x_1\\
		x_2\\
		x_3\\
		x_4\\
		\vdots\\
		x_{d-1}\\
		x_d
	\end{pmatrix}
	\cdot
	\begin{pmatrix}
		\cos{m\theta_1} \\
		\cos{m\theta_1} \\
		\cos{m\theta_2} \\
		\cos{m\theta_2} \\
		\vdots \\
		\cos{m\theta_{d/2}} \\
		\cos{m\theta_{d/2}} 
	\end{pmatrix}
	+
	\begin{pmatrix}
		-x_2\\
		x_1\\
		-x_4\\
		x_3\\
		\vdots\\
		-x_d\\
		x_{d-1}
	\end{pmatrix}
	\cdot
	\begin{pmatrix}
		\sin{m\theta_1}\\
		\sin{m\theta_1}\\
		\sin{m\theta_2}\\
		\sin{m\theta_2}\\
		\vdots\\
		\sin{m\theta_{d/2}}\\
		\sin{m\theta_{d/2}}
	\end{pmatrix},
\end{equation}
where $\cdot$ is element-wise multiplication, and $\theta_k = 10000^{-2k/d}$ is a constant. 
\fi

\section{Methodology}
\def\mW{\mathbf{W}}
\def\mWa{\mathbf{W}_{A}}
\def\mWin{\mathbf{W}_{in}}
\def\mWou{\mathbf{W}_{ou}}
\def\pin{\mathrm{Pick}_{in}}
\def\pou{\mathrm{Pick}_{ou}}
\def\ein{\mathrm{E}_{in}}
\def\eou{\mathrm{E}_{ou}}

In this chapter, we first present our reasoning process and then introduce our method. We provide intuitive and easy-to-understand reasoning in the main text, with more analyses available in the appendix.

\subsection{Analysis} 
\paragraph{Tokens Play Different Roles}
Through our experiments and analyses, we \textit{first} defined that tokens in attention mechanisms can be categorized into 4 types, and their effects on performance after elimination are shown in Figure~\ref{fig:perf_drop}.

% Figure environment removed

\textbf{Linchpins} $\mX_{lin}$: Tokens with significantly high attention scores. These tokens often appear near the current token or the first token~\cite{xiao2023streamingllm} and frequently account for over $70\%$ of the accumulated attention scores. These tokens often have a critical impact on the model's reasoning results, as they are the primary contributors to altering hidden states between layers under the residual structure~\cite{liu2024minicache} and causing outliers~\cite{bondarenko2023quantizable}.

\textbf{Context Fillers} $\mX_{con}$: Tokens near the current token and exhibit relatively high attention scores. They generally account for approximately $25\%$ of the total accumulated attention scores but with constrained maximum value. Their presence has only a limited impact on generation results, as the model's reasoning capability is affected (not large) only when a large amount of them are eliminated.

\textbf{Hidden Gems} $\mX_{hid}$: Tokens located in distant regions yet exhibiting noticeably higher attention scores. Despite their distance from the current token, these tokens exert a more pronounced impact on performance than $\mX_{con}$.

% Figure environment removed

\textbf{Small Potatoes} $\mX_{pot}$: The vast majority of tokens with sparse attention. Contribute generally nothing, with accumulated attention scores no more than $2\%$.

Intuitively, identifying $\mX_{hid}$ to enhance the model's retrieval ability is a reasonable approach. These hidden gems are expected to have high relevance with the current token, but their influence is significantly constrained due to the effects of RoPE and Softmax. Specifically, the sublinear decay ratio of RoPE at greater distances (Figure~\ref{fig:rot_max}) combined with the exponential scaling of Softmax results in $\mX_{hid}$, despite their high similarity, only barely maintaining the magnitude of attention scores as $\mX_{con}$ after the scaling of $ \mD \mP$, as shown in Figure~\ref{fig:hidden_gem}. From a mathematical perspective, leveraging the properties of RoPE and softmax, the classification of these four types of tokens corresponds to four distinct scaling behaviors of attention scores $\mA$ in both positional and magnitude spaces, as elaborated in Appendix~\ref{sec:attn_dist}.



% Figure environment removed

\paragraph{Information Is Transferred Step by Step}
Under the combined effects of RoPE and softmax, attention cannot focus on tokens that are very distant from the current token, making it impossible to directly access information from distant tokens. We conducted experiments to measure how accumulated attention scores on keywords change across layers with varying distances. We found that beyond a certain distance, the accumulated attention score on keywords becomes minimal, yet the model is still able to produce normal outputs. A reasonable explanation for this is that information is continuously propagated during the inter-layer propagation process, ultimately being received by the current token. See more details in the Appendix~\ref{sec:info_trans}


\paragraph{LLMs Are Inherently Stable}
In line with our attention elimination approach and previous findings, we observed that removing $30\%$ of the accumulated attention scores across all tokens in all layers of LLMs—including most $\mX_{con}$ and $\mX_{hid}$, as well as all $\mX_{pot}$—still allowed the model to output content stably, albeit with some performance degradation on complex tasks, as shown in Figure~\ref{fig:perf_drop}. Therefore, a \textit{moderate increase in attention scores} should also not lead to large disturbances. Based on our previous analysis, by manually identifying and amplifying $\mX_{hid}$, we obtained exciting results: the model's information comprehension and retrieval abilities improved significantly for long texts! This insight was pivotal in driving the creation of SRA.


\subsection{Scaled ReAttention}
Based on all the analyses above, we designed the \textbf{S}caled \textbf{R}e\textbf{A}ttention (SRA) technique with two loops: an inter-loop and an outer-loop. The inter-loop is responsible for reinforcing the transfer of information, achieved by selecting an intermediate subset of tokens and strengthening their connection with preceding tokens. Meanwhile, the outer-loop helps the final subset of tokens ignore the distance constraints introduced by PE, allowing them to allocate attention to distant tokens directly. Both loops first identify the regions to enhance. Then, they eliminate the majority of the attention weights among the selected tokens. The erased attention weights are amplified and redistributed to those Hidden Gems within the region. The overall framework is illustrated in Figure~\ref{fig:pipeline}.


Note that the fundamental difference between our technique and previous ones lies in the fact that, after the softmax, we increase the attention sum of certain tokens—originally limited to 1—\textbf{to exceed 1 through SRA}. These additional, intentionally introduced attentions help improve the model's performance.

\paragraph{Identify Strengthened Blocks} 
Specifically, given an attention weight matrix $\mWa = \mD \mP \mA \in \R^{n \times n}$, we first divide it into blocks and apply the SRA operations only to specific blocks and regions. Due to the impact of the attention sink~\cite{xiao2023streamingllm}, we preserve the integrity of the first $C_s$ initial tokens. For the last $C_e$ tokens, we specify that they only participate in the outer loop. To strategically enhance distant hidden gems, for the remaining intermediate tokens, we divide them evenly into $l+3$ distinct blocks, where $\mathbf{C}_m^{l+3} = [C_m^1, \ldots, C_m^{l+3}]$ and $C_m^i$ is the initial token's index for the $i$th block in $\mathbf{C}_m^{l+3}$. 

The inter-loop SRA begins at layer 0 and ends at the penultimate layer, while the outer-loop SRA starts from the second layer and also ends at the penultimate layer. For every layer, both of them select only one block each. Given the selection algorithms $\pin$ and $\pou$ for inter-loop and outer-loop respectively, the regions of attention weights selected for the $i$th layer (starting at $0$th) are as follows:
\begin{align}
\label{eq:pick_alg}
    \mWa[\pin(\mWa, i)] &= \mWa^{C_m^{i+4}:C_m^{i+5} , C_m^{i+1}:C_m^{i+3}} \nonumber\\
    \mWa[\pou(\mWa, i)] &= \mWa^{-C_e: , C_m^{i+3}:C_m^{i+4}}
\end{align}
The indexing rules here are consistent with the indexing rules of $torch.tensor$ in \textbf{PyTorch}. This hierarchical approach ensures that the inter loop focuses on refining intermediate regions, while the outer loop further consolidates and enhances these refined regions in the subsequent layer.

\paragraph{Attention Elimination and Scaled Redistribution} 
The goal of elimination is to remove the smaller Context Fillers and Small Potatoes among the enhanced tokens while preserving the Hidden Gems as much as possible. Specifically, for the $j$th enhanced block $\mWin = \mWa^{C_m^{j}:C_m^{j+1} , :}$ for inter loop, $\mWou = \mWa^{- C_e:, :}$ for outer loop, the inter-loop eliminator $\ein$ and outer-loop eliminator $\eou$ is defined as:
\begin{align}
\label{eq:elim_alg}
    \ein(\mWin, j) &= \mathrm{Whe}(\mWin > (\tau_{in} / C_m^{j}), \mWin, 0) \nonumber\\
    \eou(\mWou) &= \mathrm{Whe}(\mWou > (\tau_{ou}/C_r), \mWou, 0)
\end{align}
Here, $\mathrm{Whe}$ functions the same as $torch.where$ and $C_r = n - C_e$. If no Hidden Gems are found during the elimination process, such as all $\mWin^{:,C_m^{j-3}:C_m^{j-2}}=0$, the elimination will be skipped, and no subsequent operations will be performed. Otherwise, the erased weights will be summed in a token-wise manner and multiplied by a scaling factor, $s_{in}$ for inter loop and $s_{ou}$ for the outer loop. This amplification enhances performance by sacrificing the stability of the LLM, allowing the accumulated attention to exceed 1. Finally, the amplified erased weights will be evenly re-added on those uneliminated Hidden Gems within targeted blocks in Eq.~\ref{eq:pick_alg}. The inter-loop algorithm is exhibited in Algorithm~\ref{algo:sra_inter}, while the outer-loop one is in the Appendix~\ref{sec:outler_sra}. During inference, SRA is triggered \textit{only in the prefilling stage}.

\begin{algorithm}[h]
   \caption{Inter-loop Scaled ReAttention}
\begin{algorithmic}
    \label{algo:sra_inter}
   \STATE {\bfseries After applying Softmax on attention weights:}
   \STATE {\bfseries Input:} \textit{Attention Weights} $\mWa$,  \textit{Layer Index} $i$,  \textit{Layer Num} $l$,  \textit{Inter Threshold} $\tau_{in}$,  \textit{Inter Scaling Factor} $s_{in}$. \COMMENT{\textbf{Note}: All unspecified functions are from \textbf{PyTorch}.}
   \IF[Inter Loop]{$ (l-1) > i > 0 $}
    \STATE /* Function only on indexes having Hidden Gems */
    
    \STATE $\mathbf{idx}_{gem} = any(\mWa[\pin(\mWa,i)] > (\tau_{in} / C_m^{i+4}))$ 
    
    \STATE $\mathbf{\mW}_{eli} = \ein(\mWa[\mathbf{idx}_{gem}], i+4)$ 
    
    \STATE $\mathbf{idx}_{tar} = \pin(\mathbf{\mW}_{eli},i)$
    
    \STATE $\mW_{tar} = \mW_{eli}[\mathbf{idx}_{tar}]$
    
    \STATE /* Prepare scaled attention removal */
    \STATE $\mW_{re} = sum(\mW_{eli},dim=-1)$
    
    \STATE $\mW_{rm} = oneslike(\mW_{re})-\mW_{re}$
    \STATE $\mathbf{m}_{gem} = where(\mW_{tar} > 0, 1, 0.01)$
    
    \STATE $\mW_{add} = div(\mW_{rm}, sum(\mathbf{m}_{gem}, dim=-1))*s_{in} $
    \STATE /* Readded to original weights */
    
    \STATE $\mW_{eli}[\mathbf{idx}_{tar}] = \mW_{tar} + \mW_{add}$
    \STATE $\mWa[\mathbf{idx}_{gem}] = \mW_{eli}$
    \ENDIF 
\end{algorithmic}
\end{algorithm}

\section{Experiments}

\subsection{Setting}
Our experiments comprehensively demonstrate the effectiveness of our method from multiple perspectives. We selected commonly used model series such as LLaMA~\cite{touvron2023llama}, Mistral~\cite{jiang2023mistral}, Qwen~\cite{bai2023qwen}, and LongChat~\cite{longchat2023}, as well as their YaRN~\cite{peng2023yarn} and LandMark~\cite{mohtashami2023landmark} variants, as baseline models. First, we used methods from LandMark~\cite{mohtashami2023landmark} and LongChat~\cite{longchat2023} to evaluate the improvement in retrieval capabilities brought by SRA. Next, we tested the model's ability to summarize and understand long, complex texts on the XSUM~\cite{narayan2018xsum} dataset under GPT-4 evaluation protocol~\cite{vicuna}. We further validated the superiority of our approach through downstream tasks on publicly available long-text comprehension benchmarks, including LongBench~\cite{bai2023longbench}, LongBench v2~\cite{bai2024longbench}, InfiniteBench~\cite{zhang2024bench}. 

The configuration of SRA is not a one-size-fits-all solution. Instead, it requires dynamic tuning based on the requirements of specific tasks. Several factors influence the choice of SRA parameters, including task characteristics and variations among different baselines. In our experiments, we typically keep the total accumulated attention of SRA-strengthened tokens within the range of 1.1 to 1.4. Detailed discussions can be found in the Appendix~\ref{sec:sra_config}.
%, InfiniBench~\cite{infi},

\subsection{Reterieval Evaluation}
\label{sec:retrieval_eval}
We began by assessing the improvements in retrieval capabilities introduced by SRA within the LongChat framework, followed by an evaluation using a retrieval prompt proposed in LandMark. We modified the original retrieval prompt to increase complexity. For the \textit{PASS KEY}, we randomly generated 50 words comprising numbers and uncommon vocabulary. By varying the retrieval distance, we tested the model’s performance. Throughout the experiments, \(a\) was fixed at 8, while \(b\) was varied at intervals of 200 tokens. The prompt and results of the two tasks are shown in Figure~\ref{fig:retrieval_perf}. 

Experiments reveal a significant enhancement in the model's retrieval capabilities after incorporating SRA. For the LandMark \textit{PASS KEY} retrieval task, \textbf{LLaMA-2-7B} achieves an average improvement of 4.7\% over the original model. Additionally, compared to the LandMark variant of \textbf{LLaMA-7B}, our approach delivers an average improvement of 8.5\%, effectively enabling SRA to mitigate the decline in retrieval performance caused by its structure-altering. On the LongChat benchmark, SRA achieves a notable performance boost, with an average retrieval accuracy improvement exceeding 10\% over these original models.

% Figure environment removed

%------------------------------------------------------------
\begin{table*}[tb]
\footnotesize
\centering
\caption{ \textbf{LongBench Results.} We present the results of three open-source models evaluated on seven tasks from LongBench, both before and after applying SRA. SRA delivers consistent performance improvements without requiring any additional fine-tuning or retraining.}
%\resizebox{\linewidth}{!}
{
\begin{tabular}{l|ccccccc|c}
\toprule
\textbf{Model / Tasks}$\uparrow$  & \textbf{MFQA-EN} & \textbf{VCSUM}  & \textbf{TREC} & \textbf{SAMSum} & \textbf{LSHT} & \textbf{LCC} & \textbf{RepoBench-P} & \textbf{Average}\\ \hline

LLaMA-2-7B-Chat & 36.22 & 15 & 64.5 & 40.7 & 17.75 & 58.50 & 52.45 & 40.73\\ 

\cellcolor{Gray}\textbf{SRA} & \cellcolor{Gray}37.83 & \cellcolor{Gray}21.0 & \cellcolor{Gray}66.5 & \cellcolor{Gray}42.31 & \cellcolor{Gray}19.00 & \cellcolor{Gray}59.36 & \cellcolor{Gray}53.23 & \cellcolor{Gray}42.74(+2.01) \\

\midrule


LLaMA-3-8B-Instruct & 41.50 & 14.8 & 75.5 & 42.48 & 24.25 & 58.87 & 50.73 & 44.01\\ 

\cellcolor{Gray}\textbf{SRA} & \cellcolor{Gray}42.71 & \cellcolor{Gray}17.5 & \cellcolor{Gray}78.5 & \cellcolor{Gray}43.04 & \cellcolor{Gray}28.00 & \cellcolor{Gray}59.72 & \cellcolor{Gray}51.27 & \cellcolor{Gray}45.82(+1.81) \\

\midrule

LongChat-v1.5-7B-32k & 41.40 & 9.9 & 63.5 & 34.20 & 23.20 & 53.00 & 55.30 & 40.07\\ 

\cellcolor{Gray}\textbf{SRA} & \cellcolor{Gray}43.20 & \cellcolor{Gray}14.5 & \cellcolor{Gray}65.1 & \cellcolor{Gray}35.80 & \cellcolor{Gray}25.10 & \cellcolor{Gray}53.72 & \cellcolor{Gray}55.96 & \cellcolor{Gray}41.91(+1.84) \\

\bottomrule
\end{tabular}
}
\label{tab:longbench_v1}
\end{table*}
%------------------------------------------------------------



\subsection{Summarization Evaluation}
\label{sec:xsum}
We tested the enhancements brought by SRA on texts of different lengths using \textbf{LLaMA-3-8B-Instruct} and \textbf{LLaMA-2-13B-Chat}. Specifically, for \textbf{LLaMA-3-13B-Chat}, we started with texts of length 1000 tokens, collecting 100 cases at intervals of 500 tokens, up to a length of 4000 tokens. For \textbf{LLaMA-3-8B-Instruct}, we started with texts of length 2000 tokens, collecting 50 cases at intervals of 500 tokens, up to a length of 5000 tokens. We used GPT4o as the evaluation model following the GPT-4 evaluation protocol, comparing the outputs under SRA with the original model outputs. The results are illustrated in Figure~\ref{fig:xsum_perf}, where we show the counts of ``pure win" and ``tie" cases. Here, the ``pure win" refers to the winning number of SRA minus the winning number of the original model. 

The results indicate that the benefits of SRA become increasingly evident as text length grows, with a declining number of ties and a steadily rising count of ``pure wins". Beyond a context length of 3000 for \textbf{LLaMA-2-13B-Chat} and 3500 for \textbf{LLaMA-3-8B-Instruct}, over half of the total samples show improvements compared to the original results when SRA is applied.

% Figure environment removed


\subsection{Results on Open-Source Benchmarks}
Starting with LongBench, we selected 7 tasks including MultiFieldQA-EN (MFQA-EN), VCSUM~\cite{wu-etal-2023-vcsum}, TREC~\cite{li2002learning}, SAMSum~\cite{gliwa2019samsum}, LSHT~\cite{LSHT}, LCC~\cite{guo2023longcoder}, and RepoBench-P~\cite{liu2024lost}. The following results are illustrated in Table~\ref{tab:longbench_v1}. Our SRA technique enables an overall performance gain exceeding 1.8 across all models.

We further explored the performance of SRA on \textbf{YaRN-Mistral} within the InfiniteBench benchmark, with the results shown in Table~\ref{tab:infini_results}. With the enhancements provided by SRA, the decline in retrieval and comprehension capabilities caused by PI modifications introduced by YaRN was significantly mitigated. This improvement further unlocks the potential of methods like YaRN and other PI approaches in the application of LLMs.

\begin{table}[tb]
    \centering
    %\setstretch{0.8}
    %\scriptsize
    \footnotesize
    \caption{\textbf{InfiniteBench Results of YaRN-Mistral}. Here, \textit{Retrieve} encompasses both \textit{Retrieve.PassKey} and \textit{Retrieve.Number}. SRA notably enhances the retrieval capabilities of models trained with YaRN.}
    \begin{tabular}{lcccc}%p{2cm}p{1cm}p{1cm}p{1cm}
    \toprule
    ~ & Retrieve & En.Sum & En.MC \\
    \hline
    YaRN-Mistral & 74.66 & 9.09 & 27.95 \\
    \midrule
    \cellcolor{Gray}\textbf{SRA} & \cellcolor{Gray}81.24 & \cellcolor{Gray}12.08 & \cellcolor{Gray}37.25 \\
    \bottomrule
    \end{tabular}
    %\vspace{-1.5em}
    \label{tab:infini_results}
\end{table}

Finally, we conducted tests on the newly released LongBench v2, which includes a series of models using RoPE as their positional encoding method. The results are presented in Table~\ref{tab:longbench_v2}. The results demonstrate that SRA exhibits excellent generalizability for LLMs utilizing RoPE as their positional encoding, significantly enhancing comprehension capabilities while maintaining high compatibility with CoT prompt engineering. For smaller models, the improvements brought by SRA are particularly pronounced, with most tasks on \textbf{Llama-3.1-8B-Instruct} achieving gains of over 2\%. For larger LLMs, the most notable improvements are observed in long-text processing, with performance increases exceeding 2\% in certain tasks. This is likely because advanced LLMs are already well-optimized for handling shorter contexts effectively. Moreover, regardless of task difficulty, the enhancements achieved through SRA remain consistent, demonstrating the robustness and reliability of our method.

%To avoid disrupting the linguistic coherence of extremely long context (e.g., $>$60k), we set both the elimination factor $\tau_{in}$ and $\tau_{ou}$, and scale factor $s_{in}$ and $s_{ou}$ to relatively small values, as excessive use of SRA may lead to confusion in language capabilities.

\begin{table*}[t]
\centering
\caption{\textbf{SRA evaluation results (\%) on LongBench v2.} Results under \colorbox{Gray}{CoT} prompting are highlighted with a gray background. SRA exhibits robust compatibility and enhancement effects for models employing RoPE as the positional encoding method, especially when integrated with CoT reasoning.}
\resizebox{\linewidth}{!}{
\begin{tabular}{p{5cm}|m{0.7cm}m{0.7cm}|m{0.7cm}m{0.7cm}|m{0.7cm}m{0.7cm}|m{0.7cm}m{0.7cm}|m{0.7cm}m{0.7cm}|m{0.7cm}m{0.7cm}}
\toprule
 &  & & \multicolumn{4}{|c}{\textbf{Difficulty}} & \multicolumn{6}{|c}{\textbf{Length ($<$32k; 32k-128k; $>$128k)}} \\
\cmidrule(r){1-3} \cmidrule(lr){4-7} \cmidrule(l){8-13}
\textbf{Model} & \multicolumn{2}{c|}{\textbf{Overall}} & \multicolumn{2}{c|}{\textbf{Easy}} & \multicolumn{2}{c|}{\textbf{Hard}} & \multicolumn{2}{c|}{\textbf{Short}} & \multicolumn{2}{c|}{\textbf{Medium}} & \multicolumn{2}{c}{\textbf{Long}} \\ 
\midrule

%\texttt
{Llama-3.1-8B-Instruct} 
& 30.0 & \cellcolor{Gray}30.4 
& 30.7 & \cellcolor{Gray}36.5 
& 29.6 & \cellcolor{Gray}26.7 
& 35.0 & \cellcolor{Gray}34.4 
& 27.9 & \cellcolor{Gray}31.6 
& 25.9 & \cellcolor{Gray}21.3 \\ 

\textbf{SRA} 
& 31.3 & \cellcolor{Gray}32.0 
& 31.9 & \cellcolor{Gray}38.2 
& 30.7 & \cellcolor{Gray}28.5 
& 37.3 & \cellcolor{Gray}36.9 
& 29.3 & \cellcolor{Gray}33.1 
& 27.2 & \cellcolor{Gray}23.4 \\

\midrule

%\texttt
{Llama-3.3-70B-Instruct} & 29.8 & \cellcolor{Gray}36.2 & 34.4 & \cellcolor{Gray}38.0 & 27.0 & \cellcolor{Gray}35.0 & 36.7 & \cellcolor{Gray}45.0 & 27.0 & \cellcolor{Gray}33.0 & 24.1 & \cellcolor{Gray}27.8 \\

\textbf{SRA} 
& 31.0 & \cellcolor{Gray}37.2 
& 35.1 & \cellcolor{Gray}39.2 
& 27.9 & \cellcolor{Gray}35.8 
& 37.5 & \cellcolor{Gray}48.1 
& 28.6 & \cellcolor{Gray}34.3 
& 25.8 & \cellcolor{Gray}28.4 \\

\midrule

%\texttt
{Qwen2.5-72B-Instruct} 
& 39.4 & \cellcolor{Gray}38.8 
& 43.8 & \cellcolor{Gray}42.2 
& 36.7 & \cellcolor{Gray}36.7 
& 44.4 & \cellcolor{Gray}50.0 
& 34.0 & \cellcolor{Gray}28.8 
& 41.7 & \cellcolor{Gray}39.8 \\

\textbf{SRA} 
& 41.2 & \cellcolor{Gray}41.4 
& 44.5 & \cellcolor{Gray}43.6 
& 36.9 & \cellcolor{Gray}39.1 
& 45.2 & \cellcolor{Gray}51.3 
& 35.9 & \cellcolor{Gray}31.4 
& 43.5 & \cellcolor{Gray}41.6 \\

\midrule

%\texttt
{Mistral-Large-Instruct-2407} & 26.6 & \cellcolor{Gray}33.6 & 29.7 & \cellcolor{Gray}34.4 & 24.8 & \cellcolor{Gray}33.1 & 37.8 & \cellcolor{Gray}41.1 & 19.5 & \cellcolor{Gray}31.2 & 22.2 & \cellcolor{Gray}25.9 \\


\textbf{SRA} 
& 28.0 & \cellcolor{Gray}34.6 
& 30.5 & \cellcolor{Gray}36.2 
& 26.3 & \cellcolor{Gray}34.5 
& 39.1 & \cellcolor{Gray}43.4 
& 20.2 & \cellcolor{Gray}31.5 
& 24.3 & \cellcolor{Gray}27.2 \\

\bottomrule
\end{tabular}
}
\label{tab:longbench_v2}
\end{table*}


\subsection{Ablation Studies}
\label{sec:ablation}
In SRA, both the inter loop and outer loop are integral to its effectiveness. Even without scaling—where \( s_{in} \) and \( s_{ou} \) are set to \( 1 \)—the basic ``ReAttention" mechanism reduces perplexity during inference, as demonstrated in Table~\ref{tab:ablation_ppl}. Furthermore, extensive experiments reveal that the inter loop and outer loop enhance distinct aspects of model comprehension, as illustrated in Table~\ref{tab:ablation_tasks}. The inter loop primarily bolsters overall comprehension, making it particularly effective for tasks such as dialogue, summarization, and document understanding. In contrast, the outer loop excels in improving retrieval capabilities, especially for questions or keywords positioned near the end of prompts. Combining all of the components, SRA finally renders its superior effects.


%---------------------------------------------------
\begin{table}[tb]
    \centering
    %\setstretch{0.8}
    %\scriptsize
    \footnotesize
    \caption{\textbf{Ablation study of LLaMA-3-8B on WikiText.} We use the last 200 words to calculate the perplexity.}
    \begin{tabular}{p{2cm}|m{1cm}|m{1cm}|m{1cm}}%p{2cm}p{1cm}p{1cm}p{1cm}
    \toprule
    Word Counts & 1024 & 1536 & 2048 \\
    \midrule
    Original & 5.58 & 5.61 & 5.22  \\
    %\hline
    +inter loop & 5.57 & 5.60 & 5.21 \\
    %\hline
    +outer loop & 5.57 & 5.59 & 5.20 \\
    \midrule
    RA & 5.56  & 5.57 & 5.19 \\
    \bottomrule
    \end{tabular}
    %\vspace{-1.5em}
    \label{tab:ablation_ppl}
\end{table}
%---------------------------------------------------

%---------------------------------------------------
\begin{table}[tb]
    \centering
    %\setstretch{0.8}
    %\scriptsize
    \footnotesize
    \caption{\textbf{Ablation study of LLaMA-3-8B-Instruct on downstream tasks.}}
    \begin{tabular}{p{2.3cm}|c|c|c}%p{2cm}p{1cm}p{1cm}p{1cm}
    \midrule
    Tasks & LongChat & MFQA-EN & LSHT \\
    \midrule
    Original & 59.5 & 41.50 & 24.25\\
    %\hline
    +inter loop & 62.5 & 42.35 & 26.80 \\
    %\hline
    +outer loop & 68.9 & 41.78 & 25.20 \\
    \midrule
    SRA & 70.0 & 42.71 & 28.00\\
    \bottomrule
    \end{tabular}
    %\vspace{-1.5em}
    \label{tab:ablation_tasks}
\end{table}
%---------------------------------------------------


\subsection{Discussion of SRA Configurations}
In our experiments, we tested numerous sets of SRA parameters to validate the gains brought by SRA. Despite variations in models and tasks, we derived a generalizable approach for tuning SRA parameters. More discussions and their corresponding experiments are exhibited in the Appendix~\ref{sec:sra_config}.

From a model perspective, training methods and following tasks influence a model's sensitivity to SRA. For example, while both are based on LLaMA, the LongChat series demonstrates greater sensitivity to SRA compared to the LLaMA series. Generally, models trained for retrieval tasks require a lower elimination threshold and smaller scale factors. Excessive values for these parameters can lead to incorrect outputs even when the correct position is identified, such as retrieving the correct context but returning an incorrect number in retrieval tasks. For models trained under standard conditions, larger parameter values are needed, particularly for the scaling factor, which directly affects the enhancement achieved.

In terms of context length, longer texts generally require smaller scaling factors. This is because the robustness of LLMs is limited, and excessively large scaling factors can impair the model's language capabilities. Specifically, this manifests as fragmented sentences and incoherent expressions. While some keywords may still appear, they fail to form continuous and meaningful statements.

From a task perspective, as discussed in Sec.~\ref{sec:ablation}, the type of task significantly influences the configuration of the inter loop and outer loop scaling factors. For tasks such as QA and summarization, larger \( s_{in} \) and smaller \( s_{ou} \) are recommended. In contrast, for retrieval-based tasks, focusing the gains from SRA on the keywords in the final question—by setting a larger \( s_{ou} \)—yields better results.

\section{Limitations}
\paragraph{Variability of SRA}
Although we have established a relatively general set of guidelines, a small amount of task-specific testing remains unavoidable. This is particularly important for adjusting SRA parameters to suit different tasks. However, our experiments reveal that models within the same series generally exhibit similar characteristics, reducing the need for extensive testing to some extent. 

Excessive application of SRA can cause significant disruptions to LLMs, potentially rendering them non-functional. Under normal use, SRA is characterized by a slight increase in perplexity compared to the original model. While some negative effects are present, the positive outcomes far outweigh them. From a task perspective, this slight increase in perplexity under normal usage has no impact on the quality or accuracy of the generated content.

\paragraph{Inference Efficiency}
A notable limitation of SRA is its reliance on explicit manipulation of attention scores after Softmax, as operations before the Softmax stage would disrupt the original distribution and introduce significant interference. This explicit computation prevents the use of certain attention acceleration techniques, such as FlashAttention~\cite{dao2022flashattention}, in conjunction with SRA, leading to slower inference speeds. However, the impact of SRA's operations on pure processing time is minimal, as we only enhance a small fraction of tokens in heads with Hidden Gems and just in the prefilling stage. Comparisons are shown in Table~\ref{tab:inference_speed}. Moreover, the performance gains provided by SRA without additional training fully compensate for the impact of the extra time.

\begin{table}[tb]
    \centering
    %\setstretch{0.8}
    %\scriptsize
    \footnotesize
    \caption{\textbf{Inference Speed of LLaMA 7B.} We report the running memory (denoted as `RM') and speed in NVIDIA A100-80G.}
    \begin{tabular}{p{2cm}|p{1.5cm}|p{1.5cm}}
    \toprule
    Method & RM & Token/s \\
    \midrule
    Normal & 14.4 G &  69.2 \\
    %\hline
    Flash Attention& 13.7 G &  77.9  \\
    \midrule
    SRA & 14.6 G &  64.8  \\
    \bottomrule
    \end{tabular}
    %\vspace{-1.5em}
    \label{tab:inference_speed}
\end{table}


\section{Conclusion}
In this paper, we introduce SRA, a training-free method designed to enhance the contextual understanding capabilities of large language models. SRA achieves this by manually adjusting attention scores, amplifying the scores projected onto Hidden Gems, and trading off some model stability to improve retrieval and comprehension abilities. Through extensive experiments, we demonstrate the effectiveness of SRA across a variety of tasks, achieving significant performance improvements in retrieval and summarization tasks. Furthermore, SRA delivers notable enhancements even in open-ended long-text scenarios.




\section*{Impact Statement}
Our goal is to enhance large language models' reading comprehension and information retrieval capabilities without requiring any additional training. Our research is strongly oriented toward the industry, where cost is a crucial factor. Unlike previous research-focused work that requires significant resource investment to boost performance, our study emphasizes lightweight industrial implementation and practical deployment. In our view, our research makes an outstanding contribution.

The application scenarios for our research are highly extensive, as most large language models today are based on RoPE for positional encoding. Through a series of experiments, we have demonstrated the universality of our method. For instance, it can be applied to everyday tasks such as document summarization, inductive reasoning, long-text keyword retrieval, and memory in long and complex conversations, covering nearly all daily scenarios that require handling long texts.

For this work, the key point we need to emphasize remains the same: \textbf{no additional training is required}. Compared to the hundreds or thousands of A100 hours typically needed for training or fine-tuning, achieving immediate performance improvement through a plug-and-play method is exceptionally valuable.










\iffalse
\subsection{Algorithms}

If you are using \LaTeX, please use the ``algorithm'' and ``algorithmic''
environments to format pseudocode. These require
the corresponding stylefiles, algorithm.sty and
algorithmic.sty, which are supplied with this package.
\cref{alg:example} shows an example.

\begin{algorithm}[tb]
   \caption{Bubble Sort}
   \label{alg:example}
\begin{algorithmic}
   \STATE {\bfseries Input:} data $x_i$, size $m$
   \REPEAT
   \STATE Initialize $noChange = true$.
   \FOR{$i=1$ {\bfseries to} $m-1$}
   \IF{$x_i > x_{i+1}$}
   \STATE Swap $x_i$ and $x_{i+1}$
   \STATE $noChange = false$
   \ENDIF
   \ENDFOR
   \UNTIL{$noChange$ is $true$}
\end{algorithmic}
\end{algorithm}



% Note use of \abovespace and \belowspace to get reasonable spacing
% above and below tabular lines.

\begin{table}[t]
\caption{Classification accuracies for naive Bayes and flexible
Bayes on various data sets.}
\label{sample-table}
\vskip 0.15in
\begin{center}
\begin{small}
\begin{sc}
\begin{tabular}{lcccr}
\toprule
Data set & Naive & Flexible & Better? \\
\midrule
Breast    & 95.9$\pm$ 0.2& 96.7$\pm$ 0.2& $\surd$ \\
Cleveland & 83.3$\pm$ 0.6& 80.0$\pm$ 0.6& $\times$\\
Glass2    & 61.9$\pm$ 1.4& 83.8$\pm$ 0.7& $\surd$ \\
Credit    & 74.8$\pm$ 0.5& 78.3$\pm$ 0.6&         \\
Horse     & 73.3$\pm$ 0.9& 69.7$\pm$ 1.0& $\times$\\
Meta      & 67.1$\pm$ 0.6& 76.5$\pm$ 0.5& $\surd$ \\
Pima      & 75.1$\pm$ 0.6& 73.9$\pm$ 0.5&         \\
Vehicle   & 44.9$\pm$ 0.6& 61.5$\pm$ 0.4& $\surd$ \\
\bottomrule
\end{tabular}
\end{sc}
\end{small}
\end{center}
\vskip -0.1in
\end{table}



\subsection{Theorems and such}
The preferred way is to number definitions, propositions, lemmas, etc. consecutively, within sections, as shown below.
\begin{definition}
\label{def:inj}
A function $f:X \to Y$ is injective if for any $x,y\in X$ different, $f(x)\ne f(y)$.
\end{definition}
Using \cref{def:inj} we immediate get the following result:
\begin{proposition}
If $f$ is injective mapping a set $X$ to another set $Y$, 
the cardinality of $Y$ is at least as large as that of $X$
\end{proposition}
\begin{proof} 
Left as an exercise to the reader. 
\end{proof}
\cref{lem:usefullemma} stated next will prove to be useful.
\begin{lemma}
\label{lem:usefullemma}
For any $f:X \to Y$ and $g:Y\to Z$ injective functions, $f \circ g$ is injective.
\end{lemma}
\begin{theorem}
\label{thm:bigtheorem}
If $f:X\to Y$ is bijective, the cardinality of $X$ and $Y$ are the same.
\end{theorem}
An easy corollary of \cref{thm:bigtheorem} is the following:
\begin{corollary}
If $f:X\to Y$ is bijective, 
the cardinality of $X$ is at least as large as that of $Y$.
\end{corollary}
\begin{assumption}
The set $X$ is finite.
\label{ass:xfinite}
\end{assumption}
\begin{remark}
According to some, it is only the finite case (cf. \cref{ass:xfinite}) that is interesting.
\end{remark}
%restatable
\fi

% Acknowledgements should only appear in the accepted version.
\iffalse
\section*{Acknowledgements}

\textbf{Do not} include acknowledgements in the initial version of
the paper submitted for blind review.

If a paper is accepted, the final camera-ready version can (and
usually should) include acknowledgements.  Such acknowledgements
should be placed at the end of the section, in an unnumbered section
that does not count towards the paper page limit. Typically, this will 
include thanks to reviewers who gave useful comments, to colleagues 
who contributed to the ideas, and to funding agencies and corporate 
sponsors that provided financial support.
\fi


% In the unusual situation where you want a paper to appear in the
% references without citing it in the main text, use \nocite
\nocite{langley00}

\bibliography{example_paper}
\bibliographystyle{icml2025}

%\onecolumn
\begin{center}
{\Large \textbf{Appendix}}
\end{center}

\section{RELATED WORKS}

\paragraph{Issues with attention for long-range reasoning.} Efficient processing of long sequences is an important open question in deep learning. Attention-based transformers~\citep{vaswani2017attention} provide a scalable approach but suffer from \textit{quadratically increasing complexity in inference/memory} as the sequence length grows. While many approaches exist to alleviate this issue, e.g. efficient memory management~\citep{dao2022flashattention,dao2023flashattention} and architectural modifications~\citep{wang2020linformer, kitaev2020reformer, child2019generating, beltagy2020longformer}, the sequence length in large language models is usually kept to $2k/4k$ tokens for this reason~(e.g. Llama2~\citep{touvron2023llama}). In addition, in some long-range reasoning tasks~\citep{tay2020long} attention does not seem to provide the correct \textit{inductive bias}, leading to poor performance in addition to high computational costs.
\vspace{-3mm}
\paragraph{Success of modern recurrent layers.} Due to the issues outlined above, the community has witnessed in the last year the rise of new, drastically innovative, \textit{recurrent} alternatives to the attention mechanism, named state-space models~(SSMs). The first SSM, S4, was introduced by \cite{gu2021efficiently}, and since then, a plethora of variants have been proposed:  LiquidS4~\citep{hasani2022liquid}, DSS~\citep{gupta2022diagonal},
S4D~\citep{gu2022parameterization}, S5~\citep{smith2022simplified}, RWKV~\citep{peng2023rwkv} and RetNet~\citep{sun2023retentive} among others. These models achieve remarkable performance, surpassing all modern attention-based transformer variants by an average $20\%$ accuracy on challenging sequence classification tasks~\citep{tay2020long}. SSMs have reached outstanding results in various domains beyond toy datasets~\citep{nguyen2022s4nd,goel2022sashimi,gu2021efficiently,lu2023structured,zucchet2023online}. SSMs also were successfully applied to language modeling, and are sometimes used in combination with attention~\citep{fu2023hungry, wang2023pretraining, ma2022mega}. At inference time, all SSMs coincide with a stack of linear Recurrent Neural Networks, interleaved with MLPs and normalization. Linearity of the RNNs allows fast parallel processing with FFTs~\citep{gu2022parameterization} or parallel scans~\citep{smith2023simplified}.  
\vspace{-3mm}

\paragraph{The linear recurrent unit~(LRU).} Among modern architectures for long-range reasoning based on recurrent modules, the simplest is perhaps Linear Recurrent Unit~(LRU)~\citep{orvieto2023resurrecting}: while SSMs rely on the discretization of a structured continuous-time latent dynamical system, the LRU is directly designed for discrete-time systems~(token sequences), and combines easy hyperparameter tuning with solid performance and scalability. The only differences between the LRU and the standard RNN update $x_{k} = A x_{k-1} + B u_k$~($u$ is the input at a specific layer and $x$ is the hidden-state, then fed into a position-wise MLP) are (1) the system operates in the complex domain~(required for expressivity, see discussion in~\cite{orvieto2023resurrecting})~(2) to enhance stability and better control how quickly gradients vanish, $A$~(diagonal) is learned using polar parametrization and log-transformed magnitude and phase. Finally, (3) the recurrence is normalized through an extra optimizable parameter that scales the input to stabilize signal propagation. The parametrization of linear RNNs of~\citep{orvieto2023resurrecting} was found to be effective also in surpassing deep LSTMs and GRUs~\citep{zucchet2023gated}. We use the LRU codebase\footnote{\url{https://github.com/NicolasZucchet/minimal-LRU/tree/main/lru}} as a starting point for our experiments, when the linear RNN is learned.
\vspace{-3mm}
\paragraph{Approximation theory for MLP and non-linear RNNs.} The approximation properties of deep neural networks with ReLU activations are well studied. While recent advances concern the effect of depth~\citep{lu2017expressive}, the study by \citet{pinkus1999approximation}, as well as previous works~\citep{funahashi1989approximate,hornik1989multilayer,hornik1991approximation, barron1993universal}, already established the power of neural networks with a single hidden layer, which can approximate arbitrary continuous non-linear maps on compacts as the size of the hidden layer grows to infinity. The cleanest result is perhaps the one of \citet{barron1993universal}, that we heavily use in this paper.\\
In the context of non-linear RNN approximation of dynamical systems~(e.g. in neuroscience), the state-to-state computation can be seen as part of an MLP~(see e.g.~\citet{hanson2020universal}): we have $x_k = \sigma(A x_{k-1} + B u_k)$, where $\sigma$ is is a non-linearity. As a result, wide non-linear RNNs can in principle approximate non-linear dyamical systems, as we discuss in detail in App.~\ref{app:rw_MLP}.\\
Meanwhile, linear RNNs, where $x_k = A x_{k-1} + B u_k$, have often been considered of minor interest, as they equivalent in approximation power to convolutions~\citep{li2022approximation}~(see App.~\ref{app:rw_MLP}). In this paper we take a different approach: we show that when sufficiently wide, the linear RNNs \emph{do not} form a bottleneck, and the architecture maintains universality through the application of the pointwise MLP on the hidden state, as done in recent SSMs achieving state-of-the-art results~\citep{gu2021efficiently, smith2022simplified, orvieto2023resurrecting}. As motivated thoroughly in the paper, this architecture unlocks parallel computation, in contrast to what is possible with directly placing non-linearities in the recurrence.

\section{Approximation theory for (non-linear) RNNs}
\label{app:rw_MLP}

We recall a result on universality of MLPs already stated in the main paper.

\barron*
\barronthm*


\subsection{Guarantees for RNNs with recurrent non-linearities} Research on universality of non-linear RNNs dates back to~\citep{siegelmann1992computational}. We present here a more recent result by \citet{hanson2020universal}.

\begin{theorem}[Universality of non-linear RNNs] Consider the continuous-time non-linear dynamical system with form
\begin{equation}
    \dot{\bar{x}}(t) = \bar f(\bar x(t), u(t)),\quad \bar y(t) = h(\bar x(t)),
    \label{eq:non-linear-RNN-ct}
\end{equation}
with $\bar x(t)\in\R^{\bar N}$, $u(t)\in\R^M$. Under some technical assumptions~(bounded input, non-chaotic $f$), for any $\epsilon>0$ there exists a non-linear RNN
\begin{equation}
    \dot x(t) = -\frac{1}{\tau} x(t) + \sigma(A x(t) + Bu(t)), \quad y(t) = Cx(t),
    \label{eq:non-linear-RNNapprox-ct}
\end{equation}
for some non-polynomial $\sigma$, $\tau>0$, $A\in\R^{N\times N}$, $B\in\R^{N\times M}$, $C\in\R^{M\times N}$ that approximates the solution to Eq.~\ref{eq:non-linear-RNN-ct} up to error $\epsilon$ uniformly in time, on compact sets of inputs.
\end{theorem}
The result above typically involves taking $N$~(RNN width) to infinity.

\textit{Proof.} We briefly outline the idea behind the proof, and invite the reader to refer to~\cite{hanson2020universal} for details. Approximating the solution to Eq.~~\ref{eq:non-linear-RNN-ct} is equivalent to approximating the infinitesimal solution generator, which is a non-linear function of $(x, u)$. By Barron's Theorem~(Thm.~\ref{thm:barron_thm}), this generator can be approximated by a one-layer MLP, that is exactly the right-hand side of Eq.~\ref{eq:non-linear-RNNapprox-ct}.
\proofend

\subsection{Guarantees for linear RNNs} Simply taking out the non-linearity from the recurrence in Eq.~\ref{eq:non-linear-RNNapprox-ct} restricts the function class to convolutions. To start, recall that the linear continuous-time RNN on one-dimensional inputs
\begin{equation*}
    \dot x(t) = A x(t) + B u(t), \quad y(t) = Cx(t),
\end{equation*}
with $A\in\R^{N\times N}$, $B \in\R^{N\times M}$, $C \in\R^{1\times N}$
has solutions given by a convolution.
\begin{equation*}
    x(t) = \int_0^t C^\top e^{As}B u(t-s) ds =: \int_0^t \rho(s)^\top u(t-s) ds =: \sum_i(\rho^i\star u^i)_t.
\end{equation*}
Let us call $\hat{\mathcal{H}_N}$ the class of functionals parametrizable with linear RNNs with hidden state of dimension $N$, and $\hat{\mathcal{H}} = \cup_{N\in\mathbb{N}_+}\mathcal{H}_N$. 
\begin{equation*}
    \mathcal{H}_N :=\left\{ \{H_t
: t \in \R\}, H_t(u) = \int_{0}^{t}C^\top e^{As} Bu(t-s) ds, C\in\R^{1\times N}, A\in\R^{N\times N}, B\in \R^{N\times M}\right\}.
\end{equation*}
It turns out that convolutions are dense in the class of linear functionals. Let $\mathcal{U} = C_0(\R,\R^d)$ with norm $\|u\| = \sup_{t\in\R} \|u(t)\|_\infty$.
\begin{theorem}[Linear functionals in $C_0(\R,\R^d)$ are convolutions \citep{li2022approximation}]
Let $\{H_t
: t \in \R\}$ be a
family of continuous, linear, causal, regular, and time-homogeneous functionals on $\mathcal{U}$, i.e. such that
\begin{enumerate}
    \item (Continuous)  $\forall t\in\R$, $\sup_{\|u\|<1} H_t(u)<\infty$.
    \item (Linear) $\forall t\in\R$, $u,v\in \mathcal{U}$ and $\nu,\lambda\in \R$, we have $H_t(\lambda u + \nu v) = \lambda H_t(u) + \nu H_t(v)$.
    \item (Causal) For all $u,v\in\mathcal{U}$ such that $u(s)=v(s)$ for all $s\le t$, he have $H_t(v)=H_t(u)$.
    \item (Regular) Let $(u^n)$ be a sequence in $\mathcal{U}$ s.t. $u^n(s)\to 0$ for almost every $s\in\R$, then, for all $t\in\R$, we have $\lim_{n\to\infty} H_t(u^n)=0$.
    \item (Time Homogeneous) For all $u\in \mathcal{U}$ let $u^{\tau}_t = u(t-\tau)$, then $H_{t}(u^{\tau}) = H_{t+\tau}(u)$. 
\end{enumerate}
Then, for any $\{H_t
: t \in \R\}$ there exist a function (a kernel) $\bar \rho:\R_+\to\R^M$ such that for all $t\in\R$.
\begin{equation*}
    H_t(u) = \int_{0}^\infty \rho(s)^{\top} u(t-s) ds = \sum_i(\bar \rho^i\star u^i)_t.
\end{equation*}
\end{theorem}


\begin{theorem}[Linear RNNs can parametrize any convolution \citep{li2022approximation}]
Let $\{H_t
: t \in \R\}$ be a
family of continuous, linear, causal, regular, and time-homogeneous functionals on $\mathcal{U}$. Then,
for any $\epsilon > 0$ there exists $\{\hat H_t: t \in \R\}\in\hat{\mathcal{H}}$ such that
$$
    \sup_{t\in\R}\sup_{\|u\|<1} \|H_t(u)-\hat H_t(u)\|\le \epsilon.
$$
\label{thm:linear_rnn}
\end{theorem}
The result above typically involves taking $N$~(RNN width) to infinity.

\paragraph{This paper.} There is a sizable gap between the power of nonlinear and linear RNNs. We show in this paper that placing a nonlinearity at the output of linear RNNs~(unlocking parallel computation) allows approximation of arbitrary regular non-linear sequence-to-sequence mappings.

% \section{Reconstruction of sparse multidimensional inputs}
% \label{app:multidim_rec}

% As initialization of $\Lambda$, we are going to make use of the following lemma from~\cite{orvieto2023resurrecting},
% \begin{lemma}
%  Let $u_1,u_2$ be independent uniform random variables on the interval $[0,1]$. Let $0\le r_{\min}\le r_{\max}\le1$. Compute $\nu = -\frac{1}{2}\log\left(u_1(r_{\max}^2-r_{\min}^2)+r_{\min}^2\right)$ and $\theta = 2\pi u_2 $. Then $\exp(-\nu+i\theta)$ is uniformly distributed on the ring in $\mathbb{C}$ between circles of radii $r_{\min}$ and $r_{\max}$. We denote this distribution as $\mathbb{T}(r_{\min}, r_{\max})$.
%  \label{lemma:sampling_exp}
%  \end{lemma}

% Assumption~\ref{ass:sparse} implies the result below, which is complemented by the simulation in Fig.~\ref{fig:prop1}\&\ref{fig:prop2} in the appendix. Fig.~\ref{fig:architecture} illustrates the result through a concrete example on the MNIST dataset.
% \begin{proposition}[LRU memory] Let $M$ dimensional input sequences $u\in\R^{M\times L}$, with $L\le L_{\max}$ be sampled from a distribution $\mathcal{U}$. Assume that $\mathcal{U}$ is such that Assumption~\ref{ass:sparse} holds true almost surely. Let $E:\R^{M}\to\R^{H}$, $H=M+1$, be an affine position-wise encoder that appends the feature $1$ for all multidimensional inputs: $u\overset{E}{\to}\tilde u$. Let then $\tilde u\in\R^{H\times L}$ be the input to the following recursive LRU computation: $x_k = \Lambda x_{k-1} + B\tilde u_{:,k}$, with $B\in \C^{N\times H}$ a specific matrix~(see Eq.\eqref{eq:matrix_B}), $\Lambda=\diag(\lambda_1,\lambda_2,\dots, \lambda_N)$ and each $\lambda_i\sim\mathbb{T}(r_{\min}, r_{\max})\subseteq\C$ according to Lemma~\ref{lemma:sampling_exp}. If $N $ big enough but still way smaller than $L_{\max}$
% %(~\antonio{To be precise is a bit hard, working on it})
% , there exists a linear map that from each $x_k$~(seen as a generic vector in $\C^N$, with no access to its index) can recover the coefficients $\alpha^{u,k}$ for the decomposition $u_{:,1:k} = \sum_{i=1}^P \alpha^{u,k}_i\psi^i_{:,1:k}$, with probability one. Moreover, there exists a non-linear continuous map that from $x_k$ can recover $k$.
% \label{prop:approx}
% \end{proposition}


% Since, while making use of Assumption~\ref{ass:sparse}, we want to write $u$ as a matrix multiplication of basis functions times coefficients, it is useful to work with vectorized quantities. Let us define
% \begin{equation}
%     \vect(u_{:,1:k}) := \begin{pmatrix}
% u_{1,1:k}^\top \vspace{1mm}\\
%  u_{2,1:k}^\top \\
%  \vdots \\
%  u_{M,1:k}^\top
%     \end{pmatrix}\in\R^{kM},
% \end{equation}
% Next, let us define the matrix
% \begin{equation}
%     \Psi_k := (\vect(\psi^1_{:,1:k}),\vect(\psi^2_{:,1:k}),\cdots, \vect(\psi^P_{:,1:k}))\in\R^{kM\times P}.
% \end{equation}
% Then, Assumption~\ref{ass:sparse} implies
% \begin{equation}
%     \vect(u_{:,1:k}) = \Psi_k \alpha^{u,k}
% \end{equation}
% where $\alpha^{u,k} = (\alpha_1^{u,k}, \alpha_2^{u,k},\cdots, \alpha_P^{u,k})^\top\in\R^{P\times 1}$, for all $k$. Note that the coding vector $\alpha$ has fixed size, independent of $k$.

% We assume the encoder $E:\R^{M}\to\R^{H}$, $H=M+1$, preceding the linear recurrent unit~(LRU) already appended the feature $1$ positionwise to each multidimensional input, such that $\tilde u_{:,i} = (1,u_{:,i})\in\R^{H}$ for all $i\in\{1,2,\dots,L\}$ --- we will use this dummy feature to recover the inner state position in the sequence. We denote by $\tilde u = (\tilde u_{:,1}, \tilde u_{:,2},\cdots, \tilde u_{:,L})\in\R^{H\times L}$ the LRU input. 
% \begin{equation}
%     \tilde u =
%     \begin{pmatrix}
%     1 & 1 & 1&\cdots& 1\\
%     u_{1,1} &u_{1,2} & u_{1,3}&\cdots&u_{1,L}\\
%     \vdots&\vdots & \vdots&\ddots&\vdots\\
%     u_{M,1} &u_{M,2} & u_{M,3}&\cdots&u_{M,L}\\
%     \end{pmatrix}
% \end{equation}
% All in all, we can write
% \begin{equation}
%     \tilde u_{:,1:k} = (1_{k\times 1}, u_{:,1:k})^\top,\quad \vect(u_{:,1:k}) = \Psi_k \alpha^{u,k}
% \end{equation}

% The LRU is $x_k = \Lambda x_{k-1} + B \tilde u_k$, where $x_k\in\R^N$ and $\tilde u_k\in\R^{H}$. We pick $N = RM+1$~(where $R>0$ we will fix later), and the following input projection $B\in\C^{N\times H}$:
% \begin{equation}
%     B = \begin{pmatrix}
% 1& 0 & \cdots & 0 & 0\\
% 0 & 1_{R\times 1} & \cdots & 0_{R\times 1} & 0_{R\times 1}\\
% 0 & 0_{R\times 1} & \cdots & 0_{R\times 1} & 0_{R\times 1}\\
% \vdots & \vdots & \ddots & \vdots & \vdots\\
% 0 & 0_{R\times 1} & \cdots & 1_{R\times 1} & 0_{R\times 1}\\
% 0 & 0_{R\times 1} & \cdots & 0_{R\times 1} & 1_{R\times 1}\\
% \end{pmatrix}.
% \label{eq:matrix_B}
% \end{equation}
% The result of the LRU computation is a block diagonal version of the Vandermonde multiplication we had in the simplified setting
% \begin{equation}
%     \tilde x_k  =
%     \begin{pmatrix}
%   \sum_{i=0}^{k-1}\lambda_0^i
%   & \rvline &  & \rvline & & \rvline & &  \rvline & \\
% \hline
% & \rvline & V_{k,1} & \rvline & & \rvline & &  \rvline & \\
% \hline
%   & \rvline &  &\rvline & V_{k,2}& \rvline & &  \rvline & \\
% \hline
%   & \rvline &  &\rvline & & \rvline & \ddots &  \rvline & \\
% \hline
%   & \rvline &  &\rvline & & \rvline && \rvline  &  V_{k,M} \\
% \end{pmatrix}
%     \begin{pmatrix}
%     1\\
%     \vspace{1mm}
% u_{1,1:k}^\top \\
%     \vspace{1mm}
%  u_{2,1:k}^\top \\
%  \vdots \\
%  u_{M,1:k}^\top
%     \end{pmatrix} = \tilde V_k  \begin{pmatrix}1\\
%     \vect(u_{:,1:k})\end{pmatrix},
% \end{equation}
% where $\tilde V_k=\diag(1,V_k)$, $V_k = \diag(V_{k,1},V_{k,2},\cdots,V_{k,M})$, and $V_{k,j}$ is the Vandermonde matrix of width $k$ corresponding to the block $\Lambda_j$ in the diagonal matrix $\Lambda$.
% \begin{align}
% &\Lambda = \diag(\lambda_0, \Lambda_1,\Lambda_2,\cdots,\Lambda_M)\in\C^{(RM+1)\times(RM+1)},\\
%     &\Lambda_j = \diag(\lambda_{1,j},\lambda_{2,j},\cdots,\lambda_{R,j})\in\C^{R\times R},\\
%     &V_{k,j} = \begin{pmatrix}
%     \lambda_{1,j}^{k-1}&\lambda_{1,j}^{k-2} &\cdots& \lambda_{1,j}&1\\
%     \lambda_{2,j}^{k-1}&\lambda_{2,j}^{k-2} &\cdots& \lambda_{2,j}&1\\
%     \vdots&\vdots &\ddots&\vdots&\vdots\\
%     \lambda_{R,j}^{k-1}&\lambda_{R,j}^{k-2} &\cdots& \lambda_{R,j}&1\\
%     \end{pmatrix}\in\C^{R\times k}.
% \end{align}
% We use Assumption~\ref{ass:sparse} to write
% \begin{equation}
%     x_k =  \tilde V_k \begin{pmatrix}1\\
%     \vect(u_{:,1:k})\end{pmatrix} = \begin{pmatrix}\sum_{i=0}^{k-1}\lambda_0^i\\
%      (V_k \Psi_k) \alpha^{u,k}\end{pmatrix},
% \end{equation}
% The timestamp $k$ can be recovered in a trivial way, under the assumption that $\lambda_0\in\C$ is such that $|\lambda_0|\le 1$:
% \begin{equation*}
%     x_{k,0}=\sum_{i=0}^{k-1}\lambda_0^i = \frac{\lambda_0^k-1}{\lambda_0-1}\quad\implies\quad k = \log_{\lambda_0}(x_{k,0}(\lambda_0-1)+1).
% \end{equation*}
% Next, we show how to recover the coding coefficients $\alpha^{u,k}$. Let us define
% \begin{equation}
%     \Omega_k = V_k\Psi_k,\qquad\text{such that}\quad x_k = \Omega_k \alpha^{u,k}.
% \end{equation}
% Now, to get $\alpha^{u,k}$ one can compute a~(left) pseudoinverse
% \begin{equation}
%     \alpha^{u,k} = \Omega_k^+ x_k.
% \end{equation}
% Since $\Omega_k\in\C^{(N-1)\times P}$, with $N = RM+1$, for the system to be overdetermined, a necessary condition is $N-1=RM > P$, i.e $R\ge \lceil P/M\rceil$. To be able to recover $\alpha^{u,k}$, we need $\Omega$ to have row rank bigger or equal to the column rank. In practice, for numerical feasibility, one needs $\Sigma_k$ not to be ill-conditioned. In Figures~\ref{fig:prop1},\&~\ref{fig:prop2} we show this is achievable by using $\Lambda$ initialized very close to the unit disk, i.e. under the LRU initialization. Under the assumption that $\Omega_k$ is not too ill-conditioned, one can perfectly recover $u_{1:k}$ with a linear map $\vec(u_{1:k}) = \Psi_k\Omega_k^+ x_k$.

\newpage
\section{Details on Multidimensional Input Reconstruction}

In the main text, we showed that linear diagonal RNN computations on one-dimensional input sequences can be written in matrix form using a Vandermonde matrix~(Sec.~\ref{sec:vandermonde}). For convenience of the reader, we repeat the reasoning here: let $H=M=1$, the encoder $e$ be the identity, and $B = (1,1,\dots, 1)^\top$. Then, eq.~(\ref{eq:lin_rnn_unroll}) can be written as
\begin{equation*}
    x_k   =
    \begin{pmatrix}
    \lambda_1^{k-1}&\lambda_1^{k-2} &\cdots& \lambda_1&1\\
    \lambda_2^{k-1}&\lambda_2^{k-2} &\cdots& \lambda_2&1\\
    \vdots&\vdots &\ddots&\vdots&\vdots\\
    \lambda_N^{k-1}&\lambda_N^{k-2} &\cdots& \lambda_N&1\\
    \end{pmatrix}
    \begin{pmatrix}
    u_1 \\ u_{2} \\ \vdots \\ u_k
    \end{pmatrix}= V_k u_{1:k}^\top.
\end{equation*}
where $u_{1:k}= v_{1:k} = (v_i)_{i=1}^k \in\R^{1\times k}$, and $V_k$ is a Vandermonde matrix. As long as $N\ge k$, we can hope to recover $u_{1:k}$ by pseudoinversion of the Vandermonde:
\begin{equation*}
    v_{1:k}^\top = u_{1:k}^\top = V_k^+ x_k,
\end{equation*}

Here, we give  details on the design of input projections such that the RNN output from multidimensional inputs can also be seen as matrix multiplication.  Let us define
\begin{equation*}
    \vect(v_{1:k}) := \begin{pmatrix}
v_{1,1:k}^\top \vspace{1mm}\\
 v_{2,1:k}^\top \\
 \vdots \\
 v_{M,1:k}^\top
    \end{pmatrix}\in\R^{kM},
\end{equation*}

The matrix $B$ we are going to use in our linear diagonal RNN is
\begin{equation*}
    B = \begin{pmatrix}
 1_{N'\times 1} & \cdots & 0_{N'\times 1} & 0_{N'\times 1}\\
 0_{N'\times 1} & \cdots & 0_{N'\times 1} & 0_{N'\times 1}\\
  \vdots & \ddots & \vdots & \vdots\\
 0_{N'\times 1} & \cdots & 1_{N'\times 1} & 0_{N'\times 1}\\
 0_{N'\times 1} & \cdots & 0_{N'\times 1} & 1_{N'\times 1}\\
\end{pmatrix},
\end{equation*}
where we select $N = MN'$. With this choice, the linear diagonal RNN output can be written as
\begin{equation*}
    x_k  =
    \begin{pmatrix}
V_{k,1} & \rvline & & \rvline & &  \rvline & \\
\hline
  &\rvline & V_{k,2}& \rvline & &  \rvline & \\
\hline
  &\rvline & & \rvline & \ddots &  \rvline & \\
\hline
  &\rvline & & \rvline && \rvline  &  V_{k,M} \\
\end{pmatrix}
    \begin{pmatrix}
v_{1,1:k}^\top \\
    \vspace{1mm}
 v_{2,1:k}^\top \\
 \vdots \\
 v_{M,1:k}^\top
    \end{pmatrix} = \tilde V_k  \begin{pmatrix}1\\
    \vect(v_{:,1:k})\end{pmatrix},
\end{equation*}

where $V_k = \diag(V_{k,1},V_{k,2},\cdots,V_{k,M})$, and $V_{k,j}\in\C^{N'\times k}$ is the Vandermonde matrix corresponding to the block $\Lambda_j$ in the diagonal recurrent matrix $\Lambda$:
\begin{align*}
&\Lambda = \diag( \Lambda_1,\Lambda_2,\cdots,\Lambda_M)\in\C^{(N'M)\times(N'M)},\\
    &\Lambda_j = \diag(\lambda_{1,j},\lambda_{2,j},\cdots,\lambda_{N',j})\in\C^{N'\times N'},\\
    &V_{k,j} = \begin{pmatrix}
    \lambda_{1,j}^{k-1}&\lambda_{1,j}^{k-2} &\cdots& \lambda_{1,j}&1\\
    \lambda_{2,j}^{k-1}&\lambda_{2,j}^{k-2} &\cdots& \lambda_{2,j}&1\\
    \vdots&\vdots &\ddots&\vdots&\vdots\\
    \lambda_{N',j}^{k-1}&\lambda_{N',j}^{k-2} &\cdots& \lambda_{N',j}&1\\
    \end{pmatrix}\in\C^{N'\times k}.
\end{align*}
This construction effectively decouples each input dimension and reduces the discussion to the one-dimensional setting: invertibility of $V_k$ is guaranteed by invertibility of each block, provided $N'\ge L$ and that eigenvalues are distinct. Slight changes can be made to keep also track of the timestamp~(see Sec.~\ref{sec:main_idea}) and to adapt to the sparse setting~(see Sec.~\ref{sec:sparse}).

\newpage
\section{Expressivity Proofs}

One of our main steps involved computation of the Barron constant of the function mapping the RNN hidden state to the output.
\mainMLPsing*

Since $\Omega_k$ is a matrix, the result can be proved by computing the Barron constant of a function where the argument is the output of a linear map.


\begin{restatable}[Change of variables]{lemma}{cov}
Let $A\in\R^{p\times n}$ and $f(x) = g(A x)$, then 
$$C_f = \|A\|_2 C_g.$$
\end{restatable}

\textit{Proof.} The inverse Fourier transform formula directly leads to
\begin{equation}
    f(x) = \int_{\R^p} e^{i \langle p, A x\rangle} \mathcal{G}(\xi)d\xi.
\end{equation}
Let us now compute the Fourier Transform of $f$.
\begin{align}
    \F(\omega) &= \int_{\R^n} e^{-i\langle \omega, x \rangle} f(x) dx\\
    &= \int_{\R^n} e^{-i\langle \omega, x \rangle} \left[\int_{\R^p} e^{i \langle \xi, A x\rangle} \mathcal{G}(\xi)d\xi\right] dx\\
    &= \int_{\R^n}\int_{\R^p} e^{-i\langle \omega, x \rangle}  e^{i \langle  A^\top \xi, x\rangle} \mathcal{G}(\xi)d\xi dx\\
    &= \int_{\R^p}\left[\int_{\R^n} e^{i \langle A^\top\xi  - \omega, x\rangle}dx\right] \mathcal{G}(\xi)d\xi.
\end{align}

Recall now the definition of the Dirac delta:
\begin{equation}
\delta(z) = \frac{1}{2\pi}\int_{\R}e^{i\nu z}d\nu.
\end{equation}
Therefore
\begin{equation}
    \F(\omega) =\int_{\R^p} \delta(A^\top\xi  - \omega)\mathcal{G}(\xi)d\xi.
\end{equation}
Note that this is a singular measure in $\R^n$: lives in a linear $p$-dimensional subspace. Further, note that
\begin{align}
    C_f &= \int_{\R^n}\|\omega\|_2\cdot|\F(\omega)|d\omega\\
    &= \int_{\R^n}\|\omega\|_2\cdot\big|\int_{\R^p} \delta(A^\top\xi  - \omega)\mathcal{G}(\xi)d\xi\big|d\omega\\
    &\le \int_{\R^n}\|\omega\|_2\cdot\int_{\R^p} \delta(A^\top\xi  - \omega)|\mathcal{G}(\xi)|d\xi d\omega\\
    &\le \int_{\R^p}\left[\int_{\R^n}\|\omega\|_2 \delta(A^\top\xi  - \omega) d\omega\right] |\mathcal{G}(\xi)|d\xi\\
    &= \int_{\R^p}\|A^\top\xi\|_2\cdot  |\mathcal{G}(\xi)|d\xi\\
    &= \|A^\top\|_2\int_{\R^p}\|\xi\|_2\cdot  |\mathcal{G}(\xi)|d\xi.
\end{align}
The proof is done, since $\|A^\top\|_2 = \|A\|$~(same singular values), and $C_g = \int_{\R^p}\|\xi\|_2\cdot  |\mathcal{G}(\xi)|d\xi$.
\proofend


We now proceed proving the complexity for interpolation of sequences of Barron functions. This directly implies our main result~(Thm.\ref{thm:universal}).


\mainbarronk*

\textit{Proof.} Let us consider the following definition for $\tilde f$:
\begin{equation}
    \tilde f(x,t) = \sum_{k=1}^L f(x,k) h(t-k),
\end{equation}
where $h:\R\to\R$ is a filter~(see discussion after the proof) with support in $[-1,1]$. Compactness in the support of $h$ leads to the desired property $\tilde f(x,k) = f(x, t)$, for all $k = 1,2,\dots L$. Let us now compute the Fourier transform of $\tilde f$. Frequencies are of the form $\omega=(w,\nu)$, with $w\in\R^n$, $\nu\in\R$. 
\begin{align}
    \tF(\omega) &= \frac{1}{(2\pi)^{n+1}} \int_\R \int_{\R^n} \tilde f(x,t) e^{-i\langle w,x\rangle} e^{-i\nu t} dx dt\\
    &= \frac{1}{(2\pi)^{n+1}} \int_\R \int_{\R^n} \left[\sum_{k=1}^L f(x,k) h(t-k)\right] e^{-i\langle w,x\rangle} e^{-i\nu t} dx dt\\
    &= \frac{1}{(2\pi)^{n+1}} \sum_{k=1}^L \left[ \int_{\R^n} f(x,k)  e^{-i\langle w,x\rangle}  dx\right]\cdot \left[\int_\R h(t-k) e^{-i\nu t} dt\right]\\
    &= \sum_{k=1}^L \tF_k(w) \mathcal{H}(\nu) e^{i\nu k},
\end{align}
where $\mathcal{H}$ is the Fourier transform of $h$, and the factor $e^{i\nu k}$ comes from the shift $h(\cdot -k)$. All in all, we get
\begin{equation}
    \tF(w,\nu) = \mathcal{H}(\nu) \sum_{k=1}^L \tF_k(w)  e^{i\nu k}.
\end{equation}
Trivially, 
\begin{equation}
    |\tF(w,\nu)| \le |\mathcal{H}(\nu)| \sum_{k=1}^L |\tF_k(w)|.
\end{equation}

Therefore:
\begin{align}
    \tilde C &= \int_{\R^{n+1}} \|\omega\|_2 |\tilde{\mathcal{F}} (\omega)| d\omega\\ &\le \int_{\R^n}\int_\R \|(w,\nu)\|_2 \cdot \left[ |\mathcal{H}(\nu)|\sum_{k=1}^L |\tF_k(w)|\right] dw d\nu\\
&=\sum_{k=1}^L \int_{\R^n}\int_\R \|(w,\nu)\|_2 \cdot |\mathcal{H}(\nu)|\cdot|\tF_k(w)| dw d\nu.
    \end{align}
    
Using the triangle inequality:
\begin{align}
    \tilde C &\le \int_{\R^n}\int_\R (\|w\|_2 + |\nu| )\cdot |\mathcal{H}(\nu)|\cdot|\tF_k(w)| dw d\nu\\ &= \int_{\R^n}\int_\R \|w\|_2\cdot |\mathcal{H}(\nu)|\cdot|\tF_k(w)| dw d\nu + \int_{\R^n}\int_\R |v|\cdot |\mathcal{H}(\nu)|\cdot|\tF_k(w)| dw d\nu\\
    &= C_{f_k} C'_{h}+ C_h C_{f_k}'.
\end{align}
This concludes the proof since~(see next paragraph) $C_h$ and $C'_h$ are problem-independent bounded constants.
\hfill $\square$

The proof of the theorem above concludes stating that $C_h$ and $C'_h$ are problem-independent bounded constants. Recall that, in our proof, $h:\R\to\R$ has compact support in $[-1,1]$. We can design $h$ such that its Fourier transform has fast decay:

\begin{restatable}[\cite{tlas2022bump}]{theorem}{tlas}
For any $\delta \in (0,1)$ and any $c >0$  there is a function $h(t)$ which is $C^\infty$, real, even, nonnegative, supported in $[-1,1]$ and whose Fourier transform $\mathcal{H}(\nu)$ is monotone decreasing for $\nu \geq 0$ and satisfies the following double inequality 
\begin{equation}
      \exp(- (1+\epsilon) c \nu^\delta ) \lesssim  \mathcal{H}(\nu)  \lesssim  \, \, \exp(- (1- \epsilon) c \nu^\delta),
\end{equation}
for any $\epsilon > 0$.
\end{restatable}
This result, rooted in the Beurling-Malliavin multiplier theorem\cite{mashreghi2006beurling}, ensures that we can design $h$ on a compact support with exponentially decaying frequencies. This implies that both integrals
\begin{equation}
    C_h = \int_{\R} |\nu| |\mathcal{H} (\nu)|d\nu,\qquad 
    C_h' = \int_{\R} |\mathcal{H} (\nu)|d\nu
\end{equation}
are bounded and independent on the specific form of the function $f$ we want to approximate.




% \subsection{Approximation of a target from last token representation}

% Consider the task of approximating a \textbf{target scalar function} over the whole sequence. That is, assume that 
% \begin{equation}
%     y = f_L(u_{1:L}) = f_L(u_1,u_2,\dots, u_L).
% \end{equation}
% Let us assume this function is Barron, that is, $C_f<\infty$. As such, using $u_{1:L}$ as input of a 1HL-MLP, one needs $\frac{4r^2 C_f^2}{\epsilon^2}$ hidden neurons to approximate $f$ at level $\epsilon$ on the compact set $\|u_{1:L}\|\le r$.

% Let us now consider instead approximating $f$ using an MLP head on the last token representation provided by a linear RNN. We now ask -- \textit{how many hidden neurons do we need to reconstruct $f$ directly from $x_L$}?

% As a first simplified setting, assume the RNN hidden state is as wide as the input length: $N = L$. In this case, $x_L = V_L u_{1:L}$ where the Vandermonde matrix $V_L\in\R^{L\times L}$ under the LRU initialization is invertible with probability 1 --- i.e. has non-vanishing determinant.

% We have the following result
% \begin{theorem}
% In the setting we just described, the number of hidden neurons needed to approximate $f$ at level $\epsilon$ from the last RNN hidden state $x_L$ is
% \begin{equation}
%     H\ge \frac{4 r^2 C_f^2 \|V_L^{-1}\|^2_2}{\epsilon^2},
% \end{equation}
% where $r$ bounds the hidden state magnitude.
% \label{thm:approx_barron_last}
% \end{theorem}

% Recall the change of variables formula:
% \begin{lemma}
% Let $A$ be an invertible matrix, then
% \begin{equation}
%     \int_{\R^n} g(Ax)dx = \frac{1}{|\det(A)|}\int_{\R^n} g(u)du.
% \end{equation}
% \end{lemma}

% This formula directly implies the next result:
% \begin{theorem}[Fourier Transform Stretch]
% Let $f$ be $L^1(\R^n)$. Let $W\in\R^{n\times n}$ be an invertible matrix and let $W^{-\top}$ be the inverse of its transpose. We have
% \begin{equation}
%     \mathcal{F}(f(W\cdot))(w) = \frac{1}{|\det(W)|}  \hat f(W^{-\top} w)
% \end{equation}
% \label{thm:fourier_of_stretch}
% \end{theorem}

% \textit{Proof of Thm.~\ref{thm:approx_barron_last}}
% In our setting, $f(u_{1:L}) = f(V_L^{-1} x_L):= \bar f_L(x_L)$. Theorem~\ref{thm:fourier_of_stretch} implies
% \begin{equation}
%     \hat{\bar {f}}_L(w) = \mathcal{F}(\bar f_L))(w) = \frac{1}{|\det(V_L^{-1})|}  \hat f(V_L^{\top} w)
% \end{equation}

% Let us then proceed computing the Barron constant of $f_L$
% \begin{equation}
%     C_{\bar f_L} = \frac{1}{|\det(V_L^{-1})|} \int_{\R^d} \|w\|_2 |\hat f(V_L^{\top} w)| dw<\infty.
% \end{equation}
% Consider the change of variables $u = V_L^{\top} w$. Then,
% \begin{align}
%     C_{\bar f_L} &= \frac{1}{|\det(V_L^{-1})|} \int_{\R^d} \|V_L^{-\top} V_L^{\top}w\|_2 |\hat f(V_L^{\top} w)|dw\\ &= \frac{1}{|\det(V_L^{-1})||\det(V_L^\top)|} \int_{\R^d} \|V_L^{-\top} u\|_2 |\hat f(u)|du\\
%     &= \int_{\R^d} \|V_L^{-\top} u\|_2 |\hat f(u)|du\\
%     &= \int_{\R^d} \|V_L^{-\top}\|_2 \| u\|_2 |\hat f(u)|du\\
%     &\le \|V_L^{-\top}\|_2 C_f\\
%     &= \|V_L^{-1}\|_2 C_f
% \end{align}
% This concludes the proof.

% \proofend

% Intuitively, the theorem we just proved shows that estimating $f$ from the hidden state is easy, if the maximum eigenvalue of $V_L^{-1}$ is not too large -- that we can obtain with diagonal recurrences with eigenvalues close to the unit disk.


% \begin{theorem} Set $\tilde C_f = \frac{1}{L}\sum_{k =0}^{L-1} C_{f_k} \|V_k^{-1}\|_2$ and $L_{1}^{f,a} = \frac{1}{L}\sum_{k =0}^{L-1} 2\sqrt{\frac{a}{\pi}} \|\hat{ \bar{f}}_k\|_{L_1}$.
% In the setting we just described, with level of smoothing $a$ the number of hidden neurons needed to approximate $\tilde f(x,k)$ at level $\epsilon$ from the last RNN hidden state is
% \begin{equation}
%     H\ge \frac{4 r^2 L (\tilde C_f + L_{1}^{f,a})}{\epsilon^2},
% \end{equation}
% where $r$ bounds the hidden state magnitude and $L$ the makixum sequence length.
% \label{thm:approx_barron_all}
% \end{theorem}



% \paragraph{Warmup.} Consider as a warm-up the case where we have $f(x) = g(a^\top x)$, where $x, a\in\R^d$. Given $C_g$ the Barron constant for $g:\R\to\R$, what is the Barron constant for $f$?

% First, note that one can write
% \begin{equation}
%     f(x) = \int_{\R} e^{i\nu \langle a,x\rangle}\mathcal{G}(\nu)d\nu
% \end{equation}
% Indeed, recall the Dirac identity
% \begin{equation}
%     \delta(z) = \frac{1}{2\pi}\int_{\R}e^{i\nu z}d\nu.
% \end{equation}
% From this, we get
% \begin{align}
%     f(x) & = g(\langle a,x\rangle)\\
%     & = \int_{\R}\delta(\langle a,x\rangle-y) g(y) dy\\
%     & = \frac{1}{2\pi}\int_{\R}\int_\R e^{i (\langle a,x\rangle-y) \nu} g(y) d\nu  dy\\
%     & = \frac{1}{2\pi}\int_{\R}\int_\R e^{-iy\nu} e^{i\langle a,x\rangle \nu} g(y) d\nu  dy\\
%     & = \int_{\R} e^{i\langle a,x\rangle \nu}\left[\frac{1}{2\pi}\int_\R e^{-iy\nu}  g(y) dy\right]  d\nu\\
%     & = \int_{\R} e^{i\nu \langle a,x\rangle}\mathcal{G}(\nu)d\nu
% \end{align}

% \begin{restatable}[Change of variables]{lemma}{cov}

% \end{restatable}

% \textit{Proof.}
% let $A_i$ be the i-th row of $A$ and $y_i$ be the $i$-th element of $y$.
% \begin{align}
%     f(x) & = g(Ax)\\
%     & = \int_{\R^m}\left[\prod_{i=1}^m\delta(A_ix-y_i)\right] g(y) dy\\
%     & = \frac{1}{(2\pi)^m}\int_{\R^m}\left[\prod_{i=1}^m\int_\R e^{i (A_ix-y_i) \nu_i} d\nu_i\right] g(y)   dy\\
%     & = \frac{1}{(2\pi)^m}\int_{\R^m}\int_{\R^m} e^{i \langle v, A x-y\rangle} g(y)  d v  dy\\
%     & = \int_{\R^m} e^{i \langle v, A x\rangle} \left[\frac{1}{(2\pi)^m}\int_{\R^m}  e^{-i \langle v, y\rangle} g(y)   dy\right]  d v\\
%     & = \int_{\R^m} e^{i \langle v, A x\rangle} \mathcal{G}(v)dv.
% \end{align}
% \proofend

\newpage
\section{Additional experiments}
\subsection{Reconstruction using the Vandermonde inverse~(support to Section~\ref{sec:vandermonde})} 
\label{app:rec_vandermonde}
We consider linear diagonal RNNs with the $N$ diagonal entries of $\Lambda$ sampled inside the unit disk in $\C$, uniformly in angle in between radii $r_{\min}$ and 1. We consider the hidden state $x_L\in\C^N$ computed after $L$ RNN steps, \textit{i.e. after the sequence is completely processed}. We want to recover the input sequence from the hidden state using the Vandermonde inverse $V_L^+$~(see Sec.~\ref{sec:vandermonde}). If $N\ge L$, since under random initialization the determinant of any set of $L$ columns of $V_L$ is positive, we can in theory achieve perfect reconstruction. In practice, $V_L$ is ill conditioned --- especially if $r_{\min}$ is not close to $1$. This causes some problem in the pseudoinverse computation, which may result in imperfect reconstruction. 

In Figure~\ref{fig:MNIST_rec_random_van} we provide evidence on the MNIST dataset~(flattened image $=$ 784 tokens): we observe that, if eigenvalues are sampled with $r_{\min}=0$, by multiplying $x_L$ by $V_L^+$ we are only able to recover recent history. Instead, moving closer to the unit disk allows almost perfect reconstruction at $N=784$, and a satisfactory result already at $N=512$.

% Figure environment removed

 In Figure~\ref{fig:MNIST_van_error_avg} we clearly show that the average reconstruction error~(average over $10k$ images samples and 10 random re-samplings of the linear RNN) is decreasing both as a function of the hidden state size~(see discussion in Sec.~\ref{sec:vandermonde}) and of $r_{\min}$~($r_{\max}=1$). The same pattern is observed for the condition number of $V_L^\top V_L$. On the same figure, we show how the error is distributed over timestamps: it is clear that, for $r_{\min}\ll1$, the reconstruction only covers the last few hundreds of tokens -- a property which is liked to the bad condition number observed in this setting.

% Figure environment removed

 Last, in Figure~\ref{fig:MNIST_rec_unit_van} we show what happens when picking the $N$ diagonal entries of $\Lambda$ to be the $N$-th complex roots of $1$: as shown in~\cite{cordova1990vandermonde}, in this setting the Vandermonde condition number is $1$. We observe that we can indeed reconstruct perfectly the output for $N=784$. However, for smaller values of $N$, the reconstruction presents undesired artifacts.

% Figure environment removed

\subsection{Reconstruction under sparsity~(support to \S\ref{sec:sparse})}
\label{app:rec_sparse}


In Figures~\ref{fig:prop1} and \ref{fig:prop2} we test the discussion in Sec.~\ref{sec:sparse} in a controlled setting. We consider one-dimensional stream of $L=4096$ random inputs sparse in a basis of $P=32$ Haar wavelets~\citep{haar1911theorie}. The linear diagonal RNN has $\Lambda\in\C^{N\times N}$ with eigenvalues sampled uniformly at random from $\mathbb{T}(0.95,1)$~(Fig.~\ref{fig:prop1}) or $\mathbb{T}(0.99,1)$~(Fig.~\ref{fig:prop2}), where
$$\mathbb{T}[r_{\min}, r_{\max}] := \{\lambda\in\C \ | \ r_{\min}\le|\lambda|\le r_{\max}\}.$$
We use matrix $B = (1,1,\dots, 1)^\top$. Plotted is the rank of $V_k,\Psi_k$, $\Omega_k = V_k\Psi_k$ as $k$ increases~(see notation in Sec.~\ref{sec:sparse}). We show how the reconstruction error behaves when reconstructing $u_{1:k} = \Psi_k\alpha^{u}_k$ with $\alpha^{u}_k = \Omega_k^+ x_k$. In the figures, we plot the error for reconstruction of the tokens $(u_i)_{i=1}^k$ from $x_k$, for all $k\le L$. As $N$ gets larger the reconstruction error gets uniformly negligible. In particular, if we initialize in $\mathbb{T}(0.95,1)$ then the minimum $N$ we need for perfect reconstruction of around $N=256$. If instead we initialize closer to the unit circle, then $N=64$ is enough for perfect reconstruction. This finding is similar to the one presented in Sec.~\ref{app:rec_vandermonde}. 

On a more fundamental level, we study the condition number of the matrix $\Omega^T\Omega$, where $\Omega = V_L\Psi_L$. This condition number quantifies the numerical stability of the pseudoinverse $\Omega^+$, used in reconstruction. In Figure~\ref{fig:better_conditioning}, we show the logarithm of the condition number for a sequence of length $512$ sparse in a basis of $P$ eigenfunctions. As $r_{\min}$ gets close to $1$ the condition number decreases as we saw in \S\ref{app:rec_vandermonde}. Crucially however, the condition number also decreases as $P$ decreases.


% Figure environment removed

% Figure environment removed

% Figure environment removed

\subsection{Approximation of sequence-to-sequence maps~(ODE Systems)}
We consider approximating sequence-to-sequence maps $(v_i)_{i=1}^L\overset{T}{\mapsto} (y_i)_{i=1}^L$ defined by Runge-Kutta discretization of the flow of a controlled differential equation $\dot z_t = f(z_t, v_t), y_t = h(z_t)$, where $(v_t)_{t}$ is the input, $f$ is a non-linear multidimensional function, $h$ projects the multidimensional state $z_t$ into a one-dimensional output. An example is the \textbf{Protein Transduction~(PT)} system~\citep{vyshemirsky2008bayesian}:
 \begin{align*}
     &\dot z_1(t) = -k_1 z_1(t) - k_2 z_1(t) z_3(t) + k_3 z_4(t) + v(t)\\
     &\dot z_2(t) = k_1 z_1(t)\\
     &\dot z_3(t) = - k_2 z_1(t) z_3(t) + k_3 z_4(t) + V \frac{z_5(t)}{K_m + z_5(t)}\\
     &\dot z_4(t) = k_2 z_1(t) z_3(t) - (k_3+k_4)z_4(t)\\
     &\dot z_5(t) = k_4 z_4(t) - V\frac{z_5(t)}{K_m + z_5(t)}
 \end{align*}
\vspace{-2mm}

We identify $y_k = z_1(\Delta k)$, where $\Delta = 0.01$, and $v_k = v(\Delta k)$. We sample $(v_i)_{i=1}^L$~($L=2048$ in PT) from a linear combination of 16 base wavelets of low frequency (\textit{bias: input is random but has slow variations}), and set the ground truth $(y_i)_{i=1}^L$ to be the result of Runge-Kutta integration with stepsize $\Delta$. As hyperparameters, we use $k_1 = 0.07, k_2 = 0.6, k_3 =
0.05, k_4 = 0.3, V = 0.017, K_m = 0.3, z_1(0)=1, z_2(0) = 0, z_3(0) = 1, z_4(0) =  z_5(0) = 0$ as prescribed by~\citet{vyshemirsky2008bayesian}. Results using approximation of one linear RNN followed by a 1HL-MLP~(shared across timestamps) are shown in Fig.~\ref{fig:ptpz} and discussed in the main text. To train, we use $10k$ random (low frequency) trajectories and test on $1k$ trajectories with same distribution. Experiments run on a single A5000 using an LRU in JAX~(\url{https://github.com/NicolasZucchet/minimal-LRU/tree/main/lru}).

In this appendix, we additionally discuss performance in approximating the solution of two other controlled ODEs. Settings are same as used for PT unless stated otherwise. The first ODE~(results in Fig.~\ref{fig:res_lv}) is a \textbf{Lotka-Volterra (LV)} system~\citep{lotka1925elements, volterra1928variations}:
\begin{align*}
\dot z_1(t) &= z_1(t)(a-b\cdot z_2(t)) \\
\dot z_2(t) &= -z_2(t)(c-d \cdot z_1(t))+v(t)
\end{align*}
where $a = 1.0, b= 0.6, c=1.0, d=0.7$, and we initialize $z_1(0) = 1.0 , z_2(0) = 0.5$ as used in~\citet{dondelinger2013ode}. For integration, we use a stepsize of $\Delta =0.01$ and $y_k = z_1(\Delta k)$. Here, sequence length is again $L=2048$.

Finally, we consider an extremely \textit{challenging scenario}: the \textbf{Lorentz system~(LZ)}~\citep{lorenz1963deterministic}, which notoriously has chaotic solutions~(named strange attractor, linked to the butterfly effect):
\begin{align*}
\dot z_1(t) &= \sigma \cdot (z_2(t)-z_1(t)) \\
\dot z_2(t) &= (r-z_3(t))z_1(t)-z_2(t)+ v(t) \\
\dot z_3(t) &= z_2(t)z_1(t)-b\cdot z_3(t)
\end{align*}
With our parameter choices $\sigma = 10, r= 26, b =8/3$ and initialization $z_1(0) = -0.89229143 , z_2(0) = 1.08417925, z_3(0) = 2.34322702$~(parameter source: Wikipedia, visited September 2023), the system has a chaotic behavior as shown in Figure~\ref{fig:lorentz_ill}. We choose an integration timestep $\Delta = 0.002$ and, due to the non-linear chaotic and possibly unstable nature of the controlled attractor, consider $L=512$ in this setting.
\vspace{-3mm}
\paragraph{Note on training.} Training one layer of linear RNN + MLP on these ODEs exhibits huge variation across seeds~(see Fig.~\ref{fig:seed_variability}). Since in this paper we want to show that \textit{there exist} a Linear RNN+MLP configuration able to model non-linear sequence to sequence maps, we consider the following setup: we run each experiment and hyperparameter sweep on seeds $1-6$, and only report the best performance on the test data (train loss always lower). Based on our experience with the LRU, we conclude that instability is due to the small dimension of our model: performance on standard tasks is more stable as depth increases~\citep{orvieto2023resurrecting}. To test our theory, we limit ourselves to a linear RNN with either $N=128$ or $N=256$, followed by an MLP with one hidden layer. We grid-search hyperparameters for each model configuration and report test error for the best-performing models. Usual best-performing stepsizes are $0.003$ and $0.01$. In Lotka-Volterra experiments~(Fig.~\ref{fig:res_lv}) we train for $200$ epochs, while we train on the Lorentz System~(Fig.~\ref{fig:res_lz}) for $1000$ epochs. No weight decay, dropout or normalizations are applied. \textbf{Results are discussed in the relative figure captions and nicely validate our claims}.  \textit{Code for reproducing the experiments will be provided upon acceptance of this paper}.


% Figure environment removed

% Figure environment removed

% Figure environment removed


% Figure environment removed


%%%%%%%%%%%%%%%%%%%%%%%%%%%%%%%%%%%%%%%%%%%%%%%%%%%%%%%%%%%%%%%%%%%%%%%%%%%%%%%
%%%%%%%%%%%%%%%%%%%%%%%%%%%%%%%%%%%%%%%%%%%%%%%%%%%%%%%%%%%%%%%%%%%%%%%%%%%%%%%
% APPENDIX
%%%%%%%%%%%%%%%%%%%%%%%%%%%%%%%%%%%%%%%%%%%%%%%%%%%%%%%%%%%%%%%%%%%%%%%%%%%%%%%
%%%%%%%%%%%%%%%%%%%%%%%%%%%%%%%%%%%%%%%%%%%%%%%%%%%%%%%%%%%%%%%%%%%%%%%%%%%%%%%
\end{document}
