% !TeX root = main. tex
\begin{table}[t]
    \centering
    \caption{\textnormal{Symbols and Definitions.}}
    \label{tab:notations}
    \vspace{-0.1in}
    \begin{tabular}{c|c} \toprule
        \textbf{Symbol} & \textbf{Definition and Description}\\ \midrule
        $\graph(\nodes,\edges,\weights)$ & \makecell{Graph $\graph$ with nodeset $\nodes$,\\ Edgeset $\edges$ and edge weights $\weights$} \\
        $M$,$N$         & Number of edges and nodes of a graph \\
        $\subnode$          & Subset of nodes, i.e., $\subnode \subseteq \nodes$ \\
        $\subnode^{*}$ & The nodeset of the densest subgraph\\
        $\edges(\subnode),\weights(\subnode)$& the edgeset and total weights induced by $\subnode$\\
        $\rho \left ( \subnode \right )$ & The density of the nodeset $\subnode$\\
        $\rho^{*}$ & The density of the densest subgraph\\
        $\setndeg{\subnode}{u}$ & The degree of node $u$ in $\subnode$ \\
        % $\edges(\subnode),\weights(\subnode)$ & Edges induced by $\subnode$ and total weights in $\subnode$ \\
        $l_u, w_e$            & Load of the node $u$ and Weight of the edge $e$ \\
        % $\setndeg{\graph}{u}$ & Degree of node $u$ \\
        $f_{e}(u)$            & The weight distributed to node $v$ from edge $e$ \\
        \bottomrule
    \end{tabular}
\end{table}

In this section, we formally define the densest subgraph detection and 
locally-dense decomposition problem that we focus on. 
Table~\ref{tab:notations} summarizes the main symbols used in the paper.

Let $\graph = (\nodes, \edges, \weights)$ be an undirected graph 
with $N = |\nodes|$ vertices and $M = |\edges|$ edges.
For any $e \in \edges \subseteq \nodes \times \nodes$, 
its weight is $w_{e} \in \weights$ with $w_{e} \in \numR_{+}$ 
and $w_{e} = 1$ for the weighted and unweighted graph, respectively.
Given a node subset $\subnode \subseteq \nodes$, 
$\edges(\subnode)$ denotes the set of edges and 
$\weights(\subnode)$ denotes the total weights induced by $\subnode$,
and $\setndeg{\subnode}{u}$ is the degree of $u$ in the induced subgraph, 
i.e., the total weights of edges connected to $u$ within the set $\subnode$. Bold letters are used to represent vectors.


% \textit{Density:} 
Given the nodeset $\subnode$, 
the \emph{edge density} of the subgraph $\graph(\subnode)$ is defined as 
\begin{equation*}
    \rho(\subnode) \coloneqq \frac{\weights(\subnode)}{|\subnode|} 
% = \frac{\sum_{e \in \subnode} w_e}{|\subnode|} 
= \frac{\sum_{u \in \subnode}\setndeg{\subnode}{u}}{2 \cdot |\subnode|}.
\end{equation*}
Accordingly, we present the formal definition of the densest subgraph problem as below:
% Table~\ref{tab:notations} lists the main symbols we used in the paper.


\begin{problem}[Densest Subgraph Problem (DSP)]
    \label{prob:dsp}
    Given an undirected graph $\graph = (\nodes, \edges, \weights)$, 
    find the subset of nodes $\optset$ such that 
    $\optset = \argmax_{\subnode \subseteq \nodes} \rho(\subnode)$.
\end{problem}

Another variant of DSP is more general, which is called locally-dense decomposition.

\begin{definition}[Locally-Dense Decomposition (LDD) \cite{tatti2015density,tatti2019density}]
    Given an undirected graph $\graph=(\nodes, \edges, \weights)$, 
    it has a nested decomposition consisting of a sequence 
    $\emptyset=B_0 \subsetneqq B _1 \subsetneqq \ldots \subsetneqq B_k = \nodes$. 
    We define $B_i$ as the maximal densest subgraph properly containing $B_{i-1}$, that is,
    $$ B_i = \argmax_{{S \supsetneqq B_{i-1}}}{\frac{\weights(S)-\weights(B_{i-1})}{| S \setminus B_{i-1}|}} $$
\end{definition}

% \todo{How does the local-density decomposition come? Is it necessary for the paper? 
% There should be a brief introduction if it is.}

As we can see, DSP is a sub-problem of locally-dense decomposition, because $B_1$ is the maximal densest subgraph of the whole graph $\graph$, corresponding to the target in Problem~\ref{prob:dsp}. LDD is also closely related to convergence analysis for iterative update methods towards DSP~\cite{danisch2017large, boob2020flowless, harb2022faster}.
Formally, the locally-dense decomposition problem over $\graph$ is formulated as:
% Here we list another definition, i.e., locally-dense decomposition and corresponding problem, DSP is a subproblem of locally-dense decomposition problem and many iterative methods towards the LP dual of DSP can converge to locally-den decomposition, including Frank-Wolfe in \cite{danisch2017large}, Greedy++ in \cite{boob2020flowless} and FISTA in \cite{harb2022faster}.

\begin{problem}[Locally-dense decomposition             Problem\cite{tatti2015density,tatti2019density,danisch2017large}]
    \label{prob:locally-dense}
    Given an undirected graph $\graph=(\nodes, \edges, \weights)$, 
    find its locally-dense decomposition 
    $\emptyset=B_0 \subsetneqq B _1 \subsetneqq \ldots  \subsetneqq B_k=\nodes$.
\end{problem}

Based on the above problem definition, 
we summarize some important properties of LDD.
% Here are some important properties of locally-dense decomposition: 

\begin{property}[\cite{danisch2017large}]
    Given an undirected graph $\graph=(\nodes, \edges, \weights)$, its locally-dense decomposition is unique. 
    \label{prop:uniquity1}
\end{property}

\begin{property}[\cite{harb2022faster}]
    And for any $u \in B_i$, let 
    \begin{equation*}        \lambda_u=\lambda_i\coloneqq{\frac{\weights(\edges(W))-\weights(\edges(B_{i-1}))}{|W\setminus B_{i-1}|}},
    \end{equation*}
    there is a unique optimal $\bm{\ell}^*$ so that ${\ell_v}^*=\lambda_v$ for each node $v$. And $\lambda_1>\lambda_2>...>\lambda_k$.
    \label{prop:uniquity2}
\end{property}

\begin{property}[\cite{danisch2017large,ma2022finding}]
    For an optimal solution $(\bm{f}^*,\bm{\ell}^*)$, if there is an edge $e=(v_1,v_2)$ with ${\ell_{v_1}}^*>{\ell_{v_2}}^*$, then ${f_e}^*(v_1)=0$.
    \label{prop:one-way}
\end{property}


Properties \ref{prop:uniquity1} and \ref{prop:uniquity2} imply that in the unique locally-dense decomposition, each node in set $B_i\setminus B_{i-1}$ has the same load $\lambda_i$, and the nodeset in the inner nodeset in the inner layer is denser than one in the outer layer. Property \ref{prop:one-way} means the weight of edge e connecting two layers should be only distributed to the node with a lower load in one-way. 

In section \ref{sec:method} we will introduce an algorithm which provides an iterative operation to solve both DSP and locally-dense decomposition problem.

