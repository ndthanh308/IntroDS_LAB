% !TeX root = main. tex
In this paper, we propose \emph{\method} to detect the densest subgraph and prove it can converge to locally-dense decomposition. \emph{\method} redistributes edge weight in a locally optimal operation according to the linear programming of the DSP and quadratic programming of locally-dense decomposition. Besides, we develop a pruning technology using modified Counting Sort and prove that it is a subprocess of Greedy. We did a lot of experiments to exhibit its pruning efficiency on 26 real-world datasets and compare it with other algorithms about Greedy and Max-Flow. We also use it to prune the graph to speed up the iterative algorithms. In our experiments, \emph{\method} can converge to the optimal values both in the linear programming of the DSP and quadratic programming of locally-dense decomposition faster than other state-of-the-arts iterative algorithms, including Frank-Wolfe in \cite{danisch2017large}, Greedy++ in \cite{boob2020flowless} and FISTA in \cite{harb2022faster}. It also performs better than a version of MWU in \cite{harb2022faster}. 

Through our study, there are far more interesting topics in DSP and LDD that we can study in future work. We list them as follows:\\
\vspace{-0.15in}
\begin{enumerate}[label={\arabic*.}]
    \item What is the relationship between the size of the graph after pruning and the size of the whole graph (it's possibly related to properties of Kronecker graphs in \cite{leskovec2010kronecker}, which concerns how graphs evolve)?
    \item In our experiments, Kamada-Kawai path-length has a close relationship with LDD because we use Kamada-Kawai path-length to decide the position of nodes. How can we theoretically analyze the relationship between these two things?
    \item Can we discover the relation between iteration count $T$ and the approximate ratio?
\end{enumerate}

These problems also show that there is a close relationship among DSP, locally-dense decomposition and other theories in the graph field like DkS and locally densest subgraph in \cite{ma2022finding}. Understanding the relationship among them can help us mine dense subgraphs in more efficient and meaningful ways, and we can use dense subgraphs to discover more characteristics in the graph.