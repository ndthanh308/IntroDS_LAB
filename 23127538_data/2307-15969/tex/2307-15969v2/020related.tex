% !TeX root = main. tex
The most recent surveys~\cite{lanciano2023survey,luo2023survey} present a systematic, thorough overview and
summarization of the densest subgraph problem, and the tutorials~\cite{Gionis2015DSD, fang2022densest} also
give a comprehensive survey of the discovery of the densest subgraphs on large graphs and 
discuss the challenges and various applications.

For the graph with non-negative edge weights, the densest subgraph can be identified in polynomial time by 
solving a maximum flow problem~\cite{goldberg1984finding, gallo1989fast, khuller2009finding}; 
Charikar~\cite{charikar2000greedy} introduces a linear programming formulation of the problem and shows that 
the greedy algorithm proposed by Asashiro et al.~\cite{asahiro2000greedily} produces 
a $\frac{1}{2}$-approximation of the optimal density in linear time. 
\cite{danisch2017large} devises an efficient algorithm via convex programming, which can compute the exact locally-dense decomposition 
in real graphs with billions of edges, and proposes an $(1+\epsilon)$-approximation solution based on the Frank-Wolfe algorithm. 
Boob et al.~\cite{boob2020flowless} developed a simple iterative peeling algorithm Greedy++ to improve the quality of the subgraph over 
Charikar's greedy algorithm by drawing insights from the iterative approaches (multiplicative weights update) of convex optimization; 
the history of peeling information of nodes will help to escape the local solution to some extent. 
\cite{sawlani2020near} provided an algorithm for maintaining an $(1 - \epsilon)$-approximate (arbitrarily close to 1) densest subgraph 
within $O(\mathrm{poly}\log n)$ time over dynamic directed graphs, and extended to solve the problem on vertex-weighted static graphs. 
Feng et al.~\cite{feng2021specgreedy} proposed a generalized framework for addressing DSP and related 
problems~\cite{hooi2016fraudar, miyauchi2018finding, anagnostopoulos2020spectral, Tsourakakis2019NovelDS} and introduced SpecGreedy, 
an algorithm that leverages the graph spectral properties to a greedy peeling strategy to solve the generalized problem and speed up the detection. 
Chekuri et al.~\cite{chekuri2022densest} exploited the supermodular maximization and proposed more efficient $(1 - \epsilon)$-approximation algorithms in deterministic $\tilde{O}(m / \epsilon)$ time via approximate flow techniques for DSP, 
and gives evidence of the convergence and theoretical truthfulness of Greedy++, that is, 
it can converge to a $(1-\epsilon)$-approximation in $O(1/\epsilon^2)$ iterations; 
it also developed an $\frac{1}{2}$-approximation peeling algorithm for the densest-at-least-k subgraph. 
\cite{fazzone2022discovering} modified Greedy++ to have a quantitative certificate of the solution quality provided by the algorithm at each iteration. 
\cite{harb2022faster} proposed another iterative method using Proximal Gradient Method, which achieves $(1-\epsilon)$-approximation in $O(1/\epsilon)$, they also proposed a technique called Fractional Peeling to make use of the information in edge distribution. 
For directed graphs, the LP-based approach proposed by Charikar~\cite{charikar2000greedy} 
requires the computation of $n^2$ linear programs, and the $\frac{1}{2}$-approximation runs $O(n^3 + mn^2)$ time, 
\cite{khuller2009dense} provided more efficient implementations for these algorithms for undirected and directed graphs.

There is another research line that discovers the densest subgraph building upon some microstructures (motifs) in a graph, 
including the triangles~\cite{tsourakakis2015k, samusevich2016local}, cliques~\cite{sun2020kclist++, Tsourakakis2013Denser, fang2019efficient}, 
$k$-core~\cite{galimberti2017core}, $k$-club / $k$-plex, etc., and proposed corresponding different variants for the density measures. 
\cite{Mitzenmacher2015SLN, tsourakakis2015k} extended the DSP to the $k$-clique, and the $(p,q)$-biclique densest subgraph problems, 
which can be used to find large near-cliques. %\cite{Tsourakakis2013Denser} developed the quasi-cliques to 
%detect dense subgraphs but necessarily not fully interconnected; finding the optimal quasi-cliques is, however, NP-hard. 
%\cite{tatti2015density,tatti2019density} use k-core to detect locally-dense subgraphs in $\frac{1}{2}$-approximation guarantee. 
Tatti and Gionis~\cite{tatti2015density,tatti2019density} introduced the locally-dense graph decomposition method, 
which imposes certain insightful constraints on the $k$-core decomposition. 
\cite{fang2019efficient} proposed exact and approximate solutions by improving the flow-based exact algorithm by 
locating the densest subgraph in a specific $k$-core, which can be generalized 
by considering an arbitrary pattern graph and aiming to maximize the average number of occurrences of the pattern in the resulting subgraph. 
\cite{ma2020efficient} proposed $[x,y]$-core-based algorithms (both exact and approximation) 
with the divide-and-conquer strategy to find the densest subgraph for directed graphs.

When restrictions on the size of nodeset are imposed, the DSP also becomes NP-hard~\cite{andersen2009finding}, 
which is called the \textit{densest k-subgraph} (DkS). Its two variants called densest \textit{at-least-k subgraph} (DalkS) 
and densest \textit{at-most-k subgraph} (DamkS) are also NP-hard according to \cite{khuller2009finding,andersen2009finding}.

The dense subgraphs are used to detect communities~\cite{Chen2010Dense, costa2015milp, wong2018sdregion} and 
anomalies~\cite{prakash2010eigenspokes, beutel2013copycatch, hooi2016fraudar}. 
As one of the key characteristics, density, as well as other similar metrics like modularity~\cite{newman2006modularity}, 
associativity, and local density~\cite{qin2015locally}, are used as (part of) optimization objectives to detect community structures. 