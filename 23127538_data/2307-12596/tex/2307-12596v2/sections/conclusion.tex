\section{Conclusion}
\label{sec:conclusion}

In this study, we conducted a systematic analysis of ChatGPT-generated code to assess its reliability and identify potential code quality issues.
Our findings demonstrate that while ChatGPT can generate functional code for various programming tasks, the generated code often suffers from quality issues, such as compilation and errors, wrong outputs, maintainability problems, and performance inefficiencies.
We also explored ChatGPT's self-repairing capabilities and investigated the impact of different feedback strategies in addressing these code quality issues.

Our research provides valuable insights into the current limitations of ChatGPT and highlights the importance of considering context-aware feedback and code quality issues when utilizing AI-driven code generation tools. 
Moreover, our work offers insights for future research and development efforts aimed at enhancing the code generation capabilities of AI models like ChatGPT. 
We believe that by addressing these challenges, we can pave the way for more reliable, efficient, and maintainable AI-generated code, ultimately benefiting both experienced developers and novice programmers alike. In the future, we plan to develop more advanced prompts in both code generation and fixing to further improve the reliability of ChatGPT-generated code. 