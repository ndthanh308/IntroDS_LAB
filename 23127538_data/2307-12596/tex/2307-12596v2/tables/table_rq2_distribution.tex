\begin{table}[t]
    \centering
    \caption{Distribution of issues across difficulty levels and programming languages. P and J denotes Python and Java, respectively.}
    \label{tab:rq2_issues_distribution}
    \scalebox{0.65}{
    \begin{tabular}{lccccccccc}
    \toprule
          & \multicolumn{2}{c}{\textbf{Easy (501)}} & \multicolumn{2}{c}{\textbf{Medium (1064)}} & \multicolumn{2}{c}{\textbf{Hard (468)}} & \multirow{2}[3]{*}{\textbf{Pass (2756)}} & \multirow{2}[3]{*}{\textbf{Fail (1310)}} & \multirow{2}[3]{*}{\textbf{Sum}} \\
\cmidrule{2-7}          & \textbf{P} & \textbf{J} & \textbf{P} & \textbf{J} & \textbf{P} & \textbf{J} &       &       &  \\
    \hline
    \textbf{Compilation and Runtime Error } & 7 (1\%) & 8 (2\%) & 37 (3\%) & 32 (3\%) & 46 (10\%) & 47 (10\%) & 0 (0\%) & 177 (14\%) & 177 (4\%) \\
    \textbf{Wrong Outputs} & 47 (9\%) & 60 (12\%) & 290 (27\%) & 260 (24\%) & 229 (49\%) & 196 (42\%) & 0 (0\%) & 1082 (83\%) & 1082 (27\%) \\
    \textbf{Code Style and Maintainability} & 174 (35\%) & 230 (46\%) & 431 (41\%) & 588 (55\%) & 194 (41\%) & 313 (67\%) & 1243 (45\%) & 687 (52\%) & 1930 (47\%) \\
    \textbf{Performance and Efficiency} & 1 (0\%) & 2 (0\%) & 20 (2\%) & 16 (2\%) & 6 (1\%) & 6 (1\%) & 0 (0\%) & 51 (4\%) & 51 (1\%) \\
    \bottomrule
    \end{tabular}%
    }
\end{table}

