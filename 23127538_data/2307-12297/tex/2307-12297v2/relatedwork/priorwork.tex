\noindent The estimation of temperature can be broadly divided into two parts: transforming the output of the camera to temperatures, and correcting nonuniformity in the sensor.
Determining the transformation from camera output to temperature is called \emph{thermal calibration}. Correcting the nonuniformity in the sensor is called \emph{nonuniformity correction}~(NUC).

\subsection{Thermal calibration}\label{sec:prior:thermalDrift}
    \noindent The raw output of the IR camera is dependent on the object temperature, and the output values themselves are given in gray levels. For example, the gray levels dynamic range in the \taucamera\ is $14_{bit}$. 
The classic approach is to calibrate the camera for different ambient temperatures~\cite{Schulz1995}. 

A large dataset that has pairs of object and ambient temperatures must be collected for calibration. The gain and offset are calculated from the data per-pixel to attune for the spatially-variant nonuniformity. Thus, the calibration process usually requires considerable time and resources.

Schulz et al.~\cite{Schulz1995} used a single-point correction. Meaning that a single ambient temperature is used, a constant gain is assumed and only the offset is found. 
Riou et al.~\cite{Riou2004} suggested a two-point correction that requires two ambient temperatures, but solved for both gain and offset, and it is widely used across industrial IR cameras today.
Both methods used a linear regression to extract the gain and offset coefficients.
Nugent et al.~\cite{Nugent2013} modeled the gain and offset as polynomial in the temperature of the object and used least-squares to extract the coefficients.
Contemporary works adds prior knowledge into the calibration process. 
Liang et al.~\cite{Liang2017} found the gain and offset for a given temperature and interpolated the results for other ambient temperatures, Chang and Li~\cite{Chang2019} incorporated the integration time of each frame as prior knowledge to the calibration.

The calibration data must be collected for each camera separately, because each camera is slightly different due to the manufacturing process. This requires scientific-grade equipment, making the calibration process infeasible for most users.

\subsection{Nonuniformity correction}\label{sec:prior:nuc}
    % Figure environment removed
\noindent As stated in \cref{sec:intro}, the frames of the \ir camera suffer from spatially variant nonuniformity. The nonuniformity can be corrected for a single frame, or by combining information from multiple frames (known as scene-based).

\subsubsection{Single Frame}\label{sec:prior:nuc:singleFrame}
A given image contains information that can be exploited for different tasks, such as low frequencies~\cite{Oz2020}, recurring patches in the image~\cite{Shocher2017} or the statistical distribution of patches in the image~\cite{Shaham2019}.
Some works used a single image to correct the nonuniformity.

Scribner et al.~\cite{Scribner91} used a neural network (NN) to find the offset and gain by alternating optimization and gradient descent.
Tendero and Gilles~\cite{Tendero12} used histogram equalization across the columns in a frame, and then applied a discrete cosine transform to denoise the frame.
Cao and Tisse~\cite{Cao2014} relied on spatial dependence between adjunct pixels to estimate both the ambient temperature and the correction.
Zhao et al.~\cite{Zhao13} solved an optimization problem, with a constraint on the directional gradients of each frame.

Recent work has applied deep learning (DL) methods for single-image NUC\@.
Jian et al.~\cite{Jian2018} learned the nonuniformity pattern from the filtered high frequencies of the frames.
He et al.~\cite{He2018} trained a convolutional neural network (CNN) that outputs a corrected image end to end (E2E).
Chang et al.~\cite{ChangDeepLearning2019} constructed a multiscale network to reconstruct a corrected frame.
Saragadam et al.~\cite{Saragadam2021} solved an optimization problem with a NN as the prior, and a physical model as the constraint.
Oz et al.~\cite{Oz2022} modeled the nonuniformity and trained a network based on the physics of the acquisition model.

Single-image methods require only a single frame so they are easier to apply, but their performance is degraded compared to scene-based methods.

\subsubsection{Scene-Based}
Scene-based studies rely on the assumption that the change in ambient temperature is slower than the frame rate, and therefore the gain and offset are constant between consecutive frames.

Harris and Chiang~\cite{Harris99} calculated shift and normalization terms per pixel and updated these terms recursively when new frames arrived.
Hardie et al.~\cite{Hardie00} registered the frames and then averaged the results per pixel.
Vera and Torres~\cite{Vera05} improved the NN suggested by Scribner et al.~\cite{Scribner91} with an adaptive learning rate and a different loss function that accounts for multiframe information.
Averbuch et al.~\cite{Averbuch2007} reformulated the NUC problem to a Kalman filter.
Zuo et al.~\cite{Zuo11} estimated per-pixel \textit{irradiance} between two frames.
Papini et al.~\cite{Papini2018} approximated the gain and offset from multiple pairs of blurred and sharp images.
The common characteristic of these studies is that an update step must be performed when new frames arrive, before the correction step. The combined update and correction steps are computationally intensive and pose a constraint on the run time of the system.

A NN-based method to simultaneously estimate the scene temperature and correct the nonuniformity using multiframe information has not yet been achieved.

The present study builds on the image-acquisition model, which describes the relationship between the observed scene and the output of the camera (\cref{sec:background:imageAcquasition}).
By leveraging redundant information across multiple frames and ambient temperature data, the study develops a kernel prediction network (KPN) that uses DL techniques to estimate the temperature of each pixel (\cref{sec:methods:net}).

The efficacy of the method is demonstrated through comparisons of real measurements obtained with an uncooled \ir camera and those from a scientific radiometric camera.
These tests illustrate the method's ability to correct for nonuniformity and estimate temperatures accurately across different cameras (\cref{sec:results:realdata}).

Our main contributions consist of: (1) exploiting the redundant information between frames to simultaneously estimate the scene temperature and correct the nonuniformity using a NN; (2) imposing the physical model of the camera as a constraint on the network to enhance the temperature-estimation accuracy; (3) incorporating the ambient temperature data as an additional input to the network to further improve the accuracy of the temperature estimation; and (4) demonstrating the advantages of using multiple frames over single-frame methods through extensive experiments on synthetic and real data.