\begin{abstract}
    \ir cameras are widely used for temperature measurements in various applications, including agriculture, medicine, and security.
    Low-cost \ir cameras have the immense potential to replace expensive radiometric cameras in these applications; however, low-cost microbolometer-based \ir cameras are prone to spatially variant nonuniformity and to drift in temperature measurements, which limit their usability in practical scenarios.
    
    To address these limitations, we propose a novel approach for simultaneous temperature estimation and nonuniformity correction (NUC) from multiple frames captured by low-cost microbolometer-based \ir cameras. We leverage the camera's physical image-acquisition model and incorporate it into a deep-learning architecture termed kernel prediction network (KPN), which enables us to combine multiple frames despite imperfect registration between them. We also propose a novel offset block that incorporates the ambient temperature into the model and enables us to estimate the offset of the camera, which is a key factor in temperature estimation.
    
    Our findings demonstrate that the number of frames has a significant impact on the accuracy of the temperature estimation and NUC\@. 
    Moreover, introduction of the offset block results in significantly improved performance compared to vanilla KPN\@. 
    The method was tested on real data collected by a low-cost \ir camera mounted on an unmanned aerial vehicle, showing only a small average error of $0.27-0.54^\circ C$ relative to costly scientific-grade radiometric cameras.
    
    Our method provides an accurate and efficient solution for simultaneous temperature estimation and NUC, which has important implications for a wide range of practical applications.
\end{abstract}
\begin{IEEEkeywords}
Deep learning (DL), fixed-pattern Noise (FPN), \ir camera, microbolometer, multiframe, nonuniformity correction (NUC),  space-variant nonuniformity, temperature estimation.
\end{IEEEkeywords}
