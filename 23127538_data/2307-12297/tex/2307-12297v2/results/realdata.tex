\newcommand{\sizeRealData}{0.382}%
% Figure environment removed%
% Figure environment removed%
We validated the effectiveness of the proposed method on real data. Two cameras, \scientificCamera\ and \taucamera, were attached to a \hyperlink{https://www.dji.com/global/matrice600}{DJI Matrice 600} UAV and both captured the same scenes in nadir view at a height of $50_m$ above the ground at a vertical speed of $10_{m/s}$.
The \scientificCamera\ is a scientific-grade radiometric camera which outputs a temperature map of the scene, whereas the \taucamera\ outputs a gray-level map corresponding to the radiation flux. An image of the setup can be found in \cref{supp_uav} of the supplementary material.
Notice that the \taucamera\ used for the experiment was not the one used to collect the calibration data in \cref{sec:method}, which further strengthens the generality and robustness of the proposed method.

The frame rate for the \taucamera\ was set to $30_{Hz}$; the resolution of the \taucamera\ was $336\times256$ pixels, the focal length was $9.8_{mm}$, and the sensor size was $4.4_{mm}$ per 256 pixels (in the direction of the flight). The ground sampling distance was $\frac{50\cdot4.4}{9.8 \cdot 256}=0.087_{m/pix}$. The drone passed $\frac{10}{30}=0.33_m$ for every frame. This means that an object moved $\frac{0.33}{0.087}=3.80_{pix}$ between consecutive frames. Thus, an object could appear in $\frac{256}{3.80}\cong67$ frames. The \scientificCamera\ field of view was much larger than that of the \taucamera\, so a frame of the \scientificCamera\ contained multiple frames of the \taucamera.

The \scientificCamera\ requires accurate ambient parameters to produce a valid temperature map.
The ambient temperature and humidity were gathered from a nearby weather station ($28.4^\circ C$ and $32\%$, respectively). The emissivity was tuned using an accurate temperature sensor placed in the scene.

Both cameras were focused to infinity. 
The flight height of $50_m$ above ground ensured that all objects where within the depth of field of both cameras.
\scientificCamera\ captured $1,192$ frames at $5_{Hz}$ and the \taucamera\ captured $7,152$ frames at $30_{Hz}$.

The frames of the \taucamera\ were divided into overlapping groups of 7 frames each. We used 7 frames due to hardware limitations.
The frames of each group were registered toward the middle frame of the group.
The registration was performed by SIFT feature-matching using the Python package Kornia V0.6.10.
The registered frame groups were the input to the network.

The output of the network was the estimated temperature map of the scene. These estimated temperature maps were registered to the \scientificCamera\ temperature maps by hand-picking correspondence points. The final registration was performed using the Python package OpenCV V4.5.1.
The GT and estimated temperature maps are presented in \cref{supp_H,supp_I,supp_A,supp_O,supp_B,supp_M} in the supplementary material.

Six results are presented in \cref{fig:results:realdata} and four more are presented in \cref{supp_realdata} in the supplementary material.
We present the difference maps between the estimated and GT temperature maps, produced by the proposed method, in each subfigure. The GT maps are in gray, and the color scale from blue to red is the magnitude of the difference, with blue denoting low and red denoting high errors. The upper-left corner of the upper image displays the MAE of the difference map as a white number.

The MAE values span $0.27-0.54^\circ C$, indicating a high accuracy of temperature estimation comparable to the \scientificCamera\ precision of ($\sim0.5^\circ C$). This was obtained without applying any thermographic corrections or NUC to the \taucamera\ data, relying solely on the raw measurements of the radiation flux as gray levels. The supplementary material provides the detailed configuration of the \taucamera.

The cumulative distribution function of the MAE between the GT temperature map and the estimated temperature map is shown in \cref{fig:results:realdata:percentage}.
The dashed red lines indicate the $0.5^\circ C$ threshold. All three examples show that more than $80\%$ of the pixels have a MAE of less than $0.5^\circ C$.
This further solidifies the effectiveness of the proposed method.