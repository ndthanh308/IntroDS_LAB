\noindent We presented a novel method for simultaneous temperature estimation and NUC in \ir imaging, based on a DL method that incorporates the physical model of the sensor. 
The method uses redundant information between multiple overlapping frames to infer the scene temperature and correct nonuniformity, without requiring any calibration or external reference. 
The method also exploits prior knowledge of the camera's ambient temperature, which is measured by a built-in sensor, to improve the accuracy and robustness of the estimation.

We evaluated the performance of the method on synthetic and real data and compared it with existing methods. 
The results showed that the method can achieve high accuracy and low error, and can handle various scenarios, such as changing ambient temperature, moving objects, and complex backgrounds.

We showed that performance improves with the number of frames, highlighting the benefits of exploiting the redundant information between frames.
The training process introduced misalignments between frames, which were handled by the method and did not affect its performance.
The method can also generalize well to different camera models and settings, and can be easily adapted to different applications. This was demonstrated by real data collected with a different camera mounted on an UAV\@. The MAE with the real UAV data was $0.27-0.54^\circ C$, which is comparable to the accuracy of scientific-grade cameras.

The method offers a simple and effective solution for improving the quality and reliability of low-cost uncooled \ir imaging and can potentially enable new applications that require accurate and consistent temperature measurements.