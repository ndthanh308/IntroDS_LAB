\IEEEPARstart{T}{emperature} is an important indicator of an object's state.
For example, the temperature of a plant is important in deducing information on its well-being~\cite{ir_importance_1, ir_importance_2}. 
Long-wave \ir (LWIR) imaging, commonly termed \ir imaging, measures the thermal radiation emitted from an object. 
To avoid noise and improve accuracy, radiometric \ir cameras employ either a cooling mechanism or a sophisticated shuttering apparatus. 
Both are expensive and energy consuming, which result in a highly expensive camera.
Although \ir imaging is a well-established technique, the high cost of \ir cameras prohibits its widespread use.
There exists an alternative approach to radiometric thermal imaging involving the use of low-cost uncooled microbolometer arrays, which can facilitate the creation of inexpensive \ir cameras with low energy requirements, but with a significant loss in accuracy.
Unlike photon-counting detector arrays, microbolometer arrays gauge alterations in electrical resistance resulting from the radiation emitted from an object~\cite{bolometer}. Each microbolometer in the array is heated by the thermal radiation to a temperature that is dependent on the scene, resulting in each microbolometer having a marginally different temperature based on the observed scene and the incident angle of the radiation. The incident radiation causes a miniscule change in the resistance of the microbolometer. The temperature of the scene is reflected by the variation in resistance of each microbolometer. The infinitesimal changes in resistance detected by each microbolometer are used to create an image that corresponds to the temperature of the observed scene.

Although microbolometer arrays are a useful tool for thermal imaging, they have significant limitations. Space-variant nonuniformity and noise from various sources affect the accuracy of these arrays. The nonuniformity drifts due to the change in ambient temperature, which causes unpredictable errors in the sensor readings.
The lack of a cold shield in the uncooled camera is a prominent cause of nonuniformity~\cite{IrFundamentals}. This self-radiation effect is attributed to the camera's housing and lens, which emit thermal radiation onto the sensor. This self-radiation varies according to the ambient temperature of the camera.

Fixed-pattern noise (FPN) is another factor that contributes to nonuniformity in microbolometer arrays. The readout circuitry of these arrays is typically line-based, like charge-coupled devices. Even minor differences between line-readers on the same array can result in significant variation between lines in the resulting image \cite{Riou2004}.
Noises in the camera increase the noise equivalent differential temperature (NEDT), which refers to the minimum detectable change in scene temperature~\cite{Riou2004}. The NEDT is a measure of the sensitivity of the camera; the higher the NEDT, the less sensitive the camera is to changes in temperature.
An image of a uniform heat source (blackbody) is shown in \cref{fig:nonuniformity}. The spatially-variant nonuniformity is demonstrated by the radial patterns in the gray levels of the left subfigure. The subfigure on the right plots the gray levels along the blue dashed line, showing the impact of nonuniformity and noise on the gray levels.

A widely used application of \ir imaging is remote sensing - the process of acquiring information about an object without making physical contact with it. The information is acquired by measuring the radiation that is reflected by, or emitted from the object. The information is then used to deduce its physical properties. Remote sensing is used in a variety of fields (e.g., agriculture, geology, and meteorology).

One common use for \ir cameras is to mount them on drones. 
This setup results in high overlap between frames (\cref{sec:results:realdata}). 
The redundant information can be used to simultaneously improve the accuracy of the temperature estimation and correct nonuniformity in the frames.
% Figure environment removed
\cref{fig:approach:redundentNonUniformity} illustrates how redundant information between frames can be beneficial. The object is affected differently by the nonuniformity in each frame, which means that the true underlying temperature of the object can be extracted.

The aims of this study are to: exploit the redundancy in the data and the physical model of the camera to develop a method of estimating scene temperatures using a low-cost microbolometer-based \ir camera, and correct for nonuniformity in the frames.