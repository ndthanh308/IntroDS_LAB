\noindent The dataset used for training consisted of $12,897$ frames, and the validation set was composed of $4,723$ frames, all of which were captured by an UAV flying at a height of $70-100_m$ above various agricultural fields in Israel. Only clear and in-focus frames were selected for the dataset manually by a human user.

The noise variance was established by analyzing the measurements taken in the environmental chamber. All frames were stacked depth-wise, and the variance of each pixel was calculated, resulting in a 2D variance map. 
The mean of the variance map was $~5$ gray levels, which was used as the noise variance in the network training. 
The influence of $\tamb$ and $\tobj$ on $\sigma^2$ was determined to be insignificant.

To prevent data leakage between the training and validation sets and evaluate the network's ability to generalize to new data, the validation sets were captured at the same locations as the training sets but on different days. This validation approach was maintained across all training schemes to ensure a fair comparison between different experiments. The same split between the training and validation datasets was maintained throughout the study.