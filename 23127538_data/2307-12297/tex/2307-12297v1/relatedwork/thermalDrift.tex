\noindent The raw output of the IR camera is dependent on the object temperature, and the output values themselves are given in gray levels. For example, the gray levels dynamic range in the \taucamera\ is $14_{bit}$. 
The classic approach is to calibrate the camera for different ambient temperatures~\cite{Schulz1995}. 

A large dataset that has pairs of object and ambient temperatures must be collected for calibration. The gain and offset are calculated from the data per-pixel to attune for the spatially-variant nonuniformity. Thus, the calibration process usually requires considerable time and resources.

Schulz et al.~\cite{Schulz1995} used a single-point correction. Meaning that a single ambient temperature is used, a constant gain is assumed and only the offset is found. 
Riou et al.~\cite{Riou2004} suggested a two-point correction that requires two ambient temperatures, but solved for both gain and offset, and it is widely used across industrial IR cameras today.
Both methods used a linear regression to extract the gain and offset coefficients.
Nugent et al.~\cite{Nugent2013} modeled the gain and offset as polynomial in the temperature of the object and used least-squares to extract the coefficients.
Contemporary works adds prior knowledge into the calibration process. 
Liang et al.~\cite{Liang2017} found the gain and offset for a given temperature and interpolated the results for other ambient temperatures, Chang and Li~\cite{Chang2019} incorporated the integration time of each frame as prior knowledge to the calibration.

The calibration data must be collected for each camera separately, because each camera is slightly different due to the manufacturing process. This requires scientific-grade equipment, making the calibration process infeasible for most users.
