\IEEEPARstart{T}{emperatures} is an important indicator for the state of an object.
For example, the temperature of a plant is important in deducing information on its well-being~\cite{ir_importance_1, ir_importance_2}. 

Long-wave infrared (LWIR) imaging is a technique that measures the thermal radiation emitted from an object, commonly known as infra-red (IR) imaging. To avoid noise and improve accuracy, radiometric IR cameras are usually cooled to $200_K$ and below. The cooling apparatus, as well as complex shuttering and control systems, increase the cost of the camera considerably.
Although IR imaging is a well-established technique, the high cost of IR cameras prohibits widespread usage.

An alternative approach to radiometric thermal imaging involves the use of low-cost uncooled microbolometer arrays, which can facilitate the creation of inexpensive IR cameras with low energy requirements. Unlike photon-counting detector arrays, microbolometer arrays gauge alterations in electrical resistance resulted from the radiation emission of an object~\cite{bolometer}. Each microbolometer in the array is heated by the thermal radiation to a temperature that is reliant on the scene, resulting in each microbolometer having a marginally different temperature based on the observed scene and the incident angle of the radiation. The incident radiation causes a miniscule change in the resistance of the microbolometer. The temperature of the scene is reflected by the variation in resistance of each microbolometer. The infinitesimal changes in resistance detected by each microbolometer are used to create an image that corresponds to the temperature of the observed scene.

Although microbolometer arrays are a useful tool for thermal imaging, they have significant limitations. Space-variant nonuniformity and noise from various sources affect the accuracy of these arrays. The nonuniformity drifts due to the change in ambient temperature, which causes unpredictable errors in the sensor readings.

The lack of a cold shield in the uncooled camera is a prominent cause of nonuniformity~\cite{IrFundamentals}. This self-radiation effect is attributable to the housing and lens of the camera, which emit thermal radiation onto the sensor. The self-radiation varies according to the ambient temperature of the camera.

Fixed-pattern noise (FPN) is an additional factor that contributes to nonuniformity in microbolometer arrays. The readout circuitry of these arrays is typically line-based, like charge coupled devices. Even minor differences between line-readers on the same array can result in significant variation between lines in the resulting image \cite{Riou2004}.

Noises in the camera has the effect of increasing noise equivalent differential temperature (NEDT), which refers to the minimum detectable change in scene temperature~\cite{Riou2004}. The NEDT is a measure of the sensitivity of the camera. The higher the NEDT, the less sensitive the camera is to changes in temperature.

An image of a uniform heat source (blackbody) is shown in \cref{fig:nonuniformity}. The spatially-variant nonuniformity is demonstrated by the radial patterns in the gray levels of the left subfigure. The subfigure on the right plots the gray levels along the blue dashed line, showing the impact of the nonuniformity and the noise on the gray levels.

A widely used application of IR imaging is remote sensing. Remote sensing is the process of acquiring information about an object without making physical contact with it. The information is acquired by measuring the reflected or emitted radiation from the object. The information is then used to deduce the physical properties of the object. Remote sensing is used in a variety of fields (e.g, agriculture, geology, and meteorology).

One common use-case for IR camera is to be mounted on drones. 
This setup results in high overlap between frames (\cref{sec:results:realdata}). 
The redundant information can be used to simultaneously improve the accuracy of the temperature estimation and correct nonuniformity in the frames.
% Figure environment removed
\cref{fig:approach:redundentNonUniformity} illustrates how redundant information between frames is beneficial. The object is affected differently by the nonuniformity at each frame, which means that the true underlying temperature of the object can be extracted.

The aim of this study is twofold: exploiting the redundancy in data and the physical model of the camera to develop a method of estimating scene temperatures using a low-cost IR camera based on microbolometers, and to correct for nonuniformity in the frames.