\begin{abstract}
    Infrared (IR) cameras are widely used for temperature measurements in various applications, including agriculture, medicine, and security.
    Low-cost IR camera have an immense potential to replace expansive radiometric cameras in these applications, however low-cost microbolometer-based IR cameras are prone to spatially-variant nonuniformity and to drift in temperature measurements, which limits their usability in practical scenarios.
    
    To address these limitations, we propose a novel approach for simultaneous temperature estimation and nonuniformity correction from multiple frames captured by low-cost microbolometer-based IR cameras. We leverage the physical image acquisition model of the camera and incorporate it into a deep learning architecture called kernel estimation networks (KPN), which enables us to combine multiple frames despite imperfect registration between them. We also propose a novel offset block that incorporates the ambient temperature into the model and enables us to estimate the offset of the camera, which is a key factor in temperature estimation.
    
    Our findings demonstrate that the number of frames has a significant impact on the accuracy of temperature estimation and nonuniformity correction. Moreover, our approach achieves a significant improvement in performance compared to vanilla KPN, thanks to the offset block. 
    The method was tested on real data collected by a low-cost IR camera mounted on a UAV, showing only a small average error of $0.27^\circ C-0.54^\circ C$ relative to costly scientific-grade radiometric cameras.
    
    Our method provides an accurate and efficient solution for simultaneous temperature estimation and nonuniformity correction, which has important implications for a wide range of practical applications.
\end{abstract}
\begin{IEEEkeywords}
Deep learning, Fixed-Pattern Noise (FPN), Multiframes, Temperature estimation, Nonuniformity correction, Infrared cameras, Microbolometer, space variant nonuniformity
\end{IEEEkeywords}
