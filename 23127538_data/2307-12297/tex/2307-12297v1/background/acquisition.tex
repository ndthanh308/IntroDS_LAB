\noindent A \emph{blackbody} is an ideal Lambertian surface that emits the maximal radiation at any given wavelength.
The spectral density of radiation emitted from a blackbody is described by Planck's law~\cite{IrFundamentals}:
\begin{equation}\label{eq:acquisition:plancksLaw}
    M_\lambda(T) = \frac{2\pi h c^2}{\lambda^5}\frac{1}{\exp(\frac{hc}{\lambda k T})-1}\quad[W\cdot sr^{-1}\cdot m^{-3}]
\end{equation} where $T$ is the temperature of the blackbody in Kelvin, $\lambda$ is the radiation wavelength, $h$ is Planck's constant, $k$ the Boltzmann constant and $c$ the speed of light.

The power emitted over the entire bandwidth is found using Stefan-Boltzmann law~\cite{IrFundamentals}:
\begin{equation}\label{eq:acquisition:boltzmann:ideal}
    M(T)=\int_0^\infty{M_\lambda(T)d\lambda}=\sigma\cdot T^4\quad[W\cdot sr^{-1}\cdot m^{-2}]
\end{equation} where $\sigma$ is the Stefan-Boltzmann constant.

The equations above holds for an ideal blackbody. Real objects can never emit the maximal radiation for a given wavelength due to physical constraints (e.g., material, viewing angle). The ratio between the ideal emission and the practical emission of an object is called \emph{emissivity}.
Thus, the Stefan-Boltzmann law for radiance power of practical objects is:
\begin{equation}\label{eq:acquisition:boltzmann:practical}
    M(T)=\sigma\cdot\epsilon\cdot T^4\quad[W\cdot sr^{-1}\cdot m^{-2}]
\end{equation} where $\epsilon$ is the emissivity.

Estimating the incident power by an object on a microbolometer is done by integrating over the physical dimensions of the system on \cref{eq:acquisition:boltzmann:practical}. The incident power on the microbolometer can be written as~\cite{IrFundamentals}:
\begin{equation}\label{eq:acquisition:incidentpower}
    \phi(T)=\gamma\cdot\sigma\cdot\epsilon\cdot T^4\quad[W]
\end{equation} where $\gamma$ is a coefficient that accounts for the dimensions of the object and field of view of the camera.

The intensities of the pixels in radiometric IR cameras (i.e.\ gray levels) are linearly proportional to the incident power on the microbolometer.
To model the intensities, we consider a small environment near a reference temperature $T_0$ and expand the Stefan-Boltzmann law in \cref{eq:acquisition:incidentpower} by Taylor series. In Kelvin, the temperature of the object can be considered a small perturbation around a reference temperature, because the reference temperature is usually hundreds of Kelvin, while $\Delta T$ is usually tens of Kelvin. The Taylor expansion of \cref{eq:acquisition:incidentpower} is:
\begin{equation}\label{eq:acquisition:affine}
    \begin{split}
        I(\tobj)&=\gamma\epsilon\sigma T^4=\gamma\epsilon\sigma(\Delta T + T_0)^4\\
        &\approx 4\gamma\epsilon\sigma T_0^3\Delta T + \gamma\epsilon\sigma T_0^4\\
        &\approx g\cdot\tobj + d
    \end{split}
\end{equation}
where $I(\tobj)$ is the gray level output of the IR camera, $g=4\gamma\epsilon\sigma T_0^3, d=\gamma\epsilon\sigma T_0^4$ are the gain and offset coefficients.
Using the relation between Kelvin and Celsius, we denote $\tobj\equiv T-273.15$ in $^\circ C$.

\cref{eq:acquisition:affine} shows that the radiation is linear in scene temperature in the small environment near $T_0$, with the term $g$ dependent on the object temperature, and the term $d$ independent of the object temperature.

The incident power $\phi(\tobj)$ in \cref{eq:acquisition:incidentpower} changes the temperature of the microbolometer by a small fraction. The change in temperature also changes the electrical resistance of the microbolometer~\cite{uncooled_thermal_imaging}. By applying a constant electrical current on the microbolometer and using Ohm-like law, a mapping between the incident power and the voltage of the microbolometer can be derived~\cite{IrFundamentals}.
In a low-cost uncooled IR camera, the resistance of the microbolometer changes with the ambient temperature.

To account for the effects of the ambient temperature on the resistance of the microbolometer, the gain and offset of the IR camera are modelled as a function of the ambient temperature~\cite{Nugent2013}:
\begin{equation}\label{eq:acquisition:frame}
    I(\tobj,\tamb)  = g(\tamb) \cdot \tobj + d(\tamb)
\end{equation}

For a given ambient temperature, the gray levels of pixel $[u,v]$ can be written as:
\begin{equation}\label{eq:acquisition:affine:pixel}
    I(\tobj)[u,v]=g[u,v]\cdot\tobj[u,v]+d[u,v]
\end{equation}
The gain and offset are two dimensional (2D), and together they model the space-variant nonuniformity.

The signal-to-noise ratio of uncooled IR cameras is often low due to noises, with the most dominant noises being $\frac{1}{f}$ and electronic (Johnson) noise~\cite[Chapter 5]{uncooled_thermal_imaging}. The $\frac{1}{f}$ noise is more dominant because the camera operates at a low frequency. $\frac{1}{f}$ noise can be modelled as Gaussian~\cite{ir_1f_gaussian} with zero mean.