%!TEX root = ../main.tex

\newcommand{\invpred}{\mathit{inv}}

\newcommand{\join}{\sqcup}
\newcommand{\bigjoin}{\bigsqcup}
\newcommand{\aseq}[2]{\left(#2\right)_{#1\in\nat}}

\newcommand{\fiter}{\mathit{iter}}

\newcommand{\fprel}{\preceq}
\newcommand{\fpreldot}{\mathop{\dot{\preceq}}}
\newcommand{\fpreleq}{\mathop{\precapprox}}
\newcommand{\ctxfprel}{\preceq_\mathit{ctx}}
\newcommand{\makectxrel}[1]{#1_\mathit{ctx}}
\newcommand{\fpclosureof}[3][\fprel]{\mathrm{closure}_{#1}^{#2}(#3)}
\newcommand{\fpcompatible}[2][\fprel]{\mathrm{compatible}_{#1}(#2)}
\newcommand{\fpdecreasing}[2][\fprel]{\mathrm{decreasing}_{#1}(#2)}
\newcommand{\fpdisjoint}[2][\fprel]{\mathrm{disjoint}_{#1}(#2)}
\newcommand{\fpthresholdof}[3][\fprel]{\mathrm{threshold}_{#1}^{#2}(#3)}

\newcommand{\cval}{\mathit{cval}}
\newcommand{\myid}{\mathit{id}}
\newcommand{\myidof}[1]{\myid_{#1}}

% \newcommand{\sends}[2]{\mathsf{sends}_{#1\leadsto#2}}
\newcommand{\explain}[1]{\text{\small{(\;#1\;)}}\quad}
\newcommand{\explains}[2]{\stackrel{\text{\small{#1}}}{#2}}

\newcommand{\Bekic}{Beki\'c's Lemma\xspace}

%%%%%%
\newcommand{\simplerel}{\Subset} % \ll
\newcommand{\complexrel}{\mathrel{\Subset\!\!\!\!-}} % \ll


\newcommand{\absreach}{\mathcal{R}}
\newcommand{\keyspace}{\mathbb{K}}
\newcommand{\contents}{\mathit{C}}
% \newcommand{\reachof}[1]{\mathsf{R}(#1)}
\newcommand{\ctxupclosed}[1]{\Psi(#1)}
\newcommand{\ctnof}[1]{\mathsf{C}(#1)}
\newcommand{\inset}{\mathsf{IS}}
\newcommand{\isof}[1]{\inset(#1)}
\newcommand{\keyset}{\mathsf{KS}}
\newcommand{\ksof}[1]{\keyset(#1)}
\newcommand{\oslof}[1]{\mathsf{OS}_\mathit{left}(#1)}
\newcommand{\osrof}[1]{\mathsf{OS}_\mathit{right}(#1)}
\newcommand{\ispof}[1]{\mathsf{\overline{IS}}(#1)}
\newcommand{\kspof}[1]{\mathsf{\overline{KS}}(#1)}
\newcommand{\loopinv}[1]{\mathsf{LoopInv}(#1)}
\newcommand{\invraw}{\mathsf{Inv}}
\newcommand{\inv}[1]{\invraw(#1)}
\newcommand{\invp}[1]{\mathsf{\overline{Inv}}(#1)}
\newcommand{\ninvraw}{\mathsf{NInv}}
\newcommand{\ninv}[1]{\mathsf{NInv}(#1)}
\newcommand{\ninvp}[1]{\mathsf{NInv}^{\mkern-2mu+\mkern-4mu}(#1)}
\newcommand{\lbof}[1]{\mathsf{LB}(#1)}
\newcommand{\ubof}[1]{\mathsf{UP}(#1)}
\newcommand{\pcof}[1]{???}
\newcommand{\rof}[1]{???}
\newcommand{\Root}{\mathit{Root}}
\newcommand{\key}{\mathit{key}}
\newcommand{\pred}{\mathit{pred}}
\newcommand{\curr}{\mathit{curr}}
\newcommand{\p}{\mathit{p}}
\newcommand{\x}{\mathit{x}}
\newcommand{\y}{\mathit{y}}
\newcommand{\z}{\mathit{z}}
\newcommand{\xkey}{\mathit{xk}}
\newcommand{\ykey}{\mathit{yk}}
\newcommand{\dup}{\mathit{c}}
\newcommand{\res}{\mathit{res}}
\newcommand{\aks}{\mathcal{K}}
\newcommand{\ais}{\mathcal{M}}
\newcommand{\aisp}{\mathcal{M'}}
\newcommand{\keyof}[1]{\mathit{key}(#1)}
\newcommand{\dataof}[1]{\keyof{#1}} % TODO: remove
\newcommand{\leftof}[1]{\mathit{left}(#1)}
\newcommand{\rightof}[1]{\mathit{right}(#1)}
\newcommand{\delof}[1]{\mathit{del}(#1)}
\newcommand{\inof}[1]{\mathit{in}(#1)}
\newcommand{\dupof}[1]{\mathit{dup}(#1)}
\newcommand{\dupvalnone}{\textsc{\sc no}}
\newcommand{\dupvalleft}{\textsc{\sc left}}
\newcommand{\dupvalright}{\textsc{\sc right}}
\newcommand{\remof}[1]{\mathit{rem}(#1)}
\newcommand{\rotrof}[1]{\mathit{rotR}(#1)}
\newcommand{\rotlof}[1]{\mathit{rotL}(#1)}
\newcommand{\lockof}[1]{\mathit{lock}(#1)}
\newcommand{\acqof}[1]{\mathsf{ACQ}(#1)}
\newcommand{\holds}[1]{\acqof{\lockof{#1}}}
\newcommand{\free}[1]{\mathsf{REL}(\lockof{#1})}

\newcommand{\pto}{\mapsto}
\newcommand{\upto}{\leadsto}
\newcommand{\INTER}[3]{\begin{aligned}[t]\set{&#1}\\\:&#2\:\\\set{&#3}\end{aligned}}
\newcommand{\data}{\mathit{val}}
\newcommand{\lchild}{\mathit{left}}
\newcommand{\rchild}{\mathit{right}}
\newcommand{\del}{\mathit{del}}
\newcommand{\rem}{\mathit{rem}}
\newcommand{\rotr}{\mathit{rotR}}
\newcommand{\rotl}{\mathit{rotL}}
\newcommand{\lock}{\mathit{lock}}
\newcommand{\is}{\mathsf{IS}}
\newcommand{\isp}{\mathsf{\overline{IS}}}
\newcommand{\lb}{\mathsf{LB}}
\newcommand{\ub}{\mathsf{UP}}

\newcommand{\bst}[1]{\mathsf{BST}(#1)}
\newcommand{\akey}{\mathit{k}}
\newcommand{\nodeof}[1]{\mathsf{N}(#1)}
\newcommand{\selof}[2]{#1\mcode{.#2}}


% ----------------------------------------------------------------------------


\newcommand{\astatepp}{\mathsf{u}}
\newcommand{\astateppp}{\mathsf{v}}
\newcommand{\arel}{\mathit{R}}
\newcommand{\arelof}[1]{\arel(#1)}
\newcommand{\arelp}{\mathit{S}}
\newcommand{\arelpof}[1]{\arelp(#1)}
\newcommand{\arelpp}{\mathit{T}}
\newcommand{\arelppof}[1]{\arelpp(#1)}
\newcommand{\acontextrelof}[2]{\arel_{#1}(#2)}
\newcommand{\abort}{\top}
% \newcommand{\footprintsof}[1]{\textit{fp}(#1)}
\newcommand{\setpreds}{\mathsf{Preds}(\setstates)}
\newcommand{\setlocacts}{\mathsf{LActs}(\setstates)}
\newcommand{\predleq}{\sqsubseteq}
\newcommand{\predjoin}{\sqcup}
\newcommand{\predmeet}{\sqcap}
\newcommand{\bigpredjoin}{\bigsqcup}
\newcommand{\bigpredmeet}{\bigsqcap}
\newcommand{\predtransleq}{\mathrel{\dot\predleq}}
\newcommand{\predtransjoin}{\mathop{\dot\sqcup}}
\newcommand{\bigpredtransjoin}{\dot\bigsqcup}

\renewcommand{\highlight}[1]{\textcolor{blue}{#1}}
\newcommand{\makeColorLogic}[1]{\textcolor{blue}{#1}}
\newcommand{\makeColorLogicDep}[1]{\textcolor{red}{#1}}

\newcommand{\hoareOf}[3]{\hoareof{#1}{#2}{#3}}
\newcommand{\choareOf}[4]{\choareof{#1}{#2}{#3}{#4}}
\newcommand{\choareHighOf}[4]{\chhoareof{#1}{#2}{#3}{#4}}

\newcommand{\aprecond}{\mathit{p}}
\newcommand{\apostcond}{\mathit{q}}

\newcommand{\setcom}{\mathtt{COM}}

\newcommand{\setpredtrans}{\mathsf{PT}(\setstates)}
\newcommand{\setcapredtrans}{\mathsf{CAPT}(\setstates)}
\newcommand{\capredtransleq}{\ddot\sqsubseteq}
\newcommand{\capredtransjoin}{\ddot\sqcup}
\newcommand{\bigcapredtransjoin}{\ddot\bigsqcup}

\newcommand{\casem}[2]{\sem{#1}_{#2}}
\newcommand{\casemof}[3]{\casem{#1}{#2}(#3)}
\newcommand{\argument}[1]{\set{\text{#1}}}
\newcommand{\icasem}[2]{\sem{#1}_{#2}^{\mathrm{ind}}}
\newcommand{\icasemof}[3]{\icasem{#1}{#2}(#3)}

\newcommand{\overapprox}{\textsf{over}}
\newcommand{\overapproxof}[3]{\overapprox(\icasemof{#1}{#2}{#3})}

\newcommand{\asetpreds}{\mathit{P}}
\newcommand{\sizeof}[1]{|#1|}


% ----------------------------------------------------------------------------



\newcommand{\internal}{\scalebox{0.9}{$\mathghost$}}
\newcommand{\cor}{\mathop{\sim}}
\newcommand{\imult}{\mathop{{\internal}}}
\newcommand{\imultdef}{\mathop{\#_{\scalebox{.7}{\internal}}}}
\newcommand{\paraleq}[1]{\mathop{\preceq_{#1}}}
\newcommand{\paraclof}[2]{#1\!\uparrow_{#2}}
\newcommand{\ileq}{\mathop{<_{\internal}}}

\newcommand{\funs}{\contfunof{\amonoid}}
\newcommand{\discup}{\uplus}
\newcommand{\setflowgraphs}{\mathsf{FG}}

\newcommand{\astatex}{\mathsf{x}}
\newcommand{\astatey}{\mathsf{y}}

\newcommand{\abssem}[1]{\sem{#1}^{\#}}
\newcommand{\abssemof}[2]{\abssem{#1}(#2)}

\newcommand{\phyabs}{\mathop {\rightarrow^{\footnotesize\sharp}}}

\newcommand{\ghostconc}[1]{[\imult #1]}
\newcommand{\ghostconcof}[2]{\ghostconc{#1}(#2)}
\newcommand{\ghostabs}[1]{\ghostconc{#1}^{\sharp}}
\newcommand{\ghostabsof}[2]{\ghostabs{#1}(#2)}

\newcommand{\absimult}{\mathop{{{\internal}}^{\sharp}}}

\newcommand{\vertin}{\rotatebox{270}{$\!\!\!\in$}}
\newcommand{\verteq}{\rotatebox{90}{$=$}}

\newcommand{\up}[1]{\mathop{[#1]}}
\newcommand{\upof}[2]{\up{#1}(#2)}
\newcommand{\absup}[1]{\mathop{[#1]^{\sharp}}}
\newcommand{\absupof}[2]{\absup{#1}(#2)}


% ----------------------------------------------------------------------------


\newcommand{\localindex}{\mathsf{L}}
\newcommand{\sharedindex}{\mathsf{G}}
\newcommand{\sharedmult}{\mathop{{\mstar}_{\sharedindex}}}
\newcommand{\localmult}{\mathop{{\mstar}_{\localindex}}}
\newcommand{\sharedemp}{\emp_{\sharedindex}}
\newcommand{\localemp}{\emp_{\localindex}}
\newcommand{\setshared}{\Sigma_{\sharedindex}}
\newcommand{\setlocal}{\Sigma_{\localindex}}
\newcommand{\alocal}{\mathsf{l}}
\newcommand{\ashared}{\mathsf{g}}
\newcommand{\setstmt}{\mathtt{ST}}
\newcommand{\aconfig}{\mathsf{cf}}
\newcommand{\apc}{\mathsf{pc}}
\newcommand{\setconfig}{\mathsf{CF}}
\newcommand{\initset}[2]{\mathsf{Init}_{#1, #2}}
\newcommand{\acceptset}[1]{\mathsf{Acc}_{#1}}
\newcommand{\reachset}[1]{\mathsf{Reach}(#1)}
\newcommand{\reachsetof}[2]{\mathsf{Reach}_{#1}(#2)}
\newcommand{\pcStepOf}[3]{#1\,\xxrightarrow{\vphantom{pt}#2}\,#3}
\newcommand{\progStepRel}{\rightarrow}
\newcommand{\subModels}{\models}
\DeclareRobustCommand{\mmodels}{\mathrel{|\mkern-2mu|}\joinrel \Relbar}
\newcommand{\semCalc}{\semcalc}
\newcommand{\interop}{\rhd}


%=============================================================================%
%=============================================================================%
% Figure environment removed

\section{Introduction}
Automatic 3D reconstruction of clothed humans using image inputs has gained increasing significance due to its potential applications in a wide array of AR/VR scenarios. High-fidelity reconstructions typically depend on sophisticated capture systems, which are developed with dense camera arrays~\cite{collet2015high,joo2015panoptic,joo2018total}, programmable light-stages~\cite{Vlasic2009, guo2019relightables}, and depth sensors~\cite{newcombe2011kinectfusion,DoubleFusion,BodyFusion,dou2016fusion4d,newcombe2015dynamicfusion}. However, stringent capture environments equipped with complex hardware pose significant challenges for consumer-level applications.


In this context, considerable research effort has been dedicated to developing methods that allow for more flexible capture configurations, such as utilizing a few RGB inputs. Among these works, learning implicit functions \cite{iccv2020PIFu, saito2020pifuhd, hong2021stereopifu} has proven effective in achieving highly detailed reconstructions by integrating the advancements of deep neural networks. These methods employ large multi-layer perceptrons (MLPs) to predict the occupancy probability or truncated signed distance function (TSDF) value of every queried 3D point based on its associated local feature, which is extracted from images. They can recover a continuous surface at arbitrary resolutions without topology restrictions.


However, in typical MLP-based implicit networks, the occupancy or TSDF value at each location is solved independently with planar image features, rendering them less capable of addressing challenging cases such as occlusions. Consequently, these methods suffer from generalization and robustness issues, particularly when tackling strong occlusions caused by large motion or multiple interacting humans. 
Some follow-up studies  \cite{zheng2021deepmulticap,zheng2021pamir,huang2020arch} utilize an extra geometric model, SMPL~\cite{Loper2015}, to improve robustness by introducing strong shape priors. 
Their success typically relies on the assumption of geometrical similarity \cite{huang2020arch} between the shape prior and target reconstruction, making them intractable for handling complex cases with loose clothes and sensitive to errors in SMPL model fitting.



%\ping{this paragraph sounds like `TSDF is better than MLP/SMPL, and we use TSDF to solve the problem'. But in Sec 3, we are telling a different story, saying `MLP needs a 3D convolutional encoder'. We need to make these two sections consistent.}\sicong{I think in this paragraph we claim that the TSDF}


%We opt for Trucated Signed Distance Funtion (TSDF) volumetric representations as they are naturally suitable for convolution operations, which have shown remarkable performance for learning hierarchical features on 2D visual perception tasks \cite{SunXLW19}. 
%Meanwhile, TSDF also describes the gradual geometry change around shape surface, which is not reflected by occupancy volume. 

We instead revisit the 3D volumetric representation and resort to 3D convolutional neural networks (CNNs) for feature learning, due to their impressive performance in feature learning and the ability to incorporate spatial context. However, volumetric methods and 3D convolution involve discretization, which might raise concerns regarding whether a discretized volume can preserve subtle geometric details as continuous representations learned in implicit functions. We investigate the relationship between volume resolution and quantization error on synthetic data by converting target mesh objects to TSDF volumes, as shown in Figure~\ref{fig:quantization_error}. We observe that the quantization errors are significantly reduced by increasing volume resolution and become nearly negligible when reaching a relatively high resolution (e.g., 512 or higher). In other words, achieving fine-detailed reconstruction is not supposed to be restricted by the use of volume representations as long as a proper volume resolution is utilized. Therefore, we present a method with high-resolution feature volumes, e.g., 256 and 512, while traditional volumetric methods \cite{varol18_bodynet,gilbert2018volumetric} are often limited to much lower resolutions, such as 32 or 128.



On the other hand, an increase in volume resolution may lead to a cubic growth of memory overhead \cite{8100085}. Reducing memory costs while guaranteeing the granularity of volumetric representations is necessary for pursuing high-quality reconstruction. Thus, we adopt a coarse-to-fine approach and cull away irrelevant voxels to build a sparse high-resolution feature volume. At the coarse level, the network computes an initial TSDF by applying a U-Net with sparse 3D CNN \cite{3DSemanticSegmentationWithSubmanifoldSparseConvNet} on the sparse feature volume, which is carved by a visual hull. Through our experiments, it turns out that more than 95\% of the volume grids are discarded by the visual hull culling, making the sparse 3D CNN efficient. At the fine level, the network focuses on a narrow band near the zero-level set of the initial TSDF and discretizes the narrow band with smaller voxels. By employing this narrow-band culling, we further shrink the sampling space, resulting in a relatively small range of grid numbers (usually 300K--500K in our experiments) even with a high volume resolution of 512. The remaining voxels in the narrow band are associated with features that fuse high-frequency information from the computed normal maps upon the low-frequency shape from the coarse level to compute the TSDF at high resolution. The final mesh is then extracted from the TSDF using the Marching-Cube algorithm ~\cite{Lorensen87marchingcubes}.
% Different from the u-net sturcture to preserve global topology context, we then apply a shallow 3dcnn to compute the final TSDF $D_{final}$ which contain more local geometry detail.




% \ping{this paragraph can be expanded. It is an important contribution and often ignored by other works. stress on the novel idea of regressing blending weights instead of colors}

In addition to geometry, high-quality mesh texture is also a crucial factor contributing to visual appearance. Directly computing a color field in 3D space, as in \cite{iccv2020PIFu}, struggles to capture high-frequency texture details, while the neural radiance field (NeRF) \cite{yu2020pixelnerf} or the DoubleField~\cite{shao2022doublefield} require expensive per-instance optimization and are often unstable for sparse input images. In contrast, we adopt an image-based rendering approach to compute a texture atlas map, which is efficient and widely supported in existing computer graphics tools. 
Specifically, we compute a blending weight at each 3D point on the mesh surface to determine its color as a weighted average of the colors at its image projections. The blending weights can be computed at a relatively coarse resolution, e.g., 512 volume resolution in our case, and leave texture details to the high-resolution images, such as 1K or 2K. Unlike previous methods that generate blurry texturing results under sparse input, our method generalizes well on both synthetic and real data with just a few input views. 
Figure~\ref{fig:teaser} shows two examples reconstructed by our method. Despite the challenging garment, pose, and occlusion, our method recovers faithful shape, normal, and texture on the right.

%with a wide variety of poses and clothing styles, and it is also adaptive to handle input image with arbitrary resolutions.
%\sicong{For this concern we claim that when the resolution of dicretized volume meets certain threshold (which is 256 in our experiment), the quantization error can be neglected.} 



In summary, the main contributions of this paper are as follows:
\begin{itemize}
\vspace{-0.1in}
  \item 
  We revisit the 3D volumetric representation and demonstrate that it can support clothed human reconstruction with equal or even better performance compared to implicit representation. 
  \item 
  We develop a memory and computation-efficient method for high-resolution volumetric reconstruction using sophisticated sparse 3D CNN, coarse-to-fine estimation, and voxel culling by visual hull and narrow bands. 
  \item 
  We introduce a novel method to compute a texture atlas map, which captures rich appearance details from high-resolution input images.
  \item 
  We achieve impressive results on standard benchmark datasets Twindom and MultiHuman, significantly reducing the point-2-surface (P2S) precision to approximately 0.2cm from just six input views, with more than $50\%$ error reduction compared to the state-of-the-art methods, including DoubleField~\cite{shao2022doublefield} and PIFuHD~\cite{saito2020pifuhd}.
\end{itemize}
\section{Motivation}
\label{sec:motivation}

IGNORE THIS FILE, WILL DO IN INTRO

%!TEX root = ../main.tex

\newcommand{\astatepp}{\mathsf{u}}
\newcommand{\astateppp}{\mathsf{v}}
\newcommand{\arel}{\mathit{R}}
\newcommand{\arelof}[1]{\arel(#1)}
\newcommand{\arelp}{\mathit{S}}
\newcommand{\arelpof}[1]{\arelp(#1)}
\newcommand{\arelpp}{\mathit{T}}
\newcommand{\arelppof}[1]{\arelpp(#1)}
\newcommand{\acontextrelof}[2]{\arel_{#1}(#2)}
\newcommand{\abort}{\top}
\newcommand{\footprintsof}[1]{\textit{fp}(#1)}
%\renewcommand{\astmt}{\mathit{S}}
\newcommand{\setpreds}{\mathsf{Preds}(\setstates)}
\newcommand{\setlocacts}{\mathsf{LActs}(\setstates)}
\newcommand{\predleq}{\sqsubseteq}
\newcommand{\predjoin}{\sqcup}
\newcommand{\predmeet}{\sqcap}
\newcommand{\bigpredjoin}{\bigsqcup}
\newcommand{\bigpredmeet}{\bigsqcap}
\newcommand{\predtransleq}{\mathrel{\dot\predleq}}
\newcommand{\predtransjoin}{\mathop{\dot\sqcup}}
\newcommand{\bigpredtransjoin}{\dot\bigsqcup}
%\renewcommand{\choiceof}[2]{#1\ \locactjoin\ #2}

\renewcommand{\highlight}[1]{\textcolor{blue}{#1}}
\newcommand{\makeColorLogic}[1]{\textcolor{blue}{#1}}
\newcommand{\makeColorLogicDep}[1]{\textcolor{red}{#1}}

\newcommand{\hoareOf}[3]{\hoareof{#1}{#2}{#3}}
\newcommand{\choareOf}[4]{\choareof{#1}{#2}{#3}{#4}}
\newcommand{\choareHighOf}[4]{\chhoareof{#1}{#2}{#3}{#4}}



\newcommand{\aprecond}{\mathit{p}}
\newcommand{\apostcond}{\mathit{q}}


\newcommand{\setcom}{\mathtt{COM}}

\newcommand{\setpredtrans}{\mathsf{PT}(\setstates)}
\newcommand{\setcapredtrans}{\mathsf{CAPT}(\setstates)}
\newcommand{\capredtransleq}{\ddot\sqsubseteq}
\newcommand{\capredtransjoin}{\ddot\sqcup}
\newcommand{\bigcapredtransjoin}{\ddot\bigsqcup}

\newcommand{\casem}[2]{\sem{#1}_{#2}}
\newcommand{\casemof}[3]{\casem{#1}{#2}(#3)}
\newcommand{\argument}[1]{\set{\text{#1}}}
\newcommand{\icasem}[2]{\sem{#1}_{#2}^{\mathit{i}}}
\newcommand{\icasemof}[3]{\icasem{#1}{#2}(#3)}

\newcommand{\overapprox}{\textsf{over}}
\newcommand{\overapproxof}[3]{\overapprox(\icasemof{#1}{#2}{#3})}

\newcommand{\asetpreds}{\mathit{P}}

\newcommand{\iris}{\textsf{IRIS}}

\newcommand{\sizeof}[1]{|#1|}

\section{Semantics}
\label{sec:semantics}
We briefly recall the setup of abstract separation logic~\cite{DBLP:conf/lics/CalcagnoOY07} which we adapt slightly as a basis for our formal development.

A separation algebra $(\setstates, \mstar, \emp)$ is a commutative monoid in which the multiplication $\mstar$ is only partially defined, it is cancellative, and we have a set of units $\emp$. 
By cancellativity, we mean that if $\astate_1\mstar\astatep$ and $\astate_2\mstar\astatep$ are both defined and $\astate_1\mstar\astatep=\astate_2\mstar\astatep$, then $\astate_1=\astate_2$ follows. 
For every state $\astate\in\setstates$, we require that there is a unit $\munit\in\emp$ with $\astate\mstar\munit=\astate$. 
Moreover, for every pair of units $\munit\neq \munit'$ in $\emp$, we expect that the multiplication is undefined. 
We use $\astate\statemultdef\astatep$ to denote definedness of the multiplication. 


Predicates in the set $\setpreds=\powerset{\setstates}\cup\set{\abort}$ are sets of states or a dedicated symbol~$\abort$ that indicates a failure of a computation. 
We extend the multiplication to predicates, then called separating conjunction, by defining $\apred\mstar \apredp=\setcond{\astate\mstar\astatep}{\astate\in\apred\wedge \astatep\in\apredp\wedge \astate\statemultdef\astatep}$ for $\apred, \apredp\subseteq\setstates$ and $\apred\mstar\abort=\abort\mstar\apred=\abort$ for $\apred\in\setpreds$. 
We endow predicates with an ordering that coincides with inclusion on sets of states and has $\abort$ as the top element: $\apred\predleq\apredp$ if $\apred, \apredp\subseteq\setstates$ and~$\apred\subseteq\apredp$, and $\apred\predleq\abort$ for all $\apred\in\setpreds$. 
Then $(\setpreds, \predleq)$ is a complete lattice and we use~$\bigpredjoin\asetpreds$ to denote the least upper bound, or simply join, of a set of predicates $\asetpreds\subseteq\setpreds$. 

We say $\apred \subseteq \setstates$ is \emph{precise} if it identifies unique substates: for every $\astate \in \setstates$ there exists at most one $\astatep \in \apred$ such that $\astate \in \{\astatep\} \mstar \setstates$.

We define our programming language parametric in a set of commands $\setcom$. 
The set is expected to come with a semantics 
\begin{align*}
\sem{-}: \setcom\rightarrow \setpredtrans
\end{align*}
that assigns to each $\acom\in\setcom$ a predicate transformer $\sem{\acom}\in\setpredtrans$. 
The predicate transformers used in separation logic are functions $\sem{\acom}:\setpreds\rightarrow\setpreds$ that satisfy $\semof{\acom}{\abort}=\abort$ and $\semof{\acom}{\bigpredjoin\asetpreds}=\bigpredjoin \semof{\acom}{\asetpreds}=\bigpredjoin\setcond{\semof{\acom}{\apred}}{\apred\in\asetpreds}$ for all $\asetpreds\subseteq\setpreds$. 
They are strict in $\abort$ and distribute over arbitrary joins.  
With a pointwise lifting of the ordering on predicates, predicate transformers form a complete lattice $(\setpredtrans, \predtransleq)$ as well. 

We consider sequential while-programs over $\setcom$ of the form
\begin{align*}
\astmt\quad \defebnf\quad \acom\bnf\choiceof{\astmt_1}{\astmt_2}\bnf\seqof{\astmt_1}{\astmt_2}\bnf{\loopof{\astmt}} \enspace.
\end{align*}
Programs also have a semantics in terms of predicate transformers that is derived from the semantics of commands. 
The non-deterministic choice is the join, $\sem{\choiceof{\astmt_1}{\astmt_2}}=\sem{\astmt_1}\predtransjoin\sem{\astmt_2}$, the composition is function composition, $\sem{\seqof{\astmt_1}{\astmt_2}}=\sem{\astmt_2}\circ\sem{\astmt_1}$, and the semantics of iteration is $\sem{\loopof{\astmt}}=\bigpredtransjoin_{i\in\nat}\sem{\astmt^i}$ with $\sem{\astmt^0}$ being the identity and $\sem{\astmt^{i+1}}=\sem{\seqof{\astmt}{\astmt^i}}$.
The set of predicate transformers is closed under these constructions, and so $\sem{\astmt}\in\setpredtrans$. 

We specify program correctness with Hoare triples of the form $\hoareof{\apred}{\astmt}{\apredp}$. 
The triple is valid, denoted by $\models \hoareof{\apred}{\astmt}{\apredp}$, if $\semof{\astmt}{\apred}\predleq\apredp$. 
We reason about validity using the proof rules in \cref{Figure:PL}. Ignore the colorful parts for the moment, in particular the rules \ruleref{frame} and \ruleref{context}. We refer to the remainder as the rules of Hoare logic.  
We say that a rule is sound, if validity of the premise entails validity of the conclusion. 
% ----------------------------------------------------------------------------------
% ----------------------------------------------------------------------------------
% ----------------------------------------------------------------------------------
% ----------------------------------------------------------------------------------
\begin{lemma}\label{Lemma:SoundnessHoare}
The rules of Hoare logic are sound.
\end{lemma}
% ----------------------------------------------------------------------------------
% ----------------------------------------------------------------------------------
% ----------------------------------------------------------------------------------
% ----------------------------------------------------------------------------------

We refer to the rules in \cref{Figure:PL} without the blue annotations as the separation logic (SL) induced by $\sem{-}$. For soundness of the frame rule, it is well-known that the predicate transformers $\sem{\acom}$ need to satisfy an extra property called \emph{locality}: for all $\apred, \apredp\in\setpreds$ we need 
\begin{align}
\semof{\acom}{\apred\mstar\apredp}\predleq\semof{\acom}{\apred}\mstar\apredp.\tag{Locality}\label{Equation:Locality}
\end{align} 
% ----------------------------------------------------------------------------------
% ----------------------------------------------------------------------------------
% ----------------------------------------------------------------------------------
% ----------------------------------------------------------------------------------
Soundness of the frame rule is an immediate consequence of the fact that all constructions used to define the program semantics preserve \eqref{Equation:Locality}. 
% ----------------------------------------------------------------------------------
% ----------------------------------------------------------------------------------
% ----------------------------------------------------------------------------------
% ----------------------------------------------------------------------------------
\begin{lemma}[Soundness of \ruleref{frame}]\label{Lemma:SoundnessFrame}
Assume $\sem{\acom}$ satisfies \eqref{Equation:Locality} for all $\acom\in\setcom$. 
Then $\models\hoareof{\apred}{\astmt}{\apredp}$ implies $\models\hoareof{\apred\mstar\apredppp}{\astmt}{\apredp\mstar\apredppp}$. 
\end{lemma}

A state $\astate$ is a footprint of $\sem{\acom}$ if $\semof{\acom}{\set{\astate}}\neq \abort$, the predicate transformer does not abort on the state.  
We use $\footprintsof{\sem{\acom}}$ for the set of all footprints.   
Due to locality, the set of footprints is closed under composition: $\astate\in \footprintsof{\sem{\acom}}$ and $\astate\statemultdef\astatep$ implies $\astate\mstar\astatep\in \footprintsof{\sem{\acom}}$.  

%!TEX root = ../main.tex

\newcommand{\semcalcticol}{\vdash}
% Figure environment removed 



%%% Local Variables:
%%% mode: latex
%%% TeX-master: "../main"
%%% End:

%!TEX root = ../main.tex

\section{Context-Aware Reasoning for Smaller Footprints}\label{Section:CAReasoning}

The frame rule is key to local reasoning: it allows one to focus all attention only on a smallest footprint $\apred$ of the computation $\astmt$ and current state at hand, obtaining for free that the remainder of the state, captured by the \emph{frame} $d$, is preserved by $\astmt$. We are concerned with situations where the smallest footprint remains inherently large, thwarting any attempt at local reasoning.

What causes large footprints is the locality requirement for commands. 
If we cannot guarantee $\semof{\acom}{\astate\mstar\astatep}\predleq\semof{\acom}{\astate}\mstar\set{\astatep}$ for all states~$\astatep$, then we have to define $\semof{\acom}{\astate}=\abort$. That is, $\acom$ aborts on $\astate$ and any attempt at reasoning locally about the effect of $\acom$ on $\astate$ will fail.
%
The locality requirement, in turn, is a consequence of the fact that the frame rule is meant to hold for all possible frames. 
%
It says that, no matter which frame $\apredppp$ is added to the proof, the program has to leave it unchanged. 
%
In short, since the frame rule is context-agnostic, we need locality, and due to locality programs that affect a large part of the state inherently have large footprints.

This work starts from the idea of introducing a context-aware variant of the frame rule that justifies smaller footprints when reasoning about programs whose effect on the frame is benign. 
%
The rationale is that if the predicate $\apredppp$ to be added by framing is known, then we can relax the locality requirement, redefine the semantics in a way that aborts less often, and hence enable more local reasoning.
%
We develop this idea in a conservative extension of separation logic.

\subsection{Context-Aware Separation Logic}
We propose \emph{context-aware separation logic (CASL)} in which correctness statements $\choareof{\apredpp}{\apred}{\astmt}{\apredp}$ are Hoare triples enriched by a so-called \emph{context} $\acontext$.
%
The context is a predicate that is meant to be framed to the Hoare triple $\hoareof{\apred}{\astmt}{\apredp}$. This intuition is captured by the rule \ruleref{context} in \cref{Figure:PL}. 

The benefit of knowing the predicate that should be framed is that we can relax the locality requirement on the semantics of commands relative to that context.

A context-aware semantics for $\setcom$ is a function that assigns to each $\acom\in\setcom$ a context-aware predicate transformer $\casem{\acom}{\bullet}:\setcapredtrans=\setpreds\rightarrow\setpredtrans$. 
That is, a context-aware predicate transformer expects a context $\acontext$ as input and returns a suitable predicate transformer $\casem{\acom}{\acontext}$.  

The soundness of rule \ruleref{context} relies on the requirement that the context-aware semantics is compatible for the different choices of the context. 
\begin{definition}
Let $\acom \in \setcom$ and $\apredpp \in \setpreds$. We say that $\casem{\acom}{\apredpp}$ satisfies \emph{mediation} if
\begin{align}
\forall \apred, \apredppp\in\setpreds. \quad
  \casemof{\acom}{\apredpp}{\apred\mstar\apredppp}\predleq\casemof{\acom}{\apredpp\mstar\apredppp}{\apred}\mstar\apredppp\tag{Mediation}.\label{Equation:Mediation} 
\end{align}
\end{definition}
The mediation condition plays a similar role for \ruleref{context} as locality does for the frame rule.

% --------------------------------------------------------------------------------------------
% --------------------------------------------------------------------------------------------
% --------------------------------------------------------------------------------------------
% --------------------------------------------------------------------------------------------

% --------------------------------------------------------------------------------------------
% --------------------------------------------------------------------------------------------
% --------------------------------------------------------------------------------------------
% --------------------------------------------------------------------------------------------
We lift the context-aware semantics from commands to programs in the canonical way. Mediation is preserved by this lifting.
% --------------------------------------------------------------------------------------------
% --------------------------------------------------------------------------------------------
% --------------------------------------------------------------------------------------------
% --------------------------------------------------------------------------------------------
\begin{lemma}\label{Lemma:Mediation}
  \label{lem:lifting-mediation}
Let $\apredpp \in \setpreds$. If $\casem{\acom}{\apredpp}$ satisfies \eqref{Equation:Mediation} for all $\acom\in\setcom$, then so does
$\casem{\astmt}{\apredpp}$.   
\end{lemma}
% --------------------------------------------------------------------------------------------
% --------------------------------------------------------------------------------------------
% --------------------------------------------------------------------------------------------
% --------------------------------------------------------------------------------------------

We say that a correctness statement $\choareof{\acontext}{\apred}{\astmt}{\apredp}$ in context-aware separation logic is valid, denoted by $\models\choareof{\acontext}{\apred}{\astmt}{\apredp}$, 
if $\casemof{\astmt}{\acontext}{\apred}\predleq\apredp$. 
We write $\vdash\choareof{\acontext}{\apred}{\astmt}{\apredp}$ if a correctness statement can be derived using the proof rules in \cref{Figure:PL}. 
% ---------------------------------------------------------------------------
% ---------------------------------------------------------------------------
% ---------------------------------------------------------------------------
% ---------------------------------------------------------------------------
\begin{theorem}[Soundness of context-aware reasoning]
  \label{thm:casl-soundness}
Assume that $\casem{\acom}{\emp}$ satisfies \eqref{Equation:Locality} and $\casem{\acom}{\acontext}$ \eqref{Equation:Mediation} for all $\acom\in\setcom$, $\apredpp\in\setpreds$. 
Then $\vdash\choareof{\acontext}{\apred}{\astmt}{\apredp}$ implies $\models \choareof{\acontext}{\apred}{\astmt}{\apredp}$.
\end{theorem}
% ---------------------------------------------------------------------------
% ---------------------------------------------------------------------------
% ---------------------------------------------------------------------------
% ---------------------------------------------------------------------------
The theorem is an immediate consequence of soundness of the single rules. 
For the rules of Hoare logic and the frame rule, we apply \cref{Lemma:SoundnessHoare,Lemma:SoundnessFrame}.
For the new context-aware framing, soundness is an immediate consequence of \eqref{Equation:Mediation}. 
% ---------------------------------------------------------------------------
% ---------------------------------------------------------------------------
% ---------------------------------------------------------------------------
% ---------------------------------------------------------------------------
\begin{lemma}[Soundness of~\ruleref{context}]
  \label{lem:soundness-context}
Assume $\casem{\acom}{\bullet}$ satisfies \eqref{Equation:Mediation} for all $\acom\in\setcom$. 
Then $\models \choareOf{\acontext\mstar \apredppp}{\apred}{\astmt}{\apredp}$ implies $\models \choareOf{\acontext}{\apred\mstar \apredppp}{\astmt}{\apredp\mstar \apredppp}$.
\end{lemma}

\begin{proof}
We have $\casemof{\astmt}{\apredpp}{\apred\mstar\apredppp}\predleq\casemof{\astmt}{\apredpp\mstar\apredppp}{\apred}\mstar\apredppp\predleq\apredp\mstar\apredppp$. 
The former inequality is by \eqref{Equation:Mediation}, which holds due to \cref{Lemma:Mediation}.
The latter inequality is $\models \choareOf{\acontext\mstar \apredppp}{\apred}{\astmt}{\apredp}$.
\end{proof}
% ----------------------------------------------------------------------------------
% ----------------------------------------------------------------------------------
% ----------------------------------------------------------------------------------
% ----------------------------------------------------------------------------------
% ----------------------------------------------------------------------------------
% ----------------------------------------------------------------------------------
% ----------------------------------------------------------------------------------
% ----------------------------------------------------------------------------------
We note that the proof of \Cref{thm:casl-soundness} only relies on $\casem{-}{\apredpp'}$ to satisfy mediation for the contexts $\apredpp'$ occurring in the applications of rule \ruleref{context} used to derive $\vdash\choareof{\apredpp}{\apred}{\astmt}{\apredp}$. We demand the stricter requirement to avoid a side condition in the rule. However, \eqref{Equation:Mediation} can be weakened so that it is only required to hold for the contexts that are of interests for a particular proof.

\paragraph{Conservative extensions.}
We study CASLs that conservatively extend separation logics. That is, the CASL obtained from a constructed context-aware semantics should be both sound and complete relative to the separation logic induced by a given local semantics:
\begin{align*}
\forall \astmt, \apred, \apredp, \acontext. \quad & & \models \choareOf{\acontext}{\apred}{\astmt}{\apredp} \implies & \models \hoareOf{\apred \mstar \acontext}{\astmt}{\apredp \mstar \acontext} & \tag{Relative Soundness}\\
\forall \astmt, \apred, \apredp. \quad & & \vdash \hoareOf{\apred}{\astmt}{\apredp} \implies & \; \exists \acontext. \; \vdash \choareOf{\acontext}{\apred}{\astmt}{\apredp} & \tag{Relative Completeness}
\end{align*}

The canonical way to obtain a conservative extension is to let $\casem{-}{\emp}$ and $\sem{-}$ coincide.

\begin{lemma}
  \label{lem:conservative-extension}
  If $\casem{\acom}{\acontext}$ satisfies \eqref{Equation:Mediation} for all $\acom$ and $\acontext$, and $\casem{-}{\emp} = \sem{-}$, then the CASL induced by $\casem{-}{\bullet}$ conservatively extends the SL induced by $\sem{-}$.
\end{lemma}

\begin{proof}
  For relative soundness, note that $\semof{\astmt}{\apred \mstar \acontext} = \casemof{\astmt}{\emp}{\apred \mstar \acontext} \predleq \casemof{\astmt}{\acontext}{\apred}\mstar\acontext \predleq \apredp \mstar \acontext$. The first equality and the first inequality follow from the assumptions. The second inequality is the premise of relative soundness.

  Relative completeness immediately follows by a rule induction over the SL derivation that constructs a CASL derivation mimicking the SL derivation one-to-one with context $\acontext = \emp$.
\end{proof}

\paragraph{Framing under context.}
Observe that the \ruleref{frame} rule of CASL is restricted to
context-aware Hoare triples with an empty context.  This is a
deliberate design choice as allowing framing under non-empty contexts
imposes additional constraints on the considered predicates, the
underlying separation algebra, or both. For a more in-depth discussion
of a context-aware separation logic that permits a more liberal frame
rule, we direct the interested reader to \cref{app:locality}. That said,
we do not consider the restriction imposed on when to apply framing to be of
practical concern as one can always frame first before doing any
context-aware reasoning.

\subsection{From Separation Logic to Context-aware Separation Logic}
\label{sec:induced-casem}

We next present a general construction of a context-aware separation logic that conservatively extends a given separation logic. 
The question we will answer is how to make use of the fact that the predicate to be added by context-aware framing is known in order to reduce the footprint size. 
The guiding theme for the construction is this. 
Assume $\semof{\acom}{\apred}=\abort$ but $\semof{\acom}{\apred\mstar\apredpp}\predleq\apredp\mstar\apredpp\neq \abort$. 
The assumption says that some states in $\apred$ fail to be footprint, because executing $\acom$ on them also has an effect on neighboring states from~$\apredpp$. 
Moreover, these neighboring states do not change arbitrarily but again belong to $\apredpp$.
Formulated differently, although the states in $\apredpp$ are modified by the command, the predicate $\apredpp$ is invariant under these changes. 
This is the setting where context-aware semantics shows its strength.
Although $\semof{\acom}{\apred}$ has to abort as it cannot satisfy \eqref{Equation:Locality}, there is no need to let $\casemof{\acom}{\apredpp}{\apred}$ abort with \eqref{Equation:Mediation}. 
It can play the role of $\apredp$ above, in the sense that it  satisfies 
$\semof{\acom}{\apred\mstar\apredpp}\predleq\casemof{\acom}{\apredpp}{\apred}\mstar\apredpp$.
The consequence of the fact that $\casem{\acom}{\apredpp}$ does not abort is that the footprint is smaller than the footprint of $\sem{\acom}$. 

The key problem is thus to come up with a general definition of $\casem{\acom}{\apredpp}$ that captures the idea that $\casemof{\acom}{\apredpp}{\apred}$ should behave like $\apredp$ above. 
Our solution uses the septraction operator $\apred\septract\apredp$, which is the analogue of subtraction for separating conjunction. 
When $\apred$ and $\apredp$ are sets of states, then  $\apred\septract\apredp=\setcond{\astate\in\setstates}{\exists \astatep\in\apred.\ \astate\mstar\astatep\in\apredp}$, the operator thus takes states in $\apredp$ and removes the part that belongs to $\apred$. 
If $\apred=\abort$ or $\apredp=\abort$, then $\apred\septract\apredp=\abort$. 
Septraction is monotonic in both arguments.
It is also worth noting that septraction is deterministic in the following sense: if $\apred$ and $\apredp$ are singleton sets, then $\sizeof{\apred\septract\apredp}\leq 1$ due to cancellativity.  


% --------------------------------------------------------------------------------------
% --------------------------------------------------------------------------------------
% --------------------------------------------------------------------------------------
% --------------------------------------------------------------------------------------
\begin{definition}[Induced context-aware semantics]
Let $\sem{\acom}\in \setpredtrans$ be a predicate transformer.
We define the \emph{induced context-aware predicate transformer} $\icasem{\acom}{\bullet}$ by
\begin{align*}
\icasemof{\acom}{\apredpp}{\apred}\ =\ 
\begin{cases}
\apredpp\septract\semof{\acom}{\apred\mstar\apredpp} &\qquad \text{if \emph{over-aproximation} holds}, \\
\abort &\qquad\text{otherwise}. 
\end{cases}
\end{align*}
Over-approximation is the condition $\apredp\predleq (\apredpp\septract\apredp)\mstar\apredpp$ with $\apredp=\semof{\acom}{\apred\mstar\apredpp}$. 
By defining $\icasem{-}{\bullet}=\lambda\acom.\icasem{\acom}{\bullet}$, we turn a semantics $\sem{-}$ for $\setcom$ into a context-aware semantics. 
\end{definition}
% --------------------------------------------------------------------------------------
% --------------------------------------------------------------------------------------
% --------------------------------------------------------------------------------------
% --------------------------------------------------------------------------------------
The definition should be understood as follows: $\apredpp\septract\semof{\acom}{\apred\mstar\apredpp}$ is the modification that $\semof{\acom}{\apred\mstar\apredpp}$ applies to the $\apred$ part. 
The side condition of over-approximation makes sure we do not lose states when moving from the original to the induced semantics. 
To see that this is essential for relative soundness, consider a state $\astate$ that $\acom$ modifies to $\astatep$, and assume there is no $\astatepp$ so that $\astatep=\astate\mstar\astatepp$. 
Define $\apredpp=\set{\astate}$ and let $\apred=\emp$.  
If we omitted the side condition and just used $\apredpp\septract\semof{\acom}{\apred\mstar\apredpp}$ as the induced semantics, we would obtain $\set{\astate}\septract\set{\astatep}=\emptyset$. Then $\astatep\notin\emptyset\mstar\apredpp$, which is unsound.  
It may also be justified that all states live in the context, but then we see $\emp$ instead of $\emptyset$ as the post. 
Consider the same setting except that $\acom$ leaves $\astate$ unchanged. 
Over-approximation is readily checked and thus $\icasemof{\acom}{\apredpp}{\apred}=\set{\astate}\septract\set{\astate}\predleq\emp$. 
This is correct, as $\astate\in \emp\mstar\apredpp$.  

Our main result in this section is that the induced semantics satisfies the conditions in \cref{lem:conservative-extension}, and hence yields a conservative extension of separation logic.
% --------------------------------------------------------------------------------------
% --------------------------------------------------------------------------------------
% --------------------------------------------------------------------------------------
% --------------------------------------------------------------------------------------
\begin{proposition}
$\icasem{-}{\bullet}$ satisfies the conditions in \cref{lem:conservative-extension}.
\end{proposition}
% --------------------------------------------------------------------------------------
% --------------------------------------------------------------------------------------
% --------------------------------------------------------------------------------------
% --------------------------------------------------------------------------------------
We split the proof into several lemmas. 
Details missing here can be found in \cref{sec:pl-proofs}. 
The basic condition on a conservative extension is readily checked.
% --------------------------------------------------------------------------------------
% --------------------------------------------------------------------------------------
% --------------------------------------------------------------------------------------
% --------------------------------------------------------------------------------------
\begin{lemma}
$\sem{\acom}=\icasem{\acom}{\emp}$ for all $\acom\in\setcom$. 
\end{lemma}
% --------------------------------------------------------------------------------------
% --------------------------------------------------------------------------------------
% --------------------------------------------------------------------------------------
% --------------------------------------------------------------------------------------
For $\apredpp\neq\emp$, it requires a bit of work to show that the $\icasem{\acom}{\apredpp}$ are predicate transformers satisfying \eqref{Equation:Mediation}. 
The first step is to understand over-approximation better. 
Here is a characterization. 
% ---------------------------------------------------------------------------------
% ---------------------------------------------------------------------------------
% ---------------------------------------------------------------------------------
% ---------------------------------------------------------------------------------
% ---------------------------------------------------------------------------------
% ---------------------------------------------------------------------------------
% ---------------------------------------------------------------------------------
% ---------------------------------------------------------------------------------
\begin{lemma}
  \label{Lemma:OverDet}
  Let $\apredp \subseteq \Sigma$.
  We have $\apredp \predleq (\apredpp\septract \apredp) \mstar \apredpp$ iff for all $\astate \in \apredp$, $(\setstates\septract\astate)\predmeet\apredpp\neq\emptyset$.
\end{lemma}
% ---------------------------------------------------------------------------------
% ---------------------------------------------------------------------------------
% ---------------------------------------------------------------------------------
% ---------------------------------------------------------------------------------
\begin{proof}
  If $\apredpp = \top$ then the equivalence holds trivially. Thus, assume $\apredpp \neq \top$.
  \begin{asparaitem}
    \item[(``$\Rightarrow$'')]
      % $\Rightarrow$
      Assume $\astate \in \apredp$ and hence $\astate \in (\apredpp\septract\apredp)\mstar\apredpp$. 
      This means there is $\astatepp \in\apredpp$ and $\astatep$ with $\astate = \astatep \mstar \astatepp$. Therefore, we also have $\astatepp \in (\setstates \septract \astate)$.
    \item[(``$\Leftarrow$'')]
      % $\Leftarrow$
      Assume $\astatepp\in (\setstates\septract\astate)\cap\apredpp$ for some $\astate \in \apredp$. 
      Then there is a state $\astatep\in\setstates$ so that $\astatep\mstar\astatepp=\astate$.
      Hence, $\astatep=\astatepp\septract\astate \predleq \apredpp\septract\apredp$.
      Moreover, since $\astatepp\in\apredpp$, we get $\astate = \astatep\mstar\astatepp\predleq (\apredpp\septract\apredp)\mstar\apredpp$.
      \qedhere
  \end{asparaitem}
\end{proof}
% -----------------------------------------------------------
% -----------------------------------------------------------
% -----------------------------------------------------------
% -----------------------------------------------------------
The characterization immediately yields that over-approximation is distributive as follows.
% -----------------------------------------------------------
% -----------------------------------------------------------
% -----------------------------------------------------------
% -----------------------------------------------------------
\begin{lemma}\label{Lemma:OverDist}
Let $\bigpredjoin\asetpreds=\apredp\subseteq\setstates$.  
Then $\apredp\predleq(\apredpp\septract\apredp)\mstar\apredpp$ iff for all $\apred\in\asetpreds$ we have $\apred\predleq(\apredpp\septract\apred)\mstar\apredpp$. 
\end{lemma}
% -----------------------------------------------------------
% -----------------------------------------------------------
% -----------------------------------------------------------
% -----------------------------------------------------------
We will also need that over-approximation behaves well with respect to mediation. 
% -----------------------------------------------------------
% -----------------------------------------------------------
% -----------------------------------------------------------
% -----------------------------------------------------------
\begin{lemma}\label{Lemma:OverMediation}
$\apredp\predleq((\apredpp\mstar\apredppp)\septract\apredp)\mstar\apredpp\mstar\apredppp$ implies $\apredp\predleq(\apredpp\septract\apredp)\mstar\apredpp$. 
\end{lemma}
% -----------------------------------------------------------
% -----------------------------------------------------------
% -----------------------------------------------------------
% -----------------------------------------------------------
\begin{proof}
The non-trivial case is $\apredp\neq\abort$. 
Consider a state $\astate\in\apredp$. 
By the premise, there are states $\astate_1\in (\apredpp\mstar\apredppp)\septract\apredp$, $\astate_2\in\apredpp$, and $\astate_3\in\apredppp$ so that $\astate=\astate_1\mstar\astate_2\mstar\astate_3$. 
Then $\astate_1\mstar\astate_3\in \apredpp\septract\apredp$, since we can add $\astate_2\in\apredpp$ and arrive at $\astate\in\apredp$. 
Hence, $\astate= (\astate_1\mstar\astate_3)\mstar\astate_2\in (\apredpp\septract\apredp)\mstar\apredpp$ as desired.
\end{proof}
% -----------------------------------------------------------
% -----------------------------------------------------------
% -----------------------------------------------------------
% -----------------------------------------------------------
\begin{lemma}[Strictness]\label{Lemma:Strictness}
$\icasemof{\acom}{\apredpp}{\abort} = \abort$.
\end{lemma} 
% -----------------------------------------------------------
% -----------------------------------------------------------
% -----------------------------------------------------------
% -----------------------------------------------------------
\begin{lemma}[Distributivity]\label{Lemma:Distributivity}
$\icasemof{\acom}{\apredpp}{\bigpredjoin\asetpreds} = \bigpredjoin\icasemof{\acom}{\apredpp}{\asetpreds}$. 
\end{lemma}

% % -----------------------------------------------------------
% % -----------------------------------------------------------
% % -----------------------------------------------------------
% % -----------------------------------------------------------
% -----------------------------------------------------------
% -----------------------------------------------------------
% -----------------------------------------------------------
% -----------------------------------------------------------
% -----------------------------------------------------------
% -----------------------------------------------------------
% -----------------------------------------------------------
% -----------------------------------------------------------
% -----------------------------------------------------------
% -----------------------------------------------------------
% -----------------------------------------------------------
% -----------------------------------------------------------
% -----------------------------------------------------------
% -----------------------------------------------------------
% -----------------------------------------------------------
% -----------------------------------------------------------
\begin{lemma}[Mediation]
  \label{prop:icasem-mediation}
$\icasemof{\acom}{\apredpp}{\apred\mstar\apredppp}\predleq\icasemof{\acom}{\apredpp\mstar\apredppp}{\apred}\mstar\apredppp$, provided $\apredpp$ is precise.
\end{lemma}
\begin{proof}
The inequality is immediate if over-approximation fails for $\icasemof{\acom}{\apredpp\mstar\apredppp}{\apred}$, so assume it holds. 
By \cref{Lemma:OverMediation}, this implies over-approximation for 
$\icasemof{\acom}{\apredpp}{\apred\mstar\apredppp}$. 
We thus have:
\begin{align*}
&\ \icasemof{\acom}{\apredpp}{\apred\mstar\apredppp}\\
=&\ \apredpp\septract\semof{\acom}{\apred\mstar\apredppp\mstar\apredpp}\\
\explain{over-approximation + monotonicity of $\septract$}\predleq&\ \apredpp\septract(((\apredpp\mstar\apredppp)\septract\semof{\acom}{\apred\mstar\apredppp\mstar\apredpp})\mstar\apredpp\mstar\apredppp)\\
\explain{$\apredpp$ precise justifies $\apredpp\septract(\apredp\mstar\apredpp)\predleq\apredp$}\predleq&\ ((\apredpp\mstar\apredppp)\septract\semof{\acom}{\apred\mstar\apredppp\mstar\apredpp})\mstar\apredppp\\
=&\ \icasemof{\acom}{\apredpp\mstar\apredppp}{\apred}\mstar\apredppp.\qedhere
\end{align*} 
\end{proof}
% ----------------------------------------------------------------------------------
% ----------------------------------------------------------------------------------
% ----------------------------------------------------------------------------------
% ----------------------------------------------------------------------------------
It is worth noting that the proof of \cref{prop:icasem-mediation} does not need locality.
This means the induced semantics is mediating even if the underlying semantics it is defined from does not satisfy locality.
% ----------------------------------------------------------------------------------
% ----------------------------------------------------------------------------------
% ----------------------------------------------------------------------------------
% ----------------------------------------------------------------------------------
% ----------------------------------------------------------------------------------
% ----------------------------------------------------------------------------------
% ----------------------------------------------------------------------------------
% ----------------------------------------------------------------------------------

We conclude the section with a completeness result for the induced semantics. 
%
Whenever the original semantics leaves a part $\apredpp$ of the state unchanged, then $\icasem{-}{\apredpp}$ will transform the remainder of the state as desired.
The assumption we have to make is that we can uniquely identify the $\apredpp$ part. 
\begin{proposition}[Completeness of the induced context-aware semantics]\label{Proposition:CompletenessInducedSemantics}
Let $\apredpp$ be precise. 
Then $\semof{\acom}{\apred\mstar\apredpp}\predleq\apredp\mstar\apredpp$ implies $\icasemof{\acom}{\apredpp}{\apred}\predleq\apredp$
\end{proposition} 
\begin{proof}
For over-approximation, we have to show that every state in $\semof{\acom}{\apred\mstar\apredpp}$ has a substate in $\apredpp$, which holds by the premise.
We thus have
\begin{align*}
&\ \icasemof{\acom}{\apredpp}{\apred}\\
=&\ \apredpp\septract\semof{\acom}{\apred\mstar\apredpp}\\
\explain{premise}\predleq&\ \apredpp\septract(\apredp\mstar\apredpp)\\
\explain{$\apredpp$ precise}
\predleq&\ \apredp.\qedhere
\end{align*}
\end{proof}


%%% Local Variables:
%%% mode: latex
%%% TeX-master: "../main"
%%% End:

%!TEX root = ../main.tex

\subsection{Contextualization}
\label{sec:contextualization}

The purpose of rule \ruleref{context} is to frame out predicates that are invariant under the command of interest.
Our goal is to obtain \theLogicSeq derivations that look something like this:
\begin{align*}
	\inferrule*[left={\ruleref{consequence}}]{
		~~
		\inferrule*[left={\ruleref{context}}]{
			~~
			\inferrule*[left={\ruleref{com}}]{
				\casemof{\acom}{\apredpp}{\apred}\predleq\apredp
			}{
				~~\choareOf{\acontext}{\apred}{\acom}{\apredp}~~
			}
			~~
		}{
			\choareOf{\emp}{\apred\mstar\acontext}{\acom}{\apredp\mstar\acontext}
		}
		\\
		\apredppp\predleq\acontext
		~~
	}{
		\choareOf{\emp}{\apred\mstar\apredppp}{\acom}{\apredp\mstar\acontext}
	}
\end{align*}
But how does one determine predicates that are guaranteed to be invariant under the command?
We first tackle this problem for the original semantics $\sem{-}$ and from this derive a schema for obtaining context-aware semantics $\casem{-}{\bullet}$.
To be precise, this is the problem we address next:
\begin{quote}
	\underline{Contextualization} \\
	Given: $\semof{\acom}{\apred\mstar\apredppp}$. \\
	Determine: Predicates $\apredp$ and $\acontext$ with $\apredppp\predleq\acontext$ so that $\semof{\acom}{\apred\mstar\acontext}\predleq \apredp\mstar\acontext$.
\end{quote}
Of course the predicates $\apredp$ and $\acontext$ should be as precise as possible.
We solve this problem in a setting that is specific enough to provide helpful assumptions, yet general enough to cover frameworks like flows~\cite{DBLP:conf/esop/KrishnaSW20,DBLP:conf/tacas/MeyerWW23}
and ghost state induced by morphisms~\cite{DBLP:journals/pacmpl/Nanevski0DF19}.
In analogy to the term framing, we say we \emph{contextualize} $\acontext$.


%=============================================================================%
%=============================================================================%
%=============================================================================%
\subsubsection{The Semantics of updates}
\label{sec:contextualization:semantics}

The motivation for contextualization stems from the fact that the states in $\apredppp$ can be large.
In our examples from \cref{sec:motivation}, these states are full registries and full subtrees.
It is worth having a closer look at what forces us to maintain these rich states.

\begin{example}
	\label{ex:decompose-flow}
	The crucial moment in the BST proof from \cref{sec:motivation:flow} is this Hoare triple:
	\begin{lstlisting}[language=SPL,numbers=none,style=codeInline,keywords={free}]
	  $\annot{
	    \anode \pointsto (l,r,k) \MSTAR p \pointsto (\anodep,\anodepp,i) \MSTAR \anodep \pointsto (\pnull, \anodeppp, j)\MSTAR \apredppp
	  }$
	  $\anode$.$\key$ = $\anodep$.$\key$; $p$.$\lchild$ = $\anodep$.$\rchild$;
	  $\annot{
	    \anode \pointsto (l,r,j) \MSTAR p \pointsto (\anodeppp,\anodepp,i) \MSTAR \apredppp'
	  }$
	\end{lstlisting}
	The update modifies a pointer of $p$ and the key of $\anode$. %, and deallocates $\anodep$.
	This is the change on the states in the predicate $\apred$ introduced in the notion of contextualization.
	However, the update also has an effect on the subtree rooted at $r$ (without $p$ as the node belongs to $\apred$).
	We are interested in the contents of this subtree, the set of keys of all nodes reachable from the root.
	While the subtree does not change physically, the update changes the contents of $\apredppp$.
	In short, while the physical modification involves only few nodes, it influences the ghost state associated with a whole set of nodes.
	The phenomenon is independent of the formalism we use to describe states, be it recursive predicates, flow graphs, or morphisms.
	\qed
\end{example}

\begin{example}
	\label{ex:decompose-lin}
	Recall the linearizability proof goal from \cref{sec:motivation:lin}: \[
		\hoareof{\mcslstate(M(h)) \mstar (h, R)}{\acom}{\mcslstate(M(h)[k \mapsto v]) \mstar ((k,v) \cdot h, R')}
		\enspace .
	\]
	The linearization point $\acom$ of \code{upsert($k$, $v$)} modifies the physical representation $\mcslstate$ of the structure and appends the new key-value pair $(k,v)$ to the history $h$.
	Moreover, $\acom$ also affects the registry $R$: it linearizes all threads that are awaiting $(k,v)$ to be upserted, as dictated by their prophecy variables, resulting in the (potentially entirely) new registry $R'$.
	Here, we are interested in contextualizing the registry $R$ as part of $\apredppp$ and keeping both the physical representation as well as the history in $\apred$.
	The reason for this is that the registry update is induced by the change of the history.
	We wish to focus the proof on the part that matters, the history and its update.
	\qed
\end{example}

To capture the fact that an update involves a modification of the physical state and a modification of the ghost state, we wish to assume that the semantics of commands can be decomposed into the physical update and a separate operation that adjusts the ghost state according purely to the new physical state. 
However, distinguishing between physical and ghost state is unnecessarily strict and impractical in some cases, as seen in \cref{ex:decompose-lin}. 
Instead, we only assume that the semantics of commands decomposes according to the following equalities:
\[
	\semof{\acom}{\apred\mstar\apredppp} \ =\ \upof{\acom}{\apred\mstar\apredppp}\quad\text{and}\qquad \upof{\acom}{\apred\mstar\apredppp} \ =\ \upof{\acom}{\apred}\imult\apredppp\quad \text{if }\upof{\acom}{\apred}\neq\abort
	\enspace .
\]
Here, $\up{\acom}$ is a predicate transformer that implements the \emph{core update}.
The core update satisfies a condition similar to \eqref{Equation:Locality}, except that the ordinary multiplication $\mstar$ is replaced by a ghost multiplication $\imult$ applying the \emph{induced update} on the remaining state.
Going forward, one can think of the core and induced updates as updates to the physical and ghost state, respectively, but our results do not rely on this understanding.
The ghost multiplication is commutative and associative.
(There is no need to assume the existence of units.)
We lift the ghost multiplication to predicates in the expected way: $\apred\imult\apredp=\setcond{\astate\imult\astatep}{\astate\in\apred\wedge \astatep\in\apredp\wedge \astate\imultdef\astatep}$ and $\abort\imult\apred=\abort=\apred\imult\abort$.
We make the assumption that the result of a ghost multiplication decomposes uniquely as follows.
For $\apred_1, \apred_2\subseteq\setstates$ with $\apred_1\imult\apred_2=\apredp$, there are unique smallest predicates $\apredp_1, \apredp_2$ with $\apredp_1\mstar\apredp_2=\apredp$ so that $\apredp_1$ corresponds to $\apred_1$ and $\apredp_2$ corresponds to $\apred_2$.
This correspondence is formalized as an equivalence on states, which we have suppressed as we do not need it beyond this unique decomposition requirement.

\begin{example}
	\label{ex:imult-lin}
	For our registry example from \cref{sec:motivation:lin}, we define the core update $\up{\acom}$ for the linearization point $\acom$ to extend the history: $\upof{\acom}{(h,\emptyset) \mstar \apredppp} = ((k,v) \cdot h, \emptyset) \imult \apredppp$.
	The induced update $\imult$ takes care of linearizing threads according to new entries of the history.
	Formally, \[
		(h, R_1) \imult\, (h, R_2) \,=\, (h, R_1 \uplus R_2)
		\qquad\text{and}\qquad
		((k,v) \cdot h, R_1) \imult\, (h, R_2) \,=\, ((k,v) \cdot h, R_1 \uplus R_2')\ .
	\]
	Here, $R_2'$ is obtained from $R_2$ by changing all entries $R_2(\tid)=\anobl{k,v}$ to $\aful{k,v}$ and leaving all other entries unchanged.
	In all remaining cases, $\imult$ is undefined.
	With this, $\imult$ indeed captures our intuition of an induced update that adjusts the registry given the effect that the core update has on the history.

	For an induced update $(h_1, R_1) \imult\, (h_2, R_2) = (h_3, R_3)$ that is defined, its unique decomposition splits the resulting registry $R_3$ along the domains of $R_1$ and $R_2$ which are disjoint by definition of $\imult$.
	The decomposition is $(h_3, \project{R_3}{\domof{R_1}}) \mstar (h_3, \project{R_3}{\domof{R_2}})$ where $\project{R_3}{\domof{R_1}}$ is the projection of $R_3$ to the domain of $R_1$, and similarly for $R_2$.

	So far, we have ignored the physical representation $\mcslstate$ because it is orthogonal to the ghost state.
	The separation algebra for the overall proof will be a product of two independent separation algebras, one capturing the physical state and one the ghost state.
	The induced update $\imult$ on the ghost state separation algebra extends naturally to the product separation algebra: the core update keeps the entire physical state in $\upof{\acom}{\apred}$, the physical part of $\apredppp$ is $\emp$, and the ghost multiplication is the separating conjunction.
	\qed
\end{example}

The use of a ghost multiplication is inspired by the morphism framework in \cite{DBLP:journals/pacmpl/FarkaN0DF21} where the separation algebra of states $\Sigma$ is mapped to a separation algebra of ghost states  $\Gamma$ that has its own multiplication.
We stay within one separation algebra, which can be thought of as $\Sigma\times\Gamma$, and assume to inherit the second multiplication.


%=============================================================================%
%=============================================================================%
%=============================================================================%
\subsubsection{Solution}

We approach contextualization by abstract interpretation: we give an approximate semantics for the commands from which we can construct the desired predicates.
A particularity of our approach is that we do not want to devise an abstract domain but wish to stay in the realm of separation logic where the algebraic framework is well-developed.
Another particularity is that the semantics of commands consist of a core and an induced update, both of which we have to approximate.

We mimic the core update by an \emph{approximate core update} $\absup{\acom}$.
Like the core update, it should be a predicate transformer that satisfies $\absupof{\acom}{\apred\mstar\apredppp}=\absupof{\acom}{\apred}\imult\apredppp$ if $\absupof{\acom}{\apred}\neq\abort$.
We also expect soundness, $\upof{\acom}{\apred}\predleq\absupof{\acom}{\apred}$.

To mimic the induced update, observe that the ghost multiplication induces a family of predicate transformers $\ghostconc{\apred}$ that capture the effect of the ghost multiplication on the first operand when the second operand is fixed to be~$\apred$.
For $\apred, \apredppp\subseteq\setstates$, we define $\ghostconcof{\apred}{\apredppp}=\apredppp'$, if $\apredppp\imult\apred=\apredppp'\mstar\apred'$ is the unique decomposition.
This can be understood as currying, then a partial instantiation, and finally a masking of the result.
For $\apred=\abort$ or $\apredppp=\abort$, we define $\ghostconcof{\apred}{\apredppp}=\abort$.
It is worth noting that these functions capture the ghost multiplication without loss of information: $\apredppp\imult\apred=\ghostconcof{\apred}{\apredppp}\mstar \ghostconcof{\apredppp}{\apred}$.
To define the predicates we are after, it will be beneficial to approximate this family rather than the multiplication operator.

\begin{example}
	Consider the ghost states $\apred=((k,v) \cdot h, R_1)$ and $\apredppp=(h, R_2)$.
	What is $\apredppp'=\ghostconc{\apred}(\apredppp)$?
	To find it, first compute $((k,v) \cdot h, R_1) \imult\, (h, R_2)$ along the lines of \cref{ex:imult-lin}.
	If the ghost multiplication is undefined, we have $\apredppp'=\bot$.
	Otherwise, it yields $((k,v) \cdot h, R_1 \uplus R_2')$ with $R_2'$ being the appropriately updated registry as before.
	The $\apredppp$-portion of the unique decomposition for this ghost state gives $\apredppp'=((k,v) \cdot h, R_2')$.
	As expected, $\ghostconc{\apred}$ updates $\apredppp$ by extending its history by the new event $(k,v)$ and updating the registry to $R_2'$ accordingly.

	Similarly, $\ghostconc{\apredppp}(\apred) = ((k,v) \cdot h, R_1)$ if the ghost multiplication from above is defined (recall that $\imult$ is commutative).
	We confirm $\apred\imult\apredppp = ((k,v) \cdot h, R_1) \mstar ((k,v) \cdot h, R_2') = \ghostconc{\apredppp}(\apred) \mstar \ghostconc{\apred}(\apredppp)$.
	\qed
\end{example}

An \emph{approximate ghost multiplication} is a family of predicate transformers $\ghostabs{\apred}$.
We now proceed the other way around and use the family to define $\apred\absimult\apredppp=\ghostabsof{\apredppp}{\apred}\mstar\ghostabsof{\apred}{\apredppp}$.
Again, we expect soundness, $\apred\imult\apredppp\predleq\apred\absimult\apredppp$.

With the approximate core and induced updates in place, we can now state our solution to the contextualization problem.
Recall that we are given $\semof{\acom}{\apred\mstar\apredppp}$, and we want to determine predicates $\apredp$ and $\acontext$ with $\apredppp\predleq\acontext$ so that $\semof{\acom}{\apred\mstar\acontext}\predleq \apredp\mstar\acontext$.
We define:
\begin{alignat*}{5}
	\apredpp\ &=\ \rho^*(\apredppp)\qquad&\text{with}\qquad\rho\ &=\ \ghostabs{\apred'}\qquad \text{and}\qquad \apred'=\absupof{\acom}{\apred}
	\\
	\apredp\ &=\ \sigma(\apred')\qquad&\text{with}\qquad\sigma\ &= \ghostabs{\apredpp}.
	\end{alignat*}
By assumption, $\ghostabs{\apred'}$ is a predicate transformer (i.e., strict and a complete join morphism), and so the reflexive transitive closure $\rho^* = \lfpof{(\lambda f.\, \mathit{id} \predtransjoin \rho \circ f)}$ is well-defined.
The construction  captures our intuition about the context being stable under the (ghost) updates inflicted by the command, and it solves contextualization as promised.

\begin{theorem}
	\label{thm:contextualization}
	Consider $\semof{\acom}{\apred\mstar\apredppp}$.
	Then $\semof{\acom}{\apred\mstar\apredpp}\predleq\apredp\mstar\apredpp$ and $\apredppp\predleq\apredpp$.
\end{theorem}

It is worth noting that we only lose precision in the approximations and in the transitive closure.
The transitive closure seems to be unavoidable to make $\apredpp$ invariant under the command.
The physical update is often deterministic and does not need approximation.
Hence, the only parameter worth tuning is the precision of the approximate ghost multiplication.
We illustrate the construction of $\apredp$ and $\apredpp$ in \cref{thm:contextualization} on the registry example  from \cref{sec:motivation:lin}. 
It is worth noting that, in this example, the transitive closure does not lose information because the ghost multiplication is idempotent.

\begin{example}
	Consider $\semof{\acom}{\apred\mstar\apredppp}$ with $\apred=(h,\emptyset)$, $\apredppp=(h,R)$, and $\acom$ the linearization point of an \code{upsert($k$, $v$)}.
	For simplicity, we choose not to perform any approximation here, i.e., choose $\absup{\dontcare}=\up{\dontcare}$ and $\ghostabs{\dontcare}=\ghostconc{\dontcare}$.
	However, we note that the use of approximations can enhance proof automation by improving the convergence of solving the contextualization problem.
	For an example use of approximations, refer to \cref{sec:bst-example}.

	We now compute $\apredp$ and $\apredpp$ to solve contextualization.
	First, we have $\apred'=\absupof{\acom}{\apred}=((k,v) \cdot h, \emptyset)$.
	Then, $\ghostabs{((k,v) \cdot h, \emptyset)}((h,R))=((k,v) \cdot h, R')$ with $R'$ the updated variant of $R$.
	Because $\ghostabs{\dontcare}$ is idempotent, we obtain $\apredpp=((k,v) \cdot h, R')$.
	Finally, $\apredp=\ghostabs{(h, R)}(((k,v) \cdot h, \emptyset))=((k,v) \cdot h, \emptyset)$.
	That is, $((k,v) \cdot h, \emptyset) \mstar ((k,v) \cdot h, R')$ approximates the post image of $\acom$ under $\apred\mstar\apredppp$.
	\qed
\end{example}


%=============================================================================%
%=============================================================================%
%=============================================================================%
\subsubsection{An induced context-aware semantics}

The above solution to the contextualization problem also gives rise to a context-aware semantics based on the over-approximation principle.
The derived context-aware semantics computes the physical update $\apred'$ and applies to it the approximate ghost multiplication for the given context $\acontext$, if $\acontext$ is invariant under the update, that is, if $\acontext$ is the fixed point solution to $\rho^*(\apredppp)$.
We define the \emph{induced context-aware predicate transformer} $\icasem{\acom}{\acontext}$ for a non-empty context $\acontext\neq\emp$ by \[
	\icasemof{\acom}{\acontext}{\apred} = \begin{cases}
		\ghostabsof{\acontext}{\apred'}
		&\text{if~\:}
		\acontext = \ghostabsof{\apred'}{\acontext}
		\text{\:~where~\:}
		\apred'=\absupof{\acom}{\apred}
		\\
		\top
		&\text{otherwise}
		\enspace .
	\end{cases}
\]
For an empty context, there is no need for approximation, we simply use the original semantics, $\icasem{\acom}{\emp}=\sem{\acom}$.
Using \cref{thm:contextualization} it is easy to see that $\icasem{\acom}{\bullet}$ satisfies \eqref{Equation:Mediation}.
That is, we can instantiate \theLogicSeq with $\icasem{\acom}{\bullet}$ and obtain by \Cref{lem:conservative-extension} a conservative extension of separation logic that supports contextualization for reasoning more locally about large footprints.

\begin{theorem}
	\label{thm:approx-induced-casl-is-conservative}
	The \theLogicSeq induced by $\icasem{\acom}{\bullet}$ conservatively extends SL.
\end{theorem}

%%% Local Variables:
%%% mode: latex
%%% TeX-master: "../main"
%%% End:

%!TEX root = ../main.tex
%%% Local Variables:
%%% mode: latex
%%% TeX-master: "../main"
%%% End:

\subsection{A Concurrent Extension}
\label{sec:og}

To reason about concurrent programs in separation logic we employ the Owicki-Gries proof principle \cite{DBLP:journals/acta/OwickiG76}.
That is, we reason in two steps.
First, we verify the program code as if it was run by a single thread in isolation.
Second, we check interference freedom to ensure that the proof remains valid in the presence of other threads.
If so, the concurrent Hoare triple $\hoareOf{\apred}{\astmt}{\apredp}$ is valid, denoted by ${}\mmodels\hoareof{\apred}{\astmt}{\apredp}$, meaning that any number of threads each executing $\astmt$ and starting in $\apred$ will reach $\apredp$.

The judgments for verifying the isolated thread take the form $\thePredicates,\theInterference\semcalc\hoareOf{\apred}{\astmt}{\apredp}$.
The proof rules for these judgments (\Cref{app:og-casl}) collect the predicates that were used during the proof in the set $\thePredicates$ and the interferences in the set~$\theInterference$~\cite[Section 7.3]{DBLP:conf/popl/Dinsdale-YoungBGPY13}.
The interferences can be thought of as pairs $(\acom, \apred)$ for which rule \ruleref{com} was applied.
Recording these pairs allows to later \emph{replay} the effect of the command on other threads.

The interference freedom check ensures that, given a set of interferences $\theInterference$ and a set of predicates $\thePredicates$, no interference $(\acom, \apred)$ from $\theInterference$ can invalidate a predicate $\apredp$ from $\thePredicates$.
Intuitively, this means that replaying $\acom$ under $\apred \cap \apredp$ results in a state covered by $\apredp$.
To support per-thread local state, one has to assume that the underlying separation algebra is a product of two separation algebras defining the global and local state.
Then, the effect of the interfering command is its update to the global state, leaving the local state unchanged.
More concretely, if $\apred=(\ashared_\apred,\alocal_\apred)$ and $\apredp=(\ashared_\apredp,\alocal_\apredp)$ then we compute $\semof{\acom}{\ashared_\apred\cap\ashared_\apredp,\alocal_\apred}=(\ashared',\alocal')$ and check if $(\ashared',\alocal_\apredp)\predleq\apredp$.
If this is the case, we write $\isInterferenceFreeOf[\theInterference]{\thePredicates}$ and say that $\thePredicates$ is interference-free wrt. $\theInterference$.

The resulting Owicki-Gries proof system is sound \cite{DBLP:journals/pacmpl/MeyerWW22}.

\begin{theorem}
	\label{thm:soundness-OG}
	$\thePredicates, \theInterference\semcalc\hoareof{\apred}{\astmt}{\apredp}$
	and
	${}\isInterferenceFreeOf[\theInterference]{\thePredicates}$
	and
	$\apred\in\thePredicates$
	imply
	${}\mmodels\hoareof{\apred}{\astmt}{\apredp}$.
\end{theorem}

We develop \emph{context-aware concurrent separation logic (\theLogicOG)} whose
judgements take the form $\thePredicates, \theInterference\semcalc\choareof{\acontext}{\apred}{\astmt}{\apredp}$.
As for \theLogicSeq, $\acontext$ is meant to be framed to the pre- and postcondition.
That is, validity $\mmodels\choareof{\acontext}{\apred}{\astmt}{\apredp}$ holds iff $\mmodels\hoareof{\apred\mstar\acontext}{\astmt}{\apredp\mstar\acontext}$.
The extended program logic is as expected, we elide it here for brevity.
Refer to \Cref{app:og-casl} for more details.
This extension is sound and it is easy to obtain a conservative extension of the standard Owicki-Gries approach.

\begin{theorem}
	\label{thm:soundness-OGCASL}
	$\thePredicates, \theInterference\semcalc\choareof{\acontext}{\apred}{\astmt}{\apredp}$
	and
	${}\isInterferenceFreeOf[\theInterference]{\thePredicates}$
	and
	$\apred\in\thePredicates$
	imply
	${}\mmodels\choareof{\acontext}{\apred}{\astmt}{\apredp}$.
\end{theorem}

Since contextualization addresses atomic commands, it is equally applicable to both the sequential $\theLogicSeq$ and the concurrent $\theLogicOG$. To avoid notational clutter, we stay within $\theLogicSeq$ throughout the remainder of the paper. However, we stress that we have evaluated our approach against concurrent benchmarks, see \cref{sec:instantiation:automation}.

\input{content/ex_flows}
\section{Related Work}
\label{appsec: related work}
Bayesian causal discovery literature has primarily focused on inference in linear models with closed-form posteriors or marginalized parameters. Early works considered sampling directed acyclic graphs (DAGs) for discrete~\cite{cooper1992bayesian, madigan1995bayesian, heckerman2006bayesian} and Gaussian random variables~\cite{friedman2003being, tong2001active} using Markov chain Monte Carlo (MCMC) in the DAG space. However, these approaches exhibit slow mixing and convergence~\cite{eaton2012bayesian,grzegorczyk2008improving}, often requiring restrictions on number of parents~\cite{kuipers2017partition}. %Alternative exact dynamic programming methods are limited to small settings~\cite{koivisto2012advances}. 

Recent advances in variational inference~\cite{zhang2018advances} have facilitated graph inference in DAG space, with gradient-based methods employing the NOTEARS DAG penalty \cite{zheng2018dags}.\cite{annadani2021variational} samples DAGs from autoregressive adjacency matrix distributions, while \cite{lorch2021dibs} utilizes Stein variational approach \cite{liu2016stein} for DAGs and causal model parameters. \cite{cundy2021bcd} proposed a variational inference framework on node orderings using the gumbel-sinkhorn gradient estimator \cite{mena2018learning}. \cite{deleu2022bayesian,nishikawa2022bayesian} employ the GFlowNet framework \cite{bengio2021gflownet} for inferring the DAG posterior. Most methods, except\cite{lorch2021dibs} are restricted to linear models, while \cite{lorch2021dibs} has high computational costs and lacks DAG generation guarantees compared to our method.
% at least quadratic scaling complexity, both with respect to the number of nodes (due to the DAG penalty) as well as number of posterior samples. Our proposed approach instead has linear complexity with respect to number of posterior samples and does not require any additional DAG penalty.     

In contrast, \emph{quasi-Bayesian} methods, such as DAG bootstrap \cite{friedman2013data}, demonstrate competitive performance. DAG bootstrap resamples data and estimates a single DAG using PC \cite{spirtes2000causation}, GES \cite{chickering2002optimal}, or similar algorithms, weighting the obtained DAGs by their unnormalized posterior probabilities. Recent neural network-based works employ variational inference to learn DAG distributions and point estimates for nonlinear model parameters \cite{charpentier2022differentiable,geffner2022deep}.

%%% Local Variables:
%%% mode: latex
%%% TeX-master: "../main"
%%% End:
