%!TEX root = ../main.tex
%%% Local Variables:
%%% mode: latex
%%% TeX-master: "../main"
%%% End:


\begin{abstract}
Separation logic is often praised for its ability to closely mimic the locality of state updates when reasoning about them at the level of assertions. The prover only needs to concern themselves with the footprint of the computation at hand, i.e., the part of the state that is actually being accessed and manipulated. Modern concurrent separation logics lift this local reasoning principle from the physical state to abstract ghost state. For instance, these logics allow one to abstract the state of a fine-grained concurrent data structure by a predicate that provides a client the illusion of atomic access to the underlying state. However, these abstractions inadvertently increase the footprint of a computation: when reasoning about a local low-level state update, one needs to account for its effect on the abstraction, which encompasses a possibly unbounded portion of the low-level state. Often this gives the reasoning a global character.

We present context-aware (concurrent) separation logic (\theLogic) to provide new opportunities for local reasoning in the presence of rich ghost state abstractions. \theLogic introduces the notion of a context of a computation, the part of the concrete state that is only affected on the abstract level. Contexts give rise to a new proof rule that allows one to reduce the footprint by the context, provided the computation preserves the context as an invariant. The context rule complements the frame rule of separation logic by enabling more local reasoning in cases where the predicate to be framed is known in advance. We instantiate our developed theory for the flow framework, enabling contextual reasoning about programs manipulating general heap graphs, and describe two other applications of the logic. We have implemented the flow instantiation of the logic in a concurrency proof outline checker and used it to verify two highly-concurrent binary search tree implementations with their maintenance operations.
\end{abstract}
