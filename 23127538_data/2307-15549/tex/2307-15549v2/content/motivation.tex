%!TEX root = ../main.tex
%%% Local Variables:
%%% mode: latex
%%% TeX-master: "../main"
%%% End:

\section{Motivation and Overview}
\label{sec:motivation}

We motivate our work by demonstrating how contextual reasoning can simplify linearizability proofs for concurrent data structures.
Such proofs often require ghost state to synchronize the linearization status of all threads, particularly when dealing with future-dependent linearization points (\Cref{sec:motivation:lin}), and to relate the logical contents of the structure to its physical representation (\Cref{sec:motivation:flow}).
Reasoning about these ghost state updates is challenging because they are frequently non-local to the actual physical updates performed by the program code and may involve an unbounded number of ghost resources.
We show how to decompose these complex ghost updates into a finite \emph{core} ghost update and the remaining \emph{context}.
To prove the core ghost update, we proceed as if the context was framed.
For the context, we employ a much simpler proof argument, namely that (the assertion describing) the context is \emph{invariant} under the update.

%!TEX root = ../main.tex

\newcommand{\tombstone}{\square}
\newcommand{\mcsstate}{\mathsf{DS}}
\newcommand{\mcslstate}{\mcsstate}%{\overline{\mcsstate}}
\newcommand{\Hist}{\mathit{Hist}}
\newcommand{\Status}{\mathit{Status}}
\newcommand{\obl}{\mathsf{OBL}}
\newcommand{\ful}{\mathsf{FUL}}
\newcommand{\slt}{\mathsf{SLT}}
\renewcommand{\anobl}[1]{\obl(#1)}
\renewcommand{\aful}[1]{\ful(#1)}
\newcommand{\aslt}[1]{\slt(#1)}
\newcommand{\astatus}[1]{\mathit{status}(#1)}
\newcommand{\Valid}{\mathit{Valid}}
\newcommand{\tid}{\mathit{tid}}
\newcommand{\latest}{\mathit{latest}}

\subsection{Linearizability with Helping}
\label{sec:motivation:lin}

We illustrate contextual reasoning for the purpose of linearizability proofs of concurrent data structure operations whose linearization points are future-dependent and potentially located in other threads.
Specifically, we focus on proofs that use prophecy variables and involve \emph{helping protocols} that govern the transfer of linearizability obligations between threads~\cite{DBLP:journals/pacmpl/JungLPRTDJ20,DBLP:journals/pacmpl/PatelKSW21}.

Concretely, we consider concurrent data structures that implement a (total) map $M$ from keys $K$ to values $V$.
For simplicity, assume a dedicated \emph{tombstone} value $\tombstone \in V$ that indicates the absence of a mapped value.
There are two types of operations: \code{search(k)} retrieves the value associated with $k$ in $M$ and \code{upsert(k,v)} updates the value of $k$ in $M$ to the new value $v$. 
We represent the data structure's physical state using an abstract predicate $\mcslstate(M)$.
So the goal is to prove that the operations are linearizable subject to the expected sequential specification:
\begin{alignat*}{2}
  & \hoareof{\mcslstate(M)}{&~\mcode{search}(k)\,~~&}{\mcode{$v$}.\; \mcslstate(M) * M(k)=v}
  \\
  & \hoareof{\mcslstate(M)}{&~\mcode{upsert}(k,v)~&}{\mcslstate(M[k \mapsto v])}\ .
\end{alignat*}
To do so, we can use the history $h \in (K \times V)^*$ of key/value pairs that have been upserted thus far as an intermediate abstraction of the physical state.
To be precise, $h$ induces the abstract state $M(h)$ that evaluates every key to the latest upserted value for that key or the tombstone if there is no upsert for the key.
The core aspect of the linearizability proof is carried out at this level of abstraction \cite{DBLP:journals/pacmpl/PatelKSW21}.
(Refer to \Cref{sec:motivation:flow} to see how relating the physical state to such an abstraction can also benefit from contextual reasoning.)

We focus on the linearizability argument for search threads.
A thread executing \code{search($k$)} may return value $v$ if either $M(h)(k)=v$ holds for the history $h$ when the search started or some \code{upsert($k$,$v$)} operation linearized during the execution of the search.
In the first case, the linearization point of \code{search($k$)} is right at the start of the operation.
In the second case, its linearization point coincides with the linearization point of the interfering \code{upsert($k$,$v$)} thread.

To enable thread modular reasoning about linearizability, the proof maintains a shared ghost state component that consists of a registry $R$.
The registry is a partial map from the thread IDs of all active search threads to their \emph{linearizability status}, $\anobl{k,v}$ or $\aful{k,v}$.
Status $\anobl{k,v}$ indicates that the thread (i) is searching for key $k$, (ii) it will return value $v$, and (iii) it still has the obligation to linearize.
Status $\aful{k,v}$ is similar but indicates that the thread has fulfilled its obligation to linearize.
The choice of the return value $v$ is implemented using a prophecy variable; we elide the details here.

Overall, the ghost state for the proof is a pair $(h, R)$ consisting of the current history $h$ and the registry $R$.
The actual code induces two kinds of updates to that ghost state: spawning a new search and linearizing an upsert.
When spawning a new \code{search($k$)}, an entry $\tid \mapsto s(k,v)$ for a fresh thread ID $\tid$ is added to the registry $R$, where $v$ is the thread's prophesied return value.
If $M(h)(k)=v$ then $s$ is chosen to be $\ful$ (the thread immediately linearizes) and otherwise $s=\obl$.
That is, the resulting ghost state is $(h, R\uplus\setcompact{\tid \mapsto s(k,v)})$.

When linearizing an \code{upsert($k$, $v$)}, a new pair $(k,v)$ is appended to the history $h$.
More importantly, the registry is updated to linearize other threads that are searching for key $k$ and expect value $v$.
To be precise, the new ghost state is $\bigl((k,v) \cdot h, R'\bigr)$ where $\cdot$ is the concatenation of histories and, for all threads $\tid$, $R'(\tid)=\aful{k,v}$ if $R(\tid)=\anobl{k,v}$ and $R'(\tid)=R(\tid)$ otherwise.
Note that this means we have $M((k,v) \cdot h)(k)=M(h)[k \mapsto v]=v$, so the sequential specification of \code{search($k$)} is satisfied and the search can indeed linearize.

Now, the actual proof of \code{upsert} operations has to deal with both the physical representation and the ghost state, i.e., with assertions $\mcslstate(M(h)) \mstar (h, R)$.
Consequently, for a command $\acom$ executing the linearization point of $\mcode{upsert}(k,v)$, the proof goal will be: \[
  \hoareof{\mcslstate(M(h)) \mstar (h, R)}{\acom}{\mcslstate(M(h)[k \mapsto v]) \mstar ((k,v) \cdot h, R')}
  \enspace .
\]
Notably, the entire proof has to deal with the registry $R$ although its updates are not relevant when updating the physical representation $\mcslstate(M)$.
Nevertheless, we cannot frame $R$ because separation logic does not allow the frame to be changed by $\acom$.
\begin{quote}\itshape
In short, due to the update of the ghost state, separation logic fails to localize the reasoning about the physical update.
\end{quote}

To alleviate this shortcoming of the frame rule, we approximate the exact registry $R$.
Towards this, we define the separating conjunction for the ghost state as $(h_1, R_1) \mstar (h_2, R_2) \defeq (h_1, R_1 \uplus R_1)$ if $h_1=h_2$ and $R_1,R_2$ are disjoint, leaving it undefined in all other case.
Then, rewrite $(h, R)$ into $(h, \emptyset) \mstar (h, R)$.
We use the former conjunct to keep track of the history.
The latter conjunct we approximate by a predicate $\acontext$ that corresponds to the smallest set of ghost states containing $(h, R)$ as well as all $(h'', R'')$ that result from $(h, R)$ by applying some sequence of search and upsert ghost updates, as discussed above.
By construction, $\acontext$ is stable under $\acom$: it denotes the ghost state $(h, R)$ from the precondition as well as the ghost state $((k,v) \cdot h, R')$ from the postcondition.
Note that despite this approximation, we can recover the desired registry $R'$ from computing $(h, \emptyset) \mstar \acontext$.
This leaves us with the following new proof goal: \[
  \hoareof{\mcslstate(M(h)) \mstar (h, \emptyset) \mstar \acontext}{\acom}{\mcslstate(M(h)[k \mapsto v]) \mstar ((k,v) \cdot h, \emptyset) \mstar \acontext}
  \enspace .
\]
Now, we treat $\acontext$ like a frame and ``remove'' it from the proof.
Technically, we do not use the frame rule.
Instead, we use a new context rule.
Like the frame rule, it allows us to ignore $\acontext$ and focus on the remaining parts of the proof.
Unlike the frame rule, we allow commands to modify the resources in $\acontext$.
To that end, we keep $\acontext$ syntactically in the proof tree and ensure its stability under updates of commands like $\acom$.
The result is a \highlight{context-aware} Hoare triple: \[
  \choareHighOf{\highlight{\acontext}}
  {\mcslstate(M(h)) \mstar (h, \emptyset)}{\acom}{\mcslstate(M(h)[k \mapsto v]) \mstar ((k,v) \cdot h, \emptyset)}
  \enspace .
\]
Applying this argument to the full proof of \code{upsert} allows us to focus on the updates of the physical representation.
While moving the registry to the context does not come for free, i.e., without any proof obligation, we observe that stability arguments are typically quite simple and may even be discharged upfront by reasoning over the semantics of commands rather than specific commands, just as we did when introducing the ghost state updates.
Hence, the context-aware Hoare triple that we are left with removes the need for reasoning about the registry altogether.

%%% Local Variables:
%%% mode: latex
%%% TeX-master: "../main"
%%% End:

%!TEX root = ../main.tex

\newcommand{\treepred}{\mathsf{tree}}
\newcommand{\htreepred}{\mathsf{htree}}
\newcommand{\findSucc}{\mymathtt{findSucc}}
\newcommand{\remove}{\mymathtt{remove}}
\newcommand{\pointsto}{\mapsto}
\newcommand{\pnull}{\mathsf{null}}
\newcommand{\aframe}{\mathit{c}}

\subsection{Flow}
\label{sec:motivation:flow}

Contextual reasoning is also useful when relating the physical representation of a data structure to the ghost state that captures its logical contents.
To illustrate this, we use a binary search tree (BST) that implements a mathematical set.
In practical implementations, a BST will have distinct \code{insert} and \code{delete} operations, rather than the single \code{upsert} operation used in our high-level linearizability argument above (\cref{sec:motivation}).
We focus on \code{delete}, specifically the in-place removal of a key stored in an inner node of the tree.
This is the most interesting case of the operation.

\input{content/fig_motivation_code}

\Cref{fig:bst-remove} shows the code of the operation and illustrates how it changes the tree.
Each node in the tree is labeled with its key. The key $k$ to be removed is stored in node $\anode$.
The operation proceeds in four steps.
First, it uses the helper function $\findSucc$ to identify the left-most node $\anodep$ in the right subtree of $\anode$, as well as its parent $p$.
That is, $j$ is the next larger key stored in the tree after $k$.
We omit the definition of $\findSucc$.
The \code{assume} statement models a branching condition.
We focus on the case where $p \neq \anode$.
The operation copies the key $j$ from $\anodep$ to $\anode$, effectively removing $k$ from the structure.
Next, it unlinks $\anodep$ from the tree by setting $p$'s left pointer to the right child of $\anodep$.
This is to maintain the invariant that each key occurs at most once in the tree.
Finally, $\anodep$ is garbage collected.

Our goal is to demonstrate the functional correctness of the operation, meaning the operation updates the tree's contents from $\contents$ to $\contents \setminus \set{k}$.
A conventional proof in separation logic would use a recursive predicate to tie the data structure's physical representation to its contents $\contents$.
However, this approach has several disadvantages, especially for proof automation.
First, the prover needs to infer auxiliary inductive predicates to decompose the proof state into the footprint and the frame.
Next, the prover needs to derive auxiliary data-structure and property-specific lemmas to enable reasoning about the involved (auxiliary) predicates.
Finally, and perhaps most importantly, the proof does not easily generalize.
In a concurrent setting, threads may temporarily break the tree structure by introducing sharing, resulting in DAGs rather than trees.
Consequently, proofs can no longer rely on simple recursive predicates but require more complex machinery such as overlapping conjunctions~\cite{DBLP:conf/aplas/DockinsHA09,DBLP:conf/popl/GardnerMS12} and ramifications~\cite{DBLP:conf/popl/HoborV13}.

\smartparagraph{Node-local reasoning.}
An alternative to recursive predicates is to use indexed separating conjunction to describe unbounded heap regions~\cite{Yang01ShorrWaite, DBLP:conf/cav/0001SS16}.
These are predicates of the form $\bigmstar_{\anode \in \setnodes} \apred(x)$ and express that $\apred(\anode)$ must hold disjointly for all nodes $\anode \in \setnodes$.
The predicate $\apred(\anode)$ specifies a node-local property (e.g., constraining the values of a single points-to predicate for $\anode$).
Indexed separating conjunctions can be easily composed and decomposed along arbitrary partitions of $\setnodes$.
This greatly simplifies framing.
They can also be used to describe general graphs.
The recently proposed flow framework~\cite{DBLP:journals/pacmpl/KrishnaSW18,DBLP:conf/esop/KrishnaSW20,DBLP:conf/tacas/MeyerWW23} extends this approach so that $\apred(\anode)$ can capture global properties of the heap graph spanned by the nodes in $\setnodes$.
The approach works by augmenting every node with additional ghost information, its \emph{flow}.
Flows are computed inductively over the graph structure using a data-flow equation.
The equation can be thought of as collecting information about all possible traversals of the graph.
The definition is such that it still yields generic reasoning principles for decomposing and composing predicates similar to those for indexed separating conjunctions.

A suitable flow for verifying the functional correctness of our $\remove$ operation assigns to each node its \emph{inset}.
Intuitively, the inset of a node $\anode$ consists of the set of keys $k$ such that an operation on $k$ may traverse $\anode$ to find $k$.
\Cref{fig:bst-remove-flows} shows two search trees, before and after execution of the $\remove$ operation, with the inset of each node annotated in {\color{blue}blue}.
For example, the inset of $\anodep$ in the pre-state is the interval $\color{blue}(4,8)$ because the largest (highest up) key on the path from the $\Root$ to $\anodep$ when moving right is $4$ and the smallest key when moving left is $8$.

If we subtract from a node's inset all the insets of its children, we derive its \emph{keyset}.
For example, in the pre-state, the keyset of $p$ is $\{8\}$ and the keyset of $\anode$'s left child is $(-\infty,1]$. 
Assuming searches follow deterministic paths through the graph (as they do for binary search trees), then the keysets are pairwise disjoint~\cite{DBLP:journals/tods/ShashaG88}.
This means the keyset of a node $\anode$ consists of exactly those keys that can only be found in $\anode$ if they are stored anywhere in the structure.

%!TEX root = ../main.tex

% Figure environment removed

%%% Local Variables:
%%% mode: latex
%%% TeX-master: "../main"
%%% End:


To reason about the functional correctness of the operations on the tree, we simply maintain the following \emph{keyset invariant}:  the key stored in each node is contained in the node's keyset.
The overall contents $\contents$ of the tree is the union of all keys stored in its nodes.
The keyset invariant together with the disjointness of the keysets imply that the node-local contents are also disjoint.
Hence, any change made to the contents of a node, such as replacing its key, is reflected by a corresponding change of the global contents $\contents$.
That is, we can now reason node-locally about the overall functional correctness of the search tree operations!


\smartparagraph{Unbounded footprints.}

To enable compositional reasoning about inductive properties, the flow framework adds an additional constraint on separating conjunction: two graphs augmented with flows compose only if their flow values are consistent with the flow obtained in the composite graph~\cite{DBLP:journals/pacmpl/KrishnaSW18,DBLP:conf/esop/KrishnaSW20,DBLP:conf/tacas/MeyerWW23}.
As a consequence, the footprint of an update on the graph can be larger than the \emph{physical} footprint that encompasses the changes to the graph structure (i.e., when ignoring the auxiliary ghost state).
In fact, the full footprint can be unbounded even if the physical footprint is not.

For the $\remove$ operation, the physical footprint consists of the three nodes $\anode$, $p$, and $\anodep$ (shaded yellow in \cref{fig:bst-remove-flows}).
However, observe that moving $\anodep$'s key to $\anode$ changes the insets of all the nodes shaded in gray.
These are the nodes on the path from $\anode$ to $\anodep$ as well as all nodes on the path from $\anode$ to the right-most leaf in its left subtree.
As these paths can be arbitrarily long, the footprint of the update is unbounded.

If we attempt to reason solely about the bounded physical footprint and put everything else into the frame, the proof will fail: after the update, the physical footprint no longer composes with the frame, as the insets of the two regions are inconsistent.
In a sense, the stronger notion of graph composition forces us to reconcile with the global effect of the update immediately at the point when the update occurs. Thus, reasoning about an update with an unbounded footprint appears to entail some form of quantifier instantiation or inductive argument, which adversely affects proof automation.

\begin{quote}\itshape
New reasoning techniques are needed to effectively handle unbounded footprints.
\end{quote}

Existing works on the flow framework have either considered only updates with bounded footprint~\cite{DBLP:conf/esop/KrishnaSW20,DBLP:conf/tacas/MeyerWW23} or cases where the unbounded footprint is traversed by the program prior to the update~\cite{DBLP:journals/pacmpl/MeyerWW22}.
However, not all updates fall into these categories as our example demonstrates.
In this paper, we provide a general solution.

Finally, we note that the issue of having to reason about large footprints is not unique to the flow framework or registry-like constructs.
It has been observed in the literature that this issue arises naturally whenever rich ghost state abstractions are layered on top of the physical state, thereby inducing a stronger notion of separation~\cite{DBLP:journals/pacmpl/Nanevski0DF19, DBLP:journals/pacmpl/FarkaN0DF21}.
This is why we formulate our solution in the setting of abstract separation logic~\cite{DBLP:conf/lics/CalcagnoOY07}, so that it can apply broadly.

%%% Local Variables:
%%% mode: latex
%%% TeX-master: "../main"
%%% End:



\subsection{Contributions and Overview}

Our first contribution is \emph{context-aware (concurrent) separation logic (\theLogic)}, which we describe in \cref{Section:CAReasoning}.
\theLogic is a conservative extension of separation logic that enables local reasoning about computations with large footprints.
Hoare judgments in \theLogic take the form $\choareof{\acontext}{\apred}{\astmt}{\apredp}$.
The judgment decomposes the footprint of $\astmt$ into two parts: a core footprint $\apred$ and a \emph{context} $\acontext$.
In our registry example, the core footprint is $\mcslstate(M(h)) \mstar (h, \emptyset)$, meaning we focus on the physical state $\mcslstate(M(h))$ and maintain $(h, \emptyset)$ as minimimalistic information about the ghost state. 
The context $\acontext$ is the approximation of the registry that takes into account potential updates. 
In our flow example, the core footprint is the physical footprint of the update.
The context is a predicate describing the nodes shaded in gray.

Akin to the frame rule, if $\choareof{\acontext}{\apred}{\astmt}{\apredp}$ is valid, then $\astmt$ transforms $\apred \mstar \acontext$ to $\apredp \mstar \acontext$.
However, the frame rule has to work for all possible frames $\aframe$ and must therefore require that no state in $\aframe$ is affected by $\astmt$.
In contrast, when $\hoareof{\apred}{\astmt}{\apredp}$ is viewed in the context of $\acontext$, the logic can take advantage of the fact that $\acontext$ is known.
This enables new opportunities for local reasoning in the cases where the full footprint of $\astmt$ is large.
Intuitively, $\acontext$ is the part of the state whose ghost component may be affected by the update, but the ghost component changes in a way such that $\acontext$ is maintained.
In the registry example, the context is defined by a closure of the current registry under potential updates, and is therefore invariant under updates by construction.  
In the flow example, the important property being maintained is the keyset invariant.

Our second contribution addresses the question of how to derive appropriate context predicates $\acontext$.
More precisely, given a predicate $\apred \mstar \apredppp$ that describes the pre-states of a computation $\astmt$, the \emph{contextualization} problem is to identify predicates $\apredp$ and $\acontext$ such that $\choareof{\acontext}{\apred}{\astmt}{\apredp}$ and $\apredppp \Rightarrow \acontext$ are valid.
We propose a principled approach based on abstract interpretation that solves contextualization % for the general setting where changes to the physical state induce large ghost state footprints
(\cref{sec:contextualization}).
The crux of the approach is to derive $\apredpp$ using an abstract semantics of $\astmt$.
By tailoring the abstraction to the specific ghost state and property of interest, one can derive simple reasoning principles for showing that $\astmt$ preserves $\apredpp$.
This style of reasoning enables better proof automation compared to proving $\hoareof{\acontext \mstar \apred}{\astmt}{\acontext \mstar \apredp}$ directly in standard separation logic.

We then instantiate this abstract solution for the concrete setting of the flow framework (\cref{sec:instantiation}).
The technical challenge here is that one needs to approximate a fixed point that is computed over the graphs in the image of $\apred \mstar \apredppp$ under $\astmt$, without precise information about what these graphs look like.
Our instantiation is motivated by the observation that, in practice, the change of the flow that emanates from the core footprint simply propagates through the context.
For instance, in the example shown in \cref{fig:bst-remove-flows}, the insets in the left subtree of $\anode$ uniformly increase by $[4,6)$ and in the right subtree they uniformly decrease by $[4,6)$.
In both cases, the change preserves the keyset invariant (which is the desired $\acontext$).
We identify general conditions under which the induced flow changes can be uniformly approximated.
In effect, this allows us to replace complex inductive reasoning to infer $\acontext$ from $\apredppp$ and $\astmt$ with simple local monotonicity reasoning about how the flow changes in the core footprint.

We have implemented our approach in the concurrency proof outline checker \nekton \cite{DBLP:conf/cav/MeyerOWW23} and used it to verify three highly concurrent binary search tree implementations.
It would be difficult to achieve the same degree of proof automation (using flows or recursive predicates) without contextualization due to the aforementioned challenges (unbounded footprints, DAG structures).
