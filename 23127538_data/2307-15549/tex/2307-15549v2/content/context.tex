%!TEX root = ../main.tex
%%% Local Variables:
%%% mode: latex
%%% TeX-master: "../main"
%%% End:

\section{Context-Aware Reasoning for Smaller Footprints}
\label{Section:CAReasoning}

The frame rule is key to local reasoning: it allows one to focus all attention only on a smallest footprint $\apred$ of the computation $\astmt$ and current state at hand, obtaining for free that the remainder of the state, captured by the \emph{frame} $\apredppp$, is preserved by $\astmt$.
We are concerned with situations where the smallest footprint remains inherently large, thwarting any attempt at local reasoning.

What causes large footprints is the locality requirement for commands.
If we cannot guarantee $\semof{\acom}{\astate\mstar\astatep}\predleq\semof{\acom}{\astate}\mstar\set{\astatep}$ for all states $\astatep$, then we have to define $\semof{\acom}{\astate}=\abort$.
That is, $\acom$ aborts on $\astate$ and any attempt at reasoning locally about the effect of $\acom$ on $\astate$ will fail.
The locality requirement, in turn, is a consequence of the fact that the frame rule is meant to hold for all possible frames.
It says that, no matter which frame $\apredppp$ is added to the proof, the program has to leave it unchanged.
In short, since the frame rule is context-agnostic, we need locality, and due to locality programs that affect a large part of the state inherently have large footprints.

This work starts from the idea of introducing a context-aware variant of the frame rule that justifies smaller footprints when reasoning about programs whose effect on the frame is benign.
The rationale is that if the predicate $\apredppp$ to be added by framing is known, then we can relax the locality requirement
and hence enable more local reasoning.
We develop this idea in a conservative extension of separation logic.


%=============================================================================%
%=============================================================================%
%=============================================================================%
\subsection{Context-Aware Separation Logic}
We propose \emph{context-aware separation logic (\theLogicSeq)} in which correctness statements $\choareof{\acontext}{\apred}{\astmt}{\apredp}$ are Hoare triples enriched by a \emph{context} $\acontext$.
The context is a predicate that is meant to be framed to the Hoare triple $\hoareof{\apred}{\astmt}{\apredp}$.
This intuition is captured by the rule \ruleref{context} and becomes more evident as we define the validity of such correctness statements.

\begin{definition}[Validity of \theLogicSeq statements]
	\label{def:casl-validity}
	\(
		{}\models \choareof{\acontext}{\apred}{\astmt}{\apredp}
		~~\defifff~
		{}\models \hoareof{\apred\mstar\acontext}{\astmt}{\apredp\mstar\acontext}
		\enspace .
	\)
\end{definition}

We reason about the validity of \theLogicSeq statements using the program logic from \cref{Figure:PL} (including the \makeColorLogic{blue} parts).
We write $\vdash\choareof{\acontext}{\apred}{\astmt}{\apredp}$ if a correctness statement can be derived using this logic.
The benefit of knowing the predicate that should be framed is that we can relax the locality requirement on the semantics of commands relative to that context.
To develop this relaxation, observe that pushing and pulling predicates $\acontext$ into and from the context as captured by the new rules \ruleref{context} and \ruleref{widen}, respectively, is sound immediately by our definition of validity above.
Instead, we have to focus on the modified rule \ruleref{com}: it uses a new context-aware semantics that takes the role of the standard semantics.

A context-aware semantics is a function that assigns to each command $\acom\in\setcom$ a context-aware predicate transformer $\casem{\acom}{\bullet}:\setpreds\rightarrow\setpredtrans$.
A context-aware predicate transformer expects a context $\acontext$ as input and returns a suitable predicate transformer $\casem{\acom}{\acontext}$.
Context-aware semantics extend naturally to programs.

The soundness of rule \ruleref{com} then relies on the requirement that the context-aware semantics over-approximates the standard semantics for the different choices of the context.

\begin{definition}
	Let $\acom \in \setcom$ and $\acontext \in \setpreds$. We say that $\casem{\acom}{\acontext}$ satisfies \emph{mediation} if
	\begin{align}
		\forall \apred\in\setpreds.
		\quad
		\semof{\acom}{\apred\mstar\acontext} ~\predleq~ \casemof{\acom}{\acontext}{\apred}\mstar\acontext
		\enspace.
		\tag{Mediation}
		\label{Equation:Mediation}
	\end{align}
\end{definition}

Although we need \eqref{Equation:Mediation} for the soundness of rule \ruleref{com}, it plays a similar role for \ruleref{context} as locality does for the \ruleref{frame} rule: it allows us to push a predicate $\acontext$ into the context and focus on the remainder $\apred$ if we can guarantee that $\acontext$ is invariant under the actions of the program.

Soundness of \theLogicSeq now follows because \ruleref{com} is sound by \eqref{Equation:Mediation}, \ruleref{context} and \ruleref{widen} are sound as they exploit our validity from \cref{def:casl-validity}, and the remaining rules are sound because separation logic is sound by \cref{Lemma:SoundnessHoare}.

\begin{theorem}[Soundness of \theLogicSeq]
	\label{thm:casl-soundness}
	Assume that $\sem{\acom}$ satisfies \eqref{Equation:Locality} and $\casem{\acom}{\acontext}$ satisfies \eqref{Equation:Mediation} for all $\acom\in\setcom$, $\acontext\in\setpreds$.
	Then ${}\vdash\choareof{\acontext}{\apred}{\astmt}{\apredp}$ implies ${}\models\choareof{\acontext}{\apred}{\astmt}{\apredp}$.
\end{theorem}

Actually, to prove \Cref{thm:casl-soundness}, we only need mediation for  $\casem{-}{\acontext'}$, if $\acontext'$ is a context that occurs in an applications of rule \ruleref{com} which is used to derive $\vdash\choareof{\acontext}{\apred}{\astmt}{\apredp}$.
We pose the stricter requirement that mediation has to hold for all contexts to avoid a side condition in the rule.
However, \eqref{Equation:Mediation} can be weakened so that it is only required to hold for the contexts that are of interest for a particular proof.


% ----------------------------------------------------------------------------
% ----------------------------------------------------------------------------
\paragraph{Conservative extensions}
It is worth pointing out that the above soundness result does not rely on any correspondence, besides \eqref{Equation:Mediation}, among the standard semantics $\sem{-}$ and the new context-aware semantics $\casem{-}{\bullet}$.
While we exploit this potential for approximation for practical purposes in \Cref{sec:contextualization}, we typically start from {\theLogicSeq}s that conservatively extend separation logics.
That is, we study {\theLogicSeq}s that are both sound and complete relative to the separation logic induced by a given standard semantics:
\begin{align*}
	\forall \astmt, \apred, \apredp, \acontext. \quad & & \vdash \choareOf{\acontext}{\apred}{\astmt}{\apredp} \implies & \vdash \hoareOf{\apred \mstar \acontext}{\astmt}{\apredp \mstar \acontext} &
	\tag{Relative Soundness}
	\\
	\forall \astmt, \apred, \apredp. \quad & & \vdash \hoareOf{\apred}{\astmt}{\apredp} \implies & \; \exists \acontext. \; \vdash \choareOf{\acontext}{\apred}{\astmt}{\apredp}\ . &
	\tag{Relative Completeness}
\end{align*}

One can always obtain such a conservative extension from a separation logic induced by any given standard semantics $\sem{-}$.
The canonical way to do so is to let $\casem{-}{\emp}$ and $\sem{-}$ coincide.

\begin{theorem}[Conservative extension]
	\label{lem:conservative-extension}
	If $\casem{\acom}{\acontext}$ satisfies \eqref{Equation:Mediation} for all $\acom$ and $\acontext$, and $\casem{-}{\emp} = \sem{-}$, then the \theLogicSeq induced by $\casem{-}{\bullet}$ conservatively extends the SL induced by $\sem{-}$.
\end{theorem}
In the remainder of the section, we develop machinery for deriving suitable context-aware semantics.
