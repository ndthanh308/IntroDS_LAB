%!TEX root = ../main.tex
%%% Local Variables:
%%% mode: latex
%%% TeX-master: "../main"
%%% End:

\section{Proofs of Section~\ref{Section:CAReasoning}}
\label{sec:ctx-proofs}

\begin{proof}[Proof of \Cref{thm:casl-soundness}]
	For all rules, we show that the validity of their precondition entails the validity of their postcondition.
	From this, the overall claim follows by a straightforward rule induction of the \theLogicSeq derivation tree.

	\paragraph{Rule \ruleref{com}}
	We have $\casemof{\acom}{\acontext}{\apred}\predleq\apredp$.
	By \eqref{Equation:Mediation} then, $\semof{\acom}{\apred\mstar\acontext} \predleq \casemof{\acom}{\acontext}{\apred}\mstar\acontext \predleq \apredp\mstar\acontext$.
	That is, $\models \hoareof{\apred\mstar\acontext}{\acom}{\apredp\mstar\acontext}$ is valid.
	So $\models \choareof{\acontext}{\apred}{\acom}{\apredp}$ is valid as well, by \Cref{def:casl-validity}.

	\paragraph{Rule \ruleref{consequence}}
	We have $\apred\predleq\apred'$, $\apredp\predleq\apredp'$, and $\models \choareof{\acontext}{\apred'}{\astmt}{\apredp'}$.
	By \Cref{def:casl-validity}, the latter means $\models \hoareof{\apred'\mstar\acontext}{\astmt}{\apredp'\mstar\acontext}$.
	Since separation logic is sound, \Cref{Lemma:SoundnessSL}, we obtain $\models \hoareof{\apred\mstar\acontext}{\astmt}{\apredp\mstar\acontext}$ using \ruleref{consequence}.
	Again by \Cref{def:casl-validity}, we get $\models \choareof{\acontext}{\apred}{\astmt}{\apredp}$.

	\paragraph{Rule \ruleref{seq}}
	We have $\models \choareof{\acontext}{\apred}{\astmt_1}{\apredp}$ and $\models \choareof{\acontext}{\apredp}{\astmt_2}{\apredppp}$.
	By \Cref{def:casl-validity}, this means $\models \hoareof{\apred\mstar\acontext}{\astmt_1}{\apredp\mstar\acontext}$ and $\models \hoareof{\apredp\mstar\acontext}{\astmt_2}{\apredppp\mstar\acontext}$.
	By \Cref{Lemma:SoundnessSL}, an application of rule \ruleref{seq} gives $\models \hoareof{\apred\mstar\acontext}{\seqof{\astmt_1}{\astmt_2}}{\apredppp\mstar\acontext}$.
	Then, $\models \choareof{\acontext}{\apred}{\seqof{\astmt_1}{\astmt_2}}{\apredppp}$ follows by \Cref{def:casl-validity}.

	\paragraph{Rule \ruleref{choice}}
	We have $\models \choareof{\acontext}{\apred}{\astmt_1}{\apredp}$ and $\models \choareof{\acontext}{\apred}{\astmt_2}{\apredp}$.
	By \Cref{def:casl-validity}, this means $\models \hoareof{\apred\mstar\acontext}{\astmt_1}{\apredp\mstar\acontext}$ and $\models \hoareof{\apred\mstar\acontext}{\astmt_2}{\apredp\mstar\acontext}$.
	By \Cref{Lemma:SoundnessSL}, rule \ruleref{choice} gives $\models \hoareof{\apred\mstar\acontext}{\choiceof{\astmt_1}{\astmt_2}}{\apredp\mstar\acontext}$.
	Then, $\models \choareof{\acontext}{\apred}{\choiceof{\astmt_1}{\astmt_2}}{\apredp}$ follows by \Cref{def:casl-validity}.

	\paragraph{Rule \ruleref{loop}}
	We have $\models \choareof{\acontext}{\apred}{\astmt}{\apred}$.
	By \Cref{def:casl-validity}, this means $\models \hoareof{\apred\mstar\acontext}{\astmt}{\apred\mstar\acontext}$.
	By \Cref{Lemma:SoundnessSL}, rule \ruleref{loop} gives $\models \hoareof{\apred\mstar\acontext}{\loopof{\astmt}}{\apred\mstar\acontext}$.
	Then, $\models \choareof{\acontext}{\apred}{\astmt}{\apredp}$ follows by \Cref{def:casl-validity}.

	\paragraph{Rule \ruleref{frame}}
	We have $\models \choareof{\acontext}{\apred}{\astmt}{\apredp}$.
	By \Cref{def:casl-validity}, this means $\models \hoareof{\apred\mstar\acontext}{\astmt}{\apredp\mstar\acontext}$.
	By \Cref{Lemma:SoundnessSL}, rule \ruleref{frame} gives $\models \hoareof{\apred\mstar\acontext\mstar\apredppp}{\astmt}{\apredp\mstar\acontext\mstar\apredppp}$.
	Then, $\models \choareof{\acontext}{\apred\mstar\apredppp}{\astmt}{\apredp\mstar\apredppp}$ follows by \Cref{def:casl-validity}.

	\paragraph{Rule \ruleref{context}}
	We have $\models \choareof{\acontext\mstar\apredppp}{\apred}{\astmt}{\apredp}$.
	By the definition of validity, \Cref{def:casl-validity}, this means $\models \hoareof{\apred\mstar\acontext\mstar\apredppp}{\astmt}{\apredp\mstar\acontext\mstar\apredppp}$.
	Again by \Cref{def:casl-validity}, we get $\models \choareof{\acontext}{\apred\mstar\apredppp}{\astmt}{\apredp\mstar\apredppp}$.

	\paragraph{Rule \ruleref{widen}}
	We have $\models \choareof{\acontext}{\apred\mstar\apredppp}{\astmt}{\apredp\mstar\apredppp}$.
	By the definition of validity, \Cref{def:casl-validity}, this means $\models \hoareof{\apred\mstar\acontext\mstar\apredppp}{\astmt}{\apredp\mstar\acontext\mstar\apredppp}$.
	Again by \Cref{def:casl-validity}, we get $\models \choareof{\acontext\mstar\apredppp}{\apred}{\astmt}{\apredp}$.
\end{proof}

\begin{proof}[Proof of \Cref{lem:conservative-extension}]
	Relative soundness and relative completeness follow from a rule induction over the \theLogicSeq derivation that constructs a SL derivation mimicking the \theLogicSeq derivation one-to-one with context, and vice versa.
\end{proof}

\begin{proof}[Proof of \Cref{thm:soundness-OG}]
	See \cite[Theorem 4.2]{DBLP:journals/pacmpl/MeyerWW22}.
\end{proof}

\begin{proof}[Proof of \Cref{thm:soundness-OGCASL}]
	Analogous to the proof of \Cref{thm:casl-soundness}.
\end{proof}

\begin{proof}[Proof of \Cref{thm:contextualization}]
	As $\apredpp$ is a reflexive and transitive closure of $\apredppp$, we have $\apredppp\predleq\apredpp$. 
	It remains to prove $\semof{\acom}{\apred\mstar\apredpp}\predleq\apredp\mstar\apredpp$. 
	The interesting case is $\apredp\neq\abort\neq\apredpp$. 
	Then also $\apred'\neq\abort$ and we have 
	\begin{align*}
		&~~ \semof{\acom}{\apred\mstar\apredpp}\\
		\explain{$\upof{\acom}{\apred}\neq \abort$ by soundness of $\absup{\acom}$ and $\apred'\neq\abort$}=&~~ \upof{\acom}{\apred}\imult\apredpp\\
		\explain{Soundness of $\absup{\acom}$}\predleq&~~ \absupof{\acom}{\apred}\imult\apredpp\\
		\explain{Soundness of $\absimult$}\predleq&~~ \absupof{\acom}{\apred}\absimult\apredpp\\
		\explain{Definition of $\apred'$}=&~~ \apred'\absimult\apredpp\\
		\explain{Definition of $\absimult$}=&~~ \ghostabsof{\apredpp}{\apred'}\mstar\ghostabsof{\apred'}{\apredpp}\\
		\explain{Definition of $\apredp$ and $\rho$}=&~~ \apredp\mstar\rho(\apredpp)\\
		\explain{Definition of $\apredpp$}\predleq&~~ \apredp\mstar\apredpp
		\enspace .
		\qedhere
	\end{align*}
\end{proof}

\begin{proof}[Proof of \Cref{thm:approx-induced-casl-is-conservative}]
	The claim follows from \Cref{lem:conservative-extension}.
	We show that \Cref{lem:conservative-extension} applies.
	By definition, we have $\icasem{\acom}{\emp}=\sem{\acom}$.
	So it remains to show that $\icasem{\acom}{\apredppp}$ satisfies \eqref{Equation:Mediation} for all $\acom$ and $\apredppp$.
	Consider some $\apred\in\setpreds$.
	We establish $\semof{\acom}{\apred\mstar\apredppp} \predleq \icasemof{\acom}{\apredppp}{\apred}\mstar\apredppp$.
	For $\apredppp=\emp$, the inclusion follows immediately because $\icasem{\acom}{\emp}=\sem{\acom}$.
	Assume $\apredppp\neq\emp$.
	If $\icasemof{\acom}{\apredppp}{\apred}=\top$, then the desired inclusion holds by definition.
	So assume $\icasemof{\acom}{\apredppp}{\apred}\neq\top$.
	This means $\icasemof{\acom}{\apredppp}{\apred}=\ghostabsof{\apredppp}{\apred'}$ with $\apred'=\absupof{\acom}{\apred}\neq\top$ and $\apredppp=\ghostabsof{\apred'}{\apredppp}$.
	Observe that the latter means $\apredppp=\rho^*(\apredppp)$.

	Now, apply \Cref{thm:contextualization} to $\semof{\acom}{\apred\mstar\apredppp}$.
	We obtain $\semof{\acom}{\apred\mstar\acontext}\predleq\ghostabsof{\acontext}{\apred'}\mstar\acontext$ with $\acontext=\rho^*(\apredppp)$ and $\apredppp\predleq\acontext$.
	By the above observation, $\apredppp=\acontext$ must hold.
	That is, $\semof{\acom}{\apred\mstar\apredppp}\predleq\ghostabsof{\apredppp}{\apred'}\mstar\apredppp$.
	Hence, we arrive at $\semof{\acom}{\apred\mstar\apredppp}\predleq\icasemof{\acom}{\apredppp}{\apred}\mstar\apredppp$, as required.
	Overall, this concludes that $\icasem{\acom}{\bullet}$ induces a \theLogicSeq that is a conservative extension of SL.
\end{proof}
