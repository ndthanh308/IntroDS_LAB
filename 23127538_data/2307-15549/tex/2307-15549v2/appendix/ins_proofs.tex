%!TEX root = ../main.tex
%%% Local Variables:
%%% mode: latex
%%% TeX-master: "../main"
%%% End:

\subsection{Proofs for the Instantiation}

% -----------------------------------------------------------------------------
% -----------------------------------------------------------------------------
% -----------------------------------------------------------------------------
% -----------------------------------------------------------------------------
Let $\astate_1\imult\astate_2=\astatep$. Then there are $\astatep_1$ and $\astatep_2$ with $\astatep=\astatep_1\mstar\astatep_2$ and $\astate_1.\setnodes=\astatep_1.\setnodes$ and $\astate_2.\setnodes=\astatep_2.\setnodes$. Moreover, the flow graphs $\astatep_1$ and $\astatep_2$ are unique. 

\begin{proof}[Proof of \Cref{thm:unique-ghost-decomposition}]
	\label{proof:unique-ghost-decomposition}
	Consider $\afg_1\imult\afg_2=\afgc$ with $\afg_i=(\setnodes_i,\edges_i,\inflow_i)$.
	Choose $\afgp_1=\restrictto{\afgc}{\setnodes_1}$ and $\afgp_2=\restrictto{\afgc}{\setnodes_2}$.
	By definition, $\afg_1.\setnodes=\afgp_1.\setnodes$ and $\afg_1.\edges=\afgp_1.\edges$.
	Similarly, $\afg_2.\setnodes=\afgp_2.\setnodes$ and $\afg_2.\edges=\afgp_2.\edges$.
	Moreover, \cref{thm:restriction-vs-statemult} gives $\afgp_1\statemultdef\afgp_2$ and $\afgp_1\statemult\afgp_2=\afgc$.
	This establishes the first claim.

	It remains that the decomposition of $\afgc$ into $\afgp_1$ and $\afgp_2$ is unique.
	Towards a contradiction, assume there are $\afgp_1',\afgp_2'$ such that $\afgp_1\neq\afgp_1'$ and $\afgp_2\neq\afgp_2'$ and $\afgp_1'\statemult\afgp_2'=\afgc$.
	By the definition of the multiplication, we have $\afgp_i.\setnodes=\afgp_i'.\setnodes$ and $\afgp_i.\edges=\afgp_i'.\edges$, for $i\in\set{1,2}$.
	That is, $\afgp_1.\inflow\neq\afgp_1'.\inflow$ or $\afgp_2.\inflow\neq\afgp_2'.\inflow$.
	Wlog. assume $\afgp_1.\inflow\neq\afgp_1'.\inflow$.
	By definition, $(\afgp_1\statemult\afgp_2).\inflow=\afgc.\inflow=(\afgp_1'\statemult\afgp_2').\inflow$.
	Hence, $\afgp_1.\inflow\neq\afgp_1'.\inflow$ means that there is a pair of nodes $\anode\in\afgp_1.\setnodes,\,\anodep\in\afgp_2.\setnodes$ that witnesses the inequality, $\afgp_1.\inflow(\anodep,\anode)\neq\afgp_1'.\inflow(\anodep,\anode)$.
	From the choice of $\afgp_1$, we know that $\afgp_1.\inflow(\anodep,\anode)=\afgp_2.\outflow(\anodep,\anode)=\afgp_2.\edgesatof{\anodep}{\anode}{\afgp_2.\fvalof{\anodep}}$.
	By definition, $\afgp_2.\fvalof{\anodep}=(\afgp_1.\fval\uplus\afgp_2.\fval)(\anodep)$.
	By $\afgp_1\statemultdef\afgp_2$ then, $\afgp_2.\fvalof{\anodep}=(\afgp_1\imult\afgp_2).\fvalof{\anodep}=\afgc.\fvalof{\anodep}$.
	Combined, $\afgp_1.\inflow(\anodep,\anode)=\afgp_2.\edgesatof{\anodep}{\anode}{\afgc.\fvalof{\anodep}}$.
	Similarly, we obtain $\afgp_1'.\inflow(\anodep,\anode)=\afgp_2'.\edgesatof{\anodep}{\anode}{\afgc.\fvalof{\anodep}}$.
	Because $\afgp_2.\edges=\afgp_2'.\edges$ by assumption, we conclude $\afgp_1.\inflow(\anodep,\anode)=\afgp_1'.\inflow(\anodep,\anode)$.
	This contradicts the earlier $\afgp_1.\inflow(\anodep,\anode)\neq\afgp_1'.\inflow(\anodep,\anode)$.
\end{proof}


% -----------------------------------------------------------------------------
% -----------------------------------------------------------------------------
% -----------------------------------------------------------------------------
% -----------------------------------------------------------------------------
\begin{proof}[Proof of \Cref{Lemma:FlowAlgebra}]
	See \cite[Lemma 2]{DBLP:conf/tacas/MeyerWW23}
\end{proof}

\begin{proof}[Proof of \Cref{Lemma:MultCoincides}]
	Follows immediately from the definition of the multiplication $\mstar$.
\end{proof}


% -----------------------------------------------------------------------------
% -----------------------------------------------------------------------------
% -----------------------------------------------------------------------------
% -----------------------------------------------------------------------------
\begin{proof}[Proof of \Cref{thm:instantiation-closure}]
	Consider flow graphs $\afg,\afgp,\afgc$ with $\afg\ctxfprel\afgp$ and, $\afg\statemultdef\afgc$.
	Because $\fprel$ is an estimator along the lines of \cref{sec:instantiation}, we have $\fpcompatible[\fprel]{\afg}$, $\fpcompatible[\fprel]{\afgp}$, and $\fpcompatible[\fprel]{\afgc}$.
	Then, \cref{thm:upward-closed-framing} for $\afg,\afgp,\afgc$ yields $\afgp'\in\fpclosureof[\fprel]{\afgc}{\afgp}$ and $\afgc'\in\fpclosureof[\fprel]{\afgp}{\afgc}$ such that $\afgp'\statemultdef\afgc'$ and  $\afg \statemult \afgc \ctxfprel \afgp'\statemult\afgc'$.
	By \cref{Lemma:MultCoincides}, we have $\afgp'\statemult\afgc'=\afgp'\imult\afgc'$.
	By the definition of $\fpclosureof[\fprel]{\afgc}{\afgp}$, we have $\afgp'=(\afgp.\setnodes,\afgp.\edges,\inflow_{\afgp'})$ with $\afgp.\inflow\fprelup{\afgc.\setnodes}\inflow_{\afgp'}$.
	Together with $\afg.\inflow=\afgp.\inflow$ from $\afg\ctxfprel\afgp$, we get $\afg.\inflow\fprelup{\afgc.\setnodes}\inflow_{\afgp'}$.
	Similarly, $\afgc'=(\afgc.\setnodes,\afgc.\edges,\inflow_{\afgc'})$ with $\afgc.\inflow\fprelup{\afgp.\setnodes}\inflow_{\afgc'}$.
	This means that $\afgp$ and $\afgp'$ agree on their inflow except for the portion from $\afgc$, and similarly $\afgc$ and $\afgc'$ agree on their inflow except for the portion from $\afgp$.
	Since the ghost multiplication removes this, we obtain $\afgp\imult\afgc=\afgp'\imult\afgc'$.
	Altogether, we arrive at the desired correspondence: \[
		\afg\statemult\afgc
		~\ctxfprel~
		\afgp'\statemult\afgc'
		~=~
		\afgp'\imult\afgc'
		~=~
		\afgp\imult\afgc
		\ .
		\qedhere
	\]
\end{proof}


% -----------------------------------------------------------------------------
% -----------------------------------------------------------------------------
% -----------------------------------------------------------------------------
% -----------------------------------------------------------------------------
\begin{proof}[Proof of \Cref{thm:instantiation-physical}]
	The first claim holds by definition.
	For the second claim, assume $\absupof{\acom}{\afg}\neq\abort$ and $\afg\statemultdef\afgc$.
	We show $\absupof{\acom}{\afg\statemult\afgc}=\absupof{\acom}{\afg}\imult\afgc$.
	To that end, it suffices to show that $\upof{\acom}{\afg\statemult\afgc}$ does not abort and its states satisfy the estimator requirement.
	Indeed, then
	\begin{align*}
		&\absupof{\acom}{\afg\statemult\afgc}
		\\
		\explain{Definition}=~~&
		\upof{\acom}{\afg\statemult\afgc}
		\\
		\explain{$\absupof{\acom}{\afg}\neq\abort$ implies $\upof{\acom}{\afg}\neq\abort$}=~~&
		\upof{\acom}{\afg}\imult\afgc
		\\
		\explain{$\absupof{\acom}{\afg}\neq\abort$}=~~&
		\absupof{\acom}{\afg}\imult\afgc. 
	\end{align*}
	That $\upof{\acom}{\afg\statemult\afgc}$ does not abort follows from $\upof{\acom}{\afg}\neq\abort$.
	For the estimator requirement, let $\afgp\imult\afgc\in \upof{\acom}{\afg\statemult\afgc}=\upof{\acom}{\afg}\imult\afgc$.
	We have to show $\afg\statemult\afgc\ctxfprel\afgp\imult\afgc$.
	Since $\absupof{\acom}{\afg}\neq\abort$, we can rely on $\afg\ctxfprel\astatep$.
	Then \cref{thm:instantiation-closure} concludes the argument.
\end{proof}



% -----------------------------------------------------------------------------
% -----------------------------------------------------------------------------
% -----------------------------------------------------------------------------
% -----------------------------------------------------------------------------
\begin{proof}[Proof of \Cref{thm:instantiation-ghost}]
	\label{proof:instantiation-ghost}
	Consider some flow graphs $\afgp,\afgc$.
	If $\afgp\absimult\afgc = \top$, there is nothing to show.
	So assume $\afgp\absimult\afgc\neq\top$.
	This means $\ghostabsof{\afgc}{\afgp}\neq\top$ and $\ghostabsof{\afgp}{\afgc}\neq\top$.
	Wlog. this means that there is some flow graph $\afg$ with $\afg\ctxfprel\afgp$ and $\afg\statemultdef\afgc$, by the definition of $\ghostabs{\afgc}$.
	Then, or \cref{thm:instantiation-closure} gives $\afgp\imult\afgc=\afgp[\inflow\mapsto\inflow_\afgp]\statemult\afgc[\inflow\mapsto\inflow_\afgc]$ with $\afgp.\inflow\fprelup{\afgc.\setnodes}\inflow_\afgp$ and $\afgc.\inflow\fprelup{\afgp.\setnodes}\inflow_\afgc$.
	By definition, we conclude: \[
		\afgp\imult\afgc
		~=~
		\afgp[\inflow\mapsto\inflow_\afgp]\statemult\afgc[\inflow\mapsto\inflow_\afgc]
		~\in~
		\ghostabsof{\afgc}{\afgp}\statemult\ghostabsof{\afgp}{\afgc}
		~=~
		\astatep\absimult \astatepp
		\ .
		\qedhere
	\]
\end{proof}
