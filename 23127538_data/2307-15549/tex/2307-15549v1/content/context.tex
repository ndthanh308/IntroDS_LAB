%!TEX root = ../main.tex

\section{Context-Aware Reasoning for Smaller Footprints}\label{Section:CAReasoning}

The frame rule is key to local reasoning: it allows one to focus all attention only on a smallest footprint $\apred$ of the computation $\astmt$ and current state at hand, obtaining for free that the remainder of the state, captured by the \emph{frame} $d$, is preserved by $\astmt$. We are concerned with situations where the smallest footprint remains inherently large, thwarting any attempt at local reasoning.

What causes large footprints is the locality requirement for commands. 
If we cannot guarantee $\semof{\acom}{\astate\mstar\astatep}\predleq\semof{\acom}{\astate}\mstar\set{\astatep}$ for all states~$\astatep$, then we have to define $\semof{\acom}{\astate}=\abort$. That is, $\acom$ aborts on $\astate$ and any attempt at reasoning locally about the effect of $\acom$ on $\astate$ will fail.
%
The locality requirement, in turn, is a consequence of the fact that the frame rule is meant to hold for all possible frames. 
%
It says that, no matter which frame $\apredppp$ is added to the proof, the program has to leave it unchanged. 
%
In short, since the frame rule is context-agnostic, we need locality, and due to locality programs that affect a large part of the state inherently have large footprints.

This work starts from the idea of introducing a context-aware variant of the frame rule that justifies smaller footprints when reasoning about programs whose effect on the frame is benign. 
%
The rationale is that if the predicate $\apredppp$ to be added by framing is known, then we can relax the locality requirement, redefine the semantics in a way that aborts less often, and hence enable more local reasoning.
%
We develop this idea in a conservative extension of separation logic.

\subsection{Context-Aware Separation Logic}
We propose \emph{context-aware separation logic (CASL)} in which correctness statements $\choareof{\apredpp}{\apred}{\astmt}{\apredp}$ are Hoare triples enriched by a so-called \emph{context} $\acontext$.
%
The context is a predicate that is meant to be framed to the Hoare triple $\hoareof{\apred}{\astmt}{\apredp}$. This intuition is captured by the rule \ruleref{context} in \cref{Figure:PL}. 

The benefit of knowing the predicate that should be framed is that we can relax the locality requirement on the semantics of commands relative to that context.

A context-aware semantics for $\setcom$ is a function that assigns to each $\acom\in\setcom$ a context-aware predicate transformer $\casem{\acom}{\bullet}:\setcapredtrans=\setpreds\rightarrow\setpredtrans$. 
That is, a context-aware predicate transformer expects a context $\acontext$ as input and returns a suitable predicate transformer $\casem{\acom}{\acontext}$.  

The soundness of rule \ruleref{context} relies on the requirement that the context-aware semantics is compatible for the different choices of the context. 
\begin{definition}
Let $\acom \in \setcom$ and $\apredpp \in \setpreds$. We say that $\casem{\acom}{\apredpp}$ satisfies \emph{mediation} if
\begin{align}
\forall \apred, \apredppp\in\setpreds. \quad
  \casemof{\acom}{\apredpp}{\apred\mstar\apredppp}\predleq\casemof{\acom}{\apredpp\mstar\apredppp}{\apred}\mstar\apredppp\tag{Mediation}.\label{Equation:Mediation} 
\end{align}
\end{definition}
The mediation condition plays a similar role for \ruleref{context} as locality does for the frame rule.

% --------------------------------------------------------------------------------------------
% --------------------------------------------------------------------------------------------
% --------------------------------------------------------------------------------------------
% --------------------------------------------------------------------------------------------

% --------------------------------------------------------------------------------------------
% --------------------------------------------------------------------------------------------
% --------------------------------------------------------------------------------------------
% --------------------------------------------------------------------------------------------
We lift the context-aware semantics from commands to programs in the canonical way. Mediation is preserved by this lifting.
% --------------------------------------------------------------------------------------------
% --------------------------------------------------------------------------------------------
% --------------------------------------------------------------------------------------------
% --------------------------------------------------------------------------------------------
\begin{lemma}\label{Lemma:Mediation}
  \label{lem:lifting-mediation}
Let $\apredpp \in \setpreds$. If $\casem{\acom}{\apredpp}$ satisfies \eqref{Equation:Mediation} for all $\acom\in\setcom$, then so does
$\casem{\astmt}{\apredpp}$.   
\end{lemma}
% --------------------------------------------------------------------------------------------
% --------------------------------------------------------------------------------------------
% --------------------------------------------------------------------------------------------
% --------------------------------------------------------------------------------------------

We say that a correctness statement $\choareof{\acontext}{\apred}{\astmt}{\apredp}$ in context-aware separation logic is valid, denoted by $\models\choareof{\acontext}{\apred}{\astmt}{\apredp}$, 
if $\casemof{\astmt}{\acontext}{\apred}\predleq\apredp$. 
We write $\vdash\choareof{\acontext}{\apred}{\astmt}{\apredp}$ if a correctness statement can be derived using the proof rules in \cref{Figure:PL}. 
% ---------------------------------------------------------------------------
% ---------------------------------------------------------------------------
% ---------------------------------------------------------------------------
% ---------------------------------------------------------------------------
\begin{theorem}[Soundness of context-aware reasoning]
  \label{thm:casl-soundness}
Assume that $\casem{\acom}{\emp}$ satisfies \eqref{Equation:Locality} and $\casem{\acom}{\acontext}$ \eqref{Equation:Mediation} for all $\acom\in\setcom$, $\apredpp\in\setpreds$. 
Then $\vdash\choareof{\acontext}{\apred}{\astmt}{\apredp}$ implies $\models \choareof{\acontext}{\apred}{\astmt}{\apredp}$.
\end{theorem}
% ---------------------------------------------------------------------------
% ---------------------------------------------------------------------------
% ---------------------------------------------------------------------------
% ---------------------------------------------------------------------------
The theorem is an immediate consequence of soundness of the single rules. 
For the rules of Hoare logic and the frame rule, we apply \cref{Lemma:SoundnessHoare,Lemma:SoundnessFrame}.
For the new context-aware framing, soundness is an immediate consequence of \eqref{Equation:Mediation}. 
% ---------------------------------------------------------------------------
% ---------------------------------------------------------------------------
% ---------------------------------------------------------------------------
% ---------------------------------------------------------------------------
\begin{lemma}[Soundness of~\ruleref{context}]
  \label{lem:soundness-context}
Assume $\casem{\acom}{\bullet}$ satisfies \eqref{Equation:Mediation} for all $\acom\in\setcom$. 
Then $\models \choareOf{\acontext\mstar \apredppp}{\apred}{\astmt}{\apredp}$ implies $\models \choareOf{\acontext}{\apred\mstar \apredppp}{\astmt}{\apredp\mstar \apredppp}$.
\end{lemma}

\begin{proof}
We have $\casemof{\astmt}{\apredpp}{\apred\mstar\apredppp}\predleq\casemof{\astmt}{\apredpp\mstar\apredppp}{\apred}\mstar\apredppp\predleq\apredp\mstar\apredppp$. 
The former inequality is by \eqref{Equation:Mediation}, which holds due to \cref{Lemma:Mediation}.
The latter inequality is $\models \choareOf{\acontext\mstar \apredppp}{\apred}{\astmt}{\apredp}$.
\end{proof}
% ----------------------------------------------------------------------------------
% ----------------------------------------------------------------------------------
% ----------------------------------------------------------------------------------
% ----------------------------------------------------------------------------------
% ----------------------------------------------------------------------------------
% ----------------------------------------------------------------------------------
% ----------------------------------------------------------------------------------
% ----------------------------------------------------------------------------------
We note that the proof of \Cref{thm:casl-soundness} only relies on $\casem{-}{\apredpp'}$ to satisfy mediation for the contexts $\apredpp'$ occurring in the applications of rule \ruleref{context} used to derive $\vdash\choareof{\apredpp}{\apred}{\astmt}{\apredp}$. We demand the stricter requirement to avoid a side condition in the rule. However, \eqref{Equation:Mediation} can be weakened so that it is only required to hold for the contexts that are of interests for a particular proof.

\paragraph{Conservative extensions.}
We study CASLs that conservatively extend separation logics. That is, the CASL obtained from a constructed context-aware semantics should be both sound and complete relative to the separation logic induced by a given local semantics:
\begin{align*}
\forall \astmt, \apred, \apredp, \acontext. \quad & & \models \choareOf{\acontext}{\apred}{\astmt}{\apredp} \implies & \models \hoareOf{\apred \mstar \acontext}{\astmt}{\apredp \mstar \acontext} & \tag{Relative Soundness}\\
\forall \astmt, \apred, \apredp. \quad & & \vdash \hoareOf{\apred}{\astmt}{\apredp} \implies & \; \exists \acontext. \; \vdash \choareOf{\acontext}{\apred}{\astmt}{\apredp} & \tag{Relative Completeness}
\end{align*}

The canonical way to obtain a conservative extension is to let $\casem{-}{\emp}$ and $\sem{-}$ coincide.

\begin{lemma}
  \label{lem:conservative-extension}
  If $\casem{\acom}{\acontext}$ satisfies \eqref{Equation:Mediation} for all $\acom$ and $\acontext$, and $\casem{-}{\emp} = \sem{-}$, then the CASL induced by $\casem{-}{\bullet}$ conservatively extends the SL induced by $\sem{-}$.
\end{lemma}

\begin{proof}
  For relative soundness, note that $\semof{\astmt}{\apred \mstar \acontext} = \casemof{\astmt}{\emp}{\apred \mstar \acontext} \predleq \casemof{\astmt}{\acontext}{\apred}\mstar\acontext \predleq \apredp \mstar \acontext$. The first equality and the first inequality follow from the assumptions. The second inequality is the premise of relative soundness.

  Relative completeness immediately follows by a rule induction over the SL derivation that constructs a CASL derivation mimicking the SL derivation one-to-one with context $\acontext = \emp$.
\end{proof}

\paragraph{Framing under context.}
Observe that the \ruleref{frame} rule of CASL is restricted to
context-aware Hoare triples with an empty context.  This is a
deliberate design choice as allowing framing under non-empty contexts
imposes additional constraints on the considered predicates, the
underlying separation algebra, or both. For a more in-depth discussion
of a context-aware separation logic that permits a more liberal frame
rule, we direct the interested reader to \cref{app:locality}. That said,
we do not consider the restriction imposed on when to apply framing to be of
practical concern as one can always frame first before doing any
context-aware reasoning.

\subsection{From Separation Logic to Context-aware Separation Logic}
\label{sec:induced-casem}

We next present a general construction of a context-aware separation logic that conservatively extends a given separation logic. 
The question we will answer is how to make use of the fact that the predicate to be added by context-aware framing is known in order to reduce the footprint size. 
The guiding theme for the construction is this. 
Assume $\semof{\acom}{\apred}=\abort$ but $\semof{\acom}{\apred\mstar\apredpp}\predleq\apredp\mstar\apredpp\neq \abort$. 
The assumption says that some states in $\apred$ fail to be footprint, because executing $\acom$ on them also has an effect on neighboring states from~$\apredpp$. 
Moreover, these neighboring states do not change arbitrarily but again belong to $\apredpp$.
Formulated differently, although the states in $\apredpp$ are modified by the command, the predicate $\apredpp$ is invariant under these changes. 
This is the setting where context-aware semantics shows its strength.
Although $\semof{\acom}{\apred}$ has to abort as it cannot satisfy \eqref{Equation:Locality}, there is no need to let $\casemof{\acom}{\apredpp}{\apred}$ abort with \eqref{Equation:Mediation}. 
It can play the role of $\apredp$ above, in the sense that it  satisfies 
$\semof{\acom}{\apred\mstar\apredpp}\predleq\casemof{\acom}{\apredpp}{\apred}\mstar\apredpp$.
The consequence of the fact that $\casem{\acom}{\apredpp}$ does not abort is that the footprint is smaller than the footprint of $\sem{\acom}$. 

The key problem is thus to come up with a general definition of $\casem{\acom}{\apredpp}$ that captures the idea that $\casemof{\acom}{\apredpp}{\apred}$ should behave like $\apredp$ above. 
Our solution uses the septraction operator $\apred\septract\apredp$, which is the analogue of subtraction for separating conjunction. 
When $\apred$ and $\apredp$ are sets of states, then  $\apred\septract\apredp=\setcond{\astate\in\setstates}{\exists \astatep\in\apred.\ \astate\mstar\astatep\in\apredp}$, the operator thus takes states in $\apredp$ and removes the part that belongs to $\apred$. 
If $\apred=\abort$ or $\apredp=\abort$, then $\apred\septract\apredp=\abort$. 
Septraction is monotonic in both arguments.
It is also worth noting that septraction is deterministic in the following sense: if $\apred$ and $\apredp$ are singleton sets, then $\sizeof{\apred\septract\apredp}\leq 1$ due to cancellativity.  


% --------------------------------------------------------------------------------------
% --------------------------------------------------------------------------------------
% --------------------------------------------------------------------------------------
% --------------------------------------------------------------------------------------
\begin{definition}[Induced context-aware semantics]
Let $\sem{\acom}\in \setpredtrans$ be a predicate transformer.
We define the \emph{induced context-aware predicate transformer} $\icasem{\acom}{\bullet}$ by
\begin{align*}
\icasemof{\acom}{\apredpp}{\apred}\ =\ 
\begin{cases}
\apredpp\septract\semof{\acom}{\apred\mstar\apredpp} &\qquad \text{if \emph{over-aproximation} holds}, \\
\abort &\qquad\text{otherwise}. 
\end{cases}
\end{align*}
Over-approximation is the condition $\apredp\predleq (\apredpp\septract\apredp)\mstar\apredpp$ with $\apredp=\semof{\acom}{\apred\mstar\apredpp}$. 
By defining $\icasem{-}{\bullet}=\lambda\acom.\icasem{\acom}{\bullet}$, we turn a semantics $\sem{-}$ for $\setcom$ into a context-aware semantics. 
\end{definition}
% --------------------------------------------------------------------------------------
% --------------------------------------------------------------------------------------
% --------------------------------------------------------------------------------------
% --------------------------------------------------------------------------------------
The definition should be understood as follows: $\apredpp\septract\semof{\acom}{\apred\mstar\apredpp}$ is the modification that $\semof{\acom}{\apred\mstar\apredpp}$ applies to the $\apred$ part. 
The side condition of over-approximation makes sure we do not lose states when moving from the original to the induced semantics. 
To see that this is essential for relative soundness, consider a state $\astate$ that $\acom$ modifies to $\astatep$, and assume there is no $\astatepp$ so that $\astatep=\astate\mstar\astatepp$. 
Define $\apredpp=\set{\astate}$ and let $\apred=\emp$.  
If we omitted the side condition and just used $\apredpp\septract\semof{\acom}{\apred\mstar\apredpp}$ as the induced semantics, we would obtain $\set{\astate}\septract\set{\astatep}=\emptyset$. Then $\astatep\notin\emptyset\mstar\apredpp$, which is unsound.  
It may also be justified that all states live in the context, but then we see $\emp$ instead of $\emptyset$ as the post. 
Consider the same setting except that $\acom$ leaves $\astate$ unchanged. 
Over-approximation is readily checked and thus $\icasemof{\acom}{\apredpp}{\apred}=\set{\astate}\septract\set{\astate}\predleq\emp$. 
This is correct, as $\astate\in \emp\mstar\apredpp$.  

Our main result in this section is that the induced semantics satisfies the conditions in \cref{lem:conservative-extension}, and hence yields a conservative extension of separation logic.
% --------------------------------------------------------------------------------------
% --------------------------------------------------------------------------------------
% --------------------------------------------------------------------------------------
% --------------------------------------------------------------------------------------
\begin{proposition}
$\icasem{-}{\bullet}$ satisfies the conditions in \cref{lem:conservative-extension}.
\end{proposition}
% --------------------------------------------------------------------------------------
% --------------------------------------------------------------------------------------
% --------------------------------------------------------------------------------------
% --------------------------------------------------------------------------------------
We split the proof into several lemmas. 
Details missing here can be found in \cref{sec:pl-proofs}. 
The basic condition on a conservative extension is readily checked.
% --------------------------------------------------------------------------------------
% --------------------------------------------------------------------------------------
% --------------------------------------------------------------------------------------
% --------------------------------------------------------------------------------------
\begin{lemma}
$\sem{\acom}=\icasem{\acom}{\emp}$ for all $\acom\in\setcom$. 
\end{lemma}
% --------------------------------------------------------------------------------------
% --------------------------------------------------------------------------------------
% --------------------------------------------------------------------------------------
% --------------------------------------------------------------------------------------
For $\apredpp\neq\emp$, it requires a bit of work to show that the $\icasem{\acom}{\apredpp}$ are predicate transformers satisfying \eqref{Equation:Mediation}. 
The first step is to understand over-approximation better. 
Here is a characterization. 
% ---------------------------------------------------------------------------------
% ---------------------------------------------------------------------------------
% ---------------------------------------------------------------------------------
% ---------------------------------------------------------------------------------
% ---------------------------------------------------------------------------------
% ---------------------------------------------------------------------------------
% ---------------------------------------------------------------------------------
% ---------------------------------------------------------------------------------
\begin{lemma}
  \label{Lemma:OverDet}
  Let $\apredp \subseteq \Sigma$.
  We have $\apredp \predleq (\apredpp\septract \apredp) \mstar \apredpp$ iff for all $\astate \in \apredp$, $(\setstates\septract\astate)\predmeet\apredpp\neq\emptyset$.
\end{lemma}
% ---------------------------------------------------------------------------------
% ---------------------------------------------------------------------------------
% ---------------------------------------------------------------------------------
% ---------------------------------------------------------------------------------
\begin{proof}
  If $\apredpp = \top$ then the equivalence holds trivially. Thus, assume $\apredpp \neq \top$.
  \begin{asparaitem}
    \item[(``$\Rightarrow$'')]
      % $\Rightarrow$
      Assume $\astate \in \apredp$ and hence $\astate \in (\apredpp\septract\apredp)\mstar\apredpp$. 
      This means there is $\astatepp \in\apredpp$ and $\astatep$ with $\astate = \astatep \mstar \astatepp$. Therefore, we also have $\astatepp \in (\setstates \septract \astate)$.
    \item[(``$\Leftarrow$'')]
      % $\Leftarrow$
      Assume $\astatepp\in (\setstates\septract\astate)\cap\apredpp$ for some $\astate \in \apredp$. 
      Then there is a state $\astatep\in\setstates$ so that $\astatep\mstar\astatepp=\astate$.
      Hence, $\astatep=\astatepp\septract\astate \predleq \apredpp\septract\apredp$.
      Moreover, since $\astatepp\in\apredpp$, we get $\astate = \astatep\mstar\astatepp\predleq (\apredpp\septract\apredp)\mstar\apredpp$.
      \qedhere
  \end{asparaitem}
\end{proof}
% -----------------------------------------------------------
% -----------------------------------------------------------
% -----------------------------------------------------------
% -----------------------------------------------------------
The characterization immediately yields that over-approximation is distributive as follows.
% -----------------------------------------------------------
% -----------------------------------------------------------
% -----------------------------------------------------------
% -----------------------------------------------------------
\begin{lemma}\label{Lemma:OverDist}
Let $\bigpredjoin\asetpreds=\apredp\subseteq\setstates$.  
Then $\apredp\predleq(\apredpp\septract\apredp)\mstar\apredpp$ iff for all $\apred\in\asetpreds$ we have $\apred\predleq(\apredpp\septract\apred)\mstar\apredpp$. 
\end{lemma}
% -----------------------------------------------------------
% -----------------------------------------------------------
% -----------------------------------------------------------
% -----------------------------------------------------------
We will also need that over-approximation behaves well with respect to mediation. 
% -----------------------------------------------------------
% -----------------------------------------------------------
% -----------------------------------------------------------
% -----------------------------------------------------------
\begin{lemma}\label{Lemma:OverMediation}
$\apredp\predleq((\apredpp\mstar\apredppp)\septract\apredp)\mstar\apredpp\mstar\apredppp$ implies $\apredp\predleq(\apredpp\septract\apredp)\mstar\apredpp$. 
\end{lemma}
% -----------------------------------------------------------
% -----------------------------------------------------------
% -----------------------------------------------------------
% -----------------------------------------------------------
\begin{proof}
The non-trivial case is $\apredp\neq\abort$. 
Consider a state $\astate\in\apredp$. 
By the premise, there are states $\astate_1\in (\apredpp\mstar\apredppp)\septract\apredp$, $\astate_2\in\apredpp$, and $\astate_3\in\apredppp$ so that $\astate=\astate_1\mstar\astate_2\mstar\astate_3$. 
Then $\astate_1\mstar\astate_3\in \apredpp\septract\apredp$, since we can add $\astate_2\in\apredpp$ and arrive at $\astate\in\apredp$. 
Hence, $\astate= (\astate_1\mstar\astate_3)\mstar\astate_2\in (\apredpp\septract\apredp)\mstar\apredpp$ as desired.
\end{proof}
% -----------------------------------------------------------
% -----------------------------------------------------------
% -----------------------------------------------------------
% -----------------------------------------------------------
\begin{lemma}[Strictness]\label{Lemma:Strictness}
$\icasemof{\acom}{\apredpp}{\abort} = \abort$.
\end{lemma} 
% -----------------------------------------------------------
% -----------------------------------------------------------
% -----------------------------------------------------------
% -----------------------------------------------------------
\begin{lemma}[Distributivity]\label{Lemma:Distributivity}
$\icasemof{\acom}{\apredpp}{\bigpredjoin\asetpreds} = \bigpredjoin\icasemof{\acom}{\apredpp}{\asetpreds}$. 
\end{lemma}

% % -----------------------------------------------------------
% % -----------------------------------------------------------
% % -----------------------------------------------------------
% % -----------------------------------------------------------
% -----------------------------------------------------------
% -----------------------------------------------------------
% -----------------------------------------------------------
% -----------------------------------------------------------
% -----------------------------------------------------------
% -----------------------------------------------------------
% -----------------------------------------------------------
% -----------------------------------------------------------
% -----------------------------------------------------------
% -----------------------------------------------------------
% -----------------------------------------------------------
% -----------------------------------------------------------
% -----------------------------------------------------------
% -----------------------------------------------------------
% -----------------------------------------------------------
% -----------------------------------------------------------
\begin{lemma}[Mediation]
  \label{prop:icasem-mediation}
$\icasemof{\acom}{\apredpp}{\apred\mstar\apredppp}\predleq\icasemof{\acom}{\apredpp\mstar\apredppp}{\apred}\mstar\apredppp$, provided $\apredpp$ is precise.
\end{lemma}
\begin{proof}
The inequality is immediate if over-approximation fails for $\icasemof{\acom}{\apredpp\mstar\apredppp}{\apred}$, so assume it holds. 
By \cref{Lemma:OverMediation}, this implies over-approximation for 
$\icasemof{\acom}{\apredpp}{\apred\mstar\apredppp}$. 
We thus have:
\begin{align*}
&\ \icasemof{\acom}{\apredpp}{\apred\mstar\apredppp}\\
=&\ \apredpp\septract\semof{\acom}{\apred\mstar\apredppp\mstar\apredpp}\\
\explain{over-approximation + monotonicity of $\septract$}\predleq&\ \apredpp\septract(((\apredpp\mstar\apredppp)\septract\semof{\acom}{\apred\mstar\apredppp\mstar\apredpp})\mstar\apredpp\mstar\apredppp)\\
\explain{$\apredpp$ precise justifies $\apredpp\septract(\apredp\mstar\apredpp)\predleq\apredp$}\predleq&\ ((\apredpp\mstar\apredppp)\septract\semof{\acom}{\apred\mstar\apredppp\mstar\apredpp})\mstar\apredppp\\
=&\ \icasemof{\acom}{\apredpp\mstar\apredppp}{\apred}\mstar\apredppp.\qedhere
\end{align*} 
\end{proof}
% ----------------------------------------------------------------------------------
% ----------------------------------------------------------------------------------
% ----------------------------------------------------------------------------------
% ----------------------------------------------------------------------------------
It is worth noting that the proof of \cref{prop:icasem-mediation} does not need locality.
This means the induced semantics is mediating even if the underlying semantics it is defined from does not satisfy locality.
% ----------------------------------------------------------------------------------
% ----------------------------------------------------------------------------------
% ----------------------------------------------------------------------------------
% ----------------------------------------------------------------------------------
% ----------------------------------------------------------------------------------
% ----------------------------------------------------------------------------------
% ----------------------------------------------------------------------------------
% ----------------------------------------------------------------------------------

We conclude the section with a completeness result for the induced semantics. 
%
Whenever the original semantics leaves a part $\apredpp$ of the state unchanged, then $\icasem{-}{\apredpp}$ will transform the remainder of the state as desired.
The assumption we have to make is that we can uniquely identify the $\apredpp$ part. 
\begin{proposition}[Completeness of the induced context-aware semantics]\label{Proposition:CompletenessInducedSemantics}
Let $\apredpp$ be precise. 
Then $\semof{\acom}{\apred\mstar\apredpp}\predleq\apredp\mstar\apredpp$ implies $\icasemof{\acom}{\apredpp}{\apred}\predleq\apredp$
\end{proposition} 
\begin{proof}
For over-approximation, we have to show that every state in $\semof{\acom}{\apred\mstar\apredpp}$ has a substate in $\apredpp$, which holds by the premise.
We thus have
\begin{align*}
&\ \icasemof{\acom}{\apredpp}{\apred}\\
=&\ \apredpp\septract\semof{\acom}{\apred\mstar\apredpp}\\
\explain{premise}\predleq&\ \apredpp\septract(\apredp\mstar\apredpp)\\
\explain{$\apredpp$ precise}
\predleq&\ \apredp.\qedhere
\end{align*}
\end{proof}


%%% Local Variables:
%%% mode: latex
%%% TeX-master: "../main"
%%% End:
