%!TEX root = ../../main.tex

\subsection{Specification}
\label{sec:bst:spec}

To specify the operations of our BST implementation, we define the predicate $\bst{\abscontent}$ denoting a binary search tree with logical contents $\abscontent$.
With this understanding, an implementation is a binary search tree if its operations adhere to the following specification:
\begin{align*}
  \annot{\bst{\abscontent} \mstar -\infty\neq\key\neq\infty}~~\mcode{contains($\key$)}&~~\annot{\res.~\bst{\abscontent} \mstar \res\Leftrightarrow\key\in\abscontent}
  \\
  \annot{\bst{\abscontent} \mstar -\infty\neq\key\neq\infty}~~\mcode{~~insert($\key$)}&~~\annot{\res.~\bst{\abscontent\cup\set{\key}} \mstar \res\Leftrightarrow\key\notin\abscontent}
  \\
  \annot{\bst{\abscontent} \mstar -\infty\neq\key\neq\infty}~~\mcode{~~delete($\key$)}&~~\annot{\res.~\bst{\abscontent\setminus\set{\key}} \mstar \res\Leftrightarrow\key\in\abscontent}
  \\
  \annot{\bst{\abscontent}}~~\mcode{maintenance()}&~~\annot{\bst{\abscontent}}
  \ .
\end{align*}
%
In order to tie the logical contents of the specification to the physical state of our implementation, we define $\bst{\abscontent} \defeq \exists\setnodes.~\inv{\abscontent,\setnodes,\setnodes}$.
Predicate $\invraw$ is the structural invariant of our implementation:
\begin{align*}
  \inv{\abscontent, \setnodesp, \setnodes} ~=~~&
    \Root\in\setnodes \MSTAR \nullptr\notin\setnodes \MSTAR
    \setnodesp\subseteq\setnodes \MSTAR \abscontent=\ctnof{\setnodesp} \MSTAR {\bigmstar}_{\!\!\anode\in\setnodesp~~} \nodeof{\anode} \mstar \inv{\anode,\setnodes}
  \\
  \ninvp{\anode,\setnodes} ~=~~&
    \ninv{\anode,\setnodes} ~~\land~~
    \ctnof{\anode}\subseteq\ksof{\x} ~~\land~~
    \bigl(\isof{\anode}\neq\bot \implies \keyof{\anode}\in\isof{\anode}\bigr)
  \ .
\end{align*}
The invariant has two main ingredients.
First, it ties the expected logical contents $\abscontent$ to the physical contents $\ctnof{\setnodesp}$ of the region $\setnodesp$, as desired.
Second, it carries the resources $\nodeof{\anode}$ for all nodes $\anode\in\setnodesp$ and specifies their properties using the node-local invariant $\ninvp{\anode,\setnodes}$.
Predicate $\nodeof{\anode}$ boils down to a standard points-to predicate, we omit its definition.
Node invariant $\ninvp{\anode,\setnodes}$ strengthens $\ninv{\anode,\setnodes}$ from \cref{ex:estimator-motivation}. It additionally requires that
\begin{inparaenum}
  \item the contents are always a subset of the keyset, and
  \item reachable nodes receive at least their own key as inset.
\end{inparaenum}

We focus on verifying the restructuring in \code{removeSimple} (\cref{sec:bst:remove-simple}) and \code{removeComplex} (\cref{sec:bst:remove-complex}) as they have unbounded flow (ghost) updates.
The remaining operations are standard, we briefly comment on them afterwards (\cref{sec:bst:api-operations}).


%%% Local Variables:
%%% mode: latex
%%% TeX-master: "../../main"
%%% End:
