%!TEX root = ../../main.tex

\subsection{Remaining Operations}
\label{sec:bst:api-operations}

The proof for \code{find($\key$)} is given in \cref{fig:bst-impl}.
It is easy to see that the loop on \cref{code:find:loop-begin} maintains the invariant that it is still on the right track towards finding $\key$, $\key\in\isof{\x}$, it did not overshoot, $\key\neq\keyof{\x}$, and that $\y$ is the next node on the search path, $\y=\leftof{\x}$ or $\y=\rightof{\x}$ if $\key<\keyof{\x}$ or $\keyof{\x}<\key$, respectively.
The traversal terminates at the end of the search path, when $\y$ is $\pnull$ or $\key=\keyof{\y}$.

The proof for \code{delete} relies on the above properties that \code{find} establishes for the nodes $\x$ and $\y$ it returns.
If $\y$ is null, then $\key\in\ksof{\x}$ and $\key\notin\ctnof{\x}$.
The former means $\key\notin\ksof{\setnodes\setminus\set{\x}}$ by the keyset disjointness result due to \citet{DBLP:journals/tods/ShashaG88}, thus $\key\notin\ctnof{\setnodes\setminus\set{\x}}$.
Hence, $\key\notin\abscontent$ and returning $\false$ on \cref{code:delete:false} is correct.
If $\y$ is marked, the argument is similar.
Otherwise, $\y$ is unmarked and thus $\key\in\abscontent$.
Marking $\y$ on \cref{code:delete:mark} effectively removes $\key$ from $\abscontent$, because no other node contains $\key$ by keyset disjointness as before.
This justifies returning $\true$ on \cref{code:delete:true}.
Note that the update does not change the flow, so its physical and ghost footprint is just $\y$.

The proofs for \code{insert} and \code{contains} are similar.


%%% Local Variables:
%%% mode: latex
%%% TeX-master: "../../main"
%%% End:
