%!TEX root = ../../main.tex

\subsection{Implementation}

Our BST implementation is given in \cref{fig:bst-impl} (ignore the \textcolor{colorAnnotation}{annotations} for a moment).
The tree is comprised of \code{Node}s that contain pointers to their \code{left} and \code{right} subtrees, a \code{key} from $\keyspace$, and a \code{del} flag indicating whether the node is logically deleted.
We say that a node is marked if the \code{del} flag is raised, and unmarked otherwise.
The physical contents of a node $\anode$ then is $\ctnof{\anode}=\emptyset$ if $\delof{\anode}\lor\anode=\Root$ and $\ctnof{\anode}=\set{\keyof{\anode}}$ otherwise.
The shared variable $\Root$ is the entry point to the tree.
It is an unmarked sentinel node containing key $\infty$.

\input{content/example/fig_impl}

All operations rely on the helper \code{find}.
It takes a $\key\in\keyspace$ and searches it in a standard BST fashion starting in $\Root$.
The search terminates if it encounters $\pnull$ or a node containing $\key$.
It then returns the last two nodes $\x$ and $\y$ of the search path.
Note that this means $\x$ never contains~$\key$.

Operation \code{delete($\key$)} uses \code{find} to obtain the last two nodes $\x$ and $\y$ on the search path for $\key$.
If $\y$ is $\pnull$ or marked, then the tree does not logically contain $\key$ and the operation returns $\false$ (\cref{code:delete:false}).
Otherwise, $\y$ is unmarked.
In this case, \cref{code:delete:mark} marks it to purge it from the logical contents of the tree and \cref{code:delete:true} subsequently returns $\true$.
Mimicking concurrent implementations, $\y$ remains physically present in the tree.
The more involved physical removal is deferred to \code{maintenance}.

Operation \code{insert($\key$)} proceeds similarly.
It uses \code{find} to obtain the nodes $\x$ and $\y$.
If $\y$ is $\pnull$, a new node containing $\key$ is created and added as a child of $\x$, \cref{code:insert:less,code:insert:greater}.
If $\y$ is non-$null$, it is guaranteed to contain $\key$.
If it is marked, it is simply unmarked, \cref{code:insert:unmark}.
In both cases, $\key$ is successfully added to the contents of the tree and $\true$ is returned.
Otherwise, $\key$ is already present in the tree and $\false$ is returned, \cref{code:insert:false}.

Operation \code{contains($\key$)} simply returns whether the node $\y$ returned by \code{find} is non-null and unmarked.
Its correctness argument is similar to the pure cases of \code{delete} and \code{insert}.

The \code{maintenance} operation non-deterministically restructures the tree using \code{removeSimple} and \code{removeComplex}, which physically remove marked nodes.
We discuss the removals in detail in \cref{sec:bst:remove-simple,sec:bst:remove-complex}, respectively.
For a BST implementation with rotations, consult \cref{app:bst-rotate}.


%%% Local Variables:
%%% mode: latex
%%% TeX-master: "../../main"
%%% End:
