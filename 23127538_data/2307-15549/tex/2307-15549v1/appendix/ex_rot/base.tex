%!TEX root = ../../main.tex

\section{A BST with Rotations}
\label{app:bst-rot}
\label{app:bst-rotate}

We present a BST with rotations and prove it correct using contextualization.


%=============================================================================%
%=============================================================================%
\subsection{Specification}
\label{app:bst-rot:specification}

To specify the operations of our BST implementation, we define the predicate $\bst{\abscontent}$ denoting a binary search tree with logical contents $\abscontent$. (We assume a global root pointer that is left implicit.)
The logical contents are a subset of the keys, a totally ordered set $\keyspace$ that has minimal and maximal elements $-\infty$ and $\infty$, respectively.
With this understanding, an implementation is a binary search tree if its \code{contains}, \code{insert}, and \code{delete} operations adhere to the following specification:
\begin{align*}
  \annot{\bst{\abscontent} \mstar -\infty\neq\key\neq\infty}~~\mcode{contains($\key$)}&~~\annot{\res.~\bst{\abscontent} \mstar \res\Leftrightarrow\key\in\abscontent}
  \\
  \annot{\bst{\abscontent} \mstar -\infty\neq\key\neq\infty}~~\mcode{~~insert($\key$)}&~~\annot{\res.~\bst{\abscontent\cup\set{\key}} \mstar \res\Leftrightarrow\key\notin\abscontent}
  \\
  \annot{\bst{\abscontent} \mstar -\infty\neq\key\neq\infty}~~\mcode{~~delete($\key$)}&~~\annot{\res.~\bst{\abscontent\setminus\set{\key}} \mstar \res\Leftrightarrow\key\in\abscontent}\ .
\end{align*}
The specification of \code{contains} requires that the logical contents $\abscontent$ of the tree remain unchanged.
Moreover, the Boolean return value $\res$ must indicate whether or not the given $\key$ is contained in $\abscontent$.
Operation \code{insert} adds the given $\key$ to the contents of the tree.
Its return value indicates whether the $\key$ was successfully inserted ($\res=\true$) or if it was already present ($\res=\false$).
Similarly, \code{delete} removes the given $\key$ from the tree and indicates whether an actual deletion took place.
As is usual, all operations expect $-\infty\neq\key\neq\infty$; values $-\infty,\infty$ are for internal purposes.

Besides the above user-facing API, virtually all (concurrent) binary trees come with maintenance operations that restructure the tree.
Restructuring operations do not alter the logical contents but may rotate or remove nodes in order to speed up future accesses.
We assume a single \code{maintenance} operation which performs the desired restructuring periodically.
Its specification is as expected:
\begin{align*}
  \annot{\bst{\abscontent}}~~\mcode{maintenance()}&~~\annot{\bst{\abscontent}}\ .
\end{align*}
Interestingly, concurrent tree implementations tend to have fairly simple implementations for \code{insert} and \code{delete} but much more intricate \code{maintenance} operations.
Our implementation will mimic this: \code{delete} will simply mark nodes as logically deleted but does not attempt to remove them from the tree, the removal is performed later by the \code{maintenance} operation.

%=============================================================================%
%=============================================================================%
\subsection{Implementation}
\label{app:bst-rot:implementation}

Our sequential BST implementation is given in \cref{app:fig:api-impl} (ignore the \textcolor{colorAnnotation}{annotations} for a moment).
The nodes of the tree are of type \code{Node}.
They contain \code{left} and \code{right} pointers for their left and right subtrees, respectively, a \code{key} from $\keyspace$ that they represent, and a \code{del} flag indicating whether the node is logically deleted.
We say that a node is marked if the \code{del} flag is raised, and unmarked otherwise.
Additionally, \code{Node}s contain a ghost field \code{dup} that we use in our proofs to distinguish temporary duplicates that are inserted by rotations.
That it is a ghost field means that the implementation does not react on its value, only the ghost state may.
The shared variable $\Root$ is the entry point to the tree.
It is an unmarked sentinel node containing key $-\infty$.

\input{appendix/ex_rot/fig_api}

All operations of the tree rely on the helper \code{find}.
It takes a $\key\in\keyspace$ and searches it in a standard BST fashion: upon reaching a node $\anode$, the search terminates at $\anode$ if $\anode$ is $\pnull$ or if $\key$ equals $\selof{\anode}{key}$, continues to $\selof{\anode}{left}$ if $\key$ is less than $\selof{\anode}{key}$, and continues to $\selof{\anode}{right}$ if $\key$ is greater than $\selof{\anode}{key}$.
The helper then returns the last two nodes $\anode$, $\anodep$ on the search path, such that $\anode$ is guaranteed to be non-null and $\anodep$ is the potentially-null left or right child of $\anode$.
If $\anodep$ is non-null, then it is guarantee to contain $\key$.
Node $\anode$ never contains $\key$ (otherwise the search path would not extend to $\anodep$).

Operation \code{delete($\key$)} uses \code{find} to obtain nodes $\x$ and $\y$ with the above properties.
If $\y$ is null or marked, then $\key$ is not logically contained in the tree (because $\x$ does not contain $\key$) and $\false$ is returned, \cref{app:code:delete:false}.
Otherwise, $\y$ is unmarked.
In this case, \cref{app:code:delete:mark} marks it to purge it from the logical contents of the tree and \cref{app:code:delete:true} subsequently returns $\true$.
Note that $\y$ remains physically present in the tree.
The physical removal is deferred to the \code{maintenance} operation.

Operation \code{insert($\key$)} proceeds similarly.
It uses \code{find} to obtain nodes $\x$ and $\y$.
If $\y$ is null, a new node containing $\key$ is created and added as a child of $\x$, \cref{app:code:insert:less,app:code:insert:greater}.
If $\y$ is non-null, it is guaranteed to contain $\key$.
If it is marked, it is simply unmarked, \cref{app:code:insert:unmark}.
In both cases, $\key$ is successfully added to the contents of the tree and $\true$ is returned.
Otherwise, $\y$ is unmarked, i.e., $\key$ is already present in the tree.
Hence, the ongoing insertion fails and returns $\false$, \cref{app:code:insert:false}.

Operation \code{contains($\key$)} simply returns whether the node $\y$ returned by \code{find} is non-null and unmarked.
The arguments for \code{delete}/\code{insert} already cover why this is an appropriate result.

The \code{maintenance} operation non-deterministically invokes the helper functions \code{removeSimple}, \code{removeComplex}, and \code{rotate}.
Operations \code{removeSimple} and \code{removeComplex} perform the physical removal (unlinking) of marked nodes.
We discuss them in detail in \cref{app:bst-rot:remove-simple,app:bst-rot:remove-complex}, respectively.
Operation \code{rotate} performs standard right rotations.
We discuss it in detail in \cref{app:bst-rot:rotation}.
We ignore left rotations, they are symmetrical.


%=============================================================================%
%=============================================================================%
\subsection{Proof Methodology}
\label{app:bst-rot:invariant}

Towards verifying our implementation against the BST specification, we first develop the predicate $\bst{\abscontent}$.
We define it by \[ \bst{\abscontent} \defeq \exists\setnodes.~\inv{\abscontent,\setnodes} \ .\]
Predicate $\invraw$ is the structural invariant of our implementation that ties its physical state to the logical contents $\abscontent$ of the specification.
We use the flow framework to devise $\invraw$.

\smartparagraph{Flow Domain}
As the underlying flow monoid, we choose
\[
  \bigl( \powerset{\keyspace}\uplus\set{\bot,\top},\, \oplus,\, \bot \bigr)
  \qquad\text{with}\qquad
  \amonval\oplus\amonvalp ~\defeq~ \begin{cases}
    \amonval &\text{if }~ \amonvalp=\bot \\
    \amonvalp &\text{if }~ \amonval=\bot \\
    \top &\text{otherwise}
  \ .
  \end{cases}
\]
The flow values propagated by this flow are sets of keys $\amonval,\amonvalp\subseteq\keyspace$, or dedicated sentinel values $\bot,\top$.
Intuitively, if $\akey\in\amonval$ for the flow $\amonval$ of a node $\anode$, then \code{find} will traverse $\anode$ when searching for $\akey$, as alluded to in \cref{app:bst-rot:implementation}.
Value $\bot$ denotes that a node is unreachable from $\Root$.
Note that $\bot$ and $\emptyset$ differ: $\emptyset$ means that the node is still reachable from $\Root$, but \code{find} will not traverse it.
Value $\top$ denotes that a node has multiple reachable parents, that is, the heap graph is not a tree.
To establish this intuition, $\bot$ is neutral with respect to $\oplus$ and in all other cases $\oplus$ yields $\top$.

\smartparagraph{Physical State}
The physical state is comprised of a collection of nodes (of type \code{Node}).
We capture the resources associated with such nodes $\anode$ in a predicate $\nodeof{\anode}$.
To simplify the notation, assume that proofs are existentially closed.
This way, we can use the naming convention $f(\anode)$ to refer to the logical variable that holds the value of field $\selof{\anode}{f}$.
Then define: \[
  \nodeof{\anode} ~\defeq~ \begin{aligned}[t]
    &\selof{\anode}{left}\pto\leftof{\anode} \MSTAR
    \selof{\anode}{right}\pto\rightof{\anode} \MSTAR
    \selof{\anode}{key}\pto\keyof{\anode}  \\\!\!\!\MSTAR
    &\selof{\anode}{del}\pto\delof{\anode} \MSTAR
    \selof{\anode}{in}\pto\inof{\anode} \MSTAR
    \selof{\anode}{dup}\pto\dupof{\anode}\enspace.
  \end{aligned}
\]
Fields $\leftof{\anode}$, $\rightof{\anode}$, $\keyof{\anode}$, and $\delof{\anode}$ are as expected, they capture the left child, right child, key, and deletion flag of node $\anode$, respectively.
They give rise to the physical contents of node $\anode$:
\[
  \ctnof{\anode} ~\defeq~ \begin{cases}
    \set{\keyof{\anode}} &\text{if } \neg\delof{\anode} \lor \anode\neq\Root \\
    \emptyset &\text{otherwise}
  \ .
  \end{cases}
\]
Field $\inof{\anode}$ is the inflow of $\anode$.
Most of the time we are interested in the \emph{inset} of $\anode$, which we define as \[\isof{\anode} ~\defeq~ \bigoplus_{\anodep \in \setnodes} \inof{\anode}(\anodep) \ .\]
Field $\dupof{\anode}$ is the ghost field used for rotations.
Its possible values are $\dupvalnone$ (not a duplicate), $\dupvalleft$ (duplicate due to a left rotation), and $\dupvalright$ (duplicate due to a right rotation).


\smartparagraph{Edge Functions}
We derive edge functions from the physical representation of nodes.
That is, the edge functions $\edges$ of every flow graph are induced by the physical state of its nodes.
Intuitively, the edge functions filter the incoming flow values (search paths) according to the BST search principle from \cref{app:bst-rot:implementation}.
Formally, we define:
\begin{align*}
  \edgesatof{\anode}{\anodep}{\amonval} ~=~ \begin{cases}
    \top  &\text{if }~ \leftof{\anode}=\anodep=\rightof{\anode} \\
    \amonval\cap[-\infty,\keyof{\anode})  &\text{if }~ \anodep=\leftof{\anode} ~\land~ \dupof{\anode}\neq\dupvalleft \\
    \amonval\cap(\keyof{\anode},\infty]  &\text{if }~ \anodep=\rightof{\anode} ~\land~ \dupof{\anode}\neq\dupvalright \\
    \bot  &\text{otherwise}
  \end{cases}
\end{align*}
where we use $\bot\cap\amonval=\bot$ and $\top\cap\amonval=\top$.
The first case handles edges from a node $\anode$ to its left child $\anodep$.
The edge forwards the portion of the given flow value $\amonval$ that is smaller than $\anode$'s key.
The side condition $\dupof{\anode}\neq\dupvalleft$ prevents the edge function from forwarding flow if $\anode$ is a duplicate inserted by a left rotation. This is needed because $\anode$'s left child already receives flow from the node that $\anode$ duplicates.
Similarly, the second case forwards the portion of $\amonval$ that is larger than $\anode$'s key to its right child, provided $\anode$ is not a duplicate during a right rotation.
In all other cases, the edge function produces $\bot$.

We refer to the \emph{left outset} of a node $\anode$ as the quantity produced by the edge function $\edgesat{\anode}{\leftof{\anode}}$ for the inflow of $\anode$.
Formally, this is $\oslof{\anode} = \edgesatof{\anode}{\leftof{\anode}}{\isof{\anode}}$ if $\leftof{\anode}\neq\nullptr$ and $\oslof{\anode}=\emptyset$ otherwise.
The \emph{right outset} $\osrof{\anode}$ is defined correspondingly.
Subtracting $\anode$'s outsets from its inset yields the keys for which \code{find} terminates in $\anode$.
That is, these are the keys that could be in $\anode$ while still satisfying the BST order property for the remaining graph.
We refer to this quantity as the \emph{keyset} of $\anode$: \[
  \ksof{\anode} ~=~ \begin{cases}
    \emptyset  &\text{if }~ \isof{\anode}\in\set{\bot,\top} \\
    \isof{\anode} \setminus \bigl( \oslof{\anode} \cup \osrof{\anode} \bigr)  &\text{otherwise}
  \ .
  \end{cases}
\]

\smartparagraph{Invariant}
The structural invariant of our BST implementation is given by a predicate $\inv{\abscontent,\setnodesp,\setnodes}$, which denotes a subregion $\setnodesp\subseteq\setnodes$ of the entire structure $\setnodes$ with logical contents $\abscontent$.
The invariant carries the resources for the nodes in $\setnodesp$ and describes their properties:
\begin{align*}
  \inv{\abscontent, \setnodesp, \setnodes} ~=~~&
    \Root\in\setnodes \MSTAR \nullptr\notin\setnodes \MSTAR
    \setnodesp\subseteq\setnodes \MSTAR \abscontent=\ctnof{\setnodesp} \MSTAR \bigmstar_{\anode\in\setnodesp} \nodeof{\anode} \mstar \inv{\anode,\setnodes}
  \\
  \inv{\anode,\setnodes} ~=~~&
    \set{\leftof{\anode},\rightof{\anode}}\subseteq\setnodes\uplus\set{\nullptr} ~~\land~~
    \dupof{\anode}=\dupvalnone ~~\land~~
    \\&
    \isof{\anode} \neq \top ~~\land~~
    \ctnof{\anode}\subseteq\ksof{\anode} ~~\land~~
    \bigl(\isof{\anode}\neq\bot \implies \keyof{\anode}\in\isof{\anode}\bigr) ~~\land~~
    \\&\bigl(\anode = \Root \implies \isof{\anode} = [-\infty,\infty] \land \neg\delof{\anode} \land \keyof{\anode}=-\infty\bigr)
\end{align*}
The invariant has two main ingredients.
First, it ties the expected logical contents $\abscontent$ to the physical contents $\ctnof{\setnodesp}$ of the region $\setnodesp$, as desired.
Second, it carries the resources $\nodeof{\anode}$ for all nodes $\anode\in\setnodesp$ from the region and specifies their properties using the node-local invariant $\inv{\anode,\setnodes}$.
The node-$\anode$-local invariant requires that
\begin{inparaenum}
  \item the overall structure $\setnodes$ is self-contained, i.e., $\anode$ cannot reach nodes outside of $\setnodes$,
  \item $\anode$ is not a duplicate, i.e., duplicates are inserted only temporarily during rotation,
  \item $\anode$ has at most one parent node that is reachable from $\Root$ (up to temporary duplicates), i.e., at most one node sends flow to $\anode$,
  \item the physical contents of $\anode$ are contained in its keyset, i.e., the search paths for $\anode$'s contents reach and terminate in $\anode$,
  \item if $\anode$ has inflow, then it receives at least its own key, and
  \item $\Root$ is a sentinel node with key $-\infty$, it is never marked as deleted, and it is the entry point for all searches.
\end{inparaenum}
For brevity, we write $\inv{\abscontent,\setnodes}$ to mean $\inv{\abscontent,\setnodes,\setnodes}$.

\smartparagraph{Compositionality}
When framing or contextualizing a subregion of $\setnodes$, we employ the following compositionality of the invariant:
\begin{align*}
  \inv{\abscontent,\setnodesp_1\uplus\setnodesp_2,\setnodes}
  ~~\iff~~
  \exists\abscontent_1,\abscontent_2.~
  \inv{\abscontent_1,\setnodesp_1,\setnodes} \mstar
  \inv{\abscontent_2,\setnodesp_2,\setnodes} \mstar
  \abscontent=\abscontent_1\cup\abscontent_2\enspace.
  %
  \tag{\textsc{comp}}
  \label{app:weak-decomposition}
\end{align*}
That is, we decompose the invariant into two disjoint regions and recompose them.
However, the decomposition does not localize the reasoning because it does not localize the logical contents.
The invariant alone does not guarantee that the contents $\abscontent_1$ and $\abscontent_2$ are disjoint, despite $\setnodesp_1$ and $\setnodesp_2$ being disjoint.
To overcome this, we strengthen the decomposition by requiring $\setnodes = \setnodesp_1 \cup \setnodesp_2$:
\begin{align*}
  \begin{aligned}
    \inv{\abscontent,\setnodes,\setnodes} \mstar
    \setnodes=\setnodesp_1\uplus\setnodesp_2
    ~~\implies~~
    \exists\abscontent_1,\abscontent_2.
    \begin{aligned}[t]
      &\inv{\abscontent_1,\setnodesp_1,\setnodes} \mstar
      \inv{\abscontent_2,\setnodesp_2,\setnodes} \mstar
      \\{}\mstar{}
      &\ksof{\setnodesp_1}\cap\ksof{\setnodesp_2}=\emptyset \mstar
      \abscontent=\abscontent_1\uplus\abscontent_2\enspace.
    \end{aligned}
  \end{aligned}
  %
  \tag{\textsc{decomp}}
  \label{app:strong-decomposition}
\end{align*}
The implication only allows one to decompose the entire tree, but in return establishes that the keysets of the decomposed regions are disjoint.
From this, we conclude that $\abscontent_1$ and $\abscontent_2$ are disjoint, because the invariant guarantees $\abscontent_1\subseteq\ksof{\setnodesp_1}$ and $\abscontent_2\subseteq\ksof{\setnodesp_2}$.
That the keysets of $\setnodesp_1$ and $\setnodesp_2$ are disjoint follows from a result due to \citet{DBLP:journals/tods/ShashaG88}.
Translated to flows, the result requires that
\begin{inparaenum}
  \item all edge functions are decreasing, i.e, $\edgesatof{\anode}{\anodep}{\amonval}\leq\amonval$,
  \item the left and right outsets are disjoint, $\oslof{\anode}\cap\osrof{\anode}=\emptyset$, and
  \item only the root node receives inflow from outside the full graph.
\end{inparaenum}
These properties are ensured by $\invraw$ and the definition of the edge functions. However, the result only applies if $\invraw$ is satisfied by the entire graph $\setnodes$.
Hence, we apply the decomposition only if we are splitting the full graph.


%=============================================================================%
%=============================================================================%
\subsection{Verifying the Implementation}

We now show that the implementation from \cref{app:fig:api-impl} implements a BST along \cref{app:bst-rot:specification}.
Proof outlines for \code{find} and \code{delete} are given in form of \textcolor{colorAnnotation}{colored annotations} in \cref{app:fig:api-impl}.

The proof for \code{find($\key$)} follows our intuition from \cref{app:bst-rot:implementation}: the traversal, \crefrange{app:code:find:loop-begin}{app:code:find:loop-end}, \emph{goes with the flow} to locate the search key, $\key$.
It maintains the invariant that it is still on right track, $\key\in\isof{\x}$, $\x$ does not contain the search key, $\key\neq\keyof{\x}$, and that $\y$ is the next node on the search path, $\y=\leftof{\x}$ or $\y=\rightof{\x}$ if $\key<\keyof{\x}$ or $\keyof{\x}<\key$, respectively.
The traversal terminates at the end of the search path, if $\y$ is null ($\y\notin\setnodes$) or if $\key=\keyof{\y}$.

The proof for \code{delete} relies on the above properties that \code{find} establishes for the nodes $\x$ and $\y$ it returns.
There are three cases.
\begin{asparaenum}[(1)]
  \item
    If $\y$ is null, then we have $\key\in\ksof{\x}$ because $\key\in\isof{\x}$ and there is no outset to null.
    Moreover, $\key\notin\ctnof{\x}$ because $\key\neq\keyof{\x}$.
    We now use the keyset disjointness of the strong decomposition \eqref{app:strong-decomposition} to conclude that $\x$ is the only node that has $\key$ in its keyset, $\key\notin\ksof{\setnodes\setminus\set{\x}}$.
    Recall that $\ctnof{\setnodes\setminus\set{\x}}\subseteq\ksof{\setnodes\setminus\set{\x}}$ holds by the invariant.
    Hence, $\key$ is not contained in the tree, $\key\notin\abscontent$.
    This justifies returning $\false$ on \cref{app:code:delete:false}.
  \item
    If $\y$ is non-null and marked, we know that $\key$ flows from $\x$ to $\y$, $\key\in\isof{\y}$.
    That the search terminates in $\y$ means $\key=\keyof{\y}$.
    Together, $\key\in\ksof{\y}$.
    That $\y$ is marked means $\ctnof{\y}=\emptyset$.
    With a similar argument as before, we conclude $\key\notin\abscontent$.
    Returning $\false$ on \cref{app:code:delete:false} is again correct.
  \item
    If $\y$ is non-null and unmarked, the reasoning is similar.
    We have $\key\in\ksof{\y}\cap\ctnof{\y}$ and $\key\notin\ctnof{\setnodes\setminus\set{\y}}$.
    From \eqref{app:strong-decomposition} we get that $\key$ is stored exclusively in $\y$, $\abscontent=\ctnof{\y}\uplus\ctnof{\setnodes\setminus\set{\y}}$.
    Hence, marking $\y$ on \cref{app:code:delete:mark} effectively removes $\key$ from $\abscontent$.
    That is, after \cref{app:code:delete:mark} the state of the tree is $\inv{\abscontent\setminus\set{\key},\setnodes,\setnodes}$.
    This justifies returning $\true$ on \cref{app:code:delete:true}.
    Note that the update does not change the flow, so its physical and ghost footprint is just $\y$.
\end{asparaenum}

The proofs for \code{insert} and \code{contains} follow a similar pattern.
In the remainder of this section, we focus on \code{removeSimple}, \code{removeComplex}, and \code{rotate}.


%=============================================================================%
%=============================================================================%
\subsection{Simple Removal}
\label{app:bst-rot:remove-simple}

The maintenance operation \code{removeSimple} unlinks nodes from the tree that have been marked as deleted.
It is a ``simple'' removal because it unlinks nodes only if they have at most one child.
We expect \code{removeSimple} to leave unchanged the logical contents of the tree.
Concretely, we establish the following specification for it, as imposed by the specification of \code{maintenance}: \[
  \annot{
    \inv{\abscontent,\setnodes}
  }
  ~~
  \mcode{removeSimple()}
  ~~
  \annot{
    \inv{\abscontent,\setnodes}
  }
  \ .
\]

\input{appendix/ex_rot/fig_simple}

The implementation and proof outline of \code{removeSimple} are given in \cref{app:fig:simple-removal:impl}.
It proceeds as follows.
Starting from some arbitrary node $\x$ currently linked into the tree, the left child $\y$ of $\x$ is read out.
If $\y$ is a non-null marked node with at most one child, \code{removeSimple} tries to unlink $\y$.
In the case where $\y$ has no left child, $\y$ is unlinked by making $\y$'s right child the left child of $\x$, i.e. by updating $\selof{\x}{left}$ to $\rightof{\y}$.
The state of the tree before and after the unlinking is depicted in \cref{app:fig:simple-removal:prestate,app:fig:simple-removal:poststate} (ignore the annotations for a moment).
The case where $\y$ has no right child is symmetric.
Similarly, removing the marked right child of $\x$, \cref{app:code:simple-remove:symmetric}, is symmetric.
We omit the symmetric cases.

The interesting part of the proof is the moment where $\y$ is unlinked.
The precondition of the unlinking is on \cref{app:code:simple-remove:pre}.
It states that the tree currently represents the set $\abscontent$ and satisfies the invariant, $\inv{\abscontent,\setnodes}$.
Moreover, the precondition captures our above intuition, stating that $\y$ is the marked left child of $\x$, $\delof{\y} \mstar \leftof{\x}=\y$, and has no left child itself, $\leftof{\y}=\nullptr$.
The proof goal is to establish $\inv{\abscontent,\setnodes}$ on \cref{app:code:simple-remove:post}, that is, show the unlinking of $\y$ on \cref{app:code:simple-remove:unlink} maintains both the logical contents and the invariant.
The main challenge with this update is its unbounded footprint.
To see this, consider \cref{app:fig:simple-removal:prestate}.
It depicts the part of the tree rooted in $\x$ prior to the update.
Because $\x$ is linked into the tree, it has non-$\bot$ inflow, say $\ais\defeq\isof{\x}$.
Node $\y$ receives the portion of $\ais$ that is smaller than $\x$' key, $\isof{\y}=\oslof{\x}=\ais\cap[-\infty,\keyof{x})$.
From that inset, $\y$ forwards the portion that is larger than its key to is right child, subtree $A$ in the \cref{app:fig:simple-removal:prestate}.
Overall, this means that the inset of $A$ before unlinking $\y$ is $\ais\cap(\keyof{\y},\keyof{\x})$.
After unlinking $\y$, i.e., making subtree $A$ the left child of $\x$ as depicted in \cref{app:fig:simple-removal:poststate}, $A$ receives all the flow $\ais\cap[-\infty,\keyof{x})$ that $\y$ used to receive, not just the $\keyof{\y}$-larger portion of it.
That is, the unlinking results in subtree $A$ receiving the additional flow $\aks\defeq\ais\cap[-\infty,\keyof{\y}]$.
To see why this makes the footprint of unlinking $\y$ unbounded, observe that the nodes in $A$ all have keys larger than $\y$.
That is, the additional flow $\aks$ is forwarded to the left-most leaf in $A$.
Because $A$ can be an arbitrary tree, the footprint is unbounded.

To overcome the unbounded footprint, we contextualize subtree $A$ (and the right child of $\x$ which is irrelevant here).
To be precise, we decompose the tree described by $\inv{\abscontent,\setnodes}$ into the footprint $\inv{\abscontent_2,\set{\x,\y},\setnodes}$, consisting of nodes $\x$ and $\y$, and the context $\inv{\abscontent_1,\setnodes\setminus\set{\x,\y},\setnodes}$, consisting of the remainder of the tree, with $\abscontent=\abscontent_1\cup\abscontent_2$.
In \cref{app:fig:simple-removal:impl}, the application of rule \ruleref{context} is between \cref{app:code:simple-remove:context,app:code:simple-remove:post-footprint}: \cref{app:code:simple-remove:context} states the \colorbg{backgroundContext}{context} and the following lines reason about the \colorbg{backgroundFootprint}{footprint}.

Because the update changes the inflow of the context $\inv{\abscontent_1,\setnodes\setminus\set{\x,\y},\setnodes}$, we have to show that it tolerates the additional inflow it receives after the update.
Towards this, we capture the change in inflow using the estimator $\simplerel$ defined by: \[
  \amonval\simplerel\amonvalp
  ~~\defifff~~
  \amonval=\amonvalp
  \,~\lor~
  \bigl(\,\setc{\amonval,\amonvalp}\cap\setc{\bot,\top}=\emptyset
  ~\land~
  \amonval\subseteq\amonvalp\,\bigr)
  \ .
\]
This reflects our intuition that flow values may increase, but it prevents previously unreachable nodes (those that have a flow of $\bot$) to receive flow and it also prevents nodes from receiving flow from more sources than before (the flow increase does not result in a flow of $\top$).
The relation satisfies the requirements for an estimator.
Note that the actual update produces a flow increase of at most $\aks$ in the context, however, our proof does not rely on this upper bound.
It is readily checked that the context $\inv{\abscontent_1,\setnodes\setminus\set{\x,\y},\setnodes}$ is $\simplerel$-closed, because the keyset of a node only increases, unreachable nodes (flow of $\bot$) remain unreachable, and no node's inflow becomes $\top$.
That $\simplerel$ satisfies the requirements of an estimator is left as an exercise to the reader.

We turn to the actual update on \cref{app:code:simple-remove:unlink} within the footprint $\inv{\abscontent_2,\set{\x,\y},\setnodes}$.
The physical update unlinking $\y$ is standard.
Moreover, it is easy to see that the invariant for $\x$ is maintained.
To see that the invariant for $\y$ is maintained as well, observe that $\delof{\y}$ on \cref{app:code:simple-remove:pre-footprint} means $\y\neq\Root$ and $\ctnof{\y}=\emptyset$.
This is the annotation on \cref{app:code:simple-remove:post-footprint}, and it is $\simplerel$-closed following the same arguments as for the context.
Lastly, it remains to show that the footprint's outflow after the update is $\simplerel$-larger than prior to the update.
We already discussed that $\y$'s outflow to its right child is $\ais\cap(\keyof{\y},\keyof{\x})$ before the update and $\ais\cap[-\infty,\keyof{x})$ after it.
Because $\ais=\isof{\x}\neq\bot$, we have the desired $\ais\cap(\keyof{\y},\keyof{\x}) \simplerel \ais\cap[-\infty,\keyof{x})$.
(The outflow at $\x$'s right child does not change and $\simplerel$ is reflexive).
This concludes the proof of \code{removeSimple} along the lines of the proof strategy from \cref{sec:instantiation}.


%=============================================================================%
%=============================================================================%
\subsection{Complex Removal}
\label{app:bst-rot:remove-complex}

Operation \code{removeComplex} unlinks marked nodes from the tree that have two children and are thus not handled by \code{removeSimple}.
The specification is as expected: \[
  \annot{
    \inv{\abscontent,\setnodes}
  }
  ~~
  \mcode{removeComplex()}
  ~~
  \annot{
    \inv{\abscontent,\setnodes}
  }
  \ .
\]

\input{appendix/ex_rot/fig_complex}

\Cref{app:fig:complex-removal} gives the implementation and proof outline.
There are four steps in \code{removeComplex}.
First, it obtains an arbitrary marked node $\x$ that is reachable from $\Root$ and has two children.
Second, it uses the helper function \code{findSucc} from \cref{app:fig:find-succ} to obtain the left-most leaf $\y$ and its parent $\p$ in the subtree $B$ rooted at $\x$'s right child.\footnote{%
  To avoid distracting case distinctions, \code{findSucc} assumes $\p\neq\x$.
  We omit the case $\p=\x$ because it is much simpler: its flow update is not unbounded, it affect only the nodes $\x$ and $\y$.
}
That $\y$ is the left-most leaf in $B$ means that $\keyof{\y}$ is the next larger key after $\keyof{\x}$ in $B$.
Consequently, all search paths for keys from $(\keyof{\x},\keyof{\y})$ that reach $\x$ continue to $B$ and eventually reach $\y$.
Moreover, they terminate in $\y$ because $\y$ has no left child.
Formally, these search paths are for keys $\aks=\isof{\x}\cap(\keyof{\x},\keyof{\y})$ and they are part of $\y$'s keyset, $\aks\subseteq\ksof{\y}$.
This is the annotation on \cref{app:code:complex-remove:pre}, depicted in \cref{app:fig:complex-removal:prestate}.
Third, the contents of $\x$ and $\y$ are swapped, \crefrange{code:complex-remove:move}{code:complex-remove:del-y}.
This is the most interesting part of the proof and we discuss it in detail below.
Last, $\y$ is unlinked.
The procedure, involving $\y$ and its parent $\p$, is the same as the unlinking in \code{removeSimple}.
We will not reiterate it.

A detailed discussion of the third step, swapping the contents of $\x$ and $\y$, is in order.
To avoid confusion between the values of fields before and after the following updates, we record the values $\akey_\x=\keyof{\x}$, $\akey_\y=\keyof{\y}$, and $\ais=\isof{\x}$ from before the update, as on \cref{app:code:complex-remove:pre}.
Note that $\akey_\x<\akey_\y$.
Now, \cref{app:code:complex-remove:move} copies $\akey_\y$ into $\x$.
This is challenging due to its intricate flow update, which is visualized in \cref{app:fig:complex-removal:prestate,fig:complex-removal:poststate}.
The update 
increases the inflow of $\x$'s left subtree $A$ from $\ais\cap[-\infty,\akey_\x)$ to $\ais\cap[-\infty,\akey_\y)$.
The inflow of $\x$'s right subtree $B$, in turn, decreases from $\ais\cap(\akey_\x,\infty]$ to $\ais\cap(\akey_\y,\infty]$.
That is, the portion $\aks=\ais\cap(\akey_\x,\akey_\y]$ is redirected from $B$ to $A$.
We capture this change of inflow with the estimator $\complexrel$ defined by: \[
  \amonval\complexrel\amonvalp
  ~~\defifff~~
    \amonval\simplerel\amonvalp
    \,~\lor~
    \bigl(\,
      \setc{\amonval,\amonvalp}\cap\setc{\bot,\top}=\emptyset ~\land~ \akey_\x\notin\amonval ~\land~ \amonval\setminus\aks\subseteq\amonvalp
    \,\bigr)
  \ .
\]
The relation allows the inflow to increase arbitrarily.
Moreover, it allows the inflow to decrease by up to $\aks$.
However, decreasing the inflow may only occur for inflows that do not contain $\akey_\x$.
The side condition localizes the decrease to the subtrees of $\x$, because $\akey_\x\in\isof{\x}$.
As we will see, it prevents us from \emph{loosing} $\akey_\y$ from the inflow of $\x$ when performing the update.

For handling the update, we choose nodes $\x$ and $\y$ as the footprint and contextualize everything else.
Technically, the context is $\inv{\abscontent_1,\setnodes\setminus\set{\x,\y},\setnodes}$ and the footprint is $\inv{\abscontent_2,\set{\x,\y},\setnodes}$ with $\abscontent=\abscontent_1\cup\abscontent_2$.
We now show that the context is $\complexrel$-closed, which is significantly more involved than showing $\simplerel$-closedness for \code{removeSimple}.
Consider a contextualized node $\z\in\setnodes\setminus\set{\x,\y}$ with $\isof{\y}\neq\bot$.
The invariant states $\keyof{\z}\in\isof{\z}$.
To preserve this inclusion, despite reducing the inflow by up to $\aks$, requires $\keyof{\z}\notin\aks$.
To see this, observe that $\keyof{\z}\in\ksof{\z}$ follows from $\keyof{\z}\in\isof{\z}$ prior to the update.
Using \eqref{app:strong-decomposition} for the above context-footprint decomposition, yields $\ksof{\z}\cap\ksof{\y}=\emptyset$.
Because we already argued for $\aks\subseteq\ksof{\y}$, we get $\keyof{\z}\notin\aks$ as desired.
This line of reasoning also implies that the $\ctnof{\z}\subseteq\ksof{\z}$ part of the invariant is preserved.
Note that $\Root$ does not loose flow due to $\akey_\x\in\isof{\Root}$.
Overall, we conclude that the context $\inv{\abscontent_1,\setnodes\setminus\set{\x,\y},\setnodes}$ on \cref{app:code:complex-remove:ctx} is indeed $\complexrel$-closed.

We turn to the footprint $\inv{\abscontent_2,\set{\x,\y},\setnodes}$, \cref{app:code:complex-remove:footprint-pre}.
The physical update changing $\selof{\x}{key}$ from $\akey_\x$ to $\akey_\y$ is as expected.
It remains to discuss how it affects the flow of $\x$ and $\y$.
\Cref{thm:contextualization} prescribes that the footprint be $\complexrel$-closed after the update.
Because $\akey_\x,\akey_\y\in\ais$ prior to the update, $\complexrel$ guarantees that $\akey_\y$ remains in the inset of $\x$ after the update.
That is, we have $\inv{\emptyset,\set{\x},\setnodes}$ after the update, because $\x$ is still marked.
The inflow of $\y$, in turn, may decrease by up to $\aks$.
This is expected because $\y$'s inflow is solely due to $\x$'s outflow.
Note that relation $\complexrel$ does not remove \emph{exactly} $\aks$---we simply do not know the exact loss in inflow, and we do not care.
As a consequence, the invariant of $\y$ breaks, because it no longer receive its key $\akey_\y$ as inflow.
The subsequent actions will re-establish the invariant for $\y$.
Overall, we arrive at the postcondition on \cref{app:code:complex-remove:footprint-post-move}.

Next, \code{removeComplex} finalizes swapping the contents of $\x$ and $\y$ by swapping their \code{del} flags, \cref{app:code:complex-remove:del-x,app:code:complex-remove:del-y}.
This results in $\ctnof{\x}$ being updated to $\abscontent_2$.
Because $\abscontent_2\subseteq\set{\akey_\y}\subseteq\ksof{\x}$, we obtain the invariant $\inv{\abscontent_2,\set{\x},\setnodes}$.
The content of $\y$, on the other hand, is deleted, $\ctnof{\y}=\emptyset$.
This is the annotation on \cref{app:code:complex-remove:footprint-post}.
Note that these updates do not change the flow.
At this point, we can recompose $\x$ with context and obtain $\inv{\abscontent,\setnodes\setminus\set{\y},\setnodes}$, \cref{app:code:complex-remove:intermediate}.
It remains to incorporate $\y$.

Finally, \cref{app:code:complex-remove:unlink} unlinks $\y$.
As stated earlier, we do not present the details of the unlinking here as it is similar to \code{removeSimple}.
Overall, the unlinking results in the inflow of $\y$ to be $\bot$.
Together with $\delof{\y}$, this reestablished the invariant, $\inv{\emptyset,\set{\y},\setnodes}$.
Recomposing it with $\inv{\abscontent,\setnodes\setminus\set{\y},\setnodes}$, we arrive at the desired $\inv{\abscontent,\setnodes}$, \cref{app:code:complex-remove:post}.


%=============================================================================%
%=============================================================================%
\subsection{Rotations}
\label{app:bst-rot:rotation}

%!TEX root = ../../main.tex
%%% Local Variables:
%%% mode: latex
%%% TeX-master: "../main"
%%% End:

% Figure environment removed
  \end{minipage}
  \hfill
  \begin{minipage}[b]{.475\textwidth}
  % Figure environment removed\vspace{-3mm}
  % Figure environment removed
  \end{minipage}
\end{figure}


Operation \code{rotate} performs a standard right rotation.
We expect it not to change the logical contents of the tree.
The specification is: \[
  \annot{
    \bst{\abscontent}
  }
  ~~
  \mcode{rotate()}
  ~~
  \annot{
    \bst{\abscontent}
  }
  \ .
\]
Unlike for the previous maintenance operations, we use $\bst{\abscontent}$ instead of $\inv{\abscontent,\setnodes}$.
This is because \code{rotate} creates a new node to the heap graph that we \emph{hide} in the existential quantifier of $\bst{\abscontent}$.

\input{appendix/ex_rot/fig_rot_heap}

\Cref{app:fig:rotate} gives the implementation and proof outline for \code{rotate}.
Given nodes $\y$ and $\z$ with $\selof{\y}{left}=\z$, the goal is to move $\y$ into $\z$'s right subtree.
In line with concurrent BST implementations, \code{rotate} does not perform this update in-place, but inserts a duplicate $\dup$ of $\y$ in $\z$'s right subtree and subsequently unlinks $\y$.
This breaks the tree shape temporarily, until $\y$ is unlinked.
To handle this in the proofs we set $\dup$'s \code{dup} field to $\dupvalright$ before inserting it and then to $\dupvalnone$ at the moment when $\y$ is unlinked.
Node $\y$ is unlinked by replacing in $\y$'s parent $\x$ the child pointer to $\y$ by $\z$.
Consult \cref{app:fig:rotate-fig} for an illustration.
The figure shows that all updates are contained within the footprint $\set{\x,\y,\z,\dup}$---the subtrees $A,B,C$ are not aware of the changes.
In particular, there is no unbounded flow update.
Hence, \code{rotate} can be verified using standard arguments.
In particular, the \ruleref{frame} rule is applicable.


%%% Local Variables:
%%% mode: latex
%%% TeX-master: "../../main"
%%% End:
