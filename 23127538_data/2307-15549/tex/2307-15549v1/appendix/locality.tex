%!TEX root = ../main.tex

\section{Framing under Contexts}
\label{app:locality}

In this section, we study a variant of CASL that allows framing under non-trivial contexts. That is, we replace the rule \ruleref{frame} in \cref{Figure:PL} with the following more liberal rule:
\begin{mathpar}		
  \inferH{frame-context}{
    \choareHighOf{\highlight{\acontext}}{\apred}{\astmt}{\apredp}
  }{
    \choareHighOf{\highlight{\acontext}}{\apred\mstar\apredppp}{\astmt}{\apredp\mstar\apredppp}
  }		
\end{mathpar}

To guarantee soundness of the resulting logic, we need to require that $\casem{-}{\acontext}$ satisfies \eqref{Equation:Locality} for the considered contexts $\acontext$ under which the new frame rule is applied. We first study conditions that imply \eqref{Equation:Locality} of our induced context-aware semantics from \cref{sec:induced-casem}.

\begin{lemma}[Locality]
Assume $\sem{\acom}$ satisfies \eqref{Equation:Locality}, $\apred\mstar\apredp\predleq\apred$, and $\apredpp$ and $\apredp$ satisfy $\apredpp\septract(\apredppp\mstar\apredp)\predleq(\apredpp\septract\apredppp)\mstar\apredp$ for all $\apredppp$. 
Then 
\begin{align*} 
\icasemof{\acom}{\apredpp}{\apred\mstar\apredp}\predleq\icasemof{\acom}{\apredpp}{\apred}\mstar\apredp.
\end{align*}  
\end{lemma}
\begin{proof}
We have that $\overapproxof{\acom}{\apredpp}{\apred}$ implies $\overapproxof{\acom}{\apredpp}{\apred\mstar\apredp}$ by $\apred\mstar\apredp\predleq\apred$. 
This means if $\overapproxof{\acom}{\apredpp}{\apred\mstar\apredp}$ fails, we have $\icasemof{\acom}{\apredpp}{\apred\mstar\apredp}=\abort =\icasemof{\acom}{\apredpp}{\apred}$. 
If $\overapproxof{\acom}{\apredpp}{\apred\mstar\apredp}$ holds, we argue as follows: 
\begin{align*}
&\ \icasemof{\acom}{\apredpp}{\apred\mstar\apredp}\\
\explain{$\overapproxof{\acom}{\apredpp}{\apred\mstar\apredp}$ + definition}=&\ \apredpp\septract\semof{\acom}{\apred\mstar\apredp\mstar\apredpp}\\
\explain{Locality of $\sem{\acom}$}\predleq&\ \apredpp\septract(\semof{\acom}{\apred\mstar\apredpp}\mstar\apredp)\\
\explain{Assumption}\predleq&\ (\apredpp\septract\semof{\acom}{\apred\mstar\apredpp})\mstar\apredp\\
\explain{$\overapproxof{\acom}{\apredpp}{\apred}$ + definition}\predleq&\ \icasemof{\acom}{\apredpp}{\apred}\mstar\apredp.\qedhere
\end{align*}
\end{proof}

Hence, for \eqref{Equation:Locality} of $\icasem{-}{\bullet}$, we have to impose assumptions. 
We now give conditions under which they hold. 
The assumption $\apred\mstar\apredp\predleq\apred$ is immediately satisfied if $\apred$ is an intuitionistic predicate in the sense that $\apred\mstar\setstates\predleq \apred$. 
Distributivity of separating conjunction over septraction, $\apredpp\septract(\apredppp\mstar\apredp)\predleq(\apredpp\septract \apredppp)\mstar\apredp$ seems to be new.
It holds whenever (i) $\apredpp$ and $\apredp$ only have units as common substates, and (ii) the underlying separation algebra satisfies the following factorization property:
\[
  \label{eq:factorization}
  \arraycolsep=2pt
  \forall \astate, \astatep, \astatepp, \astateppp \in \setstates. \; \astate \mstar \astatep = \astatepp \mstar \astateppp \;\;\Rightarrow\;\; \exists \astate_{\astatepp}, \astate_{\astateppp}, \astatep_{\astatepp}, \astatep_{\astateppp}. \begin{aligned}[t]
    \astate_\astatepp \mstar \astatep_\astatepp = \astatepp &\;\land\; \astate_\astateppp \mstar \astatep_\astateppp = \astateppp \\ {} \land\; {}
    \astate_\astatepp \mstar \astate_\astateppp = \astate &\;\land\; \astatep_\astatepp \mstar \astatep_\astateppp = \astatep
\end{aligned}
\tag{Factorization} \]
This property holds, e.g., for separation algebras that admit unique factorization such as the standard heap model. However, it also holds for other commonly used models such as fractional permissions.

% -----------------------------------------------------------------------------------
% -----------------------------------------------------------------------------------
% -----------------------------------------------------------------------------------
% -----------------------------------------------------------------------------------
\begin{lemma}\label{Lemma:Septraction}
If $(\setstates, \mstar, \emp)$ satisfies \eqref{eq:factorization} and $(\setstates\septract\apredp)\sqcap (\setstates\septract\apredpp)\predleq\emp$ then $\apredpp\septract(\apredppp\mstar\apredp)\predleq (\apredpp\septract\apredppp)\mstar\apredp$.
\end{lemma}
\begin{proof}
If $\apredp=\abort$, $\apredpp=\abort$, or $\apredppp=\abort$, the inequality is immediate, so assume all are sets of states and let $\astate\in \apredpp\septract(\apredppp\mstar\apredp)$.
This means there is $\astatep\in\apredpp$ so that $\astate\mstar\astatep\in\apredppp\mstar\apredp$. 
Then there are states $\astate_{\apredppp}\in\apredppp$ and $\astate_{\apredp}\in \apredp$ so that $\astate\mstar\astatep=\astate_{\apredppp}\mstar\astate_{\apredp}$. 
The equality and \eqref{eq:factorization} yields decompositions $\astate=\astate_1\mstar\astate_2$ and $\astatep=\astatep_1\mstar\astatep_2$ so that $\astate_1\mstar\astatep_1=\astate_{\apredppp}$ and $\astate_2\mstar\astatep_2=\astate_{\apredp}$. 
Now $\astatep_2\in \setstates\septract\apredp$, as one can add $\astate_2$ to arrive at $\astate_{\apredp}\in\apredp$. 
Moreover, $\astatep_2\in \setstates\septract\apredpp$, as one can add $\astatep_1$ to arrive at $\astatep\in\apredpp$. 
The assumption now yields $\astatep_2\in\emp$. 
As a consequence, $\astate_2=\astate_{\apredp}\in\apredp$. 
Moreover, $\astatep_1=\astatep\in\apredpp$ and so $\astate_1\in \apredpp\septract \apredppp$ since $\astate_1\mstar\astatep_1=\astate_{\apredppp}\in\apredppp$.  
Hence, $\astate=\astate_1\mstar\astate_2\in (\apredpp\septract \apredppp)\mstar\apredp$ as desired. 
\end{proof}

\iffalse
One may wonder whether there are alternative context-aware semantics that impose fewer constraints. A natural candidate is the \emph{best local action} semantics of~\cite{}, which yields local predicate transformer by construction. Applied to our setting, this yields the following definition:

\subsubsection*{Completeness of CASL for invariant reasoning}

Assume that for all $\acom \in \setcom$, $\sem{\acom}$ satisfies \eqref{Equation:Locality}. Moreover, suppose we have proved that some predicate $\invpred \neq \emp$ is an invariant of all commands $\acom$. That is, for each $\acom \in \setcom$ there exist predicates $\aprecond$ and $\apostcond$ such that
\[ \models \hoareOf{\aprecond \mstar \invpred}{\acom}{\apostcond \mstar \invpred} \enspace.\]
Our goal is to define a relatively sound context-aware semantics so that we can use rule \ruleref{context} to frame the invariant $\invpred$ when reasoning about composite programs $\astmt$. When reasoning about individual commands $\acom$,  we then want to use the valid Hoare triples $\hoareOf{\aprecond \mstar \invpred}{\acom}{\apostcond \mstar \invpred}$ to discharge the premise of the rule \ruleref{com}. This style of reasoning is akin to the use of resource invariants in concurrent separation logics. The difference is that instead of using this as a mechanism for transferring ownership of shared resources between concurrently executing threads, we here use it to obtain more local proofs by reducing the footprint of the program.

To enable the above reasoning, we need a context-aware semantics $\casem{-}{\bullet}$ such that (i) $\models \hoareof{\apred}{\astmt}{\apredp}$ and $\models \choareof{\emp}{\apred}{\astmt}{\apredp}$ coincide, and (ii) $\casem{\acom}{\emp}$ satisfies \eqref{Equation:Mediation} for all $\acom$. Relative soundness then follows immediately.

The following definition will do:
\[ \casemof{\acom}{\apredpp}{\apred} \defeq \begin{cases}
    \semof{\acom}{a} & \text{if } \apredpp = \emp\\
    \bigpredjoin_{\apred' \predleq \apred} \bigpredmeet \setcond{\apostcond \mstar \apredp}{\apred' \predleq \aprecond \mstar \apredp} & \text{if } \apredpp = \invpred\\
    \top & \text{otherwise}\enspace.
    \end{cases}
\]

\begin{lemma}
  \label{lem:invariant-casem-mediating}
  $\casem{\acom}{\apredpp}$ is strict, a complete join morphism, and local for all $\acom \in \setcom$ and $\apredpp \in \setpreds$. Moreover, $\casem{\acom}{\emp}$ satisfies \eqref{Equation:Mediation}.
\end{lemma}

\begin{proof}
  For $\casem{\acom}{\emp}$ the desired properties other than mediation are inherited from $\sem{\acom}$.   To show that $\casem{\acom}{\emp}$ satisfies mediation, let $\apredppp \in \setpreds$. We focus on the case where $\apredppp = \invpred$ as the other cases are immediate:
  \begin{align*}
      & \; \casemof{\acom}{\emp}{\apred \mstar \invpred} \\
\explain{Definition} = & \; \semof{\acom}{\apred \mstar \invpred}\\
    = & \; \semof{\acom}{(\bigjoin_{\apred' \predleq \apred} \apred') \mstar \invpred}\\
\explain{Distributivity of $\mstar$ and $\predjoin$}  = & \; \semof{\acom}{\bigjoin_{\apred' \predleq \apred} (\apred' \mstar \invpred)}\\
\explain{$\sem{\acom}$ distributes over $\predjoin$} = & \; \bigpredjoin_{\apred' \predleq \apred}  \semof{\acom}{\apred' \mstar \invpred}\\
    \predleq & \; \bigpredjoin_{\apred' \predleq \apred} \bigpredmeet \setcond{\semof{\acom}{\aprecond \mstar \apredp \mstar \invpred}}{{\apred' \predleq \aprecond \mstar \apredp}}\\
\explain{Locality of $\sem{\acom}$}  \predleq & \; \bigpredjoin_{\apred' \predleq \apred} \bigpredmeet \setcond{\semof{\acom}{\aprecond \mstar \invpred} \mstar \apredp}{\apred' \predleq \aprecond \mstar \apredp} \\
\explain{Assumption}  \predleq & \; \bigpredjoin_{\apred' \predleq \apred} \bigpredmeet \setcond{\apostcond \mstar \invpred \mstar \apredp}{\apred' \predleq \aprecond \mstar \apredp} \\
\explain{$\invpred$ precise}  \predleq & \; \bigpredjoin_{\apred' \predleq \apred} \bigpredmeet \setcond{\apostcond \mstar \apredp}{\apred' \predleq \aprecond \mstar \apredp} \mstar \invpred \\
\explain{Distributivity of $\mstar$ and $\predjoin$}  = & \; \left(\bigpredjoin_{\apred' \predleq \apred} \bigpredmeet \setcond{\apostcond \mstar \apredp}{\apred' \predleq \aprecond \mstar \apredp} \right) \mstar \invpred \\
\explain{Definition} = & \; \casemof{\acom}{\invpred}{\apred} \mstar \invpred \\
= & \; \casemof{\acom}{(\emp \mstar \invpred)}{\apred} \mstar \invpred \enspace.
  \end{align*} 

  For $\casem{\acom}{\invpred}$ strictness and distributivity are immediate. For locality of $\casem{\acom}{\invpred}$ observe that
  \begin{align*}
      & \; \casem{\acom}{\invpred}(\apred \mstar \apredppp)\\
    = & \; \bigpredjoin_{\apred' \predleq \apred \mstar \apredppp} \bigpredmeet \setcond{\apostcond \mstar \apredp}{\apred'' \predleq \aprecond \mstar \apredp}\\
\explain{See below}    \predleq & \; \bigpredjoin_{\apred' \predleq \apred} \bigpredmeet \setcond{\apostcond \mstar (\apredp' \mstar \apredppp)}{\apred'' \predleq \aprecond \mstar \apredp'}\\
\explain{Distributivity of $\mstar$ with $\predjoin$, $\apredppp$ precise} \predleq & \; \left(\bigpredjoin_{\apred' \predleq \apred}\bigpredmeet \setcond{\apostcond \mstar \apredp'}{\apred' \predleq \aprecond \mstar \apredp'}\right)  \mstar \apredppp\\
\explain{Definition} = & \; \casem{\acom}{\invpred}(\apred) \mstar \apredppp \enspace. 
\end{align*}
To see the first inequality, suppose $\apred' \predleq \apred \mstar \apredppp$. Define $\apred'' = \setcond{\astate}{\astate \mstar \astate_\apredppp \in \apred' \land \astate \in \apred \land \astate_\apredppp \in \apredppp}$. Then $\apred'' \predleq \apred$ by definition. Now suppose $\apred'' \predleq \aprecond \mstar \apredp'$ for some $\apredp'$. Then $\apred'' \mstar \apredppp \predleq \aprecond \mstar \apredp' \mstar \apredppp$ by monotonicity of $\mstar$. However, we also have $\apred' \predleq \apred'' \mstar \apredppp$ by the assumption on $\apred'$ and the definition of $\apred''$. It follows that $\apred' \predleq \aprecond \mstar (\apredp' \mstar \apredppp)$.


For all remaining cases, the properties immediately follow from the fact that $\casemof{\acom}{\apredpp}{\apred} = \top$ for all $\apred$ and $\acom$.
  
\end{proof}
\fi

%%% Local Variables:
%%% mode: latex
%%% TeX-master: "../main"
%%% End:
