%!TEX root = ../main.tex
%%% Local Variables:
%%% mode: latex
%%% TeX-master: "../main"
%%% End:

\subsection{Proofs for Additional Meta Theory}


% -----------------------------------------------------------------------------
% -----------------------------------------------------------------------------
% -----------------------------------------------------------------------------
% -----------------------------------------------------------------------------
\begin{proof}[Proof of \Cref{thm:our-monoid-has-bot}]
	We show that $\bot=\monunit$ is the least element in $\amonoid$.
	Consider some $m\in\amonoid$.
	We show $\monunit \leq m$.
	By definition, we require some $x\in\amonoid$ such that $\monunit+x=m$.
	Choosing $x=m$ satisfies the requirement.
\end{proof}


% -----------------------------------------------------------------------------
% -----------------------------------------------------------------------------
% -----------------------------------------------------------------------------
% -----------------------------------------------------------------------------
\begin{proof}[Proof of \Cref{thm:our-monoid-has-addprop}]
	By definition, $\amonval\leq\amonvalp$ means $\amonval+\delta=\amonvalp$ for some $\delta\in\amonoid$.
	As a consequence, we have $\amonval+\delta+\amonvalpp=\amonvalp+\amonvalpp$.
	By definition again, $\amonval+\amonvalpp\leq\amonvalp+\amonvalpp$.
	This concludes the first claim.
	
	With this, we obtain $\amonval+\amonval'\leq\amonvalp+\amonval'$ and $\amonval'+\amonvalp\leq\amonvalp'+\amonvalp$.
	Together, this means $\amonval+\amonval'\leq\amonvalp+\amonvalp'$.
	This concludes the second claim.
\end{proof}


% -----------------------------------------------------------------------------
% -----------------------------------------------------------------------------
% -----------------------------------------------------------------------------
% -----------------------------------------------------------------------------
\begin{proof}[Proof of \Cref{thm:our-monoid-sum-vs-join}]
	Consider two chains $\achain=\amonval_0\leq\amonval_1\leq\cdots$ and $\achainp=\amonvalp_0\leq\amonvalp_1\leq\cdots$.
	We show $(\bigjoin\achain)+(\bigjoin\achainp)=\bigjoin(\achain+\achainp)$.
	The overall claim then follows from repeatedly applying the above argument.
	Because $\amonoid$ is continuous, we have:
	\begin{align*}
		(\bigjoin\achain)+(\bigjoin\achainp)
		~=~
		\bigjoin\setcond{\amonval_i + \bigjoin\achainp}{i\in\nat}
		~=~
		\bigjoin\setcond{\bigjoin\setcond{\amonval_i + \amonvalp_j}{j\in\nat}}{i\in\nat}
		\ .
	\end{align*}
	First, observe that $\amonval_i+\amonvalp_i\leq\amonval_i+\amonvalp_j$ for all $i\in\nat$ and all $j\geq i$ by \Cref{thm:our-monoid-has-addprop}.
	This means $\amonval_i+\amonvalp_i\leq\bigjoin\setcond{\amonval_i + \amonvalp_j}{j\in\nat}$ for all $i\in\nat$.
	Hence, we get:
	\begin{align*}
		\bigjoin\setcond{\amonval_i + \amonvalp_i}{i\in\nat}
		~~\leq~~
		\bigjoin\setcond{\bigjoin\setcond{\amonval_i + \amonvalp_j}{j\in\nat}}{i\in\nat}
		\ .
	\end{align*}
	Second, observe that we have $\amonval_i+\amonvalp_j\leq\amonval_k+\amonvalp_k$ with $k=\max(i,j)$ for all $i,j\in\nat$.
	This means that for every $i\in\nat$ there is some $k\in\nat$ such that $\bigjoin\setcond{\amonval_i+\amonvalp_j}{j\in\nat}\leq\amonval_k+\amonvalp_k$.
	Hence, $\bigjoin\setcond{\amonval_i+\amonvalp_j}{j\in\nat}\leq\bigjoin\setcond{\amonval_k+\amonvalp_k}{k\in\nat}$.
	Then we get:
	\begin{align*}
		\bigjoin\setcond{\bigjoin\setcond{\amonval_i + \amonvalp_j}{j\in\nat}}{i\in\nat}
		~~\leq~~&
		\bigjoin\setcond{\bigjoin\setcond{\amonval_k + \amonvalp_k}{k\in\nat}}{i\in\nat}
		\\~~=~~&
		\bigjoin\setcond{\amonval_k + \amonvalp_k}{k\in\nat}
		\ .
	\end{align*}
	Note here that the quality holds because, for every $\amonvalpp\in\amonoid$, the sequence $\amonvalpp\leq\amonvalpp\leq\cdots$ is an $\leq$-ascending chain the join of which exists and is $\amonvalpp$.
	Altogether, we arrive at:
	\begin{align*}
		\bigjoin\setcond{\bigjoin\setcond{\amonval_i + \amonvalp_j}{j\in\nat}}{i\in\nat}
		~~=~~
		\bigjoin\setcond{\amonval_i + \amonvalp_i}{i\in\nat}
		~~=~~
		\bigjoin(\achain+\achainp)
		\ .
	\end{align*}
	This concludes $(\bigjoin\achain)+(\bigjoin\achainp) = \bigjoin(\achain+\achainp)$, as desired.
\end{proof}


% -----------------------------------------------------------------------------
% -----------------------------------------------------------------------------
% -----------------------------------------------------------------------------
% -----------------------------------------------------------------------------
\begin{proof}[Proof of \Cref{thm:add-relations}]
	Consider $\amonval,\amonval',\amonvalp,\amonvalp'\in\amonoid$ with $\amonval\fprel\amonvalp$ and $\amonval'\fprel\amonvalp'$.
	By assumption, we have $\amonval+\amonval'\fprel\amonvalp+\amonval'$.
	Also by assumption, we have $\amonval'+\amonvalp\fprel\amonvalp'+\amonvalp$.
	Hence, $\amonval+\amonval'\fprel\amonvalp+\amonvalp'$ follows from $\fprel$ being transitive by \eqref{def:fpcompat:refl-trans} of $\fpcompatible[\fprel]{\aflowconstraint}$.
\end{proof}


% -----------------------------------------------------------------------------
% -----------------------------------------------------------------------------
% -----------------------------------------------------------------------------
% -----------------------------------------------------------------------------
\begin{proof}[Proof of \Cref{thm:continuous-implies-monotonic-special}]
	% Let $f \in \contfunof{\amonoid\to\amonoid}$.
	Consider some $\amonval,\amonvalp\in \amonoid$ with $\amonval \leq \amonvalp$.
	Because $(\amonoid,\leq)$ is an $\omega$-cpo by assumption, the join $\amonval \join \amonvalp$ exists.
	The join is $\amonval \join \amonvalp = \amonvalp$.
	Consequently, $f$ is defined for $\amonval \join \amonvalp$, $f(\amonval \join \amonvalp)\in\amonoid$.
	We obtain $f(\amonval \join \amonvalp)=f(\amonval) \join f(\amonvalp)$ because $f$ is $\leq$-continuous.
	This means the join $f(\amonval) \join f(\amonvalp)$ must exist as well, $(f(\amonval) \join f(\amonvalp))\in\amonoid$.
	Altogether, we conclude the first claim as follows: \(
		f(\amonvalp) = f(\amonval \join \amonvalp) = f(\amonval) \join f(\amonvalp) \geq f(\amonval)
	\).
	The second claim follows analogously.
\end{proof}


% -----------------------------------------------------------------------------
% -----------------------------------------------------------------------------
% -----------------------------------------------------------------------------
% -----------------------------------------------------------------------------
\begin{proof}[Proof of \Cref{thm:fixpoint-kleene}]
	First, we show $f^i(\bot)\leq f^{i+1}(\bot)$ for all $i\in\nat$.
	We proceed by induction.
	In the base case, $f^0(\bot)=\bot\leq f^1(\bot)$ because $\bot$ is the least element in $(\amonoid,\leq)$.
	For the induction step, we have $f^i(\bot)\leq f^{i+1}(\bot)$.
	By \Cref{thm:continuous-implies-monotonic-special}, $f$ is $\leq$-monotonic.
	Hence, we have the following by induction: $f^{i+1}(\bot)=f(f^i(\bot))\leq f(f^{i+1}(\bot))=f^{i+2}(\bot)$.
	This concludes the induction and means that $\achain = f^0(\bot)\leq f^1(\bot)\leq\cdots$ is an $\leq$-ascending chain, as desired.
	Because $(\amonoid,\leq)$ is an $\omega$-cpo, the join $\bigjoin\achain$ exists in $\amonoid$.

	\medskip
	It remains to show that $\bigjoin\achain=\lfpof{f}$ holds.
	We observe that $\bigjoin\achain$ is a fixed point of $f$:
	\begin{align*}
		&f(\bigjoin\achain)
		\\\explain{f $\leq$-continuous}=&~~
		\bigjoin\setcond{f^{i+1}(\bot)}{i\in\nat}
		\\\explain{$\bot$ least element}=&~~
		\bigjoin\setcond{f^{i}(\bot)}{i\in\nat}
		\\\explain{Def. $\achain$}=&~~
		\bigjoin\achain
	\end{align*}
	\medskip\newcommand{\dd}{\cval^\dagger}%
	We now show that $\bigjoin\achain$ is the least fixed point of $f$.
	To that end, consider another fixed point $\dd$ of $f$, i.e., $f(\dd)=\dd$.
	It suffices to show that $f^i(\bot) \leq \dd$ holds for all $i$, because this implies that the join over the $f^i(\bot)$ is at most $\dd$.
	We proceed by induction.
	In the base case, $f^0(\bot)=\bot\leq\dd$ because $\bot$ is the least element.
	For the induction step, we have $f^i(\bot) \leq \dd$.
	Because $f$ is $\leq$-monotonic as noted earlier, we obtain $f(f^i(\bot)) \leq f(\dd)$.
	Since $\dd$ is a fixed point of $f$, this means $f^{i+1}(\bot)\leq\dd$.
	This concludes the induction.
	We arrive at the desired $\lfpof{f}=\bigjoin\achain$.
\end{proof}


% -----------------------------------------------------------------------------
% -----------------------------------------------------------------------------
% -----------------------------------------------------------------------------
% -----------------------------------------------------------------------------
\begin{proof}[Proof of \Cref{thm:flow-iteration-monotonic-continuous}]
	If $f$ is $\leq$-continuous, then it is also $\leq$-monotonic by \Cref{thm:continuous-implies-monotonic-special}.
	So, it suffices to show that $f$ is $\leq$-continuous.
	Consider an $\leq$-ascending chain $\cval_0\leq\cval_1\leq\cdots$ with $\cval_i:\setnodes\to\amonoid$.
	We show $f(\bigjoin\setcond{\cval_i}{i\in\nat})=\bigjoin\setcond{f(\cval_i)}{i\in\nat}$.
	To that end, consider $\anode\in\setnodes$.
	We have:
	\begin{align*}
		&f(\bigjoin\setcond{\cval_i}{i\in\nat})(\anode)
		\\\explain{Def. $f$}=~~&
		\sum_{\anodep\in\nat\setminus\setnodes}\inflow(\anodep,\anode) + \sum_{\anodep\in\setnodes}\edgesatof{\anodep}{\anode}{(\bigjoin\setcond{\cval_i}{i\in\nat})(\anodep)}
		\\\explain{point-wise Def. $\join$}=~~&
		\sum_{\anodep\in\nat\setminus\setnodes}\inflow(\anodep,\anode) + \sum_{\anodep\in\setnodes}\edgesatof{\anodep}{\anode}{\bigjoin\setcond{\cval_i(\anodep)}{i\in\nat}}
		\\\explain{$\edgesat{\anodep}{\anode}$ continuous}=~~&
		\sum_{\anodep\in\nat\setminus\setnodes}\inflow(\anodep,\anode) + \sum_{\anodep\in\setnodes}\bigjoin\setcond{\edgesatof{\anodep}{\anode}{\cval_i(\anodep)}}{i\in\nat}
		\\\explain{\Cref{thm:our-monoid-sum-vs-join}}=~~&
		\sum_{\anodep\in\nat\setminus\setnodes}\inflow(\anodep,\anode) + \bigjoin\setcond{\sum_{\anodep\in\setnodes}\edgesatof{\anodep}{\anode}{\cval_i(\anodep)}}{i\in\nat}
		\\\explain{$\amonoid$ continuous}=~~&
		\bigjoin\setcond{\sum_{\anodep\in\nat\setminus\setnodes}\inflow(\anodep,\anode) + \sum_{\anodep\in\setnodes}\edgesatof{\anodep}{\anode}{\cval_i(\anodep)}}{i\in\nat}
		\\\explain{Def. $f$}=~~&
		\bigjoin\setcond{f(\cval_i)}{i\in\nat}
	\end{align*}
	This establishes that $f$ is $\leq$-continuous, as desired.
	
	Now, assume $\fpcompatible[\fprel]{\aflowconstraint}$.
	It remains to show that $f$ is $\fprel$-monotonic.
	To that end, consider $\cval_1,\cval_2:\setnodes\to\amonoid$ with $\cval_1\fpreldot\cval_2$.
	We show that $f(\cval_1)\fpreldot f(\cval_2)$ holds.
	For all $\anode\in\setnodes$ we have:
	\begin{align*}
		&f(\cval_1)(\anode)
		\\\explain{Def. $f$}=~~&
		\sum_{\anodep\in\nat\setminus\setnodes}\inflow(\anodep,\anode) + \sum_{\anodep\in\setnodes}\edgesatof{\anodep}{\anode}{\cval_1(\anodep)}
		\\\explain{see below}\fprel~~&
		\sum_{\anodep\in\nat\setminus\setnodes}\inflow(\anodep,\anode) + \sum_{\anodep\in\setnodes}\edgesatof{\anodep}{\anode}{\cval_2(\anodep)}
		\\\explain{Def. $f$}=~~&
		f(\cval_2)(\anode)
	\end{align*}
	where the approximation holds because we have $\edgesatof{\anodep}{\anode}{\cval_1(\anodep)}\fprel\edgesatof{\anodep}{\anode}{\cval_2(\anodep)}$ by \eqref{def:fpcompat:edges} of $\fpcompatible[\fprel]{\aflowconstraint}$, for all $\anodep\in\setnodes$, and thus the approximation is preserved under addition by \eqref{def:fpcompat:add} of $\fpcompatible[\fprel]{\aflowconstraint}$ together with \Cref{thm:add-relations}.
	In case the sums are empty, the approximation follows from $\monunit=\bot$ by \Cref{thm:our-monoid-has-bot} and $\bot\fprel\bot$ by \eqref{def:fpcompat:refl-trans} of $\fpcompatible[\fprel]{\aflowconstraint}$.
\end{proof}


% -----------------------------------------------------------------------------
% -----------------------------------------------------------------------------
% -----------------------------------------------------------------------------
% -----------------------------------------------------------------------------
\begin{proof}[Proof of \Cref{thm:flow-as-lfp}]
	Follows from \Cref{thm:flow-iteration-monotonic-continuous} together with \Cref{thm:fixpoint-kleene}.
\end{proof}


% -----------------------------------------------------------------------------
% -----------------------------------------------------------------------------
% -----------------------------------------------------------------------------
% -----------------------------------------------------------------------------
\begin{proof}[Proof of \Cref{thm:restriction-vs-statemult}]
	Proven in \cite[Proof of Lemma~4]{DBLP:conf/tacas/MeyerWW23,DBLP:journals/corr/abs-2304-04886}.
\end{proof}


% -----------------------------------------------------------------------------
% -----------------------------------------------------------------------------
% -----------------------------------------------------------------------------
% -----------------------------------------------------------------------------
\begin{proof}[Proof of \Cref{thm:outflow-vs-statemult}]
	By choice of $\anode,\anodep$ and the definition of the composition $\statemult$, we obtain $(\aflowconstraint_1\statemult\aflowconstraint_2).\outflow(\anode,\anodep)=\aflowconstraint_1.\outflow(\anode,\anodep)$.
	We immediately obtain $\aflowconstraint_1.\outflow(\anode,\anodep)=\aflowconstraint_1.\edgesatof{\anode}{\anodep}{\aflowconstraint_1.\fval(\anode)}$ from the definition of the outflow.
	This concludes the proof.
\end{proof}


% -----------------------------------------------------------------------------
% -----------------------------------------------------------------------------
% -----------------------------------------------------------------------------
% -----------------------------------------------------------------------------
\begin{proof}[Proof of \Cref{thm:transformer-vs-statemult}]
	Consider some node $\anodepp\in\nat\setminus(\aflowconstraint_1.\setnodes\cup\aflowconstraint_2.\setnodes)$.
	The we have:
	\begin{align*}
		&\transformerofof{\aflowconstraint_1\statemult\aflowconstraint_2}{(\aflowconstraint_1\statemult\aflowconstraint_2).\inflow}(\anodepp)
		\\\explain{Def. $\transformerof{\dontcare}$}=~~&
		\sum_{\anode\in\aflowconstraint_1.\setnodes\cup\aflowconstraint_2.\setnodes} (\aflowconstraint_1\statemult\aflowconstraint_2).\outflow(\anode,\anodepp)
		\\\explain{by set theory}=~~&
		\sum_{\anode\in\aflowconstraint_1.\setnodes} (\aflowconstraint_1\statemult\aflowconstraint_2).\outflow(\anode,\anodepp)
		+\sum_{\anodep\in\aflowconstraint_2.\setnodes} (\aflowconstraint_1\statemult\aflowconstraint_2).\outflow(\anodep,\anodepp)
		\\\explain{by \Cref{thm:outflow-vs-statemult}}=~~&
		\sum_{\anode\in\aflowconstraint_1.\setnodes} \aflowconstraint_1.\outflow(\anode,\anodepp)
		+\sum_{\anodep\in\aflowconstraint_2.\setnodes} \aflowconstraint_2.\outflow(\anodep,\anodepp)
		\\\explain{Def. $\transformerof{\dontcare}$}=~~&
		\transformerofof{\aflowconstraint_1}{\aflowconstraint_1.\inflow}(\anodepp)
		+\transformerofof{\aflowconstraint_2}{\aflowconstraint_2.\inflow}(\anodepp)
	\end{align*}
	This concludes the proof.
\end{proof}


% -----------------------------------------------------------------------------
% -----------------------------------------------------------------------------
% -----------------------------------------------------------------------------
% -----------------------------------------------------------------------------
\begin{proof}[Proof of \Cref{thm:inflow-leq}(i)]
	Let $\aflowconstraint=(\setnodes,\edges,\dontcare)$.
	Define $\aflowconstraint_1=\aflowconstraint[\inflow\mapsto\inflow_1]$ and $\aflowconstraint_2=\aflowconstraint[\inflow\mapsto\inflow_2]$.
	By \Cref{thm:flow-as-lfp}, the flow in $\aflowconstraint_i$ is $\aflowconstraint_i.\fval=\bigjoin\setcond{f_i^j(\bot)}{j\in\nat}$ with $f_i:(\setnodes{\to}\amonoid)\to(\setnodes{\to}\amonoid)$ defined by: \[
		f_i(\cval)(\anode) ~~=~~ \sum_{\anodep\in\nat\setminus\setnodes}\inflow_i(\anodep,\anode) + \sum_{\anodep\in\setnodes}\edgesatof{\anodep}{\anode}{\cval(\anodep)}
		\ .
	\]
	To conclude, it suffices to show that $f_1^j(\bot) \leq f_2^j(\bot)$ holds for all $j\in\nat$.
	We proceed by induction.
	In the base case, $\inflow_1\leq\inflow_2$ together with \Cref{thm:our-monoid-has-addprop} gives $f_1^0(\bot) \leq f_2^0(\bot)$ by the definition of $f_1,f_2$.
	For the induction step, we show $f_1(f_1^j(\bot)) \leq f_2(f_2^j(\bot)$.
	By induction together with $\leq$-monotonicity of $f_1$ by \Cref{thm:flow-iteration-monotonic-continuous}, we have $f_1(f_1^j(\bot)) \leq f_1(f_2^j(\bot))$.
	It remains to show $f_1(f_2^j(\bot)) \leq f_2(f_2^j(\bot))$.
	This immediately follows from the fact that $\inflow_1\leq\inflow_2$ together with \Cref{thm:our-monoid-has-addprop}, similarly to the base case.
\end{proof}

\begin{proof}[Proof of \Cref{thm:inflow-leq}(ii)]
	Let $\aflowconstraint=(\setnodes,\edges,\dontcare)$.
	Define $\aflowconstraint_1=\aflowconstraint[\inflow\mapsto\inflow_1]$ and $\aflowconstraint_2=\aflowconstraint[\inflow\mapsto\inflow_2]$.
	Let $\anodep\in\nat\setminus\setnodes$.
	We have $\transformerofof{\aflowconstraint}{\inflow_i}(\anodep)=\sum_{\anode\in\setnodes}\edgesatof{\anode}{\anodep}{\aflowconstraint_i.\fval(\anode)}$ by the definition of $\transformerof{\dontcare}$ and the outflow.
	Let $\anode\in\setnodes$.
	Part (i) of this \namecref{thm:inflow-leq} gives $\aflowconstraint_1.\fval(\anode)\leq\aflowconstraint_2.\fval(\anode)$.
	Then, we get $\edgesatof{\anode}{\anodep}{\aflowconstraint_1.\fval(\anode)}\leq\edgesatof{\anode}{\anodep}{\aflowconstraint_2.\fval(\anode)}$, because $\edges$ is $\leq$-continuous and thus $\leq$-monotonic by \Cref{thm:continuous-implies-monotonic-special}.
	Hence, \Cref{thm:our-monoid-has-addprop} yields the desired $\transformerofof{\aflowconstraint}{\inflow_1}(\anodep)\leq\transformerofof{\aflowconstraint}{\inflow_2}(\anodep)$.
\end{proof}


% -----------------------------------------------------------------------------
% -----------------------------------------------------------------------------
% -----------------------------------------------------------------------------
% -----------------------------------------------------------------------------
\begin{proof}[Proof of \Cref{thm:inflow-fprel}(i)]
	Let $\aflowconstraint=(\setnodes,\edges,\dontcare)$.
	Define $\aflowconstraint_1=\aflowconstraint[\inflow\mapsto\inflow_1]$ and $\aflowconstraint_2=\aflowconstraint[\inflow\mapsto\inflow_2]$.
	By \Cref{thm:flow-as-lfp}, the flow in $\aflowconstraint_i$ is $\aflowconstraint_i.\fval=\bigjoin\setcond{f_i^j(\bot)}{j\in\nat}$ with $f_i:(\setnodes{\to}\amonoid)\to(\setnodes{\to}\amonoid)$ defined by: \[
		f_i(\cval)(\anode) ~~=~~ \sum_{\anodep\in\nat\setminus\setnodes}\inflow_i(\anodep,\anode) + \sum_{\anodep\in\setnodes}\edgesatof{\anodep}{\anode}{\cval(\anodep)}
		\ .
	\]
	To conclude, it suffices to show that $f_1^j(\bot) \fpreldot f_2^j(\bot)$ holds for all $j\in\nat$.
	We proceed by induction.
	In the base case, $f_1^0(\bot)=\bot\fpreldot\bot=f_2^0$ because $\bot\fprel\bot$ by \eqref{def:fpcompat:refl-trans} of $\fpcompatible[\fprel]{\aflowconstraint}$.
	For the induction step, we show $f_1(f_1^j(\bot)) \fpreldot f_2(f_2^j(\bot)$.
	By induction together with $\fpreldot$-monotonicity of $f_1$ by \Cref{thm:flow-iteration-monotonic-continuous}, we have $f_1(f_1^j(\bot)) \fpreldot f_1(f_2^j(\bot))$.
	Because $\fprel$ is transitive by \eqref{def:fpcompat:refl-trans} of $\fpcompatible[\fprel]{\aflowconstraint}$, it remains to show $f_1(\cval) \fpreldot f_2(\cval)$ with $\cval=f_2^j(\bot)$.
	Assume for a moment that we have $\sum_{\anodep\in\nat\setminus\setnodes}\inflow_1(\anodep,\anode)\fprel\sum_{\anodep\in\nat\setminus\setnodes}\inflow_2(\anodep,\anode)$, for all $\anode\in\setnodes$.
	Then, we immediately get $f_1(\cval) \fpreldot f_2(\cval)$ by \eqref{def:fpcompat:add} of $\fpcompatible[\fprel]{\aflowconstraint}$ and \Cref{thm:add-relations}.
	To see the correspondence, choose $\setnodesp_1,\setnodesp_2$ such that $\setnodesp_1\uplus\setnodesp_2=\nat\setminus\setnodes$ and $\inflow_1(\anodep,\anode)=\inflow_1(\anodep,\anode)$ and $\inflow_1(\anodepp,\anode)\neq\inflow_1(\anodepp,\anode)$ for all $\anode\in\setnodes,\anodep\in\setnodesp_1,\anodepp\in\setnodesp_2$.
	We have $\sum_{\anodep\in\nat\setnodesp_1}\inflow_1(\anodep,\anode)=\sum_{\anodep\in\setnodes_1}\inflow_2(\anodep,\anode)$.
	Because $\inflow_1\fpreleq\inflow_2$, we must have $\inflow_1(\anodepp,\anode)\fprel\inflow_1(\anodepp,\anode)$.
	Hence, $\sum_{\anodepp\in\nat\setnodesp_2}\inflow_1(\anodep,\anode)=\sum_{\anodep\in\setnodes_2}\inflow_2(\anodepp,\anode)$ by \eqref{def:fpcompat:add} of $\fpcompatible[\fprel]{\aflowconstraint}$ and \Cref{thm:add-relations}.
	Again by \eqref{def:fpcompat:add}, adding the two sums maintains $\fprel$.
	That is, $\sum_{\anodep\in\nat\setminus\setnodes}\inflow_1(\anodep,\anode)\fprel\sum_{\anodep\in\nat\setminus\setnodes}\inflow_2(\anodep,\anode)$ holds, as required.
	This concludes the induction.
\end{proof}

\begin{proof}[Proof of \Cref{thm:inflow-fprel}(ii)]
	Let $\aflowconstraint=(\setnodes,\edges,\dontcare)$.
	Define $\aflowconstraint_1=\aflowconstraint[\inflow\mapsto\inflow_1]$ and $\aflowconstraint_2=\aflowconstraint[\inflow\mapsto\inflow_2]$.
	Let $\anodep\in\nat\setminus\setnodes$.
	We have $\transformerofof{\aflowconstraint}{\inflow_i}(\anode)=\sum_{\anode\in\setnodes}\edgesatof{\anode}{\anodep}{\aflowconstraint_i.\fval(\anode)}$ by the definition of $\transformerof{\dontcare}$ and the outflow.
	Let $\anode\in\setnodes$.
	Part (i) of this \namecref{thm:inflow-fprel} gives $\aflowconstraint_1.\fval(\anode)\fprel\aflowconstraint_2.\fval(\anode)$.
	Then, we get $\edgesatof{\anode}{\anodep}{\aflowconstraint_1.\fval(\anode)}\fprel\edgesatof{\anode}{\anodep}{\aflowconstraint_2.\fval(\anode)}$, because $\edges$ is $\fpreldot$-monotonic by \eqref{def:fpcompat:edges} of $\fpcompatible[\fprel]{\aflowconstraint}$ from the assumption.
	Hence, \Cref{thm:add-relations} yields the desired $\transformerofof{\aflowconstraint}{\inflow_1}(\anodep)\fprel\transformerofof{\aflowconstraint}{\inflow_2}(\anodep)$
\end{proof}


% -----------------------------------------------------------------------------
% -----------------------------------------------------------------------------
% -----------------------------------------------------------------------------
% -----------------------------------------------------------------------------
\begin{proof}[Proof of \Cref{thm:partial-fixpoint}]
	Follows from \cite[Proof of Theorem~1]{DBLP:conf/tacas/MeyerWW23,DBLP:journals/corr/abs-2304-04886}.
	We repeat the argument and adapt the proof to our use case.

	\newcommand{\F}{\alpha}
	\newcommand{\FF}{\alpha^{\dagger}}
	\newcommand{\G}{\beta}
	\newcommand{\dval}{\mathit{dval}}
	\newcommand{\theinflow}{\overline{\inflow}}
	Let $\setnodes_1\defeq\aflowconstraint_1.\setnodes$, $\setnodes_2\defeq\aflowconstraint_2.\setnodes$, and $\overline{\setnodes}\defeq\nat\setminus(\setnodes_1\cup\setnodes_2)$.
	Further, let $\theinflow\defeq(\aflowconstraint_1\statemult\aflowconstraint_2).\inflow$.
	To apply \Bekic~\cite{DBLP:conf/ibm/Bekic84e}, define the target pairing of two functions
	\begin{align*}
		\F~:~A\times B\to A
		\qquad\text{and}\qquad
		\G~:~A\times B\to B
	\end{align*}
	over the same domain $A \times B$ as the function
	\begin{align*}
	 	\pairingof{\F}{\G}~:~A\times B \to A\times B
	 	\qquad\text{with}\qquad
	 	\pairingof{\F}{\G}(a) ~\defeq~ (f(a),\, g(a))
	 	\ .
	\end{align*}
	We compute the flow of $\aflowconstraint_1\uplus\aflowconstraint_2$ as the least fixed point of a target pairing $\pairingof{f}{g}$ with
	\begin{align*}
		\F ~:&~\, ((\setnodes_1\uplus\setnodes_2) \to \amonoid) \to (\setnodes_1 \to \amonoid)
		\\\text{and}\qquad
		\G ~:&~\, ((\setnodes_1\uplus\setnodes_2) \to \amonoid) \to (\setnodes_2 \to \amonoid)
		\ .
	\end{align*}
	Function $\F$ updates the flow of the nodes $\setnodes_1$ in $\aflowconstraint_1$ depending on the flow in/inflow from $\aflowconstraint_2$.
	Function $\G$ is responsible for the flow of the nodes $\setnodes_2$ in $\aflowconstraint_2$.
	The inflow from the nodes outside $\aflowconstraint_1\uplus\aflowconstraint_2$ is constant, $\theinflow$.
	Concretely, we define $\F$ and $\G$ along the lines of the flow equation \eqref{def:flow-equation}:
	\begin{align*}
		\F(\dval)(\anode) ~\defeq~
			\sum_{\anodepp\in\overline{\setnodes}} \theinflow(\anodepp,\anode)
			+
			\sum_{\anodep\in\setnodes_1\cup\setnodes_2} (\aflowconstraint_1\uplus\aflowconstraint_2).\edgesatof{\anodep}{\anode}{\dval(\anodep)}
		\\\text{and}\qquad
		\G(\dval)(\anodep) ~\defeq~
			\sum_{\anodepp\in\overline{\setnodes}} \theinflow(\anodepp,\anodep)
			+
			\sum_{\anode\in\setnodes_1\cup\setnodes_2} (\aflowconstraint_1\uplus\aflowconstraint_2).\edgesatof{\anode}{\anodep}{\dval(\anode)}
	\end{align*}
	From \Cref{thm:flow-iteration-monotonic-continuous} for $\aflowconstraint_1\statemult\aflowconstraint_2$ we get that $\F$ and $\G$ are $\leq$-monotonic and $\leq$-continuous.
	Moreover, the \namecref{thm:flow-iteration-monotonic-continuous} also gives that $\F$ and $\G$ are $\fpreldot$-monotonic, provided $\fpcompatible[\fprel]{\aflowconstraint_1\statemult\aflowconstraint_2}$ holds.
	Observe that the above definitions guarantee:
	\begin{align*}
		(\aflowconstraint_1\statemult\aflowconstraint_2).\fval ~=~ \lfpof{\pairingof{\F}{\G}}
		\ .
	\end{align*}

	We curry function $\F$ and obtain:
	\begin{align*}
		\F ~:~ (\setnodes_2 \to \amonoid) \to (\setnodes_1 \to \amonoid) \to (\setnodes_1 \to \amonoid)
		\ .
	\end{align*}
	This gives rise to the following function:
	\begin{align*}
		\F(\cval) ~:\,&~~ (\setnodes_1 \to \amonoid) \to (\setnodes_1 \to \amonoid)
		\qquad\text{for all}\qquad
		\cval ~:~ \setnodes_2 \to \amonoid
		\\\text{with}\qquad
		\F(\cval)(\dval)(\anode) ~=&~
			\sum_{\anodepp\in\overline{\setnodes}} \theinflow(\anodepp,\anode)
			+
			\sum_{\anodep\in\setnodes_1\cup\setnodes_2} (\aflowconstraint_1\uplus\aflowconstraint_2).\edgesatof{\anodep}{\anode}{(\cval\uplus\dval)(\anodep)}
		\ .
	\end{align*}
	This function is still $\leq$-monotonic and $\leq$-continuous as well as $\fpreldot$-monotonic if $\fpcompatible[\fprel]{\aflowconstraint_1\statemult\aflowconstraint_2}$.
	Therefore, it has a least fixed point by \Cref{thm:fixpoint-kleene} so that this is well-defined:
	\begin{align*}
		\FF ~:~ (\setnodes_2 \to \amonoid) \to (\setnodes_1 \to \amonoid)
		\qquad\text{with}\qquad
		\FF(\cval) ~\defeq~ \lfpof{\F(\cval)}
		\ .
	\end{align*}

	Towards an application of \Bekic, we define
	\begin{align*}
		\pairingof{\FF}{\myid} ~:~ (\setnodes_2\to\amonoid) \to ((\setnodes_1\uplus\setnodes_2)\to\amonoid)
		\qquad\text{with}\qquad
		\myid ~:~ (\setnodes_2\to\amonoid) \to (\setnodes_2\to\amonoid)
	\end{align*}
	Now, compose this function with $\G$.
	This yields:
	\begin{align*}
		\G \circ \pairingof{\FF}{\myid} ~:~ (\setnodes_2\to\amonoid) \to (\setnodes_2\to\amonoid)
		\ .
	\end{align*}
	Now, \Bekic guarantees the correctness of the following least fixed point:
	\begin{align}
		\lfpof{\pairingof{\F}{\G}} ~=~ (\FF(\cval), \cval)
		\qquad\text{with}\qquad
		\cval ~=~ \lfpof{\G \circ \pairingof{\FF}{\myid}}
		\label{proof:partial-fixpoint:bekic-fixed-point}
		\ .
	\end{align}

	\medskip
	By $\aflowconstraint_1\statemultdef\aflowconstraint_2$ we have $(\aflowconstraint_1\statemult\aflowconstraint_2).\fval=\aflowconstraint_1.\fval\uplus\aflowconstraint_2.\fval$.
	Combined with \eqref{proof:partial-fixpoint:bekic-fixed-point} this yields:
	\begin{align*}
		\aflowconstraint_2.\fval ~=~ \lfpof{\G \circ \pairingof{\FF}{\myid}}
		\ .
	\end{align*}
	We show that $\G \circ \pairingof{\FF}{\myid}$ is equivalent to $f$.
	To that end, rewrite the curried version of $\F$:
	\begin{align*}
		&\F(\cval)(\dval)(\anode)
		\\\explain{Def. $\alpha(\cval)$}=~&
		\sum_{\anodepp\in\overline{\setnodes}} \theinflow(\anodepp,\anode)
		+
		\sum_{\anodep\in\setnodes_1\cup\setnodes_2} (\aflowconstraint_1\uplus\aflowconstraint_2).\edgesatof{\anodep}{\anode}{(\cval\uplus\dval)(\anodep)}
		\\\explain{Def. $\aflowconstraint_1\statemult\aflowconstraint_2$}=~&
		\sum_{\anodepp\in\overline{\setnodes}} \theinflow(\anodepp,\anode)
		+
		\sum_{\anodep\in\setnodes_2} \aflowconstraint_2.\edgesatof{\anodep}{\anode}{\cval(\anodep)}
		+
		\sum_{\anodep\in\setnodes_1} \aflowconstraint_1.\edgesatof{\anodep}{\anode}{\dval(\anodep)}
		\\\explain{Def. $\inflow_\cval$}=~&
		\sum_{\anodep\in\setnodes_2\cup\overline{\setnodes}} \inflow_\cval(\anodep,\anode)
		+
		\sum_{\anodep\in\setnodes_1} \aflowconstraint_1.\edgesatof{\anodep}{\anode}{\dval(\anodep)}
	\end{align*}
	Note that the last sum is the flow equation for $\aflowconstraint_1$ with its inflow updated to $\inflow_\cval$.
	By definition, this means:
	\begin{align}
		\FF(\cval)
		~=~
		\lfpof{\F(\cval)}
		~=~
		\aflowconstraint_1[\inflow\mapsto\inflow_\cval].\fval
		\label{proof:partial-fixpoint:flow-in-h1}
		\ .
	\end{align}
	With this, we conclude:
	\begin{align*}
		&(\G \circ \pairingof{\FF}{\myid})(\cval)(\anodep)
		\\\explain{Def. $\circ$}\!\!=~&
		\G(\pairingof{\FF}{\myid}(\cval))(\anodep)
		\\\explain{Def. $\G$}\!\!=~&
		\sum_{\anodepp\in\overline{\setnodes}} \theinflow(\anodepp,\anodep)
		+
		\sum_{\anode\in\setnodes_1\cup\setnodes_2} (\aflowconstraint_1\uplus\aflowconstraint_2).\edgesatof{\anode}{\anodep}{\pairingof{\FF}{\myid}(\cval)(\anode)}
		\\\explain{by $\aflowconstraint_1\statemultdef\aflowconstraint_2$}\!\!=~&
		\sum_{\anodepp\in\overline{\setnodes}} \theinflow(\anodepp,\anodep)
		+
		\sum_{\anode\in\setnodes_1} \aflowconstraint_1.\edgesatof{\anode}{\anodep}{\pairingof{\FF}{\myid}(\cval)(\anode)}
		+
		\sum_{\anode\in\setnodes_2} \aflowconstraint_2.\edgesatof{\anode}{\anodep}{\pairingof{\FF}{\myid}(\cval)(\anode)}
		\\\explain{Def. $\pairingof{\FF}{\myid}$}\!\!=~&
		\sum_{\anodepp\in\overline{\setnodes}} \theinflow(\anodepp,\anodep)
		+
		\sum_{\anode\in\setnodes_1} \aflowconstraint_1.\edgesatof{\anode}{\anodep}{\FF(\cval)(\anode)}
		+
		\sum_{\anode\in\setnodes_2} \aflowconstraint_2.\edgesatof{\anode}{\anodep}{\myid(\cval)(\anode)}
		\\\explain{by \eqref{proof:partial-fixpoint:flow-in-h1}}\!\!=~&
		\sum_{\anodepp\in\overline{\setnodes}} \theinflow(\anodepp,\anodep)
		+
		\sum_{\anode\in\setnodes_1} \aflowconstraint_1.\edgesatof{\anode}{\anodep}{\aflowconstraint_1[\inflow\mapsto\inflow_\cval].\fval(\anode)}
		+
		\sum_{\anode\in\setnodes_2} \aflowconstraint_2.\edgesatof{\anode}{\anodep}{\cval(\anode)}
		\\\explain{Def. $\outflow$}\!\!=~&
		\sum_{\anodepp\in\overline{\setnodes}} \theinflow(\anodepp,\anodep)
		+
		\sum_{\anode\in\setnodes_1} \aflowconstraint_1[\inflow\mapsto\inflow_\cval].\outflow(\anode,\anodep)
		+
		\sum_{\anode\in\setnodes_2} \aflowconstraint_2.\edgesatof{\anode}{\anodep}{\cval(\anode)}
		\\\explain{Def. $\transformerof{\dontcare}$}\!\!=~&
		\sum_{\anodepp\in\overline{\setnodes}} \theinflow(\anodepp,\anodep)
		+
		\transformerofof{\aflowconstraint_1}{\inflow_\cval}(\anodep)
		+
		\sum_{\anode\in\setnodes_2} \aflowconstraint_2.\edgesatof{\anode}{\anodep}{\cval(\anode)}
		\\\explain{Def. $f$}\!\!=~&
		f(\cval)(\anodep)
	\end{align*}
	Overall, we arrive at:
	\begin{align*}
		\aflowconstraint_2.\fval
		~=~
		\lfpof{\G \circ \pairingof{\FF}{\myid}}
		~=~
		\lfpof{f}
	\end{align*}
	Finally, $f$ is $\leq$-monotonic and $\leq$-continuous because $\FF,\G,\myid$ are.
	Moreover, $f$ is $\fpreldot$-monotonic because $\FF,\G,\myid$ are, provided $\fpcompatible[\fprel]{\aflowconstraint_1\statemult\aflowconstraint_2}$.
	This concludes the proof.
\end{proof}


% -----------------------------------------------------------------------------
% -----------------------------------------------------------------------------
% -----------------------------------------------------------------------------
% -----------------------------------------------------------------------------
\begin{proof}[Proof of \Cref{thm:upward-closed-outflow}]
	\newcommand{\theinflow}{\overline{\inflow}}
	\newcommand{\thefpin}[1]{\inflow^{\smash{\setnodes_1}}_{#1}}
	%
	We unroll the premise for flow graphs $\aflowconstraint_1,\aflowconstraint_2,\aflowconstraint_F$:
	\begin{compactenum}[(A)]
		\item\label{proof:upward-closed-outflow:multdef}
			$\aflowconstraint_1\statemultdef \aflowconstraint_F$
		\item\label{proof:upward-closed-outflow:ctxfprel}
			$\aflowconstraint_1 \ctxfprel \aflowconstraint_2$
		\item\label{proof:upward-closed-outflow:fprel-compat}
			$\fpcompatible[\fprel]{\aflowconstraint_1,\aflowconstraint_2,\aflowconstraint_F}$, i.e., \eqref{def:fpcompat:refl-trans}, \eqref{def:fpcompat:add}, \eqref{def:fpcompat:edges}, and \eqref{def:fpcompat:lfp} from \Cref{def:fpcompat}
	\end{compactenum}

	\medskip
	Let $\setnodes_1 \defeq \aflowconstraint_1.\setnodes$ and $\setnodes_F \defeq \aflowconstraint_F.\setnodes$ and $\overline{\setnodes} \defeq \nat\setminus(\setnodes_1 \cup \setnodes_F)$ and $\theinflow \defeq (\aflowconstraint_1 \statemult \aflowconstraint_F).\inflow$.
	By \eqref{proof:upward-closed-outflow:ctxfprel}, $\setnodes_1 = \aflowconstraint_2.\setnodes$.
	Define the flow graph $\aflowconstraint_{2+F}$ by \[
		\aflowconstraint_{2+F} ~\defeq~ \bigl(~
			\setnodes_1 \uplus \setnodes_F,~~
			\aflowconstraint_2.\edges \uplus \aflowconstraint_F.\edges,~~
			\theinflow
		~\bigr)
		\ .
	\]
	Note that $\aflowconstraint_{2+F}$ is well defined because $\setnodes_1 \cap \setnodes_F = \emptyset$ by \eqref{proof:upward-closed-outflow:ctxfprel}.
	Now, choose \[
		\aflowconstraint_2'=\restrictto{\aflowconstraint_{2+F}}{\setnodes_2}
		\qquad\text{and}\qquad
		\aflowconstraint_F'=\restrictto{\aflowconstraint_{2+F}}{\setnodes_F}
		\ .
	\]
	By definition, $\aflowconstraint_2'.\setnodes=\setnodes_1$ and $\aflowconstraint_F'.\setnodes=\setnodes_F$.
	Moreover, $\aflowconstraint_2'.\edges=\aflowconstraint_2.\edges$ and $\aflowconstraint_F'.\edges=\aflowconstraint_F.\edges$.
	\Cref{thm:restriction-vs-statemult} gives both $\aflowconstraint_2'\statemultdef\aflowconstraint_F'$ and $\aflowconstraint_2'\statemult\aflowconstraint_F'=\aflowconstraint_{2+F}$.
	It is easy to see $\fpcompatible[\fprel]{\aflowconstraint_2',\aflowconstraint_F',\aflowconstraint_{2+F}}$.
	
	\medskip
	We derive the flow in $\aflowconstraint_F$ and $\aflowconstraint_F'$ as a fixed point relative to the inflow provided by the transformers of $\aflowconstraint_1$ and $\aflowconstraint_2'$, respectively.
	Concretely, invoke \Cref{thm:partial-fixpoint} for $\aflowconstraint_F$ in $\aflowconstraint_1\statemult\aflowconstraint_F$ and $\aflowconstraint_F'$ in $\aflowconstraint_2'\statemult\aflowconstraint_F'$, yielding:
	\begin{align}
		\aflowconstraint_F.\fval ~=~ \lfpof{f_1} ~=&~~ \bigjoin\setcond{f_1^i(\bot)}{i\in\nat}
		\label{proof:upward-closed-outflow:flow-fp-pre}
		\\\text{and}\quad
		\aflowconstraint_F'.\fval ~=~ \lfpof{f_2} ~=&~~ \bigjoin\setcond{f_2^i(\bot)}{i\in\nat}
		\label{proof:upward-closed-outflow:flow-fp-post}
	\end{align}
	with $f_j : (\setnodes_F \to \amonoid) \to (\setnodes_F \to \amonoid)$ and $\thefpin{\cval} : (\nat\setminus\setnodes_1) \times \setnodes_1 \to \amonoid$ defined by:
	\begin{align}
		f_j(\cval)(\anode) ~&\defeq~
			\sum_{\anodepp\in\overline{\setnodes}} \theinflow(\anodepp,\anode)
			+
			\transformerofof{\aflowconstraint_j}{\thefpin{\cval}}(\anode)
			+
			\sum_{\anodep\in\setnodes_F} \aflowconstraint_F.\edgesatof{\anodep}{\anode}{\cval(\anodep)}
			\label{proof:upward-closed-outflow:frame-flow-step}
		\\
		\thefpin{\cval}(\anode,\anodep) ~&\defeq~
			\anode\in\setnodes_F ~~~?~~~
			\aflowconstraint_F.\edgesatof{\anodep}{\anode}{\cval(\anodep)}
			~~:~~
			\theinflow(\anode,\anodep)
			\label{proof:upward-closed-outflow:fp-inflow-def}
	\end{align}
	Intuitively, $\thefpin{\cval}$ is the inflow at nodes $\setnodes_1$ given the flow values $\cval$ for the nodes in $\setnodes_F$.
	That is, it is the sum of $\theinflow$ plus the flow received from $\setnodes_F$.
	Hence, by \eqref{proof:upward-closed-outflow:multdef} and choice $\aflowconstraint_2'$ we have:
	\begin{align}
		\thefpin{\aflowconstraint_F.\fval} ~=~ \aflowconstraint_1.\inflow
		\qquad\text{and}\qquad
		\thefpin{\aflowconstraint_F'.\fval} ~=~ \aflowconstraint_2'.\inflow
		\label{proof:upward-closed-outflow:inflow-frame}
		\ .
	\end{align}
	\Cref{thm:partial-fixpoint} also provides the following properties for $f_1$ and $f_2$:
	\begin{compactenum}[(A)]
		\setcounter{enumi}{3}
		\item both $f_1$ and $f_2$ are $\leq$-monotonic and $\leq$-continuous, and
			\label{proof:upward-closed-outflow:f-leq-monotonic}
		\item
			both $f_1$ and $f_2$ are $\fpreldot$-monotonic.
			\label{proof:upward-closed-outflow:f-fprel-monotonic}
	\end{compactenum}

	\medskip
	We now show $f_1^i(\cval) \fpreldot f_2^i(\cval)$ for all $\cval \leq \aflowconstraint_F.\fval$ and all $i\in\nat$.
	The claim is true for $i=0$ by \eqref{def:fpcompat:refl-trans}.
	For $i\geq 1$, we proceed by induction.
	For the base case, $i=1$, observe $\thefpin{\cval} \leq \thefpin{\aflowconstraint_F.\fval}$ by \eqref{proof:upward-closed-outflow:fp-inflow-def} together with the fact that all edge functions in $\aflowconstraint_F.\edges$ are $\leq$-continuous and thus $\leq$-monotonic by \Cref{thm:continuous-implies-monotonic-special}.
	Then, \eqref{proof:upward-closed-outflow:inflow-frame} gives $\thefpin{\cval} \leq \aflowconstraint_1.\inflow$.
	By \eqref{proof:upward-closed-outflow:ctxfprel}, this yields $\transformerofof{\aflowconstraint_1}{\thefpin{\cval}} \fpreldot \transformerofof{\aflowconstraint_2}{\thefpin{\cval}}$.
	Since this is the only part that differs in $f_1$ and $f_2$ according to \eqref{proof:upward-closed-outflow:frame-flow-step}, we obtain the desired $f_1(\cval) \fpreldot f_2(\cval)$ by \eqref{def:fpcompat:add} and \Cref{thm:add-relations}.
	(Note: should the sum be empty, we obtain the desired approximation by $\bot\fprel\bot$ from \eqref{def:fpcompat:refl-trans}.)
	For the induction step, we have $f_1^i(\cval') \fpreldot f_2^i(\cval')$ for all $\cval' \leq \aflowconstraint_F.\fval$.
	We show that $f_1^{i+1}(\cval) \fpreldot f_2^{i+1}(\cval)$ holds for all $\cval \leq \aflowconstraint_F.\fval$.
	By \eqref{proof:upward-closed-outflow:f-leq-monotonic} combined with \eqref{proof:upward-closed-outflow:flow-fp-pre}, we have $f_1(\cval) \leq f_1(\aflowconstraint_F.\fval) = \aflowconstraint_F.\fval$.
	Then, by induction, we obtain $f_1^i(f_1(\cval)) \fpreldot f_2^i(f_1(\cval))$.
	We already showed (for the base case), that $f_1(\cval) \fpreldot f_2(\cval)$ holds.
	Hence, $f_2^i(f_1(\cval)) \fpreldot f_2^i(f_2(\cval))$ because $f_2$ is $\fpreldot$-monotonic by \eqref{proof:upward-closed-outflow:f-fprel-monotonic}.
	By transitivity of $\fprel$ from \eqref{def:fpcompat:refl-trans}, we get $f_1^i(f_1(\cval)) \fpreldot f_2^i(f_2(\cval))$.
	Altogether, this concludes the induction and proves:
	\begin{align}
		\forall\, i\in\nat
		~~
		\forall\, \cval \leq \aflowconstraint_F.\fval
		. \quad
		f_1^i(\cval) ~\fpreldot~ f_2^i(\cval)
	 	\label{proof:upward-closed-outflow:flow-iteration-approx}
		\ .
	\end{align}
	Now, we invoke \Cref{thm:fixpoint-kleene} for $f_1$ and $f_2$.
	This yields the fixed points of $f_1$ and $f_2$ as the joins $\lfpof{f_1}=\bigjoin\achain$ and $\lfpof{f_2}=\bigjoin\achainp$ over the $\leq$-ascending chains $\achain \defeq f_1^0(\bot)\leq f_1^2(\bot)\leq\cdots$ and $\achainp \defeq f_2^0(\bot)\leq f_2^2(\bot)\leq\cdots$, respectively.
	We now obtain $\bigjoin\achain = f_1(\bigjoin\achain) \fpreldot f_2(\bigjoin\achain)$ from combining \eqref{proof:upward-closed-outflow:flow-fp-pre} and \eqref{proof:upward-closed-outflow:flow-iteration-approx}.
	This together with \eqref{proof:upward-closed-outflow:f-leq-monotonic}, \eqref{proof:upward-closed-outflow:f-fprel-monotonic}, and \eqref{proof:upward-closed-outflow:flow-iteration-approx}, establishes the premise of \eqref{def:fpcompat:lfp}.
	Applying the property for $\bigjoin\achain$ and $\bigjoin\achainp$ yields:
	\begin{align}
		\aflowconstraint_F.\fval
		~~\explains{\eqref{proof:upward-closed-outflow:flow-fp-pre}}{=}~~
		\lfpof{f_1}
		~~=~~
		\bigjoin\achain
		~~\explains{\eqref{def:fpcompat:lfp}}{\fpreldot}~~
		\bigjoin\achainp
		~~=~~
		\lfpof{f_2}
		~~\explains{\eqref{proof:upward-closed-outflow:flow-fp-post}}{=}~~
		\aflowconstraint_F'.\fval
		\label{proof:upward-closed-outflow:frame-flow-approx}
		\ .
	\end{align}

	\medskip
	Next, we show $\aflowconstraint_2.\inflow \fpreldot \aflowconstraint_2'.\inflow$.
	To that end, consider some $\anode\in\setnodes_2$, $\anodep\in\setnodes_F$, and $\anodepp\in\overline{\setnodes}$.
	By choice, $\aflowconstraint_1.\inflow(\anodepp,\anode)=\theinflow(\anodepp,\anode)=\aflowconstraint_2'.\inflow(\anodepp,\anode)$.
	Hence, $\aflowconstraint_2.\inflow(\anodepp,\anode)=\aflowconstraint_2'.\inflow(\anodepp,\anode)$ because \eqref{proof:upward-closed-outflow:ctxfprel} gives $\aflowconstraint_1.\inflow=\aflowconstraint_2.\inflow$.
	It remains to consider the inflow at $\anode$ from $\anodep$:
	\begin{align*}
		&\aflowconstraint_2.\inflow(\anodep,\anode)
		\\\explain{by definition}=~~&
		\aflowconstraint_F.\edgesatof{\anodep}{\anode}{\aflowconstraint_F.\fval(\anodep)}
		\\\explain{by choice of $\aflowconstraint_F'$}=~~&
		\aflowconstraint_F'.\edgesatof{\anodep}{\anode}{\aflowconstraint_F.\fval(\anodep)}
		\\\explain{by \eqref{proof:upward-closed-outflow:frame-flow-approx} and \eqref{def:fpcompat:edges}}\fprel~~&
		\aflowconstraint_F'.\edgesatof{\anodep}{\anode}{\aflowconstraint_F'.\fval(\anodep)}
		\\\explain{by definition}=~~&
		\aflowconstraint_2'.\inflow(\anodep,\anode)
		\ .
	\end{align*}
	Combining the above, we obtain:
	\begin{align}
		\aflowconstraint_2.\inflow ~\fpreleq~ \aflowconstraint_2'.\inflow
		\qquad\text{and}\qquad
		\restrictto{\aflowconstraint_2.\inflow}{\overline{\setnodes}\times\setnodes_1} ~=~ \restrictto{\aflowconstraint_2'.\inflow}{\overline{\setnodes}\times\setnodes_1}
		\label{proof:upward-closed-outflow:footprint-inflow-approx}
		\ .
	\end{align}
	Recall that $\aflowconstraint_2$ and $\aflowconstraint_2'$ differ only in the inflow.
	Hence, \eqref{proof:upward-closed-outflow:footprint-inflow-approx} together \Cref{thm:inflow-fprel} yields
	\begin{align}
		\aflowconstraint_2.\fval ~\fpreldot~ \aflowconstraint_2'.\fval
		\label{proof:upward-closed-outflow:footprint-flow-approx}
		\ .
	\end{align}

	\medskip
	Now, we are ready to show that $(\aflowconstraint_1\statemult\aflowconstraint_F).\outflow \fpreldot \aflowconstraint_{2+F}.\outflow$ holds.
	As a first step, we conclude the following relation among the transformers of $\aflowconstraint_1$ and $\aflowconstraint_2'$:
	\begin{align*}
		&\transformerofof{\aflowconstraint_1}{\aflowconstraint_1.\inflow}
		\\\explain{by \eqref{proof:upward-closed-outflow:ctxfprel}}\fpreldot~~&
		\transformerofof{\aflowconstraint_2}{\aflowconstraint_1.\inflow}
		\\\explain{by \eqref{proof:upward-closed-outflow:inflow-frame}}=~~&
		\transformerofof{\aflowconstraint_2}{\thefpin{\aflowconstraint_F.\fval}}
		\\\explain{see below}\fpreldot~~&
		\transformerofof{\aflowconstraint_2}{\thefpin{\aflowconstraint_F'.\fval}}
		\\\explain{by \eqref{proof:upward-closed-outflow:inflow-frame}}=~~&
		\transformerofof{\aflowconstraint_2}{\aflowconstraint_2'.\inflow}
	\end{align*}
	where $\transformerofof{\aflowconstraint_2}{\thefpin{\aflowconstraint_F.\fval}} \fpreldot \transformerofof{\aflowconstraint_2}{\thefpin{\aflowconstraint_F'.\fval}}$ holds because \eqref{proof:upward-closed-outflow:frame-flow-approx} gives $\aflowconstraint_F.\fval\fpreldot\aflowconstraint_F'.\fval$ which means $\thefpin{\aflowconstraint_F.\fval}\fpreldot\thefpin{\aflowconstraint_F'.\fval}$ because edges functions are $\fpreldot$-monotonic by \eqref{def:fpcompat:edges} and thus an application of \Cref{thm:outflow-vs-statemult} yields the desired property.
	By the choice of $\aflowconstraint_2'$, we obtain:
	\begin{align}
		\transformerofof{\aflowconstraint_1}{\aflowconstraint_1.\inflow}
		~\fpreldot~
		\transformerofof{\aflowconstraint_2'}{\aflowconstraint_2'.\inflow}
		\label{proof:upward-closed-outflow:footprint-inflow-transformer-pre-vs-post}
		\ .
	\end{align}
	As a second step, we conclude the following relation among the transformers of $\aflowconstraint_F$ and $\aflowconstraint_F'$, for some node $\anodepp\notin\setnodes_F$:
	\begin{align*}
		&\transformerofof{\aflowconstraint_F}{\aflowconstraint_F.\inflow}(\anodepp)
		\\\explain{Def. $\transformerof{\dontcare}$}=~~&
		\sum_{\anodep\in\setnodes_F} \aflowconstraint_F.\outflow(\anodep,\anodepp)
		\\\explain{by definition}=~~&
		\sum_{\anodep\in\setnodes_F} \aflowconstraint_F.\edgesatof{\anodep}{\anodepp}{\aflowconstraint_F.\fval(\anodep)}
		\\\explain{by \eqref{proof:upward-closed-outflow:frame-flow-approx} and \eqref{def:fpcompat:edges} and \Cref{thm:add-relations}; if empyt sum, by \eqref{def:fpcompat:refl-trans}}\fprel~~&
		\sum_{\anodep\in\setnodes_F} \aflowconstraint_F.\edgesatof{\anodep}{\anodepp}{\aflowconstraint_F'.\fval(\anodep)}
		\\\explain{by choice of $\aflowconstraint_F'$}=~~&
		\sum_{\anodep\in\setnodes_F'} \aflowconstraint_F'.\edgesatof{\anodep}{\anodepp}{\aflowconstraint_F'.\fval(\anodep)}
		\\\explain{by definition}=~~&
		\sum_{\anodep\in\setnodes_F'} \aflowconstraint_F'.\outflow(\anodep,\anodepp)
		\\\explain{Def. $\transformerof{\dontcare}$}=~~&
		\transformerofof{\aflowconstraint_F'}{\aflowconstraint_F'.\inflow}(\anodepp)
	\end{align*}
	That is,
	\begin{align}
		\transformerofof{\aflowconstraint_F}{\aflowconstraint_F.\inflow}
		~\fpreldot~
		\transformerofof{\aflowconstraint_F'}{\aflowconstraint_F'.\inflow}
		\label{proof:upward-closed-outflow:frame-inflow-transformer-pre-vs-post}
		\ .
	\end{align}
	Using the above, we arrive at the following, for some node $\anodepp\in\overline{\setnodes}$:
	\begin{align*}
		&\transformerofof{\aflowconstraint_1\statemult\aflowconstraint_F}{\theinflow}(\anodepp)
		\\\explain{by choice $\theinflow$}=~~&
		\transformerofof{\aflowconstraint_1\statemult\aflowconstraint_F}{(\aflowconstraint_1\statemult\aflowconstraint_F).\inflow}(\anodepp)
		\\\explain{by \Cref{thm:transformer-vs-statemult}}=~~&
		\transformerofof{\aflowconstraint_1}{\aflowconstraint_1.\inflow}(\anodepp)
		+\transformerofof{\aflowconstraint_F}{\aflowconstraint_F.\inflow}(\anodepp)
		\\\explain{by \eqref{proof:upward-closed-outflow:footprint-inflow-transformer-pre-vs-post} and \eqref{proof:upward-closed-outflow:frame-inflow-transformer-pre-vs-post} and \Cref{thm:add-relations}}\fpreldot~~&
		\transformerofof{\aflowconstraint_2'}{\aflowconstraint_2'.\inflow}(\anodepp)
		+\transformerofof{\aflowconstraint_F'}{\aflowconstraint_F'.\inflow}(\anodepp)
		\\\explain{by \Cref{thm:transformer-vs-statemult}}=~~&
		\transformerofof{\aflowconstraint_2'\statemult\aflowconstraint_F'}{(\aflowconstraint_2'\statemult\aflowconstraint_F').\inflow}(\anodepp)
		\\\explain{by choice $\aflowconstraint_{2+F},\aflowconstraint_2',\aflowconstraint_F$}=~~&
		\transformerofof{\aflowconstraint_{2+F}}{\theinflow}(\anodepp)
	\end{align*}
	We arrive at the desired:
	\begin{align}
		\transformerofof{\aflowconstraint_1\statemult\aflowconstraint_F}{\theinflow}
		~\fpreldot~
		\transformerofof{\aflowconstraint_2'\statemult\aflowconstraint_F'}{\theinflow}
		\label{proof:upward-closed-outflow:conclusion-transformer}
	\end{align}

	\medskip
	Observe that \eqref{proof:upward-closed-outflow:footprint-inflow-approx} immediately gives
	\begin{align}
		\aflowconstraint_2' ~\in~ \fpclosureof{\setnodes_F}{\aflowconstraint_2}
		\label{proof:upward-closed-outflow:conclusion-closure-footprint}
		\ .
	\end{align}
	It remains to argue for $\aflowconstraint_F'$.
	Recall that we have $\aflowconstraint_F.\inflow = \theinflow \uplus \restrictto{\aflowconstraint_1.\outflow}{\setnodes_1\times\setnodes_F}$ by \eqref{proof:upward-closed-outflow:multdef}.
	Moreover, we have $\aflowconstraint_F'.\inflow = \theinflow \uplus \restrictto{\aflowconstraint_2'.\outflow}{\setnodes_1\times\setnodes_F}$ by the definition of $\aflowconstraint_F'$.
	Hence, it suffices to show that the sum of inflow $\aflowconstraint_F$ receives from $\aflowconstraint_1$ is $\fprel$-related to the sum of inflow $\aflowconstraint_F'$ receives from $\aflowconstraint_2'$.
	To that end, consider some node $\anodep\in\setnodes_F$.
	Then, we conclude as follows:
	\begin{align*}
		&\sum_{\anode\in\nat\setminus\setnodes_F} \aflowconstraint_F.\inflow(\anode,\anodep)
		\\\explain{by definition}=~~&
		\sum_{\anode\in\setnodes_1} \aflowconstraint_1.\outflow(\anode,\anodep)
		~~~+\sum_{\anodepp\in\overline{\setnodes}} \aflowconstraint_F.\inflow(\anode,\anodep)
		\\\explain{Def. $\transformerof{\dontcare}$}=~~&
		\transformerofof{\aflowconstraint_1}{\aflowconstraint_1.\inflow}(\anodep)
		~~+\sum_{\anodepp\in\overline{\setnodes}} \aflowconstraint_F.\inflow(\anode,\anodep)
		\\\explain{by \eqref{proof:upward-closed-outflow:footprint-inflow-transformer-pre-vs-post}}\fprel~~&
		\transformerofof{\aflowconstraint_2'}{\aflowconstraint_2'.\inflow}(\anodep)
		~~+\sum_{\anodepp\in\overline{\setnodes}} \aflowconstraint_F.\inflow(\anode,\anodep)
		\\\explain{Def. $\transformerof{\dontcare}$}=~~&
		\sum_{\anode\in\setnodes_2'} \aflowconstraint_2'.\outflow(\anode,\anodep)
		~~~+\sum_{\anodepp\in\overline{\setnodes}} \aflowconstraint_F.\inflow(\anode,\anodep)
		\\\explain{by definition}=~~&
		\sum_{\anode\in\nat\setminus\setnodes_F'} \aflowconstraint_F'.\inflow(\anode,\anodep)
	\end{align*}
	By definition, this means:
	\begin{align}
		\aflowconstraint_F' ~\in~ \fpclosureof{\setnodes_2}{\aflowconstraint_F}
		\label{proof:upward-closed-outflow:conclusion-closure-frame}
		\ .
	\end{align}

	\medskip
	This concludes the proof, as \eqref{proof:upward-closed-outflow:conclusion-transformer}, \eqref{proof:upward-closed-outflow:conclusion-closure-footprint}, and \eqref{proof:upward-closed-outflow:conclusion-closure-frame} show the desired properties.
\end{proof}

% -----------------------------------------------------------------------------
% -----------------------------------------------------------------------------
% -----------------------------------------------------------------------------
% -----------------------------------------------------------------------------
\begin{proof}[Proof of \Cref{thm:upward-closed-framing}]
	\newcommand{\theinflow}{\inflow}
	\newcommand{\theinflowp}{\overline{\inflow}}
	\newcommand{\aflowconstraintp}{\hat{\aflowconstraint}}
	%
	Consider flow graphs $\aflowconstraint_1,\aflowconstraint_2,\aflowconstraint_F$ with $\aflowconstraint_1\statemultdef \aflowconstraint_F$ and assume $\aflowconstraint_1 \ctxfprel \aflowconstraint_2$ and $\fpcompatible[\fprel]{\aflowconstraint_1,\aflowconstraint_2,\aflowconstraint_F}$.
	Let $\setnodes_1 \defeq \aflowconstraint_1.\setnodes$ and $\setnodes_F \defeq \aflowconstraint_F.\setnodes$ and $\overline{\setnodes} \defeq \nat\setminus(\setnodes_1 \cup \setnodes_F)$ and $\theinflow \defeq (\aflowconstraint_1 \statemult \aflowconstraint_F).\inflow$.
	Observe $\aflowconstraint_2.\setnodes = \setnodes_1$ due to $\aflowconstraint_1 \ctxfprel \aflowconstraint_2$.
	Let $\edges_1\defeq\aflowconstraint_1.\edges$, $\edges_F\defeq\aflowconstraint_F.\edges$, and $\edges_2\defeq\aflowconstraint_2.\edges$.
	Invoke \Cref{thm:upward-closed-outflow} for $\aflowconstraint_1,\aflowconstraint_2,\aflowconstraint_F$ to obtain a flow graph $\aflowconstraint_{2+F}$ with:
	\begin{align}
		\aflowconstraint_{2+F} ~&=~ (\setnodes_1\uplus\setnodes_F,~ \edges_2\uplus\edges_F,~ \theinflow)
		\label{proof:upward-closed-framing:fg-2f-def}
		\\
		\restrictto{\aflowconstraint_{2\statemult F}}{\setnodes_1}\in\fpclosureof{\aflowconstraint_F}{\aflowconstraint_2}
		\quad&\text{and}\quad
		\restrictto{\aflowconstraint_{2\statemult F}}{\setnodes_F}\in\fpclosureof{\aflowconstraint_2}{\aflowconstraint_F}
		\label{proof:upward-closed-framing:fg-2f-closure}
	\end{align}
	Now, choose $\aflowconstraint_2' \defeq \restrictto{\aflowconstraint_{2+F}}{\setnodes_1}$ and $\aflowconstraint_F' \defeq \restrictto{\aflowconstraint_{2+F}}{\setnodes_F}$.
	By \Cref{thm:restriction-vs-statemult}, we have $\aflowconstraint_2'\statemultdef\aflowconstraint_F'$ and $\aflowconstraint_2'\statemult\aflowconstraint_F'=\aflowconstraint_{2+F}$.
	Furthermore, \eqref{proof:upward-closed-framing:fg-2f-closure} immediately gives $\aflowconstraint_2'\in\fpclosureof{\aflowconstraint_F}{\aflowconstraint_2}$ and $\aflowconstraint_F'\in\fpclosureof{\aflowconstraint_2}{\aflowconstraint_F}$.

	\medskip
	It remains to show $\transformerofof{\aflowconstraint_1\statemult\aflowconstraint_F}{\theinflowp} \fpreldot \transformerofof{\aflowconstraint_{2+F}}{\theinflowp}$, for all $\theinflowp\leq\theinflow$.
	Fix some $\theinflowp\leq\theinflow$.
	Define $\aflowconstraintp_1 = \restrictto{(\aflowconstraint_1\statemult\aflowconstraint_F)[\inflow\mapsto\theinflowp]}{\setnodes_1}$ and $\aflowconstraintp_F = \restrictto{(\aflowconstraint_1\statemult\aflowconstraint_F)[\inflow\mapsto\theinflowp]}{\setnodes_F}$.
	By definition:
	\begin{align}
		\aflowconstraintp_1 ~=~ (\setnodes_1,~ \edges_1,~ \theinflowp \uplus \restrictto{\aflowconstraintp_F.out}{\setnodes_F\times\setnodes_1})
		\label{proof:upward-closed-framing:fg-hh1-def}
		\quad\text{and}\quad
		\aflowconstraintp_F ~=~ (\setnodes_F,~ \edges_F,~ \theinflowp \uplus \restrictto{\aflowconstraintp_1.out}{\setnodes_1\times\setnodes_F})
		\ .
	\end{align}
	By \Cref{thm:restriction-vs-statemult} we have $\aflowconstraintp_1\statemultdef\aflowconstraintp_F$ and $\aflowconstraintp_1\statemult\aflowconstraintp_F=(\aflowconstraint_1\statemult\aflowconstraint_F)[\inflow\mapsto\theinflowp]$.
	Towards our proof goal, we apply \Cref{thm:upward-closed-outflow} for $\aflowconstraintp_1,\aflowconstraintp_F,\aflowconstraint_2$.
	Before we can do so, however, we have to show that the \namecref{thm:upward-closed-outflow} is applicable, i.e., that $\aflowconstraintp_1 \ctxfprel \aflowconstraint_2$ holds.

	From \Cref{thm:inflow-leq} and the choice of $\theinflowp$ we know $(\aflowconstraintp_1\statemult\aflowconstraintp_F).\fval \leq (\aflowconstraint_1\statemult\aflowconstraint_F).\fval$.
	Consequently, $\aflowconstraintp_F.\outflow \leq \aflowconstraint_F.\outflow$ by definition.
	This, in turn, means $\aflowconstraintp_1.\inflow \leq \aflowconstraint_1.\inflow$ by \eqref{proof:upward-closed-framing:fg-hh1-def}.
	Hence, for all $\inflow'\leq\aflowconstraintp_1.\inflow$, we have $\transformerofof{\aflowconstraint_1}{\inflow'}\fpreldot\transformerofof{\aflowconstraint_2}{\inflow'}$ by $\aflowconstraint_1\ctxfprel\aflowconstraint_2$ from the premise.
	By definition of $\transformerof{\dontcare}$, we obtain $\transformerofof{\aflowconstraintp_1}{\inflow'}\fpreldot\transformerofof{\aflowconstraint_2}{\inflow'}$ for all $\inflow'\leq\aflowconstraintp_1.\inflow$.
	That is, $\aflowconstraintp_1\ctxfprel\aflowconstraint_2$.

	Now, we are ready to apply \Cref{thm:upward-closed-outflow} to $\aflowconstraintp_1,\aflowconstraintp_F,\aflowconstraint_2$.
	We get $\aflowconstraintp_{2+F}$ with
	\begin{align}
		\aflowconstraintp_{2+F} ~&=~ (\setnodes_1\uplus\setnodes_F,~ \edges_2\uplus\edges_F,~ \theinflowp)
		\label{proof:upward-closed-framing:fg-h2f-def}
		\\
		\transformerofof{\aflowconstraintp_1\statemult\aflowconstraintp_F}{\theinflowp} ~&\fpreldot~ \transformerofof{\aflowconstraintp_{2+F}}{\theinflowp}
		\label{proof:upward-closed-framing:fg-h2f-transformer}
	\end{align}
	As noted earlier, we have $(\aflowconstraintp_1\statemult\aflowconstraintp_F)=(\aflowconstraint_1\statemult\aflowconstraint_F)[\inflow\mapsto\theinflowp]$.
	Moreover, \eqref{proof:upward-closed-framing:fg-2f-def} combined with \eqref{proof:upward-closed-framing:fg-h2f-def} gives $\aflowconstraintp_{2+F}=\aflowconstraint_{2+F}[\inflow\mapsto\theinflowp]$.
	Hence, \eqref{proof:upward-closed-framing:fg-h2f-transformer} yields $\transformerofof{\aflowconstraint_1\statemult\aflowconstraint_F}{\theinflowp}\fpreldot\transformerofof{\aflowconstraint_{2+F}}{\theinflowp}$, as required.
	This concludes $\aflowconstraint_1\statemult\aflowconstraint_F\ctxfprel\aflowconstraint_2'\statemult\aflowconstraint_F'$.
\end{proof}


% -----------------------------------------------------------------------------
% -----------------------------------------------------------------------------
% -----------------------------------------------------------------------------
% -----------------------------------------------------------------------------
\begin{proof}[Proof of \Cref{thm:easy-fprel-lfp}]
	Follows immediately by choosing $\amonval_i=f^i(\bot)$ and $\amonvalp_j=g^j(\bot)$.
\end{proof}


% -----------------------------------------------------------------------------
% -----------------------------------------------------------------------------
% -----------------------------------------------------------------------------
% -----------------------------------------------------------------------------
\begin{proof}[Proof of \Cref{thm:fprelcompatible-sub-omega-cpo}]
	Recall that $\fprel$ is a sub-$\omega$-cpo if $\fprel$
	\begin{inparaenum}
		\item is an $\omega$-cpo such that
		\item $\fprel \subseteq \leq$ and
		\item $\sup_\fprel(\achain)=\sup_\leq(\achain)$ for all $\fprel$-ascending chains $\achain=\amonval_0\fprel\amonval_1\fprel\cdots$.
	\end{inparaenum}
	(Note $\sup_\leq(M)=\bigjoin M$.)

	Now, consider functions $f,g:(\aflowconstraint.\setnodes{\to}\amonoid)\to(\aflowconstraint.\setnodes{\to}\amonoid)$ that are $\leq$-continuous and $\fpreldot$-monotonic such that we have $f^i(\bot)\leq f^{i+1}(\bot)$ and $g^i(\bot)\leq g^{i+1}(\bot)$ and $f^i(\bot)\fpreldot g^i(\bot)$ for all $i\in\nat$, as well as $\lfpof{f}\fpreldot g(\lfpof{f})$.
	Let $\achain=f^0(\bot)\leq f^1(\bot)\leq\cdots$ and $\achainp=g^0(\bot)\leq g^1(\bot)\leq\cdots$.
	By \Cref{thm:fixpoint-kleene}, $\lfpof{f}=\bigjoin\achain$ and $\lfpof{g}=\bigjoin\achainp$ exist.
	So we have:
	\begin{gather}
		\lfpof{f} ~=~ \bigjoin\achain ~\leq~ \bigjoin\achainp ~=~ \lfpof{g}
		\label{proof:fp-approx}
		\\
		\bigjoin\achain ~\leq~ g(\bigjoin\achain)
		\quad\text{and}\quad
		\bigjoin\achain ~\fpreldot~ g(\bigjoin\achain)
		\label{proof:g-iter-init}
	\end{gather}
	Moreover, we get $g^i(\bigjoin\achain) \fpreldot g^{i+1}(\bigjoin\achain)$, for all $i\in\nat$, because $g$ is $\fpreldot$-monotonic.
	As a consequence, $\achainp'=g^0(\bigjoin\achain)\fpreldot g^1(\bigjoin\achain)\fpreldot\cdots$ is a $\fpreldot$-ascending chain.
	Because $\fprel$ is an $\omega$-cpo, this means $\bigjoin\achainp'$ exists.
	By definition, we obtain $\bigjoin\achain=g^0(\bigjoin\achain)\fpreldot\bigjoin\achainp'$.
	To conclude the overall claim, it now suffices to show that $\bigjoin\achainp'=\bigjoin\achainp$ holds.
	This, in turn, holds if $\lfpof{g}=\bigjoin\achainp'$.

	\medskip
	We first show that $\bigjoin\achainp'$ is a fixed point of $g$.
	\begin{align*}
		&~g(\bigjoin\achainp')
		\\\explain{Def. $\achainp'$}=&~
		g(\bigjoin\setcond{g^i(\join\achain)}{i\in\nat})
		\\\explain{$g$ continuous}=&~
		\bigjoin\setcond{g^{i+1}(\join\achain)}{i\in\nat}
		\\\explain{by \eqref{proof:g-iter-init}}=&~
		\bigjoin\setcond{g^{i+1}(\join\achain)}{i\in\nat} \join \bigjoin\achain
		\\\explain{$g^0(\join\achain)=\join\achain$}=&~
		\bigjoin\setcond{g^i(\join\achain)}{i\in\nat}
		\\\explain{Def. $\achainp'$}=&~
		\bigjoin\achainp'
		\ .
	\end{align*}

	\medskip\newcommand{\dd}{\cval^\dagger}
	We now show that $\bigjoin\achainp'$ is the least fixed point of $g$.
	To that end, consider another fixed point $\dd$ of $g$, i.e., $g(\dd)=\dd$.
	It suffices to show that $g^i(\bigjoin\achain) \leq \dd$ holds for all $i$, because this implies that the join over the $g^i(\bigjoin\achainp')$ is at most $\dd$.
	We proceed by induction.
	In the base case, $i=0$, we have \[
		g^0(\join\achain)
		~=~
		\bigjoin\achain
		~\explains{\eqref{proof:fp-approx}}{=}~
		\lfpof{g}
		~\leq~
		\dd
	\]
	where the last approximation holds by the definition of $\lfpof{g}$ together with the fact that $\dd$ is a fixed point of $g$.
	For the induction step, assume $g^i(\bigjoin\achain)\leq\dd$.
	We have \[
		g^{i+1}(\join\achain) ~=~ g(g^{i}(\join\achain)) ~\leq~ g(\dd) ~=~ \dd
	\]
	where the approximation is by induction together with $g$ begin $\leq$-monotonic by \Cref{thm:continuous-implies-monotonic-special} and the last equality is by the fact that $\dd$ is a fixed point of $g$.
	Altogether, we conclude the desired equality: $\lfpof{g}=\bigjoin\setcond{g^i(\bigjoin\achain)}{i\in\nat}$.

	Overall, we conclude the desired $\bigjoin\achain\fpreldot\bigjoin\achainp'=\lfpof{g}=\bigjoin\achainp$.
\end{proof}


% -----------------------------------------------------------------------------
% -----------------------------------------------------------------------------
% -----------------------------------------------------------------------------
% -----------------------------------------------------------------------------
\begin{proof}[Proof of \Cref{thm:fprelcompatible-acc}]
	Recall that $(\amonoid,\leq)$ satisfies the ascending chain condition if every chain $\leq$-ascending chains $\achain=\amonval_0\leq\amonval_1\leq\cdots$ become stationary, that is, there is some $i\in\nat$ such that $\amonval_{i}=\amonval_{i+j}$ holds for all $j\in\nat$.

	Now, consider $\leq$-ascending chains $\achain=\amonval_0\leq\amonval_1\leq\cdots$ and $\achainp=\amonvalp_0\leq\amonvalp_1\leq\cdots$ with $\amonval_k\fprel\amonvalp_k$ for all $k\in\nat$.
	Let $i\in\nat$ such that $\forall i'\in\nat.~ \amonval_i=\amonval_{i+i'}$.
	Let $j\in\nat$ such that $\forall j'\in\nat.~ \amonvalp_j=\amonvalp_{j+j'}$.
	By the ascending chain condition, $i$ and $j$ exist.
	Choose $k=\max(i,j)$.
	Then, $\forall i'\in\nat.~ \amonval_k=\amonval_{k+i'}$ and $\forall j'\in\nat.~ \amonval_k=\amonvalp_{k+j'}$.
	This means that $\bigjoin\achain=\amonval_k$ and $\bigjoin\achainp=\amonvalp_k$.
	By assumption, $\amonval_k\fprel\amonvalp_k$.
	Hence, $\bigjoin\achain\fprel\bigjoin\achainp$.
	Then, \Cref{thm:easy-fprel-lfp} establishes \eqref{def:fpcompat:lfp}.
\end{proof}


% -----------------------------------------------------------------------------
% -----------------------------------------------------------------------------
% -----------------------------------------------------------------------------
% -----------------------------------------------------------------------------
\begin{proof}[Proof of \Cref{thm:trivial-choices}]
	Property $\fpcompatible[=]{\aflowconstraint}$ is trivially true.
	We show $\fpcompatible[\leq]{\aflowconstraint}$.
	Because $\leq$ is the natural order, we immediately have \eqref{def:fpcompat:refl-trans}.
	\Cref{thm:our-monoid-has-addprop} gives \eqref{def:fpcompat:add}.
	Because edge functions are $\leq$-continuous, they are also $\leq$-monotonic by \Cref{thm:continuous-implies-monotonic-special}.
	This is \eqref{def:fpcompat:edges}.
	Finally, it is easy to see that $\leq$ is a sub-$\omega$-cpo of itself, so \eqref{def:fpcompat:lfp} follows from \Cref{thm:fprelcompatible-sub-omega-cpo}.
\end{proof}
