\section{Appendix}
\subsection{Further Analysis on Different Delay Time}
An interesting phenomenon in Figure.\ref{delay_interval} is that the Vanilla model performs better than the Oracle model on samples with short delays (G1). It may be because samples with short delays have a higher percentage of observed positive samples than actual positive samples.
Specifically, the proportion of G1 to the observed positive samples is 25.8\%, which is higher than that of G1 to the true positive samples, 21.4\%. Thus, the Vanilla model focuses more on these short-delay samples and learns better about them. We analyze the training samples with different delay time in the same way as above. As shown in Fig.\ref{delay_time_train}, the Vanilla model indeed learns better on the short-delay samples of training data than the Oracle model. Moreover, we can also observe that the samples with a longer delay time on the training set have a larger training loss for the Oracle model. As the number of true positive samples is the same for these groups, this also indicates that samples with longer delays are more likely to be hard samples.

% Figure environment removed

\subsection{Case Study}
Here are two concrete cases on the Criteo dataset with MLP as the backbone. For each test sample, we find the ten most similar samples in the training set, as the label correctness of these training samples might have a strong impact on the prediction of the test sample. We calculate similarity using the L2 distance of sample embeddings in our model. As shown in Table \ref{case1} and \ref{case2}, it can be found that our method effectively corrects the labels of false negative samples without wrongly categorizing true negative samples as positive ones. Consequently, the prediction accuracy of the test sample is enhanced.

\begin{table}[h]
\centering
\caption{The first concrete case on the Criteo dataset with MLP as the backbone. The false negatives are in boldface. Our method effectively corrects the labels of false negative samples (5063626, 5630493). Meanwhile, the true negative samples (1738500, 904961, 3478664) are not corrected to positive samples.}
\label{case1}
\resizebox{0.45\textwidth}{!}{%
\begin{tabular}{cccc}
\toprule \toprule
\textbf{Test Sample ID}                                                               & \textbf{Label}      & \textbf{Vanilla Pred.}  & \textbf{ULC(ours) Pred.} \\ \midrule
\textbf{186833}                                                                       & 1          & 0.081          & 0.787           \\ \midrule \midrule
\begin{tabular}[c]{@{}c@{}}\textbf{Ten Most Similar} \\ \textbf{Training Samples}\end{tabular} & \textbf{True Label} & \textbf{Observed Label} & \textbf{Corrected Label} \\ \midrule
\textbf{5063626}                                                             & \textbf{1} & \textbf{0}     & \textbf{0.906}  \\
\textbf{4136975}                                                             & 1          & 1              & 1               \\
\textbf{3726489}                                                             & 1          & 1              & 1               \\
\textbf{1738500}                                                             & 0          & 0              & 0.000           \\
\textbf{5630493}                                                             & \textbf{1} & \textbf{0}     & \textbf{0.821}  \\
\textbf{904961}                                                              & 0          & 0              & 0.023           \\
\textbf{1980991}                                                             & 1          & 1              & 1               \\
\textbf{3424953}                                                             & 1          & 1              & 1               \\
\textbf{3478664}                                                             & 0          & 0              & 0.044           \\
\textbf{5645863}                                                             & 1          & 1              & 1      \\ \bottomrule \bottomrule     
\end{tabular}%
}
\end{table}

\begin{table}[h]
\centering
\caption{The second concrete case on the Criteo dataset with MLP as the backbone. The false negatives are in boldface. Our method effectively corrects the labels of false negative samples (5782400, 5796674, 5801703). Meanwhile, the true negative samples (remaining part) are not corrected to positive samples.}
\label{case2}
\resizebox{0.45\textwidth}{!}{%
\begin{tabular}{cccc}
\toprule \toprule
\textbf{Test Sample ID}                                                               & \textbf{Label}                    & \textbf{Vanilla Pred.}            & \textbf{ULC(ours) Pred.}              \\ \midrule
\textbf{146531}                                                                                & 1                                 & 0.061                             & 0.790                                 \\ \midrule \midrule
\textbf{\begin{tabular}[c]{@{}c@{}}Ten Most Similar \\ Training Samples\end{tabular}} & \textbf{True Label}               & \textbf{Observed Label}           & \textbf{Corrected Label}              \\ \midrule
\textbf{5782400}                                                                      & {\color[HTML]{2C3A4A} \textbf{1}} & {\color[HTML]{2C3A4A} \textbf{0}} & {\color[HTML]{2C3A4A} \textbf{0.538}} \\
\textbf{5796674}                                                                      & {\color[HTML]{2C3A4A} \textbf{1}} & {\color[HTML]{2C3A4A} \textbf{0}} & {\color[HTML]{2C3A4A} \textbf{0.677}} \\
\textbf{5801703}                                                                      & {\color[HTML]{2C3A4A} \textbf{1}} & {\color[HTML]{2C3A4A} \textbf{0}} & {\color[HTML]{2C3A4A} \textbf{0.616}} \\
\textbf{2648493}                                                                      & 0                                 & 0                                 & 0.002                                 \\
\textbf{4545673}                                                                      & 0                        & 0                        & 0.012                        \\
\textbf{686762}                                                                       & 0                                 & 0                                 & 0.000                                 \\
\textbf{214715}                                                                       & 0                                 & 0                                 & 0.000                                 \\
\textbf{5184723}                                                                      & 0                                 & 0                                 & 0.007                                 \\
\textbf{602570}                                                                       & 0                                 & 0                                 & 0.000                                 \\
\textbf{5382003}                                                                      & 0                                 & 0                                 & 0.013     \\ \bottomrule \bottomrule                           
\end{tabular}%
}
\end{table}
