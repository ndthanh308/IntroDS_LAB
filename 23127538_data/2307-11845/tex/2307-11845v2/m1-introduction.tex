\section{Introduction}
\label{sec:intro}

The banking sector is an integral part of the global economy and plays a critical role in shaping financial growth and stability. In recent years, the banking industry has experienced increasing competition from FinTech companies and the adoption of innovative technologies to drive operational efficiency. Optimizing the wide range of banking processes is an integral part of the digital transformation of banks to remain competitive in a rapidly evolving environment.

Management of the diverse document landscape within core processes, which involve documents of multiple types, structures, and languages, is a key part of banking operations. Advanced document analytics techniques promise to accelerate data extraction, reduce manual processing, and improve the quality and efficiency of decision-making.

The emergence of large foundational language models, such as GPT-4, currently characterizes the field of Natural Language Processing (NLP). This dynamic field currently resembles a ``race of models", with new Deep Neural Network architectures for language processing being published frequently. 
Rather than getting caught up in a battle of benchmarks, this study shifts the focus to real-world applications, specifically, the potential of document analytics to increase efficiency. Some document analytics models provide capabilities beyond text processing. These so-called multimodal models handle documents in which both textual and visual information is crucial for understanding \citep{oral2020}. Examples are scanned invoices \citep{baviskar2021}, money transfer orders forms \citep{oral2020, oral2022} or financial reports with tables and images \citep{kamaruddin2015, pejic2019, dong2023}.

The objectives of this research cover three areas. First, we investigate the diversity and characteristics of the banking document landscape. Next, we intend to develop and validate a proof-of-concept for document analytics, exploring textual and visual feature incorporation.
Lastly, we aim to benchmark our proposed document analytics framework against a ready-to-use large language model, specifically GPT, to elucidate the strengths and limitations of general-purpose models in the complex field of banking document analytics.


Our study offers pivotal contributions to the OR in banking literature. Through our detailed examination of the customer banking document landscape, we reveal unexplored areas for efficiency gains. Further, we demonstrate the importance of including visual features in document analytics models, specifically when using LayoutXLM. Our comparative analysis of LayoutXLM with models such as BERT and GPT provides novel insights into the power of multimodal models in banking contexts.
Addressing real-world obstacles, in particular class imbalance, we uncover a reduced dependency on exhaustive labeled data for deploying pre-trained document analytics models. Finally, we stress the urgency for banking institutions to re-evaluate their analytical paradigms, suggesting a shift towards more advanced frameworks.



