\section{Related Work}
\label{sec:related}
This section surveys related studies and sheds light on the state of affairs in two interconnected research areas: NLP applications in banking and the emerging field of multimodal document analytics. 
A list of related studies in both areas including relevant document types in banking operations is summarized in Table \ref{tab:literature-applications}.




\begin{table}[htbp]
  \centering
  \caption{Applications of NLP and Document Analytics in Banking}
  \label{tab:literature-applications}
  \setlength\tabcolsep{5pt}
  \renewcommand{\arraystretch}{1.2}
  \scriptsize
    \begin{tabularx}{\textwidth}{p{90px}Xp{100px}}
    \toprule
    \textbf{Study} & \textbf{Main Topic} & \textbf{Highlighted Documents} \\

    \midrule
    \addlinespace[1ex]
    \multicolumn{3}{l}{\textbf{Natural Language Processing in Banking}} \\
    \addlinespace[1ex]
    \citet{li2011} & Survey of text analysis techniques for corporate disclosures & financial statements, earnings releases, and conference call transcripts   \\
    \citet{Chaturvedi2014} & Sentiment analysis over banking services using online reviews & -\\
    \citet{kamaruddin2015} & Text mining framework for financial statement deviation detection & financial statements  \\
    \citet{kumar2016} & Text mining applications in finance (e.g., FOREX rate prediction) & - \\
    \citet{aureli2017} & Content analysis and text mining for companies' reputation restoration & social and environmental disclosures \\
    \citet{sumathi2017} & Opinion mining using sentiment analysis on social media texts & - \\
    \citet{pejic2019} & Overview of text mining applications in the financial sector & financial statements and legal documents  \\
    \citet{oral2019} & Extract transactions from banking documents using a BiLSTM network & banking orders \\
    \citet{lewis2019} & Highlight the importance of optimizing corporate reports analysis with NLP & balance sheet, income statement, statement of shareholders' equity, and statement of cash flows \\
    \citet{gupta2020} & Review of text mining applications in finance & transaction orders, annual reports, and financial statements \\
    \citet{decaigny2020} & Improving customer churn prediction models using electronic text messages between client and financial advisor & - \\
    \citet{baviskar2021} & Review of AI methods for automated processing of unstructured documents and mention of applications (also in banking) & invoices, passports, ID-cards, diverse application forms, and legal contracts \\
    \citet{sokolov2021} & Transformer-based ESG scoring using social media texts & - \\
    \citet{dong2023} &Keyword extraction to construct an ESG scoring system & financial reports\\
    %Nguyen et al. (2021) & Transformer model for fine-grained NER for Japanese business documents & bidding and sale documents & (X) & X & \\
    \addlinespace[1ex]
    \midrule
    \addlinespace[1ex]
    \multicolumn{3}{l}{\textbf{Multimodal Document Analytics in Banking}} \\
    \addlinespace[1ex]
    \citet{engin2019} & Multimodal deep neural networks for classification of Turkish banking documents using textual and visual features & unspecified banking order documents  \\
    \citet{oral2020} & Information extraction from banking documents using deep learning algorithms and word positional features & money transfer orders  \\
    \citet{oral2022} & Impact of different fusion techniques on information extraction from unstructured documents & money transfer orders  \\
    \bottomrule
    \end{tabularx}%
\end{table}


\subsection{NLP Applications in Banking}



In recent years, there has been an increasing number of studies in the banking sector deploying NLP techniques. These studies, which often fall under ``text mining", analyze textual banking data and gain insights through classification, clustering, information extraction, or sentiment analysis \citep{gupta2020}.
Literature reviews \citep{kumar2016, fisher2016, pejic2019, gupta2020, baviskar2021} aggregate different applications of NLP in banking and identify opportunities to use unstructured textual data.

\citet{doumpos2023} categorize AI banking research into some core areas, including bank efficiency and risk management, as well as customer-related insights \citep{doumpos2023}. In the following paragraphs, we dive into these core areas to illustrate NLP applications in banking.

\paragraph{Bank Efficiency} Several studies investigate the role of NLP in improving bank efficiency, defined as achieving the best possible results with the least amount of inputs and costs \citep{doumpos2023}. Bank efficiency can be seen from two perspectives: \textit{cost efficiency}, which seeks to minimize inputs, and \textit{profit efficiency}, which aims to increase outputs \citep{tecles2010}.

The analysis of financial statements and corporate reports, for example, through information extraction or named entity recognition (NER), is a major field where NLP demonstrates efficiency increases. Studies such as \citet{li2011} and \citet{kamaruddin2015} propose the use of NLP to analyze financial statements, helping to optimize processes and reduce costs. 
\citet{lewis2019} underscore NLP's potential to handle massive amounts of data in corporate reporting, uncovering latent attributes and tackling information overload, thus optimizing efficiency. These text-driven approaches can be applied not only to financial statements and corporate reports, but also to various other document types, as discussed in more detail in the section \ref{sec:multimod}.

Customer service also benefits from NLP-enhanced cost efficiency:
Context-aware chatbots and conversational AI provide permanent customer service, handle complex transactions, and adapt to changing customer needs, thus optimizing process efficiency and improving customer experience \citep{suhel2020, mogaji2021, petersson2023}.

On the profit side, NLP enhances revenue streams, particularly in the area of personalized marketing.
Textual data contains insights that can streamline marketing initiatives and personalized product recommendations \citep{pejic2019, chen2020}. For example, \citet{schmitz2023} support stock trading decisions to increase profits using textual data from corporate disclosures.
Furthermore, text analyses on social media can provide more personalized services \citep{sun2018, chen2019}. This is further extended by techniques such as Company2Vec to identify other companies for customer acquisition \citep{gerling2023}. These NLP systems contribute to the offering of customized banking services, leading to business and revenue growth.


\paragraph{Risk Management} Effective risk management is a cornerstone in banking. It requires a complex understanding and accurate quantification of operational, market and credit risk \citep{doumpos2023}. Many studies use NLP techniques to improve risk identification and assessment, using various textual sources such as trade orders, annual reports, and financial statements \citep{gupta2020}. 

For risk assessment and default prediction, text analytics provides innovative methods to improve predictive accuracy. \citet{stevenson2021} develop a combined small business default prediction model using traditional and textual features, highlighting the use of BERT, a transformer-based model known for its success in a variety of NLP tasks, for their textual analysis.  
Similarly, \citet{jiang2018} and \citet{mai2019} improve loan default and bankruptcy prediction by including textual information from descriptive texts or company disclosures.
Furthermore, \citet{netzer2019} find empirical evidence that borrowers who are likely to default use a specific language in their loan application. Including these textual features in a combined model increases default prediction performance.

NLP also seems promising in the area of transaction fraud detection \citep{kotios2022}.
\citet{kotios2022} demonstrate text-based fraud detection, by identifying fraudulent activities in transactions.
In this context, \citet{rodriguez2022} develop a transformer-based model, which outperforms traditional methods in several fraud scenarios.
Furthermore, \citet{craja2020} propose a hierarchical attention network that combines information from financial ratios and managerial comments within corporate annual reports to detect statement fraud.


Financial forecasting is another dimension where NLP is increasingly gaining ground, enhancing the understanding of market risk and volatile underlyings \citep{kumar2016}.
\citet{xing2018} review the emerging field of NLP-based financial forecasting, emphasizing the growing ability of NLP to improve financial market predictions. 
There are advances like Stock2Vec \citep{dang2018}, a two-stream deep learning model for short-term stock forecasting using newspaper articles in combination with historical stock prices. 
Similarly, \citet{li2021} employs an LSTM-based model for stock prediction using online news and fundamental data.


\paragraph{Customer-related Studies} This category includes studies that aim to understand customer behavior without an immediate impact on cost reductions or revenue increases. These additional insights play a critical role in understanding competition, refining product quality, improving customer satisfaction, and thus aligning banking strategies.

In this area, prevention of customer churn and early identification of potential customer loss are of great interest. \citet{decaigny2020} use text messages between financial advisors and clients to enrich customer churn prediction models, achieving the best performance using a combination of textual and structured features within convolutional neural networks. Such models can greatly support customer retention strategies.   

Sentiment analysis and opinion mining also have a considerable impact on banking. \citet{Chaturvedi2014} use automated sentiment analysis to extract customer opinions from bank reviews and to provide insight into strengths and weaknesses of particular areas of a bank (e.g., customer service). Meanwhile, \citet{sumathi2017} use social media text to create a sentiment index for decision support, highlighting the influence of public sentiment on banking strategies.

NLP also assesses companies' environmental and social behavior (ESG) through textual analyses. For example, \citet{aureli2017} applies content analysis to explore firms' reputation restoration strategies after industrial disasters. Then \citet{dong2023} uses keyword extraction from companies' annual financial reports to construct an ESG scoring system. Extending this, \citet{sokolov2021} use BERT to automate ESG scoring from unstructured social media contents, thereby improving ESG risk assessment and creating semi-autonomous ESG rating systems.




\subsection{Multimodal Document Analytics in Banking}
\label{sec:multimod}

NLP research mentions the potential for analyzing banking documents, from traditional financial documents such as financial statements and earnings releases \citep{lewis2019, li2011} to more operationally focused documents such as bank (transaction) orders \citep{oral2019, gupta2020}, ID cards, legal contracts \citep{baviskar2021} or social and environmental disclosures \citep{aureli2017} (more examples in Table \ref{tab:literature-applications}).

Previous research mainly considered documents as ordered collections of word tokens. However, the recent literature perceives banking documents as multidimensional entities that encompass visual elements and layout details \citep{engin2019, oral2020, oral2022, tavakoli2023}.
This perspective acknowledges the variety of visually rich, often multipage, documents that are heavily used in the banking industry. New methods include positional information and incorporate image pixels from the document pages into the model (to be discussed in Section \ref{sec:methodology}).
The benefit of analyzing visually rich documents to improve business operations has been demonstrated in closely related industries such as insurance, where damage claim processes can be optimized \citep{levich2023}.
This example illustrates a transition towards a new era of multimodal document analysis, also highlighting the importance of leveraging diverse banking documents.
 


\paragraph{Research on Multimodal Models and Document Analytics in Banking}
In the field of multimodal models and document analytics for the banking sector, some pioneering studies have been carried out. \citet{engin2019} explore a multimodal deep neural network for the classification of Turkish banking documents using both textual and visual features. 
They categorize document layout structures into free-form text, large tables, custom forms, and forms predefined by specific organizations \citep{engin2019}. Furthermore, \citet{oral2020} demonstrate the power of deep learning algorithms and word position features in extracting information from banking documents, specifically money transfer orders. \citet{oral2022} take a more technical approach by examining the impact of different fusion techniques on the extraction of information from unstructured documents, again focusing on money transfer orders.

However, research on multimodal document analytics in banking remains limited, often focusing on specific document types. This leaves a large number of potential applications unexplored. A wider range of document types and applications for multimodal document analytics in banking is yet missing.
On a side note, even the most recent multimodal models, such as GPT-4, remain unexplored for their potential in banking document analytics.


Our study seeks to fill this gap by first reviewing the various types of banking documents. The goal is to build a bridge between the methodological literature and practical banking applications, a connection that is currently underrepresented. In particular, our objective is to assess the layout variability and automation potential of banking documents, aspects that are rarely addressed in previous works.
Subsequently, we propose a robust multimodal document analysis method using LayoutXLM on German company register extracts. To the best of our knowledge, this is the first study that leverages a pre-trained multimodal model to conduct document analytics in the banking sector. This includes a detailed evaluation to address challenges such as class imbalance, learning curve analyses, and component-wise ablations.
Through these efforts, our study aims not only to deepen the understanding of multimodal document analysis in banking, but also to pave the way for future research in this promising field.