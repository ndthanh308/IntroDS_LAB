\section{Introduction}
\label{sec:intro}

The banking sector is an integral part of the global economy and plays a critical role in shaping financial growth and stability. In recent years, the banking industry is marked by the rise of competitive FinTech disruptors and the adoption of innovative technologies to drive operational efficiency. Optimizing the wide range of banking processes is indispensable in their digital transformation to maintain competitiveness.

Key to banking operations is the management of diverse documents within core processes, involving multiple types, structures, and languages. Here, advanced document analytics techniques promise to accelerate data extraction, reduce manual processing, and facilitate decision-making.

The rapid evolution of large foundational language models such as GPT-4 currently characterizes the field of Natural Language Processing (NLP). This dynamic field currently resembles a ``race of models", with new Deep Neural Network architectures for language processing being published frequently. % \cite{Ahmed2023}
Rather than getting caught up in a battle of benchmarks, this study shifts the focus to real-world applications, specifically the potential for efficiency increases. Some document analytics models have extended capabilities beyond text processing. These so-called multimodal models handle documents where both textual and visual information are crucial for understanding \cite{oral2020}. Examples are scanned invoices \cite{baviskar2021}, money transfer orders forms \cite{oral2020, oral2022} or financial reports with tables and images \cite{kamaruddin2015, pejic2019, dong2023}.

The main objectives of this research are twofold: First, we investigate the diversity and characteristics of the banking document landscape and assess potential efficiency gains in the document-intensive customer business. Second, we seek to establish a proof-of-concept for multimodal document analytics using LayoutXLM, a cross-lingual multimodal pre-trained model, to analyze textual and visual document information.

Our study contributes to the banking literature by exploring the complexity of the customer banking document landscape, highlighting potential efficiency increases through advanced document analytics. We seek to uncover factors that contribute to the performance of multimodal document analytics models and advocate for visual feature integration. In addition, we address key issues such as class imbalance and evaluate the practical feasibility of using LayoutXLM through a learning curve analysis. Finally, this study aims to raise awareness of state-of-the-art document analytics frameworks within banking, underscoring practical benefits for process efficiency.



