\section{Documents in Banking}
\label{sec:bankingdocs}

Beyond the technical aspects of foundational document analytics models, practical applications with high business value are crucial. Banking documents, distributed across multiple languages and layouts, align well with the multimodal, cross-lingual capabilities of LayoutXLM. To fully understand the potential of document analytics in banking, we now examine the complexity of its document landscape.


\subsection{Structure of Commercial Banking Business}
\label{sec:structure-banking}
The general structure of the commercial banking business is illustrated in Figure \ref{fig:banking-overview}. This diagram provides an overview of the key areas within a typical bank. Banking activities are first divided into the core business and other internal operations, which are not directly related to banking-specific departments (e.g., human resources).

Although internal operations and proprietary trading are integral components of banks' activities, they are outside the scope of this study. The focus is on customer-centric core operations. While other internal operations within banks are indeed document-intensive, the specificity of focusing on customer-facing processes provides a deeper understanding of the unique characteristics of the banking industry. This approach differentiates our study from more general document analytics research and allows us to directly impact process efficiency, risk management, and customer-related studies.



% Figure environment removed


This study explores the core business branch of the tree, specifically the customer business, which can be further subdivided into typical banking areas. The proposed classification for these areas in banking provides a framework for understanding the activities of a commercial bank.
\citeay{casu2015introduction} highlight typical components of the core customer business, including payment services, deposits, investments, lending services, e-banking, and additional financial services (e.g., insurance or trade finance). To account for contextual variations in banking activities and subprocesses, and to better align with our research focus, we propose a refined categorization:


\begin{enumerate}
    \item \textbf{Client Relationship Management.} In this domain, the focus lies on managing the bank-customer relationship, including the know-your-customer process and general customer requests.
    
    \item \textbf{Account and Transaction Services.} This segment includes the management of checking accounts and the facilitation of payment transactions. 
    
    \item \textbf{Investment and Savings Solutions.} Within this area, a diverse range of savings accounts and investment services are offered, including stock trading.
    
    \item \textbf{Credit and Lending Services.} This category encapsulates all activities related to the provision of loans and other forms of credit business. For corporate clients, this also includes specialized financing solutions such as factoring and leasing.
    
    \item \textbf{Trade and Guarantee Services.} This area covers additional or specialized financial services such as trade finance and guarantee business, which have unique document management requirements.
\end{enumerate}

While some banks may offer additional financial and non-financial services (e.g., insurance), these services are not considered separately in this classification since they can either be embedded in one of the presented areas or fall outside the scope of core banking services.

Our contextual framework provides a clear perspective on the core business activities of commercial banks, particularly with regard to document analytics. Each area represents a distinct domain of document-intensive processes within core banking services. It facilitates a focused exploration of the role and potential of multimodal document analytics within these domains. Therefore, it serves as the guiding principle for our investigation.




\subsection{Assessment and Categorization of Banking Documents}
In order to effectively assess the characteristics of banking documents from the presented areas, we employ certain assessment and categorization criteria. The aim is to provide a comprehensive overview to enhance the understanding of each document type and its potential for process automation.

In the course of commercial core banking, each operation or process typically requires specific documents. We have undertaken a collection of various document types to cover a wide range of banking areas. For instance, the consumer loan approval process might entail documents such as income proofs and credit ratings.
Given the variety of documents in each area of commercial banking, we conduct an evaluation for each document type. A systematic review of relevant literature, process documentation, practice guides, web resources, and consultations with banking domain experts provides a solid basis for this assessment. Each document type is evaluated against three key criteria:

\begin{itemize}

\item \textbf{Multimodality Assessment.} As proposed by \citeay{engin2019}, different structures within documents range from free-formatted texts to highly standardized forms. In our assessment, we consider the presence of unstructured texts, tabular data or forms, and figures or images. The reason for this diversity consideration is its impact on our approach to document analytics. For example, documents that primarily consist of free text might only require the application of unimodal NLP models, and documents that are composed of images may be best served by traditional unimodal image classifiers. However, those with a mixture of text, forms, and images may require multimodal methods such as LayoutXLM.

\item \textbf{Automation Potential.} Each document is analyzed based on its potential for automation, indicating the expected impact and ease of automating its processing. This assessment takes into account the current manual effort required to process each document and the possible benefits that could be derived from automation like information extraction. Starting document analytics with use cases that have a high automation potential can ensure efficient resource allocation and maximum return on investment.

\item \textbf{Layout Variability.} This criterion indicates how much the appearance of a particular document type can vary. If a document type always follows a strict format and appearance, its layout variability is considered low, making it easier to process. Conversely, a document type that appears in multiple forms has a high layout variability, potentially making its processing more challenging.
The assessment of layout variability is based on an examination of multiple examples of each document type. 

\end{itemize}



Lastly, while the additional assessment of \textit{data privacy} could have been a separate criterion, we observe that almost all banking documents contain sensitive information resulting in strict privacy and security measures. Therefore, we emphasize the importance of data privacy as a critical consideration across all document types. However, some types of documents, such as company register extracts, may be public and less sensitive. 

These assessment criteria provide a framework for understanding the current landscape of banking documents and the opportunities they present for automation and decision support. This paves the way for the effective adoption of document analytics in commercial banking operations.





\subsection{Results and Applications of Document Analytics in Banking}
Table \ref{tab:doc_assessment} provides an overview of various document-intense banking processes and the corresponding assessment of multimodality, automation potential, and layout variability. 
These processes are covered by the five customer-centric core-banking areas we previously defined in \ref{sec:structure-banking}. It should be noted that this presentation, while comprehensive, might not be exhaustive. It rather showcases representative examples of common processes in banks.




\begin{landscape}
\begin{table}[h]
\centering
\caption{Assessment of Banking Documents}
\label{tab:doc_assessment}
\setlength\tabcolsep{12pt} % Makes the columns wider
\begin{scriptsize} % Makes the text within the table even smaller
\begin{tabular}{lp{5.5cm}ccccc}
\toprule
\textbf{Area} & \textbf{Processes and Documents} & \multicolumn{3}{c}{\textbf{Multimodality}} & \textbf{Automation} & \textbf{Layout} \\
 & & \textit{Unstructured} & \textit{Tabular} & \textit{Figures or} & \textbf{Potential} & \textbf{Variability} \\
 & & \textit{Texts} & \textit{Data/Forms} & \textit{Images} & & \\
\midrule
\addlinespace[1ex]

\multirow{11}{*}{Client Relationship Management} & \textbf{Onboarding Process} &  &  & &  &  \\
 & \hspace{1em}\textit{Identification Documents} & &  &  &  &  \\
 & \hspace{2em}ID Card / Passport & & X & X & High & Low \\
 & \hspace{2em}Proof of Address & & X &  & High & Low \\
 & \hspace{2em}Company Register Extract & X & X & (X) & High & Low \\
& \hspace{2em}Articles of Association & X & & & High & Medium \\
%& \hspace{2em}Company Registration Certificate & X & & & High & Medium \\
& \hspace{2em}Organizational Chart & X & & X & High & High \\
 & \hspace{1em}KYC Form & & X & & High & Low \\
\addlinespace[1ex]
 & \textbf{Customer Service} &  &  & &  &  \\
 & \hspace{1em}Textual Service Requests/Complaints & X & & & High & High \\
 & \hspace{1em}Product Orders or Cancellation & X & (X) & & High & High \\
\midrule
\addlinespace[1ex]

\multirow{19}{*}{Account and Transaction Services} & \textbf{Account Opening Process} &  &  & &  &  \\
 & \hspace{1em}Account Opening Contract & X & X & & Medium & Medium \\
 & \hspace{1em}Authorization Form & & X & & Medium & Medium \\
 & \hspace{1em}Signature Specimen & & & X & Low & High \\
 \addlinespace[1ex]
 & \textbf{Transaction Process} &  &  & &  &  \\
 & \hspace{1em}Standard Payment Form & & X & & High & Low \\
 & \hspace{1em}Informal Payment Order & X & & & Low & High \\
\addlinespace[1ex]
 & \textbf{Account Closure Process} &  &  & &  &  \\
 & \hspace{1em}Account Closure Form & & X & & Medium & Low \\
 & \hspace{1em}Certificates of Inheritance & X & X & & Medium & High \\
 & \hspace{1em}Switch Bank Orders & X & X & & High & Medium \\
\addlinespace[1ex]
 & \textbf{Others} &  &  & &  &  \\
 & \hspace{1em}Account Statement & & X & & High & Medium \\
 & \hspace{1em}Balance Confirmation & & X & & High & Medium \\
 & \hspace{1em}Garnishment Order & X & X & & High & Medium \\
  & \hspace{1em}Invoices & & X & & High & High \\
\midrule
\addlinespace[1ex]

\multirow{11}{*}{Investment and Savings Solutions} & \textbf{Standard Savings Process} &  &  & &  &  \\
 & \hspace{1em}Savings Certificates / Passbook & & X & & Medium & Medium \\
 & \hspace{1em}Deposit Slip & & X & & Medium & Medium \\
\addlinespace[1ex]
 & \textbf{Investment Process} &  &  & &  &  \\
 & \hspace{1em}Investment Application Form & & X & & Medium & High   \\
 & \hspace{1em}Investment Advice Record & X & X & & Medium & High \\
 & \hspace{1em}Transaction Receipt & & X & & Medium & Medium \\
 & \hspace{1em}Securities Account Statements & & X & & High & Medium \\
 & \hspace{1em}Effective Securities & & & X & Low & High \\
\addlinespace[1ex]
 & \textbf{Tax Documents} &  &  & &  &  \\
 & \hspace{1em}Exemption Form & & X & & High & Low \\
\bottomrule
\end{tabular}
\end{scriptsize} % End of smaller text portion
\end{table}
\end{landscape}






\begin{landscape}
\begin{table}[h]
\centering
\caption*{Table \ref{tab:doc_assessment} (continued): Assessment of Banking Documents}
\label{tab:doc_assessment2}
\setlength\tabcolsep{12pt} % Makes the columns wider
\begin{scriptsize} % Makes the text within the table even smaller
\begin{tabular}{lp{5.5cm}ccccc}
\toprule
\textbf{Area} & \textbf{Processes and Documents} & \multicolumn{3}{c}{\textbf{Multimodality}} & \textbf{Automation} & \textbf{Layout} \\
 & & \textit{Unstructured} & \textit{Tabular} & \textit{Figures or} & \textbf{Potential} & \textbf{Variability} \\
 & & \textit{Texts} & \textit{Data/Forms} & \textit{Images} & & \\
\midrule
\addlinespace[1ex]

\multirow{19}{*}{Credit and Lending Services} & \textbf{Loan Approval Process} &  &  & &  &  \\
 & \hspace{1em}Loan Application Form & (X) & X & & High & Medium \\
 & \hspace{1em}\textit{Income Proof} &  & & &  &  \\
 & \hspace{2em}Salary Statement & (X) & X & & High & Medium \\
 & \hspace{2em}Pension Statement & X & X & & High & Medium \\
 & \hspace{2em}Financial Status Report & X & X & & High & High \\
 & \hspace{2em}Annual Company Reports & X & X & X & High & High \\
 & \hspace{1em}\textit{Credit Ratings} &  & & &  &  \\
 & \hspace{2em}Rating Form & X & X & & Medium & Medium \\
 & \hspace{2em}Third-Party Ratings & X & X & & High & High \\
 & \hspace{1em}Bank Reference & X & X & & High & High \\
 & \hspace{1em}\textit{Loan Agreement} &  & & &  &  \\
 & \hspace{2em}Loan Agreement Contract& X & X & & Medium & Medium \\
 & \hspace{2em}Loan Securities Documents & X & X & X & High & High \\
 & \hspace{2em}Loan Guarantee & X & X & & Medium & High \\
 & \hspace{2em}Land Register Extracts & X & X & (X) & High & Low \\
 & \hspace{1em}\textit{Loan Servicing} &  & & &  &  \\
 & \hspace{2em}Loan Repayment Schedule & X & X & & High & High \\
\addlinespace[1ex]
 & \textbf{Others} &  &  & &  &  \\
 & \hspace{1em}Factoring / Leasing Invoices & X & X & & High & High \\
 & \hspace{1em}Dunning Process Documents & X & X & & Low & Medium \\
 & \hspace{1em}Insolvency Documents & X & X & & Low & High \\
\midrule
\addlinespace[1ex]


\multirow{16}{*}{Trade and Guarantee Services} & \textbf{Trade Finance Application Process} &  &  & &  &  \\
 & \hspace{1em}Trade Finance Application Form & X & X & & High & Low \\
 & \hspace{1em}Trade Agreement & X & & & Medium & Medium \\
 & \hspace{1em}Commercial Invoices & X & X & & High & High \\
\addlinespace[1ex]
 & \textbf{Document Verification Process} &  &  & &  &  \\
 & \hspace{1em}Trading Documents & X & X & & High & High \\
 & \hspace{1em}Transportation Documents & X & X & & Medium & High \\
 & \hspace{1em}Insurance Documents & X & X & & Medium & High \\
 & \hspace{1em}Customs Documents & X & & & Medium & Medium \\
 & \hspace{1em}Additional Certificates & X & X & (X) & Medium & High \\
\addlinespace[1ex]
 & \textbf{Guarantee Issuance Process} &  &  & &  &  \\
 & \hspace{1em}Guarantee Document & X & X & & High & High \\
 & \hspace{1em}Letter of Credits & X & X & & High & High \\
 & \hspace{1em}Export Guarantees & X & X & & High & High \\
 & \hspace{1em}Other Guarantees & X & X & & Medium & High \\
 & \hspace{1em}Rental Payment Guarantee & X & X & & Medium & Medium \\
 & \hspace{1em}Delivery Guarantee & X & X & & Medium & Medium \\
\bottomrule
\addlinespace[0.5ex]
\multicolumn{7}{p{0.8\linewidth}}{Note: Documents can be used in multiple processes. For this study, each document is assigned to one primary process based on common usage.} % Footnote

\end{tabular}
\end{scriptsize} % End of smaller text portion
\end{table}
\end{landscape}


One important insight that emerges from our study is the recognition of inherent multimodality in banking documents. Many banking documents do not only consist of plain text, but also include various types of visual content, such as images, graphs, tables, or even handwriting. This complexity necessitates the application of multimodal models like LayoutXLM to ensure a comprehensive understanding of the document's content.

As an example, banking forms, such as loan applications or account opening forms, often contain both tabular data (sometimes to be completed by the customer) and unstructured text. Meanwhile, identification documents that are required for know-your-customer purposes, typically include personal information presented in (tabular) text format, alongside photographs, and signatures. In both cases, an unimodal approach that focuses solely on textual or visual data would be insufficient.

We identify three key use cases for application: information extraction, document page classification, and document splitting, each of which has the potential to improve the efficiency of banking operations.


\paragraph{Information Extraction}
An advanced use case is the automated extraction of structured information from unstructured documents, including elements such as entity names, numeric data, and checkbox selections. For instance, extracting customer details from ID documents or company information from company register extracts or annual company reports. Accurate extraction of this data enables banks to process applications or requests quickly and efficiently, reducing processing times and improving customer satisfaction.
The complexity of information extraction is closely related to the layout variability of the document type. For example, highly standardized forms such as tax forms and identity documents are generally easier to parse and extract information.

\paragraph{Document Page Classification}
This involves categorizing document pages into predefined document types. This is especially important when banking processes require different documents, each of which must be processed differently. By automatically identifying and then processing each document type, banks can increase operational efficiency and ensure consistent and accurate document lifecycle management. 
This approach can be particularly beneficial for document-intensive processes such as opening an account. In this case, specific documents such as account opening forms, proofs of identity, and other supporting documents can be classified and automatically routed in order to streamline the process.


\paragraph{Document Splitting}
Document splitting refers to the process of breaking multi-page documents into smaller, more manageable segments or pages, each of which represents a distinct sub-document. 
These sections may correspond to predefined document types, or they may represent individual units of information within a document type. For example, a compilation of monthly salary proof documents can be divided not only based on their categorization as ``salary proof", but also according to the specific month to which they correspond.
This process is, for instance, required when a batch of physical documents is scanned and all the pages are combined into one digital file. In this context, document splitting helps to separate and organize the collected information into individual units, making it easier to handle and process.
\\

These use cases demonstrate how document analytics can serve as a powerful tool in the banking sector. It can automate manual processes, increase accuracy, and accelerate decision-making, contributing to increased customer satisfaction and competitive advantage. Other potential applications could include document representation, visual question answering, and more, expanding the range of possibilities in this area.
