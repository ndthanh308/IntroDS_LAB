\section{Conclusion and Discussion}
\label{sec:conclusion}

%Summary of Findings and Contribution
This study analyzes the diverse banking document landscape, stressing the potential for efficiency gains through advanced document analytics. We illustrate the successful application of the multimodal, cross-lingual LayoutXLM model for information extraction from German company register extracts. With an overall token classification F1 score of approx. 80\%, LayoutXLM proves effective in managing visually-rich banking documents.
Our ablation studies confirm that the high performance is due to the model's ability to extract relevant information from layout and visual information, which underscores the importance of taking a multimodal approach toward document analytics. 
LayoutXLM also shows robustness against token class imbalance and delivers good results with little training data, supporting its suitability for banking applications.
Consequently, this study's contribution is the presentation of the practical benefits of document analytics and providing empirical evidence for the effectiveness of multimodal models in banks.

%Discussion of Results

The presented use case illustrates potential bank efficiency improvements: In the corporate client onboarding process, the automatic extraction of information from company register extracts streamlines workflows. The LayoutXLM model can distinguish current data from outdated entries. The automatic recognition of entities like the current company directors or the legal form can accelerate decision-making. Moreover, numerous other banking documents also exhibit these visual and layout cues, suggesting that they can be analyzed in a similar way with LayoutXLM for tasks such as information extraction or document classification.

%Practical Implications

Implementing a state-of-the-art document analytics workflow has the potential to significantly optimize processes, reduce processing time and improve customer service.
As a result, banks can gain a competitive advantage and drive digital transformation in a rapidly evolving financial landscape.

%Limitations and Future Work

Although the results of our study are promising, it is not without limitations that point to future research directions. Key areas for further research include handling more complex and diverse document types, especially those with high layout variability.
Future work could also address the potential superiority of multimodal large language models such as GPT-4, despite their current adoption challenges due to privacy concerns. A comparative study between GPT-4 and LayoutXLM would be an enlightening extension.