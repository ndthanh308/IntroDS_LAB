% eprint cover page suitable for DIS2023
% modified version of the sample eprint article in LaTeX  by
% M. Peskin, 9/7/00
% should work with most latex interpreters.

\pdfinclusioncopyfonts=1

\documentclass[12pt]{article}
\usepackage{graphicx} % Required for inserting images
\usepackage[colorlinks=true,linkcolor=blue]{hyperref} % Required for href
\usepackage{lineno}
\usepackage{xspace}
\usepackage{caption}
\usepackage{subcaption}

%%%%%%%%%%%%%%%%%%%%%%%%%%%%%
% Basic data for the eprint %
%%%%%%%%%%%%%%%%%%%%%%%%%%%%%
% Adjust these for your printer:
\textwidth=6.0in  
\textheight=8.25in
\leftmargin=-0.3in   
\topmargin=-0.20in

% Date, you can change this to a fixed date
\newcommand\pubdate{\today}

%%%%%%%%%%%%%%%%%%%%%%%%%
% Document style macros %
%%%%%%%%%%%%%%%%%%%%%%%%%

\def\Title#1{\begin{center} {\Large #1 } \end{center}}
\def\Author#1{\begin{center}{ \sc #1} \end{center}}
\def\Address#1{\begin{center}{ \it #1} \end{center}}
\def\andauth{\begin{center}{and} \end{center}}
\def\submit#1{\begin{center}Submitted to {\sl #1} \end{center}}

%%%% Template definitions provided by DIS organisers
\newcommand\pubblock{\rightline{\begin{tabular}{l}  \\ % Author's note number [if you need to add one] goes here
         \pubdate  \end{tabular}}}
\newenvironment{Abstract}{\begin{quotation}  }{\end{quotation}}
\newenvironment{Presented}{\begin{quotation} \begin{center} 
             PRESENTED AT\end{center}\bigskip 
      \begin{center}\begin{large}}{\end{large}\end{center} \end{quotation}}
%%%%%%%%%%%%%%%%%%%%%%%%%%%%%%%%%%%%%%%%%%%%%%%%%%%%%%%%%%%%%%%%%%%%%%%%%%%%

%%%% Variable and style definitions
% Italics
\newcommand{\tita}{\textit} 
\newcommand{\mita}{\mathit}
% Boldface
\newcommand{\tbf}{\textbf} 
\newcommand{\mbf}{\mathbf}
% Calligraphic
\newcommand{\mcal}{\mathcal}
% Roman
\newcommand{\mrm}{\mathrm}
% Colour
\newcommand{\tc}{\textcolor}
%
\newcommand{\pp}{\tita{pp}\xspace}
%
\newcommand{\alphaem}{\alpha_{\mrm{EM}}}
\newcommand{\gs}{g_{\mrm{s}}}
\newcommand{\alphas}{\alpha_{\mrm{s}}}
\newcommand{\alphasq}{\alpha_{\mrm{s}}\left(Q^{\mrm{2}}\right)}
\newcommand{\alphasmu}{\alpha_{\mrm{s}}\left(\mu^{\mrm{2}}\right)}
\newcommand{\alphasmuR}{\alpha_{\mrm{s}}\left(\muR^{\mrm{2}}\right)}
\newcommand{\alphasmz}{\alpha_{\mrm{s}}\left(m_{\mrm{Z}}\right)}
\newcommand{\betas}{\beta\left(\alpha_{\mrm{s}}\right)}
\newcommand{\betasq}{\beta\left(\alpha_{\mrm{s}}\left(Q^{\mrm{2}}\right)\right)}
%
\newcommand{\muR}{\mu_{\mrm{R}}}
\newcommand{\muF}{\mu_{\mrm{F}}}
\newcommand{\muf}{\mu_{\mrm{f}}}
%
\newcommand{\kt}{k_{\mrm{t}}}
\newcommand{\pt}{p_{\mathrm{T}}}
\newcommand{\pttrue}{p_{\mathrm{T}}^{\mrm{true}}}
\newcommand{\ptreco}{p_{\mathrm{T}}^{\mrm{reco}}}
\newcommand{\etrue}{E^{\mrm{true}}}
\newcommand{\ereco}{E^{\mrm{reco}}}
\newcommand{\et}{E_{\mathrm{T}}}
\newcommand{\vareta}{\vert \eta \vert}
%
\newcommand{\etgamma}{E_{\mathrm{T}}^{\gamma}}
\newcommand{\ptgamma}{p_{\mathrm{T}}^{\gamma}}
\newcommand{\phigamma}{\phi^{\gamma}}
\newcommand{\etagamma}{\eta^{\gamma}}
\newcommand{\etadet}{\eta_{\mrm{det}}}
\newcommand{\varetagamma}{\vert \eta^{\gamma} \vert}
\newcommand{\varetadet}{\vert\eta_{\mrm{det}}\vert}
\newcommand{\etiso}{E_{\mathrm{T}}^{\mathrm{iso}}}
\newcommand{\etisocut}{E_{\mathrm{T,cut}}^{\mathrm{iso}}}
\newcommand{\etmax}{E_{\mathrm{T}}^{\mathrm{max}}}
\newcommand{\ethad}{E_{\mathrm{T}}^{\mathrm{had}}}
\newcommand{\rbg}{R^{\mathrm{bg}}}
%
\newcommand{\ptref}{p_{\mathrm{T}}^{\mrm{ref}}}
\newcommand{\ptprobe}{p_{\mathrm{T}}^{\mrm{probe}}}
\newcommand{\ptavg}{p_{\mathrm{T}}^{\mrm{avg}}}
\newcommand{\phijet}{\phi^{\mrm{jet}}}
\newcommand{\ptjet}{p_{\mathrm{T}}^{\mrm{jet}}}
\newcommand{\pttruth}{p_{\mathrm{T}}^{\mrm{truth}}}
\newcommand{\rapidityjet}{y^{\mrm{jet}}}
\newcommand{\etajet}{\eta^{\mrm{jet}}}
\newcommand{\varrapidityjet}{\vert y^{\mrm{jet}} \vert}
\newcommand{\varrapidity}{\vert y \vert}
\newcommand{\varetajet}{\vert \eta^{\mrm{jet}} \vert}
\newcommand{\mgj}{m^{\gamma-\mrm{jet}}}
\newcommand{\dphigj}{\Delta\phi^{\gamma-\mrm{jet}}}
\newcommand{\dygj}{\vert\Delta y^{\gamma-\mrm{jet}}\vert}
\newcommand{\cosine}{\vert \cos\theta^{\ast} \vert}
%
\newcommand{\cme}{\sqrt{s}}
\newcommand{\avmu}{\langle\mu\rangle}
\newcommand{\npv}{N_{\mrm{PV}}}
%
\newcommand{\rat}{R_{\mrm{13/8}}}
\newcommand{\ratg}{R^{\gamma}_{\mrm{13/8}}}
\newcommand{\ratz}{R^{Z}_{\mrm{13/8}}}
\newcommand{\drat}{D^{\gamma / Z}_{\mrm{13/8}}}
\newcommand{\sigzb}{\sigma^{\mrm{fid}}_{Z}(13\mrm{TeV})}
\newcommand{\sigza}{\sigma^{\mrm{fid}}_{Z}(8\mrm{TeV})}
%
\newcommand{\firstlumi}{36\,\mrm{fb}^{-1}\xspace}
\newcommand{\secondlumi}{80\,\mrm{fb}^{-1}\xspace}
\newcommand{\thirdlumi}{139\,\mrm{fb}^{-1}\xspace}
\newcommand{\thirdlumibis}{138\,\mrm{fb}^{-1}\xspace}
\newcommand{\fourthlumi}{58.5\,\mrm{fb}^{-1}\xspace}
% PROPER NOUN
\newcommand{\sherpa}{S{\scriptsize{HERPA}}\xspace}
\newcommand{\sherpalo}{S{\scriptsize{HERPA}} LO\xspace}
\newcommand{\sherpanlo}{S{\scriptsize{HERPA}} NLO\xspace}
\newcommand{\pythia}{P{\scriptsize{YTHIA}}\xspace}
\newcommand{\jetphox}{J{\scriptsize{ETPHOX}}\xspace}
\newcommand{\nnlojet}{N{\scriptsize{NLOJET}}\xspace}
\newcommand{\dyturbo}{D{\scriptsize{YTURBO}}\xspace}
\newcommand{\herwig}{H{\scriptsize{ERWIG}}\xspace}
\newcommand{\powheg}{P{\scriptsize{OWHEG}}\xspace}
\newcommand{\powhegbox}{P{\scriptsize{OWHEG}} B{\scriptsize{OX}} {\normalsize{v2}}\xspace}
\newcommand{\powhegpythia}{P{\scriptsize{OWHEG}}+P{\scriptsize{YTHIA}}\xspace}
\newcommand{\powhegherwig}{P{\scriptsize{OWHEG}}+H{\scriptsize{ERWIG}}\xspace}

%%%%%%%%%%%%%%%%%%%%%%%%%%%%%%%%%%%%%%%%%%%%%%%%%%%%%%%%%%%%%%%%%%%%%%%%%%%%

\begin{document}
%\linenumbers

\begin{titlepage}

\pubblock

\vfill

\Title{Precision measurements of jet and photon \\ production at ATLAS}

\vfill

\Author{Daniel Camarero Muñoz (on behalf of the ATLAS Collaboration\footnote{\noindent Copyright 2023 CERN for the benefit of the ATLAS Collaboration. CC-BY-4.0 license.})}
\Address{Brandeis University, Waltham, United States of America}

\vfill

\begin{Abstract}
\noindent The production of jets and prompt isolated photons at hadron colliders provides stringent tests of perturbative QCD. The latest measurements performed by the ATLAS Collaboration at the LHC are presented in these proceedings. The inclusive prompt-photon production is measured for two distinct photon isolation cones, $R = 0.2$ and $0.4$, as well as for their ratio. This measurement is sensitive to gluon parton density distribution in the proton. In addition, a measurement of variables probing the properties of the multijet energy flow which are used to determine the strong coupling constant is presented. These measurements are compared to state-of-the-art NLO and NNLO predictions. Lastly, a measurement of new event-shape jet observables defined in terms of reference geometries with cylindrical and circular symmetries using the energy mover's distance is discussed.
\end{Abstract}

\vfill

\begin{Presented}
DIS2023: XXX International Workshop on Deep-Inelastic Scattering and
Related Subjects, \\
Michigan State University, USA, 27-31 March 2023 \\
     % Figure removed
\end{Presented}
\vfill

\end{titlepage}

\setlength{\parskip}{0.5em}

%-------------------------------------------------------------------------------
\section{Introduction}
\label{Sec:Intoduction}
The problem of the presence or absence of phase transition is central in statistical mechanics. To prove the existence of phase transition, the standard idea is to define a notion of contour and use \textit{Peierls' argument} \cite{Peierls.1936}. In the usual Ising model \cite{Ising_25}, particles of the system interact only with their nearest-neighbors. On ferromagnetic long-range Ising models \cite{Anderson_Yuval_69}, there is interaction between each pair of spins in the lattice. The Hamiltonian of the model is given formally by
\begin{equation*}
    H(\sigma) = - \sum_{x,y\in \Z^d}J_{xy}\sigma_x\sigma_y,
\end{equation*}
where $J_{xy}=J|x-y|^{-\alpha}$, $J>0$, $\alpha > d$. It is well-known that the Peierls' argument in dimension 2 implies phase transition for Ising models with nearest-neighbors or long-range interactions when $d\geq 2$, using correlation inequalities. For the unidimensional lattice, it was known that short-range models do not present phase transition. In the long-range case, a different behavior was expected depending on the exponent $\alpha$ (see \cite{Kac_Thompson_69}), but the problem was challenging since contours were first created as multidimensional objects.

In dimension $d=1$, phase transition was proved first in 1969 by Dyson \cite{Dyson.69}, for $\alpha \in (1,2)$, by proving phase transition in an auxiliary model and then using correlation inequalities. In 1982, Fr{\"o}hlich and Spencer \cite{Frohlich.Spencer.82} introduced a notion of one-dimensional contours and then applied the Peierls' argument to show phase transition for the critical value $\alpha = 2$. These contours were inspired by the multiscale techniques previously introduced to study the Berezinskii-Kosterlitz-Thouless transition in two-dimensional continuous spin systems \cite{FS81}. Later, Cassandro, Ferrari, Merola and Presutti  \cite{Cassandro.05} extended the contour argument previously available for $\alpha=2$ to exponents $\alpha\in (3-\frac{\ln 3}{\ln 2}, 2)$, with the additional restriction that the nearest-neighbor interaction is strong, i.e.,  ${J(1)\gg 1}$; this restriction was removed for a subclass of interactions in \cite{Bissacot.Endo.18}. Further results were obtained using contour arguments, such as the decay of correlations, cluster expansions, phase transition with random interactions, etc; some references with these results are \cite{ Cassandro.Merola.Picco.17, Cassandro.Merola.Picco.Rozikov.14, Imbrie.82, Imbrie.Newman.88, Johansson.91}. 

In the multidimensional setting ($d\geq 2$), Ginibre, Grossmann, and Ruelle, in \cite{Ginibre.Grossmann.Ruelle.66}, proved the phase transition for $\alpha > d+1$, using an enhanced version of Peierls' argument and the usual contours. Park proposed a different notion of contour for long-range systems in \cite{Park.88.I, Park.88.II}, extending the Pirogov-Sinai theory available for short-range interactions assuming $\alpha > 3d+1$, although he can also consider Potts models with his methods. Some results in the literature suggest that truly long-range effects appear only when $d < \alpha \leq d+1$, see for instance, \cite{Biskup_Chayes_Kivelson_07}. Recently, Affonso, Bissacot, Endo and Handa \cite{Affonso.2021}, inspired by the ideas from Fr{\"o}hlich and Spencer in \cite{FS81, Frohlich.Spencer.82}, introduced a version of multiscale multidimensional contour and proved phase transition by a contour argument in the whole region $\alpha > d$. They can consider long-range Ising models with deterministic decaying fields, first introduced in the context of nearest-neighbor interactions in \cite{Bissacot_Cioletti_10}. For these models, the lack of analyticity of the free energy does not imply phase transition since these models have the same free energy as the models with zero field. It is expected that fields decaying slowly imply uniqueness. In this setting, a contour argument is useful for proofs of phase transitions as well for uniqueness, some papers with models with deterministic decaying fields are \cite{Aoun_Ott_Velenik_23, Bissacot_Cass_Cio_Pres_15, Bissacot.Endo.18, Cioletti_Vila_2016}.

The Random Field Ising model (RFIM) \cite{Imry.Ma.75} is the nearest-neighbor Ising model with an additional external field acting on each site $(h_x)_{x\in\Z^d}$ that is a family of i.i.d. Gaussian random variable with mean 0 and variance 1. Formally, the Hamiltonian of the model is given by
\begin{equation*}
    H(\sigma) = - \sum_{\substack{x,y\in \Z^d \\|x-y|=1}}J\sigma_x\sigma_y  - \varepsilon\sum_{x\in\Z^d}h_x\sigma_x,
\end{equation*}
where $J>0$, $\varepsilon>0$, $\alpha > d$ and $d \geq 1$. A detailed account of the history of the phase transition problem for this model, as well as detailed proofs, was given in \cite{Bovier.06}. Here we present a brief overview.

During the 1980s, the question of the specific dimension where phase transition for the RFIM should happen attracted much attention and was a topic of heated debate. Two convincing arguments were dividing the physics community. One of them, due to Imry and Ma \cite{Imry.Ma.75}, was a non-rigorous application of the Peierls' argument together with the use of the isoperimetric inequality. The key idea of Peierls' argument is to define a notion of contour and calculate the energy cost of "erasing" each contour, i.e., the energy cost of flipping all spins inside the contour. When there is no external field, that energy necessary to flip the spins in a region $A\subset \Z^d$ is of the order of the boundary $|\partial A|$. When we add an external field, we get an extra cost depending on this field. Imry and Ma argued that this cost should be approximately $\sqrt{|A|}$, which is smaller than $|\partial A|$ for all regions only when $d\geq 3$, so this should be the region where phase transition occurs. The other argument, due to Parisi and Sourlas \cite{Parisi.Sourlas.79}, based on dimensional reduction, predicted that the $d$-dimensional RFIM would behave like the $d-2$-dimensional nearest-neighbor Ising model, therefore presenting phase transition only when $d\geq 4$. 

The question was settled by two celebrated papers showing that Imry and Ma's prediction was correct. First, in 1988, Bricmont and Kupiainen \cite{Bricmont.Kupiainen.88} showed that there is phase transition almost surely in $d\geq3$, for low temperatures and variance $\varepsilon$ small enough. Their proof uses a rigorous renormalization group analysis for the short-range case and it is considered involved. Still, they claimed that the result works for any model with a suitable contour representation and centered sub-gaussian external field. Later on, Aizenman and Wehr \cite{Aizenman.Wehr.90} proved uniqueness for $d\leq 2$. For detailed proofs of these results, we refer the reader to \cite{Bovier.06} (see also \cite{Berretti.85, Camia.18, Frohlich.Imbre.84,  Klein.Masooman.97} for more uniqueness results). 

Recently, Ding and Zhuang, see \cite{Ding2021}, provided a simpler proof of the phase transition, not using RGM. And in  \cite{Ding.Liu.Xia.22}, Ding, Liu and Xia proved that if $\beta_c(d)$ is the critical inverse of the temperature of the Ising model with no field, for all $\beta>\beta_c(d)$ there exists a critical value $\varepsilon_0(d, \beta)$ such that the RFIM with $\varepsilon \leq \varepsilon_0$ presents phase transition. 

In the present paper, we are considering a long-range Ising model with a random field, whose Hamiltonian is given formally by
\begin{equation*}
    H(\sigma) = - \sum_{x,y\in \Z^d}J_{xy}\sigma_x\sigma_y - \varepsilon\sum_{x\in\Z^d}h_x\sigma_x,
\end{equation*}
where $J_{xy}=J|x-y|^{-\alpha}$, $J, \varepsilon>0$, $\alpha > d$ and $h_x\in\mathbb{R}$, $d\geq 3$.
Until now, the only known result in the long-range setting is for the one-dimensional long-range Ising model with a random field, by Cassandro, Orlandi, and Picco \cite{Cassandro.Picco.09}. They used the contours of \cite{Cassandro.05} to show the phase transition for the model when $\alpha\in (3-\frac{\ln 3}{\ln 2}, \frac{3}{2})$, under the assumption $J(1) \gg 1$. We stress that, as remarked by Aizenman, Greenblatt, and Lebowitz \cite{Aizenman_Greenblatt_Lebowitz_2012}, although their argument does not work for the whole region for the exponent $\alpha$, the phase transition holds for values close to the critical value $\alpha=3/2$, since by the Aizenman-Wehr theorem we know that there is uniqueness for $\alpha>3/2$.

The argument from Ding and Zhuang in \cite{Ding2021}, for $d\geq3$, involves controlling the probability of a bad event, which is closely related to controlling the quantity $$\sup_{\substack{0\in A\subset\Z^d \\ A \text{ connected }}}\frac{\sum_{x\in A}h_x}{|\partial A|},$$ known as the greedy animal lattice normalized by the boundary. The greedy animal lattice normalized by the size, instead of the boundary, was extensively studied for general distributions of $(h_x)_{x\in\Z^d}$, see \cite{Cox_Gandolfi_Griffin_Kesten_93, Gandolfi_Kesten_94, Hammond_06, Martin_02}. When we normalize by the boundary, an argument by Fisher, Fr\"{o}hlich and Spencer \cite{FFS84} shows that the expected value of the greedy animal lattice is constant. In dimension $d=2$, the expected value is not finite, see \cite{Ding.Wirth.20}. The supremum is taken over connected regions containing the origin since the interiors of the usual Peierls contours are of this form.


For the long-range model, the interior of contours is not necessarily connected. In fact, long-range contours may have considerably large diameters with respect to their size, so their interiors can be very sparse. To avoid this, we define contours, strongly inspired by the $(M,a,r)$-partition in \cite{Affonso.2021}, using a multiscaled procedure that assures that the contours have no cluster with small density.  With them, we generalize the arguments by Fisher-Fr\"{o}hlich-Spencer \cite{FFS84}, and prove that the expected value of the greedy animal lattice is constant, even considering regions not necessarily connected in the supremum. Then, we prove the phase transition for $d\geq 3$. The main result of this paper is the following.
\begin{theorem*}Given $d\geq 3$, $\alpha>d$, there exists $\beta_c\coloneqq\beta(d, \alpha)$ and $\varepsilon_c\coloneqq\varepsilon(d, \alpha)$ such that, for $\beta >\beta_c$ and $\varepsilon\leq \varepsilon_c$, the extremal Gibbs measures $\mu_{\beta, \varepsilon}^+$ and $\mu_{\beta, \varepsilon}^-$ are distinct, that is, $\mu_{\beta, \varepsilon}^+ \neq \mu_{\beta, \varepsilon}^-$ $\mathbb{P}$-almost surely. Therefore the long-range random field Ising model presents phase transition.
\end{theorem*}

This paper is divided as follows. In Section 2, we define the model and the contours, and suitable generalizations to the constructions in \cite{Affonso.2021} are introduced.  In Section 3, we define two bad events of the external field and prove that they occur with a small probability.  In Section 4, we present the proof of the phase transition.
%-------------------------------------------------------------------------------

%-------------------------------------------------------------------------------
\section{Inclusive-photon production}
\label{Sec:Anal1Inclusive}
Differential cross sections are measured as functions of the photon transverse energy, $\etgamma$, in different regions of the photon pseudorapidity, $\etagamma$, for $\etgamma > 250$~GeV and $\varetagamma < 2.37$. Photons are required to be isolated both at particle and detector levels using a `fixed cone' criterion: $E^{\mathrm{iso}}_{\mathrm{T}} < E^{\mathrm{iso}}_{\mathrm{T,cut}} = 4.2 \cdot 10^{-3} \cdot \etgamma +4.8$~GeV \cite{3}. The dependence of the cross section on the isolation cone radius, $R$, is investigated by measuring the ratios of the cross sections for $R = 0.2$ and $0.4$.

\subsection{Systematic uncertainties}

The dominant systematic uncertainties affecting the measurement of the differential cross sections arise from the uncertainty on the photon energy scale, the luminosity uncertainty, the uncertainty on the $R^{\mathrm{bg}}$ correlation, and the pile-up modelling. The total systematic uncertainty varies in the range $(3-20)\%$. The measurement of the ratios of the cross sections benefits from large cancellations for the uncertainties that are independent of the isolation cone radius. The total uncertainty is typically $<1\%$.

\subsection{Theoretical predictions}

The next-to-leading-order (NLO) pQCD calculations included in this measurement were computed using the programs \jetphox $1.3.1\_2$ and \sherpa $2.2.2$. The next-to-next-to-leading-order (NNLO) pQCD predictions are calculated in the \nnlojet framework. There are several differences between these calculations which are described in detail in Section 8 of Reference \cite{3}. The total uncertainty ranges from $\approx (10 - 15)\%$ for \jetphox, and it is $\approx 20\%$ for \sherpa. For the NNLO pQCD calculation of \nnlojet, the uncertainties are in the range $(1-6)\%$, being smaller than those in the NLO pQCD prediction by a factor $2-15$. The total uncertainty in the ratios of the cross sections benefits from large cancellations, being $\approx 1.5\%$ ($\approx 1\%$) for the predictions of \jetphox and \sherpa (\nnlojet).

\subsection{Results}

The ratios of the predictions from \jetphox based on different PDFs and the data are shown in Figure \ref{fig_1_1} for $R=0.2$. The calculations of \jetphox are consistent with the measurements within uncertainties. Predictions based on MMHT2014, CT18 and NNPDF3.1 PDF sets are similar and the closest to the data for $\varetagamma < 1.37$ and $1.81 < \varetagamma < 2.37$. For $1.56 < \varetagamma < 1.81$, the predictions based on the HERAPDF2.0 PDF and ATLASpdf21 sets are the closest to the data.

% Figure environment removed

The dependence of the inclusive photon cross section on $R$ is investigated by measuring the ratios of the cross sections for $R = 0.2$ and $0.4$, shown in Figure \ref{fig_1_2}. \sherpa predictions overestimate the data, while \jetphox gives a good description. The NNLO pQCD predictions give an excellent description of the data. These measurements provide a very stringent test of pQCD with reduced experimental and theoretical uncertainties, validating the underlying theoretical description up to $\mcal{O}(\alphas^2)$.
%-------------------------------------------------------------------------------

%-------------------------------------------------------------------------------
\section{Transverse Energy-Energy Correlations}
\label{Sec:Anal2TEEC}
Differential cross sections for the TEEC and ATEEC observables as functions of $\cos\phi$ are measured and compared with several Monte Carlo (MC) predictions inclusively and in ten bins of the scalar sum of the transverse momenta of the two leading jets, $H_{\mrm{T2}}$. The selected jets are required to have $\pt > 60$~GeV and $\vareta < 2.4$. The TEEC and ATEEC measured cross sections are compared to NNLO pQCD predictions for first time. A determination of the strong coupling constant is performed from this comparison.

\subsection{Systematic uncertainties}

The systematic uncertainties arising from the jet energy scale uncertainties dominate both the TEEC and ATEEC measurements, while the MC modelling and jet energy resolution uncertainties are also important for the TEEC and ATEEC, respectively. The total systematic uncertainty is of order $2\%$ ($1\%$) for the TEEC (ATEEC).

\subsection{Theoretical predictions}

The theoretical predictions for the TEEC and ATEEC functions are calculated at leading-order (LO), NLO and NNLO in pQCD. The NNLO pQCD corrections include real–real, real–virtual and virtual–virtual finite terms, as well as single- and double-unresolved terms and the finite remainder. The renormalisation and factorisation scales are set to the scalar sum of the transverse momenta of all final-state partons, $\muR = \muF = \hat{H}_{\mrm{T}}$. The uncertainties are computed from the scale variations, the PDF uncertainties, and those in the non-perturbative corrections to the pQCD predictions.

\subsection{Results}

The comparison of the measured cross sections for the TEEC and ATEEC functions with the predictions at different orders in pQCD are shown in Figures 6 and 7 of Reference \cite{4}. The description of the data provided by the NNLO pQCD predictions is excellent, improving with respect to the NLO prediction. The reduction of the scale uncertainties up to a factor of 3 is made evident from these figures, as well as the improvement in the description.

The strong coupling constant at the scale of the pole mass of the $Z$ boson, $\alphasmz$, was determined  from the comparison of the data to the theoretical predictions by means of a $\chi^2$ fit described in Section 9 of Reference \cite{4}, for both the TEEC and ATEEC distributions. To compare the results with other experiments, the value of the energy scale $Q$ is chosen as half of the average value of $\hat{H}_{\mrm{T}}$ for each $H_{\mrm{T2}}$ bin. Figure \ref{fig_2_2} shows the values of $\alphas(Q)$ from the fit to the TEEC distribution together with the world average band provided by the Particle Data Group \cite{6} and values of $\alphas$ obtained in other analyses. The results show a good agreement between all measurements and the renormalisation group equation prediction.

% Figure environment removed
%-------------------------------------------------------------------------------

%-------------------------------------------------------------------------------
\section{Event isotropies using optimal transport}
\label{Sec:Anal3Isotropies}
The shape of the event isotropy observables $\mcal{I}^{N = 2}_\mrm{Ring}$, $\mcal{I}^{N = 128}_\mrm{Ring}$, and $\mcal{I}^{N = 16}_{\mrm{Cyl}}$ are measured in inclusive bins of $N_{\mrm{jet}}$ and $H_{\mrm{T2}}$. These observables are defined in Table 1 of Reference \cite{5}. The inclusive jet-multiplicity bins range from $N_{\mrm{jet}} \geq 2$ to $N_{\mrm{jet}} \geq 5$, and the inclusive bins of $H_{\mrm{T2}}$ are $H_{\mrm{T2}} \geq 500$~GeV, $H_{\mrm{T2}} \geq 1000$~GeV and $H_{\mrm{T2}} \geq 1500$~GeV. 

\subsection{Systematic uncertainties}

The dominant sources of systematic uncertainties are related either to the jet energy resolution or to the choice of MC model used in the unfolding. 

\subsection{Theoretical predictions}

Samples of MC simulated dijet and multijet events are used in this analysis. \pythia 8.230 is used as the nominal MC generator. Two sets of \sherpa 2.2.5 with the default AHADIC cluster hadronisation or with the \sherpa interface to the Lund string hadronisation model were used. Two sets of \herwig 7.1.3 multijet events were generated with the default cluster hadronisation model and either the default angle-ordered parton shower (PS) or alternative dipole PS. Two additional samples of dijet events with NLO matrix element accuracy were produced with \powhegbox, matched to either the \pythia 8 or angle-ordered \herwig 7 parton shower.

\subsection{Results}

The unfolded data are compared with predictions from several state-of-the-art Monte Carlo models. Good agreement is often observed between the LO and NLO Monte Carlo generators throughout the non-isotropic region of a given distribution; poorer agreement is seen as particle configurations become more isotropic.

The most inclusive measurement of $1 - \mcal{I}^{N = 128}_\mrm{Ring}$ cross-sections is shown in Figure \ref{fig_3_1}. This distribution is saturated by well-balanced dijets events and by multijet events with isotropic configurations. The \powheg predictions are found to strongly disagree with those of the other MC generators. Large differences are also found between the \herwig angle-ordered and dipole shower models. No notable differences are seen between the \sherpa hadronisation models. The most inclusive measurement of $1 - \mcal{I}^{N = 16}_{\mrm{Cyl}}$ cross-sections is shown in Figure \ref{fig_3_2}. Multijet events that cover the rapidity–azimuth plane with activity in both the central and forward regions produce the highest values for this observable. None of the MC predictions accurately describe this observable. The predictions from the \pythia and \powheg samples are consistent except at low values, where \pythia overestimates the observed cross-section.

% Figure environment removed
%-------------------------------------------------------------------------------

%-------------------------------------------------------------------------------
%\section{Summary and conclusions}
%\label{Sec:Summary}
%Measurements of the inclusive isolated-photon production cross section and of event shapes probing the properties of the multijet energy flow in $pp$ collisions at $\cme = 13$~TeV recorded by the ATLAS experiment at the LHC are presented based on $\thirdlumi$ of $2015-2018$ data.

The dependence on $R$ of the inclusive isolated-photon cross sections is measured for first time, and found to be well described by the NLO \jetphox and NNLO \nnlojet pQCD predictions. The ratios of differential cross sections provide a very stringent test of pQCD, with significantly reduced experimental and theoretical uncertainties, and validate the underlying theoretical description up to $\mcal{O}(\alphas^2)$.

The measurement of transverse energy–energy correlations and their corresponding asymmetries bennefit from the inclusion of NNLO QCD corrections to three-jet production in the theoretical predictions for the first time, allowing a reduction by a factor of 3 of the theoretical uncertainties in both the cross-section calculation for the TEEC and ATEEC distributions and in the determination of the strong coupling constant $\alphas$.

A first application of optimal transport techniques in a collider physics measurement is performed in the measurement of the multijet event isotropies. The measured data are compared with the predictions of several state-of-the-art Monte Carlo event generators. Agreement between the unfolded data and the simulated events tends to be best in balanced, dijet-like arrangements and deteriorates in more isotropic configurations.
%-------------------------------------------------------------------------------

%-------------------------------------------------------------------------------
@book {Pick,
    AUTHOR = {Agler, Jim and McCarthy, John E.},
     TITLE = {Pick interpolation and {H}ilbert function spaces},
    SERIES = {Graduate Studies in Mathematics},
    VOLUME = {44},
 PUBLISHER = {American Mathematical Society, Providence, RI},
      YEAR = {2002},
      ISBN = {0-8218-2898-3} }
      


@book{NewPick,
author={Agler, Jim and McCarthy, John E. and Young, Nicholas J.},
place={Cambridge}, series={Cambridge Tracts in Mathematics}, title={Operator Analysis: Hilbert Space Methods in Complex Analysis}, publisher={Cambridge University Press}, year={2020}, collection={Cambridge Tracts in Mathematics}}




@article {AndoNishioconvex,
    AUTHOR = {Ando, Tsuyoshi and Nishio, Katsuyoshi},
     TITLE = {Convexity properties of operator radii associated with unitary
              {$\rho $}-dilations},
   JOURNAL = {Michigan Math. J.},
  FJOURNAL = {Michigan Mathematical Journal},
    VOLUME = {20},
      YEAR = {1973},
     PAGES = {303--307},
}


@article {BadeaCassierConstrained,
    AUTHOR = {Badea, C. and Cassier, G.},
     TITLE = {Constrained von {N}eumann inequalities},
   JOURNAL = {Adv. Math.},
  FJOURNAL = {Advances in Mathematics},
    VOLUME = {166},
      YEAR = {2002},
    NUMBER = {2},
     PAGES = {260--297},
}
@article {BadCrouzKlajaradii,
    AUTHOR = {Badea, Catalin and Crouzeix, Michel and Klaja, Hubert},
     TITLE = {Spectral sets and operator radii},
   JOURNAL = {Bull. Lond. Math. Soc.},
  FJOURNAL = {Bulletin of the London Mathematical Society},
    VOLUME = {50},
      YEAR = {2018},
    NUMBER = {6},
     PAGES = {986--996},
}
@article {BergStampskewdil,
    AUTHOR = {Berger, C. A. and Stampfli, J. G.},
     TITLE = {Norm relations and skew dilations},
   JOURNAL = {Acta Sci. Math. (Szeged)},
  FJOURNAL = {Acta Universitatis Szegediensis. Acta Scientiarum
              Mathematicarum},
    VOLUME = {28},
      YEAR = {1967},
     PAGES = {191--195},
}
@article {CassierFackContractions,
    AUTHOR = {Cassier, Gilles and Fack, Thierry},
     TITLE = {Contractions in von {N}eumann algebras},
   JOURNAL = {J. Funct. Anal.},
  FJOURNAL = {Journal of Functional Analysis},
    VOLUME = {135},
      YEAR = {1996},
    NUMBER = {2},
     PAGES = {297--338},
}
@article {CassierSuciuSharpened,
    AUTHOR = {Cassier, Gilles and Suciu, Nicolae},
     TITLE = {Sharpened forms of a von {N}eumann inequality for
              {$\rho$}-contractions},
   JOURNAL = {Math. Scand.},
  FJOURNAL = {Mathematica Scandinavica},
    VOLUME = {102},
      YEAR = {2008},
    NUMBER = {2},
     PAGES = {265--282},
}
@article {CassierHarnack,
    AUTHOR = {Cassier, Gilles and Benharrat, Mohammed and Belmouhoub,
              Soumia},
     TITLE = {Harnack parts of {$\rho$}-contractions},
   JOURNAL = {J. Operator Theory},
  FJOURNAL = {Journal of Operator Theory},
    VOLUME = {80},
      YEAR = {2018},
    NUMBER = {2},
     PAGES = {453--480},
}
@article {Davisshell,
    AUTHOR = {Davis, Chandler},
     TITLE = {The shell of a {H}ilbert-space operator},
   JOURNAL = {Acta Sci. Math. (Szeged)},
  FJOURNAL = {Acta Universitatis Szegediensis. Acta Scientiarum
              Mathematicarum},
    VOLUME = {29},
      YEAR = {1968},
     PAGES = {69--86},
}
@article {DritschelScott,
    AUTHOR = {Dritschel, Michael A. and McCullough, Scott and Woerdeman,
              Hugo J.},
     TITLE = {Model theory for {$\rho$}-contractions, {$\rho\le2$}},
   JOURNAL = {J. Operator Theory},
  FJOURNAL = {Journal of Operator Theory},
    VOLUME = {41},
      YEAR = {1999},
    NUMBER = {2},
     PAGES = {321--350},
}
@article {Durszton,
    AUTHOR = {Durszt, E.},
     TITLE = {On unitary {$\rho $}-dilations of operators},
   JOURNAL = {Acta Sci. Math. (Szeged)},
  FJOURNAL = {Acta Universitatis Szegediensis. Acta Scientiarum
              Mathematicarum},
    VOLUME = {27},
      YEAR = {1966},
     PAGES = {247--250},
}
@article {Dursztfactor,
    AUTHOR = {Durszt, E.},
     TITLE = {Factorization of operators in {$\scr C\sb{\rho }$} classes},
   JOURNAL = {Acta Sci. Math. (Szeged)},
  FJOURNAL = {Acta Universitatis Szegediensis. Acta Scientiarum
              Mathematicarum},
    VOLUME = {37},
      YEAR = {1975},
    NUMBER = {3-4},
     PAGES = {195--199},
     }
@article {Holbrookonthe,
    AUTHOR = {Holbrook, John A. R.},
     TITLE = {On the power-bounded operators of {S}z.-{N}agy and {F}oia\c{s}},
   JOURNAL = {Acta Sci. Math. (Szeged)},
  FJOURNAL = {Acta Universitatis Szegediensis. Acta Scientiarum
              Mathematicarum},
    VOLUME = {29},
      YEAR = {1968},
     PAGES = {299--310},
}
@article {OkuboAndoproducts,
    AUTHOR = {Okubo, K. and Ando, T.},
     TITLE = {Operator radii of commuting products},
   JOURNAL = {Proc. Amer. Math. Soc.},
  FJOURNAL = {Proceedings of the American Mathematical Society},
    VOLUME = {56},
      YEAR = {1976},
     PAGES = {203--210},
      ISSN = {0002-9939},
}
@article {OkuboAndroconstantsrelated,
    AUTHOR = {Okubo, Kazuyoshi and Ando, Tsuyoshi},
     TITLE = {Constants related to operators of class {$C\sb{\rho }$}},
   JOURNAL = {Manuscripta Math.},
  FJOURNAL = {Manuscripta Mathematica},
    VOLUME = {16},
      YEAR = {1975},
    NUMBER = {4},
     PAGES = {385--394},
      ISSN = {0025-2611},
}
@article {OkuboSpitkovsky2x2,
    AUTHOR = {Okubo, Kazuyoshi and Spitkovsky, Ilya},
     TITLE = {On the characterization of {$2\times 2$} {$\rho$}-contraction
              matrices},
   JOURNAL = {Linear Algebra Appl.},
  FJOURNAL = {Linear Algebra and its Applications},
    VOLUME = {325},
      YEAR = {2001},
    NUMBER = {1-3},
     PAGES = {177--189},
}
@article {PagaczbetweenvonNeumann,
    AUTHOR = {Pagacz, Patryk and Pietrzycki, Pawe\l  and Wojtylak, Micha\l },
     TITLE = {Between the von {N}eumann inequality and the {C}rouzeix
              conjecture},
   JOURNAL = {Linear Algebra Appl.},
  FJOURNAL = {Linear Algebra and its Applications},
    VOLUME = {605},
      YEAR = {2020},
     PAGES = {130--157},
      ISSN = {0024-3795},
}

@article {NagyFoiasoncertainclasses,
    AUTHOR = {Sz.-Nagy, B\'{e}la and Foia\c{s}, Ciprian},
     TITLE = {On certain classes of power-bounded operators in {H}ilbert
              space},
   JOURNAL = {Acta Sci. Math. (Szeged)},
  FJOURNAL = {Acta Universitatis Szegediensis. Acta Scientiarum
              Mathematicarum},
    VOLUME = {27},
      YEAR = {1966},
     PAGES = {17--25},
}
@book {NagyFoiasbook,
    AUTHOR = {Sz.-Nagy, B\'{e}la and Foias, Ciprian and Bercovici, Hari and
              K\'{e}rchy, L\'{a}szl\'{o}},
     TITLE = {Harmonic analysis of operators on {H}ilbert space},
    SERIES = {Universitext},
   EDITION = {Second},
   EDITION = {enlarged},
 PUBLISHER = {Springer, New York},
      YEAR = {2010},
     PAGES = {xiv+474},
      ISBN = {978-1-4419-6093-1},
}
@article {Tsikvonneumann,
    AUTHOR = {Tsikalas, Georgios},
     TITLE = {A von {N}eumann type inequality for an annulus},
   JOURNAL = {J. Math. Anal. Appl.},
  FJOURNAL = {Journal of Mathematical Analysis and Applications},
    VOLUME = {506},
      YEAR = {2022},
    NUMBER = {2},
     PAGES = {Paper No. 125714, 12},
}
@article {WilliamsSchwarz,
    AUTHOR = {Williams, James P.},
     TITLE = {Schwarz norms for operators},
   JOURNAL = {Pacific J. Math.},
  FJOURNAL = {Pacific Journal of Mathematics},
    VOLUME = {24},
      YEAR = {1968},
     PAGES = {181--188},
}
@article {NagyFrench,
    AUTHOR = {Sz.-Nagy, B\'{e}la},
     TITLE = {Sur les contractions de l'espace de {H}ilbert},
   JOURNAL = {Acta Sci. Math. (Szeged)},
  FJOURNAL = {Acta Universitatis Szegediensis. Acta Scientiarum
              Mathematicarum},
    VOLUME = {15},
      YEAR = {1953},
     PAGES = {87--92},
}
@article {Bergerstampflimapping,
    AUTHOR = {Berger, C. A. and Stampfli, J. G.},
     TITLE = {Mapping theorems for the numerical range},
   JOURNAL = {Amer. J. Math.},
  FJOURNAL = {American Journal of Mathematics},
    VOLUME = {89},
      YEAR = {1967},
     PAGES = {1047--1055},
}
@book {Bergerthesis,
    AUTHOR = {Berger, Charles Arnold},
     TITLE = {N{ORMAL} {DILATIONS}},
      NOTE = {Thesis (Ph.D.)--Cornell University},
 PUBLISHER = {ProQuest LLC, Ann Arbor, MI},
      YEAR = {1963},
     PAGES = {67}, 
}
@misc{schwendevries,
  url = {https://arxiv.org/abs/2302.05389},
  author = {Schwenninger, Felix L. and de Vries, Jens},
  title = {On Abstract Spectral Constants},
  publisher = {arXiv},
  year = {2023},
  
  copyright = {arXiv.org perpetual, non-exclusive license}
}

@article {Putsand,
    AUTHOR = {Putinar, Mihai and Sandberg, Sebastian},
     TITLE = {A skew normal dilation on the numerical range of an operator},
   JOURNAL = {Math. Ann.},
  FJOURNAL = {Mathematische Annalen},
    VOLUME = {331},
      YEAR = {2005},
    NUMBER = {2},
     PAGES = {345--357},
}






@misc{RansfordOster,
  url = {https://arxiv.org/abs/2011.10422},
  author = {Clouâtre, Raphaël and Ostermann, Maëva and Ransford, Thomas},
  title = {An abstract approach to the Crouzeix conjecture},
  publisher = {arXiv},
  year = {2020},
  copyright = {Creative Commons Attribution 4.0 International}
}
@article {LiCaldwellGreen,
    AUTHOR = {Caldwell, Trevor and Greenbaum, Anne and Li, Kenan},
     TITLE = {Some extensions of the {C}rouzeix-{P}alencia result},
   JOURNAL = {SIAM J. Matrix Anal. Appl.},
  FJOURNAL = {SIAM Journal on Matrix Analysis and Applications},
    VOLUME = {39},
      YEAR = {2018},
    NUMBER = {2},
     PAGES = {769--780},
}
@article {CrouzPal,
    AUTHOR = {Crouzeix, M. and Palencia, C.},
     TITLE = {The numerical range is a {$(1+\sqrt{2})$}-spectral set},
   JOURNAL = {SIAM J. Matrix Anal. Appl.},
  FJOURNAL = {SIAM Journal on Matrix Analysis and Applications},
    VOLUME = {38},
      YEAR = {2017},
    NUMBER = {2},
     PAGES = {649--655},
}
@article {CrouxNumerangeandfunctional,
    AUTHOR = {Crouzeix, Michel},
     TITLE = {Numerical range and functional calculus in {H}ilbert space},
   JOURNAL = {J. Funct. Anal.},
  FJOURNAL = {Journal of Functional Analysis},
    VOLUME = {244},
      YEAR = {2007},
    NUMBER = {2},
     PAGES = {668--690},
}
@article {Delyon,
    AUTHOR = {Delyon, Bernard and Delyon, Fran\c{c}ois},
     TITLE = {Generalization of von {N}eumann's spectral sets and integral
              representation of operators},
   JOURNAL = {Bull. Soc. Math. France},
  FJOURNAL = {Bulletin de la Soci\'{e}t\'{e} Math\'{e}matique de France},
    VOLUME = {127},
      YEAR = {1999},
    NUMBER = {1},
     PAGES = {25--41},
}
@article {RansfordSchwenninger,
    AUTHOR = {Ransford, Thomas and Schwenninger, Felix L.},
     TITLE = {Remarks on the {C}rouzeix-{P}alencia proof that the numerical
              range is a {$(1+\sqrt2)$}-spectral set},
   JOURNAL = {SIAM J. Matrix Anal. Appl.},
  FJOURNAL = {SIAM Journal on Matrix Analysis and Applications},
    VOLUME = {39},
      YEAR = {2018},
    NUMBER = {1},
     PAGES = {342--345},
}
@article {Boundsforanalytical,
    AUTHOR = {Crouzeix, Michel},
     TITLE = {Bounds for analytical functions of matrices},
   JOURNAL = {Integral Equations Operator Theory},
  FJOURNAL = {Integral Equations and Operator Theory},
    VOLUME = {48},
      YEAR = {2004},
    NUMBER = {4},
     PAGES = {461--477},
}
@article {CrouzGreen,
    AUTHOR = {Crouzeix, Michel and Greenbaum, Anne},
     TITLE = {Spectral sets: numerical range and beyond},
   JOURNAL = {SIAM J. Matrix Anal. Appl.},
  FJOURNAL = {SIAM Journal on Matrix Analysis and Applications},
    VOLUME = {40},
      YEAR = {2019},
    NUMBER = {3},
     PAGES = {1087--1101},
}
@article {Tsiknote,
    AUTHOR = {Tsikalas, Georgios},
     TITLE = {A note on a spectral constant associated with an annulus},
   JOURNAL = {Oper. Matrices},
  FJOURNAL = {Operators and Matrices},
    VOLUME = {16},
      YEAR = {2022},
    NUMBER = {1},
     PAGES = {95--99},
}
@misc{belloyak,
  
  url = {https://arxiv.org/abs/2106.08757},
  
  author = {Bello, Glenier and Yakubovich, Dmitry},

  title = {An operator model in the annulus},
  
  publisher = {arXiv},
  
  year = {2021},
  
  copyright = {arXiv.org perpetual, non-exclusive license}
}


@article {mcculloughpascoe,
    AUTHOR = {McCullough, Scott and Pascoe, James E.},
     TITLE = {Geometric dilations and operator annuli},
   JOURNAL = {J. Funct. Anal.},
  FJOURNAL = {Journal of Functional Analysis},
    VOLUME = {285},
      YEAR = {2023},
    NUMBER = {7},
     PAGES = {Paper No. 110035}
}




@article {WA,
    AUTHOR = {Suen, Ching-Yun},
     TITLE = {{$W_A$} contractions},
   JOURNAL = {Positivity},
  FJOURNAL = {Positivity. An International Journal devoted to the Theory and
              Applications of Positivity in Analysis},
    VOLUME = {2},
      YEAR = {1998},
    NUMBER = {4},
     PAGES = {301--310},
}
@article {ScottChien,
    AUTHOR = {McCullough, Scott and Shen, Li-Chien},
     TITLE = {On the {S}zeg\H{o} kernel of an annulus},
   JOURNAL = {Proc. Amer. Math. Soc.},
  FJOURNAL = {Proceedings of the American Mathematical Society},
    VOLUME = {121},
      YEAR = {1994},
    NUMBER = {4},
     PAGES = {1111--1121},
}
@book {Paulsenbook,
    AUTHOR = {Paulsen, Vern},
     TITLE = {Completely bounded maps and operator algebras},
    SERIES = {Cambridge Studies in Advanced Mathematics},
    VOLUME = {78},
 PUBLISHER = {Cambridge University Press, Cambridge},
      YEAR = {2002},
     PAGES = {xii+300},
}
@book {Halmosproblembook,
    AUTHOR = {Halmos, Paul Richard},
     TITLE = {A {H}ilbert space problem book},
    SERIES = {Encyclopedia of Mathematics and its Applications},
    VOLUME = {17},
   EDITION = {Second},
 PUBLISHER = {Springer-Verlag, New York-Berlin},
      YEAR = {1982},
     PAGES = {xvii+369},
}
@book {Conway,
    AUTHOR = {Conway, John B.},
     TITLE = {A course in functional analysis},
    SERIES = {Graduate Texts in Mathematics},
    VOLUME = {96},
   EDITION = {Second},
 PUBLISHER = {Springer-Verlag, New York},
      YEAR = {1990},
     PAGES = {xvi+399},
}
@misc{bivariate,
  
  url = {https://arxiv.org/abs/2007.09784},
  
  author = {Crouzeix, Michel and Kressner, Daniel},
  
  keywords = {Functional Analysis (math.FA), Numerical Analysis (math.NA), FOS: Mathematics, FOS: Mathematics, 15A60, 15A16, 65F60},
  
  title = {A bivariate extension of the Crouzeix-Palencia result with an application to Fréchet derivatives of matrix functions},
  
  publisher = {arXiv},
  
  year = {2020},
  
  copyright = {arXiv.org perpetual, non-exclusive license}
}
@article {Annuluskspectral,
    AUTHOR = {Crouzeix, Michel},
     TITLE = {The annulus as a {$K$}-spectral set},
   JOURNAL = {Commun. Pure Appl. Anal.},
  FJOURNAL = {Communications on Pure and Applied Analysis},
    VOLUME = {11},
      YEAR = {2012},
    NUMBER = {6},
     PAGES = {2291--2303},
}
@article {Mccshen,
    AUTHOR = {McCullough, Scott and Shen, Li-Chien},
     TITLE = {On the {S}zeg\H{o} kernel of an annulus},
   JOURNAL = {Proc. Amer. Math. Soc.},
  FJOURNAL = {Proceedings of the American Mathematical Society},
    VOLUME = {121},
      YEAR = {1994},
    NUMBER = {4},
     PAGES = {1111--1121},
}
@incollection {badeaspectral,
    AUTHOR = {Catalin Badea and Bernhard Beckermann},
     TITLE = {Spectral Sets},
   EDITION = {\textit{Handbook of Linear Algebra}, (L. Hogben, ed.), second edition},
 PUBLISHER = {CRC Press},
      YEAR = {2013},
}
@article {Harnackcrho,
    AUTHOR = {Cassier, Gilles and Benharrat, Mohammed and Belmouhoub,
              Soumia},
     TITLE = {Harnack parts of {$\rho$}-contractions},
   JOURNAL = {J. Operator Theory},
  FJOURNAL = {Journal of Operator Theory},
    VOLUME = {80},
      YEAR = {2018},
    NUMBER = {2},
     PAGES = {453--480},
     
}
@article {manyauthorscrouz,
    AUTHOR = {Bickel, Kelly and Gorkin, Pamela and Greenbaum, Anne and
              Ransford, Thomas and Schwenninger, Felix L. and Wegert, Elias},
     TITLE = {Crouzeix's conjecture and related problems},
   JOURNAL = {Comput. Methods Funct. Theory},
  FJOURNAL = {Computational Methods and Function Theory},
    VOLUME = {20},
      YEAR = {2020},
    NUMBER = {3-4},
}
@article {Williams,
    AUTHOR = {Williams, James P.},
     TITLE = {Schwarz norms for operators},
   JOURNAL = {Pacific J. Math.},
  FJOURNAL = {Pacific Journal of Mathematics},
    VOLUME = {24},
      YEAR = {1968},
     PAGES = {181--188},
}
\addcontentsline{toc}{section}{Bibliography}
%-------------------------------------------------------------------------------

\end{document}