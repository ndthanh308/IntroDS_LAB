The shape of the event isotropy observables $\mcal{I}^{N = 2}_\mrm{Ring}$, $\mcal{I}^{N = 128}_\mrm{Ring}$, and $\mcal{I}^{N = 16}_{\mrm{Cyl}}$ are measured in inclusive bins of $N_{\mrm{jet}}$ and $H_{\mrm{T2}}$. These observables are defined in Table 1 of Reference \cite{5}. The inclusive jet-multiplicity bins range from $N_{\mrm{jet}} \geq 2$ to $N_{\mrm{jet}} \geq 5$, and the inclusive bins of $H_{\mrm{T2}}$ are $H_{\mrm{T2}} \geq 500$~GeV, $H_{\mrm{T2}} \geq 1000$~GeV and $H_{\mrm{T2}} \geq 1500$~GeV. 

\subsection{Systematic uncertainties}

The dominant sources of systematic uncertainties are related either to the jet energy resolution or to the choice of MC model used in the unfolding. 

\subsection{Theoretical predictions}

Samples of MC simulated dijet and multijet events are used in this analysis. \pythia 8.230 is used as the nominal MC generator. Two sets of \sherpa 2.2.5 with the default AHADIC cluster hadronisation or with the \sherpa interface to the Lund string hadronisation model were used. Two sets of \herwig 7.1.3 multijet events were generated with the default cluster hadronisation model and either the default angle-ordered parton shower (PS) or alternative dipole PS. Two additional samples of dijet events with NLO matrix element accuracy were produced with \powhegbox, matched to either the \pythia 8 or angle-ordered \herwig 7 parton shower.

\subsection{Results}

The unfolded data are compared with predictions from several state-of-the-art Monte Carlo models. Good agreement is often observed between the LO and NLO Monte Carlo generators throughout the non-isotropic region of a given distribution; poorer agreement is seen as particle configurations become more isotropic.

The most inclusive measurement of $1 - \mcal{I}^{N = 128}_\mrm{Ring}$ cross-sections is shown in Figure \ref{fig_3_1}. This distribution is saturated by well-balanced dijets events and by multijet events with isotropic configurations. The \powheg predictions are found to strongly disagree with those of the other MC generators. Large differences are also found between the \herwig angle-ordered and dipole shower models. No notable differences are seen between the \sherpa hadronisation models. The most inclusive measurement of $1 - \mcal{I}^{N = 16}_{\mrm{Cyl}}$ cross-sections is shown in Figure \ref{fig_3_2}. Multijet events that cover the rapidity–azimuth plane with activity in both the central and forward regions produce the highest values for this observable. None of the MC predictions accurately describe this observable. The predictions from the \pythia and \powheg samples are consistent except at low values, where \pythia overestimates the observed cross-section.

% Figure environment removed