Differential cross sections are measured as functions of the photon transverse energy, $\etgamma$, in different regions of the photon pseudorapidity, $\etagamma$, for $\etgamma > 250$~GeV and $\varetagamma < 2.37$. Photons are required to be isolated both at particle and detector levels using a `fixed cone' criterion: $E^{\mathrm{iso}}_{\mathrm{T}} < E^{\mathrm{iso}}_{\mathrm{T,cut}} = 4.2 \cdot 10^{-3} \cdot \etgamma +4.8$~GeV \cite{3}. The dependence of the cross section on the isolation cone radius, $R$, is investigated by measuring the ratios of the cross sections for $R = 0.2$ and $0.4$.

\subsection{Systematic uncertainties}

The dominant systematic uncertainties affecting the measurement of the differential cross sections arise from the uncertainty on the photon energy scale, the luminosity uncertainty, the uncertainty on the $R^{\mathrm{bg}}$ correlation, and the pile-up modelling. The total systematic uncertainty varies in the range $(3-20)\%$. The measurement of the ratios of the cross sections benefits from large cancellations for the uncertainties that are independent of the isolation cone radius. The total uncertainty is typically $<1\%$.

\subsection{Theoretical predictions}

The next-to-leading-order (NLO) pQCD calculations included in this measurement were computed using the programs \jetphox $1.3.1\_2$ and \sherpa $2.2.2$. The next-to-next-to-leading-order (NNLO) pQCD predictions are calculated in the \nnlojet framework. There are several differences between these calculations which are described in detail in Section 8 of Reference \cite{3}. The total uncertainty ranges from $\approx (10 - 15)\%$ for \jetphox, and it is $\approx 20\%$ for \sherpa. For the NNLO pQCD calculation of \nnlojet, the uncertainties are in the range $(1-6)\%$, being smaller than those in the NLO pQCD prediction by a factor $2-15$. The total uncertainty in the ratios of the cross sections benefits from large cancellations, being $\approx 1.5\%$ ($\approx 1\%$) for the predictions of \jetphox and \sherpa (\nnlojet).

\subsection{Results}

The ratios of the predictions from \jetphox based on different PDFs and the data are shown in Figure \ref{fig_1_1} for $R=0.2$. The calculations of \jetphox are consistent with the measurements within uncertainties. Predictions based on MMHT2014, CT18 and NNPDF3.1 PDF sets are similar and the closest to the data for $\varetagamma < 1.37$ and $1.81 < \varetagamma < 2.37$. For $1.56 < \varetagamma < 1.81$, the predictions based on the HERAPDF2.0 PDF and ATLASpdf21 sets are the closest to the data.

% Figure environment removed

The dependence of the inclusive photon cross section on $R$ is investigated by measuring the ratios of the cross sections for $R = 0.2$ and $0.4$, shown in Figure \ref{fig_1_2}. \sherpa predictions overestimate the data, while \jetphox gives a good description. The NNLO pQCD predictions give an excellent description of the data. These measurements provide a very stringent test of pQCD with reduced experimental and theoretical uncertainties, validating the underlying theoretical description up to $\mcal{O}(\alphas^2)$.