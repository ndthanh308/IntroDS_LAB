Differential cross sections for the TEEC and ATEEC observables as functions of $\cos\phi$ are measured and compared with several Monte Carlo (MC) predictions inclusively and in ten bins of the scalar sum of the transverse momenta of the two leading jets, $H_{\mrm{T2}}$. The selected jets are required to have $\pt > 60$~GeV and $\vareta < 2.4$. The TEEC and ATEEC measured cross sections are compared to NNLO pQCD predictions for first time. A determination of the strong coupling constant is performed from this comparison.

\subsection{Systematic uncertainties}

The systematic uncertainties arising from the jet energy scale uncertainties dominate both the TEEC and ATEEC measurements, while the MC modelling and jet energy resolution uncertainties are also important for the TEEC and ATEEC, respectively. The total systematic uncertainty is of order $2\%$ ($1\%$) for the TEEC (ATEEC).

\subsection{Theoretical predictions}

The theoretical predictions for the TEEC and ATEEC functions are calculated at leading-order (LO), NLO and NNLO in pQCD. The NNLO pQCD corrections include real–real, real–virtual and virtual–virtual finite terms, as well as single- and double-unresolved terms and the finite remainder. The renormalisation and factorisation scales are set to the scalar sum of the transverse momenta of all final-state partons, $\muR = \muF = \hat{H}_{\mrm{T}}$. The uncertainties are computed from the scale variations, the PDF uncertainties, and those in the non-perturbative corrections to the pQCD predictions.

\subsection{Results}

The comparison of the measured cross sections for the TEEC and ATEEC functions with the predictions at different orders in pQCD are shown in Figures 6 and 7 of Reference \cite{4}. The description of the data provided by the NNLO pQCD predictions is excellent, improving with respect to the NLO prediction. The reduction of the scale uncertainties up to a factor of 3 is made evident from these figures, as well as the improvement in the description.

The strong coupling constant at the scale of the pole mass of the $Z$ boson, $\alphasmz$, was determined  from the comparison of the data to the theoretical predictions by means of a $\chi^2$ fit described in Section 9 of Reference \cite{4}, for both the TEEC and ATEEC distributions. To compare the results with other experiments, the value of the energy scale $Q$ is chosen as half of the average value of $\hat{H}_{\mrm{T}}$ for each $H_{\mrm{T2}}$ bin. Figure \ref{fig_2_2} shows the values of $\alphas(Q)$ from the fit to the TEEC distribution together with the world average band provided by the Particle Data Group \cite{6} and values of $\alphas$ obtained in other analyses. The results show a good agreement between all measurements and the renormalisation group equation prediction.

% Figure environment removed