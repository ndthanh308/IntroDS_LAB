% Prompt photon production
The production of prompt photons\footnote{Photons that are not secondaries from hadron decays are considered as prompt.} with large transverse momenta ($\pt$) in proton-proton ($pp$) collisions, $pp \rightarrow \gamma + X$, provides a testing ground for perturbative QCD (pQCD). Since the dominant production mechanism in proton-proton collisions at the LHC proceeds via the $qg \rightarrow q\gamma$ process, measurements of prompt-photon production are sensitive to the gluon density in the proton.

In hadron colliders, due to the abundance of photons produced in neutral hadron decays and the contribution of fragmentation processes, prompt-photon production is studied by requiring photon isolation. This requirement is based on the amount of transverse energy ($\et$) allowed inside a cone of radius $R$ in the pseudorapidity-azimuth plane around the photon. For a detailed overview of the selection cuts employed in this measurement, please refer to Section 4 of Reference \cite{3}.

% Transverse-Energy-Energy Correlations
Multijet final states, produced in $pp$ collisions with large momentum transfer at the LHC, also provide an ideal testing ground for pQCD. Event shapes are a class of observables defined as functions of the final-state particles four momenta, which characterise the hadronic energy flow in a collision.

Transverse energy-energy correlations (TEEC) and their associated azimuthal asymmetries (ATEEC) are studied in Reference \cite{4}. The TEEC function is defined as the transverse-energy-weighted distribution of the azimuthal differences between jet pairs in the final state, while the ATEEC function is built as the difference between the forward ($\cos\phi > 0$) and backward ($\cos\phi < 0$) parts of the TEEC, as discussed in Section 1 of Reference \cite{4}. Both the TEEC and ATEEC functions are sensitive to gluon radiation and show a clear dependence on the strong coupling.

% Event isotropies using optimal transport
A novel class of event shapes, broadly called \textit{event isotropies}, was recently proposed to quantify the isotropy of collider events in terms of a Wasserstein distance metric. These distances are framed in terms of optimal transport problems, using the `Energy-Mover's Distance' (EMD), defined as \textit{the minimum amount of `work' necessary to transport one event $\mathcal{E}$ with $M$ particles into another $\mathcal{E}^{\prime}$ of equal energy with $M^{\prime}$ particles, by movements of energy $f_{ij}$ from particle $i \leq M$ in one event to particle $j \leq M^{\prime}$ in the other}, as shown in Equations (1) and (2) of Reference \cite{5}.

An event isotropy $\mathcal{I}$ is defined as a Wasserstein distance between a collider event $\mathcal{E}$ and a (quasi-)uniform radiation pattern $\mathcal{U}$, determined using the EMD: $\mathcal{I} \, (\mathcal{E}) = \mathrm{EMD} \, (\mathcal{E},\mathcal{U})$. The event isotropy $\mathcal{I}$ is bounded on $\mathcal{I} \in [0,1]$, where the least (most) isotropic events take values approaching $\mathcal{I} = 1$ $(\mathcal{I} = 0)$.

Three measurements performed at $\sqrt{s} = 13$~TeV with the $\thirdlumi$ of data recorded by the ATLAS experiment \cite{2} are presented in these conference proceedings.