%\documentclass[main.tex]{subfiles}
\documentclass[aps,prx,preprint,superscriptaddress]{revtex4-2}


\usepackage{xr-hyper}
\usepackage{hyperref}
\usepackage{graphicx}
\usepackage{latexsym}
\usepackage{amsmath,amssymb}
\usepackage{xcolor}
\usepackage[utf8]{inputenc}
\usepackage[english]{babel}


\usepackage{float}

\graphicspath{{figures_suppl/}}

\hypersetup{
    colorlinks=true,
    linkcolor=blue,
    filecolor=magenta,      
    urlcolor=red,
    pdftitle={Supplementary MoS2Au},
    pdfpagemode=FullScreen,
    }


\makeatletter
\renewcommand{\thefigure}{S\arabic{figure}}


\makeatother

% Command for editing
%\newcommand{\LR}[1]{\textcolor{red}{[LR:#1]}}
\usepackage{xr}
%\externaldocument{main}
\renewcommand{\v}[1]{\ensuremath{\mathbf{#1}}}

\makeatletter
\newcommand*{\addFileDependency}[1]{% argument=file name and extension
  \typeout{(#1)}
  \@addtofilelist{#1}
  \IfFileExists{#1}{}{\typeout{No file #1.}}
}
\makeatother

\newcommand*{\myexternaldocument}[1]{%
    \externaldocument{#1}%
    \addFileDependency{#1.tex}%
    \addFileDependency{#1.aux}%
}

\myexternaldocument{main}
\newcommand{\TODO}[1]{\textcolor{red}{[#1]}}


\begin{document}




\title{Supporting Information for 
Twist angle dependent electronic properties of exfoliated single layer MoS$_2$ on Au(111)}

\author{Ishita Pushkarna}
\affiliation{DQMP, Université de Genève, 24 Quai Ernest Ansermet, CH-1211 Geneva, Switzerland}
\affiliation{These authors contributed equally to this work.}
\author{\'{A}rp\'{a}d P\'{a}sztor}
\email{arpad.pasztor@unige.ch}
\affiliation{DQMP, Université de Genève, 24 Quai Ernest Ansermet, CH-1211 Geneva, Switzerland}
\affiliation{These authors contributed equally to this work.}
\author{Christoph Renner}
\email{christoph.renner@unige.ch}
\affiliation{DQMP, Université de Genève, 24 Quai Ernest Ansermet, CH-1211 Geneva, Switzerland}

\maketitle

\tableofcontents

\newpage
  
\section{Substrate and sample characterization}
\label{sup:sec_substrateandsample}
We obtained millimeter-sized monolayer (ML) MoS$_2$ by exfoliating 2H-MoS$_2$ single crystals onto template-stripped gold substrates. Exfoliation onto these substrates primarily resulted in large MLs, with only a few tiny thicker flakes. A typical ML is outlined in black in the optical microscope image in Figure~\ref{fig:suppl_fig1}(a). 

Exfoliated MLs were identified through the characteristic Raman spectra of MoS$_2$ on Au \cite{Velicky2020, Pollmann2021}. Due to the strong interaction with the substrate, ML MoS$_2$ has a markedly different Raman spectrum on gold (Figure~\ref{fig:suppl_fig1}(b), blue spectrum) than on SiO$_2$/Si \cite{Li2012}. The A$_{1g}$ and E$_{2g}$ modes in the ML are shifted with respect to their bulk positions on both substrates. However, the shift is different on Au, with an additional splitting of the A$_{1g}$ mode appearing around 397 cm$^{-1}$ (Figure~\ref{fig:suppl_fig1}(b), blue spectrum).
The bilayer spectrum features four distinguishable peaks (Figure~\ref{fig:suppl_fig1}(b), orange spectrum), which correspond to the combined peaks observed in bulk and ML specimen. This spectrum can be understood as a combined contribution from a bulk spectrum and from the first ML, which is affected by the substrate. The evolution of Raman peaks as a function of MoS$_2$ flake thickness is summarized in Figure~\ref{fig:suppl_fig1}(c) where we show the position of the peaks at several different locations on each flake. 

The gold substrates were characterized using X-ray diffraction (XRD) and atomic force microscopy (AFM). The XRD spectrum in Figure~\ref{fig:suppl_fig1}(d) shows a peak around 38.1\textdegree, indicating (111) orientation, and a peak near 81.81\textdegree~corresponding to the (222) reflection of the gold surface. The peaks at 34\textdegree~and 69.25\textdegree~correspond to the Si(100) substrate \cite{Krishnamurthy2014,Zhao2004}. 
The large-scale AFM image in Figure~\ref{fig:suppl_fig1}(e) reveals an ultra-flat gold surface, with a typical roughness of less than a nanometer. Disposing of such flat and freshly exposed gold surfaces is essential to exfoliate large-area MoS$_2$ MLs. The polycrystalline structure of the gold substrate is nicely resolved in smaller range AFM (Figure~\ref{fig:suppl_fig1}(f)) and STM images (Figure~\ref{fig:suppl_fig1}(g)).


% Figure environment removed


\newpage
\clearpage

\section{Determination of the twist angle}
\label{sup:sec_twistangle}

We determine the twist angle between the MoS$_2$ lattice and the Au(111) surface based on the Fourier-transform (FT) of atomically resolved STM topographies exemplified in Figure~\ref{fig:suppl_fig2}(a). We identify and select the peaks corresponding to the moiré and to the MoS$_2$ wave vectors marked by green and red circles in Figure~\ref{fig:suppl_fig2}(b), respectively, and calculate the length of each wave vector. The length of the gold lattice wave vector is given by the sum of the wave vectors of MoS$_2$ and moiré lattices. We choose the pairs of moiré and MoS$_2$ wave vectors which give the value closest to the lattice constant of gold to calculate the corresponding twist angle using supplementary equation~(1). In the example of Figure~\ref{fig:suppl_fig2}, we find a twist angle of 7.7\textdegree. The uncertainty on the twist angles determined in this way is $\pm 0.5$\textdegree, primarily limited by the precision of measuring the k-vectors in the FTs. 
\begin{equation}
    \begin{split}
    cos({\varphi})&=\frac{\vec{k}_{\text{Au}}\cdot\vec{k}_{\text{MoS}_2}}{|\vec{k}_{\text{Au}}||\vec{k}_{\text{MoS}_2}|}
    \end{split}
\end{equation}

An alternative approach to determine the twist angle is to calculate the moiré wavelength as a function of the twist angle. This allows to extract the twist angle from the plot of $|\vec{k}_\text{Moiré}|/|\vec{k}_{\text{MoS}_2}|$ as a function of twist angle (Figure~\ref{fig:suppl_fig2}(c)). 

% Figure environment removed

\newpage
\clearpage



\section{Characterization of gold grains and their interfaces}
\label{sup:sec_goldstep}
XRD shows that the gold film is composed of [111]-oriented grains. High-resolution STM topography is consistent with XRD, and provides atomic-scale insight into the tilting and rotation of the grains about their [111]-axis. In Figure~\ref{fig:suppl_fig8}, we analyze a high-resolution topographic image of a continuous MoS$_2$ ML spanning two adjacent gold grains. We can determine the step heights in each grain by successively flattening the STM image with respect to a terrace on the left-hand side grain and then with respect to a terrace on the right-hand side grain. In both cases, we extract a height difference between the terraces of about 228~pm, consistent with the step height expected for Au(111) \cite{Barth1990, Sun2008}, as shown in the lower panels of Figure~\ref{fig:suppl_fig8}(a),(b). 

% Figure environment removed

The same topographic STM image as in Figure~\ref{fig:suppl_fig8} is shown in Figure~\ref{fig:suppl_fig4}(a), but leveled to highlight the moiré patterns in the two adjacent grains. A closer look at the grain boundary region in Figure~\ref{fig:suppl_fig4}(b) clearly shows a continuous MoS$_2$ flake extending over the entire field of view. The changing moir\'e pattern is thus a direct consequence of the different orientations of the two Au(111) grains. Continuous MoS$_2$ flakes spanning Au(111) grain boundaries provide a unique platform to study the electronic properties as a function of twist angle in a single device with the same tip, excluding any spurious experimental effects that might affect the data.

% Figure environment removed


\newpage
\clearpage


\section{Nanobubbles on the surface}
\label{sup:sec_nanobubbles}
The large area topographic STM image in Figure~\ref{fig:suppl_fig3}(a) shows a single continuous MoS$_2$ ML extending over the entire image and spanning different Au(111) grains and terraces. We observe occasional bubbles, usually located around step edges (Figure~\ref{fig:suppl_fig3}(b)). They correspond to regions where the MoS$_2$ ML is decoupled from the substrate. Similar features have been observed in other studies of exfoliated TMDs \cite{Peto2019} and graphene \cite{Khestanov2016} on different substrates. These decoupled regions show purely semi-conducting $I(V)$ spectra (Figure~\ref{fig:suppl_fig3}(c), orange spectrum), whereas the surrounding regions show hybridized spectra, with a reminiscence of the semi-conducting nature of MoS$_2$ (Figure~\ref{fig:suppl_fig3}(c), blue spectrum). 

% Figure environment removed
\newpage
\clearpage


\section{Spatial mapping of VBM, CBm, and gap using $I(V, \vec{r})$ curves}
\label{sup:sec_IVfitting}
Here we show how we extract the CB and VB edges from $I(V)$ spectra by fitting the $I(V)$ curves following Zhuou et al. \cite{Zhou2016} with a slightly different model DOS. We considered a constant DOS for the purely semi-conducting decoupled MoS$_2$ areas. However, for the hybridized films, a better fit is obtained using a square-root energy-dependent semiconducting DOS (3D) in each band. To take into account the metallic background due to the hybridization, we added a non-zero constant to the DOS over the entire energy range. The VB and CB regions were then fitted separately, using independent constants for each half of the spectrum (each spectrum is divided into two parts at the approximate center of the gap near -0.5~V). This ensures a roughly equal number of data points to fit both sides. In Figure~\ref{fig:suppl_fig5}(a), we show an example of a fitted $I(V)$ spectrum.

Figure~\ref{fig:suppl_fig5}(b), (c), and (d) show the spatial mapping of VBM, CBm, and the band gap, obtained by fitting $I(V)$ curves. Both methods (fitting of $dI/dV(V)$ or $I(V)$ spectra) give the same information on band edge modulation and attest the suitability of any of the two methods.
% Figure environment removed
\newpage
\clearpage

\section{Conductance maps ($dI/dV(V,\Vec{r})$)  as a function of bias }
\label{sup:sec_bias_dep_LDOS}

  

To address the electronic or structural origin of the moiré pattern observed in STM images, we examined the $dI/dV(V,\Vec{r})$ conductance maps of a 2.2\textdegree~twist angle heterostructure as a function of bias in  Figure~\ref{fig:figure4}.  The main observations are the following: i) we do not see any significant modulation of the local density of states (LDOS) at bias voltages inside the MoS$_2$ gap (Figure~\ref{fig:figure4}(c)), probably because the tip is stabilized far outside the gap and the signal from these low energy states is too weak; ii) outside of the gap, the LDOS is modulated at the  moiré period (Figure~\ref{fig:figure4}(a),(b),(d),(e)); iii) the moiré contrast inverts at least three times in the examined energy range, once at negative bias (Figure~\ref{fig:figure4}(a)-(b)), once across the gap (Figure~\ref{fig:figure4}(b)-(d)), and once at positive bias (Figure~\ref{fig:figure4}(d)-(e)). Such contrast inversions in the conductance maps as a function of energy are not consistent with morphologic features: hills cannot swap their positions with trenches three times as a function of imaging bias. 

% Figure environment removed


The electronic origin of the observed moiré pattern can be entirely understood by comparing the $dI/dV(V)$ curves measured at a crest and a valley position: they cross each other at several biases and are nearly equal in the gap region (see Figure~\ref{fig:suppl_fig10}(a) and Figure~2(b)). Therefore, depending on the bias chosen to extract the conductance maps, the LDOS will be largest when the tip is over a crest or when it is over a valley of the moiré, leading to changing and inverted contrasts. To make sure the bias set-point does not affect these observations, we did the same analysis for a $dI/dV(V,\Vec{r})$  map acquired at a negative bias set point of -3~V. The result is exactly the same as seen in Figure~\ref{fig:suppl_fig6}. 

The changing contrast of the moiré pattern in Figures~\ref{fig:figure4} and ~\ref{fig:suppl_fig6} is a direct consequence of the different tunneling spectra measured when the STM tip sits at a crest or at a valley of the moiré pattern (Figure~\ref{fig:suppl_fig10}(a)). The moiré pattern contrast can be quantified by summing the intensity of the six peaks corresponding to the moiré pattern in the Fourier transforms (FTs) of the conductance maps extracted from $dI/dV(V,\Vec{r})$. This total intensity is plotted as a function of bias voltage in Figure~\ref{fig:figure4}(f) and Figure~\ref{fig:suppl_fig6}(f). The color code is defined in the following way: red when the conductance is larger at the moiré crest, corresponding to the contrast in Figure~\ref{fig:figure4}(a),(d); green when the conductance is larger in the moiré valley, corresponding to the contrast in Figure~\ref{fig:figure4}(b),(e); blue when the conductances are nearly the same at the moiré crest and in the moiré valley, corresponding to the contrast in Figure~\ref{fig:figure4}(c). This bias dependence of the moiré contrast can be directly compared to the numerical difference between the two conductance curves measured at the moiré crest and moiré valley positions in Figure~\ref{fig:suppl_fig10}(a). In  Figure~\ref{fig:suppl_fig10}(c), we plot $dI/dV(V, \text{crest})-dI/dV(V, \text{valley})$, with a positive (negative) result represented by a red (green) dot. This curve is remarkably similar to the one extracted from the Fourier analysis of the conductance maps and confirms the electronic origin of the moiré pattern contrast. 


% Figure environment removed


% Figure environment removed


\newpage
\clearpage



\section{Twist angle dependent average charge transfer}
\label{sec:suppl_fig_modeling}

In this model, we consider the charge transfer at a given Au-S nearest neighbor pair to be proportional to $1/d_{\text{eff}}$. We calculate the average charge transfer as:
\begin{equation}
    \overline{\Delta Q}=\frac{1}{N}\sum_{i=1}^{N} \frac{1}{d_{\text{eff}}(\vec{r}_i)}.
\end{equation}
While the average of $d_{\text{eff}}(\vec{r}_i)$ does not change as a function of twist angle (Figure~\ref{fig:suppl_fig_modeling}(a)) we clearly see in Figure~\ref{fig:suppl_fig_modeling}(b) --similarly to the model presented in the main text-- that the average charge transfer is decreasing with increasing twist angle.% in this $1/d_{\text{eff}}$-model. 
% Figure environment removed




%%%%%%%%%%%%%%%%%%%%%%%%%%%%%%%%%%%%%%%%%%%%%%%%%%%%%%%%%%%%%%%%%%%%%%%%%%%%%%%%%

 \bibliographystyle{ieeetr}
 \bibliography{biblio.bib}


\end{document}