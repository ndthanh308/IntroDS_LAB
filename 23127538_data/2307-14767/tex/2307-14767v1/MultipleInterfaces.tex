\documentclass[11pt,a4paper]{preprint}
%Replace the above by the following line to get another template style
% \documentclass[11pt,a4paper]{amsart}

\usepackage{cancel}
\usepackage{geometry}
\usepackage{soul}
%\usepackage[subtle]{savetrees}
%\usepackage[margin=2cm]{geometry}
\usepackage{tikz,amsmath, amssymb,bm,color, amsthm,amsfonts}
\usetikzlibrary{positioning, calc,chains,fit,shapes}
%\usetikzlibrary{circuits.logic.US,circuits.logic.IEC,fit}
\usepackage{enumerate}
\usepackage{comment}
\usepackage{tikz}
\usepackage{graphics}
%\usepackage[cm]{fullpage}
\usepackage{longtable}
\usepackage{mdframed}
\usepackage{caption}
\usepackage{subcaption}
\usepackage{slashbox}
\usepackage{url}
\usepackage{framed}
\usepackage{array}
\usepackage{tabu}
\usepackage{lscape}
\usepackage{multirow}
\usepackage{ulem}
\usepackage{multicol}
\usepackage{placeins}
\usepackage{cite}
\usepackage{enumitem}
\usepackage{mathtools}
%\usepackage[numbers]{natbib}
%\usepackage{mathtools}
%\usepackage{authblk}

\mdfsetup{skipabove=2pt,skipbelow=2pt}
%\setlenght {\marginparwidth }{2cm}
%\usepackage{todonotes}

%\usepackage{floatrow}
%\usepackage{adjustbox}
%\setlength{\extrarowheight}{.05ex}
%\renewcommand\thesubfigure{\roman{subfigure}}


%\newtheorem{theorem}{Theorem}[section]
%\newtheorem{lemma}[theorem]{Lemma}
%\newtheorem{observation}[theorem]{Observation}
%\newtheorem{corollary}[theorem]{Corollary}
%\newtheorem{proposition}[theorem]{Proposition}
%\newtheorem{definition}[theorem]{Definition}
\newtheorem{construction}{Construction}
%\newtheorem{conjecture}{Conjecture}
%\newtheorem{remark}[theorem]{Remark}

\newcommand{\pname}[1]{\textnormal{\textsc{#1}}}
\newcommand{\cclass}[1]{\textnormal{\textsf{#1}}}
\newcommand{\nog}{nine} % no of members in the gang!
\newcommand{\nogd}{nineteen} % no of members in the gang - for deletion/completion
\newcommand{\nogl}{eighteen} % no of members in the larger gang - for editing
\newcommand{\nogld}{thirty eight} % no of members in the larger gang - for deletion/completion
\newcommand{\diffnog}{ten} %
%\newcommand{\dominatedby}{dominated by} %
%\newcommand{\dominatingset}{dominating set} %
%\newcommand{\dominates}{dominates} %
\newcommand{\simulates}{simulates} %
\newcommand{\baseset}{base} %
\newcommand{\issimulatedby}{is simulated by} %

\newcommand{\StarSAT}{\pname{8-SAT$_{\geq 6}$}}
\newcommand{\FSAT}{\pname{4-SAT$_{\geq 2}$}}
\newcommand{\FISAT}{\pname{5-SAT$_{\geq 3}$}}
\newcommand{\SIXSAT}{\pname{6-SAT$_{\geq 4}$}}
\newcommand{\ESAT}{\pname{8-SAT$_{\geq 6}$}}
\newcommand{\KSAT}{\pname{$k$-SAT$_{\geq {k-2}}$}}
\newcommand{\KSATO}{\pname{$k$-SAT}}
\newcommand{\ESATO}{\pname{8-SAT}}
\newcommand{\FSATO}{\pname{4-SAT}}
\newcommand{\FISATO}{\pname{5-SAT}}
\newcommand{\TSAT}{\pname{3-SAT}}
\newcommand{\HED}{\pname{${H}$-free Edge Deletion}}
\newcommand{\AEE}{\pname{${A}$-free Edge Editing}}
\newcommand{\AED}{\pname{${A}$-free Edge Deletion}}
\newcommand{\TSED}{\pname{$t$-star-free Edge Deletion}}
\newcommand{\ATSED}{\pname{Annotated $t$-star-free Edge Deletion}}
\newcommand{\AFSED}{\pname{Annotated $4$-star-free Edge Deletion}}
\newcommand{\FSED}{\pname{$4$-star-free Edge Deletion}}
\newcommand{\FVSED}{\pname{$5$-star-free Edge Deletion}}
\newcommand{\HEE}{\pname{${H}$-free Edge Editing}}
\newcommand{\HEC}{\pname{${H}$-free Edge Completion}}
\newcommand{\HDEE}{\pname{${H'}$-free Edge Editing}}
\newcommand{\HDDEE}{\pname{${H''}$-free Edge Editing}}
\newcommand{\HDED}{\pname{${H'}$-free Edge Deletion}}
\newcommand{\HDEC}{\pname{${H'}$-free Edge Completion}}
\newcommand{\HBEE}{\pname{${\overline{H}}$-free Edge Editing}}
\newcommand{\HBED}{\pname{${\overline{H}}$-free Edge Deletion}}
\newcommand{\HBEC}{\pname{${\overline{H}}$-free Edge Completion}}
\newcommand{\HOEDCE}{\pname{${H_1}$-free Edge Deletion(Completion/Editing)}}
\newcommand{\HEDCE}{\pname{${H}$-free Edge Deletion(Completion/Editing)}}
\newcommand{\HEEDC}{\pname{${H}$-free Edge Editing(Deletion/Completion)}}
\newcommand{\HDEEDC}{\pname{${H'}$-free Edge Editing(Deletion/Completion)}}
\newcommand{\BFED}{\pname{Bow-free Edge Deletion}}
\newcommand{\ABFED}{\pname{Annotated Bow-free Edge Deletion}}
\newcommand{\DTIS}{\pname{Distance-3 Independent Set}}
\newcommand{\SVC}{\pname{Strong Vertex Cover}}
\newcommand{\CLIQUE}{\pname{Clique}}
\newcommand{\IS}{\pname{Independent Set}}
\newcommand{\PFS}{\pname{Propagational-$f$ Satisfiability}}
\newcommand{\RHED}{\pname{Restricted ${H}$-free Edge Deletion}}
\newcommand{\RHEC}{\pname{Restricted ${H}$-free Edge Completion}}
\newcommand{\RHDED}{\pname{Restricted ${H'}$-free Edge Deletion}}
\newcommand{\RHDEC}{\pname{Restricted ${H'}$-free Edge Completion}}
\newcommand{\RHEE}{\pname{Restricted ${H}$-free Edge Editing}}
\newcommand{\PH}{$\cclass{NP} \subseteq \cclass{coNP/poly}$}
\newcommand{\NOPH}{$\cclass{NP} \not\subseteq \cclass{coNP/poly}$}
\newcommand{\LG}{\mathcal{W}}
\newcommand{\LGD}{\mathcal{W}'}
\newcommand{\LGDD}{\mathcal{W}''}


%\let\oldvee\vee
\renewcommand\vee{\boxtimes}

\newcommand\addvmargin[1]{
  \node[fit=(current bounding box),inner ysep=#1,inner xsep=0]{};
}
\setlength{\fboxrule}{0pt}

\newcommand{\defstage}[2]{% PGD Version
  \hfill\\\smallskip\noindent%
  \begin{tabularx}{\textwidth}{|l X|}%
    \hline%
    \multicolumn{2}{|l|}{\textbf{#1}}\\%
    &#2\\\hline%
  \end{tabularx}%
%  \smallskip%
}
\setlength\extrarowheight{15pt}

\newcounter{rowcntr}[table]
\renewcommand{\therowcntr}{\thetable.\arabic{rowcntr}}

% A new columntype to apply automatic stepping
\newcolumntype{N}{>{\refstepcounter{rowcntr}\therowcntr}c}

% Reset the rowcntr counter at each new tabular
\AtBeginEnvironment{longtabu}{\setcounter{rowcntr}{0}}

\newcounter{rowcntra}[table]
\renewcommand{\therowcntra}{\arabic{rowcntra}}

% A new columntype to apply automatic stepping
\newcolumntype{M}{>{\refstepcounter{rowcntra}\therowcntra}c}

% Reset the rowcntr counter at each new tabular
\AtBeginEnvironment{tabular}{\setcounter{rowcntra}{0}}

\newcommand{\NPC}{NP-Complete}


\newcommand{\highlight}[1]{\textcolor{blue}{#1}}
\newcommand{\dhanya}[1]{\textcolor{blue}{dhanya: #1}}


%\newcommand{\XCD1}[1]{\pname{$\chi_{cd}$\ensuremath{(#1)}}}
\newcommand{\XCD}{\pname{$\chi_{cd}$}}
\newcommand{\SC}{\pname{$\omega_{s}$}}

\newcommand{\CDC}{\textsc{CD-coloring}}
\newcommand{\SCP}{\textsc{Separated-Cluster}}
\newcommand{\TD}{\textsc{Total Domination}}
\newcommand{\ISP}{\textsc{Independent Set}}
\newcommand{\CC}{\textsc{Clique Cover}}
\newcommand{\TETHS}{Further, the problem cannot be solved in time \ensuremath{2^{o(|V(G)|)}}, unless the ETH fails}
%\usetikzlibrary{positioning,chains,shapes,calc}
\usetikzlibrary{fit}
\thispagestyle{empty}
\usetikzlibrary{
  graphs,
  graphs.standard
}
\setcounter{tocdepth}{2}

\begin{document}
\title{Entropic repulsion and scaling limit for a finite number of non-intersecting subcritical FK interfaces}


\author{Lucas D'Alimonte$^*$}
\institute{$^*$Université de Fribourg \newline \url{lucas.dalimonte@unifr.ch}}

\maketitle
\begin{abstract}
This article is devoted to the study of a finite system of long clusters of subcritical 2-dimensional FK-percolation with $q\geq 1$, conditioned on mutual avoidance. We show that the diffusive scaling limit of such a system is given by a system of Brownian bridges conditioned not to intersect: the so-called \emph{Brownian watermelon}. Moreover, we give an estimate of the probability that two sets of $r$ points at distance $n$ of each other are connected by distinct clusters. As a byproduct, we obtain the asymptotics of the probability of the occurrence of a large finite cluster in a supercritical random-cluster model.
\end{abstract}
\tableofcontents

\section{Introduction}

Rigorous understanding of the behaviour of \textit{interfaces} in statistical mechanics models has been the focus of intensive study for more than 50 years, especially in the case of the Ising model. The first rigorous results were perturbative and made use of the Pirogov--Sinaï theory to show that a low temperature two-dimensional Ising interface converges, after an appropriate diffusive scaling, towards a Brownian bridge~\cite{Gallavotti, higuchi}. However, these works are restricted to the very low temperature regime, even if the belief was that the result should hold for any subcritical temperature.

In the beginning of the XXI$^\text{st}$ century, the development and the understanding of the rigorous \textit{Ornstein--Zernike theory}, first in Bernoulli percolation and later on in the context of more dependent models such as the Ising and Potts models ~\cite{ornsteinzernikebernoulli2002, ozisingcampanino,ozrandomclustercampanino}, provided a new powerful tool for a detailed study of the subcritical phase of these percolation or spin models. The structural output of this theory is the probabilistic description of long clusters (or equivalently of long interfaces as we shall see below) in terms of one-dimensional ``irreducible pieces'' behaving almost independently  (for a precise statement, we refer to Theorem~\ref{oz theorem}). In particular, the diffusive scaling limit of interfaces at any subcritical temperature could be obtained in the case of the Ising model as a quite simple byproduct of this robust theory in the work of Greenberg and Ioffe~\cite{greenbergioffe} (see~\cite{kovchegov} for the simpler case of Bernoulli percolation). 

Later on, this technique has been found to be efficient for studying interfaces interacting with their environment. Indeed, the above mentioned works deal with unconstrained (also called \emph{free}) interfaces, but recent works have been extending the study of these interfaces to broader settings in which non-trivial interactions with the environment are added.  Let us cite~\cite{Ott_2018} for the case of a defect line in the Potts model, and --- much more related to this work ---~\cite{ottvelenikwachtelioffe} for the treatment of a Potts interface above a boundary wall. These examples of interfaces interacting with their close environment have turned out to be more delicate to handle and in certain conditions have been shown to exhibit highly non-trivial behaviours such as \textit{wetting transitions}, which have been studied in~\cite{IV18}. 

% A powerful tool to study these different cases, which has been thoroughly studied in~\cite{Ve2004} and successfully implemented in~\cite{ottvelenikwachtelioffe}, is the so-called \textit{entropic repulsion} phenomenon. In words, in certain regimes of the subcritical phase, the entropy of the interface wins over the energetic gain offered by the characteristics of the conditioning, allowing to locally treat it as an infinite volume interface. 

Of the same nature is the study of a system of multiple interfaces interacting together. Such a problem will be the focus of this work. Indeed, this paper determines the scaling behaviour of a finite number of long clusters of subcritical Fortuin--Kasteleyn (FK) percolation, conditioned not to intersect. Regarding the discussion above, it can be seen as the study of a finite number of interfaces interacting together. An interesting feature of this setting is that when conditioned on not intersecting, the interaction between the clusters can turn out to be \emph{attractive}, \emph{a priori} allowing the existence of a \emph{pinning} transition --- a regime where this attraction is so strong that the clusters actually remain at a bounded distance from each other. 

In this paper, we rule out the existence of such a transition. In the fashion of~\cite{ottvelenikwachtelioffe}, we show that the behaviour of this system obeys an \emph{entropic repulsion} phenomenon: the entropy caused by the large number of possible interfaces wins over the energetic reward obtained by staying close to each other, all the way down to the critical temperature. Such a phenomenon has been previously identified in a variety of settings, for instance in the three-dimensional semi-infinite Ising model at low temperatures~\cite{frohlichpfisterSemiInfiniteIsing}, the 2+1-dimensional SOS model above a hard wall~\cite{caputomartinellitoninelliEntropicRepulsionSOS}, a 1+1-dimensional interface above an attractive field in presence of a magnetic field~\cite{Ve2004} or a supercritical Potts interface above a wall~\cite{ottvelenikwachtelioffe}, to mention but a few works studying this phenomenon. 

In this work, entropic repulsion of the FK clusters at any subcritical temperature is established in Proposition~\ref{proposition global entropic repulsion}, which is probably the most important output of this work. As a byproduct, we derive two results regarding the global behaviour of such a system of conditioned interfaces. The first one is the diffusive scaling limit of such a system, which is shown to be a system of Brownian bridges conditioned not to intersect: the so-called \emph{Brownian watermelon}. Moreover, we observe that the entropic repulsion phenomenon also allows the computation --- up to a multiplicative constant --- of the probability of the existence of such a system of interfaces. Finally, as a byproduct of the latter observation, we also obtain the asymptotics of the probability of the occurrence of a large finite connection in the supercritical random-cluster model.  

The method is in spirit close to that of~\cite{ottvelenikwachtelioffe}, but we face some additional difficulties --- essentially due to the fact that the interaction is not only between a random interface and a deterministic object as in the latter work, but between several random interfaces, making the proofs more intricate as one has to control the \emph{joint} behaviour of these interacting objects: in particular Section~\ref{section RCM} contains some estimates concerning the simultaneous behaviour of the system of clusters that appear to be new. The proofs crucially rely on the Ornstein--Zernike theory for subcritical random-cluster models, developed in~\cite{ozrandomclustercampanino}.



\subsection{Definitions of the random-cluster model and the Brownian watermelon}
\subsubsection{The random-cluster model}
The model of interest is the so-called \textit{random-cluster model} (also known as \textit{FK-percolation}). We first recall its definition and a few basic properties (we refer to~\cite{duminilcopin2017lectures} for a complete exposition).
The random-cluster model on $\Z^2$ is a model of random subgraphs of $\Z^2$. Its law is described by two parameters, $p \in [0,1]$ and $q > 0$. 

Let $G = \left( V(G), E(G) \right)$ be a finite subgraph of $\Z^2$. We denote its \textit{inner boundary} (resp. \emph{outer boundary}) by 
\begin{multline*}
\partial G = \left\lbrace x \in V(G), \exists y \notin V(G), \lbrace x,y \rbrace \in E(\Z^2) \right\rbrace
\\ (\text{resp. } 
\partial_{\mathsf{ext}} G = \left\lbrace x \notin V(G), \exists y \in V(G), \lbrace x,y \rbrace \in E(\Z^2) \right\rbrace).
\end{multline*}
A \textit{percolation configuration} on $G$ is an element $\om \in \left\lbrace 0,1 \right\rbrace^{E(G)}$. We say that an edge $e \in G$ is \textit{open} if $\om(e) = 1$ and \textit{closed} otherwise. Two vertices $x,y \in \Z^2$ are said to be connected if there exists a path of nearest neighbour vertices $x=x_0, x_1, \dots, x_n = y$ such that the edges $\lbrace x_i, x_{i+1} \rbrace$ are open for every $0 \leq i \leq n-1$. In this case, we say that the event $\lbrace x \leftrightarrow y \rbrace$ occurs.  A \textit{vertex cluster} of $\om$ is a maximal connected component of the set of vertices (it can be an isolated vertex). Given a percolation configuration $\om$, we denote by $o(\om)$ its number of open edges, and by $k(\om)$ its number of vertex clusters. 


A \textit{boundary condition} on $G$ is a partition $\eta = P_1 \cup \dots \cup P_k$ of $\partial G$. From a configuration $\om \in \left\lbrace 0,1 \right\rbrace^{E(G)}$, we create a configuration $\om^\eta$ by identifying the vertices that belong to the same $P_i$ of $\eta$. Two particular boundary conditions, that we shall call the \textit{free boundary} condition (resp. wired boundary condition), consist in the partition made of singletons (resp. of the whole set $\partial G$). We shall write $\eta= 0$ (resp. $\eta=1$) for this specific boundary condition.
\begin{Def}
Let $G =  \left( V(G), E(G) \right)$ be a finite subgraph of $\Z^2$, and $\eta$ be a boundary condition on $G$. Let $p \in [0,1]$ and $q > 0$. The random-cluster measure on $G$ with boundary condition $\eta$ is the following probability measure on percolation configurations on $G$:
\begin{equ}
\phi^\eta_{p,q,G}\left( \om \right) = \frac{1}{Z^\eta_{p,q,G}} \left(\frac{p}{1-p}\right)^{o(\om)}q^{k(\om^\eta)},
\end{equ}
where $Z^\eta_{p,q,G}>0$ is the normalisation constant ensuring that $\phi^\eta_{p,q,G}$ is indeed a probability measure. We shall refer to $Z^\eta_{p,q,G}$ as the \textit{partition function} of the model.
\end{Def}

It is classical that for $\eta = 0$ and $\eta = 1$, this measure can be extended to the whole plane $\Z^2$, by taking the weak limit of the measures $\phi^\eta_{p,q,G_n}$ over any exhaustion $\left(G_n\right)_{n \in \N}$ of $\Z^2$, and that the limit measure does not depend of the choice of the exhaustion. Below, we will simply write $\phi_{p,q}^\eta$ instead of $\phi_{p,q,\Z^2}^\eta$. 

A very fundamental feature of this model is that it undergoes a \textit{phase transition}. Namely for any $q \geq 1$, there exists a critical parameter $p_c = p_c(q) \in (0,1)$ such that:

\begin{itemize}
    \item $\forall p < p_c(q), \phi^1_{p,q}\left( 0 \leftrightarrow \infty \right) = 0$;
    \item $\forall p > p_c(q), \phi^0_{p,q} \left( 0 \leftrightarrow \infty \right) > 0$,
\end{itemize}
where $\lbrace 0 \leftrightarrow \infty \rbrace$ is the event that the cluster of 0 is infinite.

We are going to be interested in the first case --- called \textit{the subcritical regime}. In this case it is well known that the choice of boundary conditions does not affect the infinite volume measure. We thus drop $\eta$ from the notation and simply write $\phi_{p,q}$ for the unique infinite volume measure when $p<p_c(q)$. Another important feature of the subcritical random-cluster model is the existence and the positivity of the following limit:
\begin{equ}\label{Def tau}
    \tau_{p,q} := \lim_{n \rightarrow \infty} -\frac{1}{n}\log\left[\phi_{p,q}\left(0 \leftrightarrow (n, 0) \right)\right].
\end{equ}
We call this quantity the \textit{inverse correlation length} in the direction $\Vec{e_1}$. Moreover, standard subadditivity arguments yield that 
\begin{equ}
    \forall  x \in \Z^2, \phi_{p,q} \left[ x \leftrightarrow x + (n,0)  \right] \leq \e^{-\tau_{p,q} n}.
\end{equ}
Since $p,q$ will be fixed through this work, we shall simply write $\tau>0$ instead of $\tau_{p,q}$.

% \begin{Rem}
% In the following work, we will adopt the quite non-standard convention of working with \textit{edge clusters} rather than vertex clusters. An edge cluster of a percolation configuration will simply be the set of the edges of a vertex cluster. Moreover, when $x$ is a vertex of $\Z^2$, we will denote by $\mathcal{C}_x$ its edge cluster (if the vertex cluster of $x$ is $\lbrace x \rbrace$, then we set $\mathcal{C}_x = \emptyset$). Thus, unless specified, in the rest of this work, the word "cluster" has to be understood as "edge cluster". Furthermore, for $x \in \Z^2$, we shall sometimes make a slight abuse of notation by writing that $x \in \mathcal{C}$ instead of writing "$x$ is the endpoint of some edge of $\mathcal{C}$". 
% \end{Rem}

\subsubsection{The Brownian watermelon}\label{subsection bw}
The Brownian watermelon is a stochastic process that arises in various areas of probability theory, like random matrix theory~\cite{baik2007}, integrable probability~\cite{johansonn2004}, but also more recently in the study of the KPZ universality class~\cite{hammond2016brownian}.

We give a brief definition of this object, and we refer to~\cite{oconnellyor},~\cite{GRABINER1999177} and~\cite{conditionallimittheoremsfororderedrandomwalks} for the full construction and details. Let $r \geq 1$ be an integer. We define the Weyl chamber of order $r$:
\begin{equ}
    W = \left\lbrace (x_1, \dots, x_r) \in \R^r, x_1 < \dots< x_r \right\rbrace.
\end{equ}
We shall also introduce the functional Weyl chamber in the interval $[s,t]$ for $0\leq s<t$ (the set $\mathcal{C}(\R^+, \R^r)$ denotes the space of continuous functions from $\R^+$ to $\R^r$):
\begin{equ}
    \mathcal{W}_{[s,t]} = \lbrace f \in \mathcal{C}([s,t], \R^r), \forall s \leq \ell \leq t, f(\ell) \in W  \rbrace.
\end{equ}
Moreover let $\Delta$ denote the Vandermonde function, defined for any $(x_1, \dots, x_r) \in \R^r$ by:
\begin{equ}
    \Delta(x_1, \dots, x_r) = \prod_{1 \leq i < j \leq r}(x_j-x_i). 
\end{equ}
\begin{Def}[Brownian watermelon]
The Brownian watermelon with $r$ bridges is the continuous process $\big(\bw^{(r)}_t\big)_{0\leq t \leq 1}$ obtained by conditioning $r$ independent standard Brownian bridges not to intersect in $(0,1)$. It is a random object of $\mathcal{C}([0,1], \R^r)$.
\end{Def}
\begin{Rem}
Since the non-intersection event has null probability for $r$ random bridges as soon as $r \geq 2$, the latter conditioning is rigorously done by means of a Doob $h$-transformation by the harmonic function $\Delta$. We refer to~\cite{oconnellyor} and~\cite{Invarianceprinciplesforrandomwalksincones} for the details of the construction (and the fact that $\Delta$ is harmonic for a system of $r$ standard bridges). Moreover, it can be shown, by means of the Karlin--McGregor formula, that for any $0<t<1$
\begin{equ}\label{equation marginale BW}
    \PP\left[ \bw^{(r)}_t \in \mathrm{d}z\right] \propto \frac{1}{\left(t(1-t)\right)^{r^2/2}}\Delta^2(z)\e^{-\frac{|z|^2}{2t(1-t)}}\1_{z \in W}\dif z.
\end{equ}
\end{Rem}
\begin{Rem}
Alternatively, the Brownian watermelon can be built \emph{via} the following method: consider a system $(B^\eps_t)_{0 \leq t \leq 1}$ of $r$ independent standard Brownian bridges started from 0, $\eps, \dots, (r-1)\eps$ respectively. Then under the conditioning on the event $\left\lbrace B^\eps_t \in \mathcal{W}_{0,1} \right\rbrace$ (this happens with positive probability), the following weak limit exists in $\mathcal{C}([0,1], \R^r)$ when $\eps \rightarrow 0$ and is called the Brownian watermelon:
\begin{equ}
   ( B^\eps_t )_{0\leq t \leq 1} \goes{(d)}{\eps \rightarrow 0}  {(\bw^{(r)}_t)_{0\leq t\leq 1}}.
\end{equ}
For more information on this construction, see~\cite{oconnellyor} and~\cite{hammond2016brownian}.
\end{Rem}

\begin{Notationsandconventions} If $a_n$ and $b_n$ are two sequences of real numbers, we shall write $a_n \sim b_n$ when $\frac{a_n}{b_n} \goes{}{n \rightarrow \infty}{1}$. We shall also write $a_n = o(b_n)$ when $\frac{a_n}{b_n} \goes{}{n \rightarrow \infty}{0}$ and $a_n = O(b_n)$ when there exists a constant $C> 0$ such that $\vert a_n \vert \leq C\vert b_n \vert$ for all $n \geq 0$. Moreover, we shall write $a_n \asymp b_n$ whenever $a_n = O(b_n)$ and $b_n = O(a_n)$. Finally, the generic notations $c,C>0$ will denote constants depending only on $p$ and $q$, that may change from line to line during computations. 
\end{Notationsandconventions}

\subsection{Exposition of the results}

In this paper, we study the scaling limit of a system of subcritical clusters conditioned on a connection and a non-intersection event. We first start by a precise definition of these percolation events.

\begin{Def}[Connection event, Non-intersection event]
Let $x, y \in W\cap\Z^r$ and $n\geq 0$.
Then we define the multiple connection event $\con^n_{x,y}$ by
\begin{equ}
    \con^n_{x,y} = \left\lbrace \forall 1 \leq i \leq r, (0,x_i) \leftrightarrow (n,y_i) \right\rbrace.
\end{equ}
The non-intersection event will be defined by
\begin{equ}
    \nonint_{x} = \left\lbrace \forall 1 \leq i < j \leq r, ~\mathcal{C}_{(0,x_i)} \cap \mathcal{C}_{(0, x_j)} = \emptyset \right\rbrace.
\end{equ}
In the rest of this work, as $n, x, y$ will be fixed, we shall abbreviate $\con^n_{x,y}$ by $\con$ and $\nonint_x$ by $\nonint$. Moreover, we will also abbreviate $\mathcal{C}_{(0,x_i)}$ by $\mathcal{C}_i$ (recall that $\mathcal{C}_{(0,x_i)}$ denotes the cluster of the vertex $(0,x_i)$). 
\end{Def}

Our main result consists in the estimation of the probability that $\lbrace \con, \nonint \rbrace$ occurs in a subcritical random-cluster measure. 
\begin{Theorem}\label{theoreme estimation}
Let $q\geq 1$, and $0<p<p_c(q)$. Let $r\geq 1$ be a fixed integer. Then for any sequences $x_n, y_n$ of elements of $W$ satisfying $\Vert x_n \Vert, \Vert y_n \Vert = o(\sqrt{n})$, as $n \rightarrow \infty$,
\begin{equ} 
  \phi\left[\con, \nonint \right] \asymp V(x_n)V(y_n)n^{-\frac{r^2}{2}}\e^{-\tau rn},
\end{equ}
where $V$ is the function defined in Theorem~\ref{theoreme local limit srw} and the $\asymp$ is uniform in $x_n$ and $y_n$.
\end{Theorem}
\begin{Rem}
The function $V$ is not explicit. However, it is known that (see Theorem~\ref{theoreme local limit srw}) :
\begin{equ} \text{When }\min_{1\leq i \leq r-1} \lbrace |(x_n)_{i+1}- (x_{n})_i | \rbrace \goes{}{n\rightarrow\infty}{+\infty}, \text{ then } V(x_n)\sim\Delta(x_n) \text{ as } n\rightarrow \infty. \end{equ} 
Moreover, when $x,y$ are fixed elements of $W$, the statement simplifies as 
\begin{equ} 
  \phi\left[\con, \nonint \right] \asymp n^{-\frac{r^2}{2}}\e^{-\tau rn}.
\end{equ}
\end{Rem}
An interesting corollary, which is a direct consequence of Theorem~\ref{theoreme estimation} in the case $r=2$, can be obtained using the methods of~\cite{louidor}, where the same result is proved in the case of Bernoulli percolation (corresponding to $q=1$ in the random-cluster model). Let us define the \textit{truncated inverse correlation length} in the direction $\Vec{e_1}$ by 
\begin{equ}
    \tau^\mathsf{f}_p = \lim_{n \rightarrow \infty} -\frac{1}{n}\log\phi\left[ 0 \leftrightarrow (n, 0), \left| \mathcal{C}_0 \right| < \infty\right].
\end{equ}
It is well known that on $\Z^2$, whenever $p \neq p_c(q)$, one has that $\tau^\mathsf{f}_p > 0$. Moreover, it is clear that whenever $p < p_c(q)$, $\tau^\mathsf{f}_p = \tau_p$, where $\tau_p$ has been defined in~\eqref{Def tau}.
Then, Theorem~\ref{theoreme estimation} allows to compute the prefactor in the supercritical truncated correlation function. 
\begin{Cor}
Let $q\geq 1$ and $p\in (p_c, 1)$. Let $\phi$ be the unique infinite-volume random-cluster measure on $\Z^2$. Then, 
\begin{equ}
    \phi\left[ 0 \leftrightarrow (n,0), \left| \mathcal{C}_0 \right| < \infty \right] \asymp \frac{1}{n^2}\e^{-2\tau^{\mathsf{f}}_{p^*}n},
\end{equ}
where $p^*$ stands for the dual parameter of $p$ (see~\eqref{eqdualite} for the relation linking $p$ and $p^*$).
\end{Cor}
\begin{Rem}
In particular, we obtain the following equality, holding for any supercritical $p>p_c$
\begin{equ}
    \tau^\mathsf{f}_{p} = 2\tau^\mathsf{f}_{p^*} (= 2\tau_{p^*}).
\end{equ}
This is a very specific instance of duality, and such a relation is not expected to hold in higher dimensions. The result was already well known in the case of Bernoulli percolation, see for instance~\cite{correlationlengthonehalf} or~\cite[Theorem 11.24]{grimmett}.
\end{Rem}

Our second result consists in the study of the behaviour of the $r$ clusters created by conditioning on $\lbrace \con, \nonint \rbrace$. It will be formulated in terms of the \textit{envelopes} of a cluster.
\begin{Def}[Upper and lower envelopes of a cluster]\label{definition interfaces}
Let $\omega \in \con$. Then for any $0\leq k \leq n$ and $1 \leq i\leq r$ we define (see Figure~\ref{figure illustration interfaces})
\begin{equ}
    \Gamma^+_i(k) = \max\left\lbrace \ell \in \Z, (k,\ell) \in \mathcal{C}_i \right\rbrace \text{ and } \Gamma^-_i(k) = \min\left\lbrace \ell\in \Z, (k,\ell) \in \mathcal{C}_i \right\rbrace.
\end{equ}
It is clear that $\Gamma_i^\pm$ are well defined under $\con$, since the cluster of 0 is almost surely finite in the subcritical regime. We will see these quantities as functions from $[0,n]$ to $\R$ by considering the piecewise affine functions $\Gamma^\pm_i(t)$ that coincide with $\Gamma^\pm_i$ on the integers $t=k$. 
\end{Def}
Our second result is the following:
\begin{Theorem}\label{Theoreme main}
Fix $x,y \in W\cap\Z^r$ and $p\in(0,p_c(q)$. Then under the family of measures $\phi_{p,q}\left[ ~\cdot \vert \con,\nonint\right]$ (we recall that $\con, \nonint$ depend on $n$), there exists $\sigma > 0$ such that:
\begin{equ}\label{equation convergence main}
    \left( \frac{1}{\sqrt{n}}\left(\Gamma^+_1(nt), \dots, \Gamma^+_r(nt) \right)\right)_{0\leq t \leq 1} \goes{(d)}{n\rightarrow \infty}{ (\sigma\bw^{(r)}_t)_{0\leq t \leq 1}},
\end{equ}
where $\bw^{(r)}$ is the Brownian watermelon with $r$ bridges, and where the convergence holds in the space $\mathcal{C}\left([0,1],\R^r\right)$ endowed with the topology of uniform convergence. Moreover, almost surely, for all $1\leq i\leq r$,
\begin{equ}\label{equation shrink interface thm convergence}
\frac{1}{\sqrt{n}}\left\Vert \Gamma^+_i - \Gamma^-_i \right\Vert_{\infty} \goes{}{n \rightarrow \infty}{0}
\end{equ}
\end{Theorem}
\begin{Rem}
A consequence of~\eqref{equation shrink interface thm convergence} is that in the setting of Theorem~\ref{Theoreme main}, the clusters remain of width $o(\sqrt{n})$. In particular, the choice of the upper interfaces $\Gamma_i^+$ in~\eqref{equation convergence main} is arbitrary and can be replaced by any assignment of $\pm$ for the choice of interfaces to converge.
\end{Rem}
% Figure environment removed
\begin{Rem}
The result is stated for fixed $x,y \in W\cap \Z^r$. However, the careful reader may check that our method allows to treat the case where $x$ and $y$ depend on $n$. Indeed, as soon as $x_n, y_n$ are two sequences of $W\cap\Z^r$ satisfying
\begin{equ}
    \Vert x_n \Vert = o(\sqrt{n}) ~~~\text{and}~~~\Vert y_n \Vert = o(\sqrt{n}),
\end{equ}
our methods may apply and yield the same scaling limit.
\end{Rem}
\begin{Rem}\label{remarque interface}
For the reader familiar with statistical mechanics, it might seem strange that our result is formulated in terms of these envelopes and not in terms of the upper and lower interfaces running along the boundary of the clusters $\mathcal{C}_i$. However, it may be shown that the interfaces also converge to the paths of $\bw^{(r)}_t$ (as paths in $[0,1] \times \R^r$). We then chose to work with $\Gamma^\pm$ since we can use the space of continuous functions from $[0,n]$ to $\R^r$ for studying convergence questions, which is easier to treat than the space of continuous curves which would be needed when considering those interfaces. 
\end{Rem}



\subsection{Background on the random-cluster model}


We first recall some basic properties of the random-cluster model (once again we refer to~\cite{duminilcopin2017lectures} for a complete exposition). These properties are valid for any choice of parameters $p$ and $q$.
\medskip

\noindent{\bf Positive association.}
The space $\left\lbrace 0,1 \right\rbrace^{E(\Z^2)}$ can be equipped with a partial order: we say that $\om_1 \leq \om_2$ if for any $e \in E(\Z^2)$, $\om_1(e) \leq \om_2(e)$. An event $\mathcal{A}$ will be called \textit{increasing} if for any $\om_1 \leq \om_2$, $\om_1 \in \mathcal{A} \Rightarrow \om_2 \in \mathcal{A}$. The \textit{FKG inequality} then states that for any increasing events $\mathcal{A}, \mathcal{B}$, any graph $G$ and any boundary conditions $\eta$, 
\begin{equ}\tag{FKG}\label{equation FKG}
    \phi_{G,p,q}^\eta[\mathcal{A} \cap \mathcal{B}] \geq \phi^\eta_{G,p,q}[\mathcal{A}]\phi^\eta_{G,p,q}[\mathcal{B}].
\end{equ}
This property implies in particular that for any boundary conditions $\eta_1 \leq \eta_2$ (meaning that the partition $\eta_1$ is finer than $\eta_2$), for any increasing event $\mathcal{A}$,
\begin{equ}\tag{CBC}\label{Comparaison boundary conditions}
    \phi_{G,p,q}^{\eta_1}\left[\mathcal{A}\right] \leq \phi_{G,p,q}^{\eta_2}\left[\mathcal{A}\right].
\end{equ}
This property is called the \textit{comparison of boundary conditions} and may also be stated as ``$\phi^{\eta_1}$ is stochastically dominated by $\phi^{\eta_2}$".
\medskip

\noindent{\bf Duality.}
Let $(\Z^2)^* = (\frac{1}{2},\frac{1}{2}) + \Z^2$ and consider the lattice $(\Z^2)^*$ with edges between nearest neighbours. This lattice is called the \textit{dual lattice}. It has the property that for any $e \in E(\Z^2)$, there exists a unique edge $e^* \in E((\Z^2)^*)$ that crosses $e$. To a percolation configuration $\om \in \left\lbrace 0,1 \right\rbrace^{E(\Z^2)}$ we can associate a dual configuration $\om^*$ on the dual lattice by setting $\om^*(e^*) = 1 - \om(e)$. Then we remark that --- as soon as the parameters guarantee that there exists a unique Gibbs measure --- if $\om$ is sampled according to $\phi_{p,q}$, then $\om^*$ has the distribution of $\phi_{p^*,q^*}$, where 
\begin{equ}\label{eqdualite}
    q=q^* \text{ and } \frac{pp^*}{(1-p)(1-p^*)} = q.
\end{equ}
It has been proved by V. Beffara and H. Duminil-Copin in~\cite{duminilbeffara} that $p_c(q) = p^*_c(q)$, meaning that the parameter $p_c(q)$ is \textit{self-dual}. Also observe that if $\phi_{p,q}$ is subcritical, then $\phi_{p^*,q^*}$ is supercritical and vice-versa. 
\medskip

\noindent{\bf Spatial Markov property.} Let $G$ be a subgraph of $\Z^2$, and $G' \subset G$ a subgraph of $G$. Let $\xi$ be a percolation configuration on $\Z^2$. Observe that it induces a boundary condition on $G$ --- that we name $\eta(\xi)$ --- by identifying the vertices wired together by $\xi$ outside $G$, and a boundary condition on $G'$ - that we name $\eta'(\xi)$ by the same principle. Then,
\begin{equ}\tag{SMP}\label{equation smp}
    \phi_{G,p,q}^{\eta(\xi)}\left[ ~\cdot~\vert \om(e) = \xi(e), ~\forall e \notin G' \right] = \phi^{\eta'(\xi)}_{G',p,q}[\cdot].
\end{equ}


\noindent{\bf Finite energy property.} When $p \notin \lbrace 0,1 \rbrace$, there exists a constant $\eps>0$ depending only on $p$ and $q$ such that for any finite graph $G$, any finite $F \subset E(G)$, any boundary condition $\eta$, and any percolation configuration $\om_0$,
\begin{equ}
   (1-\eps)^{|F|} \geq  \phi_{G,p,q}^\eta\left[\om(e) = \om_0(e),~ \forall e \in F \right] \geq \eps^{|F|}.
\end{equ}
\medskip



\noindent{\bf Weak ratio mixing.}
In the subcritical regime, the random-cluster measure also enjoys the following \textit{weak ratio mixing property}. For two finite connected sets of edges $E_1$ and $E_2$, define their distance $d(E_1, E_2)$ as the Euclidean distance between the set of their respective endpoints. Then, for any graph $G$, any boundary condition $\eta$, any $q\geq 1$ and any $p<p_c(q)$, there exists a constant $c>0$ such that for any events $\mathcal{A}$ and $\mathcal{B}$ depending on edges of $E_1$ and $E_2$ respectively,
\begin{equ}\label{weak ratio mixing}\tag{MIX}
    \left| 1- \frac{\phi_{G,p,q}^\eta[\mathcal{A}\cap\mathcal{B}]}{\phi_{G,p,q}^\eta[\mathcal{A}]\phi_{G,p,q}^\eta[\mathcal{B}]} \right| < \e^{-cd(E_1,E_2)}.
\end{equ}
%We shall also use a more particular form of exponential mixing that has been proved for subcritical random-cluster models in~\cite[Prop 3.1]{ozrandomclustercampanino}. Namely for a set of edges $E$, define $\mathcal{N}(E)=\left\lbrace \text{All the edges of }E \text{ are closed} \right\rbrace$. Then there exists a constant $c>0$ such that for any finite sets of edges $A,B,C$,
%\begin{equ}\label{equation mixing}\tag{ExpMIX}
%    \frac{\phi_{p,q}\left[\mathcal{N}(A) \vert \mathcal{N}(B)\right]}{\phi_{p,q}\left[\mathcal{N}(A) \vert \mathcal{N}(C)\right]} \leq \exp \left( c\sum_{e\in A} e^{-\tau d(e, B \Delta C)} \right).
%\end{equ}
\medskip


\subsection{Outline of the proof}

The main idea of modern Ornstein--Zernike theory is to couple a subcritical percolation cluster conditioned on realizing a connection event $\left\lbrace x \leftrightarrow y \right\rbrace$ with a random walk started from $x$ and conditioned to reach $y$. Such a cluster is essentially a one-dimensional object. As the knowledge on conditioned random walks is very broad, in particular in terms of local limit theorems and invariance principles, such a coupling allows to derive properties of the original cluster. In our setting, we would like to couple a system of $r$ percolation clusters conditioned on $\con \cap \nonint$ with a system of $r$ random walks conditioned on a hitting event and on not intersecting each other. However, such a coupling is not immediately available in this setting and we have to rely on several comparison principles to show that the behaviours of these two types of systems are close. Once this task is accomplished, we use an invariance principle for a system of non-intersecting random walks to derive Theorem~\ref{Theoreme main}. 

 Let us be a bit more precise about the method. We first show that Ornstein--Zernike theory extends to $r$ non-intersecting clusters sampled according to $\phi^{\otimes r}$ (the product of $r$ random cluster measures on $\Z^2$) and thus interacting only through the conditioning. This allows us to derive an invariance principle for this product measure. 
 
 The next step is to transmit the results obtained for the product measure to the ``true" FK-percolation measure. As crucially observed in~\cite{ottvelenikwachtelioffe}, this can be done proving an \textit{a priori} (meaning independent of the above mentioned coupling) repulsion estimate: under the conditioned random-cluster measure, the clusters naturally move far from each other and never come near each other again. This input will then allow us to use the mixing property of subcritical FK-percolation to derive Theorem~\ref{Theoreme main}.

 
\subsection{Organization of the paper}


We first focus on Theorems~\ref{theoreme estimation} and~\ref{Theoreme main}. As explained previously, the proof consists of two independent tasks: comparing the behaviours of the percolation clusters and of a system of interacting random walks, and then obtaining the scaling limit and fine estimates on such a system of random walks. Our interest mainly being statistical mechanics, we postpone all the results about interacting random walks to Section~\ref{section marches}. In particular, it is independent of the other sections, and the reader interested in statistical mechanics only may skip this section. Section~\ref{section review oz} consists in a review of the rigorous Ornstein--Zernike results for one single subcritical cluster of FK-percolation. Section~\ref{section independent system} is devoted to the study of the scaling limit under the \textit{product measure} through a straightforward extension of the Ornstein--Zernike theory to this setting, as discussed before. Finally, Section~\ref{section RCM} is devoted to the proof of the entropic repulsion estimates, and thus of the announced result. 


\begin{Acknowledgements}We warmly thank Ioan Manolescu and Sébastien Ott for very useful and instructive discussions. We thank Romain Panis, Ulrik Thinggaard Hansen and Maran Mohanarangan for a careful reading of an early draft of this paper. The author was supported by the Swiss National Science Foundation grant n° 182237
\end{Acknowledgements}



\section{Ornstein--Zernike theory for a single subcritical FK-cluster}\label{section review oz}

In the remainder of the paper, we \textbf{fix} $q \geq 1$ and $0< p < p_c(q)$. Since these parameters will not change through the paper, we drop them from the notations and abbreviate $\phi_{p,q} := \phi$.

In this section, we review and discuss the main result of~\cite{ozrandomclustercampanino} --- the Ornstein--Zernike theorem. Schematically, this result can be described as follows. Under the conditioned measure $\phi \left[~\cdot\vert y \in \mathcal{C}_0\right]$ where $y$ is some vertex far away from 0, the cluster of $0$ has a very particular structure. Indeed, it macroscopically looks like the geodesic from 0 to $y$. Moreover, it exhibits typical Brownian bridge fluctuations around this geodesic, and is confined in a very small tube around this Brownian bridge. The result is precisely stated in Theorem~\ref{oz theorem}.

\begin{Def}[Directed random walk]\label{def directed measure}
A \textit{directed measure} on $\Z^2$ is a probability measure on $\N^* \times \Z$. If $X_1, \dots, X_n, \dots$ are independent and identically distributed random variables sampled according to a directed measure on $\Z^2$, then the distribution of process $S_n = X_1+\dots +X_n$ is called a \textit{directed random walk}. We shall call a possible realization of $(S_n)$ a \textit{directed walk} on $\Z^2$.
\end{Def}

In the remainder of the paper, we will often interpret trajectories of directed walks as real-valued functions defined on $\R^+$. Indeed, let $\nu$ be a directed measure on $\Z^2$ and $(S_n)_{n \geq 0}$ the associated directed random walk. Since $\nu(\N^* \times \Z)=1$, for any $t\geq 0$, the trajectory of $S$ almost surely intersects the vertical line $\lbrace t \rbrace \times \R$ once. Calling this point $S(t)$ provides us with a continuous and piecewise linear function: moreover this correspondence is one-to-one. We shall often use notations as $\lbrace S \in \mathcal{A} \rbrace$, where $\mathcal{A}$ is a subset of $\mathcal{C}(\R^+, \R)$. In that case, $S$ will have to be taken as the continuous function described above. Let $(S_n)_{n \geq 0}$ be a directed random walk on $\Z^2$. If $y \in \Z^2$, we introduce the event 
\begin{equ}
    \hit_y = \left\lbrace \exists n \geq 0, S_n = y \right\rbrace.
\end{equ}

\subsection{Diamond confinement and diamond decomposition}\label{subsubsection diamond decomposition}

We need a bit of vocabulary, in order to properly state the confinement property of a long subcritical cluster. Let $\delta > 0, x \in \Z^2$. Following~\cite{ozrandomclustercampanino}, we introduce the following subsets of $\Z^2$:

\begin{itemize}
    \item The $\delta$-forward cone of apex $x$ to be the set  $\mathcal{Y}^{\delta,+}_x = x + \left\lbrace (x_1, x_2) \in \Z^2, \delta x_1 \geq |x_2|    \right\rbrace.$ 
    \item The $\delta$-backward cone of apex $x$ to be the set $\mathcal{Y}^{\delta,-}_x = x + \left\lbrace (x_1,x_2) \in \Z^2, \delta x_1 \leq -|x_2|  \right\rbrace.$ 
    \item If $x,y \in \Z^2$ are such that $x_1<y_1$, the $\delta$-diamond of apexes $x, y$ is the intersection:
\begin{equ}    
    \mathcal{D}^\delta_{x,y} = \mathcal{Y}^{\delta,+}_x \cap \mathcal{Y}^{\delta,-}_y. 
\end{equ}
If $x=0$, we abbreviate the notation by $\mathcal{D}_y^\delta$.
\end{itemize}

Let $G$ be a finite subgraph of $\Z^2$ containing the vertex $0$ (we say that $G$ is a subgraph of $\Z^2$ \textit{rooted at 0}). We say that:
\begin{itemize}
    \item $(G,v)$ is $\delta$-\textit{left-confined} if there exists $x\in V(G)$ such that $G \subset \mathcal{Y}^{\delta,-}_x$.
    \item $(G,v)$ is $\delta$-\textit{right-confined} if there exists $x\in V(G)$ such that $G \subset \mathcal{Y}^{\delta,+}_x$.
    \item $G$ is $\delta$-\textit{diamond-confined} if there exist $y \in V(G)$ such that $G \subset \mathcal{D}^\delta_{y}$. In that case, we say that $\mathcal{D}^\delta_{y}$ is the diamond containing $G$. 
\end{itemize}

Observe that in the previous definitions, if the points $x,y$ do exist, they are necessarily unique. We denote the set of $\delta$-left-confined subgraphs of $\Z^2$ rooted at 0 (resp. $\delta$-right-confined subgraphs of $\Z^2$ rooted at 0, resp $\delta$-diamond-confined subgraphs of $\Z^2$ rooted at 0) by $\mathfrak{C}_L^\delta$ (resp $\mathfrak{C}_R^\delta$, resp $\mathfrak{D}^\delta$).


\begin{Def}
We now define the notion of \textit{displacement} along a left-confined, right-confined or diamond-confined subgraph of $\Z^2$. 
\begin{itemize}
    \item Let $G$ be a $\delta$-left-confined subgraph of $\Z^2$ rooted at 0. The \textit{displacement of $G$} is
    \begin{equ} 
    X^L(G) = x, 
    \end{equ}
    where $x$ is the unique vertex of $G$ such that $G \subset \mathcal{Y}_x^{-,\delta}$.
    \item Let $G$ be a $\delta$-right-confined subgraph of $\Z^2$ rooted at 0. The \textit{displacement of $G$} is 
    \begin{equ}
    X^R(G) = -x,
    \end{equ}
    where $x$ is the unique vertex of $G$ such that $G \subset \mathcal{Y}_x^{+,\delta}$.
    \item Let $G$ be a diamond-confined subgraph of $\Z^2$ rooted at 0. The \textit{displacement of $G$} is
    \begin{equ} X(G) = y, \end{equ}
    where $y$ is the only vertex of $G$ such that $G \subset \mathcal{D}^\delta_y$.
\end{itemize}
\end{Def}

In order to properly state what is a diamond decomposition of a cluster, we also need to introduce the operation of concatenation of two confined subgraphs rooted at 0. Let $G_1 \in \mathfrak{C}^\delta_L$ and $G_2 \in \mathfrak{D}^\delta$. The concatenation of $G_1$ and $G_2$, called $G_1\circ G_2$, is defined to be the subgraph
\begin{equ}
G_1\circ G_2 = G_1 \cup \left( X^L(G_1)+G_2 \right). 
\end{equ}
In the same manner, we can concatenate a $\delta$-diamond-confined rooted graph $G_1$ with a $\delta$-right-confined rooted graph by setting 
\begin{equ}
    G_1\circ G_2 = G_1\cup \left(X^R(G_2)+G_2\right).
\end{equ}
Finally observe that one can concatenate two $\delta$-diamond-confined rooted graphs by setting 
\begin{equ}
    G_1\circ G_2 = G_1\cup \left(X(G_2)+G_2\right).
\end{equ}

These definitions in hand, we can now define the \textit{diamond decompositions} of a subgraph of $\Z^2$. 


\begin{Def}[Diamond decomposition of a subgraph, skeleton of a subgraph]
Let $G$ be a finite subgraph of $\Z^2$ rooted at 0. Then, to any decomposition of the type $G=G^L\circ G_1 \circ \dots \circ G_\ell \circ G^R$ with $G^L \in \mathfrak{C}_L, G^R \in \mathfrak{C}_R, G_i \in \mathfrak{D}^\delta$ for all $i \in \lbrace 1, \dots, \ell \rbrace$, can be associated the concatenation of the confining $\delta$-left cone with all the associated $\delta$-diamonds and the confining $\delta$-right cone. We call such a subset a \textit{diamond decomposition} of $G$:

\begin{equ} 
\mathcal{D}(G) = \mathcal{Y}^{\delta, -}_{X^L(G^L)} \circ \mathcal{D}^{\delta}_{X(G_1)}  \circ \dots \circ \mathcal{D}^{\delta}_{X(G_\ell)} \circ \mathcal{Y}^{\delta,+}_{X^R(G^R)}. 
\end{equ}

Let us call $x_0 = 0, x_1 = X^L(G^L)$, $x_k = X(G_{k-1})$ for $2\leq k \leq \ell+1$, and $x_{\ell+2} = X^R(G^R)$. We then define, for $0\leq n \leq \ell+2$,
\begin{equ}
    \mathcal{S}(\mathcal{D})(G)_n = \sum_{k=0}^{n} x_k.
\end{equ}
The process $\mathcal{S}(\mathcal{D})(G)_n$ is called \textit{the skeleton} of the diamond decomposition $\mathcal{D}(G).$
\end{Def}

\begin{Rem}\label{remarque maximal skeleton}
    Observe that diamond decompositions of $G$ are not unique: as soon as there exists one of them with $\ell \geq 3$, merging inner diamonds allows one to create new (coarser) diamond decompositions of $G$. However, any finite subgraph rooted at 0 admits a unique \textit{maximal} diamond decomposition: we call it $\mathcal{D}^{\mathsf{max}}(G)$. The skeleton associated to this decomposition will by called $\mathcal{S}^{\mathsf{max}}(G)$ and referred to as the \textit{maximal} skeleton of $G$.
\end{Rem}

\begin{Rem}
    Our object of interest will be the skeleton of random diamond decompositions of subgraphs of $\Z^2$. Amongst the properties of the skeleton associated to a diamond decomposition of some rooted subgraph $G$, observe that the vertices of the skeleton of a diamond decomposition of $G$ are \textit{cone-points} of $G$, in the sense that for any $n \leq \ell+2$,
    \begin{equ}
        G \subset \mathcal{Y}_{\mathcal{S}(\mathcal{D})(G)_n}^{\delta, -} \cup \mathcal{Y}_{\mathcal{S}(\mathcal{D})(G)_n}^{\delta, +}.
    \end{equ}
    Furthermore, observe that the skeleton of a diamond decomposition of $G$ is always a finite directed walk, which motivates the terminology introduced in Definition~\ref{def directed measure}.
\end{Rem}

\begin{Rem}
The structure of the diamond decomposition is here given in the direction given by the first coordinate axis. However we see that adapting the definitions of the cones, the diamond decomposition can be defined for any direction $s \in \mathbb{S}^1$. The results of this work naturally adapt to this case, with this slight modification.
\end{Rem}



\subsection{Ornstein--Zernike theory for one subcritical cluster}

We are ready to state the main result of~\cite{ozrandomclustercampanino}, which we shall refer to as the \oz theorem. Set $\mathfrak{G}_0$ to be the set of connected subgraphs of $\Z^2$, rooted at 0.

% Figure environment removed

\begin{Theorem}[Ornstein--Zernike theorem,~\cite{ozrandomclustercampanino}]\label{oz theorem}
There exist two constants $C,c>0$ and a positive $\delta > 0$, such that the following holds. There exist two positive finite measures $\rho_L, \rho_R$ on $\mathfrak{C}_L^\delta$ and $\mathfrak{C}_R^\delta$ respectively, and a probability measure $\mathbf{P}$ on $\mathfrak{D}^\delta$ such that for any bounded function $f: \mathfrak{G}_0\rightarrow \R$, any $y  \in \mathcal{Y}_{0}^+$,

\begin{equ}\label{equation OZ 1 cluster}
  \Big\vert \e^{\tau x_1}\phi\left[f( \mathcal{C}_0) \1_{y \in \mathcal{C}_0} \right] - \sum_{\substack{\ell \geq 0 \\ G^L\in \mathfrak{C}_L^\delta \\ G^R \in \mathfrak{C}_L^\delta \\ G_1, \dots, G_\ell \in \mathfrak{D}^\delta }}\rho_L(G^L)\rho_R(G^R)\mathbf{P}(G_1) \cdots \mathbf{P}(G_\ell) f(G) \Big\vert \leq C\Vert f\Vert_{\infty}\e^{-c\norme{y}},\end{equ}
 where the sum runs over all $G^L \in \mathfrak{C}_L^\delta, G^R \in \mathfrak{C}_R^\delta, G_1, \dots, G_\ell \in \mathfrak{D}^\delta$ satisfying the relation
 \begin{equ}
     X^L(G^L) + X(G_1)+ \dots + X(G_\ell) + X^R(G^R) = y.
 \end{equ}
 We also have written $G = G^L \circ G_1 \circ \dots \circ G_\ell \circ G^R$ in the argument of $f$. Moreover, the measures $\rho_L, \rho_R, \mathbf{P}$ have exponential tails with respect to the length of the displacement: there exist $c', C' > 0$ such that
 \begin{equ}
     \max \left\lbrace \rho_L \left[\norme{X^L(G^L)} > t  \right], \rho_R \left[ \norme{X^R(G^R)} > t  \right], \mathbf{P}\left[ \norme{X(G)} > t \right]  \right\rbrace < C'\e^{-c't}.
 \end{equ} 
\end{Theorem}

In the remainder of the paper, we \textbf{fix} $\delta$ to be equal to the value given by Theorem~\ref{oz theorem}. In particular, we will not highlight the dependency anymore and we drop it from the notations. 


\begin{Rem}\label{Remarque definition loi nu}
For any $x \in \N^* \times \Z$, define the following three quantities:
\begin{itemize}
\item $\nu_L(x) = \sum_{G^L\in \mathfrak{C}_L, X^L(G^L)=x}\rho_L(G^L)$, 
\item $\nu_R(x) = \sum_{G^R\in \mathfrak{C}_R, X^R(G^R)=x}\rho_R(G^R),$ 
\item $\nu(x) = \sum_{G \in \mathfrak{D}, X(G) = x} \mathbf{P}\left(G\right)$.
\end{itemize}
Then, it is clear that $\nu$ is a directed probability measure on $\Z^2$, which has exponential tails. We define $\PP^\RW$ to be the directed random walk measure associated to $\nu$.
\end{Rem}

These definitions allow us to formulate a second version of Theorem~\ref{oz theorem} in terms of a coupling between a percolation cluster conditioned to contain a distant point and a directed random bridge.

\begin{Theorem}[Ornstein--Zernike theorem; coupling version]\label{theoreme couplage oz 1 cluster}
Let $y \in \mathcal{Y}^+_0$. There exists a probability space $(\Omega, \mathcal{F}, \Phi_{0 \rightarrow y})$ supporting a random variable $(\mathcal{C}_0, \mathcal{S})$ such that:
\begin{itemize}
    \item $\mathcal{C}_0$ has the distribution of the cluster of 0 under the measure $\phi\left[~\cdot\vert y \in \mathcal{C}_0   \right]$, ie if $C$ is a connected subgraph of $\Z^2$ containing 0,
    \begin{equ}
        \Phi_{0 \rightarrow y}\left[ \mathcal{C} = C \right] = \phi\left[\mathcal{C}_0 = C \vert y \in \mathcal{C}_0\right]
    \end{equ}
    \item $\mathcal{S}$ has the distribution of a directed random walk conditioned to hit $y$, \textit{ie} for any $\ell \geq 1$, any family $s_1, \dots, s_\ell$ of vertices of $\Z^2$,
    \begin{equ}
        \Phi_{0 \rightarrow y}\left[\mathcal{S}_1 =s_1, \dots, \mathcal{S}_\ell = s_\ell \right] \propto \nu_L(s_1)\nu_R(y-s_\ell)\prod_{k=2}^{\ell}\nu(s_k - s_{k-1}),
    \end{equ}
    where the symbol $\propto$ means that one has to normalise the latter quantity to get a proper probability measure,
    \item With probability at least $1-C\e^{-c\Vert y \Vert}$, for all $1 \leq k \leq \ell$, $\mathcal{S}_k \in \mathcal{C}_0$ and $\mathcal{S}_k$ is a renewal of $\mathcal{C}_0$. Furthermore, for any $1\leq k\leq \ell-1$, the portion of $\mathcal{C}_0$ lying between $\mathcal{S}_k$ and $\mathcal{S}_{k+1}$ is a $\delta$-diamond-confined subgraph of $\Z^2$.
\end{itemize}
\end{Theorem}
\begin{proof}
Fix some $y \in \mathcal{Y}_0^+$. We define a probability distribution on the space
\begin{equ}
    \mathfrak{C}_L \times \bigcup_{l=1}^{+\infty} \left( \prod_{k=1}^\ell \mathfrak{D}\right) \times \mathfrak{C}_R
\end{equ} 
by the formula:
\begin{multline*}
    \phi^{\mathsf{Dec}}_y\left[(G^L, G_1, \dots, G_\ell, G_R)\right] \propto  \1_{X_L(G^L)+X(G_1)+\dots+X(G_\ell)+X_R(G^R) = y}\\ \times\sum_{\substack{\ell \geq 0 \\ G^L\in \mathfrak{C}_L^\delta \\ G^R \in \mathfrak{C}_L^\delta \\ G_1, \dots, G_\ell \in \mathfrak{D}^\delta }}\rho_L(G^L)\rho_R(G^R)\mathbf{P}(G_1) \cdots \mathbf{P}(G_\ell).
\end{multline*}
Then, for any percolation event $\mathcal{A}$, we form the ratio of~\eqref{equation OZ 1 cluster} with $f=\1_\mathcal{A}$ and~\eqref{equation OZ 1 cluster} with $f=1$. We immediately get that the total variation distance between $\phi\left[~\cdot\vert y \in \mathcal{C}_0\right]$ and the pushforward of $\phi^\mathsf{Dec}_y\left[~\cdot~\right]$ by the concatenation operation is bounded by $C\e^{-c\norme{x}}$. It is classical that this yields the existence of a maximal coupling between those two measures, ie that one can construct a probability space $(\Omega, \mathcal{F}, \Phi_{0 \rightarrow y})$ supporting $(\mathcal{C}^1_0, \mathcal{C}^2_0)$ such that 
\begin{itemize}
    \item The distribution of $\mathcal{C}^1_0$ is the distribution of the cluster of 0 under $\phi\left[~\cdot\vert y \in \mathcal{C}_0\right]$,
    \item The distribution of $\mathcal{C}^2_0$ is the distribution of the concatenation $G^L\circ G^1 \circ \dots \circ G^\ell \circ G^R$ where $(G^L, G^1, \dots, G^\ell, G^R)$ are sampled according to $\phi^\mathsf{Dec}_y$,
    \item $\Phi_{0 \rightarrow y}(\mathcal{C}^1_0 \neq \mathcal{C}^2_0) \leq C\e^{-c\norme{y}}$.
\end{itemize}
Now consider the random variable $\mathcal{S}$ formed from $(G_L, G^1, \dots, G^\ell, G_R)$ by the following formula:
\begin{equ}
   \mathcal{S}_1 = X_L(G_L) ~~\text{and}~~ \mathcal{S}_k = \mathcal{S}_{k-1}+X(G^{k-1}) ~~\text{for}~~ 2\leq k \leq \ell.
\end{equ}
Then it is immediate that 
\begin{equ}
    \Phi_{(0,y)}\left[ \mathcal{S}_1 = s_1, \dots, \mathcal{S}_\ell = s_\ell \right] \propto \nu_L(s_1)\nu_R(y-s_\ell)\prod_{k=2}^{\ell}\nu(s_k-s_{k-1}).
\end{equ}
Moreover by definition, the $\mathcal{S}_k$'s are renewals of $\mathcal{C}^2_0$ and the portions of $\mathcal{C}^2_0$ lying between two consecutive $\mathcal{S}_k$'s are $\delta$-diamond-confined. Thus, $(\Omega, \mathcal{F}, \Phi_{0 \rightarrow y})$ equipped with the random variable $(\mathcal{C}^1_0, \mathcal{S})$ provides us with the desired coupling.  
\end{proof}
For now, we shall only work in the extended probability space $(\Omega, \mathcal{F}, \Phi_{0 \rightarrow y})$. Thus, each percolation configuration conditioned to contain the distant point $y$ will be sampled together with a directed random walk bridge: we call this directed random bridge \textbf{the skeleton of $\mathcal{C}_0$}; the associated diamond decomposition will be called \textbf{the diamond decomposition of $\mathcal{C}_0$}. Observe that this enlarged probability space carries extra randomness than the space supporting $\phi$: indeed, to a given percolation cluster can be associated several skeletons that are randomly chosen by the measure $\Phi_{0 \rightarrow y}$(see Figure~\ref{Figure ornstein zernike}). We adopt the terminology of~\cite{ozrandomclustercampanino} by calling the points of $\mathcal{S}$ \emph{renewals} of the cluster. Observe that due to the latter discussion, all the renewals of $\mathcal{C}$ are cone-points, but the converse is not necessarily true.

%\begin{Rem}
%Theorem~\ref{theoreme couplage oz 1 cluster} is very useful when working with \textit{conditioned} measures. It is however less general than Theorem~\ref{oz theorem} that allows us to work with unconditioned measures. The price to pay in that case is that both summands of~\eqref{equation OZ 1 cluster} are not probability measures. Let us see what happens to the skeleton of $\mathcal{C}_0$ when not conditioning of $\lbrace x \in \mathcal{C}_0 \rbrace$. Let $f$ be any bounded function on the set of directed walks Keeping the notations of Theorem~\ref{oz theorem}, one has that for any bounded function of the possible realizations of the skeleton of $\mathcal{C}_0$, 
%\begin{multline}\label{equation couplage squelette rw} \Big\vert \e^{\tau x_1} \Phi_{(0,y)}\left[ f\left(\mathcal{S}\right)\right] - \sum_{\substack{\ell \geq 0 \\ x_L + x_1 + \cdots + x_\ell + x_R = x}}f(S)\nu_L(x_L)\nu_R(x_R)\prod_{k=1}^\ell \nu(x_k) \Big\vert \\ < C\Vert f\Vert_{\infty}\e^{-c\norme{x}},
%\end{multline}
%where we set $S$ to be the directed walk given by $S_0 = 0$, $S_1 = x_L$, $S_k = x_L + x_1 + \dots + x_{k-1}$ when $k \leq \ell+1$ and $S_{\ell+2} = x_L + x_1 + \cdots + x_\ell + x_R$. It is very tempting to interpret the second term of the left-hand side of~\eqref{equation couplage squelette rw} as the expectation of $f$ under a directed random walk measure started from 0 and conditioned on hitting the vertex $x$. However, it is not really the case, because $\nu_L$ and $\nu_R$ are not probability measures. Define $\PP^{\mathsf{RW}}$ as the measure of the directed walk with increments given by $\nu$. Then,~\eqref{equation couplage squelette rw} reads as 
%\begin{multline}\label{equation couplage squelette rw}
%    \Big\vert \e^{\tau x_1} \Phi_{(0,y)}\left[ f\left(\mathcal{S}\right) \1_{x \in \mathcal{C}_0}\right] - \sum_{x_L, x_R \in \Z^2} \nu_L(x_L)\nu_R(x_R)\E_{x_L}^{\mathsf{RW}}\left[f(x_L \circ S \circ x_R)\1_{S \in \hit_{x-x_R}}\right]\Big\vert \\ < C\Vert f\Vert_{\infty}\e^{-c\norme{x}},
%\end{multline}
%where $\E_{x_L}^{\mathsf{RW}}$ is the expectation with respect to the measure $\PP^\mathsf{RW}$ started at $x_L$. In the precedent formula, we have denoted by $x_L \circ S \circ x_R$ the process which is made of the concatenation of a linear function from 0 to $x_L$, then by the directed random walk $S$ sampled according to $\PP^\RW\left[~\cdot\vert\hit_{x-x_R}\right]$ and finally by the linear function from $x_R$ to $x$.    
%\end{Rem}
In the remainder of this paper, we introduce $\PP^\RW$, the measure of the directed random walk with independent increments sampled according to $\nu$ and started from 0.
\begin{Rem}
We are often going to be interested in observables of the \textit{skeleton} of a cluster sampled according to $\Phi_{0 \rightarrow y}$. In that case, Theorem~\ref{theoreme couplage oz 1 cluster} reads as follows: let $f$ be a bounded function of the set of directed random walks. Then,
    \begin{multline}\label{equation couplage squelette rw}
        \Bigg| \Phi_{0 \rightarrow y}\left[f(\mathcal{S})\right] - \sum_{x_L, x_R}\nu_L( x_L)\nu_R(y-x_R)\E^\RW\left[f( x_L \circ S \circ x_R ) \vert S \in \hit_{x_R-x_L} \right] \Bigg| \\ \leq C\Vert f \Vert_\infty \e^{-c\Vert y \Vert_2}.
    \end{multline}
In the latter writing, the notation $x_R \circ S \circ x_L$ stands for the directed walk obtained by the concatenation of $x_L$, the trajectory of $S$, and $x_R$. In the writing $\E^\RW\left[f( x_L \circ S \circ x_R ) \vert S \in \hit_{x_R-x_L} \right]$, only $S$ is random - and has law $\PP^\RW\left[~\cdot\vert S \in \hit_{x_R-x_L}\right]$.

Finally, the unconditionnal version of~\eqref{equation couplage squelette rw} is the following. 
\begin{equ}\label{Equation unconditional skeleton coupling 1 cluster}
\Bigg\vert \e^{\tau y_1}\Phi_{0 \rightarrow y}\left[ f(\mathcal{S}) \right] - \sum_{x_L, x_R}\nu_L(x_L)\nu_R(y-x_R)\E^\RW\left[f(S)\1_{\hit_{x_R-x_L}}\right]  \Bigg\vert \leq C\Vert f\Vert_\infty \e^{-c\Vert y \Vert_2}.
\end{equ}
\end{Rem}

 The following lemma states when looking at certain families of observables of the trajectories of directed walks, it is sufficient to study the measure $\PP^\RW$ started from 0 rather than the intricate second summand of the left-hand side of~\eqref{equation couplage squelette rw}

\begin{Lemma}\label{lemme skorokhod topology}
    There exists a constant $\varsigma > 0$ such that for any $y \in \mathcal{Y}^+_0$, any two sequences $a_n, b_n$ of positive numbers going to infinity, any bounded function $f : \mathcal{C}([0,x_1], \R) \rightarrow \R$ continuous with respect to the Skorokhod topology (see~\cite{billingsley} for the definition and properties of this topology), 
    \begin{multline*}
        \Big\vert \e^{-\tau b_n y_1}\Phi_{0 \rightarrow b_n y} \left[ f\left(a_n^{-1}\mathcal{S}(\lfloor b_n t\rfloor) \right)_{t \geq 0}\right] - \varsigma\E^\RW\left[f\left(a_n^{-1} S(\lfloor b_n t \rfloor )\right)_{t \geq 0}\1_{S \in \hit_{b_n y}} \right] \Big\vert \\ \goes{}{n \rightarrow \infty}{0}.
    \end{multline*}
We have used the interpretation of directed walks as real-valued functions explained above.
\end{Lemma}
\begin{proof}
    Set $\varsigma = \nu_L(\Z^2)\nu_R(\Z^2)$. By~\eqref{Equation unconditional skeleton coupling 1 cluster}, it sufficient to prove that 
    \begin{multline*}
    \Big\vert \sum_{x_L, x_R \in \Z^2} \nu_L(x_L)\nu_R(x_R)\E^{\mathsf{RW}}_{x_L}\left[f\left(a_n^{-1}(x_L\circ S\circ x_R)(\lfloor b_n t  \rfloor )\right)_{t \geq 0}\1_{S \in \hit_{x_L-x_R}}\right] \\ - \varsigma\E^\RW\left[f\left(a_n^{-1} S(\lfloor b_n t \rfloor )\right)_{t \geq 0}\1_{S \in \hit_{b_n y}} \right] \Big\vert \goes{}{n \rightarrow \infty}{0}.
    \end{multline*}
The right-hand side can be dominated by
\begin{multline*}
    \sum_{x_L, x_R \in \Z^2} \nu_L(x_L)\nu_R(x_R) \E^\RW\Big[ \big\vert f\left(a_n^{-1}(x_L\circ (S+x_L)\circ x_R)(\lfloor b_n t  \rfloor )\right)_{t \geq 0}\1_{S \in \hit_{x_R - x_L}} \\ - f\left(a_n^{-1} S(\lfloor b_n t \rfloor )\right)_{t \geq 0}\1_{S \in \hit_{b_n y}} \Big].
\end{multline*}
Now we take advantage of the exponential tails of $\nu_L$ and $\nu_R$ by splitting the sum in two parts, the first one running over $x_L,x_R \in B(0, \log(\min(a_n, b_n)))$, and the remaining one. Thanks to the exponential tails of $\nu_L$ and $\nu_R$, the remaining one can be bounded by $2\Vert f \Vert_\infty \min(a_n,b_n)^{-c}$, which indeed goes to 0. The first part is shown to go to 0 by noticing that when $x_L,x_R \in B(0, \log(\min(a_n, b_n)))$, the Skorokhod distance between the two considered functions goes to 0. We conclude by continuity of $f$ and dominated convergence.
\end{proof}
% We shall mainly use conditional versions of equation~\eqref{equation couplage squelette rw}, that we now state as a general principle.

% \begin{Cor}\label{corollaire oz conditionnel}
% Let $\mathcal{A}, \mathcal{B}$ be two measurable events of $\mathcal{C}(\R^+, \R)$. We  assume that $\mathcal{A}$ satisfies the following assumption: when $y$ is sufficiently large,
% \begin{equ}
% \sum_{x_L,x_R \in \Z^2} \nu_L(x_L)\nu_R(x_R)\PP^\RW\left[(x_L\circ S\circ x_R) \in \mathcal{A}, S \in \hit_{x_L-x_R}\right] > \e^{-o(\Vert y\Vert)}.
% \end{equ}
% Then, there exist two constant depending on the event $\mathcal{A}$, but not on $y$, such that when $y$ is large enough,
% \begin{multline}\label{equation assumption decay of the considered event corollaire oz conditionnel}
%  \Big| \Phi_{0 \rightarrow y}\Big[ \mathcal{S} \in \mathcal{B} \vert \mathcal{S} \in \mathcal{A}, y \in \mathcal{C}_0 \Big] - \\ \sum_{x_L,x_R \in \Z^2} \nu_L(x_L)\nu_R(x_R) \PP^{\RW}\Big[x_L \circ S \circ x_R \in \mathcal{B} \vert x_L\circ S \circ x_R  \in \mathcal{A}, S \in \hit_{x_R-x_L} \Big]   \Big| \\ \leq C\e^{-c\norme{y}}. 
%  \end{multline}
% \end{Cor}

% Note that because of Lemma~\ref{lemme skorokhod topology}, if $\mathcal{A}, \mathcal{B}$ are measurable with respect to the \textit{scaled} skeleton of the cluster $\mathcal{C}_0$ such as in Theorem~\ref{Theoreme main}, then the left-hand side takes the simpler form:
% \begin{equ}
% \left| \Phi_{0 \rightarrow y}\left[ \mathcal{S} \in \mathcal{B} \vert \mathcal{S} \in \mathcal{A}, y \in \mathcal{C}_0   \right] - \PP^\RW\left[S \in \mathcal{B} \vert S\in \mathcal{A}\cap\hit_{y}\right]\right|.
% \end{equ}

% \begin{proof}[Proof of Corollary~\ref{corollaire oz conditionnel}]
% The proof is straightforward using~\eqref{equation couplage squelette rw} twice, once with $f=\1_{\mathcal{A}}$ and a second time with $f = \1_{\mathcal{A}\cap\mathcal{B}}$. Forming the ratio of these quantities and using~\eqref{equation assumption decay of the considered event corollaire oz conditionnel} to see that the denominator decays way slower than the numerator, we immediately get the result.
% \end{proof}
% \begin{Rem}
% It is classical - thanks to local limit theorems for directed random walks (see~\cite{ozrandomclustercampanino}) - that $\PP^\RW\left[S\in\hit_y\right]$ decays as fast as a polynomial in $\norme{y}$ when $y \in \mathcal{Y}^{+,\delta}$. Using the exponential tails for $\rho_L,\rho_R$, we obtain by a reasoning similar to the one used in the proof of Lemma~\ref{lemme skorokhod topology}, that 
% \begin{equ}
% \sum_{x_L,x_R \in \Z^2} \nu_L(x_L)\nu_R(x_R)\PP^\RW\left[ S \in \hit_{x_L-x_R}\right] 
% \end{equ}
% also decays as fast as a polynomial in $\Vert y\Vert$. In particular the precedent corollary is non-trivial. 
% \end{Rem}

We state two byproducts of Theorem~\ref{oz theorem}:

\begin{Cor}\label{corollaire nombre linéaire de renewals}
There exists three constants $c, C, K > 0$ such that for any $y\in \mathcal{Y}^+_0$ with $\norme{y}$ sufficiently large,
\begin{equ}
    \phi \left[ \mathcal{C}_0 \text{ has less than } K\norme{y} \text{ renewal points} \vert y \in \mathcal{C}_0 \right] < C\e^{-c\norme{y}}.
\end{equ}
\end{Cor}

%\begin{proof}
%This is a basic large deviations estimate. Indeed, let us call $\mu = \nu\left[\left\langle X, \vec{e_1} \right\rangle \right]>0$ (we recall that $\nu$ has been introduced in Remark~\ref{Remarque definition loi nu}). Set $K < \delta \mu^{-1}$. Then,
%\begin{eqnarray*}
%\sum_{r \leq K\norme{x}} \PP\left[ S_r = x \right] &\leq& \PP \left[ \left\langle S_r, \vec{e_1} \right\rangle ] = x_1 \right] \\
%&\leq& \sum_{r \leq K\norme{x}} \exp\left(c(K-\delta\mu^{-1})\norme{x}\right)\\
%&=& O\left(\exp(c(K-\delta\mu^{-1})\norme{x}\right),
%\end{eqnarray*}
%where the value of the constant $c>0$ has changed from line to line. %We used the fact that $x \in \mathcal{Y}^+_0$, so that $\delta\norme{x} \leq x_1 \leq \norme{x}$.

%But since the event  

%\begin{equ}
%\lbrace \mathcal{C}_0 \text{ has less than } K\norme{x} \text{ renewal points}\rbrace
%\end{equ}

%is measurable with respect to $\mathcal{S})$, we use Corollary~\ref{corollaire oz conditionnel} with $\mathcal{A} = \hit_x$ to obtain the result. 
%\end{proof}

\begin{Cor}\label{corollaire volume diamants}
There exists a constant $K > 0$, such that for any $y \in \mathcal{Y}^+_0$, 

\begin{equ}
    \Phi_{0 \rightarrow y}\Big[ \max_{\substack{\mathcal{D} \subset \mathcal{D}(\mathcal{C}_0) \\ \mathcal{D}\text{ diamond}}} \Vol\left(\mathcal{D}\right) > K (\log y_1)^2 \Big]  < C\norme{y}^{-c\log \norme{y}},
\end{equ}
where $\Vol$ denotes the Euclidean volume, and where the $\max$ is taken over all the diamonds appearing in the diamond decomposition of the cluster of 0 under the measure $\Phi_{(0,y)}$. 
\end{Cor}
Note that the latter bound decays faster than the inverse of any polynomial in $\norme{y}$.
%\begin{proof}
%First, note that the probability of the event 
%\begin{equ} 
%\big\lbrace \max_{\substack{\mathcal{D} \subset \mathcal{D}(\mathcal{C}_0) \\ \mathcal{D}\text{ diamond}}} \Vol\left(\mathcal{D}\right) > K (\log x_1)^2 \big\rbrace
%\end{equ}
%is in fact controlled by the probability of an event measurable with respect to $\mathcal{S})$. Indeed the volume of a diamond $\mathcal{D}$ of opening $\delta > 0$ is at most $\frac{\delta}{2}\norme{X(\mathcal{D})}^2$ (using $\sin\delta < \delta$).  Looking at Corollary~\ref{corollaire oz conditionnel}, we have to control:
%\begin{equ}
%\PP^{\RW} \left[ \max \left\lbrace \norme{X_1}, \dots, \norme{X_k} \right\rbrace > c\log x_1 \vert S \in \hit_x    \right].
%\end{equ}
%Thanks to Corollary~\ref{corollaire nombre linéaire de renewals}, we know that up to an event of exponentially small probability, the random walk has a linear in $\norme{x}$ number of renewal points. The bound then follows on classical estimates on the max of i.i.d. random variables having exponential tails and the corollary~\ref{corollaire oz conditionnel}. 
%\end{proof}
\subsection{Ornstein--Zernike in a strip with boundary conditions}

We import a few facts about Ornstein--Zernike theory that will be useful later on in our analysis. They deal with the uniformity of the Ornstein--Zernike formula in the boundary conditions and are directly imported from~\cite{ozisingcampanino, greenbergioffe}. For $y= (y_1, y_2) \in \mathcal{Y}_{0}^+$, let us call $\mathsf{Strip}_y$ the strip $\mathsf{Strip}_y= \left[0, y_1 \right] \times \Z$. In the following proposition, the probability measure $\mathbf{P}$ is the same object as in Theorem~\ref{oz theorem}.

\begin{Prop}[Uniform OZ formula in a strip]\label{prop oz boundary conditions} Let $y = (y_1,y_2)\in \mathcal{Y}_{0}^+$. 
Let $0 \ni C_{\EXT}^L$ be a finite connected subset of edges of $\mathcal{Y}^{-}_0$ and $y \ni C_{\EXT}^R$ be a finite connected subset of edges of $\mathcal{Y}_{y}^+$. Then there exist two positive and bounded measures $\rho_L^{\EXT}, \rho_R^\EXT$ on $\mathcal{C}_L$ and $\mathfrak{C}_R$ respectively such that for any bounded function $f : \mathfrak{G}_0\rightarrow \R$,
\begin{multline}\label{equation OZ boundary conditions}
  \bigg\vert \e^{\tau y_1}\phi\left[f( \mathcal{C}_0) \1\left(\lbrace\mathcal{C}_0 \cap \mathsf{Strip}_y^c = C^L_\EXT \sqcup C^R_\EXT \rbrace \right) \right] - \\ \sum_{\substack{\ell \geq 0 \\ G^L\in \mathfrak{C}_L \\ G^R \in \mathfrak{C}_L\\ G_1, \dots, G_\ell \in \mathfrak{D} }}\rho^\EXT_L(G^L)\rho^\EXT_R(G^R)\mathbf{P}(G_1) \cdots \mathbf{P}(G_\ell) f(G) \bigg\vert  \leq C\Vert f\Vert_{\infty}\e^{-c\norme{y}},
\end{multline}
 where the sum holds over all $G^L \in \mathfrak{C}_L, G^R \in \mathfrak{C}_R, G_1, \dots, G_k \in \mathfrak{D}$ satisfying the relation
 \begin{equ}
     X^L_0(G^L) + X(G_1)+ \dots + X(G_k) + X^R_x(G^R) = y,
 \end{equ}
 and where we have written $G = G^L \circ G_1 \circ \cdots \circ G_k \circ G^R $.
 Moreover, the measures $\rho^\EXT_L$ and $\rho^\EXT_R$ have exponential tails, uniformly in the sets $C^L_\EXT, C^R_\EXT$ satisfying the above-stated assumptions: indeed, there exist $c', C' > 0$ such that for any $t>0$,
 \begin{equ}
     \sup_{C^L_\EXT, C^R_\EXT} \max \left\lbrace \rho^\EXT_L(X(G^L) > t), \rho^\EXT_R(X(G^R) > t) \right\rbrace < C'\e^{-c't}.
 \end{equ}
\end{Prop}
Observe that in the latter formula, the event $\lbrace\mathcal{C}_0 \cap \mathsf{Strip}_y^c = C^L_\EXT \sqcup C^R_\EXT \rbrace$ implies that $y \in \mathcal{C}_0$.
\begin{Rem}
As done previously, for any $y \in \mathcal{Y}_{0}^+$, we shall call 
\begin{equ}
    \nu^\EXT_L(x) = \sum_{G^L \in \mathfrak{C}^\delta_L, X^L_0(G^L) = x}\rho^\EXT_L(G^L) \text{  and  } \nu^\EXT_R(x) = \sum_{G^R \in \mathfrak{C}^\delta_R, X^R_x(G^R) = x}\rho^\EXT_R(G^R)
\end{equ}
\end{Rem}
 We simply sketch the proof of the proposition, since it is a simple byproduct of the analysis of~\cite{ozrandomclustercampanino}

\begin{proof}[Proof of Proposition~\ref{prop oz boundary conditions}]
Apply the Ornstein--Zernike formula~\eqref{equation OZ 1 cluster} to the function $g(\mathcal{C}_0) = f(\mathcal{C}_0)\1\left\lbrace \mathcal{C}_0\cap \mathsf{Strip}_y^c = C^L_\EXT \sqcup C^R_\EXT \right\rbrace$. Thus, one has that $\rho_L^\EXT$ (resp. $\rho^R_\EXT$) is the restriction of $\rho_L$ (resp. $\rho_R$) to pieces of clusters compatible with $C^L_\EXT$ (resp. $C^R_\EXT$). The announced exponential decay is a byproduct of the exponential tails of $\rho_L$ and $\rho_R$.
\end{proof}

A non-trivial consequence of the latter proposition is the following estimate, appearing in~\cite[Equation (2.19)]{greenbergioffe}.  

\begin{Prop}[Ornstein--Zernike decay uniform in the boundary conditions]\label{Proposition oz uniforme boundary conditions}
There exists $\chi>0$ such that for any sets $C_\EXT^L, C_\EXT^R$ satisfying the assumptions of Proposition~\ref{prop oz boundary conditions},
\begin{equ}\label{equation oz uniform BC}
    \frac{1}{\chi} \frac{\e^{-\tau n}}{\sqrt{n}} \leq \phi\left[ 0 \overset{\mathsf{Strip}_n}{\longleftrightarrow} (n,0) \Bigg\vert 
        ~\begin{aligned} &\mathcal{C}_0 \cap \left(\Z^- \times \Z\right) = C^L_\EXT ~~\text{and} \\ &\mathcal{C}_{(n,0)} \cap \left( [n, +\infty) \times \Z\right) = C^R_\EXT \end{aligned}   \right] \leq \chi \frac{\e^{-\tau n}}{\sqrt{n}}.
\end{equ}
\end{Prop}


\section{Scaling limit for the product measure}\label{section independent system}

For our purposes, we need to develop an analog of the Ornstein--Zernike theory for $r$ non-intersecting clusters of FK-percolation. However, there is a supplementary difficulty, namely that these non-intersecting clusters are not independent, beyond the obvious interaction introduced by the conditioning. If we consider \textit{a product measure}, we can readily extend the Ornstein--Zernike theorem to $r$ clusters sampled independently according to $\phi$. This is the goal of the present section. Even though it might seem a bit strange to consider the conditioned product measure $\phi^{\otimes r}$ instead of the real conditioned random-cluster measure, we are going to see in Section~\ref{section RCM} that because of the conditioning, these two measures behave similarly "in the bulk". This is a consequence of the spatial mixing property of the subcritical random-cluster measure combined with an a priori repulsion estimate. 

In what follows, $\phi^{\otimes r}$ will always denote the measure consisting in the product of $r$ random-cluster measures $\phi$. Moreover, $\mathcal{C}_i$ will denote $\mathcal{C}_i(\omega_i)$. In particular, if $\mathcal{A}$ is an event measurable with respect to $\left(\mathcal{C}_1, \dots, \mathcal{C}_r\right)$, we have:
\begin{equ}
\phi^{\otimes r}\left[\mathcal{A}\right] = \PP \left[\left( \mathcal{C}_1(\om_1), \dots, \mathcal{C}_r(\om_r) \right) \in \mathcal{A} \right],     
\end{equ}
 where $\om_1, \dots, \om_r$ are \textbf{independent} percolation configurations sampled according to $\phi$.

The main goal of this section is the following proposition:

\begin{Prop}\label{Proposition convergence mesure produit}
Recall the definition of the envelopes of a cluster $\Gamma^{\pm}(\mathcal{C})$ introduced in Def~\ref{definition interfaces}, and their natural parametrization. Then there exists $\sigma > 0$ such that:
\begin{equ}
 \frac{1}{\sqrt{n}} \left( \Gamma^+(\mathcal{C}_1)(nt), \dots, \Gamma^+(\mathcal{C}_r)(nt) \right)_{0 \leq t \leq 1} \goes{(d) }{n \rightarrow \infty}{\left(\sigma\bw^{(r)}_t \right)_{0 \leq t \leq 1}},
 \end{equ}
where the percolation configuration is sampled under the measure $\phi^{\otimes r} \left[ ~\cdot ~ \big\vert \con,\nonint \right]$, and the convergence occurs in the space $\mathcal{C}^r([0,1])$ equipped with the topology of uniform convergence. Moreover, almost surely, for any $1\leq i\leq r$,
\begin{equ}
    \frac{1}{\sqrt{n}}\Vert \Gamma_i^+ - \Gamma_i^- \Vert_{\infty} \goes{}{n\rightarrow +\infty}{0}.
\end{equ}
\end{Prop}


The strategy of the proof is the following: we start to state an analog of the Ornstein--Zernike theorem in the case of a product measure in Section~\ref{subDefinition of the product measure and multidimensional version of Ornstein--Zernike theorem }, and use this coupling to compare the skeletons of a system of $r$ non-intersecting clusters with a system of $r$ non-intersecting directed random walks. However, there is a small difficulty while implementing this program: indeed, conditioning on the event that the \textit{clusters} do not intersect is not the same as conditioning on the event that the \textit{skeletons} of the clusters do not intersect. Moreover, while the latter event is very well described in terms of the Ornstein--Zernike coupling, it is not a priori clear how the former acts on the coupled walks. For that reason, we shall show first that under the conditioning on $\nonint \cap \con$, the clusters very soon get far from each other (this is the goal of Subsection~\ref{sub edge repulsion}), and thanks to this input we will be able to prove that "in the bulk" of the system, the conditioning of non-intersection for the clusters or for the skeletons of the clusters yield the same scaling limit. Thus we shall apply the invariance principle of Theorem~\ref{theoreme invariance principle for drw} to conclude in Subsection~\ref{sub convergence towards the brownian watermelon}.

\subsection{Definition of the product measure and multidimensional version of Ornstein--Zernike theorem}\label{subDefinition of the product measure and multidimensional version of Ornstein--Zernike theorem }

We first state a $r$-dimensional version of Theorem~\ref{oz theorem} (the Ornstein--Zernike formula). Indeed, if $\om^1, \dots, \om^r$ are $r$ independent configurations of law $\phi$ and $f : \mathfrak{G}^r \rightarrow \R$ is a bounded function, then it is an easy consequence of Theorem~\ref{oz theorem} that when $n \in \N$ is sufficiently large,
\begin{multline}\label{multidim oz}
\Bigg\vert \e^{\tau r n}\phi^{\otimes r}\left[ f(\mathcal{C}_1, \dots, \mathcal{C}_r) \1_{(\om^1, \dots, \om^r) \in \con} \right]  -  \sum_{\substack{G_1^L\in \mathfrak{C}_L \\ \dots \\ G_r^L \in \mathfrak{C}_L}} \sum_{\substack{G_1^R \in \mathfrak{C}_R \\ \dots \\ G_r^R \in \mathfrak{C}_R}}\sum_{\substack{k_1 \geq 0 \\ \dots \\ k_r \geq 0}} \sum_{G_1^1, \dots, G^{k_1}_1 \in \mathfrak{D} } \dots \sum_{G_r^1, \dots, G^{k_r}_r \in \mathfrak{D}}  \\ \bigg(\prod_{i=1}^r \rho_L(G_i^L)\rho_R(G_i^R)\mathbf{P}(G^1_i) \cdots \mathbf{P}(G^{k_i}_i)\bigg) f(G_1, \dots, G_r) \Bigg\vert 
 \leq Cr\Vert f\Vert_{\infty}\e^{-cn}.
\end{multline}
where we sum over all the $G_1^L, \dots, G_r^L \in \mathfrak{C}_L$, the $G_1^R, \dots, G_r^R \in \mathfrak{C}_R$, the $G_1^1, \dots, G_1^{k_1}, \dots, G^1_r, \\ \dots,G^{k_r}_r \in \mathfrak{D}$ such that for any $1 \leq i \leq r$,
\begin{equ}
    X^L_0(G_i^L)+ X(G^i_1)+\dots+X(G^{k_i}_i) +X^R_x(G_i^R) = n(y_i-x_i).
\end{equ}

The coupling stated in Theorem~\ref{theoreme couplage oz 1 cluster} is available in this context: we call it $\Phi^{\otimes r}_{(0,x)\rightarrow (n,y)}$. It simply consists in the product of the $r$ couplings $\Phi_{(0,x_j)\rightarrow (n,y_j)}$ given by Theorem~\ref{theoreme couplage oz 1 cluster}. Its main feature is that for any bounded function of the skeletons of a system of $r$ clusters, 
\begin{multline}\label{equation couplage squelette p rw} \Bigg\vert
 \Phi^{\otimes r}_{(0,x)\rightarrow (n,y)} \left[ f\left(\mathcal{S}_1, \dots, \mathcal{S}_r \right)\right] - \\ \sum_{\substack{x_L^1, \dots, x_L^r \\ x_R^1, \dots, x_R^r}}\left(\prod_{i=1}^r \nu_L(x_L^i)\nu_R(x_R^i)\right)\left(\mathbb{E}^{\RW}\right)^{\otimes r}\left[f(x_L^1\circ S^1\circ x_R^1, \dots, x_L^r\circ S^r \circ x_R^r)\1_{S\in \hit_{ x_R - x_L}} \right] \Bigg\vert  \\ < Cr\Vert f \Vert_{\infty}\e^{-cn},
\end{multline}
where we denoted by $(\E^\RW)^{\otimes r}$ the expectation under the measure of $r$ independent directed walks $(S^1, \dots, S^r)$ started from 0, and $\hit_{x_R - x_L}$ the event that each $S^i$ ever hits $x_R^i - x_L^i$. Note that the uniform Ornstein--Zernike coupling introduced in Proposition~\ref{prop oz boundary conditions} also holds in this context. Moreover, Lemma~\ref{lemme skorokhod topology} is also true in its $r$-dimensional version, so that it will be sufficient to study $(\E^\RW)^{\otimes r}$ when estimating probabilities for \textit{scaled} random walks.

Before working on the repulsion estimates as announced, we lower bound the probability of non-intersection and connection in the product measure.  
\begin{Lemma}
\label{lemme minoration proba con nonint mesure produit} Let $x,y \in W\cap\Z^r$. Then, there exists $c>0$ such that 
\begin{equ}\label{equation lemme minoration proba con nonint mesure produit}
    \phi^{\otimes r}\left[ \nonint, \con \right] \geq cV(x)V(y) n^{- \frac{r^2}{2}}\e^{-\tau rn},
\end{equ}
where $V$ is the function introduced in Theorem~\ref{theoreme local limit srw}.
\end{Lemma}

\begin{proof}
We use the Ornstein--Zernike coupling given by~\eqref{equation couplage squelette p rw}: indeed, up to exponential terms due to the coupling, and using the diamond confinement property,

\begin{align*}
&\phi^{\otimes r}\left[ \nonint, \con \right] \\ &= \e^{-\tau rn}\Phi^{\otimes r}_{(0,x)\rightarrow (n,y)}\left[(\mathcal{C}_1, \dots, \mathcal{C}_r) \in \nonint \right] \\
&\geq \e^{-\tau rn}\Phi^{\otimes r}_{(0,x)\rightarrow (n,y)} \left[\bigcap_{1\leq i\neq j\leq r}\lbrace\mathcal{D}(\mathcal{S}^i)\cap\mathcal{D}(\mathcal{S}^j)=\emptyset\rbrace \right] \\
&= \e^{-\tau rn}\sum_{\substack{x^1_L, \dots, x_L^r \\ x^1_R, \dots, x_R^r}}\left(\prod_{i=1}^r \nu_L(x^i_L) \nu_R(x^i_R)\right)\\  &\hspace{20pt}\times c\left(\PP^\RW\right)^{\otimes r}\left[\bigcap_{1\leq i\neq j\leq r}\lbrace \mathcal{D}\left(x^i_L \circ S^i \circ x^i_R\right)\cap \mathcal{D}\left(x^j_L \circ S^j \circ x^j_R\right) = \emptyset\rbrace , \hit_{(n,y - x)}\right].
\end{align*}
Hence the result boils down to lower bound the probability of non-intersection and connection for $r$ independent \textit{decorated} directed random walks. This is precisely the content of Lemma~\ref{lemme lower bound int con decorated random walks}. By finiteness of the measures $\nu_L, \nu_R$, we conclude that
\begin{equ}
    \phi^{\otimes r}\left[ \nonint, \con \right] \geq cV(x)V(y) n^{-\frac{r^2}{2}}\e^{-\tau rn}.
\end{equ}
\end{proof}
Observe that the latter bound reads as follows on the coupling measure:
\begin{equ}\label{eq: lower bound proba nonint conditionnelle}
    \Phi^{\otimes r}_{(0,x)\rightarrow (n,y)}\left[ \nonint \right] \geq cV(x)V(y)n^{-\frac{r^2 }{2}}.
\end{equ}

Thanks to Proposition~\ref{prop oz boundary conditions}, the same analysis holds for deriving the analog of the latter result in a strip with boundary conditions. 
\begin{Cor}\label{corollaire majoration proba ni con mesure produit BC avec ecartement suffisant}
Let $x,y \in W$. Let $x_i \ni C^L_{i,\EXT}, y_i \ni C^R_{i,\EXT}$, and assume that the family $C = \left(C^L_{i,\EXT}, C^R_{i,\EXT}\right)_{1\leq i \leq r}$ satisfies the assumptions of the uniform Ornstein--Zernike coupling given by Proposition~\ref{prop oz boundary conditions}. Then, there exists a uniform constant $\chi > 0$ such that 
\begin{equ}
    \phi^{\otimes r}\left[ \con, \nonint ~\Bigg\vert \begin{aligned}
        &\mathcal{C}_{(0,x_i)} \cap (\Z^-\times\Z) = C^L_{i, \EXT},  \\
        &\mathcal{C}_{(n,y_i)} \cap ([n,+\infty]\times\Z) = C^R_{i,\EXT}, ~~~\forall 1 \leq i \leq r
    \end{aligned} \right] \geq \frac{c}{\chi}V(x)V(y)n^{-\frac{r^2}{2}}\e^{-\tau rn}. 
\end{equ}
\end{Cor}

\subsection{Edge repulsion}\label{sub edge repulsion}

The goal of this subsection is to prove Lemma~\ref{equation entropic repulsion independent system} which we refer to as the "edge repulsion" lemma for the independent system. Beforehand we introduce the important notion of \emph{synchronized skeleton} of a system of long clusters. 

Let $\mathcal{C} = (\mathcal{C}_1, \dots, \mathcal{C}_r)$ be sampled according to $\Phi^{\otimes r}_{(0,x)\rightarrow (n,y)}$. We say that $k \in \N$ is a \textit{synchronization time} for $\mathcal{C}$ if there exists $(s^1_k, \dots, s^r_k) \in \Z^r$ such that for any $1\leq i \leq r$, one has $(k, s^i_k) \in \mathcal{S}^i$. In other words, $k$ is a synchronization time for $\mathcal{C}$ if and only if each one of the $r$ skeletons of $\mathcal{C}$ contains a point of $x$-coordinate $k$. We define the set of synchronization times of $\mathcal{C}$ by 
\begin{equ}
    \mathsf{ST}(\mathcal{C}) = \lbrace 0 \leq k_1 < k_2 < \dots < k_l \leq n\rbrace.
\end{equ}
The \textit{synchronized skeleton} of $\mathcal{C}$, called $\check{\mathcal{S}}$ is now defined to be the process defined on $\mathsf{ST}(\mathcal{C})$, taking its values in $\Z^r$, such that for any $k \in \mathsf{ST}(\mathcal{C})$,
\begin{equ}
    \check{\mathcal{S}}_k = (\mathcal{S}^1(k), \dots, \mathcal{S}^r(k)).
\end{equ}
As usual, we extend this process as a function of $\R^+$ to $\R$ by linear interpolation.
Let us observe an important property of this process (which the reason of its introduction) before turning to Lemma~\ref{Lemme entropic repulsion independent system}. 

\paragraph{\textbf{Claim}.} 
Under $\Phi^{\otimes r}_{(0,x)\rightarrow (n,y)}$, 
\begin{equ}\label{claim inclusion synchronized events}
   \left\lbrace \left(\mathcal{C}_1, \dots, \mathcal{C}_r \right) \in \nonint \right\rbrace \subseteq \left\lbrace \left( \check{\mathcal{S}}(\mathcal{C}_1), \dots, \check{\mathcal{S}}(\mathcal{C}_r)) \right) \in \mathcal{W}_n \right\rbrace.
\end{equ}
\begin{proof}
    This is an immediate consequence of the intermediate value theorem.
\end{proof}
\begin{Rem}
    Observe that this inclusion would not be true when replacing $\check{\mathcal{S}}$ by $\mathcal{S}$ (see Figure~\ref{figure mauvaise intersection}).
\end{Rem}

 % Figure environment removed

Moreover, due to the exponential tails of the increments of $\mathcal{S}$, it is clear that the increments of $\check{\mathcal{S}}$ also have exponential tails. Thus, $\check{\mathcal{S}}$ falls into the class of \textit{synchronized directed random walks}, studied in Section~\ref{subsection non-intersecting systems of synchronized directed random walks}.

Next lemma indicates that in a time less than $n-o(n)$ the clusters have been far away from at least $n^\eps$ at least once. A convenient notion for stating this result is the gap of a point $x \in W$.

\begin{Def}[Gap of a point]
    Let $x \in W$. We define its \emph{gap} to be the following quantity:
    \begin{equ}
        \Gap(x) = \min_{1\leq i \leq r-1}(x_{i+1} - x_i).
    \end{equ}
    Observe that due to the fact that $x$ lies in $W$, $\Gap(x)$ is always a positive quantity.
\end{Def}

We are ready to state the edge repulsion result for the independent system.

\begin{Lemma}[Edge repulsion for the independent system]\label{Lemme entropic repulsion independent system}
There exists an $\eps > 0$ such that the following holds. Let $T_1$ and $T_2$ be the following random variables:
\begin{equ}
    T_1 = \inf \left\lbrace k \geq 0, \Gap(\check{\mathcal{S}}_k) > n^\eps \right\rbrace.
\end{equ}
and
\begin{equ}
    T_2 = \sup \left\lbrace k \geq 0, \Gap(\check{\mathcal{S}}_k)  > n^\eps \right\rbrace
\end{equ}
Then, there exist $C,c>0$ such that when $n\geq 0$ is large enough,
\begin{equ}\label{equation entropic repulsion independent system}
    \Phi^{\otimes r}_{(0,x)\rightarrow (n,y)} \left[ \left\lbrace T_1 > n^{1-\eps}\right\rbrace \cup \left\lbrace T_2 < n - n^{1-\eps} \right\rbrace \vert  \nonint \right] < \frac{1}{c}\exp(-cn^\frac{\eps}{2}).
\end{equ}
\end{Lemma}


\begin{proof}
We prove that $\Phi^{\otimes r}_{(0,x)\rightarrow (n,y)}\left[ T_1 > n^{1-\eps} \vert\nonint \right] < \exp(-cn^\eps)$. By time reversal and a basic union bound, it will be sufficient to conclude. We roughly upper bound:

\begin{equ}\label{e: numerateur denominateur preuve edge rep indep}
    \Phi^{\otimes r}_{(0,x)\rightarrow (n,y)}\left[ T_1 > n^{1-\eps} \vert\nonint \right] \leq \frac{ \Phi^{\otimes r}_{(0,x)\rightarrow (n,y)}\left[ T_1 > n^{1-\eps} \right]  }{ \Phi^{\otimes r}_{(0,x)\rightarrow (n,y)}\left[ \nonint  \right]
}.\end{equ}

We are going to separately bound the numerator and the denominator. We start by the numerator of~\eqref{e: numerateur denominateur preuve edge rep indep}. Since $T_1$ is measurable with respect to the synchronized skeleton of $\mathcal{C}$ (which itself is measurable with respect to $\mathcal{S}$), we use the fact that the law of $\mathcal{S}$ under $\Phi^{\otimes r}_{(0,x)\rightarrow (n,y)}$ is that of a system of directed random bridges to write - up to an exponential correction due to the coupling:
\begin{equ}
    \Phi^{\otimes r}_{(0,x)\rightarrow (n,y)}\left[ T_1 > n^{1-\eps} \right]  = \PP^{\RW}_{(0,x)}\left[ T_1(\check{S}) > n^{1-\eps} \vert \hit_{(n,y)} \right].
\end{equ}
We then are exactly in the context of entropic repulsion for synchronized directed random bridges, and we refer to Corollary~\ref{cor: repulsion SDRbridges}, which asserts that the latter probability is upper bounded by $c^{-1}\exp(-cn^\eps)$ for some constant $c>0$.

By~\eqref{eq: lower bound proba nonint conditionnelle}, we have a polynomial lower bound on the denominator. Hence, up to slightly changing the value of $c$, we obtained
\begin{equ}
    \Phi^{\otimes r}_{(0,x)\rightarrow (n,y)}\left[ T_1 > n^{1-\eps} \vert\nonint \right] < \frac{1}{c}\exp(-cn^\eps),
\end{equ}
which achieves the proof.
\end{proof}

By the synchronized renewal time property, for any $1\leq i\leq r$, the cluster $\mathcal{C}_i$ intersects the line $\lbrace T_1 \rbrace \times \Z$ (resp. $\lbrace T_2 \rbrace \times \Z$) at a unique vertex of $\Z^2$, whose $y$-coordinate shall be called $\mathscr{X}_i$ (resp. $\mathscr{Y}_i$). Thus, $\mathscr{X}$ and $\mathscr{Y}$ are elements of $\Z^r$ satisfying $\mathscr{X}_1< \dots < \mathscr{X}_r$ (resp. $\mathscr{Y}_1< \dots < \mathscr{Y}_r$). We introduce the following \emph{edge-regularity} condition:

\begin{Def}[Edge-regularity property]\label{def: definition edge regularity property}
Let $\om \in \con\cap\nonint$ be a percolation configuration. We call $\om$ \emph{edge-regular} an abbreviate this event in $\mathsf{EdgeReg}$ if it satisfies the following properties:
\begin{enumerate}[(i)]
    \item $T_1 < n^{1-\eps}$ and $T_2 > n-n^{1-\eps}$,
    \item $\norme{\mathscr{X}} \leq n^{1/2 - \eps/4}$ and $ \norme{\mathscr{Y}} \leq n^{1/2 - \eps/4}$
\end{enumerate}
where $\eps>0$ is given by Lemma~\ref{Lemme entropic repulsion independent system}.
\end{Def}
We then prove that a percolation configuration sampled under $\Phi^{\otimes r}_{(0,x)\rightarrow (n,y)}\left[~\cdot\vert\nonint \right]$ is typically edge-regular.
\begin{Lemma}\label{lem: edge regularity product measure}
    There exists a small constant $c>0$ such that 
    \begin{equ}
        \Phi^{\otimes r}_{(0,x)\rightarrow (n,y)}\left[\mathsf{EdgeReg}^c\vert\nonint\right] < \frac{1}{c}\exp(-cn^{\frac{\eps}{2}}).
    \end{equ}
\end{Lemma}
\begin{proof}
    Let us work conditionally on the event $T_1 < 2n^{1-\eps}$, as it has been proved to occur with exponentially large probability in Lemma~\ref{Lemme entropic repulsion independent system}. As previously, we use the rough upper bound
    \begin{eqnarray*}
        \Phi_{(0,x)\rightarrow(n,y)}^{\otimes r}\left[ \norme{\mathscr{X}} > n^{1/2-\eps/4}\vert\nonint\right] &=& \Phi_{(0,x)\rightarrow(n,y)}^{\otimes r}\left[ \norme{\mathcal{S}(T_1)}>n^{1/2-\eps/4} \vert\nonint \right] \\
        &=& \frac{\Phi_{(0,x)\rightarrow(n,y)}^{\otimes r}\left[ \norme{\mathcal{S}(T_1)}>n^{1/2-\eps/4}\right]}{\Phi_{(0,x)\rightarrow(n,y)}^{\otimes r}\left[\nonint\right]}.
    \end{eqnarray*}
    As previously we use the lower bound~\eqref{eq: lower bound proba nonint conditionnelle} to argue that the denominator is at least polynomial, while we are going to produce a stretched-exponential upper bound on the numerator. First, observe that
    \begin{eqnarray*}
        \Phi_{(0,x)\rightarrow(n,y)}^{\otimes r}\left[ \norme{\mathscr{X}} > n^{1/2-\eps/4}\right]  &=& \PP^{\RW}_{(0,x)}\left[ \norme{S(T_1)} > n^{1/2-\eps/4} \vert\hit_{(n,y)} \right] \\ &\leq& \frac{\PP^{\RW}_{(0,x)}\left[ \norme{S(T_1)} > n^{1/2-\eps/4} \right]}{\PP^{\RW}_{(0,x)}\left[ \hit_{(n,y)}\right]}.
    \end{eqnarray*}
By Theorem~\ref{theoreme local limit srw}, the denominator is at least polynomial.
Now observe that the classical theory of large deviations for random walks allows us to produce a stretched-exponential upper bound on the numerator (remember that we work conditionally on $T_1 < n^{1-\eps}$): there exists $c>0$ such that
\begin{equ}
    \PP^{\RW}_{(0,x)}\left[\norme{S(T_1)} > n^{\frac{1-\eps}{2}+\eps/4}\right] \leq \exp(-cn^{\frac{\eps}{2}}).
\end{equ}
This proves, as usual up to small change in the constant $c$, that
\begin{equ}
\Phi_{(0,x)\rightarrow(n,y)}^{\otimes r}\left[ \norme{\mathscr{X}} > n^{1/2-\eps/4}\vert T_1 \leq  n^{1-\eps}, \nonint\right] \leq \exp(-cn^{\frac{\eps}{2}}).
\end{equ}
We conclude writing (the factor 2 comes from the terms in $T_2$ and $\mathscr{Y}$ that are handled by symmetry):
\begin{multline*}
    \Phi_{(0,x)\rightarrow(n,y)}^{\otimes r}\left[ \mathsf{EdgeReg}^c\vert\nonint\right] \leq \\ 2\left(2\Phi_{(0,x)\rightarrow(n,y)}^{\otimes r}\left[ T_1 > n^{1-\eps}\vert\nonint\right]+\Phi_{(0,x)\rightarrow(n,y)}^{\otimes r}\left[ \norme{\mathscr{X}} > n^{1/2-\eps/4}\vert T_1 \leq n^{1-\eps}, \nonint\right]\right).
\end{multline*}
Thus,
\begin{equ}
    \Phi_{(0,x)\rightarrow(n,y)}^{\otimes r}\left[ \mathsf{EdgeReg}^c\vert\nonint\right] \leq \frac{1}{c}\exp(-cn^{\frac{\eps}{2}}).
\end{equ}
\end{proof}

\subsection{Convergence towards the Brownian watermelon}\label{sub convergence towards the brownian watermelon}

As we shall explain here, the edge repulsion stated in Lemma~\ref{Lemme entropic repulsion independent system} is the main ingredient needed to show that the rescaled system, \textit{sampled under the product measure} and conditioned both on the mutual avoidance of the clusters and on the connection event converges in distribution towards the Brownian watermelon. 

The technique of proof will be used several times through the paper. Basically, it consists in splitting the system of clusters sampled under the measure $\phi^{\otimes r}[~\cdot \vert \con, \nonint]$ in two different parts (for the sake of exposition, we explain the splitting on the first half of the cluster "near the starting point" - of course, one has to do the symmetric splitting "near the arrival point"). The first part will be given by the random time $T_1$ introduced in Lemma~\ref{Lemme entropic repulsion independent system}. At this time, the clusters are far from each other, sufficiently far for the conditioning on the non-intersection of the \textit{clusters} to be asymptotically equivalent to the conditioning on the non-intersection \textit{skeletons} of the clusters. This allows us to implement the Ornstein--Zernike coupling given by~\eqref{multidim oz} for the section of the clusters which is after $T_1$ (taking into account the boundary conditions enforced by the configuration outside of the strip thank to Proposition~\ref{prop oz boundary conditions}). We conclude by applying the invariance principle for directed random walks derived in Subsection~\ref{subsection non intersecting non-synchronized rw}.

Due to the fact that we work between the random times $T_1$ and $T_2$ we need a technical input that allows us to extend the convergence as a process of the interval $(0,1)$ to the convergence as a process defined on $[0,1]$.

\begin{Lemma}\label{Lemme technique processus sto}
    Let $G_n$ be random sequence of functions of the space $\mathcal{C}([0,1], \R^r)$ and $G$ be a continuous stochastic process of $\mathcal{C}([0,1], \R^r)$. Assume that: 
    \begin{enumerate}[(i)]
        \item For any $\delta>0$, for any bounded and continuous function $f: \mathcal{C}([\delta, 1-\delta], \R^r) \rightarrow \R$,
        \begin{equ}
            \E\left[f(\restriction{G_n}{[\delta, 1-\delta]})\right] \goes{}{n \rightarrow \infty}{\E\left[f(\restriction{G}{[\delta, 1-\delta]})\right]},
        \end{equ}
        \item For all $\eps > 0$
        \begin{equ}
        \lim_{t\rightarrow 0}\sup_{n\geq 0} \PP\left[ \big\vert G_n(t) - G_n(0)\big\vert > \eps\right]   = 0
        \end{equ}
        and
        \begin{equ}
        \lim_{t\rightarrow 1}\sup_{n\geq 0} \PP\left[ \big\vert G_n(t) - G_n(1)\big\vert>\eps \right]   = 0
        \end{equ}
    \end{enumerate}
    Then, $G_n$ converges in distribution towards $G$ in the space $\mathcal{C}([0,1], \R^r)$.
\end{Lemma}

\begin{proof}[Sketch of proof of Lemma~\ref{Lemme technique processus sto}]
The proof of Lemma~\ref{Lemme technique processus sto} relies on very classical arguments and we refer to~\cite{billingsley} for details. Observe that hypothesis $(i)$ together with the fact that the family $[\delta, 1-\delta]$ is a compact exhaustion of $(0,1)$ yields the convergence of $G_n$ towards $G$ as processes from $(0,1)$ to $\R^r$. The equicontinuity of $G_n$ at 0 and 1 (hypothesis $(ii)$) then yields the desired convergence by the Arzelà-Ascoli Theorem.
\end{proof}
This technical tool in hand, we can prove the main result of this section.
\begin{proof}[Proof of Proposition~\ref{Proposition convergence mesure produit}]
In what follows, let us introduce the scaled version of $\mathcal{S}$, for any $0\leq t\leq 1$:
\begin{equ}
    \mathcal{S}_n(t) = \frac{1}{\sqrt{n}}\mathcal{S}(nt).
\end{equ}

We are going to implement the strategy given by Lemma~\ref{Lemme technique processus sto} to show that under the measure $\Phi^{\otimes r}_{(0,x)\rightarrow (n,y)}[~\cdot\vert\nonint]$, the scaled system of skeletons $\mathcal{S}_n$ converges towards the Brownian watermelon as random functions of $\mathcal{C}([0,1], \R^r)$.

We start with the proof of point (i) (the crucial part of the proof). Fix $\delta > 0$. Fix $f^\delta : \mathcal{C}([\delta, 1-\delta], \R^r) \rightarrow \R$, continuous and bounded. Our goal is to show that there exists $\sigma > 0$, independent of $\delta$, such that
\begin{equ}\label{equation a montrer convergence FDD mesure produit}
    \Phi^{\otimes r}_{(0,x)\rightarrow (n,y)}\left[ f^\delta(\restriction{\mathcal{S}_n}{[\delta, 1-\delta]})\vert \nonint \right] \goes{}{n\rightarrow\infty}{\E\left[f^\delta(\restriction{\sigma\bw^{r}}{[\delta, 1-\delta]})\right]}.
\end{equ}
For sake of notational simplicity, the restrictions of the functions $\mathcal{S}_n$ and $\bw^{(r)}$ to the interval $[\delta, 1-\delta]$ will not be made explicit anymore.  

We first observe that, by Lemma~\ref{lem: edge regularity product measure}, and using the fact that $f^\delta$ is bounded, 

\begin{equ}
    \Phi^{\otimes r}_{(0,x)\rightarrow (n,y)}\left[ f^\delta(\restriction{\mathcal{S}_n}{[\delta, 1-\delta]})\vert \nonint \right] = (1+o(1))\Phi^{\otimes r}_{(0,x)\rightarrow (n,y)}\left[ f^\delta(\restriction{\mathcal{S}_n}{[\delta, 1-\delta]})\vert \nonint, \mathsf{EdgeReg} \right].
\end{equ}
Hence, it is sufficient to establish the convergence~\eqref{equation a montrer convergence FDD mesure produit} for the measure conditioned on the configuration to be edge-regular. We recall that under this conditioning, there exist $T_1$ and $T_2$ such that:
\begin{enumerate}[(i)]
    \item $T_1 < n^{1-\eps}, T_2 > n-n^{1-\eps}$ and $T_1$ and $T_2$ are synchronized renewal times of $\mathcal{C}$.
    \item $\Gap(\mathscr{X}), \Gap(\mathscr{Y}) > n^\eps$.
    \item $\norme{\mathscr{X}}, \norme{\mathscr{Y}} < n^{1/2 - \eps/4}$.
\end{enumerate}

We chose $n$ large enough so that $n\delta > T_1$ and $n(1-\delta) < T_2$\footnote{This is the only place where we use the fact that our functions are defined on $[\delta, 1-\delta]$ and it is the reason why we follow the strategy given by Lemma~\ref{Lemme technique processus sto} instead of directly working on $[0,1]$.}.

We call $\mathsf{Strip} := [T_1, T_2]\times\Z$. We are going to use an exploration argument, by conditioning on the portion of the clusters $\mathcal{C}$ that lies outside of $\mathsf{Strip}$. To that end, for an edge-regular percolation configuration $\om \in \nonint \cap \con$, we introduce the following sets of vertices:
\begin{equ}
    \EXT_i = (\mathcal{C}_i \cup \partial_{\mathsf{ext}}\mathcal{C}_i) \cap \mathsf{Strip}^c \qquad \text{and} \qquad \EXT = \bigcup_i \EXT_i.
\end{equ}

Now observe that summing over all the possible exterior edge-regular configurations yields
\begin{multline}\label{equation conditionnement preuve invariance systeme produit}
    \Phi^{\otimes r}_{(0,x)\rightarrow (n,y)}\left[ f^\delta(\mathcal{S}_n) \vert
        \nonint, \mathsf{EdgeReg} \right] = \\ \sum_{\mathsf{Ext}} \Phi^{\otimes r}_{(T_1,\mathscr{X})\rightarrow (T_2,\mathscr{Y})}\left[f^\delta(\mathcal{S}_n) \vert \nonint, \EXT = \mathsf{Ext} \right]\\\times\Phi_{x \rightarrow y}^{\otimes r}\left[\EXT = \mathsf{Ext} \vert \nonint, \mathsf{EdgeReg}\right].
\end{multline}


Fix such an admissible edge-regular $\mathsf{Ext}$. We use the following input from Section~\ref{subsection non intersecting non-synchronized rw}. By edge-regularity of $\mathsf{Ext}$, usual properties of the coupling and Lemma~\ref{lemme repulsion globale non-synchronized RW} ensure that
\begin{equ}
    \Phi^{\otimes r}_{(T_1,\mathscr{X})\rightarrow (T_2,\mathscr{Y})}\left[\inf_{t\in [T_1, T_2]} \Gap(\mathcal{S}(t)) \leq (\log n)^3 \vert \nonint, \EXT = \mathsf{Ext} \right] \goes{}{n \rightarrow \infty}{0}.
\end{equ}
Now under the complementary event, the diamond confinement property given by the Ornstein--Zernike coupling ensures that ($\Delta$ here means the symmetric difference) the event $\lbrace \left(\mathcal{S}(t)\right)_{0\leq t \leq n} \in \mathcal{W}_{[T_1, T_2]} \rbrace \Delta\lbrace \mathcal{C}\in \nonint \rbrace$  can occur only if one of the diamonds appearing in the diamonds decompositions of the $\mathcal{C}_i$ has a volume larger than $(\log^3 n)^2$. This event has been shown in Corollary~\ref{corollaire volume diamants} to occur with probability going to 0. Thus, we proved that:
\begin{equ}
    \Phi^{\otimes r}_{(T_1,\mathscr{X})\rightarrow (T_2,\mathscr{Y})}\left[ \left\lbrace \left(\mathcal{S}(t)\right)_{0\leq t \leq n} \in \mathcal{W}_{[T_1, T_2]} \right\rbrace \Delta \left\lbrace \mathcal{C}\in \nonint \right\rbrace \vert \EXT = \mathsf{Ext}\right]   \goes{}{n\rightarrow\infty}{0}.
\end{equ}
It is an easy consequence that:
\begin{multline}\label{equation conditionnement equivalents mesure prod}
    \Big\vert  \Phi^{\otimes r}_{(T_1,\mathscr{X})\rightarrow (T_2,\mathscr{Y})}\left[ f^\delta(\mathcal{S}_n) \vert 
        \nonint,\EXT = \mathsf{Ext}  \right] - \\ \Phi_{(T_1,\mathscr{X}) \rightarrow (T_2,\mathscr{ Y})}^{\otimes r}\left[f^\delta(\mathcal{S}_n) \vert 
            \mathcal{S} \in \mathcal{W}_{[T_1, T_2]},  \EXT = \mathsf{Ext}  \right] \Big\vert \goes{}{n \rightarrow\infty}{0}.
\end{multline}
The right-hand summand of the latter formula is measurable with respect to $\mathcal{S}$, except the conditioning on $\EXT = \mathsf{Ext}$. Thanks to the uniform Ornstein--Zernike formula stated in Proposition~\ref{Proposition oz uniforme boundary conditions} and Lemma~\ref{lemme skorokhod topology} to get rid of the boundary conditions, we obtain that - uniformly on $\mathsf{Ext}$ being edge-regular:
\begin{multline}
 \Phi^{\otimes r}_{(T_1,\mathscr{X})\rightarrow (T_2,\mathscr{Y})}\left[f^\delta(\mathcal{S}_n) \vert 
            \mathcal{S} \in \mathcal{W}_{[T_1, T_2]}, \EXT = \mathsf{Ext}  \right] - \\ \left(\PP^\RW\right)^{\otimes r} \left[ f^\delta(S)~ \big\vert S \in \mathcal{W}_{[T_1, T_2]}, \hit_{(0,ny)} \right] \Big\vert  \goes{}{n \rightarrow\infty}{0}
\end{multline}
The main input of Section~\ref{section marches}, namely Theorem~\ref{theoreme invariance principle for drw} then allows us to conclude that 
\begin{equ}
    \Phi^{\otimes r}_{(0,\mathscr{X})\rightarrow (n,\mathscr{Y})}\left[f^\delta(\mathcal{S}_n) \vert 
            \mathcal{S} \in \mathcal{W}_{[T_1, T_2]}, \EXT = \mathsf{Ext}  \right]  \goes{}{n\rightarrow \infty}{\E\left[f^\delta(\sigma\bw^{(r)})\right]},
\end{equ}
for some $\sigma>0$ that depends on the distribution $\mathbf{P}$ in~\eqref{equation OZ boundary conditions}, but of course not on $\delta$. This concludes the proof of point $(i)$ of Lemma~\ref{Lemme technique processus sto}.

The point $(ii)$ - the equicontinuity of the family $\mathcal{S}_n$ at 0 and 1 -  is an easy consequence of the above arguments together with the central limit Theorem. 

By Lemma~\ref{Lemme technique processus sto}, we thus proved that $\mathcal{S}_n$ converges in distribution towards $\bw^{(r)}$ when sampled under the distribution $ \Phi^{\otimes r}_{(0,\mathscr{X})\rightarrow (n,\mathscr{Y})}\left[f^\delta(\mathcal{S}_n) \vert 
            \mathcal{S} \in \mathcal{W}_{[T_1, T_2]}, \EXT = \mathsf{Ext}  \right]$. Now, observe that the diamond confinement property and the volume estimate stated in Lemma~\ref{corollaire volume diamants} yield that
\begin{equ}
    \Phi^{\otimes r}_{(0,x)\rightarrow (n,y)}\left[ \sup_{0\leq t \leq n}\left| \Gamma^\pm(t) - \mathcal{S}(t) \right| > \log^3 n \right] \leq \exp(-c\log^2 n).
\end{equ}
This concludes the proof, by the usual observation that this decay is faster than any polynomial.
\end{proof}

\section{Brownian watermelon asymptotics for the random-cluster measure}\label{section RCM}

Now that the convergence of the rescaled clusters towards the Brownian watermelon is established in the case of the product measure, the goal of the following section is to transfer this convergence to the rescaled clusters sampled under the "real" random-cluster measure, and thus to achieve our journey towards Theorems~\ref{theoreme estimation} and~\ref{Theoreme main}. The strategy looks similar to the precedent section: indeed, we shall first prove an edge repulsion lemma in Subsection~\ref{sub edge rep RCM}. Then we shall prove in Subsection~\ref{sub global rep RCM} that the clusters remain far away from each other in the bulk. This will finally allow us to conclude that in the bulk, the conditioned random-cluster measure is close to the conditioned product measure thanks to a mixing argument, and to import the results of the precedent section to conclude the proofs in Subsection~\ref{Sub finale mixing et preuves}. The main difficulty and the reason why we needed to introduce and study the measure $\phi^{\otimes r}$ is that a coupling such as $\Phi_{x \rightarrow y}^{\otimes r}$ is not available in this setting. Hence, the random diamond decomposition and its associated skeleton given by the coupling $\Phi_{x \rightarrow y}^{\otimes r}$ do not exist anymore. We then work with the \emph{maximal} diamond decomposition and \emph{maximal} skeletons of the clusters (see remark~\ref{remarque maximal skeleton}). We draw the attention of the reader on the fact that this maximal skeleton \emph{does not} behave like a process with independent increments as it was the case in Section~\ref{section independent system}.

Let us introduce the following notation: if $\mathcal{E}$ is a set of edges of $\Z^2$ and $\eta, \om$ are two percolation configurations, we set
\begin{equ}
    \lbrace \om \overset{\mathcal{E}}{=} \eta \rbrace = \lbrace \forall e \in \mathcal{E}, \om(e) = \eta(e) \rbrace.
\end{equ}
The first  easy comparison between the infinite volume and the product measures is given by the following lemma:
\begin{Lemma}\label{lemme estimation fine proba con ni}
Let $\mathcal{E}$ be an arbitrary subset of $E(\Z^2)$ and $\eta$ an arbitrary percolation configuration on $\Z^2$. Then,
\begin{equ}\label{equation lemme estimation fine proba con ni}
    \phi\left[ \con, \nonint \vert \om \overset{\mathcal{E}}{=} \eta  \right] \geq \phi^{\otimes r }\left[ \con, \nonint \vert  \om_1 \overset{\mathcal{E}}{=} \dots \overset{\mathcal{E}}{=} \om_r \overset{\mathcal{E}}{=} \eta  \right].
\end{equ}
\end{Lemma}

\begin{proof}

It is a simple consequence of the FKG inequality applied to $\phi[~.\vert \om \overset{\mathcal{E}}{=} \eta ]$, which is a random-cluster measure on the graph $\Z^2 \setminus \mathcal{E}$ with some boundary conditions imposed by the configuration $\eta$. If $C$ is a connected set of edges of $\Z^2$, let us introduce its edge exterior boundary by:
\begin{equ}
    \partial_{\text{ext}} C = \left\lbrace \lbrace x,y\rbrace \in E(\Z^2) \cap C^c, x \text{ is the endpoint of an edge of }C\right\rbrace.
\end{equ}
Then, we write, summing over all the potential realizations of $\mathcal{C}_1, \dots, \mathcal{C}_r$ such that $\con \cap \nonint$ occur:

\begin{align*}
\phi \left[\nonint, \con\vert \om \overset{\mathcal{E}}{=}\eta \right] 
&= \sum_{C_1, \dots, C_r } \phi\left[\bigcap_{i=1}^r\lbrace \mathcal{C}_i = C_i\rbrace \vert \om \overset{\mathcal{E}}{=}\eta\right]\\
&= \sum_{C_1, \dots, C_r } \phi\left[ \bigcap_{i=1}^r \lbrace C_i\text{ is open },\partial_{\text{ext}}C_i \text{ is closed} \rbrace\vert \om \overset{\mathcal{E}}{=}\eta\right] \\
&=\sum_{C_1, \dots, C_r } \phi\left[ \bigcap_{i=1}^r \lbrace \partial_{\text{ext}}C_i \text{ is closed}\rbrace\vert \om \overset{\mathcal{E}}{=} \eta \right]\prod_{i=1}^r \phi^0_{C_i}\left[C_i \text{ is open } \vert ~\om \overset{\mathcal{E}\cap C_i }{=} \eta \right]\\
&\geq\sum_{C_1, \dots, C_r } \prod_{i=1}^r \phi\left[\partial_{\text{ext}} C_i \text{ is closed }\vert \om \overset{\mathcal{E}}{=} \eta \right]\phi_{C_i}\left[C_i \text{ is open } \vert \om \overset{\mathcal{E}\cap C_i}{=}\eta \right]\\
&= \sum_{C_1, \dots, C_r } \prod_{i=1}^r \phi\left[\mathcal{C}_i = C_i \vert \om \overset{\mathcal{E}}{=} \eta \right]\\
&=\phi^{\otimes r} \left[ \nonint,\con \vert \om \overset{\mathcal{E}}{=} \eta \right],
\end{align*}
where the inequality comes from the positive association property~\eqref{equation FKG} of the measure 
\begin{equ} 
\phi\left[~\cdot\vert \om \overset{\mathcal{E}}{=}\eta \right].
\end{equ}
\end{proof}

\begin{Rem}\label{remarque minoration ni con arbitrary BC}
Observe that the latter lemma together with Lemma~\ref{lemme minoration proba con nonint mesure produit} immediately yields that for any $x, y \in W\cap\Z^r$,
\begin{equ}\label{equation minoration proba con nonint arbitrary boundary conditions}
    \phi\left[\nonint, \con  \right] \geq cV(x)V(y)n^{-\frac{r^2}{2}}\e^{-\tau rn}.
\end{equ}
This will be of particular interest later - and is the first half of the proof of Theorem~\ref{theoreme estimation}.
\end{Rem}
While the latter bound is optimal (up to a constant), we also import a rough non-optimal upper bound.
\begin{Lemma}\label{lemme majoration rough upper bound con ni}
    Let $x, y \in W \cap \Z^2$. Then,
    \begin{equ}
        \phi\left[\con, \nonint \right] \leq \e^{-\tau rn}.
    \end{equ}
\end{Lemma}

Before turning to the proof of Lemma~\ref{lemme majoration rough upper bound con ni}, we introduce a useful notation for the rest of the paper. When $x,y \in W\cap\Z^r$, if $1\leq i \leq r$ we write $(\con,\nonint)_{\neq i}$ for the event that $\mathcal{C}_1, \dots, \mathcal{C}_{i-1},\mathcal{C}_{i+1}, \dots, \mathcal{C}_r$ realize the connection event and are non-intersecting. Observe that whenever $1\leq i \leq r$, $(\con, \nonint) \subset (\con,\nonint)_{\neq i}$, while the opposite inclusion is obviously not true.

\begin{proof}[Proof of Lemma~\ref{lemme majoration rough upper bound con ni}]
    We proceed by induction on $r$. For $r = 1$, the statement to prove is 
    \begin{equ}
        \phi\left[(0,x) \leftrightarrow (n,y) \right] \leq \e^{-\tau rn},
    \end{equ}
    which is the consequence of a well-known subbaditivity argument. 

If the statement is established with $r$ clusters, let $x, y \in \Z^{r+1}$. Then, observe that if $\con, \nonint$ occurs, then $(\con, \nonint)_{\neq r+1}$ has to occur. Summing over all the potential realizations of $\mathcal{C}_1, \dots, \mathcal{C}_r$ under $\con, \nonint$, we get:
\begin{align*}
    \phi\left[\con, \nonint\right] &= \sum_{C_1, \dots, C_r}\phi\big[(0,x_{r+1})\overset{(C_1 \sqcup \dots \sqcup C_r)^c}{\longleftrightarrow} (n, y_{r+1}) \vert \mathcal{C}_i = C_i, \forall 1\leq i\leq r \big]\\ & \hspace{196.5pt}\times \phi\left[\mathcal{C}_i = C_i, \forall 1\leq i\leq r  \right] \\
    &= \sum_{C_1, \dots, C_r} \phi^0_{(C_1 \sqcup \dots \sqcup C_r)^c}\left[(0,x_{r+1}) \leftrightarrow (n,y_{r+1})\right]\phi\left[\mathcal{C}_i = C_i, \forall 1\leq i\leq r  \right] \\
    &\leq \phi\left[(0,x_{r+1}) \leftrightarrow (n,y_{r+1})\right]\phi\left[(\nonint, \con)_{\neq r+1}\right] \\
    &\leq \e^{-\tau n}\phi\left[(\nonint, \con)_{\neq r+1}\right],
\end{align*}
where we used~\eqref{equation smp} in the second line,~\eqref{Comparaison boundary conditions} in the third line, and the case $r=1$ in the last line. The statement follows by the induction hypothesis. 
    \end{proof}
    
We next state another consequence of these two bounds, observing that they allow us to derive a diamond confinement property for the infinite volume measure conditioned on $\con \cap \nonint$, the exact analog of Corollary~\ref{corollaire volume diamants} for the conditioned measure. We formulate it for a rather particular class of boundary conditions, in order to be able to apply it later: however the reader should think about the measure $\phi$ on the strip $\mathsf{Strip}_n$ with boundary conditions given by the trace of a subcritical cluster outside of the strip. We recall that $\mathcal{D}^\mathsf{max}(\mathcal{C})$ denotes the \textit{maximal} diamond decomposition of the cluster $\mathcal{C}$, and introduce the following events:

\begin{equ}
   \BigDiam_i = \lbrace \max_{\substack{\mathcal{D}\subset \mathcal{D}^{\mathsf{max}}(\mathcal{C}_i) \\ \mathcal{D}\text{ diamond}}} \Vol(\mathcal{D}) \geq \log^2 n \rbrace ~~~~\text{  and  }~~~~ \BigDiam = \bigcup_{i=1}^r\BigDiam_i.
\end{equ}

\begin{Lemma}[Diamond confinement]\label{lemme diamond confinement mesure conditionnee} 
Let $\Ext$ be a finite set of edges such that $\Ext \cap E(\mathsf{Strip}_n)  = \emptyset$. Then there exists a constant $c>0$ such that for any $n$ large enough,
\begin{equ}
    \phi^0_{\Ext^c} \Big[\BigDiam ~\big\vert \con,\nonint\Big] \leq \exp\left(-c\left(\log n\right)^{2}\right),
\end{equ}
where the max is taken over all the diamonds appearing in the maximal diamond decomposition of the clusters $\mathcal{C}_i$. 
\end{Lemma}

\begin{proof}
We write 
\begin{equ}
 \phi^0_{\Ext^c} \Big[\BigDiam ~\big\vert \con,\nonint\Big] \leq \sum_{i = 1}^r \phi^0_{\Ext^c} \Big[ \BigDiam_i ~\big\vert \con,\nonint\Big].
\end{equ}
We fix an $i \in \lbrace 1, \dots, r \rbrace$. Now we shall focus on the numerator of the latter probability, namely on estimating
the quantity $
    \phi^0_{\Ext^c} \Big[ \BigDiam_i, \con,\nonint\Big].
$
Summing over all the potential clusters $C_1, \dots, C_{i-1}, C_{i+1}, \dots, C_r$ under $\con, \nonint$, 
\begin{multline}\label{equ: equation conditionnement preuve max diamond}
\phi^0_{\Ext^c} [ \BigDiam_i, \con,\nonint] \\ = \sum_{ C_1, \dots, C_{i-1}, C_{i+1}, \dots C_r } \phi^0_{\Ext^c}[ \BigDiam_i, \con,\nonint \vert \mathcal{C}_j = C_j]\phi^0_{\Ext^c}[ \mathcal{C}_j = C_j ],
\end{multline}
where the conditioning holds over all the $1\leq j\neq i\leq r$.
Fix such a system of clusters $C_1, \dots, C_{i-1}, C_{i+1}, \dots, C_r$, and call by convenience 
 $   \widetilde{\Ext} = \Ext\cup\left(\bigcup_{1\leq j \neq i \leq r}C_j\cup\partial_{\mathsf{ext}}C_j\right).
$
We then observe that - thanks to~\eqref{equation smp}, 
\begin{equ}
    \phi^0_{\Ext^c} \Big[ \BigDiam_i, \con, \nonint ~ \big\vert \mathcal{C}_j = C_j\Big] = \phi^0_{\widetilde{\Ext}^c} \Big[\BigDiam_i , (n,y_i) \in \mathcal{C}_i \Big].
\end{equ}
Moreover, since the events $\lbrace (n,y_i) \in \mathcal{C}_i \rbrace $ and $\BigDiam_i$ are increasing, we obtain:
\begin{equ}
    \phi^0_{\widetilde{\Ext}^c} \Big[ \BigDiam_i, (n,y_i) \in \mathcal{C}_i \Big] \leq 
     \phi\Big[ \BigDiam_i, (n,y_i) \in \mathcal{C}_i \Big]
\end{equ}
But we are now in the setting of Corollary~\ref{corollaire volume diamants}, which ensures that (the diamonds appearing in the \textit{maximal} diamond decomposition are always contained in the ones appearing in the diamond decomposition given by the Ornstein--Zernike coupling):
\begin{equ}
   \phi\Big[ \BigDiam_i, (n,y_i) \in \mathcal{C}_i \Big] \leq \e^{-\tau n}n^{-c\log n}.
\end{equ}
Coming back to ~\eqref{equ: equation conditionnement preuve max diamond}, we proved that 
\begin{equ}
    \phi^0_{\Ext^c} \Big[ \BigDiam_i, \con,\nonint\Big] \leq \e^{-\tau n - c\log^2 n}\phi^0_{\Ext^c}\left[ (\con, \nonint)_{\neq i} \right]
\end{equ}
 Using the rough upper bound given by Lemma~\ref{lemme majoration rough upper bound con ni}, we obtain:
\begin{equ}
      \phi^0_{\Ext} \Big[ \BigDiam_i, \con,\nonint\Big] \leq \e^{-(\tau rn + c\log^2 n) }.
\end{equ}
Now, thanks to Remark~\ref{remarque minoration ni con arbitrary BC} and the uniform Ornstein--Zernike decay~\eqref{equation oz uniform BC}, we bound the denominator:
\begin{equ}
    \phi^0_{\Ext^c}\left[\con,\nonint\right] \geq cn^{-\frac{r^2}{2}}\e^{-\tau rn}.
\end{equ}
We just proved that 
\begin{equ}
     \phi^0_{\Ext^c} \Big[\BigDiam ~\big\vert \con,\nonint\Big] \leq c^{-1}n^\frac{r^2}{2}\exp(-c \log^2 n)), 
\end{equ}
which yields the result for another value of $c$ sufficiently small.
\end{proof}

\subsection{Edge repulsion}\label{sub edge rep RCM}

% \color{red}
% The goal of this section is to prove Lemma~\ref{Lemme entropic repulsion independent system} in the measure $\phi\left[~\cdot~ \big\vert \con, \nonint\right]$. However we need a new definition of the random times $T_1$ and $T_2$, since $\mathcal{S}$ is not available anymore. Recall the definition of the upper and lower interfaces of a cluster $\Gamma^{\pm}(t)$. As in Subsection~\ref{subsection repulsion indep} we need to consider synchronized systems of walks. 

% Remember that, even if the skeleton $\mathcal{S}$ is not available in our context, the clusters do admit a \emph{maximal} diamond decomposition and thus a \emph{maximal} skeleton.  We thus work with the \emph{synchronized maximal skeleton} of the system of clusters $\mathcal{C}$. We call this object $\check{\mathcal{S}^{\mathsf{max}}}$. We are now able to define $T_1$ and $T_2$.

% \begin{Def}
% Let $\eps >0$. We define the two following random times:
% \begin{equ}
% T_1 = \inf \lbrace k \geq 0, \Gap(\check{\mathcal{S}^{\mathsf{max}}}) > n^\eps \rbrace, 
% \end{equ}
% and 
% \begin{equ}
% T_2 = sup \lbrace k \geq 0, \Gap(\check{\mathcal{S}^{\mathsf{max}}}) > n^\eps \rbrace.
% \end{equ}
% \end{Def}


% The analogous of Lemma~\ref{Lemme entropic repulsion independent system} is the following:

% \begin{Lemma}[Edge repulsion]\label{Entropic repulsion principle} There exists $\eps > 0$ and $c>0$ such that for any $n$ large enough,
% \begin{equ}\label{equation entropic repulsion}
%     \phi \left[ \left\lbrace T_1 > n^{1-\eps}\right\rbrace \cup \left\lbrace T_2 < n-n^{1-\eps}\right\rbrace \vert \con, \nonint \right] < \exp\left(-cn^{\eps} \right).
% \end{equ}
% \end{Lemma}

% The value of $\eps> 0$ given by Lemma~\ref{Entropic repulsion principle} will be fixed in the rest of the paper. Lemma~\ref{Entropic repulsion principle} is going to be proved by showing by induction on $r\geq 0$ that there exists a constant $c = c(r) > 0$ such that 
% \begin{equ}\label{equ: hypothese de recurrence}\tag{HR}
%  \phi \left[  T_{1,r} > n^{1-\eps} \vert \con_r, \nonint_r \right] \leq \frac{1}{c(r)}\exp(-c(r)n^{\frac{\eps}{2}}).
% \end{equ} 

% To highlight the dependency in the parameter $r\geq 0$, we thus introduce the following set of notations. Let $x,y \in W\cap\Z^{r+1}$. We define, for $1 \leq k \leq r+1$,

% \begin{equ}
% \con_k = \lbrace (0,x_1) \leftrightarrow (n,y_1), \dots, (0, x_k) \leftrightarrow (n,y_k)  \rbrace \qquad \text{and} \qquad \nonint_k = \lbrace \forall 1 \leq i \neq j \leq k, \mathcal{C}_i \cap \mathcal{C}_j = \emptyset \rbrace. 
% \end{equ}

% Moreover, we also define $T_{1,k}$ and $T_{2,k}$ to be the associated times for $\mathcal{C_1}, \dots, \mathcal{C}_k$. \lucas{maybe give a better definition}. We start by proving the following estimate.

% \begin{Lemma}\ref{lem: estimate r to r+1 clusters} Let $x,y \in W\cap\Z^{r+1}$. Assume that~\ref{equ: hypothese de recurrence} holds for some fixed $r\geq 1$. Then,
% $\phi\left[ T_{1,r} > n^\eps \vert \con_{r+1}, \nonint_{r+1} \right] < \frac{2}{c(r)}\exp(-\frac{c(r)}{2} n^{-\eps}).$
% \end{Lemma}

% \begin{proof}
% We start by writing:
% \begin{equ}
% \phi\left[ T_{1,r} > n^\eps \vert \con_{r+1}, \nonint_{r+1} \right]  = \frac{\phi\left[ T_{1,r} > n^\eps, \con_{r+1}, \nonint_{r+1} \right]}{\phi\left[\con_{r+1}, \nonint_{r+1} \right]}. 
% \end{equ}
% By~\eqref{equation minoration proba con nonint arbitrary boundary conditions}, we can upper bound this term by $c^{-1}n^{\frac{(r+1)^2}{2}}\e^{-\tau(r+1)n}phi\left[ T_{1,r} > n^\eps, \con_{r+1}, \nonint_{r+1} \right]$. Now observe that:
% \begin{multline*}
% \phi\left[ T_{1,r} > n^\eps, \con_{r+1}, \nonint_{r+1} \right] = \\ \phi\left[ (0,x_{r+1}) \leftrightarrow (n,y_{r+1}), \mathcal{C_{r+1}}\cap \mathcal{C}_j \text{ for } 1 \leq j \leq r  \vert \con_r, \nonint_r,  T_{1,r} > n^\eps \right]\times \\  \phi\left[ T_{1,r} > n^\eps \vert \con_r, \nonint_r \right] \phi\left[\con_r, \nonint_r\right].
% \end{multline*}
% Using~\eqref{hypothese de recurrence} and Lemma~\ref{lemme majoration triviale}, we have 
% \begin{equ}
% \phi\left[ T_{1,r} > n^\eps \vert \con_r, \nonint_r \right] \phi\left[\con_r, \nonint_r\right] \leq \frac{1}{c}\e^{-(\tau rn + cn^{-\eps}}.
% \end{equ} 
% Our goal is now to prove that:  
% \begin{equ}\label{equ: equation a prouver preuve recurrence}
% \phi\left[ (0,x_{r+1}) \leftrightarrow (n,y_{r+1}), \mathcal{C_{r+1}}\cap \mathcal{C}_j \text{ for } 1 \leq j \leq r  \vert \con_r, \nonint_r,  T_{1,r} > n^\eps \right] \leq \e^{-\tau n}.
% \end{equ}
% Indeed, with that estimation in hand, we put all the pieces together to obtain that:
% \begin{equ}
% \phi\left[ T_{1,r} > n^\eps, \con_{r+1}, \nonint_{r+1} \right] \leq \frac{1}{c^2}n^{\frac{(r+1)^2}{2}}\exp(-c(r)n^{-\frac{\eps}{2}},
% \end{equ}
% which conclude the proof. 
% To obtain~\eqref{equ: equation a prouver preuve recurrence}, we observe that we can repeat the argument used in the proof of Lemma~\ref{lemme estimation triviale}: indeed, exploring the clusters $\mathcal{C}_1, \dots, \mathcal{C}_r$ yields free boundary on the remaining space, making the connection between $(0,x_{r+1})$ and $(n,y_{r+1})$ less likely to occur. Thus,
%  \begin{equ}
%  \phi\left[ (0,x_{r+1}) \leftrightarrow (n,y_{r+1}), \mathcal{C}_{r+1}\cap \mathcal{C}_j \text{ for } 1 \leq j \leq r  \vert \con_r, \nonint_r,  T_{1,r} > n^\eps \right] \leq  \phi\left[ (0,x_{r+1}) \leftrightarrow (n,y_{r+1})\right] \leq \e^{-\tau n}
%  \end{equ} by the usual subadditivity argument.
% \end{proof}

% This lemma in hand, we turn to the proof of Lemma~\ref{Entropic repulsion principle}.

% \begin{proof}[Proof of Lemma~\ref{Entropic repulsion principle}]

% As explained, we proceed by induction on $r$.

% Observe that the property is trivially true for $r=1$, as $\Gap(\check{\mathcal{S}^\mathsf{max}})$ is always infinite. 

% We now assume~\eqref{hypothese de recurrence} to hold true for some $r \geq 1$. We are going to prove that there exists $c(r+1)> 0$ such that:
% \begin{equ}
% \phi\left[  \right]
% \end{equ} 


% \end{proof}
% \color{black}

The goal of this section is to prove Lemma~\ref{Lemme entropic repulsion independent system} in the measure $\phi\left[~\cdot~ \big\vert \con, \nonint\right]$. However we need an alternative definition of the random times $T_1$ and $T_2$, since $\mathcal{S}$ is not available anymore. Recall the definition of the upper and lower interfaces of a cluster $\Gamma^{\pm}(t)$.

\begin{Def}
    Fix $\eps > 0$. We define the two following random variables. 
    \begin{equ}
        T'_1 = \min\left\lbrace t \geq 0, \min_{*, \star \in \pm}\min_{1\leq i < j \leq r}\big\vert \Gamma^\star_i(t) - \Gamma^*_j(t) \big\vert > n^\eps  \right\rbrace
    \end{equ}
    and
    \begin{equ}
        T'_2 = \max\left\lbrace t \geq 0, \min_{*, \star \in \pm}\min_{1\leq i < j \leq r}\big\vert \Gamma^\star_i(t) - \Gamma^*_j(t) \big\vert > n^\eps  \right\rbrace
    \end{equ}\end{Def}
    
The analogous of Lemma~\ref{Lemme entropic repulsion independent system} is the following:

\begin{Lemma}[Edge repulsion]\label{Entropic repulsion principle} There exists $\eps > 0$ and $c>0$ such that for any $n$ large enough,
\begin{equ}\label{equation entropic repulsion}
    \phi \left[ \left\lbrace T'_1 > n^{1-\eps}\right\rbrace \cup \left\lbrace T'_2 < n-n^{1-\eps}\right\rbrace \vert \con, \nonint \right] < 2\exp\left(-cn^{\eps} \right).
\end{equ}
\end{Lemma}

The value of $\eps> 0$ given by Lemma~\ref{Entropic repulsion principle} will be fixed in the rest of the paper.



\begin{proof}

Let $\eps> 0$ - its value will be determined at the end of the proof. By symmetry, we focus on proving the following bound
\begin{equ}
    \phi\left[T'_1 > n^{1-\eps} \vert \con,\nonint \right] \leq \exp(-cn^\eps). 
\end{equ}
As in the proof of Lemma~\ref{Lemme entropic repulsion independent system}, we will conclude by time reversal and an easy union bound. For $2 \leq i \leq r$, we define the event $\mathsf{MLCP}_i$ (meaning "many left-close points") by 
\begin{equ}
    \mathsf{MLCP}_i = \lbrace \# \lbrace k \in \lbrace 0, \dots, n^{1-\eps}\rbrace, |\Gamma^{-}_i(k) - \Gamma^+_{i-1}(k)| < n^\eps \rbrace \geq \frac{1}{r} n^{1-\eps} \rbrace. 
\end{equ}
The reason for the introduction of this event is the following inclusion (that is a simple consequence of the pigeonhole principle):
\begin{equ}
    \lbrace T'_1 > n^{1-\eps }\rbrace \subset \bigcup_{i=2}^r \mathsf{MLCP}_i.
\end{equ}
Thus, by union bound
\begin{equ}
    \phi\left[T'_1 > n^{1-\eps} \vert \con, \nonint\right] \leq \sum_{i=2}^r \phi\left[\mathsf{MLCP}_i\vert\con,\nonint\right].
\end{equ}
Fix some $i \in \lbrace 2, \dots, r\rbrace$. As usual we upper bound $\phi\left[\mathsf{MLCP}_i\vert\con,\nonint\right]$ by separately bounding the numerator and the denominator of this fraction. We start with the numerator, and we write, conditioning over all the possible clusters $C_1, \dots, C_{i-1}, C_{i+1}, \dots, C_r$ under $\con, \nonint$:
\begin{equ}\label{eq: equation conditionnement clusters repulsion edge}
    \phi\left[\mathsf{MLCP}_i,\con,\nonint\right] =  \sum_{C_1, \dots, C_{i-1}, C_{i+1}, \dots, C_r}\phi\left[\mathsf{MLCP}_i, (n,y_i)\in\mathcal{C}_i \vert \mathcal{C}_j=C_j\right]\phi\left[\mathcal{C}_j = C_j\right],
\end{equ}
 where the conditioning on $\lbrace \mathcal{C}_j = C_j \rbrace$ holds over all the $1 \leq i\neq j\leq r$.
 Let us fix $C_1, \dots, C_{i-1}, C_{i+1}, \dots, C_r$ that can appear in the sum ~\eqref{eq: equation conditionnement clusters repulsion edge}.
As in the precedent proof, we define the following set of edges
\begin{equ}
    \widetilde{\Ext} = \bigcup_{1\leq j \neq i \leq r} (C_j \cup \partial _{\text{ext}}C_j).
\end{equ}
Following the previous computation and using the spatial Markov property~\eqref{equation smp}, we observe that 
\begin{equ}
    \phi\left[\mathsf{MLCP}_i, (n,y_i)\in\mathcal{C}_i \vert \mathcal{C}_j=C_j\right] = \phi^0_{\widetilde{\Ext}^c}\left[ \mathsf{MLCP}_i, (n,y_i)\in \mathcal{C}_i\right],
\end{equ}
where
\begin{multline*}
    \phi^0_{\widetilde{\Ext}^c}\left[ \mathsf{MLCP}_i,  (n,y_i)\in\mathcal{C}_i\right] =  \\ \phi^0_{\widetilde{\Ext}^c}\left[(n,y_i)\in\mathcal{C}_i, \# \lbrace k \in \lbrace 0, \dots, n^{1-\eps}\rbrace, |\Gamma^{-}_i(k) - \Gamma^+(C_{i-1})(k)| < n^\eps \rbrace \geq \frac{1}{r} n^{1-\eps}\right].
\end{multline*}
Now, observe that the event appearing on the right-hand side of the latter equation is actually increasing (the connection event is always increasing, and adding edges to the configuration can only push down $\Gamma^-_i$, making it closer to $\Gamma^+(C_{i-1})$). Thus, by ~\eqref{Comparaison boundary conditions}, we obtain:
\begin{multline}\label{equ: equation remarque increasing event}
    \phi\left[\mathsf{MLCP}_i, (n,y_i)\in\mathcal{C}_i \vert \mathcal{C}_j=C_j\right] \leq \\ \phi\left[  (n,y_i)\in\mathcal{C}_i, \# \lbrace k \in \lbrace 0, \dots, n^{1-\eps}\rbrace, |\Gamma^{-}_i(k) - \Gamma^+(C_{i-1})(k)| < n^\eps \rbrace \geq \frac{1}{r} n^{1-\eps} \right]. 
\end{multline}
We are now in the framework of the classical one-cluster Ornstein--Zernike theory - and we are going to conclude using Lemma~\ref{Confinment lemma}. Indeed, observe that - using the elementary fact that any point of the skeleton of a cluster necessarily belong to the interfaces $\Gamma^\pm$:
\begin{multline*}
    \phi\left[  (n,y_i)\in\mathcal{C}_i, \# \lbrace k \in \lbrace 0, \dots, n^{1-\eps}\rbrace, |\Gamma^{-}_i(k) - \Gamma^+(C_{i-1})(k)| < n^\eps \rbrace \geq \frac{1}{r} n^{1-\eps} \right] \\
    \leq \e^{-\tau n}\Phi_{(0,x_i)\leftrightarrow (n,y_i)}\left[\# \lbrace 0 \leq k \leq n^{1-\eps}, |\mathcal{S}^i(k) - \Gamma^+(C_{i-1})(k)|<n^\eps \rbrace \geq \frac{1}{r}n^{1-\eps} \right]. 
\end{multline*}
Since $\Gamma^+(C_{i-1})$ is the graph of a function that converges pointwise under the Brownian scaling (by Donsker's Theorem), then we can apply Lemma~\ref{Confinment lemma} \footnote{Lemma~\ref{Confinment lemma} is stated for unconditioned random walks, but as usual due to the local limit Theorem, being a bridge has a polynomial probability which is always beaten by the quantity $\exp(-cn^\eps)$} to argue that
\begin{equ}
    \Phi_{(0,x_i)\leftrightarrow (n,y_i)}\left[\# \lbrace k \in \lbrace 0, \dots, n^{1-\eps}\rbrace, |\mathcal{S}^i(k) - \Gamma^+(C_{i-1})(k)|<n^\eps \rbrace \geq \frac{1}{r}n^{1-\eps} \right] \leq \exp(-cn^\eps),
\end{equ}
provided that $\eps >0 $ is small enough. Coming back to ~\eqref{eq: equation conditionnement clusters repulsion edge}, we proved that 
\begin{equ}
    \phi\left[ \mathsf{MLCP^i(r)},\con,\nonint\right] \leq \exp(-cn^\eps - \tau n)\phi\left[ (\con, \nonint)_{\neq i}\right]. 
\end{equ}
Finally using the rough bound given by Lemma ~\ref{lemme majoration rough upper bound con ni} we proved that
\begin{equ}
    \phi\left[T'_1 > n^{1-\eps}, \con, \nonint \right] \leq \exp(-(\tau rn + cn^\eps)).
\end{equ}
We conclude by using the lower bound on $\phi\left[\con, \nonint \right]$ given by ~\eqref{equation minoration proba con nonint arbitrary boundary conditions}. We obtain
\begin{equ}
     \phi\left[T'_1 > n^{1-\eps}\vert \con, \nonint \right] \leq n^{\frac{r^2}{2}}\exp(-cn^\eps),
\end{equ}
which yields the result, up to slightly changing the value of the constant $c$.
% We first introduce a few notations. Throughout the proof we fix $\eps>0$ - the value of $\eps_0$ is going to be determined by Lemma~\ref{Confinment lemma}. We introduce the following events:
% \begin{equ}
%     \mathsf{Rep^L} = \left\lbrace \vartheta_n \leq n^{1-\eps} \right\rbrace,
% ~~~~\text{and}~~~~
%     \mathsf{Rep^R} = \left\lbrace \widetilde{\vartheta}_n \geq n-n^{1-\eps} \right\rbrace,
%  \end{equ}
%  and set
%  \begin{equ}
%     \rep = \mathsf{Rep^L} \cap \mathsf{Rep^R}.
% \end{equ}
% For the rest of the proof, due to the obvious symmetry between $\mathsf{Rep^L}$ and $\mathsf{Rep^R}$, we only focus on proving that 
% \begin{equ}
%     \phi \left[ \left(\mathsf{Rep^L}\right)^c \vert \con,\nonint \right] \leq \exp\left(-cn^{\eps}\right).
% \end{equ}
% For $2 \leq i \leq r$, we also introduce the event $\mathsf{Close}^\mathsf{L}_i$ defined by:
% \begin{equ}
%     \mathsf{Close}^\mathsf{L}_i = \left\lbrace \inf \left\lbrace t \geq 0, \left| \Gamma^-_i(t) - \Gamma^+_{i-1}(t)\right| > n^\eps \right\rbrace > n^{1-\eps} \right\rbrace,
% \end{equ}
% and notice that 
% \begin{equ}
%     \phi\left[\left(\mathsf{Rep^L}\right)^c \vert \nonint,\con \right] \leq \sum_{i=2}^r \phi\left[ \mathsf{Close}^\mathsf{L}_i \vert \nonint, \con \right].
% \end{equ}
% We are going to show that there exists $c>0$ such that for any $i \in \left\lbrace 2, \dots, r \right\rbrace$
% \begin{equ}\label{equation a prouver proba closei}
%     \phi \left[\mathsf{Close}^\mathsf{L}_i \vert \nonint, \con \right] < \e^{-cn^{\eps}}.
% \end{equ} 
% We focus on the numerator of the left-hand side of~\eqref{equation a prouver proba closei}. As done previously, we condition on the possible realizations of the clusters $\mathcal{C}_1, \dots, \mathcal{C}_{i-1}, \mathcal{C}_{i+1}, \dots, \mathcal{C}_r$ that realize the event $(\nonint, \con)_{r-1}$. 
% \begin{multline}\label{equation conditionnée lemme repulsion edge}
% \phi \left[\mathsf{Close}^\mathsf{L}_i, \nonint, \con \right]
% =\sum_{C_1, \dots, C_{i-1}, C_{i+1}, \dots, C_{r}}\phi\left[ \mathsf{Close}^\mathsf{L}_i, \con, \nonint \vert \mathcal{C}_j = C_j, \forall 1 \leq j\neq i \leq r \right]\\ \times \phi\left[\mathcal{C}_j = C_j, \forall 1 \leq j\neq i \leq r \right].
% \end{multline}

% Now let us define the following set of edges
% \begin{equ}
%     \mathsf{Ext} = \bigcup_{1\leq j \neq i \leq r} (C_j \cup \partial _{\text{ext}}C_j).
% \end{equ}
% Following the previous computation and using the spatial Markov property~\eqref{equation smp}, we have  
% \begin{multline*}
% \phi \left[ \mathsf{Close}^\mathsf{L}_i, \nonint, \con \right] 
% = \\ \sum_{C_1, \dots, C_{i-1}, C_{i+1}, \dots, C_r} \phi^0_{\Z^2 \setminus \EXT}\left[(0, x_i)\leftrightarrow(n,y_i), \sup_{0<t<n^{1-\eps}}\big\vert \Gamma^-_i(t) - \Gamma^+(C_{i-1})(t)\big\vert < n^\eps \right] \\ \times \phi\left[\mathcal{C}_j = C_j, \forall 1 \leq j\neq i \leq r\right].
% \end{multline*}
% For now we focus on the term
% \begin{equ}
%  \phi^0_{\Z^2 \setminus \EXT}\left[(0, x_i)\leftrightarrow(n,y_i), \sup_{0<t<n^{1-\eps}}\big\vert \Gamma^-_i(t) - \Gamma^+(C_{i-1})(t)\big\vert < n^\eps \right].
% \end{equ}
% Now, we notice that the event
% \begin{equ}
%     \big\lbrace (0, x_i)\leftrightarrow(n,y_i), \sup_{0<t<n^{1-\eps}}\big\vert \Gamma^-_i(t) - \Gamma^+(C_{i-1})(t)\big\vert < n^\eps \big\rbrace.
% \end{equ}
% is increasing. Indeed, observe that adding new edges to $\mathcal{C}_i$ can only push its bottom interface towards $\Gamma^+(C_{i-1})$. Then, we can make use of the monotone coupling between $\phi^0_{\EXT^c}$ and $\phi$ to obtain:
% \begin{multline}
% \phi^0_{\Z^2 \setminus \EXT}\big[ (0, x_i) \leftrightarrow(n,y_i), \sup_{0<t<n^{1-\eps}}\big\vert \Gamma^-_i(t) - \Gamma^+(C_{i-1})(t)\big\vert < n^\eps  \big] \\
% \leq \phi \big[(0, x_i) \leftrightarrow(n,y_i), \sup_{0<t<n^{1-\eps}}\big\vert \Gamma^-_i(t) - \Gamma^+(C_{i-1})(t)\big\vert < n^\eps\big].
% \end{multline}
% We are now in the framework of the classical one-cluster Ornstein--Zernike theory presented in Section~\ref{section review oz}. Let us work under the coupling $\Phi_{(0,x_i)\rightarrow(n,y_i)}$. Under this coupling, consider the random variable $\mathcal{S}$. Since with exponentially high probability, each one of the steps of $\mathcal{S}$ is a renewal of $\mathcal{C}_i$, in particular each step of $\mathcal{S}$ belongs to $\Gamma^-(\mathcal{C}_i)$. Thus,
% \begin{equ}
%     \big\lbrace \sup_{0<t<n^{1-\eps}}\big\vert \Gamma^-_i(t) - \Gamma^+(C_{i-1})(t)\big\vert < n^\eps \rbrace \subset \big\lbrace\max_{k, (\mathcal{S}_k)_1 \leq n^{1-\eps}}\big\vert \mathcal{S}_k - \Gamma^+(C_{i-1})((\mathcal{S}_k)_1) \big\vert <n^\eps \big\rbrace.
% \end{equ}
% Only for this proof, let us call temporarily $\tilde{\Gamma}^+(C_{i-1})(t)$ the function obtained by linear interpolation between the vertices $((\mathcal{S}_k)_1, \Gamma^+_{i-1}((\mathcal{S}_k)_1))$. Using the elementary fact that the supremum of the difference of two piecewise linear function with slope changes occuring at the same time can be achieved only at a slope change time (this is the reason for introducing $\tilde{\Gamma}^+(C_{i-1})$), we proved that  
% \begin{multline}
%     \phi^0_{\Z^2 \setminus \EXT}\big[ (0, x_i) \leftrightarrow(n,y_i), \sup_{0<t<n^{1-\eps}}\big\vert \Gamma^-_i(t) - \Gamma^+(C_{i-1})(t)\big\vert < n^\eps\big] \\ \leq \PP^\RW\big[\sup_{0< t \leq n^{1-\eps} }\big\vert S(t) - \tilde{\Gamma}^+(C_{i-1})  \big\vert\big].
% \end{multline}
% We thus upper bounded the left-hand side term by the probability for a directed random walk to stay confined inside a tube of diameter $n^\eps$ during a time $n^{1-\eps}$. But we proved in Lemma~\ref{Confinment lemma} that this event has a probability smaller than $\exp(-cn^\eps)$. Thus, thanks to Lemma~\ref{Confinment lemma} we obtained that - coming back to~\eqref{equation conditionnée lemme repulsion edge}:
% \begin{equ}
%     \phi \left[ \mathsf{Close}^\mathsf{L}_i, \con,\nonint\right] < \e^{-\tau n - cn^{\eps}} \phi\left[ \left( \con, \nonint \right)_{r-1}\right]. 
% \end{equ}
% Roughly upper bounding $\phi\left[ \left( \con, \nonint \right)_{r-1}\right]$ thanks to Lemma~\ref{lemme majoration rough upper bound con ni}, and using the remark~\ref{remarque minoration ni con arbitrary BC} to bound the denominator, we obtain:
% \begin{eqnarray*}
%     \phi \left[ \mathsf{Close}^\mathsf{L}_i \vert \con,\nonint \right] &\leq& \e^{-cn^\eps}n^{\frac{r^2}{2}} \\
%     &\leq& \exp\left(-cn^{\eps}\right),
% \end{eqnarray*}
% for a larger value of $c$ and provided that $n$ is large enough. This shows that:
% \begin{equ}
% \phi\left[(\mathsf{Rep^L})^c\vert \con,\nonint \right] \leq r\exp(-cn^{\eps}).
% \end{equ}
% By time reversal, the same inequality can be proved for $\tilde{\vartheta}_n$. Now a simple union bound immediately yields the proposition. 
\end{proof}

Observe that in the definition of $T'_1$ and $T'_2$, it is not the case that those random variables are necessarily synchronization times. The reason for that is that enforcing that condition would not have left us with an increasing event and we could not have obtained the bound given by ~\eqref{equ: equation remarque increasing event}. Thus let us define the actual random variables $T_1$ and $T_2$ by the following:
\begin{equ}
    T_1 = \min\lbrace k \geq T'_1, k \text{ is a synchronization time for }\mathcal{S}^{\mathsf{max}} \rbrace 
\end{equ}
and
\begin{equ}
    T_2 = \max\lbrace k \leq T'_2, k \text{ is a synchronization time for }\mathcal{S}^{\mathsf{max}} \rbrace.
\end{equ}
We observe that, as in the precedent section, the precedent result allows one to get an edge-regular configuration with large probability. We recall the notion of edge-regular configuration from the last section: we slightly modify it be calling a percolation configuration $\om \in \con,\nonint$ \emph{edge-regular} and writing $\om\in \mathsf{EdgeReg}$ if the following set of conditions is satisfied: 
\begin{enumerate}[(i)]
    \item $T_1 < n^{1-\eps}$ and $T_2 > n-n^{1-\eps}$.
    \item $\norme{\mathscr{X}}\leq n^{1/2- \eps/4}$ and $\norme{\mathscr{Y}} \leq n^{1/2 - \eps/4}$.
    \item $\Gap(\mathscr{X}) > \frac{1}{2}n^{\eps}$ and $\Gap(\mathscr{Y}) > \frac{1}{2}n^\eps$.
\end{enumerate}
As previously, we prove that typical configurations are edge-regular under the conditioning on $\con, \nonint$.
\begin{Lemma}\label{lem: edge regularity true measure}
   There exists a constant $c>0$ such that 
   \begin{equ}
       \phi\left[\mathsf{EdgeReg}^c\vert\con,\nonint\right] \leq \frac{1}{c}\exp(-cn^{\eps/2}).
   \end{equ}
\end{Lemma}
We only briefly sketch the proof since it is very similar to the one of Lemma~\ref{lem: edge regularity product measure}.
\begin{proof}
By Lemma ~\ref{lemme diamond confinement mesure conditionnee}, it is easy to see that:
\begin{equ}\label{equ: modified T_1 close to T_1}
    \phi\left[ \max\lbrace |T_1-T'_1|,|T_2-T'_2| \rbrace > (\log n)^3 \vert \con,\nonint\right] \leq \e^{-c(\log n)^2}
\end{equ}
for some small constant $c>0$. Indeed, one can use Lemma ~\ref{lemme diamond confinement mesure conditionnee} together with the fact that being a cone-point is a decreasing event. This fact established, the proof is essentially the same as the proof of Lemma~\ref{lem: edge regularity product measure}: in a time smaller than $(\log n)^2$, the clusters cannot move to a polynomial distance of their starting point by a basic large deviations estimate. 
\end{proof}
\subsection{Global repulsion }\label{sub global rep RCM}

In this subsection, we work under the measure $\phi\left[~\cdot\vert \con, \nonint\right]$, and we want to prove that between a time $o(n)$ and $n-o(n)$, the minimal gap between the clusters diverges with $n$. The event that we are going to estimate is the following "global repulsion" event: 


\begin{multline}
    \mathsf{GlobRep} := \lbrace T_1 < n^{1-\eps} \rbrace \cap \lbrace T_2 > n-n^{1-\eps} \rbrace \\ \cap \big\lbrace \min_{t \in [T_1,T_2]} \min_{1 \leq i< j \leq r} \min_{* ,\star\in \pm}\big\vert \Gamma^\star_i(t) - \Gamma^*_j(t) \big\vert > \log^2 n \big\rbrace. 
\end{multline}

The goal of this section is to prove the following statement.

\begin{Prop}[Global repulsion estimate]\label{proposition global entropic repulsion} 
    There exists $\beta > 0$, depending only on $r$, and $c>0$ such that 
    \begin{equ}\label{entropic repulsion equation}
        \phi \left[\mathsf{GlobRep} \vert \con,\nonint \right] \geq 1-cn^{-\beta}.
    \end{equ}
\end{Prop}


This lemma will be the main ingredient for the proofs of Theorems~\ref{theoreme estimation} and~\ref{Theoreme main}. 
The rest of the section is dedicated to the proof of Proposition~\ref{proposition global entropic repulsion}.
Observe that, due to Lemma~\ref{lem: edge regularity true measure} 
\begin{eqnarray*}
    \phi\left[\mathsf{GlobRep}^c\vert \nonint, \con\right] 
    &\leq& \phi[\mathsf{EdgeReg}^c \vert\nonint, \con] + \phi[\mathsf{GlobRep}^c \vert \nonint, \con, \mathsf{EdgeReg} ]\\
    &\leq& \frac{1}{c}\exp\left(-cn^{\eps/2}\right) + \phi[\mathsf{GlobRep}^c \vert \nonint, \con, \mathsf{EdgeReg}].
\end{eqnarray*}
We will focus on bounding the second term in the right-hand side of the above. We will do so by conditioning on $T_1$, $T_2$ and the shape of the clusters before $T_1$ and after $T_2$.

Recall that $\EXT$ is the trace of the clusters and their boundaries outside of the strip $\mathsf{Strip}$. Also recall the notation $\mathscr{X}$ and $\mathscr{Y}$ for the vertical positions of the renewals at the times $T_1$ and $T_2$. 
Recall that $\phi[.|\con, \nonint, \EXT]$\footnote{The conditioning on $\EXT$ contains implicitly the fact that $T_1$ and $T_2$ are renewals.} on the complement of $\EXT$ is the measure  $\phi^0_{ \EXT^c}$ conditioned on $\mathscr{X}$ being connected to $\mathscr{Y}$ by disjoint clusters contained in the respective diamonds. Write $\mathsf{Diam}_i$ for the fact that the $i$-th connection above occurs indeed in the diamond, and $\mathsf{Diam}$ for the fact the all the $\mathsf{Diam}_i$ occur simultaneously. Then 
\begin{align}
    \label{equation conditionnement trajext preuve globrep}
    \phi\big[\mathsf{GlobRep}^c\vert\nonint, &\con, \mathsf{EdgeReg}\big] \nonumber
    \\ &= \sum_{\mathsf{Ext}} \phi \left[ \mathsf{GlobRep}^c \big\vert \nonint, \con,\EXT = \mathsf{Ext} \right]\phi\left[\EXT=\mathsf{Ext}  ~\big\vert\con, \nonint,\mathsf{EdgeReg}\right]\nonumber\\
    &=\sum_{\mathsf{Ext}} \frac{\phi \left[ \mathsf{GlobRep}^c,\nonint, \con|\EXT = \mathsf{Ext} \right]}
    {\phi \left[ \nonint, \con|\EXT = \mathsf{Ext} \right]}
    \phi\left[\EXT=\mathsf{Ext}  ~\big\vert\con, \nonint,\mathsf{EdgeReg}\right],
\end{align}
where the sum runs over all possible edge-regular realizations $\mathsf{Ext}$ of $\EXT$.
We are going to focus on bouding the ratio above uniformly in all edge-regular $\Ext$. 
Due to the structure of the measure conditioned on $\lbrace \EXT = \mathsf{Ext}\rbrace$, we may write this ratio as
\begin{align}\label{equation mesure conditionnement trajext preuve globrep}
    \frac{\phi \left[ \mathsf{GlobRep}^c,\nonint, \con|\EXT = \mathsf{Ext} \right]}
    {\phi \left[ \nonint, \con|\EXT = \mathsf{Ext} \right]}=
    \frac{\phi_{\Ext^c}^0 \left[ \mathsf{GlobRep}^c,\nonint, \con, \mathsf{Diam}\right]}
    {\phi_{\Ext^c}^0 \left[\nonint, \con, \mathsf{Diam}\right]}
\end{align}

Notice that Lemma~\ref{lemme estimation fine proba con ni} and Corollary~\ref{corollaire majoration proba ni con mesure produit BC avec ecartement suffisant}  allow us to lower bound the denominator as
\begin{equ}\label{eq: lower bound denominator preuve globrep}
    {\phi_{\Ext^c}^0 \left[\nonint, \con, \mathsf{Diam}\right]}
    \geq \tfrac{1}{\chi} V(\mathscr{X})V(\mathscr{Y}) (T_2 - T_1)^{r^2/2} e^{-\tau r (T_2 - T_1)},
\end{equ}
where $\chi$ and $V(\cdot)$ were described in the aforementioned lemmas.

Finally, we claim the following upper bound on the numerator.

\begin{Lemma}\label{lem:Ioan_numerator}
    There exists constants $\beta > 0$ and $C > 0$ such that, for any edge-regular $\mathsf{Ext}$
    \begin{equ}
    \phi_{\Ext^c}^0 \left[ \mathsf{GlobRep}^c,\nonint, \con, \mathsf{Diam}\right] 
    \leq C V(\mathscr{X})V(\mathscr{Y}) (T_2 - T_1)^{-r^2/2 - \beta} e^{-\tau r (T_2 - T_1)}.
    \end{equ}
\end{Lemma}

The lemma above is the main difficulty in the proof of Proposition~\ref{proposition global entropic repulsion}; 
we postpone its proof and finish that of the proposition.

\begin{proof}[Proof of Proposition~\ref{proposition global entropic repulsion}]
    Lemma~\ref{lem:Ioan_numerator} together with the estimation~\eqref{eq: lower bound denominator preuve globrep} yield that for any edge-regular $\mathsf{Ext}$, the following holds:
    \begin{equ}
        \frac{\phi_{\Ext^c}^0 \left[ \mathsf{GlobRep}^c,\nonint, \con, \mathsf{Diam}\right]}
    {\phi_{\Ext^c}^0 \left[\nonint, \con, \mathsf{Diam}\right]} \leq \tfrac{C}{\chi}(T_2-T_1)^{-\beta}.
    \end{equ}
By edge-regularity of $\Ext$, we know that $(T_2-T_1) \geq n-2n^{1-\eps}$, so that $(T_2-T_1)^{-\beta} = n^{-\beta}(1+o(1))$. Thus, plotting this estimation into~\eqref{equation conditionnement trajext preuve globrep}, we proved that
\begin{equ}
    \phi\left[ \mathsf{GlobRep}^c\vert\nonint, \con, \mathsf{EdgeReg} \right] \leq \tfrac{C}{\chi}n^{-\beta},
\end{equ}
which concludes the proof of Proposition~\ref{proposition global entropic repulsion}.
\end{proof}

We now turn to the proof of Lemma~\ref{lem:Ioan_numerator}.
Fix some edge-regular $\mathsf{Ext}$.
When $\nonint, \con$ and $\mathsf{Diam}$ occur, write $\Gamma_i$ for the top-most path of the cluster of $\mathscr{X}_i$. This is a path of open edges contained in $\mathsf{Strip}$ (due to $\mathsf{Diam}$) connecting $\mathscr{X}_i$ to $\mathscr{Y}_i$. For any such path, write $\partial \Gamma_i$ for all the edges of $\mathsf{Strip}$ adjacent to and above $\Gamma_i$. The strategy of proof follows the following pattern: first we are going to upper bound the probability of the event $\lbrace \Gamma = \gamma \rbrace$ by the probability of a suitable event in the product measure, and then in a second step conclude using Ornstein--Zernike theory to estimate the probability of this new event in the product measure. The first step is given by the following statement, which constitutes the core of the proof. 

\begin{Lemma}\label{lem: first step Ioan numerator} For any possible realization $\gamma$ of the random set of paths $\Gamma$ under the event $\con, \nonint$, for any edge-regular $\mathsf{Ext}$
     \begin{equ}
         \phi_{\mathsf{Ext}^c}^0 \left[ \Gamma = \gamma ,\nonint, \con, \mathsf{Diam}\right] 
      \leq (1-\exp(-n^\eps))^{-r}  (\phi_{\mathsf{Ext}^c}^0)^{\otimes r} \Big[ \bigcap_{i=1}^r\lbrace \gamma_i \text{  is open}, \partial \gamma_i \text{ is closed}\rbrace\Big ].
     \end{equ}
\end{Lemma}
\begin{Rem}
    The careful reader might think that this lemma is in contradiction with the lower bound given by Lemma~\ref{lemme estimation fine proba con ni}. However, observe that even if the events $\mathsf{Diam}$ and $\lbrace \Gamma = \gamma \rbrace$ do imply the events $\con$ and $\nonint$ in the measure $\phi^0_{\mathsf{Ext}^c}$, it is not the case for the measure $(\phi^0_{\EXT^c})^{\otimes r}$.
\end{Rem}
\begin{proof}[Proof of Lemma~\ref{lem: first step Ioan numerator}]
    Fix $\mathsf{Ext}$ an edge-regular exterior configuration. Fix $\gamma$ a possible realization of the random set of paths $\Gamma$ under the events $\nonint, \con, \mathsf{Diam}$. Observe that the paths $\gamma$ satisfy the following set of properties: 
    \begin{equ}
        \gamma_i : \mathscr{X}_i \leftrightarrow \mathscr{Y}_i \qquad\text{and}\qquad (\gamma_i \cup \partial \gamma_i) \cap (\gamma_j \cap \partial \gamma_j) = \emptyset 
        \end{equ}
        for any $1\leq i \neq j \leq r$. Now we observe that:

\begin{eqnarray*}
    \phi_{\mathsf{Ext}^c}^0 \left[ \Gamma = \gamma ,\nonint, \con, \mathsf{Diam}\right] 
    &=& \prod_{i=1}^r \phi^0_{\mathsf{Ext}^c} \Big[ \Gamma_i = \gamma_i, \mathsf{Diam_i} \vert  \bigcap_{k=1}^{i-1} \lbrace \Gamma_k= \gamma_k, \mathsf{Diam}_k \rbrace\Big]  \\
    &\leq& \prod_{i=1}^r \phi^0_{\mathsf{Ext}^c} \Big[ \gamma_i \text{ is open}, \partial \gamma_i \text{ is closed} \big\vert  \bigcap_{k=1}^{i-1} \lbrace \Gamma_k= \gamma_k, \mathsf{Diam}_k \rbrace\Big].
\end{eqnarray*}
Our goal is now to prove that for any $i \in \lbrace 1, \dots, r \rbrace$,
\begin{multline}\label{equ: equation a prouver majoration chemin par mesure produit}
    \phi^0_{\mathsf{Ext}^c} \Big[ \gamma_i \text{ is open}, \partial \gamma_i \text{ is closed} \vert  \bigcap_{k=1}^{i-1} \lbrace \Gamma_k= \gamma_k, \mathsf{Diam}_k \rbrace\Big] \\ \leq (1-\exp(-n^\eps))^{-1}\phi^0_{\mathsf{Ext}^c}\left[ \gamma_i \text{ is open}, \partial \gamma_i \text{ is closed} \right].
\end{multline}
Indeed, assuming that the above inequality is true, we will have proved that
\begin{align*}
    \phi_{\mathsf{Ext}^c}^0 \big[ \Gamma = \gamma ,\nonint, \con, \mathsf{Diam}&\big] \\ &\leq \prod_{i=1}^r (1-\exp(-n^\eps))^{-1}
\phi^0_{\mathsf{Ext}}\left[ \gamma_i \text{ is open}, \partial \gamma_i \text{ is closed} \right]\\
&= (1-\exp(-n^\eps))^{-r}(\phi_{\mathsf{Ext}^c}^0)^{\otimes r} \left[ \bigcap_{i=1}^r\lbrace \gamma_i \text{  is open}, \partial \gamma_i \text{ is closed}\rbrace\right].
\end{align*}
We thus focus on ~\eqref{equ: equation a prouver majoration chemin par mesure produit}
Observe that it is obviously true for $i=1$. We then focus on the case $i >1$. We write:
\begin{multline}\label{equ: decomposition chemin ouvert et ouvert sachant fermé}
    \phi^0_{\mathsf{Ext}^c} \Big[ \gamma_i \text{ is open}, \partial \gamma_i \text{ is closed} \vert  \bigcap_{k=1}^{i-1} \lbrace \Gamma_k= \gamma_k, \mathsf{Diam}_k \rbrace\Big] = \\ \phi^0_{\mathsf{Ext}^c} \Big[ \gamma_i \text{ is open}\vert  \bigcap_{k=1}^{i-1} \lbrace \Gamma_k= \gamma_k, \mathsf{Diam}_k \rbrace\Big]\\ \times\phi^0_{\mathsf{Ext}^c} \Big[ \partial \gamma_i \text{ is closed} \vert  \gamma_i \text{ is open}, \bigcap_{k=1}^{i-1} \lbrace \Gamma_k= \gamma_k, \mathsf{Diam}_k \rbrace\Big].
\end{multline}
The first factor is easy to upper bound, as the conditioning on $\bigcap_{k=1}^{i-1} \lbrace \Gamma_k= \gamma_k, \mathsf{Diam}_k \rbrace$ enforces negative information for $\gamma_i$ to be open. Indeed, let us explore the clusters of the points $(T_1,\mathscr{X}_1), \dots, (T_1,\mathscr{X}_{i-1})$ together with their boundaries, and let us call $\mathsf{Expl}$ the set of explored edges. Because of the conditioning on $\Diam_1, \dots, \Diam_{i-1}$ and of the disjointedness of the paths of $\gamma$, it is easy to observe that $(\mathsf{Expl}\cup\Ext) \cap \gamma_i = \emptyset$. Moreover, the measure induced on the complement of these clusters by this exploration procedure is clearly $\phi^0_{(\mathsf{Expl}\cup\mathsf{Ext})^c}$. Since by~\eqref{Comparaison boundary conditions}, one has 
\begin{equ}
    \phi_{(\mathsf{Ext}\cup\mathsf{Expl})^c}^0\left[\gamma_i \text{ is open} \right] \leq \phi_{\mathsf{Ext}^c}^0\left[\gamma_i \text{ is open} \right],
\end{equ}
we get that 
\begin{equ}\label{equ: majoration premier terme proba conditionnelle chemin ouvert}
    \phi^0_{\mathsf{Ext}^c} \Big[ \gamma_i \text{ is open}\vert  \bigcap_{k=1}^{i-1} \lbrace \Gamma_k= \gamma_k, \mathsf{Diam}_k \rbrace\Big] \leq \phi_{\mathsf{Ext}}^0\left[\gamma_i \text{ is open} \right].
\end{equ}

Upper bounding the second factor is however slightly more subtle, since the the boundary conditions enforced by the conditioning \emph{a priori} favour $\partial \gamma_i$ to be closed. As previously, we explore the clusters of $\mathscr{X}_1, \dots, \mathscr{X}_{i-1}$ together with their boundaries, and call $\mathsf{Expl}$ the explored edges. By the same argument as before, those edges are disjoint of $\mathsf{Ext}$ and of $\gamma_i \cup \partial \gamma_i$, and as previously the measure induced by this exploration is $\phi^0_{(\mathsf{Expl}\cup\mathsf{Ext})^c}$. Let us also call $\mathsf{Strip}^\uparrow(\gamma_i)$ the set of vertices of $\mathsf{Strip}$ that lie above the path $\gamma_i$. In particular observe that $\partial \gamma_i \subset \mathsf{Strip}^\uparrow(\gamma_i)$.

We introduce the following events 
\begin{equ}
    \mathcal{H}^{\mathsf{L}} = \left\lbrace  \text{There exists an open dual path linking } (T_1, \mathscr{X}_i)^* \text{ to } \infty \text{ in }(\mathsf{Strip}^{c})^* \right\rbrace 
\end{equ}
and 
\begin{equ}
    \mathcal{H}^{\mathsf{R}} = \left\lbrace  \text{There exists an open dual path linking } (T_2, \mathscr{Y}_i)^* \text{ to } \infty \text{ in }(\mathsf{Strip}^{c})^* \right\rbrace. 
\end{equ}
We shall rely on the following claim, whose proof relies on standard percolation arguments.
\paragraph{\textbf{Claim 4.2}.}\label{claim 2}There exists some $c>0$ such that
\begin{equ}
    \phi^{0}_{\mathsf{Ext}^c}\left[ \mathcal{H}^\mathsf{L}\cap\mathcal{H}^\mathsf{R}\vert\gamma_i\text{ is open}  \right] \geq 1 - \exp\left(-cn^\eps\right).
\end{equ}

\begin{proof}[Proof of Claim~\ref{claim 2}] The proof of this statement relies on classical properties of the subcritical regime of the random-cluster measure, and essentially on the sharpness of the phase transition (first proved in~\cite{duminilbeffara} for the case of the square lattice).

% Figure environment removed
 Since the event $\lbrace \gamma_i \text{ is open}\rbrace$ is measurable with respect to the edges of $\Strip$, while $\mathcal{H}^\mathsf{L}\cap\mathcal{H}^\mathsf{R}$ is measurable with respect to the edges of $\Strip^c$, we explore all the configuration on $\Strip$ and call $\xi$ the induced boundary condition on $\Strip^c$. The measure induced by this exploration is then $\phi^{0/\xi}_{(\Ext\cup\mathsf{Strip})^c}$, where $0/\xi$ is the boundary condition made of $\xi$ on the vertices of the boundary of the strip and free on the boundary of $\Ext$.
We prove that $\phi^{0/\xi}_{(\Ext\cup\mathsf{Strip})^c}\left[\mathcal{H}^\mathsf{L}\right] \geq 1 - \exp(-cn^\eps)$ and the claim follows by union bound. 
Observe that~\eqref{Comparaison boundary conditions} implies that 
    \begin{multline*}
        \phi^{0/\xi}_{(\Ext\cup\mathsf{Strip})^c}\left[\mathcal{H}^\mathsf{L}\right]  \\ 
        \geq\phi^\xi_{\mathsf{Strip}^c}\left[ \partial(\Ext_i)^* \text{ is connected to }\infty\text{ by a dual open path lying in }(\lbrace T_1\rbrace \times \Z)^*  \right].
    \end{multline*}
    Now, we observe that when $\partial(\Ext_i)^*$ is \emph{not} connected to infinity by an open dual path, then there must exist at least one index $k \geq 0$ such that $(-k, x_i)$ is connected to the half-plane $\lbrace T_1 \rbrace\times \Z$ by a (primal) open path lying in the half-plane $(-\infty, T_1] \times \Z$ (see Figure~\ref{fig: figure illustration preuve appendix}). Thus, we upper bound:
    \begin{equ}\label{equ: equ preuve appendice finale somme}
         \phi^{0/\xi}_{(\Ext\cup\mathsf{Strip})^c}\left[(\mathcal{H}^\mathsf{L})^c\right] \leq \sum_{k\geq 0}\phi^\xi_{\mathsf{Strip}^c} \Big[ (-k, x_i) \leftrightarrow \lbrace T_1\rbrace\times \Z  \Big].
    \end{equ}
Let us call $\Lambda := \Lambda_{T_1 + k}((-k,x_i))$ the $L^1$ box of radius $T_1+k$ centered at the vertex $(-k,x_i)$, and observe that if $(-k,x_1)$ is connected to the half-plane $\lbrace T_1 \rbrace \times \Z$, then the center of $\Lambda$ has to be connected to its boundary. We then write that, due to~\eqref{Comparaison boundary conditions}
\begin{equ}
    \phi^\xi_{\mathsf{Strip}^c} \Big[ (-k, x_i) \leftrightarrow \lbrace T_1\rbrace\times \Z  \Big] \leq \phi^1_{\Lambda}\big[ (-k,x_i) \leftrightarrow \partial\Lambda \big].
\end{equ}
Now it is a consequence of the celebrated sharpness of the phase transition~\cite{DuminilCopin2017SharpPT} that this probability decays exponentially fast in $T_1 + k$, that is that there exists some $c>0$ such that
\begin{equ}
    \phi^1_{\Lambda}\big[ (-k,x_i) \leftrightarrow \partial\Lambda \big] \leq \e^{-c(T_1+k)}.
\end{equ}
Summing over $k\geq 0$ and plotting this in~\eqref{equ: equ preuve appendice finale somme} yields that for some finite $C>0$:
\begin{equ}
    \phi^{0/\xi}_{(\Ext\cup\mathsf{Strip})^c}\left[(\mathcal{H}^\mathsf{L})^c\right] \leq C\e^{-cT_1}.
\end{equ}
We conclude observing that $T_1 \geq \frac{n^\eps}{2\delta}$ due to the cone confinement property. 
\end{proof}




% Due to~\eqref{Comparaison boundary conditions}, one has the following stochastic domination:
% \begin{equ}
%     \phi^0_{(\mathsf{Expl}\cup\mathsf{Ext})^c}\left[ ~\cdot\vert\gamma_i \text{ is open} \right] \preccurlyeq \phi^0_{\mathsf{Ext}^c}\left[ ~\restriction{\cdot}{(\mathsf{Expl}\cup\mathsf{Ext})^c}\vert\gamma_i \text{ is open} \right]
% \end{equ}
% Let us now consider the increasing coupling between those two measures, \emph{ie} the probability measure $\Phi$ on $\lbrace 0,1\rbrace^{E(\Z^2)} \times \lbrace 0,1\rbrace^{E(\Z^2)}$ such that, if $(\om_1,\om_2)$ is sampled according to $\Phi$, then $\om_1 \sim \phi^0_{(\mathsf{Expl}\cup\mathsf{Ext})^c}\left[ ~\cdot\vert\gamma_i \text{ is open} \right]$, $\om_2 \sim \phi^0_{\mathsf{Ext}}\left[ ~\restriction{\cdot}{(\mathsf{Expl}\cup\mathsf{Ext})^c}\vert\gamma_i \text{ is open} \right]$ and $\om_1 \leq \om_2$ almost surely. The proof of the existence of such a coupling is standard, we refer to~\cite{grimmetttherandomclustermodel}. 

\paragraph{}Let us end the proof of Lemma~\ref{lem: first step Ioan numerator} using Claim~\ref{claim 2}. We rely on the following stochastic domination:
\begin{equ}
    \phi_{\Ext^c}^0\left[\restriction{\cdot}{\Strip^\uparrow(\gamma_i)} \vert \mathcal{H}^\mathsf{L}, \mathcal{H}^\mathsf{R}, \gamma_i\text{ is open}\right] \preccurlyeq \phi^0_{(\mathsf{Expl}\cup\Ext)^c}\left[\restriction{\cdot}{\Strip^\uparrow(\gamma_i)} \vert \gamma_i \text{ open}\right].
\end{equ}
Such a statement relies on standard percolation arguments: we briefly explain how to prove it. Let us consider the measure $\phi_{\Ext^c}^0\big[\cdot \vert \mathcal{H}^\mathsf{L}, \mathcal{H}^\mathsf{L}, \gamma_i\text{ is open}\big]$, and in that measure explore (by some arbitrary procedure) a dual open path $\Gamma^{*, \mathsf{L}}$ (resp. $\Gamma^{*, \mathsf{R}})$) linking $(T_1, \mathscr{X}_i)^*$ (resp. $(T_2, \mathscr{Y}_i)^*$) to infinity outside of the strip. Then one can see that the edges of $\Gamma^\mathsf{L}\cup\gamma_i\cup\Gamma^\mathsf{R}$ separate $\Z^2$ in two infinite subdomains, say $\mathcal{D}^{\uparrow}$ and $\mathcal{D}^\downarrow$, such that $\Strip^\uparrow(\gamma_i) \subset \mathcal{D}^\uparrow$ and $\Ext\cup\mathsf{Expl} \subset \mathcal{D}^{\downarrow}$ (see Figure~\ref{fig: figure illustration Hi} for illustration). Thus, the measure on $\mathcal{D}^\uparrow$ induced by this exploration is nothing but $\phi^{0/1}_{\mathcal{D}^\uparrow}$, where the boundary condition is wired along the edges of $\gamma_i$, and free elsewhere. It is a standard instance of "pushing the boundary conditions" (or~\eqref{Comparaison boundary conditions}) that this measure is stochastically dominated by $\phi^{0/1}_{(\mathsf{Expl}\cup\Ext)^c)}[\restriction{\cdot}{\mathcal{D^{\uparrow}}}]$ (where the boundary condition is wired along the edges of $\gamma_i$ and free elsewhere), which concludes the proof of that statement.
% Figure environment removed
% \paragraph{}Let us end the proof of Lemma~\ref{lem: first step Ioan numerator} using Claim~\ref{claim 2}. Indeed, let us work under the event $\om_2 \in \mathcal{H}^\mathsf{L}\cap\mathcal{H}^\mathsf{R}$. In $\om_2$, explore an infinite open dual path $\Gamma^{*, \mathsf{L}}$ (resp. $\Gamma^{*,\mathsf{R}}$) linking $(T_1, \mathscr{X}_i)^*$ (resp. $(T_2, \mathscr{Y}_i)^*$) to infinity. Then it is a geometrical fact (see Figure~\ref{fig: figure illustration Hi}) that the edges of $\Gamma^\mathsf{L} \cup \gamma_i \cup \Gamma^\mathsf{R}$ separate $\Z^2$ in two infinite domains, say $\mathcal{D}^{\uparrow}$ and $\mathcal{D}^{\downarrow}$, such that 
% \begin{itemize}
%     \item $ \partial \gamma_i \subset E(\mathcal{D}^\uparrow)$
%     \item $ \mathsf{Expl} \subset E(\mathcal{D}^\downarrow)$.
% \end{itemize}
% Thus, by~\eqref{equation smp},
% \begin{equ}
%     \Phi\left[ \partial \gamma_i \text{ is closed in }\om_2 \vert \om_2 \in \mathcal{H}^\mathsf{L}\cap\mathcal{H}^\mathsf{R} \right] = \Phi\left[ \phi_{\mathcal{D}^\uparrow}^\xi\left[ \partial \gamma_i\text{ is closed in }\om_2 \right] \vert \om_2 \in \mathcal{H}^\mathsf{L}\cap\mathcal{H}^\mathsf{R}  \right],
% \end{equ}
% with $\xi$ denoting the mixed boundary condition with closed edges on $\Gamma^\mathsf{L}\cup\Gamma^\mathsf{R}$ and open edges on $\gamma_i$. 

% But observe that by construction of the coupling, since $\om_2 \geq \om_1$, $\om_1$ necessarily induces the same boundary conditions on $\mathcal{D}^\uparrow$, and thus $\om_1$ and $\om_2$ do coincide on $\mathcal{D}^\uparrow$. Hence,
% \begin{align*}
%     \phi^0_{(\mathsf{Expl}\cup\mathsf{Ext})^c}&\left[\partial \gamma_i \text{ is closed }\right] - \phi^0_{(\mathsf{Ext})^c}\left[\partial \gamma_i  \text{ is closed }\right] \\ &= \Phi\left[ \partial \gamma_i \text{ is closed } \text{ in } \om_1 \text{ but not in } \om_2 \right] \\
%     &\leq\Phi\left[\partial \gamma_i \text{ is closed } \text{ in } \om_1, \om_2 \notin \mathcal{H}^\mathsf{L}\cap\mathcal{H}^\mathsf{R}  \right]  \\
%     &= \phi^0_{(\mathsf{Expl}\cup\mathsf{Ext})^c}\left[\partial \gamma_i \text{ is closed } \right]\Phi\left[ \om_2 \notin \mathcal{H}^\mathsf{L}\cap\mathcal{H}^\mathsf{R} \vert \partial \gamma_i \text{ is closed } \text{ in } \om_1  \right].
% \end{align*}
% We obtain that 
% \begin{equ}
%     \left\vert 1 - \frac{\phi^0_{(\mathsf{Ext})^c}\left[\partial \gamma_i  \text{ is closed }\right]}{ \phi^0_{(\mathsf{Expl}\cup\mathsf{Ext})^c}\left[\partial \gamma_i \text{ is closed }\right]} \right\vert \leq \Phi\left[ \om_2 \notin \mathcal{H}^\mathsf{L}\cap\mathcal{H}^\mathsf{R} \vert \partial \gamma_i \text{ is closed } \text{ in } \om_1  \right].
% \end{equ}
% The end of the argument follows by Claim~\ref{claim 2}. Indeed, since $\mathcal{H}^\mathsf{L}\cap \mathcal{H}^\mathsf{R}$ is measurable with respect to the edges of $\mathsf{Strip}^c$, and the event that $\partial \gamma_i$ is closed is measurable with respect to the edges of $\mathsf{Strip}$, one can explore the configuration $\om_2$ sampled according to the second marginal of $\Phi\left[ ~\cdot~\vert~\partial \gamma_i \text{ is closed in }\om_1\right]$ on the set $\mathsf{Strip}$. It yields boundary conditions on the boundary of $\mathsf{Strip}$: by Claim~\ref{claim 2} we can upper the probability that $\om_2$ does not satisfy $\mathcal{H}^\mathsf{L}\cap \mathcal{H}^\mathsf{R}$ uniformly in those boundary conditions. We thus established that
% \begin{equ}
%     \left\vert 1 - \frac{\phi^0_{(\mathsf{Ext})^c}\left[\partial \gamma_i  \text{ is closed }\right]}{ \phi^0_{(\mathsf{Expl}\cup\mathsf{Ext})^c}\left[\partial \gamma_i \text{ is closed }\right]} \right\vert \leq \exp(-cn^\eps), 
% \end{equ}
Thus we obtain that 
\begin{equ}
    \phi^0_{(\mathsf{Expl}\cup\Ext)^c}\big[ \partial \gamma_i \text{ is closed}\vert\gamma_i \text{ is open} \big] \leq \phi^0_{\Ext^c}\big[ \partial \gamma_i \text{ is closed}\vert\gamma_i \text{ is open}, \mathcal{H}^\mathsf{L}, \mathcal{H}^\mathsf{R} \big].
\end{equ}
Using the elementary fact that
\begin{equ}
    \phi^0_{\Ext^c}\big[ \partial \gamma_i \text{ is closed}\vert\gamma_i \text{ is open}, \mathcal{H}^\mathsf{L}, \mathcal{H}^\mathsf{R} \big] \leq \frac{\phi^0_{\Ext^c}\left[\partial \gamma_i \text{ is closed}\vert\gamma_i \text{ is open}\right]}{\phi^0_{\Ext^c}\big[\mathcal{H}^\mathsf{L}, \mathcal{H}^\mathsf{R}\vert \gamma_i \text{ is open}\big]}
\end{equ}
and using Claim~\ref{claim 2}, we obtain that
\begin{equ}
     \phi^0_{\Ext^c}\big[ \partial \gamma_i \text{ is closed}\vert\gamma_i \text{ is open}, \mathcal{H}^\mathsf{L}, \mathcal{H}^\mathsf{R} \big] \leq (1- \exp(-cn^\eps))^{-1}\phi^0_{\Ext^c}\left[\partial \gamma_i \text{ is closed}\vert\gamma_i \text{ is open}\right],
\end{equ}
and thus
\begin{multline}
    \phi^0_{\mathsf{Ext}^c} \Big[ \partial \gamma_i \text{ is closed} \vert  \gamma_i \text{ is open}, \bigcap_{k=1}^{i-1} \lbrace \Gamma_k= \gamma_k, \mathsf{Diam}_k \rbrace\Big] \\ \leq (1-\exp(-cn^\eps))^{-1}  \phi^0_{\mathsf{Ext}^c} \Big[ \partial \gamma_i \text{ is closed}\vert\gamma_i \text{ is open} \Big].
\end{multline}
We are now done: plotting this input together with the upper bound given by ~\eqref{equ: majoration premier terme proba conditionnelle chemin ouvert} inside the decomposition ~\eqref{equ: decomposition chemin ouvert et ouvert sachant fermé} yields that 
\begin{multline*}
    \phi^0_{\mathsf{Ext}^c} \Big[ \gamma_i \text{ is open}, \partial \gamma_i \text{ is closed} \vert  \bigcap_{k=1}^{i-1} \lbrace \Gamma_k= \gamma_k, \mathsf{Diam}_k \rbrace\Big] \\ \leq (1-\exp(-cn^\eps) )^{-1}\phi^0_{\mathsf{Ext}^c}\left[ \gamma_i \text{ is open}, \partial \gamma_i \text{ is closed} \right]
\end{multline*}
which concludes the proof as explained previously.
\end{proof}
% As explained before, under $\rep$ we condition on $T_1$ and $T_2$ satisfying the event of~\eqref{condition T1 et T2 preuve finale}. 

% We first use the same exploration argument as in the proof of Proposition~\ref{Proposition convergence mesure produit}. We import from this proof the definition of $\mathsf{TrajExt}_{[T_1,T_2]}$, and for $C \in \mathsf{TrajExt}_{[T_1,T_2]}$, we import the definitions of $\EXT = \mathsf{Ext}, \nonint, \con_{\mathscr{X}, \mathscr{Y}}, \mathscr{X}(C)$ and $\mathscr{Y}(C)$. As before, we write:


% \begin{multline}\label{equation conditionnement trajext preuve globrep}
%     \phi\left[\mathsf{GlobRep}^c \vert \nonint, \con, \rep\right] = \\ \sum_{C\in\mathsf{TrajExt}_{[T_1,T_2]}} \phi \left[ \mathsf{GlobRep}^c \vert \nonint, \con_{\mathscr{X}, \mathscr{Y}},\EXT = \mathsf{Ext} \right]\phi\left[\EXT = \mathsf{Ext} ~\big\vert\con, \nonint\right].
% \end{multline}

% Moreover, as in the proof of Proposition~\ref{Proposition convergence mesure produit}, because of Lemma~\ref{lemme diamond confinement mesure conditionnee}, we know that summing over the $\mathscr{X}(C)$'s and $\mathscr{Y}(C)$'s satisfy the following set of conditions


% only results in a loss of $o(1)$ in ~\eqref{equation conditionnement trajext preuve globrep}. 
% Fix $C\in \mathsf{TrajExt}_{[T_1,T_2]}$ satisfying~\eqref{equation condition ecartement x et y preuve vraie mesure}. We are going to use a second exploration argument to upper bound the term $\phi\left[\mathsf{GlobRep}^c \vert \nonint, \con_{\mathscr{X}, \mathscr{Y}}, \EXT = \mathsf{Ext}\right]$. To properly state it, we need a bit of geometrical notations. First of all, observe that $\Z^2\setminus \mathsf{Strip}_{[T_1,T_2]}$ has two connected components that we call $\mathsf{Strip}^L_{\EXT, T_1,T_2}$ (the left one) and $\mathsf{Strip}^R_{\EXT, T_1,T_2}$ (the right one). Now, let $\gamma^i = (\gamma^i_0, \dots, \gamma^i_\ell)$ be a path of edges of $E(\mathsf{Strip}_{[T_1,T_2]})$ such that two consecutive edges share a common endpoint, $(T_1,\mathscr{X}(C)_i)$ is an endpoint of $\gamma^i_0$ and $(T_2, \mathscr{Y}(C)_i)$ is an endpoint of $\gamma^i_\ell$. Such a path naturally splits the strip $\mathsf{Strip}_{[T_1,T_2]}$ in two infinite connected components, one that we call $\mathsf{Ext^T}(\gamma^i)$ (the top one) and the other that we call $\mathsf{Ext^B}(\gamma^i)$ (the bottom one). We define the two exterior edge boundaries of the path $\gamma^i $ by 
% \begin{equ}
%     \partial^{\#}\gamma^i = \left\lbrace e = \lbrace x,y \rbrace \in E(\mathsf{Strip}_{[T_1,T_2]}), x \text{ is the endpoint of an edge of }\gamma^i, y \in \EXT^{\#}(\gamma^i) \right\rbrace,
% \end{equ}
% for $\# \in \lbrace T,B \rbrace.$ Finally we define the following event
% \begin{equ}
%     \mathcal{N}(\gamma^i) = \left\lbrace \gamma^i \text{ is open }, \partial^B\gamma^i \text{ is closed } \right\rbrace.
% \end{equ}
% By exploring the bottom-most path of the clusters of the $(T_1, \mathscr{X}(C)_i)$ under the event \\ $\con_{\mathscr{X}, \mathscr{Y}}, \nonint$ we can associate to each cluster a unique path $\gamma^i$ joining $(T_1, \mathscr{X}(C)_i)$ to $(T_2, \mathscr{Y}(C)_i)$ inside the strip $\mathsf{Strip}_{[T_1,T_2]}$ (remember that $T_1$ and $T_2$ are synchronized renewals) such that $\mathcal{N}(\gamma^i)$ occurs. Since we work under the event $\nonint$, these paths do not share any edge in common. Moreover, if $\mathsf{GlobRep}^c$ occurs, then by the diamond confinement property stated in Lemma~\ref{lemme diamond confinement mesure conditionnee}, a pair of two such paths must have a Euclidean distance inferior to $\log^2 n + 2\log^2 n = 3\log^2 n$. 

% Let us call $\mathcal{P}$ the set of such path we are going to sum over:

% \begin{equ}
%     \mathcal{P} = \left\lbrace (\gamma^1, \dots, \gamma^r), \begin{cases} \gamma^i:(T_1,\mathscr{X}(C)_i) \overset{\mathsf{Strip}_{[T_1,T_2]}}{\longleftrightarrow} (T_2,\mathscr{Y}(C)_i) &\forall 1 \leq i\leq r, \\ \gamma^i\cap \gamma^j=\emptyset & \forall 1\leq i \neq j \leq r,\\  \exists 1 \leq j_1<j_2\leq r, d(\gamma^{j_1}, \gamma^{j_2})\leq 3\log^2 n\end{cases} \right\rbrace.
% \end{equ}

% Thus we obtain that 
% \begin{multline}\label{equation a raccourcir}
%     \phi\left[\mathsf{GlobRep}^c \vert \con_{\mathscr{X}, \mathscr{Y}},\nonint,\EXT = \mathsf{Ext} \right] \leq \\ \sum_{(\gamma^1, \dots, \gamma^r)\in \mathcal{P}}\phi\left[\mathcal{N}(\gamma^1), \dots, \mathcal{N}(\gamma^r)\vert \con_{\mathscr{X}, \mathscr{Y}}, \nonint,\EXT = \mathsf{Ext}\right].
% \end{multline}

% Let us fix some $(\gamma^1, \dots, \gamma^r) \in \mathcal{P}$. We observe that:

% \begin{equ}
%     \phi\left[\mathcal{N}(\gamma^1), \dots, \mathcal{N}(\gamma^r) \vert \con_{\mathscr{X}, \mathscr{Y}}, \nonint,\EXT = \mathsf{Ext} \right] = \frac{\phi\left[\mathcal{N}(\gamma^1),\dots, \mathcal{N}(\gamma^r) \vert \EXT = \mathsf{Ext}\right]}{\phi\left[\con_{\mathscr{X}, \mathscr{Y}}, \nonint~\big\vert \EXT = \mathsf{Ext}\right]}
% \end{equ}

% We start by bounding the numerator of this quantity. We decompose:
% \begin{equ}\label{equation decomposition intersection preuve finale}
%     \phi\left[ \mathcal{N}(\gamma^1), \dots, \mathcal{N}(\gamma^r) \vert \EXT = \mathsf{Ext}\right] = \prod_{i=1}^r \phi\left[\mathcal{N}(\gamma^i)\vert \mathcal{N}(\gamma^{i+1}), \dots, \mathcal{N}(\gamma^r),\EXT = \mathsf{Ext}\right]
% \end{equ}
% where the $i=r$ term has to be understood simply as $\phi\left[\mathcal{N}(\gamma^r)\vert\EXT = \mathsf{Ext}\right]$.

% The heuristic for the rest of the proof is the following: we would like to argue that under the measure $\phi\left[~\cdot\vert \EXT = \mathsf{Ext}\right]$, the conditioning on $\left\lbrace \mathcal{N}(\gamma^{i+1}), \dots, \mathcal{N}(\gamma^r) \right\rbrace$ brings negative information for $\gamma^i$ to be open - because of the closed edges of $\partial^B \gamma^{i+1}$ lying above $\gamma_i$. However there is one small obstruction to this argument: indeed one of the $\gamma^{i+1}, \dots, \gamma^r$ that are already open could create an arc travelling outside of the strip $\mathsf{Strip}_{[T_1,T_2]}$ and wiring vertices close to $(T_1,\mathscr{X}(C)_i)$ or $(T_1, \mathscr{Y}(C)_i)$, then favouring $\gamma^i$ to be open (see Figure~\ref{figure illustration Hi})

% We evict this obstruction by introducing the following events: for any $1 \leq i \leq r-1$, let $\mathcal{H}^L_i$ and $\mathcal{H}^R_i$ be the events:
% \begin{equ}\label{definition Hi}
%     \mathcal{H}^L_i = \big\lbrace \exists \gamma^*: (\partial C^L_{i,\EXT})^* \goes{\left( \mathsf{Strip}^\mathsf{L}_{\EXT,T_1,T_2}\right)^*}{}{ (\partial C^L_{i+1,\EXT})^*} \text{ open} \big\rbrace
% \end{equ}
% and
% \begin{equ}
%     \mathcal{H}^R_i = \big\lbrace \exists \gamma^*: (\partial C^R_{i,\EXT})^* \goes{\left( \mathsf{Strip}^\mathsf{R}_{\EXT,T_1,T_2}\right)^*}{}{ (\partial C^R_{i+1,\EXT})^*} \text{ open} \big\rbrace
% \end{equ}
% Remember that the superscript $*$ denotes the dual lattice. 
% For $i=r$, we slightly change the definition of $\mathcal{H}^{\#}_r$ in setting
% \begin{equ}
%     \mathcal{H}^L_r =   \left\lbrace \exists \gamma^*: (\partial C^L_{r,\EXT})^* \goes{\left( \mathsf{Strip}^\mathsf{L}_{\EXT,T_1,T_2}\right)^*}{}{\infty} \text{ open}\right\rbrace
% \end{equ}
% and
% \begin{equ}
%     \mathcal{H}^R_r =   \left\lbrace \exists \gamma^*: (\partial C^R_{r,\EXT})^* \goes{\left( \mathsf{Strip}^\mathsf{R}_{\EXT,T_1,T_2}\right)^*}{}{\infty} \text{ open}\right\rbrace.
% \end{equ}

% We claim that the event $\bigcap_{\# \in \lbrace L,R \rbrace} \bigcap_{i=1}^r \mathcal{H}^{\#}_i$ occurs with exponentially large probability: 

% \paragraph{\textbf{Claim 2}.} When $n$ is sufficiently large, uniformly in $C$ satisfying the conditions~\eqref{condition T1 et T2 preuve finale},
% \begin{equ}\label{equation claim proba Hi}
%     \phi\left[ \bigcap_{\# \in \lbrace L,R \rbrace} \bigcap_{i=1}^r \mathcal{H}^{\#}_i \vert \EXT = \mathsf{Ext} \right] \geq 1 - 4p\exp(-n^{\eps/2}).
% \end{equ}
% This claim relies on classical percolation arguments, and we postpone it to the end of the proof.
% % Figure environment removed
% The interest of the introduction of these events lies on the following observation. Let us name the following events:
% \begin{equ}
%     \mathcal{N}_i = \bigcap_{j=i+1}^r \mathcal{N}(\gamma^j) ~~~~\text{   and   }~~~~ \mathcal{G}_i = \bigcap_{j=i+1}^r \bigcap_{\# \in \lbrace L,R\rbrace} \mathcal{H}^\#_j.
% \end{equ}
% Then, for $i \leq r$, under the measure $\phi\left[~ \cdot\vert\EXT = \mathsf{Ext}\right]$, under the event $\mathcal{G}_i \cap \mathcal{N}_i$, we can associate to any percolation configuration a unique set $\Gamma_C$ of closed edges of $\mathsf{Strip}^\mathsf{L}_{\EXT,T_1,T_2} \sqcup \mathsf{Strip}^\mathsf{R}_{\EXT,T_1,T_2}$ such that $\Gamma_C \sqcup \partial^B \gamma^{i+1}$ is \textit{blocking} from $\gamma^{i+1}\cup\dots\cup\gamma^r$ to $\gamma^i$. By this terminology, we mean that any path of edges going from $\gamma^{i+1}\cup\dots\cup\gamma^r$ to $\gamma^i$ has to cross an edge of $\left(\Gamma_C \cup \partial^B\gamma^{i+1}\right)^*$. The construction of $\Gamma_C$ is straightforward: it is sufficient to consider the rightmost (resp. leftmost) dual paths in the definition of the $\mathcal{H}^L_j$ (resp. $\mathcal{H}^R_j$) for $i\leq j$ and to consider $\Gamma_C$ to be the union of all the primal edges intersecting these paths and of $\partial  C^L_{i,\EXT}$ and $\partial C^R_{i,\EXT}$. The existence of this set of edges is guaranteed by the occurrence of $\mathcal{H}_i$, and the fact that it is blocking is a simple geometrical observation. Furthermore the procedure above-described ensures that it is unique. Once again for an illustration, see Figure~\ref{figure illustration Hi}. 

% Now, we formally write:

% \begin{equ}
% \phi\left[ \mathcal{N}(\gamma^i) \vert \mathcal{G}_i,\mathcal{N}_i, \EXT = \mathsf{Ext}\right] = \phi\left[ \phi\left[\mathcal{N}(\gamma^i)\vert \Gamma_C, \mathcal{N}_i, \EXT = \mathsf{Ext} \right] \vert \mathcal{G}_i, \mathcal{N}_i, \EXT = \mathsf{Ext} \right], \end{equ}
% and we work on the term $\phi\left[ \mathcal{N}(\gamma^i) \vert \Gamma_C,\mathcal{N}_i,\EXT = \mathsf{Ext}\right]$.

% \begin{eqnarray*}
% \phi\left[ \mathcal{N}(\gamma^i) \vert \Gamma_C,\mathcal{N}_i,\EXT_C\right]  &=& \phi^{0}_{\Z^2 \setminus (\Gamma_C \sqcup \partial^B\gamma^{i+1})}\left[\mathcal{N}(\gamma^i)\vert \Gamma_C, \mathcal{N}_i, \EXT = \mathsf{Ext}\right] \\
% &=& \phi^{0}_{\Z^2 \setminus (\Gamma_C \sqcup \partial^B\gamma^{i+1})}\left[\mathcal{N}(\gamma^i)\vert \Gamma_C, \EXT = \mathsf{Ext}\right] \\
% &=& \phi^0_{\Z^2 \setminus (\Gamma_C \sqcup \partial^B\gamma^{i+1})}\left[ \gamma^i \text{ open} \vert \Gamma_C, \EXT = \mathsf{Ext}\right]\\&\times&\phi^0_{\Z^2 \setminus (\Gamma_C \sqcup \partial^B\gamma^{i+1})}\left[ \partial^B \gamma^i \text{ closed}\vert \Gamma_C, \EXT = \mathsf{Ext}, \gamma^i\text{ open}\right],
% \end{eqnarray*}
% where the first line comes from~\eqref{equation smp} and the second line comes from the fact that since $\Gamma_C \sqcup \partial^B\gamma^{i+1}$ is blocking from $\gamma^{i+1}, \dots, \gamma^{r}$, $\mathcal{N}(\gamma^i)$ is independent of $\mathcal{N}_i$ in the measure $\phi_{\Z^2\setminus(\Gamma_C\sqcup\partial^B\gamma^{i+1})}^0\left[~\cdot\vert \Gamma_C, \EXT = \mathsf{Ext}\right]$.

% Now, observe that because of~\eqref{Comparaison boundary conditions}, we have:

% \begin{equ}
%     \phi^0_{\Z^2\setminus(\Gamma_C\sqcup\partial^B\gamma^{i+1})} \left[ \gamma^i \text{open}\vert\Gamma_C, \EXT = \mathsf{Ext}\right] \leq \phi^{0}_{\Z^2\setminus\Gamma_C}\left[\gamma^i \text{open}\vert\Gamma_C, \EXT = \mathsf{Ext}\right],
% \end{equ}
% and
% \begin{multline}
%     \phi^0_{\Z^2 \setminus (\Gamma_C \sqcup \partial^B\gamma^{i+1})}\left[ \partial^B \gamma^i \text{ closed}\vert \Gamma_C, \EXT = \mathsf{Ext}, \gamma^i\text{ open}\right] = \\ \phi^0_{\Z^2\setminus\Gamma_C}\left[ \partial^B \gamma^i \text{ closed}\vert \Gamma_C, \EXT = \mathsf{Ext}, \gamma^i\text{ open}\right],
% \end{multline}
% because the set $\Gamma_C\sqcup\gamma^i$ is blocking from $\partial^B\gamma^{i+1}$ to $\partial^B \gamma^i$. Multiplying those two inputs yields 
% \begin{equ}
%     \phi\left[\mathcal{N}(\gamma^i)\vert \Gamma_C, \mathcal{N}_i,\EXT = \mathsf{Ext} \right] \leq \phi\left[ \mathcal{N}(\gamma^i)\vert \Gamma_C, \EXT = \mathsf{Ext}\right].
% \end{equ}
% Integrating this inequality with respect to the measure $\phi\left[~\cdot\vert \mathcal{G}_i,\mathcal{N}_i, \EXT = \mathsf{Ext}\right]$ yields
% \begin{equ}
%     \phi\left[\mathcal{N}(\gamma^i)\vert \mathcal{G}_i, \mathcal{N}_i, \EXT = \mathsf{Ext}\right] \leq \phi\left[\mathcal{N}(\gamma^i)\vert \mathcal{G}_i, \EXT = \mathsf{Ext}\right]
% \end{equ}
% Hence, coming back to~\eqref{equation decomposition intersection preuve finale}, and using~\eqref{equation claim proba Hi}, we infer that
% \begin{equ}\label{equation majoration chemins produits}
%     \phi\left[\mathcal{N}(\gamma^1),\dots,\mathcal{N}(\gamma^r)\vert\EXT = \mathsf{Ext}\right] \leq \phi^{\otimes r}\left[\mathcal{N}(\gamma^1),\dots,\mathcal{N}(\gamma^r)\vert\EXT = \mathsf{Ext}\right]\left(1+o(1)\right).
% \end{equ}

% For the sake of clarity, we drop the terms in $1+o(1)$ in our estimations. 
% Coming back to~\eqref{equation a raccourcir}, we proved that:

% \begin{equ}
%     \phi\left[\mathsf{GlobRep}^c \vert \con_{\mathscr{X}, \mathscr{Y}},\nonint,\EXT = \mathsf{Ext}\right] \leq \sum_{(\gamma^1, \dots, \gamma^r) \in \mathcal{P}} \frac{\phi^{\otimes r}\left[\mathcal{N}(\gamma^1), \dots, \mathcal{N}(\gamma^r)\vert \EXT = \mathsf{Ext}\right]}{\phi\left[\con_{\mathscr{X}, \mathscr{Y}},\nonint\vert \EXT = \mathsf{Ext}\right]}.
% \end{equ}


% Now we make the following observation. Under the measure $\phi^{\otimes r}$, it is still true that there can exist at most one $(\gamma^1, \dots, \gamma^r)$ such that $\lbrace \mathcal{N}(\gamma^1), \dots, \mathcal{N}(\gamma^r)\rbrace$ occurs. However, the occurrence of this event does not guarantee anymore that $\mathcal{C}_1, \dots, \mathcal{C}_r$ are non-intersecting (since we work under $\phi^{\otimes r}$). However, we are now able to work under the coupling $\Phi^{\otimes r}_{(T_1, \mathscr{X}(C)) \rightarrow (T_2, \mathscr{Y}(C))}\left[~\cdot\vert\EXT = \mathsf{Ext}\right]$. Remembering the definition of the \textit{synchronized} skeletons from Section~\ref{section independent system}, we observe that if there exists $(\gamma^1, \dots, \gamma^r) \in \mathcal{P}$ such that $\lbrace \mathcal{N}(\gamma^1), \dots, \mathcal{N}(\gamma^r)\rbrace$ occurs, then it must be the case that $\con$ occurs and that $ \check{\mathcal{S}}(\mathcal{C}) \in \mathcal{W}_{[T_1, T_2]}$ (indeed, all the renewals of the $r$ clusters are necessarily located on the paths $\gamma^1, \dots, \gamma^r$). Moreover because of the diamond confinement property of the clusters in the product measure, we see that one must have\footnote{As in the proof of Lemma~\ref{Entropic repulsion principle}, we use the elementary fact that the minimum of two piecewise linear function with the same slope change times must be achieved at a slope change time.} for any $\delta > 0$ and $n$ sufficiently large\footnote{Observe that the $n^\delta$ is way too large and should be replaced by a large multiple of $\log^2 n$, but is sufficient to apply Lemma~\ref{lemme repulsion globale non-synchronized RW}.}, 
% \begin{equ}
%   \check{\close}^\delta := \left\lbrace \min_{1\leq i\leq p-1}\min_{k \in \lbrace T_1, \dots,  T_2 \rbrace}\left| \check{\mathcal{S}}(\mathcal{C}_{i+1})(k) - \check{\mathcal{S}}(\mathcal{C}_i)(k) \right|  \leq n^\delta \right\rbrace.
% \end{equ}

% Thus, we proved that 

% \begin{multline}\label{equation a plotter finale preuve globrep}
%     \phi\left[\mathsf{GlobRep}^c \vert \con_{\mathscr{X}, \mathscr{Y}}, \nonint, \EXT = \mathsf{Ext}\right] \\ 
% \leq \e^{-\tau r(T_2-T_1)}\frac{\Phi^{\otimes r}_{(T_1, \mathscr{X}(C)) \rightarrow (T_2, \mathscr{Y}(C))}\left[\check{\mathcal{S}}(\mathcal{C})\in \mathcal{W}_{[T_1, T_2]}, \check{\close}^\delta\vert \EXT = \mathsf{Ext} \right] }{\phi\left[\con_{\mathscr{X}, \mathscr{Y}},\nonint\vert \EXT = \mathsf{Ext}\right]}
% \end{multline}

% The numerator is - by the uniform Ornstein--Zernike formula - the probability of bulk repulsion for a system of non-intersecting synchronized random walks. Since $C$ satisfies condition~\eqref{equation condition ecartement x et y preuve vraie mesure}, we use Lemma~\ref{lemme repulsion globale non-synchronized RW} and the local limit theorem for synchronized random walks as inputs from Section~\ref{section marches}  to obtain:

% \begin{multline}\label{estimation numerateur preuve globrep}
%     \Phi^{\otimes r}_{(T_1, \mathscr{X}(C)) \rightarrow (T_2, \mathscr{Y}(C))}\left[\check{\mathcal{S}}(\mathcal{C})\in \mathcal{W}_{[T_1, T_2]}, \check{\close}^\delta \vert \EXT = \mathsf{Ext} \right] \\ \leq cV(\mathscr{X}(C))V(\mathscr{Y}(C))(T_2-T_1)^{-\frac{r^2}{2}-\beta}.
% \end{multline}

% Our last task is to take care of the denominator. By Lemma~\ref{lemme estimation fine proba con ni} together with Corollary~\ref{corollaire majoration proba ni con mesure produit BC avec ecartement suffisant} we know that - since $\mathscr{X}(C)$ and $\mathscr{Y}(C)$ satisfy~\eqref{equation condition ecartement x et y preuve vraie mesure}: 
% \begin{equ}
%     \phi\left[\con_{\mathscr{X}, \mathscr{Y}}, \nonint \vert \EXT = \mathsf{Ext}\right] \geq \frac{1}{\chi}(T_2-T_1)^{-\frac{r^2}{2}}\e^{-\tau r (T_2-T_1)}V(\mathscr{X}(C))V(\mathscr{Y}(C)).
% \end{equ}
% Plotting this input inside~\eqref{equation a plotter finale preuve globrep} and using~\eqref{estimation numerateur preuve globrep}, we finally get that 

% \begin{equ}
%     \phi\left[ \mathsf{GlobRep}^c \vert \con_{\mathscr{X}, \mathscr{Y}}, \nonint,\EXT = \mathsf{Ext} \right] \leq C_1(T_2-T_1)^{-\beta} = C_1n^{-\beta}(1+o(1))
% \end{equ}
% for some large constant $c_1>0$, uniformly in $C$ satisfying~\eqref{equation condition ecartement x et y preuve vraie mesure} and $T_1,T_2$ satisfying~\eqref{condition T1 et T2 preuve finale}. Integrating this inequality over $C$ and then over $T_1$ and $T_2$ achieves the proof.

% It now remains to prove Claim 2 and more precisely the bound~\eqref{equation claim proba Hi}. First of all, notice that by the FKG inequality we have 
% \begin{equ}
%     \phi\left[\bigcap_{\# \in \lbrace L,R \rbrace} \bigcap_{i=1}^r \mathcal{H}^{\#}_i \vert\EXT = \mathsf{Ext}\right] \geq \prod_{\# \in \lbrace L,R\rbrace}\prod_{i=1}^r \phi\left[ \mathcal{H}^{\#}_{i}~\big\vert \EXT = \mathsf{Ext}\right].
% \end{equ}
% We prove the inequality for $\# = L$ only, since the obvious symmetry of the situation allows us to adapt the proof to the case $\# = R$ without difficulty.

% First let $i \in \left\lbrace 1,\dots,r-1 \right\rbrace$, so that 
% \begin{equ}
%     \mathcal{H}^L_i = \left\lbrace \exists \gamma^*: (\partial C^L_{i,\EXT})^* \goes{\left( \mathsf{Strip}^\mathsf{L}_{\EXT,T_1,T_2}\right)^*}{}{ (\partial C^L_{i+1,\EXT})^*} \text{ open} \right\rbrace
% \end{equ}
% Then it is easy to see that under the event $\mathcal{H}^L_i$, the arcs $\lbrace 0 \rbrace \times [x_i, x_{i+1}]$ and $\lbrace T_1\rbrace \times [\mathscr{X}(C)_i, \mathscr{X}(C)_{i+1}] $ have to be linked by a path of open edges. Since by assumption~\eqref{equation condition ecartement x et y preuve vraie mesure} one has $\mathscr{X}(C)_{i+1}-\mathscr{X}(C)_i < o(\sqrt{
% n}) $, then a simple union bound and the use of~\eqref{Comparaison boundary conditions} yield:
% \begin{eqnarray*}
% \phi\left[ \left(\mathcal{H}^L_{\EXT,i}\right)^c \vert \EXT = \mathsf{Ext} \right] &\leq& \phi\left[\lbrace 0 \rbrace \times [x_i, x_{i+1}] \leftrightarrow \lbrace T_1\rbrace \times [\mathscr{X}(C)_i, \mathscr{X}(C)_{i+1}] \vert\EXT = \mathsf{Ext} \right] \\
% &\leq& \phi\left[ \lbrace 0 \rbrace \times [x_i, x_{i+1}] \leftrightarrow \lbrace T_1\rbrace \times [\mathscr{X}(C)_i, \mathscr{X}(C)_{i+1}]\right] \\
% &\leq&\sum_{\substack{u \in \lbrace 0 \rbrace \times [x_i, x_{i+1}]\\v \in \lbrace T_1\rbrace \times [\mathscr{X}(C)_i, \mathscr{X}(C)_{i+1}] }} \phi\left[u \leftrightarrow v\right] \\
% &\leq& (x_{i+1} - x_i)o(\sqrt{n}) \e^{-\eta T_1},
% \end{eqnarray*}
% for some $\eta > 0$ (we use the fact that in the subcritical regime, the inverse correlation length is bounded away from 0 - see~\cite{ozrandomclustercampanino}). Knowing that $T_1>n^\eps$ (see~\eqref{condition T1 et T2 preuve finale}) concludes the proof. 


% Now, we work under the event $\bigcap_{i=1}^{r-1} \mathcal{H}^L_i$. Observe that if $\left( \mathcal{H}^L_r \right)^c$ occurs, under $\bigcap_{i=1}^{r-1} \mathcal{H}^L_i$, it implies the existence of an open path linking a vertex of $\lbrace T_1 \rbrace \times [\mathscr{X}(C)_r, \infty[$ to a vertex of $\lbrace T_1 \rbrace \times ]-\infty, \mathscr{X}(C)_1]$. By~\eqref{equation condition ecartement x et y preuve vraie mesure}, we know that $\mathscr{X}(C)_r - \mathscr{X}(C)_1 \geq \frac{rn^\eps}{2}.$ Then a basic union bound yields that 
% \begin{eqnarray*}
%     \phi \left[ \left(\mathcal{H}^L_r\right)^c \vert \bigcap_{i=1}^{r-1}\mathcal{H}^L_i, \EXT = \mathsf{Ext} \right] &\leq& \phi \left[ \left(\mathcal{H}^L_r\right)^c  \right] \\ &\leq& \sum_{\substack{u \in \lbrace T_1 \rbrace \times [\mathscr{X}(C)_r, +\infty) \\ v\in \lbrace T_1 \rbrace \times (-\infty, \mathscr{X}(C)_1]}} \phi\left[u \leftrightarrow v\right]
%     \\&\leq& \sum_{\substack{x>\mathscr{X}(C)_r \\ y<\mathscr{X}(C)_1}}\e^{-\eta(x-y)}\\
%     &\lesssim& \e^{-\eta(\mathscr{X}(C)_r-\mathscr{X}(C)_1)} \\
%     &\leq&C\exp\left(-\frac{\eta r n^\eps}{2}\right).
% \end{eqnarray*}

% Putting all the pieces together, we proved that
% \begin{equ}
%     \prod_{i=1}^r \phi\left[\mathcal{H}_i^L \vert \EXT = \mathsf{Ext}\right] \geq \left(1-2\exp\left( -cn^{\frac{\eps}{2}}\right)\right)^r.
% \end{equ}
% Obviously, the same symmetric proof holds for the quantities with superscript $R$ instead of $L$, so that we proved that
% \begin{eqnarray*}
%     \phi\left[\bigcap_{\# \in \lbrace L,R \rbrace}\bigcap_{i=1}^r \mathcal{H}^{\#}_i \vert \EXT = \mathsf{Ext} \right] &\geq& \left(1 - 2\exp\left(-cn^{\frac{\eps}{2}} \right)\right)^{2r} \\
%     &\geq& 1-4r\exp\left(-cn^{\frac{\eps}{2}}\right),
% \end{eqnarray*}
% provided that $n$ is large enough. This achieves the proof.


% \begin{proof}[Proof of Proposition~\ref{proposition global entropic repulsion}]

% \end{proof}
Lemma~\ref{lem: first step Ioan numerator} is crucial to the proof of Lemma ~\ref{lem:Ioan_numerator}. We turn to the proof of the latter.

\begin{proof}[Proof of Lemma~\ref{lem:Ioan_numerator}]
    We argue that the realization of the event $\lbrace \nonint, \con, \mathsf{Diam} \rbrace$ implies that the random set $\Gamma = \lbrace \Gamma_1, \dots, \Gamma_r \rbrace $ introduced in Lemma~\ref{lem: first step Ioan numerator} lies in $\Path$, where $\Path$ is the set of paths $\gamma = (\gamma_1, \dots, \gamma_r)$ satisfying the following properties.
    \begin{enumerate}[(i)]
        \item $\gamma_i$ connects $\mathscr{X}_i$ to $\mathscr{Y}_i$.
        \item $\gamma_i$ is contained in $\mathsf{Strip}$.
        \item $\gamma_i \cap \gamma_j = \emptyset$ for any $1 \leq j \neq i\leq r$.
    \end{enumerate} 

We also introduce the following subset of $\Path$:
\begin{multline*}
    \mathsf{ClosePath} = \big\lbrace \gamma \in \Path, \exists \ell_{i-1}, \ell_i \in \Z, k\in\lbrace T_1, \dots, T_2\rbrace \text{ such that } \\  (k,\ell_{i-1}) \in \gamma_{i-1}, (k, \ell_i)\in\gamma_i, |\ell_i - \ell_{i-1}| < \log^3 n   \big\rbrace.
\end{multline*}

We first argue that 
\begin{equ}\label{equ: equation a remonter preuve lemme ioan numerator}
    \phi^0_{\mathsf{Ext}^c}\left[ \mathsf{GlobRep}^c, \con, \nonint, \mathsf{Diam} \right] \leq (1 - \e^{-c(\log^2 n)})^{-1}\sum_{\gamma \in \mathsf{ClosePath}} \phi^0_{\Ext}\left[\Gamma= \gamma, \con, \nonint, \Diam \right].
\end{equ}
This fact is a consequence of the diamond confinement property proved in Lemma~\ref{lemme diamond confinement mesure conditionnee}. Indeed, observe that the clusters are confined in diamonds of maximal volume $(\log n)^2$ with exponentially large probability. Under this event, it is easy to check that if $\mathsf{GlobRep}^c$ occurs, then the top-most path of the clusters that are close one from each other cannot have a distance larger than $\log^2 n +2 \log^2 n$, and in particular belong to $\mathsf{ClosePath}$.

In the rest of the proof, for convenience we don't carry the term $(1 - \e^{-c(\log^2 n)})^{-1}$ through the calculations since it goes to 1 when $n$ goes to infinity. We then use Lemma~\ref{lem: first step Ioan numerator} to obtain:
\begin{multline*}
    \phi^0_{\mathsf{Ext}^c}\left[ \mathsf{GlobRep}^c, \con, \nonint, \mathsf{Diam} \right]  \leq \\ (1-\exp(-cn^\eps))^{-r}\sum_{\gamma\in\mathsf{ClosePath}}(\phi^0_{\Ext^c})^{\otimes r}\left[\bigcap_{i=1}^r\lbrace \gamma_i \text{ is open}, \partial \gamma_i \text{ is closed}\rbrace\right].
\end{multline*}
Even though the latter sum holds on an \emph{a priori} infinite set of paths, due to the fact that these paths are contained inside $\mathsf{Strip}$ it is a simple topological observation that at most one such $\gamma$ can exist. We obtain:
\begin{multline*}
    \phi^0_{\mathsf{Ext}^c}\left[ \mathsf{GlobRep}^c, \con, \nonint, \mathsf{Diam} \right]  \\ \leq (1+o(1))(\phi^0_{\Ext^c})^{\otimes r}\Big[ \lbrace \exists \gamma \in \mathsf{ClosePath}, \forall i\in\lbrace 1, \dots, r\rbrace, \gamma_i \text{ is open}, \partial \gamma_i \text{ is closed}\rbrace \Big].
\end{multline*}

The end the proof follows by an Ornstein--Zernike argument. Assume that there exists $\gamma \in \mathsf{ClosePath}$ such that $\gamma_i$ is open and $\partial \gamma_i$ is closed for all $i \in \lbrace 1,\dots, r \rbrace$. In the measure $(\phi^0_{\Ext^c})^{\otimes r}$, we can see that $\con_{(T_1,\mathscr{X}), (T_2,\mathscr{Y})}$ occurs. We thus work under the multidimensional Ornstein--Zernike coupling $\Phi^{0, \otimes r}_{\Ext^c, (T_1,\mathscr{X})\rightarrow(T_2, \mathscr{Y})}\big[ ~\cdot~ \big]$ and consider the random skeleton system $\mathcal{S}$ given by this coupling. 

Since the paths of $\gamma$ necessarily contain all the potential renewal points of the clusters $\mathcal{C}_i$ in $\mathsf{Strip}$, then it is the case that the synchronized skeleton $\check{\mathcal{S}}$ of the system of clusters is non-intersecting. Finally by the usual diamond confinement property of the Ornstein--Zernike coupling, since $\gamma$ satisfies the requirement of being in $\mathsf{ClosePath}$, then with probability going to 1 when $n$ goes to infinity, $\check{\mathcal{S}}$ satisfies $\inf_{t\in [T_1,T_2]} \Gap(\check{\mathcal{S}}(t)) \leq 3\log^2n + \log^2 n.$\footnote{We allow ourselves to be a bit elliptic, since the argument has already been written in the proof of Proposition~\ref{Proposition convergence mesure produit}.} Thus, we obtain by the usual Ornstein--Zernike coupling:


\begin{multline*}
    (\phi^0_{\Ext^c})^{\otimes r}\Big[ \lbrace \exists \gamma \in \mathsf{ClosePath}, \forall i\in\lbrace 1, \dots, r\rbrace, \gamma_i \text{ is open}, \partial \gamma_i \text{ is closed}\rbrace \Big] \\ \leq \e^{-\tau r (T_2-T_1)}\Phi_{(T_1,\mathscr{X}) \rightarrow (T_2, \mathscr{Y})}^{\otimes r}\left[ \check{\mathcal{S}}\in \mathcal{W}_{[T_1,T_2]}, \Gap(\check{\mathcal{S}}(t)) \leq 4\log^2n\big\vert\EXT=\Ext \right]. 
\end{multline*}

We now use once the local limit Theorem~\ref{theoreme local limit srw} and then Lemma~\ref{Lemme repulsion globale sdry} (the assumptions of the lemma are satisfied due to the edge-regularity of $\Ext$) to conclude that
\begin{multline*}
    \Phi_{(T_1,\mathscr{X}) \rightarrow (T_2, \mathscr{Y})}^{\otimes r}\left[ \check{\mathcal{S}}\in \mathcal{W}_{[T_1,T_2]}, \Gap(\check{\mathcal{S}}(t)) \leq 4\log^2n\big\vert\EXT=\Ext \right]\\ \leq CV(\mathscr{X})V(\mathscr{Y})(T_2-T_1)^{-\frac{r^2}{2}}(T_2-T_1)^{-\beta}.
\end{multline*}
Putting everything together and coming back to ~\eqref{equ: equation a remonter preuve lemme ioan numerator}, we obtain:
\begin{equ}
    \phi^0_{\Ext^c}\left[\mathsf{GlobRep}^c, \con, \nonint, \Diam\right] \leq CV(\mathscr{X})V(\mathscr{Y})(T_2-T_1)^{-\frac{r^2}{2}-\beta}\e^{-\tau r(T_2-T_1)},
\end{equ}
which concludes the proof.
\end{proof}

\subsection{The mixing argument and the proof of Theorems~\ref{theoreme estimation} and~\ref{Theoreme main} }\label{Sub finale mixing et preuves}


% Let us prove the converse inequality. The reasoning crucially relies on the idea introduced in the proof of Proposition~\ref{proposition global entropic repulsion}. Indeed, let us introduce the following events:
% \begin{equ}
%     \mathsf{Block^L} = \left\lbrace (0,x_1) \overset{\mathsf{Strip}^L_{n, \EXT}, *}{\longleftrightarrow} \dots \overset{\mathsf{Strip}^L_{n, \EXT}, *}{\longleftrightarrow} (0,x_r) \overset{\mathsf{Strip}^L_{n, \EXT}, *}{\longleftrightarrow} \infty \right\rbrace 
% \end{equ}
% and
% \begin{equ}
%     \mathsf{Block^R} = \left\lbrace (n,y_1) \overset{\mathsf{Strip}^R_{n, \EXT}, *}{\longleftrightarrow} \dots \overset{\mathsf{Strip}^R_{n, \EXT}, *}{\longleftrightarrow} (n,y_r) \overset{\mathsf{Strip}^R_{n, \EXT}, *}{\longleftrightarrow} \infty \right\rbrace.
% \end{equ}
% By this terminology, we mean that under $\mathsf{Block^L}$, $(0,x_1), (0,x_2), \dots, (0,x_r)$ belong to some open dual path that connects to infinity, all these connections occuring in the half-plane $(-\infty, 0] \times \Z$. Moreover we call $\mathsf{Block} = \mathsf{Block^L} \cap \mathsf{Block^R}$. Then,

% \begin{equ}
% \phi\left[\con,\nonint\right] \leq \phi\left[\con,\nonint \vert \mathsf{Block}\right]+\phi\left[\mathsf{Block}^c \vert \con,\nonint\right]\phi\left[\nonint, \con\right],
% \end{equ}
% which leads to (conditionally on the fact that $\phi\big[\mathsf{Block}\vert\con,\nonint\big]>0$, which will be established in the end of the proof):
% \begin{equ}\label{equation a plotter preuve finale estimation}
%     \phi\left[\con, \nonint \right] \leq \left( 1 - \phi\left[\mathsf{Block}^c \vert \con,\nonint\right] \right)^{-1}\phi\left[\con, \nonint\vert\mathsf{Block}\right]. 
% \end{equ}
% We upper bound the term $ \phi\left[\con, \nonint~\big\vert\mathsf{Block}\right]$. Under the event $\mathsf{Block}$, the exact same exploration argument leading to the bound~\eqref{equation majoration chemins produits} gives us - calling 
% \begin{equ}
%     \mathcal{P}_{x,y} = \left\lbrace (\gamma^1, \dots, \gamma^r) \text{ edge paths }, \begin{cases}
%         \gamma^i : (0,x_i) \overset{\mathsf{Strip}_n}{\longleftrightarrow} (n,y_i) &\forall 1\leq i\leq r \\
%         \gamma^i \cap \gamma^j=\emptyset &\forall 1 \leq i\neq j \leq r
%     \end{cases} \right\rbrace,
% \end{equ}
% that
% \begin{eqnarray*}
%     \phi\left[\con,\nonint \vert \mathsf{Block}\right] \leq \sum_{(\gamma^1, \dots, \gamma^r) \in \mathcal{P}_{x,y}}\phi^{\otimes r}\left[ \mathcal{N}(\gamma^1), \dots, \mathcal{N}(\gamma^r)\right]
% \end{eqnarray*}
% The end of the argument is the same as in the proof of Proposition~\ref{proposition global entropic repulsion}. Indeed let observe that under $\phi^{\otimes r}$, there can only exist one $r$-uple $(\gamma^1, \dots, \gamma^r)\in \mathcal{P}_{x,y}$ such that $\lbrace \mathcal{N}(\gamma^1), \dots, \mathcal{N}(\gamma^r) \rbrace$ occurs. Moreover, if the latter event occurs, then one has that $\con$ occurs and that the process of synchronized renewals is non-intersecting. Thus, 
% \begin{eqnarray}\label{eq majoration con ni sous block}
%     \phi\left[\con,\nonint \vert \mathsf{Block}\right] &\leq& \e^{-\tau nr}\Phi_{x \rightarrow y}^{\otimes r}\left[ \check{\mathcal{S}}(\mathcal{C}) \in \mathcal{W}_n \vert \mathsf{Block} \right] \nonumber \\
%     &\leq&C_1V(x)V(y)n^{-\frac{r^2}{2}}\e^{-\tau rn}.
% \end{eqnarray}
% where the last display comes from the uniform (in the boundary conditions) $r$-dimensional Ornstein--Zernike coupling~\eqref{equation oz uniform BC} and the local limit theorem for synchronized random walks. 

% To conclude the proof we have to show that there exists a constant $\eta > 0$ independent of $n$, such that 
% \begin{equ}\label{equation a montrer eta}
%     \phi\left[\mathsf{Block}^c \vert \con, \nonint\right] < 1-\eta.
% \end{equ}
% But this comes from classical properties of the subcritical regime. Indeed, let us write:
% \begin{eqnarray*}
% && \phi\left[\mathsf{Block} \vert \con, \nonint\right] \\ &=& \sum_{\substack{C_1, \dots, C_r \in \nonint,\con \\ 0,x \text{ are synchronized renewals }}}\phi\left[ \mathsf{Block} \vert\mathcal{C}_i =C_i, 1\leq i\leq r  \right]\phi\left[\mathcal{C}_i =C_i, 1\leq i\leq r\big\vert\nonint, \con\right] \\
% &\geq& \phi\left[\mathsf{Block}\right]\phi\left[ 0,x \text{ are synchronized renewals }\vert\nonint,\con\right],
% \end{eqnarray*}
% where the third line comes from the fact the the events $\mathcal{C}_i = C_i$ induce free boundary conditions on $\Z^2$ that favour the occurrence of $\mathsf{Block}$. Now, it is a consequence of the finite energy property that there exists some $\eta_1 > 0$ such that 
% \begin{equ}
%     \phi\left[0,x \text{ are synchronized renewals }\vert\nonint,\con\right] > \eta_1.
% \end{equ}
% Now we bound the term $\phi\left[\mathsf{Block}\right]$. First of all we observe that by~\eqref{equation FKG}, $\phi\left[\mathsf{Block}\right] > \phi\left[ \mathsf{Block^L}\right]\phi\left[\mathsf{Block^R}\right].$ We then focus on the term $\phi\left[\mathsf{Block^L}\right]$. It can be bounded by very classical percolation arguments. Let $R>0$ be a constant large enough such that $x_1, \dots, x_r \in \Lambda_R(0).$ Let $\mathcal{A}_R$ be the event that the half-annulus $\left(\Lambda_{2R}(0) \setminus \Lambda_R(0)\right)\cap \mathsf{Strip}^\mathsf{L}_{n,\EXT}$ is crossed by an open dual (half-)circuit. Then, it is clear that by~\eqref{equation FKG}:
% \begin{align*}
% &\phi\left[\mathsf{Block^L} \right] \\&\hspace{9pt}\geq \phi\big[ (0,x_1) \overset{\mathsf{Strip}^L_{n, \EXT}, *}{\longleftrightarrow} \dots \overset{\mathsf{Strip}^L_{n, \EXT}, *}{\longleftrightarrow} (0,x_r) \overset{\mathsf{Strip}^L_{n, \EXT}, *}{\longleftrightarrow} \partial \Lambda_{2R}(0), \mathcal{A}_R, \partial \Lambda _R(0) \overset{*}{\longleftrightarrow}\infty \big] \\
% &\hspace{9pt}\geq \phi\big[ (0,x_1) \overset{\mathsf{Strip}^L_{n, \EXT}, *}{\longleftrightarrow} \dots \overset{\mathsf{Strip}^L_{n, \EXT}, *}{\longleftrightarrow} (0,x_r) \overset{\mathsf{Strip}^L_{n, \EXT}, *}{\longleftrightarrow} \partial \Lambda_{2R}(0)\big]\phi\left[\mathcal{A}_R\right]\phi\big[\partial \Lambda_R(0) \overset{*}{\longleftrightarrow}\infty\big].
% \end{align*}
% The first two factors are strictly positive by finite energy, and the third one is strictly positive by very classical properties of the subcritical regime (see~\cite{duminilcopin2017lectures} for instance). This concludes the proof: plotting~\eqref{eq majoration con ni sous block} and~\eqref{equation a montrer eta} in~\eqref{equation a plotter preuve finale estimation} yields that
% \begin{equ}
%     \phi\left[ \con, \nonint \right] \leq \frac{C_1V(x)V(y)}{\eta}n^{-\frac{r^2}{2}}\e^{-\tau rn}.
% \end{equ}
We start with the proof of Theorem~\ref{theoreme estimation}. Observe that the first part of the statement, namely the estimation
\begin{equ}
    \phi\left[\nonint, \con\right] \geq c n^{-\frac{r^2}{2}}\e^{-\tau rn},
\end{equ}
has already be proved in Remark~\ref{remarque minoration ni con arbitrary BC}. Hence we only need an upper bound on $\phi\left[\nonint, \con\right]$. Observe that this task was almost already accomplished in Lemma~\ref{lem:Ioan_numerator}, up to one detail: in this lemma, one works between $T_1$ and $T_2$ and thus obtain a multiplying factor of $V(\mathscr{X})V(\mathscr{Y})$, which is at least of order $n^{2\eps r}$. This was necessary to ensure that the walks never come back close one from each other in order to implement the mixing argument leading to the proof of Theorem~\ref{Theoreme main}. Since we only need an upper bound, our task is mainly to check that the proof of Lemma~\ref{lem:Ioan_numerator} is still valid when $T_1$ and $T_2$ are replaced by random times of finite order.

\begin{proof}[Proof of Theorem~\ref{theoreme estimation}]
Let $\tilde{T}_1$ (resp. $\tilde{T}_2$) be the first (resp. the last) synchronization point of the maximal skeletons of the clusters after 0 (resp. before $n$). We also call $\tilde{\mathscr{X}}$ (resp. $\tilde{\mathscr{Y}}$) the unique vector such that $(T_1, \tilde{\mathscr{X}}_i) \in \mathcal{C}_i$ (resp. $(T_2, \tilde{\mathscr{Y}}_i)\in\mathcal{C}_i$). We already argued in the proof of Lemma~\ref{lemme diamond confinement mesure conditionnee} that $\tilde{T}_1$ and $\tilde{T}_2$ have exponential tails: for any $t \geq 0$ large enough,
\begin{equ}
    \phi\left[ \max \lbrace \tilde{T}_1, \tilde{T}_2 \rbrace >t \vert \con,\nonint\right] \leq \e^{-ct}.
\end{equ}
We also already argued that 
\begin{equ}
    \phi\left[\max\lbrace\Vert\tilde{\mathscr{X}}\Vert, \Vert\tilde{\mathscr{Y}}\Vert\rbrace > t \vert \con, \nonint \right] \leq \e^{-ct},
\end{equ}
for a possibly different value of $c>0$. We fix an arbitrary value of $t>0$ for the rest of this proof. Then, we upper bound:
\begin{eqnarray*}
    \phi\left[ \con, \nonint \right] &=& \phi\left[\con, \nonint, \max\lbrace\Vert\tilde{\mathscr{X}}\Vert, \Vert\tilde{\mathscr{Y}}\Vert\rbrace > t\right] + \phi\left[\con, \nonint, \max\lbrace\Vert\tilde{\mathscr{X}}\Vert, \Vert\tilde{\mathscr{Y}}\Vert\rbrace < t\right] \\ 
    &\leq& \phi\left[\con, \nonint \vert \max\lbrace\Vert\tilde{\mathscr{X}}\Vert, \Vert\tilde{\mathscr{Y}}\Vert\rbrace < t\right] + \e^{-ct}\phi\left[ \con, \nonint\right].
\end{eqnarray*}
Hence, we obtain that 
\begin{equ}\label{equ: majoration con nonint via conditionnement preuve finale}
    \phi\left[\con,\nonint\right] \leq \frac{1}{1-\e^{-ct}}\phi\left[\con, \nonint \vert \max\lbrace\Vert\tilde{\mathscr{X}}\Vert, \Vert\tilde{\mathscr{Y}}\Vert\rbrace < t\right].
\end{equ}
We focus on upper bounding $\phi\left[\con, \nonint \vert \max\lbrace\Vert\tilde{\mathscr{X}}\Vert, \Vert\tilde{\mathscr{Y}}\Vert\rbrace < t\right]$, and will do so using the method given by Lemma~\ref{lem: first step Ioan numerator}. Indeed, as usual we condition on the shape of the clusters outside of $\mathsf{Strip}$ to write:
\begin{multline*}
    \phi\left[\con, \nonint \vert \max\lbrace\Vert\tilde{\mathscr{X}}\Vert, \Vert\tilde{\mathscr{Y}}\Vert\rbrace < t\right] = \sum_{\Ext}\phi^0_{\Ext^c}\left[\con_{\tilde{\mathscr{X}},\tilde{\mathscr{Y}}}, \nonint \right] \\ \times \phi\left[\Ext=\Ext \vert \max\lbrace\Vert\tilde{\mathscr{X}}\Vert, \Vert\tilde{\mathscr{Y}}\Vert\rbrace < t\right].
\end{multline*}
As previously, the conditioning on $\EXT$ contains the fact that $\tilde{T}_1$ and $\tilde{T}_2$ are renewals. Let us fix some $\Ext$ such that $\max\lbrace\Vert\tilde{\mathscr{X}}\Vert, \Vert\tilde{\mathscr{Y}}\Vert\rbrace < t$. 

We use an argument very similar to that used in the proof of Lemma~\ref{lem:Ioan_numerator}. Indeed, recall the definition of the set of paths $\Gamma$ extracted from a configuration satisfying the events $\con, \nonint, \Diam$. We make the following claim, which is a variant of the statement of Lemma~\ref{lem: first step Ioan numerator}: there exists a constant $\eta>0$ depending only on the parameters $p$ and $q$, such that for any possible realization $\gamma$ of $\Gamma$,
\begin{equ}\label{equ: equivalent du lemme ioan numerateur}
    \phi^0_{\Ext^c}\left[\Gamma=\gamma, \nonint, \con\right] \leq (2-\eta)^r(\phi^0_{\Ext^c})^{\otimes r}\left[\bigcap_{i=1}^r\lbrace\gamma_i\text{ is open}, \partial \gamma_i\text{ is closed}\rbrace\right].
\end{equ}
Indeed, taking a close look at the proof of Lemma~\ref{lem: first step Ioan numerator} shows that the proof can be reproduced as soon as the equivalent of Claim~\ref{claim 2} is proved. It is indeed the case: we claim that for any boundary condition $\xi$ on $\mathsf{Strip}$,
\begin{equ}
\phi^{\xi/0}_{(\mathsf{Strip}\cup\Ext)^c}\left[\mathcal{H}^\mathsf{L}\cap\mathcal{H}^{\mathsf{R}}\right] \geq \eta >0.   
\end{equ}
The latter follows by classical properties of the subcritical phase of the random-cluster model: any vertex of the dual lattice can be connected to infinity with an open dual path in a half-plane with positive probability independently of the boundary conditions (since this fact is classical we do not write a full proof, though the argument is very similar to that leading to the proof of Claim~\ref{claim 2}, see~\cite{duminilcopin2017lectures} for instance). The conclusion of this discussion is that~\eqref{equ: equivalent du lemme ioan numerateur} holds with this choice of $\eta$.

We now conclude using the method of the proof of Lemma~\ref{lem:Ioan_numerator}: indeed by the same topological observation as in the latter proof, we argue that in the duplicate system given by the measure $(\phi^{0}_{\Ext^c})^{\otimes r}$ there can only exist one set of paths $\gamma = (\gamma_1, \dots, \gamma_r)$ satisfying the properties that they are contained in $\mathsf{Strip}$, that they are open and their top boundary is closed, and that they are linking $\tilde{\mathscr{X}}_i$ to $\tilde{\mathscr{Y}}_i$. Thus, we obtain: 
\begin{equ}\label{equ: equ majoration of con ni by a path event}
    \phi^0_{\Ext^c}\left[\con, \nonint, \Diam\right]  \leq (2-\eta)^r(\phi^0_{\Ext^c})^{\otimes r}\left[\lbrace \exists \gamma \in \mathsf{Path}, \gamma_i\text{ is open}, \partial\gamma_i \text{ is closed} \rbrace\right].
\end{equ}
We conclude using the same Ornstein--Zernike argument as in Lemma~\ref{lem: first step Ioan numerator}. Indeed, observe that the event on the right-hand side of~\eqref{equ: equ majoration of con ni by a path event} implies that in the product measure, the connection event occurs, and the synchronized skeletons are non-intersecting. Thus,
\begin{equ}
    \phi^0_{\Ext^c}\left[\con, \nonint, \Diam\right] \leq (2-\eta)^r \e^{-\tau r (T_2-T_1)}\phi^{0, \otimes r }_{ (T_1,\tilde{\mathscr{X}})\rightarrow(T_2,\tilde{\mathscr{Y}})}\left[  \check{\mathcal{S}}\in \mathcal{W}_{[T_1, T_2]} \vert \Ext=\Ext \right].
\end{equ}
We conclude using the properties of the coupling, and the Local limit Theorem~\ref{theoreme local limit srw} to upper bound the right-hand side probability by $CV(\tilde{\mathscr{X}})V(\tilde{\mathscr{Y}})(T_2-T_1)^{-\frac{r^2}{2}}$. Using the assumption on $\Ext$, we then very roughly upper bound (we can be a bit loose at this stage since we only need an upper bound)
\begin{equ}
    \max\lbrace V(\tilde{\mathscr{X}}), V(\tilde{\mathscr{Y}}) \rbrace \leq  (2\max\lbrace \Vert\tilde{\mathscr{X}}\Vert, \Vert\tilde{\mathscr{Y}}\Vert \rbrace)^{\frac{r(r-1)}{2}} \leq (2t)^{\frac{r(r-1)}{2}} \leq V(x)V(y)(2t)^{\frac{r(r-1)}{2}}
\end{equ}
Gathering everything together, we proved that uniformly in $\Ext$ satisfying the norm condition:
\begin{equ}
    \phi^0_{\Ext^c}\left[ \con,\nonint \right] \leq V(x)V(y)(2(2-\eta)t)^{\frac{r(r-1)}{2}}\e^{-\tau r(n-t)}(n-t)^\frac{r^2}{2}. 
\end{equ}
Summing over all the admissible $\Ext$ and coming back to~\eqref{equ: majoration con nonint via conditionnement preuve finale} we proved that
\begin{equ}
    \phi\left[\con, \nonint \right] \leq \tilde{C}V(x)V(y)\e^{-\tau rn}n^{-\frac{r^2}{2}},
\end{equ}
with $\tilde{C} = \frac{C\e^{\tau rt}(2t)^{\frac{r(r-1)}{2}}(2-\eta)^r}{1-\e^{-ct}}$. This concludes the proof.
\end{proof}

We then turn to the proof of Theorem~\ref{Theoreme main}. It follows from the repulsion estimate of Proposition~\ref{proposition global entropic repulsion} and the convergence of the product system stated in Proposition~\ref{Proposition convergence mesure produit}. The observation is that when $\mathsf{GlobRep}$ occurs, then it is a consequence of the mixing property of the random-cluster measure~\eqref{weak ratio mixing} that the distribution system of clusters is very close to the one of an \emph{independent} system of clusters.  


\begin{proof}[Proof of Theorem~\ref{Theoreme main}]

We follow the same pattern as in the proof of Proposition~\ref{Proposition convergence mesure produit}, and use the strategy given by Lemma~\ref{Lemme technique processus sto}. We then fix some $\delta >0$ and fix arbitrary signs $\pm$ for the $r$ envelopes. We introduce the scaled process $\Gamma^{\pm}_n(t)$, and consider a function $f : \mathcal{C}([\delta, 1-\delta], \R^r) \rightarrow \R$, continuous and bounded. As in the proof of Proposition~\ref{Proposition convergence mesure produit} we shall omit to write the restriction to the interval $[\delta, 1-\delta]$ when writing $f^\delta(\Gamma^\pm_n)$. We start by arguing that due to the boundedness of $f^\delta$ and to Lemma~\ref{lem: edge regularity true measure},

\begin{equ}
    \phi\left[ f^\delta(\restriction{\mathcal{S}_n}{[\delta, 1-\delta]})\vert \nonint,\con \right] = (1+o(1))\phi\left[ f^\delta(\restriction{\mathcal{S}_n}{[\delta, 1-\delta]})\vert \nonint,\con, \mathsf{EdgeReg} \right].
\end{equ}

As usual, we first condition on $T_1$ and $T_2$: because of the edge-regularity condition, we can chose $n$ large enough so that $n\delta>T_1$ and $n(1-\delta) < T_2$. We then condition on the (edge-regular) shape of the clusters outside of $\mathsf{Strip}:=\mathsf{Strip}_{[T_1,T_2]}$:
\begin{multline*}
    \phi\left[f^\delta(\Gamma^\pm_n) \vert \nonint, \con\right] =  \sum_{\Ext} \phi\left[f^\delta\left(\Gamma^\pm_n\right) \vert \nonint, \con_{\mathscr{X}, \mathscr{Y}}, \EXT = \mathsf{Ext} \right] \\ \times \phi\left[\EXT = \mathsf{Ext}\vert\nonint,\con_{\mathscr{X}, \mathscr{Y}}, \mathsf{EdgeReg}\right].
\end{multline*}
Let us fix some $\Ext$ which is edge-regular. 
Now, we make use of Proposition~\ref{proposition global entropic repulsion} to argue that:
\begin{multline}
    \phi\left[f^\delta(\Gamma^\pm_n) \vert \nonint, \con_{\mathscr{X}, \mathscr{Y}}, \EXT = \mathsf{Ext} \right] = \\ \left(1+o(1)\right)\phi\left[f^\delta(\Gamma^\pm_n) \vert \nonint, \con_{\mathscr{X}, \mathscr{Y}}, \EXT = \mathsf{Ext}, \mathsf{GlobRep} \right].
\end{multline}
But the mixing property~\eqref{weak ratio mixing} actually allows us to argue that:
\begin{equ}\label{equ: equation finale mixing}
    \left| \frac{\phi\left[f^\delta(\Gamma^\pm_n) \vert \nonint, \con_{\mathscr{X}, \mathscr{Y}}, \EXT = \mathsf{Ext}, \mathsf{GlobRep} \right]}{\phi^{\otimes r}\left[f^\delta(\Gamma^\pm_n) \vert \nonint, \con_{\mathscr{X}, \mathscr{Y}}, \EXT = \mathsf{Ext}, \mathsf{GlobRep} \right]} - 1\right| < \e^{-2(\log n)^2}.
\end{equ}
Indeed, to see it, we decompose the term $\phi\left[f^\delta(\Gamma^\pm_n) \vert \nonint, \con_{\mathscr{X}, \mathscr{Y}}, \EXT = \mathsf{Ext}, \mathsf{GlobRep} \right]$ as follows:
\begin{equ}
    \phi\big[f^\delta(\Gamma^\pm_n) \vert \nonint, \con_{\mathscr{X}, \mathscr{Y}}, \mathsf{GlobRep} \big] = \sum_{C_1, \dots, C_r} f^\delta(\Gamma^\pm_n)\frac{\phi^0_{\Ext^c}\left[\mathcal{C}_1=C_1, \dots, \mathcal{C}_r=C_r\right]}{\phi^0_{\Ext^c}\left[\nonint, \con_{\mathscr{X}, \mathscr{Y}}, \mathsf{Globrep} \right]},
\end{equ}
where the sum holds over the possible realizations of $C_1, \dots, C_r$ of the clusters of $\mathscr{X}$ under the measure $\phi^0_{\Ext^c}[\cdot\vert \nonint, \con_{\mathscr{X}, \mathscr{Y}}, \mathsf{GlobRep}]$. The point is that those sets are almost surely finite and have a mutual distance larger than $\delta \log^3 n$ by the diamond confinement property. We can then apply ~\eqref{weak ratio mixing} to both the numerator and the denominator of the fraction to obtain ~\eqref{equ: equation finale mixing}.

The last thing to notice is that the entropic repulsion estimate~\eqref{entropic repulsion equation} also holds for the product measure: 
\begin{equ}
    \phi^{\otimes r}\left[ \mathsf{GlobRep} \vert \con_{\mathscr{X}, \mathscr{Y}}, \nonint, \EXT = \mathsf{Ext} \right] \geq 1-cn^{-\beta},
\end{equ}
because of the usual Ornstein--Zernike coupling~\eqref{equation OZ boundary conditions} and the entropic repulsion for random walks given by Lemma~\ref{lemme entropic repulsion sdrw}. Thus, we proved that:
\begin{multline}
    \phi\left[f^\delta(\Gamma^\pm_n) \vert \nonint, \con_{\mathscr{X}, \mathscr{Y}}, \EXT = \mathsf{Ext} \right] = \\ (1+o(1))\phi^{\otimes r}\left[f^\delta(\Gamma^\pm_n) \vert \nonint, \con_{\mathscr{X}, \mathscr{Y}}, \EXT = \mathsf{Ext} \right].
\end{multline}
But now, due to the assumed edge-regularity of $\Ext$ and to Proposition~\ref{Proposition convergence mesure produit}, we know that the RHS converges towards $\E\left[f^\delta(\sigma\bw^{(r)})\right].$
Hence, 
\begin{equ}
    \phi\left[ f^\delta(\Gamma^\pm_n) \vert \nonint, \con \right] \goes{}{n \rightarrow \infty}{\E\left[f^\delta(\sigma\bw^{(r)})\right]},
\end{equ}
and so is established point $(i)$ of Lemma~\ref{Lemme technique processus sto}. As previously, point $(ii)$ is an easy consequence of the previous arguments and the central limit Theorem. This yields the desired weak convergence.
\end{proof}



\section{Local statistics of directed non-intersecting random bridges}\label{section marches}

As seen before, the study of a system of clusters subject to the non-intersection conditioning boils down - thanks to Ornstein--Zernike theory and entropic repulsion estimates - to the study of systems of non-intersecting directed random walks.  

Non intersecting random walks, and more largely random walks in cones have a very rich combinatorial and probabilistic structure. They have been studied widely throughout the last 50 years. The seminal work has been the paper of Karlin and McGregor~\cite{karlin1959} noticing a determinantal formula for the probability of $r$ random walks to intersect. However, their approach only applied to a very specific class of walks, and was combinatorial by nature; it lead to remarkable developments around integrable systems of walks (see~\cite{GRABINER1999177,johansonn2004}).

A more probabilistic treatment was started in~\cite{konig2005},~\cite{conditionallimittheoremsfororderedrandomwalks,randomwalksincones, Invarianceprinciplesforrandomwalksincones}. Indeed, in~\cite{konig2005} a definition a the random walk conditioned to stay in a cone was given in terms of a Doob $h$-transform by an harmonic function vanishing on the boundary of the cone, allowing the authors to obtain local limit theorems and invariance principles for a much broader class of random walks.  We briefly summarize the definitions and construction of the concerned objects.

The goal of this section is then to study the properties of such systems of walks, especially their behaviour under the diffusive scaling. Let us introduce the relevant object to study.  

\begin{Def}[Directed system of random walk] \label{Definition directed random walk}
Let $r\geq 1 $ be an integer. For $1 \leq i \leq r$, let $(X_n^i)_{n \geq 0}$ be an independent sequence of independent and identically distributed random variables having an exponential moment and such that $\E X_1^1 = 0$. Moreover, for $1 \leq i \leq r$, let $(\theta_n^i)_{n \geq 0}$ be an independent sequence of independent and identically $\N^*$-valued distributed random variables, also having an exponential moment, and independent of all the $X_n^i$'s. Now let us call 
\begin{equ}
\mathbf{T}^i_n = \sum_{k=1}^n \theta^i_k, ~ \mathbf{Z}^i_n = \sum_{k=1}^n X^i_k \text{ and } \mathbf{S}^i_n = \left( \mathbf{T}
^i_n, \mathbf{Z}^i_n \right).
\end{equ}
Then the system 
\begin{equ}
    \left(\mathbf{S}_n\right)_{n \geq 0} = \left(\mathbf{S}^1_n, \dots, \mathbf{S}^r_n\right)_{n \geq 0}
\end{equ}
is called a system of $r$ (non-synchronized) directed random walks. For any $(k_i,x_i)_{1\leq i \leq r} \in \left(\N\times \Z\right)^r$, we write $\mathbf{P}_{(k,x)}$ for the law of the $r$-directed random walk with $\mathbf{S}^i_0 = (k_i,x_i)$. When all the $k_i$ are equal, which will often be the case, we make the slight abuse of notation by writing $\mathbf{P}_{(k,x)}$ with $k\in\Z, x\in\Z^r$.
\end{Def}

For sake of simplicity in the proofs, and to stick to the context given by Sections~\ref{section review oz},~\ref{section independent system} and~\ref{section RCM}, we shall also make the following assumption of $\delta$-confinement:

\paragraph{\textbf{Assumption 1}.} There exists a $\delta > 0$ such that almost surely, 
\begin{equ}\label{eq: assumption 1}
    (\theta^1_1, X^1_1) \in \mathcal{Y}^{+,\delta}_0. 
\end{equ}
\begin{Rem}
    The latter assumption is only used by convenience, and the results would hold true without it. However it will be used in subsection~\ref{subsection non intersecting non-synchronized rw} to argue that non-intersection of $r$ non-synchronized random walks is implied by the non-intersection of the $\delta$-diamonds of the associated embedded system of synchronized random walks. 
\end{Rem}
We also introduce the walks analogs of the event $\con$ and $\nonint$ (see Figure~\ref{figure non-synchronized random walks}).

\begin{Def}\label{definition temps d'arret non-synchronized system}
Let $y \in W$, and let $S$ be a system of $r$ directed random walks. For $1 \leq i \leq r$, we denote by $\mathbf{S}^i$ the linear interpolation of the points $(\mathbf{T}_1^i, \mathbf{Z}_1^i), (\mathbf{T}_2^i, \mathbf{Z}_2^i), \dots $. We then form the $r$-uple $\mathbf{S} = (\mathbf{S}^1, \dots, \mathbf{S}^r)$. Finally, for any $t \geq 0$, we set $\mathbf{S}(t)$ to be the almost surely unique $r$-uple of real numbers $\mathbf{S}(t) = \mathbf{S} \cap \left( \lbrace t \rbrace \times \R^r\right)$. The hitting event is defined by:
\begin{equ}
    \hit_{(n,y)} = \left\lbrace \mathbf{S}(n) = y  \right\rbrace,
\end{equ}
and we introduce the stopping time $\boldsymbol{\rho}$ as follows:
\begin{equ}
    \boldsymbol{\rho} = \inf \left\lbrace t\geq 0, \mathbf{S}(t) \notin W\right\rbrace.
\end{equ}
Observe that with this definition, $\boldsymbol{\rho}$ does not need to be an integer. Moreover observe that 
\begin{equ}
    \lbrace \boldsymbol{\rho} > n \rbrace = \lbrace \mathbf{S} \in \mathcal{W}_n \rbrace.
\end{equ}
\end{Def}

% Figure environment removed


The goal of this section is the proof of the following result:


\begin{Theorem}[Invariance principle for directed random walks]\label{theoreme invariance principle for drw}
Let $S$ be a system of $r$ directed random walks sampled according to $\mathbf{P}_{(0,x)}\left[~.\vert \mathbf{S} \in \mathcal{W}_n, \hit_{(n,y)} \right]$. Then, there exists $\sigma> 0$ such that
\begin{equ}
    \left(\frac{1}{\sqrt{n}}\mathbf{S}\left(nt\right) \right)_{0\leq t \leq 1} \goes{(d)}{n \rightarrow \infty}{\left(\sigma\bw^{(r)}_{t}\right)_{0\leq t \leq 1}},
\end{equ}
where the convergence holds in the space $\mathcal{C}([0,1], \R^r)$ equipped with the topology of the uniform convergence, and is uniform over any pair of starting and ending points $x,y$ satisfying $\norme{x},\norme{y} = o(\sqrt{n})$. 
\end{Theorem}


In order to do so, we first study a version of the system of conditioned directed random walks which is easier to handle, where the length of the steps is still random, but is the same for each one of the walks: such systems are called \textit{synchronized}, and are the object of Subsection~\ref{subsection non-intersecting systems of synchronized directed random walks}, where the invariance principle for this type of walks is proved. Then, we derive the theorem for the non-synchronized version in Subsection~\ref{subsection non intersecting non-synchronized rw}, by using the framework of \textit{decorated random walks}, studied in Subsection~\ref{Subsection synchronized systems of random walks with random decorations}. 

In our journey to Theorem~\ref{theoreme invariance principle for drw}, we also collect a few facts about system of non-intersecting random walks bridges (whether they are synchronized or not), that we have used in Sections~\ref{section independent system} and~\ref{section RCM}.
These outputs are the local limits Theorems~\ref{theoreme local limit srw} and~\ref{gnedenko theorem}, the confinement lemma for a single walk (Lemma~\ref{Confinment lemma}), the a priori estimate of Lemma~\ref{lemme estimation proba non-synchronized RW} and finally the entropic edge repulsion and bulk repulsion for systems of random walks Lemmas~\ref{lemme repulsion globale non-synchronized RW} and~\ref{entropic repulsion non-synchronized RB}. 


\subsection{Non-intersecting systems of synchronized directed random walks}\label{subsection non-intersecting systems of synchronized directed random walks}

As observed in Sections~\ref{section independent system} and~\ref{section RCM} our arguments are often soft enough to boil down to the study of a \emph{synchronized} system of walks, where the time reference is still random but common to every walk.

%\begin{Def}[Directed random walk]
%Let $\left(\theta_k, X_k\right)_{k\geq 1}$ be a sequence of independent and identically distributed random variables that are $\N \times \Z$-valued such that $\theta_1 > 0$ almost surely and such that the $X$'s are centered. We then form the associated random walks:
%\begin{equ}
%    T_n = \sum_{k=1}^n \theta_k \text{  and  } Z_n = \sum_{k=1}^n X_k.
%\end{equ}
%Then we call the process $S_n:= (T_n, Z_n)$ a \textit{directed random walk} on $\Z^2$. Moreover, the law of the increments of such a walk will be called a \textit{directed measure} on $\Z^2$. 
%\end{Def}
%In what follows, we will always assume that both $\theta_1$ and $X_1$ have an exponential moment, meaning that there exists $c>0$ such that, for any $t>0$ sufficiently large,
%\begin{equ}
%    \PP \left[ \max(\theta_1, |X_1|) > t\right] \leq \e^{-ct}.
%\end{equ}
%Moreover we will always assume 
%\begin{Rem}
%There are two equivalent ways of seeing such an object. The first one, as suggested by the definition is to consider $(S_n)_{n \geq 0}$ as a random walk on $\Z^2$ having its first coordinate increasing as. But one could also consider it as a random walk on $\Z$ with a random time reference given by the process $(T_n)_{n \geq 0}$. Indeed, observe that if all the $\theta$'s are almost surely equal to 1, then we recover a regular one-dimensional random walk. 
%\end{Rem}

\begin{Def}[Synchronized directed random walk]\label{def synchronized system of random walks}
Let $r\geq 1$ be an integer. We consider a sequence of independent and identically distributed random variables $(\theta_k, X^1_k, \dots, X^r_k)_{k\geq 0}$ taking values in $\N \times \Z^r$. Moreover we assume that $\theta_1 > 0$ almost surely, that $X_1^1, \dots, X_1^r$ are centered, and that for any $1\leq i < j \leq r$, $X_1^i$ and $X_1^j$ are independent and identically distributed (observe that the only pair of variables that could possibly be dependent are $\theta_k$ and $X^i_k$ for any $1 \leq i \leq r$).  Now let us call:
\begin{equ}
    T_n = \sum_{k=1}^n \theta_k \text{  and  } Z^i_n = \sum_{k=1}^n X^i_k
\end{equ}
Then the system 
\begin{equ}
  \left(S_n\right)_{n \geq 0} = \left(T_n, Z^1_n, \dots, Z^r_n\right)_{n \geq 0}
\end{equ}
is called a \textit{synchronized system of directed random walks}. In what follows, we chose to see it as an object of $\N \times \Z^{r}$, which we will refer to as $\left(S_n\right)_{n\geq 0} = \left(T_n, Z_n\right)_{n \geq 0}.$ For any $(k,x) \in \N\times\Z^r$, we will denote by $\PP_{(k,x)}$ the law of the synchronized $r$-random walk started from the point $(k,x)$, \textit{i.e.} the law of $\left( (k,x)+S_n\right)_{n \geq 0}.$ 
\end{Def}

In what follows, we will always assume that both the $\theta$'s and the $X$'s have an exponential moment, meaning that there exists $c> 0$ such that for any $t>0$ sufficiently large, 
\begin{equ}
    \PP \left[ \max \left\lbrace \theta_1, \left| X^1_1 \right| \right\rbrace > t \right] \leq  \e^{-ct}.
\end{equ}
We also assume that Assumption 1 holds and~\eqref{eq: assumption 1} is still valid. 

We also introduce the following hitting event, for any $(n,y) \in \N\times\Z^r$:
\begin{equ}
    \hit_{(n,y)} = \left\lbrace \exists k \geq 0, S_k = (n,y) \right\rbrace,
\end{equ}
and we introduce the stopping time
\begin{eqnarray}
    H_{(n,y)} = \min\left\lbrace k \geq 0, S_k = (n,y) \right\rbrace.
\end{eqnarray}
Moreover, $\rho$ will denote the stopping time corresponding to the first exit of the Weyl chamber:
\begin{equ}
    \rho= \min \left\lbrace n \geq 0,  S_n \notin W \right\rbrace
\end{equ}

\begin{Rem}
We want to highlight and warn the reader about the fact that in the latter definition, the time reference for $\rho$ and the event $\hit_{(n,y)}$ has been given in terms of the increments of the walks and not in terms of projection on the $\Vec{e_1}$-axis, as in Definition~\ref{definition temps d'arret non-synchronized system}. This is for convenience in the proofs to come in the rest of this subsection. 
\end{Rem}

The main results of this subsection are two local limit theorems for a system of $r$ synchronized random walks and an invariance principle for such a system under a diffusive scaling (recall that $\Delta$ denotes the Vandermonde function). These results have already been obtained in~\cite{conditionallimittheoremsfororderedrandomwalks} and the scaling limit under a diffusive scaling has been shown to be Brownian excursion in $W$ (the Brownian watermelon). The only difference with our setting is the supplementary randomness introduced by the random variable $T$. The goal of what follows is to prove that the local limit results of~\cite{randomwalksincones} still hold in this context. The statements are the following: 

\begin{Theorem}\label{theoreme local limit srw} 
Let $\left( S_n \right)_{n \geq 0}$ be a synchronized system of random walks. There exists a function $V: W \rightarrow \R^*_+$ and a constant $C_1 > 0$ such that for any pair of sequences $(x_n)_{n \geq 0}, (y_n)_{n\geq 0}$ taking values in $W$ such that $\norme{x_n}, \norme{y_n} = o(\sqrt{n})$, when $n \goes{}{}{\infty}$, 
\begin{equ}
    \PP_{(0,x_n)} \left[ H_{(n,y_n)}<\rho, \hit_{(n,y_n)} \right] = C_1\frac{V(x_n)V(y_n)}{n^{r^2/2}}\left(1+o(1)\right),
\end{equ}
with $o(1)$ being independent of the choice of the sequences $x_n$ and $y_n$. 
%Furthermore, there exists $C_2 > 0$ such that for any pair of points $x,y \in W$ such that $\norme{u}\ = o(\sqrt{n})$, $\norme{v}= O(\sqrt{n})$,
%\begin{equ}
%    \PP_{(0,x)} \left[ H_{(n,y)}<\tau, \hit_{(n,y)} \right] = C_2\frac{V(u)\Delta(v)\e^{-\frac{-\norme{v}^2}{2n}}}{n^{r^2/2}}\left(1+o(1)\right).
%\end{equ}
Furthermore, the function $V$ satisfies the following set of properties:
\begin{enumerate}
    \item If $x,y \in W$ are such that $\left| y_{i+1} - y_i \right| > \left|x_{i+1}-x_i\right|$ for any $1 \leq i \leq r-1$, then
    \begin{equ}
        V(y) \geq V(x).
    \end{equ}
    \item When  $\Gap(x) \rightarrow \infty$, then $\frac{V(x)}{\Delta(x)} \rightarrow 1$.
    \item There exists a positive $c>0$ such that
    \begin{equ}
            V(x) \leq c \prod_{1\leq i < j \leq r}\left|1+x_j-x_i\right|
    \end{equ}
\end{enumerate}
\end{Theorem}

The second local limit result is the analog of Gnedenko's local limit theorem. It is the Theorem 5 of~\cite{randomwalksincones}.

\begin{Theorem}\label{gnedenko theorem}
Let $\left( S_n \right)_{n \geq 0}$ be a synchronized system of random walks. Then, there exists a constant $\varkappa > 0$ such that for any fixed $x \in W$,
\begin{equ}
    \sup_{y \in W} \left| n^{\frac{r(r+1)}{4}}\PP_{(0,x)}\left[ S \in \mathcal{W}_n, \hit_{(n,y)} \right] - \varkappa V(x)\Delta\left(\frac{y}{\sqrt{n}}\right)\e^{-\frac{\norme{y}^2}{2n}} \right| \goes{}{n \rightarrow \infty}{0}.   
\end{equ}
\end{Theorem}

The last theorem of this section is the invariance principle stating that properly rescaled such a system of non-intersecting synchronized directed random bridges converges towards the Brownian watermelon.
\begin{Theorem}\label{theoreme d'invariance sdrw}
Let $\left(S_n\right)_{n \geq 0}$ be a system of $r$ synchronized random walks. We study the trajectory of $S$ on $[0, H_{(n,y)}] $ under the measure 
\begin{equ}
\PP_{(0,x)} \left[ ~\cdot~ \vert H_{(n,y)} < \rho, \hit_{(n,y)} \right] .
\end{equ}
Let $\mathfrak{T}$ be the linear interpolation between the points $(T_1, S_1), \dots, (n,y)$, and 
$S(t)$ be the almost surely unique intersection $\mathfrak{T}\cap \left(\lbrace t \rbrace\times \R^r\right)$. Then, there exists $\sigma>0$ such that:
\begin{equ}
    \left( \frac{1}{\sqrt{n}} S(nt) \right)_{0 \leq t \leq 1} \goes{}{n\rightarrow \infty}{\left(\sigma\bw^{(r)}_t\right)_{0 \leq t \leq 1}},
\end{equ}
The convergence occurs in the space $\mathcal{C}\left([0,1], \R^r \right)$ endowed with the topology of uniform convergence. Moreover, it is uniform over every starting and ending points $x$ and $y$ such that $\norme{x},\norme{y} = o(\sqrt{n})$ (though we do not highlight the dependence of $x$ and $y$ on $n$ to lighten a bit the notations). 
\end{Theorem}
\begin{Rem}
These three statements have been shown to hold for systems of "regular" (meaning that $\theta_k = 1$ almost surely) non-intersecting systems of random bridges (see~\cite{conditionallimittheoremsfororderedrandomwalks, randomwalksincones, Invarianceprinciplesforrandomwalksincones}). The goal of this section is then to show that they still hold whenever the random variables $\theta_k$'s are sufficiently regular (and the condition that they have an exponential moment is by far sufficient to ensure it). This strategy has already been implemented in~\cite{ottvelenikwachtelioffe} in the case of a directed random walk conditioned to remain above a hard wall. Actually, the method developed in this latter work combined with the results of~\cite{randomwalksincones} implies Theorems~\ref{theoreme local limit srw} and~\ref{gnedenko theorem}. In Subsection~\ref{subsub Preliminary work on non-intersecting system of regular random walks}, we state a lemma about regular random walks that is sufficient to mimic the proofs of~\cite{randomwalksincones}. In Subsection~\ref{subsub Sketch of the proofs of Theorems theoreme d'invariance sdrw and theoreme local limit srw}, we briefly explain how to get the two announced results from the papers~\cite{ottvelenikwachtelioffe, randomwalksincones}. Finally in Subsection~\ref{subsub Two useful lemmas about directed random walks} we state and prove two \textit{ad hoc} lemmas that will be needed in the rest of the work, and which can be seen as \textit{entropic repulsion} estimates for random walks in two different settings.
\end{Rem}

\subsubsection{Preliminary work on non-intersecting system of regular random walks}\label{subsub Preliminary work on non-intersecting system of regular random walks} 

Recall the definitions of the Weyl chamber and the Vandermonde function from subsection~\ref{subsection bw}. Below, for $1 \leq i \leq r$, let $X^i_1, \dots, X^i_n, \dots$ be a sequence of iid random variables satisfying the important assumption that they have an exponential moment. For $1 \leq i \leq r$, $n \in \N$, we then form the associated random walks $S^i_n = X^i_1 + \dots + X^i_n$, and call $S_n = (S^1_n, \dots, S^r_n) \in \R^r$. For any $x \in W$, we denote by $\PP_x$ the law of $x+S_n$. For any $y\in W$, let us introduce the following stopping time:
\begin{equ}\label{definition tau}
    \rho_y = \min\left\lbrace n \geq 0, y+S_n \notin W \right\rbrace.
\end{equ}

We shall need a statement similar to Lemma 14 of~\cite{conditionallimittheoremsfororderedrandomwalks} but for a wider range of $y$'s under a stronger moment assumption.

\begin{Lemma}\label{lemme estimation temps d'arrêt} Let $\eps > 0$ be fixed.
There exist two positive constants $\chi > 0, C > 0$ such that for any sequence $y_n \in W$ satisfying:
\begin{equ}\tag{C1}\label{equation C1}
    \norme{y_n} = o(\sqrt{n})
\end{equ}
and
\begin{equ}\tag{C2}\label{equation C2}
    \Gap(y_n) > (\log n)^{1+\eps}, 
\end{equ}
then one has
\begin{equ}\label{equation asymptotique temps d'arret}
    \PP \left[ \rho_{y_n} > n \right] = \chi \Delta(y_n)n^{-\frac{r(r-1)}{4}}\left(1 + o(1)\right),
\end{equ}
with $o(1)$ being uniform in the choice of the sequence $y_n$ satisfying~\eqref{equation C1} and~\eqref{equation C2}. Moreover,
\begin{equ}\label{equation majoration lemme estimation temps d'arrêt}
    \PP \left[\rho_y > n \right] \leq C V(y)n^{-\frac{r(r-1)}{4}}.
\end{equ}
In the latter expression $\chi$ and $C$ depend on the law of $X$.
\end{Lemma}
Note that this statement is stronger than Lemma 14 of~\cite{conditionallimittheoremsfororderedrandomwalks}, but we work under an exponential moment assumption whereas the authors of the cited paper work under minimal moment assumptions.
The main ingredient for this lemma is the celebrated Kömlos-Major-Tusnady (KMT) coupling that allows one to couple a random walk with exponential moments with a Brownian motion with precision $O\left(\log n\right)$.

\begin{Prop}[KMT coupling]\label{prop kmt coupling}
Let $X_1, \dots, X_n, \dots$ be a sequence of centered i.i.d. real variables having an exponential moment, and $S_n = X_1 + \dots + X_n$ the associated random walk. Then, one can define a probability space supporting $S$ and a Brownian motion $B$ and two constants $C>0$, $\lambda > 0$ such that for any $y>0$ sufficiently large,
\begin{equ}\label{equation KMT}
    \PP \left[ \sup_{t\leq n} \left| S_{\lfloor t \rfloor} - B_t \right| > C\log n + y \right] \leq C\e^{-\lambda y}.
\end{equ}
\end{Prop}
\begin{proof}
This is a very classical and fundamental result of strong approximation theory. The original proof can be found in~\cite{KMT1} and~\cite{KMT2}.
\end{proof}
With this result in hand we can mimic the proof of Lemma 14 of ~\cite{conditionallimittheoremsfororderedrandomwalks} plotting the input~\eqref{equation KMT} instead of the weaker approximation given by their Lemma 12.
\begin{proof}[Proof of Lemma~\ref{lemme estimation temps d'arrêt}]
As in~\cite{conditionallimittheoremsfororderedrandomwalks}, we define for any sequence $y$ satisfying~\eqref{equation C1} and~\eqref{equation C2},
\begin{equ}
    y^{\pm} = \left( y_i \pm 2(i-1)\left(\log n\right)^{1+\eps/2}, 1 \leq i \leq r \right).
\end{equ}
Then by considering the event
\begin{equ}
    A = \left\lbrace\sup_{1 \leq i \leq r} \sup_{t\leq n} \left| S^i_{\lfloor t \rfloor} - B^i_t \right| \leq \left( \log n \right)^{1+\frac{\eps}{2}} \right\rbrace,
\end{equ}
we obtain that 
\begin{equ}
    \PP \left[\rho_y > n\right] \leq \PP\left[\rho^{bm}_{y^+} > n\right] + o\left(e^{-\log n \left((\log n)^{\eps/2} - 1 \right)}\right).
\end{equ}
In the same fashion,
\begin{equ}
    \PP \left[ \rho^{bm}_{y^-} > n \right] \leq \PP \left[ \tau_y > n\right] + o\left(e^{-\log n \left((\log n)^{\eps/2} - 1 \right)}\right).
\end{equ}
Now, because of Lemma 13 of the latter reference giving precise estimates for $\PP\left(\rho^{bm}_y > n\right)$, we know that:
\begin{equ}
    \PP\left[\rho^{bm}_{y^\pm} > n\right] = \chi \Delta(y_\pm) n^{-\frac{r(r-1)}{4}}.
\end{equ}
Now, note that $\Delta(y^\pm) = V(y)\left(1+o\left( \left(\log n\right)^{-\eps/2} \right)\right)$ because of~\eqref{equation C2}.
Therefore,  we conclude that:
\begin{eqnarray*}
    \PP \left[\rho_y > n \right] &=& \chi V(y) n^{\frac{r(r-1)}{4}}\left(1 + o\left( \left(\log n\right)^{-\eps/2} \right)\right) + o\left(e^{-\log n \left((\log n)^{\eps/2} - 1 \right)}\right)\\
    &=& \chi V(y) n^{-\frac{r(r-1)}{4}}\left(1+o(1)\right),
\end{eqnarray*}
since because of~\eqref{equation C2} we know that 
\begin{equ}
    \e^{-\log n\left((\log n)^{\eps/2} -1 \right)} = o\left(\chi V(y) n^{-\frac{r(r-1)}{4}}\right).
\end{equ}
The proof of~\eqref{equation majoration lemme estimation temps d'arrêt} goes exactly the same way as in~\cite{conditionallimittheoremsfororderedrandomwalks}.
\end{proof}


\subsubsection{Sketch of the proofs of Theorems~\ref{theoreme local limit srw} and~\ref{theoreme d'invariance sdrw} }\label{subsub Sketch of the proofs of Theorems theoreme d'invariance sdrw and theoreme local limit srw}

The supplementary ingredient used in~\cite{randomwalksincones, Invarianceprinciplesforrandomwalksincones} to extend the local limit theorem over any starting and ending points is the following repulsion lemma:

\begin{Lemma}[Edge repulsion for synchronized random walks]\label{lemme entropic repulsion sdrw}
There exists $\eps>0$ such that the following holds. Let $\left( S^i_n\right)_{n \geq 0  1 \leq i \leq r}$ be a synchronized system of directed random walks. Let
\begin{equ}\label{definition stopping time}
    \eta_n = \min \left\lbrace k \geq 0, \min_{1 \leq i<j \leq r} \left| S^i_k - S^j_k \right| > n^\eps  \right\rbrace
\end{equ}
Then there exists $c>0$ such that for any $x \in \R^r$, when $n$ is sufficiently large,
\begin{equ}
    \PP_{(0,x)}\left[ \eta_n > n^{1-\eps} \right] < \frac{1}{c}
    \exp(-cn^\eps).
\end{equ}
\end{Lemma}
\begin{proof}
This fact has been proved in~\cite[Lemma 7]{conditionallimittheoremsfororderedrandomwalks}. Their version is even stronger than our formulation of the lemma since the $n^{\eps}$ in the definition of $\eta_n$ is replaced by $n^{\frac{1}{2}-\eps}$. Moreover the fact that the result of~\cite{conditionallimittheoremsfororderedrandomwalks} holds only for regular random walks (and not directed ones) does not change anything since $\eta_n$ does not depend of the horizontal coordinate of $S$.
\end{proof}
\begin{Rem}
We observe that since this probability is stretch-exponentially small, the bound is still valid - up to a change in the constant $c$ - when conditioning the synchronized system of random walks on an event of polynomial probability. Then, anticipating in the remainder of this work and using Theorem~\ref{theoreme local limit srw}, we obtain the following Corollary - which is useful in the proof of Lemma~\ref{Lemme entropic repulsion independent system}.
\begin{Cor}\label{cor: repulsion SDRbridges}
    There exists $\eps>0$ such that the following holds. Let $\left( S^i_n\right)_{n \geq 0  1 \leq i \leq r}$ be a synchronized system of directed random walks. Then there exists $c>0$ such that for any $x,y \in \R^r$, when $n$ is sufficiently large,
    \begin{equ}
    \PP_{(0,x)}\left[ \eta_n > n^{1-\eps} \vert \hit_{(n,y)} \right] < \frac{1}{c}
    \exp(-cn^\eps).
\end{equ}
\end{Cor}    
\end{Rem}

We are now able to briefly sketch the proof of Theorem~\ref{theoreme local limit srw}. As explained before, a very similar statement can be found in~\cite{randomwalksincones}. However there are two main differences between the statement we want to prove and the latter. The first one is that there is a supplementary difficulty due to the fact that the horizontal coordinate is random with increments $\theta_k$. The second difficulty is that our local limit theorem needs uniformity in $x,y \in W$ satisfying~\eqref{equation C1} while in the latter reference, the local limit theorem is uniform over all the $x,y$ satisfying  $\min\lbrace \Gap(x), \Gap(y) \rbrace > n^{\frac{1}{2}-\eps}$, which is weaker that what we want. 

We rule out these two problems following the methods of section 5 of~\cite{ottvelenikwachtelioffe}. Since the proof is very similar - the authors consider one single direct random walk conditioned to be nonnegative but these two examples can be unified under the setting of random walks conditioned to stay on cones (see~\cite{randomwalksincones}), we do not claim to do a full proof but rather to show where the proof has to be modified to hold in this setting. The method used is then the method of~\cite[Theorem 5.1]{ottvelenikwachtelioffe}. 

\begin{proof}[Sketch of proof of Theorem~\ref{theoreme local limit srw}]

We start by writing: 
\begin{equ}\label{equation probas totales}
    \PP_{(0,x)}\left[ \hit_{(n,y)}<\rho, \hit_{(n,y)}  \right] = \sum_{k=0}^\infty \PP_{(0,x)} \left[ S_k = (n,y), \rho> k  \right],
\end{equ}
and split the sum in three different ranges for the index $k$. Let us recall that we called $\mu = \E\left[\theta_1\right]$. 

The first range of indexes in~\eqref{equation probas totales} is $ \big| k- \frac{n}{\mu} \big| > \eps n,$
which is easily handled by a large deviations estimate for the first component of the walk. Indeed,
\begin{equ}
\sum_{\left| k- \frac{n}{\mu} \right| > \eps n}\PP_{(0,x)}\left[S_k = (n,y), \rho > k\right] = o(\e^{-c\eps n}) 
\end{equ}
for some constant $c>0$.

The second range of indexes is $\eps n > \left| k -\frac{n}{\mu}\right| > A\sqrt{n}.$ In this regime, the strategy of~\cite{ottvelenikwachtelioffe} is to perform an exponential change of measure and use the following bound, valid for all $u \in W$ satisfying~\eqref{equation C1}, that we proved in Lemma~\ref{lemme estimation temps d'arrêt}:
\begin{equ}
    \PP^{h_{k,n}}_{(0,x)}\left[ \rho > k\right] <c\Delta(u)k^{-\frac{r(r-1)}{4}}.
\end{equ}
The method of~\cite{ottvelenikwachtelioffe} then shows:
\begin{equ}
    \sum_{\eps n > \left| k -\frac{n}{\mu}\right| > A\sqrt{n}}\PP_{(0,x)}\left[S_k = (n,y), \rho > k\right] \leq \frac{V(x)V(y)}{n^{\frac{r^2}{2}}}o_{A \rightarrow \infty}(1)
\end{equ}

Finally we have to deal with the range of indexes  $\left|k - \frac{\mu}{n} \right| < A\sqrt{n}.$ The idea is then to mimic the proofs of~\cite{randomwalksincones} - excepted that we also have to harvest the uniformity in $u$ satisfying~\eqref{equation C1} in~\cite[Lemma 21]{randomwalksincones}.  As in~\cite{ottvelenikwachtelioffe}, the harmonic function for the directed system only depends on the $y$-coordinate of the walk: hence we are left with the same setting, and the same arguments apply. The inputs needed are the fact that $Z_n$ is a martingale, the fact that $V(u) \sim \Delta(u)$ whenever $\Gap(u)\rightarrow \infty$ (see~\cite[Proposition 4, (d)]{conditionallimittheoremsfororderedrandomwalks}) and the normal approximation given by Proposition~\ref{prop kmt coupling}. Finally, exactly as in~\cite{conditionallimittheoremsfororderedrandomwalks}, the assumption~\eqref{equation C2} can be removed using Lemma~\ref{lemme entropic repulsion sdrw}.
\end{proof}

In the exact same manner, proofs of Theorems~\ref{gnedenko theorem} and~\ref{theoreme d'invariance sdrw} follow from the results of~\cite{randomwalksincones} using the method of~\cite{ottvelenikwachtelioffe} as described above.


\subsubsection{Toolbox: two useful lemmas about directed random walks}\label{subsub Two useful lemmas about directed random walks}

This subsubsection is devoted to the proof of two lemmas that have been useful in the proof of repulsion estimates. The first one of them states that a walk stays confined into a tiny deterministic "tube" with only exponentially small probability. The second lemma establishes that "in the bulk" of a non-intersecting system of directed random walks, it is unlikely that any two of them ever come back polynomially close one from each other. 



\begin{Lemma}[Confinement lemma for a single directed random walk]\label{Confinment lemma}
There exists $\eps_0> 0$ such that that the following holds. Fix $\eps<\eps_0$. Let $(S_n)_{n\geq 0}$ be a directed random walk, and remember that $S(t)$ denotes its linear interpolation. Let $f: \R^+ \rightarrow \R$ be any (deterministic) function satisfying the following condition: there exists $g:\R^+ \rightarrow \R$ such that the following simple convergence holds.
\begin{equ}
   \forall t \geq 0, \frac{1}{\sqrt{n}}f(nt)\goes{}{n\rightarrow\infty}{g(t)}.
\end{equ}
Then, for any $\alpha \in (0, 1]$, there exists $c>0$ such that for any $x \in \R$,
\begin{equ}
    \PP_{(0,x)}\left[ \# \left\lbrace k \in \lbrace 0, \dots, n^{1-\eps}\rbrace, \left| S(k) - f(k) \right| < n^\eps \right\rbrace > \alpha n^{1-\eps} \right] < \e^{-cn^\eps}.
\end{equ}
\end{Lemma}

\begin{proof}
This can be seen as a consequence of Donsker's Theorem. Indeed, let us write for $1 \leq k \leq n^{1-3\eps}, t^n_k = kn^{2\eps}$. Call an index $k \in \lbrace 1, \dots, n^{1-3\eps}\rbrace$ $f$-\emph{close} if there exists $\ell \in [t^n_{k-1}, t^n_{k}]$ such that $|S(\ell) - f(\ell)| < n^\eps$. Then observe that 

\begin{multline*}
     \PP_{(0,x)}\left[ \# \left\lbrace k \in \lbrace 0, \dots, n^{1-\eps}\rbrace, \left| S(k) - f(k) \right| < n^\eps \right\rbrace > \alpha n^{1-\eps} \right] \leq \\ \PP_{(0,x)}\left[ \text{ At least }\alpha n^{1-3\eps} \text{ indices are }f\text{-close}  \right].
\end{multline*}
Let $k \in \lbrace 0, \dots, n^{1-\eps}\rbrace$. By the Markov property, one can write 
\begin{equ}
    \PP_{(0,x)}\left[ k \text{ is }f\text{-close} \right] = \PP_{0}\left[ \inf_{t \in [0, n^{2\eps}]}|\tilde{S}(t) - (S(t_k^n) +f(t))| < n^\eps \right],
\end{equ}
where $\tilde{S}(\cdot)$ is started from 0 and independent of $S(t_k^n)$.
By Donsker's Theorem, 
\begin{equ}
    \PP_{0}\left[ \inf_{t \in [0, n^{2\eps}]}|\tilde{S}(t) - (S(t_k^n) +f(t))| < n^\eps \right] \goes{}{n \rightarrow \infty}{\PP\left[\inf_{t \in [0,1]}|B(t)-(Y+g(t))|<1\right]},
\end{equ}
where $Y\sim \mathcal{N}(0,k)$ is independent of the Brownian motion $B$. By independence between $B$ and $Y$, this probability can be upper bounded as follows: 
\begin{equ}
    \PP\left[\inf_{t \in [0,1]}|B(t)-(Y+g(t))|<1\right] \leq \PP\left[Y \in [-1,1]\right].
\end{equ}
Using the trivial majoration $\PP\left[ Y \in [-1,1] \right] \leq \frac{2}{\sqrt{2\pi k}}$, we obtain that 
\begin{equ}
    \PP_{(0,x)}\left[ \# \left\lbrace k \in \lbrace 0, \dots, n^{1-\eps}\rbrace, \left| S(k) - f(k) \right| < n^\eps \right\rbrace > \alpha n^{1-\eps} \right] \leq \binom{n^{1-3\eps}}{\alpha n^{1-3\eps}}\prod_{k=1}^{\alpha n^{1-3\eps}} \frac{2}{\sqrt{2\pi k}}.
\end{equ}
By Stirling's formula, it is easy to see that this quantity decays faster than any exponential of $-n^{1-3\eps}$. In particular, 
\begin{equ}
    \PP_{(0,x)}\left[ \# \left\lbrace k \in \lbrace 0, \dots, n^{1-\eps}\rbrace, \left| S(k) - f(k) \right| < n^\eps \right\rbrace > \alpha n^{1-\eps} \right] \leq \exp(-n^\eps).
\end{equ}
\end{proof}

Let us now state the announced second lemma. 

\begin{Lemma}\label{lemme repulsion bulk}
Let $S$ be a system of directed synchronized random walks. Let $\eps> 0$. Then for any $\delta > 0$ chosen sufficiently small, any points $x,y \in W$ satisfying $\norme{x}, \norme{y} = o(\sqrt{n})$, there exists $\beta > 0, C>0$ such that for any $n \geq 0$ sufficiently large, 
\begin{equ}\label{equation a démontrer bulk rep synchronized rw}
    \PP_{(0,x)} \left[ \exists t \in [n ^\eps, n-n^\eps], \Gap(S(t)) \leq n^\delta \vert S \in \mathcal{W}_n, \hit_{(n,y)} \right]  \leq Cn^{-\beta}.
\end{equ}
\end{Lemma}

\begin{proof} 
First, notice that one can actually examine only integer values of $t$ in~\eqref{equation a démontrer bulk rep synchronized rw} since the minimal distance between two synchronized piecewise linear functions is achieved at a slope change time, which by definition of $S$ is an integer. Let us introduce the following kernel:
\begin{equ}\label{Equation definition kernel}
    q_n(x,y) = \PP_{(0,x)}\left[ S \in \mathcal{W}_n, \hit_{(n,y)} \right].
\end{equ}

Let $i \in \lbrace 1, \dots, r-1\rbrace$. By a basic union bound it is sufficient to prove that
\begin{equ}
    \PP_{(0,x)} \left[ \exists k \in \lbrace n^\eps, \dots, n-n^{\eps} \rbrace , \left|S^{i+1}(k) - S^i(k) \right| \leq n^\delta \vert  S \in \mathcal{W}_n, \hit_{(n,y)} \right] \leq Cn^{-\beta}
\end{equ}
Let us introduce the following subset of $W$:
\begin{equ}
    W_{n,\delta} = \left\lbrace u \in W, \left| u_{i+1}- u_i \right| < n^\delta \right\rbrace.
\end{equ}
We make use of Theorems~\ref{theoreme local limit srw} and~\ref{gnedenko theorem}. Indeed, let us choose $n$ large enough so that for $n^\eps < k < n-n^\eps$, one has that for any $u\in W_{n,\delta}$:
\begin{equ}
    \begin{cases}
        q_n(x,y) &\geq (1-\eps)V(x)V(y)n^{-\frac{r^2}{2}} \\
        q_k(x,u) &\leq 2V(x)\Delta\left(\frac{u}{\sqrt{k}}\right)k^{-\frac{r(r+1)}{4}}\e^{-\frac{\norme{u}^2}{2k}} \\
        q_{n-k}(u,y) &\leq 2V(y)\Delta\left(\frac{u}{\sqrt{n-k}}\right)(n-k)^{-\frac{r(r+1)}{4}}\e^{-\frac{\norme{u}^2}{2(n-k)}}.
    \end{cases}
\end{equ}
Then, a basic union bound yields:
\begin{eqnarray*}
&& \PP_{(0,x)} \left[ \exists k \in \lbrace n^\eps, \dots, n-n^{\eps} \rbrace , \left|S^{i+1}(k) - S^i(k) \right| \leq n^\delta \vert  S \in \mathcal{W}_n, \hit_{(n,y)} \right] \\
&\leq& \sum_{k=n^\eps}^{n-n^{1-\eps}} \sum_{u\in W_{n,\delta}} \frac{q_k(x,u)q_{n-k}(u,y)}{q_n(x,y)} \\
&\leq& \frac{4}{1-\eps}\sum_{k=n^\eps}^{n-n^{1-\eps}} \sum_{u\in W_{n,\delta}} n^\frac{r^2}{2} (k(n-k))^{-\frac{r(r+1)}{4}}\Delta\left(\frac{u}{\sqrt{k}}\right)\Delta\left(\frac{u}{\sqrt{n-k}}\right)\e^{-\frac{\norme{u}^2}{2}\left(\frac{1}{k}+\frac{1}{n-k}\right)}.
\end{eqnarray*}
We make two observations: the first one of them is that this sum is actually symmetric around $\frac{n}{2}$, so that it is sufficient to bound it for $k$ going from $n^\eps$ to $\frac{n}{2}$. The second observation is that since $u\in W_{n,\delta}$, we have:
\begin{equ}
    \Delta\left(\frac{u}{\sqrt{k}}\right) \leq 2\norme{u}^{\frac{r(r-1)}{2}-1}n^\delta k^{-\frac{r(r-1)}{4}}.
\end{equ}
Then,
\begin{multline*}
 \PP_{(0,x)} \left[ \exists k \in \lbrace n^\eps, \dots, n-n^{\eps} \rbrace , \left|S^{i+1}(k) - S^i(k) \right| \leq n^\delta \vert  \hit_{(n,y)}, \tau > H_{(n,y)} \right] \\
\leq \frac{32}{1-\eps}\sum_{k=n^\eps}^{\frac{n}{2}}\left(\frac{n}{k(n-k)}\right)^{\frac{r^2}{2}}n^{2\delta}\underbrace{\sum_{u\in W_{n,\delta}}\norme{u}^{r(r-1)-2}\e^{-\frac{\norme{u}^2}{2k}}}_{I}.
\end{multline*}
We then evaluate the order of the sum $I$. Indeed, let us write:
\begin{equ}
    I = \sum_{\ell\geq 0}\sum_{\substack{u\in W_{n,\delta} \\ \norme{u}=\ell}} \ell^{r(r-1)-2}\e^{-\frac{\ell^2}{2k}} \lesssim \sum_{r\geq 0} n^\delta \ell^{r-2}\ell^{r(r-1)-2}\e^{-\frac{\ell^2}{2k}},
\end{equ}
where we have used the fact that when $\ell \rightarrow \infty$, if $B_\ell(0)$ denotes the $\norme{\cdot}$ ball of $\R^r$ centered at 0 and of radius $\ell$, then
\begin{equ}\label{equation estimation volume twisted weyl chamber}
    \left| W_{n,\delta} \cap \partial B_\ell(0) \right| \lesssim n^\delta \ell^{r-2}.
\end{equ}
We then compare the latter sum with the integral $\int_{x=0}^\infty x^{r^2-4}e^{-\frac{x^2}{2k}}\dif x,$
which after the change of variables $t = \frac{x^2}{2k}$, can be explicitly evaluated:
\begin{equ}
    \int_{x=0}^\infty x^{r^2-4}e^{-\frac{x^2}{2k}}\dif x = \sqrt{2}^{r^2-5}\Gamma\left( \frac{r^2-3}{2}\right)k^{\frac{r^2-3}{2}}.
\end{equ}
Plotting this input in our previous computation yields 
\begin{align*}
 \PP_{(0,x)} \Big[ \exists k \in \lbrace n^\eps, \dots, n-n^{\eps} \rbrace ,& \left|S^{i+1}(k) - S^i(k) \right| \leq n^\delta \vert  S \in \mathcal{W}_n, \hit_{(n,y)} \Big] \\
&\leq C\sum_{k=n^\eps}^{n/2}\left( \frac{n}{k(n-k)}\right)^\frac{r^2}{2}n^{3\delta}k^{\frac{r^2-3}{2}} \\
&\leq Cn^{3\delta}\sum_{k=n^\eps}^\infty k^{-\frac{3}{2}} \\
&\leq Cn^{3\delta}n^{-\frac{\eps}{2}}.
\end{align*}
Hence, whenever $\delta<\frac{\eps}{6}$, this probability decays polynomially, as announced.
\end{proof}

In the proofs of Sections~\ref{section independent system} and~\ref{section RCM}, we used this lemma under a slightly different form that we state now. 

\begin{Lemma}\label{Lemme repulsion globale sdry}
Let $S$ be a directed system of synchronized random walks. Let $x_n, y_n$ two sequences of elements of $W$ such that 
\begin{equ}
    \min\lbrace\Gap(x_n), \Gap(y_n)\rbrace \geq  n^\eps
\end{equ}
and
\begin{equ}
    \norme{x_n}, \norme{y_n} = o(\sqrt{n}).
\end{equ}
Then for any $\delta > 0$ sufficiently small,, there exists $\beta > 0$, $C> 0$ such that for $n\geq 0$ large enough, 
\begin{equ}
    \PP_{(0,x_n)} \left[ \inf_{0 \leq t \leq n }\Gap(S(t)) \leq n^\delta~\big\vert ~S \in \mathcal{W}_n, \hit_{(n,y_n)}\right] \leq Cn^{-\beta}.
\end{equ}
\end{Lemma}

\begin{proof}
All the work has been done in Lemma~\ref{lemme repulsion bulk}. Indeed, we already know that 
\begin{equ}
    \PP_{(0,x_n)} \left[ \inf_{n^\eps\leq k\leq n-n^\eps}\Gap(S(t)) \leq n^\delta~\big\vert ~S \in \mathcal{W}_n, \hit_{(n,y_n)}\right] \leq Cn^{-\beta}.
\end{equ}
It remains to control the range of indexes $k \in \lbrace 1, \dots, n^\eps\rbrace \cup \lbrace n-n^\eps, \dots, n\rbrace$ (observe that we cannot make use of the local limit theorems in this range). However it is a basic large deviations estimate: let us write it for $k \in \lbrace 0,\dots, n^\eps \rbrace$.  We roughly bound 
\begin{multline}
 \PP_{(0,x_n)} \left[ \inf_{0\leq k\leq n^\eps} \Gap(S(k)) \leq n^\delta~\big\vert ~S \in \mathcal{W}_n, \hit_{(n,y_n)}\right] \\\leq \frac{\PP_{(0,x_n)} \left[\inf_{0\leq k\leq n^\eps} \Gap(S(k)) \leq n^\delta\right]}{\PP_{(0,x_n)} \left[S \in \mathcal{W}_n, \hit_{(n,y_n)}\right]}.
\end{multline}
Now observe that for the event of the numerator to occur, one of the walks has to travel at a distance at least $\frac{1}{2}(n^\eps- n^\delta)$ of its starting point in a time $n^\eps$ which by large deviations occurs with stretched exponentially small probability as soon as $\delta < \eps$. However by Theorem~\ref{theoreme local limit srw} the denominator is of order at most polynomial, and this concludes the proof. 
\end{proof}

\subsection{Synchronized systems of random walks with random decorations}\label{Subsection synchronized systems of random walks with random decorations}

For convenience, we shall need to study synchronized systems of random walks with \textit{random decorations}. These decorations will not be specified, since we will use these results in different settings: for the first application of the result that we will prove in this section, the random decorations will be other steps of random walks (see Subsection~\ref{subsection non intersecting non-synchronized rw}), but we shall also use this result with the decorations being random pieces of subcritical FK-clusters. The important feature of these decorations are that they shall be confined into \textit{diamonds} of finite volume surrounding the increments of the walks. Recall the definitions of cones and diamonds of Section~\ref{section review oz}.


Let $\left(S_n\right)_{n \geq 0}  = \left( T_n, Z_n \right)_{n \geq 0}$ be a system of synchronized random walks. For any $1 \leq i \leq r$, $k\geq 0$ and $\delta > 0$, we define 
\begin{equ}
    \mathcal{D}^{\delta}_{i,k} = \mathcal{D}^\delta_{\left(T_k, S^i_k\right), \left(T_{k+1}, S^i_{k+1}\right)}.
\end{equ}
and 
\begin{equ}
    \mathcal{D}(S^i) = \bigcup_{k \geq 0} \mathcal{D}_{i,k}^\delta.
\end{equ}


The crucial result of this section is the following lemma - adapted from~\cite[Lemma 2.7]{Asymptoticsofeven-evencorrelationsintheIsingmodel}

\begin{Lemma}\label{lemme decorated random walks}
Let $\delta > 0$, and $x,y \in W$. Then, there exists $c>0$ such that:
\begin{equ}
  \PP_{(0,x)}\left[S \in \hit_{(n,y)}, \bigcap_{1 \leq i \neq j \leq r} \left\lbrace \mathcal{D}(S^i) \cap \mathcal{D}(S^j) = \emptyset \right\rbrace \right] >   c\PP_{(0,x)} \left[S \in \mathcal{W}_n, \hit_{(n,y)}\right] .
\end{equ}
\end{Lemma}
\begin{proof}
We need to prove that there exists some $c<1$ such that
\begin{equ}
\PP_{(0,x)}\left[ \exists 1 \leq i< j \leq r, \mathcal{D}(S^i) \cap \mathcal{D}(S^j) \neq \emptyset \vert S \in \mathcal{W}_n, \hit_{(n,y)}\right] \leq c.
\end{equ}
By the union bound, the latter probability is lesser or equal than 
\begin{equ}
    \sum_{1 \leq i \leq r-1}\PP_{(0,x)}\left[ \mathcal{D}(S^i) \cap \mathcal{D}(S^{i+1}) \neq \emptyset \vert 
    S \in \mathcal{W}_n, \hit_{(n,y)}  \right],
\end{equ}
and we now focus on the terms of this sum. Let us introduce the following family of events (recall that $\theta_k = T_{k+1}-T_k$):
\begin{equ}
    \mathcal{L}_k = \left\lbrace \left| S^{i+1}_k - S^i_k \right| < 2\delta \theta_{k+1} \right\rbrace.
\end{equ}
Observe that due to the cone-confinement property, if $\left\lbrace \mathcal{D}(S^{i+1}) \cap \mathcal{D}(S^i) \neq \emptyset \right\rbrace $, then one of the $\mathcal{L}_k$ must occur. 
Let $T>0$ be a large integer, that will be fixed later. We first argue that there exists a constant $c_1>0$ which only depends on $\delta$ such that
\begin{equ}
    \PP_{(0,x)}\left[ \bigcup_{k=1}^n \mathcal{L}_k \vert S \in \mathcal{W}_n, \hit_{(n,y)} \right] \leq \e^{+c_1T}\PP_{(0,x)}\left[ \bigcup_{k=T}^{n-T} \mathcal{L}_k\vert S \in \mathcal{W}_n, \hit_{(n,y)}\right].
\end{equ}
This is a finite-energy property, the fact that $c_1$ is uniform over $T$ comes from the cone-confinement property. Then by union bound, let us write:
\begin{eqnarray*}
  && \PP_{(0,x)}\left[\bigcup_{k=T}^{n-T} \mathcal{L}_k\vert S \in \mathcal{W}_n, \hit_{(n,y)} \right] 
  \\&\leq& \sum_{k=T}^{n-T} \PP_{(0,x)}\left[ \mathcal{L}_k \vert S \in \mathcal{W}_n, \hit_{(n,y)}\right]\\
  &\leq& \sum_{k=T}^{n-T}\sum_{\ell = 1}^{n-k} \PP_{(0,x)} \left[ \mathcal{L}_k, \theta_{k+1} = \ell \vert S \in \mathcal{W}_n, \hit_{(n,y)}\right] \\
  &\leq& \sum_{k=T}^{n-T}\sum_{\ell = 1}^{n-k} \PP_{(0,x)} \left[ \left| S^{i+1}_k - S^i_k \right| < 2\delta\ell, \theta_{k+1} = \ell \vert S \in \mathcal{W}_n, \hit_{(n,y)} \right] \\
  &\leq& \sum_{k=T}^{n-T} \sum_{\ell = 1}^{n-k}\sum_{u \in W_{\delta\ell}}\sum_{v \in W} \e^{-c_1\ell}\e^{-c_2\norme{u-v}}\frac{q_k(x,u)q_{n-l-k}(v,y)}{q_n(x,y)},
\end{eqnarray*}
where as in the proof of Lemma~\ref{Lemme repulsion globale sdry}, we have introduced the kernel 
\begin{equ}
    q_n(x,y) = \PP_{(0,x)}\left[ S \in \mathcal{W}_n, \hit_{(n,y)} \right],
\end{equ}
and the notation
\begin{equ}
    W_{\delta\ell} = \left\lbrace u \in W, \left|u_{i+1}-u_i\right| < \delta\ell \right\rbrace.
\end{equ}
Moreover, we also used the property that both the random variables $\theta_k$ and $\check{X}_k$ have an exponential moment. Let us now make use of Theorems~\ref{theoreme local limit srw} and~\ref{gnedenko theorem} to write, choosing $T>0$ large enough (still uniformly of everything else):

\begin{eqnarray*}
 && \PP_{(0,x)}\left[\bigcup_{k=T}^{n-T} \mathcal{L}_k\vert S \in \mathcal{W}_n, \hit_{(n,y)} \right] \\
  &\leq& \frac{2}{1-\eps} \sum_{k=T}^{n/2}\sum_{l=1}^{n-k}\sum_{u \in W_{\delta\ell}}\sum_{v \in W}\e^{-c_1\ell}\e^{-c_2\norme{u-v}} n^{\frac{r^2}{2}}(k(n-k))^{-\frac{r(r+1)}{4}}V\left( \frac{u}{\sqrt{k}}\right)\\&\times&V\left(\frac{v}{\sqrt{n-k}}\right)\e^{-\frac{1}{2}\left(\frac{\norme{u}^2}{k}+\frac{\norme{v}^2}{(n-l-k)}\right)} \\
  &\leq& \frac{2}{1-\eps} \sum_{k=T}^{n/2}n^{\frac{r^2}{2}}(k(n-k))^{-\frac{r(r+1)}{4}-\frac{r(r-1)}{4}}\sum_{\ell=1}^{n-k}\e^{-c_1\ell}(\delta\ell)^2\sum_{u \in W_{\delta\ell}}\sum_{v \in W}(\norme{u}\norme{v})^{\frac{r(r-1)}{2}-1}\\&\times& \e^{-\frac{1}{2}\left(\frac{\norme{u}^2}{k}+\frac{\norme{v}^2}{(n-l-k)}\right)}\e^{-c_2\norme{u-v}}\\
  &\leq&\frac{2C}{1-\eps}\sum_{k=T}^{n/2}n^{\frac{r^2}{2}}(k(n-k))^{-\frac{r(r+1)}{4}-\frac{r(r-1)}{4}}\sum_{\ell=1}^{n-k}\e^{-c_1\ell}(\delta\ell)^2\underbrace{\sum_{u \in W_{\delta\ell}}\norme{u}^{r(r-1)-2}\e^{-\frac{\norme{u}^2}{2k}}}_{I}.
\end{eqnarray*}
Let us now estimate the sum $I$. Here, we will use crucially the fact that we sum over $W_{\delta\ell}$ and not over $W$. We write
\begin{eqnarray*}
I &=& \sum_{s \geq 0}\sum_{\substack{u \in W_{\delta\ell} \\ \norme{u}=s}}s^{r(r-1)-2}\e^{-\frac{r^2}{2k}} \\
&\leq&C\delta\ell \sum_{s\geq 0}s^{r-2}s^{r(r-1)-2}e^{-\frac{r^2}{2k}}.
\end{eqnarray*}
We used once again the estimation~\eqref{equation estimation volume twisted weyl chamber} for the volume of the set we are summing over. As before, we compare this sum to the integral
$I_2 = \int_{x=0}^\infty x^{r^2-4}\e^{-\frac{x^2}{2k}}\dif x$,
which, after the appropriate change of variables $t = \frac{x^2}{2k}$, can be explicitly computed, yielding 
\begin{equ}
    I_2 = \sqrt{2}^{r^2-5}\Gamma(\frac{r^2-3}{2})k^{\frac{r^2-3}{2}}.
\end{equ}
Continuing our previous computation, we obtain that:
\begin{align*}
    \PP_{(0,x)}\Big[\bigcup_{k=T}^{n-T} \mathcal{L}_k\vert S \in \mathcal{W}_n, &\hit_{(n,y)}\Big] \\ &\leq \frac{2\widetilde{C}}{1-\eps}\sum_{k=T}^{n/2}\left( \frac{n}{k(n-k)} \right)^\frac{r^2}{2} k^{\frac{r^2-3}{2}}\sum_{\ell=1}^{n-k} \e^{-c_1\ell}(\delta\ell)^3 \\
    &\leq\frac{2\widetilde{C}}{1-\eps}\sum_{k=T}^{n/2} k^{-\frac{r^2}{2} + \frac{r^2}{2} - \frac{3}{2}} \\
    &\leq \frac{2\widetilde{C}}{1-\eps} T^{-\frac{1}{2}}.
\end{align*}
Chose $T>0$ large enough so that quantity is smaller than $\frac{1}{2}$.
We then showed that:
\begin{equ}
     \PP \left[ \bigcap_{k=1}^n \mathcal{L}_k^c \vert S \in \mathcal{W}_n, \hit_{(n,y)}\right] \geq \frac{1}{2}\e^{-c_1T},
\end{equ}
which conclude the proof, since $T>0$ has been chosen uniformly of $n$.
\end{proof}

\subsection{Non-intersecting systems of non-synchronized directed random walks}\label{subsection non intersecting non-synchronized rw}

The goal of this section is to transmit the result of Section~\ref{subsection non-intersecting systems of synchronized directed random walks} to the setting of \emph{non-synchronized} random walks. For that, we will interpret such a system as an embedded synchronized random walk carrying random decorations, and use the results of the precedent section. 

Before diving into the proof, we introduce the "embedded system of synchronized random walks" of a system of random walks. 

\begin{Def}[Embedded system of synchronized random walks]
Let  $\left(\mathbf{S}_n\right)= \left( \mathbf{T}_n, \mathbf{Z}_n \right)$ be a system of non-synchronized directed random walks. We introduce the random set of synchronization times:
\begin{equ}
    \mathsf{ST} = \left\lbrace \ell\geq 0, \exists k_1(\ell), \dots, k_p(\ell) \geq 0, \mathbf{T}_{k_1}^1 = \dots = \mathbf{T}_{k_p}^r = \ell \right\rbrace.
\end{equ}
Writing $\mathsf{ST} = \left\lbrace \ell_1 < \dots < \ell_r < \dots  \right\rbrace$, we define the "embedded system of synchronized random walks" to be the process:
\begin{equ}
   \left( \check{\mathbf{S}}_n\right)_{n \geq 0} = \left( \ell_n, \mathbf{Z}^1_{k_1(\ell_n)}, \dots, \mathbf{Z}^r_{k_p(\ell_n)} \right)_{n \geq 0}.
\end{equ}
Observe that in particular the trajectory of $\check{\mathbf{S}}$ is a subset of the trajectory of $S$, and that by definition the system $\check{\mathbf{S}}$ is synchronized. 
\end{Def}

It is clear that $\check{\mathbf{S}}$ has the distribution of a synchronized system of walks, the exponential tails of the steps being a consequence of the renewal theorem~\cite{feller15}.  

\begin{Lemma}\label{lemme estimation proba non-synchronized RW}
There exists a positive $c>0$ such that for any fixed $x,y \in W$,
\begin{equ}
    \mathbf{P}_{(0,x)}\left[\mathbf{S} \in \mathcal{W}_n, \hit_{(n,y)}\right] > c\frac{V(x)V(y)}{n^{\frac{r^2}{2}}}.
\end{equ}
\end{Lemma}
\begin{proof}
The idea is to see a sample $\mathbf{S}$ of $\mathbf{P}_{(0,x)}$ as a synchronized system of random walks $\check{\mathbf{S}}$, decorated with finite portions of directed random walks staying confined in finite volume diamonds with high probability. Then, we will argue that the probability that $\mathbf{S}$ is non-intersecting is lower bounded by the probability that the diamonds of $\check{\mathbf{S}}$ do not intersect, and then to conclude using Lemma~\ref{lemme decorated random walks} and the local limit Theorem~\ref{theoreme local limit srw} for synchronized random walks.

Indeed, observe that (the first line is due to Assumption 1 that the increments of the walk almost surely lie in the cone $\mathcal{Y}_0^{+,\delta}$):
\begin{eqnarray*}
\mathbf{P}_{(0,x)}\big[\mathbf{S} \in \mathcal{W}_n, \hit_{(n,y)}\big] &\geq& \PP_{(0,x)}\left[ \check{\mathbf{S}}\in \hit_{(n,y)}, \bigcap_{1\leq i\neq j \leq r}\left\lbrace \mathcal{D}(\check{\mathbf{S}}^i) \cap \mathcal{D}(\check{\mathbf{S}}^j) = \emptyset \right\rbrace \right]\\
&\geq& c\PP_{(0,x)}\left[\check{\mathbf{S}} \in  \mathcal{W}_n, \hit_{(n,y)}\right] \\
&\geq& cV(u)V(v)n^{-\frac{r^2}{2}},
\end{eqnarray*}
where the second inequality comes from Lemma~\ref{lemme decorated random walks} and the third one comes from the fact that $\check{S}$ has the distribution of a synchronized system of random walks, so that Theorem~\ref{theoreme local limit srw} applies. 
\end{proof}

\begin{Rem}
The exact same technique of proof can be used to show an analog of Lemma~\ref{Lemme repulsion globale sdry} for non-synchronized random walks. Indeed the probability of two non-synchronized random walks coming close one from each other can be upper bounded by the probability of two decorated synchronized random walks coming close one from each other. Making use of Lemmas~\ref{Lemme repulsion globale sdry} and~\ref{lemme estimation proba non-synchronized RW} we obtain:
\begin{Lemma}\label{lemme repulsion globale non-synchronized RW}
Let $\mathbf{S}$ be a system of non-synchronized random walks. Let $x, y$ two sequences of elements of $W$ such that 
\begin{equ}
    \Gap(x), \Gap(y) \geq  n^\eps
\end{equ}
and
\begin{equ}
    \norme{x}, \norme{y} = o(\sqrt{n}).
\end{equ}
Then for any $\delta > 0$ sufficiently small,, there exists $\beta > 0$, $C> 0$ such that for $n\geq 0$ large enough, 
\begin{equ}
    \mathbf{P}_{(0,x)} \left[ \inf_{1\leq k\leq n} \Gap(\mathbf{S}_k) \leq n^\delta~\big\vert ~\mathbf{S} \in \mathcal{W}_n, \hit_{(n,y)}\right] \leq Cn^{-\beta}.
\end{equ}
\end{Lemma}


\end{Rem}

The next step in our way to the proof of Theorem~\ref{theoreme invariance principle for drw} is then to show the repulsion estimate stated in Lemma~\ref{lemme entropic repulsion sdrw} in the setting of non-synchronized systems of non-intersecting bridges. Let $\eps > 0$. As in Sections~\ref{section independent system} and~\ref{section RCM} we introduce the following times:
\begin{equ}
    T_1(\mathbf{S}) = \min_{k \geq 0} \left\lbrace k \geq 0, \Gap(\mathbf{S}_k) > n^\eps  \right\rbrace 
\end{equ}
and
\begin{equ}    
    T_2(\mathbf{S}) = \max_{k \geq 0} \left\lbrace k \geq 0, \Gap(\mathbf{S}_k) > n^\eps  \right\rbrace .
\end{equ}

\begin{Lemma}\label{entropic repulsion non-synchronized RB}

There exists $\eps >0$ sufficiently small such that there exists a positive constant $c>0$ such that:
\begin{equ}
    \mathbf{P}_{(0,x)} \left[ T_1(\mathbf{S}) > n^{1-\eps}, T_2(\mathbf{S}) < n-n^{1-\eps} \vert \mathbf{S} \in \mathcal{W}_n, \hit_{(n,y)}\right] < \frac{1}{c}\exp\left(-cn^\eps\right),
\end{equ}
where as usual $\mathbf{P}_{(0,x)}$ is the distribution of a non-synchronized system of directed random walks. 
\end{Lemma}
\begin{proof}
As usual, denote by $\check{\mathbf{S}}$ the synchronized system of random walks embedded in $\mathbf{S}$. Then, 
\begin{align*}
 \mathbf{P}_{(0,x)} \big[ T_1(\mathbf{S}) > n^{1-\eps}, & T_2(\mathbf{S}) < n-n^{1-\eps} \vert  \mathbf{S} \in \mathcal{W}_n, \hit_{(n,y)}\big] \\
&\leq \mathbf{P}_{(0,x)} \left[ T_1(\check{\mathbf{S}}) > n^{1-\eps}, T_2(\check{\mathbf{S}}) < n-n^{1-\eps} \vert  \mathbf{S} \in \mathcal{W}_n, \hit_{(n,y)}\right] \\
&\leq 2 \frac{\mathbf{P}_{(0,x)}\left[T_1(\check{\mathbf{S}}) > n^{1-\eps}  \right]}{\mathbf{P}_{(0,x)}\left[\mathbf{S} \in \hit_{(n,y)},\boldsymbol{\rho}>n\right]} \\
&\leq \frac{2}{c}\exp(-cn^\eps)n^{\frac{r^2}{2}}(V(x)V(y))^{-1},
\end{align*}
which proves the lemma for another constant $c'<c$ provided that $n$ is large enough. The last inequality comes from Lemma~\ref{lemme entropic repulsion sdrw}.
\end{proof}
We are now ready to prove Theorem~\ref{theoreme invariance principle for drw}. The technique is very similar to the proof of Theorem~\ref{Theoreme main}. Indeed, we shall wait for a  sublinear time that the walks attain a gap of order $n^\eps$. After this time, we know that - looking at the process as a system of synchronized decorated random walks - the diamonds are very likely not to intersect so that the convergence of the synchronized embedded system towards the Brownian watermelon can be transmitted to the whole system.  
\begin{proof}[Proof of Theorem~\ref{theoreme invariance principle for drw}]
Let $\mathbf{S}$ be sampled according to the measure
\begin{equ}
\mathbf{P}_{(0,x)}\left[ ~.~\big\vert \mathbf{S} \in \mathcal{W}_n, \hit_{(n,y)}\right].
\end{equ}
Again, since we are going to work between the random times $T_1$ and $T_2$, we need to implement the strategy given by Lemma~\ref{Lemme technique processus sto}. Let $\delta > 0$ and $f^\delta : \mathcal{C}([\delta, 1-\delta], \R^r) \rightarrow \R$, continuous and bounded. We also introduce $\mathbf{S}_n(t)$ the scaled version of $\mathbf{S}$:
\begin{equ}
    \mathbf{S}_n(t)= \frac{1}{\sqrt{n}}\mathbf{S}(nt).
\end{equ}
Our goal is to show that (as usual we keep implicit the restrictions of $\mathbf{S}_n$ and $\bw^{(r)}$ to the interval $[\delta, 1-\delta]$):
\begin{equ}
    \E\left[f^\delta(\mathbf{S}_n) \vert \mathbf{S} \in \mathcal{W}_n, \hit_{(n,y)} \right] \goes{}{n \rightarrow \infty}{\E\left[f^\delta(\sigma\bw^{(r)})\right]}.
\end{equ}

As usual, we claim that - thanks to Lemma~\ref{entropic repulsion non-synchronized RB} and the usual deviation argument for random walks - with probability $1+o(1)$, there exist $T_1>0$ and $T_2< n$ two random times such that $T_1$ and $T_2$ are synchronization times for $\mathbf{S}$, and such that 
\begin{equ}\label{conditions t1 t2 marches}
    T_1 < 2n^{1-\eps} \text{ and  } T_2 > n - 2n^{1-\eps},
\end{equ}
and 
\begin{equ}\label{condition ecartement marches aleatoires}
    \begin{cases}
    \norme{\mathbf{S}(T_1)}, \norme{\mathbf{S}(T_2)} = o(\sqrt{n}), \\
        \min\lbrace \Gap(\mathbf{S}(T_1)), \Gap(\mathbf{S}(T_2)) \rbrace > \frac{1}{2}n^\eps.
    \end{cases} 
\end{equ}
In the rest of the proof, we then condition on the values of $T_1, T_2, \mathbf{S}(T_1)$ and $\mathbf{S}(T_2)$ satisfying~\eqref{conditions t1 t2 marches} and~\eqref{condition ecartement marches aleatoires}. Moreover for sake of simplicity in the proof let us call $u = (T_1, \mathbf{S}(T_1))$ and $v=(T_2,\mathbf{S}(T_2))$. As soon as $n$ is large enough so that $n\delta>T_1$ and $n(1-\delta) < T_2$, the Markov property for random walks ensures that:

\begin{multline}
    \E\Big[f^\delta(\mathbf{S}_n) \vert u,v, \mathbf{S} \in \mathcal{W}_n, \hit_{(n,y)} \Big] \\ = \E_{u}\Big[ f^\delta\Big(\mathbf{S}_n\Big(\min\Big(t-\tfrac{T_1}{n}, \tfrac{T_2}{n}\Big)\Big)\Big) \vert S \in \hit_{v}, \mathcal{W}_{T_2-T_1} \Big],
\end{multline}
where $\E_u$ denotes the expectation under the measure $\mathbf{P}_{u}$. 


Let us consider, as usual, $\check{\mathbf{S}}$ to be the synchronized system embedded into $\mathbf{S}$, and $\check{\mathbf{S}}(t)$ be its linear interpolation. By standard estimates on the max of a linear number of independent random variables with exponential tails, one gets: 
\begin{align*}
    \mathbf{P}_{(0,x)} \Big[ \sup_{0\leq t \leq n}\Big| \mathbf{S}(t) - \check{\mathbf{S}}(t) \Big| & > \log^2 n \vert \mathbf{S}\in \mathcal{W}_n, \hit_{(n,y)}  \Big] \\ &\leq \frac{\mathbf{P}_{(0,x)} \left[\sup_{0\leq t \leq n}\left| \mathbf{S}(t) - \check{\mathbf{S}}(t) \right| > \log^2 n\right]}{\mathbf{P}_{(0,x)}\left[\mathbf{S} \in \mathcal{W}_n, \hit_{(n,y)}\right]} \\
    &\leq \frac{1}{c}\exp\left(-c(\log^2 n)\right)(V(x)V(y))^{-1}n^{\frac{r^2}{2}},
\end{align*}
where we used Lemma~\ref{lemme estimation proba non-synchronized RW} for the last step. We now work under the event that
\begin{equ}
    \sup_{0\leq t \leq n}\left| \mathbf{S}(t) - \check{\mathbf{S}}(t) \right| > \log^2 n.
\end{equ}
Hence, for our purpose it is sufficient to show that:
\begin{equ}
    \E_{u}\big[ f^\delta\big(\check{\mathbf{S}}_n\big(\min\big(t-\tfrac{T_1}{n}, \tfrac{T_2}{n}\big)\big)\big) \vert \mathbf{S} \in \hit_{v}, \mathcal{W}_{T_2-T_1} \big]  \goes{}{n\rightarrow \infty}{\E\left[f^\delta(\sigma\bw^{(r)})\right]}.
\end{equ}
The next step is to replace the conditioning over $\mathbf{S}$ belonging to the non-intersection and connection event by a conditioning over $\check{\mathbf{S}}$ belonging to this event. Indeed, assuming that we managed to show that this change of conditioning was justified, the result would follow by Theorem~\ref{theoreme d'invariance sdrw}.
Our target estimate is then:
\begin{equ}
    \mathbf{P}_u\left[ \left\lbrace \mathbf{S} \in \mathcal{W}_{T_2-T_1} \right\rbrace \Delta \left\lbrace \check{\mathbf{S}}\in \mathcal{W}_{T_2-T_1} \right\rbrace \vert \check{\mathbf{S}}\in \mathcal{W}_{T_2-T_1}, \hit_v \right]  \goes{}{n \rightarrow \infty}{0}.
\end{equ}
Observe that because we work under the event $\lbrace\sup_{0\leq t \leq n}\left| \mathbf{S}(t) - \check{\mathbf{S}}(t) \right| > \log^2 n \rbrace$, then
\begin{multline}
  \mathbf{P}_u\left[ \left\lbrace \mathbf{S} \in \mathcal{W}_{[T_2-T_1]} \right\rbrace \Delta \left\lbrace \check{\mathbf{S}}\in \mathcal{W}_{T_2-T_1} \right\rbrace \right] \\ \leq \mathbf{P}_u \big[ \inf_{t \in [T_1,T_2]} \Gap(\check{\mathbf{S}}(t)) < \log^3 n \vert \check{\mathbf{S}}\in\mathcal{W}_{[T_1, T_2]}, \hit_v\big]. 
 \end{multline}
 By Lemma~\ref{Lemme repulsion globale sdry} we know that this probability decays to 0 at least polynomially fast. Thus, we proved that 
 \begin{multline}\label{equivalence des conditionnements preuve marches}
     \Big|\E_{u}\Big[ f^\delta\big(\check{\mathbf{S}}_n\big(\min\big(t-\tfrac{T_1}{n}, \tfrac{T_2}{n}\big)\big)\big) \vert \mathbf{S} \in\mathcal{W}_{[T_1, T_2]}, \hit_v \Big] \\-  \E_{u}\big[ f^\delta\big(\check{\mathbf{S}}_n\big(\min\big(t-\tfrac{T_1}{n}, \tfrac{T_2}{n}\big)\big)\big) \vert \check{\mathbf{S}}\in\mathcal{W}_{[T_1, T_2]}, \hit_v \big] \Big|  \goes{}{n\rightarrow \infty}{0}.
 \end{multline}
 Now because of~\eqref{conditions t1 t2 marches} and~\eqref{condition ecartement marches aleatoires}, Theorem~\ref{theoreme d'invariance sdrw} applies and we get that
 \begin{equ}
     \E_{u}\big[ f^\delta\big(\check{\mathbf{S}}_n\big(\min\big(t-\tfrac{T_1}{n}, \tfrac{T_2}{n}\big)\big)\big) \vert \check{\mathbf{S}}\in\mathcal{W}_{[T_1, T_2]}, \hit_v \big]  \goes{}{n\rightarrow \infty}{\E\left[f^\delta(\sigma\bw^{(r)})\right]}
 \end{equ}
 for some $\sigma> 0$. This concludes the proof of the theorem: as usual condition $(i)$ of Lemma~\ref{Lemme technique processus sto} is a simple consequence of the central limit Theorem.
 \end{proof}
\subsection{Non-intersecting system of decorated non-synchronized directed random walks}
This last subsection is devoted to show that a system of non-synchronized random bridges carrying $\delta$-diamonds around its steps has - up to some finite constant - the same probability of being non-intersecting as the system obtained when removing those diamonds. Recall the notation $\mathcal{D}(\mathbf{S})$ for the diamond envelope associated to some directed walk $\mathbf{S}$.

\begin{Lemma}\label{lemme lower bound int con decorated random walks}
    There exists some constant $c>0$ such that for any fixed $x,y \in W$,
    \begin{equ}
        \mathbf{P}_{(0,x)}\Big[\bigcap_{1\leq i\neq j \leq r}\lbrace \mathcal{D}(\mathbf{S}^i) \cap \mathcal{D}(\mathbf{S}^j) = \emptyset \rbrace, \hit_{(n,y)}\Big] \geq cV(x)V(y)n^{-\frac{r^2}{2}}.
    \end{equ}
\end{Lemma}
\begin{proof}
    The proof follows the same line as the proof of Lemma~\ref{lemme estimation proba non-synchronized RW}. Indeed, consider $\check{\mathbf{S}}$ the synchronized random walk embedded in $\mathbf{S}$. By the structure of the diamond decomposition, one has for any $1\leq i \leq r$,
    \begin{equ}
        \mathcal{D}(\mathbf{S}^i) \subset \mathcal{D}(\check{\mathbf{S}}^i).
    \end{equ}
Thus,
\begin{align*}
     \mathbf{P}_{(0,x)}\Big[\bigcap_{1\leq i\neq j \leq r}\lbrace \mathcal{D}(\mathbf{S}^i) \cap \mathcal{D}(\mathbf{S}^j) &= \emptyset \rbrace, \hit_{(n,y)}\Big] \\ &\geq \mathbf{P}_{(0,x)}\Big[\bigcap_{1\leq i\neq j \leq r}\lbrace \mathcal{D}(\check{\mathbf{S}}^i) \cap \mathcal{D}(\check{\mathbf{S}}^j) = \emptyset \rbrace, \hit_{(n,y)}\Big] \\
    &\geq c\mathbf{P}_{(0,x)}\left[ \check{\mathbf{S}} \in \mathcal{W}_n, \check{\mathbf{\rho}} > n \right] \\
    &\geq cV(x)V(y)n^{-\frac{r^2}{2}},
\end{align*}
where the third line follows by Lemma~\ref{lemme decorated random walks}, and the last line is a consequence of the local limit theorem. 
\end{proof}


\bibliographystyle{alpha}
\bibliography{biblio}

\end{document}
