\documentclass[conference]{IEEEtran}
\IEEEoverridecommandlockouts
% The preceding line is only needed to identify funding in the first footnote. If that is unneeded, please comment it out.
\usepackage{cite}
\usepackage{amsmath,amssymb,amsfonts}
\usepackage{algorithmic}
\usepackage{graphicx}
\usepackage{textcomp}
\usepackage{xcolor}
\usepackage{subfigure}
\usepackage{float}

%\def\BibTeX{{\rm B\kern-.05em{\sc i\kern-.025em b}\kern-.08em
%    T\kern-.1667em\lower.7ex\hbox{E}\kern-.125emX}}
\begin{document}

\title{A Phase-Coded Time-Domain Interleaved OTFS Waveform with Improved Ambiguity Function\\
}

% \author{\IEEEauthorblockN{Jiajun Zhu\textsuperscript{1}, Chi Zhang\textsuperscript{1}, Haoran Yin\textsuperscript{1}, Jiaojiao Xiong\textsuperscript{1}, Chao Yang\textsuperscript{2}, Yanqun Tang\textsuperscript{1, *}}
\author{\IEEEauthorblockN{Jiajun Zhu\textsuperscript{1}, Yanqun Tang\textsuperscript{1*}, Chao Yang\textsuperscript{2}, Chi Zhang\textsuperscript{1}, Haoran Yin\textsuperscript{1}, Jiaojiao Xiong\textsuperscript{1}, Yuhua Chen\textsuperscript{1}}
\IEEEauthorblockA{\textit{\textsuperscript{1}School of Electronics and Communication Engineering, Sun Yat-sen University, Shenzhen, China} \\
\textit{\textsuperscript{2}School of Automation, Guangdong University of Technology, Guangzhou, China}\\
% Email: \{zhujj59, zhangch397, yinhr6, xiongjj7\}@mail2.sysu.edu.cn, chyang513@gdut.edu.cn, *tangyq8@mail.sysu.edu.cn}
Email: zhujj59@mail2.sysu.edu.cn, \textsuperscript{*}tangyq8@mail.sysu.edu.cn}
}
\maketitle
\begin{abstract}

Integrated sensing and communication (ISAC) is a significant application scenario in future wireless communication networks, and sensing is always evaluated by the ambiguity function.
To enhance the sensing performance of the orthogonal time frequency space (OTFS) waveform, we propose a novel time-domain interleaved cyclic-shifted P4-coded OTFS (TICP4-OTFS) with improved ambiguity function.
TICP4-OTFS can achieve superior autocorrelation features in both the time and frequency domains by exploiting the multicarrier-like form of OTFS after interleaved and the favorable autocorrelation attributes of the P4 code.
Furthermore, we present the vectorized formulation of TICP4-OTFS modulation as well as its signal structure in each domain. 
Numerical simulations show that our proposed TICP4-OTFS waveform outperforms OTFS with a narrower mainlobe as well as lower and more distant sidelobes in terms of delay and Doppler-dimensional ambiguity functions, and an instance of range estimation using pulse compression is illustrated to exhibit the proposed waveform’s greater resolution. Besides, TICP4-OTFS achieves better performance of bit error rate for communication in low signal-to-noise ratio (SNR) scenarios. 
\end{abstract}

\begin{IEEEkeywords}
OTFS, P4 phase code, radar sensing, ambiguity function, ISAC.
\end{IEEEkeywords}

\section{Introduction}
% The advancement of next-generation wireless networks has underscored the growing significance of sensing capabilities, thus paving the way for the emergence of integrated sensing and communication (ISAC) \cite{ref1}. A pivotal aspect of ISAC research lies in waveform design, as it offers the potential to seamlessly integrate radar and communication functionalities on a single hardware device, facilitating spectrum sharing and enabling synergistic collaboration to unlock enhanced capabilities \cite{ref2}. 
The development of next-generation wireless networks has highlighted the increasing importance of sensing capabilities, paving the way for the introduction of integrated sensing and communication (ISAC) \cite{ref1}. Waveform design is an important part of ISAC research since it allows for the seamless integration of radar and communication features on a single hardware device, facilitating spectrum sharing and enabling synergistic collaboration to uncover greater capabilities \cite{ref2}.

% Orthogonal frequency division multiplexing (OFDM) has gained widespread adoption as a communication waveform in commercial fourth and fifth generation wireless networks, and some research has been conducted on utilizing OFDM for ISAC applications \cite{refISACOFDM1}. 
Orthogonal frequency division multiplexing (OFDM) has received extensive usage as a communication waveform in commercial fourth and fifth generation wireless networks, and some research on exploiting OFDM for ISAC applications has been undertaken \cite{refISACOFDM1}. 
% However, OFDM exhibits limitations in scenarios involving high-speed movement, which necessitates the exploration of alternative waveforms. 
However, in cases requiring high-speed movement, OFDM has limitations, necessitating the investigation of alternate waveforms. 
% In this regard, orthogonal time frequency space (OTFS) has emerged as a promising solution, capable of mitigating the adverse effects of high Doppler shifts. The growing interest from academia and industry highlights OTFS as a viable candidate for next-generation wireless networks \cite{refOTFS2}. 
In this regard, orthogonal time frequency space (OTFS) has emerged as a viable method capable of minimizing the negative consequences of significant Doppler shifts. The increased interest in OTFS from academia and industry emphasizes it as a promising candidate for next-generation wireless networks \cite{refOTFS2}.
% Notably, the distinctive feature of OTFS lies in its parameterization through delay and Doppler indices, aligning well with the physical interpretation of radar targets in terms of range and velocity. These inherent characteristics position OTFS as a natural fit for ISAC applications \cite{refOTFS1,refOTFS2,refOTFS3,refOTFS4}. 
Notably, the distinguishing aspect of OTFS is its parameterization via delay and Doppler indices, which aligns well with the physical interpretation of radar targets in terms of range and velocity. These intrinsic properties make OTFS a good fit for ISAC applications, as demonstrated by \cite{refOTFS1,refOTFS2,refOTFS3,refOTFS4}.

% Radar sensing, as an important component of ISAC, some work has focused on radar sensing applications for OTFS based on matched filter algorithms \cite{OTFSRADARmf1, OTFSRADARpdmf2, OTFSRADARpdmf3, refOTFSaf1}. And, since matched filtering is one of the most important algorithms in radar sensing, it is very necessary to study the ambiguity function for that the sensing performance is characterized by the ambiguity function and the Cramer-Rao bound \cite{yuan2023new}. 
Radar sensing is a key component of ISAC, and some work has focused on radar sensing applications for OTFS based on matched filter algorithms \cite{OTFSRADARmf1, OTFSRADARpdmf2, OTFSRADARpdmf3, refOTFSaf1}. And, because matched filtering is one of the most significant methods in radar sensing, it is critical to investigate the ambiguity function, as sensing performance is characterized by the ambiguity function and the Cramer-Rao bound \cite{yuan2023new}.
\cite{OTFSRADARmf1} proposed an efficient OTFS-based matched filter algorithm for target range and velocity estimation, which allows longer range radar and larger Doppler frequency estimation. \cite{OTFSRADARpdmf2} proposed a framework to achieve pulse Doppler radar processing using an existing OTFS communications architecture, based on which \cite{OTFSRADARpdmf3} proposed a two-dimensional correlation-based algorithm to estimate the fractional delay and Doppler parameters for radar sensing. However, there has been little research into the OTFS ambiguity function in the context of sensing performance. In a related study \cite{refOTFSaf1}, the authors proposed a random-padded OTFS modulation scheme for joint communication and radar/sensing systems. 
% This variant of OTFS demonstrated improved efficiency in radar systems by significantly reducing the sidelobes in the ambiguity function of the transmitted signal. It is worth noting, though, that the work specifically targeted a zero-padded OTFS signal, rendering it inapplicable to OTFS waveforms that do not utilize the zero-padded form.
This OTFS variation increased radar system efficiency by drastically lowering sidelobes in the ambiguity function of the transmitted signal. It should be noted, however, that the work was designed particularly for a zero-padded OTFS signal, making it inapplicable to OTFS waveforms that do not use the zero-padded form.

In this paper, we commence our investigation by examining the alignment of the OTFS frame in the time domain and analyzing the corresponding forms of the Zak transform in both analog and digital domains.
% Notably, the OTFS waveform, leveraging the inverse discrete Zak transform (IDZT) \cite{DD1}, exhibits a multicarrier structure akin to the OFDM waveform based on the inverse discrete Fourier transform (IDFT). 
Notably, the OTFS waveform, which employs the inverse discrete Zak transform (IDZT) \cite{DD1}, \cite{refZAK2}, has a multicarrier structure similar to that of the OFDM waveform, which employs the inverse discrete Fourier transform (IDFT). 
% This observation inspires us to explore the applicability of techniques employed in frequency modulated waveforms to the OTFS signal. 
This observation motivates us to investigate the OTFS signal's suitability for techniques used in frequency modulated waveforms.
% Furthermore, we establish the relationship between the ambiguity function and the signal's autocorrelation properties. Building upon this understanding, we propose the utilization of a row-column interleaver in the digital time domain to modify the alignment of the OTFS waveform, thereby enhancing its autocorrelation properties. Additionally, we incorporate a phase coding technique commonly employed in radar waveforms. Specifically, we encode P4 code sequences in a Doppler-wise manner and cyclically shift them by a number corresponding to the Doppler index, resembling the encoding mode in Multifrequency complementary phase (MCPC) schemes \cite{MCPC1}. 
Furthermore, we demonstrate a link between the ambiguity function and the signal's autocorrelation features. Based on this insight, we propose using a row-column interleaver in the digital time domain to change the alignment of the OTFS waveform, hence improving its autocorrelation features. In addition, we use a phase coding approach that is extensively used in radar waveforms. We encode delay-Doppler symbols with P4 code sequences Doppler-wise and cyclically shift them by a value corresponding to the Doppler index, similar to the encoding mode in multifrequency complementary phase (MCPC) schemes \cite{MCPC1}. 
% To facilitate further exploration of this waveform, we derive the vectorized form of the proposed time-domain interleaved cyclic-shifted P4-coded OTFS (TICP4-OTFS). This serves as a foundation for comprehensive investigation and analysis. 
We generate the vectorized form of the proposed time-domain interleaved cyclic-shifted P4-coded OTFS (TICP4-OTFS) to assist future investigation of this waveform. This lays the groundwork for further inquiry and analysis.

Through numerical simulations, we demonstrate that the strategy proposed in this paper produces a superior ambiguity function for radar sensing while also enhancing the bit error rate (BER) performance in low signal-to-noise ratio (SNR) scenarios for communication applications.
In addition, we illustrate an instance of range estimation using pulse compression, exhibiting the proposed waveform's greater resolution.
\section{Fundamental concepts}
In this section, we present the signal model for Zak-based OTFS and another equivalent implementation based on OFDM \cite{DD1}. Additionally, the definition of the ambiguity function and the P4 code are introduced.
\subsection{Signal model of OTFS}

In the subsequent analysis, we introduce a time-domain signal $s(t)$ with a time limit of $T_w=NT$ and a band limit of $B_w=M \Delta f$, where $\Delta f$ is set to $1/T$ to ensure orthogonality in the time-frequency domain.
By employing the Zak transform, we can express the Zak domain signal of $s(t)$ within the fundamental region of the delay-Doppler domain, denoted as $\mathcal R=\{\tau\in[0,T),\nu\in[0,\Delta f)\}$, as
\begin{equation}
Z_T[s(t)](\tau,\nu)=\sqrt T \sum_{ n=0}^{N-1}s(\tau+ nT)e^{-j2\pi nT\nu},\label{eq1}
\end{equation}
and the corresponding inverse transform of Zak is defined as
\begin{equation}
s(\tau+ nT)=\sqrt T\int_0^{\Delta f}Z_T[s(t)](\tau+nT,\nu)e^{j2\pi nT\nu}d\nu,\label{eq2}
\end{equation}
where $ n=0,1,\dots,N-1$. After sampling time $t$ and delay $\tau$ at intervals of $1/B_w=T/M$ and sampling Doppler at intervals of $1/T_w=1/(NT)$, we obtain the indices $q=t M/T$ for $q=0,1,\dots,NM-1$, $l=\tau M/T$ for $l=0,1,\dots,M-1$, and $k=\nu NT$ for $k=0,1,\dots,N-1$. Consequently, the digital time-domain signal model can be expressed as
\begin{equation}
s[q]=s[l+nM]=\frac{1}{\sqrt N}\sum_{k=0}^{N-1}Z[l,k]e^{j2\pi\frac {kn}{N}},\label{eq3}
\end{equation}
where $Z[l,k]$ represents the original information symbols located at the $l$-th delay and $k$-th Doppler of the delay-Doppler plane grid $\Gamma:\{l\frac{T}{M},k\frac{1}{NT}\}$ for $l=0,1,\dots,M-1$ and $k=0,1,\dots,N-1$. Additionally, $s[q]=s[l+nM]$ denotes the time samples corresponding to the $l$-th delay and the $n$-th time slot of the delay-time plane after column-wise parallel to serial (P/S) conversion.

Revisiting another implementation of OTFS in the digital domain, which is based on OFDM and proven to be equivalent to IDZT-based OTFS. Firstly, the delay-Doppler domain symbols $Z[l,k]$ are mapped into the time-frequency domain using inverse symplectic finite Fourier transform (ISFFT) as

\begin{equation}
X[m,n]=\frac{1}{\sqrt{MN}}\sum_{l=0}^{M-1}\sum_{k=0}^{N-1}Z[l,k]e^{j2\pi(\frac{kn}{N}-\frac{lm}{M})},\label{eq4}
\end{equation}
where $X[m,n]$ represents information symbols located at the $m$-th frequency and $n$-th time of the time-frequency plane grid $\Pi:\{nT,\frac{m}{T}\}$ with $m=0,1,\dots,M-1$, $n=0,1,\dots,N-1$. 

Then, using IDFT the time-frequency symbols $X[m,n]$ are mapped into the delay-time domain. Finally, the delay-time symbols $S[l,n]$ are rearranged in a column-wise P/S conversion to obtain the time-domain symbols $s[q]=s[l+nM]$. The process can be denoted as
\begin{equation}
\begin{aligned}
&S[l,n]=\frac{1}{\sqrt{M}}\sum_{m=0}^{M-1}X[m,n]e^{j2\pi\frac{ml}{M}},\\
&S[l,n]\xrightarrow{\rm{P/S}} s[l+nM]=s[q].\label{eq5}
\end{aligned}
\end{equation}


\subsection{Ambiguity function and P4 code}
To analyze the performance and characteristics of the waveform in greater detail, we introduce the concept of the ambiguity function. The ambiguity function serves as a metric to evaluate the sensing performance of a waveform, revealing the interference caused by a transmitted signal due to variations in delay $\tau$ and Doppler shift $\nu$ compared to a reference signal. In this paper, the ambiguity function is defined as
\begin{equation}
\chi (\tau,\nu)=\int_{-\infty}^{+\infty}s(t)s^{*}(t+\tau)e^{j2\pi \nu t}dt.\label{eq8}
\end{equation}

If we set $\nu=0$ or $\tau=0$, the delay and Doppler-dimensional versions of the ambiguity function can be obtained, respectively. The delay and Doppler-dimensional ambiguity function is defined as
\begin{equation}
\chi (\tau)=\int_{-\infty}^{+\infty}s(t)s^{*}(t+\tau)dt,\label{eqAFt}
\end{equation}

\begin{equation}
\chi (\nu)=\int_{-\infty}^{+\infty}S(f)S^{*}(f+\nu)df,\label{eqAFf}
\end{equation}
where $s(t)$ and $S(f)$ denote the signal in the time and frequency domain, respectively, and $(\cdot)^*$ is the conjugate operation.
It can be noted that the definition of the ambiguity function in \eqref{eqAFt} and \eqref{eqAFf} is the same as the autocorrelation function. Therefore, improving the autocorrelation properties of the signal in both the time and frequency domains is crucial to optimizing the sensing performance.

In terms of the P4 code, it has good autocorrelation properties in both the time and frequency domains for inheriting the advantages of the linear frequency modulation (LFM) waveform, which has the potential to be used to improve the ambiguity function of waveforms. The P4 code is generated conceptually by downconverting the LFM waveform to baseband using a local oscillator and then sampling it at the Nyquist rate, so that the successive samples are given by
\begin{equation}    
\phi_p = \pi(p-1)^2/P-\pi(p-1),\label{eq21}
\end{equation}
where $P$ denotes the number of samples and $p=1,2,\dots,P$.
% This means that the correlation value decays quickly with increasing distance, as shown in Fig. \ref{P4code}.


% % Figure environment removed


% Figure environment removed
\section{Proposed TICP4-OTFS}
In this section, we introduce the TICP4-OTFS signal model and derive the vectorized formulation at the transmitter side. This enables us to describe the structure of the TICP4-OTFS signal in each domain, emphasizing its distinct arrangement and properties.

\subsection{Signal model of TICP4-OTFS}
By observing the form of Zak-based OTFS in the digital domain, we can identify an intrinsic form of multicarrier modulation. Equation \eqref{eq3} can be seen as an IDFT for the Doppler indices $k$, with the results parameterized by the time indices $n$, if we treat the delay indices $l$ as constant. This observation implies that the delay-Doppler plane can be viewed as a time-frequency plane with different inter-symbol and inter-carrier spacing, where the carriers are the Doppler indices, thus indicating another paradigm of multicarrier modulation. 

For all $N$ subcarriers in the proposed TICP4-OTFS signal, we use cyclic-shifted P4 phase code sequences. Each subcarrier corresponds to $M$ symbols with different delay indices in the delay-Doppler plane grid. Each P4 code sequence, which is made up of $M$ samples, is multiplied by $M$ delay-domain symbols and cyclically shifted using different Doppler indices $k$. Additionally, a row-column interleaver is employed to modify the alignment in the digital time domain. As a result, the TICP4-OTFS signal is formulated as
\begin{equation}
s'[q]=s'[lN+n]=\frac{1}{\sqrt N}\sum_{k=0}^{N-1}Z[l,k]e^{j\phi_{[l-k]_M}}e^{j2\pi\frac {kn}{N}},\label{eq22}
\end{equation}
where $(\cdot)_M$ represents the modulo $M$ operation. In addition to the definition in \eqref{eq22}, the TICP4-OTFS signal can also be acquired using an OFDM-based method. Firstly, using ISFFT, the time-frequency domain symbols can be obtained as
\begin{equation}
X'[m,n]=\frac{1}{\sqrt{MN}}\sum_{l=0}^{M-1}\sum_{k=0}^{N-1}Z[l,k]e^{j\phi_{[l-k]_M}}e^{j2\pi(\frac{kn}{N}-\frac{lm}{M})}.\label{eq23}
\end{equation}

Then, using IDFT, we can obtain the delay-time signal as
\begin{equation}
\begin{aligned}
S'[l,n]=\frac{1}{\sqrt{M}}\sum_{m=0}^{M-1}X'[m,n]e^{j2\pi\frac{ml}{M}},\label{delay-time}
\end{aligned}
\end{equation}

Finally, the time-domain signal $s'[q]=s'[lN+n]$ can be obtained by performing a column-wise P/S conversion followed by a row-column interleaving operation, as
\begin{equation}
\begin{aligned}
S'[l,n]\xrightarrow{\rm{P/S}} s'[lN+n]=s'[q].\label{time}
\end{aligned}
\end{equation}

% In conclusion, by applying the cyclic-shifted P4 encoder and a row-column interleaver, we obtain the TICP4-OTFS samples in the digital domain. Following that, we obtain the analog domain signal of TICP4-OTFS using rectangular pulse shaping, which is commonly used in OFDM and OTFS. The modulation process is presented in Fig. \ref{block}. This allows us to investigate the autocorrelation properties and ambiguity function of the TICP4-OTFS waveform.
In conclusion, we obtain the TICP4-OTFS samples in the digital domain by using the cyclic-shifted P4 encoder and a row-column interleaver. The analog domain signal of TICP4-OTFS is then obtained using rectangular pulse shaping, allowing us to investigate the TICP4-OTFS waveform and its ambiguity function. Fig. \ref{block} depicts the modulation process. 

% Figure environment removed

% Figure environment removed
%% Figure environment removed

% Figure environment removed
%% Figure environment removed
\subsection{Vectorized formulation of TICP4-OTFS modulation}
% In this subsection, we use the following notations. Matrices are denoted by uppercase boldface letters, and vectors are represented by lowercase boldface letters. The normalized $n$-point DFT matrix is denoted as $\mathbf{F}n$, and the Hermitian transpose is denoted by $(\cdot)^H$. The identity matrix of size $k$ is denoted as $\mathbf{I}_{\rm k}$. The Kronecker product and Hadamard product operations are denoted by $\otimes$ and $\odot$, respectively.
The notations used in this subsection are as follows. Uppercase boldface letters represent matrices, while lowercase boldface letters represent vectors. The Hermitian transpose is denoted by $(\cdot)^H$ and the normalized $n$-point DFT matrix is denoted by $\mathbf{F}_{\rm n}$. The identity matrix of size $k$ is denoted by the symbol $\mathbf{I}_{\rm k}$. The Kronecker and Hadamard product operations are denoted by the symbols $\otimes$ and $\odot$, respectively.

% First, we rewrite the transmitted signal in \eqref{delay-time} using a vectorized formulation as
First, the transmitted signal in \eqref{delay-time} can be rewritten using a vectorized formulation as
\begin{equation}
\begin{aligned}
\mathbf S'_{\rm_{OTFS}}= \mathbf I_{\rm M}\mathbf Z'\mathbf{F_{\rm{N}}^{\rm{H}}},\label{eqOTFS}
\end{aligned}
\end{equation}
where $\mathbf Z'=\mathbf Z\odot\mathbf{\Delta_{\phi}}$ represents the input data in the delay-Doppler domain after applying the cyclic-shifted P4 code. 
% Here, $\mathbf Z$ is an $M\times N$ matrix consisting of $Z[l,k]$ elements, and $\mathbf{\Delta_{\phi}}$ is an $M\times N$ matrix representing the cyclic-shifted P4 code defined as
$\mathbf Z$ is a $M\times N$ matrix with $Z[l,k]$ elements, and $\mathbf{\Delta_{\phi}}$ is a $M\times N$ matrix with the cyclic-shifted P4 code described as

\begin{equation}
\begin{aligned}
\mathbf{\Delta_{\phi}}&=
\begin{bmatrix}
e^{\phi_0}&e^{\phi_{M-1}}&\dots &\\ e^{\phi_1}&e^{\phi_{0}}&\dots &\\ \vdots&\vdots&\ddots&\\ e^{\phi_{M-1}}&e^{\phi_{M-2}}&\dots &
\end{bmatrix}
\\
&=
\begin{bmatrix}

P_0&P_{M-1}&\dots &\\ P_1&P_{0}&\dots &\\ \vdots&\vdots&\ddots&\\ P_{M-1}&P_{M-2}&\dots &
\end{bmatrix}.
\end{aligned}
\end{equation}

Performing column-wise vectorization on $\mathbf S_{\rm_{OTFS'}}$ yields the $MN \times 1$ vector as
\begin{equation}
\begin{aligned}
\mathbf s'_{\rm_{OTFS}}= \rm{vec}({\mathbf S'_{\rm_{OTFS}}})=(\mathbf{F_{\rm{N}}^{\rm{H}}} \otimes\mathbf I_{\rm M})\mathbf z',
\end{aligned}
\end{equation}
where $\mathbf z' =\rm{vec}(\mathbf Z' )$. To generate the proposed TICP4-OTFS signal $\mathbf s'_{\rm_{TICP4-OTFS}}$, we introduce a row-column interleaving matrix $\mathbf T\in\mathbb{C}^{NM\times NM}$ to represent the row-column interleaving operation
\begin{equation}
\mathbf T =\begin{bmatrix}
\mathbf E_{1,1}&\mathbf E_{2,1}&\dots &\mathbf E_{N,1}\\ \mathbf E_{1,2}&\mathbf E_{2,2}&\dots &\mathbf E_{N,2}\\ \vdots&\vdots&\ddots&\vdots\\ \mathbf E_{1,M}&\mathbf E_{2,M}&\dots &\mathbf E_{N,M}
\end{bmatrix},
\end{equation}
where the $N\times M$ matrix $\mathbf E_{i,j}$ is defined as
\begin{equation}
\begin{aligned}
\mathbf E_{i,j}(i',j')=\begin{cases} 1,& \text{if}\ i'=i\ \text{and}\ j'=j \\0,& \text{otherwise}  \end{cases}.
\end{aligned}
\end{equation}

Finally, we obtain the vectorized formulation of TICP4-OTFS as $\mathbf s'_{\rm_{TICP4-OTFS}}= \mathbf T\mathbf s'_{\rm_{OTFS}}$.

\subsection{Structure analysis for TICP4-OTFS signal}
% The TICP4-OTFS transceiver is illustrated in Fig. \ref{MDPCblock}, where the cyclic-shifted CP4-coded modulated symbols are mapped to the delay-Doppler domain. Two methods are available to generate the time-domain TICP4-OTFS waveform for transmission over the channel. The first method involves using IDZT followed by a row-column interleaver. As mentioned previously, IDZT enables the mapping of delay-Doppler symbols to the time domain, resembling a multicarrier-like modulation. The row-column interleaver rearranges the time-domain samples without losing any information.
Fig. \ref{MDPCblock} depicts the TICP4-OTFS transceiver, where the cyclic-shifted CP4-coded modulated symbols are mapped to the delay-Doppler domain. There are two ways for generating the time-domain TICP4-OTFS waveform for transmission over the channel. The first approach employs IDZT, which is followed by a row-column interleaver. As previously stated, IDZT allows the mapping of delay-Doppler CP4-coded symbols to the time domain, resulting in a multicarrier-like modulation. The row-column interleaver rearranges the time-domain samples while retaining all of their information.

% To provide a clearer description, we present an alternative indirect scheme using an OFDM-based approach. Initially, an orthogonal 2D precoding technique such as the ISFFT is employed to transfer the delay-Doppler domain signal to the time-frequency domain. Subsequently, an OFDM modulator is used in each time slot to further transform the time-frequency domain signal to the delay-time domain. Finally, by employing column-wise P/S conversion followed by a row-column interleaver, the TICP4-OTFS waveform is generated.
To provide a more detailed description, we present an alternate indirect system based on OFDM. The delay-Doppler domain signal is first transferred to the time-frequency domain using an orthogonal 2D precoding approach such as the ISFFT. Following that, in each time slot, an OFDM modulator is utilized to further transform the time-frequency domain signal to the delay-time domain. Finally, the TICP4-OTFS waveform is formed by using column-wise P/S conversion followed by a row-column interleaver.
% By examining the alignment of the TICP4-OTFS signal in the delay-Doppler domain and time domain, we can observe that it exhibits similarities to an OFDM signal when interpreting the delay axis as the time axis and the Doppler axis as the frequency axis. Consequently, the time domain part of TICP4-OTFS inherits the shape of the MCPC waveform, which in turn influences the characteristics of the ambiguity function.

Examining the alignment of the TICP4-OTFS signal in the delay-Doppler domain and time domain reveals similarities to an OFDM signal when the delay axis is interpreted as the time axis and the Doppler axis as the frequency axis. As a result, the time domain part of TICP4-OTFS inherits the shape of the MCPC waveform, which determines the ambiguity function's features.

% However, due to the difference in physical meaning brought about by the modulation method, the receiver can still take advantage of the delayed-Doppler domain signal processing.
% 这里可以加上对这种通信方式的描述,比如其在时域上变成了OFDM,但是根本性质在于DD域调制信号,DD域信道建模,使得可以利用DD域的检测算法。

\section{Simulation results and discussions}
% In this section, we compare the ambiguity function and BER between the conventional OTFS and TICP4-OTFS waveforms. The simulation considers different delay-Doppler plane grids $\Gamma$, with $M=8,8$ and $N=4,8$ for both waveforms, ensuring they have the same time duration and bandwidth. 
In this section, we compare the ambiguity function and BER of the conventional OTFS and TICP4-OTFS waveforms. The simulation takes into account two delay-Doppler plane grids $\Gamma$, with $M=8,8$ and $N=4,8$ for both waveforms to ensure they have the same time duration and bandwidth. 
% The two waveforms each modulate a sequence containing $MN$ $1$ elements to analyze the ambiguity function, representing a typical radar waveform without transmitting information. Analog time-domain waveforms are obtained by applying rectangular pulse shaping with a sampling rate of 4 times the Nyquist rate.
To examine the ambiguity function, the two waveforms modulate a one-element sequence with the length of $MN$, representing a typical radar waveform without sending information. Analog time-domain waveforms are created by using rectangular pulse shaping at four times the Nyquist rate.
% For communication analysis, $MN$ 4-QAM symbols are modulated. A 4-tap channel of uniform power is employed, with delay taps and Doppler taps set as $[0,1,2,3]$ and $[0,1,2,3]$, respectively. These simulations provide insights into the ambiguity function and BER performance of the conventional OTFS and TICP4-OTFS waveforms.
4-QAM modulated symbols of $MN$ are used for communication analysis. A 4-tap uniform power channel is used, with delay and Doppler taps set to $[0,1,2,3]$ and $[0,1,2,3]$, respectively. These simulations provide insight into the ambiguity function and BER performance of conventional OTFS and TICP4-OTFS waveforms.


% Firstly, we compare the zero Doppler cut and zero delay cut of the ambiguity function between conventional OTFS and TICP4-OTFS waveforms. Fig. \ref{MDPCdelay1} demonstrates that the proposed TICP4-OTFS waveform exhibits a significantly narrower main lobe in the delay dimension, reducing it to approximately one-tenth the size of the conventional OTFS main lobe.
Firstly, we compare the zero Doppler and zero delay cuts of the ambiguity function between conventional OTFS and TICP4-OTFS waveforms. Fig. \ref{MDPCdelay1} shows that the proposed TICP4-OTFS waveform has a substantially narrower main lobe in the delay dimension, around one-tenth the size of the conventional OTFS mainlobe.
% Moreover, in the Doppler dimension, the TICP4-OTFS waveform effectively suppresses high amplitude side lobes, ensuring they are lower in magnitude and well-separated from the main lobe, as illustrated in Fig. \ref{MDPCDoppler1}. 
Moreover, the TICP4-OTFS waveform successfully suppresses high amplitude sidelobes in the Doppler dimension, guaranteeing they are smaller in magnitude and well separated from the mainlobe, as seen in Fig. \ref{MDPCDoppler1}. 

% To further analyze the joint performance of the ambiguity function in the delay-Doppler domain, we compare the two-dimensional ambiguity contours of conventional OTFS and TICP4-OTFS.
To investigate the joint performance of the ambiguity function in the delay-Doppler domain further, we compare the two-dimensional ambiguity contours of conventional OTFS and TICP4-OTFS. 
% Fig. \ref{MDPC2} clearly demonstrates that our proposed strategy transforms the ambiguity function from a column-like shape to a pegboard-like pattern. However, it should be noted that the TICP4-OTFS waveform may exhibit some coupling effects between delay and Doppler dimensions, as a result of the influence of the P4 code.
Fig. \ref{MDPC2} clearly shows how our proposed technique transforms the ambiguity function from a column-like structure to a pegboard-like pattern. However, due to the influence of the P4 code, the TICP4-OTFS waveform may display certain coupling effects between the delay and Doppler dimensions.

% Similarly, in Fig. \ref{MDPC12} and Fig. \ref{MDPC22}, we set $N=8$ to ensure it matches the value of $M$. Compared to conventional OTFS, TICP4-OTFS demonstrates narrower main lobes in both the delay and Doppler dimensions of the ambiguity function. 
Similarly, in Fig. \ref{MDPC12} and Fig. \ref{MDPC22}, we fix $N=8$ to guarantee that it corresponds to the value of $M$. TICP4-OTFS has thinner mainlobes in both the delay and Doppler dimensions of the ambiguity function than conventional OTFS. 
% Additionally, the side lobes in the Doppler dimension are notably suppressed compared to Fig. \ref{MDPCDoppler1}, e.g., all amplitudes except for the main lobe are suppressed below -3dB. This improvement can be attributed to the use of cyclic-shifted P4 codes, where the equality of the values of $M$ and $N$ keeps the Doppler-dimensional P4 codes from recurring. 
Furthermore, the sidelobes in the Doppler dimension are significantly suppressed compared to Fig. \ref{MDPCDoppler1}, e.g., all amplitudes except the mainlobe's are suppressed below -3dB. This enhancement could be attributed to the use of cyclic-shifted P4 codes, where the equality of the values of $M$ and $N$ prevents the Doppler-dimensional P4 codes from reoccurring. 
As a result, TICP4-OTFS achieves comparable phase coding performance in both the time and frequency domains.

Following that, we examine the performance of two waveforms for range estimation using the pulse compression algorithm frequently used in radar by setting the delay taps to $[1,4,7]$ and the Doppler taps to $[0,0,0]$. As seen in Fig. ref{RPC}, TICP4-OTFS exactly matches the three peaks, however OTFS mixes the three peaks together to produce a peak with a wider mainlobe due to weak resolving capacity.
% Finally, we compare the BER of the two waveforms to explore the performance of the waveforms proposed in this paper in terms of communication. The simulation results in Fig. \ref{BER} show that TICP4-OTFS has lower BER than OTFS, which, combined with the results of the ambiguity function above, implies that TICP4-OTFS has the potential for ISAC applications.

Finally, we compare the BER of the two waveforms to investigate the performance of the waveforms provided in this paper in terms of communication. The simulation findings in Fig. \ref{BER} reveal that TICP4-OTFS has a lower BER than OTFS in low SNR scenarios. When combined with the ambiguity function and BER performance, we can conclude that TICP4-OTFS may be more appropriate for ISAC applications.

% , which, when paired with the results of the ambiguity function above, implies that TICP4-OTFS has the potential for ISAC applications.


% Figure environment removed
% Figure environment removed

\section{Conclusion and future works}
In this paper, we propose a novel TICP4-OTFS to optimize the ambiguity function shape. This approach leverages the multicarrier-like form of IDZT-based OTFS after interleaved and the favorable autocorrelation properties of the P4 code. 
% In this paper, we offer a novel TICP4-OTFS for optimizing the form of the ambiguity function. This method takes advantage of the multicarrier-like shape of IDZT-based OTFS after interleaving and the P4 code's advantageous autocorrelation features. 
By applying cyclic-shifted P4 code sequences in the delay-Doppler domain and incorporating a row-column interleaver in the time domain, the TICP4-OTFS waveform is generated. 
% The TICP4-OTFS waveform is created by using cyclic-shifted P4 code sequences in the delay-Doppler domain and a row-column interleaver in the time domain. 
We also derive the vectorized formulation of this waveform, which serves as a valuable contribution for further exploration of efficient techniques and algorithms for TICP4-OTFS. 
Simulation results demonstrate that TICP4-OTFS exhibits a narrower mainlobe, and lower as well as more distant sidelobes while achieving better BER performance in low SNR scenarios. 
These favorable characteristics indicate the potential of TICP4-OTFS for application in ISAC scenarios, which will be investigated in our future work.

%\bibliographystyle{IEEEtran}
%\bibliography{IEEEexample}

% Generated by IEEEtran.bst, version: 1.14 (2015/08/26)
\begin{thebibliography}{10}
\providecommand{\url}[1]{#1}
\csname url@samestyle\endcsname
\providecommand{\newblock}{\relax}
\providecommand{\bibinfo}[2]{#2}
\providecommand{\BIBentrySTDinterwordspacing}{\spaceskip=0pt\relax}
\providecommand{\BIBentryALTinterwordstretchfactor}{4}
\providecommand{\BIBentryALTinterwordspacing}{\spaceskip=\fontdimen2\font plus
\BIBentryALTinterwordstretchfactor\fontdimen3\font minus
  \fontdimen4\font\relax}
\providecommand{\BIBforeignlanguage}[2]{{%
\expandafter\ifx\csname l@#1\endcsname\relax
\typeout{** WARNING: IEEEtran.bst: No hyphenation pattern has been}%
\typeout{** loaded for the language `#1'. Using the pattern for}%
\typeout{** the default language instead.}%
\else
\language=\csname l@#1\endcsname
\fi
#2}}
\providecommand{\BIBdecl}{\relax}
\BIBdecl

\bibitem{ref1}
Z.~Wei, H.~Qu, Y.~Wang, X.~Yuan, H.~Wu, Y.~Du, K.~Han, N.~Zhang, and Z.~Feng,
  ``{Integrated Sensing and Communication Signals Toward 5G-A and 6G: A
  Survey},'' \emph{IEEE Internet of Things Journal}, vol.~10, no.~13, pp.
  11\,068--11\,092, 2023.

\bibitem{ref2}
W.~Zhou, R.~Zhang, G.~Chen, and W.~Wu, ``{Integrated Sensing and Communication
  Waveform Design: A Survey},'' \emph{IEEE Open Journal of the Communications
  Society}, vol.~3, pp. 1930--1949, 2022.

\bibitem{refISACOFDM1}
C.~Sturm and W.~Wiesbeck, ``{Waveform Design and Signal Processing Aspects for
  Fusion of Wireless Communications and Radar Sensing},'' \emph{Proceedings of
  the IEEE}, vol.~99, no.~7, pp. 1236--1259, 2011.

\bibitem{refOTFS2}
Z.~Wei, W.~Yuan, S.~Li, J.~Yuan, G.~Bharatula, R.~Hadani, and L.~Hanzo,
  ``{Orthogonal Time-Frequency Space Modulation: A Promising Next-Generation
  Waveform},'' \emph{IEEE Wireless Communications}, vol.~28, no.~4, pp.
  136--144, 2021.

\bibitem{refOTFS1}
R.~Hadani, S.~Rakib, M.~Tsatsanis, A.~Monk, A.~J. Goldsmith, A.~F. Molisch, and
  R.~Calderbank, ``{Orthogonal Time Frequency Space Modulation},'' in
  \emph{2017 IEEE Wireless Communications and Networking Conference (WCNC)},
  2017, pp. 1--6.

\bibitem{refOTFS3}
Z.~Wei, S.~Li, W.~Yuan, R.~Schober, and G.~Caire, ``{Orthogonal Time Frequency
  Space Modulation—Part I: Fundamentals and Challenges Ahead},'' \emph{IEEE
  Communications Letters}, vol.~27, no.~1, pp. 4--8, 2023.

\bibitem{refOTFS4}
W.~Yuan, Z.~Wei, S.~Li, R.~Schober, and G.~Caire, ``{Orthogonal Time Frequency
  Space Modulation—Part III: ISAC and Potential Applications},'' \emph{IEEE
  Communications Letters}, vol.~27, no.~1, pp. 14--18, 2023.

\bibitem{OTFSRADARmf1}
P.~Raviteja, K.~T. Phan, Y.~Hong, and E.~Viterbo, ``{Orthogonal Time Frequency
  Space (OTFS) Modulation Based Radar System},'' in \emph{2019 IEEE Radar
  Conference (RadarConf)}, 2019, pp. 1--6.

\bibitem{OTFSRADARpdmf2}
K.~{Zhang}, W.~{Yuan}, S.~{Li}, F.~{Liu}, F.~{Gao}, P.~{Fan}, and Y.~{Cai},
  ``{Radar Sensing via OTFS Signaling: A Delay Doppler Signal Processing
  Perspective},'' \emph{arXiv e-prints}, p. arXiv:2301.09909, Jan. 2023.

\bibitem{OTFSRADARpdmf3}
A.~S. Bondre and C.~D. Richmond, ``{Dual-Use of OTFS Architecture for Pulse
  Doppler Radar Processing},'' in \emph{2022 IEEE Radar Conference
  (RadarConf22)}, 2022, pp. 1--6.

\bibitem{refOTFSaf1}
P.~Karpovich and T.~P. Zielinski, ``{Random-Padded OTFS Modulation for Joint
  Communication and Radar/Sensing Systems},'' in \emph{2022 23rd International
  Radar Symposium (IRS)}, 2022, pp. 104--109.

\bibitem{yuan2023new}
W.~Yuan, S.~Li, Z.~Wei, Y.~Cui, J.~Jiang, H.~Zhang, and P.~Fan, ``{New Delay
  Doppler Communication Paradigm in 6G Era: A Survey of Orthogonal Time
  Frequency Space (OTFS)},'' \emph{China Communications}, vol.~20, no.~6, pp.
  1--25, 2023.

\bibitem{DD1}
A.~F. Molisch, ``{Delay-Doppler Communications: Principles and Applications},''
  \emph{IEEE Communications Magazine}, vol.~61, no.~3, pp. 10--10, 2023.

\bibitem{refZAK2}
S.~K. Mohammed, R.~Hadani, A.~Chockalingam, and R.~Calderbank, ``{OTFS—A
  Mathematical Foundation for Communication and Radar Sensing in the
  Delay-Doppler Domain},'' \emph{IEEE BITS the Information Theory Magazine},
  vol.~2, no.~2, pp. 36--55, 2022.

\bibitem{MCPC1}
N.~Levanon, ``{Multifrequency Complementary Phase-Coded Radar Signal},''
  \emph{IEE Proceedings-Radar, Sonar and Navigation}, vol. 147, no.~6, pp.
  276--284, 2000.

\end{thebibliography}


\end{document}
