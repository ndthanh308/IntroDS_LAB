The meteoric advancement of machine learning and artificial intelligence technologies has enabled the construction of neural networks that effectively emulate the complex computations of the human brain. These deep learning models have found utility in a wide range of applications, such as computer vision, natural language processing, autonomous driving, and more. With the growing complexity and sophistication of these neural network models, the computational requirements, particularly for 32-bit operations, have exponentially increased. This heightened computational demand necessitates the exploration of more efficient alternatives, such as 16-bit operations.

However, the shift to 16-bit operations is riddled with challenges. A common standpoint within the research community argues that 16-bit operations are not ideally suited for neural network computations. This belief is mainly attributable to concerns related to numerical instability during the backpropagation phase, especially when popular optimizers like Adam are employed. This instability, more pronounced during the optimizer-mediated backpropagation process rather than forward propagation, can negatively impact the performance of 16-bit operations and compromise the functioning of the neural network model. Current optimizers predominantly operate on 32-bit precision. If these are deployed in a 16-bit environment without appropriate hyperparameter fine-tuning, the neural network models encounter difficulties during learning. This issue is particularly evident in backward propagation, which heavily relies on the optimizer. Confronted with these challenges, the objective of this research is to conduct an exhaustive investigation into the feasibility and implementation of 16-bit operations for neural network training. We propose and evaluate innovative strategies aimed at reducing the numerical instability encountered during the backpropagation phase under 16-bit environments. A significant focus of this paper is also dedicated to exploring the future possibilities of developing 16-bit based optimizers. One of the fundamental aims of this research is to adapt key optimizers such as Adam to prevent numerical instability, thereby facilitating efficient 16-bit computations. These newly enhanced optimizers are designed to not only address the issue of numerical instability but also leverage the computational advantages offered by 16-bit operations, all without compromising the overall performance of the neural network models. Through this research, our intention goes beyond improving the efficiency of neural network training; we also strive to validate the use of 16-bit operations as a dependable and efficient computational methodology in the domain of deep learning. We anticipate that our research will contribute to a shift in the prevalent perceptions about 16-bit operations and will foster further innovation in the field. Ultimately, we hope our findings will pave the way for a new era in deep learning research characterized by efficient, high-performance neural network models.