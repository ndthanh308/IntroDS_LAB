\documentclass[conference]{IEEEtran}
\IEEEoverridecommandlockouts
% The preceding line is only needed to identify funding in the first footnote. If that is unneeded, please comment it out.
\usepackage{cite}
\usepackage{amsmath,amssymb,amsfonts}
\usepackage{algorithmic}
\usepackage{graphicx}
\usepackage{textcomp}
\usepackage{xcolor}
\usepackage{booktabs}
\usepackage{array}
\def\BibTeX{{\rm B\kern-.05em{\sc i\kern-.025em b}\kern-.08em
    T\kern-.1667em\lower.7ex\hbox{E}\kern-.125emX}}
\begin{document}

\title{An Efficient Approach to Mitigate Numerical Instability in Backpropagation for 16-bit Neural Network Training
}

\author{
\\
\IEEEauthorblockN{Juyoung Yun}
\\
\IEEEauthorblockA{\textit{Department of Computer Science, Stony Brook University} \\
\textit{Department of Applied Mathematics and Statistics, Stony Brook University} \\
juyoung.yun@stonybrook.edu
\\
}
}
\maketitle

\begin{abstract}
In this research, we delve into the intricacies of the numerical instability observed in 16-bit computations of machine learning models, particularly when employing popular optimization algorithms such as RMSProp and Adam. This instability is commonly experienced during the training phase of deep neural networks, leading to disrupted learning processes and hindering the effective deployment of such models. We identify the single hyperparameter, epsilon, as the main culprit behind this numerical instability. An in-depth exploration of the role of epsilon in these optimizers within 16-bit computations reveals that a minor adjustment of its value can restore the functionality of RMSProp and Adam, consequently enabling the effective utilization of 16-bit neural networks. We propose a novel method to mitigate the identified numerical instability issues. This method capitalizes on the updates from the Adam optimizer and significantly improves the robustness of the learning process in 16-bit computations. This study contributes to better understanding of optimization in low-precision computations and provides an effective solution to a longstanding issue in training deep neural networks, opening new avenues for more efficient and stable model training.
\end{abstract}

\begin{IEEEkeywords}
Deep Learning, Low Precision, Neural Network, High Performance, Optimizer
\end{IEEEkeywords}

\section{Introduction}
The meteoric advancement of machine learning and artificial intelligence technologies has enabled the construction of neural networks that effectively emulate the complex computations of the human brain. These deep learning models have found utility in a wide range of applications, such as computer vision, natural language processing, autonomous driving, and more. With the growing complexity and sophistication of these neural network models, the computational requirements, particularly for 32-bit operations, have exponentially increased. This heightened computational demand necessitates the exploration of more efficient alternatives, such as 16-bit operations.

However, the shift to 16-bit operations is riddled with challenges. A common standpoint within the research community argues that 16-bit operations are not ideally suited for neural network computations. This belief is mainly attributable to concerns related to numerical instability during the backpropagation phase, especially when popular optimizers like Adam are employed. This instability, more pronounced during the optimizer-mediated backpropagation process rather than forward propagation, can negatively impact the performance of 16-bit operations and compromise the functioning of the neural network model. Current optimizers predominantly operate on 32-bit precision. If these are deployed in a 16-bit environment without appropriate hyperparameter fine-tuning, the neural network models encounter difficulties during learning. This issue is particularly evident in backward propagation, which heavily relies on the optimizer. Confronted with these challenges, the objective of this research is to conduct an exhaustive investigation into the feasibility and implementation of 16-bit operations for neural network training. We propose and evaluate innovative strategies aimed at reducing the numerical instability encountered during the backpropagation phase under 16-bit environments. A significant focus of this paper is also dedicated to exploring the future possibilities of developing 16-bit based optimizers. One of the fundamental aims of this research is to adapt key optimizers such as Adam to prevent numerical instability, thereby facilitating efficient 16-bit computations. These newly enhanced optimizers are designed to not only address the issue of numerical instability but also leverage the computational advantages offered by 16-bit operations, all without compromising the overall performance of the neural network models. Through this research, our intention goes beyond improving the efficiency of neural network training; we also strive to validate the use of 16-bit operations as a dependable and efficient computational methodology in the domain of deep learning. We anticipate that our research will contribute to a shift in the prevalent perceptions about 16-bit operations and will foster further innovation in the field. Ultimately, we hope our findings will pave the way for a new era in deep learning research characterized by efficient, high-performance neural network models.

\section{Related Works}
The matter of numerical precision in deep learning model training has garnered substantial attention in recent years. A milestone study by Gupta et al. \cite{Gupta2015} was among the first to explore the potential of lower numerical precision in deep learning, emphasizing the critical balance between computational efficiency and precision. They suggested that with careful implementation, lower precision models can be as effective as their higher precision counterparts while requiring less computational and memory resources. Building upon these findings, significant advancements have been made in utilizing 16-bit operations for training convolutional neural networks. Courbariaux et al. \cite{Courbariaux2016} pioneered a novel technique to train neural networks using binary weights and activations, significantly reducing memory and computational demands. Similarly, Micikevicius et al. \cite{Micikevicius2017} stressed the imperative transition from 32-bit to 16-bit operations in deep learning, given the memory and computational constraints associated with training increasingly complex neural networks. Their research demonstrated that half-precision computations offer a more memory and computation-efficient alternative without compromising the model's performance. Despite these advancements, numerical instability during backpropagation remains a persistent challenge. Bengio et al. \cite{Bengio1994} illustrated the difficulties encountered in learning long-term dependencies with gradient descent due to numerical instability. Such findings underscore the need for innovative solutions to mitigate this widespread issue. One critical component to addressing this challenge lies in the development of effective optimizers. Adam, an optimizer introduced by Kingma and Ba \cite{kingma2014adam}, is known to face issues of numerical instability when employed in lower-precision environments. Another research by Yun et al. \cite{yun2023defense}, provided a comprehensive theoretical analysis, focusing on the performance of pure 16-bit floating-point neural networks. They introduced the concepts of floating-point error and tolerance to define the conditions under which 16-bit models could approximate their 32-bit counterparts. Their results indicate that pure 16-bit floating-point neural networks can achieve similar or superior performance compared to their mixed-precision and 32-bit counterparts, offering a unique perspective on the benefits of pure 16-bit networks. Lastly, in the context of efficient neural network training, Han et al. \cite{Han2015} proposed a three-stage pipeline that significantly reduces the storage requirements of the network. Their work forms an integral part of the broader discussion on efficient neural network training, further reinforcing the relevance of our research on 16-bit operations. Through our investigation, we aim to contribute to this body of work by presenting a novel approach to addressing numerical instability issues associated with 16-bit precision in the realm of deep learning model training.

\section{Analysis}
Through theoretical analysis, we analyze which part of the neural network occurs numerical instability which can ruin the training process.

\subsection{Forward Propagation}
\noindent\textbf{Linear Network.} During forward propagation, each column of the input matrix $X$ represents a distinct training example. This matrix layout simplifies the network's computation, as each feature of every training sample can be processed in parallel, exploiting the parallel nature of matrix operations.
\begin{equation*}
X = \begin{bmatrix}
x_{11} & x_{12} & \cdots & x_{1m} \\
x_{21} & x_{22} & \cdots & x_{2m} \\
\vdots & \vdots & \ddots & \vdots \\
x_{n1} & x_{n2} & \cdots & x_{nm}
\end{bmatrix}
\end{equation*}
The weight matrix $W$ represents the strength and direction of connections between neurons. Each column of $W$ corresponds to the set of weights connecting every input neuron to a specific hidden neuron.
\begin{equation*}
W = \begin{bmatrix}
w_{11} & w_{12} & \cdots & w_{1k} \\
w_{21} & w_{22} & \cdots & w_{2k} \\
\vdots & \vdots & \ddots & \vdots \\
w_{n1} & w_{n2} & \cdots & w_{nk}
\end{bmatrix}
\end{equation*}
The bias vector $b$ provides an additional degree of freedom, allowing each neuron in the hidden layer to be activated not just based on weighted input but also based on this inherent property.
\begin{equation*}
b = \begin{bmatrix}
b_{1} \\
b_{2} \\
\vdots \\
b_{k}
\end{bmatrix}
\end{equation*}
The raw outputs or pre-activations (denoted as $Z$) are computed by multiplying the transposed weight matrix with the input matrix and adding the bias. 
\begin{equation*}
Z = W^T X + b
\end{equation*}
After computing $Z$, we need a mechanism to introduce non-linearity into our model. Without this, no matter how deep our model, it would behave just like a single-layer perceptron. This non-linearity is introduced using an activation function $\sigma(\cdot)$, applied element-wise to the matrix $Z$.
\begin{equation*}
A = \sigma(Z)
\end{equation*}
Here, each entry $a_{ij}$ in $A$ is the activated value of the $j$-th neuron in the hidden layer when the model is provided the $i$-th training example. These activations will either be used as input for subsequent layers or be the final output of the network.

In Deep Neural Network (Linear Network.), each operation in the forward pass involves straightforward mathematical operations like addition and multiplication. There is no division or complex operation that could amplify small numerical errors or lead to potential instability. \\

\noindent\textbf{Convolutional Network.} In the realm of deep learning, especially when processing image data, CNNs have gained significant prominence. The foundational blocks of CNNs involve convolving filters over input data and pooling layers \cite{lecun1998gradient}. 

The convolution layer can be detailed by observing the element-wise multiplication and summation \cite{lecun1998gradient}. Specifically, in our 3x3 input matrix \(I\) and 2x2 filter \(F\) example:
\begin{align*}
I &= \begin{bmatrix}
i_{11} & i_{12} & i_{13} \\
i_{21} & i_{22} & i_{23} \\
i_{31} & i_{32} & i_{33}
\end{bmatrix}
&
F &= \begin{bmatrix}
f_{11} & f_{12} \\
f_{21} & f_{22}
\end{bmatrix}
\end{align*}
For the top-left corner of \(I\), the convolution operation using the filter \(F\) is as follows:
\begin{equation*}
o_{11} = \begin{bmatrix}
i_{11} & i_{12} \\
i_{21} & i_{22}
\end{bmatrix} 
\odot 
\begin{bmatrix}
f_{11} & f_{12} \\
f_{21} & f_{22}
\end{bmatrix}
\end{equation*}
Where \(\odot\) represents element-wise multiplication. This implies:
\begin{equation*}
o_{11} = i_{11} \cdot f_{11} + i_{12} \cdot f_{12} + i_{21} \cdot f_{21} + i_{22} \cdot f_{22}
\end{equation*}
The filter \(F\) continues sliding across \(I\), performing similar computations for each position, resulting in the matrix:
\begin{equation*}
O = \begin{bmatrix}
o_{11} & o_{12} \\
o_{21} & o_{22}
\end{bmatrix}
\end{equation*}
Each element in the output matrix \(O\) is computed in the same fashion, with each \(o_{ij}\) being the result of convolving a 2x2 segment of \(I\) with the filter \(F\).

Pooling layers can also be represented using matrix notation, though the operation is simpler. For max pooling with a 2x2 window, given an input matrix $I$, the operation for a single position can be illustrated as \cite{maxpool}:
\begin{equation*}
P_{ij} = \max \begin{bmatrix}
I_{2i-1, 2j-1} & I_{2i-1, 2j} \\
I_{2i, 2j-1} & I_{2i, 2j}
\end{bmatrix}
\end{equation*}
For an example 4x4 input matrix being processed by a 2x2 max pooling operation:
\begin{equation*}
I = \begin{bmatrix}
i_{11} & i_{12} & i_{13} & i_{14} \\
i_{21} & i_{22} & i_{23} & i_{24} \\
i_{31} & i_{32} & i_{33} & i_{34} \\
i_{41} & i_{42} & i_{43} & i_{44}
\end{bmatrix}
\end{equation*}
The output matrix $P$ from this max pooling operation will be:
\begin{equation*}
P = \begin{bmatrix}
\max(i_{11}, i_{12}, i_{21}, i_{22}) & \max(i_{13}, i_{14}, i_{23}, i_{24}) \\
\max(i_{31}, i_{32}, i_{41}, i_{42}) & \max(i_{33}, i_{34}, i_{43}, i_{44})
\end{bmatrix}
\end{equation*}
In CNNs, actions taken during the forward pass primarily consist of basic arithmetic, such as addition and multiplication. There's an absence of division or intricate calculations that might magnify minor numerical inaccuracies or trigger potential instability. 

Especially those with a significant number of layers, the iterative multiplications during forward propagation can intermittently lead to either vanishingly small values or notably large increments in the activations. However, methods such as batch normalization\cite{batch} have been introduced to regulate and prevent these activations from attaining extreme values. Therefore, forward propagation, which involves the transmission of input data through the network to produce an output, is generally more resilient to such instabilities. This comparative robustness can be ascribed to a variety of underlying reasons:
\\
\begin{itemize}
\item Simplified Operations: Forward propagation predominantly involves elementary mathematical operations such as addition and multiplication. Thus, the results are less prone to reaching extreme values unless the input data or model parameters are improperly scaled or exhibit a high dynamic range \cite{He2015}.

\item Absence of Derivatives: Contrasting backpropagation, forward propagation does not necessitate the computation of derivatives, a process that could engender exceedingly large or small numbers, thus inducing numerical instability \cite{Rumelhart1986}.

\item Limited Propagation of Errors: Forward propagation is less likely to accumulate and propagate numerical errors throughout the network. This contrasts with backpropagation, where errors could proliferate through the derivative chain \cite{lecun1998gradient}. \\
\end{itemize}

% \subsection{Gradient Analysis}
% Figure \ref{fig:DNN} illustrates the gradient behaviors during the training epochs of a Deep Neural Network (DNN) structured with three dense layers, each comprising 2048 units and applying a custom Rectified Linear Unit (ReLU) activation function. The left subplot vividly portrays the maximum gradient values, revealing an initial peak that tends to stabilize in subsequent epochs. The right subplot, on the other hand, depicts the minimum gradient values.
% % Figure environment removed
% Most importantly, it is discernible from the graphs that the gradient values predominantly lie within the float16 numerical representation range, which is approximately between -65504 and 65504. This observation is paramount, substantiating that the 16-bit DNN manifests a minimized susceptibility to overflow occurrences during its training phase. This intrinsic robustness against overflow enhances the reliability and stability of the network's training process, supporting its effective implementation and operation.

% % Figure environment removed

% Figure \ref{fig:CNN} shows the gradient behaviors within a Convolutional Neural Network (CNN) utilizing a ResNet-56 architecture during the training epochs. The examination is focused on discerning the susceptibility of the network to gradient overflow issues, considering the computational constraints of float16 numerical representation.

% The left subplot delineates the maximum gradient values through the training epochs, uncovering a pronounced variability, notably within the initial epochs, which then navigates towards stabilization in the subsequent epochs. Concurrently, the right subplot illustrates the minimum gradient values, which, while unveiling substantial fluctuations, also trends towards stabilization across the epochs.

% A pivotal observation conveyed through these visualizations is the encapsulation of gradient values primarily within the numerical boundaries of float16 representation, ranging approximately from -65504 to 65504. This crucial observation illustrates the resilience of the 16-bit ResNet-56 architecture against potential overflow challenges during the training process. The empirical evidence presented through these gradients reinforces the practical feasibility of deploying the 16-bit CNN ResNet-56 architecture, ensuring robustness and stability throughout the training lifecycle.

\subsection{Backward Propagation}
We focused on backpropagation rather than forward propagation. The reason for this is that forward propagation consists of the product and sum of the matrix without division operations. In pursuit of a thorough mathematical scrutiny, our objective pivots on three principal tenets: firstly, to dissect and understand the operational intricacies of each optimizer; secondly, to pinpoint the exact circumstances and causes that give rise to numerical instability; and finally, to formulate effective solutions that can aptly mitigate these issues. 

\subsubsection{Gradient Analysis}
The gradient computations is crucial in the training of neural networks. Accurate gradient propagation is vital for effective learning, as it directs the optimization algorithm to adjust the model weights, thereby minimizing the loss function. This section investigates the gradient behaviors within DNN and CNN architectures to ensure the learning process's stability and efficiency. We meticulously analyze the gradient magnitudes throughout the training epochs to identify any potential numerical issues that could hinder the network's learning capabilities based on Floating Point 32-bit. Although overflow is a commonly acknowledged risk, our findings indicate that underflow might pose a more insidious threat, which we will address in subsequent sections.

\textbf{Deep Neural Network}\\
In the training progression of a DNN with a linearly connected architecture, each comprising 2048 neurons, Figure \ref{fig:DNN} presents the gradient dynamics. The left panel of the figure clearly traces the maximum gradient values, displaying an initial peak followed by a leveling off, suggesting stabilization as training progresses. In contrast, the right panel sheds light on the minimum gradient values, mapping their trajectory across the training epochs.

% Figure environment removed

% Figure environment removed

\textbf{Convolutional Neural Network}

Similarly, Figure \ref{fig:CNN} examines the gradient behavior within a CNN that utilizes the ResNet-56 architecture\cite{he2016deep} during its training epochs. The examination is centered on the network's resilience to gradient overflow, considering the computational constraints of the float16 numerical format. The left panel of the figure reveals the maximum gradient values, showing significant variability in the initial stages of training, which then tend towards stabilization in later epochs. The right panel, concurrently, depicts the minimum gradient values, which, despite noticeable fluctuations, also demonstrate a trend towards stabilization with ongoing training. At this time, the reason why the results of this experiment are small, unlike the above DNN results, is that Batch Normalization\cite{batch} is used as explained in the previous section to prevent the value from increasing.

A critical examination of the gradient behaviors in both the DNN and CNN architectures reveals that the gradients predominantly remain within the float16 numerical domain, which spans approximately from -65504 to 65504 \cite{ieee754}. This observation is crucial as it confirms that both the 16-bit DNN and CNN architectures exhibit a reduced risk of numerical overflow during training. This resilience to overflow is indicative of the architectures' inherent stability and reliability, which are imperative for the networks' operational integrity and successful deployment.

The analysis thus far has concentrated on mitigating overflow in gradient values, a vital component of numerical stability in neural network training. The consistent containment of gradient values within the float16 range underscores the robustness of the 16-bit architectures against overflow, ensuring their viability during the training process.

\subsection{Update Rule}
Although DNN and CNN are robust about overflow, there might be underflow issue during training. Given the potential for underflow, it is essential to consider its implications on the update rule, which is the cornerstone of the learning process in neural networks. The update rule is sensitive to the scale of the gradients; if the gradients are too small, they may not effect meaningful change in the parameters, leading to stalled training or suboptimal convergence. This concern is not merely theoretical but has practical ramifications for the design and implementation of neural network training algorithms. In this sub-section, we will explain about the risk of several update rules regarding underflow. To exemplify, consider the update rule in the gradient descent method for optimizing a neural network \cite{ruder2017overview}:
\begin{align*}
\theta = \theta - \eta \nabla_{\theta} J(\theta)
\end{align*}
In this equation, $\theta$ denotes the parameters of the model, $\eta$ represents the learning rate, and $\nabla_{\theta} J(\theta)$ signifies the gradient of the loss function $J(\theta)$ with respect to the parameters. Understanding such equations allows us to uncover the internal mechanics of each optimizer, thereby facilitating our quest to alleviate numerical instability.
\subsection{Mini-Batch Gradient Descent}
Mini-batch gradient descent is a variant of the gradient descent algorithm that divides the training datasets into small batches to compute errors \cite{ruder2017overview}. The update rule for the parameters in mini-batch gradient descent can be written as:
\begin{align*}
\theta =& \theta - \eta \nabla_{\theta} J(\theta; x^{(i:i+n)}; y^{(i:i+n)})
\end{align*} 
This optimizer leverages mini-batches of $m$ examples $(x^{(i:i+n)}, y^{(i:i+n)})$ from the training dataset ${x^{(i)},..., x^{(n)}}$ to update the model's parameters $\theta \in \mathbb{R}^{d}$. It aims to minimize the loss function $J(\theta)$ by taking calculated steps that are antithetical to the gradient of the function with respect to the parameters \cite{ruder2017overview}. The magnitude of these steps is dictated by the learning rate, denoted by $\eta$. It is postulated that Mini-Batch Gradient Descent can function optimally in a 16-bit environment, given the limited number of hyperparameters that can provoke numerical instability within the neural network during the process of minimizing $J(\theta)$.
\\
\begin{itemize}
\item \textbf{Assumption 3.1} \textit{Consider the input image data $x \in X{x_1,…,x_i}$, the label $y \in Y{y_1,…,y_i}$, where each $x \in \mathbb{R}$ satisfies $0<x<255$, and each $y \in \mathbb{Z}$. The normalized data $\bar{x}$ will lie in the range $0.0<\Bar{x}<1.0$. If $fp16_{min}<f(x,y)=\nabla_{\theta} J(\theta; x^{(i)}; y^{(i)}) )<fp16_{max}$, and $f(x,y)$ does not engender numerical instability $I={NaN, Inf}$ during the update of $\theta$, it can be deduced that the mini-batch gradient descent method will function properly for training the 16-bit neural network.} \\
\end{itemize}
Based on \textit{Assumption 3.1}, the number of hyperparameters in the Mini-Batch Gradient Descent that could instigate numerical instability is restricted. Consequently, we anticipate that this optimizer should function efficiently within a 16-bit neural network with default hyperparameters.

\subsection{RMSProp}
Root Mean Square Propagation, more commonly referred to as RMSProp, is an optimization methodology utilized in the training of neural networks. This technique was proposed by Geoff Hinton during his Coursera class in Lecture 6e \cite{tieleman2012lecture}. However, when implemented within a 16-bit neural network using default hyperparameters, RMSProp results in the model weights turning into $NaN$, thereby causing a complete halt in the training process. 

% Figure environment removed

Figure \ref{fig:RMS} shows that when training MNIST with a 16-bit neural network using RMSProp, the weights become NaN (white points) after a few epochs if a value less than 1e-4 is used as epsilon. To circumvent this numerical problem, it is essential to understand the source of numerical instability that RMSProp introduces in 16-bit environments. The update rule for RMSProp can be expressed as follows:
\begin{align*}
w_t =& w_{t-1} - \eta \frac{g_t}{\sqrt{v_t} + \epsilon}
\end{align*}
The 16-bit floating-point representation has a more constrained computational scope in contrast to the 32-bit variant. When values exceed or fall below the valid range of $fp16$, numerical instability emerges during the training sequence. In the context of the RMSProp optimizer, the learning rate $\eta$ typically doesn't instigate issues provided $\eta > fp16_{min}$. However, the denominator $v_t$, adaptive learning rate, and $\epsilon$ can trigger numerical instability if their values extend beyond the allowable range.
\begin{align*}
v_t =& {\begin{cases}
0 & if \: v_t < fp16_{min} \\
{\beta} {v_{t-1}} + {(1 - \beta) {{g_t}^{2}}} & otherwise\\
\infty & if \: v_t > fp16_{max}
\end{cases}}
\end{align*}
If the value of ${v_t}$ falls below $fp16_{min}$, ${v_t}$ converges to zero, which in turn affects $w_t$. The expression for $w_t$ can then be rearranged so that only $\epsilon$ remains in the denominator. Under these circumstances, the numerical instability of the optimizer is primarily induced by the default $\epsilon$ value. In TensorFlow, the default value for $\epsilon$ is set to 1e-07, and the default learning rate $\eta$ is configured as 1e-03.
\begin{align*}
w_t =& w_{t-1} - \eta \cdot {g_t} \cdot {\epsilon}^{-1}
\end{align*} 
The adjustment of weights is contingent upon the $\epsilon$ value. Nevertheless, during this process, numerical instability may emerge when this value is inversed, leading to underflow and overflow incidents in the optimizer.
\begin{align*}
\epsilon^{-1} =& 
{\begin{cases}
\infty & if \: \epsilon^{-1} > fp16_{Max} \\
\epsilon^{-1} & otherwise \\
\end{cases}}
\end{align*}
Should the value of epsilon decline to 1e-07, its reciprocal soars to 1e+07. While the 16-bit format accommodates 1e-07, it cannot handle 1e+07, leading to an overflow that escalates to $\infty$. This overflow is an inevitable outcome, independent of $g_t$, the gradient's value. The progressive unfolding of such overflow instances stirs up numerical instability, eventually compromising the integrity of the neural network. Whenever $\epsilon^{-1} > fp16_{max}$, the term $g_{t} \cdot \epsilon^{-1}$ will invariably result in $I = {\infty, -\infty}$, contingent on the sign of $g_t$.
\begin{align*}
w_t =& 
{\begin{cases}
w_{t-1} - \infty & if \: g_t > 0 \\
w_{t-1} + \infty & if \: g_t < 0 \\
\end{cases}}
\end{align*}
The current weight is represented by $w_{t} = w_{t-1} - i$ ($i \in I$) which yields ${NaN, -\infty}$ when $w_{t-1} \in {\infty, -\infty}$. Upon the occurrence of $NaN$, this optimizer ceases to function on a 16-bit neural network, a consequence of the TensorFlow default epsilon setting being at 1e-07.
\\
\begin{itemize}
\item \textbf{Assumption 3.2} \textit{We posit that the input image data $x \in X{x_1,…,x_i}$ and label $y \in Y{y_1,…,y_i}$, where all $x \in \mathbb{R}$, $0<x<255$, and all $y \in \mathbb{Z}$. The normalized data $\bar{x}$ will lie within the range of $0.0<\Bar{x}<1.0$. Similarly, the gradient $g_t$ will fall within the range $fp16_{min}<g_t<fp16_{max}$. There exist two conditions under which RMSProp can operate effectively in a 16-bit environment. Firstly, if $v_t \ne 0$, $v_t \ge fp16_{min}$ and $fp16_{min} < \sqrt{v_t} + \epsilon < fp16_{max}$, $w_t$ will not undergo $overflow$. Secondly, when $v_t < fp16_{min}$, $v_t$ becomes $0$. In this case, if $g_t \cdot \epsilon^{-1} < fp16_{max}$, RMSProp successfully evades critical numerical instability. Provided either of these conditions are met, the RMSProp optimizer will function appropriately in a 16-bit neural network.} \\ 
\end{itemize}
In the event that \textit{Assumption 3.2} holds true, the RMSProp optimizer is capable of executing effectively in 16-bit neural networks. An improper selection of the epsilon parameter could misdirect the course of learning in the 16-bit neural network, potentially causing numerical instability. This may lead to erroneous conclusions regarding the compatibility of the RMSProp optimizer with 16-bit neural networks.

\subsection{Adam} 
Adam is a sophisticated optimization algorithm that amalgamates the salient features of the RMSProp and Momentum methods. Conceived by Diederik Kingma and Jimmy Ba, the Adam optimizer implements the technique of adaptive moment estimation to iteratively refine neural network weight calculations with enhanced efficiency \cite{kingma2014adam}. 
% Figure environment removed
By extending stochastic gradient descent, Adam optimally addresses non-convex problems at an accelerated pace, requiring fewer computational resources compared to a host of other optimizers. The potency of this algorithm is particularly noticeable when dealing with large datasets, as it maintains a tighter trajectory over numerous training iterations.
\begin{align*}
w_t =& w_{t-1} - \eta \cdot \frac{\hat{m_t}}{\sqrt{\hat{v_t}} + \epsilon}
\end{align*} 
Figure \ref{fig:RMS} shows that when training MNIST with a 16-bit neural network using Adam, the weights become NaN (white points) after a few epochs if a value less than 1e-3 is used as epsilon. When considering the Adam optimizer, numerical instability arises following principles akin to those governing RMSProp. However, an additional distinguishing element in Adam is the introduction of a momentum variable, $m$. In the Adam optimizer, both $m$ and $v$ are initialized to zero. Hence, at the onset of learning, both $m_t$ and $v_t$ are inclined towards zero, undergoing an initial process to remove this bias.
\begin{align*}
m_t =& \beta_1 \cdot m_{t-1}  + (1 - \beta_1) \cdot g_t \\ 
v_t =& {\beta_2} \cdot {m_{t-1}} + {(1 - \beta_2) \cdot {{g_t}^{2}}}
\end{align*}
The velocity variable $v_t$ in the Adam optimizer is a non-negative real number. The adjusted velocity $\hat{v}t$ is calculated as the ratio of $v_t$ to $(1-\beta{2})$. Therefore, if $v_t$ is less than the maximum value permissible in 16-bit floating point precision, $fp16_{max}$, then $\hat{v_t}$ will approach zero. This condition significantly influences the calculated value of $w_t$ within the Adam optimizer.
\begin{align*}
w_t =& w_{t-1} - \eta \cdot \hat{m_t} \cdot {\epsilon^{-1}}
\end{align*}
In the event where the inverse of epsilon ($\epsilon^{-1}$) exceeds the maximum value permissible in a 16-bit floating point ($fp16_{max}$), $\epsilon^{-1}$ essentially becomes infinite. The default value of $\epsilon$ in TensorFlow is $1e-07$, while the default learning rate ($\eta$) is $1e-03$. Therefore, if $\epsilon$ is $1e-07$, its reciprocal escalates to infinity. Depending on the sign of the adjusted momentum ($\hat{m_t}$), the value of $\eta \cdot \hat{m_t} \slash \epsilon$ will either be positive or negative infinity. Consequently, irrespective of the learning rate $\eta$, $\hat{m_t} \slash \epsilon$ will belong to the set $I = {\infty, -\infty}$. As a result, the new weight update equation $w_{t} = w_{t-1} - i$ ($i \in I$) leads to a situation where $w_t$ belongs to the set ${NaN, -\infty}$ whenever $w_{t-1}$ is ${\infty, -\infty}$. This introduces numerical instability in the learning process.
\begin{align*}
w_t =& 
{\begin{cases}
w_{t-1} - \infty & if \: \hat{m_t} > 0 \\
w_{t-1} + \infty & if \: \hat{m_t} < 0 \\
\end{cases}}
\end{align*}
This process induces a numerical instability in a 16-bit neural network. In particular, if the previous weight $w_{t-1}$ is $-\infty$, the operation $w_{t-1}$ - $\infty$ returns $-\infty$, but $w_{t-1}$ + $\infty$ produces a non-numeric value ($NaN$). Consequently, the current weight $w_t$ becomes $NaN$, thereby halting training within the 16-bit Neural Network utilizing the Adam optimizer.
\\
\begin{itemize}
\item \textbf{Assumption 3.3} posits that the input image data $x \in X{x_1,…,x_i}$, label $y \in Y{y_1,…,y_i}$, where all $x \in \mathbb{R}$ satisfy $0<x<255$, and all $y \in \mathbb{Z}$. The normalized data $\bar{x}$ from $x$ will lie in the range $0.0<\Bar{x}<1.0$, and the momentum $m_t$ will fall within $fp16_{min}<m_t<fp16_{max}$. Two conditions may allow the Adam optimizer to function in a 16-bit setting. Primarily, if $\hat{v_t} \ne 0$, $\hat{v_t} \ge fp16_{min}$ and $fp16_{min} < \sqrt{\hat{v_t}} + \epsilon < fp16_{max}$, the weight $w_t$ will not overflow. Additionally, when $\hat{v_t} < fp16_{min}$, $\hat{v_t}$ becomes $0$. If $\hat{m_t} \cdot \epsilon^{-1} < fp16_{max}$, the Adam optimizer circumvents critical numerical instability. Provided one of these conditions is met, the Adam optimizer functions effectively in a 16-bit neural network. \\
\end{itemize}
Given that the Adam optimizer incorporates the method of the RMSProp optimizer, similar issues impede its performance in a 16-bit Neural Network. However, if Adam satisfies \textit{Assumption 3.3}, the optimizer can function within a 16-bit Neural Network. Proper comprehension and handling of the epsilon issue enables the utilization of a variety of optimizers even within 16-bit computations.

Our exploration into the root causes of numerical instability has led us to a singularly intriguing finding - the value of the hyperparameter, Epsilon, plays a remarkably pivotal role. This discovery, though unassuming at first glance, has far-reaching implications in our quest for numerical stability within the realm of neural network optimization. Epsilon is often introduced in the denominator of certain equations to prevent division by zero errors or to avoid the pitfalls of numerical underflow. In the context of optimizers such as RMSProp and Adam, it's usually involved in the update rule, safeguarding against drastic weight changes arising from exceedingly small gradient values. However, in a 16-bit computational environment, when the reciprocal of Epsilon becomes larger than the maximum representable number, it leads to a condition known as numerical overflow. The overflow subsequently manifests as numerical instability, disrupting the otherwise orderly progression of the learning process, and effectively halting the training of the 16-bit neural network.

\noindent\textbf{Epsilon.} Consider the function $f(x) = \frac{1}{x}$. As $x$ nears 0, $f(x)$ grows indefinitely. By adding $\epsilon$ (i.e., $f(x) = \frac{1}{x + \epsilon}$), the function remains bounded near 0. In optimization, this ensures parameter updates stay bounded, aiding numerical stability during training. However, in low precision scenarios, the presence of $\epsilon$ can induce instability, as gradients become overshadowed by this constant. This isn't unique to the Adam optimizer and adjusting Epsilon's value isn't straightforward, with its ideal value often being context-specific. We've developed a method to address this, offering a systematic approach to Epsilon tuning, ensuring stable optimization amidst numerical challenges.

\section{Method}
\subsection{Existing Method}
\textbf{Loss Scaling in Mixed Precision}: Mixed precision training \cite{Micikevicius2017} is a technique that utilizes both 16-bit and 32-bit floating-point types during training to make it more memory efficient and faster. However, the reduced precision of 16-bit calculations can sometimes lead to numerical underflow, especially in the intermediate gradient values in some optimizers such as Adam and RMSProp. To prevent this, the "loss scaling" method is employed. Loss scaling is a straightforward concept. Before performing backpropagation, the loss is scaled up by a large factor. This artificially increases the scale of the gradients and thus helps prevent underflow. After the gradients are calculated, they are scaled back down to ensure the weights are updated correctly. In essence, this process amplifies the small gradient values to make them representable in the reduced precision and then scales them back down to ensure that the actual model updates remain accurate. While mixed precision does use 16-bit calculations, it leans on 32-bit calculations for the critical parts to maintain accuracy, especially for maintaining running sums and optimizer states. This implies that we aren't fully exploiting the speed and memory benefits of 16-bit precision. According to Table \ref{tabmp}, Mixed precision based VIT-16 will not reduce GPU memory usage because it utilizes 32-bit computation for mixed precision. 
\begin{table*}[ht]
\caption{Performance comparison between 16-bit, 32-bit, and Mixed Precision 
settings for training VIT-16\cite{dosovitskiy2020image} with cifar-10 images and Adam optimizer.}
\vspace{0.25cm}
\small
\centering
% \begin{adjustbox}{width=1\textwidth}
\begin{tabular}{ccccc}
\hline
\hline
& Epsilon & Test Accuracy & Training Time (seconds) & GPU Memory Usage \\
\hline
Floating Point 16-bit & 1E-3 & 0.686 & 1255.35 & 1.234 GB \\ 
Floating Point 32-bit & 1E-3 & 0.693 & 1733.14 & 2.412 GB \\ 
Mixed Precision (MP) & 1E-3 & 0.685 & 1430.12 & 2.415 GB \\ 
\hline
\hline
\end{tabular}
% \end{adjustbox}
\label{tabmp}
\end{table*}

Contrary to the mixed precision method, a pure 16-bit neural network operates entirely in the 16-bit precision realm. There are several reasons why pure 16-bit computations can outshine both 32-bit and mixed precision. First, 16-bit operations are faster than their 32-bit and mixed precision. Second, 16-bit representations require half the memory of 32-bit, which means we can fit larger models or batch sizes into GPU memory, further enhancing performance. Yun et at\cite{yun2023defense} shows pure 16-bit neural networks are faster than both 32-bit and mixed precision. This means that mixed precision with loss scaling method cannot utilize advantages of 16-bit operations. The inefficiencies and the constant juggling between 16-bit and 32-bit in mixed precision, along with the need for methods like loss scaling, indicate that a more robust solution is needed for numerical stability in low precision environments. This is where the "Numerical Guarantee Method" comes into play. Instead of hopping between precisions or artificially scaling losses, the Numerical Guarantee Method aims to provide a stable environment for optimizers like Adam to function efficiently in pure 16-bit computations. This approach makes the training process faster and potentially more accurate than using 32-bit or mixed precision models.

\subsection{Numerical Guarantee Method}
In this study, we present a novel approach designed to alleviate numerical instability, a common problem associated with widely-used optimizers such as Adam and RMSProp, in the training of 16-bit neural networks. These optimizers have frequently encountered challenges when working with smaller epsilon values, particularly in instances involving division operations. The genesis of our method stems from an in-depth mathematical exploration of the update rules in the Adam optimizer, which revealed that instability originates during the gradient update step - a phase where the momentum term ($\hat{m}$) is divided by the sum of the square root of the velocity term ($\hat{v}$) and the epsilon parameter ($\epsilon_t$). To circumvent this instability, we propose a refined version of the gradient computation equation, which effectively prevents division by an exceedingly small epsilon value. This is achieved by introducing a maximum function that sets a lower threshold on the denominator. Consequently, the updated Adam optimizer follows the equation:
\begin{align*}
w_{t} & = w_{t-1} - \frac{\eta \cdot \hat{m_t}}{\sqrt{\max(\hat{v_t}, \epsilon)}}
\end{align*}
The modification guarantees that the denominator remains within a safer numerical range, thus facilitating a more stable gradient update. The addition of the stabilizing term, $\sqrt{\max(\hat{v_t}, \epsilon)}$, ensures the denominator does not dwindle excessively, thereby avoiding abnormally large updates. Should $\hat{v_t}$ fall below $\epsilon$, the term assumes the square root of $\epsilon$ instead. The precise value employed as the lower limit (in this case, $\epsilon$ is 1e-4) can be tailored according to the specific requirements of the neural network in consideration. Our proposed adjustment to the Adam optimizer offers a myriad of benefits, which are elaborated upon subsequently:
\\
\begin{itemize}
\item Enhanced Numerical Stability: By taking the maximum of $\hat{v_t}$ and a small constant inside the square root function, we mitigate the numerical instability that arises when $\hat{v_t}$ is very close to zero. This enhancement significantly reduces the chances of overflow and underflow during computations, which in turn increases the robustness of the optimizer.

\item Preservation of Adam's Benefits: The modification does not alter the fundamental characteristics and benefits of the Adam optimizer. It retains the benefits of Adam, such as efficient computation, suitability for problems with large data or parameters, and robustness to diagonal rescale or translations of the objective functions.

\item Ease of Implementation: The modification requires just a minor change to the original Adam algorithm, making it easy to implement in practice. This simplicity of modification allows for easy integration into existing machine learning frameworks and pipelines, thereby increasing its accessibility to practitioners.

\item Applicability to 16-bit Computations: The updated method enables Adam to work effectively on 16-bit neural networks, which was a challenge with the original Adam optimizer due to numerical instability issues. Thus, it extends the applicability of Adam to systems and applications where memory efficiency is crucial.

\item Flexibility: This modification is generalizable and can be applied to other optimizers based on the same concept as Adam, thereby providing a broad range of applications.\\
\end{itemize}

Overall, these advantages make this modification an effective solution to the problem of numerical instability in Adam and similar optimizers, particularly in the context of 16-bit computations. Our approach presents a novel way to alleviate numerical instability in 16-bit neural networks, without losing the advantages offered by advanced optimizers such as Adam and RMSProp. We will demonstrate the effectiveness of our approach through comprehensive experimental results in the following section.


\section{Results}
\subsection{Numerical Results}

% Figure environment removed


% Figure environment removed

% Figure environment removed
% Figure environment removed


% Figure environment removed

% Figure environment removed

\if11
% Figure environment removed
\fi

We present results using LM and NL1 for evaluating the approximations in \eqref{eq:approximations} and discuss the effect of 
the initial input $u_0$ and the decay rate of the form $\eta_k = \eta_0 k^{-c}$ on ILC performance.

\subsubsection{Effect of the model}
The results presented in Fig. \ref{fig:error-vs-iteration-init=xi} and Fig. \ref{fig:error-vs-iteration} show that for the iterations taken using gradient information from both the LM and the NL1, the error converges to a value of the same order of magnitude.
However, we see LM converge to a solution with lower deviation and at a faster rate compared to NL1.
This result is at first surprising given that the prediction error of the LM is one order of magnitude higher than NL1. We note however that NL1 is built with LeakyReLu activation functions, and thus its gradient is piecewise constant and discontinuous. Despite its higher prediction error, the structure of the LM is found to provide more accurate gradient information.
Different shapes and tunings of the cost function show mostly similar trends of the ILC loop using the linear and nonlinear models (data not shown).
We have observed that the system is only mildly nonlinear, and since the ILC step relies on measurements that do not depend on the models used, the fidelity of the approximations~\eqref{eq:approximations} is apparently not of critical importance. 
We hypothesize that a system with more pronounced nonlinearities would experience faster convergence with ILC steps relying on the nonlinear model for the approximations.
In Fig. \ref{fig:deviation-linearmodel} and Fig. \ref{fig:deviation-nonlinearmodel}, we show the output deviation as a function of time before and after $20$ steps of the proposed method. The deviations are computed between the output and the target trajectory depicted in Fig. \ref{fig:xy-detail-linearmodel}, where the output trajectories are plotted in $x-y$ coordinates.
Empirically, in previous studies, we found that the best rms error that can be obtained in this setup is close to the steady state values to which the ILC converges.
\subsubsection{Effect of initialization}
In Fig. \ref{fig:error-vs-iteration}, we show the results for two different initial conditions.
The results illustrate that since the underlying problem is nonlinear it is possible to be in a local minimum and not achieve a better solution, depending on the initial conditions.



\subsubsection{Effect of step size}


Next, we study the effect of step size on convergence behavior. We take $\eta_k = \eta_0 k^{-c}$ and compare the convergence.
We show the error trajectories for $c = \{0.2, 0.5, 0.9\}$ (Fig. \ref{fig:normalised-vs-iteration}).
In Fig. \ref{fig:rms-vs-c} we plot the error after $10$ and $50$ iterations for different values of $c$.
We observe that for $c$ between $0.3$ and $0.6$, we obtain fast decreases in the error without compromising the value at a steady state.


\if 01
\eb{
\begin{itemize}
    \item Comparison between different models starting from the same initial condition. 
    \item Compare to linear ILC and use the linear model for the Hessian etc. to show what we gain in simulation results
    \item Use the best-performing model and maybe do a plot on the effect of the initial condition to show the sensitivity to the initial condition, which should be part of the limitations of the method. Could make sense to compare against the linear ILC maybe? 
    \item (A side note that we could add only if we have time/space) If we have the 500ms NN as both model and system, we can show that the method (hopefully) almost converges to the optimization of the 500ms NN directly. This would be an additional motivation to use this framework for the optimization of such NN models instead of doing expensive optimization directly. 
\end{itemize}
}
\fi



\section{Discussion}
The novel approach for mitigating numerical instability presented in this work has demonstrated promising results with respect to the backpropagation phase of deep learning models when dealing with a simple Linear Layer. Our modification to the gradient computation equation within the Adam optimizer ensures that the denominator stays within a safe range, thereby providing a stable update for the gradient. This has the potential to significantly enhance the stability of the training process, and by extension, the accuracy of the resultant model. However, as in all scientific studies, this work is not without its limitations. The empirical validation of our proposed method has been conducted with image classification model such as the vision transformer models. While this forms the basis for more complex architectures, it remains only one part of training of the wide array of layers and structures currently utilized in modern deep learning architectures. Specifically, future research should aim to validate the robustness and utility of our approach across a broader spectrum of deep learning architectures. In particular, Transformer models for Natural Language Processing is a significant portion of the current deep learning landscape, particularly in language-related tasks. The complexity of these models, with their intricate hierarchies and highly non-linear transformations, may pose additional challenges that are not fully encapsulated by a Linear Layer. As such, it is imperative that further validation is conducted on these architectures to establish the generalizability of our method. Furthermore, while our focus has primarily been on mitigating numerical instability, it would also be beneficial to investigate any potential side effects this approach might have on other aspects of the model training process. For instance, it would be interesting to explore the implications for training time, memory requirements, and robustness to variations in hyperparameters. In conclusion, this work presents an exciting step forward in the pursuit of more robust and stable training methodologies for deep learning models. The road ahead, while challenging, is filled with opportunities for further innovation and refinement.

\section{Conclusions}
\section{Conclusion}
\label{sec:conclusion}
In this paper, we present the first comprehensive evaluation of multiple LLMs (Alpaca, Alpaca-LoRA, GPT-3.5) on mental health prediction tasks (binary and multi-class classification) via online text data. 
Our experiments cover zero-shot prompting, few-shot prompting, and instruction finetuning. The results reveal a number of interesting findings.
Our context enhancement strategy can robustly improve performance for all LLMs, and our mental health enhancement strategy can enhance models with large number of trainable parameters.
Meanwhile, few-shot prompting can also robustly improve model performance even by providing just one example per class.
Most importantly, our experiments show that instruction finetuning across multiple datasets can significantly boost model performance on various mental health prediction tasks at the same time. Our best finetuned model Mental-Alpaca performs on par with the state-of-the-art task-specific model Mental-RoBERTa.
We summarize our findings as a set of guidelines for future researchers, developers, and practitioners who want to empower LLMs with better knowledge of mental health for downstream tasks.



% \section*{ACKNOWLEDGMENTS}


\bibitem{shen2010spectral}
H.-W. Shen and X.-Q. Cheng, ``Spectral methods for the detection of network
  community structure: a comparative analysis,'' \emph{JSTAT}, 2010.

\bibitem{wong2018sdregion}
S.~W. Wong, C.~Pastrello, M.~Kotlyar, C.~Faloutsos, and I.~Jurisica,
  ``Sdregion: Fast spotting of changing communities in biological networks,''
  in \emph{SIGKDD'18}, 2018.

\bibitem{liu2019coupled}
Y.~Liu, L.~Zhu, P.~Szekely, A.~Galstyan, and D.~Koutra, ``Coupled clustering of
  time-series and networks,'' in \emph{SDM}.\hskip 1em plus 0.5em minus
  0.4em\relax SIAM, 2019.

\bibitem{cohen2003reachability}
E.~Cohen, E.~Halperin, H.~Kaplan, and U.~Zwick, ``Reachability and distance
  queries via 2-hop labels,'' \emph{SIAM Journal on Computing}, vol.~32, no.~5,
  pp. 1338--1355, 2003.

\bibitem{jin20093}
R.~Jin, Y.~Xiang, N.~Ruan, and D.~Fuhry, ``3-hop: a high-compression indexing
  scheme for reachability query,'' in \emph{Proceedings of the 2009 ACM SIGMOD
  International Conference on Management of data}, 2009, pp. 813--826.

\bibitem{goldberg1984finding}
A.~V. Goldberg, ``Finding a maximum density subgraph,'' 1984.

\bibitem{charikar2000greedy}
M.~Charikar, ``Greedy approximation algorithms for finding dense components in
  a graph,'' in \emph{APPROX'00}, 2000.

\bibitem{danisch2017large}
M.~Danisch, T.-H.~H. Chan, and M.~Sozio, ``Large scale density-friendly graph
  decomposition via convex programming,'' in \emph{WWW}, 2017.

\bibitem{sawlani2020near}
S.~Sawlani and J.~Wang, ``Near-optimal fully dynamic densest subgraph,'' in
  \emph{Proceedings of the 52nd Annual ACM SIGACT Symposium on Theory of
  Computing}, 2020, pp. 181--193.

\bibitem{boob2020flowless}
D.~Boob, Y.~Gao, R.~Peng, S.~Sawlani, C.~Tsourakakis, D.~Wang, and J.~Wang,
  ``Flowless: Extracting densest subgraphs without flow computations,'' in
  \emph{Proceedings of The Web Conference 2020}, 2020, pp. 573--583.

\bibitem{chekuri2022densest}
C.~Chekuri, K.~Quanrud, and M.~R. Torres, ``Densest subgraph: Supermodularity,
  iterative peeling, and flow,'' in \emph{Proceedings of the 2022 Annual
  ACM-SIAM Symposium on Discrete Algorithms (SODA)}.\hskip 1em plus 0.5em minus
  0.4em\relax SIAM, 2022, pp. 1531--1555.

\bibitem{harb2022faster}
E.~Harb, K.~Quanrud, and C.~Chekuri, ``Faster and scalable algorithms for
  densest subgraph and decomposition,'' \emph{Advances in Neural Information
  Processing Systems}, vol.~35, pp. 26\,966--26\,979, 2022.

\bibitem{khuller2009finding}
S.~Khuller and B.~Saha, ``On finding dense subgraphs,'' in \emph{Automata,
  Languages and Programming: 36th International Colloquium, ICALP 2009, Rhodes,
  Greece, July 5-12, 2009, Proceedings, Part I 36}.\hskip 1em plus 0.5em minus
  0.4em\relax Springer, 2009, pp. 597--608.

\bibitem{tatti2015density}
N.~Tatti and A.~Gionis, ``Density-friendly graph decomposition,'' in
  \emph{WWW}, 2015.

\bibitem{ma2022finding}
C.~Ma, R.~Cheng, L.~V. Lakshmanan, and X.~Han, ``Finding locally densest
  subgraphs: a convex programming approach,'' \emph{Proceedings of the VLDB
  Endowment}, vol.~15, no.~11, pp. 2719--2732, 2022.

\bibitem{tatti2019density}
N.~Tatti, ``Density-friendly graph decomposition,'' \emph{ACM Transactions on
  Knowledge Discovery from Data (TKDD)}, vol.~13, no.~5, pp. 1--29, 2019.

\bibitem{fang2019efficient}
Y.~{Fang}, K.~{Yu}, R.~{Cheng}, L.~V. {Lakshmanan}, and X.~{Lin}, ``Efficient
  algorithms for densest subgraph discovery,'' \emph{arXiv: Databases}, 2019.

\bibitem{bibby1974axiomatisations}
J.~Bibby, ``Axiomatisations of the average and a further generalisation of
  monotonic sequences,'' \emph{Glasgow Mathematical Journal}, vol.~15, no.~1,
  pp. 63--65, 1974.

\bibitem{jure2014snapnets}
J.~Leskovec and A.~Krevl, ``{SNAP Datasets}: {Stanford} large network dataset
  collection,'' \url{http://snap.stanford.edu/data}, Jun. 2014.

\bibitem{wan2019aminer}
H.~Wan, Y.~Zhang, J.~Zhang, and J.~Tang, ``Aminer: Search and mining of
  academic social networks,'' \emph{Data Intelligence}, 2019.

\bibitem{nr2015aaai}
R.~A. Rossi and N.~K. Ahmed, ``The network data repository with interactive
  graph analytics and visualization,'' in \emph{AAAI}, 2015. [Online].
  Available: \url{http://networkrepository.com}

\bibitem{ZafaraniLiu2009}
R.~Zafarani and H.~Liu, ``Social computing data repository at {ASU},'' 2009.
  [Online]. Available: \url{http://socialcomputing.asu.edu}

\bibitem{kunegis2013konect}
J.~Kunegis, ``Konect: the koblenz network collection,'' in \emph{WWW}, 2013,
  pp. 1343--1350.

\bibitem{hooi2016fraudar}
B.~Hooi, H.~A. Song, A.~Beutel, N.~Shah, K.~Shin, and C.~Faloutsos, ``Fraudar:
  Bounding graph fraud in the face of camouflage,'' in \emph{SIGKDD}, 2016.

\bibitem{veldt2021generalized}
N.~Veldt, A.~R. Benson, and J.~Kleinberg, ``The generalized mean densest
  subgraph problem,'' in \emph{Proceedings of the 27th ACM SIGKDD Conference on
  Knowledge Discovery \& Data Mining}, 2021, pp. 1604--1614.

\bibitem{lanciano2023survey}
T.~Lanciano, A.~Miyauchi, A.~Fazzone, and F.~Bonchi, ``A survey on the densest
  subgraph problem and its variants,'' \emph{arXiv preprint arXiv:2303.14467},
  2023.

\bibitem{luo2023survey}
W.~Luo, C.~Ma, Y.~Fang, and L.~V. Lakshman, ``A survey of densest subgraph
  discovery on large graphs,'' \emph{arXiv preprint arXiv:2306.07927}, 2023.

\bibitem{Gionis2015DSD}
A.~Gionis and C.~E. Tsourakakis, ``Dense subgraph discovery: Kdd 2015
  tutorial,'' in \emph{KDD}, 2015.

\bibitem{fang2022densest}
Y.~Fang, W.~Luo, and C.~Ma, ``Densest subgraph discovery on large graphs:
  applications, challenges, and techniques,'' \emph{Proceedings of the VLDB
  Endowment}, vol.~15, no.~12, pp. 3766--3769, 2022.

\bibitem{gallo1989fast}
G.~Gallo, M.~D. Grigoriadis, and R.~E. Tarjan, ``A fast parametric maximum flow
  algorithm and applications,'' \emph{SIAM Journal on Computing}, vol.~18,
  no.~1, pp. 30--55, 1989.

\bibitem{asahiro2000greedily}
Y.~Asahiro, K.~Iwama, H.~Tamaki, and T.~Tokuyama, ``Greedily finding a dense
  subgraph,'' \emph{Journal of Algorithms}, 2000.

\bibitem{feng2021specgreedy}
W.~Feng, S.~Liu, D.~Koutra, H.~Shen, and X.~Cheng, ``Specgreedy: unified dense
  subgraph detection,'' in \emph{Machine Learning and Knowledge Discovery in
  Databases: European Conference, ECML PKDD 2020, Ghent, Belgium, September
  14--18, 2020, Proceedings, Part I}.\hskip 1em plus 0.5em minus 0.4em\relax
  Springer, 2021, pp. 181--197.

\bibitem{miyauchi2018finding}
A.~Miyauchi and N.~Kakimura, ``Finding a dense subgraph with sparse cut,'' in
  \emph{CIKM}, 2018.

\bibitem{anagnostopoulos2020spectral}
A.~Anagnostopoulos, L.~Becchetti, A.~Fazzone, C.~Menghini, and
  C.~Schwiegelshohn, ``Spectral relaxations and fair densest subgraphs,'' in
  \emph{CIKM}, 2020.

\bibitem{Tsourakakis2019NovelDS}
C.~E. Tsourakakis, T.~Chen, N.~Kakimura, and J.~W. Pachocki, ``Novel dense
  subgraph discovery primitives: Risk aversion \& exclusion queries,''
  \emph{ECML-PKDD}, 2019.

\bibitem{fazzone2022discovering}
A.~Fazzone, T.~Lanciano, R.~Denni, C.~E. Tsourakakis, and F.~Bonchi,
  ``Discovering polarization niches via dense subgraphs with attractors and
  repulsers,'' \emph{Proceedings of the VLDB Endowment}, vol.~15, no.~13, pp.
  3883--3896, 2022.

\bibitem{khuller2009dense}
S.~Khuller and B.~Saha, ``On finding dense subgraphs,'' in \emph{Proceedings of
  the 36th International Colloquium on Automata, Languages and Programming:
  Part I}, ser. ICALP ’09.\hskip 1em plus 0.5em minus 0.4em\relax
  Springer-Verlag, 2009.

\bibitem{tsourakakis2015k}
C.~Tsourakakis, ``The k-clique densest subgraph problem,'' in \emph{WWW}, 2015.

\bibitem{samusevich2016local}
R.~Samusevich, M.~Danisch, and M.~Sozio, ``Local triangle-densest subgraphs,''
  in \emph{2016 IEEE/ACM International Conference on Advances in Social
  Networks Analysis and Mining (ASONAM)}.\hskip 1em plus 0.5em minus
  0.4em\relax IEEE, 2016, pp. 33--40.

\bibitem{sun2020kclist++}
B.~Sun, M.~Danisch, T.~Chan, and M.~Sozio, ``Kclist++: A simple algorithm for
  finding k-clique densest subgraphs in large graphs,'' \emph{Proceedings of
  the VLDB Endowment (PVLDB)}, 2020.

\bibitem{Tsourakakis2013Denser}
C.~Tsourakakis, F.~Bonchi, A.~Gionis, F.~Gullo, and M.~Tsiarli, ``Denser than
  the densest subgraph: extracting optimal quasi-cliques with quality
  guarantees,'' in \emph{SIGKDD}, 2013, pp. 104--112.

\bibitem{galimberti2017core}
E.~Galimberti, F.~Bonchi, and F.~Gullo, ``Core decomposition and densest
  subgraph in multilayer networks,'' in \emph{CIKM}, 2017.

\bibitem{Mitzenmacher2015SLN}
M.~Mitzenmacher, J.~Pachocki, R.~Peng, C.~Tsourakakis, and S.~C. Xu, ``Scalable
  large near-clique detection in large-scale networks via sampling,'' in
  \emph{KDD}, 2015.

\bibitem{ma2020efficient}
C.~Ma, Y.~Fang, R.~Cheng, L.~V. Lakshmanan, W.~Zhang, and X.~Lin, ``Efficient
  algorithms for densest subgraph discovery on large directed graphs,'' in
  \emph{SIGMOD}, 2020.

\bibitem{andersen2009finding}
R.~Andersen and K.~Chellapilla, ``Finding dense subgraphs with size bounds,''
  in \emph{WAW'09}.

\bibitem{Chen2010Dense}
J.~Chen and Y.~Saad, ``Dense subgraph extraction with application to community
  detection,'' \emph{IEEE TKDE}, 2010.

\bibitem{costa2015milp}
A.~Costa, ``Milp formulations for the modularity density maximization
  problem,'' \emph{European Journal of Operational Research}, 2015.

\bibitem{prakash2010eigenspokes}
B.~A. Prakash, A.~Sridharan, M.~Seshadri, S.~Machiraju, and C.~Faloutsos,
  ``Eigenspokes: Surprising patterns and scalable community chipping in large
  graphs,'' in \emph{PAKDD}, 2010.

\bibitem{beutel2013copycatch}
A.~Beutel, W.~Xu, V.~Guruswami, C.~Palow, and C.~Faloutsos, ``Copycatch:
  stopping group attacks by spotting lockstep behavior in social networks,'' in
  \emph{WWW}, 2013, pp. 119--130.

\bibitem{newman2006modularity}
M.~E. Newman, ``Modularity and community structure in networks,''
  \emph{Proceedings of the national academy of sciences}, 2006.

\bibitem{qin2015locally}
L.~Qin, R.-H. Li, L.~Chang, and C.~Zhang, ``Locally densest subgraph
  discovery,'' in \emph{SIGKDD}, 2015, pp. 965--974.

\bibitem{leskovec2010kronecker}
J.~Leskovec, D.~Chakrabarti, J.~Kleinberg, C.~Faloutsos, and Z.~Ghahramani,
  ``Kronecker graphs: an approach to modeling networks.'' \emph{Journal of
  Machine Learning Research}, vol.~11, no.~2, 2010.

\bibitem{seidman1983network}
S.~B. Seidman, ``Network structure and minimum degree,'' \emph{Social
  networks}, vol.~5, no.~3, pp. 269--287, 1983.

\bibitem{zhu2023fast}
Y.~{Zhu}, S.~{Liu}, W.~{Feng}, X.~{Cheng},``Fast Searching The Densest Subgraph And Decomposition With Local Optimality,'' \emph{arXiv: Databases}, 2023.

\end{document}
