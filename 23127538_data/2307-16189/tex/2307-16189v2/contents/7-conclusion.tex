% The field of deep learning, while highly advanced, continues to face fundamental challenges such as numerical instability. In this study, we introduced a novel approach to tackle the numerical instability frequently encountered during the backpropagation phase of model training. This involved a sophisticated modification to the Adam optimizer's gradient update equation, leading to a significant reduction in numerical instability, particularly in a 16-bit neural network training scenario. Our empirical evaluations, conducted on the MNIST dataset using a simple Deep Neural Network (DNN) consisting of a single linear layer, yielded promising results. As shown in Table 1, the modified optimization algorithm consistently demonstrated robust performance, and in most cases, exceeded the traditional Adam optimizer in terms of test accuracy in a 16-bit precision setting. Furthermore, our method ensured numerical stability across a range of epsilon values. Notably, the training time with our modified optimizer in 16-bit precision setting was about 45\% lower than that in a 32-bit precision setting, highlighting the efficiency of our approach. Despite the complexity of the problem, we found that our modified optimizer maintained numerical stability across a range of values. Importantly, we observed a noticeable reduction in the occurrence of numerical underflows and overflows, thus ensuring the robustness of the training process even with 16-bit precision. While the experimental results are encouraging, it is important to acknowledge that this study focused primarily on a simple DNN with a single Linear Layer. This signifies the necessity of further investigations to validate the effectiveness of our proposed method on more complex architectures, such as convolutional neural networks (CNNs) and Vision Transformers (ViTs). By expanding the scope of our approach across various network architectures, we aim to establish a comprehensive and robust method for mitigating numerical instability in a wide range of deep learning models. In conclusion, this work makes a significant contribution to the ongoing endeavors in enhancing deep learning methodologies, specifically focusing on improving the stability and efficiency of model training. We anticipate that our research findings will stimulate further advancements in the field, ultimately leading to the development of more reliable and robust deep learning systems. Our hope is that this innovative approach to preventing numerical instability in 16-bit neural networks will become a cornerstone in the future of efficient and stable model training.

While deep learning stands at the pinnacle of advancement, it grapples with core issues like numerical instability. Our research introduces an innovative strategy to address this instability, most often seen during the backpropagation stage of model training. Central to our approach is a nuanced alteration to the gradient update equation of the Adam optimizer. This adaptation notably diminishes the extent of numerical instability, especially in 16-bit neural network training. Our hands-on assessments, executed on the MNIST dataset using a streamlined Deep Neural Network (DNN) with a singular linear layer, presented promising outcomes. As illustrated in Table 1, the revised optimization technique showed unwavering efficacy. In many instances, it outperformed the conventional Adam optimizer concerning test accuracy within a 16-bit precision framework. Significantly, our strategy upheld numerical stability for an array of epsilon values. It's worth emphasizing that our enhanced optimizer, when set to 16-bit precision, slashed the training duration by roughly 45\% compared to its 32-bit counterpart, underscoring our method's efficiency. Even in the face of intricate challenges, our refined optimizer consistently ensured numerical stability over diverse value ranges. Of particular importance, we recorded a marked decline in numerical underflows and overflows, solidifying the training robustness, especially in the 16-bit precision context. Our trials with the Vision Transformer model for training cifar-10 shows that our approach works in complex neural networks architectures. The 16-bit models, impressively, computed data nearly 30\% faster than the 32-bit versions. More pivotal was the observation that, with the integration of the Numerical Guarantee Method, the VIT models advanced smoothly through the learning trajectory, devoid of the instability often linked with traditional 16-bit models. Although the empirical outcomes spark optimism, it's vital to underscore our study's focus on image classification via Linear DNN and the Vision Transformer models (VIT). This underscores the imperative of broader probes to assess our method's potency in sophisticated architectures and other deep learning domains, including transformer models tailored for natural language processing. By broadening our methodological horizon, we aspire to devise a universal and resilient technique to counter numerical instability across diverse deep learning models. Summarily, our research underscores a paramount stride in refining deep learning protocols, with an emphasis on enhancing model training's stability and efficiency. We're optimistic that our insights will catalyze further breakthroughs, steering the discipline towards the creation of dependable and sturdy deep learning frameworks. We envision our pioneering tactic against numerical instability in 16-bit neural networks to set a benchmark in the future narrative of proficient and unswerving model training