Video analytics are often performed as cloud services in edge settings, mainly to offload computation, and also in situations where the results are not directly consumed at the video sensors. Sending high-quality video data from the edge devices can be expensive both in terms of bandwidth and power use. In order to build a streaming video analytics pipeline that makes efficient use of these resources, it is therefore imperative to reduce the size of the video stream. Traditional video compression algorithms are unaware of the semantics of the video, and can be both inefficient and harmful for the analytics performance. In this paper, we introduce \emph{LtC}, a collaborative framework between the video source and the analytics server, that efficiently learns to reduce the video streams within an analytics pipeline. Specifically, LtC uses the full-fledged analytics algorithm at the server as a teacher to train a lightweight student neural network, which is then deployed at the video source. The student network is trained to comprehend the semantic significance of various regions within the videos, which is used to differentially preserve the crucial regions in high quality while the remaining regions undergo aggressive compression. Furthermore, LtC also incorporates a novel temporal filtering algorithm based on feature-differencing to omit transmitting frames that do not contribute new information. Overall, LtC is able to use 28-35\% less bandwidth and has up to 45\% shorter response delay compared to recently published state of the art streaming frameworks while achieving similar analytics performance.