%%
%% This is file `sample-sigconf.tex',
%% generated with the docstrip utility.
%%
%% The original source files were:
%%
%% samples.dtx  (with options: `sigconf')
%% 
%% IMPORTANT NOTICE:
%% 
%% For the copyright see the source file.
%% 
%% Any modified versions of this file must be renamed
%% with new filenames distinct from sample-sigconf.tex.
%% 
%% For distribution of the original source see the terms
%% for copying and modification in the file samples.dtx.
%% 
%% This generated file may be distributed as long as the
%% original source files, as listed above, are part of the
%% same distribution. (The sources need not necessarily be
%% in the same archive or directory.)
%%
%% Commands for TeXCount
%TC:macro \cite [option:text,text]
%TC:macro \citep [option:text,text]
%TC:macro \citet [option:text,text]
%TC:envir table 0 1
%TC:envir table* 0 1
%TC:envir tabular [ignore] word
%TC:envir displaymath 0 word
%TC:envir math 0 word
%TC:envir comment 0 0
%%
%%
%% The first command in your LaTeX source must be the \documentclass command.
\documentclass[sigconf, authorversion, nonacm]{acmart}
%% NOTE that a single column version is required for 
%% submission and peer review. This can be done by changing
%% the \doucmentclass[...]{acmart} in this template to 
%% \documentclass[manuscript,screen]{acmart}
%% 
%% To ensure 100% compatibility, please check the white list of
%% approved LaTeX packages to be used with the Master Article Template at
%% https://www.acm.org/publications/taps/whitelist-of-latex-packages 
%% before creating your document. The white list page provides 
%% information on how to submit additional LaTeX packages for 
%% review and adoption.
%% Fonts used in the template cannot be substituted; margin 
%% adjustments are not allowed.

\usepackage{algorithmic}
\usepackage{amsfonts}
\usepackage{amsmath}
\usepackage{enumitem}
\usepackage{graphicx}
\usepackage{hyperref}
\usepackage{makecell}
\usepackage{multirow}
\usepackage{subcaption}
\usepackage{textcomp}
\usepackage{url}
\usepackage{xcolor}
\usepackage{xurl}

\DeclareMathOperator*{\argmax}{argmax}
\DeclareMathOperator*{\argmin}{argmin}

%%
%% \BibTeX command to typeset BibTeX logo in the docs
\AtBeginDocument{%
  \providecommand\BibTeX{{%
    \normalfont B\kern-0.5em{\scshape i\kern-0.25em b}\kern-0.8em\TeX}}}

%% Rights management information.  This information is sent to you
%% when you complete the rights form.  These commands have SAMPLE
%% values in them; it is your responsibility as an author to replace
%% the commands and values with those provided to you when you
%% complete the rights form.
\setcopyright{acmcopyright}
\copyrightyear{2023}
\acmYear{2023}
\acmDOI{XXXXXXX.XXXXXXX}

%% These commands are for a PROCEEDINGS abstract or paper.
\acmConference[ACM MM '23]{ACM Multimedia}{October 28, 2023 – November 3, 2023}{Ottawa, Canada}
%
%  Uncomment \acmBooktitle if th title of the proceedings is different
%  from ``Proceedings of ...''!
%
%\acmBooktitle{Woodstock '18: ACM Symposium on Neural Gaze Detection,
%  June 03--05, 2018, Woodstock, NY} 
\acmPrice{15.00}
\acmISBN{978-1-4503-XXXX-X/18/06}


%%
%% Submission ID.
%% Use this when submitting an article to a sponsored event. You'll
%% receive a unique submission ID from the organizers
%% of the event, and this ID should be used as the parameter to this command.
%%\acmSubmissionID{123-A56-BU3}

%%
%% For managing citations, it is recommended to use bibliography
%% files in BibTeX format.
%%
%% You can then either use BibTeX with the ACM-Reference-Format style,
%% or BibLaTeX with the acmnumeric or acmauthoryear sytles, that include
%% support for advanced citation of software artefact from the
%% biblatex-software package, also separately available on CTAN.
%%
%% Look at the sample-*-biblatex.tex files for templates showcasing
%% the biblatex styles.
%%

%%
%% The majority of ACM publications use numbered citations and
%% references.  The command \citestyle{authoryear} switches to the
%% "author year" style.
%%
%% If you are preparing content for an event
%% sponsored by ACM SIGGRAPH, you must use the "author year" style of
%% citations and references.
%% Uncommenting
%% the next command will enable that style.
%%\citestyle{acmauthoryear}

%%
%% end of the preamble, start of the body of the document source.
\begin{document}

%%
%% The "title" command has an optional parameter,
%% allowing the author to define a "short title" to be used in page headers.
\title[Learn to Compress (LtC)]{Learn to Compress (LtC): Efficient Learning-based Streaming Video Analytics}

%%
%% The "author" command and its associated commands are used to define
%% the authors and their affiliations.
%% Of note is the shared affiliation of the first two authors, and the
%% "authornote" and "authornotemark" commands
%% used to denote shared contribution to the research.
\author{Quazi Mishkatul Alam}
\affiliation{%
  \institution{University of California}
  \city{Riverside}
  \state{CA}
  \country{USA}}
\email{qalam001@ucr.edu}

\author{Israat Haque}
\affiliation{%
  \institution{Dalhousie University}
  \city{Halifax}
  \country{Canada}}
\email{israat@dal.ca}

\author{Nael Abu-Ghazaleh}
\affiliation{%
  \institution{University of California}
  \city{Riverside}
  \state{CA}
  \country{USA}}
\email{nael@cs.ucr.edu}

%%
%% By default, the full list of authors will be used in the page
%% headers. Often, this list is too long, and will overlap
%% other information printed in the page headers. This command allows
%% the author to define a more concise list
%% of authors' names for this purpose.
\renewcommand{\shortauthors}{Quazi et al.}

%%
%% The abstract is a short summary of the work to be presented in the
%% article.
\begin{abstract}
\begin{abstract}

The Fast Reciprocal Square Root Algorithm is a well-established approximation technique consisting of two stages: first, a coarse approximation is obtained by manipulating the bit pattern of the floating point argument using integer instructions, and second, the coarse result is refined through one or more steps, traditionally using Newtonian iteration but alternatively using improved expressions with carefully chosen numerical constants found by other authors. The algorithm was widely used before microprocessors carried built-in hardware support for computing reciprocal square roots. At the time of writing, however, there is in general no hardware acceleration for computing other fixed fractional powers. This paper generalises the algorithm to cater to all rational powers, and to support any polynomial degree(s) in the refinement step(s), and under the assumption of unlimited floating point precision provides a procedure which automatically constructs provably optimal constants in all of these cases. It is also shown that, under certain assumptions, the use of monic refinement polynomials yields results which are much better placed with respect to the cost/accuracy tradeoff than those obtained using general polynomials. Further extensions are also analysed, and several new best approximations are given.

\end{abstract}

\end{abstract}

%%
%% The code below is generated by the tool at http://dl.acm.org/ccs.cfm.
%% Please copy and paste the code instead of the example below.
%%
%\begin{CCSXML}
%<ccs2012>
%   <concept>
%       <concept_id>10002951.10003227.10003251.10003255</concept_id>
%       <concept_desc>Information systems~Multimedia streaming</concept_desc>
%       <concept_significance>500</concept_significance>
%       </concept>
%   <concept>
%       <concept_id>10002951.10003227.10003241.10003244</concept_id>
%       <concept_desc>Information systems~Data analytics</concept_desc>
%       <concept_significance>500</concept_significance>
%       </concept>
% </ccs2012>
%\end{CCSXML}

%\ccsdesc[500]{Information systems~Multimedia streaming}
%\ccsdesc[500]{Information systems~Data analytics}

%%
%% Keywords. The author(s) should pick words that accurately describe
%% the work being presented. Separate the keywords with commas.
%\keywords{video streaming, analytics, learning-based compression}

% \received{20 February 2007}
% \received[revised]{12 March 2009}
% \received[accepted]{5 June 2009}

%%
%% This command processes the author and affiliation and title
%% information and builds the first part of the formatted document.
\maketitle

\section{Introduction}
\section{Introduction}
Current quantum hardware is unable to carry out universal quantum computations due to the buildup of errors that occur during the computation. 
The magnitude of the individual error is currently above the value that the Threshold Theorem requires in order to kick-start quantum error correction and fault-tolerant quantum computation~\cite[Section 10.6]{nielsen_chuang_2010}. 
Although the experimentally achieved fidelity rates are promising and the error bounds are inching closer to the required threshold, we will have to work for the foreseeable future with quantum hardware with errors that build-up during the computation.  This implies that we can only do a limited number of steps before the output of the computation has become completely uncorrelated with the intended one.

For fault-tolerant quantum computing, we repeat four steps: 
1) We apply a number of single and two-qubit quantum gates, in parallel whenever possible; 
2) We perform a syndrome measurement on a subset of the qubits; 
3) We perform fast classical computations to determine which errors have occurred and how to correct them; 
and, 4) We apply correction terms based on the classical computations.
We then repeat these four steps with a next sequence of gates. 
These four steps are essential to fault-tolerant quantum computing. 


The starting point of this work is to use the four steps outlined above, not to carry out error correction and fault-tolerant computation, but to enhance short, constant-depth, {\em uncorrected} quantum circuits that perform single qubit gates and {\em nearest-neighbor} two qubit gates. 
Since in the long run we will have to implement error-correction and fault-tolerant computation anyhow, and this is done by such a four-step process, why not make other use of this architecture? Moreover, on some of the quantum hardware platforms, these operations are already in place.
Embracing this idea we naturally arrive at the question: what is the computational power of \textit{low-depth} quantum-classical circuits organized as in the four steps outlined above? 
We thus investigate circuits that execute a small, ideally constant, number of stages, where at each stage we may apply, in parallel, single qubit gates and {\em nearest-neighbor} two qubit gates, followed by measurements, followed by low-depth classical computations of which the outcome can control quantum gates in later stages. 
It is not clear, at first, whether such circuits, especially with constant depth, can do anything remotely useful. 
But we will see that this is indeed the case: many quantum computations can be done by such circuits in constant depth. 
By parallelizing quantum computations in this way, we improve the overall computational capabilities of these circuits, as we do not incur errors on qubits that are idle, simply because qubits are not idle for a very long time. 
Furthermore, reducing the depth of quantum circuits, at the cost of increasing width, allows the circuit to be run faster even if errors occur.

The first usage of such a four-step layout, not to do error correction, but to perform computations, can be found in the paradigm of measurement-based quantum computing~\cite{gottesman1999demonstrating,raussendorf2001one,jozsa2006introduction,clark2007generalised}: 
A universal form of quantum computing where a quantum state is prepared and operations are performed by measuring qubits in different bases, depending on previous measurements and intermediate measurements.

\citeauthor{PhamSvore2013} were the first to formalize the four-step protocol for performing computations~\cite{PhamSvore2013}. They included specific hardware topologies by considering two-dimensional graphs for imposing constraints on qubit interactions. In their model, they develop circuits for particularly useful multi-qubit gates, including specifying costs in the width, number of qubits, depth, number of concurrent time steps, size, and total number of non-Identity operations.
As a result, they find an algorithm that factors integers in polylogarithmic depth.
\citeauthor{Browne:2011} showed that the main tool in the work by \citeauthor{PhamSvore2013}, the fan-out gate, can also be replaced by additional log-depth classical computations in the measurement-based quantum computing setting~\cite{Browne:2011}.

More recently, \citeauthor{Cirac:2021} introduced a scheme to implement unitary operations involving quantum circuits combined with Local Operations and Classical Communication ($\mathsf{LOCC}$) channels: $\mathsf{LOCC}$-assisted quantum circuits~\cite{Cirac:2021}. Similarly to the four-step scheme we just described, they allow for a short depth circuit to be run on the qubits, followed by one round of $\mathsf{LOCC}$, in which ancilla qubits are measured and local unitaries are applied based on the measurement outcomes. They show that in this model any 1D transitionally invariant matrix-product state (MPS) with fixed bond dimension is in the same phase of matter as the trivial state. Similar ideas can be found in~\cite{TVV_NonAbelianTopologicalOrder_2022, tantivasadakarn2021long}.

In this work, we introduce a new model, called \textit{Local Alternating Quantum-Classical Computations} ($\LAQCC$). In this model we alternate between running quantum circuits (constrained by locality), ending in the measurement of a subset of qubits, and fast classical computations based on the measurement results. The outcome of the classical computations are then used to control future quantum circuits. We allow for flexibility in this model, by giving different constraints to the power of both the quantum circuits and the classical circuits as well as the number of alternations between them. 
Most attention will be given to $\LAQCC$ containing quantum circuits of constant depth, classical circuits of logarithmic depth and at most a constant number of alternations between them. 
Any circuit constructed in this model is considered to be of constant depth. 
We restrict ourselves to logarithmic depth classical computations, as this is the first natural and non-trivial extension beyond constant-depth classical computations. 
Constant-depth classical computations do however also have an equivalent constant-depth quantum implementation.

The definition of $\LAQCC$ sharpens the original definition of \citeauthor{PhamSvore2013} by adding constraints to the intermediate classical computations. This allows us to bound the power of $\LAQCC$ from above. 

The main result of \citeauthor{Cirac:2021}, that 1D translational invariant MPS with fixed bond dimension can be prepared by $\mathsf{LOCC}$-assisted circuits, relies on local symmetries of the MPS. These symmetries allow them to prepare local states (on a constant number of qubits) and glue them together by doing one round of the appropriate entangling measurement and corrections, after which they run a round of local unitaries to get the desired result. This general scheme for preparing states that exhibit an MPS description with the appropriate local symmetries requires only geometrically local unitaries and one round of measurement and corrections an therefore is accessible in $\LAQCC$. Studying different local symmetries, known as Symmetry Protected Topological (SPT) phases of matter, to find measurement-based constant depth circuits for states is a broad ongoing field of research~\cite{TVV_NonAbelianTopologicalOrder_2022, tantivasadakarn2021long, smith2023deterministic}. 
All these schemes have a $\LAQCC$ implementation.

%$\LAQCC$-circuits also exist for general schemes of preparing local states, based on the local tensors, and gluing them together using one round of entangled measurement and corrections, based on the local symmetry. 
%The main result of \citeauthor{Cirac:2021}, that 1D translational invariant MPS with fixed bond dimension can be prepared by $\mathsf{LOCC}$-assisted circuits, relies heavily on local symmetries of the MPS and as a result also has an equivalent $\LAQCC$ implementation. 
%The corrections applied after the measurement round are local unitaries depending on the local symmetries of the MPS. 

 

%This general scheme of preparing local states, based on the local tensors, and gluing it together by doing one round of entangled measurement and corrections, based on the local symmetry, is accessible in $\LAQCC$.
Note however that \citeauthor{Cirac:2021} also suggest a circuit for the $W$-state.
This circuit uses sequentially and dependent measurement-based corrections of the ancilla qubits. 
These dependent measurements translate to sequential alternations between the quantum and classical circuits and therefore increase the total depth to linear depth, exceeding the constant-depth constraints imposed by $\LAQCC$-circuits. 

We study the power of the $\LAQCC$ model with respect to state preparation, showing that even with only constant quantum-depth and logarithmic classical depth it remains possible to prepare states with long-range entanglement.
Another surprising result is that it is unlikely that $\LAQCC$ circuits are classically simulatable. We show that any instantaneous quantum polynomial-time (IQP) circuit~\cite{Bremner2010,Shepherd2009} has an $\LAQCC$ implementation.
Classical simulation of IQP circuits implies the collapse of the polynomial hierarchy to the third level, which is not believed to be true~\cite{Bremner2017}. Therefore, we expect that $\LAQCC$ circuits are unlikely to be classically simulatable. We bound the power of $\LAQCC$ by showing that it is contained in $\QNC^1$, the class of polynomial-size, log-depth circuits.

Next, we also study the power that intermediate classical calculations can add to quantum computations, by considering a new model that alternates between polynomially many polynomial-depth quantum circuits and unbounded classical computations
We study this model by doing a complexity theoretical analysis, where we draw inspiration from the notions of complexity given by \citeauthor{RosenthalYuen:2022}, \citeauthor{MetgerYuen:2023}, and \citeauthor{Aaronson:2004}.
All three complexity notions are based on the notion of state preparation, instead of more traditional definition of complexity such as the decidability of a computational problem. 
The first two consider classes based on sequences of quantum states preparable by a polynomial-sized quantum circuit, where the circuits are uniformly generated by a computational class, for instance, the class $\mathsf{PSPACE}$, which results in the complexity class $\mathsf{StatePSPACE}$~\cite{RosenthalYuen:2022,MetgerYuen:2023}.
The third notion considers a relative complexity, where the complexity is measured between two given states, and is measured by the number of gates, from a given gate-set, required to transform one state in another state~\cite{Aaronson:2004}. 
For our definition of state preparation complexity, we drop the uniformity constraint from~\cite{RosenthalYuen:2022,MetgerYuen:2023} and define a class as $\mathsf{StateX}$, which refers to states preparable by circuits of type $\mathsf{X}$. 
As an example, if $\mathsf{X} = \QNC^0$, this results in the class $\mathsf{StateQNC^0}$, which is the set of states preparable from the $\ket{0}^n$ state by poly-size constant-depth circuits. 
This notion is similar to the relative complexity from~\cite{Aaronson:2004}, where one state is the  $\ket{0}^n$ state and instead of counting the number of gates we consider the set of states preparable by a fixed number of gates. Using this notion of complexity we show that any state preparable by an $\LAQCC^*$ circuit is also preparable by a $\mathsf{PostQPoly}$ circuit, the class of circuits of polynomial depth with an additional post-selection gate. 

All Clifford circuits have a constant-depth $\LAQCC$ implementation, implying that any stabilizer state can be implemented by a constant-depth $\LAQCC$ circuit, see Section~\ref{sec:clifford_circuits} for a proof of this statement. 
Efficient circuits for stabilizer states have been known already through measurement-based quantum computing. Therefore this paper focuses on the preparation of non-stabilizer states, and as a surprising result we find novel constant-depth protocols for four very natural classes of non-stabilizer states.
Despite the extensive research into these four classes of non-stabilizer states and the many applications of them, no efficient constant- or low-depth state preparation protocols are known yet. We specifically consider these four classes as they are all often used as initial states in other algorithms.

The first state is a uniform superposition over an arbitrary number of states. 
This state finds applications in many quantum algorithms, as they often start with a uniform superposition over multiple states. 
This superposition is often achieved by applying Hadamard gates to every qubit due to its simplicity to prepare. 
Yet, the analysis of many algorithms, such as Shor's algorithm~\cite{Shor:1997}, would benefit from a different initial superposition. 
The circuit to prepare the uniform superposition over an arbitrary number of states uses an exact version of Grover search as a subroutine, that turns a probabilistic circuit, with a known constant probability of success, into a deterministic circuit. 
We use the circuit for preparing a uniform superposition over an arbitrary number of states as a subroutine in the next two quantum state preparation protocols. 

The second state is the $W$-state, the uniform superposition over all computational basis states of Hamming-weight~$1$, a natural long-ranged entangled state that displays a fundamentally nonequivalent type of entanglement from the Greenberger–Horne–Zeilinger state~\cite{WState:2000}, for which $\LAQCC$-type constant-depth circuits were previously known~\cite{PhamSvore2013, Cirac:2021}. 
The $W$-state is often used as benchmark for new quantum hardware~\cite{Haffner2005,Neeley2010,GarciaPerez:2021}. 
A novel way to prepare the $W$-state therefore gives a new way to benchmark different quantum devices with each other. 
A circuit for preparing the $W$-state was given in~\cite{Cirac:2021}, but this implementation requires sequentially alternating measurements followed by local unitaries, which in the $\LAQCC$ model is not considered to be of constant depth. 
We improve this protocol by giving an $\LAQCC$ implementation of the $W$-state, based on a compress-uncompress method that links the one-hot and binary encoding of integers.

The third state considered is the Dicke state, a generalization of the $W$-state, a superposition over all computational basis states with Hamming-weight $k$~\cite{Dicke:1954}. 
Dicke states have relevance in various practical settings.
For instance, for quantum game theory~\cite{zdemir2007}, quantum storage~\cite{Bacon_Compress:2006,Plesch:2010}, quantum error correction~\cite{ouyang2014permutation}, quantum metrology~\cite{toth2012multipartite}, and quantum networking~\cite{prevedel2009experimental}. 
Dicke states have been used as a starting state for variational optimization algorithms, most notably Quantum Alternating Operator Ansatz (QAOA)~\cite{Hadfield2019}, to find solutions to problems such as Maximum k-vertex Cover~\cite{Brandhofer2022,cook2020quantum}.
The ground states of physical Hamiltonians describing one-dimensional chains tend to show a resemblance to Dicke states such as states resulting from the Bethe ansatz, making them an ideal starting state when investigating the ground state behavior of these Hamiltonians~\cite{TDL_BetheAnsatzDerivation:2010,B_ExcitedStateQuantumPhaseTransitions:2013,DickeTransitions:2021}. 
For instance, the algorithm by \citeauthor{van2021preparing}, who give an algorithm to prepare the Bethe ansatz eigenstates of the spin-1/2 XXZ spin chain, starts by first preparing a Dicke state~\cite{van2021preparing}. 
A Dicke-state preparation protocol based on the compress-uncompress methodology used in the $W$-state furthermore finds applications in entanglement distillation, where the entanglement of a large state is concentrated on only a few qubits. 
Efficient deterministic circuits for preparing Dicke states have been proposed by \citeauthor{bartschi2019deterministic}~\cite{bartschi2019deterministic, bartschi2022deterministic_short_depth}. 
They provide a quantum circuit of depth $\mathO(k \log(\frac{n}{k}))$, allowing arbitrary connectivity, to prepare a Dicke state, which they conjecture to be optimal when $k$ is constant. 
In this work, we provide a constant-depth $\LAQCC$ circuit below their conjectured bound already for constant $k$. 
However, this does not directly disprove their conjecture, as we allow for intermediate measurements and classical computations. 
More significantly, we even construct constant-depth $\LAQCC$ circuits for $k = \mathO(\sqrt{n})$ greatly improving their bound.
This construction extends the compress-uncompress method for the $W$-state combined with additional subroutines. 

We continue with a log-depth state preparation protocol for the Dicke-state for arbitrary $k$. 
This protocol implements an efficient transformation between the factoradic number representation and the combinatorial number representation of a positive integer. 
The combinatorial number representation relates directly to the Dicke state. 
The provided efficient transformation between number representation systems might be of independent interest. 

We conclude by modifying our protocol for preparing a Dicke-state to a protocol that prepares quantum many-body scar states in constant-depth. 
These states have low entanglement and longer coherence times than states with similar energy density.
These characteristics make many-body scar states interesting to analyze and relevant within physics.
Many-body scar states appear for instance in the AKLT model~\cite{AKLT:1987,MRBAR:2018,MRB:2018} and different spin models~\cite{SI:2019,MOBFR:2020}.
Known methods for preparing these states have polynomial-depth~\cite{Gustafson:2023}, whereas our circuit has constant depth. 

% We conclude by studying the power that intermediate classical calculations can add to quantum computations. 
% In this study, we define a new model that relaxes constant-depth quantum circuits to polynomial depth quantum circuits, log-depth classical calculations to unbounded classical computations and a constant number of alternations to a polynomial number of alternations. 
% We call this model $\LAQCC^*$. 
% We study this model by doing a complexity theoretical analysis, where we draw inspiration from the notions of complexity given by \citeauthor{RosenthalYuen:2022}, \citeauthor{MetgerYuen:2023}, and \citeauthor{Aaronson:2004}.
% All three complexity notions are based on the notion of state preparation, instead of more traditional definition of complexity such as the decidability of a computational problem. 
% The first two consider classes based on sequences of quantum states preparable by a polynomial-sized quantum circuit, where the circuits are uniformly generated by a computational class, for instance, the class $\mathsf{PSPACE}$, which results in the complexity class $\mathsf{StatePSPACE}$~\cite{RosenthalYuen:2022,MetgerYuen:2023}.
% The third notion considers a relative complexity, where the complexity is measured between two given states, and is measured by the number of gates, from a given gate-set, required to transform one state in another state~\cite{Aaronson:2004}. 
% For our definition of state preparation complexity, we drop the uniformity constraint from~\cite{RosenthalYuen:2022,MetgerYuen:2023} and define a class as $\mathsf{StateX}$, which refers to states preparable by circuits of type $\mathsf{X}$. 
% As an example, if $\mathsf{X} = \QNC^0$, this results in the class $\mathsf{StateQNC^0}$, which is the set of states preparable from the $\ket{0}^n$ state by poly-size constant-depth circuits. 
% This notion is similar to the relative complexity from~\cite{Aaronson:2004}, where one state is the  $\ket{0}^n$ state and instead of counting the number of gates we consider the set of states preparable by a fixed number of gates. Using this notion of complexity we show that any state preparable by an $\LAQCC^*$ circuit is also preparable by a $\mathsf{PostQPoly}$ circuit, the class of circuits of polynomial depth with an additional post-selection gate. 

\paragraph{Summary of results}
\begin{itemize}
    \item We give a new definition of a computational model that captures the power of the four step process: applying a constant number of layers of one- and two-qubit gates; performing a syndrome measurement; perform a fast classical computation determining corrections; apply corrections. We call this model \emph{Local Alternating Quantum Classical Computations}, or $\LAQCC$ for short. In this model we bound the allowed quantum operations, intermediate classical calculations, and number of rounds separately. In Section~\ref{sec:LAQCC_model} we define this model and give a list of operations based on results from literature contained in this computational model. In some of these operations we explicitly use that we allow for multiple, but at most constant, rounds  of corrections.
    \item  We show show that there exist $\LAQCC$ circuits that can not be weakly simulated in Section~\ref{sec:IQP_in_LAQCC}. We further show that for every $\LAQCC$ circuit there exists a $\QNC^1$ circuit simulating it perfectly, in Section~\ref{sec:LAQCC_in_QNC1}.
    \item We introduce a new type computational complexity for preparing states and show that the extension of $\LAQCC$ where we allow a polynomial number of rounds and unbounded classical computation, is contained in $\mathsf{PostQPoly}$, the class of polynomial circuits with post-selection, in Section~\ref{sec:Complexity results}.
    \item We show a protocol to prepare the uniform superposition state of size $q$ in $\LAQCC$ using $\mathO(\ceil{\log_2(q)}^2)$ qubits in Section~\ref{sec:superposition_modulo_q}. 
    \item We show a protocol to prepare the $W_n$ state in $\LAQCC$ using $\mathO(n\log(n))$ qubits in Section~\ref{sec:W_state_in_LAQCC}.
    \item We show two ways of preparing the Dicke-$(n,k)$ state. The first method is in $\LAQCC$, works up to $k = \mathO(\sqrt{n})$, uses $\mathO(n^2\log(n))$ qubits, and is found in Section~\ref{sec:dicke:small_k}. The second method is in $\LAQCC\text{-}\mathsf{LOG}$ (an extension of $\LAQCC$ allowing for logarithmic number of alterations instead of constant), works for any $k$, uses $\mathO(\text{poly}(n))$ qubits, and is found in Section~\ref{sec:Dicke_in_LAQCC_LOG}. 
    \item We extend on our $\LAQCC$ method of generating Dicke-$(n,k)$ states for $k = \mathO(\sqrt{n})$ and show a protocol to generate many-body scar states for a particular Hamiltonian in $\LAQCC$ (Section~\ref{sec:many_body_scar}). 
\end{itemize}
Summarized in a table, we provide the following state generation protocols:
\begin{table}[htb]
\centering
\begin{tabular}{l|l|l|l}
\textbf{State description} & \textbf{Width} & \textbf{Depth} & \textbf{Implementation}\\
\hline 
Uniform superposition mod $q$: $\frac{1}{\sqrt{q}} \sum_{i = 0}^{q-1}\ket{i}$ & $\mathO(\ceil{\log^2 q})$ & $\mathO(1)$ & Section~\ref{sec:superposition_modulo_q}\\

$W$-state: $\frac{1}{\sqrt{n}}\sum_{i = 0}^{n-1}\ket{e_i}$ & $\mathO(n \log n)$ & $\mathO(1)$ & Section~\ref{sec:W_state_in_LAQCC}\\

Dicke-$(n,k)$, $k = \mathO(\sqrt{n})$: $\binom{n}{k}^{-1/2}\sum_{x \in \{0,1\}^n: |x| = k} \ket{x}$ &  $\mathO(n^2\log n)$ & $\mathO(1)$ 
&Section~\ref{sec:dicke:small_k}\\

Dicke-$(n,k)$: $\binom{n}{k}^{-1/2}\sum_{x \in \{0,1\}^n: |x| = k} \ket{x}$ & $\mathO(\text{poly}(n))$ & $\mathO(\log n)$ &Section~\ref{sec:Dicke_in_LAQCC_LOG}\\

QMBS: $\ket{S_k} = \frac{1}{k! \sqrt{\mathcal N(n,k)}}(Q^\dagger)^k \ket{\Omega}$ &  $\mathO(n^2\log n)$ & $\mathO(1)$  &  Section~\ref{sec:many_body_scar}
\end{tabular}
\caption{Summary of state preparation protocols given in this paper.}
\label{tab:sate_prep}
\end{table}
In the entry for the quantum many-body scar state $Q$ denotes the raising operator and $\mathcal N(n,k)=\binom{n-k-1}{k}$. 
Section~\ref{sec:many_body_scar} will provide more details on the variables and the implementation. 

\paragraph{Organization of the paper}
\noindent We first introduce relevant preliminaries in Section~\ref{sec:preliminaries}. 
In Section~\ref{sec:LAQCC_model} we formally define the class of Local Alternating Quantum-Classical Computations ($\LAQCC$). We also show that any Clifford circuit can be implemented in constant depth $\LAQCC$ (a result based on a result from measurement-based quantum computing~\cite{jozsa2006introduction}). 
This result allows us to give many useful multi-qubit gates and routines in Section~\ref{sec:gates_created_in_LAQCC}. 
Beyond that we show that constant depth $\LAQCC$ circuits are contained in $\QNC^1$ and that any $\mathsf{IQP}$ circuit has an $\LAQCC$ implementation.
We conclude this section with an analysis of a more powerful instantiation of $\LAQCC$ and show an inclusion with respect to the class $\mathsf{PostQPoly}$, which is the class of circuits of polynomial depth with one additional post-selection gate. 
In Section~\ref{sec:state_prep_in_LAQCC} we give $\LAQCC$ circuit implementations for preparing the uniform superposition over an arbitrary number of states, the $W$-state and the Dicke state up to $k = \mathO(\sqrt{n})$. We furthermore give a log-depth circuit implementation for preparing the Dicke state for any $k$. We conclude by showing a $\LAQCC$ circuit for generating many body scar states of a particular type of Hamiltonian.



\section{Motivation and Opportunity}
%There are established video compression algorithms (e.g., HVEC, MPEG, etc.) that are used to compress consumer-oriented videos. 
The pertinent question that arises is, \emph{why do we need different compression algorithms for video analytics?}   Conventional video compression algorithms (e.g., HVEC or MPEG) are designed to maintain human-perceived visual quality; these compression algorithms uniformly degrade the quality across the video. In contrast, the goal of semantic compression is to maintain the accuracy of downstream analytics, and therefore, by using differential compression techniques it is able to achieve a better bandwidth-accuracy tradeoffs. The process of spatial and temporal compression with respect to object classification is presented in Figure \ref{fig:spatial_temporal}.
% Figure environment removed

To demonstrate the potential of semantic compression in comparison to conventional video compression in the context of video analytics, we carry out a series of experiments. Our results indicate that semantic compression offers better bandwidth-accuracy tradeoffs than conventional codecs, such as MPEG. To evaluate the performance of these compression algorithms, we calculate their F1-scores by comparing the outputs of the full-sized DNN on both the compressed and the original video data. Alternatively, this approach can be thought of as the full-sized DNN being placed at the source to act as an optimal semantic compression, which provides perfect estimates of the positions and sizes of the required regions for analytics (by the same DNN) at the server. This optimal algorithm, however, is not feasible in resource-constrained source devices, and serves only to illustrate the size of the opportunity.



% To highlight the potential of semantic compression in the context of video analytics compared , we carry out a series of experiments. Specifically, we show that semantic compression is able to achieve better bandwidth-accuracy tradeoff than conventional compression (such as MPEG).  % terms of both bandwidth efficiency and F1-score of the downstream analytics. 



% We determine the F1-score by comparing the outputs of the full-sized DNN on both the compressed and the original video data. Alternatively, 

% this can be thought of as the full-sized DNN placed at the source to act as an optimal semantic compression algorithm, which provides perfect estimates of the positions and sizes of the required regions for analytics (by the same) at the server. This optimal algorithm, however, is not feasible in resource-constrained source devices, and serves only to illustrate the magnitude of the opportunity.


% using the output of the full-sized object detector DNN on both the compressed and the original video data. We assume an optimal semantic compression algorithm that uses the full DNN at the source, which provides perfect estimates of the positions and the sizes of the regions that are needed (by the same DNN) at the server. 
% In this section, we motivate the need for a semantic compression algorithm by comparing it to traditional video codecs (specifically, MPEG) with respect to both bandwidth efficiency and accuracy of the downstream analytics. Specifically, when evaluating the accuracy, 
% Specifically, when evaluating the accuracy, we consider the output at \emph{the full-sized object detector DNN after the compressed data is delivered to it and used as input}. To estimate the potential of semantic compression, we assume we have access to the full-sized object detector DNN at the source cameras, which provides perfect estimates of the positions and the sizes of the regions that are needed (by the same DNN) at the server. 
% This optimal algorithm yields no loss in F1-score, and uses the least possible bandwidth within our scheme, and only serves to illustrate the size of the opportunity.  %We end the section by showing the amount of spatial and temporal redundancies present in the videos of our dataset.
% Also, we discuss the semantic compression opportunities that typically lie in a video. Lastly, we also show the possible bandwidth saving opportunities using spatial and temporal compressions separately, and in cascade. 


% Figure environment removed

% We use a video dataset collected from a range of real world scenarios (details presented later in Table \ref{tab:dataset}).
  Figure \ref{fig:mpeg_comparison} presents the performance of MPEG against the optimal versions of spatial, temporal, and spatial-temporal algorithms. Each MPEG bar is configured such that it consumes either similar or higher bandwidth than the semantic compression. In all cases, the MPEG suffers a 5-13\% F1-score drop for similar bandwidth use. This is a substantial loss in terms of F1-score: over the past 5 years, the improvement in the state of the art performance in ImageNet classification is less than 10\% \cite{beyer2020we, stock2018convnets}. Also, to reach a 95\% F1-score, MPEG is able to compress only 20\% of the highest quality video (e.g., the MPEG bar with 94.7\% F1-score), while spatio-temporal compression (albeit under optimal conditions) is able to compress over 80\%.

% Figure environment removed

We also use the optimal semantic compression algorithm to characterize the redundancies present along the spatial and temporal dimensions in real videos from our dataset listed in Table \ref{tab:dataset}. Generally, in these videos the object regions constitute a small fraction of the size of the field of view. Across all the videos, around 80\% of the video frames have 33\% or fewer object regions (Figure \ref{fig:spatial_cdf}). Also, approximately in 80\% of the 1-second batches sampled from these videos, only 20 frames in 30 frames-per-second (FPS) videos and 5 frames in 15 FPS videos are necessary for analytics (Figure \ref{fig:temporal_cdf}). 


% Moreover, in approximately 80\% of 1-second segments sampled from 30 FPS videos, only 20 or fewer frames suffice to reach an optimal F1-score (Figure \ref{fig:temporal_cdf}).%\nael{20 out of 30?  Earlier you say frame rate is 15 or 30, so 20 useful frames is not surprising.  Better to use percentage like you did with spatial.  Also definition of useful from a temporal perspective is not given precisely like you do for spatial.} 



% Lastly, in Figure \ref{fig:spatial_temporal_opportunity} we present the bandwidth savings opportunities using the optimal spatial, temporal, and spatial-temporal algorithms. We can see that for the drone, traffic cam, and dash cam dataset, the optimal spatial compression is able to reduce more than 20\% of the full size of the video. However, the traffic cam dataset is less sparse, so optimal spatial compression has less gain. We can also observe that the optimal temporal compression is able to reduce more than 80\% of the full size of the videos across all the dataset. Also, it is evident that it is possible to reduce the videos even more by applying spatial and temporal compressions in cascade.
% % Figure environment removed










% There are highly efficient video compression algorithms, such as MPEG, HEVC, etc.~\cite{grois2013performance} that are used to compress consumer-oriented videos. Naturally the question arises, \emph{why do we need different compression algorithms for video analytics?} In consumer-oriented videos, a human viewer may look at any part of the video, and as such, the aforementioned compression algorithms uniformly compress the entire video in order to maximize the visual appeal. However, for video analytics, compression quality should be measured in terms of the accuracy achieved by the downstream analytics.  Semantic compression recognizes that different spatio-temporal regions within the field of view contribute differently to the features needed by the analytics: for example, a background region may not be important and can be compressed aggressively without affecting accuracy (Spatial compression as can be seen in Figure~\ref{fig:i_spatial_compression}).  %As a result, any regions within the spatial-temporal volume of the video that do not contribute to the accuracy may be omitted. By leveraging this opportunity, it is possible to design a more aggressive compression algorithm for the purposes of video analytics. This algorithm will spatially compress the video frames by catering to the need of the DNN, preserving the features of the object regions in high-resolution, while degrading or even omitting the background regions (Figure \ref{fig:i_spatial_compression}). Also, 
% In addition, temporal redundancy may exist and enable additional temporal compression opportunities targeted towards preserving features rather than visual perception quality.  We may, for example, omit frames that do not have sufficient feature entropy as can be seen in Figure \ref{fig:i_temporal_compression}. 




% In this section, we motivate the use of semantic compression.   Specifically, we characterize the available compression opportunity from the perspective of semantic compression (preserving features rather than visual quality).  %In the second subsection, we discuss the tradeoffs in terms of how to structure the compression between the camera sources (clients) and the edge/cloud servers.  %We structure the presentation around a number of questions that underlie the main design decisions.

%\nael{I think this question/paragraph is boring at this point.  We already made this point twice, in abstract and intro and I dont think the current paragraph adds much that is new.  We should see if we can introduce the compression figure without this paragraph.}

% \nael{This section now is focused on characterizing the opportunity for semantic compression.  What would be ideal is if we show an experiment where regular video compression is ineffective, but I know we struggled to do that in the past and it may not be possible.}

% % Figure environment removed

%\subsection{Semantic Compression and size of the opportunity}

% There are highly efficient video compression algorithms, such as MPEG, HEVC, etc.~\cite{grois2013performance} that are used to compress consumer-oriented videos. Naturally the question arises, \emph{why do we need different compression algorithms for video analytics?} In consumer-oriented videos, a human viewer may look at any part of the video, and as such, the aforementioned compression algorithms uniformly compress the entire video in order to maximize the visual appeal. However, for video analytics, compression quality should be measured in terms of the accuracy achieved by the downstream analytics.  Semantic compression recognizes that different spatio-temporal regions within the field of view contribute differently to the features needed by the analytics: for example, a background region may not be important and can be compressed aggressively without affecting accuracy (Spatial compression as can be seen in Figure~\ref{fig:i_spatial_compression}).  %As a result, any regions within the spatial-temporal volume of the video that do not contribute to the accuracy may be omitted. By leveraging this opportunity, it is possible to design a more aggressive compression algorithm for the purposes of video analytics. This algorithm will spatially compress the video frames by catering to the need of the DNN, preserving the features of the object regions in high-resolution, while degrading or even omitting the background regions (Figure \ref{fig:i_spatial_compression}). Also, 
% In addition, temporal redundancy may exist and enable additional temporal compression opportunities targeted towards preserving features rather than visual perception quality.  We may, for example, omit frames that do not have sufficient feature entropy as can be seen in Figure \ref{fig:i_temporal_compression}. 

%\subsection{What is the size of the opportunity?}
%\textcolor{red}{can we replace the above with -- How much can we compress?}
%\textcolor{purple}{we can start with the section like "to answer the above question, we perform some investigation (please see results in Fig.{}). The results reveal that"}
% % Figure environment removed

% We conduct experiments to characterize the amount of redundancy available in real videos with respect to semantic compression.  Our video dataset is collected from a wide range of real-world scenarios: traffic cameras deployed in highways and intersections, drone footage, dash cam recordings, and surveillance videos from a parking lot, as presented in Table \ref{table:dataset}. In all of these videos, the object regions are sparser than the background regions, and constitute only a small percentage of the total frame. Throughout the four scenarios, approximately 80\% of the video frames have 33\% or fewer object regions, as can be seen in Figure \ref{fig:i_spatial_cdf}. Moreover, all the videos are captured at a high frame-rate, e.g., 15 or 30 frames-per-second (FPS). In approximately 80\% of 1-second segments sampled from these videos, 20 or fewer frames are sufficient for analytics, as can be seen in Figure \ref{fig:i_temporal_cdf}.\nael{20 out of 30?  Earlier you say frame rate is 15 or 30, so 20 useful frames is not surprising.  Better to use percentage like you did with spatial.  Also definition of useful from a temporal perspective is not given precisely like you do for spatial.} 


%\textcolor{purple}{the definition of optimal compression algorithm is not clear to me.}
% If an exact copy of the full-sized DNN were somehow placed at the source, we could use the bounding-boxes generated by this network to perform spatial and temporal compression, and the resulting compression algorithm would be completely lossless. We use this lossless compression in order to demonstrate the spatial and temporal compression opportunities in different videos in our dataset, as seen in Figure \ref{fig:i_spatial_temporal_opportunity}. Finally, by pursuing both spatial and temporal compression, we are able to take advantage of both of these opportunities.
%it is possible to achieve  gains over any single approach. 

% There are highly efficient video compression standards such as H.265 and MPEG, etc.~\cite{grois2013performance} that are used to reduce the size of consumer-oriented videos. The question naturally arises: why do we need specialized compression for video analytics?  In consumer-oriented videos, the target is to maximize the video quality as perceived by a human viewer within the constraint of the resource budget both in terms of networking or computation. Since a human viewer may be looking at any part of the video, traditional compression algorithms generally compress the field of view uniformly, for example, not distinguishing between regions that have objects from background regions~\cite{du2020server}. Optimizing beyond uniform compression requires tracking of human perception, e.g., in some application spaces such as Augmented/Virtual Reality, if the human focus can be tracked within the rendered field of view, compression may be tailored to where the user is looking catering for human perception--a type of rendering called \emph{foveated rendering}~\cite{patney2016perceptually}.  

% Similarly, in the case of analytics-oriented videos a different type of compression is required that will track the regions-of-interest to the server-side DNN, and preserve the semantics within those regions~\cite{patwa-20}. Traditional compression algorithms yield the following unsatisfying tradeoff between bandwidth and accuracy: as we compress more aggressively to reduce bandwidth consumption, equal amount of degradation is incurred at both important object regions along with background regions causing a drop in the accuracy. Therefore, a more effective semantic compression will cater to the needs of the server-side object detection DNN by non-uniformly compressing the video regions spatially: keeping the features of the object regions intact while aggressively compressing the background regions that are immaterial to the task (Figure ~\ref{fig:i_spatial_compression}). %We call this type of compression spatial compression, as illustrated in Figure . 

% A similar argument can be applied to videos that are captured in a high frame-rate. More frames in a video results in smoother viewing experience to the viewer, but do not contribute to the downstream analytics as object positions and sizes remain relatively constant between successive frames. As MPEG only reduces this temporal redundancy up to the point where it does not hamper the visual appeal, there is an opportunity to filter additional frames in video analytics \cite{li2020reducto} (Figure \ref{fig:i_temporal_compression}. Instead of relying on the pixel values of in the image, we use a novel approach that compares differences in the feature space which are more discriminative of the values of the frames. % meaningful features from these frames to determine if there are any redundancies. We call this type of compression temporal compression, as illustrated in Figure .


% We pursue this opportunity as well  Therefore, instead of applying the compression evenly across the spatial-temporal volume of the videos, in analytics, we aim to compress the regions that are not essential for the downstream analytics. It is noteworthy that, aforementioned conventional video codecs are still applicable after analytics-oriented compression to further reduce the outgoing video stream.



% \subsection{Why a learning-based compression?}
% % \subsection{How should we structure compression responsibilities between the client and the server?}
% % \nael{How is this different from the previous question? -- help me understand if my suggested question is ok.  I left the original question commented out.}
% % Decoupling the operation of video collection from the analytics renders the source-side compression logic oblivious to the server-side DNN accuracy. Therefore, independent heuristics perform poorly in streaming video analytics where the scene changes rapidly \cite{du2020server, li2020reducto, zhang2018awstream}. For example, the same heuristic that can successfully compress a video in a room may not work effectively when it is pointed towards a window facing a street. Also, depending on the query, some heuristics are proven be more effective than others \cite{li2020reducto}. In order to incorporate the server-side DNN into the compression logic, the server needs to send feedback to the source concerning how it requires the video to be compressed. 

% In literature, this feedback process is performed at different granularity. One approach is to utilize short-lived, but frequent feedback \cite{du2020server}. Although this results in a highly contextual compression algorithm, it causes a high network delay due to recurring communication between the source and the server. Another approach is to summarize the compression strategy into longer-lasting feedback, so that it can be updated only when needed \cite{li2020reducto, zhang2018awstream}. In order to prepare this feedback the server will need a sizable amount of unfiltered raw video from the source, which ramps up bandwidth usage. To overcome these issues, several properties of learning-based models come in handy.

% % Based on the previous discussion, there is a knowledge gap between the source and the server that is the root cause behind the aforementioned issues. 
% We suggest use a neural network as feedback which learns the shifting requirements of the server-side DNN in reaction to the changes in the source-side scenario. In our design this neural network is called the student network, and is prepared by distilling the knowledge \cite{zeiler2014visualizing} of the server-side teacher DNN. This student network inherits the generalization capabilities of learning-based models, which make the feedback tolerant to minor environment changes and effective through longer periods of time. Additionally, the knowledge-transfer capabilities of the neural networks \cite{pan2009survey, zhuang2020comprehensive} facilitates the student network to be updated with only a small amount of new data. Thus, a learning-based compression logic is able to reduce the network delay as well as save bandwidth usage. 

% The input-adjacent layers of the student network capture 2D patterns (e.g., corners, edges, etc.) in videos and remain relatively unchanged as the environment changes \cite{zeiler2014visualizing, liu2019fusing, kataoka2015feature}. This allows us to send feedback as low-cost updates to only the layers adjacent to the output of the student network. Finally, the size of the neural network is constant and small, a negligible fraction of the streamed video data. 
%As a result, \emph{LtC} allows the feedback to be both long-lasting and infrequent; thus, it does not adversely affect bandwidth savings or response delay.
% \nael{This section is confusing and does not make clear points.  I would rather talk about alternatives: everything at source, everything at server (with and without feedback).  Note that at this point we have not introduced our system, so the discussion has to be general.}


%\noindent\textbf{Why not tiny models?} 
% \subsection{Why not tiny models?}

% Recently there have been developments of tiny neural network models \cite{tiny} that can perform video analytics tasks on resource-limited devices. One could argue that why not use these models at the camera to avoid streaming video to remote cloud servers altogether? Although these tiny models can achieve high FPS in IoT and embedded devices, their accuracy is significantly lower than their full-sized counterparts~\cite{incremental_improv}. While these models can be useful for certain applications (e.g., face recognition, gesture detection, etc.), in this paper, we focus on critical applications (e.g., autonomous driving, unmanned flights, etc.) that require near-optimal accuracy.  
% Moreover, the server may have analytics that apply to multiple video streams or may require the video data for other purposes such as training.  

% \nael{This raises expectations that you will compare to the tiny models.}

% \textcolor{purple}{should not following definitions go to the evaluation section?}
% \subsection{What are the performance metrics?} \label{section:performance_metrics}
% We evaluate LtC as well as other baselines based on the following performance metrics: (i) accuracy, (ii) bandwidth use, and (iii) response delay. 
% \begin{itemize}[leftmargin=10pt]
%     \item \textbf{Accuracy:} We use F1 score (the harmonic mean of precision and recall) to measure the downstream analytics accuracy. We use F1 score for two reasons: i) the object and background regions are highly imbalanced; and ii) we want to penalize the false negatives. Consistent with previous approach \cite{du2020server}, we run the pipeline using raw video frames without any spatial or temporal compression (only MPEG is applied) to attain the \emph{ground truth} results. In this way, when we actually apply spatial and temporal compression, the ground truth results reflect any loss that was caused by the compression.  
%     \item \textbf{Bandwidth use:} The bandwidth use of compressed data is normalized with the size of the ground truth data, and represented as a fractional number in the range [0, 1]. This allows us to compare results run on different dataset, and also across arbitrary runtimes. 
%     \item \textbf{Response delay:} The response delay is measured in seconds, and implies the freshness of the results. It consists of the the network delay and the processing delay incurred between the video stream leaving the source and the results leaving the server. 
% \nael{This is inconsistent with our results where temporal did not save a lot (I guess probably it does not account for mpegs temporal compression).  It would be good to go back and relate to the optimal opportunity to see how close we got.}









\section{LtC Design and Implementation}
%This section describes the system design and implementation of LtC. We start off with a brief discussion of the overall architecture of LtC, following by the details of the individual components.

%\subsection{Overall Architecture of LtC}
The overall architecture of LtC is presented in Figure ~\ref{fig:architecture}. %LtC utilizes separate source and server components that collaborate to achieve compression. 
The source-side component is centered around a compact student neural network, which is able to identify the object regions in order to perform spatial compression. The student network also provides a tensor based feature vector extracted from its intermediate layers to be used for temporal filtering. As the camera captures the video, the source accumulates a fixed-size batch of frames to process at a time. %; this is advantageous if we plan to apply MPEG as a post-processing step to semantic compression. 
The student network applies both spatial and temporal compression before it sends the batch over to the server. As the batch arrives at the server, the teacher network performs analytics and looks for concept drift. Notably, the student network is pretrained on historical data allowing itself to quickly adapt to the present scenario with only a few frames.

% , and can catch up to the current scenario with minimal number of frames.

% : (1)  generates the analytics results; and (2) evaluates using a concept drift module whether a student update is needed. % evaluates the need for an update. %A concept drift module is placed at the server, which is responsible for making the decision of sending the newly trained student network back to the source. 
% Optionally, the server can use a offline pretrained module to initialize the parameters of the student network (steps A1 and A2 in the Figure). Alternatively, the first few batches of frames can be dedicated to train the student network.



% The overall architecture of LtC and its constituent modules are presented in Figure ~\ref{fig:architecture}. LtC has separate source and server components that collaborate to perform compression. The source side component centers around a small student neural network trained by the full DNN at the server (the teacher network); the student aids in the performance of both spatial and then temporal compression.   %As it camptures video, the source-side component first applies spatial compression and then temporal compression.
% Specifically, the teacher object detector DNN at the server trains the student network using a small number of frames and sends it over to the source. 
% The student network's goal is to identify regions of the image that are likely to contain objects to send those at higher resolution.  It is also used in the temporal compression algorithm; we use tensor differencing at the output of an intermediate layer (the encoder) to drive whether a frame is kept or discarded.  After compression is performed, the source sends the reduced video frames to the server, where it is analyzed by the object detector network in order to generate results. We describe the different components of the framework in more details in the remainder of this section.
%In addition, the server also trains the student network using the student-teacher training mechanism.  
%This freshly trained student network is then compared with the previous network by the concept drift module to determine if an update is warranted. In the initial iteration, both the source and the server do not posses a student network. 
%This cold start issue can be solved using one of the two ways: (1) the offline pertain module can prepare a student network using historical data beforehand, or (2) we dedicate the first few iterations to train a student network. 
% Figure environment removed

%Before going into the details of the individual modules of \emph{LtC}, we will first discuss about its overall workflow from a high level. 
% Figure.~\ref{fig:architecture} shows the overall organization of \emph{LtC}.  The system uses a small student neural network at the source to carry out both spatial and temporal compression.  This student network is trained using a few frames sent to the server using the full model at the server that is the eventual consumer of the videos (also called the teacher model).   %It is a server-driven compression mechanism, i.e., the DNN at the server influences the compression logic at the client, providing feedback. The feedback is sent in the form of a lightweight neural network model, which is developed following a unique student-teacher framework. 
% Specifically, the DNN plays the role of a teacher and distills its knowledge into a student neural network. As no compression logic is present in the camera at the beginning, it sends raw video frames to the server. Upon the arrival of the video frames at the server, the teacher DNN initiates the training procedure of the student network. Later this student network is sent back to the camera, where it is deployed to spatially and temporally compress the videos. 
% This procedure is repeated on demand if the environment changes. Specifically, a concept drift module is situated on the server, which determines if the camera-side student network requires an update.\nael{Could separate out this into a different figure.} Optionally, the student network can be prepared offline with historical data and placed on the camera before the pipeline starts. Both the camera and the server maintain a cache of previously used student networks which can reduce the cost of transferring a model if it has been previously used. In the following, we present each component of the \emph{LtC} architecture.  


\subsection{The student-teacher framework}

The student-teacher framework is at the heart of LtC's system design. This method of training enables a large neural network to distill its knowledge into a smaller network \cite{hinton2015distilling, wang2021knowledge}. As the student network specializes only on a subset of the whole data, and tackles a much simpler task than the actual analytics, its size can be kept small. Moreover, initial pretraining and subsequent updating allows for fast retraining with only a small amount of new data.


% The student can be kept compact, as it specializes only on a subset of the whole data, and solves a much simpler task. Moreover, it supports fast training with a small amount of new data as a result of pretraining and subsequent continual updates.

% The student can be small as it specializes only on a subset of the whole data, and solves a simpler task. Its training can be quick and with few new data as a result of pretraining and subsequent continual learning.

% The student is specialized to a subset of the data (the one used in its training), and it can also be solving a simpler task than the original teacher.  %It is particularly useful when we do not need the full capabilities of the larger neural network.   
%Often, both the teacher and the student networks have the same goals; however, they can sometimes be different as is the case in our application, with the student solving a simpler problem that can be viewed as a subset of the original teacher problem. 
In the context of LtC, the teacher DNN trains the student network to determine the objectness (likelihood of an object being present) in the regions within a frame.   In our approach, the video frames are divided into an array of non-overlapping equal-size regions. These regions are labeled using the results of the object detection from the teacher network $\mathcal{T}$, to be used in the training of the student network $\mathcal{S}$. 

The training starts as the camera sends $\mathcal{N}$ video frames $F^\mathcal{I}=[f^\mathcal{I}_1, f^\mathcal{I}_2, ..., f^\mathcal{I}_{\mathcal{N}}]$ to the server in $\mathcal{I}$th iteration. 
%We denote the posterior from the teacher network $\mathcal{T}$ as $P_\mathcal{T}(y | x^f_{ij})$ and the student work $\mathcal{S}$ as $P_\mathcal{S}(y | x^f_{ij})$, where $y$ is the measure of objectness. 
Upon receiving the frames, the teacher network $\mathcal{T}$ generates $C_f$ bounding boxes $BB^f=[bb^f_1, bb^f_2, ..., bb^f_{C_f}]$ around objects in frame $f$. %We filter out the bounding boxes with low confidence scores to get rid of noisy samples. Also, 
Each frame $f$ is split into a $\mathcal{L}\times\mathcal{L}$ array $X^f = [..., x^f_{ij}, ...]$ of non-overlapping same-sized regions, where $i$ and $j$ are 2D indices of a region in the frame. We denote the posterior from the teacher $\mathcal{T}$ and the student $\mathcal{S}$ networks as $P_\mathcal{T}(y | x^f_{ij})$ and $P_\mathcal{S}(y | x^f_{ij})$, respectively, where $y$ is the measure of objectness. We use \textit{Intersection over Union (IoU)} function to determine $P_\mathcal{T}(y | x^f_{ij})$ in the following manner:
\begin{equation}
    \mathcal{S}(x^f_{ij}) = P_\mathcal{T}(y | x^f_{ij}) = \begin{cases}
			1, & \max^{C_f}_{k = 1} IoU(x^f_{ij}, bb^f_k) > 0.5\\
            0, & \text{otherwise}
		 \end{cases}    
\end{equation}
%\nael{We could evaluate how successful the student is at what it does, and also how much it saves in size/complexity relative to the teacher.  I know you presented these results before, but they can make this section more interesting.  Ignore for now if there is no time.}
%\noindent 
%We use $P_T$ from $\mathcal{T}$ to learn $P_S$ from $\mathcal{S}$ by minimizing the Kullback–Leibler (KL)-divergence between the two distributions in each iteration $\mathcal{I}$.  

%\begin{equation} \label{eq:kl}
%    \mathcal{KL}(P_\mathcal{T} || P_\mathcal{S}) = \sum^{\mathcal{N}}_{f = 1 } \sum^{\mathcal{L}}_{i = 1}\sum^{\mathcal{L}}_{j = 1} P_\mathcal{T}(y | x^f_{ij}) \log \dfrac{P_\mathcal{T}(y | x^f_{ij})}{P_\mathcal{S}(y | x^f_{ij})}
%\end{equation}
In order to train a student network that approximates a pretrained teacher network whose parameters are frozen, we only have to minimize the Kullback–Leibler (KL)-divergence between the two distributions with respect to the parameters of the student network, which is equivalent to minimizing the following loss function in each iteration $\mathcal{I}$:
\begin{equation}
    \mathcal{L}(\theta_S) = -\sum^{\mathcal{N}}_{f = 1 } \sum^{\mathcal{L}}_{i = 1}\sum^{\mathcal{L}}_{j = 1} P_\mathcal{T}(y | x^f_{ij}) \log{P_\mathcal{S}(y | x^f_{ij};\theta_S)},
\end{equation}
where $\theta_S$ are the parameters of the student network.
\begin{table}[t]
\centering
\resizebox{0.8\linewidth}{!}{
\begin{tabular}{l|ccc} 
\hline
\multicolumn{1}{l}{}                 & Layer type         & Output Shape & Param \#  \\ 
\hline\hline
\multirow{7}{*}{\rotatebox{90}{Encoder}}   & Input Layer          & 28x28x3  & 0         \\
                                     & 2D Convolution Layer & 28x28x16 & 448      \\
                                     & 2D Max Pooling       & 14x14x16 & 0         \\
                                     & 2D Convolution Layer & 14x14x32 & 4640     \\
                                     & 2D Max Pooling       & 7x7x32   & 0         \\
                                     & Flatten Layer        & 1568         & 0         \\
                                     & Dense Layer          & 128           & 200832    \\ 
\hline\hline
\multirow{5}{*}{\rotatebox{90}{\hspace{7pt}Extension}} & Input Layer          & 128           & 0         \\
                                     & Dense Layer          & 32           & 4128      \\
                                     & Dense Layer          & 16          & 528      \\
                                     & Output Layer         & 1            & 17        \\
\hline\hline
\end{tabular}}
\caption{Architecture of the student network} %\mishkat{Why using resnet, explained in methodology}}
\label{tab:student_network_layers}
\end{table}

 \begin{table}[t]
\centering
\resizebox{0.8\linewidth}{!}{
\begin{tabular}{ccc} 
\hline
Network type                                                        & Param \# & Size      \\ 
\hline
Faster R-CNN ResNet-101 (Teacher) & 44M      & 196.5 MB  \\
Encoder (Student)                                                            & 205920   & 844.2 KB    \\
Extension (Student)                                                          & 4673     & 36.5 KB     \\
Student Total                                                             & 210593   & 880.7 KB    \\
\hline
\end{tabular}}
\caption{Sizes of the teacher, and the student network (broken into the encoder, the extension)}
\vspace{-0.2in}
\label{tab:relative_sizes}
\end{table}

We use the Faster R-CNN ResNet-101 ~\cite{ren2015faster} as our teacher network. %We choose this network to be consistent with DDS\cite{du2020server}, a recent system that is the closest baseline to our spatial compression algorithm. 
The student network (shown in Table \ref{tab:student_network_layers}) is inspired by the VGG-16 model~\cite{simonyan2014very}, which also uses alternating convolution and max pooling layers. % Other student network architectures are possible (we explore varying the number of layers in the evaluation section).    
The parameters of the full network as well as the student can be seen in Table~\ref{tab:relative_sizes}. Notably, the teacher is over 200x larger than the student network. 
% The training is bootstrapped either using some initial training on historical data (A1 and A2 on Figure~\ref{fig:architecture}), or alternatively by streaming the first few frames without compression to initiate the training of the first student network.  %The student network is occasionally updated after the initial deployment, if a concept drift module determines that its accuracy is dropping. %To further optimize this process, we constrain the update to the extension when only that part has changed.  



%\subsection{The encoder-extension split}
The student network $\mathcal{S}$ is divided into two sub-networks: (1) Encoder $\mathcal{U}$ and (2) Extension $\mathcal{V}$ (sizes in Figure \ref{tab:relative_sizes}). The encoder is responsible for converting the input frame-region into a feature vector. The extension is the later part, which translates the feature vector into an objectness score. In literature \cite{zhang2015design, pan2009survey}, this kind of split is most commonly performed to rapidly train a neural network for a task by reusing the features of a pretrained model on a similar task as a starting point. For related tasks, the encoder part of the network remains relatively unchanged, as it captures low-level features, such as, shapes, patterns, and other features, which do not drift significantly for similar objects. High-level features captured by the extension, such as objectness, tend to be more sensitive to environment changes. In the context of compression for video analytics, this encoder-extension split is a novel contribution of LtC. This split has two benefits: (1) it provides a vector of semantic features that is built implicitly during training, which we leverage for temporal filtering, and (2) it allows for efficient update of the student network by updating only the extension. 

%The objectness score $y$ of a patch $x^f_{ij}$ of frame $f$ is:
%\begin{equation}
%    y = \mathcal{S}(x^f_{ij}) = \mathcal{V}(\mathcal{U}(x^f_{ij}))
%\end{equation} 
%where $\mathcal{U}(x^f_{ij})$ is the feature vector.

\subsection{Temporal filtering using embedded features}

Temporal filtering traditionally involves performing a comparison between frames to see if the new frame includes sufficient new information.   This comparison typically uses manually crafted low-level features that operate directly on the input image (e.g., SIFT and HOG)~\cite{li2020reducto}. Such features can be sensitive to small changes to the input resulting from the presence of wind or similar effects that do not substantially change the semantic features of the image.

We propose a novel temporal filtering approach that uses directly embedding in the feature space of the student network to detect salient changes in the image.  Specifically, we use the differences in terms of the feature vector acquired from the encoder to judge whether a frame should be discarded.  %These features can be thought of as a collection of high-level abstract features that characterize frame-regions. 
The main advantage using this feature vector is: it is highly contextual and up-to-date compared to other manually crafted features. Moreover, it is more robust against noises in the input caused by environmental factors such as illumination or weather.

For temporal filtering, we calculate the sum of the differences between feature vectors in consecutive frames. If this difference is below a threshold value, then that pair is considered identical.  % Other distance metrics are also possible.  %Previous literature \cite{chen2015glimpse, li2020reducto} argue whether this threshold value should be static or be updated in each iteration, but that is out of the scope of this paper. 
We define a difference function between frames $f_1$ and $f_2$ as follows:
\begin{equation}
    D(f_1, f_2) = \sum^{\mathcal{L}}_{i = 1}\sum^{\mathcal{L}}_{j = 1} {\lVert \mathcal{U}(x^{f_1}_{ij}) - \mathcal{U}(x^{f_2}_{ij}) \rVert}^2_2
\end{equation}

Instead of sending $\mathcal{N}$ frames $F^\mathcal{I}$ = $[f^\mathcal{I}_1, f^\mathcal{I}_2, ..., f^\mathcal{I}_{\mathcal{N}}]$ the camera uses the difference function to divide $F^\mathcal{I}$ into $\mathcal{M}$ partitions $[p^\mathcal{I}_1, p^\mathcal{I}_2, ..., p^\mathcal{I}_{\mathcal{M}}]$ of consecutive frames in the $\mathcal{I}$th iteration, where $1 \leq \mathcal{M} \leq \mathcal{N}$ and all frames in a partition $p$ satisfy: 
\begin{equation}
    \max D(f_1, f_2) < th; \forall f_1, f_2 \in p 
\end{equation}

Lastly, we construct the list of filtered frames $\hat{F}^\mathcal{I}$ by selecting a representative frame $f$ from a partition $p$ as follows:
% one frame from each partition that minimizes the sum of the differences from all other frames in that partition. 
\begin{equation}
    f = \argmin_{x \in p} \sum_{\substack{x \neq y \\ y \in p}} D(x, y)
\end{equation}

\subsection{Spatial compression using objectness score}

Spatial compression is performed after the temporal filtering. Based on the objectness score of a frame-region given by the student network, that region is either preserved in high-quality or degraded. %The main advantage of using the student network is that we do not have to run the full-sized teacher DNN at the source.  
LtC uses the \emph{posterior} $\mathcal{V}(\mathcal{U}(x^f_{ij}))$ to define an identity function as follows:
\begin{equation}
        I(x_{ij})= \begin{cases}
            1, & \mathcal{V}(\mathcal{U}(x^f_{ij})) > 0.5\\
            0, & \text{otherwise}
        \end{cases}
\end{equation}
We keep a region $x_{ij}$ if $I(x_{ij})$ is 1, or compress it otherwise. Each frame $f$ is converted into a $\mathcal{L}\times\mathcal{L}$ array $X^f = [..., I(x_{ij})x^f_{ij}, ...]$ of non-overlapping same-sized regions, where $i$ and $j$ are 2D indices of a region in the frame.

\subsection{Updating the student network}
The student network may become less effective following a significant change in the environment or scene dynamics, leading to a concept drift. When concept drift is detected at the server, an update of the student network is triggered. A copy of the currently operating student network is maintained at the server, which is evaluated by the teacher network in each iteration to check if indeed object regions yield high objectness score. If the percentage falls below a fixed threshold, the server initiates a student-teacher training, causing the student network to change partially (freezing all other parameters but the extension) or, in rare cases, wholly. Based on the degree of change, the server then sends either the extension or the new student network to the source as an update. We observe that 85\% of the updates consist only of the extension, which is approximately 25x smaller than the encoder as can be seen in Table \ref{tab:relative_sizes}.


% as can be seen in Table \ref{tab:relative_sizes} the extension is approximately 25x smaller than the encoder.



% The student network may become less effective following a significant change in the environment or scene dynamics, leading to concept drift.  When concept drift is detected at the server, an update of the student network is triggered. 
%A concept drift module is situated at the server to determine if or when an update is warranted. 
% These updates can be sent in one of two tiers:  (1) the full student network needs to be updated; or (2) only the extension is updated.  We observe that most updates require updating only the extension; as can be seen in Table \ref{tab:relative_sizes} the extension is approximately 25x smaller than the encoder.

% For each incoming batch, the server uses the teacher network to label the regions in the frames, it uses a subset of the frames to detect concept-drift.  Based on the degree of the concept drift, the server can decide whether or not to update the student, and if so to update only the extension or the full network.%The server trains the student network and stores it in its memory in each iteration, but does not send it to the source every time. Rather, it tracks the last sent student network to the source, and compares it against the in-memory student network.
%If the performance of the new network is comparable to the network at the source, then no update is sent.  If the accuracy drop is small, within a pred-determined tolerance range, then only the extension is updated.  Finally, if the accuracy drops beyond the tolerance, the full student network is updated.  %Otherwise, the in-memory student network is sent as an update to the source. 

% \nael{Commented this out.  The concept of iteration not introduced so far.  Also, I think the formalism is unnecessary since the ideas is straightforward.}
%Adding to the previous notations, we use $(\mathcal{U}_m$, $\mathcal{V}_m)$ to represent the in-memory networks, and $(\mathcal{U}_s$, $\mathcal{V}_s)$ to indicate the last sent networks. Moreover, we use $\hat{\mathcal{X}}$ to indicate that the parameters of network $\mathcal{X}$ are frozen, and therefore, untrainable. Lastly, the operators $\texttt{train(}\mathcal{X}\texttt{)}$ and $\texttt{eval(}\mathcal{X}\texttt{)}$ are used to indicate the training and the evaluation of network $\mathcal{X}$ respectively.

%the subscript $m$ to indicate in-memory networks, and the subscript $s$ to indicate the last sent network. Moreover, we use a cap symbol ($\wedge$) over a network to indicate that the parameters of this network is frozen, and therefore, untrainable. Lastly, the operators  \texttt{train(X)} and \texttt{eval(X)} are used to indicate training and evaluation of network \texttt{X}. The mechanism of updating the student network is as follows:
%\begin{itemize}
%    \item In each iteration, $\mathcal{I}$, the server trains the in-memory networks $(\mathcal{U}_m$, $\mathcal{V}_m)$, and evaluates it to set up a baseline accuracy (\texttt{acc}).
%    \begin{equation*}
%        \ballnumber{1}\texttt{train(}\mathcal{U}_m\mathcal{V}_m\texttt{)} \hspace{10pt} \ballnumber{2}\texttt{acc} = \texttt{eval(}\mathcal{U}_m\mathcal{V}_m\texttt{)}
%    \end{equation*}
%    \item No update is sent if the last sent networks $(\mathcal{U}_s$, $\mathcal{V}_s)$ satisfy the following condition:
%    \begin{equation*}
%        \texttt{eval(}\mathcal{U}_s\mathcal{V}_s\texttt{)} \geq \texttt{acc} - \mathcal{E}
%    \end{equation*}
%    Where $\mathcal{E}$ used as a margin of tolerance.
%    \item Otherwise, only the extension $\mathcal{V}_s$ of the last sent student network is trained by keeping the parameters of the encoder $\mathcal{U}_s$ frozen. After that, the previous step is tried again, but if true, the extension $\mathcal{V}_s$ is sent to the source.
 %       \begin{equation*}
  %      \texttt{train(}\hat{\mathcal{U}_s}\mathcal{V}_s\texttt{)}
  %  \end{equation*}
 %   \item If everything else fails, the in-memory networks $(\mathcal{U}_m$, $\mathcal{V}_m)$ are sent to the source.
%\end{itemize}

\subsection{Discussion and General Properties of LtC}

A state-of-the-art spatial compression algorithm, DDS \cite{du2020server}, uses server-side feedback in order to identify the regions that require high-quality transmission. Specifically, a low-quality baseline transmission of the video is first sent to the server, which uses the full-sized DNN to identify the regions of interest that it requires in higher resolution.  It sends a request back to the source, which in turn resends those regions at a higher quality.  % the server initially performs analytics suing low-quality frames sent by the source, and incrementally refines the results by requesting high-quality patches for uncertain regions.
Although this approach is able to achieve reasonably high F1-score, it results in significant delay in server response. This delay includes the network delay incurred during multiple requests and responses, as well as the server processing delay after each response. Moreover, the number of feedback used in the process adversely affects the response delay. These delays are shown in Figure \ref{fig:i_dds_suboptimality} as a function of the number of feedback rounds used for each batch of frames. LtC solves this problem by training the student network to identify the regions of interest, and the compressed video is sent in a single shot.  Although the student network is smaller than the teacher network, LtC has the advantage of operating on the full resolution video; this is in contrast to DDS which transmits a low resolution video to detect the areas with objects.

% Figure environment removed

A recently proposed temporal filtering algorithm, Reducto \cite{li2020reducto}, uses a profiling based approach, where the server profiles batches of frames against the optimal actions (what threshold to use for filtering). %These optimal actions are applied at the source to future batches fitting the profile. 
This approach is also able to achieve a high F1-score, but has high bandwidth use as the source sends unfiltered batches to the server during the profiling, as presented in Figure \ref{fig:i_reducto_suboptimality}. This profiling happens once at the beginning and every time the environment changes, and requires a large number of frames for profiling. LtC solves this problem by using the student network for profiling, which is both long-lasting due to its generalization capabilities, and is able to quickly update itself with a few new frames by transferring knowledge from the previous environments.  Additionally, Reducto does not support spatial compression.
% Figure environment removed

In the case of moving cameras, such as, dash cam, drone, etc., the background regions are changing constantly. As a result, use of simple features (e.g., pixel values) will result in significant difference between consecutive frames, even if the objects did not move by much. The feature vector in LtC is robust to mobility related background changes, and therefore, can recognize background regions even in the presence of mobility. In Figure \ref{fig:i_features}, we observe that for a moving camera, difference values calculated using LtC's features has the highest Pearson correlation coefficient (0.92) compared to a number of other commonly used features.

% Figure environment removed

%Lastly, in Figure \ref{fig:spatial_temporal_opportunity} we use the optimal compression algorithm to showcase the relative performance of different compression strategies. We observed that across all the videos in our dataset the use of spatial and temporal compression together yields a better performance than any single strategy. 
%% Figure environment removed











% If we consider a setup where we could run a full-sized DNN at the source, it would be possible to exactly figure out where the object regions are, to achieve optimal compression without loss of accuracy. However, we assume it is infeasible to run a full-sized DNN at the source and approximation is necessary.  One option is to send low quality video to the server, and have the server determine which regions require additional resolution as in DDS~\cite{du2020server}. However, this results in loss of accuracy since the decisions are being made on low quality video, and with high delays since feedback is necessary from the server.  It also can be wasteful of bandwidth since: (1) It makes it impossible to carry out temporal semantic compression; and (2) even the baseline low quality transmission could use more bandwidth than necessary for background regions.  



% %Independent heuristics perform poorly when the scene changes rapidly \cite{du2020server, li2020reducto, zhang2018awstream}. For example, the same heuristic that can successfully compress a video in a room may not work effectively when it is pointed towards a window facing a street. Also, depending on the query, some heuristics are proven to be more effective than others \cite{li2020reducto}. Therefore, it is imperative for the source-side heuristics to adapt itself according to the requirements of the server-side DNN. This can be achieved if the server sends periodic feedback to the source, so that the server-side DNN can be incorporated into the source-side compression logic without actually hosting it.

% %\subsection{Overview of LtC}
% LtC uses an alternative approach where we have a low complexity network to discriminate between regions that have objects and regions that do not.   We use this information to differentially compress the video.  %Although some loss of accuracy may result since the discriminator network is lower in complexity we show that the spatial compression performance is competitive with feedback-based compression.  
% LtC offers a number of advantages: (1) Since the compression is at the source side, we are also able to carry out temporal compression, which we do using a novel feature based algorithm; (2) we are able to compress without feedback from the server improving response time; (3) We are potentially more bandwidth efficient because we do not need to send video at a baseline quality to the server to enable accurate decisions since our decisions are made at the source where the full video resolution is available.  LtC requires retraining when the scene dynamics change, but we introduce a lightweight retraining mechanism that makes this possible at low overhead.


% \subsection{Updating the student network}
% \nael{Could split up updating part into a separate section with a different figure.  Section 3 is massive.}

% When the scene dynamics change, the student network may no longer be effective in identifying areas with objects and we need to update the student network. A concept-drift detector at the server determines when an update is required. The updates consist of the parameter values. As a result, the size of the updates is directly affected by the architecture of the student network, and is one of the factors in how we select the architecture of the student network. The student network is significantly smaller than the full-sized DNN.\nael{Can we mention some specifics or point to a result?} 

% In selecting the student network architecture, we have to balance accuracy and overhead of updates.  At first, the analytics accuracy increases with adding layers to the student network, but eventually reaches a plateau.  In our experiments, we have observed that a student network of 4MB in size produced excellent accuracy at a reasonable size. Notably, this size is orders of magnitude smaller than the size of the video segments transferred over the duration of the operation of the model.  Moreover, our encoder-extension separation of the model further reduces the update size by updating only the extension.\nael{Removed iteration since this concept was not introduced before.} % However, considering the scenarios where the downstream bandwidth constrains the pipeline, we design the updates to be delivered in two discrete forms. 

% The encoder part of the student network is more robust than the extension when the environment changes. Thus, when typical changes in the environment occur we can update only the extension to restore the accuracy of the student model. For major changes, however, the complete model is required to be updated.  Periodically, we send a set of training frames which are split into unevenly sized segments (we used a 90\%-10\% split). The larger segment is used for student-teacher training, while the smaller segment is used for concept-drift evaluation. \nael{Should be clear in evaluation that you take these overheads into account.  The breakdown figures dont show these kinds of overheads.}

% A revoked student network due to an update might still be useful. For example, consider dashcam footage where a car moves from a city road onto a highway, then returns to the city roads. Therefore, we can cache previously used encoders and extensions at the camera and the server. %We can use the least recently used (LRU) strategy for model replacement. 
% These cached models are primarily maintained at the server, and control messages are sent to the camera so that it can synchronize itself by executing the same operations. [\textcolor{purple}{this sentence is not clear to me --  Therefore, in future discussion, it is assumed that any operations executed on them reflect at both the camera and the server.}]\nael{Yes, caching is not explained well.  How do you recognize if a previous model is needed again?  What is the reconfiguration process in that case?  You say stored at server, but there is a model cache on the client in the figure as well.  Perhaps better to remove it from this version of the paper or improve the explanation.}

% \begin{table}[t]
% \centering
% \begin{tabular}{ll} 
% \hline
% \texttt{train(N)} & Trains a network \texttt{N} using ST framework \\
% \texttt{eval(N)} &  Evaluates a network \texttt{N} \\ 
% \hline
% \texttt{use(N)} & Set \texttt{N} as running network \\
% \texttt{insert(N, C)} & Insert a network \texttt{N} in cache \texttt{C} \\
% \texttt{update(N, C)} & Update a network \texttt{N} in cache \texttt{C} \\
% \hline
% \end{tabular}
% \caption{Operators in the update routine.}
% \label{tab:operators}
% \end{table}

% Let us consider two LRU caches $\sigma = [\mathcal{U}_1, \mathcal{U}_2, ...]$ and $\phi = [\mathcal{V}_1, \mathcal{V}_2, ...]$ of maximum size $K$ that hold previously used encoders and extensions, respectively. $\tilde{\mathcal{U}} and \tilde{\mathcal{V}}$) indicate that their parameters have been frozen and will not change during training. In addition, we define a few necessary operators in Table~\ref{tab:operators} to describe the model updating scheme. Finally, the sequence of the operations is indicated with circled numbers.

% \begin{itemize}[leftmargin=10pt]
%     \item In each iteration, $\mathcal{I}$, the server uses the larger segment of the incoming video frames to train a student network. We define the encoder and the extension of this network as $\hat{\mathcal{U}}$ and $\hat{\mathcal{V}}$, respectively. This network $\hat{\mathcal{U}}\hat{\mathcal{V}}$ is always kept up-to-date with the most recent information. After training, the concept drift evaluation module evaluates this network by using the smaller segment of the incoming video frames to set up a baseline accuracy (\texttt{acc}). 
%     \begin{equation*}
%         \hspace{0.3in}\ballnumber{1}\texttt{train(}\hat{\mathcal{U}}\hat{\mathcal{V}}\texttt{)} \hspace{10pt} \ballnumber{2}\texttt{acc} = \texttt{eval(}\hat{\mathcal{U}}\hat{\mathcal{V}}\texttt{)}
%     \end{equation*}
%     \item If there exists any previous student network comprising encoders and extensions in cache $\sigma$ and $\phi$ respectively can perform as good as the up-to-date network ($\hat{\mathcal{U}}\hat{\mathcal{V}}$), then that network is used as the running network.
%     \begin{equation*}
%         \hspace{0.3in}\ballnumber{1}\exists\limits_{1 \leq i \leq \lvert \sigma \rvert} \texttt{eval(}\mathcal{U}_i\mathcal{V}_i\texttt{) > acc} \Rightarrow \texttt{use(}\mathcal{U}_i\mathcal{V}_i\texttt{)}
%     \end{equation*}
%     \item Otherwise, the previous student networks are trained by freezing the parameters of the encoders in $\sigma$, and making only the parameters of the extensions in $\phi$ trainable. If there exists any previous student network is able to match the performance of the up-to-date network ($\hat{\mathcal{U}}\hat{\mathcal{V}}$) after the training, then an update for the extension is issued and that student network is used as the running network.
%     \begin{equation*}
%         \hspace{0.3in}\ballnumber{1}\texttt{train(}\tilde{\mathcal{U}_i}\mathcal{V}_i\texttt{)}
%     \end{equation*}
%     \begin{equation*}
%         \hspace{0.3in}\ballnumber{2} \let\scriptstyle\textstyle\substack{\exists\limits_{1 \leq i \leq \lvert \sigma \rvert} \texttt{eval(}\mathcal{U}_i\mathcal{V}_i\texttt{) > acc} \\ \Rightarrow \texttt{update(}\mathcal{V}_i, \phi\texttt{)} \hspace{2pt}|\hspace{2pt} \texttt{use(}\mathcal{U}_i\mathcal{V}_i\texttt{)}}
%     \end{equation*}
%     \vspace{0.01in}
%     \item Otherwise, we need to insert the constituent encoder ($\hat{\mathcal{U}}$) and extension ($\hat{\mathcal{V}}$) of the up-to-date network ($\hat{\mathcal{U}}\hat{\mathcal{V}}$) into $\sigma$ and $\phi$, respectively, and also set it as the running network. In case of cache overflow, the least recently used networks are evicted from $\sigma$ and $\phi$.
%     \begin{equation*}
%         \hspace{0.3in}\ballnumber{1}\texttt{insert(}\hat{\mathcal{U}}, \sigma\texttt{)} \hspace{2pt}|\hspace{2pt}
%         \texttt{insert(}\hat{\mathcal{V}}, \phi\texttt{)} \hspace{2pt}|\hspace{2pt} \texttt{use(}\hat{\mathcal{U}}\hat{\mathcal{V}}\texttt{)}
%     \end{equation*}
% \end{itemize}
% \nael{This description of the cache is a lot of details on how the models are cached but the intuition of how they are used is missing.}

% Although, this is a magnitude smaller than the size of the video data, in some cases there might be shortage of available downstream bandwidth 
% gain of increasing the number of layers reaches a plateau.  


% The update can be delivered in two discrete levels: i) extension update, and ii) complete update.    

% \resizebox{.99\hsize}{!} 
% \begin{equation} 
%     \begin{cases}
%     <use, id, \square, \square>, & \argmax_i\hspace{5pt}eval(freeze(\phi^{i}_{\mathcal{E}}) + freeze(\phi^{i}_{\mathcal{X}})) > acc\\
%     abcd, & otherwise
%     \end{cases}
%     % \resizebox{.98\hsize}{!}{\argmax_i\hspace{5pt}eval(freeze(\phi^{i}_{\mathcal{E}}) + freeze(\phi^{i}_{\mathcal{X}})) > acc \implies <use, id, \square, \square>}
%     % eval(freeze(\phi^{i}_{\mathcal{E}}) + train(\phi^{i}_{\mathcal{X}})) > th     <update, i, \square, \mathcal{X}>
%     % otherwise <insert, i, \mathcal{E}, \mathcal{X}>  
% \end{equation}

% Instead of $\mathcal{N}$ frames $F^\mathcal{I}=[f^\mathcal{I}_1, f^\mathcal{I}_2, ..., f^\mathcal{I}_{\mathcal{N}}]$
% Let us say that the camera captures $\mathcal{N}_{org}$ video frames $F_{org}=[f_1, f_2, ..., f_{\mathcal{N}_{org}}]$.


% $\mathbb{I}_{f_1f_2}$ like below:
% \begin{equation}
%     \mathbb{I}_{f_1f_2} = \begin{cases}
% 			1, & \sum^{\mathcal{L}}_{i = 1}\sum^{\mathcal{L}}_{j = 1} (\mathcal{U}(x^{f_1}_{ij}) - \mathcal{U}(x^{f_2}_{ij})) > th_2 \\
%             0, & \text{otherwise}
% 		 \end{cases}    
% \end{equation}

% Let $F_{org}$ and $F_{fil}$ denote the list of frames before and after filtering respectively, and $\hat{f}$ be the last frame from $F_{org}$ that has been added to $F_{fil}$. Then, we determine the membership of a frame $f$ after $\hat{f}$ in the following manner:
% \begin{equation}
%     f \in F_{fil} \iff f \in F_{org}\hspace{5pt}\textrm{and}\hspace{5pt}\mathbb{I}_{\hat{f}f} 
% \end{equation}


% Let's say, out of $\mathcal{N}$ video frames $F^\mathcal{I}$ in iteration $\mathcal{I}$ only $\mathcal{M}$ frames are found to be unique

% , and $F^\mathcal{I_filtered}=[f^\mathcal{I}_1, ..., f^\mathcal{I}_{m - 1}, f^\mathcal{I}_{m}, ...]$ are the filtered frames.


% say out of $\mathcal{N}$ video frames $F^\mathcal{I}$ only $\mathcal{M}$ frames are found to be unique after temporal compression. 


% The sum of the differences between all the feature vectors in successive frames are calculated, and one is omitted if it goes beyond a threshold value ().

% Unlike specially crafted features (i.e. SIFT, HOG, etc.) that do not have context-specific knowledge, the feature vector captures up-to-date highly contextual information with the updates of the encoder network.  


% We use the posterior  
% , which allow us the run the extension network on only the feature vectors on the regions in the filtered frames. 
% The spatial compression comes after temporal compression in our pipeline, where the filtered frames are further reduced by omitting unnecessary regions in them. This allows us to maximize the quality of the video frame with respect to the available bandwidth as we know

% This procedure is repeated again if the environment changes beyond acceptable limit. There is a concept drift module situated at the server, which determines the need for    

% there is not compression logic present in the camerasthe camera sends raw video frames to the server. Upon receiving the video frames, 

% student nn is specific
% don't need no high res

%explain not so much update needed
%explain queu

\section{Evaluation}
\section{Evaluation} \label{sec:evaluation}

\begin{table*}[tbp]
\centering
\small
\begin{tabular}{cccccccccc}
\toprule
& \multicolumn{3}{c}{\msr} & \multicolumn{3}{c}{\negc} & \multicolumn{3}{c}{\wsj} \\
& Acc. & F1 & wF1 & Acc. & F1 & wF1 & Acc. & F1 & wF1 \\ \cmidrule(lr){2-4} \cmidrule(lr){5-7} \cmidrule(lr){8-10} 
\udel & 66.86 & 56.76 & 64.3 & \textbf{80.80} & 55.45 & 77.9 & 63.74 & 64.23 & 63.2 \\
\icsi & \underline{71.19} & 64.73 & 70.4 & 80.36 & 64.53 & \underline{78.6} & 64.62 & 64.15 & 63.4 \\
\cnts & 68.59 & 61.39 & 67.2 & 78.68 & 61.62 & 76.8 & 64.31 & 64.59 & 64.4 \\
\osu & 68.02 & 60.28 & 66.6 & 79.24 & 57.04 & 76.5 & 69.20 & 69.63 & 68.9 \\
\isg & 67.05 & 58.83 & 65.3 & 77.34 & 59.52 & 75.6 & 69.15 & 69.35 & 69.2 \\ \midrule
\bert & \textbf{71.68} & \underline{66.70} & \textbf{71.4} & 77.79 & \underline{72.87} & 77.7 & \underline{80.95} & \underline{80.93} & \underline{80.9} \\
\roberta & 70.91 & \textbf{67.53} & \underline{70.7} & \textbf{80.80} & \textbf{77.29} & \textbf{80.7} & \textbf{82.61} & \textbf{82.70} & \textbf{82.6} \\ \midrule
Average & 69.19 & 62.32 & 67.99 & 79.29 & 64.05 & 77.69 & 70.65 & 70.80 & 70.37 \\
\bottomrule
\end{tabular}
\caption{\label{tab:performance} Overall accuracy (Acc.), macro-averaged F1 (F1), and weighted-macro F1 (wF1) scores of the algorithms depicted in Section~\ref{sec:algorithm}. For instance, \msr-\udel refers to a C5.0 classifier trained on the \msr~corpus, using the feature set mentioned in \citet{greenbacker-mccoy-2009-udel}.}
%Its Acc., F1 and wF1 of this model are 66.86, 56.76, and 64.3, respectively.}
\end{table*}


In this section, we introduce the evaluation protocol and report the performance of the models.

\subsection{Implementation Details} \label{sec:implementation}

For \bert and \roberta, we used \textit{bert-base-cased} and \textit{roberta-base}, both from Hugging Face. For fine-tuning, we set the batch size to 16, the learning rate to 1e-3, the dropout rate to 0.5, and the size of the output layer to 256. We ran each model for 20 epochs and used the one that achieved the highest F1 score on the development set. The implementation details of the classic ML-based models can be found in Appendix~\ref{sec:appendixML}.

\subsection{Evaluation Protocol} \label{sec:protocol}

The main evaluation metric in the GREC-MSR shared tasks was accuracy. 
In addition to accuracy, we also report macro-F1 and weighted-macro F1. We argue that different metrics evaluate algorithms from different perspectives and provide us with different meaningful insights. 
For pragmatic tasks like REG, it makes sense to ask how well an algorithm performs on naturally distributed data which is often imbalanced. For these cases, reporting accuracy and weighted F1 are logical. 
Furthermore, analogous to other classification tasks, minority categories should not be overlooked. Take as an example the class \emph{description} in the \negc corpus, which occurs only 4\%. If a model fails to produce this class, the produced document might sound unnatural. Therefore, it is important to ensure that an algorithm is not over- or under-generating certain classes. Looking into accuracy and macro-F1 together provides insights into such cases.

\subsection{Performance of the Models}\label{subsec:overallacc}

The overall accuracy of the models, their macro F1, and their weighted-macro F1 are presented in Table \ref{tab:performance}. 
We also present the ranking of the models based on these scores in Appendix~\ref{sec:app_rank}. 


\paragraph{PLM-based Models.} The best-performing models across all corpora and metrics are PLM-based models.  In six out of nine rankings, \bert and \roberta are ranked as the top two models. The sole exception is \negc, where \bert is the second worst model. The benefit of using PLMs is the largest on the \wsj corpus. For example, \roberta improves the macro F1 score from 69.63 (i.e., the performance of the best ML-based model) to 82.70.


\paragraph{ML-based Models.} In contrast to the robust performance of the PLM models, the performance of the classic ML models is more corpus-dependent. In the case of \msr and \negc, \icsi is the best-performing model, while in the case of \wsj, it is at the bottom section of the rankings. Another interesting observation is the performance of the \udel models. In terms of accuracy, \udel has the highest performance in \negc, while it has the lowest performance in both \msr and \wsj. In terms of macro-F1 rankings, the \negc \udel model dropped from first to last place, whereas \bert improved from penultimate place to second place. In general, our ML models yielded lower scores than the original models used in the GREC study \citep{belz2009generating}. This could be attributed to a variety of factors, including differences in feature engineering and model parameters.

\paragraph{Comparing Different Metrics.} 

Upon comparing average scores across the three metrics, we observe that for \msr and \negc, PLMs are clear winners only when macro-F1 is the metric in question. However, for \wsj, PLMs are winners on all three metrics. This may be because the distribution of categories in \wsj is much more balanced than in the other two corpora.

\section{Related Work}
\section{Related Work}
\label{sec:relwork}

There has been relatively little prior work on formal verification of virtual memory.
Instead, much OS verification work has focused on minimizing reasoning about virtual memory management.
The original Verisoft project~\cite{alkassar2008verisoft,alkassar2010pervasive,alkassar2008formal,dalinger2005verification,hillebrand2005address,alkassar2008formal,starostin2010formal} relied on custom hardware which, among other things, always ran kernel code with virtual memory disabled, removing the circularity that is a key challenge of verifying actual virtual memory code: at that point page tables become a basic partial map data structure to represent user program address translations.
Other work on OS verification either never progressed far enough to address VMM verification (Verisoft XT~\cite{cohen2009vcc,cohen2010local,dahlweid2009vcc,cohen2013SOFSEM}), or uses memory-safe languages to enable safe co-habitation of a single address space by all processes (Singularity~\cite{Fahndrich2006language,Hunt2007singularity,Hunt2007sealing,Barnett2011specsharp}, Verve~\cite{Yang2010Verve}, and Tock~\cite{levy2017multiprogramming}).

The work that does address the core challenges of VMM verification is all associated with either \textsc{seL4} or \textsc{CertiKOS}.

\textsc{CertiKOS}~\cite{gu15,gu2016certikos,gu2018certikos,chen2016interrupts} is a microkernel intended for use as a hypervisor,
and its papers do not explicitly detail verification of the VMM, so we do not know the full space of which VMM functionality 
is verified, but we do know it includes the ability to map or unmap pages.
The work is clear, however, that it trusts low-level assembly fragments such as the instruction sequence which actually
switches address spaces, rather than verifying them.
The overall approach in that body of work is many layers of refinement proofs, using a
 proliferation of layers with small differences to keep most individual refinements tractable. In keeping with precursor work 
on the project from the same group~\cite{vaynberg2012compositional}, the purpose of some layers is to abstract away from 
virtual memory, so the proof is essentially a simulation proof covering for example a proof that execution with page-in on 
page faults is a valid refinement of an execution model where no paging occurs.
% Another key aspect of their approach is that the OS is written in Clight and compiled with \textsc{CompCert}~\cite{blazy2006formal,leroy2009formally,leroy2008formal}.
% CompCert's memory abstraction~\cite{leroy2008formal} assumes
% memory is a set of disjoint chunks of bytes with no overlap, so the lowest levels of CertiKOS must provide a matching 
% machine model as a layer. This prohibits virtual address aliasing, so CertiKOS cannot support simultaneous memory-mapped 
% (\texttt{mmap}) and stream-oriented (\texttt{read}/\texttt{write}) IO to a single file\todo{should we go into this detail?}, 
% and cannot use
% the common kernel design choice of mapping all physical memory into the bottom of the kernel's address space for direct access i
% while the kernel code is simultaneously mapped (and executed) at higher virtual addresses.
% This is not necessary for \textsc{CertiKOS}'s intended primary use case (a hypervisor), but means that \textsc{CertiKOS}'s
% approach cannot be used to support this functionality in other systems, without major surgery to \textsc{CompCert}.

\textsc{seL4}~\cite{Klein2009seL4,seL4TOCS,Sewell2013translation} is a formally verified L4 microkernel~\cite{Liedtke1995,Liedtke1996} (and the first verified OS kernel to run on real-world hardware), verified with a mix of refinement proofs and program logic reasoning down to the assembly level.
Because \textsc{seL4} is a microkernel, most VMM functionality actually lives in usermode and is unverified, and moreover, their hardware model omits address translation entirely and the MMU entirely~\cite{Klein2009seL4,seL4TOCS}. As a result, the limited page table management present in the microkernel treats page tables as idiosyncratic tree-maps, ignoring the risks posed by even transient inconsistencies that would crash the kernel on real hardware (like ``temporarily'' unmapping the kernel). This is mitigated primarily by manually identifying some trusted invariants (e.g., that the address range designated for the kernel is appropriately mapped) and setting up the proof to ensure those invariants are maintained (i.e., as an extra proof obligation not required by their hardware model).


One important outgrowth of the \textsc{seL4} project, not integrated into the main project's proof, was work by 
Kolanski and Klein which studied verification of code against a hardware model that \emph{did} include address translation
 --- the only work aside from ours to do so --- initially in terms of basic memory~\cite{kolanski08vstte} and subsequently 
integrating source-level types into the interpretation~\cite{kolanski09tphols}. 
They were the first work to model physical and virtual points-to assertions separately, defining virtual points-to assertions
in terms of physical points-to assertions mimicking page table walks, and defining all of their assertions as predicates on a
pair of (physical) machine memory and a page table root, an approach we improve on.

Their work has a number of significant limitations which our work addresses.
They also define their virtual points-to assertions such that a virtual points-to $p\mapsto_\mathsf{v} a$ owns the full 
lookup path to virtual address $p$. This means that given two virtual points-to assertions at the same time, such as 
$p\mapsto_\mathsf{v}a \ast p'\mapsto_\mathsf{v}b$, the memory locations traversed to translate $p$ and $p'$ must be disjoint. 
This means the logic has a peculiar limit on how many virtual points-to assertions can coexist in a proof. Since page tables 
fan out, the bottleneck is the number of entries in the root table. For their 32-bit ARMv6 example, the top-level address is 
still 4Kb (4096 bytes), and each entry (consumed entirely by a virtual points-to in their scheme) is 4 bytes, so they have a 
maximum of 1024 virtual points-tos in their ARMv6 configuration. Any assertion which implies more than that number
of virtual addresses are mapped implies false in their logic.
(They do formulate their logic over an abstract model, but every architecture would incur a similar limitation;
Na\"ively transferring their model to x86-64 4-level tables would yield a limit of 512 assertions (also a 4Kb root page, 
but 8-byte entries).

% Kolanski and Klein's points-to assertions do not model that page table entries for nearby addresses typically 
% \emph{share} entries in higher layers of the page tables --- a single L1 entry maps 4KB of memory on many architectures,
% but their logics avoid fractional ownership, so they can in fact only use a single memory location per page of memory.
% In fact, because their virtual points-to assertions contain \emph{full} ownership of all entries, even in the highest-level
% page table (L4 in our case), each entry can contribute to only a single mapped address. Thus any assertion
% in their logic that implies there are more virtual addresses mapped than entries in the top-level page table implies false.
Our definitions make use of fractional permissions throughout; Figure \ref{fig:strongvirtualpointsto}'s definition
of \lstinline|L4_L1_PointsTo| ellides the specific fractions used, but it in fact asserts 1/512 ownership of
the L1 entry, 1/($512^2$) of the L2 entry, and so on, so each entry may map the appropriate number of machine words.

As noted earlier, Kolanski and Klein's logics, by collocating both the physical ownership of the page table walk
as part of the virtual points-to itself, preempt support for changes to page tables which do not actually affect 
address translation.

The other major distinction is that Kolanski and Klein have no accounting for other address spaces.
Their logic does not deal with change of address space, and has no way to assert that certain facts hold
in another address space.
They verify only one address space manipulation: mapping a single unmapped page into the current address space (in both papers).
We verify this, as well as a change-of-address-space, which requires us to introduce assertions for talking
about other address spaces (we must know, for example, that the precondition of the code after the change must be true
in the \emph{other} address space), and to deal with the fact that the standard frame rule
for separation logic is unsound in the presence of address space changes and address-space-contingent assertions.
% The mapping write is verified in an ad hoc way by unfolding the machine semantics, because the logic lacks proper reusable rules for
% updating page tables.

Our approach in this paper uses modalities to distinguish virtual-address-based assertions that hold only in specific 
address spaces, making it possible to manipulate other address spaces, and equally critically, to \emph{change} address 
spaces while reasoning about correctness. 

Unlike our work, Kolanski and Klein prove very useful embedding theorems stating that code that does not modify page table 
entries can be verified in a VM-ignorant program logic, and that proofs in that logic can be embedded into the VM-aware logic 
(essentially by interpreting ``normal'' points-to relations as virtual points-to facts). While we have not proven such a result,
an analagous result {should} hold of our work: consider that the doubles for the \texttt{mov} instructions
that access memory behave just as one would expect for a VM-ignorant logic~\cite{Chlipala2013Bedrock}.
With our general approach to virtual points-to assertions being inspired by Kolanski and Klein, \emph{both}
 our approach and theirs could in principle be extended to account for pageable points-to assertions by adding additional 
disjunctions to an extended points-to definition; embedding ``regular'' separation logic into such a variant
is the appropriate next step to extend reasoning to usermode programs running with a kernel that may demand-page the program's
memory.

As noted throughout the paper, the inspiration for our other-space modality comes from hybrid logic~\cite{areces2001hybrid,blackburn1995hybrid,gargov1993modal,goranko1996hierarchies},
where modalities are indexed by \emph{nominals} which are names for specific individual states in a Kripke model.
We are aware of only two prior works combining hybrid logics with program logics specifically. 
Brotherston and Villard~\cite{brotherston2014parametric} demonstrated that may properties true of various 
separation logics are not definable in boolean \BI (\BBI), and showed that a hybrid extension \HyBBI allows
most such properties to be defined (e.g., the fact that separating conjunction is cancellative is unprovable 
in boolean \BI, but provable in \HyBBI). There, nominals named resources 
(roughly, but not exactly, heap fragments). 
Gordon~\cite{gordon2019modal} described a use of hybrid logic in the verification of actor programs, 
where nominals named the local state of individual actors (with such assertions stabilized with a 
rely/guarantee approach). Beyond these, there is limited work on the interaction of specifically 
\emph{hybrid logic} with substructural logics. 
Primarily there is a line of work on hybrid linear logic (\HyLL)~\cite{despeyroux2014hybrid}, 
originally used as a way to more conveniently express aspects of transition systems in linear logic. 
However, \HyLL's proof rules offer no non-trivial interactions with multiplicative connectives 
(every \HyLL proof can in fact be embedded into regular linear logic~\cite{chaudhuri2019hybrid}, 
unlike Brotherston and Villard's \HyBBI, which demonstrably increases expressive power over its base \BBI.

In both \HyLL and \HyBBI, nominals denote worlds with monoidal structure (as worlds in Kripke semantics
for either LL or \BBI necessarily have monoidal structure). Our nominals, by contrast, 
do not name worlds in the same sense with respect to Iris's CMRAs, 
but in fact \emph{classes} of worlds, because the names are locations 
(a means of \emph{selecting} resources) rather than resources.  
A key difference is that the use of nominals in those logics corresponds specifically to hypothetical 
reasoning about resources (until a nominal is connected to a current resource, in which case conclusions 
can be drawn about the current resource), which means the modalities themselves do not ``own'' resources. 
Instead, assertions under our other-space modality can and do
have resource footprints.
Pleasantly, we sidestep most of the metatheoretical complexity of those other substructural hybrid
systems by building our logic within a substructural metatheory (\iris).

\iris has been used to build other logics through pointwise lifting, notably logics that deal with weak
memory models~\cite{dang2019rustbelt,dang2022compass}. Those systems build a derived logic
whose lifting consists of functions from thread-local views of events (an operationalization of the release-acquire + nonatomic
portion of the repaired C11 memory model~\cite{lahav2017repairing}): there modalities $\Delta_\pi(P)$ and $\nabla_\pi(P)$
represent that $P$ held before or will hold after certain memory fence operations by thread $\pi$.
The definitions of those specific modalities existentially quantify over other views, related to the ``current'' view (the one where
the current thread's assertions are evaluated), and evaluate $P$ with respect to those other views. This approach to parameterizing
assertion semantics by a point of evaluation, and evaluating modalized assertions at other points, is what it means
to have a modality at all.
It is \emph{not}, however, an instance of hybrid logic, which is specifically demarcated by an assertion language where
\emph{assertions}, not their semantics, choose and name the evaluation points for modal assertions.
A hybrid extension of the aforementioned logics would include assertions which named specific views at which to evaluate
$P$, in the syntax of the assertion (e.g., $\Delta_\pi^v(P):=\lambda\_\ldotp (P\;v)$) rather than the 
$\Delta_\pi(P):= \lambda v\ldotp (\exists v_{rel}\ldotp \ownGhost{\pi}{\mathsf{RelV}(v_{rel})\;v} \ast (P\;v_{rel})))$ actually used.
Note the hybrid version takes the place to evaluate $P$ as a parameter, and therefore allows the \emph{derived} (modal) logic to explicitly
reason in terms of evaluation points, rather than hiding all points of evaluation in the internal definitions of modalities.




\section{Conclusion}
\section{Conclusion and Future Work}
In this work, I design corruption-robust algorithms for the Lipschitz contextual search problem. I present the \emph{agnostic checking} technique and demonstrate its effectiveness in designing corruption-robust algorithms. There are several open problems for future research. First, in the algorithm I propose for pricing loss, the schedule for agnostic checks is fixed upfront. Can the learner design an adaptive checking schedule for the pricing loss? Second, this work assumes the learner has knowledge of the Lipschitz constant $L$. Can the learner design efficient no-regret algorithms without knowledge of $L$? 

%%
%% The next two lines define the bibliography style to be used, and
%% the bibliography file.
\bibliographystyle{ACM-Reference-Format}
\bibliography{main}

%%
%% If your work has an appendix, this is the place to put it.
\appendix
\input{}

\end{document}
\endinput
%%
%% End of file `sample-sigconf.tex'.
