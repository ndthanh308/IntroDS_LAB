\nael{This subsection should be a discussion subsection after you introduce LtC}

\textbf{Multi-feedback-loop based approaches suffer in real-time applications.} The state-of-the-art spatial compression mechanism \cite{du2020server} uses a multi-feedback loop for each frame to figure out the locations and the sizes of object regions. In the initial phase, the camera sends low-quality frames to the server so the DNN can create a list of bounding boxes where objects might reside. The list is sent back to the camera as feedback requesting high-quality patches only for the bounding boxes mentioned in the list. This process is repeated multiple times in subsequent phases until satisfactory accuracy is achieved. The problem with this approach is that the multi-feedback loop is valid for a single frame, and depending on the number of feedback used, the results are delayed proportionally to the network round-trip-time (RTT) and server processing time. Even if just two phases are used, the response delay is essentially doubled. We aim to use a neural network model as feedback that persists through a long stream of video frames and can act in a single phase \textcolor{purple}{the concept of phase is not clear to me}.




\textbf{Profiling-based approaches hinder bandwidth savings.} The state-of-the-art temporal compression \cite{li2020reducto} mechanism profiles one-second segments of videos and map them to the best possible action \textcolor{purple}{what do you mean by action? which action are you talking about?}. Initially, the camera sends high-quality unfiltered frames for a period of time to the server. Besides performing the video analytics tasks, the server profiles video segments and registers them against the optimal action into a hashtable. The hashtable is sent back to the camera as feedback, where subsequent video segments are matched against the hashtable, and the registered action is performed. The profiling mechanism is initiated if a video segment does not match any entries in the hashtable. Effectively, both in the initial phase and any environment changes trigger the profiling mechanism. In the case of streaming video analytics, the scenario is very dynamic, and often the video segment does not match any entries in the hashtable. As a result, the camera halts the compression until the updated hashtable arrives from the server. This is a waste of bandwidth as each profiling period is usually long and needs a bulk of new frames. By leveraging the generalization capabilities of the feedback neural network model, the update mechanism is rarely called and can be finished using only a few new frames.


