\documentclass[doublecol]{epl2} 
% or \documentclass[page-classic]{epl2} for one column style

%\newcommand\emc{E=mc^{2}}

\usepackage{amsmath,amssymb}
\usepackage{bm}
\usepackage{physics}

\usepackage{hyperref}


\bibliographystyle{eplbib}

\newcommand\eq{\text{eq}}
\newcommand\st{\text{s}}
\newcommand\hcs{\text{HCS}}
\newcommand{\sgn}{\text{sgn}}
\newcommand{\fin}{\text{f}}
\newcommand{\ini}{\text{i}}

\title{Non-equilibrium memory effects: granular fluids and beyond}
%\shorttitle{Title} %Insert here a short version of the title if it exceeds 70 characters

\author{A. Patrón, B. Sánchez-Rey, C. A. Plata, and A. Prados}
%\shortauthor{F. Author \etal}

\institute{Física Teórica, Universidad de Sevilla, Apartado de Correos
  1065, E-41080 Sevilla, Spain }


\abstract{ In this perspective paper, we look into memory effects in
  out-of-equilibrium systems. To be concrete, we exemplify memory
  effects with the paradigmatic case of granular fluids, although
  extensions to other contexts such as molecular fluids with non-linear drag are also considered. The focus is put on two archetypal memory effects: the Kovacs and Mpemba effects. In brief, the first is related to imperfectly reaching a steady state---either equilibrium or
  non-equilibrium, whereas the second is related to reaching a steady
  state faster despite starting further. Connections to optimal control theory thus naturally emerge and are briefly discussed.}


\begin{document}

\maketitle



\section{Introduction}\label{sec:intro}

Under quite general conditions, many physical systems tend in the long
time limit to a state in which all trace of initial conditions is
lost. This state is often stationary, either an equilibrium state or a
non-equilibrium steady state (NESS), but it also may be a
time-dependent ``hydrodynamic'' state---in which a reduced description in terms of a few ``thermodynamic'' or
``macrosocopic'' variables accounts for the complete characterisation of the time evolution of the system.

Memory effects are intimately related to
aging~\cite{cugliandolo_evidence_1994,keim_memory_2019,jaeger_temperature_2022}. A
system displays aging when its relaxation or time correlations are not
invariant under time translation after being aged for a long waiting time; instead, they explicitly depend on such a  time. A memory effect emerges in a physical system when its time
evolution depends on the previous history, i.e.~on its initial
preparation that, in turn, depends on how it has been previously
``aged''.

A classic example of memory effect is the so-called Kovacs hump,
first reported by Kovacs when measuring the volume relaxation of polymeric
glasses~\cite{kovacs_transition_1963,kovacs_isobaric_1979}. Analogous
behaviours have been repeatedly observed in many different
contexts~\cite{chow_molecular_1983,berthier_surfing_2002,buhot_kovacs_2003,bertin_kovacs_2003,arenzon_kovacs_2004,cugliandolo_memory_2004,mossa_crossover_2004,aquino_kovacs_2006,prados_kovacs_2010,bouchbinder_nonequilibrium_2010,diezemann_memory_2011,chang_kovacs_2013,ruiz-garcia_kovacs_2014,prados_kovacs-like_2014,brey_memory_2014,
  plata_kovacs-like_2017,kursten_giant_2017,  mompo_memory_2021,peyrard_memory_2020,mandal_memory_2021,militaru_kovacs_2021,lulli_kovacs_2021,sanchez-rey_linear_2021,patron_strong_2021,godreche_glauber-ising_2022}.
Let us consider a 
quantity $P$ of a physical system in contact with a thermal bath.  Its
equilibrium value is denoted by $P_{\eq}(T)$, which is assumed to be a monotonic function. The Kovacs hump is the non-monotonic response
of the system to the two-jump protocol described below.

% Figure environment removed
Figure~\ref{fig:kovacs} shows a sketch of the Kovacs protocol and the associated Kovacs response. The system of interest is initially
equilibrated at temperature $T_{i}$, and therefrom it is aged at a
lower temperature $T_{1}<T_{i}$ in the time interval $0<t<t_{w}$. At
$t=t_{w}$, the instantaneous value of $P$ is $P(t_{w})$, and at this
point the temperature of the bath is abruptly changed to $T_{f}$, such
that $P(t_{w})=P_{\eq}(T_{f})$---thus, $T_{i}>T_{f}>T_{1}$. The system
displays the Kovacs effect when, for $t>t_{w}$, $P$ departs from its equilibrium
value, which $P$ already has as a consequence of the choice of
$T_{f}$, and presents a non-monotonic behaviour. The existence of this Kovacs hump entails that the pair
$(T,P)$ does not suffice to completely characterise the state of the
system: additional state variables are necessary.



Another example of memory effect is the Mpemba
effect~\cite{mpemba_cool_1969}. Originally, the
Mpemba effect refers to ``hot'' water freezing faster than ``cold''
water~\cite{mpemba_cool_1969,jin_mechanisms_2015}. In this context, the very existence of
the Mpemba effect is still
controversial~\cite{burridge_questioning_2016,burridge_observing_2020}. Recently, the Mpemba effect has attracted the attention of the
non-equilibrium physics community, understanding it in a generalised
way as follows. The relaxation of two samples of the same system
to a common final steady state is considered. Under certain
conditions, the sample initially further from the steady
state relaxes thereto faster than that initially closer, in contradiction with the usual Newton's law of cooling~\cite{maruyama_newtons_2021}. 

The Mpemba effect is qualitatively depicted in fig.~\ref{fig:mpemba}. Both the Mpemba effect---the hotter cools
sooner---and the inverse Mpemba effect---the colder heats sooner have
been observed in many different physical
contexts~\cite{lu_nonequilibrium_2017,lasanta_when_2017,klich_mpemba_2019,torrente_large_2019,baity-jesi_mpemba_2019,yang_non-markovian_2020,biswas_mpemba_2020,mompo_memory_2021,santos_mpemba_2020,carollo_exponentially_2021,chetrite_metastable_2021,uskokovic_and_2021,biswas_mpemba_2021,busiello_inducing_2021,takada_mpemba_2021,gomez_gonzalez_mpemba-like_2021,
  patron_strong_2021,megias_mpemba-like_2022,zhang_theoretical_2022,degunther_anomalous_2022,biswas_mpemba_2022,lin_power_2022,megias_thermal_2022,holtzman_landau_2022,kumar_anomalous_2022,yang_mpemba_2022,chorazewski_curious_2023,biswas_mpemba_2023,sun_physics_2023}. In
the theoretical studies of the Mpemba effect, two main frameworks have
been employed: the stochastic thermodynamics (or entropic)
approach~\cite{lu_nonequilibrium_2017}, and the kinetic (or thermal) approach~\cite{lasanta_when_2017}, which we will describe later in detail. In the former, distance to equilibrium is defined in probability space, e.g. with the Kullback-Leibler distance. In the latter, distance to equilibrium is monitored through the kinetic temperature, which is proportional to the average kinetic energy.\footnote{For equilibrium systems, due to the equipartition theorem, the
kinetic temperature equals the thermodynamic temperature. This is no longer the case for out-of-equilibrium states.}
% Figure environment removed



In the Mpemba effect, the system that is further from the steady state  somehow takes a shortcut and thus relaxes thereto faster than the closer one. Then,
there appears a natural connection with the general field of shortcuts
or, employing the terminology introduced in
ref.~\cite{guery-odelin_driving_2023}, swift state to state
transformations. In particular, a related problem is the optimisation
of the relaxation route to equilibrium---or to a NESS. For given
initial and final states, the minimisation of the connection time
between them by engineering the time dependence of some physical
quantities, like the temperature or the potential, is a well-defined
mathematical problem in optimal control
theory~\cite{liberzon_calculus_2012}. This
is the classic brachistochrone problem, which very recently has been
addressed for both quantum and non-equilibrium
systems~\cite{deffner_quantum_2017,plata_finite-time_2020,lam_demonstration_2021,prados_optimizing_2021,ruiz-pino_optimal_2022,patron_thermal_2022,guery-odelin_driving_2023,aghion_thermodynamic_2023,pires_optimal_2023}.





%No paragraph of organisation, since it is a letter


\section{Kovacs effect}\label{sec:Kovacs}

For systems with a master equation dynamics, there are
general results for the shape of the Kovacs hump in linear
response. These results hold under quite general conditions, basically
(i) a canonical form of the equilibrium probability distribution function (pdf), proportional to
$\exp(-\beta H)$, with $\beta=(k_{B}T)^{-1}$ and $H$ being the
system's Hamiltonian, and (ii) detailed
balance in the dynamics~\cite{prados_kovacs_2010}. With these assumptions, the form of
the Kovacs hump for the energy $E(t)=\expval{H}(t)$ is directly related to
the form of its ``direct'' relaxation function $\phi_{E}(t)$ from $T_{i}$
to $T_{f}$, with only one jump. 

From the explicit expression of the Kovacs hump in linear response, eq.~(43)
of ref.~\cite{prados_kovacs_2010}, one deduces that: (i) the Kovacs
hump is always positive, i.e. $E(t)\ge E_{\eq}(T_{f})$, (ii) there is
only one maximum of $E(t)$. Interestingly, the explicit
  expression of the Kovacs hump derived in
  ref.~\cite{prados_kovacs_2010} resembles the phenomenological
  expression written by Kovacs~\cite{kovacs_isobaric_1979}. Although
the majority of studies are done in the non-linear regime, i.e. with
large values of the temperature jumps, the behaviour described by the linear response theory, i.e. (i)
and (ii) above, a positive hump with only one maximum, is the one
found in glassy and other complex
systems~\cite{kovacs_transition_1963,kovacs_isobaric_1979,chow_molecular_1983,berthier_surfing_2002,buhot_kovacs_2003,bertin_kovacs_2003,cugliandolo_memory_2004,arenzon_kovacs_2004,mossa_crossover_2004,aquino_kovacs_2006,prados_kovacs_2010,bouchbinder_nonequilibrium_2010,diezemann_memory_2011,chang_kovacs_2013,ruiz-garcia_kovacs_2014,peyrard_memory_2020,mandal_memory_2021,
lulli_kovacs_2021,patron_strong_2021,godreche_glauber-ising_2022};
thus the term ``normal Kovacs hump'' has been coined to describe
it. The normal hump stems from the structure of the direct relaxation
function in linear response, which is a sum of exponentially
decreasing modes with positive coefficients.



%FIRST GLASSY SYSTEMS, GENERAL RESULT FOR MASTER EQUATION, MAYBE THE
%ISING MODEL (ONE FIGURE)

In glassy systems, the emergence of the Kovacs effect is often
explained as a consequence of the complex energy landscape typical
thereof. Still, the Kovacs effect has also been observed in systems
with a much simpler energy landscape. A paradigmatic case is that of
granular gases, which are intrinsically non-equilibrium systems:
energy is purely kinetic but it is continuously dissipated in
collisions. 
Therefore, an external mechanism is needed to drive the
system to a stationary state, which is always a NESS with a
non-Maxwellian velocity distribution function (vdf)~\cite{poschel_granular_2001}. The simplest one is that of the uniformly heated granular gas, in
which independent white noise forces with variance $\chi$ act on all
 particles. The kinetic temperature---here also called granular temperature---at the NESS is then a certain
function of $\chi$~\cite{van_noije_velocity_1998,montanero_computer_2000}. 

Despite the simple energy landscape, the Kovacs effect neatly appears
when a system of smooth inelastic
  hard
particles\footnote{In smooth collisions, the tangential component of the relative velocity is conserved, whereas the normal component is reversed and shrunk with the restitution coefficient $\alpha$; the energy loss is thus proportional to $1-\alpha^2$ and $\alpha=1$ corresponds to the elastic case.} is submitted to the two-jump Kovacs protocol, with the
intensity of the driving playing the role of the bath temperature,
$\chi_{i}\to \chi_{1} \to
\chi_{f}$~\cite{prados_kovacs-like_2014,trizac_memory_2014}. This
neatly implies that the instantaneous value of the kinetic temperature
$T(t)$ does not suffice to completely describe granular fluids. It is
the non-Gaussianities that are responsible for the emergence of the
Kovacs effect, and it is thus essential to incorporate them to the
physical picture. It suffices to do so in the simplest way by
including only the excess kurtosis $a_{2}(t)$---the
so-called first Sonine approximation.

More interestingly, the sign of the Kovacs hump depends on the
inelasticity. Specifically, it depends on the sign of the excess
kurtosis at the steady state, $a_{2}^{\st}$, which is negative
(positive) for small (large) inelasticity. The key point is the
cooling rate being an increasing function of $a_{2}$. By aging the
system with a very low value of the driving $\chi_{1}$, the system
falls onto the homogeneous cooling state (HCS)~\cite{brey_homogeneous_1996},
in which the granular fluid freely cools following Haff's law~\cite{haff_grain_1983}, $T(t)\propto t^{-2}$, and the excess kurtosis becomes constant and equals $a_{2}^{\hcs}$. One always has $\sgn(a_{2}^{\hcs})=\sgn(a_{2}^{\st})$ and $|a_{2}^{\hcs}| > |a_{2}^{\st}|$---the white
  noise forcing diminishes the non-Gaussian character of the vdf, thus
decreasing $\abs{a_{2}}$. Then,
just after the second jump $\chi_{1}\to\chi_{f}$ at $t=t_{w}$, despite
having the ``correct'' kinetic temperature $T_{f}$, the system is
cooling slower (faster) than at the steady state when
$a_{2}^{\hcs}-a_{2}^{\st}$, or simply $a_{2}^{\st}$, is negative
(positive), i.e. for small (large) inelasticity.

For $t>t_{w}$, the discussion above entails the following. The kinetic
temperature $T(t)$ initially increases (decreases) and passes through
a maximum (minimum) before going back to $T_{f}$ when $a_{2}^{\st}<0$
($a_{2}^{\st}>0$), i.e. for small (large) inelasticity. Therefore, the
Kovacs hump is \textit{normal}, similar to that of molecular fluids, positive
and with only one maximum for small inelasticities, whereas the Kovacs hump turns out to be \textit{anomalous},
using the terminology introduced in
ref.~\cite{prados_kovacs-like_2014}, for large
inelasticities: negative with one minimum.
Figure~\ref{fig:kovacs-granular} shows two examples of the
aforementioned behaviours.
% Figure environment removed
In the granular gas, both the normal and the anomalous Kovacs effect
persist in the linear response regime~\cite{sanchez-rey_linear_2021}.

The Kovacs effect has also been investigated in a granular fluid of
rough particles. In addition to inelastic, collisions have a
certain degree of roughness, i.e. the tangential component of the
relative velocity is not conserved in collisions. This induces a coupling between the translational and rotational degrees of freedom.
More complex Kovacs
responses emerge, which may involve several
extrema~\cite{lasanta_emergence_2019}.

The linear response theory for molecular systems~\cite{prados_kovacs_2010} has been
generalised to athermal
systems~\cite{kursten_giant_2017,plata_kovacs-like_2017}. Specifically, the relation between the Kovacs hump and the direct
relaxation function remains valid, but the latter is not necessarily a
sum of positive modes. It is this fact that makes it possible the
emergence of the anomalous Kovacs effect, at least in linear
response~\cite{sanchez-rey_linear_2021}.

Finally, it is interesting to note that the Kovacs effect has also
been recently investigated in a variety of systems, such as active
matter~\cite{kursten_giant_2017}, disordered mechanical
systems~\cite{lahini_nonmonotonic_2017}, frictional
interfaces~\cite{dillavou_nonmonotonic_2018}, fluids with non-linear
drag~\cite{patron_strong_2021}, or a levitated colloidal
nanoparticle~\cite{militaru_kovacs_2021}.


\section{Mpemba effect}\label{sec:Mpemba}

To start with, we discuss the entropic (stochastic) Mpemba effect, triggered by the seminal Lu and Raz's
work~\cite{lu_nonequilibrium_2017}. A mesoscopic system is considered
and its time evolution is analysed in terms of the pdf  of the relevant variables, which obeys a
Markovian evolution equation (master equation, Fokker-Planck equation,
etc.) with detailed balance. Distance to equilibrium is defined in terms of a functional of
the pdf, e.g. the Kullback-Leibler divergence or other norms like the
$\mathcal{L}^{1}$ or $\mathcal{L}^{2}$ norms. By expanding the
solution of the evolution equation in the eigenfunctions of the
relevant operator, the entropic Mpemba effect is found when, under appropriate
conditions, the amplitude of the slowest relaxation mode presents a
non-monotonic dependence with the
temperature~\cite{lu_nonequilibrium_2017,klich_mpemba_2019,kumar_exponentially_2020,busiello_inducing_2021,chetrite_metastable_2021,holtzman_landau_2022,kumar_anomalous_2022,biswas_mpemba_2023}.
Also, a \textit{strong} Mpemba effect has been reported, which
arises when, by adequately choosing the system parameters, the
coefficient of the slowest relaxation mode vanishes and the relaxation
to equilibrium becomes exponentially faster
\cite{lu_nonequilibrium_2017,klich_mpemba_2019,kumar_exponentially_2020,kumar_anomalous_2022}.

In the thermal (kinetic) approach, started with Lasanta~\etal's analysis of a
granular gas~\cite{lasanta_when_2017}, a fluid is considered and its
time evolution is analysed in terms of the one-particle vdf,
which evolves following a kinetic equation---Boltzmann-Fokker-Planck,
typically. The relaxation to the steady state is monitored by the
kinetic temperature.  The kinetic
approach has been employed for both granular
fluids~\cite{lasanta_when_2017,torrente_large_2019,biswas_mpemba_2020,gomez_gonzalez_mpemba-like_2021,biswas_mpemba_2021,megias_mpemba-like_2022},
in which collisions between particles are inelastic, and molecular fluids with
elastic collisions but with a non-linear drag
force~\cite{santos_mpemba_2020,patron_strong_2021,megias_thermal_2022,megias_mpemba-like_2022}. The
former relaxes to a NESS that is characterised by the
intensity of the driving applied to balance, in average, the energy
dissipated in collisions; the latter relaxes to a true equilibrium
state with a Maxwellian vdf.

There are some key differences between the stochastic and kinetic approaches to the Mpemba effect. On the one hand, the monitored quantity in the kinetic approach, the kinetic
temperature is much closer to an
experimentally measurable quantity than the abstract distance between
distributions employed in the stochastic  approach. In
addition, the thermal Mpemba effect typically takes place for short
times, far away from the final state---which in principle makes it easier to be observed. On the other hand,  the initial conditions in the kinetic approach must
be non-stationary and thus, in principle, non-trivial to implement---although for non-linear fluids it has been discussed the aging procedure to obtain
these initial conditions, which correspond to a long-lived,
metastable, non-equilibrium state~\cite{patron_strong_2021}; whereas
the initial conditions for the entropic Mpemba effect in the stochastic  approach are  equilibrium states.\footnote{Initial stationary conditions have also
  been considered in the granular case, but an unrealistic asymmetric
  driving mechanism has to be introduced to trigger the Mpemba
  effect\cite{biswas_mpemba_2021,biswas_mpemba_2022}.}



Now we focus on the kinetic approach to the Mpemba
effect. Following Prados and Trizac's analysis of
the Kovacs effect in the same
system~\cite{prados_kovacs-like_2014,trizac_memory_2014}, the first
Sonine approximation was employed in ref.~\cite{lasanta_when_2017} to analyse the emergence of the Mpemba effect in a granular fluid by incorporating the non-Gaussianities
to the physical picture. In fact, it is the non-Gaussian character of
the vdf that makes the Mpemba effect possible. If the vdf were Gaussian, the kinetic temperature $T(t)$ would obey a closed first-order differential equation, without additional variables, and neither the Mpemba effect nor any other memory effect would emerge.


% Therein, the thermal Mpemba
% effect was defined as the crossing of the curves corresponding to the
% time evolution of the kinetic temperature of two samples, A (hot) and
% B (cold), of the granular gas: although the hot sample A has the
% larger initial temperature, $T_{i,A}>T_{i,B}$, it may relax faster to
% the NESS than the cold sample $B$, $T_{A}(t)<T_{B}(t)$ after a certain
% crossing time $t_{\times}$. 



% Figure environment removed
In fig.~\ref{fig:mpemba-granular}, specific examples of both the Mpemba effect, for $T_{i,A}>T_{i,B}>T_s$, and the inverse Mpemba effect, for $T_s>T_{i,A}>T_{i,B}$, are shown. The hot sample A is prepared in an initial state with kinetic
temperature $T_{i,A}$ and excess kurtosis $a_{2,i}^{A}$, and cools
down to a NESS corresponding to a certain value of the driving
$\chi_{\st}$ following the dynamical curve $T_{A}(t)$ (red solid line). The cold sample
is prepared in an initial state with kinetic temperature
$T_{i,B}<T_{i,A}$ and excess kurtosis $a_{2,i}^{B}$, and also cools
down to the same NESS following the dynamical curve $T_{B}(t)$ (blue dashed). Again, the key point is the
cooling rate  increasing with $a_{2}$: if
$a_{2,i}^{A}>a_{2,i}^{B}$, the difference of the initial cooling rates
may become large enough to facilitate the crossing of the
corresponding time evolutions $T_{A}(t)$ and $T_{B}(t)$---at least for small
enough kinetic temperature difference $\Delta T_{i}\equiv T_{i,A}-T_{i,B}$. As the initial states are not stationary states, the initial values of the kurtosis $a_{2,i}$ can be tuned to bring the Mpemba effect about.  


As
the difference of the initial kurtosis
$\Delta a_{2,i}\equiv a_{2,i}^{A}>a_{2,i}^{B}$ increases, the range of
initial temperatures $\Delta T_{i}\equiv T_{i,A}-T_{i,B}$ for which
the Mpemba effect is observed increases. Since the cooling rate depends on the inelasticity $\alpha$, the range of temperatures for which the Mpemba effect emerges depends on the inelasticity as well; decreasing with it and  vanishing in the elastic
limit $\alpha\to 1$.

The thermal Mpemba effect has also been investigated for a gas of inelastic
rough hard spheres~\cite{torrente_large_2019}. Therein, the
Mpemba effect is giant, much larger than in the smooth granular gas.
The initially hotter sample may cool sooner,
%than the colder one, 
even
when the initial temperatures differ by more than one order of
magnitude. The largeness of the memory effect stems from the
coupling between the translational and rotational temperatures, which
are of the same order---in the smooth case, the Mpemba
effect stemmed from the coupling with the (quite small) non-Gaussianities.

It is interesting to note that the Mpemba effect has also recently
been found in a  molecular fluid, in which the collisions between
particles are elastic, with non-linear
drag $\zeta(v)=\zeta_0 (1+ \gamma\,  m v^2/2 k_B T_s)$~\cite{santos_mpemba_2020,patron_strong_2021,megias_thermal_2022}. The non-linearity is measured by a dimensionless parameter $\gamma$, and the relevance of collisions by a dimensionless collision rate $\xi^{-1}$ ($\xi=\infty$ thus corresponds to the collisionless case.) The
kinetic temperature is not constant due to the interactions with the
thermal bath---modelled as a background fluid of particles with
comparable
mass~\cite{ferrari_particles_2007,ferrari_particles_2014,hohmann_individual_2017}. The
non-linearity of the drag implies that the evolution equation for
the temperature is coupled to higher-order cumulants of the vdf, bringing about the possible emergence of memory
effects.

One key question, unanswered in previous studies in the granular
case~\cite{lasanta_when_2017, torrente_large_2019}, is the aging
procedure that may give rise to the specific initial non-equilibrium
conditions chosen to make the Mpemba effect as large as
possible. Interestingly, it is possible to give an answer for the
non-linear fluid: the hot sample must be prepared by heating it from a
much lower temperature, whereas the cold sample must be prepared by
cooling it from a much higher temperature. The quench
from a very high temperature employed for the cold sample makes it
fall in a long-lived far-from-equilibrium state~\cite{patron_nonequilibrium_2023}, over which the kinetic
temperature follows a very slow, algebraic, decay to
equilibrium---which (i) increases the magnitude of the Mpemba effect and (ii) makes it universal, in the sense that the curves corresponding to different initial temperatures, non-linearity $\gamma$, and collision rate $\xi$ collapse onto a unique master curve upon a suitable rescaling, see fig.~\ref{fig:univ-mpemba-nonlin}~\cite{patron_strong_2021}.
% Figure environment removed












\section{Optimal control}\label{sec:opt-control}

What is the fastest relaxation route between two given states, either
equilibrium, NESSs, or arbitrary ones? In general,
this is the problem of the brachistochrone, which has recently been
addressed in different physical
contexts~\cite{deffner_quantum_2017,plata_finite-time_2020,lam_demonstration_2021,prados_optimizing_2021,ruiz-pino_optimal_2022,patron_thermal_2022,guery-odelin_driving_2023,aghion_thermodynamic_2023,pires_optimal_2023}. It
is tempting to relate this problem with the Mpemba effect, since the
relaxation from the initially further from equilibrium state
overtaking that of the initially closer may be interpreted as the
former finding a shortcut to the common final state.

The thermal brachistochrone has been recently investigated in
uniformly driven granular
fluids~\cite{prados_optimizing_2021,ruiz-pino_optimal_2022}. It refers to the minimum time connection by controlling the intensity of
the stochastic forcing $\chi$. The protocols minimising the
connection time between the initial and final NESSs corresponding to initial and final kinetic temperatures
$T_{i}$ and $T_{f}$ are of bang-bang
type, i.e. they comprise
different time intervals in which the thermostat alternates between its
maximum and minimum available values~\cite{liberzon_calculus_2012}. 

In the granular fluid, the time over the brachistochrone $t_{f}$
typically beats the experimental relaxation time $t_R$ by at least one order
of magnitude---see figure~\ref{fig:accel-factor}.  Remarkably, in the usual relaxation experiment with a sudden step at $t=0$, the relaxation is never complete
in a finite time---the empirical relaxation time is defined by
estimating that the system is close ``enough'' to the final
state. On the contrary, over the brachistochrone, the system reaches exactly the final state in a finite time. 

A  similar situation, with the thermal brachistochrone given by bang-bang protocols is found in Fokker-Planck systems~\cite{prados_optimizing_2021,patron_thermal_2022}. The case of coupled harmonic oscillators that are driven from an
initial equilibrium state at temperature $T_{i}$ to a final
equilibrium state at temperature
$T_{f}$ has been analysed in detail, and an unexpected discontinuity of the minimum connection time with increasing dimension has been unveiled~\cite{patron_thermal_2022}.
% Figure environment removed




\section{Discussion}\label{sec:discussion}

% SUMMARY OF MAIN FRAMEWORKS/RESULTS
We have reviewed the emergence of non-equilibrium memory effects,
mainly in granular fluids and fluids with non-linear drag. Despite
being quite different from a fundamental point of view---collisions
 in granular fluids are inelastic, so they are
intrinsically out-of-equilibrium systems with non-Gaussian vdfs even in the stationary state, both types of
systems display the Kovacs and the Mpemba memory effects. Still, one key
difference between granular and non-linear fluids is the emergence of
the anomalous Kovacs effect in the former. Even when a non-linear drag
is present, the Kovacs effect is always normal when the stationary state
corresponds to equilibrium and the dynamics verify detailed balance.

% NECESSITY OF ADDITIONAL VARIABLES!! CUMULANTS IN THE KINETIC APPROACH
The existence of these memory effects in the relaxation of the kinetic
temperature, proportional to the average kinetic energy, of granular and non-linear
fluids stems from its evolution being coupled to additional
variables, higher-order cumulants of the velocity that measure the
deviation of the vdf from the Gaussian shape. In
other words, the kinetic temperature does not suffice to univocally
determine the macroscopic state of the system. Hence, it is essential in general to keep track of the non-Gaussianities to understand the
non-equilibrium behaviour.

The thermal and entropic approaches to the Mpemba effect have been scarcely compared~\cite{megias_kinetic_2022,biswas_mpemba_2023}. In ref.~\cite{megias_kinetic_2022}, it was shown that the thermal Mpemba effect may appear without its entropic counterpart---or vice versa---in a molecular fluid with non-linear drag. Therein, some situations appear in which the kinetic temperature overshoots the stationary value, which makes it necessary to revise the usual definition of the thermal Mpemba effect in this scenario. The authors of ref.~\cite{megias_kinetic_2022} propose a separation of the Kullback-Leibler divergence into a ``kinetic'' contribution plus a ``local-equilibrium" distribution that allows for defining a non-equilibrium temperature,  not necessarily associated with the average kinetic temperature, for any system relaxing to equilibrium. It seems worth exploring if this line of thought could lead to a unique framework for the thermal and entropic Mpemba effects.

Much progress has been made in the understanding of these memory
effects. However, there is still room for further work in this
appealing line of research. One perspective is related to their optimal
control, 
% either for minimising the
% ``negative'' consequences of an (imperfect) preparation of the initial
% state---as in the Kovacs effect---or 
e.g. for optimising the ``positive''
consequences of a (tailored) preparation of the initial state---as in
the Mpemba effect. Therein, it seems also worth
investigating possible connections between the Mpemba effect and the
optimisation of the relaxation route to equilibrium, which has
attracted a lot of attention recently from different perspectives:
e.g. the impact of a precooling strategy~\cite{gal_precooling_2020} or the
possible asymmetry between heating and
cooling~\cite{lapolla_faster_2020,van_vu_toward_2021,ibanez_heating_2023}. 












% Here it is a citation~\cite{gomez_gonzalez_mpemba-like_2021}

%% here a revision

% \revision{Insert here the text.
% See fig.~\ref{fig.1}, table~\ref{tab.1} and eq.~(\ref{eq.1}).
% }

% here a shortcut $\emc$ and again $\emc$


% \begin{equation}
% \label{eq.1}
% 0\neq1
% \end{equation}


% % Figure environment removed


% \begin{table}
% \caption{Table caption.}
% \label{tab.1}
% \begin{center}
% \begin{tabular}{lcr}
% first  & table & row\\
% second & table & row
% \end{tabular}
% \end{center}
% \end{table}



\acknowledgments
We acknowledge financial support from Grant PID2021-122588NB-I00
funded by MCIN/AEI/10.13039/ 501100011033/ and by ``ERDF A way of
making Europe'', and also from  Grant
ProyExcel\_00796 funded by Junta de Andalucía's PAIDI 2020 programme. A. Patr\'on acknowledges support from the FPU programme through Grant FPU2019-4110. C. A. Plata acknowledges the funding received from EU Horizon Europe--Marie Sk\l{}odowska-Curie 2021 programme through the Postdoctoral Fellowship with Ref.~101065902 (ORION). We are indebted with all the people with whom we have collaborated in
this exciting field of memory effects.

\bibliography{EPL-Perspective-memory-effects,Mi-biblioteca-19-jun-2023}

% \begin{thebibliography}{0}

% \bibitem{b.a}
%   \Name{Author F., Author S. \and Author T.}
%   \REVIEW{Some Rev. A}{69}{1969}{9691}.

% \bibitem{b.b}
%   \Name{Author F. \and Author S.}
%   \Book{Some Book of Interest}
%   \Editor{A. Editor}
%   \Vol{9}
%   \Publ{Publishing house, City}
%   \Year{1939}
%   \Page{666}.

% \bibitem{b.c}
%   \Editor{Editor A.}
%   \Book{Some Book of Interest}
%   \Vol{9}
%   \Publ{Publishing house, City}
%   \Year{1939}
%   \Section{A}.

% \end{thebibliography}

\end{document}

