\documentclass[12pt]{article}

\usepackage{setspace}
%\doublespacing
%\usepackage{a4wide}

\usepackage{amsthm, amssymb, amstext}
\usepackage[fleqn]{amsmath}
\usepackage{latexsym}
\usepackage[dvips]{graphicx}
\usepackage{comment}
\usepackage{hyperref}
\usepackage{mathtools}
\usepackage{enumerate} 

\usepackage{todonotes}
\usepackage{comment}
\newcommand{\todocf}[1]{\todo[size=\tiny, color=blue!10]{CF: #1}}
\newcommand{\todomj}[1]{\todo[size=\tiny, color=blue!10]{MJ: #1}}


\def\indr{\mc{I}^{(r)}(G)}
\def\indrm{\mc{I}^{(r-1)}(G)}
\def\del{\!\downarrow\!}
\def\A{\mathcal{A}}
\def\B{\mathcal{B}}
\def\C{\mathcal{C}}
\def\D{\mathcal{D}}
\def\E{\mathcal{E}}
\def\F{\mathcal{F}}
\def\I{\mathcal{I}}
\def\L{\mathcal{L}}
\def\M{\mathcal{M}}
\def\N{\mathcal{N}}
\def\S{\mathcal{S}}
\def\R{\mathcal{R}}
\def\H{\mathcal{H}}
\def\mc{\mathcal}

\newtheorem{theorem}{Theorem}
\newtheorem{lemma}{Lemma}
\newtheorem{proposition}{Proposition}
\newtheorem{observation}{Observation}
\newtheorem{corollary}{Corollary}
\newtheorem{conjecture}{Conjecture}
\newtheorem{property}{Property}
\newtheorem{claim}{Claim}
\newtheorem{problem}{Problem}

\theoremstyle{definition}
\newtheorem{definition}{Definition}[section]
\newtheorem{remark}{Remark}

\usepackage{amsthm}




\title{Three remarks on $\mathbf{W_2}$ graphs}
\author{Carl Feghali\thanks{Univ Lyon, EnsL, CNRS, LIP, F-69342, Lyon Cedex 07, France, email: \texttt{carl.feghali@ens-lyon.fr}} \\ \and Malory Marin\thanks{École Normale Supérieure de Lyon, LIP, F-69342, Lyon Cedex 07, France, email : \texttt{malory.marin@ens-lyon.fr}}}

\date{}

\begin{document}
\maketitle

\begin{abstract}
Let $k \geq 1$. A graph $G$ is $\mathbf{W_k}$ if for any $k$ pairwise disjoint independent vertex subsets $A_1, \dots, A_k$ in $G$, there exist $k$ pairwise disjoint maximum independent sets $S_1, \dots, S_k$ in $G$ such that $A_i \subseteq S_i$ for $i \in [k]$. Recognizing $\mathbf{W_1}$ graphs is co-NP-hard, as shown by Chv\'atal and Hartnell (1993) and, independently, by Sankaranarayana and Stewart (1992). Extending this result and answering a recent question of Levit and Tankus, we show that recognizing $\mathbf{W_k}$ graphs is co-NP-hard for $k \geq 2$. On the positive side, we show that recognizing $\mathbf{W_k}$ graphs is, for each $k\geq 2$, FPT parameterized by clique-width and by tree-width. Finally,  we construct graphs $G$ that are not $\mathbf{W_2}$ such that, for every vertex $v$ in $G$ and every maximal independent set $S$ in $G - N[v]$, the largest independent set in $N(v) \setminus S$ consists of a single vertex, thereby refuting a conjecture of Levit and Tankus. 
 \end{abstract}


For a positive integer $k$, a graph $G$ is $\mathbf{W_k}$ if for any $k$ pairwise disjoint independent vertex subsets $A_1, \dots, A_k$ in $G$, there exist $k$ pairwise disjoint maximum independent sets $S_1, \dots, S_k$ in $G$ such that $A_i \subseteq S_i$ for $i \in [k]$. A graph that is $\mathbf{W_1}$ is also commonly called \emph{well-covered} and has been extensively studied; see \cite{brown, chvatal, finbow, plummer,  sank} for some examples. 

Chv\'atal and Hartnell \cite{chvatal} and, independently, Sankaranarayana and Stewart \cite{sank} demonstrated that recognizing well-covered (or $\mathbf{W_1}$) graphs is co-NP-complete. However, the complexity of recognizing $\mathbf{W_k}$ graphs for each $k \geq 2$ has remained open, despite this class of graphs being introduced around the same time as the class of well-covered graphs (see the survey by Plummer \cite{plummer} for some references), and this was raised explicitly by Levit and Tankus \cite{levit}. It is worthwhile noting that a $\mathbf{W_2}$ graph without isolated vertices is $1$-well-covered, a class of graphs that has received considerable attention; see for example \cite{hartnell, levit1}. Our first contribution is to answer this question. 

\begin{theorem}
For each $k \geq 2$, recognizing $\mathbf{W_k}$ graphs is co-NP-hard. 
\end{theorem}

\begin{proof}
We shall construct a graph $H$ from an arbitrary graph $G$ in polynomial-time such that $G$ is $\mathbf{W_1}$ if and only if $H$ is $\mathbf{W_k}$. Since recognizing $\mathbf{W_1}$ graphs is co-NP-hard \cite{chvatal, sank}, this will imply the theorem. 

The graph $H$ is obtained from $G$ by replacing every vertex $v$ of $G$ by a clique $K^v$ of size $k$ and whenever there is an edge joining two vertices $u$ and $w$ in $G$, we add all possible edges between $K^u$ and $K^w$ in $H$. 

Suppose first that $G$ is $\mathbf{W_1}$. Let $A_1, \dots, A_k$ be disjoint independent vertex subsets of $H$. We must show that there exist $k$ pairwise disjoint maximum independent sets $S_1, \dots, S_k$ in $H$ such that $A_i \subseteq S_i$ for $i \in [k]$. For each vertex $v$ of $G$, let $v_1, \dots, v_k$ denote the vertices of $K^v$ in $H$, and for each independent set $I$ in $H$ and index $j \in [k]$, let $f_j(I) = \{v_j: v_i \in I \mbox{ for some } v \in V(G) \mbox{ and } i \in [k]\}$. By construction, $f_j(I)$ is independent. Since the subgraph $H_j$ of $H$ induced by $\{v_j: v \in V(G)\}$ is isomorphic to $G$ and hence is $\mathbf{W_1}$, it follows that $f_j(I)$ is contained in a maximum independent of $H_j$. Thus, we can let $T_1, \dots, T_k$ be (not necessarily disjoint) maximum independent sets in $H_1$ containing, respectively, $f_1(A_1), \dots, f_1(A_k)$. Then the $k$ subsets $A_1 \cup f_1(T_1 \setminus A_1), A_2 \cup f_2(T_2 \setminus A_2), \dots, A_k \cup f_k(T_k \setminus A_k)$ are disjoint maximum independent sets in $H$, as needed. 

Conversely, suppose $H$ is $W_k$. Let $A$ be an independent set of $G$. For each independent set $I$ of $H$, let $g(I) = \{v: v_i \in I \mbox{ for some } v \in V(G) \mbox{ and } i \in [k]\}$. Since $H$ is $\mathbf{W_k}$, the independent set $\{a_1: a \in A\}$ is contained in a maximum independent set $J$ of $H$. Then $A$ is contained in the maximum independent set $g(J)$. This completes the proof.   
\end{proof}



An instance of a \emph{parameterized problem} is a pair $(x, k)$, where $x$ is a string encoding the input and $k$ is a parameter. A parameterized problem is said to be \emph{fixed-parameter-tractable} (FPT) if it can be solved in $f(k) \cdot n^{O(1)}$ for some computable function $f$. 

In the 90's, Courcelle proved that if a graph problem can be formulated in \textit{monadic second-order logic} (MSO${}_{1}$), then it is FPT when parameterized by clique-width \cite{Courcelle1,Courcelle2,Courcelle3,Courcelle5,Courcelle6}. 
Furthermore, Courcelle \cite{Courcelle4} showed that every graph problem definable in LinEMSOL (an extension of MSO${}_1$ with the possibility of optimization with respect to some linear evaluation function) is also FPT when parameterized by clique-width. 


The latter enabled Alves \emph{et al.} \cite{Alves} to show that recognizing well-covered graphs is FPT when parameterized by clique-width. In our next contribution, we extend this result by showing that recognizing $\mathbf{W_k}$ graphs is, for each $k\geq 2$, FPT when parameterized by clique-width (here we need not supply the technical definitions of clique-width and MSO${}_{1}$).
\begin{theorem}
For each $k\geq 2$, recognizing $\mathbf{W_k}$ graphs is FPT when parameterized by clique-width.
\end{theorem}

\begin{proof}
To prove the theorem, we express the problem in LinEMSOL and then apply Courcelle's theorem. In order to achieve this, we define the formula
\begin{equation*}
\mathbf{Wk} = \quad ~\forall_{X_1,\dots,X_k\subseteq V}
 ~\mathbf{indep}_k(X_1,\dots X_k)
 \end{equation*}
 which checks if $X_1, \dots, X_k$ are pairwise disjoint independent vertex subsets of $G$, and if so, whether there exist $k$ pairwise disjoint maximal independent sets $X'_1, \dots, X'_k$ such that $X_i \subseteq X'_i$ for $i \in [k]$. Here
\begin{align*}
\mathbf{indep}_k(X_1,\dots,X_k) &=  \left(\bigwedge_{1\leq i \leq k} \mathbf{indep}(X_i) ~\wedge \bigwedge_{1\leq i < j \leq k}\textbf{disjoint}(X_i,X_k)\right) \\ \\
& \Longrightarrow \left(\exists_{X_1',\dots, X_k'\subseteq V} \bigwedge_{1\leq i \leq k} \mathbf{subset}(X_i,X_i') \wedge ~\mathbf{maximal}(X_i') \right.\\
&\qquad \left.\wedge \bigwedge_{1\leq i <j \leq k} \mathbf{disjoint}(X_i',X_j') \right)
\end{align*}
 where $\mathbf{indep}(X) = \quad \forall_{u,v\in X} ~\neg \mathbf{adj}(u,v)$ checks whether $X$ is independent, 
$\textbf{maximal}(X) = \quad \mathbf{indep}(X) \wedge \forall_{v\notin X} \exists_{v\in X} \mathbf{adj}(u,v)$ checks whether $X$ is both maximal and independent,  
$\textbf{disjoint}(X,Y) = \quad \forall_{u\in X} \forall_{v\in Y} ~(u\neq v)$ checks whether $X$ and $Y$ are disjoint and, lastly, 
$\textbf{subset}(X,X') = \quad \forall_{x\in X} ~(x\in X')$ checks whether $X$ is a subset of $X'$. We have the following claim. 




\begin{claim}\label{claim}
A graph $G$ is $\mathbf{W_k}$ if, and only if, $G$ is well-covered and $G\models  \mathbf{Wk}$.
\end{claim}

\begin{proof}
Suppose first that $G$ is $\mathbf{W_k}$. Then, by definition, $G$ is well-covered. Let $X_1,...,X_k$ be pairwise disjoint independent vertex subsets of $G$. Since $G$ is $\mathbf{W_k}$, there exist pairwise disjoint maximal (in fact, maximum) independent vertex subsets $X_1',\dots X_k'\subseteq V$ such that $X_i \subseteq X_i'$ for $i \in [k]$. Thus $G\models \mathbf{Wk}$.

Conversely, suppose that $G$ is well-covered and $G\models \mathbf{Wk}$. Let $X_1,\dots,X_k$ be $k$ pairwise disjoint independent vertex subsets in $G$. Since $G\models \mathbf{Wk}$, there are $k$ pairwise disjoint maximal independent sets $X_1',\dots,X_k'$ such that  $X_i \subseteq X_i'$ for $i \in [k]$. But $G$ being well-covered implies each $X_i'$ is also maximum and thus $G$ is $\mathbf{W_k}$ as needed. 
\end{proof}

To complete the proof: since the property of being a maximal independent set is MSO${}_1$ by our formulas, given an $n$-vertex graph $G$ of clique-width $w$, we can compute the size $\beta(G)$ of the smallest maximal independent set and the size $\alpha(G)$ of the largest maximal independent set in time $f_1(w).n^{\mathcal{O}(1)}$ by Courcelle's theorem on LinEMSOL logic, where $f_1$ is a computable function. In particular, since  a graph $G$ is well-covered iff $\alpha(G) = \beta(G)$, deciding whether $G$ is well-covered can thus be done in time $f_1(w).n^{\mathcal{O}(1)}$. 



On the other hand, we can decide if $G\models \mathbf{Wk}$ in time $f_2(w).n^{O(1)}$ using Courcelle's theorem on MSO${}_1$ logic, where $f_2$ is a computable function. %Note that the size of $\mathbf{Wk}$ increases with $k$, and therefore $f_2$ also increases with $k$.

Therefore, together with Claim \ref{claim}, there is an algorithm that decides if $G$ is $\mathbf{W_k}$ in time $f(w).n^{O(1)}$, where $f = f_1 + f_2$ completing the proof. 
\end{proof}

Note that Courcelle's first theorem was about MSO${}_2$ logic (which allows quantification over edges) and about tree-width instead of clique-width \cite{Courcelle3}. However, since MSO${}_2$ logic is an extension of MSO${}_1$, the following also holds.

\begin{corollary}
For each $k \geq 2$, recognizing $\mathbf{W_k}$ graphs is FPT when parameterized by tree-width.
\end{corollary}

Our final contribution is to refute a conjecture of Levit and Tankus concerning a characterization of $\mathbf{W_2}$ graphs. The following theorem was proved in \cite{levit}. 

\begin{theorem}
A well-covered graph $G$ is $\mathbf{W_2}$ if and only if for every vertex $v$ in $G$ and every maximal independent set $S$ in $G - N[v]$, the largest independent set in $N(v) \setminus S$ consists of a single vertex.
\end{theorem}

In their same paper, they conjectured that the same statement holds without the assumption that $G$ is well-covered. Unfortunately, this turns out to be false. Let $s$ and $t$ be any positive integers, $s < t$. Let $A$ be the disjoint union of $s$ edges and $B$ be the disjoint union of $t$ edges. Then the graph $G$ formed from $A$ and $B$ by adding all possible edges between $A$ and $B$ serves as a counterexample. 




 \bigskip



\noindent
\textbf{Acknowledgements.} Carl Feghali was supported by the French National Research Agency under research grant ANR DIGRAPHS ANR-19-CE48-0013-01









 
 
  
 \bibliography{bibliography}{}
\bibliographystyle{abbrv}
 
\end{document}


  