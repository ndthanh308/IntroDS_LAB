\vspace{-.2cm}
\section{Preliminary Observations}
\label{results}
% % Figure environment removed



In this section, we discuss our observations for the moral principles as exhibited in the GitHub heated issues. %Next, we try to understand 
%\subsection{Moral Principles in GitHub Issues}
%With definitions, we do not mean a new, enhanced version of Moral Foundation Theory. But rather, we intend to see how we can interpret moral values in the context of OSS. 
%SE communications are significantly different than other types of communications seen on other social platforms such as Twitter, Reddit, etc. Therefore, it is necessary for us to see how we can adapt the original definitions of morality to SE communications, especially the language used in \textit{toxic} SE communications as we are trying to understand the connection between toxicity and morality. We use the key concepts of each morality type to see if we can observe any patterns in the dataset. Then, we use these patterns to come up with SE-adapted definitions of morality. 
%In this section, we explain these patterns for the 5 moral foundations. It should be noted that detecting morality even in social contexts such as Twitter is a hard task, mostly in the sense of distinguishing types of morality from each other. Because of this very reason, Hoover et. al. \cite{hooverMoral2020} allow for overlapping labels (expressions of moral sentiment that are associated with multiple foundations) during the annotation process for morality detection in the Twitter corpus. We did our best to come up with clear definitions for each type of morality in the domain of SE.
We analyze the issue comment threads in our data sample to identify instances where moral principles were expressed in the text.
% Figure \ref{OSSMoralExample} shows examples of moral values as exhibited in GitHub.
Out of 695 issue comments across 100 threads, 135 exhibit at least one form of moral principle.
%A total of 695 issue comments (in 100 issue threads) were analyzed, and from those, 135 comments exhibit at least one type of morality in them.
% XX 
%In order to better clarify our observations of different patterns of conversations in these issue threads, 
In this analysis, %it is important to note that we refer to the 
people who open the issue threads to find fixes for problems are referred to as \textit{users}, and people who try to resolve the issues and close the threads as \textit{contributors}. %Users are the people who are seeking help from contributors to add new features or fix the bugs and problems that they are facing when using the products. Upon identifying these roles, 
We observed unique patterns specific to each of the identified roles.

% \textcolor{red}{@Ramtin - ADD HERE A PARA ABOUT THE DIFFERENT ROLES: I am guessing we have developers, maintainers, and users? We need to clearly define these roles.}
% In issue comments, users usually either seek help from contributors to fix the bugs and problems that they are facing when using the products, or they are asking for new features.

\noindent
\textbf{Care/Harm.}
%Based on the original definition, kindness, gentleness, and nurturance are counted as virtues in this foundation. 
A total of 48 comments exhibit this principle.
Pertaining to this category, following are the behaviors we observed: \textbf{(CH1) }users who use derogatory language or insults towards contributors while seeking assistance or requesting new features; \textbf{(CH2) }users insulting contributors mainly by trolling, possibly due to differences in ideology, opinions, etc.; \textbf{(CH3) }contributors responding using insults toward users.
% XX
% \begin{enumerate}[leftmargin=*]
%     \item[] \textbf{CH1: }users who use derogatory language or insults towards contributors while seeking assistance or requesting new features.
%     \item[] \textbf{CH2: }users insulting contributors mainly by trolling, possibly due to differences in ideology, opinions, etc.
%     \item[] \textbf{CH3: }contributors responding using insults toward users.
% \end{enumerate}

The first type (CH1) was the most common in our dataset (32 out of 48 instances). For example, in one thread, a user opens an issue saying that the name of the project is duplicated and it should change while using insults: \textit{"This name is in use and should be changed immediately. Yes, I have seen the other issues. Yes, I am opening a new one because f**k you Microsoft. You are merely trying to cast a shadow on other, truly open source projects..."}.

The second type (CH2) was also seen in a number of cases (5 out of 48 instances). For example, in one instance, a user opens an issue saying: \textit{"Revenue. F**k you guys"}, and does not add anything else to the thread. Threads similar to this are mostly the result of trolling, but they can also be the result of differences in ideology or based on previous interactions~\cite{Miller2022}.

For the third type (CH3), we saw an interesting pattern of contributors responding in a detrimental manner towards the users. This type is rare compared to other types in this category.
% In one interesting instance, a user opened up an issue explaining the need for making a variable configurable: \textit{"...Make boinc auth a configurable value in the community grid role to enable users to allocate the WU points to an account that they choose..."}. The conversation goes on to the point where the user is explaining the ethical implications of this choice that developers have made \textit{"...It is also not disrespectful or disingenuous to debate the ethical implications of such a decision. Some seem to interpret such a discussion by default as an attack on them personally..."}.
% However, the project member continues the conversion with this comment: \textit{"...I will wait for others like @cloin and @liquidat to weigh in on this. Just because you THINK one way doesn't mean that others THINK another way..."}. The conversation even escalates to the point where the developer says: \textit{"...I would have hoped that all IBMers and Red Hatters, our partners and community would be happy we are trying. Sorry that we tried to contribute to make the world a better place..."}.
% This conversation is not as severe as the other instances but still falls under the harm category nonetheless.
In one instance, a user opens an issue asking: 
\textit{"Where to f**k python2? Why, when I give the brew install python command, python3 is installed, not python2. Developers are you stoned there?"},
and surprisingly, one of the contributors responds with: 
% and interestingly, one of the contributors responds with: 
\textit{"I recommend not having intercourse with EOL software. Python 2 is no longer included in Homebrew"}. The user then responds: \textit{"You have an unfinished raw product with some problems. What can be your attitude? Treating users like sh*t"}. This issue thread exhibits the first and last patterns (CH1 and CH3).
%In another issue thread, a user suggests adding a framework to a list in the project: \textit{"I think iris should be on list too"}. However, one of the developers responds by saying: \textit{"...would honestly prefer to write a web application in assembly language before using anything written by @USER. He is a cancer to open source"}.
%All instances resembling a pattern of communication like the mentioned examples are attributed to the \texttt{Harm} foundation. 

\noindent
\textbf{Fairness/Cheating.}
%This foundation captures the ideas of justice, rights, and autonomy. Based on this definition, discrimination, and unfairness count as vices for this foundation. Discrimination can fall into different types such as nationality, culture, gender, LGBTQ+, religion, age, and race \cite{vanbreukelen2023still, Sarker2023Mitigate}. As we were going through our dataset, we did not observe these types of discrimination. Based on the results of the study done by Cheriyan et. al. \cite{cheriyanTowards2021}, racial offense is a rare type of offense in the GitHub community when compared to swearing and personal offenses. However, this does not mean that they do not exist. Therefore, for this type of foundation, we decided to look at data outside of our dataset to find instances of discrimination in GitHub.
We observed behaviors associated with this category in a total of 19 comments, which include: \textbf{(FC1) }users having unrealistic expectations from the contributors; \textbf{(FC2) }contributors failing to address user's issues due to unjustified reasons.
% \begin{enumerate}[leftmargin=*]
%     \item[] \textbf{FC1: }users having unrealistic expectations from the contributors.
%     \item[] \textbf{FC2: }contributors failing to address user's issues due to unjustified reasons.
%     %\item discrimination (based on nationality, culture, gender, LGBTQ+, religion, age, race, etc)~\cite{vanbreukelen2023still, cheriyanTowards2021, Sarker2023Mitigate}.
% \end{enumerate}

Among the instances of this category, 7 out of 19 instances exhibit the first type of communication pattern. In one of the instances representing the first type (FC1), the user has opened an issue asking about a problem in the debugger. Contributors of the projects try to help the user by providing tips on how the problem can be solved, but as the conversation goes on, the user gets frustrated and says: \textit{"@<USER1> @<USER2> Too busy to respond?"}, which in the end, leads to the issue thread being locked. This is an example of unrealistic expectations about  response time causing conflict.

For the second type (FC2), %in one instance the user has opened an issue explaining the project's name is in conflict with other projects: \textit{"Name clash with Maui Linux and MauiKit"}. One of the contributors responds sarcastically: \textit{"Perhaps the Linux project should change its name as they've conflicted with a city in Hawaii which existed long before"}. As the conversation goes on, the contributor states that: \textit{"You can be mad all you want, but let's be realistic here... this project you're fighting for so passionately, doesn't have as many stars as I have thumbs down for telling you that you're being ridiculous"}, which is an obvious resemblance of cheating morality.Or in another instance, 
for instance, this is how one of the contributors responds to a  user's issue: \textit{"How would be this a priority, the app must be ready to be shipped in a few days and basically, no one uses the smartphone in landscape mode except for watching media"}.

Another type of behavior related to \textit{Fairness} would be discrimination (based on race, culture, gender, religion, etc). In our dataset, we did not observe any instances of discrimination. Analysis of additional data may reveal discriminatory behavior in OSS, as such occurrences have been observed in previous studies~\cite{vanbreukelen2023still, cheriyanTowards2021}.
%And finally, for the final type, by looking at Cheriyan et. al. \cite{cheriyanTowards2021}'s annotated dataset, we found an issue thread with severe use of racial slurs in it. In this thread, a user posted this comment: \textit{"instead of showing a list of the reagents and picking them to switch to them it now requires you to press the item to switch between the reagents over and over and over Jesus Fucking Christ Ni**as"}. It should be noted that by analyzing more data, we should be able to find instances of religious, gender, and other types of discrimination in SE communications as well, as they have been observed and studied before. %These types of instances, if observed, fall into cheating morality as well.

\noindent
\textbf{Loyalty/Betrayal.}
%The key concepts in this foundation are patriotism and self-sacrifice for the group, or as the saying goes: "one for all, and all for one". The vices for this foundation boil down to the people who rebel against the community and betray the community and its people.
The observed patterns for this category of morality were found in 23 instances of the dataset: \textbf{(LB1) }users actively promoting and inciting rebellion against a project, encouraging others to switch to an alternative project (most common with 15 instances); \textbf{(LB2:) }contributors excluding users from the project.
% \begin{enumerate}[leftmargin=*]
%     \item[] \textbf{LB1: }users actively promoting and inciting rebellion against a project, encouraging others to switch to an alternative project (most common with 15 instances).
%     \item[] \textbf{LB2: }contributors excluding users from the project.
% \end{enumerate}

To better understand these behaviors, we provide examples for each type. For instance, in one of the issue threads, a user claims that SSL Insecure is not being respected in the project: \textit{"Whoever will find this issue and gets pissed off, because the author doesn't bother to fix it for years, ditch mitmproxy and use SSL SPLIT"}. One of the developers responds with: \textit{"I'm glad you have found something that works for you!"}. In this instance, both LB1 and LB2 were observed, because the user is encouraging people to leave this project behind, and the contributor also encourages the user to not use the project.
In another instance, a user requests the removal of the slur filter in the app, however, one of the contributors says: \textit{"If you don't like it, fork it. Stop bothering us about it, we will never fully remove the slur filter"}, which counts as excluding the user from the community.

% % Figure environment removed

% Figure environment removed

\noindent
\textbf{Authority/Subversion.}
OSS communities, like any other community, often have specific rules and guidelines that everyone has to follow. GitHub projects use the Code of Conduct (CoC) in an attempt to promote their expectations and standards of ethical behavior within the community \cite{touraniCode2017}. %When rules and hierarchies are defined in OSS communities, it becomes natural that moralities such as authority and subversion are at play in OSS communications.
The establishment of such rules and hierarchies often leads to the dynamic interplay of moralities such as authority and subversion in OSS communications. A total of 30 instances with this principle were found in our dataset.
The patterns observed for this category are: \textbf{(AS1) }users trying to rebel against authority, and questioning developers' ability to lead the community. \textbf{(AS2) }contributors (authorities) enforcing Code of Conduct when thought necessary or any other admin-privileged acts to ban, censor, or silence users (most common with 16 instances).
% \begin{enumerate}[leftmargin=*]
%     \item[] \textbf{AS1: }users trying to rebel against authority, and questioning developers' ability to lead the community
%     \item[] \textbf{AS2: }contributors (authorities) enforcing Code of Conduct when thought necessary or any other admin-privileged acts to ban, censor, or silence users (most common with 16 instances).
% \end{enumerate}

In one of the instances, a user has opened an issue, claiming that the project has stolen his code: \textit{"Don’t be a d*ck as the source code was stolen at that point and implemented on this"}.
The contributor enforces authority by saying: \textit{"please be civilized and refrain from profanities as required by our Code of Conduct, or I'll ban you from all Falconry projects. This is the last warning"}.
And the user challenges the authority by saying: \textit{"I’m not a bot but if you wish to ignore me go ahead and I’ll get a lawyer"}.
%In another interesting instance, a user explains in a civilized manner that he thinks the release of "Pyflakes" should be urgent, contrary to what the contributor believes: \textit{"Would be great to get this released ASAP...I disagree. pyflakes is currently incompatible with (arguably) the number 1 headline feature of the latest version python...I therefore think a release is very urgent...pyflakes is great and I know how thankless maintaining OS libraries like this can be. Thank you all for your hard work"}. But the contributor tags the comment with the tag "off-topic", which leads to the user saying: \textit{"Please could someone explain why my comment was marked as "off-topic"? You may disagree with it, but it is absolutely not off-topic - it relates directly to the the subject of this issue."}. Another user also proceeds to say: \textit{"it's very disappointing to see this sort of behavior on a project this important. You and the other maintainers should strongly consider adding a code of conduct to this repo and following it."}.
Additionally, there were instances where the users questioned the capabilities of contributors. For example, a user opened an issue asking about how to do a certain thing within the system. As the conversation goes on, the user responds to one of the contributors: \textit{"...I would hate to have you in charge of any security issues"}.

\noindent
\textbf{Sanctity/Degradation.}
%Based on the definitions, "disgust" is the key concept of this type of morality. 
The most commonly observed pattern in this category, in a total of 42 instances, was observed when: (\textbf{SD1}) users express their hate (disgust) toward a certain package, system, or code that the project is using, most likely due to  their personal preferences. This principle is targeted at the code/system rather than people (targeting people represents \textit{Care/Harm}).

For example, a user opened an issue in the project expressing how much the document is poorly written, and in a sense, he is \textit{disgusted} with it: \textit{"...in fact, you write a sh*t doc. I'm a real man, it's my feeling of your holy sh*t doc"}. 
Or in another instance, a user opens an issue explaining his problem with the software: \textit{"...I just tried reinstalling you buggy, sh*tty software for the third time..."}.



%In another example, a user has opened an issue after a new release of the software, saying: \textit{"did this fix the annoying pulse audio bullsh*t?"}.