%%
%% This is file `sample-manuscript.tex',
%% generated with the docstrip utility.
%%
%% The original source files were:
%%
%% samples.dtx  (with options: `manuscript')
%% 
%% IMPORTANT NOTICE:
%% 
%% For the copyright see the source file.
%% 
%% Any modified versions of this file must be renamed
%% with new filenames distinct from sample-manuscript.tex.
%% 
%% For distribution of the original source see the terms
%% for copying and modification in the file samples.dtx.
%% 
%% This generated file may be distributed as long as the
%% original source files, as listed above, are part of the
%% same distribution. (The sources need not necessarily be
%% in the same archive or directory.)
%%
%% Commands for TeXCount
%TC:macro \cite [option:text,text]
%TC:macro \citep [option:text,text]
%TC:macro \citet [option:text,text]
%TC:envir table 0 1
%TC:envir table* 0 1
%TC:envir tabular [ignore] word
%TC:envir displaymath 0 word
%TC:envir math 0 word
%TC:envir comment 0 0
%%
%%
%% The first command in your LaTeX source must be the \documentclass command.
\documentclass[sigconf]{acmart}
\usepackage{multirow}
\usepackage{tabularx}
\usepackage{enumitem}

%PUT YOUR MACROS HERE
\newcommand{\linguistic}[1]{{\textit{baseline+psycholinguistic}}}
\newcommand{\morality}[1]{{\textit{baseline+psycholinguistic+morality}}}

\acmConference[ESEC/FSE 2023]{The 31st ACM Joint European Software Engineering Conference and Symposium on the Foundations of Software Engineering}{11 - 17 November, 2023}{San Francisco, USA}

%%
%% \BibTeX command to typeset BibTeX logo in the docs
\AtBeginDocument{%
  \providecommand\BibTeX{{%
    \normalfont B\kern-0.5em{\scshape i\kern-0.25em b}\kern-0.8em\TeX}}}

%% Rights management information.  This information is sent to you
%% when you complete the rights form.  These commands have SAMPLE
%% values in them; it is your responsibility as an author to replace
%% the commands and values with those provided to you when you
%% complete the rights form.
\setcopyright{acmcopyright}
\copyrightyear{2023}
\acmYear{2023}
\acmDOI{XXXXXXX.XXXXXXX}

%% These commands are for a PROCEEDINGS abstract or paper.

\acmPrice{15.00}
\acmISBN{978-1-4503-XXXX-X/18/06}

%%
%% Submission ID.
%% Use this when submitting an article to a sponsored event. You'll
%% receive a unique submission ID from the organizers
%% of the event, and this ID should be used as the parameter to this command.
%%\acmSubmissionID{123-A56-BU3}

%%
%% For managing citations, it is recommended to use bibliography
%% files in BibTeX format.
%%
%% You can then either use BibTeX with the ACM-Reference-Format style,
%% or BibLaTeX with the acmnumeric or acmauthoryear sytles, that include
%% support for advanced citation of software artefact from the
%% biblatex-software package, also separately available on CTAN.
%%
%% Look at the sample-*-biblatex.tex files for templates showcasing
%% the biblatex styles.
%%

%%
%% The majority of ACM publications use numbered citations and
%% references.  The command \citestyle{authoryear} switches to the
%% "author year" style.
%%
%% If you are preparing content for an event
%% sponsored by ACM SIGGRAPH, you must use the "author year" style of
%% citations and references.
%% Uncommenting
%% the next command will enable that style.
%%\citestyle{acmauthoryear}

%%
%% end of the preamble, start of the body of the document source.
\begin{document}


%%
%% The "title" command has an optional parameter,
%% allowing the author to define a "short title" to be used in page headers.
\title{Exploring Moral Principles Exhibited in OSS: A Case Study on GitHub Heated Issues}

%%
%% The "author" command and its associated commands are used to define
%% the authors and their affiliations.
%% Of note is the shared affiliation of the first two authors, and the
%% "authornote" and "authornotemark" commands
%% used to denote shared contribution to the research.
% \author{Ben Trovato}
% \authornote{Both authors contributed equally to this research.}
% \email{trovato@corporation.com}
% \orcid{1234-5678-9012}
% \author{G.K.M. Tobin}
% \authornotemark[1]
% \email{webmaster@marysville-ohio.com}
% \affiliation{%
%   \institution{Institute for Clarity in Documentation}
%   \streetaddress{P.O. Box 1212}
%   \city{Dublin}
%   \state{Ohio}
%   \country{USA}
%   \postcode{43017-6221}
% }

% \author{Lars Th{\o}rv{\"a}ld}
% \affiliation{%
%   \institution{The Th{\o}rv{\"a}ld Group}
%   \streetaddress{1 Th{\o}rv{\"a}ld Circle}
%   \city{Hekla}
%   \country{Iceland}}
% \email{larst@affiliation.org}

% \author{Valerie B\'eranger}
% \affiliation{%
%   \institution{Inria Paris-Rocquencourt}
%   \city{Rocquencourt}
%   \country{France}
% }

% \author{Aparna Patel}
% \affiliation{%
%  \institution{Rajiv Gandhi University}
%  \streetaddress{Rono-Hills}
%  \city{Doimukh}
%  \state{Arunachal Pradesh}
%  \country{India}}

% \author{Huifen Chan}
% \affiliation{%
%   \institution{Tsinghua University}
%   \streetaddress{30 Shuangqing Rd}
%   \city{Haidian Qu}
%   \state{Beijing Shi}
%   \country{China}}

% \author{Charles Palmer}
% \affiliation{%
%   \institution{Palmer Research Laboratories}
%   \streetaddress{8600 Datapoint Drive}
%   \city{San Antonio}
%   \state{Texas}
%   \country{USA}
%   \postcode{78229}}
% \email{cpalmer@prl.com}

% \author{John Smith}
% \affiliation{%
%   \institution{The Th{\o}rv{\"a}ld Group}
%   \streetaddress{1 Th{\o}rv{\"a}ld Circle}
%   \city{Hekla}
%   \country{Iceland}}
% \email{jsmith@affiliation.org}

% \author{Julius P. Kumquat}
% \affiliation{%
%   \institution{The Kumquat Consortium}
%   \city{New York}
%   \country{USA}}
% \email{jpkumquat@consortium.net}

\author{Ramtin Ehsani}
\affiliation{%
  \institution{Drexel University}
  \city{Philadelphia, PA}
  \country{USA}}
\email{ramtin.ehsani@drexel.edu}

\author{Rezvaneh Rezapour}
\affiliation{%
  \institution{Drexel University}
  \city{Philadelphia, PA}
  \country{USA}}
\email{shadi.rezapour@drexel.edu}

\author{Preetha Chatterjee}
\affiliation{%
  \institution{Drexel University}
  \city{Philadelphia, PA}
  \country{USA}}
\email{preetha.chatterjee@drexel.edu}

%%
%% By default, the full list of authors will be used in the page
%% headers. Often, this list is too long, and will overlap
%% other information printed in the page headers. This command allows
%% the author to define a more concise list
%% of authors' names for this purpose.
% \renewcommand{\shortauthors}{Trovato and Tobin, et al.}

%%
%% The abstract is a short summary of the work to be presented in the
%% article.
\begin{abstract}
%Studies have shown that toxic behavior can cause contributors to leave and hinder newcomers' participation in Open Source Software (OSS) projects, especially from underrepresented communities. Thus, understanding and detecting patterns of toxic language play a crucial role in fostering OSS collaboration and inclusivity. The distinct nature of toxicity observed in OSS communities makes the use of general-purpose tools and techniques developed in other domains ineffective (e.g., entitlement and arrogance are more frequently observed on GitHub than on Reddit or Twitter). To get a fresh look at toxicity from the perspective of morality, we leverage Moral Foundations Theory to explore moral principles in the context of Software Engineering toxic communications, to analyze and understand moralities exhibited in GitHub issue threads, and to find their association to toxicity. 
%However, the distinct nature of software-engineering text and the nature of toxicity observed in OSS communities makes it challenging to use general-purpose toxicity detectors developed in other domains. 
To foster collaboration and inclusivity in Open Source Software (OSS) projects, it is crucial to understand and detect patterns of toxic language that may drive contributors away, especially those from underrepresented communities. 
Although machine learning-based toxicity detection tools trained on domain-specific data have shown promise, their design lacks an understanding of the unique nature and triggers of toxicity in OSS discussions, highlighting the need for further investigation.
In this study, we employ Moral Foundations Theory to examine the relationship between moral principles and toxicity in OSS. Specifically, we analyze toxic communications in GitHub issue threads to identify and understand five types of moral principles exhibited in text, and explore their potential association with toxic behavior. Our preliminary findings suggest a possible link between moral principles and toxic comments in OSS communications, with each moral principle associated with at least one type of toxicity. The potential of MFT in toxicity detection warrants further investigation. 
%Our preliminary results show that moral principles are exhibited in OSS, and how they are associated with different natures of toxic behaviors.
\end{abstract}

%%
%% The code below is generated by the tool at http://dl.acm.org/ccs.cfm.
%% Please copy and paste the code instead of the example below.
%%

\begin{CCSXML}
<ccs2012>
   <concept>
       <concept_id>10011007.10010940</concept_id>
       <concept_desc>Software and its engineering~Software organization and properties</concept_desc>
       <concept_significance>300</concept_significance>
       </concept>
 </ccs2012>
\end{CCSXML}

\ccsdesc[300]{Software and its engineering~Software organization and properties}
% \begin{CCSXML}
% <ccs2012>
%  <concept>
%   <concept_id>10010520.10010553.10010562</concept_id>
%   <concept_desc>Computer systems organization~Embedded systems</concept_desc>
%   <concept_significance>500</concept_significance>
%  </concept>
%  <concept>
%   <concept_id>10010520.10010575.10010755</concept_id>
%   <concept_desc>Computer systems organization~Redundancy</concept_desc>
%   <concept_significance>300</concept_significance>
%  </concept>
%  <concept>
%   <concept_id>10010520.10010553.10010554</concept_id>
%   <concept_desc>Computer systems organization~Robotics</concept_desc>
%   <concept_significance>100</concept_significance>
%  </concept>
%  <concept>
%   <concept_id>10003033.10003083.10003095</concept_id>
%   <concept_desc>Networks~Network reliability</concept_desc>
%   <concept_significance>100</concept_significance>
%  </concept>
% </ccs2012>
% \end{CCSXML}

% \ccsdesc[500]{Computer systems organization~Embedded systems}
% \ccsdesc[300]{Computer systems organization~Redundancy}
% \ccsdesc{Computer systems organization~Robotics}
% \ccsdesc[100]{Networks~Network reliability}

%%
%% Keywords. The author(s) should pick words that accurately describe
%% the work being presented. Separate the keywords with commas.
\keywords{moral principles, toxicity, open source, textual analysis}

% \received{20 February 2007}
% \received[revised]{12 March 2009}
% \received[accepted]{5 June 2009}

%%
%% This command processes the author and affiliation and title
%% information and builds the first part of the formatted document.
\maketitle

% Figure environment removed

\section{Introduction}
Automatic 3D reconstruction of clothed humans using image inputs has gained increasing significance due to its potential applications in a wide array of AR/VR scenarios. High-fidelity reconstructions typically depend on sophisticated capture systems, which are developed with dense camera arrays~\cite{collet2015high,joo2015panoptic,joo2018total}, programmable light-stages~\cite{Vlasic2009, guo2019relightables}, and depth sensors~\cite{newcombe2011kinectfusion,DoubleFusion,BodyFusion,dou2016fusion4d,newcombe2015dynamicfusion}. However, stringent capture environments equipped with complex hardware pose significant challenges for consumer-level applications.


In this context, considerable research effort has been dedicated to developing methods that allow for more flexible capture configurations, such as utilizing a few RGB inputs. Among these works, learning implicit functions \cite{iccv2020PIFu, saito2020pifuhd, hong2021stereopifu} has proven effective in achieving highly detailed reconstructions by integrating the advancements of deep neural networks. These methods employ large multi-layer perceptrons (MLPs) to predict the occupancy probability or truncated signed distance function (TSDF) value of every queried 3D point based on its associated local feature, which is extracted from images. They can recover a continuous surface at arbitrary resolutions without topology restrictions.


However, in typical MLP-based implicit networks, the occupancy or TSDF value at each location is solved independently with planar image features, rendering them less capable of addressing challenging cases such as occlusions. Consequently, these methods suffer from generalization and robustness issues, particularly when tackling strong occlusions caused by large motion or multiple interacting humans. 
Some follow-up studies  \cite{zheng2021deepmulticap,zheng2021pamir,huang2020arch} utilize an extra geometric model, SMPL~\cite{Loper2015}, to improve robustness by introducing strong shape priors. 
Their success typically relies on the assumption of geometrical similarity \cite{huang2020arch} between the shape prior and target reconstruction, making them intractable for handling complex cases with loose clothes and sensitive to errors in SMPL model fitting.



%\ping{this paragraph sounds like `TSDF is better than MLP/SMPL, and we use TSDF to solve the problem'. But in Sec 3, we are telling a different story, saying `MLP needs a 3D convolutional encoder'. We need to make these two sections consistent.}\sicong{I think in this paragraph we claim that the TSDF}


%We opt for Trucated Signed Distance Funtion (TSDF) volumetric representations as they are naturally suitable for convolution operations, which have shown remarkable performance for learning hierarchical features on 2D visual perception tasks \cite{SunXLW19}. 
%Meanwhile, TSDF also describes the gradual geometry change around shape surface, which is not reflected by occupancy volume. 

We instead revisit the 3D volumetric representation and resort to 3D convolutional neural networks (CNNs) for feature learning, due to their impressive performance in feature learning and the ability to incorporate spatial context. However, volumetric methods and 3D convolution involve discretization, which might raise concerns regarding whether a discretized volume can preserve subtle geometric details as continuous representations learned in implicit functions. We investigate the relationship between volume resolution and quantization error on synthetic data by converting target mesh objects to TSDF volumes, as shown in Figure~\ref{fig:quantization_error}. We observe that the quantization errors are significantly reduced by increasing volume resolution and become nearly negligible when reaching a relatively high resolution (e.g., 512 or higher). In other words, achieving fine-detailed reconstruction is not supposed to be restricted by the use of volume representations as long as a proper volume resolution is utilized. Therefore, we present a method with high-resolution feature volumes, e.g., 256 and 512, while traditional volumetric methods \cite{varol18_bodynet,gilbert2018volumetric} are often limited to much lower resolutions, such as 32 or 128.



On the other hand, an increase in volume resolution may lead to a cubic growth of memory overhead \cite{8100085}. Reducing memory costs while guaranteeing the granularity of volumetric representations is necessary for pursuing high-quality reconstruction. Thus, we adopt a coarse-to-fine approach and cull away irrelevant voxels to build a sparse high-resolution feature volume. At the coarse level, the network computes an initial TSDF by applying a U-Net with sparse 3D CNN \cite{3DSemanticSegmentationWithSubmanifoldSparseConvNet} on the sparse feature volume, which is carved by a visual hull. Through our experiments, it turns out that more than 95\% of the volume grids are discarded by the visual hull culling, making the sparse 3D CNN efficient. At the fine level, the network focuses on a narrow band near the zero-level set of the initial TSDF and discretizes the narrow band with smaller voxels. By employing this narrow-band culling, we further shrink the sampling space, resulting in a relatively small range of grid numbers (usually 300K--500K in our experiments) even with a high volume resolution of 512. The remaining voxels in the narrow band are associated with features that fuse high-frequency information from the computed normal maps upon the low-frequency shape from the coarse level to compute the TSDF at high resolution. The final mesh is then extracted from the TSDF using the Marching-Cube algorithm ~\cite{Lorensen87marchingcubes}.
% Different from the u-net sturcture to preserve global topology context, we then apply a shallow 3dcnn to compute the final TSDF $D_{final}$ which contain more local geometry detail.




% \ping{this paragraph can be expanded. It is an important contribution and often ignored by other works. stress on the novel idea of regressing blending weights instead of colors}

In addition to geometry, high-quality mesh texture is also a crucial factor contributing to visual appearance. Directly computing a color field in 3D space, as in \cite{iccv2020PIFu}, struggles to capture high-frequency texture details, while the neural radiance field (NeRF) \cite{yu2020pixelnerf} or the DoubleField~\cite{shao2022doublefield} require expensive per-instance optimization and are often unstable for sparse input images. In contrast, we adopt an image-based rendering approach to compute a texture atlas map, which is efficient and widely supported in existing computer graphics tools. 
Specifically, we compute a blending weight at each 3D point on the mesh surface to determine its color as a weighted average of the colors at its image projections. The blending weights can be computed at a relatively coarse resolution, e.g., 512 volume resolution in our case, and leave texture details to the high-resolution images, such as 1K or 2K. Unlike previous methods that generate blurry texturing results under sparse input, our method generalizes well on both synthetic and real data with just a few input views. 
Figure~\ref{fig:teaser} shows two examples reconstructed by our method. Despite the challenging garment, pose, and occlusion, our method recovers faithful shape, normal, and texture on the right.

%with a wide variety of poses and clothing styles, and it is also adaptive to handle input image with arbitrary resolutions.
%\sicong{For this concern we claim that when the resolution of dicretized volume meets certain threshold (which is 256 in our experiment), the quantization error can be neglected.} 



In summary, the main contributions of this paper are as follows:
\begin{itemize}
\vspace{-0.1in}
  \item 
  We revisit the 3D volumetric representation and demonstrate that it can support clothed human reconstruction with equal or even better performance compared to implicit representation. 
  \item 
  We develop a memory and computation-efficient method for high-resolution volumetric reconstruction using sophisticated sparse 3D CNN, coarse-to-fine estimation, and voxel culling by visual hull and narrow bands. 
  \item 
  We introduce a novel method to compute a texture atlas map, which captures rich appearance details from high-resolution input images.
  \item 
  We achieve impressive results on standard benchmark datasets Twindom and MultiHuman, significantly reducing the point-2-surface (P2S) precision to approximately 0.2cm from just six input views, with more than $50\%$ error reduction compared to the state-of-the-art methods, including DoubleField~\cite{shao2022doublefield} and PIFuHD~\cite{saito2020pifuhd}.
\end{itemize}
\section{Cross-Lingual Diffusion Language Model}
\label{sec:XDLMusion}

% In this section, we present our proposed language modeling objectives designed specifically for diffusion and the diffusion model applied for cross-lingual translation. These objectives cater to both monolingual and multilingual data, and they are situated within the diffusion model framework for facilitating cross-lingual translation.

In this section, we present the Cross-lingual Diffusion Language Model (XDLM), which incorporates a pre-training phase on cross-lingual data, utilizing diffusion techniques for the purpose of non-autoregressive machine translation, and a fine-tuning phase generating corresponding text from one language to another language based on the pre-trained model.

% \subsection{Preliminary}
% \subsubsection{Cross-lingual translation}
% (\irene{combine 3.1.1 and 3.1.2 as NAR machine translation, and, there is no such term called \textit{Cross-lingual translation}, all translation is cross-lingual, it should be either \textit{machine translation} or \textit{cross-lingual language model}})

% Cross-lingual translation typically involves generating an output sequence $Y=\{y_1, y_2,…, y_{|Y|}\}$ from a given input sequence $X=\{x_1,x_2,…,x_{|X|}\}$, with each sequence being in a different language. Three common generative paradigms exist for cross-lingual translation: AutoRegressive (AR) generation, Non-AutoRegressive (NAR) generation, and semi-NAR generation. Ordinarily, diffusion models employ the NAR approach for generation tasks.

% \subsubsection{Non-AutoRegressive(NAR) generation}
% The NAR generation follows the conditional probality: 
% $$
% p_{\theta}(Y|X)=\prod_{i=1}^{|Y|} p_{\theta}(y_i|X)
% $$

% Unlike AutoRegressive (AR) generation, all tokens $y_i$$(0\leq i \leq |Y|)$ in the generated sequence Y are predicted concurrently. The generation solely depends on the input sequence X, without any dependency on preceding tokens. This attribute presents a challenge in determining the length of the generated sequence. To address this issue, the prediction of the output sequence is introduced as an auxiliary task \cite{gu2017non}.

\textbf{Non-AutoRegressive (NAR) Machine Translation}
In machine translation, given the input sequence from a source language $X=\{x_1,x_2,…,x_{|X|}\}$, the task is to generate the output sequence of the translation in the target language $Y=\{y_1, y_2,…, y_{|Y|}\}$. In this work, we focus on the Non-AutoRegressive (NAR) translation setting with the diffusion model. Typically, it has the following conditional probability:  
$$
p_{\theta}(Y|X)=\prod_{i=1}^{|Y|} p_{\theta}(y_i|X).
$$

Unlike AutoRegressive (AR) text generation, all tokens $y_i$$(0\leq i \leq |Y|)$ in the generated sequence $Y$ are predicted concurrently. The generation solely depends on the input sequence $X$, without any dependency on preceding tokens. This attribute presents a challenge in determining the length of the generated sequence. To address this issue, the length prediction of the output sequence is introduced as an auxiliary task \cite{gu2017non}. And the training loss is defined as a weighted sum between the translation loss and the length prediction loss.

\textbf{Diffusion Models}
The Denoising Diffusion Probabilistic Model (DDPM) \cite{ho2020denoising} is a parametrized Markov chain, and it is trained using variational inference to generate samples that match the original input data. 
% a diffusion process for generative tasks was introduced by \cite{ho2020denoising}, yielding impressive results.
The diffusion process comprises a noise-adding forward process and a noise-removing backward process, both of which can be viewed as discrete-time Markov processes. During the forward process, the model gradually introduces random noise with different scheduled variance $\beta_1,...,\beta_t$, with the aim of generating a standard Gaussian noise $x_t$ after $t$ turns. This can be formalized as follows:
$$
q(x_{t+1}|x_t)=\mathcal{N}(x_{t+1};\sqrt{1-\beta_{t+1}}x_t,\beta_{t}\mathbf{I}).
$$

The backward process, the reverse of the forward process, attempts to reconstruct the target sequence from the standard noise. Like the forward process, this procedure is also applied incrementally and can be formalized as follows:

$$
    p(x_{t-1}|x_t)=\mathcal{N}(x_{t-1};\mu_{\theta}^{t-1},\sigma_{\theta}^{t-1}),
$$
$$
    \mu_{\theta}^{t-1}=\frac{1}{\sqrt{\alpha_{t}}}(x_t-\frac{\beta_{t}}{\sqrt{1-\overline(\alpha_{t})}}z_{\theta}(x_{t},t)), 
$$
$$
    \sigma_{\theta}^{{t-1}^2}=\frac{1-\overline{\alpha_{t-1}}}{1-\overline{\alpha_{t}}}\dot \beta_{t},
$$

where $\alpha_t=1-\beta_t, \overline{\alpha_{t}}=\prod_{i=1}^t \alpha_{i}$ and $z_\theta$ comes from the prediction of model parameterized by $\theta$. 
In this work, we apply discrete diffusion for text generating and cross-lingual translation. Based on \citet{zheng2023reparameterized}, we follow the proposed discrete diffusion model with the following routing mechanism.

$
    x_{t-1}, v_{t-1} \sim q(x_{t-1},v_{t-1}|x_t,x_0) \\
    q(v_{t-1}|x_t,x_0)=q(v_{t-1})=Bernoulli(\lambda) \\
    q(x_{t-1}|v_{t-1},x_t,x_0)= \\
    v_{t-1}x_t+(1-v^{(1)}_{t-1})q_{noise}, \quad if \quad x_t = x_0 \\
    v_{t-1}x_0+(1-v_{t-1}^{(2)})q_{noise} (x_t), \quad if \quad x_t \neq x_0 \\
$


Which models the joint distribution over both $x$ and $v$. The sampling process here also takes the reparameterized method, which improves flexibility and expressiveness compared to the original process.

% Figure environment removed
\textbf{Translation Diffusion Language Modeling (TDLM)}
% Contrary to previous language modeling objectives for diffusion models, which primarily focus on monolingual data and neglect the potential to harness cross-lingual modeling capabilities from parallel datasets, we propose a pretraining process for parallel language pairs along with a corresponding modeling objective.
Unlike previous diffusion model objectives for language modeling that primarily concentrate on monolingual data, we target to exploit cross-lingual modeling capabilities from parallel datasets. Consequently, we propose a pretraining process named Translation Diffusion Language Modeling (TDLM), aiming at enhancing cross-lingual pretraining with diffusion models. As illustrated in Figure 1, we first concatenate both source and target sentences and generate the corresponding language and position embedding sequences, and then stack them as the input to a diffusion model. 
% we select both source and target sentences, generate their corresponding language and position embedding series, and concatenate them to form the input text stream. 
In a similar vein to \citet{lin2023text}, we random mask 15\% of the tokens to the input as \cite{lample2019cross} designed, tasking the model with predicting the noise and its surrounding text based on the cross-lingual context. This denoising setting assists the model in grasping the cross-lingual context.


\vspace{-.2cm}
\section{Preliminary Observations}
\label{results}
% % Figure environment removed



In this section, we discuss our observations for the moral principles as exhibited in the GitHub heated issues. %Next, we try to understand 
%\subsection{Moral Principles in GitHub Issues}
%With definitions, we do not mean a new, enhanced version of Moral Foundation Theory. But rather, we intend to see how we can interpret moral values in the context of OSS. 
%SE communications are significantly different than other types of communications seen on other social platforms such as Twitter, Reddit, etc. Therefore, it is necessary for us to see how we can adapt the original definitions of morality to SE communications, especially the language used in \textit{toxic} SE communications as we are trying to understand the connection between toxicity and morality. We use the key concepts of each morality type to see if we can observe any patterns in the dataset. Then, we use these patterns to come up with SE-adapted definitions of morality. 
%In this section, we explain these patterns for the 5 moral foundations. It should be noted that detecting morality even in social contexts such as Twitter is a hard task, mostly in the sense of distinguishing types of morality from each other. Because of this very reason, Hoover et. al. \cite{hooverMoral2020} allow for overlapping labels (expressions of moral sentiment that are associated with multiple foundations) during the annotation process for morality detection in the Twitter corpus. We did our best to come up with clear definitions for each type of morality in the domain of SE.
We analyze the issue comment threads in our data sample to identify instances where moral principles were expressed in the text.
% Figure \ref{OSSMoralExample} shows examples of moral values as exhibited in GitHub.
Out of 695 issue comments across 100 threads, 135 exhibit at least one form of moral principle.
%A total of 695 issue comments (in 100 issue threads) were analyzed, and from those, 135 comments exhibit at least one type of morality in them.
% XX 
%In order to better clarify our observations of different patterns of conversations in these issue threads, 
In this analysis, %it is important to note that we refer to the 
people who open the issue threads to find fixes for problems are referred to as \textit{users}, and people who try to resolve the issues and close the threads as \textit{contributors}. %Users are the people who are seeking help from contributors to add new features or fix the bugs and problems that they are facing when using the products. Upon identifying these roles, 
We observed unique patterns specific to each of the identified roles.

% \textcolor{red}{@Ramtin - ADD HERE A PARA ABOUT THE DIFFERENT ROLES: I am guessing we have developers, maintainers, and users? We need to clearly define these roles.}
% In issue comments, users usually either seek help from contributors to fix the bugs and problems that they are facing when using the products, or they are asking for new features.

\noindent
\textbf{Care/Harm.}
%Based on the original definition, kindness, gentleness, and nurturance are counted as virtues in this foundation. 
A total of 48 comments exhibit this principle.
Pertaining to this category, following are the behaviors we observed: \textbf{(CH1) }users who use derogatory language or insults towards contributors while seeking assistance or requesting new features; \textbf{(CH2) }users insulting contributors mainly by trolling, possibly due to differences in ideology, opinions, etc.; \textbf{(CH3) }contributors responding using insults toward users.
% XX
% \begin{enumerate}[leftmargin=*]
%     \item[] \textbf{CH1: }users who use derogatory language or insults towards contributors while seeking assistance or requesting new features.
%     \item[] \textbf{CH2: }users insulting contributors mainly by trolling, possibly due to differences in ideology, opinions, etc.
%     \item[] \textbf{CH3: }contributors responding using insults toward users.
% \end{enumerate}

The first type (CH1) was the most common in our dataset (32 out of 48 instances). For example, in one thread, a user opens an issue saying that the name of the project is duplicated and it should change while using insults: \textit{"This name is in use and should be changed immediately. Yes, I have seen the other issues. Yes, I am opening a new one because f**k you Microsoft. You are merely trying to cast a shadow on other, truly open source projects..."}.

The second type (CH2) was also seen in a number of cases (5 out of 48 instances). For example, in one instance, a user opens an issue saying: \textit{"Revenue. F**k you guys"}, and does not add anything else to the thread. Threads similar to this are mostly the result of trolling, but they can also be the result of differences in ideology or based on previous interactions~\cite{Miller2022}.

For the third type (CH3), we saw an interesting pattern of contributors responding in a detrimental manner towards the users. This type is rare compared to other types in this category.
% In one interesting instance, a user opened up an issue explaining the need for making a variable configurable: \textit{"...Make boinc auth a configurable value in the community grid role to enable users to allocate the WU points to an account that they choose..."}. The conversation goes on to the point where the user is explaining the ethical implications of this choice that developers have made \textit{"...It is also not disrespectful or disingenuous to debate the ethical implications of such a decision. Some seem to interpret such a discussion by default as an attack on them personally..."}.
% However, the project member continues the conversion with this comment: \textit{"...I will wait for others like @cloin and @liquidat to weigh in on this. Just because you THINK one way doesn't mean that others THINK another way..."}. The conversation even escalates to the point where the developer says: \textit{"...I would have hoped that all IBMers and Red Hatters, our partners and community would be happy we are trying. Sorry that we tried to contribute to make the world a better place..."}.
% This conversation is not as severe as the other instances but still falls under the harm category nonetheless.
In one instance, a user opens an issue asking: 
\textit{"Where to f**k python2? Why, when I give the brew install python command, python3 is installed, not python2. Developers are you stoned there?"},
and surprisingly, one of the contributors responds with: 
% and interestingly, one of the contributors responds with: 
\textit{"I recommend not having intercourse with EOL software. Python 2 is no longer included in Homebrew"}. The user then responds: \textit{"You have an unfinished raw product with some problems. What can be your attitude? Treating users like sh*t"}. This issue thread exhibits the first and last patterns (CH1 and CH3).
%In another issue thread, a user suggests adding a framework to a list in the project: \textit{"I think iris should be on list too"}. However, one of the developers responds by saying: \textit{"...would honestly prefer to write a web application in assembly language before using anything written by @USER. He is a cancer to open source"}.
%All instances resembling a pattern of communication like the mentioned examples are attributed to the \texttt{Harm} foundation. 

\noindent
\textbf{Fairness/Cheating.}
%This foundation captures the ideas of justice, rights, and autonomy. Based on this definition, discrimination, and unfairness count as vices for this foundation. Discrimination can fall into different types such as nationality, culture, gender, LGBTQ+, religion, age, and race \cite{vanbreukelen2023still, Sarker2023Mitigate}. As we were going through our dataset, we did not observe these types of discrimination. Based on the results of the study done by Cheriyan et. al. \cite{cheriyanTowards2021}, racial offense is a rare type of offense in the GitHub community when compared to swearing and personal offenses. However, this does not mean that they do not exist. Therefore, for this type of foundation, we decided to look at data outside of our dataset to find instances of discrimination in GitHub.
We observed behaviors associated with this category in a total of 19 comments, which include: \textbf{(FC1) }users having unrealistic expectations from the contributors; \textbf{(FC2) }contributors failing to address user's issues due to unjustified reasons.
% \begin{enumerate}[leftmargin=*]
%     \item[] \textbf{FC1: }users having unrealistic expectations from the contributors.
%     \item[] \textbf{FC2: }contributors failing to address user's issues due to unjustified reasons.
%     %\item discrimination (based on nationality, culture, gender, LGBTQ+, religion, age, race, etc)~\cite{vanbreukelen2023still, cheriyanTowards2021, Sarker2023Mitigate}.
% \end{enumerate}

Among the instances of this category, 7 out of 19 instances exhibit the first type of communication pattern. In one of the instances representing the first type (FC1), the user has opened an issue asking about a problem in the debugger. Contributors of the projects try to help the user by providing tips on how the problem can be solved, but as the conversation goes on, the user gets frustrated and says: \textit{"@<USER1> @<USER2> Too busy to respond?"}, which in the end, leads to the issue thread being locked. This is an example of unrealistic expectations about  response time causing conflict.

For the second type (FC2), %in one instance the user has opened an issue explaining the project's name is in conflict with other projects: \textit{"Name clash with Maui Linux and MauiKit"}. One of the contributors responds sarcastically: \textit{"Perhaps the Linux project should change its name as they've conflicted with a city in Hawaii which existed long before"}. As the conversation goes on, the contributor states that: \textit{"You can be mad all you want, but let's be realistic here... this project you're fighting for so passionately, doesn't have as many stars as I have thumbs down for telling you that you're being ridiculous"}, which is an obvious resemblance of cheating morality.Or in another instance, 
for instance, this is how one of the contributors responds to a  user's issue: \textit{"How would be this a priority, the app must be ready to be shipped in a few days and basically, no one uses the smartphone in landscape mode except for watching media"}.

Another type of behavior related to \textit{Fairness} would be discrimination (based on race, culture, gender, religion, etc). In our dataset, we did not observe any instances of discrimination. Analysis of additional data may reveal discriminatory behavior in OSS, as such occurrences have been observed in previous studies~\cite{vanbreukelen2023still, cheriyanTowards2021}.
%And finally, for the final type, by looking at Cheriyan et. al. \cite{cheriyanTowards2021}'s annotated dataset, we found an issue thread with severe use of racial slurs in it. In this thread, a user posted this comment: \textit{"instead of showing a list of the reagents and picking them to switch to them it now requires you to press the item to switch between the reagents over and over and over Jesus Fucking Christ Ni**as"}. It should be noted that by analyzing more data, we should be able to find instances of religious, gender, and other types of discrimination in SE communications as well, as they have been observed and studied before. %These types of instances, if observed, fall into cheating morality as well.

\noindent
\textbf{Loyalty/Betrayal.}
%The key concepts in this foundation are patriotism and self-sacrifice for the group, or as the saying goes: "one for all, and all for one". The vices for this foundation boil down to the people who rebel against the community and betray the community and its people.
The observed patterns for this category of morality were found in 23 instances of the dataset: \textbf{(LB1) }users actively promoting and inciting rebellion against a project, encouraging others to switch to an alternative project (most common with 15 instances); \textbf{(LB2:) }contributors excluding users from the project.
% \begin{enumerate}[leftmargin=*]
%     \item[] \textbf{LB1: }users actively promoting and inciting rebellion against a project, encouraging others to switch to an alternative project (most common with 15 instances).
%     \item[] \textbf{LB2: }contributors excluding users from the project.
% \end{enumerate}

To better understand these behaviors, we provide examples for each type. For instance, in one of the issue threads, a user claims that SSL Insecure is not being respected in the project: \textit{"Whoever will find this issue and gets pissed off, because the author doesn't bother to fix it for years, ditch mitmproxy and use SSL SPLIT"}. One of the developers responds with: \textit{"I'm glad you have found something that works for you!"}. In this instance, both LB1 and LB2 were observed, because the user is encouraging people to leave this project behind, and the contributor also encourages the user to not use the project.
In another instance, a user requests the removal of the slur filter in the app, however, one of the contributors says: \textit{"If you don't like it, fork it. Stop bothering us about it, we will never fully remove the slur filter"}, which counts as excluding the user from the community.

% % Figure environment removed

% Figure environment removed

\noindent
\textbf{Authority/Subversion.}
OSS communities, like any other community, often have specific rules and guidelines that everyone has to follow. GitHub projects use the Code of Conduct (CoC) in an attempt to promote their expectations and standards of ethical behavior within the community \cite{touraniCode2017}. %When rules and hierarchies are defined in OSS communities, it becomes natural that moralities such as authority and subversion are at play in OSS communications.
The establishment of such rules and hierarchies often leads to the dynamic interplay of moralities such as authority and subversion in OSS communications. A total of 30 instances with this principle were found in our dataset.
The patterns observed for this category are: \textbf{(AS1) }users trying to rebel against authority, and questioning developers' ability to lead the community. \textbf{(AS2) }contributors (authorities) enforcing Code of Conduct when thought necessary or any other admin-privileged acts to ban, censor, or silence users (most common with 16 instances).
% \begin{enumerate}[leftmargin=*]
%     \item[] \textbf{AS1: }users trying to rebel against authority, and questioning developers' ability to lead the community
%     \item[] \textbf{AS2: }contributors (authorities) enforcing Code of Conduct when thought necessary or any other admin-privileged acts to ban, censor, or silence users (most common with 16 instances).
% \end{enumerate}

In one of the instances, a user has opened an issue, claiming that the project has stolen his code: \textit{"Don’t be a d*ck as the source code was stolen at that point and implemented on this"}.
The contributor enforces authority by saying: \textit{"please be civilized and refrain from profanities as required by our Code of Conduct, or I'll ban you from all Falconry projects. This is the last warning"}.
And the user challenges the authority by saying: \textit{"I’m not a bot but if you wish to ignore me go ahead and I’ll get a lawyer"}.
%In another interesting instance, a user explains in a civilized manner that he thinks the release of "Pyflakes" should be urgent, contrary to what the contributor believes: \textit{"Would be great to get this released ASAP...I disagree. pyflakes is currently incompatible with (arguably) the number 1 headline feature of the latest version python...I therefore think a release is very urgent...pyflakes is great and I know how thankless maintaining OS libraries like this can be. Thank you all for your hard work"}. But the contributor tags the comment with the tag "off-topic", which leads to the user saying: \textit{"Please could someone explain why my comment was marked as "off-topic"? You may disagree with it, but it is absolutely not off-topic - it relates directly to the the subject of this issue."}. Another user also proceeds to say: \textit{"it's very disappointing to see this sort of behavior on a project this important. You and the other maintainers should strongly consider adding a code of conduct to this repo and following it."}.
Additionally, there were instances where the users questioned the capabilities of contributors. For example, a user opened an issue asking about how to do a certain thing within the system. As the conversation goes on, the user responds to one of the contributors: \textit{"...I would hate to have you in charge of any security issues"}.

\noindent
\textbf{Sanctity/Degradation.}
%Based on the definitions, "disgust" is the key concept of this type of morality. 
The most commonly observed pattern in this category, in a total of 42 instances, was observed when: (\textbf{SD1}) users express their hate (disgust) toward a certain package, system, or code that the project is using, most likely due to  their personal preferences. This principle is targeted at the code/system rather than people (targeting people represents \textit{Care/Harm}).

For example, a user opened an issue in the project expressing how much the document is poorly written, and in a sense, he is \textit{disgusted} with it: \textit{"...in fact, you write a sh*t doc. I'm a real man, it's my feeling of your holy sh*t doc"}. 
Or in another instance, a user opens an issue explaining his problem with the software: \textit{"...I just tried reinstalling you buggy, sh*tty software for the third time..."}.



%In another example, a user has opened an issue after a new release of the software, saying: \textit{"did this fix the annoying pulse audio bullsh*t?"}.
\section{Experimental Results}\label{sec:results}
    \subsection{General Results}
        The basic ResSAN model is used to determine reference results which our expanded model can be compared to as it is structurally similar to ResLAN but does not possess the Lidar adaptive components of it. Further, we compare with the full-size PackNet-SAN and the unmodified NLSPN architecture. 
        As it can be seen from Tab.\,\ref{tab:sota-results}, our LiDAR-adaptive ResLAN achieves competitive performance compared to state-of-the-art standard depth completion methods, which are specialized to the unfiltered 64-beam-LiDAR. The performance differences are in the range of a few centimetres in terms of MAE, which is acceptable given the practical advantage that ResLAN can generalize to different beam patterns as will be shown below.

        Furthermore, we compared the architectures for a set of three different input types that contained 64, 32 or 16 LiDAR channels using both filter types on the metrics from the KITTI benchmark. The NLSPN model was trained for the standard depth completion task and then evaluated with different input data. As for the ResSAN models, we trained one model for each input type and tested it for the corresponding one which serve serve as the \emph{Baseline} in Tab.\,\ref{tab:overall-results}. Our ResLAN model was jointly trained for all three settings. As listed in Tab.\,\ref{tab:overall-results}, the ResLAN models outperform the challenging baseline in all metrics for FOV filtering and all but one for sparse filtering. This implies that our LiDAR adaptive model is able to outperform dedicated models in case of very sparse input depth. Fig.\,\ref{fig:comp-plot} shows this is indeed the case for 32 and even more for 16 channels. For FOV-filtered inputs with 16 channels, the ResLAN exhibits approx. $10\%$ smaller MAE than the baseline. As for the NLSPN, it becomes apparent that it is not capable of generalizing to other input types since it shows clearly worse results. The difference is especially pronounced for the FOV filtering where on average more than every fourth predicted pixel is more than $25 \%$ deviating from the ground truth\,($\delta_{1.25}$). Therefore, using a weight-adapting network in combination with differently filtered input depths allows us to train models that outperform their non-adaptive counterparts.

        \begin{table}[]
            \centering
    	    \small
            \vspace{0.4cm}
            \caption{\textbf{Depth estimation result for standard depth completion} when the ResSAN model was only trained for 64 channels and the ResLAN model for multiple tasks. The PackNet-SAN and NLSPN models were trained with the setup that was also used for our model architecture.}
            \footnotesize
            \setlength{\tabcolsep}{5pt}
            \begin{tabular}{@{}lrrrrl@{}}
            \toprule
            \multicolumn{6}{c}{\textbf{Standard LiDAR Depth Completion}}                                                                                                                         \\ \midrule
            \multicolumn{1}{l|}{Method}          & RMSE $\downarrow$            & MAE  $\downarrow$            & iRMSE $\downarrow$             & iMAE $\downarrow$ & $\delta_{1.25}$ $\uparrow$ \\
            \multicolumn{1}{l|}{}                & \multicolumn{1}{l}{{[}mm{]}} & \multicolumn{1}{l}{{[}mm{]}} & \multicolumn{1}{l}{{[}1/km{]}} & {[}1/km{]}        &                            \\ \midrule
            \multicolumn{1}{l|}{PackNet-SAN}     &  914                            &  298                            &  2.78                              &  1.4                 &  99.65 \%                          \\
            \multicolumn{1}{l|}{NLSPN}           &  \textbf{889}                            &   \textbf{263}                           &  \textbf{2.62}                              &   \textbf{1.3}                &   \textbf{99.61} \%                         \\ \midrule
            \multicolumn{1}{l|}{ResSAN (Ours)}   & 948                             &  275                            &  2.75                              &    1.4               &   99.58 \%                         \\
            \multicolumn{1}{l|}{ResLAN (Ours)} &   969                           &  283                            &   2.83                             &   1.4                &  99.56 \%                          \\ \bottomrule
            \end{tabular}
            \vspace{0.2cm}
            \label{tab:sota-results}
        \end{table}

        \begin{table}[]
    	    \centering
    	    \small
    	    \caption{\textbf{Depth estimation results of the two baseline setups and the explicit and implicit ResSAN} when evaluated on a combination of 16, 32 and 64 channel depth inputs. Please note that Specialist Methods need to train three specialized networks, one for each of the three types of inputs while our method only uses one network.}
            \footnotesize
            \setlength{\tabcolsep}{4.8pt}
            \begin{tabular}{@{}lrrrrl@{}}
                \toprule
                \multicolumn{6}{c}{\textbf{Sparse Channel Filter}}                                                                                                                                  \\ \midrule
                \multicolumn{1}{l|}{Method}        & RMSE $\downarrow$            & MAE  $\downarrow$            & iRMSE $\downarrow$             & iMAE $\downarrow$ & $\delta_{1.25}$ $\uparrow$  \\
                \multicolumn{1}{l|}{}              & \multicolumn{1}{l}{{[}mm{]}} & \multicolumn{1}{l}{{[}mm{]}} & \multicolumn{1}{l}{{[}1/km{]}} & {[}1/km{]}        &                             \\ \midrule
                \multicolumn{1}{l|}{NLSPN}         &  1396                            &  437                            & 5.54                               &  2.2                 &  98.82 \%                           \\
                \multicolumn{1}{l|}{Baseline}      & \textbf{1207}                             &  381                            & 4.41                               &  1.8                 &  \textbf{99.37} \%                           \\
                \multicolumn{1}{l|}{ResLAN (Ours)} &  1215                            &  \textbf{378}                            &  \textbf{4.27}                              &  \textbf{1.7}                 &  99.31 \%                           \\ \toprule
                \multicolumn{6}{c}{\textbf{Field-of-View Filter}}                                                                                                                                   \\ \midrule
                \multicolumn{1}{l|}{Method}        & RMSE $\downarrow$            & MAE  $\downarrow$            & iRMSE $\downarrow$             & iMAE $\downarrow$ & $\delta_{1.25}$ $\uparrow$ \\
                \multicolumn{1}{l|}{}              & \multicolumn{1}{l}{{[}mm{]}} & \multicolumn{1}{l}{{[}mm{]}} & \multicolumn{1}{l}{{[}1/km{]}} & {[}1/km{]}        &                             \\ \midrule
                \multicolumn{1}{l|}{NLSPN}         &  2738                            &  1702                            & 12.3                              &  4.3                 &  74.69 \%                           \\
                \multicolumn{1}{l|}{Baseline}      &  1556                            &  525                            &  6.8                              &  3.0                 & 98.14 \%                            \\
                \multicolumn{1}{l|}{ResLAN (Ours)} &  \textbf{1548}                            &  \textbf{519}                            &  \textbf{6.44}                              &  \textbf{2.8}                 & \textbf{98.52 \%}                            \\ \bottomrule
            \end{tabular}
            \label{tab:overall-results}
        \end{table}

        
        
        % Figure environment removed
        
        % Figure environment removed

    \subsection{Filter Effects}
        Comparing the effect of the two different types of depth input filters on the model performance, it becomes apparent that FOV filtering is the more challenging task. In that setting, reducing LiDAR channels is more detrimental to the performance than sparse filtering as it creates regions where no depth information is available. Effectively, the model is forced to perform depth prediction in these regions. These effects are highlighted in the depth images in Fig.\,\ref{fig:dense-maps} where the effect of a 16-channel sparse depth filter and a 16-channel FOV can be compared.

    \subsection{Generalization Capabilities}
        We trained three models for both filter types eaach, so the combinations and number of filtered depth inputs they receive are different. This serves the purpose of testing the generalization capabilities of the ResLAN architecture as well as the robustness to different filter settings. After training, the models were evaluated for the depth input settings they were trained for, as well as for ones they weren't exposed to. Overall, ResLAN shows good generalization capabilities. As one can gather from Fig.\,\ref{fig:explicit-comp} and Fig.\,\ref{fig:implicit-comp}, the consequences of slightly varying sets of input depth settings are limited. The most considerable deviations can be seen when the model is tasked to extrapolate. For instance, the model $\{64, 32, 16\}$ shows a noticeably higher MAE for eight-channel depth inputs than the model that was trained for it. Similar behaviour can be seen for the FOV filtering case as well for the model $\{64, 48, 32\}$ when tasked to generalize for a 16-channel input. There is no such pronounced effect for generalization tasks that lie between two filter settings the model was trained for. At most, it can be observed that models that were trained for a smaller range of filter values perform slightly better than ones that have to cover a wider range. The number of filter settings used in a fixed range does not relevantly influence the model performance, as can be seen, when comparing the two models in Fig.\,\ref{fig:implicit-comp}, which are both trained for a range of 64 to 32 channels but one with three filter settings and the other one with five.
    
    % Figure environment removed
    
    
    % Figure environment removed
\section{Threats to Validity}
\label{sec:threats}

Given the empirical nature of our study, we discuss several threats to the validity of this work according to the guidelines proposed by \cite{runeson2009guidelines}, and how these threats were partially mitigated in our study.

% Given the empirical nature of our study, potential threats can affect the study results. We classify and discuss these threats by following the recommendations suggested in \cite{Wohlin2000Experimentation}.

\textbf{Construct Validity} reflects on the extent of consistency between the operational measures of the study and the RQs. In this work, we depended on human activities, including data labelling and data extraction \& analysis, which would introduce personal bias. To reduce this threat, each step in the aforementioned human activities was conducted by two authors and a third author was involved to discuss and resolve the conflict in case of disagreement. Moreover, we also conducted a pilot data labelling to make sure that the two researchers achieved a consensus on what are code snippets in this study, which could also partially alleviate this threat. 

Another threat to the construct validity of this study is that we used mostly closed-ended questions in the industrial survey, which may affect the richness of the responses collected from the participants. However, as argued by \cite{reja2003openended}, open-ended questions have several disadvantages compared with closed-ended questions. For example, much long time to fill out the questionnaire might make participants do not participate in the survey at all. Participants may provide poor answers or even just skip when answering open-ended questions. Due to the above disadvantages, we chose to mainly use closed-ended questions in our survey. For some of the closed-ended questions, we also provided the ``Other'' field so that participants can fill in their own opinions if existing options do not cover their thoughts. Furthermore, to help participants better understand the open-ended questions in the survey, we provided two examples for each question. During the data analysis, we found that some participants only agreed with the provided examples without providing additional answers, which indicates that the provided examples may restrict participants from providing their own answers to the open-ended questions, thus affecting the richness of answers. Besides, another threat is that some of the responses from participants are written in Chinese, and translating the raw data from Chinese to English may lead to information lost or corruption. The two authors who extracted and analyzed the Chinese responses are native Chinese speakers, and the third author who is a native Chinese speaker as well was asked to check and refine the translation, which partially minimizes this threat.
% However, the provided examples come from the results of the pilot interview before the formal survey, that means the examples are provided by other participants. We argue that it could partially reduce the threat of the examples to the answers.

The last threat is concerning the size of our dataset. We collected 63 responses from our industrial survey, and we acknowledge that the small number of responses may threaten the validity of our findings. Therefore, we conjecture that we could obtain more convincing results by inviting more developers with code review experiences from more diverse communities to participate in the survey, which is also our next step.

\textbf{Internal Validity} considers the causal relationships between variables and results. A threat to internal validity in this study is that we used a programming language detection tool called Guesslang to assist us in labelling the code review comments containing code snippets. During code review, review comments might contain both literal contents and code snippets, which may affect the labelling accuracy of automatic tool (i.e., Guesslang), thus affecting the results of this study. To mitigate this threat, we only used Guesslang to help filter out review comments that most likely do not contain code snippets, in order to retrieve the review comments containing code snippets as many as possible. In addition, to make the threshold of filtering as accurate as possible, we first conducted a pilot labelling, and adjusted the threshold based on the results of the pilot labelling.

\textbf{External Validity} refers to the degree to which our study results and findings can be generalized in other cases (e.g., other projects in OpenStack and Qt communities, or projects in other communities). We selected active projects from the OpenStack and Qt communities since these two communities have made a serious investment in code review for many years and have been widely used in many studies related to code review. We argue that the selected communities and projects are representative and can increase the generalizability of our study results. 

In terms of the industrial survey, we invited developers from the OpenStack and Qt communities collected from our dataset, developers from well-known software companies in China, and developers from professional software development groups, which partially increases the external validity of the survey results. But we admitted that the findings of this study may not be generalized to all developers. In the future, we plan to invite more developers from various development groups (e.g., inner source development) to expand the scope of the industrial survey.
% We plan to invite developers in other domains in the future, which would make the results of this study more generalized.

\textbf{Reliability} refers to the replicability of a study for obtaining same or similar results. To improve the reliability, we made a research protocol with detailed procedure, which was discussed and confirmed by all the authors. Besides, all of the empirical steps in our study, including the data mining process, data labelling, and data extraction and analysis, were conducted and discussed by three authors. Furthermore, the dataset and analysis results of our study have been made publicly available online~\cite{replpack} in order to facilitate other researchers to replicate our study easily. We believe that these measures can partially alleviate this threat.
%\paragraph{Unlearning.}

The naive approach to machine unlearning is to retrain a model from scratch with each data deletion request. However, retraining is not feasible for companies with many large models or organizations with limited resources. Thus, the primary objective of machine unlearning is to provide efficient approximations to retraining. Early approaches in security and privacy attempt to achieve exact removal, where an unlearned model is identical to retraining, but are limited in model class~\citep{caoMakingSystemsForget2015, ginartMakingAIForget2019}. \citet{bourtouleMachineUnlearning2020} propose SISA, a flexible approach to exact unlearning that ``shards" a dataset, dividing it and training an ensemble of models where each can be retrained separately. More recent approaches propose approximate removal, requiring the unlearned model to be ``close" to the output of retraining. Some approximate removal methods focus on improving efficiency~\citep{wuDeltaGradRapidRetraining2020} and others try to preserve performance~\citep{wuPUMAPerformanceUnchanged2022}. While these methods apply to a large class of models, they have no formal guarantees on data removal. A second group of approximate approaches provide theoretical guarantees on the statistical indistinguishability of unlearned and retrained models. These noise-based methods leverage convex loss functions to guarantee unlearning with gradient updates~\citep{neelDescenttoDeleteGradientBasedMethods2020a} and Hessian methods~\citep{guoCertifiedDataRemoval2020, sekhariRememberWhatYou, izzoApproximateDataDeletion}. We augment this second set of approximate methods to simultaneously provide strong guarantees on data protection and preserve fairness performance while targeting a common class of models.

\paragraph{Fairness.}

There are a multitude of definitions for fairness in machine learning, such as individual fairness, multicalibration or multiaccuracy, and group fairness. Individual fairness~\citep{dwork2012fairness} posits that ``similar individuals should be treated similarly" by a model. 
On the other hand, recent work has focused on multicalibration and multiaccuracy~\citep{hebert2018multicalibration, kearns2018preventing, deng2023happymap}, where predictions are required to be calibrated across subpopulations. These subpopulation definitions can be highly expressive, containing many intersectional identities from protected groups. In this work, however, we focus on the most commonly studied form of fairness, group fairness, which seeks to balance certain statistical metrics across predefined subgroups. Group fairness literature has proposed various definitions of fairness, but the three most common definitions are Demographic Parity~\citep{zafar2017fairness, feldman2015certifying, zliobaite2015relation, calders2009building}, Equalized Odds, and Equality of Opportunity~\citep{hardt2016equality}. To achieve these definitions, there are generally three approaches to achieving group fairness: \emph{preprocessing} which attempts to correct dataset imbalance to ensure fairness~\citep{calmon2017optimized}, \emph{in-processing} which occurs during training by modifying traditional empirical risk minimization objectives to include fairness constraints~\citep{lowyStochasticOptimizationFramework2022, berk2017convex, agarwal2018reductions, martinez2020minimax}, and \emph{postprocessing} which modifies predictions to ensure fair treatment~\citep{alghamdi2022beyond, hardt2016equality}. 
In this work we focus on in-processing algorithms because they simply modify an objective to account for fairness rather than requiring an additional operation before or after each unlearning request which would also have to be made unlearnable.

\paragraph{Intersections.} Despite advancements in machine unlearning, the literature still lacks sufficient consideration of the downstream impacts of unlearning methods. While recent papers have explored the compatibility of the right to be forgotten with the right to explanation~\citep{krishna2023towards}, there is little work at the intersection of unlearning and fairness. In privacy literature, a thread of work has shown the incompatibility of group fairness with privacy~\citep{esipova2022disparate, bagdasaryan2019differential, cummings2019compatibility} but these incompatibilities arise due to privacy-specific methods, such as gradient clipping and differences in neighboring datasets. Fairness literature has studied the related problem of the influence of training data on fairness~\citep{wang2022understanding}, but does not provide any methods for unlearning. In unlearning literature, recent empirical studies have shown that unlearning can increase disparity~\citep{zhang2023forgotten}, other works have demonstrated the incompatibility of fairness and unlearning for the SISA algorithm \citep{kochno}, and one work~\citep{wang2023inductive} has provided a method to achieve removal and fairness but uses a sharding and retraining algorithm over fairness-corrected graph data for GNNs. In this paper, we propose the first efficient method which achieves fairness while being provably unlearnable without requiring retraining.
\section{Conclusion and Future Work}
In this work, I design corruption-robust algorithms for the Lipschitz contextual search problem. I present the \emph{agnostic checking} technique and demonstrate its effectiveness in designing corruption-robust algorithms. There are several open problems for future research. First, in the algorithm I propose for pricing loss, the schedule for agnostic checks is fixed upfront. Can the learner design an adaptive checking schedule for the pricing loss? Second, this work assumes the learner has knowledge of the Lipschitz constant $L$. Can the learner design efficient no-regret algorithms without knowledge of $L$? 

% \citestyle{acmauthoryear}
\bibliographystyle{ACM-Reference-Format}
\bibliography{fse, aaai22,oss,preetha}


\end{document}
\endinput
%%
%% End of file `sample-manuscript.tex'.


