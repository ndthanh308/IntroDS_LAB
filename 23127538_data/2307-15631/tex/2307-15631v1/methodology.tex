\vspace{-6pt}
\section{Moral Foundations Theory}
\label{moraldef}
With the goal of investigating instances on GitHub that represent moral principles exhibited in the text, we leverage Moral Foundations Theory (MFT)~\cite{graham2013moral}. MFT classifies human behavior into five fundamental principles that represent contrasting values: % (virtues and vices):%\footnote{\url{moralfoundations.org/}}:

\noindent
\textbf{Care/harm}: This principle is based on our general dislike of suffering, whether for ourselves or others, related to our evolution as mammals. It underlies the virtues of kindness, compassion, and gentleness, and it condemns cruelty and aggression.

\noindent
\textbf{Fairness/cheating}: This principle  %revolves around the ideas of 
relates to justice and rights, and is linked to the evolutionary process of reciprocal altruism.

\noindent
\textbf{Loyalty/betrayal}: This principle promotes patriotism, heroism, trust, and self-sacrifice for the group, rooted in our history as tribal beings. It values the virtue of "One for all, and all for one," while considering acts of betrayal towards social structures as immoral.
%This principle generates notions of patriotism, heroism, trust, and self-sacrifice for the group, stemming from our history as tribal beings. It views ``One for all, and all for one'' as virtuous, while acts of betrayal towards social structures are seen as immoral.

\noindent
\textbf{Authority/subversion}: This principle is shaped by our primate history of hierarchical social structure, resulting in virtues such as leadership, followership, and deference to authority/traditions. It may view dissent against authority as immoral.

\noindent
\textbf{Sanctity/degradation}: This principle arises from the psychology of disgust and contamination, where the concept of humans striving to live in a noble and elevated way is the key.  It reflects the %notion that the body is a temple that can be contaminated by immoral activities.
idea of the body as a temple that can be corrupted by immorality.

% A concise version of these definitions is available in Table \ref{tab:orgDefs}. 

%We use these definitions to study morality in SE communications, but since these are too broad and general, we have to adapt them to the context of SE. The process of adaptation of these definitions is explained in the next section.

\begin{comment}
% #######################################
% \newcolumntype{r}{>{\hsize=1.3\hsize}X}
\newcolumntype{t}{>{\hsize=.3\hsize}X}
% \newcolumntype{y}{>{\hsize=.35\hsize}X}
\newcolumntype{q}{>{\hsize=.7\hsize}X}


\begin{table}[h]
\small
% \setlength{\tabcolsep}{4pt}
\resizebox{\columnwidth}{!}{%
\begin{tabularx}{\columnwidth}{|t|q|}
\hline
\textbf{Primary} & \textbf{Secondary}\\
\hline
Insulting & Curse words, Profanities, Vulgarity\\
\cline{2-2}
                 & Communicative Aggression, Verbal abuse, Hate Speech, Offensive or Inappropriate Name Calling\\
\cline{2-2}
                 & Identity Attacks (Race, Religion, Nationality, Gender or Sexual-oriented Attacks)\\
\hline
Entitled & Commanding\\
\hline
Unprofessional & Threats\\
\cline{2-2}
& Sexual activities references, Flirtations\\
\cline{2-2}
& Object-directed toxicity\\
\cline{2-2}
& Irony, Mocking\\
\cline{2-2}
& Bitter frustration, Impatience, Complaining\\
\hline
Arrogant & Dissatisfaction, Oppression\\
\hline
Trolling & Joking\\
\hline
\end{tabularx}
}
\caption{Types of Toxicity}
\label{tab: ToxicTypes}
\end{table}
%#######################################

\section{Toxicity in SE}
%\subsection{Types of Toxicity}
Analyzing toxic communications in Open Source has gathered recent attention. Researchers have studied toxic behaviors  of different forms as follows: 

%The different types of toxicity observed in related works are:
\begin{itemize}[leftmargin=*]
    \item Sarker et. al. \cite{sarkerBenchmark2020,sarkerAutomated2022}:
    Curse words (profanities), Acronym of profanities (e.g., WTF), Offensive name calling, Insults, Identity attacks (race, religion, nationality, gender or sexual orientation), Threats, Sexual activities references, Flirtations
    % \item Sarker et. al. \cite{sarkerBenchmark2020}: Curse words (profanities), Acronym of profanities (e.g., WTF), Insults, Identity attacks (race, religion, nationality, gender or sexual orientation), Threats, Sexual activities references, Flirtations
    % \item Sarker et. al. \cite{sarkerAutomated2022}: Offensive name calling, Insults, Threats, Personal attacks, Flirtations, Reference to sexual activities, Swearing or cursing
    \item Cohen et. al. \cite{cohenContextualizing2021}: Complaining, Entitled, Troll, Joking, Aggressive
    \item Ferreira et. al. \cite{ferreiraSTFU2021, ferreiraHeat2022}: Positive tones (Considerateness, Appreciation, Excitement, Humility), 
    Neutral tones (Expectation, Confusion, Sincere apologies, Friendly joke, Hope to get feedback),
    Negative/uncivil tones (Commanding, Sadness, Oppression, Dissatisfaction, Criticizing oppression, Annoyance, Bitter frustration, Name calling, Mocking, Irony, Impatience, Vulgarity, Threat)
    \item Kou et. al. \cite{kouToxic2020} (toxicity in online video games): Communicative aggression (verbal abuse, hate speech, and offensive or inappropriate name), Cheating, Hostage holding (keeping others stay in an unpleasant situation), Mediocritizing, and Sabotaging
\end{itemize}

%As we were doing our literature review, we came across many definitions and types of toxicity in related works. So we decided to gather the different categories, remove the repeated ones, and categorize the similar ones into a  hierarchy. 
We compiled the different forms of toxicity as observed in related works, eliminated duplications, and organized the similar ones into a structured hierarchy. This hierarchical framework consists of two levels: Primary, Secondary, as shown in Table \ref{tab: ToxicTypes}. 

We use Miller et. al.'s categories \cite{Miller2022} at the primary level because we believe they are  comprehensive and suitable for our research. Next, we assigned the other types of toxic behaviors into secondary type based on their definitions in the papers. 
%After analyzing each type of toxicity, and comparing them with our primary category of toxicity, we came up with the categorization given in table \ref{tab: ToxicTypes}. 
For example, we assigned curse words, communicative aggression, and identity attacks as the secondary level of toxicity for the \textit{Insulting} category, because they exhibit variations of the same form of toxicity.
% Vulgarity and name calling could be categorized as types of profanities, and thus put at the tertiary level for \textit{Insulting}. 
%the same nature based on their definitions. 

%"Commanding" is another type of toxicity which is defined as: "...appears in sentences that issue a command, instructions, or a request in an abrupt way. Someone might also ask rhetorical questions to express an order or command..." \cite{ferreiraSTFU2021}. This definition is in line with the definition of "Entitlement" given in \cite{Miller2022}: "...entitled comments make demands of people or projects as if the author had an expectation due to a contractual relationship or payment". Therefore, we put "Commanding" as the secondary type of entitlement. The same kind of methodology is used to categorize the other types as well.
\end{comment}


\vspace{-6pt}
\section{Methodology} 
To understand how moral principles manifest in an OSS project's textual communication and its relation to toxicity, we analyze the GitHub developer interactions. 
Our focus is on GitHub heated issues, which often result from conflicts, as these issues can provide valuable insights into the ethical and moral considerations of participants.
% In order to understand how moral principles are exhibited in the textual communications of OSS projects,% and to perform our analysis, 
% we examine the interactions among GitHub developer. 
% More specifically, %We focus on examining moral principles 
% we believe that GitHub locked issues, which often result in issue locking due to conflicts, can provide valuable insights into moral considerations among participants and its connection to toxic language.  
%We specifically focus on GitHub locked issues as we hypothesize that it is the best place to look for morality principles as people engage in discussions that lead to locking the issue based on some conflict. 
It should be noted that we are only analyzing textual representations of moral values, and our analysis does not represent the moral values of %those we are not trying to scrutinize the personal moral values of 
people engaging in these discussions.


%In order to study the correlation between toxicity and morality in the context of SE communications, we qualitatively analyze a dataset of 100 toxic issue comments, curated by Miller et. al. \cite{Miller2022}
% We developed a suite of techniques for automatically identifying toxic SE communications. Automatic detection takes as input a text segment, either an issue comment or a code review comment, and classifies them as toxic or non-toxic.



\newcolumntype{b}{>{\hsize=1.3\hsize}X}
\newcolumntype{s}{>{\hsize=.8\hsize}X}
\newcolumntype{k}{>{\hsize=.2\hsize}X}
\definecolor{myblue}{RGB}{0, 103, 148}


\begin{table*}[t]
\centering
\small
\resizebox{\textwidth}{!}{%
\begin{tabularx}{\textwidth}{|k|s|}
\hline
   \textbf{Moral Principle --- Def.} & \textbf{As Exhibited in Issue Threads}
  \\ \hline\hline
   \textcolor{myblue}{Care/harm} --- Protecting versus hurting others & \textbf{Care:} Developer communications exhibiting kindness, e.g., experienced contributors shielding newcomers from criticism. %People taking a kind and gentle approach when communicating in the comments, whether they are contributors or developers. On the contributor side, people higher in the hierarchy showing support and compassion for other contributors, and protecting them against criticism was the observed pattern.
   
   \textbf{Harm:} Developers exchanging insults in issue threads, either in response to feature requests or bug fixes, or due to trolling or experience from past interactions~\cite{Miller2022}). %Developers insulting the contributors through their issue threads when they are asking for a new feature or they want fixes for the bugs, or they insult them completely out of nowhere (mainly because of trolling, but could also be because of ideology or previous interactions \cite{Miller2022}). In addition, the contributors themselves could also insult developers when responding to them. 
   \\ \hline
   
   \textcolor{myblue}{Fairness/cheating} --- Cooperation/ trust/ just versus cheating in interaction with objects and people & \textbf{Fairness:} Developers and contributors respecting one another’s rights (e.g., having the right to open threads and ask for features/fixes, get the help they need). 
   
   \textbf{Cheating:} %In SE communities, both contributors and developers have rights. For developers, in issue comments, it’s having the right to open threads and ask for features/fixes and get the help they need, and for contributors, it’s having the right to develop the project on their own terms and timeline. 
   Disregarding developers' rights by imposing unrealistic expectations (e.g., unfeasible project timeline), or by discriminating against them (e.g., based on gender) and not addressing their concerns.~\cite{vanbreukelen2023still}. 
   %Developers commanding contributors to do certain things and having unrealistic expectations from them counts as disregarding their rights. In addition, when contributors don’t solve developers’ issues based on their unfair reasons counts as cheating.
\\ \hline
   \textcolor{myblue}{Loyalty/betrayal} --- Ingroup commitment (to coalitions, teams, brands) versus leaving group & \textbf{Loyalty:} Acts of self-sacrifice and altruism for the community, ranging from dedicating time to participating in issues threads to defending developers and the project itself against criticisms.
   % “one for all, and all for one”. 
   
   \textbf{Betrayal:} OSS communities, like all social communities, view acts of betrayal as a breach of community morale. Betrayal is observed when developers urge others to abandon a project, or exclude them from the community. 
   %GitHub and OSS communities function like any other kind of social community in our lives (more or less). Any act of betrayal and rebellion toward the community counts as betrayal morality. Developers encouraging other developers to leave the community and use another project instead of the one they are opening an issue thread for, and contributors excluding developers from the project and community was the observed pattern of betrayal in the communications.
\\ \hline
   \textcolor{myblue}{Authority/subversion} --- Adhering to the rules of hierarchy versus challenging hierarchies & \textbf{Authority:} OSS communities often have guidelines that must be followed, e.g., Code of Conduct: Authorities enforcing CoC or any admin-privileged acts (minimizing comments, closing issues, etc.) to ban or censor the users. %Like any other community, OSS communities often have some rules and guidelines that everyone has to abide by. In the case of GitHub, it’s respecting the rules of the Code of Conduct (CoC). 
   
   \textbf{Subversion:} Developers trying to rebel against authority, and questioning contributors’ ability to lead the community.\\ 
   
   \hline
   \textcolor{myblue}{Sanctity/degradation} --- Behavioral immune system versus spontaneous reaction & \textbf{Sanctity:} Communicating with temperance and respect for one another, and encouraging others to do so as well. %Mostly seen in project members' communications. 
   
   \textbf{Degradation:} Developers expressing their hatred (disgust) toward a certain package, system, or code in the project. %, mostly because of their personal preference. 
   This behavior is targeted at code (objects) rather than people. % (in which case becomes harm morality).
   \\
  \hline
\end{tabularx}
}
\caption{Moral Principles in GitHub Heated Issues}
\label{tab:orgDefs}
\vspace{-0.8cm}
\end{table*}

% \begin{table*}[t]
% \centering
% \resizebox{\textwidth}{!}{%
% \begin{tabularx}{\textwidth}{lXX}
% \hline
%    \textbf{Moral Category} & \textbf{SE Domain Definition (First half)} & \textbf{SE Domain Definition (Finalized)}
%   \\ \hline\hline
%    Care/harm & It translates to people who either want to help and provide solutions, or the people who communicate in a cruel, ungentle way (maybe arrogance) and criticize other members.  & \\ \hline
%    Fairness/cheating & People who use reasoning to respond and solve the issues and are generally just, or people who discriminate and are unfair in their comments. &\\ \hline
%    Loyalty/betrayal (Ingroup) & People who try to keep the community together by either providing solutions, sacrificing their time, etc, or people who try to rebel against the community (individualism) &\\ \hline
%    Authority/subversion & People who adhere to the roles and policies of GitHub (CoC) or the project, or people who disobey the roles and feel entitled to have things their own way\\ \hline
%    Sanctity/degradation (Purity) & Still not sure how it translates to SE, but in general, disgust plays an important role. Disgust from mistakes, corruption, etc. Maybe, people who compliment the project VS. people who criticize the authors of the code or the faults and mistakes in the code/project? &
% \\
%   \hline
% \end{tabularx}
% }
% \caption{SE definitions of moral foundations}
% \label{tab:adaptDefs}
% \end{table*}

%\subsection{Dataset} \label{datasets}
\noindent
\textbf{Dataset.}
%We leverage a benchmark dataset of 100 GitHub toxic communications in \textit{issue comment threads} curated by Miller et al. \cite{Miller2022}. Toxicity is defined as ``rude, disrespectful, or unreasonable language that is likely to make someone leave a discussion''.
We analyze a benchmark dataset of 100 toxic GitHub issue comment threads curated by Miller et al. \cite{Miller2022}, in which toxicity is defined as "rude, disrespectful, or unreasonable language that is likely to make someone leave a discussion". 
The dataset includes information on the issue link, author, trigger, target, and the nature of toxicity for each thread of comments. Out of the 100 issue threads, 20 contain comments that were either removed or the project along with its issues were taken down from GitHub.
%As we are investigating the relationship between toxicity and morality, it is important that we use a dataset that has identified the nature and common triggers of toxicity in the comments.
%a reliable source that has identified toxic issues, and preferably, also has identified the nature and type of the toxicity in the comments. 
The 5 categories of toxic instances in this dataset are: \textit{Insulting}, \textit{Entitled}, \textit{Arrogant}, \textit{Trolling}, and \textit{Unprofessional}.

%\subsection{Procedure}
\noindent
\textbf{Procedure.}
To understand the characteristics and patterns of morality in the context of SE, we qualitatively analyze 100 issue threads and their context,
% following the method used in \cite{Miller2022},
using thematic analysis \cite{BraunThematicAnalysis}, and following the trustworthiness criteria (credibility, transferability, dependability, and confirmability) as recommended by previous research \cite{lincolnNaturalistic1985, nowellThematic2017}. 
The data analysis process was conducted iteratively and reflectively, following recommendations from qualitative analysis~\cite{qual}. %Initially, all issue comment threads were thoroughly read to gain a better understanding of the interactions among developers and the contextual circumstances of their communication. 
To enhance our understanding of the developers' interactions and the contextual factors influencing their communication, we first carefully read through all the issue comment threads. Next, the goal was to associate moral values (as defined in Section \ref{moraldef}) to each comment in the threads, if applicable. Given the substantial differences between OSS communications and other social media platforms such as Twitter and Reddit, we adapted the original definitions of morality to align them with the context of software engineering.
%Next, the goal was to create an initial set of mappings of moral values (as defined in Section \ref{moraldef}) in the context of software engineering (SE). Since OSS communications differ significantly from other social platforms like Twitter and Reddit, we adapted the original morality definitions to the software engineering context. 
Specifically, we identify key concepts of each morality type, observe patterns in the dataset, and use them to create the mapping of morality categories. 
%This process also involved identifying the presence or absence of moral values in each data instance. 
Throughout the analysis, careful notes were taken to capture patterns of morality observed in the comments. This process was repeated multiple times, with constant additions to the notes and observations, and ongoing reevaluation of the mappings in each iteration.

Note that in the process of labeling morality to the issue comments, we consciously disregard the toxicity labels to mitigate bias in our decisions, so that we do not associate a specific type of toxicity with a certain type of morality.
The result of our mapping between the morality categories and their manifestation in GitHub heated issues is presented in Table \ref{tab:orgDefs}.
%Table \ref{tab:orgDefs} shows the result of this process, i.e., our mapping between the morality categories and how those are exhibited in GitHub locked issues. %The rationale behind every definition is explained in detail with examples in the next section.


%The results that we expect from this analysis are: 1) find accurate definitions for each type of morality in the context of SE, and 2) label each data with the appropriate morality types observed in it. There are 4 steps of our analysis: (1) Manual analysis of issue comments to confirm the type of toxicity present, ensuring data reliability. (2) Identification of the types of morality exhibited in the issue comments (if applicable) based on definitions provided in the previous section. (3) Labeling of each data instance with the observed types of morality. (4) Continuous improvement and refinement of morality definitions in the context of SE, based on ongoing analysis of data, as recommended in qualitative analysis \cite{creswellQualitative2016}. This iterative process ensures that data is labeled based on the most up-to-date definitions.



%After analyzing half of the dataset, we stopped and analyzed our findings again, and after discussing each definition among ourselves, we updated definitions for each type of morality and resumed our 4-staged analysis to finalize labeling and definitions. 

%\subsection{Morality and Toxicity}


% %#######################################
% \begin{table*}[]
% \centering
% \resizebox{\textwidth}{!}{
% \begin{tabular}{|l|l|l|}
% \hline
% \multicolumn{2}{|c|}{\textbf{Dimensions}} & \multicolumn{1}{c|}{\multirow{}{}{\textbf{Explanation}}} \\ \cline{1-2}

% \textbf{Virtue} & \textbf{Vice} & \multicolumn{1}{c|}{} \\ \hline
% \begin{tabular}[c]{@{}l@{}}Care \\ (CareVirtue)\end{tabular}& \begin{tabular}[c]{@{}l@{}} Harm \\ (CareVice)\end{tabular}& Protecting versus hurting others \\ \hline
% \begin{tabular}[c]{@{}l@{}} Fairness \\ (FairnessVirtue)\end{tabular}& \begin{tabular}[c]{@{}l@{}}Cheating \\ (FairnessVice)\end{tabular}& Cooperation/ trust/ just versus cheating in interaction with objects and people \\ \hline
% \begin{tabular}[c]{@{}l@{}}Loyalty \\(IngroupVirtue)\end{tabular}& \begin{tabular}[c]{@{}l@{}}Betrayal \\ (IngroupVice)\end{tabular}& Ingroup commitment (to coalitions, teams, brands) versus leaving group \\ \hline
% \begin{tabular}[c]{@{}l@{}}Authority \\ (AuthorityVirtue)\end{tabular}& \begin{tabular}[c]{@{}l@{}}Subversion \\ (AuthorityVice)\end{tabular} & Playing within the rules of hierarchy versus challenging hierarchies \\ \hline
% \begin{tabular}[c]{@{}l@{}}Purity \\(PurityVirtue)\end{tabular}& \begin{tabular}[c]{@{}l@{}}Degradation \\ (PurityVice)\end{tabular}& Behavioral immune system versus spontaneous reaction \\ \hline
% \end{tabular}%
% }
% \caption{Principles of Moral Foundations Theory}
% \label{tab: wassa_MTF}
% \end{table*}
% %#######################################