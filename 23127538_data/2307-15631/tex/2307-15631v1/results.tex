

%\subsection{Moral Principles and Toxicity in GitHub Issues}
After gathering our observations of the exhibited patterns of morality in GitHub communications, we associated the toxicities presented in our dataset with the moral principles in each issue thread. The results generated for this mapping are shown in Figure \ref{OSSMoralToxic}. Based on this mapping, we observe  that every type of toxicity is associated with at least one form of morality. %However, for each type, there is a more significant co-occurrence among the others. 
Entitlement is mostly associated with the principle of \textit{Fairness/Cheating}, Trolling with \textit{Sanctity/Degradation}, Arrogant to \textit{Loyalty/Betrayal}, Unprofessional with \textit{Sanctity/Degradation}, and Insulting with \textit{Care/Harm}.

The relationship between the concepts of toxicity and moral principles can be observed in their respective definitions. For example, entitlement comes from disregarding people's rights for personal gain. This aligns with the \textit{Fairness/Cheating} definitions given in previous sections. %, making their relation sensible. 
The relationship between \textit{Authority/Subversion} and insulting threads implies that using insults causes the authorities to intervene more in the issues compared to the other types of toxicity.
%Another interesting observation is, %mapping can be observed in insulting comments. \textit{Sanctity/Degradation} and \textit{Care/Harm} are two of the most observed moralities in toxic threads, which was mostly observed as use of insults and profanities targeted at people or code.
The moralities most frequently observed in toxic threads are \textit{Sanctity/Degradation} and \textit{Care/Harm}, which were primarily expressed through insults directed towards code or individuals.

\vspace{-6pt}
\section{Implications and Challenges}
We found that toxic behaviors can be categorized and linked to various types of moral principles, indicating the potential for utilizing moral principles to detect toxicity in OSS communications. %that toxic natures can be categorized and mapped into different types of moral principles, meaning there is a potential in the potential usefulness of moral principles to detect toxicity  % by understanding the moral principles in SE communications. 
To this end, numerous Moral Foundations Dictionaries (MFD) can be integrated into models to detect moral values in SE texts, as demonstrated in other fields~\cite{rezapour2019enhancing,kennedyMoral2022}. 
Recent studies have explored the association between moral values and the sense of belonging in OSS communities~\cite{trinkenreich2023}.
Adapting existing morality dictionaries to SE and incorporating them into detection tools could facilitate a better understanding of several human values in OSS, including toxicity. %, which could potentially aid in promoting positive and inclusive OSS communities.
It is worth noting that our study only examined the association between moral principles and toxicity in OSS communities, and did not explore any potential causal relationships between the two. While our findings suggest that moral principles are associated with toxic interactions, it is possible that other factors may also contribute to toxic behavior in OSS communities. %Therefore, future research should aim to investigate the causal mechanisms underlying the relationship between moral principles and toxicity in OSS communities.

% CHALLENGES
We acknowledge that inferring human values based on textual communication data has its perils. For instance, previous studies show high error rate in detecting developer personalities based on textual data~\cite{vanmilPromises2021, calefatoUsing2022}. We emphasize that we are only analyzing the textual representation of moral principles, and we are not associating morality values with the people involved in the discussions.
Our findings of the exhibited moralities and their association with toxicity are limited to a relatively small dataset of GitHub issue threads. Further analysis on a large and diverse dataset is necessary to generalize the findings.  
Detecting morality even in social contexts (e.g., Twitter) is a hard task, mostly in distinguishing co-occurring moral principles~\cite{hooverMoral2020}. %Therefore, previous studies allow for overlapping labels during the annotation process to associate multiple moral principles to the expressed moralities in the texts \cite{hooverMoral2020}. 
We attempted to address that challenge by mapping and creating clear definitions of moral principles in the domain of SE (Table \ref{tab:orgDefs}).
Another important point to consider is the potential impact of cultural and social differences on the interpretation and expression of moral principles in OSS communities. Our study focused on English texts, and it is possible that different cultures and languages may have different conceptions and expressions of moral principles. Therefore, future research should aim to investigate the role of cultural and linguistic factors in shaping the expression and interpretation of moral principles in OSS communities.



% #######################################
% \newcolumntype{r}{>{\hsize=1.3\hsize}X}
% \newcolumntype{t}{>{\hsize=.3\hsize}X}
% \newcolumntype{y}{>{\hsize=.35\hsize}X}
% \begin{table}[h]
% \setlength{\tabcolsep}{4pt}

% \resizebox{\columnwidth}{!}{%
% \begin{tabularx}{\columnwidth}{|t|t|k|}
% \hline
% \textbf{Primary} & \textbf{Secondary} & \textbf{Tertiary} \\
% \hline
% Insulting & Curse words (Profanities) & Vulgarity, Acronym of Profanities\\
% \cline{2-3}
%                  & Communicative Aggression & Verbal abuse, Hate Speech, Offensive or Inappropriate Name Calling\\
% \cline{2-3}
%                  & Identity Attacks & Race, Religion, Nationality, Gender or Sexual-oriented Attacks\\
% \hline
% Entitled & Commanding & -\\
% \hline
% Unprofessional & Threats & -\\
% \cline{2-3}
% & Sexual activities references & Flirtations\\
% \cline{2-3}
% & Object-directed toxicity & -\\
% \cline{2-3}
% & Irony & Mocking\\
% \cline{2-3}
% & Bitter frustration & Impatience, Complaining\\
% \hline
% Arrogant & - & Dissatisfaction, Oppression\\
% \hline
% Trolling & Joking & -\\
% \hline
% \end{tabularx}
% }
% \caption{Types of Toxicity}
% \label{tab: ToxicTypes}
% \end{table}
%#######################################

