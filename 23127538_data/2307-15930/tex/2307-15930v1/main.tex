\documentclass[runningheads]{llncs}
%
\usepackage[T1]{fontenc}
% T1 fonts will be used to generate the final print and online PDFs,
% so please use T1 fonts in your manuscript whenever possible.
% Other font encondings may result in incorrect characters.
%
\usepackage{booktabs} 
%\usepackage{subcaption} 
\usepackage{hyperref,amsmath,amssymb}
\usepackage[utf8x]{inputenc}
\usepackage{amscd}
\usepackage{amsmath}
%\usepackage{amssymb}
\usepackage{amsthm}

\usepackage[colorinlistoftodos]{todonotes}

\usepackage{thmtools}
\usepackage{thm-restate}
\usepackage{mathtools}
\usepackage[full]{complexity}
\usepackage{longtable}

%\usepackage[usenames,dvipsnames]{xcolor}
\usepackage{xcolor}
% % tables
 \usepackage{array}

\usepackage{bbm}
\usepackage{comment}
\usepackage{enumerate}
\usepackage{floatrow}


\usepackage{parallel,enumitem}

\usepackage{xspace}
\usepackage{paralist}
\usepackage{xifthen}
\usepackage{url}
\usepackage{csquotes}
% \usepackage{graphicx}
\usepackage{wrapfig}
\usepackage{multirow}
\usepackage[binary-units=true]{siunitx}

\usepackage{tikz}
\usetikzlibrary{trees,decorations,arrows,automata,shadows,positioning,plotmarks,backgrounds,shapes}
\usetikzlibrary{calc,matrix,fit,petri,decorations.markings,decorations.pathmorphing,patterns,intersections,decorations.text}
\usepackage{pgfplots}
\usepackage{pgfplotstable}

\tikzstyle{mystate}=[state,inner sep=3pt,minimum size=20pt,line width=0.2mm]
\tikzstyle{fstate}=[state,accepting,inner sep=2pt,minimum size=3pt]
\tikzstyle{istate}=[state,initial,inner sep=2pt,minimum size=3pt]
\tikzstyle{mysquare}=[inner sep=3pt,minimum size=15pt,line width=0.2mm]
\tikzstyle{fmysquare}=[inner sep=3pt,minimum size=15pt,line width=0.5mm,accepting]
\newcommand{\SFSAutomatEdge}[5]{\path[->](#1) edge[#4,line width=0.2mm] node[#5] {\ensuremath{#2}} (#3);}
\usepackage{subcaption}
\usepackage{tabularx}
\usepackage{booktabs}
\usepackage{xfrac}

\usepackage{etoc}
\etocsettocdepth{3}

% \usepackage{minitoc}

% \usepackage{titletoc}
% 
% \newcommand\DoToC{%
%   \startcontents
%   \printcontents{}{2}{\textbf{Contents}\vskip3pt\hrule\vskip5pt}
%   \vskip3pt\hrule\vskip5pt
% }

\usepackage{microtype}
\usepackage{graphicx}

\usepackage{lineno}
%%\linenumbers

\newcommand\calF{\mathcal{F}}
\newcommand\calG{\mathcal{G}}
\newcommand\calM{\mathcal{M}}
\newcommand\calV{\mathcal{V}}
\newcommand\calU{\mathcal{U}}
\newcommand\calW{\mathcal{W}}
\newcommand\calP{\mathcal{P}}
\newcommand\calD{\mathbb{D}}
%%%%%%%%%%%%%%%%%
%% macros introduced by Luke 
\newcommand\mydef[1]{{\bf\em #1}}
%%%%%%%%%%%%%%%%%

\newcommand{\numviparams}{{| \lambda |}}
\newcommand{\scoreaccvars}[1]{s_1^{#1}, \ldots, s_{\numviparams}^{#1}}
\newcommand{\scoreaccvar}[2]{s_{#1}^{#2}}
\newcommand{\isdeterm}[1]{\text{Deterministic}({#1})}


\newcommand{\expect}[1]{\mathbb{E}\left[{#1}\right]}
\newcommand{\var}[1]{\mathbb{V}\left[ {#1} \right]}
\newcommand{\expectdist}[2]{\mathbb{E}_{#1}\left[ {#2} \right]}
\newcommand{\vardist}[2]{\mathbb{V}_{#1}\left[ {#2} \right]}
\newcommand{\cov}[2]{\mathbb{C}\text{ov}[{#1}][{#2}]}
\newcommand{\covv}[1]{\mathbb{C}\text{ov}[{#1}]}
\newcommand{\corr}[1]{\mathbb{C}\text{orr}[{#1}]}

\newcommand{\fix}[1]{\mathit{fix}\left({#1}\right)}
\newcommand{\sbr}[1]{\left\llbracket {#1} \right\rrbracket}
\newcommand{\ctxtype}[3]{{#1} \cong_\text{ctx} {#2} : {#3}}
\newcommand{\bigstep}[3]{{#1} \Downarrow_{#2} {#3}}


% PCF types
\newcommand{\bool}{\mathit{bool}}
\newcommand{\nat}{\mathit{nat}}

\newcommand{\ctx}[1]{\mathcal{C}\left[ {#1}\right] }
\newcommand{\pcft}[1]{\text{PCF}_{#1}}

\newcommand{\nfl}{\mathbb{N}_\bot}
\newcommand{\bfl}{\mathbb{B}_\bot}

% PCF constructs
\newcommand{\succc}[1]{\mathbf{succ}({#1})}
\newcommand{\succcn}[2]{\mathbf{succ}^{#1}({#2})}
\newcommand{\zero}{\mathbf{0}}
\newcommand{\zerotest}[1]{\mathbf{zero}\left({#1}\right)}
\newcommand{\pred}[1]{\mathbf{pred}\left( {#1} \right)}
\newcommand{\predn}[2]{\mathbf{pred}^{#1}\left( {#2} \right)}
\def\solvable{\#}

\newcommand{\true}{\mathbf{true}}
\newcommand{\false}{\mathbf{false}}
\newcommand{\pcffix}[1]{\mathbf{fix}\left({#1}\right)}
\newcommand{\pcffn}[3]{\mathbf{fn}~{#1}:{#2}\mathpunct{.}{#3}}
\newcommand{\pairtype}[2]{{#1} * {#2}}
\newcommand{\pairexp}[2]{\mathbf{pair}({#1}, {#2})}
\newcommand{\leftexp}[1]{\mathbf{left}({#1})}
\newcommand{\rightexp}[1]{\mathbf{right}({#1})}

\newcommand{\RationalPos}{\mathbb{Q}^{+}}

\newcommand{\meas}[1]{\mathbb{M}\left( {#1} \right) }
\newcommand{\integ}[1]{\sbr{#1}_I}

\newcommand{\notbigstep}[2]{{#1}~\cancel{\Downarrow}_{#2}}
\newcommand{\subtrace}[3]{{#1}^{{#2} \ldots {#3}}}
\newcommand{\supp}[1]{\textsf{supp}\left({#1}\right)}
\newcommand{\dom}[1]{\textsf{Dom}\left({#1}\right)}
\newcommand{\suppk}[2]{\textsf{Supp}^{#1}\left({#2}\right)}
\newcommand{\tracespace}{\bigcup_{n \in \mathbb{N}}[0, 1]^n}
\newcommand{\generictracespace}{\mathbb{T}}
\newcommand{\nnreals}{\mathbb{R}_{\geq 0}}
\newcommand{\posreals}{\mathbb{R}_{> 0}}
\newcommand{\reals}{\mathbb{R}}

\newcommand{\unrollkM}[2]{\textsf{unroll}_{#1}\left({#2}\right)}
\newcommand{\nphmcint}[5]{\Psi_\textsf{NP}\left({#1}, {#2}, {#3}, {#4}, {#5}\right)}

%SPCF constructs
\newcommand{\spcfvalues}{\Lambda^0_v}

\newcommand{\prevalueM}[1]{\textsf{value}^{-1}_{#1}(\spcfvalues{})}
\newcommand{\num}[1]{\underline{#1}}

% \theoremstyle{definition}
% \newtheorem{thm}{Theorem}
% \newtheorem{lem}{Lemma}
% \newtheorem{defn}{Definition}
% \newtheorem{conj}{Conjecture}
% \newtheorem{prop}{Proposition}

%\theoremstyle{definition}
%\newtheorem{defn}{Definition}[section]
%\newtheorem{example}[defn]{Example}
%
%
%\theoremstyle{plain}
%\newtheorem{thm}{Theorem}[section]
%\newtheorem{lem}[thm]{Lemma}
%\newtheorem{cor}[thm]{Corollary}
%\newtheorem{conj}[thm]{Conjecture}
%\newtheorem{prop}[thm]{Proposition}
%\newtheorem{remark}[thm]{Remark}

%% Proofs
%\let\oldproof\proof
%\renewcommand{\proof}{\color{blue}\oldproof}


\definecolor{codegreen}{rgb}{0,0.6,0}
\definecolor{codegray}{rgb}{0.5,0.5,0.5}
\definecolor{codepurple}{rgb}{0.58,0,0.82}
\definecolor{backcolour}{rgb}{0.95,0.95,0.92}

\lstdefinestyle{myStyle}{
    belowcaptionskip=1\baselineskip,
    breaklines=true,
    frame=none,
    basicstyle=\footnotesize\ttfamily,
    keywordstyle=\bfseries\color{green!40!black},
    commentstyle=\itshape\color{purple!40!black},
    identifierstyle=\color{blue},
    backgroundcolor=\color{gray!10!white},
    %backgroundcolor=\color{backcolour}, 
    numberstyle=\tiny\color{codegray},
    stringstyle=\color{codepurple},
    breakatwhitespace=false,                          
    keepspaces=true,                 
    numbers=left,       
    numbersep=5pt,                  
    showspaces=false,                
    showstringspaces=false,
    showtabs=false,                  
    tabsize=2,
}

% argmin/argmax
\DeclareMathOperator*{\argmax}{arg\,max}
\DeclareMathOperator*{\argmin}{arg\,min}

% Concatenation of lists
\newcommand\doubleplus{+\kern-1.3ex+\kern0.8ex}

% Program configurations
\newcommand{\tuple}[1]{\ensuremath{\langle #1 \rangle}}
% Rule based definitions
\newcommand{\Rule}[4][]{\ensuremath{\inferrule*[lab={\hypertarget{#2}{(\TirName{#2})}},#1]{#3}{#4}}}

% Calligraphic symbols
\newcommand{\calI}{{\mathcal I}} 
\newcommand{\calT}{{\mathcal T}}

%  Macro for new Y operator.
\newcommand{\yBounded}[3]{\mu^{#1}_{#2}\rvert_{#3}}

%%%%%%%%%%%%%%%%%
 
%%%%%%%%%%%%%%%%%

\newcommand{\expv}{\mathbb{E}}

\newcommand{\combTr}[2]{\left[\begin{matrix}
		#1\\
		#2
	\end{matrix} \right]}

\newcommand{\exType}[2]{\left\{\begin{matrix}
		#1\\
		#2
	\end{matrix} \right\}}
\newcommand{\myint}[1]{ [#1]}
\newcommand{\Uniform}{\ensuremath{\mathrm{Uniform}}}
\newcommand{\Normal}{\ensuremath{\mathrm{normal}}}
\DeclareMathOperator{\abs}{abs}
\DeclareMathOperator{\pdf}{pdf}

\newcommand{\intConf}[1]{\lceil#1\rceil}
\newcommand{\tr}{\boldsymbol{t}}

\newcommand{\sample}{\tt{sample}}
%\newcommand{\fix}{\texttt{fix}}
%\newcommand{\num}[1]{\underline{#1}}
\newcommand{\myif}{\texttt{if}}
\newcommand{\mylet}{\texttt{let} \, }
\newcommand{\myin}{\, \texttt{in} \,}
\newcommand{\mythen}{\, \texttt{then} \,}
\newcommand{\myelse}{\, \texttt{else} \,}
\newcommand{\score}{\tt{score}}
\newcommand{\tick}{\tt{tick}}

\newcommand{\term}{\tt{term}}
\newcommand{\pv}{\mathbf{v}}
\newcommand{\rv}{\mathbf{r}}

\newcommand{\interval}{\mathfrak{I}}

\newcommand{\typeReal}{\textbf{\textsf{R}}}

\newcommand{\symbolInt}{\myint{\cdot}}

\newcommand{\LambdaInterval}{\Lambda_{\interval}}
\newcommand{\LambdaSymbolic}{\Lambda_{\text{sym}}}

\newcommand{\toIntervalTerm}[1]{#1^{2\interval}}

%Others
\newcommand{\Sset}{\mathbb{S}}
\newcommand{\Iset}{\mathbb{I}}
\newcommand{\Rset}{\mathbb{R}}
\newcommand{\Nset}{\mathbb{N}}
\newcommand{\Zset}{\mathbb{Z}}

\newcommand{\Term}{\mathbb{T}}
\newcommand{\prob}{\mathbb{P}}
\newcommand{\expt}{\mathbb{E}}


\newcommand{\Leb}{\tt{Leb}}
\newcommand{\Red}{\tt{Red}}
\newcommand{\cost}{\text{cost}}

%\newcommand{\intervalab}[2]{\underline{[#1,#2]}}
\newcommand{\intervalab}{\underline{[a,b]}}
\newcommand{\interI}{\mathcal{I}}
\newcommand{\trans}{\mathcal{T}}

\newcommand{\iv}{\mathbb{I}}

% Programming language constructs
\newcommand{\lit}[1]{\underline{#1}}
\newcommand{\letIn}[1]{\mathsf{let}\,{#1}\,\mathsf{in}\,}
\newcommand{\fixLam}[2]{\mu {#1} {#2}.}
\newcommand{\ifElse}[3]{\mathsf{if} (#1 \le \num{0}) \, {#2} \,\mathsf{else}\, {#3}}

%%Basic notions
\newcommand{\pspace}{(\Omega,\mathcal{F},\probm)}
\newcommand{\probm}{\mathbb{P}}
\newcommand{\condexpv}[2]{{\expt}{\left[{#1} \mid {#2}\right]}}

\newcommand{\stdConf}[1]{(#1)}
%\newcommand{\intConf}[1]{\lceil#1\rceil}
%\newcommand{\intConf}[1]{(#1)}
%\newcommand{\symConf}[1]{\langle\!\langle  #1 \rangle\!\rangle}
%\newcommand\symPath[1]{(#1)}
\newcommand{\symPath}[1]{\langle\!\langle  #1 \rangle\!\rangle}
\newcommand\symConf[1]{(#1)}

\newcommand{\ifSimple}[3]{\mathsf{if}(#1, #2, #3)}
%\newcommand{\ifElse}[3]{\mathsf{if} (#1 \le 0) \, \allowbreak {#2} \, \allowbreak \mathsf{else}\, {#3}}
%\newcommand{\ifElse}[3]{\ifSimple{#1}{#2}{#3}}

%\newcommand{\trace}{\mathsf{s}}
%
%\newcommand\defn[1]{{\bf \em #1}}
\newcommand{\traces}{\mathbb{T}}
%
%\newcommand{\stdConf}[1]{(#1)}
%%\newcommand{\intConf}[1]{\lceil#1\rceil}
%\newcommand{\intConf}[1]{(#1)}
%%\newcommand{\symConf}[1]{\langle\!\langle  #1 \rangle\!\rangle}
%%\newcommand\symPath[1]{(#1)}
%\newcommand{\symPath}[1]{\langle\!\langle  #1 \rangle\!\rangle}
%\newcommand\symConf[1]{(#1)}

\newcommand{\valueSem}[1]{\mathsf{val}_{#1}} % value (semantics)
\newcommand{\weightSem}[1]{\mathsf{wt}_{#1}} % weight (semantics)
\newcommand{\measureSem}[1]{\llbracket #1 \rrbracket}
\newcommand{\posterior}{\mathsf{posterior}}


%%%%%%%%%
% 
%%%%%%%%
\newcommand{\loc}{\ell}
\newcommand{\locs}{\mathit{L}}
\newcommand{\blocs}{\mathit{L}_{\mathrm{b}}}

\newcommand{\iflocs}{\mathit{L}_{\mathrm{if}}}
\newcommand{\looplocs}{\mathit{L}_{\mathrm{while}}}

\newcommand{\alocs}{\mathit{L}_{\mathrm{a}}}
\newcommand{\wlocs}{\mathit{L}_{\mathrm{w}}}
\newcommand{\rlocs}{\mathit{L}_{\mathrm{r}}}
\newcommand{\Alocs}[1]{\mathit{L}_{\mathrm{A}}^{\mathsf{#1}}}
\newcommand{\Dlocs}{\mathit{L}_{\mathrm{nd}}}
\newcommand{\transitions}{{\rightarrow}}

%%% 
\newcommand{\plocs}{\mathit{L}_{\mathrm{p}}}
\newcommand{\tlocs}{\mathit{L}_{\mathrm{t}}}

\newcommand{\lin}{\loc_\mathrm{init}}
\newcommand{\lout}{\loc_\mathrm{out}}
\newcommand{\val}[1]{\mbox{\sl Val}_{#1}}

\newcommand{\pvars}{V_\mathrm{p}}
\newcommand{\rvars}{V_{\mathrm{r}}}
\newcommand{\pre}{\mathrm{pre}}

\newcommand{\sle}{\sqsubseteq}
\newcommand{\sge}{\sqsupseteq}

\newcommand{\lfp}{\mathrm{lfp}}
\newcommand{\gfp}{\mathrm{gfp}}

\newcommand{\rdvarjdis}{\mathcal D}
\newcommand{\sampset}{\textit{supp}}

\newcommand{\upd}{\mbox{\sl upd}}
\newcommand{\wet}{\mbox{\sl wt}}
\newcommand{\transset}{\mathfrak T}
\newcommand{\valin}{\pv_{\mathrm{init}}}
\newcommand{\ret}{\mbox{\sl ret}}

\newcommand{\win}{w_{\mathrm{init}}}

\newcommand{\sampdpd}{\overline{\Upsilon}}

\newcommand{\outmap}{\text{O}}
\newcommand{\sat}[1]{\langle #1 \rangle}
\newcommand{\monoid}{\mbox{\sl Monoid}}
\newcommand{\handelmanformat}{(\dagger)}

\newcommand{\trunc}{\mathcal{B}}

\newcommand{\ewt}{\mbox{\sl ewt}}
\newcommand{\statemap}{\text{St}}

\newcommand{\valrd}{{\mathbf{r}}}
\newcommand{\frmloc}{\ell^{\mathrm{src}}}
\newcommand{\toloc}{\ell^{\mathrm{dst}}}

\newcommand{\monomials}{\mathbf{M}}

\setlength{\textfloatsep}{1em plus 1.0pt minus 2.0pt}

%% KOSTIS: Do NOT use Natbib with LLNCS!
%\usepackage{Natbib}
\newcommand{\citet}[1]{\cite{#1}}


% Used for displaying a sample figure. If possible, figure files should
% be included in EPS format.
%
% If you use the hyperref package, please uncomment the following two lines
% to display URLs in blue roman font according to Springer's eBook style:
%\usepackage{color}
%\renewcommand\UrlFont{\color{blue}\rmfamily}
%
\sloppy

\begin{document}
%
\title{Tailoring Stateless Model Checking for Event-Driven Multi-Threaded Programs}
%
\titlerunning{Tailoring Stateless Model Checking for Event-Driven Programs}
% If the paper title is too long for the running head, you can set
% an abbreviated paper title here
%
\author{Parosh Aziz Abdulla\inst{1} \and Mohammed Faouzi Atig\inst{1}
		\and Frederik Meyer Bønneland\inst{2} \and Sarbojit Das\inst{1}
		\and Bengt Jonsson\inst{1} \and Magnus Lång\inst{1}
		\and Konstantinos Sagonas\inst{1,3}}
\institute{Uppsala University, Uppsala, Sweden \and Aalborg University, Denmark \and NTUA, Grece
}
\authorrunning{P. Abdulla \and M. Atig \and F. Bønneland \and S. Das \and B. Jonsson \and M. Lång \and K. Sagonas}


%% \author{First Author\inst{1}\orcidID{0000-1111-2222-3333} \and
%% Second Author\inst{2,3}\orcidID{1111-2222-3333-4444} \and
%% Third Author\inst{3}\orcidID{2222--3333-4444-5555}}
%% %
%% \authorrunning{F. Author et al.}
%% \institute{Princeton University, Princeton NJ 08544, USA \and
%% Springer Heidelberg, Tiergartenstr. 17, 69121 Heidelberg, Germany
%% \email{lncs@springer.com}\\
%% \url{http://www.springer.com/gp/computer-science/lncs} \and
%% ABC Institute, Rupert-Karls-University Heidelberg, Heidelberg, Germany\\
%% \email{\{abc,lncs\}@uni-heidelberg.de}}
%
\maketitle              % typeset the header of the contribution
%
\begin{abstract}
Event-driven multi-threaded programming is an important idiom for structuring concurrent computations.
Stateless Model Checking (SMC) is an effective verification technique for multi-threaded programs, especially when coupled with Dynamic Partial Order Reduction (DPOR).
%
Existing SMC techniques are often ineffective in handling event-driven programs,
since they will typically explore all possible orderings of event processing, even when events do not conflict.
We present \EventDPOR, a DPOR algorithm tailored to event-driven multi-threaded programs.
It is based on \OptimalDPOR, an optimal DPOR algorithm for multi-threaded programs; we show how it can be extended for event-driven programs.
We prove correctness of \EventDPOR for all programs, and optimality for a large subclass.
One complication is that an operation in \EventDPOR, which checks for redundancy of new executions, is NP-hard,
as we show in this paper; we address this by a sequence of inexpensive (but incomplete) tests which check for redundancy efficiently.
%
Our implementation and experimental evaluation show that, in comparison with other tools in which handler threads are simulated using locks, \EventDPOR can be exponentially faster than other state-of-the-art DPOR algorithms on a variety of programs and
% \old{, unlike other algorithms that can achieve analogous reduction,}

%% \keywords{dynamic partial order reduction \and software model checking \and event-driven programs \and systematic testing \and wakeup trees}
\end{abstract}
%

% Figure environment removed

\section{Introduction}
Automatic 3D reconstruction of clothed humans using image inputs has gained increasing significance due to its potential applications in a wide array of AR/VR scenarios. High-fidelity reconstructions typically depend on sophisticated capture systems, which are developed with dense camera arrays~\cite{collet2015high,joo2015panoptic,joo2018total}, programmable light-stages~\cite{Vlasic2009, guo2019relightables}, and depth sensors~\cite{newcombe2011kinectfusion,DoubleFusion,BodyFusion,dou2016fusion4d,newcombe2015dynamicfusion}. However, stringent capture environments equipped with complex hardware pose significant challenges for consumer-level applications.


In this context, considerable research effort has been dedicated to developing methods that allow for more flexible capture configurations, such as utilizing a few RGB inputs. Among these works, learning implicit functions \cite{iccv2020PIFu, saito2020pifuhd, hong2021stereopifu} has proven effective in achieving highly detailed reconstructions by integrating the advancements of deep neural networks. These methods employ large multi-layer perceptrons (MLPs) to predict the occupancy probability or truncated signed distance function (TSDF) value of every queried 3D point based on its associated local feature, which is extracted from images. They can recover a continuous surface at arbitrary resolutions without topology restrictions.


However, in typical MLP-based implicit networks, the occupancy or TSDF value at each location is solved independently with planar image features, rendering them less capable of addressing challenging cases such as occlusions. Consequently, these methods suffer from generalization and robustness issues, particularly when tackling strong occlusions caused by large motion or multiple interacting humans. 
Some follow-up studies  \cite{zheng2021deepmulticap,zheng2021pamir,huang2020arch} utilize an extra geometric model, SMPL~\cite{Loper2015}, to improve robustness by introducing strong shape priors. 
Their success typically relies on the assumption of geometrical similarity \cite{huang2020arch} between the shape prior and target reconstruction, making them intractable for handling complex cases with loose clothes and sensitive to errors in SMPL model fitting.



%\ping{this paragraph sounds like `TSDF is better than MLP/SMPL, and we use TSDF to solve the problem'. But in Sec 3, we are telling a different story, saying `MLP needs a 3D convolutional encoder'. We need to make these two sections consistent.}\sicong{I think in this paragraph we claim that the TSDF}


%We opt for Trucated Signed Distance Funtion (TSDF) volumetric representations as they are naturally suitable for convolution operations, which have shown remarkable performance for learning hierarchical features on 2D visual perception tasks \cite{SunXLW19}. 
%Meanwhile, TSDF also describes the gradual geometry change around shape surface, which is not reflected by occupancy volume. 

We instead revisit the 3D volumetric representation and resort to 3D convolutional neural networks (CNNs) for feature learning, due to their impressive performance in feature learning and the ability to incorporate spatial context. However, volumetric methods and 3D convolution involve discretization, which might raise concerns regarding whether a discretized volume can preserve subtle geometric details as continuous representations learned in implicit functions. We investigate the relationship between volume resolution and quantization error on synthetic data by converting target mesh objects to TSDF volumes, as shown in Figure~\ref{fig:quantization_error}. We observe that the quantization errors are significantly reduced by increasing volume resolution and become nearly negligible when reaching a relatively high resolution (e.g., 512 or higher). In other words, achieving fine-detailed reconstruction is not supposed to be restricted by the use of volume representations as long as a proper volume resolution is utilized. Therefore, we present a method with high-resolution feature volumes, e.g., 256 and 512, while traditional volumetric methods \cite{varol18_bodynet,gilbert2018volumetric} are often limited to much lower resolutions, such as 32 or 128.



On the other hand, an increase in volume resolution may lead to a cubic growth of memory overhead \cite{8100085}. Reducing memory costs while guaranteeing the granularity of volumetric representations is necessary for pursuing high-quality reconstruction. Thus, we adopt a coarse-to-fine approach and cull away irrelevant voxels to build a sparse high-resolution feature volume. At the coarse level, the network computes an initial TSDF by applying a U-Net with sparse 3D CNN \cite{3DSemanticSegmentationWithSubmanifoldSparseConvNet} on the sparse feature volume, which is carved by a visual hull. Through our experiments, it turns out that more than 95\% of the volume grids are discarded by the visual hull culling, making the sparse 3D CNN efficient. At the fine level, the network focuses on a narrow band near the zero-level set of the initial TSDF and discretizes the narrow band with smaller voxels. By employing this narrow-band culling, we further shrink the sampling space, resulting in a relatively small range of grid numbers (usually 300K--500K in our experiments) even with a high volume resolution of 512. The remaining voxels in the narrow band are associated with features that fuse high-frequency information from the computed normal maps upon the low-frequency shape from the coarse level to compute the TSDF at high resolution. The final mesh is then extracted from the TSDF using the Marching-Cube algorithm ~\cite{Lorensen87marchingcubes}.
% Different from the u-net sturcture to preserve global topology context, we then apply a shallow 3dcnn to compute the final TSDF $D_{final}$ which contain more local geometry detail.




% \ping{this paragraph can be expanded. It is an important contribution and often ignored by other works. stress on the novel idea of regressing blending weights instead of colors}

In addition to geometry, high-quality mesh texture is also a crucial factor contributing to visual appearance. Directly computing a color field in 3D space, as in \cite{iccv2020PIFu}, struggles to capture high-frequency texture details, while the neural radiance field (NeRF) \cite{yu2020pixelnerf} or the DoubleField~\cite{shao2022doublefield} require expensive per-instance optimization and are often unstable for sparse input images. In contrast, we adopt an image-based rendering approach to compute a texture atlas map, which is efficient and widely supported in existing computer graphics tools. 
Specifically, we compute a blending weight at each 3D point on the mesh surface to determine its color as a weighted average of the colors at its image projections. The blending weights can be computed at a relatively coarse resolution, e.g., 512 volume resolution in our case, and leave texture details to the high-resolution images, such as 1K or 2K. Unlike previous methods that generate blurry texturing results under sparse input, our method generalizes well on both synthetic and real data with just a few input views. 
Figure~\ref{fig:teaser} shows two examples reconstructed by our method. Despite the challenging garment, pose, and occlusion, our method recovers faithful shape, normal, and texture on the right.

%with a wide variety of poses and clothing styles, and it is also adaptive to handle input image with arbitrary resolutions.
%\sicong{For this concern we claim that when the resolution of dicretized volume meets certain threshold (which is 256 in our experiment), the quantization error can be neglected.} 



In summary, the main contributions of this paper are as follows:
\begin{itemize}
\vspace{-0.1in}
  \item 
  We revisit the 3D volumetric representation and demonstrate that it can support clothed human reconstruction with equal or even better performance compared to implicit representation. 
  \item 
  We develop a memory and computation-efficient method for high-resolution volumetric reconstruction using sophisticated sparse 3D CNN, coarse-to-fine estimation, and voxel culling by visual hull and narrow bands. 
  \item 
  We introduce a novel method to compute a texture atlas map, which captures rich appearance details from high-resolution input images.
  \item 
  We achieve impressive results on standard benchmark datasets Twindom and MultiHuman, significantly reducing the point-2-surface (P2S) precision to approximately 0.2cm from just six input views, with more than $50\%$ error reduction compared to the state-of-the-art methods, including DoubleField~\cite{shao2022doublefield} and PIFuHD~\cite{saito2020pifuhd}.
\end{itemize}
\section{Related Work}
\label{sec:relwork}

There has been relatively little prior work on formal verification of virtual memory.
Instead, much OS verification work has focused on minimizing reasoning about virtual memory management.
The original Verisoft project~\cite{alkassar2008verisoft,alkassar2010pervasive,alkassar2008formal,dalinger2005verification,hillebrand2005address,alkassar2008formal,starostin2010formal} relied on custom hardware which, among other things, always ran kernel code with virtual memory disabled, removing the circularity that is a key challenge of verifying actual virtual memory code: at that point page tables become a basic partial map data structure to represent user program address translations.
Other work on OS verification either never progressed far enough to address VMM verification (Verisoft XT~\cite{cohen2009vcc,cohen2010local,dahlweid2009vcc,cohen2013SOFSEM}), or uses memory-safe languages to enable safe co-habitation of a single address space by all processes (Singularity~\cite{Fahndrich2006language,Hunt2007singularity,Hunt2007sealing,Barnett2011specsharp}, Verve~\cite{Yang2010Verve}, and Tock~\cite{levy2017multiprogramming}).

The work that does address the core challenges of VMM verification is all associated with either \textsc{seL4} or \textsc{CertiKOS}.

\textsc{CertiKOS}~\cite{gu15,gu2016certikos,gu2018certikos,chen2016interrupts} is a microkernel intended for use as a hypervisor,
and its papers do not explicitly detail verification of the VMM, so we do not know the full space of which VMM functionality 
is verified, but we do know it includes the ability to map or unmap pages.
The work is clear, however, that it trusts low-level assembly fragments such as the instruction sequence which actually
switches address spaces, rather than verifying them.
The overall approach in that body of work is many layers of refinement proofs, using a
 proliferation of layers with small differences to keep most individual refinements tractable. In keeping with precursor work 
on the project from the same group~\cite{vaynberg2012compositional}, the purpose of some layers is to abstract away from 
virtual memory, so the proof is essentially a simulation proof covering for example a proof that execution with page-in on 
page faults is a valid refinement of an execution model where no paging occurs.
% Another key aspect of their approach is that the OS is written in Clight and compiled with \textsc{CompCert}~\cite{blazy2006formal,leroy2009formally,leroy2008formal}.
% CompCert's memory abstraction~\cite{leroy2008formal} assumes
% memory is a set of disjoint chunks of bytes with no overlap, so the lowest levels of CertiKOS must provide a matching 
% machine model as a layer. This prohibits virtual address aliasing, so CertiKOS cannot support simultaneous memory-mapped 
% (\texttt{mmap}) and stream-oriented (\texttt{read}/\texttt{write}) IO to a single file\todo{should we go into this detail?}, 
% and cannot use
% the common kernel design choice of mapping all physical memory into the bottom of the kernel's address space for direct access i
% while the kernel code is simultaneously mapped (and executed) at higher virtual addresses.
% This is not necessary for \textsc{CertiKOS}'s intended primary use case (a hypervisor), but means that \textsc{CertiKOS}'s
% approach cannot be used to support this functionality in other systems, without major surgery to \textsc{CompCert}.

\textsc{seL4}~\cite{Klein2009seL4,seL4TOCS,Sewell2013translation} is a formally verified L4 microkernel~\cite{Liedtke1995,Liedtke1996} (and the first verified OS kernel to run on real-world hardware), verified with a mix of refinement proofs and program logic reasoning down to the assembly level.
Because \textsc{seL4} is a microkernel, most VMM functionality actually lives in usermode and is unverified, and moreover, their hardware model omits address translation entirely and the MMU entirely~\cite{Klein2009seL4,seL4TOCS}. As a result, the limited page table management present in the microkernel treats page tables as idiosyncratic tree-maps, ignoring the risks posed by even transient inconsistencies that would crash the kernel on real hardware (like ``temporarily'' unmapping the kernel). This is mitigated primarily by manually identifying some trusted invariants (e.g., that the address range designated for the kernel is appropriately mapped) and setting up the proof to ensure those invariants are maintained (i.e., as an extra proof obligation not required by their hardware model).


One important outgrowth of the \textsc{seL4} project, not integrated into the main project's proof, was work by 
Kolanski and Klein which studied verification of code against a hardware model that \emph{did} include address translation
 --- the only work aside from ours to do so --- initially in terms of basic memory~\cite{kolanski08vstte} and subsequently 
integrating source-level types into the interpretation~\cite{kolanski09tphols}. 
They were the first work to model physical and virtual points-to assertions separately, defining virtual points-to assertions
in terms of physical points-to assertions mimicking page table walks, and defining all of their assertions as predicates on a
pair of (physical) machine memory and a page table root, an approach we improve on.

Their work has a number of significant limitations which our work addresses.
They also define their virtual points-to assertions such that a virtual points-to $p\mapsto_\mathsf{v} a$ owns the full 
lookup path to virtual address $p$. This means that given two virtual points-to assertions at the same time, such as 
$p\mapsto_\mathsf{v}a \ast p'\mapsto_\mathsf{v}b$, the memory locations traversed to translate $p$ and $p'$ must be disjoint. 
This means the logic has a peculiar limit on how many virtual points-to assertions can coexist in a proof. Since page tables 
fan out, the bottleneck is the number of entries in the root table. For their 32-bit ARMv6 example, the top-level address is 
still 4Kb (4096 bytes), and each entry (consumed entirely by a virtual points-to in their scheme) is 4 bytes, so they have a 
maximum of 1024 virtual points-tos in their ARMv6 configuration. Any assertion which implies more than that number
of virtual addresses are mapped implies false in their logic.
(They do formulate their logic over an abstract model, but every architecture would incur a similar limitation;
Na\"ively transferring their model to x86-64 4-level tables would yield a limit of 512 assertions (also a 4Kb root page, 
but 8-byte entries).

% Kolanski and Klein's points-to assertions do not model that page table entries for nearby addresses typically 
% \emph{share} entries in higher layers of the page tables --- a single L1 entry maps 4KB of memory on many architectures,
% but their logics avoid fractional ownership, so they can in fact only use a single memory location per page of memory.
% In fact, because their virtual points-to assertions contain \emph{full} ownership of all entries, even in the highest-level
% page table (L4 in our case), each entry can contribute to only a single mapped address. Thus any assertion
% in their logic that implies there are more virtual addresses mapped than entries in the top-level page table implies false.
Our definitions make use of fractional permissions throughout; Figure \ref{fig:strongvirtualpointsto}'s definition
of \lstinline|L4_L1_PointsTo| ellides the specific fractions used, but it in fact asserts 1/512 ownership of
the L1 entry, 1/($512^2$) of the L2 entry, and so on, so each entry may map the appropriate number of machine words.

As noted earlier, Kolanski and Klein's logics, by collocating both the physical ownership of the page table walk
as part of the virtual points-to itself, preempt support for changes to page tables which do not actually affect 
address translation.

The other major distinction is that Kolanski and Klein have no accounting for other address spaces.
Their logic does not deal with change of address space, and has no way to assert that certain facts hold
in another address space.
They verify only one address space manipulation: mapping a single unmapped page into the current address space (in both papers).
We verify this, as well as a change-of-address-space, which requires us to introduce assertions for talking
about other address spaces (we must know, for example, that the precondition of the code after the change must be true
in the \emph{other} address space), and to deal with the fact that the standard frame rule
for separation logic is unsound in the presence of address space changes and address-space-contingent assertions.
% The mapping write is verified in an ad hoc way by unfolding the machine semantics, because the logic lacks proper reusable rules for
% updating page tables.

Our approach in this paper uses modalities to distinguish virtual-address-based assertions that hold only in specific 
address spaces, making it possible to manipulate other address spaces, and equally critically, to \emph{change} address 
spaces while reasoning about correctness. 

Unlike our work, Kolanski and Klein prove very useful embedding theorems stating that code that does not modify page table 
entries can be verified in a VM-ignorant program logic, and that proofs in that logic can be embedded into the VM-aware logic 
(essentially by interpreting ``normal'' points-to relations as virtual points-to facts). While we have not proven such a result,
an analagous result {should} hold of our work: consider that the doubles for the \texttt{mov} instructions
that access memory behave just as one would expect for a VM-ignorant logic~\cite{Chlipala2013Bedrock}.
With our general approach to virtual points-to assertions being inspired by Kolanski and Klein, \emph{both}
 our approach and theirs could in principle be extended to account for pageable points-to assertions by adding additional 
disjunctions to an extended points-to definition; embedding ``regular'' separation logic into such a variant
is the appropriate next step to extend reasoning to usermode programs running with a kernel that may demand-page the program's
memory.

As noted throughout the paper, the inspiration for our other-space modality comes from hybrid logic~\cite{areces2001hybrid,blackburn1995hybrid,gargov1993modal,goranko1996hierarchies},
where modalities are indexed by \emph{nominals} which are names for specific individual states in a Kripke model.
We are aware of only two prior works combining hybrid logics with program logics specifically. 
Brotherston and Villard~\cite{brotherston2014parametric} demonstrated that may properties true of various 
separation logics are not definable in boolean \BI (\BBI), and showed that a hybrid extension \HyBBI allows
most such properties to be defined (e.g., the fact that separating conjunction is cancellative is unprovable 
in boolean \BI, but provable in \HyBBI). There, nominals named resources 
(roughly, but not exactly, heap fragments). 
Gordon~\cite{gordon2019modal} described a use of hybrid logic in the verification of actor programs, 
where nominals named the local state of individual actors (with such assertions stabilized with a 
rely/guarantee approach). Beyond these, there is limited work on the interaction of specifically 
\emph{hybrid logic} with substructural logics. 
Primarily there is a line of work on hybrid linear logic (\HyLL)~\cite{despeyroux2014hybrid}, 
originally used as a way to more conveniently express aspects of transition systems in linear logic. 
However, \HyLL's proof rules offer no non-trivial interactions with multiplicative connectives 
(every \HyLL proof can in fact be embedded into regular linear logic~\cite{chaudhuri2019hybrid}, 
unlike Brotherston and Villard's \HyBBI, which demonstrably increases expressive power over its base \BBI.

In both \HyLL and \HyBBI, nominals denote worlds with monoidal structure (as worlds in Kripke semantics
for either LL or \BBI necessarily have monoidal structure). Our nominals, by contrast, 
do not name worlds in the same sense with respect to Iris's CMRAs, 
but in fact \emph{classes} of worlds, because the names are locations 
(a means of \emph{selecting} resources) rather than resources.  
A key difference is that the use of nominals in those logics corresponds specifically to hypothetical 
reasoning about resources (until a nominal is connected to a current resource, in which case conclusions 
can be drawn about the current resource), which means the modalities themselves do not ``own'' resources. 
Instead, assertions under our other-space modality can and do
have resource footprints.
Pleasantly, we sidestep most of the metatheoretical complexity of those other substructural hybrid
systems by building our logic within a substructural metatheory (\iris).

\iris has been used to build other logics through pointwise lifting, notably logics that deal with weak
memory models~\cite{dang2019rustbelt,dang2022compass}. Those systems build a derived logic
whose lifting consists of functions from thread-local views of events (an operationalization of the release-acquire + nonatomic
portion of the repaired C11 memory model~\cite{lahav2017repairing}): there modalities $\Delta_\pi(P)$ and $\nabla_\pi(P)$
represent that $P$ held before or will hold after certain memory fence operations by thread $\pi$.
The definitions of those specific modalities existentially quantify over other views, related to the ``current'' view (the one where
the current thread's assertions are evaluated), and evaluate $P$ with respect to those other views. This approach to parameterizing
assertion semantics by a point of evaluation, and evaluating modalized assertions at other points, is what it means
to have a modality at all.
It is \emph{not}, however, an instance of hybrid logic, which is specifically demarcated by an assertion language where
\emph{assertions}, not their semantics, choose and name the evaluation points for modal assertions.
A hybrid extension of the aforementioned logics would include assertions which named specific views at which to evaluate
$P$, in the syntax of the assertion (e.g., $\Delta_\pi^v(P):=\lambda\_\ldotp (P\;v)$) rather than the 
$\Delta_\pi(P):= \lambda v\ldotp (\exists v_{rel}\ldotp \ownGhost{\pi}{\mathsf{RelV}(v_{rel})\;v} \ast (P\;v_{rel})))$ actually used.
Note the hybrid version takes the place to evaluate $P$ as a parameter, and therefore allows the \emph{derived} (modal) logic to explicitly
reason in terms of evaluation points, rather than hiding all points of evaluation in the internal definitions of modalities.



\newcommand{\tikzwrapfigbg}{%
  \begin{pgfonlayer}{background}
    \path[fill=gray!10,rounded corners]
    (current bounding box.south west) rectangle
    (current bounding box.north east);
\end{pgfonlayer}}

\section{Main Concepts and Challenges}
\label{sec:concepts}
In this section, we informally present core concepts of our approach by examples\footnote{Note
  that in the remainder of the paper, we will use the term \emph{message} to refer to what was called \emph{event} in Sections~\ref{sec:intro} and~\ref{sec:related},
  for the reason that the literature on DPOR has reserved the term \emph{event} to denote an execution of a program statement. We will also use \emph{mailbox} instead of event pool.}
%% Recall that henceforth we will use the term \emph{message} to refer to what was called \emph{event} in \cref{sec:intro}.

\hasbeenremoved{By a DPOR algorithm, we refer to an algorithm that analyses a terminating program on given input,
by exploring different executions resulting from different thread interleavings. 
It equips each execution with a transitive happens-before ordering, induced by ordering accesses performed within a message or non-handler thread, as well as conflicting accesses to a shared variable
(two accesses are \emph{conflicting} if they involve the same shared variable and one of them is a write).
The happens-before relation induces an equivalence relation on executions.
A DPOR algorithm should explore at least one execution in each equivalence class.
\hasbeenremoved{
A DPOR algorithm is \emph{correct} if it  explores at least one execution in each equivalence class.
It is \emph{optimal} if, additionally, it explores exactly one execution in each equivalence class.}
}
%% The first DPOR algorithms, starting
%% with~\cite{FG:dpor} used backtrack sets to decorate prefixes of the currently explored execution sequence with alternative continuations.
%% In~\cite{FG:dpor}, these sets consisted of single threads that initiate alternative explorations. Correctness was established by proving that
%% these backtrack sets were persistent sets, which in~\cite{abdulla2014optimal} was improved into source sets. However, as demonstrated in~\cite{abdulla2014optimal},
%% having only single threads in backtrack sets is not sufficient for avoiding to explore the same equivalence class twice, or encountering sleep set blocking during execution.


\newcommand{\evnt}[2]{\mbox{$#1$: \texttt{#2}}}
\newcommand{\hndlr}[3]{\mbox{$#1$: $#2$: \texttt{#3}}}

% Figure environment removed

\subsection{Review of \OptimalDPOR}
Our DPOR algorithm for event-driven programs is an extension of \OptimalDPOR~\cite{abdulla2014optimal}.
%, which is designed for programs consisting of interacting sequential threads.
%% The overall idea of DPOR algorithms is to detect races in explored executions, and reverse them to obtain new executions that will be explored.
%% Given a program, \OptimalDPOR first explores a maximal execution $\exseq$, and then inspects $\exseq$ to find races. From each race, an alternative execution is constructed which branches off from $\exseq$ at the point where the race occurred, and whose continuations produce executions that are inequivalent to $\exseq$. The branch is called a \emph{wakeup sequence}.
Let us illustrate \OptimalDPOR on the program snippet shown in~\cref{fig:rw}.
In this code, three threads $s$, $t$, and~$u$ access
three shared variables \texttt{x}, \texttt{y}, and \texttt{z},\footnote{Throughout
  this paper, we assume that threads are spawned by a \texttt{main} thread,
  and that all shared variables get initialized to $0$, also by the main thread.}
whereas \texttt{a}, \texttt{b}, \texttt{c}, and \texttt{d} are thread-local registers.
%% Two accesses are said to be \emph{conflicting}
%% if they involve the same shared variable and one of them is a write; if they are also adjacent in the
%% dependency ordering, they are in a \emph{race}.
%% Since there are no writes to \texttt{y} and \texttt{z} here,
%% accesses to \texttt{y} and \texttt{z} are not dependent with anything else.
%% This allows DPOR algorithms to explore considerably less executions than the~$30$
%% that a na{\"\i}ve exploration would examine.
%% Similarly to many other DPOR algorithms,
\OptimalDPOR first explores a maximal execution, which it inspects to detect races.
From each race, it constructs an initial fragment of
an alternative execution which reverses the race and branches off from the explored execution just before the race.
%% In \OptimalDPOR, such a fragment is called a \emph{wakeup sequence}.
Let us illustrate with the program in~\cref{fig:rw}.
Assume that the first execution is $\exseq_1$ (cf. the tree in~\cref{fig:rw}).
The DPOR algorithm first computes its happens-before order, denoted $\happbf{\hb}{E_1}$,
which is the transitive closure of the union of:
\begin{inparaenum}[(i)]
\item
  the \emph{program order}, which totally orders the events in each thread (small blue arrows to the left of $E_1$), and
\item
  the \emph{conflict order} which orders conflicting events: two events are conflicting if they access a common shared variable and at least one is a write
  (red arcs left of $E_1$).
\end{inparaenum}
%% Let us define a \emph{happens-before prefix} of $E_1$ as a subsequence of $E_1$ such that if an event $e$ is in $E_1$, then all
%% $\happbf{\hb}{E_1}$-predecessors of $e$ are in $E_1$.
A \emph{race} consists of two conflicting events in different threads that are adjacent in the $\happbf{\hb}{E_1}$-order.
The execution $\exseq_1$ contains two races (red arcs in~\cref{fig:rw}).
Let us consider the first race, in which the first event is \evnt{s}{x=1} and the second event is \evnt{t}{b=x}.
The alternative execution is generated by concatenating 
the sequence of events in $E_1$ that do not succeed the first event in the $\happbf{\hb}{E_1}$ order (i.e., $\evnt{t}{a\,=\,y}; \evnt{u}{c\,=\,z}$) with
the second event of the race \evnt{t}{b=x}.
%% Here, sequence not happening-after $e$ is  and the second event is 
%% $\evnt{t}{a\,=\,y}; \evnt{u}{c\,=\,z}; \evnt{t}{b=x}$ ($t.u.t$ for short), which
This forms a \emph{wakeup sequence}, which branches off from $E_1$ just before
the race, i.e., at the beginning of the exploration (green in~\cref{fig:rw}).
%% the independent events are $\evnt{t}{a\,=\,y}; \evnt{u}{c\,=\,z}$ together with their predecessors; by appending the second event, we form the wakeup sequence
%% $\evnt{t}{a\,=\,y}; \evnt{u}{c\,=\,z}; \evnt{t}{b=x}$ ($t.u.t$ for short), which branches off from $E_1$ just before the race, i.e.,
%% at the beginning of the exploration (green in~\cref{fig:rw} right).
%% The second race, between \evnt{s}{x=1} and~\evnt{u}{d=x} induces the wakeup sequence $t.u.u$ formed from the
%% independent events $\evnt{t}{a\,=\,y}; \evnt{u}{c\,=\,z}$ and the second event \evnt{u}{d\,=\,x}, also branching off at the beginning
%% (note that it does not contain the second step of~$t$ since it succeeds \evnt{s}{x=1} in the $\happbf{\hb}{E_1}$-ordering).
%%
%% The $\happbf{\hb}{E_1}$-order of $E_1$ is 
%% i.e., the order of events in a thread, and
%%   events within a thread or between conflicting accesses to a shared variable). 
%%
%%  it contains two races (marked red in~\cref{fig:rw} right).
%% It is constructed using the following principles.
%% %% that make \OptimalDPOR correct and optimal:
%% \begin{inparaenum}
%% \item[P1:]
%% It should reverse the race, i.e., contain the second event of the race (as its last), but not the first.
%% \item[P2:]
%% It should also includes all events of $E_1$ that are independent of the race, in the sense that they do not succeed the racing events
%% in the $\happbf{\hb}{E_1}$-order, together with all their predecessors in said $\happbf{\hb}{E_1}$-order
%% \end{inparaenum}
%% These principles are designed to make \OptimalDPOR correct and optimal.
%%
%% If the first execution is $\exseq_1$ (cf.~\cref{fig:rw} right), it contains two races (marked red in~\cref{fig:rw} right).
%% %% Assume that the first explored sequence
%% % (We will denote executions by the dotted sequence of scheduled thread steps.)
%% %% It then inspects $\exseq_1$ to detect races; detecting
%% For the race between \evnt{s}{x=1} and~\evnt{t}{b=x},
%% the independent events are $\evnt{t}{a\,=\,y}; \evnt{u}{c\,=\,z}$ together with their predecessors; by appending the second event, we form the wakeup sequence
%% $\evnt{t}{a\,=\,y}; \evnt{u}{c\,=\,z}; \evnt{t}{b=x}$ ($t.u.t$ for short), which branches off from $E_1$ just before the race, i.e.,
%% at the beginning of the exploration (green in~\cref{fig:rw} right).
The second race, between \evnt{s}{x=1} and~\evnt{u}{d=x} induces the wakeup sequence $t.u.u$ formed from the
sequence $\evnt{t}{a\,=\,y}; \evnt{u}{c\,=\,z}$ and the second event \evnt{u}{d\,=\,x}, also branching off at the beginning
(note that $t.u.u$ does not contain the second event \evnt{t}{b=x} of~$t$ since it succeeds \evnt{s}{x=1} in the $\happbf{\hb}{E_1}$-ordering).
%% \mbox{$s$: \texttt{x=1}} and~\mbox{$u$: \texttt{d=x}}, inducing the
When attempting to insert $t.u.u$, the algorithm will discover that this sequence is \emph{redundant}, since its events are
consistently contained in a continuation ($t.u.t.u$) of 
the already inserted wakeup sequence $t.u.t$, and it will therefore not insert $t.u.u$.
After this, the algorithm will reclaim the space for~$E_1$, extend $t.u.t$ into a maximal execution~$E_2$, 
in which races are detected that generate
two new wakeup sequences (which start in green and continue in blue), which are extended to two additional executions (cf.~\cref{fig:rw}).
%% In this way, \OptimalDPOR explores one trace in each equivalence class of this program.

\hasbeenremoved{In short, the \OptimalDPOR algorithm conceptually organizes all its explored execution sequences
in a tree $\exseqs$ that it explores in a depth-first manner (and gradually reclaims).
The nodes of $\exseqs$ correspond to execution sequences; leaves correspond to maximal explored executions and to complete wakeup sequences.}

%%\old{
%% The tree $\exseqs$ has two properties to guarantee optimality:
%% \begin{enumerate}[(1)]
%% \item
%%   Whenever the exploration of some subtree rooted at $\exseq \in \exseqs$ has finished,
%%   then for all maximal executions of form $\exseq.w$, the algorithm has explored a maximal execution $\mtclass{\exseq.w}$.
%% \item
%%   Let $\treeorder$ be the post-order on the nodes of $\exseqs$, induced by the
%%   depth-first exploration of the DPOR algorithm. For each internal node
%%   $\exseq$ in $\exseqs$, for each thread $s$ and sequence $w$ such that
%%   $\exseq.p$ is a node and $\exseq.w$ is a leaf with $\exseq.p \treeorder \exseq.w$,
%%   there is no continuation of $\exseq.p$ which is equivalent to $\exseq.w$.
%% \end{enumerate}
%% Property (1) implies correctness by letting $\exseq$ be the empty sequence.
%% Note that the exploration of some particular execution in $\mtclass{\exseq.w}$ need not be
%% performed during exploration of the subtree rooted at $\exseq$; it can be done during
%% the exploration of some preceding subtree, which also contains executions in
%% $\mtclass{\exseq.w}$.
%% Property (2) guarantees that no equivalence class will be explored twice.
%% For each execution $\exseq$, let us introduce the relation $\inwfirstseqs{E}$
%% on continuations of $E$, defined by letting
%% $v \inwfirstseqs{E} w$ denote that there are sequences $v'$ and
%% $w'$ such that $v.v' \equivafter{E} w.w'$. If $E.w$ is maximal, then it means
%% that there is a sequences $v'$ with $v.v' \equivafter{E} w$.

%% In order to to ensure that the resulting exploration tree will satisfy (2),
%% the new branches created by race analysis are organized in \emph{wakeup trees}.
%% Each prefix $E$ of the currently explored execution has a wakeup tree, denoted
%% $\wut{E}$, which contains extensions of $E$ that are or will be in $\exseqs$,
%% each of which is the root of a subtree in $\exseqs$.
%%
%% If the first thread explored after $E$ is $s$, then each sequence $w$ in
%% $\wut{E}$ must satisfy $s \notinwfirstseqs{E} w$, which guarantees that
%% any continuation of $E.w$ is inequivalent with any
%% continuation of $E.p$. The wakeup three
%% $\wut{E}$ is internally organized as a tree in which
%% this principle is maintained recursively,
%% i.e., whenever $u.p$ and $u.w$ are nodes in $\wut{E}$ with
%% $u.p \treeorder u.w$, and $u.w$ is a leaf,
%% then $s \notinwfirstseqs{E.u}{w}$. Again, 
%% this guarantees that the continuations of any two leaves of $\wut{E}$
%% are inequivalent.
%%}

%% % Figure environment removed

%% To see the ingredients that \EventDPOR requires,
%% consider first the program in \cref{fig:example1}.
%% It is a variation of the previous program, but now messages consist of
%% a single write access to the \emph{same} shared variable \texttt{x}.
%% Assume that the first explored execution is $E_1$; cf.~\cref{fig:example1}.
%% % The trace graph of $E_1$ is shown to the right of the same figure.
%% %% There the $\eom$ relation between the two messages induces the $\ppm$ relation between the two post events.
%% \EventDPOR detects that the two accesses to \texttt{x} are in a race.
%% It then constructs a wakeup sequence which leads to the second access (\texttt{a\,=\,x}) without executing the first. 
%% \EventDPOR infers that such a wakeup sequence can be constructed only by executing $p_2$ instead of $p_1$,
%% obtaining the wakeup sequence shown in blue,
%% which is inserted after the two post events in the exploration tree.
%% %% \EventDPOR needs to detect that messages $p_1$ and $p_2$ contain events that are in a race.
%% %% However, because of the FIFO discipline of message queues,
%% %% we cannot reverse the race simply by creating a new branch after $s.q$
%% %% where the order of messages $p_1$ and $p_2$ is reversed.
%% %% Instead, we have to go back to their posting events, which also have to be reversed.
%% %% Thus, the definition of races between messages (\cref{sec:prels})
%% %% needs to be designed in a way that such reordering is always possible.
%% This wakeup sequence will then be extended to the execution $E_2$.
%% Like in \OptimalDPOR, $E_1$'s memory can be reclaimed at this point.
%% %% and the algorithm will explore execution $E_2$ that starts by this wakeup sequence.
%% (Since this example is very simple, nothing more is added here.)

%% \endgroup % NEEDS TO HAVE AN EMPTY LINE BEFORE IT!

%% % Figure environment removed
%% \paragraph{Handling Conflicting Messages}

%% Let $E$ be the empty sequence $\emptyseq$, and 
%% In our setting, we have
%% \begin{itemize}
%% \item
%%   $q \infirstseqs{\emptyseq}{p.q}$, since $q.p \simeq p.q$,  and
%% \item
%%   $s.q \infirstseqs{\emptyseq}{p.q.m}$ since the former is a prefix of the latter, but
%% \item $q \notinfirstseqs{\emptyseq}{p.q.m}$, since any sequence that starts with $q$ must let $n$ be processed before $m$.
%% \end{itemize}

%% \begin{itemize}
%% \item
%%   whenever $E.p$ and $E.w$ are nodes in $\exseqs$, where $s$ is a single thread,
%%   $E.p \prec E.w$, and $u.w$ is a maximal execution sequence
%%   then $s \notinwfirstseqs{E} w$.
%% \end{itemize}
%% \footnote{\citet{abdulla2014optimal} used the notation
%% $s \in \winits{E}{w}$ to denote $s \inwfirstseqs{E} w$.
%% In contrast, we will employ $s \inwfirstseqs{E} w$, in order to emphasize that
%% our algorithm never actually computes the set $\winits{E}{w}$.}

%% \begingroup
%% \setlength{\intextsep}{0em}%
%% \setlength{\columnsep}{.75em}%
%% Note that the algorithm should also handle programs that include multiple levels of posting events.
%% \begin{wrapfigure}{h}{0.42\textwidth}
%%   \centering \footnotesize
%%   \begin{tikzpicture}[line width=1pt,framed,inner sep=1pt]
%%     \node[name=p,anchor=south west] at (-0.15,0.25) {{$s$}};
%%     \node[name=post1] at (0,0) {$\mathtt{post}(p_1,h)$};
    
%%     \draw[line width=0.5pt] ($(post1.north east)+(1pt,10pt)$)--($(post1.south east)+(1pt,-34pt)$);
%%     \draw[line width=0.5pt] ($(post1.north east)+(3pt,10pt)$)--($(post1.south east)+(3pt,-34pt)$);
    
%%     \node[name=q,anchor=south west] at (1.45,0.25) {{$q$}};
%%     \node[name=post2,anchor=west] at ($(post1.east)+(5pt,0.5pt)$) {$\mathtt{post}(p_2,h)$};

%%     \draw[line width=0.5pt] ($(post2.north east)+(1pt,10pt)$) -- ($(post2.south east)+(1pt,-34pt)$);
%%     \draw[line width=0.5pt] ($(post2.north east)+(3pt,10pt)$) -- ($(post2.south east)+(3pt,-34pt)$);
    
%%     \node[name=r,anchor=south west] at (3,0.2) {{$r$}'s messages};
%%     \node[name=m1,anchor=west] at ($(post2.east)+(5pt,0.5pt)$) {$p_1$: $\left[\mathtt{post}(p_3,h)\right]$};
%%     \node[name=m2,anchor=north west] at ($(m1.south west)+(0pt,-1pt)$) {$p_2$: $\left[\mathtt{post}(p_4,h)\right]$};
%%     \node[name=m3,anchor=north west] at ($(m2.south west)+(0pt,-1pt)$) {$p_3$: $\left[\texttt{x\,=\,1}\right]$};
%%     \node[name=m4,anchor=north west] at ($(m3.south west)+(0pt,-1pt)$) {$p_4$: $\left[\texttt{a\,=\,x}\right]$};
%%   \end{tikzpicture}
%% \end{wrapfigure}
%% To see this, consider the program on the left.
%% % in \cref{fig:postpost} below.
%% Assume that the first explored execution is $s.q.m_1.p_2.p_3.p_4$.
%% Again, there is a race, this time between messages $p_3$ and $p_4$ that write to \texttt{x}.
%% %% This induces $\ppm$ relations between $p_1$ and $p_2$ and between $s$ and $q$.
%% When reversing the race between $p_3$ and $p_4$,
%% \emph{both} the order between $p_1$ and $p_2$ and between $s$ and $q$ must be reversed,
%% resulting in the wakeup sequence $q.p.p_2.p_1.p_4.p_3$,
%% which is the second ---and last--- execution that will be explored.
%% % for this program.

%% \endgroup % NEEDS TO HAVE AN EMPTY LINE BEFORE IT!

%% % Figure environment removed
\hasbeenremoved{
  \paragraph{Non-atomic Messages}
A major complications in extending \OptimalDPOR to the event-driven execution model stems from non-atomic messages.
In general, a message consists of a sequence of statements.
We cannot reverse races between individual statements in two different messages that execute on the same handler in the same way as in standard DPOR,
since we cannot swap just individual statements without swapping the order of two messages entirely.
Such swapping can produce many other changes to the execution, stemming from swapping all events in the two messages.
\EventDPOR preserves \OptimalDPOR's principle to construct wakeup sequences which contain events that do not happen-after any event of the race,
followed by the second event of the race. However, it is often not possible to include all such events in an event-driven execution, in which
case \EventDPOR will include a maximal subsequence.
}
%% The execution of non-conflicting messages should not be ordered by the happens-before relation.
%% The problem is conceptually analogous to regarding messages as mini-threads,
%% which compete for a shared resource (the handler thread).
%% Most existing DPOR algorithms consider accesses to such shared resources
%% (typically protected by a lock) as always conflicting,
%% and will hence explore all serializations of such accesses.
%% The challenge for \EventDPOR is to avoid serialization-by-default,
%% in the case where messages can contain arbitrary (terminating) code.
%% 
%% \paragraph{Reversing Races Between Events on the Same Handler}
We illustrate this mechanism on the program at the bottom left of \cref{fig:example1new}.
%%, which extends the program above it by letting each message contain three accesses, two of which are pairwise conflicting.
Assume that the first explored execution is $E_1$. It contains two races between events in the two messages, one on \texttt{x} and one on \texttt{y}.
According to \OptimalDPOR's principle for race reversal, the race on \texttt{x} should induce an alternative execution composed of 
the sequence of events that do not happen-after the first event (i.e., {\hndlr{h}{p_1}{u\,=\,1}} {\hndlr{h}{p_2}{v\,=\,2}}) and the 
second event {\hndlr{h}{p_2}{a\,=\,x}}
(for brevity, we do not show the two post events).
However, since message execution is serialized, these events cannot form an execution.
Therefore, \EventDPOR forms the alternative execution (shown in blue) by appending the second event {\hndlr{h}{p_2}{a\,=\,x}} to a 
maximal subset of the events of $E_1$ which is closed under $\happbf{\hb}{E_1}$-predecessors
(i.e., if it contains an event $e$ then it also contains all its $\happbf{\hb}{E_1}$-predecessors), and which can form an execution that does not
contain the first event. 
Later, this wakeup sequence is extended to execution $E_2$.
%% In the general case, the alternative execution should be of form $E'.e'$,
%% where $E'$ is a maximal happens-before prefix of $E_1$  for which $E'.e'$ is an execution which does not contain the first event of the race. Here, a \emph{happens-before prefix} of $E_1$ as a subsequence of $E_1$ such that if an event $e$ is in $E_1$, then all $\happbf{\hb}{E_1}$-predecessors of $e$ are in $E_1$.
%% \begin{inparaenum}
%% \item[P1:] includes the event {\hndlr{h}{p_2}{a\,=\,x}}, but not {\hndlr{h}{p_1}{x\,=\,1}} implying that $p_2$ will be executed before $p_1$, and
%% \item[P2:] include the accesses that are independent of the race, i.e., {\hndlr{h}{p_1}{u\,=\,1}} and {\hndlr{h}{p_2}{v\,=\,2}}.
%% \end{inparaenum}
%% However, since message execution is serialized, ${\hndlr{h}{p_1}{u\,=\,1}}$ can be executed only by including all events of $p_2$, which is not possible since
%% then also the event {\hndlr{h}{p_2}{b\,=\,y}}, which is not independent of the race, would be included. We must therefore revise P2 into P2' which requires to include
%% a maximal subset of the independent events which (together with their predecessors) can form an execution when followed by  {\hndlr{h}{p_2}{a\,=\,x}} (the second event of the race).
%% Thus the wakeup sequence is {\hndlr{h}{p_2}{v\,=\,2}} {\hndlr{h}{p_2}{a\,=\,x}}, shown in blue to the right.
Let us then consider the race on \texttt{y}.
The constructed wakeup sequence should append the second event {\hndlr{h}{p_2}{b\,=\,y}} to a maximal subset of events
that do not happen-after the first event {\hndlr{h}{p_1}{y\,=\,1}}.
However, there is no execution that satisfies these constraints, since it would have to include
{\hndlr{h}{p_2}{a\,=\,x}} before its $\happbf{\hb}{E_1}$-predecessor {\hndlr{h}{p_1}{x\,=\,1}}.
The conclusion is that the race on \texttt{y} cannot (and should not) be considered for reversal, whereas that on \texttt{x} should be reversed.
More generally, if two messages executing on the same handler thread are in conflict, then a wakeup sequence is constructed consisting of only the second message up until and including its first conflicting event.
%% One way to understand this apparent asymmetry is that when the race is reversed
%% by posting the messages in the opposite order,
%% the part of $p_2$ before the racing event is not affected,
%% since it does not conflict with $p_1$.
%% On the other hand, $p_2$ can in general contain many statements after the racing event,
%% which may modify the shared variables in ways which can affect all of $p_1$
%% (i.e., the effect of $p_1$ can change completely when it is swapped with $p_2$).

When messages can branch on values read from shared variables,
reversing the order of two messages may change the control flow of each involved message.
Also in this case, \EventDPOR's principles for reversing races work fine.
We illustrate this on the program in \cref{fig:non-atomic-1},
consisting of two threads $s$ and $t$ and a handler thread $h$.
Thread $s$ posts a message $p_1$ to $h$ and thereafter writes to \texttt{x}.
Thread $t$ posts message $p_2$ to $h$
% , \revise{which is non-atomic}: it
that reads from \texttt{x} and \emph{may} then read from~\texttt{y}.
 
% Figure environment removed

Assume that the first execution is $E_1$, where $s$'s access to \texttt{x} goes last.
%% cf.~\cref{fig:non-atomic-1}.
% The corresponding trace graph is shown to the right.
The execution has two races:
one on \texttt{y} between {\evnt{p_1}{y\,=\,2}} and {\evnt{p_2}{b\,=\,y}}, and
one on \texttt{x} between {\evnt{p_2}{a\,=\,x}} and \evnt{s}{x\,=\,1}.
The race on \texttt{x} can be handled in the same way as in \OptimalDPOR:
%% The sequence of events that do not happen-after this race is $s.t.p_1$.
the wakeup sequence is $\evnt{s}{x\,=\,1}$, which branches off after the prefix $s.t.p_1$ (green in~\cref{fig:non-atomic-1}), and 
will subsequently be extended to execution~$E_2$.
% The corresponding trace graph is at the bottom left in \cref{fig:non-atomic-1}.
The race on \texttt{y} is a race between events in two messages on the same handler thread. As in the previous example, the wakeup sequence will
include the second message up until and including the first racing event, which is {\evnt{p_2}{b\,=\,y}}.
Included in the events that do not happen-after the first event is also $\evnt{s}{x\,=\,1}$, which must be placed after its predecessor {\evnt{p_2}{a\,=\,x}}, yielding the wakeup sequence
  {\evnt{p_2}{a\,=\,x}}; \evnt{s}{x\,=\,1}; {\evnt{p_2}{b\,=\,y}}, which
branches off after
  \evnt{s}{post($p_1$,$h$)}, 
  \evnt{t}{post($p_2$,$h$)}.
  This is the blue rightmost branch of the tree in~\cref{fig:non-atomic-1},
  and is later extended into the execution $E_3$.
Execution $E_3$ has a race on \texttt{x}. Its reversal produces the wakeup sequence \evnt{s}{x\,=\,1}, which is a tentative branch next
to  {\evnt{p_2}{a\,=\,x}}. However, this wakeup sequence is not in conflict with the left branch labeled  {\evnt{p_1}{b\,=\,y}}, which means that it will not
be inserted for the reason that it is equivalent to a subsequence of an execution starting with  {\evnt{p_1}{b\,=\,y}}, namely $E_2$.

%% % Figure environment removed
%% To further illustrate how races between events in messages on the same handler are treated as races between a message and an event,
%% consider the same program (in \cref{fig:non-atomic-1})
%% but assume that the two post events occur in the opposite direction
%% and the first explored sequence is $E'_1$ in \cref{fig:non-atomic-2}.
%% %% (The corresponding trace graph is on the right.)
%% Let us consider how to reverse the race on \texttt{y}.
%% This is now a race between $m_2$ and \evnt{m_1}{y\,=\,1}.
%% %
%% Note that in general it is meaningless to try to exploit that the
%% events \evnt{m_2}{a\,=\,x} and \evnt{p}{x\,=\,1},
%% which lie between the conflicting events and the postings of the two messages,
%% are not involved in the race,
%% since after $m_2$ and $m_1$ are swapped,
%% the message $m_1$ might contain some event that changes the value of~\texttt{x},
%% thereby affecting all of $m_2$ in ways that are hard to predict.
%% The sequence not happening-after the race consists of the two \texttt{post} events.
%% The wakeup sequence is constructed by first adding (the event in) $m_1$,
%% thereafter executing the rest of $m_1$ (there are no other events in this case),
%% and finally $m_2$ until the conflict arises; cf. the blue sequence in \cref{fig:non-atomic-2}.
%% Extending this wakeup sequence will lead to execution $E_1$ in \cref{fig:non-atomic-1}.

%% In the last case above, we let the wakeup sequence include events up till the point
%% when a happens-before edge which actually reverses the race appears.
%% This can sometimes be necessary in order to steer
%% the continuation of the wakeup sequence into a non-redundant path.
%% This phenomenon can be illustrated, again on the program in \cref{fig:non-atomic-1}
%% by letting the first explored sequence be
%%     \( q.p[1].m[1].p[2].m[2].n \).
%% The corresponding trace graph is shown in \cref{fig:non-atomic-2}.
%% Let us consider how to reverse the race on \texttt{y}.
%% The sequence not happening-after the race is \( q.p[1] \). The rule says that we should consider
%% all of $m$ as involved in the race, and therefore also $s[2]$. It may be tempting to add a rule
%% to remember that the events $m[1].p[2]$ were not involved in the race, in order to start the
%% subsequent execution of $m$ with these events. Hoever, in the general case, the message
%% $n$ can be very long with many statements after the write to $y$. When swapping the order
%% of $m$ and $n$, all these statements will be executed before $m$ is executed: Thus,
%% the variable $x$ may have been completely updated so that it is not meaningful to have
%% remembered how $m$ was executed in the first execution.
%% On ther other hand, we should probably mandate that
%% after the execution of $n$ we execute $m$ until a conflict with $m$ appears.

\hasbeenremoved{
  % Figure environment removed
%%\bjcom{Do we really need this example?}
Now
%% In order to illustrate further the effect of considering races between events in messages
%% as races between a message and an event,
consider the program in~\cref{fig:example2}.
Assume that the first explored execution is~$E_1$.
%% (Its trace graph is shown at the right.)
There is in fact only one race here: between  {\hndlr{h}{p_1}{a\,=\,y}} and \evnt{t}{y\,=\,1},
shown with a red arc with arrows in the figure.
This race will be reversed in the standard way.
The wakeup sequence, shown in blue, will be inserted after \evnt{s}{post($p_1,h$)} and will be extended to $E_2$.
On the other hand, note that the conflict between  {\hndlr{h}{p_1}{x\,=\,1}} and {\hndlr{h}{p_2}{x\,=\,2}} in $E_1$ is not a race,
since it should be regarded as a conflict between the message $p_1$ and the event  {\hndlr{h}{p_2}{x\,=\,2}}.
Viewed in this way, there is another dependency from $p_1$ to  {\hndlr{h}{p_2}{x\,=\,2}},
namely via the events  {\hndlr{h}{p_1}{a\,=\,y}}, \evnt{t}{y\,=\,1}, and \evnt{t}{post($p_2,h$)}.
Therefore the conflict between the accesses to \texttt{x} is not a race.
%% But the message $p_1$ is also involved in
%% $$\evnt{p_1}{a\,=\,y} \happbf{\fr}{E} \evnt{t}{y\,=\,1} \happbf{\po}{E} \evnt{t}{post($p_2,r$)} \happbf{\pb}{E} \evnt{p_2}{x\,=\,2}$$
%% so this conflict is not a race.
That this conflict is not a race can also be seen by trying to swap the accesses to \texttt{x}
and discover that this will induce a cycle in the happens-before relation.
% in the trace graph.
%% Furthermore, their posts are ordered $s \happbf{\pb}{E} m[1] \happbf{\fr}{E} q[2] \happbf{\po}{E} q[2] \happbf{\pb}{E} n[1]$ which induces the derived relation $m[2] \happbf{\eop}{E} n[1]$, indicating that the conflicts between the messages $m$ and $n$ cannot be reversed.
}

% Figure environment removed
\paragraph{Reordering Messages when Reversing Races}
\EventDPOR's principles for reversing races may necessitate
reordering of messages on handlers that are not involved in the race.
%% In order to illustrate further the effect of considering races between events in messages
%% as races between a message and an event,
Consider the program in~\cref{fig:example3}.
Assume that the first explored execution is~$E_1$, where we have omitted the initial sequence of post events of thread $t$ for  succinctness.
In $E_1$, message $p_1$ is processed before $p_2$, and $q_1$ is processed before $q_2$. There are three races in $E_1$, one on each of the shared variables \texttt{x}, \texttt{y}, \texttt{z}.
Let us consider the race on \texttt{x}, shown by the red arrow. A wakeup sequence which reverses  this race must include all events of $q_2$, since these are the
$\happbf{\hb}{E_1}$-predecessors of \evnt{q_2}{c\,=\,x}. It must also include
the write to \texttt{z} by $p_2$ since it is a $\happbf{\hb}{E_1}$-predecessor of events in $q_2$. On the other hand, it cannot include any part of the message $q_1$, since $q_1$ must now occur after $q_2$, and therefore it also cannot include the read of \texttt{y} by $p_1$ since its predecessor in $q_1$ is missing.
In summary, the wakeup sequence contains two fully processed messages $p_2$ and~$q_2$, the event \hndlr{h}{p_1}{d\,=\,1} of $p_1$, 
but no events from $q_1$. Such a wakeup sequence must branch off after the post events of $t$, i.e., from the root of the tree to the right in~\cref{fig:example3}. Later, this wakeup sequence is extended to a full execution $E_2$.
In total, the program of~\cref{fig:example3} has eight inequivalent executions (the other six are not shown).

\hasbeenremoved{It should be observed that the mechanism of triggering the reordering of messages $p_1$ and $p_2$ due to a race between accesses of $q_1$ and $q_2$ is necessary in this example, otherwise execution $E_2$ can not be reached from $E_1$ by a sequence of race reversals.}

%% An informal description of the exploration proceeds as follows.
%% \begin{enumerate}
%% \item
%%   Explore a maximal execution sequence $E$.
%% \item
%%   For each race of form $e\revrace{E}e'$ or $m\revrace{E}e'$ in $E$,
%% , we construct a sequence where the race is reversed (a so-called
%%   \emph{wakeup sequence}, and thereafter check whether some consistent sequence has previously
%%   been explored. If not, it is inserted into the wakeup tree at the appropriate point, for future exploration.
%%   More precisely, it is done as follows:
%% \begin{enumerate}
%% \item
%%   We first identify the events in $E$ which do not happen-after any of the racing events or messages.
%%   Intuitively,   these events are not affected by the race or its reversal. It clearly includes $\pre{\exseq}{\event}$, resp.\  $\pre{\exseq}{m}$,
%%   but also events that occur after $e$ (resp.\ $m$) but do not happen-after $e$ (resp.\ $m$).
%%   Let $\notsucc{\event}{\exseq}$   denote the subsequence of events that do not happen-after $e$.
%%   Define $\notsucc{m}{\exseq}$ analogously.
%%   Thus, note that for a race of form $m\revrace{E}e'$, the events preceding $e'$ in its message are in
%%   $\notsucc{m}{\exseq}$.
%%   Also note that $\ppm$ edges induced by happens-before edges appearing after $e$ are not included in the considerations. 
%% %% by the race. Intuitively, this part consists of the events that do not happen-after $e$ in  $E$,
%% %%   which we denote by . An unclear case is when $e$ is a post event, which may become $\ppm$-before another post event if a long
%% %%   enough prefix of $E$ is considered. However, such a $\ppm$-edge can occur only because of a message that happens-after $e$, so therefore such potential
%% %%   edges should not be considered when computing $\notsucc{\event}{\exseq}$.
%%  %%   Here, we must be careful since the happens-before relation increases as $E$ is extended. \revise{By consulting the proof of correctness}, we see that the
%%  %%  happens-before relation should not consider induced $\ppm$ relations that occur after $e$. Let us therefore define
%%  %% as  the sub-sequence of $\exseq$ consisting of the events $\event'$ such that
%%  %%  $\event {\happbf{}{\pre{E}{e'}.\thof{e'}}} \event'$ does not hold
%% \item
%%   We next form the new wakeup sequence.
%%   For a race of form $e\revrace{E}e'$, we start by forming $x = \notsucc{\event}{\exseq}.\event'.\event$.
%%   For a race of form $m\revrace{E}e'$, we form $x = \notsucc{m}{\exseq}.\event'.\hat{m'}.\tilde{m}$, where
%%   $\hat{m'}$ executes the remainder of $m'$ (i.e., the events after $e'$) to completion, and
%%   where $\tilde{m}$ executes $m$ until a conflict with $m'$ appears (such a conflict is guaranteed to appear
%%   whenever $m\revrace{E}e'$).
%%   The sequence $x$ may not be an execution sequence, since messages may not be processed in FIFO order, but for now let us ignore this.
%%   Equip $x$ with a happens-before relation $\happbf{}{x}$, following the usual rules. Note that
%%   $\happbf{}{x}$ may include edges that point ``backwards'', but this will be fixed in step (f) below.
%%   %% If $e$ is in a message $m$ posted by $\postev_m$, we also add
%%   %% a $\happbf{}{x}$ edge to $\postev_m$ from each post event that posts a message
%%   %% $m'$ with $m' \happbf{}{x} \event'$, since $e'$ will happen before $e$ in any continuation of $x$.
%% \item
%%   We thereafter check whether $x$ is redundant, i.e., whether some previously explored branch of $\exseqs$ has explored a sequence which is consistent with $x$. This happens if there is a prefix $E$ of $x$ and
%%   a previously explored branch $E.p$ in $\exseqs$ such that, with $x = E.v$ we have
%%   $p \inwfirstseqs{E} v$. In the \OptimalDPOR algorithm, this check is performed by maintaining sleep
%%   sets during the exploration. In our algorithm, we will also use sleep sets, but they need to be modified
%%   since the happens-before relation between post events can vary depending on the subsequent execution.
%%   We elaborate on this at the end of this section.
%% \item
%%   We then find a suitable prefix of $E$ from which $x$ should branch off. This is
%%   the longest prefix $E'$ of $\pre{\exseq}{\event}$ with
%%   $E' \infirstseqs{\emptyseq} x$.
%%   Let $e''$ be the first event in $x$ that such that
%%   $e''' \happbf{}{x} e''$ for some $e'''$ which occurs after $e''$ in $x$.
%%   Such an $e''$ may exist if 
%%   the reversal of $e\revrace{E}e'$ induces reversal of some post events.
%%   If there is no such event $e''$ in $x$, then $E' = \pre{E}{e}$, otherwise
%%   $E' = \pre{E}{e''}$. 
%% \item
%%   Define $v$ by $E'.v = x$.
%% \item
%%   Reschedule $v$ so that it is consistent with  $\happbf{}{x}$ and the ordering between post events in
%%   $E'$, and insert it into the wakeup tree at $E'$.
%% \end{enumerate}
%% \end{enumerate}
%% \begin{enumerate}
%% \item
%%   Modify $E$ by reversing the happens-before edge $e \happbf{}{E} e'$ so that
%%   it points from $e'$ to $e$.
%% \item
%%   Add any $\happbf{ppm}{E}$ edge that is induced by $e' \happbf{}{E} e$.
%% \item
%%   Let $e''$ be the first event in $E$ that is the target of an induced arrow that
%%   points backwards, i.e., of form $e''' \happbf{}{E} e''$, where
%%   $e'''$ occurs after $e''$ in $E$. If the previous steps added no induced
%%   $\ppm$-edges, then $e''$ is $e$.
%% \item
%%   Let $E'$ be the prefix of $E$ that precedes $e''$.
%% \item
%%   Letting $p$ be $\thof{e''}$, create a wakeup sequence
%%   $v$ such that $p \notinwfirstseqs{E'} v$.
%% \item
%%   Check whether the sequence $E'.v$ is redundant with the already explored
%%   execution tree. If not, insert it into the wakeup tree at $E'$.
%% \end{enumerate}

  
%% \subsection{Definition of Concepts}
%% Here is a repository with concepts.
%% Let $E$ be an execution sequence, and let $v$ and
%% $w$ be sequences of threads with $\valid Ev$ and $\valid Ew$.
%% \begin{itemize}
%% \item
%%   $v \infirstseqs{E} w$ denotes that there is a sequence $v'$ such that
%%  $v.v' \equivafter{E} w$.
%%  %%  Two execution sequences $E$ and $E'$ are 
%%  %% there is a sequence $v'$ such
%%  %%  that $E.v.v'$ and $E.w$ are execution sequences with
%% \item
%%   $v \inwfirstseqs{E} w$ denotes that there are sequences $v'$ and
%%   $w'$ such that $v.v' \equivafter{E} w.w'$.
%% \end{itemize}
%%   Intuitively, $v \infirstseqs{E} w$ if, after $E$, the sequence $v$ is a
%%   possible way to start an execution that is equivalent to $w$.
%%   Intuitively, $v  \inwfirstseqs{E} w$ if, after $E$, the sequences $v$ and $w$
%%   are ``consistent'' in the sense that they can be continued to sequences that
%%   are equivalent to each other. We see that if $v$ is a prefix of $w$, then
%%   $v \infirstseqs{E} w$, and that $v \infirstseqs{E} w$ implies $v \inwfirstseqs{E} w$.
%%   Moreoever, the relations enjoy the following properties:
%% \begin{itemize}
%% \item
%%   $v \infirstseqs{E.u} w$ if and only if $u.v \infirstseqs{E} u.w$
%% \item
%%   $v \inwfirstseqs{E.u} w$ if and only if $u.v \inwfirstseqs{E} u.w$
%% \item
%%   if $v \notinwfirstseqs{E} w$ and  $v \infirstseqs{E} v'$ and $w \infirstseqs{E} w'$, then $v' \notinwfirstseqs{E} w'$
%% \end{itemize}
%% Note that there is also a difference from the corresponding definitions in a framework without event handling threads,
%% e.g., that of~\citet{abdulla2014optimal}, in that $\infirstseqs{}$ is no longer transitive. Let us illustrate this by the example seen in Figure~\ref{fig:example1}.

\section{Computation Model}
\label{sec:model}

\hasbeenremoved{In this section, we introduce the class of event-driven programs that we consider.
We also define several semantical notions such as transitions, events, executions, and traces.}

\subsection{Programs}
\label{sec:programs}
We consider programs consisting of a finite set of \emph{threads} that interact via
a finite set of \emph{(shared) variables}.
Each thread is either a \emph{normal thread} or a \emph{handler thread}.
%% ranging over a domain $\valset$ of \emph{values} that includes a special value $\zeroval$.
%
A normal thread has a finite set of local registers and runs a deterministic code, built in a standard way
from expressions and atomic statements, using standard control
flow constructs (sequential composition, selection and bounded iteration).
Atomic statements read or write to shared variables and local registers,
including read-modify-write operations, such as \mbox{compare-and-swap}.
%% To simplify the presentation, we do not consider statements that can block
%% other threads, such as locks.
A handler thread has a \emph{mailbox} to which all threads (also handler threads) can post messages.
A mailbox has unbounded capacity, implying that the posting of a message to a mailbox can never block.
A message consists of a deterministic code, built in the same way as the code of a thread. 
We let $\post(p,h)$ denote the statement which posts the message $p$ into the mailbox of handler thread $h$.
A handler thread repeatedly extracts
a message from its mailbox, executes the code of the message to completion,
then extracts a next message and executes its code, and so on. 
Messages are extracted from the mailbox in arbitrary order.
%% , i.e., the structure of the mailbox is that of an unordered multiset.
The execution of a message is interleaved with the statements of other threads.
%% When its mailbox is empty, the handler thread waits until more messages arrive, or until the
%% program execution terminates.

The local state of a thread is a valuation of its local registers
%% (including those used by the currently executing message, if any)
together with the contents of its mailbox.
A global state of a program consists of a local state of each thread together
with a valuation of the shared variables.
The program has a unique initial state, in which mailboxes are empty.

Recall that we use \emph{message} to denote what is called \emph{event} in \cref{sec:intro}.
%% in order to avoid conflict with our use of the term event as a particular execution of a statement.

\subsection{Events, Executions, Happens-before Ordering, and Equivalence}
\label{sec:events}

We use $s,t, \ldots$ for threads,
$p,q,\ldots$ for messages and non-handler threads,
\texttt{x}, \texttt{y}, \texttt{z} for shared variables,
and \texttt{a}, \texttt{b}, \texttt{c}, \texttt{d} for local registers.
%% For uniformity, we will use the term \emph{message} to refer also to a normal thread,
%% since a normal thread can be represented as a message which is the only one executing on its handler.
%% we let each normal thread be represented as a message, which is executed by a handler thread which executes only
%% this one message, which is present in its mailbox at initialization.
%% We can then say that each statement execution is performed by a message, which executes on
%% some handler.
%% Thus, each event is in some message, which executes on some handler.
%% We then sometimes say that ``the event is executed by the message'', even if this is slightly incorrect.
We assume, %% without loss of generality,
wlog,
that the first event of a message does not access a shared variable, but only performs a local action, e.g., related to initialization
of message execution.
%% This assumption simplifies the presentation of handling the interaction between shared-variable accesses and message scheduling.
In order to simplify the presentation, we henceforth extend the term \emph{message} to refer not only to a message but also to a non-handler thread.

The execution of a program statement is an \emph{event},
which affects the global state of the program. 
An event  is denoted by a pair $\transpair pi$, where $p$ denotes the
message containing the event and $i$ is a positive integer,
denoting that the event results from the $i$-th execution step in message $p$.
\hasbeenremoved{
We call an event \emph{global} if it accesses a shared variable or posts a message.
A message is denoted by the event that posted it. 
For instance, if the second step of
message $p$ posts a message, whose first step is to post another message $q$,
then the event representing the first step of message $q$ is denoted
$\transpair{\msgof{\msgof{p}{2}}{1}}{1}$.
%% \bjcom{The following sentence may not be fully accurate. Also, skip the preceding example?}
As is customary in DPOR algorithms, we can let an event
represent the combined effect of a sequence of statements, if at most one of them affects a shared variable or the
local state of other threads.
This avoids consideration of interleavings of
local statements of different threads in the analysis.
}
An \emph{execution sequence} $\exseq$
is a finite sequence of events, starting from the initial state of the program.
%% following the execution model described in~\cref{sec:programs}.
Since thread and message codes are deterministic, an execution
sequence $\exseq$ can be uniquely characterized by the sequence
of messages (and non-handler threads) that perform execution steps in $\exseq$,
%% and each execution sequence leads to a uniquely defined program state.
%% since the code of each message is deterministic, and their interleaving is given by the execution sequence. 
where we use dot(.) as concatenation operator.
Thus $p.p.q$ denotes the
execution sequence consisting first of two events of $p$, followed by an event of~$q$. 
%% The reason for naming messages after their posting events is to be able to consider execution sequences as equivalent, i.e., having the same sets of events, even when independent messages are handled in different orders. 

We let $\enabled{E}$ denote the set of
messages that can perform a next event in the state to which $E$ leads.
A sequence $\exseq$ is \emph{maximal} if $\enabled{E} = \emptyset$.
%% i.e., no message is enabled after~$\exseq$.
We use $u,v,w, \ldots$ to range over sequences of events. 
We introduce the following notation, where $E$ is an execution sequence and $w$ is a sequence of events.
\begin{itemize}[-]
\item $\emptyseq$  denotes the empty sequence. 
\item $\valid{\exseq}{w}$ denotes that $\exseq.w$ is an execution sequence.
\item $w \remove p$ denotes the sequence
  $w$ with its first occurrence of $p$ (if any) removed.
\item $\dom{\exseq}$ denotes the set of events $\transpair{p}{i}$ in $\exseq$, that is, $\transpair{p}{i} \in \dom{\exseq}$ iff
  $\exseq$ contains at least $i$ events of $p$.
  We also write $e \in E$ to denote $e \in \dom{E}$.
\item $\nextev{\exseq}{p}$ denotes the next event to be performed by the message $p$ after the execution $E$ if $p \in \enabled{E}$,
  otherwise $\nextev{\exseq}{p}$ is undefined.
\item $\procof{\event}$ denotes the
  message that performs $e$, i.e., $e$ is of form
$\event = \transpair{\procof{\event}}{i}$ for some $i$.
%% \item   $\event \totorder{\exseq} \event'$ denotes that $\event$ occurs before $\event'$ in $\exseq$ (i.e., $\totorder{\exseq}$ is the total order on events in $\exseq$).
\item $\exseq' \prefix \exseq$ denotes that  $\exseq'$ is 
  a  (not necessarily strict) prefix of $\exseq$.
%% \item $\pre{\exseq}{\event}$, where $\event \in \exseq$ is the prefix of $\exseq$ up to, but not including, the event $\event$.
%% \item $w\lceil p$ is the prefix of $w$ that precedes $\nextev{\exseq}{p}$ if $\nextev{\exseq}p \in w$, otherwise $w\lceil p$ is $w$.
\end{itemize}
We say that \emph{$p$ starts after $E$} if $p$ has been posted in $E$, but not yet performed any events in $E$.
We say that \emph{$p$ is active after $E$} if $p$ has been posted in $E$, but not finished its execution in $E$.

%% \subsection{Happens-Before Relation and Equivalence}
%% \label{sec:hb}
\hasbeenremoved{The basis for our DPOR algorithm is the definition of a happens-before relation on the events of each execution sequence,
which captures the data and control dependencies that must be respected by any equivalent execution.}

\begin{definition}[Happens-before]
\label{def:hb-def}
Given an execution sequence $E$,
we define the \emph{happens-before relation} on $E$, denoted
$\happbf{\hb}{E}$, as the smallest irreflexive partial order on $\dom{E}$ such that
$e \happbf{\hb}{E} e'$ if $\event$ occurs before $\event'$ in $\exseq$ and either  %% $\totorder{E}$ and either
\begin{itemize}
\item
$e$ and $e'$ are performed by the same message $p$,
\item
$e$ and $e'$ access a common shared variable \texttt{x} and at least one writes to \texttt{x}, or
\item
$\procof{e'}$ is the message that is posted by $e$ and $e'$ is the first event of $\procof{e'}$. %% $\fst{\procof{e'}}$ is $e'$. %% (i.e., $e =\pbof{e'}$).
  \qed
\end{itemize}
\end{definition}
\hasbeenremoved{Intuitively, $\happbf{\po}{E}$ (\emph{program order}) is the total order of events of each message.
Note that $\happbf{\po}{E}$  does not order events of different messages relative to each other.
The relation $\happbf{\cnf}{E}$ (\emph{conflicts with}) captures data flow constraints arising from reads and writes to shared variables.
The relation $\happbf{\pb}{E}$ (\emph{posted by}) captures the causal dependency from message posting to message execution.}
%% Let $\happbf{\hb}{E}$ denote the transitive closure of the union of $\happbf{\po}{E}$, $\happbf{\cnf}{E}$, and $\happbf{\pb}{E}$.
%% \KS{This last sentence was mentioned before.}
%% The $\happbf{\hb}{E}$ relation induces in a natural way an equivalence relation on execution sequences.
The \emph{\hb-trace} (or \emph{trace} for short) of $E$ is the directed graph $(\dom{E},\ \happbf{\hb}{E})$.
%% For a message $p$, we write $p\happbf{\hb}{E} e'$ to denote that $e'$ is not in $p$ and
%% $e\happbf{\hb}{E} e'$ for some event $e$ in $p$, and similarly for $p\happbf{\cnf}{E} e'$, $p\happbf{\weakall}{E} e'$,
%% $e\happbf{\cnf}{E} p'$, etc.
%% \bjcom{Some stuff could be pruned from this paragraph}
\begin{definition}[Equivalence]
\label{def:hb-equiv}
Two execution sequences $E$ and $E'$ are 
\emph{equivalent}, denoted $E \mtequiv E'$, if they have the same trace.
We let \eqclass{E} denote the equivalence class of $E$.
  \qed
\end{definition}
Note that for programs that do not post or process messages,
%% which in our model can be represented as programs with one message per handler,
$\mtequiv$ is the standard Mazurkiewicz trace equivalence for multi-threaded programs \cite{Mazurkiewicz:traces,Godefroid:thesis,FG:dpor,abdulla2014optimal}.
We say that two sequences of events, $w$ and $w'$, with $\valid{E}{w}$ and $\valid{E}{w'}$, are
\emph{equivalent after $E$}, denoted $w \equivafter{E} w'$ if $E.w \mtequiv E.w'$.

\hasbeenremoved{
  In the event-driven execution model, the happens-before relation induces additional ordering constraints, since 
%% induced by the event-driven execution model, in which
%% When constructing executions that respect some particular happens-before relation, one must in fact also respect additional ordering constraints 
%% induced by the event-driven execution model, in which
each handler must execute its messages in some sequential order. The following \emph{saturation operation} adds such additional orderings imposed by
any ordering relation on events.
%%
\begin{definition}[Saturation]
  \label{def:weakall}
Let $E$ be a sequence of events, and $\happbf{}{E}$ be an irreflexive partial order on the events of $E$. 
We define
$\weaksatrel{\happbf{}{E}}$ as the smallest transitive relation $\happbf{\sat}{E}$ on the events of $E$ which includes
$\happbf{}{E}$ and satisfies the constraint that  whenever $e$ and $e'$ are events in different messages on the same handler,
and there is an event $e''$ in the same message as $e$ and an event $e'''$  in the same
  message as $e'$ with $e'' \happbf{\sat}{E} e'''$,
then $e \happbf{\sat}{E} e'$.
%% \begin{enumerate}[(i)]
%% \item \label{rule:saturation-1}
%% \end{enumerate}
%% Let $\edseq{E}{E'}$ denote $\weaksatrel{\hbmseq{E}{E'}}$ ($\weakall$ stands for ``event-driven''). Let $\happbf{\weakall}{E}$ denote $\edseq{E}{\varepsilon}$, where $\varepsilon$ is the empty sequence.
\qed
\end{definition}
In the above definition, note that it is not required that $e$ is distinct from $e''$, nor that $e'$ is distinct from $e'''$.
}


\hasbeenremoved{
\paragraph{\bf Event-driven Consistency.}
The event-driven consistency problem consists in checking whether, for a given  directed graph 
$(S,\happbf{\hb}{S})$ where $S$ is a set of events and $\happbf{\hb}{S}$ is a set of edges,  there is an execution sequence $E$ such that $(S,\happbf{\hb}{S})$ is the  \emph{\hb-trace} of  $E$.

\begin{theorem}
\label{thm-consistency}
The event-driven consistency problem is NP-complete.
\end{theorem}

The proof of the above theorem can be found in \cref{sec:complexity-proof-consistency}.  Given this NP-hardness result,  we define a procedure  to reverse races (\cref{sec:race-reversals-appendix}) that makes use of a saturation procedure to constrain the ordering between messages and therefore reduces the number of cases to  consider. 
}

\section{The \EventDPOR Algorithm}
\label{sec:eventdpor}

In this section, we present \emph{\EventDPOR}, a DPOR algorithm for event-driven programs.
Given a terminating program on given input,
the algorithm explores different maximal executions resulting from different thread interleavings.
\hasbeenremoved{
\EventDPOR is correct, i.e., it explores at least one execution in each equivalence class induced by $\mtequiv$.
%% \bjcom{What about the following?}
For the class of non-branching programs, it is also optimal, in the sense that it explores exactly one execution in each equivalence class.
  We first introduce essential concepts of \EventDPOR (\cref{sec:prels}) and then describe \EventDPOR itself (\cref{sec:algo:access-sets}).
Thereafter, specific parts in \EventDPOR are described:
the reversal of races (\cref{sec:race-reversals}), checking redundancy (\cref{sec:checkwi}),
and wakeup trees (\cref{sec:wakeuptrees}).
}
%% \EventDPOR is inspired by the
%% \OptimalDPOR algorithm of~\citet{optimal-dpor-jacm}, but modified according to the properties of the event-driven execution model.
%% A main difference from \OptimalDPOR is that in \EventDPOR the reversal of races may
%% require the messages on the same handler thread to execute in a different order.
%% \KS{This paragraph has repeated things which have been mentioned
%%   before... It can be trimmed.}

%% \subsection{Basic Algorithm Properties}
%% \label{sec:basic-props}
%% The \EventDPOR algorithm initially explores a maximal execution with abitrary message interleaving.
%% After exploring each maximal execution, the \EventDPOR algorithm analyses its races, which are then reversed to construct new
%% executions, that will later be extended to new maximal executions.
%% \EventDPOR explores one maximal execution sequence from each equivalence classs of \emph{Mazurkiewicz traces}.
%% The explored executions can be thought of as forming
%% a big exploration tree $\exseqs$, in which nodes correspond to explored executions, leaves correspond to maximal such sequences, and there is a total order $\treeorder$ on nodes of $\exseqs$, corresponding to the order in which they are explored.
%% %% where $\exseq \treeorder \exseq'$ if $\explore(\exseq)$ returns before $\explore(\exseq')$ according to \cref{sec:algo:access-sets}.
%% Note that, only the current maximal execution $\exseq$ is stored in its entirety
%% by the algorithm; already explored subtrees are de-allocated when the information they contain is no longer needed.

%% \bjcom{Skip these properties here?}
%% The \EventDPOR algorithm maintains the property that 
%% \begin{itemize}
%%   \item[P1:]
%%   Whenever the exploration of some subtree rooted at $\exseq \in \exseqs$ has finished,
%%   then for each maximal execution of form $\exseq.w$,
%%   the algorithm has explored an execution equivalent to $E.w$.
%% \end{itemize}
%% Property P1 guarantees correctness.
%% For non-branching programs, \EventDPOR also maintains the property that 
%% \begin{itemize}
%%   \item[P2:]
%%   After the exploration of a subtree rooted at $\exseq.p \in \exseqs$ has finished, the algorithm will thereafter not explore a maximal execution
%%   of form $E.w$, which is equivalent to some execution of form $\exseq.p.w'$.
%% \end{itemize}
%% Property P2 enforces optimality.
%% \bjcom{Revise the following}
%% Condition (2) is enforced by a check for redundancy (to be defined in \cref{def:redundancy}), which we can check precisely only for non-branching programs.
%% \KS{What is ``Condition 2''?}

\subsection{Central Concepts in \EventDPOR}
\label{sec:prels}
%% We here introduce central concepts in \EventDPOR. 
\hasbeenremoved{In this section, we define central concepts in \EventDPOR. We first define
the concepts of \emph{happens-before prefix} and \emph{weak initials},
which are used in the check for redundancy of new executions.
Thereafter, we define \emph{races}: these are used to construct new executions from already explored ones.}
%% We begin with concepts derived from the happens-before ordering and equivalence

\begin{definition}[Happens-before Prefix]
\label{def:hb-prefix}
Let $E$ and $E'$ be execution sequences.
We say that $E'$ is a {\em happens-before prefix} of $E$, denoted
$E' \mtprefix E$, if
\begin{inparaenum}[(i)]
\item
  $\dom{E'} \subseteq \dom{E}$,
\item
  $\happbf{\hb}{E'}$ is the restriction of $\happbf{\hb}{E}$ to $E'$, and
\item
  whenever $e \happbf{\hb}{E} e'$ for some $e' \in \dom{E'}$, then $e \in \dom{E'}$.
\end{inparaenum}
We let $w' \mtprefixafter{E} w$ denote that $E.w' \mtprefix E.w$.
  \qed
\end{definition}
Intuitively, $E' \mtprefix E$ denotes that the execution $E'$ is ``contained'' in the execution $E$
in such a way that it is not affected by the events in $E$ that are not in $E'$.
\footnote{The  relation $w' \mtprefixafter{E} w$ is also introduced in \citet{Maiya:tacas16}, as ``$w$ is a dependence-covering sequence of $w'$.''}
%% To illustrate, for  \cref{prog:independent-messages},
To illustrate, for the top left program of \cref{fig:example1new},
the execution $E'$ consisting of \evnt{t}{post($p_2$,$h$)} \hndlr{h}{p_2}{y\,=\,2} is a happens-before prefix of
any maximal execution of the program, since the event of $p_2$ cannot happen-after any other event than
the event that posts $p_2$, which is already in $E'$.

\begin{definition}[Weak Initials]
  \label{def:winits}
  Let $E$ be an execution sequence, and $w$ be a sequence with $\valid Ew$.
  The set $\winits{\exseq}{w}$ of \emph{weak initials of $w$ after $E$} is the set of messages $p$
  %% in $\enabled{E}$
such that $\valid \exseq p.w'$ for some $w'$ with $w \mtprefixafter{E} p.w'$.
\qed
\end{definition}
\begingroup
\setlength{\intextsep}{0em}%
\setlength{\columnsep}{.75em}%
Intuitively, $p$ is in $\winits{\exseq}{w}$ if $p$ can execute the first event in a continuation of $\exseq$ which ``contains'' $w$, in the sense of $\sqsubseteq$.
In \EventDPOR, the concept of weak initials is used to test whether a new sequence is redundant, i.e., is ``contained in'' an execution that have been explored or in
a wakeup sequence that is scheduled for exploration.
%% \cref{def:winits} provides a connection between weak initials and the properties of exploration trees given in \cref{sec:basic-props}.
%% From \cref{def:winits}, it follows that if $p \not \in \winits{\exseq}{w}$, then there can be no sequences $w', w''$ such that
%% $w.w'' \equivafter{E} p.w'$: thus any continuation of $E.w$ is guaranteed to be inequivalent to any continuation of $E.p$.
Note that in \cref{def:winits}, we can generally not choose $w'$ as $w \remove p$.
This happens, e.g., if $p$ does not occur in $w$ but instead $w$ contains another message $p'$ which executes on the same handler as $p$ and
does not conflict with $p$; in this case $w'$ must contain a completed execution of $p$ inserted before $p'$.

\begin{wrapfigure}{r}{0.46\textwidth}
  \footnotesize
  \begin{tikzpicture}[line width=1pt,framed,inner sep=1pt]
    \node[name=p,anchor=south west] at (-0.15,0.25) {{$s$}};
   \node[name=post1] at (0,0) {$\mathtt{post}(p_1,h)$};
    
    \draw[line width=0.5pt] ($(post1.north east)+(1pt,10pt)$)--($(post1.south east)+(1pt,-22pt)$);
    \draw[line width=0.5pt] ($(post1.north east)+(3pt,10pt)$)--($(post1.south east)+(3pt,-22pt)$);
    
    \node[name=q,anchor=south west] at (1.45,0.25) {{$t$}};
    \node[name=post2,anchor=west] at ($(post1.east)+(5pt,0.5pt)$) {$\mathtt{post}(p_2,h)$};
    
    \draw[line width=0.5pt] ($(post2.north east)+(1pt,10pt)$) -- ($(post2.south east)+(1pt,-22pt)$);
    \draw[line width=0.5pt] ($(post2.north east)+(3pt,10pt)$) -- ($(post2.south east)+(3pt,-22pt)$);
    
    \node[name=r,anchor=south west] at (2.7,0.25) {{$h$'s messages}};
    \node[name=m1,anchor=west] at ($(post2.east)+(8pt,0.5pt)$) {$p_1$: $\left[\texttt{x\,=\,1}\right]$};
    \node[name=m2,anchor=north west] at ($(m1.south west)+(0pt,-1pt)$) {$p_2$: $\left[\begin{array}{@{}l@{}}\texttt{y\,=\,2};\\ \texttt{z\,=\,2}\end{array}\right]$};
  \end{tikzpicture}
\vspace{-0.6cm}
  \caption{Illustrating weak initials}
  \label{prog:wi-illustration}
\end{wrapfigure}
We illustrate using the program shown on the right.
If we let $E$ be the execution $s.t$ and $w$ be the sequence $p_1$,
we have $p_2 \in \winits{\exseq}{w}$, since $w \mtprefixafter{E} p_2.p_2.p_1$.
%% If we let $E$ be \evnt{s}{post($p_1$,$h$)} \evnt{t}{post($p_2$,$h$)}, and $w$ be \hndlr{h}{p_1}{x\,=\,1},
%% we have $p_2 \in \winits{\exseq}{w}$, since $w \mtprefixafter{E} p_2.p_2.p_1$.
%% This example shows that to be able to determine whether $p_2 \in \winits{\exseq}{w}$ holds, we must know which accesses to shared variables are performed
%% by a completed execution of $p_2$ in order to determine whether any of them conflicts with the execution of $p_1$.
This illustration shows that in order to determine whether $p \in \winits{\exseq}{w}$ for a message $p$, one must know which shared-variable access will be performed by
$\nextev{E}{p}$, and, in case $p$ starts after $E$ but will execute after some other message on its handler,
also the sequences of shared-variable accesses that $p$ will perform when executing to completion.


The weak initial check problem consists in checking whether $p \in \winits{\exseq}{w}$.

\begin{theorem}
\label{thm-lowerbound}
The weak initial check problem is NP-hard. 
%\revise{This problem is NP-complete for acyclic  programs (i.e., without iterative statements) with a fixed number of thread and message creation. }
\end{theorem}

The proof of the above theorem can be found in \cref{thm-lowerbound-weak}.
In \cref{sec:checkwi}, we propose a sequence of inexpensive rendundancy checks, which have shown to be sufficient for all our benchmarks.

\endgroup % NEEDS TO HAVE AN EMPTY LINE BEFORE IT!

%% \begin{definition}[Doneset]
%%   \label{def:deneset}
%%   A doneset $\done$ is a function defined from a set of execution sequences to a set of event sequences.
%% \end{definition}
%% Let the \EventDPOR algorithm is currently exploring extensions of $E.p$.
%% In this context, $\done(\exseq)$ is the set of already explored event sequences as continuations of $E$
%% \begin{itemize}
%% \item If $p$ starts after $E$, then store the sequence of global events performed by $p$.
%% \item Otherwise store $\nextev{E}{p}$.
%% \end{itemize}

%% We can now define the notion of redundancy.
%% For an execution $\exseq$, a doneset $\done$ defined for the
%% prefixes of $\exseq$, and a message $p$, define $\since{\exseq}{\done}{p}$
%% to be the shortest suffix $w$ of $\exseq$ such that, with $\exseq = \exseq'.w$ we have $\nextev{\exseq'}{p}$ starts a
%% sequence in $\done(\exseq')$.
%% If there is no prefix $\exseq'$ of $\exseq$ where  $\nextev{\exseq'}{p}$  starts a
%% sequence in $done(\exseq')$, then $\since{\exseq}{\done}{p}$ is undefined.
%% If $\since{\exseq}{\done}{p}$ is defined, define $\before{\exseq}{\done}{p}$ by
%% $\exseq = \before{\exseq}{\done}{p}.\since{\exseq}{\done}{p}$.

%% \begin{definition}[Redundancy]
%%   \label{def:redundancy}
%%   Let $\exseq$ be an execution, and $v$ a sequence with $\valid Ev$. Let $\done$ be a mapping from prefixes of $\exseq$ to sets of messages, such that
%%   $\valid{E'}{p}$ whenever $p \in \done(E')$.
%%   We say that $v$ is \emph{redundant after $\tuple{\exseq,\done}$}, denoted
%%     $\redundant{\exseq}{\done}{v}$, if there is a message $p$, and a longest prefix $E_p$ of $E$ with $p \in \done(E_p)$ such that $p \in \winits{E_p}{w.v}$,
%%     where $E_p.w = E$.
%% %%   $E' = \before{\exseq}{\done}{p}$ and $w = \since{\exseq}{\done}{p})$.
%%   \qed
%% \end{definition}
%% %
%% Intuititively, $\redundant{\exseq}{\done}{v}$ holds if there is prefix $\exseq_p$ of $E$ and $p \in \done(\exseq_p)$, such that
%% $\exseq.v$ is a happens-before prefix of an execution which starts with $\exseq_p.p$.


%% \subsubsection*{Races}
%% \label{sec:races}
%% A central mechanism of \EventDPOR algorithm is to detect and reverse races.
%% Intuitively, a race is a conflict between two consecutive accesses to a shared variable, one of which is a write.
\begin{definition}[Races]
\label{def:races}
Let $E$ be a maximal execution sequence.
Two events $e$ and $e'$ in different messages are in a \emph{race}, denoted $e\revrace{E}e'$, if $e \happbf{\hb}{E} e'$ and
\begin{enumerate}[(i)]
\item
$e$ and $e'$ access a common shared variable and at least one is a write, and
\item  there is no event $e''$ with $e \happbf{\hb}{E} e''$ and $e'' \happbf{\hb}{E} e'$.
  \qed
\end{enumerate}
\end{definition}
Intuitively, a race arises between conflicting accesses to a shared variable, by events which are in different messages but adjacent
in the $\happbf{\hb}{E}$ order.
%% An event race $e\revrace{E}e'$ should occur in $E$ in such a way that if $\thof{e}$ is
%% suspended just before $e$, then there is an alternative continuation of the execution
%% in which $e'$ is the first event which conflicts with $e$; this is guaranteed by condition (ii).
%% Analogously,
%% a message race $p\revrace{E}e'$ should occur in $E$ in such a way that if the message $p$ is
%% suspended just before its first event in $E$, then there is an alternative continuation of the execution
%% in which $e'$ is the first event which conflicts with $p$, guaranteed by condition (ii).
%% As a particular case, if $e'$ is in another message $p'$ on the same handler as $p$, then $e'$ must be the first event in $p'$ which conflicts
%% with some event in $p$.

%% In \EventDPOR, new executions are explored by reversing races in already explored executions.
%% \EventDPOR reverses each race of form $e \revrace{E} e'$ by generating
%% an alternative execution of form $\exseq'.v$ in which the event $e'$ is performed, but not $e$.
%% to replace of $p$ after $\pre{\exseq}{\event}$ (or $\pre{\exseq}{p}$).
%% The sequence $v$ performs $e'$, the first event which conflicts with $e$ (or $p$), instead of $e$.
%% Condition (ii) of \cref{def:races} makes sure that it is possible to execute $v$ after $\pre{\exseq}{\event}$ (or $\pre{\exseq}{p}$).
%% In this alternative execution, $E'$ is a prefix of $E$, and
%% $v$, called a \emph{wakeup sequence}, is a suffix which deviates from $\exseq$.
%% A wakeup sequence consists of a \emph{notdep sequence}, denoted by $\notsucc{\event}{\exseq}$, followed by the event $e'$.
%% \begin{definition}
%%     \label{def:notdep}
%%   Let $E$ be a maximal execution sequence.
%%   \begin{itemize}
%%   \item
%%     For an event $e$ in $E$, define $\notsucc{\event}{\exseq}$, as the subsequence of $E$ consisting of the events $e'$ that occur after $\event$, such that
%%     \begin{enumerate}[(i)]
%%     \item it is not the case that $e \edseq{E}{E'} e'$ where $E'$ is $\pre{\exseq}{e}$.
%%     \end{enumerate}
%%   \item
%%     For a message $p$ and event $e$ in $E$, define $\notsucc{p}{\exseq}$
%%     as the subsequence of $E$ consisting of the events $e'$ not in $p$ that occur after the first event of $p$, such that
%%     \begin{enumerate}[(i)]
%%     \item $p \nhappbf{\hb}{E} e'$, and
%%     \item there is no event $e''$ in $E$ with $p \happbf{\hb}{E}$ and $e'' \edseq{E}{E'} e'$ where $E'$ is $\pre{\exseq}{\fst{p}}$.
%%       \qed
%%     \end{enumerate}
%%   \end{itemize}
%% \end{definition}
%% In other words, $\notsucc{\event}{\exseq}$ is formed from the events that do not happen-after $e$
%% (or $p$), but which could precede $e'$ in some execution that
%% is equivalent to $E$.
%% Note that $\notsucc{p}{\exseq}$ may include a message $p'$ that occurs after $p$ on the same handler as $p$ if $p$ and $p'$ are not in conflict; this reflects that $p'$ may occur before $p$ in an equivalent execution. On the other hand $\notsucc{\event}{\exseq}$ cannot include events that occur after $\event$ on the same handler: this reflects that $e$ is in a message that has already started before the occurrence of $e$ and cannot be suspended to schedule another message on the same handler.
%% In both cases, the sequence $v$ is constructed from the sequence $\notsucc{\event}{\exseq}$ (or $\notsucc{p}{\exseq}$)
%% as part of~\cref{alg:wakeuptree} for wakeup tree insertion, to be explained in \cref{sec:wakeup}.


%% Given an execution sequence $E$, a prefix $E'$ of $E$, and two events $e$ $e'$ in $E$, let
%% $e \hbmseq{E}{E'} e'$ denote the smallest transitive relation which includes $\happbf{\hb}{E}$ and in addition has the
%% property that $e \hbmseq{E}{E'} e'$  whenever
%% $e$ is in a message whose first event is in $E'$ and that $e'$  ccurs after $e$ on the same handler as $e$.
%% Intuitively, if $e \hbmseq{E}{E'} e'$ then $e$ must occur before $e'$ in any execution in $\mtclass{E}$ which has
%% $E'$ as a prefix.


\subsection{The \EventDPOR Algorithm}
\label{sec:algo:access-sets}

The \EventDPOR algorithm, shown as pseudocode in \cref{alg:eventdpor-access},
%% It stores generated wakeup sequences in wakeup trees: these are described in \cref{sec:wakeuptrees}.
%% for the case that each execution of a message performs the same sequence of shared-variable accesses.
performs a depth-first exploration of executions using the recursive procedure
$\explore(\exseq)$, where $\exseq$ is the currently explored execution,
which also serves as the stack of the exploration.
In addition the algorithm maintains three mappings from prefixes of $\exseq$, named $\done$, $wut$, and $\pendingwusname$.
For each prefix $\exseq'$ of $\exseq$, 
\begin{itemize}
\item $\done(\exseq')$ is a mapping whose domain is the set of messages $p$ for which the call $\explore(\exseq'.p)$ has returned.
If $p$ does not start after $E'$, then $\done(\exseq')(p)$ is the shared variable-access performed by $\nextev{E'}{p}$.
If $p$ starts after $E'$, then $\done(\exseq')(p)$ is the set of sequences of shared variable-accesses that can be performed in a completed
    execution of $p$ after $E'$.
   The information in $\done(\exseq')(p)$ is collected during the call $\explore(\exseq'.p)$
%% and entered during information collection
  (\crefrange{algacsl:collection-start}{algacsl:donesleeptree-add-ev}).
  \item $\wut{\exseq'}$ is a \emph{wakeup tree}, i.e., an ordered tree $\tuple{B,\prec}$ where
    $B$ is a prefix-closed set of sequences, whose leaves are wakeup sequences.
    For each sequence $u \in B$, the order $\prec$ orders its children (of form  $u.p$)
    by the order in which they were added to $\wut{\exseq'}$. This is also the order in which the sequences of form
    $\exseq'.u.p$ will be visited in the recursive exploration.
\hasbeenremoved{We extend $\prec$ to the post-order relation on $B$ induced by the ordering $\prec$ on children of a node.}
    %% For leaf $w \in B$ the sequence $\exseq'.w$
    %% will be explored during the call $\explore(\exseq')$ in the order given by $\prec$.}
  \item $\pendingwus{\exseq'}$ is a set of wakeup sequences $v$ that were previously being inserted into some wakeup tree $\wut{\exseq''}$, but
    were ``parked'' at the sequence $\exseq'$ because at that time there was not enough information to determine where in $\wut{\exseq''}$ to place $v$.
    Later, when a branch of $\wut{\exseq''}$ has been extended to a maximal execution,
    it should be possible to determine where to insert $v$.
 %% waiting to be inserted in an appropriate wakeup tree.
 %%    The wakeup sequences in $\pendingwus{\exseq'}$ have been previously parked there during ins is a set of wakeup sequences waiting to be inserted in an appropriate wakeup tree.
\end{itemize}
\hasbeenremoved{The already explored executions together with the sequences in the wakeup trees can be thought of as forming an exploration tree $\exseqs$.}

%% in which nodes correspond to explored executions,
%% leaves correspond to maximal such sequences,
%% and there is a total order $\treeorder$ on nodes of $\exseqs$, corresponding to the order in which they are explored.
%% where $\exseq \treeorder \exseq'$ if $\explore(\exseq)$ returns before $\explore(\exseq')$ according to \cref{sec:algo:access-sets}.
%% Note that
%% only the current maximal execution $\exseq$ is stored in its entirety by the algorithm;
%% already explored subtrees are deallocated when the information they contain is no longer needed.


Each call to $\explore(E)$ first initializes $\done(E)$ and $\pendingwus{E}$
($\wut{\exseq}$ was initialized before the call),
%% , into which information about explored subtrees of $E$ will later be inserted.
and thereafter enters one of two phases:
\emph{race detection} (\crefrange{algacsl:event-race-begin}{algacsl:event-race-end}) or
\emph{exploration} (\crefrange{algacsl:exploration-begin}{algacsl:donesleeptree-add-ev}).
%% \emph{execution exploration} (\crefrange{algacsl:exploration-begin}{algacsl:event-call-explore}), and
%% \emph{information collection} (\crefrange{algacsl:initialize-accesses}{algacsl:donesleeptree-add-ev}).
The race detection phase is invoked when $\exseq$ is a maximal execution sequence.
First, for each wakeup sequence $v$ parked at a prefix $E'$ of $E$ it invokes $\insertpendingwu{v}{E'}$ to insert $v$ into the appropriate wakeup tree
(\crefrange{algacsl:insert-parkedwus-begin}{algacsl:insert-pendingwu}).
% , to be described in \cref{sec:wakeuptrees}.
Thereafter, each race (of form $\event \revrace{\exseq} \event'$) in $\exseq$ is analyzed by $\reverserace(\exseq,e,e')$, which returns a set of
executions that reverse the race. Each such execution $E'.v$ is returned as a pair $\tuple{E',v}$, where $v$ is a wakeup sequence that should be
considered for insertion in the wakeup tree at $E'$.
%%   that it performs $\event'$ (together with all events that are needed to enable $\event'$) without performing $\event$, and furthermore is a maximal execution   with this property.
%% The function $\reverserace$ is further elaborated in \cref{sec:race-reversals}.
%%   In the execution $E'.v$, the prefix $E'$ is the maximal prefix of $E$ which does not conflict with $E'.v$ and the suffix $v$ is the wakeup sequence.
%% The subsequence $\notsucc{\event}{\exseq}$ of $\exseq$
%% consisting of the events that do not happen-after $\event$ in $\exseq$ is extracted. From this sequence, maximal executions $\exseq''$ that enable
%% $\event'$ (i.e., contain the $\happbf{\po}{E}$-predecessor of $\event'$) are extracted. Often, all events in $\notsucc{\event}{\exseq}$ can be ordered into
%% such a sequence, in which case only one such maximal $\exseq''$ need be considered.
%% %% (lines~\ref{algacsl:msg-race-assign-x} and~\ref{algacsl:normal-race-assign-x}).
%% \item Each sequence $\exseq''$ formed in the first step is extended with $\procof{\event'}$. The resulting sequence $\exseq.\procof{\event'}$ is then organized as the concatenation
%%   of a maximal consistent prefix $\exseq'$ of $\exseq$, followed by the suffix $u.\procof{\event'}$, so that
%%   $\exseq'.u.\procof{\event'} \mtequiv \exseq''.\procof{\event'}$. The sequence $v$, defined as $u.\procof{\event'}$ is the wakeup sequence.
%% %%   which, if it is not redundant, will be inserted into $\wut{\exseq'}$ .
Each wakeup sequence $v$ is checked for redundancy (\cref{algacsl:event-test}), using
the information in $\done$.
%%   according to \cref{def:redundancy}.
If $v$ is not redundant, it is inserted 
into the wakeup tree at $E'$ for future exploration (\cref{algacsl:event-insert}).
\hasbeenremoved{Wakeup tree insertion is elaborated below, and in~\cref{alg:wakeuptree}.}

\begin{algorithm}[!htp]
\Initial{$\explore(\emptyseq)$~with~$\wut{\emptyseq}=\emptytree$}
\BlankLine
\Fn{$\explore(\exseq)$\tcp*[f]{Returns access sequences of messages}}{
$\done(\exseq) := \emptyset$\;\label{algacsl:doneset-init}
$\pendingwus{\exseq} := \emptyset$\;\label{algacsl:pendingwus-init}
%% $\doneaccesses(\exseq) := \emptyset$\;\label{algacsl:doneaccessesset-init}
  \If(\tcp*[f]{When $E$ is maximal, enter race detection}){$\enabled{\exseq} = \emptyset$}{
    \label{algacsl:event-race-begin}
    \ForEach{$\exseq' \leq \exseq$}{\label{algacsl:insert-parkedwus-begin}
  %%     $\done'(\exseq') := \done(\exseq')$\tcp*{\bjcom{This line to be explained}}
      \ForEach(\tcp*[f]{Parked wakeup sequences}){$v \in \pendingwus{\exseq'}$}{
        $\insertpendingwu{v}{E'}$\tcp*{are inserted at the appropriate place}\label{algacsl:insert-pendingwu}
      }
    }
    \ForEach(\tcp*[f]{For each race in $\exseq$}){$\event,\event'\ \keyword{such that} \ \event \revrace{\exseq} \event'$\label{algacsl:race-loop}} {
%%  \ \ \keyword{and each} \ \ e,  m, m', \event'\ \keyword{such that} \ e\msgrevrace{E}{m}{m'}e'$
%%     \keyword{let} \ \mbox{$\event_{pre}$ be  in }\;
      \ForEach(\tcp*[f]{For each race reversal}){$\tuple{\exseq',v} \in \reverserace(\exseq,e,e')$\label{algacsl:rev-race}}{
%% , built from events in $\notsucc{\event}{E}$, which enables $\event'$}){maximal $\exseq''$ with $\exseq'' \mtprefix \notsucc{\event}{E}$ which contains the $\happbf{\po}{E}$-predecessor of $\event'$}{
%%      \keyword{let} \ \mbox{$\exseq'$ and $u$ be such that $\exseq.u.\procof{\event'} \mtequiv \exseq''.\procof{\event'}$ and $\exseq'$ is a maximal prefix of $E$}\label{algacsl:normal-race-assign-x}\tcp*{Find the prefix $\exseq'$ at which the wakeup sequence $u.\procof{\event'}$ will be inserted}\label{algacsl:def-msg-E}
%%      \keyword{let} \ \mbox{$v = u.\procof{\event'}$}\;
%%        \If(\tcp*[f]{Has no equivalent sequence already been explored?}){$\neg \redundant{\exseq'}{\done}{v}$\label{algacsl:event-test}}{
        \If(\tcp*[f]{If $v$ is not redundant}){$\neg \exists E'',w,p \ \mbox{ s.t. }\ E''.w = E' \land p \in \dom{\done(E'')} \land p \in \winits{E''}{w.v}$\label{algacsl:event-test}}{
          $\insertwus{v}{\exseq'}{\emptyseq}$\tcp*{insert $v$ into the wakeup tree at $E'$}\label{algacsl:event-insert}\label{algacsl:event-race-end}
          %% \keyword{let} $p$ is the next in $\exseq$ after $\exseq'$\tcp*{\bjcom{This and the next three lines to be explained}}
          %% \If{$\nextev{\exseq'}{p} = \fst{p}$} {add the run of $p$ from $\exseq$ in $\done'(\exseq')$}
          %% \lElse{add $\nextev{\exseq'}{p}$ to $\done'(\exseq')$}
        }
      }
    }
  }
\Else(\label{algacsl:event-explore-begin}\tcp*[f]{If not at a maximal execution sequence, explore...}){
  \If{$\wut{\exseq} = \emptytree$ }{\label{algacsl:exploration-begin}
    $\keyword{choose}\; p \in \enabled{\exseq}$\tcp*{... or by selecting an arbitrary $p$...}\label{algacsl:wut-empty-choose}
    $\wut{\exseq} := \tuple{\set{\emptyseq,p},\set{(p,\emptyseq)}}$\tcp*{Adapt wakeup tree accordingly}\label{algacsl:wut-empty-init}
%%     $\sizeofws := |\exseq|$\tcp*{\bjcom{This line is to be explained}}
  }
  \ForEach{message $q$ that is active after $\exseq$}{$\accesses(q) := \emptyset$\tcp*{Initialize the sequences of accesses for messages}\label{algacsl:accesses-initialize}}
  \While(\tcp*[f]{While the wakeup tree is not empty...}){$\exists q \in \wut{\exseq}$\label{algacsl:while-WuT}}{
    $\keyword{let}\; p = \min_{\prec} \{ q \in \wut{\exseq} \}$\label{algacsl:pick-WuT}\tcp*{... pick next branch, ...}
    $\wut{\exseq.p} := \subtreeafter{p}{\wut{\exseq}}$\label{algacsl:def-WuT}\tcp*{extract next wakeup tree)}
    $\keyword{let}\; \tmpaccesses = \explore(\exseq.p)$\label{algacsl:event-call-explore}\tcp*{... and make a recursive call}
    \If{$\nextev{E}{p}$ is the last event of message $p$}{\label{algacsl:collection-start}
      add $p$ to $\dom{\tmpaccesses}$ with $\tmpaccesses(p) = \set{\emptyseq}$ \label{algacsl:initialize-accesses}
    }
    \If{$\nextev{E}{p}$ performs a global access}{
      prepend $\nextev Ep$'s access to each sequence in $\tmpaccesses(p)$\label{algacsl:extend-tmpaccesses}
    }
    \ForEach{message $q$ that is active after $\exseq$}{$\accesses(q) \ \cup\!= \tmpaccesses(q)$}\label{algacsl:accumulate-accesses}
    add $p$ to the domain of $\done(\exseq)$\label{algacsl:doneset-add}\tcp*{Mark $p$ as explored}
    \If(\tcp*[f]{If $p$ starts}){$p$ starts after $E$}{
      $\done(\exseq)(p) := \accesses(p)$\tcp*{... store $p$'s accesses}\label{algacsl:donesleeptree-add}
    }
    \lElse(\tcp*[f]{... store $\nextev{E}{p}$'s access}){$\done(\exseq)(p) := \mbox{$\nextev{E}{p}$}$'s access}\label{algacsl:donesleeptree-add-ev}
    {remove all sequences of form $p.w$ from $\wut{\exseq}$}\tcp*{At end, cleanup}\label{algacsl:wut-deletebranch}
  }
  $\keyword{return}\; \accesses$\label{algacsl:exploration-end}
}
}
\caption{\EventDPOR}
\label{alg:eventdpor-access}
\end{algorithm}


The exploration phase (\crefrange{algacsl:exploration-begin}{algacsl:exploration-end})
is entered if exploration has not reached the end of a maximal execution sequence.
First, if $\wut{\exseq}$ only contains the empty sequence, then 
an arbitrary enabled message is entered into $\wut{\exseq}$ (\cref{algacsl:wut-empty-choose,algacsl:wut-empty-init}).
Thereafter, each sequence in $\wut{\exseq}$ is subject to recursive exploration.
We find the $\prec$-minimal child $p$ of the root of $\wut{\exseq}$ (\cref{algacsl:pick-WuT}),
and make the recursive call $\explore(\exseq.p)$ (\cref{algacsl:event-call-explore}).
Before the call, $\wut{\exseq.p}$ is initialized (\cref{algacsl:def-WuT}).
During the call $\explore(\exseq)$, information is also collected about the sequences of shared-variable accesses that can be performed by each message that is active after $\exseq$,
and subsequently stored in the mapping $\done$.
The information is collected in the variable $\accesses$, which
is initialized at \cref{algacsl:accesses-initialize}.
%% mapping each message that is active after $\exseq$ to the empty set of access sequences.
%% to be returned at \cref{algacsl:exploration-end}, and 
%% where it is used in the check for redundancy (\cref{algacsl:event-test}) and during wakeup tree insertion.
Each recursive call $\explore(\exseq.p)$ returns the sets of access sequences performed by messages that are active after $\exseq.p$ (\cref{algacsl:event-call-explore}).
After prepending the access performed by $\nextev{E}{p}$ to the sets of access sequences performed by $p$ (\cref{algacsl:extend-tmpaccesses}),
the sets  returned by $\explore(\exseq.p)$ are added to the corresponding sets in $\accesses$ (\cref{algacsl:accumulate-accesses}).
Finally, $p$ is added to the domain of $\done(E)$ (\cref{algacsl:doneset-add}).
If $p$ starts a message after $E$, then  $\done(\exseq)(p)$ is assigned the set of access sequences performed by $p$ (\cref{algacsl:donesleeptree-add}), otherwise only the access of
$\nextev{E}{p}$.
Thereafter, the subtree rooted at $p$ is removed from $\wut{\exseq}$ (\cref{algacsl:exploration-end}).
When all recursive calls of form $\explore(\exseq.p)$ have returned, the accumulated sets of access sequences are returned (\cref{algacsl:exploration-end}).

%% \begin{description}
%% \item[Exploration] is pursued if exploration has not reached the end of a maximal execution sequence.
%% It picks the next unexplored leaf of a wakeup tree, and extends it, through a sequence of recursive calls to $\explore()$, to a maximal execution, where the race detection mode is entered. After completing the race detection mode, exploration backtracks and moves to the next
%% %% by returning from the recursive calls to $\explore()$, and makes a new sequence of recursive calls to reach the next
%% unexplored leaf of a wakeup tree. During backtracking, information about explored subtrees and shared variable accesses performed by messages, which is needed for redundancy tests, are collected into the variable $\done$. This information collection is performed at \crefrange{algacsl:collection-start}{algacsl:donesleeptree-add-ev}.
%% \item[Race Detection] mode is entered when $\exseq$ is a maximal execution.
%%     In this mode, the races in $\exseq$ are detected and analyzed, and from each race, of form $\event \revrace{\exseq} \event'$, a set of alternative (non-maximal) executions is generated.
%%     Each alternative execution reverses the race $\event \revrace{\exseq} \event'$ in the sense that it performs $\event'$ without performing $\event$, and is structured as $E'.v$, where
%%     $E'$ is a prefix of the current execution $E$, and $v$ is an alternative continuation, called a \emph{wakeup sequence}.
%%     The sequence $E'.v$ is then subject to a \emph{redundancy test}, which checks whether $E'.v$ is contained in a previously explored execution, using information which is collected
%%     in the variable $\done$. If this test finds $E'.v$ not to be redundant, the procedure $\insertwusname$ is invoked, which, if needed, adds a new leaf to the wakeup tree of $E'$.
%% \end{description}
%% The initial call is $\explore(\emptyseq)$, upon which the wakeup tree $\wut{\emptyseq}$ is initialized with the empty sequence. $\explore$ thus enters exploration mode and
%% extends this empty sequence to a maximal execution with arbitrary message interleaving, whereafter it enters the race detection mode, and so on.

\EventDPOR  calls functions that are briefly described in the following paragraphs. 
More elaborate descriptions (with pseudocode) are in~\cref{sec:functions-appendix}.

\noindent\emph{$\reverserace({E},{\event},{\event'})$} is given a race $\event \revrace{\exseq} \event'$
in the execution $\exseq$ (\cref{algacsl:race-loop}),
and returns a set of executions that reverse the race in the sense that they 
perform the second event $e'$  of the race without performing the first one, and (except for $e'$) only contain events that are not affected by the race.
More precisely, it returns a set of pairs of form $\pair{E'}{u.e'}$, such that
\begin{inparaenum}[(i)]
\item
  %% $E'.u \mtprefix E$ and
$E'.u$ is a maximal happens-before prefix of $E$ such that $E'.u.e'$ is an execution, and
\item $\dom{E'}$ is a maximal subset of $\dom{E'.u}$ such that $E' \leq E$.
\end{inparaenum}
An illustration of the $\reverserace$ function was given for the race on \texttt{x} in the program of \cref{fig:example3}.

  \hasbeenremoved{
    As an illustration, consider the race on \texttt{x} in the program of \cref{fig:example3}.
Here, there is a unique (up to equivalence) maximal execution which reverses the race, which consists of all events that post messages, all events in messages
    $p_2$ and $q_2$, and the assignment to \texttt{d} by $p_1$. The read of \texttt{x} by~$q_2$ should be ordered last, since it corresponds to the racing event $e'$.
    Message $q_1$ is removed by the rule at \cref{algl:revrace-rule-must}, whereby also the second of event of $p_1$ is removed, since it reads from the first event in $q_1$.
  }
  
\noindent\emph{$\insertwus{v}{E'}{\emptyseq}$} inserts the wakeup sequence $v$ into the wakeup tree $\wut{E'}$. If there is already some sequence $u$ in $\wut{E'}$ such that $u \mtprefixafter{E'} v$ or
  $v \mtprefixafter{E'} u$, then the insertion leaves $\wut{E'}$ unaffected. Otherwise
  $\insertwus{v}{E'}{\emptyseq}$ attempts to find the $\prec$-minimal non-leaf sequence $u$ in $\wut{E'}$ with $u \mtprefixafter{E'} v$, and
  insert a new leaf of form $u.v'$ into $\wut{E'}$, such that $v \mtprefixafter{E'} u.v'$, which is ordered after all existing descendants of $u$ in $\wut{E'}$.
  The function finds such a $u$ by descending into $\wut{E'}$ one event at a time; from each node $u'$ it finds a next node $u'.p$ as the $\prec$-minimal child with 
  $u'.p \mtprefixafter{E'} v$. If, during this search, the message $p$ starts after $E'.u'$ it may happen that the wakeup tree does not contain enough subsequent events to determine whether
  $u'.p \mtprefixafter{E'} v$; in this case the sequence $v$ is ``parked'' at the node $u'.p$: the insertion of $v$ will be resumed when $E'.u'.p$ is extended to a maximal execution (at \cref{algacsl:insert-pendingwu} with $E'$ being $E'.u'$).

\noindent\emph{$\insertpendingwu{v}{E'}$} inserts a wakeup sequence $v$, which is parked after a prefix $E'$ of the execution $E$, into an appropriate wakeup tree.
The function first decomposes $E'$ as $E''.p$, and checks whether
$p \in \winits{\exseq''}{v}$, using 
information about the accesses of $p$ that can be found in $E$.
%% , so that the check $p \in \winits{\exseq''}{v}$ can be performed.
%% The check will be exact for non-branching programs, but possibly conservative in general.
If the check succeeds, then insertion proceeds recursively
one step further in the execution $E$, otherwise $v$ conflicts with $p$ and should be inserted into the wakeup tree after $E''$.
%% \Cref{alglpwu:former-leaf} checks whether $E''$ was the leaf that is extended to the currently explored execution.
%% If so, the insertion can return without inserting anything, in analogy
%% with how leaves are handled in wakeup tree insertion
%% (\cref{alg:wut-insert-empty} of \cref{alg:wakeuptree}).

\noindent\emph{Checking for Redundancy}
%% A frequent operation in \EventDPOR is to test whether a sequence $w$ is equivalent to an already explored execution, phrased as a test of form
%% Such a test appears at \cref{algacsl:event-test} of \cref{alg:eventdpor-access}; it is defined in \cref{def:redundancy} using the concept of weak initials (\cref{def:winits}).
Tests of form $p \in \winits{\exseq}{w}$ for a message $p$ and an execution $\exseq.w$
appear at
\cref{algacsl:event-test} and in the functions $\insertwusname$ and $\insertpendingwuname$.
If $p$ does not start after $E$, then the check can be straightforwardly performed using
sleep sets~\cite{Godefroid:thesis}.
If $p$ starts after $E$, then checking whether $p \in \winits{\exseq}{w}$ is NP-hard in the general case (see~\cref{thm-lowerbound}).
To avoid expensive calls to a decision procedure,
\EventDPOR employs a sequence of incomplete checks, starting with simple ones, and proceeding with a next test only if the preceding was not conclusive.
%% We assume that $p$ has been posted in $E$, otherwise $p \in \winits{\exseq}{w}$ is trivially false.
These tests are in order:
\begin{inparaenum}[1)]
  \item
  If $p$ is the first message (if any) on its handler in $w$, then $p \in \winits{\exseq}{w}$ is trivially true.
\item
  If the happens-before relation precludes $p$ from executing first on its handler, then $p \in \winits{\exseq}{w}$ is false; checking this may require
  $w$ to be extended so that $p$ (and possibly other messages) are executed to completion.
\item An attempt is made to construct an actual execution in which $p$ is the first message on its handler, which respects the happens-before ordering.
\item If all previous tests were inconclusive, a decision procedure is invoked as a final step.
\end{inparaenum}
  

  
\hasbeenremoved{\subsection{Reversing Races}
\label{sec:race-reversals}
\begin{algorithm}[t]
\SetAlFnt{\small\sf}
\BlankLine
\Fn{$\reverserace({E},{\event},{\event'})$}{
  \keyword{let} $E''$ be the subsequence of $E$ consisting of the events $e'''$ with $\event \nhappbf{\hb}{E} \event'''$\;\label{algl:revrace-short-init-notdep}
  \keyword{let} $S$ be a set of pairs of form $\pair{E'}{u.e'}$ which \\
  \quad for each maximal subset of $E''\cup\set{e'}$ that can be linearized \\
  \quad \quad \qquad \qquad to an execution of form $E'.u.e'$ while respecting $\happbf{hb}{E''}$ \\
  \quad contains one such execution $E'.u.e'$ where $E'$ is a maximal prefix of $E$.
  %% \keyword{let} $\event''$ be the $\happbf{\po}{E}$-predecessor of $\event'$\;

  %% $\happbf{sc}{E''} := \weaksatrel{\happbf{hb}{E''}}$\;\label{algl:revrace-short-init-ordering}
  %% \Repeat{convergence}{
  %%   \label{algl:converge-begin}
  %%   \If{a partially executed message includes an event $e'''$ with $e''' \happbf{\hb}{E''} e''$}{\label{algl:revrace-short-rule-must}
  %%     remove all other incomplete messages on same handler from $E''$\;\sdcmt{need to revisit}
  %%   }
  %%   \If{several incomplete messages execute on one handler}{\label{algl:revrace-short-rule-choose}
  %%     remove all but one of them from $E''$ (nondeterministically)\;
  %%   }
  %%   \ForEach{incomplete message $p$}{\label{algl:revrace-short-rule-last}
  %%     add relation $\happbf{sc}{E''}$ from all other messages on same handler to $p$\;
  %%     saturate\;
  %%   }
  %%   \ForEach{cycle in $\happbf{sc}{E''}$}{\label{algl:revrace-short-rule-cycle}
  %%     remove a message in the cycle\;
  %%   }
  %%   remove events that happen-after already removed events\;
  %% }\label{algl:converge-end}
  %% linearize $E''$ to an execution, respecting $\happbf{sc}{E''}$, with maximal prefix $E'$ s.t.\ $E' \leq E''$ and $E' \leq E$\; \label{algl:revrace-short-determine-prefix}
  %% define $u$ by $E'.u = E''$\;
  $\Return(S)$\;
}
\caption{Reversal of a Race.}
\label{alg:reverserace-short}
\end{algorithm}

A key procedure of \EventDPOR is $\reverserace$ which constructs new executions by analyzing and reversing a race in an explored execution.
This procedure is given a race $\event \revrace{\exseq} \event'$ in the currently explored execution $\exseq$ (at \cref{algacsl:race-loop} of \cref{alg:eventdpor-access}),
and returns a set of maximal executions that reverse the race in the sense that they 
perform the second event $e'$  of the race without performing the first one, and (except for $e'$) only contain events that are not affected by the race.
The procedure $\reverserace({E},{\event},{\event'})$, shown in \cref{alg:reverserace-short}, returns, 
for each maximal subset of $E''\cup\set{e'}$ that can be linearized to an execution that ends in $e'$, one such execution $E'.u.e'$ in which
$E'$ is a maximally long prefix of $E$; each execution $E'.u.e'$ is returned as a pair of form $\pair{E'}{u.e'}$.
}



\hasbeenremoved{
The \EventDPOR algorithm maintains the following properties
\begin{itemize}
   \item[P1:]
  whenever the exploration of some subtree rooted at some execution $\exseq.p \in \exseqs$ has completed,
  then for each maximal execution of form $\exseq.p.w$, the algorithm has explored an execution equivalent to $E.p.w$.
%% Namely, if there is no
%% prefix $E''$ of $E'$ (with $E' = E''.w$) and message $p$ with $p \in \winits{E''}{w.v}$
%% for which the subtree rooted at $E''.p$ has been explored,
%% then $E''.w.v$ is not contained in any extension of $E''.p$, implying that no maximal
%% extension of $E''.w.v$ can be equivalent to a maximal extension of $E''.p$.
%% Property P1 is the basic property which guarantees correctness. It is also used to avoid redundant exploration by enforcing that 
   \item[P2:]
whenever the exploration tree $\exseqs$ contains a node of form $\exseq.p$, then the algorithm will not add an execution of form $\exseq.w$ which is contained in
some execution of form $\exseq.p.w'$ for some $w'$, i.e., for which $p \in \winits{\exseq}{w}$.
\end{itemize}
Property P1 is the basic property which guarantees correctness. It is also used to avoid redundant exploration by enforcing P2. Such a check for redundancy is performed
before inserting a new wakeup sequence (\cref{algacsl:event-test}),
and also inside the procedure for wakeup tree insertion (\cref{alg:wakeuptree}).
}

%% The variable $\done$ maintains the following information which is needed to perform the redundancy test at \cref{algacsl:event-test}, as explained after \cref{def:winits}.
%% \begin{itemize}
%% \item
%%   For each prefix $E'$ of $E$, the domain of $\done(E')$ is
%%   set of messages $p$ for which the exploration of the subtree rooted at $E'.p$ has been completed.
%%   \item $\done(E')$ maps each message $p$ in its domain to
%% \begin{inparaenum}[(i)]
%%   \item the shared variable-access performed by $\nextev{E'}{p}$ if
%% $p$ does not start after $E'$, and
%%   \item the sequences of shared variable-accesses that can be performed in a completed
%%     execution of $p$ after $E'$, if $p$  starts after $E'$.
%% \end{inparaenum}
%% \end{itemize}
%% This information is collected and entered during information collection (\crefrange{algacsl:collection-start}{algacsl:donesleeptree-add-ev}).

%% the set of messages $p$ for which the exploration of the subtree
%% rooted at $E'.p$ has been completed, and whose range is
%%  know for which messages $p$  the exploration of the subtree
%% rooted at $E'.p$ has been completed. In addition, the redundancy test also needs to know
%% which shared variable-access (if any) it performs, i.e., whether it reads and/or writes and to which variable. In case $p$ starts after $E'$, then the redundancy tests also must know which are the possible sequences of accesses that can be performed by a completed execution of message $p$. The variable $\done$ therefore maps each prefix $E'$ of
%% the current execution to a mapping $\done(E')$ whose domain is
%% the set of messages $p$ for which the exploration of the subtree
%% rooted at $E'.p$ has been completed, and whose range is


%% Let us now go through the pseudocode of \cref{alg:eventdpor-access}.
%% The algorithm maintains the global variables $\wutname$ and $\done$ which have been described above, and $\pendingwusname$, which will be described
%% together with wakeup tree insertion in \cref{sec:wakeuptrees}. 
%% \begin{itemize}
%% \item $\wutname$, a mapping from prefixes of $\exseq$ to wakeup trees. For a prefix $E'$ of $E$, the wakeup
%%   tree $\wut{E'}$ contains sequences that continue $E'$ in a different way than in $E$, and will later be explored by the algorithm.
%%   When $\explore(\exseq)$ is first called, the wakeup tree $\wut{\exseq}$ is initialized to the empty tree. It is
%%   later extended as a result of analyzing executions that are explored during the call $\explore(E)$.
%% \item $\done$, a mapping which maps each prefix of $\exseq$ to a mapping from messages to  sequences of shared-variable accesses. For a prefix $E'$ of $E$, the set
%%   $\done(E')$ is a mapping, whose domain is the set of messages $p$ for which the subtree rooted at $E'.p$ has already been explored.
%%   If $p$ does not start after $E'$, then $\done(E')(p)$ is the shared-variable access performed by $\nextev{E'}{p}$.
%%   If $p$ starts after $E'$, then $\done(E')(p)$ is a sequence of shared-variable accesses that will be performed by a complete execution of message $p$.
%% %% \item $\doneaccesses$, a mapping from prefixes $E'$ of $\exseq$ and messages $p \in \done(E')$ that start after $\exseq'$
%% %%   (i.e., for which $\nextev{E'}{p} = \fst{p}$) to sequences of shared-variable accesses, such that $\doneaccesses(E')(p)$ is
%% \item $\pendingwusname$, a mapping from prefixes of $\exseq$ of form $\exseq'.p$, such that $p$ starts after $\exseq'$ to sets (actually queues)
%% of wakeup sequences, which are to be inserted into an appropriate wakeup tree.
%% \end{itemize}
%%
%% When $\explore(\exseq)$ is first called, the wakeup tree $\wut{\exseq}$ is initialized to the empty tree; it can
%% thereafter be extended by adding wakeup sequences that are constructed as the result of analyzing races in explored executions.
%%
%% Each call to $\explore(E)$ starts by initializing the sets $\done(E)$ and $\pendingwus{E}$.
%% %% , into which information about explored subtrees of $E$ will later be inserted.
%% Thereafter, a call to $\explore(E)$ consists of two phases:
%% race detection (\crefrange{algacsl:event-race-begin}{algacsl:event-race-end}) and
%% execution exploration (\crefrange{algacsl:exploration-begin}{algacsl:exploration-end}).
%% The race detection phase is invoked when $\exseq$ is a maximal execution sequence.
%% In this phase, first parked wakeup sequences are considered
%% (\crefrange{algacsl:insert-parkedwus-begin}{algacsl:insert-pendingwu}).
%% % , to be described in \cref{sec:wakeuptrees}.
%% Thereafter, each race in the explored execution $\exseq$ is analyzed by a call $\reverserace(\exseq,e,e')$, which returns a set of pairs $\tuple{E',v}$, each of which can form
%%  an execution $E'.v$, which reverses the race in the sense
%%   that it performs $\event'$ (together with all events that are needed to enable $\event'$) without performing $\event$, and furthermore is a maximal execution
%%   with this property. The function $\reverserace$ is further elaborated in \cref{sec:race-reversals}.
%% %%   In the execution $E'.v$, the prefix $E'$ is the maximal prefix of $E$ which does not conflict with $E'.v$ and the suffix $v$ is the wakeup sequence.
%% %% The subsequence $\notsucc{\event}{\exseq}$ of $\exseq$
%% %% consisting of the events that do not happen-after $\event$ in $\exseq$ is extracted. From this sequence, maximal executions $\exseq''$ that enable
%% %% $\event'$ (i.e., contain the $\happbf{\po}{E}$-predecessor of $\event'$) are extracted. Often, all events in $\notsucc{\event}{\exseq}$ can be ordered into
%% %% such a sequence, in which case only one such maximal $\exseq''$ need be considered.
%% %% %% (lines~\ref{algacsl:msg-race-assign-x} and~\ref{algacsl:normal-race-assign-x}).
%% %% \item Each sequence $\exseq''$ formed in the first step is extended with $\procof{\event'}$. The resulting sequence $\exseq.\procof{\event'}$ is then organized as the concatenation
%% %%   of a maximal consistent prefix $\exseq'$ of $\exseq$, followed by the suffix $u.\procof{\event'}$, so that
%% %%   $\exseq'.u.\procof{\event'} \mtequiv \exseq''.\procof{\event'}$. The sequence $v$, defined as $u.\procof{\event'}$ is the wakeup sequence.
%% %% %%   which, if it is not redundant, will be inserted into $\wut{\exseq'}$ .
%%   Each wakeup sequence $v$ is checked for redundancy (\cref{algacsl:event-test}), using
%%  the information in $\done$ as explained above.
%% %%   according to \cref{def:redundancy}.
%%   If $v$ is not redundant, it is inserted 
%%   into the wakeup tree at $E'$ for future exploration (\cref{algacsl:event-insert}).
%%   Wakeup tree insertion is elaborated in~\cref{alg:wakeuptree}.


%% The exploration phase (\crefrange{algacsl:exploration-begin}{algacsl:exploration-end})
%% is entered if exploration has not reached the end of a maximal execution sequence.
%% First, if $\wut{\exseq}$ only contains the empty sequence, then 
%% an arbitrary enabled message is entered into $\wut{\exseq}$ (\cref{algacsl:wut-empty-choose,algacsl:wut-empty-init}).
%% Thereafter, each sequence in $\wut{\exseq}$ is subject to exploration.
%% We find the left-most (i.e., minimal) single-message child $p$ of the root of $\wut{\exseq}$ (\cref{algacsl:pick-WuT}),
%% and call $\explore$ recursively for the
%% sequence $\exseq.p$ (\cref{algacsl:event-call-explore}).
%% Before the call, $\wut{\exseq.p}$ is initialized (\cref{algacsl:def-WuT}).
%% After the call, $p$ is added to $\done(E)$ (\cref{algacsl:doneset-add}),
%% and all sequences beginning with $p$ are removed from $\wut{\exseq}$ (\cref{algacsl:exploration-end}).

%% During the exploration phase, the algorithm also performs \textbf{information collection} about the sequences of shared-variable accesses that can be performed by each message; this information is stored in the mapping $\done$ and is used in the check for redundancy (\cref{algacsl:event-test}) and during wakeup tree insertion.
%% The information is collected in the variable $\accesses$ to be returned at \cref{algacsl:exploration-end}.
%% The variable $\accesses$ is initialized at \cref{algacsl:accesses-initialize}, mapping each message that has yet to perform some event(s) after $\exseq$ to the empty set of access sequences.
%% Each recursive call $\explore(\exseq.p)$ returns the set of access sequences performed by messages that are active after $\exseq.p$ (\cref{algacsl:event-call-explore}).
%% From this, the access performed by $\nextev{E}{p}$ is prepended to the concerned sequences (\cref{algacsl:extend-tmpaccesses}). If $\nextev{E}{p}$ is the first event of message $p$, then the information about the access sequences performed by $p$ is entered into $\done(\exseq)(p)$ (\cref{algacsl:donesleeptree-add}).
%% Thereafter, the access sequences in $\tmpaccesses$ are added to $\accesses$ (\cref{algacsl:accumulate-accesses}).
%% Finally, the message $p$ is added to $\done(\exseq)$ (\cref{algacsl:doneset-add}) and its subtree is reclaimed from the exploration tree (\cref{algacsl:wut-deletebranch}).
%% When all recursive calls of form $\explore(\exseq.p)$ have returned, the accumulated sets of access sequences are returned (\cref{algacsl:exploration-end}).

%% After $\explore(\exseq.p)$ has returned, its return value is stored in the mapping $\accesses$ from messages to sequences
%% of accesses. If the current event $\nextev Ep$ performs a global access, it is prepended to $\accesses(p)$ (\cref{algacsl:extend-tmpaccesses}).
%% Thereafter, if $p$ is the first event of a new message $m$, the sequences of its accesses is entered into the global variable as $\doneaccesses(\exseq)(p)$ (\cref{algacsl:donesleeptree-add}), otherwise the current set of accesses is extended \bjcom{I did not understand}.
%% %% (\cref{algacsl:doneet-addincsleeptree}).
%% After performing the recursive exploration calls to children of form $E.p$, the call $\explore(\exseq)$ returns the mapping
%% $\accesses$ resulting from the first recursive call. From the assumption that a message performs the same sequence of accesses in each
%% call, it does not matter which of the recursive calls is used for returning $\accesses$ from $\explore(\exseq)$.


%% be the set of events of $E$ that are not affected by the race (\cref{algl:revrace-init-notdep}):
%% this is the set of events $e'''$ with $\event \nhappbf{\hb}{E} \event'''$. If $E''$ can be reordered to form an execution, $\reverserace$ terminates and returns it.
%% However, there are situations in which $E''$ cannot be reordered into an execution.
%% For instance, $E''$ may contain two incompletely executed messages on the same handler because the remaining parts of these messages happen-after $e$ in $E$.
%% Since a wakeup sequence may contain at most one incompletely executed message per handler,
%% $\reverserace$ then performs a sequence of message removals and reorderings to produce a set of maximal wakeup sequences.
%%   The procedure employs the saturation operation of \cref{def:weakall} to constrain the ordering between messages on the same handler. 
%% % A simple example includes message-message races, illustrated by \cref{fig:example1new}, where all of message $p_1$ must be removed from $\exseqmay$ in order to form an execution.
%%   The procedure maintains
%% %%  a sequence $E''$, initialized to the events that could possible be in $E'.u$
%%   an ordering relation $\happbf{sc}{E''}$ on $E''$, initialized to $\weaksatrel{\happbf{hb}{E''}}$ (\cref{algl:revrace-init-ordering}).
%% It thereafter performs a sequence of steps in which messages are removed from $E''$ and/or the ordering relation $\happbf{sc}{E''}$ is extended. Some steps may be resolved nondeterministically: in such cases the procedure pursues all possible alternatives, potentially resulting in several returned sequences. The steps of \cref{alg:reverserace} are the following.
%%   \begin{description}
%%     \item[\cref{algl:revrace-rule-must}]
%%       If a partially executed message includes an event $e'''$ with $e''' \happbf{hb}{E''} e''$, then
%%       any other message on the same handler which is not completely executed in $E''$ must be removed.
%%     %% \item A message containing events that are ordered (by $\happbf{sc}{E''}$) after events in an incompletely executed message
%%     %%   on the same handler must be discarded.
%%     \item[\cref{algl:revrace-rule-choose}]
%%       If several incompletely executed messages execute on the same handler, then all except one of them must be removed. This is done nondeterministically, potentially leading to several returned wakeup sequences.
%%     \item[\cref{algl:revrace-rule-last}]
%%       Whenever a handler has an incompletely executed message $p$, any other message $p'$
%%     on that handler must be executed before $p$, represented by extending $\happbf{sc}{E''}$ from the last
%%     event of $p'$ to the first event of $p$ and then saturating.
%%   \item[\cref{algl:revrace-rule-cycle}]
%%     If $\happbf{sc}{E''}$ becomes cyclic during the filtering and ordering procedure, then each cycle should be broken by
%%       removing the events in a suitable message.
%%   \end{description}
%%     After the sequence, the resulting sequence $E''$ is linearized while respecting $\happbf{sc}{E''}$. Then the decomposition of $E''$ into $E'.u$ is determined (\cref{algl:revrace-determine-prefix}), and the event $e'$ is added at the end.



%%     \begin{itemize}
%%   \item the earliest message which is discarded, or preceded by another one on its handler,
%%   \item otherwise the entire computation preceding $\event$.
%%   \end{itemize}
%% \bjcom{We should describe how this procedure is guided by sleep sets}



\section{Correctness and Optimality}
\label{sec:correctness}
A program is defined to be \emph{non-branching} if each
message, which executes on the same handler as  some other message, performs the same sequence of accesses (reads or writes) to shared
variables during its execution, regardless of how its execution is interleaved with other threads and messages.
%% Intuitively, messages in a
%% non-branching program must not use the values it reads from shared variables to determine which shared variables to access thereafter.
%% On the other hand, a non-branching program can use values of read variables to choose between different messages that it posts.
Note that the ``non-branching'' restriction does not apply to non-handler threads nor to messages that are the only ones executing on their handler.

The following theorems state that \EventDPOR is 
\emph{correct} (explores at least one execution in each equivalence class)
for \emph{all} event-driven programs
and \emph{optimal} (explores exactly one execution in each equivalence class)
for non-branching programs.
%% In this section, we state correctness and optimality of \EventDPOR.
Proofs can be found in \cref{sec:correctness-proof}.
%% of the supplementary material.



\hasbeenremoved{
Let us now
% present the main properties of the \EventDPOR algorithm. Throughout, we
assume a particular completed execution of \EventDPOR. This execution 
consists of a number of terminated calls to $\explore(E)$ for some values 
of the parameters $E$ and $\WuT$. Let $\exseqs$ denote the set of execution 
sequences $E$ that have been explored in some call $\explore(E)$. Define 
the ordering $\treeorder$ on $\exseqs$ by letting $E \treeorder E'$ if 
$\explore(E)$ returned before $\explore(E')$. Intuitively, if one 
were to draw an ordered tree that shows how the exploration has proceeded, then 
$\exseqs$ would be the set of nodes in the tree, and $\treeorder$ would be the 
post-order between nodes in that tree. The correctness and optimality of 
\cref{alg:eventdpor-access} are stated in the following theorems.
}

\begin{theorem}[Correctness]
\label{thm:correctness}
Whenever the call to $\explore(\emptyseq)$ returns during \cref{alg:eventdpor-access},
then for all maximal execution sequences $E$, the algorithm has explored
some execution sequence in~$\eqclass{E}$.
\end{theorem}

\hasbeenremoved{
Since the initial call to the algorithm, $\explore(\emptyseq)$, starts with
the empty sequence as argument, \cref{thm:correctness}
implies that for all maximal execution sequences $E$ the algorithm
explores some execution sequence $E'$ which is in $\eqclass{E}$.
  Note also that a sequence of form $E.w$ need not have been explored inside
the call $\explore(E)$, but can have been explored in some earlier call,
of form $\explore(E'.p)$ for some prefix $E'$ of $E$.
}

\begin{theorem}[Optimality]
\label{thm:optimality}
When applied to a non-branching program,
\cref{alg:eventdpor-access} never explores two maximal execution
sequences which are equivalent.
\end{theorem}

\hasbeenremoved{
  \Cref{thm:optimality} ensures
that all maximal execution sequences reached are non-redundant.
}

%\section{Complexity Results}
\label{sec:np-complete}

In this section, we consider the following two problems.

\begin{description}
\item[Event-driven Consistency.]
The event-driven consistency problem consists in checking whether, for a given  directed graph 
$(S,\happbf{\hb}{S})$ where $S$ is a set of events and $\happbf{\hb}{S}$ is a set of edges,  there is an execution sequence $E$ such that $(S,\happbf{\hb}{S})$ is the  \emph{\hb-trace} of  $E$.
%% The event-driven consistency problem consists in checking whether, for a given  directed graph 
%% $(S,\happbf{\hb}{S})$ where $S$ is a set of events and $\happbf{\hb}{S}=\happbf{\po}{S} \cup \happbf{\cnf}{S} \cup \happbf{\pb}{S}$ is a set of labeled edges,   there is an execution sequence $E$ such that $(S,\happbf{\hb}{S})$ is the  \emph{\hb-trace} of  $E$ (i.e., $S=\dom{E}$, $ \happbf{\po}{S} =\happbf{\po}{E} $,  $\happbf{\cnf}{S} = \happbf{\cnf}{E}$  and $\happbf{\pb}{S} = \happbf{\pb}{E}$).

\item[Weak Initial Check.]
The weak initial check problem consists in checking whether  $p \in \winits{\exseq}{w}$ for a message $p$.

%\item {\bf Reversing the order of the execution of two messages.} The order reversing  problem consists in checking whether, for a given   execution sequence $E=w_1 . e_1 . w_2 . e_2 . w_3$ such that $e_1$ and $e_2$ are the two first events executed by two messages on the same thread handler, there is an  execution sequence $E'=w'_1 . e_1 . w'_2 . e_2 . w'_3$ such that $E \mtequiv E'$.
\end{description}

The following theorems summarise our results concerning these problems.
Their proofs can be found in \cref{sec:complexity-proof}.

\begin{theorem}
\label{thm-consistency}
The event-driven consistency problem is NP-complete.
\end{theorem}


\begin{theorem}
\label{thm-lowerbound}
The weak initial check problem is NP-hard.
\end{theorem}

% Their proofs can be found in \cref{sec:complexity-proof} of the supplementary material.



%  \subsection{NP-completeness of the event-driven consistency problem}
%
%\paragraph{Upper-bound} Let  $(S,\happbf{\hb}{S})$ be a directed graph (i.e., \hb-trace) 
% where $S$ is a set of events and $\happbf{\hb}{S}=\happbf{\po}{S} \cup \happbf{\cnf}{S} \cup \happbf{\pb}{S}$ is a set of labeled edges.
%To show that the event-driven consistency problem is NP, it suffices to first guess a total ordering $<_S$ between the messages on the same thread handler. Observe that we can have at most one incomplete message per handler which should be scheduled last with respect to $<_S$. We then use  the  total order relation $<_S$ to extend  the {\em program order} relation  $\happbf{\po}{S}$   into a total order relation $\happbf{\po}{}$ on the set of events executed by the same handler such that: $(1)$ $e \happbf{\po}{} e'$ if $e \happbf{\po}{S} e'$, and $(2)$
%$e \happbf{\po}{} e'$ whenever $e$ and $e'$ are events in two different messages $p$ and $p'$ on the same handler and $p <_S p'$. Finally,  the extended  {happens-before relation} $\happbf{\hb}{}=\happbf{\po}{} \cup \happbf{\cnf}{S} \cup \happbf{\pb}{S}$  is acyclic ({which  is equivalent  to checking sequential consistency} of the extended graph $(S,\happbf{\hb}{})$) if and only if  there is an execution sequence $E$ such that $(S,\happbf{\hb}{})$ is the  \emph{\hb-trace} of  $E$ (i.e., $S=\dom{E}$, $ \happbf{\po}{} =\happbf{\po}{E} $,  $\happbf{\cnf}{S} = \happbf{\cnf}{E}$  and $\happbf{\pb}{S} = \happbf{\pb}{E}$). Observe that checking the acyclicity  of the extended {happens-before relation} $\happbf{\hb}{S}$ can be done in polynomial time. Furthermore, the execution sequence $E$ can be obtained   via the linearlization of the extended {happens-before relation} $\happbf{\hb}{}$  (since the extend  the {\em program order} relation $\happbf{\po}{}$  forces the messages on the same handler to be executed one after the other).
%
%\paragraph{Lower-bound} We prove the lower bound by reduction from the problem of verifying the sequential consistency  of traces when  only the read-from relation is given. Hereafter, we call this problem VSC-read. The VSC-read problem consists in checking whether, given a directed graph $(S,\happbf{\hb}{S}=\happbf{\po}{S} \cup \happbf{\rf}{S})$
% where $S$ is a set of write and read events, $\happbf{\po}{S}$ is the program order relation that totally order all the events of each thread, and $\happbf{\rf}{S}$ is the {\em read-from} relation that maps each read event to the  write event from which it gets its value, there is an execution sequence $E$ such that $S=\dom{E}$, $ \happbf{\po}{S} =\happbf{\po}{E} $ and 
% $\happbf{\rf}{S} \subseteq \happbf{\cnf}{E}$. 
% The VSC-read problem is known to be  NP-complete in the size of the program 
%~\cite[Theorem~4.1]{DBLP:journals/siamcomp/GibbonsK97}.
%
%We now reduce the VSC-read problem to the event-driven consistency problem. Let   $(S,\happbf{\hb}{S}=\happbf{\po}{S} \cup \happbf{\rf}{S})$ be a directed graph. 
%To simplify the presentation\footnote{We assume that threads/messages are spawned/posted by a main thread, and that all shared variables get initialized to 0, also by the main thread. To make the presentation simple, we omit the events of the main thread.}, we assume w.l.o.g. that each write event is read by at least one read event. 
%The main idea of our reduction is to associate a message $p_e$  for each event $e$. The message $p_e$ will be executed by the handler $h$. The order of the execution of these messages will correspond to a linearization  of the set of event $S$ (since all these messages will be executed by the same handler $h$). However, this poses a challenge  since such  reduction from the VSC-read problem to the event-driven consistency problem will  fix the order of write events on the same
%variable (as it is implied by the conflict relation $\happbf{\cnf}{}$). To address this challenge we  rename the shared variables used by each event in $S$ and thus there will be no conflict relation between write-write events (and therefore between read-write events too). However, this leads to a new challenge which is how to make sure that between a write event $e \in S$ and a read event $e' \in S$ that is reading from $e$ there is no other scheduled write event   in $S$ on the same variable between $e$ and $e'$. To address the second challenge, we use an extra handler $h_x$ per variable $x$ that executes a number of independent messages (one  per write event on x in $S$). The order in which these messages are executed corresponds to the order in which the write events  on the same variable are scheduled. Furthermore, we make sure that each read event is scheduled after the write event it reads from and before the next scheduled write event on the same variable.
%
%Formally, for every write event $e$ in $S$ executed by a thread $t$, we create a message $p_e$ running on the thread handler $h$. The message 
%$p_e$ will be of the form [\texttt{$x_e$\,= 1};\,\texttt{$x_t$\,= 1};\, \texttt{$x_e$\,= 1};].  For a read event $e'$ in $S$ executed by a thread $t$ and reading from the write event $e$, we create a message $p_{e'}$ running on the thread handler $h$. The message 
%$p_{e'}$ will be of the form [\texttt{a=$y_e$};\,\texttt{$x_t$\,= 1};\, \texttt{a=$y_e$};]. We use the write event on $x_t$ to order the messages corresponding to events running on the same thread $t$ in $S$. In fact, we will simulate $\happbf{\po}{S}$ using $\happbf{\cnf}{S'}$ that will  totally order all the write events on $x_t$. This results in adding  a conflict relation $ \happbf{\cnf}{S'}$ between every two events corresponding to the writes on $x_t$ in two different messages $p_e$ and $p_{e'}$ iff $e \happbf{\po}{S} e'$. 
%
%The  statements on $x_e$ and $y_e$ are used to force a total order on the messages corresponding to events on the same variable such that  all the read messages are scheduled just after their corresponding write messages. To that aim we use an extra handler $h_x$ for each variable $x$ (used by the events of $S$). For each write event $e$ on the variable $x$, we create a message $p_{e,x}$ that will run the following sequence of statements [\texttt{$x_e$\,= 0};\, \texttt{$x_e$\,= 0};\,\texttt{$y_e$\,= 0};\, \texttt{$y_e$\,= 0};]. We then add a conflict relation from the first  write of $p_{e,x}$   to the first write of $p_e$ and from  the last write of $p_e$  to the second write of $p_{e,x}$. This will force the message $p_e$ to start and end before its corresponding read messages. For a read event $e'$  reading from the write event $e$,  we also add a conflict relation from the third write of $p_{e,x}$  to  the first  read of $p_{e'}$ and from the last read of $p_{e'}$ to  the last write of $p_{e,x}$. This conflict relation will force that all the message $p_{e'}$ will be executed just after the message $p_e$ without the interleaving  of  any other message that corresponds to a write event on $x$ between $p_e$ and $p_e'$.
%
%Observe that the messages $p_e$ are run one after the other (since they are on the same handler $h$). Furthermore, the constraints between the messages of the handler $h$ and the messages of the handler $h_x$ impose that the read message $p_{e'}$ is scheduled just after its corresponding write message $p_e$ and before the next scheduled write  message on the same variable.
%Let $(S',\happbf{\hb'}{S'}=\happbf{\po}{S'} \cup \happbf{\cnf}{S'}\cup  \happbf{\pb}{S'})$ be the constructed  \hb-trace from $(S,\happbf{\hb}{S}=\happbf{\po}{S} \cup \happbf{\rf}{S})$. It is then easy to see that:
%
%\begin{lemma}
% There is an execution sequence $E'$ such that $(S',\happbf{\hb}{S'})$ is the  \emph{\hb-trace} of  $E'$  if and only if there is an execution $E$ such that $S=\dom{E}$, $ \happbf{\po}{S} =\happbf{\po}{E} $ and 
% $\happbf{\rf}{S} \subseteq \happbf{\cnf}{E}$. 
%\end{lemma}
%
%\subsection{NP-hardness proof of the weak initial check problem}
%
%
%
%
%
%
%%\paragraph{Upper-bound} 
%% Let $E$ be an execution sequence,  $w$ be a sequence with $\valid Ew$, and $p$ be a given message. We assume that $p$ starts after $E$  and that  $p$ is not the first message on its handler in $w$ (see Section \ref{sec:checkwi}).  
%% 
%% We extend $w$ by completing all the unfinished messages and then  executing the message  $p$ if $p$ was not already started in $w$. Let $E'$ be the resulting execution sequence. 
%% Let  $(S',\happbf{\hb}{S'})$  be the  \hb-trace of E'. We construct from $(S',\happbf{\hb}{S'})$ another \hb-trace $(S,\happbf{\hb}{S})$ where $S$ contains all the events of S' (modulo renaming of the events) plus two extra events $e$ and $e'$. The new event $e$ (resp. $e'$) corresponds to the first  statement of a new message constructed from  the message  $\procof{\event_1}$ (resp.  $\procof{\event_2}$) by adding to it a write statement to a fresh variable $x_{new}$ at its beginning. The happens-before relation $\happbf{\hb}{S'}$ contains $\happbf{\hb}{S}$ plus a conflict relation from $e'$ to $e$ and a program order relation from $e$ to $e_1$ and from $e'$ to $e_2$. Having this conflict relation from $e'$ to $e$ will force the message $\procof{e'}$ (i.e., $\procof{\event_2}$) to be executed before  $\procof{e}$ (i.e., $\procof{\event_1}$). Finally, it suffices to check the event-driven consistency  of $(S',\happbf{\hb}{S'})$ in order to determine the existence of an  execution sequence $E'$ of the form $w'_1 . e_1 . w'_2 . e_2 . w'_3$ such that $E \mtequiv E'$.
%
%
%We  prove the lower bound by reduction from the  VSC-read problem. The reduction is similar to the one from the  event-driven consistency problem to the  weak initial check   problem except that we need to start from an execution sequence and reversing the order of two messages leads to the pattern used in the hardness  proof of the  event-driven consistency problem. The main idea of the proof is to replace the conflict relation from   $p_e$ to $p_{e,x}$ by a sequence of conflict relations that go through two particular messages $p'_{e,x}$ and $p''_{e,x}$ if they are executed in a certain order. Otherwise there is no conflict relation from $p_e$ to  $p_{e,x}$ and so the happens-before relation is acyclic by default.
% 
%We use the same set of assumptions as in the hardness proof of the  event-driven consistency problem. 
%We now reduce the VSC-read problem to the order reversing problem. Let   $(S,\happbf{\hb}{S}=\happbf{\po}{S} \cup \happbf{\rf}{S})$ be a directed graph. As in the previous proof we  associate a message $p_e$  for each event $e$. The message $p_e$ will be executed by the handler $h$. 
%For every write event $e$ in $S$ executed by a thread $t$, we have a message $p_e$. The message 
%$p_e$ is of the form [\texttt{$x_e$\,= 1};\,\texttt{$x_t$\,= 1};\, \texttt{$x'_e$\,= 1};].  For a read event $e'$ in $S$ executed by a thread $t$ and reading from the write event $e$, we have a message $p_{e'}$. The message 
%$p_{e'}$ is of the form [\texttt{a=$y_e$};\,\texttt{$x_t$\,= 1};\, \texttt{$y_{e'}$=0};]. 
%
%We use also an extra handler $h_x$ for each variable $x$ (used in the events of $S$). For each write event $e$ on the variable $x$, we have a message $p_{e,x}$ that will run the following sequence of statements [\texttt{$x_e$\,= 0};\, \texttt{$z_e$\,= 0};\,\texttt{$y_e$\,= 0};\, \texttt{$z'_e$\,= 0};]. We then add a conflict relation from  the first write of $p_{e,x}$ to the first  write of $p_e$. This will force the message $p_e$ to start after $p_{e,x}$. For a read event $e'$  reading from the write event $e$,  we also add a conflict relation from  the third write of $p_{e,x}$  to the first  read of $p_{e'}$. Observe that we do not impose a direct conflict relation from  $p_{e}$ or $p_{e'}$ to $p_{e,x}$.
%
%For each write event $e$ on the variable $x$, we have two messages $p'_{e,x}$ and $p''_{e,x}$ that run on a fresh handler $h_e$ the following sequence of statements 
%[\texttt{a\,= z};\, \texttt{$x'_e$\,= 1};] and [\texttt{$z_e$\,= 0};\, \texttt{a\,= x};] respectively. We add a conflict relation from the second last event of  $p_{e}$ to the second event of $p'_{e,x}$ and from the first write event of  
%$p''_{e,x}$  to the last event of  the second write event of  $p_{e,x}$. Observe that in the case that $p'_{e,x}$ is executed before $p''_{e,x}$, we have an indirect conflict relation from the last write of $p_e$ to  the second write of $p_{e,x}$   through $p'_{e,x}$ and $p''_{e,x}$. In the case where we execute $p''_{e,x} $ before $p'_{e,x}$, there is no  (indirect) happens-before relation from $p_{e}$ to $p_{e,x}$.
%
%
%%We add a conflict relation from the second write event of  $p_{e,x}$ to the second event of $p'_{e,x}$ and from the first write event of  
%%$p''_{e,x}$  to the last event of $p_e$. Observe that in the case that $p'_{e,x}$ is executed before $p''_{e,x}$, we have an indirect conflict relation from the second write of $p_{e,x}$ to the last write of $p_e$ through $p'_{e,x}$ and $p''_{e,x}$. In the case where we execute $p''_{e,x} $ before $p'_{e,x}$, there is no  (indirect) happens-before relation from $p_{e,x}$ to $p_e$.
%
%
%In similar manner, for each  read event $e'$ on $x$ in $S$  reading from the write event $e$, we have two messages $p'_{e',x}$ and $p''_{e',x}$ that run on a fresh handler $h_{e'}$ the following sequence of statements 
%[\texttt{a\,= z};\, \texttt{$y_{e'}$\,= 1};] and [\texttt{a=$z'_e$};\, \texttt{a\,= x};] respectively.  We add a conflict relation from the the last event of $p_{e'}$ to the second event of $p'_{e',x}$ and from the first read event of  
%$p''_{e',x}$  to the  last write event of  $p_{e,x}$. Observe that in the case that $p'_{e',x}$ is executed before $p''_{e',x}$, we have an indirect conflict relation from the last write of $p_{e'}$ to the last write of $p_{e,x}$ through $p'_{e',x}$ and $p''_{e',x}$.
%
%To set the order of all $p'_{e,x}$ and  $p''_{e,x}$ ($p'_{e',x}$ and $p''_{e',x}$), we will use  two messages $p$ and $p'$ on a fresh handler $h'$ that run the following  statements [\texttt{ $x_p$=1}; \texttt{ z=1};] and [\texttt{ $x_{p'}$=1}; \texttt{x\,= 1};] respectively. We add then a conflict relation  from the first read event of  $p'_{e,x}$ (resp. $p'_{e',x}$) to the  event of $p$ and from the  write event of  
%$p'$  to the last event of $p''_{e,x}$ (resp.  $p''_{e',x}$). Note that if $p$ is executed before $p'$ then  $p'_{e,x}$ (resp. $p'_{e',x}$) is executed before  $p''_{e,x}$ (resp. $p''_{e',x}$).
%
%
%Let $(S',\happbf{\hb'}{S'}=\happbf{\po}{S'} \cup \happbf{\cnf}{S'}\cup  \happbf{\pb}{S'})$ be the constructed  \hb-trace from $(S,\happbf{\hb}{S}=\happbf{\po}{S} \cup \happbf{\rf}{S})$. It is easy to see that there is an execution sequence $E$ such that $(S',\happbf{\hb}{S'})$ is the  \emph{\hb-trace} of  $E$ and where the message $p'$  is executed before $p$ and   $p''_{e,x}$ (resp. $p''_{e',x}$) is executed before  $p'_{e,x}$ (resp. $p'_{e',x}$).
%
%
%\begin{lemma}
%\label{lemma2}
% There is an execution sequence $E'$ such that $(S',\happbf{\hb}{S'})$ is the  \emph{\hb-trace} of  $E'$ and where the first event of message $p$ is  the  first executed event in $E$    if and only if there is an execution $E''$ such that $S=\dom{E}$, $ \happbf{\po}{S} =\happbf{\po}{E} $ and 
% $\happbf{\rf}{S} \subseteq \happbf{\cnf}{E}$. 
%\end{lemma}
%
%Imposing $p$ to be executed before $p'$ will impose that every  $p'_{e,x}$ (resp. $p'_{e',x}$) is executed before  $p''_{e,x}$ (resp. $p''_{e',x}$) and so there will be an indirect relation from the last write of  (resp. $p_e$) $p_{e'}$ to the last (second) write of $p_{e,x}$ through $p'_{e,x}$ and $p''_{e,x}$ ($p'_{e',x}$ and $p''_{e',x}$). Thus, we are in similar case as in the hardness proof of the  event-driven consistency problem.
%Furthermore, we have the first event of $E'$  can be the first event of $p$ since it is independent from any other event.
%
%
%\begin{lemma} 
%\label{lemma3}
%$p \in \winits{\emptyseq}{E}$  if and only if there is an execution sequence $E'$ such that $(S',\happbf{\hb}{S'})$ is the  \emph{\hb-trace} of  $E'$ and where the message $p$ is  executed before $p'$.
%\end{lemma}
%
%Finally, Theorem \ref{thm-lowerbound} can be see as an immediate  corollary of  Lemma \ref{lemma2}, and Lemma \ref{lemma3}.
%
%%
%%\subsection{The Order Reversing  Problem}
%%
%%In the following, we show:
%%
%%\begin{theorem}
%%The order reversing problem is NP-complete.
%%\end{theorem}
%%
%%The rest of this section is devoted to the proof of the above theorem.
%%
%%\paragraph{Upper-bound} 
%%Let   $E=w_1 . e_1 . w_2 . e_2 . w_3$ be an execution sequence  such that $e_1$ and $e_2$ are the two first events executed by two messages on the same thread handler $h$. Let  $(S,\happbf{\hb}{S})$  be the  \hb-trace of E. We construct from $(S,\happbf{\hb}{S})$ another \hb-trace $(S',\happbf{\hb}{S'})$ where $S'$ contains all the events of S (modulo renaming of the events) plus two extra events $e$ and $e'$. The new event $e$ (resp. $e'$) corresponds to the first  statement of a new message constructed from  the message  $\procof{\event_1}$ (resp.  $\procof{\event_2}$) by adding to it a write statement to a fresh variable $x_{new}$ at its beginning. The happens-before relation $\happbf{\hb}{S'}$ contains $\happbf{\hb}{S}$ plus a conflict relation from $e'$ to $e$ and a program order relation from $e$ to $e_1$ and from $e'$ to $e_2$. Having this conflict relation from $e'$ to $e$ will force the message $\procof{e'}$ (i.e., $\procof{\event_2}$) to be executed before  $\procof{e}$ (i.e., $\procof{\event_1}$). Finally, it suffices to check the event-driven consistency  of $(S',\happbf{\hb}{S'})$ in order to determine the existence of an  execution sequence $E'$ of the form $w'_1 . e_1 . w'_2 . e_2 . w'_3$ such that $E \mtequiv E'$.
%%
%%
%%\paragraph{Lower-bound} We  prove the lower bound by reduction from the  VSC-read problem. The reduction is similar to the one from the VSC-read problem to the event-driven consistency problem except that we need to start from an execution sequence and reversing the order of two messages leads to the pattern used in the previous proof. The main idea of the proof is to replace the conflict relation from   $p_e$ to $p_{e,x}$ by a sequence of conflict relations that go through two particular messages $p'_{e,x}$ and $p''_{e,x}$ if they are executed in a certain order. Otherwise there will be no conflict relation from $p_e$ to  $p_{e,x}$ and thus our happens-before relation is acyclic by default.
%% 
%%
%%We now reduce the VSC-read problem to the order reversing problem. Let   $(S,\happbf{\hb}{S}=\happbf{\po}{S} \cup \happbf{\rf}{S})$ be a directed graph. As in the previous proof we  associate a message $p_e$  for each event $e$. The message $p_e$ will be executed by the handler $h$. 
%%For every write event $e$ in $S$ executed by a thread $t$, we have a message $p_e$. The message 
%%$p_e$ is of the form [\texttt{$x_e$\,= 1};\,\texttt{$x_t$\,= 1};\, \texttt{$x'_e$\,= 1};].  For a read event $e'$ in $S$ executed by a thread $t$ and reading from the write event $e$, we have a message $p_{e'}$. The message 
%%$p_{e'}$ is of the form [\texttt{a=$y_e$};\,\texttt{$x_t$\,= 1};\, \texttt{$y_{e'}$=0};]. 
%%
%%We use also an extra handler $h_x$ for each variable $x$ (used in the events of $S$). For each write event $e$ on the variable $x$, we have a message $p_{e,x}$ that will run the following sequence of statements [\texttt{$x_e$\,= 0};\, \texttt{$z_e$\,= 0};\,\texttt{$y_e$\,= 0};\, \texttt{$z'_e$\,= 0};]. We then add a conflict relation from  the first write of $p_{e,x}$ to the first  write of $p_e$. This will force the message $p_e$ to start after $p_{e,x}$. For a read event $e'$  reading from the write event $e$,  we also add a conflict relation from  the third write of $p_{e,x}$  to the first  read of $p_{e'}$. Observe that we do not impose a direct conflict relation from  $p_{e}$ or $p_{e'}$ to $p_{e,x}$.
%%
%%For each write event $e$ on the variable $x$, we have two messages $p'_{e,x}$ and $p''_{e,x}$ that run on a fresh handler $h_e$ the following sequence of statements 
%%[\texttt{a\,= z};\, \texttt{$x'_e$\,= 1};] and [\texttt{$z_e$\,= 0};\, \texttt{a\,= x};] respectively. We add a conflict relation from the second last event of  $p_{e}$ to the second event of $p'_{e,x}$ and from the first write event of  
%%$p''_{e,x}$  to the last event of  the second write event of  $p_{e,x}$. Observe that in the case that $p'_{e,x}$ is executed before $p''_{e,x}$, we have an indirect conflict relation from the last write of $p_e$ to  the second write of $p_{e,x}$   through $p'_{e,x}$ and $p''_{e,x}$. In the case where we execute $p''_{e,x} $ before $p'_{e,x}$, there is no  (indirect) happens-before relation from $p_{e}$ to $p_{e,x}$.
%%
%%
%%%We add a conflict relation from the second write event of  $p_{e,x}$ to the second event of $p'_{e,x}$ and from the first write event of  
%%%$p''_{e,x}$  to the last event of $p_e$. Observe that in the case that $p'_{e,x}$ is executed before $p''_{e,x}$, we have an indirect conflict relation from the second write of $p_{e,x}$ to the last write of $p_e$ through $p'_{e,x}$ and $p''_{e,x}$. In the case where we execute $p''_{e,x} $ before $p'_{e,x}$, there is no  (indirect) happens-before relation from $p_{e,x}$ to $p_e$.
%%
%%
%%In similar manner, for each  read event $e'$ on $x$ in $S$  reading from the write event $e$, we have two messages $p'_{e',x}$ and $p''_{e',x}$ that run on a fresh handler $h_{e'}$ the following sequence of statements 
%%[\texttt{a\,= z};\, \texttt{$y_{e'}$\,= 1};] and [\texttt{a=$z'_e$};\, \texttt{a\,= x};] respectively.  We add a conflict relation from the the last event of $p_{e'}$ to the second event of $p'_{e',x}$ and from the first write event of  
%%$p''_{e',x}$  to  last write event of  $p_{e,x}$. Observe that in the case that $p'_{e',x}$ is executed before $p''_{e',x}$, we have an indirect conflict relation from the last write of $p_{e'}$ to the last write of $p_{e,x}$ through $p'_{e',x}$ and $p''_{e',x}$.
%%
%%To set the order of all $p'_{e,x}$ and  $p''_{e,x}$ ($p'_{e',x}$ and $p''_{e',x}$), we will use  two messages $p$ and $p'$ on a fresh handler $h'$ that run the following  statements [\texttt{ z=1};] and [\texttt{x\,= 1};] respectively. We add then a conflict relation  from the first read event of  $p'_{e,x}$ (resp. $p'_{e',x}$) to the  event of $p$ and from the  write event of  
%%$p'$  to the last event of $p''_{e,x}$ (resp.  $p''_{e',x}$). Note that if $p$ is executed before $p'$ then all $p'_{e,x}$ (resp. $p'_{e',x}$) are executed before  $p''_{e,x}$ (resp. $p''_{e',x}$) and vice-versa.
%%
%%
%%Let $(S',\happbf{\hb'}{S'}=\happbf{\po}{S'} \cup \happbf{\cnf}{S'}\cup  \happbf{\pb}{S'})$ be the constructed  \hb-trace from $(S,\happbf{\hb}{S}=\happbf{\po}{S} \cup \happbf{\rf}{S})$. It is easy to see that  there is an execution sequence $E$ such that $(S',\happbf{\hb}{S'})$ is the  \emph{\hb-trace} of  $E$ and where the message $p'$  is executed before $p$.
%%
%%\begin{lemma}
%% There is an execution sequence $E'$ such that $(S',\happbf{\hb}{S'})$ is the  \emph{\hb-trace} of  $E'$ and where the message $p$ is  executed before $p'$   if and only if there is an execution $E''$ such that $S=\dom{E}$, $ \happbf{\po}{S} =\happbf{\po}{E} $ and 
%% $\happbf{\rf}{S} \subseteq \happbf{\cnf}{E}$. 
%%\end{lemma}
%%
%\end{comment}

%% -*- mode: LaTeX; fill-column: 78; -*-

\section{Implementation} \label{sec:impl}
%========================================
\EventDPOR was implemented on top of \Nidhugg.
\Nidhugg~\cite{tacas15:tso} is a state-of-the-art stateless model checker for
C/C++ programs with Pthreads, which works at the level of the LLVM Intermediate
Representation.
% explores all their possible executions, and reports assertion violations and
% crashes, such as segmentation faults.
\Nidhugg comes with a selection of DPOR algorithms. One of them is
\OptimalDPOR, which we have used as a basis for \EventDPOR's implementation.

We have extended the data structures of \Nidhugg with the information
needed by \EventDPOR.
For instance, nodes in wakeup trees contain new information, such as the set
of parked wakeup sequences, and events in executions include the information in
$\tmpaccesses$, used to compute the \done set
as shown in \crefrange{algacsl:initialize-accesses}{algacsl:donesleeptree-add}
of~\cref{alg:eventdpor-access}.
The relation $\happbf{\hb}{E}$ is represented by a
vector clock per event, containing the set of preceding events.
When reversing races (in $\reverserace$) and checking for redundancy
(\cref{algacsl:event-test} of~\cref{alg:eventdpor-access}),
the relation $\happbf{\hb}{E}$ is extended by a saturation operation
(\cref{def:weakall} in \cref{sec:functions-appendix}) that captures ordering constrained induced by serialized message execution.
%% Transitive relations such as $\weaksatrel{\happbf{\hb}{E}}$ are represented by a
%% vector clock per event, containing the set of preceding events.
%% The saturation procedure in \cref{def:weakall} is implemented using
%% fixpoint iteration.

Concerning race reversal, instead of reversing multiple races between messages
executed on the same handler, our implementation detects and reverses only
the race induced by the first conflict, since other races cannot be reversed,
as explained using the example in \cref{fig:example1new}.
Moreover, in cases where $\reverserace$ would return several maximal executions that reverse a race, our implementation instead returns their union, even though it may not form an execution (e.g., since it may contain several incomplete executed messages on a handler). From this union, events will be removed adaptively during wakeup tree insertion to extract only those maximal executions that generate new leaves in a wakeup tree.

\hasbeenremoved{Moreover, we postpone the deletion of incomplete messages from $\exseq''$, described at
\crefrange{algl:converge-begin}{algl:converge-end} in \cref{alg:reverserace},
to the point before adding a new branch in the wakeup tree.
This avoids unnecessary creation of wakeup sequences that will not create any
new leaf in the wakeup tree, since the deletion of incomplete messages
can be guided by the redundancy checks performed during the insertion.
%
Finally, instead of computing $\happbf{sc}{E.w'}$ for each test of form
$p \in \winits{E}{w}$ at \cref{algacsl:event-test}
of \cref{alg:eventdpor-access}, we precompute $\happbf{sc}{E.w}$
before all the tests, which accelerates the redundancy check.}
%% During \textbf{Happens Before Check},
%% we consider additional scheduling constraints between messages that are
%% supposed to be added during \textbf{Witness Construction}.
%% These constraints are computed by a heuristic without doing expensive
%% \textbf{Witness Construction}. 

\section{Evaluation} \label{sec:eval}
%====================================
In this section, we evaluate the performance of our implementation and put it
into context.
%
Since currently there is no other SMC tool for event-driven programs to
compare against,\footnote{All our attempts to use \href{https://github.com/eth-sri/R4}{$R^4$} failed miserably; the tool has not been updated since~2016.}
we have created an API, in the form of a C header file, that
implements event handlers as pthread mutexes (locks) and simulates messages as
threads that wait for their event handler to be free. This API allows us to
use plain C/pthread programs to compare \EventDPOR with the \OptimalDPOR
algorithm implemented in \Nidhugg as baseline, but also with the
\emph{Lock-Aware Partial Order Reduction} (\emph{LAPOR})
algorithm~\citet{LAPOR@OOPSLA-19}, implemented in \GenMC. The \LAPOR algorithm
is often analogous to \EventDPOR w.r.t. the amount of reduction that it can
achieve when event handlers are modeled as global locks.
% Since we are comparing against \LAPOR, we have also included
We also include in our comparison the baseline DPOR algorithm of \GenMC that
tracks the modification order (\genmcmo{\small}) of shared variables.
% , and thus is analogous to the \OptimalDPOR algorithm of \Nidhugg.
%
For \Nidhugg, we used its \texttt{master} branch at the end of~2022;
% (\EventDPOR has been rebased on top of this branch.)
%
for \GenMC, we used version 0.6.1.%
%, because this is its most recent
%version that comes with a working implementation of \LAPOR.
\footnote{\GenMC v0.6.1 (released July 2021) warns that \LAPOR usage with
\genmcmo{\footnotesize} is experimental; in fact, \LAPOR support has been
dropped in more recent \GenMC versions.}
% . On the other hand, there is no \GenMC version without this warning and
% but we did not experience any problems using it.}
%
We have run all benchmarks on a Ryzen 5950X desktop running Arch Linux.
%% and used a timeout of ten hours.

%% These commands are generated in the result columns by the benchmark scripts
\newcommand\SZ{\scriptsize}
\newcommand\error{\textcolor{red}{\SZ \textdagger}\xspace}
\newcommand\timeout{\textcolor{blue}{\SZ \clock}\xspace}%\tiny\StopWatchEnd}}
\newcommand\notavail{\textcolor{gray}{\SZ n/a}\xspace}
\newcommand\bug{\SZ\color{red} bug }

\pgfplotstableset{% global config, for example in the preamble
  % these columns/<colname>/.style={<options>} things define a style
  % which applies to <colname> only.
  columns/benchmark/.style={string type,
    string replace*={_}{\_},
  },
  columns/tool/.style={string type},
  columns/lapormo_traces/.style={column name=\multicolumn{1}{r}{\lapormo{\SZ}}},
  columns/lapormo_time/.style={column name=\multicolumn{1}{r}{\lapormo{\SZ}}},
  columns/genmcmo_traces/.style={column name=\multicolumn{1}{r}{\genmcmo{\SZ}}},
  columns/genmcmo_time/.style={column name=\multicolumn{1}{r}{\genmcmo{\SZ}}},
  columns/optimal_traces/.style={column name=\multicolumn{1}{r}{\opt{\SZ}}},
  columns/optimal_time/.style={column name=\multicolumn{1}{r}{\opt{\SZ}}},
  columns/event_traces/.style={column name=\multicolumn{1}{r}{\evt{\SZ}}},
  columns/event_time/.style={column name=\multicolumn{1}{r}{\evt{\SZ}}},
  every head row/.style={before row={%
       \toprule
      & \multicolumn{4}{c}{Executions (Traces+Blocked)} & \multicolumn{4}{c}{Time (secs)}\\
      \cmidrule(r){2-5}\cmidrule(r){6-9}
      & \multicolumn{2}{c}{\GenMC} & \multicolumn{2}{c}{\Nidhugg} &
      \multicolumn{2}{c}{\GenMC} & \multicolumn{2}{c}{\Nidhugg} \\
      \cmidrule(r){2-3}\cmidrule(r){4-5}\cmidrule(r){6-7}\cmidrule(r){8-9}
    },after row=\midrule},
  every last row/.style={after row=\bottomrule},
  column type={r},
  %% Not the same column as lowercase "benchmark"
  create on use/Benchmark/.style={
    %% Breaks with _ in benchmark names
    create col/assign/.code={%
      \getthisrow{benchmark}\benchmark
      \getthisrow{n}\n
      \edef\entry{\benchmark(\n)}%
      \pgfkeyslet{/pgfplots/table/create col/next content}\entry
    },
  },
  columns/Benchmark/.style={
    column type={l},
  },
  columns={Benchmark,
    genmcmo_traces,lapormo_traces,optimal_traces,event_traces,
    genmcmo_time,lapormo_time,optimal_time,event_time},
  fixed,
  string type, %% Prevents dropping of trailing zeroes
  %% set thousands separator={},
}

We will compare implementations of different DPOR algorithms based on the
number of executions that they explore, as well as the time that this takes.
For some programs, \LAPOR also examines a fair amount
of \emph{blocked} executions (i.e., executions that cannot
be serialized and need to be aborted), which naturally affects its time
performance. In \cref{tab:eval}, we show the number of executions explored by an
entry of the form $T$+$B$, where $T$ is the number of complete traces and $B$
is the number of blocked executions. (We omit the $B$ part when
it is zero.)
% Refer to \cref{tab:eval}.

\begin{table}[t]
  \caption{Performance of different DPOR algorithm implementations.}
  \label{tab:eval}
  %% \smallertabcaptionspace
  \centering\SZ
  \setlength{\tabcolsep}{2pt}
  %\pgfplotstablevertcat{\output}{results/laban/writers.txt}
  \pgfplotstablevertcat{\output}{results/laban/posters.txt}
  \pgfplotstablevertcat{\output}{results/laban/buyers.txt}
  \pgfplotstablevertcat{\output}{results/laban/ping_pong.txt}
  %\pgfplotstablevertcat{\output}{results/laban/2PC.txt}
  \pgfplotstablevertcat{\output}{results/laban/consensus.txt}
  %\pgfplotstablevertcat{\output}{results/laban/db_cache.txt}
  \pgfplotstablevertcat{\output}{results/laban/prolific.txt}
  %\pgfplotstablevertcat{\output}{results/laban/mat_mult.txt}
  \pgfplotstablevertcat{\output}{results/laban/sparse-mat.txt}
  \pgfplotstablevertcat{\output}{results/laban/plb.txt}
  \pgfplotstabletypeset[
    every row no 3/.style={before row=\midrule},
    every row no 6/.style={before row=\midrule},
    every row no 9/.style={before row=\midrule},
    every row no 12/.style={before row=\midrule},
    every row no 15/.style={before row=\midrule},
    every row no 18/.style={before row=\midrule},
  ]{\output}
\end{table}

%% \iffref{app:artifact-link}{See \cref{app:artifact-link} for}{The
%%   supplementary material contains} a link to an artifact with all
%%   benchmarks and pre-compiled tools.

All the benchmark programs we use are parametric, typically on the number of
threads used (and thus messages posted); their parameters are shown inside
parentheses.
%
In the first program (\bench{posters}), each thread posts to a single event
handler two messages containing stores to some atomic global variable, and
then the value of this variable is checked by an assertion. This simple
program allows us to establish the baseline speed of all implementations. We
can see that \GenMC~\genmcmo{\small} is the fastest here. The reason is that 
it does not perform any checks whether the explored executions are sequentially 
consistenct, which allows it to be five times faster than \LAPOR, and seven 
to nine times faster than \Nidhugg's algorithm implementations. We can also
notice that \EventDPOR incurs a small but noticeable overhead over
\OptimalDPOR for the extra machinery that its implementation requires.

The next two benchmarks were taken from a paper by Kragl et al.~\citet{Kragl20}.
%
In \bench{buyers}, $n$ ``buyer'' threads coordinate the purchase of an item
from a ``seller'' as follows: one buyer requests a quote for the item from the
seller, then the buyers coordinate their individual contribution, and finally
if the contributions are enough to buy the item, the order is placed.
%
In \bench{ping-pong}, the ``pong'' handler thread receives messages with
increasing numbers from the ``ping'' thread, which are then acknowledged back
to the ``ping'' event handler.

Looking at~\cref{tab:eval}, we notice that, in both \bench{buyers} and
\bench{ping-pong}, all algorithms explore the same number of traces, but
\LAPOR also explores a significant number of executions that cannot be
serialized and need to be aborted. In fact, for both benchmarks, the aborted
executions significantly outnumber the traces explored.  This affects
negatively the time that \LAPOR takes, and \GenMC \lapormo{\small} becomes the
slowest implementation.  In contrast, \EventDPOR does not suffer from this
problem and shows similar scalability as baseline \GenMC and \OptimalDPOR.

With the four remaining benchmarks, we evaluate all implementations in programs
where algorithms tailored to event-driven programming, either natively
(\EventDPOR) or which are lock-aware (when handlers are implemented as locks),
have an advantage.
%
The first program (\bench{consensus}), again from the paper by Kragl et
al.~\citet{Kragl20}, is a simple \emph{broadcast consensus} protocol for $n$
nodes to agree on a common value. For each node~$i$, two threads are created:
one thread executes a \texttt{broadcast} method that sends the value of
node~$i$ to every other node, and the other thread is an event handler that
executes a \texttt{collect} method which receives~$n$ values and stores the
maximum as its decision. Since every node receives the values of all other
nodes, after the protocol finishes, all nodes have decided on the same value.
%
%% The second benchmark (\bench{db-cache}) is a key-value storage system
%% inspired from Memcached, a well known distributed cache application. There
%% are $n$ clients requesting a fixed sequence of storage accesses to a server
%% via UDP sockets (modeled as threads with event queues). On the server side
%% there is one worker thread per client to fulfill these requests. So
%% multiple worker threads on the server threads may race.
%
The next program (\bench{prolific}) is synthetic: $n$ threads send $n$
messages with an increasing number of stores to and loads from an atomic
global variable to one event handler.
%
The \bench{sparse-mat} program computes the number of non-zero elements of a
sparse matrix of dimension $m \times n$, by dividing the work into $n$ tasks
sent as messages to different handlers, which compute and join their results.
%
The last benchmark (\bench{plb}) is taken from a paper by Jhala and
Majumdar~\citet{popl07:JhalaM}. A fixed sequence of task requests is received
by the main thread. Upon receiving a task, the main thread allocates a space
in memory and posts a message with the pointer to the allocated memory that
will be served by a thread in the future.

Refer again to~\cref{tab:eval}.
%
In \bench{consensus}, all algorithms start with the same number of traces, but
\LAPOR and \EventDPOR need to explore fewer and fewer traces than the other
two algorithms, as the number of nodes (and threads) increases. Here too,
\LAPOR explores a significant number of executions that need to be aborted,
which hurts its time performance. On the other hand, \EventDPOR's handling
of events is optimal here.
%
The \bench{prolific} program shows a case where algorithms not tailored to
events (or locks) explore $(n-1)!$ traces, while \LAPOR and \EventDPOR explore
only $2^n-2$ consistent executions, when running the benchmark with
parameter $n$. It can also be noted that \EventDPOR scales \emph{much} better
than \LAPOR here in terms of time, due to the extra work that \LAPOR needs to
perform in order to check consistency of executions (and abort some of them).
%
The \bench{sparse-mat} program shows another case where algorithms that are
not tailored to events explore a large number of executions unnecessarily
(\timeout denotes timeout). This program also shows that \EventDPOR beats
\LAPOR time-wise even when \LAPOR does not explore executions that need to be
aborted.
%
Finally, \bench{plb} shows a case on which \EventDPOR and \LAPOR really
shine. These algorithms need to explore only one trace, independently of the
size of the matrices and messages exchanged, while DPOR algorithms not
tailored to event-driven programs explore a number of executions which
increases exponentially and fast.

We remark that, in all benchmarks, the inexpensive checks for redundancy were
sufficient, and \EventDPOR explored the optimal number of traces.
Results from an extended set of benchmarks appear in~\cref{app:eval-complete}.

%% -*- mode: LaTeX; fill-column: 78; -*-

\section{Concluding Remarks}
\label{sec:conclusions}

In this paper, we presented a novel SMC algorithm, \EventDPOR, tailored to the
characteristics of event-driven multi-threaded programs running under the SC
semantics. The algorithm was proven correct and optimal for event-driven
programs in which the variable accesses of events do not depend on how their
execution is interleaved with other threads.

We have implemented \EventDPOR in the \Nidhugg tool, and we will open-source
our implementation.
%
With a wide range of event-driven programs, we have shown that \EventDPOR
incurs only a moderate constant overhead over its baseline implementation
(\OptimalDPOR), it is exponentially faster than existing state-of-the-art SMC
algorithms in time and number of traces examined on programs where events'
actions do not conflict, and does not suffer from performance degradation
caused by having to examine
% a significant number of
non-serializable executions.
%
%% \bjcom{Should we include:
%% Moreover, in our benchmarks, also those that are not non-branching,
%% \EventDPOR explores only the optimal number of executions, and never
%% had to resort to a potentially expensive decision procedure.}

\EventDPOR assumes that handlers can process their events in arbitrary order.
Directions for future work include to retarget \EventDPOR for event-driven
programs with other policies (e.g., FIFO), and for specific event-driven
execution models.

\section{Reproducible Artifact}
An anonymous artifact containing the benchmarks and all the tools used in the evaluation, including our Nidhugg with Event DPOR, is available at
\url{https://doi.org/10.5281/zenodo.7929004}.


\bibliographystyle{splncs04}
\bibliography{./bibdatabase.bib}

\appendix
\clearpage
%\section{Binary Artifact}
\label{app:artifact-link}
An anonymous artifact containing the benchmarks and all the tools used in the
evaluation, including our \Nidhugg version with \EventDPOR, is available at
\url{https://drive.google.com/file/d/17BbkGYfqSy-6OsbTWWzhEY5AYr2Bs9wL/view?usp=sharing}.
The SHA256 hash of the file is
\texttt{3a5198802c3396a3ccaecd4dfb7bde4a620a5709d9a87b5d0cb3ace87bad9f39}.

%% Note that the version in the artifact does not implement the \textbf{Witness
%%   Construction} or \textbf{Decision Procedure} steps in \cref{sec:checkwi}, and
%% can misbehave on programs where these steps are required. Also, branches in
%% messages are only supported under some conditions.

\section{Detailed Descriptions of Auxiliary Functions}
\label{sec:functions-appendix}
In this section, we describe in detail the functions that are called by \EventDPOR, and were briefly described at the end of \cref{sec:algo:access-sets}
Some of these functions extend the happens-before relation $\happbf{\hb}{E}$ on an execution with
additional ordering constraints that are enforced  in the event-driven execution model, stemming from the fact that
%% induced by the event-driven execution model, in which
%% When constructing executions that respect some particular happens-before relation, one must in fact also respect additional ordering constraints 
%% induced by the event-driven execution model, in which
each handler must execute its messages in some sequential order. The following \emph{saturation operation} adds such additional orderings imposed by
any ordering relation on events.
%%
\begin{definition}[Saturation]
  \label{def:weakall}
Let $E$ be a sequence of events, and $\happbf{}{E}$ be an irreflexive partial order on the events of $E$. 
We define
$\weaksatrel{\happbf{}{E}}$ as the smallest transitive relation $\happbf{\sat}{E}$ on the events of $E$ which includes
$\happbf{}{E}$ and satisfies the constraint that  whenever $e$ and $e'$ are events in different messages on the same handler,
and there is an event $e''$ in the same message as $e$ and an event $e'''$  in the same
  message as $e'$ with $e'' \happbf{\sat}{E} e'''$,
then $e \happbf{\sat}{E} e'$.
%% \begin{enumerate}[(i)]
%% \item \label{rule:saturation-1}
%% \end{enumerate}
%% Let $\edseq{E}{E'}$ denote $\weaksatrel{\hbmseq{E}{E'}}$ ($\weakall$ stands for ``event-driven''). Let $\happbf{\weakall}{E}$ denote $\edseq{E}{\varepsilon}$, where $\varepsilon$ is the empty sequence.
\qed
\end{definition}
In the above definition, note that it is not required that $e$ is distinct from $e''$, nor that $e'$ is distinct from $e'''$.


\subsection{Reversing Races}
\label{sec:race-reversals-appendix}
A key procedure of \EventDPOR is $\reverserace({E},{\event},{\event'})$ which constructs new executions by analyzing and reversing a race in an explored execution.
This procedure is given a race $\event \revrace{\exseq} \event'$ in the currently explored execution $\exseq$ (at \cref{algacsl:race-loop} of \cref{alg:eventdpor-access}),
and returns a set of maximal executions that reverse the race.
%% in the sense that they
%% perform the second event $e'$  of the race without performing the first one, and (except for $e'$) only contain events that are not affected by the race.
More precisely, it returns a set of pairs of form $\pair{E'}{u.e'}$, such that
\begin{inparaenum}[(i)]
\item
  %% $E'.u \mtprefix E$ and
$E'.u$ is a maximal happens-before prefix of $E$ such that $E'.u.e'$ is an execution, and
\item $\dom{E'}$ is a maximal subset of $\dom{E'.u}$ such that $E' \leq E$.
\end{inparaenum}

%% Let us first define constraints that must be satisfied by $\exseq'.u.e'$.
%% Given a race $\event \revrace{\exseq} \event'$, let $\event''$ be the $\happbf{\po}{E}$-predecessor of $\event'$.
%% Define the following two subsequences of $E$:
%%   \begin{itemize}
%%   \item $\exseqmust$, defined as the subsequence  of events $\event'''$ such that
%%     $\event''' \happbf{\hb}{E} \event''$, also including $\event''$,
%%   \item $\exseqmay$, defined as the subsequence of events $\event'''$ such that
%%     $\event \nhappbf{\hb}{E} \event'''$, i.e., the events that do not happen-after $\event$.
%%   \end{itemize}
%%   Intuitively, $\exseqmust$ is the set of events that \emph{must} be present in $\exseq'.u$, for the reason that they are the predecessors that enable $\event'$. The sequence $\exseqmay$ are the events that \emph{may} be present in $\exseq'.u$ for the reason that they are not affected by the race: the events not in
%%   $\exseqmay$ happen-after $\event$ and are therefore affected by the race: when the race is reversed, they may be affected (e.g., by reading a different value).
%%   The race reversal procedure will return the set of maximal executions of form $\exseq'.u.e'$ such that
%%   $\exseqmust \sqsubseteq E'.u \sqsubseteq \exseqmay$. Of course, it need produce only one per equivalence class.


%% Let us consider two particular cases.
%%   \begin{itemize}
%%     \item
%%       If the events in $\exseqmust$ can not be ordered into an execution, then the race cannot and should not be reversed. An illustration of this phenomenon was
%%       given in \cref{fig:example1new}, where the race on \texttt{y} cannot be reversed, since $\exseqmust$ would contain the two writes to \texttt{x} which
%%       cannot be included in an execution (without the writes to \texttt{y}).
%%     \item
%%       If the events in $\exseqmay$ can be ordered into an execution (which is often the case), then that sequence will be returned as the unique maximal execution.
%% %% \bjcom{A simple illustration would be a simple event-event race}
%%   \end{itemize}
%%   If $\exseqmust$ can be reordered into an execution, but constraints imposed by the event-driven execution model prevent $\exseqmay$ from being reordered to an execution, 

The procedure $\reverserace({E},{\event},{\event'})$ is shown in \cref{alg:reverserace}. Let $E''$ be the set of events of $E$ that are not affected by the race (\cref{algl:revrace-init-notdep}):
this is the set of events $e'''$ with $\event \nhappbf{\hb}{E} \event'''$. If $E''$ can be reordered to form an execution, the code at
\crefrange{algl:revrace-init-ordering}{algl:converge-end} will have no effect;
$\reverserace$ will terminate and returns its linearization.
However, there are situations in which $E''$ cannot be reordered into an execution.
For instance, $E''$ may contain two incomplete messages on the same handler because the remaining parts of these messages happen-after $e$ in $E$.
Since an execution may contain at most one incomplete message per handler,
$\reverserace$ then performs a sequence of message removals and reorderings to produce a set of maximal wakeup sequences.
  The procedure employs the saturation operation of \cref{def:weakall} to constrain the ordering between messages on the same handler. 
% A simple example includes message-message races, illustrated by \cref{fig:example1new}, where all of message $p_1$ must be removed from $\exseqmay$ in order to form an execution.
  The procedure maintains
%%  a sequence $E''$, initialized to the events that could possible be in $E'.u$
  an ordering relation $\happbf{sc}{E''}$ on $E''$, initialized to $\weaksatrel{\happbf{hb}{E''}}$ (\cref{algl:revrace-init-ordering}).
It thereafter performs a sequence of steps in which messages are removed from $E''$ and/or the ordering relation $\happbf{sc}{E''}$ is extended. Some steps may be resolved nondeterministically: in such cases the procedure pursues all possible alternatives, potentially resulting in several returned sequences. The steps of \cref{alg:reverserace} are the following.
\begin{algorithm}[!htp]
\SetAlFnt{\small\sf}
\BlankLine
\Fn{$\reverserace({E},{\event},{\event'})$}{
  \keyword{let} $E''$ be the subsequence of $E$ consisting of the events $e'''$ with $\event \nhappbf{\hb}{E} \event'''$\;\label{algl:revrace-init-notdep}
  \keyword{let} $\event''$ be the last event performed by $\procof{\event'}$\;
  $\happbf{sc}{E''} := \weaksatrel{\happbf{hb}{E''}}$\;\label{algl:revrace-init-ordering}
  \keyword{let} $S=\{E''\}$\label{algl:revrace-lin-seq}\;
  \ForEach{$F\in S$}{
    \Repeat{each handler in $F$ has at most one incomplete message}{\label{algl:revrace-msg-del-start}
      \label{algl:converge-begin}
      \If{an incomplete message includes an event $e'''$ with $e''' \happbf{\hb}{F} e''$}{\label{algl:revrace-rule-must}
        remove all other incomplete messages on same handler from $F$\;
      }
      \If{several incomplete messages execute on one
        handler}{\label{algl:revrace-rule-choose}
        \ForEach{incomplete message $p$ in the same handler}{
          construct a sequence $U$ where all the messages except $p$ from
          this handler are removed\;
          add $U$ to the set $S$\;
        }
        delete $F$ from $S$\;
        pick another sequence $F$ from $S$\;
      }
      \ForEach{incomplete message $p$}{\label{algl:revrace-rule-last}
        add relation $\happbf{sc}{F}$ from all other messages on
        same handler to $p$ and saturate\;
      }
      \ForEach{cycle in $\happbf{sc}{F}$}{\label{algl:revrace-rule-cycle}
        remove a message in the cycle\;
      }
      remove events that follow already removed events in the $\happbf{\hb}{F}$ ordering\;
      \If{an event $e'''$ s.t. $e''' \happbf{\hb}{F} e''$ is
        deleted}{\label{algl:revrace-irreversible}
        remove $F$ from $S$ and exit the loop;
      }
    }\label{algl:converge-end}
  }\label{algl:revrace-msg-del-end}
  %$E'$ is the maximal prefix s.t. $E'.u = E''$ \ $E' \leq E''$ and
  %$E' \leq E$\; \label{algl:revrace-determine-prefix}
  \keyword{let} $WSS=\emptyset$ \tcp*[f]{set of wakeup sequences}\;
  \ForEach{$F\in S$}{
    \ForEach{two messages from the same handler that are not ordered by $\happbf{sc}{F}$}{
      add relation $\happbf{sc'}{F}$ as they appear in $F$; \label{algl:revrace-msg-order}
    }
    \Repeat{until $\happbf{sc}{F}$ and $\happbf{sc'}{F}$ together
      are acyclic}{
      nondeterministically pick two messages ordered by the relation $\happbf{sc'}{F}$ and
      reverse the order; \label{algl:revrace-next-msg-order}
    }
    topologically sort $F$ respecting
    $\happbf{sc}{F}$ and $\happbf{sc'}{F}$ \label{algl:revrace-linearize}\;
    extract largest common prefix $E'$ of $F$ and $E$\;
    add \tuple{E',.u.e'} to $WSS$, where $F=E'.u$\;
  }
  $\Return(WSS)$\;
}
\caption{Reversal of a Race.}
\label{alg:reverserace}
\end{algorithm}


\begin{description}
  \item[\cref{algl:revrace-lin-seq}]
    After the loop from \cref{algl:revrace-msg-del-start}
    to \cref{algl:revrace-msg-del-end}, the set $S$ will contain all
    the possible sequences with at most one incomplete message per handler.
  \item[\cref{algl:revrace-rule-must}]
    If an incomplete message includes an event $e'''$ with $e''' \happbf{hb}{F} e''$, then
    any other message on the same handler which is not completely executed in $F$ must be removed.
    %% \item A message containing events that are ordered (by $\happbf{sc}{E''}$) after events in an incompletely executed message
    %%   on the same handler must be discarded.
  \item[\cref{algl:revrace-rule-choose}]
    If several incomplete messages execute on the same handler, then
    finds all the possible sequence where only one of the incomplete
    messages is present and saves them to $S$.
    
  \item[\cref{algl:revrace-rule-last}]
    Whenever a handler has an incomplete message $p$, any other message $p'$
    on that handler must be executed before $p$, represented by extending $\happbf{sc}{F}$ from the last
    event of $p'$ to the first event of $p$ and then saturating.
  \item[\cref{algl:revrace-rule-cycle}]
    If $\happbf{sc}{F}$ becomes cyclic during the filtering and ordering procedure, then each cycle should be broken by
    removing the events in a suitable message.
  \item[\cref{algl:revrace-irreversible}]
    It is possible to have two or more incomplete messages from the same handler in $F$ each having at least one
    event that happens-before $e''$. Because of this reason or
    non-deterministic choice during message deletion process described
    previously, an event $e'''$ such that
    $e''' \happbf{\hb}{F} e''$ might be deleted from
    $\exseq''$. Then the algorithm removes $F$ from $S$.
  \item[\cref{algl:revrace-msg-order}]
    By adding additional relation $\happbf{sc'}{F}$,
    the algorithm determines a total order on the messages from the same handler.
  \item[\cref{algl:revrace-next-msg-order}]
    If $\happbf{sc}{F}$ and $\happbf{sc'}{F}$ together form a
    cycle, the algorithm tries to guess another order
    $\happbf{sc'}{F}$. Systematic search of $\happbf{sc'}{F}$ is a NP-complete
    problem in general case (see \cref{thm-consistency} below). But for the
    programs we have tried so far, doing \cref{algl:revrace-msg-order}
    is sufficient.
  \item[\cref{algl:revrace-linearize}]
    The sequence $u$ is linearized by topological sort procedure while
    respecting $\happbf{sc}{F}$ and $\happbf{sc'}{F}$.
\end{description}

As an illustration, consider the race on \texttt{x} in the program of \cref{fig:example3}.
Here, there is a unique (up to equivalence) maximal execution which reverses the race, which consists of all events that post messages, all events in messages
    $p_2$ and $q_2$, and the assignment to \texttt{d} by $p_1$. The read of \texttt{x} by~$q_2$ should be ordered last, since it corresponds to the racing event $e'$.
    Message $q_1$ is removed by the rule at \cref{algl:revrace-rule-must}, whereby also the second of event of $p_1$ is removed, since it reads from the first event in $q_1$.

\paragraph{\bf Event-driven Consistency.}
When describing \cref{algl:revrace-next-msg-order} above, we stated that the problem of determininig whether a given happens-before relation
can be obtained from some execution is NP-complete. This follows from NP-completeness of the event-driven consistency problem.
The event-driven consistency problem consists in checking whether, for a given  directed graph 
$(S,\happbf{\hb}{S})$ where $S$ is a set of events and $\happbf{\hb}{S}$ is a set of edges,  there is an execution sequence $E$ such that $(S,\happbf{\hb}{S})$ is the  \emph{\hb-trace} of  $E$.

\begin{theorem}
\label{thm-consistency}
The event-driven consistency problem is NP-complete.
\end{theorem}

The proof of the above theorem can be found in \cref{sec:complexity-proof-consistency}.  Given this NP-hardness result,  we define a procedure  to reverse races (\cref{sec:race-reversals-appendix}) that makes use of a saturation procedure to constrain the ordering between messages and therefore reduces the number of cases to  consider. 


%%     \begin{itemize}
%%   \item the earliest message which is discarded, or preceded by another one on its handler,
%%   \item otherwise the entire computation preceding $\event$.
%%   \end{itemize}
    %% \bjcom{We should describe how this procedure is guided by sleep sets}

\subsection{Wakeup Tree Insertion}
\label{sec:wakeuptrees}
In this section, we formally define wakeup trees, and present the procedure $\insertwusname$ for inserting wakeup sequences, and $\insertpendingwuname$ for inserting parked wakeup sequences.

%% Define an \emph{ordered tree} as a pair
%% $\tuple{B,\prec}$, where
%% $B$ (the set of \emph{nodes}) is a finite prefix-closed set of sequences of
%% messages, with the empty sequence $\emptyseq$ being the root.
%% The children of a node $u$, of form
%% $u.p$ for some set of messages $p$, are ordered by $\prec$.
%% In the tree $\tuple{B,\prec}$, such an ordering between children is
%% extended to a total order $\prec$ on~$B$ by letting
%% $\prec$ be the induced post-order relation between the nodes in $B$.
%% Thus, if the children $u.p_1$ and $u.p_2$ are ordered as
%% $u.p_1 \prec u.p_2$, then $u.p_1 \prec u.p_2 \prec u$ in the induced post-order.

\begin{definition}[Wakeup Tree]
\label{def:Wut}
A \emph{wakeup tree} is an ordered tree $\tuple{B,\prec}$, where
$B$ (the set of \emph{nodes}) is a finite prefix-closed set of sequences of
messages, with the empty sequence $\emptyseq$ being the root.
The children of a node $u$, of form
$u.p$ for some set of messages $p$, are ordered by $\prec$.
In the tree $\tuple{B,\prec}$, such an ordering between children is
extended to a total order $\prec$ on~$B$ by letting
$\prec$ be the induced post-order relation between the nodes in $B$
(i.e., if the children $u.p_1$ and $u.p_2$ are ordered as
$u.p_1 \prec u.p_2$, then $u.p_1 \prec u.p_2 \prec u$ in the induced post-order).
\qed
\end{definition}

\begin{algorithm}[t]
\newcommand{\lIfElse}[3]{\lIf{#1}{#2 \textbf{else}~#3}}
\SetAlFnt{\small\sf}
\SetKw{Continue}{continue}
%% \Initial{$\insertrec{E}{\emptyseq}{v}{\tuple{B,\prec}}$}
\BlankLine
\Fn{$\insertwus{v}{E'}{u}$}{
  \lIf{$v$ is the empty sequence \keyword{or} $u$ is a leaf in $\wut{E'}$}{\Return{}}\label{alg:wut-insert-empty}
%%   \keyword{let} $L$ be the list of children of $u$ in $\wut{E'}$ from left to right\;\label{alg:wut-insert-forever}
  \ForEach{\mbox{\rm child $u.p$ of $u$, in order (from left to right)}}{\label{algl:wut-foreach-child}
    \If(\tcp*[f]{If a new message is not started ...}){$p$ does not start after ${\exseq'.u}$}{\label{alg:no-message}
      \If{$p \in \winits{\exseq'.u}{v}$}{
        \lIfElse{$\nextev{\exseq'.u}{p} \in v$}{$\insertwus{v\remove p}{E'}{u.p}$}{\Return{}\label{alg:travers-no-message}}
      }
    }
    \Else{\label{alg:message}
      \If{$p$ is the first (if any) message on its handler in $v$}{
        \lIfElse{$\nextev{\exseq'.u}{p} \in v$}{$\insertwus{v\remove p}{E'}{u.p}$}{\Return{}\label{alg:wut-pfirst}}
      }
      \ElseIf{$p$ is fully present in $v$\label{alg:wut-pfull}} {
	      \lIfElse{$p \in \winits{\exseq'.u}{v}$}{$\insertwus{v\remove p}{E'}{u.p}$}{
         %\lElse{
%%            add the run of $p$ from $v$ in $\done'(\exseq'.u)$\bjcom{Hmmm}\;
           \Continue
         }\label{alg:wut-pfull-continue}
      }
      \Else{
        add $v$ to $\pendingwus{E'.u.p}$\;
        \Return{}\;
      }
    }
%%     add $p$ to $\done'(\exseq'.u)$\bjcom{Hmmm}
  }
%%   Construct $v'$ from events in $v$ such that $\valid{E.u}{v'}$ and $\neg \redundant{E.u}{\done'}{v'}$\;\label{algwut:new-branch}
insert $v$ as a new branch from $u$, ordered after the existing children of $u$\;\label{alg:wut-insert-branch}
  \Return{}\;
}
\caption{Insertion into Wakeup Tree.}
\label{alg:wakeuptree}
\end{algorithm}


\hasbeenremoved{The insertion of a new wakeup sequence $v$ into a wakeup tree $\wut{E'}$ should enforce property P2.
The insertion of a new wakeup sequence $v$ into a wakeup tree $\wut{E'}$ is performed by
descending step-by-step from the root of the tree. At each node $u$, it
should, for each of its children (of form $u.p$), test whether $p \in\winits{E'.u}{v'}$, where
$v'$ is obtained from $v$ by removing the events that are already in $u$.
If $p \in\winits{E'.u}{v'}$ then insertion moves to the child $u.p$. If the test fails for all
children, then $v'$ should be added as a right-most child of $u$.
}

%% whenever the exploration tree $\exseqs$ contains a node of form $\exseq.p$, then the algorithm will not add an execution of form $\exseq.w$ which is contained in
%% some execution of form $\exseq.p.w'$ for some $w'$, i.e., for which $p \in \winits{\exseq}{w}$. Such a check for redundancy is performed


%% \EventDPOR maintains a wakeup tree $\wut{E'}$ for each prefix $E'$ of the currently explored execution.
%% The algorithm strives to maintain the following two invariants.
%% \begin{enumerate}
%% \item \label{def:Wut:p1} %% modified
%%   No leaf $u$ of $\wut{E'}$ is redundant after $E'$, i.e., $\neg \redundant{E'}{\done}{u}$.
%% \item \label{def:Wut:p2} %% same
%%   For any sequences $u$, $w,$ and message $p$, such that
%%   $u.p$ is a node and $u.w$ is a leaf in $\wut{E'}$ with
%%   $u.p \prec u.w$, we have $p \not\in\winits{E'.u}{w}$.
%% \qed
%% \end{enumerate}
%% These invariants may be difficult to maintain when the check for redundancy is not exact \bjcom{Comment more?}


Insertion of a wakeup sequence $v$ into the wakeup tree $\wut{E'}$ is performed by calling the function $\insertwus{v}{E'}{u}$ with parameters $v$ and $E'$, and the parameter $u$ being the empty sequence.
The call $\insertwus{v}{E'}{\emptyseq}$ will, if $v$ conflicts with all its current leaves, extend the wakeup tree $\wut{E'}$ by a new leaf $v'$ such that $v' \mtequiv{E'} v$.
The recursive function $\insertwus{v}{E'}{u}$, shown in \cref{alg:wakeuptree},
traverses the wakeup tree $\wut{E'}$ from the root downwards, where $u$ is the current point of the traversal.
%% whether $v$ is redundant: if so, it leaves $\wut{E'}$ unchanged, if not it inserts a new leaf.
The initial call is performed with $u$ being the empty sequence.
%% The current point of the traversal is represented by the sequence $u$, hence an initial call will set this parameter to $\emptyseq$.
Each invocation of $\insertwus{v}{E'}{u}$
first checks whether a leaf has been reached or all of $v$ has already been examined, in which case nothing new should be added to $\wut{E'}$ (\cref{alg:wut-insert-empty}).
Thereafter, it considers the children of $u$ (of form $u.p$) from left to right.
For each child $u.p$, the algorithm tries to determine whether or not $p \in \winits{\exseq.u}{v}$.
If $p$ does not start after $E'.u$ then $p \in \winits{\exseq.u}{v}$ then $p \in \winits{\exseq.u}{v}$ can be checked by simple inspection at \crefrange{alg:no-message}{alg:travers-no-message} (as described in the second paragraph of \cref{sec:checkwi}).
The algorithm traverses to $u.p$ by a call to $\insertwus{v\remove p}{E'}{u.p}$ if $p \in \winits{\exseq.u}{v}$, otherwise it
considers the next child of $u$ if $p \not\in \winits{\exseq.u}{v}$.
If $p \in \winits{\exseq.u}{v}$ but $p$ does not appear in $v$, then actually no wakeup sequence need be inserted (\cref{alg:travers-no-message}).
%% The cases for which 
%%  is straight-forward, it , if $p \in \winits{\exseq.u}{v}$, insertion 
%% If $p$ does not start after $E'.u$, then the check whether $p \in \winits{\exseq.u}{v}$ is straight-forward, \revise{as described in the second paragraph of \cref{sec:checkwi}}, and
%% performed at \crefrange{alg:no-message}{alg:travers-no-message}. In case the test succeeds, but $p$ does not appear in $v$, then actually no wakeup sequence need be inserted.
If $p$ starts after $E'.u$ (\cref{alg:message}), then
\begin{itemize}
\item the case in which  $p$ is the first (if any) message on its handler in $v$, considered at \cref{alg:wut-pfirst} is performed according to the \textbf{Simple Check} in \cref{sec:checkwi};
\item if $p$ executes to completion in the sequence $v$ (\cref{alg:wut-pfull}), then $v$ contains sufficient information to decide whether $p \in \winits{\exseq.u}{v}$ using the remaining sequence of checks in \cref{sec:checkwi};
\item if none of these two cases apply, then more information is needed about which accesses $p$ performs when it is executed. Therefore the sequence $v$ is ``parked'' at the node $u.p$: the insertion of $v$ will be resumed when the node $u.p$ is extended to a maximal execution starting with $E'.u.p$, which happens at \cref{algacsl:insert-pendingwu} of \cref{alg:eventdpor-access} with $E'$ being $E'.u$.
\end{itemize}
If all children $u.p$ of $u$ have been traversed with failing tests for $p \in \winits{\exseq.u}{v}$, then $v$ is added as a new branch from $u$, ordered after the already existing children (\cref{alg:wut-insert-branch}).
%% \footnote{An optimization over the Optimal DPOR algorithm of~\cite{optimal-dpor-jacm} is that in the case where
%% $p \in \winits{\exseq.u}{v}$ but $\nextev{E.u}{p} \not\in v$ (implying that if $p \not\in \inits{\exseq.u}{v}$), the insertion leaves the
%% wakeup tree unchanged. This optimization does not miss executions,
%% since the algorithm will anyway later explore continuations of $E.u.p$ that include $E.u.v$ as a happens-before prefix.}


\begin{algorithm}[t]
\SetAlFnt{\small\sf}
\BlankLine
\Fn{$\insertpendingwu{v}{E'}$}{
  $\keyword{let} \ E''.p  \ \mbox{be} \ E'$\;
  \lIf{$v$ is the empty sequence}{\Return{}}
  \lIf{$\exseq''$ was formerly a leaf in a wakeup tree}{\Return{}}\label{alglpwu:former-leaf}
%%   $\keyword{let} \ \mbox{$\nextev{E'}{p}$ be the event following $E'$ in $E$}$\;
  \If{\label{algl:pwu-if-in-wi}$p \in \winits{\exseq''}{v}$}{
    $\keyword{let} \ q \ \mbox{be the message following $E'$ in $E$}$\;
    $\insertpendingwu{v\remove p}{E'.q}$\;\label{alglpwu:recursive-call}
   }
   \Else{
%%      \If{$\nextev{E'}{p} = \fst{p}$} {add the run of $p$ from $\exseq$ in $\done'(\exseq')$}
%%     \lElse{add $p$ to $\done'(\exseq')$}
     $\insertwus{v}{\exseq''}{\emptyseq}$\tcp*{If conflict w. next event, insert into the wakeup tree}\label{alglpwu:event-insert}
   }
}
\caption{Insertion of Parked Wakeup Sequences.}
\label{alg:pendingwus}
\end{algorithm}


It remains to define the procedure for inserting parked wakeup sequences (called at~\cref{algacsl:insert-pendingwu} of~\cref{alg:eventdpor-access}).
This insertion is described in~\cref{alg:pendingwus}, as the function
$\insertpendingwu{v}{E'}$, which inserts a wakeup sequence $v$ which is parked after a prefix $E'$ of the execution $E$.
The function first decomposes $E'$ as $E''.p$, and checks whether
$p \in \winits{\exseq''}{v}$.
Information about the accesses of $p$ can now be found in the execution $E$, so that the check $p \in \winits{\exseq''}{v}$ can be performed.
The check will be exact for non-branching programs, but possibly conservative in general.
If the check succeeds, then insertion proceeds one step further in the execution $E$ (\cref{alglpwu:recursive-call}), otherwise $v$ conflicts with $p$ and so should be inserted at the wakeup tree after $E''$ (\cref{alglpwu:event-insert}).
As an additional optimization, \Cref{alglpwu:former-leaf} checks whether $E''$ was the leaf that is extended to the currently explored execution.
If so, the insertion can return without inserting anything, in analogy
with how leaves are handled in wakeup tree insertion
(\cref{alg:wut-insert-empty} of \cref{alg:wakeuptree}).


%% \begin{itemize}
%% \item If $p$ does not start after $E'.u$ (\cref{alg:no-message}) then $p \in \winits{\exseq.u}{v}$ can always be decided from the happens-before relation $\happbf{hb}{E'.u.v}$ \bjcom{check which exec}, (\cref{alg:travers-no-message}).
%% \item If $p$ starts after $E'.u$ (\cref{alg:message}) then there are two conditions under which $p \in \winits{\exseq.u}{v}$ can be decided:
%%   \begin{inparaenum}[(1)]
%%   \item if $p$ is the first (if any) message on its handler in $v$, then $p \in \winits{\exseq.u}{v}$ is trivially true, since the first event of $p$ is never in conflict with any event (\cref{alg:wut-pfirst}),
%%     \item if $p$ executes to completion in the sequence $v$ (\cref{alg:wut-pfull}), then $v$ contains sufficient information to decide whether $p \in \winits{\exseq.u}{v}$: if yes the traversal can continue at $u.p$, if no the next child of $u$ is considered.
%%   \end{inparaenum}
%%   If none of these cases apply, then more information is needed about which accesses $p$ performs when it is executed. Therefore the sequence $v$ is parked at the node $u$: the insertion of $v$ will be resumed when the node $u.p$ is extended to a maximal execution starting with $E'.u.p$, which happens at \cref{algacsl:insert-pendingwu} of \cref{alg:eventdpor-access} with $E'$ being $E'.u$.
%% \end{itemize}
%% If all children $u.p$ of $u$ have been traversed with failing tests for $p \in \winits{\exseq.u}{v}$, then $v$ is added as a new branch from $u$, ordered after the already existing children (\cref{alg:wut-insert-branch}).
%% \footnote{An optimization over the Optimal DPOR algorithm of~\cite{optimal-dpor-jacm} is that in the case where
%% $p \in \winits{\exseq.u}{v}$ but $\nextev{E.u}{p} \not\in v$ (implying that if $p \not\in \inits{\exseq.u}{v}$), the insertion leaves the
%% wakeup tree unchanged. This optimization does not miss executions,
%% since the algorithm will anyway later explore continuations of $E.u.p$ that include $E.u.v$ as a happens-before prefix.}

\subsection{Checking for Redundancy}
\label{sec:checkwi}
%% An important operation in \EventDPOR is to test whether a sequence $w$ is redundant, i.e., whether it is equivalent to an already explored execution.
%% Such a test appears at \cref{algacsl:event-test} of \cref{alg:eventdpor-access}; it is defined in \cref{def:redundancy} using the concept of weak initials (\cref{def:winits}).
Let us now consider the problem of deciding whether $p \in \winits{\exseq}{w}$ for a message $p$ and an execution $\exseq.w$.

If $p$ does not start after $E$, then $p \in \winits{\exseq}{w}$ can be checked by simple inspection, as follows.
    If $\nextev{E}{p}$ is a local event or posts a message, then $p \in \winits{\exseq}{w}$ holds trivially.
  If $\nextev{E}{p}$ accesses a shared variable, then
  \begin{inparaenum}[(i)]
    \item if $p$ appears in $w$, we have $p \in \winits{\exseq}{w}$ precisely when there is no event $\event$ in $w$ such that
      $\event \happbf{hb}{E.w}{\nextev{E}{p}}$, and
    \item if $p$ does not appear in $w$, we have $p \in \winits{\exseq}{w}$ precisely when no event in $w$ conflicts with $\nextev{E}{p}$.
  \end{inparaenum}

If $p$ starts after $E$, then checking whether $p \in \winits{\exseq}{w}$
is NP-hard in the general case, as we show in \cref{thm-lowerbound}.
However, in many cases, the check can be performed by tests that run in polynomial time.
\EventDPOR employs the following sequence of checks, starting with simple ones, and resorting to an exact decision procedure only as a last step.
We assume that the event which posts message $p$ appears in $E$, otherwise
$p \in \winits{\exseq}{w}$ is trivially false.

%% The problem of how to determine whether $p \in \winits{\exseq}{w}$ is considered in \cref{sec:checkwi}, where a decision procedure is presented.
%% This procedure must sometimes extend $w$ by completing the execution of partially executed messages, in order to determine whether the order of message execution can
%% be changed without affecting the happens-before relation.

%% \begin{definition}[Completion]
%%   \label{def:completion}
%%   Let $E$ be an execution and $w$ be a sequence with $\valid Ew$, and let $p$ be a message which is partially executed in $E.w$.
%%   The \emph{$p$-completion} of $w$ after $E$, denoted $\completion{E}{w}{p}$ is the result of extending incompletely executed messages to completion,
%%   while respecting the following constraints:
%%   \begin{inparaenum}[(1)]
%%   \item first $p$ is extended to completion
%%   \item thereafter messages $q$ containing events that happen-before events in $p$ are extended,
%%     \item thereafter still incompletely executed messages that happen-before events completed in the previous step, and so on.
%%   \end{inparaenum}
%% \end{definition}

%% In \EventDPOR, the completion is constructed without actually executing the messages. Instead, the sequence of shared-variable accesses of a message is obtained from
%% previous or the current execution. \bjcom{More about this?}

\begin{description}
  \item[Simple Check]
  If $p$ is the first message (if any) on its handler in $w$, then $p \in \winits{\exseq}{w}$ is trivially true (recall our assumption that the first event of a message does not access a shared variable).
\item[Happens-Before Check]
  If $p$ is not the first message on its handler in $w$, 
  we check whether there is a happens-before dependency from a message $p'$ which precedes $p$ on its handler, as follows.
  \begin{enumerate}
  \item
   If $p$ is not executed to completion in $w$, we extend $w$ by a sequence of events performed by $p$ which performs all the shared-variable
  accesses that $p$ did not perform in $w$. If after this extension, some event of $p$ happens-after an event in a message $q$ on another handler which is
  not executed to completion in $w$, then $w$ is further extended by events of $q$ in the same way. If an event of $q$ again
  happens-after an event in an incomplete message on some other handler, this procedure is repeated recursively until convergence, resulting in an extension $w'$ of $w$.
%%  In this phase, it is checked whether the relation $\happbf{hb}{E.w}$ prevents the shared-variable accesses of $p$ to be performed before any other
%%   message on the same handler.
  \item
  Thereafter, the happens-before relation $\happbf{hb}{E.w'}$ is extended to include ordering constraints induced by the event-driven execution model.
  \begin{enumerate}[(i)]
    \item First $\hbwpref{E}{w'}$ is constructed as
    the smallest transitive relation which includes $\happbf{\hb}{E.w'}$ and in addition enforces
    $e \hbwpref{E}{w'} e'$  whenever $e$ is in a message whose first event is in $E$ and $e'$  occurs after $e$ on the same handler as $e$.
    \item
   Thereafter, $\happbf{sc}{E.w'}$ is defined as the saturation $\weaksatrel{\hbwpref{E}{w'}}$ of $\hbwpref{E}{w'}$ (see \cref{def:weakall}).
  \end{enumerate}
  If now $e' \happbf{sc}{E.w'} e$ for some event $e$ in $p$ and event $e'$ in a message which precedes $p$ on the same handler, then $p \in \winits{\exseq}{w}$ must be false.
  \end{enumerate}
\item[Witness Construction]
%%   If there is no event $e$ in $p$ with $p' \happbf{sc}{E.w'} e$, then
  If the Happens-Before Check was not negative, the next step is to construct an actual execution in which $p$ is the first message on its handler. 
  First, $\happbf{sc}{E.w'}$ is extended, by ordering
  the events in $p$ before any event in a message that precedes $p$ in $w'$ on the same handler, and thereafter saturated by the saturation operation $\weaksatrel{\cdot}$.
  If the result contains a cycle, then $p \in \winits{\exseq}{w}$ must be false. Otherwise
  we extend the saturated extension of $\happbf{sc}{E.w'}$ to a total order on the messages of each handler, by ordering
  messages that are still unordered to execute in the same order as they appear in $w'$. If this can be done without creating a cycle then
  $p \in \winits{\exseq}{w}$ is true.
\item[Decision Procedure]
  If a cycle is created, then a decision procedure is invoked as a final step.
\end{description}



\input{np-completeness-proofs}
\section{Proof of Correctness and Optimality}
\label{sec:correctness-proof}

In this section, we prove correctness (\cref{thm:correctness}) and optimality
(\cref{thm:optimality}) of the \EventDPOR algorithm.

\subsection{Proof of \cref{thm:correctness}}

Let us first prove \cref{thm:correctness}. This theorem follows from the more general
\cref{thm:correctness-general}, which we state and prove in this section.

Let us
% present the main properties of the \EventDPOR algorithm. Throughout, we
assume a particular completed execution of \EventDPOR. This execution 
consists of a number of terminated calls to $\explore(E)$ for some values 
of the parameters $E$ and $\WuT$. Let $\exseqs$ denote the set of execution 
sequences $E$ that have been explored in some call $\explore(E)$. Define 
the ordering $\treeorder$ on $\exseqs$ by letting $E \treeorder E'$ if 
$\explore(E)$ returned before $\explore(E')$. Intuitively, if one 
were to draw an ordered tree that shows how the exploration has proceeded, then 
$\exseqs$ would be the set of nodes in the tree, and $\treeorder$ would be the 
post-order between nodes in that tree. \cref{thm:correctness} follows from the more
general \cref{thm:correctness-general}, stated here

\begin{theorem}[Correctness of \EventDPOR]
\label{thm:correctness-general}
Whenever a call to $\explore(E)$ returns during \cref{alg:eventdpor-access},
then for all maximal execution sequences $E.w$, the algorithm has explored
some execution sequence in~$\eqclass{E.w}$.
\end{theorem}

Since the initial call to the algorithm, $\explore(\emptyseq)$, starts with
the empty sequence as argument, \cref{thm:correctness-general}
implies that for all maximal execution sequences $E$ the algorithm
explores some execution sequence $E'$ which is in $\eqclass{E}$.
Note also that a sequence of form $E.w$ need not have been explored inside
the call $\explore(E)$, but can have been explored in some earlier call,
of form $\explore(E'.p)$ for some prefix $E'$ of $E$.

The proof of~\cref{thm:correctness-general} proceeds by induction on the set $\exseqs$ of execution sequences $E$ that
are explored during the considered execution, using the ordering $\treeorder$
(i.e., the order in which the corresponding calls to $\explore(E)$ return).

We first state and prove a small lemma.

\begin{lemma}
\label{lem:WuT-exseqs}
Let $\exseqs$ be the tree of explored execution sequences. and let $\treeorder$ be
the order in which the corresponding calls to $\explore(E)$ return.
Consider some point in the execution, and let  $\wut{\exseq}$ be the wakeup tree
at $\exseq$ at that point, for some $\exseq \in \exseqs$.
\begin{enumerate}
\item \label{lem:WuT-exseqs:1}
  If $w \in \wut{\exseq}$ for some $w$, then $\exseq.w \in \exseqs$.
\item \label{lem:WuT-exseqs:2}
  If $w \prec w'$ for $w,w' \in \wut{\exseq}$ then $\exseq.w \treeorder \exseq.w'$
\end{enumerate}
\end{lemma}
%
\begin{proof}
The lemma follows by noting how the exploration from any $\exseq
\in \exseqs$ is controlled by the wakeup tree $\wut{\exseq}$ at
\crefrange{algacsl:exploration-begin}{algacsl:event-call-explore}
of~\cref{alg:eventdpor-access}.
\end{proof}

%\begin{comment}
%\bjcom{All of this goes away, since it has been defined earlier}
%
%First a definition. Let $E$ be an execution sequence, $p$ be a thread, and $v$ a  a sequence with $\valid Ev$. 
%We let $p \racefree{E} v$ denote that there is a sequence $w$ with $\valid E{p.w}$ such that $v \mtprefixafter{E} p.w$.
%%% \begin{itemize}
%%% \item
%%%   $\dom{E.v} \subseteq \dom{E.p.w}$
%%% \item
%%%   The relation $\happbf{}{E.p.w} \cup \precof{E.v}$, where $\precof{E.v}$
%%%   \begin{inparaenum}[(i)]
%%%   \item
%%%     totally orders the events in $\dom{E.v}$ by $\totorder{E.v}$ and
%%%   \item
%%%     orders all events in $\dom{E.v}$ before all events not in $\dom{E.v}$,
%%%   \end{inparaenum}
%%%   is acyclic.
%%% \end{itemize}
%Intuitively, $p \racefree{E} v$ denotes that there is a sequence starting with $p$ which includes all events of $v$ and does not interfere with the ordering on $v$.
%Concretely, if $\nextev{E}{p}$ is a write or read operation on a shared variable, then
%$p \racefree{E} v$ amounts to checking whether $p$ is independent with the longest prefix of $v$ that does not contain $p$
%(which may be $v$ itself if $v$ does not contain $p$).
%If $\nextev{E}{p}$ is a post event then $p \racefree{E} v$ holds if and only if we can transform $v$ to $p.w$ by 
%\begin{inparaenum}[(i)]
%\item
%  first moving events $e$ in $v$ with $\nextev{E}{p} \happbf{\pb^*}{E.v} e$ towards the front, but respecting $\happbf{}{E.v}$,
%  so that $p$ ends up first, and 
%\item
%  thereafter adding events, which may not happen-before (acc.\ to $\happbf{}{E.p.w}$) any events already in $v$,
%\end{inparaenum}
%such that $E.p.w$ is a valid execution sequence.
%It turns out that
%\begin{itemize}
%\item in step (i) it is sufficient to move $p$ first, and to move messages resulting from $\nextev{E}{p}$ just enough to maintain FIFO execution,
%\item in step (ii) it is sufficient add events in or happening-before (acc.\ to $\happbf{}{E.p.w}$) messages (transitively) posted from $\nextev{E}{p}$.
%\end{itemize}
%%% $p \inwfirstseqs{E} w$ under the assumption that for each handler thread $t$, if $t$ is in the middle of executing a message at the end of $E.w$, then
%%% that message has no remaining events, i.e., it is truncated at the point of execution that it has reached at the end of $E.w$.
%%% If $\nextev{E}{p}$ is a write or read event, then
%%% $p \racefree{E} w$ is the same as $p \inwfirstseqs{E} w$, but
%%% if $\nextev{E}{p}$ is a post event to a handler $t$, then we must pretend that any message executed by
%%% by $t$ at the end of $w$ has no remaining events.
%As an example, in the program of \cref{fig:non-atomic-1},
%\fix{
%we have $p \racefree{\emptyseq} q.m[1]$ since the message $n$ posted by $p$ does not conflict with $m[1]$. This
%in spite of the fact that $p \notinwfirstseqs{\emptyseq} q.m[1]$, since any continuation of $m$ will conflict
%with $n$. The point of this definition is that $p \racefree{E} w$ denotes that the trace $w$ itself has 
%no conflicts that may cause $\nextev{E}{p}$ not to be a weak initial:
%such conflicts may arise only when the message executed at the end of $w$ is continued.}
%
%\bjcom{End of what goes away}
%\end{comment}

We now continue with the proof of \cref{thm:correctness-general}.

\medskip
\noindent
{\sl Base Case:}
This case corresponds to the first execution sequence $E$ for which the call
$\explore(E)$ returns. By the algorithm, $E$ is already
maximal, so the theorem trivially holds.

\medskip
\noindent
{\sl Inductive Hypothesis:}
The theorem holds for all execution
sequences $E'$ with $E' \treeorder E$.

\medskip
\noindent
{\sl Inductive Step:}
Proof by contradiction. Let us assume that there exists an execution $E$
such that when the call to $\explore(E)$ returns,
there is a maximal execution sequence $E.w$ such that
\cref{alg:eventdpor-access} has not explored
any execution sequence in $\eqclass{E.w}$.
We will show that this leads to a contradiction.
%% \bjcom{The following should be revised (but later)}
%% The proof will follow these steps:
%% \begin{enumerate}[1.]
%%   \item \label{itm:proofstep1} Choose two conflicting events in $w$. We will 
%%                                see next how to choose them. 
%%   \item \label{itm:proofstep2} Prove that there is some execution sequence 
%%                                that was explored before(by $\treeorder$ order), 
%%                                in which we see the conflicting events in 
%%                                opposite order. This execution sequence should 
%%                                respect some happens before relation from $w$, 
%%                                which we will see later in the proof.
%%   \item \label{itm:proofstep3} The algorithm will detect the race between those 
%%                                two events and reverse. As a result it will 
%%                                explore some equivalent sequence of $w$. 
%% \end{enumerate}
%% \bjcom{end of text to be revised}
So, let $E$ be the smallest such execution in the $\treeorder$ order.
Let $\done$ be the value of the mapping $\done$ when the call to $\explore(E)$ returns.
Note that for such $w$ to exist, $\exseq$ cannot be maximal, so $\done(E)$ contains at least one message.

For each message $p$ such that $p \in \done(E')$ for some $E'$ with $E' \leq E$, 
% when the call to $\explore(E)$ returns,
define $E_p'$ to be the longest such $E'$. Thus, if 
$p \in \done(E)$ then $E_p' = E$, otherwise if
$E_p'$ is defined it is a strict prefix of~$E$ with $p \in \done(E_p')$.
It follows that $E_p'.p \treeorder E$.
We further define $w_p'$ by $\exseq = E_p'.w_p'$.
For each message $p$ such that $E_p'$ is defined and $p \in \winits{E_p'}{w_p'}$, define
\begin{itemize}
%% \item the sequence $E_p'$, such that $E_p' \leq E$ and
%%   $E_p'.p \in \exseqs$, and such that $E_p'.p$ is the last execution
%%   sequence of this form in $\exseqs$ that precedes $E$ (w.r.t.\ $\treeorder$);
\item $w_p$ as the longest prefix of $w$ such that $p \in \winits{E_p'}{w_p'.w_p}$ (such a prefix must exist since one candidate is the empty sequence),
\item $e_p$ as the first event in $w$ which is not in
  $w_p$.  Such an event $e_p$ must exist, otherwise $w_p = w$, which implies
  $p \in \winits{E_p'}{w_p'.w}$,
%%   which implies $p \racefree{\exseq}{w}$,
  which together with the Inductive Hypothesis contradicts the assumption that
  the algorithm has not explored
  any execution sequence in $\eqclass{E.w}$,
\item $w_p''$ as a sequence such that $w_p'.w_p\mtprefixafter{E_p'}{p.w_p''}$.
  %% which includes the event $e_p$.
  %% Such a sequence exists by the following argument: (* THIS IS COMPLETELY WRONG *)
  %% \bjcom{The following text must be revised. It is too abstract, and should be adapted to our
  %% current execution model}
  %% \begin{quote}
  %%   By $p \in \winits{E_p'}{w_p'.w_p}$ there is a sequence $w_p''$ with $w_p'.w_p\mtprefixafter{E_p'}{p.w_p''}$. If $w_p''$ does not include $e_p$, then since
  %% $e_p$ is enabled after $E_p'.w_p'.w_p$, the only reason why $e_p$ cannot be added after $E_p'.p.w_p''$ is that some messages have been inserted before $e_p$ in
  %% its handler thread. But after executing these messages, $e_p$ can be added to $w_q''$, while preserving the property $w_p'.w_p\mtprefixafter{E_p'}{p.w_p''}$.
  %% \end{quote}
  %% Let $u_p$ be the subsequence of $p.w_p''$ consisting of the events that are not in $w_p'.w_p$. We note that $e_p$ must happen-after some event in $u_p$,
  %% otherwise we would have $w_p'.w_p.e_p \mtprefixafter{E_p'}{p.w_p''}$, implying $p \in \winits{E_p'}{w_p'.w_p.e_p}$, contradicting the maximality of $w_p$.
  %% Among the possible sequences $w_p''$ with the $w_p'.w_p\mtprefixafter{E_p'}{p.w_p''}$ that include $e_p$,
  %% choose one with the smallest number of events in $u_p$ that happen-before $e_p$.
\end{itemize}
Among the messages $p$ for which $E_p'$ is defined and $p \in \winits{E_p'}{w_p'}$, select
$q$ such that $w_q$ is the longest prefix among $w_p$.
If there are several such messages $q$ with equally long prefixes $w_q$,
%% then among these pick $q$ with a minimal number of events in $u_q$ that happen-before $e_q$ in
%% $E_q'.q.w_q''$ (note that there is at least one such event in each $u_q$).
%% If there are still several messages $q$ with a minimal number of such events,
then among these pick $q$ such that
$E_q'.q$ is minimal with respect to $\treeorder$.
Let $w_q''$ be a sequence with $w_q'.w_q\mtprefixafter{E_q'}{q.w_q''}$. 
%% \item $E_q'$, such that $E_q'  \leq E$,
%%   $E_q'.q \in \exseqs$, and $E_q'.q$ is the last execution
%%   sequence of this form that precedes $E$ (w.r.t.\ $\treeorder$). If
%%   $q \in \done(E)$ then $E_q' = E$, otherwise $E_q'$
%%   is a strict prefix of~$E$.
%% Thus, $e_q$ is the first event in $w$ which is not in $w_q$.
  %%$w'$, \bjcom{small type error here, how to avoid?}which
  %% Such an event $e_q$ must exist, otherwise  $w_q = w$, which implies $q\in \winits{E_q'}{w_q'.w}$,
  %% which implies that there exists $w''$ s.t. $E_q'.w_q'.w\simeq E_q'.q.w''$.
  %% By I.H. $\explore(E_q'.q)$ has then explored some
  %% execution sequence equivalent to $E.w$, contrary to the assumption.
%%
%% By the preceding construction, we have $\valid{E_q'}{q.w_q''}$

Let $p'$ be the message $\procof{e_q}$ of $e_q$.
We first note that $e_q$ must be a shared-variable access. To see why,
note that if $e_q$ would start the message $p'$, then no event of the message $p'$
can be in $w_q'.w_q$. Moreover, the handler of $p'$ must be free after $E_q'.w_q'.w_q$, and $E_q'.w_q'.w_q$ must contain the event which posts $p'$.
We can simply extend $w_q''$ until it starts message $p'$
and then we have a sequence $w_q'''$ with $w_q'.w_q.e_q\mtprefixafter{E_q'}{q.w_q'''}$, contradicting the
choice of $w_q$.

There are now two cases to consider.
\begin{enumerate}
\item
  $q$ does not start a message after $E_q'$. Then $E_q'$ contains the first part of message $q$ (up until but not including $\nextev{E_q'}{q}$).
  Since $w_q'.w_q\mtprefixafter{E_q'}{q.w_q''}$, it follows that $\nextev{E_q'}{q}$ does not conflict with any event in $w_q'.w_q$, and that
  we can choose $w_q''$ as $w_q'.w_q$. The only reason for $q \not\in \winits{E_q'}{w_q'.w_q.e_q}$ (which implies $w_q'.w_q.e_q \notmtprefixafter{E_q'}{q.w_q''.e_q}$)
  is that $\nextev{E_q'}{q}$ conflicts with $e_q$.
%%  It is also clear that $\nextev{E_q'}{q}$ does not conflict with any event in $w_q''$.
  This implies that the execution $E_q'.q.w_q'.w_q.e_q$ contains a race between $\nextev{E_q'}{q}$ and $e_q$.
  Let $w_q'''$ be $w_q'.w_q.e_q$ and let $E_q'.q.w_q'''.z$ be a maximal extension of $E_q'.q.w_q'''$.
  Then $\nextev{E_q'}{q} \revrace{E_q'.q.w_q'''.z} e_q$.
By the Inductive Hypothesis, $\explore(E_q'.q)$ has then explored some sequence $E_q'.q.z'$ in $\mtclass{E_q'.q.w_q'''.z}$.
When exploring it, the race $\nextev{E_q'}{q}\revmsgrace{E_q'.q.z'}{m} e_q$ between $\nextev{E_q'}{q}$ and $e_q$ will be detected (at \cref{algacsl:race-loop}). Then $\reverserace(E_q'.q.z',\nextev{E_q'}{q},e_q)$ will return maximal executions, one of which must contain $E_q'.w_q'.w_q'.e_q$ as a happens-before prefix.
\item
  $q$ starts a message after $E_q'$.
  Since $e_q$ is a shared-variable access, it can be simply added to  $p'$ in $w_q''$, obtaining $w_q'''$.
  Since $q \not\in \winits{E_q'}{w_q'.w_q.e_q}$, $w_q''$ must contain an event $e$, which is
  not in $w_q'.w_q$, which conflicts with $e_q$.
  This implies that the execution $E_q'.q.w_q'''$ contains a race between $e$ and $e_q$.
  Let $E_q'.q.w_q'''.z$ be a maximal extension of $E_q'.q.w_q'''$.
  Then $e \revrace{E_q'.q.w_q'''.z} e_q$.
By the Inductive Hypothesis, $\explore(E_q'.q)$ has then explored some sequence $E_q'.q.z'$ in $\mtclass{E_q'.q.w_q''.z}$.
When exploring it, the race $e \revmsgrace{E_q'.q.z'}{m} e_q$ between $e$ and $e_q$ will be detected (at \cref{algacsl:race-loop}). Then $\reverserace(E_q'.q.z',e,e_q)$ will return maximal executions, one of which must contain $E_q'.w_q'.w_q'.e_q$ as a happens-before prefix.
\end{enumerate}
%% Let $\done'$ be the value of the mapping $\done$ just before the call to $\explore(E_q'.q)$.
%% \begin{claim}
%% $\neg \redundant{E_q'}{\done'}{q.w_q'''}$
%% \label{claim:nonred}
%% \end{claim}
%% \begin{proof}
%% Proof by contradiction. If $\redundant{E_q'}{\done'}{q.w_q'''}$
%% then there is a message $r$, different from $q$, and a (not necessarily strict) prefix $E_r'$ of
%% $E_q'$ such that $E_r'.r \in \exseqs$ with $E_r'.r \treeorder E_q'.q$, and
%% $r \in \winits{E_q'}{w_r'''.q.w_q''}$, where $w_r'''$ is defined by $E_r'.w_r''' = E_q'$.
%% By definition of $r \in \winits{E_q'}{w_r'''.q.w_q''}$, there is
%% a sequence $w_r''$ such that $w_r'''.q.w_q'' \mtprefixafter{E_r'} r.w_r''$.
%% From $w_q'.w_q\mtprefixafter{E_q'}{q.w_q''}$ we infer $w_r'''.w_q'.w_q \mtprefixafter{E_r'}r.w_r''$.
%% This implies that $w_r$ is at least as long as $w_q$ and that the number of preceding events for $r$ is at most that for $q$.
%% But this contradicts the fact that $q$ was selected in the preceding construction.
%% \end{proof}
%% By construction, $w_q'.w_q \mtprefixafter{E_q'} q.w_q''$, but since 
%% $q \notracefree{E_q'}{w_q'.w_q.e_q}$ we have $w_q'.w_q.e_q \notmtprefixafter{E_q'} q.w_q''$.
%% This can only happen if there is some event in $q.w_q''$ which is not in $w_q'.w_q$ (i.e., in $u_q$) which happens-before $e_q$ in $E_q'.q.w_q''$. Otherwise 
%% $w_q'.w_q.e_q \mtprefixafter{E_q'} q.w_q''$
%% would follow from $w_q'.w_q \mtprefixafter{E_q'} q.w_q''$
%% by the definition of
%% $\mtprefixafter{E_q'}$, thereby violating the construction of $w_q$.
%% Let $e_q'$ be the last event in $u_q$ which happens-before $e_q$, i.e.,
%% $e_q' \happbf{\hb}{E_q'.q.w_q''} e_q$.
%% We claim that this happens-before relation between $e_q'$ and $e_q$ must be a race. For this, we check the three conditions in~\cref{def:races}.
%% \begin{enumerate}[(i)]
%% \item We check that $e_q' \happbf{\cnf}{E_q'.q.w_q''} e_q$ by checking that $e_q' \happbf{\hb}{E_q'.q.w_q''} e_q$ cannot be labeled by anything else than $\cnf$:
%%   \begin{itemize}
%%     \item $e_q' \happbf{\po}{E_q'.q.w_q''} e_q$ is impossible, since $\valid{E_q'.w_q'.w_q}{e_q}$ and $e_q'$ is not in $E_q'.w_q'.w_q$,
%% \item $e_q' \happbf{\pb}{E_q'.q.w_q''} e_q$ is impossible for the same reason,
%%   \end{itemize}
%% \item To see that there is no $e''$ with $e_q' \happbf{\weakall}{E_q'.q.w_q''} e'' \happbf{\hb}{E_q'.q.w_q''} e_q$, we note that such an  $e''$ cannot be in
%%   $w_q'.w_q$ since it can be scheduled after $w_q'.w_q$.
%%   Thus $e''$ is in $u_q$, but this contradicts that $e_q'$ is the last event in $u_q$ which happens-before $e_q$.
%% \item If $e_q'$ and $e_q$ are in different messages $m_q'$ and $m_q$, then we cannot have $\pbof{e_q'} \happbf{\weakall}{E_q'.q.w_q''} \pbof{e_q}$ since $e_q'$ does not occur in $E_q'.w_q'.w_q$ whereas $\valid{E_q'.w_q'.w_q}{e_q}$.
%% \end{enumerate}
%% Let $E_q'.q.w_q''.z$ be a maximal extension of $E_q'.q.w_q''$.
%% %% where the happens-before edge between $e_q'$ and
%% %% $e_q$ is still a race. This can be achieved in the same manner as in Case i).
%% %% \bjcom{Proof of this fact to be supplied. Or: we do not need it by the definition of race. To be checked.}
%% %%
%% %% \bjcom{This paragraph seems to say the same thing as the four bullets above.}
%% %% By the definition of $w_q'.w_q \mtprefixafter{E_q'} q.w'$ it follows that
%% %% $e_q'$ cannot happen-before any event in $w_q'.w_q$. Thus, there is no other event $e''$ in $q.w'$ such that
%% %% $e_q' \happbf{\hb}{} e'' \happbf{\hb}{} e_q$. Furthermore, we annot have $e_q' \happbf{\po}{} e_q$ or $e_q' \happbf{\eop}{} e_q$, since then
%% %% $e_q$ could not appear after $w_q$ in $w$ before $e_q'$ has occurred.
%% %% 
%% The race between $e_q'$ and $e_q$ is either a message-event race or an
%% event-event race. Let us consider the two cases.
%% \begin{itemize}
%%   \item The race is a message-event race $m'\revmsgrace{E_q'.q.w_q''.z}{m} e_q$.
%% By the Inductive Hypothesis, $\explore(E_q'.q)$ has then explored some sequence $E_q'.q.z'$ in $\mtclass{E_q'.q.w_q''.z}$.
%% When exploring it, the race $m'\revmsgrace{E_q'.q.z'}{m} e_q$ between $m'$ and $e_q$ will be detected.
%% When handling the race, the algorithm will process each maximal feasible happens-before prefix $\pexseq$ of $E_q'.q.z'$, which does not include $m'$ nor any
%% messages of a sibling-race of $m'\revmsgrace{E_q'.q.z'}{m} e_q$ such that
%% $\pexseq.e_q$ is a feasible subsequence of $E_q'.q.z'$.
%% Let $\pexseq$ be such a prefix which includes $E_q'.w_q'.w_q'$
%% (such a prefix exists by Lemma XXX), i.e., $E_q'.w_q'.w_q \mtprefix \pexseq$.
%% \item
%%   Analogously for event-event-race.
%% \end{itemize}
%% We note that any event which happens-before $e_q$ in $\pexseq.e_q$ also happens-before $e_q$ in $E_q'.q.z'$, since the removal of $e_q'$ (and its message) cannot add any events that happen-before $e_q$.

%% \bjcom{Unclear what to do with this}
%% As the next step, we prove that reversing the race will not incur any accidental rescheduling in $E_q'.w_q'.w_q$.
%% \begin{claim}
%%   Let $post$ and $post'$ be two post events in $E_q'.w_q'.w_q$, where $post$ occurs before $post'$. Then $post'$ will not be required to happen-before $post$ in the
%%   rescheduling of the wakeup sequence.
%% \label{claim:no-post-reversal}
%% \end{claim}
%% Note that $post$ or $post'$ can be $\nextev{E_q'}{q}$ if $q$ is already in $E_q'.w_q'.w_q$.
%% \begin{proof}
%%   We make a proof by contradiction. Assume that $post$ and $post'$ are post events in $E_q'.w_q'.w_q$, where $post$ occurs before $post'$.
%%   Obviously, $post'$ is not required to happen-before $post$ in $E_q'.q.w_q''.z$. In order for $post'$ to be required to happen-before $post$
%%   in the wakeup sequence,
%%   there must be events $e_p$ and $e_p'$ in $E_q'.q.w_q''.z$ such that
%%   $post \happbf{\pb}{E_q'.q.w_q''.z} e_p$ and  $post' \happbf{\pb}{E_q'.q.w_q''.z} e_p'$ and such that
%%   $e_p'$ is required to happen-before $e_p$ in the wakeup sequence. This can happen in two ways:
%%   \begin{itemize}
%%   \item $e_p' \happbf{}{E_q'.q.w_q''} e_q$ (or $e_p'$ is $e_q$), and $e_q' \happbf{}{E_q'.q.w_q''} e_p$ (or $e_q'$ is $e_p$).
%%     Since $post$ occurs before $post'$ in $E_q'.w_q'.w_q$, also $e_p$ occurs before $e_p'$ in $E_q'.q.w_q''$. This implies that
%%     $e_q'$ strictly occurs before $e_p'$. But, since the reversed race is  between $e_q'$ and $e_q$, this contradicts the construction which says that
%%     the reversed race is between $e_q$ and the closest preceding event that happens before $e_q$, which must be $e_p'$ or occur after $e_p'$.
%%   \item \bjcom{Not clear what to do with this. Remove?} 
%%     $e_p' \happbf{}{E_q'.q.w_q''} e_q$ (or $e_q'$ is $e_q$), and $e_p$ is not in $\notsucc{m'}{E_q'.q.w_q''.z}$
%%   \item $e_p$ and $e_p'$ are again post events, but then the first case is applied recursively.
%%   \end{itemize}
%% \end{proof}
%% \begin{claim}
%%   The event $\nextev{E_q'}{q}$ happens after some later event in the wakeup sequence.
%% \label{claim:q-happens-after}
%% \end{claim}
%% \begin{proof}
%%   Proof by contradiction. Suppose $q$ does not happen after some later event in the wakeup sequence. Then the wakeup sequence has $q$ as a weak initial.
%%   But the wakeup sequence has fewer events that happen-before $e_q$ than the sequence $E_q'.q.w_q''.z$. Since by construction $E_q'.q.w_q''.z$
%%   has en minimum number of events preceding $e_q$, it cannot have $q$ after $E_q'$.
%% \end{proof}
%% \bjcom{End of: Unclear what to do with this}

Let $E_q'.w_q'.w_q'.e_q$ be reordered as $E_q'.v$.
It follows that
$q \not\in\winits{E_q'}{v}$, from the assumptions made when selecting $q$.
Moreover, there cannot be any $E'',w,p$ such that $E''.w = E_q'$ and $p \in \dom{\done(E'')}$ and $p \in \winits{E''}{w.v}$,
also by the assumptions made when selecting $q$.
Thus, the wakeup sequence $v$ will be inserted into the wakeup tree $\wut{E_q'}$ (\cref{algacsl:event-race-end}) by the call $\insertwus{v}{\exseq_q'}{\emptyseq}$.
We claim that this insertion will add a sequence of form $E.p$ with
$p \in \winits{\exseq}{w_q.\procof{e_q}}$. To see why, we consider the
definition of $\insertwus{v}{E_q'}{u}$ in \cref{alg:wakeuptree}.
We first claim that during the insertion, the sequence $u$ will always
satisfy $E_q.u \leq E$ and $v$ will satisfy $u'.w_q.\procof{e_q} \infirstseqs{E_q.u} v$,
where $u.u' = w_q'$.
This is trivially true initially. To see that
it is preserved by each iteration of the loop starting at \cref{algl:wut-foreach-child},
we consider the possible children of form $u.p$. Let $r$ be the message such that
$E_q'.u.r \leq E$ (if still $E_q'.u < E$).
We know that $E_q'.u.r$ is in $\exseqs$ when $\explore(\exseq)$ is returns.
Furthermore, for each branch $u.p$ with $E_q.u'.p \treeorder E_q.u'.r$ we have that
$p \not \in \winits{\exseq.u}{u'.w_q.\procof{e_q}}$ by the Inductive
Hypothesis and the assumption that $\exseq.w$ has not been explored. On the other hand
$r \in \winits{\exseq.u}{u'.w_q.\procof{e_q}}$, implying that either
$u.r$ is already in $\wut{\exseq_q'}$ during the insertion, in which case the
loop will move to the next iteration with invariants preserved, or
$u.r$ is not already in $\wut{\exseq_q'}$ in which case it must be added during the
current insertion and produce a branch $u.v$ such that
$u'.w_q.\procof{e_q} \infirstseqs{E_q.u} v$.
Thus, when the insertion of $v$ has completed, possibly after having been parked, the exploration tree will contain
an execution of form $E.v'$ with $w_q.\procof{e_q} \mtprefixafter{E} v'$, thereby
contradicting the assumption that $w_q$ is the longest extension of $E$ that has been explored.
This concludes the proof of the inductive step,
and~\cref{thm:correctness-general} is proven.
%% \bjcom{Still to be considered is parking}
\qed

\subsection{Proof of \cref{thm:optimality}}

Let us next prove \cref{thm:optimality}. This theorem depends on \EventDPOR being able
to the following property P:
\begin{itemize}
   \item[P:]
whenever the exploration tree $\exseqs$ contains a node of form $\exseq.p$, then the algorithm will not add an execution of form $\exseq.w$ which is contained in
some execution of form $\exseq.p.w'$ for some $w'$, i.e., for which $p \in \winits{\exseq}{w}$.
\end{itemize}
If P is enforced, then~\cref{alg:eventdpor-access} cannot explore two equivalent maximal executions. To see this, let $\exseq$ be the longest common prefix of the two executions. Let the
execution of the two, which is explored first, be of form $\exseq.p.w'$. The other execution
will then be the continuation of a wakeup sequence, which is inserted as a new sequence $w$
from the node $\exseq$ in the exploration tree $\exseqs$, and thereafter extended to
$\exseq.w.v$. But if now $\exseq.p.w' \mtequiv \exseq.w.v$, then
$\exseq.w \mtprefix \exseq.p.w'$, which implies $p \in \winits{\exseq}{w}$, which contradicts
P.

It thus remains to check that property P is enforced. By inspection of
\cref{alg:eventdpor-access}, we see that whenever a new sequence is inserted into $\exseqs$,
which happens before inserting a new wakeup sequence (\cref{algacsl:event-test}),
inside procedure \insertwusname (\cref{alg:wakeuptree}) for wakeup tree insertion, and in
the procedure \insertpendingwuname (\cref{alg:pendingwus}) for inserting parked wakeup sequences.
Furthermore, for non-branching programs the test for $p \in \winits{\exseq}{w}$,
described in \cref{sec:checkwi}, is exact.
This concludes the proof of the theorem.
\qed


%% -*- mode: LaTeX; fill-column: 78; -*-

\section{Complete Set of Benchmark Tables} \label{app:eval-complete}
%===================================================================
In this appendix, we include the complete set of benchmark results comparing
the performance of the \EventDPOR with that of the \OptimalDPOR algorithm,
with the LAPOR technique implemented in \GenMC and also with the baseline
algorithm of \GenMC which tracks the modification order (\genmcmo{\small}) of
shared variables. A subset of these results appears in the main body of the
paper.

\paragraph{Baseline Comparison}
%------------------------------
First, we measure the performance of algorithm implementations on three
programs where all algorithms explore the same number of executions.  The
first two of them are simple programs where a number of threads post racing
messages to a \emph{single} event handler. Both programs are parametric on the
number of threads (and messages posted); the value of this parameter is shown
inside parentheses. The messages of the first program (\bench{writers})
consist of a store to the same atomic global variable followed by an assertion
that checks for the value written. The second program (\bench{posters}) is
similar but between the write and the assertion check the messages also post,
to the same handler, another message with an atomic store to the same global
variable; this increases the number of executions to examine.
%
Finally, the third program (\bench{2PC}) is a two-phase commit protocol used
by a coordinator and $n$ participant threads (i.e., $n+1$ handler threads in
total) to decide whether to commit or abort a transaction, by broadcasting and
receiving messages.

\begin{table}[t]
  \caption{Performance on programs where different DPOR algorithms implemented
    in \Nidhugg and \GenMC explore the same number of complete and blocked
    executions. Times (in seconds) show the relative speed of their
    implementations.}
  \label{tab:eval:baseline}
  %% \smallertabcaptionspace
  \centering\SZ
  %\setlength{\tabcolsep}{1pt}
  \pgfplotstablevertcat{\output}{results/laban/writers.txt}
  \pgfplotstablevertcat{\output}{results/laban/posters.txt}
  \pgfplotstablevertcat{\output}{results/laban/2PC.txt}
  \pgfplotstabletypeset[
    every row no 3/.style={before row=\midrule},
    every row no 6/.style={before row=\midrule},
  ]{\output}
\end{table}

Results from running these benchmarks for increasing number of threads are
shown in~\cref{tab:eval:baseline}. As can be seen, all algorithms explore the
same number of executions here. This allows us to establish that:
\begin{enumerate}[(i)]
\item \GenMC \genmcmo{\small} is fastest overall; in particular, it is $3$--$7$
  times faster than \Nidhugg \opt{\small} and about $8$--$9$ times faster
  than \Nidhugg \evt{\small}.
\item The overhead that LAPOR incurs over its baseline implementation in
  \GenMC is significant.
% (four to more than ten times slower).
  Still, for the first two programs, which involve just one event handler and
  no blocked or aborted executions, \GenMC \lapormo{\small} beats \Nidhugg
  \evt{\small}.  However, \Nidhugg \evt{\small} is faster than \GenMC
  \lapormo{\small} on the third program (\bench{2PC}).
\item The overhead that \EventDPOR incurs over \OptimalDPOR for the extra
  machinery that its implementation requires is small but quite noticeable.
\end{enumerate}
%
The results from \bench{2PC} corroborate these conclusions. The blocked
executions in this benchmark are due to \emph{assume-blocking} and affect all
algorithms equally in terms of additional executions examined.  However,
notice that \GenMC \lapormo{\small} is affected more in terms of time overhead
compared to its baseline.

\paragraph{Performance on More Involved Event-Driven Programs}
The next two benchmarks were taken from a recent paper by Kragl et
al.~\citet{Kragl20}.
%
In \bench{buyers}, $n$ ``buyer'' threads coordinate the purchase of an item
from a ``seller'' as follows: one buyer requests a quote for the item from the
seller, then the buyers coordinate their individual contribution, and finally
if the contributions are enough to buy the item, the order is placed.
%
In \bench{ping-pong}, the ``pong'' handler thread receives messages with
increasing numbers from the ``ping'' thread, which are then acknowledged back
to the ``ping'' event handler.

\begin{table}[t]
  \caption{Performance on programs where different DPOR algorithms implemented
    in \Nidhugg and \GenMC examine the same number of traces, but LAPOR also
    explores a significant number of executions that need to be aborted.
    This negatively affects the runtime that SMC using LAPOR takes.}
  \label{tab:eval:whenblocked}
  \setlength{\tabcolsep}{2.5pt}
  \centering\SZ
  %% \pgfplotstablevertcat{\output}{results/laban/n2W.txt}
  \pgfplotstablevertcat{\output}{results/laban/buyers.txt}
  \pgfplotstablevertcat{\output}{results/laban/ping_pong.txt}
  \pgfplotstabletypeset[
    every row no 3/.style={before row=\midrule},
  ]{\output}
\end{table}

Results from running these benchmarks are shown
in~\cref{tab:eval:whenblocked}.
%
In these two programs, all algorithms explore the same number of traces, but
LAPOR also explores a significant number of executions that cannot be
serialized and need to be aborted. This negatively affects the time that SMC
using LAPOR requires; \GenMC \lapormo{\small} becomes the slowest
configuration here.  In contrast, \Nidhugg \evt{\small} shows similar
scalability as baseline \GenMC and \Nidhugg \opt{\small}.

\paragraph{Performance on Event-Driven Programs Showing Complexity
  Differences Between DPOR Algorithms}
Finally, we evaluate all algorithms in programs where algorithms tailored to
event-driven programming, either natively (\EventDPOR) or which are
lock-aware (when handlers are implemented as locks), have an advantage.
%
We use six benchmarks.
%
The first (\bench{consensus}), again from the paper by Kragl et
al.~\citet{Kragl20}, is a simple \emph{broadcast consensus} protocol for $n$
nodes to agree on a common value. For each node~$i$, two threads are created:
one thread executes a \texttt{broadcast} method that sends the value of
node~$i$ to every other node, and the other thread is an event handler that
executes a \texttt{collect} method which receives~$n$ values and stores the
maximum as its decision. Since every node receives the values of all other
nodes, after the protocol finishes, all nodes have decided on the same value.
%
The second benchmark (\bench{db-cache}) is a key-value store system inspired
from Memcached, a well known distributed cache application. There are $n$
clients requesting a fixed sequence of storage accesses to a server via UDP
sockets (modeled as threads with mailboxes). On the server side there is
one worker thread per client to fulfill these requests. So multiple worker
threads on the server threads may race.
%
The third benchmark (\bench{prolific}) is synthetic: $n$ threads send
$n$ messages with an increasing number of stores to and loads from an atomic
global variable to one event handler.
%
The fourth benchmark (\bench{sparse-mat}) computes sparseness (number of
non-zero elements) of a sparse matrix of dimension $m \times n$. The work is
divided among $n$ tasks/messages and sent to different handlers, which then
compute and join these results.
%
The fifth benchmark (\bench{mat-mult}) implements concurrent matrix
multiplication taking two matrices of dimensions $m \times k$ and $k \times n$
as inputs. The work is divided among $n$ tasks/messages and sent to different
handlers, which then compute and join these results.
%
The last benchmark (\bench{plb}) is taken from a paper by Jhala and
Majumdar~\citet{popl07:JhalaM}. The main thread receives a fixed sequence of
task requests. Upon receiving a task, the main thread allocates a space in
memory and posts a message with the pointer to the allocated memory that will
be served by a thread in the future.

%% The sixth benchmark \sdcmt{To be added}, originally in
%% \textsc{LibEel}~\cite{hotos05:CunninghamK} (a library supporting
%% asynchrounous function calls) was transformed by hand to events and event
%% handlers.  The benchmark contains $N$ {\tt listen} messages being posted to
%% a master handler. Each of them spawns one client handler and posts a {\tt
%% new\_client} message to it. The {\tt new\_client} message in each client
%% handler is racing to acquire a shared device {\tt dev} (a global variable)
%% and writes to it by posting a {\tt write} message when successful. Upon
%% failing to acquire the device, it tries again by posting another {\tt
%% new\_client} message to the same client handler. In total $n$ {\tt
%% new\_client} messages are created, which are all conflicting with each
%% other.

\begin{table}[t]
  \caption{Performance on programs that show complexity differences in the
    number of traces that different DPOR algorithms implemented in \Nidhugg
    and \GenMC explore.}
  \label{tab:eval:differences}
  \setlength{\tabcolsep}{1.5pt}
  \centering\SZ
  \pgfplotstablevertcat{\output}{results/laban/consensus.txt}
  \pgfplotstablevertcat{\output}{results/laban/db_cache.txt}
  \pgfplotstablevertcat{\output}{results/laban/prolific.txt}
  \pgfplotstablevertcat{\output}{results/laban/sparse-mat.txt}
  \pgfplotstablevertcat{\output}{results/laban/mat_mult.txt}
  \pgfplotstablevertcat{\output}{results/laban/plb.txt}
  \pgfplotstabletypeset[
    every row no 3/.style={before row=\midrule},
    every row no 6/.style={before row=\midrule},
    every row no 9/.style={before row=\midrule},
    every row no 12/.style={before row=\midrule},
    every row no 15/.style={before row=\midrule},
  ]{\output}
\end{table}

Results from running these six benchmarks are shown in~\cref{tab:eval:differences}.

In \bench{consensus}, all algorithms start with the same number of traces, but
\LAPOR and \EventDPOR need to explore fewer and fewer traces than the other
two algorithms, as the number of nodes (and threads) increases. Here too,
\LAPOR explores a significant number of executions that need to be aborted,
which hurts its time performance. On the other hand, \EventDPOR's handling
of events is optimal in this program, even though it is not non-branching.

The \bench{db-cache} program shows a case where \GenMC, both when running with
\genmcmo{\small} but also with \lapormo{\small}, is non-optimal.  In contrast,
\EventDPOR, even with employing the inexpensive redundancy checks, manages to
explore an optimal number of traces.

The \bench{prolific} program shows a case where algorithms not tailored to
events (or locks) explore $(n-1)!$ traces, while \LAPOR and \EventDPOR explore
only $2^n-2$ consistent executions, when running the benchmark with $n$ nodes.
%
We briefly explain why the number of feasible executions are $2^n-2$. Because
of the access patterns of global variables in this program, each message is
conflicting with the previous and the next messages. In an execution, these
conflicts can be represented by $n$ directed edges. So there are $2^n$
possible reorderings when both directions of each edge are considered. But two
of these reorderings are not possible because they create a cycle, hence the
$2^n-2$.
%
On this program, it can also be noted that \EventDPOR scales \emph{much}
better than \LAPOR here in terms of time, due to the extra work that \LAPOR
needs to perform in order to check consistency of executions (and abort some
of them).

The \bench{sparse-mat} program shows another case where algorithms that are
not tailored to events explore a large number of executions unnecessarily
(\timeout denotes timeout). This program also shows that \EventDPOR beats
\LAPOR time-wise even when \LAPOR does not explore executions that need to be
aborted.

Finally, \bench{plb} shows a case on which \EventDPOR and \LAPOR really
shine. These algorithms need to explore only one trace, independently of the
size of the matrices and messages exchanged, while DPOR algorithms not
tailored to event-driven programs explore a number of executions which
increases exponentially and fast.


\end{document}
