% Figure environment removed

\noindent
If one looks at the examples provided in Section \ref{S3}, s/he will find out that each single linguistic differentia is quite fine-grained, which is motivated by the need of making it precise and not ambiguous. S/he will also notice that, on the other hand, the linguistic genus is quite coarse-grained and ambiguous, being in fact an ImageNet label. This apparent contradiction is solved by observing that the meaning of any \textit{genus can be recursively generated in terms of a second more general genus plus multiple differentiae down to the genus itself.}  
Thus, for instance, the meaning of the genus \textit{String Instrument} of the label \textit{Koto} can be reconstructed as being \textit{a Device, with sound mechanism, with taut strings}. 
%
This leads to the consequence that Choices C3, C4, as carried out by the classificationist, generate a \textit{genus-differentia classification hierarchy} which suitably organizes the meaning of labels. 

The mainstream approach to the generation of this type of hierarchy is the  \textit{faceted classification} methodology \cite{SRR-67}, a tried-and-tested paradigm of developing classification schemas devised specifically to classify any type of information resource. A key innovation of this paper is the suggestion of using, as genus-differentia classification hierarchy, the \textit{WordNet} \textit{lexico-semantic hierarchy} \cite{PWN} and its evolutions, as developed in the work in Computational Linguistics. Thus, for instance, in WordNet we have the following definition of \textit{piano}:

\begin{algorithm}[tb]
    \caption{$L  = VisClassify(I)$}
    \label{alg:algorithm}
     $H$ = Tree of ($N$);\\
    $N= <id, D>$;\\    % PLEASE CHECK IF YOU WANT THIS TWO LINES HERE. AND NOW WE DONOT HAVE THE Node IN THE ALGORITHM, CHEKC IF WE NEED THIS LINE OF MODIFY THE VARIABLE.
    \textbf{Input}: Image $I$;\\
    % \textbf{Parameter}:  Linguistic hierarchy $H$ with visual differentia $D$ for every category node;\\
    \textbf{Output}: Image label $L$;
    \begin{algorithmic}[1] %[1] enables line numbers
        \STATE $R = getRoot(H)$;
        \IF {$askDif(I, R.D) = False$}
            \RETURN $L = ``Discharged"$;    %Label(I) MEANS A FUNCTION, SO I STILL USE L_v.
        % \ELSE
        \ENDIF
        \STATE $L = R.D$;
            \WHILE{$hasChild(R) = True$}
                % \IF {$askDifferentia(I, R.D) = True$}
                    \WHILE{$C \in getChild(R)$}
                        \IF {$askDif(I, C.D) = True$}
                        \STATE $R = C$;   $break$;                \ENDIF
                    \ENDWHILE
                \STATE $L = R.D$;   % DO NOT NEED BREAK HERE, SINCE WHEN THE NODE HAS NO CHILD, THE LOOP WILL BREAK.
            \ENDWHILE
            \RETURN $L$;
        % \ENDIF
    \end{algorithmic}
\end{algorithm}

\begin{table}
\setlength{\abovecaptionskip}{2pt}%    
\setlength{\belowcaptionskip}{0pt}%
	\resizebox{1\linewidth}{!}{%
\begin{tabular}[t]{ll}
\multicolumn{2}{c}{\textit{Word}: piano}                                                                                  \\ 
Sense ID:~03934354; piano\#1                                     & Sense ID:~04998511; piano\#2         \\ 
Synset = piano, forte-piano, ...                            & Synset = piano, pianissimo, ...  \\
Gloss = ``a keyboard instrument   & Gloss = ``(music) low loudness"   \\ 
that is played by depressing keys" &     \\ 
\end{tabular}}
\centering
%\caption{}
\label{pwntab}
\end{table}
\vspace{-0.3cm}
\noindent
From this definition, we know that the 
 label \textit{piano} has two unique meanings, ``piano\#1"  and ``piano\#2", called \emph{senses}, each associated with a unique sense \emph{identifier}, e.g., \texttt{03934354}. Different synonymous words for the same sense are clustered into \emph{synsets}. Further, each synset is associated with a \emph{gloss} which is a natural language definition of the concept expressed by the synset in terms of genus-differentia. For instance, the synset \{\emph{piano, forte-piano, ...}\}, associated to the sense ``piano\#1", is defined by a gloss which encodes its genus \emph{``keyboard instrument"} and its differentia \emph{``depressing keys"}.
Fig.\ref{I0} reports an abstraction of the WordNet \textit{Musical Instrument} hierarchy. In turn, in ImageNet, this hierarchy has been populated with 3660 images. Notice how in Fig.\ref{I0} it is possible to reconstruct the definition of the label \textit{Koto} provided above.



The use of lexico-semantic hierarchies as genus-differentia classifications has three main consequences. The first is that 
lexico-semantic hierarchies can be reused, at no cost, for the generation of high-quality datasets (thus generalizing the idea behind the development of ImageNet).  The second is that the follow-up application of Choices C1, C2 allows to generate datasets where the semantics of language and vision are natively aligned thus enabling a general solution to the SGP many-to-many mapping problem. The third is that, given a set of images that populate the root node, \textit{the only information needed to label an image is its visual differentia}. As the following will make clear, this last fact is the one which causes a somewhat radical departure from the current mainstream approach to image annotation.

Choice C2, as carried out by the classifier, takes in input the hierarchy generated by Choices C3, C4 and associates each single image to a node in the hierarchy. The resulting algorithm is reported in Algorithm 1. This algorithm takes in input a hierarchy $H$, e.g., the WordNet/ ImageNet lexico-semantic hierarchy in Fig.\ref{I0},  
 and an Image $I$ and produces in output the image label $L$ (Lines 3, 12). After a first initialization where it checks whether the input Image can be classified, i.e., it is in the scope of the root label $R$ (as defined by differentia from a more abstract genus) (Lines 1-4), Algorithm 1 proceeds down the tree. This is achieved via a first \textit{while} (Line 6) which recursively descends the hierarchy and a second \textit{while} (Line 7) which, at each level in the hierarchy, checks whether there is a sibling matching the image visual differentia (Line 8). Algorithm 1 applies the \textit{GetSpecific} principle \cite{get-specific},
 i.e., it stops whenever none of the siblings satisfies the test in Line 8. 
 

 Figure \ref{I0} shows the results of applying Algorithm 1 to the images in Fig \ref{I1}. Here, the images which cannot be classified in any of the nodes below the root node are collected in the yellow box and labeled \textit{Discharged}.
 %
 Let us see how the various types of annotation mistakes from Section \ref{S2} are handled.
 %
 We assume that MOI images have been previously split in multiple sub-images, one per object, with numbering increasing from left to right.  In fact, as from Table \ref{T1}, MOI images are dealt with by Choice C1.
 %
We also assume that SOIL labels have been handled when building the classification hierarchy. Thus, for instance in Fig. \ref{I0},
 the label \textit{Dulcimer}, coherently with the ImageNet differentia, has a single meaning  describing \textit{American Dulcimers}. Taking into account \textit{Chinese dulcimers} would require extending ImageNet by adding a new sense to the label \textit{Dulcimer}. Synonyms are natively handled via synsets.
 %
 MI images are moved to the label which suitably described their visual contents. Thus, for instance, Image \#364 labelled \emph{Dulcimer} in Fig.\ref{I1} (III), is discharged.
%
 SOII images are always classified higher in the hierarchy, if compared to the ImageNet position, this is because the partial view of the object does not allow for its full discrimination. Thus, for instance,
 Image, \#251 labelled \emph{Guitar} in Fig.\ref{I1} (II), is labelled as a \emph{musical instrument}. %Notice how this image cannot be annotated with a more specialized label due to absence of any further visual property. 
 %
 With SOIA images the situation is the opposite of SOII images, that is, they are classified deeper in the hierarchy. Thus, for instance, Image \#257 labelled as \emph{Guitar} in Fig.\ref{I1} (II) can be correctly labelled as an \emph{Acoustic guitar} based on  the visual property \emph{``no input jack"}.
 
Based on the above, it is possible to classify the results of the annotation of an image as follows.

\begin{itemize}
    \item \textit{Correct}. The image annotation is correct;
     \item \textit{Incorrect}. The image annotation is wrong. This case can be further refined as follows:
     \begin{itemize}
         \item \textit{Generic}. The image annotation is more general than the correct label, i.e., it does not exploit all its visual properties;
          \item \textit{Restricted}. The image annotation is more specific than the correct label, i.e., it has been generated by refining the correct label using visual properties which are not visible;
           \item \textit{Misplaced}. The image annotation is unrelated to the correct label, i.e. it has been generated by using visual properties which are alternative to those needed to select the correct label.
     \end{itemize}  
\end{itemize}
\noindent
MI images can take any type of incorrect label. SOIA images always have generic labels. SOII images always have restricted labels.

The last observation is that lexico-semantic-resources are a very good starting point but more work needs to be done to make them fully usable for image annotation.  
This is in fact ongoing work. Fig.\ref{I0} exemplifies two cases. The first is the root node whose differentia is based on sound and, as such, not recognizable visually. This applies to all differentia describing senses. The second is the differentia of the label \textit{Acoustic Guitar}. This differentia, articulated as a missing visual feature, will always empty the parent node, in this case, the label \textit{guitar}.


\begin{table*}[htp]
\centering
\setlength{\abovecaptionskip}{2pt}%    
\setlength{\belowcaptionskip}{0pt}%
	\resizebox{1\linewidth}{!}{%
\begin{tabular}{|cccccccccccc|cccccccccc|}
\hline
\multicolumn{1}{|c|}{\multirow{2}{*}{\textbf{Id.}}}  & \multicolumn{1}{c|}{\multirow{2}{*}{\textbf{GT1}}} & \multicolumn{1}{c|}{\multirow{2}{*}{\textbf{GT2}}}  &  \multicolumn{9}{c|}{\textbf{GT3 (Annotation via Differentia labels)}}                                                                                                                                      & \multicolumn{9}{c|}{\textbf{GT4 (Annotation via Category labels)}}       \\ \cline{4-21}
\multicolumn{1}{|c|}{}            & \multicolumn{1}{c|}{}                    & \multicolumn{1}{c|}{}      &    \multicolumn{1}{c|}{\textbf{Differentia}}                          & \textbf{NA$_{1.1}$}  & \textbf{NA$_{1.2}$}  & \textbf{NA$_{1.3}$}  & \textbf{NA$_{1.4}$}  & \textbf{NA$_{1.5}$}  & \textbf{NA$_{1.6}$}  & \textbf{NA$_{1.7}$}  & \multicolumn{1}{c|}{\textbf{NA$_{1.8}$} } &      \multicolumn{1}{c|}{\textbf{Categories}}          & \textbf{NA$_{2.1}$} & \textbf{NA$_{2.2}$} & \textbf{NA$_{2.3}$} & \textbf{NA$_{2.4}$} & \textbf{NA$_{2.5}$} & \textbf{NA$_{2.6}$} & \textbf{NA$_{2.7}$} & \multicolumn{1}{c|}{\textbf{NA$_{2.8}$}}       \\ \hline
\multicolumn{1}{|l|}{1}       & \multicolumn{1}{c|}{50}    & \multicolumn{1}{c|}{41}   &    \multicolumn{1}{|l|}{with Sound Mechanism}    & 33          & 12          & 27          & 25          & 28          & 29          & 12          & \multicolumn{1}{c|}{18}                    & \multicolumn{1}{l|}{Musical Instrument}  & 16          & 42          & 100         & 17          & 27          & 20          & 19          & \multicolumn{1}{c|}{26}     \\
\multicolumn{1}{|l|}{1\_1}      & \multicolumn{1}{c|}{50}      & \multicolumn{1}{c|}{123}     &  \multicolumn{1}{|l|}{with Taut Strings}                   & 46          & 97          & 71          & 133         & 112         & 83          & 62          & \multicolumn{1}{c|}{79}         & \multicolumn{1}{l|}{Stringed Instrument} & 162         & 78          & 0           & 115         & 144         & 106         & 161         & \multicolumn{1}{c|}{74}      \\
\multicolumn{1}{|l|}{1\_1\_1}      & \multicolumn{1}{c|}{50}      & \multicolumn{1}{c|}{34}      &  \multicolumn{1}{|l|}{with 6 Strings}                       & 37          & 13           & 40          & 34          & 58          & 34          & 19           & \multicolumn{1}{c|}{31}         & \multicolumn{1}{l|}{Guitar}              & 77          & 18          & 41          & 40          & 13          & 9           & 0           & \multicolumn{1}{c|}{8}       \\
\multicolumn{1}{|l|}{1\_1\_1\_1}     & \multicolumn{1}{c|}{50}      & \multicolumn{1}{c|}{40}         &  \multicolumn{1}{|l|}{with No Input Jack}                     & 66          & 77        & 53          & 54          & 26          & 67        & 68          & \multicolumn{1}{c|}{50}       & \multicolumn{1}{l|}{Acoustic Guitar}     & 13          & 81          & 66          & 43          & 70          & 70          & 82          & \multicolumn{1}{c|}{71}       \\
\multicolumn{1}{|l|}{1\_1\_1\_2}      & \multicolumn{1}{c|}{50}     & \multicolumn{1}{c|}{43}     & \multicolumn{1}{|l|}{with Input Jack}                          & 54          & 61          & 67          & 43          & 28          & 45         & 82          & \multicolumn{1}{c|}{78}       & \multicolumn{1}{l|}{Electric Guitar}     & 74          & 65          & 86          & 44          & 70          & 71          & 68          & \multicolumn{1}{c|}{74}    \\
\multicolumn{1}{|l|}{1\_1\_2}     & \multicolumn{1}{c|}{50}      & \multicolumn{1}{c|}{31}    & \multicolumn{1}{|l|}{Elliptical body with 3 or 4 strings} & 61          & 37          & 36          & 32          & 29          & 30         & 37          & \multicolumn{1}{c|}{30}          & \multicolumn{1}{l|}{Dulcimer}            & 0           & 6           & 31          & 31          & 0           & 16          & 6           & \multicolumn{1}{c|}{46}     \\
\multicolumn{1}{|l|}{1\_1\_3}      & \multicolumn{1}{c|}{50}      & \multicolumn{1}{c|}{27}   & \multicolumn{1}{|l|}{with 13 Strings}                         & 42          & 32          & 42          & 16          & 18          & 44          & 53          & \multicolumn{1}{c|}{45}           & \multicolumn{1}{l|}{Koto}                & 0           & 47          & 47          & 39          & 0           & 41          & 0           & \multicolumn{1}{c|}{43}          \\
\multicolumn{1}{|l|}{1\_2}      & \multicolumn{1}{c|}{50}     & \multicolumn{1}{c|}{47}      & \multicolumn{1}{|l|}{with Keyboard}                      & 49          & 46          & 41          & 49          & 43          & 47          & 50          & \multicolumn{1}{c|}{46}       & \multicolumn{1}{l|}{Keyboard Instrument} & 41          & 49          & 0           & 44          & 40          & 47          & 52          & \multicolumn{1}{c|}{44}       \\
\multicolumn{1}{|l|}{1\_3}         & \multicolumn{1}{c|}{50}     & \multicolumn{1}{c|}{47}    & \multicolumn{1}{|l|}{with Embouchure}             & 62          & 60          & 60          & 63          & 50          & 55          & 60          & \multicolumn{1}{c|}{59}             & \multicolumn{1}{l|}{Wind Instrument}     & 61          & 57          & 55          & 62          & 54          & 61          & 60          & \multicolumn{1}{c|}{59}      \\
\multicolumn{1}{|l|}{}     & \multicolumn{1}{c|}{0}  & \multicolumn{1}{c|}{17}       & \multicolumn{1}{|l|}{Unrecognised}                         & 0           & 15          & 13          & 1           & 58          & 12          & 7           & \multicolumn{1}{c|}{14}          & \multicolumn{1}{l|}{Unrecognised}        & 6           & 7           & 24          & 15          & 32          & 9           & 2           & \multicolumn{1}{c|}{5}    \\ \hline
\multicolumn{1}{|l|}{all}   &   \multicolumn{1}{c|}{450}              & \multicolumn{1}{c|}{450}     & \multicolumn{1}{|l|}{Krippendorff’s alpha}          & \multicolumn{8}{|c|}{0.5973}       & \multicolumn{1}{l|}{Krippendorff’s alpha}                                                                                                       & \multicolumn{8}{c|}{0.5047}      \\ \hline
\end{tabular}
}
\vspace{0.2cm}
\caption{Inter-annotator agreement evaluation with Choice C2 only. The images in GT1 have ImageNet labels. GT2, GT3, GT4 images have labels selected by the expert annotator, and by the 8+8 non-expert annotators in Groups 1 and 2, respectively. NA$_{j,i}$ stands for annotator $i$ of Group $j$. Id. is the label identifier, as from Fig. 2. The last row reports the value of Krippendorff's alpha.}
\label{tab:3}
\end{table*}


\begin{table}
\centering
\setlength{\abovecaptionskip}{2pt}%    
\setlength{\belowcaptionskip}{0pt}%
	\resizebox{1\linewidth}{!}{%
\begin{tabular}{|l|l|cc|cccccccc|}
\hline
\multicolumn{1}{|c|}{\multirow{2}{*}{\textbf{Index}}} &  \multicolumn{1}{|c|}{\multirow{2}{*}{\textbf{Categories}}}
& \multicolumn{2}{|c|} {\textbf{GT2*}}    
& \multicolumn{8}{c|}{\textbf{GT3* (Only Single-Object images)}}     \\ \cline{3-12} 
\multicolumn{1}{|c|}{}      &       & \textbf{EA$_1$}& \textbf{EA$_2$}    & \textbf{NA$_{1.1}$} & \textbf{NA$_{1.2}$} & \textbf{NA$_{1.3}$} & \textbf{NA$_{1.4}$} & \textbf{NA$_{1.5}$} & \textbf{NA$_{1.6}$} & \textbf{NA$_{1.7}$} & \multicolumn{1}{c|}{\textbf{NA$_{1.8}$}}\\ \hline


1                                                                                & \multicolumn{1}{l|}{with   Sound Mechanism}      & 17   &  17               & 16            & 2             & 5             & 11            & 11            & 11            & 4             & \multicolumn{1}{c|}{4}             \\
1\_1                                                                              & \multicolumn{1}{l|}{with Taut   Strings}           & 42 &42             & 21            & 41            & 32            & 43            & 41            & 39            & 28            & \multicolumn{1}{c|}{32}         \\
1\_1\_1                                                                           & \multicolumn{1}{l|}{with 6 Strings}                & 21 & 20            & 21            & 8             & 23            & 19            & 26            & 24            & 11            & \multicolumn{1}{c|}{15}             \\
1\_1\_1\_1                                                                        & \multicolumn{1}{l|}{with No Input   Jack}       & 21  & 22               & 31            & 34            & 29            & 28            & 18            & 32           & 33            & \multicolumn{1}{c|}{29}            \\
1\_1\_1\_2                                                                       & \multicolumn{1}{l|}{with Input Jack}                 & 22   &  22            & 24            & 32            & 31            & 21            & 20            & 21            & 39            & \multicolumn{1}{c|}{39}        \\
1\_1\_2                                                                           & \multicolumn{1}{l|}{Elliptical body with 3 or 4 strings}         & 13   &  13             & 24            & 13            & 13            & 14            & 12            & 13          & 15            & \multicolumn{1}{c|}{14}            \\
1\_1\_3                                                                           & \multicolumn{1}{l|}{with 13 Strings}             & 12  &  12                & 12            & 7             & 10            & 8             & 9             & 10            & 13            & \multicolumn{1}{c|}{9}            \\
1\_2                                                                              & \multicolumn{1}{l|}{with Keyboard}                  & 33  &  33               & 33            & 33            & 29            & 34            & 29            & 31            & 34            & \multicolumn{1}{c|}{33}       \\
1\_3                                                                              & \multicolumn{1}{l|}{with Embouchure}             & 10    &  10                & 20            & 20            & 21            & 23            & 14            & 15           & 19            & \multicolumn{1}{c|}{21}           \\
 & \multicolumn{1}{l|}{Unrecognised}                 & 11   &  11             & 0             & 12            & 9             & 1             & 22            & 6             & 6             & \multicolumn{1}{c|}{6}            \\ \hline
all    &   \multicolumn{1}{l|}{Krippendorff’s alpha}                                                                                                                & \multicolumn{2}{c|}{0.9877}                                             & \multicolumn{8}{c|}{0.7595}        \\ \hline
\end{tabular}}
\vspace{0.2cm}
\caption{Inter-annotator agreement with Choices C1, C2 (full \texttt{vTelos}), generating two GT2* and eight GT3* datasets.}
\label{tab:6}
\end{table}



\begin{table}
\centering
\setlength{\abovecaptionskip}{2pt}%    
\setlength{\belowcaptionskip}{3pt}%
	\resizebox{1\linewidth}{!}{%
\begin{tabular}{|c|l|l|l|}
\hline
\textbf{Anno.} & \textbf{Acoustic   Guitar}        & \textbf{Dulcimer}                                   & \textbf{Koto}                          \\ \hline
\textbf{NA$_{1.1}$}          & Guitar                            & Appalachian   Dulcimer                              & Biwa                                   \\
\textbf{NA$_{1.2}$}          & Guitar                            & IDK (I Don't Know)                                                 & IDK                                    \\
\textbf{NA$_{1.3}$}          & Wooden   guitar                   & IDK                                                 & Zither                                 \\
\textbf{NA$_{1.4}$}         & Wooden   guitar                   & 3 or 4 String Musical Instrument             & 13 String Koto                 \\
\textbf{NA$_{1.5}$}         & Hawaiian   Guitar                 & 3-4 Stringed Elliptical Instrument   & 13-String Instrument \\
\textbf{NA$_{1.6}$}          & 6 Stringed Instrument no Jack     & Fretted Stringed Instrument                       & 13 Stringed Instrument         \\
\textbf{NA$_{1.7}$}          & Classic   Guitar                  & Elliptical Stringed Instrument                    & Rectangular   Stringed Instrument      \\
\textbf{NA$_{1.8}$}         & Non   Powered Guitars             & Short-Stringed Music Instruments                    & Japanese   Stringed Instrument        \\ \hline
\end{tabular}}
\vspace{0.2cm}
\caption{Category Labels suggested by Group 1 annotators.}
\label{tab:4}
\end{table}