%% Author_tex.tex
%% V1.0
%% 2012/13/12
%% developed by Techset
%%
%% This file describes the coding for rsproca.cls

\documentclass[11pt]{article}

%%%% *** Do not use packages/style files that adjust lengths, control margins, column widths, etc. ***

% \usepackage[usenames,dvipsnames]{xcolor}
\usepackage{bm}
\usepackage[normalem]{ulem}
\usepackage{amsmath,amsbsy,amsfonts,amssymb,graphics,epsfig,float,mathrsfs,amsthm,braket}
\usepackage{authblk}

\textheight235truemm
\textwidth165truemm\hoffset-2.0cm
\voffset-3.cm

%%%%%%%%%%% Defining Enunciations  %%%%%%%%%%%
%\graphicspath{{./ImagesRSPA/}}
\newcommand{\upd}{\mathrm{d}}
\newcommand{\br}{\bar{r}}
\newcommand{\bc}{\bar{c}}
\newcommand{\bt}{\bar{t}}
\newcommand{\bu}{\bar{u}}
\newcommand{\bZ}{\bar{\zeta}}
\newcommand{\amax}{a_{\max}}
\newcommand{\ba}{\bar{a}}
\newcommand{\bp}{\bar{p}}
\newcommand{\bOmega}{\tilde{\Omega}}
\newcommand{\bSigma}{\bar{\sigma}}
\newcommand{\bPsi}{\bar{\psi}}
\newcommand{\chiMS}{\chi_{\mathrm{MS}}}
\newcommand{\ETwoD}{E_{2D}}
\newcommand{\Rd}{R_{\mathrm{drum}}}
\newcommand{\Tpre}{T_{\mathrm{pre}}}
\newcommand{\meff}{m_{\mathrm{eff}}}
\newcommand{\kone}{\alpha_1}
\newcommand{\ktwo}{\alpha_2}
\newcommand{\kthree}{\alpha_3}
\newcommand{\cc}{\mathrm{c.c.}}
\newcommand{\ii}{\mathrm{i}}
\newcommand{\bending}{{\mathcal B}}

\newcommand{\blue}[1]{{\color{blue} #1 }}
\newcommand{\red}[1]{{\color{red} #1 }}
\newcommand{\green}[1]{{\color{green} #1 }}
\newcommand{\dv}[1]{\emph{\color{magenta} [#1]}}
\newcommand{\saf}[1]{\emph{\color{cyan} [#1]}}
\newcommand{\rout}[1]{\red{\sout{#1}}}

\newcommand{\s}[1]{{\textsf{\textbf{#1}}}}


\begin{document}

%%%% Article title to be placed here
\title{\s{Premature jump-down mimicks nonlinear damping in nanoresonators}}
\author{ \textsf{Safvan Palathingal$^\dagger$ and Dominic Vella$^\ddagger$}}
\date{
{\it $^{\dagger}$Department of Mechanical and Aerospace Engineering,\\ Indian Institute of Technology Hyderabad, Telangana, India}\\[2ex]
{\it $^{\ddagger}$Mathematical Institute, University of Oxford,\\Woodstock Rd, Oxford, OX2 6GG, UK}\\[2ex]
\today}
%\author[1]{\textsf{Safvan Palathingal}}
%\author[2]{\textsf{Dominic Vella}}


 \maketitle
\hrule\vskip 6pt

%%%% Abstract text to be placed here %%%%%%%%%%%%
\begin{abstract}
Recent experiments on  nano-resonators in a bistable regime use the `jump-down' point between states to infer mechanical properties of the membrane or a load, but often suggest the presence of some nonlinear damping. Motivated by such experiments, we develop a mechanical model of a membrane subject to a uniform, oscillatory load and linear damping. We solve this model numerically and compare its jump-down behaviour with standard asymptotic predictions for a one-dimensional Duffing oscillator with strain stiffening. We show that the axisymmetric, but spatially-varying, problem can be mapped to the Duffing problem with coefficients determined rationally from the model's Partial Differential Equations. However, we also show that jump-down happens earlier than expected (i.e.~at lower frequency, and with a smaller oscillation amplitude). Although this premature jump-down is often interpreted  as the signature of a nonlinear damping in experiments, its appearance in numerical simulations with only linear damping suggests instead that indicate that the limitations of asymptotic results may, at least sometimes, be the cause. We therefore suggest that care should be exercised in interpreting the results of nano-resonator experiments.
\end{abstract}

\vskip 6pt
\hrule

%\maketitle



\section{Introduction}

Having a high modulus of elasticity while remaining lightweight is one of the exceptional mechanical 
properties of two-dimensional (2D) wonder materials such as graphene. These mechanical properties, together with high electrical and thermal conductivity, electron mobility, and optical transparency make such materials  attractive prospects for various engineering applications \cite{khan2017mechanical,carvalho2022review}. Moreover, the coupling between mechanical strain and electronic properties of graphene \cite{Guinea2010} make understanding the mechanics critical to realizing these applications. As a consequence, there is significant interest in developing reliable methods for characterizing its mechanical properties \cite{Akinwande2017}.

As 2D materials deform in three dimensions, they must, in general, stretch in the plane. In particular, Gauss' \emph{Theorema egregium} shows that axisymmetric deformations of a flat 2D material must be accompanied by some stretching. This coupling between bending and stretching modes of deformation in turn leads to nonlinearities in the mechanics of 2D materials: deformation-induced stretching changes the in-plane tension and hence the out-of-plane deformation. Over the last two decades, researchers have  attempted to determine the 2D modulus of these materials from the strength of the nonlinearity using a combination of modelling and experiment in some (apparently) simple loading settings  \cite{Al-Quraishi.2020,Akinwande2017}. However, these studies have produced contradictory values for the 2D modulus, corresponding to an effective Young's modulus somewhere in the range 400-1200 GPa \cite{Akinwande2017, castellanos2015mechanics}. The fact that multiple static and dynamic techniques have been developed and give different results highlights that measuring the properties of graphene is challenging, but also suggests that subtleties of the underlying mechanics of each probe may somehow confound such experiments. 

In this paper, we focus on understanding the nonlinearity that arises in dynamic oscillations of suspended flakes of graphene \cite{davidovikj_nonlinear_2017} and other 2D materials \cite{Kaisar2022}.  These oscillations help measure masses with single-atom resolution \cite{Jensen2008} but require very low damping (to give a high-quality factor). The measurement relies on fitting the experimentally measured amplitude-frequency response and so needs a model of the nonlinear coupling between stress and strain. These have usually been modelled under the assumption of a simple cubic nonlinearity and linear damping, leading to a Duffing equation \cite{nayfeh_nonlinear_2008} in a single, time-dependent, variable.  

Generally, the compliance  of such a nanoresonator (the  oscillation amplitude at jump-down relative to the maximum force)  is expected to be constant provided that the damping is viscous and hence linear in velocity \cite{LifshitzReview}. However, to fit the results of  experiments, previous authors have required a dissipation that increases with oscillation amplitude; the damping appears to be nonlinear, rather than viscous.  For instance, the data of Davidovikj \emph{et al.}~\cite{davidovikj_nonlinear_2017}, which is reproduced in fig.~\ref{fig:ExptData}, 
% Figure environment removedshows that experimentally, the jump-down compliance is not constant 
(fig.~\ref{fig:ExptData} b). Does this change in compliance necessarily signify nonlinear damping? While this may appear to be the case, a careful analysis in a controlled setting has not, to our knowledge, been performed. 




In ref.~\cite{davidovikj_nonlinear_2017}, the mismatch in the identified peak amplitudes between the analytical frequency-response curves (obtained from Duffing's equation) and those obtained  experimentally have been accounted for by allowing the quality factor to vary with the driving amplitude: essentially an unknown nonlinear damping is assumed. Alternatively, quadratic and cubic nonlinear damping forms have been proposed to account for the reduction in compliance \cite{Eichler.2011,LifshitzReview,Farokhi.2023}. However, we also note that for the effective spring system, it is known \cite{Brennan2008} that different asymptotic approaches yield different behaviours in the amplitude-response curve (and consequently might require different nonlinear damping law to explain observations).
 
Here, we investigate whether the issue could lie with the regimes of validity of the asymptotic results,  rather than the presence of unknown forms of damping in experiments. We use numerical experiments on a  model that accounts for spatial variation within the stress field (thereby going beyond the standard nonlinear spring analogy)  but uses only a linear damping. We show that the resulting amplitude-response behaviour shows persistent deviations from that predicted by asymptotic techniques;  in particular, we observe premature jump-down in the amplitude-frequency response. This behaviour would conventionally be interpreted as the effect of a nonlinear damping, even though no such damping exists. To make this discussion more concrete, we first give a more technical description of the two commonly used asymptotic approaches applied to Duffing's equation --- an ordinary differential equation, and hence without some of the complications caused by spatial, as well as temporal, variation.



\section{Approximate Analyses of the Duffing equation \label{sec:Asymptotics}}



Previous analysis of the nanoresonator problem is based on the modelling assumption that its state may be characterized by a single variable, $x(t)$, which satisfies the Duffing equation
\begin{equation}
m\frac{\upd^2x}{\upd t^2}+c\frac{\upd x}{\upd t}+k_1 x+k_3x^3=F \cos(\omega t+\phi),
\label{eqn:DuffingDim}
\end{equation} for constants $m$, $k_1$, $k_3$ and $F$ that represent a typical mass, a linear stiffness, a nonlinear stiffness and the applied forcing, respectively. Here the term $c\upd x/\upd t$ represents linear viscous damping provided that $c$ is constant.

Equation \eqref{eqn:DuffingDim} may be non-dimensionalized by using the natural frequency $\omega_0=(k_1/m)^{1/2}$ to non-dimensionalize times, i.e.~we let $T=\omega_0t$, and scaling lengths with $x_0=(k_1/k_3)^{1/2}$, i.e.~we let $X=x/x_0$. We then find that \eqref{eqn:DuffingDim} becomes
\begin{equation}
    \frac{\upd^2X}{\upd T^2}+\gamma\frac{\upd X}{\upd T}+X+X^3={\cal F} \cos(\bOmega T+\phi)
\label{eqn:DuffingND}
\end{equation} where
\begin{equation}
    \gamma=\frac{c}{(m k_1)^{1/2}},\quad {\cal F}=F\left(\frac{k_3}{k_1^3}\right)^{1/2},\quad \bOmega=\frac{\omega}{(k_1/m)^{1/2}}.
\end{equation}

The solution of the forced, nonlinear oscillation of \eqref{eqn:DuffingND} is, unsurprisingly, oscillatory;  the key quantity of interest then becomes the maximum amplitude of these oscillations; this response is usually expressed as an amplitude response curve, giving the maximum amplitude as a function of the imposed frequency measured relative to the natural frequency, $\bOmega$.


Two methods are commonly used to determine the frequency-amplitude response: Harmonic Balance (HB) and the method of Multiple Scales (MS). In HB, an ansatz $X(T)=a\cos\bOmega T$ is posed, and leads to the frequency-response relationship \cite{Jordan2007} 
\begin{equation}
    {\cal F}^2=a^2\left[\left(\bOmega^2-1-\tfrac{3}{4}a^2\right)^2+\gamma^2\bOmega^2\right]
    \label{eqn:ModelHB}
\end{equation} The alternative approach, using the method of multiple scales, leads \cite{nayfeh_nonlinear_2008} to
\begin{equation}
    {\cal F}^2=a^2\left[\left(\bOmega^2-1-\tfrac{3}{4}a^2\right)^2+\gamma^2\right].
    \label{eqn:ModelMS}
\end{equation}

The expressions in \eqref{eqn:ModelHB} and \eqref{eqn:ModelMS} may be viewed as a polynomial for $a$ as a function of the forcing frequency $\bOmega$ at fixed forcing strength, ${\cal F}$. However, thinking in this way somewhat obscures a key feature of the behaviour: that with fixed parameters the system may adopt more than one stable state (bistability). This is more easily seen if we seek instead the frequency required to give a particular response amplitude, i.e.~$\bOmega=\bOmega(a)$.
 In particular, the quartic for $\bOmega$ (or quadratic in $\bOmega^2$) may be written:
\begin{equation}
\label{eqn:DuffingQuartic}
    0=\bOmega^4-2b\bOmega^2+C
\end{equation} where
\[
b=1+\frac{3}{4}a^2-\frac{1}{2}\gamma^2(1-\chiMS)
\]
\[
C=\left(1+\frac{3}{4}a^2\right)^2-\frac{{\cal F}^2}{a^2}+ \gamma^2\chiMS
\] and $\chiMS$ indicates whether we use the method of Multiple Scales or Harmonic Balance (i.e.~$\chiMS=0$ for HB and $\chiMS=1$ for the MS approach). We therefore find that
\begin{equation}
\bOmega^2=b\pm\left(b^2-C\right)^{1/2}.
\end{equation}  Crucially, the $\pm$ above shows that there are, in general, two values of $\bOmega>0$ that give a particular displacement $a$; conversely, there may be multiple values of $a$ that can be attained at a fixed value of $\bOmega$ --- the system exhibits bistability over a range of forcing frequencies $[\bOmega_{\min},\bOmega_{\max}]$. It is readily shown that the upper limit of bistability, $\bOmega_{\max}$, is $\bOmega_{\max}=b^{1/2}$ for a hardening response ($k_3>0$); the curve $\bOmega_{\max}=b^{1/2}$ is referred to the backbone, and describes how this maximum frequency with bistability evolves with increased loading \cite{Wawrzynski2021}. As $\bOmega$ increases slowly beyond $\bOmega_{\max}$ this branch of solutions disappears and the amplitude $a$ must drop sharply --- it falls, or jumps, down --- a  point we refer to as the `jump-down' point. In dimensional terms the frequency at jump down is
\begin{equation}
    \omega_{\max}=\left(\frac{k_1}{m}\right)^{1/2}\left[1+\frac{3k_3}{4k_1}x_{\max}^2-\frac{1-\chiMS}{2}\frac{c^2}{mk_1}\right]^{1/2}, 
    \label{eqn:omegaMax}
\end{equation}   where $x_{\max}=x(\bOmega_{\max})$ is the value of the variable $x$ at jump-down, and is \emph{not}, in general, precisely equal to the maximum amplitude that is observed. 

In the example of a nanoresonator, it is natural to assume that $k_3\propto \ETwoD$, with $\ETwoD$ the stretching modulus of the membrane. Hence, \eqref{eqn:omegaMax} shows that the maximum shift in frequency observed before the response `jumps-down' to the lower branch gives a relatively sensitive measure of $\ETwoD$ --- bistability is a consequence of the nonlinearity of the system and hence its disappearance may be used to infer the degree of nonlinearity in different systems \cite{ramlan2016exploiting}. However,  this technique requires us to know the damping strength, $c$ (at least if using the Harmonic Balance technique). This has been determined previously\cite{davidovikj_nonlinear_2017} by simultaneously fitting the  amplitude at jump-down (denoted $\amax$), which can  be determined from the requirement that $b^2=C$. The value of $\amax$ is perhaps most easily expressed in terms of the jump-down compliance of the system, defined as $\gamma\amax/{\cal F}$, which is given by
\begin{equation}
    \left(\frac{\gamma\amax}{{\cal F}}\right)^2=\begin{cases}1,&\chiMS=1\\
    \frac{2\gamma^2}{3{\cal F}^2}\left\{\frac{\gamma^2}{4}-1+\left[(1-\gamma^2/4)^2+3{\cal F}^2/\gamma^2\right]^{1/2}\right\},&\chiMS=0
    \end{cases}.
    \label{eqn:IntroCompliance}
\end{equation} Note that for ${\cal F}\ll\gamma\ll1$ the two predictions agree approximately: both methods predict a constant jump-down compliance  when the loading is small. However,  \eqref{eqn:IntroCompliance} also shows that the Perturbation Method retains this constant jump-down compliance even for large forcing; the HB method, in contrast, suggests that the jump-down compliance decreases as the forcing increases. {This finding is in agreement with our numerical simulations of \eqref{eqn:DuffingND}, which shows that \eqref{eqn:omegaMax} and \eqref{eqn:IntroCompliance} with $\chiMS=0$ together give an excellent account of the position of the jump-down point.}

As a concrete example of the use of jump-down to infer properties of the underlying nonlinear response, recent experimental work  \cite{davidovikj_nonlinear_2017} has attempted to infer the stretching modulus of graphene by gradually increasing $\bOmega$ until  the amplitude of the response suddenly decreases --- a point that is assumed to correspond to `jump-down'. By equating this point with the theoretically determined value $\omega_{\max}$ in \eqref{eqn:omegaMax} values of the stretching modulus $\ETwoD$ corresponding to $E=594\pm45\mathrm{~GPa}$ were obtained. 
 
 Crucially, the two standard asymptotic methods for determining the amplitude-response curve give quite different results for the location of this jump-down point: whether the jump-down frequency in \eqref{eqn:omegaMax} depends on the damping coefficient $c$ itself depends on whether the multiple scales or harmonic balance method has been used.


% Can then also try to calculate the width of the region close to jump-down to try and understand the scalings that SP has seen.

As discussed in the Introduction, experimental data clearly shows that the jump-down compliance is not constant as the magnitude of the load increases (see figure \ref{fig:ExptData}). One interpretation of this is to postulate the presence of a nonlinear damping, as suggested by \cite{LifshitzReview} (see figure 1.5 of ref.~\cite{LifshitzReview}).  However, we have also seen that the Harmonic Balance method would predict a jump-down compliance that decreases with increasing load, see \eqref{eqn:IntroCompliance}. Nevertheless, Davidovikj \emph{et al.} \cite{davidovikj_nonlinear_2017} found that, even using the results from the Harmonic Balance method,  the observed jump-down compliance still decreases faster than expected as the forcing amplitude increases. They attribute this to some unknown nonlinear damping. In this paper, we ask the question of whether such a deduction is reasonable: does premature jump-down necessarily indicate the presence of a nonlinear damping? We show that the numerical solution of a model with purely linear damping also shows premature jump-down.



\section{Modelling and numerical results}

 We model the dynamics of the nano-resonator as a thin (i.e.~two-dimensional), circular  elastic membrane with clamped boundary at a radius $\Rd$.  The membrane is subject to a uniform pre-tension, $\Tpre$, as well as a spatially uniform, but time-varying, pressure $\bp(\bt)$. We assume that the membrane has a mass per unit area $\rho h$ (so that $h$ is a nominal thickness and $\rho$ a nominal density). As a result of the oscillating applied pressure, the membrane adopts a profile $\bZ(\br,\bt)$ that varies in both time and space. The setup is illustrated schematically in fig.~\ref{fig:Setup}.
 

% Figure environment removed

\subsection{Governing equations}

 To simplify the problem we make two key  assumptions: (i) the effects of the membrane's bending stiffness may be neglected and (ii) the stress profile within the membrane is instantaneously determined by its shape. The validity of these assumptions is discussed in more detail in Appendix \ref{sec:SimplifyingAssumptions}. With these assumptions, we use the standard equations of vertical and in-plane force balance on the membrane, presenting these first in dimensional form, before non-dimensionalizing appropriately.

\subsubsection{Dimensional problem}

The vertical force balance equation for a (zero bending stiffness) membrane  is
\begin{equation}
\rho h \frac{\partial^2 \bZ}{\partial \bt^2} + \bc h \frac{\partial \bZ}{\partial \bt}=\bp(\bt)+\bm{\nabla}\cdot\left( \mathbf{\bSigma}\cdot\bm{\nabla}\bZ \right),
    \label{Eq:governing_out_of_plane}
\end{equation} where the damping coefficient is denoted  $\bc$ and the stress tensor  $\mathbf{\bSigma}$.  The boundary conditions on the transverse displacements are:
\begin{equation}
    \left.\frac{\partial\bZ}{\partial r}\right|_{r=0}=0,\quad \bZ(\Rd)=0.
    \label{Eq:out_of_plane_bcs}
\end{equation} Here the slope, $\partial\bZ/\partial \br$, is zero at the centre, $\br=0$, since we restrict our attention to axisymmetric solutions and there is no vertical displacement at the outer edge ($\br=\Rd$) where it is clamped. 

Assuming the in-plane force balance is satisfied instantaneously (no in-plane acceleration) we also have
\begin{equation}
\bm{\nabla}.\mathbf{\bSigma}=0.
    \label{Eq:governing_in_plane}
\end{equation} To ensure that this equation is automatically satisfied in all that follows, we introduce the first integral of the Airy stress function, which we denote $\bPsi(\br,\bt)$ and is defined such that
  \begin{equation}
      \bSigma_{r r} =\frac{\bPsi}{\br},\quad  \sigma_{\theta \theta}=\frac{\partial \bPsi}{\partial \br}.
  \end{equation} The stress function $\bPsi$ is determined by the requirement that the corresponding displacement field is physically realizable --- the so-called compatibility equation
  \begin{equation}
      \br\frac{\partial}{\partial \br}\left[\frac{1}{\br}\frac{\partial(\br\bPsi)}{\partial \br}\right] =-\frac{\ETwoD}{2} \left(\frac{\partial\bZ}{\partial \br}\right)^2,
    \label{Eq:compatibility_dimensional}
  \end{equation} where  $\ETwoD = E h$, is the two-dimensional Elastic modulus of the membrane.  

While considering the boundary conditions for in-plane displacements $\mathbf{\bu}=\bu_{\bar{r}} (\br,\bt)\mathbf{e}_{\br}$, we need to account for the additional deformation ($\bu_0(\br)$) imposed prior to loading by the pre-tension, $\Tpre$. Though there is no deformation at the centre caused by pre-tension, there is a nonzero horizontal displacement ($\bu_0(\Rd)$) where the clamping is imposed.  This value remains unchanged when the sheet is later loaded with pressure. Hence we must have:  
\begin{equation}
    \bu_r(0,t)=0,\quad \bu_{\br}(\Rd,\bt)=(1-\nu)\frac{\Tpre\Rd}{\ETwoD}.
    \label{Eq:in_plane_bcs}
\end{equation} Using the Hookean constitutive relationship, these boundary conditions can be expressed in terms of the stress function as
\begin{equation}
    \lim_{\br\to0}\left[\br\frac{\partial\bPsi}{\partial \br}-\nu\bPsi\right]=0,\quad \Rd\left.\frac{\partial\bPsi}{\partial r}\right|_{\Rd}-\nu\bPsi(\Rd,\bt)=(1-\nu)\frac{\Tpre}{\ETwoD}.
    \label{eqn:BCs_Psi_Dim}
\end{equation}
 
\subsubsection{Non-dimensionalization}
\label{Sec:Non-dimensionalization}
To determine the dimensionless forms of \eqref{Eq:governing_out_of_plane}--\eqref{Eq:out_of_plane_bcs}, we introduce the dimensionless variables:
\begin{align}
    r=\frac{\br}{\Rd} , \quad 
    t=\bt  \left(\frac{\Tpre}{\rho h\Rd^2}\right)^{1/2} ,  \quad{\bm{\sigma}}= \bm{\bSigma}/\Tpre , \nonumber\\ {\zeta}({r},{t})=\bZ(\br,\bt)\left(\frac{\ETwoD}{\Tpre\Rd^2}\right)^\frac{1}{2} ,  
    %\quad \bar{u}=\frac{u}{\Rd}\frac{\ETwoD}{\Tpre},
    \quad {\psi}({r},{t})=\frac{\bPsi(\br,\bt)}{\Tpre\Rd}.
\end{align}
The dimensionless version of \eqref{Eq:governing_out_of_plane} is then
\begin{equation}
\frac{\partial^2 {\zeta}}{\partial {t}^2}+  \gamma \frac{\partial {\zeta}}{\partial {t}}={ p}({r},t)+\frac{1}{r}\frac{\partial}{\partial r}\left( {\psi}\frac{\partial{\zeta}}{\partial {r}} \right),
    %\label{Eq:governing_out_of_plane_dimensionless}
    \label{Eq:VForceBal_ND}
\end{equation}
 where \begin{equation}
{p}=\bp\frac{\Rd \ETwoD^{1/2}}{\Tpre^{3/2}},     \label{Eq:normal_p}     
 \end{equation} is the dimensionless forcing pressure and
\begin{equation}
    \gamma= \bc  \left(\frac{ h\Rd^2}{\rho  \Tpre}\right)^{1/2} ,\label{Eq:gamma}
\end{equation} is the dimensionless damping coefficient. The dimensionless  compatibility equation is given by
\begin{equation}
    {r}\frac{\partial}{\partial{r}}\left[\frac{1}{{r}}\frac{\partial({r}{\psi})}{\partial{r}}\right] =-\frac{1}{2} \left(\frac{\partial{\zeta}}{\partial{r}}\right)^2.
    \label{Eq:compatibility_dimensionless}
\end{equation}
Equations \eqref{Eq:VForceBal_ND} and \eqref{Eq:compatibility_dimensionless} are to be solved subject to the dimensionless boundary conditions:
\begin{equation}
    \left.\frac{\partial{\zeta}}{\partial {r}}\right|_{r=0}=0,\quad {\zeta}(1,t)=0, \quad 
    \lim_{{r}\to0}\left[{r}\frac{\partial{\zeta}}{\partial {r}}-\nu{\psi}\right]=0,\quad \left.\frac{\partial{\psi}}{\partial {r}}\right|_{r=1}-\nu{\psi}(1,t)=(1-\nu)
    \label{Eq:bcs_dimensionless}.
\end{equation}

In our numerics, we use a purely oscillatory forcing pressure
\begin{equation}
 p=P\cos \Omega t.
 \label{eqn:}
\end{equation} Our model does not, therefore, directly account for the experiments by Davidovikj \emph{et al.} \cite{davidovikj_nonlinear_2017} who applied a (constant) DC bias in addition to the voltage.


\subsection{Numerical method}


We determine the dynamic evolution of the membrane by solving \eqref{Eq:VForceBal_ND}--\eqref{Eq:bcs_dimensionless} numerically. 
The partial differential equation (PDE) \eqref{Eq:VForceBal_ND} is numerically solved by using the method of lines with numerical integration performed in Python using the \texttt{solve\_ivp} routine to advance the shape of the membrane in time. At each time step, the stress state must also be updated in response to the shape; this is a boundary value problem and hence is solved using \texttt{solve\_bvp}. Details of this procedure, and various convergence tests, are presented in Appendices \ref{subsec:num_implementation} and \ref{subsec:convergence}.

Our primary interest lies in the response of the membrane to an oscillating pressure field --- this mimicks the effect of an oscillating applied voltage as used experimentally by Davidovikj \emph{et al.} \cite{davidovikj_nonlinear_2017}. The numerical results show that, after an initial transient, the vibrations of the membrane reach an approximately periodic state, with some  amplitude $a$. These results also show that the value of $a$ is affected by both the applied pressure and the applied frequency. This motivates us to plot the amplitude--response curve (fig.~\ref{fig:FirstResponse}). % Figure environment removed We do this by a forward sweep, i.e., increase the frequency by small increments starting from a frequency close to zero. The first and leftmost point on the curve is obtained with this small forcing frequency, assuming that the membrane starts from rest. Subsequently, we take the initial conditions to be the membrane shape and velocity corresponding to the peak amplitude $a$ of the previous step. As we continue the forward sweep, the amplitude reaches $a_\mathrm{max}$, and a further increase in frequency ($\Omega>\Omega_\mathrm{max}$) suddenly causes the amplitude to jump down to a significantly smaller value. Therefore, we need to ensure that the increment used for the sweep is sufficiently small to capture the right jump-down frequency; this is particularly important close to the jump-down frequency. The markers in parula colours in fig.~\ref{fig:FirstResponse} correspond to forward sweeps with sufficiently small fixed increments, with each colour representing a different magnitude of pressure. We ensure that these sweeps indeed capture the jump-down frequency accurately by running an adaptive sweep depicted by brown markers in fig.~\ref{fig:FirstResponse}. In the adaptive sweep, when a jump-down occurs, we pause and go back a step and then sweep forward with a smaller increment. We do this as long as the increment size is larger than a specified tolerance  ($\Delta \bOmega>10^{-6}$). 



\section{Theory}

We now seek to determine asymptotic results for the amplitude of oscillations as a function of the forcing pressure and frequency (as well as the linear damping). For simplicity in the presentation  of this, we let
\begin{equation}
    {p}(r,t)=P\cos(\Omega t+\phi).
\end{equation} In so doing, we have introduced a phase shift, $\phi$, to the driving (rather than the response) because this simplifies the subsequent calculation --- the phase shift in the driving oscillation is thus measured relative to that in the response.

Our primary goal is to determine the frequency-amplitude response curve (analogous to figure \ref{fig:ExptData}). As outlined in \S\ref{sec:Asymptotics}, There are two common asymptotic techniques used to determine this: Multiple Scales and Harmonic Balance. We consider these in turn, giving detail of the Multiple Scales approach. {The key difficulty here (in comparison to the standard Duffing equation) is to account for the spatial, as well as temporal, variation in the membrane's deflection and stress profiles.}


\subsection{Multiple scales approach}

In the multiple scales approach, we assume that the motion takes place over a time scale of order unity (comparable to the frequency of the natural, small displacement oscillation) but also that the oscillation amplitude itself evolves over a longer, slow, time scale. We therefore introduce two time scales, $t_0=t$ and $t_1=t/\epsilon$,  where $\Omega=\omega_0+\epsilon\varsigma$ and $\epsilon\ll1$ --- here $\varsigma$ is the detuning parameter and measures how far from the resonant frequency, $\omega_0$, the oscillatory forcing is.  With these two time scales, we  note that
\begin{equation}
    \frac{\partial^2}{\partial t^2}=\frac{\partial^2}{\partial t_0^2}+2\epsilon \frac{\partial^2}{\partial t_0\partial t_1}+O(\epsilon^2).
    \label{eqn:TimeDerivative}
\end{equation}

Our aim is to understand how the frequency at which the jump-down is observed is affected by the change in the stress within the membrane caused by deformation. We have defined $\epsilon$ to be the scale of the shift in the maximum response frequency that is caused by the modification to the stress and so we expect that this perturbation to the stress enters also at $O(\epsilon)$; we therefore let
\begin{equation}
\psi(r,t_0,t_1)=r+\epsilon\psi_2(r,t_0,t_1)+...
\end{equation} where the leading order (linear) term represents the state of uniform, isotropic pre-tension that exists prior to any vibration. We expect this perturbation to the stress to be caused by the vertical deflection of the membrane whose typical size we denote $\delta$. A simple geometrical argument (see, for example, \cite{Chopin2008}) shows that the induced strain $\sim\delta^2$; since stress is proportional to strain, we therefore expect that $\delta=O(\epsilon^{1/2})$. (Alternatively, the compatibility equation \eqref{Eq:compatibility_dimensionless} leads to the same scaling.) Hence, we expect that at leading order, the deflection $\zeta=O(\epsilon^{1/2})$. This leading-order amplitude, however, evolves once $t_1=O(1)$, introducing terms at $O(\epsilon^{3/2})$ via the derivative in \eqref{eqn:TimeDerivative}. These $O(\epsilon^{3/2})$ terms will introduce a spatial dependence that cannot be balanced by the existing terms (since they are determined at higher order), and so we expect the perturbation to $\zeta(r,t_0,t_1)$ will enter at $O(\epsilon^{3/2})$, rather than $O(\epsilon)$. We therefore let 
\begin{equation}
\zeta(r,t_0,t_1)=\epsilon^{1/2}\zeta_1(r,t_0,t_1)+\epsilon^{3/2}\zeta_3(r,t_0,t_1)+... .
\end{equation} 


We must also choose scalings for the damping and the forcing amplitudes, as functions of $\epsilon$. We expect that the damping, $\gamma\partial\zeta/\partial t$, and the forcing, $P$, should enter at the same order of magnitude: the energy inputted via the forcing must be dissipated by the damping. As a result, we expect that $P\sim\epsilon^{1/2}\gamma$.  We also expect that the damping and forcing should enter the problem first through the problem for $\zeta_3$ --- the problem for $\zeta_1$ represents free oscillations of the membrane, and so does not rely on the forcing or damping. As a result, we expect $\gamma=O(\epsilon)$, $P=O(\epsilon^{3/2})$; we therefore let $\gamma=\epsilon\Gamma$ and $P=\epsilon^{3/2}\Pi$.






Having chosen these scalings we find that, at leading order:
\begin{equation}
    {\cal L}(\zeta_1)=0,
    \label{eqn:Zeta1WaveEqn}
\end{equation} where the operator ${\cal L}(\cdot)$ is defined by
\begin{equation}
    {\cal L}(\cdot):=\frac{\partial^2(\cdot)}{\partial t_0^2}-\frac{1}{r}\frac{\partial}{\partial r}\left(r\frac{\partial (\cdot)}{\partial r}\right).
\end{equation}

Equation \eqref{eqn:Zeta1WaveEqn} describes the free oscillations of the membrane (as expected based on the choice of size of $\gamma$ and $P$) and has solution
\begin{equation}
\zeta_1(r,t_0,t_1)=\left[A(t_1)e^{\ii\omega_0t_0}+\cc\right]J_0(\omega_0r).
\end{equation} Here $\cc$ denotes the complex conjugate and the (dimensionless) fundamental frequency $\omega_0\approx2.4048$ is chosen to ensure that the boundary condition $\zeta(1,t_0,t_1)=0$.


With this leading order solution for the shape, the leading order problem for the stress is
\begin{equation}
    r\frac{1}{r}\frac{\partial}{\partial r}\left[\frac{1}{r}\frac{\partial(r\psi_2)}{\partial r}\right]=-\frac{\omega_0^2}{2}\left[A(t_1)^2e^{2\ii\omega_0t_0}+|A|^2+\cc\right]\left[J_1(\omega_0r)\right]^2.
\end{equation} We therefore find that
\begin{equation}
\psi_2(r,t_0,t_1)=\left[A(t_1)^2e^{2\ii\omega_0t_0}+|A|^2+\cc\right]\, Y(r)
\label{eqn:psieqn}
\end{equation} 
where $Y(r)$ satisfies 
\begin{equation}
    r\frac{\upd}{\upd r}\left[\frac{1}{r}\frac{\upd}{\upd r}(rY)\right]=-\frac{\omega_0^2}{2}\left[J_1(\omega_0r)\right]^2,
    \label{eqn:CompatGovEq}
\end{equation} subject to boundary conditions
\begin{equation}
    \lim_{r\to0}\bigl[rY'-\nu Y\bigr]=0,\quad Y'(1)=\nu Y(1).
    \label{eqn:CompatBCsG0}
\end{equation} The function $Y(r)$ can be found analytically to be
\begin{equation}
Y(r)=\beta r+\frac{\omega_0^2r}{8} \setlength\arraycolsep{1pt}
{}_1F_2\left(\{1/2\};\{2,2\};-\omega_0^2r^2\right)
\label{eqn:Yr}
\end{equation} where $\setlength\arraycolsep{1pt}
{}_pF_q\left(\{a_1,...,a_p\};\{b_1,...,b_q\};x\right)$ is the generalized hypergeometric function \cite{NISThandbook} and 
\[
\beta=\beta(\nu)=-\frac{\omega_0^2}{8}\setlength\arraycolsep{1pt}
{}_1F_2\left(\{1/2\};\{2,2\};-\omega_0^2\right)+\frac{\omega_0^4}{32(1-\nu)}\setlength\arraycolsep{1pt}
{}_1F_2\left(\{3/2\};\{3,3\};-\omega_0^2\right)
\] is determined from the boundary conditions. Note that the constant $\beta$ depends on the Poisson ratio $\nu$; as a result the function $Y(r)$ in fact depends on $\nu$ and we therefore denote it $Y(r;\nu)$ henceforth.

Finally, at $O(\epsilon^{3/2})$, we find that
\begin{eqnarray}
{\cal L}(\zeta_3)%\frac{\partial^2\zeta_3}{\partial t_0^2}-\frac{1}{r}\frac{\partial}{\partial r}\left(r\frac{\partial \zeta_3}{\partial r}\right)
&=&-2\frac{\partial^2}{\partial t_0\partial t_1}\left[A(t_1)e^{\ii\omega_0t_0}+\cc\right]J_0(\omega_0r)+\frac{\Pi}{2}\left[e^{\ii(\omega_0t_0+\varsigma t_1+\phi)}+\cc\right]\nonumber\\&&-\omega_0\Gamma J_0(\omega_0r) \left[\ii A(t_1) e^{\ii\omega_0 t_0}+\cc\right] \nonumber\\
&&-\frac{\omega_0}{r}\frac{\partial}{\partial r}\left[Y(r;\nu)J_1(\omega_0r)\right]\left[A^3e^{3\ii\omega_0t_0}+3|A|^2Ae^{\ii\omega_0t_0}+\cc\right].
\label{eqn:MS_PDE}
\end{eqnarray}

To make progress, we use the Fredholm Alternative Theorem \cite{Keener}. In the context of a PDE like \eqref{eqn:MS_PDE}, the Fredholm Alternative Theorem states that either \eqref{eqn:MS_PDE} has a solution \emph{or} the homogeneous adjoint problem has a solution that is \emph{not} orthogonal to the RHS of \eqref{eqn:MS_PDE}. The operator ${\cal L}(\cdot)$ on the LHS of \eqref{eqn:MS_PDE} is self-adjoint with respect to the inner product $\langle u,v\rangle=\int_0^{2\pi/\omega_0}\int_0^1 r\,\overline{u(r,t_0)} v(r,t_0)~\upd r~\upd t_0$. Multiplying the RHS of \eqref{eqn:MS_PDE} by the complex conjugate of the solution of the homogeneous adjoint problem, 
 $u(r,t_0)=J_0(\omega_0r)e^{\ii\omega_0t_0}$, and integrating, we  find that the condition for a solution of \eqref{eqn:MS_PDE} to exist is that
\begin{equation}
    0=-2\ii\kone \omega_0\dot{A}+\frac{\Pi\ktwo}{2}e^{\ii(\varsigma t_1+\phi)}-\ii\omega_0\Gamma\kone A - 3\kthree|A|^2A+\cc
    \label{eqn:MS_amplODE}
\end{equation}  where the orthogonality condition gives rise to several constants:
\begin{align}
    \kone&=\int_0^1r\bigl[J_0(\omega_0r)\bigr]^2~\upd r\approx 0.1348\nonumber\\
    \ktwo&=\int_0^1rJ_0(\omega_0r)~\upd r\approx 0.2159\label{eqn:MS-Ints}\\
    \kthree(\nu)&=\omega_0\int_0^1J_0(\omega_0r)\frac{\upd}{\upd r}\left[Y(r;\nu)J_1(\omega_0r)\right]~\upd r.\nonumber
\end{align} Note that, while the constants $\kone$ and $\ktwo$ are `universal' for this problem, $\kthree$ depends on the Poisson ratio $\nu$; in the simulations presented here, we take $\nu=0.3$ and note that $\kthree(0.3)\approx0.5243$.

\subsubsection{Amplitude response}

% Figure environment removed


The solution of \eqref{eqn:MS_amplODE} may be written
\begin{equation}
    A(t_1)=\frac{a}{2}e^{\ii\varsigma t_1}
\end{equation} (so that $A+\cc=a\cos\varsigma t_1$ with $a$ the oscillation amplitude).
\begin{equation}
\frac{\Pi^2\ktwo^2}{a^2}=\kone^2\omega_0^2\Gamma^2+\left(\frac{\kone(\Omega^2-\omega_0^2)}{\epsilon}-\tfrac{3}{4}\alpha_3a^2\right)^2.
    \label{eqn:MS-AmpResponse1}
\end{equation} Recalling  the various scalings with $\epsilon$ (i.e.~the true amplitude $a_1=a\epsilon^{1/2}$, $P=\epsilon^{3/2}\Pi$ and $\gamma=\epsilon\Gamma$) and introducing a rescaled forcing frequency $\bOmega=\Omega/\omega_0$ we immediately have
\begin{equation}
\frac{\ktwo^2}{\kone^2\omega_0^4}\frac{P^2}{a_1^2}=\frac{\gamma^2}{\omega_0^2}+\left(\bOmega^2-1-\tfrac{3}{4}\frac{\kthree}{\kone\omega_0^2}a_1^2\right)^2.
    \label{eqn:MS-AmpResponse2}
\end{equation} 



Equation \eqref{eqn:MS-AmpResponse2} gives a relationship between the oscillation amplitude, $a_1$, and the rescaled forcing frequency $\bOmega$; note that the form of \eqref{eqn:MS-AmpResponse2} has precisely the same structure as \eqref{eqn:ModelMS}, which was derived from the Duffing equation --- a nonlinear ODE. However, unlike the Duffing oscillator, the constants have been derived formally via an asymptotic analysis of the governing PDEs that is valid in the limit of small oscillation amplitudes. This approach means that no fitting of the constants is required to give excellent agreement between the numerically-determined response curves and the predictions of \eqref{eqn:MS-AmpResponse2} in the limit of small amplitude oscillations (see dashed curves in fig.~\ref{fig:DetailedResponse}). 

Another advantage of our approach is that it gives more detailed information about the effects of deformation; for example, the above calculation allows us to determine the  perturbation to the stress caused by vibration: the function $Y(r;\nu)$ is defined in \eqref{eqn:psieqn} in terms of the perturbation to the stress function $\psi_2(r,t)=\bigl[\psi(r,t)-r\bigr]$. The analytical prediction of $Y(r;\nu)$, given by eqn \eqref{eqn:Yr}, is compared with an estimate of this function obtained from our numerical results in fig.~\ref{fig:StressPert}. This shows reasonable, if not perfect, agreement between the two (independent) approaches.



\subsubsection{Jump-down point}

We note that to determine predictions for the jump-down point, \eqref{eqn:MS-AmpResponse2} is most conveniently viewed as a quadratic in $\bOmega^2$ (rather than a cubic in $a_1^2$) and is precisely of the form \eqref{eqn:DuffingQuartic} given in \S\ref{sec:Asymptotics} but now with
\begin{equation}
    b=1+\frac{3}{4}\frac{\alpha_3a_1^2}{\omega_0^2\alpha_1}
\end{equation} and
\begin{equation}
    C=\left(1+\frac{3}{4}\frac{\alpha_3a_1^2}{4\omega_0^2\alpha_1}\right)^2+\frac{\gamma^2}{\omega_0^2}-\frac{\alpha_2^2}{\alpha_1^2\omega_0^4}\frac{P^2}{a_1^2}.
\end{equation} As a result of this direct correspondence, we can immediately use the standard results given in \S\ref{sec:Asymptotics} to describe the jump-down point that is predicted by this multiple scales approach. In particular, we expect that jump-down should occur when
\begin{equation}
    a_1=\amax=\frac{\alpha_2}{\alpha_1\omega_0}\frac{P}{\gamma}\approx0.666 \frac{P}{\gamma}.
\end{equation}

% Figure environment removed

While the comparison of the shape of the response curve is generally favourable, we note that the agreement breaks down badly at the jump-down point, which is where the above calculation suggests that bi-stability should be lost. (This jump-down point may be expressed explicitly using \eqref{eqn:omegaMax}, but the discrepancy is clear even without this.) This discrepancy between the jump-down point predicted by the multiple scales approach and the numerical results motivates us to now consider another approach (Harmonic Balance). 





\subsection{Harmonic Balance}

Rather than  rederive equations using the method of Harmonic Balance, we proceed here by analogy with the results from the method of multiple scales, noting the differences between the predictions of the methods of Multiple Scales and Harmonic Balance that were discussed in \S\ref{sec:Asymptotics}. From this analogy (essentially adding a $\bOmega$ to the damping term of \eqref{eqn:MS-AmpResponse2}) we have

\begin{equation}
\frac{\ktwo^2}{\kone^2\omega_0^4}\frac{P^2}{a_1^2}=\frac{\gamma^2}{\omega_0^2}\bOmega^2+\left(\bOmega^2-1-\tfrac{3}{4}\frac{\kthree}{\kone\omega_0^2}a_1^2\right)^2.
    \label{eqn:HB_AmpResponse}
\end{equation}


This result is used in the plots of fig. \ref{fig:DetailedResponse} and shows very good agreement with the numerically determined frequency--amplitude response curves. We can also use the earlier results to determine the properties of the jump-down point. In particular, we have that the constants $b$ and $C$ from \eqref{eqn:DuffingQuartic} are
\begin{equation}
    b=1+\tfrac{3}{4}\frac{\kthree}{\kone\omega_0^2}a_1^2-\frac{\gamma^2}{2\omega_0^2},\quad C=\left(1+\tfrac{3}{4}\frac{\kthree}{\kone\omega_0^2}a_1^2\right)^2-\frac{\ktwo^2}{\kone^2\omega_0^4}\frac{P^2}{a_1^2}
\end{equation} and hence jump-down should occur when $a_1=\amax$ with
\begin{equation}
0=\tfrac{3}{4}\frac{\kthree}{\kone\omega_0^2}\amax^4+\left(1    -\frac{\gamma^2}{4\omega_0^2}\right)\amax^2-\frac{\ktwo^2}{\kone^2\omega_0^2\gamma^2}P^2
\end{equation} or
\begin{equation}    a_{\max}^2=\frac{2\kone \omega_0^2}{3\kthree}\left\{\frac{\gamma^2}{4\omega_0^2}-1+\left[\left(1-\frac{\gamma^2}{4\omega_0^2}\right)^2+\frac{3\ktwo^2\kthree P^2}{\kone^3\gamma^2\omega_0^4}\right]^{1/2}\right\}
    \label{eqn:HB_AmpJumpDown}
\end{equation}
where we have taken the positive square root to ensure real roots for $\amax$.

We also note that in our simulations we consistently have $\gamma\lesssim1/4$ so that $\gamma^2/(4\omega_0^2)\lesssim 3\times10^{-3}\ll1$. Hence, \eqref{eqn:HB_AmpJumpDown} can be simplified to give
\begin{equation}
    a_{\max}^2\approx\frac{2\kone \omega_0^2}{3\kthree}\left[\left(1+\frac{3\ktwo^2\kthree }{\kone^3\omega_0^4}\frac{P^2}{\gamma^2}\right)^{1/2}-1\right].
    \label{eqn:HB_AmpJumpDown_approx}
\end{equation} Crucially, \eqref{eqn:HB_AmpJumpDown_approx} shows that the maximum amplitude at jump-down is a function only of $P/\gamma$. Moreover, when $P/\gamma\ll1$, we have that $\amax\approx\ktwo P/(\kone\gamma\omega_0)$, which precisely matches the prediction of the Multiple Scales approach (as expected). However, for $P/\gamma=O(1)$, the two approaches differ and the jump-down amplitude $\amax$ grows sub-linearly with growing $P/\gamma$. Figure \ref{fig:Collapsenew} shows that the numerical results do indeed exhibit this `premature' jump-down (the points lie below the dashed line predicted by the MS analysis). This figure also shows that the prediction of \eqref{eqn:HB_AmpJumpDown_approx} (solid red curve) predicts a large portion of this premature jump-down but also that the numerical results exhibit premature jump-down even compared to the predictions of Harmonic Balance.


% Figure environment removed

The discrepancy with increasing loading can also be observed by comparing with the prediction for the frequency at which jump-down is observed, $\bOmega_{\mathrm{max}}$, which is given by
\begin{equation}
\bOmega_{\max}^2-1=\begin{cases}
    \frac{3\ktwo^2\kthree}{4\kone^3\omega_0^4}\frac{P^2}{\gamma^2}, & \chiMS=1\\
    \frac{1}{2}\left[\left(1+\frac{3\ktwo^2\kthree}{\kone^3\omega_0^4}\frac{P^2}{\gamma^2}\right)^{1/2}-1\right]+O(\gamma^2),& \chiMS=0,
\end{cases}   
\label{eqn:MaxFrequencyAsy}
\end{equation} (where we have used the approximation $\gamma^2/\omega_0^2\ll1$ to simplify the expression for the Harmonic Balance case). Again, we also note that when $P/\gamma\ll1$ the two expressions agree. This comparison is shown in fig.~\ref{fig:Collapsenew}b with the inset showing the relative error between the jump-down point observed numerically and that predicted by HB. Crucially, this suggests that the discrepancy becomes measurable once the detuning frequency differs by about $10\%$ from the resonant frequency.




    



% % Figure environment removed


% % Figure environment removed



\section{Conclusion}

In this paper, we have developed, and solved numerically, a model for the deflections of an elastic membrane subject to a spatially-uniform, oscillatory load and linear viscous damping. Our model accounts for the changes to the membrane stress that are caused by out-of-plane membrane deformations, which leads to a strain-stiffening effect. Our numerical results generates frequency-amplitude response curves that are qualitatively similar to those determined analytically from asymptotic analyses of the Duffing oscillator. However, we have shown that neither of the standard asymptotic approaches (Multiple Scales or Harmonic Balance) satisfactorily describes the jump-down point at which the amplitude suddenly drops. While this premature `jump-down' is ordinarily  interpreted as a signature of nonlinear damping \cite{LifshitzReview,davidovikj_nonlinear_2017}, our model only included a linear damping. This result suggests that the domain of validity of asymptotic results for the jump-down point does not extend as far as previously assumed. In particular, premature jump-down should not, automatically, be inferred as being a consequence of nonlinear damping.

Our own simulations of the Duffing equation itself, an ODE, show very good agreement with the jump-down point predicted by Harmonic Balance. This suggests that the premature jump-down observed in the coupled PDEs is a feature of the spatial variation, likely caused by higher modes of the membrane being excited (since this has no analogue in the ODE system). Understanding this further, and proposing an alternative expression for use in fitting, remains a topic for future work.




% The conclusion text goes here. Things we should include are:

% \begin{itemize}
%     \item The difference between HB and PM --- but that neither adequately describes our numerical experiments.
    
%     \item We could make the point that the jump-up point could be a better signature of nonlinearity (independent of damping) than the jump-down point. (This was already made by Brennan \emph{et al.}~\cite{Brennan2008}, but may be worth re-emphasizing.)
% \end{itemize}

\enlargethispage{20pt}

 \vskip6pt

We are grateful to Lincoln College, Oxford and IIT Hyderabad for funding, Prof.~Vyasarayani for discussions and Prof.~ Steeneken for providing his group's previous experimental data (fig.~\ref{fig:ExptData}) and comments on an earlier version of this manuscript.

 \appendix
 
 \section{Validity of modelling assumptions}
\label{sec:SimplifyingAssumptions}
 
 The axisymmetric dynamic oscillations of a sheet are subject to transverse force balance, which can be written as
\begin{equation}
\rho h \frac{\partial^2 \bZ}{\partial \bt^2}=\bp[\br,\bZ (\br,\bt),\bt]-B\nabla^4\bZ+\bm{\nabla}\cdot\left( \bm{\bSigma}\cdot\bm{\nabla}\bZ \right)
    \label{Eq:trans_force_balance}
\end{equation} and the in-plane equation of motion is
\begin{equation}
\rho h \frac{\partial^2 \bm{\bu}}{\partial \bt^2}=\frac{1}{\br}\frac{\partial\left(\br\bSigma_{\br \br}\right)}{\partial \br} - \frac{\bSigma_{\bar{\theta} \bar{\theta}}}{\br}.
    \label{Eq:in_plane_force_balance}
\end{equation}
With the non-dimensionalization used throughout the main text, \eqref{Eq:trans_force_balance} becomes
\begin{equation}
\frac{\partial^2 {\zeta}}{\partial {t}^2}={ p}({r},{t})-\bending \nabla^4{\zeta}+\bm{\nabla}\cdot\left( {\bm{\sigma}}\cdot\bm{\nabla}{\zeta} \right)
    \label{Eq:trans_force_balance_2}
\end{equation}
where
\begin{equation}
 \bending =\frac{B}{\Tpre \Rd^2}
    \label{Eq:epsilon}
\end{equation} is the reciprocal of the `bendability' \cite{Davidovitch2011} of the membrane.
Now considering \eqref{Eq:in_plane_force_balance}
\begin{equation}
 \frac{1}{{r}}\frac{\partial\left({r}{\sigma}_{r r}\right)}{\partial {r}} -\frac{{\sigma}_{\theta\theta}}{{r}}= \frac{\Tpre}{\ETwoD} \frac{\partial^2 {\bm{u}}}{\partial {t}^2}
\end{equation}
Thus, the time scale for evolution of the stress is
\begin{equation}
    \left(\frac{\Tpre}{\ETwoD} \frac{\rho h \Rd^2}{\Tpre}\right)^{1/2}=\frac{\Rd}{c},
    \label{eqn:SoundTime}
\end{equation}
where $c$ is the natural wave speed
\begin{equation}
c = \sqrt{\frac{\ETwoD}{\rho h}}.
\label{Eq:wave_speed}
\end{equation}

Our neglect of bending stiffness and in-plane inertia therefore require that both $\bending\ll1$ and $\Rd/(ct_\ast)\ll1$. By taking a typical Elastic Modulus value $E=400$~GPa, $h=3\times 10^{-10}\mathrm{~m}$, $\Rd=2.5\mathrm{~\mu m}$ (based on figure 1b of \cite{davidovikj_nonlinear_2017}), $\nu=0.3$ and $\Tpre\approx0.1\mathrm{~N\,m^{-1}}$  \cite{lee_measurement_2008}, we get $\bending\approx6\times 10^{-5}$ and $\Rd/(ct_\ast)\approx0.01$; this suggests that both bending and dynamic effects can be neglected.
 
 \section{The numerical scheme}
 
 \subsection{Description of scheme}
 
 \label{subsec:num_implementation}
We obtain the dynamics of the membrane by solving \eqref{Eq:VForceBal_ND}--\eqref{Eq:bcs_dimensionless}. 
To solve the PDE  \eqref{Eq:VForceBal_ND}  numerically  by using the method of lines, we first step discretize the spatial derivative and convert it to a set of  ordinary differential equations. We do this by introducing a uniform mesh for $r$ on the interval $[0,1]$. The grid points on the mesh are given by $r_i=i/N$, where $0\leq i\leq N$ . We denote  the numerical values of $\zeta$, $\psi$ and $\psi'$ at the grid point $r_i$ by $\zeta_{i}$, $\psi_i$ and $\psi_i'$, respectively. The spatial derivatives of $\zeta$ in \eqref{Eq:VForceBal_ND}, except at points $r=0$ and $r=1$, are approximated with second-order central differences. The spatial derivatives in $\zeta$ at $r=0$ and $r=1$ are approximated by second-order forward and backward differences, respectively. {(Note that $\psi'$ is calculated directly from our solution of the compatibility equation, and hence is not calculated via finite differences.)} Thus, for $i=1,2,\dots,N-1$, \eqref{Eq:VForceBal_ND} becomes:  
\begin{equation}
    \frac{d^2 {\zeta_i}}{dt^2}+  \gamma\frac{d {\zeta_i}}{dt}={p}[{r}_i,\zeta_i,t]+\frac{\psi'_{i}}{r_i}\frac{{\zeta}_{i+1}-{\zeta}_{i-1}}{2 \Delta {r}}+\frac{\psi_{i}}{r_i}\frac{{\zeta}_{i+1}-2{\zeta}_i+{\zeta}_{i-1}}{\Delta {r}^2},
    \label{Eq:trans_force_balance_no_bending_discrete}
\end{equation}
and for $i=0$ and $N$, it becomes:
\begin{align}
    \frac{d^2 {\zeta_0}}{dt^2}+  \gamma\frac{d {\zeta_0}}{dt}&={p}[{r}_0,\zeta_0,t]+\frac{\psi'_{0}}{r_0}\frac{-3\zeta_0+4\zeta_1-\zeta_2}{2 \Delta {r}}\nonumber\\
    &+\frac{\psi_{0}}{r_0}\frac{2\zeta_0-5\zeta_1+4\zeta_2-\zeta_3}{\Delta {r}^2},    \label{Eq:trans_force_balance_no_bending_discrete_left}\\
    \frac{d^2 {\zeta_N}}{dt^2}+  \gamma\frac{d {\zeta_N}}{dt}&={p}[{r}_N,\zeta_N,t]+\frac{\psi'_{N}}{r_N}\frac{3\zeta_N-4\zeta_{N-1}+\zeta_{N-2}}{2 \Delta {r}} \nonumber\\
    &+\frac{\psi_{N}}{r_N}\frac{2\zeta_N-5\zeta_{N-1}+4\zeta_{N-2}-\zeta_{N-3}}{\Delta {r}^2},
    \label{Eq:trans_force_balance_no_bending_discrete_right}
\end{align}
respectively.  The boundary conditions, \eqref{Eq:bcs_dimensionless}, become:
\begin{align}
    \zeta_0 &= \frac{4\zeta_1-\zeta_2}{3},   \quad &{\zeta}_{N}&=0     \label{Eq:bcs_dimensionless_zeta_dicrete},\\
    {\psi_0}&=0,  &{\psi}'_{N}-\nu \, {\psi}_N&=1-\nu.
    \label{Eq:bcs_dimensionless_psi_dicrete}
\end{align} 

The PDEs are thus cast as a set of ODEs. To write this set of ODEs in matrix form, we introduce the column vectors, $\bm{\zeta}$ and  $\bm{p}$, each of length $N-1$:
\begin{equation}
    \bm{\zeta} = \left({\zeta}_1,\dots,{\zeta}_{N-1}\right)^\mathrm{T},
\end{equation}
\begin{equation}
    \bm{p} = p\left(1,\dots,1\right)^\mathrm{T}.
\end{equation} The PDEs may then be written in the form
\begin{equation}
    \frac{d^2     \bm{\zeta}}{dt^2}=-\gamma \frac{d \bm{\zeta}}{dt}+\bm{p}+\left(\bm{S}_{rr}\bm{A}+\bm{S}_{\theta\theta}\bm{B}\right)\bm{\zeta}
    \label{Eq:system_of_equations}
\end{equation}  where the matrices $\bm{S}_{rr}$ and $\bm{S}_{\theta\theta}$ represent the effects of the in-plane stress and are defined as

%and matrices of size $(N-1) \times (N-1)$:
\begin{equation}
    \bm{S}_{rr} =  \begin{pmatrix}
\frac{\psi_1}{r_1}  &  &  \\
 &\ddots  &   \\
  &   & \frac{\psi_{N-1}}{r_{N-1}} \\
\end{pmatrix},  
\end{equation} and
\begin{equation}
    \bm{S_{\theta\theta}} =  \begin{pmatrix}
 \frac{\psi'_1}{r_1}  &  &  \\
 & \ddots  &     \\
& & \frac{\psi'_{N-1}}{r_{N-1}} \\
\end{pmatrix},  
\end{equation} while the matrices $\bm{A}$ and $\bm{B}$, defined by
\begin{equation}
    \bm{A} =\frac{1}{\Delta {r}^2}  \begin{pmatrix}
-2/3 & 2/3 &  &  & &  \\
1 & -2 & 1 &   &  &  \\
  & 1 & -2 & 1 &  &  \\
 &  &\ddots &\ddots &\ddots &   \\
 &  &  & 1 & -2 & 1  \\
  &  &  &  & 1 & -2 \\
\end{pmatrix},   
\end{equation}
\begin{equation}
    \bm{B} = \frac{1}{2 \Delta {r}} \begin{pmatrix}
-4/3 & 4/3 & 1 &   & &  & \\
-1 & 0 & 1 &   &  &  & \\
  & -1 & 0 & 1 &  &  & \\
 &  &\ddots &\ddots &\ddots &  &  \\
 &  &  & -1 & 0 & 1 & \\
  &  &  &  & -1 & 0 & 1\\
  &  &  &  &  & -1 & 0\\
\end{pmatrix},   
\end{equation} calculate the second and first derivatives of $\zeta$ with respect to $r$, respectively.

At each time step, the values of $\psi$ are calculated by solving \eqref{Eq:compatibility_dimensionless} with the boundary conditions \eqref{Eq:bcs_dimensionless_psi_dicrete}. The initial conditions are:
\begin{equation}
    \bm{\zeta}(0)=0, \quad\bm{\Dot{\zeta}}(0)=0. \label{Eq:initial_conditions} 
\end{equation} 
 
 We solve \eqref{Eq:system_of_equations} by numerically integrating it in python  by using \texttt{solve\_ivp} routine with the initial conditions \eqref{Eq:initial_conditions}. We solve for $\bm{\psi}$  by using \texttt{solve\_bvp} routine for \eqref{Eq:compatibility_dimensionless} with the boundary conditions \eqref{Eq:bcs_dimensionless_psi_dicrete}. To obtain the full response curve we do  forward and backward sweeps. The response branch with larger amplitudes and the jump-down point are found from the forward sweep. Similarly, the lower branch and the jump up point are from the backward sweep.
 

 
 \subsection{Convergence study}
  \label{subsec:convergence}
% Figure environment removed
  
 We choose $N=50$ for solving  \eqref{Eq:system_of_equations}--\eqref{Eq:initial_conditions} and to obtain the numerical results presented in the main text of this work. Numerical error resulting from a coarse discretization can be minimized by taking a large $N$ but at the cost of computational time. The chosen value of  $N=50$ ensures reasonable computational time without compromising the accuracy of the results. This is validated by a convergence study in which the value of $N$ is varied --- see fig.~\ref{fig:convergence}.  Figure \ref{fig:convergence}a depicts response curves (circular markers)  at different $N$ for $\gamma = 0.1$. As expected, the circles overlap (away from the jump-down point) and approach analytical predictions (solid curves) as $N$ increases.  

Our primary interest here is in the jump-down point on the response curve. We therefore plot $\frac{a_\mathrm{fall}}{a_\mathrm{max}}$ as a function of $N$ in fig.~\ref{fig:convergence}b. We observe that as $N$ increases, $\frac{a_\mathrm{fall}}{a_\mathrm{max}}$ decreases linearly. Extrapolating $\frac{a_\mathrm{fall}}{a_\mathrm{max}}$ to $N=\infty$ ($\frac{1}{N}=0$) indicates that the numerical results at large $N$ will have a slightly smaller $a_{\mathrm{fall}}$ value than the ones reported with $N=50$. Crucially, increasing $N$ does not avert our observation of premature jump-down.







\vskip2pc

\begin{thebibliography}{99}

\bibitem{khan2017mechanical}
Khan ZH, Kermany AR, {\"O}chsner A, Iacopi F. 2017  Mechanical and
  electromechanical properties of graphene and their potential application in
  MEMS. {\em Journal of Physics D: Applied Physics} \textbf{50}, 053003.

\bibitem{carvalho2022review}
Carvalho AF, Kulyk B, Fernandes AJ, Fortunato E, Costa FM. 2022  A review on
  the applications of graphene in mechanical transduction. {\em Advanced
  Materials} \textbf{34}, 2101326.

\bibitem{Guinea2010}
Guinea F, Katsnelson MI, Geim AK. 2010  Energy gaps and a zero-field quantum
  Hall effect in graphene by strain engineering. {\em Nat. Phys.} \textbf{6},
  30--33.

\bibitem{Akinwande2017}
Akinwande D, Brennan CJ, Bunch JS, Egberts P, Felts JR, Gao H, Huang R, Kim JS,
  Li T, Li Y, Liechti KM, Lu N, Park HS, Reed EJ, Wang P, Yakobson BI, Zhang T,
  Zhang YW, Zhou Y, Zhu Y. 2017  A review on mechanics and mechanical
  properties of 2D materials—Graphene and beyond. {\em Extr. Mech. Lett.}
  \textbf{13}, 42--77.

\bibitem{Al-Quraishi.2020}
Al-Quraishi KK, He Q, Kauppila W, Wang M, Yang Y. 2020  {Mechanical testing of
  two-dimensional materials: a brief review}. {\em International Journal of
  Smart and Nano Materials} \textbf{11}, 207--246.

\bibitem{castellanos2015mechanics}
Castellanos-Gomez A, Singh V, van~der Zant HS, Steele GA. 2015  Mechanics of
  freely-suspended ultrathin layered materials. {\em Annalen der Physik}
  \textbf{527}, 27--44.

\bibitem{davidovikj_nonlinear_2017}
Davidovikj D, Alijani F, Cartamil-Bueno SJ, van~der Zant HSJ, Amabili M,
  Steeneken PG. 2017  Nonlinear dynamic characterization of two-dimensional
  materials. {\em Nature Communications} \textbf{8}, 1253.


\bibitem{Kaisar2022}
Kaisar T, Lee J, Li D, Shaw SW, Feng PXL. 2022  {Nonlinear Stiffness and
  Nonlinear Damping in Atomically Thin MoS2 Nanomechanical Resonators}. {\em
  Nano Letters} \textbf{22}, 9831--9838.


\bibitem{Jensen2008}
Jensen K, Kim K, Zettl A. 2008  An atomic-resolution nanomechanical mass
  sensor. {\em Nat. Nano.} \textbf{3}, 533--537.

\bibitem{nayfeh_nonlinear_2008}
Nayfeh AH, Mook DT. 2008 {\em Nonlinear {Oscillations}}.
John Wiley \& Sons.

\bibitem{LifshitzReview}
Lifshitz R, Cross MC. 2010 pp. 1--52.
In {\em Nonlinear Dynamics of Nanomechanical and Micromechanical Resonators},
  pp. 1--52. John Wiley \& Sons.

\bibitem{Eichler.2011}
Eichler A, Moser J, Chaste J, Zdrojek M, Wilson-Rae I, Bachtold A. 2011
  {Nonlinear damping in mechanical resonators made from carbon nanotubes and
  graphene}. {\em Nature Nanotechnology} \textbf{6}, 339--342.


\bibitem{Farokhi.2023}
Farokhi H, Rocha RT, Hajjaj AZ, Younis MI. 2023  {Nonlinear damping in
  micromachined bridge resonators}. {\em Nonlinear Dynamics} \textbf{111},
  2311--2325.


\bibitem{Brennan2008}
Brennan M, Kovacic I, Carrella A, Waters T. 2008  On the jump-up and jump-down
  frequencies of the {D}uffing oscillator. {\em J. Sound Vib.} \textbf{318},
  1250--1261.

\bibitem{Jordan2007}
Jordan DW, Smith P. 2007 {\em Nonlinear Ordinary Differential Equations}.
Oxford University Press.

\bibitem{Wawrzynski2021}
Wawrzynski W. 2021  Duffing‐type oscillator under harmonic excitation with a
  variable value of excitation amplitude and time‐dependent external
  disturbances. {\em Sci. Rep.} \textbf{11}, 2889.

\bibitem{ramlan2016exploiting}
Ramlan R, Brennan MJ, Kovacic I, Mace BR, Burrow SG. 2016  Exploiting knowledge
  of jump-up and jump-down frequencies to determine the parameters of a Duffing
  oscillator. {\em Communications in Nonlinear Science and Numerical
  Simulation} \textbf{37}, 282--291.

\bibitem{Chopin2008}
Chopin J, Vella D, Boudaoud A. 2008  The liquid blister test. {\em Proc. R.
  Soc. A} \textbf{464}, 2887--2906.

\bibitem{NISThandbook}
Olver FWJ, Lozier DW, Boziert RF, Clark CW. 2010 {\em NIST Handbook of
  Mathematical Functions}.
Cambridge University Press.

\bibitem{Keener}
Keener JP. 1988 {\em Principles of applied mathematics}.
Addison-Wesley.

\bibitem{Davidovitch2011}
Davidovitch B, Schroll RD, Vella D, Adda-Bedia M, Cerda E. 2011  A prototypical
  model for tensional wrinkling in thin sheets. {\em Proc. Natl. Acad. Sci.}
  \textbf{108}, 18227--18232.

\bibitem{lee_measurement_2008}
Lee C, Wei X, Kysar JW, Hone J. 2008  Measurement of the {Elastic} {Properties}
  and {Intrinsic} {Strength} of {Monolayer} {Graphene}. {\em Science}
  \textbf{321}, 385--388.


\end{thebibliography}

%
%\bibliographystyle{RS}
%\bibliography{refs}

\end{document}