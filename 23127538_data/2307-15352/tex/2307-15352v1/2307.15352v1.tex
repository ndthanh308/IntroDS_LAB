\documentclass[11pt]{amsart}
\usepackage{fullpage}
\usepackage{amsfonts,amssymb,amsmath,amsthm}
\usepackage{hyperref}



\theoremstyle{plain}
\newtheorem{theorem}{Theorem}[section]
 \newtheorem{corollary}[theorem]{Corollary}
 \newtheorem{lemma}[theorem]{Lemma}
 \newtheorem{proposition}[theorem]{Proposition}
 \theoremstyle{definition}
 \newtheorem{definition}[theorem]{Definition}
 \newtheorem{setting}[theorem]{Setting}
 \theoremstyle{remark}
 \newtheorem{remark}[theorem]{Remark}
 \numberwithin{equation}{section}
  \newtheorem{example}[theorem]{Example}
    \newtheorem{hypothesis}[theorem]{Hypothesis}

\usepackage{mathrsfs}
\def \sL{\mathscr L}
\def \sF {\mathscr F}

\def \bA {\mathbb A}
\def \bB {\mathbb B}
\def \bC {\mathbb C}
\def \bD {\mathbb D}
\def \bE {\mathbb E}
\def \bF {\mathbb F}
\def \bG {\mathbb G}
\def \bH {\mathbb H}
\def \bI {\mathbb I}
\def \bJ {\mathbb J}
\def \bK {\mathbb K}
\def \bL {\mathbb L}
\def \bM {\mathbb M}
\def \bN {\mathbb N}
\def \bO {\mathbb O}
%\def \bP {\mathbb P}
\def \bQ {\mathbb Q}
\def \bR {\mathbb R}
\def \bS {\mathbb S}
\def \bT {\mathbb T}
\def \bU {\mathbb U}
\def \bV {\mathbb V}
\def \bW {\mathbb W}
\def \bX {\mathbb X}
\def \bY {\mathbb Y}
\def \bZ {\mathbb Z}

\def \cA {\mathcal A}
\def \cB {\mathcal B}
\def \cC {\mathcal C}
\def \cD {\mathcal D}
\def \cE {\mathcal E}
\def \cF {\mathcal F}
\def \cG {\mathcal G}
\def \cH {\mathcal H}
\def \cI {\mathcal I}
\def \cJ {\mathcal J}
\def \cK {\mathcal K}
\def \cL {\mathcal L}
\def \cM {\mathcal M}
\def \cN {\mathcal N}
\def \cO {\mathcal O}
\def \cP {\mathcal P}
\def \cQ {\mathcal Q}
\def \cR {\mathcal R}
\def \cS {\mathcal S}
\def \cR {\mathcal R}
\def \cR {\mathcal R}
\def \cT {\mathcal T}
\def \cU {\mathcal U}
\def \cV {\mathcal V}
\def \cW {\mathcal W}
\def \cX {\mathcal X}
\def \cY {\mathcal Y}
\def \cZ {\mathcal Z}

\def \fa {\mathfrak a}
\def \fb {\mathfrak b}
\def \fc {\mathfrak c}
\def \fd {\mathfrak d}
\def \fe {\mathfrak e}
\def \ff {\mathfrak f}
\def \fg {\mathfrak g}
\def \fh {\mathfrak h}
\def \mfi {\mathfrak i}
\def \fj {\mathfrak j}
\def \fk {\mathfrak k}
\def \fl {\mathfrak l}
\def \fm {\mathfrak m}
\def \fn {\mathfrak n}
\def \fo {\mathfrak o}
\def \fp {\mathfrak p}
\def \fq {\mathfrak q}
\def \fr {\mathfrak r}
\def \fs {\mathfrak s}
\def \ft {\mathfrak t}
\def \fu {\mathfrak u}
\def \fv {\mathfrak v}
\def \fw {\mathfrak w}
\def \fx {\mathfrak x}
\def \fy {\mathfrak y}
\def \fz {\mathfrak z}

\def \fA {\mathfrak A}
\def \fB {\mathfrak B}
\def \fC {\mathfrak C}
\def \fD {\mathfrak D}
\def \fE {\mathfrak E}
\def \fF {\mathfrak F}
\def \fG {\mathfrak G}
\def \fH {\mathfrak H}
\def \fI {\mathfrak I}
\def \fJ {\mathfrak J}
\def \fK {\mathfrak K}
\def \fL {\mathfrak L}
\def \fM {\mathfrak M}
\def \fN {\mathfrak N}
\def \fO {\mathfrak O}
\def \fP {\mathfrak P}
\def \fQ {\mathfrak Q}
\def \fR {\mathfrak R}
\def \fS {\mathfrak S}
\def \fR {\mathfrak R}
\def \fR {\mathfrak R}
\def \fT {\mathfrak T}
\def \fU {\mathfrak U}
\def \fV {\mathfrak V}
\def \fW {\mathfrak W}
\def \fX {\mathfrak X}
\def \fY {\mathfrak Y}
\def \fZ {\mathfrak Z}



\def\eps{\varepsilon}
\def\R{{\mathbb R}}% real numbers
\def\C{{\mathbb C}}% complex numbers
\def\N{{\mathbb N}}% nonnegative integers
\def\Z{{\mathbf Z}}% integers
\def\T{{\mathbf T}}% torus
\mathchardef\mhyphen="2D % math hypehn
\def \ad {{\rm ad}}
\def \Ghat{\widehat{G}}
\def \Hhat{\widehat{H}}
\def \tr {{\rm tr}}
\def \Op  {{\rm Op}}
\def \id {{\rm id}}

\def\Diff{{\rm Diff }}

\def \pr {{\rm pr}} % for the projections

\def\Tend#1#2{\mathop{\longrightarrow}\limits_{#1\rightarrow#2}}

\numberwithin{equation}{section}

\begin{document}
\title[]{Quantization on Groups and G\aa rding inequality}
\author[L. Benedetto]{Lino Benedetto}
\address[L. Benedetto]{DMA, École normale supérieure, Université PSL, CNRS, 75005 Paris, France \& Univ Angers, CNRS, LAREMA, SFR MATHSTIC, F-49000 Angers, France} 
\email{lbenedetto@dma.ens.fr}
\author[C. Fermanian]{Clotilde~Fermanian~Kammerer}
\address[C. Fermanian Kammerer]{
Univ Paris Est Creteil, Univ Gustave Eiffel, CNRS, LAMA UMR8050, F-94010 Creteil, France \& Univ Angers, CNRS, LAREMA, SFR MATHSTIC, F-49000 Angers, France
}
\email{clotilde.fermanian@u-pec.fr}
\author[V. Fischer]{V\'eronique Fischer}\address[V. Fischer]
{University of Bath, Department of Mathematical Sciences, Bath, BA2 7AY, UK} 
\email{v.c.m.fischer@bath.ac.uk}


\begin{abstract} 
In this paper, we introduce Wick's quantization on groups and discuss its links with Kohn-Nirenberg's. By  quantization, we mean     an operation that associates an operator to a symbol. 
The notion of symbols for both quantizations  is based on representation theory via the group Fourier transform  and the  Plancherel theorem. 
As an application, we give a simple proof of G\aa rding inequalities for three globally symbolic pseudo-differential calculi on groups. 
 \end{abstract}

%\subjclass[2010]{43A80, 47G30, 58J40}
\keywords{Abstract harmonic analysis, 
Pseudo-differential calculus on compact and nilpotent Lie groups, G\aa rding inequality.  
}

\maketitle

\makeatletter
\renewcommand\l@subsection{\@tocline{2}{0pt}{3pc}{5pc}{}}
\makeatother

\tableofcontents


\section{Introduction}

On $\bR^n$, the links between the Kohn-Nirenberg and Wick quantizations provide a simple proof of G\aa rding inequalities; this is briefly sketched in  Appendix \ref{app} for the H\"ormander calculus on $\bR^n$
while a reference for the semi-classical is for instance Jean-Marc Bouclet's lecture notes \cite{bouclet}, see also~\cite{Zwobook,Lerner}.
In this  paper, we show how these ideas extend readily to  groups. 

We start by explaining that, as for the Kohn-Nirenberg quantization, the definition of the  Wick quantization extend naturally to groups
that  satisfy some hypotheses allowing for the definition of the group Fourier transform (based on representation theory) and   the associated Plancherel theorem. 
Using this observation, 
we then prove  G\aa rding inequalities for three globally symbolic pseudo-differential calculi on compact and graded nilpotent Lie groups. 
This topic, that is, 
G\aa rding inequalities for global pseudo-differential calculi on groups, has been the subject of many papers in recent years, see e.g. \cite{FRCras,RTGarding,CDR,CFR}.


Our results regarding G\aa rding inequalities are summarised in the three following theorems,
although their statement will use the notation for the settings and the calculi recalled later on in the paper.   
Our approach provides a simple proof of strong G\aa rding inequalities, that is, with gain of one derivative, in a context where the symbols are not scalar-valued but operator-valued.
Moreover, 
these G\aa rding inequalities are sharp in the sense that the lower bound involves only one term (here, given with gain of one derivative).


The first inequality is set on compact Lie groups and considers the symbolic pseudo-differential calculus proposed in \cite{RT,RTW},
studied in \cite{FJFA2015} and  briefly recalled in Section \ref{subsec_pseudoC_compact}. We will prove the $(\rho,\delta)$-generalisation of the following G\aa rding inequality:

\begin{theorem}
\label{thm_Gardingcomp}
Let $G$ be a connected compact Lie group. 
Let $m\in \bR$.
	Assume that the symbol $\sigma\in S^{2m+1}(G)$ satisfies the elliptic condition $\sigma \geq c (\id  +\widehat \cL)^{\frac {2m+1}2}$ for some constant $c>0$.
	Then there exists a constant $C>0$ such that 
	$$
	\forall f\in \cC^\infty(G)\qquad 
	\Re \left (\Op^{\rm KN}(\sigma) f, f\right )_{L^2(G)} \geq -C \|f\|_{H^{m}(G)}^2 .
$$	 
\end{theorem}
Above, the spaces $H^m(G)$ denotes the usual Sobolev spaces defined on any compact manifold, here $G$, 
while the definitions of  the symbol classes $S^m(G)$ and  the Laplace-Beltrami operator $\cL$ as well as its Fourier transform are recalled in Section \ref{subsec_pseudoC_compact}.

\smallskip

The next result concerns  the symbolic pseudo-differential calculus on a graded Lie group $G$ \cite{FR,FF0}, briefly recalled in Section \ref{subsec_pseudoC_graded}.
The Sobolev space $L^2_s(G)$ will be the ones adapted to this setting \cite{FR,FRSob}. We will prove the $(\rho,\delta)$-generalisation of the following G\aa rding inequality:
\begin{theorem}
\label{thm_Garding_graded}
Let $G$ be a graded nilpotent Lie group. 
Let $m\in \bR$.
	Assume that  $\sigma\in S^{2m+1}(G)$ satisfies the elliptic condition $\sigma \geq c (\id  +\widehat \cR)^{\frac {2m+1}\nu}$
	for some constant $c>0$ where $\cR$ is a positive Rockland operator of homogeneous degree $\nu$.
	Then there exists a constant $C>0$ such that 
	$$
	\forall f\in \cC^\infty_c(G)\qquad 
	\Re \left (\Op^{\rm KN}(\sigma) f, f\right )_{L^2(G)} \geq -C \|f\|_{L^2_{m}(G)}^2 .
$$	 
\end{theorem}

Still in the context of graded Lie groups, 
our method is particularly adapted to  the semi-classical counter-part of Theorem \ref{thm_Garding_graded}:
 \begin{theorem}\label{thm:garding_sc}
 Let $\sigma\in \cA_0$, that is, the symbol $\sigma$ is a smoothing symbol with $x$-compact support. 
If $\sigma$ is non-negative, then  there exists a constant $C>0$ such that 
 $$
 \forall f\in L^2(G), \quad \forall \eps\in (0,1]\qquad 
 \Re \left(\Op^{\rm KN}_\eps(\sigma) f, f\right) _{L^2(G)}\geq  -C\eps \| f\|^2_{L^2(G)}.
 $$
 \end{theorem}
 The semi-classical calculus in this setting \cite{FF1,FF2}
is recalled in Section \ref{subsec_semiclC}, 
as well as the definitions for $\cA_0$ and $\Op^{\rm KN}_\eps$.

\medskip

On a compact Lie group $G$, the pseudo-differential calculus  mentioned above with $\rho>\delta$ and $\rho\geq 1-\delta$ coincides with H\"ormander's pseudo-differential calculus defined via charts on the compact Lie group $G$ viewed as a compact manifold \cite{FJFA2015,RTW}.
However, the notion of symbols are not the same in these two calculi: the one presented here or in \cite{FJFA2015,RT,RTW}
is  global and  based on the representations of the group.
For a graded nilpotent Lie group $G$, 
the pseudo-differential calculus mentioned above coincides 
with the (global) H\"ormander calculus only when 
$G$ is abelian, that is, only when $G$ is the abelian group $(\bR^n,+)$ with $n=\dim G$.
Otherwise, although a graded nilpotent Lie group is globally diffeomorphic to $\bR^n$ as a manifold, 
the calculi will not be comparable.
At this point, we ought to  clarify what we mean by pseudo-differential calculus on a  smooth manifold $M$ 
in this paper:

\begin{definition}
\label{def_pseudo-diff_calculus}
For each $m\in \bR$, 
let $\Psi^m(M)$  be a given Fr\'echet space of continuous operators $\cD(M)\to\cD(M)$.
We say that the space $\Psi^\infty(M):=\cup_m \Psi^m(M)$ form a \emph{pseudo-differential calculus} 
when it is an algebra of operators satisfying: 
\begin{enumerate}
\item 
\label{item_def_pseudo-diff_calculus_inclusion}
The continuous inclusions
$\Psi^m (M)\subset \Psi^{m'}(M)$ hold for any $m\leq m'$.
\item
\label{item_def_pseudo-diff_calculus_product}
$\Psi^\infty(M)$ is an algebra of operators.
Furthermore if $T_1\in \Psi^{m_1}(M)$, $T_2\in \Psi^{m_2}(M)$, 
then $T_1T_2\in \Psi^{m_1+m_2}(M)$,
and the composition is continuous as a map
$\Psi^{m_1}(M)\times \Psi^{m_2}(M)\to \Psi^{m_1+m_2}(M)$.
\item
\label{item_def_pseudo-diff_calculus_adjoint}
$\Psi^\infty(M)$ is stable under taking the adjoint.
Furthermore if $T\in \Psi^{m}(M)$  then $T^*\in \Psi^{m}(M)$,
and taking the adjoint is continuous as a map
$\Psi^{m}(M)\to \Psi^{m}(M)$.
 \end{enumerate}
 \end{definition}
 

The paper is organised as follows. We start with recalling the definition of the Kohn-Nirenberg quantization on groups and introducing Wick's   (Section \ref{sec_quantization}).
Then we show that a link between these two quantizations in the symbolic calculi provides a proof of 
G\aa rding inequalities in the cases of compact Lie groups $G$ 
(Section \ref{sec_Gcomp}), and of graded nilpotent Lie groups $G$ (Section \ref{sec_Gnilp}).
In the latter case, we also study the semi-classical analogue in Section \ref{sec_SC_nilpG}. 
In the Appendix, we develop the same strategy of proof in the Euclidean case; to our knowledge, the  proof which is the closest to ours is Folland's one~\cite[Chapter 2, Section 6]{folland}.

\medskip

\noindent{\bf Acknowledgements}. Clotilde Fermanian-Kammerer thanks the Erwin Schr\"odinger Institut for its hospitality when writing this paper. Clotilde Fermanian-Kammerer and Veronique Fischer gladly acknowledge the support of  The Leverhulme Trust via Research Project Grant RPG 2020-037. 


\section{Quantizations on groups}
\label{sec_quantization}

In this section, we discuss two quantizations of operators on groups that are based on the group Fourier transform  and the associated Plancherel theorem. These latter notions require some hypotheses on the group we now list. The group  $G$ is a separable  locally compact group.
We assume that it is unimodular, that is, its left (resp. right) Haar measures are also right (resp. left) invariant. 
We also assume that it is of type I.
The paper may be read without understanding these technical hypotheses. It suffices to know that  they ensure that the Plancherel theorem  holds, and that 
 they  are naturally satisfied on Lie groups that are compact or nilpotent. 
 
\subsection{Fourier analysis}

\subsubsection{The dual set}
Recall that a (unitary) \emph{representation} $(\mathcal H_\pi, \, \pi)$ of $G$ is a pair consisting in a Hilbert space~$\mathcal H_\pi$  and a group morphism~$\pi$ from~$G$ to the set of unitary transforms on $L^2(\mathcal H_\pi)$.
In this paper, the representations will always be assumed (unitary) strongly continuous, and their Hilbert spaces separable. 
A representation is said to be {\it irreducible} if the only closed subspaces of $\mathcal H_\pi$ that are stable under~$\pi$ are $\{0\}$ and $\mathcal H_\pi$ itself. 
Two representations $\pi_1$ and $\pi_2$ are equivalent  if there exists a unitary transform $\mathbb U$ called an {\it intertwining map} that sends $\mathcal H_{\pi_1}$ on $\mathcal H_{\pi_2}$ with 
$$\pi_1=\mathbb U^{-1}\circ  \pi_2 \circ \mathbb U.$$ 
The {\it dual set} $\widehat G$ is obtained by taking the quotient of the set of irreducible representations by this equivalence relation.  We may still denote by $\pi$ the elements of $\widehat G$ and we keep in mind that different representations of the class are equivalent through intertwining operators. 


\subsubsection{Fixing a Haar measure}
We fix a Haar measure that we denote $dx$ when the variable of integration is $x\in G$, or $dy$ if the variable is $y$.  

The non-commutative convolution is given via
$$
 (f_1*f_2)(x):=\int_G f_1(y) f_2(y^{-1}x) dy,\quad x\in G.
$$
for $f_1,f_2\in {\mathcal C}_c(G)$; here 
${\mathcal C}_c (G)$ denotes the space of continuous complex-valued functions on $G$ with compact support.


\subsubsection{The Fourier transform}
\label{subsubsecFG}
The {\it Fourier transform} of an integrable function $f\in L^1(G)$ at a representation $\pi$ of $G$ is the operator acting on $\mathcal H_\pi$
 via 
 $$
 \widehat f(\pi):=
 \cF(f)(\pi) :=\int_G  f(z)\, (\pi(z))^*\, dz.
 $$
 Note that if $f_1,f_2\in \cC_c (G)$ then 
 \begin{equation}
 \label{eq_cFf1*f2}
 \widehat {f_1 * f_2} = \widehat f_2 \widehat f_1.	
 \end{equation}
 
 If $\pi_1,\pi_2$ are two equivalent representations of $G$ with $\pi_1=\mathbb U^{-1}\circ  \pi_2 \circ \mathbb U$ for some intertwining operator $\mathbb U$, then 
$$\mathcal F(f)(\pi_1) =\mathbb U^{-1}\circ \mathcal F(f)(\pi_2)\circ \mathbb U .
$$
Hence, this defines the measurable field of operators $\{{\mathcal F}(f)(\pi), \pi\in \Ghat \}$ modulo equivalence. 
Here, the unitary dual $\Ghat$ is equipped with its natural Borel structure, 
and the equivalence comes from quotienting the set of irreducible representations of $G$ together with understanding the resulting fields of operators modulo intertwiners.

\subsubsection{The Plancherel Theorem} 

Here, we recall the Plancherel Theorem due to Dixmier \cite[Ch. 18]{dixmier}. Among other results, 
it states the existence and uniqueness of the {\it Plancherel measure}, that is, the positive Borel measure $\mu$ 
on $\widehat G$ such that 
 the Plancherel formula
\begin{equation}
\label{eq_plancherel_formula}
\int_G |f(x)|^2 dx = \int_{\widehat G} \|\widehat f(\pi)\|_{HS(\mathcal H_\pi)}^2 d\mu(\pi),
\end{equation}
holds for any $f\in {\mathcal C}_c(G) $.
Here $\|\cdot\|_{HS(\mathcal H_\pi)}$ denotes the Hilbert-Schmidt norm on $\mathcal H_\pi$.  
This implies that the group Fourier transform 
is a unitary map from $L^1(G)\cap L^2(G)$ equipped with the norm of $L^2(G)$ to the Hilbert space
$$
L^2(\widehat G):=\int_{\widehat G} \mathcal H_\pi \otimes\mathcal H_\pi^*\, d\mu(\pi).
$$
We identify $L^2(\Ghat)$ with the space of $\mu$-square integrable Hilbert-Schmidt fields on $\widehat G$;  its Hilbert norm and scalar products are then given by 
\begin{align*}
	\|\tau\|_{L^2(\Ghat)}^2 &= \int_{\Ghat} \|\tau (\pi)\|_{HS(\mathcal H_\pi)}^2 d\mu(\pi), \quad \tau\in L^2(\Ghat), \\
(\tau_1,\tau_2)_{L^2(\Ghat)} 
&= \int_{\Ghat} \tr_{\cH_\pi} (\tau_1 (\pi)\ \tau_2(\pi)^*) d\mu(\pi), \quad \tau_1,\tau_2\in L^2(\Ghat).
\end{align*}
Here $\tr_{\cH_\pi}$ denotes the trace of operators on the Hilbert space $\cH_\pi$.
The group Fourier transform $\cF$ 
extends unitarily from 
 $L^2(G)$ onto $L^2(\widehat G)$.


We denote by $L^\infty(\Ghat)$ the space of measurable fields (modulo equivalence) of bounded operators
$\sigma = \{\sigma(\pi)\in \sL(\cH_\pi) 
: \pi\in \Ghat\}$ on $\Ghat$ such that 
$$
\|\sigma\|_{L^\infty(\Ghat)} := \sup_{\pi\in \Ghat} \|\sigma(\pi)\|_{\sL(\cH_\pi) }
$$
is finite; here the supremum refers to the essential supremum with respect to the Plancherel measure $\mu$ of $\Ghat$. 
In fact, $L^\infty(\Ghat)$ is naturally a Banach space and moreover a von Neumann algebra, sometimes called the von Neumann algebra of the group $G$. 
It acts naturally on $L^2(\Ghat)$ by composition on the left: 
$$
(\sigma \tau) (\pi) = \sigma(\pi) \ \tau(\pi),\quad \pi\in \Ghat, \ \sigma\in L^\infty(\Ghat) \ \mbox{and}\ \tau \in L^2(\Ghat),
$$
(it also acts on the right)
and this action is continuous
$$
\|\sigma\tau\|_{L^2(\Ghat)}\leq 
\|\sigma\|_{L^\infty(\Ghat)}
\|\tau\|_{L^2(\Ghat)}.
$$
Dixmier's Plancherel theorem implies that 
$L^\infty(\Ghat) $ is isomorphic to the von Neumann algebra $\sL(L^2(G))^G$ of linear bounded operators on $G$ that are invariant under left translations. 
The isomorphism is given by the fact that the 
the Fourier multiplier with symbol $\sigma$, i.e. the operator $f\mapsto \cF^{-1} (\sigma \widehat f)$, is an operator  in  $\sL(L^2(G))^G$.


Note that $\cF L^1(G)\subseteq L^\infty(\Ghat)$ with 
$$
\forall f\in L^1(G)\qquad 
\|\widehat f\|_{L^\infty(\Ghat)} \leq \|f\|_{L^1(G)}.
$$

\subsection{The  Kohn-Nirenberg quantization}

In this section, we recall some results related to  the symbolic quantization on groups introduced by Michael Taylor \cite{Taylor}.
When $G$ is the abelian group $\bR^n$, this is the quantization often used in the field of Partial Differential Equations and called the Kohn-Nirenberg quantization or classical quantization~\cite{ho,AG}.
 We keep this vocabulary in the group case. 

\subsubsection{The space $L^2(G\times \Ghat)$}
We may identify the tensor product $L^2(G)\otimes L^2(\Ghat)$ with the space denoted by $L^2(G\times \Ghat)$ of measurable fields $\tau=\{\tau(x,\pi) \in HS(\cH_\pi) \ : \ (x,\pi)\in G\times \Ghat\}$ of Hilbert-Schmidt operators (up to equivalence) such that the quantities 
$$
\|\tau\|_{L^2(G\times \Ghat)}^2
:=
\int_{G\times\Ghat} \|\tau(x,\pi)\|_{HS(\cH_\pi)}^2 dxd\mu(\pi)
$$
are finite. It is naturally a separable Hilbert space with norm $\|\cdot\|_{L^2(G\times \Ghat)}$ and scalar product given by
$$
(\tau_1,\tau_2)_{L^2(G\times \Ghat)}
=
\int_{G\times\Ghat} \tr_{\cH_\pi} \left( \tau_1(x,\pi) \ \tau_2(x,\pi)^*\right)  dxd\mu(\pi),
\quad \sigma_1,\sigma_2\in L^2(G\times \Ghat).
$$



By the Plancherel theorem, the Hilbert space $L^2(G\times \Ghat)$ and $L^2(G\times G)$ are isomorphic via the Fourier transform in the second variable:
$$
L^2(G\times G)\longrightarrow
L^2(G\times \Ghat) ,\qquad 
\kappa 
 \longmapsto (\id\otimes \cF) \kappa.
$$
In other words, 
any  $\tau\in L^2(G\times \Ghat)$ may be written as 
$$
\tau (x,\pi) =\widehat \kappa_{\tau,x}(\pi) 
$$ for a unique function $\kappa_\tau:(x,y)\mapsto \kappa_{\tau,x}(y) = \kappa(x,y)$   in $L^2(G\times G)$. 

\subsubsection{The quantization $\Op^{\rm KN}$ on $L^2(G\times \Ghat)$}
\label{subsubsec_OPKN}
For any $f\in \cC_c(G)$, the symbol
$$
\tau \widehat f := \{\tau(x,\pi) \pi(f) \ : \ (x,\pi)\in G\times \Ghat\}
$$
is measurable on $G\times \Ghat$ and  
satisfies
$$
\|\tau \widehat f\|_{L^2(G\times \Ghat)} \leq \|\tau \|_{L^2(G\times \Ghat)} \|\widehat f\|_{L^\infty(\Ghat)}.
$$
Hence $\tau \widehat f\in L^2(G\times \Ghat)$ and 
 we can define
$(\id\otimes \cF^{-1}) (\tau \widehat f) \in L^2(G\times G)$. 
By \eqref{eq_cFf1*f2}, 
we have:
$$
(\id\otimes \cF^{-1}) (\tau \widehat f)(x,z)
=
 \int_G f(y) \kappa_{\tau,x}(y^{-1}z)dy
=
\int_G f(zy^{-1}) \kappa_{\tau,x}(y) dy
= f*\kappa_x(z).
$$
As $f\in \cC_c(G)$, $(\id\otimes \cF^{-1}) (\tau \widehat f)(x,z)$ is in fact continuous in $z$ and it makes sense to define:
\begin{equation}
\label{eq_OpKN}
\Op^{\rm KN}(\tau) f (x):= (\id\otimes \cF^{-1}) (\tau \widehat f) (x,x)=
 f*\kappa_{\tau,x}(x),\qquad f\in \cC_c(G), \, x\in G. 
\end{equation}

It follows from the formula above that the integral kernel of $\Op^{\rm KN}(\tau)$ is given by 
$$
G\times G\ni (x,y)\longmapsto \kappa_{\tau,x}(y^{-1}x).
$$
Hence,  the operator $\Op^{\rm KN}(\tau)$ extends uniquely into a Hilbert-Schmidt operator on $L^2(G)$ with norm
$$
\|\Op^{\rm KN}(\tau)\|_{HS(L^2(G))}
=
\|\kappa_\tau\|_{L^2(G\times G)}
=
\|\tau\|_{L^2(G\times\Ghat)}
$$
Consequently, $\Op^{\rm KN}$ is an isometry from $L^2(G\times \Ghat)$ onto $HS(L^2(G))$.

\subsubsection{Extension of $\Op^{\rm KN}$ to $\cC(G, \cF L^1(G))$}

Clearly, we can extend naturally $\Op^{\rm KN}$ via \eqref{eq_OpKN} to $\cC(G, \cF L^1(G))$, that is, to the symbols 
$\sigma$ of the form $\sigma(x,\pi)=\widehat \kappa_x(\pi)$ with convolution kernel $\kappa\in \cC(G,L^1(G))$.
By injectivity of the Fourier transform, 
the two possible definitions of $\Op^{\rm KN}$ on symbols in $L^2(G\times \Ghat)$ and  $\cC(G, \cF L^1(G))$ coincide. 
Note that $\Op^{\rm KN}(\sigma)$ with $\sigma\in \cC(G, \cF L^1(G))$ will act on $\cC_c(G)$ and the Young convolution inequality implies  the following estimate for the operator norm as operators on  $L^2(G)$. 
\begin{lemma}
\label{lem_A0norm}
	If $\sigma\in \cC(G,\cF L^1(G))$ then 
	$$
	\|\Op^{\rm KN} (\sigma)\|_{\sL(L^2)} \leq 
	\int_G \sup_{x\in G} |\kappa_x(y)| dy 
	$$
\end{lemma}
\begin{proof}
Let $\kappa\in \cC(G,L^1(G))$ and  $f\in \cC(G)$. 
We have
$$
	|f*\kappa_x(x)|\leq |f|*\sup_{x'\in G}|\kappa_{x'}| (x),
	$$	
so the Young convolution inequality yields
	$$
	\sqrt{\int_G |f*\kappa_x(x)|^2 dx} \leq \| |f|*\sup_{x'\in G}|\kappa_{x'}\|_{L^2(G)}
	\leq \|f\|_{L^2(G)} \|\sup_{x'\in G}|\kappa_{x'}\|_{L^1(G)}.
	$$	
\end{proof}

If $\sigma\in \cC(G,L^1(G))$, we define
\begin{equation}
	\label{eq_A0norm}
\|\sigma\|_{\cA_0}
:=\int_G \sup_{x\in G} |\kappa_x(y)| dy, 
\qquad \sigma(x,\pi)=\cF\kappa_x(\pi),
\end{equation}
and we denote by $\cC_b(G,\cF L^1(G))$ 
the subspace of 
$\sigma\in\cC(G,\cF L^1(G))$  such that $\|\sigma\|_{\cA_0}
$ is finite. 
We also denote by $\cC_b(G,\cF L^1(G))$ the space of $\kappa\in \cC(G,L^1(G))$ such that $\int_G \sup_{x\in G} |\kappa_x(y)| dy$ is finite. 


\subsubsection{Extension of $\Op^{\rm KN}$ via the inversion formula}


We can  extend $\Op^{\rm KN}$ to a larger space of symbols than $\cC(G,\cF L^1(G))$ but acting on a smaller space of functions than $\cC_c(G)$ under some further technical assumptions. 
Indeed, let us consider
the space $\cC(G,L^\infty(\Ghat))$ of symbols  $\sigma$ that are continuous maps from $G$ to $L^\infty(\Ghat)$. 
A symbol $\sigma$ in $\cC(G,L^\infty(\Ghat))$  is naturally identified  with a measurable field (up to equivalence) of operators $\sigma=\{\sigma(x,\pi) \in \sL(\cH_\pi) \, : \, (x,\pi)\in G\times \Ghat\}$ satisfying conditions of continuity in $x$ and boundedness in $\pi$. 
We also consider, when it exists, a  space $S$ of bounded, continuous and integrable functions satisfying:
\begin{itemize}
	\item the space $S\cap \cC_c(G)$ is dense in $L^2(G)$,  and
	\item for any $f\in S$, the operators $\widehat f(\pi)$, $\pi\in \Ghat$,
are trace-class and 
the following quantity is finite:
$$
\int_{\Ghat} \tr_{\cH_\pi} |\widehat f(\pi)|d\mu(\pi)<\infty.
$$
\end{itemize} 	
As a consequence of the Plancherel formula, 
the following inversion formula holds:
$$
 f(x)
=  \int_{\widehat G} \tr_{\cH_\pi} \, \Big(\pi(x) \widehat  f(\pi)  \Big)\, d\mu(\pi) , \quad f\in S, \ x\in G,
$$
provided that $G$ is amenable. 
We will not discuss here 
these technical assumptions (existence of $S$ and amenability of $G$), but just comment on the fact that they are naturally satisfied for compact or nilpotent Lie groups with $S$ being the space of smooth functions with compact support; in the nilpotent case, we can take $S$ to be the space of Schwartz functions. 
With the inversion formula, $\Op^{\rm KN}$ extends to the quantization given for symbols $\sigma$ in $\cC(G,L^\infty(\Ghat))$ by:
$$
\Op^{\rm KN}(\sigma)f(x) = \int_{\Ghat} \tr_{\cH_\pi} \, \Big(\pi(x)\sigma(x,\pi) \widehat f(\pi)  \Big)\, d\mu(\pi) , \quad f\in S, \ x\in G.
$$
Naturally, this coincides with the quantization defined above for $\sigma\in L^2(G\times \Ghat)$ and for $\sigma \in \cC(G,\cF L^1(G))$.





\subsection{The Wick quantization}

Another natural symbolic quantization appears on the (locally compact, unimodular, type I) group $G$, in the same flavour as Wick's quantization (see~\cite{Lerner}).
For this, we start by defining the transformation $\cB=\cB_a$ associated with a continuous, square-integrable and bounded function $a$ satisfying $\| a\|_{L^2}=1$.

\subsubsection{The transformation $\cB$}
\label{subsubsec_B}
First, for each $x\in G$, we define the symbol $\sF_x$ on $G\times \Ghat$ via
$$
\sF_x := \{\sF_{x,\pi}(y) : (y,\pi)\in G\times \Ghat\}, \qquad \sF_{x,\pi}(y)=  a( yx^{-1})\pi(y)^*.
$$
We check readily that $\sF_x \in \cC(G,L^\infty(\Ghat ))$ with
$$
\sup_{y\in G} \| \sF_x (y,\cdot)\|_{L^\infty(\Ghat)} \leq  \| a\|_{L^\infty(G)}.
$$
We can now define the operator $\cB=\cB_a$ on $\cC_c(G)$ via
$$
\cB[f](x,\pi)=  \int_G f(y) \, \sF_{x,\pi}(y) dy,\quad f\in \cC_c(G), \ (x,\pi)\in G\times \Ghat.
$$
We observe that $\cB[f]$ is the field of operators  on $G\times \Ghat$ given by 
\begin{equation}\label{eq:B}
\cB[f] (x,\pi)= \cF\left( f \, a( \cdot\, x^{-1})\right)(\pi), \qquad  (x,\pi)\in G\times \Ghat. 
\end{equation}
This map has  frame's properties:

\begin{proposition}
\label{prop_B}
\begin{enumerate}
	\item For any $f\in \cC_c(G)$, $\cB[f]$ defines an element of $L^2(G\times \Ghat)$ with norm 
	$$
	\|\cB[f]\|_{L^2(G\times \Ghat)} = \|f\|_{L^2(G)}.
	$$
	\item The map $\cB$ extends uniquely to an isometry from $L^2(G)$ to
$L^2(G\times \widehat G)$ for which we keep the same notation. Its adjoint map
$\cB^* : L^2(G\times \widehat G)\to L^2(G)$ is given by 
$$
\cB^*[\tau](y)= \int_{G\times \Ghat}  \tr_{\cH_\pi} \left( \tau(x,\pi) (\sF_{x,\pi}(y))^* \right) dxd\mu(\pi), 
\qquad \tau\in L^2(G\times \Ghat), \ y\in G,
$$
in the sense that for any $f\in L^2(G)$, 
$$
(\cB^*[\tau],f)_{L^2(G)}
=\int_{G\times \Ghat}  \tr_{\cH_\pi} \left( \tau(x,\pi) \left(\cF\left ( fa( \cdot\, x^{-1})\right )(\pi)\right )^* \right) dxd\mu(\pi).
$$
If $\tau = (\id\otimes\cF) \kappa$, $\kappa\in L^2(G\times G)$, then 
$$
\cB^* [\tau](y) = \int_G \kappa_x(y) \bar  a(yx^{-1}) dx
=  (\bar a*\kappa_\cdot (y)) (y).
$$
\item We have $\cB^*\cB=\id_{L^2(G)}$ while $\cB \cB^*$ is a projection on a closed subspace of $L^2(G\times \Ghat)$.
\end{enumerate}
\end{proposition}

\begin{proof}
From~\eqref{eq:B} and the Plancherel formula \eqref{eq_plancherel_formula}, we obtain
$$
\int_{\Ghat}
\|\cB[f](x,\pi)\|_{HS(\cH_\pi)}^2
d\mu(\pi) = \|f\, a( \cdot\, x^{-1})\|_{L^2(G)}^2, 
\qquad x\in G.
$$
Integrating against $dx$ yields Part (1). 
Parts (2) and (3) follow readily. 
\end{proof}

\subsubsection{The quantization $\Op^{\rm Wick}$}
We can now define the quantization 
$$
\Op^{\rm Wick} = \Op^{{\rm Wick},a}, 
\quad \mbox{with}\quad 
\Op^{\rm Wick} (\sigma) f= \cB^* \sigma \cB[f], 
\qquad f\in L^2(G), \ \sigma\in L^\infty(G\times \Ghat).
$$
Here, $L^\infty(G\times \Ghat)$ denotes the space 
of symbols $\sigma=\{\sigma(x,\pi) \ : \ (x,\pi)\in G\times \Ghat\}$ which are bounded in $(x,\pi)\in G\times \Ghat$, 
i.e. a measurable field of operators in $(x,\pi)\in G\times \Ghat$ such that 
$$
\|\sigma\|_{L^\infty(G\times\Ghat)} : =\sup_{(x,\pi)\in G\times \Ghat} \|\sigma(x,\pi)\|_{\sL(\cH_\pi)}
$$
is finite, the supremum referring to the essential supremum for the measure $dxd\mu$ on $G\times \Ghat$. 
This is naturally a Banach space (even a von Neumann algebra). 
Moreover, it acts naturally continuously on $L^2(G\times \Ghat)$ by left  composition (and even right composition) with
$$
\|\sigma \tau \|_{L^2(G\times \Ghat)} 
\leq
\|\sigma  \|_{L^\infty(G\times \Ghat)}
\| \tau \|_{L^2(G\times \Ghat)}, 
\qquad \sigma\in L^\infty(G\times \Ghat), \ 
\tau\in L^2(G\times \Ghat).
$$
This implies that the quantization $\Op^{\rm Wick}$ is well defined:
\begin{proposition}
	The symbolic quantization $\Op^{\rm Wick}$ is well defined on $L^\infty(G\times \Ghat)$ and satisfies
$$
\forall \sigma\in L^\infty(G\times\Ghat)\qquad 
\|\Op^{\rm Wick}(\sigma)\|_{\sL(L^2(G))}\leq \|\sigma\|_{L^\infty(G\times \Ghat)}.
$$	
\end{proposition}

\begin{proof}
We have for any $f\in L^2(G)$:
\begin{align*}
	\|\Op^{\rm Wick} (\sigma)f\|_{L^2(G)}
&=\|\cB^* \sigma \cB[f]\|_{L^2(G)}
\\&\leq \|\cB^*\|_{\sL(L^2(G\times \Ghat):L^2(G))}
\|\sigma\|_{L^\infty(G\times \Ghat)}
\|\cB\|_{\sL(L^2(G): L^2(G\times \Ghat))}\|f\|_{L^2(G)}.
\end{align*}
Since $\cB$ is an isometry, the operator norms of $\cB$ and $\cB^*$ are equal to 1. 
\end{proof}

	In the case of $G=\bR^n$ and $a$ being  a Gaussian function, we recognize  $\cB$ as the Bargmann transform and $\Op^{\rm Wick}$ as the Wick quantization \cite{folland,Lerner,corobook}. This explains the notation above. 
	We  extend this vocabulary to the group case. 

As an example, we check readily that $\Op^{\rm Wick}(\id) =\id_{L^2(G)}$ where $\id$ is the symbol $\id = \{ \id_{\cH_\pi} , (x,\pi)\in G\times \Ghat)\}$.



The following computation will allow for the comparison  between the Wick and Kohn-Nirenberg quantizations on  $\cC_b(G,\cF L^1(G))$; note that a symbol in $\cC_b(G,\cF L^1(G))$ is in $L^\infty (G\times 
\Ghat)$.
\begin{lemma}
\label{lem_kW}
	If a symbol $\sigma$ is in $\cC_b(G,\cF L^1(G))$, 
	then 
	$$
	\Op^{\rm Wick} (\sigma) f (x) = f * \kappa^{\rm Wick}_x(x), \qquad f\in \cC_c(G), \, x\in G,
	$$
	where $\kappa^{\rm Wick}\in \cC_b(G, L^1(G))$ is given by:
	\begin{align*}
\kappa^{\rm Wick}_x(w) 
&=
\int_G a(w^{-1}xz^{-1})\bar a(x z^{-1}) \kappa_z(w) dz	
\\
&=
\int_G a( w^{-1}z')\bar a(z') \kappa_{z'^{-1}x}(w) dz',
\end{align*}
	and $\kappa\in \cC_b(G,L^1(G))$ is the convolution kernel of $\sigma$ in the sense that 
	$\sigma(x,\pi) = \cF\kappa_x (\pi)$.
\end{lemma}
We will call $\kappa^{\rm Wick}$ the Wick convolution kernel of $\sigma$. 

\begin{proof}
We check readily that $\kappa^{\rm Wick}\in \cC_b(G,L^1(G))$ with
$$
\int_G \sup_{x\in G}| \kappa_x^{\rm Wick}(w)| dw
\leq \int_G \sup_{x'\in G}| \kappa_{x'}(w)| dw 
\int_G |a|(w^{-1} z' )|a|(z')  dz'
\leq \int_G \sup_{x'\in G}| \kappa_{x'}(w)| dw ,
$$
by the Cauchy-Schwartz inequality since $\|a\|_{L^2(G)}=1$.


Let $f\in \cC_c(G)$ and $x\in G$. Property \eqref{eq_cFf1*f2} yields
$$
\sigma \cB[f](x,\pi) = (\widehat \kappa_x \cF (f a(\cdot \, x^{-1})))(\pi)
=\cF\left ( (f\, a(x^{-1}\cdot ))*\kappa_x\right )(\pi)
$$
 so with Part 2 of Proposition \ref{prop_B}, we obtain
\begin{align*}
\Op^{\rm Wick} (\sigma) f (x) 
&=\int_G  (f\, a(\cdot\, z^{-1} ))*\kappa_z(x) \, \bar  a(x z^{-1}) 
dz\\
&=\int_{G\times G} f(y) a(y z^{-1}) \kappa_z(y^{-1} x)\, \bar  a(xz^{-1}) 
dydz,
\end{align*}
and we recognise 
$f*\kappa^{\rm Wick}_x(x).$
\end{proof}


\subsubsection{Some properties of $\Op^{\rm KN}$ and $\Op^{\rm Wick}$}
\label{subsubsec_compOPs}
In our definitions of the quantizations, we choose to  act on the left by $\tau$ in \eqref{eq_OpKN} or equivalently to place $\kappa_x$ on the right of the convolution product in \eqref{eq_OpKN} 
 while we made choices in the writing of $\sF_{x,\pi}$.
These choices imply that our quantization interact well with the left translations $L_{x_0}$ by $x_0$ on functions, i.e. $L_{x_0}f(x) = f(x_0x)$ for any function $f$ defined on $G$, and also on symbols: $L_{x_0}\sigma (x,\pi) = \sigma(x_0 x,\pi)$. Indeed, 
we check readily that 
$$
\Op^{{\rm Wick},a}(L_{x_0} \sigma) 
=
\Op^{{\rm Wick},L_{x_0} a}(\sigma)
\quad\mbox{while}\quad 
\Op^{\rm KN}(L_{x_0}\sigma)=
L_{x_0} \Op^{\rm KN}(\sigma) L_{x_0}^{-1} 
.
$$




The Wick quantization $\Op^{\rm Wick}$ has the advantage of preserving self-adjointness and of being naturally positive.
Indeed, for any $\sigma\in L^\infty (G\times \Ghat)$, we have
$$
(\Op^{\rm Wick}(\sigma))^* = \cB^* \sigma^* \cB=\Op^{\rm Wick}(\sigma^*),
$$
so
if $\sigma$ is self-adjoint in the sense that $\sigma(x,\pi)=\sigma(x,\pi)^*$ for almost all $(x,\pi)\in G\times \Ghat$, then $\Op^{\rm Wick}(\sigma)$ is self-adjoint. 
Moreover, 
 if $\sigma$ is a non-negative symbol in the sense that the operator $\sigma(x,\pi)$ is bounded below by 0 for almost every $(x,\pi)\in G\times \Ghat$, then 
the corresponding operator acting on $L^2(G\times \Ghat)$ is also non-negative so 
\begin{equation}
	\label{positivity}
	(\Op^{\rm Wick}(\sigma)f,f)_{L^2(G)} = 
	(\sigma \cB[f],\cB[f])_{L^2(G\times\Ghat)}\geq 0 
	\qquad \sigma\in L^\infty(G\times\Ghat), \ f\in S,
\end{equation}
and
$\Op^{\rm Wick}(\sigma)$ is a non-negative operator on $L^2(G)$.

In general, the Kohn-Nirenberg quantization $\Op^{\rm KN}$ will not be positive.
However,  weaker properties of positivity  may be recovered in certain cases via  G\r  arding inequalities in pseudo-differential calculi.
The rest of this paper is devoted to showing  G\r  arding inequalities  in the case of graded nilpotent Lie groups and compact Lie groups. 


\section{G\r  arding inequality on compact  Lie groups}
\label{sec_Gcomp}

Here, $G$ is a connected compact Lie group. 
Automatically, all the technical assumptions  mentioned in Section \ref{sec_quantization} (locally compact, unimodular, type I, amenable) are satisfied.
In this case, every irreducible representation is finite dimensional, the dual set $\Ghat$ is discrete and the Plancherel measure is known explicitly:
$\mu(\{\pi\})=d_\pi$ is the dimension of $\pi\in \Ghat$, so that we have the Plancherel formula: 
$$
\|f\|_{L^2(G)}^2= \sum_{\pi\in \Ghat} d_\pi \|\widehat f(\pi)\|_{HS(\cH_\pi)}^2.
$$
A symbol is a family $\sigma = \{\sigma(x,\pi) \in \sL(\cH_\pi) \, :\, (x,\pi)\in G\times \Ghat\}$ of finite dimensional linear maps parametrised by $(x,\pi)$, each acting on the (finite dimensional) space of the representation. 
We can define the Fourier transform not only of integrable functions, but also of any distributions.



\subsection{The pseudo-differential calculus}
\label{subsec_pseudoC_compact}
In this section, we set some notation and recall briefly the global symbol classes defined on $G$ together with some properties of the pseudo-differential calculus. 
We refer to 
\cite{FJFA2015} for more details.


We fix a basis $X_1,\ldots,X_n$ for the Lie algebra $\fg$ of the group $G$. 
We keep the same notation for the associated left-invariant vector fields on $G$. For a multi-index $\alpha=(\alpha_1,\ldots,\alpha_n)\in \bN_0^n$, we set $X^\alpha=X_1^{\alpha_1}\ldots X_n^{\alpha_n}$.

We fix a scalar product on $\fg$ that is invariant under the adjoint action.
The Laplace-Beltrami operator is the differential operator
$\cL=-X_1^2-\ldots - X_n^2$
for any orthonormal basis $X_1,\ldots,X_n$ of $\fg$. 
Identified with an element of the universal enveloping algebra and keeping the same notation for a representation $\pi$ of $G$ and its infinitesimal counterpart for $\fg$, $\pi(\cL)$ is scalar when $\pi$ is irreducible.
$$
\widehat \cL(\pi):=\pi(\cL)=\lambda_\pi \id_{\cH_\pi},
$$
with $\lambda_\pi\geq 0$. In fact, $\lambda_1=0$
when $\pi$ is the trivial representation 1, while 
$\lambda_\pi>0$ when $\pi\neq 1$. 

Let $m\in \bR$ and $1\geq \rho \geq \delta\geq 0$. 
A symbol $\sigma$ is in $S_{\rho,\delta}^m(G)$
when  for any multi-indices  $\alpha,\beta$, there exists $C=C(\alpha,\beta)\geq 0$ such that 
$$
\| X^\beta \Delta^\alpha \sigma(x,\pi) \|_{\sL(\mathcal H_\pi)} \leq C (1+\lambda_\pi)^{\frac {m-\rho|\alpha|+\delta|\beta|}2};
$$
above, $\Delta^\alpha$ denotes the intrinsic difference operators (see \cite{FJFA2015,FJFA2020} for more details) or 
 the RT-difference operators (see \eqref{eq_RTdiff} below).
This yields the following semi-norm 
$$
\|\sigma\|_{S_{\rho,\delta}^m,a,b}
:=\max_{|\alpha|\leq a, |\beta|\leq b}
\sup_{(x,\pi)\in G\times \Ghat}
(1+\lambda_\pi)^{-\frac {m-\rho|\alpha|+\delta|\beta|}2}\| X^\beta \Delta^\alpha \sigma(x,\pi) \|_{\sL(\mathcal H_\pi)} .
$$
If $(\rho,\delta)=(1,0)$, we simply write 
$S^m(G)=S^m_{1,0}(G)$.

\smallskip

The following theorem summarises the main property of the classes of operators obtained by the $\Op^{\rm KN}$-quantization of the classes $S_{\rho,\delta}^m(G)$:
\begin{theorem}
\label{thm_calculus_comp}
For each $m\in \bR$, and $1\geq \rho \geq \delta\geq 0$,
equipped with the semi-norm $\|\cdot \|_{S_{\rho,\delta}^m,a,b}$, 
  $S^m_{\rho,\delta}(G)$ becomes a Fr\'echet space.
The space of operators
 $\Psi_{\rho,\delta}^m(G) := \Op^{\rm KN}(S_{\rho,\delta}^m(G))$ inherit this structure of Fr\'echet space. 
The classes of operators
 $\Psi_{\rho,\delta}^\infty(G)=\cup_{m\in \bR}\Psi_{\rho,\delta}^m(G) $ 
 is a pseudo-differential calculus in the sense of  
Definition	\ref{def_pseudo-diff_calculus}.
Moreover, we have the following properties:

\begin{enumerate}
\item The calculus $\Psi_{\rho,\delta}^\infty(G)$ acts continuously on the Sobolev spaces $H^s(G)$ in the following sense: 
if $\sigma\in S_{\rho,\delta}^m(G)$ then $\Op^{\rm KN}(\sigma)$ maps $H^s(G)$ to $H^{s-m}(G)$ for any $s\in \bR$.
Furthermore, the map $\sigma\mapsto \Op^{\rm KN}(\sigma)$ is continuous $S^m(G)\to \sL(H^s(G),H^{s-m}(G))$.
\item For any $\sigma_1\in S_{\rho,\delta}^{m_1}$
and $\sigma_2\in S_{\rho,\delta}^{m_2}$, we have
$$
\Op^{\rm KN}(\sigma_1)\Op^{\rm KN}(\sigma_2)- \Op^{\rm KN}(\sigma_1\sigma_2)
\in 
\Psi_{\rho,\delta}^{m_1+m_2-(\rho-\delta)}(G).
$$
Furthermore, the map $(\sigma_1,\sigma_2)\mapsto \Op^{\rm KN}(\sigma_1)\Op^{\rm KN}(\sigma_2)- \Op^{\rm KN}(\sigma_1\sigma_2)$ is continuous $S_{\rho,\delta}^{m_1}\times S_{\rho,\delta}^{m_2} \to \Psi_{\rho,\delta}^{m_1+m_2-(\rho-\delta)}(G)$.

\item For any $\sigma\in S_{\rho,\delta}^{m}$, we have
$$
\Op^{\rm KN}(\sigma)^*- \Op^{\rm KN}(\sigma^*)
\in 
\Psi_{\rho,\delta}^{m-(\rho-\delta)}(G).
$$
Furthermore, the map $\sigma\mapsto \Op^{\rm KN}(\sigma)^*- \Op^{\rm KN}(\sigma^*)$ is continuous $S_{\rho,\delta}^{m} \to \Psi_{\rho,\delta}^{m-(\rho-\delta)}(G)$.
\end{enumerate}
 \end{theorem}
 
 Whenever it makes sense, this calculus coincides with H\"ormander's:
 \begin{theorem}
 	When $\rho>\delta$ and $\rho\geq 1-\delta$, 
$\Psi_{\rho,\delta}^\infty(G)$
 coincides with the H\"ormander pseudo-differential calculus defined locally via charts.
 \end{theorem}

Any $\sigma\in S_{\rho,\delta}^m(G)$ admits a distributional  convolution kernel $\kappa: z\mapsto \kappa_x(z) \in \cC^\infty(G,\cD'(G))$, i.e. $\sigma(x,\pi) = \widehat \kappa_x(\pi)$ and 
$\Op^{\rm KN}(\sigma)f(x) =f*\kappa_x(x)$, $f\in \cD(G)$. 
This allows for an application of the RT-difference operators.
Let us recall that the RT-difference operator $\Delta_q$ associated to $q\in \cC^\infty(G)$ is defined via:
\begin{equation}
	\label{eq_RTdiff}
	\Delta_q \widehat \kappa = \cF (q\kappa), \qquad \kappa\in \cD'(G).
\end{equation}
The following property of RT-difference operators follows readily from \cite[Section 5]{FJFA2015}:
\begin{lemma}
\label{lem_contDelta_q}
Let $m\in \bR$ and $1\geq \rho\geq \delta\geq 0$. 
\begin{enumerate}
	\item If $q\in \cD(G)$, then the map $\sigma \mapsto \Delta_q \sigma$ is continuous  $S_{\rho,\delta}^m(G)\to S_{\rho,\delta}^m(G)$. 
Moreover, $q\mapsto \Delta_q$ is continuous $\cD(G)\to \sL(S_{\rho,\delta}^m(G))$. 
	\item The map $\sigma \mapsto \Delta_{q-q(e_G)} \sigma$ is continuous  $S_{\rho,\delta}^m(G)\to S_{\rho,\delta}^{m-(\rho-\delta)}(G)$ for any $m\in \bR$. 
Moreover, $q\mapsto \Delta_q$ is continuous $\cD(G)\to \sL(S_{\rho,\delta}^m(G), S_{\rho,\delta}^{m-(\rho-\delta)}(G))$. 
\end{enumerate}		
\end{lemma}
We observe that translating a symbol will not affect its membership to a $S_{\rho,\delta}^m(G)$-class or  the action of a difference operator:
\begin{lemma}
\label{lem_translationSm}
If $x_0\in G$ and $m\in \bR$, then for any $\sigma\in S_{\rho,\delta}^m(G)$, the symbol $L_{x_0}\sigma  = \{\sigma(x_0x,\pi) : (x,\pi)\in G\times \Ghat)\}$ is in $S_{\rho,\delta}^m(G)$. 
	Moreover, 
	$$
	\|L_{x_0} \sigma \|_{S_{\rho,\delta}^m,a,b} 
	= \| \sigma\|_{S_{\rho,\delta}^m,a,b}, 
	$$
	so
$(x_0,\sigma) \mapsto L_{x_0} \sigma $ is continuous $G\times S_{\rho,\delta}^m(G)\to S_{\rho,\delta}^m (G)$.
\end{lemma}

It implies readily the following property of convolution in the $x$-variable of a symbol:
\begin{corollary}
\label{cor_lem_translationSm}
    If $\sigma\in S_{\rho,\delta}^m(G)$ and $\varphi\in \cC^\infty(G)$, then 
denoting by $\varphi *\sigma$ the symbol
	 $$
	 \varphi *\sigma = \{\varphi *\sigma(x,\pi)\, : \, (x,\pi)\in G\times \Ghat\}, 
	 \quad 
	 \varphi*\sigma (x,\pi)= \int_G \varphi (z) \sigma(z^{-1}x,\pi)  dz,
	 $$
	 we have $\varphi *\sigma \in S_{\rho,\delta}^m(G)$.
\end{corollary}

\subsection{Proof of the G\aa rding inequality}
\label{subsec_pfGarding_comp}
Here, we prove the following $(\rho,\delta)$-generalisation of Theorem \ref{thm_Gardingcomp}:
\begin{theorem}
\label{thm_Gardingcomp_rhodelta}
Let $G$ be a connected compact Lie group. 
Let $m\in \bR$ and $1\geq \rho>\delta\geq 0$.
	Assume that the symbol $\sigma\in S_{\rho,\delta}^{m}(G)$ satisfies the elliptic condition $\sigma \geq c (\id  +\widehat \cL)^{m/2}$, 
that is,	
	$$
	\sigma(x,\pi) \geq c (1+\lambda_\pi)^{\frac m2} \id_{\cH_\pi} \qquad (x,\pi)\in G\times \Ghat,
	$$
	for some constant $c>0$.
	Then there exists a constant $C>0$ such that 
	$$
	\forall f\in \cC^\infty(G)\qquad 
	\Re \left (\Op^{\rm KN}(\sigma) f, f\right )_{L^2(G)} \geq -C \|f\|_{H^{\frac{m-(\rho-\delta)}2}(G)}^2 .
$$	 
\end{theorem}

The main step in our proof is the following analysis of the Wick quantization in the $\Psi^\infty_{\rho,\delta}$-calculus:
\begin{lemma}
\label{lem_OpW}
Here, we consider the Wick quantization $\Op^{{\rm Wick},a}$ with a smooth function 
$a:G\to \bC$. 
Let $m\in \bR$ and $1\geq \rho\geq \delta\geq 0$. 
\begin{enumerate}
	\item If $\sigma\in S_{\rho,\delta}^m(G)$ with $m\leq 0$, then 
 $\sigma\in L^\infty(G\times \Ghat)$ and $\Op^{\rm Wick}(\sigma)\in \Psi_{\rho,\delta}^m(G)$. Moreover, the map $\sigma \mapsto \Op^{\rm Wick}(\sigma)$ is continuous $S_{\rho,\delta}^m(G)\to \Psi_{\rho,\delta}^m(G)$.
	
 
 \item  If $\sigma\in S_{\rho,\delta}^m(G)$ with $m\leq 0$, 
	 then we have
	$$
	\Op^{\rm Wick}(\sigma) - \Op^{\rm KN} (|a|^2 *\sigma) \in \Psi_{\rho,\delta}^{m-(\rho-\delta)}(G).
	$$ 
Moreover, the map $\sigma \mapsto \Op^{\rm Wick}(\sigma) - \Op^{\rm KN} (|a|^2 *\sigma)$ is continuous $S_{\rho,\delta}^m(G)\to \Psi_{\rho,\delta}^{m-(\rho-\delta)}(G)$.
\end{enumerate}
\end{lemma}

\begin{proof}[Proof of Lemma \ref{lem_OpW}]
We may rephrase Lemma \ref{lem_kW} as $\Op^{\rm Wick}(\sigma) = \Op^{\rm KN}(\sigma^{\rm Wick})$
with 
$$
\sigma^{\rm Wick} (x,\pi)= \int_G \Delta_{q_z} \sigma (z^{-1}x,\pi) dz
$$
where $q_z(w) = a(w^{-1}z)\bar a(z)$.
By Lemma \ref{lem_translationSm}, Corollary \ref{cor_lem_translationSm} and Lemma \ref{lem_contDelta_q} (1), this implies Part 1.

We observe that 
$$
\int_G \Delta_{q_z(0)} \sigma (z^{-1}x,\pi)\, dz
=\int_G  |a|^{2}(z) \, \sigma(z^{-1}x,\pi)\, dz
= |a|^2*\sigma(x,\pi).
$$
Hence, Lemma \ref{lem_contDelta_q} (2) implies Part 2.
\end{proof}

 The choice of the functions
 $a$ for the Wick quantization is at the core of our proof of the G\aa rding inequality. We do it  in relation to an approximation of the identity.
By an approximation of the identity on a compact Lie group $G$, we mean a family of functions  $\varphi_t\in \cD(G)$, $t>0$, satisfying $\varphi_t(z)\geq 0$, 
$\int_G \varphi_t(z)dz =1$ for any $t>0$ 
and for any neighbourhood $V$ of the neutral element $e_G$, $\lim_{t\to 0} \int_{z\in V} \varphi_t(z) dz=1$.
We then have  
$$
\lim_{t\rightarrow0}\max_{x\in G}|\psi(x)-\varphi_t *\psi(x)|=0,
$$
for any continuous function $\psi:G\to \bC$; this extend to  continuous functions $\psi$ that are  Banach valued. 
From this, we readily obtain the  properties regarding the approximation of the identity that we will use in our proof of G\aa rding inequalities below and  summarised in the following lemma:

\begin{lemma}
  \label{lem_control}
  Let $\varphi_t$, $t>0$, be an approximation of the identity on $G$. 
  We have
$$
\forall \sigma\in S^m_{\rho,\delta}(G)\qquad 
\lim_{t\to 0}\|\sigma -  \varphi_t*\sigma\|_{S^m_{\rho,\delta}(G),a_0,b_0}
=0,
$$
for any semi-norm $\|\cdot\|_{S^m_{\rho,\delta}(G),a_0,b_0}$, $m\in \bR$, $1\geq \rho\geq \delta\geq 0$.
\end{lemma}
\begin{proof}[Proof of Lemma \ref{lem_control}]
We observe that for  any semi-norm $\|\cdot\|_{S^m_{\rho,\delta}(G),a_0,b_0}$, 
any $\varphi\in \cC^\infty(G)$ and $\sigma\in S^m_{\rho,\delta}(G)$, we have:
$$
\|\sigma -  \varphi*\sigma\|_{S^m_{\rho,\delta}(G),a_0,b_0}
=
\max_{|\alpha|\leq a_0, |\beta|\leq b_0}
\sup_{x\in G}
\|\sigma_{\alpha,\beta}(x,\cdot) -  \varphi*\sigma_{\alpha,\beta}(x,\cdot)\|_{L^\infty(\Ghat)},
$$
where
$\sigma_{\alpha,\beta}:=
(\id+\widehat \cL)^{-\frac{m-\rho|\alpha|-\delta|\beta|}2}X_x^\beta \Delta^\alpha\sigma$.
The statement  follows from the properties of the approximation of the identity recalled above.  
\end{proof}

In the proof of the G\aa rding inequality for symbol of order 0 below, 
we will choose an approximation of the identity $\varphi_t$, $t>0$, that never vanishes, i.e.  $\varphi_t(x)>0$ for any $x\in G$ and $t>0$, and then take $a:=\sqrt{\varphi_t}$.
Such an approximation of the identity $\varphi_t$ is obtained by considering the heat kernel $p_t$ \cite{varo,FJFA2015}, that is, the convolution kernel of $e^{-t \cL}$ .


\begin{proof}[Proof of Theorem \ref{thm_Gardingcomp_rhodelta} for $m=0$] 
Let $\sigma\in S_{\rho,\delta}^0(G)$ satisfying $\sigma(x,\pi)=\sigma(x,\pi)^*$ for any $(x,\pi)\in G\times \Ghat$.
The link between  the Wick and Kohn-Nirenberg quantizations (Lemma \ref{lem_OpW}) and the properties  of the pseudo-differential calculus (Theorem \ref{thm_calculus_comp}) imply
$$
\frac 12(\Op^{\rm KN}(\sigma)+\Op^{\rm KN}(\sigma)^*) 
= \Op^{\rm Wick} (\sigma) + \Op^{\rm KN} (\sigma - |a|^2*\sigma ) + \Op^{\rm KN}(\rho),
$$
with $\rho\in S_{\rho,\delta}^{-(\rho-\delta)}(G)$. Hence, we obtain:
\begin{align*}
	\Re (\Op^{\rm KN}(\sigma) f, f)_{L^2(G)}
&\geq  (\Op^{\rm Wick}(\sigma) f, f)_{L^2(G)} 
- \|\Op^{\rm KN} (\sigma -  |a|^2*\sigma )\|_{\sL(L^2(G))} \|f\|^2_{L^2(G)}
\\&\qquad -  \|\Op^{\rm KN}(\rho)\|_{\sL(H^{\frac {\rho-\delta}2}(G),H^{-\frac {\rho-\delta}2}(G))}
\|f\|^2_{H^{-\frac {\rho-\delta}2}(G)}.
\end{align*}
The property \eqref{positivity} of the Wick quantization and the hypothesis 
 $\sigma(x,\pi)\geq c$ for any $(x,\pi) \in G\times \Ghat$ yield
$$
	(\Op^{\rm Wick}(\sigma) f, f)_{L^2(G)} 
	\geq c \|f\|^2_{L^2(G)},
$$
while the properties of the pseudo-differential calculus (Theorem \ref{thm_calculus_comp}) imply that the operator norms 
$$
\|\Op^{\rm KN}(\rho)\|_{\sL(H^{\frac {\rho-\delta}2}(G)),H^{-\frac {\rho-\delta}2}(G))}
\qquad\mbox{and}\qquad 
\|\Op^{\rm KN} (\sigma -  |a|^2*\sigma )\|_{\sL(L^2(G))}$$ are finite. 
Moreover, by Lemma \ref{lem_OpW} (2), the latter is bounded up to some  constant $C_1>0$ by 
a semi-norm $\|\cdot\|_{S^m_{\rho,\delta}(G),a_0,b_0}$:
$$
\|\Op^{\rm KN} (\sigma -  |a|^2*\sigma )\|_{\sL(L^2(G))}\leq C_1
\|\sigma -  |a|^2*\sigma\|_{S^m_{\rho,\delta}(G),a_0,b_0}.
$$
We then choose $a:=\sqrt{\varphi_t}$ with $\varphi_t(x)>0$ an approximation of the identity, with $t>0$ small enough so that 
$$
C_1
\|\sigma -  \varphi_t*\sigma\|_{S^m_{\rho,\delta}(G),a_0,b_0}
\leq c;
$$
this is possible by Lemma \ref{lem_control}.
This concludes the proof of the case $\sigma\in S^0_{\rho,\delta}(G)$.
\end{proof}


In the proof above, we observe  that $\|\Op^{\rm KN}(\rho)\|_{\sL(H^{\frac {\rho-\delta}2}(G),H^{-\frac {\rho-\delta}2}(G))}$ depends on our choice of $a$, but fixing such an $a$ as above yields a constant $C>0$ as in the statement of Theorem \ref{thm_Gardingcomp_rhodelta} for $m=0$. 

\begin{proof}[Proof of Theorem \ref{thm_Gardingcomp_rhodelta} for any $m\in \bR$] 
Let $\sigma\in S_{\rho,\delta}^{m}(G)$ be such that $\sigma \geq c (\id  +\widehat \cL)^{m /2}$.
By the properties of the pseudo-differential calculus, we may write
$$
(\id+\cL)^{-m/4} \Op^{\rm KN}(\sigma) \,(\id+\cL)^{-m/4} =
\Op^{\rm KN}(\sigma_1)+
\Op^{\rm KN}(\rho_1),
$$
with $\sigma_1\in S_{\rho,\delta}^0(G)$ given via $\sigma_1(x,\pi):= (1+\lambda_\pi)^{-m/2} \sigma (x,\pi)$,  and 
$\rho_1\in S_{\rho,\delta}^{-(\rho-\delta)}(G)$.
We observe that $\sigma_1$ satisfies the hypothesis of Theorem \ref{thm_Gardingcomp_rhodelta} with $m=0$.

For any $f\in \cC^\infty(G)$, 
setting $f_1=(\id+\cL)^{m/4} f$, we have
\begin{align*}
\Re (\Op^{\rm KN}(\sigma) f, f)_{L^2(G)}
&=
\Re ((\id+\cL)^{-\frac{m}4} \Op^{\rm KN}(\sigma) \,(\id+\cL)^{-\frac{m}4}
 f_1, f_1)_{L^2(G)}	\\
 &= \Re (\Op^{\rm KN}(\sigma_1) f_1, f_1)_{L^2(G)}
 + \Re (\Op^{\rm KN}(\rho_1) f_1, f_1)_{L^2(G)}\\
 &\geq -C \|f_1\|_{H^{-\frac {\rho-\delta}2}(G)},
\end{align*}
by Theorem \ref{thm_Gardingcomp_rhodelta} with $m=0$ applied to $\sigma_1$ and the properties of the pseudo-differential calculus applied to $\rho_1$. 
The conclusion follows from $\|f_1\|_{H^{-\frac {\rho-\delta}2}(G)} \asymp \|f\|_{H^{\frac{m -(\rho-\delta)}2}(G)}$.
\end{proof}

\section{G\r  arding inequality on graded nilpotent  Lie groups}
\label{sec_Gnilp}

Here, we prove the G\r  arding inequality on a graded nilpotent  Lie group $G$. 
Before this,  we recall some definitions and notation about this class of groups and the associated pseudo-differential calculus. 


\subsection{Preliminaries on graded nilpotent groups}

A graded group $G$ is a connected simply connected nilpotent Lie group whose (finite dimensional, real) Lie algebra $\mathfrak g$ admits an $\N$-gradation into linear subspaces, 
$$
\mathfrak g = \oplus_{j=1}^\infty \mathfrak g_j 
\quad\mbox{with} \quad [\mathfrak g_i,\mathfrak g_j]\subseteq \mathfrak g_{i+j}, \;\; 1\leq i\leq j,
$$
where all but a finite number of subspaces $\mathfrak g_j $ are trivial.
We denote by $j=n_G$ the smallest integer such that all the subspaces $\mathfrak g_j $, $j>n_G$,  are trivial.  
If the first stratum $\mathfrak g_1$ generates  the whole Lie algebra, then $\mathfrak g_{j+1}= [\mathfrak g_1,\mathfrak g_j]$ for all $j\in\N_0$  and $n_G$ is the step of the group; the group $G$ is then said to be stratified, and also  (after a choice of basis or inner product for $\mathfrak g_1$)  Carnot.


\smallskip 
 
 The product law on $G$ is derived from the exponential map  $\exp_G : \mathfrak g \to G$ 
which  is a global diffeomorphism from $\mathfrak g$ onto $G$. Once a basis $X_1,\ldots,X_{n}$
for~$\mathfrak g$ has been chosen, we may  identify 
the points $(x_{1},\ldots,x_n)\in \mathbb R^n$ 
 with the points  $x=\exp(x_{1}X_1+\cdots+x_n X_n)$ in~$G$.
It allows us to define 
the (topological vector) spaces $\mathcal C^\infty(G)$ and $\mathcal S(G)$  of smooth and  Schwartz functions on $G$ identified with $\mathbb R^n$; 
note that the resulting spaces are intrinsically defined as spaces of functions on $G$ and do not depend on a choice of basis. 

The exponential map induces a Haar measure $dx$ on $G$ which is invariant under left and right translations and  defines Lebesgue spaces on~$G$.

\smallskip 



We now construct a basis adapted to the gradation. 
Set $d_j={\rm dim} \, \mathfrak g_j $ for $1\leq j\leq n_G$.
We choose a basis $\{X_1,\ldots, X_{n_1}\}$ of $\mathfrak g_1$ (this basis is possibly reduced to $\emptyset$), then 
$\{X_{n_1+1},\ldots,  X_{n_1+n_2}\}$ a basis of $\mathfrak g_2$
(possibly $\{0\}$)
and so on.
Such a basis $\mathcal B=(X_1, \cdots , X_{d_1+\cdots +d_{n_G}})$
of $\mathfrak g$ is said to be adapted to the gradation. 

\smallskip 

The Lie algebra 
 $\mathfrak g$ is  a homogeneous Lie algebra equipped with 
the family of dilations  $\{\delta_r, r>0\}$,  $\delta_r:\mathfrak g\to \mathfrak g$, defined by
$\delta_r X=r^\ell X$ for every $X\in \mathfrak g_\ell$, $\ell\in \N$
\cite{folland+stein_82,FR}.
We re-write the set of integers $\ell\in \N$ such that $\mathfrak g_\ell\not=\{0\}$
into the increasing sequence of positive integers
 $\upsilon_1,\ldots,\upsilon_n$ counted with multiplicity,
 the multiplicity of $\mathfrak g_\ell$ being its dimension.
 In this way, the integers $\upsilon_1,\ldots, \upsilon_n$ become 
 the weights of the dilations and we have $\delta_r X_j =r^{\upsilon_j} X_j$, $j=1,\ldots, n$,
 on the chosen basis of $\mathfrak g$.
 The associated group dilations are defined by
$$
\delta_r(x)=
rx
:=(r^{\upsilon_1} x_{1},r^{\upsilon_2}x_{2},\ldots,r^{\upsilon_n}x_{n}),
\quad x=(x_{1},\ldots,x_n)\in G, \ r>0.
$$
In a canonical way,  this leads to the notions of homogeneity for functions and operators. 
For instance, the Haar measure is homogeneous of degree
$$
Q:=\sum_\ell \upsilon_\ell \dim \fg_\ell,
$$
which is called the homogeneous dimension of the group. 
Another example is the vector field corresponding to an element $X\in \fg_\ell$: it  is $\ell$-homogeneous. 

In the rest of the paper, 
we assume that we have fixed a basis $X_1,\ldots, X_n$ of $\fg$ adapted to the gradation.
We keep the same notation for the associated left-invariant vector fields on $G$, 
and we denote the corresponding right invariant vector fields by $\tilde X_1, \ldots,\tilde X_n$. For a multi-index $\alpha=(\alpha_1,\ldots,\alpha_n)\in \bN_0^n$, we set $X^\alpha=X_1^{\alpha_1}\ldots X_n^{\alpha_n}$
and $\tilde X^\alpha=\tilde X_1^{\alpha_1}\ldots \tilde  X_n^{\alpha_n}$.
The differential operators $X^\alpha$ and $\tilde X^\alpha$ are homogeneous of degree 
$$
[\alpha] = \upsilon_1\alpha_1+\ldots+\upsilon_n \alpha_n.
$$
 

\subsection{The pseudo-differential calculus}
\label{subsec_pseudoC_graded}
In this section, we set some notation and recall briefly  the global symbol classes defined on $G$ together with some properties of the pseudo-differential calculus. 
We refer to 
\cite{FR} for more details.




A symbol $\sigma$ is in $S_{\rho,\delta}^m(G)$
when  for any multi-indices  $\alpha,\beta\in \bN_0^n$ and $\gamma\in \bR$, there exists $C=C(\alpha,\beta)$ such that we have for almost $(x,\pi)\in G\times \Ghat$,
\begin{equation}
	\label{eq_SmnormGnilp}
	\|\pi(\id+\cR)^{-\frac {m-\rho[\alpha] + \delta[\beta]+\gamma}{\nu}} X^\beta \Delta^\alpha \sigma(x,\pi) \pi(\id+\cR)^{\frac {\gamma}{\nu}}\|_{\sL(\mathcal H_\pi)} \leq C_{\alpha,\beta,\gamma},
\end{equation}
where $\cR$ is a (and then any) positive Rockland operator of homogeneous degree $\nu$; 
 we may assume $\gamma\in \bZ$.  
 The difference operator $\Delta^\alpha$ is the difference operator $\Delta_{x^\alpha}$ for the monomial $x^\alpha$  in the coordinates $x_j$. 
Generalising the definition in the compact setting,  the difference operator $\Delta_q$ associated to $q\in \cC^\infty(G)$ is defined via
$\Delta_q \widehat \kappa = \cF (q\kappa)$ for any $\kappa\in \cS'(G)$ for which $\kappa$ and $q\kappa$ admits a Fourier transform (see \cite{FR}).
We set $\|\sigma\|_{S^m,a,b,c} := \max_{|\alpha|\leq a, |\beta|\leq b, |\gamma|\leq c} C_{\alpha,\beta,\gamma}$ for the best constants $C_{\alpha,\beta,\gamma}$ in \eqref{eq_SmnormGnilp} and $a,b,c\in \bN_0$. If $(\rho,\delta)=(1,0)$, we simply write 
$S^m(G)=S^m_{1,0}(G)$.

The following theorem summarises the main property of the classes of operators obtained by the $\Op^{\rm KN}$-quantization of the classes $S^m(G)$;
the  Sobolev spaces $L^2_s(G)$ adapted to the graded nilpotent Lie group $G$ were studied in \cite{FR,FRSob} generalising slightly the stratified case \cite{folland75}.

\begin{theorem}
\label{thm_calculus_Gnilp}
Theorem \ref{thm_calculus_comp} holds for $G$ a graded nilpotent Lie group when replacing the symbol classes with the ones defined  above and the Sobolev spaces with $L^2_s(G)$.  
  \end{theorem}

Any $\sigma\in S^m(G)$ admits a distributional  convolution kernel $\kappa: z\mapsto \kappa_x(z) \in \cC^\infty(G,\cS'(G))$, i.e. $\sigma(x,\pi) = \widehat \kappa_x(\pi)$ and 
$\Op^{\rm KN}(\sigma)f(x) =f*\kappa_x(x)$, $f\in \cS(G)$. 
We can define difference operators via \eqref{eq_RTdiff}, 
and the statement of Lemma \ref{lem_contDelta_q} also holds in the context of graded Lie groups if $a\in \cD(G)$ is replaced with $a\in \cS(G)$. 
\begin{proof}[Sketch of the proof for Lemma \ref{lem_contDelta_q} for graded $G$]
The properties of the  convolution kernels of symbols in $S^m(G)$, for instance being Schwartz away from the origin, implies that we may assume $q\in \cD(G)$ with compact support near the origin. Let $\chi\in \cD(G)$ be such that $\chi\equiv 1$ on the support of $q$.  
Let $P^q_N(z)$ be the Taylor polynomial at order $N$ in the sense of Folland-Stein \cite{folland+stein_82,FR}
for $q$.
The estimates of  the convolution kernel  $\kappa$ of $\sigma$ imply that
$$
 \sup_{x,y\in G}|\chi(q - P^q_N) \kappa_x|(y)
$$
 will be finite for $N$ large enough, with a constant given by semi-norms in $\sigma$, 
 and similarly for with $X_y^{\beta'_2}\tilde X_y^{\beta'_2}
  y^\alpha \chi(q - P^q_N) X^\beta_x\kappa_x $.
  This implies the statement. 
 \end{proof}

\begin{remark}
	\label{rem_changescompnilp}
	It is straightforward to check that Lemma \ref{lem_translationSm}  on the  left-translations of a symbol, 
 Corollary \ref{cor_lem_translationSm} on the $x$-convolution with a symbol and then Lemma \ref{lem_OpW} on the Wick quantization in the calculus $\Psi^\infty_{\rho,\delta}(G)$ with $a\in \cS(G)$ hold on a graded group $G$.
\end{remark}


\subsection{Proof of the G\aa rding inequality}
Here, we prove the following $(\rho,\delta)$-generalisation of Theorem \ref{thm_Garding_graded}:
\begin{theorem}
\label{thm_Gardingnilp_rhodelta}
Let $G$ be a graded nilpotent Lie group. 
Let $m\in \bR$ and $1\geq \rho>\delta\geq 0$.
	Assume that the symbol $\sigma\in S_{\rho,\delta}^{m}(G)$ satisfies the elliptic condition $\sigma \geq c (\id  +\widehat \cR)^{\frac {m}\nu}$, 
		for some constant $c>0$ where $\cR$ a positive Rockland operator of homogeneous degree $\nu$.
	Then there exists a constant $C>0$ such that 
	$$
	\forall f\in \cS(G)\qquad 
	\Re \left (\Op^{\rm KN}(\sigma) f, f\right )_{L^2(G)} \geq -C \|f\|_{L^2_{\frac{m-(\rho-\delta)}2}(G)}^2 .
$$	 
\end{theorem}

The proof of  Theorem \ref{thm_Gardingnilp_rhodelta}  is a simple modification of the case of compact groups given in  Section~\ref{subsec_pfGarding_comp}.
We will need to replace the Sobolev spaces $H^s(G)$ with the  Sobolev space $L^2_s(G)$ adapted to the graded nilpotent case. Moreover, in the final argument showing that  
the case of a symbol  $\sigma\in S_{\rho,\delta}^{m}(G)$  follows from the case of a symbol of order 0, we need to  replace $\sigma_1$ with 
$$
\sigma_1 = (\id+\widehat\cR)^{-\frac m {2\nu}} \sigma  
(\id+\widehat\cR)^{-\frac m {2\nu}}.
$$
Below, we will detail the proof for a symbol $\sigma\in S_{\rho,\delta}^0(G)$ of order 0.
The main difference with the compact case will come from the construction of a suitable approximation of the identity.  
Before starting the proof, let us recall some properties of the approximations of the identity constructed using dilations on nilpotent Lie groups. 
In the proof of G\aa rding inequalities below, we will crucially use Part (3):
\begin{lemma}
\label{lem_appIGnilp}
Let $\varphi_1\in \cS(G)$ with $\int_G \varphi_1=1$.
Consider the family of integrable functions 
$\varphi_t  = t^{-Q}\varphi_1\circ \delta_{t^{-1}}$, $t>0$.
	\begin{enumerate}
		\item The family of functions 
$\varphi_t$, $t>0$,  form an approximation of identity on $L^p(G)$, $p\in [1,\infty)$ and on the space $\cC_0(G)$ of continuous functions vanishing at infinity, that is, 
$$
\lim_{t\to 0} \|\psi-\varphi_t*\psi \|_{L^p(G)}=0,
$$ 
for $\psi\in L^p(G)$ (resp.  $\cC_0(G)$) for $p\in [1,\infty)$ (resp. $p=\infty$).
\item There exist constants $C,C'>0$, depending on  the group and on $\varphi_1$, such that for any $\psi\in \cC^\infty(G)$, we have
$$
\forall t\in (0,1],\  x\in G, \qquad
|\psi(x) - \varphi_t*\psi(x)| \leq 
C t \max_{j=1,\ldots,n} \sup_{x'\in G}|\tilde X_j \psi(x')|,
$$
and 
$$
\forall t\in (0,1],\  x\in G, \qquad
|\psi(x) - \varphi_t*\psi(x)| \leq 
C' t \max_{j=1,\ldots,n} \sup_{x'\in G}|X_j \psi(x')|.
$$
\item  We have
$$
\forall \sigma\in S^m_{\rho,\delta}(G)\qquad 
\lim_{t\to 0}\|\sigma -  \varphi_t*\sigma\|_{S^m_{\rho,\delta}(G),a_0,b_0}
=0,
$$
for any semi-norm $\|\cdot\|_{S^m_{\rho,\delta}(G),a_0,b_0}$, $m\in \bR$, $1\geq \rho\geq \delta\geq 0$.
	\end{enumerate}
\end{lemma}

The heat kernels $p_t$ of a positive Rockland operators $\cR$, i.e. 
the convolution kernels of $e^{-t\cR}$, $t>0$, provides such an approximation of the identity \cite{FR,folland+stein_82}. 
When $G$ is stratified and $\cR$ is a sub-Laplacian, the heat kernels will be non-negative and never vanishing.  We observe that the heat kernel being positive and never vanishing was used in the proof of the compact case in Section \ref{subsec_pfGarding_comp}.
 These properties of the heat kernel for a general positive Rockland operator are not guaranteed in the graded case. However, in the graded case, Lemma \ref{lem_appIGnilp}  does not require an approximation of the identity that is non-negative and  never vanishing, and this lemma will be sufficient in our proof of Theorem \ref{thm_Gardingnilp_rhodelta}.




\begin{proof}[Proof of Lemma \ref{lem_appIGnilp}]
Part 1 is well known (see \cite{folland+stein_82,FR}) but will not be used in this paper.
In Part~2, the first estimate follows by the Taylor estimates due to Folland and Stein \cite{folland+stein_82,FR}.
As $X_j$ may be expressed as a sum of $\tilde X_k$ with polynomial coefficients, we obtain the second inequality only for $\psi$ with compact support, and a constant depending  on this support. We now extend this to any function $\psi$  with an argument of partition of unity. 

From the partition of unity due to Folland and Stein \cite{folland+stein_82}, there exist a function $\chi\in \cD(G)$ and a sequence $(x_j)_{j\in\bN}$ of points in $G$ such that $\sum_{j\in \bN} \chi_j(x)=1$ for all $x\in G$, where $\chi_j(x)=\chi(xx_j )$; moreover, this sum is finite for every $x\in G$, and in fact uniformly finite on  any compact set of $G$. 
We now observe setting $\psi_j (x):=\psi(xx_j^{-1})$
$$
\psi(x) - \varphi_t *\psi(x)
=
\sum_j (\psi \chi_j)(x) - \varphi_t * (\psi \chi_j)(x)
=
\sum_j \left ((\psi_j \chi) - \varphi_t *(\psi_j\chi) \right )(x x_j).
$$ 
We obtain the second estimates of Part (2)
 with the estimate for functions with a fixed compact support and the sum being uniformly finite. 


We show Part (3) by modifying the proof of Lemma \ref{lem_control}.
We write for any $\varphi\in \cS(G)$
$$	
\|\sigma -  \varphi*\sigma \|_{S^m_{\rho,\delta},a_0,b_0,c_0}
=
\max_{[\alpha]\leq a_0,[\beta]\leq b_0, |\gamma|\leq c_0} 
\max_{x\in G}
\| \sigma_{\alpha,\beta,\gamma}(x,\cdot \,) -  \varphi * \sigma_{\alpha,\beta,\gamma}(x,\cdot \,) )\|_{L^\infty(\Ghat)},
$$
where the symbol $\sigma_{\alpha,\beta,\gamma}\in L^\infty(\Ghat)$ is defined via
$$
\sigma_{\alpha,\beta,\gamma}(x,\pi):=\pi(\id+\cR)^{-\frac {m-\rho [\alpha]+\delta[\beta]+\gamma}{\nu}} X^\beta \Delta^\alpha \sigma(x,\pi) \pi(\id+\cR)^{\frac {\gamma}{\nu}}.
$$
We conclude with the second estimate in Part (2) extended to Banach valued functions $\psi$.
\end{proof}

\begin{proof}[Proof of Theorem~\ref{thm_Gardingnilp_rhodelta}]
As explained above, it suffices to show the statement for 
 $\sigma\in S_{\rho,\delta}^0(G)$ satisfying $\sigma(x,\pi)\geq c$.
We fix a function $a_1\in \cS(G)$ with $\|a_1\|_{L^2(G)}=1$, 
and set $a_t(x):=t^{-Q/2}a_1(\delta_t^{-1} x)$, $x\in G$, $t>0$.
We observe that $|a_t|^2 = t^{-Q} |a_1|^2 \circ \delta_t^{-1}$, $t>0$, is an approximation of the identity in the sense of Lemma \ref{lem_appIGnilp}. We now consider the Wick quantization $\Op^{{\rm Wick}, a}$ with $a=a_t$ to be chosen at the end of the proof.
 
The properties of the Wick quantization (see \eqref{positivity} and Lemma \ref{lem_OpW} with Remark \ref{rem_changescompnilp}) imply
\begin{align*}
	\Re (\Op^{\rm KN}(\sigma) f, f)_{L^2(G)}
&\geq  c \|f\|^2_{L^2(G)} 
- \|\Op^{\rm KN} (\sigma -  |a|^2*\sigma )\|_{\sL(L^2(G))} \|f\|^2_{L^2(G)}
 -  C
\|f\|^2_{L^2_{-\frac {\rho-\delta}2}(G)},
\end{align*}
for some constant $C>0$. 
By Theorem \ref{thm_calculus_Gnilp}, 
 there exist a constant $C_1>0$ and a semi-norm $\|\cdot\|_{S_{\rho,\delta}^0,a_0,b_0,c_0}$ (independent of $\sigma$ and $a$) such that 
$$
\|\Op^{\rm KN} (\sigma - |a|^2 *\sigma)\|_{\sL(L^2(G))}
\leq C_1  \|\sigma -  |a|^2*\sigma \|_{S_{\rho,\delta}^0,a_0,b_0,c_0}.
$$
We now choose $a=a_t$
with $t>0$ small enough so that, 
by  Lemma \ref{lem_appIGnilp} (3), the right-hand side above is $\leq c$.
 This shows the case of $\sigma\in S_{\rho,\delta}^0(G)$ and  concludes the proof of Theorem \ref{thm_Garding_graded}. 
\end{proof}

Note that the proof above also gives the following weak G\aa rding inequality, where the symol is only supposed to be non-negative.

\begin{theorem}
Let $G$ be a graded nilpotent Lie group. 
Let $m\in \bR$ and $1\geq \rho>\delta\geq 0$.
	Assume that the symbol $\sigma\in S_{\rho,\delta}^{m}(G)$ satisfies the elliptic condition $\sigma \geq 0$.
	Then, for any $\eta>0$,  
 there exists a constant $C_\eta>0$ such that
	$$
	\forall f\in \cS(G)\qquad 
	\Re \left (\Op^{\rm KN}(\sigma) f, f\right )_{L^2(G)} \geq -\eta \|f\|^2_{L^2_{\frac{m} 2}(G) }-C_\eta\|f\|_{L^2_{\frac{m-(\rho-\delta)}2}(G)}^2 .
$$	 
\end{theorem}


\section{Semi-classical G\r  arding inequality on graded nilpotent Lie groups}
\label{sec_SC_nilpG}

In this section, we show the semi-classical inequality stated in Theorem \ref{thm:garding_sc}.
Before this, we recall the definition of the semi-classical calculus and we introduce the  Wick quantization adapted to the semi-classical setting. 

\subsection{Semi-classical pseudodifferential calculus}
\label{subsec_semiclC}

The set $\mathcal A_0$ is the space of symbols
$\sigma = \{\sigma(x,\pi) : (x,\pi)\in G\times \widehat G\}$ of the form 
$$
\sigma(x,\pi)=\mathcal F \kappa_x (\pi) = \int_G \kappa_x(y) (\pi(x))^* dx, 
$$
where $(x,y)\mapsto \kappa_x(y)$ is a function of  the set  $\mathcal C_c^\infty(G,\mathcal S( G))$ of smooth and compactly supported functions on~$U$ valued in the set of Schwartz class functions. 
As before, $x\mapsto \kappa_x$ is called the convolution kernel of $\sigma$. 

With the symbol $\sigma\in\mathcal A_0$, we associate the (family of) {\it semi-classical pseudodifferential operators} 
$$
{\rm Op}_\eps(\sigma) = \Op^{\rm KN} \sigma (\cdot ,\delta_\eps \cdot) , \qquad \eps\in (0,1], 
$$ 
where the Kohn-Nirenberg quantization $\Op^{\rm KN}$ was defined in Section \ref{subsubsec_OPKN}
and $\delta_r$ denotes the action of $\mathbb{R}^+$ 
on $\widehat G$ given via
$$
\delta_r \pi (x)
=
\pi(\delta_r x),\quad x\in G,\
\pi\in \widehat G, \ r>0.
$$

In other words, we have
$$
{\rm Op}_\eps(\sigma) f(x) = \int_{\pi\in\widehat G} {\rm Tr}_{{\mathcal H}_\pi} \left(  \pi(x) \sigma(x,\delta_\eps\pi) {\mathcal F}_G f(\pi)  \right)d\mu(\pi),\;\;
f\in \mathcal S(G), \, x\in G, 
$$
or equivalently, in terms of the convolution kernel $\kappa_x =\mathcal F^{-1}_G \sigma(x,\cdot)$,
$$
{\rm Op}_\eps(\sigma) f(x) = f* \kappa^{(\eps)}_x (x),\quad 
f\in \mathcal S(G), \, x\in U,
$$
where $\kappa^{(\eps)}_x$ is the following rescaling of the convolution kernel:
$$
\kappa^{(\eps)}_x (y) := \eps^{-Q} \kappa_x(\delta_\eps ^{-1} y).
$$


 \subsection{The semi-classical Wick quantization}

 
Let $a\in \mathcal S(G)$ such that $\| a\|_{L^2}=1$.
We set
$$
a_\eps := \eps^{-\frac Q4} a \circ \delta_{\eps^{-\frac 12}},
\qquad \eps>0,
$$
so that $a_\eps\in \cS(G)$ with 
 $\| a_\eps \|_{L^2(G)}=1$.
We also set
$$
\sF_x^\eps  := \{\sF_{x,\pi}^\eps (y) : (y,\pi)\in G\times \Ghat\}, \qquad \sF_{x,\pi}^\eps (y)=  a_\eps (yx^{-1} )\delta_\eps ^{-1}\pi(y)^*,
$$
and define the operator $\cB^\eps$ on $\cS(G)$ via
$$
\cB^\eps [f](x,\pi)= \eps^{-\frac Q2}  \int_G f(y) \sF^\eps _{x,\pi}(y) dy,\quad f\in \cS(G), \ (x,\pi)\in G\times \Ghat.
$$
Note that, with respect to the operator $\cB_a$ defined in Section \ref{subsubsec_B}, we have
$$
\cB^\eps [f](x,\pi) =\eps^{-\frac Q2} \cB_{a_\eps}[f](x,\delta_{\eps}^{-1} \pi).
$$
Hence, by Proposition \ref{prop_B}, 
the map $\cB^\eps $ extends uniquely to an isometry from $L^2(G)$ to
$L^2(G\times \widehat G)$ for which we keep the same notation. Denoting by 
$\cB^{\eps,*} : L^2(G\times \widehat G)\to L^2(G)$  its adjoint map, we have $\cB^{\eps,*}\cB^\eps=\id_{L^2(G)}$ while $\cB^\eps \cB^{\eps,*}$ is a projection on a closed subspace of $L^2(G\times \Ghat)$.
We also define the semi-classical Wick quantization for $\sigma\in L^\infty(\Ghat)$ 
$$
\Op^{\rm Wick}_\eps(\sigma) :=  \cB^{\eps,*} \sigma\, \cB^\eps
$$
We can compute the convolution kernel of $\Op^{\rm Wick}_\eps(\sigma)$ as in Lemma \ref{lem_kW}:
\begin{lemma}
If $\sigma\in \cA_0$, 
then 
$$
\Op^{\rm Wick}_\eps(\sigma) = \Op^{\rm KN}_\eps (\sigma^{\eps,{\rm Wick}}),
$$
where $\sigma^{\eps,{\rm Wick}}\in \cA_0$ has the following convolution kernel 
\begin{align*}
	\kappa^{\eps,{\rm Wick}}_x(w) 
&=
\int_G a_\eps (\delta_\eps w^{-1} x z^{-1})\bar a_\eps (x z^{-1}) \kappa_z(w) dz\\
&=
\int_G a_\eps (\delta_\eps w^{-1}z')\bar a_\eps (z') \kappa_{ {z'}^{-1}x}(w) dz'\\
&=
\int_G a ( \delta_{\sqrt{\eps}} w^{-1} z')\bar a (z') \kappa_{\delta_{\sqrt{\eps}}{z'}^{-1} x}(w) dz',
\end{align*}
\end{lemma}



\begin{corollary}
\label{cor_OpepsWKN}
	We choose a function $a\in \cD(G)$ that is even, i.e.  $a(x^{-1})=a(x)$, and real valued.  
	Then for any $\sigma\in \cA_0$, there exists $C>0$ such that for all $\eps\in (0,1]$, 
	$$
	\|\Op_\eps^{\rm KN}(\sigma)-\Op^{\rm Wick}_\eps (\sigma) \|_{\sL(L^2(G))} \leq C\eps.
	$$
\end{corollary}


\begin{proof}
By Lemma \ref{lem_A0norm}, using the $\cA_0$-norm defined in \eqref{eq_A0norm}, we have
$$
\|\Op_\eps^{KN}(\sigma)-\Op^{\rm Wick}_\eps (\sigma) \|_{\sL(L^2(G))}
\leq \| \sigma - \sigma^{\eps,{\rm Wick}} \|_{\cA_0}
\leq I_1(\eps) +I_2(\eps), 
$$	
where
\begin{align*}
	I_1(\eps)&:= \int_G \sup_{x\in G}\left|\int_G |a(z)|^2 \left(\kappa_x(w) -\kappa_{\delta_{\sqrt{\eps}}{z}^{-1} x}(w) \right )dz \right| dw \\
I_2(\eps)&:= \int_G \sup_{x\in G}\left|\int_G (a(z)-a(\delta_{\sqrt \eps} w^{-1}z))\bar a(z)  \kappa_{ \delta_{\sqrt{\eps}}{z}^{-1} x}(w) dz \right| dw.
\end{align*}
By the Taylor estimates due to Folland and Stein \cite{folland+stein_82,FR}, we have:
\begin{align*}
	I_1(\eps)&= \sqrt \eps   \int_G \sup_{x\in G}\left| \sum_{[\alpha]=1}  \int_G |a(z)|^2 q_\alpha(z^{-1})dz \ 
	 \tilde X^\alpha_x \kappa_x(w)  \right| dw  + O(\eps)\\
I_2(\eps)&=\sqrt \eps \int_G \sup_{x\in G}\left|\sum_{[\alpha]=1} \int_G   \tilde X^\alpha a(z) \bar a(z)  q_\alpha (w)  \kappa_{x \delta_{\sqrt{\eps}}{z}^{-1}}(w) dz \right| dw +O(\eps)\\
&\leq \sqrt \eps \int_G \sup_{x'\in G}\sum_{[\alpha]=1}\left| \int_G   \tilde X^\alpha a(z) \bar a(z) dz \right|  \left| q_\alpha (w)  \kappa_{x'}(w)\right|  dw +O(\eps).
\end{align*}
As $a$ is even, for any polynomial $q$ satisfying $q(z^{-1})=-q(z)$ such as the coordinate polynomials $q_\alpha$ with $[\alpha]=1$, we have
$\int_G |a(z)|^2 q(z) dz=0$.
As $a$ is real valued, for any left or right invariant vector field $X$, an integration by parts shows 
 $ \int_G X a(z) \bar a(z) dz =0$. 
Consequently, $I_1(\eps)=O(\eps)$ and $I_2(\eps)=O(\eps)$. 
\end{proof}


\subsection{Proof of the semi-classical G\aa rding inequality}
We write
$$
\Re \left( \Op^{\rm KN}_\eps(\sigma )f,f\right)_{L^2(G)}
\geq 
\left( \Op^{{\rm Wick}}_\eps(\sigma )f,f\right)_{L^2(G)}
- \|\Op_\eps^{KN}(\sigma)-\Op^{\rm Wick}_\eps (\sigma) \|_{\sL(L^2(G))}\|f\|_{L^2(G)}.
$$
By the properties of the semi-classical Wick quantisation, if $\sigma\geq 0$, then 
$$
\left( \Op^{{\rm Wick}}_\eps(\sigma )f,f\right)_{L^2(G)} 
=(\sigma \cB^\eps f,\cB^\eps f)_{L^2(G\times \Ghat)} \geq 0,
$$
while by Corollary \ref{cor_OpepsWKN}, $\|\Op_\eps^{KN}(\sigma)-\Op^{\rm Wick}_\eps (\sigma) \|_{\sL(L^2(G))}=O(\eps)$.
This concludes the proof of Theorem~\ref{thm:garding_sc}.
\smallskip 

We point out that  the semi-classical case is more straightforwards because we restrict ourselves to the use of $L^2$-norm and to a gain in the semi-classical parameter $\eps$. The proof of Theorem~\ref{thm:garding_sc} is analogous to the one of Theorem~\ref{thm_Gardingnilp_rhodelta}, by taking the dilation parameter $t$ as $\sqrt\eps$. 


\appendix


\section{The Euclidean case}
\label{app}

In this section, we recall the definitions and some properties of the 
Kohn-Nirenberg and Wick quantizations in the Euclidean case $\bR^n$. 
We develop the same chain of arguments  that show  the G\aa rding inequality  for the H\"ormander calculus on $\bR^n$  as in the core of the paper.
This leads  to a proof which is close to the one of  \cite[Chapter 2, section 6]{folland}.


\subsection{Kohn-Nirenberg and Wick quantizations}

On $\bR^n$, the Kohn-Nirenberg quantization may be defined for any symbols $\sigma\in \cS'(\bR^n\times \bR^n)$ via
the formula 
$$
\Op^{\rm KN} (\sigma)f (x)
=\int_{\bR^n} e^{2i\pi x\xi } \sigma(x,\xi)\, \widehat f (\xi) \, d\xi, \qquad x\in \bR^n,\, f\in \cS(\bR^n), 
$$
where $\widehat f = \cF f$ denotes the Euclidean Fourier transform of $f$: 
$$
\cF f(\xi)=
\widehat f(\xi) = \int_{\bR^n} e^{-2i\pi x\xi }f(x)dx, \qquad \xi\in \bR^n.
$$
With the \emph{convolution kernel} $\kappa_x := \cF^{-1}\sigma(x,\cdot\,)$ of $\sigma$, this may be rewritten as 
 $$
\Op^{\rm KN} (\sigma)f (x)
=f*\kappa_x(x), \qquad x\in \bR^n,\, f\in \cS(\bR^n).
$$

Fixing a continuous, bounded and square-integrable function $a$ with $\|a\|_{L^2(\bR^n)}=1$, we 
set for any $f\in L^2(\bR^n)$ and $(x,\xi)\in \bR^n$
$$
\cB_a [f] (x,\xi) := \cF (f \, a (\cdot -x)) (\xi)
= \int_{\bR^n} f(y)\, a(y-x) \, e^{-2i\pi y\xi } dy.
$$
This defines the generalised Bargmann transform $\cB_a = \cB$. The function $a$ is usually chosen as the Gaussian function $a(x)=\pi^{-\frac d4}{\rm e}^{-\frac{|x|^2}{2}} $ \cite{corobook}.  It is a unitary transformation $L^2(\bR^n)\to L^2(\bR^n\times \bR^n)$.
Denoting by $\cB^*$ its adjoint, we define the Wick quantization $\Op^{\rm Wick} = \Op^{{\rm Wick},a}, $ for any symbol $\sigma\in L^\infty (\bR^n\times \bR^n)$ via:
$$
\Op^{\rm Wick} (\sigma) f= \cB^* \sigma \cB[f], 
\qquad f\in L^2(\bR^n).
$$
This quantization has the advantage of yielding bounded operators on $L^2$, of preserving self-adjointness:
$$
\|\Op^{\rm Wick} (\sigma)\|_{\sL(L^2(\bR^n)}
\leq \|\sigma\|_{L^\infty(\bR^n\times \bR^n)},
\qquad
\Op^{\rm Wick} (\sigma)^*=\Op^{\rm Wick} (\bar \sigma),
$$
and of preserving positivity:
$$
\sigma(x,\xi)\geq 0 \ \mbox{for all}\ (x,\xi)\in \bR^n\times\bR^n \Longrightarrow
(\Op^{\rm Wick} (\sigma)f,f)_{L^2(\bR^n)}\geq 0.
$$

The link between the Kohn-Nirenberg and Wick quantization for a bounded symbol is the following:
\begin{lemma}
\label{lem_kappaWRn}
Let $a\in \cS(\bR^n)$ with $\|a\|_{L^2(\bR^n)}=1$, and consider the associated Wick quantization.
For any symbol $\sigma\in L^\infty(\bR^n\times \bR^n)$, we have: 
$$
\Op^{\rm Wick} (\sigma) f (x) = f*\kappa^{\rm Wick}_x(x),\quad f\in \cS(\bR^n), \ x\in \bR^n, 
$$	
where $\kappa^{\rm Wick}\in \cS'(\bR^n\times \bR^n)$ is given by 
$$
\kappa_x^{\rm Wick}(y) = \int_{\bR^n} a(z-y) \bar a(z) \kappa_{x-z}(y) dz,
$$
where $\kappa_x =\cF^{-1}\sigma(x,\cdot)$ denotes the convolution kernel of $\sigma$. 
Hence, $\Op^{\rm Wick}(\sigma) =\Op^{\rm KN}(\sigma^{\rm Wick})$ where $\sigma^{\rm Wick}\in \cS'(\bR^n\times \bR^n)$ is the symbol given by $\sigma^{\rm Wick}(x,\xi)=\cF \kappa^{\rm Wick}_x(\xi)$. 
\end{lemma}



\subsection{G\aa rding inequality for H\"ormander symbols}

First, let us recall the definition of the H\"ormander classes of symbols.
\begin{definition}
	A function $\sigma\in C^\infty(\bR^n\times \bR^n)$ is a H\"ormander symbol of order $m\in \bR$ and index $(\rho,\delta)$
	when $$
\forall \alpha,\beta\in \bN_0^n \quad\exists C_{\alpha,\beta}>0
\qquad 
\forall (x,\xi)\in \bR^n\times\bR^n\qquad 
|\partial_x^\beta\partial_\xi^\alpha \sigma(x,\xi) |
\leq 
C_{\alpha,\beta} (1+|\xi|^2)^{\frac {m-\rho|\alpha|+\delta|\beta|}2}.
$$
The space of H\"ormander symbols of order $m$ and index $(\rho,\delta)$ is denoted by $S^m_{\rho,\delta}(\bR^n)$.
\end{definition}
The space $S^m_{\rho,\delta}(\bR^n)$ is naturally equipped with a structure of Fr\'echet space, inherited by the resulting class of operators
 $\Psi_{\rho,\delta}^m(\bR^n):=\Op^{\rm KN}(S_{\rho,\delta}^m(\bR^n))$. Moreover, 
the H\"ormander calculus $\cup_{m\in \bR} \Psi_{\rho,\delta}^m(\bR^n)$ is a calculus in the sense of Definition \ref{def_pseudo-diff_calculus}. In this context, 
 the link between the Kohn-Nirenberg and Wick quantizations is given in Part (2) of the following statement.
\begin{proposition}
\label{prop_OpWRn}
\begin{enumerate}
\item Let $m\in \bR$ and $1\geq \rho\geq \delta\geq 0$. 
If $\varphi\in \cS(\bR^n)$ and  $\sigma\in S_{\rho,\delta}^m(\bR^n)$
then 
$$
\varphi*\sigma:(x,\xi) \longmapsto \int_{\bR^n} \varphi(z)\sigma(x-z,\xi) = \varphi*\sigma (\, \cdot\, ,\xi)\, (x), 
$$
defines a symbol in $S_{\rho,\delta}^m(\bR^n)$.
\item 
We assume that $a\in \cS(\bR^n)$ with $\|a\|_{L^2(\bR^n)}=1$. 
If  $\sigma\in S_{\rho,\delta}^0(\bR^n)$, then 
$$
	\Op^{\rm Wick}(\sigma) - \Op^{\rm KN}(|a|^2* \sigma) \in \Psi_{\rho,\delta}^{-(\rho-\delta)}(\bR^n).
	$$
 \end{enumerate}
\end{proposition}
\begin{proof}[Sketch of the proof of Proposition \ref{prop_OpWRn}]
Part (1) is easily checked. For Part (2),
we may rephrase Lemma \ref{lem_kappaWRn}  using the notion of difference operators $\Delta_q$ 
defined formally via:
$$
(\Delta_q \sigma )(x,\xi) = \cF \left(q \cF^{-1} \sigma(x,\cdot)\right)(\xi) = \cF (q \kappa_x)(\xi) = \widehat q * \sigma_x \,(\xi), 
$$
with $\kappa_x =\cF^{-1}\sigma(x,\cdot)$ the convolution kernel of $\sigma$. We have:
$$
\Op^{\rm Wick}(\sigma) = \Op^{\rm KN}(\sigma^{\rm Wick}), 
\quad\mbox{with} \quad
\sigma^{\rm Wick} (x,\xi)= \int_{\bR^n} \Delta_{q_z} \sigma (x-z,\xi) dz,
$$
where $q_z(w) = a(z-w)\bar a(z)$.
We then conclude with the following asymptotic expansion in any symbol class $S^m_{\rho,\delta}(\bR^n)$  for a difference operator associated with $q\in \cS(\bR^n)$ 
$$
\Delta_q \sigma \ \sim \ q(0)  \sigma +  \sum_{\alpha>0} c_\alpha \partial^\alpha_x q(0) \partial_\xi^\alpha \sigma,
$$
with explicit coefficients $c_\alpha$.
\end{proof}


The properties  of the  Kohn-Nirenberg and Wick quantizations  imply the following G\aa rding inequality:
\begin{theorem}
\label{thm_gardingRn}
Let $m\in \bR$ and $1\geq \rho>\delta\geq 0$.
	Assume that the symbol $\sigma\in S_{\rho,\delta}^{m}(\bR^n)$ satisfies the elliptic condition $\sigma(x,\xi) \geq c (\id  +|\xi|^2)^{m/2}$, 
	for some constant $c>0$.
	Then there exists a constant $C>0$ such that 
	$$
	\forall f\in \cS(\bR^n)\qquad 
	\Re \left (\Op^{\rm KN}(\sigma) f, f\right )_{L^2} \geq -C \|f\|_{H^{\frac{m-(\rho-\delta)}2}}^2 .
$$	 
\end{theorem}

\begin{proof}[Sketch of the proof of Theorem \ref{thm_gardingRn}]
It suffices to show the case $m=0$.
Let $\sigma\in S_{\rho,\delta}^0(\bR^n)$ satisfying $\sigma(x,\xi)\geq c$.
The properties  of the Wick quantization
(especially preserving positivity and Proposition \ref{prop_OpWRn}) imply
\begin{align*}
	\Re (\Op^{\rm KN}(\sigma) f, f)_{L^2}
&\geq  c \|f\|^2_{L^2} 
- \|\Op^{\rm KN} (\sigma -  |a|^2*\sigma )\|_{\sL(L^2(\bR^n))} \|f\|^2_{L^2}
 -  C
\|f\|^2_{H^{\frac{\rho-\delta}2}},
\end{align*}
for some constant $C>0$. 
The operator $\Op^{\rm KN} (\sigma -  |a|^2*\sigma )$ is bounded on $L^2(\bR^n)$ with operator norm estimated by a semi-norm in $\sigma -  |a|^2*\sigma$. We may write this as:
$$
\|\Op^{\rm KN} (\sigma - |a|^2 *\sigma)\|_{\sL(L^2(G))}
\leq C_1 \max_{\substack{|\alpha|\leq a_0\\ |\beta|\leq b_0}} \sup_{x,\xi\in \bR^n}
(1+|\xi|^2)^{-\frac{\delta|\beta|-\rho|\alpha|}2} 
\left|\partial_\xi^\alpha\partial_x^\beta
    (\sigma -  |a|^2*\sigma)(x,\xi)\right|,
$$
for some $a_0,b_0\in \bN$ and $C_1>0$.
We observe that the convolution above is in the variable $x$ only, so that denoting 
$\sigma_{\alpha,\beta}:= (1+|\xi|^2)^{-\frac{\delta|\beta|-\rho|\alpha|}2} 
\ \partial_\xi^\alpha\partial_x^\beta \sigma$, we have:
$$
(1+|\xi|^2)^{-\frac{\delta|\beta|-\rho|\alpha|}2} 
\ \partial_\xi^\alpha\partial_x^\beta
    (\sigma -  |a|^2*\sigma)
    =
\sigma_{\alpha,\beta} -  |a|^2*\sigma_{\alpha,\beta} .
$$
Hence a judicious choice of $a$ in relation with an approximation of the identity will allow us to conclude. For this, we fix  a  function $a_1\in \cS(\bR^n)$ with $\|a_1\|_{L^2(\bR^n)}=1$
and set $a_t(x) :=t^{-n/2} a_1(t^{-1} x)$. 
We observe that $|a_t|^2 = t^{-n}|a_1 (t^{-1}\cdot)|^2$, $t>0$
 is an approximation of the identity.
We then choose $a=a_t$ with $t>0$ small enough so that  the right-hand side above is $\leq c$.
\end{proof}



\begin{thebibliography}{99}

\bibitem{AG}
Serge Alinhac \& Patrick G\'erard. Pseudo-differential operators and the Nash-Moser theorem. Graduate Studies in Mathematics {82}. American Mathematical Society, Providence, RI, 2007.




\bibitem{bouclet}
Jean-Marc Bouclet.
\emph{The semiclassical Garding inequality}.
Online lecture notes available at \url{https://www.math.univ-toulouse.fr/~bouclet/Notes-de-cours-exo-exam/M2/Garding.pdf}

 \bibitem{CDR}
 {Duv\'an Cardona, Julio Delgado and Michael Ruzhansky}.
 Analytic functional calculus and Gårding inequality on graded Lie groups with applications to diffusion equations, 
 Arxiv:2111.07469.


 \bibitem{CFR}
 {Duv\'an Cardona, Serena Federico and Michael Ruzhansky}.
 Subelliptic sharp Gårding inequality on compact Lie groups.
 Arxiv:2110.00838

\bibitem{corobook} Monique Combescure and Didier Robert.  Coherent states and applications in mathematical physics. {\it Theoretical and Mathematical Physics}. Springer, Dordrecht, 2nd Edition (2021). 

\bibitem{dixmier}
 {Jacques Dixmier}.
{{$C\sp*$}-algebras.}.
  {Translated from the French by Francis Jellett.},
 \textit{North-Holland Publishing Co., Amsterdam-New York-Oxford,}
 {1977}.
 


\bibitem{FF0} Clotilde Fermanian Kammerer \& V\'eronique Fischer. 
Defect measures on graded Lie groups, {\it Ann. Sc. Norm. Super. Pisa.} Vol \textbf{21} 5 (2020), p. 207-291.


\bibitem{FF1}  Clotilde Fermanian Kammerer \& V\'eronique Fischer. Semi-classical analysis on H-type groups, {\it Science China, Mathematics}, \textbf{62} No. 6: 1057-1086, 2019. 

\bibitem{FF2}  Clotilde Fermanian Kammerer \& V\'eronique Fischer. Quantum evolution and sub-laplacian operators on groups of Heisenberg type, {\it Journal of Spectral Theory}, \textbf{11} (3) (2021), p. 1313-1367

 
  \bibitem{FJFA2015} 
 V\'eronique Fischer.
 {Intrinsic pseudo-differential calculi on any compact {L}ie
              group},
 \emph{J. Funct. Anal.},
\textbf{268},
{2015},
No {11},
 pp {3404--3477}.
 
 
 \bibitem{FJFA2020} 
 V\'eronique Fischer. 
 {Differential structure on the dual of a compact {L}ie group},
 \emph{J. Funct. Anal.}, 
\textbf{279},
  {2020},
    No {3},
   pp {108555, 53}.

 \bibitem{FR} V\'eronique Fischer \& Michael Ruzansky.  {Quantization on nilpotent Lie groups},
 {\it Progress in Mathematics}, 314,
 Birkh\"auser Basel, 2016.
 
 \bibitem{FRSob} V\'eronique Fischer \& Michael Ruzansky. {Sobolev spaces on graded {L}ie groups},
 \emph{Ann. Inst. Fourier (Grenoble)},
 \textbf{67},
  {2017},
  No {4},
  pp {1671--1723}.
  
  \bibitem{FRCras} 
  V\'eronique Fischer \& Michael Ruzansky.  
  {Lower bounds for operators on graded {L}ie groups},
 \emph {C. R. Math. Acad. Sci. Paris},
\textbf{351},
  {2013},
 No {1-2},
 pp {13--18}.
 
 \bibitem{folland75}
 {Gerald Folland}.
{Subelliptic estimates and function spaces on nilpotent {L}ie
              groups},
\emph{Ark. Mat.},
\textbf {13},
 {1975},
 No {2},
pp {161--207}.
  
\bibitem{folland+stein_82}
 {Gerald Folland \& Elias Stein}.
 {Hardy spaces on homogeneous groups},
    {\it Mathematical Notes},
     {28},
 {Princeton University Press},
  {1982}.

\bibitem{folland}
 {Gerald Folland}.
 {Harmonic analysis in phase space},
\emph{Annals of Mathematics Studies}
\textbf {122},
{Princeton University Press, Princeton, NJ},
 {1989}.

 \bibitem{ho}Lars H\"{o}rmander. The Analysis of Linear Partial 
Differential  Operators I-III Springer Verlag (1983-85). 
   
\bibitem{Lerner} 
 {Nicolas Lerner},
{Metrics on the phase space and non-selfadjoint
              pseudo-differential operators},
\emph{Pseudo-Differential Operators. Theory and Applications} 
   \textbf  {3},
{Birkh\"{a}user Verlag, Basel},
 {2010},

\bibitem{RT}
 {Michael Ruzhansky \& Ville Turunen}.
 {Pseudo-differential operators and symmetries},
 \emph{Pseudo-Differential Operators. Theory and Applications},
 \textbf{2},
 {Birkh\"{a}user Verlag, Basel},
  {2010}.
  
  \bibitem{RTGarding}
 {Michael Ruzhansky \& Ville Turunen}.
{Sharp {G}\aa rding inequality on compact {L}ie groups},
\emph{J. Funct. Anal.},
 \textbf{260},
  {2011},
  No {10},
   pp. {2881--2901}.
   
\bibitem{RTW}
 {Michael Ruzhansky, Ville Turunen, and Jens  Wirth}.
 {H\"{o}rmander class of pseudo-differential operators on
              compact {L}ie groups and global hypoellipticity},
  \emph{J. Fourier Anal. Appl.},
 \textbf{20},
{2014},
 No {3}.
 


 
 \bibitem{Taylor}
 {Michael Taylor}.
  {Noncommutative harmonic analysis},
  \textit{Mathematical Surveys and Monographs},
  \textbf{22},
{American Mathematical Society, Providence, RI},
 {1986}.


 \bibitem{varo}
 {Nicolas Varopoulos, Laurent Saloff-Coste, and Thierry Coulhon.}
{Analysis and geometry on groups},
  \emph{Cambridge Tracts in Mathematics}
  \textbf{100},
{Cambridge University Press, Cambridge},
 {1992}.

 \bibitem{Zwobook}
Maciej Zworski.
 { Semiclassical analysis}, volume \textbf{138} of {\it Graduate Studies in
  Mathematics}.
\newblock American Mathematical Society, Providence, RI, 2012.
 
\end{thebibliography}


 \end{document} 
