\section{Conclusion}
 %-- provide formal understanding of how stateful defenses prevent black-box attacks.
 %-- We characterize the trade-off between detection and false positive by providing bounds on the detection rate. Our analysis shows that this trade-off depends on the distribution of attack queries, the distribution of natural queries and the properties of the feature extractor.
%-- Our analysis can help understand why some defenses work better than others and can help guide the design choices for future defenses.
In conclusion, our work offers a more formal understanding of how stateful defenses prevent black-box adversarial attacks. We outlined a crucial trade-off between detecting attack detection and false positives, and highlighted its dependence upon the distribution of attack and natural queries, and the properties of the defense's feature extractor. Our analysis can help illuminate why certain defenses perform better against black-box attacks, which can help to refine current strategies and potentially guide the design of future defenses. As the landscape of adversarial attacks and defenses evolves, our findings contribute to the development of more robust and resilient machine learning models under the realistic black-box threat model.