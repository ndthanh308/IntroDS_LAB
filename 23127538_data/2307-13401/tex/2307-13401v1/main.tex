\documentclass[conference]{IEEEtran}

%\usepackage[subtle,title=tight]{savetrees} 
%\usepackage[small,compact]{titlesec}
% \usepackage{amsthm}
% \usepackage[english]{babel}
\usepackage{ifthen}
\usepackage{xcolor}
% \usepackage{blindtext}
% \usepackage{algorithm}
% %\usepackage{subfigure}
% \usepackage{graphicx}
\usepackage{amsmath}
% \usepackage[noend]{algpseudocode}
% \usepackage{subfig}
\usepackage{xspace}
\usepackage{amssymb} 
\usepackage{multirow}
\usepackage{url}
\usepackage{hyperref}
% \usepackage{balance}
% \newtheorem{remark}{Remark}
% \usepackage{graphicx}
% \usepackage{footnote}
% \usepackage{comment}
\usepackage{array}
\newcolumntype{P}[1]{>{\centering\arraybackslash}p{#1}}

\usepackage{url}
\def\UrlBreaks{\do\/\do-}

\newcommand{\exclude}[1]{}
\newcommand{\showComments}{yes}
\newcommand{\note}[2]{
    \ifthenelse{\equal{\showComments}{yes}}{\textcolor{#1}{#2}}{}
}
\newcommand{\TODO}[1]{%
  \addcontentsline{tdo}{todo}{\protect{#1}}%
  \note{red}{TODO: #1}
}

\newcommand\numberthis{\addtocounter{equation}{1}\tag{\theequation}}

\newcommand{\frank}[1]{\note{brown}{[WW: #1]}}
\newcommand{\manya}[1]{\note{red}{[MG: #1]}}
\newcommand{\naader}[1]{\note{brown}{[NH: #1]}}
\newcommand{\kayvon}[1]{\note{violet}{[KS: #1]}}
\newcommand{\ying}[1]{\note{green}{[YZ: #1]}}


\newcommand{\psass}{\ensuremath{\mathbin{{=}}\ }}
\newcommand{\addeq}{\ensuremath{\mathbin{{+}{=}}\ }}
\newcommand{\subeq}{\ensuremath{\mathbin{{-}{=}}\ }}
\newcommand{\muleq}{\ensuremath{\mathbin{{\times}{=}}\ }}
\newcommand{\diveq}{\ensuremath{\mathbin{{\divides}{=}}\ }}
\newcommand{\eqeq}{\ensuremath{\mathbin{{=}{=}}\ }}
\newcommand{\todo}[1]{{\color{red} #1}}
\newcommand{\name}{{\sc{OEAINet}}\xspace}
\newcommand{\cc}{{\sc{c$^2$}}\xspace}

\newcommand{\fattree}{{Clos}\xspace}
\newcommand{\fattrees}{{Clos}\xspace}
\newcommand{\LBE}{{\fattree}\xspace}
\newcommand{\SBE}{{Ideal Switch}\xspace}
\newcommand{\OBE}{{Oversub. \fattree}\xspace }
\newcommand{\para}[1]{{\textbf{{#1}}}}
\newcommand{\net}{{{Big-Net}}\xspace}
\newcommand{\MP}{{MP}\xspace}
\newcommand{\allreduce}{{AllReduce}\xspace}
\newcommand{\ata}{{All-to-All}\xspace}
\newcommand{\allgather}{{AllGather}\xspace}
\newcommand{\redsca}{{Reduce-Scatter}\xspace}
\newcommand{\fbd}{{Full-Bisection Domain}\xspace}
\newcommand{\dcd}{{Direct-Connected Domain}\xspace}



\newcommand{\captionvspace}{0em}
\pagestyle{plain}


\newenvironment{CompactItemize}
  {\def\usecounter{\compactify\latexusecounter}
   \begin{itemize}}
  {\end{itemize}\let\usecounter=\latexusecounter}
\date{}
\usepackage{lipsum} % for dummy text
\usepackage{enumitem}
\setlist{nosep} % or \setlist{noitemsep} to leave space around whole list

\graphicspath{{./graphics/}}

\begin{document}

\title{
Longer Is Shorter: Making Long Paths to Improve the Worst-Case Response Time of DAG Tasks}

\author{Qingqiang He\textsuperscript{1},
Nan Guan\textsuperscript{2},
Mingsong Lv\textsuperscript{1}
\\
\\
\textsuperscript{1}The Hong Kong Polytechnic University, China\\
\textsuperscript{2}City University of Hong Kong, China\\
}


\maketitle
\begin{abstract}\
DAG (directed acyclic graph) tasks are widely used to model parallel real-time workload.
The real-time performance of a DAG task not only depends on its total workload, but also its graph structure. Intuitively, with the same
total workload, a DAG task with looser precedence constraints tends to have better real-time performance in terms of worst-case response time.
However, this paper shows that actually we can shorten the worst-case response time of a DAG task by carefully adding new edges and constructing longer paths.
We develop techniques based on the state-of-the-art DAG response time analysis techniques to properly add new edges so that the worst-case response time bound guaranteed by formal analysis can be significantly reduced.
Experiments under different parameter settings demonstrate the effectiveness of the proposed techniques.
\end{abstract}


%\begin{IEEEkeywords}
%dependency, DAG task, real-time scheduling
%\end{IEEEkeywords}


\section{Introduction}
\label{sec:introduction}
\section{Introduction}
Current quantum hardware is unable to carry out universal quantum computations due to the buildup of errors that occur during the computation. 
The magnitude of the individual error is currently above the value that the Threshold Theorem requires in order to kick-start quantum error correction and fault-tolerant quantum computation~\cite[Section 10.6]{nielsen_chuang_2010}. 
Although the experimentally achieved fidelity rates are promising and the error bounds are inching closer to the required threshold, we will have to work for the foreseeable future with quantum hardware with errors that build-up during the computation.  This implies that we can only do a limited number of steps before the output of the computation has become completely uncorrelated with the intended one.

For fault-tolerant quantum computing, we repeat four steps: 
1) We apply a number of single and two-qubit quantum gates, in parallel whenever possible; 
2) We perform a syndrome measurement on a subset of the qubits; 
3) We perform fast classical computations to determine which errors have occurred and how to correct them; 
and, 4) We apply correction terms based on the classical computations.
We then repeat these four steps with a next sequence of gates. 
These four steps are essential to fault-tolerant quantum computing. 


The starting point of this work is to use the four steps outlined above, not to carry out error correction and fault-tolerant computation, but to enhance short, constant-depth, {\em uncorrected} quantum circuits that perform single qubit gates and {\em nearest-neighbor} two qubit gates. 
Since in the long run we will have to implement error-correction and fault-tolerant computation anyhow, and this is done by such a four-step process, why not make other use of this architecture? Moreover, on some of the quantum hardware platforms, these operations are already in place.
Embracing this idea we naturally arrive at the question: what is the computational power of \textit{low-depth} quantum-classical circuits organized as in the four steps outlined above? 
We thus investigate circuits that execute a small, ideally constant, number of stages, where at each stage we may apply, in parallel, single qubit gates and {\em nearest-neighbor} two qubit gates, followed by measurements, followed by low-depth classical computations of which the outcome can control quantum gates in later stages. 
It is not clear, at first, whether such circuits, especially with constant depth, can do anything remotely useful. 
But we will see that this is indeed the case: many quantum computations can be done by such circuits in constant depth. 
By parallelizing quantum computations in this way, we improve the overall computational capabilities of these circuits, as we do not incur errors on qubits that are idle, simply because qubits are not idle for a very long time. 
Furthermore, reducing the depth of quantum circuits, at the cost of increasing width, allows the circuit to be run faster even if errors occur.

The first usage of such a four-step layout, not to do error correction, but to perform computations, can be found in the paradigm of measurement-based quantum computing~\cite{gottesman1999demonstrating,raussendorf2001one,jozsa2006introduction,clark2007generalised}: 
A universal form of quantum computing where a quantum state is prepared and operations are performed by measuring qubits in different bases, depending on previous measurements and intermediate measurements.

\citeauthor{PhamSvore2013} were the first to formalize the four-step protocol for performing computations~\cite{PhamSvore2013}. They included specific hardware topologies by considering two-dimensional graphs for imposing constraints on qubit interactions. In their model, they develop circuits for particularly useful multi-qubit gates, including specifying costs in the width, number of qubits, depth, number of concurrent time steps, size, and total number of non-Identity operations.
As a result, they find an algorithm that factors integers in polylogarithmic depth.
\citeauthor{Browne:2011} showed that the main tool in the work by \citeauthor{PhamSvore2013}, the fan-out gate, can also be replaced by additional log-depth classical computations in the measurement-based quantum computing setting~\cite{Browne:2011}.

More recently, \citeauthor{Cirac:2021} introduced a scheme to implement unitary operations involving quantum circuits combined with Local Operations and Classical Communication ($\mathsf{LOCC}$) channels: $\mathsf{LOCC}$-assisted quantum circuits~\cite{Cirac:2021}. Similarly to the four-step scheme we just described, they allow for a short depth circuit to be run on the qubits, followed by one round of $\mathsf{LOCC}$, in which ancilla qubits are measured and local unitaries are applied based on the measurement outcomes. They show that in this model any 1D transitionally invariant matrix-product state (MPS) with fixed bond dimension is in the same phase of matter as the trivial state. Similar ideas can be found in~\cite{TVV_NonAbelianTopologicalOrder_2022, tantivasadakarn2021long}.

In this work, we introduce a new model, called \textit{Local Alternating Quantum-Classical Computations} ($\LAQCC$). In this model we alternate between running quantum circuits (constrained by locality), ending in the measurement of a subset of qubits, and fast classical computations based on the measurement results. The outcome of the classical computations are then used to control future quantum circuits. We allow for flexibility in this model, by giving different constraints to the power of both the quantum circuits and the classical circuits as well as the number of alternations between them. 
Most attention will be given to $\LAQCC$ containing quantum circuits of constant depth, classical circuits of logarithmic depth and at most a constant number of alternations between them. 
Any circuit constructed in this model is considered to be of constant depth. 
We restrict ourselves to logarithmic depth classical computations, as this is the first natural and non-trivial extension beyond constant-depth classical computations. 
Constant-depth classical computations do however also have an equivalent constant-depth quantum implementation.

The definition of $\LAQCC$ sharpens the original definition of \citeauthor{PhamSvore2013} by adding constraints to the intermediate classical computations. This allows us to bound the power of $\LAQCC$ from above. 

The main result of \citeauthor{Cirac:2021}, that 1D translational invariant MPS with fixed bond dimension can be prepared by $\mathsf{LOCC}$-assisted circuits, relies on local symmetries of the MPS. These symmetries allow them to prepare local states (on a constant number of qubits) and glue them together by doing one round of the appropriate entangling measurement and corrections, after which they run a round of local unitaries to get the desired result. This general scheme for preparing states that exhibit an MPS description with the appropriate local symmetries requires only geometrically local unitaries and one round of measurement and corrections an therefore is accessible in $\LAQCC$. Studying different local symmetries, known as Symmetry Protected Topological (SPT) phases of matter, to find measurement-based constant depth circuits for states is a broad ongoing field of research~\cite{TVV_NonAbelianTopologicalOrder_2022, tantivasadakarn2021long, smith2023deterministic}. 
All these schemes have a $\LAQCC$ implementation.

%$\LAQCC$-circuits also exist for general schemes of preparing local states, based on the local tensors, and gluing them together using one round of entangled measurement and corrections, based on the local symmetry. 
%The main result of \citeauthor{Cirac:2021}, that 1D translational invariant MPS with fixed bond dimension can be prepared by $\mathsf{LOCC}$-assisted circuits, relies heavily on local symmetries of the MPS and as a result also has an equivalent $\LAQCC$ implementation. 
%The corrections applied after the measurement round are local unitaries depending on the local symmetries of the MPS. 

 

%This general scheme of preparing local states, based on the local tensors, and gluing it together by doing one round of entangled measurement and corrections, based on the local symmetry, is accessible in $\LAQCC$.
Note however that \citeauthor{Cirac:2021} also suggest a circuit for the $W$-state.
This circuit uses sequentially and dependent measurement-based corrections of the ancilla qubits. 
These dependent measurements translate to sequential alternations between the quantum and classical circuits and therefore increase the total depth to linear depth, exceeding the constant-depth constraints imposed by $\LAQCC$-circuits. 

We study the power of the $\LAQCC$ model with respect to state preparation, showing that even with only constant quantum-depth and logarithmic classical depth it remains possible to prepare states with long-range entanglement.
Another surprising result is that it is unlikely that $\LAQCC$ circuits are classically simulatable. We show that any instantaneous quantum polynomial-time (IQP) circuit~\cite{Bremner2010,Shepherd2009} has an $\LAQCC$ implementation.
Classical simulation of IQP circuits implies the collapse of the polynomial hierarchy to the third level, which is not believed to be true~\cite{Bremner2017}. Therefore, we expect that $\LAQCC$ circuits are unlikely to be classically simulatable. We bound the power of $\LAQCC$ by showing that it is contained in $\QNC^1$, the class of polynomial-size, log-depth circuits.

Next, we also study the power that intermediate classical calculations can add to quantum computations, by considering a new model that alternates between polynomially many polynomial-depth quantum circuits and unbounded classical computations
We study this model by doing a complexity theoretical analysis, where we draw inspiration from the notions of complexity given by \citeauthor{RosenthalYuen:2022}, \citeauthor{MetgerYuen:2023}, and \citeauthor{Aaronson:2004}.
All three complexity notions are based on the notion of state preparation, instead of more traditional definition of complexity such as the decidability of a computational problem. 
The first two consider classes based on sequences of quantum states preparable by a polynomial-sized quantum circuit, where the circuits are uniformly generated by a computational class, for instance, the class $\mathsf{PSPACE}$, which results in the complexity class $\mathsf{StatePSPACE}$~\cite{RosenthalYuen:2022,MetgerYuen:2023}.
The third notion considers a relative complexity, where the complexity is measured between two given states, and is measured by the number of gates, from a given gate-set, required to transform one state in another state~\cite{Aaronson:2004}. 
For our definition of state preparation complexity, we drop the uniformity constraint from~\cite{RosenthalYuen:2022,MetgerYuen:2023} and define a class as $\mathsf{StateX}$, which refers to states preparable by circuits of type $\mathsf{X}$. 
As an example, if $\mathsf{X} = \QNC^0$, this results in the class $\mathsf{StateQNC^0}$, which is the set of states preparable from the $\ket{0}^n$ state by poly-size constant-depth circuits. 
This notion is similar to the relative complexity from~\cite{Aaronson:2004}, where one state is the  $\ket{0}^n$ state and instead of counting the number of gates we consider the set of states preparable by a fixed number of gates. Using this notion of complexity we show that any state preparable by an $\LAQCC^*$ circuit is also preparable by a $\mathsf{PostQPoly}$ circuit, the class of circuits of polynomial depth with an additional post-selection gate. 

All Clifford circuits have a constant-depth $\LAQCC$ implementation, implying that any stabilizer state can be implemented by a constant-depth $\LAQCC$ circuit, see Section~\ref{sec:clifford_circuits} for a proof of this statement. 
Efficient circuits for stabilizer states have been known already through measurement-based quantum computing. Therefore this paper focuses on the preparation of non-stabilizer states, and as a surprising result we find novel constant-depth protocols for four very natural classes of non-stabilizer states.
Despite the extensive research into these four classes of non-stabilizer states and the many applications of them, no efficient constant- or low-depth state preparation protocols are known yet. We specifically consider these four classes as they are all often used as initial states in other algorithms.

The first state is a uniform superposition over an arbitrary number of states. 
This state finds applications in many quantum algorithms, as they often start with a uniform superposition over multiple states. 
This superposition is often achieved by applying Hadamard gates to every qubit due to its simplicity to prepare. 
Yet, the analysis of many algorithms, such as Shor's algorithm~\cite{Shor:1997}, would benefit from a different initial superposition. 
The circuit to prepare the uniform superposition over an arbitrary number of states uses an exact version of Grover search as a subroutine, that turns a probabilistic circuit, with a known constant probability of success, into a deterministic circuit. 
We use the circuit for preparing a uniform superposition over an arbitrary number of states as a subroutine in the next two quantum state preparation protocols. 

The second state is the $W$-state, the uniform superposition over all computational basis states of Hamming-weight~$1$, a natural long-ranged entangled state that displays a fundamentally nonequivalent type of entanglement from the Greenberger–Horne–Zeilinger state~\cite{WState:2000}, for which $\LAQCC$-type constant-depth circuits were previously known~\cite{PhamSvore2013, Cirac:2021}. 
The $W$-state is often used as benchmark for new quantum hardware~\cite{Haffner2005,Neeley2010,GarciaPerez:2021}. 
A novel way to prepare the $W$-state therefore gives a new way to benchmark different quantum devices with each other. 
A circuit for preparing the $W$-state was given in~\cite{Cirac:2021}, but this implementation requires sequentially alternating measurements followed by local unitaries, which in the $\LAQCC$ model is not considered to be of constant depth. 
We improve this protocol by giving an $\LAQCC$ implementation of the $W$-state, based on a compress-uncompress method that links the one-hot and binary encoding of integers.

The third state considered is the Dicke state, a generalization of the $W$-state, a superposition over all computational basis states with Hamming-weight $k$~\cite{Dicke:1954}. 
Dicke states have relevance in various practical settings.
For instance, for quantum game theory~\cite{zdemir2007}, quantum storage~\cite{Bacon_Compress:2006,Plesch:2010}, quantum error correction~\cite{ouyang2014permutation}, quantum metrology~\cite{toth2012multipartite}, and quantum networking~\cite{prevedel2009experimental}. 
Dicke states have been used as a starting state for variational optimization algorithms, most notably Quantum Alternating Operator Ansatz (QAOA)~\cite{Hadfield2019}, to find solutions to problems such as Maximum k-vertex Cover~\cite{Brandhofer2022,cook2020quantum}.
The ground states of physical Hamiltonians describing one-dimensional chains tend to show a resemblance to Dicke states such as states resulting from the Bethe ansatz, making them an ideal starting state when investigating the ground state behavior of these Hamiltonians~\cite{TDL_BetheAnsatzDerivation:2010,B_ExcitedStateQuantumPhaseTransitions:2013,DickeTransitions:2021}. 
For instance, the algorithm by \citeauthor{van2021preparing}, who give an algorithm to prepare the Bethe ansatz eigenstates of the spin-1/2 XXZ spin chain, starts by first preparing a Dicke state~\cite{van2021preparing}. 
A Dicke-state preparation protocol based on the compress-uncompress methodology used in the $W$-state furthermore finds applications in entanglement distillation, where the entanglement of a large state is concentrated on only a few qubits. 
Efficient deterministic circuits for preparing Dicke states have been proposed by \citeauthor{bartschi2019deterministic}~\cite{bartschi2019deterministic, bartschi2022deterministic_short_depth}. 
They provide a quantum circuit of depth $\mathO(k \log(\frac{n}{k}))$, allowing arbitrary connectivity, to prepare a Dicke state, which they conjecture to be optimal when $k$ is constant. 
In this work, we provide a constant-depth $\LAQCC$ circuit below their conjectured bound already for constant $k$. 
However, this does not directly disprove their conjecture, as we allow for intermediate measurements and classical computations. 
More significantly, we even construct constant-depth $\LAQCC$ circuits for $k = \mathO(\sqrt{n})$ greatly improving their bound.
This construction extends the compress-uncompress method for the $W$-state combined with additional subroutines. 

We continue with a log-depth state preparation protocol for the Dicke-state for arbitrary $k$. 
This protocol implements an efficient transformation between the factoradic number representation and the combinatorial number representation of a positive integer. 
The combinatorial number representation relates directly to the Dicke state. 
The provided efficient transformation between number representation systems might be of independent interest. 

We conclude by modifying our protocol for preparing a Dicke-state to a protocol that prepares quantum many-body scar states in constant-depth. 
These states have low entanglement and longer coherence times than states with similar energy density.
These characteristics make many-body scar states interesting to analyze and relevant within physics.
Many-body scar states appear for instance in the AKLT model~\cite{AKLT:1987,MRBAR:2018,MRB:2018} and different spin models~\cite{SI:2019,MOBFR:2020}.
Known methods for preparing these states have polynomial-depth~\cite{Gustafson:2023}, whereas our circuit has constant depth. 

% We conclude by studying the power that intermediate classical calculations can add to quantum computations. 
% In this study, we define a new model that relaxes constant-depth quantum circuits to polynomial depth quantum circuits, log-depth classical calculations to unbounded classical computations and a constant number of alternations to a polynomial number of alternations. 
% We call this model $\LAQCC^*$. 
% We study this model by doing a complexity theoretical analysis, where we draw inspiration from the notions of complexity given by \citeauthor{RosenthalYuen:2022}, \citeauthor{MetgerYuen:2023}, and \citeauthor{Aaronson:2004}.
% All three complexity notions are based on the notion of state preparation, instead of more traditional definition of complexity such as the decidability of a computational problem. 
% The first two consider classes based on sequences of quantum states preparable by a polynomial-sized quantum circuit, where the circuits are uniformly generated by a computational class, for instance, the class $\mathsf{PSPACE}$, which results in the complexity class $\mathsf{StatePSPACE}$~\cite{RosenthalYuen:2022,MetgerYuen:2023}.
% The third notion considers a relative complexity, where the complexity is measured between two given states, and is measured by the number of gates, from a given gate-set, required to transform one state in another state~\cite{Aaronson:2004}. 
% For our definition of state preparation complexity, we drop the uniformity constraint from~\cite{RosenthalYuen:2022,MetgerYuen:2023} and define a class as $\mathsf{StateX}$, which refers to states preparable by circuits of type $\mathsf{X}$. 
% As an example, if $\mathsf{X} = \QNC^0$, this results in the class $\mathsf{StateQNC^0}$, which is the set of states preparable from the $\ket{0}^n$ state by poly-size constant-depth circuits. 
% This notion is similar to the relative complexity from~\cite{Aaronson:2004}, where one state is the  $\ket{0}^n$ state and instead of counting the number of gates we consider the set of states preparable by a fixed number of gates. Using this notion of complexity we show that any state preparable by an $\LAQCC^*$ circuit is also preparable by a $\mathsf{PostQPoly}$ circuit, the class of circuits of polynomial depth with an additional post-selection gate. 

\paragraph{Summary of results}
\begin{itemize}
    \item We give a new definition of a computational model that captures the power of the four step process: applying a constant number of layers of one- and two-qubit gates; performing a syndrome measurement; perform a fast classical computation determining corrections; apply corrections. We call this model \emph{Local Alternating Quantum Classical Computations}, or $\LAQCC$ for short. In this model we bound the allowed quantum operations, intermediate classical calculations, and number of rounds separately. In Section~\ref{sec:LAQCC_model} we define this model and give a list of operations based on results from literature contained in this computational model. In some of these operations we explicitly use that we allow for multiple, but at most constant, rounds  of corrections.
    \item  We show show that there exist $\LAQCC$ circuits that can not be weakly simulated in Section~\ref{sec:IQP_in_LAQCC}. We further show that for every $\LAQCC$ circuit there exists a $\QNC^1$ circuit simulating it perfectly, in Section~\ref{sec:LAQCC_in_QNC1}.
    \item We introduce a new type computational complexity for preparing states and show that the extension of $\LAQCC$ where we allow a polynomial number of rounds and unbounded classical computation, is contained in $\mathsf{PostQPoly}$, the class of polynomial circuits with post-selection, in Section~\ref{sec:Complexity results}.
    \item We show a protocol to prepare the uniform superposition state of size $q$ in $\LAQCC$ using $\mathO(\ceil{\log_2(q)}^2)$ qubits in Section~\ref{sec:superposition_modulo_q}. 
    \item We show a protocol to prepare the $W_n$ state in $\LAQCC$ using $\mathO(n\log(n))$ qubits in Section~\ref{sec:W_state_in_LAQCC}.
    \item We show two ways of preparing the Dicke-$(n,k)$ state. The first method is in $\LAQCC$, works up to $k = \mathO(\sqrt{n})$, uses $\mathO(n^2\log(n))$ qubits, and is found in Section~\ref{sec:dicke:small_k}. The second method is in $\LAQCC\text{-}\mathsf{LOG}$ (an extension of $\LAQCC$ allowing for logarithmic number of alterations instead of constant), works for any $k$, uses $\mathO(\text{poly}(n))$ qubits, and is found in Section~\ref{sec:Dicke_in_LAQCC_LOG}. 
    \item We extend on our $\LAQCC$ method of generating Dicke-$(n,k)$ states for $k = \mathO(\sqrt{n})$ and show a protocol to generate many-body scar states for a particular Hamiltonian in $\LAQCC$ (Section~\ref{sec:many_body_scar}). 
\end{itemize}
Summarized in a table, we provide the following state generation protocols:
\begin{table}[htb]
\centering
\begin{tabular}{l|l|l|l}
\textbf{State description} & \textbf{Width} & \textbf{Depth} & \textbf{Implementation}\\
\hline 
Uniform superposition mod $q$: $\frac{1}{\sqrt{q}} \sum_{i = 0}^{q-1}\ket{i}$ & $\mathO(\ceil{\log^2 q})$ & $\mathO(1)$ & Section~\ref{sec:superposition_modulo_q}\\

$W$-state: $\frac{1}{\sqrt{n}}\sum_{i = 0}^{n-1}\ket{e_i}$ & $\mathO(n \log n)$ & $\mathO(1)$ & Section~\ref{sec:W_state_in_LAQCC}\\

Dicke-$(n,k)$, $k = \mathO(\sqrt{n})$: $\binom{n}{k}^{-1/2}\sum_{x \in \{0,1\}^n: |x| = k} \ket{x}$ &  $\mathO(n^2\log n)$ & $\mathO(1)$ 
&Section~\ref{sec:dicke:small_k}\\

Dicke-$(n,k)$: $\binom{n}{k}^{-1/2}\sum_{x \in \{0,1\}^n: |x| = k} \ket{x}$ & $\mathO(\text{poly}(n))$ & $\mathO(\log n)$ &Section~\ref{sec:Dicke_in_LAQCC_LOG}\\

QMBS: $\ket{S_k} = \frac{1}{k! \sqrt{\mathcal N(n,k)}}(Q^\dagger)^k \ket{\Omega}$ &  $\mathO(n^2\log n)$ & $\mathO(1)$  &  Section~\ref{sec:many_body_scar}
\end{tabular}
\caption{Summary of state preparation protocols given in this paper.}
\label{tab:sate_prep}
\end{table}
In the entry for the quantum many-body scar state $Q$ denotes the raising operator and $\mathcal N(n,k)=\binom{n-k-1}{k}$. 
Section~\ref{sec:many_body_scar} will provide more details on the variables and the implementation. 

\paragraph{Organization of the paper}
\noindent We first introduce relevant preliminaries in Section~\ref{sec:preliminaries}. 
In Section~\ref{sec:LAQCC_model} we formally define the class of Local Alternating Quantum-Classical Computations ($\LAQCC$). We also show that any Clifford circuit can be implemented in constant depth $\LAQCC$ (a result based on a result from measurement-based quantum computing~\cite{jozsa2006introduction}). 
This result allows us to give many useful multi-qubit gates and routines in Section~\ref{sec:gates_created_in_LAQCC}. 
Beyond that we show that constant depth $\LAQCC$ circuits are contained in $\QNC^1$ and that any $\mathsf{IQP}$ circuit has an $\LAQCC$ implementation.
We conclude this section with an analysis of a more powerful instantiation of $\LAQCC$ and show an inclusion with respect to the class $\mathsf{PostQPoly}$, which is the class of circuits of polynomial depth with one additional post-selection gate. 
In Section~\ref{sec:state_prep_in_LAQCC} we give $\LAQCC$ circuit implementations for preparing the uniform superposition over an arbitrary number of states, the $W$-state and the Dicke state up to $k = \mathO(\sqrt{n})$. We furthermore give a log-depth circuit implementation for preparing the Dicke state for any $k$. We conclude by showing a $\LAQCC$ circuit for generating many body scar states of a particular type of Hamiltonian.



\section{System Model}
\label{sec:model}
% !TEX program = pdflatex
% !TEX root = main.tex


\section{The Model}

We represent a series of interactions between $N$ individuals as a sequence of weighted directed networks with adjacency matrix $A^t$ for $t=0,1,2,\ldots,T$. For each $t$, its entry $A_{ij}^t$ is the outcome of interactions $i \rightarrow j$ suggesting that $i$ is ranked above $j$. This allows both cardinal and ordinal inputs. For instance, in team sports, $A_{ij}^t$ could be the number of points by which team $i$ beat team $j$, or we could simply set $A_{ij}^t=1$ to indicate that $i$ won and $j$ lost. We can include the case where individuals interact multiple times at time $t$ by summing the corresponding entries.

We assume that the values of $A_{ij}^t$ are influenced by a vector of real-valued ranks $\v{s}^t=(s_{1}^t,\dots, s_{N}^t)$, where $s_i^t$ is $i$'s skill, strength or prestige at time $t$.
To model these interactions, we follow SpringRank's approach of imagining the network as a physical system~\cite{de2018physical}. Specifically, each node $i$ is embedded in $\mathbb{R}$ at position $s_i^t$, and each directed edge $i \rightarrow j$ becomes an oriented spring with a non-zero resting length and displacement $s_i^t-s_j^t$. Since we are free to rescale latent space and the energy scale, we set the spring constant and resting length to $1$. The spring corresponding to an edge $i \rightarrow j$ at time $t$ then has energy
\be\label{eqn:staticH}
H_{ij}(s_i^t,s_j^t)=\f{1}{2} \bup{s_i^t-s_j^t-1}^{2} \, .
\ee
If there were no other effects, the total energy of the system at time $t$ would then be 
\be\label{eqn:totalstaticH}
H^t(\v{s}^t) = \sum_{i,j=1}^{N} A_{ij}^t \,H_{ij}(s_i^t,s_j^t) \, .
\ee
If we determined $\v{s}^t$ by minimizing $H^t$ for each $t$ separately, we would simply be applying the static SpringRank model separately to each ``snapshot'' of the network. This would ignore all previous (and future) interactions, and ignore the hypothesis that ranks change smoothly from one time-step to the next.

% Figure environment removed

To model this smoothness, we also assume a dependence between ranks at successive time-steps. Specifically, we extend the Hamiltonian~\eqref{eqn:totalstaticH} with an extra term that models the \emph{self-interaction} between past and current ranks,
\begin{equation}\label{eqn:selfH}
\Hself^t(\v{s}^t,\v{s}^{t-1}) 
= \frac{\kself}{2} \sum_{i=1}^N (s_i^t-s_i^{t-1})^2 \, .
\end{equation}
This can be seen as a set of additional ``self-springs'' that connect the rank of each individual with its own previous rank. The spring constant $\kself$ parametrizes how smoothly we want the ranks to change from one step to the next. In inference terms, $\kself$ is a hyperparameter which we tune using cross-validation.

Summing over all time-steps $0 < t \le T$ and adding this to the pairwise interactions at each time-step then gives a total energy

\begin{align}\label{eqn:fullH}
\Htotal(\{\v{s}^t\}) = \sum_{t=0}^T H^t(\v{s}^t) + \sum_{t=1}^T \Hself^t(\v{s}^t,\v{s}^{t-1}) \, .
\end{align}
We call this the dynamical SpringRank Hamiltonian. The optimal ranks $\v{s}^0,\v{s}^1,\ldots,\v{s}^T$ are those that minimize it.


There are two ways to minimize $\Htotal$. One is to proceed in an online way, moving forward in time. In this approach, we use the static SpringRank model Eq.~\eqref{eqn:totalstaticH} to find the initial ranks $\v{s}^0$ by minimizing $H^0(\v{s}^0)$. As in Ref.~\cite{de2018physical}, the energy is unchanged if we add a constant to all the ranks; we can break this translational symmetry by setting the mean initial rank $(1/N) \sum_{i=1}^N v_i^0$ to zero.
Then, at each subsequent time-step $t \ge 1$, we update the ranks by taking into account both the new pairwise interactions and the self-springs connecting the ranks with their previous values. Namely, given $\v{s}^{t-1}$ and $A^t$, we find the ranks $\v{s}^t$ that minimize $H^t(\v{s}^t) + \Hself^t(\v{s}^t,\v{s}^{t-1})$.

Since this is a convex function of $\v{s}^t$, we can find its minimum by setting its gradient to zero, or equivalently by balancing all the forces $v_i^t$. This yields a system of linear equations:
\begin{align}\label{eqn:fullsolution}
\rup{ D^{out,t}+D^{in,t}- \bup{A^t + (A^t)^\dagger}+\kself\id} \,\v{s}^t
&=\rup{D^{out,t}-D^{in,t}}\v{1} \nonumber \\& +\kself\, \v{s}^{t-1} \, . 
\end{align}

Here 
$D^{out,t}$ and $D^{in,t}$ are diagonal matrices whose entries are the weighted out- and in-degrees $D^{out,t}_{ii}=\sum_{j}A^t_{ij}$ and $D^{in,t}_{ii}=\sum_{j}A^t_{ji}$; 
$\dagger$ denotes the transpose; 
$\id$ is the identity matrix; 
and $\v{1}$ is the all-ones vector.

The matrix on the left side of~\Cref{eqn:fullsolution} is invertible if $\kself > 0$. In particular, its eigenvector $\v{1}$ has eigenvalue $N \kself$. Thus for each $A^t$ and each $\v{s}^{t-1}$, Eq.~\eqref{eqn:fullsolution} has a unique solution $\v{s}^t$. Overall, Eq.~\eqref{eqn:fullsolution} is similar to the regularized version of SpringRank~\cite{de2018physical} with regularization parameter $\alpha= \kself$. However, unlike the static model, there is a term on the right-hand side containing the previous ranks $\v{s}^{t-1}$, creating a Markovian dependence between successive time-steps. We refer to this model as \dsrfull\ (\dsr).

Importantly the online DSR approach does not actually minimize $\Htotal$, instead solving a sequence of minimization problems, one for each time step. To minimize $\Htotal$ instead, we set $\nabla \Htotal(\v{s}^t) = 0$, solving for the minimizers $\v{s}^t$ over all $N(T+1)$ ranks simultaneously, yielding the following system of equations (SI \Cref{sec:h_total_derive}):

\begin{align}\label{eqn:h_total}
\rup{ D^{out,t}+D^{in,t} - \bup{A^t+(A^t)^\dagger} + 2\kself\id}\,\v{s}^t 
&=\rup{D^{out,t}-D^{in,t}}\v{1} \nonumber\\ 
& +\kself \,\bup{\v{s}^{t-1} + \v{s}^{t+1}} \, . 
\end{align}
This differs from \Cref{eqn:fullH} in that the right-hand side now includes both past and future ranks (which doubles the contribution of $\kself$ on the left). We remove the terms $\v{s}^{t-1}$ and $\v{s}^{t+1}$ for $t=0$ and $t=T$ respectively. This entire system has translational symmetry, since the energy Eq.~\eqref{eqn:fullH} remains the same if we add the same constant to all ranks at all times, but we can again break this symmetry by setting the mean rank to zero.

Additionally, in contrast to \Cref{eqn:fullsolution}, the ranks at $t$ now depend on both $t-1$ and $t+1$, which themselves depend on ranks at adjacent time-steps, so that ranks are affected by interactions in both the past and the future. In computer science, methods like this where the entire history is provided to the algorithm are called \emph{offline}, to distinguish them from \emph{online} approaches that update their results in real time as data becomes available. Thus we refer to this model as \nmdsrfull\ (\nmdsr).  

The cost of solving \Cref{eqn:fullsolution} for a single time-step is the same as static SpringRank with only one additional parameter to be tuned using cross-validation, and there are $T$ such $N$-dimensional equations to be solved successively. On the other hand, \Cref{eqn:h_total} requires solving a single  system of dimension $NT$, whose operator consists of $T$ blocks, each of dimension $N\times N$. While these two approaches feature numbers of non-zero entries that are fundamentally determined by the number of total edges across all time steps, the cost of solving \dsr vs \nmdsr will depend on the particular choice of linear solver~\cite{peng2021solving}.

Philosophically, Eqns.~\eqref{eqn:fullsolution} and~\eqref{eqn:h_total} are trying to do two different things. If we are given all the data $A^0,A^1,\ldots,A^T$ and we want to infer retrospectively how each individual's rank changed over time, it makes sense to include both past and future interactions as in~\eqref{eqn:h_total} so that $s_i^t$ is affected by $i$'s entire history. 

In contrast, \eqref{eqn:fullsolution} can be viewed as modeling each individual's perceived rank at the time, based only on the interactions that have occurred so far.

In principle, one could envisage other ways to formally incorporate an explicit dependence on  $\v{s}^{t-1}$ into the model, and we provide one example in SI \Cref{sec:sidynl}. However, we found that the approaches presented in this Section provide a natural interpretation, result in good prediction performance on both real and synthetic datasets (see \Cref{sec:results}) and are computationally scalable. 

We close this section with two possible extensions to these models. First, in some settings we might have timestamps $t$ that are not successive integers $0,1,\ldots,T$. In this case, if the time interval between two successive times is $\Delta t$, one could scale the spring constant of the self-springs between time-steps as $\kself/\Delta t$. This corresponds to the fact that if we have $\Delta$ identical springs in series, each of which is stretched by $(s^t-s^{t-1})/\Delta$, their total energy is $(1/2)(\kself/\Delta)(s^t-s^{t-1})^2$. The same expression applies if the timestamps are real-valued so that $\Delta$ is not an integer.

Second, if we believe that not just the ranks themselves but their rates of change behave smoothly over time, one could add a momentum term to the Hamiltonian which is quadratic in the discrete second derivative of the ranks. Since
\begin{gather*}
\left( (s^{t+1}-s^t) - (s^t-s^{t-1}) \right)^2
= \left( s^{t+1} - 2 s^t + s^{t-1} \right)^2 \\
= 2 (s^t-s^{t-1})^2 + 2 (s^{t+1}-s^t)^2 - (s^{t+1} - s^{t-1})^2 \, ,
\end{gather*}
this is equivalent to adding a repulsive force, i.e., a spring with negative spring constant, between ranks two time-steps apart. Note that the system nevertheless remains convex: this momentum term is positive semidefinite, so adding it to~\eqref{eqn:fullH} keeps the coupling matrix positive definite except for translational symmetry. Of course, these terms are second-order in time. In the online approach, one would have to determine $\v{s}^0$ from the static model, $\v{s}^1$ from the first-order model~\eqref{eqn:fullsolution}, and then use the model including this momentum term for $\v{s}^t$ for $t \ge 2$. We have not pursued this here, but it may make sense for certain datasets.


\subsection{Moving-window SpringRank}\label{subsec:mwsr}

Before we test the various versions of \dsrfull\ defined above, we consider a simpler model as a baseline. 
The simplest way to extend SpringRank to a dynamical context is to apply the static model to the interactions in a series of ``windows,'' where in each window we sum the interactions over a series of consecutive time-steps. For instance, we can compute $\v{s}^t$ for each $t$ by applying the static model to a window of width $\tau$, i.e., replacing $A^t$ with $\sum_{t'=t}^{t+\tau-1} A^{t'}$. Since these windows overlap, the resulting estimates $\v{s}^t$ will be smooth to some extent, even without imposing an explicit dependence between $\v{s}^t$ and $\v{s}^{t-1}$. We use this method, which we call \mwsrfull\ (\mwsr), as a baseline to compare with the dynamical models presented above.

Roughly speaking, a larger $\tau$ is like a larger self-spring constant $\kself$, since it induces more overlap between windows and thus a stronger correlation between the inferred ranks. However, like a decaying-history approach, \mwsr\ assumes a particular kernel for the importance of past time-steps: namely, that all $t'$ in the window are equally important. In contrast, \dsrfull\ infers the importance of past time-steps by coupling $\v{s}^t$ with $\v{s}^{t-1}$.

However, both models have a free parameter that needs to be tuned, i.e., $\kself$ and $\tau$. A shorter window $\tau$ or smaller spring constant $\kself$ allows the ranks to respond quickly to new interactions, while a longer window or larger spring constant more tightly couples nearby estimates. This trade-off suggests the existence of an optimal window length $\tau_{\opt}$. We tune $\tau$ using a cross-validation procedure as explained in SI \Cref{sisec:tuning}.


\subsection{Generative Model and Synthetic Data}
\label{sec:genmod}

Analogous to a model presented in~\cite{de2018physical}, we propose a probabilistic generative model for dynamic data. It takes as input the ranks $\v{s}^t$ and generates a sequence of weighted directed networks with adjacency matrix $A^t$ at time $t$. One can also imagine models that generate the ranks, for instance with a random walk with Gaussian steps whose log-probability is the self-spring Hamiltonian~\eqref{eqn:selfH}, but we treat $\v{s}^t$ as an input since we want the user of this model to have control over how the ground-truth ranks vary with time.  For instance, in our experiments below we generate synthetic data where the ranks vary sinusoidally.

The generative model has two real-valued parameters: a signal-to-noise ratio or inverse temperature $\beta$, and an overall density of edges $c$. Given the ranks $\v{s}^t$, it generates weighted, directed edges between each pair of nodes $i,j$ independently, as follows. The probability $P_{ij}^t(\beta)$ of $i$ ``beating'' $j$ at time $t$, giving a directed edge $i \to j$, is a logistic function as in~\cite{de2018physical} or the Bradley-Terry-Luce model~\cite{bradley1952,luce1959}:
\bea
\nonumber P_{ij}^t(\beta)=\frac{1}{1+\e^{-2\beta(s_i^t-s_j^t)}} \, .
\eea
The number of such edges, which gives the integer weight $A_{ij}^t$, is then drawn from a Poisson distribution whose mean $\lambda_{ij}^t$ is $cP^t_{ij}\,(\beta)$: 
\be
\label{generative_poiss}
A^t_{ij} \sim \Poi\left(\lambda_{ij}^t=\frac{c}{1+\e^{-2\beta(s_i^t-s_j^t)}}\right).
\ee
Since $P_{ij}^t(\beta) + P_{ji}^t(\beta)=1$, for any pair $i,j$ the total number of interactions $A_{ij}^t + A_{ji}^t$ is Poisson-distributed with mean $c$. The rank differences $s_i^t-s_j^t$ are used only to choose the directions of these edges. This  is equivalent to a model where we define a random multigraph where the number of edges between $i$ and $j$ is $\Poi(c)$, and then we choose the direction of each edge independently according to $P_{ij}^t$.

This is different from the generative model proposed in the static case in~\cite{de2018physical}. In that model the probability that $i$ and $j$ interact depends on $s_i-s_j$ so that nodes are more likely to interact if their ranks are fairly close. This is consistent with SpringRank's assumption that if $i$ beats $j$ then $j$ is below $i$, but not too far below it (since the springs have resting length $1$). This assumption makes sense for some datasets but not for others. By generating synthetic data without this dependence, our intent is to pose a greater challenge to SpringRank by modeling (for example) round-robin tournaments where every team plays each other.

\subsection{Model Evaluation}
\label{sec:testing}

Assessing a ranking model on real datasets is not straightforward since we do not know the true values of the underlying ranks. Nevertheless, we may measure the extent to which inferred ranks are accurate in the sense that they can predict the outcome of new observations. 

There are several performance metrics that can be used for prediction evaluation. From coarse-grained measures capable of predicting the likely winner to more fine-grained measures that also estimate odds, we consider four main metrics in our experiments, detailed in \Cref{sisec:evaluation}. We measure prediction performance using a cross-validation protocol where datasets are divided into training and test sets. The training set is used for hyperparameter tuning and parameter estimation while performance is evaluated on the test set. In order to preserve the chronological ordering of the data, the test set contains future observations, i.e., observations that chronologically follow those used in training. Hyperparameters for each method are tuned using grid-search in order to maximize the performance metrics as described in SI \Cref{sisec:tuning}.





%%% Local Variables:
%%% mode: latex
%%% TeX-master: "main"
%%% End:


\section{Motivation}
\label{sec:motivation}
\section{Motivation}
\label{sec:motivation}

IGNORE THIS FILE, WILL DO IN INTRO


\section{Making Long Paths}
\label{sec:method}
\section{Method} \label{method_hybridaugment}
In this section, we formally define the problem, motivate our work and then present our proposed techniques.


\subsection{Preliminaries}
Let $\mathcal{F}(x;W)$ be an image classification CNN trained on the training set $\mathcal{T}_\text{train} = (x_{i}, y_{i})^{N}_{i=1}$  with $N$ samples, where $x$ and $y$ correspond to images and labels. The clean accuracy (CA) of $\mathcal{F}(x;W)$ is formally defined as its accuracy over a clean test set $\mathcal{T}_\text{test} = (x_{j}, y_{j})^{M}_{j=1}$. Assume two operators ${A}(\cdot)$ and ${C}(c, s)$ that adversarially attacks or corrupts a given set of images with the corruption category $c$ and severity $s$, respectively.  Let $A\mathcal{T}_\text{test}$ and $C\mathcal{T}_\text{test}$ be the adversarially attacked and corrupted versions of $\mathcal{T}_\text{test}$, and let $\mathcal{F}(x;W)$ have a robust accuracy (RA) on $A\mathcal{T}_\text{test}$ and a corruption accuracy (CRA) on $C\mathcal{T}_\text{test}$. 
The aim is to fit $\mathcal{F}(x;W)$ such that the model gains robustness (\ie. increased RA and CRA compared its the baseline version), while retaining (or improving) the clean accuracy of its baseline version trained without robustness concerns.


\noindent \textbf{What we know.} Our work builds on the following crucial observations: i) CNNs favour high-frequency content \cite{wang2020high}, ii) adversaries and corruptions often reside in high-frequency \cite{wang2020towards}, iii) images are dominated by low-frequency \cite{Saikia_2021_ICCV} and iv) models relying on low-frequency components are more robust \cite{li2022robust,wang2020towards}. The robustness-accuracy trade-off is visible; low-frequency reliant models are more robust, but tend to miss out on clean accuracy brought by the high-frequency components. 

\subsection{HybridAugment}
We hypothesize that a \textit{sweet spot} in the robustness-accuracy trade-off can be found. Unlike the \textit{hard} approaches that completely rule out the reliance on high-frequency components (i.e. low-pass filters), we propose to \textit{reduce} the reliance on them. To this end, we adopt a data augmentation approach that aims to diversify $\mathcal{T}_\text{train}$ by an operation $\mathcal{HA(\cdot)}$. Keeping the strong relation intact between labels and low-frequency content (i.e. labels come from low-frequency-component image), we propose to swap high and low-frequency components of images in a batch on-the-fly. Unlike \cite{mukai2022improving}, we \textit{do not} restrict the images to belong to the same class; this diversifies the training distribution even further while preserving the image semantics. We call this basic version of our approach \textit{HybridAugment}, which corresponds to: 
%
\begin{equation} \label{hybrid_augment_paired}
    \mathcal{HA_{P}}(x_{i}, x_{j}) = \mathcal{LF}(x_{i}) + \mathcal{HF}(x_{j})
\end{equation}
%
where $x_{i}$ is the input image and $x_{j}$ is a randomly sampled image from the whole training set, which we simply sample from the mini batch at each training iteration in practice. $\mathcal{HF}$ and $\mathcal{LF}$ operators select the high and low-frequency components of an input image, for which we use:
%
\begin{equation} \label{eq:cutoff}
\begin{split}
    \mathcal{LF}(x) = GaussBlur(x) \\
    \mathcal{HF}(x) = x - \mathcal{LF}(x)
    \end{split}
\end{equation}
%
where $GaussBlur$ is used as a low-pass filter. Note that a similar outcome is possible by using Discrete Fourier Transforms (DFT), swapping the frequency bands and then applying Inverse DFT (IDFT). We find the gaussian blur operation to be faster and better in practice. 


Inspired from \cite{chen2021amplitude}, in addition to the image-pair scheme in Eq.~\ref{hybrid_augment_paired}, we propose a single image variant of \textit{HybridAugment}. In the single image variant, instead of combining two images, $x_i$ and $x_{j}$ are obtained by applying randomly sampled augmentations to a single image. The single image variant $\mathcal{HA_{S}}$ can therefore be defined as 
%
\begin{equation} \label{hybrid_augment_single}
    \mathcal{HA_{S}}(x_{i}) = \mathcal{LF}(Aug(x_{i})) + \mathcal{HF}(\hat{Aug}(x_{i}))
\end{equation}
%
where $Aug$ and $\hat{Aug}$ correspond to two sets of randomly sampled augmentation operations. Note that paired and single versions can work in tandem ($\mathcal{HA_{PS}}$), and actually outperform single or paired image versions. 


\subsection{HybridAugment++}


The frequency analysis is a vast literature, however, two core aspects often stand out; frequency-band analysis (i.e. low, high) and the decomposition of signals into amplitude and phase. \textit{HybridAugment} covers the former and shows competitive results in various benchmarks (see Section \ref{sec:exp_hybridaugment}). The latter is investigated in $\mathcal{APR}$ \cite{chen2021amplitude}, where phase is shown to be the more relevant component for correct classification, and training models based on their phase labels and swapping amplitude components of images randomly lead to more robust models. Note that frequency-band and phase/amplitude discussions are arguably orthogonal, since frequency, phase and amplitude provide distinct characterizations of a signal: intuitively speaking, frequency, phase and amplitude can be seen as the separation of visual patterns in terms of scale, location and significance. 


We hypothesize these two approaches can be complementary; a model reliant on low-frequency and spatial information (i.e. phase) can further improve robustness. Inspired by the successes of cascaded augmentation methods \cite{hendrycks2019augmix,wang2021augmax,calian2022defending}, we unify these two core aspects into a single, hierarchical augmentation method. We refer to this method as \textit{HybridAugment++} and define its paired version as:
%
\begin{equation}
  \mathcal{HA_{P}}^{++}(x_{i}, x_{j}, x_{z}) = \mathcal{APR_{P}}(\mathcal{LF}(x_{i}), x_{z}) + \mathcal{HF}(x_{j})
\end{equation}
%
where $x_{i}$, $x_{j}$ and $x_{z}$ are images sampled from the same batch. Here, $\mathcal{APR_{P}}$~\cite{chen2021amplitude} is defined as
\begin{equation}
    \mathcal{APR_{P}}(x_{i}, x_{z}) = \mathcal{IDFT}(A_{x_{z}} \otimes e^{i. P_{x_{i}}}) \\
\end{equation}
%
where $\otimes$ is element-wise multiplication, $A$ is the amplitude and $P$ is the phase component. Similar to $\mathcal{HA}$ and $\mathcal{APR}$, we also define a single-image version of \textit{HybridAugment++} as
%
\begin{equation}
 \mathcal{HA_{S}}^{++}(x_{i}) = \mathcal{APR_{S}}(\mathcal{LF}(Aug(x_{i}))) + \mathcal{HF}(\hat{Aug}(x_{i}))
\end{equation}
%
where $\mathcal{APR_{S}}$~\cite{chen2021amplitude} is defined as
%
\begin{equation}
\mathcal{APR_{S}}(x_{i}) = \mathcal{IDFT}\left(A_{\bar{Aug}(x_{i})} \otimes e^{i. P_{\overline{Aug}\left(x_{i}\right)}}\right)    
\end{equation}
%
where $Aug$, $\hat{Aug}$, $\bar{Aug}$ and $\overline{Aug}$ are different sets of randomly sampled augmentation operations. Note that we essentially propose a framework; one can use different single and paired image augmentations, either individually or together, and can still achieve competitive results (see ablations in Section \ref{sec:exp_hybridaugment}). There are also other alternatives, such as swapping phase/amplitude first and then performing $\mathcal{HA}$, but we observe poor performance in practice; dividing the phase component into frequency-bands is not interpretable as frequencies of the phase component are not well defined. The pseudo-code of our methods can be found in the supplementary material.





\section{Real-Time Scheduling of DAG Tasks}
\label{sec:extension}
\section{Extension: Leader is not Perfectly Informed}
\subsection{Alternative Information Structure} \label{sect_noisy_signaling}


Suppose, now, that instead of perfectly learning the state $\theta$, the leader observes a noisy signal $x_L = \theta + \sigma_L \varepsilon_L$ with $\varepsilon_L$ being a standard Gaussian noise independent of $\theta$ and $\varepsilon_j$ for any $j \in F$. We deem this extension important for three reasons. First, from an applied perspective, it is more natural to assume that the leader observes a noisy signal about the state rather than its true value. Second, we document that multiplicity of rationalizable behavior is not a consequence of the one-sided dominance in the subgames. That is, the reason for multiplicity is not that the action of the leader when she is perfectly informed introduces ``too much'' common knowledge. Rather, the main result of the paper continues to hold with the (noisy) information structure that brings back the standard two-sided dominance. Finally, this extension will show that while it must be the case that the leader is fully informed to obtain efficiency, it is not enough for the leader to be better informed than the followers to obtain unique rationalizable behavior. The requirement that the noise in leader's information is smaller than the one in the followers' (i.e., $\sigma_L < \sigma_F$), is neither necessary nor sufficient for uniqueness.  

\subsection{Analysis and Rationalizable Behavior}


Call $x_L \in X_L = \RR$ the leader's type. A strategy for the leader is now a mapping $s_L: X_L \to \Action_L$. Let $S_L$ denote the strategy space for the leader. The actions and strategies of the followers remain the same as defined in Section \ref{sect_signaling_game}.
Let $\mu_L(\cdot\vl x_L) \in \Delta \left(\Theta \times X \times S \right)$ be leader $x_L$'s belief about the state and the type-strategy pairs of the followers, where $X \times S = \prod_{j\in F} (X_j \times S_j)$. Since there is no learning for the leader, the marginal of $\mu_L(\cdot \vl x_L)$ about $\theta$ has a Gaussian distribution with mean $x_L$ and variance $\sigma_L^2$.
Let $\mu_j(\cdot \vl x_j, h) \in \Delta \left(\Theta \times X_L \times X_{-j} \times S_{-j} \right)$ be follower $j$'s belief about 
the state and the type-strategy pairs of his opponents given type $x_j$ and history $h$, where $X_{-j} \times S_{-j} = \prod_{k \neq j \in F} (X_k \times S_k)$.


For type $x_L$ of the leader, exerting effort is the best response to a belief $\mu_L(\cdot \vl x_L)$ if
\begin{equation*}
    \int_{(\theta, x, s)} u(\theta, A_{-L}(s)) \dd \mu_L(\theta, x, s \vl x_L) > 0,
\end{equation*}
where $A_{-L}(s(\invest)) = \sum_{j \in F} \one(s_j(\invest) = \invest)$.
For a follower with type $x_j$, $s_j(h) = \invest$ is the best response to $\mu_j(\cdot \vl x_j, h)$ when
\begin{equation*}
    \int_{(\theta, x_L, x_{-j}, s_{-j})} u\left(\theta, A_{-j}(h, s_{-j}) \right) \dd \mu(\theta, x_L, x_{-j}, s_{-j} \vl x_j, h) > 0
\end{equation*}
where $A_{-j}(h, s_{-j}(h)) = \chi_\invest + \sum_{k \in F, \,k \neq j} \one(s_k(h) = \invest)$. The initial set of type-strategy pairs for the leader in the definition of $\Delta$-rationalizability is now given by $R_L^0 =X_L \times S_L$.


Suppose that a follower type $x$ believes that the leader uses a monotone strategy with threshold $z$; i.e., $a_L = \invest$ if and only if $x_L > z$. Then, 
upon observing $h = \invest$, type $x$'s interim belief has CDF
\begin{equation} \label{posterior_invest}
    G^\invest(\theta; \,x, z) = \frac{1}{ \Phi\left( \frac{x - z}{\sigma}\right) } \int_{-\infty}^{\theta} \frac{1}{\sigma_F} \phi\left(\frac{t- x}{\sigma_F}\right) \Phi\left(\frac{t - z}{\sigma_L}\right)\dd t,
\end{equation}
where $\sigma^2 = \sigma_F^2 + \sigma_L^2$.
Similarly, type $x$'s interim CDF under history $h = \notinvest$ is
\begin{equation} \label{posterior_not_invest}
    G^\notinvest(\theta; \,x, z) = \frac{1}{\Phi \left(\frac{z - x}{\sigma}\right) } \int_{-\infty}^{\theta}
    \frac{1}{\sigma_F} \phi\left(\frac{t-x}{\sigma_F}\right) \Phi\left(\frac{z - t}{\sigma_L}\right)\dd t.
\end{equation}
It is worth noting that $G^h(\cdot; \,x, z)$, unlike the interim beliefs $\Psi^h(\cdot; \,x, z)$ in the main model, has support over the entire real line. Thus, the subgames no longer feature one-sided dominance in the $\Delta$-rationalizability procedure.

If type $x$ believes further that other followers use monotone strategies with threshold $x_h$ under history $h$, then his payoff to choosing $\invest$ yields
\begin{equation*}
    \pi_F^h(x_j; z, x_h) = \EE_{\theta \sim G^h(\cdot; x, z) } \left[\theta - \frac{n-1}{n} \Phi\left(\frac{x_h - \theta}{\sigma_F}\right) \right]  -\frac{\chi_\notinvest}{n}.
\end{equation*}
We show in Appendix A that $\pi_F^h(x; z, x_h)$ is strictly increasing in $x$ and crosses zero once from below. Furthermore, it is strictly increasing in $z$ and strictly decreasing in $x_h$.
Type $x$'s 
conditional rank belief is given by
\begin{equation} \label{ext_rank_belief}
\begin{cases}
    R^\invest(x; z) = \prob(x_k \leq x_j \vl x_j = x, x_L > z) = \frac{1}{2} - \frac{T\left(\frac{x-z}{\sigma},  
 ~\alpha \right)}{\Phi\left(\frac{x - z}{\sigma}\right)} & \\
    ~ & \\
    R^\notinvest(x; z)  = \prob(x_k \leq x_j \vl x_j = x, x_L \leq z) = \frac{1}{2} + \frac{T\left(\frac{z - x}{\sigma}, ~\alpha \right)}{\Phi\left(\frac{ z - x}{\sigma}\right)} 
\end{cases},
\end{equation}
where $\alpha = \sigma_F/(2\sigma_L^2 + \sigma_F^2)^{1/2}$ and $T(y, a)$ is \textit{Owen's T-function}.\footnote{\label{owen_t}~Owen's T-function, first introduced by \cite{owen_1956}, is defined by 
\[
T(y,a) = \frac{1}{2\pi}\int_0^a \frac{\ee^{-(1+t^2)y^2/2}}{1+t^2} \dd t.
\]
It gives the probability of the event $\{X > y, ~0 < Y < a X \}$ when $X$ and $Y$ are independent standard Gaussian random variables. 
See \cite{savischenko_2014} and \cite{brychkov_savischenko_2016} for an overview of the function.} The derivation of (\ref{ext_rank_belief}) is given in Appendix A. When $x = x_h$, it can be shown that
\begin{equation} \label{ext_follower_fp_eqn}
    \pi_F^h(x_h; z, x_h) = \EE_{\theta \sim G^h(\cdot; \,x_h, z)}[\theta] - \frac{n-1}{n}R^h(x_h; z) - \frac{\chi_\notinvest}{n}.
\end{equation}


Now consider type $x_L$ of the leader. Suppose that leader $x_L$ believes that
followers use monotone strategies with threshold $x_\invest$ under history $h = \invest$. Then her payoff to choosing $\invest$ is
\begin{equation*}
    \pi_L(x_L; y_\invest) = x_L - \Phi\left(\frac{x_\invest -x_L}{\sigma}\right).
\end{equation*}
which is strictly increasing in $x_L$ and strictly decreasing in $x_\invest$.


The $\Delta$-rationalizability procedure again yields six sequences. We prove in Appendix A that $(\xlow^k)_{k=0}^\infty$, $(\xilow^k)_{k=0}^\infty$, and $(\xnlow^k)_{k=0}^\infty$ are increasing and bounded above, and $(\xup^k)_{k=0}^\infty$, $(\xiup^k)_{k=0}^\infty$, and $(\xnup^k)_{k=0}^\infty$ are decreasing and bounded below.
The monotone convergence theorem therefore guarantees that they converge to $\xlow$, $\xilow$, $\xnlow$, $\xup$, $\xiup$, and $\xnup$, respectively. In addition, the limits together solve the following system of equations:
\begin{equation} \label{ext_sys_eqs}
    \begin{cases}
        \pi_L(\xlow; \xilow) = 0 \\
        \pi_L(\xup; \xiup) = 0 \\
        \pi_F^\invest(\xilow; \xup, \xilow) = 0 \\ \pi_F^\invest(\xiup; \xlow, \xiup) = 0 \\
        \pi_F^\notinvest(\xnlow; \xup, \xnlow) = 0 \\
        \pi_F^\notinvest(\xnup; \xlow, \xnup) = 0
    \end{cases}.
\end{equation}

For a given pair of $(\sigma_L, \sigma_F)$, note that we can view it as a point on the ray from the origin with slope $\sigma_F/\sigma_L$. To understand which pair of $(\sigma_L, \sigma_F)$ induces unique $\Delta$-rationalizable behavior, we establish a sufficient condition on each fixed ray $\sigma_F = \gamma \sigma_L$, $\gamma \geq 0$, along which the slope parameter of Owen's T-function is given by $\alpha = \gamma/\sqrt{2 + \gamma^2}$.



\begin{proposition} \label{prop_unique_rat_ext}

The $\Delta$-rationalizable sets are $R_L^\infty = R_L^0 \setminus \overline{R}_L^\infty$ and $R_{F, \,j}^\infty = R_{F, \,j}^0 \setminus \overline{R}_{F, \,j}^\infty$, where
\begin{equation*}
     \overline{R}_L^{\infty} = \left\{(x_L, a_L) \vl~ a_L = \invest ~\text{if}~ x_L \leq \xlow ~\text{and}~ a_L = \notinvest ~\text{if}~ x_L > \xup \right\}
\end{equation*}
and
\begin{equation*}
\overline{R}_{F, \,j}^\infty =  \left\{(x_j, s_j) \vl
\text{$s_j(h) = \invest$ if $x_j \leq \underline{x}_h$ and  $s_j(h) = \notinvest$ if $x_j > \overline{x}_h$, \text{for all}~$h \in \Action_L$} \right\}
\end{equation*} 
Moreover, \\
(i) there exists $\widehat{\sigma}_L(\gamma)$ such that the game has unique $\Delta$-rationalizable behavior if $\sigma_L > \widehat{\sigma}_L(\gamma)$; \\
(ii) in the limit as $\sigma_L \to 0$ (while keeping $\sigma_F/\sigma_L = \gamma$ fixed), the game features multiplicity of $\Delta$-rationalizable behavior.
\end{proposition}


Proposition \ref{prop_unique_rat_ext} is an analog of Proposition \ref{prop_unique_rat}. It establishes that the game generally features multiplicity of rationalizable behavior. In particular, this is necessarily the case, given the ray $\gamma$, in the limit where both noises approach zero in a way that their ratio is always given by $\gamma$. Moreover, for each $\gamma$, one can increase the noises $(\sigma_F,\sigma_L)$ in a way that their ratio is given by $\gamma$ and the game features unique rationalizable behavior. Note, however, that now, even when unique rationalizable play is obtained, the thresholds the agents use depend on the noises $(\sigma_F,\sigma_L)$. This was not the case in our main model. There, as long as $\sigma_F >\widehat{\sigma}_F$, the unique rationalizable play was always the fully efficient one. Finally, the leader having more accurate information than the followers is neither necessary nor sufficient to obtain unique rationalizable behavior, since this can happen irrespective of whether $\gamma$ is greater, equal, or less than one. On the other hand, the leader being arbitrarily better informed than followers is necessary to obtain the efficient outcome.


\subsection{Discussion}

\subsubsection{Signaling Effect, Miscoordination Effect, and Multiplicity}

In a similar spirit to the analysis of the main model, one can analyze the subgame after history $h$ and consider the rationalizable profiles of followers' type-strategy pairs.  Let $z=x_L^* $, that is, $z$ be equal to the leader's threshold in the case where unique rationalizable behavior obtains. Assume that $\widehat{x}$ is the type of follower who is indifferent between choosing $\invest$ and $\notinvest$. One can rewrite Equation \ref{eq_eqm_follower} as
\begin{equation} \label{eq_f_ext}
    \EE_{\underbrace{{\theta \sim G^h(\widehat{x}; z, \widehat{x})}}_\text{signaling effect}} \left[ \theta \right] -  \underbrace{\frac{\chi_\notinvest}{n}}_{\text{externality from leader's action}}= \underbrace{\frac{n-1}{n}R^h(\widehat{x}; z)}_{\text{miss-coordination effect}} 
\end{equation} 

As we stated, in this case, Equation (\ref{eq_f_ext}) has at least one solution. To get the uniqueness of rationalizable play, it must be the case that the derivative of the conditional rank belief function is sufficiently bounded. This is not generally the case, since around $z$, for certain values of $\sigma_L $ and $\sigma_F$\footnote{~In particular, this is necessarily the case in the limit as $\sigma_L\rightarrow 0$ and $\sigma_F \rightarrow 0$ with $\sigma_F/\sigma_L= \gamma$.}, the rank belief function abruptly changes, which means that the expected payoff of the indifferent type of follower $j$ changes sign more than once. The condition of Proposition \ref{prop_unique_rat_ext} ensures that this rapid change is not enough to make the expected payoff of follower $j$ cross the $x$-axis multiple times. Similarly to the main model, if the sign change occurred more than once, the subgame would feature at least two Bayesian Nash equilibria that would correspond to the solutions of Equation \ref{eq_f_ext}. In this case, multiplicity of rationalizable behavior immediately obtains.  Such a case is given in Figure \ref{fig_invest_noisy_multiple}.

% Figure environment removed


\begin{comment}
Notice that the subgame after the leader chooses action $\invest$ has three Bayesian Nash equilibria, in which the followers use thresholds $\xilow<x_{\invest}^* < \xiup$. The leader's best responses are given by $\xlow<x_L^* <\xup$ respectively. Now, every type of leader in $[\xlow,\xup]$ finds both actions rationalizable and the same holds for every type of a follower in $[\xilow,\xiup]$\footnote{The values $(x_L^*,x_{\invest}^*)$ together with the value $x_{\notinvest}^*$ (which is not shown in the graph) correspond to the PBE of the game.}.
\end{comment}



\subsubsection{Inefficiency of the Unique Outcome}
Contrary to the main model, the extension features an inefficient outcome irrespective of the uniqueness of rationalizable play. This result obtains whenever $\sigma_L$ is bounded away from zero. In the limiting case where $\sigma_L \to 0^+$ and for $\sigma_F$ sufficiently large, we recover the unique efficient $\Delta$-rationalizable profile of the main model. It is worth noting, though, that in all cases, the extensive form game leads to outcomes at least as efficient as the ones that would obtain if the game was a simultaneous move game, a prediction consistent with existing literature. This means that the presence of the leader is always helpful, even if her information is very imprecise. This is not surprising, since the leader's action apart from information carries a benefit that spills over to followers.

% \subsubsection{Equilibrium Behavior}

% If the sufficient condition derived in Proposition \ref{prop_unique_rat_ext} is satisfied, the game features unique rationalizable behavior. This immediately implies that the game features unique equilibrium behavior. In particular, the unique rationalizable strategy profile corresponds to the unique Perfect Bayesian Equilibrium of the game, which is in monotone strategies, with thresholds for the leader and followers given by $x_L^*$, $\xistar$ and $\xnstar$. 



\subsection{A Synthesis of the Results}
One can think of the results established in Propositions 1 and 3 in the following way: In the $(\sigma_L , \sigma_F)$ space, when $\sigma_L=0$, Proposition 1 derives a necessary and sufficient condition under which the game features unique rationalizable behavior which delivers the fully efficient outcome. When $\sigma_F=0$, multiplicity immediately obtains since the followers are perfectly informed about the state. In the limit where both noises vanish and $\sigma_F/\sigma_L\to 0$,\footnote{~ This means that followers are arbitrarily better informed than the leader.} the subgame becomes a standard global game, where the leader's action carries only the positive spillover and no information. In this case, the unique rationalizable behavior features a monotone strategy for the leader and the followers with threshold $(n-1)/2n$.  When both $\sigma_L$ and $\sigma_F$ are nonzero and vanish at a rate such that their ratio is given by $\gamma$ for any $\gamma>0$, Proposition 3 establishes the multiplicity of rationalizable behavior in the limit when we move towards the origin along the fixed ray $\gamma$ and the existence of a value $\widehat{\sigma}_L(\gamma)$ such that when one is sufficiently away from the origin the game features unique rationalizable behavior. Thus, in general, the leader cannot ``discipline'' the followers on imitating her behavior, and thus, she may choose the inefficient action in the first place. 














\section{Evaluation}
\label{sec:evaluation}
\section{Evaluation} \label{sec:evaluation}

\begin{table*}[tbp]
\centering
\small
\begin{tabular}{cccccccccc}
\toprule
& \multicolumn{3}{c}{\msr} & \multicolumn{3}{c}{\negc} & \multicolumn{3}{c}{\wsj} \\
& Acc. & F1 & wF1 & Acc. & F1 & wF1 & Acc. & F1 & wF1 \\ \cmidrule(lr){2-4} \cmidrule(lr){5-7} \cmidrule(lr){8-10} 
\udel & 66.86 & 56.76 & 64.3 & \textbf{80.80} & 55.45 & 77.9 & 63.74 & 64.23 & 63.2 \\
\icsi & \underline{71.19} & 64.73 & 70.4 & 80.36 & 64.53 & \underline{78.6} & 64.62 & 64.15 & 63.4 \\
\cnts & 68.59 & 61.39 & 67.2 & 78.68 & 61.62 & 76.8 & 64.31 & 64.59 & 64.4 \\
\osu & 68.02 & 60.28 & 66.6 & 79.24 & 57.04 & 76.5 & 69.20 & 69.63 & 68.9 \\
\isg & 67.05 & 58.83 & 65.3 & 77.34 & 59.52 & 75.6 & 69.15 & 69.35 & 69.2 \\ \midrule
\bert & \textbf{71.68} & \underline{66.70} & \textbf{71.4} & 77.79 & \underline{72.87} & 77.7 & \underline{80.95} & \underline{80.93} & \underline{80.9} \\
\roberta & 70.91 & \textbf{67.53} & \underline{70.7} & \textbf{80.80} & \textbf{77.29} & \textbf{80.7} & \textbf{82.61} & \textbf{82.70} & \textbf{82.6} \\ \midrule
Average & 69.19 & 62.32 & 67.99 & 79.29 & 64.05 & 77.69 & 70.65 & 70.80 & 70.37 \\
\bottomrule
\end{tabular}
\caption{\label{tab:performance} Overall accuracy (Acc.), macro-averaged F1 (F1), and weighted-macro F1 (wF1) scores of the algorithms depicted in Section~\ref{sec:algorithm}. For instance, \msr-\udel refers to a C5.0 classifier trained on the \msr~corpus, using the feature set mentioned in \citet{greenbacker-mccoy-2009-udel}.}
%Its Acc., F1 and wF1 of this model are 66.86, 56.76, and 64.3, respectively.}
\end{table*}


In this section, we introduce the evaluation protocol and report the performance of the models.

\subsection{Implementation Details} \label{sec:implementation}

For \bert and \roberta, we used \textit{bert-base-cased} and \textit{roberta-base}, both from Hugging Face. For fine-tuning, we set the batch size to 16, the learning rate to 1e-3, the dropout rate to 0.5, and the size of the output layer to 256. We ran each model for 20 epochs and used the one that achieved the highest F1 score on the development set. The implementation details of the classic ML-based models can be found in Appendix~\ref{sec:appendixML}.

\subsection{Evaluation Protocol} \label{sec:protocol}

The main evaluation metric in the GREC-MSR shared tasks was accuracy. 
In addition to accuracy, we also report macro-F1 and weighted-macro F1. We argue that different metrics evaluate algorithms from different perspectives and provide us with different meaningful insights. 
For pragmatic tasks like REG, it makes sense to ask how well an algorithm performs on naturally distributed data which is often imbalanced. For these cases, reporting accuracy and weighted F1 are logical. 
Furthermore, analogous to other classification tasks, minority categories should not be overlooked. Take as an example the class \emph{description} in the \negc corpus, which occurs only 4\%. If a model fails to produce this class, the produced document might sound unnatural. Therefore, it is important to ensure that an algorithm is not over- or under-generating certain classes. Looking into accuracy and macro-F1 together provides insights into such cases.

\subsection{Performance of the Models}\label{subsec:overallacc}

The overall accuracy of the models, their macro F1, and their weighted-macro F1 are presented in Table \ref{tab:performance}. 
We also present the ranking of the models based on these scores in Appendix~\ref{sec:app_rank}. 


\paragraph{PLM-based Models.} The best-performing models across all corpora and metrics are PLM-based models.  In six out of nine rankings, \bert and \roberta are ranked as the top two models. The sole exception is \negc, where \bert is the second worst model. The benefit of using PLMs is the largest on the \wsj corpus. For example, \roberta improves the macro F1 score from 69.63 (i.e., the performance of the best ML-based model) to 82.70.


\paragraph{ML-based Models.} In contrast to the robust performance of the PLM models, the performance of the classic ML models is more corpus-dependent. In the case of \msr and \negc, \icsi is the best-performing model, while in the case of \wsj, it is at the bottom section of the rankings. Another interesting observation is the performance of the \udel models. In terms of accuracy, \udel has the highest performance in \negc, while it has the lowest performance in both \msr and \wsj. In terms of macro-F1 rankings, the \negc \udel model dropped from first to last place, whereas \bert improved from penultimate place to second place. In general, our ML models yielded lower scores than the original models used in the GREC study \citep{belz2009generating}. This could be attributed to a variety of factors, including differences in feature engineering and model parameters.

\paragraph{Comparing Different Metrics.} 

Upon comparing average scores across the three metrics, we observe that for \msr and \negc, PLMs are clear winners only when macro-F1 is the metric in question. However, for \wsj, PLMs are winners on all three metrics. This may be because the distribution of categories in \wsj is much more balanced than in the other two corpora.

\section{Related Work}
\label{sec:related}
\section{Related Work}
\label{appsec: related work}
Bayesian causal discovery literature has primarily focused on inference in linear models with closed-form posteriors or marginalized parameters. Early works considered sampling directed acyclic graphs (DAGs) for discrete~\cite{cooper1992bayesian, madigan1995bayesian, heckerman2006bayesian} and Gaussian random variables~\cite{friedman2003being, tong2001active} using Markov chain Monte Carlo (MCMC) in the DAG space. However, these approaches exhibit slow mixing and convergence~\cite{eaton2012bayesian,grzegorczyk2008improving}, often requiring restrictions on number of parents~\cite{kuipers2017partition}. %Alternative exact dynamic programming methods are limited to small settings~\cite{koivisto2012advances}. 

Recent advances in variational inference~\cite{zhang2018advances} have facilitated graph inference in DAG space, with gradient-based methods employing the NOTEARS DAG penalty \cite{zheng2018dags}.\cite{annadani2021variational} samples DAGs from autoregressive adjacency matrix distributions, while \cite{lorch2021dibs} utilizes Stein variational approach \cite{liu2016stein} for DAGs and causal model parameters. \cite{cundy2021bcd} proposed a variational inference framework on node orderings using the gumbel-sinkhorn gradient estimator \cite{mena2018learning}. \cite{deleu2022bayesian,nishikawa2022bayesian} employ the GFlowNet framework \cite{bengio2021gflownet} for inferring the DAG posterior. Most methods, except\cite{lorch2021dibs} are restricted to linear models, while \cite{lorch2021dibs} has high computational costs and lacks DAG generation guarantees compared to our method.
% at least quadratic scaling complexity, both with respect to the number of nodes (due to the DAG penalty) as well as number of posterior samples. Our proposed approach instead has linear complexity with respect to number of posterior samples and does not require any additional DAG penalty.     

In contrast, \emph{quasi-Bayesian} methods, such as DAG bootstrap \cite{friedman2013data}, demonstrate competitive performance. DAG bootstrap resamples data and estimates a single DAG using PC \cite{spirtes2000causation}, GES \cite{chickering2002optimal}, or similar algorithms, weighting the obtained DAGs by their unnormalized posterior probabilities. Recent neural network-based works employ variational inference to learn DAG distributions and point estimates for nonlinear model parameters \cite{charpentier2022differentiable,geffner2022deep}.


\section{Conclusion}
\label{sec:conclusion}
In this paper, through exploring the dependencies between vertices, we propose a method of adding edges and thus making longer paths in DAG tasks to optimize the worst-case response time bound.
We also apply the proposed techniques to the scheduling of multiple DAG tasks.
Experiments demonstrate that the proposed method significantly outperforms the state-of-the-art, reducing the worst-case response time bound by 21.6\% and improving the system schedulability by 22.2\% on average.

\bibliographystyle{IEEEtran}
\bibliography{reference}

\end{document}
