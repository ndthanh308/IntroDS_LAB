The most closely related work to this research is \cite{he2022bounding}, where He et al. proposed a response time bound for a DAG task using multiple long paths. Our work employs the bound of \cite{he2022bounding} as a guidance and proposes methods that systematically connect short paths into long paths by adding edges to reduce the worst-case response time bound for a DAG task.

Concerning modifying edges for DAG tasks in real-time scheduling, \cite{buttazzo2011hard} studied the timing anomaly of DAG tasks and observed that weakening the precedence constraints (i.e., removing edges) may lead to longer response time.
\cite{fonseca2017improved} considered removing edges of DAG tasks and transforming the DAG into a series-parallel graph \cite{he1987parallel} to derive a bound on the interference incurred by this task.
Different from the above works, we study the method of adding edges and focus on the response time bound and schedulability of DAG tasks.

For real-time scheduling of DAG tasks, existing approaches can be categorized into four major paradigms: federated scheduling \cite{li2014analysis, chen2016federated, baruah2015federatedconstrained, baruah2015federatedarbitrary, baruah2015federatedconditional, jiang2017semi, ueter2018reservation, jiang2021virtually, he2022bounding}, global scheduling \cite{li2013outstanding, chen2014capacity, nasri2019response, he2021response, dai2022response, he2023real}, partitioned scheduling \cite{fonseca2016response, casini2018partitioned}, and decomposition-based scheduling \cite{qamhieh2013global, saifullah2014parallel, jiang2016decomposition}.
Federated scheduling, where each DAG task is scheduled on a set of dedicated cores, is closely related to this work.
Federated scheduling was originally proposed in \cite{li2014analysis}. 
Later, federated scheduling was generalized to constrained deadline tasks \cite{baruah2015federatedconstrained}, arbitrary deadline tasks \cite{baruah2015federatedarbitrary}, and conditional DAG tasks \cite{baruah2015federatedconditional}.
A series of federated-based scheduling approaches \cite{jiang2017semi, ueter2018reservation, jiang2021virtually, he2022bounding} were proposed to address the resource-wasting problem in federated scheduling.
