\title{Cover Letter}
% \maketitle

% We thank the reviewers for the careful reading and comments, and the shepherd for the help in improving the quality of the paper. Below we provide our revision plan, and summarize the corresponding revisions we have made in the updated paper. For your convenience, we provide an annotated version of our revised paper, in which only the changes required in this round are marked in {\color{red} red}. 

% \section{Reviewer A}

% \textbf{P1: Complexity.} We include theoretical analysis of the complexity of FFT-based convolution in Section~\ref{subsec:conv_spectral_dp} and the complexity analysis of block circulant matrices (BCM) based FC layer in Appendix \ref{apdx:complexity}.

% \textbf{P2: Filtering ratio impact.} As pointed out in the  rebuttal, we have included a study of the filtering ratio on CIFAR10 (see Section~\ref{subsec:effectiveness_spectral_dp} and Table~\ref{tab:conv1-pruning}).  
% In the revision, We provide more results of filtering ratio impact on CIFAR10 in Appendix \ref{appdix:ablation_cifar_transfer} and CIFAR100 in Section~\ref{subsec:transfer_learning_ablation}.

% \textbf{P3: Lossless transform.} We correct the statement as advised by the reviewer in Section~\ref{sec:relatedwork}.



% \section{Reviewer B}

% \textbf{P1: Limitations/Shortcomings. }
% We add Section~\ref{sec:limitations} to discuss the limitations.
% As discussed in the rebuttal, there still exists the utility or accuracy gap for deep models that are trained from scratch, and there’s scope to explore other possible domain transformation techniques in addition to DFT along this direction. We also discuss the areas for the future improvement of Spectral-DP.  



% \section{Reviewer C}

% \textbf{P1: Comparison Spectral-DP against other works and De et al “DeepMind”.} As suggested by the reviewer, we have run additional experiments on CIFAR10 and CIFAR100 and compared Spectral-DP with more recent baselines including DeepMind~\cite{de2022unlocking}, AAAI21~\cite{papernot2021tempered}, ICLR21~\cite{tramer2021differentially} in Section~\ref{subsec:WRN_transfer}.

% \textbf{P2: Comparison against Luo et al. CVPR2021.} We include the results of ResNeXt compared to this work, as shown in our rebuttal.
% This has been added to Appendix~\ref{appendix:transfer_resnext} due to space limitations.
% % in Appendix \ref{appendix:transfer_resnext} due to space limitation. 

% \textbf{P3: Cases/settings where this lost information significantly hurts utility.} We discuss these cases in Section~\ref{sccratch} and Section~\ref{subsec:transfer_learning_ablation}.


% \section{Reviewer D}

% \textbf{P1: Privacy composition and sampling technique.} 
% In Section~\ref{sec:DPtogether},
% we add the information regarding the sampling operation into Algorithm~\ref{alg:PutInDL}.
% % we added the sampling step in Algorithm~\ref{alg:PutInDL}.
% We clarify the sampling technique with the supporting reference in the second paragraph of Section~\ref{sec:DPtogether}.
% We also modify Corollary~\ref{cor:composition} accordingly.
% % We addressed these issues in the rebuttal as the reviewer suggested. In the revision, we will provide and clarify the privacy composition technique and cite the related materials. We will also clarify the sampling method in Section 3.4 and modify Algorithm 4 and Corollary 2 accordingly. 

% \textbf{P2: Experiments of standard architecture.} We have conducted additional experiments that demonstrate the effectiveness of Spectral-DP by using standard neural networks architectures including Wide-ResNet and ResNeXt. In the revised manuscript, we update the results in Section~\ref{subsec:WRN_transfer} and Appendix \ref{appdix:ablation_cifar_transfer}, respectively.

% \textbf{P3: Concentrating the weights in the low bandwidth domain.} 
% In Section~\ref{sec:introduction}, we provided the corresponding justification supported by relevant references for concentrating weights in the low bandwidth region. 
% % We add the motivation of concentrating on low bandwidth in Section~\ref{sec:introduction} with references.


% \section{Additional modifications}
% \begin{enumerate}
% \item In Section~\ref{sec:introduction}, we modified the contribution and the organization of the paper as we added additional experimental results and Section~\ref{sec:limitations} in the revision.
% \item In Section~\ref{sec:approach}, to avoid confusion caused by using time domain and signal domain interchangeably, we replace time domain by spatial domain.
% \item  In Section~\ref{subsec:theoreticalanalysis}, we move the proof of Proposition 3 to Appendix \ref{Appendix:noise_reduction} due to the space limit.
% \item In Section~\ref{subsec:adaptingspectralDPin2dconv}, we move the proof of Corollary 1 to Appendix \ref{Appendix:noise_reduction2D} due to the space limit.
% \item The original content in Section~\ref{subsec:transfer_learning_ablation} about the ablation study on CIFAR10 transfer models is moved to Appendix \ref{appdix:ablation_cifar_transfer} due to the space limit.

% \end{enumerate}



We thank you for the time and effort in reviewing our revision and providing valuable comments. We have carefully considered the comments and provided a response to address each of the concerns. 

\section{Comment 1: Privacy guarantee in Cor. 2}

We clarify that the experimental number of privacy accounting does not change. All privacy accounting is based on the sampling technique in \cite{2019arXiv190810530M}. 

We are sorry for the confusion, in the prior submission, \textbf{$T$} represents the \textbf{total training iterations} of Spectral-DP. 
To avoid confusion, in this revision, we define $T_e$ as the \textbf{training epochs} of Spectral-DP. 
% In this revision, we clarified the sampling technique in Algorithm 4 and used \textbf{T to represent the training epochs} of Spectral-DP. 
As the number of \textbf{total training iterations} is computed as the product of training epochs and $N/B$ ($N$ is the size of the training dataset, $B$ is the batch size), we modified Corollary 2 by replacing $T$ with $T_e*(N/B)$. 


\section{Comment 2: Result of ResNeXt}
We have moved the result of ResNeXt to Section \ref{subsec:transfer_resnext} in the revised draft. We added Table~\ref{tab:apdx-transfer} and marked the new content in {\color{red} red}. 



