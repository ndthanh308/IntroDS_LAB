In this section, we show how to apply our algorithm in Section~\ref{sec:upper-bound} which only works for very large $t$ (or rather, does not have improved runtime for smaller $t$) to other ranges of $t$ as well, via black-box reductions.

\begin{theorem}
\label{thm:reduction_from_small_t}
Suppose for every $1 \le \ell < k$, $\cliquelist{k, \ell}$ can be solved in $\tO(\Delta_\ell^{\alpha_{k, \ell}} t^{1-\frac{\ell \alpha_{k, \ell}}{k}})$ time when $t \ge \Delta_\ell^{\gamma_{k, \ell}}$. Then for every $1 \le \ell \le k$ and $1 \le s < k$ where $\lceil \frac{k}{s} \rceil \ne \lceil \frac{\ell}{s} \rceil$, $\cliquelist{k, \ell}$ can be solved in 
$$\tO\left( \left(\Delta_{\ell}^{\frac{s}{\ell} \lceil \frac{\ell}{s}\rceil} \right)^{\alpha_{k', \ell'}} t^{1-\frac{\ell' \alpha_{k', \ell'}}{k'}} \right)$$
time for $t \ge \left(\Delta_{\ell}^{\frac{s}{\ell} \lceil \frac{\ell}{s}\rceil} \right)^{\gamma_{k', \ell'}}$, where $k' = \lceil \frac{k}{s}\rceil$ and $\ell' = \lceil \frac{\ell}{s}\rceil$. 
\end{theorem}
\begin{proof}
Let $G$ be the input of a $\cliquelist{k, \ell}$ instance. Without loss of generality, assume $G$ is $k$-partite with parts $V_1, \ldots, V_k$. Create a new $k'$-partite graph $G'$ on node parts $U_1 = V_1 \times \cdots \times V_s,  U_2 = V_{s+1} \times \cdots \times V_{2s}, \ldots, U_{k'} = V_{s(k' - 1) + 1} \times \cdots \times V_k$ (each node corresponds to a set of at most $s$ nodes). Keep a node $(v_1, v_2, \ldots, v_i)$ if and only if $(v_1, v_2, \ldots, v_i)$ forms a clique in $G$.  
Add an edge between two nodes $(v_1, v_2, \ldots, v_i)$ and $(v_1', v_2', \ldots, v_{i'}')$ belonging to two different parts if and only if the nodes $(v_1, v_2, \ldots, v_i, v_1', v_2', \ldots, v_{i'}')$ form a clique in $G$. 

Clearly, $k'$-cliques in $G'$ have one-to-one correspondence with $k$-cliques in $G$, so it suffices to list $t$ $k'$-cliques in $G'$ in order to list $t$ $k$-cliques in $G$. Furthermore, distinct $\ell'$-clique in $G'$ corresponds to distinct clique in $G$. Depending on whether an $\ell'$-clique uses a node in $U_{k'}$, it corresponds to either an $(s\ell')$-clique in $G$ or an $(s\ell'+k-sk')$-clique in $G$. Either way, it is a clique of size at most $s\ell'$. Thus, by Lemma~\ref{lem:simple_list_ub}, there are $\tO(\Delta_\ell^{\frac{s \ell'}{\ell}}) = \tO(\Delta_\ell^{\frac{s}{\ell} \lceil \frac{\ell}{s}\rceil})$ such cliques in $G$ and we can list them in $\tO(\Delta_\ell^{\frac{s}{\ell} \lceil \frac{\ell}{s}\rceil})$ time as well. 

Thus, to solve $\cliquelist{k, \ell}$ on $G$ with $t$ $k$-cliques, it suffices to solve $\cliquelist{k', \ell'}$ on $G'$ with $t$ $k'$-cliques. The theorem thus easily follows. 
\end{proof}

Let us give some examples to show how to use Theorem~\ref{thm:reduction_from_small_t}. 

First, for any $1 \le \ell \le k$, let us take the extreme example $s = 1$. In this case, the running time is exactly the running time given in Theorem~\ref{thm:k-l-large-t-listing}. On the other extreme end, $s = k - 1$. Then $k' = 2$ and $\ell' = 1$. In this extreme case, we have $\alpha_{k', \ell'} = 2$ and $\gamma_{k', \ell'} = 0$. Thus, we get an algorithm that works for any $t \ge 1$, although its running time $\tO\left(\Delta_\ell^{\frac{2(k-1)}{\ell}}\right)$ is not great. One can imagine when increasing $s$ from $1$ to $k-1$, we achieve a trade-off between the bound for $t$ and running time of the algorithm. 

Let us give the following more concrete examples. For simplicity, we assume $\omega = 2$. 

\begin{corollary}
\label{cor:k-1-listing-general}
Assume $\omega = 2$. 
Fix any integer  $k \ge 2$, and any integer $1 \le s < \frac{k}{2}$. $\cliquelist{k, 1}$ can be solved in $\tO\left(n^{\frac{2s}{k'-1}} t^{1-\frac{2}{k'(k'-1)}} \right)$ time when $t \ge n^{s(k'-1-\frac{2}{k'^2-k'-2})}$, where $k' = \lceil \frac{k}{s}\rceil$.
\end{corollary}
\begin{proof}
Apply Theorem~\ref{thm:reduction_from_small_t}. We get $k' = \lceil \frac{k}{s}\rceil\ge 3$, $\ell' = 1$, and an algorithm for $\cliquelist{k, 1}$ that runs in $\tO(\left(n^{s}\right)^{\alpha_{k', 1}} t^{1-\frac{\alpha_{k',1}}{k'}})$ time when $t \ge \left(n^{s}\right)^{\gamma_{k', 1}}$. By Theorem~\ref{thm:k_1_optimal}, when $\omega = 2$, $\alpha_{k', 1} = \frac{2}{k'-1}$ and $\gamma_{k', 1} = k'-1-\frac{2}{k'^2 - k' - 2}$. Thus, we get an 
$\tO\left(n^{\frac{2s}{k'-1}} t^{1-\frac{2}{k'(k'-1)}} \right)$ time algorithm for $t \ge n^{s(k'-1-\frac{2}{k'^2-k'-2})}$. 
\end{proof}


\begin{example}[$\cliquelist{12, 1}$]\label{eg:12_1_list}
$\cliquelist{12, 1}$ has the following running times (by setting $s = 1, 2, 3, 4$ in Corollary~\ref{cor:k-1-listing-general}):
\begin{itemize}
    \item $\tO\left(n^{\frac{2}{11}} t^{\frac{65}{66}}\right)$ when $t \ge n^{11-\frac{1}{65}}$;
    \item $\tO\left(n^{\frac{4}{5}} t^{\frac{14}{15}}\right)$ when $t \ge n^{10-\frac{1}{7}}$;
    \item $\tO\left(n^{2} t^{\frac{5}{6}}\right)$ when $t \ge n^{9-\frac{3}{5}}$;
    \item $\tO\left(n^{4} t^{\frac{2}{3}}\right)$ when $t \ge n^{6}$;
    \item $\tO\left(n^8\right)$ when $t<n^6$.
\end{itemize}
\end{example}
Figure~\ref{fig:12_1} shows a pictorial representation of the $\cliquelist{12,1}$ runtime.

% Figure environment removed

We can similarly obtain the following corollary for $\cliquelist{k, 2}$. 

\begin{corollary}
\label{cor:k-2-listing-general}
Assume $\omega = 2$. 
Fix any integer  $k \ge 2$, and any integer $2 \le s < \frac{k}{2}$. $\cliquelist{k, 2}$ can be solved in $\tO\left(m^{\frac{s}{k'-1}} t^{1-\frac{2}{k'(k'-1)}} \right)$ time when $t \ge m^{\frac{s}{2} (k'-1-\frac{2}{k'^2-k'-2})}$, where $k' = \lceil \frac{k}{s}\rceil$.
\end{corollary}
\begin{proof}
Apply Theorem~\ref{thm:reduction_from_small_t}. We get $k' = \lceil \frac{k}{s}\rceil\ge 3$, $\ell' = 1$, and an algorithm for $\cliquelist{k, 2}$ that runs in $\tO\left(\left(m^{\frac{s}{2}}\right)^{\alpha_{k', 1}} t^{1-\frac{\alpha_{k',1}}{k'}}\right)$ time when $t \ge \left(m^{\frac{s}{2}}\right)^{\gamma_{k', 1}}$. By Theorem~\ref{thm:k_1_optimal}, when $\omega = 2$, $\alpha_{k', 1} = \frac{2}{k'-1}$ and $\gamma_{k', 1} = k'-1-\frac{2}{k'^2 - k' - 2}$. The corollary then follows. 
\end{proof}
