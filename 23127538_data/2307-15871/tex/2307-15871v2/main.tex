\def\anon{0}
\def\draft{1}

\documentclass[11pt]{article}
\usepackage{color,colortbl,latexsym,amsthm,amsmath,amssymb,path,enumitem,soul,tikz,fullpage,subcaption}
\usetikzlibrary{shapes.geometric}
\usepackage[colorlinks,citecolor=blue,linkcolor=blue,urlcolor=blue,pagebackref]{hyperref}
\usepackage{algorithm}
\usepackage{algorithmicx}
\usepackage{algpseudocode}
\usepackage{soul}
\usepackage{multirow}

\usepackage{relsize}
\usepackage{diagbox}
\usepackage{comment}
\usepackage[font={small,sf}]{caption}
\newcommand{\ignore}[1]{}

\makeatother
\makeatletter
\newenvironment{breakablealgorithm}
  {%
   \begin{center}
     \refstepcounter{algorithm}%
     \hrule height.8pt depth0pt \kern2pt%
     \renewcommand{\caption}[2][\relax]{%
       {\raggedright\textbf{\ALG@name~\thealgorithm} ##2\par}%
       \ifx\relax##1\relax %
         \addcontentsline{loa}{algorithm}{\protect\numberline{\thealgorithm}##2}%
       \else %
         \addcontentsline{loa}{algorithm}{\protect\numberline{\thealgorithm}##1}%
       \fi
       \kern2pt\hrule\kern2pt
     }
  }{%
     \kern2pt\hrule\relax%
   \end{center}
  }
\makeatother




\setlist[itemize]{itemsep=-0.1em}
\setlist[enumerate]{itemsep=-0.1em}


\makeatletter
\def\th@plain{%
  \thm@notefont{}%
  \itshape %
}
\def\th@definition{%
  \thm@notefont{}%
  \normalfont %
}
\makeatother

\newtheorem{theorem}{Theorem}[section]
\newtheorem{lemma}[theorem]{Lemma}
\newtheorem{corollary}[theorem]{Corollary}
\newtheorem{observation}[theorem]{Observation}
\newtheorem{conjecture}[theorem]{Conjecture}
\newtheorem{hypothesis}[theorem]{Hypothesis}
\newtheorem{question}{Question}
\newtheorem{claim}[theorem]{Claim}
\newtheorem{definition}[theorem]{Definition}
\newtheorem{problem}[theorem]{Problem}
\newtheorem{remark}[theorem]{Remark}
\newtheorem{proposition}[theorem]{Proposition}
\newtheorem{fact}[theorem]{Fact}
\newtheorem{example}[theorem]{Example}
\usepackage{graphicx,psfrag}

\newcommand{\mycaption}[1]
        {\caption{\small #1}}
\newcommand{\placefig}[2]
        {% Figure removed}

\usepackage{listings,xcolor}
\lstset{language=Mathematica}
\lstset{basicstyle={\sffamily\footnotesize},
  numbers=left,
  lineskip=1mm,
  numberstyle=\tiny\color{gray},
  numbersep=5pt,
  breaklines=true,
  captionpos={t},
  frame={lines},
  rulecolor=\color{black},
  framerule=0.5pt,
  columns=flexible,
  tabsize=2,
  mathescape
}

\usepackage{dsfont}
\newcommand{\NN}{\ensuremath{\mathbb N}}
\newcommand{\ZZ}{\ensuremath{\mathbb Z}}
\newcommand{\RR}{\ensuremath{\mathbb R}}
\newcommand{\CC}{\ensuremath{\mathbb C}}
\newcommand{\TT}{\ensuremath{\mathbb T}}
\newcommand{\FF}{\ensuremath{\mathbb F}}
\newcommand{\DD}{\ensuremath{\mathbb D}}
\newcommand{\PP}{\ensuremath{\mathbb P}}
\newcommand{\EE}{\ensuremath{\mathbb E}}
\renewcommand{\epsilon}{\varepsilon}
\newcommand{\eps}{\varepsilon}
\newcommand{\NNnull}{\NN_{^{_0}\!}}
\newcommand{\RRplus}   {\RR^{\!^+}}
\newcommand{\RRplusnull}{\RR^{\!^+}_{^{_0}\!}}
\newcommand{\prr}       {{\mathrm{P\hspace{-0.05em}r}}}
\newcommand{\prrprime}  {{\mathrm{P\hspace{-0.05em}r}'}}
\newcommand{\pr}[1]     {{\,\prr\hspace{-0.05em}\left(#1\right)}}
\newcommand{\prprime}[1]{{\,\prrprime\hspace{-0.05em}\left(#1\right)}}
\newcommand{\ex}[1]     {{\,{\ensuremath{\mathbb E}}\!\left[#1\right]}}
\newcommand{\var}[1]    {{\,{\mathrm{var}}\!\left(#1\right)}}
\newcommand{\cond}      {\!\mid\!}

\def \poly {\mathop{\rm poly}} %
\def \polylog {\mathop{\rm polylog}}%
\def\th{{\text{\rm th}}}
\def\tO{\tilde{O}}

\def\eps{{\varepsilon}}

\newcommand{\parag}[1]{\vspace{2mm}

\noindent{\bf #1} }

\newcommand{\todo}[1]{\textcolor{magenta}{TODO: #1}}

\ifnum\draft=1
\newcommand{\mina}[1]{\textcolor{red}{(Mina: #1)}}
\newcommand{\surya}[1]{\textcolor{magenta}{(Surya: #1)}}
\newcommand{\yinzhan}[1]{\textcolor{green}{(Yinzhan: #1)}}
\newcommand{\virgi}[1]{\textcolor{red}{(Virginia: #1)}}
\else
    \newcommand{\mina}[1]{}
    \newcommand{\surya}[1]{}
    \newcommand{\yinzhan}[1]{}
    \newcommand{\virgi}[1]{}
\fi 
\newcommand\cliquelist[1]{(#1)\text{-}\mathsf{Clique}\text{-}\mathsf{Listing}}
\newcommand\cliquedet[1]{(#1)\text{-}\mathsf{Clique}\text{-}\mathsf{Detection}}
\newcommand{\exactclique}[1]{\mathsf{Exact}\text{-}#1\text{-}\mathsf{Clique}}
\newcommand\MM{\mathsf{MM}}
\newcommand\sparse{{\tt Sparse}}
\newcommand\dense{{\tt Dense}}


\title{Towards Optimal Output-Sensitive Clique Listing\\
{{or: Listing Cliques from Smaller Cliques}}}
\ifnum\anon=1
    \author{Anonymous}
\else
    \author{\normalsize Mina Dalirrooyfard\thanks{Morgan Stanley Research. \href{mailto:minad@mit.edu}{\texttt{minad@mit.edu}}. While at MIT, supported by a Google Faculty Research Award and an Akamai MIT CS Theory Group Fellowship.} \and \normalsize Surya Mathialagan\thanks{MIT. \href{mailto:smathi@mit.edu}{\texttt{smathi@mit.edu}}. Supported by the Siebel Scholars program, by DARPA under Agreement No. HR00112020023 and by NSF grant CNS-2154149.}
    \and \normalsize Virginia Vassilevska Williams\thanks{MIT. \href{mailto:virgi@mit.edu}{\texttt{virgi@mit.edu}}. Partially supported by NSF Career Award CCF-1651838, NSF Grant CCF-2129139, a Sloan Research Fellowship and a Google Faculty Research Award.} 
    \and \normalsize  Yinzhan Xu\thanks{MIT. \href{mailto:xyzhan@mit.edu}{\texttt{xyzhan@mit.edu}}. Supported by NSF Grant CCF-2129139.}}
\fi 

\date{}
\begin{document}

\maketitle 
\pagenumbering{gobble} 
\begin{abstract}
We study the problem of finding and listing $k$-cliques in an $m$-edge, $n$-vertex graph, for constant $k\geq 3$. This is a fundamental problem of both theoretical and practical importance.

Our first contribution is an algorithmic framework for finding $k$-cliques that gives the first improvement in 19 years over the old runtimes for $4$ and $5$-clique finding, as a function of $m$ [Eisenbrand and Grandoni, TCS'04]. 
With the current bounds on matrix multiplication, our algorithms run in $O(m^{1.66})$ and $O(m^{2.06})$ time, respectively, for $4$-clique and $5$-clique finding.

Our main contribution is an output-sensitive algorithm for listing $k$-cliques, for any constant $k\geq 3$. We complement the algorithm with tight lower bounds based on standard fine-grained assumptions. Previously, the only known conditionally optimal output-sensitive algorithms were for the case of $3$-cliques given by Bj\"{o}rklund, Pagh, Vassilevska W. and Zwick [ICALP'14]. If the matrix multiplication exponent $\omega$ is $2$, and if the number of $k$-cliques $t$ is large enough, the running time of our algorithms is 
$$\tilde{O}\left(\min\{m^{\frac{1}{k-2}}t^{1 - \frac{2}{k(k-2)}},n^{\frac{2}{k-1}}t^{1-\frac{2}{k(k-1)}}\}\right),$$
and this is {\em tight} under the Exact-$k$-Clique Hypothesis. This running time naturally extends the running time obtained by Bj\"{o}rklund, Pagh, Vassilevska W. and Zwick for $k=3$.

Our framework is very general in that it gives $k$-clique listing algorithms whose running times can be measured in terms of the number of $\ell$-cliques $\Delta_\ell$ in the graph for any $1\leq \ell<k$. This generalizes the typical parameterization in terms of $n$ (the number of $1$-cliques) and $m$ (the number of $2$-cliques).

If $\omega$ is $2$, and if the size of the output, $\Delta_k$, is sufficiently large, then for every $\ell<k$, the running time of our algorithm for listing $k$-cliques is 
$$\tilde{O}\left(\Delta_\ell^{\frac{2}{\ell (k - \ell)}}\Delta_k^{1-\frac{2}{k(k-\ell)}}\right).$$ 
We also show that this runtime is {\em optimal} for all $1 \leq \ell < k$ under the Exact $k$-Clique hypothesis. 
\end{abstract}

\newpage


\tableofcontents{}
\newpage 
\pagenumbering{arabic} 
\section{Introduction}
The meteoric advancement of machine learning and artificial intelligence technologies has enabled the construction of neural networks that effectively emulate the complex computations of the human brain. These deep learning models have found utility in a wide range of applications, such as computer vision, natural language processing, autonomous driving, and more. With the growing complexity and sophistication of these neural network models, the computational requirements, particularly for 32-bit operations, have exponentially increased. This heightened computational demand necessitates the exploration of more efficient alternatives, such as 16-bit operations.

However, the shift to 16-bit operations is riddled with challenges. A common standpoint within the research community argues that 16-bit operations are not ideally suited for neural network computations. This belief is mainly attributable to concerns related to numerical instability during the backpropagation phase, especially when popular optimizers like Adam are employed. This instability, more pronounced during the optimizer-mediated backpropagation process rather than forward propagation, can negatively impact the performance of 16-bit operations and compromise the functioning of the neural network model. Current optimizers predominantly operate on 32-bit precision. If these are deployed in a 16-bit environment without appropriate hyperparameter fine-tuning, the neural network models encounter difficulties during learning. This issue is particularly evident in backward propagation, which heavily relies on the optimizer. Confronted with these challenges, the objective of this research is to conduct an exhaustive investigation into the feasibility and implementation of 16-bit operations for neural network training. We propose and evaluate innovative strategies aimed at reducing the numerical instability encountered during the backpropagation phase under 16-bit environments. A significant focus of this paper is also dedicated to exploring the future possibilities of developing 16-bit based optimizers. One of the fundamental aims of this research is to adapt key optimizers such as Adam to prevent numerical instability, thereby facilitating efficient 16-bit computations. These newly enhanced optimizers are designed to not only address the issue of numerical instability but also leverage the computational advantages offered by 16-bit operations, all without compromising the overall performance of the neural network models. Through this research, our intention goes beyond improving the efficiency of neural network training; we also strive to validate the use of 16-bit operations as a dependable and efficient computational methodology in the domain of deep learning. We anticipate that our research will contribute to a shift in the prevalent perceptions about 16-bit operations and will foster further innovation in the field. Ultimately, we hope our findings will pave the way for a new era in deep learning research characterized by efficient, high-performance neural network models.


\section{Preliminaries}
\label{sec:prelim}

\input{2-prelim}

\section{Detecting Cliques}\label{sec:detection}

\input{3-detection}

\section{Lower Bounds for Listing Cliques}\label{sec:lower_bounds}
\input{4-lower-bound}


\section{Optimal Listing Algorithms for Graphs with Many \texorpdfstring{$k$}{k}-Cliques}
\label{sec:upper-bound}
\input{5-upper-bound}

\section{Extending the Algorithm to Graphs with Fewer \texorpdfstring{$k$}{k}-Cliques}
\label{sec:general-list}
In this section, we show how to apply our algorithm in Section~\ref{sec:upper-bound} which only works for very large $t$ (or rather, does not have improved runtime for smaller $t$) to other ranges of $t$ as well, via black-box reductions.

\begin{theorem}
\label{thm:reduction_from_small_t}
Suppose for every $1 \le \ell < k$, $\cliquelist{k, \ell}$ can be solved in $\tO(\Delta_\ell^{\alpha_{k, \ell}} t^{1-\frac{\ell \alpha_{k, \ell}}{k}})$ time when $t \ge \Delta_\ell^{\gamma_{k, \ell}}$. Then for every $1 \le \ell \le k$ and $1 \le s < k$ where $\lceil \frac{k}{s} \rceil \ne \lceil \frac{\ell}{s} \rceil$, $\cliquelist{k, \ell}$ can be solved in 
$$\tO\left( \left(\Delta_{\ell}^{\frac{s}{\ell} \lceil \frac{\ell}{s}\rceil} \right)^{\alpha_{k', \ell'}} t^{1-\frac{\ell' \alpha_{k', \ell'}}{k'}} \right)$$
time for $t \ge \left(\Delta_{\ell}^{\frac{s}{\ell} \lceil \frac{\ell}{s}\rceil} \right)^{\gamma_{k', \ell'}}$, where $k' = \lceil \frac{k}{s}\rceil$ and $\ell' = \lceil \frac{\ell}{s}\rceil$. 
\end{theorem}
\begin{proof}
Let $G$ be the input of a $\cliquelist{k, \ell}$ instance. Without loss of generality, assume $G$ is $k$-partite with parts $V_1, \ldots, V_k$. Create a new $k'$-partite graph $G'$ on node parts $U_1 = V_1 \times \cdots \times V_s,  U_2 = V_{s+1} \times \cdots \times V_{2s}, \ldots, U_{k'} = V_{s(k' - 1) + 1} \times \cdots \times V_k$ (each node corresponds to a set of at most $s$ nodes). Keep a node $(v_1, v_2, \ldots, v_i)$ if and only if $(v_1, v_2, \ldots, v_i)$ forms a clique in $G$.  
Add an edge between two nodes $(v_1, v_2, \ldots, v_i)$ and $(v_1', v_2', \ldots, v_{i'}')$ belonging to two different parts if and only if the nodes $(v_1, v_2, \ldots, v_i, v_1', v_2', \ldots, v_{i'}')$ form a clique in $G$. 

Clearly, $k'$-cliques in $G'$ have one-to-one correspondence with $k$-cliques in $G$, so it suffices to list $t$ $k'$-cliques in $G'$ in order to list $t$ $k$-cliques in $G$. Furthermore, distinct $\ell'$-clique in $G'$ corresponds to distinct clique in $G$. Depending on whether an $\ell'$-clique uses a node in $U_{k'}$, it corresponds to either an $(s\ell')$-clique in $G$ or an $(s\ell'+k-sk')$-clique in $G$. Either way, it is a clique of size at most $s\ell'$. Thus, by Lemma~\ref{lem:simple_list_ub}, there are $\tO(\Delta_\ell^{\frac{s \ell'}{\ell}}) = \tO(\Delta_\ell^{\frac{s}{\ell} \lceil \frac{\ell}{s}\rceil})$ such cliques in $G$ and we can list them in $\tO(\Delta_\ell^{\frac{s}{\ell} \lceil \frac{\ell}{s}\rceil})$ time as well. 

Thus, to solve $\cliquelist{k, \ell}$ on $G$ with $t$ $k$-cliques, it suffices to solve $\cliquelist{k', \ell'}$ on $G'$ with $t$ $k'$-cliques. The theorem thus easily follows. 
\end{proof}

Let us give some examples to show how to use Theorem~\ref{thm:reduction_from_small_t}. 

First, for any $1 \le \ell \le k$, let us take the extreme example $s = 1$. In this case, the running time is exactly the running time given in Theorem~\ref{thm:k-l-large-t-listing}. On the other extreme end, $s = k - 1$. Then $k' = 2$ and $\ell' = 1$. In this extreme case, we have $\alpha_{k', \ell'} = 2$ and $\gamma_{k', \ell'} = 0$. Thus, we get an algorithm that works for any $t \ge 1$, although its running time $\tO\left(\Delta_\ell^{\frac{2(k-1)}{\ell}}\right)$ is not great. One can imagine when increasing $s$ from $1$ to $k-1$, we achieve a trade-off between the bound for $t$ and running time of the algorithm. 

Let us give the following more concrete examples. For simplicity, we assume $\omega = 2$. 

\begin{corollary}
\label{cor:k-1-listing-general}
Assume $\omega = 2$. 
Fix any integer  $k \ge 2$, and any integer $1 \le s < \frac{k}{2}$. $\cliquelist{k, 1}$ can be solved in $\tO\left(n^{\frac{2s}{k'-1}} t^{1-\frac{2}{k'(k'-1)}} \right)$ time when $t \ge n^{s(k'-1-\frac{2}{k'^2-k'-2})}$, where $k' = \lceil \frac{k}{s}\rceil$.
\end{corollary}
\begin{proof}
Apply Theorem~\ref{thm:reduction_from_small_t}. We get $k' = \lceil \frac{k}{s}\rceil\ge 3$, $\ell' = 1$, and an algorithm for $\cliquelist{k, 1}$ that runs in $\tO(\left(n^{s}\right)^{\alpha_{k', 1}} t^{1-\frac{\alpha_{k',1}}{k'}})$ time when $t \ge \left(n^{s}\right)^{\gamma_{k', 1}}$. By Theorem~\ref{thm:k_1_optimal}, when $\omega = 2$, $\alpha_{k', 1} = \frac{2}{k'-1}$ and $\gamma_{k', 1} = k'-1-\frac{2}{k'^2 - k' - 2}$. Thus, we get an 
$\tO\left(n^{\frac{2s}{k'-1}} t^{1-\frac{2}{k'(k'-1)}} \right)$ time algorithm for $t \ge n^{s(k'-1-\frac{2}{k'^2-k'-2})}$. 
\end{proof}


\begin{example}[$\cliquelist{12, 1}$]\label{eg:12_1_list}
$\cliquelist{12, 1}$ has the following running times (by setting $s = 1, 2, 3, 4$ in Corollary~\ref{cor:k-1-listing-general}):
\begin{itemize}
    \item $\tO\left(n^{\frac{2}{11}} t^{\frac{65}{66}}\right)$ when $t \ge n^{11-\frac{1}{65}}$;
    \item $\tO\left(n^{\frac{4}{5}} t^{\frac{14}{15}}\right)$ when $t \ge n^{10-\frac{1}{7}}$;
    \item $\tO\left(n^{2} t^{\frac{5}{6}}\right)$ when $t \ge n^{9-\frac{3}{5}}$;
    \item $\tO\left(n^{4} t^{\frac{2}{3}}\right)$ when $t \ge n^{6}$;
    \item $\tO\left(n^8\right)$ when $t<n^6$.
\end{itemize}
\end{example}
Figure~\ref{fig:12_1} shows a pictorial representation of the $\cliquelist{12,1}$ runtime.

% Figure environment removed

We can similarly obtain the following corollary for $\cliquelist{k, 2}$. 

\begin{corollary}
\label{cor:k-2-listing-general}
Assume $\omega = 2$. 
Fix any integer  $k \ge 2$, and any integer $2 \le s < \frac{k}{2}$. $\cliquelist{k, 2}$ can be solved in $\tO\left(m^{\frac{s}{k'-1}} t^{1-\frac{2}{k'(k'-1)}} \right)$ time when $t \ge m^{\frac{s}{2} (k'-1-\frac{2}{k'^2-k'-2})}$, where $k' = \lceil \frac{k}{s}\rceil$.
\end{corollary}
\begin{proof}
Apply Theorem~\ref{thm:reduction_from_small_t}. We get $k' = \lceil \frac{k}{s}\rceil\ge 3$, $\ell' = 1$, and an algorithm for $\cliquelist{k, 2}$ that runs in $\tO\left(\left(m^{\frac{s}{2}}\right)^{\alpha_{k', 1}} t^{1-\frac{\alpha_{k',1}}{k'}}\right)$ time when $t \ge \left(m^{\frac{s}{2}}\right)^{\gamma_{k', 1}}$. By Theorem~\ref{thm:k_1_optimal}, when $\omega = 2$, $\alpha_{k', 1} = \frac{2}{k'-1}$ and $\gamma_{k', 1} = k'-1-\frac{2}{k'^2 - k' - 2}$. The corollary then follows. 
\end{proof}


\section{6-Clique Madness}
\label{sec:6clique}
In this section, we show that our algorithm in Section~\ref{sec:upper-bound} is improvable by showing a faster algorithm for $\cliquelist{6, 1}$. See Figure~\ref{fig:6_1} for a comparison of the bounds achieved by the algorithm in Section~\ref{sec:general-list} and this section. 

The new algorithm for $\cliquelist{6, 1}$ comprises two parts:  Algorithms~\ref{alg:6_clique_Algo_1} and \ref{alg:6_clique_Algo_2}. 

\begin{algorithm}
    \caption{$\cliquelist{6, 1}$ Algorithm I.}\label{alg:6_clique_Algo_1}
    \begin{algorithmic}
    \item \textbf{Input:} $(G := (V, E), n, t)$
    \item \textbf{Output:} The list of all $\le t$ $6$-cliques.
    \begin{enumerate}
    \item If $n \le 5$, list nothing and return. 
    \item Call a $K_4$ light if it is contained in at most $\rho$ $K_6$ for some $\rho \ge 1$. Clearly, there are at most $15t/\rho$ dense $K_4$. 
    \item Just as before, we can list all $K_6$ containing light $K_4$ in $\tO(\rho \MM(n^2, \frac{n^2}{\rho}, n^2))$ time. 
    \label{line:6-1-list-denseK4}
    \item Call an edge light if it is contained in at most $\lambda$ $K_6$ for some $\lambda \ge 1$. All other edges are dense. There are at most $15 t/\lambda$ dense edges. 
    \item Just as before, we can list all $K_6$ containing one light edge and one dense $K_4$ that is disjoint with the light edge in $\tO(\lambda \MM(n, \frac{t/\rho}{\lambda}, n))$ time. 
    \label{line:6-1-list-denseedge}
    \item For each node $v$ connected to  $d_v \le x$ dense edges for some $x \ge 1$, run the $\cliquelist{5, 1}$ algorithm from Corollary~\ref{cor:5_1_opt} in its neighbors connected to it by dense edges. Delete this node afterwards. 
    \item The number of remaining nodes is at most $30t/x\lambda$; recurse. 
    \end{enumerate}
    \end{algorithmic}
\end{algorithm}

\begin{remark}
    Intuitively, Algorithm~\ref{alg:6_clique_Algo_1} is similar to Algorithm~\ref{alg:optimal_kl} with one main difference: we first bound the number of 4-cliques in the graph by $O(t/\rho)$ by getting rid of light 4-cliques rather than simply bounding the number of 4-cliques by $n^4$. This idea allows us to get a better bound on $\gamma_6$ than in Theorem~\ref{thm:k_1_optimal}. This idea can also be extended to all $k \geq 6$.
\end{remark}

\begin{lemma}
\label{lem:6_clique_Algo_1}
Algorithm~\ref{alg:6_clique_Algo_1} is correct and runs in $\tO(n^4 + n^{5/2}t^{1/2}+n^{2/5}t^{14/15})$ time if $\omega = 2$. 
\end{lemma}
\begin{proof}
After Line~\ref{line:6-1-list-denseK4}, the algorithm has listed all $K_6$ containing at least one light $K_4$. After Line~\ref{line:6-1-list-denseedge}, the algorithm  has also listed all $K_6$ containing a dense $K_4$ and a disjoint light edge. Thus, after this point, only $K_6$ containing no light edges are not listed. Then clearly, the next two steps list all such $K_6$. 

The running time, excluding the recursion, is (if $\omega = 2$)
\begin{align*}
    &\tO\left(\rho \MM\left(n^2, \frac{n^2}{\rho}, n^2\right) + \lambda \MM\left(n, \frac{t/\rho}{\lambda}, n\right) + \sum_{v: d_v \le x} \left( d_v^4 + d_v^{1/2}\Delta_6(v)^{9/10} \right) \right)\\
    \le & \tO\left(\rho n^4 + \lambda n^2 + \frac{nt}{\rho}  +  (t/\lambda) x^3 + (t/\lambda)^{1/10} t^{9/10} x^{2/5} \right).
\end{align*}
The inequality is due to $\sum_{v}d_v \le O(t/\lambda)$,  $\sum_v \Delta_6(v) \le O(t)$ and H\"{o}lder's inequality. 

We also set $\lambda = \max\{1, \frac{15t}{xn}\}$, so that each recursion level decreases $n$ by a factor of at least $2$. The overall time complexity is thus within $\tO(1)$ of the time complexity of the first recursion level. The running time then becomes (assuming $\frac{15t}{xn} \ge 1$)
\begin{align*}
    \tO\left(\rho n^4 + \frac{nt}{x} + \frac{nt}{\rho}  +  x^4 n + t^{9/10}x^{1/2}n^{1/10} \right).
\end{align*}
The running time of the algorithm is thus
\begin{itemize}
    \item $\tO(n^4)$ when $t \le n^3$ by setting $\rho = 1$ and $x=1$ (even though in this setting, $\frac{15t}{xn}$ will be less than $1$ if $t < n / 15$, the running time still holds by setting $\lambda = 1$);
    \item $\tO(n^{5/2} t^{1/2})$ when $n^3 < t \le n^{63/13}$ by setting $x = \rho = n^{-3/2} t^{1/2}$;
    \item and $\tO(n^{2/5} t^{14/15})$ when $t > n^{63/13}$ by setting $x = n^{3/5} t^{1/15}$ and $\rho = n^{-18/5}t^{14/15}$. \qedhere
\end{itemize}
\end{proof}


In Algorithm~\ref{alg:6_clique_Algo_2}, we show another alternative algorithm for $\cliquelist{6, 1}$ that performs better for different ranges of $t$.  

\begin{algorithm}
    \caption{$\cliquelist{6, 1}$ Algorithm II.}\label{alg:6_clique_Algo_2}
    \begin{algorithmic}
    \item \textbf{Input:} $(G := (V, E), n, t)$
    \item \textbf{Output:} The list of all $\le t$ $6$-cliques
    \item \textbf{The Algorithm:}
    \begin{enumerate}
    \item If $n \le 5$, list nothing and return. 
    \item Call a $K_4$ light if it is contained in at most $\rho$ $K_6$ for some $\rho \ge 1$. Clearly, there are at most $15 t/\rho$ dense $K_4$. 
    \item Just as before, we can list all $K_6$ containing light $K_4$ in $\tO(\rho \MM(n^2, \frac{n^2}{\rho}, n^2))$ time. 
    \item Call an edge $e$ light if it is contained in $q_e \le x$ dense $K_4$ for some $x \ge 1$. All other edges are dense. There are at most $90 t/\rho x$ dense edges. 
    \item For every light edge $e$, we call the $\cliquelist{4, 2}$ algorithm in Section~\ref{sec:4_5_l_listing} using all edges that are disjoint with $e$ and form a dense $K_4$ with $e$. We remove edge $e$ afterwards. 
    \item For each node $v$ connected to  $d_v \le y$ dense edges for some $y \ge 1$, run the $\cliquelist{5, 1}$ algorithm from Corollary~\ref{cor:5_1_opt} in its neighbors connected to it by dense edges. Delete this node afterwards. 
    \item The number of remaining nodes is at most $180 t/\rho x y $; recurse. 
    \end{enumerate}
    \end{algorithmic}
\end{algorithm}

\begin{remark}
    Intuitively, Algorithm~\ref{alg:6_clique_Algo_2} is similar to the algorithm obtained from Theorem~\ref{thm:reduction_from_small_t} by setting $k = 6$ and $s = 2$ and reducing the problem to $\cliquelist{3, 1}$. However, instead of calling $\cliquelist{2, 1}$ (as one would in the usual $\cliquelist{3, 1}$ algorithm), we   call $\cliquelist{4, 2}$ instead. This is better because $\cliquelist{4, 2}$ takes advantage of matrix multiplication whereas $\cliquelist{2, 1}$ simply uses brute-force.
\end{remark}

\begin{lemma}
\label{lem:6_clique_Algo_2}
Algorithm~\ref{alg:6_clique_Algo_2} is correct and runs in $\tO(n^4 + n^{15/7}t^{4/7}+n^{37/21}t^{2/3}+n^{29/25}t^{4/5}+n^{9/10}t^{17/20})$ time. 
\end{lemma}
\begin{proof}
The correctness of the algorithm is almost immediate. The running time of the algorithm, excluding the recursion, is (if $\omega = 2$)
\begin{align*}
    &\tO\left(\rho \MM\left(n^2, \frac{n^2}{\rho}, n^2\right) + \sum_{e: q_e \le x} \left(q_e^{3/2} + q_e \Delta_6(e)^{2/5} + q_e^{1/2}\Delta_6(e)^{3/4} \right)+ \sum_{v: d_v \le y} \left( d_v^4 + d_v^{1/2}\Delta_6(v)^{9/10} \right) \right)\\
    \le & \tO\left(\rho n^4 + (t/\rho)x^{1/2} + (t/\rho)^{3/5}t^{2/5}x^{2/5}+(t/\rho)^{1/4}t^{3/4}x^{1/4}+ (t/\rho x) y^3 + (t/\rho x)^{1/10} t^{9/10}y^{2/5}\right).
\end{align*}
The inequality is due to 
$\sum_e q_e = O(t/\rho), \sum_v d_v = O(t/\rho x)$ and H\"{o}lder's inequality. 
We also set $\rho = \max\{1, \frac{90  t}{xyn}\}$ so that each recursion level decreases $n$ by a factor of at least $2$. The overall time complexity is thus within $\tO(1)$ of the time complexity of the first recursion level. The running time then becomes (assuming $\rho =\frac{90  t}{xyn}$)
$$\tO\left(\frac{n^3 t}{xy} + x^{3/2}yn +xy^{3/5}n^{3/5}t^{2/5}+x^{1/2}y^{1/4}n^{1/4}t^{3/4}+ y^4 n + y^{1/2}n^{1/10}t^{9/10}\right).$$
The running time of the algorithm is thus
\begin{itemize}
    \item $\tO(n^4)$ when $t \le n^{13/4}$ by setting $x = n^{3/2}, y = n^{3/4}$ (even though in this setting, $\frac{90t}{xyn}$ will be less than $1$ if $t < n^{7/4} / 90$,  the running time still holds by setting $\rho = 1$);
    \item $\tO(n^{15/7}t^{4/7})$ when $n^{13/4} < t \le n^4$ by setting $x=n^{4/7}t^{2/7}$ and $y = n^{2/7} t^{1/7}$;
    \item $\tO(n^{37/21}t^{2/3})$ when $n^4 < t \le n^{158/35}$ by setting $x=n^{22/21}t^{1/6}$ and $y = n^{4/21}t^{1/6}$;
    \item $\tO(n^{29/25}t^{4/5})$ when $n^{158/35} < t \le n^{26/5}$ by setting $x=n^{9/5}$ and $y = n^{1/25}t^{1/5}$;
    \item $\tO(n^{9/10}t^{17/20})$ when $n^{26/5} < t \le n^{6}$ by setting $x=n^{1/2}t^{1/4}$ and $y = n^{8/5}t^{-1/10}$ (note that $y$ can be $> n$ sometimes. This would mean all nodes are ``dense nodes'', and we could improve the running time by decreasing $y$ to $n$. Nevertheless, Algorithm I performs better in this regime. )
\end{itemize}
\end{proof}

Combining Lemma~\ref{lem:6_clique_Algo_1} and Lemma~\ref{lem:6_clique_Algo_2}, we obtain the following upper bound for $\cliquelist{6, 1}$. 
\begin{proposition}
If $\omega = 2$, $\cliquelist{6, 1}$ can be solved in time $$\tO\left(\min\left\{n^4 + n^{5/2}t^{1/2}+n^{2/5}t^{14/15}, n^4 + n^{15/7}t^{4/7}+n^{37/21}t^{2/3}+n^{29/25}t^{4/5}+n^{9/10}t^{17/20} \right\}\right).$$
\end{proposition}


\bibliographystyle{alpha}
\bibliography{ref}

\end{document}
