In this section, we show that our algorithm in Section~\ref{sec:upper-bound} is improvable by showing a faster algorithm for $\cliquelist{6, 1}$. See Figure~\ref{fig:6_1} for a comparison of the bounds achieved by the algorithm in Section~\ref{sec:general-list} and this section. 

The new algorithm for $\cliquelist{6, 1}$ comprises two parts:  Algorithms~\ref{alg:6_clique_Algo_1} and \ref{alg:6_clique_Algo_2}. 

\begin{algorithm}
    \caption{$\cliquelist{6, 1}$ Algorithm I.}\label{alg:6_clique_Algo_1}
    \begin{algorithmic}
    \item \textbf{Input:} $(G := (V, E), n, t)$
    \item \textbf{Output:} The list of all $\le t$ $6$-cliques.
    \begin{enumerate}
    \item If $n \le 5$, list nothing and return. 
    \item Call a $K_4$ light if it is contained in at most $\rho$ $K_6$ for some $\rho \ge 1$. Clearly, there are at most $15t/\rho$ dense $K_4$. 
    \item Just as before, we can list all $K_6$ containing light $K_4$ in $\tO(\rho \MM(n^2, \frac{n^2}{\rho}, n^2))$ time. 
    \label{line:6-1-list-denseK4}
    \item Call an edge light if it is contained in at most $\lambda$ $K_6$ for some $\lambda \ge 1$. All other edges are dense. There are at most $15 t/\lambda$ dense edges. 
    \item Just as before, we can list all $K_6$ containing one light edge and one dense $K_4$ that is disjoint with the light edge in $\tO(\lambda \MM(n, \frac{t/\rho}{\lambda}, n))$ time. 
    \label{line:6-1-list-denseedge}
    \item For each node $v$ connected to  $d_v \le x$ dense edges for some $x \ge 1$, run the $\cliquelist{5, 1}$ algorithm from Corollary~\ref{cor:5_1_opt} in its neighbors connected to it by dense edges. Delete this node afterwards. 
    \item The number of remaining nodes is at most $30t/x\lambda$; recurse. 
    \end{enumerate}
    \end{algorithmic}
\end{algorithm}

\begin{remark}
    Intuitively, Algorithm~\ref{alg:6_clique_Algo_1} is similar to Algorithm~\ref{alg:optimal_kl} with one main difference: we first bound the number of 4-cliques in the graph by $O(t/\rho)$ by getting rid of light 4-cliques rather than simply bounding the number of 4-cliques by $n^4$. This idea allows us to get a better bound on $\gamma_6$ than in Theorem~\ref{thm:k_1_optimal}. This idea can also be extended to all $k \geq 6$.
\end{remark}

\begin{lemma}
\label{lem:6_clique_Algo_1}
Algorithm~\ref{alg:6_clique_Algo_1} is correct and runs in $\tO(n^4 + n^{5/2}t^{1/2}+n^{2/5}t^{14/15})$ time if $\omega = 2$. 
\end{lemma}
\begin{proof}
After Line~\ref{line:6-1-list-denseK4}, the algorithm has listed all $K_6$ containing at least one light $K_4$. After Line~\ref{line:6-1-list-denseedge}, the algorithm  has also listed all $K_6$ containing a dense $K_4$ and a disjoint light edge. Thus, after this point, only $K_6$ containing no light edges are not listed. Then clearly, the next two steps list all such $K_6$. 

The running time, excluding the recursion, is (if $\omega = 2$)
\begin{align*}
    &\tO\left(\rho \MM\left(n^2, \frac{n^2}{\rho}, n^2\right) + \lambda \MM\left(n, \frac{t/\rho}{\lambda}, n\right) + \sum_{v: d_v \le x} \left( d_v^4 + d_v^{1/2}\Delta_6(v)^{9/10} \right) \right)\\
    \le & \tO\left(\rho n^4 + \lambda n^2 + \frac{nt}{\rho}  +  (t/\lambda) x^3 + (t/\lambda)^{1/10} t^{9/10} x^{2/5} \right).
\end{align*}
The inequality is due to $\sum_{v}d_v \le O(t/\lambda)$,  $\sum_v \Delta_6(v) \le O(t)$ and H\"{o}lder's inequality. 

We also set $\lambda = \max\{1, \frac{15t}{xn}\}$, so that each recursion level decreases $n$ by a factor of at least $2$. The overall time complexity is thus within $\tO(1)$ of the time complexity of the first recursion level. The running time then becomes (assuming $\frac{15t}{xn} \ge 1$)
\begin{align*}
    \tO\left(\rho n^4 + \frac{nt}{x} + \frac{nt}{\rho}  +  x^4 n + t^{9/10}x^{1/2}n^{1/10} \right).
\end{align*}
The running time of the algorithm is thus
\begin{itemize}
    \item $\tO(n^4)$ when $t \le n^3$ by setting $\rho = 1$ and $x=1$ (even though in this setting, $\frac{15t}{xn}$ will be less than $1$ if $t < n / 15$, the running time still holds by setting $\lambda = 1$);
    \item $\tO(n^{5/2} t^{1/2})$ when $n^3 < t \le n^{63/13}$ by setting $x = \rho = n^{-3/2} t^{1/2}$;
    \item and $\tO(n^{2/5} t^{14/15})$ when $t > n^{63/13}$ by setting $x = n^{3/5} t^{1/15}$ and $\rho = n^{-18/5}t^{14/15}$. \qedhere
\end{itemize}
\end{proof}


In Algorithm~\ref{alg:6_clique_Algo_2}, we show another alternative algorithm for $\cliquelist{6, 1}$ that performs better for different ranges of $t$.  

\begin{algorithm}
    \caption{$\cliquelist{6, 1}$ Algorithm II.}\label{alg:6_clique_Algo_2}
    \begin{algorithmic}
    \item \textbf{Input:} $(G := (V, E), n, t)$
    \item \textbf{Output:} The list of all $\le t$ $6$-cliques
    \item \textbf{The Algorithm:}
    \begin{enumerate}
    \item If $n \le 5$, list nothing and return. 
    \item Call a $K_4$ light if it is contained in at most $\rho$ $K_6$ for some $\rho \ge 1$. Clearly, there are at most $15 t/\rho$ dense $K_4$. 
    \item Just as before, we can list all $K_6$ containing light $K_4$ in $\tO(\rho \MM(n^2, \frac{n^2}{\rho}, n^2))$ time. 
    \item Call an edge $e$ light if it is contained in $q_e \le x$ dense $K_4$ for some $x \ge 1$. All other edges are dense. There are at most $90 t/\rho x$ dense edges. 
    \item For every light edge $e$, we call the $\cliquelist{4, 2}$ algorithm in Section~\ref{sec:4_5_l_listing} using all edges that are disjoint with $e$ and form a dense $K_4$ with $e$. We remove edge $e$ afterwards. 
    \item For each node $v$ connected to  $d_v \le y$ dense edges for some $y \ge 1$, run the $\cliquelist{5, 1}$ algorithm from Corollary~\ref{cor:5_1_opt} in its neighbors connected to it by dense edges. Delete this node afterwards. 
    \item The number of remaining nodes is at most $180 t/\rho x y $; recurse. 
    \end{enumerate}
    \end{algorithmic}
\end{algorithm}

\begin{remark}
    Intuitively, Algorithm~\ref{alg:6_clique_Algo_2} is similar to the algorithm obtained from Theorem~\ref{thm:reduction_from_small_t} by setting $k = 6$ and $s = 2$ and reducing the problem to $\cliquelist{3, 1}$. However, instead of calling $\cliquelist{2, 1}$ (as one would in the usual $\cliquelist{3, 1}$ algorithm), we   call $\cliquelist{4, 2}$ instead. This is better because $\cliquelist{4, 2}$ takes advantage of matrix multiplication whereas $\cliquelist{2, 1}$ simply uses brute-force.
\end{remark}

\begin{lemma}
\label{lem:6_clique_Algo_2}
Algorithm~\ref{alg:6_clique_Algo_2} is correct and runs in $\tO(n^4 + n^{15/7}t^{4/7}+n^{37/21}t^{2/3}+n^{29/25}t^{4/5}+n^{9/10}t^{17/20})$ time. 
\end{lemma}
\begin{proof}
The correctness of the algorithm is almost immediate. The running time of the algorithm, excluding the recursion, is (if $\omega = 2$)
\begin{align*}
    &\tO\left(\rho \MM\left(n^2, \frac{n^2}{\rho}, n^2\right) + \sum_{e: q_e \le x} \left(q_e^{3/2} + q_e \Delta_6(e)^{2/5} + q_e^{1/2}\Delta_6(e)^{3/4} \right)+ \sum_{v: d_v \le y} \left( d_v^4 + d_v^{1/2}\Delta_6(v)^{9/10} \right) \right)\\
    \le & \tO\left(\rho n^4 + (t/\rho)x^{1/2} + (t/\rho)^{3/5}t^{2/5}x^{2/5}+(t/\rho)^{1/4}t^{3/4}x^{1/4}+ (t/\rho x) y^3 + (t/\rho x)^{1/10} t^{9/10}y^{2/5}\right).
\end{align*}
The inequality is due to 
$\sum_e q_e = O(t/\rho), \sum_v d_v = O(t/\rho x)$ and H\"{o}lder's inequality. 
We also set $\rho = \max\{1, \frac{90  t}{xyn}\}$ so that each recursion level decreases $n$ by a factor of at least $2$. The overall time complexity is thus within $\tO(1)$ of the time complexity of the first recursion level. The running time then becomes (assuming $\rho =\frac{90  t}{xyn}$)
$$\tO\left(\frac{n^3 t}{xy} + x^{3/2}yn +xy^{3/5}n^{3/5}t^{2/5}+x^{1/2}y^{1/4}n^{1/4}t^{3/4}+ y^4 n + y^{1/2}n^{1/10}t^{9/10}\right).$$
The running time of the algorithm is thus
\begin{itemize}
    \item $\tO(n^4)$ when $t \le n^{13/4}$ by setting $x = n^{3/2}, y = n^{3/4}$ (even though in this setting, $\frac{90t}{xyn}$ will be less than $1$ if $t < n^{7/4} / 90$,  the running time still holds by setting $\rho = 1$);
    \item $\tO(n^{15/7}t^{4/7})$ when $n^{13/4} < t \le n^4$ by setting $x=n^{4/7}t^{2/7}$ and $y = n^{2/7} t^{1/7}$;
    \item $\tO(n^{37/21}t^{2/3})$ when $n^4 < t \le n^{158/35}$ by setting $x=n^{22/21}t^{1/6}$ and $y = n^{4/21}t^{1/6}$;
    \item $\tO(n^{29/25}t^{4/5})$ when $n^{158/35} < t \le n^{26/5}$ by setting $x=n^{9/5}$ and $y = n^{1/25}t^{1/5}$;
    \item $\tO(n^{9/10}t^{17/20})$ when $n^{26/5} < t \le n^{6}$ by setting $x=n^{1/2}t^{1/4}$ and $y = n^{8/5}t^{-1/10}$ (note that $y$ can be $> n$ sometimes. This would mean all nodes are ``dense nodes'', and we could improve the running time by decreasing $y$ to $n$. Nevertheless, Algorithm I performs better in this regime. )
\end{itemize}
\end{proof}

Combining Lemma~\ref{lem:6_clique_Algo_1} and Lemma~\ref{lem:6_clique_Algo_2}, we obtain the following upper bound for $\cliquelist{6, 1}$. 
\begin{proposition}
If $\omega = 2$, $\cliquelist{6, 1}$ can be solved in time $$\tO\left(\min\left\{n^4 + n^{5/2}t^{1/2}+n^{2/5}t^{14/15}, n^4 + n^{15/7}t^{4/7}+n^{37/21}t^{2/3}+n^{29/25}t^{4/5}+n^{9/10}t^{17/20} \right\}\right).$$
\end{proposition}
