\def\anon{0}
\def\draft{1}

\documentclass[11pt]{article}
\usepackage{color,colortbl,latexsym,amsthm,amsmath,amssymb,path,enumitem,soul,tikz,fullpage,subcaption}
\usetikzlibrary{shapes.geometric}
% \usepackage[hidelinks]{hyperref}
\usepackage[colorlinks,citecolor=blue,linkcolor=blue,urlcolor=red,pagebackref]{hyperref}
\usepackage{algorithm}
\usepackage{algorithmicx}
\usepackage{algpseudocode}
\usepackage{soul}
\usepackage{multirow}
% \usepackage{authblk}

\usepackage{relsize}
\usepackage{diagbox}
\usepackage{comment}
%\geometry{margin=1in}
\usepackage[font={small,sf}]{caption}
\newcommand{\ignore}[1]{}

\makeatother
\makeatletter
\newenvironment{breakablealgorithm}
  {% \begin{breakablealgorithm}
   \begin{center}
     \refstepcounter{algorithm}% New algorithm
     \hrule height.8pt depth0pt \kern2pt% \@fs@pre for \@fs@ruled
     \renewcommand{\caption}[2][\relax]{% Make a new \caption
       {\raggedright\textbf{\ALG@name~\thealgorithm} ##2\par}%
       \ifx\relax##1\relax % #1 is \relax
         \addcontentsline{loa}{algorithm}{\protect\numberline{\thealgorithm}##2}%
       \else % #1 is not \relax
         \addcontentsline{loa}{algorithm}{\protect\numberline{\thealgorithm}##1}%
       \fi
       \kern2pt\hrule\kern2pt
     }
  }{% \end{breakablealgorithm}
     \kern2pt\hrule\relax% \@fs@post for \@fs@ruled
   \end{center}
  }
\makeatother

% \def\marrow{{\marginpar[\hfill$\longrightarrow$]{$\longleftarrow$}}}
% \def\Yinzhan#1{(\textcolor{purple}{{\sc Yinzhan says: }{\marrow\sf #1}})}
% \def\Surya#1{(\textcolor{cyan}{{\sc Surya says: }{\marrow\sf #1}})}
% \def\Virgi#1{(\textcolor{red}{{\sc Virginia says: }{\marrow\sf #1}})}
% \newcommand{\todo}[1]{\textcolor{blue}{TODO: #1}}
% \newcommand{\towrite}[1]{\textcolor{magenta}{Write about: #1}}

%%%%%%%%%%%%%%%%%%%%%%%%%%%%%%%%%%%%%%%%%%%%%%%%%%%%%%%%%%%%% FORMAT
\iffalse
\setlength{\textwidth}{6.0in}
\setlength{\evensidemargin}{0.25in}
\setlength{\oddsidemargin}{0.25in}
\setlength{\textheight}{8.4in}
\setlength{\topmargin}{-0.5in} 
%\setlength{\parskip}{2mm}
\setlength{\parskip}{0mm}
\setlength{\baselineskip}{1.7\baselineskip}
\fi 

\setlist[itemize]{itemsep=-0.1em}
\setlist[enumerate]{itemsep=-0.1em}

%%%%%%%%%%%%%%%%%%%%%%%%%%%%%%%%%%%%%%%%%%%%%%%%%%%%%% THEOREMS ETC.

%theorem titles in bold
\makeatletter
\def\th@plain{%
  \thm@notefont{}% same as heading font
  \itshape % body font
}
\def\th@definition{%
  \thm@notefont{}% same as heading font
  \normalfont % body font
}
\makeatother
% up to here

\newtheorem{theorem}{Theorem}[section]
\newtheorem{lemma}[theorem]{Lemma}
\newtheorem{corollary}[theorem]{Corollary}
\newtheorem{observation}[theorem]{Observation}
\newtheorem{conjecture}[theorem]{Conjecture}
\newtheorem{hypothesis}[theorem]{Hypothesis}
\newtheorem{question}{Question}
\newtheorem{claim}[theorem]{Claim}
\newtheorem{definition}[theorem]{Definition}
\newtheorem{problem}[theorem]{Problem}
\newtheorem{remark}[theorem]{Remark}
\newtheorem{proposition}[theorem]{Proposition}
\newtheorem{fact}[theorem]{Fact}
\newtheorem{example}[theorem]{Example}
%%%%%%%%%%%%%%%%%%%%%%%%%%%%%%%%%%%%%%%%%%%%%%%%%%%%%%%%%%%% FIGURES
\usepackage{graphicx,psfrag}

\newcommand{\mycaption}[1]
        {\caption{\small #1}}
\newcommand{\placefig}[2]
        {% Figure removed}

\usepackage{listings,xcolor}
\iffalse
\lstset{language=Mathematica}
\lstset{basicstyle={\sffamily\footnotesize},
  numbers=left,
  lineskip=1mm,
  numberstyle=\tiny\color{gray},
  numbersep=5pt,
  breaklines=true,
  captionpos={t},
  frame={lines},
  rulecolor=\color{black},
  framerule=0.5pt,
  columns=flexible,
  tabsize=2
}
\fi
\lstset{language=Mathematica}
\lstset{basicstyle={\sffamily\footnotesize},
  numbers=left,
  lineskip=1mm,
  numberstyle=\tiny\color{gray},
  numbersep=5pt,
  breaklines=true,
  captionpos={t},
  frame={lines},
  rulecolor=\color{black},
  framerule=0.5pt,
  columns=flexible,
  tabsize=2,
  mathescape
}

%%%%%%%%%%%%%%%%%%%%%%%%%%%%%%%%%%%%%%%%%%%%%%%%%%%%%%%%%%  NUMBERS
\usepackage{dsfont}
\newcommand{\NN}{\ensuremath{\mathbb N}}
\newcommand{\ZZ}{\ensuremath{\mathbb Z}}
\newcommand{\RR}{\ensuremath{\mathbb R}}
\newcommand{\CC}{\ensuremath{\mathbb C}}
\newcommand{\TT}{\ensuremath{\mathbb T}}
\newcommand{\FF}{\ensuremath{\mathbb F}}
\newcommand{\DD}{\ensuremath{\mathbb D}}
\newcommand{\PP}{\ensuremath{\mathbb P}}
\newcommand{\EE}{\ensuremath{\mathbb E}}
\renewcommand{\epsilon}{\varepsilon}
\newcommand{\eps}{\varepsilon}
\newcommand{\NNnull}{\NN_{^{_0}\!}}
\newcommand{\RRplus}   {\RR^{\!^+}}
\newcommand{\RRplusnull}{\RR^{\!^+}_{^{_0}\!}}
%%%%%%%%%%%%%%%%%%%%%%%%%%%%%%%%%%%%%%%%%%%%%%%%%%%%%%% PROBABILITY
\newcommand{\prr}       {{\mathrm{P\hspace{-0.05em}r}}}
\newcommand{\prrprime}  {{\mathrm{P\hspace{-0.05em}r}'}}
\newcommand{\pr}[1]     {{\,\prr\hspace{-0.05em}\left(#1\right)}}
\newcommand{\prprime}[1]{{\,\prrprime\hspace{-0.05em}\left(#1\right)}}
\newcommand{\ex}[1]     {{\,{\ensuremath{\mathbb E}}\!\left[#1\right]}}
\newcommand{\var}[1]    {{\,{\mathrm{var}}\!\left(#1\right)}}
\newcommand{\cond}      {\!\mid\!}
%%%%%%%%%%%%%%%%%%%%%%%%%%%%%%%%%%%%%%%%%%%%%%%%%%%% PAPER SPECIFIC

\def \poly {\mathop{\rm poly}} %{ \text{\rm poly~} }
\def \polylog {\mathop{\rm polylog}}%{ \text{\rm polylog~} }
\def\th{{\text{\rm th}}}
\def\tO{\tilde{O}}

\def\eps{{\varepsilon}}

\newcommand{\parag}[1]{\vspace{2mm}

\noindent{\bf #1} }

%%%%%%%%%%%%%%%%%%%%%%%%%%%%%%%%%%%%%%%%%%%%%
% TODO and comment Shortcuts
% \newcommand{\todo}[1]{\textcolor{magenta}{TODO: #1}}
% \newcommand{\mathtodo}[1]{\textcolor{red}{Math TODO: #1}}

\ifnum\draft=1
\newcommand{\mina}[1]{\textcolor{red}{(Mina: #1)}}
\newcommand{\surya}[1]{\textcolor{magenta}{(Surya: #1)}}
\newcommand{\yinzhan}[1]{\textcolor{green}{(Yinzhan: #1)}}
\newcommand{\virgi}[1]{\textcolor{red}{(Virginia: #1)}}
\else
    \newcommand{\mina}[1]{}
    \newcommand{\surya}[1]{}
    \newcommand{\yinzhan}[1]{}
    \newcommand{\virgi}[1]{}
\fi 
%%%%%%%%%%%%%%%%%%%%%%%%%%%%%%%%%%%%%%%%%%%%%
\newcommand\cliquelist[1]{(#1)\text{-}\mathsf{Clique}\text{-}\mathsf{Listing}}
\newcommand\cliquedet[1]{(#1)\text{-}\mathsf{Clique}\text{-}\mathsf{Detection}}
\newcommand{\exactclique}[1]{\mathsf{Exact}\text{-}#1\text{-}\mathsf{Clique}}
\newcommand\MM{\mathsf{MM}}
%%%%%%%%%%%%%%%%
\newcommand\sparse{{\tt Sparse}}
\newcommand\dense{{\tt Dense}}
%%%%%%%%%%%%%%%%



\title{Listing Cliques from Smaller Cliques}
\ifnum\anon=1
    \author{Anonymous}
\else
    \author{\normalsize Mina Dalirrooyfard\thanks{Morgan Stanley Research. \texttt{minad@mit.edu}. While at MIT, supported by a Google Faculty Research Award and an Akamai MIT CS Theory Group Fellowship.} \and \normalsize Surya Mathialagan\thanks{MIT. \texttt{smathi@mit.edu}. Supported by the Siebel Scholars program, by DARPA under Agreement No. HR00112020023 and by NSF grant CNS-2154149.}
    \and \normalsize Virginia Vassilevska Williams\thanks{MIT. \texttt{virgi@mit.edu}. Partially supported by NSF Career Award CCF-1651838, NSF Grant CCF-2129139, a Sloan Research Fellowship and a Google Faculty Research Award.} 
    \and \normalsize  Yinzhan Xu\thanks{MIT. \texttt{xyzhan@mit.edu}. Supported by NSF Grant CCF-2129139.}}
\fi 

\date{}
\begin{document}

\maketitle 
\pagenumbering{gobble} 
\begin{abstract}
We study finding and listing $k$-cliques in a graph, for constant $k\geq 3$, a fundamental problem of both theoretical and practical importance.

Our main contribution is a new output-sensitive algorithm for listing $k$-cliques in graphs, for arbitrary $k\geq 3$, coupled with lower bounds based on standard fine-grained assumptions, showing that our algorithm's running time is tight. Previously, the only known conditionally optimal output-sensitive algorithms were for the case of $3$-cliques by Bj\"{o}rklund, Pagh, Vassilevska W. and Zwick [ICALP'14].

Typical inputs to subgraph isomorphism or listing problems are measured by the number of nodes $n$ or the number of edges $m$. Our framework is very general in that it gives $k$-clique listing algorithms whose running times are measured in terms of the number of $\ell$-cliques $\Delta_\ell$ in the graph for any $1\leq \ell<k$. This generalizes the typical parameterization in terms of $n$ (the number of $1$-cliques) and $m$ (the number of $2$-cliques).

If the matrix multiplication exponent $\omega$ is $2$, and if the size of the output, $\Delta_k$, is sufficiently large, then for every $\ell<k$, the running time of our algorithm for listing $k$-cliques is 
$$\tilde{O}\left(\Delta_\ell^{\frac{2}{\ell (k - \ell)}}\Delta_k^{1-\frac{2}{k(k-\ell)}}\right).$$ 
For sufficiently large $\Delta_k$, we prove that this runtime is in fact {\em optimal} for all $1 \leq \ell < k$ under the Exact $k$-Clique hypothesis. 

In the special cases of $k = 4$ and $5$, our algorithm in terms of $n$ is conditionally optimal for all values of $\Delta_k$ if $\omega = 2$. Moreover, our framework is powerful enough to provide an improvement upon the 19-year old runtimes for $4$ and $5$-clique detection  in $m$-edge graphs, as a function of $m$ [Eisenbrand and Grandoni, TCS'04]. 
\end{abstract}

\newpage


\tableofcontents{}
\newpage 
\pagenumbering{arabic} 
\section{Introduction}
\input{intro-new}


\section{Preliminaries}
\label{sec:prelim}

We first review some basic concepts from probability theory (see standard textbooks such as \cite{pollard2002user,williams1991probability} for a detailed treatment), 
%the background of Bayesian inference, and finally 
%We first review some basic concepts from probability theory, 
and then present the Bayesian probabilistic programming language and the normalised posterior distribution (NPD) problem.
%we consider in this work. 
Throughout the paper,
we denote by $\Nset$, $\Zset$ and $\Rset$ the sets of all natural numbers (including zero), integers, and real numbers, respectively.

\vspace{-1.5ex}
\subsection{Basics of Probability Theory}
%We assume familiarity with basic probability theory (see \cref{app:prelim} for details). 

A \emph{measurable space} is a pair $(U,\Sigma_U)$, where $U$ is a nonempty set and $\Sigma_U$ is a $\sigma$-algebra on $U$, i.e., a family of subsets of $U$ such that $\Sigma_U\subseteq \mathcal{P}(U)$ contains $\emptyset$ and is closed under complementation and countable union. Elements of $\Sigma_U$ are called \emph{measurable} sets. A function $f$ from a measurable space $(U_1,\Sigma_{U_1})$ to another measurable space $(U_2,\Sigma_{U_2})$ is \emph{measurable} if $f^{-1}(A)\in\Sigma_{U_1}$ for all $A\in\Sigma_{U_2}$.

A \emph{measure} $\mu$ on a measurable space $(U,\Sigma_U)$ is a mapping from $\Sigma_U$ to $[0,\infty]$ such that (i) $\mu(\emptyset)=0$ and (ii) $\mu$ 
%satisfies the
is countably additive:
%condition: 
for every pairwise-disjoint set sequence $\{A_n\}_{n\in\Nset}$ in $\Sigma_U$, it holds that $\mu(\bigcup_{n\in\Nset}A_n)=\sum_{n\in\Nset}\mu(A_n)$. We call the triple $(U,\Sigma_U,\mu)$ a \emph{measure space}. 
%If $\mu(U)\le 1$, we call $\mu$ a \emph{subprobability measure}. 
If $\mu(U)=1$, we call $\mu$ a \emph{probability measure}, and $(U,\Sigma_U,\mu)$ a \emph{probability space}.
The Lebesgue measure $\lambda$ is the unique measure on $(\Rset,\Sigma_{\Rset})$ satisfying $\lambda([a,b))=b-a$ for all valid intervals $[a,b)$ in $\Sigma_{\Rset}$. For each $n\in\Nset$, we have a measurable space $(\Rset^n,\Sigma_{\Rset^n})$ 
%such that there exists 
and
a unique product measure $\lambda_n$ on $\Rset^n$ satisfying $\lambda_n(\prod_{i=1}^n A_i)=\prod_{i=1}^n \lambda(A_i)$ for all $A_i\in\Sigma_{\Rset}$.


The \emph{Lebesgue} integral operator $\int$ is a partial operator that maps a measure $\mu$ on $(U,\Sigma_U)$ and a real-valued function $f$ on the same space $(U,\Sigma_U)$ to a real number or infinity, which is denoted by $\int f \mathrm{d}\mu$ or $\int f(x)\mu(\mathrm{d}x)$. 
The detailed definition of Lebesgue integral is somewhat technical, see \cite{rankin1968real,rudin1976principles} for more details. 
Given a measurable set $A\in\Sigma_U$, the integral of $f$ over $A$ is defined by $\int_A f(x)\mu(\mathrm{d} x):=\int f(x) \cdot [x\in A] \mu(\mathrm{d}x)$
%\begin{align*}
%\textstyle\int_A f(x)\mu(\mathrm{d} x):=\int f(x) \cdot [x\in A] \mu(\mathrm{d}x)
%\end{align*} 
where $[-]$ is the Iverson bracket such that $[\phi]=1$ if 
%the predicate 
$\phi$ is true, and $0$ otherwise. If $\mu$ is a probability measure, then we call the integral as the \emph{expectation} of $f$, denoted by $\expectdist{x\sim\mu;A}{f}$, or $\expv[f]$ when the scope is clear from the context.

For a measure $v$ on $(U,\Sigma_U)$, a measurable function $f:U\to \Rset_{\ge 0}$ is the \emph{density} of $v$ with respect to $\mu$ if $v(A)=\int f(x)\cdot [x\in A] \mu(\mathrm{d} x)$ for all measurable $A\in\Sigma_U$, and $\mu$ is called the \emph{reference measure} (most often $\mu$ is the Lebesgue measure). Common families of probability distributions on the reals, e.g., uniform, normal distributions, are measures on $(\Rset,\Sigma_{\Rset})$. Most often these are defined in terms of probability density functions with respect to the Lebesgue measure. That is, for each $\mu_D$ there is a measurable function $\text{pdf}_D:\Rset\to\Rset_{\ge 0}$ that determines it: $\mu_D(A):=\int_A \text{pdf}_D (\mathrm{d}\lambda) $. As we will see, density functions such as $\text{pdf}_D$ play an important role in Bayesian inference.

Given a probability space $\pspace$, a \emph{random variable} is an $\mathcal{F}$-measurable function $X: \Omega \rightarrow \Rset \cup \{+\infty,-\infty\}$. The expectation of a random variable $X$, denoted by $\expv(X)$, is the Lebesgue integral of $X$ w.r.t. $\probm$, i.e., $\int X\,\mathrm{d}\probm$. A \emph{filtration} of $\pspace$ is an infinite sequence $\{ \mathcal{F}_n \}_{n=0}^{\infty}$ such that for every $n\ge 0$, the triple $(\Omega, \mathcal{F}_n, \probm)$ is a probability space and $\mathcal{F}_n \subseteq \mathcal{F}_{n+1} \subseteq \mathcal{F}$. A \emph{stopping time} w.r.t. $\{ \mathcal{F}_n \}_{n=0}^{\infty}$ is a random variable $T: \Omega \rightarrow \Nset \cup \{0, \infty\}$ such that for every $n \geq 0$, the event \{$T \leq n$\} is in $\mathcal{F}_n$. 

A \emph{discrete-time stochastic process} is a sequence $\Gamma = \{X_n\}_{n=0}^\infty$ of random variables in $\pspace$. The process $\Gamma$ is \emph{adapted} to a filtration $\{ \mathcal{F}_n \}_{n=0}^{\infty}$, if for all $n \geq 0$, $X_n$ is a random variable in $(\Omega, \mathcal{F}_n, \probm)$. A discrete-time stochastic process $\Gamma=\{X_n\}_{n=0}^\infty$ adapted to a filtration $\{\mathcal{F}_n\}_{n=0}^\infty$ is a \emph{martingale} (resp. \emph{supermartingale}, \emph{submartingale})
if for all $n \geq 0$, $\expv(|X_n|)<\infty$ and it holds almost surely (i.e.,~with probability $1$) that
$\condexpv{X_{n+1}}{\mathcal{F}_n}=X_n$ (\mbox{resp. } $\condexpv{X_{n+1}}{\mathcal{F}_n}\le X_n$, $\condexpv{X_{n+1}}{\mathcal{F}_n}\ge X_n$).
See~\cite{williams1991probability} for details.
%Intuitively, a martingale is a discrete-time stochastic process, in which at any time $n$, the expected value $\condexpv{X_{n+1}}{\mathcal{F}_n}$ in the next step, given all previous values, is equal to the current value $X_n$. In a supermartingale, this expected value is less than or equal to the current value and a submartingale is defined conversely.
Applying martingales to qualitative and quantitative analysis of probabilistic programs is a well-studied technique~\cite{SriramCAV,ChatterjeeFG16,ChatterjeeNZ2017}.


\subsection{Bayesian Probabilistic Programming Language}

%We consider an imperative arithmetic probabilistic programming language. 
The syntax of our probabilistic programming language (PPL) is given in \cref{fig:syntax}, where the metavariables $S$, $B$ and $E$ stand for statements, boolean expressions and arithmetic expressions, respectively.   
Our PPL is imperative with the usual conditional and loop structures (i.e.,~\textbf{if} and \textbf{while}), as well as the following new structures: (a)~sample constructs of the form ``$\textbf{sample}\  D$'' that sample a value from a prescribed distribution $D$ over $\mathbb{R}$ and then assign this value to a sampling variable $r$; (b)~score statements of the form ``\textbf{score}($EW$)'' that weight the current execution with a value expressed by $EW$ (note that $\textit{pdf}(D,x)$ means the value of a probability density function w.r.t. $D$ at $x$);
%\footnote{Instead of the hard conditioning that refutes the execution when the observation mismatches the value of the sampling variable, we use the more general soft conditioning and assume the existence of a global weight variable initialized  to $1$.}
%for each program
(c)~probabilistic branching statements of the form
``$\textbf{if}\ \textbf{prob}(p)\dots$'' that lead to the then part with probability
$p\in (0,1]$ and to the else part with probability $1-p$. We also have sequential compositions (i.e., ";") and support return statements (i.e., \textbf{return}) that 
return the value of the program variable of interest. %The set of all statements is denoted by $Stmt$.
Note that $c,c_1,c_2\in\Rset$ are constants, and our language supports any distributions with continuous density functions and infinite supports, 
including but not limited to uniform and normal distributions. 



% Figure environment removed





Given a probabilistic program in our language, we distinguish two disjoint sets of variables in the program: (i) the set $\pvars$ of \emph{program variables} whose values are determined by assignments in the program (i.e., the expressions at the LHS of ``:="); (ii)~the set $\rvars$ of \emph{sampling variables} whose values are independently sampled from prescribed probability distributions each time they are accessed (i.e., each ``$\textbf{sample}\ D$" can be regarded as a sampling variable $r$). 




\begin{example}\label{ex:pedestrian-program}

%Consider the pedestrian random walk example~\cite{DBLP:conf/esop/MakOPW21}, a pedestrian is lost on a road, and she only knows that she is away from her house at most $3$ km. Thus, she starts to repeatedly walk a uniformly random distance of at most $1$ km in either direction, until reaching her house. Upon she arrives, an  odometer tells that she has walked $1.1$ km totally. However, this odometer was once broken and the measured distance is normally distributed around the true distance with standard deviation $0.1$ km. 
\cref{fig:pedestrian-program} shows a Bayesian probabilistic program written in our PPL language. In this program, the set of program variables is $\pvars=\{start,pos,dis,step\}$, and the set of sampling variables is $\rvars=\{ \textbf{sample uniform}(0,1)\}$. Each time $\textbf{sample uniform}(0,1)$ is executed, it samples a value uniformly from $[0,1]$ and then assigns the value to the variable $step$. 
%Thus, $step$ is associated with the probability distribution $\textbf{uniform}(0,1)$.
\qed


	
% Figure environment removed
\end{example}

\subsection{The Semantics of Our Programming Language}

%To relate variables with their values, we introduce the notion of valuations. 
Let $V$ be a finite set of variables with an implicit linear order over its elements. A \emph{valuation} on $V$ is a function $\pv: V \rightarrow \Rset$ that assigns a real value to each variable in $V$. We denote the set of all valuations on $V$ by $\val{V}$. For each $1\le i\le |V|$, we denote the value of the $i$-th variable (in the implicit linear order) in $\pv$ by $\pv[i]$, so that we can view each valuation as a real vector on $V$. A \emph{program} (resp. \emph{sampling}) valuation is a valuation on $\pvars$ (resp. $\rvars$), respectively. 
For the sake of convenience, we fix the notations in the following way, i.e., we always use $\pv\in\val{\pvars}$ to denote a program valuation, and $\rv\in\val{\rvars}$ to denote a sampling valuation; we also write $\pv[\mathit{ret}]$ for the value of the return variable in $\pv$. 



Below we present the semantics for our programming language. Existing semantics in the literature are either measure-\cite{DBLP:conf/lics/StatonYWHK16,LeeYRY20} or sampling-based  \cite{DBLP:conf/esop/MakOPW21,Beutner2022b}. To facilitate the development of our algorithm, we consider the \emph{transition-based} semantics~\cite{DBLP:conf/cav/ChakarovS13,DBLP:conf/popl/ChatterjeeFNH16} to our language and 
%To apply template-based algorithmic approaches to NPD problems, we consider  that 
treat each probabilistic program as a \emph{weighted probabilistic transition system} (WPTS). A WPTS extends a PTS  ~\cite{DBLP:conf/cav/ChakarovS13,DBLP:conf/popl/ChatterjeeFNH16} with weights and an initial probability distribution. 





%Below we present a variant of probabilistic transition systems \cite{DBLP:conf/cav/ChakarovS13}.
\begin{definition}
%[Weighted Probabilistic Transition Systems]
[WPTS]\label{def:wpts}
	A \emph{weighted probabilistic transition system} (WPTS) $\Pi$
	is a tuple
\begin{equation}\label{eq:wpts} 
\tag{\dag}
\Pi = (\pvars, \rvars,  L,\lin,\lout,\mu_{\mathrm{init}}, \rdvarjdis,\transset)%\win)
\end{equation}
for which:
	\begin{itemize}
		\item
		$\pvars$ and $\rvars$ are finite disjoint sets of \emph{program} and resp. \emph{sampling} variables.
%  (variables}) 
%  such that $\pvars\cap \rvars=\emptyset$.
    \item $\locs$ is a finite set of \emph{locations} 
  %or \emph{program counters} 
  with special locations $\lin,\lout\in \locs$. Informally, $\lin$ is the initial location and $\lout$ represents program termination. 
		\item
		$\mu_{\mathrm{init}}$ is the \emph{initial probability distribution} over $\mathbb{R}^{\pvars}$ with a finite support (denoted by $\supp{\mu_{\mathrm{init}}}$), 
  %from which the initial program valuation %$\valin$ is sampled, 
  while $\rdvarjdis$ is a function that assigns a probability distribution $\rdvarjdis(r)$ to each 
  %sampling variable 
  $r \in \rvars$. We call each $\pv\in\supp{\mu_{\mathrm{init}}}$ an \emph{initial program valuation}, and abuse the notation so that $\rdvarjdis$ also denotes the independent joint distribution of all $\rdvarjdis(r)$'s ($r\in \rvars$).
		\item 
		$\transset$ is a finite set of \emph{transitions} where
		each transition $\tau \in \transset$ is a tuple $\langle \loc, \phi, F_1,\dots,F_k \rangle$ such that 
(i) $\loc\in L$ is the \emph{source location} of the transition, 
%\item 
(ii) $\phi$ is the \emph{guard condition} which is a predicate over variables $\pvars$, %which serves as the \emph{guard condition}, 
and (iii) each $F_j:=\langle \loc'_j, p_j, \upd_j,\wet_j \rangle$ is called a \emph{weighted fork} for which (a) $\loc'_j\in L$ is the \emph{destination location} of the fork, (b) $p_j\in (0,1]$ is the probability of this fork, (c) $\upd_j:\Rset^{|\pvars|} \times \Rset^{|\rvars|} \rightarrow \Rset^{|\pvars|}$ is an {\em update function} that takes as inputs the current program and sampling valuations  and returns an updated program valuation in the next step, and (d) $\wet_j:\Rset^{|\pvars|} \times \Rset^{|\rvars|}\to [0,\infty)$ is a \emph{score function} that gives the likelihood weight of this fork depending on the current program and sampling valuations.	
\end{itemize}
\end{definition}


In a WPTS, we use update and score functions to model the update on the program variables and resp. the likelihood weight when running a basic block of statements in a program, respectively.  
%and use score functions to model  caused by the execution of the score statements (if exists) in this block. 
If there is no score statement in the block, then the score function is constantly $1$. 
We always assume that a WPTS $\Pi$ is \emph{deterministic} and \emph{total}, i.e., (i) there is no program valuation that simultaneously satisfies the guard conditions of two distinct transitions from the same source location, and (ii) the disjunction of the guard conditions of all the transitions from any source location is a tautology. 
The transformation from a probabilistic program into its WPTS can be done in a straightforward way (see e.g.~\cite{DBLP:journals/toplas/ChatterjeeFNH18,DBLP:conf/cav/ChakarovS13}). 

\begin{example}\label{ex:pedestrian-semantics} 
\cref{fig:pedestrian-wpts} shows the WPTS of the program in \cref{fig:pedestrian-program} which has two locations $\lin,\lout$. 
 %In the WPTS, 
The circle nodes represent locations and square nodes model the forking behavior of transitions. An edge entering a square node is labeled with the condition of its respective transition, while an edge entering a circle node stands for a fork, which is associated with its probability, update functions and score functions that marked by $w$.\footnote{Here we omit the update functions if the values of program variables are unchanged.} The value of $step$ is initialised to $0$. An the initial probability distribution $\mu_{\mathrm{init}}$ is determined by the joint distribution of $(start,pos,dis,step)$ where $start\sim uniform(0,3)$ and $pos,dis,step$ observe the Dirac measures $Dirac(\{start\})$, $Dirac(\{0\})$ and $Dirac(\{0\})$, respectively, e.g., the probability of the event ``$dis\in\{0\}$'' equals $1$. As $step$ simply receives values from a sampling variable, we neglect it in the WPTS.\qed
\end{example}

%\paragraph{Score-recursive WPTS.} 

We say that a WPTS is \emph{non-score-recursive} if for all transitions $\tau=\langle \loc, \phi, F_1,  F_2,\dots,F_k \rangle$ in the WPTS with each fork $F_j=\langle \loc'_j, p_j, \upd_j,\wet_j \rangle$ ($1\le j\le k$), we have that each score function $\wet_j$ is constantly $1$ (i.e., the multiplicative weight does not change) for every $\loc'_j\ne \lout$. Otherwise, the WPTS is \emph{score-recursive}.
Informally, a non-score-recursive WPTS has non-trivial score functions only on the transitions to the termination of a program, while a score-recursive WPTS has {\tt score} statements in the execution of the program. 
For example, the WPTS of the program in~\cref{sec3:pedestrian} is non-score-recursive as the nontrivial (i.e., score values not equal to $1$) {\tt score} statement only appears to the termination, while the WPTS of the program in \cref{sec3:phylogenetic} is score recursive since it has {\tt score} statements inside the loop body.
In the case of a non-score-recursive WPTS, we say that the WPTS is \emph{score-bounded} by a positive real $M>0$ if for every $\tau=\langle \loc, \phi, F_1, F_2,\dots,F_k \rangle$ in the WPTS with $F_j=\langle \loc'_j, p_j, \upd_j,\wet_j \rangle$ ($1\le j\le k$), we have that 
$|\wet_j|\le M$ whenever $\loc'_j=\lout$.


Given a program valuation $\mathbf{v}$ and a predicate $\phi$ over variables $\pvars$, we say that $\mathbf{v}$ \emph{satisfies} $\phi$ (written as $\mathbf{v}\models\phi$) if $\phi$ holds when the variables in $\phi$ are substituted by their values in $\mathbf{v}$. 
A \emph{state} 
%of the WPTS $\Pi$ 
is a pair $\Xi=(\loc, \pv)$ where $\loc \in L$ (resp. $\pv \in \Rset^{|\pvars|}$) represents the current location (resp. program valuation), respectively, while a \emph{weighted state} is a triple 
%$\Xi^w:=(\loc, \pv,w)$ 
$\Theta=(\loc, \pv, w)$ 
where $(\loc, \pv)$ is a state and $w\in [0,\infty)$ represents the multiplicative likelihood weight accumulated so far. 


 
%\paragraph{Semantics.} 
Below we specify the semantics of a WPTS. Consider a WPTS $\Pi$ in the form of \eqref{eq:wpts}. The semantics of $\Pi$ is formalized by the infinite sequence $\Gamma=\{\widehat{\Theta}_n=(\widehat{\loc}_n,\widehat{\pv}_n,\widehat{w}_n)\}_{n\ge 0}$ 
%of \emph{random weighted states} 
where each $(\widehat{\loc}_n,\widehat{\pv}_n,\widehat{w}_n)$ is the random weighted state at the $n$th execution step of the WPTS such that $\widehat{\loc}_n$ (resp. $\widehat{\pv}_n$, $\widehat{w}_n$) is the random variable for the location (resp. the random vector 
%of random variables 
for the program valuation, the random variable for the multiplicative likelihood weight) at the $n$th step, respectively. %The initial random state $\widehat{\Theta}_0$ is constant and equals $(\lin,\valin,\win)$. 
%its corresponding stochastic process $\Gamma:=\{\hat{\Xi}_n\}_{n\ge 0}$ on states.
The sequence $\Gamma$ starts with the initial random weighted state 
$\widehat{\Theta}_0=(\widehat{\loc}_0,\widehat{\pv}_0,\widehat{w}_0)$ such that $\widehat{\loc}_0$ is constantly $\lin$, $\widehat{\pv}_0\in \supp{\mu_\mathrm{init}}$ is sampled from the initial distribution $\mu_\mathrm{init}$ and the initial weight $\widehat{w}_0$ is constantly set to $1$\footnote{This follows the traditional setting in e.g.~\cite{Beutner2022b}.}. 
Then, given the current random weighted state $\widehat{\Theta}_n=(\widehat{\loc}_n,\widehat{\pv}_n,\widehat{w}_n)$ at the $n$th step, the next random weighted state $\widehat{\Theta}_{n+1}=(\widehat{\loc}_{n+1},\widehat{\pv}_{n+1},\widehat{w}_{n+1})$ is determined by:
(a) If $\widehat{\loc}_n=\lout$, then $(\widehat{\loc}_{n+1}, \widehat{\pv}_{n+1},\widehat{w}_{n+1})$ takes the same weighted state as $(\widehat{\loc}_n,\widehat{\pv}_n,\widehat{w}_n)$ (i.e., the next weighted state stays at the termination location $\lout$);
(b) Otherwise, $\widehat{\Theta}_{n+1}$ is determined by the following procedure:
\begin{itemize}
\item First, since the WPTS $\Pi$ is deterministic and total, we take the unique transition $\tau=\langle \hat{\loc}_n,\phi,F_1,\dots, F_k \rangle$ such that $\hat{\pv}_n\models\phi$. 
\item Second, we choose a fork $F_j=\langle \loc_j, p_j,\upd_j,\wet_j\rangle$ with probability $p_j$.
\item 
Third, we obtain a sampling valuation $\rv\in \supp{\rdvarjdis}$ 
%over the sampling variables $\rvars$ 
by sampling each $r \in \rvars$ independently w.r.t the probability distribution $\rdvarjdis(r).$
\item Finally, the value of the next random weighted state $(\widehat{\loc}_{n+1}, \widehat{\pv}_{n+1},\widehat{w}_{n+1})$ is determined as that of 
$(\loc'_j, \upd_j(\hat{\pv}_n,\rv),\widehat{w}_n\cdot \wet_j(\widehat{\pv}_n,\rv))$. 
\end{itemize}


Given the semantics, a \emph{program run} of the WPTS $\Pi$ is a concrete instance of $\Gamma$, i.e., an infinite sequence $\omega=\{\Theta_n\}_{n\ge 0}$ of weighted states where each $\Theta_n=(\loc_n,\pv_n,w_n)$ is the concrete weighted state at the $n$th step in this program run with location $\loc_n$, program valuation $\pv_n$ and multiplicative likelihood weight $w_n$. A state $(\loc,\pv)$ is called \emph{reachable} if there exists a program run $\omega=\{\Theta_n\}_{n\ge 0}$ such that $\Theta_n=(\loc,\pv,w_n)$ for some $n$. 


 
\begin{example}\label{ex:pedestrian-run}
Consider the WPTS in \cref{ex:pedestrian-semantics}. Consider an initial program valuation $(1,1,0)$ which means that the initial values of $start,pos,dis$ are $1,1,0$, respectively. Then starting from the initial weighted state $(\lin,(1,1,0),1)$, a program run w.r.t the WPTS semantics above could be 
\[
(\lin,(1,1,0),1)\to (\lin,(1,0.5,0.5),1)\to (\lin,(1,-0.1,1.1),1)\to (\lout,(1,-0.1,1.1),3.9894).\qed
\]
\end{example}

Given an initial program valuation $\valin$ of a WPTS, one could construct a probability space over the program runs by their probabilistic evolution described above and standard constructions such as general state space Markov chains~\cite{meyn2012markov}. We denote the probability measure in the probability space by $\probm_{\valin}(-)$ and the expectation operator by $\expectdist{\valin}{-}$.  



\subsection{Normalised Posterior Distribution}\label{sec2:NPD}


Before presenting the central problem of Bayesian probabilistic programming, i.e., analyzing normalised posterior distribution with our WPTS models, we introduce some technical concepts.

%\paragraph{Termination.}
\begin{definition}[Termination]
The \emph{termination time} of a WPTS
%The \emph{termination time} of the WPTS 
$\Pi$ 
%is a random variable $T$ defined on programs runs given 
is the random variable $T$ given by
%a program run  $\omega=\{\Xi_n=(\loc_n,\pv_n,w_n)\}_{n\in\Nset}$,
%\begin{align*}	
$T(\omega):=\text{min}\{n\in\Nset\mid \loc_n=\lout\}$ for every program run  $\omega=\{(\loc_n,\pv_n,w_n)\}_{n\ge 0}$
%\end{align*}
where $\text{min}\,\emptyset:=\infty$. That is, $T(\omega)$ is the number of steps a program run $\omega$ takes to reach the termination location $\lout$. A WPTS $\Pi$ is \emph{almost-surely terminating} (AST) if $\probm_{\valin}(T<\infty)=1$ for all initial program valuations $\valin\in \supp{\mu_{\mathrm{init}}}$.  
%in the case that the program run never terminates. 
\end{definition}




\begin{definition}[Expected Weights]\label{def:exp-wt}
 Given a WPTS $\Pi$ in the form of \eqref{eq:wpts}, a designated initial program valuation $\valin$ and a measurable subset $\calU\in\Sigma_{\Rset^{|\pvars|}}$, the \emph{expected weight} $\measureSem{\Pi}_{\valin}(\calU)$ 
%$\measureSem{\Pi}(\valin)$ 
%of $\Pi$ w.r.t $\pv$ 
is defined as
%$\measureSem{\Pi}_\calU(\valin):=\expectdist{\valin}{\widehat{w}_T}$. 
$\measureSem{\Pi}_{\valin}(\calU):=\expectdist{\valin}{[\widehat{\pv}_T\in \calU]\cdot\widehat{w}_T}$. 
\end{definition}

By definition, we have that $\widehat{\pv}_T$ (resp. $\widehat{w}_T$) is the random vector (resp. variable) of the program valuation (resp. the multiplicative likelihood weight) at termination, respectively. Thus, $\measureSem{\Pi}_{\valin}(\calU)$ is the expectation of $\widehat{w}_T$ 
%over all program runs 
that start from the state $(\lin,\valin,1)$ and end with $\widehat{\pv}_T\in\calU$. If $\calU=\Rset^{|\pvars|}$, the restriction of $\widehat{\pv}_T\in\calU$ can be removed.

Below we define the normalised posterior distribution (NPD) problem. %under our WPTS semantics. 

 
\begin{definition}[Normalised Posterior Distribution]\label{def:npd}
Given a WPTS $\Pi$ in the form of \eqref{eq:wpts},
%We write $\measureSem{\Pi}(\valin)$ iff $\calU=\Rset^{|\pvars|}$.)
%Then given a probability distribution $\mu$ over initial program valuations, 
the \emph{normalised posterior distribution} (NPD) $\posterior_\Pi$ of $\Pi$ 
%over $U$ 
is defined by:
\begin{align*}
\posterior_{\Pi}(\calU):=\measureSem{\Pi}(\calU)/Z_\Pi\mbox{ for all measurable subsets } \calU\in \Sigma_{\Rset^{|\pvars|}},   
\end{align*}	
where 
$\measureSem{\Pi}(\calU):=\int_{\calV} \measureSem{\Pi}_{\pv}(\calU)\cdot \mu_{\mathrm{init}}(\mathrm{d} \pv)$ is the \emph{unnormalised posterior distribution} w.r.t. $\calU$, $\calV:=\supp{\mu_{\mathrm{init}}}$, %is the support of $\mu_{\mathrm{init}}$
%is the integral of all expected weights with an initial program valuation $\pv\in U$, 
and $Z_\Pi:=\measureSem{\Pi}(\Rset^{|\pvars|})$ is the \emph{normalising constant}.  
The WPTS $\Pi$ is called \emph{integrable} 
%w.r.t a probability distribution (for initial program valuations) 
if we have $0<Z_{\Pi}<\infty$. 
%\pw{Shall we mention that $\measureSem{\Pi}_{\pv}(\calU)$ is an integrable function here?}
\end{definition}

%We call a WPTS $\Pi$ \emph{integrable} 
%w.r.t a probability distribution (for initial program valuations) 
%if the normalising constant is finite, i.e., ~$0<Z_{\Pi}<\infty$. %for any $\pv\in\val{\pvars}$. 
%Given an integrable program, we are interested in deriving lower and upper bounds on the normalised posterior distribution over some measurable set $U\in \Sigma_\Rset$.
\paragraph{Interval Bounds for NPD.} In this work, we consider the automated interval-bound analysis for NPD of a WPTS. Formally, we aim to derive an interval $[l,u]\subseteq [0,\infty)$ 
for an integrable WPTS $\Pi$ and any measurable sets $\calU\in\Sigma_{\Rset^{|\pvars|}}$ as tight as possible such that $l\le \posterior_{\Pi}(\calU) \le u$. 
%$l,u$ are called \emph{interval bounds} for the NPD $\posterior_{\Pi}(\calU)$. 
%To achieve this, in the following (\cref{sec:math}) we develop approaches to obtain interval bounds for expected weights as $\measureSem{\Pi}(\calU)$ and $Z_\Pi$ are integrations of expected weights over $\calV$. 
 



To achieve interval bounds for NPD, below we introduce the construction of a new WPTS $\Pi_\calU$ based on the original WPTS $\Pi$ and a measurable set $\calU\in \Sigma_{\Rset^{|\pvars|}}$.  

\paragraph{Construction of $\Pi_\calU$.} Consider a probabilistic program $P$ and its WPTS $\Pi$, given a measurable set $\calU\in\Sigma_{\Rset^{|\pvars|}}$, we construct a new program $P_\calU$ by adding a conditional branch of the form ``\textbf{if} $\pv_T\notin\calU$ \textbf{then} \textbf{score}($0$) \textbf{fi}'' immediately after the termination of $P$ and obtain the WPTS $\Pi_\calU$ of $P_\calU$. Therefore, $\Pi$ and $\Pi_\calU$ have the same initial probability distribution $\mu_{\mathrm{init}}$ and the same finite support $\calV=\supp{\mu_{\mathrm{init}}}$. The following proposition shows that interval-bound analysis for NPD can be reduced to interval-bound analysis for expected weights in the form $\llbracket \Pi\rrbracket_{\pv}(\Rset^{|\pvars|})$. 

\begin{proposition}\label{prop:unnorm-norm}
   Given a WPTS $\Pi$ in the form of \eqref{eq:wpts}, a measurable set $\calU\in\Sigma_{\Rset^{|\pvars|}}$ and the WPTS $\Pi_\calU$ constructed as above, we have that $\llbracket \Pi \rrbracket_{\pv}(\calU)=\llbracket \Pi_\calU\rrbracket_{\pv}(\Rset^{|\pvars|})$ for any $\pv\in\calV=\supp{\mu_{\mathrm{init}}}$. Furthermore,
   if there exist intervals $[l_1,u_1],[l_2,u_2]\subseteq [0,\infty)$ such that $\llbracket \Pi_\calU\rrbracket_{\pv}(\Rset^{|\pvars|})\in [l_1,u_1]$ and $\llbracket \Pi\rrbracket_{\pv}(\Rset^{|\pvars|})\in [l_2,u_2 ]$ for any $\pv\in\calV$, then we have two intervals $[l_\calU,u_\calU],[l_Z,u_Z]\subseteq [0,\infty)$ such that the unnormalised posterior distribution $\llbracket \Pi\rrbracket (\calU)\in [l_\calU,u_\calU]$ and the normalising constant $Z_\Pi\in [l_Z,u_Z]$. Moreover, if $\Pi$ is integrable, i.e., $[l_Z,u_Z]\subseteq (0,\infty)$, then we can obtain the NPD $\posterior_{\Pi}(\calU)\in [\frac{l_\calU}{u_Z},\frac{u_\calU}{l_Z}]$.\footnote{The interval bounds derived in this manner may be loose, but they are definitely correct.}  Note that by \cref{def:npd}, $l_\calU=\int_\calV l_1 \cdot\mu_{\mathrm{init}}(\mathrm{d} \pv)$, $u_\calU=\int_\calV u_1 \cdot\mu_{\mathrm{init}}(\mathrm{d} \pv)$, $l_Z=\int_\calV l_2 \cdot\mu_{\mathrm{init}}(\mathrm{d} \pv)$ and $u_Z=\int_\calV u_1 \cdot\mu_{\mathrm{init}}(\mathrm{d} \pv)$.

\end{proposition}

The proof of \cref{prop:unnorm-norm} is relegated to \cref{app:sec2-prop}. In the following, we will develop approaches to obtain interval bounds for expected weights.
%in the form $\llbracket \Pi \rrbracket_{\pv}(\Rset^{|\pvars|})$ where $\pv$ is an initial program valuation.












\section{Detecting Cliques}\label{sec:detection}

In this section, we first describe our algorithm for $\cliquedet{k, \ell}$, and then analyze its running time in some interesting cases. 

Throughout this section, we use $g(k, \ell)$ to denote our algorithm's running time exponent on the number of $\ell$-cliques of $\cliquedet{k, \ell}$, i.e., our algorithm for $\cliquedet{k, \ell}$ runs in $\tO(\Delta_\ell^{g(k, \ell)})$ time. 

\subsection{General Detection Framework}
Now we describe a generic algorithm for $\cliquedet{k, \ell}$ for $k \ge 3$ (for $k=2$, we trivially list all edges in the graph, so $g(2, 1) = 2$) in Algorithm~\ref{alg:generic_detection}.

\begin{algorithm}
\caption{Generic $
\cliquedet{k, \ell}$ algorithm.}\label{alg:generic_detection}
\begin{algorithmic}
\item \textbf{Input:} Graph $G = (V, E)$ and the list $L$ of all $\ell$-cliques. 
\item \textbf{Output:} Output {\sc yes} if $G$ contains a $k$-cliques, and {\sc no} otherwise.
\item \textbf{The Algorithm:} 
\begin{itemize}
    \item Let integers $k \ge a \geq b \geq c \ge 1$ be such that $k = a + b + c$ (the algorithm chooses $a, b, c$ optimally). Then goal is then to bound the number of $d$-cliques for $d \in \{a, b, c\}$. 
    \begin{itemize}
        \item If $d \ge \ell$, we can use Lemma~\ref{lem:simple_list_ub} to upper bound the number of $d$-cliques with $S_d = \tilde{\Theta}(\Delta_\ell^{d/\ell})$, and add these $d$-cliques to a list $L_d$ in the same time. 
        \item If $d < \ell$, for every $d$-clique $K$ with $\Delta_
        \ell (K) \leq \Delta_\ell^{x_d}$ (for some parameter $x_d \in [0, 1]$ to be chosen), we check if $K$ is in a $k$-clique by recursively running $\cliquedet{k-d, \ell-d}$ in its neighbourhood. Then, let $L_d$ denote the set of remaining $d$-cliques. Then, $S_{d} := |L_d|  = \Theta(\Delta_\ell^{1-x_d})$. The running time of this step is 
        \begin{align*}
            \tO\left(\sum_{\substack{K: d\text{-clique}\\\Delta_\ell(K) \le \Delta_\ell^{x_d}}} \Delta_\ell(K)^{g(k-d, \ell - d)}\right) & \le \tO\left(\sum_{\substack{K: d\text{-clique}\\\Delta_\ell(K) \le \Delta_\ell^{x_d}}} \Delta_\ell(K) \cdot \Delta_\ell^{x_d (g(k-d, \ell - d) - 1)}\right)\\
            & \le \tO\left(\Delta_\ell^{1+x_d (g(k-d, \ell - d) - 1)}\right). 
        \end{align*}
    \end{itemize}
    \item Finally, we conduct a usual matrix multiplication of dimensions $S_a, S_b, S_c$ in time $\MM(S_a, S_b, S_c)$ as follows. If we find a $k$-clique, output {\sc yes}, otherwise we output {\sc no.}
    \begin{itemize}
        \item Create a matrix $X$ whose rows are indexed by $a$-cliques in $L_a$ and columns are indexed by  $b$-cliques in $L_b$. Set $A[K_a, K_b] = 1$ if  the nodes of $K_a$ and $K_b$ form an $(a+ b)$-clique, and $0$ otherwise.
        \item Create a matrix $Y$ whose rows are indexed by $b$-cliques in $L_b$ and columns are indexed by $c$-cliques in $L_c$, and set the entries similarly.
        \item Compute $Z = XY$. For each pair of  remaining $a$-clique $K_a$ and $c$-clique $K_c$ that form an $(a + c)$-clique, check if $Z[K_a, K_c] > 0$. If such an entry exists, output {\sc yes}. Otherwise, output {\sc no}.
    \end{itemize}
    
\end{itemize}
\end{algorithmic}
\end{algorithm}

The correctness of this algorithm is immediate. 
We also remark that the algorithm can  be used to count the number of $k$-cliques, by replacing all the recursive calls with the counting version of the algorithm, using the matrix multiplication to count the number of $k$-cliques in the remaining graph, and properly summing up and scaling the numbers. Clearly, the counting version of the algorithm will have the same running time. 

\subsection{Examples}
\label{sec:detection_examples}
Let us give some explicit examples to illustrate the algorithm. 

\paragraph{$\cliquedet{k,1}$.} The simplest example is $\cliquedet{k,1}$ for $k \ge 3$. Let $\lfloor k/3 \rfloor \leq c \leq b \leq a \leq \lceil k/3\rceil$ be integers such that $a+b+c = k$, which is one of the possible choices of $a, b, c$ for the algorithm.
Note that $c = \lfloor k/3 \rfloor, b =  \lceil (k-1)/3\rceil, a = \lceil k/3 \rceil$. Since $a, b, c \ge \ell = 1$, the algorithm would choose to use Lemma~\ref{lem:simple_list_ub} to bound the number of cliques of sizes $a, b, c$ as $n^a, n^b, n^c$ respectively. Thus, the running time of the algorithm is $\tO(n^{\omega(a, b, c)}) = \tO(n^{\beta(k)})$, matching the previous running time \cite{eisenbrand2004complexity}.

\paragraph{$\cliquedet{k,\ell}$ for $\ell \leq \lfloor k / 3\rfloor.$}
Similar as above, let $c = \lfloor k/3 \rfloor, b =  \lceil (k-1)/3\rceil, a = \lceil k/3 \rceil$ and the algorithm would choose to use Lemma~\ref{lem:simple_list_ub} to bound the number of cliques of sizes $a, b, c$. Thus, the running time of the algorithm is  $\tO(\Delta_\ell^{\omega(a/\ell, b/\ell, c/\ell)}) \leq \tO(\Delta_\ell^{\omega(\lceil k/3 \rceil,   \lceil (k-1)/3\rceil, \lfloor k/3 \rfloor)/\ell})$. 
This running time is optimal barring improvements for $\cliquedet{k, 1}$:


\begin{table}[ht]
    \centering
    \begin{tabular}{c|c|c|c|c|c|c|c|c|c|c}
         \backslashbox{$\ell$}{$k$} &  3 & 4 & 5 & 6 & 7 & 8 & 9 & 10 & 11 & 12\\
         \hline 
         1 & 2.373 & 3.252 & 4.089 & 4.746 & 5.594 & 6.401 & 7.119 & 7.952 & 8.751 & 9.492\\
         2 & 1.408 & 1.657 & 2.058 & 2.373 & 2.797 & 3.201 & 3.559 & 3.976 & 4.376 & 4.746\\
         3 & - & 1.248 & 1.422 & 1.669 & 1.918 & 2.151 & 2.373 & 2.651 & 2.917 & 3.164\\
         4 & - & - & 1.175 & 1.298 & 1.487 & 1.657 & 1.841 & 2.028 & 2.207 & 2.373\\
         5 & - & - & - & 1.130 & 1.233 & 1.378 & 1.504 & 1.660 & 1.811 & 1.954
    \end{tabular}
    \caption{Our $\cliquedet{k, \ell}$ exponent for various values of $k, \ell$ with the best current bound on $\omega$ \cite{alman2021refined} and rectangular matrix multiplication \cite{LU18}. See also \cite{van2019dynamic} for a way to bound $\omega(a, b, c)$ for arbitrary $a, b, c > 0$ from values of $\omega(1, x, 1)$. 
    The $(k, \ell)$th entry corresponds to the exponent $\alpha$ such that the runtime to detect a $k$-clique is $\tilde{O}(\Delta_\ell^\alpha)$ , where $\Delta_\ell$ is the number of $\ell$-cliques. 
    }
    \label{tab:det_exponent}
\end{table}

\begin{proposition}
Fix any positive integers $k \ge 3$ and $\ell \le \lfloor k/3\rfloor$, and let $\beta(k) = \omega(\lceil k/3 \rceil,   \lceil (k-1)/3\rceil, \lfloor k/3 \rfloor)$.
If $\cliquedet{k, 1}$ requires $n^{\beta(k) - o(1)}$ time, then
$\cliquedet{k, \ell}$ requires $\Delta_\ell^{\beta(k)/\ell-o(1)}$ time. 
\end{proposition}
\begin{proof}
Suppose for the sake of contradiction that $\cliquedet{k, \ell}$ has an  $O(\Delta_\ell^{\beta(k)/\ell-\eps}$) time algorithm $\mathcal{A}$ for some $\eps > 0$. Then given a $\cliquedet{k, 1}$ instance, we can first use Lemma~\ref{lem:simple_list_ub} to list all $\ell$-cliques in $O(n^\ell)$ time, and the number of $\ell$-cliques is bounded by $O(n^\ell)$. Then we can use $\mathcal{A}$ to solve the $\cliquedet{k, 1}$ instance in $O((n^\ell)^{\beta(k)/\ell-\eps})=n^{\beta(k)-\eps \ell}$ time, a contradiction.
\end{proof}



\begin{example}[$\cliquedet{3, 2}$]
\label{ex:clique-det-3-2}
\em
In this case, the algorithm can only choose $a=b=c=1$, and it would naturally choose $x_a=x_b=x_c$. The time it takes to bound the number of $1$-cliques (nodes) is $\tO(\Delta_2^{1+x_a (g(2, 1) - 1)}) = \tO(m^{1+x_a})$. Then we have $S_a, S_b, S_c \le \Theta(m^{1-x_a})$. Thus, the running time for the matrix multiplication of dimensions $S_a, S_b, S_c$ is $\tO(m^{(1-x_a)\omega})$. Overall, the running time is $\tO(m^{\frac{2\omega}{\omega+1}})$ by setting $x_a = \frac{\omega-1}{\omega+1}$. This is essentially Alon, Yuster and Zwick \cite{alon1997finding}'s triangle detection algorithm for sparse graphs. 
\end{example}

\begin{example}[$\cliquedet{4, 2}$]
\label{ex:clique-det-4-2}
\em
In this case, the algorithm can only choose $a=2, b=c=1$, and it would naturally choose $x_b=x_c$. The algorithm uses Lemma~\ref{lem:simple_list_ub} to (trivially) bound the number of edges as $m$. 
The time it takes to bound the number of nodes is $\tO(\Delta_2^{1+x_b (g(3, 1) - 1)}) = \tO(m^{1+x_b(\omega-1)})$. Then we have $S_a \le \Theta(m), S_b, S_c \le \Theta(m^{1-x_b})$. Thus, the running time for the matrix multiplication of dimensions $S_a, S_b, S_c$ is $\tO(m^{\omega(1, 1-x_b, 1-x_b)})$. The algorithm chooses $x_b$ so that $1+x_b(\omega-1) = \omega(1, 1-x_b, 1-x_b)$. If we simply bound $\omega(1, 1-x_b, 1-x_b)$ by $x_b + \omega(1-x_b)$, we can get $g(4, 2) \le \frac{\omega+1}{2}$ by setting $x_b = \frac{1}{2}$. For the current best bound of rectangular matrix multiplication \cite{LU18}, we can set $x_b = 0.478$ to get an upper bound $g(4, 2) \le 1.657$. As seen in Table~\ref{table:improved_det_4_5}, this is an improvement over the previous best algorithm of Eisenbrand and Grandoni \cite{eisenbrand2004complexity}. The key difference between our algorithm and \cite{eisenbrand2004complexity}'s algorithm is that, after they perform a similar first stage, they recursively call a $\cliquedet{4, 1}$ algorithm on graphs with $S_b$ nodes, losing the information that the graph has $S_a = m$ edges to begin with. We instead utilize this information with rectangular matrix multiplication to get a better running time. 

\end{example}

\begin{example}[$\cliquedet{5, 2}$]
\label{ex:clique-det-5-2}
\em
In this case, let the algorithm  choose $a=b=2, c=1$ (the choice $a=3, b=c=1$ gives a worse bound). The algorithm uses Lemma~\ref{lem:simple_list_ub} to (trivially) bound the number of edges as $m$. 
The time it takes to bound the number of nodes is $\tO(\Delta_2^{1+x_c (g(4, 1) - 1)}) = \tO(m^{1+x_c(\omega(1, 2, 1)-1)})$. Then we have $S_a,S_b \le \Theta(m), S_c \le \Theta(m^{1-x_c})$. Thus, the running time for the matrix multiplication of dimensions $S_a, S_b, S_c$ is $\tO(m^{\omega(1, 1, 1-x_c)})$. The algorithm chooses $x_c$ so that $1+x_c(\omega(1, 2, 1)-1) = \omega(1, 1, 1-x_c)$. If we simply bound $\omega(1, 2, 1)$ by $\omega + 1$ and $\omega(1, 1, 1-x_c)$ by $2x_c + (1-x_c)\omega$, we can get $g(5, 2) \le \frac{\omega+2}{2}$ by setting $x_c = \frac{1}{2}$. For the current best bound of rectangular matrix multiplication~\cite{LU18}, we can set $x_c = 0.4698$ to get an upper bound $g(5, 2) \le 2.058$. As seen in Table~\ref{table:improved_det_4_5}, this is an improvement over the previous best known algorithm of Eisenbrand and Grandoni \cite{eisenbrand2004complexity}.
\end{example}


\begin{example}[More Small Examples]
\label{ex:more-small-examples}
\em
See Tables~\ref{tab:detection_runtime} and \ref{tab:det_exponent} for more examples of the running times of our algorithm. These running times were obtained by finding the optimal values of $a, b, c$ using dynamic programming. 

From previous examples, one might wonder whether the algorithm always sets $a, b, c$ as close to $k/3$ as possible. The following example shows that it is not the case (for $\omega = 2$). 

In $\cliquedet{8, 4}$, if the algorithm chooses $a=4, b = c = 2$, then the running time is 
$$\tO\left(\Delta_4^{1+x_b(g(6, 2)-1)}+\Delta_4^{1+x_c(g(6, 2)-1)} + \Delta_4^{\omega(1, 1-x_b, 1-x_c)}\right).$$
By setting $x_b=x_c = \frac{1}{2}$, this running time is bounded by $\tO(\Delta_4^{3/2})$ when $\omega = 2$ (See Table~\ref{tab:det_exponent} for the value of $g(6, 2)$ when $\omega = 2$). 

However, if the algorithm chooses a more balanced choice $a=b=3, c = 2$, then the running time is $$\tO\left(\Delta_4^{1+x_a(g(5, 1)-1)}+\Delta_4^{1+x_b(g(5, 1)-1)}+\Delta_4^{1+x_c(g(6, 2)-1)} + \Delta_4^{\omega(1-x_a, 1-x_b, 1-x_c)}\right).$$
One optimal way to set the parameters when $\omega = 2$ is $x_a = x_b = \frac{1}{5}$ and $x_c = \frac{3}{5}$, which only gives an $\tO(\Delta_4^{8/5})$ running time when $\omega = 2$ (See Table~\ref{tab:det_exponent} for the values of $g(5, 1)$ and $g(6, 2)$ when $\omega = 2$). 
\end{example}

\subsection{Upper Bound for \texorpdfstring{$\cliquedet{k, k - h}$}{(k, k-h)-Clique-Detection}}
\label{sec:k-h_detect_bound}

In this section, we analyze the running time of our algorithm for $\cliquedet{k, k - h}$ for some constant $h=O(1)$. For convenience, let $e_h(k) = g(k, k - h)$.

We start with the following lemma.
\begin{lemma}
\label{lem:det_exponent_monotone}
For every $k > h$, $e_h(k + 1) \le e_h(k)$. 
\end{lemma}
\begin{proof}
We prove the statement by induction. We skip the base case $k = h+1$ as it works similarly as the induction step
(except for $h=1$, in which case $e_h(2) = 2$ and $e_h(3) = \frac{2\omega}{\omega+1} \le e_h(2)$, as the algorithm handles $\cliquedet{2, 1}$ specially).  Suppose the statement is already true for all smaller $k$. 

Let $\ell = k - h$ and $\ell' = k + 1 - h$. Suppose for $\cliquedet{k, \ell}$, the optimal parameters are $a, b, c, x_a, x_b, x_c, S_a, S_b, S_c$ ($x_d$ is relevant only if $d < \ell$ for $d \in \{a, b, c\}$).  Consider $\cliquedet{k+1, \ell+1}$ with parameters $a' = a+1, b' =b, c'=c$ and $x'_{a'}, x'_{b'}, x'_{c'}, S'_{a'}, S'_{b'}, S'_{c'}$ to be determined. Let $\Delta_\ell$ be the number of $\ell$-cliques in the $\cliquedet{k, \ell}$ instance and let $\Delta'_{\ell'}$ be the number of $(\ell+1)$-cliques in the $\cliquedet{k+1, \ell+1}$ instance. 

We first compare exponents related to $S_a$ and $S'_{a'}$.
\begin{itemize}
    \item If $a \ge \ell$. Then $S_a$ in $\cliquedet{k, \ell}$ is bounded by $\tO(\Delta_\ell^{a / \ell})$. In the $\cliquedet{k + 1, \ell + 1}$ algorithm, $S_{a'}'$ is bounded  by $\tO((\Delta'_{\ell'})^{(a+1)/(\ell+1)})$, a smaller exponent. 
    \item If $a < \ell$, the exponent of the running time for bounding $S_a$ in $\cliquedet{k, \ell}$ is $1+x_a (e_h(k-a) - 1)$, and $S_a$ is bounded by $\Delta_\ell^{1-x_a}$. Let $x'_{a'}$ be equal to $x_a$ in the algorithm for $\cliquedet{k + 1, \ell + 1}$. Then notice that the exponent for running time is $1+x'_{a'}(e_h(k + 1 - a') - 1) = 1+x_a(e_h(k-a)-1)$ and the bound on $S'_{a'}$ is $(\Delta'_{\ell'})^{1-x_a}$, both with same exponents as previous bounds. 
\end{itemize}
We then compare exponents related to $S_b$ and $S'_{b'}$. 
\begin{itemize}
    \item If $b > \ell$. Then $b' = b \ge \ell+1 = \ell'$. Then $S_b$ in $\cliquedet{k, \ell}$ is bounded by $\tO(\Delta_\ell^{b / \ell})$. In the $\cliquedet{k + 1, \ell + 1}$ algorithm, $S_{b'}'$ is bounded  by $\tO((\Delta'_{\ell'})^{b/(\ell+1)})$, a smaller exponent. 
    \item If $b = \ell$. In this case, $S_b = \tO(\Delta_\ell)$ and we will have $b' < \ell'$. Let $x'_{b'} = 0$ in $\cliquedet{k + 1, \ell + 1}$. Then $S'_{b'}$ is bounded by $\tO((\Delta'_{\ell'})^1)$, the same exponent as the bound of $S_b$. Also, the cost for having this bound is $\tO((\Delta'_{\ell'})^{1+x'_{b'}(e_h(k+1-b'))}) = \tO(\Delta'_{\ell'})$, so we can ignore the cost as it is near-linear time.
    \item If $b < \ell$, the exponent of the running time for bounding $S_b$ in $\cliquedet{k, \ell}$ is $1+x_b (e_h(k-b) - 1)$, and $S_b$ is bounded by $\Delta_\ell^{1-x_b}$. Let $x'_{b'}$ be equal to $x_b$ in the algorithm for $\cliquedet{k + 1, \ell + 1}$. Then notice that the exponent for running time is $1+x'_{b'}(e_h(k + 1 - b') - 1) = 1+x_b(e_h(k-b+1)-1)$. By the induction assumption, 
    $e_h(k-b+1) \le e_h(k-b)$, so $1+x_b(e_h(k-b+1)-1)$ is upper bounded by the running time exponent of the corresponding case in $\cliquedet{k, \ell}$. Note that this case does not happen in the base case $k=h+1$, as $b < \ell = 1$ can never happen, so we can safely apply the induction assumption. 
    The bound on $S'_{b'}$ is $(\Delta'_{\ell'})^{1-x_b}$,  with the same exponent as $S_b$ in $\cliquedet{k, \ell}$. 
\end{itemize}
The comparison of the exponents related to $S_c$ and $S'_{c'}$ works similarly. Thus, $e_h(k+1) \le e_h(k)$. 
\end{proof}

\begin{proposition}
\label{prop:eh_upper_bound}
$e_h(k) = 1+O\left(1/ k^{\log_{\frac{3}{2}}(\frac{\omega}{\omega-1})}\right)$. 
\end{proposition}
\begin{proof}
Let $\ell = k - h$. 
 Let $k_0 = 100h$. For all $k \le k_0$, $e_h(k) = O(1)$. 

For $k > k_0$, we choose $a, b, c$ in our $\cliquedet{k, k - h}$ algorithm so that $\lfloor k/3\rfloor = c \le b \le a = \lceil k/3\rceil$. Clearly, $a, b, c < \ell = k - h$. The running time of the algorithm is thus 
$$\tO\left(\Delta_\ell^{1+x_a  \cdot (e_h(k-a)-1)} 
+ \Delta_\ell^{1+x_b  \cdot (e_h(k-b)-1)} 
+ \Delta_\ell^{1+x_c  \cdot (e_h(k-c)-1)} 
+ MM\left(\Delta_\ell^{1-x_a}, \Delta_\ell^{1-x_b},\Delta_\ell^{1-x_c} \right)\right).$$
By Lemma~\ref{lem:det_exponent_monotone}, $e_h(k-c) \le e_h(k-b) \le e_h(k-a)$, so the running time is bounded by 
$$\tO\left(\Delta_\ell^{1+\max\{x_a, x_b, x_c\}  \cdot (e_h(k-a)-1)} 
+MM\left(\Delta_\ell^{1-x_a}, \Delta_\ell^{1-x_b},\Delta_\ell^{1-x_c} \right)\right).$$
Set $x_a = x_b = x_c = \frac{\omega -1}{\omega + e_h(k-a) - 1}$. The running time then becomes 
$$\tO\left(\Delta_\ell^{\frac{\omega \cdot e_h(k-a)}{\omega + e_h(k-a) - 1}}\right).$$
Thus,
$e_h(k) \le \frac{\omega \cdot e_h(k-a)}{\omega + e_h(k-a) - 1}$. Consequently, $$e_h(k) - 1 \le \frac{(\omega - 1) \cdot (e_h(k-a) - 1)}{\omega + e_h(k-a) - 1} \le \frac{\omega - 1}{\omega}  \cdot (e_h(k-a) - 1) = \frac{\omega - 1}{\omega}  \cdot (e_h(k-\lceil k/3\rceil) - 1).$$
Therefore $e_h(k) - 1 \le O\left(\left(\frac{\omega - 1}{\omega}\right)^{\log_{\frac{3}{2}} k}\right) = O\left(1/ k^{\log_{\frac{3}{2}}(\frac{\omega}{\omega-1})}\right)$.
\end{proof}

We also show that our choices of $a, b, c$ are not too far away from optimal, at least when $\omega = 2$. In the following proposition, recall $e_h(k)$ is the exponent of our algorithm, instead of the best exponent for $\cliquedet{k, k - h}$. 

\begin{proposition}
$e_h(k) = 1+\Omega\left(1/ k^{\log_{\frac{3}{2}}(2)}\right)$. 
\end{proposition}
\begin{proof}
Let $\ell = k - h$,  $\rho = \log_{\frac{3}{2}}(2)$, and $f_h(k) = \frac{1}{e_h(k) - 1}$. 
 Let $k_0 = 100h$. It is not difficult to see that for all $k \le k_0$, $f_h(k) \le M k^\rho - 1$ for some sufficiently large constant $M > 1$ because our algorithm does not achieve almost linear time, i.e., it always has $e_h(k) > 1$ and thus $f_h(k) < \infty$. 
 
 Let $k > k_0$, and let $a, b, c$ be the optimal choices for $\cliquedet{k, k - h}$. We will show by induction that $f_h(k) \le M k^\rho - 1$. 
 Consider two cases. 
 
 For the first case, assume $a < \ell$. Let $x_a, x_b, x_c$ be the optimal parameters for $\cliquedet{k, k - h}$, and if there are multiple choices, we choose one set of parameters with smallest $x_a+x_b+x_c$. 
 Then, the bound of our running time is (up to $\tO(1)$ factors) 
$$\Delta_\ell^{1+x_a  \cdot (e_h(k-a)-1)} 
+ \Delta_\ell^{1+x_b  \cdot (e_h(k-b)-1)} 
+ \Delta_\ell^{1+x_c  \cdot (e_h(k-c)-1)} 
+ MM\left(\Delta_\ell^{1-x_a}, \Delta_\ell^{1-x_b},\Delta_\ell^{1-x_c}\right).$$ 

Suppose $x_a > x_b$. By Lemma~\ref{lem:det_exponent_monotone}, $e_h(k-a) \ge e_h(k-b)$. Therefore, we can slightly increase $x_b$, and the running time of the algorithm will not be worse. This contradicts with the optimality of $x_a, x_b, x_c$ and minimality of $x_a+x_b+x_c$. Thus, we must have $x_a \le x_b$. Similarly, we have $x_b \le x_c$. 

Then we can lower bound $MM\left(\Delta_\ell^{1-x_a}, \Delta_\ell^{1-x_b},\Delta_\ell^{1-x_c}\right)$ by   $\Delta_\ell^{2-x_a-x_b}$.

The optimal way to balance $\Delta_\ell^{1+x_a  \cdot (e_h(k-a)-1)}, \Delta_\ell^{1+x_b  \cdot (e_h(k-b)-1)}, \Delta_\ell^{1+x_c  \cdot (e_h(k-c)-1)}$ and $ \Delta_\ell^{2-x_a-x_b}$ is to set 
$x_a = \frac{e_h(k-b)-1}{e_h(k-a)e_h(k-b)-1}$, $x_b = \frac{e_h(k-a)-1}{e_h(k-a)e_h(k-b)-1}$ and $x_c = \min\{1, \frac{(e_h(k-a)-1)(e_h(k-b)-1)}{(e_h(k-a)e_h(k-b)-1)(e_h(k-c)-1)}\}$, which gives \[e_h(k) \ge \frac{2e_h(k-a)e_h(k-b)-e_h(k-a)-e_h(k-b)}{e_h(k-a)e_h(k-b)-1}.\] Substituting $e_h$ by $f_h$ gives the following cleaner formula:
$$f_h(k) \le 1+f_h(k-a)+f_h(k-b).$$
As the algorithm chooses the optimal $a, b, c$, we have that 
$$f_h(k) \le \max_{\substack{1 \leq c \leq b \leq a \leq k\\ a + b + c = k}} \left\{ 1 + f_h(k-a) + f_h(k-b)\right\}.$$

By Lemma~\ref{lem:det_exponent_monotone}, $f_h(k-b)$ is nondecreasing when $b$ increases, so we can pick $b$ to be as large as possible for fixed $a$. Therefore, for fixed $a$, we choose  $c = \lfloor \frac{k-a}{2} \rfloor$ and $b = \lceil \frac{k-a}{2} \rceil$. Therefore, we can rewrite
$$f_h(k) \le \max_{ k/3  \leq a \leq k-2} \left\{1 + f_h(k-a) + f_h\left(\left\lfloor \frac{k+a}{2}\right\rfloor\right)\right\}.$$

By the induction assumption, $f_h(k') \le M (k')^\rho - 1$ for all $k'<k$.

Then,
\begin{align*}
    f_h(k)&\leq \max_{k/3 \le a \le k-2} \left\{1 + f_h(k-a) + f_h\left(\left\lfloor\frac{k+a}{2}\right\rfloor\right)\right\} \\
    & \leq \max_{0 \le p \le k/3}\left\{ 1 + M \left(\frac{2k}{3} - 2p\right)^\rho + M\left(\frac{2k}{3} + p\right)^\rho - 2 \right\}\\
    & \leq Mk^\rho \cdot \max_{0 \le p' \le 1/3} \left\{\left(\frac{2}{3}-2p'\right)^\rho + \left(\frac{2}{3}+p'\right)^\rho \right\} - 1\\
    &\le Mk^\rho - 1,
\end{align*}
which completes the induction step for this case. 

For the other case, assume $a \ge \ell$. Note that we must have $b, c < \ell$ as $2\ell > k$. Let $x_b, x_c$ be the optimal parameters. Similar as before, we can assume $x_b \le x_c$. 
 Then, the bound of our running time is (up to $\tO(1)$ factors) 
\begin{align*}
&\Delta_\ell^{1+x_b  \cdot (e_h(k-b)-1)} 
+ \Delta_\ell^{1+x_c  \cdot (e_h(k-c)-1)} 
+ MM\left(\Delta_\ell^{a/\ell}, \Delta_\ell^{1-x_b},\Delta_\ell^{1-x_c}\right)\\
\ge & \Delta_\ell^{1+x_b  \cdot (e_h(k-b)-1)} 
+ \Delta_\ell^{1+x_c  \cdot (e_h(k-c)-1)} 
+ \Delta_\ell^{a/\ell + 1 - x_b}
\end{align*}
The optimal way to balance is to set $x_b = \frac{a}{\ell e_h(k-b)}$ and $x_c = \min\{1, \frac{a(e_h(k-b)-1)}{\ell e_h(k-b)(e_h(k-c)-1)}\}$. 
This gives $e_h(k) \ge \frac{ae_h(k-b)-a}{\ell e_h(k-b)}+1$. Note that it is possible that $x_b > 1$ in this setting, but if that happens, $e_h(k) > e_h(k-b)$, which by Lemma~\ref{lem:det_exponent_monotone}, can never be optimal. In terms of $f_h$, this implies that $f_h(k) \le \frac{\ell  (f_h(k-b)+1)}{a}$. 
As the algorithm chooses the optimal $a, b, c$, we have that 
$$f_h(k) \le \max_{\substack{1 \leq c \leq b < \ell \leq a \leq k\\ a + b + c = k}}  \frac{\ell  (f_h(k-b)+1)}{a}.$$
By Lemma~\ref{lem:det_exponent_monotone}, $f_h(k-b)$ is nondecreasing when $b$ increases, so we can pick $b$ to be as large as possible for fixed $a$. Therefore, for fixed $a$, we choose  $c = \lfloor \frac{k-a}{2} \rfloor$ and $b = \lceil \frac{k-a}{2} \rceil$. Thus, we can rewrite
$$f_h(k) \le \max_{ \ell \leq a \leq k-2} \frac{\ell  (f_h\left(\left\lfloor \frac{k+a}{2}\right\rfloor\right)+1)}{a} \le \max_{ \ell \leq a \leq k-2}  \left\{f_h\left(\left\lfloor \frac{k+a}{2}\right\rfloor\right)+1\right\}.$$
By induction, it can be further upper bounded by 
$$\max_{ \ell \leq a \leq k-2}  \left\{M\left(\left\lfloor \frac{k+a}{2}\right\rfloor\right)^\rho -1 +1\right\} \le M(k-1)^\rho < Mk^\rho - 1,$$
as $M, \rho > 1$. This finishes the induction step for this case. 

Overall, we have shown that $f_h(k) \le M k^\rho - 1$ for all $k$, which implies $e_h(k) = 1+\Omega\left(1/ k^{\log_{\frac{3}{2}}(2)}\right)$. 
\end{proof}

\subsection{Upper Bound for \texorpdfstring{$\cliquedet{C\ell, \ell}$}{(Cl, l)-Clique-Detection}}\label{sec:Cl_l_detectionbound}
Define a sequence of functions $(f_i)_{i \ge 0}$ as follows:
$$f_i(C) = \frac{2^i \omega^{i+1} C}{3^{i+1}(\omega-1)^i+\left(3 (2^i - 3^i) (\omega-1)^i - 2^i (\omega-1)^i \omega + 2^i \omega^{i + 1}\right)C}. $$
The functions have the following recurrence relation, whose proof we omit as it is straightforward algebra. 
\begin{claim}
$f_0(C) = \frac{\omega C}{3}$ and $f_i(C) = \frac{\omega}{1+\frac{\omega - 1}{f_{i-1}\left(\frac{2C}{3-C}\right)}}$ for $i > 0$.
\end{claim}


Then we can express the running time of $\cliquedet{C\ell, \ell}$ for sufficiently large $\ell$ in terms of the functions $f_i$:
\begin{theorem}\label{thm:det_mult_upper_bound}
Let $C > 1$ be any constant such that $\frac{1}{C} \in \left(1-\left(\frac{2}{3}\right)^i, 1-\left(\frac{2}{3}\right)^{i+1} \right]$ for some constant integer $i \ge 0$. Then for any $\ell \ge 1$ and $C\ell \le  k \le (C+o_\ell(1)) \ell$, $g(k, \ell) \le f_i(C) + o_\ell(1)$. 
\end{theorem}
\begin{proof}
We prove by induction on $i$. 

When $i = 0$, $k \ge C\ell \ge 3\ell$. Therefore, we can apply the  $\cliquedet{k,\ell}$ example in Section~\ref{sec:detection_examples} for $\ell \leq \lfloor k / 3\rfloor$  to get $g(k, \ell) \le \omega(\lceil k/3 \rceil,   \lceil (k-1)/3\rceil, \lfloor k/3 \rfloor)/\ell$. This leads to 
\begin{align*}
    g(k, \ell) &\le \omega(k/3+1, k/3+1,k/3+1) / \ell \\
    & = \frac{(k/3+1)\omega}{\ell}\\
    & \le \frac{((C+o_\ell(1)) \ell / 3 + 1) \omega}{\ell}\\
    & \le \frac{\omega C}{3} + o_\ell(1) = f_0(C) + o_\ell(1).
\end{align*}

When $i > 0$, assume the claim is correct for $i-1$. Similar to the proof of Proposition~\ref{prop:eh_upper_bound}, we choose $a, b, c$ in our $\cliquedet{k, \ell}$ algorithm so that $\lfloor k/3\rfloor = c \le b \le a = \lceil k/3\rceil$. By the same analysis, the running time exponent can then be bounded by 
$\frac{\omega \cdot g(k-a, \ell - a)}{\omega + g(k-a, \ell - a) - 1}$. Let $C' = \frac{2C}{3-C}$. It is not difficult to verify that $\frac{1}{C'} \in \left(1-\left(\frac{2}{3}\right)^{i-1}, 1-\left(\frac{2}{3}\right)^{i} \right]$. 

Also, 
\begin{align*}
    \frac{k-a}{\ell - a} &\ge \frac{k - k/3}{\ell - k/3} \ge \frac{C \ell - (C\ell) / 3}{\ell - (C \ell) / 3} = \frac{2C}{3 - C} = C',
\end{align*}
and 
\begin{align*}
    \frac{k-a}{\ell - a} &\le \frac{k - (k/3 + 1)}{\ell - (k/3 + 1)} \le \frac{(C + o_\ell(1))\ell - ((C + o_\ell(1))\ell) / 3}{\ell - ((C + o_\ell(1)) \ell) / 3} = \frac{2C + o_\ell(1)}{3 - C - o_\ell(1)} = C' + o_\ell(1).
\end{align*}
Thus, $C'(\ell - a) \le k-a \le (C'+o_\ell(1))(\ell-a)$, so $g(k-a, \ell - a) \le f_{i-1}(C') + o_\ell(1)$ by induction. Therefore, the running time exponent of $\cliquedet{k, \ell}$ can be bounded by 
\begin{align*}
    \frac{\omega \cdot g(k-a, \ell - a)}{\omega + g(k-a, \ell - a) - 1} &= \frac{\omega}{1 + \frac{\omega - 1}{g(k-a, \ell - a)}}
    \le \frac{\omega}{1 + \frac{\omega - 1}{f_{i-1}(C') + o_\ell(1)}}
     \le \frac{\omega}{1 + \frac{\omega - 1}{f_{i-1}(\frac{2C}{3-C})}} +  o_\ell(1) = f_i(C) + o_\ell(1).
\end{align*}
\end{proof}

% Figure environment removed

In Figure~\ref{fig:detection_upper_bound}, we compare the bound obtained from Theorem~\ref{thm:det_mult_upper_bound} with the actual running time of Algorithm~\ref{alg:generic_detection} computed by dynamic programming for $3 \leq k \leq 200$.
In particular, for various values of $C$, we plot the exponent of $\cliquedet{k, \lfloor k/C\rfloor}$ against the upper bound obtained from Theorem~\ref{thm:det_mult_upper_bound} (without the $o_\ell(1)$ factor). Figure~\ref{fig:detection_upper_bound} shows that the estimates given by Theorem~\ref{thm:det_mult_upper_bound} are actually quite close to the actual exponents, and the values  indeed converge to our bound. 



\section{Lower Bounds for Listing Cliques}\label{sec:lower_bounds}

In this section, we will show our conditional lower bound for $\cliquelist{k,\ell}$ under the Exact-$k$-Clique hypothesis.

\begin{theorem}[Theorem~\ref{thm:lb_intro}]\label{thm:lower_bound}
For any $k \ge 3, 1 \le \ell < k$, and $\gamma \in [0, k/\ell]$, $\cliquelist{k, \ell}$ for instances with $t = \tilde{\Theta}(\Delta_\ell^\gamma)$
requires $$\left(\Delta_\ell^{\frac{2}{\ell(k-\ell)}} t^{1 - \frac{2}{k(k-\ell)}}\right)^{1-o(1)}$$
    time, where $\Delta_\ell$ is the number of $\ell$-cliques and $t$ is the number of $k$-cliques required to list, assuming Hypothesis~\ref{hyp:exact_k_clique}.
\end{theorem}
    
\begin{proof}
First, we can assume 
$\frac{2}{\ell(k-\ell)} + \gamma(1 - \frac{2}{k(k-\ell)}) > 1$, as otherwise the lower bound is trivial. 

    Let $G=(V = V_1 \sqcup \cdots \sqcup V_k, E, w)$ be a $k$-partite Exact-$k$-Clique instance on $k \cdot n$ nodes. Without loss of generality, we assume the edge weights $w$ of $G$ are from $\FF_p$ for some sufficiently large prime $p = n^{O(k)}=n^{O(1)}$. Then we sample $x \sim \FF_p$ uniformly at random. For every $i \in [k]$, and every node $v \in V_i$, we sample $k-1$ random variables $(y_{v, j})_{j \in [k] \setminus\{i\}} \sim \FF_p$ where $\sum_{j \in [k] \setminus\{i\}} y_{v, j} = 0$ uniformly at random. Note that $(y_{v, j})_{j \in [k] \setminus\{i\}}$ are $(k-2)$-wise independent. 
    
    For every $1 \le i < j \le k$, and $(v_i, v_j) \in V_i \times V_j$, let $$w'(v_i, v_j) = x \cdot w(v_i, v_j) + y_{v_i, j} + y_{v_j, i}.$$ It is not difficult to verify that, whenever $x \ne 0$, the sets of exact-$k$-cliques in the graph with weight $w$ and with weight $w'$ are the same. 
    
    Then we partition $\FF_p$ into $s$ contiguous intervals, each of length $O(p/s)$, for some $s$ to be chosen later where $\Omega(1) \le s \le O(n^{\frac{2}{k-1}})$. Consider all combinations of intervals $(L_{i, j})_{1 \le i < j \le k}$, where $0 \in \sum_{1 \le i < j \le k} L_{i, j}$. If we fix an arbitrary choice of the first $\binom{k}{2} - 1$ intervals, their sumset is an interval of length $O(p/s)$. Thus, there is only $O(1)$ choices for the last interval in order for their sumset to contain $0$. Hence, there are only $O(s^{\binom{k}{2} - 1})$ such combinations. For each such combination, we construct an instance of $\cliquelist{k, \ell}$  as follows: create an unweighted graph $H$ such that an edge $(v_i, v_j) \in V_i \times V_j$ for $i < j$ in $G$ is added to $H$ if and only if $w'(v_i, v_j) \in L_{i, j}$. The high level idea then is to list a certain number of $k$-cliques in $H$ and verify whether any of them is an exact-$k$-clique in $G$. Clearly, this algorithm never finds an exact-$k$-clique if $G$ does not have one, so it suffices to show that when $G$ does have an exact-$k$-clique, the algorithm finds it with decent probability. 
    
    Let $(u_1, \ldots, u_k)$ be an arbitrary exact-$k$-clique in $G$. Clearly, there exists one combination of intervals such that all edges in this exact-$k$-clique are in the corresponding subgraph $H_0$. For any $i, j$ and edge $(v_i, v_j) \in V_i \times V_j$, the edge is in $H_0$ only if $w'(u_i, u_j) - w'(v_i, v_j) \in [-O(p/s), O(p/s)]$, which happens with probability $O(1/s)$ as long as $\{u_i, u_j\} \ne \{v_i, v_j\}$. The following lemma shows that the random variables $w'(u_i, u_j) - w'(v_i, v_j)$ are fairly independent. 
    
    \begin{lemma}
    \label{lem:independent}
    If $(v_1, \ldots, v_k)$ is not an exact-$k$-clique w.r.t. $w$, and $(u_1, \ldots, u_k)$ shares exactly $c$ nodes indexed by $S$ with $(v_1, \ldots, v_k)$, then the random variables $$\left\{w'(u_i,u_j) - w'(v_i, v_j)\right\}_{\substack{1 \le i < j \le k \\ i \not \in S \text{ or } j \not \in S}}$$ are independent.     
    \end{lemma}
    \begin{proof}
    By symmetry, we can assume $|S| = [c]$, and we need to show that 
        $$\left\{w'(u_i,u_j) - w'(v_i, v_j)\right\}_{\substack{c+1 \le j \le k \\ 1 \le i < j}} = \left\{ x\cdot (w(u_i,u_j) - w(v_i, v_j)) + y_{u_i, j} + y_{u_j, i} - y_{v_i, j} - y_{v_j, i}\right\}_{\substack{c+1 \le j \le k \\ 1 \le i < j}}$$
        are independent. 
        
        Define $\alpha_{i, j} = w'(u_i,u_j) - w'(v_i, v_j)$. Let $\beta$ be the sum of all the $\alpha_{i, j}$:
        \begin{align*}
            \beta &=\sum_{\substack{c+1 \le j \le k \\ 1 \le i < j}} \alpha_{i, j} 
            = \sum_{1 \le i < j \le k} \alpha_{i, j}\\
            &= x \cdot \left(\sum_{1 \le i < j \le k} \left( w(u_i, u_j) - w(v_i, v_j)\right)\right) + \sum_{i=1}^k \sum_{j \in [k] \setminus \{i\}} y_{u_i, j} - \sum_{i=1}^k \sum_{j \in [k] \setminus \{i\}} y_{v_i, j}\\
            &= x \cdot \left(\sum_{1 \le i < j \le k} \left( w(u_i, u_j) - w(v_i, v_j)\right)\right).
        \end{align*}
        Since $(u_1, \ldots, u_k)$ is an exact-$k$-clique whereas $(v_1, \ldots, v_k)$ is not, we have $\sum_{1 \le i < j \le k} \left( w(u_i, u_j) - w(v_i, v_j)\right) \ne 0$. Therefore, $\beta$ is uniformly random. 

        Showing $\left\{\alpha_{i, j}\right\}_{\substack{c+1 \le j \le k \\ 1 \le i < j}}$ are independent is equivalent to showing that the variables are independent when one of the variables is replaced with the sums of the variables. Namely, it suffices to show $\left\{\alpha_{i, j}\right\}_{\substack{c+1 \le j \le k \\ 1 \le i < j \\ (i, j) \ne (k-1, k)}} \cup \{\beta\}$ are independent. 
        
        Consider the following ordering of the variables:
        $$\beta, \alpha_{c+1, 1}, \ldots, \alpha_{c+1, c}, \alpha_{c+2, 1}, \ldots, \alpha_{c+2, c+1}, \ldots,
        \alpha_{k-1, 1}, \ldots, \alpha_{k-1, k-2}, 
        \alpha_{k, 1}, \ldots, \alpha_{k, k- 2}.$$
        Conditioned on the previous variables, all $\alpha_{j, i}$ variables in this list has an additive term $y_{u_j, i}$ that is independent of all previous variables. Thus, this list of variables is independent. 
    \end{proof}
    \begin{corollary}
    \label{cor:independent}
    For any $\ell$-clique on nodes $(v_i)_{i \in T}$ in $G$ that shares exactly $c$ nodes indexed by $S$ with $(u_1, \ldots, u_k)$, the random variables 
    $$\left\{w'(u_i,u_j) - w'(v_i, v_j)\right\}_{\substack{i, j \in T\\
    i < j\\ i \not \in S \text{ or } j \not \in S}}$$
    are independent. 
    \end{corollary}
    \begin{proof}
        By symmetry, we can assume $[T] =[\ell]$ and $[S] = [c]$. We can complete this $\ell$-clique to a nonzero $k$-clique $(v_1, \ldots, v_k)$ (we can  assume any $\ell$-clique is in some nonzero $k$-clique by adding hypothetical nodes to the graph in this analysis). 
        
        By Lemma~\ref{lem:independent}, $\left\{w'(u_i,u_j) - w'(v_i, v_j)\right\}_{\substack{c+1 \le j \le k \\ 1 \le i < j}}$ are independent, so 
        $\left\{w'(u_i,u_j) - w'(v_i, v_j)\right\}_{\substack{c+1 \le j \le \ell \\ 1 \le i < j}} $ are also independent. 
    \end{proof}
    
    Now we can compute the expected number of $\ell$-cliques in $H_0$. 
    The number of $\ell$-cliques in $G$ that share exactly $c$ nodes with $(u_1, \ldots, u_k)$ is $O(n^{\ell-c})$. By Corollary~\ref{cor:independent}, each of them is in $H_0$ with probability $O\left(1/s^{\binom{\ell}{2} - \binom{c}{2}}\right)$. Therefore, the expected number of $\ell$-cliques in $H_0$ is 
    $$O\left(\sum_{c=0}^\ell n^{\ell-c} / s^{\binom{\ell}{2} - \binom{c}{2}}\right) = O\left(n^\ell / s^{\binom{\ell}{2}}\right),$$
    since by our choice of $s = O(n^{2/(k-1)})$, we have that $s^{\binom{c}{2}} = O(n^c)$.

    
    Similarly, the expected number of $k$-cliques in $H_0$ that do not correspond to exact-$k$-cliques in $G$ is $O\left(n^k / s^{\binom{k}{2}}\right)$. 
    
    Therefore, by Markov's inequality and union bound, with probability at least $1-1/\Omega(\log n)$, the number of $\ell$-cliques in $H_0$ is at most $n^\ell \log n/ s^{\binom{\ell}{2}}$ and the number of $k$-cliques in $H_0$ that do not correspond to exact-$k$-cliques in $G$ is at most $n^k \log n / s^{\binom{k}{2}}$. 
    
    Let $s = n^{\frac{k-\gamma \ell}{\binom{k}{2} - \gamma \binom{\ell}{2}}}$, so that $\frac{n^k}{s^{\binom{k}{2}}} = \left(\frac{n^\ell}{s^{\binom{\ell}{2}}}\right)^\gamma$. We can verify that indeed $\Omega(1) \le s \le O(n^{\frac{2}{k-1}})$. In fact, since $\frac{2}{\ell(k-\ell)} + \gamma(1 - \frac{2}{k(k-\ell)}) > 1$, we can obtain a stronger upper bound $s = O\left(n^{\frac{k-\ell}{\binom{k}{2}-\binom{\ell}{2}-1}-\delta}\right)$ for $\delta>0$.

    
    
    Suppose for the sake of contradiction that there is a $\cliquelist{k, \ell}$ algorithm $\mathcal{A}$ for instances with specified $t = \tilde{\Theta}(\Delta_\ell^\gamma)$  with running time $$T(\Delta_\ell, t) = O\left(\left(\Delta_\ell^{\frac{2}{\ell(k-\ell)}} t^{1 - \frac{2}{k(k-\ell)}}\right)^{1-\eps}\right)$$
    for some $\eps > 0$. Then consider the following algorithm for Exact-$k$-Clique:
    \begin{enumerate}
        \item 
        First, hash the weights of the graph and enumerate $O(s^{\binom{k}{2} - 1})$ graphs $H$ as described earlier.
        \item Enumerate all $\ell$-cliques in $G$, and pre-compute which graphs $H$ contain each $\ell$-clique. Since each $\ell$-clique exists in $s^{\binom{k}{2}-1 -\binom{\ell}{2}}$ graphs $H$, and this list of graphs can be listed efficiently, this step costs 
        $$\tO\left(n^\ell \cdot s^{\binom{k}{2}-1 -\binom{\ell}{2}}\right) \le \tO\left(n^\ell \cdot \left(n^{\frac{k-\ell}{\binom{k}{2}-\binom{\ell}{2}-1}-\delta}\right)^{\binom{k}{2}-1 -\binom{\ell}{2}}\right)\le \tO(n^{k-\delta'})$$
        for some $\delta' > 0$.
        
        \item \label{item:lower-bound-skip} From the previous step, we have a list of $\ell$-cliques for each graph $H$. If some $H$ contains more than $n^\ell \log n/ s^{\binom{\ell}{2}}$ $\ell$-cliques, we skip it. If it contains fewer than $0.99 n^\ell / s^{\binom{\ell}{2}}$ $\ell$-cliques, we add a complete $\ell$-partite graphs with $n'$ nodes on each part, for some $n'$, so that the total number of $\ell$-cliques in the new graph reaches $0.99(n+n')^\ell / s^{\binom{\ell}{2}}$. Clearly, $n' = O(n)$. 
        \item \label{item:lower-bound-listing} For graphs $H$ which we did not skip in the previous step, we run $\mathcal{A}$ on it with $t = n^k \log n / s^{\binom{k}{2}} + 1$. For any $k$-clique listed by $\mathcal{A}$, we test whether it is an exact-$k$-clique in $G$. This step takes $s^{\binom{k}{2}-1} \cdot T(\tO(n^\ell / s^{\binom{\ell}{2}}), \tO(n^k / s^{\binom{k}{2}}))$ time.
        \item If any exact-$k$-clique is found in the previous step, we return YES for the Exact-$k$-Clique instance; otherwise, we return NO. 
    \end{enumerate}
    
    Clearly, if $G$ contains no exact-$k$-clique, our algorithm is always correct. If $G$ contains any exact-$k$-clique, let $H_0$ be the constructed graph containing it. As discussed previously, with probability $1-1/\Omega(\log n)$, the number of $\ell$-cliques in $H_0$ is at most $n^\ell \log n/ s^{\binom{\ell}{2}}$ and the number of $k$-cliques in $H_0$ that do not correspond to exact-$k$-cliques in $G$ is at most $n^k \log n / s^{\binom{k}{2}}$. In this case, we will not skip $H_0$ in Step~\ref{item:lower-bound-skip}, and listing $t = n^k \log n / s^{\binom{k}{2}} + 1$ $k$-cliques in Step~\ref{item:lower-bound-listing} guarantees an exact-$k$-clique. Thus, we will find an exact-$k$-clique with probability $1-1/\Omega(\log n)$, which can be boosted to $1-1/\poly(n)$ by repeating the algorithm $O(\log n)$ times. 
    
    Overall, this algorithm only needs time (besides the previous $\tO(n^{k-\delta'})$ time)
    \begin{align*}
        &\tilde{O}\left(s^{\binom{k}{2}-1}\cdot \left(\left(n^\ell / s^{\binom{\ell}{2}}\right)^{\frac{2}{\ell(k-\ell)}} \left(n^k / s^{\binom{k}{2}}\right)^{1-\frac{2}{k(k-\ell)}} \right)^{1-\eps}\right) =  \tilde{O}\left(n^k \cdot \left(\frac{s^{\binom{k}{2}-1}}{n^k}\right)^\eps\right).
    \end{align*}
    As $s = O(n^{\frac{2}{k-1}})$, the above running time can be further upper bounded by 
    \begin{align*}
        \tilde{O}\left(n^k \cdot \left(\frac{(n^{\frac{2}{k-1}})^{\binom{k}{2}-1}}{n^k}\right)^\eps\right) = \tO\left(n^{k-\frac{2\eps}{k-1}}\right),
    \end{align*}
    contradicting the Exact-$k$-Clique hypothesis. 
\end{proof}




\section{Optimal Listing Algorithms for Graphs with Many \texorpdfstring{$k$}{k}-Cliques}
\label{sec:upper-bound}
In this section, we give a $\cliquelist{k, 1}$ algorithm that is  optimal for graphs with many $k$-cliques under Hypothesis~\ref{hyp:exact_k_clique}. This algorithm can be seen as a generalization of the densifying and sparsifying paradigm of  \cite{bjorklund2014listing}. 

We then show how we can extend this algorithm to obtain the conditionally optimal algorithms for all $\cliquelist{k, \ell}$ for graphs with many $k$-cliques.

\subsection{Algorithm}

First, we describe the algorithm for $\cliquelist{k, 1}$ in Algorithm~\ref{alg:large_t_sparse_dense}. 

 \begin{breakablealgorithm}
        \caption{$\cliquelist{k, 1}$ Algorithm for large $t \geq n^{\gamma_k}$, where $\gamma_k$ is defined in Theorem~\ref{thm:k_1_optimal}}\label{alg:large_t_sparse_dense}
        
        
        \begin{algorithmic}
            \item \dense{}$(G:= (V, E), n, t)$:
            \begin{itemize}
            \item \textbf{Input:} Graph $G = (V, E)$ with $|V| \leq n$ and at most $t$ $k$-cliques.
            \item \textbf{Output:} List of $k$-cliques in $G$.
            \item \textbf{The Algorithm:}
        \begin{enumerate}
            \item If $n < k$, it returns no $k$-cliques.
            \item Choose a parameter $\lambda$. Let an edge be $\lambda$-light if it is in fewer than $\lambda$ $k$-cliques.
            \item Use the algorithm in Lemma~\ref{lem:simple_list_ub} to obtain a list $L$ of all $(k-2)$-cliques (there are at most $n^{k-2}$ such cliques).
            \item Initialize an empty list $T$.
            \item \label{step:matmul_sampling}Repeat the following $O(\lambda \log n)$ times:
            \begin{itemize}
                \item Sample a subset $L'$ of $L$ of size $|L|/\lambda$.
                \item Construct adjacency matrices $A$ and $\overline{A}$ where the rows are indexed by $V$ and columns are indexed by $L'$.
                \item Let $A[v, C] = 1$ if node $v$ is distinct from and adjacent to every node in the $(k-1)$-clique $C$, and set $A[v, C] = 0$ otherwise.
                \item Let $\overline{A}[v, C] = A[v, C] \cdot C$, i.e. column $C$ contains entries 0 or $C$. 
                \item Compute $B = A \cdot A^T$ and $\overline{B} = A \cdot \overline{A}^T$. This takes $O(\MM(n, |L'|, n))$ time.
                \item For every edge $(u, v) \in E$ that is $\lambda$-light, if $B[u, v] = 1$, add $(u, v, \overline{B}[u, v])$ to $T$.
            \end{itemize}
            \item Output $T$.
            \item Delete all $\lambda$-light edges from $E$ to obtain $E'$ (all $\lambda$-light edges are found in Step \ref{step:matmul_sampling} w.h.p.).
            \item Call {\tt Sparse}$(G' := (V, E'), \binom{k}{2}t/\lambda, t)$.
        \end{enumerate}
        \end{itemize}
    \item\sparse{}$(G:= (V, E), m, t)$:
       \begin{itemize}
       \item \textbf{Input:} Graph $G = (V, E)$ with $|E| \leq m$ and at most $t$ $k$-cliques.
       \item \textbf{Output:} List of $k$-cliques in $G$.
       \item \textbf{The Algorithm:}
        \begin{enumerate}
            \item If $m < \binom{k}{2}$, it returns no $k$-cliques.    
            \item Choose a parameter $x$.
            \item Find all nodes such that $\deg(v) \leq x$, and call the $\cliquelist{k-1,1}$ algorithm in the neighbourhoods of all such nodes with $n' = \deg(v)$.
            \item Delete all nodes in $V$ of degree less than $x$ to obtain set $V'$.
            \item Call {\tt Dense}$(G' := (V', E \cap (V' \times V')), 2m/x, t)$
        \end{enumerate}
    \end{itemize}
    \end{algorithmic}
    \end{breakablealgorithm}
    
 

In the $\dense$ algorithm, we use matrix multiplication to enumerate all $k$-cliques containing light edges, i.e. edges that are part of very few $k$-cliques. These edges are then removed to result in a  \emph{sparse} graph with only edges that are part of many $k$-cliques.

In the $\sparse$ algorithm, we enumerate all $k$-cliques containing low-degree nodes by recursively listing all $(k-1)$-cliques in their neighborhoods, and delete all such nodes. Deleting these nodes results in a \emph{dense} graph with only high degree nodes. While one could brute-force the $(k-1)$-cliques in the neighborhoods, our key insight is that we can instead recursively use a $\cliquelist{k-1, 1}$ algorithm to be more efficient. 

We first show the correctness of Algorithm~\ref{alg:large_t_sparse_dense} and defer its runtime analysis to Section~\ref{sec:k_1_opt}.

\paragraph{Correctness.} It is clear that the $\sparse$ algorithm finds all $k$-cliques in the neighborhoods of low-degree nodes. At the end of the algorithm, since the graph has $\Delta_\ell$ $\ell$-cliques and only nodes with degree at least $x$, there are at most $2m/x$ nodes left in the graph.

Now, we argue that the $\dense$ algorithm lists all $k$-cliques containing $\lambda$-light edges.

We argue that Step~\ref{step:matmul_sampling} finds all $\lambda$-light edges with high probability. For every $(u, v) \in E$, let $L_{u, v}$ denote the set of all $(k-2)$-cliques that form $k$-cliques with nodes $u$ and $v$. Since we sample $L'$ of size $|L|/\lambda$, the probability that $L_{u, v} \cap L' = K_{k-2}$ for any fixed $K_{k-2} \in L_{u, v}$ is \[\frac{1}{\lambda} \cdot \left(1 - \frac{1}{\lambda}\right)^{|L_{u, v}| - 1} \geq \frac{1}{\lambda} \cdot \left (1 - \frac{1}{\lambda}\right)^{\lambda-1} \geq \frac{1}{e \lambda}.\]
Therefore, by choosing $O(\lambda \log  n)$ random sets of size $|L|/\lambda$, with high probability, we find all $k$-cliques containing $\lambda$-light edges.

\paragraph{$\cliquelist{k, \ell}$ when $\ell \geq 2$.} To generalize this algorithm to $\cliquelist{k, \ell}$ for $\ell \geq 2$, we recursively use $\cliquelist{k-1, \ell-1}$ to reduce the problem to $\cliquelist{k, 1}.$ At a high level, the algorithm considers all nodes $v$ in fewer than $x$ $\ell$-cliques and recursively calls $\cliquelist{k-1,\ell-1}$ to list all $k$-cliques containing $v$. See Algorithm~\ref{alg:optimal_kl}. The correctness of Algorithm~\ref{alg:optimal_kl} can be shown as follows. 

\begin{algorithm}[ht]
    \caption{$\cliquelist{k, \ell}$ Algorithm for large $t \geq n^{\gamma_{k,\ell}}$, where $\gamma_{k, \ell}$ is defined in Theorem~\ref{thm:k-l-large-t-listing}}\label{alg:optimal_kl}
    \begin{algorithmic}
    \item \textbf{Input:} A graph $G$ and a list $L$ of all $\ell$-cliques.
    \item \textbf{Output:} All $k$-cliques in the graph.
    \item \textbf{The Algorithm:}
    \begin{enumerate}
        \item Call a node $v$ light if $\Delta_\ell(v) \leq x$, for some parameter $x$. 
        \item\label{step:kl_opt_lightnodes} For all light nodes, call $\cliquelist{k-1, \ell-1}$ in the neighbourhoods to find all $k$-cliques incident to $x$. 
        \item Delete all light nodes and incident edges from $G$. 
        \item\label{step:kl_densecall} Call the $\cliquelist{k, 1}$ algorithm $\dense(G' := (V', E'), \ell \Delta_\ell/x, t)$ (from Algorithm~\ref{alg:large_t_sparse_dense}).
    \end{enumerate}
    \end{algorithmic}
\end{algorithm}

\paragraph{Correctness.} It is clear that the algorithm lists all $k$-cliques incident to low-degree nodes. Since all remaining nodes are in at least $x$ $\ell$-cliques, and each $\ell$-cliques contains at most $\ell$ nodes, we can bound the remaining number of nodes by $\ell \Delta_\ell/x$.

To illustrate these algorithms, we first show simplified analyses of Algorithms~\ref{alg:large_t_sparse_dense} and \ref{alg:optimal_kl} for the case of $k = 4$ and $k=5$ assuming that $\omega = 2$ in Section~\ref{sec:4_5_l_listing}. We give more detailed analyses in terms of $\omega$ in Sections~\ref{sec:k_1_opt} and \ref{sec:k_l_opt}. 

\subsection{Analysis for \texorpdfstring{$k = 4$}{k = 4} and \texorpdfstring{$k = 5$}{k = 5} assuming \texorpdfstring{$\omega = 2$}{omega = 2}}\label{sec:4_5_l_listing}
In this section, we illustrate how to analyze the runtime for listing algorithm by considering the cases where $k = 4$ or $k=5$.
\begin{proposition}
    Suppose $\omega = 2$. Then, given a graph $G$ with $t$ 4-cliques,
    \begin{itemize}
        \item $\cliquelist{4,1}$ can be solved in $\tilde{O}(n^3 + n^{2/3}t^{5/6})$ if $G$ has $n$ nodes.
        \item $\cliquelist{4,2}$ can be solved in $\tilde{O}(m^{3/2} + mt^{2/5} + m^{1/2}t^{3/4})$ if $G$ has $m$ edges.
        \item $\cliquelist{4, 3}$ can be solved in $\tilde{O}(\Delta^{6/5} + \Delta t^{1/5} + \Delta^{2/3}t^{1/2})$ if $G$ has $\Delta = \Delta_3$ triangles.
     \end{itemize}  
\end{proposition}

\begin{proof}
    Consider the $\dense$ algorithm. In this case, $L$ is a list of all (up to $n^2$) edges. Therefore, the runtime of this step can be bounded by 
    \[D'(n, m, t) \leq n^2 + \lambda \log n \cdot \MM(n, m/\lambda, n) + S(6t/\lambda, t) = \tilde{O}(n^2 + \lambda n^2 + nm) + S(6t/\lambda).\]
    We can also upper bound $m$ by $n^2$ to obtain the following bound without a dependence on $n$:
\[
    D(n, t) \leq D(n, n^2, t) \leq \tilde{O}(n^3 + \lambda n^2) + S(6t/\lambda, t).
\]
assuming $\omega = 2$. 
    
Consider the $\sparse$ algorithm. In this case, we call $(3,1)$-listing, which takes time $\tO(n^2 + nt^{2/3})$. Therefore (ignoring $\tO(1)$ factors), 
\begin{align*}
    S(m, t) &\leq \sum_{v: \deg(v) \leq x} \left(\deg(v)^2 + \deg(v) \Delta_4(v)^{2/3}\right)+ D'(2m/x, m, t)\\
    &\leq  \sum_{v: \deg(v) \leq x} \left(\deg(v) \cdot x + \deg(v)^{1/3} \Delta_4(v)^{2/3} x^{2/3}\right)+ D'(2m/x, m, t)\\
    &\leq mx + m^{1/3}t^{2/3}x^{2/3} + D'(2m/x, m, t),
\end{align*}
where we applied H\"{o}lder's inequality as seen in Corollary~\ref{cor:holders_useful}. 

\paragraph{$\cliquelist{4, 1}$ analysis.} To obtain a runtime for $\cliquelist{4,1}$, we unravel the recursion in $D(n, t)$. Ignoring $\tilde{O}(1)$ factors in the following inequalities, we have
\begin{align*}
    D(n, t) &\leq n^3 + \lambda n^2 + S(6t/\lambda, t)\\
    &\leq n^3 + \lambda n^2 + \frac{tx}{\lambda} + \frac{tx^{2/3}}{\lambda^{1/3}} + D\left(\frac{12t}{\lambda x}, t\right)
\end{align*}
Choosing $\lambda = \max\{1, \frac{24t}{nx}\}$, we have $\frac{12t}{\lambda x} \leq n/2$, and the above runtime will be dominated by the first four terms up to $\tilde{O}(1)$ factors. Substituting this value of $\lambda$, we obtain a runtime of 
    \[D(n, t) \leq n^3 + \frac{tn}{x} + nx^2 + n^{1/3}t^{2/3}x.
    \]
Choosing \[x = 
    \begin{cases}
        n & t \leq n^{5/2}\\
        n^{8/3}/t^{2/3} & n^{5/2} \leq t \leq n^{14/5}\\
        n^{1/3}t^{1/6} & t \geq n^{14/5}
    \end{cases},\]
we obtain $D(n, t) = n^3 + n^{2/3}t^{5/6}$, as desired.

\paragraph{$\cliquelist{4, 2}$ analysis.} To obtain a runtime for $\cliquelist{4, 2}$, we analyze the runtime of $S(m, t)$.
Here, we use our bound $D'$ in terms of $n$, $m$ and $t$ to get a tighter analysis
(instead of just $n$ and $t$). 

\begin{align*}
    S(m, t) &\leq  mx + m^{1/3}t^{2/3}x^{2/3} + D'(2m/x,m, t)\\
    &\leq mx + m^{1/3}t^{2/3}x^{2/3} + \frac{\lambda m^2}{x^2} + \frac{m^2}{x} + S(6t/\lambda, t).
\end{align*}
Choosing $\lambda = \max\{1, \frac{12t}{m}\}$, we have that $6t/\lambda \leq m/2$. Therefore, the first four terms dominate up to $\tilde{O}(1)$ factors, so (ignoring $\tO(1)$ factors)
\begin{align*}
    S(m, t) &\leq mx + m^{1/3}t^{2/3}x^{2/3} + \frac{mt}{x^2} + \frac{m^2}{x}.
\end{align*}
By choosing 
\[
    x = \begin{cases}
        m^{1/2} &\text{if $t \leq m^{5/4}$}\\
        m/t^{2/5} &\text{if $m^{5/4} < t \leq m^{10/7}$}\\
        m^{1/4}t^{1/8} & \text{if $t > m^{10/7}$,}
    \end{cases}
\] 
we get a runtime of 
\[S(m, t) \leq m^{3/2} + mt^{2/5} + m^{1/2}t^{3/4}.\]

\paragraph{$\cliquelist{4,3}$ analysis.} Note that while we can use Algorithm~\ref{alg:optimal_kl} to bound the runtime in this case, we instead provide a more efficient algorithm shown in Algorithm~\ref{alg:4_3} for $\cliquelist{4, 3}$.

\begin{algorithm}[ht]
    \caption{$\cliquelist{4, 3}$ algorithm}\label{alg:4_3}
    \begin{algorithmic}
    \item \textbf{Input:} Graph $G = (V, E)$, and a list $L$ of all triangles in $G$.
    \item \textbf{Output:} A list of all $k$-cliques in $G$.
    \begin{enumerate}
        \item Call an edge light if it occurs in fewer than $x$ triangles, i.e. $\Delta(e) \leq x$. 
        \item\label{step:light_edges_in_triangles} For all light edges $e$, consider all pairs of nodes in its neighbourhoods to find all 4-cliques containing $e$.
        \item Delete all light edges from $G$.
        \item Call $\cliquelist{4, 2}$ algorithm $\sparse(G' := (V', E'), 3\Delta/x, t).$ 
    \end{enumerate}
    \end{algorithmic}
\end{algorithm}

The runtime of Step~\ref{step:light_edges_in_triangles} is bounded by $\sum_{e:\Delta(e) \le x} \Delta(e)^2 \leq \Delta x.$ Now, we call $\cliquelist{4, 2}$ with a graph with at most $3\Delta/x$ edges and $t$ $4$-cliques, giving a runtime of 
\[
    \left(\Delta/x\right)^{3/2} + \left(\Delta/x\right) t^{2/5} + \left(\Delta/x\right)^{1/2} t^{3/4}.
\]
Therefore, choosing $x = \max\{\Delta^{1/5}, t^{1/5}, t^{1/2}/\Delta^{1/3}\}$, we get a runtime of $\tilde{O}\left(\Delta^{6/5} + \Delta t^{1/5} + \Delta^{2/3} t^{1/2}\right).$ 

\end{proof}

\begin{proposition}
    Suppose $\omega = 2$. Then, given a graph $G$ with $t$ 5-cliques,
    \begin{itemize}
        \item $\cliquelist{5,1}$ can be solved in $\tilde{O}(n^4 + n^{1/2}t^{9/10})$ if $G$ has $n$ nodes.
        \item $\cliquelist{5,2}$ can be solved in $\tilde{O}(m^2 + m^{17/18}t^{10/18} + m^{1/3}t^{13/15})$ if $G$ has $m$ edges.
     \end{itemize}  
\end{proposition}

\begin{proof}
Consider the $\dense$ algorithm. In this case, $L$ is a list of all (up to $n^3$) triangles. Therefore, the runtime of this step is 
\[D'(n, \Delta_3, t) \leq n^3 + \lambda \log n \cdot \MM(n, \Delta_3/\lambda, n) + S(10t/\lambda, t) = \tilde{O}(n^3+n\Delta_3 + \lambda n^2) + S(10t/\lambda, t)\]
assuming $\omega = 2$.
Upper bounding $\Delta_3 \leq O(n^3)$, we get a bound without dependence on $\Delta_3$ of \[D(n, t) \leq D'(n, n^3, t) = \tilde{O}(n^4 + \lambda n^2) +S(10t/\lambda).\]
Consider the $\sparse$ algorithm. In this case, we call $\cliquelist{4, 1}$ in the neighborhoods of all low-degree nodes, so
\begin{align*}
    S(m, t) \leq \sum_{v: \deg(v) \leq x} \left(\deg(v)^3 + \deg(v)^{2/3} \Delta_5(v)^{5/6}\right) + D(2m/x, t) &\leq mx^2 + m^{1/6}t^{5/6}x^{1/2} +  D(2m/x, t)
\end{align*}
by using H\"{o}lder's inequality as in Corollary~\ref{cor:holders_useful}.

\paragraph{$\cliquelist{5, 1}$ analysis.} To obtain a runtime for $\cliquelist{5, 1}$, we analyze the runtime of $D(n, t) $ by unravelling the recursion. Therefore, we have the following inequalities (omitting $\tilde{O}(1)$ factors):
\begin{align*}
    D(n, t) &\leq n^4 + \lambda n^2 + \frac{tx^2}{\lambda} + \left(\frac{t}{\lambda}\right)^{1/6} t^{5/6}x^{1/2} + D\left(\frac{20t}{\lambda x}, t\right)
\end{align*}
By choosing $\lambda = \max\left\{5, \frac{40 t}{nx}\right\}$, we would have $\frac{20t}{\lambda x} \leq \frac{n}{2}$, and the running time will therefore be dominated by the first 4 terms. Choosing 
\begin{align*}
    x = 
    \begin{cases}
        n & \text{if } t \leq n^{19/15}\\
        n^{23/4}/t^{5/4} & \text{if } n^{19/5} \leq t \leq n^{35/9}\\
        n^{1/2}t^{1/10} & \text{if } t \geq n^{35/9}
    \end{cases},
\end{align*}
we obtain a runtime of $\tilde{O}(n^4 + n^{1/2}t^{9/10}).$

\paragraph{$\cliquelist{5, 2}$ analysis.} We now analyze the runtime of $S(m, t).$ Note that the graph has at most $\Delta_3 = O(m^{3/2})$ triangles. Here, we use $D'(n, \Delta_3, t)$ to bound the runtime instead.
Therefore, unrolling the recursion, we have (up to $\tilde{O}(1)$ factors) 
\begin{align*}
    S(m, t) &\leq mx + m^{1/6}t^{5/6}x^{1/2} + D'(m/x, m^{3/2}, t)\\
    &\leq mx + m^{1/6}t^{5/6}x^{1/2} + (m/x) \cdot m^{3/2} + (m/x)^{1/2}t^{9/10}.
\end{align*}
Setting 
\[
    x = \begin{cases}
        m^{1/2} & \text{if } t \leq m^{19/10}\\
        m^{17/18}t^{10/18} & \text{if }m^{19/10} \leq t \leq m^{55/28}\\
        m^{1/3}t^{13/15} & \text{if } t \geq m^{55/28}
    \end{cases},
\]
we get a runtime of $\tilde{O}(m^2  + m^{17/18}t^{10/18} + m^{1/3}t^{13/15}).$ 
\end{proof}

\subsection{Analysis for \texorpdfstring{$\cliquelist{k, 1}$}{(k,1)-Clique-Listing}}\label{sec:k_1_opt} 

For $k \geq 2$, define 
\begin{align}
    x_k &= k \prod_{j = 2}^k ((5 - 2j) + (j - 2)\omega)\label{eq:alpha_num}\\
    y_k &= (3-\omega)^{k-2} + \sum_{j = 2}^{k-1} (3-\omega)^{k-1-j} x_j
    \label{eq:alpha_denom}
\end{align}

The following identities are immediate. 
\begin{claim}
    For any $k \ge 3$, $x_k = x_{k-1} \cdot \frac{k}{k-1} \cdot ((5-2k)+(k-2)\omega)$ and $y_k = (3  - \omega) \cdot y_{k-1} +  x_{k-1}$.
\end{claim}

\begin{theorem}\label{thm:k_1_optimal}
    Let $\alpha_k = x_k/y_k$. For any $k \geq 2$ and large $t \geq n^{\gamma_k}$ where 
    \[\gamma_k = 
    \begin{cases}
    0 & \text{if } k = 2\\
    k \left(1 - \frac{3 - \omega}{k - \alpha_k}\right) & \text{if } k \geq 3
    \end{cases},\]  
    there exists an algorithm that lists all $t$ $k$-cliques in time $\tO(n^{\alpha_k}t^{1-\frac{\alpha_k}{k}})$. If $\omega=2$,  we have that $x_k = k$ and $y_k = \frac{k(k-1)}{2}$, therefore giving a runtime of $\tO(n^{\frac{2}{k-1}} t^{1 - \frac{2}{k(k-1)}})$ for $t \geq n^{k - 1 - \frac{2}{k^2 - k - 2}}.$
\end{theorem}

\begin{proof}
    For $k = 2$, the brute-force algorithm runs in $n^2$ time, and it is easy to check that $x_k=2$ and $y_k = 1$. Moreover, this bound holds for all values of $t$, so we can set $\gamma_k = 0.$
    
    For $k = 3$, \cite{bjorklund2014listing} give an algorithm that runs in time $O(n^\omega + n^{\frac{3(\omega - 1)}{5 - \omega}}t^{\frac{2(3-\omega)}{5 - \omega}})$, which can easily be verified to match the form of the theorem statement. Rewriting this as $O(n^\omega + n^{\alpha_3}t^{1- \frac{\alpha_3}{3}})$, it is easy to see that this term dominates exactly when $t \geq n^{3\left(1 - \frac{3-\omega}{3-\alpha_3}\right)}$, which corresponds exactly to our setting of $\gamma_3.$
    Now suppose  $k \geq 3$ and that the theorem statement is true for all $\cliquelist{r, 1}$ for all $r < k$.
    In particular, suppose the runtime of $\cliquelist{k-1, 1}$ is bounded by $$T_{k-1}(n, \Delta_{k-1}) \leq n^{\alpha_{k-1}}\Delta_{k-1}^{1 - \frac{\alpha_{k-1}}{k-1}},$$ when $\Delta_{k-1} \ge n^{\gamma_{k-1}}$ for some $\gamma_{k-1}$. In fact, since the runtime is non-decreasing in the parameter $\Delta_{k-1}$ by Lemma~\ref{lem:list_exponent_monotone}, one can bound the above runtime for any $\Delta_k$  by:
    \begin{align*}
        T_{k-1}(n, \Delta_{k-1}) \leq n^{\alpha_{k-1}}\left(n^{\gamma_{k-1}}\right)^{1 - \frac{\alpha_{k-1}}{k-1}} + n^{\alpha_{k-1}}\Delta_{k-1}^{1 - \frac{\alpha_{k-1}}{k-1}}.
    \end{align*}

    \paragraph{Runtime analysis.} Let $D(n, t)$ be the running time of $\dense(G, n, t)$, and let $S(m, t)$ be the running time of $\sparse(G, m, t)$. Note that ignoring $\tilde{O}(1)$ factors
    \begin{align*}
        D(n, t) &\le n^{k-1} + \lambda \MM(n, n^{k-2}/\lambda, n) + S\left(\binom{k}{2}t/\lambda, t\right)
    \end{align*}
    
    By the standard trick of decomposing a rectangular matrix product into smaller square matrix products, one can bound $$\MM(n, n^{k-2}/\lambda, n) \leq \left(\frac{n^{k-2}/\lambda}{n}\right) \cdot n^\omega + \left(\frac{n}{n^{k-2}/\lambda}\right)^2 \cdot \left(\frac{n^{k-2}}{\lambda}\right)^\omega = \frac{n^{k-3+\omega}}{\lambda} + \frac{n^{\omega(k-2) -2k + 6}}{\lambda^{\omega - 2}}.$$
    Therefore, we can rewrite 
    $$D(n, t) \leq n^{k-3+\omega} + \lambda^{3 - \omega} n^{\omega(k-2) -2k + 6} + S\left(\binom{k}{2}t/\lambda, t\right).$$

    For $\sparse(G, m, t)$, note that the runtime is bounded by:
    \begin{align*}
        S(m, t) &\leq \sum_{v : \deg(v) \leq x} T_{k-1}(\deg(v), \Delta_k(v)) + D(2m/x, t).\\
        &\leq \sum_{v : \deg(v) \leq x} \left(\deg(v)^{\alpha_{k-1} + \gamma_{k-1}\left(1 - \frac{\alpha_{k-1}}{k-1}\right)} + \deg(v)^{\alpha_{k-1}}\Delta_k(v)^{1 - \frac{\alpha_{k-1}}{k-1}}\right) + D(2m/x, t).
    \end{align*}
    
    One can use H\"{o}lder's inequality as in Corollary~\ref{cor:holders_useful} to bound 
    \begin{align*}
        \sum_{v: \deg(v) \leq x} \deg(v)^{\alpha_{k-1}} \Delta_k(v)^{1 - \frac{\alpha_{k-1}}{k-1}} 
        &\leq x^{\alpha_{k-1} - \frac{\alpha_{k-1}}{k-1}}\sum_{v: \deg(v) \leq x} \deg(v)^{\frac{\alpha_{k-1}}{k-1}}\Delta_k(v)^{1 - \frac{\alpha_{k-1}}{k-1}} \\
        &\leq 
        x^{\alpha_{k-1} \cdot \frac{k-2}{k-1}} \left(\sum_{v: \deg(v) \leq x} \deg(v)\right)^{\frac{\alpha_{k-1}}{k-1}} \left(\sum_{v: \deg(v) \leq x} \Delta_k(v) \right)^{1- \frac{\alpha_{k-1}}{k-1}}\\
        & \leq
        O\left(x^{\alpha_{k-1} \cdot \frac{k-2}{k-1}} m^{\frac{\alpha_{k-1}}{k-1}} 
        t^{1- \frac{\alpha_{k-1}}{k-1}}\right).
    \end{align*}
    
    Thus, we have that (once again ignoring $\tilde{O}(1)$ factors)
    $$S(m, t) \leq m \cdot x^{\alpha_{k-1} + \gamma_{k-1}\left(1 - \frac{\alpha_{k-1}}{k-1}\right) - 1} + x^{\alpha_{k-1} \cdot \frac{k-2}{k-1}} m^{\frac{\alpha_{k-1}}{k-1}} 
        t^{1- \frac{\alpha_{k-1}}{k-1}} + D(2m/x, t).$$
   
    Unravelling the runtime of $\dense(G, n, t)$, we therefore have
    \begin{align*}
        D(n, t) \leq & n^{k-3+\omega} + \lambda^{3-\omega}n^{\omega(k-2) -2k + 6}\\ & + (t/\lambda) \cdot x^{\alpha_{k-1} + \gamma_{k-1}\left(1 - \frac{\alpha_{k-1}}{k-1}\right) - 1} + x^{\alpha_{k-1} \cdot \frac{k-2}{k-1}} (t/\lambda)^{\frac{\alpha_{k-1}}{k-1}} 
        t^{1- \frac{\alpha_{k-1}}{k-1}} \\
        &+ D\left(\frac{2\cdot \binom{k}{2}t}{\lambda x}, t\right).
    \end{align*}
    If one chooses $\lambda$ and $x$ so that $\frac{2 \cdot \binom{k}{2} t}{\lambda x} \leq \frac{n}{2}$, then the runtime is dominated by the first 4 terms up to $\tilde{O}(1)$ factors. Therefore, we choose $\lambda = \max\{1, \frac{4 \cdot \binom{k}{2} t}{n x}\}$ (note that for $t \geq n^{\gamma_k}$, this value will always be equal to $\frac{4 \cdot \binom{k}{2} t}{n x}$ for our setting of $x$). Hence, ignoring $\tilde{O}(1)$ factors, this gives us for $t \geq n^{\gamma_k}$, 
    \begin{align*}
        D(n, t) &\leq n^{k-3+\omega} + \left(\frac{t}{nx}\right)^{3 - \omega}n^{\omega(k-2) -2k + 6} \\
        &+ (n\cdot x) \cdot x^{\alpha_{k-1} + \gamma_{k-1}\left(1 - \frac{\alpha_{k-1}}{k-1}\right) - 1} + x^{\alpha_{k-1} \cdot \frac{k-2}{k-1}} (n\cdot x)^{\frac{\alpha_{k-1}}{k-1}} 
        t^{1- \frac{\alpha_{k-1}}{k-1}}
    \end{align*}
    
    First, suppose that the term $\left(\frac{t}{nx}\right)^{3 - \omega}n^{\omega(k-2) -2k + 6}$
    dominates $n^{k-3+\omega}$ and the term $x^{\alpha_{k-1} \cdot \frac{k-2}{k-1}} (n \cdot x)^{\frac{\alpha_{k-1}}{k-1}} t^{1- \frac{\alpha_{k-1}}{k-1}}$ dominates $(n\cdot x) \cdot x^{\alpha_{k-1} + \gamma_{k-1}\left(1 - \frac{\alpha_{k-1}}{k-1}\right) - 1}$ (we show that this is in fact true for our choice of $\gamma_k$ later) . Then, 
    \begin{equation}\label{eq:bounding_dense}
        D(n, t) \leq \left(\frac{t}{nx}\right)^{3- \omega}n^{\omega(k-2) -2k + 6} + x^{\alpha_{k-1} \cdot \frac{k-2}{k-1}} (n \cdot x)^{\frac{\alpha_{k-1}}{k-1}}t^{1- \frac{\alpha_{k-1}}{k-1}}.
    \end{equation}
    Choosing $x$ to equate the two terms, we set 
    \begin{equation}\label{eq:x_solution_k1_runtime}
        x = \left({t^{\alpha_{k-1} - (\omega-2)(k-1)}n^{(k-1)^2 \omega - (\alpha_{k-1} + 2k^2 - 5k + 3)}}\right)^{\frac{1}{(k-1)(3+\alpha_{k-1} - \omega)}}.
    \end{equation}
    Substituting this into \eqref{eq:bounding_dense}, we have \[D(n, t) \leq n^{\frac{k\alpha_{k-1}((5-2k)+(k-2)\omega)}{(k-1) \cdot (3 + \alpha_{k-1} - \omega)}} t^{\frac{(3-\omega)((k-2)\alpha_{k-1} + (k-1))}{(k-1) \cdot (3 + \alpha_{k-1} - \omega)}}.
    \]
    Setting 
    \begin{equation}\label{eq:alpha_recursive}
    \alpha_k = \frac{k\alpha_{k-1}((5-2k)+(k-2)\omega)}{(k-1) \cdot (3 + \alpha_{k-1} - \omega)},
    \end{equation}
    it is easy to verify that the above bound is in fact of the form $n^{\alpha_k} t^{1-\frac{\alpha_k}{k}}.$ Moreover, note that
    \begin{align*}
        \alpha_k &= \frac{k\alpha_{k-1}((5-2k)+(k-2)\omega)}{(k-1) \cdot (3 + \alpha_{k-1} - \omega)}\\
        &= \frac{k \cdot \frac{x_{k-1}}{y_{k-1}} \cdot ((5-2k)+(k-2)\omega)}{(k-1) \cdot (3  - \omega +  \frac{x_{k-1}}{y_{k-1}})}\\
        &= \frac{x_{k-1} \cdot \frac{k}{k-1} \cdot ((5-2k)+(k-2)\omega)}{(3  - \omega) \cdot y_{k-1} +  x_{k-1}} = \frac{x_k}{y_k}
    \end{align*}
    as desired.
    
    \paragraph{Bound on $\gamma_k$.} Now, it suffices to show that for $t \geq n^{\gamma_k}$, for our choice of $x$,
    \begin{align}
        n^{k-3+\omega} &\leq \left(\frac{t}{nx}\right)^{3-\omega} n^{\omega(k-2) -2k + 6} = n^{\alpha_k} t^{1 - \frac{\alpha_{k}}{k}}\label{eq:first_bound}\\
        (n\cdot x) \cdot x^{\alpha_{k-1} + \gamma_{k-1}\left(1 - \frac{\alpha_{k-1}}{k-1}\right) - 1} &\leq  x^{\alpha_{k-1} \cdot \frac{k-2}{k-1}} (n\cdot x)^{\frac{\alpha_{k-1}}{k-1}} 
        t^{1- \frac{\alpha_{k-1}}{k-1}}\label{eq:second_bound}
    \end{align}
    The first inequality \eqref{eq:first_bound} is trivially satisfied since we chose $\gamma_k \geq k \left(1 - \frac{3-\omega}{k-\alpha_k}\right).$ 
    
    It suffices to show that the second inequality \eqref{eq:second_bound} is also satisfied. Rearranging, we see that it suffices to show that $x^{\gamma_{k-1}} \leq \frac{t}{n}.$
    Plugging in $x$ from \eqref{eq:x_solution_k1_runtime} and $\gamma_{k-1}$ and rearranging, we obtain that this holds as long as 
    \[ t \geq n^{\frac{(k-1) \cdot ((3 - \omega)^2 + k(\omega - 2)) + \alpha_{k-1} (k(2-\omega) + \omega - 3)}{11-(\omega - 2)\alpha_{k-1} + k(\omega - 2) - 7\omega + \omega^2}}.\]
    To show that all $t \geq n^{\gamma_k}$ satisfies the above inequality, it suffices to check that the exponent above is at most $\gamma_k$, i.e., it suffices to check that 
    \[\frac{(k-1) \cdot ((3 - \omega)^2 + k(\omega - 2)) + \alpha_{k-1} (k(2-\omega) + \omega - 3)}{(k-1-\alpha_{k-1})(\omega - 2) + (3 - \omega)^2}
    \leq \gamma_k = k \left(1 - \frac{3-\omega}{k - \alpha_k}\right).\]
    If $\omega = 2$, we can rewrite \eqref{eq:alpha_recursive} as $\alpha_k = \frac{k\alpha_{k-1}}{(k-1)(1 + \alpha_{k-1})}$ to obtain the following equivalent inequality:
    \begin{align*}
        k - (\alpha_{k-1} + 1) \leq k - \frac{(k-1)}{(k-1) + (k-2) \alpha_{k-1}} \cdot (\alpha_{k-1} + 1),
    \end{align*}
    which clearly holds since $(k-2)\alpha_{k-1} \geq 0.$
    
    When $\omega > 2$, we substitute our recursive formula for $\alpha_k$ from \eqref{eq:alpha_recursive} and rearrange to obtain that the inequality is satisfied for all $k > 0$ as long as 
    \begin{align}\label{eq:bound_alpha_for_gamma}
        (k-1)(\omega - 2) \leq \alpha_{k-1} \leq k-1 + \frac{(3-\omega)^2}{\omega - 2}.
    \end{align}
    
    \begin{claim}\label{claim:alpha_k_lb}
    For $2 \le \omega \leq 3$ and $k \geq 2$, we have $k(\omega - 2) \leq \alpha_k$.
    \end{claim}
    \begin{proof}
         We show this by induction. When $k = 2$, the equation is clearly true because $0 \leq \omega - 2 \leq 1$. 
         Therefore, $2 (\omega - 2) \leq 2,$ and the lower bound clearly holds.
    
    Now suppose $k \ge 3$ and that $\alpha_{k-1} \geq (k-1)(\omega - 2)$. Now, we want $\alpha_k \geq k (\omega - 2)$. Substituting the recursion from \eqref{eq:alpha_recursive}, we have
    \begin{align*}
        \frac{k \alpha_{k-1} ((5-2k) + (k-2)\omega)}{(k-1) \cdot (3 + \alpha_{k-1} - \omega)} \geq k (\omega - 2).
    \end{align*}
    Rearranging the equation using the fact that $3 - \omega + \alpha_{k-1}  > 0$, we have that the above equation holds if and only if
    \begin{align*}
        (3 - \omega) (\alpha_{k-1} - (k-1)(\omega - 2)) \geq 0.
    \end{align*}
    Since $\omega \leq 3$ and $\alpha_{k-1} \geq (k-1)(\omega - 2)$, we have that the equation indeed holds. 
    \end{proof}
   \begin{claim}
        For $2 \leq \omega \leq 3$ and $k \geq 2$, $\alpha_{k} \leq k.$
   \end{claim}
   \begin{proof}
       We proceed by induction. First, we note that $\alpha_2 = 2$ and $\alpha_3 = \frac{3(\omega - 1)}{5 - \omega} \leq 3$ since $\omega \leq 3$. 
        Now, suppose $\alpha_{k-1} \leq k-1$. Then, note that
        \begin{align*}
            \alpha_k &\leq \frac{k \alpha_{k-1} ((5-2k) + (k-2)\omega)}{(k-1) \cdot (3 + \alpha_{k-1} - \omega)} \leq \frac{k ((5-2k) + (k-2)\omega)}{3 + \alpha_{k-1} - \omega}. 
        \end{align*}
        Therefore, $\alpha_k \leq k$ as long as 
        \begin{align*}
            &(5-2k) + (k-2)\omega \leq 3 + \alpha_{k-1} - \omega\\
            \iff & \alpha_{k-1} \geq (k-1)(\omega - 2),
        \end{align*}
        which is indeed true by Claim~\ref{claim:alpha_k_lb}.
   \end{proof}
   Therefore, the bounds in \eqref{eq:bound_alpha_for_gamma} indeed hold, thereby completing the proof.
\end{proof}

Using the bound of Theorem~\ref{thm:k_1_optimal} and note that the runtime of $\cliquelist{k, \ell}$ is monotone with respect to $t$ (Lemma~\ref{lem:list_exponent_monotone}), we immediately get the following corollaries. 
\begin{corollary}[Theorem~\ref{thm:4_1_opt}]\label{cor:4-1-opt}
    Given a graph on $n$ nodes, one can list $t$ 4-cliques in $$\tilde{O}\left(n^{\omega + 1} + n^{\frac{4(\omega - 1)(2\omega - 3)}{\omega^2 - 5 \omega + 12}}t^{1 - \frac{(\omega - 1)(2\omega - 3)}{\omega^2 - 5 \omega + 12}}\right)$$ time. If $\omega = 2$, the runtime is $\tilde{O}(n^3 + n^{2/3}t^{5/6}).$
\end{corollary}

\begin{corollary}[Theorem~\ref{thm:5_1_opt}]\label{cor:5_1_opt}
     Given a graph on $n$ nodes, one can list $t$ 5-cliques in $$\tilde{O}\left(n^{\omega + 2} + n^{\frac{5(\omega - 1)(2\omega - 3)(3\omega - 5)}{48-47\omega + 16\omega^2 - \omega^3}}t^{1 - \frac{(\omega - 1)(2\omega - 3)(3\omega - 5)}{48-47\omega + 16\omega^2 - \omega^3}}\right)$$
    time. If $\omega = 2$, the runtime is $\tilde{O}(n^4 +n^{1/2}t^{9/10}).$
\end{corollary}
    
\subsection{Analysis for \texorpdfstring{$\cliquelist{k, \ell}$}{(k, l)-Clique-Listing} for \texorpdfstring{$\ell \geq 2$}{l >= 2}}\label{sec:k_l_opt}
We have shown an algorithm for $\cliquelist{k, 1}$ that is conditionally optimal when $\Delta_k \geq n^{\gamma_k}$. Now, we use this to show that there exists a $\cliquelist{k, \ell}$ algorithm for all $\ell$ that is conditionally optimal for $\Delta_k \geq n^{\gamma_{k, \ell}}$, for some $0 \leq \gamma_{k, \ell} < \frac{k}{\ell}.$ First, we define the following variable 
\[z_{k,\ell} = x_k \sum_{i=0}^{\ell-1}  \frac{k-\ell}{k-i-1} \cdot  \frac{y_{k-i}}{x_{k-i}}\]

where $x_k$ and $y_k$ are just as defined in \eqref{eq:alpha_num} and \eqref{eq:alpha_denom}. From this definition, the following identity is immediate.
\begin{claim}\label{claim:zkl_recursive}
For $\ell \ge 2$ and $k > \ell$, 
$z_{k,\ell} =\frac{x_k}{x_{k-1}} z_{k-1,\ell-1} +  \frac{k-\ell}{k-1}y_k$. 
\end{claim}

\begin{theorem}
\label{thm:k-l-large-t-listing}
    Fix any constant integers $k-1 \ge \ell \ge 1$.
    Let $\alpha_{k,\ell} = x_k/z_{k,\ell}$. Then, there exists
    some $\gamma_{k, \ell} = (1 - \epsilon_{k,\ell})k/\ell$ for
    $\epsilon_{k,\ell} > 0$ such that for large $t \geq n^{\gamma_k}$ 
    there exists an algorithm that lists all $t$ $k$-cliques given the $\ell$-cliques in time $\tO(\Delta_\ell^{\alpha_{k,\ell}}t^{1-\frac{\ell \alpha_{k,\ell}}{k}})$. 
    
    If $\omega = 2$, we have $x_k = k$ and $z_{k, \ell} = \frac{k\ell(k-\ell)}{2}$, giving a runtime of $\tO(\Delta_\ell^{\frac{2}{\ell(k-\ell)}}t^{1 - \frac{2}{k(k-\ell)}})$ for all $t \geq n^{\gamma_{k, \ell}}$, where
    \[\gamma_{k, \ell} = \frac{k(k^2 - 2k - 1)}{\ell(k^2 - k - \ell - 1)}.
    \]
\end{theorem}

\begin{proof}
    We show this inductively on $\ell$. For $\ell = 1$, we have $z_{k, 1} = y_k$, and this simply reduces to Theorem~\ref{thm:k_1_optimal}.
    
    For some $\ell > 1$, suppose that the theorem statement is true for all $\ell' < \ell$ and $k' > \ell'$. In particular, we assume that $\cliquelist{k',\ell'}$ takes time \[\tO(\Delta_{\ell'}^{\alpha_{k',\ell'}}\Delta_{k'}^{1 - \frac{\ell' \alpha_{k', \ell'}}{k'}})\]
    for $\Delta_{k'} \geq \Delta_{\ell'}^{\gamma_{k', \ell'}},$ and that the runtime is 
    \[\tO\left(\Delta_{\ell'}^{\alpha_{k',\ell'} + \gamma_{k',\ell'} \left(1 - \frac{\ell' \alpha_{k', \ell'}}{k'}\right)}\right)\]
    for $\Delta_{k'} \leq \Delta_{\ell'}^{\gamma_{k', \ell'}}.$
    We may assume this because the runtime is non-decreasing in $\Delta_{k'}$ by Lemma~\ref{lem:list_exponent_monotone}. 

    Recall that, 
    at a high level, Algorithm~\ref{alg:optimal_kl} first lists all $k$-cliques containing nodes that are contained in at most $x$ $\ell$-cliques, and deletes all such nodes. Now, there are at most $k\Delta_\ell/x$ nodes left in the graph, and we call the $\dense$ algorithm from Algorithm~\ref{alg:large_t_sparse_dense}.
    
    \paragraph{Runtime analysis.} Fix any $k > \ell$. In Step~\ref{step:kl_opt_lightnodes} of the algorithm, the runtime is given by (omitting $\tO(1)$ factors):
    \begin{align}
        &\sum_{v: \Delta_\ell(v) \leq x}
        \left(\Delta_\ell(v)^{\alpha_{k-1,\ell-1} + \gamma_{k-1,\ell-1} \left(1 - \frac{(\ell-1)\alpha_{k-1,\ell-1}}{k-1}\right)} + \Delta_\ell(v)^{\alpha_{k-1,\ell-1}} \Delta_k(v)^{1 - \frac{(\ell-1)\alpha_{k-1,\ell-1}}{k-1}}
        \right)\nonumber\\
        &\leq \Delta_\ell x^{\alpha_{k-1,\ell-1} + \gamma_{k-1,\ell-1} \left(1 - \frac{(\ell-1)\alpha_{k-1,\ell-1}}{k-1}\right) - 1} + \Delta_\ell^{\frac{(\ell-1)\alpha_{k-1,\ell-1}}{k-1}} t^{1 - \frac{(\ell-1)\alpha_{k-1,\ell-1}}{k-1}} x^{\frac{(k-\ell)\alpha_{k-1,\ell-1}}{k-1}},\label{eq:k_l_holders}   
    \end{align}
    where we use H\"{o}lder's inequality to bound the second term. Suppose for now that $t$ is large enough so that the second term dominates. 
    
    In Step~\ref{step:kl_densecall}, by Theorem~\ref{thm:k_1_optimal}, the runtime can be bounded by $(k\Delta_\ell/x)^{\alpha_k} t^{1 - \frac{\alpha_k}{k}}$  up to $\tilde{O}(1)$ factors, if we have
    \begin{equation}\label{eq:k_l_secondbound}
        t \geq (\Delta_\ell/x)^{\gamma_k}.
    \end{equation}
    
    Suppose that $t$ is large enough so that this inequality holds. Then, the runtime of the algorithm is
    \begin{align*}
        \Delta_\ell^{\frac{(\ell-1)\alpha_{k-1,\ell-1}}{k-1}} t^{1 - \frac{(\ell-1)\alpha_{k-1,\ell-1}}{k-1}} x^{\frac{(k-\ell)\alpha_{k-1,\ell-1}}{k-1}} +  \left(\frac{\Delta_\ell}{x}\right)^{\alpha_k} t^{1 - \frac{\alpha_k}{k}}.
    \end{align*}
    Choosing \begin{equation}\label{eqn:x_k_l_listing}
    x = \Delta_\ell^{\frac{(k-1) \alpha_k - (\ell-1)  \alpha_{k-1,\ell-1} }{(k-1)\alpha_k + (k-\ell)\alpha_{k-1,\ell-1}}} 
    t^{\frac{k(\ell-1) \alpha_{k-1,\ell-1} - (k-1) \alpha_k}{k((k-1)\alpha_k + (k-\ell)\alpha_{k-1,\ell-1})}},\end{equation}
    we get a runtime of (omitting $\tO(1)$ factors)
    \[
        \Delta_\ell^{\frac{\alpha_k \alpha_{k-1,\ell-1} (k-1)}{\alpha_k(k-1) + \alpha_{k-1,\ell-1} (k-\ell))}} t^{1 - \frac{\ell}{k} \cdot \frac{\alpha_k \alpha_{k-1,\ell-1} (k-1)}{\alpha_k(k-1) + \alpha_{k-1,\ell-1} (k-\ell))}},
    \]
    therefore giving 
    \begin{align*}
        \alpha_{k,\ell} &= \frac{\alpha_k \alpha_{k-1,\ell-1} (k-1)}{\alpha_k(k-1) + \alpha_{k-1,\ell-1} (k-\ell)}\\
        &= \frac{\frac{x_k}{y_k}\cdot \frac{x_{k-1}}{z_{k-1,\ell-1}}}{\frac{x_k}{y_k} + \frac{k-\ell}{k-1} \cdot \frac{x_{k-1}}{z_{k-1,\ell-1}}}\\
        &= \frac{x_k}{\frac{x_k}{x_{k-1}} z_{k-1,\ell-1} + \frac{k-\ell}{k-1}\cdot y_k} = \frac{x_k}{z_{k,\ell}}
    \end{align*}
    by Claim~\ref{claim:zkl_recursive}.
    
    \paragraph{Bound on $\gamma_{k, \ell}$ if $\omega = 2$.} 
    If $\ell = 1$, then $\gamma_{k, 1}$ matches the value of $\gamma_k$ we obtained from Theorem~\ref{thm:k_1_optimal}. Thus, we assume $\ell > 1$ and the bound for all $\ell' < \ell$ holds. 
    Recall that when $\omega = 2$, $\alpha_k = \frac{2}{k-1}$ and $\alpha_{k-1, \ell-1} = \frac{2}{(\ell-1)(k-\ell)}$. Therefore, substituting this into \eqref{eqn:x_k_l_listing}, we obtain
    \[
        x = \Delta_\ell^{\frac{(k-\ell-1)(\ell - 1)}{\ell(k-\ell)}} t^{\frac{\ell - 1}{k(k-\ell)}}.
    \]
    First, we check that \eqref{eq:k_l_secondbound} holds. In fact,
    \begin{align*}
        &t \geq \left(\frac{\Delta_\ell}{x}\right)^{\gamma_k} = \left(\Delta_\ell \cdot \Delta_\ell^{-\frac{(k-\ell-1)(\ell - 1)}{\ell(k-\ell)}} t^{-\frac{\ell - 1}{k(k-\ell)}}\right)^
        {\gamma_k}\\
        \iff & t^{1 + \frac{\gamma_k(\ell - 1)}{k(k-\ell)}} \geq \Delta_\ell^{\frac{(k-1)\gamma_k}{\ell(k-\ell)}}\\
        \iff & t \geq \Delta_\ell^{\frac{k(k-1)\gamma_k}{k\ell(k- \ell) + \ell (\ell - 1)\gamma_k}}.
    \end{align*}
    Substituting $\gamma_k = k - 1 - \frac{2}{k^2 - k - 2}$ from Theorem~\ref{thm:k_1_optimal}, we get the inequality
    $t \geq \Delta_\ell^{ \frac{k(k^2 - 2k - 1)}{\ell(k^2 - k - \ell - 1)}} = \Delta_\ell^{\gamma_{k, \ell}}$, which is indeed true by our choice of $\gamma_{k, \ell}$. 
    
    Now, it suffices to show that if $t \geq \Delta_\ell^{\gamma_{k ,\ell}}$, then the second term dominates in \eqref{eq:k_l_holders}. In fact, the second term dominates as long as 
    \begin{align*}
        &t \geq \Delta_\ell x^{\gamma_{k-1,\ell-1} - 1}\\
        \iff &t \geq \Delta_\ell^\frac{1+\frac{(k-\ell-1)(\ell - 1)}{\ell(k-\ell)}(\gamma_{k-1,\ell-1}-1)}{1 - \frac{\ell - 1}{k(k-\ell)}(\gamma_{k-1,\ell-1}-1)} = \Delta_\ell^{\frac{k(k-1)(k-3)}{\ell(k^2 - 3k + 3 - \ell)}},
    \end{align*}
    where the last equality holds because $\gamma_{k-1,\ell-1} = \frac{(k-1)((k-1)^2 - 2(k-1)-1)}{(\ell-1) ((k-1)^2 - k - \ell + 1)}$
    by induction. 
    Hence, it suffices to show that $\gamma_{k,\ell}$ is at least the exponent on the right-hand side.
    \begin{align*}
        &\gamma_{k, \ell} = \frac{k(k^2 - 2k - 1)}{\ell(k^2 - k - \ell - 1)} \geq \frac{k(k-1)(k-3)}{\ell(k^2 - 3k + 3 - \ell)}\\
        \iff & 1 - \frac{k-\ell}{k^2-k-1-\ell} \geq 1 - \frac{k-\ell}{k^2 - 3k - \ell + 3}\\
        \iff & k^2 - k - 1 - \ell \geq k^2 - 3k - \ell + 3\\
        \iff & k \geq 2,
    \end{align*}
    which is true since $k \geq 3.$
    
    \paragraph{Bound on $\gamma_{k, \ell}$ if $\omega > 2$.} In this case, we show that there exists some $\epsilon_{k,\ell} > 0$ such that $\gamma_{k,\ell} \leq \frac{k}{\ell}(1 - \epsilon_{k, \ell})$.

    
    First, consider \eqref{eq:k_l_secondbound}. Note that one can rewrite 
    \begin{align*}
       x = \Delta_\ell^{1 - \frac{\alpha_{k,\ell}}{\alpha_k}} t^{\frac{\ell \alpha_{k,\ell}}{k\alpha_k} - \frac{1}{k}}.
    \end{align*}
    Rewriting $q = \frac{\alpha_{k, \ell}}{\alpha_k}$, and substituting this into \eqref{eq:k_l_secondbound}, we obtain
    \begin{align*}
        t \geq \left(\frac{\Delta_\ell}{x}\right)^{\gamma_k} = \left(\frac{\Delta_\ell^q}{t^{\frac{\ell q}{k} - \frac{1}{k}}}\right)^{\gamma_k} \iff t \geq \Delta_\ell^{\frac{kq \gamma_k}{\ell q \gamma_k + k - \gamma_k}}.
    \end{align*}
By choosing $\epsilon_1 = \frac{k - \gamma_k}{\ell q \gamma_k + k - \gamma_k}$ (which is positive since $k > \gamma_k$ by Theorem~\ref{thm:k-l-large-t-listing}, it is easy to check that the right-hand side is equal to $\Delta_\ell^{\frac{k}{\ell}(1 - \epsilon_1)}$.

Now, consider \eqref{eq:k_l_holders}. For the second term to dominate, we can rewrite the inequality as 
\begin{align*}
    t \geq \Delta_\ell \cdot x^{\gamma_{k-1,\ell-1} - 1} = \Delta_\ell^{1 + (\gamma_{k-1,\ell-1} - 1)(1-q)} t^{\frac{1}{k} \cdot (\gamma_{k-1,\ell-1} - 1)(\ell q - 1)}.
\end{align*}
Rearranging this, we see that we require
\begin{align*}
    t \geq \Delta_\ell^{\frac{1 + (\gamma_{k-1,\ell-1} - 1) (1 - q)}{1 - \frac{1}{k}(\gamma_{k-1,\ell-1} - 1)(\ell q - 1)}}.
\end{align*}
By the induction hypothesis, we know there exists some $0 < \epsilon' < 1$ such that $\gamma_{k-1,\ell-1} = \frac{k-1}{\ell - 1}(1 - \epsilon').$ Therefore, substituting this into the above equation and rearranging, we require
\begin{align}\label{eq:gamma_kl_holderbound}
    t \geq \Delta_\ell^{\frac{k}{\ell}\left(1 - \frac{(k-1)(\ell - 1)\epsilon'}{(q\ell - 1)(k-1) \epsilon' + \ell((k-1) - q(k-\ell))}\right)}.
\end{align}
Let $\epsilon_{num} = (k-1)(\ell-1)\epsilon'$ and and $\epsilon_{den} = (q\ell - 1)(k-1) \epsilon' + \ell((k-1) - q(k-\ell))$. Clearly, since $k \geq 3$ and $\ell \geq 2$, $\epsilon_{num} > 0$. Now, consider two cases.
\begin{itemize}
    \item $q\ell - 1 \geq 0$. Then, since $\epsilon' > 0$, we have $\epsilon_{den} \geq \ell((k-1) - q(k-\ell)) > 0$ since $q = \frac{\alpha_{k,\ell}}{\alpha_k} = \frac{y_k}{z_{k,\ell}} < \frac{k-1}{k-\ell}$ by Claim~\ref{claim:zkl_recursive}.
    \item $q\ell - 1 < 0$. Then, since $\epsilon' < 1,$ $\ell \geq 2,$ $q > 0$ and $k \geq 3$
    \begin{align*}
        \epsilon_{den} &> (q\ell - 1)(k-1)  + \ell((k-1) - q(k-\ell))\\
        & = q\ell (\ell - 1) + (\ell-1)(k-1) > 0.
    \end{align*}
\end{itemize}

    Therefore, let $\epsilon_2 = \frac{\epsilon_{num}}{\epsilon_{den}}$. Clearly, $\epsilon_2 > 0$. Then, if $t \geq \Delta_\ell^{\frac{k}{\ell}(1 - \epsilon_2)}$, then \eqref{eq:gamma_kl_holderbound} holds. Hence, we can pick $\epsilon_{k, \ell} = \min(\epsilon_1, \epsilon_2) > 0$ to ensure both conditions \eqref{eq:k_l_holders} and \eqref{eq:k_l_secondbound} hold.
\end{proof}


\section{Extending the Algorithm to Graphs with Fewer \texorpdfstring{$k$}{k}-Cliques}
\label{sec:general-list}
\input{general_list}

\section{6-Clique Madness}
\label{sec:6clique}
\input{6-clique}

\bibliographystyle{alpha}
\bibliography{ref}

\end{document}