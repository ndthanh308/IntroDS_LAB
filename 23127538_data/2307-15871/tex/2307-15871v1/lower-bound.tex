
In this section, we will show our conditional lower bound for $\cliquelist{k,\ell}$ under the Exact-$k$-Clique hypothesis.

\begin{theorem}[Theorem~\ref{thm:lb_intro}]\label{thm:lower_bound}
For any $k \ge 3, 1 \le \ell < k$, and $\gamma \in [0, k/\ell]$, $\cliquelist{k, \ell}$ for instances with $t = \tilde{\Theta}(\Delta_\ell^\gamma)$
requires $$\left(\Delta_\ell^{\frac{2}{\ell(k-\ell)}} t^{1 - \frac{2}{k(k-\ell)}}\right)^{1-o(1)}$$
    time, where $\Delta_\ell$ is the number of $\ell$-cliques and $t$ is the number of $k$-cliques required to list, assuming Hypothesis~\ref{hyp:exact_k_clique}.
\end{theorem}
    
\begin{proof}
First, we can assume 
$\frac{2}{\ell(k-\ell)} + \gamma(1 - \frac{2}{k(k-\ell)}) > 1$, as otherwise the lower bound is trivial. 

    Let $G=(V = V_1 \sqcup \cdots \sqcup V_k, E, w)$ be a $k$-partite Exact-$k$-Clique instance on $k \cdot n$ nodes. Without loss of generality, we assume the edge weights $w$ of $G$ are from $\FF_p$ for some sufficiently large prime $p = n^{O(k)}=n^{O(1)}$. Then we sample $x \sim \FF_p$ uniformly at random. For every $i \in [k]$, and every node $v \in V_i$, we sample $k-1$ random variables $(y_{v, j})_{j \in [k] \setminus\{i\}} \sim \FF_p$ where $\sum_{j \in [k] \setminus\{i\}} y_{v, j} = 0$ uniformly at random. Note that $(y_{v, j})_{j \in [k] \setminus\{i\}}$ are $(k-2)$-wise independent. 
    
    For every $1 \le i < j \le k$, and $(v_i, v_j) \in V_i \times V_j$, let $$w'(v_i, v_j) = x \cdot w(v_i, v_j) + y_{v_i, j} + y_{v_j, i}.$$ It is not difficult to verify that, whenever $x \ne 0$, the sets of exact-$k$-cliques in the graph with weight $w$ and with weight $w'$ are the same. 
    
    Then we partition $\FF_p$ into $s$ contiguous intervals, each of length $O(p/s)$, for some $s$ to be chosen later where $\Omega(1) \le s \le O(n^{\frac{2}{k-1}})$. Consider all combinations of intervals $(L_{i, j})_{1 \le i < j \le k}$, where $0 \in \sum_{1 \le i < j \le k} L_{i, j}$. If we fix an arbitrary choice of the first $\binom{k}{2} - 1$ intervals, their sumset is an interval of length $O(p/s)$. Thus, there is only $O(1)$ choices for the last interval in order for their sumset to contain $0$. Hence, there are only $O(s^{\binom{k}{2} - 1})$ such combinations. For each such combination, we construct an instance of $\cliquelist{k, \ell}$  as follows: create an unweighted graph $H$ such that an edge $(v_i, v_j) \in V_i \times V_j$ for $i < j$ in $G$ is added to $H$ if and only if $w'(v_i, v_j) \in L_{i, j}$. The high level idea then is to list a certain number of $k$-cliques in $H$ and verify whether any of them is an exact-$k$-clique in $G$. Clearly, this algorithm never finds an exact-$k$-clique if $G$ does not have one, so it suffices to show that when $G$ does have an exact-$k$-clique, the algorithm finds it with decent probability. 
    
    Let $(u_1, \ldots, u_k)$ be an arbitrary exact-$k$-clique in $G$. Clearly, there exists one combination of intervals such that all edges in this exact-$k$-clique are in the corresponding subgraph $H_0$. For any $i, j$ and edge $(v_i, v_j) \in V_i \times V_j$, the edge is in $H_0$ only if $w'(u_i, u_j) - w'(v_i, v_j) \in [-O(p/s), O(p/s)]$, which happens with probability $O(1/s)$ as long as $\{u_i, u_j\} \ne \{v_i, v_j\}$. The following lemma shows that the random variables $w'(u_i, u_j) - w'(v_i, v_j)$ are fairly independent. 
    
    \begin{lemma}
    \label{lem:independent}
    If $(v_1, \ldots, v_k)$ is not an exact-$k$-clique w.r.t. $w$, and $(u_1, \ldots, u_k)$ shares exactly $c$ nodes indexed by $S$ with $(v_1, \ldots, v_k)$, then the random variables $$\left\{w'(u_i,u_j) - w'(v_i, v_j)\right\}_{\substack{1 \le i < j \le k \\ i \not \in S \text{ or } j \not \in S}}$$ are independent.     
    \end{lemma}
    \begin{proof}
    By symmetry, we can assume $|S| = [c]$, and we need to show that 
        $$\left\{w'(u_i,u_j) - w'(v_i, v_j)\right\}_{\substack{c+1 \le j \le k \\ 1 \le i < j}} = \left\{ x\cdot (w(u_i,u_j) - w(v_i, v_j)) + y_{u_i, j} + y_{u_j, i} - y_{v_i, j} - y_{v_j, i}\right\}_{\substack{c+1 \le j \le k \\ 1 \le i < j}}$$
        are independent. 
        
        Define $\alpha_{i, j} = w'(u_i,u_j) - w'(v_i, v_j)$. Let $\beta$ be the sum of all the $\alpha_{i, j}$:
        \begin{align*}
            \beta &=\sum_{\substack{c+1 \le j \le k \\ 1 \le i < j}} \alpha_{i, j} 
            = \sum_{1 \le i < j \le k} \alpha_{i, j}\\
            &= x \cdot \left(\sum_{1 \le i < j \le k} \left( w(u_i, u_j) - w(v_i, v_j)\right)\right) + \sum_{i=1}^k \sum_{j \in [k] \setminus \{i\}} y_{u_i, j} - \sum_{i=1}^k \sum_{j \in [k] \setminus \{i\}} y_{v_i, j}\\
            &= x \cdot \left(\sum_{1 \le i < j \le k} \left( w(u_i, u_j) - w(v_i, v_j)\right)\right).
        \end{align*}
        Since $(u_1, \ldots, u_k)$ is an exact-$k$-clique whereas $(v_1, \ldots, v_k)$ is not, we have $\sum_{1 \le i < j \le k} \left( w(u_i, u_j) - w(v_i, v_j)\right) \ne 0$. Therefore, $\beta$ is uniformly random. 

        Showing $\left\{\alpha_{i, j}\right\}_{\substack{c+1 \le j \le k \\ 1 \le i < j}}$ are independent is equivalent to showing that the variables are independent when one of the variables is replaced with the sums of the variables. Namely, it suffices to show $\left\{\alpha_{i, j}\right\}_{\substack{c+1 \le j \le k \\ 1 \le i < j \\ (i, j) \ne (k-1, k)}} \cup \{\beta\}$ are independent. 
        
        Consider the following ordering of the variables:
        $$\beta, \alpha_{c+1, 1}, \ldots, \alpha_{c+1, c}, \alpha_{c+2, 1}, \ldots, \alpha_{c+2, c+1}, \ldots,
        \alpha_{k-1, 1}, \ldots, \alpha_{k-1, k-2}, 
        \alpha_{k, 1}, \ldots, \alpha_{k, k- 2}.$$
        Conditioned on the previous variables, all $\alpha_{j, i}$ variables in this list has an additive term $y_{u_j, i}$ that is independent of all previous variables. Thus, this list of variables is independent. 
    \end{proof}
    \begin{corollary}
    \label{cor:independent}
    For any $\ell$-clique on nodes $(v_i)_{i \in T}$ in $G$ that shares exactly $c$ nodes indexed by $S$ with $(u_1, \ldots, u_k)$, the random variables 
    $$\left\{w'(u_i,u_j) - w'(v_i, v_j)\right\}_{\substack{i, j \in T\\
    i < j\\ i \not \in S \text{ or } j \not \in S}}$$
    are independent. 
    \end{corollary}
    \begin{proof}
        By symmetry, we can assume $[T] =[\ell]$ and $[S] = [c]$. We can complete this $\ell$-clique to a nonzero $k$-clique $(v_1, \ldots, v_k)$ (we can  assume any $\ell$-clique is in some nonzero $k$-clique by adding hypothetical nodes to the graph in this analysis). 
        
        By Lemma~\ref{lem:independent}, $\left\{w'(u_i,u_j) - w'(v_i, v_j)\right\}_{\substack{c+1 \le j \le k \\ 1 \le i < j}}$ are independent, so 
        $\left\{w'(u_i,u_j) - w'(v_i, v_j)\right\}_{\substack{c+1 \le j \le \ell \\ 1 \le i < j}} $ are also independent. 
    \end{proof}
    
    Now we can compute the expected number of $\ell$-cliques in $H_0$. 
    The number of $\ell$-cliques in $G$ that share exactly $c$ nodes with $(u_1, \ldots, u_k)$ is $O(n^{\ell-c})$. By Corollary~\ref{cor:independent}, each of them is in $H_0$ with probability $O\left(1/s^{\binom{\ell}{2} - \binom{c}{2}}\right)$. Therefore, the expected number of $\ell$-cliques in $H_0$ is 
    $$O\left(\sum_{c=0}^\ell n^{\ell-c} / s^{\binom{\ell}{2} - \binom{c}{2}}\right) = O\left(n^\ell / s^{\binom{\ell}{2}}\right),$$
    since by our choice of $s = O(n^{2/(k-1)})$, we have that $s^{\binom{c}{2}} = O(n^c)$.

    
    Similarly, the expected number of $k$-cliques in $H_0$ that do not correspond to exact-$k$-cliques in $G$ is $O\left(n^k / s^{\binom{k}{2}}\right)$. 
    
    Therefore, by Markov's inequality and union bound, with probability at least $1-1/\Omega(\log n)$, the number of $\ell$-cliques in $H_0$ is at most $n^\ell \log n/ s^{\binom{\ell}{2}}$ and the number of $k$-cliques in $H_0$ that do not correspond to exact-$k$-cliques in $G$ is at most $n^k \log n / s^{\binom{k}{2}}$. 
    
    Let $s = n^{\frac{k-\gamma \ell}{\binom{k}{2} - \gamma \binom{\ell}{2}}}$, so that $\frac{n^k}{s^{\binom{k}{2}}} = \left(\frac{n^\ell}{s^{\binom{\ell}{2}}}\right)^\gamma$. We can verify that indeed $\Omega(1) \le s \le O(n^{\frac{2}{k-1}})$. In fact, since $\frac{2}{\ell(k-\ell)} + \gamma(1 - \frac{2}{k(k-\ell)}) > 1$, we can obtain a stronger upper bound $s = O\left(n^{\frac{k-\ell}{\binom{k}{2}-\binom{\ell}{2}-1}-\delta}\right)$ for $\delta>0$.

    
    
    Suppose for the sake of contradiction that there is a $\cliquelist{k, \ell}$ algorithm $\mathcal{A}$ for instances with specified $t = \tilde{\Theta}(\Delta_\ell^\gamma)$  with running time $$T(\Delta_\ell, t) = O\left(\left(\Delta_\ell^{\frac{2}{\ell(k-\ell)}} t^{1 - \frac{2}{k(k-\ell)}}\right)^{1-\eps}\right)$$
    for some $\eps > 0$. Then consider the following algorithm for Exact-$k$-Clique:
    \begin{enumerate}
        \item 
        First, hash the weights of the graph and enumerate $O(s^{\binom{k}{2} - 1})$ graphs $H$ as described earlier.
        \item Enumerate all $\ell$-cliques in $G$, and pre-compute which graphs $H$ contain each $\ell$-clique. Since each $\ell$-clique exists in $s^{\binom{k}{2}-1 -\binom{\ell}{2}}$ graphs $H$, and this list of graphs can be listed efficiently, this step costs 
        $$\tO\left(n^\ell \cdot s^{\binom{k}{2}-1 -\binom{\ell}{2}}\right) \le \tO\left(n^\ell \cdot \left(n^{\frac{k-\ell}{\binom{k}{2}-\binom{\ell}{2}-1}-\delta}\right)^{\binom{k}{2}-1 -\binom{\ell}{2}}\right)\le \tO(n^{k-\delta'})$$
        for some $\delta' > 0$.
        
        \item \label{item:lower-bound-skip} From the previous step, we have a list of $\ell$-cliques for each graph $H$. If some $H$ contains more than $n^\ell \log n/ s^{\binom{\ell}{2}}$ $\ell$-cliques, we skip it. If it contains fewer than $0.99 n^\ell / s^{\binom{\ell}{2}}$ $\ell$-cliques, we add a complete $\ell$-partite graphs with $n'$ nodes on each part, for some $n'$, so that the total number of $\ell$-cliques in the new graph reaches $0.99(n+n')^\ell / s^{\binom{\ell}{2}}$. Clearly, $n' = O(n)$. 
        \item \label{item:lower-bound-listing} For graphs $H$ which we did not skip in the previous step, we run $\mathcal{A}$ on it with $t = n^k \log n / s^{\binom{k}{2}} + 1$. For any $k$-clique listed by $\mathcal{A}$, we test whether it is an exact-$k$-clique in $G$. This step takes $s^{\binom{k}{2}-1} \cdot T(\tO(n^\ell / s^{\binom{\ell}{2}}), \tO(n^k / s^{\binom{k}{2}}))$ time.
        \item If any exact-$k$-clique is found in the previous step, we return YES for the Exact-$k$-Clique instance; otherwise, we return NO. 
    \end{enumerate}
    
    Clearly, if $G$ contains no exact-$k$-clique, our algorithm is always correct. If $G$ contains any exact-$k$-clique, let $H_0$ be the constructed graph containing it. As discussed previously, with probability $1-1/\Omega(\log n)$, the number of $\ell$-cliques in $H_0$ is at most $n^\ell \log n/ s^{\binom{\ell}{2}}$ and the number of $k$-cliques in $H_0$ that do not correspond to exact-$k$-cliques in $G$ is at most $n^k \log n / s^{\binom{k}{2}}$. In this case, we will not skip $H_0$ in Step~\ref{item:lower-bound-skip}, and listing $t = n^k \log n / s^{\binom{k}{2}} + 1$ $k$-cliques in Step~\ref{item:lower-bound-listing} guarantees an exact-$k$-clique. Thus, we will find an exact-$k$-clique with probability $1-1/\Omega(\log n)$, which can be boosted to $1-1/\poly(n)$ by repeating the algorithm $O(\log n)$ times. 
    
    Overall, this algorithm only needs time (besides the previous $\tO(n^{k-\delta'})$ time)
    \begin{align*}
        &\tilde{O}\left(s^{\binom{k}{2}-1}\cdot \left(\left(n^\ell / s^{\binom{\ell}{2}}\right)^{\frac{2}{\ell(k-\ell)}} \left(n^k / s^{\binom{k}{2}}\right)^{1-\frac{2}{k(k-\ell)}} \right)^{1-\eps}\right) =  \tilde{O}\left(n^k \cdot \left(\frac{s^{\binom{k}{2}-1}}{n^k}\right)^\eps\right).
    \end{align*}
    As $s = O(n^{\frac{2}{k-1}})$, the above running time can be further upper bounded by 
    \begin{align*}
        \tilde{O}\left(n^k \cdot \left(\frac{(n^{\frac{2}{k-1}})^{\binom{k}{2}-1}}{n^k}\right)^\eps\right) = \tO\left(n^{k-\frac{2\eps}{k-1}}\right),
    \end{align*}
    contradicting the Exact-$k$-Clique hypothesis. 
\end{proof}

