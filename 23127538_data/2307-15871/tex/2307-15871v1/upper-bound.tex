In this section, we give a $\cliquelist{k, 1}$ algorithm that is  optimal for graphs with many $k$-cliques under Hypothesis~\ref{hyp:exact_k_clique}. This algorithm can be seen as a generalization of the densifying and sparsifying paradigm of  \cite{bjorklund2014listing}. 

We then show how we can extend this algorithm to obtain the conditionally optimal algorithms for all $\cliquelist{k, \ell}$ for graphs with many $k$-cliques.

\subsection{Algorithm}

First, we describe the algorithm for $\cliquelist{k, 1}$ in Algorithm~\ref{alg:large_t_sparse_dense}. 

 \begin{breakablealgorithm}
        \caption{$\cliquelist{k, 1}$ Algorithm for large $t \geq n^{\gamma_k}$, where $\gamma_k$ is defined in Theorem~\ref{thm:k_1_optimal}}\label{alg:large_t_sparse_dense}
        
        
        \begin{algorithmic}
            \item \dense{}$(G:= (V, E), n, t)$:
            \begin{itemize}
            \item \textbf{Input:} Graph $G = (V, E)$ with $|V| \leq n$ and at most $t$ $k$-cliques.
            \item \textbf{Output:} List of $k$-cliques in $G$.
            \item \textbf{The Algorithm:}
        \begin{enumerate}
            \item If $n < k$, it returns no $k$-cliques.
            \item Choose a parameter $\lambda$. Let an edge be $\lambda$-light if it is in fewer than $\lambda$ $k$-cliques.
            \item Use the algorithm in Lemma~\ref{lem:simple_list_ub} to obtain a list $L$ of all $(k-2)$-cliques (there are at most $n^{k-2}$ such cliques).
            \item Initialize an empty list $T$.
            \item \label{step:matmul_sampling}Repeat the following $O(\lambda \log n)$ times:
            \begin{itemize}
                \item Sample a subset $L'$ of $L$ of size $|L|/\lambda$.
                \item Construct adjacency matrices $A$ and $\overline{A}$ where the rows are indexed by $V$ and columns are indexed by $L'$.
                \item Let $A[v, C] = 1$ if node $v$ is distinct from and adjacent to every node in the $(k-1)$-clique $C$, and set $A[v, C] = 0$ otherwise.
                \item Let $\overline{A}[v, C] = A[v, C] \cdot C$, i.e. column $C$ contains entries 0 or $C$. 
                \item Compute $B = A \cdot A^T$ and $\overline{B} = A \cdot \overline{A}^T$. This takes $O(\MM(n, |L'|, n))$ time.
                \item For every edge $(u, v) \in E$ that is $\lambda$-light, if $B[u, v] = 1$, add $(u, v, \overline{B}[u, v])$ to $T$.
            \end{itemize}
            \item Output $T$.
            \item Delete all $\lambda$-light edges from $E$ to obtain $E'$ (all $\lambda$-light edges are found in Step \ref{step:matmul_sampling} w.h.p.).
            \item Call {\tt Sparse}$(G' := (V, E'), \binom{k}{2}t/\lambda, t)$.
        \end{enumerate}
        \end{itemize}
    \item\sparse{}$(G:= (V, E), m, t)$:
       \begin{itemize}
       \item \textbf{Input:} Graph $G = (V, E)$ with $|E| \leq m$ and at most $t$ $k$-cliques.
       \item \textbf{Output:} List of $k$-cliques in $G$.
       \item \textbf{The Algorithm:}
        \begin{enumerate}
            \item If $m < \binom{k}{2}$, it returns no $k$-cliques.    
            \item Choose a parameter $x$.
            \item Find all nodes such that $\deg(v) \leq x$, and call the $\cliquelist{k-1,1}$ algorithm in the neighbourhoods of all such nodes with $n' = \deg(v)$.
            \item Delete all nodes in $V$ of degree less than $x$ to obtain set $V'$.
            \item Call {\tt Dense}$(G' := (V', E \cap (V' \times V')), 2m/x, t)$
        \end{enumerate}
    \end{itemize}
    \end{algorithmic}
    \end{breakablealgorithm}
    
 

In the $\dense$ algorithm, we use matrix multiplication to enumerate all $k$-cliques containing light edges, i.e. edges that are part of very few $k$-cliques. These edges are then removed to result in a  \emph{sparse} graph with only edges that are part of many $k$-cliques.

In the $\sparse$ algorithm, we enumerate all $k$-cliques containing low-degree nodes by recursively listing all $(k-1)$-cliques in their neighborhoods, and delete all such nodes. Deleting these nodes results in a \emph{dense} graph with only high degree nodes. While one could brute-force the $(k-1)$-cliques in the neighborhoods, our key insight is that we can instead recursively use a $\cliquelist{k-1, 1}$ algorithm to be more efficient. 

We first show the correctness of Algorithm~\ref{alg:large_t_sparse_dense} and defer its runtime analysis to Section~\ref{sec:k_1_opt}.

\paragraph{Correctness.} It is clear that the $\sparse$ algorithm finds all $k$-cliques in the neighborhoods of low-degree nodes. At the end of the algorithm, since the graph has $\Delta_\ell$ $\ell$-cliques and only nodes with degree at least $x$, there are at most $2m/x$ nodes left in the graph.

Now, we argue that the $\dense$ algorithm lists all $k$-cliques containing $\lambda$-light edges.

We argue that Step~\ref{step:matmul_sampling} finds all $\lambda$-light edges with high probability. For every $(u, v) \in E$, let $L_{u, v}$ denote the set of all $(k-2)$-cliques that form $k$-cliques with nodes $u$ and $v$. Since we sample $L'$ of size $|L|/\lambda$, the probability that $L_{u, v} \cap L' = K_{k-2}$ for any fixed $K_{k-2} \in L_{u, v}$ is \[\frac{1}{\lambda} \cdot \left(1 - \frac{1}{\lambda}\right)^{|L_{u, v}| - 1} \geq \frac{1}{\lambda} \cdot \left (1 - \frac{1}{\lambda}\right)^{\lambda-1} \geq \frac{1}{e \lambda}.\]
Therefore, by choosing $O(\lambda \log  n)$ random sets of size $|L|/\lambda$, with high probability, we find all $k$-cliques containing $\lambda$-light edges.

\paragraph{$\cliquelist{k, \ell}$ when $\ell \geq 2$.} To generalize this algorithm to $\cliquelist{k, \ell}$ for $\ell \geq 2$, we recursively use $\cliquelist{k-1, \ell-1}$ to reduce the problem to $\cliquelist{k, 1}.$ At a high level, the algorithm considers all nodes $v$ in fewer than $x$ $\ell$-cliques and recursively calls $\cliquelist{k-1,\ell-1}$ to list all $k$-cliques containing $v$. See Algorithm~\ref{alg:optimal_kl}. The correctness of Algorithm~\ref{alg:optimal_kl} can be shown as follows. 

\begin{algorithm}[ht]
    \caption{$\cliquelist{k, \ell}$ Algorithm for large $t \geq n^{\gamma_{k,\ell}}$, where $\gamma_{k, \ell}$ is defined in Theorem~\ref{thm:k-l-large-t-listing}}\label{alg:optimal_kl}
    \begin{algorithmic}
    \item \textbf{Input:} A graph $G$ and a list $L$ of all $\ell$-cliques.
    \item \textbf{Output:} All $k$-cliques in the graph.
    \item \textbf{The Algorithm:}
    \begin{enumerate}
        \item Call a node $v$ light if $\Delta_\ell(v) \leq x$, for some parameter $x$. 
        \item\label{step:kl_opt_lightnodes} For all light nodes, call $\cliquelist{k-1, \ell-1}$ in the neighbourhoods to find all $k$-cliques incident to $x$. 
        \item Delete all light nodes and incident edges from $G$. 
        \item\label{step:kl_densecall} Call the $\cliquelist{k, 1}$ algorithm $\dense(G' := (V', E'), \ell \Delta_\ell/x, t)$ (from Algorithm~\ref{alg:large_t_sparse_dense}).
    \end{enumerate}
    \end{algorithmic}
\end{algorithm}

\paragraph{Correctness.} It is clear that the algorithm lists all $k$-cliques incident to low-degree nodes. Since all remaining nodes are in at least $x$ $\ell$-cliques, and each $\ell$-cliques contains at most $\ell$ nodes, we can bound the remaining number of nodes by $\ell \Delta_\ell/x$.

To illustrate these algorithms, we first show simplified analyses of Algorithms~\ref{alg:large_t_sparse_dense} and \ref{alg:optimal_kl} for the case of $k = 4$ and $k=5$ assuming that $\omega = 2$ in Section~\ref{sec:4_5_l_listing}. We give more detailed analyses in terms of $\omega$ in Sections~\ref{sec:k_1_opt} and \ref{sec:k_l_opt}. 

\subsection{Analysis for \texorpdfstring{$k = 4$}{k = 4} and \texorpdfstring{$k = 5$}{k = 5} assuming \texorpdfstring{$\omega = 2$}{omega = 2}}\label{sec:4_5_l_listing}
In this section, we illustrate how to analyze the runtime for listing algorithm by considering the cases where $k = 4$ or $k=5$.
\begin{proposition}
    Suppose $\omega = 2$. Then, given a graph $G$ with $t$ 4-cliques,
    \begin{itemize}
        \item $\cliquelist{4,1}$ can be solved in $\tilde{O}(n^3 + n^{2/3}t^{5/6})$ if $G$ has $n$ nodes.
        \item $\cliquelist{4,2}$ can be solved in $\tilde{O}(m^{3/2} + mt^{2/5} + m^{1/2}t^{3/4})$ if $G$ has $m$ edges.
        \item $\cliquelist{4, 3}$ can be solved in $\tilde{O}(\Delta^{6/5} + \Delta t^{1/5} + \Delta^{2/3}t^{1/2})$ if $G$ has $\Delta = \Delta_3$ triangles.
     \end{itemize}  
\end{proposition}

\begin{proof}
    Consider the $\dense$ algorithm. In this case, $L$ is a list of all (up to $n^2$) edges. Therefore, the runtime of this step can be bounded by 
    \[D'(n, m, t) \leq n^2 + \lambda \log n \cdot \MM(n, m/\lambda, n) + S(6t/\lambda, t) = \tilde{O}(n^2 + \lambda n^2 + nm) + S(6t/\lambda).\]
    We can also upper bound $m$ by $n^2$ to obtain the following bound without a dependence on $n$:
\[
    D(n, t) \leq D(n, n^2, t) \leq \tilde{O}(n^3 + \lambda n^2) + S(6t/\lambda, t).
\]
assuming $\omega = 2$. 
    
Consider the $\sparse$ algorithm. In this case, we call $(3,1)$-listing, which takes time $\tO(n^2 + nt^{2/3})$. Therefore (ignoring $\tO(1)$ factors), 
\begin{align*}
    S(m, t) &\leq \sum_{v: \deg(v) \leq x} \left(\deg(v)^2 + \deg(v) \Delta_4(v)^{2/3}\right)+ D'(2m/x, m, t)\\
    &\leq  \sum_{v: \deg(v) \leq x} \left(\deg(v) \cdot x + \deg(v)^{1/3} \Delta_4(v)^{2/3} x^{2/3}\right)+ D'(2m/x, m, t)\\
    &\leq mx + m^{1/3}t^{2/3}x^{2/3} + D'(2m/x, m, t),
\end{align*}
where we applied H\"{o}lder's inequality as seen in Corollary~\ref{cor:holders_useful}. 

\paragraph{$\cliquelist{4, 1}$ analysis.} To obtain a runtime for $\cliquelist{4,1}$, we unravel the recursion in $D(n, t)$. Ignoring $\tilde{O}(1)$ factors in the following inequalities, we have
\begin{align*}
    D(n, t) &\leq n^3 + \lambda n^2 + S(6t/\lambda, t)\\
    &\leq n^3 + \lambda n^2 + \frac{tx}{\lambda} + \frac{tx^{2/3}}{\lambda^{1/3}} + D\left(\frac{12t}{\lambda x}, t\right)
\end{align*}
Choosing $\lambda = \max\{1, \frac{24t}{nx}\}$, we have $\frac{12t}{\lambda x} \leq n/2$, and the above runtime will be dominated by the first four terms up to $\tilde{O}(1)$ factors. Substituting this value of $\lambda$, we obtain a runtime of 
    \[D(n, t) \leq n^3 + \frac{tn}{x} + nx^2 + n^{1/3}t^{2/3}x.
    \]
Choosing \[x = 
    \begin{cases}
        n & t \leq n^{5/2}\\
        n^{8/3}/t^{2/3} & n^{5/2} \leq t \leq n^{14/5}\\
        n^{1/3}t^{1/6} & t \geq n^{14/5}
    \end{cases},\]
we obtain $D(n, t) = n^3 + n^{2/3}t^{5/6}$, as desired.

\paragraph{$\cliquelist{4, 2}$ analysis.} To obtain a runtime for $\cliquelist{4, 2}$, we analyze the runtime of $S(m, t)$.
Here, we use our bound $D'$ in terms of $n$, $m$ and $t$ to get a tighter analysis
(instead of just $n$ and $t$). 

\begin{align*}
    S(m, t) &\leq  mx + m^{1/3}t^{2/3}x^{2/3} + D'(2m/x,m, t)\\
    &\leq mx + m^{1/3}t^{2/3}x^{2/3} + \frac{\lambda m^2}{x^2} + \frac{m^2}{x} + S(6t/\lambda, t).
\end{align*}
Choosing $\lambda = \max\{1, \frac{12t}{m}\}$, we have that $6t/\lambda \leq m/2$. Therefore, the first four terms dominate up to $\tilde{O}(1)$ factors, so (ignoring $\tO(1)$ factors)
\begin{align*}
    S(m, t) &\leq mx + m^{1/3}t^{2/3}x^{2/3} + \frac{mt}{x^2} + \frac{m^2}{x}.
\end{align*}
By choosing 
\[
    x = \begin{cases}
        m^{1/2} &\text{if $t \leq m^{5/4}$}\\
        m/t^{2/5} &\text{if $m^{5/4} < t \leq m^{10/7}$}\\
        m^{1/4}t^{1/8} & \text{if $t > m^{10/7}$,}
    \end{cases}
\] 
we get a runtime of 
\[S(m, t) \leq m^{3/2} + mt^{2/5} + m^{1/2}t^{3/4}.\]

\paragraph{$\cliquelist{4,3}$ analysis.} Note that while we can use Algorithm~\ref{alg:optimal_kl} to bound the runtime in this case, we instead provide a more efficient algorithm shown in Algorithm~\ref{alg:4_3} for $\cliquelist{4, 3}$.

\begin{algorithm}[ht]
    \caption{$\cliquelist{4, 3}$ algorithm}\label{alg:4_3}
    \begin{algorithmic}
    \item \textbf{Input:} Graph $G = (V, E)$, and a list $L$ of all triangles in $G$.
    \item \textbf{Output:} A list of all $k$-cliques in $G$.
    \begin{enumerate}
        \item Call an edge light if it occurs in fewer than $x$ triangles, i.e. $\Delta(e) \leq x$. 
        \item\label{step:light_edges_in_triangles} For all light edges $e$, consider all pairs of nodes in its neighbourhoods to find all 4-cliques containing $e$.
        \item Delete all light edges from $G$.
        \item Call $\cliquelist{4, 2}$ algorithm $\sparse(G' := (V', E'), 3\Delta/x, t).$ 
    \end{enumerate}
    \end{algorithmic}
\end{algorithm}

The runtime of Step~\ref{step:light_edges_in_triangles} is bounded by $\sum_{e:\Delta(e) \le x} \Delta(e)^2 \leq \Delta x.$ Now, we call $\cliquelist{4, 2}$ with a graph with at most $3\Delta/x$ edges and $t$ $4$-cliques, giving a runtime of 
\[
    \left(\Delta/x\right)^{3/2} + \left(\Delta/x\right) t^{2/5} + \left(\Delta/x\right)^{1/2} t^{3/4}.
\]
Therefore, choosing $x = \max\{\Delta^{1/5}, t^{1/5}, t^{1/2}/\Delta^{1/3}\}$, we get a runtime of $\tilde{O}\left(\Delta^{6/5} + \Delta t^{1/5} + \Delta^{2/3} t^{1/2}\right).$ 

\end{proof}

\begin{proposition}
    Suppose $\omega = 2$. Then, given a graph $G$ with $t$ 5-cliques,
    \begin{itemize}
        \item $\cliquelist{5,1}$ can be solved in $\tilde{O}(n^4 + n^{1/2}t^{9/10})$ if $G$ has $n$ nodes.
        \item $\cliquelist{5,2}$ can be solved in $\tilde{O}(m^2 + m^{17/18}t^{10/18} + m^{1/3}t^{13/15})$ if $G$ has $m$ edges.
     \end{itemize}  
\end{proposition}

\begin{proof}
Consider the $\dense$ algorithm. In this case, $L$ is a list of all (up to $n^3$) triangles. Therefore, the runtime of this step is 
\[D'(n, \Delta_3, t) \leq n^3 + \lambda \log n \cdot \MM(n, \Delta_3/\lambda, n) + S(10t/\lambda, t) = \tilde{O}(n^3+n\Delta_3 + \lambda n^2) + S(10t/\lambda, t)\]
assuming $\omega = 2$.
Upper bounding $\Delta_3 \leq O(n^3)$, we get a bound without dependence on $\Delta_3$ of \[D(n, t) \leq D'(n, n^3, t) = \tilde{O}(n^4 + \lambda n^2) +S(10t/\lambda).\]
Consider the $\sparse$ algorithm. In this case, we call $\cliquelist{4, 1}$ in the neighborhoods of all low-degree nodes, so
\begin{align*}
    S(m, t) \leq \sum_{v: \deg(v) \leq x} \left(\deg(v)^3 + \deg(v)^{2/3} \Delta_5(v)^{5/6}\right) + D(2m/x, t) &\leq mx^2 + m^{1/6}t^{5/6}x^{1/2} +  D(2m/x, t)
\end{align*}
by using H\"{o}lder's inequality as in Corollary~\ref{cor:holders_useful}.

\paragraph{$\cliquelist{5, 1}$ analysis.} To obtain a runtime for $\cliquelist{5, 1}$, we analyze the runtime of $D(n, t) $ by unravelling the recursion. Therefore, we have the following inequalities (omitting $\tilde{O}(1)$ factors):
\begin{align*}
    D(n, t) &\leq n^4 + \lambda n^2 + \frac{tx^2}{\lambda} + \left(\frac{t}{\lambda}\right)^{1/6} t^{5/6}x^{1/2} + D\left(\frac{20t}{\lambda x}, t\right)
\end{align*}
By choosing $\lambda = \max\left\{5, \frac{40 t}{nx}\right\}$, we would have $\frac{20t}{\lambda x} \leq \frac{n}{2}$, and the running time will therefore be dominated by the first 4 terms. Choosing 
\begin{align*}
    x = 
    \begin{cases}
        n & \text{if } t \leq n^{19/15}\\
        n^{23/4}/t^{5/4} & \text{if } n^{19/5} \leq t \leq n^{35/9}\\
        n^{1/2}t^{1/10} & \text{if } t \geq n^{35/9}
    \end{cases},
\end{align*}
we obtain a runtime of $\tilde{O}(n^4 + n^{1/2}t^{9/10}).$

\paragraph{$\cliquelist{5, 2}$ analysis.} We now analyze the runtime of $S(m, t).$ Note that the graph has at most $\Delta_3 = O(m^{3/2})$ triangles. Here, we use $D'(n, \Delta_3, t)$ to bound the runtime instead.
Therefore, unrolling the recursion, we have (up to $\tilde{O}(1)$ factors) 
\begin{align*}
    S(m, t) &\leq mx + m^{1/6}t^{5/6}x^{1/2} + D'(m/x, m^{3/2}, t)\\
    &\leq mx + m^{1/6}t^{5/6}x^{1/2} + (m/x) \cdot m^{3/2} + (m/x)^{1/2}t^{9/10}.
\end{align*}
Setting 
\[
    x = \begin{cases}
        m^{1/2} & \text{if } t \leq m^{19/10}\\
        m^{17/18}t^{10/18} & \text{if }m^{19/10} \leq t \leq m^{55/28}\\
        m^{1/3}t^{13/15} & \text{if } t \geq m^{55/28}
    \end{cases},
\]
we get a runtime of $\tilde{O}(m^2  + m^{17/18}t^{10/18} + m^{1/3}t^{13/15}).$ 
\end{proof}

\subsection{Analysis for \texorpdfstring{$\cliquelist{k, 1}$}{(k,1)-Clique-Listing}}\label{sec:k_1_opt} 

For $k \geq 2$, define 
\begin{align}
    x_k &= k \prod_{j = 2}^k ((5 - 2j) + (j - 2)\omega)\label{eq:alpha_num}\\
    y_k &= (3-\omega)^{k-2} + \sum_{j = 2}^{k-1} (3-\omega)^{k-1-j} x_j
    \label{eq:alpha_denom}
\end{align}

The following identities are immediate. 
\begin{claim}
    For any $k \ge 3$, $x_k = x_{k-1} \cdot \frac{k}{k-1} \cdot ((5-2k)+(k-2)\omega)$ and $y_k = (3  - \omega) \cdot y_{k-1} +  x_{k-1}$.
\end{claim}

\begin{theorem}\label{thm:k_1_optimal}
    Let $\alpha_k = x_k/y_k$. For any $k \geq 2$ and large $t \geq n^{\gamma_k}$ where 
    \[\gamma_k = 
    \begin{cases}
    0 & \text{if } k = 2\\
    k \left(1 - \frac{3 - \omega}{k - \alpha_k}\right) & \text{if } k \geq 3
    \end{cases},\]  
    there exists an algorithm that lists all $t$ $k$-cliques in time $\tO(n^{\alpha_k}t^{1-\frac{\alpha_k}{k}})$. If $\omega=2$,  we have that $x_k = k$ and $y_k = \frac{k(k-1)}{2}$, therefore giving a runtime of $\tO(n^{\frac{2}{k-1}} t^{1 - \frac{2}{k(k-1)}})$ for $t \geq n^{k - 1 - \frac{2}{k^2 - k - 2}}.$
\end{theorem}

\begin{proof}
    For $k = 2$, the brute-force algorithm runs in $n^2$ time, and it is easy to check that $x_k=2$ and $y_k = 1$. Moreover, this bound holds for all values of $t$, so we can set $\gamma_k = 0.$
    
    For $k = 3$, \cite{bjorklund2014listing} give an algorithm that runs in time $O(n^\omega + n^{\frac{3(\omega - 1)}{5 - \omega}}t^{\frac{2(3-\omega)}{5 - \omega}})$, which can easily be verified to match the form of the theorem statement. Rewriting this as $O(n^\omega + n^{\alpha_3}t^{1- \frac{\alpha_3}{3}})$, it is easy to see that this term dominates exactly when $t \geq n^{3\left(1 - \frac{3-\omega}{3-\alpha_3}\right)}$, which corresponds exactly to our setting of $\gamma_3.$
    Now suppose  $k \geq 3$ and that the theorem statement is true for all $\cliquelist{r, 1}$ for all $r < k$.
    In particular, suppose the runtime of $\cliquelist{k-1, 1}$ is bounded by $$T_{k-1}(n, \Delta_{k-1}) \leq n^{\alpha_{k-1}}\Delta_{k-1}^{1 - \frac{\alpha_{k-1}}{k-1}},$$ when $\Delta_{k-1} \ge n^{\gamma_{k-1}}$ for some $\gamma_{k-1}$. In fact, since the runtime is non-decreasing in the parameter $\Delta_{k-1}$ by Lemma~\ref{lem:list_exponent_monotone}, one can bound the above runtime for any $\Delta_k$  by:
    \begin{align*}
        T_{k-1}(n, \Delta_{k-1}) \leq n^{\alpha_{k-1}}\left(n^{\gamma_{k-1}}\right)^{1 - \frac{\alpha_{k-1}}{k-1}} + n^{\alpha_{k-1}}\Delta_{k-1}^{1 - \frac{\alpha_{k-1}}{k-1}}.
    \end{align*}

    \paragraph{Runtime analysis.} Let $D(n, t)$ be the running time of $\dense(G, n, t)$, and let $S(m, t)$ be the running time of $\sparse(G, m, t)$. Note that ignoring $\tilde{O}(1)$ factors
    \begin{align*}
        D(n, t) &\le n^{k-1} + \lambda \MM(n, n^{k-2}/\lambda, n) + S\left(\binom{k}{2}t/\lambda, t\right)
    \end{align*}
    
    By the standard trick of decomposing a rectangular matrix product into smaller square matrix products, one can bound $$\MM(n, n^{k-2}/\lambda, n) \leq \left(\frac{n^{k-2}/\lambda}{n}\right) \cdot n^\omega + \left(\frac{n}{n^{k-2}/\lambda}\right)^2 \cdot \left(\frac{n^{k-2}}{\lambda}\right)^\omega = \frac{n^{k-3+\omega}}{\lambda} + \frac{n^{\omega(k-2) -2k + 6}}{\lambda^{\omega - 2}}.$$
    Therefore, we can rewrite 
    $$D(n, t) \leq n^{k-3+\omega} + \lambda^{3 - \omega} n^{\omega(k-2) -2k + 6} + S\left(\binom{k}{2}t/\lambda, t\right).$$

    For $\sparse(G, m, t)$, note that the runtime is bounded by:
    \begin{align*}
        S(m, t) &\leq \sum_{v : \deg(v) \leq x} T_{k-1}(\deg(v), \Delta_k(v)) + D(2m/x, t).\\
        &\leq \sum_{v : \deg(v) \leq x} \left(\deg(v)^{\alpha_{k-1} + \gamma_{k-1}\left(1 - \frac{\alpha_{k-1}}{k-1}\right)} + \deg(v)^{\alpha_{k-1}}\Delta_k(v)^{1 - \frac{\alpha_{k-1}}{k-1}}\right) + D(2m/x, t).
    \end{align*}
    
    One can use H\"{o}lder's inequality as in Corollary~\ref{cor:holders_useful} to bound 
    \begin{align*}
        \sum_{v: \deg(v) \leq x} \deg(v)^{\alpha_{k-1}} \Delta_k(v)^{1 - \frac{\alpha_{k-1}}{k-1}} 
        &\leq x^{\alpha_{k-1} - \frac{\alpha_{k-1}}{k-1}}\sum_{v: \deg(v) \leq x} \deg(v)^{\frac{\alpha_{k-1}}{k-1}}\Delta_k(v)^{1 - \frac{\alpha_{k-1}}{k-1}} \\
        &\leq 
        x^{\alpha_{k-1} \cdot \frac{k-2}{k-1}} \left(\sum_{v: \deg(v) \leq x} \deg(v)\right)^{\frac{\alpha_{k-1}}{k-1}} \left(\sum_{v: \deg(v) \leq x} \Delta_k(v) \right)^{1- \frac{\alpha_{k-1}}{k-1}}\\
        & \leq
        O\left(x^{\alpha_{k-1} \cdot \frac{k-2}{k-1}} m^{\frac{\alpha_{k-1}}{k-1}} 
        t^{1- \frac{\alpha_{k-1}}{k-1}}\right).
    \end{align*}
    
    Thus, we have that (once again ignoring $\tilde{O}(1)$ factors)
    $$S(m, t) \leq m \cdot x^{\alpha_{k-1} + \gamma_{k-1}\left(1 - \frac{\alpha_{k-1}}{k-1}\right) - 1} + x^{\alpha_{k-1} \cdot \frac{k-2}{k-1}} m^{\frac{\alpha_{k-1}}{k-1}} 
        t^{1- \frac{\alpha_{k-1}}{k-1}} + D(2m/x, t).$$
   
    Unravelling the runtime of $\dense(G, n, t)$, we therefore have
    \begin{align*}
        D(n, t) \leq & n^{k-3+\omega} + \lambda^{3-\omega}n^{\omega(k-2) -2k + 6}\\ & + (t/\lambda) \cdot x^{\alpha_{k-1} + \gamma_{k-1}\left(1 - \frac{\alpha_{k-1}}{k-1}\right) - 1} + x^{\alpha_{k-1} \cdot \frac{k-2}{k-1}} (t/\lambda)^{\frac{\alpha_{k-1}}{k-1}} 
        t^{1- \frac{\alpha_{k-1}}{k-1}} \\
        &+ D\left(\frac{2\cdot \binom{k}{2}t}{\lambda x}, t\right).
    \end{align*}
    If one chooses $\lambda$ and $x$ so that $\frac{2 \cdot \binom{k}{2} t}{\lambda x} \leq \frac{n}{2}$, then the runtime is dominated by the first 4 terms up to $\tilde{O}(1)$ factors. Therefore, we choose $\lambda = \max\{1, \frac{4 \cdot \binom{k}{2} t}{n x}\}$ (note that for $t \geq n^{\gamma_k}$, this value will always be equal to $\frac{4 \cdot \binom{k}{2} t}{n x}$ for our setting of $x$). Hence, ignoring $\tilde{O}(1)$ factors, this gives us for $t \geq n^{\gamma_k}$, 
    \begin{align*}
        D(n, t) &\leq n^{k-3+\omega} + \left(\frac{t}{nx}\right)^{3 - \omega}n^{\omega(k-2) -2k + 6} \\
        &+ (n\cdot x) \cdot x^{\alpha_{k-1} + \gamma_{k-1}\left(1 - \frac{\alpha_{k-1}}{k-1}\right) - 1} + x^{\alpha_{k-1} \cdot \frac{k-2}{k-1}} (n\cdot x)^{\frac{\alpha_{k-1}}{k-1}} 
        t^{1- \frac{\alpha_{k-1}}{k-1}}
    \end{align*}
    
    First, suppose that the term $\left(\frac{t}{nx}\right)^{3 - \omega}n^{\omega(k-2) -2k + 6}$
    dominates $n^{k-3+\omega}$ and the term $x^{\alpha_{k-1} \cdot \frac{k-2}{k-1}} (n \cdot x)^{\frac{\alpha_{k-1}}{k-1}} t^{1- \frac{\alpha_{k-1}}{k-1}}$ dominates $(n\cdot x) \cdot x^{\alpha_{k-1} + \gamma_{k-1}\left(1 - \frac{\alpha_{k-1}}{k-1}\right) - 1}$ (we show that this is in fact true for our choice of $\gamma_k$ later) . Then, 
    \begin{equation}\label{eq:bounding_dense}
        D(n, t) \leq \left(\frac{t}{nx}\right)^{3- \omega}n^{\omega(k-2) -2k + 6} + x^{\alpha_{k-1} \cdot \frac{k-2}{k-1}} (n \cdot x)^{\frac{\alpha_{k-1}}{k-1}}t^{1- \frac{\alpha_{k-1}}{k-1}}.
    \end{equation}
    Choosing $x$ to equate the two terms, we set 
    \begin{equation}\label{eq:x_solution_k1_runtime}
        x = \left({t^{\alpha_{k-1} - (\omega-2)(k-1)}n^{(k-1)^2 \omega - (\alpha_{k-1} + 2k^2 - 5k + 3)}}\right)^{\frac{1}{(k-1)(3+\alpha_{k-1} - \omega)}}.
    \end{equation}
    Substituting this into \eqref{eq:bounding_dense}, we have \[D(n, t) \leq n^{\frac{k\alpha_{k-1}((5-2k)+(k-2)\omega)}{(k-1) \cdot (3 + \alpha_{k-1} - \omega)}} t^{\frac{(3-\omega)((k-2)\alpha_{k-1} + (k-1))}{(k-1) \cdot (3 + \alpha_{k-1} - \omega)}}.
    \]
    Setting 
    \begin{equation}\label{eq:alpha_recursive}
    \alpha_k = \frac{k\alpha_{k-1}((5-2k)+(k-2)\omega)}{(k-1) \cdot (3 + \alpha_{k-1} - \omega)},
    \end{equation}
    it is easy to verify that the above bound is in fact of the form $n^{\alpha_k} t^{1-\frac{\alpha_k}{k}}.$ Moreover, note that
    \begin{align*}
        \alpha_k &= \frac{k\alpha_{k-1}((5-2k)+(k-2)\omega)}{(k-1) \cdot (3 + \alpha_{k-1} - \omega)}\\
        &= \frac{k \cdot \frac{x_{k-1}}{y_{k-1}} \cdot ((5-2k)+(k-2)\omega)}{(k-1) \cdot (3  - \omega +  \frac{x_{k-1}}{y_{k-1}})}\\
        &= \frac{x_{k-1} \cdot \frac{k}{k-1} \cdot ((5-2k)+(k-2)\omega)}{(3  - \omega) \cdot y_{k-1} +  x_{k-1}} = \frac{x_k}{y_k}
    \end{align*}
    as desired.
    
    \paragraph{Bound on $\gamma_k$.} Now, it suffices to show that for $t \geq n^{\gamma_k}$, for our choice of $x$,
    \begin{align}
        n^{k-3+\omega} &\leq \left(\frac{t}{nx}\right)^{3-\omega} n^{\omega(k-2) -2k + 6} = n^{\alpha_k} t^{1 - \frac{\alpha_{k}}{k}}\label{eq:first_bound}\\
        (n\cdot x) \cdot x^{\alpha_{k-1} + \gamma_{k-1}\left(1 - \frac{\alpha_{k-1}}{k-1}\right) - 1} &\leq  x^{\alpha_{k-1} \cdot \frac{k-2}{k-1}} (n\cdot x)^{\frac{\alpha_{k-1}}{k-1}} 
        t^{1- \frac{\alpha_{k-1}}{k-1}}\label{eq:second_bound}
    \end{align}
    The first inequality \eqref{eq:first_bound} is trivially satisfied since we chose $\gamma_k \geq k \left(1 - \frac{3-\omega}{k-\alpha_k}\right).$ 
    
    It suffices to show that the second inequality \eqref{eq:second_bound} is also satisfied. Rearranging, we see that it suffices to show that $x^{\gamma_{k-1}} \leq \frac{t}{n}.$
    Plugging in $x$ from \eqref{eq:x_solution_k1_runtime} and $\gamma_{k-1}$ and rearranging, we obtain that this holds as long as 
    \[ t \geq n^{\frac{(k-1) \cdot ((3 - \omega)^2 + k(\omega - 2)) + \alpha_{k-1} (k(2-\omega) + \omega - 3)}{11-(\omega - 2)\alpha_{k-1} + k(\omega - 2) - 7\omega + \omega^2}}.\]
    To show that all $t \geq n^{\gamma_k}$ satisfies the above inequality, it suffices to check that the exponent above is at most $\gamma_k$, i.e., it suffices to check that 
    \[\frac{(k-1) \cdot ((3 - \omega)^2 + k(\omega - 2)) + \alpha_{k-1} (k(2-\omega) + \omega - 3)}{(k-1-\alpha_{k-1})(\omega - 2) + (3 - \omega)^2}
    \leq \gamma_k = k \left(1 - \frac{3-\omega}{k - \alpha_k}\right).\]
    If $\omega = 2$, we can rewrite \eqref{eq:alpha_recursive} as $\alpha_k = \frac{k\alpha_{k-1}}{(k-1)(1 + \alpha_{k-1})}$ to obtain the following equivalent inequality:
    \begin{align*}
        k - (\alpha_{k-1} + 1) \leq k - \frac{(k-1)}{(k-1) + (k-2) \alpha_{k-1}} \cdot (\alpha_{k-1} + 1),
    \end{align*}
    which clearly holds since $(k-2)\alpha_{k-1} \geq 0.$
    
    When $\omega > 2$, we substitute our recursive formula for $\alpha_k$ from \eqref{eq:alpha_recursive} and rearrange to obtain that the inequality is satisfied for all $k > 0$ as long as 
    \begin{align}\label{eq:bound_alpha_for_gamma}
        (k-1)(\omega - 2) \leq \alpha_{k-1} \leq k-1 + \frac{(3-\omega)^2}{\omega - 2}.
    \end{align}
    
    \begin{claim}\label{claim:alpha_k_lb}
    For $2 \le \omega \leq 3$ and $k \geq 2$, we have $k(\omega - 2) \leq \alpha_k$.
    \end{claim}
    \begin{proof}
         We show this by induction. When $k = 2$, the equation is clearly true because $0 \leq \omega - 2 \leq 1$. 
         Therefore, $2 (\omega - 2) \leq 2,$ and the lower bound clearly holds.
    
    Now suppose $k \ge 3$ and that $\alpha_{k-1} \geq (k-1)(\omega - 2)$. Now, we want $\alpha_k \geq k (\omega - 2)$. Substituting the recursion from \eqref{eq:alpha_recursive}, we have
    \begin{align*}
        \frac{k \alpha_{k-1} ((5-2k) + (k-2)\omega)}{(k-1) \cdot (3 + \alpha_{k-1} - \omega)} \geq k (\omega - 2).
    \end{align*}
    Rearranging the equation using the fact that $3 - \omega + \alpha_{k-1}  > 0$, we have that the above equation holds if and only if
    \begin{align*}
        (3 - \omega) (\alpha_{k-1} - (k-1)(\omega - 2)) \geq 0.
    \end{align*}
    Since $\omega \leq 3$ and $\alpha_{k-1} \geq (k-1)(\omega - 2)$, we have that the equation indeed holds. 
    \end{proof}
   \begin{claim}
        For $2 \leq \omega \leq 3$ and $k \geq 2$, $\alpha_{k} \leq k.$
   \end{claim}
   \begin{proof}
       We proceed by induction. First, we note that $\alpha_2 = 2$ and $\alpha_3 = \frac{3(\omega - 1)}{5 - \omega} \leq 3$ since $\omega \leq 3$. 
        Now, suppose $\alpha_{k-1} \leq k-1$. Then, note that
        \begin{align*}
            \alpha_k &\leq \frac{k \alpha_{k-1} ((5-2k) + (k-2)\omega)}{(k-1) \cdot (3 + \alpha_{k-1} - \omega)} \leq \frac{k ((5-2k) + (k-2)\omega)}{3 + \alpha_{k-1} - \omega}. 
        \end{align*}
        Therefore, $\alpha_k \leq k$ as long as 
        \begin{align*}
            &(5-2k) + (k-2)\omega \leq 3 + \alpha_{k-1} - \omega\\
            \iff & \alpha_{k-1} \geq (k-1)(\omega - 2),
        \end{align*}
        which is indeed true by Claim~\ref{claim:alpha_k_lb}.
   \end{proof}
   Therefore, the bounds in \eqref{eq:bound_alpha_for_gamma} indeed hold, thereby completing the proof.
\end{proof}

Using the bound of Theorem~\ref{thm:k_1_optimal} and note that the runtime of $\cliquelist{k, \ell}$ is monotone with respect to $t$ (Lemma~\ref{lem:list_exponent_monotone}), we immediately get the following corollaries. 
\begin{corollary}[Theorem~\ref{thm:4_1_opt}]\label{cor:4-1-opt}
    Given a graph on $n$ nodes, one can list $t$ 4-cliques in $$\tilde{O}\left(n^{\omega + 1} + n^{\frac{4(\omega - 1)(2\omega - 3)}{\omega^2 - 5 \omega + 12}}t^{1 - \frac{(\omega - 1)(2\omega - 3)}{\omega^2 - 5 \omega + 12}}\right)$$ time. If $\omega = 2$, the runtime is $\tilde{O}(n^3 + n^{2/3}t^{5/6}).$
\end{corollary}

\begin{corollary}[Theorem~\ref{thm:5_1_opt}]\label{cor:5_1_opt}
     Given a graph on $n$ nodes, one can list $t$ 5-cliques in $$\tilde{O}\left(n^{\omega + 2} + n^{\frac{5(\omega - 1)(2\omega - 3)(3\omega - 5)}{48-47\omega + 16\omega^2 - \omega^3}}t^{1 - \frac{(\omega - 1)(2\omega - 3)(3\omega - 5)}{48-47\omega + 16\omega^2 - \omega^3}}\right)$$
    time. If $\omega = 2$, the runtime is $\tilde{O}(n^4 +n^{1/2}t^{9/10}).$
\end{corollary}
    
\subsection{Analysis for \texorpdfstring{$\cliquelist{k, \ell}$}{(k, l)-Clique-Listing} for \texorpdfstring{$\ell \geq 2$}{l >= 2}}\label{sec:k_l_opt}
We have shown an algorithm for $\cliquelist{k, 1}$ that is conditionally optimal when $\Delta_k \geq n^{\gamma_k}$. Now, we use this to show that there exists a $\cliquelist{k, \ell}$ algorithm for all $\ell$ that is conditionally optimal for $\Delta_k \geq n^{\gamma_{k, \ell}}$, for some $0 \leq \gamma_{k, \ell} < \frac{k}{\ell}.$ First, we define the following variable 
\[z_{k,\ell} = x_k \sum_{i=0}^{\ell-1}  \frac{k-\ell}{k-i-1} \cdot  \frac{y_{k-i}}{x_{k-i}}\]

where $x_k$ and $y_k$ are just as defined in \eqref{eq:alpha_num} and \eqref{eq:alpha_denom}. From this definition, the following identity is immediate.
\begin{claim}\label{claim:zkl_recursive}
For $\ell \ge 2$ and $k > \ell$, 
$z_{k,\ell} =\frac{x_k}{x_{k-1}} z_{k-1,\ell-1} +  \frac{k-\ell}{k-1}y_k$. 
\end{claim}

\begin{theorem}
\label{thm:k-l-large-t-listing}
    Fix any constant integers $k-1 \ge \ell \ge 1$.
    Let $\alpha_{k,\ell} = x_k/z_{k,\ell}$. Then, there exists
    some $\gamma_{k, \ell} = (1 - \epsilon_{k,\ell})k/\ell$ for
    $\epsilon_{k,\ell} > 0$ such that for large $t \geq n^{\gamma_k}$ 
    there exists an algorithm that lists all $t$ $k$-cliques given the $\ell$-cliques in time $\tO(\Delta_\ell^{\alpha_{k,\ell}}t^{1-\frac{\ell \alpha_{k,\ell}}{k}})$. 
    
    If $\omega = 2$, we have $x_k = k$ and $z_{k, \ell} = \frac{k\ell(k-\ell)}{2}$, giving a runtime of $\tO(\Delta_\ell^{\frac{2}{\ell(k-\ell)}}t^{1 - \frac{2}{k(k-\ell)}})$ for all $t \geq n^{\gamma_{k, \ell}}$, where
    \[\gamma_{k, \ell} = \frac{k(k^2 - 2k - 1)}{\ell(k^2 - k - \ell - 1)}.
    \]
\end{theorem}

\begin{proof}
    We show this inductively on $\ell$. For $\ell = 1$, we have $z_{k, 1} = y_k$, and this simply reduces to Theorem~\ref{thm:k_1_optimal}.
    
    For some $\ell > 1$, suppose that the theorem statement is true for all $\ell' < \ell$ and $k' > \ell'$. In particular, we assume that $\cliquelist{k',\ell'}$ takes time \[\tO(\Delta_{\ell'}^{\alpha_{k',\ell'}}\Delta_{k'}^{1 - \frac{\ell' \alpha_{k', \ell'}}{k'}})\]
    for $\Delta_{k'} \geq \Delta_{\ell'}^{\gamma_{k', \ell'}},$ and that the runtime is 
    \[\tO\left(\Delta_{\ell'}^{\alpha_{k',\ell'} + \gamma_{k',\ell'} \left(1 - \frac{\ell' \alpha_{k', \ell'}}{k'}\right)}\right)\]
    for $\Delta_{k'} \leq \Delta_{\ell'}^{\gamma_{k', \ell'}}.$
    We may assume this because the runtime is non-decreasing in $\Delta_{k'}$ by Lemma~\ref{lem:list_exponent_monotone}. 

    Recall that, 
    at a high level, Algorithm~\ref{alg:optimal_kl} first lists all $k$-cliques containing nodes that are contained in at most $x$ $\ell$-cliques, and deletes all such nodes. Now, there are at most $k\Delta_\ell/x$ nodes left in the graph, and we call the $\dense$ algorithm from Algorithm~\ref{alg:large_t_sparse_dense}.
    
    \paragraph{Runtime analysis.} Fix any $k > \ell$. In Step~\ref{step:kl_opt_lightnodes} of the algorithm, the runtime is given by (omitting $\tO(1)$ factors):
    \begin{align}
        &\sum_{v: \Delta_\ell(v) \leq x}
        \left(\Delta_\ell(v)^{\alpha_{k-1,\ell-1} + \gamma_{k-1,\ell-1} \left(1 - \frac{(\ell-1)\alpha_{k-1,\ell-1}}{k-1}\right)} + \Delta_\ell(v)^{\alpha_{k-1,\ell-1}} \Delta_k(v)^{1 - \frac{(\ell-1)\alpha_{k-1,\ell-1}}{k-1}}
        \right)\nonumber\\
        &\leq \Delta_\ell x^{\alpha_{k-1,\ell-1} + \gamma_{k-1,\ell-1} \left(1 - \frac{(\ell-1)\alpha_{k-1,\ell-1}}{k-1}\right) - 1} + \Delta_\ell^{\frac{(\ell-1)\alpha_{k-1,\ell-1}}{k-1}} t^{1 - \frac{(\ell-1)\alpha_{k-1,\ell-1}}{k-1}} x^{\frac{(k-\ell)\alpha_{k-1,\ell-1}}{k-1}},\label{eq:k_l_holders}   
    \end{align}
    where we use H\"{o}lder's inequality to bound the second term. Suppose for now that $t$ is large enough so that the second term dominates. 
    
    In Step~\ref{step:kl_densecall}, by Theorem~\ref{thm:k_1_optimal}, the runtime can be bounded by $(k\Delta_\ell/x)^{\alpha_k} t^{1 - \frac{\alpha_k}{k}}$  up to $\tilde{O}(1)$ factors, if we have
    \begin{equation}\label{eq:k_l_secondbound}
        t \geq (\Delta_\ell/x)^{\gamma_k}.
    \end{equation}
    
    Suppose that $t$ is large enough so that this inequality holds. Then, the runtime of the algorithm is
    \begin{align*}
        \Delta_\ell^{\frac{(\ell-1)\alpha_{k-1,\ell-1}}{k-1}} t^{1 - \frac{(\ell-1)\alpha_{k-1,\ell-1}}{k-1}} x^{\frac{(k-\ell)\alpha_{k-1,\ell-1}}{k-1}} +  \left(\frac{\Delta_\ell}{x}\right)^{\alpha_k} t^{1 - \frac{\alpha_k}{k}}.
    \end{align*}
    Choosing \begin{equation}\label{eqn:x_k_l_listing}
    x = \Delta_\ell^{\frac{(k-1) \alpha_k - (\ell-1)  \alpha_{k-1,\ell-1} }{(k-1)\alpha_k + (k-\ell)\alpha_{k-1,\ell-1}}} 
    t^{\frac{k(\ell-1) \alpha_{k-1,\ell-1} - (k-1) \alpha_k}{k((k-1)\alpha_k + (k-\ell)\alpha_{k-1,\ell-1})}},\end{equation}
    we get a runtime of (omitting $\tO(1)$ factors)
    \[
        \Delta_\ell^{\frac{\alpha_k \alpha_{k-1,\ell-1} (k-1)}{\alpha_k(k-1) + \alpha_{k-1,\ell-1} (k-\ell))}} t^{1 - \frac{\ell}{k} \cdot \frac{\alpha_k \alpha_{k-1,\ell-1} (k-1)}{\alpha_k(k-1) + \alpha_{k-1,\ell-1} (k-\ell))}},
    \]
    therefore giving 
    \begin{align*}
        \alpha_{k,\ell} &= \frac{\alpha_k \alpha_{k-1,\ell-1} (k-1)}{\alpha_k(k-1) + \alpha_{k-1,\ell-1} (k-\ell)}\\
        &= \frac{\frac{x_k}{y_k}\cdot \frac{x_{k-1}}{z_{k-1,\ell-1}}}{\frac{x_k}{y_k} + \frac{k-\ell}{k-1} \cdot \frac{x_{k-1}}{z_{k-1,\ell-1}}}\\
        &= \frac{x_k}{\frac{x_k}{x_{k-1}} z_{k-1,\ell-1} + \frac{k-\ell}{k-1}\cdot y_k} = \frac{x_k}{z_{k,\ell}}
    \end{align*}
    by Claim~\ref{claim:zkl_recursive}.
    
    \paragraph{Bound on $\gamma_{k, \ell}$ if $\omega = 2$.} 
    If $\ell = 1$, then $\gamma_{k, 1}$ matches the value of $\gamma_k$ we obtained from Theorem~\ref{thm:k_1_optimal}. Thus, we assume $\ell > 1$ and the bound for all $\ell' < \ell$ holds. 
    Recall that when $\omega = 2$, $\alpha_k = \frac{2}{k-1}$ and $\alpha_{k-1, \ell-1} = \frac{2}{(\ell-1)(k-\ell)}$. Therefore, substituting this into \eqref{eqn:x_k_l_listing}, we obtain
    \[
        x = \Delta_\ell^{\frac{(k-\ell-1)(\ell - 1)}{\ell(k-\ell)}} t^{\frac{\ell - 1}{k(k-\ell)}}.
    \]
    First, we check that \eqref{eq:k_l_secondbound} holds. In fact,
    \begin{align*}
        &t \geq \left(\frac{\Delta_\ell}{x}\right)^{\gamma_k} = \left(\Delta_\ell \cdot \Delta_\ell^{-\frac{(k-\ell-1)(\ell - 1)}{\ell(k-\ell)}} t^{-\frac{\ell - 1}{k(k-\ell)}}\right)^
        {\gamma_k}\\
        \iff & t^{1 + \frac{\gamma_k(\ell - 1)}{k(k-\ell)}} \geq \Delta_\ell^{\frac{(k-1)\gamma_k}{\ell(k-\ell)}}\\
        \iff & t \geq \Delta_\ell^{\frac{k(k-1)\gamma_k}{k\ell(k- \ell) + \ell (\ell - 1)\gamma_k}}.
    \end{align*}
    Substituting $\gamma_k = k - 1 - \frac{2}{k^2 - k - 2}$ from Theorem~\ref{thm:k_1_optimal}, we get the inequality
    $t \geq \Delta_\ell^{ \frac{k(k^2 - 2k - 1)}{\ell(k^2 - k - \ell - 1)}} = \Delta_\ell^{\gamma_{k, \ell}}$, which is indeed true by our choice of $\gamma_{k, \ell}$. 
    
    Now, it suffices to show that if $t \geq \Delta_\ell^{\gamma_{k ,\ell}}$, then the second term dominates in \eqref{eq:k_l_holders}. In fact, the second term dominates as long as 
    \begin{align*}
        &t \geq \Delta_\ell x^{\gamma_{k-1,\ell-1} - 1}\\
        \iff &t \geq \Delta_\ell^\frac{1+\frac{(k-\ell-1)(\ell - 1)}{\ell(k-\ell)}(\gamma_{k-1,\ell-1}-1)}{1 - \frac{\ell - 1}{k(k-\ell)}(\gamma_{k-1,\ell-1}-1)} = \Delta_\ell^{\frac{k(k-1)(k-3)}{\ell(k^2 - 3k + 3 - \ell)}},
    \end{align*}
    where the last equality holds because $\gamma_{k-1,\ell-1} = \frac{(k-1)((k-1)^2 - 2(k-1)-1)}{(\ell-1) ((k-1)^2 - k - \ell + 1)}$
    by induction. 
    Hence, it suffices to show that $\gamma_{k,\ell}$ is at least the exponent on the right-hand side.
    \begin{align*}
        &\gamma_{k, \ell} = \frac{k(k^2 - 2k - 1)}{\ell(k^2 - k - \ell - 1)} \geq \frac{k(k-1)(k-3)}{\ell(k^2 - 3k + 3 - \ell)}\\
        \iff & 1 - \frac{k-\ell}{k^2-k-1-\ell} \geq 1 - \frac{k-\ell}{k^2 - 3k - \ell + 3}\\
        \iff & k^2 - k - 1 - \ell \geq k^2 - 3k - \ell + 3\\
        \iff & k \geq 2,
    \end{align*}
    which is true since $k \geq 3.$
    
    \paragraph{Bound on $\gamma_{k, \ell}$ if $\omega > 2$.} In this case, we show that there exists some $\epsilon_{k,\ell} > 0$ such that $\gamma_{k,\ell} \leq \frac{k}{\ell}(1 - \epsilon_{k, \ell})$.

    
    First, consider \eqref{eq:k_l_secondbound}. Note that one can rewrite 
    \begin{align*}
       x = \Delta_\ell^{1 - \frac{\alpha_{k,\ell}}{\alpha_k}} t^{\frac{\ell \alpha_{k,\ell}}{k\alpha_k} - \frac{1}{k}}.
    \end{align*}
    Rewriting $q = \frac{\alpha_{k, \ell}}{\alpha_k}$, and substituting this into \eqref{eq:k_l_secondbound}, we obtain
    \begin{align*}
        t \geq \left(\frac{\Delta_\ell}{x}\right)^{\gamma_k} = \left(\frac{\Delta_\ell^q}{t^{\frac{\ell q}{k} - \frac{1}{k}}}\right)^{\gamma_k} \iff t \geq \Delta_\ell^{\frac{kq \gamma_k}{\ell q \gamma_k + k - \gamma_k}}.
    \end{align*}
By choosing $\epsilon_1 = \frac{k - \gamma_k}{\ell q \gamma_k + k - \gamma_k}$ (which is positive since $k > \gamma_k$ by Theorem~\ref{thm:k-l-large-t-listing}, it is easy to check that the right-hand side is equal to $\Delta_\ell^{\frac{k}{\ell}(1 - \epsilon_1)}$.

Now, consider \eqref{eq:k_l_holders}. For the second term to dominate, we can rewrite the inequality as 
\begin{align*}
    t \geq \Delta_\ell \cdot x^{\gamma_{k-1,\ell-1} - 1} = \Delta_\ell^{1 + (\gamma_{k-1,\ell-1} - 1)(1-q)} t^{\frac{1}{k} \cdot (\gamma_{k-1,\ell-1} - 1)(\ell q - 1)}.
\end{align*}
Rearranging this, we see that we require
\begin{align*}
    t \geq \Delta_\ell^{\frac{1 + (\gamma_{k-1,\ell-1} - 1) (1 - q)}{1 - \frac{1}{k}(\gamma_{k-1,\ell-1} - 1)(\ell q - 1)}}.
\end{align*}
By the induction hypothesis, we know there exists some $0 < \epsilon' < 1$ such that $\gamma_{k-1,\ell-1} = \frac{k-1}{\ell - 1}(1 - \epsilon').$ Therefore, substituting this into the above equation and rearranging, we require
\begin{align}\label{eq:gamma_kl_holderbound}
    t \geq \Delta_\ell^{\frac{k}{\ell}\left(1 - \frac{(k-1)(\ell - 1)\epsilon'}{(q\ell - 1)(k-1) \epsilon' + \ell((k-1) - q(k-\ell))}\right)}.
\end{align}
Let $\epsilon_{num} = (k-1)(\ell-1)\epsilon'$ and and $\epsilon_{den} = (q\ell - 1)(k-1) \epsilon' + \ell((k-1) - q(k-\ell))$. Clearly, since $k \geq 3$ and $\ell \geq 2$, $\epsilon_{num} > 0$. Now, consider two cases.
\begin{itemize}
    \item $q\ell - 1 \geq 0$. Then, since $\epsilon' > 0$, we have $\epsilon_{den} \geq \ell((k-1) - q(k-\ell)) > 0$ since $q = \frac{\alpha_{k,\ell}}{\alpha_k} = \frac{y_k}{z_{k,\ell}} < \frac{k-1}{k-\ell}$ by Claim~\ref{claim:zkl_recursive}.
    \item $q\ell - 1 < 0$. Then, since $\epsilon' < 1,$ $\ell \geq 2,$ $q > 0$ and $k \geq 3$
    \begin{align*}
        \epsilon_{den} &> (q\ell - 1)(k-1)  + \ell((k-1) - q(k-\ell))\\
        & = q\ell (\ell - 1) + (\ell-1)(k-1) > 0.
    \end{align*}
\end{itemize}

    Therefore, let $\epsilon_2 = \frac{\epsilon_{num}}{\epsilon_{den}}$. Clearly, $\epsilon_2 > 0$. Then, if $t \geq \Delta_\ell^{\frac{k}{\ell}(1 - \epsilon_2)}$, then \eqref{eq:gamma_kl_holderbound} holds. Hence, we can pick $\epsilon_{k, \ell} = \min(\epsilon_1, \epsilon_2) > 0$ to ensure both conditions \eqref{eq:k_l_holders} and \eqref{eq:k_l_secondbound} hold.
\end{proof}
