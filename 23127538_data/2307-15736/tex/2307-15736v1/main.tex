\documentclass[11pt,letterpaper]{article}
\pdfoutput=1
\usepackage{jheppub} 
\usepackage{mathrsfs}
\usepackage{slashed}
\usepackage{caption}
\usepackage{epstopdf}
\usepackage[normalem]{ulem}
\usepackage[bottom]{footmisc}
\usepackage{subcaption}
\usepackage{bbold}
\usepackage{titlesec}
\usepackage{threeparttable}
\usepackage{booktabs}
\usepackage{changepage}
\usepackage[utf8]{inputenc}
\usepackage{dsfont}
\usepackage{grffile}
\usepackage{graphicx}  % needed for figures
\usepackage{dcolumn}   % needed for some tables
\usepackage{bm}        % for math
\usepackage{amssymb}   % for math
\usepackage{setspace}
\usepackage{amsmath, amssymb, setspace}
\usepackage{array}
\usepackage{booktabs}
\usepackage{caption}
\usepackage{indentfirst}
\usepackage{float}
\usepackage{lmodern}
\usepackage{multirow}
\usepackage{soul}
\usepackage[normalem]{ulem}
\usepackage{braket}
\usepackage{comment}
\usepackage{appendix}  
\usepackage{feynmp-auto}
\graphicspath{{figs/}}

% \newcommand{\TY}[1]{{\color{magenta} [TY: #1] }}
% \newcommand{\TS}[1]{{\color{red} [TS: #1] }}
% \newcommand{\vt}[1]{{\color{blue} [VT: #1] }}

\newcommand{\eg}{{\it e.g.}}

\title{\huge{Hunting Nonstandard Neutrino Interactions and Leptoquarks in Dark Matter Experiments}} 

\author[a]{Thomas Schwemberger,}
\author[b,c,d]{Volodymyr Takhistov,}
\author[a,e]{Tien-Tien Yu}

\affiliation[a]{Department of Physics and Institute for Fundamental Science, University of Oregon\\
Eugene, Oregon 97403, USA
}
\affiliation[b]{International Center for Quantum-field Measurement Systems for Studies of the Universe and Particles (QUP, WPI),
High Energy Accelerator Research Organization (KEK), \\ Oho 1-1, Tsukuba, Ibaraki 305-0801, Japan}
\affiliation[c]{Theory Center, Institute of Particle and Nuclear Studies (IPNS), High Energy Accelerator Research Organization (KEK), Tsukuba 305-0801, Japan}
\affiliation[d]{Kavli Institute for the Physics and Mathematics of the Universe (WPI), \\ The University of Tokyo Institutes for Advanced Study, The University of Tokyo, \\ Kashiwa, Chiba 277-8583, Japan}
\affiliation[e]{Center for Cosmology and Particle Physics, Department of Physics, New York University, New York, NY 10003, USA}

\emailAdd{tschwem2@uoregon.edu}
\emailAdd{vtakhist@post.kek.jp}
\emailAdd{tientien@uoregon.edu} 

\abstract{
Next generation direct dark matter (DM) detection experiments will be have unprecedented capabilities to explore coherent neutrino-nucleus scattering (CE$\nu$NS) complementary to dedicated neutrino experiments.
We demonstrate that future DM experiments can effectively probe nonstandard neutrino interactions (NSI) mediated by scalar fields in the
scattering of solar and atmospheric neutrinos. 
We set first limits on $S_1$ leptoquark models that result in sizable $\mu-d$ and $\tau-d$ sector neutrino NSI CE$\nu$NS contributions using LUX-ZEPLIN (LZ) data. As we show, near future DM experiments reaching $\sim \mathcal{O}(100)$ton-year exposure, such as argon-based ARGO and xenon-based DARWIN, can probe parameter space of leptoquarks beyond the reach of current and planned collider facilities. We also analyze for the first time prospects for testing NSI in lead-based detectors.
We discuss the ability of leptoquarks in the parameter space of interest to also explain the neutrino masses and $(g-2)_\mu$ observations.
}

%\pacs{} 
%\date{}

\begin{document}

\preprint{KEK-QUP-2023-0007, KEK-TH-2516, KEK-Cosmo-0309, IPMU23-0009}
 \maketitle
\flushbottom
 
\section{Introduction}
Dark matter (DM) constitutes $\sim85\%$ of all matter in the Universe and understanding its nature is one of the most important goals of science. Significant efforts have been devoted over the past several decades to detect the scattering of particle DM from the Galactic DM halo with target nuclei in underground laboratories. Historically, the focus has been on probing interactions of weakly-interacting massive particles (WIMPs) with masses $\sim 100$~GeV-TeV, using large ton-scale experiments based on liquid xenon, \eg~LUX-ZEPLIN (LZ)~\cite{LZ:2022ufs}, XENONnT~\cite{XENON:2023sxq}, and liquid argon, \eg~DEAP-3600~\cite{DEAP:2019yzn} and DarkSide-50~\cite{DarkSide-50:2022qzh}, detectors. Not only are these detectors very sensitive to the scattering of DM particles with Standard Model (SM) targets, they can also be sensitive to and provide new insights into additional phenomena both within and beyond the SM (BSM). 

As direct DM detection experiments continue to improve their sensitivity, they will inevitably encounter an irreducible background stemming from coherent elastic neutrino-nucleus scattering (CE$\nu$NS).
Predicted decades ago~\cite{Freedman:1973yd,Drukier:1984vhf}, CE$\nu$NS interactions have been directly observed by COHERENT in 2017 using neutrinos generated by pions that decay at rest~\cite{COHERENT:2017ipa}.\footnote{Efforts to observe CE$\nu$NS using reactor anti-neutrinos are also being pursued~\cite{CONUS:2021dwh,CONNIE:2016nav,NUCLEUS:2019igx,Billard:2016giu}.}
Interactions of neutrinos originating from natural sources, such as solar neutrinos from the Sun and atmospheric neutrinos from cosmic ray collisions with the atmosphere, are 
indistinguishable on an event-by-event basis from DM-induced scattering events depositing energy within detectors~(\eg ~\cite{Billard:2014, Essig:2018tss,Gelmini:2018ogy}). 
The degeneracy between neutrino and DM scattering results in the so-called ``neutrino fog" (formerly known as the ``neutrino floor").
Analyses of event statistics and temporal modulation, such as with sidereal day, can help disentangle the degeneracy.

Since the discovery of neutrino oscillations~\cite{Super-Kamiokande:1998kpq}, a multitude of experiments have confirmed and further developed the SM picture of neutrino interactions and 3-flavor oscillations~\cite{ParticleDataGroup:2022pth}.
With neutrino physics entering the precision era in a broad, world-wide program including a variety of upcoming major next generation experiments such as DUNE and Hyper-Kamiokande, significant attention is being devoted to probing new neutrino physics.
Sizable novel ``non-standard'' neutrino interactions (NSI) can readily appear~(see \eg~\cite{Proceedings:2019qno} for review) in broad classes of motivated models, such as those based on SM extensions by an additional $U(1)$ gauge symmetry and new mediators (\eg~\cite{Farzan:2016wym}). 

Along with precision tests of SM parameters, observations of CE$\nu$NS, such as data from COHERENT, allow for a unique probe of neutrino NSI and related theoretical models~\cite{Barranco:2005yy,Scholberg:2005qs,Miranda:2020tif,Liao:2017uzy,Giunti:2019xpr,Khan:2021wzy,Denton:2020hop,Farzan:2018gtr,Arcadi:2019uif,Billard:2018jnl,Bertuzzo:2021opb,delaVega:2021wpx,Banerjee:2021laz,Cadeddu:2020nbr,Flores:2020lji,Denton:2018xmq}.  
Upcoming next generation DM experiments will have sensitivity to CE$\nu$NS that allows for unique opportunities to test novel neutrino interactions in complementary ways to dedicated neutrino experiments. Studies already initiated exploration of the reach of DM detectors to neutrino NSI~\cite{Harnik:2012ni,Cerdeno:2016sfi} looking at novel scalar, pseudoscalar, vector, and axial-vector neutrino interactions. Limits on solar neutrino NSI from observations of previous DM experiments have also been explored,\footnote{Neutrino NSI can also be readily probed through electron scattering in DM experiments, which has been recently exploited in attempt to address excess in the XENON1T experiment data~\cite{Boehm:2020ltd}. However, in this work we focus solely on CE$\nu$NS interactions.} such as from LUX experiment data~\cite{Bertuzzo:2017tuf}.  

In this work, we expand on previous studies and demonstrate the power of DM experiments to probe neutrino NSI. Our analysis, focusing primarily on heavy mediators of novel neutrino interactions, advances the results in the literature in several important ways. While SM interactions of atmospheric neutrinos have been explored in DM experiments~\cite{Newstead:2020fie}, we study for the first time their potential NSI effects.
Furthermore, we explore first constraints on leptoquarks (LQs) and associated theories that can lead to sizable solar and atmospheric neutrino NSI effects in DM experiments, and for the first time use LZ data to set constraints on LQs. We discuss the complementarity of LQ searches in DM experiments and other venues, especially colliders, and connect them with models aiming to address flavor physics observations~\cite{angelescu_closing_2018, Lee:2021jdr, FileviezPerez:2021lkq}, $(g-2)_\mu$, and neutrino mass generation \cite{Dorsner:2017wwn, zhang_radiative_2021, PhysRevD.LQ3}.
Our analyses are complementary to studies of NSI in DM experiments associated with light mediators, such as~Ref.~\cite{LI2022115737, PhysRevD.106.013001,Harnik:2012ni,deNiverville:2015mwa,Cerdeno:2016sfi} or ~Ref.~\cite{Schwemberger:2022fjl}, the latter of which demonstrated that low-threshold ($\sim$ eV) DM detectors are sensitive to light ($\sim$ keV - MeV) mediators where the low energy limit provides an enhancement of the recoil rate.  

The paper is organized as follows. In the Sec. \ref{sec:flux}, we describe the neutrino flux, its flavor composition, and its signal in DM detectors. In Sec. \ref{sec:BSM} we show the effects of neutrino NSIs and the effective theories that produce such interactions. In Sec. \ref{sec:LQ} we discuss the details of a UV completion of the effective models as LQs including effects on neutrino masses, $(g-2)_\mu$, and neutrino oscillations. In Sec. \ref{sec:disc_reach} we demonstrate the ability of DM detectors to observe the effective models described in Sec. \ref{sec:BSM}, calculate constraints from the LZ experiment, and project the discovery reach of future DM detectors based on xenon, argon, and lead.

\section{Standard Model Neutrino Scattering}\label{sec:flux}

\subsection{Interaction rates and CE\texorpdfstring{$\nu$}{v}NS}
\label{sec:rates}

In DM experiments, the recoil rate of neutrinos as a function of the nuclear recoil energy depends on 
both the physics of the scattering process encapsulated in the cross-section, neutrino energy and flux, as well as the properties of the detector such as target material, size, efficiency, energy threshold, and energy resolution. The general differential recoil rate is given by
\begin{equation}\label{eq:gen_scattering}
    \frac{dR}{dE'_R} = N_T \int_{E_\nu^{\rm min}} \frac{d\sigma}{dE'_R} \frac{dN_\nu}{dE_\nu}dE_\nu~,
\end{equation}
where $dN_\nu/dE_\nu$ is the neutrino flux, $N_T$ is the number of target nuclei per unit mass (ton) of detector material, $d\sigma/d{E'}_R$ is process cross-section and for SM CE$\nu$NS it is given by Eq.~\eqref{eq:SM_cross}. Here,
primed energies represent the scattering energy of the interaction while the unprimed energies are those actually measured by the detector. The effects of new physics on the interactions will be explored in Sec.~\ref{sec:BSM}.
From kinematics, the minimum required neutrino energy to induce a recoil energy $E_N^{\prime}$ of a nucleus with mass $m_N$ is 
\begin{equation}
\label{eq:emin}
E_\nu^{\rm min} =  \dfrac{1}{2}\Big({E'}_R+\sqrt{{E'}_R^2 + 2{E'}_R m_N}\Big)~.
\end{equation} 

We incorporate the detection efficiency and energy resolution of the detector by integrating over $E'_R$ convoluted with the efficiency and a Gaussian function representing uncertainty
\begin{equation}
    \frac{dR}{dE_R} = \int_0^\infty \frac{dR}{dE'_R} \frac{\eta(E'_R)}{\zeta(E'_R)\sqrt{2\pi}}e^{-\frac{(E_R - E'_R)^2}{2\zeta^2(E'_R)}}dE'_R\, ,
\end{equation}
where for $\eta(E'_R)$ we will consider data-driven efficiency function of Eq.~\eqref{eq:efficiency} and $\zeta(E'_R)$ is the energy resolution that we conservatively assume to be $\mathcal{O}(10)\%$ for all energies of interest.\footnote{Our analysis can be readily adapted for other efficiency and resolution functions.}

Focusing on the data analysis from LZ experiment as a characteristic target~\cite{LZ:2022ufs}, we model the detection efficiency with a hyperbolic tangent as
\begin{equation}\label{eq:efficiency}
    \eta(E) = \frac{\eta_0}{1+\zeta}\left(\tanh{\xi\left(\frac{E_R}{E_{\rm th}}-1\right)}+\zeta\right)~,
\end{equation} 
where $\eta_0$ is the maximum efficiency at high energy, $\xi$ sets the rate of the efficiency rise, $E_{\rm th}$ is the threshold energy, and $\zeta$ enforces that the efficiency vanishes at zero recoil energy. We fix $\xi = 2$, which requires $\zeta \approx 0.964$. To approximately match the LZ efficiency, we use $\eta_0 = 0.9$ such that the only remaining parameter is $E_{\rm th}$, which we vary between 1 and 5 keV. We display the efficiency for a selection of threshold energies in Fig.~\ref{fig:expeff}. 

% Figure environment removed

Neutrinos that do not possess sufficient energy to discriminate individual nuclear constituents interact with the whole target nucleus via CE$\nu$NS mediated by the weak neutral current~\cite{Freedman:1973yd,Drukier:1984vhf}. For low momentum transfer $|q|$ satisfying $1/|q| \gg R$, where $R$ is the nuclear radius, the target nucleons interact in phase. CE$\nu$NS is particularly relevant for neutrino sources with $E_{\nu} < 50$~MeV, above which other interactions become significant (see \eg~\cite{Formaggio:2012cpf} for review). Within the SM, CE$\nu$NS interactions are given by
\begin{equation}\label{eq:SM_cross}
    \left.\frac{d\sigma_N}{dE_R}\right\vert_{SM} \hspace{0.5mm} = \dfrac{G_F^2}{4\pi}F^2(E_R) Q_v^2 m_N\left(1-\frac{m_N E_R}{2E_\nu^2}\right)\, ,
\end{equation}
where $G_F$ is the Fermi constant, $Q_v = N - Z(1-4s^2_w)$ is the
weak nuclear charge, $N$ is the number of neutrons, $Z$ is the number of protons, and $s_w$ is the sine of the Weinberg angle. 
For the atomic mass number $A = N + Z$ and $m_n = 931$~MeV, the nucleus mass is $m_N \approx A m_n$. 
Here, we employ the Helm nuclear form factor \cite{LEWIN199687, PhysRev.104.1466}
\begin{equation}
F^2(E_R) = \Big[\dfrac{3 j_1(q R_{\rm eff})}{q R_{\rm eff}}\Big]^2 e^{-(q s)^2}~,
\end{equation} 
where $j_1$ denotes spherical Bessel function, $q = \sqrt{2 m_N E_R}$ is the transferred 3-momentum, $R_{\rm eff} =\sqrt{(1.23A^{1/3} - 0.6)^2 + 7\pi^2r_0^2/3 - 5s^2}$ is the effective nuclear radius, $s=0.9$ fm is the skin thickness and $r_0=0.52$ fm is fit numerically from muon scattering data.
With $s_w^2 = 0.232$~\cite{ParticleDataGroup:2022pth} CE$\nu$NS interactions scale as $\sim N^2$, signifying sizable enhancement of interaction rates for heavy target materials such xenon, argon, or lead. 

\subsection{Neutrino sources}

A variety of neutrino sources can contribute to irreducible background ({\it i.e.} giving rise to ``neutrino floor/fog'')  in direct DM detection experiments, including solar, reactor, geo-,
diffusive supernovae background (DSNB) as well as atmospheric neutrinos. 
For reference, we consider benchmark detector located at SNOLab in Sudbury, Canada that is among likely sites to host next generation DM experiments.\footnote{The SNOLab experimental site is located at 46$^{\circ}$28'19'' N, 81$^{\circ}$11'12'' W with a 6010 meter water equivalent depth.}

The integrated fluxes for various neutrino components and their uncertainties (rounded to percent level) are listed in Table \ref{tab:nu_flux}. Contributions from atmospheric, reactor and geo-neutrinos will depend on the location of the experiment.
Depending on the experimental energy threshold for nuclear recoil energy $E_r$, only fluxes resulting in neutrinos with maximum energy $E_{\nu}^{\rm max} > E_{\nu}^{\rm min}$, where $E_{\nu}^{\rm min}$ is given by Eq.~\eqref{eq:emin}, will contribute. In Table~\ref{tab:nu_flux} we display the minimum required energy thresholds in xenon-based and argon-based DM experiments\footnote{Considering dominant isotopes, we take $A(Z)$ = 130 (54) for xenon and $A(Z)$ = 40(18) for argon.} to detect a particular neutrino source flux. 
We display the combined solar, DSNB and atmospheric neutrino flux contributions in Fig.~\ref{fig:flux}, along with shaded bands depicting their respective uncertainties. 

The keV-level threshold recoils in representative xenon-based experiment require that $E_{\nu}^{\rm min} \gtrsim \mathcal{O}$(few$\times$MeV), such that solar (hep and $^8$B) background contributions dominate in the region of interest for non-standard neutrino interactions. We include the subdominant atmospheric neutrinos as they are useful in specific UV complete NSI models. A representative figure for scalar-mediated non-standard neutrino interactions are displayed in Fig.~\ref{fig:recoil_rates}, further discussed below in Sec.~\ref{sec:BSM}.

 \begin{table*}[tbp]
\setlength{\extrarowheight}{2pt}
  \setlength{\tabcolsep}{10pt}
  \begin{center}
  \begin{threeparttable}
	\begin{tabular}{|l|l|c|c|l|}  \hline
	~~~~~~Neutrino & ~~~~~~~Total Flux & $E_{\nu}^{\rm max}$ & $E_{\rm th}^{\rm Xe}$ ($E_{\rm th}^{\rm Ar}$) & ~~~~Reference \\ 
    ~~~~Contribution & ~~~~~~~[cm$^{-2}$ s$^{-1}$]               & [MeV] & [keV] & ~~~~~(model) \\ \hline
	\hline
    Solar ($\nu_e$, pp)       & $5.98 (1 \pm 0.01)  \times 10^{10}$&   $0.42$  & $2.99(9.81)\times 10^{-3}$ & \cite{Vinyoles:2016djt} (B16-GS98) \\ \hline
	Solar ($\nu_e$, pep)      & $1.44 (1 \pm  0.01) \times 10^{8}$ &   $1.45$  & $3.42(11.2)\times 10^{-2}$ & \cite{Vinyoles:2016djt} (B16-GS98)\\ \hline
    Solar ($\nu_e$, hep)      & $7.98 (1 \pm  0.30) \times 10^{3}$ &   $18.77$ & $5.74(18.9)$               & \cite{Vinyoles:2016djt} (B16-GS98) \\ \hline
    Solar ($\nu_e$, $^{7}$Be) & $4.93 (1 \pm  0.06) \times 10^{9}$ &   $0.39$  & $1.20(3.96)\times 10^{-2}$ & \cite{Vinyoles:2016djt} (B16-GS98) \\ \hline
    Solar ($\nu_e$, $^{8}$B)  & $5.46 (1 \pm  0.12) \times 10^{6}$ &   $16.80$ & $4.58(15.1)$               & \cite{Vinyoles:2016djt} (B16-GS98) \\ \hline
    Solar ($\nu_e$, $^{13}$N) & $2.78 (1 \pm  0.15) \times 10^{8}$ &   $1.20$  & $2.33(7.65)\times 10^{-2}$ & \cite{Vinyoles:2016djt} (B16-GS98) \\ \hline
    Solar ($\nu_e$, $^{15}$O) & $2.05 (1 \pm  0.17) \times 10^{8}$ &   $1.73$  & $0.0486(0.160)$            & \cite{Vinyoles:2016djt} (B16-GS98) \\ \hline
    Solar ($\nu_e$, $^{17}$F) & $5.29 (1 \pm  0.20) \times 10^{6}$ &   $1.74$  & $0.0492(0.161)$            & \cite{Vinyoles:2016djt} (B16-GS98) \\ \hline
    \hline
    Atm. ($\nu_e$)\tnote{1}   & $1.27 (1 \pm  0.50)\times 10^{1}$ & $944$    & $1.47(5.00)\times 10^4$     & \cite{Battistoni:2005pd} (FLUKA) \\ \hline   
    Atm. ($\overline{\nu}_e$) & $1.17 (1 \pm  0.50)\times 10^{1}$ & $944$    & $1.47(5.00)\times 10^4$     & \cite{Battistoni:2005pd} (FLUKA) \\ \hline   
    Atm. ($\nu_{\mu}$)        & $2.46 (1 \pm  0.50)\times 10^{1}$ & $944$    & $1.47(5.00)\times 10^4$     & \cite{Battistoni:2005pd} (FLUKA) \\ \hline   
    Atm. ($\overline{\nu}_{\mu}$)\tnote{2} & $2.45 (1 \pm  0.50)\times 10^{1}$ & $944$ & $1.47(5.00)\times 10^4$ & \cite{Battistoni:2005pd} (FLUKA) \\ \hline
    \hline
DSNB ($\nu_e$)\tnote{3}      & $4.55 (1 \pm  0.50)\times 10^{1}$ & $36.90$ & $22.1(72.7)$ & \cite{Horiuchi:2008jz} (th.~avrg.)\tnote{1} \\ \hline
DSNB ($\overline{\nu}_e$)    & $2.73 (1 \pm  0.50)\times 10^{1}$ & $57.01$ & $52.8(174)$  & \cite{Horiuchi:2008jz} (th.~avrg.)\tnote{1} \\ \hline
DSNB ($\nu_x$)\tnote{4}      & $1.75 (1 \pm  0.50)\times 10^{1}$ & $81.91$ & $109(359)$   & \cite{Horiuchi:2008jz} (th.~avrg.)\tnote{1} \\ \hline
\hline
Reactor ($\overline{\nu}_e$, $^{235}$U)\tnote{5} & $1.88(1 \pm 0.08) \times 10^5$  & 10.00  & $1.62(5.33)$ & \cite{Gelmini:2018ogy}(combined)\tnote{6} \\
\hline
\hline
Geo. ($\overline{\nu}_e$, $^{40}$K)   & $2.19 (1 \pm 0.17) \times 10^{7}$ &  1.32  & $2.83(9.29)\times 10^{-2}$ & \cite{huang:2013geomodel} (global)\tnote{6} \\
\hline
Geo. ($\overline{\nu}_e$, $^{238}$U)  & $4.90 (1 \pm 0.20) \times 10^{6}$ &  3.99  & $0.259(0.849)$             & \cite{huang:2013geomodel} (global)\tnote{6} \\
\hline
Geo. ($\overline{\nu}_e$, $^{232}$Th) & $4.55 (1 \pm 0.26) \times 10^{6}$ &  2.26  & $8.29(27.2)\times 10^{-2}$ & \cite{huang:2013geomodel} (global)\tnote{7} \\
\hline
 	\end{tabular}
\begin{tablenotes}
\item[1] The cut-off of Fluka analysis.
\item[2] Average of several theoretical models.
\item[3] Fermi-Dirac spectrum with neutrino temperature of $T_\nu = 3, 5, 8$~MeV for $\nu_e$, $\overline{\nu}_e$, $\nu_x$, respectively.
\item[4] $\nu_x$ is the total contribution from all other neutrinos and antineutrinos.
\item[5] Only the most dominant element is considered. 
\item[6] Combined result from multiple nearby reactors.
\item[7] Global Earth model, incorporates several theoretical models.
\end{tablenotes}
\caption{\label{tab:nucomponents} Neutrino flux components that contribute to the coherent neutrino scattering background in direct detection experiments at the  SNOLab location.~Contributions from solar, atmospheric, diffuse supernovae, reactor as well as geo--neutrinos are shown.}
    \label{tab:nu_flux}
\end{threeparttable}
  \end{center}
\end{table*}

\underline{Solar neutrinos}:~The majority of neutrinos reaching the Earth's surface are produced in nuclear fusion processes in the Sun (for review see \eg ~\cite{Haxton:2012wfz,Bahcall:2004pz}). The flux and energy spectrum of solar neutrinos depends on the specific nuclear reaction chain step that produces them. The vast majority of Sun's energy, nearly 99\%, originates from proton-proton (pp) chain reactions, producing pp, hep, pep,
$^7$Be, $^8$B neutrinos. The remaining $\sim1\%$ of Sun's energy originates from Carbon-Nitrogen-Oxygen (CNO) cycle, producing $^{13}$N, $^{15}$O, $^{17}$F neutrinos. 

For $E_{\nu} \lesssim 20$ MeV energies, solar neutrinos dominate background for direct DM detection experiments. 
In this work we employ solar neutrino fluxes of high metallicity solar model B16-GS98~\cite{Vinyoles:2016djt}, favored by recent Borexino neutrino observation data analysis~\cite{BOREXINO:2022abl}. 
The dominant neutrino background contribution for a range of parameters stems from $^8$B neutrinos.  

% Figure environment removed

\underline{DSNB}:~For neutrino energies around $20~\text{MeV} \lesssim E_{\nu} \lesssim 50$~MeV~DSNB significantly contributes to neutrino background fluxes~\cite{Beacom:2010kk}. 
Around  99\% of the gravitational binding energy is released in the form of neutrinos in core-collapsing supernova, resulting in $\sim10^{58}$ neutrinos emitted yielding about $\sim 10^{53}$~ergs in energy. The DSNB denotes combined contributions of neutrinos originating from all the historic core-collapse supernovae events. The DSNB flux is formed from convolution of the redshift-dependent core-collapse supernovae rate, which is found from initial stellar mass function and star formation rate, as well as emitted core-collapse neutrino spectrum.
Approximating the emitted neutrino spectra as thermal, we consider DSNB neutrinos characterized by temperature $T_{\nu}$ averaged for each species over theoretical models in Ref.~\cite{Horiuchi:2008jz}.
Significant efforts are underway by neutrino experiments, especially Super-Kamiokande~\cite{Super-Kamiokande:2021jaq}, to detect the DSNB in the near future.

\underline{Atmospheric neutrinos}:~For $E_{\nu} \gtrsim 50$ MeV, 
neutrino contributions become dominated by neutrinos produced from cosmic ray collisions with nuclei in the atmosphere~(see \eg~\cite{Gaisser:2002jj} for review). Such interactions produce copious amounts of $\pi^\pm$ and $K^\pm$ mesons that predominantly decay to $\mu^+ + \nu_\mu$ and $\mu^- + \bar{\nu}_\mu$, followed by muon decay to electrons and pairs of neutrinos resulting in a $\nu_\mu$ and $\bar{\nu}_\mu$ flux twice that of $\nu_e$ and $\bar{\nu}_e$.
The atmospheric neutrino flux carries an $\mathcal{O}(30\%)$ theoretical uncertainty, which we take to be 50\% conservatively, due to the uncertainty in the cosmic ray flux and its propagation in the magnetic fields of the Earth and solar system. In addition to this uncertainty, the flux varies significantly though predictably with the position of the detector and the phase in the solar cycle~\cite{PhysRevD.105.043001}.

\underline{Neutrino Oscillations}: As shown in Table~\ref{tab:nu_flux}, solar neutrinos are produced as electron neutrinos.  
Since these neutrinos have a much longer baseline than those in reactor experiments, neutrino oscillations result in a significant fraction of neutrinos reaching the detector with $\mu$- or $\tau$-components, allowing experiments to constrain interactions of second and third generation neutrinos. In the two-flavor neutrino mixing approximation the conversion probability is given by 
\begin{equation} \label{eq:mixing}
    P(\nu_i \rightarrow \nu_j) = \sin^2(2\theta_{ij})\sin^2(\pi L/L_{osc})\, ,
\end{equation}
where $i,j$ label the neutrino flavor. For $\nu_e-\nu_\mu$ mixing, we consider~\cite{ParticleDataGroup:2022pth}  $\sin^2(\theta_{12})=0.31$, $\Delta m_{12}^2=7.5 \times 10^{-5}{\rm eV}^2$, and $L_{osc} = 2.48{~\rm km} \times E_\nu / \Delta m_{12}^2$ with $E_\nu$ in GeV. Here, $L$ is the baseline distance in km that for solar neutrinos is the Earth-Sun distance. 

\section{Nonstandard Neutrino Interactions} \label{sec:BSM}

There are several formalisms to incorporate general interactions of neutrinos (see \eg~\cite{10.21468/SciPostPhysProc.2.001}). We follow simplified models outlined in Ref.~\cite{Cerdeno:2016sfi}, also employed in e.g. Ref.~\cite{lindner_coherent_2017, Schwemberger:2022fjl}, by introducing new mediators that couple both to charged fermions and neutrinos. 
Specifically, we focus on scalar mediators with masses above $\sim 1$ TeV, which allows us to map our general interaction to the conventional NSI formalism of effective interactions~\cite{10.21468/SciPostPhysProc.2.001}.

The effective Lagrangian for the scalar mediator $\phi$ is given by 

\begin{equation}
    \mathcal{L} \supset \phi \left( g_{\nu S}\bar{\nu}_R\nu_L + g_{qS}\bar{q}q \right) - \frac{1}{2}m_S^2\phi^2\, ,
\end{equation}
where $m_S$ is the mediator's mass and $g_{\nu S}$, $g_{q S}$ represent couplings to neutrinos and quarks, respectively.
In general, the neutrino and quark couplings are independent. However, as the scattering rates will always take the form $g_{\nu_\ell S} g_{qS}$, for a given model we will assume $g_{\nu_\ell S} = g_{qS} \equiv g_{q\ell}$ where $\ell$ is the neutrino flavor and $q$ can be an $u$ or $d$ quark. In general, $q$ can also be a second or third generation quark, though as we discuss in Sec. \ref{sec:disc_reach}, DM detectors are primarily sensitive to couplings of first generation quarks. These couplings are not generation-independent and while the form of the cross-section is flavor-independent, the fluxes vary as discussed in Sec. \ref{sec:flux}. This means that for a comprehensive treatment of a scalar coupled to multiple flavors one would need to calculate the scattering of each flavor independently and combine the signals. 

In this work, we will consider either flavor independent couplings of a generalized scalar interaction, or what is often called a ``minimal coupling'' scenario, in which all couplings are zero except the one of interest that couples a particular flavor of neutrino with a specific quark. We adopt the latter in Sec.~\ref{sec:LQ} where we discuss a UV complete model. In the flavor-independent case, this leads to the differential cross-section
\begin{equation}
    \left.\frac{d\sigma_N}{dE_R}\right\vert_{S} = \frac{F^2(E_R) Q_S^2 E_R m_N^2}{4\pi E_\nu^2(2E_Rm_N + m_S^2)^2}~,
\end{equation}
where $m_N$ and $m_S$ are the nuclear and scalar masses and $Q_S \approx g_{\nu S} g_{q S} (14.7A - 3.3Z)$ is the effective coupling to a nucleus with atomic mass $A$ and number $Z$~\cite{Bertuzzo:2017tuf}.

The scalar's heavy mass allows us to integrate out the massive field and introduce an effective four-fermion interaction. Hence, in terms of conventional NSI formalism~\cite{10.21468/SciPostPhysProc.2.001} 
\begin{equation} \label{L_eff}
    \mathcal{L} \supset G_F\sum_{q,\ell}\epsilon_{q\ell}^S \bar{\nu}_\ell\nu_\ell \bar{q}q\, ,
\end{equation}
where the interaction is now a dimension six operator suppressed by two powers of the large mass scale. We parameterize the coupling constant relative to the weak force as
\begin{equation} \label{eps}
    \epsilon_{q\ell}^S = \frac{g_{\nu_\ell S} g_{qS}}{m_S^2 G_F}\, .
\end{equation}
%
Hence, we can probe and constrain a single parameter $\epsilon_{q\ell^S}$, which encapsulates mass and couplings of the model.

With this effective vertex, the CE$\nu$NS cross-section becomes

\begin{equation} \label{eq:n_cross}
    \left.\frac{d\sigma}{dE_R}\right\vert_{S} \hspace{0.5mm}  = \frac{F^2(E_R)g_{q\ell}^4 q_S^2 E_R m_N^2}{4\pi E_\nu^2 M_S^4} = \frac{F^2(E_R)G_F^2 \epsilon_{q\ell}^2 q_S^2 E_R m_N^2}{4\pi E_\nu^2}\, ,
\end{equation}

\noindent where $E_R$ is the energy of the recoiling quark and $q_S$ is the effective scalar charge of the nucleus. In the flavor/generation independent case, we have $q_S=Q_S/(g_{\nu S} g_{qS}) \approx 14.7A - 3.3Z$.
A sample of the differential recoil rate, both from the SM and NSI is shown in Fig. \ref{fig:recoil_rates}.

% Figure environment removed

\section{Case Study: Scalar Leptoquarks} \label{sec:LQ}

We will demonstrate how models of leptoquarks (LQs), which interact with baryons and leptons, provide an example application of our formalism. LQs are theoretically well-motivated and naturally arise in the context of Grand Unification theories such as $SU(5)$~\cite{Georgi:1974sy} and $SU(10)$~\cite{Fritzsch:1974nn,Georgi:1974my}, as well as the Pati-Salam models that unify quark and leptons~\cite{Pati:1974yy}. Further, they can manifest in $R$-parity violating models of supersymmetry~\cite{Hall:1983id,Barbier:2004ez}.

LQs have been recently connected to possible observational anomalies in semi-leptonic $B$ decays~\cite{HFLAV:2019otj,LHCb:2021trn,angelescu_closing_2018,Lee:2021jdr}. In addition, scalar LQs have been posed as solutions to the neutrino mass problem~\cite{Dorsner:2017wwn, PhysRevD.LQ3} and lepton flavor mixing. If the LQ couples to muons, it provides a contribution to the anomalous magnetic moment at one loop order and may explain the anomalous magnetic moment of the muon $(g-2)_\mu$~\cite{zhang_radiative_2021,dorsner_muon_2020, FileviezPerez:2021lkq,gherardi_low-energy_2021}.

Here, we consider LQs in the large coupling regime, which are are of interest for observations and provide a heavy scalar that couples to both quarks and neutrinos~\cite{Schmaltz:2018nls, DORSNER20161}. 
To map the LQ models to the NSIs discussed above, we make two observations. 
The first is that because the mediator carries lepton and quark quantum numbers, the interactions proceed via $s$ and $u$-channel diagrams rather than the $t$-channel of uncharged scalar interactions (see Fig.~\ref{fig:LQ_scattering}). Once the massive fields are integrated out, a Fierz transformation leads to the same effective four-fermion vertex. We derive the translated couplings in section~\ref{sec:couplings}. The second concerns the chirality of the outgoing neutrino. In the NSIs introduced previously, all neutrinos are left handed SM neutrinos, but in order to mediate the interaction with a LQ, the outgoing neutrino must be right-handed (RH). 

% Figure environment removed

\subsection{The \texorpdfstring{$S_1$}{S1} leptoquark}

We will focus on scalar LQs as the simplest realization of scalar NSIs. 
As a concrete realization, we consider the $S_1 = (\bar{3}, 1, 1/3)$ leptoquark, where parenthesis represent transformation under SM $\left( SU(3), SU(2), U(1)\right)$ symmetries. The additional relevant Lagrangian terms due to $S_1$ are~\cite{AristizabalSierra:2007nf, Babu:2019mfe, Babu:2020hun}: 
\begin{equation}\label{eq:S_1}
    \mathcal{L}_{S_1} \supset  -\frac{1}{2}m_{S_1}^2S_1^2+ \lambda_L^{i\ell} S_1 \bar{Q}^c_i \epsilon L_\ell + \lambda_R^{i\ell} S_1^* \bar{u}^i_R \bar{\ell}_R + \tilde{\lambda}^{i\ell}_R S_1^* \bar{d}^i_R \bar{\nu}_R^\ell + h.c.\, ,
\end{equation}
where $Q_i=(u_L,d_L)_i$ and $L_\ell=(\ell_L,\nu_{\ell L})$ are the left-handed quark and lepton doublets respectively, $\ell$ is the lepton flavor, $\epsilon$ is the Levi-Civita tensor, and $i$ is the quark generation index. From expansion of the $SU(2)$-doublets in Eq.~\eqref{eq:S_1}, we note that the same Yukawa interaction that couples charged leptons to up-type quarks will couple neutrinos to down-type quarks. Thus, by constraining neutrino scattering, we also place constraints on the charged lepton $S_1$ couplings. The same is not true of the coupling to right handed leptons as indicated by the tilde on the RH neutrino coupling, $\tilde{\lambda}^{i\ell}_R$, which distinguishes it from the charged RH lepton coupling, $\lambda_R^{i\ell}$. Throughout, we do not consider possible effects of diquark terms, which could lead to nucleon decay. Such terms can be readily restricted by additional symmetries, however detailed model implementation is beyond the scope of the present work.
  
As stressed in Refs.~\cite{Diaz:2017lit,Schmaltz:2018nls}, without a model-independent reason dictating a flavor pattern of LQ couplings all nine quark-lepton generation pairings are possible. Strong constraints exist on the first generation couplings, $\lambda^{qe}$, particularly from atomic parity violating experiments~\cite{Roberts:2014bka,Babu:2019mfe,Qweak:2018tjf,Schmaltz:2018nls}. Here, we shall focus on LQ couplings to the $\mu$ and $\tau$ flavored leptons. 
For simplicity, we will also assume the LQ couples only to one-generation (sometimes called the ``minimal leptoquark"), either $\mu$ or $\tau$ leptons.\footnote{The flavor changing constraints in this case are much weaker and both couplings may be simultaneously allowed as long as one is marginally smaller than the other.}

The interactions of $S_1$ LQ can lead to distinct phenomenological consequences.
Through the coupling to $\mu$-sector the $S_1$ LQ can directly contribute to the magnetic moment of the muon, via processes shown in Fig.~\ref{fig:g-2_diagram}.
For $m_{S_1} \gg m_t$ the resulting contribution to $a_\mu = (g-2)_\mu/2$ is~\cite{Bauer_muon}
\begin{equation}\label{eq:g-2}
    a^{S_1}_\mu = \sum_i \frac{m_\mu m_{u_i}}{4\pi^2m_{S_1}^2}\left(\ln{m_{S_1}^2/m_{u_i}^2} - 7/4\right)\Re(\lambda^R_{i2}\lambda^L_{i2}) - \frac{m_\mu^2}{32\pi^2m_{S_1}^2}\left(\sum_i |\lambda^R_{i2}|^2 + \sum_i |\lambda^L_{i2}|^2\right)\, ,
\end{equation}
which provides a target region of interest in the allowed $\lambda -m_{S_1}$ parameter space. 
% Figure environment removed

Since we are interested in coupling to protons and neutrons, we restrict ourselves to couplings to first generation of SM quarks. We are considering neutrino-nuclear scattering in or near the elastic regime, and so the parton distribution functions (PDFs) are dominated by the first generation quarks. This neglects the ``sea quarks'' which provide trace amounts of anti-quarks of the first generation as well as (even fewer) strange quarks and anti-quarks. In principle, these sea quarks provide an effective charge allowing detection of NSIs coupled to strange quarks or, as in the case of the $\tilde{R}_2$ LQ, NSIs which only couple neutrinos to anti-quarks (or quarks to anti-neutrinos), though the PDF suppression is such that neutrino scattering is not a competitive means to set bounds. Thus, we consider scalar $S_1$ LQ with couplings $\lambda^{q\ell}$, where $q=u,d$ and $\ell=\mu,\tau$. 

\subsection{From leptoquarks to NSI} \label{sec:couplings}

To map the $S_1$ LQ to the scalar NSI parameters, we need to find the expressions for $\epsilon^{q\ell}$ and $q_S$ in terms of the LQ parameters.
Since we are considering massive scalars of $\mathcal{O}$(TeV), we can integrate out the LQs from the previous subsection. Expanding out the $SU(2)$ doublet in Eq.~\eqref{eq:S_1}, integrating out the $S_1$ LQ, and taking the terms with neutrino interactions, we find 

\begin{equation}
    \mathcal{L} \supset \frac{\lambda_L^{d\ell}\tilde{\lambda}_R^{d\ell}}{2m_{S_1}^2}(\bar{d}^c_L\nu_{L\ell})(\bar{d}^c_R\nu_{R\ell})\equiv \frac{(\lambda^{d\ell})^2}{2m_{S_1}^2}(\bar{d}^c_L\nu_{L\ell})(\bar{d}^c_R\nu_{R\ell})\, ,
\end{equation}
where in the second equality
we define $(\lambda^{d\ell})^2\equiv \lambda_L^{d\ell}\lambda_R^{d\ell}$.  Rearranging fermion fields to match Eq.~\eqref{L_eff} and \eqref{eps} via a Fierz transform induces another factor of two and we find that 
\begin{equation}
    \epsilon^{d\ell}_{S_1} = \frac{(\lambda^{d\ell})^2}{4G_Fm_{S_1}^2}\, .
\end{equation}
For models with multiple scalars, we are sensitive to the sum of these NSI parameters. 
Note the interactions only involve down-type quarks, which is accounted for by a change in the effective scalar charge $q_S$. Following Eq. (3.7) of Ref.~\cite{Bertuzzo:2017tuf}, we have 
\begin{equation}
    q_{S, S_1} = \frac{1}{m_d}(Z m_p f_d^p + N m_n f_d^n)\approx 3.74 A - 0.75Z\, ,
\end{equation}
where $f_d^p = 0.0153$ ($f_d^n = 0.0191$) is the form factor capturing the low-energy coupling between the scalar and proton (neutron) for the down quark. Note that $q_{S,S_1}$ is a factor of $\sim 4$ smaller than the flavor-independent case. 

Lastly, we are considering the ``minimal LQ" model in which the $S_1$ couples to a single flavor of neutrinos, such that the neutrino-nuclear scattering signal only involves a single neutrino-flavor. Note that this is contrast to SM CE$\nu$NS, which involve all flavors.  

\subsection{Model extension and neutrino mass}

% Figure environment removed

There are several scalar LQs that can interact with neutrinos in SM at renormalizable level. Following Ref.~\cite{Babu:2019mfe}, we discuss a simple model extension of the $S_1$ LQ with a second scalar LQ, $\tilde{R}_2 = (3, 2, 1/6)$, that is an $SU(2)$ doublet, which can readily result in neutrino mass generation in conjunction with $S_1$. The considered LQs can arise in a variety of theoretical frameworks, such as models of $R$-parity violating supersymmetry~\cite{Hall:1983id,Barbier:2004ez}.
The additional Lagrangian terms due to $\tilde{R}_2$ are
\begin{equation}\label{R_2}
    \mathcal{L}_{\tilde{R}_2} \supset \lambda^{i\ell}\bar{d}^i_R\tilde{R}_2\epsilon L_\ell + \tilde{\lambda}^{i\ell}\bar{Q}^c_i\tilde{R}^\dagger_2\bar{\nu}_R^\ell + h.c. -\frac{1}{2}m_{\tilde{R}_2}^2\tilde{R}_2^2 + \mu H^{\dagger} \tilde{R}_2 S_1^{\ast} + h.c.~,
\end{equation}
where the $\mu$-term mixes $S_1$ and $\tilde{R}_2$ LQs and $H = (1,2,1/2)$ is the SM Higgs doublet.

At loop-level, $S_1$ and $\tilde{R}_2$ LQs generate neutrino mass, as displayed in Fig.~\ref{fig:nu_mass}.
Expanding the $SU(2)$ doublets, the necessary terms to generate neutrino NSI and masses are

\begin{equation}
    \mathcal{L}_{S_1, \tilde{R}_2} \supset \lambda^{i\ell} \left(\nu_\ell \bar{d}_i R^- - \ell \bar{d}_i R^+\right) + \lambda_L^{i\ell} \left(\nu_\ell d_i - \ell \bar{u}^c_i\right) S_1 - \mu R^- H^0 S_1^* + h.c.
\end{equation}
The mixing term implies the existence of a new LQ mass basis with eigenvalues
\begin{equation}
    m_{1,2}^2 = \frac{1}{2}\left(m_{R^+}^2 + m_{S_1}^2 \mp \sqrt{(m_{R^-}^2 - m_{S_1}^2)^2 + 4\mu^2v^2}\right)
\end{equation}
and a mixing angle 
\begin{equation}
    \tan(2\theta) = \frac{-\sqrt{2}\mu v}{m_{S_1}^2 - m_{R^-}^2}~,
\end{equation}
where $v$ is the Higgs vev and $m_{R^-,S_1}$ include the bare masses and the mass induced by spontaneous symmetry breaking. The neutrino mass matrix is given by

\begin{equation} \label{eq:m_nu}
    M_\nu = \frac{3\sin(2\theta)}{32\pi^2}\log\left(\frac{m_1^2}{m_2^2}\right) (\lambda M_d \lambda_L^T + \lambda_L M_d \lambda^T)
\end{equation}
where $M_d$ is a diagonal matrix of down-type quark masses. In the case of a LQ coupled only to down-type quarks and leptons, Eq.~\eqref{eq:m_nu} simplifies to 
\begin{equation}
    (M_\nu)^{\alpha \beta} = \frac{3\sin(2\theta)}{16\pi^2}\log\left(\frac{m_1^2}{m_2^2}\right) m_d \lambda_L^\alpha \lambda^\beta\, ,
\end{equation}
and similarly for LQs coupled to only strange or bottom quarks with $m_d \to m_{s,b}$. 

We note that with a multitude of parameters determining the neutrino mass above, there is degeneracy in LQ coupling and mass parameter space. From cosmological observations, the sum of neutrino masses given by the eigenvalues of $M_\nu$ is constrained to $\sum m_\nu < 0.13$ eV by cosmic microwave background radiation (CMB) data from Planck and the Dark Energy Survey (DES)~\cite{PhysRevD.107.023531}. 

\subsection{Neutrino oscillation effects}

In Sec. \ref{sec:flux} we gave the two-flavor approximation for neutrino oscillations in vacuum. In this work, we use the \texttt{NuOscProbExact} code~\cite{bustamante2019nuoscprobexact} to include the full three-flavor oscillations of solar neutrinos, which we propagate in vacuum along the baseline from the Sun to Earth, and of atmospheric neutrinos, where we also include the matter effects from propagation through the Earth. These neutrino oscillations thus contribute a significant flux of second- and third-generation neutrinos.

The introduction of NSI can also affect the neutrino oscillations discussed in Sec.~\ref{sec:flux} as well as neutrino propagation in medium. In the case of vector NSI, the contributions to the matter potential are such that the Hamiltonian is
\begin{equation}
    {\cal H_{\rm vNSI,matter}}\simeq E_\nu+\frac{MM^\dagger}{2E_\nu}\pm(V_{\rm SI}+V_{\rm NSI})\, ,
\end{equation}
where the neutrino mass matrix $M$, SM matter potential $V_{\rm SI}$, and the vector NSI matter potential contributions $V_{\rm NSI}$ are 3$\times$3 matrices. The matter potential becomes significant only when the neutrino energy, $E_\nu$, or matter density is large: $2E_\nu V\gtrsim \Delta m_{ij}^2$~\cite{Ge:2018uhz}.
 
In contrast, scalar NSI, which are the focus of this work, induce a correction to the neutrino mass matrix, which scales with the matter density. This leads to a neutrino oscillation probability that is sensitive to the matter density variations along the propagation baseline~\cite{Ge:2018uhz}. 
To see the effect of scalar NSI on neutrino oscillations,
we can write, in the flavor basis, $M_{eff} = M + M_{\rm NSI}$ where $M$ is the standard neutrino mass matrix and $M_{\rm NSI} = \sum_f n_f G_F \epsilon^S_{f\ell}$ with $\epsilon^S_{f\ell}$ defined as in Eq.~\eqref{eps} and $n_f$ is the fermion number density. We focus on flavor-diagonal NSIs, so $M_{\rm NSI}$ is diagonal in the flavor basis.\footnote{In general, $\epsilon^f_{\alpha\beta}$ are also possible.} Transforming to the mass basis and rotating the phasing matrix into the NSI term, $M_{eff}$ is given in Eq.~\eqref{eq:m_eff} where $P$ is the diagonal rephasing matrix.
\begin{equation}\label{eq:m_eff}
    M_{eff} = \mathcal{U}D_\nu\mathcal{U}^\dagger + P^\dagger M_{\rm NSI} P~.
\end{equation}
Following the notation of Ref.~\cite{Ge:2018uhz}, we take $\sqrt{|\Delta m^2_{31}|}$ as a characteristic mass scale and parameterize the NSI mass contribution as
\begin{equation}\label{eq:del_M}
    P^\dagger M_{\rm NSI} P = \delta M = \sqrt{|\Delta m^2_{31}|} (\eta)_{i,j}
\end{equation}
where $(\eta)_{i,j}$ are matrix elements in the mass basis.
For neutrinos propagating in matter with a scalar NSI, the effective Hamiltonian is 
\begin{equation}\label{eq:H_nsi}
    \mathcal{H}_{{\rm sNSI}, 
\rm matter} = E_\nu + \frac{M_{eff}M_{eff}^\dagger}{2E_\nu} + V_{\rm SI}
\end{equation}
where $V_{\rm SI}$ is the SM matter potential from charged current interactions. For antineutrinos, $V_{\rm SI}$ changes sign.

We estimate the impact of oscillations modified by NSI on our sensitivity to scalar mediators coupled to each neutrino flavor by considering a simplified approach as follows.
Though we focus on NSI that are diagonal in the flavor basis, the effect on oscillations is maximized for NSI that are minimal in the mass basis (\emph{i.e.} only one element of $\delta M$ is non-zero and that non-zero term is on the diagonal). We then consider a simplified model for the propagation considering for reference atmospheric neutrinos, based on a constant Earth density of $5.5~{\rm g/cm}^3$ and assuming an initially isotropic distribution of atmospheric neutrinos.\footnote{More detailed analysis can be performed using models of Earth such as PREM~\cite{DZIEWONSKI:1981297}.} 

We modify the~ \texttt{NuOscProbExact} code~\cite{bustamante2019nuoscprobexact} to account for the scalar NSI Hamiltonian as defined in Eq.~\eqref{eq:H_nsi}. As a simplifying approximation, we assume a spherically symmetric neutrino flux at the Super-Kamiokande site and average the oscillation probabilities assuming Normal Ordering by integrating over all propagation paths through the hemisphere containing the Earth. The resulting fraction of the atmospheric flux in each flavor is plotted in Fig.~\ref{fig:ratios} with and without effects of NSI. We note that NSI do not significantly alter the number of $\nu_\mu$ or $\nu_\tau$ relevant for statistical analysis, though it does provide additional features in the distributions that could improve expected sensitivity to NSI.

% Figure environment removed

For illustrative purposes, in Fig.~\ref{fig:ratios} (left panel) we display NSI effects that are significantly larger by multiple orders compared to the parameters for which we project sensitivity and constraints below in Sec.~\ref{sec:disc_reach}. Translating Eq.~\eqref{eq:del_M} to the $\epsilon$ NSI notation we employ throughout this work, we obtain $(P^\dagger M_{NSI} P)_{i,j} = P^\dagger(\sum_f n_f G_F \epsilon^f_{i,j})P = \sqrt{|\Delta m^2_{31}|} (\eta)_{i,j}$. Considering characteristic inputs, for the $\epsilon\lesssim 10^{-2}$ of interest assuming coupling to a single fermion (the down quark) that has an average number density of $\sim 10^{25}~{\rm cm}^{-3}$ and using $\sqrt{|\Delta m^2_{31}|}\approx0.05$~eV, neglecting the $\mathcal{O}(1)$ factors from the rephasing matrix, we obtain $\eta \sim 1.8\times10^{-11}$. At such small $\eta$ values the impact on the ratios of neutrino fluxes displayed in Fig.~\ref{fig:ratios} is for our purposes negligible. Combined with the minimal effect of even a large scalar NSI on the atmospheric neutrino flux, we neglect the effect of scalar NSI on neutrino oscillations in matter both for solar and atmospheric neutrinos in our analysis.\footnote{We note that scalar NSI also bare some relation to mass-varying neutrinos~\cite{Barger:2005mn}. For solar neutrinos we do not expect that the NSI variations we consider would lead to a violation of the adiabaticity condition over sizable range of neutrino energies, which are far smaller than those discussed in Ref.~\cite{Barger:2005mn}.
}

\section{Discovery Reach of Dark Matter Detectors}\label{sec:disc_reach}

We are interested in calculating the potential of a DM direct detection experiment to uncover signals of neutrino NSIs, with the SM CE$\nu$NS as our effective background. In the following analysis, we will present our results in terms of the general NSI parameters $\epsilon^S_{q\ell}$ and in terms of the LQ model parameters, $\lambda_{S_1}^{d\ell}, m_{S_1}$, from the previous section. For generality, we will consider that we are dominated by statistics and do not focus on making predictions in the context of more detector-specific systematic backgrounds, such as radioactive decay in or near the detector, or any other instrumental noise. 
We also do not consider the presence of a DM signal in this analysis, although Ref.~\cite{Bertuzzo:2017tuf} studied the effect of NSIs on the neutrino fog. 

\subsection{Statistical analysis}

Since we have reduced the NSI cross-section to one free parameter, $\epsilon_{q\ell}^S$, we use a likelihood ratio method to project the significance of simulated data. This follows the method used for the magnetic moment calculation in~Ref.~\cite{Schwemberger:2022fjl}, which was also dependent on only one parameter. We take as our likelihood function
\begin{equation}\label{likelihood}
    \mathcal{L}(\epsilon, \Vec{\phi}) = \frac{e^{-\Sigma_{j=1}^{n_\nu}(\eta_{\epsilon,j} + \eta_{SM,j})}}{N!} \times \prod_{j=1}^{n_\nu}\mathcal{L}(\phi_j) \times \prod_{i=1}^{N}\left(\sum_{j=1}^{n_\nu}\left( \eta_{\epsilon, j} f_{\epsilon,j}(E_i) + \eta_j f_{SM,j}(E_i) \right)\right)\, ,
\end{equation}
where we have dropped the labels on $\epsilon=\epsilon_{q\ell}^S$ for simplicity, $n_\nu$ label the neutrino fluxes and $N$ labels the energy bins. Here, $\Vec{\phi}$ are nuisance parameters normalizing the neutrino flux channels. They are allowed to fluctuate weighted by Gaussian distribution, 
\begin{equation}
\mathcal{L}(\phi_j)=\frac{1}{\sigma_j\sqrt{2\pi}}e^{-\frac{1}{2}\left(\frac{\phi_j-\phi_{0,j}}{\phi_j}\right)^2}    \, ,
\end{equation}
where the width of their uncertainties, $\sigma_j$ is given in Table~\ref{tab:nu_flux}. The functions $f_{\epsilon, j}(E_i)$ and $f_{SM, j}(E_i)$ are the distribution functions (\emph{i.e.} the normalized rates) for the number of events in the $i$-th energy bin from the $j$-th neutrino flux from NSI ($\epsilon$) or SM scattering. $\eta_{\epsilon, j}$ and $\eta_{SM, j}$ are the number of events predicted from the NSI or SM in each channel ($j$) based on the nuisance parameters. The latter depends only on the normalization of the neutrino flux components while the former depends on both the flux normalization and the NSI parameter ($\epsilon$). The product over $i$ runs over the number of events in the simulated data.

For a particular data set, we maximize the likelihood with $\epsilon=0$ as our null hypothesis and again with $\epsilon$ free as our signal. If we find a maximum for $\epsilon<0$ or the null likelihood is greater than the signal likelihood, we assign the significance $\sigma=0$. If $\mathcal{L}(\epsilon \neq 0)/\mathcal{L}(\epsilon = 0) \equiv \lambda > 1$, we have a test statistic $t = 2\log(\lambda)$ and significance $\sigma = \sqrt{t}$. This analysis is run on 600 of sets of pseudo-data generated by Poisson fluctuating the expected rates for a given $\epsilon_{q\ell}^S$ and exposure. The average of these data sets is then used for the projections in Sec.~\ref{sec:projections}.

\subsection{Results} \label{sec:projections}

% Figure environment removed

In this section, we present the results of the likelihood procedure described in the previous section. We show the 3$\sigma$ and 5$\sigma$ contours for $\epsilon$ in the exposure-$\epsilon$ plane in Fig.~\ref{fig:thresholds}, for a flavor universal scalar ($\epsilon_{q\ell}^S$, left) and a minimal scalar ($\epsilon_{d\ell}^S$, right). For the flavor universal scalar, we show the effects of varying the detector energy threshold. The yellow curves show the reach using the efficiency from the LZ experiment~\cite{LZ:2022ufs}. Improving the detector efficiency can lead to a significantly stronger reach, and we see that lowering the detector threshold from $E_{\rm th}=2$ keV to 1 keV improves the reach on $\epsilon$ by about a factor of few. Since $P_{e\to\tau} > P_{e\to\mu}$ when averaged over the solar neutrino spectrum, there will be $\sim 3$ $\nu_\tau$ for every 2 $\nu_\mu$, thus we expect $\tau$ constraints to be stronger by a factor $\sim \sqrt{3/2}$. This is confirmed by the results in Fig.~\ref{fig:thresholds} where we also show the results of an analysis of only atmospheric neutrinos which are useful to constrain interactions with anti-neutrinos such as the $\tilde{R}_2$ LQ. We further display how the results will improve as atmospheric neutrino flux uncertainty is further reduced, as far down as optimistic $\mathcal{O}(1\%)$-level.

% Figure environment removed
We also present our results in terms of the scalar LQ $S_1$ parameters, $\lambda$ and $m_{S_1}$, in Fig.~\ref{fig:LQ_reach} for a few select detector exposures. Our work, for the first time, shows the constraints from the LZ experiment in black. In addition, we show constraints from the LHC searches looking for LQ pair production (PP) and Drell-Yan (DY) production using 36 fb$^{-1}$ of data~\cite{PhysRevD.99.032014, sirunyan_search_2018}. As seen in Fig.~\ref{fig:LQ_reach}, the DM direct detection constraints are competitive with the LHC constraints for the $\lambda^{d\mu}$, but do not access the parameter region relevant for $(g-2)_\mu$ shown in gray. The DM direct detection constraints are significantly stronger than the LHC constraints for $\lambda^{d\tau}$, where the mass and short lifetime of the charged $\tau$ weaken the LHC constraints~\cite{Schmaltz:2018nls}, and the IceCube constraints coming from atmospheric neutrinos~\cite{PhysRevD.104.072006}. For both scenarios, there are constraints from perturbative unitarity~\cite{allwicher_perturbative_2021}, which provide the strongest constraints on $\lambda_{d\tau}$ for $m_{S_1}\gtrsim 3$ TeV.

In addition to xenon, other DM experiments can also have enhanced sensitivity to CE$\nu$NS and related new physics. In general, the effective charge of the nucleus grows as the square of its mass, so for a given detector target mass, larger nuclei are expected to result in more events even though there are fewer target nuclei. This competes with the kinematic effects whereby higher intensity but lower energy neutrinos are often pushed below threshold in detectors with heavy nuclei. 

In Fig.~\ref{fig:Ar_Pb_NSI_reach} we display the 3$\sigma$ and 5$\sigma$ discovery reach contours for the NSI parameter $\epsilon$ in the exposure-$\epsilon$ plane for argon and lead-based detectors. For estimating the sensitivity to scalar NSIs, while the details depend on the specific experimental configuration, we assume for simplicity the same threshold and efficiency parameters as for xenon-based detector.
Note that argon does not see much improvement as the threshold falls below 5 keV (the differences seen above are statistical noise). Argon has a relatively light nucleus and the peak induced by boron-8 is at higher energy, so lowering the threshold does not increase the number events as efficiently until the oxygen-16 neutrinos have an effect (below $\sim$ 0.5 keV). For the heavier lead nucleus, recoils induced by boron-8 neutrinos are partially cut by the threshold, so improving the threshold makes a much larger number of recoils accessible. The lower panels of Fig.~\ref{fig:Ar_Pb_NSI_reach} show the sensitivity to minimal couplings assuming the 2 keV threshold and efficiency.

In Fig.~\ref{fig:scalar_Y_Pb}, we translate our results into sensitivity projections for scalar LQ $S_1$ parameters, $\lambda$ and $m_{S_1}$, as before. We find that a detector constructed from either material, argon or lead, and reaching multi-ton-year exposure can be sensitive and improve existing bounds on the $S_1$ LQ couplings.

% Figure environment removed

% Figure environment removed

\section{Conclusions}
Direct DM detection experiments constitute a central pillar in the search for DM and will continue to increase their exposures and lower thresholds and backgrounds. As we demonstrate, this will allow such experiments to constrain or reveal new physics in the space of neutrino interactions, complementary to conventional neutrino telescopes. We set new limits on LQs associated with scalar neutrino NSI using latest data of direct DM detection experiments, in particular LZ. We find that near-future DM detectors can probe parameter space of LQs that is out of reach of the LHC. We discuss how upcoming measurements improving uncertainties for atmospheric neutrino could probe LQ and NSI parameter space. The studied LQs are of particular interest for interpreting the observations of $(g-2)_\mu$ and explaining neutrino masses. Our analysis can be extended to variety of other models and similar methodology could also be applied to vector-like LQs, which may be associated with measurements and anomalies in flavor physics.
Such a realization of massive scalar mediated NSIs provides a new benchmark for testing new physics beyond the SM at planned near-future experiments.

\acknowledgments
\addcontentsline{toc}{section}{Acknowledgments}

We thank Pouya Asadi, Kaladi Babu, Bhupal Dev, Bhaskar Dutta, Motoi Endo, Danny Marfatia for helpful discussions. T.S. and T.-T.Y. were supported in part by NSF CAREER grant PHY-1944826. T.-T.Y. also thanks the CCPP at NYU for hospitality and support. V.T.
was supported in part by the World Premier International Research Center Initiative (WPI), MEXT, Japan and also by JSPS KAKENHI grant No. 23K13109.

\bibliography{refs}
\addcontentsline{toc}{section}{Bibliography}
\bibliographystyle{JHEP}

\end{document}
