%%%%%%%%%%%%%%%%%%%%%%%%%%%%%%%%%%%%%%%%%%%%%%%%%%%%%%%%%%%%%%%%%%%%%
%% This is a (brief) model paper using the achemso class
%% The document class accepts keyval options, which should include
%% the target journal and optionally the manuscript type. 
%%%%%%%%%%%%%%%%%%%%%%%%%%%%%%%%%%%%%%%%%%%%%%%%%%%%%%%%%%%%%%%%%%%%%

\documentclass[journal=nalefd,manuscript=article,layout=traditional]{achemso}

%achemso template default was "journal=jacsat", "nalefd" is supposed to be Nano Letters

%%%%%%%%%%%%%%%%%%%%%%%%%%%%%%%%%%%%%%%%%%%%%%%%%%%%%%%%%%%%%%%%%%%%%
%% Place any additional packages needed here.  Only include packages
%% which are essential, to avoid problems later. Do NOT use any
%% packages which require e-TeX (for example etoolbox): the e-TeX
%% extensions are not currently available on the ACS conversion
%% servers.
%%%%%%%%%%%%%%%%%%%%%%%%%%%%%%%%%%%%%%%%%%%%%%%%%%%%%%%%%%%%%%%%%%%%%
\usepackage[version=3]{mhchem} % Formula subscripts using \ce{}

%%%%%%%%%%%%%%%%%%%%%%%%%%%%%%%%%%%%%%%%%%%%%%%%%%%%%%%%%%%%%%%%%%%%%
%% If issues arise when submitting your manuscript, you may want to
%% un-comment the next line.  This provides information on the
%% version of every file you have used.
%%%%%%%%%%%%%%%%%%%%%%%%%%%%%%%%%%%%%%%%%%%%%%%%%%%%%%%%%%%%%%%%%%%%%
%%\listfiles

%%%%%%%%%%%%%%%%%%%%%%%%%%%%%%%%%%%%%%%%%%%%%%%%%%%%%%%%%%%%%%%%%%%%%
%% Place any additional macros here.  Please use \newcommand* where
%% possible, and avoid layout-changing macros (which are not used
%% when typesetting).
%%%%%%%%%%%%%%%%%%%%%%%%%%%%%%%%%%%%%%%%%%%%%%%%%%%%%%%%%%%%%%%%%%%%%

% CROSS-REFERENCE FORMATTING
% For use with the cleveref package
% Define the format of Figure, Table, Equation, and Section cross-references in the text

% ACHEMSO PACKAGE FORMATTING OPTIONS
\SectionNumbersOn    % turn off section numbering
%\SectionsOff         % turn off section headers
%\AbstractOff         % turn off display of abstract

\usepackage[capitalize]{cleveref}
\crefname{figure}{Fig.}{Figs.}
\Crefname{figure}{Figure}{Figures}
\crefname{table}{Tab.}{Tabs.}
\Crefname{table}{Table}{Tables}
\crefname{equation}{Eq.}{Eqs.}
\Crefname{equation}{Equation}{Equations}
\crefname{section}{Sec.}{Secs.}
\Crefname{section}{Section}{Sections}

\renewcommand\thesection{S\arabic{section}}
\renewcommand{\thefigure}{S\arabic{figure}}
\renewcommand{\theequation}{S\arabic{equation}}
\renewcommand{\thetable}{S\arabic{table}}

\usepackage{subcaption}
\usepackage{siunitx, soul, xcolor}
\newsavebox{\measurebox}
\usepackage{gensymb}
\usepackage{indentfirst}

\usepackage[export]{adjustbox}

%select font attributes for coptions
\DeclareCaptionFont{capt}{\fontseries{n}\fontfamily{phv}\selectfont}
%naturally you can define more caption fonts with different attributes to use different fonts for different kind of captions, table, figure,...

%select attribute for figure caption
%\captionsetup[figure]{font=capt}
\captionsetup[subfigure]{labelfont=capt,position=top,singlelinecheck=false}

\newcommand{\bv}[1]{\mathbf{#1}}

\let\oldst\st
\renewcommand{\st}[1]{{\textcolor{blue}{\oldst{#1}}}}

%colors temporally for editing purposes 
\newcommand{\red}[1]{{\textcolor{red}{#1}}} % Professor
\newcommand{\green}[1]{{\textcolor{teal}{#1}}} % Shuyu
\newcommand{\blue}[1]{{\textcolor{blue}{#1}}} % Igor

\usepackage{subfiles} % Best loaded last in the preamble

%%%%%%%%%%%%%%%%%%%%%%%%%%%%%%%%%%%%%%%%%%%%%%%%%%%%%%%%%%%%%%%%%%%%%
%% Meta-data block
%% ---------------
%% Each author should be given as a separate \author command.
%%
%% Corresponding authors should have an e-mail given after the author
%% name as an \email command. Phone and fax numbers can be given
%% using \phone and \fax, respectively; this information is optional.
%%
%% The affiliation of authors is given after the authors; each
%% \affiliation command applies to all preceding authors not already
%% assigned an affiliation.
%%
%% The affiliation takes an option argument for the short name.  This
%% will typically be something like "University of Somewhere".
%%
%% The \altaffiliation macro should be used for new address, etc.
%% On the other hand, \alsoaffiliation is used on a per author basis
%% when authors are associated with multiple institutions.
%%%%%%%%%%%%%%%%%%%%%%%%%%%%%%%%%%%%%%%%%%%%%%%%%%%%%%%%%%%%%%%%%%%%%

\author{Shuyu Cheng}
\affiliation{Department of Physics, The Ohio State University, Columbus, Ohio 43210, United States}
\author{M. Nrisimhamurty}
\affiliation{Department of Physics, The Ohio State University, Columbus, Ohio 43210, United States}
\author{Tong Zhou}
\affiliation{Department of Physics, University at Buffalo, Buffalo, New York 14260, United States}
\author{N\'uria Bagu\'es}
\affiliation{Department of Materials Science and Engineering, The Ohio State University, Columbus, Ohio 43210, United States}
\author{Wenyi Zhou}
\affiliation{Department of Physics, The Ohio State University, Columbus, Ohio 43210, United States}
\author{Alexander J. Bishop}
\affiliation{Department of Physics, The Ohio State University, Columbus, Ohio 43210, United States}
\author{Igor Lyalin}
\affiliation{Department of Physics, The Ohio State University, Columbus, Ohio 43210, United States}
\author{Chris Jozwiak}
\affiliation{Advanced Light Source, Lawrence Berkeley National Laboratory, Berkeley, California 94720, United States}
\author{Aaron Bostwick}
\affiliation{Advanced Light Source, Lawrence Berkeley National Laboratory, Berkeley, California 94720, United States}
\author{Eli Rotenberg}
\affiliation{Advanced Light Source, Lawrence Berkeley National Laboratory, Berkeley, California 94720, United States}
\author{David W. McComb}
\email{mccomb.29@osu.edu}
\affiliation{Department of Materials Science and Engineering, The Ohio State University, Columbus, Ohio 43210, United States}
\author{Igor \v{Z}uti\'c}
\email{zigor@buffalo.edu}
\affiliation{Department of Physics, University at Buffalo, Buffalo, New York 14260, United States}
\author{Roland K. Kawakami}
\email{kawakami.15@osu.edu}
\affiliation{Department of Physics, The Ohio State University, Columbus, Ohio 43210, United States}

%%%%%%%%%%%%%%%%%%%%%%%%%%%%%%%%%%%%%%%%%%%%%%%%%%%%%%%%%%%%%%%%%%%%%
%% The document title should be given as usual. Some journals require
%% a running title from the author: this should be supplied as an
%% optional argument to \title.
%%%%%%%%%%%%%%%%%%%%%%%%%%%%%%%%%%%%%%%%%%%%%%%%%%%%%%%%%%%%%%%%%%%%%

\title{Supporting Information: Epitaxial Kagome Thin Films as a Platform for Topological Flat Bands}

%%%%%%%%%%%%%%%%%%%%%%%%%%%%%%%%%%%%%%%%%%%%%%%%%%%%%%%%%%%%%%%%%%%%%
%% Some journals require a list of abbreviations or keywords to be
%% supplied. These should be set up here, and will be printed after
%% the title and author information, if needed.
%%%%%%%%%%%%%%%%%%%%%%%%%%%%%%%%%%%%%%%%%%%%%%%%%%%%%%%%%%%%%%%%%%%%%

%%%%%%%%%%%%%%%%%%%%%%%%%%%%%%%%%%%%%%%%%%%%%%%%%%%%%%%%%%%%%%%%%%%%%
%% The manuscript does not need to include \maketitle, which is
%% executed automatically.
%%%%%%%%%%%%%%%%%%%%%%%%%%%%%%%%%%%%%%%%%%%%%%%%%%%%%%%%%%%%%%%%%%%%%

\begin{document}

\maketitle

\newpage

\tableofcontents

\newpage


\section{Methods}

 \subsection{Molecular Beam Epitaxy Growth of CoSn Thin Films}

The epitaxial growth of CoSn thin films was performed in an ultra-high vacuum chamber with a base pressure of 2$\times10^{-9}$\,Torr.
Prior to the growth, the MgO(111) or 4H-SiC(0001) substrates were degassed \textit{in-situ} at 500\,$^\circ$C for 20 minutes.
After annealing, the nucleation layer of 5\,nm CoSn was grown at 500\,$^{\circ}$C on 4H-SiC(0001) substrates, or at 470\,$^{\circ}$C on MgO(111) substrates.
After the growth of the nucleation layer, the sample was allowed to cool down to 100\,$^{\circ}$C in 60 minutes.
A $15-20$\,nm continuation layer was grown at 100\,$^{\circ}$C, followed by a temperature ramp to 300\,$^{\circ}$C at 12\,$^{\circ}$C/min.
Finally, a $5-10$\,nm terminating layer was grown on top of the continuation layer at 300\,$^{\circ}$C to improve the surface crystallinity.
The Co and Sn materials were evaporated from Knudson cells, and the typical growth rates were 0.78\,\AA/min~for Co, and 1.91\,\AA/min~for Sn, respectively.
The deposition rates were measured using a quartz crystal deposition monitor.

  \subsection{Scanning Transmission Electron Microscopy (STEM)}

The cross-sectional STEM samples were prepared by Ga ion milling using an FEI Helios Nano Lab 600 Dual Beam focused ion beam (FIB) operated at 30\,kV and 5\,kV.  
Final cleaning passes to remove any amorphous damage layers created in the FIB were performed in a Fischione Nanomill with 900\,V and then 500\,V Ar ions at Cryogenic temperature. 

HAADF-STEM imaging was performed using a probe-corrected Themis-ZTM at 300\,kV. 
Images were collected using the drift-corrected-frame-integration (DCFI) acquisition method within the Thermo Scientific Velox Software.

   \subsection{Angle-Resolved Photoemission Spectroscopy Measurements}

The ARPES experiments were performed at Beamline 7.0.2 (MAESTRO) of the Advanced Light Source (ALS). 
The samples were transferred from the growth chamber into the ultra-high vacuum (UHV) suitcase and shipped to the ALS, then transferred from the UHV suitcase into the ARPES system.
During the entire sample transfer procedure, the samples were kept under vacuum without exposure to air, to ensure a clean surface for ARPES experiments.
The ARPES experiments were performed at T = 6\,K. 
The photoemitted electrons were collected using a Scienta Omicron R4000 hemispherical electron analyzer, which provides energy and momentum resolution better than 30\,meV, and 0.01\,\AA$^{-1}$, respectively.



  \subsection{Density Functional Theory (DFT) Calculations}

The geometry optimization and electronic structure calculations were performed using the first-principles method based on density functional theory (DFT) with the projector-augmented-wave (PAW) formalism, as implemented in the Vienna \textit{ab initio} simulation package (VASP)~\cite{kresse1996efficient}. 
All calculations were carried out with a plane-wave cutoff energy of 550\,eV and 15$\times$15$\times$11 Monkhorst-Pack grids were adopted for the first Brillouin zone integral. 
The Perdew-Burke-Ernzerhof generalized-gradient approximation (GGA) was used to describe the exchange and correlation functional~\cite{perdew1996generalized}. 
The convergence criterion for the total energy is 10$^{-6}$\,eV. 
All the atoms in the unit cell are allowed to move until the Hellmann-Feynman force on each atom is smaller than 0.01\,eV/\AA. 
The lattice constants of CoSn are a = b = 5.275\,\AA~and c = 4.263\,\AA, taken from the experiments.

  \subsection{Transport Measurements}

The transport measurements were performed using a Quantum Design physical properties measurement system (PPMS) in DC resistivity mode.
The 35\,nm CoSn sample was patterned into 100\,$\mu$m-wide, 300\,$\mu$m-long Hall bar devices.
The longitudinal resistance was measured using 4-probe geometry.

\newpage

\section{RHEED and XRD of CoSn on MgO(111)}

% Figure environment removed

Figure~\ref{fig:RHEED_MgO} shows the \textit{in-situ} RHEED pattern of a 35\,nm CoSn sample grown on MgO(111) substrate.
The streaky RHEED pattern indicates epitaxial growth and two-dimensional surfaces with finite terrace width.
Figure~\ref{fig:XRD_MgO} shows the XRD data of the 35nm\,CoSn sample grown on MgO(111) substrate.
CoSn (0002) and (0004) peaks show up at 42.49\,$^\circ$ and 92.87\,$^\circ$, respectively.
The out-of-plane lattice constant extracted from the XRD scan is 4.252\,\AA, which is nearly identical to the lattice constant of CoSn grown on 4H-SiC(0001) substrates, and consistent with the previously reported values from bulk crystals~\cite{sales2021tuning} and sputtered thin films~\cite{thapaliya2021high}.


\newpage

\section{Additional STEM Data}

% Figure environment removed

Figure~\ref{fig:STEM_Supp} shows the atomic-resolution HAADF-STEM image of a CoSn(0001) thin film grown on 4H-SiC(0001) substrate viewed along 4H-SiC[1$\bar{1}$00] direction.
Across the film, the predominant stacking sequence is alternating stacking of one Co$_3$Sn layer and one Sn$_2$ layer, which is expected for CoSn (see the left panel of Figure~\ref{fig:STEM_Supp}).
However, in some regions near the interface, the stacking fault happens, resulting in a stacking sequence with additional Co$_3$Sn layers, as shown in the right panel of Figure~\ref{fig:STEM_Supp}.
Additionally, several-nanometer surface steps, the vertical shift of layer sequence, distortions, and contrast variations of the lattice are observed across the film suggesting the presence of defects and a complex nanostructure.

Using the lattice constant of the substrate as a reference for calibration, fast Fourier transform (FFT) analysis of the STEM image over a few nanometer scale yields lattice constants of $a$ = 5.36$\pm$0.24\,\AA~and $c$ = 4.25$\pm$0.12\,\AA~for CoSn.
The out-of-plane lattice constant $c$ agrees well with our XRD result of $c$ = 4.254\,\AA, and is consistent with previously reported values from bulk crystals~\cite{sales2021tuning} and thin films~\cite{thapaliya2021high}.
Meanwhile, the in-plane lattice constant $a$ is 1.5\% larger than the bulk value of $a$ = 5.279\,\AA~\cite{sales2021tuning}, but the difference is within the uncertainty of FFT results. 
Previous DFT calculations suggest that tensile strain shifts the flat bands downwards with respect to the Fermi level~\cite{kang2020topological}.
Along the direction of growth, there is no obvious change in the lattice constants observed in STEM, suggesting that the strain is relaxed within the first few nanometers of growth.

\newpage

\section{Additional ARPES Data}

    \subsection{ARPES Spectrum of the CoSn/MgO(111) Sample}

% Figure environment removed

Figure~\ref{fig:ARPES_MgO} shows the ARPES spectrum of the CoSn(0001) thin film grown on MgO(111) substrate.
Generally, the spectrum exhibits similar features as compared to the CoSn(0001) thin film grown on 4H-SiC(0001) substrates.
In the $k_z=\pi$ (mod 2$\pi$) plane, FB1 is observed around the H points, 0.03\,eV below the Fermi level, while FB2 spreads over almost the entire BZ with a binding energy of about -0.3\,eV (Figure~\ref{fig:FB_97eVLH_MgO}).
In the $k_z=0$ (mod 2$\pi$) plane, only FB2 is observed at 0.3\,eV below Fermi level.
Compared to the CoSn/4H-SiC(0001) sample, the differences in binding energies of individual flat bands are within the resolution of the experiment ($\sim$15\,meV).

    \subsection{k$_z$ Dependence of ARPES Spectrum}

% Figure environment removed

To determine the high-symmetry planes of the BZ, we measured the ARPES spectrum along the $\Gamma$-A-$\Gamma$ direction by varying the photon energies, as shown in Figure~\ref{fig:kz}.
Along $\Gamma$-A-$\Gamma$ direction, the binding energy of FB2 shows periodic variation, with highest energies corresponding to $\Gamma$ points, and lowest energies corresponding to A points~\cite{kang2020topological}.

    \subsection{Polarization Dependence of ARPES Spectrum}

% Figure environment removed
    
Figure~\ref{fig:Polarization} shows the polarization dependence of the ARPES spectrum of CoSn(0001) on SiC(0001).
In the $k_z = \pi$ (mod $2\pi$) plane, FB1 can be seen using s-polarized photons (Figure~\ref{fig:FB_97eVLH_S}), while it is not visible in the spectrum using p-polarized light (Figure~\ref{fig:FB_97eVLV}).
In the $k_z = 0$ (mod $2\pi$) plane, s-polarized light mostly highlights the upper branch of FB2 at the K points (Figure~\ref{fig:FB_128eVLH_S}), while p-polarized light mostly highlights the lower branch of FB2 (Figure~\ref{fig:FB_128eVLV}).
At the A and $\Gamma$ points, FB2 shows negligible polarization dependence. 
  
The polarization dependence of ARPES spectrum originates from the matrix element effect, which is sensitive to the orbital nature of the bands~\cite{hufner2013photoelectron}. 
The observation above is consistent with the DFT results that FB1 and the upper branch of FB2 mainly originate from the $d_{xy}$ and $d_{x^2-y^2}$ orbitals, while the lower branch of FB2 mainly originates from the $d_{xz}$ and $d_{yz}$ orbitals.

    \subsection{``Flatness" of the Flat Bands}

% Figure environment removed

To examine the ``flatness" of FB1, we extracted the energy distribution curves (EDCs) from the red box of Figure~\ref{fig:97eV_HLH}.
By fitting the EDCs with the product of the Lorentzian function and Fermi-Dirac distribution, we found that the binding energy of FB1 is around -0.04\,eV, as shown in Figure~\ref{fig:EDC_Stack_H}.
The effective mass $m^{*}$ is then calculated from the formula:
\begin{equation}
\frac{1}{m^{*}}=\frac{1}{\hbar^2}\frac{d^2E}{dk^2} \label{eq1}
\end{equation}
After fitting in the peak positions versus the corresponding momentum with a quadratic relationship, the $m^{*}$ is determined to be 16.7\,$m_0$ along L-H direction, where $m_0$ is the mass of free electrons.
The enhancement of effective mass comes from the dispersionless nature of FB1.

\begin{table*}[h]
\begin{tabular*}{\textwidth}{c @{\extracolsep{\fill}} ccccc}
 Position & A & L & $\Gamma$ & M \\\hline
 Binding Energy (eV) & -0.31 & -0.32 & -0.28 & -0.29 \\
 FWHM (eV) & 0.20 & 0.28 & 0.22 & 0.33 \\
\end{tabular*}
\caption{\label{tab:table1} Summary of binding energies and FWHM of FB2 at high-symmetry points.}
\end{table*}

To examine the ``flatness" of FB2, we took EDCs from representative high-symmetry points, as shown in Figure~\ref{fig:EDC_FB2}.
Using Lorentzian fitting, we extracted the binding energies and the full width at half maximum (FWHM) of FB2 at the  A, L, $\Gamma$, and M point, and summarize them in Table~\ref{tab:table1}.
At all these points, FB2 is located at $\sim$0.3\,eV below the Fermi level, and has an FWHM ranging between 0.20\,eV and 0.33\,eV.
The variation of binding energy is within 0.03\,eV, signifying that the FB2 is non-dispersive across almost the entire BZ.

\newpage

\section{Additional DFT Calculations}

We performed additional DFT calculations with several representative doping levels varying from 1.0$\times$10$^{22}$\,holes/cm$^3$ to 1.0$\times$10$^{22}$\,electrons/cm$^3$, as shown in Figure~\ref{fig:DFT_Supp}.

% Figure environment removed

\newpage

\section{Estimation of Longitudinal Conductivity and Hall Conductivity}

The calculation of transport properties was performed by numerical integration on a three-dimensional grid with 0.01\,\AA$^{-1}$ step size.
By assuming an isotropic relaxation time $\tau_{i}$ for each band, we calculated $\displaystyle\frac{\sigma_{xx}^{(i)}}{\tau_{i}}$, $\displaystyle\frac{\sigma_{zz}^{(i)}}{\tau_{i}}$, and $\displaystyle\frac{\sigma_{H}^{(i)}}{\tau_{i}^2}$, which only depend on the band dispersion $\displaystyle\varepsilon^{(i)}(\textbf{k})$:
\begin{equation}\label{eq:equationS1}
\frac{\sigma_{xx}^{(i)}}{\tau_{i}}=-\frac{e^2}{4\pi^3\hbar^2}\int\left(\frac{\partial\varepsilon}{\partial k_{x}}\right)^{2}\frac{\partial f}{\partial \varepsilon}d\textbf{k}
\end{equation}
\begin{equation}\label{eq:equationS2}
\frac{\sigma_{zz}^{(i)}}{\tau_{i}}=-\frac{e^2}{4\pi^3\hbar^2}\int\left(\frac{\partial\varepsilon}{\partial k_{z}}\right)^{2}\frac{\partial f}{\partial \varepsilon}d\textbf{k}
\end{equation}
\begin{equation}\label{eq:equationS2}
\frac{\sigma_{H}^{(i)}}{\tau_{i}^2}=-\frac{e^3}{4\pi^3\hbar^4}\int\left[\left(\frac{\partial\varepsilon}{\partial k_{x}}\right)^{2} \left(\frac{\partial^{2}\varepsilon}{\partial k_{y}^{2}}\right) -\left(\frac{\partial\varepsilon}{\partial k_{x}}\right)\left(\frac{\partial\varepsilon}{\partial k_{y}}\right)\left(\frac{\partial^{2}\varepsilon}{\partial k_{x}\partial k_{y}}\right)\right]\frac{\partial f}{\partial \varepsilon}d\textbf{k}
\end{equation}
The results are summarized in Table~\ref{tab:tableS2}.
Taking typical relaxation times of 10$^{0}\sim$10$^{1}$\,fs in metals~\cite{palenskis2018phonon}, the in-plane longitudinal conductivity $\sigma_{xx}$ is on the order of 10$^{5}\sim$10$^{6}$\,S/m, which agrees with the experimental result of 9.5$\times$10$^{5}$\,S/m.
Furthermore, the calculation results suggest that the out-of-plane longitudinal conductivity $\sigma_{zz}$ is likely to be more than 1 order of magnitude larger than the in-plane conductivity $\sigma_{xx}$ for bands II, III, and IV.
This is consistent with the observed large anisotropic conductivity in bulk CoSn~\cite{huang2022flat}.

\begin{table*}[h]
\begin{tabular*}{\textwidth}{c @{\extracolsep{\fill}} ccccc}
 Band index                     & I    & II    & III (FB1)   & IV  \\\hline
 $\sigma_{xx}$/$\tau$ (10$^3$\,/($\Omega\cdot$m$\cdot$fs))        & 122 & 1.3  & 0.8  & 9.4   \\
 $\sigma_{zz}$/$\tau$ (10$^3$\,/($\Omega\cdot$m$\cdot$fs))        & 849 & 1035  & 537 & 11.3    \\
 $\sigma_{H}$/$\tau^2$ (m/(($\Omega^{2}\cdot$C$\cdot$fs$^{2}$)))    & -16  & 0.7 & -0.02  & -2.4      \\
\end{tabular*}
\caption{\label{tab:tableS2} Calculated transport properties from band structures.}
\end{table*}

\newpage

\bibliography{CoSn.bib}

\end{document}