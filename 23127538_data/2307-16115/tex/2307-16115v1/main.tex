\documentclass[conference]{IEEEtran}
% \IEEEoverridecommandlockouts
% The preceding line is only needed to identify funding in the first footnote. If that is unneeded, please comment it out.
%\usepackage{cite}
\usepackage{amsmath,amssymb,amsfonts}
\newtheorem{theorem}{Theorem}
\newtheorem{example}{Example}
\newtheorem{proposition}{Proposition}
\usepackage{framed,subfigure}
\usepackage{graphicx}
\usepackage{textcomp}
\usepackage{multirow}
\usepackage{xcolor}
\usepackage{makecell}
\usepackage{hyperref}
\usepackage{paralist}
\usepackage{fullpage}
\usepackage{times}
\usepackage{fancyhdr,graphicx,amsmath}
\usepackage[ruled,vlined,linesnumbered]{algorithm2e}
\usepackage{bm}
\usepackage{float}
\let\oldthebibliography=\thebibliography
\let\endoldthebibliography=\endthebibliography
\renewenvironment{thebibliography}[1]{%
\begin{oldthebibliography}{#1}%
    \setlength{\parskip}{0\baselineskip}% change this value to adjust
    \setlength{\itemsep}{0\baselineskip}% change this value to adjust
}%
{%
\end{oldthebibliography}%
}
  \setlength{\textfloatsep}{0.3ex}
    \setlength{\intextsep}{0.3ex}
    \setlength{\dbltextfloatsep}{0ex}
    \addtolength{\topskip}{-1mm}
    \addtolength{\parskip}{-0.1mm}
 \setlength{\lineskip}{-0.24mm}
   \addtolength{\abovedisplayskip}{-2mm}
     \addtolength{\belowdisplayskip}{-1.9mm}
     \addtolength{\itemsep}{-1mm}
    \addtolength{\textheight}{6mm}
    \addtolength{\topmargin}{-3mm}
       \addtolength{\leftmargin}{8mm}
 %         \addtolength{\rightmargin}{8mm}
	% \setlength{\abovecaptionskip}{1mm}
	%\setlength{\belowcaptionskip}{-1mm}
   
%\def\BibTeX{{\rm B\kern-.05em{\sc i\kern-.025em b}\kern-.08em
%    T\kern-.1667em\lower.7ex\hbox{E}\kern-.125emX}}
\begin{document}

\title{IWEK: An Interpretable What-If Estimator for Database Knobs} 

\author{\IEEEauthorblockN{Yu Yan, Hongzhi Wang, Jian Geng, Jian Ma,Geng Li, Zixuan Wang, Zhiyu Dai, Tianqing Wang}
\IEEEauthorblockA{ \textit{Harbin Institute of Technology Harbin, China ; Huawei, China} \\
yuyan@hit.edu.cn,wangzh@hit.edu.cn\\} }

%yuyan@hit.edu.cn,wangzh@hit.edu.cn,1253791041@qq.com,1161906117@qq.com,wangyuzh@hit.edu.cn,21B903037@stu.hit.edu.cn
% wangtianqing2@huawei.com


%\vspace{-5mm}
\maketitle

\begin{abstract}
    The knobs of modern database management systems have significant impact on the performance of the systems. With the development of cloud databases, an estimation service for knobs is urgently needed to improve the performance of database. Unfortunately, few attentions have been paid to estimate the performance of certain knob configurations. To fill this gap, we propose IWEK, an interpretable \& transferable what-if estimator for database knobs. To achieve interpretable estimation, we propose linear estimator based on the random forest for database knobs for the explicit and trustable evaluation results. Due to its interpretability, our estimator capture the direct relationships between knob configuration and its performance, to guarantee the high availability of database.  We design a two-stage transfer algorithm to leverage historical experiences to efficiently build the knob estimator for new scenarios. Due to its lightweight design, our method can largely reduce the overhead of collecting training data and could achieve cold start knob estimation for new scenarios. Extensive experiments on YCSB and TPCC show that our method performs well in interpretable and transferable knob estimation with limited training data. Further, our method could achieve efficient estimator transfer with only 10 samples in TPCC and YSCB.
\end{abstract}

\section{Introduction}
\label{introduction}
    \section{Introduction}
Current quantum hardware is unable to carry out universal quantum computations due to the buildup of errors that occur during the computation. 
The magnitude of the individual error is currently above the value that the Threshold Theorem requires in order to kick-start quantum error correction and fault-tolerant quantum computation~\cite[Section 10.6]{nielsen_chuang_2010}. 
Although the experimentally achieved fidelity rates are promising and the error bounds are inching closer to the required threshold, we will have to work for the foreseeable future with quantum hardware with errors that build-up during the computation.  This implies that we can only do a limited number of steps before the output of the computation has become completely uncorrelated with the intended one.

For fault-tolerant quantum computing, we repeat four steps: 
1) We apply a number of single and two-qubit quantum gates, in parallel whenever possible; 
2) We perform a syndrome measurement on a subset of the qubits; 
3) We perform fast classical computations to determine which errors have occurred and how to correct them; 
and, 4) We apply correction terms based on the classical computations.
We then repeat these four steps with a next sequence of gates. 
These four steps are essential to fault-tolerant quantum computing. 


The starting point of this work is to use the four steps outlined above, not to carry out error correction and fault-tolerant computation, but to enhance short, constant-depth, {\em uncorrected} quantum circuits that perform single qubit gates and {\em nearest-neighbor} two qubit gates. 
Since in the long run we will have to implement error-correction and fault-tolerant computation anyhow, and this is done by such a four-step process, why not make other use of this architecture? Moreover, on some of the quantum hardware platforms, these operations are already in place.
Embracing this idea we naturally arrive at the question: what is the computational power of \textit{low-depth} quantum-classical circuits organized as in the four steps outlined above? 
We thus investigate circuits that execute a small, ideally constant, number of stages, where at each stage we may apply, in parallel, single qubit gates and {\em nearest-neighbor} two qubit gates, followed by measurements, followed by low-depth classical computations of which the outcome can control quantum gates in later stages. 
It is not clear, at first, whether such circuits, especially with constant depth, can do anything remotely useful. 
But we will see that this is indeed the case: many quantum computations can be done by such circuits in constant depth. 
By parallelizing quantum computations in this way, we improve the overall computational capabilities of these circuits, as we do not incur errors on qubits that are idle, simply because qubits are not idle for a very long time. 
Furthermore, reducing the depth of quantum circuits, at the cost of increasing width, allows the circuit to be run faster even if errors occur.

The first usage of such a four-step layout, not to do error correction, but to perform computations, can be found in the paradigm of measurement-based quantum computing~\cite{gottesman1999demonstrating,raussendorf2001one,jozsa2006introduction,clark2007generalised}: 
A universal form of quantum computing where a quantum state is prepared and operations are performed by measuring qubits in different bases, depending on previous measurements and intermediate measurements.

\citeauthor{PhamSvore2013} were the first to formalize the four-step protocol for performing computations~\cite{PhamSvore2013}. They included specific hardware topologies by considering two-dimensional graphs for imposing constraints on qubit interactions. In their model, they develop circuits for particularly useful multi-qubit gates, including specifying costs in the width, number of qubits, depth, number of concurrent time steps, size, and total number of non-Identity operations.
As a result, they find an algorithm that factors integers in polylogarithmic depth.
\citeauthor{Browne:2011} showed that the main tool in the work by \citeauthor{PhamSvore2013}, the fan-out gate, can also be replaced by additional log-depth classical computations in the measurement-based quantum computing setting~\cite{Browne:2011}.

More recently, \citeauthor{Cirac:2021} introduced a scheme to implement unitary operations involving quantum circuits combined with Local Operations and Classical Communication ($\mathsf{LOCC}$) channels: $\mathsf{LOCC}$-assisted quantum circuits~\cite{Cirac:2021}. Similarly to the four-step scheme we just described, they allow for a short depth circuit to be run on the qubits, followed by one round of $\mathsf{LOCC}$, in which ancilla qubits are measured and local unitaries are applied based on the measurement outcomes. They show that in this model any 1D transitionally invariant matrix-product state (MPS) with fixed bond dimension is in the same phase of matter as the trivial state. Similar ideas can be found in~\cite{TVV_NonAbelianTopologicalOrder_2022, tantivasadakarn2021long}.

In this work, we introduce a new model, called \textit{Local Alternating Quantum-Classical Computations} ($\LAQCC$). In this model we alternate between running quantum circuits (constrained by locality), ending in the measurement of a subset of qubits, and fast classical computations based on the measurement results. The outcome of the classical computations are then used to control future quantum circuits. We allow for flexibility in this model, by giving different constraints to the power of both the quantum circuits and the classical circuits as well as the number of alternations between them. 
Most attention will be given to $\LAQCC$ containing quantum circuits of constant depth, classical circuits of logarithmic depth and at most a constant number of alternations between them. 
Any circuit constructed in this model is considered to be of constant depth. 
We restrict ourselves to logarithmic depth classical computations, as this is the first natural and non-trivial extension beyond constant-depth classical computations. 
Constant-depth classical computations do however also have an equivalent constant-depth quantum implementation.

The definition of $\LAQCC$ sharpens the original definition of \citeauthor{PhamSvore2013} by adding constraints to the intermediate classical computations. This allows us to bound the power of $\LAQCC$ from above. 

The main result of \citeauthor{Cirac:2021}, that 1D translational invariant MPS with fixed bond dimension can be prepared by $\mathsf{LOCC}$-assisted circuits, relies on local symmetries of the MPS. These symmetries allow them to prepare local states (on a constant number of qubits) and glue them together by doing one round of the appropriate entangling measurement and corrections, after which they run a round of local unitaries to get the desired result. This general scheme for preparing states that exhibit an MPS description with the appropriate local symmetries requires only geometrically local unitaries and one round of measurement and corrections an therefore is accessible in $\LAQCC$. Studying different local symmetries, known as Symmetry Protected Topological (SPT) phases of matter, to find measurement-based constant depth circuits for states is a broad ongoing field of research~\cite{TVV_NonAbelianTopologicalOrder_2022, tantivasadakarn2021long, smith2023deterministic}. 
All these schemes have a $\LAQCC$ implementation.

%$\LAQCC$-circuits also exist for general schemes of preparing local states, based on the local tensors, and gluing them together using one round of entangled measurement and corrections, based on the local symmetry. 
%The main result of \citeauthor{Cirac:2021}, that 1D translational invariant MPS with fixed bond dimension can be prepared by $\mathsf{LOCC}$-assisted circuits, relies heavily on local symmetries of the MPS and as a result also has an equivalent $\LAQCC$ implementation. 
%The corrections applied after the measurement round are local unitaries depending on the local symmetries of the MPS. 

 

%This general scheme of preparing local states, based on the local tensors, and gluing it together by doing one round of entangled measurement and corrections, based on the local symmetry, is accessible in $\LAQCC$.
Note however that \citeauthor{Cirac:2021} also suggest a circuit for the $W$-state.
This circuit uses sequentially and dependent measurement-based corrections of the ancilla qubits. 
These dependent measurements translate to sequential alternations between the quantum and classical circuits and therefore increase the total depth to linear depth, exceeding the constant-depth constraints imposed by $\LAQCC$-circuits. 

We study the power of the $\LAQCC$ model with respect to state preparation, showing that even with only constant quantum-depth and logarithmic classical depth it remains possible to prepare states with long-range entanglement.
Another surprising result is that it is unlikely that $\LAQCC$ circuits are classically simulatable. We show that any instantaneous quantum polynomial-time (IQP) circuit~\cite{Bremner2010,Shepherd2009} has an $\LAQCC$ implementation.
Classical simulation of IQP circuits implies the collapse of the polynomial hierarchy to the third level, which is not believed to be true~\cite{Bremner2017}. Therefore, we expect that $\LAQCC$ circuits are unlikely to be classically simulatable. We bound the power of $\LAQCC$ by showing that it is contained in $\QNC^1$, the class of polynomial-size, log-depth circuits.

Next, we also study the power that intermediate classical calculations can add to quantum computations, by considering a new model that alternates between polynomially many polynomial-depth quantum circuits and unbounded classical computations
We study this model by doing a complexity theoretical analysis, where we draw inspiration from the notions of complexity given by \citeauthor{RosenthalYuen:2022}, \citeauthor{MetgerYuen:2023}, and \citeauthor{Aaronson:2004}.
All three complexity notions are based on the notion of state preparation, instead of more traditional definition of complexity such as the decidability of a computational problem. 
The first two consider classes based on sequences of quantum states preparable by a polynomial-sized quantum circuit, where the circuits are uniformly generated by a computational class, for instance, the class $\mathsf{PSPACE}$, which results in the complexity class $\mathsf{StatePSPACE}$~\cite{RosenthalYuen:2022,MetgerYuen:2023}.
The third notion considers a relative complexity, where the complexity is measured between two given states, and is measured by the number of gates, from a given gate-set, required to transform one state in another state~\cite{Aaronson:2004}. 
For our definition of state preparation complexity, we drop the uniformity constraint from~\cite{RosenthalYuen:2022,MetgerYuen:2023} and define a class as $\mathsf{StateX}$, which refers to states preparable by circuits of type $\mathsf{X}$. 
As an example, if $\mathsf{X} = \QNC^0$, this results in the class $\mathsf{StateQNC^0}$, which is the set of states preparable from the $\ket{0}^n$ state by poly-size constant-depth circuits. 
This notion is similar to the relative complexity from~\cite{Aaronson:2004}, where one state is the  $\ket{0}^n$ state and instead of counting the number of gates we consider the set of states preparable by a fixed number of gates. Using this notion of complexity we show that any state preparable by an $\LAQCC^*$ circuit is also preparable by a $\mathsf{PostQPoly}$ circuit, the class of circuits of polynomial depth with an additional post-selection gate. 

All Clifford circuits have a constant-depth $\LAQCC$ implementation, implying that any stabilizer state can be implemented by a constant-depth $\LAQCC$ circuit, see Section~\ref{sec:clifford_circuits} for a proof of this statement. 
Efficient circuits for stabilizer states have been known already through measurement-based quantum computing. Therefore this paper focuses on the preparation of non-stabilizer states, and as a surprising result we find novel constant-depth protocols for four very natural classes of non-stabilizer states.
Despite the extensive research into these four classes of non-stabilizer states and the many applications of them, no efficient constant- or low-depth state preparation protocols are known yet. We specifically consider these four classes as they are all often used as initial states in other algorithms.

The first state is a uniform superposition over an arbitrary number of states. 
This state finds applications in many quantum algorithms, as they often start with a uniform superposition over multiple states. 
This superposition is often achieved by applying Hadamard gates to every qubit due to its simplicity to prepare. 
Yet, the analysis of many algorithms, such as Shor's algorithm~\cite{Shor:1997}, would benefit from a different initial superposition. 
The circuit to prepare the uniform superposition over an arbitrary number of states uses an exact version of Grover search as a subroutine, that turns a probabilistic circuit, with a known constant probability of success, into a deterministic circuit. 
We use the circuit for preparing a uniform superposition over an arbitrary number of states as a subroutine in the next two quantum state preparation protocols. 

The second state is the $W$-state, the uniform superposition over all computational basis states of Hamming-weight~$1$, a natural long-ranged entangled state that displays a fundamentally nonequivalent type of entanglement from the Greenberger–Horne–Zeilinger state~\cite{WState:2000}, for which $\LAQCC$-type constant-depth circuits were previously known~\cite{PhamSvore2013, Cirac:2021}. 
The $W$-state is often used as benchmark for new quantum hardware~\cite{Haffner2005,Neeley2010,GarciaPerez:2021}. 
A novel way to prepare the $W$-state therefore gives a new way to benchmark different quantum devices with each other. 
A circuit for preparing the $W$-state was given in~\cite{Cirac:2021}, but this implementation requires sequentially alternating measurements followed by local unitaries, which in the $\LAQCC$ model is not considered to be of constant depth. 
We improve this protocol by giving an $\LAQCC$ implementation of the $W$-state, based on a compress-uncompress method that links the one-hot and binary encoding of integers.

The third state considered is the Dicke state, a generalization of the $W$-state, a superposition over all computational basis states with Hamming-weight $k$~\cite{Dicke:1954}. 
Dicke states have relevance in various practical settings.
For instance, for quantum game theory~\cite{zdemir2007}, quantum storage~\cite{Bacon_Compress:2006,Plesch:2010}, quantum error correction~\cite{ouyang2014permutation}, quantum metrology~\cite{toth2012multipartite}, and quantum networking~\cite{prevedel2009experimental}. 
Dicke states have been used as a starting state for variational optimization algorithms, most notably Quantum Alternating Operator Ansatz (QAOA)~\cite{Hadfield2019}, to find solutions to problems such as Maximum k-vertex Cover~\cite{Brandhofer2022,cook2020quantum}.
The ground states of physical Hamiltonians describing one-dimensional chains tend to show a resemblance to Dicke states such as states resulting from the Bethe ansatz, making them an ideal starting state when investigating the ground state behavior of these Hamiltonians~\cite{TDL_BetheAnsatzDerivation:2010,B_ExcitedStateQuantumPhaseTransitions:2013,DickeTransitions:2021}. 
For instance, the algorithm by \citeauthor{van2021preparing}, who give an algorithm to prepare the Bethe ansatz eigenstates of the spin-1/2 XXZ spin chain, starts by first preparing a Dicke state~\cite{van2021preparing}. 
A Dicke-state preparation protocol based on the compress-uncompress methodology used in the $W$-state furthermore finds applications in entanglement distillation, where the entanglement of a large state is concentrated on only a few qubits. 
Efficient deterministic circuits for preparing Dicke states have been proposed by \citeauthor{bartschi2019deterministic}~\cite{bartschi2019deterministic, bartschi2022deterministic_short_depth}. 
They provide a quantum circuit of depth $\mathO(k \log(\frac{n}{k}))$, allowing arbitrary connectivity, to prepare a Dicke state, which they conjecture to be optimal when $k$ is constant. 
In this work, we provide a constant-depth $\LAQCC$ circuit below their conjectured bound already for constant $k$. 
However, this does not directly disprove their conjecture, as we allow for intermediate measurements and classical computations. 
More significantly, we even construct constant-depth $\LAQCC$ circuits for $k = \mathO(\sqrt{n})$ greatly improving their bound.
This construction extends the compress-uncompress method for the $W$-state combined with additional subroutines. 

We continue with a log-depth state preparation protocol for the Dicke-state for arbitrary $k$. 
This protocol implements an efficient transformation between the factoradic number representation and the combinatorial number representation of a positive integer. 
The combinatorial number representation relates directly to the Dicke state. 
The provided efficient transformation between number representation systems might be of independent interest. 

We conclude by modifying our protocol for preparing a Dicke-state to a protocol that prepares quantum many-body scar states in constant-depth. 
These states have low entanglement and longer coherence times than states with similar energy density.
These characteristics make many-body scar states interesting to analyze and relevant within physics.
Many-body scar states appear for instance in the AKLT model~\cite{AKLT:1987,MRBAR:2018,MRB:2018} and different spin models~\cite{SI:2019,MOBFR:2020}.
Known methods for preparing these states have polynomial-depth~\cite{Gustafson:2023}, whereas our circuit has constant depth. 

% We conclude by studying the power that intermediate classical calculations can add to quantum computations. 
% In this study, we define a new model that relaxes constant-depth quantum circuits to polynomial depth quantum circuits, log-depth classical calculations to unbounded classical computations and a constant number of alternations to a polynomial number of alternations. 
% We call this model $\LAQCC^*$. 
% We study this model by doing a complexity theoretical analysis, where we draw inspiration from the notions of complexity given by \citeauthor{RosenthalYuen:2022}, \citeauthor{MetgerYuen:2023}, and \citeauthor{Aaronson:2004}.
% All three complexity notions are based on the notion of state preparation, instead of more traditional definition of complexity such as the decidability of a computational problem. 
% The first two consider classes based on sequences of quantum states preparable by a polynomial-sized quantum circuit, where the circuits are uniformly generated by a computational class, for instance, the class $\mathsf{PSPACE}$, which results in the complexity class $\mathsf{StatePSPACE}$~\cite{RosenthalYuen:2022,MetgerYuen:2023}.
% The third notion considers a relative complexity, where the complexity is measured between two given states, and is measured by the number of gates, from a given gate-set, required to transform one state in another state~\cite{Aaronson:2004}. 
% For our definition of state preparation complexity, we drop the uniformity constraint from~\cite{RosenthalYuen:2022,MetgerYuen:2023} and define a class as $\mathsf{StateX}$, which refers to states preparable by circuits of type $\mathsf{X}$. 
% As an example, if $\mathsf{X} = \QNC^0$, this results in the class $\mathsf{StateQNC^0}$, which is the set of states preparable from the $\ket{0}^n$ state by poly-size constant-depth circuits. 
% This notion is similar to the relative complexity from~\cite{Aaronson:2004}, where one state is the  $\ket{0}^n$ state and instead of counting the number of gates we consider the set of states preparable by a fixed number of gates. Using this notion of complexity we show that any state preparable by an $\LAQCC^*$ circuit is also preparable by a $\mathsf{PostQPoly}$ circuit, the class of circuits of polynomial depth with an additional post-selection gate. 

\paragraph{Summary of results}
\begin{itemize}
    \item We give a new definition of a computational model that captures the power of the four step process: applying a constant number of layers of one- and two-qubit gates; performing a syndrome measurement; perform a fast classical computation determining corrections; apply corrections. We call this model \emph{Local Alternating Quantum Classical Computations}, or $\LAQCC$ for short. In this model we bound the allowed quantum operations, intermediate classical calculations, and number of rounds separately. In Section~\ref{sec:LAQCC_model} we define this model and give a list of operations based on results from literature contained in this computational model. In some of these operations we explicitly use that we allow for multiple, but at most constant, rounds  of corrections.
    \item  We show show that there exist $\LAQCC$ circuits that can not be weakly simulated in Section~\ref{sec:IQP_in_LAQCC}. We further show that for every $\LAQCC$ circuit there exists a $\QNC^1$ circuit simulating it perfectly, in Section~\ref{sec:LAQCC_in_QNC1}.
    \item We introduce a new type computational complexity for preparing states and show that the extension of $\LAQCC$ where we allow a polynomial number of rounds and unbounded classical computation, is contained in $\mathsf{PostQPoly}$, the class of polynomial circuits with post-selection, in Section~\ref{sec:Complexity results}.
    \item We show a protocol to prepare the uniform superposition state of size $q$ in $\LAQCC$ using $\mathO(\ceil{\log_2(q)}^2)$ qubits in Section~\ref{sec:superposition_modulo_q}. 
    \item We show a protocol to prepare the $W_n$ state in $\LAQCC$ using $\mathO(n\log(n))$ qubits in Section~\ref{sec:W_state_in_LAQCC}.
    \item We show two ways of preparing the Dicke-$(n,k)$ state. The first method is in $\LAQCC$, works up to $k = \mathO(\sqrt{n})$, uses $\mathO(n^2\log(n))$ qubits, and is found in Section~\ref{sec:dicke:small_k}. The second method is in $\LAQCC\text{-}\mathsf{LOG}$ (an extension of $\LAQCC$ allowing for logarithmic number of alterations instead of constant), works for any $k$, uses $\mathO(\text{poly}(n))$ qubits, and is found in Section~\ref{sec:Dicke_in_LAQCC_LOG}. 
    \item We extend on our $\LAQCC$ method of generating Dicke-$(n,k)$ states for $k = \mathO(\sqrt{n})$ and show a protocol to generate many-body scar states for a particular Hamiltonian in $\LAQCC$ (Section~\ref{sec:many_body_scar}). 
\end{itemize}
Summarized in a table, we provide the following state generation protocols:
\begin{table}[htb]
\centering
\begin{tabular}{l|l|l|l}
\textbf{State description} & \textbf{Width} & \textbf{Depth} & \textbf{Implementation}\\
\hline 
Uniform superposition mod $q$: $\frac{1}{\sqrt{q}} \sum_{i = 0}^{q-1}\ket{i}$ & $\mathO(\ceil{\log^2 q})$ & $\mathO(1)$ & Section~\ref{sec:superposition_modulo_q}\\

$W$-state: $\frac{1}{\sqrt{n}}\sum_{i = 0}^{n-1}\ket{e_i}$ & $\mathO(n \log n)$ & $\mathO(1)$ & Section~\ref{sec:W_state_in_LAQCC}\\

Dicke-$(n,k)$, $k = \mathO(\sqrt{n})$: $\binom{n}{k}^{-1/2}\sum_{x \in \{0,1\}^n: |x| = k} \ket{x}$ &  $\mathO(n^2\log n)$ & $\mathO(1)$ 
&Section~\ref{sec:dicke:small_k}\\

Dicke-$(n,k)$: $\binom{n}{k}^{-1/2}\sum_{x \in \{0,1\}^n: |x| = k} \ket{x}$ & $\mathO(\text{poly}(n))$ & $\mathO(\log n)$ &Section~\ref{sec:Dicke_in_LAQCC_LOG}\\

QMBS: $\ket{S_k} = \frac{1}{k! \sqrt{\mathcal N(n,k)}}(Q^\dagger)^k \ket{\Omega}$ &  $\mathO(n^2\log n)$ & $\mathO(1)$  &  Section~\ref{sec:many_body_scar}
\end{tabular}
\caption{Summary of state preparation protocols given in this paper.}
\label{tab:sate_prep}
\end{table}
In the entry for the quantum many-body scar state $Q$ denotes the raising operator and $\mathcal N(n,k)=\binom{n-k-1}{k}$. 
Section~\ref{sec:many_body_scar} will provide more details on the variables and the implementation. 

\paragraph{Organization of the paper}
\noindent We first introduce relevant preliminaries in Section~\ref{sec:preliminaries}. 
In Section~\ref{sec:LAQCC_model} we formally define the class of Local Alternating Quantum-Classical Computations ($\LAQCC$). We also show that any Clifford circuit can be implemented in constant depth $\LAQCC$ (a result based on a result from measurement-based quantum computing~\cite{jozsa2006introduction}). 
This result allows us to give many useful multi-qubit gates and routines in Section~\ref{sec:gates_created_in_LAQCC}. 
Beyond that we show that constant depth $\LAQCC$ circuits are contained in $\QNC^1$ and that any $\mathsf{IQP}$ circuit has an $\LAQCC$ implementation.
We conclude this section with an analysis of a more powerful instantiation of $\LAQCC$ and show an inclusion with respect to the class $\mathsf{PostQPoly}$, which is the class of circuits of polynomial depth with one additional post-selection gate. 
In Section~\ref{sec:state_prep_in_LAQCC} we give $\LAQCC$ circuit implementations for preparing the uniform superposition over an arbitrary number of states, the $W$-state and the Dicke state up to $k = \mathO(\sqrt{n})$. We furthermore give a log-depth circuit implementation for preparing the Dicke state for any $k$. We conclude by showing a $\LAQCC$ circuit for generating many body scar states of a particular type of Hamiltonian.






\section{Overview}
\label{sec:model}
%!TEX root = ./robust_pe.tex
%\newpage

\section{Robust Policy Evaluation as Policy Optimization}\label{sec_pe_as_po}
This section adopts a policy optimization viewpoint towards policy evaluation, by first formulating the robust policy evaluation problem as a Markov decision problem of the nature.
We then identify a few key structures of the formulated MDP that will prove useful for our subsequent development. 

Consider a MDP of nature, denoted by $\mathfrak{M}$,  defined as follows.
The state space is given by $\cS$, and
the set of possible actions  at any state $s \in \cS$ is given by $ \cD_s \subseteq \Delta_{\cS}^{\abs{\cA}}$ (cf. \eqref{def_ambiguity_set_structure}).
%We will write $\DD = [\DD_{a_1}, \ldots, \DD_{a_{\abs{\cA}}}]$ for any $\DD \in \Delta_{\cS}^{\abs{\cA}}$.
%Equivalently, any possible action $\DD \in \cD_s$ specifies $\abs{\cA}$ elements in $\Delta_{\cS}$,  each denoted as $\DD_{a}$ for every $a \in \cA$.
At state $s \in \cS$, upon making an action $\DD \in \cD_s$, the conditional distribution of the next state $s' \in \cS$ is given by 
\begin{align}\label{transit_kernel_of_nature_mdp}
\mathfrak{P}(s' | s, \DD) \coloneqq  \tsum_{a \in \cA} \sbr{(1-\zeta) \overline{\PP}_{s,a}(s')  + \zeta \DD_{a}(s')} \vartheta(a|s).
\end{align}
%Clearly, the above transition kernel $\mathfrak{P}$ is affine with respect to the action of the nature  $\DD$. 
Finally, the cost function associated with $(s, \DD)$ for any $\DD \in \cD_s$ is given by 
\begin{align*}
\mathfrak{C}(s) \coloneqq - \tsum_{a \in \cA} \vartheta(a|s) c(s,a),
\end{align*}
 and the discount factor is set as $\gamma$.

A non-randomized policy of the nature is denoted as $\pi: \cS \to \cD_s$. 
%It is clear that $\pi$ uniquely determines $\DD^{\pi} \in \cD$ defined in \eqref{eq_cD_set}.
%Let us denote $\DD^{\pi(s)} = \pi(s)$, and 
%Let us define $\DD^{\pi(s)}  = \pi(s)$ for any policy $\pi$.
For any policy $\pi$ and $s \in \cS$, let us define $\DD^{\pi(s)} \equiv \pi(s) \in \cD_s$.
For notational clarity we will sometimes use these two quantities interchangeably. 
We then define
\begin{align}\label{kernel_defined_by_nature_policy}
\PP^{\pi}_{s,a} \coloneqq (1-\zeta) \overline{\PP}_{s,a}  + \zeta \DD^{\pi(s)}_{a}, ~ (s,a) \in \cS \times \cA.
\end{align}
%as the transition kernel of the original planner when the nature's policy is $\pi$.
Consequently, from \eqref{transit_kernel_of_nature_mdp} it holds that 
\begin{align}\label{state_transit_of_nature_given_policy}
\mathfrak{P}(s'|s,  \pi(s)) =  \tsum_{a \in \cA} \sbr{(1-\zeta) \overline{\PP}_{s,a}(s')  + \zeta \DD^{\pi(s)}_{a}(s')} \vartheta(a|s)
= \tsum_{a \in \cA} \PP^{\pi}_{s,a}(s') \vartheta(a|s).
\end{align}
We define the value function of policy $\pi$ as 
\begin{align*}
%\label{eq_def_value_func_nature}
\cV^{\pi} (s) \coloneqq 
\EE \sbr{\tsum_{t=0}^\infty \gamma^t \mathfrak{C}(s_t) \big| s_0 = s, s_{t+1} \sim \mathfrak{P}(\cdot| s_t, \pi(s_t) ) , t \geq 0}, ~~ \forall s \in \cS,
\end{align*}
and the goal of the nature is to find the optimal policy of 
\begin{align}\label{eq_def_optmal_value_nature}
\textstyle
\min_{\pi \in \Pi} \cV^{\pi} (s),
\end{align}
where $\Pi: s \mapsto \cD_s$ is the set of non-randomized stationary policies of the nature.

\begin{lemma}\label{lemma_value_correspondence}
For any $\pi \in \Pi$, we have 
\begin{align}\label{eq_nature_value_as_player_value}
\cV^{\pi}(s) = - V^{\vartheta}_{\PP^{\pi}}(s), ~ \forall s \in \cS,
\end{align}
with $V^{\vartheta}_{\PP^{\pi}}$ defined in \eqref{eq_standard_value_function}.
In addition, let $\cV^*$ denote the optimal value function of \eqref{eq_def_optmal_value_nature}.
Then 
\begin{align}\label{nature_opt_as_robust_value}
\cV^*(s) = - V^{\vartheta}_r(s), ~ \forall s \in \cS,
\end{align}
where $V^{\vartheta}_r$ is defined as in \eqref{eq_def_robust_value}.
\end{lemma}

\begin{proof}
It is clear that the $\cV^{\pi}$ satisfies the following dynamic programming equation 
\begin{align*}
\cV^{\pi} (s) & = \mathfrak{C}(s) + \gamma \tsum_{s' \in \cS} \mathfrak{P}(s'|s,  \pi(s)) \cV^{\pi}(s') \\
 & = - \tsum_{a \in \cA} \vartheta(a|s) c(s,a)
 + \gamma  \tsum_{s' \in \cS} \tsum_{a \in \cA} \PP^{\pi}_{s,a}(s') \vartheta(a|s) \cV^{\pi}(s'), ~ \forall s \in \cS.
\end{align*}
where the last equality follows from \eqref{state_transit_of_nature_given_policy}.
The above relation implies that $- \cV^{\pi}$ is the fixed point of operator 
\begin{align*}
(\cT^{\pi} V)(s) = 
\tsum_{a \in \cA} \vartheta(a|s) c(s,a)
 + \gamma  \tsum_{s' \in \cS} \tsum_{a \in \cA} \PP^{\pi}_{s,a}(s') \vartheta(a|s) V(s'), ~ \forall s \in \cS.
\end{align*}
On the other hand, it is known that $V^{\vartheta}_{\PP^{\pi}}$ is the unique fixed point of $\cT^{\pi}$, from which
we obtain \eqref{eq_nature_value_as_player_value}.
% That is, the value function of the nature's policy $\cV^{\pi}$ corresponds to the negative value function of the policy $\pi$ within $\cM_{\PP^{\pi}}$.
In addition,  Bellman optimality condition of MDP \eqref{eq_def_optmal_value_nature} yields  
\begin{align*}
\cV^*(s) & = \min_{\DD \in \cD_s} \mathfrak{C}(s)  + \gamma \tsum_{s' \in \cS} \mathfrak{P}(s' |s, \DD) \cV^*(s') \\ 
& = \min_{\DD \in \cD_s} -  \tsum_{a \in \cA} \vartheta(a|s) c(s,a) 
+ \gamma \tsum_{s' \in \cS} \tsum_{a \in \cA}\sbr{(1-\zeta) \overline{\PP}_{s,a}(s')  + \zeta \DD_{a}(s')}\vartheta(a|s) \cV^*(s') \\
& = \min_{\PP \in \cP_s} -  \tsum_{a \in \cA} \vartheta(a|s) c(s,a) 
+ \gamma \tsum_{s' \in \cS} \tsum_{a \in \cA} \PP_{a}(s') \vartheta(a|s) \cV^*(s') , ~ \forall s \in \cS.
%\\
%& =  - \tsum_{a \in \cA} \vartheta(a|s) c(s,a) 
%+ \gamma  \tsum_{a \in \cA}   \vartheta(a|s)  \tsum_{s' \in \cS} \min_{\PP_{s,a} \in \cP_{s,a}} \PP_{s,a}(s') \cV^*(s'), ~ \forall s \in \cS.
\end{align*}
Clearly, $-\cV^*$ is the fixed point of operator 
\begin{align}\label{def_robust_ballmen_op}
(\cT V)(s) = \max_{\PP \in \cP_s} \tsum_{a \in \cA} \vartheta(a|s) c(s,a) 
+ \gamma \tsum_{s' \in \cS} \tsum_{a \in \cA} \PP_{a}(s') \vartheta(a|s) V(s'), ~ \forall s \in \cS. 
\end{align}
On the other hand, it is well known that $V^{\vartheta}_{r}$ is the unique fixed point of $\cT V$ \cite{wiesemann2013robust}. 
Consequently we obtain \eqref{nature_opt_as_robust_value}.
%.
%That is, the optimal value function of the nature \eqref{eq_def_value_func_nature} corresponds to the negative robust value function $V^{\pi}_r$ of the policy,
%as both are the (unique) solution of the above dynamic programming equation.
\end{proof}


In view of the above observations, the robust policy evaluation problem \eqref{eq_def_robust_value} can be equivalently solved by solving a Markov decision process \eqref{eq_def_optmal_value_nature}  of nature with finite state space and continuous action space. 
To this end, let us define the following problem:
\begin{align}\label{policy_opt_obj_nature}
\textstyle
\min_{\pi \in \Pi} \cbr{f(\pi) \coloneqq \EE_{s \sim \rho} \sbr{\cV^{\pi}(s)}},
\end{align}
where $\rho$ is any distribution with full support over $\cS$.
Our end goal is to develop efficient methods that can be applied to solve \eqref{policy_opt_obj_nature}.


The state-action value function of the nature, also know as the Q-function, is defined by
\begin{align}\label{def_q_func_nature}
\cQ^{\pi}(s, \DD) & \coloneqq 
\EE \sbr{\tsum_{t=0}^\infty \gamma^t \mathfrak{C}(s_t) \big| s_0 = s, s_1 \sim   \mathfrak{P}(\cdot| s, \DD ), s_{t+1} \sim \mathfrak{P}(\cdot| s_t, \pi(s_t) ), t \geq 1 } . 
%\\
%& =  \mathfrak{C}(s) + \gamma \EE_{s' \sim   \mathfrak{P}(\cdot| s, \DD )} \sbr{\cV^{\pi}(s')} \\
%& = \mathfrak{C}(s) +  
% \gamma \tsum_{s' \in \cS} \tsum_{a \in \cA}\sbr{(1-\zeta) \overline{\PP}_{s,a}(s')  + \zeta \DD_{s,a}(s')}\vartheta(a|s) \cV^{\pi}(s')  \\
%& =  \mathfrak{C}(s) +  
% \gamma  \tsum_{a \in \cA} \vartheta(a|s) 
% \inner{(1-\zeta) \overline{\PP}_{s,a} + \zeta \DD_{s,a}}{\cV^{\pi}} \\
% & = 
% \mathfrak{C}(s) + \gamma \inner{(1-\zeta) \overline{\PP}_s + \zeta \DD}{\cV^{\pi}_{\vartheta, s}}, 
\end{align}
Clearly one also has 
\begin{align}\label{relation_q_and_v}
\cQ^\pi(s,\DD) = \mathfrak{C}(s) + \gamma \EE_{s' \sim \mathfrak{P}(\cdot | s,\DD)} \sbr{\cV^{\pi}(s')}.
\end{align}
%for any $ \DD \in \cD_s$, 
%where in the last equality we define $\cV^{\pi}_{\vartheta, s} = \vartheta(\cdot|s) \otimes \cV^{\pi} \in \RR^{\abs{\cA} \abs{\cS}}$.
We next show that $Q^{\pi}(s, \cdot)$ is indeed an affine function over $\cD_s$, an immediate yet important property that we will exploit in the ensuing development. 

\begin{proposition}\label{prop_q_structure}
For any $\DD \in \cD_s$, we have 
\begin{align*}
\cQ^{\pi}(s, \DD)
=  \mathfrak{C}(s) + \gamma \inner{(1-\zeta) \overline{\PP}_s + \zeta \DD}{\cV^{\pi}_{\vartheta, s}}, 
\end{align*}
where $\cV^{\pi}_{\vartheta, s} \coloneqq \vartheta(\cdot|s) \otimes \cV^{\pi} \in \RR^{ \abs{\cS} \abs{\cA}}$.
%and $\overline{\PP}_s \coloneqq [\overline{\PP}_{s, a_1}, \ldots, \overline{\PP}_{s, a_{\abs{\cA}}}] \in \Delta_{\cS}^{\abs{\cA}}$.
\end{proposition}

\begin{proof}
We have 
\begin{align*}
\cQ^{\pi}(s, \DD)
& =  \mathfrak{C}(s) + \gamma \EE_{s' \sim   \mathfrak{P}(\cdot| s, \DD )} \sbr{\cV^{\pi}(s')} \\
& = \mathfrak{C}(s) +  
 \gamma \tsum_{s' \in \cS} \tsum_{a \in \cA}\sbr{(1-\zeta) \overline{\PP}_{s,a}(s')  + \zeta \DD_{a}(s')}\vartheta(a|s) \cV^{\pi}(s')  \\
& =  \mathfrak{C}(s) +  
 \gamma  \tsum_{a \in \cA} \vartheta(a|s) 
 \inner{(1-\zeta) \overline{\PP}_{s,a} + \zeta \DD_{a}}{\cV^{\pi}} \\
 & = 
 \mathfrak{C}(s) + \gamma \inner{(1-\zeta) \overline{\PP}_s + \zeta \DD}{\cV^{\pi}_{\vartheta, s}},
\end{align*}
which completes the proof.
\end{proof}

%\yan{need to define $d_{\rho}^{\pi}$ within the perf diff lemma}
Our ensuing discussions repeatedly make use of the discounted visitation measure, defined as $d_{\rho}^{\pi}(s) = (1-\gamma) \tsum_{s' \in \cS} \rho(s') \tsum_{t=0}^\infty \gamma^t \mathtt{P}^{\pi}(s_t = s| s_0=s')$ for every $s \in \cS$, where $\mathtt{P}^{\pi}(s_t = s| s_0=s')$ denotes the probability of reaching state $s$, if running $\vartheta$ starting from $s'$ within MDP $\cM_{\PP^\pi}$.
In particular, we write $d_{s}^{\pi}$ when $\rho$ has support $\cbr{s}$. 
We next establish the difference of values for two policies of nature. 

\begin{lemma}\label{lemma_perf_diff}
For a pair of policies $\pi, \pi'$, and any $s\in \cS$,  we have
\begin{align}\label{eq_perf_diff}
\cV^{\pi'}(s) - \cV^{\pi}(s) = \frac{1}{1-\gamma}
\EE_{s' \sim d_{s}^{\pi'}} \sbr{
\cQ^{\pi}(s', \pi'(s')) - 
\cQ^{\pi}(s', \pi(s'))
}
\end{align}
Equivalently, by defining $\phi^{\pi}( s, \pi'(s)) 
%\coloneqq \cQ^{\pi}(s, \pi'(s)) - 
%\cQ^{\pi}(s, \pi(s)) 
\coloneqq \gamma \zeta \inner{\pi'(s) - \pi(s)}{\cV^{\pi}_{\vartheta, s}}$, then 
\begin{align}\label{eq_perf_diff_linearized}
\cV^{\pi'}(s) - \cV^{\pi}(s) = \frac{1}{1-\gamma}
\EE_{s' \sim d_{s}^{\pi'}} \sbr{\phi^{\pi}(s', \pi'(s'))}
\end{align}
\end{lemma}


\begin{proof}
%The proof of \eqref{eq_perf_diff} follows standard steps of performance difference lemma for finite MDPs \cite{lan2021policy, kakade2002approximately} and hence is omitted here.
%\yan{need to expand on this one}
%Let $\xi_(s)$ denote the 
Let $\xi'(s) = \cbr{s_0 = s, \pi'(s_0), s_1, \pi'(s_1), \ldots} $ denote the trajectory generated by $\pi'$ within $\mathfrak{M}$. 
That is 
\begin{align*}
s_{t+1} \sim \mathfrak{P}(\cdot|s_t, \pi'(s_t)),
\end{align*}
or equivalently, in view of \eqref{state_transit_of_nature_given_policy}, that
\begin{align}\label{state_transition_distribution_equivalence}
s_{t+1} \sim \tsum_{a \in \cA} \vartheta(a|s_t) \PP^{\pi'}_{s_t,a} (\cdot) .
\end{align}
We then obtain 
\begin{align*}
\cV^{\pi'}(s) - \cV^{\pi}(s)
& \overset{(a)}{=} \EE_{\xi'(s)} \sbr{\tsum_{t=0}^\infty \gamma^t \rbr{ \mathfrak{C}(s_t) + \gamma \cV^{\pi}(s_{t+1}) - \cV^{\pi}(s_t)}  + \cV^\pi(s_0) }   - \cV^\pi(s)  \\
& \overset{(b)}{=} \EE_{\xi'(s)} \sbr{\tsum_{t=0}^\infty \gamma^t \rbr{ \cQ^{\pi}(s_t, \pi'(s_t)) - \cV^{\pi}(s_t)}  } \\
& \overset{(c)}{=} \frac{1}{1-\gamma} \EE_{s' \sim d_s^{\pi'}} \sbr{\cQ^{\pi}(s', \pi'(s')) - \cV^\pi(s')},
\end{align*}
where $(a)$ follows from the definition of $\cV^{\pi'}(s)$, 
 $(b)$  follows from $s_0 = s$ and \eqref{relation_q_and_v},
and $(c)$ follows from \eqref{state_transition_distribution_equivalence} and the definition of $d_s^{\pi'}$.
Then \eqref{eq_perf_diff} follows  by noting that $\cV^{\pi}(s) = \cQ^\pi(s, \pi(s))$. 
Finally, \eqref{eq_perf_diff_linearized} follows from \eqref{eq_perf_diff} and Proposition \ref{prop_q_structure}.
\end{proof}

Interested readers might find the formulated MDP of nature challenging upon initial examination. 
In particular, as nature has a continuous action space, even evaluating the state-action value function \eqref{def_q_func_nature} seems to be challenging, a crucial quantity for policy improvement. 
It is also unclear whether one should and how to parameterize the policy of nature. 
In the next section, we proceed to develop the first-order robust policy evaluation (FRPE) method that exploits the structural properties established in this section and overcomes the aforementioned difficulties.
























\section{Estimation Model}\label{sec:rank}

In this section, we introduce the knob estimator in detail. To avoid the influence of explosive knob space, we select the important knobs at first with the adaptive knob ranking (AKR) shown in Section~\ref{sec:ranking} and build the interpretable linear knob estimator (IKE) based on random forest in Section~\ref{sec:estimate}. 


\subsection{Adaptive Knob Ranking}\label{sec:ranking}
Since only a subsets of adjustable knobs have a great impact on the certain estimation task, we design an adaptive knob ranking mechanism to eliminate knobs with less impact on the performance. The knob ranking task is defined as follows:

Given a knob set $K$, a K-P training set for a certain scenario $S$, $D= [(x_1, y_1), ...(x_n,y_n)]$, where $x_i$ is certain knob configuration, such as $[knob1 = a, knob2 = b...,]$, and $y_i$ is the performance on $S$ with $x_i$ as the knob configuration. The goal of knob ranking is to train a model from $D$ to compute the importance of knobs to filter out the important knobs of $K$ for knob estimation.

Specifically, the knob ranking faces one significant trade-off between the accuracy and the time consumption of training data. On the one hand, ranking is a sub-important prior task for database in which we should save resource utilization as possible. On the other hand, it may lead to model underfitting to learn the ranking result of multiple knobs with limited training data. Existing ranking methods~\cite{hutter2014efficient} which depend on fixed model could not efficiently resolve the above trade-off. 

\begin{algorithm}
    \SetAlgoNoLine
    \LinesNumbered
    \SetKwInOut{Input}{input}
    \Input{
        Set of regression models($M$), Dataset($D$), Knob candidates ($K$) \\
    }
    \SetKwInOut{Output}{output}
    \Output{
        Knob importance($W$) \\
    }
    \BlankLine
    $X,y \leftarrow D$ \\
    \For(){ each $m_i \in M$}{
        // \textit{Get the performance of the model}\\ 
        $y' \leftarrow m_i(D)$\\
        $s \leftarrow$ \textit{$R^2$ value between $y''$ and $y$} \\
         \For(){each $k_j \in K$ }{
             $X' \leftarrow$ \textit{Randomly rearrange values of $k_j$ in X}\\
             $y'' \leftarrow m_i(X')$\\
             $s' \leftarrow $ \textit{$R^2$ value between $y''$ and $y$}\\
             $W_{j} \leftarrow W_{j} + (s-s')*s$\\
         }
    }
  return $W$ 
\caption{Knob Importance Rank}\label{alg:knobrank}
\end{algorithm}

To meet the above trade-off, we design a stacking ensemble learning~\cite{stacking} based method for knob importance ranking, which integrates the weighted average ranking of multiple models as the final ranking results. Compared to the single model, our method achieves stronger robustness to the various scenarios due to the integrated models. For ensemble model construction, we utilize the model's performance as the weight to judge whether the model is suitable. Here, we use $R^2$ metric between observed label and its predicted label to measure the performance, i.e. $R^2 =1-\frac{\sum_{D}(y_i - \widehat{y_i})^2}{\sum_{D} (y_i - \overline{y})^2}$, where $\widehat{y_i}$ is the predicted label, $y_i$ is the observed label and $\overline{y}$ is the mean of observed labels. Then, we assign high weights to well-perform models and low weights to poor-perform models in the ensemble. 

The knob ranking algorithm is shown in Algorithm~\ref{alg:knobrank}. In Lines 1-4, we calculate the performance of a specific model using the $R^2$ metric, which measures the goodness-of-fit between predicted and actual values. In Lines 6-8, we randomly shuffle the values of a knob to obtain shuffled data $X'$, and then compute the performance $s'$ of the model. The more important the knob is, the lower the value of $s'$ is, since the model's decision heavily relies on important knobs. In Line 9, we compute the difference between the performance before and after data shuffle and add it to $R_j$, representing the importance of knob $k_j$. Note that in Line 9, we multiply $(s-s')$ by $s$ to assign a higher weight to models that perform better, ensuring the quality of the final knob importance ranking. Because well-performing models can better utilize the important knobs for decision-making, the reliability of the ranking of knob importance is higher.



\begin{algorithm}
  \SetAlgoNoLine
  \LinesNumbered
  \SetKwInOut{Input}{input}
  \Input{
      Dataset($D={X,y}$) 
  }
  \SetKwInOut{Output}{output}
  \Output{
      The parameters of Interpretable Estimator($w$) \\
  }
  \BlankLine
  $R \leftarrow \varnothing$ \\
  // \textit{Get the optimal random forest structure by the Bayesian optimization}\\ 
  $F \leftarrow Bayesian(D)$
  
  \For(){$ t \in F$}{
    \For(){$p \in t$}{
    // \textit{add the path of tree to R}\\
        $R.append(p)$ \\
    }   
  }
   // \textit{init a binary vector according to the number of rules}\\ 
   $V \leftarrow vector( len(D.X), len(R))$\\
   \For(){ $i \in range(0,len(D.X))$}{
     \For(){$j \in range(0, len(R)$} {
         \uIf{$R_j.true(D.X_i$)}{$V_{ij} = 1$}
         \uElse{$V_{ij}=0$}
         }
   }
  $w \leftarrow  arg min(\frac{1}{k}* \sum_{i = 1}^k (D.y_i - D.y'(V_i))^2 + \lambda * ||w||_1)$ // \textit{lasso regression} \\ 
    
return $w$ \\
\caption{The Training Algorithm of IKR}\label{alg:knobestimator}
\end{algorithm}



\subsection{Interpretable Knob Estimator}\label{sec:estimate}

We then propose IKE with the goal of estimatint the performance of certain knob configuration. We first define the knob estimation problem as follows.

\textbf{Input:} A scenario $S$ and k knob configuration $x = [knob1 = a, knob2 = b,...]$ for $S$.

\textbf{Output:} The performance of input knob configuration $y = ATE(x)$ under $S$.

For the above problem, we develop an interpretable estimator based on random forest. We employ the random forest as the basic model for three reasons: (i) we could only gather limited K-P training data due to its huge time consumption of knob revision and query execution. Compared to deep learning methods, the random forest has lower training data demand~\cite{randomf2}; (ii) Random forests are composed of decision trees. naturally, a path from the root to a leaf in a decision tree can be considered as a set of rules. Thus, it is suitable for random forest to use these rules as explanations; and  (iii) The random forest is more robust to the scenario changes~\cite{randomforest} while some black-box model may produce a dramatic result on the new scenarios.
% Figure environment removed
Based on the random forest denoted as $F = \{t_1, t_2, ...\} $, we gather a knob rule set by traversing all the paths of the trees in $F$. As an example in Figure~\ref{fig:tree}, each $t_i$ is a simple-rule tree, and each path of $t_i$ can be used to generate an knob rule, e.g., $r = shared\_buffers < 400MB \& work\_mem <400MB$. These rules represent the relationships between knobs, i.e., the knobs in same path have a co-influence. Since lasso regression has the powerful ability in finding superior variables~\cite{lasso}, we utilizes the lasso regression to further fit the relationships between these rules and the database performance, pruning useless rules and assigning high weight to the high-quality rules. Then, we gather the direct relationships between the knob rules and their influence to the database performance.







The specific algorithm is shown in Algorithm~\ref{alg:knobestimator}, containing two main stages, random forest model training (Lines 1-3) and interpretable rule fitting (Lines 4-20). Line 3 trains the random forest by utilizing the Bayesian optimization~\cite{auto}, which could find the optimal parameters of random forest such as the number of trees and the depth of trees for the current training set. Lines 4-9 iteratively collect all the rules from the trained random forest. Lines 11-19 transform the input from the knob configuration to the rules. We utilize a binary vector $V$ to identify whether the rule is satisfied by the input knob configuration. 





Then, Line 20 fits the relationships between the knob rules and the database performance, where $y_i$ is the performance label of current tuning task, $k$ is the number of rules, $w$ is the weight vector of rules, and $V_{i}$ is a binary vector. If current knob configuration satisfies the $j$th rule, the corresponding $V_{ij} = 1$. Otherwise, $V_{ij} = 0$. In lasso model, each rule $r_j$ corresponds to a weight $w_i$, by which a knob configuration satisfied $r_i$ will influence the database performance by the weight of the rule. 



% Figure environment removed

     % Figure environment removed

  Then, the specific model is shown in Equation~\ref{linear} as a weighted sum of rules. Due to the simple form, the influences of each knob could be easily reflected by the weight of rules. 

  \begin{equation}\label{linear}
    y'(x) = \sum_{i = 1}^{p} w_i * v_i
  \end{equation}
  
  
Furthermore, we can compute the weight of each knob and visually show its impact, or the joint impact of some knobs on the database performance.
In Figure~\ref{tpcc}, we show an example for the weight of a significant knob 'commit\_delay' of Postgresql under the default workload configuration of TPCC.

 

\section{Transfer Model} \label{sec:transfer}
While our interpretable ML model could effectively estimate the knob performance under various scenarios, the learning model have to collect multiple training sets for model training. This process may take hours (or days) time since a large number of queries need to be executed to gather the performance label of knob configuration. In order to achieve the knob estimator transfer, we design a two-stage transfer learning method based on ER in this section. We first overview the transfer learning approach in Section~\ref{sec:know}. We propose ranking transfer and estimator transfer mechanisms in Section~\ref{sec:rtran} and~\ref{sec:estimator}, respectively. At last, we introduce the overall transfer learning algorithm in Section~\ref{sec:zeroalgor}.

\subsection{The Overview of Two-stage Transfer}\label{sec:know}



As shown in Figure~\ref{fig:transferstructure}, the two-stage knob transfer approach performs transfer estimation for new scenario defined as $O$ with limited training data, containing the ranking transfer mechanism in Section~\ref{sec:rtran} and the estimator transfer mechanism in Section~\ref{sec:estimator}. The ranking transfer mechanism is in steps 1-5. Step 1 collects the fingerprint from database log, and step 2 matches the similar experiences according to the fingerprint. Steps 3-4 calculate the transferred knob ranking results from the top-$k$ experiences. Step 5 delivers the ranking transfer results to support the knob estimator transfer. The estimator transfer mechanism is in steps 6-13. Steps 6-10 obtain the K-P distribution of $O$ and experiences. Step 11 calculate similarities of the K-P distributions. Steps 12-13 utilizes the K-P distribution to match the similar knob estimator experiences and construct the transfer knob estimator. In the remaining of this section, we introduce the design of ranking transfer and estimator transfer mechanisms in detail.


\subsection{Ranking Transfer Mechanism}\label{sec:rtran}

In this section, we propose the ranking transfer mechanism that aims to match similar knob ranking experiences.


To match the similar experiences for ranking, we encode the features of the scenario into a fingerprint. Generally speaking, existing works utilize some fine-grained and embedding approaches, like the tree network~\cite{marcus2019plan} and GPT~\cite{trummer2022codexdb}. Even though these methods could effectively catch the detailed query and data features, these embedding approaches need a huge amount of training data to gather the embeddings of features, resulting in high computational overhead. To obtain the features with limited training data, we select statistic features of the scenarios to efficiently find knob ranking and estimator with similar scenarios. Next, we introduce our design criteria of the fingerprint in detail.

We design the fingerprint of scenarios from the aspect of the knobs. Roughly, we need to consider two types of knobs~\cite{knobsurvey}: resource-knobs (such as memory and concurrency knobs) and execution-knobs (such as join and index knobs). Resource-knobs and execution-knobs are particularly relevant to workload execution of scenarios, so we design two kinds of statistic features that capture the resource and the execution features, i.e., the ratio of SUID (select, insert, update and delete) and the ratio of different pysical operators. 

Thus, we define a fingerprint $f$ as a vector concatenated by the SUID vector ($v_1$) and the operator vector($v_2$), i.e. $f$ = $<v_1, v_2>$. $v_1$ consists of the ratio of selection queries, update queries, insertion queries and deletion queries. The ratio of SUIDs identifies whether current scenario is memory intensive, CPU intensive or disk extensive. The demand of memory, CPU and disk determines the importance of resource-orient knobs. $v_2$ consists of the ratio of physical operators (including scan, sort, etc.). Different from the SUIDs, the operators of scenarios determine the scope of knob execution. For example, the enable\_index\_scan may be useful for a scenario with index scan while not useful for the scenario without index scan.

% Figure environment removed


We illustrate fingerprint with an example for the workload on YCSB shown in Figure~\ref{fig:finger}, which is a read-heavy workload with large amount of index scan operations. This workload is memory intensive, disk intensive and index intensive. Then, the corresponding kinds of knobs (such as shared\_buffers of Postgresql) are important for this scenario. Clearly, all the above features could be directly collected from the database log without consuming DBMS resources. 




After obtaining the fingerprint of scenarios, we use the similarity function to match the similar knob ranking experiences. Since fingerprints are composed of a series of ratios, the similarity should focus on the gap between each dimension. As shown in Formula~\ref{equ:sim1}, we utilize the Euclidean distance~\cite{euclidean} to measure the gap of the corresponding dimensions with type float.

\begin{equation}
  \label{equ:sim1}
    dis\_ranking(f_1, f_2) = \sqrt{\sum(f_1[i]-f_2[i])^2}  
  \end{equation}

\subsection{Estimator Transfer Mechanism}\label{sec:estimator}


Similar to the ranking transfer mechanism, we also consider to construct the features and similarity function for the knob estimator transfer. If we directly utilize the comprehensive  features of scenarios to match the similar knob estimator. The major challenge is the irregular features (like the various table design and workload structure) of scenarios caused by their too many detailed and complex feature, including workload, data table and hardwares. Existing works utilize the large model~\cite{trummer2022codexdb} to gather unify feature representations of scenarios for model transfer. Although the large deep learning models could obtain some effective feature representations of different scenarios, their black-box nature and multiple neural layers bring unstable performance and huge time consumption, respectively. 

Thus, in this section, we propose a unified and stabled feature based on splines for the knob estimator transfer. Our core idea is to abandon complex scenarios features and calculate the similarity between two scenarios from the aspect of K-P data distribution instead. Specifically, if the $O$ shares the similar K-P data distributions with the historical estimator experiences. It means that they have similar performance trend under same knob configurations. we could transfer the knob estimator to $O$. As shown in Figure~\ref{fig:transferstructure}, the K-P distribution similarity gathering approach for knob estimators consists of four main parts, sampling strategy (step 6), K-P points collection (steps 7-10), feature calculation(step 11) and similarity measure (step 12). We then introduce them respectively. 


\textbf{Sampling Strategy:} For the $O$, we guarantee the sampling efficiency by the uniform sampling algorithm and controlling the sampling space.  (i) We utilize the Latin HyperCube Sampling to achieve the uniformity of the multiple dimension of knobs. (ii) Sampling Space: Different from existing works~\cite{ibtune}, our sampling strategy is based on the important knobs obtained by the transfer ranking transfer mechanism in Section~\ref{sec:rtran}. After determining the sampling space and sampling algorithm, we can collect a high quality samples ($S$) in the $O$, as the basis for calculating K-P distribution similarity.  %We collect N points under the adjustable knob space of object scenario. Then we utilize these samples to learn the knob-performance distribution (or called estimator distribution) of the object scenario

\textbf{K-P Points Collection:} To fairly compare the similarity among experiences, we utilize the same samples ($S$) to process the experiences with the $O$. Firstly, we directly collect the performance label of $S$ on $O$ to construct the K-P data. For the experiences, we utilize the trained historical estimator to predict the performance label of $S$. For adapting to the experiences, we reshape the dimensions of the input knob configurations according to the following rules.


The knob set $L_1$ of input $x_1$ of $S$ has three kinds of relationships with the knob set $L_2$ of input $x_2$ of estimator experiences $m$, fully contained (F), partially contained (P), not contained (N). We describe how to deal with $F$, $P$, $N$ separately.(i) If the input of  and historical estimator satisfy $L_2 \subset L_1$, we simply cut $x_1$ by $L_2$. Since the removed knobs are unimportant according to the historical ranking experience, we can simply ignore the effect of these unimportant knobs. (ii) For partially contained, we reduce the uninvolved knobs and fill the gaps with the default configuration of $O$. For example, $L_1 = [knob1, knob2, knob3]$ and $ x_2 = [knob1 = d, knob2 = e, knob4 = f]$, we cut $knob3$ of $x_1$ and reshape the $x_1$ as $[knob1 = a, knob2 = b, knob4 = O(knob4)]$, where $O(knob4)$ represents the default configuration of knob4 on scenario $O$. (iii) For the case 3, we return the zero due to its totally mismatching.

Then, we obtain the K-P point set for the $O$ and all the experiences.

\textbf{Feature Calculation:} Based  on the K-P points, some natural distribution features exist, such as the mean and variance. However, these simple features could only measure the distance between point sets and ignore the differences of data distribution trends. To efficiently fit the trend feature of K-P points, we employ the spline interpolation method~\cite{linear2}  due to the powerful fitting ability of spline interpolation of describing the distribution, e.g., some simple splines could fit the complex curve distribution~\cite{linear}. Then, the coefficients of the spline function are used as the distribution features of the point set. %\textcolor{red}{For an example for interpolation results, $y=ax_1 + bx_2$, the [a,b] is the coefficients of splines ($f_1 = x_1, f_2 = x_2$).}
Compared to the simple statistics, these coefficients can describe the trend of K-P distribution. Next, we introduce how to calculate the direction distance between these coefficients.

\textbf{Similarity Measure:} As shown in Formula~\ref{equ:cosine}, we utilize the cosine distance~\cite{cosine} to calculate the similarity of two statistic feature. Since we focus on the direction distance between two statistic features under interpolation model. The similar direction means the similar performance distribution of two estimator.%\footnote{what is the motivation of cosine motivation?}


\begin{equation}
  \label{equ:cosine}
    dis\_estimator(d_1, d_2) =\frac{\sum d_1[i]*d_2[i] }{\sqrt {\sum (d_1[i])^2} \sqrt{\sum (d_2[i])^2}} 
  \end{equation}

\begin{algorithm}
    \caption{knob estimator transfer}\label{alg:zero}
    \SetAlgoNoLine
    \LinesNumbered
    \SetKwInOut{Input}{input}
    \Input{
        experience repository ($ER$), fingerprint of $O$ ($f$), the number of experience ($K$), the number of samples($N$).
    }
    \SetKwInOut{Output}{output}
    \Output{
        \ $M$ is the transfered estimator of $O$
    }
    \BlankLine
    
    $E \leftarrow \textit{Find nearest top-k experiences in ER by f}$ \\
    $weights \leftarrow \textit{assign the weights of E by f's similarity.}$\\
    $k = \sum_{i}{ weights_i * E_i.k}$ // \textit{ranking transfer by the weighted average experiences}\\
    $S \leftarrow LHS(N, k)$ // \textit{sample N points under ranking results}\\ 
    $d_1 \leftarrow \textit{collect K-P points of O under S.}$\\
    $similarity \leftarrow \varnothing $\\
    \For(){$e \in E$}{
    $d_2 = \{S, e.IKE(S)\}$ // \textit{obtain the K-P points of experiences }\\
    $similarity.add(dis\_estimator(d_1, d_2))$ \\
    }
    $mweights \leftarrow \textit{assign weight based on the similarity}$\\
    $M \leftarrow \sum_{i} {mweight_i * E_i.m }$ // \textit{estimator transfer by the weighted average experiences}\\ 
    $\textit{return M}$\\
\end{algorithm}



\subsection{Transfer Learning Algorithm}\label{sec:zeroalgor}
In this section, we introduce the overall transfer knob estimation algorithm based on the above feature design and similarity design.%\footnote{challenge and the basic idea}

As shown in Algorithm~\ref{alg:zero}, our method consists of two main stages. In the first stage (Lines 1-3), we perform the transfer of knob ranking by matching the fingerprints. The fingerprints contain some statistic features of current scenario which determine the important knobs. Line 1 finds the top-k experiences by $f$. Line 2 assigns the weight for $E$ according to the similarity of $f$. Line 3 calculates the final ranking by the weighted average of top-$K$ experiences.


In the second stage (Lines 4-12), base on the top-$K$ experiences, we perform estimator transfer by matching the similar K-P distribution. Line 4 samples $N$ points based on the ranking results for $O$. Line 5 collects the performance label for $O$ and returns the K-P dataset for $O$. Lines 6-10 calculate the similarities of K-P points. Line 11 assigns weight for the knob estimator of experiences. Line 12 calculates the transfer estimator for the $O$ according to the weight average of experiences.


Overall, the only database accessing operation of our algorithm is the collection of labeled data for $O$. Such operation could be performed on the cloned instance to avoid affecting the efficiency of online database service. 

% Figure environment removed



\section{The Evaluation Of IWEK} \label{sec:eval}
    In this section, we conduct extensive experiments to test the performance of IWEK. Firstly, in Section~\ref{sec:setting}, we introduce our experimental setup, including the dataset settings and evaluation metrics. We compare the proposed interpretable knob estimator with two typical regression model in Section~\ref{sec:inter}. We experimentally evaluate the performance of transfer learning in Section~\ref{sec:zero}. We evaluate the performance of robustness of IWEK under various $K$ values in Section~\ref{sec:tune}.


\subsection{Experimental Setup}\label{sec:setting}

All experiments were conducted on Postgresql v15.0 and Docker 20.10.19 with a container of 2GB memory, 4 processor cores and 50MB/s hard disk speed. Then we introduce the datasets and metrics of our experiments. 

\noindent \underline{\textbf{Experimental Datasets}} We utilize the open source benchmarks, YCSB and TPCC, implemented by \cite{DifallahPCC13} to evaluate our method, which are widely used in existing works~\cite{2019An}. YCSB and TPCC are designed for testing the performance of OLTP wordload which is more sensitive to the knob changes. As shown in Table~\ref{tab:workload}, we set up 16 scenarios to evaluate the performance of the IWEK. Specifically, we set various configurations in data scale and transaction operation ratios for TPCC (like tpcc-1 [NewOrder=45\%, Payment=40\%, OrderStatus=5\%, Delivery=5\%, StockLevel=5\%]) and YCSB (like ycsb-1 [ReadRecord=50\%, InsertRecord=5\%, ScanRecord=15\%, UpdateRecord=10\%, DeleteRecord=10\%, ReadModifyWriteRecord=10\%]), respectively. 

\begin{table}[h!]	
	\centering 
	\caption{The setting of scenarios}  
	\label{tab:workload}  
\begin{tabular}{|p{0.5cm}|p{1.0cm}|p{0.8cm}|p{3.5cm}|} 
    \hline
	No.&name & scale & transaction ratios \\
    \hline
	1&tpcc-1 & 1GB & 45\%,40\%,5\%,5\%,5\%  \\% 1
    \hline
	2&tpcc-2  & 1GB & 5\%,45\%,5\%,40\%,5\%  \\% 2
    \hline
	3&tpcc-3  & 1GB & 20\%,10\%,50\%,15\%,5\% \\% 18
    \hline
        4&tpcc-4  & 3GB & 60\%,20\%,10\%,5\%,5\% \\% 10
    \hline
        5&tpcc-5  & 3GB & 10\%,20\%,10\%,30\%,30\%  \\% 11
    \hline
        6&tpcc-6  &3GB & 20\%,10\%,50\%,15\%,5\%  \\% 12
    \hline
	7&tpcc-7  & 5GB & 45\%,40\%,5\%,5\%,5\%  \\% 5
    \hline
	8&tpcc-8  & 5GB & 5\%,45\%,5\%,40\%,5\%  \\% 6
    \hline
	9&ycsb-1  & 1GB & 50\%,5\%,15\%,10\%,10\%,10\%  \\% 3
    \hline
	10&ycsb-2  & 1GB & 20\%,5\%,15\%,25\%,10\%,25\%  \\% 4
    \hline
	11&ycsb-3  & 1GB & 20\%,50\%,10\%,10\%,5\%,5\% \\% 19
    \hline
	12&ycsb-4  & 1GB & 20\%,10\%,15\%,20\%,10\%,25\%  \\% 22
    \hline
	13&ycsb-5  & 3GB & 10\%,5\%,15\%,10\%,30\%,30\%  \\% 14
    \hline
	14&ycsb-6  & 3GB & 30\%,10\%,20\%,20\%,10\%,10\% \\% 15
    \hline
	15&yscb-7  & 5GB & 50\%,5\%,15\%,10\%,10\%,10\%  \\% 7
    \hline
	16&ycsb-8  & 5GB &20\%,5\%,15\%,25\%,10\%,25\%  \\% 8



    \hline
    
\end{tabular}
\end{table}


 \noindent \underline{\textbf{Metric:}} In this paper, we utilize three metrics to evaluate the performance of IWEK, the mean prediction error, the Pearson correlation coefficient and the accuracy of knob estimation. 
 The mean prediction error is used to measure the error between the predicted label and real label defined in Formula~\ref{equ:error}.  

\begin{equation}\label{equ:error}
   error = \frac{1}{n} \sum_1^n (y(x_i) - IWEK (x_i) )^2
\end{equation}


The Pearson correlation coefficient is used to measure the correlation between the predicted label and real label defined in Formula~\ref{equ:peason}. $Cov(y_t, IWEK(X))$ is the covariance of the predicted label and real label, and $\sigma_{y_t}$ and $\sigma_{IWEK(X)}$ are the standard deviation of the predicted label and real label, respectively.

\begin{equation}\label{equ:peason}
    error = 1 - \frac{Cov(y_t, IWEK(X))}{\sigma_{y_t} * \sigma_{IWEK(X)}}
\end{equation}

To verify the practicality, we use IWEK for a real binary classification task. Users could use IWEK to compare their old knob configuration and the new knob configuration. This is a classification task to clarify which is the better knob configuration. For evaluating the accuracy of this task, we randomly generate the pairs of knobs, such as $[x_1, x_2]$. The specific formula of the prediction accuracy is shown in the Formula~\ref{equ:acc}, where $TP$ is the number of correctly classified knob pairs, and $T$ is the total number of knob pairs.

\begin{equation}\label{equ:acc}
	accuracy = \frac{TP}{T}
 \end{equation}

\noindent \underline{\textbf{Baselines:}} We compared our knob estimator with two baselines: lasso regression and multilayer perceptron regression (MLP). Lasso regression is a typical regression model with low requirements for the size of the training data, and is insensitive to noise and capable of filtering out irrelevant features\cite{lasso}. MLP can better capture the nonlinear relationship between features and also demonstrate good performance in some non-linear tasks~\cite{MLP}.



\subsection{The Effectiveness for Interpretable Estimator}\label{sec:inter} 

In this section, we evaluate the performance of our interpretable knob estimator under the extensive workloads of Table~\ref{tab:workload}. To test the performance of IWEK with limited training set, we only sample 100 knob-performance points for each workload by utilizing the Latin Hypercube Sampling algorithm. Then we utilize 70 points as the training set and 30 ones as the test set. From the test set, we randomly sample 100 pairs to evaluate the accuracy of our estimator. 




Figure~\ref{a} illustrates the Pearson correlation coefficients of our IKE model compared to two baseline models, MLP and lasso. Overall, our IKE model demonstrates superior predictive performance in the majority of cases. The average Pearson coefficient for IKE reaches 0.93, whereas it is 0.68 for MLP and 0.78 for lasso. Even in the worst-case scenario (tpcc-6), our IKE model still achieves a correlation coefficient of 0.78. In contrast, due to the complex network structure of MLP, it struggles to converge with only 70 training data, resulting in instances where the coefficient becomes p=-0.3, such as tpcc3. Additionally, MLP performs poorly in tpcc-2 (p=0.36) and tpcc-6 (p=0.34). However, if MLP manages to converge, it can achieve a high correlation coefficient, as seen in tpcc-5 (p=0.98). On the other hand, lasso exhibits relatively stable performance with an average coefficient of 0.78 but still lags behind our IKE model significantly. 
% Figure environment removed


Figure~\ref{b} illustrates the classification performance of our IKE, MLP, and lasso models. It is evident that our IKE model demonstrates superior performance compared to MLP and lasso in most scenarios. On average, IKE achieves an accuracy of 80\% (with a minimum accuracy of 69\%), surpassing MLP with an average accuracy of 68\% (minimum accuracy of 48\%) and lasso with an average accuracy of 72\% (minimum accuracy of 57\%). Even in the worst-case scenario, IKE still achieves a respectable accuracy of 69\%. In contrast, the MLP model exhibits notably low accuracy in tpcc-2 (53\%) and ycsb-5 (48\%), approaching random selection results. On the other hand, the lasso model maintains comparatively stable accuracy across different scenarios. Although lasso has limitations in accurately fitting complex knob-performance data (maximum accuracy of 87\%, average accuracy of 72\%), its linear structure enables it to capture the relative relationship among various knob configurations. 

In summary, from the comparison results, IWEK demonstrates good performance across various workloads in the open-source benchmark TPCC and YCSB, as evidenced by correlation coefficient, prediction accuracy.



\subsection{The Performance of Transfer Learning}\label{sec:zero}


% Figure environment removed

In this section, we test the performance of the estimator transfer approach with parameter $N = 10$ and $K = 3$, on metrics of the similarity relationships, the Pearson correlations and the classification accuracy. 

Figure~\ref{fig:heatmap} shows the heat of similarity relationships of K-P distribution. The closer to blue the square is, the less similar the corresponding two scenes are, and while the closer to yellow the square is, the more similar the corresponding two scenes are. It is evident that similar scenario types (e.g. TPCC and TPCC) exhibit higher similarity, while the similarity between different scenarios types (e.g. TPCC and YCSB) is evidently lower.

Figure~\ref{fig:recall} shows the average Top-3 recall of our ranking transfer methods under all the scenarios. We have observed that our ranking transfer achieved a recall rate of 66.6\% in most scenarios. Specifically, for tpcc-1 and tpcc-2, our transfer ranking method successfully identified all the important knobs. However, in the case of tpcc-5 and ycsb-6, only one important knob was recalled. This might be due to the balanced transaction ratios in these scenarios, which make them sensitive to multiple types of knobs, making it challenging to accurately rank them.


% Figure environment removed

Then, we test the performance of the transfer estimation, on the Pearson correlation coefficient and binary classification accuracy. The "origin" represents the relationship between the predicted labels and observed labels obtained by using 70 data points from the current scenario as training samples and 30 data points as testing samples, and "transfer" refers to the prediction results obtained by transfer knob estimator (N = 10, K = 3). For the transfer learning of certain scenario (like tpcc-1), we utilize the remaining scenarios (tpcc-2 to ycsb-8) as the experiences. We can visually observe that in new scenario, our method obtain the effective transferred predicted labels that is close to the "origin". 





% Figure environment removed

Figure~\ref{fig:transfer_zero} shows the Pearson coefficient of the origin estimator and the transferred estimator. As seen in the Figure~\ref{fig:transfer_zero_a}, the transfer estimator can exhibit good performance in most scenarios, achieving an average correlation coefficient of 0.845. This verifies the effectiveness of our transfer algorithm. In addition, as shown in Figure~\ref{fig:transfer_zero_b}, the performance of the transfer estimator in binary classification is still comparable to that of the original estimator with an average accuracy rate of 78.81\%. Even in some scenario like ycsb-5, the transfer estimator outperforms the origin estimator. This result demonstrate that weighted sum of $K$ historical experiences may lead to better classification accuracy.
% Figure environment removed

In addition, the scatterplots in Figure~\ref{fig:transfer} show the points distribution of transfer estimator and the origin estimator. We take the real performance label as the $x$-axis and the predicted performance label as the $y$-axis. Figure~\ref{fig:transfer_a} shows the prediction results on TPCC, with origin error $0.00456$ and transfer error $0.0192$. We observe that the predicted points of transfer model are all concentrated around $y=x$ with only 10 samples. Also, the transfer prediction of YCSB performs well with transfer error $0.0365$.
% Figure environment removed
\subsection{The Evaluation of Robustness}\label{sec:tune}
In this section, we evaluate the performance of IWEK with $K = 1-6$ to test the robustness of IWEK, containing the robustness of ranking transfer and estimator transfer. 

Figure~\ref{fig:robustrecall} illustrates the average ranking transfer results for K ranging from 1 to 6. It is evident that our ranking transfer method exhibits a high level of robustness when faced with changes in K. That is because our ranking transfer assigns weights according to similarity. Then the experiences with the highest similarity receive the greatest weight, while those dissimilar experiences are assigned lower weights.  

Figure~\ref{fig:avg_correlations_trend} shows the average performance of estimator transfer approach under all the scenarios of Table~\ref{tab:workload} as the number of reused historical experiences varies. According to the figure, we observe that the model transfer performs the best when three historical experiences are reused. However, the performance declines when fewer or more than three experiences are reused. This is because when reusing fewer than three experiences, the robustness of model transfer cannot be effectively ensured, while reusing more than three experiences leads to incorporating unrelated models, thereby lowering overall performance. 


As shown in Figure~\ref{fig:avg_accuracy}, the average accuracy of 16 scenarios is more robust to the changes of parameter $K$. The average accuracy only changes from 74.5\% to the 78.5\% with $K$ = 1-6. This is because the accuracy metric focuses on the relative performance between the two knob configurations. Even if there is an error between the predicted label and the true label, the relative relationship of two configuration may still be predicted accurately. 

Overall, our model is robust to changes of $K$, indicating that our transfer model can assign proper weights for experiences.




% Figure environment removed




\section{Related Works}
    \label{related}
       \section{Related Work}
\label{appsec: related work}
Bayesian causal discovery literature has primarily focused on inference in linear models with closed-form posteriors or marginalized parameters. Early works considered sampling directed acyclic graphs (DAGs) for discrete~\cite{cooper1992bayesian, madigan1995bayesian, heckerman2006bayesian} and Gaussian random variables~\cite{friedman2003being, tong2001active} using Markov chain Monte Carlo (MCMC) in the DAG space. However, these approaches exhibit slow mixing and convergence~\cite{eaton2012bayesian,grzegorczyk2008improving}, often requiring restrictions on number of parents~\cite{kuipers2017partition}. %Alternative exact dynamic programming methods are limited to small settings~\cite{koivisto2012advances}. 

Recent advances in variational inference~\cite{zhang2018advances} have facilitated graph inference in DAG space, with gradient-based methods employing the NOTEARS DAG penalty \cite{zheng2018dags}.\cite{annadani2021variational} samples DAGs from autoregressive adjacency matrix distributions, while \cite{lorch2021dibs} utilizes Stein variational approach \cite{liu2016stein} for DAGs and causal model parameters. \cite{cundy2021bcd} proposed a variational inference framework on node orderings using the gumbel-sinkhorn gradient estimator \cite{mena2018learning}. \cite{deleu2022bayesian,nishikawa2022bayesian} employ the GFlowNet framework \cite{bengio2021gflownet} for inferring the DAG posterior. Most methods, except\cite{lorch2021dibs} are restricted to linear models, while \cite{lorch2021dibs} has high computational costs and lacks DAG generation guarantees compared to our method.
% at least quadratic scaling complexity, both with respect to the number of nodes (due to the DAG penalty) as well as number of posterior samples. Our proposed approach instead has linear complexity with respect to number of posterior samples and does not require any additional DAG penalty.     

In contrast, \emph{quasi-Bayesian} methods, such as DAG bootstrap \cite{friedman2013data}, demonstrate competitive performance. DAG bootstrap resamples data and estimates a single DAG using PC \cite{spirtes2000causation}, GES \cite{chickering2002optimal}, or similar algorithms, weighting the obtained DAGs by their unnormalized posterior probabilities. Recent neural network-based works employ variational inference to learn DAG distributions and point estimates for nonlinear model parameters \cite{charpentier2022differentiable,geffner2022deep}.

\section{Conclusion}
\label{conclusion}
    \section{Conclusion and Future Work}
In this work, I design corruption-robust algorithms for the Lipschitz contextual search problem. I present the \emph{agnostic checking} technique and demonstrate its effectiveness in designing corruption-robust algorithms. There are several open problems for future research. First, in the algorithm I propose for pricing loss, the schedule for agnostic checks is fixed upfront. Can the learner design an adaptive checking schedule for the pricing loss? Second, this work assumes the learner has knowledge of the Lipschitz constant $L$. Can the learner design efficient no-regret algorithms without knowledge of $L$? 



\newpage
\bibliographystyle{ieeetr}
\bibliography{main}


\end{document}
