The development of cloud database brings a new era of database service, DaaS~\cite{daas}. It becomes increasingly convenient for small companies and individual users to own their database services. However, diverse users pose potential risks to the database performance. Specifically, the tuning operations of non-expert users may result in negative influence for databases. Moreover, existing learning-based self-driving methods for database may also bring the potential performance degradation due to its black box model~\cite{reinforcement}. An efficient estimator of the knob tuning is in demand to improve the availability of databases.  

Unfortunately, few attentions haven been given to the estimation model for the database knobs. Existing works~\cite{reinforcement,gaosi} for database knobs focus on searching optimal knobs for certain workload. To fill this gap, we develop an \emph{\underline{i}nterpretable \underline{w}hat-if \underline{e}stimator for database \underline{k}nobs}, IWEK, under dynamic workload. Totally different from the knob tuning~\cite{reinforcement,gaosi}, IWEK focuses on constructing an interpretable model to fit the relationships between knobs and its performance, for efficiently knob evaluation. Specifically, designing a estimator for database knobs faces the following challenges: 


\textbf{Explosive Knob Space:} With the development of database management, the relational databases (such as Postgresql, openGauss, and MySQL) already have hundreds of database configuration knobs. Moreover, the number of knobs in Postgresql has increased from less than 100 in 2000 to about 600 items in 2020, and the growth rate is still maintained~\cite{van2017automatic}. It is expensive for DBAs to find the effective knobs in the search space. Also, the large-scale knobs bring combination explosion to knob evaluation space which lead to large resource consumption in constructing knob estimator.

\textbf{Availability:} An available knob estimator should be both robust and interpretable. Robustness ensures that the knob estimator always provides accurate results for various data, workload and environment. Interpretability makes the knob estimator trustable and easy to tune. However, deep learning models possess impressive learning capabilities; nonetheless, they may introduce potential risks to databases due to their unstable performance~\cite{survey}. On the other hand, rule-based approaches, although stable, are unable to capture the complex relationships between knob configuration and database performance.

\textbf{Limited training data:} It is difficult to collect enough training data for knob estimation due to two main reasons. On the one hand, collecting the knob-performance (K-P) training data has a large time consumption in knob revision and workload execution. On the other hand, modifying the knob configuration may bring potential risks to the database, such as the `fysnc' of Postgresql. Thus, we face an important challenge to construct an accurate knob estimator with limited training data. 


To overcome the above challenges, we propose interpretable what-if estimator for database knobs, \textsf{IWEK}. Here what-if means that we can obtain the synthesize performance of certain database without real evaluation to minimize the impact on the online service performance. IWEK solve above challenges as follows.

We observe that among hundreds of adjustable knobs in DBMS, only a small share play the vital role in improving database efficiency for specific workload.  For example, the 'max\_wal\_size' of Postgresql (a popular open source database) has low-impact on the OLAP workload. Thus, we design an ensemble learning based knob ranking algorithm to filter out the important knobs to reduce the \textbf{explosive knob space}. 

We develop an interpretable linear estimator based on Random Forest~\cite{random} to capture the complex relationships between knob configurations and the performance. With such a trusted and interpretable estimator, \textbf{high availability} is ensured. 

We design transfer mechanism to obtain not only knob importance but also estimators from stored experiences in database tuning without model training. As a result, we obtain the estimator with \textbf{limited training data}. 

 To the best of our knowledge, IWEK is the first systematic study for the estimation of database knobs. IWEK could help both practitioners and researchers to make stable knob tuning under dynamic workloads. The technical contributions of this paper are summarized as follows. 

\begin{itemize}
    \item To effectively find important knobs, we design an adaptive knob ranking mechanism shown in Section~\ref{sec:ranking} which integrates the weighted average ranking of multiple models as the final ranking results. 
    
    \item To fit the complex relationships between knobs and database performance, we establish an interpretable linear estimator based on the random forest shown in Section~\ref{sec:estimate}.
    
    \item  We propose a two-stage transfer learning that supports cold start on a limited training set, including the ranking transfer in Section~\ref{sec:rank} and estimator transfer~\ref{sec:estimate}.
        
    \item To clarify the effectiveness of the proposed model, we conduct extensive experiments in Section~\ref{sec:eval} on two popular benchmark, TPCC~\footnote{http://www.tpc.org/tpcc/} and YCSB~\footnote{https://github.com/brianfrankcooper/YCSB}. Experimtnal results demonstrate that IWEK outperforms existing approaches and achieves high robustness.
\end{itemize}