\documentclass[10pt,twocolumn,letterpaper]{article}
\usepackage{iccv}
\usepackage{times}

\usepackage[T1]{fontenc}
\usepackage{graphicx}
\usepackage{amsmath}
\usepackage{amssymb}
\usepackage{xspace}
\usepackage{booktabs}
\usepackage{multirow}
\usepackage[dvipsnames]{xcolor}
\usepackage{multirow}

\usepackage[pagebackref=true,breaklinks=true,letterpaper=true,colorlinks,bookmarks=false]{hyperref}
\usepackage[capitalise]{cleveref}
\usepackage{enumitem}

\iccvfinalcopy
\def\iccvPaperID{6874}
\ificcvfinal\pagestyle{empty}\fi

% suppress annoying warnings
\hfuzz=5.002pt
\hbadness=10000

\makeatletter
\renewcommand{\paragraph}{%
  \@startsection{paragraph}{4}%
  {\z@}{0.25em}{-1em}%
  {\normalfont\normalsize\bfseries}%
}
\makeatother

% change skip above the caption
\setlength{\abovecaptionskip}{2pt plus 3pt minus 2pt}

\newcommand{\datasetlong}{Replay\xspace}
\newcommand{\dataset}{Replay\xspace}
\newcommand{\RSnote}[1]{{\color{purple}{\bf RS$^\circ$: }#1}}

\title{\datasetlong: \\
Multi-modal Multi-view Acted Videos for Casual Holography\vspace{-1.8ex}}
\author{Roman Shapovalov$^*$
\qquad Yanir Kleiman$^*$
\qquad Ignacio Rocco$^*$
\qquad David Novotny\\
Andrea Vedaldi
\qquad Changan Chen$^\dagger$
\qquad Filippos Kokkinos
\qquad Ben Graham
\qquad Natalia Neverova\\
Meta \qquad UT Austin$^\dagger$ \qquad $^*$\textit{equal contribution}\\
%{\tt\small \{romansh, yanirk, irocco, dnovotny, vedaldi, fkokkinos, benjamingraham, nneverova\}@meta.com}
{\tt\small \url{https://replay-dataset.github.io/}}
%{\tt\small changan@cs.utexas.edu}
}

\begin{document}
\maketitle
\ificcvfinal\thispagestyle{empty}\fi

\begin{abstract}
We introduce \emph{\datasetlong}, a collection of multi-view, multi-modal videos of humans interacting socially. Each scene is filmed in high production quality, from different viewpoints with several static cameras, as well as wearable action cameras, and recorded with a large array of microphones at different positions in the room. Overall, the dataset contains over 4000 minutes of footage and over 7 million timestamped high-resolution frames annotated with camera poses and partially with foreground masks.
The \dataset dataset has many potential applications, such as novel-view synthesis, 3D reconstruction, novel-view acoustic synthesis, human body and face analysis, and training generative models. We provide a benchmark for training and evaluating novel-view synthesis, with two scenarios of different difficulty. Finally, we evaluate several baseline state-of-the-art methods on the new benchmark.
%The videos in the HDT dataset are filmed in 4K and production quality, 
%and include camera calibrations to eliminate camera distortion and color balance discrepancies between sensors. 
\end{abstract}

% Figure environment removed

\section{Introduction}
Automatic 3D reconstruction of clothed humans using image inputs has gained increasing significance due to its potential applications in a wide array of AR/VR scenarios. High-fidelity reconstructions typically depend on sophisticated capture systems, which are developed with dense camera arrays~\cite{collet2015high,joo2015panoptic,joo2018total}, programmable light-stages~\cite{Vlasic2009, guo2019relightables}, and depth sensors~\cite{newcombe2011kinectfusion,DoubleFusion,BodyFusion,dou2016fusion4d,newcombe2015dynamicfusion}. However, stringent capture environments equipped with complex hardware pose significant challenges for consumer-level applications.


In this context, considerable research effort has been dedicated to developing methods that allow for more flexible capture configurations, such as utilizing a few RGB inputs. Among these works, learning implicit functions \cite{iccv2020PIFu, saito2020pifuhd, hong2021stereopifu} has proven effective in achieving highly detailed reconstructions by integrating the advancements of deep neural networks. These methods employ large multi-layer perceptrons (MLPs) to predict the occupancy probability or truncated signed distance function (TSDF) value of every queried 3D point based on its associated local feature, which is extracted from images. They can recover a continuous surface at arbitrary resolutions without topology restrictions.


However, in typical MLP-based implicit networks, the occupancy or TSDF value at each location is solved independently with planar image features, rendering them less capable of addressing challenging cases such as occlusions. Consequently, these methods suffer from generalization and robustness issues, particularly when tackling strong occlusions caused by large motion or multiple interacting humans. 
Some follow-up studies  \cite{zheng2021deepmulticap,zheng2021pamir,huang2020arch} utilize an extra geometric model, SMPL~\cite{Loper2015}, to improve robustness by introducing strong shape priors. 
Their success typically relies on the assumption of geometrical similarity \cite{huang2020arch} between the shape prior and target reconstruction, making them intractable for handling complex cases with loose clothes and sensitive to errors in SMPL model fitting.



%\ping{this paragraph sounds like `TSDF is better than MLP/SMPL, and we use TSDF to solve the problem'. But in Sec 3, we are telling a different story, saying `MLP needs a 3D convolutional encoder'. We need to make these two sections consistent.}\sicong{I think in this paragraph we claim that the TSDF}


%We opt for Trucated Signed Distance Funtion (TSDF) volumetric representations as they are naturally suitable for convolution operations, which have shown remarkable performance for learning hierarchical features on 2D visual perception tasks \cite{SunXLW19}. 
%Meanwhile, TSDF also describes the gradual geometry change around shape surface, which is not reflected by occupancy volume. 

We instead revisit the 3D volumetric representation and resort to 3D convolutional neural networks (CNNs) for feature learning, due to their impressive performance in feature learning and the ability to incorporate spatial context. However, volumetric methods and 3D convolution involve discretization, which might raise concerns regarding whether a discretized volume can preserve subtle geometric details as continuous representations learned in implicit functions. We investigate the relationship between volume resolution and quantization error on synthetic data by converting target mesh objects to TSDF volumes, as shown in Figure~\ref{fig:quantization_error}. We observe that the quantization errors are significantly reduced by increasing volume resolution and become nearly negligible when reaching a relatively high resolution (e.g., 512 or higher). In other words, achieving fine-detailed reconstruction is not supposed to be restricted by the use of volume representations as long as a proper volume resolution is utilized. Therefore, we present a method with high-resolution feature volumes, e.g., 256 and 512, while traditional volumetric methods \cite{varol18_bodynet,gilbert2018volumetric} are often limited to much lower resolutions, such as 32 or 128.



On the other hand, an increase in volume resolution may lead to a cubic growth of memory overhead \cite{8100085}. Reducing memory costs while guaranteeing the granularity of volumetric representations is necessary for pursuing high-quality reconstruction. Thus, we adopt a coarse-to-fine approach and cull away irrelevant voxels to build a sparse high-resolution feature volume. At the coarse level, the network computes an initial TSDF by applying a U-Net with sparse 3D CNN \cite{3DSemanticSegmentationWithSubmanifoldSparseConvNet} on the sparse feature volume, which is carved by a visual hull. Through our experiments, it turns out that more than 95\% of the volume grids are discarded by the visual hull culling, making the sparse 3D CNN efficient. At the fine level, the network focuses on a narrow band near the zero-level set of the initial TSDF and discretizes the narrow band with smaller voxels. By employing this narrow-band culling, we further shrink the sampling space, resulting in a relatively small range of grid numbers (usually 300K--500K in our experiments) even with a high volume resolution of 512. The remaining voxels in the narrow band are associated with features that fuse high-frequency information from the computed normal maps upon the low-frequency shape from the coarse level to compute the TSDF at high resolution. The final mesh is then extracted from the TSDF using the Marching-Cube algorithm ~\cite{Lorensen87marchingcubes}.
% Different from the u-net sturcture to preserve global topology context, we then apply a shallow 3dcnn to compute the final TSDF $D_{final}$ which contain more local geometry detail.




% \ping{this paragraph can be expanded. It is an important contribution and often ignored by other works. stress on the novel idea of regressing blending weights instead of colors}

In addition to geometry, high-quality mesh texture is also a crucial factor contributing to visual appearance. Directly computing a color field in 3D space, as in \cite{iccv2020PIFu}, struggles to capture high-frequency texture details, while the neural radiance field (NeRF) \cite{yu2020pixelnerf} or the DoubleField~\cite{shao2022doublefield} require expensive per-instance optimization and are often unstable for sparse input images. In contrast, we adopt an image-based rendering approach to compute a texture atlas map, which is efficient and widely supported in existing computer graphics tools. 
Specifically, we compute a blending weight at each 3D point on the mesh surface to determine its color as a weighted average of the colors at its image projections. The blending weights can be computed at a relatively coarse resolution, e.g., 512 volume resolution in our case, and leave texture details to the high-resolution images, such as 1K or 2K. Unlike previous methods that generate blurry texturing results under sparse input, our method generalizes well on both synthetic and real data with just a few input views. 
Figure~\ref{fig:teaser} shows two examples reconstructed by our method. Despite the challenging garment, pose, and occlusion, our method recovers faithful shape, normal, and texture on the right.

%with a wide variety of poses and clothing styles, and it is also adaptive to handle input image with arbitrary resolutions.
%\sicong{For this concern we claim that when the resolution of dicretized volume meets certain threshold (which is 256 in our experiment), the quantization error can be neglected.} 



In summary, the main contributions of this paper are as follows:
\begin{itemize}
\vspace{-0.1in}
  \item 
  We revisit the 3D volumetric representation and demonstrate that it can support clothed human reconstruction with equal or even better performance compared to implicit representation. 
  \item 
  We develop a memory and computation-efficient method for high-resolution volumetric reconstruction using sophisticated sparse 3D CNN, coarse-to-fine estimation, and voxel culling by visual hull and narrow bands. 
  \item 
  We introduce a novel method to compute a texture atlas map, which captures rich appearance details from high-resolution input images.
  \item 
  We achieve impressive results on standard benchmark datasets Twindom and MultiHuman, significantly reducing the point-2-surface (P2S) precision to approximately 0.2cm from just six input views, with more than $50\%$ error reduction compared to the state-of-the-art methods, including DoubleField~\cite{shao2022doublefield} and PIFuHD~\cite{saito2020pifuhd}.
\end{itemize}
\section{Related Work}
\label{appsec: related work}
Bayesian causal discovery literature has primarily focused on inference in linear models with closed-form posteriors or marginalized parameters. Early works considered sampling directed acyclic graphs (DAGs) for discrete~\cite{cooper1992bayesian, madigan1995bayesian, heckerman2006bayesian} and Gaussian random variables~\cite{friedman2003being, tong2001active} using Markov chain Monte Carlo (MCMC) in the DAG space. However, these approaches exhibit slow mixing and convergence~\cite{eaton2012bayesian,grzegorczyk2008improving}, often requiring restrictions on number of parents~\cite{kuipers2017partition}. %Alternative exact dynamic programming methods are limited to small settings~\cite{koivisto2012advances}. 

Recent advances in variational inference~\cite{zhang2018advances} have facilitated graph inference in DAG space, with gradient-based methods employing the NOTEARS DAG penalty \cite{zheng2018dags}.\cite{annadani2021variational} samples DAGs from autoregressive adjacency matrix distributions, while \cite{lorch2021dibs} utilizes Stein variational approach \cite{liu2016stein} for DAGs and causal model parameters. \cite{cundy2021bcd} proposed a variational inference framework on node orderings using the gumbel-sinkhorn gradient estimator \cite{mena2018learning}. \cite{deleu2022bayesian,nishikawa2022bayesian} employ the GFlowNet framework \cite{bengio2021gflownet} for inferring the DAG posterior. Most methods, except\cite{lorch2021dibs} are restricted to linear models, while \cite{lorch2021dibs} has high computational costs and lacks DAG generation guarantees compared to our method.
% at least quadratic scaling complexity, both with respect to the number of nodes (due to the DAG penalty) as well as number of posterior samples. Our proposed approach instead has linear complexity with respect to number of posterior samples and does not require any additional DAG penalty.     

In contrast, \emph{quasi-Bayesian} methods, such as DAG bootstrap \cite{friedman2013data}, demonstrate competitive performance. DAG bootstrap resamples data and estimates a single DAG using PC \cite{spirtes2000causation}, GES \cite{chickering2002optimal}, or similar algorithms, weighting the obtained DAGs by their unnormalized posterior probabilities. Recent neural network-based works employ variational inference to learn DAG distributions and point estimates for nonlinear model parameters \cite{charpentier2022differentiable,geffner2022deep}.
\section{Dataset Description}
\label{sec:dataset}


In this section, we describe our dataset. 
\dataset contains \num{1209} unique roots.
A root refers to the first element in a package name. 
For example, in the package name "com.example.mypackage", "com" is the root.

We also collected data on the number of fields in the package names of the apps in our dataset. 
A field refers to a dot-separated element in a package name.
For example, in the package name "com.example.mypackage", there are three fields: "com", "example", and "mypackage".
Results are visible in Table~\ref{table:num_of_fields}.
There are 733 package names with only one field, \num{14368} package names with two fields, etc.

\begin{table}
    \centering
    \caption{Number of package names per field in \dataset}
    \begin{adjustbox}{width=.7\columnwidth,center}
        \begin{tabular}{lr|lr}
            \hline
            Fields & Count & Fields & Count \\ \hline
            with 1 field & \num{733} & with 5 fields & \num{100}\\ 
            with 2 fields & \num{14368} & with 6 fields & \num{6}\\ 
            with 3 fields & \num{4231} & with 7 fields & \num{3}\\ 
            with 4 fields & \num{720} & with 8 fields & \num{1}\\
            \hline \hline
            \multicolumn{2}{l|}{Total} & \multicolumn{2}{r}{\libsAfterRefinement}
        \end{tabular}
    \end{adjustbox}
    \label{table:num_of_fields}
\end{table}



There are significantly fewer package names with four or more fields. 
The number of package names with one field is relatively low compared to the others.
This suggests that many package names in the dataset follow a standard naming convention with a domain name followed by one or more subpackages.
The presence of package names with four or more fields may indicate the use of more complex or specialized naming conventions\footnote{Examples of libraries are:  
\href{https://mvnrepository.com/artifact/riddley/riddley}{riddley}, 
\href{https://mvnrepository.com/artifact/jakarta.annotation}{jakarta.annotation}, 
\href{https://mvnrepository.com/artifact/com.vogle.sbpayment}{com.vogle.sbpayment}, 
\href{https://mvnrepository.com/artifact/pl.robakowski.jersey.bootstrap}{pl.robakowski.jersey.bootstrap}, 
\href{https://mvnrepository.com/search?q=de.tudresden.inf.lat.jsexp}{de.tudresden.inf.lat.jsexp}, 
\href{https://mvnrepository.com/artifact/de.hs_rm.cs.vs.tools.vocabularygenerator}{de.hs\_rm.cs.vs.tools.vocabularygenerator}, 
\href{https://mvnrepository.com/artifact/eu.adlogix.com.google.api.ads.dfp}{eu.adlogix.com.google.api.ads.dfp}, 
\href{https://mvnrepository.com/artifact/us.gov.dot.faa.ang.c55/huggs}{us.gov.dot.faa.ang.c55.gradle.huggs}.
}.


Table~\ref{table:top_ten} presents the top 10 most frequent roots and the top 10 most frequent fields found. 
In the first two columns, we can see that the root "com" is by far the most frequent, with more than \num{10000} occurrences. 
The second most frequent root is "net", with \num{1265} occurrences. 
In the second two columns, which represents the most frequent fields, including the roots, we can see that the field "com" is still the most frequent. 
The second two columns do not differ much from the first two columns, except for the two fields "gradle" and "android" that now appear.
This could indicate that Android libraries are prevalent in the dataset.
It is confirmed in the last two columns, which represent the most frequent fields without the roots. 
After "gradle" and "android", the third most frequent field is "sdk", with 138 occurrences.
We see a shift in the most prevalent fields. 
Instead of roots, we now see fields such as "sdk", "maven", "plugin(s)", "api", "tools", and "common".
This may be indicative of the types of libraries.
Overall, the results suggest that most package names are from the "com" domain and that Android libraries are well represented.


\begin{table}
    \centering
    \caption{Top 10 roots and fields present in \dataset}
    \begin{adjustbox}{width=.9\columnwidth,center}
        \begin{tabular}{c|c|c|c|c|c}
            \hline
            \multicolumn{2}{c|}{\textbf{Top 10 roots}} & \multicolumn{2}{c|}{\textbf{Top 10 most used fields}} & \multicolumn{2}{c}{\textbf{Top 10 most used fields w/o roots}}\\ \hline
            \textbf{Root} & \textbf{Count} & \textbf{Field} & \textbf{Count} & \textbf{Field} & \textbf{Count} \\ \hline \hline
            com & \num{10520} & com & \num{10551} & gradle & \num{250} \\ \hline
            net & \num{1265} & net & \num{1273} & android & \num{239} \\ \hline
            de & \num{917} & de & \num{918} & sdk & \num{138} \\ \hline
            cn & \num{663} & cn & \num{663} & plugin & \num{120} \\ \hline
            dev & \num{458} & dev & \num{465} & maven & \num{108} \\ \hline
            me & \num{411} & me & \num{413} & plugins & \num{93} \\ \hline
            eu & \num{223} & gradle & \num{254} & api & \num{60} \\ \hline
            ru & \num{202} & android & \num{240} & oss & \num{57} \\ \hline
            fr & \num{188} & eu & \num{224} & tools & \num{54} \\ \hline
            ch & \num{181} & ru & \num{203} & common & \num{39} \\ \hline
        \end{tabular}
    \end{adjustbox}
    \label{table:top_ten}
\end{table}
\section{Experiments}
% \haizhou{Follow the same way of introduction as we did in Section2.}
% \noindent In this section, we will introduce datasets and experimental setups that we used. Then we evaluate our method, other self-supervised methods, and supervised methods under different distribution shifts (\ie, concept shifts and covariate shifts) under common settings (\ie, transductive, inductive settings). It has to note that we focus on node-level tasks (\eg, node classification) in this work. As for graph-level tasks, we leave it as our future work and some simple experiments can be found in Appendix~\ref{app:graph_classification}. 
In this section, we first introduce the experimental setup including datasets, training, and evaluation protocol in Section~\ref{sec:dataset}~and~\ref{sec:unsupervised}. 
% Next, we present our experimental setup and conduct extensive experiments to evaluate our method in Section~\ref{sec:unsupervised}. 
We then perform an ablation study to demonstrate the effectiveness of each proposed component in Section~\ref{sec:ablation}. 
Additionally, we analyze the impact of important hyper-parameters in Section~\ref{sec:sensitivity}. 
Subsequently, we integrate our method with various encoding models, showcasing the model-agnostic nature of our recipe in Section~\ref{sec:other_models}. 
Finally, we provide some qualitative results such as feature visualization in Section~\ref{sec:vis}.
It is important to note that we focus on node-level tasks (\eg, node classification) in this work. As for graph-level tasks, we leave it as our future work, while some simple experiments are also provided in Appendix~\ref{app:graph_classification}.

\subsection{Datasets}\label{sec:dataset}
There exist some benchmarks for evaluating graph out-of-distribution generalization~\cite{good,ji2022drugood,gds}. 
Among them, GOOD~\cite{good} is the most representative and comprehensive benchmark that curates more diverse graph datasets with diverse tasks, including single/multi-task graph classification, graph regression, and node classification involving more distribution shifts (\ie, concept shifts and covariate shifts). Hence in this work, we follow the evaluation protocol proposed in \cite{good}. Furthermore, we validate the effectiveness of our method in the datasets (\ie, Amazon-Photo, Elliptic) that are used in EERM~\cite{eerm}. The statistics and detailed introduction to these datasets can be found in Table~\ref{tab:dataset} and Appendix~\ref{app:datasets}.

\begin{table*}[htp]
\caption{The descriptions of datasets. ``Domain-Level'' means splitting by graphs, ``Time-Aware'' denotes splitting according to chronological order.``Word'' and ``Degree'' represent splitting according to word diversity and node degree respectively. ``Language'' means splitting by user language, suggesting the prediction should not be impacted by the language the user use. ``University'' denotes splitting according to the domain university, implying that the prediction of webpages should be based on word contents and link connections rather than university features. ``Color'' means that nodes are split according to node differences in covariate shift and color-label correlations in concept shift.}
\label{tab:dataset}
\centering
\begin{tabular}{cccccccc}
\toprule
Datasets     & Network Type        & \#Nodes & \#Edges & \#Attributes &\#Classes& Train/Val/Test Split     & Metric   \\
% Cora         & Artificial Transformation & 2,703   &         &              &         &                      & Accuracy \\
Amazon-Photo\footnotemark
             & Co-purchasing network      & 7,650   & 119,081   & 755          & 10      & Domain-Level         & Accuracy \\
Elliptic\footnotemark  
             & Bitcoin transactions       & 203,769 & 234,355   & 165          & 2       & Time-Aware           & F1-Score \\
GOOD-Cora    & Scientific publications    & 19,793  & 126,842   & 8,710         & 70      & Word/Degree          & Accuracy \\
% GOOD-Arxiv   & arXiv papers               & 169,343 & 2,315,598 & 128          & 40      & Time/Degree          & Accuracy \\
GOOD-Twitch  & Gamer network              & 34,120  & 892,346   & 128          & 2       & Language             & ROC-AUC  \\
GOOD-CBAS    & A BA-house graph           & 700     & 3,962     & 4             & 4       & Color                & Accuracy \\
GOOD-WebKB   & Webpage network            & 617     & 1,138     & 1,703         & 5       & University           & Accuracy \\
\bottomrule
\end{tabular}
\end{table*}
\footnotetext[5]{This dataset is adopted from~\cite{yang2016revisiting}. \cite{eerm} constructs ten graphs with different environment id’s for each graph.} 
\footnotetext[6]{The original is available on \hyperlink{https://www.kaggle.com/ellipticco/elliptic-data-set}{https://www.kaggle.com/ellipticco/elliptic-data-set}}

\subsection{Unsupervised Representation Learning}\label{sec:unsupervised}
\subsubsection{Transductive Setting}~\label{sec:trans}
% \noindent\textbf{Baselines.}\quad We conduct experiments with 12 baselines which consist of three categories: supervised methods and self-supervised generative methods, self-supervised contrastive methods. Specifically, we compare with three supervised baselines: empirical risk minimization~(ERM)~\cite{erm}, invariant risk minimization (IRM)~\cite{irm}, and a recent proposed graph OOD method dubbed EERM~\cite{eerm}. We also compare various unsupervised node-level representation learning methods: three self-supervised generative methods including GAE~\cite{gae}, VGAE~\cite{gae}, GraphMAE~\cite{gmae} and seven self-supervised contrastive methods: DGI~\cite{dgi}, MVGRL~\cite{mvgrl}, GRACE~\cite{grace}, RoSA~\cite{rosa}, BGRL~\cite{bgrl}, COSTA~\cite{costa}, SwAV~\cite{swav}. The descriptions of these methods can be found in Appendix~\ref{app:baselines}.
In this subsection, we focus on validating our proposed algorithm under the transductive setting, where the test nodes will participate in message passing~\cite{gilmer2017neural} during training following~\cite{good}. 

\noindent\textbf{Baselines.} We conduct experiments with 12 baselines from three categories: (i)~supervised methods, including empirical risk minimization~(\textbf{ERM})~\cite{erm}, invariant risk minimization (\textbf{IRM})~\cite{irm}, and a recent proposed graph OOD method \textbf{EERM}~\cite{eerm}; (ii)~self-supervised generative methods including Graph Autoencoder (\textbf{GAE})~\cite{gae}, Variational Graph Autoencoder (\textbf{VGAE})~\cite{gae}, Self-Supervised Masked Graph Autoencoders (\textbf{GraphMAE})~\cite{gmae}; (iii)~self-supervised contrastive methods including Deep Graph Infomax (\textbf{DGI})~\cite{dgi}, Contrastive Multi-View Representation Learning on Graphs (\textbf{MVGRL})~\cite{mvgrl}, Deep Graph Contrastive Representation Learning (\textbf{GRACE})~\cite{grace}, A Robust Self-Aligned Framework for Node-Node Graph Contrastive Learning (\textbf{RoSA})~\cite{rosa}, Bootstrapped Representation Learning on Graphs (\textbf{BGRL})~\cite{bgrl}, Covariance-Preserving Feature Augmentation for Graph Contrastive Learning (\textbf{COSTA})~\cite{costa}, Unsupervised Learning of Visual Features by Contrasting Cluster Assignments (\textbf{SwAV})~\cite{swav}. The detailed descriptions of these baselines can be found in Appendix~\ref{app:baselines}.

\noindent\textbf{Experimental setup.} We use the same graph encoder across different datasets for a fair comparison following~\cite{good}. We use grid search to find other hyper-parameters (\eg, learning rate, epochs) for different methods. For all experiments, we select the best checkpoints for ID and OOD tests according to results on ID and OOD validation sets following~\cite{good}, respectively. Experimental details and hyper-parameter selections are provided in Appendix~\ref{app:hyper}. For evaluating unsupervised methods, a linear classifier will be built on the frozen trained encoder after finishing pre-training. The reported results are the mean performance with standard deviation after 10 runs following~\cite{good}.

\noindent\textbf{Analysis.}\quad Based on the experimental results listed in Table~\ref{tab:trans_concept} and \ref{tab:trans_covariate}, we can draw the following conclusions: firstly, we find strong self-supervised methods (\eg, GRACE, BGRL, COSTA) are more robust to distribution shifts (concept shift in Table~\ref{tab:trans_concept} and covariate shift in Table~\ref{tab:trans_covariate}) compared to supervised methods. For instance, on GOOD-CBAS and GOOD-WebKB datasets, GRACE surpasses the best supervised method by large margins (over 6\% absolute improvement). Interestingly, we find the methods designed for OOD generalization (\ie, IRM) and graph OOD generalization (\ie, EERM) do not attain superior performance than the standard ERM on most of the datasets. For example, EERM shows superior OOD performance compared to ERM in only one experiment, and IRM outperforms ERM in four out of ten experiments across the conducted evaluations. This phenomenon is also observed in \cite{good,ahuja2020empirical,rosenfeld2021risks}, showcasing the challenge of achieving invariant prediction in non-Euclidean graph settings. 

Furthermore, our method surpasses other SOTA self-supervised methods on the OOD test set of all datasets by a considerable margin while achieving comparable performance in the in-distribution test set. For instance, on small datasets such as GOOD-CBAS and GOOD-WebKB, our method outperforms GRACE\footnote{MARIO is built up on GRACE according to our recipe. So, we make a comparison with GRACE here.} by over 2\% absolute accuracy on the OOD test set. On larger datasets such as GOOD-Cora and GOOD-Twitch, our method still outperforms other methods which shows its superiority. For instance, under covariate shift, MARIO surpasses other methods by over 7\% absolute accuracy on the GOOD-Twitch OOD test set. These statistics prove the effectiveness of our design.


\begin{table*}[htp]
\caption{Experimental results of all methods under concept shift. The bold font means the top-1 performance and the underline represents the second performance across the unsupervised methods. 'ID' represents in-distribution test performance and 'OOD' means out-of-distribution test performance. (OOM: out-of-memory on a GPU with 24GB memory)}
\label{tab:trans_concept}
\centering
\scalebox{0.95}{
\begin{tabular}{l|cc|cc|cc|cc|cc}
\toprule
\toprule
\multirow{3}{*}{concept shift} & \multicolumn{4}{c|}{GOOD-Cora}                   & \multicolumn{2}{c|}{GOOD-CBAS} & \multicolumn{2}{c|}{GOOD-Twitch} & \multicolumn{2}{c}{GOOD-WebKB} \\
                           & \multicolumn{2}{c}{word} & \multicolumn{2}{c|}{degree}& \multicolumn{2}{c|}{color}    & \multicolumn{2}{c|}{language}   & \multicolumn{2}{c}{university} \\
                           & ID         & OOD         & ID          & OOD          & ID            & OOD           & ID             & OOD            & ID            & OOD            \\
\midrule
ERM                        & 66.38±0.45 & 64.44±0.18  & 68.60±0.40  & 60.76±0.34   & 89.79±1.39    & 83.43±1.19    & 80.80±1.00     & 56.92±0.92     & 62.67±1.53    & 26.33±1.09     \\
IRM                        & 66.42±0.41 & 64.29±0.31  & 68.57±0.35  & 61.45±0.24   & 89.64±1.21    & 82.29±1.14    & 78.87±1.04     & 59.30±1.79     & 62.67±1.10    & 26.88±1.42     \\
EERM                       & 65.10±0.44 & 62.45±0.19  & 66.95±0.44  & 56.58±0.25   & 79.07±2.12    & 64.50±1.01    & OOM            & OOM            & 62.50±2.01    & 28.07±3.23      \\
\midrule
% Random-Init                & 37.53±1.74 & 32.12±1.24  & 37.82±1.71  & 27.74±1.14   &               &               &                &                & 60.33±2.21    & 27.07±1.70     \\
GAE                        & 60.65±0.89 & 58.00±0.55  & 62.59±1.11  & 53.44±0.80   & 75.28±1.36    & 68.07±2.05    & 81.25±0.81     & 51.51±1.05     & 62.17±3.34    & 25.78±1.85     \\
VGAE                       & 63.19±0.53 & 60.35±0.47  & 61.65±0.66  & 54.28±0.28   & 76.50±0.50    & 59.07±0.56    & 80.46±0.53     & 55.56±4.53     & 62.50±2.38    & 24.40±2.57     \\
GraphMAE                   & \underline{66.44±0.46} & \underline{64.87±0.30}  & 67.95±0.46  & 59.41±0.39   & 89.14±0.89    & 82.93±0.93    & 80.05±0.64     & 59.38±1.49     & 61.83±3.37    & 29.27±2.15     \\
DGI                        & 63.33±0.56 & 60.71±0.49  & 65.93±1.02  & 55.83±0.53   & 91.22±1.47    & 85.00±1.66    & 80.05±0.87     & 59.16±1.88     & 61.83±2.83    & 28.63±1.92      \\
MVGRL                      & OOM        & OOM         & OOM         & OOM          & 88.57±1.15    & 76.50±1.17    & OOM            & OOM            & 62.00±3.79    & 28.26±4.20     \\
GRACE                      & 65.61±0.61 & 63.92±0.44  & \textbf{68.59±0.35}  & 60.15±0.45   & 92.00±1.39    & 88.64±0.67    & \textbf{83.43±0.63}     & \underline{60.45±1.46}     & 64.00±3.43    & \underline{34.86±3.43}  \\
RoSA                       & 64.06±0.67 & 62.44±0.39  & 67.07±0.65  & 57.68±0.44   & 90.78±2.27    & 85.93±2.14    & 82.39±0.42     & 57.45±2.16     & 64.17±4.10    & 32.20±2.15     \\
BGRL                       & 65.18±0.43 & 63.43±0.45  & 66.83±0.80  & 59.63±0.38   & 92.36±1.16    & 87.14±1.60    & 82.52±0.60     & 55.48±1.48     & 63.67±2.33    & 31.47±3.43     \\
COSTA                      & 65.05±0.80 & 62.37±0.45  & 66.76±0.87  & 55.73±0.36   & \underline{93.50±2.62}    & \underline{89.29±3.11}    & 83.15±0.30 & 55.03±3.22     & 61.66±2.58    & 32.39±2.13 \\
% ArCL                       &            &             & 67.64±0.57  & 59.71±0.44   &               &               &                &                & 65.00±3.94    & 35.41±1.97 \\      
SwAV                       & 62.22±0.53 & 59.79±0.53  & 64.65±0.94  & 55.06±0.39   & 89.00±0.79    & 81.72±0.66    & \underline{83.32±0.15}     & 59.69±1.97     & \underline{65.17±3.76}    & 29.36±2.01    \\
\midrule
MARIO                       & \textbf{67.11±0.46} & \textbf{65.28±0.34}  & \underline{68.46±0.40}  & \textbf{61.30±0.28}   & \textbf{94.36±1.21}    & \textbf{91.28±1.10}    & 82.31±0.54     & \textbf{63.33±1.72}     & \textbf{65.67±2.81}    & \textbf{37.15±2.37}     \\
\bottomrule
\end{tabular}}
\end{table*}

\begin{table*}[htp]
\caption{Experimental results of all methods under covariate shift. The bold font means the top-1 performance and the underline represents the second performance across the unsupervised methods. 'ID' represents in-distribution test performance and 'OOD' means out-of-distribution test performance. (OOM: out-of-memory on a GPU with 24GB memory)}
\label{tab:trans_covariate}
\centering
\scalebox{0.95}{
\begin{tabular}{l|cc|cc|cc|cc|cc}
\toprule
\toprule
\multirow{3}{*}{covariate shift} & \multicolumn{4}{c|}{GOOD-Cora}                                   & \multicolumn{2}{c|}{GOOD-CBAS} & \multicolumn{2}{c|}{GOOD-Twitch} & \multicolumn{2}{c}{GOOD-WebKB} \\
                           & \multicolumn{2}{c}{word} & \multicolumn{2}{c|}{degree}& \multicolumn{2}{c|}{color}    & \multicolumn{2}{c|}{language}   & \multicolumn{2}{c}{university} \\
                           & ID         & OOD         & ID          & OOD          & ID            & OOD           & ID             & OOD            & ID            & OOD            \\
\midrule
ERM                        & 70.50±0.41 & 64.69±0.33  & 72.46±0.49  & 55.53±0.50   & 92.00±3.08    & 77.57±1.29    & 70.98±0.41     & 49.35±5.09     & 39.34±1.79    & 14.52±3.14   \\
IRM                        & 70.48±0.26 & 64.53±0.57  & 71.98±0.34  & 53.72±0.46   & 90.86±2.41    & 78.86±1.67    & 69.81±0.95     & 49.11±2.82     & 38.52±3.30    & 13.97±2.80     \\
EERM                       & OOM        & OOM         & OOM         & OOM          & 65.00±2.57    & 57.43±3.60    & OOM            & OOM            & 46.07±4.55    & 27.40±7.65     \\
\midrule
GAE                        & 56.63±0.79 & 48.93±0.93  & 66.30±0.88  & 34.01±0.87   & 73.00±2.16    & 60.86±3.01    & 67.24±1.23     & 47.65±2.49     & 45.08±6.32    & 28.02±6.29    \\
VGAE                       & 62.02±0.66 & 54.12±0.86  & 69.41±0.57  & 44.20±1.29   & 62.29±2.04    & 63.29±1.11    & 66.99±1.43     & \underline{50.48±4.58}     & 48.85±4.68    & 20.87±6.69     \\
GraphMAE                   & 68.14±0.43 & 64.00±0.33  & \textbf{73.36±0.56}  & 53.75±0.55   & 67.28±3.03    & 67.28±1.49    & 68.84±1.20     & 48.02±2.79     & 48.03±4.34    & 30.00±8.09     \\
DGI                        & 60.85±0.75 & 57.03±0.67  & 68.97±0.41  & 41.75±0.88   & 69.57±4.09    & 59.71±3.43    & 68.43±1.05     & 44.83±1.61     & 48.52±5.04    & 21.11±7.50     \\
MVGRL                      & OOM        & OOM         & OOM         & OOM          & 65.00±1.94    & 64.15±0.77    & OOM            & OOM           & \textbf{54.10±5.39}    & 16.59±6.51     \\
GRACE                      & \underline{68.77±0.33} & \underline{64.21±0.41}  & 72.69±0.34  & \underline{56.10±0.63}   & \underline{93.57±1.83}    & \underline{89.29±3.40}    & \underline{71.12±0.87} & 46.21±1.54 & 49.67±5.82    & 28.10±4.68    \\
RoSA                       & 68.19±0.56 & 62.48±0.61  & 71.04±0.62  & 52.72±0.79   & 84.71±4.14    &79.14±3.51     & 70.58±0.36     & 45.83±1.72     & 52.30±4.24    & \underline{34.24±7.92}     \\
BGRL                       & 67.23±0.43 & 61.33±0.36  & 72.11±0.39  & 49.15±0.73   & 89.00±2.56    & 79.86±3.29    & \textbf{71.43±0.53}     & 43.86±0.94     & 51.80±5.55    & 30.32±7.61    \\
COSTA                      & 65.28±0.60 & 60.33±0.53  & 70.65±0.62  & 54.03±0.28   & 92.29±1.59    & 82.71±2.74    & 69.29±1.37     & 49.07±2.13     & 50.49±3.01    & 29.84±4.75   \\
SwAV                       & 63.29±1.01 & 56.98±0.94  & 70.27±0.73  & 43.00±0.52   & 89.57±1.12    & 81.43±1.69    & 69.19±0.93     & 49.37±2.96     & 49.84±4.82    & 30.55±6.72   \\
\midrule
MARIO                       & \textbf{69.99±0.54} & \textbf{65.06±0.34}  & \underline{72.73±0.43}  & \textbf{57.73±0.45}  & \textbf{94.57±2.46}    & \textbf{91.00±2.48}     & 68.31±0.78 & \textbf{57.37±1.37}     & \underline{53.94±3.23}    & \textbf{35.24±4.98}   \\
\bottomrule
\end{tabular}}

\end{table*}

\subsubsection{Inductive Setting}
In this subsection, we conduct experiments under the inductive settings, where the test nodes are kept unseen during training. This setting is more suitable for domain generalization.
% But we think it is more convincing that conduct experiments under inductive settings which means test nodes are unseen during training. This setting is more appropriate for domain generalization.

\noindent\textbf{Baselines:} For GOOD-WebKB and GOOD-CBAS datasets, we adopt ERM, IRM, GraphMAE, and GRACE as our baselines. And for Amazon-Photo and Elliptic datasets, we select ERM, EERM, and GRACE as our baselines.

\noindent\textbf{Experimental setup:} For GOOD-WebKB and GOOD-CBAS datasets, we use the same model configuration in Section~\ref{sec:trans}.
% Besides, we add experiments on Amazon-Photo dataset~\cite{yang2016revisiting} and Elliptic~\cite{elliptic} dataset in this subsection. 
For Amazon-Photo dataset~\cite{yang2016revisiting} and Elliptic~\cite{elliptic} dataset, they consist of many snapshots (training data and testing data use different snapshots) which are naturally inductive. For Amazon-Photo dataset, we use 2-layer GCN~\cite{gcn} as the encoder and for elliptic dataset, we use 5-layer GraphSAGE~\cite{sage} as encoder following~\cite{eerm}.

% Figure environment removed

\noindent\textbf{Analysis:}
According to Figure~\ref{fig:amazon},\ref{fig:elliptic},\ref{fig:ind_con},\ref{fig:ind_cov}, we can draw following conclusions:
firstly, based on Figure~\ref{fig:amazon}, it is evident that our method outperforms other representative supervised and self-supervised methods on all test graphs (T1$\sim$T8). This superiority is reflected in the larger median value of our method compared to others. For instance, MARIO achieves over a 3\% absolute improvement compared to ERM in terms of the mean value of eight median values. Additionally, our method demonstrates higher stability across different random initializations, as indicated by the closer proximity of the first and third quartile values to the median value~(\eg, the difference of first and third quartile values of ERM, EERM, GRACE and MARIO are 4.2, 3.3, 6.7 and 1.0 on T8 respectively which indicates MARIO is much more stable than other methods). Furthermore, our method exhibits consistent performance across different graphs (\eg, The standard deviation of median values on T1$\sim$T8 for ERM, EERM, GRACE, and MARIO are 0.4, 1.1, 1.2, and 0.3, respectively.), indicating its robustness to environmental variations and its ability to extract invariant features: $g(G^e) \approx g(G^{e'})$ for all $e, e' \in \mathcal{E}^\text{train}$. In summary, our method showcases enhanced OOD generalization capabilities.
% $g(G^e)g(G^e^\prime)$ where $any e, e^\prime in \mathcal{E}^{train}$

Secondly, from the results presented in Figure~\ref{fig:elliptic}, we can observe that our method averagely harvests 10.9\% absolute improvement over GRACE and 12.5\% absolute improvement over EERM in terms of F1 scores on Elliptic dataset. This demonstrates the effectiveness of our method in handling distribution shifts and improving performance compared to existing approaches. It is worth noting that GRACE's performance worsens over time, indicating its inability to handle distribution shifts effectively. In contrast, our method consistently achieves better F1 scores, except for T9, which is caused by the dark market shutdown occurred after T7~\cite{elliptic}. The emergence of such an event introduces significant variations in data distributions, which subsequently results in performance degradation for all methods. Indeed, this event serves as an unpredictable external factor that introduces significant challenges for models trained on limited training data. The results indicate that the performance heavily depends on available training data. Nonetheless, our approach outperforms other methods even in such an extreme case. This highlights the effectiveness of our method in addressing distribution shifts and improving generalization performance.

Finally, based on the observations from Figure~\ref{fig:ind_con} and Figure~\ref{fig:ind_cov} MARIO demonstrates the best performances on both ID and OOD test sets for GOOD-WebKB and GOOD-CBAS datasets, under both concept shift and covariate shift. Notably, MARIO outperforms other methods by more than 3\% and 10\% absolute improvement on GOOD-WebKB and GOOD-CBAS, respectively, under covariate shift. We can draw similar conclusions as discussed in Section~\ref{sec:trans}. Even under the inductive setting, our method continues to demonstrate excellent OOD generalization capabilities and achieves comparable or even improved in-distribution test performance. These statistical results further validate the effectiveness of our method in handling distribution shifts and enhancing generalization performance.

Overall, the observations we have made provide strong evidence of the great capacity of our method for handling distribution shifts, validating its effectiveness and potential for real-world applications.



% Figure environment removed

% Figure environment removed


% Figure environment removed


\subsection{Ablation Studies}\label{sec:ablation}
\noindent Table~\ref{tab:aba} provides a detailed analysis of the effect of each component according to our proposed recipe for improving OOD generalization in graph contrastive learning. Let's examine the different variants of our method and their impact on performance.
Specifically, MARIO~(w/o ad) represents MARIO without  adversarial augmentation. MARIO~(w/o cmi) denotes we only maximize the mutual information between positive pairs without considering conditional mutual information. MARIO~(w/o cmi, ad) means a vanilla graph contrastive method that is similar to GRACE. 

From Table~\ref{tab:aba}, we can find MARIO~(w/o cmi) lags far behind MARIO on OOD test set which demonstrates appropriately minimizing the redundant information (\ie, conditional mutual information) is essential to improve OOD generalization of GCL methods. And adversarial augmentation can also boost OOD generalization because it can approximately serve as a supermum operator to learn more invariant features  discussed in Section~\ref{sec:aug}. Based on the analysis of these variants, it is evident that the proposed improvements on data augmentation and contrastive loss in the recipe are both effective in enhancing graph OOD generalization. Each component contributes to the overall performance improvement, and their combination leads to a stronger self-supervised graph learner in terms of graph OOD generalization. 

In short, the findings from Table~\ref{tab:aba} support the rationale behind your proposed recipe and provide empirical evidence of the effectiveness of each proposed component. By incorporating these enhancements, our method achieves superior performance in handling distribution shifts and improving graph OOD generalization in graph contrastive learning.
\begin{table*}[htp]
\caption{Ablation studies for MARIO by masking each component.}
\label{tab:aba}
\centering
\scalebox{0.9}{
\begin{tabular}{l|cc|cc|cc|cc|cc}
\toprule
\toprule
\multirow{3}{*}{concept shift} & \multicolumn{4}{c|}{GOOD-Cora}                       & \multicolumn{2}{c|}{GOOD-CBAS} & \multicolumn{2}{c|}{GOOD-Twitch} & \multicolumn{2}{c}{GOOD-WebKB} \\
                           & \multicolumn{2}{c}{word} & \multicolumn{2}{c|}{degree}& \multicolumn{2}{c|}{color}    & \multicolumn{2}{c|}{language}   & \multicolumn{2}{c}{university} \\
                           & ID         & OOD         & ID          & OOD          & ID            & OOD           & ID             & OOD            & ID            & OOD            \\
\midrule
MARIO                      & \textbf{67.11±0.46} & \textbf{65.28±0.34}  & \textbf{68.46±0.40}  & \textbf{61.30±0.28}      & \textbf{94.36±1.21}  & \textbf{91.28±1.10}    & 82.31±0.54     & \textbf{63.33±1.72}     & \textbf{65.67±2.81}    & \textbf{37.15±2.37}     \\
MARIO(w/o ad)              & 66.23±0.53 & 64.02±0.18  & 67.88±0.38  & 60.46±0.29   & 93.21±1.25    & 90.29±0.91    & 82.42±0.73     & 60.50±1.02     & 64.83±2.83    & 36.51±3.25    \\
MARIO(w/o cmi)             & 65.32±0.60 & 63.51±0.32  & 68.14±0.32  & 61.19±0.34   & 94.15±1.23    & 90.57±1.96    & \textbf{82.51±0.56}     & 61.41±2.63     & 64.50±4.35    & 35.78±2.53     \\
MARIO(w/o cmi, ad)         & 64.67±0.55 & 63.11±0.32  & 67.95±0.65  & 60.01±0.57   & 93.36±1.66    & 89.64±1.73    & 81.90±0.75     & 60.12±1.60     & 64.17±3.67    & 34.13±2.38     \\
\bottomrule
\end{tabular}}
\end{table*}
% & 65.32±0.60 & 63.51±0.32 exchange 64.67±0.55 & 63.11±0.32
% 68.14±0.32       id ood test: 60.95±0.43       ood ood test: 61.19±0.34


\subsection{Sensitivity Analysis}\label{sec:sensitivity}
\noindent In this subsection, we will analyze some important hyper-parameters of our method. We conduct sensitivity analysis on GOOD-WebKB dataset with concept shift, we chose two sensitive hyper-parameters (\ie, the coefficient $\gamma$ of condition mutual information in Equation~\ref{equ:cmi} and the number of prototypes $|C|$ in Equation~\ref{equ:pq}). The coefficient of CMI range in $[0.001, 0.01, 0.1, 0.5, 1]$ and the number of prototypes $|C|$ ranges in $[10, 50, 100, 200, 300]$. From Figure~\ref{fig:sensitivity}, we can observe that $\gamma$ reaches 0.1 and $|C|$ reaches 100 or 200 can achieve the best OOD test accuracy. Both higher and lower values of $\gamma$ result in suboptimal performance. This finding aligns with previous research such as DIB~\cite{dib}, indicating that an appropriate compression level is crucial for achieving optimal performance. Extremely high or low compression values are not ideal. 

Regarding the number of prototypes $|C|$, based on the results shown in Figure~\ref{fig:sensitivity}, it is found that setting $|C|=100$ leads to the best performance in terms of OOD test accuracy. This choice provides a moderate number of pseudo labels, which is beneficial for the learning process. 

Based on the sensitivity analysis, we determined that setting $\gamma=0.1$ and $|C|=100$ on most datasets. These hyperparameter values strike a balance between compression level and the number of prototypes, resulting in improved graph OOD generalization.
% Figure environment removed


\subsection{Integrated with Other Models}\label{sec:other_models}
% Figure environment removed

\begin{table}[htp]
\caption{Results of different learning approaches with different encoding models (\ie, GCN, GraphSAGE, GAT).}
\label{tab:others}
\centering
\scalebox{0.9}{
\begin{tabular}{cc|cc|cc}
\toprule
\toprule
\multirow{3}{*}{Model}& \multirow{3}{*}{Method} & \multicolumn{2}{c|}{GOOD-CBAS} & \multicolumn{2}{c}{GOOD-WebKB} \\
                & & \multicolumn{2}{c|}{color}    & \multicolumn{2}{c}{university} \\
                &   & ID          & OOD         & ID          & OOD            \\
\midrule
\multirow{3}{*}{GCN} 
&ERM               & 89.79±1.39 & 83.43±1.19  &  62.67±1.53 & 26.33±1.09         \\
&GRACE             & 92.00±1.39 & 88.64±0.67  &  64.00±3.43 & 34.86±3.43        \\
&MARIO             & 94.36±1.21 & 91.28±1.10  &  65.67±2.81 & 37.15±2.37        \\ \bottomrule
\multirow{3}{*}{SAGE} 
&ERM               & 95.07±1.51 & 75.14±1.19  & 73.67±2.08  & 46.33±3.42       \\
&GRACE             & 95.29±1.11 & 74.43±2.36  & 70.50±5.06  & 49.54±3.83        \\
&MARIO             & 96.00±1.07 & 76.29±3.01  & 71.00±3.82  & 51.74±4.63        \\ \bottomrule
\multirow{3}{*}{GAT} 
&ERM               & 78.64±3.63 & 72.93±2.64  & 61.33±3.71  & 28.99±2.63        \\
&GRACE             & 84.57±1.79 & 78.36±1.60  & 59.50±2.36  & 35.78±3.26        \\
&MARIO             & 84.93±1.95 & 80.43±1.89  & 62.17±4.78  & 38.17±3.10        \\
\bottomrule
\end{tabular}}
\end{table}



\noindent In the subsection, we demonstrate the model-agnostic nature of the recipe by integrating it with various graph neural network (GNN) models, including GCN, GraphSAGE, and GAT.

From Table~\ref{tab:others}, it can be observed that regardless of the specific GNN model used as the encoder, our method consistently achieves the best performance on the OOD test set. This indicates the effectiveness and robustness of our method across different GNN models.
By achieving superior performance across different GNN models, MARIO demonstrates its versatility and ability to improve the OOD generalization of various graph neural models. This highlights the broad applicability and effectiveness of our recipe in enhancing the performance of different GNN encoders.

Furthermore, we integrate our recipe with other GCL methods in Appendix~\ref{app:other_methods}. The results demonstrate our recipe can boost the OOD generalization ability of various GCL methods which means our recipe can serve as a plug-in for many current classical GCL methods.

% Figure environment removed

\subsection{Visualization}\label{sec:vis}
\subsubsection{Metric Score Curves}
We present metric score curves for ERM and MARIO, including training, ID validation, ID testing, OOD validation, and OOD testing accuracy, in Figure~\ref{fig:curve2}. Notably, MARIO demonstrates superior convergence with approximately 10\% absolute improvement on the OOD test set compared to ERM. Furthermore, MARIO effectively narrows the performance gap between in-distribution and out-of-distribution performance, showcasing its efficacy in enhancing OOD generalization for graph data. More metric score curves can be found in Appendix~\ref{app:curves}.


\subsubsection{Feature Visualization}
In order to assess the quality of learned embeddings, we adopt t-SNE~\cite{tsne} to visualize the node embedding on GOOD-Cora dataset (concept shift in word domain) using random-init of GCN, EERM, GRACE, and MARIO, where different classes have different colors in Figure~\ref{fig:vis}. For clarity, we select eight classes with the largest number of nodes to enhance the informativeness and interpretability of the visualization. We can observe that the 2D projection of node embeddings learned by MARIO has a better separation of clusters, which indicates the model can help learn representative features for downstream tasks. It has to note that we depict both ID nodes and OOD nodes in the same figure. 

Besides, we also separately visualize ID nodes and OOD nodes in the different figures in the Appendix~\ref{app:feature}. And we can find MARIO performs a clearer separation of clusters whether on ID nodes or OOD nodes compared to other methods.



%% -*- mode: LaTeX; fill-column: 78; -*-

\section{Concluding Remarks}
\label{sec:conclusions}

In this paper, we presented a novel SMC algorithm, \EventDPOR, tailored to the
characteristics of event-driven multi-threaded programs running under the SC
semantics. The algorithm was proven correct and optimal for event-driven
programs in which the variable accesses of events do not depend on how their
execution is interleaved with other threads.

We have implemented \EventDPOR in the \Nidhugg tool, and we will open-source
our implementation.
%
With a wide range of event-driven programs, we have shown that \EventDPOR
incurs only a moderate constant overhead over its baseline implementation
(\OptimalDPOR), it is exponentially faster than existing state-of-the-art SMC
algorithms in time and number of traces examined on programs where events'
actions do not conflict, and does not suffer from performance degradation
caused by having to examine
% a significant number of
non-serializable executions.
%
%% \bjcom{Should we include:
%% Moreover, in our benchmarks, also those that are not non-branching,
%% \EventDPOR explores only the optimal number of executions, and never
%% had to resort to a potentially expensive decision procedure.}

\EventDPOR assumes that handlers can process their events in arbitrary order.
Directions for future work include to retarget \EventDPOR for event-driven
programs with other policies (e.g., FIFO), and for specific event-driven
execution models.


{\small\bibliographystyle{ieee_fullname}\bibliography{vedaldi_general,vedaldi_specific,local}}

\beginsupplement
\begin{refsection}

\section{Pseudocode Example of Cumulative Disruption Algorithm} \label{sec:psuedocode}

For readers seeking a succinct code-like description of our cumulative disruption curve algorithm, we have included \cref{lst:psuedocode}.

\begin{lstlisting}[label=lst:psuedocode, language=Python, caption=Pseudocode for disruption algorithm]
disruption = []
for c in communities:
    remaining = 0
    original = 0
    removeCommunity(c)
    for user in users:
        if degree(user) > 0:
            remaining += degree(user)
            original += originalDegree(user)
    disruption += [1 - (remaining / original)]
\end{lstlisting}

Note that when calculating disruption on large networks, it is much more efficient to cache the size of the smallest community that each user participates in. We can then sort all users by the order in which they will be removed, and avoid computationally expensive references to a graph or adjacency matrix for each removal-step in the algorithm.

\section{Applications to Unipartite Networks} \label{sec:unipartite}

Our influence metric is intended for settings with clearly defined communities. For example, participation in subreddits, membership on a Mastodon server, or committing to a software code repository, all discretely identify users as members of those explicitly-bounded groups. However, network data is often presented in a unipartite configuration such as users following other users. If it is still desirable to delineate communities and measure their influence in these settings, then they can be converted into compatible bipartite networks using the following procedure:

\begin{enumerate}
    \item Apply a context-appropriate community detection algorithm to label each user as belonging to one community

    \item Create a vertex for each community

    \item Replace all user-user edges with user-to-community edges, where the edge weight is equal to the number of unipartite edges each user had to other nodes in that community

    \item Apply our influence metric to the resulting bipartite graph
\end{enumerate}

An example of this procedure is illustrated in \cref{fig:unipartite}, using a unipartite Watts-Strogatz small-world network (100 nodes, 5 neighbors, rewiring probability of 5\%), and label-propagation for community detection. The unipartite graph is shown in the top-left with community labels visualized with color. It is converted to a bipartite representation shown in the upper-right, and the effect of removing each community is illustrated in the bottom frame.

% Figure environment removed






\section{Calculating the Area Under the Disruption Curve} \label{sec:auc_explanation}

For \cref{fig:real_networks_auc,fig:toy_networks_auc,fig:assortativity_auc} we use the area under the disruption curve as a single-variable summary of how centralized a network is around its largest communities. To calculate the AUC, we use a trapezoidal approximation in logarithmic space.

We chose a trapezoidal approximation to calculate the area even with limited sample points from real-world networks. Integration is possible for purely analytic disruption curve simulations as in \cref{sec:analytic_simulations}, but this is not feasible for our non-Erd\H{o}s-R\'{e}nyi networks, so we use a trapezoidal approximation for all synthetic networks for consistency.

We measure the AUC in logarithmic space, because measuring in linear space would heavily weight the influence of the smallest communities that are removed last, and our primary interest is in examining the influence of the largest communities on the broader population. 

\section{Synthetic Network Topology Details} \label{sec:toy_examples}

We measure centralization on a variety of synthetic networks introduced in \cref{sec:disruption_toy}. In this section, we include further description and visualization of the synthetic networks used.

Bipartite Near-Star networks are analogous to a unipartite star network with duplicate edges, but in a bipartite setting. Starting with a unipartite star, replace each edge from the hub to a leaf with a two-path from the hub community to a new ``user" vertex, to the leaf community. Duplicate edges from the unipartite hub to leaves are converted into multiple users that share a community, and serve to break ties when pruning communities for disruption curves. This is illustrated in \cref{fig:star}.

% Figure environment removed

For our ``Powerlaw" networks we follow a bipartite configuration model. We first create vertices representing the desired number of communities and users. We then draw from a powerlaw distribution with an assigned $\gamma$ exponent, and assign the drawn degree to each community. Then, we create a corresponding number of edges, wiring each community to users drawn uniformly at random without replacement. This yields networks where communities follow a powerlaw degree distribution, while users follow a normal degree distribution.

Bipartite community-user networks can be visualized in a flat plane, as in \cref{fig:centralization-pl}, or as a multi-layer graph, as in \cref{fig:pl-toy}. A multi-layer representation may be beneficial for representing inter-community relationships that are not explained by shared users, such as Mastodon federation agreements, or shared moderator staff in two subverses. However, these multiplex relationships were deemed out-of-scope for our current work.

% Figure environment removed




\begin{comment}
  #data structure for the dispersion metric
  D = np.zeros(nm)

  #calculate dispersion
  cumu_sum=0
  for n in np.arange(0,nm):
    cumu_sum += n*pn[n]
    #calculate U_n
    if pn[n]==0:
      continue
    Pnpm = Pnm[n,:]/np.sum(Pnm[n,:])
    U=0
    for m in np.arange(0,mm):
      if(np.sum(Pnm[:,m])>0):
        Pnmp = Pnm[:,m]/np.sum(Pnm[:,m])
        prob = np.sum(Pnmp[n:-1])
        U+=Pnpm[m]*prob**(m-1)
    D[n] = n*pn[n]*(1-U)/(cumu_sum-n*pn[n]*U)
\end{comment}

\section{Mathematical Analysis of Disruption in Random Networks} \label{sec:analytic_simulations}

We here calculate the disruption curves for random bipartite networks parameterized by their joint-degree distribution. This approach therefore fixes the distribution $\lbrace g_m \rbrace$ of communities $m$ per user, the distribution $\lbrace p_n \rbrace$ of community size $n$, and the joint-distribution $P_{n,m}$ for the degree of the node and community involved in a random bipartite link. Beyond these constraints, the networks are fully random but allow us to explore the role of heterogeneous connectivity at the user and community level as well as the impact of correlations between both levels.

We wish to calculate the disruption $D(n)$ involved when removing communities of size $n'<n$ in these random networks. By definition of the bipartite network, we know that $np_n$ edges are removed when removing communities of size $n$. Once again, we define disruption as the fraction of \textit{remaining} edges disrupted by communities of size $n$ during the pruning process. It is thus given by the number of edges that belong to communities of size $n$ minus the fraction $u_n$ of those that are the sole edge of the corresponding users (since these users are removed in the pruning) divided by the number of edges belonging to communities of size equal or smaller than $n$ minus the $u_nnp_n$ users removed. We write:

\vspace{2em}
\begin{equation}
    D(n) = \frac{
            \eqnmarkbox[NavyBlue]{bigedges}{np_n}
            -
            \eqnmarkbox[OliveGreen]{prunededges}{u_nnp_n}
        }{
            \eqnmarkbox[WildStrawberry]{remainingedges}{\sum_{n'\leq n}n'p_{n'}}
            -
            \eqnmarkbox[OliveGreen]{prunededges2}{u_nnp_n}
        } \; .
    % Here's Laurent's original expression
    %D(n) = \frac{np_n-u_nnp_n}{-u_nnp_n + \sum_{n'\leq n}n'p_{n'}} \; .
\end{equation}
\annotate[yshift=1em]{above,left}{bigedges}{Edges to comms. of size n}
\annotate[yshift=1em]{above,right}{prunededges}{Edges to removed users}
%\annotate[yshift=-1em]{below,right}{prunededges2}{Edges for removed users}
\annotate[yshift=-0.5em]{below}{remainingedges}{Edges to comms. n or smaller}
\vspace{2em}

The quantity $u_n$ can also be defined as the probability that a random user of a community of size $n$ has no community smaller than $n$. It can therefore be calculated like so:

\vspace{1em}
\begin{equation}
    u_n = \mathlarger{\sum}_m 
        \eqnmarkbox[NavyBlue]{users_in_n_with_m}{\frac{P_{n,m}}{\sum_{m'}P_{n,m'}}}
        \left(
            \eqnmarkbox[OliveGreen]{users_with_m_larger_than_n}{\frac{\sum_{n'\geq n} P_{n',m}}{\sum_{n'}P_{n',m}}}
        \right)^{m-1} \; .
    %u_n = \mathlarger{\sum}_m \frac{P_{n,m}}{\sum_{m'}P_{n,m}} \left(\frac{\sum_{n'\geq n} P_{n',m}}{\sum_{n'}P_{n',m}}\right)^{m-1} \; .
    \label{eq:un}
\end{equation}
\annotate[yshift=1em]{above,right}{users_in_n_with_m}{Fraction of users in comm. \\ \sffamily \footnotesize size n that have m edges}
\annotate[yshift=-0.5em]{below,left}{users_with_m_larger_than_n}{Fraction of users with m edges\\ \sffamily \footnotesize in comms. larger than size n}
\vspace{2.5em}

In the previous equation, we sum over every possible type of node in a community of size $n$, which will have a number of \textit{other} communities $m-1$ proportional to $P_{n,m}$, and ask for all of these communities to be larger or equal to $n$, which will be proportional to the sum of $P_{n',m}$ over all $n'$ larger or equal to $n$. Normalizing the probabilities appropriately yields Eq. (\ref{eq:un}) as written.

Note that these equations assume that edges are unweighted, and that there are no duplicate edges, which is what we expect from an infinite random simple graph. In our real-world data sets there are often duplicate edges (for example, one user following several different users on a Mastodon instance), which we compress to weighted edges for convenience.

Despite this difference between the analytical expression and real socio-technical networks, the analysis of random infinite graphs can be useful to test how disruption is impacted by simple network statistics such as degree distributions or correlations in the joint community-user degree matrix $P_{n,m}$. 

In a simple experiment, we create a random Erd\H{o}s-R\'{e}nyi-like bipartite network and correlated equivalent networks with the same degree distributions and variable community-user degree matrices $P_{n,m}$. The random network has a simple $P^{\textrm{rand}}_{n,m} \propto np_n mg_m$ (normalized) which we can modify manually. To do so, we calculate the maximally correlated $P^{\textrm{max}}_{n,m}$ by assigning users with highest degrees $m_{\textrm{max}}$ to the largest communities available before doing the same to users with the next higher degree and so on all the way down. We can do the same to calculate $P^{\textrm{min}}_{n,m}$ by assigning users with the lowest degree to the largest communities and working our way up in the user degree distribution. We can then create arbitrary community-user degree matrix $P_{n,m}$ by interpolating between linearly with $(1-\rho) P^{\textrm{rand}}_{n,m} + \rho P^{\textrm{max}}_{n,m}$ or $(1-\rho) P^{\textrm{rand}}_{n,m} + \rho P^{\textrm{min}}_{n,m}$.

Our results are shown in \cref{fig:assortivity_random_networks}. We find that positive user-community degree correlations increase disruption and therefore \textit{centralizes} the resulting socio-technical network. Conversely, negative correlations decreases correlations and \textit{decentralizes} the network. That being said, the relative effect of correlations is relatively small as the networks are still otherwise completely random.

% Figure environment removed

\section{Further Analysis of Assortativity} \label{sec:supplemental_assortativity}

There are multiple interpretations of degree assortativity in a bipartite setting. The linear correlation between user degrees and community degrees measures whether high-degree users are likely to be connected to high-degree communities. In our network definitions edges represent activity, like follow relationships or participation in conversations, so this measures whether active users are likely to be connected to communities with lots of activity. However, a second metric of interest is whether large communities are likely to be connected to other large communities, or in other words, the  assortativity of a unipartite-projected community-community graph. This can also be broken into two sub-cases: assortativity of community size (do communities with many users share users with other high-population communities), and assortativity of degree (do communities with lots of activity share users with other high-activity communities).

These three notions of assortativity are not independent; we might expect that users with lots of activity are active in communities with high populations, and may act as bridges between multiple communities with high activity and high population. However, the three metrics are not guaranteed to correlate and should be measured separately.

While rewiring to promote user-community degree assortativity, we also plotted the changes in community-community degree assortativity, shown in \cref{fig:assortivity_user_vs_community}. Strikingly, the community assortativity \textit{decreases} as we rewire to promote user assortativity. This is because as we rewire edges to focus user connections on the largest communities we implicitly decrease the number of edges between communities. This also matches the changes in disruption in \cref{fig:assortativity_auc}: increasing assortativity may reconnect large and insular communities with the rest of the network, briefly increasing their influence, but continued assortativity rewiring also cuts bridges to and between smaller communities, yielding a sparse network that is far less centralized.

% Figure environment removed

To further explore the relationship between these types of assortativity, we also rewired networks in the reverse direction: for randomly selected pairs of edges, we rewired those edges to \textit{decrease} user to community activity assortativity. We have plotted the change in disruption curves (\cref{fig:disassortative_auc}) and correlation between assortativity metrics (\cref{fig:disassortivity_user_vs_community}). In most networks, decreasing activity assortativity lowers centralization, although the effect diminishes as the network topology more closely approximates a random network. The one exception is the Penumbra; this network has such sparse inter-community connections that any perturbation of edges increases the cross-community links and therefore \textit{increases} centralization.

% Figure environment removed

% Figure environment removed

\section{Cumulative Impact on Giant Component Size} \label{sec:giant_components}

Some readers may be interested in how removing large communities influences the giant component size on each network. This is closely related to the cumulative population size in the top sub-plots of \cref{fig:real_networks_size_comparison} and \cref{fig:toy_networks_size_comparison}. Intuition suggests that the size of the giant component will be inversely proportional to the number of cumulative communities removed; as more large communities are pruned, the giant component should shrink. This relationship holds so long as the remaining communities are interlinked, but falters once a ``bridge" community is removed and the giant component splinters. Therefore, sparsely connected networks where bridges are more prominent will have a chaotic giant component size, while more densely connected networks will present a smooth curve until most communities are pruned. This relationship is illustrated in \cref{fig:real_giant_component}. Most curves are smooth until the tail of the distribution, with two notable exceptions: Voat's giant component changes once the largest insular communities are removed (see \cref{fig:voat_render}), and the Penumbra's curve is much ``spikier" as a result of its highly sparse structure.

% Figure environment removed

Measuring the change in giant component size captures some of the same features as our disruption metric. In particular, removing large insular communities may not change the giant component size if the community is completely isolated from the giant component, so this captures some aspect of both the size and topological role of a community. However, the impact of a community is boolean: if it touches the giant component, then removing the community will shrink the giant component by the size of that community. There is no distinction between a minimally integrated and tightly integrated community. Measuring the impact of a community in terms of fraction of edges severed, rather than component vertex size, offers finer insight into the interplay between size distribution and network structure.



\section{Comparison to Network Bottlenecking} \label{sec:cheeger}

The Cheeger number \cite{cheeger} is a single-valued metric representing how large of a ``bottleneck" inhibits conductance across a graph. It is typically written as:

\vspace{2em}
\begin{equation}
    h(G) = \min \left\{
        \frac{
                \eqnmarkbox[NavyBlue]{cheeger_crossedges}{|\partial A|}
            }{
                \eqnmarkbox[OliveGreen]{cheeger_alledges}{|A|}
            }
        : \eqnmarkbox[WildStrawberry]{cheeger_subset}{A \subseteq V(G)}, 
        \eqnmarkbox[Plum]{cheeger_bounds}{0 < |A| \leq \frac{1}{2} |V(G)|}
    \right\} 
\end{equation}
\annotate[yshift=1.2em]{above}{cheeger_crossedges}{Edges crossing the boundary of A}
\annotate[yshift=-0.2em]{below}{cheeger_alledges}{All edges in+across A}
\annotate[yshift=0.8em]{above}{cheeger_subset}{A is a subset of vertices of G}
\annotate[yshift=-2em]{below,left}{cheeger_bounds}{A contains at most half of all vertices}
\vspace{2em}

Our measurement of how much a community influences a larger population, and the Cheeger measurement of whether a community is a ``bottleneck" bear some conceptual similarities. Therefore, we compare our metric to the Cheeger number in two ways. First, we create a ``local Cheeger number," following an identical equation $\frac{|\partial A|}{|A|}$, but where $A$ is defined as the set of communities we are pruning, rather than via a global search. Second, we estimate bounds on the global Cheeger value of the graph. Since evaluating the graph conductance of all possible subsets of vertices is an NP-hard problem \cite{kaibel2004expansion}, it is impractical to directly measure the Cheeger constant on most large graphs. Fortunately, the Cheeger inequality offers upper and lower bounds on the Cheeger number based on the second eigenvalue of the normalized Laplacian of the adjacency matrix of G as follows:

$$\lambda_2/2 \leq h(G) \leq \sqrt{2\lambda_2}$$

Since they are sparse, these bounds can be calculated even on large real-world datasets. 
Unfortunately, in our tests the bounds are quite wide (see \cref{fig:cheeger}), limiting the utility of this approximation. We have plotted a comparison of the ``local" Cheeger number, bounds of the global Cheeger number, and our disruption metric, for a variety of simulated networks.

% Figure environment removed

\printbibliography[heading=subbibliography]
\end{refsection}

\end{document}