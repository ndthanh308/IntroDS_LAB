
\documentclass[11pt]{article}

%\usepackage{my_let_head}
\usepackage{fancyhdr}
\usepackage{amsmath}
\usepackage{amssymb}
\usepackage{amsthm}
\usepackage{graphicx}

\usepackage{color}

\pagestyle{plain}
\renewcommand{\headrulewidth}{0.0pt}
\renewcommand{\footrulewidth}{0.0pt}

\renewcommand{\rmdefault}{phv}\renewcommand{\sfdefault}{phv}

\def\FirstAuthor{\noindent Yuga Iguchi}
\def\SecondAuthor{\noindent Alexandros Beskos}
\def\ThirdAuthor{\noindent Matthew Graham}

\def\Me    {Mr Yuga Iguchi}
\def\What    {PhD Candidate}
\def\Department {Department of Statistical Science}
\def\College    {University College London}
% \def\AddressOne   {1-19 Torrington Place}
% \def\City       {London, UK}
% \def\ZIP        {WC1E 6BT}
\def\Email   {yuga.iguchi.21@ucl.ac.uk}

%\def\toperson{\noindent Dr. L. Nash}
\def\toplace{\noindent \textit{Journal of the Royal Statistical Society, Series B (Statistical Methodology)}}
%\def\inrefto{\noindent Paper 05.070}



%%%%%%%%%%%%%%%%%%%%%%%%%%%%%%%% MARGINS %%%%%%%%%%%%%%%%%%%%%%%%%%%%%%%%%%%%
\textwidth       6.75in \textheight      10.25in \oddsidemargin
-.25in \evensidemargin  -.25in \topmargin       -1.00in
\parindent       5ex

\pagestyle{fancy} \lfoot{\scriptsize} \cfoot{} \rfoot{ } \lhead{
} \chead{} \rhead{ }

\begin{document}
\vspace{-0.75in} \hfill \vspace{-1.6in}
  {\footnotesize\begin{tabular}{ll}
\hspace{1.0 in } & \\
               &\Me\\
               &\What\\
               &\Department\\
               &\College\\
               % \\
               % &\AddressOne\\
%               &\AddressTwo\\
               %&\City\\
               % &\ZIP\\
               &\Email\\
               %& {\scriptsize\URL} \\
\\
\end{tabular}}


\vspace{1.75in}

\noindent \today

%\toperson

\toplace


 \vspace{0.60in}

\noindent Dear Sir/Madam

\vspace{0.20in}

\noindent Please find attached our paper \emph{`Parameter Inference for Degenerate Diffusion Processes'} for consideration of publication in the Journal of the Royal Statistical Society, Series B (Statistical Methodology).

We study parametric inference for hypo-elliptic \mbox{Stochastic} Differential Equations (SDEs), i.e., SDEs containing smooth components not directly driven by Brownian motion. Recently, several works \cite{dit:19, glot:21, iguchi:22} have looked at the development of careful parameter inference methods which avoid biases (characterising attempts based on immediate transfer of methodology from elliptic SDEs)  for a particular class of hypo-elliptic SDEs, where the noise in the rough components directly propagates into the smooth components. However, inference methods have yet to be developed for further important classes of hypo-elliptic SDEs used in applications, including, e.g., the family of generalised Langevin equations (GLEs). 
We focus on these underexplored classes, where 
now smooth components are not directly affected by rough components. We note that parameter inference for GLEs has recently been required in practical applications (see e.g.~\cite{vr:22}), however the  approach followed in the referenced paper \cite{vr:22} leads to biased estimates. 

Our main contribution is to establish a general {recipe} to carry out (unbiased) parameter inference for a wide class of hypo-elliptic SDEs. Our work focuses on what we call \emph{highly degenerate} SDEs, and proposes a tailored time-discretisation scheme that plays a key role in avoiding biases. We then show asymptotic normality of the contrast estimator based upon the scheme in a high-frequency full observation regime. The unbiasedness property is also verified via simulation studies in the practical scenario where only smooth components are observed. We believe that our work can have an important impact in the research area of parametric inference for SDEs, providing new contributions both from an analytical and practical point of view.  

\vspace{0.2 in}

\noindent Yours sincerely,

\vspace{0.2 in}

\FirstAuthor

\SecondAuthor

\ThirdAuthor

\bibliographystyle{abbrv}
\bibliography{letter} 
\end{document}





