% Powerset
\DeclareMathOperator{\pwr}{\mathscr{P}}

% Standard number sets
\newcommand{\N}{\mathbb{N}}
\newcommand{\Z}{\mathbb{Z}}
\newcommand{\Q}{\mathbb{Q}}
\newcommand{\R}{\mathbb{R}}
\newcommand{\C}{\mathbb{C}}

% Grammar definitions
\newcommand{\cceq}{\mathrel{\mathop{::=}}}

% Logical operators
\DeclareMathOperator{\limplies}{\rightarrow}
\DeclareMathOperator{\liff}{\leftrightarrow}
\DeclareMathOperator{\lneg}{\neg}
\DeclareMathOperator{\thesis}{\vdash}
\DeclareMathOperator{\inj}{in}
\let\th\thesis

% Restriction operator (upward harpoon with different spacing)
\newcommand{\rst}{{\upharpoonright}}

% Shorter overline to put above parentheses
\newcommand{\shorteroverline}[1]{\mkern 4mu\overline{\mkern-4mu#1\mkern-4mu}\mkern 4mu}

% Overline abbreviations
\let\co\overline
\let\dl\overline
\let\sdl\shorteroverline

% Graphic style for data structure fields
\newcommand{\data}[1]{\mathtt{#1}}

% Utility for optional arguments
\newcommand{\ifemp}[3]{\ifthenelse{\isempty{#1}}{#2}{#3}}

% Graphic style for inference rule labels
\newcommandx{\rl}[2][1]{%
    \ifemp{#2}%
        {\ensuremath{\mathtt{#1}}}%
        {\ensuremath{\mathtt{#2}_{#1}}}%
}

% Decorated rewriting arrow
\newcommandx{\rwr}[2][1,2]{\mathrel{%
    \ifemp{#1}%
        {\ifemp{#2}%
            {\longrightarrow}%
            {\stackrel{#2}{\longrightarrow}}%
        }%
        {\ifemp{#2}%
            {\longrightarrow_{\data{#1}}}%
            {{\stackrel{\raisebox{-2pt}{\ensuremath{\scriptstyle {#2}}}}{\longrightarrow}}_{\data{#1}}}%
        }%
}}

% Named atoms
\newcommand{\nm}[2]{{#2}^{#1}}
% alternative, requires stackengine
% \newcommand{\nm}[2]{\stackunder[1.5pt]{#2}{\scriptscriptstyle #1}}

% Graph constructions
\newcommand{\djg}[1]{\llbracket {#1} \rrbracket}
\newcommand{\axg}[1]{\llbracket {#1} \rrbracket}
\newcommand{\waxg}[1]{\left\llbracket {#1} \right\rrbracket}

\newcommand{\blwk}[1]{\data{Wk}^{\data{bl}}_{#1}}
\newcommand{\blid}[1]{\data{Id}^{\data{bl}}_{#1}}

\newcommand{\bldjg}[1]{\llparenthesis {#1} \rrparenthesis}
\newcommand{\blaxg}[1]{\llparenthesis {#1} \rrparenthesis}
\newcommand{\wblaxg}[1]{\left\llparenthesis {#1} \right\rrparenthesis}


% Edge-branch relation
\newcommand{\adin}{\mathrel{\Yleft}}

% High vphantom
\newcommand{\hvp}{\vphantom{\frac{0}{0}}}

% blg environment
\makeatletter
\let\blg@defaultnm\nm
\newcommand{\blg@markname}[1]{--blg-mark-#1}
\newcommand{\blg@branchnm}[2]{\blg@defaultnm{#1}{\tikzmarknode{\blg@markname{#1}}{#2}}}
\newsavebox{\blg@branchbox}
\newcommand{\blgbranch}[2]{%
    \begingroup%
    \let\nm\blg@branchnm%
    \savebox{\blg@branchbox}{\(#1\)}%
    \prfassumption{%
        \begin{tikzpicture}[remember picture, thick, black]
            \node[outer sep=0, inner xsep=0, inner ysep=-.2em] (branch) at (0,0) {\usebox{\blg@branchbox}};
            \foreach \x / \y in {#2}
                \draw ([yshift=.2em]branch.north -| {\blg@markname{\x}}.north) .. controls ++(0,1em) and ++(0,1em) .. ([yshift=.2em]branch.north -| {\blg@markname{\y}}.north);
        \end{tikzpicture}%
    }%
    \endgroup%
}
\makeatother
