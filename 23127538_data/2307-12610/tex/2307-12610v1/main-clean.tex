\documentclass[aps,twocolumn,
amsmath,prd,superscriptaddress,notitlepage
]{revtex4-1}

\usepackage[utf8]{inputenc}
\usepackage[T1]{fontenc}
\usepackage{dcolumn}
\usepackage{bm}
\usepackage{amsmath}
\usepackage{graphicx} 
\usepackage{tabularx}
\usepackage{multirow}
\usepackage{rotating} 
\usepackage{makecell}
\usepackage{gensymb}
\usepackage[normalem]{ulem} 
\usepackage{amssymb}
\usepackage{textgreek}
\usepackage[normalem]{ulem}
\usepackage[usenames, dvipsnames]{xcolor}
\usepackage{hyperref}
\hypersetup{colorlinks=true, linkcolor=blue, citecolor=blue, urlcolor=blue}



\newcommand\blue[1]{{\color{blue}#1}}
\newcommand\red[1]{{\color{red}#1}}
\newcommand\org[1]{{\color{orange}#1}}
\newcommand\MA[1]{{\color{magenta}#1}}
\newcommand\MF[1]{{\color{green}#1}}
\newcommand\purple[1]{{\color{purple}#1}}

\begin{document}
\title{Lattice dynamics in the intermetallic LaFeSi and the derived superconducting compounds LaFeSiH and LaFeSiO}



\affiliation{CNRS, Universit\'e Grenoble Alpes, Institut N\'eel, 38042 Grenoble, France}
\affiliation{Department of Physics, School of Engineering, University of Petroleum and Energy Studies (UPES), Dehradun, Uttarakhand 248007, India}

\affiliation{Univ. Bordeaux, CNRS, Bordeaux INP, ICMCB, UMR 5026, F-33600 Pessac, France}




\author{Samar Layek (equal contribution)}
\affiliation{CNRS, Universit\'e Grenoble Alpes, Institut N\'eel, 38042 Grenoble, France}
\affiliation{Department of Physics, School of Engineering, University of Petroleum and Energy Studies (UPES), Dehradun, Uttarakhand 248007, India}
\author{Mads Fonager Hansen (equal contribution)}
\affiliation{CNRS, Universit\'e Grenoble Alpes, Institut N\'eel, 38042 Grenoble, France}
\author{Jean-Baptiste Vaney}
\affiliation{Univ. Bordeaux, CNRS, Bordeaux INP, ICMCB, UMR 5026, F-33600 Pessac, France}
\author{Pierre Toulemonde}
\affiliation{CNRS, Universit\'e Grenoble Alpes, Institut N\'eel, 38042 Grenoble, France}
\author{Sophie Tenc\'e}
\affiliation{Univ. Bordeaux, CNRS, Bordeaux INP, ICMCB, UMR 5026, F-33600 Pessac, France}
\author{Philippe Boullay}
\affiliation{Normandie Universit\'e, ENSICAEN, UNICAEN, CNRS, CRISMAT, 14000 Caen, France}
\author{Andres Cano}
\affiliation{CNRS, Universit\'e Grenoble Alpes, Institut N\'eel, 38042 Grenoble, France}
\author{Marie-Aude M\'easson}
\affiliation{CNRS, Universit\'e Grenoble Alpes, Institut N\'eel, 38042 Grenoble, France}
%%%%%%%%%%%%%%%%%%%%%%%%%%%%%%%%%%%%%%%%%%%%%%%%%%%%%%%%%%%%%

\date{\today}

\begin{abstract}
The intermetallic LaFeSi and the derived superconducting compounds LaFeSiH and LaFeSiO$_{1-\delta}$ have been investigated by polarized Raman spectroscopy. The frequency and symmetry of the Raman phonons modes are well-reproduced by \textit{ab-initio} calculations. The ionic character of the spacer in this series of compounds and its coupling with the FeSi layers as compared to As-based compounds are discussed. Already at room temperature, Fano-shape modes are reported in the A$_{1g}$ channel while an intriguing doubling of the Fe-based B$_{1g}$ phonon is measured in LaFeSiH. Origins of these observations are discussed based on electron diffraction data and the different scenarios for the origin of such splitting are explored. Furthermore, there is no signature of a structural transition nor long range magnetic ordering in LaFeSiH down to 9~K.

\end{abstract}




\maketitle

\section{Introduction} 

Iron-based superconductors provide a distinct realization of unconventional superconductivity \cite{Hosono2015,Alloul2016}. At present, there is a renewed interest in these systems owing to the intriguing topological features which can supplement their superconductivity \cite{Zhang2018,zhang2019}. This further motivates the search for alternative materials in this class, where the Fe layer is combined with elements other than As or Se, such as non-toxic crystallogens. At present, there are four successful examples fulfilling this requirement, YFe$_2$Ge$_2$ \cite{Zou2014, Kim2015, Chen2016}, LaFeSiH \cite{Bernardini2018,Bhattacharyya2020,Hansen2023}, LaFeSiF \cite{Vaney2022}, and LaFeSiO \cite{Hansen2022}. The discovery of superconductivity in these compounds, and in particular LaFeSiO, challenges the quasi-universal correlation between the anion height and T$_{c}$, which has been put forth by several authors \cite{Okabe2010,Mizuguchi2010,Sun2018,Lee2012}. The anion height in LaFeSiO represents a remarkably compressed Fe-layer, and with a T$_{c}$ of 10 K, this compound falls outside the otherwise established correlation \cite{Hansen2022}. This, along with calculations showing a loss of the electron-like pockets around the M-point in LaFeSiO and LaFeSiF$_{1/8}$, further motivates the investigation of these materials \cite{Hansen2022,Vaney2022}. Since the crystal structure plays an important in role in determining the electronic properties, especially in these superconductors with local inversion-symmetry breaking \cite{Khim2021}, probing the dynamic structure becomes an interesting avenue of study, complementary to the crystal structure. The substitution of Si for As induces a compression of the Fe-layer, as expected, considering the difference in atomic radii. The compounds LaFeSiH and LaFeSiO show notably different anion heights despite having very similar T$_{c}$ values.

In this article, we focus on the FeSi-based superconductors and their LaFeSi precursor and report a lattice dynamics study 
by means of polarized Raman spectroscopy combined with density-functional-theory (DFT) calculations. The symmetry properties of the measured phonons, deduced from their selection rules, together with their frequencies are compatible with the calculations. 
In addition, we observe Fano-shape modes with various interaction intensities. Additionally, an intriguing doubling of Fe-based mode in LaFeSiH is measured. 

\section{Methods}

\subsection{Synthesis}

Single crystals of the intermetallic LaFeSi were extracted from the bulk of an arc-melted La$_{35}$Fe$_{35}$Si$_{30}$ sample. The superconducting LaFeSiH single crystals were then obtained by hydrogenation of LaFeSi single crystals at 250°C under a gas flow of H$_2$ for 4h. These crystals are stable in atmospheric condition and crystallize in the tetragonal ZrCuSiAs-type structure which has been confirmed using X-ray diffraction on both phases. Both crystals have sub-millimeter sized plate-like morphology with the larger faces perpendicular to the crystallographic \textit{c}-axis, due to the layered nature of the crystal structure. The face orientation has also been confirmed from the single crystal X-ray diffraction measurements. Typical size of the crystals were 1$\mathrm{\times}$1 mm$^{2}$ and $\sim$~20 \textmu m thickness.
The samples of LaFeSiO were obtained by heating powders of LaFeSi at 320 \textdegree{}C in 20\%/80\% O$_{2}$/Ar flow for 72h. Microscopic crystals were chosen from a powder and the faces were assumed to be parallel to the (001) plane as is the case for the other compounds. In the case of LaFeSiO, we measured polycrystaline samples having single-crystal domains large enough to gain well-defined polarized Raman spectra. The occupancy of the oxygen site has previously been found to be around 0.9, which is also in the range of the LaFeSiO samples measured in this study.


\subsection{Raman experiments} 

The Raman spectra of small single crystals were measured under a microscope attached to a WITec (Model: alpha 300R) confocal Raman spectrometer. A 532~nm laser line was used for the excitation in the Raman measurements presented here. The laser power was always kept below 1 mW in order to avoid local overheating and/or damaging the crystals throughout the experiments. Macro-Raman as a function of temperature was performed on a triple-stage spectrometer and in a cryo-free cryostat down to 9~K. Different polarizations of the light and orientations of the samples were investigated to access the symmetries of the excitations.

\subsection{Single crystal 3D electron diffraction} 

Precession electron diffraction tomography (PEDT) was performed with a JEOL F200 cold-FEG transmission electron microscope operated at 200 kV, equipped with a NANOMEGAS DigiStar precession module and a GATAN RIO16 camera. Step-by-step PEDT data collection was performed with a home-made GATAN Digital Micrograph script using a goniometer tilt step of 1 degree and a precession semi-angle set to 1.1 or 1.4 degree.
Samples for PEDT investigations were prepared by smoothly crushing powder under ethanol in an agate mortar and depositing drops of the mixture onto a holey carbon membrane supported by a Cu grid. PEDT data was processed using the programs PETS 2.0 \cite{Palatinus2019} and Jana2020 \cite{Petricek2014}. Crystallographic details of data reduction and dynamical refinement results are given in Supplementary Materials.

\subsection{DFT calculations}

The DFT calculations were performed with the Quantum {\sc ESPRESSO} package \cite{Giannozzi2009} using the norm-conserving pseudopotentials from the PseudoDojo library \cite{VanSetten2018}. We used the Perdew-Burke-Ernzerhof form of the generalized gradient approximation \cite{Perdew1996}. The calculations were converged with a Monkhorst-Pack mesh of 13$\times$13$\times$7 $k$-points and a 125~Ry cutoff for the wavefunctions with a 0.01~Ry smearing. We used the experimental lattice parameters reported in \cite{Welter1992,Bernardini2018, Hansen2022} and optimized the internal coordinates of the La and Si atoms. 


\section{Results}

\subsection{Phonon modes and Raman selection rules}

We first recall the nature of the $\Gamma$-point phonons expected in our systems and their Raman selection rules. The systems under consideration represent the FeSi-based counterparts of previous 111 and 1111 Fe-based superconductors whose crystal structure corresponds to the $P4/nmm$ space group ($D_{4h}$ point group). The contribution of the different atoms to the different $\Gamma$-point phonons is summarized in Table~\ref{tab2} \cite{Ivantchev2000}.   
In the case of the LaFeSi precursor, we then have two A$_{1g}$, one B$_{1g}$ and three E$_g$ Raman-active modes. 
In LaFeSiH and LaFeSiO, there is an additional B$_{1g}$ Raman-active mode as well as an E$_g$ mode. 

In our experiments, the \textit{c}-axis of the samples is parallel to Poynting vector of light. 
The E$_g$ modes are Raman inactive for this configuration. 
As a result, we then probe a total of 2A$_{1g}$+ B$_{1g}$
and 2A$_{1g}$+ 2B$_{1g}$ Raman-active modes in LaFeSi and LaFeSiH/O, respectively.
These modes are associated with out-of-plane displacements of the atoms. In the following, we denote these modes as $A_{1g}^{\rm La/Si}$ and $B_{1g}^{\rm Fe/H/O}$ according to the main contribution to these displacements. However, for our discussion below, it will be important to keep in mind that these modes generally have a mixed character with dominant and subdominant displacements of different atoms. 
Further selection rules according to the polarization of the light are summarized in Table~\ref{table_1}.  





\begin{table}[h!]
\footnotesize 
\begin{tabular}{ c c c } \hline \hline
Atom (Wyckoff position) & \textbf{${\Gamma}$}-point modes \\ \hline 
La  ($2c$) & \textbf{A$_{1g}$}+A$_{2u}$+\textbf{E$_g$}+E$_u$ \\ 
Fe  ($2b$) & \textbf{B$_{1g}$}+A$_{2u}$+\textbf{E$_g$}+E$_u$ \\ 
Si  ($2c$) & \textbf{A$_{1g}$}+A$_{2u}$+\textbf{E$_g$}+E$_u$ \\ 
H/O  ($2a$) & \textbf{B$_{1g}$}+A$_{2u}$+\textbf{E$_g$}+E$_u$ \\ \hline \hline
\end{tabular}
\caption{Atoms and $\Gamma$-point phonons to which they contribute in LaFeSi and LaFeSiH/O. The Raman-active modes are indicated in bold.}
\label{tab2}
\end{table}



\begin{table}[h!]
\footnotesize 
\begin{tabular}{c c c} \hline \hline
 Polarization geometry && Raman-active phonons \\ \hline 
$c(aa)\bar{{c}}$ && A$_{1g}$+B$_{1g}$\\  
$c(a'a')\bar{{c}}$ && A$_{1g}$+B$_{2g}$ \\
$c(ba)\bar{{c}}$ && A$_{2g}$+B$_{2g}$ \\ 
$c(b'a')\bar{{c}}$ && A$_{2g}$+B$_{1g}$ \\\hline \hline
\end{tabular}
\caption{Raman selection rules in the $D_{4h}$ point group in Porto notation. $a'$ stands for the (110) axis. \label{table_1} }
\end{table}



\subsection{LaFeSi precursor}

Figure \ref{Fig1} presents the polarized Raman spectra obtained on single crystals of the LaFeSi precursor at room temperature. 
The symmetry of the Raman-active phonons behind the observed peaks is deduced by changing the polarization geometry according to Table~\ref{table_1}. Thus, we clearly identify the two $A_{1g}$ modes associated with La and Si as well as the $B_{1g}$ mode associated with Fe. This identification is additionally supported by the DFT calculations (see table \ref{t:summary}), which provide the main atomic character of these modes. 
 

% Figure environment removed



\begin{table*}[t!]
\begin{tabular}{cccc|cccc|ccccc} 
\hline \hline
 \multicolumn{4}{c|}{LaFeSi} & \multicolumn{4}{c|}{LaFeSiH} & \multicolumn{4}{c}{LaFeSiO} \\ 
 \multicolumn{2}{c}{Frequency (cm$^{-1}$)} && Mode & \multicolumn{2}{c}{Frequency (cm$^{-1}$)} && Mode & \multicolumn{2}{c}{Frequency (cm$^{-1}$)} && Mode \\ 
Experimental & Calculated && symmetry & Experimental & Calculated && symmetry & Exp. & Cal. && symmetry  \\ \hline
       & 96 
       && $E_g$                &     & 98 
       && $E_g$             &   & 79 && $E_u$ \\
 112.6 & 123 
 && $A_{1g}^{\rm La}$ &      & 115 
 && $E_u$             &   & 106 && $A_{2u}$\\
       & 148 
       && $E_{u}$           & 146.4 & 120 
       && $A_{1g}^{\rm La}$&   & 123 && $E_g$ \\   
       & 152 
       && $A_{2u}$             &      & 122 
       && $A_{2u}$ &164.4 & 156 && $B_{1g}^{\rm Fe}$  \\ 
       & 209 
       && $E_g$              &    & 178 
       && $E_g$             &  & 179 && $E_{g}$ \\
205.6 & 218 
&& $B_{1g}^{\rm Fe}$  & 222.8, 228.5 & 228 
&& $B_{1g}^{\rm Fe}$&186 & 186 && $A_{1g}^{\rm La}$ \\   
250.0 & 313 
&& $A_{1g}^{\rm Si}$ & 279   & 289 
&& $A_{1g}^{\rm Si}$& -- & 272 && $B_{1g}^{\rm O}$ \\
    & 322 
    && $A_{2u}$             &    & 315 
    && $A_{2u}$           &    & 282 && $A_{2u}$ \\
    & 364 
    && $E_u$               &    & 403 
    && $E_u$               &    & 295 && $E_u$\\
    & 372 
    && $E_g$               &    & 407 
    && $E_{g}$             &260 & 297 && $A_{1g}^{\rm Si}$\\
    &           &&                     & 809.2 & 792 
    && $B_{1g}^{\rm H}$ &    & 389 && $A_{2u}$\\
    &           &&                     &       & 809 
    &&  $E_u$           &    & 414 && $E_g$ \\
    &           &&                     &       & 822 
    && $A_{2u}$         &    & 414 && $E_u$\\
    &           &&                     &       & 847 
    &&  $E_g$           &    & 433 && $E_{g}$\\
    \hline \hline 
\end{tabular}
\caption{Frequencies of the different $\Gamma$-point phonons of the investigated silicides together with their symmetries. 
In the case of the Raman-active modes observed in our experimental setup (i.e. $A_{1g}$ and $B_{1g}$), the main atomic displacements are indicated with the superscripts.
\label{t:summary}}
\end{table*}

\subsection{Superconducting LaFeSiH} 

Figure \ref{Fig2} shows the polarized Raman spectra obtained from single crystals of LaFeSiH. In addition to the two A$_{1g}$ modes associated with La and Si, two $B_{1g}$ modes associated with Fe and H are observed. 
The corresponding frequencies are in reasonable agreement with the calculations (Cf. \ref{t:summary}), which again provide the main atomic character of these modes. 


% Figure environment removed



% Figure environment removed

In the measured spectra, however, the $B_{1g}$(Fe) mode at $\sim$222~cm$^{-1}$ appears as a double-peak feature. This feature is surprising because this mode is a single, not degenerate mode, and thus cannot split due to a symmetry breaking. Also, the rest of the modes are relatively far from that frequency (see Table \ref{t:summary}). In order to better understand this feature, we measured the Raman spectrum as a function of the temperature. 
Figure \ref{Fig3} shows the result obtained at 9~K, 100~K and 300~K.
The double-peak feature persists down to the lowest temperature as seen Fig.\ref{Fig3}(a).
The corresponding width and frequency difference remain both quite constant. The possible origin of the double-peak feature will be discussed further in Sec. \ref{disc}.  
In addition, no new mode or new splitting appears as a function of temperature and in none of the four probed symmetries. This is consistent with the absence of a structural transition associated with magnetism, as proven by M\"{o}ssbauer and NMR spectroscopies and neutron powder diffraction measurement [unpublished]. The E$_g$ modes were shown to be strongly affected by magnetism in BaFe$_2$As$_2$ \cite{Chauviere2009}. This symmetry is nevertheless not accessible in our configuration.



\subsection{Superconducting LaFeSiO} 



The polarized Raman spectra obtained from LaFeSiO are shown in Fig. \ref{Fig4}. First, in the parallel polarized configuration we clearly observe two A$_{1g}$ modes, at 187 cm$^{-1}$ and 260~cm$^{-1}$. The mode measured at 165 cm$^{-1}$ is assigned to the B$_{1g}$ mode of Fe. 
For these measurements, since the orientation of the polarization of light in the (\textit{ab})-plane of the crystal is not well-defined, we do not observe a full extinction of the B$_{1g}$ modes. 
Nevertheless the symmetry can be assigned since the signal is significantly reduced for the Fe B$_{1g}$ mode in c(a$_{0\degree{}}$a$_{0\degree}$)$\bar{\mathrm{c}}$ as compared to c(a$_{45\degree{}}$a$_{45\degree}$)$\bar{\mathrm{c}}$. 
The intensity of the A$_{1g}$ modes are indeed expected to remain constant in parallel polarization when changing the angle in the (\textit{ab})-plane, as is also observed here. 



The B$_{1g}$ mode mainly associated with the O displacements is not immediately identifiable given the present data. This mode is expected at 272~cm$^{-1}$ according to our calculations and should then be observable in the c(b$_{45\degree{}}$a$_{45\degree{}}$)$\bar{\mathrm{c}}$ configuration like the B$_{1g}$-Fe mode. We do observe intensity at 262 cm$^{-1}$. 
However, since we also observe intensity at 182 cm$^{-1}$, this can also be attributed to "leakage" of the A$_{1g}$ modes of La (marked with a losange in Fig.~\ref{Fig4}). 


In addition, a broad peak, marked with a circle, is measured at 672~cm$^{-1}$. It seems to be active in the A$_{1g}$ channel. 
When considering leakage of the E$_g$ modes, the closest E$_g$ mode is calculated to be at 433 cm$^{-1}$, excluding this possibility. This high-energy feature may thus be related to a two-phonon process.


% Figure environment removed 


\section{Discussion}
\label{disc}




The measured Raman spectra of the three silicides under consideration are summarized in Fig. \ref{Fig5} (see also Table \ref{t:summary}). 
We first discuss the modes associated with the FeSi layer. 

% Figure environment removed

We note that the frequency of the $B_{1g}$-Fe mode (in blue Fig.\ref{Fig5}) in LaFeSi and LaFeSiH is comparable to that of the reference compounds LaFeAsO (201~cm$^{-1}$) and SmFeAsO (208~cm$^{-1}$) reported in \cite{Hadjiev2008}. This mode, however, becomes considerably softer in LaFeSiO. The latter is likely connected to the reduced distance between the Si and the Fe plane that materializes in this system.  
The $A_{1g}$-Si mode (in red Fig.\ref{Fig5}), however, is considerably harder compared to the $A_{1g}$-As one. In fact, the $B_{1g}$-Fe and $A_{1g}$-As modes are quasi degenerate in the above arsenides, while the frequencies of the $B_{1g}$-Fe- and the $A_{1g}$-Si modes are quite different in the silicides with the latter being harder. The non-degeneracy of these modes suggests that the coupling of the FeSi layer to the spacer is more important in the silicides. 
The above trends are confirmed in the calculations where, in addition, we observe significant changes depending on the atomic positions.

Next, we discuss the modes associated with the spacer. The $A_{1g}$ mode mainly associated with relative displacements of the La atom (in green Fig.\ref{Fig5}) becomes progressively harder from LaFeSi, to LaFeSiH then to LaFeSiO. 
This can be related to the insertion of the light element within the spacer and its electronegativity. The observed trend then suggests that the ionic character of the spacer increases with increasing electronegativity of the inserted element, thereby hardening the $A_{1g}$-La mode.  
At the same time, we also note that the frequency of this mode in LaFeSiO is higher compared to that in LaFeAsO. 
This can be associated with the subdominant contribution of the Si displacements, which again suggests an enhanced coupling between the FeSi layer and the spacer. 

In general, the $A_{1g}$ Raman peaks in the silicides display a Fano shape that is more apparent compared to their arsenide counterparts, as reported in Table~\ref{table_3}. This can be due to an enhanced electron-phonon interaction. In addition, we observe a significant broadening of the $A_{1g}$(La) mode from 5 cm$^{-1}$ in the LaFeSi precursor to $\sim$ 10-11 cm$^{-1}$ in LaFeSiH and LaFeSiO. 

\begin{table}[!h]
\begin{tabular}{l|cc}
Compound & A$_{1g}$(La) & A$_{1g}$(Si) \\ \hline
LaFeSi   & -5.6(3)   & -9.5(9)   \\
LaFeSiH  & -17(2)  & -17(1)     \\
LaFeSiO  & -6.1(5)   & -22(2)    
\end{tabular}
\caption{Fano parameter q on the two A$_{1g}$ modes of La and Si. The fits are shown in figures S1-S3, along with additional fits. 
For LaFeSi and LaFeSiH the fits are done selecting A$_{1g}$ whereas for LaFeSiO the B$_{1g}$ contamination of the signal made the fits better when selecting for A$_{1g}$ + B$_{1g}$.}
\label{table_3}
\end{table}

The most puzzling feature of the measured Raman spectra, however, is the double-peak feature observed at $224$~cm$^{-1}$ in LaFeSiH. According to the calculations, this feature should correspond a single B$_{1g}$ mode associated mainly with Fe displacements (but also involving H). We note that such a double-peak feature is robust, in the sense that we systematically observe it in all the measured samples at the same frequency with similar relative intensities. So even if Raman spectroscopy remains a surface sensitive technique (penetration depth of light of about 40~nm in Fe-based superconductors), it is unlikely due to a surface effect. The doubling clearly appears only after hydrogenation since this B$_{1g}$ mode in precursor LaFeSi is narrow, with a FWHM of 4.6 cm$^{-1}$.  
On this basis, it is thus difficult to envision that a second B$_{1g}$ peak could originate from parts of the sample that are not hydrogenated. For similar reasons, the possible contamination of the surface by oxygen atoms inserted in LaFeSi instead of hydrogen seems unlikely as the energy of the B$_{1g}$(Fe) mode in LaFeSiO is significantly lower.  

Since the B$_{1g}$ mode is non-degenerate, a symmetry breaking cannot explain a doubling. However, if there are two slightly different environments for the Fe atoms, the H atoms, or both, this may result in slightly different B$_{1g}$ frequencies. 
From X-ray single-crystal diffraction measurements, no splitting of the Fe atom positions is observed, nor an unusually high atomic displacement parameter (ADP) compared to those of the other elements \cite{Bernardini2018}. Consequently, the second B$_{1g}$ Raman peak might be related with abnormalities on the H or Si site rather than to the Fe site. \\



In order to further refine the crystal structure analysis of LaFeSiH, we used single crystal electron diffraction (PEDT). We generally confirm the LaFeSiH structure obtained by single crystal X-ray diffraction. Particularly, no peculiar abnormality is observed for the atomic position of Fe. As for H sites, the refinement is consistent with a full hydrogen occupancy within 25\% accuracy. Nevertheless the Fourier difference map exhibits residual density close to the Si sites (see Fig. \ref{FigPEDT}.a). This may either indicate that a small portion of Si is moved towards the Fe layer or that the space group P4/nmm needs to be revisited. This first hypothesis would not explain our reproducible double B$_{1g}$ mode.      
 

% Figure environment removed 

 
The residual density close to Si sites point towards a possibility that the n-glide plane may not be so robust in LaFeSiH (see Supplementary Material and Fig. \ref{FigPEDT} b)).
This possibility suggested by PEDT analysis is further confirmed by X-ray diffraction on single crystals. In all large crystals measured prior to Raman spectroscopy investigation, the presence of extra spots forbidden by the space group $\rm P4/nmm$ indicates a possible violation of the n-glide plane (see Supplementary Materials Fig 5). However, the intensity of these reflections are very weak (about 1/200 of the main peaks intensity and not visible on small single crystals \cite{Bernardini2018}), thus we could not take them into account for the refinement and the average structure solution considering the symmetry $\rm P4/nmm$ still provides very good reliability factors. Interestingly, the apparent violation of the n-glide plane already exists in the pristine silicide, as these additional super-reflections are also encountered in LaFeSi. This suggests that this deviation from the ideal structure does not result from hydrogen insertion.         
Taking these two observations into account, as well as the doubling of the Raman B$_{1g}$ mode, we tentatively revisit the LaFeSiH structure by selecting a subgroup of $\rm P4/nmm$. The subgroups of rank 1 that have lost the n-glide plane are the first to be considered:
$\rm P\overline{4}m2$, $\rm P\overline{4}2_1m$, $\rm P42_12$ and $\rm P4mm$. These structures are particularly of interest for the superconducting state since global inversion center symmetry is broken~\cite{Bauer2012}.\\

From refinements based on PEDT data, only the space group $\rm P4mm$, with 2 Si sites, shows a slightly different structure from the one obtained using the space group $\rm P4/nmm$. In $\rm P4mm$, we expect only two B$_1$ modes (same selection rules as for the B$_{1g}$ modes in P4/nmm) and no other Raman active mode in crossed polarisation. Thus, this space group can not account for our Raman results. 
Two sub-groups of higher rank and without the n-glide mirror symmetry are consistent with PEDT data:  P2$_1$2$_1$2 and P2$_1$/m. For P2$_1$2$_1$2 no active modes are expected in the configurations we used, hence this space group can be excluded. In P2$_1$/m, 12 Raman modes equivalent to A$_{1g}$+B$_{1g}$ (A$_g$) are expected with equivalent selection rules, so again, excluding this space group, based on the Raman results. Then, further measurements are needed to conclude about the subtle space group of LaFeSiH. 



In the case of LaFeSiO, the Raman spectra do not present such clear double-peak feature. However, the B$_{1g}$ peak at 164~cm$^{-1}$ has a relatively large FWHM of 10~cm$^{-1}$ as shown in the inset of Figure \ref{Fig5}(c), preventing us from a definitive conclusion on the absence of doubling of the B$_{1g}$ modes.




\section{Conclusion}

In summary, the lattice dynamics of Fe-based silicides superconductors LaFeSiH and LaFeSiO and their precursor LaFeSi were studied using Raman spectroscopy and \textit{ab-initio} calculations.   
Comparison with arsenides compounds point to stronger coupling of the spacer to the FeSi layers in silicides ones. The effect of the insertion of light element in the precursor from H to O suggests that the ionic character of the spacer is increased in turn. Strong electron-phonon interactions are highlighted by Fano-shape and width of La and Si phonon modes. Finally an intriguing doubling of a non-degenerate B$_{1g}$ mode is reported in LaFesiH, which requires a subtle change of space group for this compound as compared to $P4/nmm$. 

%\bibliography{Biblio-generale}
%\begin{thebibliography}{42}
%merlin.mbs apsrev4-1.bst 2010-07-25 4.21a (PWD, AO, DPC) hacked
%Control: key (0)
%Control: author (8) initials jnrlst
%Control: editor formatted (1) identically to author
%Control: production of article title (-1) disabled
%Control: page (0) single
%Control: year (1) truncated
%Control: production of eprint (0) enabled
\begin{thebibliography}{28}%
\makeatletter
\providecommand \@ifxundefined [1]{%
 \@ifx{#1\undefined}
}%
\providecommand \@ifnum [1]{%
 \ifnum #1\expandafter \@firstoftwo
 \else \expandafter \@secondoftwo
 \fi
}%
\providecommand \@ifx [1]{%
 \ifx #1\expandafter \@firstoftwo
 \else \expandafter \@secondoftwo
 \fi
}%
\providecommand \natexlab [1]{#1}%
\providecommand \enquote  [1]{``#1''}%
\providecommand \bibnamefont  [1]{#1}%
\providecommand \bibfnamefont [1]{#1}%
\providecommand \citenamefont [1]{#1}%
\providecommand \href@noop [0]{\@secondoftwo}%
\providecommand \href [0]{\begingroup \@sanitize@url \@href}%
\providecommand \@href[1]{\@@startlink{#1}\@@href}%
\providecommand \@@href[1]{\endgroup#1\@@endlink}%
\providecommand \@sanitize@url [0]{\catcode `\\12\catcode `\$12\catcode
  `\&12\catcode `\#12\catcode `\^12\catcode `\_12\catcode `\%12\relax}%
\providecommand \@@startlink[1]{}%
\providecommand \@@endlink[0]{}%
\providecommand \url  [0]{\begingroup\@sanitize@url \@url }%
\providecommand \@url [1]{\endgroup\@href {#1}{\urlprefix }}%
\providecommand \urlprefix  [0]{URL }%
\providecommand \Eprint [0]{\href }%
\providecommand \doibase [0]{http://dx.doi.org/}%
\providecommand \selectlanguage [0]{\@gobble}%
\providecommand \bibinfo  [0]{\@secondoftwo}%
\providecommand \bibfield  [0]{\@secondoftwo}%
\providecommand \translation [1]{[#1]}%
\providecommand \BibitemOpen [0]{}%
\providecommand \bibitemStop [0]{}%
\providecommand \bibitemNoStop [0]{.\EOS\space}%
\providecommand \EOS [0]{\spacefactor3000\relax}%
\providecommand \BibitemShut  [1]{\csname bibitem#1\endcsname}%
\let\auto@bib@innerbib\@empty
%</preamble>
\bibitem [{\citenamefont {Hosono}\ and\ \citenamefont
  {Kuroki}(2015)}]{Hosono2015}%
  \BibitemOpen
  \bibfield  {author} {\bibinfo {author} {\bibfnamefont {H.}~\bibnamefont
  {Hosono}}\ and\ \bibinfo {author} {\bibfnamefont {K.}~\bibnamefont
  {Kuroki}},\ }\href {\doibase 10.1016/j.physc.2015.02.020} {\bibfield
  {journal} {\bibinfo  {journal} {Physica C: Superconductivity and its
  Applications}\ }\textbf {\bibinfo {volume} {514}},\ \bibinfo {pages} {399}
  (\bibinfo {year} {2015})}\BibitemShut {NoStop}%
\bibitem [{\citenamefont {Alloul}\ and\ \citenamefont
  {Cano}(2016)}]{Alloul2016}%
  \BibitemOpen
  \bibfield  {author} {\bibinfo {author} {\bibfnamefont {H.}~\bibnamefont
  {Alloul}}\ and\ \bibinfo {author} {\bibfnamefont {A.}~\bibnamefont {Cano}},\
  }\href {\doibase 10.1016/j.crhy.2015.11.004} {\bibfield  {journal} {\bibinfo
  {journal} {Comptes Rendus Physique}\ }\textbf {\bibinfo {volume} {17}},\
  \bibinfo {pages} {1} (\bibinfo {year} {2016})}\BibitemShut {NoStop}%
\bibitem [{\citenamefont {Zhang}\ \emph {et~al.}(2018)\citenamefont {Zhang},
  \citenamefont {Yaji}, \citenamefont {Hashimoto}, \citenamefont {Ota},
  \citenamefont {Kondo}, \citenamefont {Okazaki}, \citenamefont {Wang},
  \citenamefont {Wen}, \citenamefont {Gu}, \citenamefont {Ding},\ and\
  \citenamefont {Shin}}]{Zhang2018}%
  \BibitemOpen
  \bibfield  {author} {\bibinfo {author} {\bibfnamefont {P.}~\bibnamefont
  {Zhang}}, \bibinfo {author} {\bibfnamefont {K.}~\bibnamefont {Yaji}},
  \bibinfo {author} {\bibfnamefont {T.}~\bibnamefont {Hashimoto}}, \bibinfo
  {author} {\bibfnamefont {Y.}~\bibnamefont {Ota}}, \bibinfo {author}
  {\bibfnamefont {T.}~\bibnamefont {Kondo}}, \bibinfo {author} {\bibfnamefont
  {K.}~\bibnamefont {Okazaki}}, \bibinfo {author} {\bibfnamefont
  {Z.}~\bibnamefont {Wang}}, \bibinfo {author} {\bibfnamefont {J.}~\bibnamefont
  {Wen}}, \bibinfo {author} {\bibfnamefont {G.~D.}\ \bibnamefont {Gu}},
  \bibinfo {author} {\bibfnamefont {H.}~\bibnamefont {Ding}}, \ and\ \bibinfo
  {author} {\bibfnamefont {S.}~\bibnamefont {Shin}},\ }\href {\doibase
  10.1126/science.aan4596} {\bibfield  {journal} {\bibinfo  {journal}
  {Science}\ }\textbf {\bibinfo {volume} {360}},\ \bibinfo {pages} {182}
  (\bibinfo {year} {2018})}\BibitemShut {NoStop}%
\bibitem [{\citenamefont {Zhang}\ \emph {et~al.}(2019)\citenamefont {Zhang},
  \citenamefont {Wang}, \citenamefont {Wu}, \citenamefont {Yaji}, \citenamefont
  {Ishida}, \citenamefont {Kohama}, \citenamefont {Dai}, \citenamefont {Sun},
  \citenamefont {Bareille}, \citenamefont {Kuroda}, \citenamefont {Kondo},
  \citenamefont {Okazaki}, \citenamefont {Kindo}, \citenamefont {Wang},
  \citenamefont {Jin}, \citenamefont {Hu}, \citenamefont {Thomale},
  \citenamefont {Sumida}, \citenamefont {Wu}, \citenamefont {Miyamoto},
  \citenamefont {Okuda}, \citenamefont {Ding}, \citenamefont {Gu},
  \citenamefont {Tamegai}, \citenamefont {Kawakami}, \citenamefont {Sato},\
  and\ \citenamefont {Shin}}]{zhang2019}%
  \BibitemOpen
  \bibfield  {author} {\bibinfo {author} {\bibfnamefont {P.}~\bibnamefont
  {Zhang}}, \bibinfo {author} {\bibfnamefont {Z.}~\bibnamefont {Wang}},
  \bibinfo {author} {\bibfnamefont {X.}~\bibnamefont {Wu}}, \bibinfo {author}
  {\bibfnamefont {K.}~\bibnamefont {Yaji}}, \bibinfo {author} {\bibfnamefont
  {Y.}~\bibnamefont {Ishida}}, \bibinfo {author} {\bibfnamefont
  {Y.}~\bibnamefont {Kohama}}, \bibinfo {author} {\bibfnamefont
  {G.}~\bibnamefont {Dai}}, \bibinfo {author} {\bibfnamefont {Y.}~\bibnamefont
  {Sun}}, \bibinfo {author} {\bibfnamefont {C.}~\bibnamefont {Bareille}},
  \bibinfo {author} {\bibfnamefont {K.}~\bibnamefont {Kuroda}}, \bibinfo
  {author} {\bibfnamefont {T.}~\bibnamefont {Kondo}}, \bibinfo {author}
  {\bibfnamefont {K.}~\bibnamefont {Okazaki}}, \bibinfo {author} {\bibfnamefont
  {K.}~\bibnamefont {Kindo}}, \bibinfo {author} {\bibfnamefont
  {X.}~\bibnamefont {Wang}}, \bibinfo {author} {\bibfnamefont {C.}~\bibnamefont
  {Jin}}, \bibinfo {author} {\bibfnamefont {J.}~\bibnamefont {Hu}}, \bibinfo
  {author} {\bibfnamefont {R.}~\bibnamefont {Thomale}}, \bibinfo {author}
  {\bibfnamefont {K.}~\bibnamefont {Sumida}}, \bibinfo {author} {\bibfnamefont
  {S.}~\bibnamefont {Wu}}, \bibinfo {author} {\bibfnamefont {K.}~\bibnamefont
  {Miyamoto}}, \bibinfo {author} {\bibfnamefont {T.}~\bibnamefont {Okuda}},
  \bibinfo {author} {\bibfnamefont {H.}~\bibnamefont {Ding}}, \bibinfo {author}
  {\bibfnamefont {G.~D.}\ \bibnamefont {Gu}}, \bibinfo {author} {\bibfnamefont
  {T.}~\bibnamefont {Tamegai}}, \bibinfo {author} {\bibfnamefont
  {T.}~\bibnamefont {Kawakami}}, \bibinfo {author} {\bibfnamefont
  {M.}~\bibnamefont {Sato}}, \ and\ \bibinfo {author} {\bibfnamefont
  {S.}~\bibnamefont {Shin}},\ }\href {\doibase 10.1038/s41567-018-0280-z}
  {\bibfield  {journal} {\bibinfo  {journal} {Nature Physics}\ }\textbf
  {\bibinfo {volume} {15}},\ \bibinfo {pages} {41} (\bibinfo {year}
  {2019})}\BibitemShut {NoStop}%
\bibitem [{\citenamefont {Zou}\ \emph {et~al.}(2014)\citenamefont {Zou},
  \citenamefont {Feng}, \citenamefont {Logg}, \citenamefont {Chen},
  \citenamefont {Lampronti},\ and\ \citenamefont {Grosche}}]{Zou2014}%
  \BibitemOpen
  \bibfield  {author} {\bibinfo {author} {\bibfnamefont {Y.}~\bibnamefont
  {Zou}}, \bibinfo {author} {\bibfnamefont {Z.}~\bibnamefont {Feng}}, \bibinfo
  {author} {\bibfnamefont {P.~W.}\ \bibnamefont {Logg}}, \bibinfo {author}
  {\bibfnamefont {J.}~\bibnamefont {Chen}}, \bibinfo {author} {\bibfnamefont
  {G.}~\bibnamefont {Lampronti}}, \ and\ \bibinfo {author} {\bibfnamefont
  {F.~M.}\ \bibnamefont {Grosche}},\ }\href {\doibase 10.1002/pssr.201409418}
  {\bibfield  {journal} {\bibinfo  {journal} {physica status solidi (RRL) -
  Rapid Research Letters}\ }\textbf {\bibinfo {volume} {8}},\ \bibinfo {pages}
  {928} (\bibinfo {year} {2014})}\BibitemShut {NoStop}%
\bibitem [{\citenamefont {Kim}\ \emph {et~al.}(2015)\citenamefont {Kim},
  \citenamefont {Ran}, \citenamefont {Mun}, \citenamefont {Hodovanets},
  \citenamefont {Tanatar}, \citenamefont {Prozorov}, \citenamefont {Bud'ko},\
  and\ \citenamefont {Canfield}}]{Kim2015}%
  \BibitemOpen
  \bibfield  {author} {\bibinfo {author} {\bibfnamefont {H.}~\bibnamefont
  {Kim}}, \bibinfo {author} {\bibfnamefont {S.}~\bibnamefont {Ran}}, \bibinfo
  {author} {\bibfnamefont {E.}~\bibnamefont {Mun}}, \bibinfo {author}
  {\bibfnamefont {H.}~\bibnamefont {Hodovanets}}, \bibinfo {author}
  {\bibfnamefont {M.}~\bibnamefont {Tanatar}}, \bibinfo {author} {\bibfnamefont
  {R.}~\bibnamefont {Prozorov}}, \bibinfo {author} {\bibfnamefont
  {S.}~\bibnamefont {Bud'ko}}, \ and\ \bibinfo {author} {\bibfnamefont
  {P.}~\bibnamefont {Canfield}},\ }\href {\doibase
  10.1080/14786435.2015.1004378} {\bibfield  {journal} {\bibinfo  {journal}
  {Philosophical Magazine}\ }\textbf {\bibinfo {volume} {95}},\ \bibinfo
  {pages} {804} (\bibinfo {year} {2015})}\BibitemShut {NoStop}%
\bibitem [{\citenamefont {Chen}\ \emph {et~al.}(2016)\citenamefont {Chen},
  \citenamefont {Semeniuk}, \citenamefont {Feng}, \citenamefont {Reiss},
  \citenamefont {Brown}, \citenamefont {Zou}, \citenamefont {Logg},
  \citenamefont {Lampronti},\ and\ \citenamefont {Grosche}}]{Chen2016}%
  \BibitemOpen
  \bibfield  {author} {\bibinfo {author} {\bibfnamefont {J.}~\bibnamefont
  {Chen}}, \bibinfo {author} {\bibfnamefont {K.}~\bibnamefont {Semeniuk}},
  \bibinfo {author} {\bibfnamefont {Z.}~\bibnamefont {Feng}}, \bibinfo {author}
  {\bibfnamefont {P.}~\bibnamefont {Reiss}}, \bibinfo {author} {\bibfnamefont
  {P.}~\bibnamefont {Brown}}, \bibinfo {author} {\bibfnamefont
  {Y.}~\bibnamefont {Zou}}, \bibinfo {author} {\bibfnamefont {P.~W.}\
  \bibnamefont {Logg}}, \bibinfo {author} {\bibfnamefont {G.~I.}\ \bibnamefont
  {Lampronti}}, \ and\ \bibinfo {author} {\bibfnamefont {F.~M.}\ \bibnamefont
  {Grosche}},\ }\href {\doibase 10.1103/PhysRevLett.116.127001} {\bibfield
  {journal} {\bibinfo  {journal} {Physical Review Letters}\ }\textbf {\bibinfo
  {volume} {116}},\ \bibinfo {pages} {127001} (\bibinfo {year}
  {2016})}\BibitemShut {NoStop}%
\bibitem [{\citenamefont {Bernardini}\ \emph {et~al.}(2018)\citenamefont
  {Bernardini}, \citenamefont {Garbarino}, \citenamefont {Sulpice},
  \citenamefont {{N{\'u}{\~n}ez-Regueiro}}, \citenamefont {Gaudin},
  \citenamefont {Chevalier}, \citenamefont {M{\'e}asson}, \citenamefont
  {Cano},\ and\ \citenamefont {Tenc{\'e}}}]{Bernardini2018}%
  \BibitemOpen
  \bibfield  {author} {\bibinfo {author} {\bibfnamefont {F.}~\bibnamefont
  {Bernardini}}, \bibinfo {author} {\bibfnamefont {G.}~\bibnamefont
  {Garbarino}}, \bibinfo {author} {\bibfnamefont {A.}~\bibnamefont {Sulpice}},
  \bibinfo {author} {\bibfnamefont {M.}~\bibnamefont
  {{N{\'u}{\~n}ez-Regueiro}}}, \bibinfo {author} {\bibfnamefont
  {E.}~\bibnamefont {Gaudin}}, \bibinfo {author} {\bibfnamefont
  {B.}~\bibnamefont {Chevalier}}, \bibinfo {author} {\bibfnamefont {M.-A.}\
  \bibnamefont {M{\'e}asson}}, \bibinfo {author} {\bibfnamefont
  {A.}~\bibnamefont {Cano}}, \ and\ \bibinfo {author} {\bibfnamefont
  {S.}~\bibnamefont {Tenc{\'e}}},\ }\href {\doibase 10.1103/PhysRevB.97.100504}
  {\bibfield  {journal} {\bibinfo  {journal} {Physical Review B}\ }\textbf
  {\bibinfo {volume} {97}} (\bibinfo {year} {2018}),\
  10.1103/PhysRevB.97.100504}\BibitemShut {NoStop}%
\bibitem [{\citenamefont {Bhattacharyya}\ \emph {et~al.}(2020)\citenamefont
  {Bhattacharyya}, \citenamefont {Rodi{\`e}re}, \citenamefont {Vaney},
  \citenamefont {Biswas}, \citenamefont {Hillier}, \citenamefont {Bosin},
  \citenamefont {Bernardini}, \citenamefont {Tenc{\'e}}, \citenamefont
  {Adroja},\ and\ \citenamefont {Cano}}]{Bhattacharyya2020}%
  \BibitemOpen
  \bibfield  {author} {\bibinfo {author} {\bibfnamefont {A.}~\bibnamefont
  {Bhattacharyya}}, \bibinfo {author} {\bibfnamefont {P.}~\bibnamefont
  {Rodi{\`e}re}}, \bibinfo {author} {\bibfnamefont {J.-B.}\ \bibnamefont
  {Vaney}}, \bibinfo {author} {\bibfnamefont {P.~K.}\ \bibnamefont {Biswas}},
  \bibinfo {author} {\bibfnamefont {A.~D.}\ \bibnamefont {Hillier}}, \bibinfo
  {author} {\bibfnamefont {A.}~\bibnamefont {Bosin}}, \bibinfo {author}
  {\bibfnamefont {F.}~\bibnamefont {Bernardini}}, \bibinfo {author}
  {\bibfnamefont {S.}~\bibnamefont {Tenc{\'e}}}, \bibinfo {author}
  {\bibfnamefont {D.~T.}\ \bibnamefont {Adroja}}, \ and\ \bibinfo {author}
  {\bibfnamefont {A.}~\bibnamefont {Cano}},\ }\href {\doibase
  10.1103/PhysRevB.101.224502} {\bibfield  {journal} {\bibinfo  {journal}
  {Physical Review B}\ }\textbf {\bibinfo {volume} {101}},\ \bibinfo {pages}
  {224502} (\bibinfo {year} {2020})}\BibitemShut {NoStop}%
\bibitem [{\citenamefont {Hansen}\ \emph {et~al.}(2023)\citenamefont {Hansen},
  \citenamefont {Vaney}, \citenamefont {De~Rango}, \citenamefont {Sala{\"u}n},
  \citenamefont {Tenc{\'e}}, \citenamefont {Nassif},\ and\ \citenamefont
  {Toulemonde}}]{Hansen2023}%
  \BibitemOpen
  \bibfield  {author} {\bibinfo {author} {\bibfnamefont {M.}~\bibnamefont
  {Hansen}}, \bibinfo {author} {\bibfnamefont {J.-B.}\ \bibnamefont {Vaney}},
  \bibinfo {author} {\bibfnamefont {P.}~\bibnamefont {De~Rango}}, \bibinfo
  {author} {\bibfnamefont {M.}~\bibnamefont {Sala{\"u}n}}, \bibinfo {author}
  {\bibfnamefont {S.}~\bibnamefont {Tenc{\'e}}}, \bibinfo {author}
  {\bibfnamefont {V.}~\bibnamefont {Nassif}}, \ and\ \bibinfo {author}
  {\bibfnamefont {P.}~\bibnamefont {Toulemonde}},\ }\href {\doibase
  10.1016/j.jallcom.2023.169281} {\bibfield  {journal} {\bibinfo  {journal}
  {Journal of Alloys and Compounds}\ }\textbf {\bibinfo {volume} {945}},\
  \bibinfo {pages} {169281} (\bibinfo {year} {2023})}\BibitemShut {NoStop}%
\bibitem [{\citenamefont {Vaney}\ \emph {et~al.}(2022)\citenamefont {Vaney},
  \citenamefont {Vignolle}, \citenamefont {Demourgues}, \citenamefont {Gaudin},
  \citenamefont {Durand}, \citenamefont {Labrug{\`e}re}, \citenamefont
  {Bernardini}, \citenamefont {Cano},\ and\ \citenamefont
  {Tenc{\'e}}}]{Vaney2022}%
  \BibitemOpen
  \bibfield  {author} {\bibinfo {author} {\bibfnamefont {J.-B.}\ \bibnamefont
  {Vaney}}, \bibinfo {author} {\bibfnamefont {B.}~\bibnamefont {Vignolle}},
  \bibinfo {author} {\bibfnamefont {A.}~\bibnamefont {Demourgues}}, \bibinfo
  {author} {\bibfnamefont {E.}~\bibnamefont {Gaudin}}, \bibinfo {author}
  {\bibfnamefont {E.}~\bibnamefont {Durand}}, \bibinfo {author} {\bibfnamefont
  {C.}~\bibnamefont {Labrug{\`e}re}}, \bibinfo {author} {\bibfnamefont
  {F.}~\bibnamefont {Bernardini}}, \bibinfo {author} {\bibfnamefont
  {A.}~\bibnamefont {Cano}}, \ and\ \bibinfo {author} {\bibfnamefont
  {S.}~\bibnamefont {Tenc{\'e}}},\ }\href {\doibase 10.1038/s41467-022-29043-8}
  {\bibfield  {journal} {\bibinfo  {journal} {Nature Communications}\ }\textbf
  {\bibinfo {volume} {13}},\ \bibinfo {pages} {1462} (\bibinfo {year}
  {2022})}\BibitemShut {NoStop}%
\bibitem [{\citenamefont {Hansen}\ \emph {et~al.}(2022)\citenamefont {Hansen},
  \citenamefont {Vaney}, \citenamefont {Lepoittevin}, \citenamefont
  {Bernardini}, \citenamefont {Gaudin}, \citenamefont {Nassif}, \citenamefont
  {M{\'e}asson}, \citenamefont {Sulpice}, \citenamefont {Mayaffre},
  \citenamefont {Julien}, \citenamefont {Tenc{\'e}}, \citenamefont {Cano},\
  and\ \citenamefont {Toulemonde}}]{Hansen2022}%
  \BibitemOpen
  \bibfield  {author} {\bibinfo {author} {\bibfnamefont {M.~F.}\ \bibnamefont
  {Hansen}}, \bibinfo {author} {\bibfnamefont {J.-B.}\ \bibnamefont {Vaney}},
  \bibinfo {author} {\bibfnamefont {C.}~\bibnamefont {Lepoittevin}}, \bibinfo
  {author} {\bibfnamefont {F.}~\bibnamefont {Bernardini}}, \bibinfo {author}
  {\bibfnamefont {E.}~\bibnamefont {Gaudin}}, \bibinfo {author} {\bibfnamefont
  {V.}~\bibnamefont {Nassif}}, \bibinfo {author} {\bibfnamefont {M.-A.}\
  \bibnamefont {M{\'e}asson}}, \bibinfo {author} {\bibfnamefont
  {A.}~\bibnamefont {Sulpice}}, \bibinfo {author} {\bibfnamefont
  {H.}~\bibnamefont {Mayaffre}}, \bibinfo {author} {\bibfnamefont {M.-H.}\
  \bibnamefont {Julien}}, \bibinfo {author} {\bibfnamefont {S.}~\bibnamefont
  {Tenc{\'e}}}, \bibinfo {author} {\bibfnamefont {A.}~\bibnamefont {Cano}}, \
  and\ \bibinfo {author} {\bibfnamefont {P.}~\bibnamefont {Toulemonde}},\
  }\href {\doibase 10.1038/s41535-022-00493-z} {\bibfield  {journal} {\bibinfo
  {journal} {npj Quantum Materials}\ }\textbf {\bibinfo {volume} {7}},\
  \bibinfo {pages} {86} (\bibinfo {year} {2022})}\BibitemShut {NoStop}%
\bibitem [{\citenamefont {Okabe}\ \emph {et~al.}(2010)\citenamefont {Okabe},
  \citenamefont {Takeshita}, \citenamefont {Horigane}, \citenamefont
  {Muranaka},\ and\ \citenamefont {Akimitsu}}]{Okabe2010}%
  \BibitemOpen
  \bibfield  {author} {\bibinfo {author} {\bibfnamefont {H.}~\bibnamefont
  {Okabe}}, \bibinfo {author} {\bibfnamefont {N.}~\bibnamefont {Takeshita}},
  \bibinfo {author} {\bibfnamefont {K.}~\bibnamefont {Horigane}}, \bibinfo
  {author} {\bibfnamefont {T.}~\bibnamefont {Muranaka}}, \ and\ \bibinfo
  {author} {\bibfnamefont {J.}~\bibnamefont {Akimitsu}},\ }\href {\doibase
  10.1103/PhysRevB.81.205119} {\bibfield  {journal} {\bibinfo  {journal}
  {Physical Review B}\ }\textbf {\bibinfo {volume} {81}},\ \bibinfo {pages}
  {205119} (\bibinfo {year} {2010})}\BibitemShut {NoStop}%
\bibitem [{\citenamefont {Mizuguchi}\ \emph {et~al.}(2010)\citenamefont
  {Mizuguchi}, \citenamefont {Hara}, \citenamefont {Deguchi}, \citenamefont
  {Tsuda}, \citenamefont {Yamaguchi}, \citenamefont {Takeda}, \citenamefont
  {Kotegawa}, \citenamefont {Tou},\ and\ \citenamefont
  {Takano}}]{Mizuguchi2010}%
  \BibitemOpen
  \bibfield  {author} {\bibinfo {author} {\bibfnamefont {Y.}~\bibnamefont
  {Mizuguchi}}, \bibinfo {author} {\bibfnamefont {Y.}~\bibnamefont {Hara}},
  \bibinfo {author} {\bibfnamefont {K.}~\bibnamefont {Deguchi}}, \bibinfo
  {author} {\bibfnamefont {S.}~\bibnamefont {Tsuda}}, \bibinfo {author}
  {\bibfnamefont {T.}~\bibnamefont {Yamaguchi}}, \bibinfo {author}
  {\bibfnamefont {K.}~\bibnamefont {Takeda}}, \bibinfo {author} {\bibfnamefont
  {H.}~\bibnamefont {Kotegawa}}, \bibinfo {author} {\bibfnamefont
  {H.}~\bibnamefont {Tou}}, \ and\ \bibinfo {author} {\bibfnamefont
  {Y.}~\bibnamefont {Takano}},\ }\href {\doibase 10.1088/0953-2048/23/5/054013}
  {\bibfield  {journal} {\bibinfo  {journal} {Superconductor Science and
  Technology}\ }\textbf {\bibinfo {volume} {23}},\ \bibinfo {pages} {054013}
  (\bibinfo {year} {2010})}\BibitemShut {NoStop}%
\bibitem [{\citenamefont {Sun}\ \emph {et~al.}(2018)\citenamefont {Sun},
  \citenamefont {Quan}, \citenamefont {Jin}, \citenamefont {Huang},
  \citenamefont {Wu}, \citenamefont {Zhao}, \citenamefont {Gu}, \citenamefont
  {Yin},\ and\ \citenamefont {Chen}}]{Sun2018}%
  \BibitemOpen
  \bibfield  {author} {\bibinfo {author} {\bibfnamefont {R.~J.}\ \bibnamefont
  {Sun}}, \bibinfo {author} {\bibfnamefont {Y.}~\bibnamefont {Quan}}, \bibinfo
  {author} {\bibfnamefont {S.~F.}\ \bibnamefont {Jin}}, \bibinfo {author}
  {\bibfnamefont {Q.~Z.}\ \bibnamefont {Huang}}, \bibinfo {author}
  {\bibfnamefont {H.}~\bibnamefont {Wu}}, \bibinfo {author} {\bibfnamefont
  {L.}~\bibnamefont {Zhao}}, \bibinfo {author} {\bibfnamefont {L.}~\bibnamefont
  {Gu}}, \bibinfo {author} {\bibfnamefont {Z.~P.}\ \bibnamefont {Yin}}, \ and\
  \bibinfo {author} {\bibfnamefont {X.~L.}\ \bibnamefont {Chen}},\ }\href
  {\doibase 10.1103/PhysRevB.98.214508} {\bibfield  {journal} {\bibinfo
  {journal} {Physical Review B}\ }\textbf {\bibinfo {volume} {98}},\ \bibinfo
  {pages} {214508} (\bibinfo {year} {2018})}\BibitemShut {NoStop}%
\bibitem [{\citenamefont {Lee}\ \emph {et~al.}(2012)\citenamefont {Lee},
  \citenamefont {Kihou}, \citenamefont {Iyo}, \citenamefont {Kito},
  \citenamefont {Shirage},\ and\ \citenamefont {Eisaki}}]{Lee2012}%
  \BibitemOpen
  \bibfield  {author} {\bibinfo {author} {\bibfnamefont {C.}~\bibnamefont
  {Lee}}, \bibinfo {author} {\bibfnamefont {K.}~\bibnamefont {Kihou}}, \bibinfo
  {author} {\bibfnamefont {A.}~\bibnamefont {Iyo}}, \bibinfo {author}
  {\bibfnamefont {H.}~\bibnamefont {Kito}}, \bibinfo {author} {\bibfnamefont
  {P.}~\bibnamefont {Shirage}}, \ and\ \bibinfo {author} {\bibfnamefont
  {H.}~\bibnamefont {Eisaki}},\ }\href {\doibase 10.1016/j.ssc.2011.12.012}
  {\bibfield  {journal} {\bibinfo  {journal} {Solid State Communications}\
  }\textbf {\bibinfo {volume} {152}},\ \bibinfo {pages} {644} (\bibinfo {year}
  {2012})}\BibitemShut {NoStop}%
\bibitem [{\citenamefont {Khim}\ \emph {et~al.}(2021)\citenamefont {Khim},
  \citenamefont {Landaeta}, \citenamefont {Banda}, \citenamefont {Bannor},
  \citenamefont {Brando}, \citenamefont {Brydon}, \citenamefont {Hafner},
  \citenamefont {K{\"u}chler}, \citenamefont {{Cardoso-Gil}}, \citenamefont
  {Stockert}, \citenamefont {Mackenzie}, \citenamefont {Agterberg},
  \citenamefont {Geibel},\ and\ \citenamefont {Hassinger}}]{Khim2021}%
  \BibitemOpen
  \bibfield  {author} {\bibinfo {author} {\bibfnamefont {S.}~\bibnamefont
  {Khim}}, \bibinfo {author} {\bibfnamefont {J.~F.}\ \bibnamefont {Landaeta}},
  \bibinfo {author} {\bibfnamefont {J.}~\bibnamefont {Banda}}, \bibinfo
  {author} {\bibfnamefont {N.}~\bibnamefont {Bannor}}, \bibinfo {author}
  {\bibfnamefont {M.}~\bibnamefont {Brando}}, \bibinfo {author} {\bibfnamefont
  {P.~M.~R.}\ \bibnamefont {Brydon}}, \bibinfo {author} {\bibfnamefont
  {D.}~\bibnamefont {Hafner}}, \bibinfo {author} {\bibfnamefont
  {R.}~\bibnamefont {K{\"u}chler}}, \bibinfo {author} {\bibfnamefont
  {R.}~\bibnamefont {{Cardoso-Gil}}}, \bibinfo {author} {\bibfnamefont
  {U.}~\bibnamefont {Stockert}}, \bibinfo {author} {\bibfnamefont {A.~P.}\
  \bibnamefont {Mackenzie}}, \bibinfo {author} {\bibfnamefont {D.~F.}\
  \bibnamefont {Agterberg}}, \bibinfo {author} {\bibfnamefont {C.}~\bibnamefont
  {Geibel}}, \ and\ \bibinfo {author} {\bibfnamefont {E.}~\bibnamefont
  {Hassinger}},\ }\href {\doibase 10.1126/science.abe7518} {\bibfield
  {journal} {\bibinfo  {journal} {Science}\ }\textbf {\bibinfo {volume}
  {373}},\ \bibinfo {pages} {1012} (\bibinfo {year} {2021})}\BibitemShut
  {NoStop}%
\bibitem [{\citenamefont {Palatinus}\ \emph {et~al.}(2019)\citenamefont
  {Palatinus}, \citenamefont {Br{\'a}zda}, \citenamefont {Jel{\'i}nek},
  \citenamefont {Hrd{\'a}}, \citenamefont {Steciuk},\ and\ \citenamefont
  {Klementov{\'a}}}]{Palatinus2019}%
  \BibitemOpen
  \bibfield  {author} {\bibinfo {author} {\bibfnamefont {L.}~\bibnamefont
  {Palatinus}}, \bibinfo {author} {\bibfnamefont {P.}~\bibnamefont
  {Br{\'a}zda}}, \bibinfo {author} {\bibfnamefont {M.}~\bibnamefont
  {Jel{\'i}nek}}, \bibinfo {author} {\bibfnamefont {J.}~\bibnamefont
  {Hrd{\'a}}}, \bibinfo {author} {\bibfnamefont {G.}~\bibnamefont {Steciuk}}, \
  and\ \bibinfo {author} {\bibfnamefont {M.}~\bibnamefont {Klementov{\'a}}},\
  }\href {\doibase 10.1107/S2052520619007534} {\bibfield  {journal} {\bibinfo
  {journal} {Acta Crystallographica Section B Structural Science, Crystal
  Engineering and Materials}\ }\textbf {\bibinfo {volume} {75}},\ \bibinfo
  {pages} {512} (\bibinfo {year} {2019})}\BibitemShut {NoStop}%
\bibitem [{\citenamefont {Pet{\v r}{\'i}{\v c}ek}\ \emph
  {et~al.}(2014)\citenamefont {Pet{\v r}{\'i}{\v c}ek}, \citenamefont {Du{\v
  s}ek},\ and\ \citenamefont {Palatinus}}]{Petricek2014}%
  \BibitemOpen
  \bibfield  {author} {\bibinfo {author} {\bibfnamefont {V.}~\bibnamefont
  {Pet{\v r}{\'i}{\v c}ek}}, \bibinfo {author} {\bibfnamefont {M.}~\bibnamefont
  {Du{\v s}ek}}, \ and\ \bibinfo {author} {\bibfnamefont {L.}~\bibnamefont
  {Palatinus}},\ }\href {\doibase 10.1515/zkri-2014-1737} {\bibfield  {journal}
  {\bibinfo  {journal} {Zeitschrift f\"ur Kristallographie - Crystalline
  Materials}\ }\textbf {\bibinfo {volume} {229}},\ \bibinfo {pages} {345}
  (\bibinfo {year} {2014})}\BibitemShut {NoStop}%
\bibitem [{\citenamefont {Giannozzi}\ \emph {et~al.}(2009)\citenamefont
  {Giannozzi}, \citenamefont {Baroni}, \citenamefont {Bonini}, \citenamefont
  {Calandra}, \citenamefont {Car}, \citenamefont {Cavazzoni}, \citenamefont
  {Ceresoli}, \citenamefont {Chiarotti}, \citenamefont {Cococcioni},
  \citenamefont {Dabo}, \citenamefont {Dal~Corso}, \citenamefont {{de
  Gironcoli}}, \citenamefont {Fabris}, \citenamefont {Fratesi}, \citenamefont
  {Gebauer}, \citenamefont {Gerstmann}, \citenamefont {Gougoussis},
  \citenamefont {Kokalj}, \citenamefont {Lazzeri}, \citenamefont
  {{Martin-Samos}}, \citenamefont {Marzari}, \citenamefont {Mauri},
  \citenamefont {Mazzarello}, \citenamefont {Paolini}, \citenamefont
  {Pasquarello}, \citenamefont {Paulatto}, \citenamefont {Sbraccia},
  \citenamefont {Scandolo}, \citenamefont {Sclauzero}, \citenamefont
  {Seitsonen}, \citenamefont {Smogunov}, \citenamefont {Umari},\ and\
  \citenamefont {Wentzcovitch}}]{Giannozzi2009}%
  \BibitemOpen
  \bibfield  {author} {\bibinfo {author} {\bibfnamefont {P.}~\bibnamefont
  {Giannozzi}}, \bibinfo {author} {\bibfnamefont {S.}~\bibnamefont {Baroni}},
  \bibinfo {author} {\bibfnamefont {N.}~\bibnamefont {Bonini}}, \bibinfo
  {author} {\bibfnamefont {M.}~\bibnamefont {Calandra}}, \bibinfo {author}
  {\bibfnamefont {R.}~\bibnamefont {Car}}, \bibinfo {author} {\bibfnamefont
  {C.}~\bibnamefont {Cavazzoni}}, \bibinfo {author} {\bibfnamefont
  {D.}~\bibnamefont {Ceresoli}}, \bibinfo {author} {\bibfnamefont {G.~L.}\
  \bibnamefont {Chiarotti}}, \bibinfo {author} {\bibfnamefont {M.}~\bibnamefont
  {Cococcioni}}, \bibinfo {author} {\bibfnamefont {I.}~\bibnamefont {Dabo}},
  \bibinfo {author} {\bibfnamefont {A.}~\bibnamefont {Dal~Corso}}, \bibinfo
  {author} {\bibfnamefont {S.}~\bibnamefont {{de Gironcoli}}}, \bibinfo
  {author} {\bibfnamefont {S.}~\bibnamefont {Fabris}}, \bibinfo {author}
  {\bibfnamefont {G.}~\bibnamefont {Fratesi}}, \bibinfo {author} {\bibfnamefont
  {R.}~\bibnamefont {Gebauer}}, \bibinfo {author} {\bibfnamefont
  {U.}~\bibnamefont {Gerstmann}}, \bibinfo {author} {\bibfnamefont
  {C.}~\bibnamefont {Gougoussis}}, \bibinfo {author} {\bibfnamefont
  {A.}~\bibnamefont {Kokalj}}, \bibinfo {author} {\bibfnamefont
  {M.}~\bibnamefont {Lazzeri}}, \bibinfo {author} {\bibfnamefont
  {L.}~\bibnamefont {{Martin-Samos}}}, \bibinfo {author} {\bibfnamefont
  {N.}~\bibnamefont {Marzari}}, \bibinfo {author} {\bibfnamefont
  {F.}~\bibnamefont {Mauri}}, \bibinfo {author} {\bibfnamefont
  {R.}~\bibnamefont {Mazzarello}}, \bibinfo {author} {\bibfnamefont
  {S.}~\bibnamefont {Paolini}}, \bibinfo {author} {\bibfnamefont
  {A.}~\bibnamefont {Pasquarello}}, \bibinfo {author} {\bibfnamefont
  {L.}~\bibnamefont {Paulatto}}, \bibinfo {author} {\bibfnamefont
  {C.}~\bibnamefont {Sbraccia}}, \bibinfo {author} {\bibfnamefont
  {S.}~\bibnamefont {Scandolo}}, \bibinfo {author} {\bibfnamefont
  {G.}~\bibnamefont {Sclauzero}}, \bibinfo {author} {\bibfnamefont {A.~P.}\
  \bibnamefont {Seitsonen}}, \bibinfo {author} {\bibfnamefont {A.}~\bibnamefont
  {Smogunov}}, \bibinfo {author} {\bibfnamefont {P.}~\bibnamefont {Umari}}, \
  and\ \bibinfo {author} {\bibfnamefont {R.~M.}\ \bibnamefont {Wentzcovitch}},\
  }\href {\doibase 10.1088/0953-8984/21/39/395502} {\bibfield  {journal}
  {\bibinfo  {journal} {Journal of Physics: Condensed Matter}\ }\textbf
  {\bibinfo {volume} {21}},\ \bibinfo {pages} {395502} (\bibinfo {year}
  {2009})}\BibitemShut {NoStop}%
\bibitem [{\citenamefont {{van Setten}}\ \emph {et~al.}(2018)\citenamefont
  {{van Setten}}, \citenamefont {Giantomassi}, \citenamefont {Bousquet},
  \citenamefont {Verstraete}, \citenamefont {Hamann}, \citenamefont {Gonze},\
  and\ \citenamefont {Rignanese}}]{VanSetten2018}%
  \BibitemOpen
  \bibfield  {author} {\bibinfo {author} {\bibfnamefont {M.}~\bibnamefont {{van
  Setten}}}, \bibinfo {author} {\bibfnamefont {M.}~\bibnamefont {Giantomassi}},
  \bibinfo {author} {\bibfnamefont {E.}~\bibnamefont {Bousquet}}, \bibinfo
  {author} {\bibfnamefont {M.}~\bibnamefont {Verstraete}}, \bibinfo {author}
  {\bibfnamefont {D.}~\bibnamefont {Hamann}}, \bibinfo {author} {\bibfnamefont
  {X.}~\bibnamefont {Gonze}}, \ and\ \bibinfo {author} {\bibfnamefont {G.-M.}\
  \bibnamefont {Rignanese}},\ }\href {\doibase 10.1016/j.cpc.2018.01.012}
  {\bibfield  {journal} {\bibinfo  {journal} {Computer Physics Communications}\
  }\textbf {\bibinfo {volume} {226}},\ \bibinfo {pages} {39} (\bibinfo {year}
  {2018})}\BibitemShut {NoStop}%
\bibitem [{\citenamefont {Perdew}\ \emph {et~al.}(1996)\citenamefont {Perdew},
  \citenamefont {Burke},\ and\ \citenamefont {Ernzerhof}}]{Perdew1996}%
  \BibitemOpen
  \bibfield  {author} {\bibinfo {author} {\bibfnamefont {J.~P.}\ \bibnamefont
  {Perdew}}, \bibinfo {author} {\bibfnamefont {K.}~\bibnamefont {Burke}}, \
  and\ \bibinfo {author} {\bibfnamefont {M.}~\bibnamefont {Ernzerhof}},\ }\href
  {\doibase 10.1103/PhysRevLett.77.3865} {\bibfield  {journal} {\bibinfo
  {journal} {Physical Review Letters}\ }\textbf {\bibinfo {volume} {77}},\
  \bibinfo {pages} {3865} (\bibinfo {year} {1996})}\BibitemShut {NoStop}%
\bibitem [{\citenamefont {Welter}\ \emph {et~al.}(1992)\citenamefont {Welter},
  \citenamefont {Venturini},\ and\ \citenamefont {Malaman}}]{Welter1992}%
  \BibitemOpen
  \bibfield  {author} {\bibinfo {author} {\bibfnamefont {R.}~\bibnamefont
  {Welter}}, \bibinfo {author} {\bibfnamefont {G.}~\bibnamefont {Venturini}}, \
  and\ \bibinfo {author} {\bibfnamefont {B.}~\bibnamefont {Malaman}},\ }\href
  {\doibase 10.1016/0925-8388(92)90045-B} {\bibfield  {journal} {\bibinfo
  {journal} {Journal of Alloys and Compounds}\ }\textbf {\bibinfo {volume}
  {189}},\ \bibinfo {pages} {49} (\bibinfo {year} {1992})}\BibitemShut
  {NoStop}%
\bibitem [{\citenamefont {Ivantchev}\ \emph {et~al.}(2000)\citenamefont
  {Ivantchev}, \citenamefont {Kroumova}, \citenamefont {Madariaga},
  \citenamefont {{P{\'e}rez-Mato}},\ and\ \citenamefont
  {Aroyo}}]{Ivantchev2000}%
  \BibitemOpen
  \bibfield  {author} {\bibinfo {author} {\bibfnamefont {S.}~\bibnamefont
  {Ivantchev}}, \bibinfo {author} {\bibfnamefont {E.}~\bibnamefont {Kroumova}},
  \bibinfo {author} {\bibfnamefont {G.}~\bibnamefont {Madariaga}}, \bibinfo
  {author} {\bibfnamefont {J.~M.}\ \bibnamefont {{P{\'e}rez-Mato}}}, \ and\
  \bibinfo {author} {\bibfnamefont {M.~I.}\ \bibnamefont {Aroyo}},\ }\href
  {\doibase 10.1107/S0021889800007135} {\bibfield  {journal} {\bibinfo
  {journal} {Journal of Applied Crystallography}\ }\textbf {\bibinfo {volume}
  {33}},\ \bibinfo {pages} {1190} (\bibinfo {year} {2000})}\BibitemShut
  {NoStop}%
\bibitem [{\citenamefont {Chauvi{\`e}re}\ \emph {et~al.}(2009)\citenamefont
  {Chauvi{\`e}re}, \citenamefont {Gallais}, \citenamefont {Cazayous},
  \citenamefont {Sacuto}, \citenamefont {M{\'e}asson}, \citenamefont {Colson},\
  and\ \citenamefont {Forget}}]{Chauviere2009}%
  \BibitemOpen
  \bibfield  {author} {\bibinfo {author} {\bibfnamefont {L.}~\bibnamefont
  {Chauvi{\`e}re}}, \bibinfo {author} {\bibfnamefont {Y.}~\bibnamefont
  {Gallais}}, \bibinfo {author} {\bibfnamefont {M.}~\bibnamefont {Cazayous}},
  \bibinfo {author} {\bibfnamefont {A.}~\bibnamefont {Sacuto}}, \bibinfo
  {author} {\bibfnamefont {M.~A.}\ \bibnamefont {M{\'e}asson}}, \bibinfo
  {author} {\bibfnamefont {D.}~\bibnamefont {Colson}}, \ and\ \bibinfo {author}
  {\bibfnamefont {A.}~\bibnamefont {Forget}},\ }\href {\doibase
  10.1103/PhysRevB.80.094504} {\bibfield  {journal} {\bibinfo  {journal}
  {Physical Review B}\ }\textbf {\bibinfo {volume} {80}} (\bibinfo {year}
  {2009}),\ 10.1103/PhysRevB.80.094504}\BibitemShut {NoStop}%
\bibitem [{\citenamefont {Hadjiev}\ \emph {et~al.}(2008)\citenamefont
  {Hadjiev}, \citenamefont {Iliev}, \citenamefont {Sasmal}, \citenamefont
  {Sun},\ and\ \citenamefont {Chu}}]{Hadjiev2008}%
  \BibitemOpen
  \bibfield  {author} {\bibinfo {author} {\bibfnamefont {V.~G.}\ \bibnamefont
  {Hadjiev}}, \bibinfo {author} {\bibfnamefont {M.~N.}\ \bibnamefont {Iliev}},
  \bibinfo {author} {\bibfnamefont {K.}~\bibnamefont {Sasmal}}, \bibinfo
  {author} {\bibfnamefont {Y.-Y.}\ \bibnamefont {Sun}}, \ and\ \bibinfo
  {author} {\bibfnamefont {C.~W.}\ \bibnamefont {Chu}},\ }\href {\doibase
  10.1103/PhysRevB.77.220505} {\bibfield  {journal} {\bibinfo  {journal}
  {Physical Review B}\ }\textbf {\bibinfo {volume} {77}},\ \bibinfo {pages}
  {220505} (\bibinfo {year} {2008})}\BibitemShut {NoStop}%
\bibitem [{\citenamefont {Momma}\ and\ \citenamefont
  {Izumi}(2011)}]{Momma2011}%
  \BibitemOpen
  \bibfield  {author} {\bibinfo {author} {\bibfnamefont {K.}~\bibnamefont
  {Momma}}\ and\ \bibinfo {author} {\bibfnamefont {F.}~\bibnamefont {Izumi}},\
  }\href {\doibase 10.1107/S0021889811038970} {\bibfield  {journal} {\bibinfo
  {journal} {Journal of Applied Crystallography}\ }\textbf {\bibinfo {volume}
  {44}},\ \bibinfo {pages} {1272} (\bibinfo {year} {2011})}\BibitemShut
  {NoStop}%
\bibitem [{\citenamefont {Bauer}\ and\ \citenamefont
  {Sigrist}(2012)}]{Bauer2012}%
  \BibitemOpen
  \bibinfo {editor} {\bibfnamefont {E.}~\bibnamefont {Bauer}}\ and\ \bibinfo
  {editor} {\bibfnamefont {M.}~\bibnamefont {Sigrist}},\ eds.,\ \href {\doibase
  10.1007/978-3-642-24624-1} {\emph {\bibinfo {title} {Non-{{Centrosymmetric
  Superconductors}}}}},\ \bibinfo {series} {Lecture {{Notes}} in {{Physics}}},
  Vol.\ \bibinfo {volume} {847}\ (\bibinfo  {publisher} {{Springer Berlin
  Heidelberg}},\ \bibinfo {address} {{Berlin, Heidelberg}},\ \bibinfo {year}
  {2012})\BibitemShut {NoStop}%
\end{thebibliography}%



\end{document}







