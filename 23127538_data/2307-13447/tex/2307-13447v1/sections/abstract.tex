% Two key challenges in physical human-robot collaboration are: (i) the non-stationarity created by humans updating their behaviour; \theo{I don't think ii is addressed in this work} (ii) the ad-hoc coordination necessitated when robots encounter previously unseen humans. These challenges affect environmental transitions and hinder human-robot collaboration. We created a novel, principled, meta-learning framework to explore how robots could better predict human behaviour, and thereby deal with issues of non-stationarity. We used this framework to develop Behaviour-Transform (BeTrans). BeTrans is a conditional transformer that enables a robot to adapt quickly to new humans with non-stationary behaviour. We trained BeTrans on simulated humans with different systematic biases in physical collaborative settings. We used an original customisable environment to show BeTrans adapts faster to previously unseen non-stationary simulated humans than SOTA techniques. 
A key challenge in human-robot collaboration is the non-stationarity created by humans due to changes in their behaviour. This alters environmental transitions and hinders human-robot collaboration. We propose a principled meta-learning framework to explore how robots could better predict human behaviour, and thereby deal with issues of non-stationarity. On the basis of this framework, we developed Behaviour-Transform (BeTrans). BeTrans is a conditional transformer that enables a robot agent to adapt quickly to new human agents with non-stationary behaviours, due to its notable performance with sequential data. We trained BeTrans on simulated human agents with different systematic biases in collaborative settings. We used an original customisable environment to show that BeTrans effectively collaborates with simulated human agents and adapts faster to non-stationary simulated human agents than SOTA techniques.