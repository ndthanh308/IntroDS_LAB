\documentclass[acmtog,anonymous=False,review=False,nonacm]{acmart}
\fancyhead{}


\settopmatter{printacmref=false} % Removes citation information below abstract
\renewcommand\footnotetextcopyrightpermission[1]{} % removes footnote with conference information in first column
\pagestyle{plain} % removes running headers

\acmSubmissionID{1234}

\usepackage{booktabs} % For formal tables
\usepackage{bbding}
\usepackage{multirow}
% TOG prefers author-name bib system with square brackets
\citestyle{acmauthoryear}
%\setcitestyle{nosort,square} % nosort to allow for manual chronological ordering
 
\usepackage[ruled]{algorithm2e} % For algorithms
\renewcommand{\algorithmcfname}{ALGORITHM}
\SetAlFnt{\small}
\SetAlCapFnt{\small}
\SetAlCapNameFnt{\small}
\SetAlCapHSkip{0pt}



\usepackage{color}
\definecolor{applegreen}{rgb}{0.55, 0.71, 0.0}
\definecolor{autumnorange}{rgb}{0.87, 0.61, 0.33}
\newcommand{\onethousand}[1]{{\color{autumnorange}#1}}

\newcommand{\hb}[1]{{\color{purple}            {#1}}}
\newcommand{\hbc}[1]{{\color{cyan}            {[HB: #1]}}}



% Metadata Information
\acmJournal{TOG}
%\acmVolume{38}
%\acmNumber{4}
%\acmArticle{39}
%\acmYear{2019}
%\acmMonth{7}

% Copyrigh
%\setcopyright{acmcopyright}
%\setcopyright{acmlicensed}
%\setcopyright{rightsretained}
%\setcopyright{usgov}
%\setcopyright{usgovmixed}
%\setcopyright{cagov}
%\setcopyright{cagovmixed}

% DOI
%\acmDOI{0000001.0000001_2}

% Paper history
%\received{February 2007}
%\received{March 2009}
%\received[final version]{June 2009}
%\received[accepted]{July 2009}


% Document starts
\begin{document}
% Title portion
\title{Learning One-Quarter Headshot 3D GANs from a Single-View Portrait Dataset with Diverse Body Poses}

% DO NOT ENTER AUTHOR INFORMATION FOR ANONYMOUS TECHNICAL PAPER SUBMISSIONS TO SIGGRAPH 2019!
\thanks{Copyright © 2025 IEEE. Personal use of this material is permitted. Permission from IEEE must be obtained for all other uses, in any current or future media, including reprinting/republishing this material for advertising or promotional purposes, creating new collective works, for resale or redistribution to servers or lists, or reuse of any copyrighted component of this work in other works.}

\author{Yiqian Wu}
\affiliation{%
     \institution{State Key Lab of CAD\&CG, Zhejiang University}
     \city{Hangzhou}
     \country{China}
     }
\orcid{0000-0002-2432-809X}
\email{onethousand@zju.edu.cn}

\author{Hao Xu}
\affiliation{%
     \institution{State Key Lab of CAD\&CG, Zhejiang University}
     \city{Hangzhou}
     \country{China}
     }
\orcid{ }
\email{ }

\author{Xiangjun Tang}
\affiliation{%
     \institution{State Key Lab of CAD\&CG, Zhejiang University}
     \city{Hangzhou}
     \country{China}
     }
\orcid{ }
\email{ }
\author{Yue Shangguan}
\affiliation{%
     \institution{University of Texas at Austin}
     \city{AUSTIN}
     \country{United States}
     }
\orcid{ }
\email{ }
\author{Hongbo Fu}
\affiliation{%
     \institution{Hong Kong University of Science and Technology}
     \city{Hong Kong}
     \country{China}
     }
\orcid{ }
\email{hongbofu@cityu.edu.hk}

\author{Xiaogang Jin}
\authornote{Corresponding author.}
\affiliation{%
     \institution{State Key Lab of CAD\&CG, Zhejiang University; ZJU-Tencent Game and Intelligent Graphics Innovation Technology Joint Lab}
     \city{Hangzhou}
     \country{China}
     }
\orcid{0000-0001-7339-2920}
\email{jin@cad.zju.edu.cn}


\makeatletter
\let\@authorsaddresses\@empty
\makeatother

\begin{abstract}
Graph Neural Networks (GNNs) have proven to be effective in processing and learning from graph-structured data.
However, previous works mainly focused on understanding single graph inputs while many real-world applications require pair-wise analysis for graph-structured data (e.g., scene graph matching, code searching, and drug-drug interaction prediction).
To this end, recent works have shifted their focus to learning the interaction between pairs of graphs.
Despite their improved performance, these works were still limited in that the interactions were considered at the node-level, resulting in high computational costs and suboptimal performance.
To address this issue, we propose a novel and efficient graph-level approach for extracting interaction representations using co-attention in graph pooling. 
Our method, Co-Attention Graph Pooling (CAGPool), exhibits competitive performance relative to existing methods in both classification and regression tasks using real-world datasets, while maintaining lower computational complexity.

\end{abstract}
%
% The code below should be generated by the tool at
% http://dl.acm.org/ccs.cfm
% Please copy and paste the code instead of the example below.
%
\begin{CCSXML}
<ccs2012>
   <concept>
       <concept_id>10010147.10010371.10010382.10010385</concept_id>
       <concept_desc>Computing methodologies~Image-based rendering</concept_desc>
       <concept_significance>500</concept_significance>
       </concept>
 </ccs2012>
\end{CCSXML}

\ccsdesc[500]{Computing methodologies~Image-based rendering}

% 
%


\keywords{Portrait generation, 3D-aware GANs}

\begin{teaserfigure}
  \centering
  % Figure removed
  \caption{
  Our 3DPortraitGAN can generate one-quarter headshot 3D avatars and output portrait images (a and d) of a single identity using camera poses and body poses from reference images (c and f). The real reference images (c and f) are sampled from our \textit{$\it{360}^{\circ}$PHQ} dataset. Shapes (b and e) are iso-surfaces extracted from the density field of each portrait using marching cubes. We demonstrate that 3DPortraitGAN can generate canonical portrait images from all camera angles by showcasing the ${360}^{\circ}$ yaw angle exploration results in (g).
  % \hbc{Is it necessary to give the details (in the following sentence) here? Does it suggest different settings (e.g., different values of $\psi$) would lead to different results?} 
  % \onethousand{[Yiqian: It is common in EG3D and downstream applications to use $\psi=0.7$. since a higher value ($\psi >= 0.8 $) results in more variety but also more distortion. The conditional values of the generator could be removed. We emphasize that in our teaser, we use only one latent code to indicate that the resulting portraits belong to a single identity. ]}
  %\hbc{Okay, but this additional detail is not well connected with the previous sentences, since you didn't mention either $\psi$ or `latent code'.}
  %\onethousand{[Yiqian: Removed the details for clarity. ]}
  % For all results shown in this figure, we use truncation with $\psi=0.7$
  % condition the generator on the average camera parameters and neutral body pose, 
  % and use only one latent code.
  }
\label{fig:teaser}
\end{teaserfigure}



\maketitle


\section{Introduction}
Deep learning models have been widely used in many applications.
For example, BERT~\citep{devlin_bert_2019}, GPT-3~\citep{brown_language_2020}, and T5~\citep{raffel_exploring_2020} achieved state-of-the-art~(SOTA) results on different natural language processing~(NLP) tasks. 
For computer vision~(CV), Transformer-like models such as ViT~\citep{dosovitskiy_image_2021} and Swin Transformer~\citep{liu_swin_2021} deliver excellent accuracy performance upon multiple tasks. 


At the same time, training deep learning models has been a critical problem troubling the community due to the long training time, especially for those large models with billions of parameters~\citep{brown_language_2020}. 
In order to enhance the training efficiency, researchers propose some manually designed parallel training strategies~\citep{narayanan_efficient_2021,shazeer_mesh-tensorflow_2018,xu_gspmd_2021}. 
However, selecting, tuning, and combining these strategies require extensive domain knowledge in deep learning models and hardware environments. With the increasing diversity of modern hardware architectures~\cite{flynn_very_1966,flynn_computer_1972} and the rapid development of deep learning models, these manually designed approaches are bringing heavier burdens to developers. 
Hence, \emph{automatic parallelism} is introduced to automate the parallel strategy searching for training models.


There are two main categories of parallelism in deep learning models: inter-layer parallelism~\citep{huang_gpipe_2019,narayanan_pipedream_2019,narayanan_memory-efficient_2021,fan_dapple_2021,li_chimera_2021,lepikhin_gshard_2021,du_glam_2022,fedus_switch_2022} and intra-layer parallelism~\citep{li_pytorch_2020,narayanan_efficient_2021,rasley_deepspeed_2020,fairscale_authors_fairscale_2021}. 
Inter-layer parallelism partitions the model into disjoint sets on different devices without slicing tensors. 
Alternatively, intra-layer parallelism partitions tensors in a layer along one or more axes and distributes them across different devices.


Current automatic parallelism techniques focus on optimizing strategies within these two categories. However, they treat these two categories separately. 
Some methods~\citep{zhao_vpipe_2022,jia_exploring_2018,cai_tensoropt_2022,wang_supporting_2019,jia_beyond_2019,schaarschmidt_automap_2021,liu_colossal-auto_2023} overlook potential opportunities for inter- or intra-layer parallelism, the others optimize inter- and intra-layer parallelism hierarchically and sequentially~\citep{narayanan_pipedream_2019,fan_dapple_2021,he_pipetransformer_2021,tarnawski_efficient_2020,tarnawski_piper_2021,zheng_alpa_2022}. 
As a result, current automatic parallelism techniques often fail to achieve the global optima and instead become trapped in local optima. 
Therefore, a unified inter- and intra-layer approach is needed to enhance the effectiveness of automatic parallelism.


This paper aims to find the optimal parallelism strategy while simultaneously considering inter- and intra-layer parallelism. 
It enables us to search in a more extensive strategy space where the globally optimal solution lurk. 
However, unifying inter- and intra-layer parallelism in automatic parallelism brings us two challenges. 
Firstly, to adopt a unified perspective on the inter- and intra-layer automatic parallelism, we should not formalize them with separate formulations as prior works. Therefore, how can we express these parallelism strategies in a unified formulation? 
Secondly, previous methods take a long time to obtain the solution with a limited strategy space. Therefore, how can we ensure that the best solution can be obtained in a reasonable time while expanding the strategy space?


To solve the above challenges, we propose UniAP. For the first challenge, UniAP adopts the mixed integer quadratic programming~(MIQP)~\citep{lazimy_mixed_1982} to search for the globally optimal parallel strategy automatically. 
It unifies the inter- and intra-layer automatic parallelism in a single MIQP formulation. 
For the second challenge, our complexity analysis and experimental results show that UniAP can obtain the globally optimal solution in a significantly shorter time.


The contributions of this paper are summarized as follows: 
\begin{itemize}
    \item We propose UniAP, the first framework to unify inter- and intra-layer automatic parallelism in model training.
    \item The optimal parallel strategies discovered by UniAP exhibit scalability on training throughput and strategy searching time.
    \item The experimental results show that UniAP speeds up model training on four Transformer-like models by up to 1.70$\times$ and reduces the strategy searching time by up to 16$\times$, compared with the SOTA method.
\end{itemize}

% \section{Preliminaries}
% \paragraph{Input feature attribution methods.}
% Consider a linear model $f(x) = w_1 x_1 + w_2 x_2$. To explain which feature is more important for predicting the value of f(x), we can compare their coefficients. If $w_1 = 1000$ and $w_2 = 0.01$, we can say that $x_1$ would be weighed more than $x_2$. This type of explanation assumes that the values of $x_1$ and $x_2$ are of the same order. This is true in the case of most inputs to the neural network models, for example image pixels. 
% Gradients are the general way of discussing the coefficient with respect to a particular feature to discuss its importance.
% \textbf{Element-wise product of gradient into input} \textsc{grad $\odot$ input} \cite{Shrikumar2016NotJA}, provides global importance about the input feature in the model's output. 
% \cite{} have used it show the feature importance in attention models. 
% , as compared to just the gradient.
% details of computing the attribution with math 
% Assume $\mathbf{x}$ is a real-valued input feature vector (for any modality). For discrete inputs, real-valued vector obtained after passing the feature through a look-up embedding.

% , but there is no clear superior attribution technique over another. 

% Instead of considering attributions over pixels, \textbf{XRAI} \cite{Kapishnikov2019XRAIBA} computes the effective attributions of integrated gradients over overly segmented image. The image is segmented based on similarity such as color, which makes the segment boundaries align with the edges. The segmentation is done at multiple scales to obtain a set of overlapping image segments.
% Assume that attribution mask over an image $I$ of size ${H\times W}$ is $A$ of the same size. 
% Using graph-based segmentations over multiple scale parameters, we obtain a set of segments $\mathcal{S}$. 
% Let a pixel be indexed by $i$ in the original image. For a segment $s$, the gain can be calculated by $g_s = \sum_{i \in s\backslash M} \frac{A_i}{area(s\backslash M)}$. 
% The segment with maximum gain is selected as  attribution to update the XRAI saliency set $\mathcal{M}$.
% The process is repeated with the remaining segments until the area of the mask set is equal to that of the image. 
% While this method seems to produce slightly better visual attributions over other variants of IG, it is sensitive to the size of segmentation scales and dilation factor. We consider  $XRAI(\cdot)$ to denote this attribution method for visual attribution analysis in \S \ref{subsec:visual_attr}.   
% which create grainy regions. 
% However, this method depends on the size of segmentation scales selected for computation. Further, dilation added to the final attribution masks to include edges may depict an inflated version of model's actual feature importance. 
% In this work, $XRAI(\cdot)$ denotes that this attribution method is applied.
%  \vspace{-0.5em}
\section{Related Work}
%  \vspace{-0.3em}
\label{sec:related_work}
\paragraph{Interpretability and explainability } Recent work in multimodal explainability in autonomous vehicles \cite{gilpin-2021-multimodal} uses symbolic explanations to debug and process outputs out of sub-components.
In contrast, we address the challenge of post-hoc multimodal interpretability for any existing end-to-end trained differentiable policies. \textsc{grad $\odot$ input}~\cite{Shrikumar2016NotJA},  a simple and modality-agnostic attribution that works on par with recent methods~\cite{Ancona2017AUV}. We use this method to compute multimodal attribution at inputs to the fusion layer to weigh how each modality contributes to the decision-making. 
% as it has been shown to work at par compared to the recent gradient-based attribution techniques~\cite{Ancona2017AUV}.
While \textsc{grad $\odot$ input} is a modality-agnostic starting point for attributions, 
it is not easy to understand, especially for images. Among recent works to improve visual attribution  \cite{Smilkov2017SmoothGradRN, Simonyan2014DeepIC, ig, sturmfels2020visualizing, Xu_2020_CVPR, Kapishnikov2021GuidedIG, Kapishnikov2019XRAIBA},  we use XRAI~\cite{Kapishnikov2019XRAIBA} for vision-specific analysis as it produces visually intuitive attributions by relying on regions, not individual pixels. 
% \cite{Smilkov2017SmoothGradRN} proposed ways to visually sharpen these vanilla gradient-based attributions. ~\cite{Simonyan2014DeepIC}  applying Gaussian noise perturbations over averaged over a sufficient number of samples.
% describe IG
% IG \cite{ig} and path methods have been studied as a cost-sharing method called Aumann-Shapley. 
% Attribution based on IG preserves axiomatic properties like \textit{sensitivity} and \textit{implementation invariance}.
% While IG aggregate the gradients on sampling inputs on a straight line between the baseline and the input, there are several paths possible in higher dimensional spaces and corresponding different attribution.
% Recent works build on IG to obtain more visually intuitive attributions, like SHAP Deep Explainer~\cite{sturmfels2020visualizing}, Blur IG ~\cite{Xu_2020_CVPR}, Guided IG~\cite{Kapishnikov2021GuidedIG} and XRAI~\cite{Kapishnikov2019XRAIBA}. Qualitatively, XRAI showed visually intuitive attributions by relying on regions and not individual pixels.  
% Interpretability using gradient-based attribution techniques is quite similar to adversarial attacks \cite{Goodfellow2015ExplainingAH} and adversarial training for robustness \cite{Bai2021RecentAI}, as both fundamentally rely on gradient of the input feature with respect to the output. 
% Do we need a figure to show the difference in attributions with just gradient vs gradxinput? 
\vspace{-0.8em}
\paragraph{Language-driven task benchmarks}

There are many benchmarks to study an agent’s ability to follow natural language instructions \cite{ALFRED20, padmakumar2022teach, gu2022vision,  mahmoudieh2022zero}. 
% While most existing settings apply only to either navigation \cite{} or manipulation \cite{}, 
% we conside one of the benchmarks which handles both, that is,
% navigation (Anderson et al., 2018; Chen
% et al., 2019), object manipulation (Misra et al.,
% 2017; Zhu et al., 2017) and embodied reasoning
% (Das et al., 2018a; Gordon et al., 2018). 
ALFRED \cite{ALFRED20} serves as a suitable testbed for this analysis as these tasks require both high reasoning for navigation and manipulation. ALFRED dataset provides visual demonstrations collected through PDDL planning in 3D Unity household environments and natural language description of the high-level goal and low-level instructions annotated by MTurkers. 
The benchmarks provide evaluation metrics for the overall task goal completion success rate (SR) and those weighted by the expert's path length (PLWSR)
% over seen and unseen tasks
and have reported a huge gap in the performance of learning algorithms and humans at these tasks. 
% ALFRED  is a benchmarking environment that provides natural language instructions annotated by MTurkers on egocentric visual sequences of actions taken for everyday household tasks. As ALFRED is a simulated environment on Unity3D game engine, the visual demonstrations are collected based on PDDL planning. 

\vspace{-0.8em}
\paragraph{End-to-end Learned Policies} We investigate the end-to-end learned policies for the task, such that, the gradient can be attributed at a task level. While we do not discuss modular yet differentiable policies like \cite{min2021film} \cite{DBLP:journals/corr/ZhouC15}, tying the gradient across multiple modular learned components is a direction for future work.
% as 
% tying the gradient among modular learned components in future work. 
In our work, we consider the checkpoints of policies trained on the ALFRED dataset. Broadly, these policies are of two types: (a) sequence-to-sequence models, that are, the one proposed with ALFRED dataset (Baseline) \cite{ALFRED20} and Modular Object-Centric Approach (MOCA) \cite{Singh2021FactorizingPA}, (b) transformer-based models, that are Episodic Transformers (ET) \cite{pashevich2021episodic}, and Hierarchical Tasks via Unified Transformers (HiTUT) \cite{Zhang2021HierarchicalTL}. Refer Table~\ref{tab:policiesarch} to compare architectural details \footnote{Previous action is modeled with learned embedding look-up in all these policies.}.
% \textbf{Seq2Seq(Baseline)} \cite{ALFRED20} is a single-stream Seq-to-Seq model with progress monitoring, processing the visual frames through  a frozen ResNet-18 encoder, language through bi-LSTM and soft attention and fusion of the latent visual, language and previous action encodings through an LSTM layer.
%%%% The visual frames are encoded by a frozen ResNet-18 encoder. The language instruction tokens are processed with a bi-LSTM and soft attention. The latent encodings for visual, language and previous action are passed through an LSTM.
% \textbf{MOCA} \cite{Singh2021FactorizingPA} presents a factorized model into two, i.e. interactive perception and action policy. The inputs to the action policy model are language encoding from bi-LSTM, visual embedding from a pretrained ResNet-18, and previous action embedding; all concatenated as input to an LSTM with residual connection.
% \textbf{Episodic Transformers} \cite{pashevich2021episodic} proposes a transformer architecture that encodes the language instructions and the sequence of visual observations and actions to predict subsequent actions per visual frame. Visual observations are processed through pretrained ResNet-50, language tokens passed through a transformer encoder pre-trained with synthetic language targets,  and action are encoded by embedding look-up. 

% Please add the following required packages to your document preamble:
% \usepackage{booktabs}
% Please add the following required packages to your document preamble:
% \usepackage{booktabs}
% Please add the following required packages to your document preamble:
% \usepackage{booktabs}
\begin{table}[t]
\centering
%  \vspace{-1em}
\caption{Policies trained on ALFRED Dataset and their architectures for each modality}
\label{tab:policiesarch}
\begin{tabular}{@{}llll@{}}
\toprule
Policies & Visual                                                                       & Language                       & Fusion                                                                   \\ \midrule
Baseline \cite{ALFRED20} & Frozen ResNet-18                                                             & Learned Embedding, Bi-LSTM     & LSTM                                                                     \\
MOCA \cite{Singh2021FactorizingPA}    & \begin{tabular}[c]{@{}l@{}}Frozen ResNet-18\\ + Dynamic Filters\end{tabular} & Learned Embedding, Bi-LSTM     & \begin{tabular}[c]{@{}l@{}}LSTM with \\ residual connection\end{tabular} \\
ET \cite{pashevich2021episodic}      & Frozen ResNet-50                                                             & Learned Embedding, Transformer & Transformer Encoder                                                      \\
HiTUT \cite{Zhang2021HierarchicalTL}   & Frozen MaskRCNN                                                              & Learned Embedding, FC, LN      & Transformer Encoder                                                  \\ \bottomrule
\end{tabular}
\vspace{-0.2em}
\end{table}
% EmBERT




 
%  provide spurious 
%  explanations and 
%  may not 
%  In cases where the attribution may 
%  this method depends on the underlying attribution methods such as IG. 



% Figure environment removed




\section{Dataset}
\label{sec: Dataset}
In this section, we will describe our data processing pipeline. We focus on utilizing single-view portrait images from the internet due to their easy accessibility and abundance. However, utilizing face recognition and face reconstruction methods to gather annotated data from all camera angles is challenging, since the necessary facial features required for accurate recognition may be obscured. Therefore, we propose to use body features (e.g., shoulders) that are more distinctive to collect data.
%
In particular, we introduce a novel data processing method based on an off-the-shelf body reconstruction method to extract camera parameters and body poses from in-the-wild images, enabling us to obtain aligned portraits.

%
We begin by making the initial assumption that we have a human body SMPL \cite{DBLP:journals/tog/LoperM0PB15} template mesh in the local space, denoted as $M$, with the standard body shape. 
Its neck joint is aligned to the origin point $[0,0,0]$, and there is no additional global rotation or translation performed on $M$. We denote the template mesh with body pose parameters $\vec{\theta} \in \mathbb{R}^{69}$ as $M(\vec{\theta})$.
%
As we aim to solely preserve the head, neck, and shoulders region of the input portrait, we only consider the neck pose $p_n \in \mathbb{R}^{3}$ and head pose $p_h \in \mathbb{R}^{3}$ in $\vec{\theta}$, while disregarding the remaining body pose. Thus, we define neck pose and head pose as $p = [p_n,p_h] \in \mathbb{R}^{6}$ and denote the template mesh with neck and head pose $p$ as $M(p)$.
%
Regarding camera settings, we assume that our camera is always positioned on a sphere with radius $r = 2.7$, directed towards a fixed point. Additionally, intrinsic camera parameters are fixed as constant values.


Given an in-the-wild portrait image, our aim is to find the camera parameters, neck pose, and head pose of the portrait, allowing the rendering result of the template mesh $M$ to be aligned with the head, neck, and shoulders region of the input portrait.
%
Using an off-the-shelf body reconstruction method, 3DCrowdNet \cite{DBLP:conf/cvpr/ChoiMPL22}, we extract the input portrait's SMPL parameters (global rotation $rot$ and translation $trans$, shape parameters $\vec{\beta} \in \mathbb{R}^{10}$, and pose parameters $\vec{\theta} \in \mathbb{R}^{69}$), resulting in an estimated mesh $\tilde{M}(trans, rot, \vec{\beta},\vec{\theta})$ in the world space (with a fixed camera).
% 
We apply the neck pose and head pose of the estimated mesh to our template mesh, resulting in $M(p)$. Then we compute the transformation matrix that could transform $\tilde{M}(trans, rot, \vec{\beta},\vec{\theta})$ to align its head, neck, and shoulders joints with those of $M(p)$. 
%
Next, we apply the same transformation matrix to the fixed camera and normalize its camera parameters according to our camera assumption, obtaining the final camera parameters. 
%
The final camera parameters, denoted as $c \in \mathbb{R}^{25}$, comprises an extrinsic camera matrix $e \in \mathbb{R}^{16}$ and an intrinsic camera matrix $k \in \mathbb{R}^{9}$. Note that $k$ remains fixed as a constant matrix.
%
Then the raw image is cropped and aligned based on the obtained parameters, resulting in an aligned image denoted as $I$ (see Fig. \ref{fig: dataset_sample}). 
%


% Figure environment removed


To filter out the images with inaccurately estimated camera parameters, we render the mesh $M(p)$ on $I$ using the camera parameters $c$. We then manually examine the rendering results and remove any images where the mesh rendering is not well-aligned with $I$, as well as blurry or noisy images.
%
However, due to the limitations of the body reconstruction method, we encounter cases where the neck and head rendering results are not aligned with $I$, even though shoulder reconstruction is more accurate. During manual selection, we also avoid using the neck and head alignment as a criterion for manual selection and only consider the shoulder alignment.
%
As a result, the estimated camera parameters can render the template mesh in the local space to have aligned shoulders with the portraits. However, the estimated neck and head pose $p$ is ``coarse'' and inaccurate.
%
Therefore, instead of being used directly as the training label, this coarse body pose is only employed to calculate a regularization loss during the early stages of the training process, which we will explain in later sections.




In sum, we collect 41,767 raw portrait images from Pexels\footnote{\hyperref[]{https://www.pexels.com}} and Unsplash\footnote{\hyperref[]{https://unsplash.com}}, finally getting \textbf{54,000} aligned images as our \textit{$\it{360}^{\circ}$PHQ} dataset.  Samples of these images can be found in Fig. \ref{fig: dataset_sample} as well as the supplementary file.
The datasets are augmented by a horizontal flip.
%
We convert the camera positions in our dataset to the spherical coordinate system ($\mu$ and $\nu$, or yaw and pitch), and visualize the distribution of camera positions in Fig. \ref{fig: data_distribution}. 
%Our dataset exhibits a wide range of camera distributions, and the images in our dataset are high-quality, and feature variations in gender, age, race, expression, and lighting.
Our dataset contains a diverse set of camera distributions, and the images are of high quality, with variations in gender, age, race, expression, and lighting.
%
The analysis of the distribution of semantic attributes (gender, race, age, etc) can be found in the supplementary file. 







% Given an in-the-wild portrait image, our goal is to find the camera pose and body pose of the portrait, these parameters can then be used to render the $M(\vec{\theta})$ that aligned with the input portrait.
% We first employ an off-the-shelf body reconstruction method \cite{DBLP:conf/cvpr/ChoiMPL22} to extract the input portrait's SMPL parameters (global rotation $rot$ and translation $trans$, shape parameters $\vec{\beta} \in \mathbb{R}^{10}$ and pose parameters $\vec{\theta} \in \mathbb{R}^{69}$), getting an estimated mesh $\tilde{M}(trans, rot, \vec{\beta},\vec{\theta})$ in the world space.
% %
% Then we apply the pose parameters $\vec{\theta}$ of the estimated mesh to our template mesh as $M(\vec{\theta})$.
% % 
% The body reconstruction method \cite{DBLP:conf/cvpr/ChoiMPL22} outputs a mesh $\tilde{M}(trans, rot, \vec{\beta},\vec{\theta})$ that is globally rotated and translated (in the world space), along with a fixed camera to render it. Our aim is to determine the camera parameter to render the mesh $M(\vec{\theta})$ (in the local space) onto the portrait image, resulting in similar rendering outcomes as the estimated mesh.


% To accomplish this, we first translate $\tilde{M}(trans, rot, \vec{\beta},\vec{\theta})$ to align its neck joint to the origin point $[0,0,0]$, obtaining the translation matrix $T$. We then compute the rotation matrix that rotates $\tilde{M}(trans, rot, \vec{\beta},\vec{\theta})$ to align it with $M(\vec{\theta})$, obtaining the rotation matrix $R$.
% %
% Next, we apply the transformation matrix $RT$ to the fixed camera and normalize its camera pose according to our camera assumption, obtaining the final camera parameter. 
% %
% The final camera parameter, denoted as $c \in \mathbb{R}^{25}$, comprises an extrinsic camera matrix $e \in \mathbb{R}^{16}$ and an intrinsic camera matrix $k \in \mathbb{R}^{9}$. Note that $k$ remains fixed as a constant matrix.





% As a result, for each in-the-wild portrait image, we obtain its pose parameters $\vec{\theta}$, and camera pose $c$.
% %In this paper, we omit the shape parameters $\vec{\beta}$. 
% As we solely aim to preserve the head, neck, and shoulders region of the input portrait, we only consider the neck pose $p_n \in \mathbb{R}^{3}$ and head pose $p_h \in \mathbb{R}^{3}$ in $\vec{\theta}$, while disregarding the remaining body pose.
% %
% As a result, we define neck pose and head pose $p = [p_n,p_h] \in \mathbb{R}^{6}$ as the body pose, and denote the template mesh with body pose $p$ as $M(p)$ throughout the rest of the paper.
% Then all the raw images are cropped and aligned based on the obtained parameters, resulting in an aligned image denoted as $I$ (see Fig. \ref{fig: dataset_sample}). 












\section{Methodology}
\label{sec: Methodology}
We first present a comprehensive overview of our proposed approach. Firstly, we introduce the design of our body pose-aware discriminator in Sec. \ref{sec: Discriminator}, which is capable of extracting body poses and corresponding scores from the input images.
%
Subsequently, we elaborate on our 3DPortraitGAN in  Sec. \ref{sec: 3DPortraitGAN}, including the pose predictor in generator (Sec. \ref{sec: pose sampling}), the generator backbone (Sec. \ref{sec: EG3D Backbone}) and our deformation module (Sec. \ref{sec: Deformation Module}).
%
Finally, we discuss the losses utilized in our training process in  Sec. \ref{sec: Losses}, along with the details of the training in  Sec. \ref{sec: Training Details}.


% Figure environment removed



\subsection{Body Pose-aware Discriminator}
\label{sec: Discriminator}
In Sec. \ref{sec: Dataset}, we mentioned that the body poses in our dataset are inaccurate and cannot be directly used for training.
Drawing inspiration from the pose-free 3D-aware generator, Pof3D \cite{shi2023pof3d}, we propose employing the discriminator to predict more accurate body pose $\hat{p}$ from real/generated images.

Taking a real image $I_{real}$ as an example, we denote its camera parameters in the \textit{$\it{360}^{\circ}$PHQ} dataset as $c_{real}$.
As shown in Fig. \ref{fig: discriminator}, the convolutional layers first extract features from image $I_{real}$. Then the features and the camera parameters are fed into a pose predictor branch $\Gamma_D$, yielding a predicted body pose:
\begin{equation}
	\begin{split}
    \label{eqn: pose_predict_D}
     \hat{p}_{real} &= \Gamma_D (\mathrm{Conv}(I_{real}), c_{real}),
     \end{split}
\end{equation}
where $\mathrm{Conv}$ denotes the convolutional layers.
We observe that $\Gamma_D$ faces difficulty in accurately predicting symmetry poses for symmetry images. To explicitly maintain the symmetry of $\Gamma_D$, we propose an explicit symmetry strategy for $\Gamma_D$.
%
Specifically, once fed into $\Gamma_D$, we flip the input image horizontally when the spherical coordinate $\mu$ of its camera position falls within the range of $[-\frac{1}{2}\pi,\frac{1}{2}\pi]$ (which indicates that the camera is on the left-hand side of the subject, see Fig. \ref{fig: data_distribution}). We also flip the resulting predicted body pose so as to obtain the final predicted body pose. This operation guarantees that two symmetrical images have the symmetry value of the predicted body pose.




Next, we feed $c_{real}$ and $\hat{p}_{real}$ into a pose feature mapping network, and the image features are fed into an image feature mapping network. Then the outputs of the two feature mapping networks are multiplied to get the final score of the discriminator:
\begin{equation}
	\begin{split}
    \label{eqn: discriminator_real}
     score_{real} & = D(I_{real}\vert c_{real}, \hat{p}_{real} ), 
     \end{split}
\end{equation}
where $D$ denotes the discriminator. 

Likewise, $D$ extract scores and body pose from generated images as:
\begin{equation}
	\begin{split}
    \label{eqn: discriminator_gen}
    \hat{p}_{gen} &= \Gamma_D (\mathrm{Conv}(I_{gen}), c_{gen}), \\
     score_{gen} & = D(I_{gen}\vert c_{gen}, \hat{p}_{gen} ),
     \end{split}
\end{equation}
where $I_{gen}$ refers to the image generated by our generator (which we will describe in the following section) using camera parameters $c_{gen}$ and a specific body pose. $c_{gen}$ is sampled from the \textit{$\it{360}^{\circ}$PHQ} dataset, while $\hat{p}_{gen}$ denotes the predicted body pose of $I_{gen}$.



% Figure environment removed



\subsection{3DPortraitGAN}
\label{sec: 3DPortraitGAN}



\subsubsection{Pose Sampling}
\label{sec: pose sampling}
In this paper, we propose to generate image $I_{gen}$ using latent code $z$, camera parameters $c_{gen}$ and a certain body pose $p_{gen}$ as :
\begin{equation}
	\begin{split}
    \label{eqn: generate}
     I_{gen} = G(z, c_{gen}, p_{gen}) \sim P(I_{gen} \vert z, c_{gen}, p_{gen}).
     \end{split}
\end{equation}
In order to train our generator, we need to sample camera parameters and body poses from the pose distribution of \textit{$\it{360}^{\circ}$PHQ} dataset. 


However, in \textit{$\it{360}^{\circ}$PHQ} dataset, while the camera parameters we extract from real images are somewhat precise (after our manual selection), the body poses are ``coarse'' and cannot be used as training data (Sec. \ref{sec: Dataset}).
%
Taking inspiration from Pof3D \cite{shi2023pof3d}, we employ a pose prediction network $\Gamma_G$ to estimate the conditional distribution of body pose based on randomly sampled latent codes and camera parameters drawn from the \textit{$\it{360}^{\circ}$PHQ} dataset. More specifically, we employ a predictor $\Gamma_G$  to predict body pose from both latent code and camera parameters:
\begin{equation}
	\begin{split}
    \label{eqn: pose_predict_G}
     {p}_{gen} &= \Gamma_G (z, c_{gen})  \sim P({p}_{gen} \vert z, c_{gen}).
     \end{split}
\end{equation}
Similarly to $\Gamma_D$, we want the predicted body poses that correspond to symmetry camera poses to have symmetry distribution. Thus we apply the explicit symmetry strategy to $\Gamma_G$.
We horizontally flip the camera parameters $c_{gen}$ with $\mu \in [-\frac{1}{2}\pi,\frac{1}{2}\pi]$, and then the predicted ${p}_{gen}$ is flipped to obtain the final predicted body pose.  



Then we can generate images from $z$ and $c_{gen}$ by:
\begin{equation}
	\begin{split}
    \label{eqn: generate_new}
     I_{gen} = G(z, c_{gen}, \Gamma_G (z, c_{gen})) \sim P(I_{gen} \vert z, c_{gen}, \Gamma_G (z, c_{gen})).
     \end{split}
\end{equation}
In the next section, we will introduce how our generator renders image ${I}_{gen}$ with certain camera parameters ${c}_{gen}$ and body pose ${p}_{gen}$ in detail.

% % Figure environment removed






\subsubsection{EG3D Backbone}
\label{sec: EG3D Backbone}
As shown in Fig. \ref{fig: pipeline},  we utilize EG3D \cite{DBLP:conf/cvpr/ChanLCNPMGGTKKW22} as our backbone. 
After predicting body pose ${p}_{gen}$ from latent code $z$ and camera parameters $c_{gen}$, 
% we input $z$ into the mapping network, while using $c_{gen}$ and  ${p}_{gen}$ as the conditional input of the mappin network. 
%
we input $z$ into the mapping network, where ${p}_{gen}$ and $c_{gen}$ serve as its conditional inputs.
The resulting $w$ latent code is then used to modulate the generator to synthesize feature maps, which are then reshaped to tri-planes.
In this paper, we assume that our tri-planes are ``normalized'' and canonical. 
This indicates that if we directly render the results using the tri-planes, we will obtain a canonical human body geometry representation with a neutral body pose.


During volumetric rendering, each point $\bold{x} = (x,y,z)$ on the ray is projected onto the tri-planes as $(x,y),(y,z),(z,y)$. These projections are used to sample features from the tri-planes, 
which are then fed into a decoder to obtain the color and density of point $\bold{x}$ and to perform volumetric rendering. 
After that, the rendered image is fed to a super-resolution network to obtain the final high-resolution image.



\subsubsection{Deformation Module}
\label{sec: Deformation Module}
In Sec. \ref{sec: EG3D Backbone}, we assume that the tri-planes generated by our generator are canonical. 
Given a body pose $p_{gen}$, to achieve a final portrait that conforms to $p_{gen}$, we utilize the deformed mesh $M(p_{gen})$ to produce a deformation field that maps each sampled point $\bold{x} = (x, y, z)$ in the observation space to a corresponding point $\bold{x}' = (x', y', z') = (x + \Delta x, y + \Delta y, z + \Delta z) $ in the canonical space, where $ \Delta \bold{x} =  (\Delta x, \Delta y,  \Delta z) $ denotes the deformation field value.


In RigNeRF \cite{DBLP:conf/cvpr/AtharXSSS22}, the deformation field is computed using 3DMM, and a neural network is utilized to predict the residual non-rigid deformation value. 
To enhance the efficiency of our training process, we exclude the NN-based non-rigid deformation training employed in RigNeRF and directly use a canonical SMPL mesh $M(0)$ ($0$ refer to neutral body pose) and the deformed SMPL mesh $M(p_{gen})$ to compute a deformation field. Similar to RigNeRF, the SMPL deformation field value at a point $\bold{x}$ on the ray is defined as follows:
\begin{equation} 
	\begin{split}
    \label{eqn: original SMPLDef}
    & \Delta \bold{x}  = SMPLDef(\bold{x}, p_{gen}) = \frac{SMPLDef(\hat{\bold{x}},p_{gen})}{\exp(\Vert\bold{x} , \hat{\bold{x}}\Vert^2)}, \\
    & SMPLDef(\hat{\bold{x}},p_{gen}) = \hat{\bold{x}}_{M(0)} - \hat{\bold{x}}, \enspace
     \end{split}
\end{equation}
where $\hat{\bold{x}}$ is the closest point on the deformed mesh $M(p_{gen})$ to $\bold{x}$, and  $\Vert\bold{x} , \hat{\bold{x}}\Vert^2$ is the Euclidean Distance between $\bold{x}$ and $\hat{\bold{x}}$. $\hat{\bold{x}}_{M(0)}$ is the position of point $\hat{\bold{x}}$ on $M(0)$.
%As a result, the deformation direction of point $\bold{x}$ is the translation direction of its closest point on the deformed SMPL mesh. Moreover, the magnitude of the deformation is determined by the distance between $\bold{x}$ and $\hat{\bold{x}}$.

However, the deformation field in RigNeRF may encounter issues when the NN-based non-rigid deformation is disabled and the body pose $p_{gen}$ is large. 
% 
The computation of RigNeRF's deformation field $\Delta \bold{x}$ depends only on the translation of the point, ignoring the relative positioning between the sample point and the mesh. This approach produces an ``offset face'' as depicted in Fig. \ref{fig: ablation_deform}.


To tackle this issue, as shown in Fig. \ref{fig: pipeline}, we utilize a deformation field that accounts for the positional relationship between $\bold{x}$ and its nearest face on the mesh:
\begin{equation}
\begin{split}
\label{eqn: new SMPLDef}
& \Delta \bold{x}  = SMPLDef(\bold{x},p_{gen}) = \left\{
\begin{aligned}
&\check{\bold{x}} - \bold{x}, \enspace \quad  \Vert\bold{x}, \hat{f}\Vert^2<\alpha\\
&0, \enspace \quad \Vert\bold{x}, \hat{f}\Vert^2 \geq \alpha \\
\end{aligned}
\right. \\
&\bold{x}  = C_{\hat{f}}(u,v,h) ,\quad  \check{\bold{x}}  = C_{\hat{f}}^{M(0)}(u,v,h), \enspace 
\end{split}
\end{equation}
where $\hat{f}$ is the face on $M(p_{gen})$ that is closest to $\bold{x}$, and  $ \Vert\bold{x}, \hat{f}\Vert^2$ denotes the Euclidean distance between $\bold{x}$ and  $\hat{f}$, $\alpha$ is a hyper-parameter that controls the ``thickness'' of the geometry beyond the mesh (we empirically set $\alpha$ as 0.25).

We first obtain the local coordinate system $C_{\hat{f}}$ of $\hat{f}$ using its vertices, which yields the local coordinates $(u,v,h)$ of $\bold{x}$ in $C_{\hat{f}}$.  Next, we obtain the local coordinate system $C_{\hat{f}}^{M(0)}$ of the same face on the template mesh $M(0)$ and use $(u,v,h)$ to compute the new global coordinates $\check{\bold{x}}$. 
In other words, if $\bold{x}$ is close to the mesh, its position relative to its closest face on the mesh remains unchanged.




\subsection{Losses}
\label{sec: Losses}

\subsubsection{Discriminator Loss}
We define the loss of the discriminator as:
\begin{equation}
\begin{split}
    \label{eqn: D discriminator loss}
    L_D = & -\mathbb{E} [\log(1-D(I_{gen}\vert c_{gen}, \hat{p}_{gen} ))] \\
          & - \mathbb{E} [\log (D(I_{real}\vert c_{real}, \hat{p}_{real} ) ] \\
          &  +  \lambda \mathbb{E} [ ||\nabla_{I_{real}} D(I_{real}\vert c_{real}, \hat{p}_{real} ) ||_2] + \lambda_p L_{p},
    \end{split}
\end{equation}
where $\lambda \mathbb{E} [ ||\nabla_{I_r} D(I_{real}\vert c_{real}, \hat{p}_{real} ) ||_2]$ is the gradient penalty, $\lambda_{p}$ represents the weight of $L_{p}$, $L_{p}$ is the body pose loss, which is used to optimize the pose predictor $\Gamma_D$ in the discriminator as:
\begin{equation}
	\begin{split}
    \label{eqn: body pose loss}
    L_{p} & = L_2({p}_{gen}, \hat{p}_{gen}),
     \end{split}
\end{equation}
where ${p}_{gen}$ could be regarded as the ground truth body pose of $I_{gen}$ (since ${p}_{gen}$ is used to perform deformation), and $\hat{p}_{gen}$ is the body pose that is predicted by the discriminator from $I_{gen}$, $L_2$ denotes the $L_2$ distance.


\begin{table*}[t]
\begin{tabular}{@{}ccccccc@{}}
\toprule
                         & Image            & $L_{preg}$                    & $L_{rear}$            & Neural Rendering Resolution     & $\Gamma_G$              \\ \midrule
\multirow{3}{*}{Stage 1} & 0$\sim$0.2M          & \CheckmarkBold & \CheckmarkBold  (w/o normalization)  & $64^2$                          & training          \\
                         & 0.2M$\sim$0.8M       & \XSolidBrush   & \CheckmarkBold  (w/ normalization)   & $64^2$                          & training        \\
                         & 0.8M$\sim$4M     & \XSolidBrush   & \CheckmarkBold  (w/ normalization)   & $64^2$                          & training         \\\midrule
Stage 2                  & 4M$\sim$10M  & \XSolidBrush   & \CheckmarkBold  (w/ normalization)   & $64^2$                          & freeze           \\ \midrule
\multirow{2}{*}{Stage 3} & 10M$\sim$11M & \XSolidBrush   & \CheckmarkBold  (w/ normalization)   & increase from $64^2$ to $128^2$ & freeze       \\ 
                         & 11M$\sim$14M & \XSolidBrush   & \CheckmarkBold  (w/ normalization)   & $128^2$                         & freeze       \\\bottomrule
\end{tabular}
\caption{The training details of our three-stage training. M means million images.}
  \label{tab: training_details}
\end{table*}



\subsubsection{Generator Loss}
We define the generator's loss as follows:
\begin{equation}
\begin{split}
    \label{eqn: G discriminator loss}
    L_G = & - \mathbb{E} [\log (D(I_{gen}\vert c_{gen}, \hat{p}_{gen} )) ] \\
    & + \lambda_{preg} L_{preg} + \lambda_{rear} L_{rear},
    \end{split}
\end{equation}
where $L_{preg}$ represents the body pose regularization loss, and $L_{rear}$ represents the rear-view depth regularization loss, $\lambda_{preg}$ and $\lambda_{rear}$ represents the weights of the regularization losses.
The body pose regularization loss $L_{preg}$ will be only employed in the very early stage of our training process, we provide more details in Sec. \ref{sec: Training Details}.


 \subsubsection{Body Pose Regularization Loss}
Although our network can learn the relative body pose between different images, it struggles to predict the absolute body pose due to the lack of prior information. The predicted pose can be viewed as a ``deviated'' pose, which has an offset from the true value. This offset does not affect camera parameters prediction (as in Pof3D \cite{shi2023pof3d}) since the camera system can be globally rotated. However, we use an SMPL model in the canonical space, which means a ``deviated'' body pose will result in an unnatural mesh deformation. To address this issue, we propose using $L_{preg}$ to constrain the value of the predicted pose as follows:
\begin{equation}
	\begin{split}
    \label{eqn: body pose reg loss}
    {p}_{gen} &= \Gamma_G (z, c_{gen}), \\
     L_{preg} & =  L_2({p}_{gen}, p_{coarse}), 
     \end{split}
\end{equation}
where $p_{coarse}$ represents the coarse body pose in the \textit{$\it{360}^{\circ}$PHQ} dataset, 
and $c_{gen}$ and $p_{coarse}$ are from the same real image.





 \subsubsection{Rear-view Depth Regularization Loss}
We observe the presence of a face on the back of the head (refer to Fig. \ref{fig: rear-view-reg} in ablation studies). Despite the generator's ability to learn the texture of hair at the back of the head, the face geometry is apparent in the shape extracted by the marching cubes algorithm. Such an artifact also exists in the comparison results in Rodin \cite{DBLP:journals/corr/abs-2212-06135}, where they adapted the official implementation of EG3D to 360-degree generation and retrain EG3D using their rendered dataset.
%
We attribute this phenomenon to the geometric ambiguity when using single-view images as training data. The discriminator can only assess the rendered images, leading to an underdetermined problem in characterizing the geometry of the back of the head. In the absence of constraints, our model may converge to a suboptimal solution marked by front-back symmetric geometry.








To tackle this issue, we introduce a rear-view depth regularization loss, which is incorporated into the generator's training to prevent it from falling into suboptimal solutions. Specifically, we generate the depth image of $M(0)$'s occiput (the SMPL mesh with neutral body pose) from the camera parameters $c_{back}$, where $\mu = -\frac{1}{2} \pi$ and $\nu = \frac{1}{2} \pi$. We then apply depth image constraints to the image generated by the generator $G$ from $c_{back}$:
\begin{equation}
	\begin{split}
    \label{eqn: rear-view depth regularization loss}
     L_{rear} & =  L_2(m \odot depth_{M(0)}, m \odot depth_{G}). \\ 
     \end{split}
\end{equation}
Here, $depth_{M(0)}$ represents $M(0)$'s depth image, $depth_{G}$ represents the depth image produced by the generator, $m$ represents the valid mask of $M(0)$'s depth image, and $\odot$ denotes element-wise multiplication. 




\subsubsection{Overrall Loss}
Our full objective function is:
\begin{equation}
\begin{split}
    \label{eqn: full objective}
    L =  L_D + L_G 
    \end{split}.
\end{equation}







 \subsection{Training Details}
\label{sec: Training Details}
Our model is trained on 8 NVIDIA A40 GPUs. It takes 7 days to train our full model.
Taking into account the computational cost of the deformation field, we retain only the SMPL faces that fall within the bounding box of the volumetric rendering. 
We set gamma as 1.0, and the resolution of the training dataset is $256^2$.
The body pose predictors $\Gamma_G$ and $\Gamma_D$ are composed of fully-connected layers with leaky ReLU as the activation functions. 

The proposed training strategy for our 3DPortraitGAN is composed of three stages, as shown in Tab. \ref{tab: training_details}. 
%to first warm up our networks, then finetune our model using the rebalanced dataset, and finally increase the neural rendering resolution.


\paragraph{Stage1 - Warm Up} 
    The first stage is the warm-up period, from 0 to 4M images.
    %
    We employ the regularization loss $L_{preg}$ during the initial phase of this stage. 
    Specifically, we employ the $L_{preg}$ regularization loss to the first 0.2M images while linearly decaying $\lambda_{preg}$ to 0 over the subsequent 0.2M images. This helps prevent the coarse body pose from negatively affecting the entire training process. 
    %
    Meanwhile,  we compute $L_{rear}$ regularization loss by directly utilizing $depth_{M(0)}$ and $depth_{G}$ according to Eq. (\ref{eqn: rear-view depth regularization loss}) from 0 to 0.8M images. After generator $G$ has acquired a rudimentary understanding of portrait geometry, we normalize $depth_{M(0)}$ with $depth_{G}$ to match their mean depth values in order to enhance head depth diversity.
    %
    We utilize the swapping regularization method proposed by EG3D \cite{DBLP:conf/cvpr/ChanLCNPMGGTKKW22}. Specifically, we begin by randomly swapping the conditioning pose of $G$'s mapping network with another random pose with 100\% probability, then the swapping probability is linearly decayed to 60\% over the first 1M images. For the remainder of this stage, we maintain a 60\% swapping probability. The neural rendering resolution remains fixed at $64^2$.
     
    
    
\paragraph{Stage2 - Low Neural Rendering Resolution} 
    From 4M to 10M images, we encounter issues with the collapse of $\Gamma_G$ during training. 
    % %
    Therefore, we freeze $\Gamma_G$ while continuing training 3DPortraitGAN for 10M images.
    We randomly swap the conditioning pose of $G$'s mapping network with another random pose with 60\% probability during this stage, and neural rendering resolution remains fixed at $64^2$.
    
    
\paragraph{Stage3 - Increase Neural Rendering Resolution}  
    This stage represents the training from 10M images to 14M images of our full training process.
    %
    In this stage, we gradually increase the neural rendering resolution of 3DPortraitGAN while other training settings are identical to those of Stage 2.
    %
    Specifically, we linearly increase the neural rendering resolution from $64^2$ to $128^2$ from 10M to 11M images, then we retain the neural rendering resolution as $128^2$ until finishing training.
    We randomly swap the conditioning pose of $G$'s mapping network with another random pose with 60\% probability during this stage.
    



    

\section{Experimental prospects for \Afb  and \Rq at ILC250 and ILC500}
\label{sec:results}

Three different scenarios have been studied: reconstruction without TPC kaon ID, a reconstruction using TPC Kaon ID (via \dEdx) for charge measurement as well as adding \dEdx in the flavour tagging and a reconstruction using TPC Kaon ID (via \dNdx) for charge measurement as well as adding \dNdx in the flavour tagging, being the later an estimation as described in section \ref{dNdxsection}. These three scenarios are covered for both 250 and 500 GeV, for the cases of $P_{\mathrm{e^{-}e^{+}}}=(-0.8,+0.3)$ and $P_{\mathrm{e^{-}e^{+}}}=(+0.8,-0.3)$. A comparison for each case using only statistical uncertainties has also been plotted. The results are summarised in Fig. \ref{fig:results}.


% Figure environment removed



\checkednote{\subsection{Definition of a measurement}}
In compressed sensing and coding is typically defined as a measurement of an analog flux through some type of coding projection or in our example a mask. As has been shown by multiple works, this analog model of light does not account for the poisson noise inherent in any real measurements and leads to counter intuitive behavior of coding approaches.

If we instead think of our system as measuring photons through different codes, the code behavior makes intuitive sense. A photon measured through a mask with many open pixels carries less information about the scene than one captured through a raster mask because our measurement is ambiguous regarding which pixel in the mask was the origin of the photon. In a raster mask every photon can be uniquely assigned to one pixel. In essence, more photons do not equal more information.

The implication of this well documented problem become ever more important in the age of low noise and photon counting cameras where Poisson noise dominates all measurements. It is wide reaching since the projection process we study here in a specific coding experiment is part of the design of any camera. In other words: Any camera or vision system has to project data from a high dimensional scene space down into a lower dimensional sensor space where it encounters Poisson noise and then uses those noisy measurements to make inferences about the scene. 

\Xpolish{This paper has shown that the challenges for computational imaging under Poisson noise. Algorithms based on the AGN noise assumption are problematic on the modern sensors. However, if the task requires no reconstruction but direct feature extraction, the Selective Sensing using the Optical Neural Networks model can find the optimal coding methods. We have shown the feasibility of the Selective Sensing via simulations and experiments, and it demonstrated a promising classification performance on the MNIST handwritten number dataset. Also, it is robust in the application scenarios where their noise level is hard to estimate. Furthermore, the Selective Sensing provides an motivation for proposed optical ANNs or ANNs with optical layers which allow us to optimize the coding schemes wherever the Poisson noise happens. }
{Our paper highlights the challenges of computational imaging under Poisson noise and its impact on algorithms based on the AGN noise assumption. We find that for compressible measurements, and especially tasks that involve direct feature extraction instead of signal reconstruction, a Selective Sensing approach using task optimized codes provides a viable coding solution. Through simulations and experiments, we demonstrate the feasibility of Selective Sensing and its promising classification performance on the MNIST handwritten number dataset. It is also robust in application scenarios with difficult-to-estimate noise levels. Our ONN method represents a method that can generate these selective measurements. Furthermore, Selective Sensing motivates the development of optical ANNs or ANNs with optical layers to globally optimize imaging systems.}

\Xpolish{On the other hand, there are some limitations in our project. First, we employed the AGN model and reparameterization trick for the model training, which is only an approximation to the noise at the sensor. Second, our test set in the experiments only contains 10 numbers, which may not be representative enough. There is also an inconsistency as the model was trained by simulated data but tested with experimental data. Last but not the least, the Photon Distribution Factor rescales the masks $\B{M}$, but its value changes during the training and it is not evolved in the back-propagation. In general, the optimization of the ONN model still has some defects and we still need to improve the results by using better optimization methods and more experimental data.}
{Despite the promising results of our project, there are some limitations that must be acknowledged. First, we used a Gaussian noise model with reparameterization to train our model, which is only an approximation of the actual quantization noise at the sensor. Second, our test set consisted of only 10 numbers, which may not provide a comprehensive evaluation of the model's performance. Additionally, we noted an inconsistency in that the model was trained using simulated data but tested with experimental data. Lastly, the Photon Distribution Factor rescales the masks $\B{M}$, but its value changes during training and is not evolved during back-propagation. These limitations highlight the need for further improvements in the optimization of the ONN model, such as using more advanced optimization methods and larger sets of experimental data. Our work highlights the importance of the integration of imaging hardware and signal processing. In single photon accurate imaging systems, comprehensibility and sparsity of the data can be exploited to far greater effect during the measurement, as opposed to post processing.}


\section{Conclusion and Future Work}
\label{sec: Conclusion and Future Work}
This paper explores formal method-based reachability analysis of variable-length time series regression neural networks (NNs) using approximate Star methods in the context of predictive maintenance, which is crucial with the rise of Industry 4.0 and the Internet of Things. The analysis considers sensor noise introduced in the data. Evaluation is conducted on two datasets, employing a unified reachability analysis that handles varying features and variable time sequence lengths while analyzing the output with acceptable upper and lower bounds. Robustness and monotonicity properties are verified for the TEDS dataset. Real-world datasets are used, but further research is needed to establish stronger connections between practical industrial problems and performance metrics. The study opens new avenues for exploring perturbation contributions to the output and extending reachability analysis to 3-dimensional time series data like videos. Future work involves verifying global monotonicity properties as well, and including more predictive maintenance and anomaly detection applications as case studies. \newblue{The study focuses solely on offline data analysis and lacks considerations for real-time stream processing and memory constraints, which present fascinating avenues for future research.}
\paragraph{\textbf{Acknowledgements.}}
The material presented in this paper is based upon work supported by the National Science Foundation (NSF) through grant numbers 1910017, 2028001, 2220418, 2220426, and 2220401, and the Defense Advanced Research Projects Agency (DARPA) under contract number FA8750-18-C-0089 and FA8750-23-C-0518, and the Air Force Office of Scientific Research (AFOSR) under contract number FA9550-22-1-0019 and FA9550-23-1-0135. Any opinions, findings, conclusions, or recommendations expressed in this paper are those of the authors and do not necessarily reflect the views of AFOSR, DARPA, or NSF. We also want to thank our colleagues, Tianshu and Barnie for their valuable feedback.
 


\bibliographystyle{ACM-Reference-Format}
\bibliography{main-bibliography}

\end{document}
