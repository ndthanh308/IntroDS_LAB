\appendix 
\section{ASlib scenarios}\label{sec:App1}
% \subsection{ASlib scenarios}
 In this section we explore 9 ASlib scenarios: ASP-POTASSCO, CSP-MiniZinc-Time-2016, GRAPHS-2015, MAXSAT-PMS-2016,  PROTEUS-2014, SAT11-INDU, SAT12-ALL, BNSL-2016 and SAT18-EXP-ALGO. For each scenario we fit an AIRT model and conduct a smaller analysis compared to the OpenML-Weka example in Section~\ref{subsec:openml}. Using the fitted model, we plot the latent trait curves. Then we compute the strengths and weaknesses of algorithms on the dataset difficulty spectrum for  $\epsilon = 0$  and $\epsilon = 0.05$. By visualizing this spectrum, we see which algorithms have strengths for easy problems and which ones are better suited for difficult problems. Similarly, we see their weaknesses as well. Using 10-fold cross validation, we evaluate airt, topset and Shapley algorithm portfolios and examine the mean performance gap as explained previously.     

% Figure environment removed


\subsubsection{ASP\_POTASSCO}
Figure~\ref{fig:asppoassco} shows the analysis for ASP\_POTASSCO scenario. Algorithm \textit{clasp/2.1.3/h3-n1} is the weakest in the portfolio as we can see from the strengths and weaknesses figure and the latent trait curves. Algorithm \textit{clasp/2.1.3/h1-n1} is better suited for difficult problems as seen by the hump in the latent trait curve around $ \delta \approx  1 $. Many algorithms perform well for very easy problems as seen by the leftmost part of the strengths in the problem difficulty spectrum. {\color{black} A point of interest about these curves is that some curves, including that of  \textit{clasp/2.1.3/h1-n1}, have a turning point around  $ \delta \approx 0.75$ followed by a positive slope, signifying an improvement in performance, culminating at $ \delta \approx 1.25$ before decreasing again. This shows locally anomalous behavior for $\delta$ in that region for certain algorithms.} The cross-validated mean performance gap of different algorithm portfolios show that   for $n \in \{1, \ldots, 5\}$ airt is similar to either Shapley or topset, but for  $n \in \{6, \ldots, 9\}$ airt has a lower mean performance gap. However, the standard errors show that the differences are not significant. 

\subsubsection{CSP\_MiniZinc\_2016}
Figure~\ref{fig:cspminizinc} shows the analysis for CSP\_MiniZinc\_2016. {\color{black} The latent trait curves are spread out well and thus show high variability. This has resulted in a sparse set of strengths and weaknesses. Even though there are many algorithms, only a few exhibit strengths and similarly only a few have weaknesses at other places apart from the rightmost end, which has the most difficult problems.} As seen from the latent trait curves and the strengths and weaknesses figure, algorithm \textit{LCG-Glucose-UC-free} shows continued strength for difficult and semi-difficult problems. Algorithm \textit{MZN/Gurobi-free} is better suited for easy and very difficult problems. The weakest algorithm is \textit{Picat-CP-fd}, which is weak for easy and semi-difficult problems. While many algorithms are good for easy problems, both \textit{LCG-Glucose-UC-free} and \textit{LCG-Glucose-free} displays strengths for a large region of the problem space for $\epsilon = 0.05$.  The cross-validated mean performance gap graphs show that airt and topset behave similarly, while Shapley has higher mean performance gaps initially but converges with airt and topset for higher $n$.


\subsubsection{Graphs\_2015}
Figure~\ref{fig:graphs} shows the analysis for Graphs\_2015. From the latent trait curves and the strengths and weaknesses figure we see that \textit{glasgow2} and  \textit{glasgow3} are suited for a large part of the  problem space. \textit{supplemantallad} is good for easy and very difficult problems and many algorithms have strengths for easy problems. The weakest algorithm is \textit{vf2} as seen from the weaknesses spectrum. {\color{black} For dataset difficulty $\delta \lessapprox 0.5$, all algorithms apart from \textit{vf2} perform well. However, after that point, the algorithms diverge in their performance as seen from the curves.} The cross-validated mean performance gap of different portfolios show that airt has the smallest performance gap for most $n$. 


\subsubsection{MAXSAT-PMS-2016}
Figure~\ref{fig:maxsatpms} shows the analysis for MaxSAT-PMS-2016 scenario. Immediately we see variety in the latent trait curves. Some curves have low performance values for most part of the space, which is different from the other scenarios we examined so far. Some curves have varying behavior with curved sections. In the strengths diagram, we see many algorithms having strengths for easier problems.  {\color{black} Of the algorithms, 15 have strengths for dataset difficulty $\delta \leq 0$ when $\epsilon = 0.05$. These are the easy problems. For the easy problems, any of these algorithms would give good performances. Only 8 algorithms have strengths for $0 \leq \delta \leq 1$ and of these only 5 have strengths for $\delta > 1$.} \textit{LMHS-2016} and \textit{maxhs-b} are better suited for harder problems. In the weaknesses space we see that \textit{CCLS2akms} and \textit{CCEHC2akms} are very weak algorithms. The cross-validated mean performance gap shows that airt performs better compared to the other two portfolios. The topset portfolio has a sudden jump at $n = 6$, possibly due to including a volatile algorithm, which gets mitigated with subsequent algorithm additions to the portfolio. 



% Figure environment removed

\subsubsection{PROTEUS-2014}
The latent curves of PROTUES-2014, shown in Figure~\ref{fig:proteus} have many wiggles. {\color{black} Four curves achieve local minima at dataset difficulty $\delta \approx -0.2$. After that point, their performance increase for some part of the dataset difficulty spectrum, i.e., as the dataset difficulty increases, the performance of these algorithms get better. Thus, these algorithms are locally anomalous. They are not anomalous throughout the spectrum, but they have regions of locally anomalous behaviour.} Algorithms \textit{claspcnf\_support, claspcnf\_direct} and \textit{claspcnf\_directorder} display strengths for a large part of the problem space including difficult problems. Algorithm \textit{gecode} is the weakest algorithm as seen by the latent trait curves and the strengths and weaknesses diagram. The cross-validated mean performance gap curves show that airt performs better than the other two portfolios. The standard errors for both topset and airt are very low making them not clearly visible in the diagram.    


\subsubsection{SAT11-INDU}
Figure~\ref{fig:sat11indu} shows the analysis of  SAT11-INDU. We see that most algorithms have similar-shaped latent trait curves. {\color{black} We do not know if the algorithms were preselected, which might account for this behaviour. The similarity of the curves implies some similarity of performance between the algorithms. } In the strengths diagram many algorithms have strengths for easy and semi-difficult problems. In the weaknesses diagram, we see a curious occurrence: many algorithms display weaknesses in the middle of the spectrum as well as on the difficult end of the spectrum. This is because the curves are packed together for most part of the problem space. Algorithm \textit{glucose\_2} occupies the highest proportion of the latent trait. From the strengths and weaknesses figure we see that \textit{QuteRSat\_2011-05-12\_fixed} is strong for difficult problems. Notably \textit{minisathackcontrasat\_2011-03-02} is weak for easy problems.  The cross-validated performance gap curves show that Shapley performs better than the others, but the standard errors of the 3 portfolios overlap for most values of $n$.  


% Figure environment removed

\subsubsection{ SAT12-ALL}

SAT12-ALL scenario contains SATzilla 2012 competition \citep{Xu2012} results on algorithm performance. Figure \ref{fig:sat12} shows the latent trait curves, strengths and weaknesses and performance comparison of different portfolios. The curves have diverse characteristics: some curves have an initial downward trend showing that they are weak for most part of the space but later trend upward indicating that they perform better for more difficult test instances. {\color{black} Another set of curves give good performances for easy problems with $\delta \lessapprox -1$ and decrease in performance after that.} Algorithms \textit{mphaseSATm} and \textit{mphaseSAT} are strong for a large part of the problem space including difficult instances. Algorithms \textit{spear-sw} and \textit{eagleup} are weak for most parts of the space. The cross-validated mean performance gap curves show that airt has a lower gap compared to the other 2 portfolios. 



\subsubsection{ BNSL-2016}
Figure \ref{fig:bnsl} shows the analysis for BNSL-2016 scenario. {\color{black} Algorithms ilp-141 and ilp-141-nc have similar latent trait curves. Similarly, ilp-162 and ilp-162-nc are also similar. Furthermore, astar-ec and astar-ed3 have similar curves. Lastly, cpbayes and astar-comp have somewhat similar curves.} Algorithms \textit{cpbayes} and \textit{astar-comp} display strengths for easy and very difficult problems while \textit{astar-ec} and \textit{astar-ed3} are weak for most of the problem space. {\color{black} Algorithms ilp-141, ilp-141-nc, ilp-162 and ilp-162-nc have strengths for a large part of the problem space.} Algorithm portfolio comparison shows that airt achieves good performance. 



% Figure environment removed

\subsubsection{ SAT18-EXP-ALGO}
Figure \ref{fig:sat18} shows the analysis for SAT18\_EXP\_ALGO scenario. The latent trait curves are somewhat similar, but not too similar as in SAT11\_INDU. The algorithm \textit{YalSAT}, depicted by a gray shade, is weak for easier instances and strong for difficult instances. Hence is comes up in both strengths and weakness diagrams. Remarkably, the latent trait curve appears at the bottom on the left hand side and bends and ends up at the top at the right-most side. {\color{black} Another upward bend is observed at $\delta \approx -0.5$ by Maple\_CM\_Dist algorithm showing a unique strength of this algorithm.} The airt portfolio  achieves good performance for this scenario. 


% Figure environment removed




