\section{Discussion}\label{sec:discu}
In this section, we discuss possible extensions to improve \ourtool~and hardware fuzzing in general.

\textbf{Dynamically Identifying Optimal Solutions.} 
Though the reset strategy works well in practice, it is a heuristic that modifies the PSO algorithm. 
In the future, we plan to model and analyze the impact of resetting threshold ($\beta$) on the capability of vulnerability detection and design space exploration of \ourtool{}. Moreover, we will also evaluate the performance of other meta-heuristic algorithms, such as simulated annealing~\cite{bertsimas1993simulated} and differential evolution~\cite{price2013differential}, and identify the ideal one for hardware fuzzing.

\textbf{Leveraging Prior Knowledge for Initialization.} 
One key way to improve its performance is by initializing with a good starting point~\cite{simpson2017penalising}. \ourtool{} initializes particles' positions randomly. However, we can use the static schemes of existing fuzzers to initialize the particles. 
For example, the static probabilities from the profiling state of \thehuzz{} can be used as the initial position of particles to select the mutation operators. 