\section{Discussion}\label{sec:discu}
In this section, we discuss possible extensions to improve \ourtool~and hardware fuzzing in general.

\textbf{Dynamically Identifying Optimal Solutions.} Since the optimal solutions (selection probabilities of mutation operators and instruction types for seed generation) change based on the coverage achieved, we develop a reset strategy that guides \ourtool{} to search optimal probabilities for selecting mutation operators and seed generation. 
Though it works well in practice, it is a heuristic that modifies the PSO algorithm. There are other optimization techniques, such as Bayesian optimization~\cite{fine2003coverage} and reinforcement learning~\cite{sutton2018reinforcement}, that intrinsically aim to find optimal sequences of solutions. However, incorporating these techniques in hardware fuzzing is non-trivial and creates critical challenges related to scalability.

\textbf{Leveraging Prior Knowledge for Initialization.} Fuzzing is fundamentally a searching process to find vulnerabilities in target hardware. Therefore, one key way to improve its performance is by initializing with a good starting point~\cite{simpson2017penalising}. \ourtool{} initializes particles' positions randomly. However, we can use the static schemes of existing fuzzers to initialize the particles in \ourtool{}. 
For example, the static probabilities from the profiling state of \thehuzz{} can be used as the initial position of particles to select the mutation operators. 
Moreover, since many processors are designed based on the same ISAs, using tests that trigger vulnerabilities in one processor as seeds may find variants of the vulnerabilities in other processors faster~\cite{ibrahim2022microarchitectural}.

\textbf{Mutation operators} are critical to the performance of a fuzzer. However, existing hardware fuzzers adapt mutation operators from software fuzzers (such as AFL)
instead of investigating mutation operators specialized for hardware~\cite{kande2022thehuzz,rfuzz,chen2023hypfuzz,muduli2020hyperfuzzing,ragab_bugsbunny_2022,canakci2021directfuzz,hur2021difuzzrtl,fuzzhwlikesw}. Therefore, an interesting direction for future work can be to develop hardware-specialized mutation operators and use PSO to select them, to further improve the performance of hardware fuzzers. 