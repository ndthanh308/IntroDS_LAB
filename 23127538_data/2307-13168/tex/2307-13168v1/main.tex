\documentclass{article}
\usepackage{graphicx} % Required for inserting images

\usepackage{arxiv}

\usepackage{amssymb}
\usepackage{amsthm}
\usepackage{amsmath}
\usepackage{dsfont}
\usepackage{xcolor}
\usepackage{booktabs} % nice tabular
\usepackage{graphicx}
\usepackage{appendix}
\usepackage{bm}
\usepackage{algorithm}
\usepackage{algpseudocode}
\usepackage{adjustbox}
\usepackage{caption}
\usepackage{subcaption}
\usepackage{soul}
% \usepackage[top=3.5cm,bottom=3.5cm,left=2.5cm,right=2.5cm]{geometry}

\usepackage{everypage}

% \setlength{\topmargin}{0 mm}
% \setlength{\oddsidemargin}{5 mm}
% \setlength{\evensidemargin}{5 mm}
% \setlength{\textwidth}{150 mm}
% \setlength{\textheight}{210 mm}

\newcommand{\R}{\mathds{R}}
\newcommand{\C}{\mathds{C}}
\newcommand{\cb}{{\boldsymbol{c}}}
\newcommand{\alphab}{{\vec{\alpha}}}

\newcommand{\tc}[1]{\textcolor{red}{#1}}
\newcommand{\tcb}[1]{\textcolor{blue}{#1}}

\usepackage{todonotes}
\newcommand{\ap}[1]{\todo{\textit{AP says:} #1}}
\newcommand{\sg}[1]{\todo{\textit{SG says:} #1}}

\newtheorem{remark}{Remark}

\title{A practical approach to determine minimal quantum gate durations using amplitude-bounded quantum controls}


\date{July 21, 2023}
% \date{}

\author{ Stefanie G{\"u}nther\thanks{Corresponding author}, ~ N.~Anders Petersson \\
	Center for Applied Scientific Computing\\
 Lawrence Livermore National Laboratory, Livermore CA 94550\\
	\texttt{\{guenther5, petersson1\}@llnl.gov} 
}

\renewcommand{\headeright}{}
\renewcommand{\undertitle}{}
% \renewcommand{\shorttitle}{Minimal gate durations}



\begin{document}

\maketitle

\begin{abstract}
We present an iterative scheme to estimate the minimal duration in which a quantum gate can be realized while satisfying hardware constraints on the control pulse amplitudes. The scheme performs multiple numerical optimal control cycles to update the gate duration based on the resulting energy norm of the optimized pulses. We provide multiple numerical examples that each demonstrate fast convergence towards a gate duration that is close to the quantum speed limit, given the control pulse amplitude bound.
\end{abstract}

\keywords{{quantum optimal control \and pulse-level control \and minimal quantum gate duration, amplitude-bounded quantum control}}

\section{Introduction}
Quantum optimal control utilizes numerical optimization tools to design pulses that drive quantum devices. It's potential to improve fundamental operations in Quantum Information Science has by now been demonstrated on various applications, e.g., for quantum state preparation \cite{Rojan2014arbitrary},  applications to quantum error correction \cite{waldherr2014quantum, gaitan2008quantum}, and the realization of logical quantum gate operations \cite{doi:10.1080/09500340802344933, PhysRevLett.89.188301, cho2023direct}, see e.g., \cite{glaser2015training, koch2016controlling} for an overview.
% % For example, pulse-level optimal control can prepare any random unitary operation with high fidelity and shorter durations than the realization with primitive gate sets, \cite{cho2023direct}. 
% Typically, gradient-based techniques are applied, that iteratively update control pulses to minimize the mismatch between a desired quantum operation and the operation realized by the current controls. 
% Various numerical tools to simulate and optimize quantum dynamics have been developed, such as the Python-based quantum toolbox QuTiP~\cite{johansson2012qutip} and Krotov package~\cite{goerz2019krotov}, as well as the High-Performance centric C++ software Quandary \cite{gunther2021quandary}, which is used for the numerical results in this paper. 
In the current Noisy Intermediate-Scale Quantum (NISQ) era, it is desirable to design control pulses with minimal gate duration, such that the quantum operation can be performed before decoherent processes collapse the quantum information into a classical state.
However, finding the minimal duration in which such an operation is realizable is challenging. When control pulse amplitudes are unlimited, various lower bounds known as intrinsic \textit{quantum speed limits} (QSL) have been established for the time-scale over which a quantum system can evolve, see, e.g., the overview in \cite{deffner2017quantum}.
Such speed limits can be computed analytically for simple model problems \cite{hegerfeldt2013driving, khaneja2001timeoptimal, khaneja2002subriemannian}, and can be estimated numerically for more complex systems by performing multiple optimal control cycles, sweeping over a range of gate durations.
Indeed it has been demonstrated on specific examples that such limits align with the shortest durations for which numerical optimal control techniques converge, when control pulse amplitudes are unlimited \cite{hegerfeldt2013driving, caneva2009quantumspeedlimit, ashhab2012speed}. 
In practice, however, control pulses are generally subjected to hardware constraints, most commonly given in terms of a maximum drive strength as defined by hardware waveform generating devices. 
When hardware constraints bound the control pulse amplitude, the time-scale required to realize a quantum operation can be significantly slower than the theoretical (intrinsic) QSL. 
For unitary operations in the closed system setting, 
%\textcolor{blue}{\st{i.e., unitary  gates,}} 
an extrinsic speed limit that takes the maximum control pulse amplitude into account has been derived in \cite{arenz2017roles}. 
While the lower bound matches the minimum gate duration remarkably well for small single-qubit cases, and the extension in \cite{lee2018dependence} to general N-level system demonstrates qualitatively reasonable scaling properties for larger numbers of qubits, the lower bound is not sharp in the sense that the actual gate duration is no longer precisely estimated as system sizes increase. 
Currently, a general technique to find minimal gate durations when control pulse amplitudes are bounded by hardware constraints is not available. 
Instead, it is common practice to determine the minimum gate duration by trial and error through multiple optimal control cycles, each using a different pulse duration and a different initial guess for the control pulse. Henceforth, the latter approach will be referred to as the brute-force method.

% \st{[seifert2022time] presents an approach to automate this process using}
In ~\cite{seifert2022time}, a re-seeding technique was proposed that adjusts gate durations based on previous success or failure of the optimal control iterations, combined with a bisection of the resulting gate durations. 
% Alternatively, the pulse duration can be added as an optimization parameter directly, however, due to the oscillatory nature of the quantum dynamics, such time-minimization often leads to only small updates to the initial guess for the final gate time when gradient-based optimization methods are used.
However, imposing parameter bounds during the optimization updates often leads to non-convex optimization landscapes \cite{ge2022optimization} and slow or stagnant optimization progress, such that optimization cycles need to be performed multiple times for various different initial guesses.

In this paper, we propose an alternative scheme to numerically find minimal time-scales in which a unitary operation can be realized under amplitude-constrained control fields, by re-scaling the gate duration based on the optimized control pulse amplitude. Instead of directly imposing box constraints on the control pulse, we include a penalty term in the objective function that minimizes the control pulse energy during the optimization. After solving the unconstrained and penalized optimization problem for a given gate duration, we re-scale the dynamics to a new duration where control pulses satisfy the amplitude constraints. Those stretched (or squeezed) control pulses then serve as an initial guess for the next optimization cycle for the new gate duration.
We present numerical evidence for system of up to three coupled qudits where this iterative scheme rapidly converges to a final gate duration close to the QSL, yielding control pulses that satisfy hardware-specific amplitude bounds while minimizing the gate time duration. 

While the approach presented here is agnostic to the underlying Hamiltonian model and optimization strategy, in Section \ref{sec:optimalcontrol} we present the system and control Hamiltonian models that are used in our numerical experiments. Section \ref{sec:mintime} describes the proposed iterative scheme for finding minimal gate durations and corresponding amplitude-constrained control pulses. Section \ref{sec:results} presents the numerical results, and conclusions are drawn in Section \ref{sec:conclusion}.

\section{The quantum optimal control problem}\label{sec:optimalcontrol}
Given a unitary target gate $V^{\textrm{target}}$ that represents a logical quantum operation, the goal of quantum optimal control is to design optimal pulses that drive any initial quantum state $\phi(0)$ at time $t=0$ to the unitarily transformed target state $\phi(T)=V^{\textrm{target}}\phi(0)$ at time $T>0$. To that end, gradient-based optimization methods are typically applied to minimize a cost function that measures the distance between the target and the driven evolution in terms of the gate infidelity,
\begin{align}\label{eq:finaltime_cost}
    \min J_{cost}\left(U(T)\right) = 1-\left| \frac{1}{N} \mbox{Tr} \left(U^\dagger(T) V^{\textrm{target}}\right) \right|^2,
\end{align}
where the columns of $U(T)$ describe the evolved quantum state at final time $T$ for a basis of initial states $U(0) = [e_0,\dots, e_{N-1}]$, where $e_i$ are the canonical basis vectors. We here consider closed quantum systems modeled by Schroedinger's equation, such that the evolved dynamics satisfy
\begin{align}\label{eq:mastereq}
 \frac{d U(t)}{dt} = &-iH(t)U(t), \quad 0\leq t \leq T,\quad U(0) = I.
\end{align}
The Hamiltonian,
\begin{align}\label{eq:schroedinger}
    H(t) = H_{sys} + H_c(t), %\sum_{q=1}^Q f_q(t)H_q
\end{align}
decomposes into a time-independent system part, $H_{sys}$, and a time-varying control part, $H_c(t)$. In the laboratory-frame of reference, the control Hamiltonian is given by $\sum_q f_q(t)(a_q + a_q^\dagger)$, where the real-valued control pulse satisfies $f_q(t) = 2\, \mbox{Re}\{c_q(t) e^{i t \omega^{d}} \}$, where $\omega^d$ is the drive frequency. To slow down the time scales of the state dynamics and the control pulse, we apply the rotating wave approximation (RWA) and select the frequency of rotation to equal the drive frequency, $\omega^{rot} = \omega^d$.
In the rotating frame, we consider a general system Hamiltonian modeling $Q\geq 1$ coupled superconducting qudits,
\begin{align}
  H_{sys} = \sum_{q=1}^Q {\left(\omega_q - \omega^{rot}\right)} a_q^\dagger a_q - \frac{\xi_q}{2} a_q^{\dagger}a_q^{\dagger}a_q a_q + \sum_{p>q} J_{pq} \left(a_p^\dagger a_q + a_pa_q^\dagger\right),
\end{align}
where $\omega_q$ denotes the 0-1 transition frequency of qudit $q$, $\xi_q$ is the self-Kerr coefficient, and $J_{pq}$ denotes the dispersive coupling coefficient between qudits $p$ and $q$. Further, $a_q$ ($a_q^\dagger$) denote the lowering (raising) operator for qudit $q$. 
% We note that the above form of the Jaynes-Cummings coupling term assumes that all rotational frequencies are equal; the general form implemented in Quandary includes the time-dependent factors {$\exp(\pm i (\omega^{rot}_p - \omega^{rot}_q) t)$ that pre-multiply the terms {$a_p^\dagger a_q$ and $a_pa_q^\dagger$, respectively}.}
The action of external control pulses on qudit $q$ in the rotating frame is given by the control Hamiltonian
\begin{align}
   H_c(t) = \sum_{q=1}^Q c_q(t) a_q + c_q^*(t) a_q^\dagger.
\end{align} 
where $c_q(t) = p_q(t) + iq_q(t)$ denote the control pulse for qubit $q$ in the rotational frame and $c_q^*(t)$ denotes its complex conjugate. 
%, and corresponding control Hamiltonian $H_q= a_q + a_q^\dagger$.
For notational simplicity, we now drop the index $q$ denoting each subsystem (qudit), noting that it is straight forward to apply the proposed method to systems with multiple qudits.

To parameterize the control pulse $c(t)$ in the rotating frame, we here choose B-spline basis functions:
\begin{align}\label{eq:bsplinebasis}
  c(t) &= \sum_{s=1}^{N_s}\alpha_{s} B_s(t),
\end{align}
where $\alpha_s = \alpha_s^{real} + i\alpha_s^{imag} \in \C$ are the control parameters that are being optimized for; $B_s(t)$ are fixed, piece-wise quadratic B-spline basis functions \cite{Unser97} with local support in time.
% , giving a total number of $2N_sQ$ real-valued control parameters ${\color{red}\vec{\alpha}}\in \R^{2N_sQ}$. 
In the following we will refer to the set of control parameters $\alphab := \{\alpha_s^{real}, \alpha_s^{imag}\}_{s=1}^{N_s}$ as the control vector. It determines the control pulse $c = c(t;\alphab)$ in the control Hamiltonian and hence the solution operator $U = U(t;\alphab)$ of Schr\"odinger's equation \eqref{eq:mastereq}.\footnote{In the following, the dependence on $\alphab$ will be suppressed for improved readability.}
Parameterizing the control functions using B-splines provides a compact alternative to discretizing the control functions on the same time step as the underlying dynamical system. Since the number of control parameters is independent of the number of time steps for solving Schroedinger's equation, the resulting number of optimization parameters can be significantly smaller. 
Further, the resulting control pulses can be readily applied to the control hardware, without the need for smoothing or filtering a stair step control pulse or a bang-bang ansatz. As we shall see in the following section, parameterizing the controls via basis functions further enables us to readily stretch or compress a control pulse when its duration is changed. 
% In contrast to other control parameterizations using basis functions (e.g. Fourier modes or Legendre polynomials, see e.g. \cite{caneva2011chopped}), the B-spline wavelets are local in time. Hence, each control parameter only influences the envelope function locally. 



%%%%%%%%%%%%%%%%%%%%%%%%%%%%%%%%%%%%%%%%%%%%%%%%%%%%%%%%%
\section{Time-scaling iteration to find minimal gate duration under pulse amplitude constraints}\label{sec:mintime}

When optimal controls are applied in practice, the pulses are typically subject to amplitude bounds as specified by the pulse-generating hardware, such as an Arbitrary Waveform Generator (AWG). Typically those constraints are given in terms of a maximum drive strength $b_{max}$ such that optimized pulses need to satisfy
\begin{align}
    \max_{t\in[0,T]} |c(t)| \leq b_{max}.
\end{align}
% \begin{remark}
% It is vital to carefully calibrate control pulses within the allowed amplitude range to ensure the proper functioning and reliability of the quantum processor. 
While the amplitude bound varies depending on the specific AWG, we here consider a maximum amplitude bound of $b_{max}{/2\pi} = 40$ MHz for the numerical results presented below.\footnote{
    Because we have scaled Schr\"odinger's equation to make $\hbar = 1$, the unit of energy becomes angular frequency. As a result, the unit of the control pulse amplitude $|c(t)|$ is also angular frequency.}
% \end{remark} 
Standard optimal control approaches incorporate such constraints into the optimization procedure by imposing box-constraints on the control parameter vector, e.g., by projecting the gradient-based update steps onto the given bounds, or by using interior point or barrier methods that augment the objective function and enforce the constraint though penalty terms \cite{nocedal2006numerical}. However, when control parameters hit the bounds, optimization convergence can deteriorate due to non-convex optimization landscapes \cite{russell2016quantum}. As a result, the constrained optimization problem is often much harder to solve than the unconstrained one. In practice, it is often required to repeat multiple optimization cycles, each for a different randomized initial guesses of the control parameters and different pulse durations, to increase the chances of finding a global optima.

% Given the constrained optimization problem, a minimal gate duration $T$ can then be found be either adding $T$ as an optimization parameter directly, or by performing multiple constrained control cycles sweeping over a range of final gates times. 

% Time-optimal control refers the task of designing control pulses that realize $V^{target}$ in the shortest time possible. 
In this section, we propose an iterative scheme to automatically adjust the gate duration based on the energy norm of the optimized control pulse.
% solving the unconstrained optimization problem. 
Instead of explicitly imposing box-constraints on the control parameter vector during the optimization procedure, we add a penalty term to the objective function that aims to minimize the control pulse energy. The iterative scheme performs a sequence of such unconstrained optimal control iterations, each using an updated gate duration that is selected based on the control pulse energy in the previous iteration. 

In each iteration $k$ of the proposed scheme, we solve the following augmented and unconstrained optimal control problem for a given gate duration $T^k$, 
\begin{align}\label{eq:optimproblem}
    % \min_{c(t)} & \quad J_{cost}\left(U(T^k)\right) + \gamma J_{energy}(c(t)).
    \min_{\alphab} & \quad J_{\textrm{infid}}\left(U(T^k;\alphab)\right) + \gamma J_{\textrm{energy}}(c(t;\alphab)).
\end{align}
Here, $\gamma>0$ is a parameter that weights the contribution of the control energy term, \begin{align}\label{eq:energynorm}
    J_{\textrm{energy}}(c(t{;\alphab})) = \frac{1}{T^k}\int_0^{T^k} |c(t;\alphab)|^2 \, dt,
\end{align}
relative to the infidelity, $J_{\textrm{infid}}$, in the objective function.

Performing an optimization cycle that minimizes the weighted sum of the infidelity at the final time \eqref{eq:finaltime_cost} and the control pulse energy \eqref{eq:energynorm} yields control pulses with the smallest energy norm possible, while realizing the given target unitary at time $T^k$. We can assume that a gradient-based optimal control technique can solve this unconstrained optimization problem, given that the weight $\gamma$ is chosen appropriately. However, since box-constraints are not imposed during the optimization, the resulting control pulses, while being minimal in terms of their energy, might  exceed the given hardware bounds $b_{max}$.
Figure \ref{fig:H4_nobounds_dmax} demonstrates the relation between the pulse duration $T$ and the resulting maximum control amplitude
\begin{align}
    c_{max} = \max_{t\in [0,T]} |c_*(t)|,
\end{align}
for optimized unconstrained control pulses $c_*(t)$, on a 4-level qudit QFT gate (problem specifications are given in Section \ref{sec:results}). {We observe that the fidelity remains close to 100\% for durations $T\geq T_{\textrm{QSL}}$, for $T_{\textrm{QSL}} \approx 20$ ns, which we interpret as the quantum speed limit for this setup. More importantly, we observe that the maximum control amplitude scales as $1/T$.}

% AP read this far on 7/21/2023

% Figure environment removed

%%%%%%%%%%%%%%%%%%%
% AP had red comments up to here.
%%%%%%%%%%%%%%%%%%%

We make use of the scaling $c_{max}\sim 1/T$ to find a minimal gate duration for given amplitude constraints by scaling the current gate duration $T^k$ by the ratio between the current maximum control amplitude resulting from solving the unconstrained optimization problem \eqref{eq:optimproblem}, and the given hardware constraint $b_{max}$, i.e., we propose the update scheme 
\begin{align}\label{eq:scaling}
    T^{k+1} = sT^k, \quad \text{where}\quad s = \frac{c_{max}}{b_{max}}.
\end{align}
Consider the scaled time variable 
\begin{align}
\tau(t) = {st} \quad \Rightarrow \quad \tau \in [0,T^{k+1}] \quad \text{for} \quad t\in [0,T^k],
\end{align}
and corresponding scaled evolution $\tilde U(\tau) := U(\tau/s) = U(t)$. The scaled unitary $\tilde U(\tau)$ satisfies the scaled Schr\"odinger equation
\begin{align}
    \dot{\tilde U}(\tau) &= \frac{d U(\tau/s)}{d \tau} = \frac{1}{s}\dot U(\tau/s) = -\frac{i}{s}H(\tau/s)\tilde U(\tau) 
   \quad \text{for} \quad \tau\in (0,T^{k+1}),
\end{align}
and, hence, $\tilde U(\tau)$ evolves under the scaled Hamiltonian
\begin{align}
    \tilde H(\tau) := \frac{1}{s}H\left(\tau/s\right) = \frac{1}{s}H_{sys} + \tilde c(\tau)  a + \tilde c(\tau)^* a^\dagger
\end{align}
with the control pulse
\begin{align}
    \tilde c(\tau) := \frac{1}{s} c(\tau/s) 
\end{align} 
The scaled pulse stretches (if $s>1$) or compresses (if $s<1$) the original pulse into the time domain $[0,T^{k+1}]$ while preserving its integral,
\begin{equation}
    \int_{0}^{T^k} |c(t)|\, \mathrm{d}t = \int_0^{T^{k+1}} |\tilde{c}(\tau) |\, \mathrm{d}\tau. 
\end{equation}
Preserving the control pulse integral can be further motivated by solving the control problem analytically for a model problem, see Appendix~\ref{app:displacement}. By parameterizing the control functions in terms of  B-spline basis functions as in \eqref{eq:bsplinebasis}, the re-scaled controls can be easily computed by scaling the parameter vector $\tilde {\alpha} = s^{-1}{\alpha}$, and re-evaluating the B-spline basis functions at $\tau/s$, hence stretching or compressing their envelops, see Figure \ref{fig:rescale}.
%exemplifies the control pulse scaling to new time domains. 
% Figure environment removed
By design (choice of the scaling parameter $s$ in \eqref{eq:scaling}), the scaled control pulses satisfy the hardware bounds exactly with 
\begin{align}
    \max_{\tau \in [0,T^{k+1}]} |\tilde c(\tau)| = b_{max}
\end{align}
Further, if the dynamics had obeyed the scaled system Hamiltonian $\frac{1}{s}H_{sys}$, the scaled controls would have realized the target gate at time $T^{k+1}$.
However, because the actual system Hamiltonian is $H_{sys}$, we only use the scaled control pulse as an initial guess to solve the optimization problem \eqref{eq:optimproblem} for the updated gate duration, $T^{k+1}$. This procedure is repeated until the scaling factor remains constant at $s=1$, indicating that the minimal gate duration $T^*$ has been found and the corresponding controls pulse $c_*(t)$ satisfies the hardware bounds $|c_*(t)|\leq b_{max}$. Note that as $s\to 1$, the scaled control pulses provide better and better initial guesses for solving the optimization problem. As a result, fewer and fewer optimization iterations are required.

In the proposed time-scaling scheme, the gate duration increases whenever the control pulse exceeds the amplitude bound, and decreases whenever the pulse is below the same bound. However, many iterations of the time-scaling scheme would be needed to precisely determine the minimal gate duration that meets the amplitude bound with equality. To obtain a practically useful stopping criteria, we therefore introduce a \textit{range} of acceptable maximum control pulse amplitudes $[b_{max}-\delta_b, b_{max}]$ for $\delta_b >0$.
The time-scaling iteration then terminates if $c_{max}\in [b_{max}-\delta_b, b_{max}]$; otherwise the gate duration and control coefficients are re-scaled as described above. The resulting time-scaling iteration is summarized in Algorithm~\ref{alg:mintime}. 
%
\begin{algorithm}[htb]
\caption{Minimizing the gate duration}\label{alg:mintime}
\begin{algorithmic}[1]
\Require Specify the range of acceptable maximum control amplitudes: $[b_{max} - \delta_b, b_{max}]$.
\Require Pick an initial gate duration $T^0>0$ and select an initial guess for the control vector $\vec{\alpha}^0$.
\For{$k=0,1,2,\dots$}
\State Starting from the initial guess $\vec{\alpha}_k$, solve the unconstrained optimization problem 
\begin{align*}
   \vec{\alpha}_* = \arg\min_{\vec{\alpha}} J_{\mathrm{infid}}\left(U(T^k;\vec{\alpha})\right) + \gamma J_{\mathrm{energy}}\left(c(t; \vec{\alpha})\right)
\end{align*}
   where $U(T^k;\vec{\alpha})$ solves $\dot U(t;\vec{\alpha}) = -iH(t;\vec{\alpha})U(t;\vec{\alpha})$ with $U(0;\vec{\alpha})=I$.
\If{$\max_{t\in [0,T^k]} \,|c(t, \vec{\alpha}_*)| \in [b_{max} - \delta_b, b_{max}]$}
    \State Success!
    \State \Return $T^k, \vec{\alpha}_*$
\Else 
    %\State Evaluate the scaling factor $s = \max_t|d(t, \alpha_*)|/b_{max}$
    \State Update the gate duration and the initial guess for next control vector: 
        \begin{align*}
            T_{k+1} = sT_k, \quad \text{and} \quad  \vec{\alpha}_{k+1} = \frac{1}{s}\vec{\alpha}_*, \quad
            \text{where} \quad s = \frac{\max_{t\in[0,T_k]}|c(t; \vec{\alpha}_*)|}{b_{max}}
        \end{align*}  
\EndIf
\EndFor
\end{algorithmic}
\end{algorithm}


%%%%%%%%%%%%%%%%%%%%%%%%%%%%%%%%%%%%%%%%%%%%%%%%%%%%
\section{Numerical results} \label{sec:results}
In this section, we verify on multiple test cases that the iterative time-scaling scheme yields gate durations that are close to the smallest possible time in which the target gate can be realized, when hardware constraints are in place on  the control pulse amplitude. We here choose a maximum allowable amplitude bound of $b_{max}{/2\pi} = 40$ MHz in the rotating frame, and accept the resulting control pulses (and final times) whenever the maximum control pulse amplitudes are within the range of $[35,40]$ MHz.
The test cases are described in Table \ref{tab:testcases}, consisting of two single-qudit cases with $3$ and $4$ energy levels, one two-qubit case with $2$ energy levels per qubit, and two three-qubit cases with 2 energy levels each. The qubits are dispersively coupled to the nearest neighbor(s) in a one-dimensional chain with coupling strength $J=5$ MHz. System frequencies for each test case are shown in Table \ref{tab:systemparams}. 
\begin{table}[htb]
    \centering
 \begin{tabular}{@ { } l l l @ { }}
   \toprule
      Name  &   System &  Target gate \\
   \midrule
      QFT$_4$ & one qudit, 4 levels   &  $QFT_d |k\rangle = \frac{1}{\sqrt{d}} \sum_{j=0}^{d-1} e^{ i2\pi kj/d} |j\rangle $, $k=0,\ldots,d-1$       \vspace{2mm}\\
   % \midrule
      SWAP02 & one qudit, 3 levels & $\begin{pmatrix} 
          0 & 0 & 1 \\  
          0 & 1 & 0 \\  
          1 & 0 & 0 \\  
        \end{pmatrix} \in \R^{3\times 3}$ \vspace{2mm} \\
      CNOT & \begin{tabular}{@{}l@{}}two qubits\\ coupling $1\leftrightarrow 2$ at 5MHz \end{tabular} & $\begin{pmatrix} 
          1 & 0 & 0 & 0 \\  
          0 & 1 & 0 & 0 \\  
          0 & 0 & 0 & 1 \\  
          0 & 0 & 1 & 0 \  
        \end{pmatrix} \in \R^{4\times 4}$ \vspace{2mm} \\
      CCNOT &  \begin{tabular}{@{}l@{}}three-qubit chain\\ coupling $1\leftrightarrow 2$ at 5MHz \\ coupling $2\leftrightarrow 3$ at 5MHz\end{tabular}  &  $\begin{pmatrix} 
          1  \\  
            & \ddots   \\  
            &        & 1  \\  
            &        &   & 0 & 1 \\  
            &        &   & 1 & 0 
        \end{pmatrix} \in \R^{8\times 8}$ \vspace{2mm}\\
      SWAP chain& \begin{tabular}{@{}l@{}}three-qubits chain\\ coupling $1\leftrightarrow 2$ at 5MHz \\ coupling $2\leftrightarrow 3$ at 5MHz \end{tabular} & $\begin{pmatrix} 
          1  \\  
            & 0 &   &   & 1 \\  
            &   & 1 &   \\  
            &   &   & 0 &  & & 1 \\  
            & 1  &  &   & 0 \\  
            &    &  &   &   & 1 \\  
            &    &  & 1 &   &   & 0 \\  
            &    &  &   &   &   &  & 1 \\  
        \end{pmatrix}  \in \R^{8\times 8} $ \\
   \bottomrule
  \end{tabular} 
  \vspace{2mm}
  \caption{Test case specifications and target gates.}
  \label{tab:testcases}
\end{table}
%
\begin{table}[htb]
\begin{adjustbox}{max width=1.1\textwidth,center}
    \centering
 \begin{tabular}{@ { } l  l | l l l l l @ { }}
   \toprule
      Name  & $dim(H)$ & $\omega_q/2\pi$ & $\xi_q/2\pi$ & $J_{pq}/2\pi$ & $\omega^{rot}/2\pi$  & $\Delta_{B}$ [ns] \\
   \midrule
      QFT$_4$ & $d=4$ & 4.914 & 0.33 & - & 4.584 & 0.3 \\
      SWAP02 & $d=3$ & 5.12 & 0.34 & - & 4.78 & 0.3\\
      CNOT & $d=2\times 2$ & 5.12, 5.06 & 0.34, 0.34 & 0.005 & 5.09 & 1.65 \\
      CCNOT & $d=2\times 2\times 2$ & 5.18, 5.12, 5.06 & 0.34, 0.34, 0.34 & 0.005, 0.0, 0.005 & 5.12 & 1.65 \\
      SWAP chain& $d=2\times 2\times 2$ & 5.18, 5.12, 5.06 & 0.34, 0.34, 0.34 & 0.005, 0.0, 0.005 & 5.12 & 1.65\\
   \bottomrule
  \end{tabular} 
 \end{adjustbox}
 \vspace{2mm}
  \caption{System parameters.}
  \label{tab:systemparams}
\end{table}


We solve the inner optimization problem in Algorithm \ref{alg:mintime} using L-BFGS as implemented in the quantum control software Quandary \cite{gunther2021quandary}. 
In order to ensure that the inner optimization cycle generates control pulses with minimal energy norm, we choose the penalty coefficient to be $\gamma = 1.0$, and stop the optimization iterations if the norm of the gradient drops below a threshold of $10^{-5}$. In order to ensure that an optimal point with minimal energy norm has been reached, we note that it is important to base the stopping criterion on the norm of the gradient, rather than on a small gate infidelity. To stabilize the inner optimization convergence, we further add a Tikhonov regularization term with parameter $\gamma_{1} = 10^{-2}$, and penalize the second derivative of the state population with parameter $\gamma_{2} = 10^{-2}$; as described in the user's guide to Quandary~\cite{gunther2021quandary}. During the initialization phase of Algorithm \ref{alg:mintime} ($k=0$), we assign the elements of the initial control vector from a uniform probability distribution $\vec{\alpha}_0\sim {\cal U}(-0.9 b_{max},0.9 b_{max})$. 


For each test case, we deploy Algorithm \ref{alg:mintime} for a wide range of initial gate durations, $T^0$. Figure \ref{fig:mintime_results} shows the convergence history of the intermediate gate durations $T^k$ (left column), as well as the corresponding maximum control pulse amplitudes $d_{max}$ (right column).
%for each outer iteration of Algorithm \ref{alg:mintime}. 
Across all test cases, we observe that an optimal gate duration can be found within very few iterations (often only two, but never more than 8 iterations), meaning that only a small number of optimal control cycles are required to determine the minimal gate duration, for a given bound on the control amplitude. This provides a clear advantage over the standard (brute-force) approach of sweeping over a wide range of gate durations, which can easily require tens to hundreds of optimal control cycles to be completed. 
The resulting final gate times at the last cycle of the time-scaling scheme are indicated by gray bands in Figure~\ref{fig:mintime_results}, and are also provided in Table~\ref{tab:mintime_vs_fromscratch}. The width of these bands is determined from the user-prescribed range of allowable maximum amplitudes $[b_{max}-\delta_b, b_{max}]$.

\AddThispageHook{\thispagestyle{empty}} % Remove page number and header 
% Figure environment removed
% }

To evaluate how well the iterative time-scaling scheme recovers the actual minimal gate duration for a given amplitude bound, 
we compare the resulting gate durations with a brute-force approach that performs a sweep of optimizations for a wide range of (fixed) gate durations, while enforcing box constraints on the control vector using a projected L-BFGS method. Figure \ref{fig:optim_from_scratch} shows statistics of optimized gate fidelities, each gathered over 10 different optimizations starting from random initial control vectors. Here, variations in the resulting gate fidelities can be attributed to deterioration of the optimization progress, related to the imposed box-constraints. Nevertheless, we compare the resulting minimal gate durations as observed in Figure \ref{fig:optim_from_scratch} with those produced by Algorithm \ref{alg:mintime} as in Figure \ref{fig:mintime_results}.  Table \ref{tab:mintime_vs_fromscratch} summarizes the resulting gate durations, demonstrating that the time-scaling scheme converges to final gate durations that are remarkably close to those obtained with the brute force method. The range of resulting gate durations as produced by Algorithm \ref{alg:mintime} is a direct consequence of the stopping criteria based on a prescribed range of allowed amplitudes $[b_{max} - \delta_b, b_{max}]$. The range in gate duration hence depends on the user-controlled parameter $\delta_b$. In the last column, we compare the shortest gate durations with the ones obtained from the brute-force optimization approach. We observe a relative difference \st{of} below $7\%$ for all test cases.  

% Figure environment removed

\begin{table}[htb]
    \centering
 \begin{tabular}{@ { } l | l l | c @ { }}
   \toprule
      Case  & Algorithm~\ref{alg:mintime} & Brute force & Smallest rel.~diff.\\
   \midrule
      QFT$_4$ & 19ns to 23ns & 18ns to 21ns & 5\%\\
      SWAP02 & 18ns to 23ns & 18ns & 0\%\\
      CNOT & 68ns to 78ns & 70ns & -2\% \\
      CCNOT & 200ns to 225ns & 190ns & 5\% \\
      SWAP chain & 205ns to 239ns & 190ns & 7\% \\
   \bottomrule
  \end{tabular}
  \vspace{2mm}
  \caption{Minimal gate durations with gate fidelity $\geq 99.9\%$, as achieved from the time-scaling scheme in Algorithm \ref{alg:mintime} (Figure \ref{fig:mintime_results}), and brute-force optimization with box constraints on the control vector (Figure \ref{fig:optim_from_scratch}).}
  %sweeping over a range of gate durations and multiple initial parameter guesses (reference).}
  \label{tab:mintime_vs_fromscratch}
\end{table}


\section{Conclusion}\label{sec:conclusion}
This paper presents a practical scheme for approximating the minimal duration for realizing a unitary gate while satisfying given amplitude bounds on the control pulses. The scheme performs a sequence of unconstrained optimal control cycles, each minimizing the gate fidelity alongside an additional penalty term for the control pulse amplitudes. After each cycle, the gate duration is scaled to a new value for which the scaled controls satisfy the amplitude bounds. We provide numerical evidence of convergence to a final gate duration that is close to the shortest achievable duration obtained from sweeping over a large number of gate durations. Updating the gate duration based on the previous control amplitudes converges to the minimal gate duration in a few optimization cycles, drastically reducing the computational costs for practical usage of time-optimal quantum control, which often requires tens to hundreds of optimization cycles to sweep over a range of gate durations in a trial and error fashion.
We demonstrate that the proposed scheme converges for a wide range of initial gate durations, indicating that almost no prior knowledge of the minimal gate duration is required.
The proposed technique is straightforward to be implemented on top of an existing quantum optimal control code, where we note that it is beneficial to parameterize the control pulses by basis functions such that the pulse amplitudes can easily be adjusted. Further, the proposed method is agnostic to the underlying system and control Hamiltonian models, as well as the target unitary gate operation, making the time-scaling iteration an easy to implement and practically useful scheme for reducing the durations of quantum gate operations.

As illustrated in Figure \ref{fig:H4_nobounds_dmax}, the control pulse amplitude can be a non-monotonic function of the gate duration. We conjecture that this phenomena is the underlying cause of the non-monotonic convergence (over- and under-shooting the gate duration) of our algorithm, and we conjecture that convergence may be further improved by damping the updates of the duration. This will be subject of further investigations.


% \textit{It is also important to realize that the speed limit was obtained for implementing a unitary transformation in a closed quantum system. It is independent on whether pure or mixed initial states, such as thermal states, are considered. However, if one operates on a time scale at which the interactions with the surrounding environment cannot be neglected anymore, decoherence and thus the temperature of the environment play an important role. The development of speed limits to implement unitary gates in such open quantum systems, for which the dynamics is typically no longer unitary, is a challenging problem that goes beyond the scope of this study. }

%%%%%%%%%%%%%%%%%%%%%%%%%%%%%%%%%%%%%%%%%%%%%%%%%%%%
\section*{Acknowledgments}
The authors would like to acknowledge Dr.~Jonathan Dubois at Lawrence Livermore National Laboratory, CA, who provided expertise and partial funding resources during the preparation of this paper.
% \section*{Auspices}


This work was performed under the auspices of the U.S. Department of Energy by Lawrence Livermore National Laboratory under Contract DE-AC52-07NA27344. LLNL-JRNL-851967.



\appendix

%%%%%%%%%%%%%%%%%%%%%%%%%%%%%%%%%%%%%%%%%%%%%%%%%%%%
% \section{A control pulse acting on a single qubit} \label{app:displacement}
%\section{\textcolor{blue}{Two-level model problem demonstrates invariance of the cost function with respect to the control pulse integral}}

\section{The infidelity is invariant to the control pulse integral in a qubit model problem}

\label{app:displacement}

To gain further insight into the essential properties of a control pulse, we
consider the model problem of a driven two-level system {in the rotating frame}, 
\begin{equation}
    H(t) = c(t) a^\dagger + c^*(t) a,\quad c(t): \mathbb{R} \mapsto \mathbb{C}.
\end{equation}
We consider the displacement transformation~\cite{GerryKnight-05},
\begin{equation}
    U(t) = D(\beta(t)) \widetilde{U}(t),\quad \beta(t):\,\mathbb{R}\mapsto \mathbb{C},
\end{equation}
where the unitary transformation operator is
\begin{equation}
    D(\beta) = \exp(\beta a^\dagger - \beta^* a),\quad D(0) = I.
\end{equation}
Thus, by taking $\beta(0)=0$, we have $U(0) = \widetilde{U}(0)$ and the transformed unitary solution operator satisfies
\begin{align}
    \dot{\widetilde{U}}(t) &= -i \widetilde{H}(t)\widetilde{U}(t),\quad \widetilde{U}(0) = I,\\
    \widetilde{H}(t) &= \left( c(t) - i \frac{d\beta(t)}{dt}\right)a^\dagger + \left( c^*(t) + i \frac{d\beta^*(t)}{dt}\right)a,
\end{align}
where we have omitted terms of the form $\delta(t)I$ in $\widetilde{H}(t)$ because they only change the global phase in $\widetilde{U}$. We conclude that $\widetilde{U}(t) = I$, for all $t\geq 0$, if $\beta(t)$ satisfies the ordinary differential equation
\begin{equation}
    \frac{d\beta(t)}{dt} = - ic(t),\quad \beta(0)=0,\quad\Leftrightarrow\quad 
    \beta(t) = - i \int_0^t c(\tau) \, d\tau.
\end{equation}
In this case, we have $U(T) = D(\beta(T))\widetilde{U}(T) = D(\beta(T))$, {and} the infidelity at time $t=T$ satisfies 
\begin{equation}
    J_{\mathrm{infid}} = 1 - \left| \frac{1}{N} \mbox{Tr}\left( V_{\textrm{target}}^\dagger D(\beta(T))\right)
    \right|^2.
\end{equation}
% \textcolor{blue}{which only depends on the control pulse \textit{integral}, rather than its actual shape.} 
{For a given duration $T$, we note that the infidelity only depends on the integral $\beta(T) = -i\int_0^T c(\tau) \, d\tau$. As a result, any control function $\widetilde{c}(t)$ with the same integral over time yields the same infidelity.}

If we scale time, $t' = s t$, and define $\beta'(t') = \beta(t)$, the infidelity with respect to $V_{\textrm{target}}$ becomes invariant to the scaling as long as $\beta'(T') = \beta(T)$. The scaling implies
\begin{equation}
    \beta'(T') = -i\int_0^{T'} c'(\tau') \, d\tau' 
    = -i\int_0^T c'(s \tau) s\, d\tau.
\end{equation}
Hence, $\beta'(s T) = \beta(T)$ if the scaled control pulse satisfies
\begin{equation}
    c'(s t) = \frac{c(t)}{s}.
\end{equation}
For example, if $s>1$ the duration of the control pulse is increased and its amplitude is decreased, such that the time integral of the control pulse remains unchanged. 


\bibliographystyle{plain}
\bibliography{mybib.bib}

\end{document}
