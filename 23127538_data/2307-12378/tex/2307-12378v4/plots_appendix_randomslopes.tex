\label{sec.generalized_results}
For the general setup, we consider the covariates to be the union of item-based features and client-based features, i.e., \( x_{ij} = (1, v_j, u_i) \). We include random effects for clients on item-based features and vice versa. Therefore, the model is given by
\[
y_{ij} = x_{ij}^\top \beta + v_j^\top a_i + u_i^\top b_j + \varepsilon_{ij},
\]
where \( a_i \overset{\text{i.i.d.}}{\sim} \mathcal{N}_3(0, \Sigma_a) \), \( b_j \overset{\text{i.i.d.}}{\sim} \mathcal{N}_3(0, \Sigma_b) \), and \( \varepsilon_{ij} \overset{\text{i.i.d.}}{\sim} \mathcal{N}(0, \sigma^2_e) \), with
\[
\Sigma_a = \begin{bmatrix}
1 & 0.2 & 0.2\\
0.2 & 1 & 0.2\\
0.2 & 0.2 & 1
\end{bmatrix}, \quad
\Sigma_b = \begin{bmatrix}
1 & 0.2 & 0.2\\
0.2 & 1 & 0.2\\
0.2 & 0.2 & 1
\end{bmatrix}, \quad \text{and} \quad \sigma^2_e = 1.
\]

We chose \( \beta = (0.1, 0.2, 0.3, 0.4, 0.5) \), and generated \( (v_j, u_i) \overset{\text{i.i.d.}}{\sim} \mathcal{N}_4(0, I) \).

In Figures~\ref{fig:beta_and_sigma2e_general_setup} and~\ref{fig:sigma_ab_general_setup}, we present the consistency of our algorithm in estimating the model parameters under the general setup. The results are similar to those presented earlier, i.e., the accuracy of our proposed algorithm approaches that of the maximum likelihood estimates as \( N \) increases. In Figure~\ref{fig:estimation_time_general_setup}, we compare the estimation time of our proposed algorithm with existing approach in the general setup; once again, the conclusions are consistent with those reported previously.

% Figure environment removed
% Figure environment removed
% Figure environment removed