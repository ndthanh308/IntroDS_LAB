\usepackage[utf8]{inputenc}
\usepackage{amssymb}
\usepackage{graphicx,float}
\usepackage{paralist}
\usepackage{hyperref}
\usepackage{caption,subcaption}
\usepackage{siunitx,tabularx}


\usepackage{dcolumn,booktabs,paralist,float}
\usepackage{appendix}

\usepackage{natbib}
\usepackage{amsthm}
\usepackage{mathtools} %do robienia mini macierzy
%           linki w spisie tresci
\usepackage[customcolors]{hf-tikz} % do kolorowych macierzy
\usetikzlibrary{patterns}
\usetikzlibrary{matrix,decorations.pathreplacing}
\usepackage{csvsimple} 
\usepackage{multirow}
\usepackage{thmtools, thm-restate}
\newcommand\independent{\protect\mathpalette{\protect\independenT}{\perp}}
\def\independenT#1#2{\mathrel{\rlap{$#1#2$}\mkern2mu{#1#2}}}
\usepackage{graphicx}% http://ctan.org/pkg/graphicx
\usepackage{amsmath}
\usepackage{hyperref}
\DeclareMathOperator*{\argmax}{arg\,max}
\DeclareMathOperator*{\argmin}{arg\,min}
%http://ctan.org/pkg/amsmath
\usepackage{centernot}
\usepackage{caption}
\usepackage{subcaption}
\usepackage[english]{babel}
\usepackage{algorithm}
\usepackage{algpseudocode}
\usepackage{mathtools}

\newcommand{\expect}{\operatorname{E}\expectarg}
\DeclarePairedDelimiterX{\expectarg}[1]{[}{]}{%
  \ifnum\currentgrouptype=16 \else\begingroup\fi
  \activatebar#1
  \ifnum\currentgrouptype=16 \else\endgroup\fi
}

\newcommand{\innermid}{\nonscript\;\delimsize\vert\nonscript\;}
\newcommand{\activatebar}{%
  \begingroup\lccode`\~=`\|
  \lowercase{\endgroup\let~}\innermid 
  \mathcode`|=\string"8000
}
\newcommand{\sumdot}{\text{\tiny$\bullet$}}
\newcommand{\nid}{N_{i\sumdot}}
\newcommand{\ndj}{N_{\sumdot j}}
\newcommand{\Ex}{\mathbb{E}}
\newcommand{\Prob}{\mathbb{P}}
\newcommand{\ols}{\mathrm{OLS}}
\newcommand{\gls}{\mathrm{GLS}}
\newcommand{\wh}{\widehat}
\newcommand{\cov}{\mathrm{Cov}}
\DeclareMathOperator{\tr}{tr}
\newtheorem{theorem}{Theorem}
\newtheorem{lemma}{Lemma}
\newtheorem{solution}{Proof}
\usepackage{relsize}
\makeatletter
\newcommand{\distas}[1]{\mathbin{\overset{#1}{\kern\z@\sim}}}%
\newsavebox{\mybox}\newsavebox{\mysim}
\setcounter{MaxMatrixCols}{20}
\newcommand{\distras}[1]{%
  \savebox{\mybox}{\hbox{\kern1pt$\scriptstyle#1$\kern1pt}}%
  \savebox{\mysim}{\hbox{$\sim$}}%
  \mathbin{\overset{#1}{\kern\z@\resizebox{\wd\mybox}{\ht\mysim}{$\sim$}}}%
}

\newcommand\Suyash[1]{{\color{blue}Suyash: ``#1''}}
\newcommand\Disha[1]{{\color{red}Suyash: ``#1''}}
\pgfkeys{tikz/mymatrixenv/.style={decoration={brace},every left delimiter/.style={xshift=8pt},every right delimiter/.style={xshift=-8pt}}}
\pgfkeys{tikz/mymatrix/.style={matrix of math nodes,nodes in empty cells,left delimiter={[},right delimiter={]},inner sep=1pt,outer sep=1.5pt,column sep=2pt,row sep=2pt,nodes={minimum width=20pt,minimum height=10pt,anchor=center,inner sep=0pt,outer sep=0pt}}}
\pgfkeys{tikz/mymatrixbrace/.style={decorate,thick}}

\newcommand*\mymatrixbraceright[4][m]{
    \draw[mymatrixbrace] (#1.west|-#1-#3-1.south west) -- node[left=2pt] {#4} (#1.west|-#1-#2-1.north west);
}
\newcommand*\mymatrixbraceleft[4][m]{
    \draw[mymatrixbrace] (#1.east|-#1-#2-1.north east) -- node[right=2pt] {#4} (#1.east|-#1-#2-1.south east);
}
\newcommand*\mymatrixbracetop[4][m]{
    \draw[mymatrixbrace] (#1.north-|#1-1-#2.north west) -- node[above=2pt] {#4} (#1.north-|#1-1-#3.north east);
}
\newcommand*\mymatrixbracebottom[4][m]{
    \draw[mymatrixbrace] (#1.south-|#1-1-#2.north east) -- node[below=2pt] {#4} (#1.south-|#1-1-#3.north west);
}
\newcommand{\tran}{\mathsf{T}}
\newcommand{\sse}{\sigma^2_e}
\newcommand{\bsb}{\boldsymbol{b}}
\newcommand{\bsa}{\boldsymbol{a}}
\makeatother
