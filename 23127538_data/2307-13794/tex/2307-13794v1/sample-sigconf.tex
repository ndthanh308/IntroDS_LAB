\documentclass[sigconf]{acmart}
\usepackage{algorithm}
\usepackage{textcomp}
%\usepackage{amsmath}
\usepackage{enumitem}
\usepackage{dblfloatfix}
\usepackage{lipsum}
\usepackage{soul}
%\usepackage{fancyvrb}
\usepackage{xcolor}
\usepackage{verbatimbox,caption,float,lipsum}
\newfloat{Code}
\usepackage{algpseudocode}
\usepackage{algorithmic}
%\usepackage{longtable}
\usepackage{wrapfig}
\usepackage{gensymb}
\usepackage{textcomp}
%\usepackage{amsmath}
\usepackage{algorithm}
%\usepackage{graphicx}
%\usepackage{caption}
\usepackage{listings}
%\usepackage{array}
\usepackage{subcaption}
\usepackage{booktabs}
\usepackage{amsmath}
\usepackage{enumitem}
\usepackage{dblfloatfix}
\usepackage{lipsum}
\usepackage{soul}
%\usepackage{fancyvrb}
\usepackage{xcolor}

%\usepackage{amsmath,amssymb,amsfonts}
\usepackage{algorithmic}
%\usepackage{graphicx}
\usepackage{textcomp}
\usepackage{xcolor}
\usepackage{amsfonts}
%\usepackage{booktabs}
\usepackage{float}
%\usepackage{graphicx}
\usepackage{relsize}
\usepackage{grffile}
\usepackage{algorithm}
%\usepackage{algpseudocode}
\usepackage{algorithmic}
% http://ctan.org/pkg/algorithmicx
%\usepackage{booktabs} % For formal tables


% Copyright
%\setcopyright{none}
%\setcopyright{acmcopyright}
%\setcopyright{acmlicensed}
\setcopyright{rightsretained}
%\setcopyright{usgov}
%\setcopyright{usgovmixed}
%\setcopyright{cagov}
%\setcopyright{cagovmixed}


% DOI
%\acmDOI{xx.xxx/xxx_x}

% ISBN
%\acmISBN{979-8-4007-0228-0/23/08}

%Conference
%\acmConference[RACS'23]{ACM RACS}{August 06-10, 2023}{Poland}
%\acmYear{2023}
%\copyrightyear{2023}


%\acmArticle{4}
%\acmPrice{15.00}

\copyrightyear{2023}
\acmYear{2023}
\setcopyright{acmlicensed}\acmConference[RACS '23]{International Conference on Research in Adaptive and Convergent Systems}{August 6--10, 2023}{Gdansk, Poland}
\acmBooktitle{International Conference on Research in Adaptive and Convergent Systems (RACS '23), August 6--10, 2023, Gdansk, Poland}
\acmPrice{15.00}
\acmDOI{10.1145/3599957.3606250}
\acmISBN{979-8-4007-0228-0/23/08}


\begin{document}
\title{Integration of Digital Twin and Federated Learning for Securing Vehicular Internet of Things}



\author{Deepti Gupta}

\affiliation{%
  \institution{Texas A\&M University - Central Texas}
  \streetaddress{1001 Leadership Pl,}
  \city{}
  \state{Texas, USA}
  \country{}
  \postcode{76549}
}
\email{deepti.mrt@gmail.com}

\author{Shafika Showkat Moni}

\affiliation{%
  \institution{Embry-Riddle Aeronautical University}
  \streetaddress{P.O. Box 1212}
  \city{Daytona Beach}
  \state{Florida, USA}
  \country{}
  \postcode{43017-6221}
}
\email{shafika1403@gmail.com}

\author{Ali Saman Tosun}
\affiliation{%
  \institution{University of North Carolina at Pembroke}
  \city{North Carolina, USA}
  \country{}
  }
\email{ali.tosun@uncp.edu}

%\author{Lavanya Elluri}

%\affiliation{%
  %\institution{Texas A\&M University - Central Texas}
 % \streetaddress{1001 Leadership Pl,}
  %\city{}
  %\state{Texas, USA}
 % \country{}
 % \postcode{76549}
%}
%\email{elluri@tamuct.edu}






\begin{abstract}

In the present era of advanced technology, the Internet of Things (IoT) plays a crucial role in enabling smart connected environments. This includes various domains such as smart homes, smart healthcare, smart cities, smart vehicles, and many others. The IoT facilitates the integration and interconnection of devices, enabling them to communicate, share data, and work together to create intelligent and efficient systems. With ubiquitous smart connected devices and systems, a large amount of data associated with them is at a prime risk from malicious entities (e.g., users, devices, applications) in these systems. Innovative technologies, including cloud computing, Machine Learning (ML), and data analytics, support the development of anomaly detection models for the Vehicular Internet of Things (V-IoT), which encompasses collaborative automatic driving and enhanced transportation systems. However, traditional centralized anomaly detection models fail to provide better services for connected vehicles due to issues such as high latency, privacy leakage, performance overhead, and model drift.


Recently, Federated Learning (FL) has gained significant recognition for its ability to address data privacy concerns in the IoT domain. In the context of V-IoT, which involves autonomous vehicles and intelligent transportation systems with connected vehicles communicating with various sensors and devices, FL is used to develop an anomaly detection model. Current technology, the Digital Twin (DT), proves beneficial in addressing uncertain crises and data security issues by creating a virtual replica that simulates various factors, including traffic trajectories, city policies, and vehicle utilization. This enables the system to facilitate efficient and inclusive decision-making. However, the effectiveness of a V-IoT DT system heavily relies on the collection of long-term and high-quality data to make appropriate decisions. Consequently, its advantages may be limited when confronted with urgent crises like the COVID-19 pandemic. 



This paper introduces a Hierarchical Federated Learning (HFL) based anomaly detection model for V-IoT, aiming to enhance the accuracy of the model. Our proposed model integrates both DT and HFL approaches to create a comprehensive system for detecting malicious activities using an anomaly detection model. Additionally, real-world V-IoT use case scenarios are presented to demonstrate the application of the proposed model.



\end{abstract}

%
% The code below should be generated by the tool at
% http://dl.acm.org/ccs.cfm
% Please copy and paste the code instead of the example below.
%
\begin{CCSXML}
<ccs2012>
 <concept>
  <concept_id>10010520.10010553.10010562</concept_id>
  <concept_desc>Computer systems organization~Embedded systems</concept_desc>
  <concept_significance>500</concept_significance>
 </concept>
 <concept>
  <concept_id>10010520.10010575.10010755</concept_id>
  <concept_desc>Computer systems organization~Redundancy</concept_desc>
  <concept_significance>300</concept_significance>
 </concept>
 <concept>
  <concept_id>10010520.10010553.10010554</concept_id>
  <concept_desc>Computer systems organization~Robotics</concept_desc>
  <concept_significance>100</concept_significance>
 </concept>
 <concept>
  <concept_id>10003033.10003083.10003095</concept_id>
  <concept_desc>Networks~Network reliability</concept_desc>
  <concept_significance>100</concept_significance>
 </concept>
</ccs2012>
\end{CCSXML}

\ccsdesc[500]{Computer systems organization~Embedded systems}
\ccsdesc[300]{Computer systems organization~Redundancy}
\ccsdesc{Computer systems organization~Robotics}
\ccsdesc[100]{Networks~Network reliability}


\keywords{Vehicular Internet of Things, Hierarchical Federated Learning, Digital Twin, Anomaly Detection Model}
\maketitle
The problem of the presence or absence of phase transition is central in statistical mechanics. To prove the existence of phase transition, the standard idea is to define a notion of contour and use \textit{Peierls' argument} \cite{Peierls.1936}. In the usual Ising model \cite{Ising_25}, particles of the system interact only with their nearest-neighbors. On ferromagnetic long-range Ising models \cite{Anderson_Yuval_69}, there is interaction between each pair of spins in the lattice. The Hamiltonian of the model is given formally by
\begin{equation*}
    H(\sigma) = - \sum_{x,y\in \Z^d}J_{xy}\sigma_x\sigma_y,
\end{equation*}
where $J_{xy}=J|x-y|^{-\alpha}$, $J>0$, $\alpha > d$. It is well-known that the Peierls' argument in dimension 2 implies phase transition for Ising models with nearest-neighbors or long-range interactions when $d\geq 2$, using correlation inequalities. For the unidimensional lattice, it was known that short-range models do not present phase transition. In the long-range case, a different behavior was expected depending on the exponent $\alpha$ (see \cite{Kac_Thompson_69}), but the problem was challenging since contours were first created as multidimensional objects.

In dimension $d=1$, phase transition was proved first in 1969 by Dyson \cite{Dyson.69}, for $\alpha \in (1,2)$, by proving phase transition in an auxiliary model and then using correlation inequalities. In 1982, Fr{\"o}hlich and Spencer \cite{Frohlich.Spencer.82} introduced a notion of one-dimensional contours and then applied the Peierls' argument to show phase transition for the critical value $\alpha = 2$. These contours were inspired by the multiscale techniques previously introduced to study the Berezinskii-Kosterlitz-Thouless transition in two-dimensional continuous spin systems \cite{FS81}. Later, Cassandro, Ferrari, Merola and Presutti  \cite{Cassandro.05} extended the contour argument previously available for $\alpha=2$ to exponents $\alpha\in (3-\frac{\ln 3}{\ln 2}, 2)$, with the additional restriction that the nearest-neighbor interaction is strong, i.e.,  ${J(1)\gg 1}$; this restriction was removed for a subclass of interactions in \cite{Bissacot.Endo.18}. Further results were obtained using contour arguments, such as the decay of correlations, cluster expansions, phase transition with random interactions, etc; some references with these results are \cite{ Cassandro.Merola.Picco.17, Cassandro.Merola.Picco.Rozikov.14, Imbrie.82, Imbrie.Newman.88, Johansson.91}. 

In the multidimensional setting ($d\geq 2$), Ginibre, Grossmann, and Ruelle, in \cite{Ginibre.Grossmann.Ruelle.66}, proved the phase transition for $\alpha > d+1$, using an enhanced version of Peierls' argument and the usual contours. Park proposed a different notion of contour for long-range systems in \cite{Park.88.I, Park.88.II}, extending the Pirogov-Sinai theory available for short-range interactions assuming $\alpha > 3d+1$, although he can also consider Potts models with his methods. Some results in the literature suggest that truly long-range effects appear only when $d < \alpha \leq d+1$, see for instance, \cite{Biskup_Chayes_Kivelson_07}. Recently, Affonso, Bissacot, Endo and Handa \cite{Affonso.2021}, inspired by the ideas from Fr{\"o}hlich and Spencer in \cite{FS81, Frohlich.Spencer.82}, introduced a version of multiscale multidimensional contour and proved phase transition by a contour argument in the whole region $\alpha > d$. They can consider long-range Ising models with deterministic decaying fields, first introduced in the context of nearest-neighbor interactions in \cite{Bissacot_Cioletti_10}. For these models, the lack of analyticity of the free energy does not imply phase transition since these models have the same free energy as the models with zero field. It is expected that fields decaying slowly imply uniqueness. In this setting, a contour argument is useful for proofs of phase transitions as well for uniqueness, some papers with models with deterministic decaying fields are \cite{Aoun_Ott_Velenik_23, Bissacot_Cass_Cio_Pres_15, Bissacot.Endo.18, Cioletti_Vila_2016}.

The Random Field Ising model (RFIM) \cite{Imry.Ma.75} is the nearest-neighbor Ising model with an additional external field acting on each site $(h_x)_{x\in\Z^d}$ that is a family of i.i.d. Gaussian random variable with mean 0 and variance 1. Formally, the Hamiltonian of the model is given by
\begin{equation*}
    H(\sigma) = - \sum_{\substack{x,y\in \Z^d \\|x-y|=1}}J\sigma_x\sigma_y  - \varepsilon\sum_{x\in\Z^d}h_x\sigma_x,
\end{equation*}
where $J>0$, $\varepsilon>0$, $\alpha > d$ and $d \geq 1$. A detailed account of the history of the phase transition problem for this model, as well as detailed proofs, was given in \cite{Bovier.06}. Here we present a brief overview.

During the 1980s, the question of the specific dimension where phase transition for the RFIM should happen attracted much attention and was a topic of heated debate. Two convincing arguments were dividing the physics community. One of them, due to Imry and Ma \cite{Imry.Ma.75}, was a non-rigorous application of the Peierls' argument together with the use of the isoperimetric inequality. The key idea of Peierls' argument is to define a notion of contour and calculate the energy cost of "erasing" each contour, i.e., the energy cost of flipping all spins inside the contour. When there is no external field, that energy necessary to flip the spins in a region $A\subset \Z^d$ is of the order of the boundary $|\partial A|$. When we add an external field, we get an extra cost depending on this field. Imry and Ma argued that this cost should be approximately $\sqrt{|A|}$, which is smaller than $|\partial A|$ for all regions only when $d\geq 3$, so this should be the region where phase transition occurs. The other argument, due to Parisi and Sourlas \cite{Parisi.Sourlas.79}, based on dimensional reduction, predicted that the $d$-dimensional RFIM would behave like the $d-2$-dimensional nearest-neighbor Ising model, therefore presenting phase transition only when $d\geq 4$. 

The question was settled by two celebrated papers showing that Imry and Ma's prediction was correct. First, in 1988, Bricmont and Kupiainen \cite{Bricmont.Kupiainen.88} showed that there is phase transition almost surely in $d\geq3$, for low temperatures and variance $\varepsilon$ small enough. Their proof uses a rigorous renormalization group analysis for the short-range case and it is considered involved. Still, they claimed that the result works for any model with a suitable contour representation and centered sub-gaussian external field. Later on, Aizenman and Wehr \cite{Aizenman.Wehr.90} proved uniqueness for $d\leq 2$. For detailed proofs of these results, we refer the reader to \cite{Bovier.06} (see also \cite{Berretti.85, Camia.18, Frohlich.Imbre.84,  Klein.Masooman.97} for more uniqueness results). 

Recently, Ding and Zhuang, see \cite{Ding2021}, provided a simpler proof of the phase transition, not using RGM. And in  \cite{Ding.Liu.Xia.22}, Ding, Liu and Xia proved that if $\beta_c(d)$ is the critical inverse of the temperature of the Ising model with no field, for all $\beta>\beta_c(d)$ there exists a critical value $\varepsilon_0(d, \beta)$ such that the RFIM with $\varepsilon \leq \varepsilon_0$ presents phase transition. 

In the present paper, we are considering a long-range Ising model with a random field, whose Hamiltonian is given formally by
\begin{equation*}
    H(\sigma) = - \sum_{x,y\in \Z^d}J_{xy}\sigma_x\sigma_y - \varepsilon\sum_{x\in\Z^d}h_x\sigma_x,
\end{equation*}
where $J_{xy}=J|x-y|^{-\alpha}$, $J, \varepsilon>0$, $\alpha > d$ and $h_x\in\mathbb{R}$, $d\geq 3$.
Until now, the only known result in the long-range setting is for the one-dimensional long-range Ising model with a random field, by Cassandro, Orlandi, and Picco \cite{Cassandro.Picco.09}. They used the contours of \cite{Cassandro.05} to show the phase transition for the model when $\alpha\in (3-\frac{\ln 3}{\ln 2}, \frac{3}{2})$, under the assumption $J(1) \gg 1$. We stress that, as remarked by Aizenman, Greenblatt, and Lebowitz \cite{Aizenman_Greenblatt_Lebowitz_2012}, although their argument does not work for the whole region for the exponent $\alpha$, the phase transition holds for values close to the critical value $\alpha=3/2$, since by the Aizenman-Wehr theorem we know that there is uniqueness for $\alpha>3/2$.

The argument from Ding and Zhuang in \cite{Ding2021}, for $d\geq3$, involves controlling the probability of a bad event, which is closely related to controlling the quantity $$\sup_{\substack{0\in A\subset\Z^d \\ A \text{ connected }}}\frac{\sum_{x\in A}h_x}{|\partial A|},$$ known as the greedy animal lattice normalized by the boundary. The greedy animal lattice normalized by the size, instead of the boundary, was extensively studied for general distributions of $(h_x)_{x\in\Z^d}$, see \cite{Cox_Gandolfi_Griffin_Kesten_93, Gandolfi_Kesten_94, Hammond_06, Martin_02}. When we normalize by the boundary, an argument by Fisher, Fr\"{o}hlich and Spencer \cite{FFS84} shows that the expected value of the greedy animal lattice is constant. In dimension $d=2$, the expected value is not finite, see \cite{Ding.Wirth.20}. The supremum is taken over connected regions containing the origin since the interiors of the usual Peierls contours are of this form.


For the long-range model, the interior of contours is not necessarily connected. In fact, long-range contours may have considerably large diameters with respect to their size, so their interiors can be very sparse. To avoid this, we define contours, strongly inspired by the $(M,a,r)$-partition in \cite{Affonso.2021}, using a multiscaled procedure that assures that the contours have no cluster with small density.  With them, we generalize the arguments by Fisher-Fr\"{o}hlich-Spencer \cite{FFS84}, and prove that the expected value of the greedy animal lattice is constant, even considering regions not necessarily connected in the supremum. Then, we prove the phase transition for $d\geq 3$. The main result of this paper is the following.
\begin{theorem*}Given $d\geq 3$, $\alpha>d$, there exists $\beta_c\coloneqq\beta(d, \alpha)$ and $\varepsilon_c\coloneqq\varepsilon(d, \alpha)$ such that, for $\beta >\beta_c$ and $\varepsilon\leq \varepsilon_c$, the extremal Gibbs measures $\mu_{\beta, \varepsilon}^+$ and $\mu_{\beta, \varepsilon}^-$ are distinct, that is, $\mu_{\beta, \varepsilon}^+ \neq \mu_{\beta, \varepsilon}^-$ $\mathbb{P}$-almost surely. Therefore the long-range random field Ising model presents phase transition.
\end{theorem*}

This paper is divided as follows. In Section 2, we define the model and the contours, and suitable generalizations to the constructions in \cite{Affonso.2021} are introduced.  In Section 3, we define two bad events of the external field and prove that they occur with a small probability.  In Section 4, we present the proof of the phase transition.
Previous work has shown that integration over learned functions is a powerful tool. It has been used in many disparate applications, from constraining monotonic functions \cite{monotonic} to computing the output of a network over a distribution of parameters \cite{bayesian}. The entire field of Neural ODEs is predicated on the ability to integrate a learned dynamics function over time \cite{neuralode, ffjord, graphode}. However, all of these methods rely on \textit{numerical} integration, which is computationally expensive and not exact.

Another work in the same vein as our method is Hamiltonian Neural Networks \cite{hamiltonian}, which parameterises a vector field which conserves energy by formulating it as the symplectic gradient of an energy function. However, this method is only used as a means of constraining the learned function, not as a technique for integration.
\section{Hierarchical Federated Learning based Anomaly Detection Model}
\label{FL-VIoT}

%FL can be widely adopted in the V-IoT domain to train various ML models, including prediction analysis and anomaly detection, by collecting data from vehicles in a privacy-preserving environment. This approach offers potential advantages such as low latency, high efficiency, data privacy, and improved security mechanisms. For the RPM use case, a FL based anomaly detection model \cite{gupta2021hierarchical} is proposed, leveraging edge computing to execute these models locally without sharing patients' data. Subsequently, this research is enhanced for multi-user scenarios by developing a Hierarchical Federated Learning Model (HFL). This model enables the aggregation of gradients at multiple levels to accommodate multiple participants by utilizing the edge computing and DT technologies.. Figure \ref{fig:HFL} presents an overview of this approach. To train the generated V-IoT sensor data, we use \textsc{FedTimeDis} LSTM \cite{gupta2021hierarchical} approach for the connected automotive environment. 



FL can be widely adopted in the V-IoT domain to train various ML models, such as prediction analysis and anomaly detection, by collecting data from vehicles in a privacy-preserving environment. This approach offers several advantages, including low latency, high efficiency, data privacy, and improved security mechanisms. For RPM use case, a FL-based anomaly detection model~\cite{gupta2021hierarchical} is proposed. This model leverages edge computing to execute the anomaly detection models locally on the edge devices without sharing patients' data with a centralized server. To further enhance the capabilities of the FL model for multi-user scenarios, HFL approach is developed. HFL allows the aggregation of gradients at multiple levels, enabling the participation of multiple entities while leveraging edge computing and DT technologies. Figure~\ref{fig:HFL} provides an overview of this approach. In this research, HFL approach is used to develop anomaly detection model for the V-IoT systems. 


In the connected automotive environment, there are various types of anomalies, such as traffic congestion, collision detection, malicious attacks, vehicle breakdown, traffic violations and driver fatigue or distraction. The detection and timely response to these anomalies can contribute to improving safety, efficiency, and overall performance in connected vehicle environments. Detecting and understanding anomalies in the V-IoT can lead to enhanced safety, security, performance optimization, and better management of traffic and resources. It allows for proactive decision-making and timely interventions to ensure a smoother and more efficient functioning of the connected vehicle ecosystem.


To develop a HFL based anomaly detection model for the V-IoT, we define the objectives, performance metrics, and requirements for the anomaly detection model. Then, gather relevant data from vehicles in the V-IoT, which may include sensor data, vehicle telemetry, weather statistics, traffic light data and historical records. Ensure that the data collection process preserves privacy and follows ethical guidelines, which is provided by city policies. After that, clean and preprocess the collected data to remove noise, handle missing values, and normalize the features. This step is crucial for preparing the data for further analysis and training. In next step, design the hierarchical architecture for FL in the V-IoT. Where, we need to determine the levels of aggregation such as vehicle-level, region-level, or vendor-level, based on the collaboration requirements and privacy considerations. 


To train the data, we utilize the FedTimeDis LSTM~\cite{gupta2021hierarchical} approach, which is specifically designed for the connected automotive environment. Now, each smart vehicle performs local training using their own data. This training is done in a privacy-preserving manner, where data remains on the local device and only model updates (e.g., gradients) are shared. To perform gradient aggregation at each level of the hierarchy to combine the model updates from different participants. This aggregation process ensures that the collective knowledge of the participating entities is utilized to improve the overall anomaly detection model. After developing the model,  evaluate the performance of the aggregated anomaly detection model using evaluation metrics such as accuracy, precision, recall, or F1-score. Refine the model if necessary by adjusting hyperparameters, incorporating feedback, or retraining with additional data. This developed HFL-based anomaly detection model can be deployed in a real-world V-IoT environment and the performance of the deployed model can be monitored continuously. Incorporate new data, update the model periodically, and iterate on the anomaly detection process to enhance its accuracy, efficiency, and robustness.






%Moni please write about smart vehicle sensors, you can tell about V2V,V2I, V2D, After that you can tell about why anomaly detection model is necessary and what types are anomalies we need to identify? tell some examples of anomalies in V-IoT
%The connected automotive environment consists of the 


\begin{algorithm}[!t]
%\SetAlgoLined
\caption{Anomaly Detection of all data nodes $N$}
\label{algo2}
\begin{algorithmic}[1]
\STATE{Collect vehicular the data $D_i$ from all data nodes $N$}
\STATE{Preparing the data $D_i$ by converting into numerical form}
\STATE{Normalize the data $D_i$}
\STATE{Create the sequences sets $(X^n,y^n)$ of data based on correlations}\\
\STATE{Take these input sequence sets $(X^n,y^n)$, where $n= 1,2,$\dots$,N$ from $N$ data nodes, initial model parameter $w_{z}$, local minibatch size $J$, number of local epochs $H$, learning rate $\alpha$, number of rounds $Q$, $h$ hidden layer.}

\STATE{Split local dataset $D_i$ to mini batches of size $J$ which are included into the set $J_i$ and fed horizontally to four LSTM cells.}

\FOR{each local epoch $j$ from 1 to $H$}
\FOR{batch $(X, y)$ $\in$ $J$}
\STATE${h\textsubscript{t} = LSTM(h\textsubscript{t-1}, x\textsubscript{t}, w\textsuperscript{n})}$
\STATE{$y$\textsuperscript{n} = ${\sigma(W\textsuperscript{FC}h\textsubscript{2nd}+Bias)}$}\\
\STATE{$u$\textsuperscript{n} = $w$\textsuperscript{n} - ${w_{z}^n}$ }\\
\STATE{$w_{z}^n$ $\leftarrow$ $w_{z}^n$  + $\frac{\alpha}{N}$ ${\mathlarger{\sum}}\textsubscript{$n$ $\in$ $D_i$} u\textsuperscript{n}$}
\ENDFOR
\ENDFOR 
\STATE{Update weights $w_{z}^n$ to federated cloudlet server and start training again until minimizing the error to build the anomaly detection model.}
\end{algorithmic}
\end{algorithm}










 



% Figure environment removed





\section{System Model}
\label{system}

In this section, we introduce our proposed model, which aims to identify anomalies and enhance the security of the V-IoT systems. The model comprises six distinct phases, each involving data exchange and collaboration among different entities. The overall system architecture is illustrated in Figure~\ref{fig:system}, providing a visual representation of the data flow and interaction between the components.


% Figure environment removed
The six phases of our proposed model are as follows:

\begin{itemize}
    \item \textit{Initial Phase}: During the initial phase of our proposed model, the smart vehicle begins collecting various types of data, including manufacturing data, driver perception data, and external entities data. By collecting these various types of data, the smart vehicle aims to gather comprehensive information about its own performance, the driver's behavior, and the external environment. 

    \item \textit{Functional Phase}: During the functional phase of our model, the entities within the V-IoT system transition into operational mode. This phase is divided into two sub-phases, each serving specific purposes. In the first sub-phase, the focus is on collecting data from the vehicle IoT sensors. These sensors, which are activated and operational, capture various types of information such as vehicle diagnostics, and performance metrics. The collected data is then transmitted to the vendor cloudlet, a cloud-based infrastructure specifically designed to handle V-IoT data. Simultaneously, in the second sub-phase, additional data is collected on the vendor cloudlet. This includes a wide range of data sources such as weather statistics, city regulations and policies, traffic light information, and camera data. These supplementary data sources provide contextual information about the external environment in which the vehicles operate.
    
    By collecting data from both the vehicle IoT sensors and other relevant sources on the vendor cloudlet, a comprehensive and multi-dimensional dataset is created.

    \item \textit{Analytic Phase}: Once the data is collected from the V-IoT system, it is transmitted to the simulated environment for further analysis. This phase involves the transition of data from the physical space to the simulated space, where advanced data analytics techniques are applied. In the simulated environment, a DT is developed for each entity within the V-IoT system. A DT is a virtual representation of a physical entity, in this case, the vehicles and other components of the V-IoT system. The DT is created based on the generated data collected from the previous phases. The data analytics process is then performed on the vehicle DT. Various analytical techniques and algorithms are applied to gain insights and extract valuable information from the data. These analytics help in understanding the behavior, performance, and patterns within the V-IoT system.
    
    By leveraging the DT and conducting data analytics, it becomes possible to identify and understand anomalies within the V-IoT system. Anomalies can include unusual behavior, deviations from normal patterns, or any abnormal activities that may indicate potential security or operational issues.


    \item \textit{Identifying Anomaly Phase}: In this phase, the simulated data $D_i$ from all the data nodes that has been processed and prepared in the previous phase is passed through a pipeline to feed the anomaly detection model. The data is carefully curated and transformed to be compatible with the model's input requirements. The anomaly detection model is developed using suitable machine learning algorithms and follows the Algorithm~\ref{algo2}. These algorithms are trained on the prepared data to learn the patterns and characteristics of normal behavior within the V-IoT system. The model aims to distinguish between normal and anomalous patterns based on the input data. During the training process, the model undergoes iterations to optimize its performance and enhance its ability to accurately detect anomalies. This involves adjusting the model's parameters, fine-tuning the algorithms, and validating the model's performance using appropriate evaluation metrics. Once the training is completed, the anomaly detection model is ready to be deployed and utilized. It collaborates with other models that are part of the subsequent phases, working together to enhance the accuracy and effectiveness of anomaly detection in the V-IoT system.


    
    %In this phase, the simulated data is passed through the pipeline to feed the anomaly detection model. The data is trained using suitable machine learning algorithms to develop an anomaly detection model. Once the training is completed, the model collaborates with other models that are part of the next phase

    \item \textit{Collaborative Phase}: Indeed, in this phase, the collaboration of multiple anomaly detection models take place to improve the accuracy rate of anomaly detection in the V-IoT system. By combining the weights of multiple ADM models, the overall effectiveness of anomaly detection can be significantly enhanced. Each model may have its own unique approach, algorithm, or specialization in detecting specific types of anomalies. By leveraging the strengths and capabilities of different models, a more comprehensive and robust anomaly detection system can be established. The collaboration among anomaly detection models involves exchanging information, sharing insights, and aggregating their detection results. This collaborative process allows for a holistic analysis of the system's behavior and the identification of anomalies from multiple perspectives.


  
    %As mentioned, this phase involves the collaboration of multiple Anomaly Detection Models (ADM) to improve the accuracy rate. The ADM models work together to enhance the overall effectiveness of anomaly detection in the system.

    \item \textit{Reporting and Decision Phase}: After an anomalous scenario is detected by the anomaly detection model, it is crucial to report the anomaly to the relevant stakeholders, including the user, vendor, and device. This phase plays a vital role in facilitating informed decision-making and taking necessary actions to ensure the safety and security of the automotive connected environment. Reporting the anomaly to the user is essential as it enables them to be aware of the detected anomaly and take appropriate measures. This could involve alerting the user through notifications, messages, or visual indicators, providing them with information about the anomaly and any recommended actions they should take. Notifying the vendor is also crucial as it allows them to be aware of the anomaly and take the necessary steps to address the issue. This could involve investigating the root cause of the anomaly, analyzing the data collected, and implementing corrective measures to prevent similar anomalies in the future.
    
    Overall, this phase of reporting anomalies is a critical component of the anomaly detection process in the V-IoT system. It helps to minimize the risks associated with anomalous events, enables proactive decision-making, and contributes to maintaining a safe and reliable automotive ecosystem.
    
    
    
    %After an anomalous scenario is detected using the Anomaly Detection Model (ADM), this phase involves reporting the anomaly to the user, vendor, and device. By notifying all relevant parties, this phase enables informed decision-making to ensure the safety of the automotive connected environment.
\end{itemize}

By employing our proposed system model, the V-IoT environment can detect anomalies in real-time and enable prompt responses to the system. This improves overall security and privacy, enhances the efficiency of the transportation system, and improves the driving experience for individuals.

The following section presents a use case scenario of V-IoT, where our proposed system model is employed to detect anomalies.





\section{Use Case Scenario}
\label{use case}
V-IoT is unfolding in many ways where users receive better services and take advantage of autonomous vehicle. The proposed system model where we present the integration of FL and DT, which can be used to secure V-IoT applications, for example, ITSs, cooperative autonomous driving, connected car services, smart city integration, collision avoidance systems and intelligent traffic control etc. In this section, we present a use-case scenario for securing V-IoT by developing an anomaly detection model. The Figure~\ref{fig:anomaly} shows the use case based on our proposed system model, which is discussed in Section~\ref{system}. 

In a smart city, the adoption of IoT technologies and the deployment of smart vehicles are guided by common city policies and regulations. These policies ensure uniformity and standardization across different regions within the smart city. Each region within the smart city may have multiple vendors launching their smart vehicles, contributing to the overall intelligent transportation ecosystem. The presence of multiple vendors in different regions allows for a diverse range of smart vehicles with varying features, technologies, and capabilities. These vehicles may be equipped with advanced sensors, communication systems, and intelligent algorithms to enhance their functionality and contribute to the overall smart city objectives. 


In Figure~\ref{fig:anomaly}, the depicted scenario showcases the presence of two different vendors, namely \textit{vendor-1} and \textit{vendor-2}, operating within \textit{region-1} of the V-IoT system. \textit{Vendor-1} has two smart vehicles, SV-1 and SV-3, while \textit{vendor-2} has one smart vehicle, SV-2. Additionally, \textit{region-1} consists of various data points, including sensor data, city policies and regulations, traffic lights, and weather statistics. To enable efficient data processing and anomaly detection in \textit{region-1}, a cloudlet is launched specifically for this region. The cloudlet serves as a localized computing resource that can host and deploy DTs of each smart vehicle within the region. These DTs not only receive data from their respective smart vehicles (SV-1, SV-2, and SV-3) but also incorporate data from other sources within the region, such as sensor data, city policies and regulations, traffic lights, and weather statistics.


In this use case, we deploy DTs on the cloudlet to reduce the gap between physical objects and their digital representations which are generally hosted in the cloud servers. These cloudlets are hosted by the regions and basically a regional cloud that can better cloud services nearer to the user. Cloudlets~\cite{saini2022chapter} are small-scale, mobility-enhanced cloud data centers that sit at the network's edge. The cloudlet's primary goal is to support furious resource and interactive mobile applications by delivering strong computing resources to mobile devices with reduced latency. A wireless local area network with single hop at comparatively higher speed, allows User Equipments (UEs) to connect to the computing resources in the neighboring cloudlet. By leveraging the cloudlet in \textit{region-1}, the DTs can effectively analyze and process the combined data from multiple sources. This integration of data from various entities allows for a holistic understanding of the V-IoT environment within \textit{region-1}. The DTs can leverage this comprehensive data to enhance anomaly detection capabilities, identify patterns, and detect any deviations or anomalies in the behavior of the smart vehicles or the overall V-IoT system. The deployment of DTs within the cloudlet of \textit{region-1} enables localized processing and analysis, reducing latency and enhancing real-time anomaly detection. The collaboration of the DTs with the cloudlet infrastructure facilitates efficient information exchange and enables timely response to any detected anomalies. 

DT of SV-1 within \textit{region-1} plays a crucial role in the development of the anomaly detection model. The data collected by SV-1's DT is utilized to train the initial anomaly detection model specific to \textit{region-1}. This model focuses on detecting anomalies within the local context and behavior of the vehicles and infrastructure within \textit{region-1}. To enhance the accuracy and effectiveness of the anomaly detection model, collaboration is encouraged among multiple models. In this case, the anomaly detection models of \textit{region-1} have the capability to collaborate with each other using the concept of FL. FL allows the models to share their knowledge and insights while maintaining data privacy and security. By aggregating the local models' learnings through weight aggregation techniques, a more robust and accurate anomaly detection model can be obtained.


Moreover, collaboration is not limited to models within the same region. The anomaly detection model of \textit{region-1} can also collaborate with the anomaly detection model of \textit{region-2} by using HFL concept. This collaboration is facilitated by exchanging gradients on a federated multi-cloud server. The gradients represent the model parameters that are shared and utilized to improve the models' performance collectively. By leveraging the collaboration capabilities of FL and the exchange of gradients on the federated multi-cloud server, the anomaly detection models of different regions can benefit from each other's insights and experiences. This cross-region collaboration enhances the overall effectiveness of anomaly detection in the V-IoT system by incorporating knowledge from diverse geographical areas and vehicle behaviors.

In summary, the integration of FL and the exchange of gradients enable collaboration among anomaly detection models at different levels by utilizing DT. This collaboration improves the accuracy and robustness of the models, both within the same region and across different regions, leading to more effective anomaly detection in the V-IoT system.





%The anomaly detection model of \textit{region-1} can also collaborate with anomaly detection model of \textit{region-2} by exchanging gradients on federated multi-cloud server. 








%Figure \ref{fig:anomaly} illustrates that \textit{region-1} has two different vendors, \textit{vendor-1} (SV-1, SV-3) and \textit{vendor-2} (SV-2). Region-1 has also other data points including sensor data, city policies and regulations, traffic lights and weather statistics. In this use case, we launch the cloudlet of \textit{region-1} where DT of each smart vehicle can deploy, this DT will not only recive the smart vehicle data but also other data from their region. 








%\subsection{FL for Predictive Analysis for V-IoT}
%In smart vehicle environment, predictive analytics is an effective data-driven application. V-IoT sensors and other environmental sensors such as camera, traffic light, motion sensors generate tremendous amount of data. This data is processed by Machine Learning algorithms to predict the on-road decisions in real-time and also used to predicting maintenance for vehicle. In current scenario, these autonomous vehicles use their own models for predictive analysis, however, it may provide bias, limited and imprecise results. To improve the efficiency of this application, ML models should train their data in collaborative manner. FL is a privacy-preserving approach, which is suitable to implement predictive analysis use case scenario. 



%In the vehicular domain, other data points such as traffic light, camera, city policies, regulations, and weather data merge to


%data privacy is a critical concern and we cannot merge one patient's data with another patient's DT, so training of datasets cannot be performed on patient's DT in a collaborative manner. We cannot run any anomaly detection model on a patient's DT, so we design FL based anomaly detection model and deploy this model on an edge cloudlet associated with the patient's DT. 
\section{Conclusion}\label{sec:conclusion}

This paper presents our empirical domain knowledge distillation framework using ChatGPT and discusses our observations from the framework application experiments in the autonomous driving domain. The key finding is that: 1) with proper design of prompt engineering and execution flow, fully automated domain knowledge (in the ontology format) distillation is possible. However, due to the randomness in the response and the butterfly effect, the quality of fully automated distillation results is not guaranteed. To address this, we develop a web-based assistant to enable manual supervision and early intervention at runtime. We hope our findings and tools inspire future research toward revolutionizing the engineering processes of knowledge-based systems across domains.
\bibliographystyle{ACM-Reference-Format}
\bibliography{References}
\end{document}
