\section{Hierarchical Federated Learning based Anomaly Detection Model}
\label{FL-VIoT}

%FL can be widely adopted in the V-IoT domain to train various ML models, including prediction analysis and anomaly detection, by collecting data from vehicles in a privacy-preserving environment. This approach offers potential advantages such as low latency, high efficiency, data privacy, and improved security mechanisms. For the RPM use case, a FL based anomaly detection model \cite{gupta2021hierarchical} is proposed, leveraging edge computing to execute these models locally without sharing patients' data. Subsequently, this research is enhanced for multi-user scenarios by developing a Hierarchical Federated Learning Model (HFL). This model enables the aggregation of gradients at multiple levels to accommodate multiple participants by utilizing the edge computing and DT technologies.. Figure \ref{fig:HFL} presents an overview of this approach. To train the generated V-IoT sensor data, we use \textsc{FedTimeDis} LSTM \cite{gupta2021hierarchical} approach for the connected automotive environment. 



FL can be widely adopted in the V-IoT domain to train various ML models, such as prediction analysis and anomaly detection, by collecting data from vehicles in a privacy-preserving environment. This approach offers several advantages, including low latency, high efficiency, data privacy, and improved security mechanisms. For RPM use case, a FL-based anomaly detection model~\cite{gupta2021hierarchical} is proposed. This model leverages edge computing to execute the anomaly detection models locally on the edge devices without sharing patients' data with a centralized server. To further enhance the capabilities of the FL model for multi-user scenarios, HFL approach is developed. HFL allows the aggregation of gradients at multiple levels, enabling the participation of multiple entities while leveraging edge computing and DT technologies. Figure~\ref{fig:HFL} provides an overview of this approach. In this research, HFL approach is used to develop anomaly detection model for the V-IoT systems. 


In the connected automotive environment, there are various types of anomalies, such as traffic congestion, collision detection, malicious attacks, vehicle breakdown, traffic violations and driver fatigue or distraction. The detection and timely response to these anomalies can contribute to improving safety, efficiency, and overall performance in connected vehicle environments. Detecting and understanding anomalies in the V-IoT can lead to enhanced safety, security, performance optimization, and better management of traffic and resources. It allows for proactive decision-making and timely interventions to ensure a smoother and more efficient functioning of the connected vehicle ecosystem.


To develop a HFL based anomaly detection model for the V-IoT, we define the objectives, performance metrics, and requirements for the anomaly detection model. Then, gather relevant data from vehicles in the V-IoT, which may include sensor data, vehicle telemetry, weather statistics, traffic light data and historical records. Ensure that the data collection process preserves privacy and follows ethical guidelines, which is provided by city policies. After that, clean and preprocess the collected data to remove noise, handle missing values, and normalize the features. This step is crucial for preparing the data for further analysis and training. In next step, design the hierarchical architecture for FL in the V-IoT. Where, we need to determine the levels of aggregation such as vehicle-level, region-level, or vendor-level, based on the collaboration requirements and privacy considerations. 


To train the data, we utilize the FedTimeDis LSTM~\cite{gupta2021hierarchical} approach, which is specifically designed for the connected automotive environment. Now, each smart vehicle performs local training using their own data. This training is done in a privacy-preserving manner, where data remains on the local device and only model updates (e.g., gradients) are shared. To perform gradient aggregation at each level of the hierarchy to combine the model updates from different participants. This aggregation process ensures that the collective knowledge of the participating entities is utilized to improve the overall anomaly detection model. After developing the model,  evaluate the performance of the aggregated anomaly detection model using evaluation metrics such as accuracy, precision, recall, or F1-score. Refine the model if necessary by adjusting hyperparameters, incorporating feedback, or retraining with additional data. This developed HFL-based anomaly detection model can be deployed in a real-world V-IoT environment and the performance of the deployed model can be monitored continuously. Incorporate new data, update the model periodically, and iterate on the anomaly detection process to enhance its accuracy, efficiency, and robustness.






%Moni please write about smart vehicle sensors, you can tell about V2V,V2I, V2D, After that you can tell about why anomaly detection model is necessary and what types are anomalies we need to identify? tell some examples of anomalies in V-IoT
%The connected automotive environment consists of the 


\begin{algorithm}[!t]
%\SetAlgoLined
\caption{Anomaly Detection of all data nodes $N$}
\label{algo2}
\begin{algorithmic}[1]
\STATE{Collect vehicular the data $D_i$ from all data nodes $N$}
\STATE{Preparing the data $D_i$ by converting into numerical form}
\STATE{Normalize the data $D_i$}
\STATE{Create the sequences sets $(X^n,y^n)$ of data based on correlations}\\
\STATE{Take these input sequence sets $(X^n,y^n)$, where $n= 1,2,$\dots$,N$ from $N$ data nodes, initial model parameter $w_{z}$, local minibatch size $J$, number of local epochs $H$, learning rate $\alpha$, number of rounds $Q$, $h$ hidden layer.}

\STATE{Split local dataset $D_i$ to mini batches of size $J$ which are included into the set $J_i$ and fed horizontally to four LSTM cells.}

\FOR{each local epoch $j$ from 1 to $H$}
\FOR{batch $(X, y)$ $\in$ $J$}
\STATE${h\textsubscript{t} = LSTM(h\textsubscript{t-1}, x\textsubscript{t}, w\textsuperscript{n})}$
\STATE{$y$\textsuperscript{n} = ${\sigma(W\textsuperscript{FC}h\textsubscript{2nd}+Bias)}$}\\
\STATE{$u$\textsuperscript{n} = $w$\textsuperscript{n} - ${w_{z}^n}$ }\\
\STATE{$w_{z}^n$ $\leftarrow$ $w_{z}^n$  + $\frac{\alpha}{N}$ ${\mathlarger{\sum}}\textsubscript{$n$ $\in$ $D_i$} u\textsuperscript{n}$}
\ENDFOR
\ENDFOR 
\STATE{Update weights $w_{z}^n$ to federated cloudlet server and start training again until minimizing the error to build the anomaly detection model.}
\end{algorithmic}
\end{algorithm}










 



% Figure environment removed




