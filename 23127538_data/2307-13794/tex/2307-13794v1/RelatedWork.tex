\section{Related Work and Background}
\label{related}
%This section discusses fundamental concepts and background essential to comprehend the research contributions. It describes the concept of V-IoT, security and privacy issues, DT and FL model.

This section provides an in-depth discussion of fundamental concepts and background information that are essential to understand the research contributions. It covers key topics such as the concept of the V-IoT, security and privacy concerns, as well as FL models and DT.



%\subsection{Related Work}


Recently, there has been a growing interest in various technologies such as cloud computing, edge computing, ML, FL, and DT in both academic and industrial sectors. These technologies are seen as promising solutions for enabling smart cities and ITSs. Lu et al.~\cite{lu2020blockchain} proposed an asynchronous federated framework to implement secure and effective data sharing in the  Internet of Vehicles (IoV). In this approach, each vehicle serves as FL client and shares data with an aggregation server at macro BS (MBS). Vehicles can request a variety of services, including traffic prediction and path selection to the MBS. The MBS develops a shared global model based on accumulated vehicular datasets. Next, the MBS transforms the sharing process into a computing task and resolves the sharing request of vehicles by using an actor-critic reinforcement learning framework.


Chai et al.~\cite{chai2020hierarchical} proposed a hierarchical blockchain-enabled FL scheme for IoV. They presented a feasibility analysis of adapting the hierarchical model to manage large-scale vehicles. In this scheme, each vehicle serves as an FL  client and uses its hardware resources to implement local learning. Road side units (RSUs) are responsible for collecting transactions from vehicles within their communication region in a blockchain framework. Each RSU compute the FL model and append into the blockchain framework to ensure security. The blockchain framework is shared to all RSUs and vehicles in IoV. Shrivastava et al.~\cite{shrivastava2021designing} presented a brief survey on Security in V-Iot using blockchain. In addition, several security models for protecting IoT devices in other domains are discussed in~\cite{gupta2020access, gupta2020learner, gupta2021game, gupta2021future, aslan2021intelligent, ozkan2021comprehensive, gupta2021detecting, gupta2021hierarchical, akbarfam2023dlacb, moni2022secure, pei2022personalized}. 



\subsection{Vehicular Internet of Things}
%The vehicular Internet of Things (V-IoT) can be characterized as a platform that facilitates information exchange between a vehicle and its surroundings through vehicle-to-vehicle (V2V), vehicle to infrastructure (V2I), and vehicle to everything (V2X) communication.

V-IoT can be described as a platform that enables the exchange of information between vehicles and their surroundings. This communication is facilitated through various channels, including vehicle-to-vehicle (V2V), vehicle-to-infrastructure (V2I), and vehicle-to-everything (V2X) communication. These interactions allow vehicles to connect with other vehicles, infrastructure elements, and various entities in their environment, creating a networked ecosystem that enhances safety, efficiency, and overall driving experience. Yang et al.~\cite{yang2014overview} put forward an abstract network model for the IoV. Their research focuses on discussing the necessary technologies to establish the IoV framework. They explore various applications that can be built upon existing technologies, highlighting the potential of IoV in different domains. V-IoT enables drivers, pedestrians, and other vehicles to utilize the data produced by vehicular ad hoc networks (VANETs) with the aid of roadside infrastructure ~\cite{moni2020efficient, moni2022crease}. V-IoT integrates the IoT technology with ITSs to improve transportation efficiency and security. It is anticipated that the V-IoT will play a vital role in enabling connected, shared, autonomous, and electric future mobility. This article~\cite{ji2020survey} conducted an extensive literature review focusing on the fundamental aspects of the IoV. They covered essential information related to IoV, including basic VANET technology, different network architectures employed in IoV systems, and typical applications of IoV. 

% Figure environment removed



\subsection{Federated Learning}

FL is an approach that places a strong emphasis on privacy by allowing ML models to be trained locally on individual devices, without the need to share the underlying data with a centralized server. This decentralized training process, depicted in Figure \ref{fig:center}, ensures that sensitive data remains on the devices where it is generated, reducing the risk of privacy breaches. By adopting FL, these privacy risks can be mitigated, as the data remains securely stored on the local devices, and only aggregated model updates are shared with the central server. This way, FL strikes a balance between data privacy and the need for accurate model training, making it a reliable solution for privacy-preserving ML in various applications. Du et al.~\cite{du2020federated} conducted a comprehensive survey of existing studies on FL and its use in wireless IoT. Then, they highlighted the potential benefits of FL in addressing the unique requirements and complexities of vehicular IoT environments. 

Mothukuri et al.~\cite{mothukuri2021federated} introduced a novel approach for anomaly detection in IoT networks using FL. Their proposed method leverages decentralized on-device data to proactively identify intrusions in IoT networks.  



%FL is an approach that prioritizes privacy by allowing machine learning models to be trained locally on individual devices without sharing the data with a centralized server, as depicted in the figure \ref{fig:center}. Deep Learning (DL) relies on labeled data to train accurate models capable of performing structured tasks like classification and prediction. However, the process of labeling a dataset comes with an additional privacy cost. Identifying individual data points can have severe consequences for the users who generated that data. The utilization of labeled data can potentially reveal the source, posing risks to security and privacy. In this context, FL has emerged as a reliable solution for preserving privacy in such scenarios.





\subsection{Digital Twin}
DT is referred to as a virtual representation of the real-world entity, devices, machines, process, or other abstraction.  Physical sensors, computer programs, machine learning algorithms, and software models are used to simulate real-time digital models of the physical entity. Tao et al.~\cite{tao2018digital} mentioned that It is considered as one of the most promising enabling technologies for realizing smart manufacturing and Industry 4.0. Wang et al.~\cite{wang2023survey} conducted a comprehensive review of the Internet of Digital Twins (IoDT), focusing on various aspects such as system architecture, enabling technologies, and security/privacy concerns. It facilitates real-time interaction, close monitoring, and reliable communication between the digital model and its physical counterpart. DT is considered to be one of the most promising technology due to its advanced capabilities, intelligent services, and bridging the gap between the digital model and its physical counterpart. For instance, the DT of a vehicle can observe the driving pattern of the driver and communicate, interact, and share this information with other DTs to notify the driver about possible issues or emergencies  on road. 



Previous research has introduced several anomaly detection models based on FL in various domains. These models have been deployed either on centralized cloud servers or edge devices. Additionally, the concept of digital twins (DT) has been utilized to identify anomalies in the industrial domain. However, as mentioned earlier, these anomaly detection models often suffer from low accuracy rates due to limited volumes of data available for training. Consequently, a robust anomaly detection model that can provide effective security and privacy solutions for protecting V-IoT systems is still lacking. To bridge this gap, our proposed integrated approach-based anomaly detection model offers a novel perspective for detecting anomalies in the vehicular domain. By combining the strengths of different technologies, such as FL and DT, we aim to enhance the accuracy and effectiveness of anomaly detection in the V-IoT systems. Our approach provides a comprehensive solution that leverages collaborative learning among distributed entities and utilizes the virtual replicas created by digital twins to simulate and analyze various factors contributing to anomalies. We strongly believe that our integrated approach presents a promising solution to address the challenges of anomaly detection in the vehicular domain.
% Figure environment removed
