\section{Conclusion and Future Work}
\label{conclusion}


In this paper, we have discussed the relevance of FL in V-IoT environments and its potential impact. We started by providing background information on V-IoT, FL, and DT technologies. We emphasized the importance of anomaly detection models in the V-IoT domain and also presented the outline of HFL based anomaly detection model. To address the challenges and opportunities in the V-IoT, we proposed a system model that integrates DT and FL. This model leverages the collaborative learning capabilities of FL and the bridging capabilities of DT between the physical system and its virtual representation. We outlined the key components and phases of our proposed model, emphasizing data exchange, anomaly detection, and security.

To illustrate the practical application of our proposed model, we presented a use case scenario in which the model is employed to detect anomalies in the V-IoT environment. The scenario showcased the data collection, processing, anomaly detection, and collaborative response aspects of our model, highlighting its potential benefits in ensuring safety and efficiency in the V-IoT systems.

Overall, this work aims to contribute to the advancement of FL, DT, and V-IoT research. By introducing our proposed model and presenting a use case scenario, we provide a foundation for further exploration, development, and practical implementation of HFL and DT in the V-IoT environments. We believe that this paper will facilitate the progress of these fields and stimulate further research in the area of HFL-based anomaly detection by utilizing DT in the V-IoT.

%With the increasing interest in FL from various domain, a discourse on the utilization of FL in V-IoT environments becomes significant. In this paper, we have examined the background of V-IoT, FL, and DT. We have highlighted the significance of HFL and presented a system model incorporating DT and FL. Furthermore, we have presented a use case scenario to demonstrate the practical application of our proposed model. We believe that this work has the potential to accelerate the research process for FL, DT and V-IoT. By introducing the proposed model and presenting its use case scenario, we aim to contribute to the advancement of these fields and provide a foundation for further exploration and development.