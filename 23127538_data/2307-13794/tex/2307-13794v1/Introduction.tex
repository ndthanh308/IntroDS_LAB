\section{Introduction}

The proliferation of Internet of Things (IoT) technology has become increasingly prevalent in our daily lives, primarily driven by the advancements in low latency and high-speed cellular networks. This technological progress has facilitated seamless connectivity and communication between various smart devices, enabling them to exchange data and interact in real-time. As a result, IoT has found widespread applications in diverse domains such as smart homes, healthcare, transportation, industrial automation, and more, enhancing efficiency, convenience, and automation in our day-to-day activities. As a result of such enormous growth, the number of IoT and connected devices is expected to increase to 60 billion by 2025~\cite{moni2021scalable}. A significant part of this rapid increase is likely linked to the Vehicular Internet of Things (V-IoT). V-IoT comprises connected vehicles, Road Side Units (RSUs), sensors, base stations, edge servers, cloud servers, and other devices capable of data sharing and communication with humans. This information generated through V-IoT will play a vital role in traffic management, traffic safety, infotainment services, smart city, and Intelligent Transportation Systems (ITSs), as shown in Figure\ref{fig:IoV}. 




%a DT of a vehicle can collect real-time information e.g. as speed, direction, local traffic information, weather conditions, and information from other DTs of vehicles, and this synchronized information can be utilized to accurately predict future events on roads.

%Federated Learning (FL) is a distributed machine learning approach that prioritizes collaborative learning and privacy without sharing the raw data with a centralized server. FL allows multiple learning agents to collaborate their computing capabilities efficiently and securely to provide better quality of services. Deployment of FL cloud in V-IoT can facilitate the vehicles, RSUs, base station, and other connected devices to to improve the learning efficiency for intelligent environment sensing, intelligent networking, cooperative autonomous driving, and intelligent processing of massive amounts of vehicular data.



Federated Learning (FL) is an innovative approach to Machine Learning (ML) that emphasizes collaborative learning and privacy preservation by avoiding the need to share raw data with a centralized server. In FL, multiple learning agents can efficiently and securely collaborate their computing capabilities to achieve an improved quality of services. The deployment of FL for V-IoT enables vehicles, Roadside Units (RSUs), base stations, and other connected devices to enhance learning efficiency in various aspects such as intelligent environment sensing, intelligent networking, cooperative autonomous driving, and intelligent processing of large volumes of vehicular data. By leveraging FL, V-IoT can benefit from collective intelligence while preserving data privacy and promoting efficient knowledge sharing among the connected entities. 

%add papers based on FL based V-IoT

% Figure environment removed


Recent research studies~\cite{du2020federated, li2023energy, pokhrel2020federated, chai2020hierarchical, lu2020blockchain} have made significant contributions in securing V-IoT environment from anomalies and malicious entities. These studies have proposed and developed various methods and techniques, leveraging the power of FL, to enhance the security and privacy of the V-IoT systems. Additionally, energy-efficient models have been designed, utilizing the FL approach, to optimize energy consumption in V-IoT deployments. The collective findings of these research works contribute to the ongoing efforts in establishing robust security mechanisms and energy efficiency strategies in the V-IoT domain. Despite advances in FL based models for V-IoT system, there are still growing concerns about safety, security and privacy of users. Therefore, this system requires a comprehensive and robust anomaly detection approach to detect anomalous behavior of various entities effectively. 

A Digital Twin (DT) enables connectivity, interaction, and synchronization between the physical entity and its virtual representation in real time. The DT is considered to be one of the most promising technology due to its advanced capabilities, intelligent services, and bridging the gap between the digital model and its physical counterpart. A digital model of a vehicle can be placed at the edge computing node that can serve as an edge middleware in the V-IoT. With the help of this cloud-edge computing paradigm, large-scale data analysis, storage, and modeling are made possible. In addition to that, a huge volume of geographically dispersed information shared by many DTs can be aggregated to derive synthesized information with effectiveness. By creating a virtual replica that mimics real-world conditions, the DT of V-IoT enables simulations and analysis of crucial factors like traffic trajectories, city policies, and vehicle utilization. This virtual representation enhances decision-making processes and assists in developing effective strategies for managing crises while ensuring the security of data within the V-IoT system. 

In this paper, we integrate both DT and FL technologies into the V-IoT framework. Our objective is to harness the advanced computing capabilities offered by DT and leverage the collaborative learning potential of FL to address the security and privacy challenges present in V-IoT systems. By combining these technologies, we aim to enhance the overall performance, efficiency, and privacy preservation in the V-IoT environments. The utilization of DT technology can significantly enhance the efficiency of the anomaly detection model by incorporating data from various sources such as smart sensors, traffic light data, weather statistics, vehicle data, and city policies. The integration of data from multiple vendors enables the provision of more accurate and expedited service delivery for the V-IoT system. In our proposed model, DT facilitates data synchronization and weight aggregation in a synchronous manner, reducing the wait time during FL process. This streamlined approach allows participants to efficiently share their data with the model, ultimately improving the overall performance and effectiveness of the FL-based anomaly detection system. 

In our research, we have implemented a Hierarchical Federated Learning (HFL) approach to develop an anomaly detection model. For example, in \textit{region-1}, where \textit{vendor-1} operates with two smart vehicles, these vehicles collaborate to build their local anomaly detection model. Similarly, in \textit{region-2}, where Vendor-2 operates with five smart vehicles, those vehicles collaborate to construct their own local anomaly detection model. This hierarchical approach allows smart vehicles from multiple regions to collaborate at different levels, enabling the development of robust anomaly detection models for the V-IoT within the same smart city.

By leveraging this HFL approach, our research aims to identify anomalies and enhance the security of the V-IoT systems. Through collaborative learning and data aggregation at different levels, we can improve the accuracy and reliability of the anomaly detection models, ultimately ensuring the integrity and safety of the V-IoT ecosystem. The main contribution of this paper are as follows-
\begin{itemize}

    \item In our research, we have identified a research gap in the development of FL based anomaly detection models specifically tailored for the V-IoT domain.
    
    \item We present the concept of a Hierarchical Federated Learning (HFL) based anomaly detection model.
    
    \item We propose a system model where we integrate both the emerging DT and FL technologies. This approach provides a powerful framework for enhanced collaboration, learning, and decision-making in the V-IoT domain, ultimately leading to improved performance, efficiency, and security.

    \item We also present a use case scenario to demonstrate the feasibility of our proposed model. 

 \end{itemize}
The remainder of this paper is organized as follows. Section~\ref{related} presents the literature review on DT and FL technologies in vehicular internet domains. We also discuss about the V-IoT, DT, and FL in this section. The concept of a Hierarchical Federated Learning (HFL) based anomaly detection model is presented in Section~\ref{FL-VIoT}. Section~\ref{system} presents the proposed system model for identifying anomalies and securing V-IoT. We discuss a use case scenario to demonstrate the practical application of our proposed model in Section~\ref{use case}. Conclusion and future work are discussed in Section~\ref{conclusion}.




