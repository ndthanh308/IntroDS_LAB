\section{System Model}
\label{system}

In this section, we introduce our proposed model, which aims to identify anomalies and enhance the security of the V-IoT systems. The model comprises six distinct phases, each involving data exchange and collaboration among different entities. The overall system architecture is illustrated in Figure~\ref{fig:system}, providing a visual representation of the data flow and interaction between the components.


% Figure environment removed
The six phases of our proposed model are as follows:

\begin{itemize}
    \item \textit{Initial Phase}: During the initial phase of our proposed model, the smart vehicle begins collecting various types of data, including manufacturing data, driver perception data, and external entities data. By collecting these various types of data, the smart vehicle aims to gather comprehensive information about its own performance, the driver's behavior, and the external environment. 

    \item \textit{Functional Phase}: During the functional phase of our model, the entities within the V-IoT system transition into operational mode. This phase is divided into two sub-phases, each serving specific purposes. In the first sub-phase, the focus is on collecting data from the vehicle IoT sensors. These sensors, which are activated and operational, capture various types of information such as vehicle diagnostics, and performance metrics. The collected data is then transmitted to the vendor cloudlet, a cloud-based infrastructure specifically designed to handle V-IoT data. Simultaneously, in the second sub-phase, additional data is collected on the vendor cloudlet. This includes a wide range of data sources such as weather statistics, city regulations and policies, traffic light information, and camera data. These supplementary data sources provide contextual information about the external environment in which the vehicles operate.
    
    By collecting data from both the vehicle IoT sensors and other relevant sources on the vendor cloudlet, a comprehensive and multi-dimensional dataset is created.

    \item \textit{Analytic Phase}: Once the data is collected from the V-IoT system, it is transmitted to the simulated environment for further analysis. This phase involves the transition of data from the physical space to the simulated space, where advanced data analytics techniques are applied. In the simulated environment, a DT is developed for each entity within the V-IoT system. A DT is a virtual representation of a physical entity, in this case, the vehicles and other components of the V-IoT system. The DT is created based on the generated data collected from the previous phases. The data analytics process is then performed on the vehicle DT. Various analytical techniques and algorithms are applied to gain insights and extract valuable information from the data. These analytics help in understanding the behavior, performance, and patterns within the V-IoT system.
    
    By leveraging the DT and conducting data analytics, it becomes possible to identify and understand anomalies within the V-IoT system. Anomalies can include unusual behavior, deviations from normal patterns, or any abnormal activities that may indicate potential security or operational issues.


    \item \textit{Identifying Anomaly Phase}: In this phase, the simulated data $D_i$ from all the data nodes that has been processed and prepared in the previous phase is passed through a pipeline to feed the anomaly detection model. The data is carefully curated and transformed to be compatible with the model's input requirements. The anomaly detection model is developed using suitable machine learning algorithms and follows the Algorithm~\ref{algo2}. These algorithms are trained on the prepared data to learn the patterns and characteristics of normal behavior within the V-IoT system. The model aims to distinguish between normal and anomalous patterns based on the input data. During the training process, the model undergoes iterations to optimize its performance and enhance its ability to accurately detect anomalies. This involves adjusting the model's parameters, fine-tuning the algorithms, and validating the model's performance using appropriate evaluation metrics. Once the training is completed, the anomaly detection model is ready to be deployed and utilized. It collaborates with other models that are part of the subsequent phases, working together to enhance the accuracy and effectiveness of anomaly detection in the V-IoT system.


    
    %In this phase, the simulated data is passed through the pipeline to feed the anomaly detection model. The data is trained using suitable machine learning algorithms to develop an anomaly detection model. Once the training is completed, the model collaborates with other models that are part of the next phase

    \item \textit{Collaborative Phase}: Indeed, in this phase, the collaboration of multiple anomaly detection models take place to improve the accuracy rate of anomaly detection in the V-IoT system. By combining the weights of multiple ADM models, the overall effectiveness of anomaly detection can be significantly enhanced. Each model may have its own unique approach, algorithm, or specialization in detecting specific types of anomalies. By leveraging the strengths and capabilities of different models, a more comprehensive and robust anomaly detection system can be established. The collaboration among anomaly detection models involves exchanging information, sharing insights, and aggregating their detection results. This collaborative process allows for a holistic analysis of the system's behavior and the identification of anomalies from multiple perspectives.


  
    %As mentioned, this phase involves the collaboration of multiple Anomaly Detection Models (ADM) to improve the accuracy rate. The ADM models work together to enhance the overall effectiveness of anomaly detection in the system.

    \item \textit{Reporting and Decision Phase}: After an anomalous scenario is detected by the anomaly detection model, it is crucial to report the anomaly to the relevant stakeholders, including the user, vendor, and device. This phase plays a vital role in facilitating informed decision-making and taking necessary actions to ensure the safety and security of the automotive connected environment. Reporting the anomaly to the user is essential as it enables them to be aware of the detected anomaly and take appropriate measures. This could involve alerting the user through notifications, messages, or visual indicators, providing them with information about the anomaly and any recommended actions they should take. Notifying the vendor is also crucial as it allows them to be aware of the anomaly and take the necessary steps to address the issue. This could involve investigating the root cause of the anomaly, analyzing the data collected, and implementing corrective measures to prevent similar anomalies in the future.
    
    Overall, this phase of reporting anomalies is a critical component of the anomaly detection process in the V-IoT system. It helps to minimize the risks associated with anomalous events, enables proactive decision-making, and contributes to maintaining a safe and reliable automotive ecosystem.
    
    
    
    %After an anomalous scenario is detected using the Anomaly Detection Model (ADM), this phase involves reporting the anomaly to the user, vendor, and device. By notifying all relevant parties, this phase enables informed decision-making to ensure the safety of the automotive connected environment.
\end{itemize}

By employing our proposed system model, the V-IoT environment can detect anomalies in real-time and enable prompt responses to the system. This improves overall security and privacy, enhances the efficiency of the transportation system, and improves the driving experience for individuals.

The following section presents a use case scenario of V-IoT, where our proposed system model is employed to detect anomalies.




