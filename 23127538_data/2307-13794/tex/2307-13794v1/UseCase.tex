\section{Use Case Scenario}
\label{use case}
V-IoT is unfolding in many ways where users receive better services and take advantage of autonomous vehicle. The proposed system model where we present the integration of FL and DT, which can be used to secure V-IoT applications, for example, ITSs, cooperative autonomous driving, connected car services, smart city integration, collision avoidance systems and intelligent traffic control etc. In this section, we present a use-case scenario for securing V-IoT by developing an anomaly detection model. The Figure~\ref{fig:anomaly} shows the use case based on our proposed system model, which is discussed in Section~\ref{system}. 

In a smart city, the adoption of IoT technologies and the deployment of smart vehicles are guided by common city policies and regulations. These policies ensure uniformity and standardization across different regions within the smart city. Each region within the smart city may have multiple vendors launching their smart vehicles, contributing to the overall intelligent transportation ecosystem. The presence of multiple vendors in different regions allows for a diverse range of smart vehicles with varying features, technologies, and capabilities. These vehicles may be equipped with advanced sensors, communication systems, and intelligent algorithms to enhance their functionality and contribute to the overall smart city objectives. 


In Figure~\ref{fig:anomaly}, the depicted scenario showcases the presence of two different vendors, namely \textit{vendor-1} and \textit{vendor-2}, operating within \textit{region-1} of the V-IoT system. \textit{Vendor-1} has two smart vehicles, SV-1 and SV-3, while \textit{vendor-2} has one smart vehicle, SV-2. Additionally, \textit{region-1} consists of various data points, including sensor data, city policies and regulations, traffic lights, and weather statistics. To enable efficient data processing and anomaly detection in \textit{region-1}, a cloudlet is launched specifically for this region. The cloudlet serves as a localized computing resource that can host and deploy DTs of each smart vehicle within the region. These DTs not only receive data from their respective smart vehicles (SV-1, SV-2, and SV-3) but also incorporate data from other sources within the region, such as sensor data, city policies and regulations, traffic lights, and weather statistics.


In this use case, we deploy DTs on the cloudlet to reduce the gap between physical objects and their digital representations which are generally hosted in the cloud servers. These cloudlets are hosted by the regions and basically a regional cloud that can better cloud services nearer to the user. Cloudlets~\cite{saini2022chapter} are small-scale, mobility-enhanced cloud data centers that sit at the network's edge. The cloudlet's primary goal is to support furious resource and interactive mobile applications by delivering strong computing resources to mobile devices with reduced latency. A wireless local area network with single hop at comparatively higher speed, allows User Equipments (UEs) to connect to the computing resources in the neighboring cloudlet. By leveraging the cloudlet in \textit{region-1}, the DTs can effectively analyze and process the combined data from multiple sources. This integration of data from various entities allows for a holistic understanding of the V-IoT environment within \textit{region-1}. The DTs can leverage this comprehensive data to enhance anomaly detection capabilities, identify patterns, and detect any deviations or anomalies in the behavior of the smart vehicles or the overall V-IoT system. The deployment of DTs within the cloudlet of \textit{region-1} enables localized processing and analysis, reducing latency and enhancing real-time anomaly detection. The collaboration of the DTs with the cloudlet infrastructure facilitates efficient information exchange and enables timely response to any detected anomalies. 

DT of SV-1 within \textit{region-1} plays a crucial role in the development of the anomaly detection model. The data collected by SV-1's DT is utilized to train the initial anomaly detection model specific to \textit{region-1}. This model focuses on detecting anomalies within the local context and behavior of the vehicles and infrastructure within \textit{region-1}. To enhance the accuracy and effectiveness of the anomaly detection model, collaboration is encouraged among multiple models. In this case, the anomaly detection models of \textit{region-1} have the capability to collaborate with each other using the concept of FL. FL allows the models to share their knowledge and insights while maintaining data privacy and security. By aggregating the local models' learnings through weight aggregation techniques, a more robust and accurate anomaly detection model can be obtained.


Moreover, collaboration is not limited to models within the same region. The anomaly detection model of \textit{region-1} can also collaborate with the anomaly detection model of \textit{region-2} by using HFL concept. This collaboration is facilitated by exchanging gradients on a federated multi-cloud server. The gradients represent the model parameters that are shared and utilized to improve the models' performance collectively. By leveraging the collaboration capabilities of FL and the exchange of gradients on the federated multi-cloud server, the anomaly detection models of different regions can benefit from each other's insights and experiences. This cross-region collaboration enhances the overall effectiveness of anomaly detection in the V-IoT system by incorporating knowledge from diverse geographical areas and vehicle behaviors.

In summary, the integration of FL and the exchange of gradients enable collaboration among anomaly detection models at different levels by utilizing DT. This collaboration improves the accuracy and robustness of the models, both within the same region and across different regions, leading to more effective anomaly detection in the V-IoT system.





%The anomaly detection model of \textit{region-1} can also collaborate with anomaly detection model of \textit{region-2} by exchanging gradients on federated multi-cloud server. 








%Figure \ref{fig:anomaly} illustrates that \textit{region-1} has two different vendors, \textit{vendor-1} (SV-1, SV-3) and \textit{vendor-2} (SV-2). Region-1 has also other data points including sensor data, city policies and regulations, traffic lights and weather statistics. In this use case, we launch the cloudlet of \textit{region-1} where DT of each smart vehicle can deploy, this DT will not only recive the smart vehicle data but also other data from their region. 








%\subsection{FL for Predictive Analysis for V-IoT}
%In smart vehicle environment, predictive analytics is an effective data-driven application. V-IoT sensors and other environmental sensors such as camera, traffic light, motion sensors generate tremendous amount of data. This data is processed by Machine Learning algorithms to predict the on-road decisions in real-time and also used to predicting maintenance for vehicle. In current scenario, these autonomous vehicles use their own models for predictive analysis, however, it may provide bias, limited and imprecise results. To improve the efficiency of this application, ML models should train their data in collaborative manner. FL is a privacy-preserving approach, which is suitable to implement predictive analysis use case scenario. 



%In the vehicular domain, other data points such as traffic light, camera, city policies, regulations, and weather data merge to


%data privacy is a critical concern and we cannot merge one patient's data with another patient's DT, so training of datasets cannot be performed on patient's DT in a collaborative manner. We cannot run any anomaly detection model on a patient's DT, so we design FL based anomaly detection model and deploy this model on an edge cloudlet associated with the patient's DT. 