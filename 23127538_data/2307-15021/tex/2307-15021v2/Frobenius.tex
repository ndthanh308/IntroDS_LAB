\section{New properties of some Frobenius extensions} \label{sec:frob}

\subsection{Demazure operators and Frobenius extensions} \label{subsec:frobextdemazure}

By a famous theorem of Chevalley and Shephard--Todd \cite{ShephardTodd}, if $I \subset S$ is finitary, then $R$ is a free graded module over $R^I$ of finite rank, and $R^I$ is itself a polynomial ring. An upgraded version of this theorem was proven by Demazure \cite{Demazure}, who gave an explicit description of the basis of $R$ over $R^I$ using Demazure operators. %demonstrated that $R^I \subset R$ is a Frobenius extension.
%\LP{Before it said that Demazure proved that this is Frobenius. This seems a bit of a stretch, does not seem to consider Frobenius extension. Reworded.}


\begin{defn}\label{defn:frobext}  An inclusion $A \subset B$ of graded commutative rings is called a \emph{graded Frobenius extension of degree $\ell$} if it comes equipped with an \emph{nondegenerate} $A$-linear map $\pa^B_A \colon B \to A$ called the \emph{(Frobenius) trace}, which is homogeneous of degree $-2\ell$. Here, $\pa^B_A$ is called nondegenerate if there exist homogeneous finite bases $\{c_i\}$ and $\{d_i\}$ of $B$ as a free $A$-module such that \begin{equation} \label{tracenondeg} \pa^B_A(c_i d_j) = \delta_{ij}.\end{equation}
\end{defn}

Note that $\pa^B_A$ must be a surjective map $B \to A$: by $A$-linearity, the image is an $A$-submodule of $A$, and by \eqref{tracenondeg}, this image contains $1$.

The following theorem is a direct consequence of the results in \cite{Demazure}. An English-language exposition can be found in \cite[Th 24.36]{GBM}. We also re-explain the proof in the next section, because we will adapt it to prove new results.

\begin{thm}\label{thm.FrobI}  If $I \subset S$ is finitary and $\pa_I \colon R \to R^I$ is surjective, then $R^I \subset R$ is a graded Frobenius extension of degree $\ell(I)$, with trace map $\pa_I$.
\end{thm}

A more general version of this theorem is the following. Recall the definition of $\pa^I_J$ from \eqref{paJIdefn}.

\begin{thm}[{\cite[Th 24.40]{GBM} \label{thm:FrobExtIJ}}] If $I \subset J \subset S$ are both finitary and generalized Demazure surjectivity holds, then $R^J \subset R^I$ is graded Frobenius of degree $\ell(J) - \ell(I)$, with trace map $\pa^I_J$.
\end{thm}

 We will be recalling the techniques used to prove Theorem \ref{thm:FrobExtIJ}, because we need similar techniques to prove our new results. We begin by justifying that $\pa^J_I$ has the correct codomain, since a priori, $\pa^J_I$ is just an endomorphism of $R$. This is a consequence of Lemma~\ref{lem:sendsJtoI}, with the same proof, but we rewrite it in this simpler setting.

\begin{lem} \label{lem:paIJ} The map $\pa^I_J$ sends $R^I$ to $R^J$. Moreover, $\pa^I_J \colon R^I \to R^J$ is surjective. \end{lem}

\begin{proof} Because $\pa_I: R \to R^I$ is surjective, any element of $R^I$ has the form $\pa_I(f)$ for some $f \in R$. By \eqref{composeDem} we have $\pa^I_J(\pa_I(f)) = \pa_J(f)$, which lives in $R^J$ by Lemma~\ref{lem:imageDem}.

For any $g \in R^J$ there is some $f \in R$ such that $\pa_J(f) = g$. Then $g = \pa^I_J(\pa_I(f))$, so $g$ is in the image of $\pa^I_J$. \end{proof}

Note that
\begin{equation*} \label{compatible} \pa^I_K =\pa^J_K \circ  \pa^I_J \end{equation*}
whenever $I \subset J \subset K$.

\subsection{Almost dual bases}

We recall the proof of Theorem \ref{thm:FrobExtIJ}, following the exposition of \cite[Sec. 24.3]{GBM}.

Recall from Definition~\ref{defn:frobext} that a Frobenius extension (with trace map $\pa$) necessitates the existence of a pair of dual bases $\{c_i\}$ and $\{d_i\}$ for which $\pa(c_i d_j) = \delta_{ij}$. How would one construct dual bases for the extensions $R^I \subset R$, or the extensions $R^J \subset R^I$ when $I \subset J \subset S$? Why is the assumption of generalized Demazure surjectivity enough to imply that dual bases exist?

There is extensive literature on dual bases for the Frobenius extension $R^S \subset R$ when $R = \R[x_1, \ldots, x_n]$ and $W = S_n$. Outside of this case (and even for the geometric realization in type $A$, which is spanned by roots), there is no explicit construction of dual bases in the literature to our knowledge. It remains an important open problem to find a combinatorial construction of dual bases!

However, what is easier to find is an explicit construction for almost dual bases.

\begin{defn}\label{def:gradedfrobext} 
 



Let $A\subset B$ be a graded extension of positively graded algebras over a field $\Bbbk$, for which $B$ is free of finite rank over $A$.  Assume that both are spanned by the identity element in degree zero. Let $\mg_A$ be the graded Jacobson radical of $A$, which is the ideal spanned by all elements of strictly positive degree.

Let $\pa_A^B:B\to A$ be an $A$-linear map of degree $-2\ell$. Finite homogeneous bases $\{c_i\}$ and $\{d_i\}$ (bases of $B$ as an $A$-module) are called \emph{almost dual bases} if
\begin{equation}\label{eq:almostdual} \pa_A^B(c_i d_j) \equiv \delta_{ij} + \mg_A. \end{equation}
\end{defn}



\begin{lem} \label{lem:eqgivesADB} Let $A$ and $B$ as in \Cref{def:gradedfrobext}.
Suppose that $\{c_i\}_{i=1}^n$ and $\{d_i\}_{i=1}^n$ are subsets of homogeneous elements of 
$B$ which satisfy \eqref{eq:almostdual} and suppose $B$ is free over $A$ of rank $n$. Then $\{c_i\}$ and $\{d_i\}$  are almost dual bases.
\end{lem}

\begin{proof}
It is enough to show that $\{c_i\}$ and $\{d_i\}$ are $A$-bases of $B$. We index the element of  $\{c_i\}_{i=1}^n$  in ascending order of their degrees, i.e.,  such that $\deg(c_i)\leq \deg(c_j)$ for $i<j$. We have $\deg(d_i)=2\ell-\deg(c_i)$ and, by degree reasons, we observe that \begin{equation}\label{realmeaning}\pa_A^B(c_id_j)=\begin{cases}0&\text{ if }i<j\\
1& \text{ if }i=j\\
f\in \mg_A&\text{ if }i>j
\end{cases}.\end{equation}

By assumption there exists a basis $\{x_i\}_{i=1}^n$ of $B$ over $A$. We can write $c_i=\sum \gamma_{ij}x_j$ and $d_i=\sum \delta_{ij}x_j$, where $\gamma:=(\gamma_{ij})$ and $\delta:=(\delta_{ij})$ are $n\times n$-matrices with coefficients in $A$. 
Then \eqref{realmeaning} says that the product $\gamma X \delta^{t}$ is a lower unitriangular matrix, where $X$ is the $n\times n$ matrix with $(i,j)$-entries $\pa^B_A(x_ix_j)$. In particular, both $\gamma$ and $\delta^t$ are invertible. It follows that $\{c_i\}$ and $\{d_i\}$ are bases.
\end{proof}




\begin{lem} \label{lem:ADBgivesDB} Suppose that $\{c_i\}$ and $\{d_i\}$ are almost dual bases of $B$ over $A$. Then $\{c_i\}$ and $\{d'_i\}$ are dual bases, for some $\{d'_i\}$. Moreover, $d_i \equiv d'_i$ modulo the ideal in $B$ generated by $\mg_A$. In particular, the extension $A\subset B$ is graded Frobenius.
\end{lem}

\begin{proof}
This follows by a straightforward application of the Gram--Schmidt algorithm. We can fix an indexing of the basis $\{c_i\}_{i=1}^n$ such that $\deg(c_i)\leq \deg(c_j)$ if $i<j$ and then set $d_n'=d_n$, $d_{n-1}'=d_{n-1} -\pa_A^B(c_nd_{n-1})d_n$, %$d_{n-2}'=d_{n-2} -\pa_A^B(c_{n-1}d_{n-2})d_{n-1}-\pa_A^B(c_{n}d_{n-2})d_n-\pa_A^B(c_nd_{n_1})\pa_A^B(c_{n-1}d_{n-2})d_n$ 
and so on. 
\end{proof}

\begin{rem} Suppose $S$ is finitary. Let $(R^S_+)$ denote the ideal in $R$ generated by all elements of $R^S$ in positive degree, and let $C = R / (R^S_+)$ be the so-called \emph{coinvariant ring}. Then $\Bbbk \subset C$ is a Frobenius extension, with trace map $C \to \Bbbk$ induced by $\pa_S$. Almost dual bases for $R^S \subset R$ descend to actual dual bases for $\Bbbk \subset C$. \end{rem}

Here is Demazure's construction of almost dual bases. First a quick lemma.

\begin{lem} Suppose that $x \in W$ and $s \in \rightdes(x)$. For any $f, g \in R$ we have \begin{equation} \label{eq:canshiftsover} \pa_x(\pa_s(f)g) = \pa_x(f \pa_s(g)).\end{equation}
\end{lem}

\begin{proof} The two key ingredients to this proof are the Leibniz rule
\begin{equation*} \pa_s(fg) = \pa_s(f) g + s(f) \pa_s(g) = \pa_s(f) g + s(f \pa_s(g)), \end{equation*}
and the anti-invariance of $\pa_s$
\begin{equation} \label{antiinvarianceofpas} \pa_s(s(f)) = -\pa_s(f). \end{equation}
Both are basic consequences of the definition of $\pa_s$.

We claim that $\pa_x(s(f)) = -\pa_x(f)$ as well, when $s \in \rightdes(x)$. This follows from \eqref{antiinvarianceofpas} because $x = w.s$ for some $w \in W$, and $\pa_x = \pa_w \pa_s$. We also claim that $\pa_x \pa_s = 0$, which follows for the same reason. Thus
\begin{equation*} 0 = \pa_x(\pa_s(fg)) = \pa_x(\pa_s(f) g + s(f \pa_s(g))) = \pa_x(\pa_s(f) g) - \pa_x(f \pa_s(g)).\qedhere \end{equation*}
\end{proof}

\begin{proof}[Proof of Theorem~\ref{thm.FrobI}]
%Now fix $I \subset S$ finitary. 
By generalized Demazure surjectivity, we can assume the existence of an element $P_I \in R$ of degree $2\ell(I)$ such that
\begin{equation} \pa_I(P_I) = 1. \end{equation}
%We will apply the above lemma to $x = w_I$ below.



Let us compute $\pa_I(\pa_y(P_I) \pa_z(P_I))$ for various $y, z \in W_I$. If $\ell(y) + \ell(z) > \ell(I)$ then the result is zero for degree reasons. If $\ell(y) + \ell(z) < \ell(I)$ then the result is in $\mg_{R^I}$ for degree reasons. If $\ell(y) + \ell(z) = \ell(I)$ we perform the following analysis. Using \eqref{eq:canshiftsover} over a reduced expression for $y$, we obtain
\begin{equation*} \pa_I(\pa_y(P_I) \pa_z(P_I)) =
\pa_I(P_I \pa_{y^{-1}} \pa_z(P_I)) =
\pa_I(P_I) \pa_{y^{-1}} \pa_z(P_I) = \pa_{y^{-1}} \pa_z(P_I). \end{equation*}
In the second equality we pulled out $\pa_{y^{-1}} \pa_z(P_I)$ since it is a scalar (degree zero). If $y^{-1} . z = w_I$ then this scalar is $1$. Otherwise, the lengths do not add up and $\pa_{y^{-1}} \pa_z = 0$.

To conclude our calculation, we have
\begin{equation*} \pa_I(\pa_y(P_I) \pa_z(P_I)) \equiv \delta_{y^{-1} . z = w_I} + \mg_{R^I}. \end{equation*}
Here $\delta_{y^{-1} . z = w_I}$ is one if $y^{-1}.z = w_I$ (implying that the lengths add up) and zero otherwise.
Then by Proposition~\ref{prop:bases} below together with Lemma~\ref{lem:eqgivesADB}, the sets $\{\pa_y(P_I)\}_{y \in W_I}$ and $\{\pa_z(P_I)\}_{z \in W_I}$ are almost dual bases for $R$ over $R^I$, when indexed appropriately. To match bases we should set $z = y^\circ$ where $y^{-1} . y^\circ = w_I$.
\end{proof}




\begin{prop}\label{prop:bases}
    Let $I\subset J$ be finitary subsets of $S$. Then $R^I$ is free over $R^J$ with basis $\{\pa_I\pa_y(P_J)\}$ where $y$ ranges over minimal coset representatives in $W_I\backslash W_J$.
\end{prop}
\begin{proof}
In the case when $I=\emptyset$, this is proved in \cite[Prop. 5]{Demazure} if generalized Demazure surjectivity holds (the torsion index defined in \cite[\S 5]{Demazure} is indeed $1$ in this case).  Thus we can write any $f \in R$ uniquely as $f = \sum g_x \pa_x(P_J)$, where $g_x \in R^J$ and the sum is indexed over $x \in W_J$.

Let now $I$ be arbitrary, and $f \in R^I$. As above we can write $f =\sum g_x \pa_x(P_J)$. We have $\pa_s(f)=0$ for any $s \in I$, and $\pa_s(f) = \sum g_x \pa_s \pa_x(P_J)$. Either $\pa_s \pa_x = 0$ or $\pa_s \pa_x = \pa_{sx}$; we conclude that
\[ 0 = \pa_s(f) = \sum_{sx > x} g_x \pa_{sx}(P_J).\]
By linear independence we conclude that $g_x = 0$ whenever $sx>x$. Repeating this argument for all $s \in I$, we deduce $g_x = 0$ unless $I \subset \leftdes(x)$, so that $x = w_I . y$ for some minimal coset representative $y \in W_J$.

In conclusion, we can write 
\[ f =\sum g_{w_I.y} \pa_I \pa_y (P_J)\] and the claim follows.
\end{proof}

\begin{proof}[Proof of Theorem \ref{thm:FrobExtIJ}]
$I \subset J \subset S$ are finitary subsets.  
We claim that 
\[ \{\pa_I \pa_y(P_J)\}_{y \in W_I \backslash W_J} \qquad \text{and}  \qquad \{ \pa_I \pa_z(P_J)\}_{z \in W_I \backslash W_J}\]
satisfy \eqref{eq:almostdual}. Here we are using $\pa_y$ and $\pa_z$ to mean  $\pa_{\mi{y}}$ and $\pa_{\mi{z}}$, where $\mi{y}$ and $\mi{z}$ are the minimal coset representatives for the cosets $y$ and $z$. We do so to avoid cluttering the page with underlines. By Lemma~\ref{lem:eqgivesADB} and Proposition~\ref{prop:bases} the condition \eqref{eq:almostdual} are enough for the two sets to be almost dual bases for $R^I$ over $R^J$, when indexed appropriately.  
%are almost dual bases for $R^I$ over $R^J$, when indexed appropriately. 


First, observe that these bases consist of homogeneous elements of $R^I$, being in the image of $\pa_I$. Second, we have
\begin{equation*} \pa^I_J{\color{red}\big(}\pa_I \pa_y(P_J) \pa_I \pa_z(P_J){\color{red}\big)} = \pa^I_J{\color{red}\big(}\pa_I {\color{green}\big(}\pa_y(P_J) \pa_I \pa_z(P_J) {\color{green}\big)}  {\color{red}\big)} = \pa_J{\color{green}\big(}\pa_y(P_J) \pa_I \pa_z(P_J) {\color{green}\big)}. \end{equation*}
The first equality holds because $\pa_I \pa_z(P_J)$ is in $R^I$ and $\pa_I$ is $R^I$-linear. The second equality holds because $\pa^I_J \circ \pa_I = \pa_J$. Now apply \eqref{eq:canshiftsover} with $x = w_J$ to deduce that
\begin{equation*} \pa_J(\pa_y(P_J) \pa_I \pa_z(P_J)) = \pa_J(P_J \pa_{y^{-1}} \pa_I \pa_z(P_J)). \end{equation*}
Using arguments similar to the previous proof, we deduce that 
\begin{equation} \label{dualpairingIJ} \pa^I_J(\pa_I \pa_y(P_J) \pa_I \pa_z(P_J)) \equiv \delta_{y^{-1} . w_I . z = w_J} + \mg_{R^J}. \end{equation}
Thus to index dual bases to match, we set $z = y^\circ$ where $y^{-1} . w_I . y^\circ = w_J$. Indeed, $y \mapsto y^\circ$ is a bijection on minimal left coset representatives for $W_I \backslash W_J$.
\end{proof}

\begin{rem} To our knowledge, this proof first appeared in \cite[Sec. 24.3]{GBM}. We have filled in some of the missing details above.
\end{rem}

\subsection{New results on almost dual bases}

Suppose that $I, J \subset S$ are such that $I \cup J$ is finitary. By Theorem~\ref{thm:FrobExtIJ}, $R^I \subset R^{I \cap J}$ is a Frobenius extension, so there exist dual bases relative to $\pa^{I \cap J}_I$. However, a stronger statement is true, to whit:
\begin{equation} \label{eq:star}\tag{$\star$} \begin{array}{c}\text{There are dual bases for $R^{I \cap J}$ over $R^I$  for which}\\\text{one of the bases lives in the subring $R^{J}$.}\end{array} \end{equation}
This condition first arose in \cite{ESWFrob}, where it was also denoted $(\star)$. It is a prerequisite for the diagrammatic technology of \cite{ESWFrob} to apply to a square of Frobenius extensions.  The condition \eqref{eq:star} may fail without the assumption that $I \cup J$ is finitary. See Example~\ref{affin}. 

In Theorem~\ref{thm:dualbasisinimage} we prove a stronger result, of which \eqref{eq:star} is a special case (see Example~\ref{ex.star}). In a sequel to this paper we construct a basis for morphisms between singular Soergel bimodules. Theorem~\ref{thm:dualbasisinimage} is crucial in our proof that the purported basis is linearly independent. Some examples and additional discussion can be found in the next section.

First, a quick lemma.

\begin{lem} \label{lem:mipsendsJtoK} Let $p$ be an $(I,J)$-coset and let $K = \leftred(p)$. Then $\pa_{\mi{p}}$ sends $R^J$ to $R^K$. \end{lem}

\begin{proof} We give two quick proofs. For the first proof, note that $R^J$ is the image of $\pa_J$, and that $\mi{p}. w_J$ has $K$ in its left descent set. Then $\pa_{\mi{p}} \pa_J = \pa_K \pa_w$ for some $w \in W$, and the image lives in $R^K$.

For the second proof, recall from Theorem~\ref{thmA} that $p = [[I \supset K]] . p'$, where $p'$ is a particular $(K,J)$-coset. By Equation \eqref{upper}, we see that  $\ma{p} = (w_I w_K^{-1}).\ma{p'}$, and with Equation \eqref{mapmip} we obtain  $\ma{p'} = \mi{p} . w_J$. Thus $\pa_{p'} = \pa_{\mi{p}}$, and it is a map from $R^J$ to $R^K$ by Lemma~\ref{lem:sendsJtoI}. \end{proof}

\begin{lem} \label{lem:almostdualexpl}  Let $I, J, L \subset S$ be such that $I \cup J \subset L$ and $L$ is finitary. Pick $P_I, P_L \in R$ such that $\pa_I(P_I) = 1$ and $\pa_L(P_L) = 1$. Let $p$ be an $(I,J)$-coset contained in $W_L$, and let $K = \leftred(p)$. Let $y \in W_I$ be a minimal representative for its coset $W_{K} y$, and let $z \in W_L$ be arbitrary. We say that $z$ is \emph{dual} to $y$ (relative to $p$) if
\begin{equation} \label{eq:circdual} y^{-1} . \mi{p} . w_J . z = w_L. \end{equation}
Then $z$ is dual to $y$ if and only if $z = y^{\circ}$, where
\begin{equation}\label{eq:ycirc} y^{\circ} = w_J \mi{p}^{-1} y w_L. \qedhere\end{equation}
Moreover,
\begin{equation} \label{eq:leadstoalmostdual} \pa^{K}_I(\pa_{K} \pa_y(P_I) \cdot \pa_{\mi{p}} \pa_J \pa_z(P_L)) = \delta_{z,y^{\circ}} + \mg_{I}. \end{equation}
Here, $\mg_I$ represents the ideal of positive degree elements in $R^I$.
\end{lem}




\begin{proof}
We use an argument similar to those in the previous section.
%Fix homogeneous elements $P_I, P_L \in R$ such that $\pa_I(P_I) = 1$ and $\pa_L(P_L) = 1$. 
We consider the pairing
\[ \pa^K_I(\pa_K \pa_y(P_I) \pa_{\mi{p}} \pa_J \pa_z(P_L)).\]
Here, $y$ will range over minimal right coset representatives in $W_K \backslash W_I$; 
we have already seen in the previous section that $\{\pa_K \pa_y(P_I)\}$ is one of a pair of almost dual bases for $R^K$ over $R^I$.

By Lemma~\ref{lem:mipsendsJtoK}, $\pa_{\mi{p}} \pa_J$ has image contained in $R^K$. Since $\pa_K$ is $R^K$-linear, we have
\begin{align*}  \pa^K_I{\color{red}\big(}\pa_K \pa_y(P_I) &\pa_{\mi{p}} \pa_J \pa_z(P_L){\color{red}\big)} =
\pa^K_I{\color{red}\big(}\pa_K {\color{green}\big(}\pa_y(P_I) \pa_{\mi{p}} \pa_J \pa_z(P_L) {\color{green}\big)} {\color{red}\big)} \\ & =
\pa_I{\color{green}\big(}\pa_y(P_I) \pa_{\mi{p}} \pa_J \pa_z(P_L) {\color{green}\big)}. \end{align*}
Since $y \in W_I$, we can use \eqref{eq:canshiftsover} to prove
\begin{equation*} \pa_I(\pa_y(P_I) \pa_{\mi{p}} \pa_J \pa_z(P_L)) =
\pa_I(P_I \pa_{y^{-1}} \pa_{\mi{p}} \pa_J \pa_z(P_L)). \end{equation*}

If $y^{-1}. \mi{p} . w_J . z$ is not a reduced composition then the result is zero. So assume that $y^{-1} . \mi{p} . w_J . z = y^{-1} \mi{p} w_J z$. If $\ell(y^{-1} . \mi{p} . w_J . z) > \ell(L)$ then the result is zero for degree reasons, and if $\ell(y^{-1} . \mi{p} . w_J . z) < \ell(L)$ then the result lives in $\mg_{R^I}$ for degree reasons. If $\ell(y^{-1} . \mi{p} . w_J . z) = \ell(L)$ then we must have $y^{-1} . \mi{p} . w_J . z = w_L$, in which case
\begin{equation*} \pa_I(P_I \pa_{y^{-1}} \pa_{\mi{p}} \pa_J \pa_z(P_L)) = \pa_I(P_I) = 1. \end{equation*}
Thus we deduce that
\begin{equation} \pa^K_I(\pa_K \pa_y(P_I) \pa_{\mi{p}} \pa_J \pa_z(P_L)) = \delta_{y^{-1} . \mi{p} . w_J . z = w_L} + \mg_{R^I}. \end{equation}

Since $y$ runs over the minimal right coset representatives for $W_K\backslash W_I$, the element $y^{-1}$ runs over the minimal left coset representatives for $ W_I/W_K$. Thus by \cite[Lemma 2.12]{EKo} the composition $x = y^{-1}.\mi{p}.w_J$ is always reduced, and ranges amongst all elements $x$ of $p$ for which $\mi{p} w_J \le x \le \ma{p}$. Since $x \le \ma{p} \le w_L$, there exists a unique $z$ such that $x.z = w_L$. Moreover, we have $z=y^\circ$ for $y^\circ=w_J\mi{p}^{-1}yw_L$.
\end{proof}

% Thus $\{\pa_K \pa_y(P_I)\}$ and $\{\pa_{\mi{p}} \pa_J \pa_{y^\circ}(P_L)\}$ are almost dual bases. 
% Here $y$ ranges among minimal coset representatives for right cosets in $W_K \backslash W_I$, and $y^{-1} \mi{p} w_J . y^{\circ} = w_L$.  More explicitly, we have \eqref{eq:ycirc} as claimed.


\begin{thm}\label{thm:almostdualinimage}
Let $I, J, L \subset S$ be such that $I \cup J \subset L$ and $L$ is finitary. Let $p$ be an $(I,J)$-coset contained in $W_L$, and let $K = \leftred(p)$. Then 
\begin{equation}\label{eq:almostdualbasis}
    \{\pa_K \pa_y(P_I)\},\quad \{\pa_{\mi{p}} \pa_J \pa_{y^\circ}(P_L)\}
\end{equation} are almost dual bases of $R^K$ over $R^I$, where $y$ ranges among minimal coset representatives for right cosets in $W_K \backslash W_I$ and $y^{\circ}$ is as in \eqref{eq:ycirc}.
% \begin{equation}\label{eq:ycirc} y^{\circ} = w_J \mi{p}^{-1} y w_L. \end{equation}
\end{thm}
\begin{proof}
It follows by \Cref{lem:almostdualexpl} that the two sets satisfy \eqref{eq:almostdual}. We also notice that $|W_K\backslash W_I|=\rank_{R^I}R^K$ by \Cref{prop:bases}, so they are almost dual bases by \Cref{lem:eqgivesADB}.
\end{proof}


\begin{thm} \label{thm:dualbasisinimage}
Let $I, J, L \subset S$ be such that $I \cup J \subset L$ and $L$ is finitary. Let $p$ be an $(I,J)$-coset contained in $W_L$, and let $K = \leftred(p)$. Then there exists a pair of dual bases for $R^K$ over $R^I$, where one of the bases is in the image of $\pa_{\mi{p}}$ when viewed as a map $R^J \to R^K$.
\end{thm}
\begin{proof}
    By Lemma~\ref{lem:ADBgivesDB}, the claim follows from Theorem~\ref{thm:almostdualinimage} since the second basis in \eqref{eq:almostdualbasis} is in the image of $\pa_{\mi{p}}$.
\end{proof}


\subsection{Additional remarks and examples}

First, some examples of Theorem~\ref{thm:dualbasisinimage}.

\begin{ex}
Let $W=S_4$ and $S=\{s_1,s_2,s_3\}$. Let $I=\{s_1,s_3\}$ and consider the $(I,I)$-coset $p$ with minimal element $s_2$, so that $\partial_{\mi{p}}=\partial_2$. Let $P_S\in R$ with $\partial_S(P_S)=1$ and let $P_I=\partial_{w_I^{-1}w_S}(P_S)=\partial_{2132}(P_S)$. Notice that $\partial_I(P_I)=\partial_I\partial_{w_I^{-1}w_S}(P_S)=1$.

We have $\leftred(p)=\emptyset$. We can apply $\partial_{\mi{p}}$ to a strict subset of the basis $\{\partial_{13}\partial_z(P_S)\}$ of $R^I$ over $R^S$ (where $z$ ranges over $W_I \backslash W$) to obtain a  basis of $R$ over $R^I$. In fact, we have
\begin{align*}
\partial_2\partial_{13}\partial_2(P_S) &=P_I\\
\partial_2\partial_{13}\partial_{21}(P_S) =\partial_3\partial_{2132}(P_S)&=\partial_3(P_I)\\
\partial_2\partial_{13}\partial_{23}(P_S) =\partial_1\partial_{2132}(P_S)&=\partial_1(P_I)\\
\partial_2\partial_{13}\partial_{231}(P_S) =\partial_{13}\partial_{2132}(P_S)&=\partial_{13}(P_I)=1.
\end{align*}
\end{ex}

\begin{ex}\label{ex.star} Let $I, J \subset S$ be such that $I \cup J$ is finitary, and let $p$ be the $(I,J)$-coset containing the identity. Then $\leftred(p) = I \cap J$ and $\pa_{\mi{p}}$ is the identity map. Thus Theorem~\ref{thm:dualbasisinimage} is the same statement as \eqref{eq:star} in this case.
\end{ex}

One can think of Theorem~\ref{thm:dualbasisinimage} as being the correct generalization of \eqref{eq:star} beyond identity double cosets.

For sake of discussion, let us isolate the conclusion of Theorem~\ref{thm:dualbasisinimage} from its hypotheses.% Let $p$ be an $(I,J)$-coset and . Consider the following statement.
\begin{equation}\label{eq:starstar}\tag{$\star \star$}\begin{array}{c}
\text{For all $(I,J)$-cosets $p$, letting $K=\leftred(p)$, there exists a}\\
\text{pair of dual bases for $R^K$ over $R^I$ where one of the bases is }\\
\text{in the image of $\pa_{\mi{p}}$ when viewed as a map $R^J \to R^K$.}
\end{array}\end{equation}

Our final task is to discuss potential generalizations of Theorem~\ref{thm:dualbasisinimage} to infinite Coxeter groups. Recall that \eqref{eq:star} came with the hypothesis that the ambient parabolic subgroup $I \cup J$ is finitary, and \eqref{eq:starstar} was proven in Theorem~\ref{thm:dualbasisinimage}  under the hypothesis that the ambient parabolic subgroup $L$ is finitary. Both \eqref{eq:star} and \eqref{eq:starstar} make sense without these finiteness hypotheses, but may or may not be true depending on the realization. We are interested in additional hypotheses on the realization which might imply these conclusions.

For illustration let us discuss \eqref{eq:star} in the example of a dihedral group with simple reflections $s$ and $t$. We focus on the statement
\begin{equation}\label{eq:diehdralstar}\tag{$\star_{s,t}$} \qquad \begin{array}{c}\text{There are dual bases for $R$ over $R^s$  for which}\\\text{one of the bases lives in the subring $R^t$.}\end{array} \end{equation}

\begin{lem} In the context above of the dihedral group, \eqref{eq:diehdralstar} is equivalent to the existence of a polynomial $P$ with $\pa_s(P) = 1$ and $\pa_t(P) = 0$, i.e. the existence of a \emph{fundamental weight}. \end{lem}

\begin{proof} If $P$ exists then $\{1,P\}$ is a basis for $R$ over $R^s$ living in $R^t$. The dual basis is $\{\al_s - P, 1\}$. Note that $\{\al_s - P, 1\}$ does not live in $R^t$ for $m_{st} > 2$. Only one of the two bases is expected to live in $R^t$.

The converse is obvious. \end{proof}

\begin{rem} Note that Demazure surjectivity implies the existence of a polynomial $P$ for which $\pa_s(P) = 1$, but not one which simultaneously satisfies $\pa_t(P) = 0$.\end{rem}

Whether or not \eqref{eq:diehdralstar} holds for the infinite dihedral group depends on the choice of realization.

\begin{ex} Any realization of a finite dihedral group is also a realization of the infinite dihedral group. Here \eqref{eq:diehdralstar} holds by Theorem~\ref{thm:dualbasisinimage}. \end{ex}

\begin{ex}\label{affin} Consider the realization spanned by simple roots, whose Cartan matrix is the usual affine Cartan matrix of type $\tilde{A}_1$. Then $\alpha_s + \alpha_t \in R^{st}$, and spans the linear terms in both $R^s$ and $R^t$. Consequently, there is no fundamental weight, and \eqref{eq:diehdralstar} fails. \end{ex}

Let us isolate a condition on the realization which is equivalent to \eqref{eq:diehdralstar} and separates these two examples.

\begin{lem} \label{lem:starifHsnotHt} Continuing the notation above, let $H_s$ and $H_t$ denote the hyperplanes in the realization $V$ which are fixed by $s$ and $t$ respectively. Then \eqref{eq:diehdralstar} holds if and only if $H_s \ne H_t$. \end{lem}

\begin{proof} If $H_s = H_t$ then no fundamental weight can exist, as e.g. in Example~\ref{affin}.

If $H_s \ne H_t$ then $V = H_s + H_t$, so $R$ is generated by the subrings $R^s$ and $R^t$. Then the linear polynomials in $R$ are spanned by those in $R^s$ and $R^t$. By Demazure surjectivity there is some linear polynomial $f + g$ with $\pa_s(f+g) = 1$, where we assume $f \in R^s$ and $g \in R^t$. Then $\pa_s(g) = 1$ as well, so $g$ is a fundamental weight. \end{proof}

Thus the distinctness of the hyperplanes $H_s$ and $H_t$ is equivalent to \eqref{eq:diehdralstar}, which is a special case of \eqref{eq:starstar}. This motivates why one might expect a relationship between \eqref{eq:starstar} and reflection faithfulness.

\begin{defn} For a reflection $r$ (i.e. a $W$-conjugate of a simple reflection), let $H_r \subset V$ denote the subspace of $V$ fixed by $r$.  A realization is called \emph{reflection faithful} if $H_r$ is a hyperplane for all reflections $r$, and $H_r \ne H_{r'}$ whenever $r \ne r'$. \end{defn}

This condition was the original condition imposed by Soergel for Soergel bimodules to be well-behaved \cite{Soe07}, and is common in the literature (e.g., \cite{EWhodge}). For example, Libedinsky \cite{LL} uses reflection faithfulness to guarantee that double leaves form a basis for morphisms between Bott-Samelson bimodules.  Just as reflection faithfulness is the condition commonly used to guarantee the proper behavior of Soergel bimodules, \eqref{eq:starstar} is what we use in future work to guarantee proper behavior of singular Soergel bimodules.

\begin{conj} \label{conj:dualbasisinimage} Suppose the realization is reflection faithful. Then \eqref{eq:starstar} holds for all double cosets $p$. \end{conj}

\begin{rem} In the dihedral setting, one thinks of Conjecture~\ref{conj:dualbasisinimage} as being a melange of Lemma~\ref{lem:starifHsnotHt} with the additional ingredient of conjugation by $\mi{p}$. \end{rem}

\begin{rem} Perhaps reflection faithfulness is equivalent to \eqref{eq:starstar}, though it seems likely that a weaker condition would suffice. It would be interesting to find a condition on the distinctness of certain hyperplanes which is equivalent to \eqref{eq:starstar} in general. \end{rem}


