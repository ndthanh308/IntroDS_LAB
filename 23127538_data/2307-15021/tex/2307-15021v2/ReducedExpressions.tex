\section{Singular reduced expressions and the Matsumoto theorem} \label{sec:singularrex}

This chapter is a review of the main results of \cite{EKo} on singular expressions.
The only thing in this chapter which is not contained in \cite{EKo} is Notation \ref{dotforcosets}.


\subsection{Some notation} \label{ssec:notation}
 
Throughout we fix a Coxeter system $(W,S)$. If $I\subset S$, then $W_I$ is the parabolic subgroup generated by $I$. If $W_I$ is finite we call $I$ \emph{finitary}, and we denote its longest element by $w_I$. We let $\ell$ denote the length function on $W$ and we set $\ell(I):=\ell(w_I)$. 



If no confusion is possible, we use shorthand for subsets $I \subset S$. We might write $stu$ rather than $\{s,t,u\} \subset S$, and $\hat{s}$ instead of $S \setminus \{s\}$. When $S = \{s_1, \ldots, s_n\}$, we may write $\{2,4,5\}$ or $245$ rather than $\{s_2, s_4, s_5\}$, and $\hat{2}$ instead of $S \setminus \{s_2\}$. If $s \notin I$ then we write $Is$ for $I \sqcup \{s\}$ and if $s \in I$ we write $I \setminus s$ for $I \setminus \{s\}$. 


For $J, I\subset S$ finitary, a \emph{$(I,J)$-coset} is an element $p$ in 
 $W_{I}\backslash W/W_{J}$. When we write ``the coset $p$" we  mean the triple $(p,I,J)$. It might happen that  $(p,I,J)\neq (p',I',J')$, even though $p=p'$ as subsets of $W$, and we distinguish between $p$ and $p'$ in this case. If $p$ is a $(I,J)$-coset we denote by $\overline{p}\in W$ and $\underline{p}\in W$ the unique maximal and minimal elements in the Bruhat order in the set $p$. 
 

For $x, y, z \in W$ we write $x.y = z$ to indicate the statement that $xy = z$ and $\ell(x) + \ell(y) = \ell(z)$. Thus we write $x.y = xy$ to imply that the lengths add when $x$ and $y$ are multiplied.  We may say that $x.y$ is a \emph{reduced composition}, in analogy to reduced expressions.

 
\subsection{Expressions and reduced expressions}
 
 \begin{defn}  A (singular) \emph{multistep  expression} is a sequence of finitary subsets of $S$ of the form 
 \begin{equation}\label{multistep}
 L_{\bullet}=[[ I_0\subset K_1\supset I_1\subset K_2 \supset \cdots \subset K_m\supset I_m]].
 \end{equation}
 By convention, if we write a multistep expression as $[[K_1 \supset I_1 \subset \cdots]]$, this means that $I_0 = K_1$,
and similarly $[[\cdots \subset K_m]]$ means that $I_m = K_m$. \end{defn}

Internal equalities in multistep expressions can be absorbed harmlessly. For example, the multistep expressions
\begin{equation} [[ \ldots K_i \supset I_i = K_{i+1} \supset I_{i+1} \subset \ldots ]] \quad \text{ and } \quad [[ \ldots K_i \supset I_{i+1} \subset \ldots ]] \end{equation}
are equivalent for all practical purposes, such as for the formulas \eqref{reduced} and \eqref{represents}. One may as well assume that multistep expressions have no internal equalities.

\begin{defn} A (singular) \emph{(singlestep) expression} is a sequence of finitary subsets of $S$ of the form
\begin{equation} I_{\bullet} = [I_0, I_1, \ldots, I_d], \end{equation} such that, for each $1 \le k \le d$, either $I_k = I_{k-1}s$ or $I_k = I_{k-1} \setminus s$ for some $s \in S$. The number $d$ is called the \emph{width} of $I_{\bullet}$. \end{defn}

To each singlestep expression one can associate one\footnote{Or many if one allows internal equalities in \eqref{multistep}.} multistep  expression by remembering its local maxima and minima. Singlestep expressions are more useful for purposes of generators and relations, while multistep expressions get to the heart of the matter faster.

\begin{defn} Let $L_{\bullet}$ be a multistep expression of the form \eqref{multistep}.% We define the \emph{length} of $L_\bullet$ as
%\begin{equation}\label{explength}\ell(L_\bullet):=\ell(w_{K_1})-\ell(w_{I_1})+\ell(w_{K_2})-\cdots -\ell(w_{I_{m-1}})+\ell(w_{K_m}).\end{equation}
Temporarily, let $w_{L_{\bullet}} \in W$ denote the element
\begin{equation} w_{L_{\bullet}} := w_{K_1}w_{I_1}^{-1}w_{K_2}\cdots w_{I_{m-1}}^{-1}w_{K_m}. \end{equation}
We say that $L_\bullet$ is \emph{reduced} if
 \begin{equation}\label{reduced}
\ell(w_{L_{\bullet}})=\ell(K_1)-\ell(I_1)+\ell(K_2)-\cdots -\ell(I_{m-1})+\ell(K_m)..
  \end{equation}
In this case we say that $L_{\bullet}$ \emph{expresses} the $(I_0, I_m)$-coset
\begin{equation}\label{represents}
p = W_{I_0}w_{L_{\bullet}}W_{I_m},    
\end{equation}
and we write $L_{\bullet} \expr p$. We write $L_{\bullet} \expr M_{\bullet}$ if  $L_{\bullet}$ and $M_{\bullet}$ express the same double coset $p$.
We say that a (singlestep) expression is \emph{reduced} if the corresponding multistep expression is. \end{defn}

Let us note  that if $L_{\bullet}$ is a reduced expression for $p$, then the maximal element $\ma{p}$ satisfies
\begin{equation} \label{map} \ma{p} = w_{L_{\bullet}} = w_{K_1} . (w_{I_1}^{-1}w_{K_2}) . (\cdots) . (w_{I_{m-1}}^{-1}w_{K_m}). \end{equation}
We stop using the notation $w_{L_{\bullet}}$, and instead start using $\ma{p}$.

By \cite[Proposition 2.31]{EKo} (building\footnote{Part of the work done in \cite{EKo} is to prove that the definition of a reduced expression given here, first stated in \cite{EKo}, is equivalent to the more technical definition given by Williamson.} on a result \cite[Proposition 1.3.4]{WillThesis} of Williamson), any double coset $p$ has a reduced expression.

\begin{rem} In formulas like the above we often write $w_K w_I^{-1}$ when $I \subset K$. Since $w_I$ is an involution, the inverses are not necessary. We include them when they are helpful for computing the length of an element, e.g. $(w_K w_I^{-1}) . w_I = w_K$. \end{rem}

\begin{rem} For more discussion of non-reduced expressions, see \S\ref{ssec:nonreduced}. \end{rem}

\subsection{Concatenation}

\begin{defn} If $I_{\bullet} = [I_0, \ldots, I_d]$ and $K_{\bullet} = [K_0, \ldots, K_e]$ then these expressions are \emph{composable} if $I_d = K_0$, in which case their composition or concatenation is
\[ I_{\bullet} \circ K_{\bullet} := [I_0, \ldots, I_d, K_1, \ldots, K_e]= [I_0, \ldots, I_{d-1}, K_0,  \ldots, K_e].\] \end{defn}

\begin{prop}[{\cite[Proposition 4.3]{EKo}}]\label{prop:concat} Let $I_{\bullet} \expr p$ and $K_{\bullet} \expr q$ be composable reduced expressions, with $J := K_0 = I_d$. Then $I_{\bullet} \circ K_{\bullet}$ is reduced if and only if
\[ \ma{p} w_J^{-1} \ma{q} = \ma{p} . (w_J^{-1} \ma{q}) = (\ma{p} w_J^{-1}) . \ma{q}. \] 
Moreover\footnote{This last statement is only stated implicitly in \cite{EKo}, and follows immediately from unraveling the equality $p * q = r$.}, $I_{\bullet} \circ K_{\bullet} \expr r$ where 
\begin{equation}\label{upper} \ma{r} = \ma{p} w_J^{-1} \ma{q}. \end{equation}
\end{prop}

\begin{notation} \label{dotforcosets} This notation is not found in \cite{EKo}. For an $(I,J)$-coset $p$, a $(J,K)$-coset $q$, and a $(I,K)$-coset $r$, let us write $p . q = r$ if we can find $I_{\bullet}$ and $K_{\bullet}$ such that
\[ I_{\bullet} \expr p, \quad K_{\bullet} \expr q, \quad I_{\bullet} \circ K_{\bullet} \expr r\]
are all reduced expressions. We say that $r$ is a \emph{reduced composition} of $p$ and $q$. In this case we also write $I_{\bullet} \circ K_{\bullet} = I_{\bullet} . K_{\bullet}$. By the previous proposition, $p . q = r$ if and only if $\ma{r} = \ma{p}. (w_J^{-1} \ma{q}) = (\ma{p} w_J^{-1}) . \ma{q}$.

We also permit ourselves to mix and match cosets and their reduced expressions using this notation. If $I_{\bullet}$ is a reduced expression for $p$, then $I_{\bullet} . q$  represents a reduced composition of $I_{\bullet}$ with some reduced expression for $q$ (the choice being, presumably, not relevant). We may write $I_{\bullet} . q \expr p . K_{\bullet} \expr r$ as well.
\end{notation}

\begin{defn}
We say that $I_{\bullet}$ is a \emph{contiguous subword} of $L_{\bullet}$ if there exist expressions $J_{\bullet}$ and $K_{\bullet}$ such that $L_{\bullet} = J_{\bullet} \circ I_{\bullet} \circ K_{\bullet}$.
\end{defn}

\begin{prop}[{\cite[Proposition 3.12]{EKo}}] \label{prop:reverse}  A contiguous subword of a reduced expression is reduced. Reversing the order of a reduced expression yields a reduced expression.\end{prop}





 \subsection{Redundancy and the core} \label{ssec:core}

\begin{defn} Let $p$ be a (finitary) $(I,J)$-coset. The \emph{left redundancy of $p$}, denoted $\leftred(p)$, is the subset of $I$ defined as follows:
\begin{equation} \leftred(p) := I \cap \mi{p} J \mi{p}^{-1}. \end{equation}
The \emph{right redundancy of $p$} is a subset of $J$ defined similarly:
\begin{equation} \rightred(p) := J \cap \mi{p}^{-1} I \mi{p}. \end{equation}
\end{defn}

There is a surjective map $W_I \times W_J \to p$ sending $(x,y) \mapsto x \mi{p} y$. The fibers of this map are torsors over the (left or right) redundancy. We have
\begin{equation} \label{mapmip} \ma{p} = (w_I w_{\leftred(p)}^{-1}) . \mi{p} . w_J = w_I . \mi{p} . (w_{\rightred(p)}^{-1} w_J).  \end{equation}
For details, see \cite[Lemma 2.12]{EKo}. The ideas above are originally due to Kilmoyer and Howlett.

\begin{defn}\label{core} Let $p$ be a $(I,J)$-coset. The \emph{core} of $p$ is the $(\leftred(p),\rightred(p))$-coset $p^{\core}$ with minimal element $\mi{p^{\core}}=\mi{p}$. \end{defn}





In \cite[Lemma 4.27]{EKo}, it is proven that the core is well-defined, and that $\leftred(p^{\core}) = \leftred(p)$ and $\rightred(p^{\core}) = \rightred(p)$. The main theorem of Section 4.9 in \cite{EKo} is that $p$ has a reduced expression which factors through its core.

\begin{thm}[{\cite[Cor. 4.27 and Prop. 4.28]{EKo}}]\label{thmA}\label{thm:core} Let $p$ be a $(I,J)$-coset. Then $p$ has a reduced expression of the form $[I,I\setminus s,\ldots,J]$ if and only if $s\not \in \leftred(p)$. Moreover, for any reduced expression $M_{\bullet}$ of $p^{\core}$, the (multistep) expression
\begin{equation} \label{throughcore} [[I \supset \leftred(p)]] \circ M_{\bullet} \circ [[\rightred(p) \subset J]] \end{equation}
is a reduced expression for $p$. \end{thm}

Using Notation \ref{dotforcosets} we may write
\begin{equation} p \expr [[I \supset \leftred(p)]] . p^{\core} . [[\rightred(p) \subset J]]. \end{equation}



\subsection{Addable and removable elements}

\begin{defn} For an element $w \in W$, let $\leftdes(w) = \{ s \in S \mid sw < w \}$, commonly called the \emph{left descent set} of $w$. Similarly, $\rightdes(w) = \{ s \in S \mid ws < w \}$ is the \emph{right descent set}.

For a double coset $p$, let $\leftdes(p)$ and $\rightdes(p)$, the \emph{left and right descent sets} of $p$, denote the (ordinary) left and right descent sets of $\ma{p}$. \end{defn}

Let $p$ be an $(I,J)$-coset. By \cite[Lemma 2.12(5)]{EKo}, $I \subset \leftdes(p)$ and $J \subset \rightdes(p)$. Moreover, for any $w \in p$, $w = \ma{p}$ if and only if $I \subset \leftdes(w)$ and $J \subset \rightdes(w)$.

\begin{thm}[{\cite[Cor. 4.20 and Prop. 4.21]{EKo}}] \label{thm:addremove} Let $p$ be an $(I,J)$-coset. Then $p$ has a reduced expression of the form $[I,Is, \ldots, J]$ if and only if $s \in \leftdes(p) \setminus I$. It has a reduced expression of the form $[I,\ldots,Js,J]$ if and only if $s \in \rightdes(p) \setminus J$. Moreover, $p$ has some reduced expression of the form 
\[[[I \subset \leftdes(p)]] . N_\bullet.[[\rightdes(p)\supset J]],\] where $N_\bullet$ is any reduced expression of the $(\leftdes(p),\rightdes(p))$-coset containing $p$.
\end{thm}

\begin{proof} See \cite[Corollaries 4.20 and 4.27, Propositions 4.21 and 4.28]{EKo}. The statement about $[I,\ldots,Js,J]$ is not found in \cite{EKo}, but can be easily deduced from the statement about $[I,Is,\ldots,J]$ by applying Proposition \ref{prop:reverse} to reverse the order of the word. \end{proof}

In other words, every double coset $p$ admits a reduced expression which begins and ends going up as much as possible, i.e. to the descent sets of $p$. On the other hand, by \Cref{thm:core} every double coset $p$ admits a reduced expression which begins and ends going down as much as possible, i.e. to the redundancy sets of $p$.








\subsection{Braid relations and Matsumoto's theorem}

Another convenient way to keep track of an expression $I_{\bullet} = [I_0, \ldots, I_d]$ is to write down not the parabolic subsets $I_k$, but the sequence of simple reflections which were added and removed. If $I_k = I_{k-1} s$ then we write $+s$, and if $I_k = I_{k-1} \setminus s$ then we write $-s$. For example, we have
\begin{equation} \label{plusminusnotation} [I + s + t - u + v] := [I, Is, Ist, Ist\setminus u, Istv \setminus u]. \end{equation}
By convention, this notation implies that $s, t \notin I$ and $u \in Ist$, and that all the subsets of $S$ appearing in the expression are finitary.

Now we list the \emph{singular braid relations}, which are local transformations one can apply to contiguous subwords of reduced expressions.
\begin{subequations} \label{braidrelns}
\begin{itemize}
    \item The \emph{up-up relation} is 
\begin{equation} \label{upup} [L+s+t] \expr [L+t+s] \end{equation}
for any $s, t \notin L$ with $Lst$ finitary.
\item The \emph{down-down relation} is
\begin{equation} \label{downdown} [L-s-t] \expr [L-t-s]\end{equation}
for any $s, t \in L$ with $L$ finitary.
\item The \emph{commuting (switchback) relation} is
\begin{equation} [L + s - t] \expr [L - t + s] \end{equation}
for any $s \notin L$ and $t \in L$ satisfying
\begin{itemize} \item $Ls$ is finitary, and
\item $s$ and $t$ are in different connected components of $Ls$. \end{itemize}
\item The \emph{switchback relation} is
\begin{equation} \label{switchback} [L + s - t] \expr [L - u_1 + u_0 - u_2 + u_1 - \cdots + u_{\delta-2} - u_{\delta} + u_{\delta-1}], \end{equation}
 for $s \notin L$, $Ls$ finitary, and $t \in L$ satisfying that $t \ne w_{Ls}  s  w_{Ls}$. Here $(u_i)$ is the ``rotation sequence'' (see below) associated to the triple $(Ls,s,t)$, with $u_0 = s$ and $u_{\delta} = t$.
\end{itemize}
\end{subequations}

The switchback relation is complicated enough to merit additional discussion, but let us first state the main theorem about braid relations.

\begin{thm}[{Singular Matsumoto Theorem \cite[Thm. 5.30]{EKo}}] \label{thm:matsumoto}  Any two reduced expressions for the same double coset are related by a sequence of singular braid relations. \end{thm}

Let us describe the switchback relation succinctly in type $A$. If $s$ and $t$ are not in the same connected component of $Ls$ then the switchback relation becomes merely the commuting relation. First consider the case when $Ls = S$ and $W = S_n$.


\begin{ex} Let $W = S_n$ and $S = \{s_1, \ldots, s_{n-1}\}$ be the usual simple reflections. Let $s = s_a$ and $t = s_b$, and $L = S \setminus s_a$. If $a + b = n$ then $t = w_S s  w_S$ and there is no switchback relation. Otherwise, if $a+b < n$ we set $c = a+b$, and if $a + b > n$ we set $c = a+b - n$. Then the switchback relation is
\begin{equation} \label{typeAswitchback} [L + s_a - s_b] \expr [L - s_c + s_a - s_b + s_c]. \end{equation}
Both sides are reduced expressions for the $(\hat{s_a}, \hat{s_b})$-coset containing $w_S$.
\end{ex}

The general case in type $A$ is below.

\begin{ex} Let $W = S_m$ and $Ls$ be an arbitrary parabolic subset. Let $I$ denote the connected component of $Ls$ containing $s$, and suppose $t \in I$. Then $W_I \cong S_n$ for some $n \le m$, and the switchback relation for $s$ and $t$ is precisely as in the previous example. \end{ex}

In arbitrary types, the rotation sequence is a particular walk around the Dynkin diagram, using conjugation by longest elements of parabolic subgroups, see \cite[Def. 5.10]{EKo}. The number $\delta$ is $1$ for the case of the commuting switchback relation. Otherwise, $\delta$ is equal to $2$ in type $A$, at most $3$ in types $BCD$, and caps out at $\delta = 11$ in type $H_4$. For explicit details on the switchback relation see \cite[Chapter 5]{EKo}.


\subsection{Non-reduced expressions} \label{ssec:nonreduced}

In \cite[Section 1.3]{EKo} the authors define the singular Coxeter monoid, a category where the morphisms are double cosets, and where a singlestep expression represents a composition of generating morphisms. In this way, any expression expresses some double coset, see \cite[Def. 1.27 and Eq. 1.15]{EKo}. When the expression is not reduced, the formula for this double coset might not agree with \eqref{represents}. Reduced expressions are exactly the expressions of minimal length expressing a coset, for a notion of length defined in \cite[Section 3.4]{EKo}.

 % \begin{defn} Given a multistep expression $L_{\bullet}$ of the form \eqref{multistep}, the \emph{length} of $L_{\bullet}$ is
% \begin{equation} \label{length} \ell(L_{\bullet}) = \ell(w_{K_1})-\ell(w_{I_1})+\ell(w_{K_2})-\cdots -\ell(w_{I_{m-1}})+\ell(w_{K_m}). \end{equation}
% \end{defn}

% \begin{prop} An expression for a double coset $p$ is a reduced expression if and only if it has minimal length amongst all expressions for $p$. \end{prop}


The \emph{(singular) $*$-quadratic relation} is
\begin{equation} \label{starquad} [L - s + s] \expr [L] \end{equation}
for $s \in L$. The left-hand side is a non-reduced expression, expressing the same double coset as the right-hand side. The main result on non-reduced expressions is below.

\begin{thm}[{\cite[Thm. 5.31]{EKo}}] \label{thm:notreduced}  Any expression $I_{\bullet}$ can be transformed into a reduced expression by applying braid relations and applying the $*$-quadratic relation in only one direction, replacing the left-hand side of \eqref{starquad} with the right-hand side. \end{thm}

