\section{Introduction}


This paper tells a completely algebraic story, though our motivations come from geometry and categorification. For sake of simplicity we focus on type $A$ in this introduction, but we work with arbitrary Coxeter groups in the body of the text, with connections to geometry (as stated) only for Weyl groups.

Let $R = \mathbb{C}[x_1, \ldots, x_n]$ be the polynomial ring in $n$ variables, equipped with its action of $W = S_n$. Let $R^W$ denote the subring of $W$-invariant polynomials, and for each $1 \leq i \leq n-1$, let $R^i$ denote the subring of polynomials invariant under the simple reflection $s_i := (i,i+1)$. There is a \emph{Demazure operator} or \emph{divided difference operator}
\begin{equation} \label{eq:paidefintro} \pa_i \co R \to R^i, \qquad \pa_i(f) = \frac{f - s_i(f)}{x_i - x_{i+1}}. \end{equation}
In similar fashion, one also has an operator
\begin{equation} \pa_W \co R \to R^W, \qquad \pa_W(f) = \frac{\sum_{w \in W} (-1)^{\ell(w)} w(f)}{\Pi_{i < j} (x_i - x_j)}. \end{equation}
These operators are now ubiquitous in combinatorics and representation theory.

Demazure operators appear when considering pushforwards in cohomology, along forgetful maps between partial flag varieties. There is also a version using equivariant cohomology of a point, for various parabolic subgroups. The geometric underpinning of Demazure operators is well-trodden; instead of rehashing it here, we refer the reader unfamiliar with this story to \cite{BGG,Brion}.

The key observation with which we begin our story is a foundational result of Demazure \cite{Demazure}: that $\pa_W$ lies within the subalgebra generated by $\pa_i$. To make this
statement more precise, let $\iota_i \co R^i \to R$ denote the inclusion map, and let $D_i := \iota_i \circ \pa_i$. Then $D_i$ satisfies the same formula as \eqref{eq:paidefintro},
but with different codomain, and the operators $D_i$ can be composed. Similarly, we set $D_W \co R \to R$ to be $\pa_W$ composed with the inclusion map $\iota_W \co R^W \to R$. The
subalgebra in $\End_{R^W}(R)$ generated by $D_i$ is called the \emph{nilCoxeter algebra}. Demazure proved a presentation for the nilCoxeter algebra: the relations are
\begin{equation} D_i^2 = 0, \qquad D_i D_{i+1} D_i = D_{i+1} D_i D_{i+1}, \qquad D_i D_j = D_j D_i \text{ if } j \ne i \pm 1. \end{equation} 
Because these relations are similar to
those of the symmetric group itself (e.g. this presentation falls into the framework of generalized Hecke algebra, see \cite[Chapter 7]{Humphreys}), the nilCoxeter algebra has a
basis $\{D_w\}_{w \in W}$ where $D_w$ is defined by applying various operators $D_i$ along a reduced expression for $w$. Moreover, Demazure proves that $D_W = D_{w_0}$, the basis
element associated to the longest element of $W$.

For any subset $I \subset S = \{1, \ldots, n-1\}$ of the simple reflections, there is a parabolic subgroup $W_I$, and a subring $R^I$ of $W_I$-invariant polynomials. By
the same token, one has an operator $\pa_I \co R \to R^I$ and $D_I \co R \to R$, and we have $D_I= D_{w_I}$ where $w_I$ is the longest element of $W_I$.

By rephrasing $\pa_i$ as an endomorphism $D_i$ of $R$, one has rudely cut the ring $R^i$ out of the picture. But $D_i = \iota_i \circ \pa_i$ is a composition of two distinct maps of $R^W$-modules. Geometrically, $\pa_i$ is a proper pushforward in cohomology, while $\iota_i$ is a pullback. When a linear transformation factors, it is often helpful to zoom in on the compositional factors independently. Indeed, Demazure's relation $D_i^2 = 0$ is a consequence of a more zoomed-in relation: $\pa_i \circ \iota_i = 0$.
The braid relations for $D_i$ are also consequences of more zoomed-in relations, called the \emph{switchback relations} in \cite{EKo}.

In this paper we are interested in studying the following category, a generalization of the nilCoxeter algebra. It describes all linear maps one could obtain by composing pullbacks and pushforwards between partial flag varieties.

\begin{defn} Let $\Dem$ be the following (non-full) subcategory of finite dimensional vector spaces over $\mathbb{C}$. Its objects are $R^I$ for $I \subset S$. Its morphisms are generated by inclusion maps $\iota^I_J \co R^J \to R^I$ and Demazure operators $\pa^I_J \co R^I \to R^J$, for $I \subset J$. \end{defn}
	
Above, $\pa^I_J$ is the restriction of $D_{w_J w_I^{-1}}$ to $R^I$; one can show its image lies in $R^J$.

Before motivating this category, let us state the main results of \Cref{sec:demazure}, which are analogous to those of Demazure but significantly more refined. We provide:
\begin{itemize}
\item A presentation of $\Dem$ by generators (namely $\iota^I_J$ and $\pa^I_J$) and relations,
\item An operator $\pa_p : R^J \to R^I$ for each double coset $p \in W_I \backslash W / W_J$,
\item A proof that $\{\pa_p\}$ is a basis for $\Hom_{\Dem}(R^J,R^I)$, as $p$ ranges among such double cosets.
\end{itemize}


\begin{rem} The map $\pa^I_J$ is $\pa_q$, where $q \in W_J \backslash W / W_I$ is the double coset containing the identity. The map $\iota^I_J$ is $\pa_p$, where $p \in W_I \backslash W / W_J$ is the double coset containing the identity (see Example \ref{ex:3.10}). \end{rem}

\begin{rem} A simple variant on Demazure's presentation, replacing $D_i^2 = 0$ with $D_i^2 = D_i$, will describe pushforwards and pullbacks in equivariant $K$-theory instead of equivariant homology (of a point). Everything stated above holds in the $K$-theoretic context as well, mutatis mutandis. However, most statements below are special to the homology setting. \end{rem}

	
\begin{rem} We use different and more streamlined notation in the body of the paper. The category $\Dem$ above matches a category $\Dem'$ in the text, while $\Dem$ is defined differently (though they are proven to be equivalent). \end{rem}

A composition of the generating morphisms can be parametrized by a word $[I_0, I_1, \ldots, I_d]$ in the subsets of $S$, where either $I_i \subset I_{i+1}$ or $I_i \supset I_{i+1}$ for each $0 \le i < d$. Such a word is called a \emph{(singular) expression} or a \emph{double coset expression}.

An ordinary expression $(s_{i_1}, s_{i_2}, \ldots, s_{i_d})$ is reduced if and only if the composition $D_{i_1} D_{i_2} \cdots D_{i_d}$ is nonzero in the nilCoxeter algebra, in
which case it is equal to $D_w$ for some $w$. One might ask a similar question in $\Dem$: which compositions of the generating maps $\pa^I_J$ and $\iota^I_J$ are nonzero? This
gives a valid notion of a \emph{reduced expression} for a double coset.

Reduced expressions for double cosets were introduced in Williamson's thesis \cite{WillThesis}, where one purpose was to produce resolutions of singularities in orbit
closures for partial flag varieties (analogous to Bott-Samelson resolutions, associated to ordinary reduced expressions, in the full flag variety). The theory of reduced expressions was greatly expanded in recent work \cite{EKo}, and several equivalent definitions are given which are easier to work with. In this paper we show that the definition of reduced expressions given loosely above agrees with the definitions in \cite{WillThesis,EKo}. 
In \Cref{sec:singularrex} we recall from \cite{EKo} the results we need.

\begin{rem} Given a double coset expression, one can also associate a singular Bott-Samelson bimodule, as in Williamson's thesis \cite{WillThesis}. Each reduced expression for a double coset $p$ gives a bimodule for which the indecomposable singular Soergel bimodule $B_p$ is a direct summand. In subsequent work, we will use the Demazure operators of this paper to explicitly construct the inclusion from $B_p$ to the Bott-Samelson bimodule of certain reduced expressions, when $p$ is a maximal double coset in a finite parabolic subgroup.
\end{rem}


One of the primary uses of Demazure operators is the following. There is an $R^W$-bilinear pairing $R \times R \to R^W$, where
\begin{equation} (f,g) := \pa_W(fg). \end{equation}
Demazure \cite{Demazure} proves that this is a perfect pairing by constructing dual bases. In modern parlance, this makes $R$ into a \emph{Frobenius extension} over $R^W$; one implication is that induction and restriction between $R$-modules and $R^W$-modules are 
biadjoint functors (up to grading shift). Similar statements can be made about the extension $R^I \subset R$, or even the extension $R^J \subset R^I$ when $I \subset J$. All of these Frobenius structures are shadows of Poincar\'{e} duality for proper maps between the corresponding partial flag varieties.

Suppose $i \ne j \in S$. Here is a statement which is not about the composition of operators $D_i D_j$, but really about the composition $\pa_i \circ\iota_j$: we can choose dual bases
for $R$ over $R^i$, such that one of the bases lives in $R^j$. More generally, the following property was called $(\star)$ in \cite{ESWFrob}: for any $I, J \subset S$, one can
choose dual bases for $R^{I \cap J}$ over $R^I$ such that one of the bases lives in $R^J$. The condition $(\star)$ was essential to the proper functioning of a diagrammatic calculus developed in \cite{ESWFrob}, for computing natural transformations between compositions of induction and restriction functors. These compositions of functors are isomorphic to taking tensor product with certain bimodules, called \emph{singular Bott-Samelson bimodules}.

In subsequent work \cite{KELP3} we will further develop this diagrammatic calculus, providing a basis for morphisms between singular Bott-Samelson bimodules, and a crucial input to
our result is the main result of \Cref{sec:frob}, \Cref{thm:dualbasisinimage}. This can be viewed as a generalization of $(\star)$ for each double coset in $W_I \backslash W / W_J$. \Cref{thm:dualbasisinimage},
when applied to the double coset $p \in W_I \backslash W /W_J$ containing the identity, is exactly the condition $(\star)$, but for other cosets it is a similar but more technical
statement. To state it precisely involves knowing additional attributes of double cosets (e.g. redundancy subgroups), so we do not state the result in the introduction.

{\bf Acknowledgments.} NL was partially supported by FONDECYT-ANID grant 1230247. BE was partially supported by NSF grant DMS-2201387. 