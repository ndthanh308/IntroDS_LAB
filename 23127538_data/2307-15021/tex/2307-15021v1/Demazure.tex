\section{Demazure operators and double cosets} \label{sec:demazure}

The goal of this chapter is to define Demazure operators associated to double cosets, and relate them to the theory of singular reduced expressions.

\subsection{Realizations}

Let $(W,S)$ be a Coxeter system and $\Bbbk$ be a domain. Let $V$ be a \emph{realization} of $W$ over $\Bbbk$. In particular $V$ is a free $\Bbbk$-module, equipped with a set of roots $\{\al_s\}_{s \in S}$ in $V$ and coroots $\{\al_s^{\vee}\}_{s \in S}$ in $V^* = \Hom_{\Bbbk}(V,\Bbbk)$, for which the formula
\begin{equation} s(v) = v - \langle \al_s^\vee, v\rangle \al_s \end{equation}
defines an action of $W$ on $V$. A realization is one way to generalize the reflection representation of $V$. See \cite[Section 3.1]{Soergelcalculus} for additional technical details.

\begin{ex} Because it is easier than the reflection representation, we encourage the reader to think about the permutation representation of $S_n$ as their running example. Thus $W = S_n$ acts in the usual way on $V := \R[x_1,\ldots, x_n]$, and $\al_{s_i} = x_i - x_{i+1}$. \end{ex}

We make three technical assumptions throughout the body of the text.

The first assumption is that our realization is faithful, i.e. $V$ is a faithful representation of $W$. Faithfulness is a requirement for the algebraic category of (singular) Soergel bimodules to behave as expected (i.e. for the Soergel--Williamson Hom formula to hold). When the realization is not faithful, the diagrammatic version of the Hecke category should still behave well, though the diagrammatic version of the singular Hecke category has not yet been fully defined.

The second assumption is that our realization is \emph{balanced}, see \cite[Def. 3.6]{Bendihedral}. Let $a_{st} := \langle \al_s^\vee, \al_t \rangle$. For example, when $m_{st} = 3$, being balanced is equivalent to
\begin{equation} a_{st} = a_{ts} = -1. \end{equation}
In a general realization with $m_{st} = 3$, one might have $a_{st} = q$ and $a_{ts} = q^{-1}$ for any invertible element $q \in \Bbbk$. There are important unbalanced realizations, see \cite{EQuantumI}. Our techniques should apply to the unbalanced case as well, but would require a significant amount of extra bookkeeping, see e.g. \cite[Chapter 5]{EQuantumI}.

The third assumption, \emph{generalized Demazure surjectivity}, is discussed throughout this chapter. It is essential for the proper behavior and even the well-definedness of both the algebraic and diagrammatic Hecke categories.

What happens when these assumptions fail will be addressed in \S\ref{subsec:nastier}.

\subsection{Basics of Demazure operators}

Let $R = \Sym(V)$ be the symmetric algebra of $V$, graded so that $\deg V = 2$. For each $I \subset S$ finitary, one can consider the subring $R^I$ of $W_I$-invariant polynomials. We permit ourselves the usual shorthand from \S\ref{ssec:notation}, using notation like $R^{st}$ instead of $R^{\{s,t\}}$ and $R^{Is}$ instead of $R^{I \sqcup \{s\}}$.

For each simple reflection one can define an $R^s$-linear map $\pa_s \co R \to R^s$ of degree $-2$ via the formula
\begin{equation} \pa_s(f) := \frac{f - sf}{\al_s}. \end{equation}
These well-known operators are called \emph{Demazure operators} or \emph{divided difference operators}. Demazure operators  satisfy the \emph{nil-quadratic relation}
\begin{equation} \label{nilquad} \pa_s^2 = 0.\end{equation}
They also (by the balanced assumption) satisfy the braid relations: for example, when $m_{st} = 3$ one has
\begin{equation} \label{dembraid} \pa_s \pa_t \pa_s = \pa_t \pa_s \pa_t. \end{equation}

The Demazure operators can all be viewed as living inside $\End_{\Bbbk}(R)$, where they generate the so-called \emph{nilCoxeter algebra}. This graded algebra has a presentation with generators $\{\pa_s\}_{s \in S}$ (all of degree $-2$). The relations are the braid relations and the nil-quadratic relations. It also has a basis $\{\pa_x\}$ indexed by $x \in W$, where
\[ \pa_x := \pa_{s_1} \pa_{s_2} \cdots \pa_{s_d} \]
using any reduced expression $x = s_1 s_2 \cdots s_d$. These operators satisfy the formula
\begin{equation} \label{composeDem} \pa_x \circ \pa_y = \begin{cases} \pa_{xy} & \text{ if } xy = x.y, \\ 0 & \text{ else}. \end{cases} \end{equation}
If instead one considers the composition $\pa_{s_1} \pa_{s_2} \cdots \pa_{s_d}$ for a non-reduced expression, one gets zero.




Recall that $\leftdes(x)$ denote the left descent set of $x \in W$. The kernel of $\pa_s$ is precisely $R^s$. Using this and \eqref{composeDem} one can see that the following three statements are equivalent, for an element $x \in W$:
\begin{enumerate} \item $s \in \leftdes(x)$.
\item $\pa_s \pa_x = 0$.
\item $\pa_x(f) \in R^s$ for all $f \in R$.
\end{enumerate}
The following lemma is an immediate consequence.
\begin{lem} \label{lem:imageDem} The following are equivalent, for a finitary subset $I \subset S$.
\begin{enumerate} \item $I \subset \leftdes(x)$.
\item $\pa_s \pa_x = 0$ for all $s \in I$.
\item $\pa_x(f) \in R^I$ for all $f \in R$.
\end{enumerate}
\end{lem}

\begin{notation} Let $I \subset S$ be finitary, and recall that $w_I$ is the longest element of $W_I$. Then set $\pa_I := \pa_{w_I}$. \end{notation}

By Lemma \ref{lem:imageDem}, $\pa_I$ is a linear map from $R$ to $R^I$, and is $R^I$-linear since each $\pa_s$ is $R^s$-linear. An alternate description of $\pa_I$ was given by Demazure \cite[Proposition 3b]{Demazure}. For an English-language exposition, see \cite[Chapter 24, p. 463-464]{GBM}.

\begin{thm} \label{thm:Demazure} For any $I \subset S$ finitary we have
\begin{equation} \pa_I(f) = \frac{\sum_{w \in W_I} (-1)^{\ell(w)} w(f)}{\prod_{\alpha \in \Phi^+_I} \alpha}. \end{equation}
The denominator is the product of the positive roots for $W_I$, i.e. the set $$\Phi^+_I:=\{w(\alpha_s)\, \vert \, s\in I, w\in W_I \ \mathrm{and}\ \ell(ws)=
\ell(w)+1 \}.$$ \end{thm}

\begin{defn} A realization satisfies \emph{Demazure surjectivity} if $\pa_s \colon R \to R^s$ is surjective for all $s \in S$. It satisfies \emph{generalized Demazure surjectivity} if $\pa_I \colon R \to R^I$ is surjective for all finitary $I \subset S$. \end{defn}

\begin{lem} \label{lem:productofrootsworks} Generalized Demazure surjectivity holds whenever the realization is faithful and $\Bbbk$ is a field of characteristic zero. \end{lem} 

\begin{proof} Let $\mu_I$ denote the product of the positive roots for $W_I$. 
When the realization is faithful, the positive roots are in bijection with the set of reflections in $W_I$.  Then the span of $\mu_I$ affords the sign representation. One can deduce from Theorem \ref{thm:Demazure} that $\pa_I(\mu_I)$ equals the size of $W_I$, viewed as an element of $\Z \subset \Bbbk$. Since the image of $\pa_I$ is an $R^I$-submodule of $R^I$, and contains an invertible scalar, it must be everything.\end{proof}



The reason to care about generalized Demazure surjectivity is that it is equivalent to a more interesting property: that the ring extension $R^I \subset R$ is graded Frobenius, with trace map $\pa_I$, when $I$ is finitary. Moreover, when $I \subset J \subset S$ are both finitary, $R^J \subset R^I$ is a graded Frobenius extension, with trace map
\begin{equation} \label{paJIdefn} \pa^I_J := \pa_{w_J w_I^{-1}}. \end{equation}
For more details, see \S\ref{sec:frob}.

\subsection{NilCoxeter algebra as associated graded} \label{ss:nilCoxeter}

For $(W,S)$ a Coxeter system, let $(W,*,S)$ denote the \emph{Coxeter $*$-monoid}. It is generated by $s \in S$, modulo the braid relations and the \emph{$*$-quadratic relation}
\begin{equation} \label{starquadordinary} s * s = s. \end{equation}
Its elements agree with those of $W$, where $x \in W$ corresponds to $s_1 * s_2 * \cdots * s_d$ for a reduced expression. It is known that
\begin{equation} \label{starinequality} \ell(w * x) \le \ell(w) + \ell(x), \end{equation}
with equality if and only if $w * x = wx$ if and only if $w.x = wx$, see \cite[Lem. 3.2]{EKo}.
When this monoid is linearized into an algebra, it is %an example of a generalized Hecke algebra
equal to the Hecke algebra at $v=0$ \cite[Ch. 1]{Mathas}, known as the \emph{$0$-Hecke algebra}.

We wish to draw the reader's attention to the following observation. The $0$-Hecke algebra is filtered, where each element of $W$ lives in a degree equal to its length. This follows from \eqref{starinequality}. Thus one can consider the associated graded algebra. A priori, the associated graded algebra need not inherit the associated graded of the presentation of $(W,*,S)$, but in this case it does. If $\pa_s$ represents the image of $s$ in the associated graded, then \eqref{starquadordinary} is replaced by \eqref{nilquad}, while the braid relations are preserved. Thus the nilCoxeter algebra is the associated graded of the $0$-Hecke algebra.


In the rest of this chapter we consider a singular version of this phenomenon. In \cite[Definition 1.22]{EKo}, the Coxeter $*$-monoid is generalized to a category $\SC$, called the \emph{singular Coxeter monoid}, with one object for each finitary parabolic subset $I \subset S$. The morphisms from $J$ to $I$ are given by $(I,J)$-cosets, and composition acts on maximal elements by $*$-multiplication. In particular, $\End_{\SC}(\mt)$ is the Coxeter $*$-monoid. 

\begin{rem} One should not expect there to be a category similar to $\SC$ but where $\End(\mt)$ is the Coxeter group $W$ (rather than the $*$-monoid). This is because ``cosets are not invertible.'' \end{rem}

In the rest of this chapter we prove that the associated graded of $\SC$ matches a category constructed with Demazure operators, and we associate a Demazure operator to any double coset. As a consequence, the braid relations like \eqref{switchback} give rise to relations between Demazure operators.



\subsection{The nilCoxeter algebroid}


We will define a category  built out of Demazure operators and give three different  versions of it, that we will call $\Dem, \Dem'$ and $\Dem''$. We assume generalized Demazure surjectivity throughout.

\begin{defn} Let $p$ be an $(I,J)$-coset. Let $\pa_p := \pa_{\ma{p} w_J^{-1}}$, viewed as a linear map $R^J \to R^I$. \end{defn}

A priori, $\pa_{\ma{p} w_J^{-1}}$ is a linear map $R \to R$, but by the following lemma, it restricts to a map $R^J \to R^I$. The notation $\pa_p$ (as opposed to $\pa_{\ma{p} w_J^{-1}}$) indicates that we have restricted the domain of the function to $R^J$.

\begin{lem} \label{lem:sendsJtoI} The map $\pa_p$ sends $R^J$ to $R^I$. \end{lem}

\begin{proof} 
Since $\im\pa_J = R^J$ and $(\ma{p}w_J\inv).w_J=\ma{p}=w_I.(w_I\inv\ma{p})$, we have
\begin{equation} \im \pa_p= \im (\pa_p \circ \pa_{J}) = \im \pa_{\ma{p}}=\im (\pa_I \circ \pa_{w_I\inv \ma{p}})\subset \im \pa_I = R^I.\qedhere\end{equation}
\end{proof}

The following example is crucial.

\begin{ex}\label{ex:3.10} Suppose $I \subset J$ and that $p$ is the minimal $(I,J)$-coset, i.e., the double coset containing the identity element. Then $\ma{p} = w_J$, so $\pa_p = \pa_{\id}$ is the identity operator on $R$, and induces the inclusion map $R^J \hookrightarrow R^I$. If $q$ is the minimal $(J,I)$-coset, then $\ma{q} = w_J$ and $\pa_q = \pa_{w_J w_I^{-1}} = \pa^I_J$.
\end{ex}

\begin{defn} Let $(W,S)$ be a Coxeter system and $R$ be the polynomial ring of some realization. The \textit{nilCoxeter algebroid}  $\Dem = \Dem(W,S,R)$ is the following subcategory of $\Bbbk$-vector spaces. 
The objects are finitary subsets $I \subset S$, associated with the vector space $R^I$. 
The morphism space from $J$ to $I$ consists of $\Bbbk$-linear combinations of the operators $\pa_p$ associated to $(I,J)$-cosets $p$.  \end{defn}

Two important properties are not obvious from this definition.
\begin{itemize}
\item The set $\{\pa_p\}$ is linearly independent as $p$ ranges over all $(I,J)$-cosets. 
\item When $p$ is an $(I,J)$-coset and $q$ is a $(J,K)$-coset, then $\pa_p \circ \pa_q$ is in the span of Demazure operators for $(I,K)$-cosets. Thus composition is well-defined on $\Dem$.
\end{itemize}
To justify these properties, we will describe the same subcategory of vector spaces in a different way, and then prove that the two categories agree.

\begin{defn} \label{def:demprime}  Let $(W,S)$ be a Coxeter system and $R$ be the polynomial ring of some realization. Let $\Dem' = \Dem'(W,S,R)$ denote the following category, which is a subcategory of vector spaces. The objects are finitary subsets $I \subset S$, associated with the vector space $R^I$. The morphism space from $I$ to $J$ consists of linear combinations of compositions of the following generating morphisms:
\begin{enumerate}
    \item Whenever $Is$ is finitary, a morphism $Is \to I$ corresponding to the inclusion of rings $R^{Is} \hookrightarrow R^I$.
    \item Whenever $Is$ is finitary, a morphism $I \to Is$ corresponding to the Demazure operator $\pa^I_{Is} \co R^I \to R^{Is}$.
\end{enumerate}
That is, the morphisms in $\Dem'$ are all linear combinations of maps obtained as compositions of inclusion maps and Frobenius trace maps.
\end{defn}

\begin{notation} For a singular expression $I_{\bullet} = [I_0, \ldots, I_d]$, let $\pa_{I_{\bullet}}$ denote the corresponding composition of inclusion maps and Frobenius trace maps, a linear transformation from $R^{I_d}$ to $R^{I_0}$.\end{notation}

\begin{ex} Associated to the expression $[\mt, s, \mt, t, \mt, u, \mt]$ we have the endomorphism of $R$ given by $\pa_s \pa_t \pa_u$. \end{ex}

\begin{ex} Let $\{s,t,u\}$ be the simple reflections in type $A_3$, with $m_{su} = 2$. Associated to the expression $[st, s, su]$ we have the map $\pa_s \pa_t \co R^{su} \to R^{st}$.\newline
Associated to both $[s,\mt,t,\mt,s]$ and $[s,st,s]$ we have the map $\pa_s \pa_t \co R^s \to R^s$. \end{ex}

It is easier to describe the operator $\pa_{I_{\bullet}}$ using multistep expressions.

\begin{lem} Suppose that $I_{\bullet}$ corresponds to the multistep expression
\[ [[ I_0\subset K_1\supset I_1\subset K_2 \supset \cdots \subset K_m\supset I_m]]. \]
Then we have
\begin{equation} \label{multistepdemazure} \pa_{I_{\bullet}} = \pa_{w_{K_1} w_{I_1}^{-1}} \circ \pa_{w_{K_2} w_{I_2}^{-1}} \circ \cdots \circ \pa_{w_{K_m} w_{I_m}^{-1}}. \end{equation} \end{lem}

\begin{proof} We read the expression from right to left, and let $k$ range from $m$ down to $1$. The parabolic subgroup grows from $I_k$ to $K_k$, so one applies $\pa^{I_k}_{K_k} = \pa_{w_{K_k} w_{I_k}^{-1}}$. Then the parabolic subgroup shrinks from $K_k$ to $I_{k-1}$, which corresponds to an inclusion map (nothing happens). \end{proof}

\begin{prop} \label{prop:demred} If $I_{\bullet} \expr p$ is a reduced expression then $\pa_{I_{\bullet}} = \pa_p$. Otherwise, $\pa_{I_{\bullet}} = 0$. \end{prop}

\begin{proof} Recall from \eqref{map} that 
\[\ma{p} = w_{K_1} w_{I_1}^{-1} w_{K_2} \cdots w_{I_{m-1}}^{-1} w_{K_m}.\]
By definition (see \eqref{reduced}), if $I_{\bullet}$ is reduced then the lengths add in the expression
\[ (w_{K_1} w_{I_1}^{-1}) . (w_{K_2} w_{I_2}^{-1}) . \cdots . (w_{K_m} w_{I_m}^{-1}) = \ma{p} w_{I_m}^{-1}. \]
Thus by \eqref{multistepdemazure} and \eqref{composeDem}, $\pa_{I_{\bullet}} = \pa_p$ when $I_{\bullet}$ is reduced. Otherwise the lengths do not add, and the result is zero by \eqref{composeDem}. \end{proof}


\begin{cor}\label{cor:delpindep}
The set $\{\pa_p\}$, where $p$ ranges over all $(I,J)$-cosets, is a linearly independent collection of maps $R^J \to R^I$.
\end{cor}

\begin{proof}
Suppose there is a linear relation of the form $\sum a_p \pa_p = 0$, ranging over $(I,J)$ cosets $p$. By \cite[Prop. 4.13]{EKo}, we can precompose any $(I,J)$ coset $p$ with $[[J,\mt]]$ and postcompose with $[[\mt, I]]$, to obtain a reduced expression for the $(\mt, \mt)$ coset $\{\ma{p}\}$. Thus
\begin{equation} 0 = \pa_{[[\mt, I]]} \circ \left( \sum a_p \pa_p \right) \circ \pa_{[[J,\mt]]} = \sum a_p \pa_{\ma{p}}. \end{equation}
Since ordinary Demazure operators are linearly independent, we deduce that $a_p = 0$ for all $p$, as desired. 
\end{proof}



\begin{cor} Let $p$ be a $(J,I)$-coset and $q$ be a $(K,J)$-coset.
Then we have
\begin{equation} \label{composedemazureforcosets} \pa_q \circ \pa_p = \begin{cases} \pa_{q . p} & \text{ if } \ma{p} w_J^{-1} \ma{q} = (\ma{p} w_J^{-1}).\ma{q} = \ma{p} . (w_J^{-1} \ma{q}), \\ 0 & \text{ else}. \end{cases}\end{equation}
\end{cor}

\begin{proof} This follows immediately from Propositions \ref{prop:demred} and \ref{prop:concat}. \end{proof}

\begin{cor} \label{cor:DDequal} The nilCoxeter algebroid $\Dem$ is equal (as a subcategory of vector spaces) to $\Dem'$. Thus it is well-defined (i.e. closed under composition). The morphisms $\{\pa_p\}$ (ranging over all double cosets) form a basis for morphisms in $\Dem$. \end{cor} 

\begin{proof} The objects of  $\Dem$ and $\Dem'$ being the same and the composition of morphisms for both categories just being the composition of functions, to show their equality we only need to prove an equality of Hom spaces. 

Every double coset has a reduced expression, thus $\pa_p$ is a morphism in $\Dem'$ for all $p$. This implies that the morphisms in $\Dem$ are contained in those of $\Dem'.$  Meanwhile, by Example~\ref{ex:3.10}, $\Dem$ contains the generators in $\Dem'$, and by \eqref{composedemazureforcosets} every composition of the generators is either $\pa_p$, for some $p$, or zero. Thus the morphisms in $\Dem'$ are contained in those of $\Dem$.

The set $\{\pa_p\}$ spans all morphisms by definition of $\Dem$, and is linearly independent by Corollary \ref{cor:delpindep}. \end{proof}

\begin{thm} \label{thm:dempresent} Suppose the realization is faithful and balanced. The nilCoxeter algebroid has a presentation, with generators as in Definition \ref{def:demprime}. The relations are:
\begin{enumerate}
    \item The braid relations \eqref{braidrelns}.
    \item The \emph{(singular) nil-quadratic relation}
    \begin{equation} \pa_{[Is,I,Is]} = 0. \end{equation}
\end{enumerate}
A basis for $\Hom_{\Dem}(I,J)$ is $\{\pa_p\}$ ranging over all $(I,J)$-cosets $p$. Moreover, $\Dem$ is isomorphic to the associated graded of the singular Coxeter monoid with respect to the length filtration. \end{thm}

\begin{proof}
Let $\Dem''$ denote the category with this presentation. To show there is a full functor $\Dem'' \to \Dem'$ (sending generators to generators) we need to confirm that the relations hold in $\Dem' = \Dem$.

The nil-quadratic relation holds because  $\pa^I_{Is}$ is zero on the subring $R^{Is}$. This is because $\pa^I_{Is} = \pa_{w_{Is}} w_I^{-1}$ and $s \in \rightdes(w_{Is}w_I^{-1})$, so $\pa^I_{Is}$ begins with $\pa_s$. Meanwhile, $R^{Is}$ is contained in $R^s$ and is killed by $\pa_s$. 

The braid relations hold by Proposition~\ref{prop:demred}, since both sides of a braid relation are a reduced expression for the same double coset.

Let us define a morphism $\pa^{''}_p \in \Dem''$ by
\[ \pa^{''}_p := \pa^{''}_{I_{\bullet}} \]
for some arbitrarily chosen reduced expression $I_{\bullet} \expr p$ (where $\pa^{''}_{I_{\bullet}}$ is now the obvious composition of generators in $\Dem''$). By Theorem \ref{thm:matsumoto}, all such reduced expressions are related by the braid relations, so $\pa^{''}_p$ is well-defined in $\Dem''$. By Proposition \ref{prop:demred}, $\pa^{''}_p$ in $\Dem''$ is sent to $\pa_p$ in $\Dem$. By Corollary \ref{cor:delpindep}, $\{\pa^{''}_p\}$ is linearly independent in $\Dem''$ (since it is sent by a linear functor to a linearly independent set).

Meanwhile, if $I_{\bullet}$ is not reduced, then by Theorem \ref{thm:notreduced} we can apply a series of braid relations to get to an expression %$I_{\bullet}'$ 
containing $[Js,J,Js]$ inside for some $J$ and $s$. Applying the relations in $\Dem''$, we see that $\pa_{I_{\bullet}} = 0$. Thus every composition of the generators in $\Dem''$ is either $\pa^{''}_p$ for some $p$, or zero. Therefore $\{\pa^{''}_p\}$ is a spanning set for $\Dem''$.

We conclude that $\{\pa^{''}_p\}$ is a basis for morphism spaces in $\Dem''$, and therefore the functor to $\Dem'$ is fully faithful.


Just as the monoid $(W,*,S)$ was filtered by length, so too the category $\SC$ is filtered by length, see \cite[\S 3.4]{EKo}. Composition in the associated graded of (the linearization of) $\SC$ of the images of $p$ and $q$ would give $p * q$ if the lengths add, and $0$ otherwise. Thus the composition rule agrees with \eqref{composedemazureforcosets}. The natural functor from $\Dem''$ to the associated graded is therefore well-defined and evidently fully faithful.
\end{proof}

\begin{rem} 
The category $\SC$ has a presentation with the braid relations and the $*$-quadratic relation $[Is,I,Is] = [I]$, see \cite[Theorem 5.32]{EKo}. Taking the $\Bbbk$-linearization of $\SC$ and then taking its associated graded, the presentation of $\SC$ becomes the presentation of $\Dem''$.
\end{rem} 

\subsection{Aside on cohomology and \texorpdfstring{$K$}{K}-theory}
 
When $(W,S)$ is a Weyl group of a complex semisimple Lie group $G$ with Borel subgroup $B$, the operators $\pa_I$ have a geometric construction. We set $R = H_B^*$ to be the $B$-equivariant cohomology of a point.
Let $P_I$ denote the parabolic subgroup of $G$ corresponding to $I \subset S$. Then $R^I = H_{P_I}^*$ is the $P_I$-equivariant cohomology of a point. The map from the $P_I$-classifying space to the $B$-classifying space is a $P_I/B$ bundle, and $P_I/B$ is a smooth projective variety. Therefore, there is a proper pushforward map $H_B^* \to H_{P_I}^*$, and using relative Poincar\'{e} duality, this is a Frobenius extension. Indeed, the proper pushforward map agrees with the Demazure operator $\pa_I$.

It is common instead to consider the equivariant $K$-theory of a point. Let $R_K := K_B^*$ and $R_K^I := K_{P_I}^*$. The proper pushforward in $K$-theory is a different operator, commonly called a ($K$-theoretic) Demazure operator or a Lusztig-Demazure operator. This operator satisfies the $*$-quadratic relation $\pa_s \circ \pa_s = \pa_s$, rather than the nil-quadratic relation.

It is straightforward to adapt the arguments of this chapter to construct a category whose morphisms are built from $K$-theoretic Demazure operators instead of ordinary Demazure operators. This category will be equivalent to the linearization of $\SC$, rather than the associated graded of the linearization.


\subsection{Asides on other kinds of realizations} \label{subsec:nastier}

In this technical series of remarks, we address what is known and likely to happen when the realization is not faithful, or when it is unbalanced.

When the realization is not faithful, the operators $\{\pa_x\}$ need not be linearly independent, and additional relations are needed to present the nilCoxeter algebra. The map $\pa_I$ is often zero (and thus generalized Demazure surjectivity will fail), when the realization is not faithful upon restriction to $W_I$. There is a full but not necessarily faithful functor $\Dem'' \to \Dem$, by the same proof. It is $\Dem''$ which should be the associated graded of the singular Coxeter monoid.

When the realization is not balanced and $m_{st}$ is odd, the braid relations \eqref{dembraid} will involve an invertible scalar. For example, if $m_{st} = 3$ and $a_{st} = q$ and  $a_{ts} = q^{-1}$, then $\pa_s \pa_t \pa_s = -q \pa_t \pa_s \pa_t$. See \cite[Claim A.7]{Bendihedral} for details.

As a consequence, different reduced expressions for $x \in W$ will produce Demazure operators which agree up to rescaling. To define $\pa_x$, we fix a choice of reduced expression. Relations like \eqref{composeDem} are understood to only hold up to (computable) scalars.

The operators $\{\pa_x\}_{x \in W}$ still span the nilCoxeter algebra. The literature does not contain a proof (for faithful, unbalanced realizations) that they are linearly independent, although a proof along the lines of Demazure's proof in \cite[Corollaire 1]{Demazure} should suffice. Demazure expresses $\pa_x$ as a linear combination of operators $w \in W$ over the fraction field of $R$, and uses this to deduce linear independence. Given the state of the literature, let us say that the linear independence of Demazure operators is conjectural for faithful unbalanced representations.

Generalized Demazure surjectivity should continue to hold even for faithful unbalanced realizations. The proof of Lemma \ref{lem:productofrootsworks} will not work, as positive roots are not well-defined. However, there is a distinguished collection of lines spanned by roots. One should replace the product of all positive roots with the product of one root from each root line, an element which is well-defined up to invertible scalar. With this modification the proof of Lemma \ref{lem:productofrootsworks} should work, though this is only conjectural.

It is unknown if there is a variant on the Coxeter $*$-monoid whose associated graded is the nilCoxeter algebra for an unbalanced realization.

Consider an unbalanced realization, with trace maps $\pa^I_J$ defined so that \eqref{compatible} holds. One can define $\Dem$ and $\Dem'$ in the same way as above. If $I_{\bullet} \expr p$, then $\pa_{I_{\bullet}}$ and $\pa_p$ will agree up to scalar. Consequently Corollary \ref{cor:DDequal} will still hold. The nil-quadratic relation will hold in $\Dem$, as will the up-up and down-down braid relations, but the switchback relation will only hold up to a scalar. Defining $\Dem''$ with the appropriate scalars built into the switchback relations, the proof of Theorem \ref{thm:dempresent} holds mutatis mutandis except the proof that $\Dem'' \to \Dem'$ is faithful, which relies on the conjecture above that $\{\pa_p\}$ is linearly independent.

Finally, for $I \subset J \subset K$ in the balanced case, one has
\begin{equation*} \label{eq:compatibleearly} \pa^I_K = \pa^J_K \circ \pa^I_J. \end{equation*}
In the unbalanced setting, this may only hold up to scalar, as currently stated. However, \eqref{eq:compatibleearly} is a desirable feature for chains of Frobenius extensions. Consequently, one should not use \eqref{paJIdefn}, but should redefine $\pa^I_J$ as some scalar multiple of $\pa_{w_J w_I^{-1}}$, in order that \eqref{eq:compatibleearly} holds. For an example of this sort of bookkeeping, see \cite[\S 3.2]{EQuantumI}. %\end{rem}


\begin{rem} See \cite{EWLocalized} for further details on how the failure of faithfulness or balancedness affects the Hecke category. \end{rem}

