\section{THEORETICAL GUARANTEES} \label{sec: theory}

% This section analyzes the theoretical properties of Algorithm~\ref{alg: mask}. Theorem~\ref{thm: conservative} shows that, under certain conditions in the infinite-trajectory regime, distributionally intervening the initial state $\St[1]$ maintains \emph{conservativeness}: the algorithm will never incorrectly mask an observation that causally influences the expert. Theorem~\ref{thm: no_more_conservative} and Proposition~\ref{prop: strictly_less_conservative} illustrate that intervening $\St[1]$ reduces the potential overconservativeness in the masking algorithm. Specifically, Theorem~\ref{thm: no_more_conservative} states that if an observation $\Ob[][\oi]$ would be correctly masked under the causal model $\sys$, intervening $\St[1]$ will produce a new causal model $\sysi$ under which $\Ob[][\oi]$ is also masked. Proposition~\ref{prop: strictly_less_conservative} provides an illustrative class of systems for which an observation will only be masked in the intervened causal model $\sysi$, showing that intervening on $\St[1]$ produces a strictly better mask.
In this section, we delve into the theoretical properties of Algorithm~\ref{alg: mask}. Theorem~\ref{thm: conservative} demonstrates that if we intervene on the initial state $\St[1]$ and meet certain conditions in the infinite-trajectory regime, the algorithm remains \emph{conservative}, ensuring that no observation that causally influences the expert is mistakenly masked. Additionally, Theorem~\ref{thm: no_more_conservative} and Proposition~\ref{prop: strictly_less_conservative} highlight the effectiveness of intervening on $\St[1]$ in mitigating overconservativeness in the masking algorithm. Specifically, Theorem~\ref{thm: no_more_conservative} asserts that the correctly masked observations under the original causal model $\sys$ will also be masked under the intervened causal model $\sysi$. Proposition~\ref{prop: strictly_less_conservative} showcases a particular set of systems where the intervention only results in masks under $\sysi$, providing compelling evidence that the masking algorithm is more effective after intervening on $\St[1]$.

\begin{toappendix}
    We first introduce a series of auxiliary lemmas.
\begin{lemma} \label{lem: faithful}
    Consider an interventionally absolutely continuous SCM $\M$ with a faithful causal graph $\G$ that contains a directed path from $X$ to $Y$. Then provided a set $\bZ$ contains all ancestors of $X$ but none of its descendants, then for any assignment $\bz$ to $\bZ$ there exist values $x, x'$ such that
    \[
        \Pnorm{\Pd{y \mid \doc(x), \bz} - \Pd{y \mid \doc(x'), \bz}} > 0,
    \]
    viewed as induced measures over $Y$.
\end{lemma}

\begin{proof}
    As $\bZ$ contains no descendants of $X$, it cannot block the directed path between $X$ and $Y$ and hence the Causal Markov Condition does not declare $X$ and $Y$ independent. Faithfulness stipulates that $X$ and $Y$ are therefore dependent given $\bz$, so there exist $x,x'$ such that
    \[
        \Pnorm{\Pd{y \mid x, \bz} - \Pd{y \mid x', \bz}} > 0.
    \]

    The second rule of do calculus states that we can exchange observation and intervention if $X$ and $Y$ are independent given $\bz$ in the causal graph $\G_{\underline{X}}$ obtained by removing outgoing edges from $X$. If we remove outgoing edges from $X$, the only remaining paths between $X$ and $Y$ must contain an edge $X \leftarrow Z$ for some variable $Z$. This makes $Z$ an ancestor of $X$, and therefore $Z$ is included in $\bZ$, and both paths of the form $X \leftarrow Z \leftarrow J$ and $X \leftarrow Z \cau J$ are blocked by $\bZ$. This means that $X$ and $Y$ are $d$-separated by $\bZ$ in $\G_{\underline{X}}$, and we can apply the second do-calculus rule to conclude that
    \begin{align*}
        \Pd{y \mid \doc(x), \bz} =\ &\Pd{y \mid x, \bz}, \\
        \Pd{y \mid \doc(x'), \bz} =\ &\Pd{y \mid x', \bz},
    \end{align*}
    and hence
    \[
        \Pnorm{\Pd{y \mid \doc(x), \bz} - \Pd{y \mid \doc(x'), \bz}} > 0.
    \]
\end{proof}

\begin{lemma} \label{lem: measure_zero}
    Consider a set $E \subseteq \R$ where for each $x \in E$, there exists a ball $B(x, \epsilon_x)$ which contains no point in $E$. Then $E$ has measure zero with respect to the standard Lebesgue measure on $\R$.
\end{lemma}

\begin{proof}
    As $E$ is a subset of $\R$, it is Lindel{\"o}f, and the cover of $E$ by the collection of balls $\{B(x, \epsilon_x) \mid x \in E\}$ has a finite subcover. Enumerate this subcover as $I_i$; we then have
    \begin{align*}
        \lambda(E) &= \lambda(E \cap (\cup_i I_i)) \leq \sum_i \lambda(E \cap I_i) = 0,
    \end{align*}
    as each $E \cap I_i$ contains only a singleton.
\end{proof}

\begin{lemma} \label{lem: deriv_abs_value_swap}
    Let $f(x)$ be a differentiable function of $x \in \R$ at some $\bar x \in \R$ with $f(\bar x) = 0$. Then 
\[
\frac{d}{dx} \biggr \rvert_{\bar{x}^+} |f(x)| = \Bigg| \frac{d}{dx} \biggr \rvert_{\bar{x}} f(x) \Bigg|
\quad \mathrm{and} \quad
\frac{d}{dx} \biggr \rvert_{\bar{x}^-} |f(x)| = - \Bigg| \frac{d}{dx} \biggr \rvert_{\bar{x}} f(x) \Bigg|.
\]
\end{lemma}
\begin{proof}
    We prove the first result as the second follows similarly. Expanding the derivative:
    \begin{align*}
    \frac{d}{dx} \biggr \rvert_{\bar{x}^+} |f(x)|
    &= \lim_{\delta \to 0^+} \frac{|f(\bar x + \delta)| - |f(\bar x)|}{\delta} \\
    &= \lim_{\delta \to 0^+} \frac{|f(\bar x + \delta) - f(\bar x)|}{\delta} \\
    &= \left| \lim_{\delta \to 0^+} \frac{f(\bar x + \delta) - f(\bar x)}{\delta} \right| \\
    &= \Bigg| \frac{d}{dx} \biggr \rvert_{\bar{x}} f(x) \Bigg|,
    \end{align*}
    where moving the limit inside the absolute value is permissible by differentiability of $f$ at $\bar x$ and continuity of absolute value.
\end{proof}

\begin{lemma} \label{lem: deriv_swap}
    Let $f(x): \R \to M(Y)$ continuously map real numbers $x$ to a measure over the values assumed by a random variable $Y$. Then we have that
    \[
        \frac{d}{db} \biggr \rvert_{\bar b} \frac{d}{d \lambda} \left( \int_a^b f(x) dx \right)
        = \frac{d}{d \lambda} f(\bar b)
    \]
    almost everywhere over the domain of $Y$. Here $\int_a^b f(x) dx$ denotes Lebesgue integration against $\cU(a,b)$, $\frac{d}{d \lambda}$ is the Radon-Nikodym derivative with respect to the standard Lebesgue measure on $\R$, and $\frac{d}{db} \bigr \rvert_{\bar b}$ denotes the standard real analysis derivative evaluated at $\bar b$.
\end{lemma}
\begin{proof}
    We can expand the definition of the outer derivative
    \begin{align*}
        \frac{d}{db} \biggr \rvert_{\bar b} \frac{d}{d \lambda} \left( \int_a^b f(x) dx \right)
        &= \lim_{b \to \bar b} \frac{1}{b - \bar b} \left(\frac{d}{d \lambda} \left( \int_a^{\bar b} f(x) dx \right) - \frac{d}{d \lambda} \left( \int_a^b f(x) dx \right) \right) \\
        &= \lim_{b \to \bar b} \frac{1}{b - \bar b} \left(\frac{d}{d \lambda} \left( \int_b^{\bar b} f(x) dx \right) \right). \\
        &= \lim_{b \to \bar b} \frac{d}{d \lambda} \left( \frac{1}{b - \bar b} \int_b^{\bar b} f(x) dx \right).
    \end{align*}
    Take an arbitrary $\epsilon > 0$. We want to show $\exists \delta > 0$ such that for all $\bar b - \delta < b < \bar b + \delta$, we have that
    \begin{align*}
        \left\|  \frac{1}{b - \bar b} \left( \int_b^{\bar b} f(x) dx \right) - f(\bar b) \right\|_1 < \epsilon,
    \end{align*}
    where the Radon-Nikodym derivative $\frac{d}{d \lambda}$ is absorbed into the $L_1$ norm definition on measures. By continuity of $f$, we can always choose a $\delta$ small enough for this inequality to hold.
\end{proof}

\vspace*{1cm}

We now prove the main theoretical results.

\end{toappendix}

All subsequent theory relies on Assumptions~\ref{ass: time_invariance}-\ref{ass: abs_cont}, and for brevity we defer proofs and auxiliary lemmas to the appendix. We now introduce the main conservativeness theorem and provide a short proof sketch.

\begin{theoremrep} \label{thm: conservative}
    In the faithful system causal model $\sys$, assume that the measure-valued function ${\hi[1] \mapsto \Pdi{v \mid \doc(\bZ = \bz), \hi[1]}}$ is continuous for any set of nodes $\bZ$ and $V \not \in \bZ$.

    Let there exist a causal edge ${\Ob[t][\oi] \cau \Ac[t'][\ai]}$ in $\Gs$ for some $t,t' \in \N$, $t' \geq t$, and indices $\oi \in [\obdim]$ and $\ai \in [\acdim]$. Then in the interventional causal model $\sysi$ where the initial state distribution $\Pdint{\st[1]}$ has everywhere-nonzero density on $\stspace$, $\Ob[][\oi]$ is almost surely not masked by Algorithm~\ref{alg: mask} for almost every uniform parameterization of $\Hi[1]$ as the number of trajectories $N \to \infty$; i.e., \eqref{eqn: mask} correctly evaluates to true.
\end{theoremrep}
\begin{customproofsketch}
% How to handle almost all uniform parameterizations? With respect to what measure?
By Assumptions~\ref{ass: time_invariance}~and~\ref{ass: reaction_horizon}, we can WLOG consider $t=1$ with $t' \in [H]$. If $\Ob[1][\oi] \cau \Ac[t'][\ai]$, by Assumption~\ref{ass: expert_attend} there exists an edge $\St[1][\si] \to \Ob[1][\oi]$ for some $\si$. We show that in the SCM $\Msi$ where we intervene distributionally on $\St[1]$, we have that $\St[1][\si] \nPI \Ob[1][\oi]$ and $\St[1][\si] \nPI \Ac[t'][\ai]$. The arguments are similar, so we informally sketch the proof for the former.

To show that $\St[1][\si]$ and $\Ob[1][\oi]$ are dependent, it suffices to find a particular pair of states $\St[1][\si] = \alpha, \alpha'$ which induce different probability measures $\Pd{\ob[1][\oi] \mid \doc(\St[1][\si] = \alpha)}$ (resp. $\alpha'$) over $\Ob[1][\oi]$. 
We marginalize out the random seed $w$ from our original measure of interest $\Pd{\ob[1][\oi] \mid \doc(\St[1][\si] = \alpha)}$ via the integral
\begin{align*}
\Pd{\ob[1][\oi] \mid \doc(\St[1][\si] = \alpha)} = \int_a^b \Pd{\ob[1][\oi] \mid \doc(\St[1][\si] = \alpha), \hi[1]} p(w) d \hi[1],
\end{align*}
where we model $\hi \sim \cU(a,b)$. Note that the right-hand integral above in fact yields a measure over $\ob[1][\oi]$. We now aim to show that the statement
\begin{align}
\begin{aligned}
    \exists \alpha, \alpha' \textrm{ s.t. }
    \Pnorm{\int_a^b \big[&\Pd{\ob[1][\oi] \mid \doc(\St[1][\si] = \alpha), \hi[1]} - \\
                        &\Pd{\ob[1][\oi] \mid \doc(\St[1][\si] = \alpha'), \hi[1]} \big] d \hi} > 0 \label{eq: sketch2} 
\end{aligned}
\end{align}
holds Lebesgue-almost everywhere for $(a,b) \in \R^2$. By faithfulness of $\sys$ and the path from $\St[1][\si]$ to $\Ob[1][\oi]$, do-calculus rules yield that for any random seed $\Hi=\hi$ there exist an $\alpha, \alpha'$ such that
\begin{align} \label{eq: sketch1}
    \Pnorm{\Pd{\ob[1][\oi] \mid \doc(\St[1][\si] = \alpha), \hi[1]} - \Pd{\ob[1][\oi] \mid \doc(\St[1][\si] = \alpha'), \hi[1]}} > 0.
\end{align}
We then analyze the sensitivity of \eqref{eq: sketch2} with respect to the integration bounds $a$ and $b$. Namely, for any $(\bar a, \bar b)$ where the left-hand side of \eqref{eq: sketch2} vanishes, \eqref{eq: sketch1} yields that there exists an open ball around $\bar b$ in which \eqref{eq: sketch2} holds everywhere except $(\bar a, \bar b)$. An argument from Fubini's theorem then shows that \eqref{eq: sketch2} holds for almost all $(a,b)$. Appealing to the consistency of Hoeffding's independence test concludes the proof.
\end{customproofsketch}

\begin{appendixproof}
    By Assumptions~\ref{ass: time_invariance}~and~\ref{ass: reaction_horizon}, we can WLOG consider $t=1$ with $t' \in [H]$. If $\Ob[1][\oi] \cau \Ac[t'][\ai]$, by Assumption~\ref{ass: expert_attend} there exists an edge $\St[1][\si] \to \Ob[1][\oi]$ for some $\si$. We now want to show that in the SCM $\Msi$ where we intervene distributionally on $\St[1]$, we have that $\St[1][\si] \nPI \Ob[1][\oi]$ and $\St[1][\si] \nPI \Ac[t'][\ai]$. The arguments are similar, so we will just state the proof for the former.

    We want to show that $\St[1][\si]$ and $\Ob[1][\oi]$ are not independent in $\Msi$. Note that in the modified structural assignment for $\St[1][\si]$ in $\Msi$, $\St[1][\si]$ is distributed with everywhere-nonzero density on $\stspace$. Therefore checking the desired independence is equivalent to showing
    \begin{align} \label{eqn: conservative_inequality_1}
        \Pnorm{\Pd{\ob[1][\oi] \mid \doc(\St[1][\si] = \alpha)} - \Pd{\ob[1][\oi] \mid \doc(\St[1][\si] = \alpha')}} > 0
    \end{align}
    for some $\alpha, \alpha' \in \R$ with $\alpha \neq \alpha'$. Here, $\Pd{\ob[1][\oi] \mid \cdot}$ denotes a probability measure over $\ob[1][\oi]$. The $\doc$ statement captures our ability to intervene on the initial state, decoupling any potential correlational influence from $\Hi[1]$.

    By Lemma~\ref{lem: faithful}, we have that for any particular value $\hi[1]$ of $\Hi[1]$,
    \begin{align*}
        \Pnorm{\Pd{\ob[1][\oi] \mid \doc(\St[1][\si] = \alpha), \hi[1]} - \Pd{\ob[1][\oi] \mid \doc(\St[1][\si] = \alpha'), \hi[1]}} > 0,
    \end{align*}
    for some $\alpha$, $\alpha'$. This is equivalent to
    \begin{align}
         \Pnormi{h(\alpha, \alpha', \hi[1])} \not\equiv \zf \quad \forall \hi[1],  \label{eqn: h_nonzero}
    \end{align}
    where we define
    \begin{align*}
        h(\alpha, \alpha', \hi[1]) \defeq \Pd{\ob[1][\oi] \mid \doc(\St[1][\si] = \alpha), \hi[1]} - \Pd{\ob[1][\oi] \mid \doc(\St[1][\si] = \alpha'), \hi[1]},
    \end{align*}
    and $\zf$ denotes an identically zero function over $\alpha, \alpha'$. Note that $h(\alpha, \alpha', \hi[1])$ specifies a signed measure over $\ob[1][\oi]$. Now observe that
    \begin{align*}
    \Pd{\ob[1][\oi] \mid \doc(\St[1][\si] = \alpha)} = \int \Pd{\ob[1][\oi] \mid \doc(\St[1][\si] = \alpha), \hi[1]} p(\hi[1]) d\mu(\hi[1]),
    \end{align*}
	where $\mu$ is a probability measure on the unobserved variable $\hi[1]$ which we will instantiate shortly, and $p(\hi[1])$ denotes the probability density of $\Hi[1]$, i.e. the Radon-Nikodym derivative of the measure $\Pdi{\hi[1]}$. Note that the result of this integral is still a signed measure over $\ob[1][\oi]$. So we have that showing our desired inequality \eqref{eqn: conservative_inequality_1} is equivalent to showing
    \begin{align*}
        \Pnorm{\int h(\alpha, \alpha', \hi[1]) p(\hi[1]) d\mu(\hi[1])} \not\equiv \zf
    \end{align*}
    as a function of $\alpha$, $\alpha'$ for ``almost all'' measures $\mu$---as there is no natural measure on the space of measures, we have formalized this assuming a uniform distribution $\Hi[1] \sim \cU(a,b)$. Note that the outer norm computes the $L_1$ norm of a signed measure over $\ob[1][\oi]$. For notational convenience, we will now define the concatenation $z = [\alpha, \alpha']$, with $z \in \R^2$. We can now concretely refine $\mu$ in the above statement, using our new $z$-notation, to showing that
    \begin{align}
        g_{a}^b(z) \defeq \Pnorm{\int_a^b h(z, \hi[1]) d\hi[1]} \not\equiv \zf \label{eqn: conservative_inequality_2} 
    \end{align}
    as a function of $z$ for almost every $(a,b)$; i.e., the subset of $(a,b)$ parameter space where $g_a^b(z) \equiv \zf$ over $z$ is measure zero with respect to the standard Lebesgue measure in $\mathbb{R}^2$. Note that we drop the $p(\hi[1])$ factor since for the uniform distribution this is a constant which factors out.

    This can be analyzed by taking sections were we fix $a$ and consider the set of $b$'s where $g_a^b (z) \equiv \zf$; if this set has measure zero, then the overall set of Cartesian pairs $(a,b)$ where $g_a^b(z) \equiv \zf$ can be shown to have measure zero by the following argument. Observe that $g_a^b(z)$ is continuous in $a, b$ and $z$ by Assumption~\ref{ass: abs_cont} and integral properties; then the inverse image of $\{0\}$ under $g$ is a Borel subset of $A \subset \R^4$, recalling that $z \in \R^2$. The projection of this Borel subset on to the $(a,b)$ plane is measurable (but not necessarily Borel). Then if each fixed-$a$ slice is measure zero, the overall set is measure zero by Fubini's theorem.

    Correspondingly, we fix any particular $a$ and drop it from the subscript of $g_a^b$ for simplicity. Consider a particular $\bar b$ where $g^{\bar b}(z) \equiv \zf$ as a function of $z$. We expand the $L_1$ norm in \eqref{eqn: conservative_inequality_2} as
    \begin{align}
        g^b(z) = \int \biggl \lvert \frac{d}{d\lambda} \bigg( \int_a^b h(z, \hi[1]) d\hi[1] \bigg) \biggr \rvert d \lambda
        = \int \left| f^b_z(\ob[1][\oi]) \right| d \lambda, \label{eqn: deriv_expansion}
    \end{align}
    using the interventional absolute continuity assumption to invoke the Radon-Nikodym derivative on our signed measure over $\ob[1][\oi]$ with respect to the standard Lebesgue measure $\lambda$. We denote the resulting density function by $f^b_z(\ob[1][\oi])$. Note that since $g^{\bar b}(z) \equiv \zf$, we have that $f^{\bar b}_z(\ob[1][\oi]) = 0$ for almost all $z$ and $\ob[1][\oi]$.

    Note that $f^b_z(\ob[1][\oi])$ is differentiable with respect to $b$ due to the assumed continuity of maps on $\hi[1]$ in the theorem statement. We now differentiate \eqref{eqn: deriv_expansion} with respect to $b$ at $\bar b$. Due to the absolute value in \eqref{eqn: deriv_expansion}, we must take care to differentiate from above and below and show both these cases are nonzero. As they follow similarly, we show the case for above:
    \begin{align}
		\frac{d}{db} \biggr \rvert_{\bar{b}^+} g^b(z) &= \frac{d}{db} \biggr \rvert_{\bar b^+} \int \biggl \lvert \frac{d}{d\lambda} \bigg( \int_a^b h(z, \hi[1]) d\hi[1] \bigg) \biggr \rvert d \lambda \label{eqn: step1} \\
 		&= \int \frac{d}{db} \biggr \rvert_{\bar b^+} \biggl \lvert \frac{d}{d\lambda} \bigg( \int_a^b h(z, \hi[1]) d\hi[1] \bigg) \biggr \rvert d \lambda \label{eqn: step2}  \\
 		&= \int \biggl \lvert \frac{d}{db} \biggr \rvert_{\bar b} \frac{d}{d\lambda} \bigg( \int_a^b h(z, \hi[1]) d\hi[1] \bigg) \biggr \rvert d \lambda \label{eqn: step3}  \\
 		&= \int \biggl \lvert \frac{d}{d\lambda} h(z, \bar b) \biggr \rvert d \lambda \label{eqn: step4}  \\
 		&= \Pnorm{h(z, \bar b)} \label{eqn: step5}  \\
 		&\not\equiv \zf, \qquad\qquad \text{(as a function of $z$)} \label{eqn: step6}
    \end{align}
    where \eqref{eqn: step2} follows from boundedness of the Radon-Nikodym derivative of $h(z, \bar b)$, \eqref{eqn: step3} follows from applying Lemma~\ref{lem: deriv_abs_value_swap} to $f^b_z(\ob[1][\oi])$ with respect to $b$, \eqref{eqn: step4} follows from Lemma~\ref{lem: deriv_swap}, \eqref{eqn: step5} follows from differentiability of $f_z^b(\ob[1][\oi])$ with respect to $b$, and \eqref{eqn: step6} follows from \eqref{eqn: h_nonzero}.

    Proceeding similarly, we can show that both
    \begin{align*}
    \frac{d}{db} \biggr \rvert_{\bar{b}^+} g^b(z) \not \equiv \zf \quad \mathrm{and} \quad
    \frac{d}{db} \biggr \rvert_{\bar{b}^-} g^b(z) \not \equiv \zf.
    \end{align*}
    It is then immediate that there exists a ball $B(\bar b, \epsilon_{\bar b})$ such that $g^b(z) \not \equiv \zf$ for all $b \in B(\bar b, \epsilon_{\bar b}) \setminus \bar b$. Applying Lemma~\ref{lem: measure_zero} concludes that for a fixed $a$, the set of $b$ for which \eqref{eqn: conservative_inequality_2} is violated is measure zero. Hence by the above Fubini argument, for almost every uniform measure $\mathcal{U}(a,b)$ on $\hi[1]$, we have that \eqref{eqn: conservative_inequality_1} holds for some $\alpha, \alpha'$. Therefore $\St[1][\si] \nPI \Ob[1][\oi]$ in the interventional distribution on $\St[1][\si]$.

    A similar argument shows that $\St[1][\si] \nPI \Ac[t'][\ai]$. By absolute continuity of the induced interventional distributions, we now have that Hoeffding's independence test is consistent, and hence the dependences are detected with probability $1$ as $N \to \infty$. Therefore $\Ob[1][\oi] \pcau \Ac[t'][\ai]$, and $\maski_{\oi}$ evaluates to false \eqref{eqn: mask} as $N \to \infty$. 
    % TODO: cite consistency
\end{appendixproof}

% Theorem~\ref{thm: conservative} provides the conservativeness guarantee that we expected: if an observation causally affects some expert action, Algorithm~\ref{alg: mask} correctly keeps this observation unmasked. As discussed in Section~\ref{sec: derivation}, this is immediate by the faithfulness of $\sys$ when we do not intervene on $\St[1]$ and allow the initial state to be generated naturally from $\Hi[1]$. The effort of Theorem~\ref{thm: conservative} is devoted to showing that this property still holds in the interventional system $\sysi$ where we assign $\St[1] \sim \Pdint{\st[1]}$.
Theorem~\ref{thm: conservative} guarantees that Algorithm~\ref{alg: mask} maintains conservativeness by correctly preserving unmasked observations that causally impact expert actions. This outcome is consistent with the discussion in Section~\ref{sec: derivation}, where we observed that the faithfulness of $\sys$ ensures the correctness of the algorithm when we do not intervene on $\St[1]$ and allow the initial state to be naturally generated from $\Hi[1]$. Theorem~\ref{thm: conservative} establishes that this property also holds in the interventional system $\sysi$, where we assign $\St[1] \sim \Pdint{\st[1]}$.

We now theoretically demonstrate the benefits of intervening with $\Pdint{\st[1]}$. Specifically, we show that this intervention reduces the excess conservatism in the masking algorithm by removing income edges from $\Hi[1]$ in the causal graph, thereby eliminating a potential avenue of confounding.

\begin{theoremrep} \label{thm: no_more_conservative}
    Let $\mask$ denote the potential-cause test evaluated by Algorithm~\ref{alg: mask} on the distribution induced by the non-interventional system $\sys$, and let $\maski$ be the original test on the interventional system $\sysi$ where $\Pdint{\st[1]}$ has everywhere-nonzero density on $\stspace$. If $\mask_{\oi}$ correctly evaluates to true for a particular $\oi \in [\obdim]$, then $\maski_{\oi}$ also evaluates to true almost surely as the number of trajectories $N \to \infty$.
\end{theoremrep}
\begin{appendixproof}
    If $\mask_{\oi}$ evaluates to true, then for any $\si \in [\stdim]$, $\ai \in [\acdim]$, and $t' \in [H]$, we have that either $\St[1][\si] \PI_{\Ms} \Ob[1][\oi]$ or $\St[1][\si] \PI_{\Ms} \Ac[\ai][t']$, where $\PI_{\Ms}$ denotes independence in the distribution induced by the non-interventional SCM $\Ms$. It suffices to show that both these independencies hold in the distribution induced by $\Msi$. As both arguments follow similarly, we consider showing that $\St[1][\si] \PI_{\Msi} \Ob[1][\oi]$.

    As we are given $\St[1][\si] \PI_{\Ms} \Ob[1][\oi]$, it is immediate by faithfulness that there exists no collider-free path from $\St[1][\si]$ to $\Ob[1][\oi]$ in $\Gs$. Since $\Gsi$ is simply $\Gs$ with the incoming edges to $\St[1]$ removed, it holds that there is no collider-free path between $\St[1][\si]$ and $\Ob[1][\oi]$ in $\Gsi$. Therefore $\St[1][\si] \PI_{\Msi} \Ob[1][\oi]$, and as $N \to \infty$ this is correctly detected with probability $1$ by the consistency of Hoeffding's test.
\end{appendixproof}

Theorem~\ref{thm: no_more_conservative} assures us that intervening with $\Pdint{\st[1]}$ does not lead to more conservative masking than the original system. We now provide a specific class of SCMs for which the intervention strictly improves the mask.

\begin{propositionrep} \label{prop: strictly_less_conservative}
    Let $\maski$ and $\mask$ be as in Theorem~\ref{thm: no_more_conservative}, and consider a particular observation index $\oi \in [\obdim]$ such that the only incoming edge to $\Ob[1][\oi]$ is $\Hi[1] \cau \Ob[1][\oi]$. Then if in $\Gs$ there exists the fork $\St[1][\si] \leftarrow \Hi[1] \cau \Ob[1][\oi]$ for some $\si \in [\stdim]$ and a directed path from $\St[1][\si]$ to some $\Ac[t][\ai]$, with $t \in [H], \ai \in [\acdim]$, $\maski_{\oi}$ correctly masks the $\oi \tth$ observation almost surely as the number of trajectories $N \to \infty$ while $\mask_{\oi}$ does not.
\end{propositionrep}

\begin{appendixproof}
	We first show that $\mask$ does not mask $\oi$ and take all causal and probabilistic statements to refer to the unintervened causal model $\sys$. By faithfulness, the fork ${\St[1][\si] \leftarrow \Hi[1] \rightarrow \Ob[1][\oi]}$ in $\Gs$ produces a statistical dependence $\St[1][\si] \,{\nPI_{\Ms}}\, \Ob[1][\oi]$ in the probability distribution induced by $\Ms$. Similarly, the directed path from $\St[1][\si]$ to $\Ac[t][\ai]$ yields ${\St[1][\si]\, {\nPI_{\Ms}}\, \Ac[t][\ai]}$. By consistency of Hoeffding's test, as ${N \to \infty}$ we get that ${^{(1,t)} D_{\si,\ai}^{\oi}}$ evaluates to true almost surely \eqref{eqn: causal_check} and thus ${\Ob[1][\oi] \pcau \Ac[t][\ai]}$ by \eqref{eqn: potential_cause}. Therefore $\mask_{\oi}$ is not masked \eqref{eqn: mask}.

    We now show that $\maski$ does mask $\oi$ and take all causal and probabilistic statements to refer to the \emph{intervened} causal model $\sysi$. Since $\Hi[1]$ only has outgoing edges, and the edge from $\Hi[1] \to \St[1][\si']$ is removed in $\Gsi$ for every $\si' \in [\stdim]$, there exists no path from $\St[1][\si']$ to $\Ob[1][\oi]$ in $\Gsi$, and therefore $\St[1][\si'] \PI_{\Msi} \Ob[1][\oi]$ in the probability distribution induced by $\Msi$. As $N \to \infty$ this independence is detected by Hoeffding's test, and since $\si'$ was arbitrary ${^{(1,t')} D_{\si',\ai'}^{\oi}}$ is false for every $\si' \in [\stdim]$, $\ai' \in [\acdim]$, and $t' \in [H]$. Therefore $\Ob[1][\oi] \npcau \Ac[t'][\ai']$ for any $\ai' \in [\acdim],t' \in [H]$, and \eqref{eqn: mask} evaluates to true. Therefore $\maski_{\oi}$ is masked.
\end{appendixproof}

% In summary, Theorem~\ref{thm: conservative} shows that masking with the intervened initial state $\Pdint{\st[1]}$ maintains conservatism; Theorem~\ref{thm: no_more_conservative} states that intervening on $\Pdint{\st[1]}$ is no more conservative than masking with the unintervened causal model; and Proposition~\ref{prop: strictly_less_conservative} shows that intervening on $\Pdint{\st[1]}$ results in a strictly less conservative mask for a certain class of systems.
