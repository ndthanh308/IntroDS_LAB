%%%%% GENERAL MATH COMMANDS
% Reals
\newcommand{\R}{{\mathbb R}}
% Integers
\newcommand{\Z}{{\mathbb Z}}
% Naturals
\newcommand{\N}{{\mathbb N}}
% Expectation
\DeclareMathOperator*{\E}{\mathbb{E}}
% ^th notation
\newcommand{\tth}{^{\text{th}}}
% Small dots for integer range [a .. b]
\newcommand{\sdots}{\,..\,}

% := sign
\newcommand{\defeq}{\vcentcolon=}
% Zero function
\newcommand{\zf}{\mathbf{0}}

% Argmin and argmax definitions
\DeclareMathOperator*{\argmax}{arg\,max}
\DeclareMathOperator*{\argmin}{arg\,min}


%%%%% EXPERIMENT NOTATION
\newcommand{\cp}{CartPole\xspace}
\newcommand{\re}{Reacher\xspace}

\newcommand{\masked}{\textsc{Masked}\xspace}
\newcommand{\bc}{\textsc{BcVanilla}\xspace}
\newcommand{\bcd}{\textsc{BcManual}\xspace}

\newcommand{\net}{f_{\theta}}
\newcommand{\compnet}{g_{\theta}}
\newcommand{\enc}{\psi_e}
\newcommand{\dec}{\psi_d}
\newcommand{\vae}{\psi}
\newcommand{\vaemask}{\tilde \psi}


%%%%% PROBLEM STATEMENT NOTATION 
% Following commands take two optional arguments: time (subscript) and index (superscript)
% E.g. \st[5][2].
% The uppercase versions refer to random variables and lowercase to specific values
\newcommandtwoopt{\St}[2][t][]{{S_{#1}^{#2}}} % State
\newcommandtwoopt{\st}[2][t][]{{s_{#1}^{#2}}}
\renewcommandtwoopt{\Im}[2][t][]{{I_{#1}^{#2}}} % Image 
% \newcommandtwoopt{\im}[2][t][]{{\textsl{\textsc{i}}_{#1}^{#2}}}
% \newcommandtwoopt{\im}[2][t][]{{\textit{\textsc{i}}_{#1}^{#2}}}
\newcommandtwoopt{\im}[2][t][]{{\scriptstyle I}_{#1}^{#2}}
\newcommandtwoopt{\Ob}[2][t][]{{O_{#1}^{#2}}} % Observation
\newcommandtwoopt{\ob}[2][t][]{o_{#1}^{#2}}
\newcommandtwoopt{\Ac}[2][t][]{{A_{#1}^{#2}}} % Action
\newcommandtwoopt{\ac}[2][t][]{{a_{#1}^{#2}}}
%\newcommand{\Hi}[1][t]{{W_{#1}}} % History
%\newcommand{\hi}[1][t]{{w_{#1}}}
\newcommand{\Hi}[1][t]{{W}} % History
\newcommand{\hi}[1][t]{{w}}

% Collection of all trajectories
\newcommand{\traj}{{^{i}\bm{\tau}}}
\newcommand{\trajs}{{^{(1..N)}\bm{\tau}}}

% Spaces
\newcommand{\stspace}{{\mathcal{S}}}
\newcommand{\imspace}{{\mathcal{I}}}
\newcommand{\obspace}{{\mathcal{O}}}
\newcommand{\acspace}{{\mathcal{A}}}

% Space dimensions
\newcommand{\stdim}{{d_{\stspace}}}
\newcommand{\imdim}{{d_{\imspace}}}
\newcommand{\obdim}{{d_{\obspace}}}
\newcommand{\acdim}{{d_{\acspace}}}

% State, observation, action indices
\newcommand{\si}{\mathscr{s}}
\newcommand{\oi}{\mathscr{o}}
\newcommand{\ai}{\mathscr{a}}

% The boolean subvariable for masking
\newcommand{\Dfull}{{^{(t,t')} D_{\si,\ai}^{\oi}}}

\newcommand{\mask}{m} % Mask on unintervened system (ONLY USE IN THEORY)
\newcommand{\maski}{\widetilde{m}} % Mask under interventional system (USE IN EXPERIMENTAL SECTION, THIS IS WHAT WE ACTUALLY EVALUATE)

%%%%% DO CALCULUS AND PROBABILITY NOTATION

\newcommand{\Pd}[1]{P\big(#1\big)}  % Probability distribution / measure, takes argument
\newcommand{\Pdi}[1]{P(#1)} % Inline version
\newcommand{\Pnorm}[1]{\Big\| #1 \Big\|_1} % L_1 norm of probability measure
\newcommand{\Pnormi}[1]{\| #1 \|_1}  % Inline version

\newcommand{\Pdint}[1]{\tilde{P}(#1)} % Interventional probability distribution / measure

\newcommand{\cU}{\mathcal U} % Caligraphic u, for uniform distribution


\newcommand{\doc}{{\mathrm{do}}} % Do calculus intervention
\newcommand{\cau}{\rightarrow} % Causal arrow in true causal graph
\newcommand{\ncau}{\not\rightarrow} % Non-causal arrow
\newcommand{\pcau}{\dashrightarrow} % Potential cause detected by algorithm
\newcommand{\npcau}{\not\dashrightarrow} % No potential cause detected by algorithm

\newcommand{\PI}{\mathrel{\perp\mspace{-10mu}\perp}} % Probabilistic independence
\newcommand{\nPI}{\centernot{\PI}} % Not probabilistically independent (i.e. dependent)

\newcommand{\G}{\mathcal{G}} % Causal graph
\newcommand{\Gi}{\widetilde{\mathcal{G}}} % Interventional causal graph
\newcommand{\M}{\mathcal{M}} % Structural causal model
\newcommand{\Mi}{\widetilde{\mathcal{M}}} % Interventional structural causal model

\newcommand{\Gs}{\mathcal{G}_s} % Causal graph for our system
\newcommand{\Gsi}{\widetilde{\mathcal{G}}_s} % Interventional causal graph for our system
\newcommand{\Ms}{\mathcal{M}_s} % Structural causal model for our system
\newcommand{\Msi}{\widetilde{\mathcal{M}}_s} % Interventional structural causal model for our system

\newcommand{\sys}{\langle \Ms, \Gs \rangle}
\newcommand{\sysi}{\langle \Msi, \Gsi \rangle}

\newcommand{\pa}[1]{\mathbf{pa}_{#1}} % Parent set of node

\newcommand{\bV}{\mathbf V} % Set of nodes variables in SCM
\newcommand{\bv}{\mathbf v} % Set of node values in SCM
\newcommand{\bU}{\mathbf U} % Set of noise variables (endogenous
\newcommand{\cF}{\mathcal F} % Causal functions SCM
\newcommand{\SCM}{\langle \bV, \bU, \cF \rangle} % Shorthand for SCM tuple

\newcommand{\bX}{\mathbf X} % Arbitrary subset of nodes in SCM
\newcommand{\bx}{\mathbf x} % Corresponding values to nodes

\newcommand{\bY}{\mathbf Y} % Arbitrary subset of nodes in SCM
\newcommand{\by}{\mathbf y} % Corresponding values to nodes

\newcommand{\bZ}{\mathbf Z} % Arbitrary subset of nodes in SCM
\newcommand{\bz}{\mathbf z} % Corresponding values to nodes


%%%% For Appendix %%%%
\newcommand{\xt}{x_t}
\newcommand{\dxt}{\dot{x}_t}
\newcommand{\ddxt}{\ddot{x}_t}
\newcommand{\tht}{\theta_t}
\newcommand{\dtht}{\dot{\theta}_t}
\newcommand{\ddtht}{\ddot{\theta}_t}
\newcommand{\xtn}{x_{t+1}}
\newcommand{\dxtn}{\dot{x}_{t+1}}
\newcommand{\thtn}{\theta_{t+1}}
\newcommand{\dthtn}{\dot{\theta}_{t+1}}

\newcommand{\mplp}{\frac{m_p l_p} {2}}
\newcommand{\temp}{\frac{f_t + \mplp \dtht^2 \sin \tht} {m_c + m_p}}
\newcommand{\nume}{g \sin \tht - \zeta_t \cos \tht}
\newcommand{\deno}{l_p \left( \frac{4}{3} - \frac{m_p (\cos \tht)^2)} {m_c + m_p} \right)}

\newcommand{\Starget}{S_\mathrm{target}}
\newcommand{\Unif}{\mathrm{Unif}}
%%%%  %%%%  %%%%  %%%%


\theoremstyle{plain}
\newtheoremrep{theorem}{Theorem}
\newtheoremrep{proposition}[theorem]{Proposition}
\newtheoremrep{lemma}[theorem]{Lemma}
\newtheoremrep{corollary}[theorem]{Corollary}

\theoremstyle{definition}
\newtheorem{definition}{Definition}
\newtheorem{assumption}{Assumption}

\theoremstyle{remark}
\newtheorem{remark}{Remark}

\newenvironment{customproofsketch}{\indent \indent \textit{Proof sketch (informal): }}{\hfill $\blacksquare$ \vspace{0.2cm}}


%%%%% Theorems
%\newtheoremstyle{mytheoremstyle} % name
    %{\topsep}                    % Space above
    %{\topsep}                    % Space below
    %{\itshape}                   % Body font
    %{}                           % Indent amount
    %{\scshape}                   % Theorem head font
    %{.}                          % Punctuation after theorem head
    %{.5em}                       % Space after theorem head
    %{}  % Theorem head spec (can be left empty, meaning ‘normal’)
%\theoremstyle{mytheoremstyle}
%\theoremstyle{plain}
%\newtheorem{theorem}{Theorem}
%\newtheorem{proposition}{Proposition}
%\newtheorem{assumption}{Assumption}
%\newtheorem{definition}{Definition}
%\newtheorem{lemma}{Lemma}
