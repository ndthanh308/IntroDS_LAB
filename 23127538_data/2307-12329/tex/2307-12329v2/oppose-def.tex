%% local definitions
%% use \textrm for roman and \textbf for bold in math
%% use \mbox for boxes in math mode
%% use \imath and \jmath for dotless letters in math mode

%% local defs


%% standard defs

\newcommand*{\Ga}{\alpha}
\newcommand*{\Gb}{\beta}
\newcommand*{\Gd}{\delta}
\newcommand*{\GD}{\Delta}
\newcommand*{\Ge}{\epsilon}
\newcommand*{\Gg}{\gamma}
\newcommand*{\GG}{\Gamma}
\newcommand*{\Gk}{\kappa}
\newcommand*{\Gl}{\lambda}
\newcommand*{\GL}{\Lambda}
\newcommand*{\Gm}{\mu}
\newcommand*{\Gom}{\omega}
\newcommand*{\Gr}{\rho}
\newcommand*{\Gs}{\sigma}
\newcommand*{\Gt}{\tau}
\newcommand*{\Gth}{\theta}
\newcommand*{\Gf}{\phi}
\newcommand*{\Gp}{\psi}

%% bold math, does not work with Charter fonts
\usepackage{bm}
\newcommand{\bmr}[1]{\bm{\mathrm{#1}}}

%% Delimiters, using mathtools
%% \left and \right sometimes give poor spacing
%% See http://bit.ly/2wjMAHR 
%% NOTE: must use \protect inside figure captions, eg, \protect\abs{}

\DeclarePairedDelimiter\abs{\lvert}{\rvert}
\DeclarePairedDelimiter\norm{\lVert}{\rVert}
\DeclarePairedDelimiter\angb{\langle}{\rangle}
\DeclarePairedDelimiter\lrb{\lbrack}{\rbrack}
\DeclarePairedDelimiter\lr{\lparen}{\rparen}
\DeclarePairedDelimiter\lrbr{\lbrace}{\rbrace}

\makeatletter
\let\oldabs\abs \def\abs{\@ifstar{\oldabs}{\oldabs*}}
\let\oldnorm\norm \def\norm{\@ifstar{\oldnorm}{\oldnorm*}}
\let\oldangb\angb \def\angb{\@ifstar{\oldangb}{\oldangb*}}
\let\oldlrb\lrb \def\lrb{\@ifstar{\oldlrb}{\oldlrb*}}
\let\oldlr\lr \def\lr{\@ifstar{\oldlr}{\oldlr*}}
\let\oldlrbr\lrbr \def\lrbr{\@ifstar{\oldlrbr}{\oldlrbr*}}
\makeatother

%% Coloring for text, useful to highlight revisions
%% See https://ctan.org/pkg/xcolor?lang=en, using x11names

\newcommand{\txtcolor}{Goldenrod3} % Goldenrod3, Brown2
\newcommand{\tclr}[1]{\textcolor{\txtcolor}{#1}}

%% Common notation
%% spacing varies in DeclareMathOperator v def
%% choose according to which looks better

\DeclareMathOperator{\E}{E}
\newcommand*{\dd}{\textrm{d}}
%\newcommand*{\ee}{\textrm{e}}

%% Eq and Fig numbering

\newcommand*{\EEq}[1]{Eqn~\ref{eq:#1}}
\newcommand*{\Eq}[1]{eqn~\ref{eq:#1}}
\newcommand*{\Eqq}[1]{eqns~\ref{eq:#1}}
\newcommand*{\Eqs}[2]{eqns~\ref{eq:#1}--\ref{eq:#2}}
\newcommand*{\Eqp}[1]{(Eq.~\ref{eq:#1})}

\newcommand*{\brr}[1]{\overline{#1}}
\newcommand*{\dovr}[2]{\frac{\dd #1}{\dd #2}}
\newcommand*{\ddovr}[2]{\frac{\dd^2 #1}{\dd #2^2}}
\newcommand*{\prt}{\partial}
\newcommand*{\povr}[2]{\frac{\prt #1}{\prt #2}}
\newcommand*{\vct}[1]{\hbox{\bf #1}}

\newcommand*{\Figure}[1]{Figure~\ref{fig:#1}}
\newcommand*{\Fig}[1]{Fig.~\ref{fig:#1}}

%% horiz rules for text notes

\newcommand*{\boldrule}{\hrule height 1.2pt}
\newcommand*{\noterule}{\medskip\boldrule\medskip}	% for notes
\newcommand{\noterulenote}[1]{\bigskip\boldrule\nobreak\medskip\nobreak%
	{\bf\noindent #1}\nobreak%
	\medskip\nobreak\boldrule\medskip}
\newcommand*{\noterulenotec}[1]{\bigskip\boldrule\nobreak\medskip\nobreak%
	\centerline{\bf{\noindent #1}}\nobreak%
	\medskip\nobreak\boldrule\medskip}
\newcommand*{\dnoterule}{\noterulenote{}}
