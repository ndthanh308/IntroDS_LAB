\documentclass[aps, notitlepage, superscriptaddress, pre, reprint]{revtex4-1}

\usepackage{graphicx, hyperref, natbib, titlesec, paralist, amsmath, amssymb, color,array, caption, subcaption, multirow}
\def\*#1{\mathbf{#1}}
\captionsetup{justification=raggedright, singlelinecheck=false}

\newcommand{\AP}[1]{\textcolor{red}{#1}}
\newcommand{\CA}[1]{\textcolor{blue}{#1}}
\begin{document}

\title{Obstructed swelling and fracture of hydrogels}

\author{Abigail Plummer}
\thanks{A.P. and C.A. contributed equally to this work.}
\affiliation{Princeton Center for Complex Materials, Princeton University, Princeton, NJ 08544}

\author{Caroline Adkins}
\thanks{A.P. and C.A. contributed equally to this work.}
\affiliation{Department of Civil and Environmental Engineering, Princeton University, Princeton, NJ 08544}

\author{Jean-Fran\c{c}ois Louf}
\affiliation{Department of Chemical and Biological Engineering, Princeton University, Princeton, NJ 08544}
\affiliation{Department of Chemical Engineering,  Auburn University, Auburn, AL 36849}

\author{Andrej Ko\v{s}mrlj}
\thanks{To whom correspondence may be addressed. \\Email: \tt{andrej$@$princeton.edu} or \tt{ssdatta$@$princeton.edu}.}
\affiliation{Department of Mechanical and Aerospace Engineering, Princeton University, Princeton, NJ 08544}
\affiliation{Princeton Materials Institute, Princeton University, Princeton, NJ 08544}

\author{Sujit S. Datta}
\thanks{To whom correspondence may be addressed. \\Email: \tt{andrej$@$princeton.edu} or \tt{ssdatta$@$princeton.edu}.}
\affiliation{Department of Chemical and Biological Engineering, Princeton University, Princeton, NJ 08544}

\date{\today}

\begin{abstract}

%\noindent \textit{\textbf{Abstract}}- \\
Obstructions influence the growth and expansion of bodies in a wide range of settings---but isolating and understanding their impact can be difficult in complex environments. Here, we study obstructed growth/expansion in a model system accessible to experiments, simulations, and theory: hydrogels swelling around fixed cylindrical 
obstacles with varying geometries. When the obstacles are large and widely-spaced, hydrogels swell around them and remain intact. In contrast, our experiments reveal that when the obstacles are narrow and closely-spaced, hydrogels unexpectedly fracture as they swell. We use finite element simulations to map the magnitude and spatial distribution of stresses that build up during swelling at equilibrium, providing a route toward predicting when this phenomenon of self-fracturing is likely to arise. Applying lessons from indentation theory, poroelasticity, and nonlinear continuum mechanics, we also develop a theoretical framework for understanding how the maximum principal tensile and compressive stresses that develop during swelling are controlled by obstacle geometry and material parameters. These results thus help to shed light on the mechanical principles underlying growth/expansion in environments with obstructions.
%\\ 

%\noindent\textit{\textbf{Significance statement}}-\\ Consider a body that is increasing in size by e.g., biological growth, fluid uptake, chemical reactions, or phase transitions. This process often does not happen in isolation. Instead, in many scenarios---in everyday life, and in agriculture, biology, and manufacturing---the body encounters obstructions as it grows/expands. How do obstructions affect growth/expansion? Here, we address this question using studies of hydrogels, polymer networks that swell in water, in obstacle arrays. Remarkably, we find that when the hydrogel is sufficiently obstructed, it tears itself apart as it swells. We develop theoretical and computational tools to understand the magnitude and distribution of swelling-induced stresses that lead to this phenomenon, helping to provide a framework to predict and control obstructed growth/expansion.\\

%\noindent\textit{\textbf{Keywords}}- \\
%swelling $|$ obstacles $|$ hydrogels $|$ fracture $|$ elasticity %3-5 keywords
\end{abstract}

\maketitle
Many growth and expansion processes are sculpted through confinement by rigid obstructions. Familiar examples include muffins rising into their characteristic shape during baking \cite{ding2019thermomechanical}, trees growing around boulders \cite{taylor2021mechanism}, and even cities expanding around inhospitable geographic features \cite{borsdorf2020urban}. Obstructed growth and expansion also play pivotal roles---both harmful and beneficial---in many practical applications. For example, excessive tissue growth around metal mesh tubes inserted into blood vessels is a common, but life-threatening, complication of stenting \cite{byrne2015stent, kuhl2007computational, cheng2021finite}; conversely, the expansion of spray foam into cracks and in between walls underlies the thermal insulation of many energy-efficient buildings \cite{cabeza2010experimental}. More broadly, obstructed growth and expansion critically influence the emergence of form and function across diverse non-living and living systems, ranging from hydrogels added to soil for water retention to  biofilms and biological tissues in complex environments~\cite{louf_under_2021, beebe2000functional, alben2022packing, chu2018self, streichan2014spatial, bengough1997biophysical, alric2022macromolecular}. Therefore, we ask: are there general principles that dictate how obstructions influence growth and expansion? And if so, how do we discover them?


% Figure environment removed

When the growth/expansion of a body is resisted by surrounding obstructions, large and spatially-nonuniform
stresses can develop, influencing subsequent growth/expansion in turn~\cite{ambrosi2019growth, goriely}. Being able to understand the distribution and magnitude of these stresses is thus a necessary step in the development of widely-applicable, predictive models of growth and expansion. Unfortunately, due to the complexity of real-world materials and environments, model systems in which the coupling between growth and stress can be systematically studied are scarce. Here, we address this issue using studies of spherical hydrogel beads swelling in 3D-printed obstacle arrays with defined geometries. Hydrogels are cross-linked networks of hydrophilic polymers that can absorb large amounts of water and swell while still retaining integrity. As a result, they are extensively used in biomedical, environmental, and manufacturing
applications, and have well-characterized and highly-tunable properties such as their degree of swelling and elasticity \cite{zhang2017advances, oyen_mechanical_2014, deligkaris2010hydrogel, guo2020hydrogels}. Indeed, the comprehensive theoretical literature on hydrogel swelling makes computations of shape and internal stresses accessible for a variety of boundary conditions. While much work has focused on the case of a hydrogel swelling while adhered to another material~\cite{tanaka1987mechanical, hong2008theory, hong2009inhomogeneous,kang2010variational,  marcombe2010theory, amar2010swelling,dervaux2012mechanical, kim2010dynamic, cho2019crack}, non-adhered swelling around obstructions has received limited attention, primarily being considered in the context of indentation testing~\cite{hui2006contact, hong2008theory, hu2010using, bouklasfem}. 

Our study extends this body of work. First, we use experiments to directly visualize how hydrogel swelling is altered by obstacles of systematically-varying geometries. When obstacles are positioned further apart, the hydrogel swells through the spaces between them and maintains its integrity. By contrast, when the obstacles are closer together, we observe a surprising phenomenon: the hydrogel fractures, repeatedly tearing itself apart as it swells! We then use theory and simulations to rationalize these observations, and importantly, quantify and understand the distribution of stresses. Taken together, our work provides a prototypical example of obstructed growth/expansion and uncovers unexpected swelling behaviors and mechanical instabilities that can result during this process---highlighting the rich physics waiting to be explored in this area of soft mechanics.


\section{Experiments}
Our experimental platform is schematized in Fig. \ref{fig:schematic}A and detailed in \textit{Materials and Methods}. To define the obstacles, we 3D-print four rigid cylindrical columns of radius $r_{\rm{obs}}$ to be placed an equal distance $r_{\rm{ctr}}$ from a central point; in the experiments, we vary $r_{\rm{obs}}$ and $r_{\rm{ctr}}$ between $2.5-15$~mm and $3.5-25$~mm, respectively. The cylinders are securely attached to horizontal, parallel laser-engraved acrylic plates spaced vertically by a fixed amount, $\Delta z=40$~mm. Importantly, these plates are transparent, permitting direct visualization of a hydrogel as it swells between the cylindrical obstacles and parallel plates. Hence, at the beginning of each experiment, we place a spherical polyacrylamide hydrogel bead of initial radius $\sim 6$~mm (characterized in SI Appendices A, B) in the center and submerge the entire apparatus in a bath of ultrapure milli-Q water---thereby initiating swelling, which we track using a camera focused on the top plate. 

We first examine the case of obstacles that are spaced further apart. As the hydrogel swells, it contacts the top and bottom plates, as well as the cylinders, and continues to swell through the space between them (Fig.~\ref{fig:schematic}B, Movie S1). It eventually reaches an unchanging four-lobed equilibrium shape that reflects the balance between the osmotic pressure driving swelling and the elastic stresses arising from the polymer network and hydrogel-obstacle contact~\cite{louf_poroelastic_2021, louf_under_2021}. 

The case of closer-spaced obstacles of the same size is dramatically different. We observe similar behavior to the previous case of less confinement at early times: the hydrogel contacts the surrounding surfaces and swells through the space between them. As these lobes continue to swell, however, cracks abruptly form at the hydrogel surface (48~h in Fig.~\ref{fig:schematic}C), reflecting the development of large stresses during obstructed swelling. Remarkably, this process then continues, resulting in elaborate, multi-step fracturing of the hydrogel as it swells, repeatedly ejecting fragments of the hydrogel outward (Fig.~\ref{fig:schematic}C, Movie S2). This process eventually stops as the hydrogel reaches its final equilibrium degree of swelling. Several other representative examples of fracturing hydrogels with different obstacle geometries are shown in Movies S3 and S4. The fracturing process varies significantly between samples, reflecting the acute sensitivity of crack formation and propagation on the random imperfections in the hydrogel and the complex topography arising from previous fracturing~\cite{gdoutos2020fracture,wang2022hidden, li2023crack}. To our knowledge, this is the first report of fracture induced by obstructed swelling.


% Figure environment removed

These observations suggest that, when sufficiently obstructed, a growing/expanding body can tear itself apart. To characterize the dependence of this phenomenon on confinement, we repeat these experiments with obstacles of varying sizes $r_{\rm{obs}}$ and spacings $r_{\rm{ctr}}$. We observe a similar phenomenon in all cases, as summarized in the state diagram in Fig. \ref{fig:threshold}. When the obstacles are far apart, the hydrogel swells and retains its integrity ($\square$), while when the obstacles are closer i.e., $r_{\rm{ctr}}$ is below a threshold value, the hydrogel repeatedly self-fractures as it swells ($\times$). This threshold value also depends on the size of the obstacles. One might expect that fracturing is promoted when the hydrogel has less free space to swell into i.e., larger $r_{\rm{obs}}$; in stark contrast to this expectation, we find that for a given $r_{\rm{ctr}}$, fracturing occurs \textit{below} a threshold value of $r_{\rm{obs}}$. We rationalize this intriguing observation using theory and simulations in the following sections. The two geometric parameters $r_{\rm{ctr}}$ and $r_{\rm{obs}}$ thus delineate a boundary between swelling without fracture and obstructed swelling causing fracture, as shown by the light and dark green regions, respectively, in Fig. \ref{fig:threshold}. 

To demonstrate the generality of this phenomenon, we repeat our experiments in fully 3D granular packings, which more closely mimic the obstructions experienced by hydrogels in many applications \cite{misiewicz2022characteristics, krafcik2017improved, kapur1996hydrodynamic, dolega2017cell, lee2019dispersible, woodhouse1991effect, demitri2013potential, guilherme2015superabsorbent, lejcus2018swelling, wei2013effect}. Our previous experiments \cite{louf_under_2021} using this platform used a solvent that promoted only slight hydrogel swelling, and therefore accessed the case of weak confinement; however, repeating these experiments with a solvent that promotes more hydrogel swelling indeed gives rise to self-fracturing, as shown in Fig. S4 (SI Appendix C). Thus, this phenomenon of obstructed growth/expansion causing fracture arises not just in idealized geometries, but also in more realistic complex spaces. 

\section{Theory and simulations}
What are the essential physical principles that govern this phenomenon? To address this question, we develop finite element simulations that incorporate the energetic penalty of contacting obstructions into a model of hydrogel swelling based on classic Flory-Rehner theory. Our goal is to better understand the complex distribution of stresses that arises during obstructed swelling, not to quantitatively capture all the features of the experiments; as such, we use a simplified two-dimensional (2D) model that permits straightforward visualization of stresses and thereby enables us to develop an intuitive and analytic description of the underlying physics. 

We describe stretching of the hydrogel polymer network and solvent-polymer interactions with the commonly-used free energy density \cite{floryrehner, flory, hong2008theory, hong2009inhomogeneous, kang2010variational, bouklasfem}:
\begin{align}
    \frac{F_{\rm{en}}}{A_0 k_B T}&=\frac{n_p}{2}  \left( F_{iK} F_{iK} - 2 - 2 \ln(\det(\*F))\right) \nonumber \\&+\frac{1}{\Omega} \left(\Omega C \ln \left( \frac{\Omega C}{1+\Omega C} \right) + \chi \frac{\Omega C}{1+\Omega C} \right). \label{eq:freeen}
\end{align}
The first term describes the elastic free energy: $n_p$ is the number of polymer chains per unit dry reference area $A_0$ and $F_{iK}=\frac{\partial x_i}{\partial X_K}$ is the deformation gradient tensor, with $F_{iK} F_{iK}=\mathrm{tr}(\*F^T \*F)=\sum_i \lambda_i^2$ in terms of principal stretches $\lambda_i$. Following standard conventions, deformed/current configurations are denoted by lowercase letters and reference/initial configurations are denoted by capital letters unless otherwise noted \cite{goriely}. The second term describes the mixing free energy: $\Omega$ is the area of a solvent molecule, $C$ is the nominal concentration of solvent (number of solvent molecules per unit reference area), and $\chi$ is the Flory-Huggins interaction parameter. Both terms are scaled by the product of the Boltzmann constant and temperature, $k_B T$. 

Given that the hydrogel network is held together by permanent cross-links between its polymer chains, we additionally assume that it only changes volume by uptake of solvent, which allows us to express concentrations in terms of the deformation: $\det(\*F)=1+\Omega C$. We require the chemical potential $\mu$ to be zero on the boundary of the hydrogel to mimic submerging it in pure water, as in the experiments. To impose this boundary condition, we perform the standard Legendre transform of Eq. \eqref{eq:freeen} and derive a new free energy $\hat{F}_{\rm{en}}$ in terms of $\mu$ as $\hat{F}_{\rm{en}}(x_i, \mu) =F_{\rm{en}}(x_i, C)-\mu C$ \cite{hong2009inhomogeneous}.  Finally, we model contact by imposing an energy penalty for overlap between the hydrogel and obstacles; as detailed in \textit{Materials and Methods}, our results are insensitive to the choice of this parameter. Note that the chemical potential boundary condition $\mu=0$ is enforced in the regions of contact between the hydrogel and the obstacles (see SI Appendix E for further discussion).

% Figure environment removed
%Obstacle radii and separations are given in terms of $r^*$, the 2D equilibrium hydrogel radius in the absence of confinement. Thus, $r_{\rm{ctr}}/r^* = 1$ permits unobstructed swelling. 

To visualize and track hydrogel deformation during obstructed swelling, along with the concomitant development of internal stresses, we implement this model in FEniCS \cite{alnaes2015fenics}, an open-source finite element platform. Our simulations consider a circular hydrogel swelling around four fixed obstacles. Although our primary focus is the stress distribution in the hydrogel at equilibrium, we simulate the full dynamics of obstructed swelling to provide numerical stability and ensure that there is a realistic path from the initial to the final equilibrium swollen state (SI Appendix E). 

We begin by examining hydrogel swelling around obstacles that have the same size, but varying spacing---just as in the experiments shown in Fig. \ref{fig:schematic}(B,C). Four representative examples, varying from the case of weak to strong confinement, are shown in Fig. \ref{fig:sims1}(A-D). The corresponding maximum tensile and compressive principal stresses on the line connecting the hydrogel center to an obstacle center are plotted in panels (E,F). We quantify these variations using $\Delta/r^*\equiv\left(r^*-r_{\rm{ctr}}\right)/r^*$, where $r^*$ is the equilibrium, fully-swollen, unconfined radius of the hydrogel and $\Delta$ is the difference between $r^*$ and the distance from the center to the closest obstacle $r_{\rm{ctr}}$.  The case of weak confinement in (A) has $\Delta/r^*=0.02$, increasing to $\Delta/r^*=0.6$ for the case of strong confinement in (D). As seen by the color maps of solvent concentration and principal stresses in panel~(A), in this limit of weak confinement, the hydrogel is barely deformed and the resultant stresses are not visible when plotted on the same scale as (B--D). Thus, as we describe in Sec. \ref{sec:weak}, the hydrogel stresses can be captured analytically using linear theories in this regime. Increasing hydrogel confinement causes the magnitudes of the stresses to increase (panels E,F), but they still remain primarily localized near the obstacles; as discussed in Sec. \ref{sec:intermediate1} and Sec. \ref{sec:intermediate2}, the largest compressive stresses can still be described by linear theory, but the largest tensile stresses require consideration of a geometric nonlinearity. As the separation between obstacles decreases further, large compressive stresses span the entirety of the hydrogel (panel C), eventually triggering a symmetry-breaking instability (panel D) as discussed in Sec. \ref{sec:strong}.  Finally, while our model is not suitable to treat fracture directly, we discuss the connection between our calculations of stresses and the experimental observations of swelling-induced self-fracture in Sec. \ref{sec:comparison}.

\subsection{Weak confinement}\label{sec:weak}
Consider a hydrogel disk that has swollen around obstacles to reach equilibrium. The chemical potential is spatially uniform and therefore all solvent transport has stopped. Nonetheless, due to contact with the obstacles, the distribution of solvent is inhomogeneous through the hydrogel: solvent preferentially enters the uncompressed lobes of the hydrogel between the obstacles, as shown by the yellow color in Fig.~\ref{fig:schematic}B~\cite{louf_poroelastic_2021}. Now, consider another hydrogel that was first swollen, unobstructed, to its equilibrium size, and then slowly and incrementally squeezed by an identical set of obstacles moving towards its center, acting as four indenters. Solvent must exit the hydrogel where it is indented by the obstacles; recall our condition $1+ \Omega C= \det(\*F)$. For the same final obstacle geometry, these two scenarios must have identical solvent distributions and stress profiles at equilibrium. Thus, the long time limit of obstructed swelling can be treated as an indentation problem, which is well-studied in the limit of small deformations. Making this analogy between obstructed swelling and indentation allows us to apply lessons from a large body of literature on linear contact mechanics and poroelasticity to derive expressions for the stress tensor in the hydrogel \cite{hui2006contact, hu2010using, goriely2016stress, weickenmeier2016mechanics}.

Assuming that deformations relative to the fully swollen state are small, we linearize Eq. \eqref{eq:freeen} to find effective linear elastic parameters (SI Appendix F, \cite{hu2011indentation, bouklasporo}). In particular, for a 2D hydrogel, the effective Poisson's ratio $\nu$ and Young's modulus $E$ are given by
\begin{align}
    \nu&=1-2 n_p \Omega\left(n_p \Omega \left(1+\frac{1}{\lambda_0^2}\right) + \frac{1}{\lambda_0^2 (\lambda_0^2-1)}-\frac{2 \chi}{\lambda_0^4}\right)^{-1}\label{poisson},\\
     E&= 2(1+\nu) n_p k_B T \label{youngs},
\end{align}
where $\lambda_0$ is the principal stretch corresponding to the fully swollen state. The expression for the Poisson's ratio can be understood as its value for an incompressible 2D solid, $\nu=1$, minus a correction---reflecting the fact that the compressibility of the swollen hydrogel is generated via solvent transport. This linearization also yields the shear modulus, $G_0=n_p k_B T$, which we use to normalize stresses throughout this paper. 

Given these effective elastic parameters, we solve for the stresses in the hydrogel as a 2D linear contact mechanics problem (SI Appendix G, \cite{barber2018contact,johnson}).
This approach provides expressions for the stress tensor $\sigma_{ij}$ along the line directly beneath the top obstacle as a function of $y$ as shown in Fig. \ref{fig:contactparams}B:
%, where varying $y$ from $0$ to $r^*$ takes us from the center of the hydrogel to the surface of the circular indenter

    \begin{align}
    \sigma_{xx}&=\frac{2 \zeta}{\pi} \Bigg(\frac{2 r^{* 3}}{(r^{*2}+y^2)^2} -\frac{1}{r^*}-\frac{2 (r^*-y)}{a^2}\nonumber\\&+ \frac{2 (r^*-y)^2 +a^2}{a^2 \sqrt{(r^*-y)^2 +a^2}} \Bigg),\label{eq:sxx}\\
    \sigma_{yy}&=\frac{2 \zeta}{\pi} \left( \frac{1}{r^*+y} -\frac{1}{r^*}+\frac{2 r^* y^2 }{(r^{*2}+y^2)^2}+ \frac{1}{\sqrt{(r^*-y)^2 + a^2}} \right),\label{eq:syy}\\
    \sigma_{xy}&=0,
    \end{align}
where $\zeta<0$ is the force applied to the indenters and the half contact width $a$ is
\begin{equation}
    a=\sqrt{- \frac{4 \zeta}{E \pi \left( \frac{1}{r_{\rm{obs}}}+\frac{1}{r^*}\right)}},
\end{equation}
as defined in Fig. \ref{fig:contactparams}B. Note that $y$ is defined with respect to the hydrogel's unobstructed, fully-swollen state and ranges from $-r^*$ to $r^*$.

% Figure environment removed


Given these analytical expressions, we next ask: Under what confinement regimes (as parameterized by $\Delta/r^*$) does this linearized theory work, and when does it break down? And since we would like to gain an intuitive understanding of obstructed swelling-induced fracture, how does the maximal principal tensile stress --- which can be used to approximate material strength \cite{gdoutos2020fracture}--- vary with confinement? To address these questions, we first re-cast Eqs.~\eqref{eq:sxx}--\eqref{eq:syy} in terms of $\Delta/r^*$ to facilitate direct comparison with the results of the nonlinear simulations. To do so, we integrate the strain $u_{yy}=\frac{\sigma_{yy}}{Y}-\frac{\nu}{Y} \sigma_{xx}$ using Eqs. \eqref{eq:sxx}--\eqref{eq:syy} to find the displacement at the surface of the indenter relative to the center of the hydrogel, and expand the result to linear order in $a/r^*$. This procedure gives
\begin{equation}
    \Delta = -\frac{\zeta}{E \pi} \left(\ln \left(\frac{16 r^{*2}}{a^2}\right) + \frac{1}{2} (\pi -6 - \pi \nu) \right),
\end{equation}
which we then invert, apply Eqs.~\eqref{poisson}--\eqref{youngs}, and substitute the resulting expression for $\zeta(\Delta)$ into Eqs.~\eqref{eq:sxx}--\eqref{eq:syy}. The resulting expressions for $\sigma_{xx}(\Delta)$ and $\sigma_{yy}(\Delta)$ cannot be expressed in terms of elementary functions, so we omit them, but are shown by the solid lines in Fig.~\ref{fig:contactparams} for the illustrative case of $r_{\rm{obs}}/r^*=0.3$; for comparison, the symbols show the results of the full nonlinear simulations. 

As expected, when $\Delta/r^* \ll 1$, the linearized indentation solution agrees well with the nonlinear simulation results, while as $\Delta/r^*$ increases, the discrepancy between the two becomes more apparent. In particular, as exemplified by the data for $\Delta/r^*=0.16$ (dark blue circles) and $\Delta/r^*=0.13$ (yellow squares) in Fig.~\ref{fig:contactparams}, the linear solution underestimates the compression at both the center $y=0$ and boundary $y=r^*$ of the hydrogel, as well as the tension that builds up at $y/r^* \approx 0.75$ (Fig. \ref{fig:contactparams}A, inset). Indeed, though tension ($\sigma_{xx}>0$) does appear in the linear solutions for small $\Delta/r^*$, it disappears with increasing $\Delta/r^*$, in contrast to the simulation results (see arrow in Fig. \ref{fig:sims1}C, for example). 

Since we are interested in fracture behavior, we thus focus our attention on the maximal value of this tensile stress for the same illustrative case of $r_{\rm{obs}}/r^*=0.3$. By symmetry, we expect the largest stresses to lie beneath each obstacle, since stresses must go to zero at the hydrogel boundary away from obstacles in the weak confinement regime. Moreover, because the straight edges of the finite element mesh can introduce spurious tensile forces at the edge of the hydrogel-obstacle contact (described further in SI Appendix I), we plot the maximum and minimum values of the principal stresses along the $x=0$ line shown in the schematic inset of Fig.~\ref{fig:contactparams}B. The results are displayed in Fig.~\ref{fig:sims1}(E,F). As noted in Fig.~\ref{fig:contactparams}A, the linear theory only predicts the presence of tension for small confinement before deviating from the nonlinear simulation results at $\Delta/r^*\gtrsim0.02$, as shown by the solid line and points in Fig.~\ref{fig:sims1}E, respectively. Interestingly, however, linear theory captures the maximum compressive stress over a broader range of confinement, shown by the solid line in Fig. \ref{fig:sims1}F, which agrees well with the simulation data up to $\Delta/r^* \approx 0.2$. 

Thus, while linear indentation theory can predict both tensile and compressive stresses during obstructed swelling in weak confinement, it underpredicts both for larger deformations---suggesting that the assumptions made in the linear theory are no longer valid. We revisit these assumptions for both tension and compression in the next two sections, respectively. 

\subsection{Tension beyond the linear regime}\label{sec:intermediate1}
The linear theory in the previous section relies on a number of assumptions that can fail as deformations increase:
\begin{enumerate}
\item The effective elastic parameters [Eqs. \eqref{poisson}--\eqref{youngs}] are independent of strain,
\item The stress is linearly related to the strain,
\item The strain tensor is linear in the displacements.
\end{enumerate}
To assess the validity of these assumptions, we compare the results of our full nonlinear simulations to those of more complex elastic models that incorporate \emph{material}/\emph{geometric} nonlinearities.

First, we explore the limits of assumption 1 by relaxing assumptions 2 \& 3. Specifically, we compare the hydrogel simulation results to those of a compressible neo-Hookean elastic material with elastic parameters given by Eqs. \eqref{poisson}--\eqref{youngs}. As detailed in SI Appendix F, the neo-Hookean model closely reproduces the stress profiles of the hydrogel simulations over a broad range of $\Delta/r^*$ up to $\approx 0.4$, well beyond the limits of the linear theory at $\sim 0.02$. Therefore, nonlinearities due to the effective elastic parameter mapping can be neglected up to this point. 

Next, we explore the limits of assumption 2 by relaxing assumption 3. Specifically, we use a St. Venant-Kirchoff elastic model with strain tensor $u_{ij}=\frac{1}{2} \left( \frac{\partial u_i}{\partial x_j} + \frac{\partial u_j}{\partial x_i} + \frac{\partial u_{k}}{\partial x_i} \frac{\partial u_k}{\partial x_j}\right)$; thus, the strain tensor is nonlinear in displacements, but we still require that the hydrogel material follows a linear constitutive law. Note that derivatives are taken with respect to coordinates in the unobstructed, fully-swollen state, denoted here with lowercase letters for simplicity of presentation. Intriguingly, as detailed in SI Appendix H, the St. Venant-Kirchhoff model quantitatively reproduces the maximum principal tensile stress in the hydrogel simulations up to $\Delta/r^*\approx 0.1$, well beyond the limit of the linear theory at $\approx 0.02$. Hence, the excess tension $\sigma_{xx}$ that develops beneath the obstacles just beyond the linear regime is driven by \emph{geometric} nonlinearity, and does not require a nonlinear constitutive relationship. At even larger displacements $\Delta/r^* > 0.1$, the St. Venant-Kirchoff simulations are unstable and we expect that both geometric and material nonlinearities contribute to the tensile stress. %Geometric nonlinearities are significant in materials experiencing large rotations---in our system, these rotations appear as the hydrogel wraps itself around the obstacle.

How exactly does geometric nonlinearity generate tension during obstructed swelling? We answer this question using an illustrative argument reminiscent of the derivation of the F\"oppl-von K\'arm\'an equation \cite{landau}. As detailed in \emph{Materials and Methods}, we first find the variation of the integrated St. Venant-Kirchhoff strain-energy function with respect to displacements, which can be written in terms of the second Piola-Kirchhoff stress tensor, $\sigma_{ij}^{PK}$. In 2D, this quantity gives the stress component in a material direction $i$ perpendicular to a line that has unit length and normal $j$ in the reference configuration \cite{audoly2010elasticity}. We then make approximations specific to our obstacle geometry. Ultimately, we find
\begin{equation}
    \sigma_{xx}^{PK} \approx -\frac{\partial \sigma_{yy}^{PK}}{\partial y}\left(1+\frac{\Delta}{r^*} \right)r_{\rm{obs}}.\label{eq:sigxx}
\end{equation}
Since $\sigma_{yy}^{PK}$ becomes more negative as $y$ increases, $\frac{\partial \sigma_{yy}^{PK}}{\partial y}<0$ (Fig. \ref{fig:contactparams}B), and therefore $\sigma_{xx}^{PK}>0$ indicating tension. Thus, geometric nonlinearity generates tension beneath an indenter perpendicular to the indentation direction, qualitatively matching our simulations.

\subsection{Compression beyond the linear regime}\label{sec:intermediate2}
A notable result shown in Fig.~\ref{fig:sims1} is that while linear indentation theory predicts tensile stresses for $\Delta/r^*<0.02$, it captures the compressive stresses over a broader range, up to $\Delta/r^*\approx0.2$. Which nonlinearities drive the deviations that arise at even larger displacements? We answer this question by following the same procedure as in the previous section, detailed further in SI Appendix F \& H. In contrast to the case of tension, we 
do not find any parameters for which the St. Venant-Kirchoff model is more accurate than the linear model. Furthermore, as shown in Fig. \ref{fig:sims1}F, past the linear regime, the compressive (Cauchy) stress underneath the top obstacle scales like that of a compressible neo-Hookean elastic material experiencing uniaxial compression, with the principal stretch parallel to indentation set to $\lambda_1=1-\Delta/r^*$ and the principal stretch perpendicular to the indentation set to 1:
\begin{equation}
\sigma_{yy} \sim G_0 \left(1-\frac{\Delta}{r^*}\right) + \frac{2 G_0 \nu}{1-\nu} \frac{\ln\left(1-\frac{\Delta}{r^*}\right)}{\left(1-\frac{\Delta}{r^*}\right)} - \frac{G_0}{\left(1-\frac{\Delta}{r^*}\right)}.\label{eq:nh}
\end{equation}
Thus, unlike the case of tension, deviations from the linear theory do not arise from geometric nonlinearities and can instead be attributed to \emph{material} nonlinearities. 

\subsection{Symmetry-breaking instability}\label{sec:strong}
A striking phenomenon arises in our simulations as the separation between obstacles decreases further: as shown in Fig.~\ref{fig:sims1}D, the hydrogel displays a symmetry-breaking instability and swells preferentially along a diagonal. Why does this instability arise? Inspecting the spatial distribution of compressive stresses provides a clue. As the hydrogel swells in increasing amounts of confinement, its central core becomes increasingly compressed (see, e.g., Fig.~\ref{fig:sims1}C--D). Compressing this circular core along a \emph{single} axis, forming an ellipse, requires less energy than does compressing this core isotropically. Thus, as confinement increases, one expects the case of asymmetric swelling to be energetically preferred, leading to this instability---as described further in SI Appendix J.


% Figure environment removed

\section{Comparison between simulations and experiments}\label{sec:comparison}
Our theoretical analyses and simulations capture the essential features of the hydrogel deformation observed in experiments (e.g., Figs.~\ref{fig:schematic}B and \ref{fig:sims1}C). They also enabled us to explore how stresses develop during obstructed swelling more generally (Figs.~\ref{fig:sims1}--\ref{fig:contactparams}). While our model is not suitable to directly treat the swelling-induced self-fracture observed experimentally, the simulated stresses help rationalize this phenomenon. To this end, we compare the simulations to the experimental state diagram shown in Fig.~\ref{fig:threshold} by plotting the maximum principal tensile and compressive stresses as a function of $r_{\rm{obs}}$ and $r_{\rm{ctr}}$. The results, shown in Fig.~\ref{fig:sims4}, bear a compelling resemblance to the experimental results. In particular, the convexity and shape of the experimental fracture boundary are similar to the simulated contours of maximum principal stress. Indeed, appealing to a commonly-used fracture criterion for brittle materials \cite{gdoutos2020fracture}, we expect that hydrogel fracture occurs when the maximum tensile stress exceeds a threshold --- whose exact value would establish the position of the experimental fracture boundary in Fig.~\ref{fig:sims4}A. We conjecture that this threshold is reached prior to the symmetry-breaking instability revealed by the simulations, as we do not observe it in our experiments; experiments using tougher hydrogels than those studied here may be a useful way to probe this deformation mode in future work. 

\section{Discussion}
Despite the ubiquity of obstructed growth and expansion in our everyday lives, how exactly this process generates large, spatially non-uniform stresses in a body --- and how these stresses influence its subsequent growth/expansion in turn --- has remained challenging to systematically study. One reason is the lack of model systems in which this intricate coupling between growth and stress can be probed both experimentally and theoretically. We addressed this need by studying the swelling of spherical hydrogel beads in obstacle arrays of tunable geometries. Our experiments revealed a striking phenomenon: under weak confinement, a hydrogel retains its integrity and assumes a symmetric, four-lobed clover-like shape, while in stronger confinement, it repeatedly tears itself apart as it swells. We elucidated the underlying physics by adopting established models of hydrogel swelling to map the tensile and compressive stresses arising during swelling. In particular, we found that when a hydrogel is weakly deformed, stresses are well-described by linear indentation theory, while as a hydrogel is increasingly deformed, geometric and material nonlinearities engender large tensile and compressive stresses tangential and normal to the obstacles, respectively, driving the hydrogel towards fracture. 

Because our study represents a first step toward fully unravelling the mechanics of obstructed growth and expansion, it necessarily has limitations. For example, while the experimental results shown in Fig.~\ref{fig:threshold} revealed a fracture threshold that varies with obstacle geometry, quantitative comparison to theory and simulation will require more precise control over the system geometry and dimensionality~\cite{joshimorpho, joshi2023effect}, both in experiments and in the model, along with a more detailed treatment of the microscopic processes underlying fracturing~\cite{lai2018probing,kuna2013finite, mao2018theory,wang2012delayed, mao2018theory, baumberger2020environmental, yang2022rate}. Moreover, many more experimental trials near this threshold and with hydrogels of varying mechanical properties, along with high-resolution imaging of crack propagation~\cite{leslie2021gel}, will be useful in characterizing the details of the fracturing process, which likely depend on the presence of microscopic imperfections in the hydrogel. Finally, although here we restricted our attention to the case of rigid obstacles, many scenarios involve growth/expansion around \emph{compliant} constraints---e.g., the development of biofilms, tissues, and organs in the body~\cite{helmlinger1997solid, zhang2021morphogenesis, fortune2022biofilm,goodwin2019smooth}, with potential implications for biological function~\cite{mobius2015obstacles, atis2019microbial}. Extending our study to the case of deformable obstacles would therefore be a useful direction for future work.

Our results may be especially relevant to diverse applications of hydrogels, and other swellable materials, that frequently involve their confinement in tight and tortuous spaces. For example, driven by growing demands for food and water, hydrogels are increasingly being explored as additives to soils to absorb and release water to plants and therefore reduce the burden for irrigation~\cite{misiewicz2022characteristics, woodhouse1991effect, demitri2013potential, guilherme2015superabsorbent, lejcus2018swelling}. They are also widely adopted in other applications, such as oil recovery, construction, mechanobiology, and filtration, all of which involve hydrogel swelling in confinement~\cite{krafcik2017improved, kapur1996hydrodynamic, dolega2017cell, lee2019dispersible}. A common assumption made in all these cases is that the hydrogel retains integrity as it swells; however, our study indicates that these applications should be evaluated for the possibility of swelling-induced self-fracture. Indeed, fracture could lead to the production and dispersal of many small hydrogel fragments, potentially reducing their utility and leading to environmental contamination. This process should therefore be carefully considered in a wide range of real-world contexts. 


\section*{Materials and Methods}
\subsubsection*{Experimental details}
To create the obstacle array, we 3D print cylindrical columns in Clear v4 resin using a Form3+ industrial 3D printer (Formlabs), and cut the acrylic plates using an Epilog Laser Mini 24 laser cutter and engraving system.  We secure the columns to the acrylic plates using a twist-and-lock mechanism. The hydrogels are polyacrylamide beads (``water gel beads" obtained from Jangostor) and are stored in a screw cap container prior to experiments; as such, they experience some slight swelling due to ambient humidity. The hydrogel beads have varying sizes and colors, but all appear to have the same swelling behavior and beads of similar sizes are used for experiments (SI Appendix A). These hydrogel beads were extensively characterized in our previous experiments~\cite{louf_under_2021, louf_poroelastic_2021}, which provide additional detail. 

Early in the swelling process (e.g. 2 h in Fig. \ref{fig:schematic}B,C), each hydrogel appears out of focus since it has not yet made contact with the top plate (focal plane for imaging), and cusps are visible on its surface due to differential swelling as water enters the hydrogel from its outer surface~\cite{tanaka1986kinetics, tanaka1987mechanical, bertrand2016dynamics, curatolo2017transient}. We verify that the hydrogel beads swell to an equilibrium shape without rupturing when no obstacles are present (either with or without the 40~mm $z$-confinement), indicating the transient stresses that engender these cusps are insufficient to drive fracture as in other less tough gels~\cite{de2016preparation, leslie2021gel}. Because the plates and obstacles are made of acrylic or a polymeric resin, respectively, we do not expect or see evidence of adhesion between the polyacrylamide hydrogel surface and the confining surfaces. For experiments in which we image the entire process of obstructed swelling, we verify that the hydrogel reaches an equilibrium shape (in the cases of no fracturing) when its size/shape does not noticeably change for at least three hours. In other experiments where we do not image the entire process of obstructed swelling, we verify that equilibrium is reached by waiting an additional 12 hours. Each symbol in Fig. \ref{fig:threshold} represents a single experiment.


\subsubsection*{Simulation details}
We create meshes using FEniCS's built-in mesh generation function with cell size set such that there are 30 vertices along the radius. We confirmed that this mesh resolution was sufficient for numerical convergence. We set $n_p \Omega=0.001$ and $\chi=0.3$. The penalty function is integrated over the hydrogel boundary, and is given by 
\begin{equation}
    \frac{F_{\rm{pen}}}{2 \pi r_0 k_B T}= \frac{p}{4 \pi r_0} \sum_i \langle r_{\rm{obs}}^2-(\*x-\*x_i)^2 \rangle_+^2,
\end{equation}
where $p$ is the penalty strength, $\*x_i$ is position of the center of the $i^{\rm{th}}$ obstacle, $r_0$ is the dry reference radius of the hydrogel, and the sum is over all obstacles. The brackets $\langle \cdots \rangle_+$ take the positive part of the argument, defined as $\langle x \rangle_+= \frac{x + |x|}{2}$. Thus, when evaluated at positions away from any obstacles, the penalty function is zero, but takes a large positive value inside the obstacles. We set $p=6.25 \pi/r_0^4$ to generate the data shown in this text, and have verified that using $p=62.5 \pi/r_0^4$ produces the same results.

Following the suggestions of Refs.~\cite{zhang2009finite, bouklasfem, angbouklas}, we use the backwards Euler scheme for time integration (see SI Appendix E for further discussion of dynamics), Taylor-Hood mixed elements (quadratic elements for the displacement field and linear elements for the chemical potential field), and early time ramping of the chemical potential boundary condition to ensure numerical stability. The Newton-Raphson method is used at each time step, and equilibrium is defined by when zero iterations are required for a step to complete. 

To find the maximum/minimum principal Cauchy stresses, we first calculate the eigenvalues of the stress tensor for a given displacement field. We project these eigenvalue fields onto a function space of discontinuous Lagrange elements of order 1. We then compare the eigenvalues defined on this mesh. The largest positive eigenvalue is the maximum principal (tensile) stress $\sigma_{\rm{max}}$, and the most negative eigenvalue is the minimum principal (compressive) stress, $\sigma_{\rm{min}}$. To find the minimum and maximum stresses beneath an obstacle (data in Fig. \ref{fig:sims1}(E,F)), we instead project the stress tensor onto the vertical line directly beneath the top obstacle. The minimum value of $\sigma_{yy}$ and the maximum value of $\sigma_{xx}$ along this line are plotted as the below top obstacle minimum and maximum stresses respectively.

\subsubsection*{Tension generated by geometric nonlinearity}

We describe the argument leading up to Eq. \eqref{eq:sigxx} in more detail, which demonstrates how tension can appear beneath an obstacle when a nonlinear strain tensor is used. The variation of the integrated St. Venant-Kirchhoff strain-energy function with respect to displacements in terms of the second Piola-Kirchhoff stress tensor is 
\begin{equation}
    \delta W= \int \sigma_{ij}^{PK} \delta u_{ij} dA= \int \sigma_{ij}^{PK}\left(\frac{\partial \delta u_i}{\partial x_j} + \frac{\partial u_k}{\partial x_i}\frac{\partial \delta u_k}{\partial x_j} \right) dA.
\end{equation}
Upon integrating by parts, assuming we cannot vary the displacements at the boundaries due to the presence of obstacles, we find
\begin{align}
   & \delta W= \int \left( \frac{\partial \sigma_{ij}^{PK}}{\partial x_j}\delta u_i +\frac{\partial}{\partial x_j}\left(\sigma_{ij}^{PK} \frac{\partial u_k}{\partial x_i} \right) \delta u_k\right) dA,  \nonumber \\
    &= \int \left( \frac{\partial \sigma_{ij}^{PK}}{\partial x_j}\delta u_i +\frac{\partial \sigma_{ij}^{PK}}{\partial x_j}\frac{\partial u_k}{\partial x_i} \delta u_k + \sigma_{ij}^{PK} \frac{\partial^2 u_k}{\partial x_i x_j} \delta u_k \right)dA. 
\end{align}
Next, in order to make approximations specific to our geometry, we explicitly list all the terms that appear for a two-dimensional solid. Since we will examine stresses directly beneath an obstacle, we set $\sigma_{xy}^{PK}=0$ by symmetry, yielding:
\begin{align}
    \delta W&\approx\int dA \delta u_x \Bigg( \frac{\partial \sigma_{xx}^{PK}}{\partial x}\left(1+\frac{\partial u_x}{\partial x} \right) +\frac{\partial \sigma_{yy}^{PK}}{\partial y}\frac{\partial u_x}{\partial y}\nonumber \\&+\sigma_{xx}^{PK} \frac{\partial^2 u_x}{\partial x^2}+\sigma_{yy}^{PK} \frac{\partial^2 u_x}{\partial y^2} \Bigg)  \nonumber\\
    &+ \int dA \delta u_y \Bigg( \frac{\partial \sigma_{yy}^{PK}}{\partial y}\left(1+\frac{\partial u_y}{\partial y}\right) +\frac{\partial \sigma_{xx}^{PK}}{\partial x}\frac{\partial u_y}{\partial x} \nonumber \\&+\sigma_{xx}^{PK} \frac{\partial^2 u_y}{\partial x^2}+\sigma_{yy}^{PK} \frac{\partial^2 u_y}{\partial y^2} \Bigg). 
\end{align}
In equilibrium, the coefficients of $\delta u_x$ and $\delta u_y$ must be zero (in the absence of body forces). We focus our attention on the coefficient of $\delta u_y$. If we consider the internal stresses directly beneath the obstacle, we can set $\partial u_y/\partial x$ to zero by symmetry. We approximate the deformation as an affine contraction in the $y$ direction, which sets $\partial u_y/\partial y\approx \Delta/r^*$ and $\partial^2 u_y/\partial y^2\approx 0$. We assume that the curvature of the $y$ displacement, $\frac{\partial^2 u_y}{\partial x^2}$, is determined by the curvature of the obstacle, $1/r_{\rm{obs}}$. With these substitutions, the equilibrium condition becomes
\begin{equation}
     \frac{\partial \sigma_{yy}^{PK}}{\partial y}\left(1+\frac{\Delta}{r^*} \right)+  \frac{\sigma_{xx}^{PK}}{r_{\rm{obs}}} \approx 0.
\end{equation}
Solving for $\sigma_{xx}^{PK}$, we find
\begin{equation}
    \sigma_{xx}^{PK} \approx -\frac{\partial \sigma_{yy}^{PK}}{\partial y}\left(1+\frac{\Delta}{r^*} \right)r_{\rm{obs}}.\label{eq:sigxx2}
\end{equation}
Just beyond the linear regime, $\sigma_{yy}^{PK}$ should follow the same trends as $\sigma_{yy}$ predicted by the linear theory. In Fig. \ref{fig:contactparams}B, we observe that $\sigma_{yy}$ decreases as $y$ increases ($\partial \sigma_{yy}/\partial y<0$) until it plateaus at $y/r^* \approx 1$. Thus, $\partial \sigma_{yy}/\partial y$ reaches its minimum a small distance away from the obstacle boundary, and our argument predicts that the largest geometric nonlinearity-generated tensions will appear there. This location is indeed where the greatest tensile stresses appear in simulations (arrow in Fig. \ref{fig:sims1}C). We can also compare the magnitude of the tension predicted by Eq. \eqref{eq:sigxx2} using $\sigma_{yy}$ from the linear theory with the tension measured in hydrogel simulations--- however, we find an estimate of the tension that is approximately five times larger than the maximum tension found via simulations and scales more gradually with increasing $\Delta/r^*$. This discrepancy is not surprising given the approximations made in this calculation.


\begin{acknowledgments}
It is a pleasure to acknowledge John Kolinski, Amaresh Sahu, and Carolyn Bull for helpful discussions, as well as funding from the NSF through the Princeton MRSEC DMR-2011750, the Project X Innovation Fund, a Princeton Materials Science Postdoctoral Fellowship (A.P.), and the Camille Dreyfus Teacher-Scholar Program of the Camille and Henry Dreyfus Foundation (S.S.D). \\

%\noindent Author contributions: A.P., A.K., and S.S.D. designed research; A.P., C.A., and J.-F. L. performed research; A.P., C.A., J.-F. L., A.K., and S.S.D. contributed new reagents/analytic tools; A.P., C.A., A.K., and S.S.D. analyzed data; and A.P., A.K., and S.S.D. wrote the paper.

\end{acknowledgments}

%\bibliography{biblio}
%  template.tex for Biometrics papers
%
%  This file provides a template for Biometrics authors.  Use this
%  template as the starting point for creating your manuscript document.
%  See the file biomsample.tex for an example of a full-blown manuscript. 
%  
%\documentclass[useAMS,referee,usenatbib]{biom}
\documentclass[useAMS,referee,usenatbib]{biom}

\usepackage{setspace}
\singlespacing

\def\bSig\mathbf{\Sigma}
\newcommand{\VS}{V\&S}
\newcommand{\tr}{\mbox{tr}}

\usepackage{amsmath}
\usepackage{amsfonts}
\usepackage{graphicx}
\usepackage{xcolor}
\usepackage{mathtools}


%%%%%%%%%%%%%%%%%%%%%%%%%%%%%%%%%%%%%%%%%%%%%%%%%%%%%%%%%%%%%%%%%%%%%

\title[Dynamic factor model for ILD]{A Continuous-Time Dynamic Factor Model for Intensive Longitudinal Data Arising from Mobile Health Studies}

\author
{Madeline R. Abbott\emailx{mrabbott@umich.edu} \\
Department of Biostatistics, University of Michigan--Ann Arbor, Ann Arbor, Michigan, U.S.A.
\and
Walter H. Dempsey\emailx{wdem@umich.edu} \\
Department of Biostatistics, University of Michigan--Ann Arbor, Ann Arbor, Michigan, U.S.A.
\and
Inbal Nahum-Shani\emailx{inbal@umich.edu} \\
Institute for Social Research, University of Michigan--Ann Arbor, Ann Arbor, Michigan, U.S.A.
\and
Cho Y. Lam\emailx{cho.lam@hci.utah.edu} \\
Department of Population Health Sciences and Huntsman Cancer Institute, University of Utah, \\ Salt Lake City, UT, U.S.A.
\and
David W. Wetter\emailx{david.wetter@hci.utah.edu} \\
Department of Population Health Sciences and Huntsman Cancer Institute, University of Utah, \\ Salt Lake City, UT, U.S.A.
\and
Jeremy M. G. Taylor\emailx{jmgt@umich.edu} \\
Department of Biostatistics, University of Michigan--Ann Arbor, Ann Arbor, Michigan, U.S.A.}


\begin{document}

\date{   }

\pagerange{\pageref{firstpage}--\pageref{lastpage}} 
  
\label{firstpage}

\begin{abstract}
Intensive longitudinal data (ILD) collected in mobile health (mHealth) studies contain rich information on multiple outcomes measured frequently over time that have the potential to capture short-term and long-term dynamics. Motivated by an mHealth study of smoking cessation in which participants self-report the intensity of many emotions multiple times per day, we describe a dynamic factor model that summarizes the ILD as a low-dimensional, interpretable latent process. This model consists of two submodels: (i) a measurement submodel---a factor model---that summarizes the multivariate longitudinal outcome as lower-dimensional latent variables and (ii) a structural submodel---an Ornstein-Uhlenbeck (OU) stochastic process---that captures the temporal dynamics of the multivariate latent process in continuous time. We derive a closed-form likelihood for the marginal distribution of the outcome and the computationally-simpler sparse precision matrix for the OU process. We propose a block coordinate descent algorithm for estimation. Finally, we apply our method to the mHealth data to summarize the dynamics of 18 different emotions as two latent processes. These latent processes are interpreted by behavioral scientists as the psychological constructs of positive and negative affect and are key in understanding vulnerability to lapsing back to tobacco use among smokers attempting to quit.
\end{abstract}

\begin{keywords}
dynamic factor model, intensive longitudinal data, mobile health, Ornstein-Uhlenbeck stochastic process
\end{keywords}

\maketitle

\section{Introduction}
\label{s:intro}

Intensive longitudinal data (ILD) can capture rapid changes in outcomes over time. In mobile health (mHealth) studies, information about multiple longitudinal outcomes is often collected with the aim of understanding the temporal dynamics of unobservable constructs related to mental or physical health. Our work is motivated by an mHealth study of smoking cessation in which the intensity of emotions over time was collected from current smokers attempting to quit. Participants self-reported the intensity of 18 different emotions up to four times per day over 10 days, resulting in a substantial quantity of rich data. For smoking cessation researchers, understanding the temporal dynamics of the latent psychological states that underlie these emotions is of scientific interest.

The volume and complexity of ILD, however, make them challenging to analyze since longitudinal outcomes are often measured irregularly across many individuals; thus statistical methods must be able to handle the high volume of irregularly spaced data. At the same time, the frequent measurements in ILD create many opportunities to discover new information, particularly if the latent constructs of interest vary rapidly. We present a dynamic factor model that is motivated by the need to model multiple longitudinal outcomes measured frequently over time in a flexible yet interpretable manner. Our proposed model consists of two submodels: (i) a measurement submodel---a factor model---that summarizes the multiple observed longitudinal outcomes as lower-dimensional latent factors and (ii) a structural submodel---an Ornstein-Uhlenbeck (OU) stochastic process---that captures the evolution of the multiple correlated latent factors over time. Together, these components of our dynamic factor model are flexible enough to capture the variability in the longitudinal outcomes while avoiding use of a non-parametric or other many-parameter model that inhibits interpretability. In addition to improving interpretability, the low-dimensional nature of the structural submodel also greatly reduces computational complexity, as opposed to fitting a high-dimensional stochastic process directly to the observed outcomes.

One standard approach to modeling changes in multiple correlated longitudinal variables over time is to use an autoregressive (AR) model. These models, which are called vector autoregressive (VAR) models when data are multivariate, have been widely used to model observed outcomes as well as latent variables. For example, \citet*{dunson_2003}, \citet*{cui_2014}, and \citet{tran_2019} have proposed related methods in which observed longitudinal outcomes are summarized as time-varying lower-dimensional latent variables. The correlation of these latent variables is then modeled with AR or VAR processes. VAR models, however, are specified for balanced data. This situation is often not realistic in the case of ILD, which generally consists of irregularly-measured outcomes, and can lead to biased estimates in cases where the assumption is made but does not hold.

Mixed models have been proposed as alternatives to discrete-time processes for modeling the evolution of latent variables over time and have been previously used in combination with factor models. Unlike the AR and VAR processes, mixed models do not require balanced data. Existing work has focused both on the development of mixed models for modeling the evolution of a single latent factor over time (e.g., \citealp{roy_2000}; \citealp{proust_2006}; \citealp*{proust_lima_2013}) or multiple latent factors (e.g., \citealp{liu_2019}; \citealp*{wang_2013}). Overall, these mixed model-based approaches are useful tools for capturing smooth trends in latent factors. In our application, however, we aim to develop a method that can capture the correlation between and rapid variation in multiple latent emotional constructs over time.

The OU process, which can be thought of as a continuous-time analog of the AR or VAR process, is a stochastic process well-suited for capturing rapid variations over time. Existing work has frequently focused on using the OU process or integrated OU process to model longitudinal outcomes that have been directly observed (or observed with measurement error); e.g., \citet*{taylor_1994}; \citet*{sy_1997}; \citet*{oravecz_2009}; \citet*{oravecz_2016}.

In more recent work, the OU process has also been used in the context of latent variable models. \citet{tran_2021a} propose a latent linear mixed model that summarizes multiple observed longitudinal outcomes as a smaller number of latent factors while accounting for the serial correlation between repeated measurements over time via an OU process. This work differs from ours, however, in that the fixed effects that capture the association between the observed covariates and the latent factors are of primary interest; the OU process is incorporated into the structural mixed model as a tool for accounting for serial correlation between repeated observations.

Most closely related to our proposed approach is work by \citet{tran_2021b}. Like us, they propose a longitudinal latent variable model that consists of two parts: a measurement submodel to summarize observed outcomes as lower dimensional latent factors and an OU process as the structural submodel for the latent factors. While we differ in the exact specification of the measurement submodel, our chosen models are related. Key distinctions between this existing work and the approach presented in this manuscript are in the model parameterization and computational approach. Tran et al. (2021b) take a Bayesian approach, which requires sampling values of the latent process at each measurement occasion. In the ILD setting, we need approaches that can scale to large numbers of repeated measurements. Here, we choose to work in the frequentist framework.  As a result of taking a maximum likelihood-based approach, we can directly maximize the marginal log-likelihood of the observed longitudinal outcome.  Furthermore, this framework enables us to employ various algebraic and computational strategies to make estimation faster, resulting in a method more suitable for ILD.

In this work, we fill a gap in the existing literature by proposing an Ornstein-Uhlenbeck factor (OUF) model that captures the temporal dynamics between rapidly-varying correlated latent factors observed via multiple longitudinal outcomes and an estimation algorithm with the computational efficiency to handle ILD. Our novel contributions include (i) a closed-form likelihood for the marginal distribution of the observed outcome, (ii) the derivation of the computationally-simpler sparse precision matrix for the multivariate OU process, (iii) identifiability constraints imposed via scaling constants, and (iv) a block coordinate descent algorithm for estimation and inference in a maximum likelihood framework.

The remainder of this paper is organized as such:  In Section \ref{s:motivating_data}, we describe the motivating mHealth data; in Section \ref{s:method}, we present our novel method and contributions; in Section \ref{s:sim_study}, we demonstrate the performance of our method via simulation; in Section \ref{s:data_app}, we illustrate our model via a scientifically--meaningful application to mHealth data; and in Section \ref{s:discussion}, we provide a discussion.

\section{Motivating data}\label{s:motivating_data}

Data motivating this work come from an mHealth study of smoking cessation \citep{potter_2023}. In this observational study, current smokers (N = 218) who were attempting to quit were followed for 10 days. During these 10 days, ecological momentary assessments (EMAs), which enable repeated sampling of individuals' current states and contexts in real time, were used to track participants' emotions as they were experienced in a high-frequency manner. Specifically, participants were prompted to respond to a series of questions sent to their smartphones multiple per day at random occasions; the original study design intended for individuals to receive up to four EMAs per day. The EMAs contained a set of questions that assessed the current intensity of multiple emotions measured on a 5-point Likert scale. We focus on a set of 18 emotions consisting of both positive and negative emotions that attempts to capture the distinct-but-correlated underlying emotional states of positive and negative affect (i.e., summary measures of overall positive and negative feeling). The resulting data contain frequent measurements of a substantial number of longitudinal outcomes, where the number of measurement occasions per person ranges from 2 to 47 (mean = 17). Note that these data are only the subset of the full study data that were available at the time of drafting this manuscript.  Additional details on the study procedures can be found in \cite{potter_2023}. The high rate of measurement enables us to capture rapid changes in emotions over time. To illustrate the dynamics of these responses, Figure \ref{fig:real_dat} shows the responses to emotion-related EMA questions over time for one participant in the study. Understanding the dynamics of smokers' latent emotional states that underlie the measured responses is of scientific interest among smoking cessation researchers and behavioral scientists more broadly.


% Figure environment removed
        

\section{Methods}\label{s:method}

In this section, we present the OUF model that jointly models multiple observed longitudinal outcomes and the lower dimensional latent factors assumed to generate the observed longitudinal outcomes. Our proposed model consists of two submodels: a measurement submodel and a structural submodel.

\subsection{Measurement submodel}

Let $\bm{Y}_i(t) = [Y_{i1}(t), Y_{i2}(t), ..., Y_{iK}(t)]^{\top}$ be a $K \times 1$ vector of measured longitudinal outcomes for individual $i, i = 1, ..., N$, at time $t$. Assume that individual $i$ has longitudinal outcomes measured at $n_i$ occasions. Using the measurement submodel, we model the observed longitudinal outcome $\bm{Y}_i(t)$ as
\begin{equation}
\begin{aligned}
\label{eq:meas_submodel}
    \bm{Y}_{i}(t) = \bm{\Lambda} \bm{\eta}_i(t) + \bm{u}_i + \bm{\epsilon}_i(t)
\end{aligned}
\end{equation}

\noindent where $\bm{\eta}_i(t)$ is a vector of $p$ time-varying latent factors (where $p < K$); $\bm{\Lambda}$ is a $K \times p$-dimensional time-invariant loadings matrix with elements $\lambda_{k,j}$ that captures the degree of association between the latent factors and observed longitudinal outcomes; $\bm{u}_i \sim N(0, \bm{\Sigma}_u)$ is a vector of length $K$ of random intercepts; and $\bm{\epsilon}_i(t) \sim N(0, \bm{\Sigma}_{\epsilon})$ is a vector representing measurement error, where $\bm{\Sigma}_{\epsilon}$ is assumed to be a diagonal matrix.

This model builds upon a standard factor model but also includes (i) a random intercept and (ii) a multivariate model for the evolution of the correlated latent processes $\bm{\eta}_i(t)$ (described in Section \ref{ss:structural_submod}).  We assume that $\bm{\Sigma}_u$ is diagonal, as we include this term to account for the longitudinal correlation in the repeated measurements but then model the correlated change in outcomes through the structural submodel.  Allowing a non-diagonal $\bm{\Sigma}_u$ is possible, but we opt not to do so to avoid the substantial increase computational cost associated with estimation of these extra parameters.  We also assume that $\bm{\Lambda}$ contains many structural zeros such that each row of the loadings matrix contains only one non-zero element; this structure means that each observed outcome is a measurement of only a single underlying latent factor. The decision to incorporate structural zeros in the loadings matrix is supported by behavioral science concepts (i.e., Positive and Negative Affect Schedule; PANAS \citep*{watson_1988}), which classify a given emotion as a measurement of a specific category of emotional state.  Learning the location of the structural zeros, rather than pre-specifying them, is a possible direction for future work.

\subsection{Structural submodel}\label{ss:structural_submod}

The structural submodel captures the evolution of the latent factors, $\bm{\eta}_i(t)$, over time. We use a multivariate OU process, which can be understood as a continuous-time analog of a VAR process and has the ability to capture rapid temporal variation. Here, we assume a bivariate OU process ($p = 2$) for illustrative purposes. The stochastic differential equation definition of the bivariate OU process is
\begin{equation}
\begin{aligned}
    d \begin{bmatrix} \eta_{i1}(t) \\ \eta_{i2}(t) \end{bmatrix} =  - \underbrace{\begin{bmatrix} \theta_{11} & \theta_{12} \\ \theta_{21} & \theta_{22} \end{bmatrix}}_{\coloneqq\bm{\theta}} \begin{bmatrix} \eta_{i1}(t) \\ \eta_{i2}(t) \end{bmatrix} dt + \underbrace{\begin{bmatrix} \sigma_{11} & 0 \\ 0 & \sigma_{22} \end{bmatrix}}_{\coloneqq\bm{\sigma}} d\begin{bmatrix} W_{i1}(t) \\ W_{i2}(t) \end{bmatrix}
\end{aligned}
\end{equation}

\noindent where the diagonal elements of matrix $\bm{\theta}$ capture the mean-reverting tendency of the latent factors (where the mean is assumed to be $0$) and the off-diagonal elements of $\bm{\theta}$ capture correlation between the latent factors.  The diagonal elements of $\bm{\theta}$ are required to be positive.  The matrix $\bm{\sigma}$, with elements $\sigma_{11}$ and $\sigma_{22} > 0$, describes the volatility of the process, where $W_{i1}(t)$ and $W_{i2}(t)$ are both standard Brownian motion. In general, the standard definition of the OU process allows $\bm{\sigma}$ to take non-zero values in the off-diagonal. By restricting $\bm{\sigma}$ to be a simpler diagonal matrix here, we consider the Brownian motion terms as separate noise processes for each latent variable and thus capture all correlation between the latent processes through the $\bm{\theta}$ matrix. We also require that all eigenvalues of the $\bm{\theta}$ matrix have a positive real part; this constraint ensures a mean-reverting process (see \cite{vatiwutipong_2019}).  

The multivariate stochastic process provides advantages over a simpler univariate stochastic process as it captures correlation between multiple variables over time; however, the multivariate nature does increase the complexity of the model and thus the computational burden. We address this complexity in the next section.


\subsubsection{Marginal covariance and precision matrices for the OU process}\label{ss:marginal_ou}

Rather than taking a Bayesian strategy or relying on the complete-data likelihood and taking an expectation-maximization (EM) approach to estimation, we directly maximize the likelihood of the observed data.  Direct maximization of the marginal likelihood allows us to avoid repeatedly calculating values of the latent factors at each measurement occasion (via posterior sampling in a Bayesian framework or via complex integrals in the E-step of the EM algorithm). Thus, our approach is more scalable to the ILD setting.

%complex integration that would be required in the E-step of the EM algorithm and does not require computing the expectation of the latent factors at each measurement occasion, making our approach more scalable to the ILD setting.

In order to carry out our estimation algorithm (described in Section \ref{ss:est_approach}), we require the marginal covariance matrix of the OU process. \citet{vatiwutipong_2019} present a form of the conditional variance and cross-covariance function for the OU process but provide these functions in integral forms that are not amenable to likelihood-based inference. To avoid approximations resulting from numerical integration, (i) we derive an analytic form of the conditional covariance function and (ii) we account for the additional uncertainty of an unknown initial state by deriving the analytic form of the marginal covariance function. For a stationary OU process with known initial state at time $t_0 = 0$, $\bm{\eta}(t_0)$, the conditional mean at time $t$ is $\mathbb{E}\{\bm{\eta}(t) | \bm{\eta}(t_0)\} = e^{-\bm{\theta} t} \bm{\eta}(t_0)$.  Assuming a marginal mean of 0, the conditional and marginal covariance functions follow:

\begin{lemma}\label{def:conditional_cross_cov}
The analytic form of the OU conditional covariance at times $s$ and $t$, where $s \leq t$, is
\begin{equation*} 
\begin{aligned}
\label{eq:conditional_cross_cov}
    Cov\{\eta(s), \eta(t)| \eta(t_0 = 0)\} &= vec^{-1}\Big\{(\theta \oplus \theta)^{-1} \big[ e^{s (\theta \oplus \theta)} - I \big] e^{-[t\theta \oplus s\theta]} vec\{ \sigma \sigma^{\top} \} \Big\}
\end{aligned}
\end{equation*}
\end{lemma}

\begin{lemma}\label{def:marginal_cross_cov}
The analytic form of the OU marginal covariance at times $s$ and $t$, $s \leq t$, is
\begin{equation*}
\begin{aligned}
\label{eq:marginal_cross_cov}
    Cov\{ \eta(s), \eta(t) \} = vec^{-1} \Big\{ (\theta \oplus \theta)^{-1} \big[ e^{(\theta \oplus \theta)s - (\theta t \oplus \theta s)} \big] vec\{\sigma \sigma^{\top}\} \Big\}
\end{aligned}
\end{equation*}
\end{lemma}

\noindent Here, $\oplus$ denotes the Kronecker sum, defined for square matrices $\bm{A}$ and $\bm{B}$ of sizes $a$ and $b$, respectively, as $\bm{A} \oplus \bm{B} = \bm{A} \otimes \bm{I}_b + \bm{I}_a \otimes \bm{B}$; and the $vec\{ \bm{A} \}$ operation consists of stacking the columns of matrix $\bm{A}$ into a column vector. For details on the derivations of these results, see Section A.1 and A.2 of the supplementary material.

In addition to the marginal covariance function of the OU process, we derive the precision matrix. Due to the Markov property of the OU process, the precision matrix is block tri-diagonal and thus much simpler to calculate than the dense covariance matrix.

\begin{lemma}\label{def:precision_matrix}

Let $\bm{\Omega}$ be the precision matrix of the OU process observed at $n$ occasions and define the stationary variance as $\bm{V} := vec^{-1} \big\{ (\bm{\theta} \oplus \bm{\theta})^{-1} vec\{\bm{\sigma} \bm{\sigma}^{\top}\} \big\}$. Then $\bm{\Omega}$ has the structure
\begin{equation}
\begin{aligned}
\bm{\Omega} = \begin{bmatrix}
            \bm{\Omega}_{11} & \bm{\Omega}_{12} & 0 & \cdots & 0 \\
            \bm{\Omega}_{12}^{\top} & \bm{\Omega}_{22} & \bm{\Omega}_{23} & \cdots & 0 \\
            0 & \bm{\Omega}_{23}^{\top} & \bm{\Omega}_{33} & \ddots & \vdots \\
            \vdots & \vdots & \ddots & \ddots & \bm{\Omega}_{n-1,n} \\
            0 & 0 & \cdots & \bm{\Omega}_{n-1,n}^{\top} & \bm{\Omega}_{nn} \\
         \end{bmatrix}
\end{aligned}
\end{equation}

\noindent and each block indexed by $j$ for $1 < j < n$ in the tri-diagonal matrix is
\begin{equation}
\begin{aligned}
    & \bm{\Omega}_{11} = \big[ \bm{V} - \bm{V}  e^{-\bm{\theta}^{\top} (t_2 - t_1)} \bm{V}^{-1} e^{-\bm{\theta} (t_2 - t_1)} \bm{V} \big]^{-1} \\
    & \bm{\Omega}_{j,j+1} = -\big[\bm{V} - \bm{V} e^{-\bm{\theta}^{\top} (t_{j+1} - t_j)} \bm{V}^{-1} e^{-\bm{\theta} (t_{j+1} - t_j)}  \bm{V} \big]^{-1}  \bm{V} e^{-\bm{\theta}^{\top} (t_{j+1} - t_j)} \bm{V}^{-1} \\
    & \bm{\Omega}_{jj} = \bm{V}^{-1} + \bm{V}^{-1} e^{-\bm{\theta} (t_j - t_{j-1})} \bm{V} \big[\bm{V} - \bm{V} e^{-\bm{\theta}^{\top} (t_j - t_{j-1})} \bm{V}^{-1} e^{-\bm{\theta} (t_j - t_{j-1})} \bm{V} \big]^{-1} \bm{V} e^{-\bm{\theta}^{\top} (t_j - t_{j-1})} \bm{V}^{-1} \\ & \hspace{0.9cm} + \big[\bm{V} - \bm{V} e^{-\bm{\theta}^{\top} (t_{j+1} - t_j)} \bm{V}^{-1} e^{-\bm{\theta} (t_{j+1} - t_j)} \bm{V} \big]^{-1} \bm{V} e^{-\bm{\theta}^{\top} (t_{j+1} - t_j)} \bm{V}^{-1} e^{-\bm{\theta} (t_{j+1} - t_j)} \\
    & \bm{\Omega}_{nn} = \bm{V}^{-1} + \bm{V}^{-1} e^{-\bm{\theta} (t_n - t_{n-1})} \bm{V} \big[\bm{V} - \bm{V} e^{-\bm{\theta}^{\top} (t_n - t_{n-1})} \bm{V}^{-1} e^{-\bm{\theta} (t_n - t_{n-1})} \bm{V} \big]^{-1} \bm{V} e^{-\bm{\theta}^{\top} (t_n - t_{n-1})} \bm{V}^{-1}
\end{aligned}
\end{equation}
\end{lemma}

\noindent The derivation for each block is given in Section A.3 of the supplementary material. Later, during estimation, we take advantage of the sparse precision matrix to simplify computation. This sparsity becomes particularly advantageous as the number of individuals and observations per individual in a dataset increase. 

\subsection{Joint distribution and likelihood}

Together, the measurement and structural submodels imply that the observed longitudinal outcomes are normally distributed with mean 0 and covariance $\bm{\Sigma}_i^* := Var(\bm{Y}_i) = (\bm{I}_{n_i} \otimes \bm{\Lambda}) Var(\bm{\eta}_i) (\bm{I}_{n_i} \otimes \bm{\Lambda})^{\top} + \bm{J}_{n_i} \otimes \bm{\Sigma}_u + \bm{I}_{n_i} \otimes \bm{\Sigma}_{\epsilon}$, where $\bm{I}_{n_i}$ is an $n_i \times n_i$ identity matrix and $\bm{J}_{n_i}$ is an $n_i \times n_i$ matrix of ones.  We estimate the OUF model by minimizing the following function, which equal to twice the negative log-likelihood up to a constant: $-2logL(\bm{Y}) = \sum_{i = 1}^{N} log|\bm{\Sigma}_i^{*}| + \sum_{i = 1}^{N} \bm{Y}_i^{\top} \bm{\Sigma}_i^{*-1} \bm{Y}_i$.

\subsection{Identification issues}\label{ss:identification}

Before fitting our model, we must make additional assumptions to address identifiability issues common to factor models. Because both $\bm{\Lambda}$ and $\bm{\eta}_i(t)$ are unknown, multiplying $\bm{\Lambda}$ by some matrix, say $\bm{A}$, and multiplying $\bm{\eta}_i(t)$ by $\bm{A}^{-1}$ will result in the same model. To make a factor model identifiable, constraints must be placed on either the loadings matrix or the latent factors. \citet{aguilar_2000} and \citet{carvalho_2008}, for example, make the standard assumption of requiring the loadings matrix to be triangular while \citet{tran_2019}, for example, fix the variance of the latent factors at 1. We also fix the scale of the latent factors but propose a novel approach for doing so. Letting $\bm{\eta}_i$ be the $(p \times n_i)$-length vector of latent variables values stacked over measurement occasions, we constrain $Var(\bm{\eta}_i)$ to have diagonal elements equal to 1. This constraint means that the OU process must have a stationary variance equal to 1. By fixing the scale of the latent factors, we can allow the elements of the loadings matrix $\bm{\Lambda}$ to vary almost freely during estimation. For a generic $\bm{\Lambda}$ (without structural zeros), the only constraint on the loadings matrix is that the sign of the first element must be positive.  Together these constraints make our model identifiable; the constraint on the OU process identifies the scale and the constraint on the first element of the loadings matrix identifies the direction.  Because we later make the simplifying assumption that $\bm{\Lambda}$ contains structural zeros with a single non-zero loading per row, flipping the signs on both the loadings and the latent factors results in the same model; we choose to keep the signs that correspond to the most relevant interpretation of the model given the application.  Another constraint could be added to require that one loading per column of $\bm{\Lambda}$ is positive; this would avoid sign flipping.  We revisit our approach to selecting the correct sign later in the context of rescaling the OU process.

To impose this identifiability constraint, we use a set of $p$ constants to re-scale the OU process parameters. We summarize this identifiability constraint for the bivariate OU process as:
\begin{lemma}\label{def:rescale_theta_sigma}
Using a pair of positive scalar constants $c_1$ and $c_2$, we can re-scale an arbitrary OU process parameterized by $\bm{\theta}$ and $\bm{\sigma}$ to have stationary variance of 1, where this re-scaled OU process is parameterized by $\bm{\theta}^*$ and $\bm{\sigma}^*$ according to
\begin{equation}\label{eq:rescale_theta_sigma}
    \begin{bmatrix} \theta^*_{11} & \theta^*_{12} \\ \theta^*_{21} & \theta^*_{22} \end{bmatrix} = \begin{bmatrix} \theta_{11} & \frac{c_1}{c_2} \theta_{12} \\ \frac{c_2}{c_1} \theta_{21} & \theta_{22} \end{bmatrix} \text{  and  }
    \begin{bmatrix} \sigma^*_{11} & 0 \\ 0 & \sigma^*_{22} \end{bmatrix} = \begin{bmatrix} c_1 \sigma_{11} & 0 \\ 0 & c_2 \sigma_{22} \end{bmatrix}
\end{equation}
\end{lemma}


\noindent In Section A.4 of the supplementary material, we show why this re-scaling approach works for any mean-reverting OU process. This constraint can also be extended to OU processes of higher dimensions.

Although this identifiability assumption allows us to identify the magnitude of the loadings in the factor model, it does so only up to a sign change. Consider again the case of a bivariate OU process.  The likelihood for our model is equivalent for pairs of scaling constants $(c_1 = 1, c_2 = 1)$ and $(c_1 = 1, c_2 = -1)$. In practice, the model would be the same under both pairs of scaling constants (and so we restrict $c_1$ and $c_2$ to be positive during estimation) but interpretation of model parameters would differ. After estimation, the signs on estimated model parameters can easily be flipped to match the most relevant interpretation of the data by multiplying estimates of $\bm{\Lambda}$ and $\bm{\theta}$ by a $p \times p$ matrix with the constants along the diagonal.

\subsection{Estimation algorithm}\label{ss:est_approach}

To fit this model, we take an iterative approach to estimation in which we directly maximize the marginal likelihood of our observed longitudinal outcome using a block coordinate descent algorithm and rely on simpler existing models to inform the initial parameter estimates. In the block coordinate descent algorithm, we split parameters into two different blocks: one block for parameters in the measurement submodel ($\bm{\Lambda}$, $\bm{\Sigma}_{u}$, $\bm{\Sigma}_{\epsilon}$) and the other for parameters in the structural submodel ($\bm{\theta}$, $\bm{\sigma}$). Note that each element of these blocks is actually a matrix of parameters.  Within each block-wise iteration, we minimize the log-likelihood with respect to one block of parameters, given the current estimates of the other block of parameters, using Newton algorithms as implemented in \texttt{R}'s \texttt{stats} package \citep{R_stats}. By updating parameters in blocks, we can leverage the availability of analytic gradients for parameters in the measurement submodel. The Kronecker structure of the covariance matrix for each individual’s longitudinal outcomes $\bm{Y}_i$ allows us to derive these analytic gradients. We present the general structure of the gradient here:
\begin{lemma}\label{def:fa_gradients}
The gradient of the log-likelihood for a single individual with respect to one of the measurement submodel parameters, $\Theta_j$, has the general form 
\begin{equation*}
\begin{aligned}
    \frac{\partial logL(\bm{Y}_i)}{\partial\Theta_j} = -\frac{1}{2} \Bigg[ tr \Bigg\{ \Big( I - \bm{\Sigma}_i^{*-1} \bm{Y}_i \bm{Y}_i^{\top} \Big) \Sigma_i^{*-1} \frac{\partial \bm{\Sigma}_i^{*}}{\partial \Theta_j}  \Bigg\} \Bigg]
\end{aligned}
\end{equation*}
where the exact form of $\frac{\partial \bm{\Sigma}_i^{*}}{\partial \Theta_j}$ depends on the specific parameter; either $\lambda_k$, $\sigma_{u_k}$, or $\sigma_{\epsilon_k}$.
\end{lemma}

\noindent The complete set of analytic gradients is given in Section A.5 of the supplementary material.  The computational advantage of using the analytic gradient, as opposed to a numerical approach to differentiation, is particularly notable as the number of longitudinal outcomes---and thus parameters in the measurement submodel---increases.

Prior to maximizing the marginal likelihood, we use a cross-sectional factor model to initialize $\bm{\Lambda}$, $\bm{\theta}$, and $\bm{\sigma}$, and use linear mixed models to initialize $\bm{\Sigma}_u$ and $\bm{\Sigma}_{\epsilon}$. Then, we iteratively update parameter estimates using the following block coordinate descent algorithm:
\begin{enumerate}
    \item \textit{Initialize estimates of $\bm{\Lambda}^{(0)}, \bm{\Sigma}_u^{(0)}, \bm{\Sigma}_{\epsilon}^{(0)}, \bm{\theta}^{(0)}, \bm{\sigma}^{(0)}$. Measurement submodel parameters are always initialized empirically; for structural submodel parameters, two sets of initial estimates are considered---an empirical set of values estimated from cross-sectional factor scores and a default set of values.  The set of values that corresponds to the higher log-likelihood given the current data is used.}
    \item \textit{Set iteration index $r = 1$ and convergence indicator $\delta = 0$. While $\delta = 0$,}
    \begin{enumerate}
        \item \textit{Update block 1 (measurement submodel): $$\bm{\Lambda}^{(r)}, \bm{\Sigma}_u^{(r)}, \bm{\Sigma}_{\epsilon}^{(r)} = \underset{\bm{\Lambda}, \bm{\Sigma}_u, \bm{\Sigma}_{\epsilon}}{argmax}\big\{ logL(\bm{\Lambda}, \bm{\Sigma}_u, \bm{\Sigma}_{\epsilon} | Y; \bm{\theta}^{(r-1)}, \bm{\sigma}^{(r-1))}) \big\}.$$  Maximization is done via a Newton-type algorithm using analytic gradients (Lemma \ref{def:fa_gradients}).}
        \item \textit{Update block 2 (structural submodel): $$\bm{\theta}^{(r)}, \bm{\sigma}^{(r)} = \underset{\bm{\theta}, \bm{\sigma}}{argmax}\big\{ logL(\bm{\theta}, \bm{\sigma} | Y; \bm{\Lambda}^{(r)}, \bm{\Sigma}_u^{(r)}, \bm{\Sigma}_{\epsilon}^{(r)}) \big\}.$$ Maximization is done via a quasi-Newton algorithm using numerical gradients.}
        \item \textit{Using Lemma \ref{def:rescale_theta_sigma}, re-scale OU parameters to satisfy the identifiability constraint.}
        \item \textit{Check for block-wise convergence: Let $\bm{\Theta}$ be a vector containing all elements of $\bm{\Lambda}$, $\bm{\Sigma}_u$, $\bm{\Sigma}_{\epsilon}$, $\bm{\theta}$, and $\bm{\sigma}$. Then, calculate $$\delta = \max \Big\{I\big\{|\bm{\Theta}^{(r)} - \bm{\Theta}^{(r-1)}|/\bm{\Theta}^{(r)} < 10^{-6}\big\}, \ I\big\{logL(\bm{\Theta}^{(r)} | \bm{Y}) - logL(\bm{\Theta}^{(r-1)} | \bm{Y}) < 10^{-6}\big\}\Big\}$$ where all operations on $\bm{\Theta}$ are element-wise.}
        \item \textit{Update} $r$: $r = r+1$
    \end{enumerate}
    \item \textit{Estimate Fisher Information-based standard errors from a numerical approximation to the Hessian of the log-likelihood, $\frac{\partial^2}{\partial\Theta \partial\Theta^{\top}}logL(\bm{\Lambda}^{(r)}, \bm{\Sigma}_u^{(r)}, \bm{\Sigma}_{\epsilon}^{(r)}, \bm{\theta}^{(r)} | Y).$}
\end{enumerate}

Note that when estimating standard errors, the parameterization of the likelihood differs slightly: the likelihood now depends on only one of the parameter matrices in the structural submodel, $\bm{\theta}$, and not the other, $\bm{\sigma}$. This change in parameterization is a result of the identifiability constraint that is placed on the stationary variance of the OU process. Since we are no longer conditioning on fixed measurement submodel parameters in this step, we restrict $\bm{\sigma}$ to be a function of $\bm{\theta}$, where this function is derived from the identifiability constraint; thus, the likelihood is not over-parameterized. Standard error estimates for $\bm{\sigma}$ can be calculated via parametric bootstrap. By sampling values of $\bm{\theta}$ from a Normal distribution defined by its point estimate and estimated covariance matrix, bootstrapped values of $\bm{\sigma}$ are calculated as a function of $\bm{\theta}$ and a confidence interval can be estimated based on the empirical distribution. More details on the parameterization of the log-likelihood for standard error estimation are in Section A.6 of the supplementary material. 

To increase the computational efficiency of this estimation algorithm, we (i) leverage the Markov property of the OU process and use the computationally-simpler sparse precision matrix derived in Lemma \ref{def:precision_matrix} rather than the dense covariance matrix, (ii) take advantage of tractable analytic gradients for the measurement submodel given in Lemma \ref{def:fa_gradients}, avoiding the need to calculate computationally expensive numerical gradients, and (iii) implement portions of our algorithm in C++.

\section{Simulation study}\label{s:sim_study}

\subsection{Data generation}

We conduct a simulation study to assess the bias and variance of estimates produced by our model. We assume that there are $K = 4$ longitudinal outcomes recorded over time for $N = 200$ individuals. For individual $i$, these longitudinal outcomes are measured at $n_i$ different occasions where $n_i \sim Uniform(10, 20)$. The gap times between measurement occasions are drawn from a $Uniform(0.1, 2)$ distribution. We consider simulated data in three different settings in which the true bivariate OU process has varying degrees of autocorrelation (see Section A.8 of the supplementary material for details). Using each true OU process, we generate the observed longitudinal outcomes by drawing from $\bm{Y}_i \sim N(0, \bm{\Sigma}_i^*)$ where $\bm{\Sigma}_i^*$ is defined using
\begin{equation}
\begin{aligned}
    \bm{\Lambda} = \begin{bmatrix}
    1.2 & 0 \\
    1.8 & 0 \\
    0 & -0.4 \\
    0 & 2
    \end{bmatrix} \text{,  }
    \bm{\Sigma}_u = \begin{bmatrix}
    1.1 & 0 & 0 & 0 \\
    0 & 1.3 & 0 & 0 \\
    0 & 0 & 1.4 & 0 \\
    0 & 0 & 0 & 0.9
    \end{bmatrix} \text{, and  }
    \bm{\Sigma}_{\epsilon} = \begin{bmatrix}
    0.6 & 0 & 0 & 0 \\
    0 & 0.5 & 0 & 0 \\
    0 & 0 & 0.4 & 0 \\
    0 & 0 & 0 & 0.7
    \end{bmatrix}.
\end{aligned}
\end{equation}

\noindent When fitting this model, we assume that the structural zeros within the loadings matrix and random intercept covariance matrix are known.

Importantly, some of the parameter values used to generate the data are different from the parameters that will be estimated by the model; this difference is a side-effect of the identifiability assumption. While unbiased estimates of $\bm{\Sigma}_u$ and $\bm{\Sigma}_{\epsilon}$ will match the values used in data generation, the values of $\bm{\Lambda}$ and the OU process parameters $\bm{\theta}$ and $\bm{\sigma}$ will differ. As a result of the re-scaling approach for identification described in Section \ref{ss:identification}, the estimated OU process has a stationary variance of 1. The additional variation present in the OU process during data generation must be absorbed by the loadings matrix $\bm{\Lambda}$. Specifically, the data-generating loadings matrix will be re-scaled according to $\bm{\Lambda} \bm{D}$ where $\bm{D}:= \sqrt{diag\{ V(\bm{\theta}, \bm{\sigma})\}}$ and $\bm{V}$ is the stationary variance of the OU process as defined in Lemma \ref{def:precision_matrix}.  $\bm{\Lambda} \bm{D}$ will be estimated by our algorithm.  The data-generating OU parameters $\bm{\theta}$ and $\bm{\sigma}$ will be re-scaled according to scalar constants chosen such that the stationary variance of the re-scaled OU process is equal to 1 via Lemma \ref{def:rescale_theta_sigma}.  True parameter values indicated in the simulation results have all been re-scaled to match the values targeted by our estimation algorithm. In setting 2, the true OU process used to generate data does have a stationary variance equal to 1 and thus the target parameter values do match the data-generating parameter values.

\subsection{Simulation results}\label{ss:sim_results}

In each of the three settings, we generate 1,000 datasets and carry out the estimation algorithm described in Section \ref{ss:est_approach}. Final point estimates are shown in Figure \ref{fig:point_ests} and information-based standard errors are summarized in Figure \ref{fig:se}. In all settings, we consistently recover unbiased estimates of the true values and find that the average of the standard errors is similar to the empirical standard deviation of the point estimates, indicating that confidence intervals will have close to nominal coverage. In a rare case, numerical issues result in a negative variance estimate; this specific case is discussed in Section A.9 of the supplementary material. 

% Figure environment removed

% Figure environment removed

\subsection{Model selection}\label{ss:model_selection}

We next carry out a simulation study in which we evaluate the ability of Akaike information criterion (AIC) and Bayesian information criterion (BIC) to correctly select the true number of latent factors among the fitted models. Formulas for AIC and BIC are given in Section A.7 of the supplementary material. Assuming the same true measurement submodel parameters as before, we now generate data from five different factor models: a one-factor model, a two-factor model with low signal (i.e., high correlation between latent factors), a two-factor model with high signal (i.e., low correlation between latent factors), a three-factor model with low signal, and a three-factor model with high signal.  Data-generating parameter values are given in the Section A.8 of the supplementary material.  For 100 datasets generated from each of these true models, we fit a one-, two-, and three-factor model and compare fit indices.  We do not consider a four-factor model in this simulation study because our data only contain four longitudinal outcomes and so fitting a four-factor model would no longer fall into the dimension-reduction setting in which factor models are generally used.

We present model selection results in Table \ref{tab:mod_selection}.  In both the high and low signal settings, the model with the lowest AIC and BIC most often has the same number of factors as the true model used to generate the data. For models fit to data generated from a true model with three factors, BIC incorrectly selects a model with two factors more often than AIC. This difference make sense given the increased penalty that BIC places on model complexity. For datasets of this size ($N = 200$), estimation becomes more difficult as the number of factors increases and so our algorithm did not converge for a few simulated datasets (see Section A.9 of the supplementary material for details). 

\begin{table}
    \centering
    % Figure removed
    \caption{For datasets generated under each true model, we summarize the percent of times that the model-selection metric chose the fitted model with the indicated number of factors. When generating data from models with 2 and 3 factors, we considered two different settings: a high signal setting in which latent factors have lower correlation and a low signal setting in which latent factors have high correlation. The settings in which the fitted model has the same number of factors as the true data-generating model are emphasized with bold orange text. These results are presented for datasets on which the algorithm either converged or reached the maximum number of iterations (200) for all three models. See Section A.9 of the supplementary material for more details.}\label{tab:mod_selection} 
\end{table}


\section{Application to mHealth emotion data}\label{s:data_app}

For illustrative purposes, we apply our method to the data on momentary emotions collected in the mHealth study. Using the OUF model, we summarize the longitudinal responses to 18 emotion-related questions as two latent factors interpreted as positive and negative affect. Positive and negative affect are two distinct-but-correlated emotional states known to be key in understanding smoking habits (e.g., \citet{minami_2014}, \citet{langdon_2016}, \citet{leventhal_2013}, \cite{baker_2004}). The proposed model accounts for both the rapid temporal variation in these states and their correlation over time.  In these data, positive affect was measured by how strongly individuals agreed with feeling happy, joyful, enthusiastic, active, calm, determined, grateful, proud, and attentive; negative affect was measured by how strongly individuals agreed with feeling sad, scared, disgusted, angry, shameful, guilty, irritable, lonely, and nervous.

When applying the OUF model, the assumed structural zeros within the loadings matrix result in positive emotions loading onto one of the latent variables, $\bm{\eta}_1(t)$, and negative emotions loading onto the other, $\bm{\eta}_2(t)$. Point estimates and 95\% confidence intervals are in Figure \ref{fig:real_dat_results}. Measures of happiness, joy, and enthusiasm are most strongly correlated with positive affect and measures of sadness and irritability are most strongly correlated with negative affect. We use the estimated parameters of the OU process to understand the latent dynamics of positive and negative affect by plotting the degree of correlation for these two latent variables across varying time intervals between consecutive observations (see Figure \ref{fig:real_dat_autocor}). We see that positive and negative affect are negatively correlated as expected, and that the correlation between the latent states decays slowly.


We also fit a univariate OUF model and a trivariate OUF model and compare these models to the bivariate OUF model.  In the univariate OUF model, all emotions are assumed to be generated from a single common underlying factor; in the trivariate OUF model, we further divide the positive emotions into two latent factors that we call high arousal positive affect (measured by feeling grateful, proud, enthusiastic, active, determined, attentive) and no-to-low arousal positive affect (measured by feeling calm, happy, and joyful), while the negative emotions are still assumed to be generated from one latent factor.  Coefficient estimates from these fitted models are given in Section B.1 of the supplementary material.


Both AIC and BIC indicate that, of the three models considered, the two factor model fits best: $AIC_{\text{1 factor}} = 123,309$ vs. $AIC_{\text{2 factors}} = 121,069$ vs. $AIC_{\text{3 factors}} = 124,957$ and $BIC_{\text{1 factor}} = 123,791$ vs. $BIC_{\text{2 factor}} = 121,577$ vs. $BIC_{\text{3 factor}} = 125,509$.  We provide more details on the calculation of AIC and BIC in Section B.1 of the supplementary material.  Psychological literature supports our conclusion that two factors represent our data better than one as it suggests that positive and negative affect are not opposites, rather they capture distinct-but-correlated components of psychological state (\citealp*{reich_2003}). The lower AIC and BIC of the two factor model compared to the three factor model suggest that the emotions corresponding to high arousal positive affect and no-to-low arousal positive affect are not different enough to justify the additional complexity of the three factor model given the current data.

% Figure environment removed

% Figure environment removed


\section{Discussion}\label{s:discussion}

We developed a dynamic OUF model that combines a factor model to summarize multivariate observed longitudinal outcomes as lower dimensional latent factors and an OU process to describe the temporal evolution of the latent factors in continuous time. By using the OU process, instead of a discrete time approach such as a VAR process, our model can be applied to irregularly-measured ILD commonly produced by mHealth studies. The OU process captures rapid variations in the correlated latent factors over time, in contrast to a multivariate mixed model that is more suitable for capturing smooth trends over time. To fit our model, we use a block coordinate descent algorithm to directly maximize the log-likelihood of the observed multivariate longitudinal outcome.  We derive both the close-form likelihood of the measured outcome and the sparse precision matrix for the multivariate OU process.  Our block coordinate descent algorithm leverages analytic gradients for a subset of parameters to improve computational efficiency. Finally, we applied our method to study the dynamics of emotions among smokers attempting to quit.

Through the marginal distribution of the multivariate OU process, we parameterize our likelihood in terms of the standard OU drift ($\bm{\theta}$) and volatility ($\bm{\sigma}$) parameters. Having estimates for these parameters enables us to understand the dynamics of the latent factors, including generating new trajectories using the estimated values and examining the decay in the trajectories' correlation over time. Through examination of decay in correlation over time, our method could help inform the design of future studies that aim to collect ILD by providing insight into how frequently the longitudinal outcomes must be measured in order to capture the correlation between them.

In our simulation study in Section \ref{s:sim_study}, we generated data under true OU processes that showed reasonably slow decay in correlation over time given the intervals between measurements. We found that estimation of the OU parameters is difficult if correlation decays quickly relative to gaps between measurements. When longitudinal outcomes are measured frequently enough that correlation between consecutive measurements is captured, our method consistently returns unbiased estimates of the OU process parameters. If this method were applied to data in which longitudinal outcomes are not measured often enough to capture the correlation, estimation would be difficult. Like all statistical methods, when enough signal exists in the data, our method works well. It does, however, require studies to be designed such that longitudinal outcomes are measured with sufficient frequency that the correlation between consecutive measurements is captured.

Although we use the sparse OU precision matrix, leverage the availability of analytic gradients for the measurement submodel parameters, and implement a portion of our algorithm in C++, the computation time of our estimation algorithm increases rapidly as the number of longitudinal outcomes increases. We successfully fit our model to a dataset containing 18 longitudinal outcomes but this does require approximately 27 hours. In order to make application of our model to datasets with larger numbers of longitudinal outcomes feasible, computational efficiency must be increased.  However, our proposed marginal likelihood-based method has substantial computational benefits when compared to alternative methods.  In comparison to the Bayesian approach proposed for fitting a similar model in \cite{tran_2021b}, our approach requires less computation time.  In a simulation study with $K = 4$ longitudinal outcomes measured at 10-20 occasions on $N = 200$ individuals, we found that estimation via our block coordinate descent algorithm required approximately 5\% of the time required by the Bayesian approach proposed in \cite{tran_2021b} given the same computing resources.  More details on this comparison are given in Section C.2 of the supplementary material.

Finally, the mHealth dataset to which we applied our method also contains information on demographic characteristics and on the timing of cigarette use. Including baseline covariates in either the measurement or structural submodel would be a useful extension. In behavioral science, specific emotional states, such as negative affect or craving, are expected to be correlated with cigarette use and so future work could involve combining our OUF model with a submodel for event-time outcomes. Our model could also be modified to account for treatment or for drift in the OU process to better capture the dynamics of the latent processes after a key event such as smoking cessation or relapse.


\section*{Acknowledgements}

This work was supported by the National Institutes of Health [grant numbers F31DA057048, P30CA042014, P50DA054039, R01DA039901, R01MD010362, T32CA083654, U01CA229437]; and the Huntsman Cancer Foundation. The National Institutes of Health had no role in the study design, collection, analysis or interpretation of the data, writing the manuscript, or the decision to submit the paper for publication. The content is solely the responsibility of the authors and does not necessarily represent the official views of the National Institutes of Health or the Huntsman Cancer Foundation. This work is not peer reviewed. The authors declare no conflicts of interest. 


\section*{Supporting Information}

Supplementary material is available with this paper.  Example code and simulated data are available on Github at https://github.com/madelineabbott/OUF.

\vspace*{-8pt}


\bibliographystyle{biom} 
\bibliography{references.bib}


\label{lastpage}

\end{document}

\end{document}
