%% ****** Start of file apstemplate.tex ****** %
%%
%%
%%   This file is part of the APS files in the REVTeX 4.2 distribution.
%%   Version 4.2a of REVTeX, January, 2015
%%
%%
%%   Copyright (c) 2015 The American Physical Society.
%%
%%   See the REVTeX 4 README file for restrictions and more information.
%%
%
% This is a template for producing manuscripts for use with REVTEX 4.2
% Copy this file to another name and then work on that file.
% That way, you always have this original template file to use.
%
% Group addresses by affiliation; use superscriptaddress for long
% author lists, or if there are many overlapping affiliations.
% For Phys. Rev. appearance, change preprint to twocolumn.
% Choose pra, prb, prc, prd, pre, prl, prstab, prstper, or rmp for journal
%  Add 'draft' option to mark overfull boxes with black boxes
%  Add 'showkeys' option to make keywords appear
%\documentclass[aps,reprint,groupedaddress, amsmath,amssymb]{revtex4-2}
\documentclass[prd,reprint,amsmath,amssymb,superscriptaddress]{revtex4-2}
%\documentclass[aps,prl,preprint,superscriptaddress]{revtex4-2}
%\documentclass[aps,prl,reprint,groupedaddress]{revtex4-2}

% You should use BibTeX and apsrev.bst for references
% Choosing a journal automatically selects the correct APS
% BibTeX style file (bst file), so only uncomment the line
% below if necessary.
%\bibliographystyle{apsrev4-2}


\newcommand{\be}{\begin{equation}}
\newcommand{\ee}{\end{equation}}
\newcommand{\ba}{\begin{align}}
\newcommand{\eda}{\end{align}}
\newcommand{\nn}{\nonumber}
%underline sottolineato
%\usepackage[utf8]{inputenc}

%%\usepackage{multicol}
\usepackage{bbm}
%\usepackage{amsmath}
\usepackage{todonotes}\topmargin -1.2cm

\usepackage[export]{adjustbox}
\usepackage{graphicx}
\usepackage{epsfig}
\usepackage{braket}
%\usepackage{amssymb}
%\usepackage[font=small,labelfont=bf]{caption}
\usepackage{capt-of}
\newcommand{\om}{\omega}
\newcommand{\omi}{\omega^{-1}}
\newcommand{\ppm}{\pi^{\pm}}
\newcommand{\pmp}{\pi^{\mp}}
\newcommand{\mpm}{\\cM^{\pm}}
\newcommand{\mmp}{\\cM^{\mp}}
\newcommand{\hp}{^{+}}
\newcommand{\hm}{^{-}}
\newcommand{\hpm}{^{\pm}}
\newcommand{\hmp}{^{\mp}}
\newcommand{\tp}{\tilde{p}}
\newcommand{\uu}{_{1}}
\newcommand{\ud}{_{2}}
\newcommand{\ua}{_{a}}
\newcommand{\ub}{_{b}}
\newcommand{\us}{_{s}}
\newcommand{\2}{^{2}}
\newcommand{\cN}{\mathcal{N}}
\newcommand{\cM}{\mathcal{M}}
\newcommand{\cK}{\mathcal{K}}
\newcommand{\cJ}{\mathcal{J}}
\newcommand{\cP}{\mathcal{P}}
\newcommand{\cA}{\mathcal{A}}
\newcommand{\cB}{\mathcal{B}}
\newcommand{\ul}{_{\lambda}}
\newcommand{\hl}{^{\lambda}}
 \newcommand{\up}{_{p}}
  \newcommand{\ut}{_{t}}
 \newcommand{\upp}{_{\pi}}
  \newcommand{\np}{\not{p}}
    \newcommand{\nk}{\slash{k}}
%    \newcommand{\er}{\eqref{#1} }
  \def\er{\eqref}
  \newcommand{\bel}[1]{\be\label{#1}}
  \newcommand{\bal}[1]{\ba\label{#1}}
  \newcommand{\dv}{\text{d}}
  \newcommand{\tr}{\text{tr}}
  \newcommand{\qs}{\sqrt{s}}
  \newcommand{\wpp}{\widehat{p}^{\, \prime}}
  \newcommand{\vpp}{\vec{p}^{\, \prime}}
  \newcommand{\wip}{\widehat{p}}
  \newcommand{\wik}{\widehat{k}}
  \newcommand{\esk}{\,\rule[-5pt]{0.4pt}{12pt}\,{}}
  
  \renewcommand\slash[1]{\not \! #1}
  \newcommand{\tnp}{\slash{\tilde{p}}}
  
 \newcommand{\bp}{\slash{p}}
 \newcommand{\bk}{\slash{k}}
 \newcommand{\bl}{\not{l}}
 \newcommand{\nldt}[1]{\overset{\approx}{\cN}      \ul{\phantom{\big{|}}}^{\!\!\!\!\!\!(#1)}}
  
  
\newcommand{\p}{\partial}

%\newcommand{\twosidep}[1]{\stackrel{\;\leftrightarrow}{\,\p}_{#1}}
\newcommand{\leftsidep}[1]{\stackrel{\;\leftarrow}{\,\p}_{#1}}
\newcommand{\rightsidep}[1]{\stackrel{\;\rightarrow}{\,\p}_{#1}}

%\intertext{and}
%^{\top} for ^T
%\Big{|}_{m\ub\2 = m_{p}\2}   Beschränkt

%a\underset{\substack{p'=p,\\
%p^{\prime 2}\!=p\2 =m\upp\2}}{\big{|\qquad\quad}} 

\usepackage{hyperref}
%\usepackage{chngcntr}
%For EqNumber 1.1 , 1.2 ....
\numberwithin{equation}{section}
\renewcommand{\theequation}{\arabic{section}.\arabic{equation}}
%\numberwithin{figure}{section}
%\numberwithin{table}{section}   
\usepackage[noabbrev]{cleveref}    

\bibliographystyle{utphys}
%\bibliographystyle{apsrev4-2}

%\allowdisplaybreaks
%\underline{off-shell}

\newcommand{\PL}{\color[rgb]{0,0,1}}

\begin{document}

% Use the \preprint command to place your local institutional report
% number in the upper righthand corner of the title page in preprint mode.
% Multiple \preprint commands are allowed.
% Use the 'preprintnumbers' class option to override journal defaults
% to display numbers if necessary
%\preprint{}

%Title of paper
\title{Soft-photon theorem for pion-proton elastic scattering revisited}

% repeat the \author .. \affiliation  etc. as needed
% \email, \thanks, \homepage, \altaffiliation all apply to the current
% author. Explanatory text should go in the []'s, actual e-mail
% address or url should go in the {}'s for \email and \homepage.
% Please use the appropriate macro foreach each type of information

% \affiliation command applies to all authors since the last
% \affiliation command. The \affiliation command should follow the
% other information
% \affiliation can be followed by \email, \homepage, \thanks as well.
%\author{...}
%\email[]{Your e-mail address}
%\homepage[]{Your web page}
%\thanks{}
%\altaffiliation{}

\author{Piotr Lebiedowicz}
%\orcid{0000-0003-1963-6263}
\email{Piotr.Lebiedowicz@ifj.edu.pl}
\affiliation{Institute of Nuclear Physics Polish Academy of Sciences, 
Radzikowskiego 152, PL-31342 Krak{\'o}w, Poland}

\author{Otto Nachtmann}
\email{O.Nachtmann@thphys.uni-heidelberg.de}
\affiliation{Institut f\"ur Theoretische Physik, Universit\"at Heidelberg,
Philosophenweg 16, D-69120 Heidelberg, Germany}

\author{Antoni Szczurek}
%\orcid{0000-0001-5247-8442}
\email{Antoni.Szczurek@ifj.edu.pl}
\affiliation{Institute of Nuclear Physics Polish Academy of Sciences, 
Radzikowskiego 152, PL-31342 Krak{\'o}w, Poland}
\affiliation{College of Natural Sciences, 
Institute of Physics, University of Rzesz{\'o}w, 
Pigonia 1, PL-35310 Rzesz{\'o}w, Poland.}

%\date{\today}

\begin{abstract}
We discuss the reactions $\pi p \to \pi p$ and $\pi p \to \pi p \gamma$ from a general quantum field theory (QFT) point of view.
We consider the pion-proton elastic scattering 
both off shell and on shell. 
The on-shell amplitudes for $\pi^{\pm} p \to \pi^{\pm} p$ scattering
are described by two invariant amplitudes,
while the off-shell amplitudes contain eight invariant amplitudes.
We study the photon emission amplitudes
in the soft-photon limit where the photon energy $\omega \to 0$.
The expansion of the $\pi^{\pm} p \to \pi^{\pm} p \gamma$ amplitudes 
to the orders
$\omega^{-1}$ and $\omega^{0}$ is derived.
These terms can be expressed by the on-shell invariant amplitudes
and their partial derivatives with respect to $s$ and $t$.
The term of order $\omega^{-1}$ is well known from the literature.
Our term of order $\omega^{0}$ is new.
The formulas given for the amplitudes in the limit 
$\omega \to 0$ are valid for both real and virtual photons.
We also discuss the behaviour of the corresponding cross-sections.
\end{abstract}

% insert suggested keywords - APS authors don't need to do this
%\keywords{}

%\maketitle must follow title, authors, abstract, and keywords
\maketitle

% body of paper here - Use proper section commands
% References should be done using the \cite, \ref, and \label commands
%%\section{}
% Put \label in argument of \section for cross-referencing
%\section{\label{}}
%%\subsection{}
%%\subsubsection{}

% If in two-column mode, this environment will change to single-column
% format so that long equations can be displayed. Use
% sparingly.
%\begin{widetext}
% put long equation here
%\end{widetext}

% figures should be put into the text as floats.
% Use the graphics or graphicx packages (distributed with LaTeX2e)
% and the \includegraphics macro defined in those packages.
% See the LaTeX Graphics Companion by Michel Goosens, Sebastian Rahtz,
% and Frank Mittelbach for instance.
%
% Here is an example of the general form of a figure:
% Fill in the caption in the braces of the \caption{} command. Put the label
% that you will use with \ref{} command in the braces of the \label{} command.
% Use the figure* environment if the figure should span across the
% entire page. There is no need to do explicit centering.

% % Figure environment removed

% Surround figure environment with turnpage environment for landscape
% figure
% \begin{turnpage}
% % Figure environment removed
% \end{turnpage}

% tables should appear as floats within the text
%
% Here is an example of the general form of a table:
% Fill in the caption in the braces of the \caption{} command. Put the label
% that you will use with \ref{} command in the braces of the \label{} command.
% Insert the column specifiers (l, r, c, d, etc.) in the empty braces of the
% \begin{tabular}{} command.
% The ruledtabular enviroment adds doubled rules to table and sets a
% reasonable default table settings.
% Use the table* environment to get a full-width table in two-column
% Add \usepackage{longtable} and the longtable (or longtable*}
% environment for nicely formatted long tables. Or use the the [H]
% placement option to break a long table (with less control than 
% in longtable).
% \begin{table}%[H] add [H] placement to break table across pages
% \caption{\label{}}
% \begin{ruledtabular}
% \begin{tabular}{}
% Lines of table here ending with \\
% \end{tabular}
% \end{ruledtabular}
% \end{table}

% Surround table environment with turnpage environment for landscape
% table
% \begin{turnpage}
% \begin{table}
% \caption{\label{}}
% \begin{ruledtabular}
% \begin{tabular}{}
% \end{tabular}
% \end{ruledtabular}
% \end{table}
% \end{turnpage}

\section{Introduction}
\label{sec:Introduction}

In this paper we shall study the $\pi p \to \pi p$ and $\pi p \to \pi p \gamma$ reactions. 

Let $\omega$ be the energy of the photon in the overall  c.m. system. We are interested in the limit $\omega\to 0$, that is, the soft-photon limit. In a seminal paper \cite{Low:1958sn} F. E. Low considered the scattering of a charged on an uncharged scalar particle with photon radiation. He showed that the pole term of order $\omega^{-1}$ in the amplitude comes exclusively from the photon emission of the external particles. The extension of this result to reactions with an arbitrary number of external particles was considered in \cite{Weinberg:1965nx}.
This result was extended to spin 1/2 particles
by Burnett and Kroll in \cite{Burnett:1967km},
and generalized subsequently by Bell and Van Royen
\cite{Bell:1969yw} to particles of arbitrary spin.
Up to the present many authors have studied
the soft photon limit; see for instance 
\cite{Gribov:1966hs,Burnett:1967km,Bell:1969yw,
Lipatov:1988ii,
DelDuca:1990gz,Gervais:2017yxv,Bern:2014vva,Lysov:2014csa,
Bonocore:2021cbv}.
There is in general agreement on the pole term proportional to $\omi$ in the amplitudes for photon emission. 

In \cite{Low:1958sn} also an expression for the term of order $\om^0$ for the reaction studied there, is given.
In \cite{Lebiedowicz:2021byo} we reconsidered this problem. We studied the reactions
\begin{align}
\label{1.1}
& \pi\hm +\pi^{0} \to\pi\hm +\pi^{0}\,,
\nonumber\\
& \pi\hm +\pi^{0} \to\pi\hm +\pi^{0}+\gamma\,.
\end{align}
We worked in the framework of QCD and treated electromagnetism to lowest relevant order. As theoretical tools we used the Ward-Takahashi identity \cite{Ward:1950xp,Takahashi:1957xn} 
and the Landau conditions giving the position of singularities in amplitudes; 
see \cite{Landau:1959fi} and chapter 18 of \cite{Bjorken:1965}.
To our surprise, we found a result for the $\om^0$ term different from the one given in \cite{Low:1958sn}. We analyzed this difference and found that in \cite{Low:1958sn} energy-momentum conservation was not taken into account correctly. 
In Appendices~A and B of \cite{Lebiedowicz:2021byo} 
we also gave a critical discussion of the terms of order $\om^0$ presented in \cite{Burnett:1967km,Lipatov:1988ii,
Gervais:2017yxv,Bern:2014vva,Lysov:2014csa},
all of which have problems, according to our analysis. 

In our present paper we continue our studies of soft-photon production in hadronic reactions. 
In \cite{Lebiedowicz:2022nnn,Lebiedowicz:2023mhe,Lebiedowicz:2023rgc} 
we treated bremsstrahlung and central-exclusive production of soft photons in proton-proton collisions at high energies. 
There we used the tensor-Pomeron model \cite{Ewerz:2013kda}
as theoretical tool. 
Now we shall study the reactions
\begin{align}
\label{1.2}
&\ppm + p \to \ppm + p\,,
\\
\label{1.3}
&\ppm + p \to \ppm + p+\gamma\,,
\end{align}
using only rigorous QFT methods. We are again interested in the limit $\om\to 0$, where $\om$ is the photon energy in the c.m. system. 
We shall work in the framework of QCD, the theory of hadrons, and consider electromagnetism to lowest relevant order. Our aim is to calculate the terms of order $\omi$ and $\om^0$ in the amplitudes for the reactions \eqref{1.3}. As theoretical tools we shall only use general relations of quantum field theory (QFT), the symmetry relations of QCD,
the Ward-Takahashi identity \cite{Ward:1950xp,Takahashi:1957xn},
and the Landau conditions \cite{Landau:1959fi,Bjorken:1965}.

Our paper is organized as follows.
In Sec.~\ref{sec:2} we discuss the amplitudes for the $\ppm p\to \ppm p$ scattering, both off shell and on shell. 
In Sec.~\ref{sec:3} we study the reactions $\ppm p \to \ppm p\gamma$, first from a general point of view, and then in the limit of the photon energy $\omega\to 0$. We derive the expansion of these amplitudes to the orders $\omi$ and $\om^0$. 
Section~\ref{sec:4} deals with the cross sections for $\ppm p\to \ppm p\gamma$ for $\om\to 0$, 
and in Sec.~\ref{sec:5} we present our conclusions. 
In Appendix~\ref{app:A} we collect kinematic relations. 
Appendix~\ref{app:B} gives a detailed discussion of the pion and proton propagators and the $\gamma\pi\pi$ and $\gamma pp$ vertex functions. 
In Appendix~\ref{app:C} we present the details of our calculations for the $\ppm p\to \ppm p\gamma$ amplitudes. 

Throughout our paper we use the metric 
and $\gamma$-matrix conventions of \cite{Bjorken:1965}. 
The Levi-Civita symbol $\varepsilon_{\mu\nu\rho\sigma}$ 
is normalized as $\varepsilon_{0123}=1$.

Since our paper is rather long we give here a short guide 
to important equations. The off-shell and on-shell matrix elements for
$\ppm p \to \ppm p$ are given in (\ref{2.16}) and (\ref{2.23}), respectively.
The matrix element for $\ppm p \to \ppm p \gamma$ can be found in (\ref{3.5})
followed by the important discussion of energy-momentum conservation;
see Fig.~\ref{fig:2}.
In (\ref{3.16}) we define the matrix function $\mathcal{N}^{\lambda}$
for $\pi^{-} p \to \pi^{-} p \gamma$ which is then the main object
of our studies.
How to obtain the standard matrix elements from $\mathcal{N}^{\lambda}$
is shown in (\ref{3.17}).
The final results for $\mathcal{N}^{\lambda}$ to the orders
$\omega^{-1}$ and $\omega^{0}$ are given in (\ref{3.41})--(\ref{3.47}).
The expansions for $\omega \to 0$ of the cross sections
for $\ppm p \to \ppm p \gamma$ are given in (\ref{4.15}).

\section{The amplitudes for the on- and off-shell scattering $\ppm p \to \ppm p$}
\label{sec:2}
In this section we discuss general properties of the reactions
\begin{equation}\label{2.1}
\ppm\,(\tp\ua)+p \,(\tp\ub)\to \ppm\,(\tp\uu)+p \,(\tp\ud)
\end{equation}
off shell and on shell.
In relations which are true for both cases we shall denote
the momenta for the off- and on-shell case 
with a tilde, e.g., $\tp_{a}, \dots ,\tp_{2}$ in \eqref{2.1}. 
The corresponding on-shell momenta will be denoted without a tilde. 
In relations which are true only on shell, the momenta
will be denoted without a tilde.
Thus, the on-shell scattering reactions are denoted as
\begin{equation}\label{2.2}
\ppm\,(p\ua)+p\,(p\ub)\to \ppm\,(p\uu)+p\,(p\ud)\,.
\end{equation}
To define the amplitudes $\cM^{(0)\pm}$ for the off-shell processes \eqref{2.1} we consider the connected part of the following four-point functions
\begin{align}\label{2.3}
G_{4c}^{\pm}&(x\uu, x\ud , x\ua,x\ub)=
\braket{0|T(\varphi\hpm(x\uu)\varphi\hmp(x\ua)\psi(x\ud)\overline{\psi}(x\ub)|0}_{c}.
\end{align}
We have then as defining equation for $\cM^{(0)\pm}$
\begin{align}\label{2.4}
&
i(2\pi)^{4}\delta^{(4)}\left(\tp\ua +\tp\ub -\tp\uu -\tp\ud\right)
iS_{F}(\tp\ud )i\Delta_{F}\left(\tp\uu\right)\nn\\
&\quad \times 
\cM^{(0)\pm}\left(\tp\uu ,\tp\ud , \tp\ua ,\tp \ub\right)
iS_{F}(\tp\ub )i\Delta_{F}\left(\tp\ua\right)\nn\\
&
=\int d^{4}x\uu\, d^{4}x\ud\, d^{4}x\ua\,d^{4}x\ub\nn\\
&\quad \times \exp\left[ i\tp\uu x\uu+i\tp\ud x\ud-i\tp\ua x\ua-i\tp\ub x\ub\right]\nn\\ 
&\quad \times G_{4c}^{\pm}\left(x_{1},x_{2},x_{a},x_{b}\right)\,.
\end{align}
The propagators $S_{F}$ and $\Delta_F$ for protons and pions, respectively, are discussed in Appendix~\ref{app:B}, Sec. \ref{subsec:B1}. Given the function $G_{4c}$ the matrix-valued amplitudes $\cM^{(0)\pm}$ are well defined everywhere since the propagators $S_F$ and $\Delta_F$ are finite and nearly everywhere non zero away from the respective mass shells for protons and pions. 
The on-shell amplitudes $\cM^{(0)\pm}(p_{1},\,p_{2},\,p_{a},\,p_{b})$ are then uniquely defined as
the limit of the off-shell amplitudes $\cM^{(0)\pm}(\tp\uu ,\,\tp\ud ,\, \tp\ua ,\,\tp \ub)$ as $\tilde{p}_{j}\to p_{j}$. 

Now we consider the on-shell reactions 
\begin{equation}
\label{2.5}
\ppm\,(p\ua)+ p\,(p\ub ,\lambda\ub )\to \ppm\,(p\uu)+p\,(p\ud ,\lambda\ud )\,,
\end{equation}
where $\lambda_b \,, \lambda_2\in\lbrace1/2, \, -1/2\rbrace$ are the spin indices and we have $p_{a}^{0},\,p_{b}^{0},\,p_{1}^{0},\,p_{2}^{0}>0$.
The standard reduction formulas for pions and protons 
(see, e.g., \cite{Bjorken:1965}) 
give for the ${\mathcal T}$-matrix element
\begin{align}\label{2.6}
&\braket{\ppm(p\uu), \, p(p\ud ,\lambda\ud)|{\mathcal T}|\ppm(p\ua), p(p\ub ,\lambda\ub)} \nn\\
&\quad =\bar{u}(p\ud ,\lambda\ud)\cM^{(0)\pm}(p_{1},\,p_{2},\,p_{a},\,p_{b})u(p\ub ,\lambda\ub)\,.
\end{align}

The kinematics of the off-shell reaction \eqref{2.1} is our next topic. 
We have from \eqref{2.4} the energy-momentum conservation equation
\begin{equation}
\label{2.7}
\tp\ua +\tp\ub =\tp\uu +\tp \ud\,.
\end{equation}
We denote the invariant off-shell squared masses by
\begin{equation}
\label{2.8}
m_{a}^{2}=\tp_{a}^{2}\;,\quad
m_{b}^{2}=\tp_{b}^{2}\;,\quad
m_{1}^{2}=\tp_{1}^{2}\;,\quad
m_{2}^{2}=\tp_{2}^{2}\,.
\end{equation}
From \eqref{2.7} we see that there are three independent momenta which we choose as follows
\begin{align}
\label{2.9}
\tp_{s}=\tp\ua +\tp\ub =\tp\uu +\tp\ud\,,\nn\\
\tp_{t}=\tp\ua -\tp\uu =\tp\ud -\tp\ub\,,\nn\\
\tp_{u}=\tp\ua -\tp\ud =\tp\uu -\tp\ub\,.
\end{align}
We set
\begin{equation}
\label{2.10}
\tilde{s}=\tp_{s}^2\,,\quad
\tilde{t}=\tp_{t}^2\,,\quad
\tilde{u}=\tp_{u}^2\,,
\end{equation}
where
\begin{equation}
\label{2.11}
\tilde{s} + \tilde{t} + \tilde{u} 
= m_{a}^2 + m_{b}^2 + m_{1}^2 + m_{2}^2 \,.
\end{equation}

For the on-shell process \eqref{2.2} we have the relations \eqref{2.9} and \eqref{2.10} without tildes 
and the squared masses \eqref{2.8} are
\begin{align}
\label{2.12}
m_{a}^{2}=m_{1}^{2}=m_{\pi}^{2}\,, \quad
m_{b}^{2}=m_{2}^{2}=m_{p}^{2}\,.
\end{align}

Further kinematic relations are given in Appendix~\ref{app:A}.

The symmetries $P$, $C$, and $T$ hold in QCD and we recall the corresponding transformations of the pion and proton fields in Appendix~\ref{app:B}. 
Using \eqref{B2}--\eqref{B4} and \eqref{B13}--\eqref{B15} we find the following relations 
for $\cM^{(0)\pm}$~\eqref{2.4}.
\\
\\
$P$ invariance:
\begin{align}
\label{2.13}
&\cM^{(0)\pm}(\tp\uu ,\,\tp\ud , \,\tp\ua ,\,\tp \ub)\nn\\
&\quad =\gamma_{0}\cM^{(0)\pm}(\mathcal{P}\tp\uu ,\mathcal{P}\tp\ud , \mathcal{P}\tp\ua ,\mathcal{P}\tp \ub)\gamma_{0}\,,
\intertext{$C$ invariance:}
\label{2.14}
&\cM^{(0)\pm}(\tp\uu ,\,\tp\ud , \,\tp\ua ,\,\tp \ub)\nn\\
&\quad =S(C)\left[\cM^{(0)\mp}(\tp\uu, -\tp\ub ,\,\tp\ua , -\tp\ud)\right]^{\top}S^{-1}(C) \nn \\
&\quad =S(C)\left[\cM^{(0)\pm}(-\tp\ua, -\tp\ub ,-\tp\uu , -\tp\ud)\right]^{\top}S^{-1}(C)\,,
\intertext{$T$ invariance:}
\label{2.15}
&\cM^{(0)\pm}(\tp\uu ,\,\tp\ud ,\, \tp\ua ,\,\tp \ub)\nn \\
&\quad =S(T)\left[\cM^{(0)\pm}(\mathcal{P}\tp\ua, \mathcal{P}\tp\ub ,\mathcal{P}\tp\uu , \mathcal{P}\tp\ud)\right]^{\top}S^{-1}(T)\,.
\end{align}
Here $S(C)$, $S(T)$ and $\mathcal{P}$ are defined in \eqref{B5} and \eqref{B8}, respectively.

We can now write the following general expansions for $\cM^{(0)\pm}$ taking already into account $P$ invariance \eqref{2.13}
\begin{align}
\label{2.16}
&\cM^{(0)\pm}(\tp\uu ,\,\tp\ud ,\, \tp\ua ,\,\tp \ub)\nn\\
& \quad =
\cM_{1}^{\pm}+\tnp_{s}\cM_{2}^{\pm}+\tnp_{t}\cM_{3}^{\pm}+\tnp_{u}\cM_{4}^{\pm}\nn\\
&\qquad +i\sigma_{\mu\nu}\tp_{s}{}^{\mu}\tp_{t}{}^{\nu}\cM_{5}^{\pm}+i\sigma_{\mu\nu}\tp_{s}{}^{\mu}\tp_{u}{}^{\nu}\cM_{6}^{\pm}\nn\\
&\qquad +i\sigma_{\mu\nu}\tp_{t}{}^{\mu}\tp_{u}{}^{\nu}\cM_{7}^{\pm}\nn\\
&\qquad +i\gamma_{\mu}\gamma_{5}\varepsilon^{\mu\nu\rho\sigma}\tp_{s\nu}\tp_{t\rho}\tp_{u\sigma}\cM_{8}^{\pm}\,.
\end{align}
Here the invariant amplitudes $\cM_{j}^{\pm}$ can only depend on $\tilde{s}$, $\tilde{t}$,
and the invariant squared masses \eqref{2.8}
\begin{equation}\label{2.17}
\cM_{j}^{\pm}=\cM_{j}^{\pm}(\tilde{s},\,\tilde{t},\,m_{1}\2,\, m_{2}\2,\, m\ua\2,\,m\ub\2)\,.
\end{equation}
Both, the $C$ as well as the $T$ invariance relation, \eqref{2.14} and \eqref{2.15}, respectively, require
\begin{align}\label{2.18}
&\cM_{j}^{\pm}(\tilde{s},\,\tilde{t},\,m_{1}\2,\, m_{2}\2,\, m\ua\2,\,m\ub\2)\nn\\
&\quad =\cM_{j}^{\pm}(\tilde{s},\,\tilde{t},\,m_{a}\2,\, m\ub\2,\,m\uu\2,\,m\ud\2)\,,\nn\\
&\text{for} \; j=1,\,2,\,4,\,5,\,7,\,8\,,
\end{align}
and
\begin{align}\label{2.19}
&\cM_{j}^{\pm}(\tilde{s},\,\tilde{t},\,m_{1}\2,\, m_{2}\2,\, m\ua\2,\,m\ub\2)\nn\\
&\quad =-\cM_{j}^{\pm}(\tilde{s},\,\tilde{t},\,m_{a}\2,\, m\ub\2,\,m\uu\2,\,m\ud\2)\,,\nn\\
&\text{for} \; j=3,\,6\,.
\end{align}

Now we consider the ${\mathcal T}$-matrix element \eqref{2.6} 
for the on-shell process \eqref{2.5}. 
Inserting in \eqref{2.6} the on-shell version of $\cM^{(0)\pm}$ 
from \eqref{2.16} we get
\begin{align}\label{2.20}
&\braket{\ppm(p\uu), \, p(p\ud ,\lambda\ud)|{\mathcal T}|\ppm(p\ua), \, p(p\ub ,\lambda\ub)}\nn\\
&\quad =\bar{u}(p\ud ,\lambda\ud)\Big{[} \cM_{1}^{(\text{on})\pm}
+\slash{p}_{s}\cM_{2}^{(\text{on})\pm}
+\slash{p}_{u}\cM_{4}^{(\text{on})\pm}\nn\\
&\qquad
+i\sigma_{\mu\nu}p_{s}{}^{\mu}p_{t}{}^{\nu}\cM_{5}^{(\text{on})\pm}
+i\sigma_{\mu\nu}p_{t}{}^{\mu}p_{u}{}^{\nu}\cM_{7}^{(\text{on})\pm}\nn\\
&\qquad
+i\gamma_{\mu}\gamma_{5}\varepsilon^{\mu\nu\rho\sigma} 
p_{s\nu}p_{t\rho}p_{u\sigma}\cM_{8}^{(\text{on})\pm}\Big{ ]} u(p\ub ,\lambda\ub)\,,
\end{align}
where
\begin{align}
\label{2.21}
&\cM_{j}^{(\text{on})\pm}=\cM_{j}^{\pm}(s,t,m_{\pi}\2 ,m_{p}\2 ,m_{\pi}\2 ,m_{p}\2 )\;,\nn\\
& j=1,\dots , 8\,.
\end{align}
From \eqref{2.19} we see that on shell we have
\begin{align}\label{2.22}
\cM_{3}^{(\text{on})\pm}=0\,, \qquad 
\cM_{6}^{(\text{on})\pm}=0\,.
\end{align}
Therefore, we have omitted $\cM_{3}^{(\text{on})\pm}$ and $\cM_{6}^{(\text{on})\pm}$ in \eqref{2.20}. For the on-shell processes we can further simplify the ${\mathcal T}$-matrix element \eqref{2.20} by using the relations \eqref{A5}--\eqref{A9} from Appendix~\ref{app:A}. 
This gives
\begin{align}
\label{2.23}
&\braket{\ppm (p\uu), \,  p(p\ud ,\lambda\ud)|{\mathcal T}|\ppm(p\ua), \, p(p\ub ,\lambda\ub)} \nn\\
& \quad = \bar{u}(p\ud , \lambda\ud)\Big{[}A^{(\text{on})\pm}(s,t)\nn\\
& \qquad +\frac{1}{2}(\slash{p}_{a}+\slash{p}_{1})
B^{(\text{on})\pm}(s,t)\Big{]}u(p\ub,\lambda\ub )\,,
\end{align}
where
\begin{align}
\label{2.24}
&A^{(\text{on})\pm}(s,t)=\cM_{1}^{(\text{on})\pm}+m_{p}\cM_{2}^{(\text{on})\pm}\nn-m_{p}\cM_{4}^{(\text{on})\pm}\nn\\
&\quad +(-s+m_{p}\2 +m_{\pi}\2)\cM_{5}^{(\text{on})\pm}\nn\\
&\quad +(s+t-m_{p}\2 -m_{\pi}\2)\cM_{7}^{(\text{on})\pm}\nn \\
&\quad -m_{p}(2s+t-2m_{p}\2 -2m_{\pi}\2)\cM_{8}^{(\text{on})\pm}\,,\\
\nn\\
\label{2.25}
&B^{(\text{on})\pm}(s,t)=\cM_{2}^{(\text{on})\pm}+\cM_{4}^{(\text{on})\pm}
+2m_{p}\cM_{5}^{(\text{on})\pm} \nn\\
&\quad -2m_{p}\cM_{7}^{(\text{on})\pm}
+(4m_{p}^{2}-t)\cM_{8}^{(\text{on})\pm}\,.
\end{align}
With \eqref{2.23} we recover a standard form 
for the on-shell $\ppm p\to \ppm p$ scattering amplitudes;
see, e.g., Chap.~18.10 of \cite{Bjorken:1965}. 
We see from \eqref{2.24} and \eqref{2.25} that in the on-shell processes \eqref{2.2} only two combinations of the eight invariant amplitudes describing 
the corresponding off-shell processes \eqref{2.1} 
can be measured. 
In the next section, where we discuss 
the reactions $\ppm p\to \ppm p\gamma$, we will, however, have to deal with the off-shell amplitudes \eqref{2.16} with all eight invariant amplitudes. 
This presents one challenge of our present investigation. 

For completeness we recall also the standard forms 
of the $\pi p$ amplitudes obtained when going from the Dirac to the Pauli spinors for the protons. 
We have
\begin{align}\label{2.26}
&u(p,\lambda)=\sqrt{p^{0}+m_{p}}\left(\begin{array}{l}
\chi_{\lambda}\\
\\
\dfrac{\vec{\sigma}\cdot \vec{p}}{p^{0}+m_{p}}\chi_{\lambda}
\end{array}\right)\,,\nn\\
& \chi_{\lambda}^{\dagger}\,\chi_{\lambda}=1 
\quad(\text{no summation over}\, \lambda)\,.
\end{align}
We consider now the reactions \eqref{2.5} 
in the c.m. system with $\theta$ the scattering angle; 
see Fig.~\ref{fig:1}.
%--------------------------------------------
% Figure environment removed
%--------------------------------------------

We set here
\begin{align}\label{2.27}
&E=p\ub^{0}=p\ud^{0}=\frac{1}{2\sqrt{s}}(s+m_{p}\2-m_{\pi}\2)\,,\nn\\
&\vec{n}=\frac{\vec{p}\ua\times\vec{p}\uu}{|\vec{p}\ua\times\vec{p}\uu|}\,,
\end{align}
and obtain
\begin{align}\label{2.28}
&p\ua^{0}=\sqrt{s}-E\,,\nn\\
&|\vec{p}\ua|=|\vec{p}\ub|=|\vec{p}\uu|=|\vec{p}\ud|=\sqrt{E\2-m_{p}\2}\,,\nn\\
&\cos \theta = \frac{\vec{p}\uu\cdot\vec{p}\ua}{|\vec{p}\ua|\2}\,,\nn\\
&t=-2(E\2-m_{p}\2)(1-\cos \theta)\,,\nn\\
&|\vec{p}\ua\times\vec{p}\uu|=(E\2-m_{p}\2)\sin \theta\,.
\end{align}
Inserting all this in \eqref{2.23} we get
\begin{align}\label{2.29}
&\braket{\ppm(p\uu), \, p(p\ud ,\lambda\ud)|{\mathcal T}|\ppm(p\ua),\, p(p\ub ,\lambda\ub)}\nn\\
&\quad =8\pi\sqrt{s}\;\chi_{\lambda_{2}}^{\dagger}\Big{[}f^{\pm}(\sqrt{s},\theta)+i \vec{\sigma}\cdot \vec{n}\, g^{\pm}(\sqrt{s},\theta)\Big{]}\chi_{\lambda_{b}}\,,
\end{align}
where
\begin{align}\label{2.30}
&f^{\pm}(\sqrt{s},\theta)=\frac{1}{8\pi\sqrt{s}}
\bigg{\lbrace}(E+m_{p})\Big{[}A^{(\text{on})\pm}(s,t)\nn \\
&\quad+(\sqrt{s}-m_{p})B^{(\text{on})\pm}(s,t)\Big{]}\nn\\
&\quad +\cos \theta (E-m_{p})\Big{[}-A^{(\text{on})\pm}(s,t)\nn\\
&\quad
+(\sqrt{s}+m_{p})B^{(\text{on})\pm}(s,t)\Big{]}\bigg{\rbrace}\,,\\
\label{2.31}
&g^{\pm}(\sqrt{s},\theta)=-\sin \theta\frac{E-m_{p}}{8\pi\sqrt{s}}\nn\\
& \quad \times \Big{[}-A^{(\text{on})\pm}(s,t)+(\sqrt{s}+m_{p})B^{(\text{on})\pm}(s,t)\Big{]}\ .
\end{align}
With \eqref{2.29} we have another standard form of the $\pi p$ scattering amplitude; see for instance Chap.~16.3 of \cite{Nachtmann:1990ta}.

\section{The reaction $\pi^{-}p\to\pi^{-} p \gamma$}
\label{sec:3}

In this section we will study in detail the reaction
\begin{equation}
\pi^{-}\,(p\ua)+p\,(p\ub,\lambda\ub)\to \pi^{-}\,(p'\uu)+p\,(p'\ud,\lambda'\ud)+\gamma\,(k , \varepsilon)\,.
\label{3.1}
\end{equation}
Here $\varepsilon$ is the polarization vector of the photon and $\lambda\ub\,,\,\lambda'_{2}\in \lbrace 1/2 , -1/2\rbrace$ are the spin indices of the protons.
We are interested in the soft-photon limit and want to compare the amplitude for \eqref{3.1} to that of the process \eqref{2.5} without photon
\begin{equation}
\label{3.2}
\pi^{-}\,(p\ua)+p\,(p\ub ,\lambda\ub)\to\pi^{-}\,(p\uu)+p\,(p\ud ,\lambda\ud)\,.
\end{equation}
The energy-momentum conservation requires for \eqref{3.1}~and \eqref{3.2}
\begin{align}\label{3.3}
p\ua+p\ub&=p'\uu+p'\ud +k\,,
\intertext{and}
\label{3.4}
p\ua+p\ub&=p\uu+p\ud\,,
\end{align}
respectively.
In order not to overload the notation we consider 
here at first only $\pi^{-}p$ scattering.
But at the end of this section we give the results
for both reactions $\pi^{-} p \to \pi^{-} p \gamma$
and $\pi^{+} p \to \pi^{+} p \gamma$.
% and mention the changes to be done when going from $\pi^{-}p$ to $\pi^{+}p$ scattering at the end. 

The ${\mathcal T}$-matrix element for the reaction \eqref{3.1} 
is given by
\begin{align}
\label{3.5}
&\braket{\pi^{-}(p'\uu), \, p(p'\ud,\lambda'\ud),\,\gamma(k,\varepsilon)|{\mathcal T}|\pi^{-}(p\ua),\,p (p\ub,\lambda\ub)}\nn\\
&\quad =\varepsilon{}^{*}_{\lambda}\,\bar{u}(p'\ud ,\lambda'\ud) \cM^{\lambda}(p'\uu , p'\ud ,k, p\ua, p\ub)\, u(p\ub , \lambda\ub)\ .
\end{align}
For real photon emission we have $k^2=0$. But we shall consider $\cM^{\lambda}$ for arbitrary $k$ corresponding to real or virtual photon emission or absorption. 

As mentioned in the introduction we treat 
the reaction \eqref{3.1} in full QCD but only to leading order in the electromagnetic interaction. The precise definition of $\bar{u} \cM^{\lambda} u$ 
in \eqref{3.5} using the framework of QFT 
is given in Appendix~\ref{app:C};
see Eqs.~\eqref{C2} and \eqref{3.17} below.

We want to study the amplitude \eqref{3.5} 
in the limit \mbox{$k \to 0$}. 
We see immediately from \eqref{3.3} that with changing $k$ also $p'\uu$ and $p'\ud$ must change. Setting in \eqref{3.5} $p'\uu=p\uu$ and $p'\ud=p\ud$ we would for $k\neq 0$ violate energy-momentum conservation. Therefore, we proceed as in \cite{Lebiedowicz:2021byo},
where in Sec.~III a detailed discussion
of this issue for the reaction $\pi \pi \to \pi \pi \gamma$
is given.

We set
\begin{equation}\label{3.6}
p'\uu=p\uu-l\uu\,,\quad p'\ud=p\ud-l\ud\,,
\end{equation}
and we get for $l_{1,2}$ the conditions
\begin{align}\label{3.7}
l\uu+l\ud&=k\,,\nn\\
(p\uu-l\uu)\2&=m_{\pi}\2\,,\nn\\
(p\ud-l\ud)\2&=m_{p}\2\,,
\end{align}
which we can also write as
\begin{align}\label{3.8}
l\uu+l\ud&=k\,,\nn\\
2\,(p\uu\cdot l\uu)&=l_{1}\2\,,\nn\\
2\,(p\ud\cdot l\ud)&=l_{2}\2\,.
\end{align}
These are 6 equations for the 8 unknown components of 
$l\uu$ and $l\ud$. We expect that the solution of \eqref{3.6} will have 2 free parameters and this is indeed the case; see~\cite{Lebiedowicz:2021byo}. 
As~discussed there we shall also here consider the situation that the momenta $p'\uu$ and $p'\ud$ in \eqref{3.1} are close to the momenta $p\uu$ and $p\ud$, respectively, in \eqref{3.2}. 
That is, we shall assume that all components of $k$, $l\uu$, and $l\ud$, are of order $\omega$ and then study 
the limit $\omega\to 0$. 
We want to extract the terms of order $\omega^{-1}$ 
and $\omega^{0}$ from the amplitude \eqref{3.5}.
For this we need $l\uu$ and $l\ud$ as solutions of \eqref{3.8} only to order $\omega$. 
Neglecting, therefore, $l\uu\2$ and $l\ud\2$ in \eqref{3.8} we get a system of linear equations for $l\uu$ and $l\ud$
\begin{align}
\label{3.9}
l\uu+l\ud&=k\,,\nn\\
p\uu\cdot l\uu&=0\,,\nn\\
p\ud\cdot l\ud&=0\,.
\end{align}
To obtain the solution of \eqref{3.9} we go to the c.m. system of the reactions \eqref{3.1} and \eqref{3.2} where we have 
\begin{align}
\label{3.9a}
&\vec{p}\ua+\vec{p}\ub=\vec{p}\uu+\vec{p}\ud=0\,;\\
\label{3.9b}&(p\uu{}^{\mu})=\left(\begin{array}{l}
p\uu^{0}\\
\\
|\vec{p}\uu|\,\widehat{p}\uu
\end{array}\right),\quad |\, \widehat{p}\uu|=1\;,
\nn\\
&(p\ud{}^{\mu})=\left(\begin{array}{l}
p\ud^{0}\\
\\
-|\vec{p}\ud|\,\widehat{p}\uu
\end{array}\right);
\\
\label{3.10}
&p\uu^{0}=\frac{1}{2\sqrt{s}}(s+m_{\pi}^{2}-m_{p}\2)\;,\nn\\
&p\ud^{0}=\frac{1}{2\sqrt{s}}(s+m_{p}^{2}-m_{\pi}\2)\;,\nn\\
&|\vec{p}\uu|=|\vec{p}\ud|=\sqrt{(p\uu^{0})\2-m_{\pi}\2}\,.\\
\intertext{We set}
\label{3.11}
&(k^{\mu})=
\left( \begin{array}{l}
k^{0}\\
k_{\parallel}\,\widehat{p}\uu+\vec{k}_{\perp}
\end{array}\right) , \nn\\
&k^{0}=\omega\,,\qquad \vec{k}_{\perp}\cdot\widehat{p}\uu=0\,.
\end{align}
The solution of \eqref{3.9} reads then
\begin{align}
\label{3.12}
(l\uu{}^{\mu})&=\left( \begin{array}{l}
\dfrac{p\ud\cdot k}{\sqrt{s}} \\
\dfrac{p_{1}^{0}}{|\vec{p}\uu|\,\sqrt{s}}(p\ud\cdot k )\widehat{p}\uu +\vec{l}_{1\perp}
\end{array}\right),\\
\label{3.13}
(l\ud{}^{\mu})&=\left( \begin{array}{l}
\dfrac{p\uu\cdot k}{\sqrt{s}}\\
\vec{k}-\dfrac{p_{1}^{0}}{|\vec{p}\uu|\,\sqrt{s}}(p\ud\cdot k )\widehat{p}\uu -\vec{l}_{1\perp}
\end{array}\right),
\end{align}
where
\begin{equation}
\label{3.14}
\vec{l}_{1\perp}\cdot\widehat{p}_{1}=0\,.
\end{equation}
The vector $\vec{l}_{1\perp}$ remains undetermined and corresponds 
to the 2 free parameters of the solution of \eqref{3.9}. 
In the following we require $\vec{l}_{1\perp}$ to be of order $\omega$. 
With \eqref{3.12}--\eqref{3.14} 
we have given the solution of \eqref{3.8} 
correct to order $\omega$; see Fig.~\ref{fig:2}.
%--------------------------------------------
% Figure environment removed
%--------------------------------------------

Our simple kinematic discussion clearly shows that it makes no sense trying to expand the amplitude \eqref{3.5} with respect to the photon momentum $k$ keeping $p'\uu=p\uu$ and $p'\ud= p\ud$ fixed to their values in \eqref{3.2}. The momentum configuration $p\uu$, $p\ud$, $k$ leads \underline{outside} of the physical region for $k\neq 0$, since then we would have from \eqref{3.4}
\begin{equation}
\label{3.15}
p\ua+p\ub\neq p\uu +p\ud +k\,.
\end{equation}

Now we come to another problem in the comparison of the amplitudes for \eqref{3.1} and \eqref{3.2}. In \eqref{3.5} we have the spinor $\bar{u}(p'\ud , \lambda'\ud)$ which must change with changing~$p'\ud$. We~could fix, by some arbitrary convention, 
these spinors for each $p'\ud$ in order to have a uniquely defined function for the ${\mathcal T}$-matrix element in \eqref{3.5}. 
We avoid this arbitrariness by considering 
instead the following matrix function
\begin{align}
\label{3.16}
&\mathcal{N}^{\lambda}(p'\uu, p'\ud, k, p\ua ,p\ub)\nn\\
&\quad =\sum_{\lambda'\ud , \lambda\ub} u(p'\ud ,\lambda'\ud)\bar{u}(p'\ud ,\lambda'\ud) \nn\\
&\qquad \times \mathcal{M}^{\lambda}(p'\uu, p'\ud, k, p\ua ,p\ub) u(p\ub ,\lambda\ub)\bar{u}(p\ub ,\lambda\ub)\nn\\
&\quad =(\slash{p}'\ud +m_{p})
\mathcal{M}^{\lambda}(p'\uu, p'\ud, k, p\ua ,p\ub)(\slash{p}\ub +m_{p})\,.
\end{align}
$\mathcal{N}^{\lambda}$ is unambiguously defined and contains all information on the matrix elements \eqref{3.5} for all spin configurations, since we have
\begin{align}
\label{3.17}
&\bar{u}(p'\ud ,\lambda'\ud) \mathcal{M}^{\lambda}u(p\ub ,\lambda\ub )=\frac{1}{(2m_{p})\2}\bar{u}(p'\ud ,\lambda'\ud)\mathcal{N}^{\lambda}u(p\ub ,\lambda\ub)\,.
\end{align}

In the following we shall, therefore, mainly work with $\mathcal{N}^{\lambda}$ instead of $\mathcal{M}^{\lambda}$.
The definition of $\mathcal{N}^{\lambda}$
using the reduction formulas of QFT is given in \eqref{C2}.
We have five types of diagrams for~$\mathcal{N}^{\lambda}$; 
see Fig.~\ref{fig:3}.
%-----------------------------------------------
% Figure environment removed
%----------------------------------------------
The diagrams Figs.~\ref{fig:3}(a)
and~\ref{fig:3}(b) correspond to the emission of the photon from the external $\pi^{-}$ lines. 
In Figs.~\ref{fig:3}(c) and \ref{fig:3}(d) we have the diagrams for photon emission by the external proton lines. 
Fig.~\ref{fig:3}(e) corresponds to the structure term, that is, all the remaining photon emissions.
We have
\begin{equation}
\label{3.18}
\mathcal{N}\ul=\cN\ul^{(a)}+\cN\ul^{(b)}+\cN\ul^{(c)}+\cN\ul^{(d)}+\cN\ul^{(e)}\,,
\end{equation}
and gauge invariance requires
\begin{equation}
\label{3.19}
k^{\lambda}\mathcal{N}_{\lambda}=0\ .
\end{equation}
Note that in the diagrams of Figs.~\ref{fig:3}(a)--(d) we have the complete $\pi$ and $p$ propagators, the complete $\gamma\pi\pi$ and $\gamma pp$ vertex functions, and the off-shell $\pi p\to \pi p$ scattering amplitudes.
We have
\begin{align}
\label{3.20}
\cN\ul^{(a)}&=-e (\slash{p}'\ud+m_{p})\mathcal{M}^{(0,a)}(\slash{p}\ub+m_{p})\nn\\
&\quad \times \Delta_{F}\big{[}(p\ua-k)\2\big{]}\widehat{\Gamma}\ul^{(\gamma\pi\pi)}(p\ua-k, \, p\ua)\,,\\
\nn \allowdisplaybreaks\\
\label{3.21}
\cN\ul^{(b)}&=-e\widehat{\Gamma}\ul^{(\gamma\pi\pi)}(p'\uu,\,p'\uu+k)\Delta_{F}\big{[}(p'\uu+k)\2\big{]}\nn\\
&\quad \times 
(\slash{p}'\ud+m_{p})\mathcal{M}^{(0,b)}(\slash{p}\ub+m_{p})\,,\\
\nn \allowdisplaybreaks\\
\label{3.22} 
\cN\ul^{(c)}&=e (\slash{p}'\ud+m_{p})\mathcal{M}^{(0,c)}S_{F}(p\ub -k)\nn\\
&\quad \times \widehat{\Gamma}\ul^{(\gamma pp)}(p\ub-k,\,  p\ub)(\slash{p}\ub+m\up)\,,\\ \allowdisplaybreaks
\nn \\
\label{3.23}
\cN\ul^{(d)}&= e (\slash{p}'\ud+m_{p})\widehat{\Gamma}\ul^{(\gamma pp)}(p'\ud,\,p'\ud+k)\nn\\
&\quad \times S_{F}(p'\ud+k)\mathcal{M}^{(0,d)}(\slash{p}\ub+m\up)\,.\allowdisplaybreaks
\end{align}
Here the propagators $\Delta_{F}$, $S_{F}$,
and the vertex functions $\widehat{\Gamma}\ul^{(\gamma\pi\pi)}$, $\widehat{\Gamma}\ul^{(\gamma pp)}$ are defined and discussed in Appendix~\ref{app:B} and we have set for the off-shell~$\pi^{-} p\to \pi^{-} p$ amplitudes occurring 
\begin{align}
\label{3.24}
\mathcal{M}^{(0,a)}&=\mathcal{M}^{(0)}(p'\uu,\, p'\ud,\, p\ua-k ,\, p\ub )\,,\\
\label{3.25}
\mathcal{M}^{(0,b)}&=\mathcal{M}^{(0)}(p'\uu+k,\,p'\ud,\, p\ua ,\, p\ub )
\,,\\
\label{3.26}
\mathcal{M}^{(0,c)}&=\mathcal{M}^{(0)}(p'\uu,\, p'\ud,\, p\ua ,\, p\ub -k)\,,\\
\label{3.27}
\mathcal{M}^{(0,d)}&=\mathcal{M}^{(0)}(p'\uu,\, p'\ud+k,\, p\ua ,\, p\ub )\,.
\end{align}
From the generalized Ward identities \eqref{B33} and \eqref{B55} we get
\begin{align}
\label{3.28}
k^{\lambda}\cN\ul^{(a)}&=\phantom{-}e\cN^{(0,a)}=\phantom{-}e(\slash{p}'\ud+m\up)\cM^{(0,a)}(\slash{p}\ub+m\up)\,,\\
\label{3.29}
k^{\lambda}\cN\ul^{(b)}&=-e\cN^{(0,b)}=-e(\slash{p}'\ud+m\up)\cM^{(0,b)}(\slash{p}\ub+m\up)\,,\\
\label{3.30}
k^{\lambda}\cN\ul^{(c)}&=-e\cN^{(0,c)}=-e(\slash{p}'\ud+m\up)\cM^{(0,c)}(\slash{p}\ub+m\up)\,,\\
\label{3.31}
k^{\lambda}\cN\ul^{(d)}&=\phantom{-}e\cN^{(0,d)}=\phantom{-}e(\slash{p}'\ud+m\up)\cM^{(0,d)}(\slash{p}\ub+m\up)\,.
\end{align}
The structure term $\cN\ul^{(e)}$ will be determined to the order $\omega^{0}$ from the gauge-invariance relation \eqref{3.19} which, together with \eqref{3.18} and \eqref{3.28}-\eqref{3.31} gives
\begin{align}
\label{3.32}
k^{\lambda}\cN\ul^{(e)}
&=
-k^{\lambda}\big{[}\cN\ul^{(a)}+\cN\ul^{(b)}+\cN\ul^{(c)}+\cN\ul^{(d)}\big{]}\nn\\
&=
e\big{[}-\cN^{(0,a)}+\cN^{(0,b)}+\cN^{(0,c)}-\cN^{(0,d)}\big{]}\,.
\end{align}

The calculation of $\cN\ul^{(a)},\dots ,\cN\ul^{(d)}$ up to the order $\omega^{0}$ is, in principle, straightforward but very lengthy and tedious. We shall here only sketch the calculation of $\cN\ul^{(a)}$. The complete calculations for  $\cN\ul^{(a)},\dots ,\cN\ul^{(d)}$ are presented in Appendix~\ref{app:C}.

In $\cN\ul^{(a)}$ \eqref{3.20} the momentum $p\ua$ is on shell, 
$p\ua\2=m_{\pi}\2$. 
Therefore, we can use the result from \eqref{B41} which implies
\begin{align}
\label{3.33}
&\Delta_{F}\big{[}(p\ua-k)\2\big{]}\,\widehat{\Gamma}\ul^{(\gamma\pi\pi)}(p\ua-k ,\,p\ua)\nn\\
&\quad =\frac{(2p\ua -k)\ul}{-2p\ua\cdot k +k\2+i\varepsilon}+{\mathcal O}(\omega)\,.
\end{align}
This determines the terms of order $\omega^{-1}$ and $\omega^{0}$ in $\Delta_{F}\,\widehat{\Gamma}\ul^{(\gamma\pi\pi)}$. 
From \eqref{3.20}, \eqref{3.24}, and \eqref{3.28} 
we see that we have to calculate
\begin{align}
\label{3.34}
\cN^{(0,a)}&=(\slash{p}'\ud
+m\up)\cM^{(0,a)}(\slash{p}\ub+m\up)
\intertext{to the orders $\omega^{0}$ and $\omega^{1}$ 
in order to obtain $\cN\ul^{(a)}$ 
to the orders $\omega^{-1}$ and $\omega^{0}$. 
For the off-shell amplitude \eqref{3.24}}
\label{3.35}
\cM^{(0,a)}&=\cM^{(0)}(p'\uu ,\,p'\ud,\,p\ua-k,\,p\ub)
\end{align}
we have to use \eqref{2.16} where the momenta $\tilde{p}_{s}$, $\tilde{p}_{t}$, $\tilde{p}_{u}$ are given by
\begin{align}
\label{3.36}
\tilde{p}_{s} &=p\ua+p\ub-k=p_{s}-k\,,\nn\\
\tilde{p}_{t}&=p\ud-p\ub-l\ud=p_{t}-l\ud\,,\nn\\
\tilde{p}_{u}&=p\uu-p\ub-l\uu=p_{u}-l\uu\,.
\end{align}
%========================================
\begingroup
\allowdisplaybreaks
%========================================
Here and in the following $p_{s}$, $p_{t}$, $p_{u}$ are the momenta corresponding to the on-shell $\pi p\to \pi p$ reaction \eqref{3.2}:
\begin{align}
\label{3.37}
&p_{s}=p\ua+p\ub\,,\nn\\
&p_{t}=p\ud-p\ub\,,\nn\\
&p_{u}=p\uu-p\ub\,, \nn\\
&s=p_{s}\2\,,\quad t=p_{t}\2\,,\quad u=p_{u}\2\,.
\end{align}
We have from \eqref{3.35}, \eqref{3.36}
\begin{align}
&\tilde{s}=\tilde{p}_{s}\2=(p_{s}-k)\2=s-2p_{s}\cdot k+{\mathcal O}(\omega\2)\,,\nn\\
&\tilde{t}=\tilde{p}_{t}\2=(p_{t}-l_{2})\2=t-2p_{t}\cdot l_{2}+{\mathcal O}(\omega\2)\,,\nn\\
&m\ua\2=(p\ua-k)\2=m_{\pi}\2-2p_{a}\cdot k+{\mathcal O}(\omega\2)\ ,\nn\\
&m\uu\2=m_{\pi}\2\,,\nn\\
&m\ub\2=m\ud\2=m\up\2\ ,
\end{align}
and, therefore, from \eqref{2.16} and \eqref{2.17}
\begin{align}
\label{3.39}
\cN^{(0,a)}&=(\slash{p}'\ud+m\up)\Big{\lbrace}\cM\uu^{(a)}+(\slash{p}_{s}-\slash{k})\cM\ud^{(a)}\nn\\
&\quad +(\slash{p}_{t}-\!\not{l}\ud)\cM_{3}^{(a)}+(\slash{p}_{u}-\!\not{l}\uu)\cM_{4}^{(a)}\nn\\
&\quad +i\sigma_{\mu\nu}(p_{s}-k)^{\mu}(p_{t}-l\ud)^{\nu}\cM_{5}^{(a)}\nn\\
&\quad +i\sigma_{\mu\nu}(p_{s}-k)^{\mu}(p_{u}-l\uu)^{\nu}\cM_{6}^{(a)}\nn\\
&\quad +i\sigma_{\mu\nu}(p_{t}-l\ud)^{\mu}(p_{u}-l\uu)^{\nu}\cM_{7}^{(a)}\nn\\
&\quad +i\gamma_{\mu}\gamma_{5} 
\varepsilon^{\mu\nu\rho\sigma}(p_{s}-k)_{\nu}(p_{t}-l\ud)_{\rho}(p_{u}-l\uu)_{\sigma}\cM_{8}^{(a)}\Big{\rbrace}\nn\\
&\quad \times (\slash{p}\ub+m\up)
+{\mathcal O}(\omega^{2})\,,
\end{align}
where
\begin{align}\label{3.40}
\cM_{j}^{(a)}&=\cM_{j}(\tilde{s},\,\tilde{t},\,m_{\pi}\2,\,m\up\2,\,m\ua\2,\,m\up\2)\nn\\
&=\cM_{j}(s-2p_{s}\cdot k,\, t-2p_{t}\cdot l\ud,\,m_{\pi}\2,
m\up\2,
\nn\\
&\qquad \quad \; m_{\pi}\2-2p\ua\cdot k,\,m_{p}\2)+{\mathcal O}(\omega\2)\nn\\
&=\cM_{j}^{(\text{on})}(s,t,\,m_{\pi}\2,\,m\up\2,\,m_{\pi}\2,\,m\up\2)\nn\\
&\quad -2(p_{s}\cdot k)\, \cM_{j},_{s}(s,t,\,m_{\pi}\2,\,m\up\2,\,m_{\pi}\2,\,m\up\2)\nn\\
&\quad -2(p_{t}\cdot l\ud)\, \cM_{j},_{t}(s,t,\,m_{\pi}\2,\,m\up\2,\,m_{\pi}\2,\,m\up\2)\nn\\
&\quad -2(p_{a}\cdot k)\, \cM_{j},_{m_{a}\2}(s,t,\,m_{\pi}\2,\,m\up\2,\,m_{a}\2,\,m\up\2)\Big{|}_{m_{a}\2 = m_{\pi}\2} \nn\\
&\quad+{\mathcal O}(\omega\2)\,,  \\
(j&=1,\dots , 8)\,. \nn
\end{align}
Here  $\cM_{j},_{s}$,  $\cM_{j},_{t}$, $\cM_{j},_{m_{a}\2}$ denote the partial derivatives of $\cM_{j}$ with respect to $s$, $t$, and $m\ua\2$.

Now we insert \eqref{3.40} into \eqref{3.39} and put together the terms of order $\omega^{0}$ and $\omega^{1}$. This gives a lengthy expression for $\cN^{(0,a)}$ 
which is given in \eqref{C11} of Appendix~\ref{app:C}.
Multiplying with \eqref{3.33} we get for $\cN\ul^{(a)}$ 
(\ref{3.20}) the terms of order 
$\omega^{-1}$ and $\omega^{0}$
as given in \eqref{C22}. 
In an analogous way we calculate in Appendix~\ref{app:C} 
the terms of order $\omega^{-1}$ and $\omega^{0}$ 
for $\cN\ul^{(b)}$, $\cN\ul^{(c)}$, and $\cN\ul^{(d)}$, 
Eqs.~\eqref{3.21}--\eqref{3.23}. 
Finally we determine the structure term $\cN\ul^{(e)}$.
This term is non-singular for $\omega\to 0$ and to the order $\omega^{0}$ it can be uniquely determined from \eqref{3.32} using the results for $\cN^{(0,a)},\dots ,\cN^{(0,d)}$ 
to the orders $\omega^{0}$ and $\omega^{1}$; 
see Appendix~\ref{app:C}.

We give now the final result for $\cN\ul$ \eqref{3.16} to the orders $\omega^{-1}$ and $\omega^{0}$. 
Let $A^{(\text{on})\mp}(s,t)$ and $B^{(\text{on})\mp}(s,t)$ be the invariant amplitudes for $\pi^{\mp}p$ elastic scattering as defined in \eqref{2.23}--\eqref{2.25}.
Here and in the following we indicate again by the superscript 
$-\,(+)$ if we are dealing with $\pi^{-}p \;(\pi^{+}p)$ scattering. We set:
\begin{align}
\label{3.41}
A,_{s}^{\!\!(\text{on})\mp}(s,t)&=\frac{\partial}{\partial s}A^ {(\text{on})\mp}(s,t)\,,\nn\\
A,_{t}^{\!\!(\text{on})\mp}(s,t)&=\frac{\partial}{\partial t}A^ {(\text{on})\mp}(s,t)\,,\nn\\
B,_{s}^{\!\!(\text{on})\mp}(s,t)&=\frac{\partial}{\partial s}B^ {(\text{on})\mp}(s,t)\,,\nn\\
B,_{t}^{\!\!(\text{on})\mp}(s,t)&=\frac{\partial}{\partial t}B^ {(\text{on})\mp}(s,t)\,.
\end{align}
We get then
\begin{equation}
\label{3.42}
\cN\ul^{(\pi^{-}p\to\pi^{-} p\gamma)}=\cN\ul^{(a+b+e1)-}+\cN\ul^{(c+d+e2)-}\,,
\end{equation}
where
\\
\\
\\
\begin{widetext}
\begin{align}\label{3.43}
\cN\ul^{(a+b+e1)-}=&-e(\slash{p}'_{2}+m\up)\bigg{[}A^{(\text{on})-}+\frac{1}{2}(\slash{p}\ua+\slash{p}\uu-\!\not{l}\uu)B^{(\text{on})-}\bigg{]}(\slash{p}_{b}+m\up)\bigg{[}\frac{(2p\ua-k)\ul}{-2p\ua\cdot k+k\2}+\frac{(2p'\uu+k)\ul}{2p'\uu\cdot k+k\2}\bigg{]}\nn\\
&-e(\slash{p}\ud+m\up)\Bigg{\lbrace}\bigg{[}A,_{s}^{\!\!(\text{on})-}+\frac{1}{2}(\slash{p}\ua+\slash{p}\uu)
B,_{s}^{\!\!(\text{on})-}\bigg{]}\bigg{[}(-2p_{s}\cdot k)\frac{p_{a\lambda}}{(-p\ua\cdot k)}-2p_{s\lambda}\bigg{]}\nn\\
&+\bigg{[}A,_{t}^{\!\!(\text{on})-}+\frac{1}{2}(\slash{p}\ua+\slash{p}\uu)B,_{t}^{\!\!(\text{on})-}\bigg{]}(-2p_{t}\cdot l\ud)
\bigg{[}\frac{p_{a\lambda}}{(-p\ua\cdot k)}+\frac{p_{1\lambda}}{(p\uu\cdot k)}\bigg{]}\nn\\
&+B^{(\text{on})-}\bigg{[}\frac{1}{2}\slash{k}\Big{(}\frac{p_{a\lambda}}{(p\ua\cdot k)}+\frac{p_{1\lambda}}{(p\uu\cdot k)}\Big{)}-\gamma\ul\bigg{]}\Bigg{\rbrace}(\slash{p}\ub+m\up)+\mathcal{O}(\omega)\,.
\\
\label{3.44}
\cN\ul^{(c+d+e2)-}=&\;
e(\slash{p}'_{2}+m\up)\bigg{[}A^{(\text{on})-}+\frac{1}{2}(\slash{p}\ua+\slash{p}'\uu)B^{(\text{on})-}\bigg{]}(\slash{p}_{b}+m\up)\frac{(2p\ub-k)\ul}{-2p\ub\cdot k+k\2}\nn\\
&+e\frac{(2p'\ud+k)\ul}{2p'\ud\cdot k+k\2}(\slash{p}'\ud+m\up)
\bigg{[}A^{(\text{on})-}+\frac{1}{2}(\slash{p}\ua+\slash{p}'\uu)B^{(\text{on})-}\bigg{]}(\slash{p}\ub+m\up)\nn\\
&+e(\slash{p}\ud+m\up)\bigg{[}A,_{s}^{\!\!(\text{on})-}+\frac{1}{2}(\slash{p}\ua+\slash{p}\uu)B,_{s}^{\!\!(\text{on})-}\bigg{]}\bigg{[}2(p_{s}\cdot k)\frac{p_{b\lambda}}{(p\ub\cdot k)}-2p_{s\lambda}\bigg{]}(\slash{p}\ub+m\up)\nn\\
&+e(\slash{p}\ud+m\up)\bigg{[}A,_{t}^{\!\!(\text{on})-}+\frac{1}{2}(\slash{p}\ua+\slash{p}\uu)B,_{t}^{\!\!(\text{on})-}\bigg{]}
\bigg{[}-2(p_{t}\cdot l\uu)\frac{p_{b\lambda}}{(p\ub\cdot k)}+2(p_{t}\cdot l\uu)\frac{p_{2\lambda}}{(p\ud\cdot k)}\bigg{]}(\slash{p}\ub+m\up)\nn\\
&
+e(\slash{p}_{2}+m\up)\bigg{[}A^{(\text{on})-}+\frac{1}{2}(\slash{p}\ua+\slash{p}\uu)B^{(\text{on})-}\bigg{]}(k\ul-\slash{k}\gamma\ul)(\slash{p}\ub+m_{p})\frac{1}{(-2p\ub\cdot  k)}\nn\\
&-e\frac{1}{(2p\ud\cdot  k)}(\slash{p}\ud+m\up) (k\ul-\gamma\ul\slash{k})\bigg{[}A^{(\text{on})-}+\frac{1}{2}(\slash{p}\ua+\slash{p}\uu)B^{(\text{on})-}\bigg{]}(\slash{p}_{b}+m\up)\nn\\
&+e(\slash{p}\ud+m\up)\bigg{[}A^{(\text{on})-}+\frac{1}{2}(\slash{p}\ua+\slash{p}\uu)B^{(\text{on})-}\bigg{]}\bigg{[}m\up(k\ul-\nk\gamma\ul)+(p_{b\lambda}\nk-(p\ub\cdot k)\gamma\ul)\bigg{]}(\slash{p}_{b}+m\up) \nn\\
&\quad \times 
\frac{F\ud(0)}{m\up}\frac{1}{(-2p\ub\cdot k)}\nn\\
&-e\frac{F\ud(0)}{m\up}\frac{1}{(2p\ud\cdot k)}(\slash{p}_{2}+m\up)\nn\\
&\quad \times \bigg{[}m\up(k\ul-\gamma\ul\nk)+(p_{2\lambda}\nk-(p\ud\cdot k)\gamma\ul)\bigg{]}
\bigg{[}A^{(\text{on})-}+\frac{1}{2}(\slash{p}\ua+\slash{p}\uu)B^{(\text{on})-}\bigg{]}(\slash{p}\ub+m\up)+\mathcal{O}(\omega)\,.
\end{align}
\end{widetext}
Here the momenta $p_{a}$, $p\ub$, $p_{1}$, $p_{2}$, $p'_{1}$, $p'_{2}$, $k$, $l\uu$, $l\ud$ are as defined in \eqref{3.3}, \eqref{3.4}, \eqref{3.12}, and \eqref{3.13}, respectively.
The definition of $p_{s}$, $p_{t}$, $p_{u}$, $s$, $t$, $u$ is given in \eqref{3.37}.

To get the corresponding expressions for the reaction $\pi^{+}p \to p^{+}p\gamma$ we make the following replacements in \eqref{3.43} and \eqref{3.44}:
\begin{align}
\label{3.45}
&\mathcal{N}\ul^{(a+b+e1)+}=-\cN^{(a+b+e1)-}\Big{|}_{\mathop{^{A^{(\text{on})-}\to A^{(\text{on})+}}                  
_{B^{(\text{on})-}\to B^{(\text{on})+}}}}\,,\\
\label{3.46}
&\mathcal{N}\ul^{(c+d+e2)+}=\cN^{(c+d+e2)-}\Big{|}
_{\mathop{^{A^{(\text{on})-}\to A^{(\text{on})+}}                  
_{B^{(\text{on})-}\to B^{(\text{on})+}}}}\,.
\end{align}
Of course, these replacements also apply to the derivatives $A,_{s}^{\!\!(\text{on})},\dots, B,_{t}^{\!\!(\text{on})}$.
Note the minus sign in \eqref{3.45} which is due to the opposite charge of $\pi^{-}$ and $\pi^{+}$ which gives \eqref{B25}.
We get then
\begin{equation}
\label{3.47}
\cN^{(\pi^{+}p\to\pi^{+}p\gamma)}\ul=\mathcal{N}\ul^{(a+b+e1)+}+\mathcal{N}\ul^{(c+d+e2)+}\,.
\end{equation}
The main results of our paper are the above expressions for $\cN^{(\pi^{-}p\to\pi^{-}p\gamma)}\ul$ and $\cN^{(\pi^{+}p\to\pi^{+}p\gamma)}\ul$, Eqs.~\eqref{3.42}--\eqref{3.47}, which give these matrix amplitudes to the orders $\omega^{-1}$ and $\omega^{0}$ in the expansion for $\omega\to 0$. We have written \eqref{3.43} and \eqref{3.44} in a (hopefully) transparent form where we have not strictly grouped together the $\omega^{-1}$ and the $\omega^{0}$ terms.
Of course, this could easily be done by substituting $p'\ud=p\ud-l\ud\,$, $\,p'\uu=p\uu-l\uu$ and making suitable expansions, e.g.,
\begin{align}
\label{3.48}
\frac{(2p'\uu+k)\ul}{2p'\uu\cdot k +k\2}=&\;\frac{p_{1\lambda}}{(p\uu\cdot k)}-\frac{p_{1\lambda}}{2(p\uu\cdot k)\2}
\big{(}-2(l\uu\cdot k)+k\2\big{)}
\nn\\
\nn\\
&+\frac{1}{2(p\uu\cdot k)}(-2l\uu+k)\ul
+{\mathcal O}(\omega)\,.
\end{align}
Note that $k$, $l\uu$ and $l\ud$ are of order $\omega$. 
But we think that substituting such expansions in \eqref{3.43} and \eqref{3.44} would not increase their legibility. 

We give the explicit expansions of $\cN\ul$ 
in powers of $\omega$ in the following only for the case of real proton emission, that is, for $k\2=0$. 
We write
\begin{align}
\label{3.49}
&\cN\ul^{(\ppm p\to \ppm p\gamma)} \equiv \cN\ul\hpm \nn\\
&\quad 
=\frac{1}{\omega}\widehat{\cN}\ul^{(0)\pm}+\widehat{\cN}\ul^{(1)\pm}+{\mathcal O}(\omega)\,,\\
\nn \\
\label{3.50}
&\widehat{\cN}\ul^{(0)\pm}=\widehat{\cN}\ul^{(a+b+e1)(0)\pm}+\widehat{\cN}\ul^{(c+d+e2)(0)\pm}\,,\\
\nn\\
\label{3.51}
&\widehat{\cN}\ul^{(1)\pm}=\widehat{\cN}\ul^{(a+b+e1)(1)\pm}+\widehat{\cN}\ul^{(c+d+e2)(1)\pm}\,.
\end{align}
Neglecting some gauge terms we obtain from \er{3.42}--\er{3.48}
\begin{widetext}
\bal{3.52}
\widehat{\cN}\ul^{(a+b+e1)(0)\pm}=&\pm e(\bp\ud+m\up)\Big{[}A^{(\text{on})\pm}+\frac{1}{2}(\bp\ua +\bp\uu )B^{(\text{on})\pm}\Big {]}(\bp\ub +m\up)\;
\omega\Big{[}-\frac{p_{a\lambda}}{p\ua\cdot k}+\frac{p_{1\lambda}}{p\uu\cdot k}\Big{]}\,,
\\
\nn \\ 
\label{3.53}
\widehat{\cN}\ul^{(a+b+e1)(1)\pm}=
&\pm e(\bp\ud+m\up)\Big{[}A^{(\text{on})\pm}+\frac{1}{2}(\bp\ua +\bp\uu )B^{(\text{on})\pm}\Big {]}(\bp\ub +m\up)\frac{1}{(p\uu\cdot k)\2}\Big{[}p_{1\lambda}(l\uu\cdot k)-l_{1\lambda}(p\uu\cdot k)\Big{]}\nn\\
&\pm e(-\!\not{l}\ud)\Big{[}A^{(\text{on})\pm}+\frac{1}{2}(\bp\ua +\bp\uu )B^{(\text{on})\pm}\Big {]}(\bp\ub +m\up)
\Big{[}-\frac{p_{a\lambda}}{p\ua \cdot k}+\frac{p_{1\lambda}}{p\uu \cdot k}\Big{]}\nn\\
&\pm e(\bp\ud+m\up)\Big{[} \frac{1}{2}(-\!\not{l}\uu)B^{(\text{on})\pm}\Big {]}(\bp\ub +m\up)
\Big{[}-\frac{p_{a\lambda}}{p\ua \cdot k}+\frac{p_{1\lambda}}{p\uu \cdot k}\Big{]}\nn\\
&\pm e (\bp\ud +m\up)\bigg{\lbrace}\Big{[}A,_{s}^{\!\!(\text{on})\pm}
+\frac{1}{2}(\bp\ua+\bp\uu)
B,_{s}^{\!\!(\text{on})\pm}\Big{]}
\Big{[} 2(p\us \cdot k) \frac{p_{a\lambda}}{p\ua\cdot k}-2p_{s\lambda}\Big{]}\nn\\
&+\Big{[}A,_{t}^{\!\!(\text{on})\pm}+\frac{1}{2}(\bp\ua+\bp\uu)B,_{t}^{\!\!(\text{on})\pm}\Big{]}
2(p\ut \cdot l\ud)
\Big{[}
\frac{p_{a\lambda}}{p\ua \cdot k}-\frac{p_{1\lambda}}{p\uu \cdot k}\Big{]}\nn\\
&+B^{(\text{on})\pm}\Big{[}\frac{1}{2}\bk \Big{(}\frac{p_{a\lambda}}{p\ua \cdot k}+\frac{p_{1\lambda}}{p\uu \cdot k}\Big{)}-\gamma_{\lambda}\Big{]}\bigg{\rbrace}(\bp\ub+m\up )
\,,
\\
\nn\\
\label{3.54}
\widehat{\cN}\ul^{(c+d+e2)(0)\pm}=&\;
e(\bp\ud+m\up)
\Big{[}A^{(\text{on})\pm}+\frac{1}{2}(\bp\ua +\bp\uu)
       B^{(\text{on})\pm}\Big {]}(\bp\ub +m\up)\;
\omega\Big{[}
-\frac{p_{b\lambda}}{p\ub \cdot k}+\frac{p_{2\lambda}}{p\ud \cdot k}\Big{]}\,, 
\\
\nn\\
\label{3.55}
\widehat{\cN}\ul^{(c+d+e2)(1)\pm}=&\;
e(\bp\ud+m\up)
\Big{[}A^{(\text{on})\pm}+\frac{1}{2}(\bp\ua +\bp\uu )B^{(\text{on})\pm}\Big {]}(\bp\ub +m\up)
\frac{1}{(p\ud\cdot k)\2}\Big{[}p_{2\lambda} (l\ud \cdot k )-l_{2\lambda}(p\ud \cdot k)\Big{]}\nn\\
&+e(-\!\not{l}\ud)
\Big{[}A^{(\text{on})\pm}+\frac{1}{2}(\bp\ua +\bp\uu )B^{(\text{on})\pm}\Big {]}(\bp\ub +m\up)
\Big{[}
-\frac{p_{b\lambda}}{p\ub \cdot k}+\frac{p_{2\lambda}}{p\ud \cdot k}\Big{]}
\nn\\
&+e(\bp\ud +m\up)
\frac{1}{2}(-\!\not{l}\uu) B^{(\text{on})\pm}(\bp\ub +m\up)\Big{[}
-\frac{p_{b\lambda}}{p\ub \cdot k}+\frac{p_{2\lambda}}{p\ud \cdot k}\Big{]}
\nn\\
&+e(\bp\ud +m\up)
\Big{[}A,_{s}^{\!\!(\text{on})\pm}+\frac{1}{2}(\bp\ua+\bp\uu)B,_{s}^{\!\!(\text{on})\pm}\Big{]}(\bp\ub +m\up)
\Big{[}2(p\us\cdot k)\frac{p_{b\lambda}}{p\ub \cdot k}-2p_{s \lambda}\Big{]}
\nn\\
&+e(\bp\ud +m\up)\Big{[}A,_{t}^{\!\!(\text{on})\pm}+\frac{1}{2}(\bp\ua+\bp\uu)B,_{t}^{\!\!(\text{on})\pm}\Big{]}(\bp\ub +m\up)
2(p\ut \cdot l\uu)\Big{[}-\frac{p_{b\lambda}}{p\ub \cdot k}+\frac{p_{2\lambda}}{p\ud \cdot k}\Big{]}\nn \allowdisplaybreaks\\
&+e(\bp\ud +m\up)
\Big{[}A^ {(\text{on})\pm}+\frac{1}{2}(\bp\ua+\bp\uu)B^ {(\text{on})\pm}\Big{]}
(k\ul -\bk \gamma\ul)(\bp\ub +m\up)\frac{1}{(-2p\ub\cdot k)}\nn \\
&-e\frac{1}{(2p\ud\cdot k)}(\bp\ud +m\up)(k\ul -\gamma\ul \bk)
\Big{[}A^ {(\text{on})\pm}+\frac{1}{2}(\bp\ua+\bp\uu)B^ {(\text{on})\pm}\Big{]}(\bp\ub +m\up)\nn\\
&+e(\bp\ud + m\up)\Big{[}A^ {(\text{on})\pm}+\frac{1}{2}(\bp\ua+\bp\uu)B^ {(\text{on})\pm}\Big{]}
\Big{[}
m\up(k\ul-\bk\gamma\ul)+\big{(}p_{b\lambda}\bk -(p\ub\cdot k )\gamma\ul\big{)}\Big{]}(\bp\ub+m\up) \nn\\
&\quad \times 
\frac{F\ud(0)}{m\up}\frac{1}{(-2p\ub\cdot k)}\nn\\
&-e\frac{F\ud(0)}{m\up}\frac{1}{(2p\ud\cdot k)}(\bp\ud + m\up)\Big{[} m\up (k\ul -\gamma\ul \bk)+\big{(}p_{2\lambda}\bk -(
p\ud \cdot k)\gamma\ul \big{)}\Big{]}\nn\\
&\quad \times \Big{[}A^ {(\text{on})\pm}+\frac{1}{2}(\bp\ua+\bp\uu)B^ {(\text{on})\pm}\Big{]}(\bp\ub +m\up)\,.
\end{align}
\end{widetext}

Finally we emphasize that the $\omega^{-1}$ terms
from \eqref{3.49}, \eqref{3.50},
\eqref{3.52}, and \eqref{3.54} are, of course, of the standard form as given in \cite{Low:1958sn,Weinberg:1965nx}. 
But our complete expressions to order $\omega^{-1}$ and $\omega^{0}$,
\eqref{3.42}--\eqref{3.47}, \eqref{3.49}--\eqref{3.55}
are, to our knowledge, new.

\section{Cross sections for $\ppm p \to \ppm p \gamma $ \\for $\omega\to 0$}
\label{sec:4}

In Sec.~\ref{sec:3} we have discussed the matrix amplitudes
\begin{equation}
\label{4.1}
\cN\ul^{(\ppm p\to \ppm p \gamma)}
\equiv
\cN\ul\hpm(p'\uu , p'\ud  , k ,  p\ua ,  p\ub)\,;
\end{equation}
see \eqref{3.16}, \eqref{3.42}, and \eqref{3.47}.
We consider now the reactions (cf. \eqref{3.1})
\bel{4.1a}
\ppm\,(p\ua)+p\,(p\ub , \lambda\ub )\to \ppm\,(p'\uu)+p\,(p'\ud , \lambda'\ud) +\gamma\,(k,\varepsilon)
\ee
with real photon emission. 
We have then from \er{3.5} and \er{3.17} with $k\2 =0$
\bal{4.2}
&\braket{\ppm (p'\uu), \, p(p'\ud , \lambda'\ud) ,\, \gamma (k,\varepsilon)|{\mathcal T}|\ppm (p\ua),\, p(p\ub , \lambda\ub )}\nn\\
&\quad =\varepsilon\ul^{*}\frac{1}{(2m\up)\2}
\bar{u}(p'\ud , \lambda'\ud)
\cN\hpm\ul(p'\uu , p'\ud  , k , p\ua ,  p\ub) u(p\ub , \lambda\ub)\,.
\end{align}
The differential cross sections for unpolarized protons in the initial state and no observation of the proton and photon polarizations 
in the final state are then
\bal{4.3}
&\dv \sigma (\ppm p\to \ppm p \gamma) \nn\\
&\quad =\frac{1}{2w(s,m\up\2 , m\upp\2)}
\frac{\dv^{3}k}{(2\pi)^{3}2k^{0}}\,
\frac{\dv^{3}p'\uu}{(2\pi)^{3}2p\uu^{\prime 0}}\,
\frac{\dv^{3}p'\ud}{(2\pi)^{3}2p\ud^{\prime 0}}\nn\\
&\qquad \times (2\pi)^{4}\delta^{(4)}(p'\uu+p'\ud+k-p\ua-p\ub)\nn\\
&\qquad \times \frac{1}{(2m\up)\2}\frac{1}{2}(-g^{\lambda\mu})\;
\tr \big{[}\cN\ul\hpm\overline{\cN_{\mu}\hpm}\, 
\big{]}\,,
\end{align}
where
\bel{4.4}
w(x,y,z)=[x\2+y\2+z\2-2xy-2yz-2xz]^{1/2}\,.
\ee

Now we go to the overall c.m. system where we have
\bal{4.5}
p\ua^{0}+p\ub^{0}&=p\uu^{\prime 0}+p\ud^{\prime 0}+\omega=\sqrt{s}\,,\nn\\
\vec{p}\ua+\vec{p}\ub&=\vec{p}^{\, \prime}\uu+\vec{p}^{\, \prime}\ud+\vec{k}=0\,,\nn\\
 p\uu^{ \prime 0}&=\sqrt{|\vec{p}^{\, \prime}\uu |\2 + m\upp\2}\,, \nn\\
 p\ud^{ \prime 0}&=\sqrt{|\vec{p}^{\, \prime}\ud |\2 + m\up\2}\nn\\
 &=\Big{[}|\vec{p}^{\, \prime}\uu |\2 +2|\vec{p}^{\, \prime}\uu|\omega (\widehat{p}^{\, \prime}\uu \cdot \widehat{k})+\omega\2+m\up\2\Big{]}^{1/2}\,.
 \end{align}
Here and in the following we set
 \bal{4.6}
&\widehat{p}\ua=\vec{p}\ua / |\vec{p}\ua |,\quad \widehat{p}^{\, \prime}\uu=\vec{p}^{\, \prime}\uu/ |\vec{p}^{\, \prime}\uu|,\quad \widehat{k}=\vec{k}/|\vec{k}| \,, \nn\\
&\omega =k^{0}=|\vec{k}|\,.
\end{align}
For fixed values of $\sqrt{s}$, $\widehat{p}\ua$, $\widehat{p}^{\, \prime}\uu$, $\omega$, and $\widehat{k}$, which we choose as independent variables, we can determine from \er{4.5} $|\vec{p}^{\, \prime}\uu|$, $\vec{p}^{\, \prime}\ud$, $p\uu^{\prime 0}$, and $p\ud^{\prime 0}$. 
The result for $|\vec{p}^{\, \prime}\uu|$ is
 \bal{4.7}
 |\vec{p}^{\, \prime}\uu|=&\frac{1}{2\big{[}(\qs-\omega)\2-\omega\2(\wpp\cdot \wik )\2\big{]}}\nn\\
& \times \bigg{\lbrace} \Big{[}\Big{(}(\qs-\omega)\2-(\omega\2+m\up\2-m\upp\2)\Big{)}\2\omega\2(\wpp\uu\cdot \wik)\2
\nn\\
&\quad +\Big{(}(\qs -\omega)\2-\omega\2(\wpp\uu\cdot\wik)\2\Big{)}
\nn\\
&\quad \times \Big{(}(\qs -\omega)^{4}
-2(\qs-\omega)\2(\omega\2+m\up\2+m\upp\2)\nn\\
&\quad +(\omega\2+m\up\2-m\upp\2)\2\Big{)}
 \Big{]}^{1/2}\nn\\
&\quad -\Big{[}(\qs-\omega)\2-(\omega\2+m\up\2-m\upp\2)\Big{]}\omega(\wpp\uu\cdot\wik)
 \bigg{\rbrace}\,.
 \end{align}
From \er{4.3} we get the following differential cross section with respect to $\omega$ and the solid angles of $\wik$ and $\wpp\uu$:
 \bal{4.8}
&\dv\sigma(\ppm p\to \ppm p\gamma)=\frac{1}{2^{4}(2\pi)^{5}w(s,m\up\2 ,m\upp\2)}\nn\\
& \quad \times \omega \, \dv\omega \, \dv \Omega_{\wik} \, \dv \Omega_{\wpp\uu} \, 
\mathcal{J}(s,\omega , \wpp\uu\cdot \wik)\,
\mathcal{K}\hpm (s,\omega,\widehat{p}\ua, \wpp\uu, \wik)
\,,
 \end{align}
where
 \begin{align}
 \label{4.9}
&\mathcal{K}\hpm (s,\omega,\widehat{p}\ua , \wpp\uu , \wik)
 =
 \frac{1}{(2m\up)\2}\frac{1}{2}(-g^{\lambda\mu})
 \;\tr(\cN\ul\hpm\overline{\cN_{\mu}\hpm})\,,
\\
\label{4.10} 
& \mathcal{J} (s,\omega, \wpp\uu \cdot\wik)
= |\vpp\uu|\2\Big{[}p^{\prime 0}_{2}|\vpp\uu|+
 p^{\prime 0}_{1}\big{(}|\vpp\uu|+\omega(\wpp\uu\cdot \wik)\big{)}\Big{]}^{-1}\,.
\end{align}
So far all this is completely general. 
Now we consider~$\mathcal{K}\hpm$ for $\omega\to 0$.
From our results \er{3.49}--\er{3.55} 
we know the expansions of $\cN\ul\hpm$ for $\omega\to 0$, that is, 
we know the terms of order $\omega^{-1}$ and $\omega^{0}$. 
In evaluating $\mathcal{K}\hpm (s,\omega,\widehat{p}\ua, \wpp\uu, \wik)$ 
we keep $\wpp\uu$ fixed. 
Thus, we must use the expansions for $\cN\ul\hpm$ setting $\wpp\uu=\wip\uu$
which means we choose $\vec{l}_{1\perp}=0$ in \er{3.12} and \er{3.13} 
and use the resulting four-vectors $l\uu$ and $l\ud$ 
in \er{3.53} and \er{3.55}. 
With this we get
 \bal{4.11}
\cN\ul\hpm(p'\uu , p'\ud , k, p\ua , p\ub)
=& \frac{1}{\omega}\widehat{\cN}^{(0)\pm}
(s,\widehat{p}\ua, \wpp\uu, \wik) \nn\\
&+ \widehat{\cN}^{(1)\pm} (s,\widehat{p}\ua , \wpp\uu , \wik) + {\mathcal O}(\omega)\,,\\
 \label{4.12} 
 \mathcal{K}\hpm (s,\omega,\widehat{p}\ua , \wpp\uu , \wik)=&\frac{1}{\omega\2}\Big{\lbrace}\widehat{\mathcal{K}}^{(0)\pm} (s,\widehat{p}\ua , \wpp\uu , \wik)\nn\\
&+ \omega \widehat{\mathcal{K}}^{(1)\pm}
(s,\widehat{p}\ua, \wpp\uu, \wik)
+ {\mathcal O}(\omega\2)\Big{\rbrace}\,,
\end{align} 
where
 \bal{4.13}
\widehat{\mathcal{K}}^{(0)} (s,\widehat{p}\ua , \wpp\uu , \wik)=&\frac{1}{(2m\up)\2}\frac{1}{2}(-g^{\lambda\mu})\;
\tr\Big{[}\widehat{\cN}\ul^{(0)\pm}\overline{\widehat{\cN}_{\mu}^{(0)\pm}}\,\Big{]}\,,\\
 \label{4.14}
\widehat{\mathcal{K}}^{(1)} (s,\widehat{p}\ua , \wpp\uu , \wik)
=&\frac{1}{(2m\up)\2}\frac{1}{2}(-g^{\lambda\mu})\nn\\
 &\times \tr \Big{[}\widehat{\cN}\ul^{(0)\pm}\overline{\widehat{\cN}_{\mu}^{(1)\pm}}+\widehat{\cN}\ul^{(1)\pm}\overline{\widehat{\cN}_{\mu}^{(0)\pm}}\,\Big{]}\,.
 \end{align}
Inserting \er{4.12}--\er{4.14} in \er{4.8} we obtain
\bal{4.15}
&\frac{\omega\, \dv\sigma(\ppm p\to \ppm p\gamma)}{\dv\omega \, \dv \Omega_{\wik} \, \dv \Omega_{\wpp\uu} }=\frac{1}{2^{4}(2\pi)^{5}
w(s,m\up\2 ,m\upp\2)}\nn\\
&\quad \times \mathcal{J} (s,\omega , \wpp\uu \cdot\wik)\Big{\lbrace}\widehat{\mathcal{K}}^{(0)\pm} (s,\widehat{p}\ua , \wpp\uu , \wik) \nn\\
&\qquad + \omega\, \widehat{\mathcal{K}}^{(1)\pm}
(s,\widehat{p}\ua , \wpp\uu , \wik)
+ {\mathcal O}(\omega\2)\Big{\rbrace}\,.
 \end{align}
This is our final result for the cross section for unpolarized initial protons and no observation of the proton and photon polarizations in the final state. The cross section $\omega \,\dv\sigma(\ppm p\to \ppm p\gamma)/(\dv \omega\, \dv\Omega_{\widehat{k}}\, \dv\Omega_{\widehat{p}'_{1}})$
is given by a known phase-space factor times
the $\widehat{\cK}$ term 
which is expanded in powers of $\omega$ for $\omega\to 0$. The terms of order $\omega^{0}$ and $\omega^{1}$ are completely fixed by the on-shell amplitudes of $\ppm p \to \ppm p$. Eq.~\er{4.15} contains all information we can get from the soft-photon theorem derived in Sect.~\ref{sec:3} if polarizations are not considered. For the case of polarized protons in the initial state and/or polarization observations of the proton and the photon in the final state the soft-photon theorem of Sec.~\ref{sec:3} provides, of course, further information.

%========================================
\endgroup
%========================================
 
\section{Conclusions}
\label{sec:5}

In this article we have considered soft-photon production in 
$\ppm$-proton elastic scattering. That is, we studied the reactions 
$\ppm p\to \ppm p$ and $\ppm p\to \ppm p\gamma$ 
for the photon energy $\omega \to 0$. 
The terms of order $\omega^{-1}$ in the amplitudes for $\ppm p\to \ppm p\gamma$ are well known \cite{Low:1958sn,Weinberg:1965nx} 
and we reproduced them in our calculations. 
The evaluation of the terms of order $\omega^{0}$ is the central topic of our paper and our result there is new, as far as we know. 
The main issues which we had to consider and the main results which we obtained were the following.
\begin{itemize}
%-----------------------------------
\item[(i)] 
%-----------------------------------
In going from
\bal{5.1}
\quad \ppm\,(p\ua)+p\,(p\ub) &\to \ppm\,(p\uu)+p\,(p\ud)
\intertext{to}
\label{5.2}
\quad \ppm\,(p\ua)+p\,(p\ub) &\to \ppm\,(p'\uu)+p\,(p'\ud)+\gamma(k)
\end{align}
for $k\neq 0$ a corresponding change of the momenta $(p\uu , p\ud)\to (p'\uu , p'\ud)$ has to be taken into account.
Keeping $p'\uu=p\uu$ and $p'\ud =p\ud$ 
for $k\neq 0$ violates energy-momentum conservation; 
see \mbox{\er{3.6}--\er{3.14}} and Fig.~\ref{fig:2}.
%-----------------------------------
\item[(ii)]
%-----------------------------------
In the diagrams for \er{5.2} we have to deal with the \underline{off-shell} amplitudes for $\ppm p \to \ppm p$; 
see Fig.~\ref{fig:3}. Whereas the on-shell amplitude 
for $\pi^{-}p\to \pi^{-}p$ scattering is described 
by two invariant amplitudes, 
$A^{-}(s,t)$ and $B^{-}(s,t)$, see \er{2.23}--\er{2.25}, the off-shell amplitude for $\pi^{-}p \to \pi^{-}p$ contains eight invariant amplitudes; 
see \er{2.16}--\er{2.19}. 
The same is, of course, true for $\pi^{+}p\to \pi^{+}p$. In the calculation of $\ppm p \to \ppm p \gamma$ all these eight off-shell invariant amplitudes have to be taken into account. 
It came as a surprise to us that the final result for the terms of order $\omega^{-1}$ and $\omega^{0}$ in the $\ppm p \to \ppm p \gamma$ amplitudes can be completely expressed in terms of the on-shell amplitudes $A\hpm (s,t)$, $B\hpm(s,t)$, 
and their partial derivatives with respect to $s$ and $t$;
see \er{3.42}--\er{3.48}.
%-----------------------------------
\item[(iii)] 
%-----------------------------------
Our results for the amplitudes of $\ppm p \to \ppm p \gamma$ were then used to discuss the behavior of the corresponding cross section for $\omega \to 0$; see \er{4.15}.
%-----------------------------------
\item[(iv)]
%-----------------------------------
The formulas 
(\ref{3.42})--(\ref{3.47})
which we give for the amplitudes in the limit 
$\omega \to 0$ are valid both for real photons ($k\2 = 0$) 
and for virtual photons ($k\2 \neq 0$).
%-----------------------------------
\item[(v)] 
%-----------------------------------
We emphasize that our results represent a strict theorem in the framework of QCD which applies at all c.m. energies~$\sqrt{s}$. 
Thus, the theorem can be tested experimentally both at low values of $\sqrt{s}$, available for instance 
in the HADES experiment
\cite{Rathod_PhD_thesis} 
at GSI, Darmstadt, and at high energy pion-proton scattering. 
\end{itemize}

We end with an outlook. Using the methods which we have developed 
for $\pi p$ scattering with soft-photon production one could also tackle proton-proton scattering with soft-photon production. We see no problem of principle there, except that the calculations will be very long and complex. Will there be at the end a theorem analogous to the $\pi p$ scattering case? To answer this question will need much more work. 

\begin{acknowledgments}
The authors thank C. Ewerz for reading the main part of the manuscript and for useful suggestions.
This work was partially supported by
the Polish National Science Centre under Grant
No. 2018/31/B/ST2/03537.
\end{acknowledgments}

% Specify following sections are appendices. Use \appendix* if there only one appendix.
%\appendix
%\section{}

%%\begin{appendix}
\appendix

%\appendix % label section numbers alphabetically: "A", "B", etc
%\counterwithin*{equation}{section} 
% reset 'equation' counter whenever '\section' is executed
\renewcommand\theequation{\thesection\arabic{equation}} 
% how to display the equation "number" - ohne "dot", also C1 statt C.1

\section{Kinematic Relations}
\label{app:A}

%========================================
\begingroup
\allowdisplaybreaks
%========================================

In this appendix we collect useful formulas which we need in the main chapters.

For the momenta $\tilde{p}\ua$, $\tilde{p}\ub$, $\tilde{p}\uu$, $\tilde{p}\ud$ of the general off-shell reaction \er{2.1} we note in addition to \er{2.7}--\er{2.11} the following relations
\begin{align}
\label{A1}
\tilde{p}\ua &=\frac{1}{2}(\tilde{p}_{s}+\tilde{p}_{t}+\tilde{p}_{u})\,,\nn\\
\tilde{p}\ub &=\frac{1}{2}(\tilde{p}_{s}-\tilde{p}_{t}-\tilde{p}_{u})\,,\nn\\
\tilde{p}\uu &=\frac{1}{2}(\tilde{p}_{s}-\tilde{p}_{t}+\tilde{p}_{u})\,,\nn\\
\tilde{p}\ud &=\frac{1}{2}(\tilde{p}_{s}+\tilde{p}_{t}-\tilde{p}_{u})\,;\\
\label{A2}\nn\\
\tilde{p}_{s}+\tilde{p}_{u} &=\tilde{p}\ua +\tilde{p}\uu \,, \nn\\
\tilde{p}_{s}-\tilde{p}_{u} &=\tilde{p}\ub +\tilde{p}\ud \,;\\
\nn\\
\label{A3}
\tilde{p}\ua \cdot\tilde{p}\ub &=\frac{1}{2}(\tilde{s}-m\ua\2-m\ub\2)\ ,\nn\\
\tilde{p}\uu \cdot\tilde{p}\ud &=\frac{1}{2}(\tilde{s}-m\uu\2-m\ud\2)\ ,\nn\\
\tilde{p}\ua \cdot\tilde{p}\uu &=\frac{1}{2}(-\tilde{t}+m\ua\2+m\uu\2)\ ,\nn\\
\tilde{p}\ub \cdot\tilde{p}\ud &=\frac{1}{2}(-\tilde{t}+m\ub\2+m\ud\2)\ ,\nn\\
\tilde{p}\ua \cdot\tilde{p}\ud &=\frac{1}{2}(-\tilde{u}+m\ua\2+m\ud\2)\ ,\nn\\
&=\frac{1}{2}(\tilde{s}+\tilde{t}-m\ub\2-m\uu\2)\,,\nn\\
\tilde{p}\uu \cdot\tilde{p}\ub &=\frac{1}{2}(-\tilde{u}+m\ub\2+m\uu\2)\ ,\nn\\
&=\frac{1}{2}(\tilde{s}+\tilde{t}-m\ua\2-m\ud\2)\,.
\end{align}
For three four vectors $a$, $b$, $c$ we have
\begin{align}
\label{A4}
i\gamma_{\mu}\gamma_{5}\varepsilon^{\mu\nu\rho\sigma}a_{\nu}b_{\rho}c_{\sigma}&=
\frac{1}{6}(
 \slash{a}\slash{b}\slash{c}\,
+\slash{c}\slash{a}\slash{b}\,
+\slash{b}\slash{c}\slash{a}\nn\\
&\quad -\slash{b}\slash{a}\slash{c}\,
-\slash{a}\slash{c}\slash{b}\,
-\slash{c}\slash{b}\slash{a})\,.
\end{align}

For on-shell momenta $p\ua$, $p\ub$, $p\uu$, $p\ud$ in the reactions~\eqref{2.2} we have for the masses \eqref{2.12} and we get the following relations with Dirac spinors 
$u(p\ub)$ and $\bar{u}(p\ud)$:
\begin{align}
\label{A5}
&\bar{u}(p\ud) \!\! \slash{p}_{s}u(p\ub)
=\bar{u}(p\ud)\Big{[}m_{p}+\frac{1}{2}(\slash{p}\ua+\slash{p}\uu)\Big{]}u(p\ub)\,,\\
\label{A6}
&\bar{u}(p\ud) \!\! \slash{p}_{u}u(p\ub)
=\bar{u}(p\ud)\Big{[}-m_{p}+\frac{1}{2}(\slash{p}\ua+\slash{p}\uu)\Big{]}u(p\ub)\,,\\
\label{A7}
&\bar{u}(p\ud) i\sigma_{\mu\nu}p_{s}{}^{\mu}p_{t}{}^{\nu}u(p\ub)\nn\\
&\quad
=\bar{u}(p\ud)\Big{[}-s+m_{p}\2+m_{\pi}\2+m_{p}(\slash{p}\ua+\slash{p}\uu)\Big{]}u(p\ub)\,,\\
\label{A8}
&\bar{u}(p\ud) i\sigma_{\mu\nu}p_{t}{}^{\mu}p_{u}{}^{\nu}u(p\ub)\nn\\
&\quad
=\bar{u}(p\ud)\Big{[}s+t-m_{p}\2-m_{\pi}\2-m_{p}
(\slash{p}\ua+\slash{p}\uu)\Big{]}u(p\ub)\,,\\
\label{A9}
&\bar{u}(p\ud)i\gamma_{\mu}\gamma_{5}\varepsilon^{\mu\nu\rho\sigma}p_{s\nu}p_{t\rho}p_{u\sigma}u(p\ub)\nn\\
&\quad
=\bar{u}(p\ud)\Big{[}-m_{p}(2s+t-2m_{p}\2-2m_{\pi}\2)\nn\\
&\quad \quad +\frac{1}{2}(\slash{p}\ua+\slash{p}\uu)(4m_{p}\2-t)
\Big{]}u(p\ub)\,.
\end{align}

Useful relations for the momenta of the on-shell reaction
\eqref{2.2}, $\pi p \to \pi p$, are as follows:
\begin{align}
\label{A10}
&
-\frac{1}{2}(p_{a} + p_{1}, p_{b} + p_{2})
-(p_{b} \cdot p_{2})+m_{p}^{2} = -s + m_{p}^{2} + m_{\pi}^{2}\,,\nn\\
&
\frac{1}{2}(p_{a} + p_{1}, p_{b} + p_{2})
-(p_{b} \cdot p_{2})+m_{p}^{2} = s + t - m_{p}^{2} - m_{\pi}^{2}\,,\nn\\
&
(p_{a} + p_{1}, p_{b} + p_{2}) = 2s + t - 2m_{p}^{2} - 2m_{\pi}^{2}\,,\nn\\
&
m_{p}^{2} + (p_{b} \cdot p_{2}) 
= \frac{1}{2} (4 m_{p}^{2} - t)\,.
\end{align}

\section{Propagators and Vertices}
\label{app:B}

In this appendix we recall the symmetry relations for fields, propagators and vertices which are valid in QCD. This is standard material from QFT; 
see, e.g., \cite{Bjorken:1965} 
and \cite{Nachtmann:1990ta,Weinberg:1995_I}. 
Then we use these relations and the generalized Ward identities to derive the general forms of propagators, photon-vertex functions and of their products as we need them for our calculations.

\subsection{$P$, $C$, and $T$ relations for fields and the general forms of the pion and proton propagators}
\label{subsec:B1}

We start by considering the renormalized field operators for the proton and for the pions:
\begin{equation}
\label{B1}
\psi(\vec{x},t)\;,\quad \varphi^{\pm}(\vec{x},t)\;,\quad \varphi^{0}(\vec{x},t)\ .
\end{equation}
In QCD these fields are not the fundamental ones which are the quark and gluon fields. But so called interpolating local fields \eqref{B1} with the correct quantum numbers exist and can easily be constructed in the QCD framework. These constructions are not unique but this is no problem for us. For the pions interpolating fields are obtained from the divergence of the axial currents with suitable normalization factors. For the proton such fields have, for instance, be given in \cite{Chung:1981wm,Ioffe:1981kw}.

From the fields in \eqref{subsec:B1}
$\psi$ annihilates protons and creates antiprotons, 
$\varphi^{\pm}$ annihilate $\pi^{\pm}$ and create $\pi^{\mp}$, $\varphi^{0}$ annihilates and creates $\pi^{0}$. 
We work in QCD where parity ($P$), charge conjugation ($C$),
and time reversal ($T$) are good symmetries. 
We also consider $\Theta=CPT$ which always is a good symmetry.
With the corresponding transformations: $U(P)$, $U(C)$ unitary, $V(T), V(\Theta)$ antiunitary, we have the following relations
\begin{align}
\label{B2}
&U(P)\psi(\vec{x},t)U^{-1}(P)=\gamma_{0}\psi(-\vec{x},t)\,,\nn\\
&U(P)\varphi^{\pm}(\vec{x},t)U^{-1}(P)=-\varphi^{\pm}(-\vec{x},t)\,,\nn\\
&U(P)\varphi^{0}(\vec{x},t)U^{-1}(P)=-\varphi^{0}(-\vec{x},t)\,, \\
\nn\\
\label{B3}
&U(C)\psi(\vec{x},t)U^{-1}(C)=S(C)\overline{\psi}^{\top}(\vec{x},t)\,,\nn\\
&U(C)\varphi^{\pm}(\vec{x},t)U^{-1}(C)=\varphi^{\mp}(\vec{x},t)\,,\nn\\
&U(C)\varphi^{0}(\vec{x},t)U^{-1}(C)=\varphi^{0}(\vec{x},t)\,,
\\
\nn\\
\label{B4}
&\big{(}V(T)\psi(\vec{x},t)V^{-1}(T)\big{)}^{\dagger}=S(T)\overline{\psi}^{\top}(\vec{x},-t)\,,\nn\\
&\big{(}V(T)\varphi^{\pm}(\vec{x},t)V^{-1}(T)\big{)}^{\dagger}=-\varphi^{\mp}(\vec{x},-t)\,,\nn\\
&\big{(}V(T)\varphi^{0}(\vec{x},t)V^{-1}(T)\big{)}^{\dagger}=-\varphi^{0}(\vec{x},-t)\,,
\\
\nn\\
\label{B4a}
&\big{(}V(\Theta)\psi(x)V^{-1}(\Theta)\big{)}^{\dagger}=-\gamma_{5}\psi(-x)\,,\nn\\
&\big{(}V(\Theta)\varphi^{\pm}(x)V^{-1}(\Theta)\big{)}^{\dagger}=\varphi^{\pm}(-x)\,,\nn\\
&\big{(}V(\Theta)\varphi^{0}(x)V^{-1}(\Theta)\big{)}^{\dagger}=\varphi^{0}(-x)\,.
\end{align}
Here $S(C)$ and $S(T)$ are given by
\begin{align}
\label{B5}
S(C)=i\gamma\2\gamma^{0}\,, \qquad
S(T)=i\gamma\2\gamma_{5}\,.
\end{align}
All these are standard relations of QFT; 
see for instance \cite{Nachtmann:1990ta,Weinberg:1995_I}. 
For the electromagnetic current operator 
$\mathcal{J}^{\lambda}(x)$ we have
in QCD with $q(x)$ the quark fields
and $Q_{q}$ their charges in units of the proton charge 
$e=\sqrt{4\pi \alpha_{\rm em}}$
\bel{B6}
\cJ^{\lambda}(x)=e\, \sum_{q} Q_{q}\,\bar{q}(x)\,\gamma^{\lambda}\,q(x) 
\,.
\ee
The current is conserved
\bel{B6a}
\frac{\partial}{\partial x^{\lambda}}\cJ^{\lambda}(x)=0\,,
\ee
and the transformation properties of this current operator are
%======================================
\endgroup
%======================================
\bal{B7}
&U(P)\cJ\hl (\vec{x},t)U^{-1}(P) = \mathcal{P}\hl{}_{\lambda'} \cJ^{\lambda'}(-\vec{x},t)\,,\nn\\
&U(C)\cJ\hl (\vec{x},t)U^{-1}(C) = - \cJ^{\lambda}(\vec{x},t)\,,\nn\\
&\big{(}V(T)\cJ\hl (\vec{x},t)V^{-1}(T)\big{)}^{\dagger} = \mathcal{P}\hl{}_{\lambda'} \cJ^{\lambda'}(\vec{x},-t)\,,\nn\\
&\big{(}V(\Theta)\cJ\hl (x)V^{-1}(\Theta)\big{)}^{\dagger} = - \cJ^{\lambda}(-x)\,,
\end{align}
where
\bel{B8}
(\mathcal{P}\hl{}_{\lambda'}) = \text{diag}(1,\, -1,\, -1, \, -1) \,.
\ee

Our next step is to discuss the general structure of the pion and proton propagators following from the above 
$P$, $C$, $T$, and $\Theta$ relations. 
From Poincar{\'e} and $CPT$ invariance one finds that the propagators for $\pi\hm$ and $\pi\hp$ are the same and we have
\bal{B9}
i\int\frac{\dv^{4}p}{(2\pi)^{4}} e^{-ip(x-y)}\Delta_{F}(p\2 )
&
= \braket{0|T\left( \varphi\hm (x)\,\varphi\hp (y) \right)|0}\nn\\
&
= \braket{0|T\left( \varphi\hp (x)\,\varphi\hm (y) \right)|0}
\,.
\end{align}
Here $T$ denotes the time-ordered product. 
The $P$, $C$, and $T$ relations following for the propagator from \er{B2}--\er{B4} are automatically fulfilled by \er{B9}. 
The normalization conditions 
for the renormalized pion propagator are
\bal{B10}
&\Delta_{F}^{-1}(p\2 )\big{|}_{p\2 = m_{\pi}\2} =0\,,\nn\\
\frac{\partial}{\partial p\2}
&\Delta_{F}^{-1}(p\2 )\big{|}_{p\2 = m_{\pi}\2} =1\,.
\end{align}
In QCD we find from the Landau conditions
\cite{Landau:1959fi,Bjorken:1965}
that $\Delta_{F}^{-1}(p\2 )$ is analytic at 
$p\2 = m_{\pi}\2$ with the nearest singularity at $p\2=(3m\upp)\2$ where the cut on the real $p\2$~axis starts. Therefore, we can write the general form for  $\Delta_{F}^{-1}(p\2 )$ as
\bal{B11}
 \Delta_{F}^{-1}(p\2 )&=p\2 -m\upp\2
 +(p\2 -m\upp\2)\2 \, C(p\2 -m\upp\2)
 \end{align}
with $C(p\2 - m\upp\2)$ an analytic function
for $|p\2 - m\upp\2|<8m\upp\2$,
\bel{B11a}
C(p\2 -m\upp\2)=c_{0}+c_{1}(p\2 -m\upp\2)+\dots \ .
\ee
The self-energy function of the pion
is\\
$-(p\2 -m\upp\2)\2 \, C(p\2 -m\upp\2)$.
 
Now we turn to the proton propagator 
\bal{B12}
i\int\frac{\dv^{4}p}{(2\pi)^{4}} &e^{-ipx}S_{F}(p)
=\braket{0|T\left( \psi(x)\,\overline{\psi}(0) \right)|0}\,.
\end{align}
From the $P$-invariance relation for the proton field \er{B2} we get
\bal{B13}
\braket{0|T \left( \psi (\vec{x},t)\,\overline{\psi}(0) \right)|0}
&=
\gamma_{0}
\braket{0|T \left( \psi (-\vec{x},t)\,\overline{\psi}(0) \right)|0}\gamma_{0}\,,
\nn\\
S_{F}(\vec{p},p^{0})& =\gamma_{0}S_{F}(-\vec{p},p^{0})\gamma_{0}\,.
\end{align}
The $C$ and $T$ relations, 
\er{B3} and \er{B4}, give
\bel{B14}
S_{F}(p)=S(C)S^{\top}_{F}(-p)S^{-1}(C)\,,
\ee
and
\bel{B15}
S_{F}(\vec{p},p^{0})=S(T)S^{\top}_{F}(-\vec{p},p^{0})S^{-1}(T)\,,
\ee
respectively. The most general ansatz for $S_{F}^{-1}(p)$ 
compatible with \er{B13}--\er{B15} is
\bel{B16}
S_{F}^{-1}(p)=C_{0}(p\2)+C_{1}(p\2)\!\bp \,.
\ee
The functions $C_{0}(p\2)$ and $C_{1}(p\2)$ are analytic 
at $p\2 = m_{p}\2$ with the nearest singularity at 
\bel{B17}
p\2=(m\up +m\upp)\2 \,.
\ee
To discuss the normalization conditions 
for the proton propagator as it is usual for fermions 
(see, e.g., \cite{Weinberg:1995_I}) 
we write in \er{B16} 
\mbox{$p^2 = \slash{p}^2$}
and consider $S_{F}^{-1}$ as a function~of~$\bp$
\bel{B18}
S_{F}^{-1}(\bp)=C_{0}(\bp\2)+C_{1}(\bp\2) \!\bp \,.
\ee
The proton mass is given by the zero point of the matrix valued function $S_{F}^{-1}(\bp)$
\bel{B19}
S_{F}^{-1}(\bp)\big{|}_{\slash{\;p} = m\up}=0\,,
\ee
and the normalization condition is
\bel{B20}
\frac{\partial}{\partial\!\!\bp}
S_{F}^{-1}(\bp)\big{|}_{\slash{\;p} = m\up}=1\,.
\ee
From \er{B17}--\er{B20} 
we find that \mbox{$S_{F}^{-1}(\bp)$} 
must have the general form
\bal{B20a}
S_{F}^{-1}(\bp)&=\bp -m\up -\Sigma(\bp) \nn\\
&=
(\bp -m\up)\Big{[} 1+
(\bp -m\up)(a\up \!\bp +b\up m\up)\Big{]}\,,
\end{align}
where $\Sigma(\bp)$ is the proton self-energy function 
and $a\up$, $b\up$ are functions of $p\2-m\up\2$, 
analytic at $p\2 -m\up\2=0$. That is, we have
\bal{B20b}
a\up \equiv a(p\2-m\up\2)=& \,a_0+(p\2-m\up\2)a_{1}\nn\\
&+(p\2-m\up\2)\2 a\ud +\dots ,\nn\\
\nn\\
b\up \equiv b(p\2-m\up\2)=&\, b_0+(p\2-m\up\2)b_{1}\nn\\
&+(p\2-m\up\2)\2 b\ud +\dots ,
\end{align}
where the expansions are valid for
\bel{B20c}
|p\2 - m\up\2 |< (m\up + m\upp)\2 -m\up\2 = 2m\up m\upp + m\upp\2 \,;
\ee
see \er{B17}.

The last task is to calculate $S_{F}(p)$ from \er{B20a} and this gives
\bal{B21}
S_{F}(p)=&\frac{1}{\bp-m\up +i\varepsilon}\nn\\
&+\Big{\lbrace}-(\bp +m\up)\Big{[} a\up + (p\2-m\up\2)a\up\2\nn\\
&\qquad +m\up\2(a\up\2-b\up\2 )\Big{]} +m\up(a\up-b\up)
\Big{\rbrace}\nn\\
&\times \Big{\lbrace}1+2(a\up p\2 -b\up m\up\2 )+ (a\up p\2-b\up m\up\2 )\2\nn\\
&\qquad -p\2m\up\2 (a\up-b\up)\2\Big{\rbrace}^{-1}\,.
\end{align}

\subsection{Vertex functions and generalized Ward identity for pions}
\label{sec:B2}
We consider first the $\gamma \pi\pi$ vertex functions 
which are defined by
\begin{align}\label{B22}
&\braket{0|T \big{(}\varphi\hpm(y)\cJ\ul(x)\varphi\hmp(z)\big{)}|0}\nn\\
&\quad =\int\frac{\dv^{4}p'}{(2\pi)^{4}}\, \frac{\dv^{4}p}{(2\pi)^{4}}e^{-ip'(y-x)}e^{-ip(x-z)}\nn\\
&\qquad \times i\Delta_{F}(p^{\prime 2})\Big{[}-\Gamma\ul^{(\gamma\ppm\ppm)}(p', p)\Big{]}i\Delta_{F}(p^{2})\,.
\end{align}
Using the relations \er{B3} and \er{B4a} we get from the definition \er{B22} and from $\Theta$ and $C$ invariance
\bal{B23}
\Gamma\ul^{(\gamma\pi\hp\pi\hp)}(p,p')&=\phantom{-}\Gamma\ul^{(\gamma\pi\hm\pi\hm)}(-p',-p)\,,\\
\label{B24}
\Gamma\ul^{(\gamma\pi\hm\pi\hm)}(p',p)&=-\Gamma\ul^{(\gamma\pi\hp\pi\hp)}(p,p')\,,\\
\label{B25}
\Gamma\ul^{(\gamma\pi\hp\pi\hp)}(p',p)&=-\Gamma\ul^{(\gamma\pi\hm\pi\hm)}(p',p)\,,
\end{align}
respectively.
Therefore, we can set
\begin{align}
\label{B26}
\Gamma\ul^{(\gamma\pi\hm\pi\hm)}(p',p)
&=-\Gamma\ul^{(\gamma\pi\hp\pi\hp)}(p',p) \nn \\
&=e\, \widehat{\Gamma}\ul^{(\gamma\pi\pi)}(p',p)\,,
\end{align}
where $\widehat{\Gamma}\ul$ satisfies
\bel{B27}
\widehat{\Gamma}\ul^{(\gamma\pi\pi)}(p',p)=\widehat{\Gamma}\ul^{(\gamma\pi\pi)}(p,p')=-\widehat{\Gamma}\ul^{(\gamma\pi\pi)}(-p',-p)\,.
\ee
The most general ansatz for $\widehat{\Gamma}\ul$ is, therefore, 
\bal{B28}
&\widehat{\Gamma}\ul^{(\gamma\pi\pi)}(p',p)\nn\\
&\quad
=(p'+p) \ul A\Big{[}p^{\prime\, 2}-m\upp\2 ,\,p\2-m\upp\2 ,\, (p'-p)\2\Big{]}\nn\\
&\qquad +(p'-p)\ul B\Big{[}p^{\prime\, 2}-m\upp\2 ,\,p\2-m\upp\2 ,\, (p'-p)\2\Big{]}\,.
\end{align}
Here $A$ and $B$ are analytic functions for
\bal{B29}
|p^{\prime\, 2}-m\upp\2 |&<8m\upp\2\,,\nn\\
|p\2-m\upp\2|&<8m\upp\2\,,\nn\\
|(p' -p)\2|&<4m\upp\2 \,,
\end{align}
which satisfy:
\bal{B30}
&A\Big{[}p^{\prime\,  2}-m\upp\2 ,\,p\2-m\upp\2 ,\, (p'-p)\2\Big{]}\nn\\
&\quad =A\Big{[}p^{  2}-m\upp\2 ,\,p^{\prime\,  2}-m\upp\2 ,\, (p'-p)\2\Big{]}\,, \nn\\
\nn\\
&B\Big{[}p^{\prime\,  2}-m\upp\2 ,\,p\2-m\upp\2 ,\, (p'-p)\2\Big{]}\nn\\
&\quad =-B\Big{[}p^{  2}-m\upp\2 ,\,p^{\prime\,  2}-m\upp\2 ,\, (p'-p)\2\Big{]}\,.
\end{align}
Due to the analyticity properties of $A$ and $B$ \er{B29} we can write
\bal{B31}
&B\Big{[}p^{\prime\,  2}-m\upp\2 ,\,p\2-m\upp\2 ,\, (p'-p)\2\Big{]}\nn\\
&\quad =(p^{\prime\,  2}-p\2)\widetilde{B} \Big{[}p^{\prime\,  2}-m\upp\2 ,\,p^{  2}-m\upp\2 ,\, (p'-p)\2\Big{]}\,,
\end{align}
where $\widetilde{B}$ is symmetric under the exchange $p'\leftrightarrow p$. \\
The most general ansatz for 
$\widehat{\Gamma}\ul^{(\gamma\pi\pi)}$ \er{B28} 
reads then
\bal{B32}
&\widehat{\Gamma}\ul^{(\gamma\pi\pi)}(p',p)\nn\\
&=(p'+p)\ul A\Big{[}p^{\prime\,  2}-m\upp\2 ,\,p\2-m\upp\2 ,\, (p'-p)\2\Big{]}\nn\\
&\quad +(p'-p)\ul (p^{\prime\, 2}-p\2)\widetilde{B} \Big{[}p^{\prime\,  2}-m\upp\2 ,\,p^{  2}-m\upp\2 ,\, (p'-p)\2\Big{]}
\end{align}
with $A$ and $\widetilde{B}$ symmetric 
under $p'\leftrightarrow p$.

Now we study the consequences of the generalized Ward identity: 
\bel{B33}
(p'-p)^{\lambda}\, \widehat{\Gamma}\ul^{(\gamma\pi\pi)}(p',p)=
\Delta_{F}^{-1}(p^{\prime\, 2})-\Delta_{F}^{-1}(p\2)\,.
\ee
We can ask if \er{B33} holds for our local interpolating pion fields. 
The answer is yes. The proof of \er{B33} just relies on current conservation \er{B6a}, the locality of the pion fields, and their charge assignment. From these one finds the following commutation relations
\bal{B33a}
&\Big{[}\cJ^{0}(x),\, \varphi\hpm (y)\Big{]}\delta (x^{0}-y^{0})=0
\qquad \text{for }x\neq y\ ,\nn\\
&\Big{[}Q,\, \varphi\hpm (y)\Big{]}=\mp\,  e \, \varphi\hpm(y) \,,
\end{align}
where $Q$ is the charge operator
\bel{B33b}
Q=\int \dv^{3} x\,  \cJ^{0}(\vec{x}, x^{0})\,.
\ee
And with these relations one can prove the generalized Ward identity; see \cite{Takahashi:1957xn} 
and chapter~10.4 of \cite{Weinberg:1995_I}.
\begin{widetext}
Inserting now into \er{B33} Eqs.~\er{B11} and \er{B32} we get
\bal{B34}
&\Big{[}(p^{\prime\,  2}-m\upp\2 )-(p\2-m\upp\2)\Big{]}
A\Big{[}p^{\prime\,  2}-m\upp\2 ,\,p\2-m\upp\2 ,\, (p'-p)\2\Big{]}\nn\\
&\quad +(p'-p)\2\Big{[}(p^{\prime\,  2}-m\upp\2 )-(p\2-m\upp\2)\Big{]}
\widetilde{B} \Big{[}p^{\prime\,  2}-m\upp\2 ,\,p^{  2}-m\upp\2 ,\, (p'-p)\2\Big{]}\nn\\
& =(p^{\prime\,  2}-m\upp\2 )+(p^{\prime\,  2}-m\upp\2 )\2 \, C(p^{\prime\,  2}-m\upp\2 )
-(p\2-m\upp\2 )-(p\2-m\upp\2 )\2\,   
C(p\2-m\upp\2 )\,.
\end{align}
\end{widetext}
Considering $p^{\prime\,  2}=p\2 \rightarrow m\upp\2$ and then $p' =p$ we get
\bel{B35}
A(0,0,0)=1
\ee
which gives from \er{B32}
\bel{B36}
\widehat{\Gamma}\ul^{(\gamma pp)}(p',p)\big{|}
_{\mathop{^{p'=p}                  
_{p^{\prime 2} = p\2 = m\upp\2}}}
=2p\ul\,.
\ee
This is exactly the normalization condition for this vertex function.

Now we consider the case
\bel{B37}
p\2=m\upp\2\;,\quad p'=p-k\,,
\ee
which gives
\bel{B37a}
p'-p=-k\;,\quad p^{\prime\, 2}-m\upp\2 = -2p \cdot k+k\2\,.
\ee
We are interested in the quantity
\bel{B38}
\Delta_{F}\big{[}(p-k)\2\big{]}\, \widehat{\Gamma}\ul^{\gamma\pi\pi}(p-k,\, p)\,.
\ee
From \er{B11} and \er{B34} we get then
\bal{B39}
&A\Big{[}p^{\prime\,  2}-m\upp\2 ,\, 0 ,\, k\2\Big{]}+k\2\widetilde{B} \Big{[}p^{\prime\,  2}-m\upp\2 ,\,0,\, k\2\Big{]}\nn\\
&\quad =1+(p^{\prime\,  2}-m\upp\2 )\;  C (p^{\prime\,  2}-m\upp\2 )\nn\\
&\quad =\Delta_{F}^{-1}(p^{\prime\,  2})
\,(p^{\prime\,  2}-m\upp\2 +i\varepsilon)^{-1}\,,
\\
\nn\\
\label{B40}
&A\Big{[}p^{\prime\,  2}-m\upp\2 ,\, 0 ,\, k\2\Big{]}
=\Delta_{F}^{-1}(p^{\prime\,  2})(p^{\prime\,  2}-m\upp\2 +i\varepsilon)^{-1}\nn\\
&\qquad \qquad \qquad \qquad \quad \;-k\2\widetilde{B} \Big{[}p^{\prime\,  2}-m\upp\2 ,\,0,\, k\2\Big{]}\,.
\end{align}
Inserting this in \er{B32} and using \er{B37} we find finally
\bal{B41}
&\Delta_{F}\big{[}(p-k)\2\big{]}\, 
\widehat{\Gamma}\ul^{(\gamma \pi\pi)}(p-k,\, p)\big{|}_{p\2=m\upp\2}\nn\\
&\quad =\frac{(2p-k)\ul}{-2p\cdot k + k\2 +i\varepsilon}
-\big{[}(2p-k)\ul\,  k\2 -k\ul (2p\cdot k -k\2)\big{]}\nn\\
&\qquad \times \big{[}-2p\cdot k +k\2+i\varepsilon \big{]}^{-1}\nn\\
&\qquad \times \big{[} 1+(-2p\cdot k +k\2)\; C(-2p\cdot k + k\2)\big{]}^{-1}\nn\\
&\qquad \times
\widetilde{B}\big{[}-2p\cdot k +k\2 ,\, 0,\, k\2\big{]}\,.
\end{align}
We also consider the case
\bel{B42}
p^{\prime \, 2}=m\upp\2 \, ,\quad p=p'+k
\ee
which gives
\bel{B43}
p'-p=-k\, ,\quad p\2-m\upp\2=2p'\cdot k +k\2 \,.
\ee
Here we study
\bel{B44}
\widehat{\Gamma}\ul^{(\gamma \pi\pi)}(p',\, p'+k)
\;
\Delta_{F}\big{[}(p'+k)\2\big{]}\,.
\ee
We find, in a completely analogous way as shown above for the derivation of \er{B41} the following result:
\bal{B45}
&\widehat{\Gamma}\ul^{(\gamma \pi\pi)}(p', \, p'+k)
\;\Delta_{F}\big{[}(p'+k)\2\big{]}\nn\\
&\quad =\frac{(2p'+k)\ul}{2p'\cdot k + k\2 +i\varepsilon}
-\big{[}(2p'+k)\ul\,  k\2 -k\ul (2p'\cdot k +k\2)\big{]}\nn\\
&\qquad \times \big{[}2p'\cdot k +k\2+i\varepsilon \big{]}^{-1}\nn\\
&\qquad \times \big{[} 1+(2p'\cdot k +k\2)\; C(2p'\cdot k + k\2)\big{]}^{-1}\nn\\
&\qquad \times
\widetilde{B}(0,\, 2p'\cdot k +k\2 ,\, k\2)\ .
\end{align}

The results \er{B41} and \er{B45} are used 
in Sec.~\ref{sec:3} and Appendix~\ref{app:C}.

\subsection{Vertex function and generalized Ward identity for the proton}
\label{subsec:B3}

The defining equation for the $\gamma pp$ vertex function is
\bal{B46}
&\braket{0|T\big{(}\psi (y)\,\cJ_{\mu}(x)\, \overline{\psi}(z)\big{)}|0}\nn\\
%&\big{\langle} 0 \,|\, T\, \big{(}\psi (y)\,\cJ_{\mu}(x)\, \overline{\psi}(z)\big{)}|\, 0 \big{\rangle}\nn\\
&\quad =\int\frac{\dv^{4}p'}{(2\pi)^{4}}\, \frac{\dv^{4}p}{(2\pi)^{4}}e^{-ip'(y-x)}e^{-ip(x-z)}\nn\\
&\qquad \times iS_{F}(p')\Big{[}-\Gamma_{\mu}^{(\gamma pp)}(p', p)\Big{]}iS_{F}(p)\,,
\end{align}
where $\cJ_{\mu}$ is given in \er{B6}. We set
\bel{B47}
\Gamma_{\mu}^{(\gamma pp)}(p', p)=-e\, \widehat{\Gamma}_{\mu}^{(\gamma pp)}(p', p)\,.
\ee
The $P$, $C$, and $T$ relations, see \er{B2}--\er{B8}, 
lead to
\bal{B48}
\widehat{\Gamma}^{(\gamma pp)\mu}(p', p)&
=\cP^{\mu}{}_{\nu}\gamma_{0}\widehat{\Gamma}^{(\gamma pp)\nu}(\cP p',\cP  p)\gamma_{0}\,,\\
\label{B49}
\widehat{\Gamma}^{(\gamma pp)\mu}(p', p)&=-S(C)\Big{(}\widehat{\Gamma}^{(\gamma pp)\mu}(-p,-p')\Big{)}^{\top}S^{-1}(C)\,,\\
\label{B50}
\widehat{\Gamma}^{(\gamma pp)\mu}(p', p)&=\cP^{\mu}{}_{\nu} S(T)\Big{(}\widehat{\Gamma}^{(\gamma pp)\nu}(\cP p , \cP p')\Big{)}^{\top}S^{-1}(T)\,,
\end{align}
respectively. 
The most general ansatz for 
$\widehat{\Gamma}_{\mu}^{(\gamma pp)}$ 
compatible with $P$, $C$, and $T$ invariance, which hold in QCD, is then obtained as follows. $P$ invariance \er{B48} requires for $\widehat{\Gamma}_{\mu}^{(\gamma pp)}$ the structure
\bal{B51}
&\widehat{\Gamma}^{(\gamma pp)\mu}(p', p)=(p'+p)^{\mu} A_{1}+
(p'-p)^{\mu} A_{2}\nn\\
&\quad +\Big{[} g^{\mu\rho} A_{3}+(p'+p)^{\mu}(p'+p)^{\rho} A_{4}\nn\\
&\qquad +(p'+p)^{\mu}(p'-p)^{\rho} A_{5}\nn\\
&\qquad +(p'-p)^{\mu}(p'+p)^{\rho} A_{6}\nn\\
&\qquad +(p'-p)^{\mu}(p'-p)^{\rho} A_{7}\Big{]}\gamma_{\rho} \nn\\
&\quad +\Big{[}g^{\mu\rho}(p'+p)^{\sigma} A_{8}+g^{\mu\rho}(p'-p)^{\sigma} A_{9}\nn\\
&\qquad+(p'+p)^{\mu}(p'+p)^{\rho}(p'-p)^{\sigma} A_{10}\nn\\
&\qquad+(p'-p)^{\mu}(p'+p)^{\rho}(p'-p)^{\sigma} A_{11}\Big{]} i\sigma_{\rho\sigma}\,,
\end{align}
where
\bel{B52}
A_{j}
= A_{j}\big{[}p^{\prime\, 2}-m\up\2 ,\; p\2 - m\up\2 , \; (p' - p)\2 \big{]}
\ee
are analytic functions for 
\bal{B53}
|p^{\prime\, 2}-m\up\2 |&< 2m\up m\upp + m\upp\2\ ,\nn\\
|p\2 - m\up\2 |&< 2m\up m\upp + m\upp\2\ ,\nn\\
|(p' - p)\2 |&< 4m\upp\2 \,.
\end{align}
From $C$ and $T$ invariance \er{B49}, \er{B50}, we find that $ A_{j}$ must be symmetric under the exchange~\mbox{$p'\leftrightarrow p$}
\mbox{for $j=1,\, 3 ,\,4,\, 7,\, 9,\, 10$} 
and antisymmetric
\mbox{for $j = 2,\, 5,\, 6,\, 8,\, 11$}. 
Therefore we can set
\bal{B54}
& A_{j}\Big{[}p^{\prime\, 2}-m\up\2 ,\; p\2-m\up\2 , \; (p' -p)\2\Big{]}\nn\\
&\quad =( p^{\prime\, 2}-p\2)\widetilde{ A}_{j}\Big{[}p^{\prime\, 2}-m\up\2 ,\; p\2-m\up\2 , \; (p' -p)\2\Big{]}\,,\nn\\
& \text{for } j=2,\, 5,\, 6,\, 8,\, 11\,,
\end{align}
where the $\widetilde{ A}_{j}$ are symmetric under  $p'\leftrightarrow p$.

Now we explore the consequences of the generalized Ward identity
\bel{B55}
(p'-p)^{\mu}\,\widehat{\Gamma}_{\mu}^{(\gamma pp)} (p', p) = S_{F}^{-1}(p')-S_{F}^{-1}(p)\,.
\ee
The proof of the generalized Ward identity \er{B55} for the local interpolating proton field is analogous to the proof for the pion field; see \er{B33}--\er{B33b}. Of course, we have to take into account that the proton fields are Fermi fields.
Inserting in \er{B55} $\widehat{\Gamma}_{\mu}^{(\gamma pp)}$ from \er{B51}, \er{B54}, 
and $S_{F}^{-1}(p)$ from \er{B20a} we find the following relations, setting
\bel{B56}
q = p'-p \,,
\ee
\bal{B57}
&(p^{\prime 2}-p\2) A_{1} +q\2 (p^{\prime 2}-p\2)\widetilde{ A}\ud\nn\\
&\quad =-2m\up (p^{\prime 2}a_{p'}-p\2 a\up)\nn\\
&\qquad +m\up\Big{[} (p^{\prime 2}+m\up\2)b_{p'}-(p\2+m\up\2)b\up\Big{]}\,,\\
\nn\\
\label{B58}
& A_{3}+(p^{\prime 2}-p\2)\widetilde{ A}_{5}+q\2  A_{7}\nn\\
&\quad =1+ \frac{1}{2}(p^{\prime 2}+m\up\2)a_{p'}-m\up\2 b_{p'}\nn\\
&\qquad + \frac{1}{2}(p^{2}+m\up\2)a_{p}-m\up\2 b_{p}
\,,\\
\nn\\
\label{B59}
&(p^{\prime 2}-p\2)\Big {[} A_{4}+q\2\widetilde{ A}_{6}\Big{]}\nn\\
&\quad =\frac{1}{2}(p^{\prime 2}+m\up\2)a_{p'}-m\up\2 b_{p'}- \frac{1}{2}(p^{2}+m\up\2)a_{p}+m\up\2 b_{p}
\,,\\
\label{B60}
&\widetilde{ A}_{8}- A_{10}-q\2 \widetilde{ A}_{11}=0\,.
\end{align}
We can use \er{B57}--\er{B60} 
for expressing $A_{1}$, $A_{3}$, $A_{4}$, 
and $\widetilde{A}_{8}$ in terms of the other functions, 
$\widetilde{A}_{2}$, $\widetilde{A}_{5}$, 
$\widetilde{A}_{6}$, $A_{7}$, 
$A_{9}$, $A_{10}$, $\widetilde{A}_{11}$,
and $a\up$, $b\up$ \er{B20b}:
\bal{B61}
& A_{1}(p^{\prime 2}-m\up\2 , \, p\2-m\up\2 ,\, q\2)\nn\\
&\quad =-q\2 \widetilde{ A}_{2}(p^{\prime 2}-m\up\2 , \, p\2-m\up\2 ,\, q\2)\nn\\
&\qquad -m\up(a_{p'}+a\up)+\frac{1}{2} m\up(b_{p'}+b\up)\nn\\
&\qquad -m\up(p^{\prime 2} +p\2)\frac{a_{p'}-a\up}{p^{\prime 2} -p\2} \nn\\
&\qquad +\frac{m\up}{2}(p^{\prime 2} +p\2+2m\up\2)\,\frac{b_{p'}-b\up}{p^{\prime 2} -p\2} \,,\\  \nn \\
\label{B62}
& A_{3}(p^{\prime 2}-m\up\2 , \, p\2-m\up\2 ,\, q\2)\nn\\
&\quad =1+\frac{1}{2}(p^{\prime 2}+m\up\2)a_{p'}-m\up\2 b_{p'}\nn\\
&\qquad + \frac{1}{2}(p^{2}+m\up\2)a_{p}-m\up\2 b_{p}
\nn\\
&\qquad -(p^{\prime 2}-p\2)\2\widetilde{ A}_{5}(p^{\prime 2}-m\up\2 , \, p\2-m\up\2 ,\, q\2)\nn\\
&\qquad -q\2 { A}_{7}(p^{\prime 2}-m\up\2 , \, p\2-m\up\2 ,\, q\2)\,,\\
\nn 
 \\
\label{B63}
& A_{4}(p^{\prime 2}-m\up\2 , \, p\2-m\up\2 ,\, q\2)\nn \\
&\quad =
\frac{1}{4}(a_{p'}+a\up)
+\frac{1}{4}(p^{\prime 2}+p\2+2m\up\2)\, \frac{a_{p'}-a\up}{p^{\prime 2}-p\2}  \nn\\
&\qquad 
-m\up\2
\; \frac{b_{p'}-b\up}{p^{\prime 2}-p\2}
-q\2 \widetilde{ A}_{6}(p^{\prime 2}-m\up\2 ,\,
p\2-m\up\2 ,\, q\2)\,,
\end{align}
\begin{align}
\label{B64}
&\widetilde{ A}_{8}(p^{\prime 2}-m\up\2 , \, p\2-m\up\2 ,\, q\2)\nn\\
&\quad = A_{10}(p^{\prime 2}-m\up\2 , \, p\2-m\up\2 ,\, q\2)\nn  \\
&\qquad +q\2
\widetilde{ A}_{11}(p^{\prime 2}-m\up\2 , \, p\2-m\up\2\ ,\, q\2)\,. 
\end{align}

%========================================
%\newpage %p.17
%========================================
Now we shall consider the vertex function 
$\widehat{\Gamma}_{\mu}^{(\gamma pp)}$ \\
between on-shell-proton spinors
\bal{B65}
&\bar{u}(p')\widehat{\Gamma}_{\mu}^{(\gamma pp)\mu}(p',p)u(p)\,,\nn \\
&p^{\prime 2}= p\2=m\up\2\,,\nn\\
&p^{\prime 0}, \,p^{0}>0\,.
\end{align}
From \er{B51} we get
\bal{B66}
&\bar{u}(p')\,  \widehat{\Gamma}_{\mu}^{(\gamma pp)\mu}(p',p)u(p) \nn \\
&\quad =\bar{u}(p')\bigg{\lbrace}(p'+p)^{\mu}\Big{[} A_{1}(0,0,q\2)\nn\\
&\qquad +2m\up  A_{4}(0,0,q\2)+q\2  A_{10}(0,0,q\2)\Big{]}\nn\\
&\qquad +\gamma^{\mu} A_{3}(0,0,q\2)
+i\sigma^{\mu\rho}q_{\rho} A_{9}(0,0,q\2)\bigg{\rbrace}u(p)\,.
\end{align}
Using Gordon's identity
\bel{B67}
\bar{u}(p')\Big {[}(p'+p)^{\mu}-2m\up \gamma^{\mu}+i\sigma^{\mu\nu}(p'-p)_{\nu}\Big{]}u(p)=0
\ee
and the relations \er{B20b}, \er{B61}--\er{B63}, 
we arrive at
\bal{B68}
&\bar{u}(p')\, \widehat{\Gamma}^{(\gamma pp)\mu}(p', p)u(p)\nn\\
&\quad =\bar{u}(p')\Big{[} \gamma^{\mu}F_{1}(q\2)+\frac{i}{2m\up}\sigma^{\mu\nu}q_{\nu}F_{2}(q\2)\Big{]}u(p)\,,
\end{align}
where
\bal{B69}
F_{1}(q\2)&=1-q\2 
\Big{[}2m\up\widetilde{A}_{2}(0,0,q\2)
+4m\up\2 \widetilde{A}_{6}(0,0,q\2)
\nn\\
&\quad 
+A_{7}(0,0,q\2) -2m\up A_{10}(0,0,q\2) \Big{]} \,,
\nn\\ 
\nn\\
F_{2}(q\2)&=2m\up\Big{\lbrace} A_{9}(0,0,q\2)+m\up(a_{0}-b_{0})
\nn\\
&\quad +q\2 \Big{[} \widetilde{A}_{2}(0,0,q\2)+2m\up \widetilde{A}_{6}(0,0,q\2)
\nn\\
&\quad -A_{10}(0,0,q\2)\Big{]}\Big{\rbrace}\,.
\end{align}
Here we have also used the expansions of $a\up$ and $b\up$ from \er{B20b}.

With \er{B68} and \er{B69} we have recovered the standard expression for the $\gamma p p $ vertex on shell. We have the correct normalization 
\bel{B70} 
F_{1}(0)=1\,,
\ee
and for $F_{2}(0)$ we have
\bal{B71}
F_{2}(0)&=2m\up \big{[}A_{9}(0,0,0)+m\up (a_{0}-b_{0})\big{]}\nn\\
&=\frac{\mu\up}{\mu_{N} }-1=1.7298\dots \,,
\end{align}
where $\mu_{ N}$ is the nuclear magneton $e/(2m\up)$,
and $\mu\up$ is the magnetic moment of the proton.

Now we consider the vertex function $\widehat{\Gamma}^{(\gamma pp)\mu}$ \er{B51} for an on-shell-proton momentum $p$ and $p'=p-k$ where all components of $k$ are supposed to be of order $\omega$. We will study the expansion of $\widehat{\Gamma}^{(\gamma pp)\mu}$ in powers of $\omega$ for $\omega\to 0$. 
We have, thus, 
\bal{B72}
&p\2=m\2\,, \qquad p'=p-k\,,\nn\\
&p^{\prime 2 }-p\2 = p^{\prime 2 }-m\up\2 =-2p\cdot k +k\2\,,\nn \\
&p'+p=2p-k\,.
\end{align}
From \er{B51} and \er{B54} we get now
\bal{B73}
&\widehat{\Gamma}^{(\gamma pp)\mu}(p-k, p)
=(2p-k)^{\mu}A\uu
(-2p\cdot k + k\2 , 0, 0)\nn\\
&\quad + \gamma^{\mu} A_{3}
(-2p\cdot k + k\2 , 0, 0)\nn\\
&\quad+ (2p-k)^{\mu}(2p-k)^{\nu}\gamma_{\nu}
A_{4}
(-2p\cdot k + k\2 , 0, 0)\nn\\
&\quad -i\sigma^{\mu\nu}\, 2p_{\nu}\, 2(p\cdot k)\widetilde{A}_{8}(0,0,0)\nn\\
&\quad -i\sigma^{\mu\nu}\, k_{\nu}A_{9}(0,0,0)\nn\\
&\quad -i\sigma_{\rho\sigma}p^{\rho}k^{\sigma}4p^{\mu}A_{10}(0,0,0)\nn\\
&\quad + {\mathcal O}(\omega\2)\,.
\end{align}
With \er{B20b}, \er{B61}--\er{B63} we get
\bal{B74}
&(2p-k)^{\mu}A\uu
(-2p\cdot k + k\2 , 0, 0)\nn\\
&\quad =(2p-k)^{\mu}\bigg{\lbrace} -2m\up a_{0}-2m\up^{3}a\uu
+m\up b_{0} + 2m\up^{3} b\uu\nn\\
&\qquad +(-2p\cdot k)\Big{[}-2m\up a\uu +m\up b\uu -2m\up^{3}a\ud +2m\up^{3}b\ud\Big{]}\bigg{\rbrace}\nn\\
&\qquad + {\mathcal O}(\omega\2)\,,\\
\nn\\
\label{B75}
&\gamma^{\mu}A_{3}(-2p\cdot k ,0,0)\nn\\
&\quad =
\gamma^{\mu}\bigg{\lbrace} 1+2m\up\2(a_{0}-b_{0} )\nn\\
&\qquad +(-2p\cdot k) \Big{[} \frac{1}{2}a_{0} +m\up\2 a\uu -m\up\2 b\uu\Big{]} \bigg{\rbrace}
\nn\\
&\qquad + {\mathcal O}(\omega\2)\,,\\
\nn\\
\label{B76}
&(2p-k)^{\mu}(2p-k)^{\nu}\gamma_{\nu}A_{4}(-2p\cdot k ,0,0)\nn\\
&\quad =\Big{[}4p^{\mu}\!\! \bp -2k^{\mu}\!\! \bp -2p^{\mu}\!\! \bk\Big {]}
\bigg{\lbrace}\frac{1}{2}a_{0}+m\up\2 a\uu-m\up\2 b\uu\nn\\
&\qquad +(-2p\cdot k)\Big{[}\frac{1}{2}a_{1}+m\up\2 a_{2}-m\up\2 b\ud\Big{]}\bigg{\rbrace}\nn\\
&\qquad + {\mathcal O}(\omega\2)\,.
\end{align}
Putting everything together we find from \er{B73}--\er{B76}
\bal{B77}
&\widehat{\Gamma}^{(\gamma pp)\mu}(p-k, p)\, u(p)\nn\\
&\quad=\bigg{\lbrace}
(2p-k)^{\mu}(-m\up a_{0} +m\up b_{0})\nn\\
&\qquad +\gamma^{\mu}(1+2m\up\2 a_{0}-2m\up\2 b_{0})\nn\\
&\qquad -2p^{\mu}\!\! \bk\big{(}\frac{1}{2}a_{0}+m\up\2 a\uu -m\up\2 b\uu\big{)}\nn\\
&\qquad -i\sigma^{\mu\nu} k_{\nu}A_{9}(0,0,0)\nn\\
&\qquad +2p^{\mu}(\bp\bk -\bk\bp)A_{10}(0,0,0)\nn\\
&\qquad +(-2p\cdot k)\Big{[} (2p-k)^{\mu}(-m\up a\uu +m\up b\uu)\nn\\
&\qquad +\gamma^{\mu}\big{(}\frac{1}{2}a_{0}+m\up\2 a\uu-m\up\2 b\uu\big{)}\nn\\
&\qquad -(\gamma^{\mu}\!\! \bp\;-\bp\gamma^{\mu})\widetilde{A}_{8}(0,0,0)\Big{]}\bigg{\rbrace}u(p)
+ {\mathcal O}(\omega\2)\,.
\end{align}

Now we shall derive the expansion of 
\bel{B78}
S_{F}(p-k)\, \widehat{\Gamma}^{(\gamma pp)\mu}(p-k, p)\,  u(p)\ee
for $\omega\to 0$, where $p$ and $k$ are as in \er{B72}.
Here we use $S_{F}$ from \er{B21}, expanded in $\omega$ with the help of \er{B20b}, 
and multiply with 
$\widehat{\Gamma}^{(\gamma pp)\mu}(p-k, p)\, u(p)$ 
from~\er{B77}.\\
The result is
\bal{B79}
&S_{F}(p-k)\widehat{\Gamma}^{(\gamma pp)\mu}(p-k, p) \, u(p)\nn\\
&\quad =\bigg{\lbrace}\frac{\bp+m\up-\bk}{-2p\cdot k+k\2+i\varepsilon}\Big{[}\gamma^{\mu}-i\sigma^{\mu\nu}k_{\nu}\Big{(}A_{9}(0,0,0)\nn\\
&\qquad +m\up(a_{0}-b_{0})\Big{)}\Big{]}\bigg{\rbrace}u(p)+{\mathcal O}(\omega)\nn\\
&\quad=\frac{\bp+m\up-\bk}{-2p\cdot k+k\2+i\varepsilon}\Big{[}\gamma^{\mu}-i\sigma^{\mu\nu}k_{\nu}\frac{1}{2m\up}F_{2}(0)\Big{]}u(p) \nn\\
&\qquad +{\mathcal O}(\omega)\,,
\end{align}
where in the last step we use \er{B71}.

We find it convenient to go from \er{B79} to an equivalent matrix relation. With $\lambda\in \lbrace 1/2 , \, -1/2\rbrace$, the proton's spin indices, we have
\bal{B80}
&S_{F}(p-k)\, \widehat{\Gamma}^{(\gamma pp)\mu}(p-k, p) (\bp+m\up)\nn\\
&\quad = \sum_{\lambda}S_{F}(p-k)\, \widehat{\Gamma}^{(\gamma pp)\mu}(p-k, p)u(p,\lambda)\bar{u}(p,\lambda)\,,\nn\\
&S_{F}(p-k)\, \widehat{\Gamma}^{(\gamma pp)\mu}(p-k, p) u(p,\lambda)\nn\\
&\quad =\frac{1}{2m\up}S_{F}(p-k)\, \widehat{\Gamma}^{(\gamma pp)\mu}(p-k, p)(\bp+m\up)u(p,\lambda)\,.
\end{align}
The result \er{B79} is then equivalent to 
\bal{B81}
&S_{F}(p-k)\, \widehat{\Gamma}^{(\gamma pp)\mu}(p-k, p)(\bp+m\up)\nn\\
&\quad =\frac{\bp+m\up-\bk}{-2p\cdot k+k\2+i\varepsilon}\Big{[}\gamma^{\mu}-\frac{i}{2m\up}\sigma^{\mu\nu}k_{\nu}F\ud(0)\Big{]} ( \bp+m\up)\nn\\
&\qquad
+{\mathcal O}(\omega)\,.
\end{align}

Finally we apply the $T$ transformation to \er{B80}. From \er{B15} and \er{B50} we get
\bal{B82}
&S_{F}(p-k)\, \widehat{\Gamma}^{(\gamma pp)\mu}(p-k, p)(\bp+m\up)\nn\\
&\quad =S(T)S^{\top}_{F}[\cP (p-k)] S^{-1}(T)\nn\\
&\qquad \times \cP^{\mu}{}_{\nu}S(T)\Big{(}\widehat{\Gamma}^{(\gamma pp)\nu}\big{(}\cP p, \cP (p-k)\big{)}\Big{)}^{\top}S^{-1}(T)\nn\\
&\qquad \times S(T)(\cP^{\rho}{}_{\lambda}p^{\lambda}\gamma_{\rho}+m\up)^{\top}S^{-1}(T)\,,
\end{align}
where $S(T)$ and $\cP$ are given in \er{B5} and \er{B8}, respectively. 
Next we make in \er{B82} the replacements
\bal{B83}
p\to\cP p'\,,\quad k\to -\cP k\,, \quad p-k\to \cP (p'+k)\,,
\end{align}
where with $p$ also $p'$ is an on-shell-proton momentum. 
From \er{B81} and \er{B82} we get then
\bal{B84}
&\cP^{\mu}{}_{\nu} S(T)\Big{[} (\bp' +m\up)\widehat{\Gamma}^{(\gamma pp)\nu}(p', p'+k)S_{F}(p'+k)\Big{]}^{\top}S^{-1}(T)\nn\\
&\quad =\frac{\big{(}\cP(p'+k)\big{)}^{\rho}\gamma_{\rho}+m\up}{2p'\cdot k +k\2+i\varepsilon}\Big{[}\gamma^{\mu}+\frac{i}{2m\up}\sigma^{\mu\nu}(\cP k )_{\nu} F\ud (0)\Big{]}\nn\\
&\qquad \times \Big{[} (\cP p')^{\sigma}\gamma_{\sigma}+m\up\Big{]}+{\mathcal O}(\omega)\,,\\
\nn\\
\label{B85}
&(\bp' +m\up)\widehat{\Gamma}^{(\gamma pp)\mu} (p', p'+k)S_{F}(p'+k)\nn\\
&\quad = (\bp'+m\up)\Big{[}\gamma^{\mu}-\frac{i}{2m\up}\sigma^{\mu\nu}k_{\nu} F\ud (0)\Big{]}\nn\\
&\qquad \times \frac{\bp'+m\up +\bk}{2p'\cdot k +k\2+i\varepsilon}
+{\mathcal O}(\omega)\,.
\end{align}
We also recall the following useful relations 
from \er{B1} and \er{B2} of \cite{Lebiedowicz:2022nnn}:
\bal{B86}
&\frac{\bp + m\up -\!\! \bk}{(p-k)\2 -m\up\2 +i\varepsilon}\Big{[} \gamma^{\mu}-\frac{i}{2m\up}\sigma^{\mu\nu}k_{\nu} F\ud (0)\Big{]}(\bp+m\up)\nn\\
&\quad =\frac{1}{-2p\cdot k +k\2 +i\varepsilon}\bigg{\lbrace}2p^{\mu}-k^{\mu}
%\nn\\
%&\qquad
+\big{(}1+F_{2}(0)\big{)}(k^{\mu}-\!\bk \gamma^{\mu})\nn\\
&\qquad + \frac{F\ud(0)}{2m\up}\Big{[}2\big{(}p^{\mu}\!\!\bk -(p \cdot k)\gamma^{\mu}\big{)}-(\bk k^{\mu}-k\2\gamma^{\mu}\Big{]}
\bigg{\rbrace} \nn\\
&\qquad \times
(\bp+m\up)\,,\\
\nn\\
\label{B87}
&(\bp'+m\up)
\Big{[}\gamma^{\mu}-\frac{i}{2m\up}\sigma^{\mu\nu}k_{\nu} F\ud (0)\Big{]}\frac{\bp' + m\up +\!\! \bk}{(p'+k)\2 -m\up\2 +i\varepsilon}\nn\\
&\quad =(\bp'+m\up)\, \frac{1}{2p'\cdot k +k\2 +i\varepsilon}
\bigg{\lbrace}2p^{\prime\mu}+k^{\mu}\nn\\
&\qquad -\big{(}1+F\ud(0)\big{)}
(k^{ \mu}-\gamma^{\mu}\!\!\bk)\nn\\
&\qquad -\frac{F\ud(0)}{2m\up}\Big{[}
2\big{(}p^{\prime\mu}\!\! \bk - (p'\cdot k) \gamma^{\mu}\big{)}
+(k^{\mu} \!\!\bk -k\2\gamma^{\mu})\Big{]}\bigg{\rbrace}\,.
\end{align}
In \er{B86} and \er{B87} $p$ and $p'$ 
are on-shell-proton momenta and $k$ is arbitrary.

%========================================
%p.19
%========================================

\section{Definition and details of the calculations for the $\ppm p \to \ppm p \gamma$ amplitudes}
\label{app:C}

%======================================
\begingroup
\allowdisplaybreaks
%======================================
Here we consider the reactions
\bel{C1}
\ppm\,(p\ua)+ p\,(p\ub)\to \ppm\,(p'\uu)+p\,(p'\ud)+\gamma(k)
\ee
for $\ppm$ and the proton on shell and $\gamma$ on or off shell. The matrix amplitudes $\cN\ul\equiv\cN^{-}\ul$ from \er{3.16} and the analogous $\cN\ul^{+}$ are defined as follows:
\bal{C2}
\cN\hpm\ul &= 
\int \dv^{4}x\uu\, \dv^{4}x\ud\, \dv^{4}x\ua\, \dv^{4}x\ub \nn\\
&\quad
\times (\slash{p}'\ud+m\up)e^{ip'\ud x\ud}
(-i\! \slash{\rightsidep}{x_{2}}+m\up)\nn\\
&\quad \times
e^{ip'\uu x\uu}(\stackrel{\rightarrow}{\square}_{x\uu}+m_{\pi}\2)\nn\\
&\quad \times 
\braket{0|T\big{(}\varphi\hpm(x\uu)\varphi\hmp(x\ua)\psi(x\ud)\overline{\psi}(x\ub)\big{(}-\mathcal{J}\ul (0)\big{)}\big{)}|0}_{c}\nn\\
&\quad \times (\stackrel{\leftarrow}{\square}_{x\ua}+m_{\pi}\2)e^{-ip\ua x\ua}\nn\\
&\quad \times
(i\! \slash{\leftsidep}{x_{b}} + m\up)e^{-ip'\ub x\ub}
(\slash{p}\ub +m\up)\,.
\end{align}
Here the fields and the current are as in \eqref{B1} and \er{B6}, respectively, 
$c$ stands for the connected part,
and we use the reduction formula; 
see, e.g.,~\cite{Bjorken:1965}.

Now we discuss the details of the calculation for 
\mbox{$\cN\ul^{(a)}$, $\dots$, $\cN\ul^{(e)}$} 
corresponding to the diagrams of 
Fig.~\ref{fig:3}(a), $\dots$, Fig.~\ref{fig:3}(e).
 We have then from \er{3.18}
\bel{C3}
\cN^{-}\ul\equiv \cN\ul = \cN\ul^{(a)}+\cN\ul^{(b)}+\cN\ul^{(c)}+\cN\ul^{(d)}+\cN\ul^{(e)}\,.
\ee
In the following calculations we shall treat explicitly the $\pi^{-}p\to\pi^{-} p \gamma$ scattering and omit, for brevity, the minus superscript in the expressions.
In Fig.~\ref{fig:3}(a)--(d) 
the off-shell $\pi^{-}p\to \pi^{-} p$ amplitude 
is occurring which is given in \er{2.16}, \er{2.17}. 
We shall need the eight scalar amplitudes 
$\cM\uu, \dots ,\cM_{8}$ ($\cM_{j} \equiv \cM_{j}^{-}$) 
on shell and their partial derivatives
\bal{C4}
&\cM_{j}^{(\text{on})}=\cM_{j}
(s,t,\,m_{\pi}\2,\,m\up\2,\,m_{\pi}\2,\,m\up\2)\,,\nn\\
&\cM_{j},_{s}=\frac{\partial}{\partial s}\cM_{j}
(s,t,\,m_{\pi}\2,\,m\up\2,\,m_{\pi}\2,\,m\up\2)\,,\nn\\
&\cM_{j},_{t}=\frac{\partial}{\partial t}\cM_{j}
(s,t,\,m_{\pi}\2,\,m\up\2,\,m_{\pi}\2,\,m\up\2)\,,\nn\\
&\cM_{j},_{m\uu\2}=\frac{\partial}{\partial m\uu\2}\cM_{j}
(s,t,\,m_{1}\2,\,m\up\2,\,m_{\pi}\2,\,m\up\2)\Big{|}_{m\uu\2 = m_{\pi}\2} \,,\nn\\
&\cM_{j},_{m\ud\2}=\frac{\partial}{\partial m\ud\2}\cM_{j}
(s,t,\,m_{\pi}\2,\,m\ud\2,\,m_{\pi}\2,\,m\up\2)\Big{|}_{m\ud\2 = m_{p}\2} \,,\nn\\
&\cM_{j},_{m\ua\2}=\frac{\partial}{\partial m\ua\2}\cM_{j}
(s,t,\,m_{\pi}\2,\,m\up\2,\,m\ua\2,\,m\up\2)\Big{|}_{m\ua\2 = m_{\pi}\2} \,,\nn\\
&\cM_{j},_{m\ub\2}=\frac{\partial}{\partial m\ub\2}\cM_{j}
(s,t,\,m_{\pi}\2,\,m\up\2,\,m_{\pi}\2,\,m\ub\2)\Big{|}_{m\ub\2 = m_{p}\2} \,,\nn\\
&(j=1, \dots, 8)\,.
\end{align}
\newpage
From \er{2.24} and \er{2.25} we find then
\bal{C5}
A,_{s}^{\!\!(\text{on})}&=\cM_{1},_{s}+m\up\cM_{2},_{s}-m\up\cM_{4},_{s}\nn\\
&\quad +(-s+m\up\2 +m_{\pi}\2)\cM_{5},_{s}\nn\\
&\quad +(s+t-m\up\2 -m_{\pi}\2)\cM_{7},_{s}\nn\\
&\quad -m\up (2s+t-2m\up\2-2m\upp\2)\cM_{8},_{s}\nn\\
&\quad -\cM_{5}^{(\text{on})}+\cM_{7}^{(\text{on})}-2m\up \cM_{8}^{(\text{on})}\,,
\\
\nn\\
\label{C6}
A,_{t}^{\!\!(\text{on})}&=\cM_{1},_{t}+m\up\cM_{2},_{t}-m\up\cM_{4},_{t}\nn\\
&\quad +(-s+m\up\2 +m_{\pi}\2)\cM_{5},_{t}\nn\\
&\quad +(s+t-m\up\2 -m_{\pi}\2)\cM_{7},_{t}\nn\\
&\quad -m\up (2s+t-2m\up\2-2m\upp\2)\cM_{8},_{t}\nn\\
&\quad +\cM_{7}^{(\text{on})}-m\up \cM_{8}^{(\text{on})}\,,\\
\nn\\
\label{C7}
B,_{s}^{\!\!(\text{on})}&=\cM_{2},_{s}+\cM_{4},_{s}+2m\up\cM_{5},_{s}\nn\\
&\quad -2m\up\cM_{7},_{s}+(4m\up\2 -t)\cM_{8},_{s}\,,\\
\nn\\
\label{C8}
B,_{t}^{\!\!(\text{on})}&=\cM_{2},_{t}+\cM_{4},_{t}+2m\up\cM_{5},_{t}\nn\\
&\quad -2m\up\cM_{7},_{t}+(4m\up\2 -t)\cM_{8},_{t}
-\cM_{8}^{(\text{on})}\,.
\end{align}

In the following calculations the momenta 
$p\ua$, $p\ub$, $p\uu$, $p\ud$, $p_{s}$, $p_{t}$, $p_{u}$, 
as well as $s$, $t$, 
refer to the $\pi^{-}p \to \pi^{-}p$ on-shell reaction; 
see \eqref{3.2} and \eqref{3.37}.

\subsection{The term $(a+b+e1)$}
\label{subsec:C1}

Now we discuss $\cN\ul^{(a)}$ corresponding to 
the diagram of Fig.~\ref{fig:3}(a). 
From \er{3.20}, \er{3.33}, and \er{3.34} we get
\bel{C9}
\cN\ul^{(a)}=-e\, \cN^{(0,a)}\left[\frac{(2p\ua -k)\ul}{-2p\ua \cdot k+k\2}+{\mathcal O}(\omega)\right]\,,
\ee
where $\cN^{(0,a)}$ is given in \eqref{3.39}. In order to simplify the expression for $\cN^{(0,a)}$ we use the following results which are easily checked.
\bal{C10}
&(\bp'\ud +m\up)(\bp_{s}-\bk)(\bp\ub +m\up) \nn\\
&\quad =(\bp'\ud +m\up)\Big{[} m\up+\frac{1}{2}(\bp\ua +\bp'\uu-\bk)\Big{]} (\bp\ub+m\up)\,,
\nn\\ 
\nn\\
&(\bp'\ud +m\up)(\bp_{t}-\!\not{l}\ud)(\bp\ub +m\up)=0\,,
\nn\\
\nn\\
&(\bp'\ud +m\up)(\bp_{u}-\!\not{l}\uu)(\bp\ub +m\up)\nn\\
&\quad =(\bp'\ud +m\up)\Big{[}- m\up+\frac{1}{2}(\bp\ua +\bp'\uu-\bk)\Big{]} (\bp\ub+m\up)\,,
\nn\\
\nn\\
&(\bp'\ud +m\up)i\sigma_{\mu\nu}(p_{s}-k)^{\mu}(p\ut -l\ud)^{\nu}(\bp\ub +m\up)\nn\\
&\quad 
 =(\bp'\ud +m\up)\Big{[}- \frac{1}{2}(p\ua +p'\uu-k\, ,\, p\ub+p'\ud)\nn\\
&\qquad
-(p\ub\cdot p'\ud)+m\up\2 +m\up (\bp\ua+\bp'\uu-\bk)\Big{]} (\bp\ub+m\up)\,,
\nn\\
\nn\\
&(\bp'\ud +m\up)i\sigma_{\mu\nu}(p_{s}-k)^{\mu}(p_{u} -l\uu)^{\nu}(\bp\ub +m\up)\nn\\
&\quad 
 =(\bp'\ud +m\up)\frac{1}{2}(p\ua +p'\uu-k\, ,\, p'\ud-p\ub) (\bp\ub+m\up)\,,
 \nn\\
\nn\\
  &(\bp'\ud +m\up)i\sigma_{\mu\nu}(p_{t}-l\ud)^{\mu}(p_{u}-l\uu)^{\nu}(\bp\ub +m\up)
 \nn\\
 &\quad =(\bp'\ud +m\up)\Big{[}\frac{1}{2}(p\ua +p'\uu-k\, ,\, p\ub+p'\ud)\nn\\
&\qquad
-(p\ub\cdot p'\ud)+m\up\2 -m\up (\bp\ua+\bp'\uu-\bk)\Big{]} (\bp\ub+m\up)\,,
\nn\\
\nn\\
&(\bp'\ud +m\up)i  \gamma_{\mu}\gamma_{5}\varepsilon^{\mu\nu\rho\sigma}(p_{s}-k)_{\nu}(p_{t} -l\ud)_{\rho}
(p_{u}-l\uu)_{\sigma} \nn \\
&\times (\bp\ub +m\up)
=
(\bp'\ud +m\up)\Big{[}-m\up(p\ua +p'\uu-k\, ,\, p\ub+p'\ud)\nn\\
 &\qquad 
 +(\bp\ua+\bp'\uu -\bk)\big{(}m\up\2 +(p'\ud \cdot p\ub)\big{)}\Big{]}
  (\bp\ub+m\up)\,.
\nn
\\
  \end{align}
Inserting \er{C10} into \er{3.39} we obtain with $\cM_{j}^{(a)}$ as in \er{3.40}
\bal{C11}
&\cN^{(0,a)}=(\bp'\ud+m\up)\bigg{\lbrace}\cM\uu^{(a)}+m\up\cM\ud^{(a)}-m\up\cM_{4}^{(a)}\nn\\
&\quad +\Big{[} - \frac{1}{2}(p\ua +p'\uu-k\, ,\, p\ub+p'\ud)-(p\ub\cdot p'\ud)+m\up\2\Big{]}\cM_{5}^{(a)}\nn\\
&\quad +\frac{1}{2}(p\ua +p'\uu-k\, ,\, p'\ud-p\ub)\cM_{6}^{(a)}\nn\\
&\quad +\Big{[}  \frac{1}{2}(p\ua +p'\uu-k\, ,\, p\ub+p'\ud)-(p\ub\cdot p'\ud)+m\up\2\Big{]}\cM_{7}^{(a)}\nn\\
&\quad -m\up(p\ua +p'\uu-k\, ,\, p\ub+p'\ud)\cM_{8}^{(a)}\nn\\
&\quad +\frac{1}{2}(\bp\ua +\bp'\uu -\bk)\Big{[}\cM_{2}^{(a)}+\cM_{4}^{(a)}+2m\up\cM_{5}^{(a)}\nn\\
&\quad -2m\up\cM_{7}^{(a)}+\big{(}2m\up\2+2(p'\ud\cdot p\ub)\big{)}\cM_{8}^{(a)}\Big{]}\bigg{\rbrace}
(\bp\ub+m\up)
\nn\\
& \quad
+{\mathcal O}(\omega\2)\,.
\end{align}

Our next topic is to discuss $\cN\ul^{(b)}$ 
corresponding to the diagram of Fig.~\ref{fig:3}(b). 
Here we have from \er{3.21} and \er{B45}
\bel{C12}
\cN\ul^{(b)}=-e\, \frac{(2p'\uu+k)\ul}{2p'\uu\cdot k + k\2} \cN^{(0,b)}+{\mathcal O}(\omega)\,,
\ee
where
\bal{C13}
\cN^{(0,b)}&=(\bp'\ud +m\up)\cM^{(0,b)}(\bp\ub +m\up)\,,
\nn\\ 
\cM^{(0,b)}&=\cM^{(0)}
(p'\uu+k\, ,\, p'\ud \, ,\,  p\ua\, ,\,  p\ub)\,;
\end{align}
see \er{3.25} and \er{3.29}.
Here $\cM^{(0,b)}$ is the off-shell amplitude for $\pi^{-} p\to\pi^{-} p$ as in \er{2.16}, \er{2.17} with the appropriate momenta inserted.
\newpage
This gives
\bal{C14}
&\cN^{(0,b)}=(\bp'\ud+m\up)\Big{\lbrace}\cM_{1}^{(b)}+\bp\us\cM_{2}^{(b)}\nn\\
&\quad +(\bp\ut-\!\not{l}\ud)\cM_{3}^{(b)}+
(\bp_{u}+\!\not{l}\ud)\cM_{4}^{(b)}\nn\\
&\quad +i\sigma_{\mu\nu}p\us{}^{\mu}(p\ut -l\ud)^{\nu}\cM_{5}^{(b)}\nn\\
&\quad +i\sigma_{\mu\nu}p\us{}^{\mu}(p_{u}+l\ud)^{\nu}\cM_{6}^{(b)}\nn\\
&\quad +i\sigma_{\mu\nu}(p\ut -l\ud)^{\mu}(p_{u}+l\ud)^{\nu}\cM_{7}^{(b)}\nn\\
&\quad +i  \gamma_{\mu}\gamma_{5}\varepsilon^{\mu\nu\rho\sigma}p_{s\nu}(p\ut -l\ud)_{\rho}(p_{u} +l\ud)_{\sigma}\cM_{8}^{(b)}\Big{\rbrace}(\bp\ub+m\up)\nn\\
&\quad +{\mathcal O}(\omega\2)\,,
\end{align}
where
\bal{C15}
\cM_{j}^{(b)}&
=\cM_{j}(s,t-2p\ut\cdot l\ud , m\upp\2 +2p\uu\cdot k , m\up\2 , m\upp\2 , m\up\2)  \nn\\
&\quad +{\mathcal O}(\omega\2) \nn\\
&
=\cM_{j}^{(\text{on})}-2(p\ut\cdot l\ud)\cM_{j},_{t}+2(p\uu\cdot k)\cM_{j},_{m\uu\2}\nn\\
&\quad +{\mathcal O}(\omega\2)\,, \nn\\
(j&=1, \dots, 8)\,.
\end{align}
We have the following relations:
\bal{C16}
&(\bp'\ud +m\up)\bp\us (\bp\ub +m\up)\nn\\
&\quad =
 (\bp'\ud +m\up)\Big{[} m\up + 
 \frac{1}{2}(\bp\ua + \bp'\uu +\bk)\Big{]} (\bp\ub +m\up)\,,\nn\\
 \nn\\
&(\bp'\ud +m\up)(\bp\ut -\!\not{l}\ud) (\bp\ub +m\up)=0\,, \nn\\
\nn\\
&(\bp'\ud +m\up)(\bp_{u} +\!\not{l}\ud) (\bp\ub +m\up) 
\nn\\
&\quad=
 (\bp'\ud +m\up)\Big{[}- m\up + \frac{1}{2}(\bp\ua +\bp'\uu +\bk)\Big{]} (\bp\ub +m\up)\,,\nn\\
\nn\\
& (\bp'\ud +m\up) i\sigma_{\mu\nu}p_{s}{}^{\mu}(p\ut -l\ud)^{\nu}(\bp\ub +m\up)
\nn\\
&\quad=
(\bp'\ud +m\up)\Big{[}- \frac{1}{2}(p\ua +p'\uu+k\, ,\,
 p\ub+p'\ud) \nn\\
&\qquad
-(p\ub\cdot p'\ud)+m\up\2 +m\up (\bp\ua+\bp'\uu+\bk)\Big{]} (\bp\ub+m\up)\,,\nn\\
\nn\\
& (\bp'\ud +m\up) i\sigma_{\mu\nu}p_{s}{}^{\mu}(p_{u} +l\ud)^{\nu}(\bp\ub +m\up)
\nn\\
&\quad=
(\bp'\ud +m\up)\frac{1}{2}(p\ua +p'\uu+k\, ,\, p'\ud-p\ub) 
 (\bp\ub+m\up)\,,\nn\\ 
\nn\\
& (\bp'\ud +m\up) i\sigma_{\mu\nu}(p\ut -l\ud)^{\mu}(p_{u}+l\ud)^{\nu}(\bp\ub +m\up)
\nn\\
&\quad=
(\bp'\ud +m\up)
\Big{[} \frac{1}{2}(p\ua +p'\uu+k\, ,\, p\ub+p'\ud) \nn\\
&\qquad
-(p\ub\cdot p'\ud)+m\up\2 -m\up (\bp\ua+\bp'\uu+\bk)\Big{]} (\bp\ub+m\up)\,,
\nn\\
 \nn\\
&(\bp'\ud +m\up) i \gamma_{\mu}\gamma_{5}\varepsilon^{\mu\nu\rho\sigma}p_{s\nu}(p\ut -l\ud)_{\rho}(p_{u} +l\ud)_{\sigma}(\bp\ub+m\up)
\nn\\
&\quad =
(\bp'\ud +m\up)\Big{[} -m\up (p\ua +p'\uu+k\, ,\, p\ub+p'\ud) \nn\\
&\qquad 
+ (\bp\ua+\bp'\uu+\bk)\big{(} m\up\2+(p'\ud\cdot p\ub)\big{)}\Big{]}
(\bp\ub+m\up)\,. 
\end{align}
Inserting \er{C16} into \er{C14} we get
\bal{C17}
&\cN^{(0,b)}=(\bp'\ud+m\up)\bigg{\lbrace}
\cM_{1}^{(b)}+m\up\cM_{2}^{(b)}-m\up\cM_{4}^{(b)}\nn\\
&\quad +\Big{[}-\frac{1}{2}(p\ua +p'\uu+k\, ,\, p\ub+p'\ud)-(p\ub\cdot p'\ud)+m\up\2 \Big{]} \cM_{5}^{(b)}\nn\\
&\quad +\frac{1}{2}(p\ua +p'\uu+k\, ,\, p'\ud-p\ub) \cM_{6}^{(b)}\nn\\
&\quad+\Big{[}\frac{1}{2}(p\ua +p'\uu+k\, ,\, p\ub+p'\ud)-(p\ub\cdot p'\ud)+m\up\2 \Big{]} \cM_{7}^{(b)}\nn\\
&\quad -m\up(p\ua +p'\uu+k\, ,\, p\ub+p'\ud)\cM_{8}^{(b)}\nn\\
&\quad+ \frac{1}{2}(\bp\ua +\bp'\uu +\bk)\Big{[} \cM_{2}^{(b)}+ \cM_{4}^{(b)}+2m\up \cM_{5}^{(b)}\nn\\
&\quad -2m\up\cM_{7}^{(b)}
+\big{(} 2m\up\2+2(p'\ud\cdot p\ub)\big{)}\cM_{8}^{(b)}\Big{]}\bigg{\rbrace}(\bp\ub+m\up)\nn\\
&\quad+{\mathcal O}(\omega\2)\,.
\end{align}

Now we define an amplitude $\cN\ul^{(e1)}$ which shall be regular for $\omega=0$ and give the complement needed by gauge invariance for $\cN\ul^{(a)}+\cN\ul^{(b)}$.
That is, we require
\bel{C18}
k^{\lambda}\left(\cN\ul^{(a)}+\cN\ul^{(b)}+\cN\ul^{(e1)}\right) 
= 0\,.
\ee
We shall find that \er{C18} gives a unique solution for $\cN\ul^{(e1)}$ to order $\omega^{0}$.

From \er{3.28} and \er{3.29} we get
\bel{C19}
k^{\lambda}\cN\ul^{(e1)}
=-e\left( \cN^{(0,a)}- \cN^{(0,b)}\right)\,.
\ee
Inserting here $\cN^{(0,a)}$ and $\cN^{(0,b)}$ from \er{C11} and \er{C17}, respectively, we find
%============================
%\begin{widetext}
%============================
\bal{C20}
&k^{\lambda}\cN\ul^{(e1)}=-e(\bp'\ud+m\up)\bigg{\lbrace}\nn\\
&-2(p\us\cdot k)\cM_{1},_{s}-2(p\ua\cdot k)\cM_{1},_{m\ua\2}-2(p\uu\cdot k)\cM_{1},_{m\uu\2}\nn\\
&+m\up\Big{[}-2(p\us\cdot k)\cM_{2},_{s}-2(p\ua\cdot k)\cM_{2},_{m\ua\2}-2(p\uu\cdot k)\cM_{2},_{m\uu\2}\Big{]}\nn\\
&-m\up\Big{[}-2(p\us\cdot k)\cM_{4},_{s}-2(p\ua\cdot k)\cM_{4},_{m\ua\2}-2(p\uu\cdot k)\cM_{4},_{m\uu\2}\Big{]}\nn\\
&+\Big{[} -\frac{1}{2}(p\ua +p'\uu\, ,\, p\ub+p'\ud)-(p\ub\cdot p'\ud)+m\up\2\Big{]}\nn\\
&\quad \times \Big{[}-2(p\us\cdot k)\cM_{5},_{s}-2(p\ua\cdot k)\cM_{5},_{m\ua\2}-2(p\uu\cdot k)\cM_{5},_{m\uu\2}\Big{]}\nn\\
&+(k\, ,\, p\ub +p'\ud ) \cM_{5}^{\text{(on)}}\nn\\
&+\frac{1}{2}(p\ua+p'\uu\, ,\, p'\ud -p\ub)\Big{[}-2(p\ua\cdot k)\cM_{6},_{m\ua\2}-2(p\uu\cdot k)\cM_{6},_{m\uu\2}\Big{]}\nn\\
&+\Big{[}\frac{1}{2}(p\ua +p'\uu\, ,\, p\ub+p'\ud)-(p\ub\cdot p'\ud)+m\up\2\Big{]}\nn\\
&\quad \times \Big{[}-2(p\us\cdot k)\cM_{7},_{s}-2(p\ua\cdot k)\cM_{7},_{m\ua\2}-2(p\uu\cdot k)\cM_{7},_{m\uu\2}\Big{]}\nn\\
&-(k\, ,\,p\ub +p'\ud ) \cM_{7}^{\text{(on)}}
%\nn\\
%&
-m\up(p\ua+p'\uu\, ,\, p\ub+p'\ud) \nn\\
& \quad \times
\Big{[}-2(p\us\cdot k)\cM_{8},_{s}% 
-2(p\ua\cdot k)\cM_{8},_{m\ua\2}
%\nn\\
%&\quad
-2(p\uu\cdot k)\cM_{8},_{m\uu\2}\Big{]}\nn\\
&+2m\up(k\, ,\, p\ub +p'\ud ) \cM_{8}^{\text{(on)}}\nn\\
&+\frac{1}{2}(\bp\ua+\bp'\uu)
\Big{[}-2(p\us\cdot k)\cM_{2},_{s}-2(p\ua\cdot k)\cM_{2},_{m\ua\2}
\nn\\
&\quad -2(p\uu\cdot k)\cM_{2},_{m\uu\2}-2(p\us\cdot k)\cM_{4},_{s}
\nn\\
&\quad -2(p\ua\cdot k)\cM_{4},_{m\ua\2}-2(p\uu\cdot k)\cM_{4},_{m\uu\2} 
\nn\\
&+2m\up\big{(}-2(p\us\cdot k)\cM_{5},_{s}-2(p\ua\cdot k)\cM_{5},_{m\ua\2}-2(p\uu\cdot k)\cM_{5},_{m\uu\2}\big{)}\nn\\
&-2m\up\big{(}-2(p\us\cdot k)\cM_{7},_{s}-2(p\ua\cdot k)\cM_{7},_{m\ua\2}-2(p\uu\cdot k)\cM_{7},_{m\uu\2}\big{)}\nn\\
&+\big{(}2m\up\2+2(p'\ud\cdot p\ub)\big{)}\big{(}-2(p\us\cdot k)\cM_{8},_{s}-2(p\ua\cdot k)\cM_{8},_{m\ua\2}\nn\\
&\quad-2(p\uu\cdot k)\cM_{8},_{m\uu\2}\big{)} \Big{]}\nn\\
&-\bk\Big{[} \cM_{2}^{\text{(on)}}+ \cM_{4}^{\text{(on)}}+2m\up \cM_{5}^{\text{(on)}}-2m\up \cM_{7}^{\text{(on)}}\nn\\
&\quad+\big{(}2m\up\2+2(p'\ud\cdot p\ub)\big{)}\cM_{8}^{\text{(on)}}\Big{]}\bigg{\rbrace}(\bp\ub+m\up)
+{\mathcal O}(\omega\2)\,.
\end{align}
%============================
%\end{widetext}
%============================
The r.h.s of \er{C20} is, to order $\omega$, a homogeneous linear function of $k\, $. Therefore, we get from \er{C20}, up to the order $\omega^{0}$, a unique solution for $\cN\ul^{(e1)}$:
\vspace{-0.3cm}
\bal{C21}
&\cN\ul^{(e1)}=-e(\bp\ud+m\up)\Big{\lbrace}\nn\\
&-2p_{s\lambda}\cM_{1},_{s}-2p_{a\lambda}\cM_{1},_{m\ua\2}-2p_{1\lambda}\cM_{1},_{m\uu\2}\nn\\
&+m\up\Big{[}-2p_{s\lambda}\cM_{2},_{s}-2p_{a\lambda}\cM_{2},_{m\ua\2}-2p_{1\lambda}\cM_{2},_{m\uu\2}\Big{]}\nn\\
&-m\up\Big{[}-2p_{s\lambda}\cM_{4},_{s}-2p_{a\lambda}\cM_{4},_{m\ua\2}-2p_{1\lambda}\cM_{4},_{m\uu\2}\Big{]}\nn\\
%
&+\Big{[} -\frac{1}{2}(p\ua +p\uu\, ,\, p\ub+p\ud)-(p\ub\cdot p\ud)+m\up\2\Big{]}\nn\\
&\quad \times \Big{[}-2p_{s\lambda}\cM_{5},_{s}-2p_{a\lambda}\cM_{5},_{m\ua\2}-2p_{1\lambda}\cM_{5},_{m\uu\2}\Big{]}\nn\\
&+(p\ub +p\ud )_{\lambda}\cM_{5}^{\text{(on)}}\nn\\
&+\frac{1}{2}(p\ua +p\uu\, ,\, p\ud-p\ub)\Big{[}-2p_{a\lambda}\cM_{6},_{m\ua\2}-2p_{1\lambda}\cM_{6},_{m\uu\2}\Big{]}\nn\\
&+\Big{[}\frac{1}{2}(p\ua +p\uu\, ,\, p\ub+p\ud)-(p\ub\cdot p\ud)+m\up\2\Big{]}\nn\\
&\quad \times \Big{[}-2p_{s\lambda}\cM_{7},_{s}-2p_{a\lambda}\cM_{7},_{m\ua\2}-2p_{1\lambda}\cM_{7},_{m\uu\2}\Big{]}\nn\\
&-(p\ub +p\ud )_{\lambda}\cM_{7}^{\text{(on)}}
%\nn\\
%&
-m\up(p\ua+p\uu\, ,\, p\ub+p\ud) \nn\\
&\quad \times
\Big{[}-2p_{s\lambda}\cM_{8},_{s}-2p_{a\lambda}\cM_{8},_{m\ua\2}
-2p_{1\lambda}\cM_{8},_{m\uu\2}\Big{]}
\nn\\
&+2m\up (p\ub +p\ud )_{\lambda}\cM_{8}^{\text{(on)}}
\nn\\
&+\frac{1}{2}(\bp\ua+\bp\uu)
\Big{[}-2p_{s\lambda}\cM_{2},_{s}-2p_{a\lambda}\cM_{2},_{m\ua\2}
-2p_{1\lambda}\cM_{2},_{m\uu\2}
\nn\\
&\quad -2p_{s\lambda}\cM_{4},_{s}-2p_{a\lambda}\cM_{4},_{m\ua\2}
-2p_{1\lambda}\cM_{4},_{m\uu\2} 
\nn\\
&\quad +2m\up\big{(}-2p_{s\lambda}\cM_{5},_{s}-2p_{a\lambda}\cM_{5},_{m\ua\2}-2p_{1\lambda}\cM_{5},_{m\uu\2}\big{)}\nn\\
&\quad -2m\up\big{(}-2p_{s\lambda}\cM_{7},_{s}-2p_{a\lambda}\cM_{7},_{m\ua\2}-2p_{1\lambda}\cM_{7},_{m\uu\2}\big{)}\nn\\
&\quad +\big{(} 2m\up\2+2(p\ud\cdot p\ub)\big{)} \big{(}-2p_{s\lambda}\cM_{8},_{s}-2p_{a\lambda}\cM_{8},_{m\ua\2}\nn\\
&\quad -2p_{1\lambda}\cM_{8},_{m\uu\2}\big{)}\Big{]}\nn\\
&-\gamma\ul\Big{[} \cM_{2}^{\text{(on)}}+ \cM_{4}^{\text{(on)}}+2m\up \cM_{5}^{\text{(on)}}-2m\up \cM_{7}^{\text{(on)}}\nn\\
&\quad +\big{(}2m\up\2+2(p\ud\cdot p\ub)\big{)}\cM_{8}^{\text{(on)}}\Big{]}
\Big{\rbrace}(\bp\ub+m\up)
+{\mathcal O}(\omega)\,.
\end{align}

Our next task is to derive the expansion
in $\omega$ for $\cN\ul^{(a)}$ 
\er{C9} using \er{C11} and \er{3.40}. 
Similarly we treat $\cN\ul^{(b)}$ \er{C12} 
using \er{C13}--\er{C15}, and \er{C17}. 
The results are as follows.
\bal{C22}
&\cN\ul^{(a)}=-e(\bp'\ud +m\up)\bigg{\lbrace} 
\cM_{1}^{\text{(on)}}+m\up \cM_{2}^{\text{(on)}}-m\up \cM_{4}^{\text{(on)}}\nn\\
&+\Big{[}-\frac{1}{2}(p\ua +p'\uu\, ,\, p\ub+p'\ud)-(p\ub\cdot p'\ud)+m\up\2\Big{]}\cM_{5}^{\text{(on)}}\nn\\
&+\Big{[}\frac{1}{2}(p\ua +p'\uu\, ,\, p\ub+p'\ud)-(p\ub\cdot p'\ud)+m\up\2\Big{]}\cM_{7}^{\text{(on)}}\nn\\
&-m\up (p\ua +p'\uu\, ,\, p\ub+p'\ud)\cM_{8}^{\text{(on)}}\nn\\
&+\frac{1}{2}(\bp\ua+\bp'\uu)\Big{[}\cM_{2}^{\text{(on)}}+\cM_{4}^{\text{(on)}}
+2m\up\cM_{5}^{\text{(on)}}-2m\up \cM_{7}^{\text{(on)}}\nn\\
&+\left( 2m\up\2+2(p'\ud\cdot p\ub)\right) 
\cM_{8}^{\text{(on)}}\Big{]}\bigg{\rbrace}
(\bp\ub+m\up)\frac{(2p\ua -k)\ul}{-2p\ua\cdot k +k\2}
\nn\\
&-e(\bp\ud+m\up)
\bigg{\lbrace}-2(p\us\cdot k)\cM_{1},_{s}-2(p\ut\cdot l\ud)\cM_{1},_{t}\nn\\
& \quad  -2(p\ua\cdot k)\cM_{1},_{m\ua\2}\nn\\
&+m\up\big{(}-2(p\us\cdot k)\cM_{2},_{s}-2(p\ut\cdot l\ud)\cM_{2},_{t}-2(p\ua\cdot k)\cM_{2},_{m\ua\2}\big{)}\nn\\
&-m\up\big{(}-2(p\us\cdot k)\cM_{4},_{s}-2(p\ut\cdot l\ud)\cM_{4},_{t}-2(p\ua\cdot k)\cM_{4},_{m\ua\2}\big{)}\nn\\
&+\Big{[}-\frac{1}{2}(p\ua +p\uu\, ,\, p\ub+p\ud)-(p\ub\cdot p\ud)+m\up\2\Big{]}\nn\\
&\quad \times \Big{[}-2(p\us\cdot k)\cM_{5},_{s}-2(p\ut\cdot l\ud)\cM_{5},_{t}-2(p\ua\cdot k)\cM_{5},_{m\ua\2}\Big{]}\nn\\
&+\frac{1}{2}(k\, ,\, p\ub+ p\ud)\cM_{5}^{\text{(on)}}
\nn\\
&+\frac{1}{2}(p\ua +p\uu\,, \,p\ud-p\ub)\big{(}-2(p\ua \cdot k)\cM_{6},_{m\ua\2}\big{)}\nn\\
&+\Big{[}\frac{1}{2}(p\ua +p\uu\, ,\, p\ub+p\ud)-(p\ub\cdot p\ud)+m\up\2\Big{]}\nn\\
&\quad \times \Big{[}-2(p\us \cdot k)\cM_{7},_{s}-2(p\ut\cdot l\ud)\cM_{7},_{t}-2(p\ua\cdot k)\cM_{7},_{m\ua\2}\Big{]}\nn\\
&-\frac{1}{2}(k\, ,\, p\ub+ p\ud)\cM_{7}^{\text{(on)}}
-m\up(p\ua +p\uu\, ,\, p\ub+p\ud)
\nn\\
&\quad \times \Big{[}-2(p\us\cdot k)\cM_{8},_{s}-2(p\ut\cdot l\ud)\cM_{8},_{t}-2(p\ua\cdot k)\cM_{8},_{m\ua\2}\Big{]}\nn\\
&+m\up (k\, ,\, p\ub+ p\ud)\cM_{8}^{\text{(on)}}+\frac{1}{2}(\bp\ua +\bp \uu) \nn \\
& \quad \times
\Big{[}
-2(p\us\cdot k)\cM_{2},_{s}-2(p\ut\cdot l\ud)\cM_{2},_{t}-2(p\ua\cdot k)\cM_{2},_{m\ua\2}\nn\\
&\qquad \;\;\, 
-2(p\us\cdot k)\cM_{4},_{s}-2(p\ut\cdot l\ud)\cM_{4},_{t}-2(p\ua\cdot k) \cM_{4},_{m\ua\2}
\nn\\
&+2m\up \big{(}-2(p\us\cdot k)\cM_{5},_{s}-2(p\ut\cdot l\ud)\cM_{5},_{t}-2(p\ua\cdot k)\cM_{5},_{m\ua\2}\big{)}\nn\\
&-2m\up \big{(}-2(p\us\cdot k)\cM_{7},_{s}-2(p\ut\cdot l\ud)\cM_{7},_{t}-2(p\ua\cdot k)\cM_{7},_{m\ua\2}\big{)}\nn\\
&
+\big{(}2m\up\2+2(p\ud\cdot p\ub)\big{)}\nn\\
&\quad \times \big{(}-2(p\us\cdot k)\cM_{8},_{s}-2(p\ut\cdot l\ud)\cM_{8},_{t}-2(p\ua\cdot k)\cM_{8},_{m\ua\2}\big{)}
\Big{]}
\nn\\
&-\frac{1}{2}\bk \Big{[}\cM_{2}^{\text{(on)}}+\cM_{4}^{\text{(on)}}+2m\up\cM_{5}^{\text{(on)}}-2m\up \cM_{7}^{\text{(on)}}\nn\\
&\quad +\big{(}2m\up\2+2(p\ud\cdot p\ub)\big{)}\cM_{8}^{\text{(on)}}\Big{]}\bigg{\rbrace}
(\bp\ub+m\up)\frac{2p_{a\lambda}}{(-2p\ua\cdot k )}
\nn\\
&+{\mathcal O}(\omega)
\,.\\
\nn\\
\label{C23}
&\cN\ul^{(b)}=-e(\bp'\ud +m\up)\bigg{\lbrace}
\cM_{1}^{\text{(on)}}+m\up \cM_{2}^{\text{(on) }}-m\up \cM_{4}^{\text{(on)}}
\nn\\
&+\Big{[}-\frac{1}{2}(p\ua +p'\uu\, ,\, p\ub+p'\ud)-(p\ub\cdot p'\ud)+m\up\2\Big{]}\cM_{5}^{\text{(on)}}\nn\\
&+\Big{[}\frac{1}{2}(p\ua +p'\uu\, ,\, p\ub+p'\ud)-(p\ub\cdot p'\ud)+m\up\2\Big{]}\cM_{7}^{\text{(on)}}\nn\\
&-m\up (p\ua +p'\uu\, ,\, p\ub+p'\ud)\cM_{8}^{\text{(on)}}\nn\\
&+\frac{1}{2}(\bp\ua+\bp'\uu)\Big{[}\cM_{2}^{\text{(on)}}+\cM_{4}^{\text{(on)}}+2m\up\cM_{5}^{\text{(on)}}
-2m\up \cM_{7}^{\text{(on)}}\nn\\
&\quad
+\big{(} 2m\up\2+2(p'\ud\cdot p\ub)\big{)} 
\cM_{8}^{\text{(on)}}\Big{]}\bigg{\rbrace}
(\bp\ub+m\up)\frac{(2p'\uu+k)\ul}{2p'\uu\cdot k +k\2}
\nn\\
&-e(\bp\ud+m\up)
\bigg{\lbrace}-2(p\ut\cdot l\ud)\cM_{1},_{t}+2(p\uu\cdot k)\cM_{1},_{m\uu\2}\nn\\
&+m\up\Big{(}-2(p\ut\cdot l\ud)\cM_{2},_{t}+2(p\uu\cdot k)\cM_{2},_{m\uu\2}\Big{)}\nn\\
&-m\up\Big{(}-2(p\ut\cdot l\ud)\cM_{4},_{t}+2(p\uu\cdot k)\cM_{4},_{m\uu\2}\Big{)}\nn\\
&+\Big{[}-\frac{1}{2}(p\ua +p\uu\, ,\, p\ub+p\ud)-(p\ub\cdot p\ud)+m\up\2\Big{]}\nn\\
&\quad \times \Big{[}-2(p\ut\cdot l\ud)\cM_{5},_{t}+2(p\uu\cdot k)\cM_{5},_{m\uu\2}\Big{]}\nn\\
&-\frac{1}{2}(k\, , \, p\ub+ p\ud)\cM_{5}^{\text{(on)}}\nn\\
&+\frac{1}{2}(p\ua +p\uu\, ,\, p\ud-p\ub)\,  2(p\uu \cdot k)\cM_{6},_{m\uu\2}\nn\\
&+\Big{[}\frac{1}{2}(p\ua +p\uu\, ,\, p\ub+p\ud)-(p\ub\cdot p\ud)+m\up\2\Big{]}\nn\\
&\quad \times \Big{[}-2(p\ut\cdot l\ud)\cM_{7},_{t}+2(p\uu\cdot k)\cM_{7},_{m\uu\2}\Big{]}\nn\\
&+\frac{1}{2}(k\, , \,  p\ub+ p\ud)\cM_{7}^{\text{(on)}}
-m\up(p\ua +p\uu\, ,\, p\ub+p\ud)\nn\\
&\quad \times \Big{[}-2(p\ut\cdot l\ud)\cM_{8},_{t}+2(p\uu\cdot k)\cM_{8},_{m\uu\2}\Big{]}\nn\\
&-m\up (k\, , \,  p\ub+ p\ud)\cM_{8}^{\text{(on)}}\nn\\
&+\frac{1}{2}(\bp\ua +\bp \uu)
\Big{[}-2(p\ut\cdot l\ud)\cM_{2},_{t}+2(p\uu\cdot k )\cM_{2},_{m\uu\2}\nn\\
&-2(p\ut\cdot l\ud)\cM_{4},_{t}+2(p\uu\cdot k)\cM_{4},_{m\uu\2}\nn\\
&+2m\up \Big{(}-2(p\ut\cdot l\ud)\cM_{5},_{t}+2(p\uu\cdot k)\cM_{5},_{m\uu\2}\Big{)}\nn\\
&-2m\up \Big{(}-2(p\ut\cdot l\ud)\cM_{7},_{t}+2(p\uu\cdot k)\cM_{7},_{m\uu\2}\Big{)}\nn\\
&
+\big{(}2m\up\2+2(p\ud\cdot p\ub)\big{)}\Big{(}-2(p\ut\cdot l\ud)\cM_{8},_{t}+2(p\uu\cdot k)\cM_{8},_{m\uu\2}\Big{)}\Big{]}\nn\\
&+\frac{1}{2}\bk \Big{[}\cM_{2}^{\text{(on)}}+\cM_{4}^{\text{(on)}}+2m\up\cM_{5}^{\text{(on)}}-2m\up \cM_{7}^{\text{(on)}}\nn\\
&\quad+\big{(}2m\up\2+2(p\ud\cdot p\ub)\big{)}\cM_{8}^{\text{(on)}}\Big{]}\bigg{\rbrace}
(\bp\ub+m\up)\frac{2p_{1\lambda}}{2p\uu\cdot k }
+{\mathcal O}(\omega)\,.
\end{align}
The expressions of \er{C21}--\er{C23} look frightening. 
But it turns out that the sum 
\bel{C24}
\cN\ul^{(a+b+e1)-}=\cN\ul^{(a)}+\cN\ul^{(b)}+\cN\ul^{(e1)}
\ee
is rather simple. We note first that in the sum \er{C24} the terms with
 $\cM_{j},_{m\uu\2}$ and  $\cM_{j},_{m\ua\2}$, that is, the derivatives in off-shell directions drop out. The remaining terms in the sum \er{C24} can be greatly simplified using \er{A10} and the following relations which are valid to the order $\omega$:
\begin{widetext}
\bal{C25}
&\frac{1}{2}(l\uu \, ,\, p\ub +p\ud)+\frac{1}{2}(l\ud\, ,\,  p\ua +p\uu)+(l\ud\cdot p\ub)+\frac{1}{2}(k\, ,\, p\ub +p\ud)-2(p\us\cdot k)=-(k\cdot p\ua)\,, \nn\\
&-\frac{1}{2}(l\uu\, ,\, p\ub 
+p\ud)-\frac{1}{2}(l\ud\, ,\, p\ua +p\uu)
+(l\ud\cdot p\ub)
+2(k \, ,\, p\ua +p\ub)
+2(l\ud \, ,\, p\ud - p\ub)
-\frac{1}{2}(k \, ,\, p\ub+p\ud)=(k\cdot p\ua)\,, \nn\\
&(l\uu \, ,\, p\ub +p\ud)+(l\ud\, ,\, p\ua +p\uu)-2(l\ud\cdot p\ut)-4(k\cdot p\us)+ (k\, ,\,p\ub+p\ud)=-2(k\cdot p\ua)\,.
\end{align}
In the proof of \er{C25} we make frequent use of \er{3.9}. 
With the help of \er{2.24}, \er{2.25}, \er{C5}--\er{C8}, and \er{C25} we can now greatly simplify the sum \er{C24} of the frightening expressions \er{C21}--\er{C23}. 
The result is given in \er{3.43}.
\end{widetext}

\subsection{The term $(c+d+e2)$}
\label{subsec:C2}

The term $\cN\ul^{(c)}$ corresponding to 
the diagram of Fig.~\ref{fig:3}(c) is defined in \er{3.22}. 
From \er{3.6}, \er{3.26}, and \er{3.30} we have
\begin{align}
\label{C26}
k^{\lambda}\cN\ul^{(c)}&=-e\,\cN^{(0,c)}\nn\\
& =-e (\bp'\ud +m\up )\cM^{(0,c)}(\bp\ub +m\up )\,, \\
\label{C27}
\cM^{(0,c)}&=\cM^{(0)}(p'\uu\, ,\, p'\ud \, ,\, p\ua\, ,\, p\ub -k)\,,\nn\\
p'\uu &= p\uu-l\uu \,, \nn \\
p'\ud &= p\ud-l\ud \,.
\end{align}
With \er{2.16} and \er{2.17} we find
\bal{C28}
&\cN^{(0, c)}=(\bp'\ud +m\up)\bigg{\lbrace}  
\cM_{1}^{(c)}
+\Big{[}m\up +\frac{1}{2}(\bp\ua +\bp'\uu-\bk)\Big{]}\cM_{2}^{(c)}
\nn\\
&+\bk \cM_{3}^{(c)}
+\Big{[}-m\up +\frac{1}{2}(\bp\ua +\bp'\uu+\bk)\Big{]}\cM_{4}^{(c)}\nn\\
&+\Big{[}-\frac{1}{2}(p\ua +p'\uu -k\,  ,\, p\ub+p'\ud)-(p\ub\cdot p'\ud)+m\up\2\nn\\
&\quad +m\up(\bp\ua +\bp'\uu-\bk)-\frac{1}{2}(p\ub-p'\ud\, ,\, k) \nn\\
&\quad 
-\dfrac{1}{4}\Big{(}(\bp\ua+\bp'\uu)\!\bk \;
-\bk (\bp\ua+\bp'\uu)\Big{)}\Big{]}\cM_{5}^{(c)}\nn\\
&+\Big{[}-\frac{1}{2}(p\ua +p'\uu\, ,\, p\ub-p'\ud)
\nn\\
&\quad 
-\dfrac{1}{4}\Big{(}(\bp\ua+\bp'\uu)\!\bk \;
-\bk (\bp\ua+\bp'\uu)\Big{)}\Big{]}\cM_{6}^{(c)}\nn\\
&+\Big{[}\frac{1}{2}(p\ua +p'\uu +k\,  ,\, p\ub+p'\ud)-(p\ub\cdot p'\ud)+m\up\2\nn\\
&\quad -m\up(\bp\ua +\bp'\uu+\bk)
-\frac{1}{2}(p\ub-p'\ud\, ,\, k) 
\nn\\
&\quad +\dfrac{1}{4}\Big{(}(\bp\ua+\bp'\uu)\!\bk \;
-\bk (\bp\ua+\bp'\uu)\Big{)}\Big{]}\cM_{7}^{(c)}\nn\\
&+\Big{[}-m\up (p\ua +p'\uu\, ,\, p\ub+p'\ud)+(\bp\ua+\bp'\uu)\big{(}m\up\2 +(p'\ud\cdot p\ub)\big{)}\nn\\
&\quad+(p\ua +p'\uu\, ,\, p'\ud)\bk - (p'\ud\cdot k )(\bp\ua +\bp'\uu)\nn\\
&-\dfrac{1}{2}m\up\Big{(}(\bp\ua+\bp'\uu)\!\bk \; -\bk (\bp\ua+\bp'\uu)\Big{)}\Big{]}\cM_{8}^{(c)}\bigg{\rbrace}
(\bp\ub+m\up) \nn \\
&+{\mathcal O}(\omega\2)\,,
\end{align}
where
\bal{C29}
\cM_{j}^{(c)}&=
\cM_{j}\Big{(}s-2(p\us\cdot k), \, 
t+2(p\ut \cdot l\uu)\ ,\nn\\
&\qquad \qquad m\upp\2 , \, m\up\2 , \, m\upp\2 , \, 
m\up\2-2(p\ub\cdot k)\Big{)}+{\mathcal O}(\omega\2)\nn\\
&=\cM_{j}^{\text{(on)}}-2(p\us\cdot k)\cM_{j},_{s}
+2(p\ut\cdot l\uu)\cM_{j},_{t}\nn\\
&\quad -2(p\ub\cdot k)\cM_{j},_{m\ub\2}
+{\mathcal O}(\omega\2)\ ,\nn\\
(j&=1, \dots , 8)\,;
\end{align}
see \eqref{C4}.
%Anfang Seite C-37, C Teil3
From \er{B81} and \er{B86} we get
\bal{C30}
&S_{F}(p\ub -k) \widehat{\Gamma}\ul^{(\gamma pp)} (p\ub - k\, ,\,  p\ub)(\bp\ub + m\up)\nn\\
&=\frac{\bp\ub + m\up - \!\!\bk}{-2p\ub\cdot k + k\2 + i\varepsilon}\Big{[}\gamma\ul - i\sigma_{\lambda\nu} k^{\nu}\frac{1}{2m\up}F\ud (0) \Big{]} (\bp\ub + m\up)\nn\\
&\quad +{\mathcal O}(\omega)\nn\\
&=\frac{1}{-2p\ub\cdot k + k\2 + i\varepsilon}\bigg{\lbrace}2p_{b\lambda}-k\ul
+(k\ul - \! \bk \gamma\ul)\big{[}1+F\ud (0)\big{]}\nn\\
&\quad + \frac{F\ud(0)}{m \up}\Big{[}p_{b\lambda} \!\! \bk - (p\ub \cdot k)\gamma\ul\Big{]}\bigg{\rbrace}(\bp\ub + m\up)\nn\\
&\quad +{\mathcal O}(\omega)\,.
\end{align}
Inserting this in \er{3.22} we obtain
\bel{C31}
\cN\ul^{(c)}=\cN\ul^{(c1)}+\cN\ul^{(c2)}+\cN\ul^{(c3)}+{\mathcal O}(\omega)\,,
\ee
where
\bal{C32}
\cN\ul^{(c1)}&=e(\bp'\ud + m\up)\cM^{(0,c)}(\bp\ub +m\up)\frac{(2p\ub -k)\ul}{-2p\ub\cdot k + k\2}\,,\\
\label{C33}
\cN\ul^{(c2)}&=e(\bp\ud + m\up)\cM^{(0)}(k\ul -\! \bk\gamma\ul)(\bp\ub +m\up)\nn\\
&\quad \times \frac{1}{-2p\ub\cdot k}\big{[} 1+F\ud (0)\big{]}\,,\\
\label{C34}
\cN\ul^{(c3)}&=e(\bp\ud + m\up)\cM^{(0)}\big{[}p_{b\lambda}\! \bk - (p\ub \cdot k)\gamma\ul\big{]} (\bp\ub + m\up)
\nn\\
&\quad \times \frac{1}{-2p\ub\cdot k}\frac{F\ud (0)}{m\up}\,.
\end{align}
Inserting here \er{2.16}, \er{2.17} we get
\bal{C35}
\cN\ul^{(c1)}&=e \, \cN^{(0, c)}\frac{(2p\ub -k)\ul}{-2p\ub\cdot k + k\2}+{\mathcal O}(\omega)\,,\\
\label{C36}
\cN\ul^{(c2)}&=e(\bp\ud +m\up)\Big{\lbrace} \cM\uu^{(\text{on})}+\bp\us\cM\ud^{(\text{on})}
+ \bp_{u} \cM_{4}^{(\text{on})}\nn\\
&\quad +i\sigma_{\mu\nu}p_{s}^{\mu}p\ut^{\nu}\cM_{5}^{(\text{on})}
+i\sigma_{\mu\nu}p_{t}^{\mu}p_{u}^{\nu}\cM_{7}^{(\text{on})}\nn\\
&\quad +i\gamma_{\mu}\gamma_{5}\varepsilon^{\mu\nu\rho\sigma}p_{s\nu}p_{t\rho}p_{u\sigma}\cM_{8}^{(\text{on})}\Big{\rbrace}\nn\\
&\quad \times (k\ul -\bk \gamma\ul)(\bp\ub + m\up)\frac{1+F\ud(0)}{-2p\ub \cdot k}\,,\\
\label{C37}
\cN\ul^{(c3)}&=e(\bp\ud +m\up)\Big{\lbrace} \cM\uu^{(\text{on})}+\bp\us\cM\ud^{(\text{on})}
+ \bp_{u} \cM_{4}^{(\text{on})}\nn\\
&\quad +i\sigma_{\mu\nu}p_{s}^{\mu}p\ut^{\nu}\cM_{5}^{(\text{on})}
+i\sigma_{\mu\nu}p_{t}^{\mu}p_{u}^{\nu}\cM_{7}^{(\text{on})}\nn\\
&\quad +i\gamma_{\mu}\gamma_{5}\varepsilon^{\mu\nu\rho\sigma}p_{s\nu}p_{t\rho}p_{u\sigma}\cM_{8}^{(\text{on})}\Big{\rbrace}\nn\\
&\quad \times \big{[}p_{b\lambda}\bk - (p\ub \cdot k)\gamma\ul\big{]}
 (\bp\ub + m\up)\frac{F\ud (0)}{m\up}\frac{1}{(-2p\ub\cdot k)}\,.
\end{align}

Now we discuss $\cN\ul^{(d)}$, corresponding to 
the diagram of Fig.~\ref{fig:3}(d), 
as defined in \er{3.23}. 
From \er{3.6}, \er{3.27}, and \er{3.31} we find
\bal{C38}
k^{\lambda}\cN\ul^{(d)}&=e\, \cN^{(0,d)}\nn\\
&=e(p'\ud +m\up)\cM^{(0,d)}(\bp\ub +m\up)\,,
\\
\label{C39}
\cM^{(0,d)}&=\cM^{(0)}(p'\uu\,,\, p'\ud + k\,,\, p\ua\,,\, p\ub )\,,
\nn\\
p'\uu&=p\uu-l\uu\,, \nn \\
p'\ud&= p\ud-l\ud\,.
\end{align}
With \er{2.16}, \er{2.17}, and \er{C4}, we find here
\bal{C40}
&\cN^{(0,d)}=
(\bp'\ud + m\up) \Big{\lbrace}\cM\uu^{(d)}
+\Big{[} 
m\up +\frac{1}{2}(\bp\ua +\bp'\uu+\bk)\Big{]}\cM_{2}^{(d)}\nn\\
&+\bk \cM_{3}^{(d)}
+\Big{[} 
-m\up +\frac{1}{2}(\bp\ua +\bp'\uu-\bk)\Big{]}\cM_{4}^{(d)}\nn\\
&+\Big{[}-\frac{1}{2}(p\ua +p'\uu +k\,  ,\, p\ub+p'\ud)-(p\ub\cdot p'\ud)+m\up\2\nn\\
&\quad +m\up(\bp\ua +\bp'\uu+\bk)-\frac{1}{2}(p\ub-p'\ud\, ,\, k) \nn\\
&\quad -\dfrac{1}{4}\Big{(}(\bp\ua+\bp'\uu)\!\bk \;-\bk (\bp\ua+\bp'\uu)\Big{)}\Big{]}\cM_{5}^{(d)}\nn\\
&+\Big{[}-\frac{1}{2}(p\ua +p'\uu\, ,\, p\ub-p'\ud)\nn\\
&\quad +\dfrac{1}{4}
\Big{(}(\bp\ua+\bp'\uu)\!\bk \;-\bk (\bp\ua+\bp'\uu)\Big{)}\Big{]}\cM_{6}^{(d)}
\nn\\
&+\Big{[}\frac{1}{2}(p\ua +p'\uu -k\,  ,\, p\ub+p'\ud)-(p\ub\cdot p'\ud)+m\up\2
\nn\\
&\quad -m\up(\bp\ua +\bp'\uu-\bk)-\frac{1}{2}(p\ub-p'\ud\, ,\, k) \nn\\
&\quad +\dfrac{1}{4}\Big{(}(\bp\ua+\bp'\uu)\!\bk \; -\bk (\bp\ua+\bp'\uu)\Big{)}\Big{]}\cM_{7}^{(d)}\nn\\
&+\Big{[}-m\up (p\ua +p'\uu\, ,\, p\ub+p'\ud)+(\bp\ua+\bp'\uu)\big{(}m\up\2 +(p'\ud\cdot p\ub)\big{)}\nn\\
&\quad + (p\ub\cdot k )(\bp\ua +\bp'\uu)-(p\ua +p'\uu\, ,\, p\ub)\bk\nn\\
&\quad -\dfrac{1}{2}m\up\Big{(}(\bp\ua+\bp'\uu)\!\bk \; -\bk (\bp\ua+\bp'\uu)\Big{)}\Big{]}\cM_{8}^{(d)}\bigg{\rbrace}(\bp\ub+m\up)
\nn\\
&+{\mathcal O}(\omega\2)\,,
\end{align}
where
\bal{C41}
\cM_{j}^{(d)}&=\cM_{j}\big{(}s, \, t+2p\ut \cdot l\uu , \, 
  m\upp\2 ,\,  m\up\2+2 p\ud\cdot k,
  m\upp\2 ,\,  m\up\2 \big{)}
  \nn\\
&\quad +{\mathcal O}(\omega\2)\nn\\
&=\cM_{j}^{\text{(on)}}+2(p\ut\cdot l\uu)\cM_{j},_{t}\ +2(p\ud\cdot k)\cM_{j},_{m\ud\2}\nn\\
&\quad +{\mathcal O}(\omega\2)\,.
\end{align}
From \er{B85} and \er{B87} we find
\bal{C42}
&(\bp'\ud+m\up)\, \widehat{\Gamma}
\ul^{(\gamma pp)}(p'\ud\, ,\, p'\ud +k)S_{F}(p'\ud + k)\nn\\
&=(\bp'\ud +m\up)\Big{[}\gamma\ul -\frac{i}{2m\up}\sigma_{\lambda\nu} k^{\nu}F\ud(0)\Big{]}
\frac{\bp'\ud +m\up + \! \bk}{2p'\ud\cdot k +k\2 +i\varepsilon} \nn\\
&\quad +{\mathcal O}(\omega)\nn\\
&=\frac{1}{2p'\ud\cdot k +k\2 +i\varepsilon}(\bp'\ud +m\up)\bigg{\lbrace}2p'_{2\lambda}+k\ul\nn\\
&\quad -(k\ul -\gamma\ul \!\bk) \big{[}1+F_{2}(0)\big{]}\nn\\
&\quad -\frac{F_{2}(0)}{m\up} \big{(}p'_{2\lambda}\!\bk - (p'\ud\cdot k) \gamma\ul\big{)}\bigg{\rbrace}+{\mathcal O}(\omega)\,.
\end{align}
Inserting \er{C42} into \er{3.23} we obtain
\bal{C43}
\cN\ul^{(d)}&=\cN\ul^{(d1)}+\cN\ul^{(d2)}+\cN\ul^{(d3)}+{\mathcal O}(\omega)\,,
\end{align}
where
\bal{C44}
\cN\ul^{(d1)}=&\;e\frac{2p'_{2\lambda}+k\ul}{2p'\ud\cdot k +k\2 } (\bp'\ud +m\up) \cM^{(0,d)}(\bp\ub +m\up)\,,\\
\label{C45}
\cN\ul^{(d2)}=&-e\big{[}1+F\ud(0)\big{]}\frac{1}{2 p\ud \cdot k} \nn \\
&\times (\bp\ud +m\up)(k\ul-\gamma\ul \!\bk)\cM^{(0)}(\bp\ub +m\up)\,,\\
\label{C46}
\cN\ul^{(d3)}=&-e\frac{F\ud(0)}{m\up}\, \frac{1}{2p\ud\cdot k}\nn\\
&\times (\bp\ud+m\up) \big{(}p_{2\lambda} \! \bk - (p\ud\cdot k) \gamma\ul\big{)}
\cM^{(0)}(\bp\ub +m\up) \,.
\end{align}
With \er{2.16}, \er{2.17}, \er{C4}, \er{C38}--\er{C41} 
we find
\bal{C47}
\cN\ul^{(d1)}=&\;
e\frac{(2p'\ud +k)\ul}{2p'\ud\cdot k +k\2 }\; 
\cN^{(0,d)}\,,\\
\label{C48}
\cN\ul^{(d2)}=&-e\big{[}1+F\ud(0)\big{]}\, \frac{1}{2p\ud\cdot k}(\bp\ud+m\up) (k\ul-\gamma\ul \bk)\nn\\
&\times \Big{\lbrace}\cM\uu^{(\text{on})}
+\bp\us\cM\ud^{(\text{on})}
+\bp_{u} \cM_{4}^{(\text{on})}\nn\\
&+i\sigma_{\mu\nu}p_{s}{}^{\mu}p\ut{}^{\nu}\cM_{5}^{(\text{on})}
+i\sigma_{\mu\nu}p_{t}{}^{\mu}p_{u}{}^{\nu}\cM_{7}^{(\text{on})}\nn\\
&+i\gamma_{\mu}\gamma_{5}\varepsilon^{\mu\nu\rho\sigma}p_{s\nu}p_{t\rho}p_{u\sigma}\cM_{8}^{(\text{on})}\Big{\rbrace} (\bp\ub + m\up)
\,,\\
\label{C49}
\cN\ul^{(d3)}=&-e\frac{F\ud(0)}{m\up}\, \frac{1}{2p\ud\cdot k}(\bp\ud+m\up) \big{[}p_{2\lambda}\bk - (p\ud\cdot k) \gamma\ul\big{]}\nn\\
&\times \Big{\lbrace}\cM\uu^{(\text{on})}
+\bp\us\cM\ud^{(\text{on})}
+\bp_{u} \cM_{4}^{(\text{on})}\nn\\&
+i\sigma_{\mu\nu}p_{s}{}^{\mu}p\ut{}^{\nu}\cM_{5}^{(\text{on})}
+i\sigma_{\mu\nu}p_{t}{}^{\mu}p_{u}{}^{\nu}\cM_{7}^{(\text{on})}\nn\\
&+i\gamma_{\mu}\gamma_{5}\varepsilon^{\mu\nu\rho\sigma}p_{s\nu}p_{t\rho}p_{u\sigma}\cM_{8}^{(\text{on})}\Big{\rbrace} (\bp\ub + m\up)\,.
\end{align}

Now we come to the determination of $\cN\ul^{(e)}$ which corresponds to the diagram of Fig.~\ref{fig:3}(e). 
We have from \er{3.32}
\bal{C50}
k^{\lambda}\cN\ul^{(e)}&=-e\Big{(}\cN^{(0,a)}-\cN^{(0,b)}\Big{)}
+e\Big{(}\cN^{(0,c)}-\cN^{(0,d)}\Big{)}\,.
 \end{align}
$-e\Big{(}\cN^{(0,a)}-\cN^{(0,b)}\Big{)}$ 
is given in \er{C19}, \er{C20}, 
and to order $\omega$ was found 
to be a homogeneous linear function of $k$.
The same is true for $e\Big{(}\cN^{(0,c)}-\cN^{(0,d)}\Big{)}$ for which we find from \er{C28} and \er{C40}
\bal{C51}
&e\Big{[}\cN^{(0,c)}-\cN^{(0,d)}\Big{]}=e(\bp'\ud +m\up)\bigg{\lbrace}
\cM\uu^{(c)}-\cM\uu^{(d)}\nn\\
&+\Big{[}m\up+\frac{1}{2}(\bp\ua +\bp'\uu)\Big{]}\Big{[}\cM\ud^{(c)}-\cM\ud^{(d)}\Big{ ]}\nn\\
&-\bk\cM\ud^{(\text{on})}
+\Big{[}-m\up+\frac{1}{2}(\bp\ua +\bp'\uu)\Big{]}\Big{[}\cM_{4}^{(c)}-\cM_{4}^{(d)}\Big{ ]}\nn\\
&+\bk\cM_{4}^{(\text{on})}
+\Big{[}-\frac{1}{2}(p\ua +p'\uu\,  ,\, p\ub+p'\ud)-(p\ub\cdot p'\ud)\nn\\
&\quad +m\up\2
+m\up(\bp\ua +\bp'\uu)\Big{]}\Big{[}\cM_{5}^{(c)}-\cM_{5}^{(d)}\Big{]}\nn\\
 &+\Big{[}( k\,  ,\, p\ub+p'\ud)-2m\up\bk\Big{]}
 \cM_{5}^{(\text{on})}\nn\\
 &-\frac{1}{2}(p\ua +p'\uu\,  ,\, p\ub+p'\ud)\Big{[}\cM_{6}^{(c)}-\cM_{6}^{(d)}\Big{]}\nn\\
 &
 +\Big{[}\frac{1}{2}(p\ua +p'\uu \,  ,\, p\ub+p'\ud)-(p\ub\cdot p'\ud)\nn\\
&\quad +m\up\2
-m\up(\bp\ua +\bp'\uu)\Big{]}\Big{[}\cM_{7}^{(c)}-\cM_{7}^{(d)}\Big{]}\nn\\
&+\Big{[}(k\, , \, p\ub +p'\ud )-2m\up \bk \Big{]}
\cM_{7}^{(\text{on})}\nn\\
& +\Big{[}-m\up(p\ua +p'\uu \,  ,\, p\ub+p'\ud)
+(\bp\ua+\bp'\uu)
\nn\\
&\quad \times \big{(}m\up\2+(p'\ud\cdot p\ub)\big{)}\Big {]}\Big{[}\cM_{8}^{(c)}-\cM_{8}^{(d)}\Big{]}\nn\\
& +\Big{[}(p\ua +p'\uu \,  ,\, p\ub+p'\ud)\bk
-(\bp\ua+\bp'\uu)(k\, , \, p\ub +p'\ud )\Big{]}\cM_{8}^{(\text{on})}\bigg{\rbrace}\nn\\
&\quad \times (\bp\ub +m\up) 
+ {\mathcal O}(\omega\2)\,.
\end{align}
From \er{C29} and \er{C41} we have
\bal{C52}
&\cM_{j}^{(c)}-\cM_{j}^{(d)}
=-2(p\us\cdot k)\cM_{j},_{s}\nn\\
&\quad -2(p\ub\cdot k)\cM_{j},_{m\ub\2}-2(p\ud\cdot k)\cM_{j},_{m\ud\2}+{\mathcal O}(\omega\2)\ ,\nn\\
&(j=1, \dots , 8)\,.
\end{align} 
Therefore, correct up to order $\omega$, we can everywhere in \er{C51} replace $p'\uu$ by $p\uu$ and $p'\ud$ by $p\ud$. This, indeed, gives for $\Big{[}\cN^{(0,c)}-\cN^{(0,d)}\Big{]}$ a homogeneous linear function in $k$ with corrections starting at order $\omega\2$.
Inserting \er{C19}, \er{C20}, and \er{C51} into \er{C50} we find for $\cN\ul^{(e)}$, up to order $\omega^{0}$, a unique solution which is given by
 \bel{C53}
 \cN\ul^{(e)}=\cN\ul^{(e1)}+\cN\ul^{(e2)}\,.
 \ee
Here $\cN\ul^{(e1)}$ is as in \er{C21} and
 \bal{C54}
&\cN\ul^{(e2)}=e(\bp\ud+m\up)\bigg{\lbrace}
-2p_{s\lambda}\cM_{1},_{s}-2p_{b\lambda}\cM_{1},_{m\ub\2}\nn\\
&-2p_{2\lambda}\cM_{1},_{m\ud\2}+\Big{[}m\up+\frac{1}{2}(\bp\ua +\bp\uu)\Big{]} \nn \\
&\quad \times
\Big{[}-2p_{s\lambda}\cM_{2},_{s}-2p_{b\lambda}\cM_{2},_{m\ub\2}-2p_{2\lambda}\cM_{2},_{m\ud\2}\Big{]}\nn\\
&-\gamma\ul\cM\ud^{(\text{on})}\nn\\
&+\Big{[}-m\up+\frac{1}{2}(\bp\ua +\bp\uu)\Big{]}\nn\\
&\quad\times\Big{[}-2p_{s\lambda}\cM_{4},_{s}-2p_{b\lambda}\cM_{4},_{m\ub\2}-2p_{2\lambda}\cM_{4},_{m\ud\2}\Big{]}\nn\\
&+\gamma\ul\cM_{4}^{(\text{on})}\nn\\
&+\Big{[}-\frac{1}{2}(p\ua +p\uu\,  ,\, p\ub+p\ud)-(p\ub\cdot p\ud)+m\up\2+m\up(\bp\ua +\bp\uu)\Big{]}\nn\\
&\quad \times \Big{[}-2p_{s\lambda}\cM_{5},_{s}-2p_{b\lambda}\cM_{5},_{m\ub\2}-2p_{2\lambda}\cM_{5},_{m\ud\2}\Big{]}\nn\\
&+\Big{[}(p\ub +p\ud)\ul -2m\up\gamma\ul\Big{]}\cM_{5}^{(\text{on})}\nn\\
&-\frac{1}{2}(p\ua +p\uu\,  ,\, p\ub-p\ud)
\Big{[}-2p_{b\lambda}\cM_{6},_{m\ub\2}-2p_{2\lambda}\cM_{6},_{m\ud\2}\Big{]}\nn\\
&+\Big{[}\frac{1}{2}(p\ua +p\uu\,  ,\, p\ub+p\ud)-(p\ub\cdot p\ud)+m\up\2-m\up(\bp\ua +\bp\uu)\Big{]}\nn\\
&\quad \times \Big{[}-2p_{s\lambda}\cM_{7},_{s}-2p_{b\lambda}\cM_{7},_{m\ub\2}-2p_{2\lambda}\cM_{7},_{m\ud\2}\Big{]}\nn\\
&+\Big{[}(p\ub +p\ud)\ul -2m\up\gamma\ul\Big{]}\cM_{7}^{(\text{on})}\nn\\
&+\Big{[}-m\up(p\ua +p\uu \,  ,\, p\ub+p\ud)+(\bp\ua +\bp\uu)\big{(}m\up\2 +(p\ud\cdot p\ub)\big{)}\Big{]}\nn\\
&\quad \times \Big{[}-2p_{s\lambda}\cM_{8},_{s}-2p_{b\lambda}\cM_{8},_{m\ub\2}-2p_{2\lambda}\cM_{8},_{m\ud\2}\Big{]}\nn\\
&+\Big{[}(p\ua +p\uu \,  ,\, p\ub+p\ud)\gamma\ul -(\bp\ua +\bp\uu)(p\ub +p\ud)\ul\Big{]}\cM_{8}^{(\text{on})}\bigg{\rbrace}\nn\\
&\quad\times(\bp\ub +m\up )+{\mathcal O}(\omega)\,.
\end{align}
 
Now the task is to determine $\cN\ul^{(c)}$,
\er{C31}--\er{C37}.
After some straightforward but lengthy calculations we obtain the following results.
\bel{C55}
\cN\ul^{(c1)}=\widetilde{\cN}\ul^{(c1)}+\nldt{c1}\,,
\ee
where, using \er{2.24}, \er{2.25}, and \er{C4}--\er{C8}, we find
 \bal{C56}
&\widetilde{\cN}\ul^{(c1)}=e(\bp'\ud+m\up)\Big{[}A^{(\text{on})}
+\frac{1}{2}(\bp\ua +\bp'\uu)B^{(\text{on})}\Big{]}\nn\\
& \quad \times (\bp\ub +m\up)\frac{(2p\ub-k)\ul}{-2p\ub\cdot k +k\2 }\nn\\
& +e(\bp\ud+m\up)\bigg{\lbrace}\bk \Big{[} -\frac{1}{2}\cM_{2}^{(\text{on})}+ \frac{1}{2}\cM_{4}^{(\text{on})}\nn\\
&\quad -m\up\cM_{5}^{(\text{on})}-m\up \cM_{7}^{(\text{on})}+(p\ua+p\uu \, ,\, p\ud)\cM_{8}^{(\text{on})}\Big{]}\nn\\
&\quad -(\bp\ua +\bp\uu)(p\ub \cdot l\ud + p\ud \cdot k) \cM_{8}^{(\text{on})}\nn\\
 &\quad+\Big{[}\frac{1}{2}(p\ub+p\ud \, , \, l\uu + k)+\frac{1}{2}(p\ua +p\uu \, , \, l\ud) + (p\ub \cdot l\ud)\nn\\
 &\qquad -\frac{1}{2}(p\ub - p\ud \, , \, k )\Big{]}\cM_{5}^{(\text{on})}\nn\\
 &\quad+\Big{[}\frac{1}{2}(p\ub+p\ud \, , \, l\ud)-\frac{1}{2}(p\ua +p\uu \, , \, l\ud) + (p\ub \cdot l\ud)\nn\\
 &\qquad -\frac{1}{2}(p\ub - p\ud \, , \, k )\Big{]}\cM_{7}^{(\text{on})}\nn\\
  &\quad+\Big{[}(p\ub+p\ud \, , \, l\uu )+(p\ua+p\uu\, , \, l\ud )\Big{]}m\up \cM_{8}^{(\text{on})}\nn\\
 &\quad+\Big{[}(\bp\ua +\bp\uu)\!\bk \;- \bk (\bp\ua + \bp\uu)\Big{]}\Big{[}-\frac{1}{4}\cM_{5}^{(\text{on})}+\frac{1}{4}\cM_{7}^{(\text{on})}\nn\\
&\qquad -\frac{1}{2}m\up \cM_{8}^{(\text{on})}\Big{]}\bigg{\rbrace} (\bp\ub +m\up )\frac{2p_{b\lambda}}{(-2p\ub \cdot k)}\nn\\
&+e(\bp\ud +m\up)\bigg{\lbrace}-2(p\us \cdot k)
\Big{[}A^{(\text{on})}_{, s}+\cM_{5}^{(\text{on})}
-\cM_{7}^{(\text{on})}\nn\\
&\quad +2m\up\cM_{8}^{(\text{on})}+\frac{1}{2}(\bp\ua + \bp\uu)B,_{s}^{\!\!(\text{on})}\Big{]}\nn\\
&\quad +2(p\ut \cdot l\uu)\Big{[}A^{(\text{on})}_{, t}-\cM_{7}^{(\text{on})}+m\up \cM_{8}^{(\text{on})}\nn\\
&\quad +\frac{1}{2}(\bp\ua + \bp\uu)( B,_{t}^{\!\!(\text{on})}+\cM_{8}^{(\text{on})})\Big{]}\bigg{\rbrace}
(\bp\ub +m\up )\frac{2p_{b\lambda}}{(-2p\ub \cdot k)}\nn\\
&+{\mathcal O}(\omega)\,,
\end{align}
\bal{C57}
&\overset{\approx}{\cN}\ul^{(c1)}=
 e(\bp\ud+m\up)2p_{b\lambda}\bigg{\lbrace}
\cM_{1, m\ub\2}\nn\\
 &+\Big{[}m\up +\frac{1}{2}(\bp\ua + \bp\uu)\Big{]}\cM_{2, m\ub\2}\nn\\
  &+\Big{[}-m\up +\frac{1}{2}(\bp\ua + \bp\uu)\Big{]}\cM_{4, m\ub\2}\nn\\
 &+\Big{[}-\frac{1}{2}(p\ua+p\uu \, , \, p\ub+p\ud)-(p\ub \cdot p\ud)+m\up\2\nn\\
 &\quad +m\up (\bp\ua + \bp\uu)\Big{]}\cM_{5, m\ub\2}\nn\\
 &-\frac{1}{2}(p\ua+p\uu \, , \, p\ub-p\ud)\cM_{6, m\ub\2}\nn\\
&+\Big{[}\frac{1}{2}(p\ua+p\uu \, , \, p\ub+p\ud)-(p\ub \cdot p\ud)+m\up\2\nn\\
&\quad - m\up (\bp\ua + \bp\uu)\Big{]}\cM_{7, m\ub\2}\nn
  \\ 
&+\Big{[}-m\up(p\ua+p\uu \, , \, p\ub+p\ud)+(\bp\ua + \bp\uu) (m\up\2 +p\ud \cdot p\ub)\Big{]}
\nn\\
&
\quad 
\times \cM_{8, m\ub\2}\bigg{\rbrace}(\bp\ub+m\up)+{\mathcal O}(\omega)\,,\\
\nn\\
\label{C58}
 &\cN\ul^{(c2)}=e(\bp\ud+m\up)\bigg{\lbrace}
 \Big{[}A^{(\text{on})}+\frac{1}{2}(\bp\ua +\bp\uu)B^{(\text{on})}\Big{]}(k\ul -\!\bk\gamma\ul)\nn\\
 &+ \Big{[}\cM_{2}^{(\text{on})}-\cM_{4}^{(\text{on})}+2m\up \cM_{5}^{(\text{on})} +2m\up \cM_{7}^{(\text{on})}\nn\\
 &\quad -(2s+t-2m\up\2 -2m\upp\2)\cM_{8}^{(\text{on})}\nn\\
 &\quad
 + (\bp\ua +\bp\uu)\Big{(}\cM_{5}^{(\text{on})}-\cM_{7}^{(\text{on})}+2m\up \cM_{8}^{(\text{on})} \Big{)}
 \Big{]}\nn\\
 &\times \big{(}\!\bk \,p_{b\lambda}-(p\ub \cdot k)\gamma\ul\big{)}\bigg{\rbrace}(\bp\ub +m\up)\frac{1+F\ud (0)}{(-2p\ub \cdot k)}\,,
\\ \nn\\
\label{C59}
&\cN\ul^{(c3)}=e(\bp\ud+m\up)\bigg{\lbrace}\cM_{1}^{(\text{on})}+(-s-m\up\2 +m\upp\2)\cM_{5}^{(\text{on})}\nn\\
 &+(s+t-3m\up\2 -m\upp\2)\cM_{7}^{(\text{on})}\nn\\
 &+\frac{1}{2}(\bp\ua + \bp\uu)\Big{(}\cM_{2}^{(\text{on})}+\cM_{4}^{(\text{on})}-t\cM_{8}^{(\text{on})}\Big{)}\bigg{\rbrace}\nn\\
&\times \Big{[} p_{b\lambda}\!\bk 
-(p\ub \cdot k)\gamma\ul\Big{]}(\bp\ub+m\up)
\frac{F\ud (0)}{m\up}\frac{1}{(-2p\ub \cdot k)} \,.
 \end{align}
%---------------------------------------------
Putting everything together 
we find for $\cN\ul^{(c)}$ \er{C31} the following:
%---------------------------------------------
 \bal{C60}
&\cN\ul^{(c)}=e(\bp'\ud+m\up)\Big{[}A^{(\text{on})}+\frac{1}{2}(\bp\ua +\bp'\uu)B^{(\text{on})}\Big{]}\nn\\
 &\quad \times (\bp\ub +m\up)\frac{(2p\ub -k)\ul}{-2p\ub \cdot k + k\2}\nn\\
 &+e(\bp\ud+m\up)\bigg{\lbrace}
 \Big{[}A^{(\text{on})}+\frac{1}{2}(\bp\ua +\bp\uu)B^{(\text{on})}\Big{]}(k\ul -\bk\gamma\ul)\nn\\       
 &\quad -2(p\us \cdot k)\Big{[} A,_{s}^{\!\!(\text{on})} +\frac{1}{2}(\bp\ua +\bp\uu)    
 B,_{s}^{\!\!(\text{on})}\Big{]} 2p_{b\lambda}\nn\\
  &\quad +2(p\ut \cdot l\uu)\Big{[} A,_{t}^{\!\!(\text{on})} +\frac{1}{2}(\bp\ua +\bp\uu)    
 B,_{t}^{\!\!(\text{on})}\Big{]} 2p_{b\lambda}\bigg{\rbrace} \nn\\
 &\quad \times (\bp\ub +m\up)\frac{1}{(-2p\ub \cdot k )}\nn\\
&+ e(\bp\ud+m\up)\Big{[}A^{(\text{on})}+\frac{1}{2}(\bp\ua +\bp\uu)B^{(\text{on})}\Big{]}\nn\\
&\quad \times \Big{[}m\up (k\ul -\bk\gamma\ul)+ \big{(} p_{b\lambda}\bk -(p\ub \cdot k)\gamma\ul\big{)}\Big{]}\nn\\
&\quad \times (\bp\ub+m\up)\frac{F\ud (0)}{m\up}\frac{1}{(-2p\ub \cdot k)}\nn\\
&+e(\bp\ud+m\up)\bigg{\lbrace}\Big{[}-2 \cM_{7}^{(\text{on})} + 2m\up \cM_{8}^{(\text{on})} \nn \\
&\quad
+ (\bp\ua +\bp\uu)\cM_{8}^{(\text{on})} \Big{]}
p_{b\lambda}\nn\\
&\quad +\Big{[}\cM_{2}^{(\text{on})}-\cM_{4}^{(\text{on})}+2m\up \cM_{5}^{(\text{on})} +2m\up \cM_{7}^{(\text{on})}\nn     \\
 &\quad  -( 2s+t-2m\up\2 -2m\upp\2)\cM_{8}^{(\text{on})}\nn\\
  &\quad +(\bp\ua +\bp\uu)  (\cM_{5}^{(\text{on})}-\cM_{7}^{(\text{on})}+2m\up \cM_{8}^{(\text{on})})\Big{]}\frac{1}{2}\gamma\ul\bigg{\rbrace}\nn\\
 &\quad \times(\bp\ub +m\up)\nn\\
 &+\nldt{c1}+{\mathcal O}(\omega)\,.
 \end{align}
Note that $\nldt{c1}$ \er{C57} contains exclusively the derivatives of the amplitudes $\cM_{j}$ with respect to $m\ub\2$ which lead off the mass shell.

The next task is to determine 
$\cN\ul^{(d)}$ from \er{C43}--\er{C49}.
Again, after a straightforward but lengthy calculation we get the following.
\bel{C61}
\cN\ul^{(d1)}=\widetilde{\cN}\ul^{(d1)}+\nldt{d1}\,,
\ee
where
\newpage
\bal{C62}
&\widetilde{\cN}\ul^{(d1)}=e\frac{(2p'\ud +k)\ul}{2p'\ud \cdot k +k\2 }(\bp'\ud +m\up)
\bigg{\lbrace} \cM_{1}^{(\text{on})} +2(p\ut \cdot l\uu)\cM_{1},_{t}\nn\\
&+\Big{[}m\up + \frac{1}{2}(\bp\ua +\bp'\uu +\bk)\Big{]}\Big{[}\cM_{2}^{(\text{on})} +2(p\ut \cdot l\uu)\cM_{2},_{t}\Big{]}\nn\\
&+\Big{[}-m\up + \frac{1}{2}(\bp\ua +\bp'\uu -\bk)\Big{]}\Big{[}\cM_{4}^{(\text{on})} +2(p\ut \cdot l\uu)\cM_{4},_{t}\Big{]}\nn\\
 &+\Big{[}-\frac{1}{2}(p\ua+p'\uu+k \, , \, p\ub+ p'\ud)-(p\ub \cdot p'\ud)+m\up\2\nn\\
 &\quad+ m\up (\bp\ua + \bp'\uu+\bk)-\frac{1}{2}(p\ub-p'\ud \, , \, k)\nn\\
 &\quad -\frac{1}{4}\Big{(} (\bp\ua +\bp'\uu)\!\bk \;-\bk (\bp\ua +\bp'\uu)\Big{)} \Big{]} \nn \\
 &\quad \times 
    \Big{[}\cM_{5}^{(\text{on})} +2(p\ut \cdot l\uu)\cM_{5},_{t}\Big{]}\nn\\
          &  +\Big{[}\frac{1}{2}(p\ua+p'\uu-k \, , \, p\ub+p'\ud)-(p\ub \cdot p'\ud)+m\up\2\nn\\
  &\quad - m\up (\bp\ua + \bp'\uu-\bk)-\frac{1}{2}(p\ub-p'\ud \, , \, k)\nn\\
    &\quad +\frac{1}{4}\Big{(} (\bp\ua +\bp'\uu)\!\bk \;-\bk (\bp\ua +\bp'\uu)\Big{)}
    \Big{]}\nn\\
    &\quad \times \Big{[}\cM_{7}^{(\text{on})} +2(p\ut \cdot l\uu)\cM_{7},_{t}\Big{]}\nn\\
      &  +\Big{[}-m\up (p\ua+ p'\uu \, , \, p\ub+p'\ud)+ (\bp\ua + \bp'\uu)(m\up\2 + p'\ud\cdot p\ub)\nn\\
  &\quad +(p\ub \cdot k)(\bp\ua + \bp'\uu)-(p\ua+ p'\uu \, , \, p\ub)\bk\nn\\
  &\quad -\frac{1}{2}m\up \Big{(}(\bp\ua + \bp'\uu)\!\bk \;-\bk (\bp\ua + \bp'\uu)\Big{)}\Big{]}\nn\\
    &\quad \times \Big{[}\cM_{8}^{(\text{on})} +2(p\ut \cdot l\uu)\cM_{8},_{t}\Big{]} \bigg{\rbrace}  (\bp\ub + m\up)+{\mathcal O}(\omega)\,,
    \end{align}
    
\bal{C63}
&\nldt{d1}=e\,2p_{2\lambda}(\bp\ud +m\up)\bigg{\lbrace}\cM_{1, m\ud\2}\nn\\
&+\Big{[}m\up + \frac{1}{2}(\bp\ua +\bp\uu )\Big{]}\cM_{2, m\ud\2}\nn\\
&+\Big{[}-m\up + \frac{1}{2}(\bp\ua +\bp\uu )\Big{]}\cM_{4, m\ud\2}\nn\\
 &+\Big{[}-\frac{1}{2}(p\ua+p\uu \, , \, p\ub+p\ud)-(p\ub \cdot p\ud)+m\up\2\nn\\
 &\quad 
 +m\up (\bp\ua + \bp\uu)\Big{]}\cM_{5, m\ud\2}\nn\\ 
 &+\Big{[}-\frac{1}{2}(p\ua+p\uu \, , \, p\ub-p\ud)\Big{]}\cM_{6, m\ud\2}\nn\\
 &+\Big{[}\frac{1}{2}(p\ua+p\uu \, , \, p\ub+p\ud)-(p\ub \cdot p\ud)+m\up\2- m\up (\bp\ua + \bp\uu)\Big{]}\nn\\
  &\quad \times \cM_{7, m\ud\2}\nn\\
 &  +\Big{[}-m\up(p\ua+p\uu \, , \, p\ub+p\ud)+(\bp\ua + \bp\uu) \big{(}m\up\2 +(p\ud \cdot p\ub)\big{)}\Big{]}\nn\\
 &\quad \times
 \cM_{8, m\ud\2}\bigg{\rbrace}
 (\bp\ub+m\up) + {\mathcal O}(\omega)\,.
 \end{align}
%--------------------------------------------------------
Note that $\nldt{d1}$ contains exclusively 
the derivatives $\cM_{j},_{m\ud\2}$ $(j=1, \dots , 8)$
which lead off the mass shell.
With \er{2.24}, \er{2.25}, and \er{C5}--\er{C8} 
we obtain
\newpage
%--------------------------------------------------------
\bal{C64}
&\widetilde{\cN}\ul^{(d1)}=e\frac{(2p'\ud +k)\ul}{2p'\ud\cdot  k +k\2 }(\bp'\ud +m\up)\nn\\
   &\quad \times \Big{\lbrace} A^{(\text{on})} 
   +\frac{1}{2}(\bp\ua +\bp'\uu) B^{(\text{on})}\Big{\rbrace}(\bp\ub +m\up)\nn\\
 &+e\frac{p_{2\lambda}}{p\ud \cdot k}\,2(p\ut \cdot l\uu)(\bp\ud +m\up)\nn\\
 &\quad \times \Big{\lbrace} A,_{t}^{\!\!(\text{on})} 
  +\frac{1}{2}(\bp\ua +\bp\uu) B,_{t}^{\!\!(\text{on})}\Big{\rbrace}(\bp\ub +m\up)\nn\\
 &+e\frac{p_{2\lambda}}{p\ud\cdot k}(\bp\ud +m\up)
 \!\bk
 \nn\\
 &\quad \times \Big{\lbrace}\frac{1}{2}\cM_{2}^{(\text{on})}-\frac{1}{2}\cM_{4}^{(\text{on})}+m\up \cM_{5}^{(\text{on})} +m\up \cM_{7}^{(\text{on})}\nn\\
 &\qquad -(p\ua +p\uu\, , \, p\ub)\cM_{8}^{(\text{on})}\nn\\
 &\qquad +\frac{1}{2}(\bp\ua +\bp\uu )(\cM_{5}^{(\text{on})}-\cM_{7}^{(\text{on})}+2m\up \cM_{8}^{(\text{on})})\Big{\rbrace} \nn\\
&\qquad \times (\bp\ub +m\up)\nn\\
 &+e \,p_{2\lambda}(\bp\ud +m\up)(-2\cM_{7}^{(\text{on})}+2m\up \cM_{8}^{(\text{on})}\nn\\
& \quad +(\bp\ua +\bp\uu) \cM_{8}^{(\text{on})})(\bp\ub +m\up)+{\mathcal O}(\omega)\,,
 \end{align}
 \bal{C65}
&\cN\ul^{(d2)}=-e\big{[}1+F\ud (0)\big{]}\frac{1}{2p\ud \cdot k}(\bp\ud +m\up)\nn\\
 & \times \bigg{\lbrace}(k\ul -\gamma\ul \! \bk)\Big{[}A^{(\text{on})}+\frac{1}{2}(\bp\ua +\bp\uu)B^{(\text{on})}\Big{]}\nn\\
 &\quad +\Big{[} p_{2\lambda}\bk -(p\ud \cdot k)\gamma\ul\Big{]} \Big{[}\cM_{2}^{(\text{on})}-\cM_{4}^{(\text{on})}\nn\\
 &\qquad +\cM_{5}^{(\text{on})}\big{(}2m\up +(\bp\ua +\bp\uu)\big{)}\nn\\
  &\qquad +\cM_{7}^{(\text{on})}\big{(}2m\up -(\bp\ua +\bp\uu)\big{)}\nn\\
  & \qquad -\cM_{8}^{(\text{on})}\big{(}2s + t -2m\up\2 -2m\upp\2 -2m\up(\bp\ua +\bp\uu)\big{)}\Big{]} \bigg{\rbrace}\nn\\
  &\quad\times (\bp\ub +m\up)+{\mathcal O}(\omega)\,,
  \end{align}
 \bal{C66}
&\cN\ul^{(d3)}=-e\frac{F\ud(0)}{m\up}\frac{1}{2p\ud\cdot k}(\bp\ud +m\up)\Big{[} p_{2\lambda}\!\bk -(p\ud \cdot k)\gamma\ul\Big{]} \nn\\
&\times \bigg{\lbrace} \cM_{1}^{(\text{on})}+\cM_{2}^{(\text{on})}\frac{1}{2}  (\bp\ua +\bp\uu) +\cM_{4}^{(\text{on})}    \frac{1}{2}  (\bp\ua +\bp\uu) \nn\\
&\quad+\cM_{5}^{(\text{on})}(-s +m\up\2  +m\upp\2 -2m\up\2)\nn\\
&\quad+\cM_{7}^{(\text{on})}(s +t-m\up\2  -m\upp\2 -2m\up\2)\nn\\
&\quad+\cM_{8}^{(\text{on})}(-\frac{1}{2}t)(\bp\ua +\bp\uu) \bigg{\rbrace}(\bp\ub +m\up)+{\mathcal O}(\omega)\,.\end{align}
\mbox{Collecting things together we get from 
\er{C43}, \er{C61}--\er{C66}}
\bal{C67}
&\cN\ul^{(d)}=\widetilde{\cN}\ul^{(d1)}+ \cN\ul^{(d2)}+ \cN\ul^{(d3)}
+\nldt{d1}+{\mathcal O}(\omega)\ \nn\\
&=e \frac{(2p'\ud+k)\ul}{2p'\ud\cdot k +k\2}(\bp'\ud +m\up)\nn\\
 &\quad \times \Big{[}A^{(\text{on})}+\frac{1}{2}(\bp\ua +\bp'\uu)B^{(\text{on})}\Big{]}(\bp\ub +m\up)\nn\\
 &+e\frac{p_{2\lambda}}{p\ud \cdot k} 2(p\ut\cdot l\uu)(\bp\ud +m\up)\nn\\
 &\quad \times \Big{[} A,_{t}^{\!\!(\text{on})} +\frac{1}{2}(\bp\ua +\bp\uu) B,_{t}^{\!\!(\text{on})}\Big{]} (\bp\ub +m\up)\nn\\
 &-e\frac{1}{2p\ud\cdot k}(\bp\ud +m\up) (k\ul -\gamma\ul \!\bk)\nn\\
 &\quad \times \Big{[}A^{(\text{on})}+\frac{1}{2}(\bp\ua +\bp\uu)B^{(\text{on})}\Big{]} (\bp\ub +m\up)\nn\\
 &-e\frac{F\ud (0)}{m\up}\frac{1}{2p\ud \cdot k}(\bp\ud +m\up) \nn\\
 &\quad \times \Big{[}m\up (k\ul -\gamma\ul \!\bk)+\big{(} p_{2\lambda}\bk -(p\ud \cdot k)\gamma\ul\big{)}\Big{]}\nn\\
 &\quad \times \Big{[}A^{(\text{on})}+\frac{1}{2}(\bp\ua +\bp\uu)B^{(\text{on})}\Big{]} (\bp\ub +m\up)\nn\\
 &+e \,p_{2\lambda}(\bp\ud +m\up)\Big{[} -2\cM_{7}^{(\text{on})}+2m\up\cM_{8}^{(\text{on})}\nn\\
 &\quad+(\bp\ua +\bp\uu) \cM_{8}^{(\text{on})}\Big{]}(\bp\ub +m\up)\nn\\
 &+e(\bp\ud +m\up)\gamma\ul \Big{[}\frac{1}{2}\cM_{2}^{(\text{on})} -\frac{1}{2}\cM_{4}^{(\text{on})}\nn\\
 &\quad+\cM_{5}^{(\text{on})}\big{(}m\up+ \frac{1}{2}(\bp\ua +\bp\uu)\big{)}
 +\cM_{7}^{(\text{on})}\big{(}m\up-\frac{1}{2}(\bp\ua +\bp\uu)\big{)}\nn\\
 &\quad +\cM_{8}^{(\text{on})}\big{(}-\frac{1}{2}(2s + t - 2 m\up\2 - 2m\upp\2)+m\up(\bp\ua +\bp\uu)\big{)}\Big{]}\nn\\
&\quad \times (\bp\ub +m\up)\nn\\
&+\nldt{d1} +{\mathcal O}(\omega)\,.
  \end{align}
 
Now we study the term
 \bal{C68}
 &\cN\ul^{(e3)}=\cN\ul^{(e2)}+ \nldt{c1}+ \nldt{d1}\,.
 \end{align}
From  \er{C54}, \er{C57}, and \er{C63} 
we get
\bal{C68A}
&\cN\ul^{(e3)}=e (\bp\ud +m\up)\bigg{\lbrace}-2p_{s\lambda}\Big{[}\cM_{1},_{s}\nn\\
 &+ \Big{(}m\up+ \frac{1}{2}(\bp\ua +\bp\uu )\Big{)}\cM_{2},_{s}\nn\\
 &+ \Big{(}-m\up+ \frac{1}{2}(\bp\ua +\bp\uu )\Big{)}\cM_{4},_{s}\nn\\
 &+ \Big{(}-\frac{1}{2}(p\ua +p\uu\, , \, p\ub + p\ud )-(p\ub \cdot p\ud ) + m\up\2 \nn\\
  &\quad +m\up (\bp\ua +\bp\uu)\Big{)}\cM_{5},_{s}\nn\\
   & + \Big{(}\frac{1}{2}(p\ua +p\uu\, , \, p\ub + p\ud )-(p\ub \cdot p\ud ) + m\up\2 \nn\\
 &\quad  -m\up (\bp\ua +\bp\uu)\Big{)}\cM_{7},_{s}\nn\\
   & + \Big{(}-m\up (p\ua +p\uu\, , \, p\ub + p\ud )+ \big{(}m\up\2 +(p\ud \cdot p\ub )\big{)} (\bp\ua +\bp\uu) \Big{)} \nn\\
 &\quad \times \cM_{8},_{s}\Big{]}\nn\\
 &-\gamma\ul \Big{[}\cM_{2}^{(\text{on})}
 -\cM_{4}^{(\text{on})}+2m\up\cM_{5}^{(\text{on})}
 +2m\up \cM_{7}^{(\text{on})}\nn\\
 &\quad -(p\ua +p\uu\, , \, p\ub + p\ud) 
 \cM_{8}^{(\text{on})}\Big{]}\nn\\
 &+(p\ub + p\ud)\ul \Big{[}\cM_{5}^{(\text{on})}+\cM_{7}^{(\text{on})}
 - (\bp\ua +\bp\uu) \cM_{8}^{(\text{on})}\Big{]}\bigg{\rbrace} \nn\\
& \times (\bp\ub +m\up)
+ {\mathcal O}(\omega)\,,
\end{align}\\
and with \er{A10}, \er{C5}, and \er{C7}
\bal{C69}
 &\cN\ul^{(e3)}=e (\bp\ud +m\up)\bigg{\lbrace}-2p_{s\lambda}\Big{[}\cM_{1},_{s}
+m\up \cM_{2},_{s}-m\up \cM_{4},_{s} \nn\\
&+(-s+m\up\2 +m\upp\2 )\cM_{5},_{s}
+ (s+t-m\up\2 -m\upp\2 )\cM_{7},_{s}\nn \\
&-m\up  (2s+t-2m\up\2 -2m\upp\2 )\cM_{8},_{s}\nn \\
&+\frac{1}{2}(\bp\ua +\bp\uu)\Big{(}\cM_{2},_{s}+\cM_{4},_{s}+2m\up \cM_{5},_{s} \nn\\
&-2m\up \cM_{7},_{s} +(4m\up\2 -t) \cM_{8},_{s}\Big{)}\Big{]}\nn\\
&-\gamma\ul \Big{[}\cM_{2}^{(\text{on})}-\cM_{4}^{(\text{on})}+2m\up\cM_{5}^{(\text{on})}+2m\up \cM_{7}^{(\text{on})}\nn\\
 &\quad -   (2s+t-2m\up\2 -2m\upp\2 )    \cM_{8}^{(\text{on})}\Big{]}\nn\\
 &+(p\ub + p\ud)\ul \Big{[}\cM_{5}^{(\text{on})}+\cM_{7}^{(\text{on})}- (\bp\ua +\bp\uu)   \cM_{8}^{(\text{on})}\Big{]}\bigg{\rbrace} \nn\\
& \times(\bp\ub +m\up) 
+ {\mathcal O}(\omega)\,,\\
\label{C70}
 &\cN\ul^{(e3)}=e (\bp\ud +m\up)\bigg{\lbrace}-2p_{s\lambda}\Big{[}A,_{s}^{\!\!(\text{on})}
 +\frac{1}{2}(\bp\ua +\bp\uu)
 B,_{s}^{\!\!(\text{on})}\Big{]}\nn\\
  &-\gamma\ul \Big{[}\cM_{2}^{(\text{on})}-\cM_{4}^{(\text{on})}+2m\up\cM_{5}^{(\text{on})}+2m\up \cM_{7}^{(\text{on})}\nn\\
 &\quad -(2s+t-2m\up\2 -2m\upp\2) 
 \cM_{8}^{(\text{on})}\Big{]}\nn \\
 &+(p\ub + p\ud)\ul \Big{[}\cM_{5}^{(\text{on})}+\cM_{7}^{(\text{on})}- (\bp\ua +\bp\uu)   \cM_{8}^{(\text{on})}\Big{]}\nn
\\
  &-2(p\uu + p\ud)\ul \Big{[}\cM_{5}^{(\text{on})}-\cM_{7}^{(\text{on})}+2m\up  \cM_{8}^{(\text{on})}\Big{]}\bigg{\rbrace} \nn\\
&\times (\bp\ub +m\up) + {\mathcal O}(\omega)\,.
\end{align}

Now we can collect everything together and calculate
\bal{C71}
\cN\ul^{(c+d+e2)}&=\cN\ul^{(c)}+\cN\ul^{(d)}+\cN\ul^{(e2)}\nn\\
&=\widetilde{\cN}\ul^{(c1)}   +\cN\ul^{(c2)}+\cN\ul^{(c3)} \nn\\
&\quad +\widetilde{\cN}\ul^{(d1)}+\cN\ul^{(d2)}+\cN\ul^{(d3)}+\cN\ul^{(e3)}\,;
\end{align}
see \er{C31}--\er{C37}, \er{C43}--\er{C49}, \er{C55}, \er{C60}, \er{C61}, \er{C68}, and \er{C70}. 
The result is given in \er{3.44}
\bel{C72}
\cN\ul^{(c+d+e2)-}\equiv \cN\ul^{(c+d+e2)}\,.
\ee
 
%======================================
\endgroup
%======================================
%%\end{appendix}

% Create the reference section using BibTeX:
\bibliography{refs}

\end{document}
%
% ****** End of file apstemplate.tex ******