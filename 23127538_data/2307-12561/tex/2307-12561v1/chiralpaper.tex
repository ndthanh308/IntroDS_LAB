\documentclass[aps,prd,nofootinbib %twocolumn
]{revtex4}
\usepackage{amsmath}
\usepackage{amssymb}

\usepackage{slashed}

\usepackage{graphicx}
\usepackage{subfigure}
\graphicspath{{fig/}}
\usepackage{color}
\renewcommand{\thefootnote}{\fnsymbol{footnote}}

\begin{document}
\title{Majorana Phase And Matter Effects In Neutrino Chiral Oscillation}
\author{Ming-Wei Li$^{1,}$\footnote{limw2021@sjtu.edu.cn}, Zhong-Lv Huang$^{1,}$\footnote{huangzhonglv@sjtu.edu.cn}, 
	Xiao-Gang He$^{1,2,}$\footnote{hexg@sjtu.edu.cn}}
\affiliation{$^1$Tsung-Dao Lee Institute, KLPAC and SKLPPC Laboratories, School of Physics and Astronomy, Shanghai Jiao Tong University, Shanghai 200240\\
$^2$Department of Physics, National Taiwan University, Taipei 10617}



\begin{abstract}
Due to finite masses and mixing, for neutrinos propagation in space-time, there is a chiral oscillation between left- and right- chiral neutrinos,  besides the usual oscillation between different generations. The probability of chiral oscillation is suppressed by a factor of $m^2/E^2$ making the effect small for relativistic neutrinos. However, for non-relativistic neutrinos, this effects can be significant. In matter, the equation of motion is modified. When neutrinos produced in weak interaction pass through the matter, the effective energies are split into two different ones depending on the helicity of the neutrino. This results in different oscillation behavior for neutrinos with different helicity, in particular there is a new resonant effect related to the helicity state of neutrino different than the usual MSW effect. For Majorana neutrinos, chiral oscillation also depends on Majorana phases.
\end{abstract}


\maketitle


\section{ Neutrino chiral oscillation}

The neutrino oscillation has been observed by many experiments \cite{ParticleDataGroup:2022pth}. Most of the studies have concentrated on neutrino oscillations between generations with the same chirality. In this paper we study neutrino oscillations between different chiral states. In the standard model (SM), active neutrinos have left-handed chiral interaction and therefore neutrinos produced are left-chiral states. 
It has been shown that the Hamiltonian $H$ deduced from the Dirac equation does not commute with left-chiral weak interaction which causes left-chiral neutrinos to oscillate into right-chiral neutrinos \cite{Bittencourt:2020xen,Ge:2020aen,Kimura:2021qlh,Bittencourt:2022hwn,Blasone:2022fgj}.
The matter effects also change neutrino oscillation behavior for both usual the MSW effect \cite{Wolfenstein:1977ue,Mikheyev:1985zog} and the chiral oscillations \cite{Pantaleone:1992xh}. We find that for left-chiral neutrino produced in weak interaction in matter, each of the  eigen-energy is split into two different ones depending on the helicity of the neutrino. This results in different oscillation behavior for neutrinos with different helicity, in particular there is a new resonant effect related to the helicity state of neutrino different than the MSW effect. For Majorana neutrinos, we find that neutrino chiral oscillation in matter can reveal information about Majorana phases which is drastically different from the usual oscillation. In the following discussion, we provide some details of our findings.

The evolution of neutrinos in free space propagation is governed by the Dirac equation
\begin{equation}
	(i \slashed{\partial}- m) \psi = 0\;,
\end{equation}
which gives a Hamiltonian $
H=\gamma^0\boldsymbol{\gamma}\cdot\mathbf{p}+m\gamma^0
=(\mathbf{p}\cdot\boldsymbol{\Sigma})\gamma^5+m\beta$.
One obtains the evolution operator
\begin{eqnarray}
U(t)=e^{-iHt} 
=\cos(Et)-i{(\mathbf{p}\cdot\boldsymbol{\Sigma})\gamma^5+m\beta\over E} \sin(Et)\;,
\end{eqnarray}
and the wave function evolves as $\psi(t) = U(t)\psi(0)$. Here $\boldsymbol{\Sigma}= \gamma^5\gamma^0\boldsymbol{\gamma} $ is the spin  operator. We have used the chiral representation for the $\gamma^\mu$ matrices,
\begin{equation}
	\gamma^0=\begin{pmatrix}
		0&I\\
		I&0
	\end{pmatrix}\;,\;\;
	\gamma^i=\begin{pmatrix}
		0&\sigma^i\\
		-\sigma^i&0
	\end{pmatrix}\;,\;\;
	\gamma^5=\begin{pmatrix}
		-I&0\\
		0&I
	\end{pmatrix}\;.\;\;
\end{equation}


The wave function $\psi^{h}(t,\mathbf{x})$, evolved from the momentum eigenstate  $\psi^h(\mathbf{x})$ produced at $t=0$ with momentum $\mathbf{p}$, {will be $
	\psi^{h}(t,\mathbf{x})=U(t)\psi^h(\mathbf{x})
	= \psi^h(\mathbf{x}) e^{-i\bar Et}
$, and the positive energy and negative energy $\bar E$ wave functions are
\begin{eqnarray}
	\psi^{h(\bar E=+E)}(\mathbf{x}) = 
	\frac{1}{\sqrt{2E}}
	\begin{pmatrix}
		\sqrt{E-h\cdot p}\ u^h\\
		\sqrt{E+h\cdot p}\ u^h
	\end{pmatrix}e^{i\mathbf{p}\cdot\mathbf{x}}\;,\;\;\;\;\psi^{h(\bar E= -E)}(\mathbf{x}) = 
	\frac{1}{\sqrt{2E}}
	\begin{pmatrix}
		\sqrt{E+h\cdot p}\ u^h\\
		- \sqrt{E-h\cdot p}\ u^h
	\end{pmatrix}e^{i\mathbf{p}\cdot\mathbf{x}}\;,
\end{eqnarray}
where $E=\sqrt{p^2 +m^2}$, 
$ h=\pm1 $ and $u^h$ are the helicity states with  $(\mathbf{p}\cdot\boldsymbol{\Sigma})u^h = (h\cdot p) u^h$. The wave function is normalized as $\int\mathrm{d}x\left( \psi'(\mathbf{x})\right)^\dagger \psi(\mathbf{x})/V =\delta_{p'p}$.


The wave function $\psi$ can be decomposed into left-handed $\psi_L = (1-\gamma_5)\psi/2$ and right-handed $\psi_R= (1+\gamma_5)\psi/2$, $\psi = \psi_L + \psi_R$. The left-chiral and right-chiral neutrinos are entangled during propagation by
$ i \slashed{\partial}\psi_L - m \psi_R = 0$ and $ i \slashed{\partial}\psi_R - m \psi_L = 0$. 
If a pure $\psi_L$ is produced, like standard weak interaction neutrino production, at some later time, $\psi_L$ will oscillate into $\psi_R$ or vise versa.  At $ t=0 $, a left-chiral normalized neutrino wave function would be
\begin{equation}
	\psi^h_{L}(\mathbf{x})
	=\begin{pmatrix}u^h\\0\end{pmatrix}
	e^{i\mathbf{p}\cdot\mathbf{x}}\;,\;\;
	\psi^h_{R}(\mathbf{x})
	=\begin{pmatrix}0\\u^h\end{pmatrix}
	e^{i\mathbf{p}\cdot\mathbf{x}}\;.
\end{equation}
At time $ t $ the evolution of $\psi^h_{L}(t,\mathbf{x})$  would be
$	\psi^h_{L}(t,\mathbf{x})=U(t)\psi^h_{L}(\mathbf{x})
=\left( \cos(Et)+i(h\cdot p/ E)\sin(Et)\right)\psi^h_{L}(\mathbf{x})
-i(m/E)\sin(Et)\psi^h_{R}(\mathbf{x})
$.

When probed at time $ t $ by the normalized the  state $\psi^h_{L(R)}$ at $ \mathbf{x}=\mathbf{L} $, one would have
\begin{equation}
	\psi^h_{L}(\mathbf{x-L})=e^{-i\mathbf{p}\cdot\mathbf{L}}
	\psi^h_L(\mathbf{x})
	=\begin{pmatrix}u^h\\0\end{pmatrix}
	e^{i\mathbf{p}\cdot(\mathbf{x-L})}\;,\;\;
	\psi^h_{R}(\mathbf{x-L})=e^{-i\mathbf{p}\cdot\mathbf{L}}
	\psi^h_R(\mathbf{x})
	=\begin{pmatrix}0\\u^h\end{pmatrix}
	e^{i\mathbf{p}\cdot(\mathbf{x-L})}\;, 
\end{equation}
leading to the following left-chiral to left-chiral and left-chiral to right-chiral neutrino oscillation probabilities 
\begin{eqnarray}
	\label{free}
	P(\nu^h_L\rightarrow\nu^h_L)=|\left\langle\psi_{L}^h(\mathbf{x})|\psi^h_{L}(t,\mathbf{x})\right\rangle|^2= 1-{m^2\over E^2} \sin^2(Et)\;,\;\;
	P(\nu^h_L\rightarrow\nu^h_R)=|\left\langle\psi_{R}^h(\mathbf{x})|\psi^h_{L}(t,\mathbf{x})\right\rangle|^2= {m^2\over E^2} \sin^2(Et)\;.
\end{eqnarray}
One sees that a pure left-handed state has partially evolved into right-handed state. A chiral oscillation has occurred.

For high energy neutrinos with $m/E\ll1$, the probability for chiral oscillation is small, and the oscillation is the same as the Dirac neutrino $\psi^h$ justifying the usual neutrino oscillation treatment. But for non-relatvisitice neutrinos, like late time cosmology neutrinos whose energies are very low, the chiral oscillation probability can reach 1/2 on the average. The chiral oscillation has a significant consequence on the detection of the cosmic neutrino background even with one generation Dirac neutrino~\cite{Ge:2020aen}.

\section{The matter effects on neutrino chiral oscillation\label{simple}}
Due to $W$ and $Z$ exchanges, Dirac neutrinos interaction in SM will change the behavior of neutrino oscillation. The effective interaction Lagrangian in matter is given by
$	\mathcal{L}_{\mathrm{int}} = 
	-\bar{\nu}_L  j_L^\mu\gamma_\mu  \nu_{L}\;. 
$
Here $j_L^\mu$ is the effective matter current which neutrino can interact.  In the rest frame of the homogeneous, isotropic, unpolarized, electrical neutrality medium, $j_L^\mu = (\rho, \mathbf{0})$, and the $Z$ contribution from electron and proton would cancel out so that $ \rho$ is a diagonal matrix $\operatorname{diag}\left\{\rho_W+\rho_Z, \rho_Z, \rho_Z \right\} $ with $ \rho_W=\sqrt{2}G_F N_e $ and $ \rho_Z=-G_F N_n/\sqrt{2} $ where $N_e, N_n$ are the number density of electron and neutron so that the elements of $\rho$ could be positive or negative. When study neutrino oscillation in matter, one should include $\mathcal{L}_{int}$.

Neutrinos may be Majorana particles. Considering that the neutrinos may have masses from Type I, 
%\cite{Minkowski:1977sc,Yanagida:1979as,Gell-Mann:1979vob,Glashow:1979nm,Mohapatra:1979ia}, 
II 
%$\cite{Magg:1980ut,Schechter:1980gr,Cheng:1980qt,Mohapatra:1980yp} 
and III Seesaw models 
%\cite{Foot:1988aq,Ma:1998dn} Seesaw model 
separately or loop contributions 
%{\color{cyan}\cite{Weinberg:1979sa}} 
and some forms of combinations, the most general effective Lagrangian for these Seesaw neutrinos in matter will be given by 
\begin{eqnarray}
	\label{general lagrangian}
\begin{aligned}
	\mathcal{L}=&\bar\nu_{L} i\slashed{\partial} \nu_{L} 
	+ \bar{N}_{R} i\slashed{\partial} N_R
	-\frac{1}{2}\left( \begin{pmatrix}
		\bar{\nu}^c_L & \bar{N}_R
	\end{pmatrix}
	\begin{pmatrix}
		M_L & M_D^T\\ M_D & M_R
	\end{pmatrix}
	\begin{pmatrix}
		\nu_L\\ N_R^c
	\end{pmatrix}+\text{h.c.}\right)
	-\begin{pmatrix}
		\bar{\nu}_L & \bar{N}_R^c
	\end{pmatrix}
	\begin{pmatrix}
		j_L^\mu & j_{RL}^\mu\\  j_{RL}^{\mu\dagger} & j_R
	\end{pmatrix}\gamma_\mu
	\begin{pmatrix}
		\nu_L\\ N_R^c
	\end{pmatrix}\\
	=&\bar \psi_L i\slashed{\partial} \psi_L
	-\frac{1}{2}\left(\bar\psi_L^c \mathcal{M}\psi_L+\text{h.c.}\right)
	-\bar{\psi}_{L}J^\mu\gamma_\mu\psi_L\;.
\end{aligned}
\end{eqnarray}
$J^\mu_{RL}$ and $J^\mu_R$ are  beyond SM contributions which vanish in Type I, II and III Seesaw models.  $N_R$ are possible right-handed neutrinos.
$\mathcal{M}$ can be diagonalized by a unitary matrix $V$ in the form of $ V^T\mathcal{M}V=\widehat{M}=\operatorname{diag}\{m_1,m_2,m_3,M_1,M_2,M_3\}  = \operatorname{diag}\{\widehat{M}_l, \widehat{M}_h\}$ for three generations of light and heavy neutrinos  to obtain the mass eigenstate neutrinos,  $ \psi^m_{L}=V^\dagger \psi_{L} $. 

In the mass eigenstate basis, assuming the momenta of all generation neutrinos are the same, the initially SM interaction produced active left-handed neutrino states are given by
\begin{equation}
	\psi_{Li}^h=\begin{pmatrix}
		V_{i1}^*\begin{pmatrix}
			u^{h}\\
			0
		\end{pmatrix}\\
		...\\
		V_{i6}^*\begin{pmatrix}
			u^{h}\\
			0
		\end{pmatrix}
	\end{pmatrix}\;,\;\;\;
	(\psi_{Li}^h)^c=\begin{pmatrix}
		V_{i1}\begin{pmatrix}
			0\\
			u^{h}
		\end{pmatrix}\\
		...\\
		V_{i6}\begin{pmatrix}
			0\\
			u^{h}
		\end{pmatrix}
	\end{pmatrix}\;,
\end{equation}
where we have used $ (\psi^m_L)^c=((\psi^m)^c)_R $ and $ (\psi^m)^c=\psi^m $. $\psi_{Li}$ are the light neutrino for $i=1,2,3$ and the heavy neutrino for $i=4,5,6$, respectively.  We have
\begin{eqnarray}
	\mathcal{L}=\frac{1}{2}\left( 
	\bar\psi^m (i\slashed{\partial}-\widehat{M}) \psi^m
	\right) 
	-\bar\psi^m\widetilde{J}^\mu\gamma_\mu\frac{1-\gamma_5}{2}\psi^m\;,
\end{eqnarray}
and the equation of motion is
\begin{eqnarray}
	(i\slashed{\partial}-\widehat{M}) \psi^m-
	\widetilde{J}^\mu\gamma_\mu\frac{1-\gamma_5}{2}\psi^m
	+(\widetilde{J}^\mu)^*\gamma_\mu\frac{1+\gamma_5}{2}\psi^m=0
	\;.
\end{eqnarray}
Here $\psi^m = \psi^m_L + (\psi^m_L)^c$ and $\widetilde{J}^\mu= V^\dagger J^\mu V$. So the Hamiltonian $H$ for the general case is
\begin{eqnarray}
	\label{general Hamiltonian}
	H=(\mathbf{p}\cdot\boldsymbol{\Sigma})\gamma^5
	+ \beta \begin{pmatrix}
		\widehat{M}_l&0\\
		0&\widehat{M}_h
	\end{pmatrix}
	+ \widetilde{J}^\mu \gamma^0\gamma_\mu\frac{1-\gamma^5}{2}
	-\left(\widetilde{J}^\mu\right)^*\gamma^0\gamma_\mu\frac{1+\gamma^5}{2}\;.
\end{eqnarray}
From the expression of Hamiltonian, we can easily see that the helicity is conserved when neutrinos pass through the matter.

From the above, one can easily recover the usual matter oscillation formalism with $J^\mu_L = (\rho, \mathbf{0})$ in the relativistic case $p\gg M>m\gg\rho$. Keeping the leading effect in this limit, one obtains,
\begin{equation}
	H_{\mathrm{eff}}=p+\frac{M^\dagger M}{2p}
	-h\cdot \rho\;.
\end{equation}
Then we can find that matter effect would influence the contribution from mixing angle and mass square. 
Note that
for $h=-1$ helicity, it is the usual  leading order neutrino oscillation in matter which can cause matter induced MSW resonant effect. But for $h=+1$ or $\rho<0$, the matter effects are different.

To explicit the chiral oscillation, let us consider the simple case of just one light and one heavy neutrinos with seesaw mass matrix $\widehat{M} = \operatorname{diag}\{m,M\}$ so that the active neutrino $ \nu_{L} $ and sterile neutrino $ N_R $ are just one generation, respectively.
Then we can parameterize the mixing matrix as
\begin{equation}
	V=\begin{pmatrix}
		V_{a1}&V_{a2}\\
		V_{s1}&V_{s2}
	\end{pmatrix} 
	=\begin{pmatrix}
		\cos \theta & e^{i \eta}\sin \theta  \\
		-\sin \theta\;\;\; & e^{i \eta}\cos \theta 
	\end{pmatrix}\;.
\end{equation}
$\theta$ is the mixing angle between light and heavy neutrinos and $\eta$ is a Majorana phase which does not show up in the usual approximation for neutrino oscillation.

With $j^\mu_{R, RL} = 0$ for the case of having just SM interactions in matter, we have
\begin{equation}
	\widetilde{J}^\mu\gamma_\mu= V^\dagger J^\mu V\gamma_\mu
	=\begin{pmatrix}
		\frac{\rho}{2}(1+\cos2\theta)&\frac{\rho}{2}e^{i\eta}\sin2\theta\\
		\frac{\rho}{2}e^{-i\eta}\sin2\theta&\frac{\rho}{2}(1-\cos2\theta)
	\end{pmatrix}\gamma_0\;.
\end{equation}
The Hamiltonian for this system becomes
\begin{equation}
	\label{Dirac+Majorana Ham}
	H=\begin{pmatrix}
		\frac{\rho}{2}(1+\cos2\theta)-\mathbf{p}\cdot\boldsymbol{\sigma}
		&m&\frac{\rho}{2}e^{i\eta}\sin2\theta&0\\
		m&\mathbf{p}\cdot\boldsymbol{\sigma}-\frac{\rho}{2}(1+\cos2\theta)&0&-\frac{\rho}{2}e^{-i\eta}\sin2\theta\\
		\frac{\rho}{2}e^{-i\eta}\sin2\theta&0&
		\frac{\rho}{2}(1-\cos2\theta)-\mathbf{p}\cdot\boldsymbol{\sigma}&M\\
		0&-\frac{\rho}{2}e^{i\eta}\sin2\theta&M&\mathbf{p}\cdot\boldsymbol{\sigma}-\frac{\rho}{2}(1-\cos2\theta)
	\end{pmatrix}\;.
\end{equation}

It is interesting to note that one can obtain analytic eigen-values of the above $H$ and therefore study the oscillation behaviors in details. The eigen-values are given by
\begin{equation}
\label{general-energy}
\begin{array}{ll}
	E_{1h}=-\sqrt{A_1-A_2}\;,&
	E_{2h}=\sqrt{A_1-A_2}\;,\\
	E_{3h}=-\sqrt{A_1+A_2}\;,\;\;&
	E_{4h}=\sqrt{A_1+A_2}\;,
\end{array}
\end{equation}
where
\begin{equation}
	\label{eigen H}
\begin{aligned}
	A_{1}=&\frac{m^2+M^2}{2}+\left(h\cdot p-\frac{\rho}{2}\right)^2+\frac{\rho^2}{4}\\
	A_{2}=&\frac{\sqrt{
			\Big((m^2-M^2)\cos2\theta- 2 \rho (h\cdot p-\tfrac{\rho}{2}) \Big)^2
			+\Big(\left(m^2-M^2\right)^2
			+ \rho^2\left( m^2+M^2-2mM\cos2\eta\right)\Big)
			\sin^2 2\theta
	}}{2}
\end{aligned}
\end{equation}
From the above one can obtain the evolution matrix $U(t)$. The elements in $U(t)$ are known analytic functions of $ m $, $ M $, $p$, $\rho$, $\theta$ and $\eta$. The important features we would like to mention are that, there are resonant chiral oscillation for a given matter density and also the Majorana phase $\eta$ can affect chiral rotations. We discuss different cases in the following.

If one just considers two active Majorana neutrino, such as $ \nu_e $ and $ \nu_\mu $ in Type-II seesaw model, we can get the results from the case of one active and one sterile neutrino directly with the replacement
\begin{equation}
	\label{replace}
	\nu_{L}\rightarrow\nu_{eL}\;,\;\;N_{R}^c\rightarrow\nu_{\mu L} \;,\;\;
	h\cdot p\rightarrow h\cdot p-\rho_Z \;,\;\; \rho\rightarrow \rho_W\;.
\end{equation}
We will give some more detailed examples in following discussion.

Before discussing the general case, we discuss how the pure Dirac neutrino and type II Seesaw neutrino oscillations are affected by matter effects.
Setting $\rho = 0$, $M_{L, R} =0$, and $M=m$, and combining $\nu_L + N_R$ to have a  Dirac state $\psi$, one recovers the free space Dirac neutrino oscillation case that we discussed in the previous sections. If $\rho$ is not zero, we have
$H = (\mathbf{p}\cdot\boldsymbol{\Sigma})\gamma^5+ m\beta+\rho (1-\gamma_5 )/2$
with eigen-values given by
\begin{eqnarray}
E_1= {\rho\over 2} + E_h\;,\;\;E_2= {\rho\over 2} - E_h
\;,
\end{eqnarray}
where $ E_h = \sqrt{m^2 + (h\cdot p-\rho/2)^2} $,  and the corresponding eigen-states
\begin{equation}
	\psi_{1}=\frac{1}{\sqrt{2E_h}}
	\begin{pmatrix}
		\sqrt{E_h-(h\cdot p-\frac{\rho}{2})}\ u^h\\
		\sqrt{E_h+(h\cdot p-\frac{\rho}{2})}\ u^h
	\end{pmatrix}\;,\;\;
	\psi_{2}=\frac{1}{\sqrt{2E_h}}
	\begin{pmatrix}
		\sqrt{E_h+(h\cdot p-\frac{\rho}{2})}\ u^h\\
		-\sqrt{E_h-(h\cdot p-\frac{\rho}{2})}\ u^h
	\end{pmatrix}\;.
\end{equation}
$E_1$ and $E_2$ correspond to the positive and negative energy cases in free space propagation. Note that 
each of the original energy is splited into two levels depending on the helicity of the states.

After obtaining the evolution operator $U(t)=\exp(-iHt)$, one can easily derive the oscillation probabilities for given helicity and chirality with,
\begin{eqnarray}
	\label{interaction Dirac amplitude}
	P(\nu_L^{h}\rightarrow \nu_L^{h})=
	1-\frac{m^2}{E_h^2}\sin^2(E_h t)\;,\;\;
	P(\nu_L^h\rightarrow \nu_R^h)
	=\frac{m^2}{E_h^2}\sin^2(E_h t)\;.
\end{eqnarray}
One can also obtain the similar results for a right-chiral initial state.

Setting $M=0$, $M_D =0$, $j^{\mu}_{RL}=0$, and $ \theta=0 $, one obtains the pure active left-handed Majorana neutrino case which can be realized in Type II seesaw model. In this case Hamiltonian $H$ is given by $H =(\mathbf{p}\cdot\boldsymbol{\Sigma})\gamma^5+ m\beta -\rho\gamma^5$. 
The positive eigen-energy is $E_h = \sqrt{m^2 + (h\cdot p-\rho)^2}$ and the negative one is $-E_h$.
Then the chiral oscillation amplitudes will be modified with $\rho/2$ replaced by $\rho$ in Eq.(\ref{interaction Dirac amplitude}),
and the expression of chiral oscillation $P(\nu^h_{L}\rightarrow \nu^h_{L})$ and $P(\nu^h_{L}\rightarrow (\nu^h_{L})^c)$ would be same as Eq.(21) with the replacement $\nu_R^h$ to $(\nu^h_{L})^c$.
The resonant ehanced chiral oscillation occurs at $p = \rho$, which is different to Eq.(\ref{interaction Dirac amplitude}). 

In the usual case, $p\gg m\gg\rho$, the contribution of $\rho$ is very small so that Eq.(\ref{interaction Dirac amplitude}) degenerate to Eq.(\ref{free}). While in the dense matter, considering the case that $ p $ and $ \rho $ are comparable, the matter effect can make a large contribution to chiral oscillation. For example, considering the matter effect in the internal of neutron star, 
$\rho=-G_F N_n/\sqrt{2} =-(3.82\times10^{-14}\mathrm{eV})\cdot a/(\mathrm{g}/\mathrm{cm}^3)$
where the mass density $a$ can be as large as $10^{15} \mathrm{g/cm}^3$ \cite{Ozel:2016oaf}, leading to a not so small $\rho$. 
For $h=-1$ there's a resonant density when $\rho/2$ ({or} $\rho$) is equal to momentum for Dirac neutrino (or Type II Seesaw Neutrino) leading to $E_h = m$. Then the chiral oscillation probability become the largest. While as for $h=+1$, there  is no such a resonant effect.

\section{Effects of Majorana phase in neutrino chiral oscillation}

For neutrino chiral oscillation, the Majorana phases can also play a role unlike the usual neutrino oscillation. Let us study the simple case of general seesaw neutrino oscillation in Sect. II. In vacuum, setting $\rho=0$ in Eq. (11) one obtains the general seesaw neutrino oscillation probabilities for one light and one heavy neutrinos in vacuum as 
\begin{equation}
	\label{free prob}
\begin{aligned}
	&P(\nu_{L}^h\rightarrow \nu_{L}^h)
	=\left(\cos^2\theta\cos(E_m t)
	+\sin^2\theta\cos(E_M t)\right)^2
	+p^2\left(\frac{\cos^2\theta}{E_m}\sin(E_m t)
	+\frac{\sin^2\theta}{E_M}\sin(E_M t)\right)^2\\
	&P(\nu_{L}^h\rightarrow (N^h_{R})^c)
	=\frac{\sin^2 2\theta}{4}\left( \left(\cos(E_m t)-\cos(E_M t)\right)^2
	+p^2\left(\frac{\sin(E_m t)}{E_m}-\frac{\sin(E_M t)}{E_M}\right)^2\right)\\
	&P(\nu_{L}^h\rightarrow (\nu_{L}^h)^c)
	=\frac{m^2}{(E_m)^2}\cos^4\theta\sin^2(E_mt)
	+\frac{mM}{2E_mE_M}\sin^2 2\theta\cos2\eta\sin(E_mt)\sin(E_Mt)
	+\frac{M^2}{(E_M)^2}\sin^4\theta\sin^2(E_Mt)\\
	&P(\nu_{L}^h\rightarrow N_{R}^h)
	=\frac{\sin^2 2\theta}{4}\left(\frac{m^2}{(E_m)^2}\sin^2(E_m t)-\frac{2mM}{E_m E_M}\cos2\eta\sin(E_m t)\sin(E_Mt)
	+\frac{M^2}{(E_M)^2}\sin^2(E_Mt)\right)\;,
\end{aligned}
\end{equation}
where $E_m = \sqrt{m^2 + p^2}$ and $E_M = \sqrt{M^2+p^2}$.
The last two equations provide information about chiral rotation. 

We see that the Majorana phase $\eta$ appear in the last two equations of Eq.(\ref{free prob}) for the chiral oscillation probabilities.  This offers the possibility to obtain information about Majorana phase using neutrino chiral oscillation. We show this fact by an illustrative example in Fig. \ref{free fig} with the non-relativistic value for neutrino momentum $p=0.01$eV,  and $m=0.01\mathrm{eV}$, $M=1\mathrm{eV}$. For the mixing angle we take a  canonical form $\sin\theta=\sqrt{m/M}=0.1$, and let the Majorana phase have two different values $\eta=0, \pi/4$.
Note that the parameter $m$ is the light neutrino mass which satisfied with the constraints from experiment \cite{Planck:2018vyg}.
% Figure environment removed

Combined with the Eq.(\ref{free prob}), we know that the chiral oscillation probability would be superposition of three  trigonometric function, whose frequencies are $ 2E_m $, $ E_M $ and $ 2E_M $, respectively.
As for the oscillation $ \nu_{L}^h\rightarrow (\nu_{L}^h)^c $, the contribution from $ 2E_M $ frequency term is much smaller than other terms as it's suppressed by $ \sin^4\theta $, so we can only see the $ 2E_m $ and $ E_M $ frequency terms in Fig. \ref{free fig}(a) $ \nu_{L}^h\rightarrow (\nu_{L}^h)^c $. And in Fig. \ref{free fig}(b) $ \nu_{L}^h\rightarrow (\nu_{L}^h)^c $, $ E_M $ frequency term vanishes as $ \eta=\pi/4 $, so we can only see the $ 2E_m $  frequency term in oscillation $ \nu_{L}^h\rightarrow (\nu_{L}^h)^c $. While the contributions of three terms are of the same order of magnitude when it comes to oscillation $ \nu_{L}^h\rightarrow N_R $, so we can see the superposition of three terms in Fig. \ref{free fig}(a) $ \nu_{L}^h\rightarrow N_R $. And in Fig. \ref{free fig}(b)  $ \nu_{L}^h\rightarrow N_R $ we can just see the superposition of $ 2E_m $ and $ 2E_M $ frequency terms as $ E_M $ term vanishes when $ \eta=\pi/4 $. One can clear see that the oscillation patterns do depend on the values  of the Majorana phase $\eta$. 
However, such effects may be very challenging to observe since
the term proportional to $\cos2\eta$ vanished when taking time average, which is also true  in more generation.
The time average oscillation probabilities which do not depend on the Majorana phase are given by
\begin{equation}
\begin{aligned}
&P(\psi_{Li}^h\rightarrow\psi_{Lj}^h)
		=\sum_{k}\left|V_{ik}^*V_{jk} \right|^2\left(\frac{1}{2}+\frac{p^2}{2E_k^2}\right)\;,\;\;
		P(\psi_{Li}^h\rightarrow(\psi_{Lj}^h)^c)
		=\sum_{k}\left|V_{ik}^*V_{jk}^*\right|^2\frac{m_k^2}{2E_k^2}\;.
\end{aligned}
\end{equation}
Besides, we would like to  emphasis that the results in Eq.(\ref{free prob}) can also apply to the oscillation between two active Majorana neutrinos in Type-II seesaw model by replacing the parameters and $ \nu_{L}$, $ N_{R}^c $ to $\nu_{eL}$, $\nu_{\mu L} $, respectively.

One anticipates that the oscillation patterns in matter will be more complicated than those in free space. An interesting finding is that in this
case, the chiral oscillation pattern in matter dependence on $\eta$ does not vanish even time average is taken. We illustrate this with the limit $p\gg M>m\gg\rho$. In this case we have
\begin{equation}
\begin{aligned}
	&P(\nu^h_{L}\rightarrow (\nu^h_{L})^c)
	=\frac{(m^2-M^2)^2}{8A_2^2}\left(
	\frac{(m^2\cos^4\theta+M^2\sin^4\theta)}{p^2}
	-\frac{\rho(4m^2\cos^6\theta-4M^2\sin^6\theta+mM\cos2\eta\cos2\theta\sin^2 2\theta)}{p(m^2-M^2)} \right .\\
	%&\hspace{2.3cm}
	&\qquad\qquad\qquad\qquad\left . +\frac{2\rho^2(2m^2\cos^4\theta+2M^2\sin^4\theta+mM\cos2\eta\sin^2 2\theta)}{(m^2-M^2)^2}\right)\;,\\
	&P(\nu^h_{L}\rightarrow N^h_{R})
	=\frac{(m^2-M^2)^2\sin^2 2\theta}{32A_2^2}
	\left(\frac{m^2+M^2}{p^2}
	-\frac{2\rho(m^2+M^2-2mM\cos2\eta)\cos2\theta}{p(m^2-M^2)} \right .\\
	&\hspace{2.9cm}\left . +\frac{4\rho^2\left( m^2+M^2-2mM\cos2\eta\right)}{(m^2-M^2)^2}
	\right)\;.
\end{aligned}
\end{equation}
The probabilities for the other two oscillation modes are
\begin{equation}
	\label{usual}
	P(\nu^h_{L}\rightarrow \nu^h_{L})
	=\frac{1}{2}+\frac{\cos^2 2\theta_{\mathrm{eff}}}{2} - P(\nu^h_{L}\rightarrow (\nu^h_{L})^c)
	\;,\;\;\;
	P(\nu^h_{L}\rightarrow (N^h_{R})^c)
	=\frac{\sin^2 2\theta_{\mathrm{eff}}}{2} - P(\nu^h_{L}\rightarrow N^h_{R})\;.
\end{equation}
where 
	$
	\cos2\theta_{\mathrm{eff}}
	=\left( (M^2-m^2)\cos2\theta+ 2 \rho (h\cdot p-\rho/2)\right) /(2A_2)$.
We do see the oscillation probabilities dependence one $ \eta $ even taking the time average. Note that the effect vanishes if any of the $m$, $M$ and $\theta$ is zero. 

We have carried a more detailed numerical analysis to see the matter and Majorana phase effects on the oscillation pattern in matter after taking the time average, which are shown in Fig. \ref{matter effect} and \ref{Majorana12 fig}.
  
In Fig. \ref{matter effect}, we show the effects of a non-zero $\rho$ effects on different states of helicity $h=\pm 1$ with $\eta = 0$. 
In the $\nu_{L}\rightarrow \nu_L$ oscillation  picture, the blue region shows the smallest survival probability, which corresponds to the significant disappearance probability. And in other three pictures, the red region shows the considerable appearance probability.  In the $\nu_{L}\rightarrow N_R^c$ oscillation picture of Fig. \ref{matter effect}(a), when energy is large enough, we can see the explicit matter effect in red region, which is consistent with MSW effect for $\rho >0$. But for $\rho <0$ there is no MSW effect. 
Besides, in low energy region, we can see that the MSW effect would be suppressed, and the $\nu_{L}\rightarrow N_R$  oscillation would show the matter effect explicitly. As for low matter density and low energy region, the probability for $\nu^h_{L}\rightarrow (\nu^h_{L})^c $ oscillation becomes significant, which is consistent with the vacuum non-relativistic results.
It is interesting to note that there are two resonant regions in Fig. \ref{matter effect}(b).  The resonant region in $\nu^h_{L}\rightarrow (\nu^h_{L})^c$ oscillation is consistent with the vacuum case before. While the resonant region in $\nu^h_{L}\rightarrow N_R$ oscillation is additional resonant effect. This is because the leading order in probability is proportional to $ \rho^2/A_2^2 $ where the expression of $ A_2 $ is in Eq.(\ref{eigen H}), and the bending direction is opposite to the MSW effect.
 

% Figure environment removed

As a last example, we consider time average probability of two active Majorana neutrinos, $ \nu_e $ and $ \nu_\mu $, oscillation case to see how Majorana phase affect oscillation pattern. The formulae can be obtained by replacement in Eq.(\ref{replace}) from the above and $M$ and $m$ are now the two tiny neutrino mass in SM. The results are shown in Fig. \ref{Majorana12 fig}.  We can see that in this case, the Majorana phase dependence is significant. 
% Figure environment removed


\section{Discussions and conclusions}


We have studied the effects of Majorana phase and matter effects on neutrino chiral oscillation in vacuum and in matter. 
There is a chiral oscillation between left- and right- chiral neutrinos,  besides the usual oscillation between different generations. The probability of chiral oscillation is suppressed by a factor of $m^2/E^2$ in vacuum. For relativistic neutrinos, chiral oscillation effects are therefore small. But for non-relativistic neutrinos, this effects can be significant. In matter, the effects can be magnified if the momentum and matter density are close to each other. We have found that the neutrino effective eigen-energies are split into two different ones depending on the helicity of the neutrino. This results in different oscillation behavior for neutrinos with different helicity, in particular there is a new resonant effect related to the helicity state of neutrino different than the usual MSW effect. We have also shown by analytic calculation for some limiting cases and also detailed numerical calculation that for Majorana neutrinos, chiral oscillation also depends on Majorana phases. We will present some of the interesting new phenomenological implications for neutrino chiral oscillation in cosmology and astrophysics elsewhere.\footnote{Work in preparation.}


\section*{Acknowledgments}
This work was supported in part by Key Laboratory for Particle Physics, Astrophysics and Cosmology, Ministry of Education, and Shanghai Key Laboratory for Particle Physics and Cosmology (Grant No. 15DZ2272100), and in part by the NSFC (Grant Nos. 11735010, 11975149, and 12090064). XGH was supported in part by the MOST (Grant No. MOST 106-2112-M-002-003-MY3 ).

\begin{thebibliography}{99}
	%\cite{ParticleDataGroup:2022pth}
\bibitem{ParticleDataGroup:2022pth}
R.~L.~Workman \textit{et al.} [Particle Data Group],
%``Review of Particle Physics,''
PTEP \textbf{2022}, 083C01 (2022)
doi:10.1093/ptep/ptac097
%1331 citations counted in INSPIRE as of 22 Jul 2023

%\cite{Bittencourt:2020xen}
\bibitem{Bittencourt:2020xen}
V.~A.~S.~V.~Bittencourt, A.~E.~Bernardini and M.~Blasone,
%``Chiral oscillations in the non-relativistic regime,''
Eur. Phys. J. C \textbf{81}, no.5, 411 (2021)
doi:10.1140/epjc/s10052-021-09209-2
[arXiv:2009.00084 [hep-ph]].
%7 citations counted in INSPIRE as of 15 May 2023

%\cite{Ge:2020aen}
\bibitem{Ge:2020aen}
S.~F.~Ge and P.~Pasquini,
%``Parity violation and chiral oscillation of cosmological relic neutrinos,''
Phys. Lett. B \textbf{811}, 135961 (2020)
doi:10.1016/j.physletb.2020.135961
[arXiv:2009.01684 [hep-ph]].
%8 citations counted in INSPIRE as of 03 Jul 2023

%\cite{Kimura:2021qlh}
\bibitem{Kimura:2021qlh}
K.~Kimura and A.~Takamura,
%``New CP Phase and Exact Oscillation Probabilities of Dirac Neutrino derived from Relativistic Equation,''
[arXiv:2101.03555 [hep-ph]].
%5 citations counted in INSPIRE as of 18 Jul 2023

%\cite{Bittencourt:2022hwn}
\bibitem{Bittencourt:2022hwn}
V.~A.~S.~V.~Bittencourt, A.~E.~Bernardini and M.~Blasone,
%``Chiral oscillations,''
EPL \textbf{139}, no.4, 44002 (2022)
doi:10.1209/0295-5075/ac8446
%3 citations counted in INSPIRE as of 03 Jul 2023

%\cite{Blasone:2022fgj}
\bibitem{Blasone:2022fgj}
M.~Blasone, V.~A.~S.~V.~Bittencourt and A.~E.~Bernardini,
%``Chiral oscillations in three-flavor neutrino mixing,''
PoS \textbf{CORFU2021}, 065 (2022)
doi:10.22323/1.406.0065
%1 citations counted in INSPIRE as of 03 Jul 2023

%\cite{Wolfenstein:1977ue}
\bibitem{Wolfenstein:1977ue}
L.~Wolfenstein,
%``Neutrino Oscillations in Matter,''
Phys. Rev. D \textbf{17}, 2369-2374 (1978)
doi:10.1103/PhysRevD.17.2369
%5752 citations counted in INSPIRE as of 20 Jul 2023

%\cite{Mikheyev:1985zog}
\bibitem{Mikheyev:1985zog}
S.~P.~Mikheyev and A.~Y.~Smirnov,
%``Resonance Amplification of Oscillations in Matter and Spectroscopy of Solar Neutrinos,''
Sov. J. Nucl. Phys. \textbf{42}, 913-917 (1985)
%4053 citations counted in INSPIRE as of 19 Jul 2023

%\cite{Pantaleone:1992xh}
\bibitem{Pantaleone:1992xh}
J.~T.~Pantaleone,
%``Dirac neutrinos in dense matter,''
Phys. Rev. D \textbf{46}, 510-523 (1992)
doi:10.1103/PhysRevD.46.510
%178 citations counted in INSPIRE as of 07 Jul 2023

%\cite{Ozel:2016oaf}
\bibitem{Ozel:2016oaf}
F.~\"Ozel and P.~Freire,
%``Masses, Radii, and the Equation of State of Neutron Stars,''
Ann. Rev. Astron. Astrophys. \textbf{54}, 401-440 (2016)
doi:10.1146/annurev-astro-081915-023322
[arXiv:1603.02698 [astro-ph.HE]].
%921 citations counted in INSPIRE as of 21 Jul 2023

%\cite{Planck:2018vyg}
\bibitem{Planck:2018vyg}
N.~Aghanim \textit{et al.} [Planck],
%``Planck 2018 results. VI. Cosmological parameters,''
Astron. Astrophys. \textbf{641}, A6 (2020)
[erratum: Astron. Astrophys. \textbf{652}, C4 (2021)]
doi:10.1051/0004-6361/201833910
[arXiv:1807.06209 [astro-ph.CO]].
%11098 citations counted in INSPIRE as of 21 Jul 2023

%\cite{Chamel:2008ca}
\bibitem{Chamel:2008ca}
N.~Chamel and P.~Haensel,
%``Physics of Neutron Star Crusts,''
Living Rev. Rel. \textbf{11}, 10 (2008)
doi:10.12942/lrr-2008-10
[arXiv:0812.3955 [astro-ph]].
%410 citations counted in INSPIRE as of 13 Jul 2023
\end{thebibliography}
	



	
	
\end{document}
