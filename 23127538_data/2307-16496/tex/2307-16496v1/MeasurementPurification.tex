\documentclass[12pt]{iopart}
\usepackage{graphicx}% Include figure files
\newcommand{\red}[1]{\textcolor{red}{#1}}
\newcommand{\blue}[1]{\textcolor{blue}{#1}}
\newcommand{\non}{\nonumber}

\begin{document}

\title{Generic eigenstate preparation via measurement-based purification}

\author{Jia-shun Yan}
\address{School of Physics, Zhejiang University, Hangzhou 310027, Zhejiang, China}

\author{Jun Jing}
\address{School of Physics, Zhejiang University, Hangzhou 310027, Zhejiang, China}
\ead{jingjun@zju.edu.cn}

\begin{abstract}
Out of the general thought, a quantum system can be prepared into a target eigenstate through repeated measurements on a coupled ancillary qubit rather than direct transitions in the Hamiltonian. In this work, we find that the positive operator-valued measures (POVMs) on the system, which is induced by the projective measurement on the qubit, can filter out the unwanted states except the target one. We discuss the measurement-based purification of entanglement in which maximally entangled states (Bell states and GHZ states) can be distilled from the maximally mixed states, and demonstrate the significant acceleration of a stimulated Raman adiabatic passage (STIRAP). Our scheme is not limited to the nondegenerate systems and allows arbitrary eigenstate generation. It offers a promising way to a generic state-preparation algorithm, enriching the functionalities of general quantum measurement.
\end{abstract}

\vspace{2pc}
\noindent{\it Keywords}: quantum measurement, eigenstate preparation, entanglement purification, stimulated Raman adiabatic passage

\section{Introduction}

Quantum state preparation is a basic and crucial premise for plenty of modern quantum applications, including but not limited to measurement-based quantum computation~\cite{OneWay,MeasurementComputation}, quantum teleportation~\cite{QuantumTeleportation,QuantumTeleportation2}, quantum dense coding~\cite{DenseCoding}, and quantum cryptography~\cite{Cryptography}. Preparing eigenstates, especially for a complex Hamiltonian, is of great importance in various fields, such as quantum chemistry~\cite{FermionicHamEigenstates,QuantumChemistry} and condensed-matter physics~\cite{ManybodyLocalization,ManybodyRMP}. Various powerful tools have been applied in pushing the system of interest into a target eigenstate, including dissipative preparation of entanglement~\cite{Dissipative,PreparationByMarkovProcess}, variational quantum algorithms~\cite{VQA,VQE}, and shortcuts to adiabaticity~\cite{STAThreelevel,STA,STAQi}. Among them, distilling an initial mixed state into a desired state of a high fidelity distinguishes itself since any quantum system is inevitably coupled to an external environment. It is therefore reasonable to find that state purification and entanglement purification have developed as key technologies in quantum information and quantum computation~\cite{PurificationViaNoisyChannels,MixedStateEntanglement,QubitPurification,PurificationZhong}.

Frequent quantum measurements over a noncommutative operator with respect to the Hamiltonian could freeze the measured quantum system at an eigenstate by affecting in an asymptotic way the system dynamics, known as quantum Zeno effect~\cite{ZenoParadox}. When the measurement operator becomes parametric dependent, the measured system could be steered to a target state from either a pure state~\cite{AharonovZeno,NonselectiveAharonovZeno} or a mixed state~\cite{MeasurementEvolution,NonselectiveMeasurementEvolution} through a finite number of measurements with nonvanishing measurement intervals. As the dimension of the system becomes larger, however, it is more difficult to implement direct measurements on the system. Alternatively, quantum engineering could be investigated through indirect measurements on an ancillary system. In general, a projective measurement or postselection on the ancillary system gives rise to a positive operator-valued measure (POVM) on the interested system~\cite{MeasurementCharging,Naimark}, which can be navigated to a target state with a finite probability~\cite{MeasurementPurification}. The indirect measurement method has a wide range of applications associated with state purification, such as cooling a resonator to its ground state~\cite{MeasurementCooling,MeasurementCoolingPRL,ExternalLevelCooling,ExpOneModeCooling,HeraldedControlMotion}, enhancing the bath spin polarization~\cite{SpinBathState,Polarization}, and charging a quantum battery~\cite{NonselevtiveCharging,MeasurementCharging}. Nevertheless, the purification of the system with repeated measurements on the ancillary system is under certain constraints, e.g., the target states cannot be arbitrarily chosen~\cite{MeasurementPurification}, the target system is required to be nondegenerate~\cite{MeasurementEigensolver}, and all the system eigenstates have to be connected directly or indirectly through given transitions~\cite{MeasurementSteering}. It is then desired to find a generic scheme capable of distilling a degenerate or nondegenerate system from an arbitrary mixed state into an eigenstate with a limited number of ancillary systems.

In this work, we propose a general framework that an interested system can be purified to an arbitrary eigenstate by repeatedly measuring a coupled ancillary qubit. Rather than transferring the system population to the target state, we build up transitions among the other states to shuffle their populations and use projective measurements to filter them out. By virtue of the population renormalization in comply with the non-unitary operations, there would be a unique population rise for the target state. We apply our framework into entanglement purification by preparing a double-qubit system into Bell states and a three-qubit system into Greenberger-Horne-Zeilinger (GHZ) states. It exemplifies creating maximally entangled states~\cite{SingletPreparation,QuantumEntanglement} from maximally mixed states. In comparison to the scheme based on the nonselective measurements~\cite{MeasurementSteering}, in which each transition channel towards the target state is accompanied with a detector (ancillary qubit), our framework requires only one ancillary qubit. Moreover, it avoids the three-body interaction and the frequent shifting between the system Hamiltonian and the system-detector interaction Hamiltonian. Also our framework of state purification can be combined with the standard stimulated Raman adiabatic passage (STIRAP), by which a complete population transfer is promoted even in a diabatic passage.

The rest part of this work is structured as follows. In Sec.~\ref{ProbPurification}, a generic framework is introduced for eigenstate preparation based on repeated measurements on the ancillary qubit. A necessary condition for state purification is established through defining a purification operator in the system space, given knowledge about the energy spectrum of the system. In Sec.~\ref{Uniqueness}, we show that the steady state of the system in the limit of an infinite number of measurements is exactly the same as the target state with a given purification operator. In Sec.~\ref{UniquenessTotal}, our framework is extended to a more general system Hamiltonian and a necessary condition is provided about the eigenstate preparation. In Sec.~\ref{Sec:PrepareEntangledState}, we apply our framework to prepare a singlet Bell state of a pair of qubits and the GHZ state in a three-qubit system. In Sec.~\ref{Sec:STIRAP}, we present two hybrid models consisting of STIRAP and state purification in a three-level system to demonstrate an accelerated adiabatic passage. We summarize our work in Sec.~\ref{Sec:Conclusion}.

\section{Theoretical framework}\label{Sec:Model}

In general, two necessary conditions have to be fulfilled for state purification through quantum measurements. The first condition is that the population over the target state would be ever increased upon a desired measurement outcome until approaching unit. The second condition is that the target state is the unique state approached by the system under a sufficient number of measurements. In this section, we first illustrate our scheme on increasing the population over the target state and the coincidence between the steady state and the target state when the system Hamiltonian in the interaction picture is in a specified formation. Then the system Hamiltonian is relieved to a general form, by which we discuss the condition of state purification through measurements.

\subsection{Purification operator and probabilistic purification}\label{ProbPurification}

% Figure environment removed

In our framework of state preparation and purification by measurement, the target state $|\Psi_{\rm target}\rangle$ is a given eigenstate of the system Hamiltonian $H_S$, i.e., $H_S|\Psi_{\rm target}\rangle=\lambda|\Psi_{\rm target}\rangle$, where $\lambda$ is the eigenvalue. The free Hamiltonian for the ancillary qubit is $H_A$. Suppose we can build up a purification operator $Q$ as shown in Fig.~\ref{Fig:Qoperator} and in the interaction picture with respect to $H_0=H_S+H_A$, the effective or purification Hamiltonian can be written as
\begin{equation}\label{Ham}
H_P=g_a\left(\sigma_a^\dagger Q+\sigma_aQ^\dagger\right)=g_a\left[\begin{array}{cc}0&Q \\ Q^\dagger&0 \end{array}\right].
\end{equation}
Here $g_a$ is the coupling strength between the system and the ancillary qubit. $\sigma_a\equiv|\varphi\rangle\langle\varphi_\perp|$ and $\sigma_a^\dagger\equiv|\varphi_\perp\rangle\langle\varphi|$ are transition operators for the ancillary qubit about the initial state $|\varphi\rangle$ and its orthogonal counterpart $|\varphi_\perp\rangle$. $\langle\varphi|\varphi_{\perp}\rangle=0$. To purify the system into the target state, the system operator $Q$ is constructed by connecting all the unwanted eigenstates of the system $H_S$, which is assumed to be time-independent for simplicity. In another word, $Q$ is required to annihilate the target state
\begin{equation}\label{PurificationCondition}
Q|\Psi_{\rm target}\rangle=0.
\end{equation}
And it is not necessary to have $Q^\dagger|\Psi_{\rm target}\rangle=0$, i.e., $Q$ could involve with the target state although it does not contain a directed path towards $|\Psi_{\rm target}\rangle$.

Given the Hamiltonian in Eq.~(\ref{Ham}), the joint time-evolution operator of both system and ancillary qubit for a period of $\tau$ is then
\begin{equation}\label{Utot}
U(\tau)=\left[\begin{array}{cc}C^T(\tau)&S^\dagger(\tau)\\S(\tau)&C(\tau)\end{array}\right],
\end{equation}
where the Kraus operators are
\begin{equation}\label{KrausOperator}
C(\tau)=\sum_{k=0}^N\frac{(-i\tau)^{2k}}{(2k)!}\left(Q^\dagger Q\right)^k,\quad S(\tau)=\sum_{k=0}^N\frac{(-i\tau)^{2k+1}}{(2k+1)!}\left(Q^\dagger Q\right)^kQ^\dagger,
\end{equation}
respectively. The time evolution of the whole system under the Hamiltonian $H_P$ is repeatedly interrupted by the projective measurement about the initial state of the ancillary system $M_\varphi\equiv|\varphi\rangle\langle\varphi|$. Any two neighboring measurements can be arbitrarily spaced in time, so that our method is essentially robust against the systematic error about the measurement moments. After $m-1$ rounds of evolution and successful measurement, the whole system state is $\rho_{\rm tot}=\rho_s(t)\otimes|\varphi\rangle\langle\varphi|$, where $t=\sum_{j=1}^{m-1}\tau_j$ with $\tau_j$'s indicating the measurement intervals. Then the whole system state becomes
\begin{equation}\label{rhotot}
\rho_{\rm tot}(t+\tau_m)=\frac{M_\varphi U(\tau_m)\rho_s\otimes|\varphi\rangle\langle\varphi|U^\dagger(\tau_m)M_\varphi}{P_\varphi^{(m)}}
=\rho_s(t+\tau_m)\otimes|\varphi\rangle\langle\varphi|
\end{equation}
after another round of evolution and measurement lasting $\tau_m$, where $P_\varphi^{(m)}\equiv{\rm Tr}[C(\tau_m)\rho_s(t)C^\dagger(\tau_m)]$ represents the success probability of the $m$th round. In particular, if the measurement outcome is as desired, i.e., the ancillary qubit is reset as the initial state, then the system and the ancillary qubit are decoupled from each other as in Eq.~(\ref{rhotot}). According to the Naimark's dilation theorem~\cite{Naimark}, the projective measurements performed on the ancillary qubit induce a POVM $\mathcal{M}(\tau)[\mathcal{O}]\equiv C(\tau)\mathcal{O}C^\dagger(\tau)$ acting on the system. Then the system state in Eq.~(\ref{rhotot}) can be expressed as
\begin{equation}\label{rhos}
\rho_s(t+\tau_m)=\frac{C(\tau_m)\rho_s(t)C^\dagger(\tau_m)}{P_\varphi^{(m)}},
\end{equation}
according to the time-evolution operator in Eq.~(\ref{Utot}).

Since the purification operator annihilates the target state $Q|\Psi_{\rm target}\rangle=0$, we have $C(\tau_m)|\Psi_{\rm target}\rangle=|\Psi_{\rm target}\rangle$ according to Eq.~(\ref{KrausOperator}), i.e., the POVM $\mathcal{M}(\tau_m)$ does not directly change the population over the target state
\begin{equation}
\langle\Psi_{\rm target}|\mathcal{M}(\tau_m)[\rho_s(t)]|\Psi_{\rm target}\rangle=\langle\Psi_{\rm target}|\rho_s(t)|\Psi_{\rm target}\rangle.
\end{equation}
As $C(\tau)$ is a non-unitary operator, the system state should be renormalized by the measurement probability $P_\varphi^{(m)}$. Then $1/P_\varphi\geq1$ in either Eq.~(\ref{rhotot}) or Eq.~(\ref{rhos}) acts as a gain factor raising the target-state population. It follows with a purification inequality:
\begin{equation}
\langle\Psi_{\rm target}|\rho_s(t+\tau_m)|\Psi_{\rm target}\rangle\geq\langle\Psi_{\rm target}|\rho_s(t)|\Psi_{\rm target}\rangle.
\end{equation}
A successful measurement then suggests that the ancillary qubit in its initial state heralds a population rise over the target state.

After $m$ rounds of measurements, the overlap between the system state and the target state $F_m\equiv\langle\Psi_{\rm target}|\rho_s(\sum_{j=1}^m\tau_j)|\Psi_{\rm target}\rangle$, i.e., the target-state fidelity, could be expressed as
\begin{equation}\label{fidelity}
F_m=\frac{F_{m-1}}{P_\varphi^{(m)}}=\frac{F_{m-2}}{P_\varphi^{(m)}P_\varphi^{(m-1)}}=\cdots=\frac{\langle\Psi_{\rm target}|\rho_s(0)|\Psi_{\rm target}\rangle}{\prod_{j=1}^{m}P_\varphi^{(j)}}.
\end{equation}
Therefore the system can be gradually purified under more and more rounds of evolution and measurement, provided that its initial population over the target state is not vanishing. Nevertheless, the outcome of our framework is probabilistic as indicated by the denominator of the last equivalence in Eq.~(\ref{fidelity}), i.e., the success probability $P_s=\prod_{j=1}^{m}P_\varphi^{(j)}$. One can find that $P_s$ is lower-bounded by the initial population over the target state.

\subsection{Uniqueness of purified state}\label{Uniqueness}

Under a sufficient number $m$ of measurements, the measurement probability of the ensued rounds would approach unit $P_\varphi^{(j\geq m)}\rightarrow1$, otherwise the target-state fidelity will keep increasing with no upper bound. In this situation, the ancillary qubit is freezed at the initial state $|\varphi\rangle$, the system approaches a steady state, and the success probability $P_s$ becomes invariant in time. We can show that the steady state is exactly the target state $|\Psi_{\rm target}\rangle$.

Note that the Kraus operator $C(\tau)$ in Eq.~(\ref{KrausOperator}) is Hermitian and $C(\tau_m)|\Psi_{\rm target}\rangle=|\Psi_{\rm target}\rangle$. Then the operator can always be expanded as
\begin{equation}
C(\tau_m)=|\Psi_{\rm target}\rangle\langle\Psi_{\rm target}|+ {\sum_k}'\epsilon_k(\tau_m)|\psi_k(\tau_m)\rangle\langle\psi_k(\tau_m)|
\end{equation}
where $|\psi_k(\tau_m)\rangle$'s and $\epsilon_k(\tau_m)$'s are instantaneous eigenstates and eigenvalues of $C(\tau_m)$, respectively. ${\sum_k}'$ indicates the summation over all degrees of freedom except the target state, whose eigenvalue is one. Consequently, the measurement probability of the $m$th round can be written as
\begin{equation}\label{UnitProb}
\eqalign{
P_\varphi^{(m)}&={\rm Tr}\left[C(\tau_m)\rho_s(t)C^\dagger(\tau_m)\right] \\
&=F_{m-1}+{\sum_k}'\epsilon_k^2(\tau_m)\langle\psi_k(\tau_m)|\rho_s\left(\sum_{j=1}^{m-1}\tau_j\right)|\psi_k(\tau_m)\rangle.
}
\end{equation}
The second part on the right hand side of Eq.~(\ref{UnitProb}) is associated with $\tau_m$-dependent populations on the instantaneous eigenstates. And $\tau_m$ can be randomly chosen in our framework. A $\tau_m$-independent and close-to-unit measurement probability $P_\varphi^{(j\geq m)}\rightarrow1$ therefore requires the vanishing populations over $|\psi_k(\tau_m)\rangle$, suggesting that the system has been successfully prepared at the target state $\rho_s(t)=|\Psi_{\rm target}\rangle\langle\Psi_{\rm target}|$.

Alternatively, one can find that the square of the eigenvalues of the Kraus operator $C(\tau_m)$ are always lower than or equivalent to one. We have
\begin{equation}
\eqalign{
\epsilon_k^2(\tau_m)&=\left|C(\tau_m)|\psi_k(\tau_m)\rangle\right|^2
=\left|\langle\varphi|U(\tau_m)|\varphi\rangle|\psi_k(\tau_m)\rangle\right|^2\\
&\leq\left|U(\tau_m)|\varphi\rangle|\psi_k(\tau_m)\rangle\right|^2=1.
}
\end{equation}
It means that most populations on these eigenstates are reduced by measurements, except the target state and some special states satisfying $\epsilon_k^2(\tau_m)=1$ for the $m$th measurement performed at the moment $t+\tau_m$. However, since the measurement intervals for the free joint evolutions can be randomly chosen, the protection over the population of such states cannot continue for a sufficient number of evolution-measurement rounds. Only the target-state population will eventually survive.

Our state-purification framework therefore promises the uniqueness of the purified state, as long as the population on the target state is not vanishing at the initial time. The POVM induced by the purification operator acts as a population sieve to filter out the populations on unwanted states.

\subsection{Purification by general Hamiltonian}\label{UniquenessTotal}

Under certain conditions, our purification framework can be generalized to accommodate the effective or purification Hamiltonian beyond the compact form in Eq.~(\ref{Ham}). In the Schr\"odinger picture, we can consider the full Hamiltonian as
\begin{equation}\label{fullHam}
H=H_0+H_P=H_S+H_A+H_P,
\end{equation}
which consists of the system Hamiltonian, the ancillary-qubit Hamiltonian, and the purification Hamiltonian given by Eq.~(\ref{Ham}). For simplicity, $H$ is assumed to be time independent. $H_A=\omega_a\sigma_a^{\dagger}\sigma_a$ with $\omega_a$ the characteristic frequency of the ancillary qubit. With the same instantaneous projective measurement $M_{\varphi}$ over the initial state of the ancillary qubit, the non-unitary evolution operator for a single period of free evolution can be expressed by a series of expansion:
\begin{equation}\label{EffU}
\tilde{U}(\tau)\equiv M_\varphi e^{-iH\tau}M_\varphi=\sum_n\frac{(-i\tau)^n}{n!}M_\varphi H^nM_\varphi.
\end{equation}
Note $\tilde{U}(\tau)$ reduces to $C(\tau)\otimes|\varphi\rangle\langle\varphi|$ if $H$ becomes $H_P\otimes I_A$ in the interaction picture, where $I_A$ is the identity matrix of ancillary qubit. Recalling the definition of the purification Hamiltonian $H_P=g_a(\sigma_a^\dagger Q+\sigma_aQ^\dagger)$, we have $M_\varphi H_PM_\varphi=0$. Then only the items consisting of an even number of $H_P$ can survive in expansion. For the $n$th order in Eq.~(\ref{EffU}), it is found that $M_\varphi H^nM_\varphi=H_{\rm eff}^{(n)}M_\varphi$, where
\begin{equation}\label{iterative}
H_{\rm eff}^{(n)}=H_{\rm eff}^{(n-1)}\tilde{H}+H^{n-1}H_P=H_0\tilde{H}^{n-1}+\sum_{k=1}^{n-1}H^kH_P\tilde{H}^{n-k-1}
\end{equation}
with $\tilde{H}\equiv H_0-H_P$. For instances, we have
\begin{equation}\label{EffHam}
\eqalign{
M_\varphi &HM_\varphi=H_0M_\varphi, \\
M_\varphi &H^2M_\varphi=\left(H_0^2+H_P^2\right) M_\varphi, \\
M_\varphi &H^3M_\varphi=\left(H_0^3+H_0H_P^2+H_P^2H_0+H_PH_0H_P\right)M_\varphi.
}
\end{equation}
According to Eqs.~(\ref{fullHam}), (\ref{EffU}), and (\ref{iterative}), and the special fact that every item in $H_{\rm eff}^{(n)}$ has an even number of $H_P$, one can find that the effective Hamiltonian is block diagonal in the ancillary qubit space spanned by $\{|\varphi\rangle, |\varphi_{\perp}\rangle\}$:
\begin{equation}
H_{\rm eff}^{(n)}=\left(\begin{array}{cc}V^{(n)} & 0 \\ 0 & W^{(n)}\end{array}\right)=V^{(n)}\otimes|\varphi\rangle\langle\varphi|+W^{(n)}|\varphi_\perp\rangle\langle\varphi_\perp|.
\end{equation}
where $V^{(n)}=\langle\varphi|H_{\rm eff}^{(n)}|\varphi\rangle$ and $W^{(n)}=\langle\varphi_\perp|H_{\rm eff}^{(n)}|\varphi_\perp\rangle$ are conditional operators living in the target system, associated respectively with the desired measurement outcome and the undesired one.

More relevant to our framework, one can find that $Q^\dagger$ and $Q$ appear by ordered pairs in each order of $V^{(n)}$. In particular, $V^{(n)}=H_S^n+D_Q^{(n)}$, where $H_S^n$ is the system Hamiltonian to the $n$th power and $D_Q^{(n)}=D_Q^{(n)}(H_S, Q^\dagger, Q)$ is a function of $H_S$ and ordered pairs of $Q^\dagger$ and $Q$. For example, when $n=3$, we have
\begin{equation}
V^{(3)}=\left\langle\varphi\left|H_{\rm eff}^{(3)}\right| \varphi\right\rangle=H_S^3+g_a^2\left(H_S Q^\dagger Q + \omega_a Q^\dagger Q +Q^{\dagger} Q H_S+Q^{\dagger} H_S Q\right).
\end{equation}
According to the definition about the purification operator in Eq.~(\ref{PurificationCondition}), we have $D_Q^{(n)}|\Psi_{\rm target}\rangle=0$ and $D_Q^{(n)}|\Psi_k\rangle={\sum_l}'\beta_{lk}^{(n)}|\Psi_l\rangle$, where ${\sum_l}'$ represents the summation over all eigenstates of $H_S$ except the target state and $\beta_{lk}^{(n)}$ is the overlap coefficient between $|\Psi_k\rangle$ and $|\Psi_l\rangle$ under $D_Q^{(n)}$.

Then the POVM induced by measuring the initial state of the ancillary qubit $|\varphi\rangle$ gives rise to the system state
\begin{equation}
\eqalign{
\rho_s(t+\tau)&\sim\mathcal{M}[\rho_s(t)]=\sum_{n,m}\frac{(-i\tau)^n(i\tau)^m}{n!m!}V^{(n)}\rho_s(t)V^{(m)}\\
&=\sum_{n,m}\frac{(-i\tau)^n(i\tau)^m}{n!m!}\left[H_S^n+D_Q^{(n)}\right]\rho_s(t)\left[H_S^m+D_Q^{(m)}\right].
}
\end{equation}
The population over each eigenstate of the target system reads
\begin{equation}\label{population}
\eqalign{
\langle\Psi_k|\mathcal{M}[\rho_s(t)]|\Psi_k\rangle=&\rho_{kk}(t)+{\sum_l}'\bigg[e^{i\lambda_k\tau}\alpha_{lk}^*(\tau)\rho_{lk}(t)+e^{-i\lambda_k\tau}\alpha_{lk}(\tau)\rho_{kl}(t)\\
&+{\sum_j}'\alpha_{jk}^*(\tau)\alpha_{lk}(\tau)\rho_{jl}(t)\bigg],
}
\end{equation}
where $\rho_{ij}(t)\equiv\langle\Psi_i|\rho_s(t)|\Psi_j\rangle$ and $\alpha_{lk}(\tau)\equiv\sum_n\frac{(i\tau)^n}{n!}\beta_{lk}^{(n)}$. The measurement probability of the current round can be obtained by summing over Eq.~(\ref{population}) for every eigenstate $|\Psi_k\rangle$
\begin{equation}
P_\varphi=\sum_k\langle\Psi_k|\mathcal{M}[\rho_s(t)]|\Psi_k\rangle=1+\chi(\tau),
\end{equation}
where
\begin{equation*}
\chi(\tau)={\sum_{l,k}}'\bigg[e^{i\lambda_k\tau}\alpha_{lk}^*(\tau)\rho_{lk}(t)
+e^{-i\lambda_k\tau}\alpha_{lk}(\tau)\rho_{kl}(t)+{\sum_j}'\alpha_{jk}^*(\tau)\alpha_{lk}(\tau)\rho_{jl}(t)\bigg]
\end{equation*}
is a function of the density-matrix elements in the space orthogonal to the target state. Generally, we have $-1\leq \chi(\tau)\leq0$. As discussed in Sec.~\ref{Uniqueness}, $P_\varphi$ is close to unit and the target system approaches a steady state after a sufficient number of measurements. In this situation, $\chi(\tau)\rightarrow0$. Assuming the magnitude of the coherent elements is negligible in comparison to the population when $\rho_s(t)$ becomes invariant with time, then $\chi(\tau)\rightarrow0$ renders
\begin{equation*}
{\sum_k}'\left[e^{i\lambda_k\tau}\alpha_{kk}^*(\tau)\rho_{kk}(t)+e^{-i\lambda_k\tau}\alpha_{kk}(\tau)\rho_{kk}(t)
+{\sum_{l}}'|\alpha_{lk}(\tau)|^2\rho_{ll}(t)\right]=0.
\end{equation*}
Since it is independent of $\tau$, then the populations on the unwanted eigenstates have to be vanishing: ${\sum_k}'\rho_{kk}(t)=0$. It also indicates that the steady state under measurements is coincident with the target one.

\section{Preparation of entanglement}\label{Sec:PrepareEntangledState}

Quantum information processing often requires entangled or superposition states as a resource, e.g., the working qubits in the quantum teleportation protocol, there is a clear need to generate such states. A quantum system with channels to external environment, however, would be generally ended with a mixed state~\cite{PurificationInTeleportation,ErrorCorrection}. It is thus interesting and important to distill or generate entangled states from such a quantum system. In this section, we apply our state-purification scheme to prepare Bell states and GHZ state from the maximally mixed states $\rho_s(0)=I/d$~\cite{MixedState}, where $d$ is the dimension of the system and $I$ is the identity matrix. In other words, the initial system population has an even distribution over all the eigenstates.

\subsection{Bell state preparation}\label{Sec:BellState}

% Figure environment removed

We first choose the singlet Bell state $|\Psi_{\rm target}\rangle=|\Psi_-\rangle=(|01\rangle-|10\rangle)/\sqrt{2}$ as our target state. For a double-qubit system with a strong $XX$ coupling~\cite{XXCoupling}, the system Hamiltonian can be written as ($\hbar=1$)
\begin{equation}\label{SystemHam}
H_S=\omega_1\sigma_1^+\sigma_1^-+\omega_2\sigma_2^+\sigma_2^-+g_s\sigma_1^x\sigma_2^x,
\end{equation}
where $\omega_i$ and $\sigma_i^{\pm}$ are respectively the bare frequency and the transition operators for the $i$th qubit and $g_s$ represents the coupling strength between qubit-$1$ and qubit-$2$. Under the resonant condition, $\omega_1=\omega_2=\omega_0$, the target state is one of the system eigenstate $H_S|\Psi_{\rm target}\rangle=(\omega_0-g_s)|\Psi_{\rm target}\rangle$. To purify the system into the target state, the purification Hamiltonian can be chosen as
\begin{equation}
H_P=g_a\left(\sigma_a^\dagger Q+\sigma_aQ^\dagger\right), \quad Q=\sigma_1^++\sigma_2^+,
\end{equation}
which can be straightforwardly verified to satisfy the condition $Q|\Psi_{\rm target}\rangle=0$. The initial state of the ancillary qubit is set as the excited state $|\varphi\rangle=|e\rangle$. Thus $\sigma_a=\sigma_a^+=|e\rangle\langle g|$, $\sigma_a^\dagger=\sigma_a^-=|g\rangle\langle e|$, and the ancillary-qubit Hamiltonian is $H_A=\omega_a\sigma_a^-\sigma_a^+$. The interaction between two system qubits is assumed to be much stronger than that between system qubits and the ancillary qubit, $g_s\gg g_a$, so it is reasonable to neglect the rapidly-shifting terms $\sigma_a^+\sigma_i^+$ and $\sigma_a^-\sigma_i^-$ in the interaction Hamiltonian under the rotating-wave approximation. Then the full Hamiltonian $H=H_S+H_A+H_P$ reads
\begin{equation}
H=\sum_{i=1,2}\omega_i\sigma_i^+\sigma_i^- + \omega_a\sigma_a^-\sigma_a^++g_s\sigma_1^x\sigma_2^x+g_a\left[\sigma_a^-(\sigma_1^++\sigma_2^+)+ {\rm H.c.}\right].
\end{equation}

The rest three eigenstates of the system Hamiltonian in Eq.~(\ref{SystemHam}) are
\begin{equation}\label{BellEigenstates}
\eqalign{
|\Psi_+\rangle=&\frac{1}{\sqrt{2}}(|01\rangle+|10\rangle), \\
|\Phi_-\rangle=&\frac{1}{\xi_-}\left[\left(\sqrt{g_s^2+\omega_0^2}-\omega_0\right)|00\rangle-g_s|11\rangle\right], \\
|\Phi_+\rangle=&\frac{1}{\xi_+}\left[\left(\sqrt{g_s^2+\omega_0^2}+\omega_0\right)|00\rangle+g_s|11\rangle\right],
}
\end{equation}
where $\xi_\pm\equiv\sqrt{g_s^2+(\omega_0\pm\sqrt{g_s^2+\omega_0^2})^2}$ are the normalization coefficients. And their eigenvalues are $\omega_0+g_s$, $\omega_0-\sqrt{\omega_0^2+g_s^2}$, and $\omega_0+\sqrt{\omega_0^2+g_s^2}$, respectively. In the eigenbasis of the system, the purification operator can be rewritten as
\begin{equation}\label{Q_BellState}
\eqalign{
Q=&\frac{\xi_+}{\sqrt{2(g_s^2+\omega_0^2)}}|\Phi_+\rangle\langle\Psi_+|
+\frac{\xi_-}{\sqrt{2(g_s^2+\omega_0^2)}}|\Phi_-\rangle\langle\Psi_+|\\
&+\frac{\sqrt{2}g_s}{\xi_+}|\Psi_+\rangle\langle\Phi_+|-\frac{\sqrt{2}g_s}{\xi_-}|\Psi_+\rangle\langle\Phi_-|,
}
\end{equation}
which involves with the transitions among $|\Phi_+\rangle$, $|\Phi_-\rangle$, and $|\Psi_+\rangle$ as shown in Fig.~\ref{Fig:BellStates}(a). The coefficient for each transition in the operator $Q$ indicates the variation rate of the eigenstates' population except the target one.

The system states after a certain number of rounds of evolution and projective measurement $M_{\varphi}$ with random measurement intervals are demonstrated in Fig.~\ref{Fig:BellStates}(b)-(d). The initial state [see Fig.~\ref{Fig:BellStates}(b)] is a maximally mixed state with population equally distributed on all the four eigenstates. After $M=20$ measurements [see Fig.~\ref{Fig:BellStates}(c)], the population over the target state $|\Psi_-\rangle$ has been raised from $0.25$ to $0.55$ (over one half); the population over $|\Psi_+\rangle$ becomes almost vanishing; and the populations over $|\Phi_-\rangle$ and $|\Phi_+\rangle$ are about $0.07$ and $0.38$, respectively. These results can be understood by the transition rates presented in $Q$ operator. For example, the transition rate for $|\Phi_+\rangle\rightarrow|\Psi_+\rangle$ in Eq.~(\ref{Q_BellState}) is $\sqrt{2}g_s/\xi_+$, whose magnitude is the smallest one among all the rates, leading to an inefficient population transfer from $|\Phi_+\rangle$ to the other states. After $M=200$ measurements [see Fig.~\ref{Fig:BellStates}(d)], all the populations have been cumulated onto the target state, which means the other states are filtered out by the measurement-induced POVM. In particular, it is found that the state fidelity is $F\approx0.98$ and the success probability is $P_s\approx26\%$.

% Figure environment removed

Our framework of measurement-based purification is much simpler than the state steering based on the nonselective measurements~\cite{MeasurementSteering}, which employs one ancillary qubit for every transition from unwanted states to the target state and also involves with the three-body interactions. Alternatively, the number of ancillary qubits can be reduced at the cost of extra unitary rotations with respect to the system Hamiltonian, which requires frequently turning on and off the interaction between the system and the ancillary qubit. In contrast, our framework contains only a single ancillary qubit and two-body interactions, because of no direct transitions to the target state.

The target state is not unique in our framework, which is determined by the formulation of the purification operator. For the basic condition in Eq.~(\ref{PurificationCondition}), one can follow three main ideas to design the purification operator for a desired target state: (i) $Q=\sum_k'a_k|\Psi_{\rm target}\rangle\langle\Psi_k|$, collecting the transitions from all the other eigenstates to the target state; (ii) $Q=\sum_k'a_k|\Psi_k\rangle\langle\Psi_{k+1}|$, building transitions between every pair of neighboring states ordered in a certain way, e.g., the annihilation operator of a resonator; (iii) $Q=\sum_k'a_k|\Psi_k\rangle\langle\Psi_k|$, mapping all the other eigenstates to themselves, i.e., a collection of projective operators. Here $a_k$'s are arbitrary and nonvanishing coefficients.

For the two-qubit system in Eq.~(\ref{SystemHam}), all the eigenstates in Eq.~(\ref{BellEigenstates}) could be prepared via the measurement-based purification. The purification performance as well as the purification operators could be efficiently simulated in digital quantum circuits~\cite{51QubitsMeasurementVQE}. We here follow the third idea of the preceding discussion to demonstrate the purification process, which avoids setting up transitions among unwanted states and those towards the target state. And for simplicity, we suppose all the self-projectors are the same in weight. Then the purification operators for the eigenstates in Eq.~(\ref{BellEigenstates}) could be constructed as
\begin{equation}\label{fourQ}
Q_{\Psi} = I-|\Psi\rangle\langle\Psi|,\quad\Psi\in\{\Psi_-,\Psi_+,\Phi_-,\Phi_+\}.
\end{equation}

Figure~\ref{Fig:AllEigenStates} shows the population dynamics about four eigenstates of Hamiltonian~(\ref{SystemHam}) with various target states whose purification operators are given by Eq.~(\ref{fourQ}). It is found that the system can be prepared as the desired target eigenstates within $M=50$ rounds of free evolution and measurement. They are ended with a state-fidelity over $F=0.98$ and a success probability over $P_s=25\%$. The populations over the unwanted states decrease gradually to zero at almost the same rate.

\subsection{GHZ state preparation}

% Figure environment removed

Our measurement-based purification framework adapts to generate entangled state for multiple qubit system. In the absence of a transversal magnetic field, we consider a one-dimensional chain of three spins-$1/2$ linked by the nearest-neighbor Ising bonds, i.e., $H_S=J(\sigma_1^z\sigma_2^z+\sigma_2^z\sigma_3^z)$, where $J$ is the coupling strength~\cite{IsingModel}. As a typical degenerate system, the eigenbasis is not unique. We set the target state as one of the maximally entangled state for this discrete system:
\begin{equation}\label{GHZ}
|{\rm GHZ}\rangle=|\Psi_{1,+}\rangle=\frac{1}{\sqrt{2}}(|000\rangle+|111\rangle).
\end{equation}
And the other eigenstates of the system can also be chosen as the general GHZ states:
\begin{equation}
\eqalign{
|\Psi_{1,-}\rangle=&\frac{1}{\sqrt{2}}(|000\rangle-|111\rangle),\quad |\Psi_{2,\pm}\rangle=\frac{1}{\sqrt{2}}(|010\rangle\pm|101\rangle)\\
|\Psi_{3,\pm}\rangle=&\frac{1}{\sqrt{2}}(|001\rangle\pm|100\rangle),\quad |\Psi_{4,\pm}\rangle=\frac{1}{\sqrt{2}}(|011\rangle\pm|110\rangle).
}
\end{equation}

To filter out the populations on these unwanted eigenstates, the purification operator can be constructed by a collection of self-projectors as
\begin{equation}\label{GHZQ}
\eqalign{
Q&=I-|\Psi_{1,+}\rangle\langle\Psi_{1,+}|\\
&=|\Psi_{1,-}\rangle\langle\Psi_{1,-}| + |\Psi_{2,+}\rangle\langle\Psi_{2,+}| + |\Psi_{2,-}\rangle\langle\Psi_{2,-}| + |\Psi_{3,+}\rangle\langle\Psi_{3,+}|\\
&+|\Psi_{3,-}\rangle\langle\Psi_{3,-}| + |\Psi_{4,+}\rangle\langle\Psi_{4,+}| + |\Psi_{4,-}\rangle\langle\Psi_{4,-}|.
}
\end{equation}
For this system, the effective Hamiltonian can be the purification Hamiltonian in the formation of Eq.~(\ref{Ham}), where $\sigma_a^\dagger=\sigma_a^-=|e\rangle\langle g|$. The system states before measurement and after $M=10$ rounds of measurements are shown in Fig~\ref{Fig:GHZState}. It is found that even from the maximally mixed state, the system can be purified to the valuable GHZ state by several random measurements. After $M=10$ measurements, the state fidelity is close to unit and the success probability is $P_s=12.5\%$, which is equivalent to the initial population on the target state.

In the preceding examples of entangled state preparation, either Bell state or GHZ state, the interested system is supposed to be initialized as a maximally mixed state with a maximal von Neumann entropy $S[\rho_s(0)]=\log(d)$~\cite{VonNeumann}. This choice is surely a ``hard mode'' for state purification. In a less extreme condition, e.g., when the system starts from a finite-temperature state or even a pure state (with a nonvanishing overlap with the target state), the required number of rounds of evolution and measurement would be greatly reduced. For example, if the three-qubit system is initialized as $|000\rangle$, which is a superposed state involving merely $|\Psi_{1,-}\rangle$ and $|\Psi_{1,+}\rangle=|\Psi_{\rm target}\rangle$, then the purification operator can be as simple as $Q=|\Psi_{1,-}\rangle\langle\Psi_{1,-}|$. Through only two rounds of measurements, the system could be prepared as the GHZ state with a near-to-unit fidelity.

\section{Acceleration of adiabatic passage}\label{Sec:STIRAP}

% Figure environment removed

The conventional stimulated Raman adiabatic passage in a three-level system is used to faithfully transfer the population on an eigenstate to another one with dark states~\cite{STAThreelevel}. It could be realized by properly driving the transitions in the system as shown in Fig.~\ref{Fig:StirapModel}(a). Under the assumption that the external driving fields are resonant with the corresponding frequency splittings between the driven levels, the system Hamiltonian in the interaction picture can be written as~\cite{AdiabaticBattery}
\begin{equation}\label{StirapHam}
H_1 = \Omega_{12}(t)\sigma_{12}^x+\Omega_{23}(t)\sigma_{23}^x,
\end{equation}
where $\sigma_{12}^x\equiv|\varepsilon_1\rangle\langle\varepsilon_2|+|\varepsilon_2\rangle\langle\varepsilon_1|$ and $\sigma_{23}^x\equiv|\varepsilon_2\rangle\langle\varepsilon_3|+|\varepsilon_3\rangle\langle\varepsilon_2|$ are the transition operators in the three-level system. One of eigenstates of the system
\begin{equation}
|E_0(t)\rangle=\frac{1}{\sqrt{2}}
\left[\frac{\Omega_{23}(t)}{\Delta(t)}|\varepsilon_1\rangle-\frac{\Omega_{12}(t)}{\Delta(t)}|\varepsilon_3\rangle\right]
\end{equation}
constitutes the time-dependent adiabatic path for state engineering. In particular, when the system is initialized at the ground state $|\varepsilon_1\rangle$, a perfect population transfer to $|\varepsilon_3\rangle$ could be realized by a slowly-decreasing field $\Omega_{23}(t)$ and a slowly-increasing field $\Omega_{12}(t)$. If the Rabi frequencies of these fields are rapidly varying with time, then the system evolution can deviate significantly from the adiabatic path, resulting in an incomplete population transfer to $|\varepsilon_3\rangle$. In this situation, we can apply our state-purification scheme to complete and accelerate STIRAP.

In our theoretical framework, STIRAP can be combined with a purification Hamiltonian consisting of only one resonant driving field $\Omega_{12}(t)$ for the chosen target state $|\Psi_{\rm target}\rangle=|\varepsilon_3\rangle$, i.e., the system Hamiltonian now reads
\begin{equation}\label{HSstirap}
H_S= \Omega_{12}(t)\sigma_{12}^x.
\end{equation}
In accordance with state purification, the target state is an eigenstate of the system Hamiltonian $H_S|\varepsilon_3\rangle=0$. As shown in Fig.~\ref{Fig:StirapModel}(b), the purification operator associated with the ancillary qubit can be set as $Q=\sigma_{12}^-$ and the initial state of the qubit is the ground state $|\varphi\rangle=|g\rangle$. Then the purification Hamiltonian is
\begin{equation}
H_P=g_a\left(\sigma_a^+\sigma_{12}^-+\sigma_a^-\sigma_{12}^+\right)
\end{equation}
and the full Hamiltonian of our scheme in the rotating frame with respect to $H_0=\sum_{i=1}^3\varepsilon_i|\varepsilon_i\rangle\langle\varepsilon_i|+\omega_a\sigma_a^+\sigma_a^-$ reads
\begin{equation}
H=H_S+H_P=\Omega_{12}(t)\sigma_{12}^x+g_a\left(\sigma_a^+\sigma_{12}^-+\sigma_a^-\sigma_{12}^+\right),
\end{equation}
where the ancillary qubit is assumed to be resonant with the splitting between $|\varepsilon_1\rangle$ and $|\varepsilon_2\rangle$, i.e., $\omega_a=\varepsilon_2-\varepsilon_1$. For the system Hamiltonian in Eq.~(\ref{HSstirap}), the three eigenstates are
\begin{equation}
\eqalign{
|E_0\rangle&=|\varepsilon_3\rangle, \\
|E_-\rangle&=\frac{1}{\sqrt{2}}(|\varepsilon_1\rangle-|\varepsilon_2\rangle), \\
|E_+\rangle&=\frac{1}{\sqrt{2}}(|\varepsilon_1\rangle+|\varepsilon_2\rangle),
}
\end{equation}
by which the purification operator $Q$ could be rewritten as
\begin{equation}\label{Q_STIRAP}
Q=|E_-\rangle\langle E_+|-|E_+\rangle\langle E_-|+|E_+\rangle\langle E_+|-|E_-\rangle\langle E_-|.
\end{equation}
And it is straightforward to confirm the purification condition $Q|\varepsilon_3\rangle=0$. Transitions presented in the operator $Q$ are demonstrated in Fig.~\ref{Fig:StirapModel}(c), where there is no transition towards the target state $|\varepsilon_3\rangle$.

% Figure environment removed

To explore the purification-induced acceleration of the population transfer when the system deviates from adiabatic evolution, we propose two hybrid models combining STIRAP and state-purification. During the stage of STIRAP, the system is driven by the Hamiltonian~(\ref{StirapHam}) with $\Omega_{12}(t)=\Omega_0f(t)$ and $\Omega_{23}(t)=\Omega_0[1-f(t)]$, where $\Omega_0$ represents the maximal magnitude of driving strength and $f(t)$ is a dimensionless function that satisfies $f(0)=0$ and $f(t_c)=1$ with a desired control time $t_c$. Here we take the hyperbolic sine function $f(t)=\sinh(ct/\tau)$, where $c$ is a scaling factor for the boundary conditions. During the state purification, the driving field between $|\varepsilon_2\rangle$ and $|\varepsilon_3\rangle$ is temporally switched off, i.e., $\Omega_{23}=0$, and the Rabi frequency of the driving field between $|\varepsilon_1\rangle$ and $|\varepsilon_2\rangle$ is set as a constant $\Omega_{12}=\Omega_0$. For the first hybrid model, the adiabatic evolutions and the purification processes present alternatively on stage. In particular, the original control time $t_c$ for both $\Omega_{12}(t)$ and $\Omega_{23}(t)$ can be divided into $M$ parts. A round of state purification consisting of a free evolution lasting a random $\tau_i$ and an instantaneous measurement on the ancillary qubit is performed at the end of each part. Taking account the time for $M$ rounds of state purification, the full running time for the first hybrid model is $t_c^{(1)}=t_c+\tau_p$, where $\tau_p=\sum_{i=1}^M\tau_i$. For the second hybrid model, the state purification is performed after the accelerated (diabatic and unfaithful) STIRAP is completed. In another word, the final state of STIRAP is the initial state for starting the purification process.

In Fig.~\ref{Fig:STIRAP}, we demonstrate the dynamics of the population on $|\varepsilon_3\rangle$ using the two hybrid models. To show the power of our state-purification scheme, we also present two results under the pure strategy of STIRAP. One is unfaithful with a shorter running time (see the red-dotted line) and another one is faithful with a much longer time (see the orange dot-dashed line). The strategy of Hybrid~1 indicates that the STIRAP is divided to a certain number of parts and concatenated with discrete rounds of state purification. And ``Hybrid~2'' describes that an intact STIRAP is followed with the purification by measurements. It is found that the strategy of Hybrid~1 (see the blue solid line) prevails the other strategies in the running time for population transfer. In particular, ``Hybrid~1'', ``Hybrid~2'', and ``$t_c^{(2)}$'' cost $6.9/\Omega_0$, $10.0/\Omega_0$, and $13.6/\Omega_0$ in time to achieve $F=0.99$, respectively. With no assistance by the state purification, one can find that the same running time for the ``$t_c^{(1)}$'' strategy is too short to achieve a faithful population transfer. Both hybrid strategies overwhelm the faithful STIRAP of the ``$t_c^{(2)}$'' strategy and the success probabilities for ``Hybrid~1'' and ``Hybrid~2'' are respectively about $P_s=69\%$ and $P_s=84\%$.

\section{Discussion and Conclusion}\label{Sec:Conclusion}

Motivated by the inverse engineering or steering with directly performing a dense sequence of measurements on the interested system along a predesigned path~\cite{AharonovZeno,NonselectiveAharonovZeno}, our framework of POVM on the interested system by indirected measurements on the ancillary system provides a much broader regime for purification by measurements. It also suggests that using local operations can control a much larger coupled system. Previous works on projection-based purification~\cite{MeasurementPurification,Nakazato2004,Nakazato2005} devoted to optimize the measurement intervals to enhance the population of the target state given a certain interaction. The strategy places a severe constraint over the target states and suffers from the purification inefficiency under a significant systematic error about the measurement intervals. Our scheme in this work, however, is mainly based on the purification operators and is capable of preparing an arbitrary eigenstate of the system with random time intervals. So that it is naturally robust against the errors with respect to the measurement moment and does not need hybrid quantum-classical feedback control with observing the system state~\cite{QuantumClassiclHybrid,HardwareEfficient}. In contrast to the steering protocol based on the nonselective measurements, our protocol does not involve the complex three-body interaction and only requires a single ancillary qubit.

In summary, we present an eigenstate purification framework by repeatedly measuring the ancillary qubit coupled to the system. The purification operator is built up without direct transitions towards the target state. Thus the measurements on the initial state of the ancillary qubit induce positive operator-valued measures that can purify the system into an arbitrary chosen eigenstate by filtering out the populations over all the other states. In qubit systems, we apply our purification scheme to generate Bell states and GHZ state. Combined with the conventional STIRAP protocol in a three-level system, we realize a much accelerated adiabatic population transfer with a high success probability. Our scheme can serve as a promising candidate for error correction when the system state deviates from the desired one due to the environment-induced decoherence, which is of great interest in the era of noisy intermediate-scale quantum. Much broadly, our scheme contributes to the universal state preparation for a multi-particle system.

\section*{Acknowledgments}

We acknowledge financial support from the National Natural Science Foundation of China (Grant No. 11974311).


\section*{References}

\bibliographystyle{iopart-num}
\bibliography{ref}

\end{document}
