In this section, we provide a detailed case study in precision tracking via a nonlinear high-fidelity simulator of a physical system.
\subsection{The system}

As a case study we use a 2-axis high precision motion system depicted in Fig. \ref{fig:andromeda}. This system contains an internal closed-loop controller, it takes reference trajectories as inputs and produces tool-tip trajectories as outputs.
In this work, we use three different models to instantiate Algorithm~\ref{alg:cap}. All model the system response in discrete time, with a sample rate of $400\,\mathrm{Hz}$. For each case, we report $\sigma$, the standard deviation of the prediction error of the model, for input trajectories with acceleration up to $3\,\mathrm{m\,s^{-2}}$, when compared to experimental data. The three models are:
\begin{enumerate}
	\item[LM] A discrete-time linear model, in a state space lifted representation. $\sigma = 236.4\,\mathrm{\mu m}$.
	\item[NL1] A nonlinear ANN model, with an input layer capturing $200\,\mathrm{ms}$ of input history, and LeakyReLu activation functions. 
    $\sigma = 16.50\,\mathrm{\mu m}$.
	\item[NL2] A nonlinear ANN model, with an input layer capturing $500\,\mathrm{ms}$ of input history, and LeakyReLu activation functions. This model is, for the purposes of simulations, considered to be the ground truth. 
    $\sigma = 11.27\,\mathrm{\mu m}$.
\end{enumerate}
For a more detailed description of the system, model design and accuracy of the models see \cite{balula2022data}. 

% Figure environment removed

In the following numerical results, we use either LM or NL1 to derive the gradient \eqref{eq:gradient} and Hessian \eqref{eq:hessian} information.
Since the structure of both models is kwown, one can take symbolic derivatives of the output in respect to the input to obtain the gradient and Hessian. %
The quality of the derivative information depends on the quality of the model derivatives, but it is not directly affected by the non-repeating disturbance, that is only used to determine the point of linearization. %
Due to the structure of both models, the Hessian evaluates to zero. In all simulations, the model NL2 is used in lieu of the true system, i.e., for the evaluations of $f(u)$ but not its gradients. 


\subsection{Cost and constraints}
The optimization problem \eqref{eq:original-problem} used in the case study is
\begin{mini!}{u, p}{J(u, p) = \sum_{i = 0}^N \|p(i) - \Xi(i)\|^2_{Q_a}  + \|\partial^2 u(i) \|^2_{R_a}}{}{} %
\addConstraint{p-f(u)}{= 0}{}
\addConstraint{p(i)}{ \in \mathcal{W}}{,\quad i = 1, \dots, N }
\addConstraint{|\partial u(i)|}{ \leq \mathtt{v_{max} }}{,\quad i = 1, \dots, N - 2 }
\addConstraint{|\partial^2 u(i)|}{ \leq \mathtt{a_{max} }}{,\quad i = 1, \dots, N - 3 }
\addConstraint{|\partial^3 u(i)|}{ \leq \mathtt{j_{max} }}{,\quad i = 1, \dots, N - 4 }
,\end{mini!}
where $N$ is the number of points in the target trajectory $\Xi = \{\Xi_i\}_{i=1}^N \subseteq \Reals^2$,
$p = \{p_i\}_{i=1}^N \subseteq \Reals^2$ and $\mathcal{W} \subseteq \Reals^2$ is the workspace.
The first term of the cost function penalizes deviations of the output with respect to the target geometry, while the second term regularizes the input by penalizing the acceleration.
We use $Q_a = 10^{6} I$, $R_a=10^{-2} I$ to reflect the different order of magnitude of the input acceleration and output deviation. For the constraints we use 
$\mathtt{v_{max}} = 2$, $\mathtt{a_{max}} = 2$, $\mathtt{j_{max}} = 500$, derived from the physical limits of the machine. The number of points $N$ in the target trajectory (see Fig.~\ref{fig:xy-detail-linearmodel}) is $314$, which corresponds to $0.785\,\mathrm{s}$ time discretization between sample points given the sample rate of $400\,\mathrm{Hz}$. As the target geometry we use the outline of the letter 'r' from the ETH Zurich logo, shown in the inset of Fig. \ref{fig:xy-detail-linearmodel}.
The values for the velocity, acceleration, and jerk limits are derived from the physical limits of the machine.


