\documentclass[11pt]{amsart}
\usepackage{amssymb}
\usepackage{tikz}
\usepackage{youngtab}
\usepackage[T1]{fontenc}\usepackage{palatino} % better font
\usepackage[mathscr]{euscript}
\usepackage{stmaryrd} % gives double bracket symbols
\usepackage[pagebackref]{hyperref} % this makes life easier for readers
\hypersetup{colorlinks,linkcolor=blue,urlcolor=blue,citecolor=blue,%
  linktocpage}  % options for hyperref (pagebackref MUST be loaded earlier)
\usepackage[alphabetic,abbrev,nobysame]{amsrefs} % load AFTER hyperref! 

%%%%%%%%%%%%%%%%%%%%%%%%%%%%%%%%%%%%%%%%%%%%%%%%%%%%%%%%%%%%
%% BEGIN: Partition-drawing code swiped from Zajj Daugherty
%% For partitions. Takes in a list of row lengths. 

\newcommand{\Part}[1]{
 \foreach \x [count=\s from 1] in {#1}{
 	{\ifnum\s=1
		\draw (0,\s-1)--(\x,\s-1); 
		\fi}
   \draw (0,\s) to (\x,\s);
   \foreach \y in {0, ..., \x} {\draw (\y,\s)--(\y,\s-1);}
 }}

\def\UNIT{.18} \newcommand{\PART}[1]{
\begin{tikzpicture}[xscale=\UNIT, yscale=-\UNIT] 
	\Part{#1}
\end{tikzpicture}
}
%% END: Partition-drawing code
%%%%%%%%%%%%%%%%%%%%%%%%%%%%%%%%%%%%%%%%%%%%%%%%%%%%%%%%%%%%
\newcommand{\epic}{
\begin{tikzpicture}[scale = 0.35,thick, baseline={(0,-1ex/2)}]
\tikzstyle{vertex} = [shape = circle, minimum size = 4pt, inner sep = 1pt] 
\node[vertex] (G--2) at (1.5, -1) [shape = circle, draw,fill=black] {}; 
\node[vertex] (G--1) at (0.0, -1) [shape = circle, draw,fill=black] {}; 
\node[vertex] (G-1) at (0.0, 1) [shape = circle, draw,fill=black] {}; 
\node[vertex] (G-2) at (1.5, 1) [shape = circle, draw,fill=black] {}; 
\draw (G--2) .. controls +(-0.5, 0.5) and +(0.5, 0.5) .. (G--1); 
\draw (G-1) .. controls +(0.5, -0.5) and +(-0.5, -0.5) .. (G-2); 
\end{tikzpicture} 
}

\newcommand{\onepic}{
\begin{tikzpicture}[scale = 0.35,thick, baseline={(0,-1ex/2)}] 
\tikzstyle{vertex} = [shape = circle, minimum size = 4pt, inner sep = 1pt] 
\node[vertex] (G--1) at (0.0, -1) [shape = circle, draw,fill=black] {}; 
\node[vertex] (G-1) at (0.0, 1) [shape = circle, draw,fill=black] {}; 
\draw (G-1) .. controls +(0, -1) and +(0, 1) .. (G--1); 
\end{tikzpicture}
}

%%%%%%%%%%%%%%%%%%%%
% theorems and such
%%%%%%%%%%%%%%%%%%%%
\swapnumbers % theorem numbers on the LEFT! 
\newtheorem{thm}{Theorem}[section]
\newtheorem*{thm*}{Theorem}
\newtheorem{lem}[thm]{Lemma}
\newtheorem*{lem*}{Lemma}
\newtheorem{prop}[thm]{Proposition}
\newtheorem*{prop*}{Proposition}
\newtheorem{cor}[thm]{Corollary}
\newtheorem*{cor*}{Corollary}
\newtheorem{conj}[thm]{Conjecture}
\newtheorem*{conj*}{Conjecture}

\theoremstyle{definition}
\newtheorem{defn}[thm]{Definition}
\newtheorem*{defn*}{Definition}
\newtheorem{example}[thm]{Example}
\newtheorem*{example*}{Example}
\newtheorem{rmk}[thm]{Remark}
\newtheorem*{rmk*}{Remark}
\newtheorem{que}{Question}
\newtheorem*{que*}{Question}

%%%%%%%%%
% macros
%%%%%%%%%
\newcommand{\C}{\mathbb{C}} % complex numbers
\newcommand{\Z}{\mathbb{Z}} % integers
\newcommand{\UU}{\mathbf{U}} % bold U
\newcommand{\TL}{\operatorname{TL}} % Temperley-Lieb alg
\newcommand{\End}{\operatorname{End}} % endomorphisms
\newcommand{\Hom}{\operatorname{Hom}} % homomorphisms
\newcommand{\gl}{\mathfrak{gl}} % general linear Lie alg
\newcommand{\fsl}{\mathfrak{sl}} % special linear Lie alg
\newcommand{\ov}{\overline} % overline short form
\newcommand{\Schur}{\mathbf{S}} % Schur algebra
\newcommand{\bil}[2]{\langle #1, #2 \rangle}
\newcommand{\qbinom}[2]{\genfrac{[}{]}{0pt}{}{#1}{#2}}
\newcommand{\M}{\mathscr{M}} % maximal vectors
\newcommand{\D}{\mathscr{D}} % diagram algebra
\newcommand{\Cat}{\operatorname{LW}} % no of increasing lattice walks
\newcommand{\Tab}{\operatorname{Tab}} % tableaux
\newcommand{\HH}{\mathcal{H}} % Hecke algebra
\newcommand{\Sym}{\mathfrak{S}} % symmetric group
\newcommand{\onto}{\twoheadrightarrow}
\newcommand{\into}{\hookrightarrow}
\newcommand{\dbracket}[1]{\llbracket #1 \rrbracket} % double brackets
\newcommand{\stirlingii}{\genfrac{\{}{\}}{0pt}{}}

\renewcommand{\labelenumi}{(\alph{enumi})}
\parskip=2pt

\allowdisplaybreaks

\title[Origins of the Temperley--Lieb algebra]%
      {Origins of the Temperley--Lieb algebra: early history}
\author{Stephen Doty}
\email{doty@math.luc.edu, tonyg@math.luc.edu}
\author{Anthony Giaquinto}
%\email{tonyg@math.luc.edu}
\address{\parbox{\linewidth}{Department of Mathematics and Statistics,
  Loyola University Chicago,\\ Chicago, IL 60660 USA}}
\subjclass{Primary 16T30, 16G99, 17B37}
\keywords{Diagram algebras, Schur--Weyl duality,
  Temperley--Lieb algebras, quantized enveloping algebras}
\begin{document}
\begin{abstract}\noindent
We give an historical survey of some of the original basic algebraic
and combinatorial results on Temperley--Lieb algebras, with a focus on
certain results that have become folklore.
\end{abstract}
\maketitle
\tableofcontents

\section*{Introduction}\noindent
The Temperley--Lieb algebra $\TL_n(\delta)$ was introduced in
\cite{TL} in connection with certain problems in mathematical physics.
It reappeared in the 1980s as a certain von Neumann algebra in the
spectacular work of Vaughan Jones
\cites{Jones:83,Jones:85,Jones:86,Jones,Jones:91} on subfactors and
knots.  Kauffman \cites{K:87,K:88,K:90} (see also \cite{Kauffman})
realized it as a diagram algebra and as a quotient algebra of the
group algebra of Artin's braid group. Birman and Wenzl
\cite{Birman-Wenzl} showed that it is isomorphic to a subalgebra of
the Brauer algebra \cite{Brauer}; this also follows from Kauffman's
results.

This paper is an historical survey of the most fundamental algebraic
and combinatorial results on these algebras and their
representations. When writing \cites{DG:PTL,DG:orthog}, we found it
challenging to track down original references for various basic
results that have become folklore. The purpose of this paper is to
document what we found, hopefully aiding subsequent researchers
working in this area. Our focus is somewhat different from that of the
excellent survey article \cite{Ridout-StAubin}. Furthermore, we wish
to draw attention to the book \cite{GHJ}, a reference which seems to
be almost universally ignored by authors writing papers in this area.
Since our focus is on early history, we do not attempt to survey more
recent important developments such as categorification
\cites{BFK,Stroppel,FKS} or the many generalizations of
Temperley--Lieb algebras that exist in abundance in the literature.


\section{The Temperley--Lieb algebra}\label{s:1}\noindent
In this section $\Bbbk$ is a commutative ring with $1$. The
Temperley--Lieb algebra appeared originally in \cite{TL} in
connection with the Potts model in mathematical physics. For any
positive integer $n$ and any element $\delta$ in the ground ring
$\Bbbk$, $\TL_n(\delta)$ is the unital algebra defined by generators
$e_1, \dots, e_{n-1}$ subject to the relations
\begin{equation}\label{e:TL}
  \begin{aligned}
  e_i^2 = \delta e_i, \quad
  e_i e_j e_i = e_i \text{ if } |i-j|=1, \quad
  e_i e_j = e_j e_i \text{ if } |i-j|>1 .
\end{aligned}
\end{equation}
The unit element $1$ of the algebra is identified with the empty
product of generators.  There is an algebra isomorphism $\TL_n(\delta)
\cong \TL_n(-\delta)$ defined on generators by $e_i \mapsto -e_i$. We
also observe that $\TL_{n-1}(\delta)$ is isomorphic to the subalgebra
of $\TL_n(\delta)$ generated by $e_1, \dots, e_{n-2}$.

\begin{rmk}
The above description by generators and relations is not given in
\cite{TL}, but can be found, for instance, in the book \cite{Baxter}.
\end{rmk}

Our first task is to show that $\TL_n = \TL_n(\delta)$ has finite rank
over $\Bbbk$ (hence is finite-dimensional if $\Bbbk$ is a field). We
follow an argument sketched in Jones \cite{Jones:83}.  If $w_1, w_2$
are words in the generators $e_1, \dots, e_{n-1}$ then they are
\emph{equivalent} (written $w_1 \sim w_2$) if they are equal up to a
power of $\delta$.  Say that a word $w = e_{i_1} \cdots e_{i_l}$ in
$\TL_n$ is \emph{reduced} if it has minimal possible length in its
equivalence class.

\begin{lem}[Jones' Lemma]
  If $w = e_{i_1} \cdots e_{i_l}$ is a reduced word in $\TL_n$ then $m
  := \max\{i_1, \dots, i_l\}$ occurs only once in the sequence $(i_1,
  \dots, i_l)$.
\end{lem}


\begin{proof}
This is proved by induction on the length. The base case is trivial.
Let $w$ be a reduced word. Suppose for contradiction that $e_m$
appears at least twice in $w$, where $m$ is the maximal index that
appears. Then $w = w_1e_mw_2e_mw_3$. We may assume that $w_2$ does not
contain $e_m$.

If $w_2$ does not contain $e_{m-1}$ then $e_m$ commutes with all the
$e_i$ appearing in $w_2$, so after commuting the rightmost $e_m$ to
the left of $w_2$, the length of $w$ can be shortened using the
equivalence $e_m^2 \sim e_m$. Contradiction.

The remaining case is that $w_2$ contains $e_{m-1}$. Now $w_2$ is
reduced since $w$ is. By the inductive hypothesis, $w_2 = w_4 e_{m-1}
w_5$ where $w_4$, $w_5$ are words on $e_1, \dots, e_{m-2}$. Thus $w_4$
can be commuted to the left and $w_5$ to the right, and the length of
$w$ can once again be shortened using the equivalence $e_m e_{m-1} e_m
\sim e_m$. Contradiction.
\end{proof}

It follows immediately from Jones' Lemma by induction on $n$ that
there are only finitely many reduced words in $\TL_n$, hence that
$\TL_n$ has finite rank over $\Bbbk$. 


By Jones' Lemma, if $w$ is a reduced word in which $e_m$ is the
generator of maximum index, then by commuting $e_m$ as far to the
right as possible, we have
\[
w = w' (e_m e_{m-1} \cdots e_{m-l}), \qquad l\ge 0
\]
where $w'$ is a reduced word in which the generator of maximum index
is strictly smaller than $m$. Induction then leads to the following.

\begin{thm}[Jones' normal form]\label{t:Jones}
  Any reduced word $w$ in $\TL_n$ may be written in the form
  \[
  w = (e_{j_1}e_{j_1-1} \cdots e_{k_1}) (e_{j_2}e_{j_2-1} \cdots
  e_{k_2}) \cdots (e_{j_r}e_{j_r-1} \cdots e_{k_r})
  \]
  where $0 < j_1 < \cdots < j_r < n$, $0 < k_1 < \cdots < k_r < n$,
  and $j_i \ge k_i$ for all $i$. The index $j_r$ is the maximum index
  appearing in $w$.
\end{thm}

To each word in Jones normal form as above we may associate a
piecewise linear increasing (planar) path from $(0,0)$ to $(n,n)$:
\begin{multline*}
(0,0) \to (j_1,0) \to (j_1,k_1) \to (j_2,k_1) \to (j_2,k_2) \to \cdots
\\ \to (j_r,k_{r-1}) \to (j_r,k_r) \to (n,k_r) \to (n,n)
\end{multline*}
on the integer lattice $\Z \times \Z$ which does not cross the
diagonal.  Such paths are known as \emph{Dyck paths}.  For instance,
the word $(e_3e_2e_1)(e_4e_3)(e_5e_4)$ in $\TL_6$ corresponds to the
Dyck path
\[
\begin{tikzpicture}[scale = 0.5,thick, baseline={(0,-1ex/2)}] 
  \draw[very thick]
  (0,0)--(3,0)--(3,1)--(4,1)--(4,3)--(5,3)--(5,4)--(6,4)--(6,6);
  \draw[step=1,dashed,gray,very thin] (0,0) grid (6,6);
\end{tikzpicture}
\]
from $(0,0)$ in the lower left corner to $(6,6)$ in the upper right
corner. This map from words to Dyck paths is a bijection because
each such walk is determined by its corner points, and the normal form
of the word can be reconstructed from the coordinates of those points.

To count the number of Dyck paths to $(n,n)$ it is useful to consider
a slightly more general question. For any integer $p$ satisfying $0
\le 2p \le n$, define a \emph{lattice walk} to $(n-p,p)$ to be a
piecewise linear increasing path from $(0,0)$ to $(n-p,p)$ on the
integer lattice $\Z \times \Z$ which does not cross the diagonal. In
particular, Dyck paths are lattice walks to $(n,n)$. Let
\[
\Cat_{n,\,p} =
\begin{minipage}{3.2in}
number of lattice walks from $(0,0)$ to $(n-p,p)$.
\end{minipage}
\]
In this notation, $\Cat_{2n,n}$ gives the number of Dyck paths to
$(n,n)$. Any lattice walk to $(n-p,p)$ must pass through either
$(n-p-1,p)$ or $(n-p,p-1)$, so
\begin{equation}\label{e:recurrance}
  \Cat_{n,\,p} = \Cat_{n-1,\,p} + \Cat_{n-1,\,p-1}
\end{equation}
where $\Cat_{n,-1} = 0$.  We clearly have $\Cat_{n,\,0} = 1$. Also,
$\Cat_{2p-1,\,p} = 0$ as walks are not allowed to cross the
diagonal. With these boundary conditions the recurrence
\eqref{e:recurrance} is easily solved, giving the formula
\begin{equation}
  \Cat_{n,\,p} = \binom{n}{p}-\binom{n}{p-1}
\end{equation}
where we interpret $\binom{n}{-1}$ as zero, as usual. The set of words
in normal form spans $\TL_n$, so we have
\begin{equation}\label{e:upper-bound}
  \text{rank}_\Bbbk \TL_n \le \Cat_{2n,n} = \binom{2n}{n}-\binom{2n}{n-1} =
  \frac{1}{n+1}\binom{2n}{n}.
\end{equation}
This upper bound is the $n$th Catalan number. Linear independence of
the set of words in normal form will be proved in the next section,
thus showing that the set of such words is a basis of $\TL_n$, and the
upper bound on rank is precise.

%\begin{rmk}
%Some authors denote $\Cat_{n,p}$ by the symbol $\stirlingii{n}{p}$,
%but as this is also used for Stirling numbers of the second kind, we
%will stick to $\Cat_{n,p}$.
%\end{rmk}



\section{Diagrammatics}\label{s:dia}\noindent
We continue to work over an arbitrary commutative ring $\Bbbk$, where
$\delta$ is a fixed element of $\Bbbk$.  We now introduce a diagram
algebra $\D_n(\delta)$, based on planar diagrams called $n$-diagrams.
It will turn out that $\D_n(\delta)$ is isomorphic to
$\TL_n(\delta)$.

An $n$-\emph{diagram} is a planar graph, with $2n$ vertices consisting
of $n$ marked points on each of two parallel lines. Each point in the
graph is the endpoint of precisely one edge, and the edges can be
drawn by non-intersecting arcs which lie entirely between the
lines. If we label the vertices along one line by the set $\mathbf{n}
= \{1, \dots, n\}$ and by $\mathbf{n}' = \{1', \dots, n'\}$
correspondingly along the other line, where $\mathbf{n} \cap
\mathbf{n}' = \emptyset$, then we may identify an $n$-diagram $D$ with
a set partition $\{B_1, \dots, B_n\}$ of $\mathbf{n} \sqcup
\mathbf{n}'$ in which each subset (block) has cardinality two. For
example, the $8$-diagram
\[
D \;=\;
\begin{tikzpicture}[scale = 0.35,thick, baseline={(0,-1ex/2)}] 
  \tikzstyle{vertex} = [shape = circle, minimum size = 4pt, inner sep = 1pt,
    fill=black] 
\node[vertex] (G--8) at (10.5, -1) [shape = circle, draw] {}; 
\node[vertex] (G--7) at (9.0, -1) [shape = circle, draw] {}; 
\node[vertex] (G--6) at (7.5, -1) [shape = circle, draw] {}; 
\node[vertex] (G-8) at (10.5, 1) [shape = circle, draw] {}; 
\node[vertex] (G--5) at (6.0, -1) [shape = circle, draw] {}; 
\node[vertex] (G--2) at (1.5, -1) [shape = circle, draw] {}; 
\node[vertex] (G--4) at (4.5, -1) [shape = circle, draw] {}; 
\node[vertex] (G--3) at (3.0, -1) [shape = circle, draw] {}; 
\node[vertex] (G--1) at (0.0, -1) [shape = circle, draw] {}; 
\node[vertex] (G-1) at (0.0, 1) [shape = circle, draw] {}; 
\node[vertex] (G-2) at (1.5, 1) [shape = circle, draw] {}; 
\node[vertex] (G-7) at (9.0, 1) [shape = circle, draw] {}; 
\node[vertex] (G-3) at (3.0, 1) [shape = circle, draw] {}; 
\node[vertex] (G-4) at (4.5, 1) [shape = circle, draw] {}; 
\node[vertex] (G-5) at (6.0, 1) [shape = circle, draw] {}; 
\node[vertex] (G-6) at (7.5, 1) [shape = circle, draw] {}; 
\draw[] (G--8) .. controls +(-0.5, 0.5) and +(0.5, 0.5) .. (G--7); 
\draw[] (G-8) .. controls +(-1, -1) and +(1, 1) .. (G--6); 
\draw[] (G--5) .. controls +(-0.8, 0.8) and +(0.8, 0.8) .. (G--2); 
\draw[] (G--4) .. controls +(-0.5, 0.5) and +(0.5, 0.5) .. (G--3); 
\draw[] (G-1) .. controls +(0, -1) and +(0, 1) .. (G--1); 
\draw[] (G-2) .. controls +(1, -1) and +(-1, -1) .. (G-7); 
\draw[] (G-3) .. controls +(0.5, -0.5) and +(-0.5, -0.5) .. (G-4); 
\draw[] (G-5) .. controls +(0.5, -0.5) and +(-0.5, -0.5) .. (G-6); 
\end{tikzpicture}
\]
corresponds to the set partition
\[
\{ \{1,1'\}, \{2,7\}, \{3,4\},
\{5,6\}, \{8,6'\}, \{2',5'\}, \{3',4'\}, \{7',8'\} \}.
\]
In the literature, edges in $n$-diagrams are also called
\emph{strands} or \emph{links}. Some authors refer to edges connecting
two vertices in the top or bottom row as \emph{cups} or \emph{caps},
respectively, and to edges connecting a top vertex to a bottom one as
\emph{through strings} or \emph{propagating strands}.


We define a multiplication on the set of $n$-diagrams as follows. If
$D_1$, $D_2$ are given $n$-diagrams, we stack $D_1$ on top of $D_2$,
identifying the middle lines and their vertices. This results in a
graph with zero or more loops in the middle, and we define
\begin{equation}\label{e:mult-rule}
D_1 D_2 = \delta^L \, D_3
\end{equation}
where $L$ is the number of loops and $D_3$ is the $n$-diagram obtained
by removing the middle data (lines, loops, and vertices). Write
\[
\D_n(\delta) = \Bbbk\text{-linear span of the set of all $n$-diagrams}.
\]
With the multiplication rule given above, $\D_n(\delta)$ is an
associative algebra over $\Bbbk$.


Let $\hat{e}_i= \{\{i,i+1\}, \{i',(i+1)'\}\} \sqcup \{ \{j,j'\}: j \ne
i, i+1 \}$ ($i = 1, \dots, n-1$) and $\hat{1}= \{ \{i,i'\}: i = 1,
\dots, n \}$.
%\begin{align*}
%\hat{1} &= \{ \{i,i'\}: i = 1, \dots, n \},\\ \hat{e}_i &=
%\{\{i,i+1\}, \{i',(i+1)'\}\} \sqcup \{ \{j,j'\}: j \ne i, i+1 \}.
%\end{align*}
Then
\[
\hat{e}_i \;=\;
\onepic \cdots \onepic\quad \epic \quad\onepic \cdots \onepic
\]
and one checks from the multiplication rule \eqref{e:mult-rule} that
the $\hat{e}_i$ ($1 \le i \le n-1$) satisfy the defining relations
\eqref{e:TL} for $\TL_n(\delta)$. Moreover, $\hat{1} d = d = d \hat{1}$
for all $n$-diagrams $d$. It follows that there is a unique algebra
morphism
\begin{equation}\label{e:morphism}
  \sigma: \TL_n(\delta) \to \D_n(\delta)
\end{equation}
such that $\sigma(1) = \hat{1}$, $\sigma(e_i) = \hat{e}_i$ for all $i
= 1, \dots, n-1$.



In order to count the number of $n$-diagrams, it is again fruitful to
consider a more general problem: counting the number of half-diagrams.
Cutting a diagram by a line halfway between (and parallel to) its
defining parallel lines divides the diagram into two half-diagrams. We
conventionally reflect the bottom half-diagram across its line of
marked points, so that all half-diagrams are oriented with the links
lying below the line.  A half-diagram coming from an $n$-diagram has
$p$ links (arcs connecting two vertices) and $n-2p$ defects (arcs with
one vertex), where $0 \le 2p \le n$.

\begin{lem}\label{l:half-count}
  $\Cat_{n,p} = $ the number of half-diagrams on $n$ vertices with $p$
  links.
\end{lem}

\begin{proof}
The set of all half-diagrams on $n$ vertices with $p$ links is in
bijection with the set of lattice walks (see \S\ref{s:1}) from $(0,0)$
to $(n-p,p)$. Reading a half-diagram from left to right, a walker
moves up at the $k$-th step if the $k$-th marked point closes a link,
and moves right otherwise. For example, the half-diagram
\[
\begin{tikzpicture}[scale = 0.35,thick, baseline={(0,-1ex/2)}] 
  \tikzstyle{vertex} = [shape = circle, minimum size = 4pt, inner sep = 1pt,
    fill=black] 
\node[vertex] (G-1) at (0.0, 1) [shape = circle, draw] {}; 
\node[vertex] (G-2) at (1.5, 1) [shape = circle, draw] {}; 
\node[vertex] (G-3) at (3.0, 1) [shape = circle, draw] {}; 
\node[vertex] (G-4) at (4.5, 1) [shape = circle, draw] {}; 
\node[vertex] (G-5) at (6.0, 1) [shape = circle, draw] {}; 
\node[vertex] (G-6) at (7.5, 1) [shape = circle, draw] {};
\node[vertex] (G-7) at (9.0, 1) [shape = circle, draw] {};
\node[vertex] (G-8) at (10.5, 1) [shape = circle, draw] {};
\draw[] (G-8) -- (10.5,0); 
\draw[] (G-1) -- (0,0); 
\draw[] (G-2) .. controls +(1, -1) and +(-1, -1) .. (G-7); 
\draw[] (G-3) .. controls +(0.5, -0.5) and +(-0.5, -0.5) .. (G-4); 
\draw[] (G-5) .. controls +(0.5, -0.5) and +(-0.5, -0.5) .. (G-6); 
\end{tikzpicture}
\]
with $8$ vertices and $3$ links corresponds to the lattice walk
\[
\begin{tikzpicture}[scale = 0.5,thick, baseline={(0,-1ex/2)}] 
\draw[very thick] (0,0)--(3,0)--(3,1)--(4,1)--(4,3)--(5,3);
\draw[step=1,dashed,gray,very thin] (0,0) grid (5,5);
\end{tikzpicture}
\]
from $(0,0)$ to $(5,3) = (n-p,p)$. The half-diagram may be
reconstructed from the lattice walk, so this is a bijection as claimed. 
\end{proof}


Kauffman observed that the set of $n$-diagrams is in a natural
bijection with the set of half-diagrams with $2n$ vertices and $n$
links. This bijection is visualized by the following picture:
\[
\begin{tikzpicture}[scale = 0.25,thick, baseline={(0,-1ex/2)}]
  \tikzstyle{vertex} = [shape = circle, minimum size = 4pt, inner sep = 1pt,
    fill=black] 
  \draw[thick] (0,-1)--(5,-1)--(5,1)--(0,1)--(0,-1);
  \fill[lightgray] (0,-1) rectangle (5,1);
  \node[vertex] (B-1) at (1, -1) [shape = circle, draw] {};
  \node[vertex] (B-2) at (2, -1) [shape = circle, draw] {};
  \node[vertex] (B-3) at (3, -1) [shape = circle, draw] {};
  \node[vertex] (B-4) at (4, -1) [shape = circle, draw] {};
  \node[vertex] (T-1) at (1, 1) [shape = circle, draw] {};
  \node[vertex] (T-2) at (2, 1) [shape = circle, draw] {};
  \node[vertex] (T-3) at (3, 1) [shape = circle, draw] {};
  \node[vertex] (T-4) at (4, 1) [shape = circle, draw] {};
\end{tikzpicture}
\quad \mapsto \quad
\begin{tikzpicture}[scale = 0.25,thick, baseline={(0,-1ex/2)}]
  \tikzstyle{vertex} = [shape = circle, minimum size = 4pt, inner sep = 1pt,
    fill=black] 
  \draw[thick] (0,-1)--(5,-1)--(5,1)--(0,1)--(0,-1);
  \fill[lightgray] (0,-1) rectangle (5,1);
  \node[vertex] (B-1) at (1, -1) [shape = circle, draw] {};
  \node[vertex] (B-2) at (2, -1) [shape = circle, draw] {};
  \node[vertex] (B-3) at (3, -1) [shape = circle, draw] {};
  \node[vertex] (B-4) at (4, -1) [shape = circle, draw] {};
  \node[vertex] (T-1) at (1, 1) [shape = circle, draw] {};
  \node[vertex] (T-2) at (2, 1) [shape = circle, draw] {};
  \node[vertex] (T-3) at (3, 1) [shape = circle, draw] {};
  \node[vertex] (T-4) at (4, 1) [shape = circle, draw] {};
  \node[vertex] (T-5) at (6, 1) [shape = circle, draw] {};
  \node[vertex] (T-6) at (7, 1) [shape = circle, draw] {};
  \node[vertex] (T-7) at (8, 1) [shape = circle, draw] {};
  \node[vertex] (T-8) at (9, 1) [shape = circle, draw] {};
  \draw[] (B-4) .. controls +(2, -2) and +(0,-0.5) .. (T-5);
  \draw[] (B-3) .. controls +(4, -4) and +(0,-0.5) .. (T-6);
  \draw[] (B-2) .. controls +(6, -6) and +(0,-0.5) .. (T-7);
  \draw[] (B-1) .. controls +(8, -8) and +(0,-0.5) .. (T-8);
\end{tikzpicture}
\]
In other words, draw an $n$-diagram in a rectangle, and rotate its
bottom edge through an angle of $180^\circ$, with its vertex at the
upper right corner of the rectangle. Edges are stretched accordingly
to maintain the planarity. For example,
\[
\begin{tikzpicture}[scale = 0.35,thick, baseline={(0,-1ex/2)}] 
  \tikzstyle{vertex} = [shape = circle, minimum size = 4pt, inner sep = 1pt,
    fill=black] 
\node[vertex] (G--4) at (4.5, -1) [shape = circle, draw] {}; 
\node[vertex] (G-4) at (4.5, 1) [shape = circle, draw] {}; 
\node[vertex] (G--3) at (3.0, -1) [shape = circle, draw] {}; 
\node[vertex] (G--2) at (1.5, -1) [shape = circle, draw] {}; 
\node[vertex] (G--1) at (0.0, -1) [shape = circle, draw] {}; 
\node[vertex] (G-3) at (3.0, 1) [shape = circle, draw] {}; 
\node[vertex] (G-1) at (0.0, 1) [shape = circle, draw] {}; 
\node[vertex] (G-2) at (1.5, 1) [shape = circle, draw] {}; 
\draw[] (G-4) .. controls +(0, -1) and +(0, 1) .. (G--4); 
\draw[] (G--3) .. controls +(-0.5, 0.5) and +(0.5, 0.5) .. (G--2); 
\draw[] (G-3) .. controls +(-1, -1) and +(1, 1) .. (G--1); 
\draw[] (G-1) .. controls +(0.5, -0.5) and +(-0.5, -0.5) .. (G-2); 
\end{tikzpicture}
\quad \mapsto \quad
%\begin{tikzpicture}[scale = 0.35,thick, baseline={(0,-1ex/2)}] 
%  \tikzstyle{vertex} = [shape = circle, minimum size = 4pt, inner sep = 1pt,
%    fill=black] 
%\node[vertex] (G--4) at (4.5, -1) [shape = circle, draw] {}; 
%\node[vertex] (G-4) at (4.5, 1) [shape = circle, draw] {}; 
%\node[vertex] (G--3) at (3.0, -1) [shape = circle, draw] {}; 
%\node[vertex] (G--2) at (1.5, -1) [shape = circle, draw] {}; 
%\node[vertex] (G--1) at (0.0, -1) [shape = circle, draw] {}; 
%\node[vertex] (G-3) at (3.0, 1) [shape = circle, draw] {}; 
%\node[vertex] (G-1) at (0.0, 1) [shape = circle, draw] {}; 
%\node[vertex] (G-2) at (1.5, 1) [shape = circle, draw] {}; 
%\draw[] (G-4) .. controls +(0, -1) and +(0, 1) .. (G--4); 
%\draw[] (G--3) .. controls +(-0.5, 0.5) and +(0.5, 0.5) .. (G--2); 
%\draw[] (G-3) .. controls +(-1, -1) and +(1, 1) .. (G--1); 
%\draw[] (G-1) .. controls +(0.5, -0.5) and +(-0.5, -0.5) .. (G-2);
%\node[vertex] (T-5) at (6,1) [shape = circle, draw] {};
%\node[vertex] (T-6) at (7.5,1) [shape = circle, draw] {};
%\node[vertex] (T-7) at (9,1) [shape = circle, draw] {};
%\node[vertex] (T-8) at (10.5,1) [shape = circle, draw] {};
%  \draw[] (G--4) .. controls +(2, -2) and +(0,-0.5) .. (T-5);
%  \draw[] (G--3) .. controls +(4, -4) and +(0,-0.5) .. (T-6);
%  \draw[] (G--2) .. controls +(6, -6) and +(0,-0.5) .. (T-7);
%  \draw[] (G--1) .. controls +(8, -8) and +(0,-0.5) .. (T-8);
%\end{tikzpicture}
%\quad \cong \quad
\begin{tikzpicture}[scale = 0.35,thick, baseline={(0,-1ex/2)}] 
  \tikzstyle{vertex} = [shape = circle, minimum size = 4pt, inner sep = 1pt,
    fill=black] 
\node[vertex] (G-4) at (4.5, 1) [shape = circle, draw] {}; 
\node[vertex] (G-3) at (3.0, 1) [shape = circle, draw] {}; 
\node[vertex] (G-1) at (0.0, 1) [shape = circle, draw] {}; 
\node[vertex] (G-2) at (1.5, 1) [shape = circle, draw] {};
\node[vertex] (T-5) at (6,1) [shape = circle, draw] {};
\node[vertex] (T-6) at (7.5,1) [shape = circle, draw] {};
\node[vertex] (T-7) at (9,1) [shape = circle, draw] {};
\node[vertex] (T-8) at (10.5,1) [shape = circle, draw] {};
\draw[] (G-1) .. controls +(0.5, -0.5) and +(-0.5, -0.5) .. (G-2);
\draw[] (G-4) .. controls +(0.5, -0.5) and +(-0.5, -0.5) .. (T-5);
\draw[] (T-6) .. controls +(0.5, -0.5) and +(-0.5, -0.5) .. (T-7);
\draw[] (G-3) .. controls +(1.5, -1.5) and +(-1.5, -1.5) .. (T-8);
\end{tikzpicture}
\]
This process of mapping $n$-diagrams to half-diagrams (with $2n$
vertices and $n$ links) is clearly reversible, hence defines a
bijection.  Thus, $\Cat_{2n,n}$ counts this number, which is the same
as the upper bound on the rank in~\eqref{e:upper-bound}.





Let $D_n$ be the underlying monoid of $\D_n(1)$. This is the set
of $n$-diagrams, with the multiplication rule specialized at $\delta =
1$. Let $M_n$ be the monoid defined by the presentation \eqref{e:TL}
again with $\delta =1$, so that the relation $e_i^2 = \delta e_i$ is
replaced by $e_i^2 = e_i$. To show that $\TL_n(\delta)$ is isomorphic
to $\D_n(\delta)$ as algebras, it clearly suffices to show that $M_n$
and $D_n$ are isomorphic as monoids. (They are both twisted semigroup
algebras in the sense of \cite{Wilcox}.)

\begin{example}\label{x:factoring}
We illustrate the proof of the first claim in the next result by means
of an example, in which we show how to construct the diagram
\[
d \; = \;
\begin{tikzpicture}[scale = 0.35,thick, baseline={(0,-1ex/2)}] 
\tikzstyle{vertex}=[shape=circle, minimum size=4pt, inner sep=1pt, fill=black]
\node[vertex] (G--7) at (9.0, -1) [shape = circle, draw] {}; 
\node[vertex] (G--2) at (1.5, -1) [shape = circle, draw] {}; 
\node[vertex] (G--6) at (7.5, -1) [shape = circle, draw] {}; 
\node[vertex] (G--3) at (3.0, -1) [shape = circle, draw] {}; 
\node[vertex] (G--5) at (6.0, -1) [shape = circle, draw] {}; 
\node[vertex] (G--4) at (4.5, -1) [shape = circle, draw] {}; 
\node[vertex] (G--1) at (0.0, -1) [shape = circle, draw] {}; 
\node[vertex] (G-7) at (9.0, 1) [shape = circle, draw] {}; 
\node[vertex] (G-1) at (0.0, 1) [shape = circle, draw] {}; 
\node[vertex] (G-6) at (7.5, 1) [shape = circle, draw] {}; 
\node[vertex] (G-2) at (1.5, 1) [shape = circle, draw] {}; 
\node[vertex] (G-3) at (3.0, 1) [shape = circle, draw] {}; 
\node[vertex] (G-4) at (4.5, 1) [shape = circle, draw] {}; 
\node[vertex] (G-5) at (6.0, 1) [shape = circle, draw] {}; 
\draw[] (G--7) .. controls +(-1, 1) and +(1, 1) .. (G--2); 
\draw[] (G--6) .. controls +(-0.7, 0.7) and +(0.7, 0.7) .. (G--3); 
\draw[] (G--5) .. controls +(-0.5, 0.5) and +(0.5, 0.5) .. (G--4); 
\draw[] (G-7) .. controls +(-1, -1) and +(1, 1) .. (G--1); 
\draw[] (G-1) .. controls +(1, -1) and +(-1, -1) .. (G-6); 
\draw[] (G-2) .. controls +(0.5, -0.5) and +(-0.5, -0.5) .. (G-3); 
\draw[] (G-4) .. controls +(0.5, -0.5) and +(-0.5, -0.5) .. (G-5); 
\end{tikzpicture}
\]
as a product of generators. The key is to number the strands in $d$ in
the natural reading order. Identify the strand which lies above and to
the left of all others and number it by $1$. Removing that strand from
the diagram, repeat the process on the resulting diagram, numbering
the selected strand by $2$, etc.  In the given $d$, the numbering
is:
\begin{center}
\begin{tabular}{l|ccccccc}
strand & 1&2&3&4&5&6&7\\ \hline
vertices & \{2,3\}&\{4,5\}&\{1,6\}&\{7,1'\}&\{2',7'\}&\{3',6'\}&\{4',5'\}
\end{tabular}
\end{center}
The diagram $d$ is constructed by the following product of generator
diagrams.
\begin{alignat*}{3}
\text{strands 1, 2} \quad &
\begin{tikzpicture}[scale = 0.35,thick, baseline={(0,-1ex/2)}] 
\tikzstyle{vertex}=[shape=circle, minimum size=4pt, inner sep=1pt, fill=black] 
\node[vertex] (G--7) at (9.0, -1) [shape = circle, draw] {}; 
\node[vertex] (G-7) at (9.0, 1) [shape = circle, draw] {}; 
\node[vertex] (G--6) at (7.5, -1) [shape = circle, draw] {}; 
\node[vertex] (G-6) at (7.5, 1) [shape = circle, draw] {}; 
\node[vertex] (G--5) at (6.0, -1) [shape = circle, draw] {}; 
\node[vertex] (G--4) at (4.5, -1) [shape = circle, draw] {}; 
\node[vertex] (G--3) at (3.0, -1) [shape = circle, draw] {}; 
\node[vertex] (G--2) at (1.5, -1) [shape = circle, draw] {}; 
\node[vertex] (G--1) at (0.0, -1) [shape = circle, draw] {}; 
\node[vertex] (G-1) at (0.0, 1) [shape = circle, draw] {}; 
\node[vertex] (G-2) at (1.5, 1) [shape = circle, draw] {}; 
\node[vertex] (G-3) at (3.0, 1) [shape = circle, draw] {}; 
\node[vertex] (G-4) at (4.5, 1) [shape = circle, draw] {}; 
\node[vertex] (G-5) at (6.0, 1) [shape = circle, draw] {}; 
\draw[] (G-7) .. controls +(0, -1) and +(0, 1) .. (G--7); 
\draw[] (G-6) .. controls +(0, -1) and +(0, 1) .. (G--6); 
\draw[] (G--5) .. controls +(-0.5, 0.5) and +(0.5, 0.5) .. (G--4); 
\draw[] (G--3) .. controls +(-0.5, 0.5) and +(0.5, 0.5) .. (G--2); 
\draw[] (G-1) .. controls +(0, -1) and +(0, 1) .. (G--1); 
\draw[] (G-2) .. controls +(0.5, -0.5) and +(-0.5, -0.5) .. (G-3); 
\draw[] (G-4) .. controls +(0.5, -0.5) and +(-0.5, -0.5) .. (G-5); 
\end{tikzpicture}
& \quad \hat{e}_2\hat{e}_4
\\
\text{strand 3}\quad &
\begin{tikzpicture}[scale = 0.35,thick, baseline={(0,-1ex/2)}] 
\tikzstyle{vertex}=[shape=circle, minimum size=4pt, inner sep=1pt, fill=black] 
\node[vertex] (G--7) at (9.0, -1) [shape = circle, draw] {}; 
\node[vertex] (G-7) at (9.0, 1) [shape = circle, draw] {}; 
\node[vertex] (G--6) at (7.5, -1) [shape = circle, draw] {}; 
\node[vertex] (G--5) at (6.0, -1) [shape = circle, draw] {}; 
\node[vertex] (G--4) at (4.5, -1) [shape = circle, draw] {}; 
\node[vertex] (G--3) at (3.0, -1) [shape = circle, draw] {}; 
\node[vertex] (G--2) at (1.5, -1) [shape = circle, draw] {}; 
\node[vertex] (G--1) at (0.0, -1) [shape = circle, draw] {}; 
\node[vertex] (G-1) at (0.0, 1) [shape = circle, draw] {}; 
\node[vertex] (G-2) at (1.5, 1) [shape = circle, draw] {}; 
\node[vertex] (G-3) at (3.0, 1) [shape = circle, draw] {}; 
\node[vertex] (G-4) at (4.5, 1) [shape = circle, draw] {}; 
\node[vertex] (G-5) at (6.0, 1) [shape = circle, draw] {}; 
\node[vertex] (G-6) at (7.5, 1) [shape = circle, draw] {}; 
\draw[] (G-7) .. controls +(0, -1) and +(0, 1) .. (G--7); 
\draw[] (G--6) .. controls +(-0.5, 0.5) and +(0.5, 0.5) .. (G--5); 
\draw[] (G--4) .. controls +(-0.5, 0.5) and +(0.5, 0.5) .. (G--3); 
\draw[] (G--2) .. controls +(-0.5, 0.5) and +(0.5, 0.5) .. (G--1); 
\draw[] (G-1) .. controls +(0.5, -0.5) and +(-0.5, -0.5) .. (G-2); 
\draw[] (G-3) .. controls +(0.5, -0.5) and +(-0.5, -0.5) .. (G-4); 
\draw[] (G-5) .. controls +(0.5, -0.5) and +(-0.5, -0.5) .. (G-6); 
\end{tikzpicture}
& \quad \hat{e}_1\hat{e}_3\hat{e}_5
\\
\text{strand 4} \quad &
\begin{tikzpicture}[scale = 0.35,thick, baseline={(0,-1ex/2)}] 
\tikzstyle{vertex}=[shape=circle, minimum size=4pt, inner sep=1pt, fill=black] 
\node[vertex] (G--7) at (9.0, -1) [shape = circle, draw] {}; 
\node[vertex] (G--6) at (7.5, -1) [shape = circle, draw] {}; 
\node[vertex] (G--5) at (6.0, -1) [shape = circle, draw] {}; 
\node[vertex] (G--4) at (4.5, -1) [shape = circle, draw] {}; 
\node[vertex] (G--3) at (3.0, -1) [shape = circle, draw] {}; 
\node[vertex] (G--2) at (1.5, -1) [shape = circle, draw] {}; 
\node[vertex] (G--1) at (0.0, -1) [shape = circle, draw] {}; 
\node[vertex] (G-1) at (0.0, 1) [shape = circle, draw] {}; 
\node[vertex] (G-2) at (1.5, 1) [shape = circle, draw] {}; 
\node[vertex] (G-3) at (3.0, 1) [shape = circle, draw] {}; 
\node[vertex] (G-4) at (4.5, 1) [shape = circle, draw] {}; 
\node[vertex] (G-5) at (6.0, 1) [shape = circle, draw] {}; 
\node[vertex] (G-6) at (7.5, 1) [shape = circle, draw] {}; 
\node[vertex] (G-7) at (9.0, 1) [shape = circle, draw] {}; 
\draw[] (G--7) .. controls +(-0.5, 0.5) and +(0.5, 0.5) .. (G--6); 
\draw[] (G--5) .. controls +(-0.5, 0.5) and +(0.5, 0.5) .. (G--4); 
\draw[] (G--3) .. controls +(-0.5, 0.5) and +(0.5, 0.5) .. (G--2); 
\draw[] (G-1) .. controls +(0, -1) and +(0, 1) .. (G--1); 
\draw[] (G-2) .. controls +(0.5, -0.5) and +(-0.5, -0.5) .. (G-3); 
\draw[] (G-4) .. controls +(0.5, -0.5) and +(-0.5, -0.5) .. (G-5); 
\draw[] (G-6) .. controls +(0.5, -0.5) and +(-0.5, -0.5) .. (G-7); 
\end{tikzpicture}
& \quad \hat{e}_2\hat{e}_4\hat{e}_6
\\
\text{strand 5} \quad &
\begin{tikzpicture}[scale = 0.35,thick, baseline={(0,-1ex/2)}] 
\tikzstyle{vertex}=[shape=circle, minimum size=4pt, inner sep=1pt, fill=black] 
\node[vertex] (G--7) at (9.0, -1) [shape = circle, draw] {}; 
\node[vertex] (G-7) at (9.0, 1) [shape = circle, draw] {}; 
\node[vertex] (G--6) at (7.5, -1) [shape = circle, draw] {}; 
\node[vertex] (G--5) at (6.0, -1) [shape = circle, draw] {}; 
\node[vertex] (G--4) at (4.5, -1) [shape = circle, draw] {}; 
\node[vertex] (G--3) at (3.0, -1) [shape = circle, draw] {}; 
\node[vertex] (G--2) at (1.5, -1) [shape = circle, draw] {}; 
\node[vertex] (G-2) at (1.5, 1) [shape = circle, draw] {}; 
\node[vertex] (G--1) at (0.0, -1) [shape = circle, draw] {}; 
\node[vertex] (G-1) at (0.0, 1) [shape = circle, draw] {}; 
\node[vertex] (G-3) at (3.0, 1) [shape = circle, draw] {}; 
\node[vertex] (G-4) at (4.5, 1) [shape = circle, draw] {}; 
\node[vertex] (G-5) at (6.0, 1) [shape = circle, draw] {}; 
\node[vertex] (G-6) at (7.5, 1) [shape = circle, draw] {}; 
\draw[] (G-7) .. controls +(0, -1) and +(0, 1) .. (G--7); 
\draw[] (G--6) .. controls +(-0.5, 0.5) and +(0.5, 0.5) .. (G--5); 
\draw[] (G--4) .. controls +(-0.5, 0.5) and +(0.5, 0.5) .. (G--3); 
\draw[] (G-2) .. controls +(0, -1) and +(0, 1) .. (G--2); 
\draw[] (G-1) .. controls +(0, -1) and +(0, 1) .. (G--1); 
\draw[] (G-3) .. controls +(0.5, -0.5) and +(-0.5, -0.5) .. (G-4); 
\draw[] (G-5) .. controls +(0.5, -0.5) and +(-0.5, -0.5) .. (G-6); 
\end{tikzpicture}
& \quad \hat{e}_3\hat{e}_5
\\
\text{strands 6, 7} \quad &
\begin{tikzpicture}[scale = 0.35,thick, baseline={(0,-1ex/2)}] 
\tikzstyle{vertex}=[shape=circle, minimum size=4pt, inner sep=1pt, fill=black] 
\node[vertex] (G--7) at (9.0, -1) [shape = circle, draw] {}; 
\node[vertex] (G-7) at (9.0, 1) [shape = circle, draw] {}; 
\node[vertex] (G--6) at (7.5, -1) [shape = circle, draw] {}; 
\node[vertex] (G-6) at (7.5, 1) [shape = circle, draw] {}; 
\node[vertex] (G--5) at (6.0, -1) [shape = circle, draw] {}; 
\node[vertex] (G--4) at (4.5, -1) [shape = circle, draw] {}; 
\node[vertex] (G--3) at (3.0, -1) [shape = circle, draw] {}; 
\node[vertex] (G-3) at (3.0, 1) [shape = circle, draw] {}; 
\node[vertex] (G--2) at (1.5, -1) [shape = circle, draw] {}; 
\node[vertex] (G-2) at (1.5, 1) [shape = circle, draw] {}; 
\node[vertex] (G--1) at (0.0, -1) [shape = circle, draw] {}; 
\node[vertex] (G-1) at (0.0, 1) [shape = circle, draw] {}; 
\node[vertex] (G-4) at (4.5, 1) [shape = circle, draw] {}; 
\node[vertex] (G-5) at (6.0, 1) [shape = circle, draw] {}; 
\draw[] (G-7) .. controls +(0, -1) and +(0, 1) .. (G--7); 
\draw[] (G-6) .. controls +(0, -1) and +(0, 1) .. (G--6); 
\draw[] (G--5) .. controls +(-0.5, 0.5) and +(0.5, 0.5) .. (G--4); 
\draw[] (G-3) .. controls +(0, -1) and +(0, 1) .. (G--3); 
\draw[] (G-2) .. controls +(0, -1) and +(0, 1) .. (G--2); 
\draw[] (G-1) .. controls +(0, -1) and +(0, 1) .. (G--1); 
\draw[] (G-4) .. controls +(0.5, -0.5) and +(-0.5, -0.5) .. (G-5); 
\end{tikzpicture}
& \quad \hat{e}_4
\end{alignat*}
To construct strands 1, 2 use $\hat{e}_2\hat{e}_4$. Then follow by
$\hat{e}_1\hat{e}_3\hat{e}_5$ to complete strand 3, followed by
$\hat{e}_2\hat{e}_4\hat{e}_6$ to complete strand 4, followed by
$\hat{e}_3\hat{e}_5$ to complete strand 5. Finally, follow by
$\hat{e}_4$ to complete strands 6, 7 and the process is finished.
Hence $d = (\hat{e}_2\hat{e}_4)(\hat{e}_1\hat{e}_3\hat{e}_5)
(\hat{e}_2\hat{e}_4\hat{e}_6)(\hat{e}_3\hat{e}_5)(\hat{e}_4)$. Using
commutation relations it is easy to rewrite this in the form
\[
d = (\hat{e}_2\hat{e}_1)(\hat{e}_4\hat{e}_3\hat{e}_2)
(\hat{e}_5\hat{e}_4\hat{e}_3)(\hat{e}_6\hat{e}_5\hat{e}_4)
\]
which is the Jones normal form of $d$. 
\end{example}

\begin{thm}[Kauffman \cite{K:90}*{Thm.~4.3}]\label{t:Kauffman}
  Every diagram $d$ in $D_n$ is expressible as a product of the
  diagrams $\hat{e}_i$ ($1 \le i \le n-1$).  Thus the map $\sigma$ in
  equation \eqref{e:morphism} induces a monoid isomorphism $M_n \cong
  D_n$.
\end{thm}

\begin{proof}
The proof of the first claim is illustrated by
Example~\ref{x:factoring} above (cf.~also \cite{K:90}*{Figure 16}).
It is clear from the example that every element of $D_n$ has a Jones
normal form. Thus, the map $\sigma$ restricts to a surjection $\sigma:
M_n \to D_n$. But $M_n$ and $D_n$ are finite sets such that $|M_n| \le
|D_n|$, by Theorem~\ref{t:Jones}, so $|M_n| = |D_n|$ and $\sigma$ is
injective, hence is an isomorphism.
\end{proof}

By the remarks above, we have the following immediate consequence.

\begin{cor}
  Over any commutative ring $\Bbbk$, for any $\delta \in \Bbbk$, the
  map $\sigma$ is a $\Bbbk$-algebra isomorphism $\TL_n(\delta) \cong
  \D_n(\delta)$.
\end{cor}

Note that it also follows from this analysis that the Jones normal
form of a diagram is unique. Henceforth, we identify $\TL_n(\delta)$
with $\D_n(\delta)$ and $e_i$ with $\hat{e}_i$ for all $i$.

It follows from the diagrammatic interpretation in this section that
$\TL_n(\delta)$ may be identified with the subalgebra of Brauer's
centralizer algebra (on $n$ strands, with parameter $\delta$) spanned
by its planar diagrams. In \cite{Birman-Wenzl}, Birman and Wenzl found
a presentation of Brauer's algebra that implies the same result.


\section{The Jones algebra}\noindent
Jones worked with a slightly different form of the Temperley--Lieb
algebra, in which the generators $e_i$ are rescaled to become
idempotents.  Let $\Bbbk$ be a commutative ring with $1$. The Jones
algebra $A_n(\beta)$ of rank $n$ with parameter $\beta$ in $\Bbbk$ is
the unital $\Bbbk$-algebra defined by the generators $u_1, \dots,
u_{n-1}$ and the relations
\begin{equation}\label{e:Jones}
  \begin{aligned}
  u_i^2 = u_i, \quad
  \beta u_i u_j u_i = u_i \text{ if } |i-j|=1, \quad
  u_i u_j &= u_j u_i \text{ if } |i-j|>1 .
\end{aligned}
\end{equation}
By setting $e_i = \delta u_i$ for all $i$ we recover the defining
relations \eqref{e:TL} from those in \eqref{e:Jones} if and only if
$\beta = \delta^2$. Thus, we have the following.

\begin{prop}\label{p:TL-iso}
  Suppose that $\Bbbk$ is a commutative ring. For any $\delta$ in
  $\Bbbk$, there is a $\Bbbk$-algebra isomorphism $A_n(\delta^2) \cong
  \TL_n(\delta)$ given by $u_i \mapsto e_i$ for all $i$.
\end{prop}

\begin{rmk}
Jones \cite{Jones:83} studied the \emph{index} $[M:N]$ of a subfactor
$N \subset M$ of type $\mathrm{II}_1$. The algebras $N$, $M$ are
typically infinite dimensional simple von Neumann algebras sharing a
common identity and with a trace. In this context, Jones constructed
an infinite sequence of self-adjoint projections $u_i$ satisfying the
relations \eqref{e:Jones}, where $\beta = [M:N]$ is the index of the
subfactor. The index is always a positive real number and Jones proved
that
\[
\text{either } [M:N] \in \{4\cos^2(\tfrac{\pi}{n}): n = 3,4,5, \dots\}
\text{ or } [M:N] \in [4,\infty].
\]
This is a complete classification of all possible values.  See
\cite{EK} for a recent survey of connections between subfactors and
mathematical physics.

The finite dimensional algebras $A_n(\beta)$ appeared in \cite{GHJ} in
connection with a pair $N \subset M$ of split semisimple finite
dimensional algebras over a field $\Bbbk$. The pair is determined, up
to isomorphism, by an inclusion matrix $\Lambda$ with nonnegative
integer entries. The matrix $\Lambda$ may be encoded as a graph, the
Bratteli diagram of the pair, and it turns out that $[M:N]$ is the
square of the Euclidean norm of the graph. It follows that $[M:N] \le
4$ if and only if the Bratteli diagram of the pair $N \subset M$ is a
Coxeter graph of type A, D, or E.
\end{rmk}


Now we consider the issue of semisimplicity of the Jones algebra when
$\Bbbk$ is a field, following \cite{GHJ}.  Let $x$ be an indeterminate
and $\Z[x]$ the ring of polynomials in $x$ with integer
coefficients. Define polynomials $P_n(x)$ in $\Z[x]$ for each integer
$n \ge 0$ by the recursion
\begin{equation}
\begin{gathered}
  P_0(x)=1, \qquad P_1(x)=1,\\ P_{n+1}(x) = P_n(x)-xP_{n-1}(x)
  \quad \text{if $n \ge 1$}.
\end{gathered}
\end{equation}
%The first few values of the $P_n(x)$ are tabulated below:
%\[
%\begin{alignedat}{4}
%%P_2(x) &= 1-x,& P_3(x) &= 1-2x,\\ P_4(x) &= 1-3x+x^2,& P_5(x) &=
%1-4x+3x^2,\\ P_6(x) &= 1-5x+6x^2-x^3,\qquad & P_7(x) &= 1-6x+10x^2-4x^3.
%\end{alignedat}
%\]
The $P_n(x)$ are closely related to the Chebyshev polynomials of the
second kind \cite{Benkart-Moon}.

Here is the semisimplicity result. It was originally obtained over
$\C$ by Jones, and extended to arbitrary fields in the cited
reference.

\begin{thm}[\cite{GHJ}*{Prop.~2.8.5(a)}]\label{t:ss}
  Suppose that $\Bbbk$ is a field, $0 \ne \beta$, and $P_1(\beta^{-1})
  P_2(\beta^{-1}) \cdots P_{n-1}(\beta^{-1}) \ne 0$ in $\Bbbk$.  Then
  the algebra $A_n(\beta)$ is split semisimple over $\Bbbk$.
\end{thm}

Let $q \in \Bbbk$. For a positive integer $n$, the classical Gaussian
integer $\dbracket{n}_q$ is
\[
  \dbracket{n}_q = 1+q+q^2 + \cdots +q^{n-1}.
\]  
If $q \ne 1$, it can be written in the form $\dbracket{n}_q =
(1-q^n)/(1-q)$ but the definition of $\dbracket{n}_q$ makes perfect
sense at $q = 1$, where it evaluates to the integer $n$. It is
customary to set $\dbracket{0}_q = 0$. Let
\[
\dbracket{n}_q^! = \dbracket{1}_q \cdots \dbracket{n-1}_q
\dbracket{n}_q = \textstyle \prod_{k=1}^n \dbracket{k}_q
\]
if $n>0$, and set $\dbracket{0}_q^! = 1$. 

Now choose $q$ in $\Bbbk$ such that $q \ne 0$, $q \ne -1$, and $\beta
= q+2+q^{-1}$. (Replace $\Bbbk$ by a suitable quadratic extension if
necessary.) It follows by a simple induction that
\begin{equation}
P_n(\beta^{-1}) = \frac{1+q+q^2 + \cdots + q^n}{(1+q)^n} =
\frac{\dbracket{n+1}}{(1+q)^n}.
\end{equation}
This was observed in Prop.~2.8.3(iv) of \cite{GHJ}.  Then
Theorem~\ref{t:ss} gives the following corollary.

\begin{cor}\label{c:ss}
  Suppose that $q \ne 0$, $q \ne -1$ where $q$ is in the field
  $\Bbbk$. With $\beta = q+2+q^{-1}$, the Jones algebra $A_n(\beta) =
  A_n(q+2+q^{-1})$ is split semisimple over $\Bbbk$ whenever
  $\dbracket{n}^!_q \ne 0$.
\end{cor}


If $\beta = q + 2 + q^{-1}$ then $\beta^{1/2} = \pm(q^{1/2} +
q^{-1/2})$, provided that a square root of $q$ exists in $\Bbbk$.
This, in light of Proposition~\ref{p:TL-iso}, gives the following
restatement of Corollary~\ref{c:ss}.

\begin{cor}
  Let $\Bbbk$ be a field containing a square root $q^{1/2}$ of $q$,
  where $q \ne 0$, $q \ne -1$. If $\dbracket{n}^!_q \ne 0$ then
  $\TL_n(\pm(q^{1/2}+q^{-1/2}))$ is split semisimple over $\Bbbk$.
\end{cor}


Now set $v = q^{1/2}$ so that $q = v^2$. We will always assume $q=v^2$
from now on. Then
\[
  \dbracket{n}_q = \dbracket{n}_{v^2} = 1 + v^2 + v^4 + \cdots +
  v^{2(n-1)} = v^{n-1} \textstyle \sum_{k=0}^{n-1} v^{-(n-1)+2k}.
\]
The \emph{balanced} form $[n]_v$ of the Gaussian integer corresponding
to $n$ is defined by
\[
[n]_v = \textstyle \sum_{k=0}^{n-1} v^{-(n-1)+2k}. 
\]
The definition of $[n]_v$ makes sense when $v=1$, in which case it
evaluates to $n$.  Notice that if $v^2 \ne 1$ then we can write $[n]_v
= \frac{v^n - v^{-n}}{v-v^{-1}}$.  We define $[n]_v^!  = [1]_v
\cdots[n-1] [n]_v$ and set $[0]^!_v = 1$.  The balanced and classical
forms of Gaussian integers are related by
\begin{equation}
  \dbracket{n}_q = v^{n-1} [n]_v  \qquad (\text{for } q=v^2).
\end{equation}
As $[n]_v$ and $\dbracket{n}_q$ are the same up to a power of $v$, the
preceding corollary may be restated in the following form.

\begin{cor}\label{c:ss-crit}
  Let $\Bbbk$ be a field, $0 \ne v \in \Bbbk$, where $0 \ne
  v+v^{-1}$. If $[n]_v^! \ne 0$ then $\TL_n(\pm(v+v^{-1}))$ is split
  semisimple over $\Bbbk$.
\end{cor}

See \cite{DG:orthog} for a new elementary proof of this result. The
recent paper \cite{AST} gives a very different proof using tilting
modules, and gives a complete classification. In many of the early
references, e.g., \cites{GHJ,Martin,Westbury,Benkart-Moon},
semisimplicity criteria were formulated in a more complicated way than
the simple condition in Corollary~\ref{c:ss-crit}.






\section{$\TL_n$ as a quotient of the Iwahori--Hecke algebra}%
\label{s:Hecke}\noindent
In this section we let $\Bbbk$ be a commutative unital ring. It will
be convenient to work with a two-parameter version of the
Iwahori--Hecke algebra $\HH_n(x_1,x_2)$ of type A, in order to avoid
choosing one normalization over another. The algebra $\HH_n(x_1,x_2)$
appeared in \cites{BW, Bigelow}; in particular, in \cite{BW} it is
shown to be in Schur--Weyl duality with a two-parameter quantized
enveloping algebra considered by Takeuchi \cite{Takeuchi}.

Let $x_1$, $x_2$ be elements of $\Bbbk$. The algebra $\HH_n(x_1,x_2)$
is defined by the generators $T_1, \dots, T_{n-1}$ subject to the
relations
\begin{equation}\label{e:Hecke}
\begin{gathered}
  T_iT_jT_i = T_jT_iT_j \text{ if } |i-j|=1; \qquad
  T_iT_j = T_jT_i \text{ if } |i-j|>1 \\
  (T_i-x_1)(T_i-x_2) = 0 .
\end{gathered}
\end{equation}
We assume that $x_1x_2 \ne 0$. This implies that the $T_i$ are
invertible, and the $T_i^{-1}$ are given by
\begin{equation}
T_i^{-1} = (T_i-x_1-x_2)/(x_1x_2).
\end{equation}
Whenever $x_1x_2 \ne 0$, $\HH_n(x_1,x_2)$ depends, up to isomorphism,
only on the negative ratio $-x_2/x_1$.  The original one-parameter
version of this algebra was $\HH_n(-1,q)$; it was used for instance in
papers of Dipper and James \cites{DJ1,DJ2,DJ3,DJ4} and Murphy
\cites{Murphy1,Murphy2}.  Other popular one-parameter versions are
$\HH_n(1,-q)$ and $\HH_n(-v,v^{-1})$, where $v^2=q$. (The latter,
which is probably used most frequently these days, requires the
existence of a square root $v = q^{1/2}$ in $\Bbbk$.)


We wish to construct the Temperley--Lieb algebra $\TL_n(\delta)$ as a
suitable quotient of $\HH_n(x_1,x_2)$. In the one-parameter case this
was first done in \cite{Jones}. To do it more generally, we make the
linear substitution
\begin{equation}\label{e:subst}
e_i = aT_i+b
\end{equation}
where $a\ne 0$, $b$ are scalars yet to be determined, and where $e_i^2
= \delta e_i$.  The quadratic relation $(T_i-x_1)(T_i-x_2)=0$ in
\eqref{e:Hecke} is equivalent to
\[
T_i^2 = (x_1+x_2)T_i - x_1x_2.
\]
Combining this with \eqref{e:subst}, we obtain
\begin{align*}
  e_i^2 &= a^2T_i^2 + 2abT_i + b^2 \\
  &= a^2\big( (x_1+x_2)T_i - x_1x_2 \big) + 2abT_i + b^2 \\
  &= \big( a^2(x_1+x_2) + 2ab \big) T_i - a^2x_1x_2 + b^2 .
\end{align*}
Thus, in order to make $e_i^2 = \delta e_i$ we are forced to solve the
system
\begin{align*}
  a^2(x_1+x_2) + 2ab &= \delta a \\
  -a^2x_1x_2 + b^2 &= \delta b
\end{align*}
of two equations in three unknowns. Since $a \ne 0$ the first equation
in the system is equivalent to
\[
a(x_1+x_2) + 2b = \delta 
\]
and substituting this into the second equation (thus eliminating
$\delta$) gives
\[
-a^2x_1x_2 + b^2 = ab(x_1+x_2) + 2b^2 
\]
which is equivalent to
\[
0 = a^2x_1x_2 + ab(x_1+x_2) + b^2 = (ax_1+b)(ax_2+b).
\]
This proves the following result.


\begin{thm}\label{t:e_i}
  In order to have $e_i = aT_i+b$ with $a \ne 0$ satisfy $e_i^2 =
  \delta e_i$ it is necessary to choose $b$ so that $b = -ax_1$
  or $b = -ax_2$. Then $\delta = a(x_1+x_2)+2b$ is given by
  \[
  \delta = a(x_2-x_1) \;\text{ or }\;
  \delta = a(x_1-x_2)
  \]
  respectively. Hence we must have
  \[
  e_i = a(T_i - x_1) \;\text{ or }\;  e_i = a(T_i - x_2)
  \]
  respectively. 
\end{thm}

\begin{proof}
The expression $\delta = a(x_1+x_2)+2b$ simplifies to $\delta =
a(x_1+x_2)-2ax_1 = a(x_2-x_1)$ in the first case; the other case is
similar.
\end{proof}

Since $a \ne 0$, the linear change of variable given by $T_i \mapsto
a(T_i-x_1)$ or $T_i \mapsto a(T_i-x_2)$ is an automorphism of
$\HH_n(x_1,x_2)$. Since these choices are symmetric under the
interchange $x_1 \leftrightarrow x_2$, the only actual choice we have
is in the scalar $a \ne 0$.

Notice that $a$ cancels out if one wishes to get a projection. More
precisely, we have the following.

\begin{cor}
Keep the same hypotheses as in Theorem~\ref{t:e_i}. If we set $u_i =
\delta^{-1} e_i$, so that $u_i$ is idempotent, then we have
\[
u_i = \frac{e_i}{\delta} = \frac{a(T_i - x_1)}{a(x_2-x_1)} = \frac{T_i -
  x_1}{x_2-x_1}
\]
or 
\[
u_i = \frac{e_i}{\delta} = \frac{a(T_i - x_2)}{a(x_1-x_2)} = \frac{T_i -
  x_2}{x_1-x_2}.
\]
\end{cor}

\begin{rmk}
Prop.~2.10.7 in \cite{GHJ} describes $\HH_n(-1,q)$ by an interesting
alternative presentation in terms of the idempotents $u_1, \dots,
u_{n-1}$ as generators.
\end{rmk}

In order to obtain the algebra $\TL_n(\delta)$ as a proper quotient
algebra by a nonzero ideal, we have to impose the additional relations
\begin{equation}
e_i e_{i \pm 1} e_i - e_i = 0
\end{equation}
for any $i$ for which they make sense. The left hand sides of these
relations are cubic expressions that generate the kernel of the
(surjective) quotient map $\HH_n(x_1,x_2) \twoheadrightarrow
\TL_n(\delta)$. The kernel kills any representation indexed by a
partition of more than two parts. (According to \cite{GHJ}*{\S2.11},
this observation was made by R.~Steinberg.)


Working within the traditional version $\HH_n(-1,q)$ of the
Iwahori--Hecke algebra, which satisfies the quadratic relation $T_i^2
= (q-1)T_i + q$, Jones \cite{Jones} defines projections $u_i$ by the
identity
\[
u_i = \frac{T_i+1}{q+1} = \frac{T_i-(-1)}{q-(-1)}
\]
In that approach, the
kernel relations become
\[
u_i u_{i\pm 1} u_i = \beta u_i
\]
where $\beta^{-1} = 2+q+q^{-1} = (\pm(q^{1/2}+q^{-1/2}))^2$. This
leads to the following result.



\begin{thm}[Jones]\label{t:Jones2}
  The Jones algebra $A_n(\beta)$ is isomorphic to the quotient of
  $\HH_n(-1,q)$ by the relations
  \[
  T_i T_{i+1} T_i + T_i T_{i+1} + T_{i+1}T_i + T_i + T_{i+1} + 1 = 0
  \]
  for all $1 \le i \le n-2$. The morphism $\HH_n(-1,q)
  \twoheadrightarrow A_n(\beta)$ such that
  \[
  T_i \mapsto u_i = \frac{T_i+1}{q+1} \quad \text{for all $i$}
  \]
  is surjective, and the $u_i$ satisfy the defining relations
  \eqref{e:Jones} for $A_n(\beta)$.
\end{thm}


\begin{rmk}
(i)
Assume that a square root $q^{1/2}$ of $q$ exists in
$\Bbbk$. Combining the isomorphism $\HH_n(-q^{-1/2},q^{1/2}) \cong
\HH_n(-1,q)$ with the above result also constructs
$\TL_n(\pm(q^{1/2}+q^{-1/2}))$ as a quotient of
$\HH_n(-q^{-1/2},q^{1/2})$.

(ii) Many authors (e.g., \cite{HMR}) prefer to replace $q$ by $q^2$
and construct $\TL_n(\pm(q+q^{-1}))$ as a quotient of
$\HH_n(-q^{-1},q) \cong \HH_n(-1,q^2)$. As far as we know,
$\TL_n(\pm(q+q^{-1}))$ cannot be constructed as a quotient of the
traditional form $\HH_n(-1,q)$.
\end{rmk}



Kauffman \cite{K:90} (see also \cite{Kauffman}) constructed
$\TL_n(\delta)$ directly as a quotient of the group algebra of the
Artin braid group in type A. This of course follows from
Theorem~\ref{t:Jones2}, as $\HH_n(x_1,x_2)$ is isomorphic to the
quotient of that group algebra obtained by imposing the quadratic
relations $(T_i-x_1)(T_i-x_2)=0$, for all~$i$.



\section{Representations of $\TL_n$}\noindent
Before discussing representations, we provide a number of bijections
that underlie the combinatorics of Temperley--Lieb algebras. To start,
notice that a pair $(n-p,p)$ may be identified with a partition of at
most two parts, which in turn may be identified with its Young
diagram. The Bratteli diagram associated to Temperley--Lieb
combinatorics is the infinite graph constructed inductively as
follows:
\begin{itemize}
\item Start with the empty partition $\emptyset$ in level zero.
\item For each partition $\lambda=(\lambda_1,\lambda_2)$ in some
  level, draw a vertical edge to the partition
  $(\lambda_1+1,\lambda_2)$ and, if $\lambda_1>\lambda_2$, a diagonal
  edge to the partition $(\lambda_1,\lambda_2+1)$.
\end{itemize}
We illustrate the Bratteli diagram in Figure \ref{Bratteli}.
%%%%%%%%%%%%%%%%%%%%%%%%%%%%%%%
%% BEGIN: Bratteli Diagram
%%
% Figure environment removed
%%%%%%%%%%%%%%%%%%%%%%%%%%%%%%%%%

A $1$-\emph{factor} is a sequence of $f = (f_1, \dots, f_n)$ such that
each $f_i = \pm 1$ and the partial sums $f_1 + \cdots + f_i \ge 0$ for
all $i$. For each $i$ with $f_i=1$, let $j$ be the smallest index (if
any) for which $i < j \le n$ and $f_i + \cdots + f_j = 0$. Whenever
this happens, the indices $(i,j)$ are said to be \emph{paired};
otherwise the index $i$ is \emph{unpaired}. Here are the promised
bijections.

\begin{lem}\label{l:bijections}
  For any $n$, $p$ such that $0 \le 2p \le n$, the following sets are
  all in bijective correspondence with one another:
  \begin{enumerate}\renewcommand{\labelenumi}{(\roman{enumi})}
  \item The set of half-diagrams on $n$ vertices with $p$ links.
  \item The set of lattice walks from $(0,0)$ to $(n-p,p)$.
  \item The set of paths in the Bratteli diagram
    from $\emptyset$ to $(n-p,p)$.
  \item The set of standard tableaux of shape $(n-p,p)$.
  \item The set of $1$-factors of length $n$ with $p$ pairings.
  \end{enumerate}
\end{lem}

\begin{proof}
The bijection between the sets in (i), (ii) is Lemma
\ref{l:half-count}.  The bijection between the sets in (ii), (iii) is
obtained by matching (horizontal, vertical) segments in a lattice walk
with (vertical, diagonal) edges in a Bratteli path. The bijection
between the sets in (ii), (iv) comes from numbering each unit-length
segment in a lattice walk, in order. Write the numbers into the boxes
of a Young diagram of shape $(n-p,p)$ so that horizontal segments are
recorded in row one, and vertical segments in row two.  For instance,
the tableau
\[
\small \young(12358,467) 
\]
corresponds to the lattice walk appearing in the proof of
Lemma~\ref{l:half-count}.  Note that the numbers are entered in order
in each row from left to right; this always produces a standard
tableau. Finally, a bijection between the sets in (i), (v) is easily
obtained by matching links with paired vertices and defects with
unpaired ones. 
\end{proof}

Any of the sets in Lemma \ref{l:bijections} may be used to index the
representations of Temperley--Lieb algebras. We will use the set of
half-diagrams in part (i) as our preferred indexing set from now on.




If $h$ is a half-diagram on $n$ vertices and $d$ an $n$-diagram, we
stack $d$ above $h$ and apply the diagrammatic multiplication rule
\eqref{e:mult-rule} to obtain
\begin{equation}\label{e:half-m-rule}
  d h = \delta^N \, h'
\end{equation}
for a unique half-diagram $h'$ (obtained by discarding the loops and
identified vertices and retaining links) and some integer $N \ge 0$
(the number of discarded loops). 
For example,
\[
\begin{tikzpicture}[scale = 0.35,thick, baseline={(0,-1ex/2)}] 
\tikzstyle{vertex} = [shape = circle, minimum size = 4pt,
    inner sep = 1pt, fill=black] 
\node[vertex] (G--6) at (7.5, -1) [shape = circle, draw] {}; 
\node[vertex] (G--5) at (6.0, -1) [shape = circle, draw] {}; 
\node[vertex] (G--4) at (4.5, -1) [shape = circle, draw] {}; 
\node[vertex] (G-6) at (7.5, 1) [shape = circle, draw] {}; 
\node[vertex] (G--3) at (3.0, -1) [shape = circle, draw] {}; 
\node[vertex] (G-1) at (0.0, 1) [shape = circle, draw] {}; 
\node[vertex] (G--2) at (1.5, -1) [shape = circle, draw] {}; 
\node[vertex] (G--1) at (0.0, -1) [shape = circle, draw] {}; 
\node[vertex] (G-2) at (1.5, 1) [shape = circle, draw] {}; 
\node[vertex] (G-5) at (6.0, 1) [shape = circle, draw] {}; 
\node[vertex] (G-3) at (3.0, 1) [shape = circle, draw] {}; 
\node[vertex] (G-4) at (4.5, 1) [shape = circle, draw] {}; 
\draw[] (G--6) .. controls +(-0.5, 0.5) and +(0.5, 0.5) .. (G--5); 
\draw[] (G-6) .. controls +(-1, -1) and +(1, 1) .. (G--4); 
\draw[] (G-1) .. controls +(1, -1) and +(-1, 1) .. (G--3); 
\draw[] (G--2) .. controls +(-0.5, 0.5) and +(0.5, 0.5) .. (G--1); 
\draw[] (G-2) .. controls +(0.9, -0.9) and +(-0.9, -0.9) .. (G-5); 
\draw[] (G-3) .. controls +(0.5, -0.5) and +(-0.5, -0.5) .. (G-4); 
\end{tikzpicture} \quad \times \quad
\begin{minipage}{2.8cm}
\begin{tikzpicture}[scale = 0.35,thick, baseline={(0,-1ex/2)}] 
\tikzstyle{vertex} = [shape = circle, minimum size = 4pt,
    inner sep = 1pt, fill=black] 
\node[vertex] (G-6) at (7.5, 1) [shape = circle, draw] {}; 
\node[vertex] (G-1) at (0.0, 1) [shape = circle, draw] {}; 
\node[vertex] (G-2) at (1.5, 1) [shape = circle, draw] {}; 
\node[vertex] (G-5) at (6.0, 1) [shape = circle, draw] {}; 
\node[vertex] (G-3) at (3.0, 1) [shape = circle, draw] {}; 
\node[vertex] (G-4) at (4.5, 1) [shape = circle, draw] {};
\draw[] (G-6) -- (7.5,0);
\draw[] (G-1) -- (0,0);
\draw[] (G-2) .. controls +(0.9, -0.9) and +(-0.9, -0.9) .. (G-5); 
\draw[] (G-3) .. controls +(0.5, -0.5) and +(-0.5, -0.5) .. (G-4); 
\end{tikzpicture}
\end{minipage} \quad = \quad
\begin{minipage}{2.8cm}
\begin{tikzpicture}[scale = 0.35,thick, baseline={(0,-1ex/2)}] 
\tikzstyle{vertex} = [shape = circle, minimum size = 4pt,
    inner sep = 1pt, fill=black] 
\node[vertex] (G-6) at (7.5, 1) [shape = circle, draw] {}; 
\node[vertex] (G-1) at (0.0, 1) [shape = circle, draw] {}; 
\node[vertex] (G-2) at (1.5, 1) [shape = circle, draw] {}; 
\node[vertex] (G-5) at (6.0, 1) [shape = circle, draw] {}; 
\node[vertex] (G-3) at (3.0, 1) [shape = circle, draw] {}; 
\node[vertex] (G-4) at (4.5, 1) [shape = circle, draw] {};
\draw[] (G-1) .. controls +(1.3, -1.3) and +(-1.3, -1.3) .. (G-6);
\draw[] (G-2) .. controls +(0.9, -0.9) and +(-0.9, -0.9) .. (G-5); 
\draw[] (G-3) .. controls +(0.5, -0.5) and +(-0.5, -0.5) .. (G-4); 
\end{tikzpicture}
\end{minipage} 
\]
as one can see by considering the configuration 
\[
\begin{gathered}
\begin{tikzpicture}[scale = 0.35,thick, baseline={(0,-1ex/2)}] 
\tikzstyle{vertex} = [shape = circle, minimum size = 4pt,
    inner sep = 1pt, fill=black] 
\node[vertex] (G--6) at (7.5, -1) [shape = circle, draw] {}; 
\node[vertex] (G--5) at (6.0, -1) [shape = circle, draw] {}; 
\node[vertex] (G--4) at (4.5, -1) [shape = circle, draw] {}; 
\node[vertex] (G-6) at (7.5, 1) [shape = circle, draw] {}; 
\node[vertex] (G--3) at (3.0, -1) [shape = circle, draw] {}; 
\node[vertex] (G-1) at (0.0, 1) [shape = circle, draw] {}; 
\node[vertex] (G--2) at (1.5, -1) [shape = circle, draw] {}; 
\node[vertex] (G--1) at (0.0, -1) [shape = circle, draw] {}; 
\node[vertex] (G-2) at (1.5, 1) [shape = circle, draw] {}; 
\node[vertex] (G-5) at (6.0, 1) [shape = circle, draw] {}; 
\node[vertex] (G-3) at (3.0, 1) [shape = circle, draw] {}; 
\node[vertex] (G-4) at (4.5, 1) [shape = circle, draw] {}; 
\draw[] (G--6) .. controls +(-0.5, 0.5) and +(0.5, 0.5) .. (G--5); 
\draw[] (G-6) .. controls +(-1, -1) and +(1, 1) .. (G--4); 
\draw[] (G-1) .. controls +(1, -1) and +(-1, 1) .. (G--3); 
\draw[] (G--2) .. controls +(-0.5, 0.5) and +(0.5, 0.5) .. (G--1); 
\draw[] (G-2) .. controls +(0.9, -0.9) and +(-0.9, -0.9) .. (G-5); 
\draw[] (G-3) .. controls +(0.5, -0.5) and +(-0.5, -0.5) .. (G-4); 
\end{tikzpicture}\\
\begin{tikzpicture}[scale = 0.35,thick, baseline={(0,-1ex/2)}] 
\tikzstyle{vertex} = [shape = circle, minimum size = 4pt,
    inner sep = 1pt, fill=black] 
\node[vertex] (G-6) at (7.5, 1) [shape = circle, draw] {}; 
\node[vertex] (G-1) at (0.0, 1) [shape = circle, draw] {}; 
\node[vertex] (G-2) at (1.5, 1) [shape = circle, draw] {}; 
\node[vertex] (G-5) at (6.0, 1) [shape = circle, draw] {}; 
\node[vertex] (G-3) at (3.0, 1) [shape = circle, draw] {}; 
\node[vertex] (G-4) at (4.5, 1) [shape = circle, draw] {};
\draw[] (G-6) -- (7.5,0);
\draw[] (G-1) -- (0,0);
\draw[] (G-2) .. controls +(0.9, -0.9) and +(-0.9, -0.9) .. (G-5); 
\draw[] (G-3) .. controls +(0.5, -0.5) and +(-0.5, -0.5) .. (G-4); 
\end{tikzpicture}
\end{gathered}
\]
obtained by the usual stacking procedure. This example shows,
incidentally, that the action does not always preserve the number of
links, although the number of links in $h'$ cannot be less than that
in $h$.

Equation \eqref{e:half-m-rule} defines a left action of
$\TL_n(\delta)$ on the set of half-diagrams on $n$ vertices. Thus, the
$\Bbbk$-linear span $\hat{H}$ of that set is a $\TL_n(\delta)$-module.
It is clear that whenever $h$, $h'$ are related as in
\eqref{e:half-m-rule}, the number of links in $h'$ is always at least
that number in $h$.  Let $\hat{H}^{\ge p}$ denote the span of the
half-diagrams on $n$ vertices with at most $p$ links. Then we have a
filtration of $\TL_n(\delta)$-modules
\begin{equation}
  \hat{H} = \hat{H}^{\ge 0} \supseteq \hat{H}^{\ge 1} \supseteq \cdots
  \supseteq \hat{H}^{\ge l} \supseteq 0
\end{equation}
where $l = \lfloor n/2 \rfloor$ is the integer part of $n/2$. Set
$\hat{H}^{\ge l+1} = 0$. For each $p$ satisfying $0 \le p \le l$, let
\[
H(n-2p) := \hat{H}^{\ge p}/\hat{H}^{\ge p+1} 
\]
be the $p$th successive quotient in the filtration.  Notice that
$n-2p$ is the number of \emph{defects} in a set of coset
representatives for the quotient $\hat{H}^{\ge p}/\hat{H}^{\ge p+1}$.
If we abuse notation by denoting a coset $h+\hat{H}^{\le p+1} $ by its
chosen representative $h$, the action of an $n$-diagram $d$ on
$H(n-2p)$ is defined by:
\begin{equation}\label{e:quo-act}
d h = 
\begin{cases}
  \delta^N \, h' & \text{ if $h$, $h'$ in \eqref{e:half-m-rule} have
    the same number of links}\\ 0 & \text{ otherwise.}
\end{cases}
\end{equation}
With this convention, $H(n-2p)$ has a basis consisting of the
half-diagrams with $p$ links (equivalently, with $n-2p$ defects) and
the rule \eqref{e:quo-act} defines its $\TL_n(\delta)$-module
structure. 



Set $\Lambda(n) = \{n, n-2, \dots, n-2l\}$ where $l = \lfloor n/2
\rfloor$ as above. We have thus constructed $\TL_n(\delta)$-modules
$H(m)$ for each $m$ in the indexing set $\Lambda(n)$. This
construction works over any commutative ring $\Bbbk$, for any $\delta
\in \Bbbk$.

Graham and Lehrer \cite{GL:96} introduced axiomatics for the study of
\emph{cellular} algebras. An algebra is cellular if it has a finite
basis (a \emph{cellular basis}) which is combinatorially linked to its
representations in a certain sense. The Temperley--Lieb algebra is a
particularly simple example; its diagram basis discussed in Section
\ref{s:dia} is in fact a cellular basis. The cellular anti-involution
$*$ is given by reflecting a diagram across its axis of symmetry with
respect to the parallel lines determined by the vertices. The general
theory of cellular algebras provides a very convenient framework for
studying the $\TL_n(\delta)$-modules, especially when $\Bbbk$ is a
field.

Any cellular algebra has an associated family of \emph{cell modules},
each of which is equipped with a certain bilinear form.  The $H(m)$
for $m$ in $\Lambda(n)$ are the cell modules for $\TL_n(\delta)$. Let
$\varphi_m(-,-)$ be the associated bilinear form on $H(m)$, and set
$L(m):= H(m)/\mathrm{rad}(\varphi_m)$, for each $m$ in $\Lambda(n)$,
where
\[
\mathrm{rad}(\varphi_m) = \{h \in H(m): \varphi_m(h,h')=0, \text{ for
  all } h' \in H(m)\}.
\]
The bilinear form $\varphi_m$ may be computed diagrammatically.  If
$h$ is a half-diagram, let $h^*$ be the result of reflecting $h$
across the line containing its vertices.  Given half-diagrams $h$,
$h'$ with $m$ defects, let $h^*/h'$ be the configuration obtained by
stacking $h^*$ above $h'$. We say that $h^*/h'$ is \emph{defect
preserving} if every defect in one of the half-diagrams is connected
by a path to a defect in the other half-diagram, after corresponding
vertices are identified. Then $\varphi_n(h,h')$ is given by
\begin{equation}
  \varphi_m(h,h') = 
  \begin{cases}
    \delta^N & \text{if $h^*/ h'$ is defect preserving}\\
    0 & \text{otherwise}
  \end{cases}
\end{equation}
where $N$ is the number of loops in $h^*/h'$ (after corresponding
vertices are identified). The form $\varphi_m$ is associative:
$\varphi_m(th,h') = \varphi_m(h,t^*h')$, for any $t$ in
$\TL_n(\delta)$.

\begin{example}
If $n=6$ then we have the following, which illustrate the various
cases that can occur.
\begin{enumerate}\renewcommand{\labelenumi}{(\roman{enumi})}
\item \quad $\varphi_0\big(
\begin{tikzpicture}[scale = 0.35,thick, baseline={(0,1ex)}] 
\tikzstyle{vertex} = [shape = circle, minimum size = 4pt,
    inner sep = 1pt, fill=black] 
\node[vertex] (G-6) at (7.5, 1) [shape = circle, draw] {}; 
\node[vertex] (G-1) at (0.0, 1) [shape = circle, draw] {}; 
\node[vertex] (G-2) at (1.5, 1) [shape = circle, draw] {}; 
\node[vertex] (G-5) at (6.0, 1) [shape = circle, draw] {}; 
\node[vertex] (G-3) at (3.0, 1) [shape = circle, draw] {}; 
\node[vertex] (G-4) at (4.5, 1) [shape = circle, draw] {};
\draw[] (G-5) .. controls +(0.5, -0.5) and +(-0.5, -0.5) .. (G-6);
\draw[] (G-1) .. controls +(0.9, -0.9) and +(-0.9, -0.9) .. (G-4); 
\draw[] (G-2) .. controls +(0.5, -0.5) and +(-0.5, -0.5) .. (G-3); 
\end{tikzpicture}
\; , \;
\begin{tikzpicture}[scale = 0.35,thick, baseline={(0,1ex)}] 
\tikzstyle{vertex} = [shape = circle, minimum size = 4pt,
    inner sep = 1pt, fill=black] 
\node[vertex] (G-6) at (7.5, 1) [shape = circle, draw] {}; 
\node[vertex] (G-1) at (0.0, 1) [shape = circle, draw] {}; 
\node[vertex] (G-2) at (1.5, 1) [shape = circle, draw] {}; 
\node[vertex] (G-5) at (6.0, 1) [shape = circle, draw] {}; 
\node[vertex] (G-3) at (3.0, 1) [shape = circle, draw] {}; 
\node[vertex] (G-4) at (4.5, 1) [shape = circle, draw] {};
\draw[] (G-5) .. controls +(0.5, -0.5) and +(-0.5, -0.5) .. (G-6);
\draw[] (G-3) .. controls +(0.5, -0.5) and +(-0.5, -0.5) .. (G-4); 
\draw[] (G-1) .. controls +(0.5, -0.5) and +(-0.5, -0.5) .. (G-2); 
\end{tikzpicture}
\big) \; = \; \delta^2$.

\item \quad $\varphi_2\big(
\begin{tikzpicture}[scale = 0.35,thick, baseline={(0,1ex)}] 
\tikzstyle{vertex} = [shape = circle, minimum size = 4pt,
    inner sep = 1pt, fill=black] 
\node[vertex] (G-6) at (7.5, 1) [shape = circle, draw] {}; 
\node[vertex] (G-1) at (0.0, 1) [shape = circle, draw] {}; 
\node[vertex] (G-2) at (1.5, 1) [shape = circle, draw] {}; 
\node[vertex] (G-5) at (6.0, 1) [shape = circle, draw] {}; 
\node[vertex] (G-3) at (3.0, 1) [shape = circle, draw] {}; 
\node[vertex] (G-4) at (4.5, 1) [shape = circle, draw] {};
\draw[] (G-1) .. controls +(0.5, -0.5) and +(-0.5, -0.5) .. (G-2);
\draw[] (G-3) -- (3.0,0);
\draw[] (G-6) -- (7.5,0);
\draw[] (G-4) .. controls +(0.5, -0.5) and +(-0.5, -0.5) .. (G-5); 
\end{tikzpicture}
\; , \;
\begin{tikzpicture}[scale = 0.35,thick, baseline={(0,1ex)}] 
\tikzstyle{vertex} = [shape = circle, minimum size = 4pt,
    inner sep = 1pt, fill=black] 
\node[vertex] (G-6) at (7.5, 1) [shape = circle, draw] {}; 
\node[vertex] (G-1) at (0.0, 1) [shape = circle, draw] {}; 
\node[vertex] (G-2) at (1.5, 1) [shape = circle, draw] {}; 
\node[vertex] (G-5) at (6.0, 1) [shape = circle, draw] {}; 
\node[vertex] (G-3) at (3.0, 1) [shape = circle, draw] {}; 
\node[vertex] (G-4) at (4.5, 1) [shape = circle, draw] {};
\draw[] (G-5) -- (6.0,0);
\draw[] (G-6) -- (7.5,0);
\draw[] (G-3) .. controls +(0.5, -0.5) and +(-0.5, -0.5) .. (G-4); 
\draw[] (G-1) .. controls +(0.5, -0.5) and +(-0.5, -0.5) .. (G-2); 
\end{tikzpicture}
\big) \; = \; \delta$.

\item \quad $\varphi_2\big(
\begin{tikzpicture}[scale = 0.35,thick, baseline={(0,1ex)}] 
\tikzstyle{vertex} = [shape = circle, minimum size = 4pt,
    inner sep = 1pt, fill=black] 
\node[vertex] (G-6) at (7.5, 1) [shape = circle, draw] {}; 
\node[vertex] (G-1) at (0.0, 1) [shape = circle, draw] {}; 
\node[vertex] (G-2) at (1.5, 1) [shape = circle, draw] {}; 
\node[vertex] (G-5) at (6.0, 1) [shape = circle, draw] {}; 
\node[vertex] (G-3) at (3.0, 1) [shape = circle, draw] {}; 
\node[vertex] (G-4) at (4.5, 1) [shape = circle, draw] {};
\draw[] (G-1) .. controls +(0.5, -0.5) and +(-0.5, -0.5) .. (G-2);
\draw[] (G-3) -- (3.0,0);
\draw[] (G-4) -- (4.5,0);
\draw[] (G-5) .. controls +(0.5, -0.5) and +(-0.5, -0.5) .. (G-6); 
\end{tikzpicture}
\; , \;
\begin{tikzpicture}[scale = 0.35,thick, baseline={(0,1ex)}] 
\tikzstyle{vertex} = [shape = circle, minimum size = 4pt,
    inner sep = 1pt, fill=black] 
\node[vertex] (G-6) at (7.5, 1) [shape = circle, draw] {}; 
\node[vertex] (G-1) at (0.0, 1) [shape = circle, draw] {}; 
\node[vertex] (G-2) at (1.5, 1) [shape = circle, draw] {}; 
\node[vertex] (G-5) at (6.0, 1) [shape = circle, draw] {}; 
\node[vertex] (G-3) at (3.0, 1) [shape = circle, draw] {}; 
\node[vertex] (G-4) at (4.5, 1) [shape = circle, draw] {};
\draw[] (G-5) -- (6.0,0);
\draw[] (G-6) -- (7.5,0);
\draw[] (G-3) .. controls +(0.5, -0.5) and +(-0.5, -0.5) .. (G-4); 
\draw[] (G-1) .. controls +(0.5, -0.5) and +(-0.5, -0.5) .. (G-2); 
\end{tikzpicture}
\big) \; = \; 0$.
\end{enumerate}
One sees this by looking respectively at the three stack
configurations $h^*/ h'$
\[
\begin{gathered}
\begin{tikzpicture}[scale = 0.35,thick, baseline={(0,1ex)}] 
\tikzstyle{vertex} = [shape = circle, minimum size = 4pt,
    inner sep = 1pt, fill=black] 
\node[vertex] (G-6) at (7.5, 0) [shape = circle, draw] {}; 
\node[vertex] (G-1) at (0.0, 0) [shape = circle, draw] {}; 
\node[vertex] (G-2) at (1.5, 0) [shape = circle, draw] {}; 
\node[vertex] (G-5) at (6.0, 0) [shape = circle, draw] {}; 
\node[vertex] (G-3) at (3.0, 0) [shape = circle, draw] {}; 
\node[vertex] (G-4) at (4.5, 0) [shape = circle, draw] {};
\draw[] (G-6) .. controls +(-0.5, 0.5) and +(0.5, 0.5) .. (G-5);
\draw[] (G-4) .. controls +(-0.9, 0.9) and +(0.9, 0.9) .. (G-1); 
\draw[] (G-3) .. controls +(-0.5, 0.5) and +(0.5, 0.5) .. (G-2); 
\end{tikzpicture}
  \\
\begin{tikzpicture}[scale = 0.35,thick, baseline={(0,-1ex)}] 
\tikzstyle{vertex} = [shape = circle, minimum size = 4pt,
    inner sep = 1pt, fill=black] 
\node[vertex] (G-6) at (7.5, 0) [shape = circle, draw] {}; 
\node[vertex] (G-1) at (0.0, 0) [shape = circle, draw] {}; 
\node[vertex] (G-2) at (1.5, 0) [shape = circle, draw] {}; 
\node[vertex] (G-5) at (6.0, 0) [shape = circle, draw] {}; 
\node[vertex] (G-3) at (3.0, 0) [shape = circle, draw] {}; 
\node[vertex] (G-4) at (4.5, 0) [shape = circle, draw] {};
\draw[] (G-5) .. controls +(0.5, -0.5) and +(-0.5, -0.5) .. (G-6);
\draw[] (G-3) .. controls +(0.5, -0.5) and +(-0.5, -0.5) .. (G-4); 
\draw[] (G-1) .. controls +(0.5, -0.5) and +(-0.5, -0.5) .. (G-2); 
\end{tikzpicture}
\end{gathered}
\qquad \quad
\begin{gathered}
\begin{tikzpicture}[scale = 0.35,thick, baseline={(0,1ex)}] 
\tikzstyle{vertex} = [shape = circle, minimum size = 4pt,
    inner sep = 1pt, fill=black] 
\node[vertex] (G-6) at (7.5, 0) [shape = circle, draw] {}; 
\node[vertex] (G-1) at (0.0, 0) [shape = circle, draw] {}; 
\node[vertex] (G-2) at (1.5, 0) [shape = circle, draw] {}; 
\node[vertex] (G-5) at (6.0, 0) [shape = circle, draw] {}; 
\node[vertex] (G-3) at (3.0, 0) [shape = circle, draw] {}; 
\node[vertex] (G-4) at (4.5, 0) [shape = circle, draw] {};
\draw[] (G-2) .. controls +(-0.5, 0.5) and +(0.5, 0.5) .. (G-1);
\draw[] (G-5) .. controls +(-0.5, 0.5) and +(0.5, 0.5) .. (G-4); 
\draw[] (G-3) -- (3.0,1);
\draw[] (G-6) -- (7.5,1);
\end{tikzpicture}
  \\
\begin{tikzpicture}[scale = 0.35,thick, baseline={(0,-1ex)}] 
\tikzstyle{vertex} = [shape = circle, minimum size = 4pt,
    inner sep = 1pt, fill=black] 
\node[vertex] (G-6) at (7.5, 0) [shape = circle, draw] {}; 
\node[vertex] (G-1) at (0.0, 0) [shape = circle, draw] {}; 
\node[vertex] (G-2) at (1.5, 0) [shape = circle, draw] {}; 
\node[vertex] (G-5) at (6.0, 0) [shape = circle, draw] {}; 
\node[vertex] (G-3) at (3.0, 0) [shape = circle, draw] {}; 
\node[vertex] (G-4) at (4.5, 0) [shape = circle, draw] {};
\draw[] (G-5) -- (6.0,-1);
\draw[] (G-6) -- (7.5,-1);
\draw[] (G-3) .. controls +(0.5, -0.5) and +(-0.5, -0.5) .. (G-4); 
\draw[] (G-1) .. controls +(0.5, -0.5) and +(-0.5, -0.5) .. (G-2); 
\end{tikzpicture}
\end{gathered}
\qquad \quad
\begin{gathered}
\begin{tikzpicture}[scale = 0.35,thick, baseline={(0,1ex)}] 
\tikzstyle{vertex} = [shape = circle, minimum size = 4pt,
    inner sep = 1pt, fill=black] 
\node[vertex] (G-6) at (7.5, 0) [shape = circle, draw] {}; 
\node[vertex] (G-1) at (0.0, 0) [shape = circle, draw] {}; 
\node[vertex] (G-2) at (1.5, 0) [shape = circle, draw] {}; 
\node[vertex] (G-5) at (6.0, 0) [shape = circle, draw] {}; 
\node[vertex] (G-3) at (3.0, 0) [shape = circle, draw] {}; 
\node[vertex] (G-4) at (4.5, 0) [shape = circle, draw] {};
\draw[] (G-6) .. controls +(-0.5, 0.5) and +(0.5, 0.5) .. (G-5);
\draw[] (G-4) -- (4.5,1);
\draw[] (G-3) -- (3.0,1);
\draw[] (G-2) .. controls +(-0.5, 0.5) and +(0.5, 0.5) .. (G-1); 
\end{tikzpicture}
  \\
\begin{tikzpicture}[scale = 0.35,thick, baseline={(0,-1ex)}] 
\tikzstyle{vertex} = [shape = circle, minimum size = 4pt,
    inner sep = 1pt, fill=black] 
\node[vertex] (G-6) at (7.5, 0) [shape = circle, draw] {}; 
\node[vertex] (G-1) at (0.0, 0) [shape = circle, draw] {}; 
\node[vertex] (G-2) at (1.5, 0) [shape = circle, draw] {}; 
\node[vertex] (G-5) at (6.0, 0) [shape = circle, draw] {}; 
\node[vertex] (G-3) at (3.0, 0) [shape = circle, draw] {}; 
\node[vertex] (G-4) at (4.5, 0) [shape = circle, draw] {};
\draw[] (G-5) -- (6.0,-1);
\draw[] (G-6) -- (7.5,-1);
\draw[] (G-3) .. controls +(0.5, -0.5) and +(-0.5, -0.5) .. (G-4); 
\draw[] (G-1) .. controls +(0.5, -0.5) and +(-0.5, -0.5) .. (G-2); 
\end{tikzpicture}
\end{gathered}
\]
depicted above. Notice that the first two configurations are defect
preserving, but the third is not.
\end{example}

The following is immediate from the results of \cite{GL:96}. 

\begin{thm}
Let $\Lambda_0(n) = \{m \in \Lambda(n): \varphi_m \ne 0\}$. Suppose
that $\Bbbk$ is a field and $\delta \in \Bbbk$. Then
\[
  \{ L(m) : m \in \Lambda_0(n) \}
\]
is a complete set, up to isomorphism, of simple
$\TL_n(\delta)$-modules. Furthermore, each simple module is absolutely
simple.
\end{thm}


Over a field, a cellular algebra is semisimple if and only if all of
its cell modules are simple. Thus we obtain the following, by combining
Corollary \ref{c:ss-crit} with the above.

\begin{cor}
  Let $\Bbbk$ be a field, and suppose that $0 \ne v \in \Bbbk$ such
  that $[n_v^! \ne 0$. If $\delta = \pm(v+v^{-1}) \ne 0$ then $\{H(m): m
  \in \Lambda(n)\}$ is a complete set of simple
  $\TL_n(\delta)$-modules, up to isomorphism.
\end{cor}

When $n$ is even and $\delta=0$ the algebra $\TL_n(\delta) = \TL_n(0)$
is not semisimple, as $\varphi_0 = 0$. If $\Bbbk$ is a field, it is
known \cite{GL:96} that $\Lambda_0(n) = \Lambda(n) \setminus \{0\}$ in
this situation.

When $\TL_n(\delta)$ isn't semisimple, its representations over a
field have been understood for a long time. The blocks are known, and
the structure of the indecomposable projective modules are known as
well \cites{GHJ,Martin,GW}.



\section{Schur--Weyl duality}\noindent
Let $\Bbbk$ be a field in this section. Fix $0 \ne v$ in $\Bbbk$ and
assume that $v-v^{-1} \ne 0$.  Let $\UU = \UU(\fsl_2)$ be the
quantized enveloping algebra \cites{Lusztig,Kassel,Jantzen} associated
to the Lie algebra $\fsl_2$.  By definition, $\UU$ is the associative
algebra with $1$ generated by symbols $E$, $F$, $K^{\pm 1}$ subject to
the defining relations
\begin{equation}
\begin{gathered}
  K K^{-1} = 1 = K^{-1}K\\
  K E K^{-1} = v^2 E, \quad K F K^{-1} = v^{-2} F \\
  EF - FE = \frac{K-K^{-1}}{v-v^{-1}}.
\end{gathered}
\end{equation}
The algebra $\UU$ is a Hopf algebra, with coproduct $\Delta: \UU \to
\UU \otimes \UU$ given on generators by
\begin{equation*}\label{e:coprod}
\begin{gathered}
  \Delta(E) = E \otimes 1 + K \otimes E, \quad
  \Delta(F) = F \otimes K^{-1} + 1 \otimes F\\
  \Delta(K) = K \otimes K .
\end{gathered}
\end{equation*}
The counit $\epsilon: \UU \to \Bbbk$ is given by
\begin{equation*}
     \epsilon(E) = \epsilon(F) = 0, \quad \epsilon(K) = 1. 
\end{equation*}
We omit the definition of the antipode. 



If $v$ is not a root of unity, let $V(n)$ be the simple $\UU$-module
(of type $\mathbf{1}$) of dimension $n+1$; see \cite{Jantzen}*{2.6}.
Let $V= V(1)$ be the natural (or ``vector'') module. Then $\UU$ acts
on $V^{\otimes n}$ by means of the iterated coproduct
$\Delta^{(n)}$, defined inductively by
\[
\Delta^{(2)} = \Delta, \qquad
\Delta^{(k+1)} = (\Delta \otimes 1^{\otimes (k-1)}) \Delta^{(k)}.
\]
Thus, $V^{\otimes n}$ is a $\UU$-module.
\begin{thm}\label{t:SWD}
Let $\Bbbk$ be a field, and $0 \ne v \in \Bbbk$. Assume that $v$ is
not a root of unity. Let $\delta = \pm(v+v^{-1})$. Then there is an
action of $\TL_n(\delta)$ on $V^{\otimes n}$ that commutes with the
action of $\UU$, and these commuting actions induce algebra
surjections
\[
\UU \to \End_{\TL_n(\delta)}(V^{\otimes n}), \quad
\TL_n(\delta) \to \End_{\UU}(V^{\otimes n})
\]
the second of which is actually an isomorphism.  Furthermore,
\[
V^{\otimes n} \cong \textstyle \bigoplus_{k \in \Lambda(n)} V(k) \otimes H(k)
\]
is a decomposition into simple $\UU \otimes \TL_n(\delta)$-modules,
where
\[
\Lambda(n) = \{n, n-2, \dots, n-2l\}, \quad l = \lfloor n/2
\rfloor.
\]
\end{thm}

If $v$ is transcendental, this may be deduced from Jimbo \cite{Jimbo}
with some work, using the results of Section \ref{s:Hecke}. In Jimbo's
result, the algebra $\HH_n(v^{-1},-v)$ acts (usually non-faithfully)
on $V^{\otimes n}$, and by passing to the corresponding quotient by
the kernel of that action one obtains a faithful action of
$\TL_n(\delta)$.

Jimbo's version of Schur--Weyl duality was generalized in
\cites{Martin:92,DPS:98}; it even works at roots of unity. Thus, the
first part of the above result holds more generally as well.

\begin{rmk}
If $[n]_v^! \ne 0$ and $\delta = \pm(v+v^{-1})\ne 0$, the paper
\cite{DG:orthog} gives a stronger version of semisimple Schur--Weyl
duality in which $\UU$ is replaced by the rank one $q$-Schur algebra
in degree $n$ (which acts faithfully).
\end{rmk}



%%%%%%%%%%%%%%%%%%%%%%%%%%%%%%%%%%%%%%%%%%%%%%%%%%%%%%%%%%%%%
% bibliography using amsrefs package 
%%%%%%%%%%%%%%%%%%%%%%%%%%%%%%%%%%%%%%%%%%%%%%%%%%%%%%%%%%%%%
\begin{bibdiv}
  \begin{biblist}
    
\bib{AST}{article}{
   author={Andersen, Henning Haahr},
   author={Stroppel, Catharina},
   author={Tubbenhauer, Daniel},
   title={Cellular structures using $U_q$-tilting modules},
   journal={Pacific J. Math.},
   volume={292},
   date={2018},
   number={1},
   pages={21--59},
%   issn={0030-8730},
%   review={\MR{3708257}},
%   doi={10.2140/pjm.2018.292.21},
}


\bib{Baxter}{book}{
   author={Baxter, Rodney J.},
   title={Exactly solved models in statistical mechanics},
   publisher={Academic Press, Inc. [Harcourt Brace Jovanovich, Publishers],
   London},
   date={1982},
%   pages={xii+486},
%   isbn={0-12-083180-5},
%   review={\MR{690578}},
}
  
\bib{Benkart-Moon}{article}{
   author={Benkart, Georgia},
   author={Moon, Dongho},
   title={Tensor product representations of Temperley--Lieb algebras and
   Chebyshev polynomials},
   conference={
      title={Representations of algebras and related topics},
   },
   book={
      series={Fields Inst. Commun.},
      volume={45},
      publisher={Amer. Math. Soc., Providence, RI},
   },
   date={2005},
   pages={57--80},
%   review={\MR{2146240}},
}

    

\bib{BW}{article}{
  author={Benkart, Georgia}, author={Witherspoon, Sarah},
  title={Representations of two-parameter quantum groups and
    Schur--Weyl duality},
  conference={ title={Hopf algebras}, },
  book={
    series={Lecture Notes in Pure and Appl. Math.},
    volume={237},
    publisher={Dekker, New York},
  },
  date={2004},
  pages={65--92},
%   review={\MR{2051731}},
}

\bib{BFK}{article}{
   author={Bernstein, Joseph},
   author={Frenkel, Igor},
   author={Khovanov, Mikhail},
   title={A categorification of the Temperley-Lieb algebra and Schur
   quotients of $U(\germ{sl}_2)$ via projective and Zuckerman functors},
   journal={Selecta Math. (N.S.)},
   volume={5},
   date={1999},
   number={2},
   pages={199--241},
%   issn={1022-1824},
%   review={\MR{1714141}},
%   doi={10.1007/s000290050047},
}


\bib{Bigelow}{article}{
  author={Bigelow, Stephen},
  title={Braid groups and Iwahori-Hecke algebras},
  conference={ title={Problems on mapping class groups and related topics},
  },
  book={
    series={Proc. Sympos. Pure Math.},
    volume={74},
    publisher={Amer. Math. Soc., Providence, RI}, },
  date={2006},
  pages={285--299},
 %  review={\MR{2264547}},
 %  doi={10.1090/pspum/074/2264547},
}


\bib{Birman-Wenzl}{article}{
   author={Birman, Joan S.},
   author={Wenzl, Hans},
   title={Braids, link polynomials and a new algebra},
   journal={Trans. Amer. Math. Soc.},
   volume={313},
   date={1989},
   number={1},
   pages={249--273},
%   issn={0002-9947},
%   review={\MR{992598}},
%   doi={10.2307/2001074},
}

\bib{Brauer}{article}{
   author={Brauer, Richard},
   title={On algebras which are connected with the semisimple continuous
   groups},
   journal={Ann. of Math. (2)},
   volume={38},
   date={1937},
   number={4},
   pages={857--872},
%   issn={0003-486X},
%   review={\MR{1503378}},
%   doi={10.2307/1968843},
}

\bib{DJ1}{article}{
   author={Dipper, Richard},
   author={James, Gordon},
   title={Representations of Hecke algebras of general linear groups},
   journal={Proc. London Math. Soc. (3)},
   volume={52},
   date={1986},
   number={1},
   pages={20--52},
%   issn={0024-6115},
%   review={\MR{812444}},
%   doi={10.1112/plms/s3-52.1.20},
}


\bib{DJ2}{article}{
   author={Dipper, Richard},
   author={James, Gordon},
   title={Blocks and idempotents of Hecke algebras of general linear groups},
   journal={Proc. London Math. Soc. (3)},
   volume={54},
   date={1987},
   number={1},
   pages={57--82},
%   issn={0024-6115},
%   review={\MR{872250}},
%   doi={10.1112/plms/s3-54.1.57},
}

		
\bib{DJ3}{article}{
   author={Dipper, Richard},
   author={James, Gordon},
   title={The $q$-Schur algebra},
   journal={Proc. London Math. Soc. (3)},
   volume={59},
   date={1989},
   number={1},
   pages={23--50},
%   issn={0024-6115},
%   review={\MR{997250}},
%   doi={10.1112/plms/s3-59.1.23},
}
		

		
\bib{DJ4}{article}{
   author={Dipper, Richard},
   author={James, Gordon},
   title={$q$-tensor space and $q$-Weyl modules},
   journal={Trans. Amer. Math. Soc.},
   volume={327},
   date={1991},
   number={1},
   pages={251--282},
%   issn={0002-9947},
%   review={\MR{1012527}},
%   doi={10.2307/2001842},
}
		

\bib{DG:PTL}{article}{
   author={Doty, Stephen},
   author={Giaquinto, Anthony},
   title={The partial Temperley--Lieb algebra and its representations},
%  eprint={https://arXiv.org/abs/2208.04296},
   status={preprint (to appear, J. of Combinatorial Algebra), %
   \href{https://arXiv.org/abs/2208.04296}{arXiv:2208.04296}},
   year={2022},
}

\bib{DG:orthog}{article}{
   author={Doty, Stephen},
   author={Giaquinto, Anthony},
   title={An orthogonal realization of representations of the
     Temperley--Lieb algebra},
   status={preprint, %
     \href{https://arXiv.org/abs/2306.17352}{arXiv:2306.17352}},
   year={2023},
}

\bib{DPS:98}{article}{
   author={Du, Jie},
   author={Parshall, Brian},
   author={Scott, Leonard},
   title={Quantum Weyl reciprocity and tilting modules},
   journal={Comm. Math. Phys.},
   volume={195},
   date={1998},
   number={2},
   pages={321--352},
%   issn={0010-3616},
%   review={\MR{1637785}},
%   doi={10.1007/s002200050392},
}

\bib{EK}{article}{
   author={Evans, David E.},
   author={Kawahigashi, Yasuyuki},
   title={Subfactors and mathematical physics},
   status={preprint, %
     \href{https://arXiv.org/abs/2303.04459}{arXiv:2303.04459}},
   year={2023},
}

\bib{FKS}{article}{
   author={Frenkel, Igor},
   author={Khovanov, Mikhail},
   author={Stroppel, Catharina},
   title={A categorification of finite-dimensional irreducible
   representations of quantum $\germ{sl}_2$ and their tensor products},
   journal={Selecta Math. (N.S.)},
   volume={12},
   date={2006},
   number={3-4},
   pages={379--431},
%   issn={1022-1824},
%   review={\MR{2305608}},
%   doi={10.1007/s00029-007-0031-y},
}


%\bib{FP:20}{article}{
%   author={Flores, Steven M.},
%   author={Peltola, Evalina},
%   title={Higher-spin quantum and classical Schur--Weyl duality
%     for $\mathfrak{sl}_2$},
%   eprint={arXiv:2008.06038},
%   status={preprint},
%   year={2020},
%}

\bib{GL:96}{article}{
   author={Graham, J. J.},
   author={Lehrer, G. I.},
   title={Cellular algebras},
   journal={Invent. Math.},
   volume={123},
   date={1996},
   number={1},
   pages={1--34},
%   issn={0020-9910},
%   review={\MR{1376244}},
%   doi={10.1007/BF01232365},
}
	

\bib{GHJ}{book}{
   author={Goodman, Frederick M.},
   author={de la Harpe, Pierre},
   author={Jones, Vaughan F. R.},
   title={Coxeter graphs and towers of algebras},
   series={Mathematical Sciences Research Institute Publications},
   volume={14},
   publisher={Springer-Verlag, New York},
   date={1989},
%   pages={x+288},
%   isbn={0-387-96979-9},
%   review={\MR{999799}},
%   doi={10.1007/978-1-4613-9641-3},
}

%\bib{GV}{article}{
%   author={Gainutdinov, A. M.},
%   author={Vasseur, R.},
%   title={Lattice fusion rules and logarithmic operator product expansions},
%   journal={Nuclear Phys. B},
%   volume={868},
%   date={2013},
%   number={1},
%   pages={223--270},
%   issn={0550-3213},
%   review={\MR{3001127}},
%   doi={10.1016/j.nuclphysb.2012.11.004},
%}

\bib{GW}{article}{
   author={Goodman, Frederick M.},
   author={Wenzl, Hans},
   title={The Temperley-Lieb algebra at roots of unity},
   journal={Pacific J. Math.},
   volume={161},
   date={1993},
   number={2},
   pages={307--334},
%   issn={0030-8730},
%   review={\MR{1242201}},
}
	
    
\bib{HMR}{article}{
   author={Halverson, Tom},
   author={Mazzocco, Manuela},
   author={Ram, Arun},
   title={Commuting families in Hecke and Temperley--Lieb algebras},
   journal={Nagoya Math. J.},
   volume={195},
   date={2009},
   pages={125--152},
%   issn={0027-7630},
%   review={\MR{2552957}},
%   doi={10.1017/S0027763000009740},
}

\bib{Jantzen}{book}{
   author={Jantzen, Jens Carsten},
   title={Lectures on quantum groups},
   series={Graduate Studies in Mathematics},
   volume={6},
   publisher={American Mathematical Society, Providence, RI},
   date={1996},
%   pages={viii+266},
%   isbn={0-8218-0478-2},
%   review={\MR{1359532}},
%   doi={10.1090/gsm/006},
}

\bib{Jimbo}{article}{
   author={Jimbo, Michio},
   title={A $q$-analogue of $U({\germ g}{\germ l}(N+1))$, Hecke algebra,
     and the Yang--Baxter equation},
   journal={Lett. Math. Phys.},
   volume={11},
   date={1986},
   number={3},
   pages={247--252},
%   issn={0377-9017},
%   review={\MR{841713}},
%   doi={10.1007/BF00400222},
}

\bib{Jones:83}{article}{
   author={Jones, V. F. R.},
   title={Index for subfactors},
   journal={Invent. Math.},
   volume={72},
   date={1983},
   number={1},
   pages={1--25},
%   issn={0020-9910},
%   review={\MR{696688}},
%   doi={10.1007/BF01389127},
}

\bib{Jones:85}{article}{
   author={Jones, V. F. R.},
   title={A polynomial invariant for knots via von Neumann algebras},
   journal={Bull. Amer. Math. Soc. (N.S.)},
   volume={12},
   date={1985},
   number={1},
   pages={103--111},
%   issn={0273-0979},
%   review={\MR{766964}},
%   doi={10.1090/S0273-0979-1985-15304-2},
}


\bib{Jones:86}{article}{
   author={Jones, V. F. R.},
   title={Braid groups, Hecke algebras and type ${\rm II}_1$ factors},
   conference={
      title={Geometric methods in operator algebras},
      address={Kyoto},
      date={1983},
   },
   book={
      series={Pitman Res. Notes Math. Ser.},
      volume={123},
      publisher={Longman Sci. Tech., Harlow},
   },
   date={1986},
   pages={242--273},
%   review={\MR{866500}},
}

\bib{Jones}{article}{
   author={Jones, V. F. R.},
   title={Hecke algebra representations of braid groups and link
   polynomials},
   journal={Ann. of Math. (2)},
   volume={126},
   date={1987},
   number={2},
   pages={335--388},
%   issn={0003-486X},
%   review={\MR{908150}},
%   doi={10.2307/1971403},
}

\bib{Jones:91}{book}{
   author={Jones, Vaughan F. R.},
   title={Subfactors and knots},
   series={CBMS Regional Conference Series in Mathematics},
   volume={80},
   publisher={%Published for the Conference Board of the Mathematical
     %Sciences, Washington, DC; by the
     American Mathematical Society, Providence, RI},
   date={1991},
%   pages={x+113},
%   isbn={0-8218-0729-3},
%   review={\MR{1134131}},
%   doi={10.1090/cbms/080},
}


\bib{Kassel}{book}{
   author={Kassel, Christian},
   title={Quantum groups},
   series={Graduate Texts in Mathematics},
   volume={155},
   publisher={Springer-Verlag, New York},
   date={1995},
%   pages={xii+531},
%   isbn={0-387-94370-6},
%   review={\MR{1321145}},
%   doi={10.1007/978-1-4612-0783-2},
}

\bib{K:87}{article}{
   author={Kauffman, Louis H.},
   title={State models and the Jones polynomial},
   journal={Topology},
   volume={26},
   date={1987},
   number={3},
   pages={395--407},
%   issn={0040-9383},
%   review={\MR{899057}},
%   doi={10.1016/0040-9383(87)90009-7},
}

\bib{K:88}{article}{
   author={Kauffman, Louis H.},
   title={Statistical mechanics and the Jones polynomial},
   conference={
      title={Braids},
      address={Santa Cruz, CA},
      date={1986},
   },
   book={
      series={Contemp. Math.},
      volume={78},
      publisher={Amer. Math. Soc., Providence, RI},
   },
   date={1988},
   pages={263--297},
%   review={\MR{975085}},
%   doi={10.1090/conm/078/975085},
}
		  

\bib{K:90}{article}{
   author={Kauffman, Louis H.},
   title={An invariant of regular isotopy},
   journal={Trans. Amer. Math. Soc.},
   volume={318},
   date={1990},
   number={2},
   pages={417--471},
%   issn={0002-9947},
%   review={\MR{958895}},
%   doi={10.2307/2001315},
}
	


\bib{Kauffman}{book}{
   author={Kauffman, Louis H.},
   title={Knots and physics},
   series={Series on Knots and Everything},
   volume={53},
   edition={4},
   publisher={World Scientific Publishing Co. Pte. Ltd., Hackensack, NJ},
   date={2013},
%   pages={xviii+846},
%   isbn={978-981-4383-01-1},
%   review={\MR{3013186}},
%   doi={10.1142/8338},
}

\bib{Lusztig}{book}{
   author={Lusztig, George},
   title={Introduction to quantum groups},
   series={Progress in Mathematics},
   volume={110},
   publisher={Birkh\"{a}user Boston, Inc., Boston, MA},
   date={1993},
%   pages={xii+341},
%   isbn={0-8176-3712-5},
%   review={\MR{1227098}},
}


%\bib{Kauffman:survey}{article}{
%   author={Kauffman, L. H.},
%   title={Combinatorial knot theory and the Jones polynomial},
%   journal={J. Knot Theory Ramifications},
%   volume={31},
%   date={2023},
%   number={11},
%   pages={Paper No. 2340011, 49},
%   issn={},
%   review={},
%   doi={10.1142/S0218216523400114},
%}


\bib{Martin}{book}{
   author={Martin, Paul Purdon},
   title={Potts models and related problems in statistical mechanics},
   series={Series on Advances in Statistical Mechanics},
   volume={5},
   publisher={World Scientific Publishing Co., Inc., Teaneck, NJ},
   date={1991},
%   pages={xiv+344},
%   isbn={981-02-0075-7},
%   review={\MR{1103994}},
%   doi={10.1142/0983},
}


\bib{Martin:92}{article}{
   author={Martin, Paul Purdon},
   title={On Schur-Weyl duality, $A_n$ Hecke algebras and quantum ${\rm
   sl}(N)$ on $\bigotimes^{n+1}{\bf C}^N$},
   conference={
      title={Infinite analysis, Part A, B},
      address={Kyoto},
      date={1991},
   },
   book={
      series={Adv. Ser. Math. Phys.},
      volume={16},
      publisher={World Sci. Publ., River Edge, NJ},
   },
   date={1992},
   pages={645--673},
%   review={\MR{1187568}},
%   doi={10.1142/S0217751X92003975},
}
	


		
\bib{Murphy1}{article}{
   author={Murphy, G. E.},
   title={On the representation theory of the symmetric groups and
   associated Hecke algebras},
   journal={J. Algebra},
   volume={152},
   date={1992},
   number={2},
   pages={492--513},
%   issn={0021-8693},
%   review={\MR{1194316}},
%   doi={10.1016/0021-8693(92)90045-N},
}

\bib{Murphy2}{article}{
   author={Murphy, G. E.},
   title={The representations of Hecke algebras of type $A_n$},
   journal={J. Algebra},
   volume={173},
   date={1995},
   number={1},
   pages={97--121},
%   issn={0021-8693},
%   review={\MR{1327362}},
%   doi={10.1006/jabr.1995.1079},
}

%\bib{P-SA}{article}{
%   author={Provencher, Guillaume},
%   author={Saint-Aubin, Yvan},
%   title={The idempotents of the ${\rm TL}_n$-module $\bigotimes^n\Bbb{C}2$
%   in terms of elements of ${\rm U}_q\germ{sl}_2$},
%   journal={Ann. Henri Poincar\'{e}},
%   volume={15},
%   date={2014},
%   number={11},
%   pages={2203--2240},
%   issn={1424-0637},
%   review={\MR{3268828}},
%   doi={10.1007/s00023-013-0297-x},
%}


\bib{Ridout-StAubin}{article}{
   author={Ridout, David},
   author={Saint-Aubin, Yvan},
   title={Standard modules, induction and the structure of the
   Temperley-Lieb algebra},
   journal={Adv. Theor. Math. Phys.},
   volume={18},
   date={2014},
   number={5},
   pages={957--1041},
%   issn={1095-0761},
%   review={\MR{3281274}},
}


\bib{Stroppel}{article}{
   author={Stroppel, Catharina},
   title={Categorification of the Temperley-Lieb category, tangles, and
   cobordisms via projective functors},
   journal={Duke Math. J.},
   volume={126},
   date={2005},
   number={3},
   pages={547--596},
%   issn={0012-7094},
%   review={\MR{2120117}},
%   doi={10.1215/S0012-7094-04-12634-X},
}

\bib{Takeuchi}{article}{
   author={Takeuchi, Mitsuhiro},
   title={A two-parameter quantization of ${\rm GL}(n)$ (summary)},
   journal={Proc. Japan Acad. Ser. A Math. Sci.},
   volume={66},
   date={1990},
   number={5},
   pages={112--114},
%   issn={0386-2194},
%   review={\MR{1065785}},
}


\bib{TL}{article}{
   author={Temperley, H. N. V.},
   author={Lieb, E. H.},
   title={Relations between the ``percolation'' and ``colouring'' problem
   and other graph-theoretical problems associated with regular planar
   lattices: some exact results for the ``percolation'' problem},
   journal={Proc. Roy. Soc. London Ser. A},
   volume={322},
   date={1971},
   number={1549},
   pages={251--280},
%   issn={0962-8444},
%   review={\MR{498284}},
%   doi={10.1098/rspa.1971.0067},
}

	

\bib{Westbury}{article}{
   author={Westbury, B. W.},
   title={The representation theory of the Temperley-Lieb algebras},
   journal={Math. Z.},
   volume={219},
   date={1995},
   number={4},
   pages={539--565},
%   issn={0025-5874},
%   review={\MR{1343661}},
%   doi={10.1007/BF02572380},
}


\bib{Wilcox}{article}{
   author={Wilcox, Stewart},
   title={Cellularity of diagram algebras as twisted semigroup algebras},
   journal={J. Algebra},
   volume={309},
   date={2007},
   number={1},
   pages={10--31},
%   issn={0021-8693},
%   review={\MR{2301230}},
%   doi={10.1016/j.jalgebra.2006.10.016},
}

\end{biblist}
\end{bibdiv}
\end{document}

%%%%%%%%%%%%%%%%%%%%%%%%%%%%%%%%%%%%%%%%%%%%%%%%%%%%%%%%%%%%%%%%%





