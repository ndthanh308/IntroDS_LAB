\documentclass[11pt]{amsart}
\usepackage{amssymb}
\usepackage{tikz,tikz-cd}
\usetikzlibrary{intersections}
\usepackage{youngtab}
\usepackage[T1]{fontenc}\usepackage{palatino} % better font
\usepackage[mathscr]{euscript}
\usepackage{stmaryrd} % gives double bracket symbols
\usepackage[pagebackref]{hyperref} % this makes life easier for readers
\hypersetup{colorlinks,linkcolor=blue,urlcolor=blue,citecolor=blue,%
  linktocpage}  % options for hyperref (pagebackref MUST be loaded earlier)
\usepackage[alphabetic,abbrev,nobysame]{amsrefs} % load AFTER hyperref! 

%%%%%%%%%%%%%%%%%%%%%%%%%%%%%%%%%%%%%%%%%%%%%%%%%%%%%%%%%%%%
%% BEGIN: Partition-drawing code swiped from Zajj Daugherty
%% For partitions. Takes in a list of row lengths. 

\newcommand{\Part}[1]{
 \foreach \x [count=\s from 1] in {#1}{
 	{\ifnum\s=1
		\draw (0,\s-1)--(\x,\s-1); 
		\fi}
   \draw (0,\s) to (\x,\s);
   \foreach \y in {0, ..., \x} {\draw (\y,\s)--(\y,\s-1);}
 }}

\def\UNIT{.18} \newcommand{\PART}[1]{
\begin{tikzpicture}[xscale=\UNIT, yscale=-\UNIT] 
	\Part{#1}
\end{tikzpicture}
}
%% END: Partition-drawing code
%%%%%%%%%%%%%%%%%%%%%%%%%%%%%%%%%%%%%%%%%%%%%%%%%%%%%%%%%%%%
\newcommand{\epic}{
\begin{tikzpicture}[scale = 0.35,thick, baseline={(0,-1ex/2)}]
\tikzstyle{vertex} = [shape = circle, minimum size = 4pt, inner sep = 1pt] 
\node[vertex] (G--2) at (1.5, -1) [shape = circle, draw,fill=black] {}; 
\node[vertex] (G--1) at (0.0, -1) [shape = circle, draw,fill=black] {}; 
\node[vertex] (G-1) at (0.0, 1) [shape = circle, draw,fill=black] {}; 
\node[vertex] (G-2) at (1.5, 1) [shape = circle, draw,fill=black] {}; 
\draw (G--2) .. controls +(-0.5, 0.5) and +(0.5, 0.5) .. (G--1); 
\draw (G-1) .. controls +(0.5, -0.5) and +(-0.5, -0.5) .. (G-2); 
\end{tikzpicture} 
}

\newcommand{\onepic}{
\begin{tikzpicture}[scale = 0.35,thick, baseline={(0,-1ex/2)}] 
\tikzstyle{vertex} = [shape = circle, minimum size = 4pt, inner sep = 1pt] 
\node[vertex] (G--1) at (0.0, -1) [shape = circle, draw,fill=black] {}; 
\node[vertex] (G-1) at (0.0, 1) [shape = circle, draw,fill=black] {}; 
\draw (G-1) .. controls +(0, -1) and +(0, 1) .. (G--1); 
\end{tikzpicture}
}

%%%%%%%%%%%%%%%%%%%%
% theorems and such
%%%%%%%%%%%%%%%%%%%%
\swapnumbers % theorem numbers on the LEFT! 
\newtheorem{thm}{Theorem}[section]
\newtheorem*{thm*}{Theorem}
\newtheorem{lem}[thm]{Lemma}
\newtheorem*{lem*}{Lemma}
\newtheorem{prop}[thm]{Proposition}
\newtheorem*{prop*}{Proposition}
\newtheorem{cor}[thm]{Corollary}
\newtheorem*{cor*}{Corollary}
\newtheorem{conj}[thm]{Conjecture}
\newtheorem*{conj*}{Conjecture}

\theoremstyle{definition}
\newtheorem{defn}[thm]{Definition}
\newtheorem*{defn*}{Definition}
\newtheorem{example}[thm]{Example}
\newtheorem*{example*}{Example}
\newtheorem{rmk}[thm]{Remark}
\newtheorem*{rmk*}{Remark}
\newtheorem{que}{Question}
\newtheorem*{que*}{Question}

%%%%%%%%%
% macros
%%%%%%%%%
\newcommand{\C}{\mathbb{C}} % complex numbers
\newcommand{\Q}{\mathbb{Q}} % rationals
\newcommand{\Z}{\mathbb{Z}} % integers
\newcommand{\N}{\mathbb{N}} % natural numbers
\newcommand{\K}{\mathbb{K}} % general field
\newcommand{\UU}{\mathbf{U}} % bold U
\newcommand{\A}{\mathscr{A}} % Jones algebra
\newcommand{\TL}{\operatorname{TL}} % Temperley-Lieb alg
\newcommand{\End}{\operatorname{End}} % endomorphisms
\newcommand{\Hom}{\operatorname{Hom}} % homomorphisms
\newcommand{\gl}{\mathfrak{gl}} % general linear Lie alg
\newcommand{\fsl}{\mathfrak{sl}} % special linear Lie alg
\newcommand{\ov}{\overline} % overline short form
\newcommand{\Schur}{\mathbf{S}} % Schur algebra
\newcommand{\bil}[2]{\langle #1, #2 \rangle}
\newcommand{\qbinom}[2]{\genfrac{[}{]}{0pt}{}{#1}{#2}}
\newcommand{\M}{\mathscr{M}} % maximal vectors
\newcommand{\D}{\mathscr{D}} % diagram algebra
\newcommand{\Cat}{\operatorname{LW}} % no of increasing lattice walks
\newcommand{\Tab}{\operatorname{Tab}} % tableaux
\newcommand{\HH}{\mathbf{H}} % Hecke algebra
\newcommand{\Sym}{\mathfrak{S}} % symmetric group
\newcommand{\onto}{\twoheadrightarrow}
\newcommand{\into}{\hookrightarrow}
\newcommand{\dbracket}[1]{\llbracket #1 \rrbracket} % double brackets
\newcommand{\stirlingii}{\genfrac{\{}{\}}{0pt}{}}
\newcommand{\tr}{\operatorname{tr}} % trace
\newcommand{\Tr}{\operatorname{Tr}} % Markov trace
\newcommand{\E}{\mathcal{E}} % conditional expectation
\newcommand{\ZZ}{\mathcal{Z}} % Laurent polynomials

\renewcommand{\labelenumi}{(\alph{enumi})}
\parskip=2pt

\allowdisplaybreaks

\title[Origins of the Temperley--Lieb algebra]%
      {Origins of the Temperley--Lieb algebra: early history}
\author{Stephen Doty}
\email{doty@math.luc.edu, tonyg@math.luc.edu}
\author{Anthony Giaquinto}
%\email{tonyg@math.luc.edu}
\address{\parbox{\linewidth}{Department of Mathematics and Statistics,
  Loyola University Chicago,\\ Chicago, IL 60660 USA}}
\subjclass{Primary 16T30, 16G99, 17B37}
\keywords{Diagram algebras, Schur--Weyl duality,
  Temperley--Lieb algebras, quantized enveloping algebras}
\begin{document}
\begin{abstract}\noindent
We give an historical survey of some of the original basic algebraic
and combinatorial results on Temperley--Lieb algebras, with a focus on
certain results that have become folklore.
\end{abstract}
\maketitle
\tableofcontents

\section*{Introduction}\noindent
The Temperley--Lieb algebra $\TL_n(\delta)$ was was introduced in
\cite{TL} in connection with certain problems in mathematical physics.
It reappeared in the 1980s as a certain von Neumann algebra in the
spectacular work of Vaughan Jones
\cites{Jones:83,Jones:85,Jones:86,Jones,Jones:91} on subfactors and
knots.  Kauffman \cites{K:87,K:88,K:90} (see also \cite{Kauffman})
realized it as a diagram algebra. Birman and Wenzl \cite{Birman-Wenzl}
showed that it is isomorphic to a subalgebra of the Brauer algebra
\cite{Brauer}; this also follows from Kauffman's results.

This paper is an historical survey of the most fundamental algebraic
and combinatorial results on these algebras and their
representations. When writing \cites{DG:PTL,DG:orthog}, we found it
challenging to track down original references for various basic
results that have become folklore. The purpose of this paper is to
document what we found, hopefully aiding subsequent researchers
working in this area. Our focus is somewhat different from that of the
excellent survey article \cite{Ridout-StAubin}. Furthermore, we wish
to draw attention to the book \cite{GHJ}, a reference which seems to
be almost universally ignored by authors writing papers in this area.
Since our focus is on early history, we do not attempt to survey more
recent important developments such as categorification
\cites{BFK,Stroppel,FKS} or the many generalizations of
Temperley--Lieb algebras that exist in abundance in the literature.

In the early papers the ground field was always the complex field
$\C$. The book \cite{GHJ} replaced $\C$ by an arbitrary field
$\K$, which is sometimes assumed to be of characteristic
zero. Many basic properties of Temperley--Lieb algebras are valid even
more generally, where the field $\K$ is replaced by an
arbitrary (unital) commutative ring, and we work in that context
whenever possible.


\subsection*{Acknowledgments}
The authors thank Fred Goodman and an anonymous referee for
helpful comments and suggestions on a previous version of this paper.


\section{The Temperley--Lieb algebra}\label{s:1}\noindent
In this section $\Bbbk$ is a commutative ring with $1$. The
Temperley--Lieb algebra appeared originally in \cite{TL} in connection
with the Potts model in mathematical physics. For any positive integer
$n$ and any element $\delta$ in the ground ring $\Bbbk$,
$\TL_n(\delta)$ is the unital $\Bbbk$-algebra defined by generators
$e_1, \dots, e_{n-1}$ subject to the relations
\begin{equation}\label{e:TL}
  \begin{aligned}
  e_i^2 = \delta e_i, \quad
  e_i e_j e_i = e_i \text{ if } |i-j|=1, \quad
  e_i e_j = e_j e_i \text{ if } |i-j|>1 .
\end{aligned}
\end{equation}
The unit element $1$ of the algebra is identified with the empty
product of generators.  There is an algebra isomorphism $\TL_n(\delta)
\cong \TL_n(-\delta)$ defined on generators by $e_i \mapsto -e_i$.

\begin{rmk}
The above description by generators and relations is not explicitly
given in \cite{TL}, but can be found, for instance, in the book
\cite{Baxter}.
\end{rmk}

Our first task is to show that $\TL_n = \TL_n(\delta)$ has a finite
spanning set over $\Bbbk$ (hence is finite-dimensional if $\Bbbk$ is a
field). We follow an argument sketched in Jones \cite{Jones:83}.
Words $w$, $w'$ in the generators $e_1, \dots, e_{n-1}$ are
\emph{equivalent} (written as $w_1 \sim w_2$) if they are equal up to
a factor which is a power of $\delta$.  Say that a word $w = e_{i_1}
\cdots e_{i_l}$ in $\TL_n$ is \emph{reduced} if it has minimal
possible length in its equivalence class.

%Say that a word $w = e_{i_1} \cdots e_{i_l}$ in $\TL_n$ is
%\emph{reduced} if it has minimal possible length with respect to the
%shortening rules: $e_ie_{i\pm 1}e_i \to e_i$, $e_ie_j=e_je_i$ if
%$|i-j|>1$, $e_i^2 \to e_i$.

\begin{lem}[Jones' Lemma]
  If $w = e_{i_1} \cdots e_{i_l}$ is a reduced word in $\TL_n$ then $m
  := \max\{i_1, \dots, i_l\}$ occurs only once in the sequence $(i_1,
  \dots, i_l)$.
\end{lem}


\begin{proof}
This is proved by induction on the length. The base case is trivial.
Let $w$ be a reduced word. Suppose for contradiction that $e_m$
appears at least twice in $w$, where $m$ is the maximal index that
appears. Then $w = w_1e_mw_2e_mw_3$. We may assume that $w_2$ does not
contain $e_m$.

If $w_2$ does not contain $e_{m-1}$ then $e_m$ commutes with all the
$e_i$ appearing in $w_2$, so after commuting the rightmost $e_m$ to
the left of $w_2$, the length of $w$ can be shortened using the
equivalence $e_m^2 \sim e_m$. Contradiction.

The remaining case is that $w_2$ contains $e_{m-1}$. Now $w_2$ is
reduced since $w$ is. By the inductive hypothesis, $w_2 = w_4 e_{m-1}
w_5$ where $w_4$, $w_5$ are words on $e_1, \dots, e_{m-2}$. Thus $w_4$
can be commuted to the left and $w_5$ to the right, and the length of
$w$ can once again be shortened using the equivalence $e_m e_{m-1} e_m
\sim e_m$. Contradiction.
\end{proof}

It follows immediately from Jones' Lemma by induction on $n$ that
there are only finitely many reduced words in $\TL_n$, hence that
$\TL_n$ has finite spanning set over $\Bbbk$. 


By Jones' Lemma, if $w$ is a reduced word in which $e_m$ is the
generator of maximum index, then by commuting $e_m$ as far to the
right as possible, and commuting subsequent generators of smaller
index as far to the left as possible, we have
\[
w = w' (e_m e_{m-1} \cdots e_{m-l}), \qquad l\ge 0
\]
where $w'$ is a reduced word in which the generator of maximum index
is strictly smaller than $m$. Induction then leads to the following.

\begin{thm}[Jones' normal form]\label{t:Jones}
  Any reduced word $w$ in $\TL_n$ may be written in the form
  \[
  w = (e_{j_1}e_{j_1-1} \cdots e_{k_1}) (e_{j_2}e_{j_2-1} \cdots
  e_{k_2}) \cdots (e_{j_r}e_{j_r-1} \cdots e_{k_r})
  \]
  where $0 < j_1 < \cdots < j_r < n$, $0 < k_1 < \cdots < k_r < n$,
  and $j_i \ge k_i$ for all $i$. The index $j_r$ is the maximum index
  appearing in $w$.
\end{thm}

To each word in Jones normal form as above we may associate a
piecewise linear increasing (planar) path from $(0,0)$ to $(n,n)$:
\begin{multline*}
(0,0) \to (j_1,0) \to (j_1,k_1) \to (j_2,k_1) \to (j_2,k_2) \to \cdots
\\ \to (j_r,k_{r-1}) \to (j_r,k_r) \to (n,k_r) \to (n,n)
\end{multline*}
on the integer lattice $\Z \times \Z$ which does not cross the
diagonal.  Such paths are known as \emph{Dyck paths}.  For instance,
the word $(e_3e_2e_1)(e_4e_3)(e_5e_4)$ in $\TL_6$ corresponds to the
Dyck path
\[
\begin{tikzpicture}[scale = 0.5,thick, baseline={(0,-1ex/2)}] 
  \draw[very thick]
  (0,0)--(3,0)--(3,1)--(4,1)--(4,3)--(5,3)--(5,4)--(6,4)--(6,6);
  \draw[step=1,dashed,gray,very thin] (0,0) grid (6,6);
\end{tikzpicture}
\]
from $(0,0)$ in the lower left corner to $(6,6)$ in the upper right
corner. This map from words to Dyck paths is a bijection because
each such walk is determined by its corner points, and the normal form
of the word can be reconstructed from the coordinates of those points.

To count the number of Dyck paths to $(n,n)$ it is useful to consider
a slightly more general question. For any integer $p$ satisfying $0
\le 2p \le n$, define a \emph{lattice walk} to $(n-p,p)$ to be a
piecewise linear increasing path from $(0,0)$ to $(n-p,p)$ on the
integer lattice $\Z \times \Z$ which does not cross the diagonal. In
particular, Dyck paths are lattice walks to $(n,n)$. Let
\[
\Cat_{n,\,p} =
\begin{minipage}{3.2in}
number of lattice walks from $(0,0)$ to $(n-p,p)$.
\end{minipage}
\]
In this notation, $\Cat_{2n,n}$ gives the number of Dyck paths to
$(n,n)$. Any lattice walk to $(n-p,p)$ must pass through either
$(n-p-1,p)$ or $(n-p,p-1)$, so
\begin{equation}\label{e:recurrance}
  \Cat_{n,\,p} = \Cat_{n-1,\,p} + \Cat_{n-1,\,p-1}
\end{equation}
where $\Cat_{n,-1} = 0$.  We clearly have $\Cat_{n,\,0} = 1$. Also,
$\Cat_{2p-1,\,p} = 0$ as walks are not allowed to cross the
diagonal. With these boundary conditions the recurrence
\eqref{e:recurrance} is easily solved, giving the formula
\begin{equation}\label{e:Cnp}
  \Cat_{n,\,p} = \binom{n}{p}-\binom{n}{p-1}
\end{equation}
where we interpret $\binom{n}{-1}$ as zero, as usual. The set of words
in normal form spans $\TL_n$, and the $n$th Catalan number
\begin{equation}\label{e:upper-bound}
  \Cat_{2n,n} = \binom{2n}{n}-\binom{2n}{n-1} =
  \frac{1}{n+1}\binom{2n}{n}
\end{equation}
gives its cardinality. Linear independence of words in normal form
will be proved in the next section, thus showing that the set of such
words is in fact a basis of $\TL_n$ and thus $\TL_n$ is free as a
$\Bbbk$-module, of rank $\Cat_{2n,n}$.

%\begin{rmk}
%Some authors denote $\Cat_{n,p}$ by the symbol $\stirlingii{n}{p}$,
%but as this is also used for Stirling numbers of the second kind, we
%will stick to $\Cat_{n,p}$.
%\end{rmk}



\section{Diagrammatics}\label{s:dia}\noindent
We continue to work over an arbitrary commutative ring $\Bbbk$, where
$\delta$ is a fixed element of $\Bbbk$.  Following Kauffman, we now
introduce a diagram algebra $\D_n(\delta)$, based on planar diagrams
called $n$-diagrams.  It will turn out that $\D_n(\delta)$ is
isomorphic to $\TL_n(\delta)$.

An $n$-\emph{diagram} is a planar graph, with $2n$ vertices consisting
of $n$ marked points on each of two parallel lines. Each point in the
graph is the endpoint of precisely one edge, and the edges can be
drawn by non-intersecting arcs which lie entirely between the
lines. If we label the vertices along one line by the set $\mathbf{n}
= \{1, \dots, n\}$ and by $\mathbf{n}' = \{1', \dots, n'\}$
correspondingly along the other line, where $\mathbf{n} \cap
\mathbf{n}' = \emptyset$, then we may identify an $n$-diagram $D$ with
a set partition $\{B_1, \dots, B_n\}$ of $\mathbf{n} \sqcup
\mathbf{n}'$ in which each subset (block) has cardinality two. For
example, the $8$-diagram
\[
D \;=\;
\begin{tikzpicture}[scale = 0.35,thick, baseline={(0,-1ex/2)}] 
  \tikzstyle{vertex} = [shape = circle, minimum size = 4pt, inner sep = 1pt,
    fill=black] 
\node[vertex] (G--8) at (10.5, -1) [shape = circle, draw] {}; 
\node[vertex] (G--7) at (9.0, -1) [shape = circle, draw] {}; 
\node[vertex] (G--6) at (7.5, -1) [shape = circle, draw] {}; 
\node[vertex] (G-8) at (10.5, 1) [shape = circle, draw] {}; 
\node[vertex] (G--5) at (6.0, -1) [shape = circle, draw] {}; 
\node[vertex] (G--2) at (1.5, -1) [shape = circle, draw] {}; 
\node[vertex] (G--4) at (4.5, -1) [shape = circle, draw] {}; 
\node[vertex] (G--3) at (3.0, -1) [shape = circle, draw] {}; 
\node[vertex] (G--1) at (0.0, -1) [shape = circle, draw] {}; 
\node[vertex] (G-1) at (0.0, 1) [shape = circle, draw] {}; 
\node[vertex] (G-2) at (1.5, 1) [shape = circle, draw] {}; 
\node[vertex] (G-7) at (9.0, 1) [shape = circle, draw] {}; 
\node[vertex] (G-3) at (3.0, 1) [shape = circle, draw] {}; 
\node[vertex] (G-4) at (4.5, 1) [shape = circle, draw] {}; 
\node[vertex] (G-5) at (6.0, 1) [shape = circle, draw] {}; 
\node[vertex] (G-6) at (7.5, 1) [shape = circle, draw] {}; 
\draw[] (G--8) .. controls +(-0.5, 0.5) and +(0.5, 0.5) .. (G--7); 
\draw[] (G-8) .. controls +(-1, -1) and +(1, 1) .. (G--6); 
\draw[] (G--5) .. controls +(-0.8, 0.8) and +(0.8, 0.8) .. (G--2); 
\draw[] (G--4) .. controls +(-0.5, 0.5) and +(0.5, 0.5) .. (G--3); 
\draw[] (G-1) .. controls +(0, -1) and +(0, 1) .. (G--1); 
\draw[] (G-2) .. controls +(1, -1) and +(-1, -1) .. (G-7); 
\draw[] (G-3) .. controls +(0.5, -0.5) and +(-0.5, -0.5) .. (G-4); 
\draw[] (G-5) .. controls +(0.5, -0.5) and +(-0.5, -0.5) .. (G-6); 
\end{tikzpicture}
\]
corresponds to the set partition
\[
\{ \{1,1'\}, \{2,7\}, \{3,4\},
\{5,6\}, \{8,6'\}, \{2',5'\}, \{3',4'\}, \{7',8'\} \}.
\]
In the literature, edges in $n$-diagrams are also called
\emph{strands} or \emph{links}. Some authors refer to edges connecting
two vertices in the top or bottom row as \emph{cups} or \emph{caps},
respectively, and to edges connecting a top vertex to a bottom one as
\emph{through strings} or \emph{propagating strands}.


We define a multiplication on the set of $n$-diagrams as follows. If
$D_1$, $D_2$ are given $n$-diagrams, we stack $D_1$ on top of $D_2$,
identifying the middle lines and their vertices. This results in a
graph with zero or more loops in the middle, and we define
\begin{equation}\label{e:mult-rule}
D_1 D_2 = \delta^L \, D_3
\end{equation}
where $L$ is the number of loops and $D_3$ is the $n$-diagram obtained
by removing the middle data (lines, loops, and vertices). Write
\[
\D_n(\delta) = \Bbbk\text{-linear span of the set of all $n$-diagrams}.
\]
With the multiplication rule given above, $\D_n(\delta)$ is an
associative algebra over $\Bbbk$. The set of $n$-diagrams is a
$\Bbbk$-basis of $\D_n(\delta)$.


Let $\hat{e}_i= \{\{i,i+1\}, \{i',(i+1)'\}\} \sqcup \{ \{j,j'\}\mid j
\ne i, i+1 \}$ ($i = 1, \dots, n-1$) and $\hat{1}= \{ \{i,i'\}\mid i =
1, \dots, n \}$.
%\begin{align*}
%\hat{1} &= \{ \{i,i'\}: i = 1, \dots, n \},\\ \hat{e}_i &=
%\{\{i,i+1\}, \{i',(i+1)'\}\} \sqcup \{ \{j,j'\}: j \ne i, i+1 \}.
%\end{align*}
Then
\[
\hat{e}_i \;=\;
\onepic \cdots \onepic\quad \epic \quad\onepic \cdots \onepic
\]
and one checks from the multiplication rule \eqref{e:mult-rule} that
the $\hat{e}_i$ ($1 \le i \le n-1$) satisfy the defining relations
\eqref{e:TL} for $\TL_n(\delta)$. Moreover, $\hat{1} d = d = d \hat{1}$
for all $n$-diagrams $d$. It follows that there is a unique algebra
morphism
\begin{equation}\label{e:morphism}
  \sigma: \TL_n(\delta) \to \D_n(\delta)
\end{equation}
such that $\sigma(1) = \hat{1}$ and $\sigma(e_i) = \hat{e}_i$ for all $i
= 1, \dots, n-1$.



In order to count the number of $n$-diagrams, it is again fruitful to
consider a more general problem: counting the number of half-diagrams.
Cutting a diagram by a line halfway between (and parallel to) its
defining parallel lines divides the diagram into two half-diagrams. We
conventionally reflect the bottom half-diagram across its line of
marked points, so that all half-diagrams are oriented with the links
lying below the line.  A half-diagram coming from an $n$-diagram has
$p$ links (arcs connecting two vertices) and $n-2p$ defects (arcs with
one vertex), where $0 \le 2p \le n$.

\begin{lem}\label{l:half-count}
  $\Cat_{n,p} = $ the number of half-diagrams on $n$ vertices with $p$
  links.
\end{lem}

\begin{proof}
The set of all half-diagrams on $n$ vertices with $p$ links is in
bijection with the set of lattice walks (see \S\ref{s:1}) from $(0,0)$
to $(n-p,p)$. Reading a half-diagram from left to right, a walker
moves up at the $k$-th step if the $k$-th marked point closes a link,
and moves right otherwise. For example, the half-diagram
\[
\begin{tikzpicture}[scale = 0.35,thick, baseline={(0,-1ex/2)}] 
  \tikzstyle{vertex} = [shape = circle, minimum size = 4pt, inner sep = 1pt,
    fill=black] 
\node[vertex] (G-1) at (0.0, 1) [shape = circle, draw] {}; 
\node[vertex] (G-2) at (1.5, 1) [shape = circle, draw] {}; 
\node[vertex] (G-3) at (3.0, 1) [shape = circle, draw] {}; 
\node[vertex] (G-4) at (4.5, 1) [shape = circle, draw] {}; 
\node[vertex] (G-5) at (6.0, 1) [shape = circle, draw] {}; 
\node[vertex] (G-6) at (7.5, 1) [shape = circle, draw] {};
\node[vertex] (G-7) at (9.0, 1) [shape = circle, draw] {};
\node[vertex] (G-8) at (10.5, 1) [shape = circle, draw] {};
\draw[] (G-8) -- (10.5,0); 
\draw[] (G-1) -- (0,0); 
\draw[] (G-2) .. controls +(1, -1) and +(-1, -1) .. (G-7); 
\draw[] (G-3) .. controls +(0.5, -0.5) and +(-0.5, -0.5) .. (G-4); 
\draw[] (G-5) .. controls +(0.5, -0.5) and +(-0.5, -0.5) .. (G-6); 
\end{tikzpicture}
\]
with $8$ vertices and $3$ links corresponds to the lattice walk
\[
\begin{tikzpicture}[scale = 0.5,thick, baseline={(0,-1ex/2)}] 
\draw[very thick] (0,0)--(3,0)--(3,1)--(4,1)--(4,3)--(5,3);
\draw[step=1,dashed,gray,very thin] (0,0) grid (5,5);
\end{tikzpicture}
\]
from $(0,0)$ to $(5,3) = (n-p,p)$. The half-diagram may be
reconstructed from the lattice walk, so this is a bijection as claimed. 
\end{proof}


Kauffman observed that the set of $n$-diagrams is in a natural
bijection with the set of half-diagrams with $2n$ vertices and $n$
links. This bijection is visualized by the following picture:
\[
\begin{tikzpicture}[scale = 0.25,thick, baseline={(0,-1ex/2)}]
  \tikzstyle{vertex} = [shape = circle, minimum size = 4pt, inner sep = 1pt,
    fill=black]
  \path[use as bounding box] (0, -1) rectangle (5, 1);
  \draw[thick] (0,-1)--(5,-1)--(5,1)--(0,1)--(0,-1);
  \fill[lightgray] (0,-1) rectangle (5,1);
  \node[vertex] (B-1) at (1, -1) [shape = circle, draw] {};
  \node[vertex] (B-2) at (2, -1) [shape = circle, draw] {};
  \node[vertex] (B-3) at (3, -1) [shape = circle, draw] {};
  \node[vertex] (B-4) at (4, -1) [shape = circle, draw] {};
  \node[vertex] (T-1) at (1, 1) [shape = circle, draw] {};
  \node[vertex] (T-2) at (2, 1) [shape = circle, draw] {};
  \node[vertex] (T-3) at (3, 1) [shape = circle, draw] {};
  \node[vertex] (T-4) at (4, 1) [shape = circle, draw] {};
\end{tikzpicture}
\quad \mapsto \quad
\begin{tikzpicture}[scale = 0.25,thick, baseline={(0,-1ex/2)}]
  \tikzstyle{vertex} = [shape = circle, minimum size = 4pt, inner sep = 1pt,
    fill=black]
  \path[use as bounding box] (0, -4) rectangle (5, 1.5);
  \draw[thick] (0,-1)--(5,-1)--(5,1)--(0,1)--(0,-1);
  \fill[lightgray] (0,-1) rectangle (5,1);
  \node[vertex] (B-1) at (1, -1) [shape = circle, draw] {};
  \node[vertex] (B-2) at (2, -1) [shape = circle, draw] {};
  \node[vertex] (B-3) at (3, -1) [shape = circle, draw] {};
  \node[vertex] (B-4) at (4, -1) [shape = circle, draw] {};
  \node[vertex] (T-1) at (1, 1) [shape = circle, draw] {};
  \node[vertex] (T-2) at (2, 1) [shape = circle, draw] {};
  \node[vertex] (T-3) at (3, 1) [shape = circle, draw] {};
  \node[vertex] (T-4) at (4, 1) [shape = circle, draw] {};
  \node[vertex] (T-5) at (6, 1) [shape = circle, draw] {};
  \node[vertex] (T-6) at (7, 1) [shape = circle, draw] {};
  \node[vertex] (T-7) at (8, 1) [shape = circle, draw] {};
  \node[vertex] (T-8) at (9, 1) [shape = circle, draw] {};
  \draw[] (B-4) .. controls +(2, -2) and +(0,-0.5) .. (T-5);
  \draw[] (B-3) .. controls +(4, -4) and +(0,-0.5) .. (T-6);
  \draw[] (B-2) .. controls +(6, -6) and +(0,-0.5) .. (T-7);
  \draw[] (B-1) .. controls +(8, -8) and +(0,-0.5) .. (T-8);
\end{tikzpicture}
\]
In other words, draw an $n$-diagram in a rectangle, and rotate its
bottom edge through an angle of $180^\circ$, with its vertex at the
upper right corner of the rectangle. Edges are stretched accordingly
to maintain the planarity. For example,
\[
\begin{tikzpicture}[scale = 0.35,thick, baseline={(0,-1ex/2)}] 
  \tikzstyle{vertex} = [shape = circle, minimum size = 4pt, inner sep = 1pt,
    fill=black] 
\node[vertex] (G--4) at (4.5, -1) [shape = circle, draw] {}; 
\node[vertex] (G-4) at (4.5, 1) [shape = circle, draw] {}; 
\node[vertex] (G--3) at (3.0, -1) [shape = circle, draw] {}; 
\node[vertex] (G--2) at (1.5, -1) [shape = circle, draw] {}; 
\node[vertex] (G--1) at (0.0, -1) [shape = circle, draw] {}; 
\node[vertex] (G-3) at (3.0, 1) [shape = circle, draw] {}; 
\node[vertex] (G-1) at (0.0, 1) [shape = circle, draw] {}; 
\node[vertex] (G-2) at (1.5, 1) [shape = circle, draw] {}; 
\draw[] (G-4) .. controls +(0, -1) and +(0, 1) .. (G--4); 
\draw[] (G--3) .. controls +(-0.5, 0.5) and +(0.5, 0.5) .. (G--2); 
\draw[] (G-3) .. controls +(-1, -1) and +(1, 1) .. (G--1); 
\draw[] (G-1) .. controls +(0.5, -0.5) and +(-0.5, -0.5) .. (G-2); 
\end{tikzpicture}
\quad \mapsto \quad
%\begin{tikzpicture}[scale = 0.35,thick, baseline={(0,-1ex/2)}] 
%  \tikzstyle{vertex} = [shape = circle, minimum size = 4pt, inner sep = 1pt,
%    fill=black] 
%\node[vertex] (G--4) at (4.5, -1) [shape = circle, draw] {}; 
%\node[vertex] (G-4) at (4.5, 1) [shape = circle, draw] {}; 
%\node[vertex] (G--3) at (3.0, -1) [shape = circle, draw] {}; 
%\node[vertex] (G--2) at (1.5, -1) [shape = circle, draw] {}; 
%\node[vertex] (G--1) at (0.0, -1) [shape = circle, draw] {}; 
%\node[vertex] (G-3) at (3.0, 1) [shape = circle, draw] {}; 
%\node[vertex] (G-1) at (0.0, 1) [shape = circle, draw] {}; 
%\node[vertex] (G-2) at (1.5, 1) [shape = circle, draw] {}; 
%\draw[] (G-4) .. controls +(0, -1) and +(0, 1) .. (G--4); 
%\draw[] (G--3) .. controls +(-0.5, 0.5) and +(0.5, 0.5) .. (G--2); 
%\draw[] (G-3) .. controls +(-1, -1) and +(1, 1) .. (G--1); 
%\draw[] (G-1) .. controls +(0.5, -0.5) and +(-0.5, -0.5) .. (G-2);
%\node[vertex] (T-5) at (6,1) [shape = circle, draw] {};
%\node[vertex] (T-6) at (7.5,1) [shape = circle, draw] {};
%\node[vertex] (T-7) at (9,1) [shape = circle, draw] {};
%\node[vertex] (T-8) at (10.5,1) [shape = circle, draw] {};
%  \draw[] (G--4) .. controls +(2, -2) and +(0,-0.5) .. (T-5);
%  \draw[] (G--3) .. controls +(4, -4) and +(0,-0.5) .. (T-6);
%  \draw[] (G--2) .. controls +(6, -6) and +(0,-0.5) .. (T-7);
%  \draw[] (G--1) .. controls +(8, -8) and +(0,-0.5) .. (T-8);
%\end{tikzpicture}
%\quad \cong \quad
\begin{tikzpicture}[scale = 0.35,thick, baseline={(0,-1ex/2)}] 
  \tikzstyle{vertex} = [shape = circle, minimum size = 4pt, inner sep = 1pt,
    fill=black] 
\node[vertex] (G-4) at (4.5, 1) [shape = circle, draw] {}; 
\node[vertex] (G-3) at (3.0, 1) [shape = circle, draw] {}; 
\node[vertex] (G-1) at (0.0, 1) [shape = circle, draw] {}; 
\node[vertex] (G-2) at (1.5, 1) [shape = circle, draw] {};
\node[vertex] (T-5) at (6,1) [shape = circle, draw] {};
\node[vertex] (T-6) at (7.5,1) [shape = circle, draw] {};
\node[vertex] (T-7) at (9,1) [shape = circle, draw] {};
\node[vertex] (T-8) at (10.5,1) [shape = circle, draw] {};
\draw[] (G-1) .. controls +(0.5, -0.5) and +(-0.5, -0.5) .. (G-2);
\draw[] (G-4) .. controls +(0.5, -0.5) and +(-0.5, -0.5) .. (T-5);
\draw[] (T-6) .. controls +(0.5, -0.5) and +(-0.5, -0.5) .. (T-7);
\draw[] (G-3) .. controls +(1.5, -1.5) and +(-1.5, -1.5) .. (T-8);
\end{tikzpicture}
\]
This process of mapping $n$-diagrams to half-diagrams (with $2n$
vertices and $n$ links) is clearly reversible, hence defines a
bijection.  Thus, $\Cat_{2n,n}$ counts this number, and 
\begin{equation}
  \text{rank}_{\Bbbk} \D_n(\delta) = \Cat_{2n,n}.
\end{equation}
This is again the $n$th Catalan number. It appeared already in
\eqref{e:upper-bound}, as the number of words in $\TL_n(\delta)$ in
Jones normal form.


Kauffman \cite{K:90} found an algorithm that expresses a given
$n$-diagram by a reduced expression as a product of the
$\hat{e}_i$.

Start by drawing the diagram inside its enclosing rectangle. As
strands do not cross, the diagram partitions the rectangle into a
disjoint union of open regions.  Number the regions according to the
natural ``reading'' order: left to right within top to bottom.  For
example, the following picture gives the ordering
\[
\begin{tikzpicture}[scale = 0.75,thick, baseline={(0,-1ex/2)}] 
\tikzstyle{vertex} = [shape=circle, minimum size=4pt, inner sep=1pt,fill=black] 
\node[vertex] (G--10) at (13.5, -2) [shape = circle, draw] {}; 
\node[vertex] (G--7) at (9.0, -2) [shape = circle, draw] {}; 
\node[vertex] (G--9) at (12.0, -2) [shape = circle, draw] {}; 
\node[vertex] (G--8) at (10.5, -2) [shape = circle, draw] {}; 
\node[vertex] (G--6) at (7.5, -2) [shape = circle, draw] {}; 
\node[vertex] (G-10) at (13.5, 2) [shape = circle, draw] {}; 
\node[vertex] (G--5) at (6.0, -2) [shape = circle, draw] {}; 
\node[vertex] (G-1) at (0.0, 2) [shape = circle, draw] {}; 
\node[vertex] (G--4) at (4.5, -2) [shape = circle, draw] {}; 
\node[vertex] (G--3) at (3.0, -2) [shape = circle, draw] {}; 
\node[vertex] (G--2) at (1.5, -2) [shape = circle, draw] {}; 
\node[vertex] (G--1) at (0.0, -2) [shape = circle, draw] {}; 
\node[vertex] (G-2) at (1.5, 2) [shape = circle, draw] {}; 
\node[vertex] (G-9) at (12.0, 2) [shape = circle, draw] {}; 
\node[vertex] (G-3) at (3.0, 2) [shape = circle, draw] {}; 
\node[vertex] (G-8) at (10.5, 2) [shape = circle, draw] {}; 
\node[vertex] (G-4) at (4.5, 2) [shape = circle, draw] {}; 
\node[vertex] (G-7) at (9.0, 2) [shape = circle, draw] {}; 
\node[vertex] (G-5) at (6.0, 2) [shape = circle, draw] {}; 
\node[vertex] (G-6) at (7.5, 2) [shape = circle, draw] {};
\draw[blue] (-0.5,2) -- (14,2);
\draw[blue] (-0.5,-2) -- (14,-2);
\draw[blue] (-0.5,2) -- (-0.5,-2);
\draw[blue] (14,2) -- (14,-2);
\draw[name path=i] (G--10) .. controls +(-1.2, 1.2) and +(1.2, 1.2) .. (G--7); 
\draw[name path=j] (G--9) .. controls +(-0.5, 0.5) and +(0.5, 0.5) .. (G--8); 
\draw[name path=h] (G-10) .. controls +(-2, -3) and +(3, 3) .. (G--6); 
\draw[name path=b] (G-1) .. controls +(2, -3) and +(-3, 3) .. (G--5); 
\draw[name path=d] (G--4) .. controls +(-0.5, 0.5) and +(0.5, 0.5) .. (G--3); 
\draw[name path=a] (G--2) .. controls +(-0.5, 0.5) and +(0.5, 0.5) .. (G--1); 
\draw[name path=c] (G-2) .. controls +(3, -3) and +(-3, -3) .. (G-9); 
\draw[name path=e] (G-3) .. controls +(2, -2) and +(-2, -2) .. (G-8); 
\draw[name path=f] (G-4) .. controls +(1.2, -1.2) and +(-1.2, -1.2) .. (G-7); 
\draw[name path=g] (G-5) .. controls +(0.5, -0.5) and +(-0.5, -0.5) .. (G-6);

\node at (5.5,1.7) {\footnotesize\bf1};
\node at (3.9,1.6) {\footnotesize\bf2};
\node at (2.6,1.5) {\footnotesize\bf3};
\node at (1.3,1.3) {\footnotesize\bf4};
\node at (0.5,-0.5) {\footnotesize\bf5};
\node at (11.5,-0.5) {\footnotesize\bf6};
\node at (10.3,-1.6) {\footnotesize\bf7};
\end{tikzpicture}
\]
for its enclosed $10$-diagram, except that four regions, one at the
top and three along the bottom, have not been numbered. The unnumbered
regions (for which the entire upper or lower boundary is a segment of
the rectangle) don't matter for Kauffman's algorithm.

Next, let $\ell_i$ be the vertical line bisecting the rectangle with
vertices at the nodes $i$, $i+1$, $i'$, $(i+1)'$. This line always
crosses an even number (possibly zero) of strands in the
diagram. Connect the intersection points in consecutive pairs by a
dashed line segment along $\ell_i$. Do this for each $i = 1, \dots,
n-1$. Here is the above diagram with its connections.
\[
\begin{tikzpicture}[scale = 0.75,thick, baseline={(0,-1ex/2)}] 
\tikzstyle{vertex} = [shape=circle, minimum size=4pt, inner sep=1pt,fill=black] 
\node[vertex] (G--10) at (13.5, -2) [shape = circle, draw] {}; 
\node[vertex] (G--7) at (9.0, -2) [shape = circle, draw] {}; 
\node[vertex] (G--9) at (12.0, -2) [shape = circle, draw] {}; 
\node[vertex] (G--8) at (10.5, -2) [shape = circle, draw] {}; 
\node[vertex] (G--6) at (7.5, -2) [shape = circle, draw] {}; 
\node[vertex] (G-10) at (13.5, 2) [shape = circle, draw] {}; 
\node[vertex] (G--5) at (6.0, -2) [shape = circle, draw] {}; 
\node[vertex] (G-1) at (0.0, 2) [shape = circle, draw] {}; 
\node[vertex] (G--4) at (4.5, -2) [shape = circle, draw] {}; 
\node[vertex] (G--3) at (3.0, -2) [shape = circle, draw] {}; 
\node[vertex] (G--2) at (1.5, -2) [shape = circle, draw] {}; 
\node[vertex] (G--1) at (0.0, -2) [shape = circle, draw] {}; 
\node[vertex] (G-2) at (1.5, 2) [shape = circle, draw] {}; 
\node[vertex] (G-9) at (12.0, 2) [shape = circle, draw] {}; 
\node[vertex] (G-3) at (3.0, 2) [shape = circle, draw] {}; 
\node[vertex] (G-8) at (10.5, 2) [shape = circle, draw] {}; 
\node[vertex] (G-4) at (4.5, 2) [shape = circle, draw] {}; 
\node[vertex] (G-7) at (9.0, 2) [shape = circle, draw] {}; 
\node[vertex] (G-5) at (6.0, 2) [shape = circle, draw] {}; 
\node[vertex] (G-6) at (7.5, 2) [shape = circle, draw] {};
\draw[blue] (-0.5,2) -- (14,2);
\draw[blue] (-0.5,-2) -- (14,-2);
\draw[blue] (-0.5,2) -- (-0.5,-2);
\draw[blue] (14,2) -- (14,-2);
\draw[name path=i] (G--10) .. controls +(-1.2, 1.2) and +(1.2, 1.2) .. (G--7); 
\draw[name path=j] (G--9) .. controls +(-0.5, 0.5) and +(0.5, 0.5) .. (G--8); 
\draw[name path=h] (G-10) .. controls +(-2, -3) and +(3, 3) .. (G--6); 
\draw[name path=b] (G-1) .. controls +(2, -3) and +(-3, 3) .. (G--5); 
\draw[name path=d] (G--4) .. controls +(-0.5, 0.5) and +(0.5, 0.5) .. (G--3); 
\draw[name path=a] (G--2) .. controls +(-0.5, 0.5) and +(0.5, 0.5) .. (G--1); 
\draw[name path=c] (G-2) .. controls +(3, -3) and +(-3, -3) .. (G-9); 
\draw[name path=e] (G-3) .. controls +(2, -2) and +(-2, -2) .. (G-8); 
\draw[name path=f] (G-4) .. controls +(1.2, -1.2) and +(-1.2, -1.2) .. (G-7); 
\draw[name path=g] (G-5) .. controls +(0.5, -0.5) and +(-0.5, -0.5) .. (G-6);
\path[name path=v1] (0.75,-2)--(0.75,2);
\path[name path=v2] (2.25,-2)--(2.25,2);
\path[name path=v3] (3.75,-2)--(3.75,2);
\path[name path=v4] (5.25,-2)--(5.25,2);
\path[name path=v5] (6.75,-2)--(6.75,2);
\path[name path=v6] (8.25,-2)--(8.25,2);
\path[name path=v7] (9.75,-2)--(9.75,2);
\path[name path=v8] (11.25,-2)--(11.25,2);
\path[name path=v9] (12.75,-2)--(12.75,2);

\path[name intersections={of=a and v1,by=a1}];
\path[name intersections={of=b and v1,by=b1}];
\draw[red,densely dashed] (a1)--(b1);

\path[name intersections={of=b and v2,by=b2}];
\path[name intersections={of=c and v2,by=c2}];
\draw[red,densely dashed] (b2)--(c2);

\path[name intersections={of=d and v3,by=d3}];
\path[name intersections={of=b and v3,by=b3}];
\draw[red,densely dashed] (d3)--(b3);

\path[name intersections={of=c and v3,by=c3}];
\path[name intersections={of=e and v3,by=e3}];
\draw[red,densely dashed] (c3)--(e3);

\path[name intersections={of=f and v4,by=f4}];
\path[name intersections={of=e and v4,by=e4}];
\draw[red,densely dashed] (f4)--(e4);

\path[name intersections={of=b and v4,by=b4}];
\path[name intersections={of=c and v4,by=c4}];
\draw[red,densely dashed] (b4)--(c4);

\path[name intersections={of=f and v5,by=f5}];
\path[name intersections={of=g and v5,by=g5}];
\draw[red,densely dashed] (f5)--(g5);

\path[name intersections={of=e and v5,by=e5}];
\path[name intersections={of=c and v5,by=c5}];
\draw[red,densely dashed] (e5)--(c5);

\path[name intersections={of=f and v6,by=f6}];
\path[name intersections={of=e and v6,by=e6}];
\draw[red,densely dashed] (f6)--(e6);

\path[name intersections={of=c and v6,by=c6}];
\path[name intersections={of=h and v6,by=h6}];
\draw[red,densely dashed] (c6)--(h6);

\path[name intersections={of=e and v7,by=e7}];
\path[name intersections={of=c and v7,by=c7}];
\draw[red,densely dashed] (e7)--(c7);

\path[name intersections={of=h and v7,by=h7}];
\path[name intersections={of=i and v7,by=i7}];
\draw[red,densely dashed] (h7)--(i7);

\path[name intersections={of=h and v8,by=h8}];
\path[name intersections={of=c and v8,by=c8}];
\draw[red,densely dashed] (h8)--(c8);

\path[name intersections={of=i and v8,by=i8}];
\path[name intersections={of=j and v8,by=j8}];
\draw[red,densely dashed] (i8)--(j8);

\path[name intersections={of=h and v9,by=h9}];
\path[name intersections={of=i and v9,by=i9}];
\draw[red,densely dashed] (h9)--(i9);
\end{tikzpicture}
\]
Label each connecting segment on $\ell_i$ by $\hat{e}_i$.  Each
numbered region $k$ has an associated word $w_k(d)$ obtained as the
product of its connection labels in order from left to right. Then
\[
w(d) := w_1(d) \cdots w_r(d) \qquad(\text{$r$ is the number of
  numbered regions})
\]
is a reduced word corresponding to the given diagram $d$. In the
example depicted above, the reduced word is
\[
w(d) = (\hat{e}_5)(\hat{e}_4\hat{e}_6)(\hat{e}_3\hat{e}_5\hat{e}_7)
(\hat{e}_2\hat{e}_4\hat{e}_6\hat{e}_8)(\hat{e}_1\hat{e}_3)
(\hat{e}_7\hat{e}_9)(\hat{e}_8) .
\]
Although not needed, it is easy to apply commutation relations to
rewrite this in its Jones normal form, which in this case is
\[
(\hat{e}_5\hat{e}_4\hat{e}_3\hat{e}_2\hat{e}_1)
(\hat{e}_6\hat{e}_5\hat{e}_4\hat{e}_3)(\hat{e}_7\hat{e}_6)
(\hat{e}_8\hat{e}_7)(\hat{e}_9\hat{e}e_8).
\]
That the word $w(d)$ constructs the original diagram $d$ is shown in
Figure~\ref{fig1}. In that figure, the individual diagrams are the
$w_j(d)$, each of which corresponds to a product of generators.
% Figure environment removed


\begin{thm}[Kauffman \cite{K:90}*{Thm.~4.3}]\label{t:Kauffman}
  Every $n$-diagram $d$ is expressible as a product of the diagrams
  $\hat{e}_i$ ($1 \le i \le n-1$).  If $\Bbbk=\Z[x]$, where $x$ is an
  indeterminate, the map $\sigma$ in equation \eqref{e:morphism} gives
  an isomorphism $\TL_n(x) \cong \D_n(x)$, as $\Z[x]$-algebras.
\end{thm}

\begin{proof}
The first claim follows from Kauffman's algorithm, as illustrated
above (cf.~also \cite{K:90}*{Figure 16} or \cite{PS}*{Thm.~26.10}). It
implies that the map $\sigma$ is surjective. Thus, there is some word
$w$ in $\TL_n(x)$ for which $\sigma(w) = d$.  Furthermore, there is a
reduced word $w'$ such that $w = x^r w'$ for some nonnegative integer
$r$. But $\sigma(w') = x^{r'}d'$, where $d'$ is an $n$-diagram and
$r'$ a nonnegative integer. Hence we have
\[
d = \sigma(w) = x^{r+r'} d'.
\]
Thus $r=r'=0$, $w=w'$, and $d=d'$. By Theorem \ref{t:Jones} we may
assume that $w$ is in normal form.  Finally, any non-trivial linear
relation holding among the normal form elements of $\TL_n(x)$ would
induce a corresponding non-trivial relation among the $n$-diagrams in
$\D_n(x)$, which would be contradictory. This shows that $\sigma$ is
injective, thus an isomorphism.
\end{proof}

Now we return to the general case.

\begin{cor}
Over any commutative ring $\Bbbk$, for any $\delta$ in $\Bbbk$, the
natural map $\sigma$ is an isomorphism $\TL_n(\delta) \cong
\D_n(\delta)$ of $\Bbbk$-modules. This isomorphism sends the set
$\Omega$ of reduced words in Jones normal form to the set
$n$-diagrams. In particular, $\TL_n(\delta)$ is free as a
$\Bbbk$-module with basis $\Omega$, and $\mathrm{rank}_\Bbbk
\TL_n(\delta) = \Cat_{2n,n} = \frac{1}{n+1}\binom{2n}{n}$.
\end{cor}

\begin{proof}
Regard $\Bbbk$ as a $\Z[x]$-algebra via the specialization map $\Z[x]
\to \Bbbk$ sending $\sum a_i x^i$ to $a_i \delta^i$, for any $a_i \in
\Z$.  Then
\[
\Bbbk \otimes_{\Z[x]} \TL_n(x) \cong \Bbbk \otimes_{\Z[x]} \D_n(x).
\]
In other words, by standard identifications, we have
\[
\TL_n(\delta) \cong \D_n(\delta)
\]
as $\Bbbk$-algebras. This proves the first claim. The other claims
follow.
\end{proof}

Henceforth, we identify $\TL_n(\delta)$ with $\D_n(\delta)$ and $e_i$
with $\hat{e}_i$ for all $i$.

\begin{rmk}
(i) Kauffman works over $\C$ in \cite{K:90}, but his argument works
  the same over $\Z[x]$.

(ii) It follows from the diagrammatic interpretation that
  $\TL_{n-1}(\delta)$ is isomorphic to the subalgebra of
  $\TL_n(\delta)$ generated by $e_1, \dots, e_{n-2}$.

(iii) The diagrammatic interpretation also implies that
  $\TL_n(\delta)$ may be identified with the subalgebra of Brauer's
  centralizer algebra (on $n$ strands, with parameter $\delta$)
  spanned by its planar diagrams. In \cite{Birman-Wenzl}, Birman and
  Wenzl found a presentation of Brauer's algebra that also implies
  this identification.
\end{rmk}





\section{The basic construction}\noindent
The Jones basic construction \cites{Jones:83,Jones:86} was originally
applied to inclusions of von Neumann algebras. It was generalized in
\cite{GHJ}, which we follow here. We work over a field $\Bbbk$ in this
section.  Suppose that $N \subset M$ is a given inclusion of
$\Bbbk$-algebras such that $1_N = 1_M$. Then $M \subset L$, where $L
= \End_N(M)$, is another such inclusion, where $M$ is regarded as a
right $N$-module.  Iterating this idea produces a tower
\begin{equation}
  M_0 \subset M_1 \subset \cdots \subset M_i \subset M_{i+1}
  \subset \cdots
\end{equation}
of $\Bbbk$-algebras, where $M_0=N$, $M_1=M$, and $M_{i+1}
= \End_{M_{i-1}}(M_i)$ for all $i \ge 1$. The \emph{rank}
$\mathrm{rk}(M_k|M_0)$ of $M_k$ over $M_0$ is the smallest number (in
$\N \cup \{\infty\}$) of generators of $M_k$ viewed as a right
$M_0$-module. The \emph{index} $[M:N]$ of $N$ in $M$ is the growth
rate
\[
[M:N] = \limsup_{k\to \infty} \big( \mathrm{rk}(M_k|M_0) \big)^{1/k} . 
\]
If $N \subset M$ is an inclusion of semisimple algebras then
(\cite{GHJ}*{Cor.~2.1.2}) either $[M:N] = 4\cos^2(\pi/q)$ for some
integer $q \ge 3$ or $[M:N] \ge 4$.

Now suppose that $N \subset M$ is an inclusion of finite dimensional
split semisimple $\Bbbk$-algebras. (These are called ``multi-matrix
algebras'' in the terminology of \cite{GHJ}.)  A \emph{trace} on $M$
is a linear map $\tr: M \to \Bbbk$ such that $\tr(xy)=\tr(yx)$ for all
$x,y \in M$. It is \emph{nondegenerate} if the corresponding bilinear
form $(x,y) \mapsto \tr(xy)$ is nondegenerate. Assume that there
exists a nondegenerate trace $\tr$ on $M$ whose restriction to $N$ is
nondegenerate. (This is always true if $\Bbbk$ has characteristic zero.)
Then there is a unique $\Bbbk$-linear map $\E: M \to N$, called a
\emph{conditional expectation}, such that
\[
\begin{alignedat}{2}
  & \E(y)=y && \text{for all } y \in N \\
  & \E(y_1xy_2) = y_1 \E(x) y_2 && \text{for all } x \in M, y_1,y_2 \in N \\
  & \tr(\E(x)) = \tr(x) &\qquad\qquad & \text{for all } x \in M. 
\end{alignedat}
\]
Of course $\E$ is an element of $L = \End_N(M)$.  In this situation,
$L$ is generated by $M$ and $\E$. More precisely, $L$ is generated as a
vector space by all $x_1 \E x_2$, where $x_1,x_2 \in M$.

In general, traces and conditional expectations do not propagate up
the tower.  To obtain such a property it is necessary to consider
Markov traces.  Given $\beta \ne 0$ in $\Bbbk$, a \emph{Markov trace of
modulus} $\beta$ on $N \subset M$ is a nondegenerate trace $\tr$ on
$M$ with nondegenerate restriction to $N$ for which there exists a
(necessarily unique) trace $\Tr$ on $L = \End_N(M)$ such that
\[
\begin{minipage}{1in}
\begin{align*}
  \Tr(x) &= \tr(x) \\
  \beta \Tr(x \E) &= \tr(x) 
\end{align*}
\end{minipage}\qquad\qquad
\text{for all $x \in M$.}
\]
Assuming that a Markov trace $\tr$ of some modulus $0 \ne \beta \in
\Bbbk$ exists on the pair $N \subset M$ of finite dimensional split
semisimple algebras, the authors of \cite{GHJ} show that the trace
propagates up the tower to give Markov traces $\tr_k$ on $M_{k-1}
\subset M_k$ and conditional expectations $\E_k: M_k \to M_{k-1}$, for
each $k\ge 1$. Note that $\E_k$ belongs to $M_{k+1}$. 

\begin{thm}[\cite{GHJ}] \label{t:tower}
Assume that $N \subset M$ is an inclusion of finite dimensional split
semisimple $\Bbbk$-algebras, on which there exists a Markov trace $\tr$
of modulus $\beta$, where $0 \ne \beta \in \Bbbk$. For each $k \ge 1$,
let $\tr_k$, $\E_k$ be as above. Then
\begin{enumerate}
\item $M_k$ is generated by $M_1=M$ and $\E_1, \dots, \E_{k-1}$.
\item The idempotents $\E_1, \dots, \E_{k-1}$ satisfy the relations
  \begin{align*}
    \beta \E_i \E_j \E_i = \E_i \quad &\quad \text{ if } |i-j|=1 \\
    \E_i\E_j=\E_j\E_i &\quad \text{otherwise.}
  \end{align*}
\end{enumerate}
\end{thm}

\begin{rmk}
(i) In \cites{Jones:83,Jones:86} the ground field is $\Bbbk = \C$ and
  the inclusion $N \subset M$ is an inclusion of von Neumann algebras.
  See \cites{Jones:09,EK} for surveys of connections with quantum
  topology and mathematical physics.

(ii) The basic construction was initially applied to construct the Jones
  polynomial \cite{Jones:85} in knot theory. Further applications can
  be found in
  \cites{Birman-Wenzl,Wenzl:annals,Wenzl:93,Kadison,Halverson-Ram}.
  
(iii) In the context of Theorem \ref{t:tower}, the authors of
  \cite{GHJ} show that the pair $N \subset M$ is determined, up to
  isomorphism, by an inclusion matrix $\Lambda$ with nonnegative
  integer entries. The matrix $\Lambda$ may be encoded as a graph, the
  Bratteli diagram of the pair, and $[M:N]$ is the square of the
  Euclidean norm of the graph. It follows that $[M:N] \le 4$ if and
  only if the Bratteli diagram of the pair $N \subset M$ is a Coxeter
  graph of type A, D, or E.
\end{rmk}


Theorem \ref{t:tower} focuses attention on the $\Bbbk$-algebra
$\A_n(\beta)$ defined by the generators $1, u_1, \dots, u_{n-1}$
subject to the relations
\begin{equation}\label{e:Jones}
  \begin{aligned}
  u_i^2 = u_i, \quad
  \beta u_i u_j u_i = u_i \text{ if } |i-j|=1, \quad
  u_i u_j &= u_j u_i \text{ if } |i-j|>1 .
\end{aligned}
\end{equation}
We call this algebra the \emph{Jones algebra}.
We now consider the issue of semisimplicity of the Jones algebra,
following \cite{GHJ}.  Let $x$ be an indeterminate and $\Z[x]$ the
ring of polynomials in $x$ with integer coefficients. Define
polynomials $P_n(x)$ in $\Z[x]$ for each integer $n \ge 0$ by the
recursion
\begin{equation}
\begin{gathered}
  P_0(x)=1, \qquad P_1(x)=1,\\ P_{n+1}(x) = P_n(x)-xP_{n-1}(x)
  \quad \text{if $n \ge 1$}.
\end{gathered}
\end{equation}
%The first few values of the $P_n(x)$ are tabulated below:
%\[
%\begin{alignedat}{4}
%%P_2(x) &= 1-x,& P_3(x) &= 1-2x,\\ P_4(x) &= 1-3x+x^2,& P_5(x) &=
%1-4x+3x^2,\\ P_6(x) &= 1-5x+6x^2-x^3,\qquad & P_7(x) &= 1-6x+10x^2-4x^3.
%\end{alignedat}
%\]
The analysis in \cite{GHJ} suggests that the $P_n(x)$ are closely
related to the Chebyshev polynomials of the second kind. The precise
relation was made explicit in \cite{Benkart-Moon}.

Say that a trace $\tr_n:\A_n(\beta) \to \Bbbk$ is \emph{normalized} if
it takes the identity to the identity in $\Bbbk$.  Here is the
semisimplicity result. It was originally obtained over $\C$ by Jones,
and extended to arbitrary fields in the cited reference.


\begin{thm}[\cite{GHJ}*{Prop.~2.8.5}]\label{t:ss}
  Suppose that $\Bbbk$ is a field, $0 \ne \beta$, and $P_1(\beta^{-1})
  P_2(\beta^{-1}) \cdots P_{n-1}(\beta^{-1}) \ne 0$ in $\Bbbk$.  Then:
  \begin{enumerate}
  \item The Jones algebra $\A_n(\beta)$ is split semisimple over
    $\Bbbk$.
  \item There exists a unique normalized trace $\tr_n:\A_n(\beta) \to
    \Bbbk$ such that for any $1 \le j \le n-1$,
    \[
    \beta \tr_n(wu_j) = \tr_n(w)
    \]
    for all $w$ in the subalgebra generated by $1, u_1, \dots,
    u_{j-1}$. Furthermore, $\tr_n$ is nondegenerate if
    $P_n(\beta^{-1}) \ne 0$.
  \item The natural map $\A_{n-1}(\beta) \to \A_n(\beta)$ is injective
    and $\tr_n$ extends $tr_{n-1}$.
  \end{enumerate}
\end{thm}


The detailed proof of Theorem~\ref{t:ss} in \cite{GHJ} reveals
that the associated conditional expectation $\E_n: \A_n(\beta) \to
\A_{n-1}(\beta)$ satisfies the identities
\begin{equation}\label{e:CE-on-gens}
  \E_n(u_{n-1}) = \beta^{-1} 1\quad\text{and}\quad
  \E_n(u_j) = u_j \text{ for all } 1\le j < n-1.
\end{equation}
The first equality follows from $\E_n u_{n-1} \E_n = \beta^{-1} \E_n$ and
the latter is by definition of conditional expectation. Furthermore,
$\E_n$ satisfies
\begin{equation}\label{e:CE}
  u_{n-1} \E_n(u_{n-1} x) = \beta^{-1} u_{n-1} x, \quad \text{ for all
  } x \in \A_n(\beta).
\end{equation}
Finally, by setting $w=1$ in part (b) of Theorem~\ref{t:ss} we obtain
\begin{equation}\label{e:trace}
  \tr_n(u_j) = \beta^{-1} \text{ for all $1\le j \le n-1$}.
\end{equation}

\begin{rmk}
It may be useful to keep in mind that the conditional expectation
$\E_n$, viewed as an endomorphism of $\A_n(\beta)$, is identified
with the idempotent $u_n$ inside the next algebra $\A_{n+1}$ in the
tower construction.
\end{rmk}





\section{Relation between $\A_n$ and $\TL_n$}\noindent
We continue to assume that $\Bbbk$ is a field in this section.
If $\beta \ne 0$, by setting $e_i = \delta u_i$ for all $i$ we see
that relations \eqref{e:TL} and \eqref{e:Jones} are formally
equivalent if and only if $\beta = \delta^2$. Thus, we have the
following.

\begin{prop}\label{p:TL-iso}
  Suppose that $\Bbbk$ is a field. For any $\delta \ne 0$ in
  $\Bbbk$, there is a $\Bbbk$-algebra isomorphism $\A_n(\delta^2) \cong
  \TL_n(\delta)$ given by $u_i \mapsto \delta^{-1} e_i$ for all $i$.
\end{prop}

In other words, as long as $\beta \ne 0$, the Jones algebra is an
equivalent form of the Temperley--Lieb algebra in which the generators
have been rescaled to idempotents.

\begin{rmk}
On the other hand, it follows from its defining presentation
\eqref{e:Jones} that for all $n>2$, $\A_n(\beta)$ collapses to the zero
algebra at $\beta=0$, while the dimension of $\TL_n(0)$ is the $n$th
Catalan number. Also, $\A_2(0) \ncong \TL_2(0)$ is clear. So the
Jones algebra gives no information about $\TL_n(0)$.
\end{rmk}


We now consider the implications of Proposition~\ref{p:TL-iso} for
traces and conditional expectations under Kauffman's diagrammatic
interpretation of the Temperley--Lieb algebra. Always assuming that $0
\ne \beta = \delta^2$, we see that the trace $\tr_n$ and conditional
expectation $\E_n$ considered at the end of the previous section carry
over under the isomorphism $\A_n(\delta^2) \cong \TL_n(\delta)$ to give
a trace and conditional expectation on $\TL_n(\delta)$, that we will
denote by the same symbols.  It is convenient to renormalize so that
the trace still takes identity to identity. With that renormalization,
it turns out that for any $n$-diagram $d$,
\begin{equation}\label{e:diag-tr}
  \tr_n(d) = \delta^{-n}\bar{d} \quad\text{and}\quad
  \E_n(d) = \delta^{-1}\bar{d}^{(n)}
\end{equation}
where $\bar{d}$ is the diagram obtained from $d$ by drawing
non-intersecting curves outside the enclosing rectangle from vertex
$i$ to vertex $i'$ for all $i = 1,\dots, n$, and $\bar{d}^{(n)}$ is
the diagram obtained from $d$ by drawing a single such curve from
vertex $n$ to vertex $n'$.  (See \cite{Kauffman-Lins}*{p.~10}.) In
this process, loops are replaced by $\delta$. We can visualize
$\bar{d}$ and $\bar{d}^{(n)}$ by the pictures:
\[
\bar{d} \;=\; 
\begin{tikzpicture}[scale = 0.25,thick, baseline={(0,-1ex/2)}]
  \tikzstyle{vertex} = [shape=circle,minimum size=4pt,inner sep=1pt,fill=black]
  \path[use as bounding box] (0, -1) rectangle (5, 1);
  \draw[thick] (0,-1)--(5,-1)--(5,1)--(0,1)--(0,-1);
  \fill[lightgray] (0,-1) rectangle (5,1);
  \node[vertex] (B-1) at (1, -1) [shape = circle, draw] {};
  \node[vertex] (B-2) at (2, -1) [shape = circle, draw] {};
  \node[vertex] (B-3) at (3, -1) [shape = circle, draw] {};
  \node[vertex] (B-4) at (4, -1) [shape = circle, draw] {};
  \node[vertex] (T-1) at (1, 1) [shape = circle, draw] {};
  \node[vertex] (T-2) at (2, 1) [shape = circle, draw] {};
  \node[vertex] (T-3) at (3, 1) [shape = circle, draw] {};
  \node[vertex] (T-4) at (4, 1) [shape = circle, draw] {};
  \draw[] (B-4) .. controls +(2, -2) and +(2,2) .. (T-4);
  \draw[] (B-3) .. controls +(4, -4) and +(4,4) .. (T-3);
  \draw[] (B-2) .. controls +(6, -6) and +(6,6) .. (T-2);
  \draw[] (B-1) .. controls +(8, -8) and +(8,8) .. (T-1);
\end{tikzpicture}
\qquad\qquad\text{and}\qquad\; \bar{d}^{(n)} \;=\;
\begin{tikzpicture}[scale = 0.25,thick, baseline={(0,-1ex/2)}]
  \tikzstyle{vertex} = [shape = circle, minimum size = 4pt,
    inner sep=1pt,fill=black] 
  \draw[thick] (0,-1)--(5,-1)--(5,1)--(0,1)--(0,-1);
  \fill[lightgray] (0,-1) rectangle (5,1);
  \node[vertex] (B-1) at (1, -1) [shape = circle, draw] {};
  \node[vertex] (B-2) at (2, -1) [shape = circle, draw] {};
  \node[vertex] (B-3) at (3, -1) [shape = circle, draw] {};
  \node[vertex] (B-4) at (4, -1) [shape = circle, draw] {};
  \node[vertex] (T-1) at (1, 1) [shape = circle, draw] {};
  \node[vertex] (T-2) at (2, 1) [shape = circle, draw] {};
  \node[vertex] (T-3) at (3, 1) [shape = circle, draw] {};
  \node[vertex] (T-4) at (4, 1) [shape = circle, draw] {};
  \draw[] (B-4) .. controls +(2, -2) and +(2,2) .. (T-4);
\end{tikzpicture}
\]
respectively.
It follows from equation \eqref{e:diag-tr} that on generators the maps
$\E_n$ and $\tr_n$ satisfy the identities in equations
\eqref{e:CE-on-gens} and \eqref{e:trace}, respectively, with $\beta$
replaced by $\delta$ and $u_i$ replaced by $e_i$ for all~$i$.

Now we consider implications for the semisimplicity of
$\TL_n(\delta)$. Our goal is to recast the hypothesis of
Theorem~\ref{t:ss} in a more palatable form.  Let $q \in \Bbbk$. For a
positive integer $n$, the classical Gaussian integer $\dbracket{n}_q$
is
\[
  \dbracket{n}_q = 1+q+q^2 + \cdots +q^{n-1}.
\]  
If $q \ne 1$, it can be written in the form $\dbracket{n}_q =
(1-q^n)/(1-q)$ but the definition of $\dbracket{n}_q$ makes perfect
sense at $q = 1$, where it evaluates to the integer $n$. It is
customary to set $\dbracket{0}_q = 0$. Let
\[
\dbracket{n}_q^! = \dbracket{1}_q \cdots \dbracket{n-1}_q
\dbracket{n}_q = \textstyle \prod_{k=1}^n \dbracket{k}_q
\]
if $n>0$, and set $\dbracket{0}_q^! = 1$. 

Now choose $q$ in $\Bbbk$ such that $q \ne 0$, $q \ne -1$, and $\beta
= q+2+q^{-1}$. (Replace $\Bbbk$ by a suitable quadratic extension if
necessary.) It follows by a simple induction that
\begin{equation}
P_n(\beta^{-1}) = \frac{1+q+q^2 + \cdots + q^n}{(1+q)^n} =
\frac{\dbracket{n+1}}{(1+q)^n}.
\end{equation}
This was observed in Prop.~2.8.3(iv) of \cite{GHJ}.  Then
Theorem~\ref{t:ss} gives the following corollary.

\begin{cor}\label{c:ss}
  Suppose that $q \ne 0$, $q \ne -1$ where $q$ is in the field
  $\Bbbk$. With $\beta = q+2+q^{-1}$, the Jones algebra $\A_n(\beta) =
  \A_n(q+2+q^{-1})$ is split semisimple over $\Bbbk$ whenever
  $\dbracket{n}^!_q \ne 0$.
\end{cor}


If $\beta = q + 2 + q^{-1}$ then $\beta^{1/2} = \pm(q^{1/2} +
q^{-1/2})$, provided that a square root of $q$ exists in $\Bbbk$.
This, in light of Proposition~\ref{p:TL-iso}, gives the following
restatement of Corollary~\ref{c:ss}.

\begin{cor}
  Let $\Bbbk$ be a field containing a square root $q^{1/2}$ of $q$,
  where $q \ne 0$, $q \ne -1$. If $\dbracket{n}^!_q \ne 0$ then
  $\TL_n(\pm(q^{1/2}+q^{-1/2}))$ is split semisimple over $\Bbbk$.
\end{cor}


Now set $v = q^{1/2}$ so that $q = v^2$. We will always assume $q=v^2$
from now on. Then
\[
  \dbracket{n}_q = \dbracket{n}_{v^2} = 1 + v^2 + v^4 + \cdots +
  v^{2(n-1)} = v^{n-1} \textstyle \sum_{k=0}^{n-1} v^{-(n-1)+2k}.
\]
The \emph{balanced} form $[n]_v$ of the Gaussian integer corresponding
to $n$ is defined by
\[
[n]_v = \textstyle \sum_{k=0}^{n-1} v^{-(n-1)+2k}. 
\]
The definition of $[n]_v$ makes sense when $v=1$, in which case it
evaluates to $n$.  Notice that if $v^2 \ne 1$ then we can write $[n]_v
= \frac{v^n - v^{-n}}{v-v^{-1}}$.  We define $[n]_v^!  = [1]_v
\cdots[n-1] [n]_v$ and set $[0]^!_v = 1$.  The balanced and classical
forms of Gaussian integers are related by
\begin{equation}
  \dbracket{n}_q = v^{n-1} [n]_v  \qquad (\text{for } q=v^2).
\end{equation}
As $[n]_v$ and $\dbracket{n}_q$ are the same up to a power of $v$, the
preceding corollary may be restated in the following form.

\begin{cor}\label{c:ss-crit}
  Let $\Bbbk$ be a field, $0 \ne v \in \Bbbk$, where $0 \ne
  v+v^{-1}$. If $[n]_v^! \ne 0$ then $\TL_n(\pm(v+v^{-1}))$ is split
  semisimple over $\Bbbk$.
\end{cor}

See \cite{DG:orthog} for a new elementary proof of this result. The
recent paper \cite{AST} gives a very different proof based on tilting
modules. In many of the early references, e.g.,
\cites{GHJ,Martin,Westbury}, semisimplicity criteria were
formulated in a more complicated way than the simple condition in
Corollary~\ref{c:ss-crit}.



\section{$\TL_n$ as a quotient of the Iwahori--Hecke algebra}%
\label{s:Hecke}\noindent
Over the complex field $\C$, the observation that $\TL_n(v+v^{-1})$ is
isomorphic to a quotient of the Iwahori--Hecke algebra goes back (at
least) to Jones \cite{Jones}.  The following result is a slight
extension (with a different normalization) of
\cite{GHJ}*{Prop.~2.11.1}.

\begin{prop}\label{p:TL-alt}
Let $\Bbbk$ be a commutative unital ring with $v \in \Bbbk$ a fixed
invertible element. Set $\gamma_i = e_i - v^{-1}$ for all $i =
1,\dots, n-1$.  The Temperley--Lieb algebra $\TL_n(\delta)$, with
parameter $\delta = v+v^{-1}$, is the algebra defined by the
generators $\gamma_1, \dots, \gamma_{n-1}$ subject to the relations
\begin{enumerate}
\item $(\gamma_i+v^{-1})(\gamma_i-v)=0$.
\item $\gamma_i\gamma_{i+1}\gamma_i = \gamma_{i+1}\gamma_i\gamma_{i+1}$.
\item $\gamma_i \gamma_j = \gamma_j\gamma_i$ if $|i-j|>1$.
\item $v^3\gamma_i\gamma_{i+1}\gamma_i + v^2(\gamma_i\gamma_{i+1} +
  \gamma_{i+1}\gamma_i) + v(\gamma_i+\gamma_{i+1}) + 1 = 0$. 
\end{enumerate}
\end{prop}


\begin{proof}
One easily checks that the relation $e_i^2=\delta e_i$ is equivalent
to (a). Thus, we may replace $\gamma_i^2$ by $(v-v^{-1})\gamma_i + 1$
in the expansion
\begin{align*}
e_i&e_{i+1}e_i  \\ &= \gamma_i\gamma_{i+1}\gamma_i +
v^{-1}(\gamma_i\gamma_{i+1} + \gamma_{i+1}\gamma_i + \gamma_i^2) +
v^{-2}(2\gamma_i + \gamma_{i+1}) + v^{-3}
\intertext{to obtain the simplification}
&= \gamma_i\gamma_{i+1}\gamma_i +
v^{-1}(\gamma_i\gamma_{i+1} + \gamma_{i+1}\gamma_i) +
v^{-2}(\gamma_i + \gamma_{i+1}) + v^{-3} + \gamma_i + v^{-1}.
\end{align*}
It follows that the relation $e_ie_{i+1}e_i - e_i = 0$ is equivalent to
\[
\gamma_i\gamma_{i+1}\gamma_i + v^{-1}(\gamma_i\gamma_{i+1} +
\gamma_{i+1}\gamma_i) + v^{-2}(\gamma_i+\gamma_{i+1}) + v^{-3} = 0.
\]
This in turn is equivalent to (d). Interchanging $i$ and $i+1$ in the
argument shows that the relation $e_{i+1}e_ie_{i+1} - e_{i+1}= 0$ is
equivalent to
\[
\gamma_{i+1}\gamma_i\gamma_{i+1} + v^{-1}(\gamma_i\gamma_{i+1} +
\gamma_{i+1}\gamma_i) + v^{-2}(\gamma_i+\gamma_{i+1}) + v^{-3} = 0.
\]
Comparing the last two equivalences shows that (b) holds. Finally, (c)
is clear. On the other hand, if one starts with elements $\gamma_i$
satisfying relations (a)--(d) then by setting $e_i = \gamma_i+v^{-1}$
the defining relations \eqref{e:TL} for $\TL_n(\delta)$ may be
deduced.
\end{proof}

We continue to work over a commutative ring $\Bbbk$ with $1$.
Recall \cite{Jimbo} (see also \cite{Lu:Hecke}) that the Iwahori--Hecke
algebra $\HH_n$ of type A may be defined as the $\Bbbk$-algebra with
$1$ on generators $T_1, \dots, T_{n-1}$ subject to the relations
\begin{equation}\label{e:Hecke}
\begin{gathered}
  (T_i+v^{-1})(T_i-v) = 0\\
  T_iT_jT_i = T_jT_iT_j \text{ if } |i-j|=1\\
  T_iT_j = T_jT_i \text{ if } |i-j|>1. 
\end{gathered}
\end{equation}
The generators $T_i$ are invertible, with
\begin{equation*}
T_i^{-1} = T_i+v^{-1}-v.
\end{equation*}
We immediately have the following consequence of
Proposition~\ref{p:TL-alt}; compare with \cite{GHJ}*{Cor.~2.11.2}.

\begin{cor}\label{c:TL-quo}
Suppose that $\Bbbk$ is a unital commutative ring containing an
invertible element $v$, and that $\delta = v+v^{-1}$.  There exists a
surjective algebra homomorphism
\[
\psi_n: \HH_n \to \TL_n(\delta)
\]
defined by $\psi_n(T_i)=\gamma_i=e_i-v^{-1}$ for $i = 1, \dots, n-1$.
If $n=1$ or $n=2$ it is an isomorphism.  If $n \ge 3$, the kernel of
$\psi_n$ is the two-sided ideal of $\HH_n$ generated by
\[
v^3 T_1T_2T_1 + v^2(T_1T_2+T_2T_1) + v(T_1+T_2) + 1. 
\]
Furthermore, the diagram (in which the horizontal maps are the canonical
inclusions)
\[
\begin{tikzcd}
\HH_n \arrow[r] \arrow[d, "\psi_n"] & \HH_{n+1} \arrow[d, "\psi_{n+1}"] \\
\TL_n(\delta) \arrow[r] & \TL_{n+1}(\delta)
\end{tikzcd}
\]
is commutative.
\end{cor}


\begin{proof}
The existence of $\psi_n$ follows from the definition of $\HH_n$ and
Proposition~\ref{p:TL-alt}. It is easy to check that $\psi_1$ and
$\psi_2$ are isomorphisms, and that the kernel of $\psi_n$ for $n \ge
3$ is generated by all
\[
x_i = v^3 T_iT_{i+1}T_i + v^2(T_iT_{i+1}+T_{i+1}T_i) + v(T_i+T_{i+1}) + 1
\]
for $i = 1, \dots, n-2$. Each $T_i$ is invertible, with $T_i^{-1} =
T_i+v^{-1}-v$. From the braid relations for the $T_i$ we have
\[
(T_1T_2 \cdots T_{n-1}) T_k (T_{n-1}^{-1} \cdots T_2^{-1}T_1^{-1}) = T_{k+1}
\]
for all $k = 1, \dots, n-2$.
Thus,
\[
(T_1T_2 \cdots T_{n-1}) x_k (T_{n-1}^{-1} \cdots T_2^{-1}T_1^{-1}) = x_{k+1}
\]
for all $k = 1, \dots, n-3$. This shows that the kernel is generated
by $x_1$, as required. The final claim is obvious.
\end{proof}



\begin{rmk}\label{r:HH2}
The original version of $\HH_n$ in the literature
(see \cites{KL:79,DJ1,DJ2,DJ3,DJ4,Murphy1,Murphy2})
was defined with the quadratic relation 
\[
(T_i+1)(T_i-q) = 0 \quad\text{where}\quad q=v^2
\]
and with the remaining relations the same. This leads to an isomorphic
algebra, but many formulas look different. To effect comparisons
between different versions, it is convenient to define (see \cites{BW,
  Bigelow}) the two-parameter Iwahori--Hecke algebra $\HH_n(q_1,q_2)$
to be the $\Bbbk$-algebra with $1$ defined by generators $T_1, \dots,
T_{n-1}$ with the defining relations
\begin{equation*}
  \begin{gathered}
    (T_i-q_1)(T_i-q_2) = 0 \\
  T_iT_jT_i = T_jT_iT_j \text{ if } |i-j|=1 \\
  T_iT_j = T_jT_i \text{ if } |i-j|>1 .
\end{gathered}
\end{equation*}
In this notation, the original version of $\HH_n$ is $\HH_n(-1,q)$ and
the version defined in \eqref{e:Hecke} is $H_n(-v^{-1},v)$, where
$q=v^2$. The algebra map defined by $T_i \mapsto v^{-1}T_i$ defines an
isomorphism
\[
\HH_n(-v^{-1},v) \cong \HH_n(-1,v^2).
\]
Assuming that $q_1$ and $q_2$ are invertible in $\Bbbk$ and setting $q
= -q_2/q_1$, one has an algebra isomorphism
\[
\HH_n(-1,q) \cong \HH_n(q_1,q_2)
\]
defined by $T_i \mapsto -q_1^{-1}T_i$ for all $i$. This means that
$\TL_n(v+v^{-1})$ can be constructed as a quotient of any version of
$\HH_n$, provided only that the eigenvalues of the $T_i$ are
invertible in $\Bbbk$ (and $q=v^2$).
\end{rmk}




\section{Representations of $\TL_n$}\noindent
In this section, $\Bbbk$ is a commutative unital ring and $\delta \in
\Bbbk$, unless stated otherwise.
We fix $n$ and $\delta$ and sometimes write $\TL_n = \TL_n(\delta)$.

We begin with a number of bijections that underlie the combinatorics
of Temperley--Lieb algebras. In Section~\ref{s:dia} we considered
lattice walks to $(n-p,p)$, where $0\le 2p \le n$. Notice that the pair
$(n-p,p)$ in such a walk may be identified with a partition of at most
two parts, which in turn may be identified with its Young diagram.
The Bratteli diagram associated to Temperley--Lieb combinatorics is
the infinite graph constructed inductively as follows:
\begin{itemize}
\item Start with the empty partition $\emptyset$ in level zero.
\item For each partition $\lambda=(\lambda_1,\lambda_2)$ in some
  level, draw a vertical edge to the partition
  $(\lambda_1+1,\lambda_2)$ and, if $\lambda_1>\lambda_2$, a diagonal
  edge to the partition $(\lambda_1,\lambda_2+1)$.
\end{itemize}
We illustrate the Bratteli diagram in Figure \ref{Bratteli} below.

A $1$-\emph{factor} is a sequence of $f = (f_1, \dots, f_n)$ such that
each $f_i = \pm 1$ and the partial sums $f_1 + \cdots + f_i$ are
nonnegative, for all $i$. For each $i$ with $f_i=1$ in a $1$-factor
$f$, let $j$ be the smallest index (if any) for which $i < j \le n$
and $f_i + \cdots + f_j = 0$. Whenever this happens, the indices
$(i,j)$ are said to be \emph{paired}; otherwise the index $i$ is
\emph{unpaired}.

%%%%%%%%%%%%%%%%%%%%%%%%%%%%%%%
%% BEGIN: Bratteli Diagram
%%
% Figure environment removed
%%%%%%%%%%%%%%%%%%%%%%%%%%%%%%%%%

Here are the promised bijections.

\begin{lem}\label{l:bijections}
  For any $n$, $p$ such that $0 \le 2p \le n$, the following sets are
  all in bijective correspondence with one another:
  \begin{enumerate}\renewcommand{\labelenumi}{(\roman{enumi})}
  \item The set of half-diagrams on $n$ vertices with $p$ links.
  \item The set of lattice walks from $(0,0)$ to $(n-p,p)$.
  \item The set of paths in the Bratteli diagram
    from $\emptyset$ to $(n-p,p)$.
  \item The set of standard tableaux of shape $(n-p,p)$.
  \item The set of $1$-factors of length $n$ with $p$ pairings.
  \end{enumerate}
\end{lem}

\begin{proof}
The bijection between the sets in (i), (ii) is Lemma
\ref{l:half-count}.  The bijection between the sets in (ii), (iii) is
obtained by matching (horizontal, vertical) segments in a lattice walk
with (vertical, diagonal) edges in a Bratteli path. The bijection
between the sets in (ii), (iv) comes from numbering each unit-length
segment in a lattice walk, in order. Write the numbers into the boxes
of a Young diagram of shape $(n-p,p)$ so that horizontal segments are
recorded in row one, and vertical segments in row two.  For instance,
the tableau
\[
\small \young(12358,467) 
\]
corresponds to the lattice walk appearing in the proof of
Lemma~\ref{l:half-count}.  Note that the numbers are entered in order
in each row from left to right; this always produces a standard
tableau. Finally, a bijection between the sets in (i), (v) is easily
obtained by matching links with paired vertices and defects with
unpaired ones. 
\end{proof}



In this section, we prefer to use the set of half-diagrams from part
(i) of Lemma~\ref{l:bijections}, but that indexing set may be replaced
by any of the others. We note that half-diagrams are called ``planar
involutions'' in \cite{GL:96}. We need the following notation.  Set
\[
\Lambda = \Lambda(n) = \{n, n-2, \dots, n-2l\}
\]
where $l$ is the integer part of $n/2$. Notice that the map $n-2p
\mapsto (n-p,p)$ for $0\le 2p \le n$ defines a bijection between
$\Lambda$ and the set of two-part partitions of $n$. For each $\lambda
\in \Lambda$, let
\[
M(\lambda) = \text{ the set of half-diagrams on $n$ vertices with
  $\lambda$ defects}.
\]
Given any $(s,t) \in M(\lambda)\times M(\lambda)$, let $t^*$ be the
reflection of $t$ across the line containing its vertices. Place $s$
directly above $t^*$. There is one and only one way to
connect the defects in $s$ to the defects in $t^*$ so as to make an
$n$-diagram. Let
\[
C^\lambda_{s,t} = \text{ the $n$-diagram obtained by this process.}
\]
Then the disjoint union $\bigsqcup_{\lambda\in \Lambda}
\{C^\lambda_{s,t} \mid s,t \in M(\lambda)\}$ is the basis consisting
of all $n$-diagrams.  Finally, let
\[
*: \TL_n \to \TL_n
\]
be the linear extension of the map that reflects a given diagram
across its axis of symmetry with respect to the parallel lines
determined by its vertices. Write $d^*$ for the image of a diagram $d$
under this map. Then
\[
d^{**}=d \quad\text{and}\quad (d_1d_2)^* = d_2^*d_1^*
\]
for all $n$-diagrams $d_1$, $d_2$. In other words, the map $*$ is an
algebra anti-involution of $\TL_n$. We have $(C^\lambda_{s,t})^* =
C^\lambda_{t,s}$ for all $s,t \in M(\lambda)$, $\lambda \in \Lambda$.
By Lemma~\ref{l:bijections} and equation~\eqref{e:Cnp}, the
cardinality of $M(\lambda)$ is given by
\begin{equation}\label{e:cardMlam}
  |M(\lambda)| = \Cat_{n,p} = \binom{n}{p} - \binom{n}{p-1} \qquad
  \text{if } n-2p = \lambda
\end{equation}
for each $\lambda$ in $\Lambda$.


Graham and Lehrer \cite{GL:96} introduced axiomatics for the study of
\emph{cellular} algebras. Let $\Bbbk$ be a commutative ring with
$1$. An associative unital $\Bbbk$-algebra $A$ is cellular if it has a
\emph{cell datum} $(\Lambda,M,C,*)$, satisfying:
\begin{enumerate}
\item[(C1)] $\Lambda$ is a finite poset, and for each $\lambda \in
  \Lambda$, $M(\lambda)$ is a finite indexing set such that $C:
  \bigsqcup_{\lambda\in \Lambda} M(\lambda) \times M(\lambda) \to
  \Lambda$ is an injective map with image a $\Bbbk$-basis of $A$.
  (One writes $C^\lambda_{S,T}$ for $C(S,T)$ if $S,T \in M(\lambda)$.)
  
\item[(C2)] $*$ is a $\Bbbk$-linear anti-involution of $A$ such that
  $(C^\lambda_{S,T})^* = C^\lambda_{T,S}$, for all $S,T \in
  M(\lambda)$ and all $\lambda \in \Lambda$.

\item[(C3)] Let $A(<\lambda)$ be the $\Bbbk$-span of the $C^\mu_{U,V}$
  such that $\mu < \lambda$.  If $\lambda \in \Lambda$ and $S,T \in
  M(\lambda)$ then for any $a \in A$ we have
  \[
  aC^\lambda_{S,T} \equiv \sum_{S' \in M(\lambda)} r_a(S',S)
  C^\lambda_{S',T} \pmod{A(<\lambda)}
  \]
  where $r_a(S',S) \in \Bbbk$ is independent of $T$.
\end{enumerate} 
It is easy to check the following (see \cite{GL:96}*{Example 1.4}).

\begin{prop}
  Let $\Bbbk$ be a unital commutative ring, $\delta \in \Bbbk$. Then
  the datum $(\Lambda,M,C,*)$ defined above is a cell datum for the
  algebra $\TL_n = \TL_n(\delta)$. In other words, $\TL_n$ is
  cellular, and the basis of $n$-diagrams is a cellular basis.
\end{prop}

The theory of cellular algebras provides a convenient framework for
studying the $\TL_n(\delta)$-modules, especially when $\Bbbk$ is a
field.

We now construct some $\TL_n$-modules diagrammatically (the
construction does not depend of the theory of cellular algebras).  If
$h$ is a half-diagram on $n$ vertices and $d$ an $n$-diagram, we stack
$d$ above $h$ and apply the diagrammatic multiplication rule
\eqref{e:mult-rule} to obtain
\begin{equation}\label{e:half-m-rule}
  d h = \delta^N \, h'
\end{equation}
for a unique half-diagram $h'$ (obtained by discarding the loops and
identified vertices and retaining links) and some integer $N \ge 0$
(the number of discarded loops). 
For example,
\[
\begin{tikzpicture}[scale = 0.35,thick, baseline={(0,-1ex/2)}] 
\tikzstyle{vertex} = [shape = circle, minimum size = 4pt,
    inner sep = 1pt, fill=black] 
\node[vertex] (G--6) at (7.5, -1) [shape = circle, draw] {}; 
\node[vertex] (G--5) at (6.0, -1) [shape = circle, draw] {}; 
\node[vertex] (G--4) at (4.5, -1) [shape = circle, draw] {}; 
\node[vertex] (G-6) at (7.5, 1) [shape = circle, draw] {}; 
\node[vertex] (G--3) at (3.0, -1) [shape = circle, draw] {}; 
\node[vertex] (G-1) at (0.0, 1) [shape = circle, draw] {}; 
\node[vertex] (G--2) at (1.5, -1) [shape = circle, draw] {}; 
\node[vertex] (G--1) at (0.0, -1) [shape = circle, draw] {}; 
\node[vertex] (G-2) at (1.5, 1) [shape = circle, draw] {}; 
\node[vertex] (G-5) at (6.0, 1) [shape = circle, draw] {}; 
\node[vertex] (G-3) at (3.0, 1) [shape = circle, draw] {}; 
\node[vertex] (G-4) at (4.5, 1) [shape = circle, draw] {}; 
\draw[] (G--6) .. controls +(-0.5, 0.5) and +(0.5, 0.5) .. (G--5); 
\draw[] (G-6) .. controls +(-1, -1) and +(1, 1) .. (G--4); 
\draw[] (G-1) .. controls +(1, -1) and +(-1, 1) .. (G--3); 
\draw[] (G--2) .. controls +(-0.5, 0.5) and +(0.5, 0.5) .. (G--1); 
\draw[] (G-2) .. controls +(0.9, -0.9) and +(-0.9, -0.9) .. (G-5); 
\draw[] (G-3) .. controls +(0.5, -0.5) and +(-0.5, -0.5) .. (G-4); 
\end{tikzpicture} \quad \times \quad
\begin{minipage}{2.8cm}
\begin{tikzpicture}[scale = 0.35,thick, baseline={(0,-1ex/2)}] 
\tikzstyle{vertex} = [shape = circle, minimum size = 4pt,
    inner sep = 1pt, fill=black] 
\node[vertex] (G-6) at (7.5, 1) [shape = circle, draw] {}; 
\node[vertex] (G-1) at (0.0, 1) [shape = circle, draw] {}; 
\node[vertex] (G-2) at (1.5, 1) [shape = circle, draw] {}; 
\node[vertex] (G-5) at (6.0, 1) [shape = circle, draw] {}; 
\node[vertex] (G-3) at (3.0, 1) [shape = circle, draw] {}; 
\node[vertex] (G-4) at (4.5, 1) [shape = circle, draw] {};
\draw[] (G-6) -- (7.5,0);
\draw[] (G-1) -- (0,0);
\draw[] (G-2) .. controls +(0.9, -0.9) and +(-0.9, -0.9) .. (G-5); 
\draw[] (G-3) .. controls +(0.5, -0.5) and +(-0.5, -0.5) .. (G-4); 
\end{tikzpicture}
\end{minipage} \quad = \quad
\begin{minipage}{2.8cm}
\begin{tikzpicture}[scale = 0.35,thick, baseline={(0,-1ex/2)}] 
\tikzstyle{vertex} = [shape = circle, minimum size = 4pt,
    inner sep = 1pt, fill=black] 
\node[vertex] (G-6) at (7.5, 1) [shape = circle, draw] {}; 
\node[vertex] (G-1) at (0.0, 1) [shape = circle, draw] {}; 
\node[vertex] (G-2) at (1.5, 1) [shape = circle, draw] {}; 
\node[vertex] (G-5) at (6.0, 1) [shape = circle, draw] {}; 
\node[vertex] (G-3) at (3.0, 1) [shape = circle, draw] {}; 
\node[vertex] (G-4) at (4.5, 1) [shape = circle, draw] {};
\draw[] (G-1) .. controls +(1.3, -1.3) and +(-1.3, -1.3) .. (G-6);
\draw[] (G-2) .. controls +(0.9, -0.9) and +(-0.9, -0.9) .. (G-5); 
\draw[] (G-3) .. controls +(0.5, -0.5) and +(-0.5, -0.5) .. (G-4); 
\end{tikzpicture}
\end{minipage} 
\]
as one can see by considering the configuration 
\[
\begin{gathered}
\begin{tikzpicture}[scale = 0.35,thick, baseline={(0,-1ex/2)}] 
\tikzstyle{vertex} = [shape = circle, minimum size = 4pt,
    inner sep = 1pt, fill=black] 
\node[vertex] (G--6) at (7.5, -1) [shape = circle, draw] {}; 
\node[vertex] (G--5) at (6.0, -1) [shape = circle, draw] {}; 
\node[vertex] (G--4) at (4.5, -1) [shape = circle, draw] {}; 
\node[vertex] (G-6) at (7.5, 1) [shape = circle, draw] {}; 
\node[vertex] (G--3) at (3.0, -1) [shape = circle, draw] {}; 
\node[vertex] (G-1) at (0.0, 1) [shape = circle, draw] {}; 
\node[vertex] (G--2) at (1.5, -1) [shape = circle, draw] {}; 
\node[vertex] (G--1) at (0.0, -1) [shape = circle, draw] {}; 
\node[vertex] (G-2) at (1.5, 1) [shape = circle, draw] {}; 
\node[vertex] (G-5) at (6.0, 1) [shape = circle, draw] {}; 
\node[vertex] (G-3) at (3.0, 1) [shape = circle, draw] {}; 
\node[vertex] (G-4) at (4.5, 1) [shape = circle, draw] {}; 
\draw[] (G--6) .. controls +(-0.5, 0.5) and +(0.5, 0.5) .. (G--5); 
\draw[] (G-6) .. controls +(-1, -1) and +(1, 1) .. (G--4); 
\draw[] (G-1) .. controls +(1, -1) and +(-1, 1) .. (G--3); 
\draw[] (G--2) .. controls +(-0.5, 0.5) and +(0.5, 0.5) .. (G--1); 
\draw[] (G-2) .. controls +(0.9, -0.9) and +(-0.9, -0.9) .. (G-5); 
\draw[] (G-3) .. controls +(0.5, -0.5) and +(-0.5, -0.5) .. (G-4); 
\end{tikzpicture}\\
\begin{tikzpicture}[scale = 0.35,thick, baseline={(0,-1ex/2)}] 
\tikzstyle{vertex} = [shape = circle, minimum size = 4pt,
    inner sep = 1pt, fill=black] 
\node[vertex] (G-6) at (7.5, 1) [shape = circle, draw] {}; 
\node[vertex] (G-1) at (0.0, 1) [shape = circle, draw] {}; 
\node[vertex] (G-2) at (1.5, 1) [shape = circle, draw] {}; 
\node[vertex] (G-5) at (6.0, 1) [shape = circle, draw] {}; 
\node[vertex] (G-3) at (3.0, 1) [shape = circle, draw] {}; 
\node[vertex] (G-4) at (4.5, 1) [shape = circle, draw] {};
\draw[] (G-6) -- (7.5,0);
\draw[] (G-1) -- (0,0);
\draw[] (G-2) .. controls +(0.9, -0.9) and +(-0.9, -0.9) .. (G-5); 
\draw[] (G-3) .. controls +(0.5, -0.5) and +(-0.5, -0.5) .. (G-4); 
\end{tikzpicture}
\end{gathered}
\]
obtained by the usual stacking procedure. This example shows,
incidentally, that the action does not always preserve the number of
defects, although the number of defects in $h'$ cannot exceed the
number in $h$.

Let $\hat{H}$ be the $\Bbbk$-linear span of the set
$\bigsqcup_{\lambda \in \Lambda} M(\lambda)$. That is, $\hat{H}$ is
the span of the set of half-diagrams on $n$ vertices. The linear
extension of the action defined in \eqref{e:half-m-rule} makes
$\hat{H}$ into a $\TL_n(\delta)$-module. For each $\lambda$ in
$\Lambda$, let
\[
\hat{H}^{\le \lambda} = \text{ $\Bbbk$-span of } \textstyle
  \bigsqcup_{\mu \le \lambda} M(\mu), \quad \hat{H}^{< \lambda} =
\text{ $\Bbbk$-span of } \textstyle \bigsqcup_{\mu < \lambda}
  M(\mu)
\]
where $\mu$ ranges over $\Lambda$ in both unions.  Since the
$\TL_n$-action cannot increase the number of defects, these spans are
$\TL_n$-submodules of $\hat{H}$. For each $\lambda \in \Lambda$, set
\[
H(\lambda) := \hat{H}^{\le \lambda}/\hat{H}^{< \lambda}. 
\]
A basis for $H(\lambda)$ is $\{h+\hat{H}^{< \lambda} \mid h \in
M(\lambda)\}$.  If we abuse notation by denoting a coset $h+\hat{H}^{<
  \lambda}$ by its chosen representative $h$, for any $h \in
M(\lambda)$, then the action of an $n$-diagram $d$ on $h$ is given by:
\begin{equation}\label{e:quo-act}
d h = 
\begin{cases}
  \delta^N \, h' & \text{ if } h' \in M(\lambda)\\ 0 & \text{ otherwise}
\end{cases}
\end{equation}
with $N$, $h'$ as in \eqref{e:half-m-rule}.
With this convention, the set $M(\lambda)$ is a basis of $H(\lambda)$
and the rule \eqref{e:quo-act} defines its $\TL_n(\delta)$-module
structure.

Graham and Lehrer prove that any cellular algebra has an associated
family of \emph{cell modules}, which may be constructed abstractly
from its chosen cellular basis.



\begin{prop}\label{p:Hm}
  Let $\delta$ be an element of a unital commutative ring $\Bbbk$.
  For any $\lambda \in \Lambda$, the module $H(\lambda)$ is isomorphic
  to the abstract cell module indexed by $\lambda$ in the theory of
  cellular algebras. 
\end{prop}

A proof of this fact is implicit in \cite{GL:96}. Alternatively, the
reader may prefer to construct the needed isomorphism directly, which
is not difficult.  Note that the rank of $H(\lambda)$ over $\Bbbk$ is
computed in equation \eqref{e:cardMlam}.


We take a moment to consider some general facts on cellular algebras.
If $A$ is cellular with cell datum $(\Lambda,M,C,*)$, let $W(\lambda)$
be the cell module indexed by $\lambda$ in $\Lambda$. There is an
associated bilinear form $\varphi_\lambda$ on $W(\lambda)$. Let
$\Lambda_0 := \{\lambda \in \Lambda \mid \varphi_\lambda \ne 0\}$.  By
\cite{GL:96}, if the ground ring $\Bbbk$ is a field then the collection
\[
  \{ L(\lambda):= W(\lambda)/\text{rad}(\varphi) \mid \lambda \in
  \Lambda_0 \}
\]
gives a complete set, up to isomorphism, of simple
$A$-modules. Furthermore, each simple module is absolutely simple.
Finally, $A$ is (split) semisimple over a field if and only if all of
its cell modules are simple.

Return now to the consideration of $A=\TL_n$. Combining the final
sentence of the preceding paragraph with Corollary \ref{c:ss-crit}
gives the following.

\begin{prop}
  Let $\Bbbk$ be a field, and suppose that $0 \ne v \in \Bbbk$ such
  that $[n]_v^! \ne 0$. If $\delta = \pm(v+v^{-1}) \ne 0$ then
  $\{H(\lambda)\mid \lambda \in \Lambda\}$ is a complete set of simple
  $\TL_n(\delta)$-modules, up to isomorphism.
\end{prop}

Let $\varphi_\lambda(-,-)$ be the bilinear form associated to
$H(\lambda)$.  We claim that $\varphi_\lambda$ may be computed
diagrammatically.  If $h$ is a half-diagram, let $h^*$ be the result
of reflecting $h$ across the line containing its vertices.  Given $h,
h' \in M(\lambda)$, where $\lambda \in \Lambda$, let $h^*|h'$ be the
configuration obtained by stacking $h^*$ above $h'$. We say that
$h^*|h'$ is \emph{defect preserving} if every defect in one of the
half-diagrams is connected by a path to a defect in the other
half-diagram, after corresponding vertices are identified. Then
$\varphi_\lambda(h,h')$ is given by
\begin{equation}
  \varphi_\lambda(h,h') = 
  \begin{cases}
    \delta^N & \text{if $h^*|h'$ is defect preserving}\\
    0 & \text{otherwise}
  \end{cases}
\end{equation}
where $N$ is the number of loops in $h^*|h'$ (after corresponding
vertices are identified). The form $\varphi_\lambda$ is associative:
$\varphi_\lambda(th,h') = \varphi_\lambda(h,t^*h')$, for any $t$ in
$\TL_n(\delta)$.

\begin{example}
If $n=6$ then we have the following, which illustrate the various
cases that can occur.
\begin{enumerate}\renewcommand{\labelenumi}{(\roman{enumi})}
\item \quad $\varphi_0\big(
\begin{tikzpicture}[scale = 0.35,thick, baseline={(0,1ex)}] 
\tikzstyle{vertex} = [shape = circle, minimum size = 4pt,
    inner sep = 1pt, fill=black] 
\node[vertex] (G-6) at (7.5, 1) [shape = circle, draw] {}; 
\node[vertex] (G-1) at (0.0, 1) [shape = circle, draw] {}; 
\node[vertex] (G-2) at (1.5, 1) [shape = circle, draw] {}; 
\node[vertex] (G-5) at (6.0, 1) [shape = circle, draw] {}; 
\node[vertex] (G-3) at (3.0, 1) [shape = circle, draw] {}; 
\node[vertex] (G-4) at (4.5, 1) [shape = circle, draw] {};
\draw[] (G-5) .. controls +(0.5, -0.5) and +(-0.5, -0.5) .. (G-6);
\draw[] (G-1) .. controls +(0.9, -0.9) and +(-0.9, -0.9) .. (G-4); 
\draw[] (G-2) .. controls +(0.5, -0.5) and +(-0.5, -0.5) .. (G-3); 
\end{tikzpicture}
\; , \;
\begin{tikzpicture}[scale = 0.35,thick, baseline={(0,1ex)}] 
\tikzstyle{vertex} = [shape = circle, minimum size = 4pt,
    inner sep = 1pt, fill=black] 
\node[vertex] (G-6) at (7.5, 1) [shape = circle, draw] {}; 
\node[vertex] (G-1) at (0.0, 1) [shape = circle, draw] {}; 
\node[vertex] (G-2) at (1.5, 1) [shape = circle, draw] {}; 
\node[vertex] (G-5) at (6.0, 1) [shape = circle, draw] {}; 
\node[vertex] (G-3) at (3.0, 1) [shape = circle, draw] {}; 
\node[vertex] (G-4) at (4.5, 1) [shape = circle, draw] {};
\draw[] (G-5) .. controls +(0.5, -0.5) and +(-0.5, -0.5) .. (G-6);
\draw[] (G-3) .. controls +(0.5, -0.5) and +(-0.5, -0.5) .. (G-4); 
\draw[] (G-1) .. controls +(0.5, -0.5) and +(-0.5, -0.5) .. (G-2); 
\end{tikzpicture}
\big) \; = \; \delta^2$.

\item \quad $\varphi_2\big(
\begin{tikzpicture}[scale = 0.35,thick, baseline={(0,1ex)}] 
\tikzstyle{vertex} = [shape = circle, minimum size = 4pt,
    inner sep = 1pt, fill=black] 
\node[vertex] (G-6) at (7.5, 1) [shape = circle, draw] {}; 
\node[vertex] (G-1) at (0.0, 1) [shape = circle, draw] {}; 
\node[vertex] (G-2) at (1.5, 1) [shape = circle, draw] {}; 
\node[vertex] (G-5) at (6.0, 1) [shape = circle, draw] {}; 
\node[vertex] (G-3) at (3.0, 1) [shape = circle, draw] {}; 
\node[vertex] (G-4) at (4.5, 1) [shape = circle, draw] {};
\draw[] (G-1) .. controls +(0.5, -0.5) and +(-0.5, -0.5) .. (G-2);
\draw[] (G-3) -- (3.0,0);
\draw[] (G-6) -- (7.5,0);
\draw[] (G-4) .. controls +(0.5, -0.5) and +(-0.5, -0.5) .. (G-5); 
\end{tikzpicture}
\; , \;
\begin{tikzpicture}[scale = 0.35,thick, baseline={(0,1ex)}] 
\tikzstyle{vertex} = [shape = circle, minimum size = 4pt,
    inner sep = 1pt, fill=black] 
\node[vertex] (G-6) at (7.5, 1) [shape = circle, draw] {}; 
\node[vertex] (G-1) at (0.0, 1) [shape = circle, draw] {}; 
\node[vertex] (G-2) at (1.5, 1) [shape = circle, draw] {}; 
\node[vertex] (G-5) at (6.0, 1) [shape = circle, draw] {}; 
\node[vertex] (G-3) at (3.0, 1) [shape = circle, draw] {}; 
\node[vertex] (G-4) at (4.5, 1) [shape = circle, draw] {};
\draw[] (G-5) -- (6.0,0);
\draw[] (G-6) -- (7.5,0);
\draw[] (G-3) .. controls +(0.5, -0.5) and +(-0.5, -0.5) .. (G-4); 
\draw[] (G-1) .. controls +(0.5, -0.5) and +(-0.5, -0.5) .. (G-2); 
\end{tikzpicture}
\big) \; = \; \delta$.

\item \quad $\varphi_2\big(
\begin{tikzpicture}[scale = 0.35,thick, baseline={(0,1ex)}] 
\tikzstyle{vertex} = [shape = circle, minimum size = 4pt,
    inner sep = 1pt, fill=black] 
\node[vertex] (G-6) at (7.5, 1) [shape = circle, draw] {}; 
\node[vertex] (G-1) at (0.0, 1) [shape = circle, draw] {}; 
\node[vertex] (G-2) at (1.5, 1) [shape = circle, draw] {}; 
\node[vertex] (G-5) at (6.0, 1) [shape = circle, draw] {}; 
\node[vertex] (G-3) at (3.0, 1) [shape = circle, draw] {}; 
\node[vertex] (G-4) at (4.5, 1) [shape = circle, draw] {};
\draw[] (G-1) .. controls +(0.5, -0.5) and +(-0.5, -0.5) .. (G-2);
\draw[] (G-3) -- (3.0,0);
\draw[] (G-4) -- (4.5,0);
\draw[] (G-5) .. controls +(0.5, -0.5) and +(-0.5, -0.5) .. (G-6); 
\end{tikzpicture}
\; , \;
\begin{tikzpicture}[scale = 0.35,thick, baseline={(0,1ex)}] 
\tikzstyle{vertex} = [shape = circle, minimum size = 4pt,
    inner sep = 1pt, fill=black] 
\node[vertex] (G-6) at (7.5, 1) [shape = circle, draw] {}; 
\node[vertex] (G-1) at (0.0, 1) [shape = circle, draw] {}; 
\node[vertex] (G-2) at (1.5, 1) [shape = circle, draw] {}; 
\node[vertex] (G-5) at (6.0, 1) [shape = circle, draw] {}; 
\node[vertex] (G-3) at (3.0, 1) [shape = circle, draw] {}; 
\node[vertex] (G-4) at (4.5, 1) [shape = circle, draw] {};
\draw[] (G-5) -- (6.0,0);
\draw[] (G-6) -- (7.5,0);
\draw[] (G-3) .. controls +(0.5, -0.5) and +(-0.5, -0.5) .. (G-4); 
\draw[] (G-1) .. controls +(0.5, -0.5) and +(-0.5, -0.5) .. (G-2); 
\end{tikzpicture}
\big) \; = \; 0$.
\end{enumerate}
One sees this by looking respectively at the three stack
configurations $h^*|h'$
\[
\begin{gathered}
\begin{tikzpicture}[scale = 0.35,thick, baseline={(0,1ex)}] 
\tikzstyle{vertex} = [shape = circle, minimum size = 4pt,
    inner sep = 1pt, fill=black] 
\node[vertex] (G-6) at (7.5, 0) [shape = circle, draw] {}; 
\node[vertex] (G-1) at (0.0, 0) [shape = circle, draw] {}; 
\node[vertex] (G-2) at (1.5, 0) [shape = circle, draw] {}; 
\node[vertex] (G-5) at (6.0, 0) [shape = circle, draw] {}; 
\node[vertex] (G-3) at (3.0, 0) [shape = circle, draw] {}; 
\node[vertex] (G-4) at (4.5, 0) [shape = circle, draw] {};
\draw[] (G-6) .. controls +(-0.5, 0.5) and +(0.5, 0.5) .. (G-5);
\draw[] (G-4) .. controls +(-0.9, 0.9) and +(0.9, 0.9) .. (G-1); 
\draw[] (G-3) .. controls +(-0.5, 0.5) and +(0.5, 0.5) .. (G-2); 
\end{tikzpicture}
  \\
\begin{tikzpicture}[scale = 0.35,thick, baseline={(0,-1ex)}] 
\tikzstyle{vertex} = [shape = circle, minimum size = 4pt,
    inner sep = 1pt, fill=black] 
\node[vertex] (G-6) at (7.5, 0) [shape = circle, draw] {}; 
\node[vertex] (G-1) at (0.0, 0) [shape = circle, draw] {}; 
\node[vertex] (G-2) at (1.5, 0) [shape = circle, draw] {}; 
\node[vertex] (G-5) at (6.0, 0) [shape = circle, draw] {}; 
\node[vertex] (G-3) at (3.0, 0) [shape = circle, draw] {}; 
\node[vertex] (G-4) at (4.5, 0) [shape = circle, draw] {};
\draw[] (G-5) .. controls +(0.5, -0.5) and +(-0.5, -0.5) .. (G-6);
\draw[] (G-3) .. controls +(0.5, -0.5) and +(-0.5, -0.5) .. (G-4); 
\draw[] (G-1) .. controls +(0.5, -0.5) and +(-0.5, -0.5) .. (G-2); 
\end{tikzpicture}
\end{gathered}
\qquad \quad
\begin{gathered}
\begin{tikzpicture}[scale = 0.35,thick, baseline={(0,1ex)}] 
\tikzstyle{vertex} = [shape = circle, minimum size = 4pt,
    inner sep = 1pt, fill=black] 
\node[vertex] (G-6) at (7.5, 0) [shape = circle, draw] {}; 
\node[vertex] (G-1) at (0.0, 0) [shape = circle, draw] {}; 
\node[vertex] (G-2) at (1.5, 0) [shape = circle, draw] {}; 
\node[vertex] (G-5) at (6.0, 0) [shape = circle, draw] {}; 
\node[vertex] (G-3) at (3.0, 0) [shape = circle, draw] {}; 
\node[vertex] (G-4) at (4.5, 0) [shape = circle, draw] {};
\draw[] (G-2) .. controls +(-0.5, 0.5) and +(0.5, 0.5) .. (G-1);
\draw[] (G-5) .. controls +(-0.5, 0.5) and +(0.5, 0.5) .. (G-4); 
\draw[] (G-3) -- (3.0,1);
\draw[] (G-6) -- (7.5,1);
\end{tikzpicture}
  \\
\begin{tikzpicture}[scale = 0.35,thick, baseline={(0,-1ex)}] 
\tikzstyle{vertex} = [shape = circle, minimum size = 4pt,
    inner sep = 1pt, fill=black] 
\node[vertex] (G-6) at (7.5, 0) [shape = circle, draw] {}; 
\node[vertex] (G-1) at (0.0, 0) [shape = circle, draw] {}; 
\node[vertex] (G-2) at (1.5, 0) [shape = circle, draw] {}; 
\node[vertex] (G-5) at (6.0, 0) [shape = circle, draw] {}; 
\node[vertex] (G-3) at (3.0, 0) [shape = circle, draw] {}; 
\node[vertex] (G-4) at (4.5, 0) [shape = circle, draw] {};
\draw[] (G-5) -- (6.0,-1);
\draw[] (G-6) -- (7.5,-1);
\draw[] (G-3) .. controls +(0.5, -0.5) and +(-0.5, -0.5) .. (G-4); 
\draw[] (G-1) .. controls +(0.5, -0.5) and +(-0.5, -0.5) .. (G-2); 
\end{tikzpicture}
\end{gathered}
\qquad \quad
\begin{gathered}
\begin{tikzpicture}[scale = 0.35,thick, baseline={(0,1ex)}] 
\tikzstyle{vertex} = [shape = circle, minimum size = 4pt,
    inner sep = 1pt, fill=black] 
\node[vertex] (G-6) at (7.5, 0) [shape = circle, draw] {}; 
\node[vertex] (G-1) at (0.0, 0) [shape = circle, draw] {}; 
\node[vertex] (G-2) at (1.5, 0) [shape = circle, draw] {}; 
\node[vertex] (G-5) at (6.0, 0) [shape = circle, draw] {}; 
\node[vertex] (G-3) at (3.0, 0) [shape = circle, draw] {}; 
\node[vertex] (G-4) at (4.5, 0) [shape = circle, draw] {};
\draw[] (G-6) .. controls +(-0.5, 0.5) and +(0.5, 0.5) .. (G-5);
\draw[] (G-4) -- (4.5,1);
\draw[] (G-3) -- (3.0,1);
\draw[] (G-2) .. controls +(-0.5, 0.5) and +(0.5, 0.5) .. (G-1); 
\end{tikzpicture}
  \\
\begin{tikzpicture}[scale = 0.35,thick, baseline={(0,-1ex)}] 
\tikzstyle{vertex} = [shape = circle, minimum size = 4pt,
    inner sep = 1pt, fill=black] 
\node[vertex] (G-6) at (7.5, 0) [shape = circle, draw] {}; 
\node[vertex] (G-1) at (0.0, 0) [shape = circle, draw] {}; 
\node[vertex] (G-2) at (1.5, 0) [shape = circle, draw] {}; 
\node[vertex] (G-5) at (6.0, 0) [shape = circle, draw] {}; 
\node[vertex] (G-3) at (3.0, 0) [shape = circle, draw] {}; 
\node[vertex] (G-4) at (4.5, 0) [shape = circle, draw] {};
\draw[] (G-5) -- (6.0,-1);
\draw[] (G-6) -- (7.5,-1);
\draw[] (G-3) .. controls +(0.5, -0.5) and +(-0.5, -0.5) .. (G-4); 
\draw[] (G-1) .. controls +(0.5, -0.5) and +(-0.5, -0.5) .. (G-2); 
\end{tikzpicture}
\end{gathered}
\]
depicted above. Notice that the first two configurations are defect
preserving, but the third is not.
\end{example}

If $n>0$ is even, the bilinear form $\varphi_0$ satisfies a special
property: $\varphi_0(h,h')$ is a positive power of $\delta$, for any
$h,h' \in M(0)$. Hence, if $\delta=0$ then $\varphi_0=0$. Further
analysis reveals the following.

\begin{thm}[\cite{GL:96}*{Cor.~(6.8)}]
  Let $\delta \in \Bbbk$ where $\Bbbk$ is a field. Then
  \[
  \Lambda_0 =
  \begin{cases}
    \Lambda \setminus \{0\} & \text{ if $n>0$ is even and
      $\delta=0$}, \\
    \Lambda & \text{ otherwise}
  \end{cases}
  \]
  gives the indexing set for the isomorphism classes of simple
  $\TL_n(\delta)$-modules.
\end{thm}

In particular, this shows that $\TL_n(0)$ is not semisimple over a
field whenever $n>0$ is even. The representation theory of $\TL_n(0)$
over $\C$ is of great interest in mathematical physics.


Corollary~\ref{c:ss-crit} gave a sufficient condition for
semisimplicity of $\TL_n(\delta)$ in the case when $\delta \ne 0$. We
now give a complete answer to the semisimplicity question.

\begin{thm}
  Let $\Bbbk$ be a field, and fix $0 \ne v \in \Bbbk$. Set $\delta =
  \pm(v+v^{-1})$. Then:
  \begin{enumerate}
  \item If $\delta \ne 0$ then $\TL_n(\delta)$ is semisimple
    if and only if $[n]^!_v \ne 0$ in $\Bbbk$.
  \item If $\delta = 0$ then $\TL_n(0)$ is semisimple if and
    only if
    \[
    \begin{cases}
      \text{$n=0$ or $n$ is odd} & \text{ if $\Bbbk$ has characteristic $0$}\\
      \text{$n=0$ or $n \in \{1,3, \dots, 2p-1\}$} & \text{ if
        $\Bbbk$ has characteristic $p>0$}.
    \end{cases}
    \]
  \end{enumerate}
\end{thm}

Part (a) goes back to \cite{Westbury}, while part (b) was proved in
\cite{Martin}; see also \cite{Ridout-StAubin}.  An easy but somewhat
more sophisticated recent proof (depending on Schur--Weyl duality and
the theory of tilting modules) covering all cases is given in
\cite{AST}*{Prop.~5.1}.


\begin{rmk}
The condition $[n]^!_v \ne 0$ in part (a) is satisfied if and only if
$v^2$ is not an $r$th root of unity unless $r>n$, or $v= \pm1$ and the
characteristic of $\Bbbk$ is strictly greater than $n$.
%
%  NOTE TO TONY, TO BE DELETED: There is something clever going on
%  here. We are using the fact that $\TL_n(\delta) \cong
%  \TL_n(-\delta)$ to be able to include the case $v = -1$, which is
%  explicitly excluded for ``technical reasons'' in \cite{AST}.
\end{rmk}



When $\TL_n(\delta)$ isn't semisimple, its representations over a
field have been understood for a long time. The blocks are known, and
the structure of the indecomposable projective modules are known as
well \cites{GHJ,Martin,GW}; see also \cite{Ridout-StAubin}.  




\section{Schur--Weyl duality}\noindent
Let $\Bbbk$ be a field in this section. Fix $0 \ne v$ in $\Bbbk$ and
assume that $v-v^{-1} \ne 0$ (i.e., $v \ne \pm 1$). The
Drinfeld--Jimbo quantized enveloping algebra associated to the Lie
algebra $\gl_2$ is the associative $\Bbbk$-algebra $\tilde{\UU}:=
\UU(\gl_2)$ with $1$ generated by symbols $E$, $F$, $K_1^{\pm 1}$,
$K_2^{\pm1}$ subject to the defining relations
\begin{equation}
\begin{alignedat}{-1}
  & K_1K_2 = K_2K_1, && K_iK_i^{-1} = 1 = K_i^{-1}K_i \quad(i=1,2)\\
  & K_1 E K_1^{-1} = v E, && K_2 E K_2^{-1} = v^{-1} E \\
  & K_1 F K_1^{-1} = v^{-1} F, && K_2 F K_2^{-1} = v F \\
  & EF - FE = \frac{K-K^{-1}}{v-v^{-1}}, && \text{where $K:= K_1K_2^{-1}$}. 
\end{alignedat}
\end{equation}
The algebra $\tilde{\UU}$ is a Hopf algebra with coproduct $\Delta:
\tilde{\UU} \to \tilde{\UU} \otimes \tilde{\UU}$ and counit $\epsilon:
\tilde{\UU} \to \Bbbk$ given on generators by
\begin{equation}\label{e:coprod}
\begin{alignedat}{-1}
  \Delta(E) &= E \otimes 1 + K \otimes E\\
  \Delta(F) &= F \otimes K^{-1} + 1 \otimes F\\
  \Delta(K_i) &= K_i \otimes K_i~(i=1,2)\\
    \epsilon(E) = \epsilon(F) &= 0, \quad \epsilon(K_i) = 1~(i=1,2).
\end{alignedat}
\end{equation}
We omit the definition of the antipode as it isn't needed here.

\begin{rmk}
There are other ways to define a Hopf algebra structure on
$\tilde{\UU}$; see \cite{Jantzen}*{\S3.8}. For our purposes, this
choice doesn't matter.
\end{rmk}

Let $\UU = \UU(\fsl_2)$ be the subalgebra of $\tilde{\UU} =
\UU(\gl_2)$ generated by $E$, $F$, and $K^{\pm1}$.  The algebra $\UU$
inherits a Hopf algebra structure from that on $\tilde{\UU}$, by
restriction, given by formulas \eqref{e:coprod}.  It is easy to see
that the relations
\begin{equation}\label{e:UU}
\begin{gathered}
  K K^{-1} = 1 = K^{-1}K\\
  K E K^{-1} = v^2 E, \quad K F K^{-1} = v^{-2} F \\
  EF - FE = \frac{K-K^{-1}}{v-v^{-1}}.
\end{gathered}
\end{equation}
are a set of defining relations for $\UU$.  The original definition of
$\UU$ is due to \cites{Drinfeld,Jimbo:85}; the version above is the
slightly modified one given by Lusztig \cite{Lusztig:89}. By letting
$v$ be an element of the ground field instead of taking it to be an
indeterminate, we are following the approach given in the first few
chapters of Jantzen's book \cite{Jantzen}; the books
\cites{Lusztig,Kassel} are also useful general references.



If $v$ is not a root of unity, let $V(n)$ be the simple $\UU$-module
(of type $\mathbf{1}$) of dimension $n+1$; see \cite{Jantzen}*{2.6}.
Let $V= V(1)$ be the natural (or ``vector'') module. Then $\UU$ acts
on $V^{\otimes n}$ by means of the iterated coproduct
$\Delta^{(n)}$, defined inductively by
\[
\Delta^{(2)} = \Delta, \qquad
\Delta^{(k+1)} = (\Delta \otimes 1^{\otimes (k-1)}) \Delta^{(k)}.
\]
Thus, $V^{\otimes n}$ is a $\UU$-module. Under the assumption that $v$
is not a root of unity, it is semisimple as a $\UU$-module. (See
e.g.~\cite{Kassel}*{Thm.~VII.2.2}.)



We now define an action of $\TL_n = \TL_n(\delta)$ on $V^{\otimes n}$,
where $\delta = \pm(v+v^{-1})$. By the quantum Clebsch--Gordon rule
(see \cite{Jimbo:85} or \cite{Kassel}*{\S VII.7}), we have a
decomposition
\begin{equation}\label{e:VV-decomp}
  V \otimes V \cong V(2) \oplus V(0), \quad \text{as $\UU$-modules}.
\end{equation}
The simple $\UU$-module $V=V(1)$ has a basis $\{x_0,x_1\}$ of weight
vectors as in \cite{Kassel}*{Thm.~VI.3.5}. Thus $V \otimes V$ has
basis
\[
\{x_{0,0}, x_{0,1}, x_{1,0}, x_{1,1}\},
\]
where we set $x_{i,j} = x_i \otimes x_j$.  It is easy to check that
the $1$-dimensional summand $V(0)$ in the decomposition
\eqref{e:VV-decomp} is spanned by the vector
\[
z_0 = x_{0,1} - v x_{1,0} = x_0 \otimes x_1 - v x_1 \otimes x_0
\]
and this vector is uniquely determined up to (nonzero) scalar multiple. 

The standard inner product
on $V$ (defined by $\bil{x_i}{x_j} = \delta_{i,j}$ (Kronecker delta)
may be extended to $V \otimes V$ by $\bil{x_{i,j}}{x_{k,l}} =
\delta_{i,k}\delta_{j,l}$.  Then the orthogonal projection of
$V\otimes V$ onto $V(0)$ is given by the matrix
\[
\dot{u} = \frac{1}{v+v^{-1}} \,
\begin{bmatrix}
  0&0&0&0\\
  0 & v^{-1} & -1 & 0\\
  0 & -1 & v & 0\\
  0&0&0&0
\end{bmatrix}
\]
with respect to the above displayed basis. Let $u$ be the linear
endomorphism of $V \otimes V$ given by the matrix $\dot{u}$. Set
$\delta = \pm(v+v^{-1})$ as above. For $i=1,\dots, n-1$, let $e_i$ in
$\TL_n = \TL_n(\delta)$ act on $V^{\otimes n}$ as
\begin{equation}\label{e:e_i}
  e_i = \pm(v+v^{-1})\, 1^{\otimes i-1} \otimes u \otimes 1^{\otimes
    n-i-1}
\end{equation}
where $u$ operates on the copy of $V \otimes V$ embedded in tensor
positions $i,i+1$. Then clearly $e_i$ is a $\UU$-endomorphism of
$V^{\otimes n}$.

\begin{lem}\label{l:TL-action}
The $e_i$ ($i=1,\dots, n-1$ defined by \eqref{e:e_i} satisfy the
defining relations of $\TL_n(\delta)$, with $\delta= \pm(v+v^{-1})$.
Hence, there is a representation $\rho: \TL_n(\delta)
\to \End_\Bbbk(V^{\otimes n})$.
\end{lem}

\begin{proof}
The proof is by a tedious yet elementary calculation, which we omit.
(It suffices to check one of the sign choices for $\delta$, thanks to
the isomorphism $\TL_n(\delta) \cong \TL_n(-\delta)$.)  
\end{proof}

\begin{rmk}
If we define $u_i = 1^{\otimes i-1} \otimes u \otimes 1^{\otimes
  n-i-1}$ for $i = 1,\dots, n-1$ then we get a representation of the
Jones algebra $\A_n((v+v^{-1})^2)$. This can be checked directly and
it is consistent with Proposition~\ref{p:TL-iso}.
\end{rmk}



If $v$ is transcendental, the next result may be deduced from Jimbo
\cite{Jimbo} with some work, using the results of Section
\ref{s:Hecke} and the notation of Remark~\ref{r:HH2}. In Jimbo's
result, the algebra $\HH_n(v^{-1},-v)$ acts (usually non-faithfully)
on $V^{\otimes n}$, and by passing to the corresponding quotient by
the kernel of that action one obtains a faithful action of
$\TL_n(\delta)$.  As \cite{Jimbo} did not include a proof, we will
sketch a proof in a slightly more general context.  First, we recall
the notation
\[
\Lambda(n) = \{n, n-2, \dots, n-2l\}, \quad\text{where } l = \lfloor n/2
\rfloor
\]
from the preceding section.



\begin{thm}[Schur--Weyl duality if $v$ is not a root of unity]\label{t:SWD}
Let $\Bbbk$ be a field, and $0 \ne v \in \Bbbk$. Assume that $v$ is
not a root of unity. Let $\delta = \pm(v+v^{-1})$. Then the above
action of $\TL_n(\delta)$ on $V^{\otimes n}$ commutes with the action
of $\UU = \UU(\fsl_2)$, and these commuting actions induce algebra
surjections
\[
\UU \to \End_{\TL_n(\delta)}(V^{\otimes n}), \quad
\TL_n(\delta) \to \End_{\UU}(V^{\otimes n})
\]
the second of which is actually an isomorphism.  Furthermore,
\[
V^{\otimes n} \cong \textstyle \bigoplus_{k \in \Lambda(n)} V(k) \otimes H(k)
\]
is a decomposition into simple $\UU \otimes \TL_n(\delta)$-modules.
\end{thm}


\begin{proof}
Since $e_i$ is a $\UU$-morphism for each $i$, the actions of $\UU$ and
$\TL_n(\delta)$ commute. Thus, the representation $\rho$ in
Lemma~\ref{l:TL-action} induces an algebra morphism
\begin{equation}\label{e:induced-map}
  \TL_n(\delta) \to \End_\UU(V^{\otimes n}).
\end{equation}
We claim that the dimensions of $\End_\UU(V^{\otimes n})$ and
$\TL_n(\delta)$ are equal, , which implies that the map
\eqref{e:induced-map} is an isomorphism.

Let $m_{n,\lambda}:= [V^{\otimes n}: V(\lambda)]$ be the composition
factor multiplicity of $V(\lambda)$ in $V^{\otimes n}$, as
$\UU$-modules. Then it suffices to show that
\[
m_{n,\lambda} = \dim_\Bbbk H(\lambda)
\quad \text{for each $\lambda \in \Lambda(n)$}.
\]
Again applying the quantum Clebsch--Gordon rule, for any
$\lambda$ in $\Lambda(k-1)$ we have
\[
V(\lambda)\otimes V = V(\lambda)\otimes V(1) \cong
\begin{cases}
  V(\lambda+1) \oplus V(\lambda-1) & \text{ if } \lambda > 0\\
  V(\lambda+1) = V(1) & \text{ if } \lambda=0.
\end{cases}
\]
It follows that the multiplicity $m_{n,\lambda}$ is equal to the
number of paths from $\emptyset$ to $(n-p,p)$ in the Bratteli diagram
(see Figure~\ref{f:Bratteli}), where $\lambda = n-2p$.  By
Lemma~\ref{l:bijections}, $m_{n,\lambda}$ is the number of
half-diagrams on $n$ nodes. This implies the claim.

The final step is to apply Jacobson's density theorem
\cites{Jacobson,Lang} to conclude that $V^{\otimes n}$
satisfies the double-centralizer property as a $\UU$-module. In other
words, the natural map
\[
\UU \to \End_{\TL_n(\delta)}(V^{\otimes n}) \quad \text{is surjective}.
\]
Here we identify $\TL_n(\delta)$ with $\End_\UU(V^{\otimes n})$ by
means of the isomorphism in \eqref{e:induced-map}.  This completes the
proof of both surjectivity statements in Theorem~\ref{t:SWD}. The
final claim in Theorem~\ref{t:SWD} is a standard consequence of the
semisimplicity of $V^{\otimes n}$, so the proof is complete.
\end{proof}

The following consequence of this analysis, which realizes the simple
$\TL_n(\delta)$-modules in tensor space, is worthy of note.

\begin{cor}
  Suppose that $0 = v \in \Bbbk$ is not a root of unity. Let $\delta =
  \pm(v+v^{-1})$. Let
  \[
  V^{\otimes n} = \textstyle \bigoplus_{\lambda \in \pm\Lambda(n)} V^{\otimes
    n}_\lambda
  \]
  be the weight space decomposition.  Let $\mathrm{Max}(\lambda)$ be
  the kernel of the restriction of $E$ to $V^{\otimes n}_\lambda$, for
  each $\lambda \in \Lambda(n)$. Then $\mathrm{Max}(\lambda)$ is the
  space of all maximal vectors of weight $\lambda$, and
  $\mathrm{Max}(\lambda) \cong H(\lambda)$ as $\TL_n(\delta)$-modules.
\end{cor}

\begin{proof}
Each nonzero vector in $\mathrm{Max}(\lambda)$ generates a copy of
$V(\lambda)$ in $V^{\otimes n}$, as $\UU$-modules. Furthermore, the
action of any $e_i$ preserves the weight of a weight vector, so
$\mathrm{Max}(\lambda)$ is a $\TL_n(\delta)$-submodule.
\end{proof}


Takeuchi \cite{Takeuchi} introduced a two-parameter version of the
quantized enveloping algebra of $\gl_n$. Under suitable hypotheses,
Benkart and Witherspoon \cite{BW} give an extension of Jimbo's
Schur--Weyl duality to the two-parameter case.


\begin{rmk}
If $[n]_v^! \ne 0$ and $\delta = \pm(v+v^{-1})\ne 0$, the paper
\cite{DG:orthog} replaces $\UU$ by the quantum Schur algebra
$S_v(2,n)$ of \cite{DJ3} in homogeneous degree $n$ (which acts
faithfully) and constructs an orthogonal basis of maximal vectors
(with respect to the induced inner product on tensor space). This
gives an orthogonal decomposition of tensor space $V^{\otimes n}$ as
$S_v(2,n)$-modules.  This leads to a combinatorial proof of a
slightly stronger version of Theorem~\ref{t:SWD} in which the
hypothesis that $v$ is not a root of unity is weakened to the
condition $[n]_v^! \ne 0$.
\end{rmk}


By excluding roots of unity, Theorem~\ref{t:SWD} does not apply to the
important case $\delta=0$, which corresponds to $v = \pm \sqrt{-1}$ in
$\Bbbk$.  However, Jimbo's version of Schur--Weyl duality was
generalized to include the case of $v\in \Bbbk$ being
non-transcendental (including the case where $v$ is a root of unity)
in \cites{Martin:92,DPS:98}. Thus, the first part of
Theorem~\ref{t:SWD} is known to hold more generally.  For the sake of
completeness, we include a brief sketch of a proof of this general
version of Schur--Weyl duality. For this we need to work with the
divided power form (introduced by Lusztig) of the quantized enveloping
algebra, which we introduce next.

Let $t$ be an indeterminate, $\ZZ = \Z[t,t^{-1}]$ the ring of integral
Laurent polynomials, and $\Q(t)$ its field of fractions. \emph{We now
change notation} and let $\UU$ be the $\Q(t)$-algebra defined by
the usual generators $E,F,K,K^{-1}$ with relations the ones in
\eqref{e:UU}, with $v$ replaced by $t$.  Fix an invertible element
$\xi \in \Bbbk$ (where for the moment $\Bbbk$ is a unital commutative
ring) and regard $\Bbbk$ as a $\ZZ$-algebra via the specialization
morphism $\ZZ \to \Bbbk$ sending each Laurent polynomial $f(t,t^{-1})$
to $f(\xi,\xi^{-1})$.  Then $\UU_\xi = \UU_\xi(\fsl_2)$ is the algebra
\[
\UU_\xi := \UU_\ZZ \otimes_{\ZZ} \Bbbk
\]
where $U_\ZZ$ is the $\ZZ$-submodule of $\UU$ generated by $K$,
$K^{-1}$ and the quantum divided powers $E^m/[m]_t^!$, $F^m/[m]_t^!$
(for $m > 0$). The algebra $\tilde{\UU}_\xi =\UU_\xi(\gl_2)$ is
defined similarly in terms of the analogous $\ZZ$-form
$\tilde{\UU}_\ZZ$. The restriction of $\Delta$, $\epsilon$ to
$\UU_\ZZ$ or $\tilde{\UU}_\ZZ$ makes it a coalgebra over $\ZZ$; in
fact it is a Hopf algebra. So $\UU_\xi$ and $\tilde{\UU}_\xi$ are Hopf
algebras.  Let
\[
V_\xi(m) = V_{\ZZ}(m) \otimes_\ZZ \Bbbk
\]
where $V_\ZZ(m)$ is the $U$-submodule of $V(m)$ generated by a maximal
vector. These modules are the (type $\mathbf{1}$) Weyl modules for
$\UU_\xi$.

We may similarly obtain $\TL_n(\delta)$ for $\delta =
\pm(\xi+\xi^{-1})$ by tensoring its $\ZZ$-form $\TL_{n,\ZZ} =
\TL_{n,\ZZ}(\pm(t+t^{-1}))$ by $\Bbbk$, where $\TL_{n,\ZZ}$ is the
$\ZZ$-algebra defined by the generators and relations in \eqref{e:TL},
with $\delta=\pm(t+t^{-1})$. The action of $\TL_{n,\ZZ}$ on
$V_\ZZ^{\otimes n}$ described earlier induces an action of
$\TL_n(\delta)$, where $\delta = \pm(\xi+\xi^{-1})$, on
$V_\xi^{\otimes n} \cong (V_\ZZ^{\otimes n}) \otimes_\ZZ \Bbbk$, and
this action commutes with the action of $\UU_\xi$.





\begin{thm}[Schur--Weyl duality, general case]
  Let $0 \ne \xi \in \Bbbk$ where $\Bbbk$ is a field.  Set $\delta =
  \pm(\xi+\xi^{-1})$. Let $V_\xi = V_\xi(1)$. Then the commuting
  actions of $\UU_\xi = \UU_\xi(\fsl_2)$ and $\TL_n(\delta)$ on
  $V_\xi^{\otimes n}$ satisfy Schur--Weyl duality; that is, each of
  the induced algebra maps
  \[
  \UU_\xi \to \End_{\TL_n(\delta)}(V_\xi^{\otimes n}), \quad
  \TL_n(\delta) \to \End_{\UU_\xi}(V_\xi^{\otimes n})
  \]
  is surjective. Moreover, the latter map is an isomorphism.  
\end{thm}


\begin{proof}
By a special case of \cite{DPS:98}, there are commuting actions of
$\tilde{\UU}_\xi = \UU_\xi(\gl_2)$ and $\HH_n(-1,\xi^2)$ on
$V_\xi^{\otimes n}$ which satisfy Schur--Weyl duality, where we employ
the two-parameter notation as in Remark~\ref{r:HH2}. So the induced
algebra maps
\[
\tilde{\UU}_\xi \to \End_{\HH_n(-1,\,\xi^2)}(V_\xi^{\otimes n}), \quad
\HH_n(-1,\xi^2) \to \End_{\tilde{\UU}_\xi}(V_\xi^{\otimes n})
\]
are both surjective. We need to argue that $\tilde{\UU}_\xi =
\UU_\xi(\gl_2)$ may be replaced by $\UU_\xi = \UU_\xi(\fsl_2)$ and
$\HH_n(-1,\xi^2)$ by $\TL_n(\delta)$ without altering the
statement. The first such replacement is trivial, and follows from the
fact that the images of $\UU_\xi(\fsl_2)$ and $\UU_\xi(\gl_2)$ in
$\End_\Bbbk(V_\xi^{\otimes n})$ are the same, because the two algebras
differ by a generator that acts as scalars on $V_\xi^{\otimes n}$.

The other replacement is not so trivial. First, argue that
$\HH_n(-1,\xi^2)$ may be replaced by $\HH_n(-v^{-1},v)$; see Remark
\ref{r:HH2}. Then apply \cite{Haerterich} to see that the dimensions
of the kernel and image of the Hecke algebra action is invariant
regardless of the choice of field or specialization $t \mapsto \xi$.
It follows that the image of that action is isomorphic to
$\TL_n(\delta)$. The result now follows by Corollary~\ref{c:TL-quo}.
\end{proof}


\begin{rmk}
If we let $\Bbbk=\C$ and take $\xi = \pm \sqrt{-1}$ then we obtain a
version of Schur--Weyl duality that applies to $\TL_n(0)$, showing in
particular that $\TL_n(0) \cong \End_{\UU_\xi}(V_\xi^{\otimes
  n})$. The same remark applies more generally in any field containing
a square root of $-1$.
\end{rmk}




%%%%%%%%%%%%%%%%%%%%%%%%%%%%%%%%%%%%%%%%%%%%%%%%%%%%%%%%%%%%%
% bibliography using amsrefs package 
%%%%%%%%%%%%%%%%%%%%%%%%%%%%%%%%%%%%%%%%%%%%%%%%%%%%%%%%%%%%%
\begin{bibdiv}
  \begin{biblist}

%\bib{Andersen}{article}{
%   author={Andersen, Henning Haahr},
%   title={Tensor products of quantized tilting modules},
%   journal={Comm. Math. Phys.},
%   volume={149},
%   date={1992},
%   number={1},
%   pages={149--159},
%   issn={0010-3616},
%   review={\MR{1182414}},
%}
	
    
\bib{AST}{article}{
   author={Andersen, Henning Haahr},
   author={Stroppel, Catharina},
   author={Tubbenhauer, Daniel},
   title={Cellular structures using $U_q$-tilting modules},
   journal={Pacific J. Math.},
   volume={292},
   date={2018},
   number={1},
   pages={21--59},
%   issn={0030-8730},
%   review={\MR{3708257}},
%   doi={10.2140/pjm.2018.292.21},
}


\bib{Baxter}{book}{
   author={Baxter, Rodney J.},
   title={Exactly solved models in statistical mechanics},
   publisher={Academic Press, Inc. [Harcourt Brace Jovanovich, Publishers],
   London},
   date={1982},
%   pages={xii+486},
%   isbn={0-12-083180-5},
%   review={\MR{690578}},
}

%\bib{BLM}{article}{
%   author={Beilinson, A. A.},
%   author={Lusztig, G.},
%   author={MacPherson, R.},
%   title={A geometric setting for the quantum deformation of ${\rm GL}_n$},
%   journal={Duke Math. J.},
%   volume={61},
%   date={1990},
%   number={2},
%   pages={655--677},
%   issn={0012-7094},
%   review={\MR{1074310}},
%   doi={10.1215/S0012-7094-90-06124-1},
%}
	

  
\bib{Benkart-Moon}{article}{
   author={Benkart, Georgia},
   author={Moon, Dongho},
   title={Tensor product representations of Temperley--Lieb algebras and
   Chebyshev polynomials},
   conference={
      title={Representations of algebras and related topics},
   },
   book={
      series={Fields Inst. Commun.},
      volume={45},
      publisher={Amer. Math. Soc., Providence, RI},
   },
   date={2005},
   pages={57--80},
%   review={\MR{2146240}},
}

    

\bib{BW}{article}{
  author={Benkart, Georgia}, author={Witherspoon, Sarah},
  title={Representations of two-parameter quantum groups and
    Schur--Weyl duality},
  conference={ title={Hopf algebras}, },
  book={
    series={Lecture Notes in Pure and Appl. Math.},
    volume={237},
    publisher={Dekker, New York},
  },
  date={2004},
  pages={65--92},
%   review={\MR{2051731}},
}

\bib{BFK}{article}{
   author={Bernstein, Joseph},
   author={Frenkel, Igor},
   author={Khovanov, Mikhail},
   title={A categorification of the Temperley-Lieb algebra and Schur
   quotients of $U(\germ{sl}_2)$ via projective and Zuckerman functors},
   journal={Selecta Math. (N.S.)},
   volume={5},
   date={1999},
   number={2},
   pages={199--241},
%   issn={1022-1824},
%   review={\MR{1714141}},
%   doi={10.1007/s000290050047},
}


\bib{Bigelow}{article}{
  author={Bigelow, Stephen},
  title={Braid groups and Iwahori-Hecke algebras},
  conference={ title={Problems on mapping class groups and related topics},
  },
  book={
    series={Proc. Sympos. Pure Math.},
    volume={74},
    publisher={Amer. Math. Soc., Providence, RI}, },
  date={2006},
  pages={285--299},
 %  review={\MR{2264547}},
 %  doi={10.1090/pspum/074/2264547},
}


\bib{Birman-Wenzl}{article}{
   author={Birman, Joan S.},
   author={Wenzl, Hans},
   title={Braids, link polynomials and a new algebra},
   journal={Trans. Amer. Math. Soc.},
   volume={313},
   date={1989},
   number={1},
   pages={249--273},
%   issn={0002-9947},
%   review={\MR{992598}},
%   doi={10.2307/2001074},
}

\bib{Brauer}{article}{
   author={Brauer, Richard},
   title={On algebras which are connected with the semisimple continuous
   groups},
   journal={Ann. of Math. (2)},
   volume={38},
   date={1937},
   number={4},
   pages={857--872},
%   issn={0003-486X},
%   review={\MR{1503378}},
%   doi={10.2307/1968843},
}

\bib{DJ1}{article}{
   author={Dipper, Richard},
   author={James, Gordon},
   title={Representations of Hecke algebras of general linear groups},
   journal={Proc. London Math. Soc. (3)},
   volume={52},
   date={1986},
   number={1},
   pages={20--52},
%   issn={0024-6115},
%   review={\MR{812444}},
%   doi={10.1112/plms/s3-52.1.20},
}


\bib{DJ2}{article}{
   author={Dipper, Richard},
   author={James, Gordon},
   title={Blocks and idempotents of Hecke algebras of general linear groups},
   journal={Proc. London Math. Soc. (3)},
   volume={54},
   date={1987},
   number={1},
   pages={57--82},
%   issn={0024-6115},
%   review={\MR{872250}},
%   doi={10.1112/plms/s3-54.1.57},
}

		
\bib{DJ3}{article}{
   author={Dipper, Richard},
   author={James, Gordon},
   title={The $q$-Schur algebra},
   journal={Proc. London Math. Soc. (3)},
   volume={59},
   date={1989},
   number={1},
   pages={23--50},
%   issn={0024-6115},
%   review={\MR{997250}},
%   doi={10.1112/plms/s3-59.1.23},
}
		

		
\bib{DJ4}{article}{
   author={Dipper, Richard},
   author={James, Gordon},
   title={$q$-tensor space and $q$-Weyl modules},
   journal={Trans. Amer. Math. Soc.},
   volume={327},
   date={1991},
   number={1},
   pages={251--282},
%   issn={0002-9947},
%   review={\MR{1012527}},
%   doi={10.2307/2001842},
}

%\bib{Donkin:gf}{book}{
%   author={Donkin, Stephen},
%   title={Rational representations of algebraic groups},
%   series={Lecture Notes in Mathematics},
%   volume={1140},
%   note={Tensor products and filtration},
%   publisher={Springer-Verlag, Berlin},
%   date={1985},
%   pages={vii+254},
%   isbn={3-540-15668-2},
%   review={\MR{804233}},
%   doi={10.1007/BFb0074637},
%}


%\bib{Donkin}{article}{
%   author={Donkin, Stephen},
%   title={On tilting modules for algebraic groups},
%   journal={Math. Z.},
%   volume={212},
%   date={1993},
%   number={1},
%   pages={39--60},
%   issn={0025-5874},
%   review={\MR{1200163}},
%   doi={10.1007/BF02571640},
%}

%\bib{D:09}{article}{
%   author={Doty, Stephen},
%   title={Schur-Weyl duality in positive characteristic},
%   conference={
%      title={Representation theory},
%   },
%   book={
%      series={Contemp. Math.},
%      volume={478},
%      publisher={Amer. Math. Soc., Providence, RI},
%   },
%   date={2009},
%   pages={15--28},
%   review={\MR{2513263}},
%   doi={10.1090/conm/478/09316},
%}
		

\bib{DG:PTL}{article}{
   author={Doty, Stephen},
   author={Giaquinto, Anthony},
   title={The partial Temperley--Lieb algebra and its representations},
   journal={J. Comb. Algebra},
   volume={7},
   number={3/4},
   pages={401--439},  
   year={2023},
}


\bib{DG:orthog}{article}{
   author={Doty, Stephen},
   author={Giaquinto, Anthony},
   title={An orthogonal realization of representations of the
     Temperley--Lieb algebra},
   status={preprint, %
     \href{https://arXiv.org/abs/2306.17352}{arXiv:2306.17352}},
   year={2023},
}

\bib{Drinfeld}{article}{
   author={Drinfel\cprime d, V. G.},
   title={Hopf algebras and the quantum Yang-Baxter equation},
   journal={Soviet Math. Dokl.},
   volume={32},
   date={1985},
   number={1},
   pages={254--258},
%   review={\MR{802128}},
}



%\bib{Du:95}{article}{
%   author={Du, Jie},
%   title={A note on quantized Weyl reciprocity at roots of unity},
%   journal={Algebra Colloq.},
%   volume={2},
%   date={1995},
%   number={4},
%   pages={363--372},
%   issn={1005-3867},
%   review={\MR{1358684}},
%}

\bib{DPS:98}{article}{
   author={Du, Jie},
   author={Parshall, Brian},
   author={Scott, Leonard},
   title={Quantum Weyl reciprocity and tilting modules},
   journal={Comm. Math. Phys.},
   volume={195},
   date={1998},
   number={2},
   pages={321--352},
%   issn={0010-3616},
%   review={\MR{1637785}},
%   doi={10.1007/s002200050392},
}

\bib{EK}{article}{
   author={Evans, David E.},
   author={Kawahigashi, Yasuyuki},
   title={Subfactors and mathematical physics},
   journal={Bull. Amer. Math. Soc. (N.S.)},
   volume={60},
   number={4},
   date={2023},
   pages={459--482},
}

\bib{FKS}{article}{
   author={Frenkel, Igor},
   author={Khovanov, Mikhail},
   author={Stroppel, Catharina},
   title={A categorification of finite-dimensional irreducible
   representations of quantum $\germ{sl}_2$ and their tensor products},
   journal={Selecta Math. (N.S.)},
   volume={12},
   date={2006},
   number={3-4},
   pages={379--431},
%   issn={1022-1824},
%   review={\MR{2305608}},
%   doi={10.1007/s00029-007-0031-y},
}


%\bib{FP:20}{article}{
%   author={Flores, Steven M.},
%   author={Peltola, Evalina},
%   title={Higher-spin quantum and classical Schur--Weyl duality
%     for $\mathfrak{sl}_2$},
%   eprint={arXiv:2008.06038},
%   status={preprint},
%   year={2020},
%}

\bib{GL:96}{article}{
   author={Graham, J. J.},
   author={Lehrer, G. I.},
   title={Cellular algebras},
   journal={Invent. Math.},
   volume={123},
   date={1996},
   number={1},
   pages={1--34},
%   issn={0020-9910},
%   review={\MR{1376244}},
%   doi={10.1007/BF01232365},
}
	

\bib{GHJ}{book}{
   author={Goodman, Frederick M.},
   author={de la Harpe, Pierre},
   author={Jones, Vaughan F. R.},
   title={Coxeter graphs and towers of algebras},
   series={Mathematical Sciences Research Institute Publications},
   volume={14},
   publisher={Springer-Verlag, New York},
   date={1989},
%   pages={x+288},
%   isbn={0-387-96979-9},
%   review={\MR{999799}},
%   doi={10.1007/978-1-4613-9641-3},
}

%\bib{GV}{article}{
%   author={Gainutdinov, A. M.},
%   author={Vasseur, R.},
%   title={Lattice fusion rules and logarithmic operator product expansions},
%   journal={Nuclear Phys. B},
%   volume={868},
%   date={2013},
%   number={1},
%   pages={223--270},
%   issn={0550-3213},
%   review={\MR{3001127}},
%   doi={10.1016/j.nuclphysb.2012.11.004},
%}

\bib{GW}{article}{
   author={Goodman, Frederick M.},
   author={Wenzl, Hans},
   title={The Temperley-Lieb algebra at roots of unity},
   journal={Pacific J. Math.},
   volume={161},
   date={1993},
   number={2},
   pages={307--334},
%   issn={0030-8730},
%   review={\MR{1242201}},
}
	
    
%\bib{HMR}{article}{
%   author={Halverson, Tom},
%   author={Mazzocco, Manuela},
%   author={Ram, Arun},
%   title={Commuting families in Hecke and Temperley--Lieb algebras},
%   journal={Nagoya Math. J.},
%   volume={195},
%   date={2009},
%   pages={125--152},
%   issn={0027-7630},
%   review={\MR{2552957}},
%   doi={10.1017/S0027763000009740},
%}

\bib{Halverson-Ram}{article}{
   author={Halverson, Tom},
   author={Ram, Arun},
   title={Characters of algebras containing a Jones basic construction: the
   Temperley-Lieb, Okada, Brauer, and Birman-Wenzl algebras},
   journal={Adv. Math.},
   volume={116},
   date={1995},
   number={2},
   pages={263--321},
%   issn={0001-8708},
%   review={\MR{1363766}},
%   doi={10.1006/aima.1995.1068},
}

\bib{Haerterich}{article}{
   author={H\"{a}rterich, Martin},
   title={Murphy bases of generalized Temperley--Lieb algebras},
   journal={Arch. Math. (Basel)},
   volume={72},
   date={1999},
   number={5},
   pages={337--345},
%   issn={0003-889X},
%   review={\MR{1680384}},
%   doi={10.1007/s000130050341},
}

\bib{Jacobson}{book}{
   author={Jacobson, Nathan},
   title={Basic algebra. II},
   publisher={W. H. Freeman and Co., San Francisco, Calif.},
   date={1980},
%   pages={xix+666},
%   isbn={0-7167-1079-X},
%   review={\MR{571884}},
}

\bib{Jantzen}{book}{
   author={Jantzen, Jens Carsten},
   title={Lectures on quantum groups},
   series={Graduate Studies in Mathematics},
   volume={6},
   publisher={American Mathematical Society, Providence, RI},
   date={1996},
%   pages={viii+266},
%   isbn={0-8218-0478-2},
%   review={\MR{1359532}},
%   doi={10.1090/gsm/006},
}

\bib{Jimbo:85}{article}{
   author={Jimbo, Michio},
   title={A $q$-difference analogue of $U({\germ g})$ and the Yang-Baxter
   equation},
   journal={Lett. Math. Phys.},
   volume={10},
   date={1985},
   number={1},
   pages={63--69},
%   issn={0377-9017},
%   review={\MR{797001}},
%   doi={10.1007/BF00704588},
}

\bib{Jimbo}{article}{
   author={Jimbo, Michio},
   title={A $q$-analogue of $U({\germ g}{\germ l}(N+1))$, Hecke algebra,
     and the Yang--Baxter equation},
   journal={Lett. Math. Phys.},
   volume={11},
   date={1986},
   number={3},
   pages={247--252},
%   issn={0377-9017},
%   review={\MR{841713}},
%   doi={10.1007/BF00400222},
}

\bib{Jones:83}{article}{
   author={Jones, V. F. R.},
   title={Index for subfactors},
   journal={Invent. Math.},
   volume={72},
   date={1983},
   number={1},
   pages={1--25},
%   issn={0020-9910},
%   review={\MR{696688}},
%   doi={10.1007/BF01389127},
}

\bib{Jones:85}{article}{
   author={Jones, V. F. R.},
   title={A polynomial invariant for knots via von Neumann algebras},
   journal={Bull. Amer. Math. Soc. (N.S.)},
   volume={12},
   date={1985},
   number={1},
   pages={103--111},
%   issn={0273-0979},
%   review={\MR{766964}},
%   doi={10.1090/S0273-0979-1985-15304-2},
}


\bib{Jones:86}{article}{
   author={Jones, V. F. R.},
   title={Braid groups, Hecke algebras and type ${\rm II}_1$ factors},
   conference={
      title={Geometric methods in operator algebras},
      address={Kyoto},
      date={1983},
   },
   book={
      series={Pitman Res. Notes Math. Ser.},
      volume={123},
      publisher={Longman Sci. Tech., Harlow},
   },
   date={1986},
   pages={242--273},
%   review={\MR{866500}},
}

\bib{Jones}{article}{
   author={Jones, V. F. R.},
   title={Hecke algebra representations of braid groups and link
   polynomials},
   journal={Ann. of Math. (2)},
   volume={126},
   date={1987},
   number={2},
   pages={335--388},
%   issn={0003-486X},
%   review={\MR{908150}},
%   doi={10.2307/1971403},
}

\bib{Jones:91}{book}{
   author={Jones, Vaughan F. R.},
   title={Subfactors and knots},
   series={CBMS Regional Conference Series in Mathematics},
   volume={80},
   publisher={%Published for the Conference Board of the Mathematical
     %Sciences, Washington, DC; by the
     American Mathematical Society, Providence, RI},
   date={1991},
%   pages={x+113},
%   isbn={0-8218-0729-3},
%   review={\MR{1134131}},
%   doi={10.1090/cbms/080},
}


\bib{Jones:09}{article}{
   author={Jones, Vaughan},
   title={On the origin and development of subfactors and quantum topology},
   journal={Bull. Amer. Math. Soc. (N.S.)},
   volume={46},
   date={2009},
   number={2},
   pages={309--326},
%   issn={0273-0979},
%   review={\MR{2476415}},
%   doi={10.1090/S0273-0979-09-01244-0},
}
		


\bib{Kadison}{article}{
   author={Kadison, Lars},
   title={Algebraic aspects of the Jones basic construction},
   journal={C. R. Math. Rep. Acad. Sci. Canada},
   volume={15},
   date={1993},
   number={5},
   pages={223--228},
%   issn={0706-1994},
%   review={\MR{1250711}},
}
		
%\bib{Kaneda}{article}{
%   author={Kaneda, Masaharu},
%   title={Based modules and good filtrations in algebraic groups},
%   journal={Hiroshima Math. J.},
%   volume={28},
%   date={1998},
%   number={2},
%   pages={337--344},
%   issn={0018-2079},
%   review={\MR{1637330}},
%}

\bib{Kassel}{book}{
   author={Kassel, Christian},
   title={Quantum groups},
   series={Graduate Texts in Mathematics},
   volume={155},
   publisher={Springer-Verlag, New York},
   date={1995},
%   pages={xii+531},
%   isbn={0-387-94370-6},
%   review={\MR{1321145}},
%   doi={10.1007/978-1-4612-0783-2},
}

\bib{K:87}{article}{
   author={Kauffman, Louis H.},
   title={State models and the Jones polynomial},
   journal={Topology},
   volume={26},
   date={1987},
   number={3},
   pages={395--407},
%   issn={0040-9383},
%   review={\MR{899057}},
%   doi={10.1016/0040-9383(87)90009-7},
}

\bib{K:88}{article}{
   author={Kauffman, Louis H.},
   title={Statistical mechanics and the Jones polynomial},
   conference={
      title={Braids},
      address={Santa Cruz, CA},
      date={1986},
   },
   book={
      series={Contemp. Math.},
      volume={78},
      publisher={Amer. Math. Soc., Providence, RI},
   },
   date={1988},
   pages={263--297},
%   review={\MR{975085}},
%   doi={10.1090/conm/078/975085},
}
		  

\bib{K:90}{article}{
   author={Kauffman, Louis H.},
   title={An invariant of regular isotopy},
   journal={Trans. Amer. Math. Soc.},
   volume={318},
   date={1990},
   number={2},
   pages={417--471},
%   issn={0002-9947},
%   review={\MR{958895}},
%   doi={10.2307/2001315},
}
	


\bib{Kauffman}{book}{
   author={Kauffman, Louis H.},
   title={Knots and physics},
   series={Series on Knots and Everything},
   volume={53},
   edition={4},
   publisher={World Scientific Publishing Co. Pte. Ltd., Hackensack, NJ},
   date={2013},
%   pages={xviii+846},
%   isbn={978-981-4383-01-1},
%   review={\MR{3013186}},
%   doi={10.1142/8338},
}


%\bib{Kauffman:survey}{article}{
%   author={Kauffman, L. H.},
%   title={Combinatorial knot theory and the Jones polynomial},
%   journal={J. Knot Theory Ramifications},
%   volume={31},
%   date={2023},
%   number={11},
%   pages={Paper No. 2340011, 49},
%   issn={},
%   review={},
%   doi={10.1142/S0218216523400114},
%}


\bib{Kauffman-Lins}{book}{
   author={Kauffman, Louis H.},
   author={Lins, S\'{o}stenes L.},
   title={Temperley--Lieb recoupling theory and invariants of $3$-manifolds},
   series={Annals of Mathematics Studies},
   volume={134},
   publisher={Princeton University Press, Princeton, NJ},
   date={1994},
%   pages={x+296},
%   isbn={0-691-03640-3},
%   review={\MR{1280463}},
%   doi={10.1515/9781400882533},
}


\bib{KL:79}{article}{
   author={Kazhdan, David},
   author={Lusztig, George},
   title={Representations of Coxeter groups and Hecke algebras},
   journal={Invent. Math.},
   volume={53},
   date={1979},
   number={2},
   pages={165--184},
%   issn={0020-9910},
%   review={\MR{560412}},
%   doi={10.1007/BF01390031},
}


\bib{Lang}{book}{
   author={Lang, Serge},
   title={Algebra},
   series={Graduate Texts in Mathematics},
   volume={211},
   edition={3},
   publisher={Springer-Verlag, New York},
   date={2002},
%   pages={xvi+914},
%   isbn={0-387-95385-X},
%   review={\MR{1878556}},
%   doi={10.1007/978-1-4613-0041-0},
}


\bib{Lusztig:89}{article}{
   author={Lusztig, G.},
   title={Modular representations and quantum groups},
   conference={
      title={Classical groups and related topics},
      address={Beijing},
      date={1987},
   },
   book={
      series={Contemp. Math.},
      volume={82},
      publisher={Amer. Math. Soc., Providence, RI},
   },
   date={1989},
   pages={59--77},
%   review={\MR{982278}},
%   doi={10.1090/conm/082/982278},
}

%\bib{Lusztig:cbII}{article}{
%   author={Lusztig, G.},
%   title={Canonical bases arising from quantized enveloping algebras. II},
%   note={Common trends in mathematics and quantum field theories (Kyoto,
%   1990)},
%   journal={Progr. Theoret. Phys. Suppl.},
%   number={102},
%   date={1990},
%   pages={175--201 (1991)},
%   issn={0375-9687},
%   review={\MR{1182165}},
%   doi={10.1143/PTPS.102.175},
%}

\bib{Lusztig}{book}{
   author={Lusztig, George},
   title={Introduction to quantum groups},
   series={Progress in Mathematics},
   volume={110},
   publisher={Birkh\"{a}user Boston, Inc., Boston, MA},
   date={1993},
%   pages={xii+341},
%   isbn={0-8176-3712-5},
%   review={\MR{1227098}},
}

\bib{Lu:Hecke}{book}{
   author={Lusztig, G.},
   title={Hecke algebras with unequal parameters},
   series={CRM Monograph Series},
   volume={18},
   publisher={American Mathematical Society, Providence, RI},
   date={2003},
%   pages={vi+136},
%   isbn={0-8218-3356-1},
%   review={\MR{1974442}},
%   doi={10.1090/crmm/018},
}



\bib{Martin}{book}{
   author={Martin, Paul Purdon},
   title={Potts models and related problems in statistical mechanics},
   series={Series on Advances in Statistical Mechanics},
   volume={5},
   publisher={World Scientific Publishing Co., Inc., Teaneck, NJ},
   date={1991},
%   pages={xiv+344},
%   isbn={981-02-0075-7},
%   review={\MR{1103994}},
%   doi={10.1142/0983},
}


\bib{Martin:92}{article}{
   author={Martin, Paul Purdon},
   title={On Schur-Weyl duality, $A_n$ Hecke algebras and quantum ${\rm
   sl}(N)$ on $\bigotimes^{n+1}{\bf C}^N$},
   conference={
      title={Infinite analysis, Part A, B},
      address={Kyoto},
      date={1991},
   },
   book={
      series={Adv. Ser. Math. Phys.},
      volume={16},
      publisher={World Sci. Publ., River Edge, NJ},
   },
   date={1992},
   pages={645--673},
%   review={\MR{1187568}},
%   doi={10.1142/S0217751X92003975},
}


	
%\bib{Mathieu}{article}{
%   author={Mathieu, Olivier},
%   title={Filtrations of $G$-modules},
%   journal={Ann. Sci. \'{E}cole Norm. Sup. (4)},
%   volume={23},
%   date={1990},
%   number={4},
%   pages={625--644},
%   issn={0012-9593},
%   review={\MR{1072820}},
%}
	

		
\bib{Murphy1}{article}{
   author={Murphy, G. E.},
   title={On the representation theory of the symmetric groups and
   associated Hecke algebras},
   journal={J. Algebra},
   volume={152},
   date={1992},
   number={2},
   pages={492--513},
%   issn={0021-8693},
%   review={\MR{1194316}},
%   doi={10.1016/0021-8693(92)90045-N},
}

\bib{Murphy2}{article}{
   author={Murphy, G. E.},
   title={The representations of Hecke algebras of type $A_n$},
   journal={J. Algebra},
   volume={173},
   date={1995},
   number={1},
   pages={97--121},
%   issn={0021-8693},
%   review={\MR{1327362}},
%   doi={10.1006/jabr.1995.1079},
}

%\bib{P-SA}{article}{
%   author={Provencher, Guillaume},
%   author={Saint-Aubin, Yvan},
%   title={The idempotents of the ${\rm TL}_n$-module $\bigotimes^n\Bbb{C}2$
%   in terms of elements of ${\rm U}_q\germ{sl}_2$},
%   journal={Ann. Henri Poincar\'{e}},
%   volume={15},
%   date={2014},
%   number={11},
%   pages={2203--2240},
%   issn={1424-0637},
%   review={\MR{3268828}},
%   doi={10.1007/s00023-013-0297-x},
%}

%\bib{Paradowski}{article}{
%   author={Paradowski, Jan},
%   title={Filtrations of modules over the quantum algebra},
%   conference={
%      title={Algebraic groups and their generalizations: quantum and
%      infinite-dimensional methods},
%      address={University Park, PA},
%      date={1991},
%   },
%   book={
%      series={Proc. Sympos. Pure Math.},
%      volume={56},
%      publisher={Amer. Math. Soc., Providence, RI},
%   },
%   date={1994},
%   pages={93--108},
%   review={\MR{1278729}},
%   doi={10.1090/pspum/056.2/1278729},
%}



\bib{PS}{book}{
   author={Prasolov, V. V.},
   author={Sossinsky, A. B.},
   title={Knots, links, braids and 3-manifolds},
   series={Translations of Mathematical Monographs},
   volume={154},
%   note={An introduction to the new invariants in low-dimensional topology;
%   Translated from the Russian manuscript by Sossinsky [Sosinski\u{\i}]},
%   publisher={American Mathematical Society, Providence, RI},
   date={1997},
%   pages={viii+239},
%   isbn={0-8218-0588-6},
%   review={\MR{1414898}},
%   doi={10.1090/mmono/154},
}


\bib{Ridout-StAubin}{article}{
   author={Ridout, David},
   author={Saint-Aubin, Yvan},
   title={Standard modules, induction and the structure of the
   Temperley-Lieb algebra},
   journal={Adv. Theor. Math. Phys.},
   volume={18},
   date={2014},
   number={5},
   pages={957--1041},
%   issn={1095-0761},
%   review={\MR{3281274}},
}


%\bib{Ringel}{article}{
%   author={Ringel, Claus Michael},
%   title={The category of modules with good filtrations over a
%   quasi-hereditary algebra has almost split sequences},
%   journal={Math. Z.},
%   volume={208},
%   date={1991},
%   number={2},
%   pages={209--223},
%   issn={0025-5874},
%   review={\MR{1128706}},
%   doi={10.1007/BF02571521},
%}

\bib{Stroppel}{article}{
   author={Stroppel, Catharina},
   title={Categorification of the Temperley-Lieb category, tangles, and
   cobordisms via projective functors},
   journal={Duke Math. J.},
   volume={126},
   date={2005},
   number={3},
   pages={547--596},
%   issn={0012-7094},
%   review={\MR{2120117}},
%   doi={10.1215/S0012-7094-04-12634-X},
}

\bib{Takeuchi}{article}{
   author={Takeuchi, Mitsuhiro},
   title={A two-parameter quantization of ${\rm GL}(n)$ (summary)},
   journal={Proc. Japan Acad. Ser. A Math. Sci.},
   volume={66},
   date={1990},
   number={5},
   pages={112--114},
%   issn={0386-2194},
%   review={\MR{1065785}},
}


\bib{TL}{article}{
   author={Temperley, H. N. V.},
   author={Lieb, E. H.},
   title={Relations between the ``percolation'' and ``colouring'' problem
   and other graph-theoretical problems associated with regular planar
   lattices: some exact results for the ``percolation'' problem},
   journal={Proc. Roy. Soc. London Ser. A},
   volume={322},
   date={1971},
   number={1549},
   pages={251--280},
%   issn={0962-8444},
%   review={\MR{498284}},
%   doi={10.1098/rspa.1971.0067},
}


%\bib{Wang}{article}{
%   author={Wang, Jian Pan},
%   title={Sheaf cohomology on $G/B$ and tensor products of Weyl modules},
%   journal={J. Algebra},
%   volume={77},
%   date={1982},
%   number={1},
%   pages={162--185},
%   issn={0021-8693},
%   review={\MR{665171}},
%   doi={10.1016/0021-8693(82)90284-8},
%}

\bib{Wenzl:annals}{article}{
   author={Wenzl, Hans},
   title={On the structure of Brauer's centralizer algebras},
   journal={Ann. of Math. (2)},
   volume={128},
   date={1988},
   number={1},
   pages={173--193},
%   issn={0003-486X},
%   review={\MR{951511}},
%   doi={10.2307/1971466},
}


\bib{Wenzl:93}{article}{
   author={Wenzl, Hans},
   title={Subfactors and invariants of $3$-manifolds},
   conference={
      title={Operator algebras, mathematical physics, and low-dimensional
      topology},
      address={Istanbul},
      date={1991},
   },
   book={
      series={Res. Notes Math.},
      volume={5},
      publisher={A K Peters, Wellesley, MA},
   },
   date={1993},
   pages={301--324},
%   review={\MR{1259073}},
}

\bib{Westbury}{article}{
   author={Westbury, B. W.},
   title={The representation theory of the Temperley-Lieb algebras},
   journal={Math. Z.},
   volume={219},
   date={1995},
   number={4},
   pages={539--565},
%   issn={0025-5874},
%   review={\MR{1343661}},
%   doi={10.1007/BF02572380},
}


%\bib{Wilcox}{article}{
%   author={Wilcox, Stewart},
%   title={Cellularity of diagram algebras as twisted semigroup algebras},
%   journal={J. Algebra},
%   volume={309},
%   date={2007},
%   number={1},
%   pages={10--31},
%   issn={0021-8693},
%   review={\MR{2301230}},
%   doi={10.1016/j.jalgebra.2006.10.016},
%}

\end{biblist}
\end{bibdiv}
\end{document}

%%%%%%%%%%%%%%%%%%%%%%%%%%%%%%%%%%%%%%%%%%%%%%%%%%%%%%%%%%%%%%%%%






		

		
