\documentclass[runningheads]{llncs}
\pdfoutput=1
\usepackage[utf8]{inputenc} % allow utf-8 input
\usepackage[T1]{fontenc}    % use 8-bit T1 fonts
\usepackage{hyperref}       % hyperlinks
\usepackage{url}            % simple URL typesetting
\usepackage{booktabs}       % professional-quality tables
\usepackage{amsfonts}       % blackboard math symbols
\usepackage{nicefrac}       % compact symbols for 1/2, etc.
\usepackage{microtype}      % microtypography
\usepackage{xcolor}         % colors
\usepackage{textcomp}

\usepackage{booktabs} % For formal tables
\usepackage{amsmath,amssymb} % define this before the line numbering.
\usepackage{graphicx}
\usepackage{multirow}
\usepackage{soul,color}
\usepackage{bm}
\usepackage{dsfont}
\usepackage{enumitem}
\usepackage{algorithmic}
\usepackage[linesnumbered,ruled,vlined]{algorithm2e}
\usepackage{booktabs}
\usepackage{caption}
% \usepackage{subcaption}
\usepackage{url}
\usepackage{cite}
% \newtheorem{prop}{Proposition}
% \newtheorem{theorem}{Theorem}
% \newtheorem{lemma}{Lemma}
% \newtheorem{remark}{Remark}
% \newtheorem{definition}{Definition}
% \newtheorem{corollary}{Corollary}
\usepackage{wrapfig, blindtext}
\usepackage{etoolbox}
\usepackage{bbm}
\usepackage{float} 
\usepackage{makecell}
\usepackage{setspace}
\usepackage{amsfonts}
\usepackage{subfig}


\newcommand{\fdl}{federated learning\ }
\newcommand{\Fdl}{Federated learning\ }
\newcommand\revise{\colorbox{yellow}{\textcolor{red}{REVISE}}}
\newcommand\att{\colorbox{yellow}{\textcolor{red}{ATTENTION}}}
\newcommand{\etal}{et al.}
\newcommand{\ra}[1]{\renewcommand{\arraystretch}{#1}}

\DeclareMathOperator*{\argmax}{arg\,max}
\DeclareMathOperator*{\argmin}{arg\,min}

\newcommand{\pt}{pre-training\ }
\newcommand{\Pt}{Pre-training\ }
\newcommand{\llm}{large language model\ }
\newcommand{\LLM}{Large Language Model\ }
\newcommand{\LLMs}{Large Language Models\ }
\newcommand{\synm}{synonymous\ }
\newcommand{\hlc}[2][yellow]{{%
    \colorlet{foo}{#1}%
    \sethlcolor{foo}\hl{#2}}%
}

\renewcommand\UrlFont{\color{blue}\rmfamily}

\begin{document}
%
\title{Low-Parameter Federated Learning with Large Language Models}
%
\titlerunning{Low-Parameter Federated Learning with Large Language Models}
% If the paper title is too long for the running head, you can set
% an abbreviated paper title here
%

\author{Jingang Jiang  \and Xiangyang Liu \and
Chenyou Fan}
%
\authorrunning{J. Jiang et al.}
% First names are abbreviated in the running head.
% If there are more than two authors, 'et al.' is used.
%

\institute{South China Normal University, Guangdong, China \\
\email{\{2022024923,2022024952\}@m.scnu.edu.cn, fanchenyou@scnu.edu.cn}}



\maketitle              % typeset the header of the contribution
%
\begin{abstract}
% We study few-shot Natural Language Understanding (NLU) tasks with Large Language Models (LLMs) in a federated learning scenarios. This is a practical yet challenging task due to the insufficient labeled data and limited communication capacities in common Federated Learning (FL) scenarios, such as learning with mobile devices. Recent studies show that LLMs for NLU can be prompted to perform few-show NLU tasks, such as sentiment analysis and arithmetic reasoning. 
% % However, the enormous parameter size leads to substantial communication costs in federated learning, rendering the implementation of the conventional FL approaches such as FedAvg unfeasible. 
% However, the huge sizes of LLMs result in formidable computation and communication costs so that classical FL schemes such as FedAvg can be impractical, especially with mobile devices.
% To address these challenges, we propose a novel method called Low-Parameter Federated Learning (LP-FL), which combines the few-shot prompt learning capabilities inherited from LLMs with our proposed efficient communication and federating techniques. 
% % Our approach  enables federated clients to assign soft labels to unlabeled data by utilizing the knowledge embedded in the global model. This is achieved through iterative and semi-supervised training, effectively leveraging both labeled and unlabeled data to tackle data insufficiency. 
% % To reduce communication costs, our LP-FL utilize the Low-Rank Adaptation (LoRA) technique to achieve efficient parameter fine-tuning and federation. To further tackle the data insufficiency issue, we design a training progress which continually expands the dataset by incorporating both labeled and soft-labeled data. 
% Our LP-FL approach addresses the data insufficiency issue by enabling federated clients to assign soft labels to unlabeled data using the knowledge gradually learned and emerged in the global model. This is achieved through iterative soft-label assigning which  expands the labeled set continually through the FL process. Additionally, to reduce computation and communication costs, LP-FL utilizes the Low-Rank Adaptation (LoRA) technique for compact learnable parameter construction, efficient local model fine-tuning, and affordable global model federation. 
% %We also design a training process that continually expands the dataset by incorporating both labeled and soft-labeled data.
% We found that our proposed LP-FL can solve the sentiment analysis task in various FL settings consistently better than Full-Parameter Federated Learning (FP-FL). Our method can equal or even surpass centralized training in few-shot settings due to its resistance to over-fitting.
We study few-shot Natural Language Understanding (NLU) tasks with Large Language Models (LLMs) in federated learning (FL) scenarios. It is a challenging task due to limited labeled data and communication capacities in FL, especially with mobile devices. Recent studies show LLMs can be prompted to perform few-shot NLU tasks like sentiment analysis and arithmetic reasoning. However, the huge sizes of LLMs result in high computation and communication costs, making classical FL schemes impractical. To address these challenges, we propose Low-Parameter Federated Learning (LP-FL). LP-FL combines few-shot prompt learning from LLMs with efficient communication and federating techniques. Our approach enables federated clients to assign soft labels to unlabeled data using gradually learned knowledge from the global model. Through iterative soft-label assigning, we continually expand the labeled set during the FL process. Additionally, to reduce computation and communication costs, LP-FL utilizes the Low-Rank Adaptation (LoRA) technique for compact learnable parameter construction, efficient local model fine-tuning, and affordable global model federation. LP-FL consistently outperforms Full-Parameter Federated Learning (FP-FL) in sentiment analysis tasks across various FL settings. Its resistance to over-fitting allows LP-FL to equal or surpass centralized training in few-shot scenarios.

% 联邦环境下各客户端无法收集到足够多的标签数据用来训练一直是研究者们关注的问题,大模型prompt learning在包括情感分析在内的很多文本任务表现出不错的few-shot能力,但是大模型的参数规模太大导致联邦通信成本高昂,因此我们设计了一个既能很好利用prompt learning的few-shot能力,又能降低通信成本的训练方法(Fed-LoRA)。我们设计了一个迭代的半监督的训练过程,每个客户端先利用本地的少量有标签数据用类似继续mlm预训练的方式微调模型,为了降低通信成本,我们使用lora参数高效微调,联邦lora的参数,客户端使用联邦后的全局模型给无标签数据标注软标签,使用原本的标签数据和标注后的软标签数据继续训练,利用规模越来越大的数据集训练几代模型。在情感分析任务上,最后我们的方法相比与集中式训练下(cen-lora)的模型只损失2%左右的性能,甚至在传统文本分类的全参数微调上也有一定提高。



\keywords{Federated Learning \and Large Language Model \and Prompt Learning \and Low-Rank Adaptation \and Sentiment Analysis.}
\end{abstract}
%
%
%

\section{Introduction}
Current quantum hardware is unable to carry out universal quantum computations due to the buildup of errors that occur during the computation. 
The magnitude of the individual error is currently above the value that the Threshold Theorem requires in order to kick-start quantum error correction and fault-tolerant quantum computation~\cite[Section 10.6]{nielsen_chuang_2010}. 
Although the experimentally achieved fidelity rates are promising and the error bounds are inching closer to the required threshold, we will have to work for the foreseeable future with quantum hardware with errors that build-up during the computation.  This implies that we can only do a limited number of steps before the output of the computation has become completely uncorrelated with the intended one.

For fault-tolerant quantum computing, we repeat four steps: 
1) We apply a number of single and two-qubit quantum gates, in parallel whenever possible; 
2) We perform a syndrome measurement on a subset of the qubits; 
3) We perform fast classical computations to determine which errors have occurred and how to correct them; 
and, 4) We apply correction terms based on the classical computations.
We then repeat these four steps with a next sequence of gates. 
These four steps are essential to fault-tolerant quantum computing. 


The starting point of this work is to use the four steps outlined above, not to carry out error correction and fault-tolerant computation, but to enhance short, constant-depth, {\em uncorrected} quantum circuits that perform single qubit gates and {\em nearest-neighbor} two qubit gates. 
Since in the long run we will have to implement error-correction and fault-tolerant computation anyhow, and this is done by such a four-step process, why not make other use of this architecture? Moreover, on some of the quantum hardware platforms, these operations are already in place.
Embracing this idea we naturally arrive at the question: what is the computational power of \textit{low-depth} quantum-classical circuits organized as in the four steps outlined above? 
We thus investigate circuits that execute a small, ideally constant, number of stages, where at each stage we may apply, in parallel, single qubit gates and {\em nearest-neighbor} two qubit gates, followed by measurements, followed by low-depth classical computations of which the outcome can control quantum gates in later stages. 
It is not clear, at first, whether such circuits, especially with constant depth, can do anything remotely useful. 
But we will see that this is indeed the case: many quantum computations can be done by such circuits in constant depth. 
By parallelizing quantum computations in this way, we improve the overall computational capabilities of these circuits, as we do not incur errors on qubits that are idle, simply because qubits are not idle for a very long time. 
Furthermore, reducing the depth of quantum circuits, at the cost of increasing width, allows the circuit to be run faster even if errors occur.

The first usage of such a four-step layout, not to do error correction, but to perform computations, can be found in the paradigm of measurement-based quantum computing~\cite{gottesman1999demonstrating,raussendorf2001one,jozsa2006introduction,clark2007generalised}: 
A universal form of quantum computing where a quantum state is prepared and operations are performed by measuring qubits in different bases, depending on previous measurements and intermediate measurements.

\citeauthor{PhamSvore2013} were the first to formalize the four-step protocol for performing computations~\cite{PhamSvore2013}. They included specific hardware topologies by considering two-dimensional graphs for imposing constraints on qubit interactions. In their model, they develop circuits for particularly useful multi-qubit gates, including specifying costs in the width, number of qubits, depth, number of concurrent time steps, size, and total number of non-Identity operations.
As a result, they find an algorithm that factors integers in polylogarithmic depth.
\citeauthor{Browne:2011} showed that the main tool in the work by \citeauthor{PhamSvore2013}, the fan-out gate, can also be replaced by additional log-depth classical computations in the measurement-based quantum computing setting~\cite{Browne:2011}.

More recently, \citeauthor{Cirac:2021} introduced a scheme to implement unitary operations involving quantum circuits combined with Local Operations and Classical Communication ($\mathsf{LOCC}$) channels: $\mathsf{LOCC}$-assisted quantum circuits~\cite{Cirac:2021}. Similarly to the four-step scheme we just described, they allow for a short depth circuit to be run on the qubits, followed by one round of $\mathsf{LOCC}$, in which ancilla qubits are measured and local unitaries are applied based on the measurement outcomes. They show that in this model any 1D transitionally invariant matrix-product state (MPS) with fixed bond dimension is in the same phase of matter as the trivial state. Similar ideas can be found in~\cite{TVV_NonAbelianTopologicalOrder_2022, tantivasadakarn2021long}.

In this work, we introduce a new model, called \textit{Local Alternating Quantum-Classical Computations} ($\LAQCC$). In this model we alternate between running quantum circuits (constrained by locality), ending in the measurement of a subset of qubits, and fast classical computations based on the measurement results. The outcome of the classical computations are then used to control future quantum circuits. We allow for flexibility in this model, by giving different constraints to the power of both the quantum circuits and the classical circuits as well as the number of alternations between them. 
Most attention will be given to $\LAQCC$ containing quantum circuits of constant depth, classical circuits of logarithmic depth and at most a constant number of alternations between them. 
Any circuit constructed in this model is considered to be of constant depth. 
We restrict ourselves to logarithmic depth classical computations, as this is the first natural and non-trivial extension beyond constant-depth classical computations. 
Constant-depth classical computations do however also have an equivalent constant-depth quantum implementation.

The definition of $\LAQCC$ sharpens the original definition of \citeauthor{PhamSvore2013} by adding constraints to the intermediate classical computations. This allows us to bound the power of $\LAQCC$ from above. 

The main result of \citeauthor{Cirac:2021}, that 1D translational invariant MPS with fixed bond dimension can be prepared by $\mathsf{LOCC}$-assisted circuits, relies on local symmetries of the MPS. These symmetries allow them to prepare local states (on a constant number of qubits) and glue them together by doing one round of the appropriate entangling measurement and corrections, after which they run a round of local unitaries to get the desired result. This general scheme for preparing states that exhibit an MPS description with the appropriate local symmetries requires only geometrically local unitaries and one round of measurement and corrections an therefore is accessible in $\LAQCC$. Studying different local symmetries, known as Symmetry Protected Topological (SPT) phases of matter, to find measurement-based constant depth circuits for states is a broad ongoing field of research~\cite{TVV_NonAbelianTopologicalOrder_2022, tantivasadakarn2021long, smith2023deterministic}. 
All these schemes have a $\LAQCC$ implementation.

%$\LAQCC$-circuits also exist for general schemes of preparing local states, based on the local tensors, and gluing them together using one round of entangled measurement and corrections, based on the local symmetry. 
%The main result of \citeauthor{Cirac:2021}, that 1D translational invariant MPS with fixed bond dimension can be prepared by $\mathsf{LOCC}$-assisted circuits, relies heavily on local symmetries of the MPS and as a result also has an equivalent $\LAQCC$ implementation. 
%The corrections applied after the measurement round are local unitaries depending on the local symmetries of the MPS. 

 

%This general scheme of preparing local states, based on the local tensors, and gluing it together by doing one round of entangled measurement and corrections, based on the local symmetry, is accessible in $\LAQCC$.
Note however that \citeauthor{Cirac:2021} also suggest a circuit for the $W$-state.
This circuit uses sequentially and dependent measurement-based corrections of the ancilla qubits. 
These dependent measurements translate to sequential alternations between the quantum and classical circuits and therefore increase the total depth to linear depth, exceeding the constant-depth constraints imposed by $\LAQCC$-circuits. 

We study the power of the $\LAQCC$ model with respect to state preparation, showing that even with only constant quantum-depth and logarithmic classical depth it remains possible to prepare states with long-range entanglement.
Another surprising result is that it is unlikely that $\LAQCC$ circuits are classically simulatable. We show that any instantaneous quantum polynomial-time (IQP) circuit~\cite{Bremner2010,Shepherd2009} has an $\LAQCC$ implementation.
Classical simulation of IQP circuits implies the collapse of the polynomial hierarchy to the third level, which is not believed to be true~\cite{Bremner2017}. Therefore, we expect that $\LAQCC$ circuits are unlikely to be classically simulatable. We bound the power of $\LAQCC$ by showing that it is contained in $\QNC^1$, the class of polynomial-size, log-depth circuits.

Next, we also study the power that intermediate classical calculations can add to quantum computations, by considering a new model that alternates between polynomially many polynomial-depth quantum circuits and unbounded classical computations
We study this model by doing a complexity theoretical analysis, where we draw inspiration from the notions of complexity given by \citeauthor{RosenthalYuen:2022}, \citeauthor{MetgerYuen:2023}, and \citeauthor{Aaronson:2004}.
All three complexity notions are based on the notion of state preparation, instead of more traditional definition of complexity such as the decidability of a computational problem. 
The first two consider classes based on sequences of quantum states preparable by a polynomial-sized quantum circuit, where the circuits are uniformly generated by a computational class, for instance, the class $\mathsf{PSPACE}$, which results in the complexity class $\mathsf{StatePSPACE}$~\cite{RosenthalYuen:2022,MetgerYuen:2023}.
The third notion considers a relative complexity, where the complexity is measured between two given states, and is measured by the number of gates, from a given gate-set, required to transform one state in another state~\cite{Aaronson:2004}. 
For our definition of state preparation complexity, we drop the uniformity constraint from~\cite{RosenthalYuen:2022,MetgerYuen:2023} and define a class as $\mathsf{StateX}$, which refers to states preparable by circuits of type $\mathsf{X}$. 
As an example, if $\mathsf{X} = \QNC^0$, this results in the class $\mathsf{StateQNC^0}$, which is the set of states preparable from the $\ket{0}^n$ state by poly-size constant-depth circuits. 
This notion is similar to the relative complexity from~\cite{Aaronson:2004}, where one state is the  $\ket{0}^n$ state and instead of counting the number of gates we consider the set of states preparable by a fixed number of gates. Using this notion of complexity we show that any state preparable by an $\LAQCC^*$ circuit is also preparable by a $\mathsf{PostQPoly}$ circuit, the class of circuits of polynomial depth with an additional post-selection gate. 

All Clifford circuits have a constant-depth $\LAQCC$ implementation, implying that any stabilizer state can be implemented by a constant-depth $\LAQCC$ circuit, see Section~\ref{sec:clifford_circuits} for a proof of this statement. 
Efficient circuits for stabilizer states have been known already through measurement-based quantum computing. Therefore this paper focuses on the preparation of non-stabilizer states, and as a surprising result we find novel constant-depth protocols for four very natural classes of non-stabilizer states.
Despite the extensive research into these four classes of non-stabilizer states and the many applications of them, no efficient constant- or low-depth state preparation protocols are known yet. We specifically consider these four classes as they are all often used as initial states in other algorithms.

The first state is a uniform superposition over an arbitrary number of states. 
This state finds applications in many quantum algorithms, as they often start with a uniform superposition over multiple states. 
This superposition is often achieved by applying Hadamard gates to every qubit due to its simplicity to prepare. 
Yet, the analysis of many algorithms, such as Shor's algorithm~\cite{Shor:1997}, would benefit from a different initial superposition. 
The circuit to prepare the uniform superposition over an arbitrary number of states uses an exact version of Grover search as a subroutine, that turns a probabilistic circuit, with a known constant probability of success, into a deterministic circuit. 
We use the circuit for preparing a uniform superposition over an arbitrary number of states as a subroutine in the next two quantum state preparation protocols. 

The second state is the $W$-state, the uniform superposition over all computational basis states of Hamming-weight~$1$, a natural long-ranged entangled state that displays a fundamentally nonequivalent type of entanglement from the Greenberger–Horne–Zeilinger state~\cite{WState:2000}, for which $\LAQCC$-type constant-depth circuits were previously known~\cite{PhamSvore2013, Cirac:2021}. 
The $W$-state is often used as benchmark for new quantum hardware~\cite{Haffner2005,Neeley2010,GarciaPerez:2021}. 
A novel way to prepare the $W$-state therefore gives a new way to benchmark different quantum devices with each other. 
A circuit for preparing the $W$-state was given in~\cite{Cirac:2021}, but this implementation requires sequentially alternating measurements followed by local unitaries, which in the $\LAQCC$ model is not considered to be of constant depth. 
We improve this protocol by giving an $\LAQCC$ implementation of the $W$-state, based on a compress-uncompress method that links the one-hot and binary encoding of integers.

The third state considered is the Dicke state, a generalization of the $W$-state, a superposition over all computational basis states with Hamming-weight $k$~\cite{Dicke:1954}. 
Dicke states have relevance in various practical settings.
For instance, for quantum game theory~\cite{zdemir2007}, quantum storage~\cite{Bacon_Compress:2006,Plesch:2010}, quantum error correction~\cite{ouyang2014permutation}, quantum metrology~\cite{toth2012multipartite}, and quantum networking~\cite{prevedel2009experimental}. 
Dicke states have been used as a starting state for variational optimization algorithms, most notably Quantum Alternating Operator Ansatz (QAOA)~\cite{Hadfield2019}, to find solutions to problems such as Maximum k-vertex Cover~\cite{Brandhofer2022,cook2020quantum}.
The ground states of physical Hamiltonians describing one-dimensional chains tend to show a resemblance to Dicke states such as states resulting from the Bethe ansatz, making them an ideal starting state when investigating the ground state behavior of these Hamiltonians~\cite{TDL_BetheAnsatzDerivation:2010,B_ExcitedStateQuantumPhaseTransitions:2013,DickeTransitions:2021}. 
For instance, the algorithm by \citeauthor{van2021preparing}, who give an algorithm to prepare the Bethe ansatz eigenstates of the spin-1/2 XXZ spin chain, starts by first preparing a Dicke state~\cite{van2021preparing}. 
A Dicke-state preparation protocol based on the compress-uncompress methodology used in the $W$-state furthermore finds applications in entanglement distillation, where the entanglement of a large state is concentrated on only a few qubits. 
Efficient deterministic circuits for preparing Dicke states have been proposed by \citeauthor{bartschi2019deterministic}~\cite{bartschi2019deterministic, bartschi2022deterministic_short_depth}. 
They provide a quantum circuit of depth $\mathO(k \log(\frac{n}{k}))$, allowing arbitrary connectivity, to prepare a Dicke state, which they conjecture to be optimal when $k$ is constant. 
In this work, we provide a constant-depth $\LAQCC$ circuit below their conjectured bound already for constant $k$. 
However, this does not directly disprove their conjecture, as we allow for intermediate measurements and classical computations. 
More significantly, we even construct constant-depth $\LAQCC$ circuits for $k = \mathO(\sqrt{n})$ greatly improving their bound.
This construction extends the compress-uncompress method for the $W$-state combined with additional subroutines. 

We continue with a log-depth state preparation protocol for the Dicke-state for arbitrary $k$. 
This protocol implements an efficient transformation between the factoradic number representation and the combinatorial number representation of a positive integer. 
The combinatorial number representation relates directly to the Dicke state. 
The provided efficient transformation between number representation systems might be of independent interest. 

We conclude by modifying our protocol for preparing a Dicke-state to a protocol that prepares quantum many-body scar states in constant-depth. 
These states have low entanglement and longer coherence times than states with similar energy density.
These characteristics make many-body scar states interesting to analyze and relevant within physics.
Many-body scar states appear for instance in the AKLT model~\cite{AKLT:1987,MRBAR:2018,MRB:2018} and different spin models~\cite{SI:2019,MOBFR:2020}.
Known methods for preparing these states have polynomial-depth~\cite{Gustafson:2023}, whereas our circuit has constant depth. 

% We conclude by studying the power that intermediate classical calculations can add to quantum computations. 
% In this study, we define a new model that relaxes constant-depth quantum circuits to polynomial depth quantum circuits, log-depth classical calculations to unbounded classical computations and a constant number of alternations to a polynomial number of alternations. 
% We call this model $\LAQCC^*$. 
% We study this model by doing a complexity theoretical analysis, where we draw inspiration from the notions of complexity given by \citeauthor{RosenthalYuen:2022}, \citeauthor{MetgerYuen:2023}, and \citeauthor{Aaronson:2004}.
% All three complexity notions are based on the notion of state preparation, instead of more traditional definition of complexity such as the decidability of a computational problem. 
% The first two consider classes based on sequences of quantum states preparable by a polynomial-sized quantum circuit, where the circuits are uniformly generated by a computational class, for instance, the class $\mathsf{PSPACE}$, which results in the complexity class $\mathsf{StatePSPACE}$~\cite{RosenthalYuen:2022,MetgerYuen:2023}.
% The third notion considers a relative complexity, where the complexity is measured between two given states, and is measured by the number of gates, from a given gate-set, required to transform one state in another state~\cite{Aaronson:2004}. 
% For our definition of state preparation complexity, we drop the uniformity constraint from~\cite{RosenthalYuen:2022,MetgerYuen:2023} and define a class as $\mathsf{StateX}$, which refers to states preparable by circuits of type $\mathsf{X}$. 
% As an example, if $\mathsf{X} = \QNC^0$, this results in the class $\mathsf{StateQNC^0}$, which is the set of states preparable from the $\ket{0}^n$ state by poly-size constant-depth circuits. 
% This notion is similar to the relative complexity from~\cite{Aaronson:2004}, where one state is the  $\ket{0}^n$ state and instead of counting the number of gates we consider the set of states preparable by a fixed number of gates. Using this notion of complexity we show that any state preparable by an $\LAQCC^*$ circuit is also preparable by a $\mathsf{PostQPoly}$ circuit, the class of circuits of polynomial depth with an additional post-selection gate. 

\paragraph{Summary of results}
\begin{itemize}
    \item We give a new definition of a computational model that captures the power of the four step process: applying a constant number of layers of one- and two-qubit gates; performing a syndrome measurement; perform a fast classical computation determining corrections; apply corrections. We call this model \emph{Local Alternating Quantum Classical Computations}, or $\LAQCC$ for short. In this model we bound the allowed quantum operations, intermediate classical calculations, and number of rounds separately. In Section~\ref{sec:LAQCC_model} we define this model and give a list of operations based on results from literature contained in this computational model. In some of these operations we explicitly use that we allow for multiple, but at most constant, rounds  of corrections.
    \item  We show show that there exist $\LAQCC$ circuits that can not be weakly simulated in Section~\ref{sec:IQP_in_LAQCC}. We further show that for every $\LAQCC$ circuit there exists a $\QNC^1$ circuit simulating it perfectly, in Section~\ref{sec:LAQCC_in_QNC1}.
    \item We introduce a new type computational complexity for preparing states and show that the extension of $\LAQCC$ where we allow a polynomial number of rounds and unbounded classical computation, is contained in $\mathsf{PostQPoly}$, the class of polynomial circuits with post-selection, in Section~\ref{sec:Complexity results}.
    \item We show a protocol to prepare the uniform superposition state of size $q$ in $\LAQCC$ using $\mathO(\ceil{\log_2(q)}^2)$ qubits in Section~\ref{sec:superposition_modulo_q}. 
    \item We show a protocol to prepare the $W_n$ state in $\LAQCC$ using $\mathO(n\log(n))$ qubits in Section~\ref{sec:W_state_in_LAQCC}.
    \item We show two ways of preparing the Dicke-$(n,k)$ state. The first method is in $\LAQCC$, works up to $k = \mathO(\sqrt{n})$, uses $\mathO(n^2\log(n))$ qubits, and is found in Section~\ref{sec:dicke:small_k}. The second method is in $\LAQCC\text{-}\mathsf{LOG}$ (an extension of $\LAQCC$ allowing for logarithmic number of alterations instead of constant), works for any $k$, uses $\mathO(\text{poly}(n))$ qubits, and is found in Section~\ref{sec:Dicke_in_LAQCC_LOG}. 
    \item We extend on our $\LAQCC$ method of generating Dicke-$(n,k)$ states for $k = \mathO(\sqrt{n})$ and show a protocol to generate many-body scar states for a particular Hamiltonian in $\LAQCC$ (Section~\ref{sec:many_body_scar}). 
\end{itemize}
Summarized in a table, we provide the following state generation protocols:
\begin{table}[htb]
\centering
\begin{tabular}{l|l|l|l}
\textbf{State description} & \textbf{Width} & \textbf{Depth} & \textbf{Implementation}\\
\hline 
Uniform superposition mod $q$: $\frac{1}{\sqrt{q}} \sum_{i = 0}^{q-1}\ket{i}$ & $\mathO(\ceil{\log^2 q})$ & $\mathO(1)$ & Section~\ref{sec:superposition_modulo_q}\\

$W$-state: $\frac{1}{\sqrt{n}}\sum_{i = 0}^{n-1}\ket{e_i}$ & $\mathO(n \log n)$ & $\mathO(1)$ & Section~\ref{sec:W_state_in_LAQCC}\\

Dicke-$(n,k)$, $k = \mathO(\sqrt{n})$: $\binom{n}{k}^{-1/2}\sum_{x \in \{0,1\}^n: |x| = k} \ket{x}$ &  $\mathO(n^2\log n)$ & $\mathO(1)$ 
&Section~\ref{sec:dicke:small_k}\\

Dicke-$(n,k)$: $\binom{n}{k}^{-1/2}\sum_{x \in \{0,1\}^n: |x| = k} \ket{x}$ & $\mathO(\text{poly}(n))$ & $\mathO(\log n)$ &Section~\ref{sec:Dicke_in_LAQCC_LOG}\\

QMBS: $\ket{S_k} = \frac{1}{k! \sqrt{\mathcal N(n,k)}}(Q^\dagger)^k \ket{\Omega}$ &  $\mathO(n^2\log n)$ & $\mathO(1)$  &  Section~\ref{sec:many_body_scar}
\end{tabular}
\caption{Summary of state preparation protocols given in this paper.}
\label{tab:sate_prep}
\end{table}
In the entry for the quantum many-body scar state $Q$ denotes the raising operator and $\mathcal N(n,k)=\binom{n-k-1}{k}$. 
Section~\ref{sec:many_body_scar} will provide more details on the variables and the implementation. 

\paragraph{Organization of the paper}
\noindent We first introduce relevant preliminaries in Section~\ref{sec:preliminaries}. 
In Section~\ref{sec:LAQCC_model} we formally define the class of Local Alternating Quantum-Classical Computations ($\LAQCC$). We also show that any Clifford circuit can be implemented in constant depth $\LAQCC$ (a result based on a result from measurement-based quantum computing~\cite{jozsa2006introduction}). 
This result allows us to give many useful multi-qubit gates and routines in Section~\ref{sec:gates_created_in_LAQCC}. 
Beyond that we show that constant depth $\LAQCC$ circuits are contained in $\QNC^1$ and that any $\mathsf{IQP}$ circuit has an $\LAQCC$ implementation.
We conclude this section with an analysis of a more powerful instantiation of $\LAQCC$ and show an inclusion with respect to the class $\mathsf{PostQPoly}$, which is the class of circuits of polynomial depth with one additional post-selection gate. 
In Section~\ref{sec:state_prep_in_LAQCC} we give $\LAQCC$ circuit implementations for preparing the uniform superposition over an arbitrary number of states, the $W$-state and the Dicke state up to $k = \mathO(\sqrt{n})$. We furthermore give a log-depth circuit implementation for preparing the Dicke state for any $k$. We conclude by showing a $\LAQCC$ circuit for generating many body scar states of a particular type of Hamiltonian.


\section{Related Work}
\label{appsec: related work}
Bayesian causal discovery literature has primarily focused on inference in linear models with closed-form posteriors or marginalized parameters. Early works considered sampling directed acyclic graphs (DAGs) for discrete~\cite{cooper1992bayesian, madigan1995bayesian, heckerman2006bayesian} and Gaussian random variables~\cite{friedman2003being, tong2001active} using Markov chain Monte Carlo (MCMC) in the DAG space. However, these approaches exhibit slow mixing and convergence~\cite{eaton2012bayesian,grzegorczyk2008improving}, often requiring restrictions on number of parents~\cite{kuipers2017partition}. %Alternative exact dynamic programming methods are limited to small settings~\cite{koivisto2012advances}. 

Recent advances in variational inference~\cite{zhang2018advances} have facilitated graph inference in DAG space, with gradient-based methods employing the NOTEARS DAG penalty \cite{zheng2018dags}.\cite{annadani2021variational} samples DAGs from autoregressive adjacency matrix distributions, while \cite{lorch2021dibs} utilizes Stein variational approach \cite{liu2016stein} for DAGs and causal model parameters. \cite{cundy2021bcd} proposed a variational inference framework on node orderings using the gumbel-sinkhorn gradient estimator \cite{mena2018learning}. \cite{deleu2022bayesian,nishikawa2022bayesian} employ the GFlowNet framework \cite{bengio2021gflownet} for inferring the DAG posterior. Most methods, except\cite{lorch2021dibs} are restricted to linear models, while \cite{lorch2021dibs} has high computational costs and lacks DAG generation guarantees compared to our method.
% at least quadratic scaling complexity, both with respect to the number of nodes (due to the DAG penalty) as well as number of posterior samples. Our proposed approach instead has linear complexity with respect to number of posterior samples and does not require any additional DAG penalty.     

In contrast, \emph{quasi-Bayesian} methods, such as DAG bootstrap \cite{friedman2013data}, demonstrate competitive performance. DAG bootstrap resamples data and estimates a single DAG using PC \cite{spirtes2000causation}, GES \cite{chickering2002optimal}, or similar algorithms, weighting the obtained DAGs by their unnormalized posterior probabilities. Recent neural network-based works employ variational inference to learn DAG distributions and point estimates for nonlinear model parameters \cite{charpentier2022differentiable,geffner2022deep}.
In this section, we describe how to learn repair strategies from the  unsafe programs and edits collected in Section~\ref{sec:data}. We define a \dsl (Section~\ref{subsec:dsl}) to express repair strategies that take an \pdg of an unsafe program  as input and generate a safe program as output. The DSL is expressive and can even express bad strategies that don't generalize well to programs in the wild. We provide examples of such bad strategies and good strategies that generalize well  (Section~\ref{subsec:examples}). We learn good repair strategies  in a data-driven manner using an example-based synthesis algorithm (Section~\ref{subsec:synthesis}). %Finally, given a new unsafe program and a set of learned repair strategies, we apply these strategies and generate  candidate repairs (Section~\ref{subsec:applying}).


%Our goal is to use the collected data to learn high-level general repair strategies. We learn these repair strategies over a joint representation of the \astree with the annotations inferred from the \sa tool (the representation referred to as \pdg ahead).  These inferred \sa tool annotations allow us to take the advantage of rich semantic information while performing \unsure{repairs}. Figure ~\ref{fig:example1-pdg} shows an example \pdg corresponding to the unsafe code shown in Figure ~\ref{fig:unsafememberex}. We develop a powerful \dsl that can utilize the annotations in the \pdg structure and learns repair strategies using a deductive synthesis algorithm. More specifically, strategies in this \dsl operate over the \pdg structure of unseen code-snippets and suggest appropriate edits correspondingly. \aksays{The following sentence can be removed if space becomes a constraint.} Section~\ref{subsec:dsl} describes the \dsl, Section~\ref{subsec:synthesis} talks about the synthesis algorithm, and Section~\ref{subsec:applying} demonstrates strategies in this \dsl can be applied. 


\subsection{\dsl for repair strategies}
\label{subsec:dsl}
We introduce a novel \dsl to express repair strategies in Figure~\ref{fig:fixing-dsl}.
%that use the knowledge of program semantics annotated on \pdg instead of just using the syntactic program structure and in-turn are more expressive and generalize better. These strategies take the an unsafe-program as input and return candidate repair programs by performing tree-edit-operations.
At a high level, the strategies define a three-step process where  they provide a computation to identify the edit-location node \editloc, a computation to identify the child index $\editindex$ of \editloc where repair happens, and a computation to generate the AST that must be placed at  index $\editindex$ of \editloc for the repair. The main part of these computations involve traversing paths of the input unsafe program \prog.
%The edit-operation can either be inserting a syntactic-child at \editloc (\insertsc) or replacing a syntactic-child with another tree at \editloc (\replace). The index at the \astree-node $\editloc$ where the insertion or replacement occurs is called the edit-index (\editindex). The tree that is inserted or replaces another existing tree at the \editloc is materialized hierarchically for the given example by defining abstract program structure using a combination of constant structure and references to \astree-nodes in the existing program \prog. These \astree-nodes are called reference-locations (\refloc). To find these locations (\editloc, \refloc) in a given program, the strategies abstractly store \textit{traversals} which materialize into a \textit{concrete} \astree-node in the given programs.  

%\naman{todo - talk about traversal in the introduction, background etc.}
\input{dsl}

%We present our \dsl in Figure ~\ref{fig:fixing-dsl}. 
%The DSL is a list of definitions for various non-terminals in the grammar. For each non-terminal, we define a corresponding type and a set of production rules. Each production rule is either a fixed expression, or an operator applied to other non-terminals or fixed-expressions in the grammar.  
The top-level production rule of the DSL defines strategies, \strategy, with type \newtextsc{Strategy}. 
%A \node is either the source node (\prog.source) or an application of \traversal on another \node. 
\gettraversal, \getclauses, and \getindex are all functions that take a \node $n$ as input and return a \node, \bool, and \integer as output respectively. The edit-AST, \eastree, is similar to a syntactic variant of \astree (i.e. no semantic edges) which we define in Section~\ref{sec:data} with one addition. It has reference nodes that, when applying the strategy to the input \pdg of \prog, are materialized from sub-trees of this \pdg, where the root nodes of these sub-trees are identified by traversing paths in the input. 
%Finally, edge-type (denoted by \edgetype) is an enumeration describing the type of edge, i.e. syntactic or semantic, and parent or child, as defined in Section~\ref{sec:data}. 

% Given these types, we now define the operators used in our \dsl. 
The strategy \strategy is of two types, \insertsc or \replace. \DMethod{Insert}{\I{L}}{\I{I}}{\I{O}}\ declaratively expresses the computation that computes the edit-location \editloc by traversing the path supplied in \I{L}, then computes \editindex, the index of edit-location,  by evaluating \I{I}(\editloc), and inserts the materialization of \I{O} as a syntactic-child \astree at index \editindex of the edit-location \editloc. \DMethod{Replace}{\I{L}}{\I{I}}{\I{O}}\ is similar and performs a replacement instead of an insertion.
%computes the \editloc and \editindex, and replaces syntactic-child of \editloc at \editindex with \I{O}.
%The insertion and replacement operations modify the nodes $\mathcal{N}$ and edges $\mathcal{E}$ of the \astree (Figure~\ref{fig:astsyntax}) appropriately. 

\node (\I{L}) is either the node corresponding to the source of vulnerability (\prog.source) or the target of the path corresponding to the traversal \DMethod{ApplyTraversal}{L}{\I{F}$_k\ o\ $\I{F}$_{k-1}\ o\ \cdots$\I{F}$_0$}. 
Here, each \I{F}$_i$ 
is a function that takes a node 
$n$ as input, performs a traversal from $n$, and returns the traversal's target node $n'$. 
Thus, \T{ApplyTraversal} can be recursively defined as \DMethod{ApplyTraversal}{\I{F}$_0$(L)}{\I{F}$_k\ o\ $\I{F}$_{k-1}\ o\ \cdots$\I{F}$_1$}\ if $k>0$ and \I{F}$_0$(L) otherwise. 

\newtextsc{GetTraversal} (\I{F}) defines a function that takes a node $n$ and returns a node $n'$ reachable from $n$ and can be of two types. Given $n$, the \DMethod{GetEdge}{\I{ET}}{\I{I}}\ operator first finds the possible single-edge traversals of type \I{ET} and indexes it using \I{I}. Specifically, if edge type \I{ET} is a parent then it returns the parent of $n$. Otherwise, 
it finds a set of $N$ of nodes that are connected with $n$ via the edge type \I{ET}, i.e., $N = \mathcal{E}(n, \I{ET})$, and returns the node $N[I(\I{n})]$ at the index given by $I$. In contrast, $\DMethod{GetKleeneStar}{\I{ET}}{\I{C}}(n)$  performs a \newtextsc{KleeneStarTraversal} that iteratively traverses edges of type \I{ET}, staring from input node $n$, until it reaches an edge whose target  node $n^{i}$ satisfies the condition defined by the clause \I{C}. Formally, \newtextsc{KleeneStarTraversal} can be defined recursively as $KE(n_1,ET,C) = \I{C}(n_1)? n_1 : \left(let\ t\in\mathcal{E}(n_1,ET)\ in\ KE(t,ET,C)\right)$. Here, the node $t$, which is target of an edge with source $n_1$ and type $ET$,  is chosen non-deterministically and our implementation resolves this non-determinism through a breadth-first search.
%A \traversal is a relation between nodes $n_1$ and $n_2$ such that there is an edge or a sequence of edges between them. 

\newtextsc{GetIndex} (\I{I}) defines a function that takes a node $n$ and returns a \integer. It is either a constant function that returns a fixed integer $z$ or a \DMethod{GetOffsetIndex}{\I{L}, \I{z}}. \DMethod{GetOffsetIndex}{\I{L}, \I{z}}\ takes a node $n$ as input and returns an integer $DO(n,L)+z$, where $DO(n_1,n_2)$ returns the index of syntactic child of $n_2$ who is a syntactic ancestor of $n_1$. 

\eastree (\I{O}) defines the edit \astree with reference nodes which, given an input program \prog, are materialized to a concrete \astree. The \eastree can either be a \T{ConstantAST} or a \T{ReferenceAST}. Specifically, \DMethod{ConstantAST}{$\tau$}{\I{value}}{\I{O}$_1$}{\I{O}$_2$}{$\cdots$}{{\I{O}$_k$}}\ returns an \eastree that has a type $\tau$, string representation \I{value}, and is recursively constructed with sub-trees \I{O}$_1 \cdots$ \I{O}$_k$ as syntactic children, each of which can either be a \T{ConstantAST} or a \T{ReferenceAST}. The \DMethod{ReferenceAST}{\I{L}}, when applying the strategy, finds a node $n$ in \prog by traversing the path described in \I{L} and returns a copy of the (syntactic) sub-tree of \prog rooted at $n$. %Next, we show examples of strategies written in this DSL and how to learn them automatically.

% Finally, note that the traversals can be composed by applying multiple \T{ApplyTraversal} operators sequentially. We use this key insight into developing our learning from examples setup. 
\newcommand{\newwrapbox}[2]{\adjustbox{margin=1pt 1.3pt, bgcolor=white, frame=1pt, cframe=#1, color=#1}{#2}}
% Figure environment removed


\lstMakeShortInline[columns=fixed]@
\subsection{Example of strategies in our \dsl}
\label{subsec:examples}
Figure~\ref{fig:repair-strategy-ex1} describes   two possible repair strategies that are sufficient to repair the motivating example in Figure~\ref{fig:vulnerabilty-example1}. We first describe the good strategy in Figure~\ref{fig:strat1}, referred to as \strategyone,  and then compare it with the bad strategy \strategytwo in Figure~\ref{fig:strat2}. 

Given the program \prog in Figure~\ref{fig:vulnerabilty-example1}(a) as input, the strategy \strategyone
performs a replacement at index \I{I} of edit-location $L_e$ with the materialization of \I{O} (line 20 of \strategyone).
This process requires first finding the "semantic location" node \semloc. %The semantic location for \prog is shown in red in Figure~\ref{fig:example1-pdg}.
To this end, the strategy 
first  traverses a path from the node annotated as \T{source} by \sa  using \DMethod{GetKleeneStar}\ in Line~\ref{lst:line:semkleene} of \strategyone.  This \newtextsc{KleeneStarTraversal} starts from \T{source}, traverses semantic dataflow edges, and stops at a node 
corresponding to an identifier being used as the function name in a function call. 
 For the input program $P$, the traversal takes the semantic-child-edges 1-7 (Figure~\ref{fig:example1-pdg}) and stops at @foo@ in Line~\ref{lst:line:callerId-sink} of Figure~\ref{fig:vulnerabilty-example1}(a). Next, to reach the edit-location $L_e$, the strategy uses a \newtextsc{KleeneStarTraversal} that starts from \semloc, traverses syntactic parent edges,  and stops when it reaches a \blockstmt. For $P$, this traversal sets $L_e$  as the node corresponding to the \blockstmt between Lines~\ref{lst:line:handlers-run} and ~\ref{lst:line:handlers-run-end} of Figure~\ref{fig:vulnerabilty-example1}(a). Next, in Line~\ref{lst:line:offseteditindex} of \strategyone, the index \I{I} is set to the index corresponding to the  syntactic child of the edit-location $L_e$ who is an ancestor of the semantic location $L_s$ . For $P$, this index  materializes into $13$; the edge  outgoing from blue \blockstmt in Figure~\ref{fig:example1-pdg} to an ancestor of semantic location (shown in red) has label \T{ch:13}. Next, we materialize the \eastree defined in Line~\ref{lst:line:eastree} of \strategyone by  materializing the  reference-nodes. The \eastree \I{O} serializes into @if (REF1.hasOwnProperty(REF2)) { REF3 } @ where @REF1@, @REF2@, and @REF3@ correspond to \T{ReferenceAST} operators with locations as \reflocone, \refloctwo, and \reflocthree. \refloctwo traverses semantic-parent edges  from \semloc (Line~\ref{lst:line:goodref}) and materialize into @data.id@. Similarly, \reflocone and \reflocthree traverse syntactic children edges and materialize into @handlers@ and @foo(data);@ respectively. Thus, the \eastree \I{O} materializes  into @if (handlers.hasOwnProperty(data.id)) { foo(data); }@, which is the required repair. 

%When \strategyone is given the program \prog in Figure~\ref{fig:vulnerabilty-example1}(a) as input, then it first  traverses a path from the source node to the "semantic location"  \semloc using \DMethod{GetKleeneStar}{"SemChild"}{\DMethod{GetClause}{"Expr"}}\ in Line~\ref{lst:line:semkleene}. This leads to a \newtextsc{KleeneStarTraversal} with the stopping condition $\lambda n.\mathcal{T}[n] = \text{"CallExpr"}$. For the input program, the traversal skips through the semantic-child-edges 1-7 and reaches @foo@ in Line~\ref{lst:line:callerId-sink}. Next, it applies another \newtextsc{KleeneStarTraversal} starting from \semloc to reach \editloc in Line~\ref{lst:line:synkleene}. This traversal skips over syntactic-parent-edges and reaches the \blockstmt between Lines~\ref{lst:line:handlers-run} and ~\ref{lst:line:handlers-run-end}. Next, in Line~\ref{lst:line:offseteditindex}, the index \I{I} is computed as \DMethod{GetOffsetIndex}{Ls, 0}\ which means to pick the child-index of \editloc that has \semloc as its descendent. For our example, this index would materialize into the statement number in the block statement containing @foo@, which turns out to be $13$. Next, we instantiate the \eastree in Line~\ref{lst:line:eastree} which hierarchically defines the children-nodes or reference-nodes. The \eastree \I{O} deserializes into @if (REF1.hasOwnProperty(REF2)) { REF3 } @ where @REF1@, @REF2@, and @REF3@ correspond to \T{ReferenceAST} operators with locations as \reflocone, \refloctwo, and \reflocthree. \reflocone and \refloctwo use the semantic-parent edge traversals from \semloc (Line~\ref{lst:line:goodref}) and materialize into @handlers@ and @data.id@. \reflocthree performs a syntactic-child edge traversal from \editloc and materializes into @foo(data);@ thus materializing the entire \eastree \I{O} into @if (handlers.hasOwnProperty(data.id)) { foo(data); }@, i.e. the required repair. 

Now consider the repair strategy \strategytwo in Figure~\ref{fig:strat2}. This strategy shares a similar structure with the earlier strategy but differs in the way traversals and the index $\I{I}$ are computed. There are four key differences
\begin{enumerate}
    \item In order to reach \semloc from \prog.source, \strategytwo performs the \T{EdgeTraversal} using semantic-child edge seven times in Line~\ref{lst:line:nosemkleene}. The number of semantic edges varies widely across programs and prevents generalization to other scenarios. \T{KleeneStarTraversal} operator instead uses \newtextsc{Clauses} over nodes to find the edit-location.
    \item To reach $L_e$ from \semloc, \strategytwo performs the \T{EdgeTraversal} using syntactic-parent edge seven times in Line~\ref{lst:line:nosynkleene}. Consider a program that instead assigns output of the function-call @let out = foo(data)@. \strategytwo will find \assignexpr as the edit-location and fail to generalize whereas \strategyone will appropriately adjust and take four parent steps.
    \item In order to compute the index at which replacement needs to occur, \strategytwo uses a \DMethod{ConstantIndex}{13}\ in Line~\ref{lst:line:consteditindex} of Figure~\ref{fig:strat2}, which effectively assumes that replacement should always occur at 13$^{th}$ child of $L_e$ and again doesn't generalize. \strategyone on the other hand uses of \T{GetOffsetIndex} operator to instead compute index dynamically for a given input program
    \item In order to materialize reference nodes, \strategytwo uses syntactic edge traversals (Line~\ref{lst:line:badref} of Figure~\ref{fig:strat2}) which assume definite structure about the structure of the program (@GetConstant(7)@ used as syntactic child index to solve a long-ranged-dependency). \strategyone instead uses semantic-parent edges to capture the semantics here and produces a better generalizing repair.
\end{enumerate} 

\lstDeleteShortInline@

\noindent These programs highlight that our \dsl is expressive enough to perform complicated non-local repairs in a generic manner. At the same time, while many strategies can repair a given program, all applicable strategies are not equally good. A key realization is that we \emph{prefer shorter traversal functions} (\newtextsc{KleeneStarTraversal}\ over a long sequence of \newtextsc{EdgeTraversal}). Similarly, we \textit{prefer the traversals with none or small constants}. For example, we prefer \DMethod{GetOffsetIndex}{\semloc}{0}\  over \DMethod{GetConstant}{13}\ and semantic-parent traversal over syntactic-parent traversal with index \DMethod{GetConstant}{7}. %Finally, we also \emph{prefer strategies that share traversals across localizing \editloc and \refloc}.
We use these insights to guide the search in our synthesis algorithm.

% The strategy (\strategy) is defined by 
% performs this localization using an edit path (\editpath). We define a path (\genpath) in the strategy as a sequence of edges in the \prog. An edge is either a syntactic \astree edge or a semantic \taintpropedge in either direction (i.e. towards parent or child). 
% The localized node in the \prog is called edit location (\editloc). Next, the strategy either inserts or replaces a child of the edit location with a new \astree. This new \astree can either be a constant node or reference a node in the original \prog using a reference path (\rfpath). Figure ~\ref{fig:approach-notations} summarizes the notations 

% This \dsl was created so we can use the \sa annotations seamlessly and is guided by how humans fix such vulnerabilities. A line of previous works~\cite{} manually write repair patterns for fixing code. Our \dsl-based approach is strictly more general as it can perform various kinds of repairs and the exact repair strategies are learned from data. Moreover, we make effective use of high-level patterns and domain insights, and annotations. So while these other approaches tend to be simplistic and \textbf{either do not generalize well or over-generalize (generate a large number of false positives)}, concrete instantiations of strategies in our \dsl are better at capturing the high-level repair intent better. Following we describe the terminologies used in the repair \dsl.

% \lstMakeShortInline[columns=fixed]@
% The top level rule in our \dsl defines the Repair Strategy (denoted by \strategy). It is parameterized by the type of vulnerability the strategy fixes and the edit \edit. We consider two kinds of edits, either an insert operation or a replace operation. This means that the edit \edit either inserts an \astree child or replaces an \astree child with another \astree. Since we are fixing taint-flow vulnerabilities, we found these two operations to be sufficient. However, our \dsl can be expanded to also handle delete operations \aksays{Why can't we say that delete is replacement with an empty tree?}. In Figure ~\ref{fig:vulnerabilty-example1}, the fix applied in replaces the \astree corresponding to @handlers[callerId](data)@ (line ~\ref{lst:line:callerId-sink}, Figure ~\ref{fig:unsafememberex}) with the if statement in lines ~\ref{lst:line:fix-start}-\ref{lst:line:fix-end}, (Figure ~\ref{fig:safememberex}) and depicts a replace edit. 

% %\paragraph{Edit (\edit)} Since we are solving source-sink-sanitizer vulnerabilities, our \edit either inserts child \astree at edit-locations (denoted by \editloc) or replaces a child with another \astree (at \editloc). %This \editloc is a node in the \pdg which is reachable from the vulnerability source (as provided by the \astree) by traversing syntactic (\astree) or semantic edges in \pdg. %Once the \editloc is found, the new \astree (either replacing the existing child being inserted as a child) can be   

% Notice that in the \pdg, @handlers[callerId](data)@ is a child of the \blockstmt (parenthesis block between lines ~\ref{lst:line:handlers-run}-\ref{lst:line:handlers-run-end} and marked in blue in Figure ~\ref{fig:example1-pdg}). So while applying the fix, we replace the \textit{$k^{th}$} child of \blockstmt with the \ifstmt. We call the node in the \pdg where the edit operation applies as the edit location (\editloc). When a strategy applies, it has to determine this edit location based on the \pdg structure. Our \dsl defines an edit path (denoted by \editpath) to find edit location. In Figure ~\ref{fig:example1-pdg}, starting from the source-node @event@ (marked in orange), we take 7 semantic edges (reaching @callerId@) and then after hopping four synactic parent edges we reach the edit location \aksays{The notion of semantic edges should be defined and explained before this.}. This sequence of edge traversal defines our edit path. More generally, our \dsl considers the \editpath to be a set of semantic edges followed by a set of syntactic edges. This constraint on the paths allows expressivity to learn general strategies while also keeping the search space small. The semantic edges in \editpath allows navigating to ``somewhere close'' to sink location. Next once semantic edges are traversed, \editloc is reached by traversing a set of syntactic edges. Additionally, since the number of semantic edges might vary across examples, our \dsl allows a powerful  operator that navigates an indefinite number of semantic edges. This formulation helps our strategies to generalize well across widely different sets of programs. 

% %\paragraph{Edit Location (\editloc)} Edit location is the node in the \pdg where the edit operation (i.e. insertion or replacement of a child node) applies. \editloc is reachable from the vulnerability source found by traversing the edit-path (\editpath) in the \pdg.  So, in our running example, "$\dots$ function (data){$\dots$}" node (marked in blue) is the edit root and is reachable from the source via first traversing the semantic edges followed by traversing to "syntactic parent" four times. 
% Our kleene-star operator navigates indefinite semantic child edges until a ``stopping node'' (parameterized by a stopping condition) is reached. This stopping-condition is defined by a set of predicates applied on a \aksays{an} \astree node. We find that simplistic predicates about \newtextsc{ASTType} or \newtextsc{ASTValue} of \astree node and its neighbours suffice in locating this stopping node. For our running example, the stopping condition is the conjunction of the predicates @ASTType(node.parent) = IndexExpr@, @ASTType(node.parent.parent) = MethodCallExpr@. The stopping node lies on the taint-flow path from source to sink and therefore is quite relevant to the insert or replace operations (being a proxy for the semantic information of the vulnerability). Therefore, we call the stopping node as the semantic location (\semloc). In our running example, @calledId@, the sink-node is also the \semloc.

% As described above, our \dsl either inserts a new \astree, or it replaces an existing \astree with another \astree. An \astree is defined by three properties -- type, value and an array of children \astree. One can construct such an \astree by concretely initializing it using specific types and values for the tree and its descendants. However, a constant \astree cannot generalize because the fix depends on existing variables in the source code. Therefore, in addition to a constant \astree, our \dsl also allows referring to any existing node in the \astree. This referral is computed by traversing a path from the semantic location (\semloc defined above) to the node to reference \aksays{What is node to reference?}. The corresponding path is known as reference path. For e.g. the condition @handers.hasOwnProperty(callerId)@ which is used in the fix refers to @handlers@ and @callerId@ nodes in the tree and combines them in a constant \callexpr \astree. 

% We described edit paths and reference paths above. More generally a path is a sequence of edges in the \pdg where the edge can be one of syntactic or semantic or ancestral. When selecting a child edge, we also need to store \textit{which child} to select and it is determined by an index. Figure ~\ref{fig:repair-strategy-ex1} shows the entire strategy that fixes the example in Figure ~\ref{fig:vulnerabilty-example1}.
% \lstDeleteShortInline@

% \paragraph{Semantic Location (\semloc)} The node at which Kleene-Star traversal of semantic edges stops is called semantic location. This node is a key component in the repair since this node is a proxy for the semantically important values that would be necessary for making the edit. The sink node, callerId, is also the \semloc in our example. %Additionally, the \editloc is near this node and 

% \paragraph{Paths (\dslpath)} A path is described as a sequence of edges in the \pdg. The edges can be syntactic parent or child edges, semantic child edges, and ancestor edges in the \pdg.

% \paragraph{Index (\dslindex)}
% While selecting a child edge or while determining where the \concinsertcode needs to be inserted or replaced with, we need some index of which child to follow. Generally, this index is a constant value however it can be computed as an offset from the \semloc descendent direction as computed with the \newtextsc{OffsetFrom} operator. \naman{explain the requirement of offset with example}

%\paragraph{\astree} \astree is the tree-representation of the editcode that will replace some existing child of \editloc or will be inserted at some child indices of \editloc. One possible way to construct this \astree is to concretely initialize it using specific values and types. However, a constant \astree cannot generalize because the inserted code almost always depends on specific variables and the structures in the code. Therefore, in addition to a constant structure, the \astree can also refer to existing elements in the \pdg. This referral is again found using a path (\dslpath) traversal from the \semloc, the intuition being that it is a good \textit{proxy for the semantics of the vulnerability} and necessary variables to refer would be close to it.  


\subsection{Synthesizing \dsl strategies from examples}
\label{subsec:synthesis}
%\naman{The discussion about anti-unification would go in related work I presume?}

Given this high-level \dsl, we will now describe our example-based synthesis algorithm. 
We take as input a set of unsafe programs and edits generated as output at the end of data collection step (Section~\ref{sec:data}). 
Let $\{(\prog_{1},\edit_1),(\prog_{2},\edit_2),\dots$ $,(\prog_{n},\edit_n)\}$. 
Here, $\prog_{i}$ is the $i^{th}$ unsafe program and $\edit_i$ is the corresponding edit. Edit ($\edit$) contains the \astree-node of the edit-location ($\edit$.loc), the \textit{concrete} \astree of the edit-program ($\edit$.editprog), and the type of edit i.e. \insertsc or \replace ($\edit$.type). We use these to learn high-level repair strategies in our \dsl. 
\lstMakeShortInline[columns=fixed]@
Our goal is to combine specific paths, learned over examples that share similar repairs in different semantic and syntactic contexts, to obtain general strategies. Our repair strategies abstractly learn the following:
% \begin{enumerate}
%     \item the traversal for localizing edit-locations (\editloc) and reference-locations (\refloc). 
%     \item template-repair-program-representations using the reference-traversals . 
% \end{enumerate}
\begin{enumerate}
    \item Traversals for localizing edit-locations (\editloc) and reference-locations (\refloc). For example, @Ls@ in Line~\ref{lst:line:semkleene} (Strategy \strategyone) depicts a \T{KleeneTraversal} abstraction we can learn from examples having a variable number of semantic-edges. Similarly, @I@ in Line~\ref{lst:line:offseteditindex} (of \strategyone) depicts a generalized index expression we can learn from examples.
    \item \eastree which use reference-traversals. For example, @O@ in Line~\ref{lst:line:goodstratO} demonstrates templated-program-structure that we can learn from examples (say by generalizing from the witnessed guards @handlers.has(data)@  and @events.storage.has(event.name)@).
\end{enumerate}
\lstDeleteShortInline@


% In particular, we wish to abstract over examples that share similar repairs in different semantic or syntactic contexts. Consider the example abstractions below: 
% \begin{enumerate}
%     \item  
%     \item Line~\ref{lst:line:eastree} in Figure~\ref{fig:strat1} depicts 
%     the guard condition in Figure~\ref{fig:safememberex} @handlers.hasOwnProperty(data.id)@ can be abstracted with another guard @eventHandlers._storage.hasOwnProperty(event.name)@ into an abstract template @REF1.hasOwnProperty(REF2)@ where @REF1@ and @REF2@ are \T{ReferenceAST} have use traversals
% \end{enumerate}
% For example, . 
%Consider the example in Figures~\ref{fig:static-witnessing-1},~\ref{fig:static-witnessing-2}, and ~\ref{fig:static-witnessing-3}. They describe three different kinds of syntactic repairs and are not candidates to merge. Instead, 

We depict our synthesis algorithm in Figure~\ref{fig:strategy-learning}. At a high-level, our synthesis algorithm, first pre-processes the inputs, storing the required \textit{concrete} traversals. Next, it performs ranked pair-wise merging over the processed edits to synthesize strategies.
%We merge non-terminals recursively by deductively choosing production rules to enumerate and merging the non-terminals appearing in the productions. %During this recursion, it learns the \textit{traversals} and program templates abstractly. 

\noindent \textbf{Pre-processing.} In this step, given the programs and edits, we store the concrete traversals required for learning \editloc and \refloc (Line~\ref{algo:line:preprocess}). Naively computing all such traversals is very expensive and also leads to \textit{bad strategies}. Here, based on the insights from Section~\ref{subsec:examples}, we only compute the traversals that lead to shorter  
traversals
%\textit{abstract traversals}
which generalize better. In addition, we also share traversals between between \editloc and \refloc. Pre-processing has following three key steps:
\begin{enumerate}
    \item \textbf{Edit Traversals.} We compute the traversals between \prog.source and \editloc (Line~\ref{algo:line:conceditloc} of Figure~\ref{fig:strategy-learning}) that have the form of a sequence of semantic-edges followed by a sequence of syntactic-edges. This allows abstracting variable-length sequences of semantic-edge traversals as a \kleeneedge (corresponding to an abstract \newtextsc{KleeneTraversal}). We implement this using \newtextsc{BiDirecBFS} method at Line ~\ref{algo:line:bidirecbfs}. For every edit-traversal ($\I{T}_e$), we define {\em semantic-location} (\semloc for brevity) as the last-node on the semantic (dataflow) traversal before traversing the syntactic-edges.
    \item \textbf{Compressing Edit Traversals.} We compress these edit-traversals using the \newtextsc{Compress} method in Line~\ref{algo:line:compress}. It takes in a sequence of (syntactic or semantic) edges as input, greedily combines the consecutive edges with the same edge-type (\edgetype) into a \kleeneedge. The \kleeneedge is constructed using the edge type \edgetype, and a set of clauses $\clause_i$ that satisfy the target node of \kleeneedge. These clauses are either $\lambda n. \mathcal{T}(n) = \tau$ that check the type  or $\lambda n. \mathcal{T}(F_i(n)) = \nu$ that check the type of a neighbor. \newtextsc{Compress} returns a sequence of edges or \kleeneedges as output. 
    \item \textbf{Reference Traversals.} For every node of the edit-program, we locate nodes in the \pdg with the same \textit{value} using a \newtextsc{LevelOrderBFS} until a max-depth (Line~\ref{algo:line:maxlevel}). We perform this traversal from \semloc (defined in (1) above). We thus share parts of traversals between locating \editloc and \refloc which helps in learning \textit{better strategies}. The motivation behind using \semloc is that the expressions necessary for repair will be close to \semloc as it lies on the information-flow path. 
\end{enumerate}
%Specifically, for \editloc, we find the traversals between \prog.source and \editloc (Line~\ref{algo:line:conceditloc}) that first navigate a set of semantic-edges followed by a set of syntactic-edges. We implement this using the \newtextsc{BiDirecBFS} function at Line ~\ref{algo:line:bidirecbfs}. %It traverses semantic-edges from the source, syntactic-edges from the edit-location, and returns the intersecting traversals. 
%For every edit-traversal ($\I{T}_e$), we define semantic-location (\semloc for brevity) as the last-node on the semantic (dataflow) traversal before navigating a syntactic-edge. 
%Next, we compress the traversals greedy by combining consecutive edges of the same edge-type (\edgetype) into a \kleeneedge using the \newtextsc{Compress} method in Line~\ref{algo:line:compress}. Every \kleeneedge stores the \edgetype, and a set of clauses $\clause_i : i \in {1,\dots,n}$ that satisfy the end-node of \kleeneedge. These clauses are either $\lambda n. \mathcal{T}(n) = \nu$, i.e. a clause on the type of \semloc or $\lambda n. \mathcal{T}(F_i(n)) = \nu$, i.e. a clause on the type of a syntactic-neighbour of \semloc. We then compute traversals for finding reference locations. Here, instead of computing traversals from the source-node, we instead compute traversals from the semantic-locations. The expressions referenced in repairs are usually close to the \semloc (as it lies on the information-flow path and thus is affiliated with variables likely necessary for building the repair). This traversal-sharing optimizes the search and generalization of our strategies.

\noindent \textbf{Strategy Synthesis.} Given the edits and the associated traversal meta-data, we synthesize the strategy by pair-wise merging  (Line~\ref{algo:line:callmerge}). \newtextsc{MergeEdits}, the top-level synthesis method, takes a pair of edits as inputs and returns a list of strategies satisfying the example edits. We synthesize the strategies recursively using a deductive search over the non-terminals of the DSL (Figure~\ref{fig:fixing-dsl}). Specifically, to synthesize an expression corresponding to a non-terminal, we deduce which production to use and recursively synthesize the non-terminals given by its production-rule. This has the following key components: 
\begin{enumerate}
    \item \newtextsc{MergeEdits}: It takes pairs of edits as inputs and returns the strategy. It recursively synthesizes the traversal (for \editloc), index, and \eastree. It combines and returns them using the edit-type. 
    \item \newtextsc{MergeTraversal}: It takes two concrete traversals (sequence of edges or \kleeneedges) as inputs and returns the abstracted traversal. by merging elements in the sequence.
    \item \newtextsc{MergeEdge}: It takes two edges or \kleeneedges as inputs and returns a \T{GetKleeneTraversal} or \T{GetEdgeTraversal}. We combine two \kleeneedges using their edge-types and intersecting the clauses stored during pre-processing. We combine two edges using their edge-types, and recursively combining their indices.
    \item \newtextsc{MergeIndex}: It takes two integer indices as inputs and returns an abstracted index. If the two input indices are equal, we return a \T{GetConstant} operator with the input index value. Otherwise, we compute offset as the difference between input-index and index of child of $n$ which has \semloc as descendent (computed by $DO(n, \semloc)$). We return this offset if they are equal and an empty-list otherwise. 
    \item \newtextsc{MergeProg}: It takes two programs as input and returns a list of \eastree, where each list element can materialize into the input programs. If the top-level node in the programs have equal values and types, we combine them as a \T{ConstantAST}. Otherwise, we recursively combine their children. Finally, we merge the reference-traversals corresponding to the input programs and combine them into \T{ReferenceAST}.
\end{enumerate}

Our synthesis procedure is inspired by anti-unification~\cite{anti-unification} and we abstract the paths and edit-programs across different examples. Specifically, our \T{KleeneTraversal} and \T{OffsetIndex} functions allow generalization across paths having different number of edges and different indices where naive abstractions fail. Similarly, \eastree also resemble anti-unification over tree-edits. However, again we use traversals over syntactic and longer-context semantic-edges, for better generalizations  and repairs. 

Finally, note that while we perform pair-wise merges over the edits, the strategy synthesis algorithm can be extended to merge bigger cluster of edits together as well. However, from our experience, we find that the pair-wise merging performs well and is sufficient for our experiments. 
%by recursively synthesizing values corresponding to the non-terminals in our \dsl. 
%Thus, to synthesize a strategy, we synthesize traversals for the \editloc. To synthesize a traversal, we check if the two traversals contain an equal number of edges. Next, we try to merge the corresponding pairs of edges on each traversal. To merge an edge, if it is simply a syntactic or semantic edge, we instantiate a \T{GetEdge} operator appropriately. However, if the two edges are \kleeneedge, then we instantiate a \T{GetKleeneEdge} operator where the clauses are constructed by intersecting the clauses computed in \newtextsc{Compress} step. It additionally ensures that the number of clauses after the intersection is more than $1$ to prevent over-generalizing strategies. In order to merge strategies, we next merge the edit-programs (\edit.editprog). We first combine the programs as a \T{ConstantAST} by checking whether the type and value match and then recursively merge the children and finally take the cartesian product of children. Next, based on the concrete reference traversals, we merge them and construct \T{ReferenceAST} \eastree. 



%Note that in order to learn repair strategies, we need to synthesize paths corresponding to all locations (\location) used in the strategy. We use \location in three production rules in our \dsl (\locationone, \locationtwo, \locationthree in Rules~\ref{dslrule:strategy},~\ref{dslrule:index},~\ref{dslrule:node}). Naively, trying to synthesize these paths is very expensive and will lead to non-generalizing strategies.
%Here, based on common fix-patterns for these vulnerabilities, we reduce the search space by enforcing structure over the paths we learn. Additionally, we also share paths between the three locations. 

%We find that performing these 

%At a high-level, our synthesis algorithm takes these unsafe-programs and edits as inputs, preprocesses them to store relevant meta-data, clusters them and then recursively enumerates over the non-terminals in the \dsl. How


% \aksays{This section is very dense. It would be good to take a small example and illustrate the key steps visually.}
\input{synthesisalgo2}

% Given this high-level \dsl, we will now describe our synthesis algorithm. We build a top-down synthesis algorithm that learns strategies in this \dsl through a \pbe approach. We receive a set of unsafe codes and concrete edits \unsure{the terminology of concrete edits can be confusing for the paper. Basically, anologous to all things in \dsl, we have concrete edits, paths, etc.}. Let $\{(\prog_{1},\concedit_1),(\prog_{2},\concedit_2),\dots$ $,(\prog_{n},\concedit_n)\}$ be the data we collect from our data collection step where $\prog_{i}$ is the ith unsafe code snippets and $\concedit_i$ is the corresponding concrete edit. $\concedit_i$ contains the concrete edit-location $\conceditloc_i$, the \astree of the editcode $\concinsertcode_i$, and the type of edit i.e. \insertsc or \replace.

% Given this data, we instantiate our algorithm to learn repair strategies. Our algorithm takes in these set of examples and learns a set of ($\{\strategy_1,\strategy_1,\dots,\strategy_k\}$) that are supposed to cover the training examples. Later, these repair strategies, when given an unsafe program \pdg will generate the edit that needs to be applied. The sketch of our synthesis and learning algorithm is presented in Figure~\ref{fig:strategy-learning}.

% %%O := \{\(\strategy\sb{1},\strategy\sb{2},\dots,\strategy\sb{k}\)\}
\HUGE DONT USE THIS - USE synthesisalgo2
% Figure environment removed

% \unsure{\textbf{Terminology for reference!!:} Letters with overlines are concrete elements ($\concedit$ is a concrete edit, $\conceditpath$ is a concrete edit path) while the letters without lines are abstract elements that can generalize over examples ($\edit$ is an abstract edit, $\editpath$ is an abstract edit path). Following, we again define the various abbreviations used in the algorithm
% \begin{itemize}
%     \item \prog is the \pdg containing \sa annotations
%     \item \concedit is the concrete edit which itself contains editcode, edit-location, edit-type, and indices
%     \item \concinsertcode is the concrete editcode that is either inserted or replaces some existing region in the unsafe code. \concinsertcode itself can be represented as an \astree
%     \item \conceditloc is the concrete edit location i.e. where the edit takes place in a \prog
%     \item \concpath is a concrete path as a sequence of edges
%     \item \conceditpath is a concrete edit path from source to \conceditloc
%     \item semLoc or semantic location is the stopping node of \conceditpath
%     \item \concrefpath is a concrete reference path from semantic location to a particular node matching value of a \astree node in \concinsertcode
%     \item edit-type refers to whether edit is \insertsc or \replace
% \end{itemize}}

% \spsays{Consider breaking these into subsections and have a running example. for instance, first section can be running bidirectional bfs, other could be pairwise merging, and so on..} 

% Our top-level \newtextsc{Learn} method receives the programs and concrete edits as inputs (Line~\ref{algo:line:learn}). This method first invokes the \newtextsc{PreProcessConcEdit} method which computes edit-paths and reference-paths in the $\concedit$. Next, \newtextsc{Learn} method ranks edits based on the similarity of their $\concinsertcode$ and in that order tries to combine edits together in a pairwise manner using the \newtextsc{MergeEdit} method.

% \newtextsc{PreProcessConcEdit} method (in Line~\ref{algo:line:preprocess}) stores the relevant paths in the $\concedit$ structure that will be useful during pairwise-merging. It first computes a set of edit paths from source to sink using a bi-directional breadth-first search (\newtextsc{BiDirectBFS}) (Line~\ref{algo:line:conceditloc}) and stores it in the edit. Note that during this \newtextsc{BiDirectBFS}, it traverses only semantic edges from the source and only syntactic edges from the edit location. This naturally leads to paths that follow the required pattern of a set of semantic edges followed by a set of syntactic edges. Note that edit-paths also contain the semantic-locations in the semLoc field. Then for every edit-node in the edit code and every semantic location in the edit path, it stores reference paths from semantic locations to nodes in the \astree having the same value as edit-node (Line~\ref{algo:line:concrefpath}. These paths are computed using a \newtextsc{MaxLevelBFS} until a certain depth and storing the satisfying nodes. Note that during implementation, we memoize the path-finding steps to avoid repeating computations.

% \newtextsc{PairSimilar} method computes a score for every pair of edits in the edit set. To compute the similarity for a given pair of edits it performs \newtextsc{ASTSimilarity} on their editcodes. 

% \newtextsc{MergeEdit} is the top-level method of our deductive top-down synthesis algorithm. The merging procedure is intuitive. For every element in the edit, it recursively calls \newtextsc{Merge} operation on the elements and then assembles an edit using their outputs. Specifically, this method first ensures that edits are of the same type (\insertsc or \replace. Then it obtains a set of candidate edit-paths by calling the \newtextsc{MergeEditPaths} method on the concrete edit paths stored in the input concrete edits. Next for every candidate edit-path, it finds a candidate editcode using the \newtextsc{MergeEditCode} method. Finally, for every editpath and editcode pair, it assembles the final edit (Line~\ref{algo:line:assembleedit})
    
% \newtextsc{MergeEditPath} method tries to merge two concrete editpaths. It first computes a set of intersecting predicates over the semantic-locations of the two paths. Using the predicates, it builds the \semkleene edge. Finally, over the remaining set of edges in the edit-path, it calls the \newtextsc{MergePath} method which inturn merges all the edges successively (Line~\ref{algo:line:mergepath})


% \subsection{Applying the learned strategies}
% \label{subsec:applying}
% Given an unsafe-program and a set of repair strategies, we apply each strategy to the program to generate candidate repairs. 
% To apply a single repair strategy, we use the definitions of operators described in Section~\ref{subsec:dsl} to generate candidate repair programs. We find that, in practice, we obtain a few distinct repairs and we return them as the output of our system. 

%Once these high level repair strategies are learnt, applying them is natural. For an unsafe program, strategy \strategy ingests the \pdg of the program. Then it tries to build an edit, by first locating the edit location (using the edit path \editpath) and building the \astree recursively depending on whether it is a constant or reference tree. If edit location and \astree are generated, the edit operation is applied appropriately based on whether it is an \insertsc or \replace edit. Otherwise, if either of location or \astree is not generated then no edit is applied. 
% \subsection{Scribblings}
% Each node $n$ of the AST has an identifier $\mathit{id}\in\mathbb{N}$. The AST is characterized by a set $\mathcal{N}$ of node $\mathit{id}$s, i.e., $\mathcal{N}=\{\mathit{id}_0,\ldots\mathit{id}_k\}$. We have a map $\mathcal{T}$ from nodes to their types, i.e., $\mathcal{T}(n)=\tau$, where the primitive types $\tau$ include {\sc MethodCallExpr}, {\sc IndexExpr}, etc. We also have a set $\mathcal{E}$ of edges, where each edge is $(n_1,n_2,ET,z)$. Here, $n_1$ is a source node, $n_2$ is a target node, $ET$ is the type of edge (syntactic parent, syntactic child, semantic parent, or semantic child), and $z$ is a child's index (set to $-1$ if the edge is a parent edge). 

% The strategy $S$ is of two types, insert an AST $O$ at index $I$ of location $L$, $\mathit{Insert}(L(\mathit{source}),I,O)$, and replace the AST at index $I$ of location $L$ with $O$, $\mathit{Replace}(L(\mathit{source}),I,O)$. Each $L(n)$ takes a node $n$ and traverses a path to reach a location, i.e.,  $L(n)=\mathit{ApplyPath}(n,F_k\circ\ldots\circ F_0)$, where the output node is $F_k(F_{k-1}(\ldots F_0(n)\ldots)$. Each $F$ instantiates the edge traversal function $TE[I,ET,C]$ with an index $I$, an edge type $ET$, and an optional clause $C$ (relevant for KleeneEdge). We define $TE[I,ET](n)$ as $let\ i=I(n)\ in\ let\ N=\mathcal{E}(n,ET)\ in\ N[i]$, which gets an integer index $i$ of children of $n$, gets a set $N$ of nodes by dereferencing edges of type $ET$ from $n$ and returns the $i^{th}$ child of $n$.
% $F$ can also be $TE[ET,C](n)\equiv KE(n,ET,C)$.
%  The KleeneEdge $KE$ keeps dereferencing edges of type $ET$ till it hits a node where a clause $C$ holds, i.e., $KE(n,ET,C)$ is defined as $C(n)? n : \left(let\ t=\mathcal{E}(n,ET)\ in\ KE(t,ET,C)\right)$. The node $t$ which is target of an edge with type $ET$ and $n$ as a source here is chosen non-deterministically and our implementation resolves this non-determinism through a breadth-first search. A clause is a conjunction of predicates of the form $\lambda n. \mathcal{T}(L(n))=\tau$. The index $I$ is either an integer $z$ or of the form $\lambda n.DO(n,L(\mathit{source}))+z$, where $DO(n_1,n_2)$ returns the index of syntactic child of $n_2$ who is a syntactic ancestor of $n_1$. 
 
%  A strategy can fail to apply if in $DO(n_1,n_2)$ there is no path from $n_2$ to $n_1$,
\section{Experiments}
% \haizhou{Follow the same way of introduction as we did in Section2.}
% \noindent In this section, we will introduce datasets and experimental setups that we used. Then we evaluate our method, other self-supervised methods, and supervised methods under different distribution shifts (\ie, concept shifts and covariate shifts) under common settings (\ie, transductive, inductive settings). It has to note that we focus on node-level tasks (\eg, node classification) in this work. As for graph-level tasks, we leave it as our future work and some simple experiments can be found in Appendix~\ref{app:graph_classification}. 
In this section, we first introduce the experimental setup including datasets, training, and evaluation protocol in Section~\ref{sec:dataset}~and~\ref{sec:unsupervised}. 
% Next, we present our experimental setup and conduct extensive experiments to evaluate our method in Section~\ref{sec:unsupervised}. 
We then perform an ablation study to demonstrate the effectiveness of each proposed component in Section~\ref{sec:ablation}. 
Additionally, we analyze the impact of important hyper-parameters in Section~\ref{sec:sensitivity}. 
Subsequently, we integrate our method with various encoding models, showcasing the model-agnostic nature of our recipe in Section~\ref{sec:other_models}. 
Finally, we provide some qualitative results such as feature visualization in Section~\ref{sec:vis}.
It is important to note that we focus on node-level tasks (\eg, node classification) in this work. As for graph-level tasks, we leave it as our future work, while some simple experiments are also provided in Appendix~\ref{app:graph_classification}.

\subsection{Datasets}\label{sec:dataset}
There exist some benchmarks for evaluating graph out-of-distribution generalization~\cite{good,ji2022drugood,gds}. 
Among them, GOOD~\cite{good} is the most representative and comprehensive benchmark that curates more diverse graph datasets with diverse tasks, including single/multi-task graph classification, graph regression, and node classification involving more distribution shifts (\ie, concept shifts and covariate shifts). Hence in this work, we follow the evaluation protocol proposed in \cite{good}. Furthermore, we validate the effectiveness of our method in the datasets (\ie, Amazon-Photo, Elliptic) that are used in EERM~\cite{eerm}. The statistics and detailed introduction to these datasets can be found in Table~\ref{tab:dataset} and Appendix~\ref{app:datasets}.

\begin{table*}[htp]
\caption{The descriptions of datasets. ``Domain-Level'' means splitting by graphs, ``Time-Aware'' denotes splitting according to chronological order.``Word'' and ``Degree'' represent splitting according to word diversity and node degree respectively. ``Language'' means splitting by user language, suggesting the prediction should not be impacted by the language the user use. ``University'' denotes splitting according to the domain university, implying that the prediction of webpages should be based on word contents and link connections rather than university features. ``Color'' means that nodes are split according to node differences in covariate shift and color-label correlations in concept shift.}
\label{tab:dataset}
\centering
\begin{tabular}{cccccccc}
\toprule
Datasets     & Network Type        & \#Nodes & \#Edges & \#Attributes &\#Classes& Train/Val/Test Split     & Metric   \\
% Cora         & Artificial Transformation & 2,703   &         &              &         &                      & Accuracy \\
Amazon-Photo\footnotemark
             & Co-purchasing network      & 7,650   & 119,081   & 755          & 10      & Domain-Level         & Accuracy \\
Elliptic\footnotemark  
             & Bitcoin transactions       & 203,769 & 234,355   & 165          & 2       & Time-Aware           & F1-Score \\
GOOD-Cora    & Scientific publications    & 19,793  & 126,842   & 8,710         & 70      & Word/Degree          & Accuracy \\
% GOOD-Arxiv   & arXiv papers               & 169,343 & 2,315,598 & 128          & 40      & Time/Degree          & Accuracy \\
GOOD-Twitch  & Gamer network              & 34,120  & 892,346   & 128          & 2       & Language             & ROC-AUC  \\
GOOD-CBAS    & A BA-house graph           & 700     & 3,962     & 4             & 4       & Color                & Accuracy \\
GOOD-WebKB   & Webpage network            & 617     & 1,138     & 1,703         & 5       & University           & Accuracy \\
\bottomrule
\end{tabular}
\end{table*}
\footnotetext[5]{This dataset is adopted from~\cite{yang2016revisiting}. \cite{eerm} constructs ten graphs with different environment id’s for each graph.} 
\footnotetext[6]{The original is available on \hyperlink{https://www.kaggle.com/ellipticco/elliptic-data-set}{https://www.kaggle.com/ellipticco/elliptic-data-set}}

\subsection{Unsupervised Representation Learning}\label{sec:unsupervised}
\subsubsection{Transductive Setting}~\label{sec:trans}
% \noindent\textbf{Baselines.}\quad We conduct experiments with 12 baselines which consist of three categories: supervised methods and self-supervised generative methods, self-supervised contrastive methods. Specifically, we compare with three supervised baselines: empirical risk minimization~(ERM)~\cite{erm}, invariant risk minimization (IRM)~\cite{irm}, and a recent proposed graph OOD method dubbed EERM~\cite{eerm}. We also compare various unsupervised node-level representation learning methods: three self-supervised generative methods including GAE~\cite{gae}, VGAE~\cite{gae}, GraphMAE~\cite{gmae} and seven self-supervised contrastive methods: DGI~\cite{dgi}, MVGRL~\cite{mvgrl}, GRACE~\cite{grace}, RoSA~\cite{rosa}, BGRL~\cite{bgrl}, COSTA~\cite{costa}, SwAV~\cite{swav}. The descriptions of these methods can be found in Appendix~\ref{app:baselines}.
In this subsection, we focus on validating our proposed algorithm under the transductive setting, where the test nodes will participate in message passing~\cite{gilmer2017neural} during training following~\cite{good}. 

\noindent\textbf{Baselines.} We conduct experiments with 12 baselines from three categories: (i)~supervised methods, including empirical risk minimization~(\textbf{ERM})~\cite{erm}, invariant risk minimization (\textbf{IRM})~\cite{irm}, and a recent proposed graph OOD method \textbf{EERM}~\cite{eerm}; (ii)~self-supervised generative methods including Graph Autoencoder (\textbf{GAE})~\cite{gae}, Variational Graph Autoencoder (\textbf{VGAE})~\cite{gae}, Self-Supervised Masked Graph Autoencoders (\textbf{GraphMAE})~\cite{gmae}; (iii)~self-supervised contrastive methods including Deep Graph Infomax (\textbf{DGI})~\cite{dgi}, Contrastive Multi-View Representation Learning on Graphs (\textbf{MVGRL})~\cite{mvgrl}, Deep Graph Contrastive Representation Learning (\textbf{GRACE})~\cite{grace}, A Robust Self-Aligned Framework for Node-Node Graph Contrastive Learning (\textbf{RoSA})~\cite{rosa}, Bootstrapped Representation Learning on Graphs (\textbf{BGRL})~\cite{bgrl}, Covariance-Preserving Feature Augmentation for Graph Contrastive Learning (\textbf{COSTA})~\cite{costa}, Unsupervised Learning of Visual Features by Contrasting Cluster Assignments (\textbf{SwAV})~\cite{swav}. The detailed descriptions of these baselines can be found in Appendix~\ref{app:baselines}.

\noindent\textbf{Experimental setup.} We use the same graph encoder across different datasets for a fair comparison following~\cite{good}. We use grid search to find other hyper-parameters (\eg, learning rate, epochs) for different methods. For all experiments, we select the best checkpoints for ID and OOD tests according to results on ID and OOD validation sets following~\cite{good}, respectively. Experimental details and hyper-parameter selections are provided in Appendix~\ref{app:hyper}. For evaluating unsupervised methods, a linear classifier will be built on the frozen trained encoder after finishing pre-training. The reported results are the mean performance with standard deviation after 10 runs following~\cite{good}.

\noindent\textbf{Analysis.}\quad Based on the experimental results listed in Table~\ref{tab:trans_concept} and \ref{tab:trans_covariate}, we can draw the following conclusions: firstly, we find strong self-supervised methods (\eg, GRACE, BGRL, COSTA) are more robust to distribution shifts (concept shift in Table~\ref{tab:trans_concept} and covariate shift in Table~\ref{tab:trans_covariate}) compared to supervised methods. For instance, on GOOD-CBAS and GOOD-WebKB datasets, GRACE surpasses the best supervised method by large margins (over 6\% absolute improvement). Interestingly, we find the methods designed for OOD generalization (\ie, IRM) and graph OOD generalization (\ie, EERM) do not attain superior performance than the standard ERM on most of the datasets. For example, EERM shows superior OOD performance compared to ERM in only one experiment, and IRM outperforms ERM in four out of ten experiments across the conducted evaluations. This phenomenon is also observed in \cite{good,ahuja2020empirical,rosenfeld2021risks}, showcasing the challenge of achieving invariant prediction in non-Euclidean graph settings. 

Furthermore, our method surpasses other SOTA self-supervised methods on the OOD test set of all datasets by a considerable margin while achieving comparable performance in the in-distribution test set. For instance, on small datasets such as GOOD-CBAS and GOOD-WebKB, our method outperforms GRACE\footnote{MARIO is built up on GRACE according to our recipe. So, we make a comparison with GRACE here.} by over 2\% absolute accuracy on the OOD test set. On larger datasets such as GOOD-Cora and GOOD-Twitch, our method still outperforms other methods which shows its superiority. For instance, under covariate shift, MARIO surpasses other methods by over 7\% absolute accuracy on the GOOD-Twitch OOD test set. These statistics prove the effectiveness of our design.


\begin{table*}[htp]
\caption{Experimental results of all methods under concept shift. The bold font means the top-1 performance and the underline represents the second performance across the unsupervised methods. 'ID' represents in-distribution test performance and 'OOD' means out-of-distribution test performance. (OOM: out-of-memory on a GPU with 24GB memory)}
\label{tab:trans_concept}
\centering
\scalebox{0.95}{
\begin{tabular}{l|cc|cc|cc|cc|cc}
\toprule
\toprule
\multirow{3}{*}{concept shift} & \multicolumn{4}{c|}{GOOD-Cora}                   & \multicolumn{2}{c|}{GOOD-CBAS} & \multicolumn{2}{c|}{GOOD-Twitch} & \multicolumn{2}{c}{GOOD-WebKB} \\
                           & \multicolumn{2}{c}{word} & \multicolumn{2}{c|}{degree}& \multicolumn{2}{c|}{color}    & \multicolumn{2}{c|}{language}   & \multicolumn{2}{c}{university} \\
                           & ID         & OOD         & ID          & OOD          & ID            & OOD           & ID             & OOD            & ID            & OOD            \\
\midrule
ERM                        & 66.38±0.45 & 64.44±0.18  & 68.60±0.40  & 60.76±0.34   & 89.79±1.39    & 83.43±1.19    & 80.80±1.00     & 56.92±0.92     & 62.67±1.53    & 26.33±1.09     \\
IRM                        & 66.42±0.41 & 64.29±0.31  & 68.57±0.35  & 61.45±0.24   & 89.64±1.21    & 82.29±1.14    & 78.87±1.04     & 59.30±1.79     & 62.67±1.10    & 26.88±1.42     \\
EERM                       & 65.10±0.44 & 62.45±0.19  & 66.95±0.44  & 56.58±0.25   & 79.07±2.12    & 64.50±1.01    & OOM            & OOM            & 62.50±2.01    & 28.07±3.23      \\
\midrule
% Random-Init                & 37.53±1.74 & 32.12±1.24  & 37.82±1.71  & 27.74±1.14   &               &               &                &                & 60.33±2.21    & 27.07±1.70     \\
GAE                        & 60.65±0.89 & 58.00±0.55  & 62.59±1.11  & 53.44±0.80   & 75.28±1.36    & 68.07±2.05    & 81.25±0.81     & 51.51±1.05     & 62.17±3.34    & 25.78±1.85     \\
VGAE                       & 63.19±0.53 & 60.35±0.47  & 61.65±0.66  & 54.28±0.28   & 76.50±0.50    & 59.07±0.56    & 80.46±0.53     & 55.56±4.53     & 62.50±2.38    & 24.40±2.57     \\
GraphMAE                   & \underline{66.44±0.46} & \underline{64.87±0.30}  & 67.95±0.46  & 59.41±0.39   & 89.14±0.89    & 82.93±0.93    & 80.05±0.64     & 59.38±1.49     & 61.83±3.37    & 29.27±2.15     \\
DGI                        & 63.33±0.56 & 60.71±0.49  & 65.93±1.02  & 55.83±0.53   & 91.22±1.47    & 85.00±1.66    & 80.05±0.87     & 59.16±1.88     & 61.83±2.83    & 28.63±1.92      \\
MVGRL                      & OOM        & OOM         & OOM         & OOM          & 88.57±1.15    & 76.50±1.17    & OOM            & OOM            & 62.00±3.79    & 28.26±4.20     \\
GRACE                      & 65.61±0.61 & 63.92±0.44  & \textbf{68.59±0.35}  & 60.15±0.45   & 92.00±1.39    & 88.64±0.67    & \textbf{83.43±0.63}     & \underline{60.45±1.46}     & 64.00±3.43    & \underline{34.86±3.43}  \\
RoSA                       & 64.06±0.67 & 62.44±0.39  & 67.07±0.65  & 57.68±0.44   & 90.78±2.27    & 85.93±2.14    & 82.39±0.42     & 57.45±2.16     & 64.17±4.10    & 32.20±2.15     \\
BGRL                       & 65.18±0.43 & 63.43±0.45  & 66.83±0.80  & 59.63±0.38   & 92.36±1.16    & 87.14±1.60    & 82.52±0.60     & 55.48±1.48     & 63.67±2.33    & 31.47±3.43     \\
COSTA                      & 65.05±0.80 & 62.37±0.45  & 66.76±0.87  & 55.73±0.36   & \underline{93.50±2.62}    & \underline{89.29±3.11}    & 83.15±0.30 & 55.03±3.22     & 61.66±2.58    & 32.39±2.13 \\
% ArCL                       &            &             & 67.64±0.57  & 59.71±0.44   &               &               &                &                & 65.00±3.94    & 35.41±1.97 \\      
SwAV                       & 62.22±0.53 & 59.79±0.53  & 64.65±0.94  & 55.06±0.39   & 89.00±0.79    & 81.72±0.66    & \underline{83.32±0.15}     & 59.69±1.97     & \underline{65.17±3.76}    & 29.36±2.01    \\
\midrule
MARIO                       & \textbf{67.11±0.46} & \textbf{65.28±0.34}  & \underline{68.46±0.40}  & \textbf{61.30±0.28}   & \textbf{94.36±1.21}    & \textbf{91.28±1.10}    & 82.31±0.54     & \textbf{63.33±1.72}     & \textbf{65.67±2.81}    & \textbf{37.15±2.37}     \\
\bottomrule
\end{tabular}}
\end{table*}

\begin{table*}[htp]
\caption{Experimental results of all methods under covariate shift. The bold font means the top-1 performance and the underline represents the second performance across the unsupervised methods. 'ID' represents in-distribution test performance and 'OOD' means out-of-distribution test performance. (OOM: out-of-memory on a GPU with 24GB memory)}
\label{tab:trans_covariate}
\centering
\scalebox{0.95}{
\begin{tabular}{l|cc|cc|cc|cc|cc}
\toprule
\toprule
\multirow{3}{*}{covariate shift} & \multicolumn{4}{c|}{GOOD-Cora}                                   & \multicolumn{2}{c|}{GOOD-CBAS} & \multicolumn{2}{c|}{GOOD-Twitch} & \multicolumn{2}{c}{GOOD-WebKB} \\
                           & \multicolumn{2}{c}{word} & \multicolumn{2}{c|}{degree}& \multicolumn{2}{c|}{color}    & \multicolumn{2}{c|}{language}   & \multicolumn{2}{c}{university} \\
                           & ID         & OOD         & ID          & OOD          & ID            & OOD           & ID             & OOD            & ID            & OOD            \\
\midrule
ERM                        & 70.50±0.41 & 64.69±0.33  & 72.46±0.49  & 55.53±0.50   & 92.00±3.08    & 77.57±1.29    & 70.98±0.41     & 49.35±5.09     & 39.34±1.79    & 14.52±3.14   \\
IRM                        & 70.48±0.26 & 64.53±0.57  & 71.98±0.34  & 53.72±0.46   & 90.86±2.41    & 78.86±1.67    & 69.81±0.95     & 49.11±2.82     & 38.52±3.30    & 13.97±2.80     \\
EERM                       & OOM        & OOM         & OOM         & OOM          & 65.00±2.57    & 57.43±3.60    & OOM            & OOM            & 46.07±4.55    & 27.40±7.65     \\
\midrule
GAE                        & 56.63±0.79 & 48.93±0.93  & 66.30±0.88  & 34.01±0.87   & 73.00±2.16    & 60.86±3.01    & 67.24±1.23     & 47.65±2.49     & 45.08±6.32    & 28.02±6.29    \\
VGAE                       & 62.02±0.66 & 54.12±0.86  & 69.41±0.57  & 44.20±1.29   & 62.29±2.04    & 63.29±1.11    & 66.99±1.43     & \underline{50.48±4.58}     & 48.85±4.68    & 20.87±6.69     \\
GraphMAE                   & 68.14±0.43 & 64.00±0.33  & \textbf{73.36±0.56}  & 53.75±0.55   & 67.28±3.03    & 67.28±1.49    & 68.84±1.20     & 48.02±2.79     & 48.03±4.34    & 30.00±8.09     \\
DGI                        & 60.85±0.75 & 57.03±0.67  & 68.97±0.41  & 41.75±0.88   & 69.57±4.09    & 59.71±3.43    & 68.43±1.05     & 44.83±1.61     & 48.52±5.04    & 21.11±7.50     \\
MVGRL                      & OOM        & OOM         & OOM         & OOM          & 65.00±1.94    & 64.15±0.77    & OOM            & OOM           & \textbf{54.10±5.39}    & 16.59±6.51     \\
GRACE                      & \underline{68.77±0.33} & \underline{64.21±0.41}  & 72.69±0.34  & \underline{56.10±0.63}   & \underline{93.57±1.83}    & \underline{89.29±3.40}    & \underline{71.12±0.87} & 46.21±1.54 & 49.67±5.82    & 28.10±4.68    \\
RoSA                       & 68.19±0.56 & 62.48±0.61  & 71.04±0.62  & 52.72±0.79   & 84.71±4.14    &79.14±3.51     & 70.58±0.36     & 45.83±1.72     & 52.30±4.24    & \underline{34.24±7.92}     \\
BGRL                       & 67.23±0.43 & 61.33±0.36  & 72.11±0.39  & 49.15±0.73   & 89.00±2.56    & 79.86±3.29    & \textbf{71.43±0.53}     & 43.86±0.94     & 51.80±5.55    & 30.32±7.61    \\
COSTA                      & 65.28±0.60 & 60.33±0.53  & 70.65±0.62  & 54.03±0.28   & 92.29±1.59    & 82.71±2.74    & 69.29±1.37     & 49.07±2.13     & 50.49±3.01    & 29.84±4.75   \\
SwAV                       & 63.29±1.01 & 56.98±0.94  & 70.27±0.73  & 43.00±0.52   & 89.57±1.12    & 81.43±1.69    & 69.19±0.93     & 49.37±2.96     & 49.84±4.82    & 30.55±6.72   \\
\midrule
MARIO                       & \textbf{69.99±0.54} & \textbf{65.06±0.34}  & \underline{72.73±0.43}  & \textbf{57.73±0.45}  & \textbf{94.57±2.46}    & \textbf{91.00±2.48}     & 68.31±0.78 & \textbf{57.37±1.37}     & \underline{53.94±3.23}    & \textbf{35.24±4.98}   \\
\bottomrule
\end{tabular}}

\end{table*}

\subsubsection{Inductive Setting}
In this subsection, we conduct experiments under the inductive settings, where the test nodes are kept unseen during training. This setting is more suitable for domain generalization.
% But we think it is more convincing that conduct experiments under inductive settings which means test nodes are unseen during training. This setting is more appropriate for domain generalization.

\noindent\textbf{Baselines:} For GOOD-WebKB and GOOD-CBAS datasets, we adopt ERM, IRM, GraphMAE, and GRACE as our baselines. And for Amazon-Photo and Elliptic datasets, we select ERM, EERM, and GRACE as our baselines.

\noindent\textbf{Experimental setup:} For GOOD-WebKB and GOOD-CBAS datasets, we use the same model configuration in Section~\ref{sec:trans}.
% Besides, we add experiments on Amazon-Photo dataset~\cite{yang2016revisiting} and Elliptic~\cite{elliptic} dataset in this subsection. 
For Amazon-Photo dataset~\cite{yang2016revisiting} and Elliptic~\cite{elliptic} dataset, they consist of many snapshots (training data and testing data use different snapshots) which are naturally inductive. For Amazon-Photo dataset, we use 2-layer GCN~\cite{gcn} as the encoder and for elliptic dataset, we use 5-layer GraphSAGE~\cite{sage} as encoder following~\cite{eerm}.

% Figure environment removed

\noindent\textbf{Analysis:}
According to Figure~\ref{fig:amazon},\ref{fig:elliptic},\ref{fig:ind_con},\ref{fig:ind_cov}, we can draw following conclusions:
firstly, based on Figure~\ref{fig:amazon}, it is evident that our method outperforms other representative supervised and self-supervised methods on all test graphs (T1$\sim$T8). This superiority is reflected in the larger median value of our method compared to others. For instance, MARIO achieves over a 3\% absolute improvement compared to ERM in terms of the mean value of eight median values. Additionally, our method demonstrates higher stability across different random initializations, as indicated by the closer proximity of the first and third quartile values to the median value~(\eg, the difference of first and third quartile values of ERM, EERM, GRACE and MARIO are 4.2, 3.3, 6.7 and 1.0 on T8 respectively which indicates MARIO is much more stable than other methods). Furthermore, our method exhibits consistent performance across different graphs (\eg, The standard deviation of median values on T1$\sim$T8 for ERM, EERM, GRACE, and MARIO are 0.4, 1.1, 1.2, and 0.3, respectively.), indicating its robustness to environmental variations and its ability to extract invariant features: $g(G^e) \approx g(G^{e'})$ for all $e, e' \in \mathcal{E}^\text{train}$. In summary, our method showcases enhanced OOD generalization capabilities.
% $g(G^e)g(G^e^\prime)$ where $any e, e^\prime in \mathcal{E}^{train}$

Secondly, from the results presented in Figure~\ref{fig:elliptic}, we can observe that our method averagely harvests 10.9\% absolute improvement over GRACE and 12.5\% absolute improvement over EERM in terms of F1 scores on Elliptic dataset. This demonstrates the effectiveness of our method in handling distribution shifts and improving performance compared to existing approaches. It is worth noting that GRACE's performance worsens over time, indicating its inability to handle distribution shifts effectively. In contrast, our method consistently achieves better F1 scores, except for T9, which is caused by the dark market shutdown occurred after T7~\cite{elliptic}. The emergence of such an event introduces significant variations in data distributions, which subsequently results in performance degradation for all methods. Indeed, this event serves as an unpredictable external factor that introduces significant challenges for models trained on limited training data. The results indicate that the performance heavily depends on available training data. Nonetheless, our approach outperforms other methods even in such an extreme case. This highlights the effectiveness of our method in addressing distribution shifts and improving generalization performance.

Finally, based on the observations from Figure~\ref{fig:ind_con} and Figure~\ref{fig:ind_cov} MARIO demonstrates the best performances on both ID and OOD test sets for GOOD-WebKB and GOOD-CBAS datasets, under both concept shift and covariate shift. Notably, MARIO outperforms other methods by more than 3\% and 10\% absolute improvement on GOOD-WebKB and GOOD-CBAS, respectively, under covariate shift. We can draw similar conclusions as discussed in Section~\ref{sec:trans}. Even under the inductive setting, our method continues to demonstrate excellent OOD generalization capabilities and achieves comparable or even improved in-distribution test performance. These statistical results further validate the effectiveness of our method in handling distribution shifts and enhancing generalization performance.

Overall, the observations we have made provide strong evidence of the great capacity of our method for handling distribution shifts, validating its effectiveness and potential for real-world applications.



% Figure environment removed

% Figure environment removed


% Figure environment removed


\subsection{Ablation Studies}\label{sec:ablation}
\noindent Table~\ref{tab:aba} provides a detailed analysis of the effect of each component according to our proposed recipe for improving OOD generalization in graph contrastive learning. Let's examine the different variants of our method and their impact on performance.
Specifically, MARIO~(w/o ad) represents MARIO without  adversarial augmentation. MARIO~(w/o cmi) denotes we only maximize the mutual information between positive pairs without considering conditional mutual information. MARIO~(w/o cmi, ad) means a vanilla graph contrastive method that is similar to GRACE. 

From Table~\ref{tab:aba}, we can find MARIO~(w/o cmi) lags far behind MARIO on OOD test set which demonstrates appropriately minimizing the redundant information (\ie, conditional mutual information) is essential to improve OOD generalization of GCL methods. And adversarial augmentation can also boost OOD generalization because it can approximately serve as a supermum operator to learn more invariant features  discussed in Section~\ref{sec:aug}. Based on the analysis of these variants, it is evident that the proposed improvements on data augmentation and contrastive loss in the recipe are both effective in enhancing graph OOD generalization. Each component contributes to the overall performance improvement, and their combination leads to a stronger self-supervised graph learner in terms of graph OOD generalization. 

In short, the findings from Table~\ref{tab:aba} support the rationale behind your proposed recipe and provide empirical evidence of the effectiveness of each proposed component. By incorporating these enhancements, our method achieves superior performance in handling distribution shifts and improving graph OOD generalization in graph contrastive learning.
\begin{table*}[htp]
\caption{Ablation studies for MARIO by masking each component.}
\label{tab:aba}
\centering
\scalebox{0.9}{
\begin{tabular}{l|cc|cc|cc|cc|cc}
\toprule
\toprule
\multirow{3}{*}{concept shift} & \multicolumn{4}{c|}{GOOD-Cora}                       & \multicolumn{2}{c|}{GOOD-CBAS} & \multicolumn{2}{c|}{GOOD-Twitch} & \multicolumn{2}{c}{GOOD-WebKB} \\
                           & \multicolumn{2}{c}{word} & \multicolumn{2}{c|}{degree}& \multicolumn{2}{c|}{color}    & \multicolumn{2}{c|}{language}   & \multicolumn{2}{c}{university} \\
                           & ID         & OOD         & ID          & OOD          & ID            & OOD           & ID             & OOD            & ID            & OOD            \\
\midrule
MARIO                      & \textbf{67.11±0.46} & \textbf{65.28±0.34}  & \textbf{68.46±0.40}  & \textbf{61.30±0.28}      & \textbf{94.36±1.21}  & \textbf{91.28±1.10}    & 82.31±0.54     & \textbf{63.33±1.72}     & \textbf{65.67±2.81}    & \textbf{37.15±2.37}     \\
MARIO(w/o ad)              & 66.23±0.53 & 64.02±0.18  & 67.88±0.38  & 60.46±0.29   & 93.21±1.25    & 90.29±0.91    & 82.42±0.73     & 60.50±1.02     & 64.83±2.83    & 36.51±3.25    \\
MARIO(w/o cmi)             & 65.32±0.60 & 63.51±0.32  & 68.14±0.32  & 61.19±0.34   & 94.15±1.23    & 90.57±1.96    & \textbf{82.51±0.56}     & 61.41±2.63     & 64.50±4.35    & 35.78±2.53     \\
MARIO(w/o cmi, ad)         & 64.67±0.55 & 63.11±0.32  & 67.95±0.65  & 60.01±0.57   & 93.36±1.66    & 89.64±1.73    & 81.90±0.75     & 60.12±1.60     & 64.17±3.67    & 34.13±2.38     \\
\bottomrule
\end{tabular}}
\end{table*}
% & 65.32±0.60 & 63.51±0.32 exchange 64.67±0.55 & 63.11±0.32
% 68.14±0.32       id ood test: 60.95±0.43       ood ood test: 61.19±0.34


\subsection{Sensitivity Analysis}\label{sec:sensitivity}
\noindent In this subsection, we will analyze some important hyper-parameters of our method. We conduct sensitivity analysis on GOOD-WebKB dataset with concept shift, we chose two sensitive hyper-parameters (\ie, the coefficient $\gamma$ of condition mutual information in Equation~\ref{equ:cmi} and the number of prototypes $|C|$ in Equation~\ref{equ:pq}). The coefficient of CMI range in $[0.001, 0.01, 0.1, 0.5, 1]$ and the number of prototypes $|C|$ ranges in $[10, 50, 100, 200, 300]$. From Figure~\ref{fig:sensitivity}, we can observe that $\gamma$ reaches 0.1 and $|C|$ reaches 100 or 200 can achieve the best OOD test accuracy. Both higher and lower values of $\gamma$ result in suboptimal performance. This finding aligns with previous research such as DIB~\cite{dib}, indicating that an appropriate compression level is crucial for achieving optimal performance. Extremely high or low compression values are not ideal. 

Regarding the number of prototypes $|C|$, based on the results shown in Figure~\ref{fig:sensitivity}, it is found that setting $|C|=100$ leads to the best performance in terms of OOD test accuracy. This choice provides a moderate number of pseudo labels, which is beneficial for the learning process. 

Based on the sensitivity analysis, we determined that setting $\gamma=0.1$ and $|C|=100$ on most datasets. These hyperparameter values strike a balance between compression level and the number of prototypes, resulting in improved graph OOD generalization.
% Figure environment removed


\subsection{Integrated with Other Models}\label{sec:other_models}
% Figure environment removed

\begin{table}[htp]
\caption{Results of different learning approaches with different encoding models (\ie, GCN, GraphSAGE, GAT).}
\label{tab:others}
\centering
\scalebox{0.9}{
\begin{tabular}{cc|cc|cc}
\toprule
\toprule
\multirow{3}{*}{Model}& \multirow{3}{*}{Method} & \multicolumn{2}{c|}{GOOD-CBAS} & \multicolumn{2}{c}{GOOD-WebKB} \\
                & & \multicolumn{2}{c|}{color}    & \multicolumn{2}{c}{university} \\
                &   & ID          & OOD         & ID          & OOD            \\
\midrule
\multirow{3}{*}{GCN} 
&ERM               & 89.79±1.39 & 83.43±1.19  &  62.67±1.53 & 26.33±1.09         \\
&GRACE             & 92.00±1.39 & 88.64±0.67  &  64.00±3.43 & 34.86±3.43        \\
&MARIO             & 94.36±1.21 & 91.28±1.10  &  65.67±2.81 & 37.15±2.37        \\ \bottomrule
\multirow{3}{*}{SAGE} 
&ERM               & 95.07±1.51 & 75.14±1.19  & 73.67±2.08  & 46.33±3.42       \\
&GRACE             & 95.29±1.11 & 74.43±2.36  & 70.50±5.06  & 49.54±3.83        \\
&MARIO             & 96.00±1.07 & 76.29±3.01  & 71.00±3.82  & 51.74±4.63        \\ \bottomrule
\multirow{3}{*}{GAT} 
&ERM               & 78.64±3.63 & 72.93±2.64  & 61.33±3.71  & 28.99±2.63        \\
&GRACE             & 84.57±1.79 & 78.36±1.60  & 59.50±2.36  & 35.78±3.26        \\
&MARIO             & 84.93±1.95 & 80.43±1.89  & 62.17±4.78  & 38.17±3.10        \\
\bottomrule
\end{tabular}}
\end{table}



\noindent In the subsection, we demonstrate the model-agnostic nature of the recipe by integrating it with various graph neural network (GNN) models, including GCN, GraphSAGE, and GAT.

From Table~\ref{tab:others}, it can be observed that regardless of the specific GNN model used as the encoder, our method consistently achieves the best performance on the OOD test set. This indicates the effectiveness and robustness of our method across different GNN models.
By achieving superior performance across different GNN models, MARIO demonstrates its versatility and ability to improve the OOD generalization of various graph neural models. This highlights the broad applicability and effectiveness of our recipe in enhancing the performance of different GNN encoders.

Furthermore, we integrate our recipe with other GCL methods in Appendix~\ref{app:other_methods}. The results demonstrate our recipe can boost the OOD generalization ability of various GCL methods which means our recipe can serve as a plug-in for many current classical GCL methods.

% Figure environment removed

\subsection{Visualization}\label{sec:vis}
\subsubsection{Metric Score Curves}
We present metric score curves for ERM and MARIO, including training, ID validation, ID testing, OOD validation, and OOD testing accuracy, in Figure~\ref{fig:curve2}. Notably, MARIO demonstrates superior convergence with approximately 10\% absolute improvement on the OOD test set compared to ERM. Furthermore, MARIO effectively narrows the performance gap between in-distribution and out-of-distribution performance, showcasing its efficacy in enhancing OOD generalization for graph data. More metric score curves can be found in Appendix~\ref{app:curves}.


\subsubsection{Feature Visualization}
In order to assess the quality of learned embeddings, we adopt t-SNE~\cite{tsne} to visualize the node embedding on GOOD-Cora dataset (concept shift in word domain) using random-init of GCN, EERM, GRACE, and MARIO, where different classes have different colors in Figure~\ref{fig:vis}. For clarity, we select eight classes with the largest number of nodes to enhance the informativeness and interpretability of the visualization. We can observe that the 2D projection of node embeddings learned by MARIO has a better separation of clusters, which indicates the model can help learn representative features for downstream tasks. It has to note that we depict both ID nodes and OOD nodes in the same figure. 

Besides, we also separately visualize ID nodes and OOD nodes in the different figures in the Appendix~\ref{app:feature}. And we can find MARIO performs a clearer separation of clusters whether on ID nodes or OOD nodes compared to other methods.





\section{Conclusion} % Good for now
We have validated that pre-trained large language models (LLMs) can attain reasonable classification accuracy in sentiment classification tasks with minimal labeled samples across different clients with federated learning paradigm. 
% In the context of natural low-resource settings, where a substantial number of mobile devices lack access to an adequate amount of labeled data, our approach enables maximum device participation in federated training.
Our proposed Low-Parameter Federated Learning (LP-FL) permits federated clients to assign soft labels to unlabeled data, utilizing the evolving knowledge from the global model. 
Given the inherent constraints of mobile devices, such as limited computational resources and unreliable network environments, we achieve comparable, and sometimes superior performance to Full-Parameter Federated Learning (FP-FL) by fine-tuning and federating only a fraction of the model parameters at each local client.  Our FP-FL approach establishes an effective learning framework for leveraging LLMs in the FL environment and shows promising applications over mobile devices. 


{
\bibliographystyle{splncs04}
\bibliography{egbib,fed,plm}
}



\end{document}
