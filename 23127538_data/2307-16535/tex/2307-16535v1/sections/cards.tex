%-------------------------------------------------------------------------------
\section{Cybersecurity Cards (Version 1)}
\label{sec:cards}
%-------------------------------------------------------------------------------

The weaknesses of existing solutions we have identified regarding the limitations of accessibility,
steep learning curves and a perceived requirement of specialist knowledge/expertise
must be ameliorated by a new solution that provides an answer to the following
research questions. Specifically, can the new solution:

\begin{questions}[leftmargin=*,align=left]%[itemindent=0.18in]
    \item Provide introductory cybersecurity knowledge to novice users?
    \item Provide material for expressing interpretation and documentation of key cybersecurity topics, which can support independent learning and self-efficacy?
    \item Act as an index for the CyBOK knowledge base which provides an interface for discussion on key cybersecurity topics?
    \item Provide links between key cybersecurity topics, allowing the generation of concepts which can capture various cybersecurity scenarios?
\end{questions}
% \end{enumerate}

To answer these research questions, we propose a novel approach to represent key cybersecurity
knowledge, utilising specially designed playing cards. In recent years, it has
been shown that specially designed cards used as a tool for education attributes to positive outcomes in the space of
learning, attitudes and critical thinking skills~\cite{kordaki2016computer,kordaki2017digital}. With a large portion
of the general population familiar with card-based games, it is clear that a playing card format is a good medium,
both physically and digitally, with relatively intuitive logic if they are designed well. A subset of the cards
we propose in this section have been made available for viewing and can be found
online\footnote{\url{https://anonymous.4open.science/r/cybersecurity_cards-9F00/}}. The full set of cards will be made
publicly available under the CC BY-NC-SA Creative Commons license after the research project ends in January 2024.

% Figure environment removed


\subsection{Focus of Knowledge}

In order to establish a good grounding for an evaluation of a card-based solution to address the proposed research
questions, an important factor is to determine what information the card aims to represent. In this work, we leverage
the information found in the CyBOK knowledge base and produce the cybersecurity cards to address the limitations of
the knowledge base. Specifically, the focus of knowledge we chose to take is in the domain of software engineering,
given recent increases in threats pertaining to this domain~\cite{saxena2020impact,pranggono2021covid} and the
observation of finding people with skills in secure coding reported as the most difficult task~\cite{willetts2014cyber}.

The topics that fall within the scope of the focus of knowledge in this work were derived from a filtering and
selection procedure, where the CyBOK knowledge tree was traversed from Knowledge Areas to Topics. Given that not all
KAs in CyBOK are related to the domain of software engineering, we selected topics that related to this theme.
As highlighted in Figure~\ref{fig:cybokfulltree}, we chose the following knowledge trees to traverse:
{\em Human Factors}, {\em Malware \& Attack Technologies}, {\em Software Security}, {\em Web \& Mobile Security} and the
{\em Secure Software Lifecycle}. For each KA, we follow each branch of the tree to the leaf nodes. For each of the
leaves, we pick up the topic and provide a better description of that name to create the first version of the card. If
there are duplicates or highly similar cards, these are merged together. Any leaves that fall within Topics that, for
example, may not follow a single predominant theme and do not relate to the selected focus of knowledge are cut. For
example, {\em Software \& Platform Security} category are disdrectly related to the scope of this work whereas others are
not, such as {\em Forensics} which involves identifying and analysing data to support legal proceedings. This process
was carried out by two academic researchers in cybersecurity who decided on the final first set of cards in consensus.

%\begin{table*}
%  % \begin{center}
%    \begin{tabularx}{\linewidth}{m{0.1\linewidth}m{0.125\linewidth}cm{0.60\linewidth}}
%      \toprule
%      \textbf{Class} & \textbf{General Card} & \textbf{Symbol} & \textbf{Description} \\
%      \midrule
%      \multirow{3}{\linewidth}{Vulnerability} &
%                      Environment & % Figure removed & Badly configured execution environment can leak out information and allow malicious users to attack the services above it \\ &
%                      Code & % Figure removed & Badly written code can allow malicious users to alter the functionality, getting extra information or crashing the services \\ &
%                      Human & % Figure removed & Bad practice by careless user or bad defence against corrupted internal user can allow malicious users to gain insider knowledge or privilege through social and psychological skills \\
%      \midrule
%      \multirow{8}{\linewidth}{Attack} &
%                      Memory & % Figure removed & Targets the memory and storage of modern computers or hides its existence in the memory in fileless status \\ &
%                      Injection & % Figure removed & Alters original functionality or adds extra executions on the original services \\ &
%                      Race Condition & % Figure removed & Abuses race conditions to get undesired output \\ &
%                      Side Channel & % Figure removed & Studies the outside behaviour to get information of the execution paths, resources used or other hidden details for further attack \\ &
%                      Authentication & % Figure removed & Breaks normal authentication process to perform actions with fake or privileged identity \\ &
%                      Web & % Figure removed & Targets web services or redirects attacks towards other users by abusing web services \\ &
%                      Human Factor & % Figure removed & Trick users to providing information of some privileged access or resources. Also refers to insiders purposely doing damage to internal services \\ &
%                      System & % Figure removed & Spreads through computers without user notice in the form of malicious software or bundle of attack commands that target basic system functionality and settings \\
%      \midrule
%      \multirow{4}{\linewidth}{Defence} &
%                      Detection & % Figure removed & Detect and notify the administrator about possible attacks happening. Allows live defence and tracing for forensic purposes \\ &
%                      Mitigation & % Figure removed & Decrease the possibility or effect of certain attacks \\ &
%                      Prevention & % Figure removed & Stop some types of attack \\ &
%                      Education & % Figure removed & Provide education to users to decrease the chance of being targeted \\
%      \bottomrule
%    \end{tabularx}
%    \caption{\centering List of General Cards (Version 1)}
%    \label{table:basecards}
%  % \end{center}
%\end{table*}


\subsection{Design and Structure}

The playing cards format is easy to handle in both a physical or digital medium~\cite{altice2014playing} and are
designed to be handled individually or combined into a set of cards (known as a {\em deck}). The deck of cards is first
split into three distinct classes: \textbf{Vulnerability}, \textbf{Attack} and \textbf{Defence}, following suit from
common threat modelling approaches and security catalogues that encompass the vulnerability-attack-defence
dichotomy~\cite{kordy2014attack,ferro2021human,tatam2021review,cve,martin2011cwe}. Within each of these classes, the
deck is further split into {\em \textbf{General}} and {\em \textbf{Detailed}} cards (Figure~\ref{fig:carddesign}).
General cards represent a type of one of the classes (e.g. {\em Code}), while Detailed cards provide lower-level,
concrete examples that relate to the General card.

In terms of design (Figure~\ref{fig:carddesign}), attack and vulnerability cards are encased in a square border, with
attack cards filled with a solid red colour while vulnerability cards a signalled with a diagonal line to separate
white from red. The defence cards resemble an octogonal shape (akin to a shield) which has a white fill colour. The
diagonal separation of white and red in vulnerability cards aims to highlight vulnerabilities being the center of
the attack-defence dichotomy.

General cards in the deck include: a title representing the KA topic; a type represented
by a symbol (see \circled{1}); and a description related to the topic it represents. Detailed cards are assigned a
unique identifier in the form {\em $a\_Bi$} in the top right corner \circled{2}, where $a$ refers to the class of the
Detailed card, $B$ referring to the first letter of the General card it is categorised under and $i$ acting as an
index number in the set of $N$ Detailed cards for the category $B$. The symbol for the Detailed card's category is
made opaque in the center of the card, behind the description of the Detailed card. Attack cards also contain a
description of the impact of the attack. Defence cards contain a target symbol to help further identify the
vulnerability it aims to protect against attacks. Vulnerability cards also describe an attack vector with associated
attack card identifiers, as well as the consequence of the vulnerability with associated defence card identifiers
\circled{3}. In total, there are 124 cards in the deck which is composed of 30 vulnerability cards, 32 attack cards and
47 defence cards, each of which are categorised under one of the 15 General cards.

\subsubsection{Relationships Between Cards}
Given that cybersecurity stems from conflict between attackers and defenders targeting one or more vulnerabilities,
creating a capturable {\em many-to-many relationship} is essential when introducing cybersecurity concepts. Thus, the
cybersecurity cards should represent this relationship, where multiple attacks can target multiple vulnerabilities that
can, in turn, be mitigated or countered by multiple defences. While CyBOK does contain information about these
relationships, it is hard to infer these from implicit references within the reference material.
% how do we derive the cybok implicit relationships and link this to the example in next paragraph
In Figure~\ref{fig:carddesign}, we can see that the vulnerability card {\em "Blind Trust of User Input"} is linked to a
set of identifier codes \circled{3} for a number of related attacks and defences. This example shows a link to the
{\em "Command Injection"} and {\em "Memory Thief"} attacks. The first attack involves the execution of unauthorised
commands to a system (which may be input by a user) and the second involves the stealing of confidential information
from memory. In contrast, two defence examples are shown as links to the attacks and the vulnerability, which include
{\em "Code Assertions"} and {\em "Input Sanitisation"} which involve monitoring code execution and sanitising user
input to eliminate malicious code or escape characters, respectively.

\subsection{Expected Usage}

We anticipate that the structuring and indexing employed by the design of the deck of
cards will provide opportunities for easy exploration of complex
cybersecurity topics. Thus, an individual with little-to-no experience in this
area could start with one known or familiar topic, typically a vulnerability, and
use the linkable relationships between cards to reach other topics. We believe this would allow
a more natural feel for progression and may appear less daunting compared
to trawling through an encyclopedic knowledge base such as CyBOK. The use of a playing cards
format is one that is interactive and can facilitate communication between
participants in exploration exercises and allows one to claim ownership of
what is being communicated by holding the card representing a piece of cybersecurity
knowledge. Using General cards as a starting point to understand core
information, and then progressing on to more concrete attack-defence scenarios, might
provide a more gentle learning pathway for novice users.