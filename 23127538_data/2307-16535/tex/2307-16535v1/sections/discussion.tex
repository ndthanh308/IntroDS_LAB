%-------------------------------------------------------------------------------
\section{Discussion}
\label{sec:discussion}
%-------------------------------------------------------------------------------

The aim of this research is to evaluate how the cybersecurity cards provided a clear communication of key
cybersecurity topics and an interface for discussing them, leveraging the information from the CyBOK knowledgebase. To
investigate this, we were guided by the research questions proposed in Section~\ref{sec:cards}. In this paper, we
provide an evaluation of two versions of the cybersecurity cards and subsequent discussion will encompass both sets of
results and follow each research question in turn.

\subsection*{\normalsize RQ1: Do the cybersecurity cards provide introductory cybersecurity knowledge to novice users?}

Cybersecurity is often overlooked as a subject due to issues such as perceived technical difficulty, steep learning
curves and a requirement of specialist knowledge and/or expertise. In industry, for example, the lack of cybersecurity
professionals has been linked to a lack of practical cybersecurity content within learning
materials~\cite{caldwell2013plugging}. While CyBOK aims to rectify this learning gap, traversing the knowledgebase and
understanding key concepts and topics in cybersecurity requires prior knowledge, as evidenced by the fact that the
primary usage of CyBOK involves curricula development for higher education programmes~\cite{hallett2018mirror}, and thus
may not be considered accessible for novice individuals. In this work, we show that our cybersecurity cards achieved a
positive result with regard to providing introductory knowledge of key cybersecurity topics to novice users. Specifically,
we found that most participants from the first workshop in higher education agreed (Section~\ref{sec:provideknowledge1})
with this, and our second workshop involving school children -- who described themselves as having much less
cybersecurity and secure coding experience/skills -- also agreeing with this (Section~\ref{sec:provideknowledge2}).

\subsection*{\normalsize RQ2: Do the cybersecurity cards provide material for expressing interpretation of key topics that supports independent learning and self-efficacy?}

One of the concerns surrounding cybersecurity relates to the preconception of steep learning curves and a requirement of
specialist knowledge and expertise~\cite{asen2019you,bahizad2020risks}, which is a problem that is not adequately
managed by the CyBOK knowledgebase. From our evaluation of the cards, we found that most of our participants in higher
education from the first workshop agreed they were able to understand key cybersecurity topics independently (Section~\ref{sec:independentlearning1}).
Interestingly, we found that half of the primary and secondary participants in the second workshop agreed with this (Section~\ref{sec:independentlearning2}),
with the remainder equally neutral or disagreeing. It appears that removing the general cards and replacing them with a
glossary may have impacted this may have impacted this, backed up by statements from the participants saying that nobody
reads the glossary. Thus, an important consideration is how this information could be better presented. In terms of
self-efficacy, we found that the higher education participants agreed that even without a cybersecurity expert present
in the group, they were able to access cybersecurity knowledge solely using our cybersecurity cards. For the primary and
secondary school participants, more participants agreed on this compared to understanding topics independently.

\subsection*{\normalsize RQ3: Do the cards act as an index for the CyBOK knowledgebase, which provides an interface for discussion on key cybersecurity topics?}

Given that the primary use cases of the CyBOK knowledgebase relate to the construction of educative teaching
material and the fact this requires existing specialist knowledge on topics within the CyBOK KAs to be
developed on further~\cite{hallett2018mirror}, it is clear that it is not easy to traverse the CyBOK
knowledgebase. Other cybersecurity learning solutions which make use of
a playing card format~\cite{denning2013control,thomas2019educational,anvik2019program} also show some success
in this aspect, for example arguing that a gamified cards approach can help provide a reachable foundation for
providing digestible cybersecurity information. In this work, while we also make use of a playing cards format, the
intended use is not in the context of games which have shown to have limitations with regard to perceptions of quality
and effort related to learning game mechanics. In any case, many of these existing approaches are either designed for
those with existing cybersecurity knowledge~\cite{swann2021open,trickel2017shell,cybokctf} or do not make use of a
peer-reviewed and well-established knowledge foundation such as CyBOK~\cite{denning2013control,anvik2019program}. It has
been shown that the value of security information and how well it is understood or perceived as actionable depends
strongly on the source~\cite{redmiles2020comprehensive,rader2015identifying}. The first version of our cybersecurity
cards show that the majority of participants agreed they were able to discuss key topics with cybersecurity experts that
periodically checked in on them, as well as with other members in their group (Section~\ref{sec:provideinterface1}).
For the second version of the cards, we found that most participants also agreed, with more strongly agreeing compared
to those in further education using the previous cards. While some participants in both cases disagree with this (1
strongly disagreeing in both workshops), this may be due to reasons such as simply not wanting to discuss topics with
the experts. Interestingly, in the second workshop, we found that more participants disagreed that the cards enabled
them to hold discussions with others in their group. This could be due to a lack of interest in doing so, or potentially
they simply did not know they could do that. Ultimately, CyBOK has been shown to lack depth, particularly in the aspect
of practical experience such as discussing key topics, which is essential for mastering cybersecurity
skills~\cite{manson2014case,hallett2018mirror}. From our evaluation, we found that our cards can provide this to even
novice users of different age groups.

\subsection*{\normalsize RQ4: Do the cards provide links between key cybersecurity topics, allowing for the capture of various scenarios?}

In the evaluation questionnaire, we asked participants whether the cards provided them with knowledge about
the relationships between attacks, defences and vulnerabilities (Section~\ref{sec:provideknowledge1}). We
found that the majority of participants agreed with this. This is further strengthened given that most
participants in the first workshop also agreed the cards promoted discussion of key cybersecurity topics with both
experts and other members in their groups (Section~\ref{sec:provideinterface1}), with similar results seen in the
second workshop with primary and secondary school participants (Section~\ref{sec:provideinterface2}). While the
CyBOK knowledge base does encapsulate the attack-defence-vulnerability dichotomy, it is difficult to identify
whether some topics focus on a single predominant theme or whether it spans across various
themes~\cite{gonzalez2022exploring}. Furthermore, it has been shown that because of this difficulty, these links could
be identified from series of keywords which are only meaningfully extracted via specialised algorithms such as topic
model analysis~\cite{gonzalez2022exploring,hallett2018mirror}. With regard to other learning approaches for
cybersecurity, Capture The Flag (CTF) activities manage to highlight links between attacks, defences and 
vulnerabilities as well by prioritising the focus of finding vulnerabilities which can help users understand
how they present themselves, how they can be targeted and how to protect against attacks which target
them~\cite{trickel2017shell,swann2021open}. Interestingly, some CTFs have been designed leveraging CyBOK as
its foundational information source or inspiration~\cite{cybokctf}. However, the disadvantage to these
approaches is that CTF competitions rely on technical expertise of computer systems, with activities
involving the use of command-line tools or requiring, at a minimum, a basic understanding of cybersecurity
concepts in order to progress in finding vulnerabilities~\cite{mcdaniel2016capture}. We show that our
cybersecurity cards provide the introductory cybersecurity knowledge for novice users, whilst also allowing
them to devise various cybersecurity scenarios that encapsulate the vulnerability-attack-defence dichotomy.

% TODO:!!
\subsection{Reported Limitations and Future Work}

From the evaluation of the first version of the cards, we identified some limitations from participant responses. First,
some participants felt the overall number of cards was too high and some cards (e.g. General cards) were not really used.
In the second version, this prompted a minimisation of the deck to improve physical handling, whilst also adhering to
the primary goals, and the replacement of General cards by a glossary that can be referred to whenever needed. In the
second workshop using the updated cards which replaces the General cards with a glossary, we found that there were no
suggestions regarding the number of cards causing any difficulties related to usage. However, some participants felt the
glossary was not useful and in some cases was not used at all. Because of this, a point of future work would look into
how such information could be portrayed differently. One potential solution to this could be to make use of additional
introductory cards (Figure~\ref{fig:introcard}), which may help provide this information in a different way using the
deck of cards already being interacted with, compared to how the glossary was perceived.

% Figure environment removed

Second, the layout and written content of first version of the cards were described as difficult to go through by
participants in the first workshop. For the second version of the cards, we improved the contrast of the red colour,
as well as the layout of card elements (e.g. positioning of symbols) to help improve clarity. In the second workshop,
none of the participants explicitly suggested any design changes to the cards. However, the number of participants who
agreed that the cards provided knowledge on the relationships between attacks, defences and vulnerabilities could be
improved. One potential design change that may improve this result is to better visibly distinguish between the
different classes of cards. In previous work, it has been studied whether different colours of warnings can affect one's
perception of risk~\cite{young1991increasing,leonard1999does}. For example, the same red color could be used for attack cards as it is typically associated with
danger and a higher perceived relative amount of risk. An orange or amber colour is typically used for warnings and
the colour yellow for caution, which could be used to signal the vulnerability cards as they


Second, the layout of the cards and written content were identified as
difficult to go through, and the colour-coding was also suggested to be updated. A new version should think about
colour contrast and the layout of various card elements (such as positioning of symbols) to improve clarity. Finally,
the links between the cards should also be revised in a new version of the cards, given that some participants stated
they only focused on attacks and defences. One of the limitations of other playing cards
approaches~\cite{denning2013control,anvik2019program} was that attacks are typically highlighted first, which
does not help users understand how attacks present themselves via vulnerabilities and how to protect against
them (defences). While this could be due to participants with limited knowledge thinking in terms of attacks vs.
defences and neglecting related vulnerabilities, it may also be that they find it hard to make links between
attacks and defences from the use of codes within the vulnerability card. This issue of readability to form
these links from vulnerabilities may be preventing them from making use of other vulnerability cards as a starting
point. In either case, the focus of centering on vulnerabilities is an important point, and a clear and concise
design that focuses on improving this will likely play a pivotal role in improving not only independent learning
and self-efficacy, but also enable the devising of various cybersecurity scenarios.