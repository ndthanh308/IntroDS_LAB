%-------------------------------------------------------------------------------
\section{Related Work}
\label{sec:related}
%-------------------------------------------------------------------------------

It has been shown that those in cybersecurity careers require a set of skills, involving
the abilities to carry out various tasks at any time in non-traditional environments and
adapt to the dynamic nature of these environments~\cite{niccs2015}.
Everyday users of any computer system gather information from
a wide array of sources, ranging from peer networks to (reputable) websites and television.
However, the amount of information that flows in a highly connected environment is
guaranteed to have a proportion of disinformation and the value of security information
depends strongly on its source~\cite{rader2015identifying}. Furthermore, strong information
is only useful so long as it is well understood and perceived as actionable~\cite{redmiles2020comprehensive}.
One factor that impacts how well security information is understood relates to having a correct
mindset that can adapt to a variety of highly dynamic and complex scenarios~\cite{thomas2019educational}.
Unfortunately, for novice individuals, cybersecurity may appear intimidating or confusing and
may require more specialist knowledge or guidance to promote adequate learning.

% Talk about CyBOK and other knowledge bases. Refer to hallett's paper to promote CyBOK.
To address this, various curriculum guidelines and knowledge frameworks have been developed for
cybersecurity, covering a range of fundamental topics ranging from software, hardware, networks and
cyber-physical systems. The Joint Task Force (JTF) on Cybersecurity Education propose a draft of
curricular guidance on cybersecurity to support educational efforts~\cite{bishop2017cybersecurity}.
They design a framework model for a body of knowledge that covers six knowledge areas which
several concepts span over, targeting specific disciplines and application areas which pertain to
the demographic of cybersecurity professionals. The National Initiative for Cybersecurity Education
(NICE)~\cite{newhouse2017national} is a cybersecurity workforce framework which aims to provide
a foundation for describing and sharing information about knowledge, skills and abilities in
cybersecurity to strengthen an organisation's cybersecurity. The NCSC propose a Certified Master's
Program that defines several pathways to address knowledge and skill gaps in cybersecurity education,
which describe what topics must be covered and to what depth~\cite{ncscmsc}. While all these frameworks
tend to agree on key cybersecurity topics that must be understood, they only promote greater
emphasis on a subset of topics. For example, NICE covers a wide range of key topics but gaps exist
such as with topics related to cyber-physical systems and human factors, and the NCSC Certified
Master's Program guidelines do not put much emphasis on important topics such as software security, but
in contrast with others place emphasis on attacks and defences. The CyBOK project, developed by the
NCSC, codifies foundational knowledge into 21 knowledge areas in 5 categories. It has been shown
that, in comparison with other knowledge frameworks, CyBOK covers a wide range of knowledge areas
and does not have gaps that are present with some other frameworks~\cite{hallett2018mirror}.

% Then talk about cybersecurity learning platforms and cards, highlighting some limitations of those
% approaches which will back up this work

Aside from knowledge frameworks, such as CyBOK, cybersecurity information and how it is perceived and
understood has also shown to be presented in other ways.
Thomas et al.~\cite{thomas2019educational} proposed the use of a collectible card game (CCG)
as a means of teaching cybersecurity to high school students, given the benefits of
prevalence culturally to all age groups (familiarity) and encouraging the understanding
of competitive strategy and mistake-making as a way of learning~\cite{turkay2012collectible}.
Anvik et al.~\cite{anvik2019program} proposed the use of a web-based card game for learning
programming and cybersecurity concepts, using simple vocabulary to create ubiquitous learning
experiences. Denning et al.~\cite{denning2013control} proposed the use of a tabletop card game,
Control-Alt-Hack, with the aim of providing awareness training for cybersecurity,
arguing that playing card games can provide a reachable foundation for providing digestible
cybersecurity information to large audiences. As well as card games, other gamified approaches
include the likes of Capture The Flag (CTF) activities, for example, which provide competitive
exercises for finding vulnerabilities in computer systems and applications and have shown to be
a valuable learning tool~\cite{trickel2017shell,cybokctf,swann2021open}. While these gamified
approaches show to be successful for different demographics and use cases, there are limitations.
For example, card game approaches such as Control-Alt-Hack~\cite{denning2013control} do not cover
a broad range of what has been identified by knowledge frameworks as key topics. Furthermore, the
use of a games format does have limitations with regard to perceptions of quality, given the nature
of modern computer games, for example, as well as some may perceiving learning bespoke game
mechanics as tedious~\cite{denning2014practical}. With regard to other approaches, such as CTF
activities, while they have shown to be a very successful learning approach, this assumes that
configurations are done properly and can be easily administrated as well as rewarding to participants.
As well as this, they also require some degree of knowledge in cybersecurity, as well as more specific
knowledge of computer systems (e.g. command-line utilities) and specific software in order to be able
to participate well~\cite{mcdaniel2016capture}.