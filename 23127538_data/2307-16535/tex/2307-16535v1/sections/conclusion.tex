%-------------------------------------------------------------------------------
\section{Conclusion}
\label{sec:conclusion}
%-------------------------------------------------------------------------------

Cybersecurity is a complex subject area that is constantly changing due to the
dynamic nature of vulnerabilities, attacks and defences. Unfortunately, because
of this, many users of computer systems who may have little-to-no knowledge of
cybersecurity are left vulnerable. Many organisations provide basic cybersecurity
training and implement best-practices, but the question as to whether users truly
understand the impact of these measures and the consequences of not adhering to them
remains.

Existing knowledge bases such as CyBOK have the potential to provide key information,
but often fall into a trap of being a vast body of knowledge that is hard to traverse
and understand relationships between the key topics they aim to provide access to.
In this work, we propose an approach that makes use of a playing cards format for introducing
and interfacing with cybersecurity topics. The novel contributions of this work include
designing a deck of cybersecurity cards that act as a more accessible and digestible index
for learning key cybersecurity topics informed by the CyBOK knowledge base, facilitate
independent learning of key topics and promote understanding of the various relationships
found in the cybersecurity ecosystem, addressing the limitations of existing solutions/activities
that aim to facilitate learning. Upon evaluation, we found that the cybersecurity
cards did in fact provide an accessible knowledge base and facilitate independent
learning of key cybersecurity topics. Further, we also found that relationships between
vulnerabilities, attacks and defences were understood by participants. While most
responses promote the success of our approach, there were limitations with respect to
the design of the cards which impacted the learning of some participants. Addressing
these limitations, we then propose the next version of our card deck (Version 2)
focusing mainly on the design of the cards to improve physical handling, as well as
better highlighting the links between vulnerabilities, attacks and defences.

In conclusion, our proposed cybersecurity cards provide a comprehensive and effective tool that
allows for the learning of key cybersecurity topics which also promotes discussion. This design
not only allows for non-technical learners to gain introductory knowledge of
cybersecurity, but also promotes creative independent learning and allows users
of the cards to analyse and devise various cybersecurity scenarios and how vulnerabilities,
attacks and defences may play a role in them.