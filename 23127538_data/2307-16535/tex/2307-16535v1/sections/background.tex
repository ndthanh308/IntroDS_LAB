%-------------------------------------------------------------------------------
\section{Background}
\label{sec:background}
%-------------------------------------------------------------------------------

The need for practical and easy-to-learn cybersecurity learning material is a constant problem which stems from
the evolving nature of vulnerabilities, threats, attacks and defences as the number of connected
users and devices scales. In recent years, the number of critical cybersecurity
incidents have increased significantly, correlating with increasing numbers of
online users during the Covid-19 pandemic, for example, as well as an increase in the
adoption of various connected computer systems in day-to-day activities. Among these
incidents, research shows that around 95\% of cybersecurity breaches occur as a
result of human error~\cite{wef2020} and that organisations lack the sophistication, interest and/or
knowledge to handle these threats~\cite{ciscoreport2020,sophos2021}.

It has been shown that those in cybersecurity careers require a set of skills, involving the
abilities to carry out various tasks at any time in non-traditional environments, and adapt to
the dynamic nature of these environments~\cite{niccs2015}. In the domain of software engineering,
basic cybersecurity training such as password best-practices and multi-factor authentication are
employed for individuals to conform to, with the aim of alleviating concerns and mitigating the
potential for liabilities that arise as a result of cybersecurity-related
incidents~\cite{walters2011assessing,alotaibi2018review,sarginson2020securing}. In the domain of
software engineering, for example, simply employing these best practices are not enough as there are
additional requirements of ensuring developers code securely. It has been identified that a large
number of Android applications contain security-related code snippets copied and pasted from Stack Overflow,
of which nearly 98\% contained at least one insecure code snippet~\cite{fischer2017stack}. This was also shown to
be true upon scanning open-source projects in different programming languages~\cite{hong2021dicos}. The
value of security information depends strongly on its source~\cite{rader2015identifying} and reputable information
sources are only useful, so long as they are well-understood and perceived as
actionable~\cite{thomas2019educational,redmiles2020comprehensive}. While sites such as StackOverflow are reputable
for providing actionable solutions to problems that are asked on them, it is clear that the security of solutions
are not well understood, either by the responder or those who interpret them. Thus, are software developers
truly {\em informed} to the extent they actually understand why certain security measures are put in place, and the
true impact of not adhering to best-practices and secure coding. For novice individuals, such as those who write
and/or deploy software code without formal software engineering training, cybersecurity may appear intimidating or
confusing, and may require more specialist knowledge or guidance to promote adequate learning and understanding.

To address this, various curricula guidelines and knowledge frameworks have been developed for
cybersecurity, covering a range of fundamental topics ranging from software and hardware security, to networks and
cyber-physical systems. The Joint Task Force (JTF) on Cybersecurity Education propose a draft of
curricular guidance on cybersecurity to support educational efforts~\cite{bishop2017cybersecurity}.
They design a framework model for a body of knowledge that covers six knowledge areas which
several concepts span over, targeting specific disciplines and application areas which pertain to
the demographic of cybersecurity professionals. The National Initiative for Cybersecurity Education
(NICE)~\cite{newhouse2017national} is a cybersecurity workforce framework which aims to provide
a foundation for describing and sharing information about knowledge, skills and abilities in
cybersecurity to strengthen an organisation's cybersecurity. The NCSC propose a Certified Master's
Program that defines several pathways to address knowledge and skill gaps in cybersecurity education,
which describe what topics must be covered and to what depth~\cite{ncscmsc}. While all these frameworks
tend to agree on key cybersecurity topics that must be understood, they only promote greater
emphasis on a subset of topics. For example, NICE covers a wide range of key topics but gaps exist
such as with topics related to cyber-physical systems and human factors, and the NCSC Certified
Master's Program guidelines do not put much emphasis on important topics such as software security, but
in contrast with others place emphasis on attacks and defences. The CyBOK knowledge base is decomposed
into 21 knowledge areas (KAs) (as of version 1.1), each introduced by a reference document and a set
of topics presented as a branch of the overall {\em Knowledge Tree}
(Figure~\ref{fig:cybokfulltree})~\cite{cybokwebsite}. Each of these knowledge areas are organised into a hierarchy
of 5 categories. It has been shown that, in comparison with other knowledge frameworks, CyBOK covers a wider range
of knowledge areas and does not have gaps that are present with other frameworks~\cite{hallett2018mirror}.
% Ultimately, the goal of these bodies of knowledge is to provide
% both a taxonomy of various cybersecurity topics and references which can be used to facilitate the development
% and review of cybersecurity curricula and learning materials.

% Figure environment removed

For each KA in CyBOK, there is a collection of reports that form an encyclopedic collection of
knowledge of key concepts backed up by state-of-the-art academic literature.
The key concepts that are covered in each KA are known as {\em Topics}. Some Topics
decomposed further into a set of more specialised subjects ({\em Sub-Topics}).
%For example,
%{\em "SQL Injection"} is a portrayed as a {\em Sub-Topic} of
%{\em "Injection Vulnerabilities"} (Figure~\ref{fig:cyboktree}).
At its core, the CyBOK Knowledge Tree aims to provide a systematic approach to traversing an exhaustive list of various
cybersecurity topics and mapping to the tree facilitates links to other resources.

%% Figure environment removed

While CyBOK facilitates a body of knowledge which attributes to the production of material for cybersecurity
education and professional training, there are some weaknesses which may render it an inaccessible resource to
more novice individuals such as those in the domain of software engineering who write or deploy code with no formal
software engineering training. First, the links between meaning and relationships among topics and sub-topics
vary across the entire Knowledge Tree, which prevents easy expression of various cybersecurity
scenarios. Second, the material across the CyBOK knowledge base and its indexing structure is not easy to traverse
for novice users. Gonzalez et al.~\cite{gonzalez2022exploring} show that it would be difficult for novice individuals
to infer the links between various topics, given that some follow either a single predominant theme or span several
topics themselves. Ultimately, to support novice users as well as those more experienced, key cybersecurity knowledge
provided by knowledgebases such as CyBOK require adequate presentation that can faciliate independent learning whilst
also providing a suitable interface for discussion of various cybersecurity scenarios to make the links between meaning
and relationships among topics.

Aside from knowledge frameworks, such as CyBOK, cybersecurity information and how it
is perceived and understood has also been presented in other ways.
Capture the Flag (CTF) activities provide a series of competitive exercises used to find
vulnerabilities in computer systems and applications, and have shown to be a valuable
learning tool~\cite{trickel2017shell,cybokctf,swann2021open}. Thomas et
al.~\cite{thomas2019educational} propose the use of a collectible
card game (CCG) as a means of teaching cybersecurity to high school students, given the benefits of
prevalence culturally to all age groups (familiarity) and encouraging the understanding
of competitive strategy and mistake-making as a way of learning~\cite{turkay2012collectible}.
Anvik et al.~\cite{anvik2019program} propose the use of a web-based card game for learning
programming and cybersecurity concepts, using simple vocabulary to create ubiquitous learning
experiences. Denning et al.~\cite{denning2013control} propose the use of a tabletop card game,
Control-Alt-Hack, with the aim of providing awareness training for cybersecurity,
arguing that playing card games can provide a reachable foundation for providing digestible
cybersecurity information to large audiences. However, while these gamified approaches show
various levels of success, there are limitations. For example, card game approaches such as
Control-Alt-Hack~\cite{denning2013control} do not cover a broad range of key cybersecurity topics,
such as those identified by knowledge frameworks such as CyBOK, and do not adequately highlight
the links between vulnerabilities, attacks and defences. Specifically, attacks are typically
highlighted first, which does not help users understand how attacks present themselves (opportunistic
vulnerability targeting) and how to protect against them. While CTF activities, for example, are
beneficial in this aspect~\cite{swann2021open,trickel2017shell}, a key disadvantage pertains to
novice users wherein competitions rely on technical expertise and the ability to traverse computer systems
using various command-line tools and other bespoke applications~\cite{ford2017capture}, or requiring (at
a minimum) a basic understanding of cybersecurity concepts in order to progress in finding vulnerabilities~\cite{mcdaniel2016capture}.