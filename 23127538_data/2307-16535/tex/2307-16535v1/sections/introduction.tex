%-------------------------------------------------------------------------------
\section{Introduction}
\label{sec:introduction}
%-------------------------------------------------------------------------------

% In the information age, reliance on computer-based systems is constantly increasing.
Cybersecurity remains a fundamental concern to users of computer
systems, with security often being overlooked due to its portrayal as a subject
pertaining to issues of perceived technical difficulty, steep learning curves and a requirement of specialist
knowledge and/or
expertise~\cite{harknett2009cybersecurity,hoffman2011thinking,asen2019you}. While the security foundations of
computer-based systems have improved over time, limiting the potential for, or mitigating the effects of, attacks
arising from  vulnerabilities, requires the involvement of all users of these systems (e.g. the general
population) and is a necessary step to improve the understanding of cybersecurity~\cite{adams1999users}. Furthermore,
the increasing complexity and diversity of the threat landscape for
cybersecurity~\cite{wall2017crime,kaloudi2020ai,bahizad2020risks} further substantiates the need for improving
understanding of cybersecurity.

In the domain of software engineering, practical solutions to achieve this include activities such as the documentation
of vulnerabilities  of computer systems and updating respective knowledge bases. Open databases such as the Common
Vulnerabilities and Exposures (CVE)~\cite{cve} and Common Weakness Enumeration (CWE)~\cite{martin2011cwe}, have played
a pivotal role in raising the awareness of known vulnerabilities such that appropriate defensive measures can be
developed or updated. While these reference databases are well maintained, they may still appear
complex to the general population and may contribute to the already existing problems of inaccessibility and
specialist requirements that are pinned against the topic of cybersecurity. Because of this, several knowledgebases
have been developed to inform and underpin cybersecurity education and
training~\cite{martinintroduction,newhouse2017national,ncscmsc}, which aim to address these issues at a high-school or
higher-education level. Although they may be a useful learning resource for providing key cybersecurity knowledge, their
primary purpose is to be used by those who are already knowledgeable in cybersecurity to develop further curricula to
teach those who may have little-to-no knowledge of cybersecurity. Furthermore, among these knowledgebases, there may be
some key topics which are not covered, or the format and density of the knowledgebase may not be perceived as accessible
to novice users, both of which directly impacting one's ability to understand key cybersecurity topics but also make
links between these topics to capture real-world cybersecurity scenarios.

In this paper, we aim to address these limitations by proposing the use of a playing cards format as a medium to
provide: introductory knowledge of key cybersecurity topics, acting as an index for the CyBOK
knowledgebase~\cite{cybokwebsite,martinintroduction}; support independent learning and self-efficacy; and allow for
links to be made between key cybersecurity topics to capture real-world scenarios. The novelty of this work is three-fold.
First, we present the design principles for the cybersecurity cards to meet these provisions and develop an initial deck
of cards. Second, we provide an evaluation of these cards, carried out using workshop activity to devise cybersecurity
scenarios using participants at the level of education which existing knowledgebases target. The output of this evaluation
is a second revised deck of the cybersecurity cards. Third, we carried out the same workshop but with a different
demographic to the first, with participants at a primary and secondary school level, to not only compare and contrast
the findings from the first workshop but to also propose a third and final deck of cybersecurity cards.
%We first present the design principles
%for the cybersecurity cards to adequately meet these provisions and develop an initial deck of cards. The cards were
%then evaluated in a workshop where participants would interact with the cards to devise various cybersecurity scenarios.
%Observing the results and gathering insights from this workshop, we next provide a second version of these cards
%
%In this paper, we aim to facilitate access to and the communication of cybersecurity topics,
%to alleviate existing concerns and address the aforementioned requirements. Specifically,
%we propose the use of playing cards as a medium to represent key
%cybersecurity knowledge points. The advantages of using a playing cards format include:
%{\em accessible handling} in a physical medium~\cite{altice2014playing} and providing {\em digestible}
%information via a {\em tangible index}~\cite{malone2008use,setiaji2021development,arnab2015mapping} for
%cybersecurity knowledge bases.
%Each card proposed is designed to be self-contained, but allows for links to be made
%with other cards to build various cybersecurity scenarios surrounding the relationships
%between vulnerabilities, attacks and defences. We first present the design principles
%for the cybersecurity cards based on these requirements and developed an initial deck of these cards. The cards were evaluated in a workshop
%where interacting participants were split into small groups and supported by cybersecurity expert to help with
%using the cards whenever required. During the workshop, the cybersecurity experts were available at all times and would periodically check in
%on participants to make sure they understood what they were meant to be doing.
%% such that each group could hold discussions with varying perspectives provided by each expert.
%Overall, we found that the majority of participants in the workshop agreed that the
%cards provided them with knowledge of key cybersecurity topics and wider scope of these topics.
%As well as this, they also noted the cards provoked and promoted independent learning and
%self-efficacy, whilst also providing them access to key topics and making links between
%vulnerabilities, attacks and defences.
%Although the results were promising and demonstrate
%the success of our playing cards approach, there were some limitations reported by participants,
%including: the number of cards being too large and hard to handle; lack of use for certain
%types of cards; and the links between vulnerabilities, attacks and defences being unclear
%in some areas. Finally, to address these limitations, we propose several updates to the design of the
%cards and present the revised Version 2.

The remainder of this paper is organised as follows. Section~\ref{sec:background}
provides background and related work, as well as the selection
procedure we applied to the production of our cybersecurity cards using the CyBOK knowledge base and the
limitations of other approaches. The design principles
applied to the cybersecurity cards, as well as the initial implementation (Version 1), are described
in Section~\ref{sec:cards}. An evaluation of Version 1 of the cards and discussion of the results is provided in
Section~\ref{sec:evaluation} along with Version 2 of the cards. In Section~\ref{sec:cards1}, the evaluation and
discussion of the results from the second workshop using Version 2 of the cards is provided, ending with the design
of Version 3 of the cards. Finally, the paper concludes in Section~\ref{sec:conclusion}.