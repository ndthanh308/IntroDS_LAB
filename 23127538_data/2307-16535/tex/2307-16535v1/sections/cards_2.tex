%-------------------------------------------------------------------------------
\section{Cybersecurity Cards -- Version 2}
\label{sec:cards1}
%-------------------------------------------------------------------------------

Upon review of the results from the evaluation questionnaire, the
cybersecurity cards were in places redesigned and the new deck is hereafter
referred to as {\em Version 2}. As well as providing the subset of the Version 1 cards for viewing, the equivalent
subset for Version 2 of the cards have also been made available and can be found
online\footnote{\url{https://anonymous.4open.science/r/cybersecurity_cards-9F00/} (anonymous repository for double-blind reviewing purposes)}. The full set of cards will be made publicly available under the CC BY-NC-SA Creative Commons license after the research project ends in January 2024.

% \MMcomment{Potentially have a table to showcase the key comparison points / differences between v1.0 and v1.1}

\subsection{Structural Redesign}

One of the limitations that required addressing was with regard to
the number of cards in the deck, which some participants felt was
too high and {\em "difficult to handle and understand (PM7)"}. As described
in Section~\ref{sec:cards}, Version 1 of the deck has 109 cards in total,
consisting of 32 attack cards, 30 vulnerability cards and 47 defence cards,
categorised under one of the 15 General cards. In contrast with a standard
deck of playing cards used for the likes of Solitaire and Poker, which has 52 cards,
Version 1 of our cards has more than double this amount resulting in it being
perceived by participants as hard to handle. Therefore, reducing the number of cards in
Version 2 of the cards to a more familiar number will likely be better received, whilst
still retaining the goals from Version 1 of providing introductory knowledge of cybersecurity,
facilitating independent learning and self-efficacy, and providing an interface for
discussion of key topics.

In Version 2, the first major revision is reducing the number
of cards in the deck. The first decision made with regard to this was the removal of General
cards, given that participants specified the least used cards to be the General
cards which denote a {\em type} of attack, defence or vulnerability card. One participant
stated that {\em "general cards gave some idea about the content (PF6)"}, however this does
not justify their need to be placed within the deck. Instead, Version 2 of the cards come
with a {\em glossary} document which provides a description of each of
the General cards, alongside a symbol associated with the type in the top left of the card
to help improve readability and visibility. This reduces the number of cards in Version 2
to a total of 70 cards, made up of 20 attack, 20 vulnerability and 30 defence cards. On each
of the attack, vulnerability and defence cards, the symbol for its associated type found in the
glossary is in the top left corner (Figure~\ref{fig:manytomanycardsv2}).
By providing a glossary, users can refer to this sheet as a form of
guidance when looking at devising particular cybersecurity scenarios, as
well as addressing the lack of usage of general cards and poor handling due to
having a large deck. The descriptions of each of the types of attack, defence or vulnerability
in the glossary improve on the grammar and readability from those present in the General cards in
Version 1. Furthermore, while the symbol icons stay the same, we also propose two new classes of vulnerability.
First, {\em Human} is split into {\em User} and {\em Management} which captures the impact of
vulnerabilities from careless or malicious user insiders, but also bad understandings from management in
organisations. Second, we provide a {\em System} vulnerability type which relates to computer infrastructure,
to better capture attacks and defences that target the computer system, as opposed to assuming that this is
simply a result of badly written code or poorly configured execution environments as observed in Version 1.
The glossary can be found in Appendix~\ref{app:glossary}.

In terms of general design, examples of Version 2 cards can be seen in
Figure~\ref{fig:manytomanycardsv2}. The red background is made
slightly lighter making it easier on the eyes, with the symbols
watermarked in the background of the card also made lighter to
maintain focus on the content of the card whilst also still
highlighting the card type. Further, the fonts across the cards
were changed to improve clarity. Finally, defence cards are now
more akin to a stop sign which helps distinguish them with the
attack cards.


% Figure environment removed

\subsection{Relationship Clarification}

The next revision made in Version 2 of the cards is a clarification
of the links between the cards to highlight relationships between
various vulnerabilities, attacks and defences, which was described to be inadequate. As shown in
Figure~\ref{fig:manytomanycardsv2}, the card identifiers which were
initially positioned in the top right corner (Figure~\ref{fig:carddesign})
are now replaced with a symbol and a card number related to each type
which is positioned in the top left corner \circled{1}. As previously mentioned,
this symbol corresponds to its associated type which is defined in the glossary
(Appendix~\ref{app:glossary}). The card number is the number of the card within a type.
For example, {\em Command / Data Injection} is card number 2 for the {\em Injection}
attack type.

As well as this, instead of related attacks and defence codes listed at the bottom of
the vulnerability cards, attack and defence cards explicitly showcase related
vulnerability cards denoted by the relevant type symbol(s) made darker along with
the card number(s) for the type of vulnerability. Similarly, vulnerability cards
use the same format for showcasing the related attacks and defences
\circled{2}+\circled{3}. The defence cards do not explicitly refer to
specific attacks for which they target as, given the dynamic nature of
cybersecurity and evolving threat landscapes, a single defence strategy
could be employed to defend against multiple attacks or mitigate the chance
for vulnerabilities to be exploited. Instead, links are given to vulnerabilities
but the aim behind this is to encourage card users to intuitively tackle
attack vs. defence strategies by learning from the content within the cards and
explicitly highlight the relationships to various attacks and defences.


\subsection{Evaluation of Version 2}

To evaluate the second version of the cards, we make use of the same workshop format described in
Section~\ref{sec:evaluation}. However, the difference here pertains to the demographic of the participants. In this
second workshop, we recruited 23 participants with ages ranging from 10 to 15 years old (mean 12.8 years), of which
7 were female and 16 were male, from either primary or secondary school. When asked to describe any experience and/or
skills they may already have with cybersecurity, one participant stated they know the basics, one saying that
passwords are important, one stating they know about hacking, and all other participants saying they have either no
experience (17) or are unsure (3). When asked about experience or skills with coding, two participants stated they had
limited experience with Python programming, six stating they have used Scratch or other block-based visual programming
tools, with the remainder (15) having no programming experience. Furthermore, only 2 participants stated they had
experience with secure coding, with one neutral and the remainder either disagreeing or strongly disagreeing with this.
After the workshop, the participants were given the same questionnaire to answer as those received in the first workshop
(Appendix~\ref{app:questionnaire}). For subsequent discussion, we will discuss the results of the evaluation on the
second version of the cards on the same themes as the first version presented in Section~\ref{sec:results1}: (1)
{\em providing a knowledge base, including: cybersecurity concepts, scope and relationships
(vulnerability-attack-defence)}; (2) {\em independent learning and self-efficacy}; (3) {\em providing an interface for
discussion on key cybersecurity topics}; and (4) any {\em drawbacks of using the cards}.

\subsubsection{Providing a Knowledge Base}
\label{sec:provideknowledge2}

For the first theme of determining whether the cards provided introductory cybersecurity knowledge that supports
learning in a well-documented manner, we look at the four items (\textbf{a--d}) in Question 1 of the questionnaire in
Appendix~\ref{app:questionnaire}. For individual concepts \textbf{(a)}, 17 of the participants agreed that the cards
provided them with knowledge of cybersecurity concepts, terminology and topics (6 somewhat agree, 4 agree and 7 strongly
agree) with 3 participants scoring neutral and 3 somewhat disagreeing with this. For wider scope \textbf{(b)}, 17
participants agreed (3 somewhat agree, 6 agree and 8 strongly agree) that the cards provided them with this knowledge,
with 2 disagreeing and 4 remaining neutral. For relationships between attacks, defences and vulnerabilities \textbf{(c)},
16 agreed (6 somewhat agree, 3 agree and 7 strongly agree) that the cards facilitated the understanding of the links
between cards that capture the vulnerability-attack-defence dichotomy that is present in cybersecurity, with 4 neutral
and 3 disagreeing. Finally, with regarding to cybersecurity terminology \textbf{(d)}, 14 participants agreed (4 somewhat
agree, 4 agree and 6 strongly agree) that the cards provided them with knowledge on terminology. Three participants
disagreed with this and the remaining 6 were neutral.

%% questions 1(a-d)
%% 1a -- 3 somehwat disagree, 3 neutral, 6 somewhat agree, 4 agree, 7 strongly agree
%% 1b -- 2 disagree, 4 neutral, 3 somewhat agree, 6 agree, 8 strongly agree
%% 1c -- 3 disagree, 4 neutral, 6 somewhat agree, 3 agree, 7 strongly agree
%% 1d -- 3 disagree, 6 neutral, 4 somewhat agree, 4 agree, 6 strongly agree


\subsubsection{Independent Learning and Self-Efficacy}
\label{sec:independentlearning2}

The theme of independent learning and self-efficacy relate to items (\textbf{e,f}) in Question 1. The first item
asked participants if the cards helped them undertake independent learning of cybersecurity \textbf{(e)}. We found that
for the second version of the cards, 11 participants agreed (2 somewhat agree, 4 agree, 5 strongly agree), 7 were
neutral and 5 disagreeing (1 strongly disagreeing). For the second item, we asked if the cards provided access to
cybersecurity knowledge without the presence of an expert. We found that 13 participants agreed (3 somewhat agree, 3
agree and 7 strongly agree), with 5 participants neutral to this and 5 disagreeing (1 strongly disagreeing).

%% questions 1(e-f)
%% 1e -- 1 strongly disagree, 4 disagree, 7 neutral, 2 somewhat agree, 4 agree, 5 strongly agree
%% 1f -- 1 strongly disagree, 4 disagree, 5 neutral, 3 somewhat agree, 3 agree, 7 strongly agree


\subsubsection{Providing an Interface for Discussion}
\label{sec:provideinterface2}

The final theme explored relates to the cards providing an interface for discussing key cybersecurity topics and relates
to items (\textbf{g, h}) in Question 1. The first question related to this theme asked participants if the cards enabled
them to discuss cybersecurity topics with an expert \textbf{g}. We found that 14 participants agreed with this (3
somewhat agree, 5 agree and 6 strongly agree), with 6 participants remaining neutral to this and 3 disagreeing (1
strongly disagreeing). The final question in this theme asked them if the cards enabled them to hold discussions on key
cybersecurity topics with others in their group \textbf{h}. We found that 14 participants agree with this (4 somewhat
agree, 4 agree and 6 strongly agree), with 5 participants neutral and 4 disagreeing with this.

%% questions 1(g-h)
%% 1g -- 1 strongly disagree, 2 disagree, 6 neutral, 3 somewhat agree, 5 agree, 6 strongly agree
%% 1h -- 4 disagree, 5 neutral, 4 somewhat agree, 4 agree, 6 strongly agree


\subsubsection{Understanding Drawbacks of Version 2}
\label{sec:limitations2}

In a similar manner to Version 1 of the cards, the next three questions were designed to gather understanding on any
limitations of the cards and whether there may be any improvements that can be made. Given that the same investigation
was performed for Version 1, the aim here is to observe whether the design and structural changes we made using the
findings to revise Version 1 of the cards have allowed for better fulfilment of the goals of the research questions
described in Section~\ref{sec:cards}. In Question 2, we asked participants whether there were any categories or subset
of cards not used by them in the workshop. The difference in this question compared to the same questionnaire employed
for evaluating Version 1 is due to the removal of the General cards and that they are replaced with a Glossary. Thus, in
Appendix~\ref{app:questionnaire}, we refer to {\em Question 2 (Version 2 Evaluation)} which includes the Glossary as an
option, as well as cards within a specific category for each of the classes. We found that the least used item was the
Glossary (7 participants) and Race Condition (7), with the next least used cards being Injection (6) and Memory (6) attack
cards, Mitigation (6) and Education (6) defence cards, and System and Environment vulnerability cards. Looking at
Question 3 to understand why these types of cards (excluding the glossary) in particular were not used, some
participants mentioned that they ``didn't need them'' (PM8), were ``unnecessary'' (PM20) or ``don't know'' (PM9). One
reason for this may be due to cards such as race conditions sound more intimidating and/or complex, in particular for
those are less experienced (in this case participants from primary or secondary school, in comparison with students from
higher education backgrounds in Version 1). With regard to the glossary, some participants stated that ``nobody reads
the glossary'' (PF10), suggesting that this was not needed and the cards alone for some participants were enough to
facilitate learning and understanding of key cybersecurity topics. Interestingly, some participants had stated that
they had ``used all the cards'' (PM4, PF6), with some others giving a similar reason yet selecting many (if not all)
checkboxes for all cards. In Question 4, we asked participants if there were any other improvements to the cards they
would like to suggest. Most participants (14) said ``No'' to this question, however one participant suggested to use
``simple wording'', which may correlate with why certain cards (e.g. Race Condition or Memory attacks) were not used.
As well as this, three participants suggested to make the cards ``easier to use'' (PM2,PM14), however any suggestions as
to how they might think this could be achieved were not elaborated.

% Question 2 - which category/subset cards not used, tick all boxes
%% Note: This question is different as we no longer have general/detailed cards, we refer to
%%       "Question 2 (Version 2) of cards".
%%
% Glossary - 7
% Attack cards (in general) (fully red cards) - 4
% Vulnerability cards (in general) (cards with half red/half white diagonal) - 4
% Defence cards (in general) (white cards with red border) - 5
% Attack -- Injection cards - 6
% Attack -- Memory cards - 6
% Attack -- Race condition cards - 7
% Attack -- Side channel cards - 5
% Attack -- Authentication cards - 5
% Attack -- Web cards - 4
% Attack -- System cards - 5
% Attack -- Human factors cards - 4
% Defence -- Detection cards - 5
% Defence -- Mitigation cards - 6
% Defence -- Education cards - 6
% Defence -- Prevention cards - 4
% Vulnerability -- Code cards - 5
% Vulnerability -- System cards - 6
% Vulnerability -- Environment cards - 6
% Vulnerability -- User cards - 2
% Vulnerability -- Management cards - 4
% Other - "I used all of them", "other", "no"

%\subsubsection{Future Work}