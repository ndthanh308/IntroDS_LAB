% \documentclass[conference]{IEEEtran}
\documentclass[10pt,conference]{IEEEtran}

% Packages
\usepackage{cite}
\usepackage{amsmath,amssymb,amsfonts}
\usepackage{algorithmic}
\usepackage{graphicx}
\usepackage{textcomp}
\usepackage{xcolor}

\newcommand{\vlad}[1]   {\noindent \textcolor{cyan}{[#1 -Vlad]}}
\newcommand{\majid}[1]   {\noindent \textcolor{orange}{[#1 -Majid]}}
\newcommand{\waseem}[1]   {\noindent \textcolor{blue}{[#1 -Waseem]}}
\newcommand{\niraj}[1]   {\noindent \textcolor{pink}{[#1 -Niraj]}}
\newcommand{\tommi}[1]   {\noindent \textcolor{magenta}{[#1 -Tommi]}}

% Document content
\begin{document}

\makeatletter
\newcommand{\linebreakand}{%
  \end{@IEEEauthorhalign}
  \hfill\mbox{}\par
  \mbox{}\hfill\begin{@IEEEauthorhalign}
}
\makeatother


%\title{Full Stack Quantum Computing in Practice: Ecosystems, Stakeholders and Challenges}
\title{Full-Stack Quantum Software in Practice: Ecosystem, Stakeholders and Challenges}

% Authors
\author{
\IEEEauthorblockN{Vlad Stirbu}
\IEEEauthorblockA{%\textit{Department of XYZ} \\
\textit{University of Jyväskylä}\\
Jyväskylä, Finland \\
vlad.a.stirbu@jyu.fi}
\and
\IEEEauthorblockN{Majid Haghparast}
\IEEEauthorblockA{%\textit{Department of XYZ} \\
\textit{University of Jyväskylä}\\
Jyväskylä, Finland \\
majid.m.haghparast@jyu.fi}
\linebreakand
\IEEEauthorblockN{Muhammad Waseem}
\IEEEauthorblockA{%\textit{Department of XYZ} \\
\textit{University of Jyväskylä}\\
Jyväskylä, Finland \\
muhammad.m.waseem@jyu.fi}
\and
\IEEEauthorblockN{Niraj Dayama}
\IEEEauthorblockA{%\textit{Department of XYZ} \\
\textit{University of Jyväskylä}\\
Jyväskylä, Finland \\
niraj.r.dayama@jyu.fi}
\and
\IEEEauthorblockN{Tommi Mikkonen}
\IEEEauthorblockA{%\textit{Department of XYZ} \\
\textit{University of Jyväskylä}\\
Jyväskylä, Finland \\
tommi.j.mikkonen@jyu.fi}
% \and 
% \IEEEauthorblockN{Author 3}
% \IEEEauthorblockA{\textit{Department of XYZ} \\
% \textit{University of ABC}\\
% City, Country \\
% email1@example.com}

% \and
% \IEEEauthorblockN{Author 4}
% \IEEEauthorblockA{\textit{Department of XYZ} \\
% \textit{University of ABC}\\
% City, Country \\
% email2@example.com}
}

\maketitle

% Abstract
\begin{abstract}
The emergence of quantum computing has introduced a revolutionary paradigm capable of transforming numerous scientific and industrial sectors. Nevertheless, realizing the practical utilization of quantum software in real-world applications presents significant challenges. Factors such as variations in hardware implementations, the intricacy of quantum algorithms, the integration of quantum and traditional software, and the absence of standardized software and communication interfaces hinder the development of a skilled workforce in this domain. This paper explores tangible approaches to establishing quantum computing software development process and addresses the concerns of various stakeholders. By addressing these challenges, we aim to pave the way for the effective utilization of quantum computing in diverse fields.
\end{abstract}

% Keywords
\begin{IEEEkeywords}
quantum computing, software development process, operations, quantum software engineering
\end{IEEEkeywords}

\maketitle

% The problem of the presence or absence of phase transition is central in statistical mechanics. To prove the existence of phase transition, the standard idea is to define a notion of contour and use \textit{Peierls' argument} \cite{Peierls.1936}. In the usual Ising model \cite{Ising_25}, particles of the system interact only with their nearest-neighbors. On ferromagnetic long-range Ising models \cite{Anderson_Yuval_69}, there is interaction between each pair of spins in the lattice. The Hamiltonian of the model is given formally by
\begin{equation*}
    H(\sigma) = - \sum_{x,y\in \Z^d}J_{xy}\sigma_x\sigma_y,
\end{equation*}
where $J_{xy}=J|x-y|^{-\alpha}$, $J>0$, $\alpha > d$. It is well-known that the Peierls' argument in dimension 2 implies phase transition for Ising models with nearest-neighbors or long-range interactions when $d\geq 2$, using correlation inequalities. For the unidimensional lattice, it was known that short-range models do not present phase transition. In the long-range case, a different behavior was expected depending on the exponent $\alpha$ (see \cite{Kac_Thompson_69}), but the problem was challenging since contours were first created as multidimensional objects.

In dimension $d=1$, phase transition was proved first in 1969 by Dyson \cite{Dyson.69}, for $\alpha \in (1,2)$, by proving phase transition in an auxiliary model and then using correlation inequalities. In 1982, Fr{\"o}hlich and Spencer \cite{Frohlich.Spencer.82} introduced a notion of one-dimensional contours and then applied the Peierls' argument to show phase transition for the critical value $\alpha = 2$. These contours were inspired by the multiscale techniques previously introduced to study the Berezinskii-Kosterlitz-Thouless transition in two-dimensional continuous spin systems \cite{FS81}. Later, Cassandro, Ferrari, Merola and Presutti  \cite{Cassandro.05} extended the contour argument previously available for $\alpha=2$ to exponents $\alpha\in (3-\frac{\ln 3}{\ln 2}, 2)$, with the additional restriction that the nearest-neighbor interaction is strong, i.e.,  ${J(1)\gg 1}$; this restriction was removed for a subclass of interactions in \cite{Bissacot.Endo.18}. Further results were obtained using contour arguments, such as the decay of correlations, cluster expansions, phase transition with random interactions, etc; some references with these results are \cite{ Cassandro.Merola.Picco.17, Cassandro.Merola.Picco.Rozikov.14, Imbrie.82, Imbrie.Newman.88, Johansson.91}. 

In the multidimensional setting ($d\geq 2$), Ginibre, Grossmann, and Ruelle, in \cite{Ginibre.Grossmann.Ruelle.66}, proved the phase transition for $\alpha > d+1$, using an enhanced version of Peierls' argument and the usual contours. Park proposed a different notion of contour for long-range systems in \cite{Park.88.I, Park.88.II}, extending the Pirogov-Sinai theory available for short-range interactions assuming $\alpha > 3d+1$, although he can also consider Potts models with his methods. Some results in the literature suggest that truly long-range effects appear only when $d < \alpha \leq d+1$, see for instance, \cite{Biskup_Chayes_Kivelson_07}. Recently, Affonso, Bissacot, Endo and Handa \cite{Affonso.2021}, inspired by the ideas from Fr{\"o}hlich and Spencer in \cite{FS81, Frohlich.Spencer.82}, introduced a version of multiscale multidimensional contour and proved phase transition by a contour argument in the whole region $\alpha > d$. They can consider long-range Ising models with deterministic decaying fields, first introduced in the context of nearest-neighbor interactions in \cite{Bissacot_Cioletti_10}. For these models, the lack of analyticity of the free energy does not imply phase transition since these models have the same free energy as the models with zero field. It is expected that fields decaying slowly imply uniqueness. In this setting, a contour argument is useful for proofs of phase transitions as well for uniqueness, some papers with models with deterministic decaying fields are \cite{Aoun_Ott_Velenik_23, Bissacot_Cass_Cio_Pres_15, Bissacot.Endo.18, Cioletti_Vila_2016}.

The Random Field Ising model (RFIM) \cite{Imry.Ma.75} is the nearest-neighbor Ising model with an additional external field acting on each site $(h_x)_{x\in\Z^d}$ that is a family of i.i.d. Gaussian random variable with mean 0 and variance 1. Formally, the Hamiltonian of the model is given by
\begin{equation*}
    H(\sigma) = - \sum_{\substack{x,y\in \Z^d \\|x-y|=1}}J\sigma_x\sigma_y  - \varepsilon\sum_{x\in\Z^d}h_x\sigma_x,
\end{equation*}
where $J>0$, $\varepsilon>0$, $\alpha > d$ and $d \geq 1$. A detailed account of the history of the phase transition problem for this model, as well as detailed proofs, was given in \cite{Bovier.06}. Here we present a brief overview.

During the 1980s, the question of the specific dimension where phase transition for the RFIM should happen attracted much attention and was a topic of heated debate. Two convincing arguments were dividing the physics community. One of them, due to Imry and Ma \cite{Imry.Ma.75}, was a non-rigorous application of the Peierls' argument together with the use of the isoperimetric inequality. The key idea of Peierls' argument is to define a notion of contour and calculate the energy cost of "erasing" each contour, i.e., the energy cost of flipping all spins inside the contour. When there is no external field, that energy necessary to flip the spins in a region $A\subset \Z^d$ is of the order of the boundary $|\partial A|$. When we add an external field, we get an extra cost depending on this field. Imry and Ma argued that this cost should be approximately $\sqrt{|A|}$, which is smaller than $|\partial A|$ for all regions only when $d\geq 3$, so this should be the region where phase transition occurs. The other argument, due to Parisi and Sourlas \cite{Parisi.Sourlas.79}, based on dimensional reduction, predicted that the $d$-dimensional RFIM would behave like the $d-2$-dimensional nearest-neighbor Ising model, therefore presenting phase transition only when $d\geq 4$. 

The question was settled by two celebrated papers showing that Imry and Ma's prediction was correct. First, in 1988, Bricmont and Kupiainen \cite{Bricmont.Kupiainen.88} showed that there is phase transition almost surely in $d\geq3$, for low temperatures and variance $\varepsilon$ small enough. Their proof uses a rigorous renormalization group analysis for the short-range case and it is considered involved. Still, they claimed that the result works for any model with a suitable contour representation and centered sub-gaussian external field. Later on, Aizenman and Wehr \cite{Aizenman.Wehr.90} proved uniqueness for $d\leq 2$. For detailed proofs of these results, we refer the reader to \cite{Bovier.06} (see also \cite{Berretti.85, Camia.18, Frohlich.Imbre.84,  Klein.Masooman.97} for more uniqueness results). 

Recently, Ding and Zhuang, see \cite{Ding2021}, provided a simpler proof of the phase transition, not using RGM. And in  \cite{Ding.Liu.Xia.22}, Ding, Liu and Xia proved that if $\beta_c(d)$ is the critical inverse of the temperature of the Ising model with no field, for all $\beta>\beta_c(d)$ there exists a critical value $\varepsilon_0(d, \beta)$ such that the RFIM with $\varepsilon \leq \varepsilon_0$ presents phase transition. 

In the present paper, we are considering a long-range Ising model with a random field, whose Hamiltonian is given formally by
\begin{equation*}
    H(\sigma) = - \sum_{x,y\in \Z^d}J_{xy}\sigma_x\sigma_y - \varepsilon\sum_{x\in\Z^d}h_x\sigma_x,
\end{equation*}
where $J_{xy}=J|x-y|^{-\alpha}$, $J, \varepsilon>0$, $\alpha > d$ and $h_x\in\mathbb{R}$, $d\geq 3$.
Until now, the only known result in the long-range setting is for the one-dimensional long-range Ising model with a random field, by Cassandro, Orlandi, and Picco \cite{Cassandro.Picco.09}. They used the contours of \cite{Cassandro.05} to show the phase transition for the model when $\alpha\in (3-\frac{\ln 3}{\ln 2}, \frac{3}{2})$, under the assumption $J(1) \gg 1$. We stress that, as remarked by Aizenman, Greenblatt, and Lebowitz \cite{Aizenman_Greenblatt_Lebowitz_2012}, although their argument does not work for the whole region for the exponent $\alpha$, the phase transition holds for values close to the critical value $\alpha=3/2$, since by the Aizenman-Wehr theorem we know that there is uniqueness for $\alpha>3/2$.

The argument from Ding and Zhuang in \cite{Ding2021}, for $d\geq3$, involves controlling the probability of a bad event, which is closely related to controlling the quantity $$\sup_{\substack{0\in A\subset\Z^d \\ A \text{ connected }}}\frac{\sum_{x\in A}h_x}{|\partial A|},$$ known as the greedy animal lattice normalized by the boundary. The greedy animal lattice normalized by the size, instead of the boundary, was extensively studied for general distributions of $(h_x)_{x\in\Z^d}$, see \cite{Cox_Gandolfi_Griffin_Kesten_93, Gandolfi_Kesten_94, Hammond_06, Martin_02}. When we normalize by the boundary, an argument by Fisher, Fr\"{o}hlich and Spencer \cite{FFS84} shows that the expected value of the greedy animal lattice is constant. In dimension $d=2$, the expected value is not finite, see \cite{Ding.Wirth.20}. The supremum is taken over connected regions containing the origin since the interiors of the usual Peierls contours are of this form.


For the long-range model, the interior of contours is not necessarily connected. In fact, long-range contours may have considerably large diameters with respect to their size, so their interiors can be very sparse. To avoid this, we define contours, strongly inspired by the $(M,a,r)$-partition in \cite{Affonso.2021}, using a multiscaled procedure that assures that the contours have no cluster with small density.  With them, we generalize the arguments by Fisher-Fr\"{o}hlich-Spencer \cite{FFS84}, and prove that the expected value of the greedy animal lattice is constant, even considering regions not necessarily connected in the supremum. Then, we prove the phase transition for $d\geq 3$. The main result of this paper is the following.
\begin{theorem*}Given $d\geq 3$, $\alpha>d$, there exists $\beta_c\coloneqq\beta(d, \alpha)$ and $\varepsilon_c\coloneqq\varepsilon(d, \alpha)$ such that, for $\beta >\beta_c$ and $\varepsilon\leq \varepsilon_c$, the extremal Gibbs measures $\mu_{\beta, \varepsilon}^+$ and $\mu_{\beta, \varepsilon}^-$ are distinct, that is, $\mu_{\beta, \varepsilon}^+ \neq \mu_{\beta, \varepsilon}^-$ $\mathbb{P}$-almost surely. Therefore the long-range random field Ising model presents phase transition.
\end{theorem*}

This paper is divided as follows. In Section 2, we define the model and the contours, and suitable generalizations to the constructions in \cite{Affonso.2021} are introduced.  In Section 3, we define two bad events of the external field and prove that they occur with a small probability.  In Section 4, we present the proof of the phase transition.
% \input{Background}

\section{Introduction}

Quantum computing holds great promise as a revolutionary technology that has the potential to transform various fields. By harnessing the principles of quantum mechanics, quantum computers can perform complex calculations and solve problems that are currently intractable for classical computers. This promises breakthroughs in areas such as cryptography, optimization, drug discovery, materials science, and machine learning. Quantum computing's ability to leverage quantum mechanics properties like superposition, interference and entanglement can unlock exponential speedups and enable more accurate simulations of quantum systems.

The development of quantum software faces numerous challenges that need to be addressed for harnessing the power of quantum computing effectively. Firstly, the limited availability and instability of quantum hardware pose significant obstacles. Quantum computers are prone to errors and noise, necessitating the development of robust error correction techniques. Additionally, quantum programming languages and tools are still in their nascent stages, requiring advancements to facilitate efficient software development. Furthermore, the scarcity of skilled quantum software developers and a lack of standardization hinder the widespread adoption of quantum software. As quantum systems scale, the complexity of designing and optimizing quantum algorithms increases, demanding novel approaches to algorithm design and optimization. Addressing these challenges is crucial for realizing the full potential of quantum computing and enabling the development of practical quantum software applications.

%\waseem{The summary of the related work should  go here.....}

This paper explores the challenges and approaches to establishing a quantum computing software development process. It highlights the obstacles in realizing practical utilization of quantum software, such as hardware variations, algorithm complexity, integration with traditional software, and the lack of standardized interfaces. Furthermore, the paper emphasizes the need to address these challenges to enable effective utilization of quantum computing.

\section{Background}
\label{background}

\subsection{Qubit implementation}

The current candidates for building general-purpose quantum computers, as listed in Table \ref{tab:quantum_qubit_tech}, fall under the category of Noisy Intermediate-Scale Quantum (NISQ) systems. Although these quantum computers are not yet advanced enough to achieve fault-tolerance or reach the scale required for quantum supremacy, they provide an experimentation platform to develop new generations of hardware, develop quantum algorithms and validate quantum technology in real world usecases. Whether a quantum computer is general-purpose or specialized, the selection of quantum qubit implementation technology can significantly enhance hardware efficiency for specific problem classes. To make effective use of the hardware, application developers must consider these differences when designing and optimizing the software's functionality and operations.

\begin{table*}[htbp]
\caption{Quantum Computing Qubit Implementation Technologies}
\label{tab:quantum_qubit_tech}
\centering
% \renewcommand{\arraystretch}{1.3}
\begin{tabular}{|l|p{7cm}|p{7cm}|}
\hline
\textbf{Qubit Technology} & \textbf{Description} & \textbf{Applicability} \\
\hline
Superconducting & Tiny superconducting materials are cooled to extremely low temperatures to manifest their quantum properties. & General-purpose quantum computing, suitable for various types of problems. \\
\hline
Trapped Ion & Ions are trapped within electromagnetic fields. & General-purpose quantum computing, with potential for high coherence and low error rates. \\
\hline
Topological & A new approach to quantum computing that leverages the properties of topological states of matter to create qubits. Unlike other qubit technologies, which typically rely on individual particles like ions or electrons, topological qubits are based on collective properties of an ensemble of particles. & General-purpose quantum computing, aimed at achieving fault-tolerant operations. \\
\hline
Photonic & Quantum information is stored in photons that can be manipulated and transmitted over long distances. & General-purpose quantum computing, suitable for communication and cryptography applications. \\
\hline
Annealing & Special purpose quantum computers designed to solve optimization problems. & Specialized quantum computing, specifically targeted at optimization and sampling problems. \\
\hline
\end{tabular}
\end{table*}

\subsection{Quantum algorithms}

%[Outline of algorithms that promise advantages over classical approaches. Not too deep, just references. The intent is to introduce algorithm developers as special kind of stakeholders, different than software developers.]

Quantum algorithms are computational techniques specifically designed to harness the unique properties of quantum systems \cite{Montanaro2016}. They offer significant advantages over classical algorithms in certain computational tasks. One key advantage is the ability to solve complex problems exponentially faster. For example, Shor's algorithm enables efficient factoring of large numbers, posing a potential threat to current encryption methods. Also, Grover's algorithm provides substantial speedup in searching large databases. Moreover, quantum algorithms can address optimization problems more effectively, leading to improved solutions in areas like portfolio optimization, logistics, and drug discovery.

\subsection{Software}

A typical quantum program performs a specialized task as part of a larger classical program, see Fig. \ref{fig:model}. The quantum program is submitted as a batch task to a classical computer that controls the operation of the quantum computer. The classical computer schedules the task execution and provides the result to the classical program when the job completes.

% Figure environment removed

% [Low-level/circuit-level: QisKit/Cirq]
Application developer use tools like Qiskit\footnote{https://qiskit.org} and Cirq\footnote{https://quantumai.google/cirq} for writing, manipulating and optimizing quantum circuits. These Python libraries allow researchers and application developers to interact with nowadays' NISQ computers, allowing them to run quantum programs on a variety of simulators and hardware designs, abstracting away the complexities of low-level operations and allowing researchers and developers to focus on algorithm design and optimization.

% [toolkits: TensorFlow Quantum/PennyLane]
Tools like TensorFlow Quantum\footnote{https://www.tensorflow.org/quantum} and PennyLane\footnote{https://pennylane.ai} play a crucial role in facilitating the development of machine learning quantum software. These frameworks provide the high-level abstractions and interfaces that bridge the gap between quantum computing and classical machine learning. They allow researchers and developers to integrate quantum algorithms seamlessly into machine learning development process by providing access to quantum simulators and hardware, as well as offering a range of quantum-friendly classical optimization techniques. TensorFlow Quantum leverages the power of Google's TensorFlow ecosystem, enabling the combination of classical and quantum neural networks for hybrid quantum-classical machine learning models. PennyLane offers a unified framework for developing quantum machine learning algorithms, supporting various quantum devices and seamlessly integrating them with classical machine learning libraries. These tools provide a foundation for researchers to explore and experiment with quantum machine learning, accelerating the progress and adoption of quantum computing in the field of machine learning.

%[tools: Jupyter notebooks, simulators]
Jupyter Notebooks and quantum simulators play a vital role in supporting developers of quantum programs. Jupyter provides an interactive and collaborative environment where developers can write, execute, and visualize their quantum code in an accessible manner. They allow for the combination of code, explanatory text, and visualizations, making it easier to experiment, iterate, and document the development process. Quantum simulators, on the other hand, enable developers to simulate the behavior of quantum systems without the need for physical quantum hardware. These simulators provide a valuable testing ground for verifying and debugging quantum algorithms, allowing developers to gain insights into their performance and behavior before running them on actual quantum devices. Developers can iterate quickly, gain a deeper understanding of quantum concepts, and refine their quantum programs efficiently.

Traditional cloud computing providers, such as AWS Bracket\footnote{https://aws.amazon.com/braket/}, Azure Quantum\footnote{https://learn.microsoft.com/en-us/azure/quantum/}, Google Quantum AI\footnote{https://quantumai.google} or IBM Quantum\footnote{https://quantum-computing.ibm.com}, offer comprehensive quantum development services. These services are designed to optimize the development process, with integrated tools like Jupyter notebooks and task schedulers. Developers can create quantum applications and algorithms across multiple hardware platforms simultaneously. This approach ensures flexibility, allowing fine-tune algorithms for specific systems while maintaining the ability to develop applications that are compatible with various quantum hardware platforms.

\subsection{Operations}

% [SDLC]
The software development lifecycle (SDLC) of quantum programs involves a series of stages tailored to the unique challenges of quantum computing \cite{sdlc}. It typically begins with requirements gathering and problem formulation, where developers identify the specific problem that the quantum program aims to solve. During algorithm design, the developers design quantum algorithms that leverage the unique capabilities of quantum systems. The designed algorithm implementation translates the algorithm into quantum code using quantum programming languages and frameworks like Qiskit or Cirq. After implementation, the program undergoes rigorous testing and debugging, using quantum simulators to validate its functionality and behavior. The tested program is executed on actual quantum hardware, with careful consideration given to the limitations and noise inherent in quantum systems. Finally, ongoing maintenance and optimization are crucial, as quantum hardware, software frameworks, and algorithms evolve rapidly.

% [simulate HW noise, virtualization]
Simulators and virtualization offer significant advantages to quantum computing from an operations perspective. Simulators provide a virtual environment for testing and debugging quantum programs without the need for physical quantum hardware. Ops teams can validate code, identify errors, and optimize performance in a controlled and reproducible manner. Simulators also allow ops teams to simulate larger-scale quantum systems than currently available in physical hardware, providing insights into the behavior and scalability of quantum programs. Additionally, virtualization techniques enable the efficient allocation and management of quantum resources, allowing multiple users to access and share quantum computing resources securely. Ops teams can provision virtualized quantum environments, manage access controls, and monitor resource utilization effectively.

\section{Full Stack Quantum Computing}

In this section we explore the \textit{full stack} quantum computing from two perspectives: development process - looking at how they are developed, and composition - looking at how quantum applications are structurally organised and the factors that need to be considered when operationalizing the execution of applications utilizing quantum computing components.

\subsection{Development process}

% Figure environment removed

The SDLC of applications incorporating quantum technology involves streams of activities encompassing both classical and quantum components, see Fig. \ref{fig:quantum-sdlc}. At the top level, the classical software development process begins by identifying user needs and deriving system requirements. These requirements are transformed into a design and implemented, followed by verification against the requirements and validation against user needs. Once the software system enters the operational phase, any detected anomalies are used to inform potential new system requirements, if necessary. Concurrently, a dedicated track for quantum components is followed within the SDLC, specific to the implementation of quantum technology. The requirements for these components are converted into a design, which is subsequently implemented, verified, and integrated into the larger software system. The development occurs on simulators running on classical computers, which can simulate the noise characteristic of actual quantum hardware. During the operational phase, the quantum software components are executed on real hardware. Scheduling ensures efficient utilization of scarce quantum hardware, while monitoring capabilities enable the detection of anomalies throughout the process.

This workflow enables the development of products that include quantum technology using both plan-based and iterative development practices. However, when it comes to the DevOps aspects of quantum computing \cite{quantum-devops}, it becomes crucial to focus on practices and activities that facilitate effective monitoring of the quantum components operating in the production environment. %This entails implementing monitoring mechanisms to ensure the smooth functioning and performance of quantum components throughout their deployment and execution.

\subsection{Composition}

% Figure environment removed

From an architecture perspective, we can identify the following three layers: user, infrastructure and hardware (depicted in Fig. \ref{fig:relationships}). The \textit{user} software refers to the end user programs and the components developed by third parties, such as general purpose (e.g. Qiskit Terra\footnote{https://github.com/Qiskit/qiskit-terra}) or specialised (e.g. OpenFermion\footnote{https://github.com/quantumlib/OpenFermion} or TensorFlow Quantum\footnote{https://www.tensorflow.org/quantum}) libraries of quantum algorithms and circuits (e.g. Cirq and Qiskit). The \textit{infrastructure} layer contains the software needed to develop (e.g. simulators), test under realistic scenarios (e.g. simulate the noise of NISQ hardware) and run quantum programs at scale (e.g. task schedulers). The \textit{hardware} layer contains the software specific for each hardware architecture, such as the software that drives the control circuits.

\section{Goals, challenges and future research directions}

% Our exploration of full-stack quantum computing focuses on identifying the challenges and difficulties in quantum software development. By leveraging the principles and practices of continuous software engineering, such as DevOps, which enable small, multidisciplinary teams to iterate quickly and deliver high quality traditional software, we aim to pinpoint the specific components and interfaces that facilitate the transfer and application of these practices in the context of quantum software applications. Through this exercise, we seek to enhance our understanding of the pain points and opportunities for improvement in quantum software development, ultimately striving to foster the seamless integration of best practices from traditional software engineering into the emerging field of quantum computing.

% The first observation is that, despite the currently high operational costs, the quantum-specific software and hardware components are relatively small compared to the overall system in which quantum technology is employed. Specifically, the design and implementation steps encompass quantum software development, while the execution step involves running the quantum components on real hardware. This characteristic aligns with previous observations made in software systems incorporating machine learning (ML) technology \cite{ml-technical-depth}. The main challenges arise in two key areas: technical challenges arise from integrating classical and quantum components, while process challenges emerge from aligning the technical solution with user needs and requirements. These observations highlight the need to address technical and process-related hurdles in order to successfully integrate quantum technology into software systems while effectively meeting user expectations.

% The second observation is that, from a development perspective, the quantum software \textit{debugging} is fundamentally different than for classical software. The black box nature of the quantum computer, with its limited observability, limits the debugging capabilities. Although new quantum debugging techniques are developed \cite{debugging}, they are far from the ability to stop the execution and inspecting its state at any point in time that is typically found in classical computing. Overcoming these limitations require new development approaches that require modular software development and reliable intermediate verification.

% The third observation highlights the presence of multiple stakeholders that contribute various software and hardware components at both the classical and quantum levels. While most stakeholders focus on specific areas like quantum algorithm or hardware development, influential entities such as Google and IBM have a significant presence and influence across the entire technology stack. They are driven by diverse economic and technological interests, which can either align or conflict with one another. Similar to the design principles behind the internet \cite{tussle}, the full-stack of quantum software must be designed to accommodate these inherent conflicts by establishing well-defined trust boundaries and open interfaces. This approach that works along the tussles among the stakeholders is crucial for fostering the development of a robust commercial environment that encourages continuous investments from both public and private entities \cite{qtm}.

%\textbf{Goals}: 
Our exploration of full-stack quantum computing focuses on identifying the challenges and difficulties in quantum software development. By leveraging the principles and practices of continuous software engineering, such as DevOps, which enable small, multidisciplinary teams to iterate quickly and deliver high-quality traditional software, we aim to pinpoint the specific components and interfaces that facilitate the transfer and application of these practices in the context of quantum software applications. Through this exercise, we seek to enhance our understanding of the pain points and opportunities for improvement in quantum software development, ultimately striving to foster the seamless integration of best practices from traditional software engineering into the emerging field of quantum computing \cite{ml-technical-depth}.

%\textbf{Challenges}: 
The main challenges emerge from two areas: technical -- integrating classical and quantum components, and process -- aligning the technical solution with user needs and requirements. These observations highlight the need to address technical and process-related hurdles in order to successfully utilize quantum technology while effectively meeting user expectations. From a development perspective, the quantum software debugging is fundamentally different than for classical software. The black box nature of the quantum computer, with its limited observability, limits the debugging capabilities. Although new quantum debugging techniques are developed \cite{debugging}, they are far from the ability to stop the execution and inspect its state at any point in time that is typically found in classical computing. Overcoming these limitations require new development approaches that require modular software development and reliable intermediate verification.

%\textbf{Future Research Directions}: 
Multiple stakeholders contribute various software and hardware components at both the classical and quantum levels. While most stakeholders focus on specific areas like quantum algorithm or hardware development, influential entities such as Google and IBM have a significant presence and influence across the entire technology stack. They are driven by diverse economic and technological interests, which can either align or conflict with one another. Similar to the design principles behind the internet \cite{tussle}, the full-stack of quantum software must be designed to accommodate these inherent conflicts by establishing well-defined trust boundaries and open interfaces. This approach that works along the tussles among the stakeholders is crucial for fostering the development of a robust commercial environment that encourages continuous investments from both public and private entities \cite{qtm}.


% Similar tussles as for the design of internet \cite{tussle} should be considered when designing quantum software systems...

% \section{Conceptual Case-study exercise}
% \label{Case study}
% We consider a conceptual exercise of a case study that discusses how the quantum computing software development process compares to the traditional software development process.   

\section{Conclusion}

Despite the novelty and the fundamentally new approach of quantum computing, the software development shares many characteristics with classical software engineering. Making reliable quantum software requires careful design that incorporates the best practices from classical computing, while focusing the development effort on specific high value components that improve the development experience and lower the operational costs.

\section*{Acknowledgement}
This work has been supported by the Academy of Finland (project DEQSE 349945) and Business Finland (project TORQS 8582/31/2022).

% \section{Research Method}
\label{researchmethod}
% \section{Results of RL active flow control}\label{sec:Results}

In this section, we discuss the converge of the RL algorithms for the three FM and PM cases (\S\ref{subsec:Convergence}) and evaluate their drag reduction performance (\S\ref{Result_drag_reduction}). A parametric analysis of the effect of NARX memory length is presented (\S\ref{subsec:Nfs}) and the isolated effect of including past actions as observations during the RL training and control (\S\ref{subsec:past_actions}). Studies of reward function (\S\ref{subsec:Rewards_Study}), sensor placement (\S\ref{subsec:Sensor_study}) and generalisability to Reynolds number changes (\S\ref{subsec:Res}) are presented, followed by a comparison of SAC and TQC algorithms (\S\ref{subsec:SACvsTQC}). 

\subsection{Convergence of learning}\label{subsec:Convergence}

We perform RL with the maximum entropy TQC algorithm to discover control policies for the three cases shown in figure \ref{fig:Case_Demo}, which maximise the net-power-saving reward function given by \req{eq: PowerR}. During the learning stage, each episode (1 DNS simulation) corresponds to $200$ non-dimensional time units.  To accelerate learning, $65$ environments run in parallel.


Figure \ref{fig:Learning_Curve} shows the learning curves of the three cases.  Table \ref{tab:LearningConvergence} shows the number of episodes needed for convergence and relevant parameters for each case.
It can be observed from the curve of episode reward that the RL agent is updated after every 65 episodes, i.e. $1$ iteration, where the episode reward is defined as 
\begin{equation}
R_{ep} = \sum_{k=1}^{N_k} r_{k},
\label{eq:Epi_R}
\end{equation}
where $k$ denotes the $k^{th}$ RL step in one episode and $N_k$ is the total number of samples in one episode.
The root mean square (RMS) value of the drag coefficient, $C_D^{RMS}$, at the asymptotic regime of control, is also shown to demonstrate convergence, defined as 
$C_D^{RMS} = \sqrt { (\mathcal{D}(\langle C_D\rangle_{env}))^2 }$,
where the operator $\mathcal{D}$ detrends the signal with a $9^{th}$-order polynomial and removes the transient part, and $\langle ~ \rangle_{env}$ denotes the average value of parallel environments in a single iteration. 

% Figure environment removed

\begin{table}
  \begin{center}
\def~{\hphantom{0}}
  \begin{tabular}{lcccccc}
    
      Environment  & Algorithm  &  $N_{c}$ & $R_{ep,c}$ & (Layers, Neurons) & $N_{fs}$ & Number of Inputs \\ 
       FM-Static   & TQC & $325$ & $37.72$ & (3,512) & $0$ & $64p_t+2a_{t-1}$\\
       PM-Static   & TQC & $1235$ & $21.87$ & (3,512) & $0$ & $64p_t+2a_{t-1}$\\
       PM-Dynamic  & TQC & $715$ & $34.35$ & (3,512) & $27$ & $N_{fs} (64p_t+2a_{t-1})$\\
  \end{tabular}
  \caption{Number of episodes $N_{c}$ required for RL convergence in different environments. The episode reward $R_{ep,c}$ at the convergence point, the configuration of NN and the dimension of inputs are presented for each case. $N_{fs}$ is the finite-horizon length of past actions-measurements.}
  \label{tab:LearningConvergence}
  \end{center}
\end{table}

In figure \ref{fig:Learning_Curve}, it can be noticed that in the FM environment, RL converges after approximately $325$ episodes ($5$ iterations) to a   {nearly} optimal policy using a static   {feedback} controller. As will be shown in \S\ref{Result_drag_reduction}, this policy is globally optimal since the vortex shedding is fully attenuated and the jets converge to zero mass flow actuation, thus recovering the unstable base flow and the minimum drag state.  However, with the same static   {feedback} controller in a PM environment (POMDP), the RL agent fails to discover the   {nearly} optimal solution, requiring around $1235$ episodes for convergence but only obtaining a relatively low episode reward.
Introducing a dynamic   {feedback} controller in the PM environment, the RL agent convergences to a near-optimal solution in 735 episodes. The dynamic   {feedback} controller trained by RL achieves a higher episode reward (34.35) than the static   {feedback} controller in the PM case (21.87), which is close to the FM case (37.72). The learning curves illustrate that using a finite horizon of past actions-measurements ($N_{fs} = 27$) to train a dynamic   {feedback} controller in the PM case improves learning in terms of speed of convergence and accumulated reward achieving nearly optimal performance with only wall pressure measurements. 


\subsection{Drag reduction with dynamic RL controllers} \label{Result_drag_reduction}

% Figure environment removed

The trained controllers for the cases shown in figure \ref{fig:Case_Demo} are evaluated to obtain the results shown in figure \ref{fig:TQC_FMPM}.   {Evaluation tests are performed for 120 non-dimensional time units to show both transient and asymptotic dynamics of the closed-loop system.}
Control is applied at $t=0$ with the same initial condition for each case, i.e. steady vortex shedding with average drag coefficient $\langle C_{D0}\rangle \approx 1.45$ (baseline without control). Consistent with the learning curves, the difference in control performance in the three cases can be observed both from the drag coefficient $C_D$ and the actuation $Q_1$.
  {The drag reduction is quantified by a ratio $\eta$ using the asymptotic time-averaged drag coefficient with control $C_{Da} = \langle C_{D}\rangle_{t \in [80,120]}$, the drag coefficient $C_{Db}$ of the base flow (details presented in Appendix \ref{App:BaseFlow}), and the baseline time-averaged drag coefficient without control $\langle C_{D0}\rangle$, as
\begin{equation}
\eta = \frac{\langle C_{D0}\rangle - C_{Da}}{\langle C_{D0}\rangle - C_{Db}} \times 100\%.
\label{eq:drag_reduction}
\end{equation}}

\begin{itemize}

\item {\bf FM-Static:} With a static   {feedback} controller trained in a full-measurement environment, a drag reduction of $\eta = 101.96\%$ is obtained with respect to the base flow (steady unstable fixed point; maximum drag reduction). This indicates that an RL controller informed with full-state information can entirely stabilise the vortex shedding and cancel the unsteady part of the pressure drag.

\item {\bf PM-Static:} A static/memoryless controller in a partial-measurement environment leads to performance degradation and a drag reduction of   {$\eta = 56.00\%$} in the asymptotic control stage, i.e. after $t=80$, compared to the performance of ``FM-Static''. This performance loss can also be observed from the control actuation curve, as $Q_1$ oscillates with a relatively large fluctuation in ``PM-Static'' while it stays about zero in the ``FM-Static'' case. 
The discrepancy between FM and PM environments using a static   {feedback} controller reveals the challenge of designing a controller with a POMDP environment. The RL agent cannot fully identify the dominant dynamics with only partial measurements on the   {downstream} surface of the bluff body, resulting in sub-optimal control behaviour.

\item{\bf PM-Dynamic:} With a dynamic   {feedback} controller (NARX model presented in \S\ref{subsec:PM_Dynamic}) in a partial-measurement environment, the vortex shedding is stabilised and the dynamic   {feedback} controller achieves   {$\eta = 97.00\%$} of the maximum drag reduction after time $t=60$. Although there are minor fluctuations in the actuation $Q_1$, the energy spent in the synthetic jets is significantly lower compared to the ``PM-Static'' case. Thus, a dynamic   {feedback} controller in PM environments can achieve nearly optimal drag reduction, even if the RL agent only collects information from pressure sensors on the   {downstream} surface of the body. The improvement in control indicates that the POMDP due to the PM condition of the sensors can be reduced to an approximate MDP by training a dynamic   {feedback} controller with a finite horizon of past actions-measurements. Furthermore, high-frequency action oscillations, which can be amplified with static   {feedback} controllers, are attenuated in the case of dynamic   {feedback} control. These encouraging and unexpected results support the effectiveness and robustness of model-free RL control in practical flow control applications, in which sensors can only be placed on a solid surface/wall.

\end{itemize}


% Figure environment removed

In figure \ref{fig:Contour}, snapshots of the velocity magnitude   {$|\boldsymbol{u}| = \sqrt{u^2+v^2}$} are presented for ``Baseline'' without control, ``PM-Static'', ``PM-Dynamic'' and ``FM-Static'' control cases. Snapshots are captured at $t=100$ in the asymptotic regime of control. A vortex-shedding structure of different strengths can be observed in the wake of all three controlled cases. In ``PM-Static'', the recirculation area is lengthened compared to the baseline flow, corresponding to base pressure recovery and pressure drag reduction. A longer recirculation area can be noticed in ``PM-Dynamic'' due to the enhanced attenuation of vortex shedding and pressure drag reduction. The dynamic   {feedback} controller in the PM case renders a $326.22\%$ increase of recirculation area with respect to the baseline flow, while only a $116.78\%$ increase is achieved by a static   {feedback} controller. The ``FM-Static'' case has the longest recirculation area, and the vortex shedding is almost fully stabilised, which is consistent with the drag reduction shown in figure \ref{fig:TQC_FMPM}.

% Figure environment removed

Figure \ref{fig:Obs} presents first- and second-order base pressure statistics for the baseline case without control and PM cases with control. In figure \ref{fig:Obs}(a), the time-averaged value of base pressure, $\overline{p}$, demonstrates the base pressure recovery after control is applied. Due to flow separation and recirculation, the time-averaged base pressure is higher at the middle of the   {downstream surface}, which is retained with control. The base pressure increase is directly linked to pressure drag reduction, which quantifies the control performance of both static and dynamic   {feedback} controllers. Up to $49.56\%$ of pressure increase at the centre of the   {downstream surface}  is obtained in the ``PM-Dynamic'' case, while only $21.15\%$ can be achieved by a static   {feedback} controller. In figure \ref{fig:Obs}(b), the base pressure RMS is shown. For the baseline flow, strong vortex-induced fluctuations of the base pressure can be noticed around the top and bottom   {on the downstream surface} of the bluff body. In the ``PM-Static'' case, the RL controller   {partially suppresses} the vortex shedding, leading to a sub-optimal reduction of the pressure fluctuation. The sensors close to the top and bottom corners are also affected by the synthetic jets, which change the RMS trend for the two top and bottom measurements. In the ``PM-Dynamic'' case,  the pressure fluctuations are nearly zero for all the measurements on the   {downstream surface}, highlighting the success of vortex shedding suppression by a dynamic RL controller in a PM environment.

% Figure environment removed

The differences between static and dynamic controllers in PM environments are further elucidated in figure \ref{fig:Action_analysis} by examining  the time series of pressure differences $\Delta p_t$ from surface sensors (control input) and control actions $a_{t-1}$ (output). The pressure differences are calculated from sensor pairs at $y=\pm y_{sensor}$, where $y_{sensor}$ is defined in Eq. \req{eq:Probe_base}. For $N=64$, there are 32 time series of $\Delta p_t$ for each case. 
%
During  the initial stages of control ($t \in [0,11]$), the control actions are similar  for the two PM cases and they deviate for $t>11$, resulting in discernible control performance at the asymptotic regime. 
At the initial stages, the controllers operate in nearly anti-phase to $\Delta p_t$, in order to eliminate the antisymmetric pressure component due to vortex shedding. The inability of the static controller to have a frequency dependent amplitude (and phase), manifests as well through the amplification of high frequency noise. For $t>11$, the static feedback controller continues to operate in nearly anti-phase to the pressure difference, resulting in partial stabilisation of unsteadiness. However, the dynamic feedback controller adjusts its phase and amplitude significantly, which attenuates the antisymmetric fluctuation of base pressure and drives $\Delta p_t$ to near zero. 

% Figure environment removed

Figure \ref{fig:ContourComparision} shows instantaneous vorticity contours for PM-Dynamic and PM-Static cases, showing both the similarities and discrepancies between the two cases. At $t=2$, flow is expelled from the bottom jet for both cases, generating a clockwise vortex, termed V1. This V1 vortex, shown in black, works against the primary counter-clockwise vortex labelled as P1, depicted in red, emerging from the bottom surface. At $t=5.5$, a secondary vortex, V2, forms from the jets to oppose the primary vortex shedding from the top surface (labelled as P2). 
%
 At $t=13$, the suppression of the two primary vortices near the bluff body is evident in both cases, indicated by their less tilted shapes compared to the previous time instances. At $t=13$, the PM-Dynamic adjusted the phase of the control signal, which corresponds to a marginal action at this time instance at figure \ref{fig:Action_analysis}. Consequently, no additional counteracting vortex is formed in PM-Dynamic. However, in the PM-Static scenario, the jets generate a third vortex, labelled V3, which emerges from the top surface. This corresponds to a peak in the action of the PM-Static controller at this time. The inability of the PM-Static controller to adapt the amplitude/phase of the input/output behaviour results in suboptimal performance.

\subsection{Horizon of the finite-history sufficient statistic}\label{subsec:Nfs}

A parametric study on the horizon of the finite history in NARX (equation \req{eq:NARX}), i.e. the number of frames stacked $N_{fs}$, is presented in this section. Since the NARX model uses a finite horizon of past actions-measurements in  \req{eq:Sufficient_statistic}, the horizon of the finite history affects the convergence of the approximation \citep{yu_near_2008}. This approximation affects the optimisation during the learning of RL because it determines whether the RL agent can observe sufficient information to converge to an optimal policy. 

Since vortex shedding is the dominant instability to be controlled, the choice of $N_{fs}$ should intuitively link to the timescale of the vortex shedding period. The ``frames'' of observations are obtained every RL step ($0.5$ time units), while the vortex shedding period is $t_{vs}\approx6.85$ time units. Thus, $N_{fs}$ is rounded to integer values for different numbers of vortex shedding periods, as shown in table \ref{tab:Frame_Stack}.


% Figure environment removed

\begin{table}
  \setlength{\tabcolsep}{12pt}
  \begin{center}
\def~{\hphantom{0}}
  \begin{tabular}{ccc}
      Number of  & Non-dimensional &  History length \\
      VS periods &    time units          &  ($N_{fs}$)         \\ [3pt]
      \hline
       0.5   & 3.43 & 7 \\
       1   & 6.85 & 14 \\
       2  & 13.70 & 27 \\
       3 & 20.55 & 41\\
       4 & 27.40 & 55\\
       5 & 34.25 & 68\\
  \end{tabular}
  \caption{Correspondence between the number of vortex shedding (VS) periods and frame stack (history) length in samples $N_{fs}$. The RL control step size is $t_a =0.5$, and $N_{fs}$ is rounded to an integer.}
  \label{tab:Frame_Stack}
  \end{center}
\end{table}

The results of time-averaged drag coefficients $\langle C_{D}\rangle$ after control and the average episode rewards $\langle R_{ep}\rangle$ in the final stage of training are presented in figure \ref{fig:Frame_Stack}. As $N_{fs}$ increases from 0 to 27, the performance of RL control improves, resulting in a lower $\langle C_{D}\rangle$ and a higher $\langle R_{ep}\rangle$. $N_{fs}=2$ is specially examined because the latent dimension of the vortex shedding limit cycle is 2. However, the control performance with $N_{fs}=2$ is marginally improved to the one with $N_{fs}=0$, i.e. a static   {feedback} controller. This result indicates that the horizon consistent with the vortex shedding dimension is not long enough for the finite horizon of past action measurements. The optimal history length to achieve stabilisation of the vortex shedding   {in PM environments} is 27 samples, which are equivalent to 13.5 convective time units or $\sim 2$ vortex shedding periods. 

With $N_{fs}=41$ and $N_{fs}=55$, the drag reduction and episode rewards drop slightly compared to $N_{fs}=27$. The decline in performance is non-negligible as $N_{fs}$ increases further to 68. This decline shows that excessive inputs to the neural networks (see table \ref{tab:LearningConvergence}), may impede training because more parameters need to be tuned or larger neural networks need to be trained. 

\subsection{Observation sequence with past actions}\label{subsec:past_actions}

Past actions (exogenous terms in NARX) facilitate reducing a POMDP to an MDP problem, as discussed in \S\ref{subsec:PM_Dynamic}. In the near-optimal control of a PM environment using a dynamic   {feedback} controller with inputs $\left( o_t, o_{t-1}, ..., o_{t-N_{fs}} \right)$, a sequence of observations $o_t = \left \{ p_t, a_{t-1}\right \}$ at step $t$ is constructed to include pressure measurements and actions. In the FM environment, due to the introduction of one-step delayed action due to the first-order-hold interpolation given by \req{eq:FOH_action}, the inclusion of the past action along with the current pressure measurement, meaning $o_t = \left \{ p_t, a_{t-1} \right \}$, is required even when the sensors are placed in the wake and cover the wavemaker region. 

Figure \ref{fig:ActionInObs} presents the control performance for the same environment with and without past actions included.
In the FM case, there is no apparent difference between RL control with $o_t = \left \{ p_t, a_{t-1} \right \}$ or $o_t = \left \{ p_t \right \}$, which indicates that the inclusion of the past action is negligible to the performance. This is the case when the RL sampling frequency is sufficiently faster than the timescale of the vortex shedding dynamics. 
In PM cases, if exogenous action terms are not included in the observations but only the finite history of pressure measurements is used, the RL control fails to converge to a near-optimal policy, with only   {$\eta = 67.45\%$}  drag reduction. With past actions included, the drag reduction of the same environment increases up to   {$\eta = 97.00\%$}. 

The above results show that in PM environments, sufficient statistics cannot be constructed only from the finite history of measurements. Missing state information needs to be reconstructed by both state-related measurements and control actions. 

% Figure environment removed

\subsection{Reward study}
\label{subsec:Rewards_Study}

In \S\ref{Result_drag_reduction}, a power-based reward function given by \req{eq: PowerR} has been implemented, and stabilising controllers can be learned by RL, as shown. In this section, RL control results with other forms of reward functions (introduced in \S\ref{subsec:Reward}) are provided and discussed.

% Figure environment removed

The control performance of RL control with the different reward functions is evaluated based on the drag coefficient $C_D$ shown in figure \ref{fig:Reward_Study}. Static   {feedback} controllers are trained in FM environments, and dynamic   {feedback} controllers are trained in PM environments. In FM cases, control performance is not sensitive to the choice of reward function (power or force-based).  
In PM cases, the discrepancies between RL-step time-averaged and instantaneous rewards can be observed in the asymptotic regime of control. The controllers with both rewards (power or force-based) achieve nearly optimal control performance, but there is some unsteadiness in the cases using instantaneous rewards due to slow statistical convergence of the rewards and limited correlation to the partial observations.

All four types of reward functions studied in this work achieve nearly optimal drag reduction around $100\%$. However, the energy-based reward (``PowerR'') offers an intuitive reward design, attributable to its physical properties and the dimensionally consistent addition of the constituent terms of the reward function. Further enhancing its practicality, since the power of the actuator can be directly measured, it avoids the necessity for hyperparameter tuning, as in the force-based reward. Additionally, the results show similar performance with both time-averaged between RL steps and instantaneous rewards, avoiding the necessity for faster sampling for the calculation of the rewards. This choice of reward function can be extended to various RL flow control problems and can be beneficial to experimental studies.


\subsection{Sensor configuration study with partial measurements}\label{subsec:Sensor_study}

% Figure environment removed

In the PM environment, the configuration of sensors (number and location on the downstream surface) may also affect the information contained in the observations and thus control performance. 
Control results of drag coefficient $C_D$ for different sensor configurations in PM-dynamic cases are presented in figure \ref{fig:Sensor_config}. In the configuration with $N = 2$, two sensors are placed at $y=\pm 0.25$, and for $N = 1$, only one sensor is placed at $y = 0.25$. Other configurations are consistent with equation \req{eq:Probe_base}. 

The $C_D$ curves in figure \ref{fig:Sensor_config} show that, as the number of sensors is reduced from 64 to 2, RL control achieves the same level of performance with minor discrepancies due to randomness in different learning cases. However, if RL control uses observations from only one sensor at $y = 0.25$, performance degradation can be observed in the asymptotic stage with 19.79\% on average less drag reduction. The sub-figure presents the relationship between the number of sensors and asymptotic drag coefficient $\langle C_D \rangle$. These results indicate a limit on sensor configuration for the use of the NARX-modeled controller to stabilise the vortex shedding. 

% Figure environment removed

To understand the cause of performance degradation in the $N=1$ case, the pressure measurements from two sensors in both baseline and PM-Dynamic cases are presented in figure \ref{fig:Pressure2Sensors}. In the baseline case, two sensors are placed at the same location as the $N=2$ case ($y=\pm 0.25$) only for observations. It can be observed that the pressure measurements from two sensors are anti-symmetric since they are placed symmetrically on the downstream surface.
In the PM-Dynamic case, the NARX controller is used, and control is applied at $t=0$. In this closed-loop system, the anti-symmetric relationship between two sensors (from the symmetric position) is broken by the control actuation, and no correlation is evident. This can be seen during the transient dynamics, e.g. in $t \in [0,10]$. Therefore, when the number of sensors is reduced to $N=1$ by removing one sensor from the $N=2$ case, the dynamic feedback from the removed sensor cannot be fully reflected by the remaining sensor in the closed-loop system. This loss of information affects the fidelity of the control response to the dynamics of the sensor-removing side, causing suboptimal drag reduction in the $N=1$ scenario.

It should be noted that the configuration of 64 sensors is not necessary for control, as $N = 2$ or $N = 16$ also achieves nearly optimal performance. The number of sensors $N = 64$ in PM-Static environments is used for comparison with the FM-Static configuration (Eq. \ref{eq:Probe_wake}), which eliminates the effect from different input dimensions between two static cases. Also, 64 sensors sufficiently cover the downstream surface of the bluff body to avoid missing spatial information. 
The optimal configuration of sensors can be tuned with optimisation techniques such as \cite{paris_robust_2021}, but the results in figure \ref{fig:Sensor_config} indicate that RL adapts with nearly optimal performance to non-optimised sensor placement in the present environment.

\subsection{Performance of RL controllers to unseen $Re$} \label{subsec:Res}

% Figure environment removed

The RL controller is tested at different Reynolds numbers, in order to examine its generalisability to environment changes. The controllers have been trained at $Re=100$ with both FM and PM conditions, and tested at $Re= 80, 90, 100, 110, 120, 150$. The controllers were further trained at $Re=150$, denoted as continual learning (CL), and tested again at $Re=150$. 

As shown in figure \ref{fig:Res}, in both ``PM-Dynamic'' and ``FM-Static'' cases, the RL controllers are able to reduce drag by $\eta=64.68\%$ in the worst case, when $Re$ is close to the training point at $Re=100$, i.e. the test cases with $Re= 80, 90, 100, 110, 120$. 
However, when applying the controllers trained at $Re=100$ to an environment at $Re=150$, the drag reduction drops to $\eta=41.98\%$ and $\eta = 74.04\%$ in PM-Dynamic and FM-Static cases, respectively.

Performing CL at $Re=150$, the drag reduction is improved to $\eta = 78.07\%$ in PM-Dynamic after 1105 training episodes while $\eta = 88.13\%$ in FM-Static after 390 episodes, with the same RL parameters as the training at $Re=100$.
Overall, the results of these tests indicate that the RL-trained controllers can achieve significant drag reduction in the vicinity of the training point (i.e. $\pm\%20$ $Re$ change). If the test point is far from the training point, a CL procedure can be implemented to achieve nearly optimal control.

\subsection{TQC vs SAC}\label{subsec:SACvsTQC}

% Figure environment removed

Control results with TQC and SAC are presented in figure \ref{fig:TQCvsSAC} in terms of $C_D$. TQC shows a more robust control performance. In the case of FM, SAC might demonstrate a slightly more stable transient behaviour attributed to the fact that the quantile regression process in TQC introduced complexity to the optimisation process. Both controllers achieved an identical level of drag reduction in the FM case. 

However, in the context of the PM cases, it is observed that TQC outperforms SAC in drag reduction with both static and dynamic   {feedback} controllers. For static   {feedback} control, TQC achieved an average drag reduction of   {$\eta = 56.00\%$}, compared to the   {$\eta = 46.31\%$}  reduction achieved by SAC. The performance under dynamic   {feedback} control conditions is more compelling, where TQC fully reduced the drag, achieving   {$\eta = 97.00\%$}  of drag reduction, reverting it to a near-base-flow scenario. In contrast, SAC managed to achieve an average drag reduction of   {$\eta = 96.52\%$}.

The fundamental mechanism for updating Q-functions in RL involves selecting the maximum expected Q-functions among possible future actions. This process, however, can potentially lead to overestimation of certain Q-functions \citep{hasselt_double_2010}. In POMDP, this overestimation bias might be exacerbated due to the inherent uncertainty arising from the partial-state information. Therefore, the Q-learning-based algorithm, when applied to POMDPs, might be more prone to choosing these overestimated values, thereby affecting the overall learning and decision-making process.

As mentioned in \S\ref{subsec:SACTQC}, the core benefit of TQC under these conditions can be attributed to its advanced handling of the overestimation bias of rewards. By constructing a more accurate representation of possible returns, TQC provides a more accurate Q-function approximation than SAC. This process of modulating the probability distribution of the Q-function assists TQC in managing the uncertainties inherent in environments with only partial-state information. In this case, TQC can adapt more robustly to changes and uncertainties, leading to better performance in both static and dynamic feedback control tasks.
% \section{Conclusion}
\label{sec:conclusion}


We presented DCA, an algorithm to address disparity in outcomes of ranking processes using compensatory bonus points. We showed that DCA, by relying on a sampling-based approach, successfully reduces disparity in a wide range of settings, while being significantly more efficient than state-of-the-art approaches, running in sub-linear time. This makes DCA a good candidate for iterative processes that would allow users to identify the ranking function that best fits their needs while checking for its fairness impacts and the required compensatory bonus points.  


Our approach relies on the use of compensatory bonus points, a departure from previous work, which has mostly focused on modifying the ranking function directly, or on the use of quotas. A significant advantage of compensatory bonus points is that they are transparent, interpretable, and easily explainable to all stakeholders.


% \section{Conclusion}\label{sec:conclusion}

This paper presents our empirical domain knowledge distillation framework using ChatGPT and discusses our observations from the framework application experiments in the autonomous driving domain. The key finding is that: 1) with proper design of prompt engineering and execution flow, fully automated domain knowledge (in the ontology format) distillation is possible. However, due to the randomness in the response and the butterfly effect, the quality of fully automated distillation results is not guaranteed. To address this, we develop a web-based assistant to enable manual supervision and early intervention at runtime. We hope our findings and tools inspire future research toward revolutionizing the engineering processes of knowledge-based systems across domains.

\bibliographystyle{ieeetr}
\bibliography{References}
\end{document}
