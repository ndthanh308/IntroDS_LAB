\documentclass[a4paper,11pt]{amsart}

\usepackage[left=3.0cm, right=3.0cm, top=2.50cm, bottom=2.50cm]{geometry}


\usepackage{graphicx}%
\usepackage{multirow}%
\usepackage{amsmath,amssymb,amsfonts}%
\usepackage{amsthm}%
\usepackage{mathrsfs}%
\usepackage[title]{appendix}%
%\usepackage{xcolor}%
\usepackage{textcomp}%
%\usepackage{manyfoot}%
%\usepackage{booktabs}%
\usepackage{enumitem}
%\usepackage{algorithm}%
%\usepackage{algorithmicx}%
%\usepackage{algpseudocode}%
%\usepackage{listings}%
%%%%%%%%%%%%%%%%%%%%%%% DEFINITIONS %%%%%%%%%%%%%%%%%%%%%%%

\theoremstyle{plain}
\newtheorem{theorem}{Theorem}[section]
\newtheorem{proposition}[theorem]{Proposition}
\newtheorem{lemma}[theorem]{Lemma}
\newtheorem{corollary}[theorem]{Corollary}
\newtheorem{fact}[theorem]{Fact}

\theoremstyle{definition}
\newtheorem{definition}[theorem]{Definition}
\newtheorem{example}[theorem]{Example}
\newtheorem{remark}[theorem]{Remark}

\begin{document}

%\tolerance=6000
\numberwithin{equation}{section}

% Sans serif operators
\def\L{\mathsf{L}}
\def\U{\mathsf{U}}

% Calligraphic operators
\def\LL{\mathcal{L}}
\def\UU{\mathcal{U}}
\def\SS{\mathcal{S}}
\def\TT{\mathcal{T}}

% Enumeration / itemize
\setlist[itemize]{label=$\circ$, itemsep=1pt, topsep=2pt}
\setlist[enumerate]{itemsep=1pt, topsep=2pt}

\title[Relational correspondences for $L$-fuzzy rough approximations]%
{Relational correspondences for $L$-fuzzy rough approximations defined on De~Morgan Heyting algebras}

\author[J.~J{\"a}rvinen]{Jouni J{\"a}rvinen}
\address[J.~J{\"a}rvinen]{Software Engineering, LUT School of Engineering Science, Mukkulankatu 19, 15210 Lahti, Finland}
\email{jouni.jarvinen@luf.fi}


\author{Michiro Kondo}
\address[M.~Kondo]{Department of Mathematics, School of System Design and Technology, Tokyo Denki University, Senju Asahi-cho 5, Adachi, Tokyo, Japan}
\email{mkondo@mail.dendai.ac.jp}


\keywords{rough set, $L$-fuzzy set, $L$-fuzzy relation, relational correspondence, De~Morgan Heyting algebra}


\begin{abstract}
Bin Pang, Ju-Sheng Mi, and Wei Yao introduced in 2019 the $L$-fuzzy rough approximation operations on De Morgan Heyting algebras. 
They presented axiomatic characterizations of $L$-fuzzy rough approximation operators corresponding to mediate, Euclidean and
adjoint $L$-fuzzy relations. In this paper, we extend these results by providing uniforms methods for obtaining characterization theorems for
$L$-fuzzy rough approximation operators. As an application of our results, we are able to give an answer to the problem left by Pang, Mi, and Yao.
We also correct a misunderstanding related to crisp positive alliance relations appearing in the literature.
\end{abstract}
    


\maketitle

\section{Introduction}

Rough sets were introduced by Z.~Pawlak in \cite{Pawlak82} to deal with concepts that cannot be defined precisely in terms of our knowledge.


In rough set theory, knowledge about objects $U$ is given in terms of an \emph{indistinguishability relation} $E$, which is an equivalence
$E$ on $U$ interpreted so that $x \, E \, y$ if $x$ and $y$ cannot be distinguished in terms of our in knowledge.
In terms of an indistinguishability relation $E$, we define for each $X \subseteq U$ the \emph{lower} and \emph{upper $E$-approximations}
by setting
\[ X_E = \{ x \in U \mid [x]_E \subseteq U\} \quad \text{and} \quad X^E = \{ x \in U \mid [x]_E \cap X \neq \emptyset \}, \]
where $[x]_E$ is the $E$-class of the element $x \in U$. It is well-known that $E$-approximations have the following properties for all $X \subseteq U$:

\begin{enumerate}[label = {(\roman*)}]
\item $X_E \subseteq X \subseteq X^E$,
\item $(X_E)^E \subseteq X \subseteq (X^E)_E$
\item $X_E = (X_E)_E = (X_E)^E$ \ and \ $X^E = (X^E)^E = (X^E)_E$.
\end{enumerate}


In the literature, numerous studies can be found  in which equivalences are replaced by different types of so-called
\emph{information relations} reflecting, for instance, similarity or preference between the elements of $U$;
see \cite{Orlowska1998}, for example. Let $U$ be a set and let $R$ be a binary relation on $U$. For any $x \in U$, we denote $R(x) = \{ y \mid x \, R \, y\}$. 
For all $X \subseteq U$, the \emph{lower} and \emph{upper approximations} of $X$ are defined by
\[ X_R = \{ x \in U \mid R(x) \subseteq X \} \mbox{ \quad and \quad }  X^R = \{ x \in U \mid R(x) \cap X \ne \emptyset \}, \]
respectively. The set $X_R$ may be interpreted as the set of elements that are \textit{certainly} in $X$, because  all elements 
to which $x$ is $R$-related are in $X$. Analogously, $X^R$ can be considered as the set of all elements that are \textit{possibly} in $X$, since in $X$ there
is at least one element to which $x$ is $R$-related.

Let us now introduce some general properties of binary relations. A binary relation $R$ on $U$ is
\begin{enumerate}[label = {(\roman*)}]
\item \emph{serial}, if for all $x \in U$, there is $y \in U$ such that $x \, R \, y$, 
\item \emph{reflexive}, if $x \, R \, x$ for all $x \in U$,
\item \emph{symmetric}, if $x \,R \, y$, then $y \, R \, x$ for all $x,y \in U$,
\item \emph{transitive}, if $x \, R \, y$ and $y \, R \, z$, then $x \, R \, z$ for all $x,y,z \in U$,
\item \emph{mediate}, if $x \, R \, y$ for some $x,y \in U$, there is $z \in U$ such that $x \, R \, z$ and $z \, R \, y$,
\item \emph{Euclidean}, if $x \, R \, y$ and $x \, R \, z$, then $y \, R \, z$ for all $x,y,z \in U$.
\end{enumerate}

\medskip\noindent%
Let $R$ be a binary relation on $U$. The following correspondence results can be found in the literature; see \cite{Jarvinen2005,Yao1996,Zhu2007}, for instance:
\begin{enumerate}[label = {(\roman*)}]
\item $R$ is serial $\iff U^R = U \iff \emptyset_R = \emptyset \iff (\forall X \subseteq U) \, X_R \subseteq X^R$,
\item $R$ is reflexive $\iff  (\forall X \subseteq U) \, X_R \subseteq X \iff (\forall X \subseteq U) \, X \subseteq X^R$,
\item $R$ is symmetric $\iff  (\forall X \subseteq U) \, (X_R)^R \subseteq X \iff (\forall X \subseteq U) \, X \subseteq (X^R)_R$,
\item $R$ is transitive $\iff (\forall X \subseteq U) \,
(X^R)^R \subseteq X^R \iff (\forall X \subseteq U) \,
X_R \subseteq (X_R)_R$.
\item $R$ is mediate $\iff (\forall X \subseteq U) \,
X^R \subseteq (X^R)^R \iff  (\forall X \subseteq U) \,
(X_R)_R \subseteq X_R$.
\item $R$ is Euclidean $\iff (\forall X \subseteq U) \,
X^R \subseteq (X^R)_R \iff  (\forall X \subseteq U) \,
(X_R)^R \subseteq X_R$.
\end{enumerate}

In recent years, research on constructing approximation operators from fuzzy relations has been carried \cite{DuboisPrade1990,radzikowska2004fuzzy}.
Wu and Chang \cite{WuZang2004} considered a setting in which $U$ and $W$ are two finite non-empty universes of discourse and $R$ a fuzzy relation from $U$ to $W$,
that is, $R(x,y)$ belongs to $[0,1]$ for all $x \in U$ and $y \in W$. For any set $A \subseteq W$, the upper and lower approximations of $A$ are fuzzy sets on $U$ defined by 
\begin{align*}
\overline{R}(A) &= \bigvee_{y \in W} ( R(x,y) \wedge A(y) ) \\
\underline{R}(A) &= \bigwedge_{y \in W} ( (1 - R(x,y)) \vee A(y) ) .
\end{align*}
In that paper, the authors presented correspondences for serial, reflexive, symmetric and transitive fuzzy relations in terms of the above approximation operations.

In \cite{Pang2019}, the authors generalized the setting by considering
$L$-fuzzy relation, where $L$ is a Heyting algebra provided with an antitone involution $'$. Three new types (mediate, Euclidean, adjoint) of $L$-fuzzy rough approximation operators were characterized.


In this paper, we extend these results and provide a uniform method to get the characterization theorems of $L$-fuzzy rough approximation operators. 
Concretely:
\begin{enumerate}[label = {(\roman*)}]
\item For an $L$-fuzzy relation $R$ and a finite sequence $\SS$ of $L$-fuzzy rough approximation operators $\L$ and $\U$, if $\U (1_x) \le \SS (1_x)$ for all 
$x\in U$, then $\U A \le \SS A$ for all $A \in \mathcal{F}_L(U)$.
\item Let $\UU$ be an $L$-fuzzy operator on $U$ and let each $\SS_j$, $1 \leq j \leq n$, be an $L$-fuzzy operator on $U$ such that $\UU \leq S_j$.
Then there exists a unique $L$-fuzzy relation $R$ on $U$ such that $\U A = \UU(A)$ for all $A \in \mathcal{F}_L(U)$
if and only if
\[
\UU(\overline{a} \wedge {\bigvee}_i A_i) = \overline{a} \wedge {\bigvee}_i \big (\UU(A_i) \wedge \SS_1(A_i) \wedge \cdots \wedge \SS_n(A_i) \big)
\]
for $a \in L$ and $\{A_i\}_{i \in I} \subseteq \mathcal{F}_L(U)$. 

\item Let $\UU$ be an $L$-fuzzy operator on $U$ and let each $\TT_k$, $1 \leq k \leq m$, be an $L$-fuzzy operator on $U$ such that
$\UU \geq \TT_k$. Then there exists a unique $L$-fuzzy relation $R$ on $U$ such that $\U A = \UU(A)$ for all $A \in \mathcal{F}_L(U)$
if and only if
\[
\UU(\overline{a} \wedge {\bigvee}_i A_i) = \overline{a} \wedge {\bigvee}_i (\UU(A_i) \vee \TT_1(A_i) \vee \cdots \vee \TT_m(A_i) )
\]
for $a \in L$ and $\{A_i\}_{i \in I} \subseteq \mathcal{F}_L(U)$. 

\end{enumerate}

These statements cover all the results obtained in \cite{Pang2019} and, moreover, they provide many characterization theorems of other types of $L$-fuzzy
rough approximation operators in the uniform way. We also correct a misunderstanding related to positive alliance relations appearing in the literature
(see Example~\ref{exa:alliance}).
In addition, as an application of our result, we also give an answer to the problem left in \cite{Pang2019}.

\section{Approximation operators in De~Morgan Heyting algebras} \label{Sec:Basics}

We begin by recalling some definitions from \cite{Castano11}.
A \emph{Heyting algebra} is an algebra 
\[ (H, \vee, \wedge, \Rightarrow, 0, 1)\] 
of type $(2,2,2,0,0)$ for which
$(H, \vee, \wedge, 0,1)$ is a bounded distributive lattice and $\Rightarrow$ is the operation of \emph{relative pseudocomplementation},
that is, for $a,b, c \in H$, $a \wedge c \leq b$ iff $c \leq a \Rightarrow b$.

A \emph{De~Morgan Heyting algebra} is an algebra 
\[ \mathbf{L} = (L, \vee, \wedge, \Rightarrow , ', 0, 1)
\] 
of type $(2, 2, 2, 1, 0, 0)$ such that 
$(L, \vee, \wedge, \Rightarrow, 0, 1)$ is a Heyting algebra and $'$ is an antitone involution, that is, 
$(x \wedge y)' = x' \vee y'$ and $x'' = x$ for all $x \in L$.

A complete lattice $L$ satisfies the \emph{join-infinite distributive law} if for any $S \subseteq L$ and $x \in L$,
\begin{equation*}\label{Eq:JID} \tag{JID}
x \wedge \big ( \bigvee S \big ) = \bigvee \{ x \wedge y \mid y \in S \}.
\end{equation*}
The dual condition is the \emph{meet-infinite distributive law}, (MID).  It is well known that a complete lattice defines a Heyting algebra
if and only if it satisfies (JID). Each De~Morgan Heyting algebra $\mathbf{L}$ defined on a complete lattice $L$ satisfies both (JID) and (MID),
because $'$ is an order-isomorphism between $(L,\leq)$ and its dual $(L,\geq)$. In this work, De~Morgan Heyting algebras defined on a complete lattice
are called \emph{complete De~Morgan Heyting algebras}.

Let $\mathbf{L}$ be a complete De~Morgan Heyting algebra and $U$ a universe.  An \emph{$L$-fuzzy set} on $U$ is a mapping $A \colon U \to L$.
We often drop the word `fuzzy' and speak about $L$-sets. The family of all $L$-sets on $U$ is denoted by $\mathcal{F}_L(U)$.

The set $\mathcal{F}_L(U)$ may be ordered \textit{pointwise} by setting for $A,B \in \mathcal{F}_L(U)$, $A \leq B$ if and only if  $A(x) \leq B(x)$ for all $x \in U$.
If $\mathbf{L}$ is a complete De~Morgan Heyting algebra, then $\mathcal{F}_L(U)$ forms a complete De~Morgan Heyting algebra such that for all
$\{A_i\}_{i \in I} \subseteq \mathcal{F}_L(U)$ and $x \in U$,
\[
\big ({\bigvee}_i A_i \big ) (x)  = {\bigvee}_i  A_i(x) \quad \text{ and } \quad
\big ({\bigwedge}_i A_i \big ) (x)  = {\bigwedge}_i  A_i(x)
\]
The operations $\Rightarrow$ and $'$ are defined for $A,B \in \mathcal{F}_L(U)$ and $x \in U$ by
\[ 
(A \Rightarrow B)(x) = A(x) \Rightarrow B(x) \qquad  \text{ and } \qquad A'(x) = A(x)' .
\] 
The map $\mathbf{0} \colon x \mapsto 0$ is the smallest and  $\mathbf{1} \colon x \mapsto 1$ is the greatest element of $\mathcal{F}_L(U)$, respectively.

An \emph{$L$-fuzzy relation $R$ on $U$} is a mapping $U \times U \to L$. We often use the term ``$L$-relation'' instead of ``$L$-fuzzy relation''.
The following definition of $L$-approximations can be found in \cite{Pang2019}.

\begin{definition}
Let $\mathbf{L}$ be a complete De~Morgan Heyting algebra, $R$ an $L$-relation on $U$ and $A \in \mathcal{F}_L(U)$. The
\emph{upper $L$-fuzzy approximation} and \emph{lower $L$-fuzzy approximation of $A$} are defined by  
\begin{align*}
\U(A)(x) = {\bigvee}_y \big ( R(x,y) \wedge A(y) \big ) \quad \text{ and } \quad \L(A)(x) = {\bigwedge}_y \big ( R(x,y)' \vee A(y) \big ),
\end{align*}  
respectively.
\end{definition}
\noindent%
If there is no danger of confusion, we may denote $\U(A)$ and $\L(A)$ simply by $\U A$ and $\L A$. In addition, 
$L$-fuzzy approximations are called simply $L$-approximations.

In \cite{Pang2019}, the following properties of $L$-approximations generalizing well-known properties of crisp rough approximations are proved.

\begin{proposition} \label{Prop:Basic} Let $U$ be a set, $L$ a complete De~Morgan Heyting algebra and $R$ an $L$-relation on $U$.
For $\{A_i\}_{i \in I} \subseteq \mathcal{F}_L(U)$, $A \in \mathcal{F}_L(U)$ and $x \in U$, the following assertions hold:
\begin{enumerate}[label = {\rm (\arabic*)}]
\item $\L\mathbf{0} = \mathbf{0}$ \  and  \ $\U \mathbf{1} = \mathbf{1}$;
\item $(\U A)' = \L(A')$  \ and  \ $(\L A)' = \U(A')$;
\item $\U({\bigvee}_i A_i) = {\bigvee}_i   \U(A_i)$ \ and  \quad $\L ({\bigwedge}_i A_i ) = {\bigwedge}_i  \L(A_i)$;
\item $\U( {\bigwedge}_i A_i) \leq {\bigwedge}_i  \U(A_i)$ \ and  \ $\L( {\bigvee}_i A_i ) \geq {\bigvee}_i   \L(A_i)$.
\end{enumerate}
\end{proposition}

\begin{remark} Radzikowska and Kerre \cite{radzikowska2004fuzzy} defined $L$-approximations in a complete residuated lattice $(L,\vee,\wedge,\odot,\to,0,1)$,
where $\to$ is the left residuum of $\odot$, by setting
\[ \U^*(A)(x) = {\bigvee}_y ( R(x,y) \odot A(y) ) \quad \text{ and } \quad \L^*(A)(x) = {\bigwedge}_y ( R(x,y) \to A(y) ). \]
Because a complete Heyting algebra $\mathbf{L}$ forms a residuated lattice such that $\odot = \wedge$ and $\to$ equals $\Rightarrow$,
$\U = \U^*$ and the lower approximation is defined as $\L^*(A)(x) = {\bigwedge}_y ( R(x,y) \Rightarrow A(y) )$.

Note that in a De~Morgan algebra $\mathbf{L}$, an implication $\to$ can be defined by setting for all $x,y \in L$,
\[ x \to y := x' \vee y. \]
The algebra $(L,\vee,\wedge,\to,0,1)$ does not generally form a residuated lattice. Let us consider the
four-element lattice $0 < a,b < 1$ in which $a$ and $b$ are incomparable, denoted $a \| b$. If the operation $'$ is defined by $0' = 1$, $a' = a$,
$b' = b$ and $1 ' = 0$, then $b \to 0 = b' \vee 0 = b' = b$. Now $a \wedge b = 0$, but $a \nleq b \to 0 = b$.

Note also that the implications $\Rightarrow$ and $\to$ are incomparable. For instance, in three-element chain $0 < u < 1$, the De~Morgan negation can
be defined by $0' = 1$, $u' = u$ and $1' = 0$. This means that $u \to u = u' \vee u = u \vee u = u$ and $u \Rightarrow u = 1$. Thus,
$u \to u \leq u \Rightarrow u$. On the other hand, $u \Rightarrow 0 = 0$ and $u \to 0 = u' \vee 0 = u$, that is, 
$u \Rightarrow 0 \leq u \to 0$.

The implication $\to$ can be extended pointwise to $\mathcal{F}_L(U)$. This means that the lower approximation can be expressed in terms of $\to$ by
\[  \L A (x) = {\bigwedge}_y ( R(x,y) \to A(y) ). \]
If one defines $\L$ in terms of $\to$ (instead of $\Rightarrow$), the operators $\L$ and $U$ are 
dual, as shown in Proposition~\ref{Prop:Basic}(2). 
\end{remark}

For $a\in L$, we define a `constant' $L$-set $\overline{a}$ by 
\[ \overline{a}(x) = a \text{ for all $x \in U$}.\]
This means that  $\overline{0} = \mathbf{0}$ and  $\overline{1} = \mathbf{1}$. For any $x \in U$, we define a map $I_x$ by
\[
I_x(y) = 
\begin{cases}
1 & \text{if $x =y$},\\
0 & \text{otherwise}.
\end{cases} 
\]
The idea is that the map $I_x$ corresponds to the singleton $\{x\}$. It is noted in \cite{Pang2019} that each $A \in \mathcal{F}_L(U)$ can be written in
two ways:
\begin{equation}\label{eq:widehat}
A = {\bigvee}_x \big ( \overline{A(x)} \wedge I_x \big )  = {\bigwedge}_x \big ( \overline{A(x)} \vee (I_x)' \big ). 
\end{equation}
Note that $(I_x)'$ corresponds the set-theoretical complement $U \setminus \{x\}$ of the singleton $\{x\}$, and
the latter equality is clear by Lemma~{3.17} of \cite{Pang2019}.
The following facts were also proved in \cite{Pang2019}.

\begin{lemma} \label{lem:bar_prop}
Let $L$ be a complete De~Morgan Heyting algebra and $R$ an $L$-relation on $U$. For all $a \in L$ and $A \in \mathcal{F}_L(U)$,
\begin{enumerate}[label = {\rm (\arabic*)}]
\item $\U(\overline{a}) \leq \overline{a} \leq \L(\overline{a})$;
\item $\U(\overline{a} \wedge A) = \overline{a} \wedge \U(A)$ \ and \ $\L(\overline{a} \vee A) = \overline{a} \vee \L(A)$;
\item $\U(\overline{a} \vee A) \leq \overline{a} \vee \U(A)$ \ and \ $\L(\overline{a} \wedge A) \geq \overline{a} \wedge \L(A)$.
\end{enumerate}
\end{lemma}

Let $\mathcal{S}$ be a finite combination of rough approximation operators $\L$ and $\U$.
Because the operators $\L$ and $\U$ are order-preserving, the operator $\mathcal{S}$ is order-preserving.
From this it follows that
\[ {\bigvee}_i \mathcal{S}(A_i) \leq \mathcal{S} \big ( {\bigvee}_i A_i \big ) \ \text{ and } \
   {\bigwedge}_i \mathcal{S}(A_i) \geq \mathcal{S} \big ( {\bigwedge}_i A_i \big ) . \]

\begin{lemma} \label{lem:step}
For all $a \in L$,
\[
\bar{a} \wedge \mathcal{S}(I_x) \leq \mathcal{S}(\bar{a} \wedge I_x).
\]
\end{lemma}

\begin{proof} We prove the claim by induction.
If $n = 1$, then the two cases $\bar{a} \wedge \U(I_x) \leq \U(\bar{a} \wedge I_x)$ and $\bar{a} \wedge \L(I_x) \leq \L(\bar{a} \wedge I_x)$
are clear by Lemma~\ref{lem:bar_prop}(2) by setting $A = I_x$.

Suppose that the claim holds for all combinations consisting of $n$ $\L$ and $\U$ operators. Let $\mathcal{S}$ be a combination $n + 1$
operators. Then $\mathcal{S} = \U \circ \mathcal{S}_1$ or $\mathcal{S} = \L \circ \mathcal{S}_2$, where $\mathcal{S}_1$ and $\mathcal{S}_2$ are combinations
of $\U$ and $\L$ of length $n$.

If $\mathcal{S} = \U \circ \mathcal{S}_1$, then
\[ \overline{a} \wedge \mathcal{S}(I_x) = \overline{a} \wedge \U(\mathcal{S}_1(I_x)) = \U(\overline{a} \wedge \mathcal{S}_1(I_x)) \leq
\U(\mathcal{S}_1(\bar{a} \wedge I_x)) = \mathcal{S}(\bar{a} \wedge I_x).
\]

If $\mathcal{S} = \L \circ \mathcal{S}_2$, then
\[ \overline{a} \wedge \mathcal{S}(I_x) = \overline{a} \wedge \L(\mathcal{S}_2(I_x)) \leq \L(\overline{a} \wedge \mathcal{S}_2(I_x)) \leq
\L(\mathcal{S}_2(\bar{a} \wedge I_x)) = \mathcal{S}(\bar{a} \wedge I_x). 
\qedhere
\]
\end{proof}

\begin{proposition} \label{prop:corr_help}
Let $\mathcal{S}$ be a finite combination of rough approximation operators $\L$ and $\U$.
If $\U(I_x) \leq \mathcal{S}(I_x)$ for all $x \in U$, then $\U(A) \leq \mathcal{S}(A)$ for all $A \in \mathcal{F}_L(A)$.
\end{proposition}

\begin{proof} Assume that $\U(I_x) \leq \mathcal{S}(I_x)$ for all $x \in U$. Then,
\begin{align*}
\U(A) & = \U \big ({\bigvee}_x (\overline{A(x)} \wedge I_x) \big)            & \text{(by \eqref{eq:widehat})}  \\
  & = {\bigvee}_x \U(\overline{A(x)} \wedge I_x)                             & \text{(by Prop.~\ref{Prop:Basic}(3))}  \\
  & = {\bigvee}_x \big (\overline{A(x)} \wedge \U(I_x) \big)              & \text{(by Lemma \ref{lem:bar_prop}(2))} \\
  & \leq {\bigvee}_x \big (\overline{A(x)} \wedge \mathcal{S}(I_x) \big)  & \text{(by assumption)} \\
  & \leq {\bigvee}_x \mathcal{S}(\overline{A(x)} \wedge I_x )                & \text{(by Lemma~\ref{lem:step})} \\
  & \leq \mathcal{S} \big ({\bigvee}_x \overline{A(x)} \wedge I_x \big )     & \text{($\mathcal{S}$ preserves $\leq$)} \\
  & \leq \mathcal{S}(A)                                                      &  \text{(by \eqref{eq:widehat})} 
\end{align*} 
\end{proof}


\section{Correspondence results}

In this section, we assume that $\mathbf{L}$ is a complete De~Morgan Heyting algebra, $U$ is a universe, and $R$ is an $L$-relation on $U$.

\subsection{Serial relations}

\begin{definition}
An $L$-relation $R$ is \emph{serial} if for all $x \in U$, $\bigvee_y R(x,y) = 1$. 
\end{definition}

Our next lemma gives a simple characterization of seriality.

\begin{lemma} 
An $L$-relation $R$ is serial if and only if $\U(\mathbf{1}) = \mathbf{1}$.
\end{lemma}

\begin{proof} Because
  \[ \U(\mathbf{1})(x) = {\bigvee}_y (R(x,y) \wedge \mathbf{1}(y)) = {\bigvee}_y R(x,y) , \]
the claim is obvious.  
\end{proof}

In \cite{radzikowska2004fuzzy}, it is proved that if $(L,\vee,\wedge,\odot,\to, 0, 1)$ is a complete residuated lattice, then
an $L$-relation $R$ is serial if and only if $\L^*(A) \leq U^*(A)$ for all $L$-sets $A$.
The following example shows that similar correspondence does not hold for De~Morgan Heyting algebras.

\begin{example} \label{Exa:serial1}
Let $L$ be the 4-element lattice $0 < a,b < 1$, where $a \| b$. If we define $0'=1$, $a'=a$, $b'=b$, and $1' = 0$, then $\mathbf{L}$ forms a complete
De~Morgan Heyting algebra. Let $U = \{x,y\}$. We define $R(x,x) = R(y,y) = a$ and $R(x,y) = R(y,x) = b$. It is now clear that $R$ is
serial, because $R(x,x) \vee R(x,y) = a \vee b = 1$ and $R(y,x) \vee R(y,y) = b \vee a = 1$.

Let $U = \{a,b\}$ and define an $L$-set $A \colon U \to L$ by $A(x) = b$ and $A(y) = a$. Now
\begin{gather*} \L(A) (x) = (R(x,x)' \vee A(x)) \wedge (R(x,y)' \vee A(y)) = (a' \vee b) \wedge (b' \vee a) \\
  = (a \vee b) \wedge (b \vee a) = 1 \wedge 1 = 1.
\end{gather*}
and
\begin{gather*}
  \U(A) (x) = (R(x,x) \wedge A(x)) \vee (R(x,y) \wedge A(y)) = (a \wedge b) \vee (b \wedge a) = 0 \vee 0 = 0.
\end{gather*}
Therefore, $\L(A) \nleq \U(A)$.

Note also that in $L$, $a \Rightarrow b = b$ and $b \Rightarrow a = a$. This gives that
\[ \L^*(A) = (R(x,x) \Rightarrow A(x)) \wedge (R(x,y) \Rightarrow A(y)) = (a \Rightarrow b) \wedge (b \Rightarrow a) = a \wedge b = 0. \]
This means that $\L^*(A) \leq \U^*(A) = \U(A)$. 
\end{example}

If $R$ is such that for all $x \in U$, there exists $y$ such that $R(x,y) = 1$, then the relation is serial.
The converse is not true. For instance, the $L$-relation $R$ of Example~\ref{Exa:serial1} is serial, but $R(x,y) \neq 1$ for all $x,y \in U$.

\begin{lemma} \label{lem:serial_implication}
If for all $x \in U$, there is $y \in U$ such that $R(x,y) = 1$, then $\L A \leq \U A$ for 
any $A \in \mathcal{F}_L(U)$. 
\end{lemma}

\begin{proof} Let $x \in U$. There exists $z \in U$ such that $R(x,z) = 1$. Now
  \[ \L(A)(x) = {\bigwedge}_y \big ( R(x,y)' \vee A(y) \big ) \leq R(x,z)' \vee A(z) = 1' \vee A(z) = 0 \vee A(z) = A(z) .\]
On the other hand,
\[ \U(A)(x) = {\bigvee}_y \big ( R(x,y) \wedge A(y) \big ) \geq R(x,z) \wedge A(z) = 1 \wedge A(z) = A(z) .\]
We have that $\L(A)(x) \leq A(z) \leq \U(A)(x)$ and the claim is proved.
\end{proof}

The converse implication does not hold, as shown in our following example.

\begin{example} \label{exa:serial_not_one}
Let $L$ be the 4-element lattice with $0 < a,b < 1$ and $a \| b$. If we define $0'=1$, $a'=b$, $b'=a$, and $1' = 0$, then $\mathbf{L}$ forms a complete
De~Morgan Heyting algebra, which is actually a Boolean algebra. This means that the implication $a \to b = a' \vee b$ coincides with
$a \Rightarrow b$. Hence, for any $L$-relation $R$, $\L^*(A) = \L(A)$ for all $A \in \mathcal{F}_L(U)$. 

Because Boolean algebras form residuated lattices, we have that $\L(A) = \L^*(A) \leq \U^*(A) = \U(A)$ for all $A \in \mathcal{F}_L(U)$. This is also
true for the $L$-relation $R$ defined in Example~\ref{Exa:serial1} which is such that $R(x,y) \neq 1$ for all $x,y \in U$.
\end{example}


For any $x,y \in U$, $\U(I_y)(x) = \bigvee_z (R(x,z) \wedge I_y(z))$. Because $I_y(z) = 1$ iff $z = y$, we obtain the following
equality which will be used frequently in this work.
\begin{equation}
  (\forall x,y \in  U) \, \U(I_y)(x) = R(x,y) .
\end{equation}
In addition, $\L(I_y)(x) = {\bigwedge}_z  ( R(x,z)' \vee I_y(x) )$. If $z \neq y$, $R(x,z)' \vee I_y(x) = R(x,y)'$ and
if $z = y$, then $R(x,z)' \vee I_y(x) = 1$. Thus,
\[
\L(I_y)(x) = \big ( {\bigwedge}_{z \neq y} R(x,z)' \big ) \wedge 1 = 
\big ( {\bigwedge}_{z \neq y} R(x,z)' \big ) =
\big ( {\bigvee}_{z \neq y} R(x,z) \big )'. 
\]

As the above results show, it seems difficult to characterize the condition
\begin{equation} \label{eq:serial}
(\forall A \in \mathcal{F}_L(U)) \, \L A \leq \U A,
\end{equation}
in terms of an $L$-relation. We end our study of seriality by considering the condition
\begin{equation} \label{eq:seriality2} 
 (\forall x,y \in U) \, R(x,y)' \leq {\bigvee}_{z \neq y} R(x,z).
\end{equation}
  
Obviously, \eqref{eq:seriality2} generalizes the notion of seriality of a binary relation $\rho$ on $U$:
if $x \, \rho \, y$ does not hold, then there exits $z \neq y$ such that $x \, R \, z$. Condition \eqref{eq:seriality2} is
equivalent to
\[ (\forall x,y \in U) \, \Big ( {\bigvee}_{z \neq y} R(x,z) \Big )' \leq R(x,y), \]
which can be easily seen equivalent to
\[ (\forall x \in U) \, \L(I_x) \leq \U(I_x). \]
Unfortunately, \eqref{eq:seriality2} is not equivalent to \eqref{eq:serial}, as can be seen in the following example.

\begin{example}
Let $L$ be the four-element lattice $0 < a,b < 1$ such that $a \| b$. Define $0' = 1$, $a' = a$, $b' = b$, and $1' = 0$. We set $U = \{x,y\}$
and define the $L$-relation $R$ on $U$ by $R(x,x) = R(x,y) = a$ and $R(y,x) = R(y,y) = b$.

Let us consider the $L$-set $\mathbf{0}$ defined by $x \mapsto 0$. We have that
\begin{gather*}
  \L (\mathbf{0})(x) = {\bigwedge}_w (R(x,w)' \vee \mathbf{0}(w)) = 
  (R(x,x)' \vee \mathbf{0}(x)) \wedge (R(x,y)' \vee \mathbf{0}(y)) = \\
  R(x,x)' \wedge R(x,y)' = a' \vee a' = a \vee a = a
\end{gather*}
and
\begin{gather*}
 \U(\mathbf{0})(x) = {\bigvee}_w (R(x,w) \wedge \mathbf{0}(w)) = 0.  
\end{gather*}
This means that $\L (\mathbf{0}) \nleq  \U(\mathbf{0})$. On the other hand,
\begin{align*}
  \U(I_x)(x) &= R(x,x) = a, \\
  \U(I_x)(y) &= R(y,x) = b, \\
  \U(I_y)(x) &= R(x,y) = a, \\
  \U(I_y)(y) &= R(y,y) = b
\end{align*}
and
\begin{align*}
  \L(I_x)(x) &= {\bigwedge}_z ( R(x,z)' \vee I_x(z)) = (R(x,x)' \vee I_x(x)) \wedge (R(x,y)' \vee I_x(y)) \\
             &= 1 \wedge R(x,y)' = R(x,y)' = a' = a, \\
  \L(I_x)(y) &= {\bigwedge}_z ( R(y,z)' \vee I_x(z)) = (R(y,x)' \vee I_x(x)) \wedge (R(y,y)' \vee I_x(y)) \\
             &= 1 \wedge R(y,y)' = R(y,y)' = b' = b, \\
  \L(I_y)(x) &= {\bigwedge}_z (R(x,z)' \vee I_y(z)) = (R(x,x)' \vee I_y(x)) \wedge (R(x,y)' \vee I_y(y)) \\
             &= R(x,x)' \wedge 1 = R(x,x)' = a' = a, \\
  \L(I_y)(y) &= {\bigwedge}_z (R(y,z)' \vee I_y(z)) = (R(y,x)' \vee I_y(x)) \wedge (R(y,y)' \vee I_y(y)) \\
             &= R(y,x)' \wedge 1 = R(y,x)' = b' = b. 
\end{align*}
We have shown that $\L(I_x) \leq \U(I_x)$ for all $x \in U$. This means that \eqref{eq:serial} is not equivalent to
\eqref{eq:seriality2}.
\end{example}


\subsection{Reflexive and symmetric relations}
\begin{definition}
An $L$-relation $R$ is \emph{reflexive}, if for all $x \in U$, $R(x,x)=1$.
\end{definition}

The following lemma gives a characterization of reflexivity.

\begin{lemma} \label{lem:refl}
An $L$-relation $R$ is reflexive if and only if for all $x \in U$, $I_x \leq \U({I_x})$.
\end{lemma}

\begin{proof} 
If $R$ is reflexive, then for all $x \in U$, $\U({I_x})(x) = R(x,x) = 1 = I_x(x)$.
If $R$ is not reflexive, then there is $x \in U$ such that $R(x,x) \neq 1$.
Now $I_x(x) = 1$ and $\U(I_x)(x) = R(x,x) \neq 1$, giving $I_x \nleq \U({I_x})$.
\end{proof}

\begin{proposition} \label{prop:reflexive}
The following are equivalent:
\begin{enumerate}[label = {\rm (\arabic*)}]
\item $R$ is reflexive;
\item $(\forall A \in \mathcal{F}_L(U)) \, A \leq \U A$;
\item $(\forall A \in \mathcal{F}_L(U)) \, \L A \leq A$.
\end{enumerate}
\end{proposition}

\begin{proof}
Suppose that $R$ is reflexive. Then by Lemma~\ref{lem:refl}, $I_x \leq \U({I_x})$ for all $x \in U$.
We have that for $A \in \mathcal{F}_L(U)$,
\begin{align*} \textstyle
  A &= {\bigvee}_x \Big ( \overline{A(x)} \wedge I_x \Big ) \leq {\bigvee}_x \Big ( \overline{A(x)} \wedge \U({I_x}) \Big )
  = {\bigvee}_x \U \Big ( \overline{A(x)} \wedge I_x \Big ) \\
  & =  \U \Big ( {\bigvee}_x ( \overline{A(x)} \wedge {I_x}) \Big ) = 
  \U A.
\end{align*}
Thus, (1) implies (2). Suppose that (2) holds. Then for all $x \in U$,
$I_x \leq \U({I_x})$, which by proposition~\ref{prop:corr_help} yields that $R$ is reflexive. Thus, also (2) implies (1).

We prove that (2) and (3) are equivalent. Assume (2) holds. Then for any $A \in \mathcal{F}_L(U)$,  $A' \leq \U(A') = (\L A)'$, which implies $\L A \leq A$.
Similarly, if (3) holds, then $\L(A') \leq A'$ implies $A \leq (\L(A'))' = \U(A)$.
\end{proof}



A binary relation $\rho$ is symmetric whenever for all $x,y \in U$, 
$x \, \rho \, y$ implies $y \, \rho \, x$. For a fuzzy relation $R$,
this is typically generalized to a condition $R(x,y) \leq R(y,x)$, which actually means that $R(x,y) = R(y,x)$.
Here we introduce another definition based on the fact that for a binary relation $\rho$, the symmetry condition can be expressed in form
$(x \, \rho^c \, y) \vee (y \rho \, x)$, where $\rho^c$ denotes the complement of the relation $\rho$, that is,
$\rho^c = (U \times U) \setminus \rho$. We say that an $L$-relation $R$ \emph{symmetric} if
\begin{equation} \tag{symm} \label{eq:symm}
 (\forall x,y \in U) \, R(x,y)' \vee R(y,x) = 1.
\end{equation}
  
\begin{example} Let us consider the three-element chain $\mathbf{3}$ 
in which $0 < u < 1$ and let $U = \{x,y\}$. Let us define a
$\mathbf{3}$-fuzzy relation $R$ on $U$ in such a way that $R(x,y) = R(y,x) = u$, $R(x,x) = 0$ and $R(y,y) = 1$. Then, $R$
is symmetric in traditional ``fuzzy sense''. But now $R(x,y)' \vee R(y,x) = u \vee u = u$.

On the other hand, let $L$ be the 4-element lattice in which $0 < a,b, < 1$. The operation $'$ is defined as $0'=1$ and $a' = b$. If $R$ is defined by  $R(x,x) = 0$, $R(x,y) = a$, $R(y,x) = b$ and $R(y,y) = 1$, then clearly \eqref{eq:symm} holds for all pairs of elements of $U$.
But now $R(x,y) \neq R(y,x)$, so $R$ is not symmetric in traditional ``fuzzy sense''.

These observations mean that \eqref{eq:symm} is independent from the usual definition of symmetry of fuzzy relations, and
it can be viewed as an alternative generalization of ``crisp symmetry''.
\end{example}

\begin{lemma} \label{lem:symm}
An $L$-relation $R$ is symmetric if and only if for all $x \in U$, 
$I_x \leq \L \U (I_x)$.
\end{lemma}

\begin{proof} For all $x \in U$,
  \[ \L \U (I_x)(x) = {\bigwedge}_y ( R(x,y)' \vee \U (I_x)(y) ) = {\bigwedge}_y (R(x,y)' \vee R(y,x)).\]
If $R$ is symmetric, then $\L \U (I_x)(x) = 1$, which means $\L \U (I_x)(x) \geq I_x (x)$. On the other hand,
if  $ \L \U (I_x) \geq I_x$, then $ \L \U (I_x)(x) = 1$ and 
$\bigwedge_y (R(x,y)' \vee R(y,x)) = 1$.
This gives that for each $y$, $R(x,y)' \vee R(y,x) = 1$.
\end{proof}

We can also write the following characterization of symmetric $L$-relations.

\begin{proposition} Let $\mathbf{L}$ complete De~Morgan Heyting algebra and $R$ an $L$-relation on $U$. Then the following are equivalent:
\begin{enumerate}[label = {\rm (\arabic*)}]
\item $R$ is symmetric;
\item $(\forall A \in \mathcal{F}_L(U)) \, A \leq \L \U A$;
\item $(\forall A \in \mathcal{F}_L(U)) \, \U \L A \leq A$.
\end{enumerate}
\end{proposition}

\begin{proof}
Suppose that $R$ is symmetric. Then by Lemma~\ref{lem:symm}, 
$I_x \leq \L \U (I_x)(x)$ for each $x \in U$. We have that
\begin{align*} 
A & = {\bigvee}_x \big (\overline{A(x)} \wedge I_x \big) \leq {\bigvee}_x \big (\overline{A(x)} \wedge  \L \U (I_x) \big )
\leq {\bigvee}_x \L \big (\overline{A(x)} \wedge \U ({I_x}) \big ) \\
& = {\bigvee}_x \L \U \big ( \overline{A(x)} \wedge {I_x} \big )
\leq \L \U \Big ({\bigvee}_x (\overline{A(x)} \wedge {I_x}) \Big ) = \L \U A.
\end{align*}
This means that (1) implies (2). If we set $A = I_x$ in (2), we get that $I_x \leq \L \U (I_x)$ for $x \in U$.
Hence, $R$ is symmetric. The equivalence of (2) and (3) follows easily by the duality of approximation operations.
\end{proof}

\subsection{Transitive and mediate relations}

An $L$-relation is \emph{transitive}, if $R(x, z) \wedge R(z, y) \leq R(x, y)$ for all $x, y, z \in U$. Obviously, this
is equivalent to that $\bigvee_z (R(x, z) \wedge R(z, y)) \leq R(x, y)$ for all $x, y \in U$.

\begin{lemma} \label{lem:trans}
An $L$-relation $R$ is transitive if and only if for all $x \in U$, $\U \U (I_x) \leq \U (I_x)$.
\end{lemma}

\begin{proof} For all $x \in U$,
\[ \U \U (I_x)(y) = {\bigvee}_z \big (R(y,z) \wedge \U (I_x)(z) \big ) = {\bigvee}_z (R(y,z) \wedge R(z,x)) ,\] 
which is below $R(y,x) = \U(I_x)(y)$ by the transitivity of $R$. 

On the other hand, for all $x,y \in U$,
\[ \U \U (I_y)(x) = {\bigvee}_z ( R(x,z) \wedge R(z,y) ) \quad \text{and} \quad \U (I_y)(x) = R(x,y).\]
If $\U \U (I_y) \leq \U (I_y)$, then the transitivity condition holds for $x$ and $y$, which completes the proof. 
\end{proof}

The following proposition characterizes transitive $L$-relations in terms of approximation operations.

\begin{proposition} \label{prop:transitive}
Let $\mathbf{L}$ be a complete De~Morgan Heyting algebra and $R$ an $L$-relation on $U$. Then the following are equivalent:
\begin{enumerate}[label = {\rm (\arabic*)}]
\item $R$ is transitive;
\item $(\forall A \in \mathcal{F}_L(U)) \, \U \U A \leq \U A$;
\item $(\forall A \in \mathcal{F}_L(U)) \, \L A \leq \L \L A$.
\end{enumerate}
\end{proposition}

\begin{proof} Assume that $R$ is transitive. By Lemma~\ref{lem:trans}, $\U \U (I_x) \leq \U (I_x)$ for $x \in U$. We get
\begin{align*}
\U \U A &=  \U \U \Big ( \Big ({\bigvee}_x (\overline{A(x)} \wedge I_x) \Big) \Big ) = \U \Big ( {\bigvee}_x \U ( \overline{A(x)} \wedge I_x) \Big )
= \U \Big ( {\bigvee}_x ( \overline{A(x)} \wedge \U (I_x) \Big ) \\
& =  \Big ( {\bigvee}_x ( \overline{A(x)} \wedge \U \U (I_x) \Big ) 
\leq {\bigvee}_x (\overline{A(x)} \wedge \U (I_x)) = 
\U \Big ({\bigvee}_x (\overline{A(x)} \wedge I_x) \Big ) \\
& = \U A.
\end{align*}
Thus, (2) holds. Conversely, if (2) holds, then $\U \U (I_x) \leq \U (I_x)$ for all $x \in U$ and $R$ is transitive.
The equivalence of (2) and (3) is clear.
\end{proof}
 
In \cite{Pang2019}, an $L$-relation $R$ is called \emph{mediate} if $x,y  \in U$,  $R(x,y) \le \bigvee_z ( R(x,z) \wedge R(z,y) )$.

\begin{lemma} \label{lem:mediate}
An $L$-relation $R$ is mediate if and only if for all $x \in U$, 
$\U \U (I_x) \geq \U (I_x)$.
\end{lemma}

\begin{proof} For all $x,y \in U$,
\[
\U (I_y(x)) \le \U \U (I_y)(x) \iff R(x,y) \le {\bigvee}_z (R(x,z) \wedge R(z,y)), 
\]
which completes the proof.
\end{proof}

We may now write the following proposition.

\begin{proposition} \label{prop:mediate}
Let $\mathbf{L}$ be a complete De~Morgan Heyting algebra and $R$ an $L$-relation on $U$. Then the following are equivalent:
\begin{enumerate}[label = {\rm (\arabic*)}]
\item $R$ is mediate;
\item $(\forall A \in \mathcal{F}_L(U)) \, \U A \leq \U \U A$;
\item $(\forall A \in \mathcal{F}_L(U)) \, \L \L A \leq \L A$.
\end{enumerate}
\end{proposition}

\begin{proof} Assume that $R$ mediate. By Lemma~\ref{lem:mediate}, $\U (I_x) \leq \mathcal{S} (I_x) = \U \U (I_x)$ for all $x \in U$, 
where $\mathcal{S}$ denotes the  combination $\U \U$. Therefore,
$\U A \leq \mathcal{S}(A) = \U \U A$ for all $A \in \mathcal{F}_L$
follows directly from Proposition~\ref{prop:corr_help}. Therefore, (2) implies (1). By applying $A = I_x$, it immediate
that (2) implies (1). 
 
The equivalence of (2) and (3) can be proved as in our earlier proofs. 
\end{proof}

\subsection{Euclidean and adjoint relations}

A binary relation $\rho$ is Euclidean if $x \, \rho \, y$ and $x \, \rho \, z$ imply $y \, \rho \, z$. Obviously,
$x \, \rho \, y$ and $x \, \rho \, z$ imply also $z \, \rho \, y$. As noted in \cite{Pang2019}, this is equivalent to
that $x \, \rho \, z$ and $z \, \rho^c \, y$ imply $x \, \rho^c \, y$. An $L$-relation $R$ on $U$ is called
\emph{Euclidean} if for all $x,y,z \in U$, $R(x,y)' \geq  R(x,z) \wedge R(z,y)'$. This is equivalent to the condition
\begin{equation} \label{eq:euc} \tag{euc}
(\forall x,y \in L) \, R(x,y)' \geq {\bigvee}_z (R(x,z) \wedge R(z,y)' ) .
\end{equation}

\begin{lemma} \label{lem:Euc}
An $L$-relation $R$ is Euclidean if and only if for all $x \in U$, $\U (I_x) \leq \L \U (I_x)$.
\end{lemma}

\begin{proof} Let $x,y \in U$. As we have noted, $\U(I_y)(x) = R(x,y)$ and
$\L\U(I_y)(x) = \bigwedge_z (R(x,z)' \vee \U(I_y)(z)) = \bigwedge_z (R(x,z)' \vee R(z,y))$. 
If $R$ is Euclidean, then $R(x,y)' \geq \bigvee_z (R(x,z) \wedge R(z,y)' )$, which implies
\begin{align*}
R(x,y) & = R(x,y)'' \leq \big ( {\bigvee}_z (R(x,z) \wedge R(z,y)' ) \big)' =
{\bigwedge}_z (R(x,z) \wedge R(z,y))' \\
& = {\bigwedge}_z (R(x,z)' \vee R(z,y)'') = {\bigwedge}_z (R(x,z)' \vee R(z,y)),
\end{align*}
This means $\U(I_y) \leq \L\U(I_y)$.

On the other hand, if $\U(I_y) \leq \L\U(I_y)$, then $\U(I_y)(x) \leq \L \U (I_y)(x)$ and $R(x,y) \leq \bigwedge_z (R(x,z)' \vee R(z,y))$.
Applying De Morgan operation $'$ for the both sides of the relation, we
obtain that \eqref{eq:euc} holds. 
\end{proof}

The next proposition characterizes Euclidean $L$-relations in terms of approximations.

\begin{proposition} \label{prop:Eucl}
Let $\mathbf{L}$ be a complete De~Morgan Heyting algebra and $R$ an $L$-relation on $U$. Then the following are equivalent:
\begin{enumerate}[label = {\rm (\arabic*)}]
\item $R$ is Euclidean;
\item $(\forall A \in \mathcal{F}_L(U)) \, \U A \leq \L\U A$;
\item $(\forall A \in \mathcal{F}_L(U)) \, \U \L A \leq \L A$.
\end{enumerate}
\end{proposition}


\begin{proof} If $R$ is Euclidean, then $\U (I_x) \leq \L \U (I_x)$ for all $x \in U$. By Proposition~\ref{prop:corr_help},
$\U A \leq \L \U A$ for all $A \in \mathcal{F}_L$. Thus, (2) implies (1). By applying $A = I_x$, we see that (2) implies (1). 
The equivalence of (2) and (3) is obvious.
\end{proof}

In \cite{Pang2019}, an $L$-relation $R$ on $U$ is called \emph{adjoint} if for all $x,y \in U$,
\[
R(x,y)' \ge \bigwedge_{z} \bigvee_{w\neq y} (R(x,z)' \vee R(z,w)).
\]

\begin{lemma} \label{lem:adjoint}
An $L$-relation $R$ is adjoint if and only if for all $x \in U$, $\U (I_x) \leq \U \L (I_x)$.
\end{lemma}

\begin{proof}
For all $x,y \in U$, we have the following equality:
\begin{align*}
  {\bigwedge}_{z} {\bigvee}_{w\neq y} (R(x,z)' \vee R(z,w)) 
  & = {\bigwedge}_{z} \big ( R(x,z)' \vee {\bigvee}_{w\neq y} R(z,w) \big ) \\
  & = {\bigwedge}_{z} \big ( R(x,z)' \vee {\bigvee}_{w} (R(z,w) \wedge {I_y}'(w)  \big) \\
  & = {\bigwedge}_{z} \big ( R(x,z)' \vee \U({I_y}')(z)  \big )\\
  & = \L \U({I_y}')(x) \\
  & = (\U \L ({I_y})(x))'.  
\end{align*}
If $R$ is adjoint, then
\[ (\U (I_y)(x))' = R(x,y)' \geq ( \U \L({I_y})(x)) ',\]
which is equivalent to $\U (I_y)(x) \leq \U \L ({I_y})(x)$. Hence,  $\U(I_y) \leq \U \L ({I_y})$ for all $y \in U$.
Conversely, if $\U (I_y) \leq  \U \L (({I_y})$ for all $y \in U$, 
then by the above equality, $R$ is adjoint.
\end{proof}

\begin{proposition} \label{prop:adjoint}
Let $\mathbf{L}$ be a complete De~Morgan Heyting algebra and $R$ an $L$-relation on $U$. Then the following are equivalent:
\begin{enumerate}[label = {\rm (\arabic*)}]
\item $R$ is adjoint;
\item $(\forall A \in \mathcal{F}_L(U)) \, \U A \leq \U \L A$;
\item $(\forall A \in \mathcal{F}_L(U)) \, \L \U A \leq \L A$.
\end{enumerate}
\end{proposition}

\begin{proof} If $R$ is adjoint, then $\U (I_x) \leq  \U \L (I_x)$ for all $x \in U$. Proposition~\ref{prop:corr_help} gives that
$\U A \leq \U \L A$ for all $A \in \mathcal{F}_L$. Thus, (1) implies (2). By applying $A = I_x$, we see that (2) implies (1). 
 
The equivalence of (2) and (3) is clear by the duality.
\end{proof}

A binary relation $\rho$ is functional when each $x \in U$ is $\rho$-related to at most one element. 
This means that $x \, \rho \, y$ implies that for all $z \in (U \setminus \{y\})$,
$x \, \rho^c \, y$. Thus, an $L$-relation $R$ on $U$ is \emph{functional} if $R(x,y) \leq  R(x,z)'$  for all $z \neq y$. This is equivalent to
\[ R(x,y) \leq  \bigwedge_{z \neq y} R(x,z)' .\]

\begin{lemma} 
An $L$-relation $R$ is functional if and only if for all $x \in U$, $\U (I_x) \leq \L (I_x)$.
\end{lemma}


\begin{proof} For all $x,y \in U$, $\U (1_y)(x) = R(x,y)$ and
\begin{align*}
  \L (1_y)(x) &= {\bigwedge}_z (R(x,z)' \vee (1_y)(z)) = 
  \Big ( {\bigwedge}_{z \neq y} (R(x,z)' \vee 0) \Big ) \wedge (R(x,y)' \vee 1) \\
  &= \Big ( {\bigwedge}_{z \neq y} R(x,z)' \Big ) \wedge 1 = {\bigwedge}_{z \neq y} R(x,z)' .
\end{align*}
Because for all $y \in U$, $\U (I_y) \leq \L (I_y)$ is equivalent to that $\U (I_y)(x) \leq \L (I_y)(x)$ for all $x,y \in U$, the claim is proved. 
\end{proof}

\begin{proposition} \label{prop:func}
Let $\mathbf{L}$ be a complete De~Morgan Heyting algebra and $R$ an $L$-relation on $U$. Then the following are equivalent:
\begin{enumerate}[label = {\rm (\arabic*)}]
\item $R$ is functional;
\item $(\forall A \in \mathcal{F}_L(U)) \, \U A \leq \L A$.
\end{enumerate}
\end{proposition}

\begin{proof}
If $R$ is functional, then $\U (I_x) \leq \L (I_x)$ for all $x \in U$. By Proposition~\ref{prop:corr_help}, we obtain
$\U A \leq \L A$ for all $A \in \mathcal{F}_L$.
Conversely, if $\U A \leq \L A$ for any $A \in \mathcal{F}_L(U)$, then $\U (I_x) \leq \L (I_x)$ for all $x \in U$. 
\end{proof}

`Positive alliance' relations were defined in \cite{Zhu2007} by stating that a binary relation $\rho$ is a
positive alliance if for any elements $x,y \in U$ such that $x \, \rho^c \, y$, there is $z \in U$ satisfying
$x \, \rho \, z$, but $z \, \rho^c \, y$.

It is clear that each reflexive relation is positive alliance, because if $a \, \rho^c \, b$, then
$a \, \rho \, a$ and $a \, \rho^c \, b$ hold trivially.

The following facts can be found in \cite{Ma2015}, but for the sake of completeness we give a proof.

\begin{lemma} Let $\rho$ be a serial and transitive binary relation on $U$. Then the following conditions hold:
\begin{enumerate}[label = {\rm (\arabic*)}]
\item $\rho$ is a positive alliance;
\item for all $X \subseteq U$, $(X^\rho)_\rho \subseteq X^\rho$.
\end{enumerate}
\end{lemma}

\begin{proof} (1) Suppose $a \, \rho^c \, b$. Because $\rho$ is serial, there is $c$ such that $a \, \rho \, c$.
Now $c \, \rho \, b$ is not possible, because that would imply $a \, \rho \, b$, contradicting  $a \, \rho^c \, b$.
Thus, $c \, \rho^c \, b$.

(2) If $\rho$ is a serial binary relation on $U$, then $X_\rho \subseteq X^\rho$ for all $X \subseteq U$. In particular,
$(X^\rho)_\rho \subseteq (X^\rho)^\rho$. If $\rho$ is also transitive, then $(X^\rho)^\rho \subseteq X^\rho$.
Thus, if $\rho$ is serial and transitive,  $(X^\rho)_\rho \subseteq X^\rho$.
\end{proof}

On the other hand, it is clear that if $\rho$ is a positive alliance, then it is serial. However, there are
positive alliance relations that are not transitive.

\begin{example}
Let $U = \{a,b,c\}$ and let $\rho = \{ (a,b), (b,b), (c,a), (c,c) \}$. Then $\rho$ is serial, but not transitive
because $c \, \rho \, a$ and $a \, \rho \, b$, but $c \, \rho^c \, b$. Now we have that:
\begin{itemize}
\item $a \, \rho^c a$, and there is $b$ such $a \, \rho \, b$ and $b \, \rho^c \, a$;
\item $a \, \rho^c c$, and there is $b$ such $a \, \rho \, b$ and $b \, \rho^c \, c$;
\item $b \, \rho^c a$, and there is $b$ such $b \, \rho \, b$ and $b \, \rho^c \, a$;
\item $b \, \rho^c c$, and there is $b$ such $b \, \rho \, b$ and $b \, \rho^c \, c$;
\item $c \, \rho^c b$, and there is $c$ such $c \, \rho \, c$ and $c \, \rho^c \, b$.  
\end{itemize}
Hence, $\rho$ is a positive alliance.
\end{example}

It is clear that for all $x,y \in U$, $y \in \{x\}^\rho \iff y \, \rho \, x$. Therefore,
\[ y \in (\{x\}^\rho)^c \iff y \, \rho^c \, x \qquad \text{and} \qquad y \in  ( (\{x\}^\rho )^c )^\rho \iff
(\exists z) y \, \rho \, z  \ \& \ z \, \rho^c \, x .\]
As proved in \cite{Zhu2007}, $\rho$ is a positive alliance if and only if $(\{x\}^\rho)^c \subseteq ( (\{x\}^\rho  )^c  )^\rho$
for all $x \in U$. Note that the latter condition is equivalent to that $(\{x\}^\rho)_\rho \subseteq \{x\}^\rho$ for all $x \in U$.

\begin{example} \label{exa:alliance}
It is commonly accepted (see \cite{Ma2015,Zhu2007}, for instance) that for any binary relation $\rho$, the following are equivalent:
\begin{enumerate}[label = {\rm (\arabic*)}]
\item $\rho$ is a positive alliance;
\item for all $X \subseteq U$, $(X^\rho)_\rho \subseteq X^\rho$.
\end{enumerate}
It is now clear that (2) implies (1) by the fact that $\rho$ is a positive alliance if and only if  $(\{x\}^\rho)_\rho \subseteq \{x\}^\rho$ for all $x \in U$.
Next we give a counter example showing that (1) does not imply (2). 

Let the relation $\rho$ be given in Figure~\ref{Fig:relation}.
% Figure environment removed

\noindent%
The relation $\rho$ is a positive alliance, because:
\begin{itemize}
\item $1 \, \rho^c 1$, and there is $4$ such $1 \, \rho \, 4$ and $4 \, \rho^c \, 1$;
\item $1 \, \rho^c 2$, and there is $4$ such $1 \, \rho \, 4$ and $4 \, \rho^c \, 2$;
\item $1 \, \rho^c 3$, and there is $4$ such $1 \, \rho \, 4$ and $4 \, \rho^c \, 3$;
\item $2 \, \rho^c 1$, and there is $2$ such $2 \, \rho \, 2$ and $2 \, \rho^c \, 1$;
\item $2 \, \rho^c 4$, and there is $2$ such $2 \, \rho \, 2$ and $2 \, \rho^c \, 4$;
\item $3 \, \rho^c 3$, and there is $1$ such $3 \, \rho \, 1$ and $1 \, \rho^c \, 3$;
\item $3 \, \rho^c 4$, and there is $2$ such $3 \, \rho \, 2$ and $2 \, \rho^c \, 4$;
\item $4 \, \rho^c 1$, and there is $4$ such $4 \, \rho \, 4$ and $4 \, \rho^c \, 1$;
\item $4 \, \rho^c 2$, and there is $4$ such $4 \, \rho \, 4$ and $4 \, \rho^c \, 2$;
\item $4 \, \rho^c 3$, and there is $4$ such $4 \, \rho \, 4$ and $4 \, \rho^c \, 3$.  
\end{itemize}
We have that
\[ \rho(1) = \{4\}, \quad \rho(2) = \{2,3\}, \quad \rho(3) = \{1,2\}, \quad \rho(4) = \{4\}. \]
Let us consider the set $\{3,4\}$. Now $\{3,4\}^\rho = \{1,2,4\}$ and $(\{3,4\}^\rho)_\rho = \{1,2,4\}_\rho = \{1,3,4\}$. This means that
\[ (\{3,4\}^\rho)_\rho \nsubseteq \{3,4\}^\rho. \]
This obviously implies that there is a positive alliance $\rho$ such that
\[ (\{3,4\}^\rho)^c \nsubseteq ( (\{3,4\}^\rho)_\rho )^c =  ((\{3,4\}^\rho)^c)^\rho ,\]
contradicting Theorem~2(5) in \cite{Ma2015}.
\end{example}

For $L$-relations, we can present the following generalization. An $L$-relation $R$ is said to be a \emph{positive alliance} if
\[ R(x,y)' \leq {\bigvee}_z (R(x,z) \wedge R(z,y)' ). \]
We can now write the following lemma characterizing positive alliance relations in terms of the approximations of the identity functions.

\begin{lemma} \label{lem:alliance}
An $L$-relation $R$ is a positive alliance if and only if for all $x \in U$, $\U (I_x) \geq \L \U (I_x)$.
\end{lemma}

\begin{proof}
Let $x,y \in I$. We have
\[ \L \U (I_y)(x)  = {\bigwedge}_z (R(x,z)' \vee \U (I_y(z)) ) = {\bigwedge}_z \big (R(x,z)' \vee R(z,y) \big ) .\]
Thus,
\[ (\L \U (I_y)(x) )'  = \Big ( {\bigwedge}_z (R(x,z)' \vee R(z,y)) \Big )' = {\bigvee}_z \big (R(x,z) \wedge R(z,y)' \big ) .\]
Now, $\U (I_x) \geq \L \U (I_x)$ for all $x \in U$ if and only if $\U (I_y)(x) \geq \L \U (I_y)(x)$ for all $x,y \in U$ if and only if
$(\U (I_y)(x))' \leq (\L \U ((I_y)(x))'$ for all $x,y \in U$, and the equivalence follows from this. 
\end{proof}

\section{From $L$-approximations to $L$-relations}

In the previous section, we defined the $L$-approximations $\L A$ and $\U A$ for any $A \in \mathcal{F}_L(U)$ in terms of an $L$-relation $R$ on $U$. 
In this section, we consider a converse problem whether we can define an $L$-relation of certain type
for a dual pair of $L$-fuzzy operators. As we already noted in the previous section, for every $L$-relation $R$ on $U$, we have 
\[ \U (I_y)(x) = R(x,y), \]
for all $x, y\in U$ and  $A \in \mathcal{F}_L(U)$. This provides a `rule' for defining relations when upper approximations are known.
We call any map on $\mathcal{F}_L(U)$ as an \emph{$L$-fuzzy operator} on $U$. 

\begin{lemma} \label{lem:equations}
Let $\UU$ be an $L$-fuzzy operator. For any $a \in L$, $A \in \mathcal{F}_L(U)$ and $\{A_i\}_{i\in I} \subseteq \mathcal{F}_L(U)$ the following are equivalent:
\begin{enumerate}[label = {\rm (\arabic*)}]
\item $\UU(A) = \bigvee_x \big (\overline{A(x)} \wedge \UU (1_x) \big )$;
\item $\UU (\overline{a} \wedge \bigvee_i A_i) = \overline{a} \wedge \bigvee_i \UU(A_i)$.
\end{enumerate}
\end{lemma}  


\begin{proof}
(1)$\Rightarrow$(2): Suppose that (1) holds. Then,
\begin{align*}
\UU (\overline{a} \wedge A) &= {\bigvee}_x \big (\overline{(\overline{a} \wedge A)(x)} \wedge \UU (I_x) \big ) 
= {\bigvee}_x \big (\overline{a \wedge A(x)} \wedge \UU (I_x) \big ) \\
&= {\bigvee}_x \big ( \overline{a} \wedge \overline{A(x)}) \wedge \UU(I_x) \big )  
= \overline{a} \wedge {\bigvee}_x \big (\overline{A(x)}) \wedge \UU(I_x) \big )  \\
&= \overline{a} \wedge \UU(A)
\end{align*}
and 
\begin{align*}
\UU \big ({\bigvee}_i A_i \big ) &= {\bigvee}_x \big (\overline{ \big ({\bigvee}_i A_i \big )(x)} \wedge \UU (I_x) \big ) 
= {\bigvee}_x \big ({\bigvee}_i \overline{ (A_i)(x)} \wedge \UU (I_x) \big ) \\
&= {\bigvee}_x {\bigvee}_i \left (\overline{A_i(x)} \wedge \UU (I_x) \right)  
= {\bigvee}_i {\bigvee}_x \left (\overline{A_i(x)} \wedge \UU (I_x) \right)  \\
&= {\bigvee}_i \, \UU (A_i).  
\end{align*}
Combining these two we obtain that
\[ \UU \left ( \overline{a} \wedge {\bigvee}_i A_i \right ) = \overline{a} \wedge \UU \left ({\bigvee}_i A \right ) = \overline {a} \wedge {\bigvee}_i \, \UU (A_i).\]

\noindent%
(2)$\Rightarrow$(1): Suppose that (1) holds. Since $A = \bigvee_x (\overline{A(x)} \wedge I_x)$, we have that
\begin{align*}
\UU (A) & = \UU \left ( {\bigvee}_x \left ( \overline{A(x)} \wedge I_x \right ) \right ) 
 = \UU \left ( \overline{1} \wedge {\bigvee}_x \left (\overline{A(x)} \wedge I_x \right ) \right ) \\  
& = \overline{1} \wedge {\bigvee}_x \, \UU \left (\overline{A(x)} \wedge I_x \right ) 
 = {\bigvee}_x \UU \left (\overline{A(x)} \wedge I_x \right ) \\
&= {\bigvee}_x \left ( \overline{A(x)} \wedge \UU (I_x) \right ).
\qedhere
\end{align*}
\end{proof}


\begin{theorem} \label{thm:main1}
Let $\UU$ be an $L$-fuzzy operator on $U$ and let each $\SS_j$, $1 \leq j \leq n$, be an $L$-fuzzy operator on $U$ such that $\UU \leq S_j$.
Then there exists a unique $L$-fuzzy relation $R$ on $U$ such that $\U(A) = \UU(A)$ for all $A \in \mathcal{F}_L(U)$
if and only if
\begin{equation}\label{eq:multiple} 
\UU(\overline{a} \wedge {\bigvee}_i A_i) = \overline{a} \wedge {\bigvee}_i \left (\UU(A_i) \wedge \SS_1(A_i) \wedge \cdots \wedge \SS_n(A_i) \right )
\end{equation}
for $a \in L$ and $\{A_i\}_{i \in I} \subseteq \mathcal{F}_L(U)$. 
\end{theorem}

\begin{proof} 
$(\Rightarrow)$ Since $\UU \leq \SS_j$, for all $j \leq n$, we have $\UU \leq \SS_1 \wedge \cdots \wedge \SS_n$ and hence
\[ \UU = \UU \wedge \SS_1 \wedge \cdots \wedge \SS_n .\]
Suppose that there is an $L$-relation $R$ such that $\UU(A) = \U(A)$ for all $A \in \mathcal{F}_L(U)$. Then,    
\begin{align*}
  \UU \big (\overline{a}\wedge {\bigvee}_i A_i \big ) & = \U \big (\overline{a} \wedge {\bigvee}_i A_i \big )
  = \Big (\overline{a} \wedge \big ({\bigvee}_i \U (A_i) \big ) \Big ) = \overline{a}\wedge {\bigvee}_i \U (A_i) \\
& =\overline{a}\wedge {\bigvee}_i \UU (A_i) = \overline{a} \wedge {\bigvee}_i (\UU (A_i) \wedge \SS_1 (A_i) \wedge \cdots \wedge \SS_n(A_i)).
\end{align*}

\noindent%
$(\Leftarrow)$ Suppose that \eqref{eq:multiple} holds for $a \in L$ and $\{A_i\}_{i \in I} \subseteq \mathcal{F}_L(U)$. This gives
\[ \UU \big ( {\bigvee}_i \, A_i \big ) = \UU \big (\overline{1} \wedge {\bigvee}_i \, A_i \big ) =
\overline{1} \wedge {\bigvee}_i \, \UU(A_i) = {\bigvee}_i \, \UU(A_i) .\]
and $\UU (\overline{a} \wedge A) = \overline{a} \wedge \UU(A)$ for all $a \in L$ and $A \in \mathcal{F}_L(U)$, which are essential properties
of upper approximation operators. 

Let us define an $L$-relation $R$ on $U$ by setting for all $x,y \in U$, 
\[ R(x,y) = \UU(I_y)(x) .\]
Then, $\U (1_y)(x) = R(x,y) = \UU(1_y)(x)$ for all $x,y \in U$, and hence $\U(I_x) = \UU(1_x)$ for all $x\in U$. For $A \in \mathcal{F}_L(U)$, 
\begin{align*}
\UU(A) &= \UU \big ({\bigvee}_x \Big ( \overline{A(x)} \wedge I_x \big ) \Big )
= {\bigvee}_x \UU \big (\overline{A(x)} \wedge I_x \big ) \\
&= {\bigvee}_x  (\overline{A(x)} \wedge \UU(I_x)  ) 
= {\bigvee}_x (\overline{A(x)} \wedge \U (I_x) ) \\
&= {\bigvee}_x \U \big (\overline{A(x)} \wedge I_x \big )
= \U \Big ({\bigvee}_x \big (\overline{A(x)} \wedge I_x \big) \Big) \\
&= \U A.
\end{align*}
It is also clear that the induced relation $R$ is unique, since if there is an $L$-relation $Q$ such that $\U_Q (A) = \UU (A)$ for all $A \in \mathcal{F}_L(U)$, we have
$R(x,y) = \UU(I_y)(x) = \U_Q (1_y)(x) = Q(x,y)$ for all $x,y\in U$ and thus $R=Q$. 
\end{proof}

Let $\UU$ and $\LL$ be $L$-fuzzy operators. 
We say that $\UU$ and $\LL$ are \emph{dual} if $\UU(A') = (\LL(A))'$ for all $A \in \mathcal{F}_L(U)$. \label{page:dual}
Note that this is equivalent to $\LL(A') = (\UU(A))'$ for all $A \in \mathcal{F}_L(U)$. Note that we used the notion of dual operations already in
Section~\ref{Sec:Basics}, even it was not defined there. The dual operators define each other uniquely, that is,
\[ \UU(A) = (\LL(A'))' \quad \text{and} \quad \LL(A) = (\UU(A'))' , \]
for all $A \in \mathcal{F}_L(U)$. 

\begin{remark}
Suppose that the $L$-fuzzy operators $\UU$ and $\LL$ on $U$ are dual. Theorem~\ref{thm:main1} can be equivalently expressed in terms of $\LL$. Indeed, let
for any $1 \leq j \leq n$, each $\SS_j$ be an $L$-fuzzy operator on $U$ such that $S_j(A) \leq \LL(A)$ for any $A \in \mathcal{F}_L(A)$. Then there exists a unique
$L$-fuzzy relation $R$ on $U$ such that $\L A = \LL (A)$ and 
$\U A = \UU(A)$ for all $A \in \mathcal{F}_L(U)$ if and only if
\[
\LL \big ( \overline{a} \vee {\bigwedge}_i A_i \big ) = \overline{a} \vee {\bigwedge}_i \left( \LL(A_i) \vee \SS_1(A_i) \vee \cdots \vee \SS_n (A_i) \right )  
\]
for all $a\in L$ and  $A \in \mathcal{F}_L(U)$. 
\end{remark}

As a corollary of \ref{thm:main1}, we can get easily the following results, which were presented in \cite{Pang2019} (see Theorems~4.2--4.6). Note that case (e)
is a new result.

\begin{corollary}
Let $\UU$ and $\LL$ be dual $L$-fuzzy operators on $U$.
\begin{enumerate}[label = {\rm (\arabic*)}]
\item There exists a unique $L$-relation $R$ on $U$ such that $\UU$ and $\LL$ coincide with the upper and lower approximation operators of $R$, respectively, if and only if
\[
\UU (\overline{a} \wedge {\bigvee}_i A_i) = \overline{a} \wedge {\bigvee}_i \UU(A_i) 
\]
for all $a \in L$ and $\{A_i\}_{i \in I} \subseteq \mathcal{F}_L(U)$. 

\item There exists a unique mediate $L$-relation $R$ on $U$ such that $\UU$ and $\LL$ coincide with the upper and lower approximation operators of $R$, respectively, if and only if
\[
\UU(\overline{a} \wedge {\bigvee}_i A_i) = \overline{a} \wedge {\bigvee}_i   (\UU(A_i) \wedge \UU (\UU(A_i))  ) 
\]
for all $a \in L$ and $\{A_i\}_{i \in I} \subseteq \mathcal{F}_L(U)$. 

\item There exists a unique Euclidean $L$-relation $R$ on $U$ such that $\UU$ and $\LL$ coincide with the upper and lower approximation operators of $R$, respectively, if and only if
\[
\UU(\overline{a} \wedge {\bigvee}_i A_i) = \overline{a} \wedge {\bigvee}_i  (\UU(A_i) \wedge \LL (\UU(A_i))   ) 
\]
for all $a \in L$ and $\{A_i\}_{i \in I} \subseteq \mathcal{F}_L(U)$.

\item There exists a unique adjoint $L$-relation $R$ on $U$ such that $\UU$ and $\LL$ coincide with the upper and lower approximation operators of $R$, respectively, if and only if
\[
\UU(\overline{a} \wedge {\bigvee}_i A_i) = \overline{a} \wedge {\bigvee}_i   (\UU(A_i) \wedge \UU (\LL(A_i))   ) 
\]
for all $a \in L$ and $\{A_i\}_{i \in I} \subseteq \mathcal{F}_L(U)$.

\item There exists a unique functional $L$-relation $R$ on $U$ such that $\UU$ and $\LL$ coincide with the upper and lower approximation operators of $R$, respectively, if and only if
\[
\UU(\overline{a} \wedge {\bigvee}_i A_i) = \overline{a} \wedge {\bigvee}_i  (\UU(A_i) \wedge \LL(A_i)  ) 
\]
for all $a \in L$ and $\{A_i\}_{i \in I} \subseteq \mathcal{F}_L(U)$. 
\end{enumerate}
\end{corollary}

\begin{proof} (1) This follows directly from Theorem~\ref{thm:main1} by selecting $n = 1$ and $\SS_1 = \U$.

(2) By Proposition~\ref{prop:mediate}, $R$ is mediate if and only if $\U A \leq \U \U A$ for any $A \in \mathcal{F}_L(U)$ from which the result follows by
Theorem~\ref{thm:main1} by setting $n = 1$ and $\SS_1 = \U \U$. 

(3) The relation $R$ is Euclidean if and only if for all $A \in \mathcal{F}_L(U)$, $\U A \leq \L \U A$ by Proposition~\ref{prop:Eucl}. If we set $n = 1$ and
$\SS_1 = \L \U$, the claim follows from Theorem~\ref{thm:main1}.

(4) By Proposition~\ref{prop:adjoint}, $R$ is adjoint if and only if for all $A \in \mathcal{F}_L(U)$, $\U A \leq \U\L A$ and the claim follows from this by
Theorem~\ref{thm:main1}.

(5) The relation $R$ is functional if and only if $\U A \leq \L A$ for any $L$-set $A$ on $U$ by Proposition~\ref{prop:func}. The result is now clear by this.
\end{proof} 

Using Theorem~\ref{thm:main1} we can also give conditions for compositions of properties (cf.~Theorems 4.8, 4.9, 4.10, and 4.12 in  \cite{Pang2019}). The proof
of the following proposition is clear.

\begin{corollary}
Let $\UU$ and $\LL$ be dual $L$-fuzzy operator on $U$.
There exists a unique mediate, Euclidean and adjoint $L$-relation $R$ on $U$ such that $\UU$ and $\LL $ coincide with the upper and lower approximation operators of $R$,
respectively, if and only if
\[
\UU(\overline{a} \wedge {\bigvee}_i A_i) = \overline{a} \wedge {\bigvee}_i (\UU(A_i) \wedge \UU (\UU(A_i)) \wedge \LL(\UU(A_i)) \wedge \UU(\LL(A_i))) 
\]
for all $a \in L$ and $\{A_i\}_{i \in I} \subseteq \mathcal{F}_L(U)$. 
\end{corollary}

In  \cite{Pang2019}, the authors presented the following problem: ``Using single axioms to characterize $L$-fuzzy rough approximation operators corresponding to compositions
of serial, reflexive, symmetric, transitive, mediate, Euclidean and adjoint L-fuzzy relations.'' Next we solve this problem so that only the parts concerning serial and symmetric relations remains open. 

Our following theorem is closely related to Theorem~\ref{thm:main1}.

\begin{theorem} \label{thm:main2}
Let $\UU$ be an $L$-fuzzy operator on $U$ and let each $\TT_k$, $1 \leq k \leq m$, be an $L$-fuzzy operator on $U$ such that
$\UU \geq \TT_k$. Then there exists a unique $L$-fuzzy relation $R$ on $U$ such that $\U A = \UU(A)$ for all $A \in \mathcal{F}_L(U)$
if and only if
\begin{equation}\label{eq:multiple2} 
\UU(\overline{a} \wedge {\bigvee}_i A_i) = \overline{a} \wedge {\bigvee}_i (\UU(A_i) \vee \TT_1(A_i) \vee \cdots \vee \TT_m(A_i) )
\end{equation}
for $a \in L$ and $\{A_i\}_{i \in I} \subseteq \mathcal{F}_L(U)$. 
\end{theorem}

\begin{proof} 
$(\Rightarrow)$ Since $\UU \geq \TT_k$, for all $k \leq m$, we have $\UU \leq \TT_1 \vee \cdots \vee \TT_m$ and hence
\[ \UU = \UU \vee \TT_1 \vee \cdots \vee \TT_m .\]
If there is an $L$-relation $R$ such that $\U A = \UU A$ for all $A \in \mathcal{F}_L(U)$, then    
\begin{align*}
  \UU \big (\overline{a} \wedge {\bigvee}_i A_i \big ) & = 
  \U  \big (\overline{a} \wedge {\bigvee}_i A_i \big)
  = \big (\overline{a} \wedge \U \big ({\bigvee}_i A_i \big) \big ) = \overline{a}\wedge {\bigvee}_i \U (A_i) \\
& =\overline{a} \wedge {\bigvee}_i \UU (A_i) = 
\overline{a} \wedge {\bigvee}_i (\UU(A_i) \vee \TT_1 (A_i) \vee \cdots \vee \TT_m(A_i) ).
\end{align*}

\noindent%
The direction $(\Leftarrow)$ is proved as in Theorem~\ref{thm:main1}. 
\end{proof}

Now we can write the following corollary of Theorem~\ref{thm:main2}. 

\begin{corollary}
Let $\UU$ and $\LL$ be dual $L$-fuzzy operators on $U$.
\begin{enumerate}[label = {\rm (\arabic*)}]

\item There exists a unique reflexive $L$-relation $R$ on $U$ such that $\UU$ and $\LL$ coincide with the upper and lower approximation operators of $R$, respectively, if and only if
\[
\UU(\overline{a} \wedge {\bigvee}_i A_i) = \overline{a} \wedge {\bigvee}_i \left (\UU(A_i) \vee  A_i) \right ) 
\]
for all $a \in L$ and $\{A_i\}_{i \in I} \subseteq \mathcal{F}_L(U)$. 

\item There exists a unique transitive $L$-relation $R$ on $U$ such that $\U$ and $\L$ coincide with the upper and lower approximation operators of $R$, respectively, if and only if
\[
\UU(\overline{a} \wedge {\bigvee}_i A_i) = \overline{a} \wedge {\bigvee}_i (\UU(A_i) \vee \UU (\UU (A_i)) ) 
\]
for all $a \in L$ and $\{A_i\}_{i \in I} \subseteq \mathcal{F}_L(U)$. 
\end{enumerate}
\end{corollary}

\begin{proof} (1) By Proposition~\ref{prop:reflexive}, $R$ is reflexive if and only if $A \leq \U A$ for any $A \in \mathcal{F}_L(U)$ from which the result follows by
Theorem~\ref{thm:main2} by setting $n = 1$ and $\TT_1(A) = A$. 

(2) By Proposition~\ref{prop:transitive}, $R$ is transitive if and only if $\U \U A \leq \U A$ for any $A \in \mathcal{F}_L(U)$ from which the result follows by
Theorem~\ref{thm:main2} by setting $n = 1$ and $\TT_1 = \U \U$. 
\end{proof}

Let $\UU$ be an $L$-fuzzy operator on $U$. Assume that each $\SS_j$, $1 \leq j \leq n$, is an $L$-fuzzy operator on $U$ such that $\UU \leq \SS_j$ and
suppose each $\TT_k$, $1 \leq k \leq m$ is an $L$-fuzzy operator on $U$ such that $\UU \geq \TT_k$. Now we have that
\begin{equation} \label{eq:combined}
  \UU = (\UU \wedge \SS_1 \wedge \cdots \wedge \SS_n) \vee (\TT_1 \vee \cdots \vee \TT_m).
\end{equation}
We can now write the following theorem. Its proof is clear, because $(\Rightarrow)$ part follows from \eqref{eq:combined},
and $(\Leftarrow)$ can be proved as in Theorem \ref{thm:main1}.

\begin{theorem} \label{thm:main3}
Let $\UU$ be an $L$-fuzzy operator on $U$ and let each $\SS_j$, $1 \leq j \leq n$ and $\TT_k$, $1 \leq k \leq m$, be $L$-fuzzy operators on $U$ such that
$\UU \leq \SS_j$ and $\UU \geq \TT_k$. Then there exists a unique $L$-fuzzy relation $R$ on $U$ such that $\U A = \UU(A)$ for all $A \in \mathcal{F}_L(U)$
if and only if
\[
\UU(\overline{a} \wedge {\bigvee}_i A_i) = \overline{a} \wedge {\bigvee}_i ( (\UU(A_i) \wedge \SS_1(A_i) \wedge \cdots \wedge \SS_n(A_i))
\vee (\TT_1(A_i) \vee \cdots \vee \TT_m(A_i)) )
\]
for $a \in L$ and $\{A_i\}_{i \in I} \subseteq \mathcal{F}_L(U)$. \qed 
\end{theorem}

We end this work by the following example of a single condition for a combination of certain relation types.

\begin{corollary}
Let $\UU$ and $\LL$ be dual $L$-fuzzy operators on $U$. There exists a reflexive, transitive, mediate, Euclidean and adjoint $L$-relation $R$ on $U$ such that $\UU$ and $\LL$
coincide with the upper and lower approximation operators of $R$,
respectively, if and only if
\[
 \UU(\overline{a} \wedge {\bigvee}_i A_i) 
 = \overline{a} \wedge {\bigvee}_i 
\big( 
(\UU(A_i) \wedge \UU (\UU(A_i)) \wedge \LL(\UU(A_i)) \wedge \UU(\LL(A_i)) )
\vee ( A_i \vee \UU(\UU(A_i)))
\big )
\]
for all $a \in L$ and $\{A_i\}_{i \in I} \subseteq \mathcal{F}_L(U)$. 
\end{corollary}


%\bibliographystyle{amsalpha}
%\bibliography{literature}


\providecommand{\bysame}{\leavevmode\hbox to3em{\hrulefill}\thinspace}
\providecommand{\MR}{\relax\ifhmode\unskip\space\fi MR }
% \MRhref is called by the amsart/book/proc definition of \MR.
\providecommand{\MRhref}[2]{%
  \href{http://www.ams.org/mathscinet-getitem?mr=#1}{#2}
}
\providecommand{\href}[2]{#2}
\begin{thebibliography}{MLM15}

\bibitem[CMS11]{Castano11}
Valeria Casta{\~n}o and Marcela Mu{\~n}oz~Santis, \emph{Subalgebras of
  {H}eyting and {D}e~{M}organ {H}eyting {A}lgebras}, Studia Logica \textbf{98}
  (2011), 123--139.

\bibitem[DP90]{DuboisPrade1990}
Didier Dubois and Henri Prade, \emph{Rough fuzzy sets and fuzzy rough sets},
  International Journal of General Systems \textbf{17} (1990), 191--209.

\bibitem[J{\"a}r05]{Jarvinen2005}
Jouni J{\"a}rvinen, \emph{Properties of rough approximations}, J. Adv. Comput.
  Intell. Intell. Inform. \textbf{9} (2005), 502--505.

\bibitem[MLM15]{Ma2015}
Zhouming Ma, Jinjin Li, and Jusheng Mi, \emph{Some minimal axiom sets of rough
  sets}, Information Sciences \textbf{312} (2015), 40 -- 54.

\bibitem[Or{\l}98]{Orlowska1998}
Ewa Or{\l}owska, \emph{Introduction: What you always wanted to know about rough
  sets}, Incomplete Information: Rough Set Analysis (Ewa Or{\l}owska, ed.),
  Physica-Verlag, Heidelberg, 1998, pp.~1--20.

\bibitem[Paw82]{Pawlak82}
Zdzis{\l}aw Pawlak, \emph{Rough sets}, International Journal of Computer and
  Information Sciences \textbf{11} (1982), 341--356.

\bibitem[PMY19]{Pang2019}
Bin Pang, Ju-Sheng Mi, and Wei Yao, \emph{{$L$}-fuzzy rough approximation
  operators via three new types of {$L$}-fuzzy relations}, Soft Computing
  \textbf{23} (2019), no.~22, 11433 -- 11446.

\bibitem[RK04]{radzikowska2004fuzzy}
Anna~Maria Radzikowska and Etienne~E Kerre, \emph{Fuzzy rough sets based on
  residuated lattices}, Transactions on Rough Sets II \textbf{II} (2004),
  278--296.

\bibitem[WZ04]{WuZang2004}
Wei-Zhi Wu and Wen-Xiu Zhang, \emph{Constructive and axiomatic approaches of
  fuzzy approximation operators}, Information Sciences \textbf{159} (2004),
  233--254.

\bibitem[YL96]{Yao1996}
Y.Y. Yao and T.Y. Lin, \emph{Generalization of rough sets using modal logics},
  Intelligent Automation \& Soft Computing \textbf{2} (1996), 103--119.

\bibitem[Zhu07]{Zhu2007}
William Zhu, \emph{Generalized rough sets based on relations}, Information
  Sciences \textbf{177} (2007), no.~22, 4997 -- 5011.

\end{thebibliography}

\end{document}
