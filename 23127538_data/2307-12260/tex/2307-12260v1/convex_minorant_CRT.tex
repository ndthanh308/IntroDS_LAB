%!TEX root = MST_brownian.tex

% \begin{alert}Decide about the orientation of time in this secion. It could make sense to use $Z^\tau$, $\tau\ge 0$,  rather than $Z^\lambda$, $\lambda \le 0$. 

% $Z^\tau_-$

% Insert ?
% \begin{compactitem}
%   \item \citet{Pitman1999b} coalescent random forests ? discrete version of \cite{AlPi1998a}
%   \item \cite{Bertoin2012a} which contains the origins of the cut tree in \cite{BeMi2013a}
% \end{compactitem}


% \end{alert}


In this section, we study the convex minorant tree of a standard Brownian excursion. We prove Theorem~\ref{thm:limit_mst_surplus} in the case where $s=0$ which says that $\CMT(\exc,\bU)$ is distributed like the Brownian continuum random tree, and Theorem~\ref{thm:additive_coalescent} which relates $\CMT(\exc,\bU)$ to the additive coalescent. We are interested here in the case of excursions, and the natural range of interest for $Z^\lambda(\omega)$ is then $\lambda \in (-\infty, 0]$, and we shall therefore rather work with $(Z^{-\tau}(\omega))_{\tau \ge 0}$, which also turns out to be a cadlag process (see Lemma~\ref{lem:properties_Z}). We still occasionally use the parameterization with $\lambda$.

% In the new encoding of the Brownian continuum random tree, we can express the distances using an explicit Hausdorff measure for the geodesics; this is the topic of Section~\ref{sub:geodesics_length}.

\subsection{A fragmentation connected to Brownian motion} % (fold)
\label{sub:a_fragmentation_connected_to_brownian_motion}

The properties of $Z^\lambda$ is intimately related to the following operators. For $\lambda \in \R$, define the operator $\Psi_\lambda$ as follows: for a function $f$ continuous on an interval $D\subseteq \R_+$, and $t \in D$ 
\begin{equation}\label{eq:def_shear}
  \Psi_\lambda f(t):=f(t)+\lambda t -\inf\{f(s)+\lambda s: s\in D, s\le t\}\,.
\end{equation}
Then, $Z^\lambda(\omega)=\{s\in D: \Psi_\lambda \omega (s) = 0\}$. 
The family of operators $(\Psi_\lambda)_{\lambda \in \R}$ enjoys the following composition property, which is a straightforward reformulation of the arguments leading to Theorem~1 i) of \cite{Bertoin2000a}. For $t\ge 0$, let $\frak S_t$ denote the shift operator defined by $\frak S_t f(s)= f(t+s)$, for all $s\ge 0$. 
\begin{lem}\label{lem:shear_composition}
  Let $f$ be a continuous function on $D\subseteq \R_+$ and suppose that, for some $\lambda \in \R$ and $t\in D$, we have $\Psi_\lambda f(t)=0$. Then, for all $h,s\ge 0$ with $t+s\in D$ one has
  \[\frak S_t \Psi_{\lambda-h} f (s) = \Psi_{\lambda-h}f(t+s) = \Psi_{-h} \frak S_t \Psi_\lambda f(s)\,.\]
  In particular, $\Psi_{\lambda-h}f(t)=0$ for all $h\ge 0$. 
\end{lem}

We now go back to the case where $f=\exc$ is a Brownian excursion and write $Z^{\lambda}=Z^\lambda(\exc)$ (in this case, $D=[0,1]$). Lemma~\ref{lem:shear_composition} implies for instance that $Z^{-\tau}=Z^{-\tau}(\exc)$ is non-decreasing in $\tau$ for the inclusion, and thus induces a fragmentation in the sense that the connected components of its complement split as $\tau$ increases. For any $x\in [0,1)$ let $I^\tau(x)$ be the maximal interval of the form $[a,b)$ containing $x$ such that for $(a,b)\cap Z^{-\tau} = \varnothing$. For $x,y\in [0,1)$, we let $x\sim_\tau y$ if $I^\tau(x)=I^\tau(y)$. Observe that, for every $\tau\ge 0$, the collection of $I^\tau(x)$ forms a partition of $[0,1)$. 

By Lemma~\ref{lem:properties_Z}, a.s, for every $\tau\ge 0$, $[0,1]\setminus Z^{-\tau}$ consists in countably many open intervals, whose lengths we denote by $F_1(\tau), F_2(\tau), \dots$ in the decreasing order. Then, let $F(\tau)=(F_i(\tau))_{i\ge 1}$. 
The main result of \citet{Bertoin2000a} is that the process $(F(\tau), \tau\ge 0)$ has the same distribution as another remarkable fragmentation introduced by Aldous and Pitman \cite{AlPi1998a}, where a Brownian continuum random tree is logged along its skeleton at the points of an (independent) Poisson point process of unit intensity; the process of interest is the sequence of sorted masses of the fragments. This shows in particular that, up to a time change, the time reversal of $(F(\tau))_{\tau\ge 0}$ is the classical standard additive coalescent. 

Although there is no obvious coupling between the two representations directly in the continuous, this shows that the fragmentation of $[0,1]$ constructed by Bertoin corresponds to a fragmentation of a certain Brownian continuum random tree. This section will show that (one choice for) this tree is the convex minorant tree $\CMT(\exc,\bU)$. We will also identify the collection of points/times where/when it should be cut and thereby, provide a coupling between the two representations. 

The rest of the section is organized as follows. In Section~\ref{sec:dynamics}, we make explicit the correspondence between $\CMT(\exc,\bU)$ and the dynamics related to the process $Z^{-\tau}$ described above. In Section~\ref{sub:the_genealogy_and_the_marked_cut_tree}, following \citet{BeMi2013a}, we introduce the cut tree which encodes the genealogy of the fragmentation $(F(\tau))_{\tau\ge 0}$. The cut tree is a crucial ingredient since it provides the link between the fragmentation and the recovery of ``the tree what was logged'' through the \emph{inverse cut tree transform} that has been studied in \cite{AdBrGo2010,BrWa2017b}. In Section~\ref{sub:distribution_CMT}, we make the connection between the cut tree, the inverse transform and $\CMT(\exc,\bU)$ and complete the proofs of Theorem~\ref{thm:limit_mst_surplus} (with $s=0$) and Theorem~\ref{thm:additive_coalescent}. 


\subsection{Making the dynamics explicit}
\label{sec:dynamics}

In this section, we provide another point of view on the convex minorant tree that makes explicit its relation with the fragmentation of $[0,1]$ induced by $Z^{-\tau}=Z^{-\tau}(\exc)$.  

\begin{lem}\label{lem:fragment_of_x}Almost surely, the following holds for every point $x\in [0,1]$. Let $(t_i)_{i\ge 0}$ and $(\gamma_i)_{i\ge 0}$ be the vertices and the slopes of the convex minorant of $\exc$ on $[0,x]$. Then, setting $\gamma_{-1}=0$ for convenience,  we have for all $i\ge 0$, 
\begin{compactenum}[i)]
  \item $\inf I^{\tau}(x) = t_i$ for all $\tau \in [\gamma_{i-1}, \gamma_{i})$, and 
  \item $\sup I^{\tau}(x) = z_i$ for $\tau = \gamma_{i-1}$. 
\end{compactenum}
\end{lem}
\begin{proof}We work on a set $\Omega^\star$ of probability one where all the events of Lemma~\ref{lem:no_exception_convex} occur, in particular, the slopes $(\gamma_i)_{i\ge 0}$ are strictly increasing for every $x\in [0,1]$. The rest of the proof is deterministic, and we proceed by induction on $i\ge 0$. 

Write $\exc^\lambda$ for the function $s\mapsto \exc(s)+\lambda s$. 
For $i=0$, by construction of the convex minorant, for every $\tau \in [0,\gamma_0)$, $\exc^{-\tau}$ is positive on $(0,x]$ and thus $\inf I^{\tau}(x)=0=t_0$. For $\tau=\gamma_0$ we have $\exc^{-\gamma_0}(t_1)=\exc^{-\gamma_0}(z_1)=0$, and $\exc^{-\gamma_0}(s)>0$ for $s\in (t_1,z_1)$. It follows that $t_1,z_1\in Z^{-\gamma_0}$ and that $\sup I^{\gamma_0}(x)=z_1$. 

Suppose now that, for some $j\ge 0$, the claims in \emph{i)} and \emph{ii)} both hold for all $0\le i\le j$, and that $t_{j+1},z_{j+1}\in Z^{-\gamma_{j}}$. By expressing $\Psi_{-\gamma_j-h}\exc$ for $h\ge 0$ in terms of $\Psi_{-\gamma_j}\exc$, Lemma~\ref{lem:shear_composition} allows us to proceed. First note that the vertices of the convex minorant of $\Psi_{-\gamma_{j}}\exc$ on $[0,x]$ that are in $[t_{j+1},1]$ are precisely $(t_{j+k})_{k\ge 1}$ and the corresponding slopes are $(\gamma_{j+k}-\gamma_j)_{k\ge 1}$. The argument we have just used for $j=0$ applies to $\frak S_{t_{j+1}}\Psi_{-\gamma_j}\exc$ and yields that for all $h\in [0, \gamma_{j+1}-\gamma_j)$, we have $\inf I^{\gamma_j +h}(x) = t_{i+1}$ and $\sup I^{\gamma_{i+1}}(x)=z_{i+1}$. Furthermore, for $h=\gamma_{j+1}-\gamma_j$, $t_{j+2}$ and $z_{j+2}$ are both zeros of $\Psi_{-\gamma_{j+1}}\exc$, while the latter is positive on $(t_{j+2},z_{j+2})$. This completes the proof.
\end{proof}

For $x,y\in [0,1)$, define $\tau(x,y)=\sup\{\tau\ge 0: x\sim_{\tau} y\}$. Note that, by the left-continuity of $Z^\lambda$, we have $Z^{-\tau(x,y)}\cap [x,y]\ne \varnothing$. Recall the definition of $\xi_m$ from Section~\ref{sub:recursive_convex_minorants}, which is also the point $\ju(t_m)$ as defined in Remark~\ref{rem:points_absolute}. 

\begin{lem}\label{lem:cut_point-time}
Almost surely for every $x\ne y\in [0,1)$, we have the following: let $(t_i)_{i\ge 0}$ be the vertices of the convex minorant of $\exc$ on $[0,\max\{x,y\}]$. Then, $m:=\min\{i\ge 1: t_i>\min\{x,y\}\}<\infty$, and:
\begin{compactenum}[i)]
  \item $Z^{-\tau(x,y)}\cap [x,y]$ consists of the single point $\kappa(x,y)=t_m$ that we call a cut point;
  \item $\tau(x,y)=\gamma_{m-1}$;
  % \item $I^{\tau(x,y)-}(x)=I^{\tau(x,y)-}(y)=[t_{m-1}, z_m)$; and
  \item $I^{\tau(x,y)}(\min\{x,y\})=[t_{m-1}, t_m)$ and $I^{\tau(x,y)}(\max\{x,y\})=[t_m, z_m)$.
\end{compactenum} 
Furthermore, we let $\eta(x,y)=\xi_m=\ju(t_m) \in (t_{m-1},t_m)$; conditionally on $I^{\tau(x,y)}(\min\{x,y\})=S$, $\eta(x,y)$ is uniformly distributed on $S$.
\end{lem}
\begin{proof}The set of probability one is $\Omega^\star$ where all the events of Lemma~\ref{lem:no_exception_convex} occur for every point of $[0,1]$. The points \emph{i)} to \emph{iii)} are straightforward consequences of Lemma~\ref{lem:fragment_of_x}, applied to the fragment containing $I^\tau(\max\{x,y\})$ until the time when it does not contain $\min\{x,y\}$ any longer. The statement concerning the distribution of $\eta(x,y)$ is a consequence of fact that $\eta(x,y)$ is then $\ju(t_m)$, which is uniform in $[t_{m-1},t_m]$ conditionally on $t_{m-1},t_m$.
\end{proof}

Observe that Lemma~\ref{lem:cut_point-time} implies that, almost surely for every $x\ne y$, we have
\begin{equation}\label{eq:frag_at_cuttime}
I^{\tau(x,y)-}(x)=\bigcap_{\tau<\tau(x,y)} I^{\tau}(x) = I^{\tau(x,y)}(x) \sqcup I^{\tau(x,y)}(y)\,. 
\end{equation}

We are now ready to move on to the main objective of this section, namely proving that both $\llb x,y\rrb$ and $d(x,y)$ may be defined using an alternative binary decomposition where the intervals containing a pair of marked points are split at the corresponding cut point, just as in \eqref{eq:frag_at_cuttime} above.

Let $\cU_2=\bigcup_{n\ge 0} \{0,1\}^n$, where it is understood that $\{0,1\}^0=\{\varnothing\}$. 
Fix now $x,y\in (0,1)$. We define recursively $(\Pi_u,\tau_u,\kappa_u, A_u, B_u)_{u\in \cU_2}$, where $\Pi_u$ is an interval,  $A_u\le B_u$ are two points in the closure of $\Pi_u$, and the values $\tau_u\in \R_+$, $\kappa_u\in [0,1]$ are always such that $\tau_u=\tau(A_u,B_u)$, $\kappa_u=\kappa(A_u,B_u)$. It is understood that all these random variables depend on $x,y$, so we actually have $\Pi_u(x,y), \tau_u(x,y), \kappa_u(x,y)$, $A_u(x,y)$, $B_u(x,y)$, for $u\in \cU_2$, but we usually omit the reference to $x,y$. Set $\Pi_\varnothing=(0,1)$, $\tau_\varnothing(x,y)=\tau(x,y)$, $\kappa_\varnothing(x,y)=\kappa(x,y)$ and $A_\varnothing=\min\{x,y\}$, $B_\varnothing = \max\{x,y\}$. Let $\Pi_0=I^{\tau_\varnothing}(A_\varnothing)$ and $\Pi_1=I^{\tau_\varnothing}(B_\varnothing)$. 

Assuming that we have defined $(\Pi_u, \tau_u, \kappa_u, A_u, B_u)$ for some $u\in \cU_2$ we then set $\Pi_{u0}=I^{\tau_u}(A_u)$, $\Pi_{u1}=I^{\tau_u}(B_u)$, $A_{u0}=\min\{A_u,\eta(A_u,B_u)\}$, $B_{u0}=\max\{A_u,\eta(A_u,B_u)\}$ and $A_{u1}=\kappa_u=\inf \Pi_{u1}$, $B_{u1}=B_u$. We finally define $\tau_{ui}=\tau(A_{ui},B_{ui})$ and $\kappa_{ui}=\kappa(A_{ui},B_{ui})$ for $i\in \{0,1\}$. 

\begin{lem}\label{lem:cut_points}Almost surely for every $x,y\in[0,1]$, for every $u\in \cU_2$, we have $A_u,B_u\in \llb 0,x\rrb \cup \llb 0,y\rrb$.
\end{lem}
\begin{proof}This is a straightforward induction. For $u=\varnothing$, we have $\{A_\varnothing,B_\varnothing\}=\{x,y\}$ and the claim holds by definition. Assume now that it holds for some $u\in \cU$. We have $\{A_{u0}, B_{u0}, A_{u1}, B_{u1}\}=\{A_u,B_u, \kappa(A_u,B_u), \eta(A_u,B_u)\}$. By Lemma~\ref{lem:cut_point-time}, $\kappa(A_u,B_u)$ is a vertex on the convex minorant of $\exc$ on the interval $[0,\max\{A_u,B_u\}]$ and thus lies in $\llb 0, \max\{A_u,B_u\}\rrb \subseteq \llb 0,x\rrb \cup \llb 0,y \rrb$ by the induction hypothesis and Lemma~\ref{lem:geodesic_restriction}. The same holds for $\eta(A_u,B_u)$ by Lemma~\ref{lem:cut_point-time} and the definition of $\llb \cdot, \cdot \rrb$.
\end{proof}

To avoid any difficulties, we define $\hat d_\lambda$ only for almost every pair of points. This will be enough to exhibit the dynamic properties we have in mind, and settle the foundations for the coupling of Section~\ref{sec:coupling} that allows to identify the law of $\CMT(X,\bU)$.  
In the following, $\overline{\Pi}_u$ denotes the closure of $\Pi_u$. Let 
\[\Pi(x,y) = \bigcap_{n\ge 0} \bigcup_{|u|=n} \overline{\Pi}_u 
\qquad \text{and} \qquad {}
\hat d(x,y) = \limsup_{n\to \infty} \sum_{|u|=n} |\Pi_u|^{1/2}\,. 
\]
The set $\Pi(x,y)$ is well-defined and non-empty, but it is so far unclear whether $\hat d(x,y)$ is finite. 
% \begin{alert}If $\Pi(x,y)$ and $\hat d(x,y)$ are only defined for a.e.\ $(x,y)$, we cannot have something a.s.\ for every pair. FIX
% \end{alert}
\begin{prop}\label{pro:distance_pairs}For any $x,y\in [0,1]$, we have almost surely
\begin{compactenum}[i)]
  \item $\Pi(x,y) = \llb x,y\rrb$, and
  \item $\hat d(x,y) = d(x,y)$.
\end{compactenum}
\end{prop}
\begin{proof}\emph{i)} Recall that, by definition, $\llb x,y \rrb$ is the union of $\llb 0,x\rrb \cap [x\wedge y, 1]$ and $\llb 0,y\rrb \cap [x\wedge y, 1]$. We follow the binary decomposition defining $\Pi(x,y)$; for each $n\ge 0$, let $0^n$ be the left-most node in $\cU_2$ at level $n$; we agree that, in this context, $0^0=\varnothing$. We show that, for each $n\ge 0$, the two sets $\Pi(x,y)$ and $\llb x,y\rrb$ coincide on $[\kappa_{0^n},1]$; we will then show that $\kappa_{0^n}=\kappa_{0^n} \downarrow x\wedge y$ as $n\to\infty$.   

For $n=0$, we have $\kappa_{0^0}=\kappa_{\varnothing}=\kappa(x,y)$, $A_\varnothing = \min\{x,y\}$ and $B_\varnothing = \max\{x,y\}$. By Lemma~\ref{lem:cut_point-time} \emph{i)} and the definition of $\llb 0, B_\varnothing\rrb$ in Equation~\eqref{def:0x}, the set $\Pi(x,y)\cap [\kappa_{0^0}, 1]$ is contained in $\llb 0, B_\varnothing\rrb$; furthermore, since $x\wedge y \le \min\{x,y\}=A_\varnothing$, it is also the case that $\Pi(x,y) \cap [\kappa_{0^0}, 1]$ is  contained in $\llb 0,B_\varnothing\rrb \cap [x\wedge y, 1]$. On the other hand, by Lemma~\ref{lem:cut_points}, $\kappa(x,y)\in \llb 0, \max\{x,y\}\rrb$, and one easily sees that $\Pi(x,y)\cap [\kappa_{0^0},1]=\llb 0, B_\varnothing\rrb \cap [\kappa_{0^0},1]$. Indeed, we may now expand $\Pi_{1}$ on the right using the recurrence relation: writing $0^i1^j$ for the node at level $i+j$ in $\cU_2$ obtained by walking $i$ steps left, and then $j$ steps right from the root, and it should be plain that the points $\kappa_{1^k}$, $k\ge 0$, are simply the vertices of the convex minorant of $\exc$ on $[0,B_\varnothing]$ that are larger than $\kappa_\varnothing = \kappa_{0^0}$. It follows that the sets $\Pi_{10}, \Pi_{1^20},\dots, \Pi_{1^i0}, \dots $ all explicitly appear in the decomposition defining $\llb 0, B_\varnothing\rrb$ on the interval $[\kappa_{0^0},1]$. 

Now for any $n\ge 1$, assuming that we have treated the part of $\Pi(x,y)$ lying in $[\kappa_{0^n},1]$, we are left with the portion of $\Pi(x,y)$ that lies in $[0,\kappa_{0^n}]$, which is constructed from $\Pi_{0^{n+1}}$. By Lemma~\ref{lem:cut_points}, we have $A_{0^{n+1}},B_{0^{n+1}}\in \llb 0,x\rrb \cup \llb 0, y\rrb$, and we have $\kappa_{0^{n+1}}=\kappa(A_{0^{n+1}},B_{0^{n+1}})$. To the right, we have the set $\Pi_{0^{n+1}1}$, that we may expand from the right using the recurrence relation. The arguments above imply that the $\kappa_{0^n1^k}$, $k\ge 0$, are the vertices of the convex minorant on the interval $[0,B_{0^{n+1}}]$ that are at least $\kappa_{0^{n+1}}$. Therefore, $\Pi(x,y)$ and $\llb x,y\rrb$ coincide on $[\kappa_{0^{n+1}}, B_{0^{n+1}}]$ and thus on $[\kappa_{0^{n+1}}, \kappa_{0^n}]$, and in turn on $[\kappa_{0^{n+1}}, 1]$ by the induction hypothesis. 

% On the right, the set $\Pi \cap [\kappa(x,y), \max\{x,y\}]$ is contained in $\llb 0,\max\{x,y\}\rrb$, and this $x\wedge y \le \min\{y,y\}$, $\Pi\cap [\kappa(x,y),\max\{x,y\}]$ is also contained in $\llb 0,\max\{x,y\}\rrb \cap [x\wedge y,\max\{x,y\}]$. Then, a simple induction yields that for every $u_n=0\dots0$, with $|u_n|=n$, we have the same on the interval $\Pi_{u_n}$: the set $\llb[\kappa_{u_n}, B_n\rrb]$ is 

% The set $\llb \kappa(A_{u_n},B_{u_n}), B_{u_n}\rrb$ is constructed as $\Pi \cap [\kappa_{A_{u_n}, B_{u_n}}, 1] \cap \Pi_{u_n1}$.
% \begin{alert}Need a simple argument for $\Leb(\Pi_{u_n})\to 0$: This is a bit of a pain to write: essentially, either (1) for some finite $N\in \N$, and all $n\ge N$, we have $\Pi_{u_n}$ is the interval containing two random points, and the the length decreases as a product of iid beta; $N$ is the first time when $A_{u_n}\not\in \{x,y\}$. Or (2) $N=\infty$ and then $\min\{x,y\}=x\wedge y$ so we are fine as well. 

% We can avoid the problem by taking only dealing with almost every $x,y$... 
%  \end{alert} 

% with $|v_n|=n$ and $v_n=0\dots 0$, we have 
Then, note that for each $n\ge 0$, $x\wedge y \in \Pi_{0^n}$. To see this, it suffices to note that for each $n \ge 0$, one of $(A_{0^n}, B_{0^n})$ or $(B_{0^n},A_{0^n})$ lies in $\llb 0, x\rrb \times \llb 0,y \rrb$. This is clearly true for $n=0$, and carries on because at each step we replace $\max\{A_{0^n}, B_{0^n}\}$ by $\eta(A_{0^n}, B_{0^n})$ which lies in $\llb 0, \max\{A_{0^n}, B_{0^n}\}\rrb$. Lemma~\ref{lem:geodesic_restriction} them implies that $\inf \Pi_{0^n} \in \llb 0,x\rrb \cap \llb 0,y\rrb$, which proves the claim. Since $\inf \Pi_{0^n}$ is non-decreasing, it would suffice to prove that $\diam(\Pi_{0^n})=\Leb(\Pi_{0^n})\to 0$ in order to show that  $\kappa_{0^n} \to x\wedge y$, which would complete the proof of \emph{i)}. So let us now this why $\Leb(\Pi_{0^n})\to 0$. For every $u$, we have $A_u\in [\inf \Pi_u, \kappa_u]$. Then, either $A_u<\eta_u$ and $|\Pi_{u0}|\le |\Pi_u| \cdot U$ where $U$ is uniformly random on $[0,1]$, or $A_u\ge \eta_u$, and then $\Pi_{u0}$ contains two uniform random points so that, $|\Pi_{u00}|\le |\Pi_u| \cdot M$, where $M$ is a Beta$(\tfrac 12, 1)$ random variable by Lemma~\ref{lem:convex_minorant_law1}. Since all the random variables are independent, it is straightforward that $|\Pi_{0^n}|\to 0$ with probability one as $n\to\infty$. 

\emph{ii)} The correspondence between the sets that are used to define $\Pi(x,y)$ and $\llb x,y\rrb$ in the proof of \emph{i)}, also yields a way of rewriting the sums which proves that $\hat d(x,y)=d(x,y)$. We omit the details.
\end{proof}

% \begin{alert}For the Brownian excursion, it is more natural to see the fragmentation direction since it gives $\R_+$ as the time interval; for the Brownian with parabolic drift, this is the other way around since we want something that is closer to random graphs. The problem is that we do not want two different definitions for stuff that are different, and coalescence should be in the same direction. 
% \end{alert}

% For the Brownian excursion, the relevant range for $\lambda$ is $\R_-=(-\infty, 0]$; but this range might be different in other situations. For instance, for the Brownian motion with parabolic driff $X^0$, the natural range is $\R$. 



% \begin{alert}The process of Bertoin was right-continuous in the direction of fragmentation: essentially, there is a first time when the cut is there, and at that time it is already there; in the recursive definition for the sets and distances, if one of the points lies in $Z$, there is just some part of the tree that disappears. There is still this stuff about using countably many points, and the "right" points are precisely those some of $Z$ (not all of $Z$, which is uncountable, but enough to recover $Z$). Initially, this was using independent uniforms, which is nice because we know the laws, but then we loose the obvious fact that everything is measurable with respect to only $e,\bU$, although it is.
% \end{alert}

% \begin{lem}Almost surely, $\Leb(Z)=0$.
% \end{lem}

% \begin{lem}\label{lem:shift_Z}
% Suppose that $t\in Z^\lambda$ for some $t\in [0,1]$ and $\lambda \le 0$. Then, for $s\ge 0$ and $h\ge 0$, we have 
% \[e^{\lambda-h}_{t+s}-\underline{e}^{\lambda-h}_{t+s}
% =f_s^{(\lambda,t)}- h s -\inf\{f_r^{(\lambda,t)}-h r: r\le s\}\,\]
% where $f_r^{(\lambda,t)}:=e^{\lambda}_{t+r}-\underline e^{\lambda}_{t+r}$.
% \end{lem}





% It is crucial to observe that Lemma~\ref{lem:cut_point-time} implies that $\kappa(x,y)\in \sL$ with probability one. In particular, there exists some 


% In this section, we provide the connection with the representation of the fragmentation dual to the additive coalescent provided by Bertoin \cite{Bertoin2000a}. We slightly adapt the presentation of Bertoin in terms of a Brownian excursion. Let $(e_s)_{s\in [0,1]}$ be a Brownian excursion. For $t\ge 0$, define 
% \[e^{t}_s:=e_s - t s \qquad \text{and} \qquad \underline{e}^t_s:=\inf\{e^t_r: 0\le r\le s\}\,.\]
% Then, for $t\ge 0$, let $Z^t:=\{s\in [0,1]: e^t_s = \underline{e}^t_s\}$. Observe that for every $t,h\ge 0$, we have  $Z^t\subseteq Z^{t+h}$ so that $(Z^t)_{t\ge 0}$ defines a refining fragmentation of the interval $[0,1]$. For $x,y\in [0,1]$ we write $x\sim_t y$ if there is no point of $Z^t$ between $x$ and $y$, and let 
% \[I^t(x)=\{y\in [0,1]: y\sim_t x\} = (L_t(x), R_t(x))\,,\]
% where $L_t(x)=\sup \{[0,x]\cap Z^t\}$ and $R_t(x)=\inf\{[x,1]\cap Z^t\}$.
% \JF{`` if there is no point of $Z^t$'', ne semble pas convenir, car si $x$ est un point de $Z^t$, et mettons que $x$ est isolé dans $Z_t$, alors $x$ est $\sim_t$ aux gens à sa droite et à ceux à sa gauche pour cette définition... On dirait que la relation $x\sim_t y$ devrait être définie uniquement sur  $[0,1]\setminus Z^t$} 
% For every $t\ge 0$, $[0,1]\setminus Z_t$ consists in a union of coutably many open intervals whose length sum to one. \JF{mettre ça avant la définition de $I^t?$} We denote their ranked sequence of lenghts by $(F_1(t)\ge F_2(t)\ge \dots $. The main result of Bertoin \cite{Bertoin2000a} is that the process $(F_i(t),\JF{i\geq 1})_{t\ge 1}$ has the same distribution as the fragmentation (\JF{sizes}) of the Brownian continuum random tree studied by \citet{AlPi1998a}, \JF{that we recall now}. 



% Our construction of the recursive convex minorants is intimately related to the fragmentation of Bertoin, and thus to the fragmentation of a Brownian continuum random tree. More precisely, the convex minorants provide an exploration of the fragmentation of Bertoin. This is made formal by the following proposition {\red (see Figure~XXX)}: 

% \begin{alert}The notation with the $t$ and $t_i$ is a bit unfortunate...
% \end{alert}

% \begin{prop}\label{pro:Bertoin-GCM}Let $(t_i,z_i,\gamma_i)_{i\ge 0}$ be the vertices, intercepts and slopes of the greatest convex minorant of $e$ on the interval $[0,x]$. Then, with probability one, for all $i\ge 1$, and all $t\in [\gamma_i, \gamma_{i+1})$ we have
% \[L_{t}\JF{(x)}=t_i, \qquad \text{and} \qquad R_{t}\JF{(x)}=\inf\{s> t_i: e_s \le e_{t_i}+t (s-t_i) \}\,,\]
% so that $\gamma_i$ is the time of the $i$-th jump of the process $(L_t(x))_{t\ge 0}$, and $z_i=R_{t_i}\JF{(x)}$. 
% \end{prop}
% {\red 
% \begin{proof}Well, should be clear: picture ?
% \end{proof}
% \JF{Faut peut-être rappeler que les points extrémaux du greatest common minorant peuvent être énumérés à partir de 0}
% \begin{cor}\label{cor:cut-points-times}Suppose that $y<x$ and let $(t_i,\gamma_i)_{i\ge 0}$ be the vertices and slopes of the greatest convex minorant of $e$ on the interval $[0,x]$. Let $\tau(x,y)=\sup\{t\ge 0: y\in I^t(x)\}=\sup\{t\ge 0: x\in I^t(y)\}$. Let $i^\star=\sup\{i:t_i\le y\}$. Then, almost surely, $\tau(x,y)=\gamma_{i^\star}$ and $\kappa(x,y):=\inf I^{\tau(x,y)}(x)=t_{i^\star}$.
% \end{cor}
% }




% \begin{prop}\label{pro:dist_marked_fragmentation}

% \end{prop}


% subsection a_fragmentation_connected_to_brownian_motion (end)

\black

\subsection{The cut tree and the reconstruction problem} % (fold)
\label{sub:the_genealogy_and_the_marked_cut_tree}

The fragmentation we have presented in Section~\ref{sub:a_fragmentation_connected_to_brownian_motion} has a remarkable genealogy, which can be encoded into a \emph{cut tree} introduced by \citet{BeMi2013a}, and which turns out to be distributed like a Brownian continuum random tree. 

Let $(\zeta_i)_{i\ge 1}$ be i.i.d.\ uniform points in $[0,1]$, which are also independent of $(e,\bU)$. Almost surely, for all $i\ne j$, we have $i\sim_0j$. Then, for distinct $i,j\ge 1$ let $\tau_{ij}=\inf\{\tau\ge 0: \zeta_i \not \sim_\tau \zeta_j\}$ be the first time when $\zeta_i$ and $\zeta_j$ are separated by a point of $Z^{-\tau}$. Then, we define a function $\delta$ on $\N_0\times \N_0$ as follows:
\begin{equation}\label{eq:def_cut-tree}
  \delta(0,i)=\int_{0}^\infty |I^\tau(\zeta_i)| d\lambda 
  \qquad \text{and}\qquad
  \delta(i,j) = \int_{\tau_{ij}}^\infty |I^\tau(\zeta_i)|d\tau + \int_{\tau_{ij}}^\infty |I^\tau(\zeta_j)| d\tau \,,
\end{equation}
where $|\cdot|$ denotes the Lebesgue measure on $[0,1]$. 
It is known that $\delta$ defines a real tree \cite{BeMi2013a}: let $\sC$ denote the completion of $\N_0$ with respect to $\delta$, and let $\nu$ denote the weak limit of probability rescaled counting measure on $\{0, 1,2,\dots, n\}$; then $(\sC,\delta, \nu, 0)$ is a measured real tree rooted at $0$ that we call the \emph{cut tree}; $\N_0=\{0,1,2, \dots\}$ should be seen as a collection of marks in $\sC$. The measured tree $(\sC, \delta, \nu, 0)$ is distributed like a Brownian continuum random tree, and the collection of points $\N\subseteq \sC$ is an i.i.d.\ sequence with common distribution $\nu$ \cite{BeMi2013a,AdDiGo2019a,BrWa2017b}. 



% {\nic We let $i\wedge j$ denote point of the unique path between $i$ and $j$ in $\sC$ at minimum distance from $0$. Then 
% \begin{equation}\label{eq:def_branch-point_in_cut-tree}
%  \delta(i,i\wedge j) = \int_{\tau_{ij}}^\infty |I^\tau(V_i)| d\tau
%  \qquad \text{and} \qquad 
%   \delta(j,i\wedge j) = \int_{\tau_{ij}}^\infty |I^\tau(V_j)| d\tau 
%  \qquad \text{and} \qquad 
% \end{equation}
% }

For each $s\in [0,1]$, let $\Gamma_\tau(s):=\{i\in \N: \zeta_i \sim_\tau s\}$. Then, each $i\in \N$ is the image of $\zeta_i$ in the cut tree $\sC$ in the sense that $\Gamma_\tau(\zeta_i)$ converges in $\sC$ as $\tau\to \infty$ to the singleton $\{i\}$ (see \cite{AdDiGo2019a}). Every branch point of $\sC$ corresponds to a fragmentation event, just as reflected by the definition in \eqref{eq:def_cut-tree}. For any $i,j\in \N$, let $i\curlywedge j$ be the common ancestor of $i$ and $j$ in $\sC$, that is the point at distance 
\[\int_0^{\tau_{ij}} |I^\tau(\zeta_i)| d\tau=\int_0^{\tau_{ij}} |I^\tau(\zeta_j)|d\tau\]
from $0$ on the paths between $0$ and $i$, and between $0$ and $j$. Here, $i\curlywedge_\sC j$ corresponds to the (unique) fragmentation event that occurs at time $\tau_{ij}$, and that separates $\zeta_i$ from $\zeta_j$. Let $\sC_{i\curlywedge j}^i$ and $\sC_{i\curlywedge j}^j$ be the two subtrees of $\sC$ above the point $i\curlywedge j$ that contain respectively $i$ and $j$; then for every $k\in \N$ we have $\zeta_i\sim_{\tau_{ij}} \zeta_k$ precisely if $k\in \sC_{i\curlywedge j}^i$. Furthermore, the interval $I^{\tau_{ij-}}(\zeta_i)=I^{\tau_{ij-}}(\zeta_j)$ which contains all the $\zeta_k$ for which $\zeta_k\sim_{t} \zeta_i$ for all $i<\tau_{ij}$ splits into the two intervals $I^{\tau_{ij}}(\zeta_i)$ and $I^{\tau_{ij}}(\zeta_j)$ by the removal of the unique point of $Z^{-\tau_{ij}}$ lying in the interior of $I^{\tau_{ij-}}(\zeta_i)$. 

Observe that the cut tree is only constructed from the process of masses of the fragments containing a sequence of i.i.d.\ uniform points; this is crucial since the ``identities" of the fragments seen as subsets of $[0,1]$ retain some information (for instance, only neighbouring intervals can merge). More precisely, we can do so using only the process of masses, by exchangeability of $(\zeta_i)_{i\ge 1}$. 


If we see the fragmentation $(F(\tau))_{\tau\ge 0}$ from the point of view of Aldous and Pitman in \cite{AlPi1998a}, the cut tree is the genealogy of the fragmentation of a Brownian continuum random tree, and it is natural to try to ask whether one can recover the initial tree $(\sT,d,\mu)$ from $\mathfrak C = (\sC,\delta, \nu, 0)$, or if not, what minimal additional information is necessary. This question has been studied by \citet*{BrWa2017b} and \citet*{AdDiGo2019a} (see also~\cite{AdBrHo2014a} for a partial result). Quite naturally, since the cut tree is constructed from the process of masses only, the locations of the cuts are lost, and reconstruction is impossible without additional information. The main result of \cite{BrWa2017b,AdDiGo2019a} is that these locations is the only information that is lost, and that one can recover $(\sT,d,\mu)$ from $(\sC,\delta,\nu)$ plus this additional information. 


Since the fragmentation is binary, for every fragmentation event, there should correspond two points, one in each of the two fragments created. It turns out that these points can be given through their images in $\sC$: the additional information comes in the form of a countable collection of marks in the cut tree, and the only relevant information to us is its distribution conditionally on $(\sC,\delta,\nu)$. Let $\Br(\sC)$ denote the set of branch points of $\sC$. Almost surely, for each $b\in \Br(\sC)$, there are precisely three connected components to $\sC\setminus \{b\}$, and we denote by $\sC_b'$ and $\sC_b''$ the two which are not containing $0$, agreeing that $\nu(\sC_b')>\nu(\sC_b'')$. Let $\bV=\{(V_b',V_b''): b\in \Br(\sC)\}$ be an independent family of random variables such that, for each $b\in \Br(\sC)$, $(V_b',V_b'')$ has distribution 
\begin{equation}\label{eq:def_distribution_hook-points}
\frac{\nu(\cdot \cap \sC_b')}{\nu(\sC_b')} \otimes \frac{\nu(\cdot \cap \sC_b'')}{\nu(\sC_b'')}\,.
\end{equation}
The inverse cut tree transform then goes as follows: there exists a (measurable) map $\Phi$ that associates, to a pair $(\mathfrak C, \bV)$ a measured real tree that is distributed like a Brownian CRT. We will verify that $\CMT(\exc,\bU)$ turns out to be $\Phi(\mathfrak C, \bV)$ for a suitable collection $\bV$, but for now, let us describe the procedure if $\bV$ is given and has the distribution described above (this follows \cite{AdDiGo2019a}). 


For $i,j\in \N$, we can recursively identify a collection of branch points in $\sC$, which are meant to correspond to the cut points on the path between $\zeta_i$ and $\zeta_j$ in $\sT$. With this goal in mind, we now define a collection $(C_u, p_u^0, p_u^1)$, $u\in \cU_2$, where $C_u$ is a subtree of $\sC$, and $p_u^0,p_u^1\in C_u$. First set $C_\varnothing = \sC$ and let $p_\varnothing^0=i$, $p_\varnothing^1=j$. Then, given $(C_u, p_u^0, p_u^1)$, and writing $b=p_u^0\curlywedge p_u^1$, let $C_{u0}$ (resp.\ $C_{u1}$) be the one among $\sC_b'$ and $\sC_b''$ which contains $p_u^0$ (resp.\ $p_u^1$). Let $p_{u0}^0=p_u^0$, $p_{u1}^1=p_u^1$ and then define $p_{u0}^1$ (resp.\ $p_{u1}^0$) be the one of $V_b'$ and $V_b''$ that lies in $C_{u0}$ (resp.\ $C_{u1}$). Then for each $n\ge 0$ define 
\[Y_n(i,j)=\sqrt{\frac \pi 2} \cdot \sum_{|u|=n} \nu(C_u)^{1/2}\,.\]
% Then $Y_n(i,j)$ converges almost surely to a finite limit $Y(i,j)$, and the collection $(Y(i,j):i,j\in \N)$. More generally, for $u\in \cU$, one may define 
% \[Y_n^u=\sum_{|w|, u\le w} \mu(\sC_w)\,.\]
% Then $Y_n^u\to Y^u$ almost surely. 
Almost surely for all $i,j$, $Y_n(i,j)\to Y(i,j)$ as $n\to \infty$. 
Then the collection of random variables $(Y(i,j):i,j\in \N)$ has the same distribution as $(\delta(i,j), i,j\in \N)$. Seen as a matrix of pairwise distances, this defines uniquely an isometry class of a random compact real tree, which is a Brownian continuum random tree.  

% {\red 
% \begin{equation}\label{eq:def_cut-point-from_cut-tree}
%   Y^0(i,j) = \lim_{n\to\infty} \sqrt{\frac \pi 2} \cdot \sum_{|u|=n, 0\preceq u} \nu(C_u)^{1/2} 
%   \qquad \text{and} \qquad 
%   Y^1(i,j) = \lim_{n\to\infty} \sqrt{\frac \pi 2} \cdot \sum_{|u|=n, 1\preceq u} \nu(C_u)^{1/2}\,,
% \end{equation}
% are respectively the distances between $i$ and $j$ and the branch point $i\wedge j$; they are used to identify in $\sT$ the image of the branch point $i\wedge j$, to prove that we get the Poisson point process. 
% }

% \begin{alert}HERE BE A BIT CAREFUL: the law of the matrix identifies uniquely a random real tree (a probability measure on a GP-equivalence classes of compact real trees). The annealed measure is that of the CRT. However, the random tree that must be a measurable function of $e,\bU$. 
% \end{alert} 




% subsection the_genealogy_and_the_marked_cut_tree (end)

\subsection{The convex minorant tree as the inverse cut tree transform} % (fold)
\label{sub:distribution_CMT}

% \JF{On pourrait dire qu'en discret, on le voit directement par notre papier sur Prim, et aussi par le papier Minmin-Me}

From the previous considerations, proving that $\CMT(\exc,\bU)$ is indeed a Brownian CRT boils down to verifying that it can be seen as obtained from the inverse cut tree transform from $\mathfrak C$ using a certain collection of points that we will denote by $\{(\beta_b',\beta_b''), b\in \Br(\sC)\}$. Our collection is in part constructed as a measurable function of $\exc$ alone, and the main task consists in verifying that it has indeed the same distribution as $\bV$ defined above in \eqref{eq:def_distribution_hook-points}.

%The points $(\beta_b',\beta_b'')$, $b\in \Br(\sC)$, are constructed from points in $[0,1]$ using $(\exc,\bU)$. 
We start with a canonical exploration of the fragmentation. We construct a process $(S_u,\tau_u, \epsilon_u)_{u\in \cU}$ where $S_u$ is a half-open interval of $[0,1)$, $\tau_u$ is the unique time when there exists $x\in [0,1]$ such that $S_u=I^{\tau_u}(x)$ (that is the interior of $S_u$ is a connected component of $[0,1)\setminus Z^{-\tau_u}$); furthermore, writing $\ell_u=|S_u|$, $\epsilon_u$ is a continuous function on $[0,\ell_u]$ with $\epsilon_u(0)=\epsilon_u(\ell_u)=0$ and $\epsilon_u(r)>0$ on $[0,\ell_u]$. It will also be convenient to write $a_u=\inf S_u\in S_u$. The precise order in which the intervals and times are associated with the elements of $\cU$ is key to control the independence structure which turns out to be crucial. 

We first set $S_\varnothing = [0,1)$, $\tau_\varnothing =0$, and $\epsilon_\varnothing = \exc$; we then have $\ell_\varnothing=1$ and $a_\varnothing=0$. For $x\in [0,1)$ and $t\ge 0$, let $R_t(x)=\sup I^t(x)$. The process $(R_t(a_\varnothing))_{t\ge \tau_\varnothing}=(R_t(0))_{t\ge 0}$ has countably many negative jumps. We let $\ell_1>\ell_2>\dots\ge 0$ denote their ranked sizes (in absolute value); then $\sum_i \ell_i = 1$ almost surely. For each $i\ge 1$, we let $\tau_i$ be the unique $t\ge \tau_\varnothing=0$ with $R_{t-}(a_\varnothing)-R_t(a_\varnothing)=\ell_i$, and define $S_i=[R_{\tau_i}, R_{\tau_i-})$; one then has $a_i=R_{\tau_i}$. We then let $\epsilon_i:[0,\ell_i]\to \R_+$ be defined for $r\ge 0$ by 
\[\epsilon_i(r)=e^{\tau_i}(a_i+r) \I{0\le r\le \ell_i}\,.\]

Now, for each $u\in \cU$, given $S_u$ and $\tau_u$, let $(\tau_{ui},\ell_{ui})_{i\ge 1}$ denote the times $\tau_{ui}\ge \tau_u$ and sizes of the jumps of the process $(R_t(a_u))_{t\ge \tau_u}$ sorted in such a way that $\ell_{u1}>\ell_{u2}>\dots \ge 0$. Write $S_{ui}=[R_{\tau_{ui}}(a_u), R_{\tau_{ui}-}(a_u))$, $a_{ui}=\inf S_{ui}$, $\ell_{ui}=|S_{ui}|$ and define $\epsilon_{ui}:[0,\ell_{ui}]\to \R_+$ for $r\ge 0$ by 
\[\epsilon_{ui}(r)
=e^{\tau_{ui}}(a_{ui}+r) \I{r\le \ell_{ui}} 
= \epsilon_u^{\tau_{ui}-\tau_u}\Bigg(\sum_{j\ge 1} \ell_{uj} \I{a_{uj}<a_{ui}}+r\Bigg)\I{0\le r\le \ell_{ui}}\,.\]

Let $\cF_\varnothing$ be the sigma-algebra generated by $(R_t(0))_{t\ge 0}$. Then $(S_i, \ell_i, \tau_i, a_i)_{i\ge 1}$ is $\cF_\varnothing$-measurable while, conditionally on $\cF_\varnothing$, the $(\epsilon_i)_{i\ge 1}$ are independent Brownian excursions of durations $\ell_1>\ell_2>\dots \ge 0$. More generally, let $\cF_u$ be the sigma-algebra generated by $\{(R_t(a_v))_{t\ge \tau_v}, v\preceq u\}$. Then $(S_{vi},\ell_{vi}, \tau_{vi}, a_{vi})_{v\preceq u, i\ge 1}$ is $\cF_u$-measurable while, conditionally on $\cF_u$, the functions $(\epsilon_{ui})_{i\ge 1}$ are independent Brownian excursions of durations $\ell_{u1}>\ell_{u2}>\dots\ge 0$.

The recursive exploration $(S_u, \tau_u, \epsilon_u)_{u\in \cU}$ we have just defined yields a canonical recursive spinal decomposition of the cut tree $\sC$; by canonical we mean that the random points that are used are constructed from $\exc$ only. We say that a point $s\in [0,1]$ has an image $x\in \sC$ if $\Pi_t(s)=\overline{\{j\in \N:\zeta_j \sim_t s\}}$ decreases to the singleton $\{x\}$ as $t\to\infty$. We let $\sC_\varnothing=\sC$ and $b_\varnothing=0$. Working towards the definition of $(\beta_b',\beta''_b)$, $b\in \Br(\sC)$, we start by defining a collection $\eta_u$, $u\in \cU$. In the following, for $x,y\in \sC$, $\llb x,y\rrb_\sC$ denotes the range of the unique geodesic in $\sC$ between $x$ and $y$. 


\begin{lem}\label{lem:images_marks}With probability one, the points $(a_u)_{u\in \cU}$ have images in $\sC$ that we denote by $(\eta_u)_{u\in \cU}$. They are defined inductively and satisfy:
\begin{compactitem}[\textbullet]
  \item $\eta_\varnothing$ is the image of $a_\varnothing$;
  \item given $\sC_u$ and $\eta_u\in \sC_u$ the points $b_{ui}$ are the points of $\llb b_u, \eta_u\rrb_\sC$ at distance $\int_{\tau_u}^{\tau_{ui}} |I^t(a_u)|dt$ from $b_u$;
  \item $\sC_{ui}$ is the subtree of $\sC_u\setminus \{b_{ui}\}$ which contains neither $b_u$ nor $\eta_u$;
  \item $\eta_{ui}$ is the image of $a_{ui}$ in $\sC$, which turns out to be in $\sC_{ui}$.
\end{compactitem}
Furthermore, the family $(\eta_u)_{u\in \cU}$ is independent and for each $u\in \cU$, $\eta_u$ has distribution $\nu(\,\cdot \, \cap \sC_u)/\nu(\sC_u)$.
\end{lem}

% \begin{alert}
% Attention with the $\llbracket x,y\rrbracket$; we need something else since this is in the cut tree... $\boldsymbol [\!\boldsymbol[x,y \boldsymbol ]_\sC$ versus $[a,b]$, or $[[a,b]]$, $\langle x,y\rangle$
% except for $b_\varnothing$, the $b_u$ are all branch points; $b_{ui}$ is the common ancestor of $\eta_u$ and $\eta_{ui}$ 
  
% $\sC_u$ is the subtree of $\sC$ induced by the $\{i\in \N: \zeta_i\in S_u\}$ 

% $i\wedge_\sC j$
% \end{alert}

\begin{proof}The proof is by induction. It is proved in \cite{Bertoin2000a} that the process $(|I^t(0)|)_{t\ge 0}$ has the same distribution as the process $(|I^t(\zeta_1)|)_{t\ge 0}$. Therefore with $\Pi_t(0):=\overline{\{i\in \N: \zeta_i\in I^t(0)\}}$, we have almost surely $\sup\{\delta(i,j): i,j\in \Pi_t(0)\}\to 0$ as $t\to\infty $ so that there is a limit point that we denote by $\eta_\varnothing$ such that $\Pi_t(0)\to \{\eta_\varnothing$\}; by definition $\eta_\varnothing$ is the image of $a_\varnothing$ in $\sC$. It also follows that $\eta_\varnothing$ has distribution $\nu$ in $\sC$, since $1$, the image of $\zeta_1$ in $\sC$, does. The points $b_i$, $i\ge 1$, are precisely the branch points of $\sC$ along the segment $\llb 0,\eta_\varnothing\rrb_\sC$, sorted in the decreasing order of the masses $\nu(\sC_i)=\ell_i$ of the subtrees of $\sC$ hanging from the segment. 

Observe now that for $u\in \cU$ and $i\ge 1$, conditionally on $\cF_u$, the process $(I^{\tau_{ui}+t}(a_{ui}))_{t\ge 0}$ is precisely the process of masses of the fragment containing $0$ in the fragmentation of the excursion $\epsilon_{ui}$. As a consequence, the image $\eta_{ui}$ of $a_{ui}$ is well-defined. Furthermore, the distribution of $\eta_{ui}$ is the rescaled mass measure $\nu$ in the image of $S_{ui}$ in $\sC$, which is precisely $\sC_{ui}$. Finally, conditionally on $\cF_u$, the functions $(\epsilon_{u_i})_{i\ge 1}$ are independent, and so are the $(\eta_{ui})_{i\ge1}$: for any collection of bounded continuous functionals $(f_i)_{i\ge1}$, we have
\[\bE\Bigg[\prod_{i\ge 1} f_i(\eta_{ui})~\bigg|~\cF_u\Bigg]=\prod_{i\ge 1} \int_{\sC_{ui}} f_i(x_i) \frac{\nu(dx_i)}{\nu(\sC_{ui})}\,.\]
The claim follows by induction. 
\end{proof}
\begin{rem}Observe that, $\sC_u$ is the complete subtree of $\sC$ induced by $\{i\in \N: \zeta_i\in S_u\}$. Furthermore, except for $b_\varnothing$, the $b_u$ are all branch points in $\sC$; more precisely $b_{ui}$ is the common ancestor of $\eta_u$ and $\eta_{ui}$, namely $b_{ui}=\eta_u \curlywedge \eta_{ui}$.
\end{rem}

The collection $(\eta_u)_{u\in \cU}$ only provides part of the marks we shall need in the cut tree $\sC$. The remaining marks are the images of the random points constructed using the sequence of uniform random variables $\bU$, and which are associated to the local minima of $\exc$. 

\begin{lem}\label{lem:vertices_branchpoints}There is a one-to-one correspondence between the local minima of $\exc$ and the branch points of the cut tree $\sC$: every branch point of $\sC$ is of the form $\eta_u\curlywedge \eta_{ui}$ for $u\in \cU$ and $i\ge 1$, and the corresponding local minimum is $a_{ui}$.
\end{lem}
\begin{proof}
For each $t\in \sL(e)$, let $t_0=0<t_1<t_2< \dots <t_i=t$ be the vertices of the convex minorant of $\exc$ on the interval $[0,t]$. Let $z_0=1>z_1>\dots> z_k$ and $\gamma_1<\gamma_2<\dots< \gamma_k$ be the corresponding intercepts and slopes. Then, at time $\gamma_k$ the interval $[t_{k-1},z_k)$ is split into the pair $[t_{k-1},t_k)$, $[t_k,z_k)$. Let $j_1=\inf\{j\in \N: \zeta_{j_1}\in [t_{k-1},t_k)\}$ and $j_2=\inf\{j\in \N: \zeta_{j_2}\in [t_k,z_k)\}$. To make the correspondence more explicit, we exhibit the two points $\eta_u$ and $\eta_{ui}$ in $\sC$ such that the branch point corresponding to $t$ is $\eta_u\curlywedge \eta_{ui}=j_1 \curlywedge j_2$. It shall be noted that the branch point corresponding to $t=t_i$ is not the image of $t$ in the cut tree $\sC$, the latter being almost surely the leaf $\eta_{ui}$ that we will exhibit. The path to follow in $\cU$ is given by the convex minorant. Let $i_1\ge 1$ be the unique index such that $\tau_{i_1}=\gamma_1$; then, let $i_2$ be the unique index such that $\tau_{i_1i_2}=\gamma_2$, and so on which yields a point $u=i_1i_2\dots i_{k-1}$ with $\tau_u=\gamma_{k-1}$, and $a_u=t_{k-1}$. Finally, let $i$ be the unique index such that $\tau_{ui}=\gamma_k$; then we have $a_{u}=t_k=t$ while $z_k=R_{\tau_{ui}-}$. The images $\eta_u$ and $\eta_{ui}$ of $t_u$ and $t_{ui}$ in $\sC$ are such that $\eta_u \curlywedge \eta_{ui}$ is the branch point $j_1\curlywedge j_2$.
  
Conversely, the sequence of sets $\{\llb 0, \eta_u\rrb_\sC: |u|\le n\}$, $n\ge 1$, increases to $\sC$ and thus exhausts all the branch points. In particular, every branch point $b$ of $\sC$ is of the form $\eta_u\curlywedge \eta_{ui}$ for some $u\in \cU$ and $i\ge 1$. Now, for such a branch point, $a_{ui}\in [0,1]$ is the local minimum of $\exc$ that separates the points from $[a_u,a_{ui})$ from $S_{ui}$ at time $\tau_{ui}$. 
\end{proof}

Finally, we complete the definition of the set of marks in the cut tree $\sC$. Consider a branch point $b$ of $\sC$; by Lemma~\ref{lem:vertices_branchpoints}, it is of the form $\eta_u\curlywedge \eta_{ui}$ for some $(u,i)\in \cU\times \N$ and $a_{ui}$ is the corresponding local minimum. Recall now the join point $\ju(a_{ui})$ associated to $a_{ui}\in \sL$ (Remark~\ref{rem:points_absolute} on page~\pageref{rem:points_absolute}). Observe that, by construction, at time $\tau_{ui}$, the two intervals that get separated are $S_{ui}$ to the right, and $[a_u,a_{ui})$, to the left. The subtree of $\sC$ above the branch point $b$ is therefore the completion of $\{i\in \N: \zeta_i\in [a_u, \sup S_{ui})\}$, and two intervals $[a_u,a_{ui})$ and $S_{ui}$ correspond to the two subtrees of $\sC$ above the branch point $b$, that we previously denoted by $\sC_b'$ and $\sC_b''$. 

Recall that $\cF_u$ is the sigma-algebra generated by $\{(R_t(a_v))_{t\ge \tau_v}, v\preceq u\}$, and that, as a consequence, $a_u, a_{ui}$ and $S_{ui}$ are $\cF_u$-measurable. By Lemma~\ref{lem:images_marks}, conditionally on $\cF_u$, the image $\eta_{ui}$ of $a_{ui}$ in $\sC$ is distributed like $\nu(\cdot \cap \sC_{ui})/\nu(\sC_{ui})$. Let $(\beta_b',\beta_b'')\in \sC_b'\times \sC_b''$ be the pair of points formed by $\eta_{ui}$ and the image of $\ju(a_{ui})$ in $\sC$ (which a.s.\ exists since $\ju(a_{ui})$ is uniform in $[a_u,a_{ui})$). Then, conditionally on $\cF_u$, and by Lemma~\ref{lem:images_marks}, the collection $(\beta_b',\beta_b'')$, $b\in \Br(\sC)$ has the same distribution as $\bV$:

\begin{lem}\label{lem:full_dist_marked_cuttree}
The marked cut tree $(\sC, \{\beta_b',\beta_b'': b\in \Br(\sC))$ is such that:
\begin{compactenum}[i)]
    \item $\{(\beta_b',\beta_b''): b\in \Br(\sC)\}$ are independent conditionally on $\sC$, and 
    \item for each $b$, $(\beta_b',\beta_b'')$ are independent random variables with distribution $\nu(\,\cdot\, \cap \sC'_b)/\nu(\sC_b')\otimes \nu(\,\cdot\, \cap \sC_b'')/\nu(\sC''_b)$.
\end{compactenum}
\end{lem}

% \begin{alert}The previous lemma says that the marks have the correct distribution. Remains to check that the reconstruction is the same, that is that with probability one, $\CMT(e,\bU)$ equals $\Phi(\sC,\bV)$. This is done by identifying the distance between any two of the points $\zeta_i,\zeta_j$ as coming from the routing between $i$ and $j$. 
% \end{alert}

% The question of the reconstruction of the tree that is fragmented in the Aldous--Pitman fragmentation has been studied by \citet{BrWa2017b} and \citet{AdDiGo2019a}. What is meant here by reconstruction is the following: the cut tree $\sC$ alone does not contain all the necessary information, and the question is to encode some additional information that is sufficient. This additional information comes in the form of marks at all the branch points of the cut tree. 


% What matters to identify the law of $\sT(e,\bU)$ as that of the Brownian CRT is the distribution of these marks. The fact that the marks have the correct distribution is straightforward from Lemma~\ref{lem:images_marks}, and we will mostly focus on describing the construction to show that the tree we construct is indeed the one obtained from the inverse cut tree transform from the cut tree $\sC$ and the marks $\eta_u$.

% {\red These marks allow to define the routing; for each $i,j\in \N$, $i\ne j$, the points needed to recover the path between $\zeta_i$ and $\zeta_j$ in the original tree; just need to verify that for each $i,j$, we have the distance that is defined is almost surely the same so that the inverse cut tree and our tree are almost surely isometric. Will need Proposition~12 of \cite{AdDiGo2019a} which identifies the law of the marks in the cut tree in order to reconstruct; Theorem~16 from \cite{AdDiGo2019a} states that the "routing metric" on $\N$ is the same as the distance in the original tree between $\zeta_i$ and $\zeta_j$, also with the branch points.

% \begin{prop}\label{pro:CMTisInverseCut}With probability one, for every $i,j\in \N$, $d(\zeta_i,\zeta_j)=d_\sC(i,j)$. As a consequence, 
% $\sT(e,\bU)$ is a Brownian continuum random tree.
% \end{prop}

% Also transport the branch points of $\sC$ together with the corresponding times; then the pair tree/collection of (point,time) is distributed like a Brownian CRT together with a Poisson point process of intensity... and therefore this provides a coupling between the Aldous--Pitman fragmentation and the representation by Bertoin as a fragmentation related to Brownian motion.

% For any $x\in \sC$, let $\sC_x$ denote the set of points of $\sC$ above $x$, that is $\{z\in \sC: x\in \llb 0,z\rrb\}$. When $x$ is a branch point and $y\in \sC_x$, we let $\sC_x^y$ denote the subtree above the point $x$ that contains $y$: $\sC_x^y:=\{z\in \sC_x: x\not\in \llb y,z\rrb \}$.

% For any $i,j\in \N$, we have 
% \[\tau_{ij}=\int_{\llb 0, i\wedge j\rrb} \frac{l(dz)}{\nu(\sC_z)}\,.\]
% }

The points $(\beta'_b,\beta''_b)$, $b\in \sC$, now being defined, we are ready to verify that $\Phi(\mathfrak C, \bbeta)$ and $\CMT(\exc,\bU)$ are almost surely isometric. The arguments above should already make this pretty clear: indeed, for each $b\in \Br(\sC)$, the set of marks $\{\beta_b',\beta_b''\}$ is precisely the image in $\sC$ the set of points $\{\ju(a_{ui}), a_{ui}\}$ which are identified at time $\tau_{ui}$. To make this formal, fix any $i,j\in \N$, and consider $Y_n(i,j)$ and $\hat d_n(\zeta_i,\zeta_j)$. The choice of the marks $(\beta_b',\beta''_b)$, $b\in \Br(\sC)$, is precisely made so that, for every $n\ge 1$, sorting the sets $\{\nu(C_u), |u|=n\}$ and $\{|\Pi_u|, |u|=n\}$ in decreasing order yields the same sequence, and therefore
\[Y_n(i,j)= \sqrt{\frac \pi 2} \sum_{|u|=n} \nu(C_u)^{1/2} = \sqrt{\frac \pi 2}\sum_{|u|=n} |\Pi_u|^{1/2} = \hat d_n(\zeta_i,\zeta_j)=d_n(\zeta_i,\zeta_j)\,,\]
where the last step follows from Proposition~\ref{pro:distance_pairs}.
Taking the limit as $n\to\infty$, this implies that, for each $k\ge 1$ the metric spaces $(\{\zeta_i, 1\le i\le k\}, d)$ and $([k], Y)$ are isometric (with the correspondence $(i,\zeta_i)$, $i\in [k]$). Since $(\{\zeta_i, 1\le i\le k\},d)$ increases to $\CMT(\exc,\bU)$ (Proposition~\ref{pro:length_measure}), the claim follows by taking the limit as $k\to\infty$.  

% \begin{alert}
% One is an equivalence class, the other $\CMT(e,\bU)$ we want to be a metric space... should be introduce $[\CMT(e,\bU)]$ for the equivalence class ?

% Maybe put somewhere that, with probability one, $\CMT(e,\bU)$ is the completion of $(\{\xi_i, i\in \N\}, d)$. 
% \end{alert}

% % subsection the_inverse_cut_tree_transform (end)
% \red
% \subsection{The Aldous--Pitman fragmentation and Bertoin's representation} % (fold)
% \label{sub:the_aldous_pitman_fragmentation_and_bertoin_s_representation}

% Our main objective in this section is to prove the following theorem:

% \begin{thm}\label{thm:CRT}Let $e$ be a standard Brownian excursion and $\bU=(U_i)_{i\ge 1}$ be an independent collection of i.i.d.\ random variables uniform on $[0,1]$. Then, $\sT(e,\bU)$ is a Brownian CRT. 
% \end{thm}

% This identifies the main building block for the constructions to come, but the proof also sheds new light on the relationship between the Aldous--Pitman fragmentation of a Brownian continuum random tree \cite{AlPi1998a}, Bertoin's representation by drifting a Brownian excursion \cite{Bertoin2000a} via the cut tree and the inverse cut tree transform \cite{BrWa2017b,AdDiGo2019a}. 

% {\nic 
% \begin{lem}\label{lem:vertices_local-min}Almost surely, the set of vertices of all the convex minorant of $e$ on $[0,x]$ for $x\in [0,1]$ coincide with the local minima of $e$.
% \end{lem}
% \begin{proof}Clearly, any local minimum $t\in \sL$ is also some vertex of some convex minorant. To see this, consider the convex minorant on $[0,t]$, then, by Lemma~\ref{lem:no_exception_convex}, the vertices are $t_0=0<t_1<\dots<t_k=t$ for some natural number $k$. For any $x\in (t,z_k)$ the point $t=t_k$ is a vertex of the convex minorant on $[0,x]$ by {\red Lemma XXX that has been removed (saying that if $y\in (t_k,z_k)$, then $t_k\in \cV_y$.)}. Conversely, for any vertex $t_i>0$ of the convex minorant on $[0,x]$ for some $x\in [0,1]$, we have $e(s)\ge e(t_i)+\gamma_{i-1}(s-t_i)$ for all $s$ in some neighborhood of $t_i$, and with probability one one also has $e(s)\ge e(t_i)$ on some (potentially smaller) neighborhood of $t_i$. {\red We conclude since the collection of vertices is countable; WELL small issue: we have that the local minima are countable...}
% \end{proof}
% }

Finally, we are ready to prove Theorem~\ref{thm:additive_coalescent} which shows that the convex minorant tree provides a coupling between the two classical constructions of the additive coalescent by Aldous \& Pitman \cite{AlPi1998a} on the one hand, and Bertoin \cite{Bertoin2000a} on the other. Let $\cP=\{(\pi(x),\slo(x)): x\in \sL(e)\}$. 

% For any $x\in (0,1]$, consider the vertices of the convex minorant of $e$ on $[0,x]$ and let $\lambda_x$ denote the supremum of the slopes. In particular, if $x=t_i$ is a vertex of the convex minorant, then $\lambda_x$ is the slope of the face on $[t_{i-1},t_i]$ and $\lambda_x<\infty$. Furthermore, $\lambda_x<\infty$ if and only if $x$ is the vertex of some convex minorant, or equivalently by Lemma~\ref{lem:vertices_local-min}, if $x\in \sL$.

% Consider the collection of points $\cP=\{(\pi(x),\lambda_x): \lambda_x<\infty\}$ which is a countable subset of $\sT\times \R_+$ by the observation above. Let $\cP_t=\{\pi(x):\lambda_x\le -t\}\subseteq \sT$. The following proposition provides a coupling between two famous representations of the fragmentation dual to the additive coalescent, the one by Aldous-Pitman \cite{AlPi1998a} based on the logging of the Brownian continuum random tree by a Poisson point process of intensity the length measure, and the one by Bertoin \cite{Bertoin2000a}. 
% For each $\lambda\ge 0$, let $\bgamma^\lambda=(\gamma^\lambda_i)_{i\ge 1}$ be the sequence of lengths of the intervals of $[0,1]\setminus \cP_t$. 

% \begin{alert}Better define $\cP=\{(\pi(x),-\gamma_x): x\in \sL(e)\}$, where $\gamma_x$ is the slope of the face just before $x\in \sL$, which exists by Lemma~\ref{lem:no_exception_convex}. Then we simply need to verify that, almost surely for every $i,j\in \N$, we have $\gamma_{\zeta_i\wedge \zeta_j} = \tau_{ij}$ where
% \[
% \tau_{ij}=\int_{\llb \rho, i\wedge j\rrb} \frac 1 {\nu(\sC_z)} dz\,,
% \]
% and the point in $\llb \zeta_i,\zeta_j\rrb$ at distance $\hat d(i,i\wedge j)$ from $\zeta_i$ is $\zeta_i\wedge \zeta_j$. Then, Theorem~16 of \cite{AdDiGo2019a} shows that $(\sT, \cP)$ has the distribution of a Brownian CRT together with a Poisson point process.
  
% \end{alert}

% \begin{thm}[Aldous--Pitman and Bertoin]\label{thm:additive_coalescent_v2}
% Conditionally on the tree $\frak T=(\sT,d,\mu,0)$:
% \begin{compactenum}[i)]
%   \item the collection of points $\cP$ is a Poisson point process with intensity measure $\ell \otimes dt$ on $\sT\times \R_+$,
%   \item for any $t\ge0$, $\bgamma^{-t}$ is the sequences of sorted lengths of intervals of $[0,1]\setminus Z^{-t}$, where $Z^\lambda:=\{t\in [0,1]: \Xi_\lambda[e]=0\}\}$; thus $(\bgamma^{-t})_{t\ge 0}$ is the process constructed by Bertoin in \cite{Bertoin2000a} from $e$, and
%   \item for any $t\ge 0$, the masses of the connected components of $\sT\setminus \cP_t$ are given by $\bgamma^{-t}$, thus $(\bgamma^{-t})_{t\ge 0}$ is distributed like the process constructed by Aldous--Pitman in \cite{AlPi1998a} from $(\sT,d)$.
% \end{compactenum}
% \end{thm}
\begin{proof}[Proof of Theorem~\ref{thm:additive_coalescent}]Observe that, by Lemma~\ref{lem:vertices_branchpoints}, with probability one, all the local minima of $\exc$ are of the form $\eta_u\curlywedge \eta_{ui}$ defined in Lemma~\ref{lem:cut_point-time}, and therefore, almost surely, $\cP=\{(\pi(a_{ui}), \tau_{ui}): u\in \cU,i\in \N\}$. Since, a.s.\ for all $u\in \cU$, we have $\ell_u>0$, this can equivalently be put as $\cP=\{(\pi(\kappa(\zeta_i,\zeta_j)), \tau(\zeta_i,\zeta_j)): i\ne j \in \N\}$. From there, the claim is an easy consequence of Theorem~16, and Corollaries~17-18 of \cite{AdDiGo2019a} (it is even simpler since we do not need to infer the $\tau(\zeta_i,\zeta_j)$ from the cut tree, they can be read directly from the fragmentation). 
% Consider any $(u,i)\in \cU\times \N$. By construction, $a_{ui}$ is a point on the segment $\llb \zeta_i,\zeta_j\rrb$, and it lies at distance $\delta_C(i, i\wedge j)$ from $i$. Furthermore, by Lemma~\ref{lem:cut_point-time}, $\tau(\zeta_i,\zeta_j)$ is the slope of the last face of the convex minorant of $\exc$ on $[0,\kappa(\zeta_i,\zeta_j)]$. By Theorem~16, and Corollaries~17-18 of \cite{AdDiGo2019a}, it follows that $\cP$ is a Poisson point process with intensity $\ell\otimes dt$ on $\Skel(\sT)\times \R_+$. The fact that the masses of $\sT\setminus \cP_t$ are precisely the lengths of the intervals of $[0,1]\setminus Z^{-t}$ is clear by construction. 
%  \nic This is the only part that is not clear from the construction. The idea for the proof is the following: the times are correct because of the coupling with the cut tree. Now, for the positions of the cut, one must prove that independently of the times, they are independent uniform. This can be done on any interval of the tree, and then prove some independence between disjoint intervals. For two uniform leaves, this is a simple calculation: take $x,y$ independent uniform random variables in $[0,1]$, then at the time of the split, we have two subexcursions of masses $m_1$ and $m_2$ that are components of some Dirichlet$(\tfrac 12, \tfrac 12, \tfrac 12)$. The distance of the two subsegments are $\sqrt{m_1} R_1$ and $\sqrt{m_2} R_2$ for two independent Rayleigh random variables. So we are interested in the distribution of 
% \[\frac{\sqrt{m_1}R_1}{\sqrt{m_1}R_1 + \sqrt{m_2}R_2}\]
% conditionally on $\sqrt{m_1}R_1 + \sqrt{m_2}R_2$. A simple calculation shows that $\sqrt{m_1}R_1$ is exponential with rate $\sqrt 2$. By independence, the result follows for the location of the first point on $\llb x,y\rrb$. {\red From this, one should be able to show that (1) if we wait longer and have $k$ points, then they are independent uniforms, and (2) that this extends to intervals with end points on the skeleton. I am confident that this is not too bad, but I do not have all the arguments yet.} 
\end{proof}


% subsection the_aldous_pitman_fragmentation_and_bertoin_s_representation (end)

% subsection the_recursive_convex_minorant (end)