%!TEX root = MST_brownian.tex

\begin{lem}\label{lem:properties_Z}Let $\omega\in \cC(\R_+, \R)$. Then
\begin{compactenum}[i)]
  \item the process $(Z^\lambda(\omega))_{\lambda \in \R}$ non-increasing and right-continuous with left-limits;
	\item for every $\lambda \in \R$, $Z^\lambda(\omega)$ and $Z^{\lambda-}$ are both closed;
	\item the set $\{\lambda \in \R: Z^{\lambda-}(\omega)\setminus Z^\lambda(\omega) \ne \varnothing\}$ is countable.
\end{compactenum}
\end{lem}
\begin{proof}
\emph{i)} The monotony is a consequence of Lemma~\ref{lem:shear_composition}, this implies the existence of the left and right limits $\cap_{h>0} Z^{\lambda-h}$ and $\cup_{h>0} Z^{\lambda+h}$, respectively. The right-continuity follows by continuity of the maps $\lambda\mapsto \omega^\lambda$ and $\underline \omega^\lambda$: if $s\in Z^{\lambda-h}$ for all $h>0$, then $\omega(s)+(\lambda-h) s = \inf\{\omega(r)+(\lambda-h)r: 0\le r\le s\}$ for all $h>0$, and thus this also holds for $h=0$. \emph{ii)} The fact that $Z^\lambda$ is closed is an easy consequence of the continuity of $\omega$. The monotony shows that $Z^{\lambda-}$ is a decreasing limit of closed sets, and is thus closed. \emph{iii)} Since $Z^\lambda$ and $Z^{\lambda-}$ are both closed for every $\lambda\in\R$, if $\lambda$ is such that $Z^{\lambda-}\setminus Z^\lambda\ne \varnothing$, then there exists $\epsilon>0$ and $x=x_\lambda\in Z^{\lambda-}$ with $d(x, Z^\lambda)>\epsilon$.  It follows that 
\[\{\lambda \in \R: Z^{\lambda-}\setminus Z^{\lambda}\ne \varnothing\} = \bigcup_{n\ge 1} \{\lambda\in \R: \dH(Z^{\lambda-}, Z^\lambda)>1/n\}\,.\]
For each $n\ge 1$, there must exist for each $\lambda$ a ball of radius $1/n$, and the collection of these balls must be disjoint. 
For each $n\ge 1$, any collection of open balls of radius $1/n$ must be countable, and therefore any set in the right-hand side above is countable. The claim follows.
\end{proof}

\begin{lem}[Continuity properties of the metric $d$]\label{lem:continuity_distance}Let $d(\cdot,\cdot)$ be the pseudo-metric on $[0,1]$ defined from the pair $(\exc,\bU)$ used in the construction of $\CMT(\exc, \bU)$. Almost surely, 
\begin{compactenum}[i)]
    \item the map $d(0, \cdot)$ is continuous almost everywhere, but 
    \item for every $x\in \sL(\exc)\cap (0,1)$, the map $d(0,\cdot)$ is not left-continuous at $x$, and
    \item for every $x\in \sL(\exc)\cap (0,1)$, the map $d(0,\cdot)$ is neither left- nor right-continuous at $\ri(x)$.
\end{compactenum}
\end{lem}
\begin{proof}
\emph{i)} Let $x$ be uniformly random in $[0,1]$, then a.s.\ the vertices of the convex minorant of $\exc$ on $[0,x]$, $(t_i(x))_{i\ge 0}$ and the corresponding intercepts $(z_i(x))_{i\ge 0}$ are such that $t_i(x)< x < z_i(x)$. 
Furthermore, there exists a sequence of local minima $z_n>x$ with $z_n \downarrow x$ such that the vertices of the convex minorant of $\exc$ on $[0,z_n]$ are precisely $\cV_{z_n}=\cV_x \cup \{z_n\}$. 
Then, for any $i\ge 1$, $\sup\{d(x,t): t\in (t_i,z_i)\} \le D_i \cdot |t_i-z_i|^{1/2}$, where $(D_i)_{i\ge 1}$ are random variables distributed like the diameter of a continuum random tree of unit mass (which are not independent). It follows that 
\begin{align*}
\pc{\sup\{d(x,t): t\in (t_i,z_i)\} > |z_i-t_i|^{1/4}} 
&\le \pc{D_i> |z_i-t_i|^{-1/4}} \\
&\le \exp(-|z_i-t_i|^{-1/4}/2v)\,,
\end{align*}
for some constant $v>0$. It follows by the Borel--Cantelli Lemma that a.s.\ $\sup\{d(x,t): t\in (t_i,z_i)\}\le |z_i-t_i|^{1/4}$ for all but finitely many values of $i$, so that $|d(0,t)-d(0,x)|\le d(x,t) \to 0$ as $t\to x$. 

\emph{ii)} 
Let $x\in \sL\cap (0,1)$; then a.s. there are only finitely many vertices in $\cV_x$, and $x=t_i(x)$ for some $i\ge 1$. Let $z_n$ be a sequence of local minima with $z_n \in (t_{i-1},t_i)$ and $z_n\uparrow t_i$ as $n\to\infty$. Then, for all $n_0$ large enough, the vertices of the convex minorant of $\exc$ on $[0,z_n]$ are exactly $\{t_j, j<i\} \cup \{z_n\}$. For each $n\ge n_0$, the point $\ju(z_n)$ is uniform in $(t_{i-1}, z_n)$ and $\ju(x)$ is uniform in $(t_{i-1},t_i)$. With probability one, there exists a subsequence $(n_j)_{j\ge 1}$ such that $0<\ju(z_{n_j})-t_{i-1} < \tfrac 12 (\ju(x)-t_{i-1})$. In particular, since $\llb 0,x\rrb$ a.s.\ has an accumulation point at $\ju(x)$, we have $\sup d(0,z_{n_j}) = \sup d(0,\ju(z_{n_j})) < d(0,\ju(x)) = d(0,x)$. It follows that, for any $\epsilon>0$, $\inf\{d(0,s): s\in(x-\epsilon, x)\} < d(0,x)$.   

\emph{iii)} For $x\in \sL \cap (0,1)$, the point $\ri(x)$ is some intercept, and the proof that $d(0,\cdot)$ is not left-continuous at $\ri(x)$ is the same as in \emph{ii)}. For the lack of right-continuity at $\ri(x)$, this is also similar, but relies on the fact that one may find a sequence of local minima $z_n$ in $(\ri(x),1)$ with $z_n \downarrow \ri(x)$ such that $\cV_{z_n}=\cV_x \cup \{z_n\}$. The same argument as above can then be used by considering the random points $\ju(z_n)$, which are independent, and uniform in $[x,z_n]$.
\end{proof}


\begin{lem}[Surplus and area under the curve]\label{lem:area} Let $\exc$ be a Brownian excursion. 
Consider the subset $D$ of $[0,1]^2\times \R$ of points $(x,y,\lambda)$ such that $[x,y]\cap Z^\lambda(\exc)= \varnothing$. Then, the 3-dimensional volume of $D$ is equal to $\int_0^1 \exc(x) dx$.
\end{lem}
%Eventuellement dire que l'argument n'utilise pas la loi de ${\sf e}$, mais des propriétés

\begin{proof}Recall the recursive decomposition of  Section~\ref{sub:distribution_CMT}. Then, the set $D$ can be decomposed into countably many portions (with disjoint interior) $D_u$, $u\in \cU$, as follows: 
\[D_u:= \{(x,y,\lambda): a_u\le x<y< R_{-\lambda}(a_u), \lambda\le -\tau_u\}\,.\]
There is a corresponding decomposition of the set $\{(s,t): s\in [0,1], 0\le t\le \exc(s)\}$ also into portions with disjoint interior, $E_u=\{(s,\exc(a_u)-\lambda (s-a_u)): a_u\le s\le R_{-\lambda}(a_u), -\lambda\ge \tau_u\}$, for $u\in \cU$. We show that, for each $u\in \cU$, 
\[\int_{E_u} dsdt = \int_{D_u} dxdyd\lambda\,.\]
We treat the case $u=\varnothing$, the others are just the same, up to the more complicated notation. First observe that the left-hand side above with $u=\varnothing$ is precisely the area under the function $f$ given by, for $i\ge 1$, 
\[f(s)=\exc(a_i)+ \tau_i (s-a_i)\qquad a_i=R_{\tau_i}\le s<R_{\tau_i-}\,.\]
Now, since each point $z=(s, f(s))$ can be represented in polar coordinates as $z=\rho(\theta) e^{i\theta}$, or alternatively by the pair $(-\lambda, R_{-\lambda}(0))$, where $-\lambda=f(s)/s$ is the slope of the line from $0$ to $z$, we have 
\[\int_{E_\varnothing} dsdt = \int_0^1 f(s)ds = \frac 1 2 \int_0^{\pi/2} \rho(\theta)^2 d\theta = \int_{D_\varnothing} dxdyd\lambda\,.\]
The claim follows by summing the contributions for $u\in \cU$. 
% We only give the main lines, and leave the details to the reader.\par
%   Observe Fig. \ref{fig:area-volume}. 
%   Consider a continuous function $f:[0,1]\to \R^+$.
%   Each point $z = (x,f(x))$ of the curve can be represented in polar coordinate $z= H(\theta)\exp(i \theta)$, or, alternatively, by a pair $(\lambda, L(\lambda))$ where $\lambda = f(x)/x$ is the slope of the line from 0 to $z$, and $L(\lambda)=x$.\par
%   The first function $f$ on Fig. \ref{fig:area-volume} has the following property ${\cal P}$: the union of the segments $[0,z]$ for $z$ on the curve is equals to the set of points below the curve. For such a function $f$, $Z_\lambda(f)=\{L(\lambda)\}$. In this situation, we have
%   \[\int_0^1 f(s) ds =\frac{1}2 \int_0^{\pi/2} H(\theta)^2\, d\theta= \frac{1}2 \int_0^\infty L^2(\lambda) d\lambda=\int 1_{x<y}1_{[x,y]\cap Z_\lambda=\varnothing} dx\,dy\,d\lambda.\]\par
%   On the second figure, is represented an irregular function, which does not satisfy ${\cal P}$. Consider the subset $V_{(0,0)}$ of points $z'$ below the curve that belongs to some segment $[0,z]$ with  $z=(x,f(x))$ and such that $[0,z]$ is entirely below the curve (these $z'$ are elements of one of the 5 triangles represented in the figure).
%   The set $V_{(0,0)}$ is the set of points under the curve of $f$, visible from $(0,0)$.  On this example, it can be decomposed in 5 triangles, these triangles being delimitated by the 4 local minima visible from $(0,0)$, between $6$ slopes, $\lambda_0=0<\lambda_1<\cdots \lambda_6$; the area corresponding to these $z'$ can be computed as explained above, and it corresponds to $\int 1_{x<y}1_{[x,y]\cap Z_\lambda=\varnothing} 1_{[0,y]\cap Z_\lambda=\varnothing}  dx\,dy\,d\lambda$ where this last condition $[0,y]\cap Z_\lambda=\varnothing$ corresponds to the condition ``to be visible from 0''.
%   The points that are not visible from 0, can be seen by watching from the local minima, above the slopes represented in the picture, on which the construction can be iterated (one then, take the set of points visible from a local minima $(x_j,f(x_j))$ above the slope between $(0,0)$ and $(x_j,f(x_j))$, and whose abscissa corresponds to $[x_j,x_{j+1}]$.
  % % Figure environment removed
\end{proof}