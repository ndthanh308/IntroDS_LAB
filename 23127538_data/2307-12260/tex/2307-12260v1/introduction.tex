%!TEX root = MST_brownian.tex


\subsection{Main results}

For a connected graph $G=(V,E)$, together with distinct positive weights associated to the edges, the minimum weight spanning tree is the unique connected spanning subgraph of $G$ that minimizes the total sum of the edge weights. 
The classical random model consists in taking the complete graph on $[n]:=\{1,2,\dots, n\}$ and independent and identically distributed (i.i.d.) random weights $w_e$, $e\in \binom{[n]}2$, uniform on $[0,1]$. 
%\JF{je pense qu'il faudrait prendre autre chose comme notation que $W_e$: plus loin, $W$ est un browien, $e$ apparaît comme une excursion, et une exponentielle. Ça fait trop... $w_e$? $u_e$?}
Then let $M_n$ denote the corresponding minimum spanning tree (MST) rooted at $\rho_n=1$. It has been proved by \citet*{AdBrGoMi2013a} that, seen as a metric space, $M_n$ admits a scaling limit in the following sense: Let $d_n$ be the graph distance on $M_n$, 
%meaning that $d_n(u,v)$ is the number of edges on the unique path between $u$ and $v$ in $M_n$; 
let $\mu_n$ be the counting measure on $[n]$. Then, there exists a (non-trivial) compact measured metric space $(\sM,d)$, a point $\rho\in \sM$, and a Borel probability measure $\mu$ on $(\sM,d)$ such that
\begin{equation}\label{eq:scaling_limit_mst}
  (M_n, n^{-1/3} d_n, n^{-1}\mu_n, \rho_n) \xrightarrow[n\to\infty]{} (\sM,d,\mu, \rho)\,  
\end{equation}
in distribution, in the sense of Gromov--Hausdorff--Prokhorov. The main result of this paper is to provide an explicit representation of the measured metric space $(\sM,d,\mu)$ using a Brownian motion, and a countable collection of i.i.d.\ uniform random variables, and to initiate the study of some of its properties and consequences. To do so, we introduce a new general class of tree-like structures constructed from functions in a way that differs from the classical contour function encoding.

% \begin{alert}Switch: 
% \begin{compactenum}
%   \item uniform weights $V_e$, $w_e$ ?
%   \item Excursion/Brownian motion: $\sf e$, Brownian motion $\sf W$ or the less ugly $\textup e$, and  $\textup W$ ? Or we could keep everything with $W$ for Brownian motion, and $W^{\sc e}$ for the excursion ? Or $\mathbf e, \mathbf W$: bold $e$ is ok, but bold $W$ is not.
% \end{compactenum}
% \end{alert}

The study of trees and their encoding has a long history. A prominent example is the now classical encoding of trees from a height or contour function which defines a tree-like metric $d$ from a continuous function using the recursive structure of its level sets. 
%NIC In this setting this explains for instance the connection local times and continuous space branching processes and superprocesses \cite{LeLe1998a,LeGall1991} \JF{Je ne comprends pas trop pourquoi parler de local time ou de superprocesses ici}. 
The representation is intimately related to branching processes and fragmentations related to heights, and thus to the process of local times of the height function \cite{LeLe1998a,LeGall1991,Miermont2003}. 
Notable examples include the Brownian continuum random tree \cite{Aldous1991b} seen as encoded by a Brownian excursion \cite{Aldous1993a,Legall1993}, and Lévy trees \cite{LeLe1998a}. 

Our construction differs radically. The tree will be associated to a continuous function $\omega$ defined on an interval $D\subseteq \R$ using the tree-like structure of the family of greatest convex minorants of the graph of $\omega$ on the intervals $[0,x]$, $x\in D$. Furthermore, while the classical height function encoding provides a metric $d_\omega$ that is continuous on $D^2$, the metrics we construct are discontinuous at every local minimum, and the information contained in the encoding function is greatly shuffled when $\omega$ is irregular. We nonetheless hope to demonstrate that the proposed construction provides a convincing point of view for a number of natural problems involving dynamics, in particular those related to remarkable coalescent and fragmentation processes. 

In the present paper, we focus on the very specific case of Brownian like functions, but the reader will easily be convinced that the procedure should apply more generally to càdlàg functions that have only positive jumps such as spectrally positive Lévy processes which will be studied elsewhere. We will in particular define a \emph{convex minorant tree} $\CMT(\exc,\bU)$ from a Brownian excursion $\exc=(\exc_s)_{s\in[0,1]}$ and an independent family of uniform random variables $\bU=(U_i)_{i\ge 1}$. Formally, $\CMT(\exc,\bU)$ will be 
a compact pointed measure metric space that we initially define together with a metric $d$ on $[0,1]$. 

Defining $\CMT(\exc,\bU)$ is an important building block towards the definition of our main object of interest, where we replace the Brownian excursion $\exc$ by another Brownian-like path. Let $(W_s)_{s\ge 0}$ be a standard (linear) Brownian motion on $\R_+$ and for $\lambda\in \R$, and $s\geq 0$, define the Brownian motion with parabolic drift by 
\BEN\label{eq:BwPd}
X^\lambda_s=W_s - \frac{s^2}2 + \lambda s.\EEN 
We usually write $X:=X^0$ when $\lambda=0$. Our main result is the following: 
\begin{thm}\label{thm:limit_mst_Kn}
%Let $\rho_n=1$ be the root of $M_n$. 
As $n\to\infty$, we have the following convergence in distribution for the Gromov--Hausdorff--Prokohorov topology:
\[(M_n, n^{-1/3} d_n, n^{-1}\mu_n, \rho_n) \xrightarrow[n\to\infty]{} \CMT(X,\bU)\,.\]
\end{thm}
We call $\CMT(X,\bU)$ the Brownian parabolic tree. In particular, the limit appearing in \eqref{eq:scaling_limit_mst} is such~that
\[(\sM,d, \mu, \rho) \eqdist \CMT(X,\bU)\,.\]
Its structure and properties provide a way to make explicit computations. For instance, the Hausdorff dimension of $(\sM,d)$ was still unknown, and we show directly
\begin{thm}\label{thm:compact_dimH}Almost surely, the space $\CMT(X,\bU)$ is compact and has Hausdorff dimension $3$. 
\end{thm} 

$\CMT(\exc, \bU)$ is one of the central objects of this paper, together with its variants. 
The following results can be seen as consequences of Theorem~\ref{thm:limit_mst_Kn}. For a natural number $s\ge 0$, we let $C_n^s$ denote a uniformly random connected graph on $[n]$ with $n-1+s$ edges. Assuming that the edge weights on this component are i.i.d.\ uniform on $[0,1]$, $C_n^s$ possesses an a.s.\ unique minimum spanning tree that we denote by $T_n^s$. It is a consequence of \cite{AdBrGoMi2013a} that, for any $s\ge 0$, the graphs $T_n^s$ considered as metric spaces equipped with the graph distance $d_n^s$ and the counting measure on the nodes $\mu_n^s$ have a limit when suitably rescaled. The following theorem provides an explicit representation of these limits. 
For $s\ge 0$, let $\exc^{(s)}$ be a process on $[0,1]$ whose distribution is characterized by (for all $f:\cC([0,1])\to \R$ bounded continuous) 
\[\Ec{f(\exc^{(s)})} = \frac{\Ec{f(\exc) \cdot (\int_0^1 \exc(u)du)^s}} {\Ec{(\int_0^1 \exc(u) du)^s}}\,,\]
where $\exc$ is a standard normalized Brownian excursion. Let $\rho_n^s=1$ be the root of $T_n^s$.
\begin{thm}\label{thm:limit_mst_surplus}
For any natural number $s\ge 0$, we have the following convergence in distribution for the Gromov--Hausdorff--Prokhorov topology:
\[(T_n^s, n^{-1/2} d_n^s, \mu_n^s, \rho_n^s) \xrightarrow[n\to\infty]{} \CMT(\exc^{(s)}, \bU)\,.\]
In particular, for $s=0$, this implies that $\CMT(\exc, \bU)$ is a Brownian continuum random tree.
\end{thm}
The last claim when $s=0$ follows from simple observations: first $\exc^{(0)}$ is simply a standard Brownian excursion; second $T_n^0=C_n^0$ since the latter is already a tree, which must then be uniform, and it is well-known that such trees converge to the Brownian continuum random tree \cite{Aldous1991b,Aldous1993a,Legall1993}. 

Let us to back to the case of the Brownian motion with parabolic drift $X$. The construction of the convex minorant tree inherently captures some hidden dynamics. The explanation shall come later, and we will for now only present some facts. For $\lambda\in \R$ and $t\ge 0$, let
\[B^\lambda_t:=X^\lambda_t - \underline X^\lambda_t \qquad \textrm{ and }\qquad Z^\lambda=\{s\in \R_+: B^\lambda_s=0\}\,.\]
The process $(Z^\lambda)_{\lambda \in \R}$ is non-increasing for the inclusion, and therefore induces a coalescent of $\R_+$: the intervals of $\R_+\setminus Z^\lambda$ can be a.s.\ indexed in decreasing order of their lengths as $\bgamma^\lambda=(\gamma^\lambda_1, \gamma_2^\lambda, \dots)$. It is known \cite{BrMa2015a,Armendariz2001} that the process of the lengths of the intervals $(|\bgamma^\lambda|)_{\lambda \in \R}$ is the standard multiplicative coalescent constructed by Aldous \cite{Aldous1997}. However, the space $\CMT(X,\bU)$ being constructed as $(\sM,d,\mu,\rho)$ from the completion of a random metric $d$ on $\R_+$, it comes with a canonical injection $\pi:\R_+\to \sM$ that allows to transport $Z^\lambda$ into $\sM$. As a consequence, as $\lambda$ varies, the points of $Z^\lambda$ actually also induce a coalescent/fragmentation of $\CMT(X,\bU)$ in the sense that $\pi(Z^\lambda)$ is a non-increasing set of points in $\sM$. We shall now explore more precisely this process. 

In the construction of $\CMT(X,\bU)$, the entries in $\bU$, which are i.i.d.\ uniform random variables, are assigned to the local minima of $X$. For an interval $I\subseteq \R_+$, let $\bU|_I$ denote the sequence of those entries that are assigned to local minima lying in $I$ (in the same order as in $\bU$). For each $i\ge 1$, let 
\[\tilde e^\lambda_i(s):=B^\lambda(s+\inf \gamma_i^\lambda) \I{0\le s \le |\gamma^\lambda_i|}\,.\]
Let $\mathfrak F^\lambda=(\CMT(\tilde e^\lambda_i, \bU|_{\gamma_i^\lambda}), i\ge 1)$ be the collection of convex minorant trees of the excursions $\tilde e^\lambda_i$, $i\ge 1$. Let now $\cS$ be an independent Poisson point process with intensity a half on $\R_+\times \R_+ \times \R$. There exists a measurable function of $(X, \bU, \cS)$ that yields, for each $\lambda\in \R$, a collection of measured metric spaces $\mathfrak G^\lambda$ obtained from $\mathfrak F^\lambda$ by identifying the points $\pi(x)$ and $\pi(y)$ for each $(x,y,t)\in \cS$ such that $t\le \lambda$ and no point of $Z^t$ lies in the closed interval between $x$ and $y$ (i.e., $x$ and $y$ are in the same interval of the fragmentation at time $t$). Almost surely, there are only finitely many points of $\cS$ satisfying these constraints for each $\lambda\in\R$ and $i\ge 1$. 

We now define some discrete analogs, which are more classical. Let $E^n$ denote $\binom{[n]}{2}$. For each $p\in [0,1]$ write $E^n_p:=\{e\in E^n: w_e\le p\}$ so that the graph $G(n,p)=([n], E^n_p)$ is a classical Erd\H{o}s--Rényi random graph, and the process $(G_n^p)_{p\in [0,1]}$ is non-decreasing (in the sense of inclusion of edge sets). The regime of interest is the one when 
\begin{equation}\label{eq:def_pnlambda}
p=p_n(\lambda):=\frac 1n+\frac{\lambda} {n^{4/3}}.
\end{equation} 
Let $C^{n,\lambda}_i$ be the $i$-th largest connected component of $G(n,p_n(\lambda))$, breaking ties using the minimum label. Let $\mathfrak G^{n,\lambda}=(\mathfrak G^{n,\lambda}_i, i\ge 1)$, where
\[\mathfrak G^{n,\lambda}_i=(C^{n,\lambda}_i,n^{-1/3} d^{n,\lambda}_i, n^{-2/3} \mu^{n,\lambda}_i)\]
denotes the corresponding measured metric space, where $d^{n,\lambda}_i$ is the graph distance, and $\mu_i^{n,\lambda}$ denote the counting measure on (the vertex set of) $C^{n,\lambda}_i$. One may similarly define the minimum spanning forest $\mathfrak F^{n,\lambda}=(\mathfrak F^{n,\lambda}_i, i\ge 1)$, where
\[\mathfrak F^{n,\lambda}_i= (C_i^{n,\lambda}, n^{-1/3} \delta^{n,\lambda}_i, n^{-2/3} \mu^{n,\lambda}_i)\]
and $\delta^{n,\lambda}_i$ is the graph distance on the minimum spanning tree of $C_i^{n,\lambda}$ (constructed from the same collection of weights $(w_e)$).

% \begin{thm}[Continuum Kruskal dynamics]Let $\mathfrak F^\lambda=(\CMT(\tilde e^\lambda_i), i\ge 1)$. Then one has 
% \begin{itemize}
%   \item for each $\lambda\in \R$, $\mathfrak F^\lambda$ is isometric to the collection of metric spaces $\mathfrak M$ minus $Z^\lambda$
%   \item for each $\lambda_1<\lambda_2<\dots <\lambda_k$, $(\mathfrak F^{\lambda_1}, \dots, \mathfrak F^{\lambda_k})$ is the scaling limit of $(\mathfrak F^{n,p_n(\lambda_1)}, \dots, \mathfrak F^{n, p_n(\lambda_k)})$.
% \end{itemize}
% \end{thm}

Then the processes $(\mathfrak F^\lambda)_{\lambda \in \R}$ and $(\mathfrak G^\lambda)_{\lambda\in \R}$ enjoy some continuum Kruskal and Erd\H{o}s--Rényi dynamics reflecting the evolution of $\mathfrak F^{n,\lambda}$ and $\mathfrak G^{n,\lambda}$, respectively, in following sense:
\begin{thm}\label{thm:dynamics_X}
For each $\lambda$, and each $i\ge 1$, $\mathfrak F^\lambda_i$ is isometric to the subet of $\sM$ induced by $\gamma^\lambda_i$. Furthermore we have, for any $k\ge 1$ and $\lambda_1<\lambda_2<\dots<\lambda_k$, jointly
\begin{align*}
  (\mathfrak G^{n,\lambda_1}, \mathfrak G^{n,\lambda_2},\dots, \mathfrak G^{n,\lambda_k}) &\xrightarrow[n\to\infty]{} (\mathfrak G^{\lambda_1}, \mathfrak G^{\lambda_2}, \dots, \mathfrak G^{\lambda_k}) \quad and,\\
  (\mathfrak F^{n,\lambda_1}, \mathfrak F^{n,\lambda_2},\dots, \mathfrak F^{n,\lambda_k}) &\xrightarrow[n\to\infty]{} (\mathfrak F^{\lambda_1}, \mathfrak F^{\lambda_2}, \dots, \mathfrak F^{\lambda_k})\,,
\end{align*}
in distribution, where, in each case, the convergence holds with respect to the product Gromov--Hausdorff--Prokhorov topology on sequences of measured metric spaces. 
\end{thm}

Theorem~\ref{thm:dynamics_X} provides an explicit coupling for the standard metric coalescent dynamics constructed by Rossignol in \cite{Rossignol2017a} (see also \cite{AdBrGoMi2019a}). In particular, this shows that $\CMT(X,\bU)$ is the right object to lift the multiplicative coalescent defined by Aldous to the level of metric spaces (as well as its augmented version \cite{BhBuWa2014a}). We are not interested here in verifying that there is indeed a natural Markov semigroup acting on measured metric spaces that formalizes these dynamics; such a Markov processes is constructed and studied in \cite{AdBrGoMi2019a} (see also \cite{Frilet2021}). 

There is also an analog to Theorem~\ref{thm:dynamics_X} replacing $\CMT(X,\bU)$ by $\CMT(\exc,\bU)$ which is relevant to the additive coalescent. The following notation are intentionally similar to that used previously; it shall always be clear to which case we refer. For each $\lambda\le 0$, let $\exc^\lambda(s)=\exc(s)+\lambda s$, and $\underline \exc^{\lambda}(s)=\inf\{\exc^\lambda(r): 0\le r\le s\}$. Write $Z^\lambda=\{s\in [0,1]: \exc^\lambda(s)=\underline \exc^\lambda(s)\}$. The process $Z^\lambda$ is non-increasing in $\lambda$ and induces a coalescent of $[0,1]$, as $\lambda$ varies in $(-\infty,0]$. Let $(\gamma^\lambda_i)_{i\ge 1}$ denote the sequence of lengths of the intervals of $[0,1]\setminus Z^\lambda$, in decreasing order. It is known since the results of \citet{Bertoin2000a} that the process $t\mapsto (\gamma^{-t}_i)_{i\ge 1}$ for $t\geq 0$ is the fragmentation dual to the standard additive coalescent introduced by \citet{AlPi1998a}. Just as before $\CMT(\exc,\bU)$ is a random measured real tree $\mathfrak T=(\sT,d, \mu)$ defined through the completion of a random metric $d$ on $[0,1]$, and we let $\pi:[0,1]\to \sT$ denote the canonical injection. 
%\JF{$\sT$ est défini implicitement? est-ce que tu veux dire,  $\CMT(e,\bU)$ is a (mesured?) random tree  $(\sT,d, \mu)$}
 This allows one to transport $Z^\lambda$ in $\sT$ and therefore, to see $\sT\setminus \pi(Z^{-t})$, $t\ge 0$, as a fragmentation of $\CMT(\exc,\bU)$. 

Formally, for each $\lambda\le 0$ and $i\ge 1$, let
\[\exc^\lambda_i(s):=(\exc^\lambda(\inf \gamma^\lambda_i+s)-\underline \exc^\lambda(s))\I{0\le s\le |\gamma^\lambda_i|}\,.\]
Let $\mathfrak T^\lambda_i:=\CMT(\exc_i, \bU^\lambda_i)=(\sT^\lambda_i, d_i^\lambda, \mu_i^\lambda)$. The following theorem provides an explicit coupling between the representations of the fragmentation that is dual to the additive coalescent due to Aldous \& Pitman on the one hand \cite{AlPi1998a}, and to Bertoin \cite{Bertoin2000a} on the other. It also provides another point of view on some recent results of \citet{KoTh2023a}. Define $\cP:=\{(\pi(x), - \lambda): x\in Z^{\lambda-}\setminus Z^\lambda, \pi(x)\in \Skel(\sT), x\in [0,1], \lambda\le 0\}$, where $\Skel(\sT)$ is the skeleton of $\sT$ that we define here as the set of points $u\in \sT$ such that $\sT\setminus \{u\}$ has at least two connected components. 
 
\begin{thm}[Aldous--Pitman vs Bertoin]\label{thm:additive_coalescent}
Let $\exc$ be a normalized Brownian excursion and recall that $\mathfrak T=(\sT,d,\mu)=\CMT(\exc,\bU)$.
Almost surely, for all $\lambda\leq 0$ and all $i\ge 1$, $\mathfrak T_i^\lambda$ is isometric to the subtree of $\sT$ induced by $\gamma^\lambda_i$. Furthermore
\begin{compactenum}[i)]
  \item for each $\lambda\le 0$ and each $i\ge 0$, $|\gamma^\lambda_i|=\mu(\sT^\lambda_i)$,
  \item conditionally on $(\sT,d)$, $\cP$ is a Poisson point process of unit intensity on $\Skel(\sT)\times \R_+$.
\end{compactenum}
Hence, the process $(\bgamma^{-t})_{t\ge 0}$ is precisely the Aldous--Pitman fragmentation of the Brownian CRT $\mathfrak T$. 
\end{thm}

% \begin{alert}Just a tiny problem: $Z^\lambda$ is not countable, so $\cP$ is not clearly a Poisson point process (the length measure is sigma-finite...). But the reason why $Z^\lambda$ is not countable is simply because of the structure of $\R_+$. We should extract the "relevant" points from $Z^\lambda$ first, these are the ones that are really separating two intervals. It seems that taking the points $x$ which are in $Z^{\lambda-}$ but not in the closure of $Z^\lambda$ should work
% \end{alert}



% \begin{alert}Find good notations to distinguish the cases $\CMT(X^0,\bU)$ and $\CMT(e,\bU)$. So far, we use the same notation for zero set and excursion lengths.

% Attention to the definition of $\cP$: the set $Z$ is not countable, need to remove the points that are only there because of the struture on $\R_+$. 
% \end{alert}




 


% We conjecture that the construction maybe adapted naturally (distances in $\CMT(e,\bU)$ and $\CMT(X^0,\bU)$ are specific to the Brownian regime) to prove that 
% \begin{itemize}
%   \item the scaling limits of the minimum spanning trees of the multiplicative random graphs may be obtained as $\CMT^(X^*, \bU)$
%   \item inhomogeneous continuum random trees are obtained as $\CMT^*(e^*,\bU)$ for the exchangeable bridges of Bertoin \cite{Bertoin2001a} and Miermont \cite{Miermont2001}
%   \item $\CMT^*$ provides the way to couple the representation of Aldous \& Pitman \cite{AlPi2000a} 
% \end{itemize}

\subsection{Motivation and history of related results} % (fold)
\label{sec:motivation_and_history}

% {\red 
% \begin{itemize}
%   \item history of the MST complete graph, the questions and the CRT
%   \item The construction of \cite{AdBrGoMi2013a} already relies on the connection with random graphs and their scaling limits \cite{AdBrGo2012a}, and thus implicitly on the multiplicative coalescent. 
%   \item Explain the discrete intuition: Kruskal's algorithm and the fact that edges connect random points in the distinct connectec components that are themselves size-biased; in other words, there is an underlying discrete coalescent, and the metric is inductively constructed by connecting the fragments by adding an edge between random points. The idea is to construct a metric space from these dynamics: for each time, have countably many tree-like metric spaces, when two merge construct a metric space on the disjoint union by connecting two independent uniform random points. Doing this would preserve the increasing property (we never mess with the metric by adding to many connections that we then need to remove), but this requires to understand every step of the process and the fact that the times of connections are dense raises some additional difficulty (the idea of adding too many edges and then remove allows to work "in batch" and to understand only what happens in non vanishing time intervals).  
% \end{itemize}
% }

It was already known from the work of \citet*{AdBrGoMi2013a} that $\mathfrak M_n = (M_n,n^{-1/3} d_n, n^{-1}\mu_n, \rho_n)$ converges in distribution. The proof relies on a Cauchy sequence argument for the distribution of $\mathfrak M_n$, and is thus essentially existential. In particular, it does not provide an explicit construction of the limit. The novelty of Theorem~\ref{thm:limit_mst_Kn} lies in the identification of the limit as the convex minorant tree $\CMT(X,\bU)$. Note that, by results of \citet{AdSe2021a}, this is also the scaling limit of random 3-regular graphs.

\medskip
\noindent\textsc{About the scaling limit of $\mathfrak M_n$.}
In order to understand the underlying issues, let us be more specific about the approach used in \cite{AdBrGoMi2013a}. The general idea is to analyse the minimum spanning tree using Kruskal's algorithm \cite{Kruskal1956}. This algorithm proceeds by adding the edges by increasing order of weights to an initially empty graph, provided doing so does not create a cycle. Since in the random setting, the order is uniformly random, the (conditional) distribution according to which the edges are added at each step is straightforward, and the difficulty consists in avoiding the cycles. So one may try to first add edges regardless of whether they create cycles or not, with the hope to be able to deal with that issue later on. One shall do this up to a threshold for the weights that ensures that there are not too many cycles (or dealing with them would be hard), but that the connected components are already fairly large (or we have basically gathered no information). These two competing constraints lead to the choice of keeping only edges with weight at most $p_n(\lambda)=\tfrac 1 n + \lambda n^{-4/3}$ with $\lambda\in \R$ large.  

This $p_n(\lambda)$ happens to be precisely the critical window of the random graphs.  The scaling limit of $G(n,p_n(\lambda))$, seen as the sequence of compact metric spaces $\mathfrak G^{n,p_n(\lambda)}$ is known from the results of \citet*{AdBrGo2012a} who built on the pioneering work of Aldous who had previously obtained the scaling limit for the vector of the sizes of the connected components \cite{Aldous1997}. The analysis in \cite{AdBrGoMi2013a} relies on the fact that, (1) given a connected component of the random graph, one may obtain a tree distributed like its minimum spanning tree by breaking cycles randomly (removing uniformly edges, unless they disconnect the component), and (2) that a similar procedure works on the scaling limit. 
% From there, a significant part of the work consists in controlling what happens as the parameter $\lambda\to\infty$, and in particular to verify that the minimum spanning tree of the largest connected component is a good approximation for $M_n$, even though is total mass is only $O(n^{2/3})$. 
% This part uses refinements of the arguments in \cite{AdBrRe2009} for the diameter of $M_n$. 
This forward/backward procedure provides some geometric information but it is inherently tricky to track it precisely. This explains why it does not lead to an explicit construction of the limit in terms of simple building blocks, or also why the Hausdorff dimension (Proposition~\ref{pro:space_compact}) remained unknown. Furthermore, this approach is fundamentally incapable of providing any result about the behavior at different times, since the cycle breaking procedure removes all cycles. 

\medskip
\noindent\textsc{Related results on the MST.}
Let us mention that Angel and Senizergues are currently finishing a paper in which they study the scaling limit of the local limit of $M_n$, that was described by \citet{Ad2013a}. As the local weak limit of $M_n$ is an infinite tree, their object $\cal M$ is not compact; still $\cal M$ has Hausdorff dimension 3, and it also seems to be the local limit of $\CMT(X,\bU)$. It is our understanding that they also plan to study a ``mesoscopic'' limit that would be an analog to the self-similar CRT of Aldous \cite{Aldous1991} for~$\mathfrak M_n$. 

\medskip
\noindent\textsc{About the scaling limit of random graphs.} It is known that the critical random graphs have a scaling limit \cite{AdBrGo2012a}, which has been constructed for each $\lambda\in \R$ in \cite{AdBrGo2012a} (see also \cite{AdBrGo2010}): for each $i\ge 1$, a connected component is built as the tree with height process $\tilde e^\lambda_i$, in which cycles are created by identifying pairs of points whose locations are given by a Poisson point process under $\tilde e^\lambda_i$. The tree is genuinely different from $\CMT(\tilde e^\lambda_i, \bU|_{\gamma^\lambda_i})$ that we use here. However, the marginals described in Theorem~\ref{thm:dynamics_X} of course correspond. For instance, the number of pairs of points that are identified must have the same distribution conditionally on the excursion. One quickly verifies that (Lemma~\ref{lem:area}), for $\lambda\in \R$ and $i\ge 1$, the average number of pairs given $\tilde e^\lambda_i$ (which also determines all the $\gamma^r_j$ which are subsets of $\gamma^\lambda_i$), is
\[\int_{\gamma^\lambda_i} \tilde e^\lambda_i(s)ds = \frac 1 2 \int_{-\infty}^\lambda \sum_{j\ge 1} |\gamma^{r}_j|^2 \I{\gamma^r_j\subseteq \gamma^\lambda_i} dr\,.\]
The spanning subtree used in \cite{AdBrGo2012a} is discovered by a depth-first search; quite recently, \citet{MiSe2022a} have studied the construction of these scaling limits from a breadth-first exploration. The procedures used in \cite{Aldous1997,AdBrGo2012a,MiSe2022a} are not consistent as $\lambda\in \R$ varies, and the objects obtained for two different values of $\lambda$ have no reason to be close, and do not relate simply to any dynamics.

% \begin{alert}Check the scaling factor for the intensity of the Poisson process; need to match the one in \cite{AdBrGo2010,AdBrGo2012a}, where it is a factor one under the curve when $X^n$ is the Lukasiewiecz walk (or $1/2$ with the height process instead).
% \end{alert}

\medskip
\noindent\textsc{About the limit Erd\H{o}s--Rényi and Kruskal dynamics.} 
Since there is an obvious process version for the entire structure at the discrete level, the question of the dynamics for the limit objects (continuum forests or graphs) is quite natural. 
First it is known from results of \citet{Armendariz2001} and \citet{BrMa2015a} that the process $X=X^0$ defined in \eqref{eq:BwPd} encodes the standard multiplicative coalescent, and thus permits to obtain a coupling of the limit of the sizes of the connected components (see also \cite{MaRa2017a}). A minor modification also yields a coupling of both the sizes and the number of extra edges via an explicit construction of the augmented multiplicative coalescent constructed by \citet*{BhBuWa2014a} (see also the recent point of view by \citet{CoLi2023a,CoLi2023b}). The metrics require the new point of view of the convex minorant tree. We emphasize that what we mean here by dynamics is a process in $\lambda$ whose marginals are the scaling limits for fixed $\lambda$, and that we do not consider the question of the existence of nice Markov semigroup acting on sequences of compact measured metric spaces; this question is addressed in \cite{AdBrGoMi2019a,Rossignol2017a}. 

Let us now say a few words about the case of the (standard) additive coalescent. It was first introduced by Aldous and Pitman \cite{AlPi1998a} as the time reversal of the fragmentation process where a Brownian continuum random tree is split as time goes using a Poisson point process. Bertoin \cite{Bertoin2000a} then observed that one obtains the same fragmentation process by cutting the unit interval at the times where a Brownian excursion plus an increasing linear drift touches its running infimum. These two constructions have been connected in a number of ways at the discrete level, starting with \citet{ChLo2002} who used a representation based on hashing with linear probing \cite{KoWe1966a,Knuth1973b}; the construction of $\CMT(\exc,\bU)$ can be seen as a scaling limit for the tree appearing there. \citet{BrMa2015a} and \citet{MaWa2018a} provide alternative approaches. Quite recently, the two processes have been coupled directly in the continuous by \citet{KoTh2023a}, just as Theorem~\ref{thm:additive_coalescent}. Let us also emphasize the fact that Theorem~\ref{thm:additive_coalescent} is a by-product of the same construction used for Theorem~\ref{thm:dynamics_X}: this shows that the standard additive and multiplicative coalescent are, even when considered at the enriched metric level, very strongly related since they are two versions of the same construction applied two different functions ($\exc$ and $X$, respectively). 

\subsection{Intuition and techniques} % (fold)
\label{sub:intuition_and_techniques}


% \begin{alert}
% 1) In discrete the coalescent giving the minimum spanning tree is simple: random edges

% 2) one may expect that in the limit, the object should be constructed in a similar way. Following the dynamics: from the multiplicative coalescent, each time two components merge, identify two random points. 

% 3) Difficulty, the range is $\R$, and connections happen densely in any time interval

% 4) Will rely on the "linearization" of the coalescent. For instance, the linearization simplifies some aspects, but seriously complicates others. For instance, the distribution of the processes encoding the discrete coalescent are not very nice any more. But once we know the coalescing events, the connections are easy. The components to the right are connected randomly to the left (there is a collection of discrete uniforms here, say $\bU^n$)

% 5) There is a scaling limit where the coalescents are also given by a simple process. 

% 6) It remains to construct something that has the required dynamics (again, the dynamics are easy, but one should start from "something"). The uniforms are the scaling limit of $\bU^n$.

% 7) Prove that we have a construction that makes sense; this involves a novel class of tree-like metric spaces defined from excursions

% 8) Prove that the law is the correct one, and we do this by coupling.
% \end{alert}

We shall now try to convey the main ideas that underlie the construction of our scaling limits. The intuition comes from the discrete setting, and we shall explain why the relevant objects should have continuum analogs, and how these limits could be formally defined. There is no very simple axiomatic definition of the minimum spanning tree, at least none that seems suitable to a direct analysis, and one is lead to track the evolution of a construction algorithm in order to obtain the minimum spanning tree. While the construction in \cite{AdBrGoMi2013a} relies on Kruskal's algorithm which grows a forest \cite{Kruskal1956}, our approach is based on a combination of algorithms by Kruskal and Prim \cite{Prim1957}, which grows a tree containing a given vertex. 
% Compared with the approach in \cite{AdBrGoMi2013a}, rather than adding to many edges and fix the issues afterwards, we shall only add the edges that are really part of the minimum spanning tree. 

Fix $n\ge 1$. Prim's algorithm proceeds as follows. Let $v_1=1$. We define the order of vertices $v_2,v_3,\dots, v_n$ iteratively. For every $j=1,\dots, n$, we let $V_j=\{v_1,\dots, v_j\}$. For $i=2,\dots, n$, let $e_i$ be the edge between $V_{i-1}$ and $[n]\setminus V_{i-1}$ that has the smallest weight. Write $e_i=\{u_i,v_i\}$ with $u_i\in V_{i-1}$ and $v_i\in [n]\setminus V_{i-1}$. Then the minimum spanning tree $M_n$ is the graph on $[n]$ with edge set $\{e_2,\dots, e_n\}$. The order $v_1,v_2,\dots, v_n$ is called the Prim order. It turns out that, for any $p\in [0,1]$, the connected components of $G(n,p)=([n], E_p^n)$, where $E_p^n=\{e\in E^n: w_e\le p\}$ are intervals in the Prim order (that is, the vertex set of each connected components is $\{v_a,v_{a+1},\cdots,v_b\}$ for some $1\leq a\leq b \leq n$). In particular, as $p$ increases, only adjacent intervals may merge. 

Now, consider the graph consisting of edges with weights (strictly) lower than $w_{e_i}$, $\{e: w_e<w_{e_i}\}$, that is just before the edge $e_i$ is added. Let $L_i$ be the connected component containing $v_{i-1}$ in this graph. These are precisely the connected components that merge when $e_i$ is added. For $p\in [0,1]$, let $\cF_p$ denote the sigma-algebra generated by the events $\{w_e\le p, e\in E\}$. The following is straightforward:
\begin{lem}\label{lem:discrete_merges}
For each $2\le i\le n$, conditionally on $\cF_{w_{e_i}-}$, the vertex $u_i$ is uniformly random in $L_i$. 
\end{lem}

In other words, in this discrete representation in which the vertices are placed in the Prim order $v_1,\cdots,v_n$, conditionally on the sequence of intervals that merge, the edges that are part of the minimum spanning tree precisely connect a uniform random vertex in the left interval to the left-most vertex in the right interval. Still in this discrete representation, determining the distribution of the sequence of pairs of intervals that merge together is not quite as easy any longer. Fortunately, in the limit, it is given explicitly by the rather nice process $\R_+\setminus Z^\lambda$.
One might thus hope that, in the limit, one should be able to construct the scaling limit of the minimum spanning tree as follows: for each $\lambda\in \R$, each interval $\gamma^\lambda_i$ should be associated to a continuum random tree, and as $\lambda$ increases, these trees should merge using an analog of the discrete dynamics: each time two intervals merge, a uniformly random point in the left interval and the left-most point of the right one should be identified; the minimum spanning tree should then be the limit as $\lambda\to \infty$ (which would indeed be a tree since $\gamma^\lambda_1\uparrow (0,\infty)$ as $\lambda\to\infty$).

While these dynamics are reasonable, they do not really provide a clear path towards a construction: while at the discrete level, the addition of edges does create some length, identifying points in the limit does not, and it remains to understand from what the length emerges. A natural idea consists in constructing the length using some kind of local time arising from the process $(Z^\lambda)_{\lambda \in \R}$. With this objective in mind, let us go back to the discrete setting.  
For any $i,j\in [n]$, we may find all the nodes on the path between $i$ and $j$ in the minimum spanning tree as follows. For some $p\in [0,1]$, it is convenient to write $i\sim_{n,p} j$ if $i$ and $j$ lie in the same connected component of the graph with edges of weight at most $p$: let $i<j\in [n]$ and let $p(i,j)=\inf\{p: i\sim_{n,p} j\}$. The path between $i$ and $j$ must go through the unique edge $e_k=\{u_k,v_k\}$ with weight $p(i,j)$; then, at time $p(i,j)-$ we are left with two connected components, each containing a pair of points ($u_k$ and $i$ on the one hand, and $v_k$ and $j$ on the other) that should each be connected by a path. Proceeding recursively, the process eventually terminates and yields precisely the collection of nodes which are on the path between $i$ and $j$, and the distance $d_n(i,j)$ is then simply the cardinality of that set (minus one). 

This approach is amenable to an extension to the continuous setting, that we expose here informally. For $x,y\in \R_+$, let $x\sim_\lambda y$ if there is no point of $Z^\lambda$ in the closed interval between $x$ and $y$. Let $I^\lambda(x):=\{y\in \R_+: x\sim_\lambda y\}$. Take now $x<y$ for convenience. Let $\lambda(x,y)=\inf\{\lambda \in \R: x\sim_\lambda y\}$. It turns out that $Z^{\lambda(x,y)}$ almost surely contains a single point in $[x,y]$, that we denote by $\kappa(x,y)$. Then, just before $x$ and $y$ get connected, we have two distinct intervals $I^{\lambda(x,y)}(x)$, and $I^{\lambda(x,y)}(y)$, which are separated by the point $\kappa(x,y)$. The discrete setting suggests that one should choose a uniformly random point $\eta(x,y)$ in $I^{\lambda(x,y)}$ (this is where the uniforms in $\bU$ are used). Then, the two points $\eta(x,y)$ and $\kappa(x,y)$ should be the continuous analog of the extremities of the maximum weight edge on the path between $x$ and $y$. Proceeding recursively by looking for the path between $x$ and $\eta(x,y)$ in $I^\lambda(x,y)(x)$ on the left, and the path between $\kappa(x,y)$ and $y$ in $I^{\lambda(x,y)}(y)$ on the right should yield a random subset of $\R_+$ containing all the points used to go from $x$ to $y$, that should resemble some kind of random Cantor set, and the distance between $x$ and $y$ should be some Hausdorff measure of that set. Our main objective is now to verify that this intuition can be turned into formal definitions,
but also that the objects constructed are indeed the ones we are looking for.

% % subsection overview_of_the_results (end)

\subsection{Organization of the paper} % (fold)
\label{sec:plan_of_the_paper}


The paper is organized as follows. In Section~\ref{sec:recursive_convex_minorants}, we discuss recursive convex minorants, the associated trees and their properties. In particular, it is there that we define the convex minorant trees $\CMT(\exc,\bU)$ and $\CMT(X,\bU)$. In Section~\ref{sec:a_new_point_of_view_on_the_continuum_random_tree}, we prove that the tree  $\CMT(\exc,\bU)$ is a Brownian CRT, and we exhibit the coupling mentioned above between the representations of the fragmentation dual to the additive coalescent by Adous--Pitman \cite{AlPi1998a} on the one hand, and by Bertoin \cite{Bertoin2000a} on the other. In Section~\ref{sec:compactness}, we prove that $\CMT(X,\bU)$ is almost surely compact. In Section~\ref{sec:mass_hausdorff}, we construct the mass measure and use it to lower bound the Hausdorff dimension. Finally, Section~\ref{sec:coupling} is devoted to proving that the Brownian parabolic tree $\CMT(X,\bU)$ is distributed like scaling limit of the minimum spanning tree. 
%{\nicc Auxialiary technical results are found in the Appendix.} %A more detailed table of contents follows. 

% subsection plan_of_the_paper (end)

{\small
\setlength{\cftbeforesecskip}{3pt}
\tableofcontents
}

\section{Notation}
\label{sec:notation}

Let $\bW$ be the Wiener measure on $\cC(\R_+,\R)$, the set of continuous functions $f:\R_+\to \R$; this is the law of standard Brownian motion $(W_t)_{t\ge 0}$ starting at $0$. For a continuous process $\omega = (\omega_t)_{t\ge 0}$, we let $\underline \omega$ and $\overline \omega$ denote respectively the running infimum and supremum processes: $\underline \omega_t :=\inf\{\omega_s : 0\le s\le t\}$ and $\overline \omega_t = \sup\{\omega_t: 0\le s \le t\}$.

 % and $\tilde \bW$ the law of Brownian motion with parabolic drift $X$ given by $X_t=W_t-t^2/2$. 
 % %Let $\cF_t$ denote the filtration generated by $(W_s, s\le t)$, completed to contain all the events of probability $0$. 
 

Let $\N=\{1,2,\dots\}$ and $\N_0=\N=\cup\{0\}$. Let $\cU=\bigcup_{n\ge 0} \N^n$ be the set of finite words on $\N$. The empty word, denoted by $\varnothing$, is the only element of $\N^0$.
We see the elements of $\cU$ as words on $\N$. For $u\in \N^n$ and $i\in \N$, we let $ui$ denote the element of $\N^{n+1}$ obtained by appending $i$ after $u$, so if $u=(u_1, u_2, \dots, u_k)$, $ui=(u_1,\dots, u_n, i)$. We see $\cU$ as a tree rooted at $\varnothing$, where the natural genealogical order denoted by $\preceq$ is such that we have $u\preceq v$ if $u$ is a prefix of $v$, potentially $u=v$. 
%For $u\in \N^n$ and $v\in \N^m$ we let $uv\in \N^{n+m}$ denote the concatenation of $u$ and $v$.  The children of a node $u\in \cU$ are the $ui$, $i\ge 1$.
Similarly, we let $\cU_2= \bigcup_{n\ge 0} \{0,1\}^n$. 

