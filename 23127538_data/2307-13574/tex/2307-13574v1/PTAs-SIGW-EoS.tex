\documentclass[prd,aps,11pt, onecolumn, superscriptaddress, floatfix, nofootinbib, amsmath, amssymb, preprintnumbers]{revtex4-1}


\usepackage{dcolumn}% Align table columns on decimal point
\usepackage{verbatim}
%\documentclass{scrartcl}
\usepackage{breakurl}
\usepackage{lmodern}
\usepackage{subcaption}

\pdfoutput=1
\usepackage{graphicx}
\usepackage{bm}
\usepackage{amsmath}
\usepackage{amsfonts}
\usepackage{amssymb}
\usepackage{multirow}
\usepackage[colorlinks,linktocpage,linkcolor=cyan,citecolor=cyan]{hyperref}
\usepackage{adjustbox}
\usepackage{array}
\usepackage[capitalise]{cleveref}
\usepackage{acro}
\usepackage[utf8]{inputenc}
\usepackage{CJKutf8}
\usepackage{amssymb}

\usepackage{tikz}
\usepackage{pgfplots}
\pgfplotsset{compat=newest,every axis plot/.append style={line width=1pt}}
%\usepackage[citestyle=phys]{biblatex}



\renewcommand{\baselinestretch}{1.5}



\crefname{figure}{Fig.}{Figs.}
\Crefname{figure}{Fig.}{Figs.}
\def\({\left(}
\def\){\right)}
\def\[{\left[}
\def\]{\right]}
\newcommand{\Mc}{\mathcal{M}_c}
\newcommand{\pc}{{\rm pc}}
\newcommand{\GeV}{{\rm GeV}}
\newcommand{\m}{{\rm m}}
\newcommand{\yr}{{\rm yr}}
\newcommand{\Hz}{{\rm Hz}}
\newcommand{\Mpc}{{\rm Mpc}}
\newcommand{\be}{{\begin{eqnarray}}}
	\newcommand{\ee}{{\end{eqnarray}}}
\newcommand{\h}{{\rm km~s^{-1}Mpc^{-1}}}
\newcommand{\Gpcyr}{{\rm Gpc^{-3}~yr^{-1}}}
\newcommand{\overbar}[1]{\mkern 1.5mu\overline{\mkern-1.5mu#1\mkern-1.5mu}\mkern 1.5mu}
\newcommand{\mpl}{m_\mathrm{Pl}}
\newcommand{\fnl}{f_\mathrm{NL}}
\newcommand{\abs}[1]{{\left \vert #1 \right \vert}}
%\newcommand{\overbar}[1]{\mkern 1.25mu\overline{\mkern-1.25mu#1\mkern-1.25mu}\mkern 1.25mu}
\newcommand{\cT}{\mathcal{T}}
\newcommand{\cP}{\mathcal{P}}
\newcommand{\cR}{\mathcal{R}}
\newcommand{\cH}{\mathcal{H}}
\newcommand{\cS}{\mathcal{S}}
\newcommand{\cO}{\mathcal{O}}
\newcommand{\cF}{\mathcal{F}}
\newcommand{\cD}{\mathcal{D}}
\newcommand{\cC}{\mathcal{C}}
\newcommand{\bk}{\bm{k}}
\newcommand{\bq}{\bm{q}}
\newcommand{\bp}{\bm{p}}
\newcommand{\bx}{\bm{x}}
\newcommand{\bn}{\mathbf{n}}
\newcommand{\bD}{\mathbf{D}}
\newcommand{\ud}{\mathrm{d}}
\newcommand{\uin}{\mathrm{in}}
\newcommand{\uRD}{\mathrm{RD}}
\newcommand{\uGW}{\mathrm{gw}}
\newcommand{\uc}{\mathrm{crit}}
\newcommand{\Beq}{\begin{align}}
	\newcommand{\Eeq}{\end{align}}
\newcommand{\sunset}{Z}
\newcommand{\fourvertex}{C}
\newcommand{\dif}{\,\mathrm{d}}



\DeclareAcronym{SW}{
	short = SW ,
	long = Sachs-Wolfe ,
	short-plural =  ,
}
\DeclareAcronym{BH}{
	short = BH ,
	long = black hole ,
	short-plural = s ,
}
\DeclareAcronym{SNR}{
	short = SNR ,
	long = signal-to-noise ratio ,
	short-plural = s ,
}
\DeclareAcronym{IMRPPv2}{
	short = ,
	long = {\normalsize IMRP}{\footnotesize HENOM}{\normalsize P}v2 ,
	short-plural = ,
}
\DeclareAcronym{SFR}{
	short = SFR ,
	long = star formation rate ,
	short-plural =  ,
}
\DeclareAcronym{IMR}{
	short = IMR ,
	long = inspiral-merger-ringdown ,
	short-plural =  ,
}
\DeclareAcronym{ABH}{
	short = ABH ,
	long  = astrophysical black hole,
	short-plural = s ,
}
\DeclareAcronym{GW}{
	short = GW ,
	long = gravitational wave ,
	short-plural = s ,
}
\DeclareAcronym{SIGW}{
	short = SIGW ,
	long = scalar-induced gravitational wave ,
	short-plural = s ,
}
\DeclareAcronym{GWB}{
	short = GWB ,
	long = gravitational-wave background ,
	short-plural = s ,
}
\DeclareAcronym{CBC}{
	short = CBC ,
	long = compact binary coalescence ,
	short-plural = s ,
}
\DeclareAcronym{BBH}{
	short = BBH ,
	long = binary black hole ,
	short-plural = s ,
}
\DeclareAcronym{PBH}{
	short = PBH ,
	long = primordial black hole ,
	short-plural = s ,
}
\DeclareAcronym{LIGO}{
	short =LIGO ,
	long = Laser Interferometer Gravitational-Wave Observatory ,
	short-plural = ,
}
\DeclareAcronym{LVK}{
	short = LVK ,
	long = {Advanced LIGO, Virgo and KAGRA} ,
	short-plural = ,
}
\DeclareAcronym{ET}{
	short = ET ,
	long  = Einstein Telescope,
	short-plural =  ,
}
\DeclareAcronym{CE}{
	short = CE ,
	long  = Cosmic Explorer,
	short-plural =  ,
}
\DeclareAcronym{LISA}{
	short = LISA ,
	long  = Laser Interferometer Space Antenna,
	short-plural =  ,
}
\DeclareAcronym{BBO}{
	short = BBO ,
	long  = big bang observer,
	short-plural =  ,
}
\DeclareAcronym{DECIGO}{
	short = DECIGO ,
	long  = Deci-hertz Interferometer Gravitational wave Observatory,
	short-plural =  ,
}
\DeclareAcronym{SKA}{
	short = SKA ,
	long  = Square Kilometre Array,
	short-plural =  ,
}
\DeclareAcronym{PTA}{
	short = PTA ,
	long = pulsar timing array ,
	short-plural = s ,
}
\DeclareAcronym{FRW}{
	short = FRW ,
	long = Friedmann-Robertson-Walker ,
	short-plural =  ,
}
\DeclareAcronym{CMB}{
	short = CMB ,
	long = cosmic microwave background ,
	short-plural =  ,
}
\DeclareAcronym{RD}{
	short = RD,
	long  = radiation-dominated ,
	short-plural =  ,
}
\DeclareAcronym{MD}{
	short = MD,
	long  = matter-dominated ,
	short-plural =  ,
}
\DeclareAcronym{HD}{
	short = HD,
	long  = Hellings-Downs ,
	short-plural =  ,
}
\DeclareAcronym{SMBH}{
	short = SMBH ,
	long  = supper-massive black hole ,
	short-plural = s ,
}
\DeclareAcronym{SGWB}{
	short = SGWB ,
	long  = stochastic gravitational-wave background ,
	short-plural = s ,
}
\DeclareAcronym{NG}{
	short = NANOGrav ,
	long  = North American Nanohertz Observatory for Gravitational Waves ,
	short-plural =  ,
}
\DeclareAcronym{PSD}{
	short = PSD ,
	long  = power spectral density ,
	short-plural = s ,
}
\DeclareAcronym{PDF}{
	short = PDF ,
	long  = probability distribution function ,
	short-plural = s ,
}
\DeclareAcronym{BBN}{
	short = BBN ,
	long  = big-bang nucleosynthesis ,
	short-plural =  ,
}
\DeclareAcronym{EoS}{
	short = EoS ,
	long  = equation of state ,
	short-plural =  ,
}

%\usepackage{subfigure}
\usepackage{subcaption}
\renewcommand\thesubfigure{\alph{subfigure}} % 将子图的标签格式改为 a, b, c 等





\begin{document}
	
	
	\title{Exploring the Equation of State of the Early Universe: Insights from BBN, CMB, and PTA Observations}


 	\author{Zhi-Chao Zhao}
	\affiliation{Department of Applied Physics, College of Science, China Agricultural University,
		Qinghua East Road, Beijing 100083, People's Republic of China}
  
	
	\author{Qing-Hua Zhu}
	\affiliation{Department of Physics, Chongqing University, Chongqing 401331, People's Republic of China}
  
	
	\author{Sai Wang}
	\email{Corresponding author: wangsai@ihep.ac.cn}
 \affiliation{Theoretical Physics Division, Institute of High Energy Physics, Chinese Academy of Sciences, Beijing 100049, People's Republic of China}
	\affiliation{School of Physical Sciences, University of Chinese Academy of Sciences, Beijing 100049, People's Republic of China}
	
	\author{Xin Zhang}
	\email{Corresponding author:  zhangxin@mail.neu.edu.cn}
	\affiliation{Key Laboratory of Cosmology and Astrophysics (Liaoning) \& College of Sciences, Northeastern
University, Shenyang 110819, People's Republic of China}
	\affiliation{National Frontiers Science Center for Industrial Intelligence and Systems Optimization, Northeastern University, Shenyang 110819, People's Republic of China}
\affiliation{Key Laboratory of Data Analytics and Optimization for Smart Industry (Ministry of Education),
Northeastern University, Shenyang 110819, People's Republic of China}
	


 
\begin{abstract} 

Strong evidence for a gravitational-wave background (GWB) has been reported in the nano-Hertz band. Interpreting the origin of this background to be scalar-induced gravitational waves (SIGWs), we explore the equation of state (EoS) of the early universe by performing Bayes parameter inferences across the big-bang nucleosynthesis (BBN), cosmic microwave background (CMB), and pulsar timing array (PTA) joint observations for the first time. Assuming a monochromatic power spectrum for primordial curvature perturbations, we obtain the spectral amplitude $A\sim10^{-3}-10^{-1}$ and spectral peak frequency $f_\ast\sim10^{-7}-10^{-6}$ Hz. We find that the radiation domination with EoS $w=1/3$ is compatible with the current observational data, the kination domination with EoS $w=1$ is not forbidden, while the early matter domination with EoS $w=0$ is excluded at more than $2\sigma$ confidence level. These results can be tested with future observations. 

\end{abstract}
	
	
\maketitle
	
\acresetall
%%%%%%%%%%%%%%%%%%%%%%%%%%%%%%%%%%%%%%%%%%%%%%%


\section{Introduction}


The \acp{EoS} is a critical parameter characterizing the thermal evolution process of the universe \cite{dodelson2003modern}. 
However, little is known about \ac{EoS} of the early universe during the era between the inflation and radiation domination epochs. 
Assuming an adiabatic fluid, the density $\rho$ and pressure $P$ can be obtained via the Boltzmann approach \cite{Maggiore:2018sht}. 
In this context, it can be inferred that in the non-relativistic approximation, i.e., $p^2 \ll E^2$, the \ac{EoS} parameter is $w = P/\rho = 0$, and in the relativistic limit, i.e., $E^2 \simeq p^2$, we have $w= 1/3$. 
Therefore, in practical scenarios, it is anticipated that ordinary matter should yield an \ac{EoS} parameter $w$ ranging from $0$ to $1/3$. 
For example, in the case of thermal plasma exhibiting strong interaction of QCD, the evolution of the relativistic degrees of freedom results in the variation of $w$ within such a range \cite{Drees:2015exa,Saikawa:2018rcs,Carr:2019kxo}. 
If the early universe is filled with a scalar matter field $\varphi$, the \ac{EoS} parameter is primarily determined by the effective potential $V(\varphi)$ of the scalar field \cite{Turner:1983he,Poulin:2018dzj,Vikman:2004dc}. 
Recently, the oscillating scalar field model has been employed to resolve the Hubble tension \cite{Poulin:2018cxd,Hill:2020osr}, as well as to address the small-scale crisis of large-scale structure \cite{Hlozek:2014lca,Marsh:2015xka}. 
In addition, an epoch between the end of inflation and the onset of radiation domination may be early-matter dominated due to \acp{PBH}, oscillons, and so on \cite{Dalianis:2020gup,Domenech:2020ssp,Lozanov:2022yoy,Bhaumik:2020dor,Haque:2021dha}. 
For the early universe, it is reasonable to anticipate that \ac{EoS} may exhibit a mixed behavior, as described in the aforementioned scenarios. 
It is also well-motivated to explore the value of the \ac{EoS} parameter $w$ through phenomenological studies \cite{Hajkarim:2019nbx,Domenech:2019quo,Domenech:2020kqm,Domenech:2020ers}. 


Since the universe is transparent to gravitational waves \cite{Flauger:2019cam}, \ac{EoS} of the early universe can be directly probed via gravitational waves produced in the early universe. 
It is known that the \acp{SIGW} could be nonlinearly produced by linear cosmological curvature perturbations when the latter reentered the Hubble horizon after the end of inflation \cite{Ananda:2006af,Baumann:2007zm,Mollerach:2003nq,Assadullahi:2009jc,Espinosa:2018eve,Kohri:2018awv}. 
In middle 2023, the \ac{PTA} data releases showed significant evidence for a \ac{GWB} in the nano-Hertz band \cite{Xu:2023wog,Antoniadis:2023ott,NANOGrav:2023gor,Reardon:2023gzh}. 
This signal has been interpreted as \acp{SIGW} by the authors of Refs.~\cite{Franciolini:2023pbf,Inomata:2023zup,Cai:2023dls,Wang:2023ost,Liu:2023ymk,Abe:2023yrw,Ebadi:2023xhq,Figueroa:2023zhu,Yi:2023mbm,Madge:2023cak,Firouzjahi:2023lzg,Zhu:2023faa,You:2023rmn,Ye:2023xyr,HosseiniMansoori:2023mqh,Balaji:2023ehk,Jin:2023wri,Das:2023nmm}. 
Since the energy-density fraction spectrum of \acp{SIGW} is sensitive to the parameter of \ac{EoS} of the early universe \cite{Hajkarim:2019nbx,Domenech:2019quo,Domenech:2020kqm}, if such an interpretation could be confirmed in the future, we would anticipate that \acp{SIGW} are a powerful probe of the early universe, particularly for \ac{EoS} (see review in Ref.~\cite{Domenech:2021ztg}).  


In this work, for the first time, we will infer the \ac{EoS} parameter of the early universe during an era between the inflation and radiation domination epochs by conducting Bayesian analysis for the \ac{NG} 15-year dataset \cite{NANOGrav:2023gor}. 
Other parameters of the model will also be inferred simultaneously. 
Comparing our results with the existing ones in the literature \cite{NANOGrav:2023hvm,Antoniadis:2023zhi}, we will demonstrate the significant impacts of the \ac{EoS} parameter on the inference of these other parameters.  
On the other hand, once produced in the early universe, \acp{SIGW} behaved like additional relativistic ingredients, thereby increasing the expansion rate of the early universe. 
Therefore, we take into account the upper bounds on the effective number of relativistic species in the early universe from observations of \ac{BBN} \cite{Cooke:2013cba} and \ac{CMB} \cite{Clarke:2020bil}. 
Analyzing such a joint dataset will further refine the above results by more tightly constraining the parameter space.  


The remainder of the paper is arranged as follows. 
In \cref{sec:theory}, we develop a theory of SIGWs considering different \ac{EoS} parameters of the early universe. 
In \cref{sec:method}, we describe the methodology of data analysis used in this work, and present the results of parameter inferences as well as their physical implications. 
In \cref{sec:conclusion}, we provide concluding remarks. 




\section{SIGWs produced in the early universe with different EoS}\label{sec:theory}

Limited by our current observational capabilities on such small scales in cosmology, we have little knowledge of the state of matter in the early universe. In this paper, by assuming an \acp{EoS} $P = w \rho$ with varying values of $w \in [0,1]$ to phenomenologically describe the early-time universe preceding the radiation-dominated epoch, we study \acp{SIGW} produced both in the early-time epoch and radiation-dominated epoch, as well as the transition from the former to the latter. 


The perturbed spatially-flat \ac{FRW} metric in conformal coordinates is given by 
\begin{eqnarray}
{\rm d} s^2 & = & a^2 (\eta) \left\{ - \left(1 + 2 \phi\right)  {\rm d} t^2 + \left[ \left(1 - 2 \psi\right) \delta_{i   j} + \frac{1}{2} h_{i   j} \right] {\rm d} x^i  {\rm d} x^j \right\}\ ,
\end{eqnarray}
where $\eta$ is the conformal time, $a (\eta)$ is the scale factor, $\phi$ and $\psi$ are the Newtonian potential and curvature perturbations, respectively, known as the linear scalar metric perturbations, and $h_{i   j}$ describes the second-order tensor metric perturbations, i.e., \acp{SIGW}. 
Due to absence of anisotropic stress, we have $\psi\simeq\phi$. 
In the universe filled with perfect fluids with the \ac{EoS} parameter $w$, the equation of motion of \acp{SIGW} is given by
\begin{eqnarray}
h_{i   j}'' + 2\mathcal{H}h_{i   j}' - \Delta h_{i   j} & = & - 4 \Lambda_{i   j}^{a   b} \mathcal{S}_{a   b} ~,\label{eom1}
\end{eqnarray}
where $\Lambda^{a   b}_{i   j}$ is the transverse-traceless operator, $\mathcal{H}$ is the conformal Hubble parameter, and the source term $\mathcal{S}_{a   b}$ consisting of the linear scalar perturbations $\phi$ is given by
\begin{eqnarray}
\mathcal{S}_{a   b} & = & \frac{2 (5 + 3 w)}{3 (1 + w)} \partial_a \phi \partial_b \phi + \frac{4}{3 (1 + w) \mathcal{H}} (\partial_a \phi \partial_b \phi' + \partial_a \phi' \partial_b \phi') + \frac{4}{3 (1 + w) \mathcal{H}^2} \partial_a \phi' \partial_b \phi' \ .  \label{src}
\end{eqnarray}
The evolution of $\phi$ is determined by the master equation as follows 
\begin{eqnarray}
\phi'' + 3 (1 + w) \mathcal{H} \phi' - w \Delta \phi & = & 0 \ ,  \label{eom2}
\end{eqnarray}
where we have used the adiabatic speed of sound $c_s^2 = w$, and focus on  $0\leq w \leq 1$ for phenomenological studies. 


We follow the scenario in Ref.~\cite{Kohri:2018awv} for an extensive study of \acp{SIGW} during the transition from the early-time epoch of the universe to the radiation domination. 
The scale factor changes in different epochs and can be given by 
\begin{eqnarray}
a (\eta) & \simeq & \left\{\begin{array}{ll}
\left( \frac{\eta}{\eta_R} \right)^{\frac{2}{1 + 3 w}} & \eta > \eta_R\\
\frac{\eta}{\eta_R} & \eta < \eta_R
\end{array}  \right.  ~,
\end{eqnarray}
where $\eta_{R}$ is the transition time.
This also leads to a transition of Hubble parameter from $\mathcal{H}= 2 [(1 + 3 w) \eta]^{- 1}$ in the early-time epoch to $\mathcal{H}= \eta^{- 1}$ in the radiation-dominated epoch. 
Therefore, upon the above setup, we obtain the solutions of $h_{i   j}$ in momentum space based on Eq.~(\ref{eom1}), i.e., 
\begin{eqnarray}
  h_{i j, \bm{k}}^{(\text{ET})} &=& (k \bar{\eta})^{\beta} Y_{\beta} (k \eta) \int_0^{\eta} \frac{\dif \bar{\eta}}{k} \left[ \frac{\pi}{2} x^{1 - \beta} J_{\beta} (k \bar{\eta}) \Lambda^{a b}_{i j} (\hat{k}) \mathcal{S}_{a b, \bm{k}}^{(\text{ET})} \right] \nonumber\\
  &&\quad - (k \eta)^{\beta} J_{\beta} (k \eta) \int_0^{\eta} \frac{\dif \bar{\eta}}{k} \left[ (k \bar{\eta})^{1 - \beta} Y_{\beta} (k \bar{\eta}) \Lambda^{a b}_{i j} (\hat{k}) \mathcal{S}_{a b, \bm{k}}^{(\text{ET})} \right]\ , 
  \label{h1}\\
  h_{i j, \bm{k}}^{(\text{RD})} &=& \frac{\sin (k \eta)}{k \eta} h_{i j, \bm{k}}^{(\text{RD}, 0)} + \frac{\cos (k \eta)}{k \eta} h^{(\text{RD}, 1)}_{i j, \bm{k}} + \frac{\sin (k \eta)}{k \eta} \int_{\eta_R}^{\eta} \bar{\eta}  \cos (k \bar{\eta}) \Lambda^{a b}_{i j} (\hat{k}) \mathcal{S}_{a b, \bm{k}}^{(\text{RD})} \dif \bar{\eta} \nonumber\\
  &&\quad - \frac{\cos (k \eta)}{k \eta} \int_{\eta_R}^{\eta} \bar{\eta}   \sin   (k \bar{\eta}) \Lambda^{a b}_{i j} (\hat{k}) \mathcal{S}_{a b, \bm{k}}^{(\text{RD})} \dif \bar{\eta}\ , \label{h2}
\end{eqnarray}
where $Y_n(x)$ and $J_n(x)$ are Bessel functions of second and first kind, respectively, $\beta \equiv - 3 (1 - w) / [2 (1 + 3 w)]$ and the superscripts (ET) and (RD) represent ``early-time'' and ``radiation domination'', respectively.  
Here, we require continuity and differentiability of $h_{ij,\bm k}$, i.e. $h^{({\rm ET})}_{i   j, \bm{k}} = h_{i   j, \bm{k}}^{({\rm RD})}$ and ${h^{({\rm ET})}_{i   j, \bm{k}}}' {= h^{({\rm RD})}_{i   j, \bm{k}}}'$ at $\eta_R$, such that the time-independent quantities $h_{i   j, \bm{k}}^{({\rm RD}, 0)}$ and $h^{({\rm RD}, 1)}_{i   j, \bm{k}}$ can be determined as 
\begin{eqnarray}
  h^{({\rm RD}, 0)}_{i   j, \bm{k}} & = & k \eta_R \cos (k \eta_R) \left. {h_{i   j, \bm{k}}^{({\rm ET})}}' \right|_{\eta = \eta_R} + \left[\cos (k \eta_R) + k \eta_R \sin (k \eta_R)\right] \left. h_{i   j, \bm{k}}^{({\rm ET})} \right|_{\eta = \eta_R} \ , \\
  h^{({\rm RD}, 1)}_{i   j, \bm{k}} & = & k \eta_R \cos (k \eta_R) \left. h_{i   j, \bm{k}}^{({\rm ET})} \right|_{\eta = \eta_R} - \left[ \left. h_{i   j, \bm{k}}^{({\rm ET})} \right|_{\eta = \eta_R} + k \eta_R   \left. {h_{i   j, \bm{k}}^{({\rm ET})}}' \right|_{\eta = \eta_R} \right] \sin (k \eta_R) \ . \label{hrd} 
\end{eqnarray}
The expressions for the source terms in Eqs.~(\ref{h1}) and (\ref{h2}), denoted as $\mathcal{S}_{a b}^{({\rm ET})}$ and $\mathcal{S}_{a b}^{({\rm RD})}$, are provided in Eq.~(\ref{src}) with $\phi_{\bm{k}}^{({\rm ET})}$ and $\phi_{\bm{k}}^{({\rm RD})}$ defined for the early-time and radiation dominated epochs, respectively. 
Upon solving Eq.~(\ref{eom2}), we can derive their corresponding expressions, namely,
\begin{eqnarray}
\phi_{\bm{k}}^{({\rm ET})} & = &  _0 F_1 \left( \frac{7 + 9 w}{2 (1 + 3) w} ; - \frac{1}{4} w   k^2 \eta^2 \right) \Phi_{\bm{k}}\ ,
  \\
\phi_{\bm{k}}^{({\rm RD})} & = &   \left[ \frac{- \sqrt{3} k \eta \cos \left( {k \eta}/{\sqrt{3}} \right) + 3 \sin \left( {k \eta}/{\sqrt{3}} \right)}{(k \eta)^3} \right] \phi_{\bm{k}}^{({\rm RD}, 0)} \nonumber\\
 &  & + \left[ \frac{- 3 \cos \left( {k \eta}/{\sqrt{3}} \right) - \sqrt{3} k \eta \sin \left( {k \eta}/{\sqrt{3}} \right)}{(k \eta)^3} \right] \phi^{({\rm RD}, 1)}_{\bm{k}}\ , 
\end{eqnarray}
where $\Phi_{\bm{k}}$ stands for a random variable related to the primordial curvature perturbations, and $_0 F_1$ is the confluent hypergeometric function. 
The continuity and differentiability of $\phi_{\bm k}$ determine the expressions of $\phi_{\bm{k}}^{({\rm RD}, 0)}$ and $\phi_{\bm{k}}^{({\rm RD}, 1)}$ as follows 
\begin{eqnarray}
  \phi_{\bm{k}}^{({\rm RD}, 0)} & = & \left[ - \frac{((k \eta_R)^2 -
  9)}{\sqrt{3}} \cos \left( \frac{k \eta_R}{\sqrt{3}} \right) + 3 k \eta \sin
  \left( \frac{k \eta_R}{\sqrt{3}} \right) \right] \left.
  \phi_{\bm{k}}^{({\rm ET})} \right|_{\eta_R}  \nonumber\\
  &  & + \left[ \sqrt{3} k \eta_R \cos \left( \frac{k \eta_R}{\sqrt{3}}
  \right) + (k \eta)^2 \sin \left( \frac{k \eta_R}{\sqrt{3}} \right) \right]
  \left. {\phi_{\bm{k}}^{({\rm ET})}}' \right|_{\eta_R} \ , \\
  \phi_{\bm{k}}^{({\rm RD}, 1)} & = & \left[ \frac{- (k \eta_R)^3 + k \eta_R ((k
  \eta_R)^2 - 18) \cos \left( {2 k \eta_R}/{\sqrt{3}} \right) + \sqrt{3}
  (4 (k \eta_R)^2 - 9) \sin \left( {2 k \eta_R}/{\sqrt{3}} \right)}{6 \cos
  \left( {k \eta_R}/{\sqrt{3}} \right) + 2 \sqrt{3} k \eta_R \sin \left(
  {k \eta_R}/{\sqrt{3}} \right)} \right] \left. \phi_{\bm{k}}^{({\rm ET})}
  \right|_{\eta_R} \nonumber\\
  &  & + \left[ \frac{- 6 (k \eta_R)^2 \cos \left( {2 k \eta_R}/{\sqrt{3}}
  \right) - \sqrt{3} k \eta_R ((k \eta_R)^2 - 3) \sin \left( {2 k
  \eta_R}/{\sqrt{3}} \right)}{6 \cos \left( {k \eta_R}/{\sqrt{3}} \right) +
  2 \sqrt{3} k \eta_R \sin \left( {k \eta_R}/{\sqrt{3}} \right)} \right] \left.
  {\phi_{\bm{k}}^{({\rm ET})}}' \right|_{\eta_R} \ .
\end{eqnarray}
It is found that the early-time epoch leaves an imprint on the evolution of linear scalar perturbations during the radiation-dominated epoch, which subsequently affects the generation of gravitational waves. This suggests that \acp{SIGW} in the context of an early-time epoch cannot be simplistically understood as simply a superposition of waves across all epochs.
The additional influence of scalar perturbations on gravitational waves underscores the non-linear characteristics of Einstein's theory of gravity.


The energy-density fraction spectrum of \acp{SIGW} at $k\eta \gg 1$ during radiation dominated epoch is defined as \cite{Ananda:2006af,Baumann:2007zm,Mollerach:2003nq,Assadullahi:2009jc,Espinosa:2018eve,Kohri:2018awv} 
\begin{eqnarray}
  \Omega_{\rm{GW}} (k,\eta) & = & \frac{1}{48\mathcal{H}^2} \int \frac{ {\rm d}^3 k'}{(2 \pi)^3} \left( \frac{1}{2} \langle k^2 h_{i   j, \bm{k}}^{({\rm{RD}})} h_{i   j, \bm{k}'}^{({\rm{RD}})} {+ h_{i   j, \bm{k}}^{({\rm{RD}})}}' {h_{i   j, \bm{k}'}^{({\rm{RD}})}}' \rangle \right) 
  \nonumber\\
  & = & \frac{1}{48} \left( \frac{k}{\mathcal{H}} \right)^2 (\mathcal{P}_h (k,\eta)) + \bar{\mathcal{P}}_h (k,\eta) \ , \label{eds}
\end{eqnarray}
where $\mathcal{P}_h (k,\eta)$ and $\bar{\mathcal{P}}_h (k,\eta)$ are defined with the two-point correlations of $h_{i   j}$ and $h_{i   j}'$, respectively, 
\begin{eqnarray}
  \langle h_{i   j, \bm{k}}^{({\rm{RD}})} h_{i   j, \bm{\bar{k}}}^{({\rm{RD}})} \rangle & = & 2 (2 \pi)^3 \delta \left( \bm{k} + \bm{\bar{k}} \right) \frac{2 \pi^2}{k^3} \mathcal{P}_h (k, \eta) \ ,
  \label{cor1}\\
  \langle {h_{i   j, \bm{k}}^{({\rm{RD}})}}' {h_{i   j, \bm{\bar{k}}}^{({\rm{RD}})}}' \rangle & = & 2 (2 \pi)^3 \delta \left( \bm{k} + \bm{\bar{k}} \right) \frac{2 \pi^2}{k} \bar{\mathcal{P}}_h (k, \eta)) \ .
  \label{cor2}
\end{eqnarray}
In the small-scale limit, where $k \eta \gg 1$, it is permissible to use $\bar{\mathcal{P}}_h (k) =\mathcal{P}_h (k)$. 
The quantities $h_{i j, \bm{k}}^{({\rm{RD}})}$ and ${h_{i j, \bm{k}}^{({\rm{RD}})}}'$ in Eqs.~(\ref{cor1}) and (\ref{cor2}) are determined by Eq.~(\ref{hrd}). 
The stochastic nature of $h_{i   j, \bm{k}}^{({\rm{RD}})}$ 
is attributed to that of $\Phi_{\bm{k}}$. 
The two-point correlations of $\Phi_{\bm{k}}$ can be formally expressed as 
\begin{eqnarray}
  \left\langle \Phi_{\bm{k}} \Phi_{\bm{\bar{k}}} \right\rangle & = & (2 \pi)^3 \delta \left( \bm{k} + \bm{\bar{k}} \right) \left[ \frac{3 (1 + w)}{5 + 3 w} \right]^2 \frac{2 \pi^2}{k^3} \mathcal{P}_{\zeta} (k) ~,
\end{eqnarray}
where $\mathcal{P}_{\zeta} (k)$ is the power spectrum of primordial curvature perturbations $\zeta$.
For simplicity, we adopt a monochromatic spectrum, i.e., $\mathcal{P}_\zeta=A k_\ast\delta(k-k_\ast)$. 
Such a spectrum might be related the formation of \acp{PBH} \cite{Hawking:1971ei}, and has been extensively studied in literature (see review in Refs.~\cite{Carr:2020xqk,Sasaki:2018dmp} and references therein). 




% Figure environment removed






In Fig.~\ref{F1}, we depict the energy-density fraction spectrum for various parameters. For the \ac{EoS} parameter $w=1/3$, the outcome corresponds to \acp{SIGW} in a purely radiation-dominated epoch \cite{Baumann:2007zm,Kohri:2018qtx}. For the \ac{EoS} parameter $w=0$, the energy-density fraction spectrum experiences an enhancement due to the transition to a radiation-dominated epoch. This enhancement mechanism pertaining to the early-time matter domination model has also been discussed in Ref.~\cite{Inomata:2019ivs}.



To conduct parameter inferences in the following section, we consider the energy-density fraction spectrum of \acp{SIGW} in the present universe, which is given by 
 \begin{equation}
     h^{2}\Omega_{\mathrm{GW},0}(k) \simeq h^{2}\Omega_{\mathrm{r},0} \times \Omega_{\mathrm{GW}}(k,\eta) \ ,\label{eq:ogw0}
 \end{equation}
 where the expression of $\Omega_{\mathrm{GW}}(k,\eta)$ has been given in Eq.~(\ref{eds}), and the physical energy-density fraction of radiations in the present universe is given as $h^{2}\Omega_{\mathrm{r},0} \simeq 4.2\times10^{-5}$ with the dimensionless Hubble constant $h=0.6766$ measured by Planck satellite  \cite{Planck:2018vyg}. 





\section{Parameter inferences from the PTA, BBN and CMB datasets}\label{sec:method}



% Figure environment removed


% Figure environment removed





In this study, we consider two distinct combinations of datasets. The first dataset constitutes solely of \ac{NG} 15-year \ac{PTA} data \cite{NANOGrav:2023gor}, whereas the second dataset further incorporates the \ac{BBN} \cite{Cooke:2013cba} and \ac{CMB} \cite{Clarke:2020bil} constraints on the effective number of relativistic species.
In terms of the integrated energy-density fraction, defined as $\int_{k_{\mathrm{min}}}^{\infty}d\ln k \ h^{2}\Omega_{\mathrm{GW},0}(k)$, the upper limits are $1.3\times10^{-6}$ for \ac{BBN} \cite{Cooke:2013cba} and $2.9\times10^{-7}$ for \ac{CMB} \cite{Clarke:2020bil}. 
The lower bound of the integral is denoted as $k_{\rm min} = 2 \pi f_{\rm min}$, where $f_{\rm min}$ is $1.5\times10^{-11}\mathrm{Hz}$ for \ac{BBN} and $3\times10^{-17}$ for \ac{CMB} \cite{Maggiore:2018sht}. 
The \ac{BBN} and \ac{CMB} data have also been taken into account in studies of \acp{SIGW} \cite{Zhu:2023faa} and phase-transition gravitational waves \cite{Bringmann:2023opz}. 
For the energy-density fraction spectrum of \acp{SIGW}, as illustrated in Eq.~(\ref{eq:ogw0}), we examine three scenarios with the parameter $k_\ast\eta_{\mathrm{R}}$ set to $10$, $10^{2}$, and $10^{3}$, respectively. 
Consequently, the parameter space under investigation is defined by $\log_{10}A$, $\log_{10}(f_\ast/\mathrm{Hz})$, and $w$. 
Throughout this work, we take $k_\ast=2\pi f_\ast$. 
We apply uniform priors for these parameters, i.e., $\log_{10} A\in [-3.5,0]$, $\log_{10}(f_\ast/\mathrm{Hz})\in[-9,-4]$, and $w\in[0,1]$.
Guided by Ref.~\cite{NANOGrav:2023hvm}, we perform Bayesian parameter inferences across this parameter space, considering both dataset combinations. The \ac{BBN} and \ac{CMB} constraints are incorporated by assigning negative infinity to the log-likelihood if the integrated energy-density fraction exceeds the specified upper limits.



By performing Bayes analysis, we show the results for the posteriors of independent parameters in
Figs.~\ref{fig:posteriors1} and \ref{fig:posteriors2}. 
Here, we have considered the \ac{NG} 15-year \ac{PTA} data for the former, while further incorporated the \ac{BBN} and \ac{CMB} bounds for the latter. 
Other settings are the same for the scenarios considered in this work. 


When we take into account the \ac{PTA} data only, based on Fig.~\ref{fig:posteriors1}, we find that the parameter $A$ is bounded from lower, but not from upper, i.e., $A\gtrsim10^{-3}$. 
Meanwhile, the peak frequency $f_\ast$ is bounded from $\sim10^{-7}$ Hz to $\sim10^{-5}$ Hz, indicating a micro-Hertz frequency band. 
Reminding that \ac{PTA} is sensitive to the nano-Hertz band, i.e., $\sim10^{-9}-10^{-7}$ Hz, and considering the spectral profiles in Fig.~\ref{F1}, we find that only the infrared tails of the energy-density fraction spectra can fit well the \ac{NG} 15-year dataset. 
Depending on scenarios, loose constraints have been shown for the \ac{EoS} parameter $w$. 


In contrast, incorporating the \ac{BBN} and \ac{CMB} bounds have refined the above results by more tightly constraining the parameter space, as shown in Fig.~\ref{fig:posteriors2}. 
The parameter region, which makes the integrated energy-density fraction to exceed the specified upper limits, has been excluded in our data analysis. 
For all scenarios, therefore, we find that the spectral amplitude $A$ is not only bounded from lower, but also from upper, i.e., $A\sim10^{-3}-10^{-1}$. 
The peak frequency $f_\ast$ is also more tightly constrained, i.e., $f_{\ast}\sim10^{-7}-10^{-6}$ Hz. 
In addition, we still obtain loose constraints on the \ac{EoS} parameter $w$. 
However, the most important result may be concerned to exclusions of the early matter domination with $w=0$ at a more-than-$2\sigma$ confidence level. 


As shown in Figs.~\ref{fig:posteriors1} and \ref{fig:posteriors2}, the radiation domination with the \ac{EoS} parameter $w=1/3$ is compatible with our dataset combinations for the scenarios considered in this work, and an epoch of kination domination with $w=1$ \cite{Gouttenoire:2021jhk} is not forbidden. 





\section{Conclusions and discussions}\label{sec:conclusion}


In this work, we conducted a study of the equation of state of the early universe, by assuming the \ac{SIGW} interpretation of the recent \ac{PTA} data releases. 
We computed the energy-density fraction spectrum of \acp{SIGW} during the transition process from an arbitrary $w\in[0,1]$ to $w=1/3$. 
See Fig.~\ref{F1} for illustrating examples. 
To perform Bayes parameter inferences, besides the \ac{NG} 15-year \ac{PTA} dataset, we analyzed the dataset combination incorporating this dataset with the \ac{BBN} and \ac{CMB} bounds. 
To our knowledge, it is the first time to perform Bayes data analysis for the considered topic. 
The results for the posteriors of model parameters have been shown in Figs.~\ref{fig:posteriors1} and \ref{fig:posteriors2}. 
For the energy-density fraction spectrum of \acp{SIGW}, we found the parameter region, i.e., the spectral amplitude $A\sim10^{-3}-10^{-1}$ and spectral peak frequency $f_\ast\sim10^{-7}-10^{-6}$ Hz, allowed by the joint dataset combination. 
In particular, we found that the early matter domination with the \ac{EoS} parameter $w=0$ is excluded at the $>2\sigma$ confidence level. 
However, we showed that the radiation domination with $w=1/3$ is still compatible with our joint analysis. 
In addition, we found that the kination domination with $w=1$ for the early universe is not forbidden by the current dataset combinations. 
The results of this work can be further tested with future observations. 



We generalized the theory of \acp{SIGW} to characterize a transition from the early-time epoch with an arbitrary $w$ to the radiation-dominated epoch. 
In fact, the production of \acp{SIGW} in the early universe with an arbitrary $w$ has been studied in Refs.~\cite{Hajkarim:2019nbx,Domenech:2019quo, Domenech:2020kqm}. 
In these models, gravitational waves propagate freely during subsequent radiation-dominated epoch, but are no longer produced. 
However, for \acp{SIGW}  produced during the early matter-dominated epoch and subsequent radiation-dominated epoch, there could be an enhanced production of \acp{SIGW} due to a sudden transition from the early-matter domination to the radiation domination \cite{Inomata:2019ivs}. 
It has been shown that the energy-density fraction spectrum of \acp{SIGW} produced via this enhancement mechanism can dominate the total spectrum. Therefore, to study \acp{SIGW} in a comprehensive way, we considered \acp{SIGW} produced in both the early-time epoch with an arbitrary $w \in [0,1]$ and subsequent radiation-dominated epoch with $1/3$, following the approach of Refs.~\cite{Kohri:2018awv,Inomata:2019ivs}. 




Our findings revealed that the \ac{EoS} parameter $w=0$ is inconsistent with the joint dataset examined in this study. 
This result also showed the importance of \ac{BBN} and \ac{CMB} bounds for studies of \acp{SIGW} produced in the early universe with varying values of \ac{EoS}. 
Given that the parameter interval of $f_\ast$ is deduced to be roughly $10^{-7}-10^{-6}$ Hz, the aforementioned result suggested that an extended period of early-matter domination, lasting in excess of $10^{7}$ seconds (thereby yielding $k_{\ast}\eta_{\mathrm{R}} \gg 1$), is not corroborated by the data. 
In actuality, when the universe was approximately $10^{7}$ seconds old, its temperature would be around $10^{-3}$ MeV. 
This temperature is lower than that which prevailed during \ac{BBN} \cite{dodelson2003modern}. 
Given the fact that early-matter domination pertains to the cosmic evolutionary process spanning from the cessation of inflation to the commencement of radiation domination, our research offered fresh perspectives for investigations into early universe physics \cite{Dalianis:2020gup,Domenech:2020ssp,Lozanov:2022yoy,Bhaumik:2020dor,Haque:2021dha}. 



The results of this work relied on the assumption that the evidence of \ac{GWB} in the \ac{PTA} band can be interpreted with the theory of \acp{SIGW}. 
However, such an assumption is not necessarily true, since there are also alternative interpretations, as suggested by Refs.~\cite{NANOGrav:2023hvm,Antoniadis:2023zhi}. 
More generally, we should consider these possible origins of \ac{GWB} to make conclusive information of \ac{EoS} of the early universe. 
In particular, the astrophysical origin of \ac{GWB}, i.e., the inspiralling \ac{SMBH} binaries, should be taken into account in the realistic data analysis \cite{NANOGrav:2023hfp,Antoniadis:2023zhi}. 
Considering the above ingredients as well as large uncertainties of the current \ac{PTA} observations, it is challenging to provide confident conclusions to \ac{EoS} of the early universe at the current stage. 
Nevertheless, our study still showed that \acp{SIGW} could be a powerful probe of the early-universe physics. 






	\begin{acknowledgements}
		We acknowledge Yan-Heng Yu for helpful discussions. S.W. is supported by the National Natural Science Foundation of China (Grant NO. 12175243). Z.C.Z. is supported by the National Natural Science Foundation of China (Grant NO. 12005016). X.Z. is supported by the National SKA Program of China (Grants Nos. 2022SKA0110200 and 2022SKA0110203) and the National Natural Science Foundation of China (Grants Nos. 11975072, 11835009, and 11805031). 
	\end{acknowledgements}
	


	
	
	\bibliography{PGW}
	%\bibliographystyle{aasjournal}
	%\bibliographystyle{unsrt}
        %\bibliographystyle{apsrev4-1}
	
	
	
\end{document}



