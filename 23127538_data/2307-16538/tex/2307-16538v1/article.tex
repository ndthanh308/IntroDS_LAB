% ****** Start of file apssamp.tex ******
%
%   This file is part of the APS files in the REVTeX 4.2 distribution.
%   Version 4.2a of REVTeX, December 2014
%
%   Copyright (c) 2014 The American Physical Society.
%
%   See the REVTeX 4 README file for restrictions and more information.
%
% TeX'ing this file requires that you have AMS-LaTeX 2.0 installed
% as well as the rest of the prerequisites for REVTeX 4.2
%
% See the REVTeX 4 README file
% It also requires running BibTeX. The commands are as follows:
%
%  1)  latex apssamp.tex
%  2)  bibtex apssamp
%  3)  latex apssamp.tex
%  4)  latex apssamp.tex
%
\documentclass[%
reprint,
%superscriptaddress,
%groupedaddress,
%unsortedaddress,
%runinaddress,
%frontmatterverbose, 
%preprint,
%onecolumn,
%preprintnumbers,
%nofootinbib,
%nobibnotes,
%bibnotes,
 amsmath,amssymb,
 aps,
prl,
%pra,
%prb,
%rmp,
%prstab,
%prstper,
%floatfix,
]{revtex4-2}

\usepackage{graphicx}% Include figure files
\usepackage{dcolumn}% Align table columns on decimal point
\usepackage{bm}% bold math
%\usepackage{hyperref}% add hypertext capabilities
%\usepackage[mathlines]{lineno}% Enable numbering of text and display math
%\linenumbers\relax % Commence numbering lines

\usepackage[T2A]{fontenc}
\usepackage[utf8]{inputenc}

\usepackage[usenames]{color}
\usepackage{colortbl}
\usepackage{amsmath}
\usepackage{amssymb}



%\usepackage[showframe,%Uncomment any one of the following lines to test 
%%scale=0.7, marginratio={1:1, 2:3}, ignoreall,% default settings
%%text={7in,10in},centering,
%%margin=1.5in,
%%total={6.5in,8.75in}, top=1.2in, left=0.9in, includefoot,
%%height=10in,a5paper,hmargin={3cm,0.8in},
%]{geometry}

\begin{document}

\preprint{APS/123-QED}

\title{Phase diagram of ferromagnetic semiconductor. The origin of superparamagnetism.}% Force line breaks with \\
%\thanks{A footnote to the article title}%

%Ising model of ferromagnetic semiconductor. Density of state approach.
%Phase diagram of ferromagnetic semiconductor. Statistical origin of superparamagnetic phase.

\author{N. A. Bogoslovskiy}
\author{P. V. Petrov}
 \email{pavel.petrov@gmail.com}
\author{N. S. Averkiev}
\affiliation{Ioffe Institute, Russian Academy of Science, 194021 St. Petersburg, Russia}


\date{\today}% It is always \today, today,
             %  but any date may be explicitly specified

\begin{abstract}

We study the theoretical model of a ferromagnetic semiconductor as a system of randomly distributed
Ising spins with a long-range exchange interaction. Using the density-of-states approach, we
analytically obtain the magnetic susceptibility and heat capacity over a wide range of temperatures
and magnetic fields. It is shown that the finite system of spins in magnetic field less than
a certain critical field is in a superparamagnetic state due to thermodynamic fluctuations.
The complex phase structure of a ferromagnetic semiconductor is discussed.

%\begin{description}
%\item[Usage]
%Secondary publications and information retrieval purposes.
%\item[Structure]
%You may use the \texttt{description} environment to structure your abstract;
%use the optional argument of the \verb+\item+ command to give the category of each item. 
%\end{description}
\end{abstract}

%\keywords{Suggested keywords}%Use showkeys class option if keyword
                              %display desired
\maketitle

During the past years, significant progress has been made in the field of ferromagnetic semiconductor materials. 
In the pioneering Ohno's work~\cite{10.1063/1.118061}, it was demonstrated that ferromagnetism in GaAs
doped with Mn is associated precisely with the properties 
of the doped semiconductor, and not with the presence of MnAs inclusions. 
Since that moment, the list of ferromagnetic semiconductor materials has been constantly expanding,
while the experimentally observed values of the Curie temperature 
have reached room temperature \cite{PhysRevLett.99.127201, PhysRevLett.117.227202}. 
On the other hand, a clear theoretical description of ferromagnetism in semiconductors has not yet been achieved. 
Various mechanisms of ferromagnetic exchange between impurities have been proposed: exchange mediated by delocalized holes \cite{PhysRevB.63.195205}, 
percolation of bound magnetic polarons \cite{PhysRevLett.88.247202} and hopping mechanism \cite{PhysRevLett.99.227205}.
But even for GaAs:Mn, the most studied to date, there is no theory covering all the main experimental observations. 
It is not even clear whether the valence or conduction bands are responsible for ferromagnetism. \cite{10.1038/nmat3317}.

One of the most significant differences of semiconductor ferromagnetic
materials is the random distribution of interacting spins, while in
conventional magnetics they are located at the nodes of a regular lattice.
Spin systems with spacial disorder have already been studied theoretically, but mostly
with antiferromagnetic sign of interaction and by means of numerical simulation
\cite{PhysRevLett.48.344, McLenaghan_1984, doi:10.1063/1.335065,
bogoslovskiy2019impurity, bogoslovskiy2021spin}.

In a number of works \cite{OHNO1999110, sawicki2010experimental, PhysRevB.94.075205, PhysRevMaterials.1.054401, Yuan_2018, PhysRevB.97.115201, 10.1063/5.0031605}, 
it was experimentally shown that, a ferromagnetic transition in doped semiconductors has a complex nature. As the concentration
of magnetic impurities increases, a material first passes from the paramagnetic to the superparamagnetic, and then to the ferromagnetic phase. 
The most probable scenario of such a transition is follows. 
Due to the random distribution of magnetic impurities in the sample, there are regions with a local concentration 
higher than a certain critical concentration. Such regions we call clusters for brevity.
The exchange interaction aligns the spins inside the clusters in one direction.
Due to thermodynamic fluctuations, which could not be neglected in finite systems,
the magnetic moment of such clusters is not fixed, and they behave like superparamagnets.
As the concentration increases the growing clusters merge in one macroscopic ferromagnet.
The purpose of our work is to theoretically investigate the physical properties of these clusters depending on temperature,
magnetic field and the number of spins in the cluster.

In the Letter, we show that the statistical approach makes it possible
to analytically calculate the density of states $g(E, M)$ of the cluster of randomly distributed spins
as a function of the total exchange energy $E$ and magnetic moment $M$. It should be
emphasized that, in contrast to the one-electron density of states,
here we are talking about the states of the entire
system of spins, the total exchange energy, and the total magnetic moment
of the system. A similar approach has already been used to study the Ising
problem on a regular lattice both numerically~\cite{PhysRevLett.86.2050} and analitically~\cite{e23121665}.
If the density of states is known, it is easy to find the partition function
and other physical properties of the system.

We consider a finite system of $N$ randomly distributed spins rigidly fixed in space. 
We use an Ising model with a ferromagnetic long-range interaction $J(r)$ to describe the energy of the system.
Each spin can be in one of two states with a magnetic moment $\mu s$ where $s=\pm1$. 
Then the Hamiltonian of the system is given by:
\begin{equation}
H = -\frac{1}{2} \sum_{i \ne j} J(r_{ij}) s_i s_j - \sum_{i} \mu B s_i 
\label{hamiltonian}
\end{equation}

The first term is the total exchange energy
\begin{equation}
E = -\frac{1}{2} \sum_{i \ne j} J(r_{ij}) s_i s_j = 
-\frac{1}{2} \sum_i J_i s_i,
\label{exchange_energy}
\end{equation}
where $- J_i s_i = - s_i \sum_{j \ne i} J(r_{ij})s_j$ is the exchange interaction energy of the spin $i$ with all other spins. 
All calculations here are performed with the hydrogen-like dependence of
the exchange energy $J$ on the distance \cite{1964Gorkov, PhysRev.134.A362} in a three-dimensional space.
\begin{equation}
J(r) = J_0 \left( \frac{r}{a} \right)^{5/2} \exp{\left(-\frac{2r}{a} \right)},
\label{hydrogen}
\end{equation}
where $a$ is the Bohr radius. However, the solution can be easily generalized to a wider class of functions $J(r)$ (RKKY-type, for instance),
as well as to an arbitrary finite space dimension.

The exchange energy is the sum of $N$ random identically distributed energies $-J_i s_i$.
In accordance with the central limit theorem, the distribution of $E$ converges to the normal distribution as the number of spins in the system increases.
It is known that for the finite sum of non-gaussian random variables the distribution tails deviate from the normal one~\cite{PhysRevLett.89.070201}.
However, in one of the previous papers, we have shown numerically that if the spin concentration $na^3$ is high enough, 
the one-spin energy $-J_i s_i$ has a normal distribution \cite{bogoslovskiy2021spin}. 
In that case, the gaussian approximation is applicable.
The critical concentration could be estimated using the central limit theorem in Lindeberg's formulation.
The distribution of a sum of random variables is normal if none of the terms makes a dominant contribution to the sum.
The maximum contribution to the one-spin energy comes from the terms for which the value of $4\pi r^2 J(r)$ is maximal.
For $J(r)$ of the form (\ref{hydrogen}), this distance is $9/4a$.
If it is greater than the average distance to the nearest neighbor, 
then the contribution of the nearest neighbor is not dominant, and the one-spin energy distribution is normal.
Using the formula for average distance to the nearest neighbor $\overline{r} = \Gamma(4/3)/(4\pi n/3)^{1/3}$ from \cite{RevModPhys.15.1}, 
one can estimate the critical concentration $n_{c}a^3 = 0.015.$

We denote the number of “down” spins by $q$, the number of “up” spins by $N - q$, and
the dimensionless magnetic moment per one spin by $m =\frac{M}{\mu N} =1-\frac{2q}{N}$.
The system of $N$ spins in the Ising model has $2^N$ possible states, and
the number of states with a fixed value of the magnetic moment is equal to binomial coefficient $\binom{N }{q}$.
If we assume that these states are normally distributed in energy, the density of states with a given $m$ is
\begin{equation}
g_m(E) = \binom{N}{\frac{N(1-m)}{2}} \frac {1} {\sqrt {2 \pi } \sigma_m} 
\exp \Biggl( {- \frac { \left ( E - \overline E_m \right ) ^2} {2 \sigma_m^2}} \Biggr)
\label{gauss}
\end{equation}
In order to determine the density of states, it is necessary to derive
the average energy $\overline {E_m}$ and the variance $\sigma_m$ for each $m$ averaged.
Averaging over configurations, one can replace the sum over discretely located spins
by an integral over space with a uniform distribution of the magnetic moment with a density of $nm$.
The average energy in the limit of large $N$ is
\begin{gather}
\overline E_m = -\frac{N}{2} \overline {s_i} \overline {J_i} =
- \frac{mN}{2} \int \limits_0^\infty {n m J(r) 4\pi r^2 dr} =
- \frac{m^2 N}{2} \overline {J_{1}}  
\nonumber  \\
\overline {J_{1}} = \frac{945 \pi}{2^{8}} \sqrt {\frac {\pi}{2}} J_0 n a^3
\label{mean_E}
\end{gather}

Using the same line of reasoning, after cumbersome calculations, we obtain the following expressions for the variance
\begin{gather} 
\sigma_m^2 = \overline {E^2} - \overline {E}^2 =
\left( 1 - m^4 \right) N \sigma^2_1 \nonumber \\
 \sigma^2_1 = \frac{7!}{2^{15}} \pi J_0^2 n a^3
\label{sigma}
\end{gather}

It is important that all further calculations do not depend on the form of $J(r)$. 
It is only necessary that the values $\overline {J_{1}}$ and $\sigma_1$ be finite and the concentration exceeds $n_c$.

Using the Stirling formula the binomial coefficient in (\ref{gauss}) can be rewritten as
\begin{equation}
\binom{N}{\frac{N(1-m)}{2}} = \sqrt{\frac{2}{\pi N}} \frac {1}{\sqrt{1 - m^2}} \exp \left( N p(m) \right)
\end{equation}
Here we introduce the notation
\begin{equation} 
\nonumber
p(m) = \ln 2 - \frac{1-m}{2} \ln(1-m) - \frac{1+m}{2} \ln(1+m)
\end{equation}

For convenience we introduce a dimensionless energy per one spin $e = \frac{E} {N \overline {J_1}}$
and a dimensionless standart deviation $s_1 = \frac{\sigma_1}{\overline {J_1}}$.  
For large $N$, we assume that the average magnetic moment $m$ varies continuously in the range from -1 to 1, 
and determine the density of states in terms of energy and magnetic moment.
\begin{equation}
\begin{aligned}
g(e, m) = \frac{N}{2 \pi s_1 (1 - m^2) \sqrt{1 + m^2}} \times  \\
\exp \left ( { N p(m) - \frac { N \left ( e + m^2 / 2 \right ) ^2} {2 s_1^2 (1-m^4)}} \right ) 
\label{gauss2}
\end{aligned}
\end{equation}
% Figure environment removed
It is noteworthy that $N^{-1} \ln g(e, m)$ in the limit of large $N$ is universal and does not depend on $N$ (figure \ref{g_em}).

The formulas (\ref{gauss}--\ref{sigma}) were independently verified by two numerical methods, the Wang-Landau and the direct sampling method. 
The Wang-Landau algorithm \cite{PhysRevLett.86.2050, doi:10.1119/1.1707017} is a non-Markovian random walk in the phase space, taking into account the statistics of previous visits. 
The calculations were carried out using parallel computing, the density of states was calculated separately for the limited set values of $m$~\cite{10.18721/JPM.161.301}. 
Random walks are performed by simultaneously flipping two randomly chosen antiparallel oriented spins in order to keep $m$ constant. 
Our calculations for $N=8192$ showed that  $g(e, m)$ for each $m$ have the form of a normal distribution with insignificant deviations on the distribution tails. 
The calculated dependences of the average energy and variance of distributions on $m$ agree well with the theoretical
values (\ref{mean_E}) and (\ref{sigma}).
% Figure environment removed

The direct sampling method consists in sequential calculation of the system energies with
a random spin configuration, but with a fixed value of $m$. 
After the accumulation of a sufficiently large number of samples, the first four central moments of the distribution were calculated using the obtained samples of energy. 
The total energy of the spin system (\ref{exchange_energy}) can be represented as $E(\mathcal{S}) = 1/2 (\mathcal{J}\times \mathcal{S})\cdot \mathcal {S}$. 
Here $\mathcal{J}$ is the matrix of interaction energies $J_{i,j}$ of spins $i$ and $j$, $\mathcal{S}$ is the column of spin variables. 
Using parallel computing on the GPU with the implementation of CUDA technology for the julia language \cite{besard2018juliagpu},
we were able to significantly increase the performance of scalar product calculations
and increase the size of the system up to $N=32768$. 
Obtained dependences of average energy $\overline{E}(\mathcal{S})$ and variance $\sigma^2 = \overline {E^2}(\mathcal{S}) - \overline{E}(\mathcal{S})^2 $ also agree 
with the theoretical formulas (\ref{mean_E}, \ref{sigma}), while the third and fourth moments do not depend on $m$ and are equal to $0 \pm 0.01$ and $3 \pm 0.02$, respectively as expected for a normal distribution.
The combined results of numerical computations in comparision with theoretical equations (\ref{mean_E}) and (\ref{sigma})
are presented on figure~\ref{fig:mpr}.

In what follows we consider a system in which the density of states is given by (\ref{gauss2}). 
Let the system have temperature $T$ and be in an external magnetic field $B$.
The probability for the system to be in a certain state can be described by the Boltzmann distribution with energy $E - \mu B N m$.
Here, as above, $E$ denotes only the exchange energy. 
For convenience, we introduce a dimensionless temperature $t = kT / \overline {J_1}$ and a dimensionless magnetic field $\beta = \mu B / \overline {J_1}$.
In this notation the probability density for the system to have energy $e$ and magnetic moment $m$ at temperature $t$ and in an external magnetic field $\beta$ is
\begin{equation}
\begin{aligned}
f (e, m, t, \beta) = \frac{1}{Z(t, \beta)} \frac{N}{2 \pi s_1 (1 - m^2) \sqrt{1 + m^2}} \times \\ 
 \exp \left ( { N p(m) - \frac { N \left ( e + m^2 / 2 \right ) ^2} {2 s_1^2 (1-m^4)}} - \frac {N \left( e - \beta m \right)} {t} \right ) 
\end{aligned}
\label{probability}
\end{equation}
Here $Z=\iint g(e, m) e^{-N(e-\beta m)/t} de\,dm$ is the partition function. After intergration over energy
\begin{equation}
\begin{aligned}
Z(t, \beta) = \sqrt{\frac{N}{2 \pi}}  \int \frac{1}{\sqrt{1 - m^2}}  \times \qquad \qquad \\
\exp {\left ( N p(m) + N\frac {s_1^2 (1 - m^4) + m^2 t + 2 \beta m t} {2 t^2} \right )} dm
\label{partition}
\end{aligned}
\end{equation}

Let us demonstrate the connection between our method and conventional approaches
such as the Curie-Weiss theory and the Landau theory.
In the case of large $N$ the integral over $m$ in (\ref{partition}) can be calculated analytically using the Laplace's method.
The value $m_0$ which correspond to the maximum of the exponent can be found from 
\begin{equation}
- \frac{2 s_1^2 m_0^3}{t^2} + \frac{m_0}{t} + \frac{\beta}{t} + \frac{1}{2} ln \left( \frac{1-m_0}{1+m_0} \right) = 0
\label{m0}
\end{equation}

Equation (\ref{m0}) can have one or three roots in depends on $t$ and $\beta$.
First we consider the case when the equation (\ref{m0}) has one root. 
The average magnetic moment calculated by the Laplace's method is $\overline{M} (t, \beta) = \mu N m_0$. 
The magnetic susceptibility can be obtained by dividing the variables and differentiating the equation (\ref{m0}).
\begin{equation} 
\chi = \frac{\partial \overline{M}}{\partial B}  = 
\frac{\mu^2 N}{\overline{J_1}}  \frac {(1-m_0^2)}{t - (1-m_0^2) + \frac{6 s_1^2}{t} (m_0^2 -m_0^4)}
\label{chi}
\end{equation}

In weak magnetic fields $\beta \ll 1$ the value of $m_0$ is small and the magnetic susceptibility (\ref{chi}) converges to
\begin{equation}
\chi = \frac{\mu^2 N}{\overline{J_1} (t - 1)} 
\end{equation}

Note that this expression coincides with the Curie-Weiss law with the Curie temperature $t_c = 1$.

After integration (\ref{partition}) using the Laplace's method
\begin{gather}
Z(t, \beta) = \sqrt {\frac{t^2}{ \left( 6 s_1^2 m_0^2 - t \right) (1 - m_0^2) + t^2}} \times \nonumber \\
\exp {\left ( N \left( \frac {s_1^2 (1-m_0^4) + m_0^2 t + 2 \beta m_0 t} {2 t^2} +  p(m_0) \right) \right)}
\label{Z0}
\end{gather}

In the limit of large $N$, the pre-exponential factor in the expression (\ref{Z0}) could be discarded.
In a weak magnetic field $m_0 \ll 1$ and $p(m_0) \approx \ln 2 - \frac{m_0^2}{2} - \frac{m_0^4}{12}$.
In these approximations, the thermodynamic free energy $F\,=\,-kT\,ln\,Z$ is
\begin{equation}
\begin{aligned}
F(t, \beta) =  N \overline {J_1} \biggl(- t \ln 2 - \frac {s_1^2}{2t} + \quad \\
+ \frac{t - 1}{2} m_0^2 + \left( \frac {s_1^2}{2t} + \frac{1}{12} \right) m_0^4  - \beta m_0 \biggr)
\label{Landau}
\end{aligned}
\end{equation}

% Figure environment removed

This expression coincides with the Landau's theory of phase transitions \cite{landau2013statistical},
and the average magnetic moment $m_0$ has the meaning of the order parameter.
The phase transition from the paramagnetic to the ferromagnetic phase occurs at a temperature $t = 1$, when the coefficient of $m_0^2$ changes its sign.
It is noteworthy that our model is applicable for arbitrary values of the magnetic field, not only in the limit
of low fields as in the Landau theory.

If (\ref{m0}) has three roots, the exponent (\ref{partition}) has two local maxima. 
Let's denote the corresponding roots of the equation (\ref{m0}) as $m_+$ and $m_{-}$. 
In this case, the partition function can also be calculated using the Laplace's method similar to (\ref{Z0}). 
The partition function is expressed as the sum of two terms, which we denote as $Z_+$ and $Z_{-}$, respectively.
In this notation the average magnetic moment is
\begin{equation}
\overline{M} = \mu N \frac{m_+ Z_+ + m_- Z_-}{Z_+ + Z_-}
\label{mean_M}
\end{equation}

The Laplace's method is not applicable in the vicinity of the phase transition.
However, the average magnetic moment and susceptibility at an arbitrary temperature can be
calculated numerically. $$\overline m
(t, \beta) = \iint m f (e, m, t, \beta) de~dm; \;\;\;\;\;\; \chi = \frac{\mu^2 N}{J_1} \frac{\partial \overline{m}}{\partial \beta} $$

Similarly, we find the average exchange energy per spin and the heat capacity.
$$ \overline e (t, \beta) = \iint e f (e, m, t, \beta) de~dm; \;\;\;\;\;\;\;\; C= kN \frac{\partial \overline e}{ \partial t} $$
% Figure environment removed
Figure \ref{magnetic_susceptibility} shows the magnetic susceptibility.
The white line is the maximum versus temperature at given magnetic field.
In high fields, the maximum shifts to higher temperatures and noticeably broadens.
In weak magnetic fields, the maximum shifts strongly down in temperature. 
This drastically differs from the behaviour of the infinite system where the maximum
converges to $t = 1$ for small $\beta$ (black dashed line on figure \ref{magnetic_susceptibility}). This effect strongly depends on $N$.
If we define the critical field $\beta_c$ as a field at which the maximum of magnetic susceptibility crosses the line $t = 1$,
then $\beta_c \rightarrow 0$ as the $N \rightarrow \infty$ 

The figure \ref{heat} shows the heat capacity in coordinates ($t, \beta$), the white line is also the position of the maximum. 
It is noteworthy that, in contrast to the magnetic susceptibility, the maximum of the heat capacity is close to $t =1$ 
and constant in low magnetic fields. At the point $\beta = \beta_c$ the heat capacity maximum abruptly starts
to depend on $\beta$ and decreases. At high fields, it shifts towards high temperatures, as the maximum of magnetic susceptibility.

Assuming that the phase transition correspond to the maxima of magnetic susceptibility and heat capacity, 
one can plot the phase diagram shown in figure \ref{phase}.
It should be noted that the positions of the maxima of the heat capacity and magnetic susceptibility do not coincide.
The maximum of the heat capacity is associated with the parallel orientation of individual spins inside a cluster.
The maximum of the magnetic susceptibility is associated with the orientation of the magnetic moment of the cluster by the magnetic field.
In weak magnetic fields only the average square of the magnetic moment changes, while the average magnetic
moment remains close to zero \cite{bogoslovskiy2021spin, demishev2022spin}.
This is a consequence of the fact that we are considering a system with a large but finite number of spins $N$
whose properties are governed by thermodynamic fluctuations.
% Figure environment removed

The dependence on $N$ can be understood from (\ref{Z0}) and (\ref{mean_M}).
The two terms in (\ref{mean_M}) $m_+Z_+$ and $m_-Z_-$ exponentially depend on $N$.
This means that the ratio between them is strongly dependent on the size of the system.
At small $N$, two terms are comparable, thermodynamic fluctuations are large, and the cluster is in the superparamagnetic state.
As $N$ increases, the number of states with a magnetic moment directed along the magnetic field becomes much larger,
and the cluster becomes ferromagnetic. This behavior is illustrated on figure \ref{DOS} where the probability density
function in diffirent phases is plotted using (\ref{probability}). In the case of an infinite system the probability
maxima are just $\delta$-functions. If $N$ is finite, the system has non-zero probability to be
in the number of states around $m_0$ or $m_+$ and $m_-$ maxima and switch between them.

The critical field $\beta_c$, above which the superparamagnetic phase does not exist,
decreases with increasing $N$ according to a power law (see inset on figure \ref{phase}).
In a semiconductor with magnetic impurities,
many oriented magnetic moments of individual clusters create a Weiss molecular field.
As the concentration of magnetic impurities increases, both the average number of spins in these clusters
and the Weiss field increase.
At some point, the value of this field exceeds the value of the critical field
$\beta_c$ and the semiconductor passes into the ferromagnetic
state.

We also would like to note that from (\ref{mean_E}) the average exchange energy linearly depends
on the spin concentration for an arbitrary $J(r)$.
Therefore, the model predicts that the Curie temperature $T_c(n) = \overline{J_1}$ linearly depends on the spin concentration, 
at least in the region where $J(r)$ is independent of $n$.
% Figure environment removed

In conclusion, an analytical model is developed that describes a system of a finite number of randomly distributed spins, 
taking into account the long-range Ising-type exchange interaction and the magnetic field.
The phase diagram in coordinates ($t, \beta$) explains the complex nature of the phase transition in ferromagnetic semiconductors
and qualitatively agrees with experimental results.
The origin of the intermediate superparamagnetic phase and its relation to thermodynamic fluctuations
in finite-size systems are explained.

We acknowledge support from Russian Science Foundation (Grant No. 23-22-00333)
The numerical calculation were performed using computational resources of the supercomputer center in Peter the Great Saint-Petersburg Polytechnic University Supercomputing Center.
%\vspace*{1.5in} 


% The \nocite command causes all entries in a bibliography to be printed out
% whether or not they are actually referenced in the text. This is appropriate
% for the sample file to show the different styles of references, but authors
% most likely will not want to use it.
% \nocite{*}
\newpage

%\bibliography{article}% Produces the bibliography via BibTeX.
\begin{thebibliography}{30}%
\makeatletter
\providecommand \@ifxundefined [1]{%
 \@ifx{#1\undefined}
}%
\providecommand \@ifnum [1]{%
 \ifnum #1\expandafter \@firstoftwo
 \else \expandafter \@secondoftwo
 \fi
}%
\providecommand \@ifx [1]{%
 \ifx #1\expandafter \@firstoftwo
 \else \expandafter \@secondoftwo
 \fi
}%
\providecommand \natexlab [1]{#1}%
\providecommand \enquote  [1]{``#1''}%
\providecommand \bibnamefont  [1]{#1}%
\providecommand \bibfnamefont [1]{#1}%
\providecommand \citenamefont [1]{#1}%
\providecommand \href@noop [0]{\@secondoftwo}%
\providecommand \href [0]{\begingroup \@sanitize@url \@href}%
\providecommand \@href[1]{\@@startlink{#1}\@@href}%
\providecommand \@@href[1]{\endgroup#1\@@endlink}%
\providecommand \@sanitize@url [0]{\catcode `\\12\catcode `\$12\catcode
  `\&12\catcode `\#12\catcode `\^12\catcode `\_12\catcode `\%12\relax}%
\providecommand \@@startlink[1]{}%
\providecommand \@@endlink[0]{}%
\providecommand \url  [0]{\begingroup\@sanitize@url \@url }%
\providecommand \@url [1]{\endgroup\@href {#1}{\urlprefix }}%
\providecommand \urlprefix  [0]{URL }%
\providecommand \Eprint [0]{\href }%
\providecommand \doibase [0]{https://doi.org/}%
\providecommand \selectlanguage [0]{\@gobble}%
\providecommand \bibinfo  [0]{\@secondoftwo}%
\providecommand \bibfield  [0]{\@secondoftwo}%
\providecommand \translation [1]{[#1]}%
\providecommand \BibitemOpen [0]{}%
\providecommand \bibitemStop [0]{}%
\providecommand \bibitemNoStop [0]{.\EOS\space}%
\providecommand \EOS [0]{\spacefactor3000\relax}%
\providecommand \BibitemShut  [1]{\csname bibitem#1\endcsname}%
\let\auto@bib@innerbib\@empty
%</preamble>
\bibitem [{\citenamefont {Ohno}\ \emph {et~al.}(1996)\citenamefont {Ohno},
  \citenamefont {Shen}, \citenamefont {Matsukura}, \citenamefont {Oiwa},
  \citenamefont {Endo}, \citenamefont {Katsumoto},\ and\ \citenamefont
  {Iye}}]{10.1063/1.118061}%
  \BibitemOpen
  \bibfield  {author} {\bibinfo {author} {\bibfnamefont {H.}~\bibnamefont
  {Ohno}}, \bibinfo {author} {\bibfnamefont {A.}~\bibnamefont {Shen}}, \bibinfo
  {author} {\bibfnamefont {F.}~\bibnamefont {Matsukura}}, \bibinfo {author}
  {\bibfnamefont {A.}~\bibnamefont {Oiwa}}, \bibinfo {author} {\bibfnamefont
  {A.}~\bibnamefont {Endo}}, \bibinfo {author} {\bibfnamefont {S.}~\bibnamefont
  {Katsumoto}},\ and\ \bibinfo {author} {\bibfnamefont {Y.}~\bibnamefont
  {Iye}},\ }\bibfield  {title} {\bibinfo {title} {{(Ga,Mn)As: A new diluted
  magnetic semiconductor based on GaAs}},\ }\href
  {https://doi.org/10.1063/1.118061} {\bibfield  {journal} {\bibinfo  {journal}
  {Applied Physics Letters}\ }\textbf {\bibinfo {volume} {69}},\ \bibinfo
  {pages} {363} (\bibinfo {year} {1996})}\BibitemShut {NoStop}%
\bibitem [{\citenamefont {Pan}\ \emph {et~al.}(2007)\citenamefont {Pan},
  \citenamefont {Yi}, \citenamefont {Shen}, \citenamefont {Wu}, \citenamefont
  {Yang}, \citenamefont {Lin}, \citenamefont {Feng}, \citenamefont {Ding},
  \citenamefont {Van},\ and\ \citenamefont {Yin}}]{PhysRevLett.99.127201}%
  \BibitemOpen
  \bibfield  {author} {\bibinfo {author} {\bibfnamefont {H.}~\bibnamefont
  {Pan}}, \bibinfo {author} {\bibfnamefont {J.~B.}\ \bibnamefont {Yi}},
  \bibinfo {author} {\bibfnamefont {L.}~\bibnamefont {Shen}}, \bibinfo {author}
  {\bibfnamefont {R.~Q.}\ \bibnamefont {Wu}}, \bibinfo {author} {\bibfnamefont
  {J.~H.}\ \bibnamefont {Yang}}, \bibinfo {author} {\bibfnamefont {J.~Y.}\
  \bibnamefont {Lin}}, \bibinfo {author} {\bibfnamefont {Y.~P.}\ \bibnamefont
  {Feng}}, \bibinfo {author} {\bibfnamefont {J.}~\bibnamefont {Ding}}, \bibinfo
  {author} {\bibfnamefont {L.~H.}\ \bibnamefont {Van}},\ and\ \bibinfo {author}
  {\bibfnamefont {J.~H.}\ \bibnamefont {Yin}},\ }\bibfield  {title} {\bibinfo
  {title} {Room-temperature ferromagnetism in carbon-doped {ZnO}},\ }\href
  {https://doi.org/10.1103/PhysRevLett.99.127201} {\bibfield  {journal}
  {\bibinfo  {journal} {Phys. Rev. Lett.}\ }\textbf {\bibinfo {volume} {99}},\
  \bibinfo {pages} {127201} (\bibinfo {year} {2007})}\BibitemShut {NoStop}%
\bibitem [{\citenamefont {Saadaoui}\ \emph {et~al.}(2016)\citenamefont
  {Saadaoui}, \citenamefont {Luo}, \citenamefont {Salman}, \citenamefont {Cui},
  \citenamefont {Bao}, \citenamefont {Bao}, \citenamefont {Zheng},
  \citenamefont {Tseng}, \citenamefont {Du}, \citenamefont {Prokscha},
  \citenamefont {Suter}, \citenamefont {Liu}, \citenamefont {Wang},
  \citenamefont {Li}, \citenamefont {Ding}, \citenamefont {Ringer},
  \citenamefont {Morenzoni},\ and\ \citenamefont
  {Yi}}]{PhysRevLett.117.227202}%
  \BibitemOpen
  \bibfield  {author} {\bibinfo {author} {\bibfnamefont {H.}~\bibnamefont
  {Saadaoui}}, \bibinfo {author} {\bibfnamefont {X.}~\bibnamefont {Luo}},
  \bibinfo {author} {\bibfnamefont {Z.}~\bibnamefont {Salman}}, \bibinfo
  {author} {\bibfnamefont {X.~Y.}\ \bibnamefont {Cui}}, \bibinfo {author}
  {\bibfnamefont {N.~N.}\ \bibnamefont {Bao}}, \bibinfo {author} {\bibfnamefont
  {P.}~\bibnamefont {Bao}}, \bibinfo {author} {\bibfnamefont {R.~K.}\
  \bibnamefont {Zheng}}, \bibinfo {author} {\bibfnamefont {L.~T.}\ \bibnamefont
  {Tseng}}, \bibinfo {author} {\bibfnamefont {Y.~H.}\ \bibnamefont {Du}},
  \bibinfo {author} {\bibfnamefont {T.}~\bibnamefont {Prokscha}}, \bibinfo
  {author} {\bibfnamefont {A.}~\bibnamefont {Suter}}, \bibinfo {author}
  {\bibfnamefont {T.}~\bibnamefont {Liu}}, \bibinfo {author} {\bibfnamefont
  {Y.~R.}\ \bibnamefont {Wang}}, \bibinfo {author} {\bibfnamefont
  {S.}~\bibnamefont {Li}}, \bibinfo {author} {\bibfnamefont {J.}~\bibnamefont
  {Ding}}, \bibinfo {author} {\bibfnamefont {S.~P.}\ \bibnamefont {Ringer}},
  \bibinfo {author} {\bibfnamefont {E.}~\bibnamefont {Morenzoni}},\ and\
  \bibinfo {author} {\bibfnamefont {J.~B.}\ \bibnamefont {Yi}},\ }\bibfield
  {title} {\bibinfo {title} {Intrinsic ferromagnetism in the diluted magnetic
  semiconductor $\mathrm{Co}:\mathrm{TiO}_{2}$},\ }\href
  {https://doi.org/10.1103/PhysRevLett.117.227202} {\bibfield  {journal}
  {\bibinfo  {journal} {Phys. Rev. Lett.}\ }\textbf {\bibinfo {volume} {117}},\
  \bibinfo {pages} {227202} (\bibinfo {year} {2016})}\BibitemShut {NoStop}%
\bibitem [{\citenamefont {Dietl}\ \emph {et~al.}(2001)\citenamefont {Dietl},
  \citenamefont {Ohno},\ and\ \citenamefont {Matsukura}}]{PhysRevB.63.195205}%
  \BibitemOpen
  \bibfield  {author} {\bibinfo {author} {\bibfnamefont {T.}~\bibnamefont
  {Dietl}}, \bibinfo {author} {\bibfnamefont {H.}~\bibnamefont {Ohno}},\ and\
  \bibinfo {author} {\bibfnamefont {F.}~\bibnamefont {Matsukura}},\ }\bibfield
  {title} {\bibinfo {title} {Hole-mediated ferromagnetism in tetrahedrally
  coordinated semiconductors},\ }\href
  {https://doi.org/10.1103/PhysRevB.63.195205} {\bibfield  {journal} {\bibinfo
  {journal} {Phys. Rev. B}\ }\textbf {\bibinfo {volume} {63}},\ \bibinfo
  {pages} {195205} (\bibinfo {year} {2001})}\BibitemShut {NoStop}%
\bibitem [{\citenamefont {Kaminski}\ and\ \citenamefont
  {Das~Sarma}(2002)}]{PhysRevLett.88.247202}%
  \BibitemOpen
  \bibfield  {author} {\bibinfo {author} {\bibfnamefont {A.}~\bibnamefont
  {Kaminski}}\ and\ \bibinfo {author} {\bibfnamefont {S.}~\bibnamefont
  {Das~Sarma}},\ }\bibfield  {title} {\bibinfo {title} {Polaron percolation in
  diluted magnetic semiconductors},\ }\href
  {https://doi.org/10.1103/PhysRevLett.88.247202} {\bibfield  {journal}
  {\bibinfo  {journal} {Phys. Rev. Lett.}\ }\textbf {\bibinfo {volume} {88}},\
  \bibinfo {pages} {247202} (\bibinfo {year} {2002})}\BibitemShut {NoStop}%
\bibitem [{\citenamefont {Sheu}\ \emph {et~al.}(2007)\citenamefont {Sheu},
  \citenamefont {Myers}, \citenamefont {Tang}, \citenamefont {Samarth},
  \citenamefont {Awschalom}, \citenamefont {Schiffer},\ and\ \citenamefont
  {Flatt\'e}}]{PhysRevLett.99.227205}%
  \BibitemOpen
  \bibfield  {author} {\bibinfo {author} {\bibfnamefont {B.~L.}\ \bibnamefont
  {Sheu}}, \bibinfo {author} {\bibfnamefont {R.~C.}\ \bibnamefont {Myers}},
  \bibinfo {author} {\bibfnamefont {J.-M.}\ \bibnamefont {Tang}}, \bibinfo
  {author} {\bibfnamefont {N.}~\bibnamefont {Samarth}}, \bibinfo {author}
  {\bibfnamefont {D.~D.}\ \bibnamefont {Awschalom}}, \bibinfo {author}
  {\bibfnamefont {P.}~\bibnamefont {Schiffer}},\ and\ \bibinfo {author}
  {\bibfnamefont {M.~E.}\ \bibnamefont {Flatt\'e}},\ }\bibfield  {title}
  {\bibinfo {title} {Onset of ferromagnetism in low-doped
  $\mathrm{Ga}_{1\ensuremath{-}x}\mathrm{Mn}_{x}\mathrm{As}$},\ }\href
  {https://doi.org/10.1103/PhysRevLett.99.227205} {\bibfield  {journal}
  {\bibinfo  {journal} {Phys. Rev. Lett.}\ }\textbf {\bibinfo {volume} {99}},\
  \bibinfo {pages} {227205} (\bibinfo {year} {2007})}\BibitemShut {NoStop}%
\bibitem [{\citenamefont {Samarth}(2012)}]{10.1038/nmat3317}%
  \BibitemOpen
  \bibfield  {author} {\bibinfo {author} {\bibfnamefont {N.}~\bibnamefont
  {Samarth}},\ }\bibfield  {title} {\bibinfo {title} {{Battle of the bands}},\
  }\href {https://doi.org/10.1038/nmat3317} {\bibfield  {journal} {\bibinfo
  {journal} {Nature Materials}\ }\textbf {\bibinfo {volume} {11}},\ \bibinfo
  {pages} {360} (\bibinfo {year} {2012})}\BibitemShut {NoStop}%
\bibitem [{\citenamefont {Bhatt}\ and\ \citenamefont
  {Lee}(1982)}]{PhysRevLett.48.344}%
  \BibitemOpen
  \bibfield  {author} {\bibinfo {author} {\bibfnamefont {R.~N.}\ \bibnamefont
  {Bhatt}}\ and\ \bibinfo {author} {\bibfnamefont {P.~A.}\ \bibnamefont
  {Lee}},\ }\bibfield  {title} {\bibinfo {title} {Scaling studies of highly
  disordered spin-\textonehalf{} antiferromagnetic systems},\ }\href
  {https://doi.org/10.1103/PhysRevLett.48.344} {\bibfield  {journal} {\bibinfo
  {journal} {Phys. Rev. Lett.}\ }\textbf {\bibinfo {volume} {48}},\ \bibinfo
  {pages} {344} (\bibinfo {year} {1982})}\BibitemShut {NoStop}%
\bibitem [{\citenamefont {McLenaghan}\ and\ \citenamefont
  {Sherrington}(1984)}]{McLenaghan_1984}%
  \BibitemOpen
  \bibfield  {author} {\bibinfo {author} {\bibfnamefont {I.~R.}\ \bibnamefont
  {McLenaghan}}\ and\ \bibinfo {author} {\bibfnamefont {D.}~\bibnamefont
  {Sherrington}},\ }\bibfield  {title} {\bibinfo {title} {The homogeneously
  random {Ising} antiferromagnet: a computer study},\ }\href
  {https://doi.org/10.1088/0022-3719/17/9/011} {\bibfield  {journal} {\bibinfo
  {journal} {Journal of Physics C: Solid State Physics}\ }\textbf {\bibinfo
  {volume} {17}},\ \bibinfo {pages} {1531} (\bibinfo {year}
  {1984})}\BibitemShut {NoStop}%
\bibitem [{\citenamefont {Ghazali}\ and\ \citenamefont
  {Diep}(1985)}]{doi:10.1063/1.335065}%
  \BibitemOpen
  \bibfield  {author} {\bibinfo {author} {\bibfnamefont {A.}~\bibnamefont
  {Ghazali}}\ and\ \bibinfo {author} {\bibfnamefont {H.~T.}\ \bibnamefont
  {Diep}},\ }\bibfield  {title} {\bibinfo {title} {Spin ordering in a random
  antiferromagnetic {Heisenberg} spin system: Numerical simulation},\ }\href
  {https://doi.org/10.1063/1.335065} {\bibfield  {journal} {\bibinfo  {journal}
  {Journal of Applied Physics}\ }\textbf {\bibinfo {volume} {57}},\ \bibinfo
  {pages} {3427} (\bibinfo {year} {1985})}\BibitemShut {NoStop}%
\bibitem [{\citenamefont {Bogoslovskiy}\ \emph {et~al.}(2019)\citenamefont
  {Bogoslovskiy}, \citenamefont {Petrov},\ and\ \citenamefont
  {Averkiev}}]{bogoslovskiy2019impurity}%
  \BibitemOpen
  \bibfield  {author} {\bibinfo {author} {\bibfnamefont {N.}~\bibnamefont
  {Bogoslovskiy}}, \bibinfo {author} {\bibfnamefont {P.}~\bibnamefont
  {Petrov}},\ and\ \bibinfo {author} {\bibfnamefont {N.}~\bibnamefont
  {Averkiev}},\ }\bibfield  {title} {\bibinfo {title} {The impurity magnetic
  susceptibility of semiconductors in the case of direct exchange interaction
  in the {Ising} model},\ }\href
  {https://doi.org/https://doi.org/10.1134/S1063783419110088} {\bibfield
  {journal} {\bibinfo  {journal} {Physics of the Solid State}\ }\textbf
  {\bibinfo {volume} {61}},\ \bibinfo {pages} {2005} (\bibinfo {year}
  {2019})}\BibitemShut {NoStop}%
\bibitem [{\citenamefont {Bogoslovskiy}\ \emph {et~al.}(2021)\citenamefont
  {Bogoslovskiy}, \citenamefont {Petrov},\ and\ \citenamefont
  {Averkiev}}]{bogoslovskiy2021spin}%
  \BibitemOpen
  \bibfield  {author} {\bibinfo {author} {\bibfnamefont {N.}~\bibnamefont
  {Bogoslovskiy}}, \bibinfo {author} {\bibfnamefont {P.}~\bibnamefont
  {Petrov}},\ and\ \bibinfo {author} {\bibfnamefont {N.}~\bibnamefont
  {Averkiev}},\ }\bibfield  {title} {\bibinfo {title} {Spin-fluctuation
  transition in the disordered {Ising} model},\ }\href
  {https://doi.org/https://doi.org/10.1134/S0021364021180077} {\bibfield
  {journal} {\bibinfo  {journal} {JETP Letters}\ }\textbf {\bibinfo {volume}
  {114}},\ \bibinfo {pages} {347} (\bibinfo {year} {2021})}\BibitemShut
  {NoStop}%
\bibitem [{\citenamefont {Ohno}(1999)}]{OHNO1999110}%
  \BibitemOpen
  \bibfield  {author} {\bibinfo {author} {\bibfnamefont {H.}~\bibnamefont
  {Ohno}},\ }\bibfield  {title} {\bibinfo {title} {Properties of ferromagnetic
  iii–v semiconductors},\ }\href
  {https://doi.org/https://doi.org/10.1016/S0304-8853(99)00444-8} {\bibfield
  {journal} {\bibinfo  {journal} {Journal of Magnetism and Magnetic Materials}\
  }\textbf {\bibinfo {volume} {200}},\ \bibinfo {pages} {110} (\bibinfo {year}
  {1999})}\BibitemShut {NoStop}%
\bibitem [{\citenamefont {Sawicki}\ \emph {et~al.}(2010)\citenamefont
  {Sawicki}, \citenamefont {Chiba}, \citenamefont {Korbecka}, \citenamefont
  {Nishitani}, \citenamefont {Majewski}, \citenamefont {Matsukura},
  \citenamefont {Dietl},\ and\ \citenamefont {Ohno}}]{sawicki2010experimental}%
  \BibitemOpen
  \bibfield  {author} {\bibinfo {author} {\bibfnamefont {M.}~\bibnamefont
  {Sawicki}}, \bibinfo {author} {\bibfnamefont {D.}~\bibnamefont {Chiba}},
  \bibinfo {author} {\bibfnamefont {A.}~\bibnamefont {Korbecka}}, \bibinfo
  {author} {\bibfnamefont {Y.}~\bibnamefont {Nishitani}}, \bibinfo {author}
  {\bibfnamefont {J.~A.}\ \bibnamefont {Majewski}}, \bibinfo {author}
  {\bibfnamefont {F.}~\bibnamefont {Matsukura}}, \bibinfo {author}
  {\bibfnamefont {T.}~\bibnamefont {Dietl}},\ and\ \bibinfo {author}
  {\bibfnamefont {H.}~\bibnamefont {Ohno}},\ }\bibfield  {title} {\bibinfo
  {title} {Experimental probing of the interplay between ferromagnetism and
  localization in {(Ga, Mn) As}},\ }\href
  {https://doi.org/https://doi.org/10.1038/nphys1455} {\bibfield  {journal}
  {\bibinfo  {journal} {Nature Physics}\ }\textbf {\bibinfo {volume} {6}},\
  \bibinfo {pages} {22} (\bibinfo {year} {2010})}\BibitemShut {NoStop}%
\bibitem [{\citenamefont {Zhou}\ \emph {et~al.}(2016)\citenamefont {Zhou},
  \citenamefont {Li}, \citenamefont {Yuan}, \citenamefont {Rushforth},
  \citenamefont {Chen}, \citenamefont {Wang}, \citenamefont {B\"ottger},
  \citenamefont {Heller}, \citenamefont {Zhao}, \citenamefont {Edmonds},
  \citenamefont {Campion}, \citenamefont {Gallagher}, \citenamefont {Timm},\
  and\ \citenamefont {Helm}}]{PhysRevB.94.075205}%
  \BibitemOpen
  \bibfield  {author} {\bibinfo {author} {\bibfnamefont {S.}~\bibnamefont
  {Zhou}}, \bibinfo {author} {\bibfnamefont {L.}~\bibnamefont {Li}}, \bibinfo
  {author} {\bibfnamefont {Y.}~\bibnamefont {Yuan}}, \bibinfo {author}
  {\bibfnamefont {A.~W.}\ \bibnamefont {Rushforth}}, \bibinfo {author}
  {\bibfnamefont {L.}~\bibnamefont {Chen}}, \bibinfo {author} {\bibfnamefont
  {Y.}~\bibnamefont {Wang}}, \bibinfo {author} {\bibfnamefont {R.}~\bibnamefont
  {B\"ottger}}, \bibinfo {author} {\bibfnamefont {R.}~\bibnamefont {Heller}},
  \bibinfo {author} {\bibfnamefont {J.}~\bibnamefont {Zhao}}, \bibinfo {author}
  {\bibfnamefont {K.~W.}\ \bibnamefont {Edmonds}}, \bibinfo {author}
  {\bibfnamefont {R.~P.}\ \bibnamefont {Campion}}, \bibinfo {author}
  {\bibfnamefont {B.~L.}\ \bibnamefont {Gallagher}}, \bibinfo {author}
  {\bibfnamefont {C.}~\bibnamefont {Timm}},\ and\ \bibinfo {author}
  {\bibfnamefont {M.}~\bibnamefont {Helm}},\ }\bibfield  {title} {\bibinfo
  {title} {Precise tuning of the curie temperature of {(Ga,Mn)As-based}
  magnetic semiconductors by hole compensation: Support for valence-band
  ferromagnetism},\ }\href {https://doi.org/10.1103/PhysRevB.94.075205}
  {\bibfield  {journal} {\bibinfo  {journal} {Phys. Rev. B}\ }\textbf {\bibinfo
  {volume} {94}},\ \bibinfo {pages} {075205} (\bibinfo {year}
  {2016})}\BibitemShut {NoStop}%
\bibitem [{\citenamefont {Yuan}\ \emph {et~al.}(2017)\citenamefont {Yuan},
  \citenamefont {Xu}, \citenamefont {H\"ubner}, \citenamefont {Jakiela},
  \citenamefont {B\"ottger}, \citenamefont {Helm}, \citenamefont {Sawicki},
  \citenamefont {Dietl},\ and\ \citenamefont
  {Zhou}}]{PhysRevMaterials.1.054401}%
  \BibitemOpen
  \bibfield  {author} {\bibinfo {author} {\bibfnamefont {Y.}~\bibnamefont
  {Yuan}}, \bibinfo {author} {\bibfnamefont {C.}~\bibnamefont {Xu}}, \bibinfo
  {author} {\bibfnamefont {R.}~\bibnamefont {H\"ubner}}, \bibinfo {author}
  {\bibfnamefont {R.}~\bibnamefont {Jakiela}}, \bibinfo {author} {\bibfnamefont
  {R.}~\bibnamefont {B\"ottger}}, \bibinfo {author} {\bibfnamefont
  {M.}~\bibnamefont {Helm}}, \bibinfo {author} {\bibfnamefont {M.}~\bibnamefont
  {Sawicki}}, \bibinfo {author} {\bibfnamefont {T.}~\bibnamefont {Dietl}},\
  and\ \bibinfo {author} {\bibfnamefont {S.}~\bibnamefont {Zhou}},\ }\bibfield
  {title} {\bibinfo {title} {Interplay between localization and magnetism in
  {(Ga,Mn)As and (In,Mn)As}},\ }\href
  {https://doi.org/10.1103/PhysRevMaterials.1.054401} {\bibfield  {journal}
  {\bibinfo  {journal} {Phys. Rev. Mater.}\ }\textbf {\bibinfo {volume} {1}},\
  \bibinfo {pages} {054401} (\bibinfo {year} {2017})}\BibitemShut {NoStop}%
\bibitem [{\citenamefont {Yuan}\ \emph {et~al.}(2018)\citenamefont {Yuan},
  \citenamefont {Wang}, \citenamefont {Xu}, \citenamefont {H\"{u}bner},
  \citenamefont {B\"{o}ttger}, \citenamefont {Jakiela}, \citenamefont {Helm},
  \citenamefont {Sawicki},\ and\ \citenamefont {Zhou}}]{Yuan_2018}%
  \BibitemOpen
  \bibfield  {author} {\bibinfo {author} {\bibfnamefont {Y.}~\bibnamefont
  {Yuan}}, \bibinfo {author} {\bibfnamefont {M.}~\bibnamefont {Wang}}, \bibinfo
  {author} {\bibfnamefont {C.}~\bibnamefont {Xu}}, \bibinfo {author}
  {\bibfnamefont {R.}~\bibnamefont {H\"{u}bner}}, \bibinfo {author}
  {\bibfnamefont {R.}~\bibnamefont {B\"{o}ttger}}, \bibinfo {author}
  {\bibfnamefont {R.}~\bibnamefont {Jakiela}}, \bibinfo {author} {\bibfnamefont
  {M.}~\bibnamefont {Helm}}, \bibinfo {author} {\bibfnamefont {M.}~\bibnamefont
  {Sawicki}},\ and\ \bibinfo {author} {\bibfnamefont {S.}~\bibnamefont
  {Zhou}},\ }\bibfield  {title} {\bibinfo {title} {Electronic phase separation
  in insulating {(Ga, Mn) As} with low compensation: super-paramagnetism and
  hopping conduction},\ }\href {https://doi.org/10.1088/1361-648X/aaa9a7}
  {\bibfield  {journal} {\bibinfo  {journal} {Journal of Physics: Condensed
  Matter}\ }\textbf {\bibinfo {volume} {30}},\ \bibinfo {pages} {095801}
  (\bibinfo {year} {2018})}\BibitemShut {NoStop}%
\bibitem [{\citenamefont {Gluba}\ \emph {et~al.}(2018)\citenamefont {Gluba},
  \citenamefont {Yastrubchak}, \citenamefont {Domagala}, \citenamefont
  {Jakiela}, \citenamefont {Andrearczyk}, \citenamefont {\ifmmode~\dot{Z}\else
  \.{Z}\fi{}uk}, \citenamefont {Wosinski}, \citenamefont {Sadowski},\ and\
  \citenamefont {Sawicki}}]{PhysRevB.97.115201}%
  \BibitemOpen
  \bibfield  {author} {\bibinfo {author} {\bibfnamefont {L.}~\bibnamefont
  {Gluba}}, \bibinfo {author} {\bibfnamefont {O.}~\bibnamefont {Yastrubchak}},
  \bibinfo {author} {\bibfnamefont {J.~Z.}\ \bibnamefont {Domagala}}, \bibinfo
  {author} {\bibfnamefont {R.}~\bibnamefont {Jakiela}}, \bibinfo {author}
  {\bibfnamefont {T.}~\bibnamefont {Andrearczyk}}, \bibinfo {author}
  {\bibfnamefont {J.}~\bibnamefont {\ifmmode~\dot{Z}\else \.{Z}\fi{}uk}},
  \bibinfo {author} {\bibfnamefont {T.}~\bibnamefont {Wosinski}}, \bibinfo
  {author} {\bibfnamefont {J.}~\bibnamefont {Sadowski}},\ and\ \bibinfo
  {author} {\bibfnamefont {M.}~\bibnamefont {Sawicki}},\ }\bibfield  {title}
  {\bibinfo {title} {Band structure evolution and the origin of magnetism in
  {(Ga,Mn)As}: From paramagnetic through superparamagnetic to ferromagnetic
  phase},\ }\href {https://doi.org/10.1103/PhysRevB.97.115201} {\bibfield
  {journal} {\bibinfo  {journal} {Phys. Rev. B}\ }\textbf {\bibinfo {volume}
  {97}},\ \bibinfo {pages} {115201} (\bibinfo {year} {2018})}\BibitemShut
  {NoStop}%
\bibitem [{\citenamefont {Takeda}\ \emph {et~al.}(2020)\citenamefont {Takeda},
  \citenamefont {Ohya}, \citenamefont {Pham}, \citenamefont {Kobayashi},
  \citenamefont {Saitoh}, \citenamefont {Yamagami}, \citenamefont {Tanaka},\
  and\ \citenamefont {Fujimori}}]{10.1063/5.0031605}%
  \BibitemOpen
  \bibfield  {author} {\bibinfo {author} {\bibfnamefont {Y.}~\bibnamefont
  {Takeda}}, \bibinfo {author} {\bibfnamefont {S.}~\bibnamefont {Ohya}},
  \bibinfo {author} {\bibfnamefont {N.~H.}\ \bibnamefont {Pham}}, \bibinfo
  {author} {\bibfnamefont {M.}~\bibnamefont {Kobayashi}}, \bibinfo {author}
  {\bibfnamefont {Y.}~\bibnamefont {Saitoh}}, \bibinfo {author} {\bibfnamefont
  {H.}~\bibnamefont {Yamagami}}, \bibinfo {author} {\bibfnamefont
  {M.}~\bibnamefont {Tanaka}},\ and\ \bibinfo {author} {\bibfnamefont
  {A.}~\bibnamefont {Fujimori}},\ }\bibfield  {title} {\bibinfo {title}
  {{Direct observation of the magnetic ordering process in the ferromagnetic
  semiconductor $\mathrm{Ga}_{1\ensuremath{-}x}\mathrm{Mn}_{x}\mathrm{As}$ via
  soft x-ray magnetic circular dichroism}},\ }\href
  {https://doi.org/10.1063/5.0031605} {\bibfield  {journal} {\bibinfo
  {journal} {Journal of Applied Physics}\ }\textbf {\bibinfo {volume} {128}},\
  \bibinfo {pages} {213902} (\bibinfo {year} {2020})}\BibitemShut {NoStop}%
\bibitem [{\citenamefont {Wang}\ and\ \citenamefont
  {Landau}(2001)}]{PhysRevLett.86.2050}%
  \BibitemOpen
  \bibfield  {author} {\bibinfo {author} {\bibfnamefont {F.}~\bibnamefont
  {Wang}}\ and\ \bibinfo {author} {\bibfnamefont {D.~P.}\ \bibnamefont
  {Landau}},\ }\bibfield  {title} {\bibinfo {title} {Efficient, multiple-range
  random walk algorithm to calculate the density of states},\ }\href
  {https://doi.org/10.1103/PhysRevLett.86.2050} {\bibfield  {journal} {\bibinfo
   {journal} {Phys. Rev. Lett.}\ }\textbf {\bibinfo {volume} {86}},\ \bibinfo
  {pages} {2050} (\bibinfo {year} {2001})}\BibitemShut {NoStop}%
\bibitem [{\citenamefont {Kryzhanovsky}\ \emph {et~al.}(2021)\citenamefont
  {Kryzhanovsky}, \citenamefont {Litinskii},\ and\ \citenamefont
  {Egorov}}]{e23121665}%
  \BibitemOpen
  \bibfield  {author} {\bibinfo {author} {\bibfnamefont {B.}~\bibnamefont
  {Kryzhanovsky}}, \bibinfo {author} {\bibfnamefont {L.}~\bibnamefont
  {Litinskii}},\ and\ \bibinfo {author} {\bibfnamefont {V.}~\bibnamefont
  {Egorov}},\ }\bibfield  {title} {\bibinfo {title} {Analytical expressions for
  {Ising} models on high dimensional lattices},\ }\href
  {https://www.mdpi.com/1099-4300/23/12/1665} {\bibfield  {journal} {\bibinfo
  {journal} {Entropy}\ }\textbf {\bibinfo {volume} {23}} (\bibinfo {year}
  {2021})}\BibitemShut {NoStop}%
\bibitem [{\citenamefont {{Gor'kov}}\ and\ \citenamefont
  {{Pitaevskii}}(1964)}]{1964Gorkov}%
  \BibitemOpen
  \bibfield  {author} {\bibinfo {author} {\bibfnamefont {L.~P.}\ \bibnamefont
  {{Gor'kov}}}\ and\ \bibinfo {author} {\bibfnamefont {L.~P.}\ \bibnamefont
  {{Pitaevskii}}},\ }\href@noop {} {\bibfield  {journal} {\bibinfo  {journal}
  {Soviet Physics Doklady}\ }\textbf {\bibinfo {volume} {8}},\ \bibinfo {pages}
  {788} (\bibinfo {year} {1964})}\BibitemShut {NoStop}%
\bibitem [{\citenamefont {Herring}\ and\ \citenamefont
  {Flicker}(1964)}]{PhysRev.134.A362}%
  \BibitemOpen
  \bibfield  {author} {\bibinfo {author} {\bibfnamefont {C.}~\bibnamefont
  {Herring}}\ and\ \bibinfo {author} {\bibfnamefont {M.}~\bibnamefont
  {Flicker}},\ }\href {https://doi.org/10.1103/PhysRev.134.A362} {\bibfield
  {journal} {\bibinfo  {journal} {Phys. Rev.}\ }\textbf {\bibinfo {volume}
  {134}},\ \bibinfo {pages} {A362} (\bibinfo {year} {1964})}\BibitemShut
  {NoStop}%
\bibitem [{\citenamefont {Tribelsky}(2002)}]{PhysRevLett.89.070201}%
  \BibitemOpen
  \bibfield  {author} {\bibinfo {author} {\bibfnamefont {M.~I.}\ \bibnamefont
  {Tribelsky}},\ }\bibfield  {title} {\bibinfo {title} {General exact solution
  to the problem of the probability density for sums of random variables},\
  }\href {https://doi.org/10.1103/PhysRevLett.89.070201} {\bibfield  {journal}
  {\bibinfo  {journal} {Phys. Rev. Lett.}\ }\textbf {\bibinfo {volume} {89}},\
  \bibinfo {pages} {070201} (\bibinfo {year} {2002})}\BibitemShut {NoStop}%
\bibitem [{\citenamefont {Chandrasekhar}(1943)}]{RevModPhys.15.1}%
  \BibitemOpen
  \bibfield  {author} {\bibinfo {author} {\bibfnamefont {S.}~\bibnamefont
  {Chandrasekhar}},\ }\bibfield  {title} {\bibinfo {title} {Stochastic problems
  in physics and astronomy},\ }\href {https://doi.org/10.1103/RevModPhys.15.1}
  {\bibfield  {journal} {\bibinfo  {journal} {Rev. Mod. Phys.}\ }\textbf
  {\bibinfo {volume} {15}},\ \bibinfo {pages} {1} (\bibinfo {year} {1943})},\
  \bibinfo {note} {{A}ppendix VII}\BibitemShut {NoStop}%
\bibitem [{\citenamefont {Landau}\ \emph {et~al.}(2004)\citenamefont {Landau},
  \citenamefont {Tsai},\ and\ \citenamefont {Exler}}]{doi:10.1119/1.1707017}%
  \BibitemOpen
  \bibfield  {author} {\bibinfo {author} {\bibfnamefont {D.~P.}\ \bibnamefont
  {Landau}}, \bibinfo {author} {\bibfnamefont {S.-H.}\ \bibnamefont {Tsai}},\
  and\ \bibinfo {author} {\bibfnamefont {M.}~\bibnamefont {Exler}},\ }\bibfield
   {title} {\bibinfo {title} {A new approach to monte carlo simulations in
  statistical physics: Wang-landau sampling},\ }\href
  {https://doi.org/10.1119/1.1707017} {\bibfield  {journal} {\bibinfo
  {journal} {American Journal of Physics}\ }\textbf {\bibinfo {volume} {72}},\
  \bibinfo {pages} {1294} (\bibinfo {year} {2004})}\BibitemShut {NoStop}%
\bibitem [{\citenamefont {Bogoslovskiy}\ \emph {et~al.}(2023)\citenamefont
  {Bogoslovskiy}, \citenamefont {Petrov},\ and\ \citenamefont
  {Averkiev}}]{10.18721/JPM.161.301}%
  \BibitemOpen
  \bibfield  {author} {\bibinfo {author} {\bibfnamefont {N.}~\bibnamefont
  {Bogoslovskiy}}, \bibinfo {author} {\bibfnamefont {P.}~\bibnamefont
  {Petrov}},\ and\ \bibinfo {author} {\bibfnamefont {N.}~\bibnamefont
  {Averkiev}},\ }\bibfield  {title} {\bibinfo {title} {{Analytical and
  numerical calculations of the magnetic properties of a system of disordered
  spins in the {Ising} model}},\ }\href {https://doi.org/10.18721/JPM.161.301}
  {\bibfield  {journal} {\bibinfo  {journal} {St. Petersburg State
  Polytechnical University Journal. Physics and Mathematics}\ }\textbf
  {\bibinfo {volume} {16}},\ \bibinfo {pages} {7} (\bibinfo {year}
  {2023})}\BibitemShut {NoStop}%
\bibitem [{\citenamefont {Besard}\ \emph {et~al.}(2018)\citenamefont {Besard},
  \citenamefont {Foket},\ and\ \citenamefont {De~Sutter}}]{besard2018juliagpu}%
  \BibitemOpen
  \bibfield  {author} {\bibinfo {author} {\bibfnamefont {T.}~\bibnamefont
  {Besard}}, \bibinfo {author} {\bibfnamefont {C.}~\bibnamefont {Foket}},\ and\
  \bibinfo {author} {\bibfnamefont {B.}~\bibnamefont {De~Sutter}},\ }\bibfield
  {title} {\bibinfo {title} {Effective extensible programming: Unleashing
  {Julia} on {GPUs}},\ }\bibfield  {journal} {\bibinfo  {journal} {IEEE
  Transactions on Parallel and Distributed Systems}\ }\href
  {https://doi.org/https://doi.org/10.1109/TPDS.2018.2872064}
  {https://doi.org/10.1109/TPDS.2018.2872064} (\bibinfo {year}
  {2018})\BibitemShut {NoStop}%
\bibitem [{\citenamefont {Landau}\ and\ \citenamefont
  {Lifshitz}(2013)}]{landau2013statistical}%
  \BibitemOpen
  \bibfield  {author} {\bibinfo {author} {\bibfnamefont {L.}~\bibnamefont
  {Landau}}\ and\ \bibinfo {author} {\bibfnamefont {E.}~\bibnamefont
  {Lifshitz}},\ }\href {https://books.google.ru/books?id=VzgJN-XPTRsC} {\emph
  {\bibinfo {title} {Statistical Physics: Volume 5}}},\ \bibinfo {number} {т.
  5}\ (\bibinfo  {publisher} {Elsevier Science},\ \bibinfo {year}
  {2013})\BibitemShut {NoStop}%
\bibitem [{\citenamefont {Demishev}\ \emph {et~al.}(2022)\citenamefont
  {Demishev}, \citenamefont {Samarin}, \citenamefont {Karasev}, \citenamefont
  {Grigoriev},\ and\ \citenamefont {Semeno}}]{demishev2022spin}%
  \BibitemOpen
  \bibfield  {author} {\bibinfo {author} {\bibfnamefont {S.}~\bibnamefont
  {Demishev}}, \bibinfo {author} {\bibfnamefont {A.}~\bibnamefont {Samarin}},
  \bibinfo {author} {\bibfnamefont {M.}~\bibnamefont {Karasev}}, \bibinfo
  {author} {\bibfnamefont {S.}~\bibnamefont {Grigoriev}},\ and\ \bibinfo
  {author} {\bibfnamefont {A.}~\bibnamefont {Semeno}},\ }\bibfield  {title}
  {\bibinfo {title} {Spin fluctuations and a spin-fluctuation phase transition
  in the magnetically ordered phase of manganese monosilicide},\ }\href
  {https://doi.org/https://doi.org/10.1134/S0021364022600677} {\bibfield
  {journal} {\bibinfo  {journal} {JETP Letters}\ }\textbf {\bibinfo {volume}
  {115}},\ \bibinfo {pages} {673} (\bibinfo {year} {2022})}\BibitemShut
  {NoStop}%
\end{thebibliography}%

\end{document}
%
% ****** End of file apssamp.tex ******

