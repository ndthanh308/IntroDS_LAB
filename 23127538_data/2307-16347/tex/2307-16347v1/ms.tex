%\documentclass[aps,pra,superscriptaddress,showpacs,twocolumn,draft]{revtex4}
\documentclass[aps,prl,twocolumn,superscriptaddress]{revtex4}
\usepackage{graphicx}
\usepackage{subfigure}
\usepackage{epstopdf}
%\usepackage{subfigure}
%\usepackage{epsfig}
\usepackage{amsmath}
\usepackage{amssymb}
\usepackage{amsfonts}
\usepackage{mathrsfs}
\usepackage{theorem}
\usepackage{bm}
\usepackage{url}
\usepackage[T1]{fontenc}
\usepackage{csquotes}
\MakeOuterQuote{"}
\usepackage{algorithm}
\usepackage{algorithmicx}
\usepackage{algpseudocode}
\renewcommand{\thealgorithm}{\!\!:}
\usepackage{dcolumn}
\usepackage{color}
%\usepackage[colorlinks,citecolor=blue]{hyperref}
%\usepackage{authblk}
%\usepackage{cite}
%%%%%%%%%%%%%%%%%%%%%%%%%%%%%%%%%%%%%%%%%%%%%%%%%%%%%%%%%%%%%%%%%%%%%%%%%%%%%%%%%%%%%%%%%%%%%%%%%%%%%%%%%%%%%%%%%%%%%%%%%%%%%%%%%%%%%%%%%%\usepackage{graphicx}
\newcommand{\red}{\color{red}}
\newcommand{\blu}{\color{blue}}
\newcommand{\blk}{\color{black}}
\definecolor{ngreen}{rgb}{0.2,0.6,0.2}
\newcommand{\grn}{\color{ngreen}}
\newcommand{\hmw}[1]{{\color{ngreen} \bf [[{#1}]]}}
\newcommand{\dwb}[1]{{\color{blue} \bf [[{#1}]]}}
\definecolor{ngold}{rgb}{0.7,0.6,0.2}
\newcommand{\gold}{\color{ngold}}
\newcommand{\blh}[1]{{\color{ngold} \bf [[{#1}]]}}
%%%%%%%%%%%%%%%%%%%%%%%%%%%%%%%%%%%%%%%%%%%%%%%%%%%%%%%%%%%%%%%%%%%%%%%%%%%%%%
%% define mathematical words via abbreviations.
\def\vec#1{\mathbf{#1}} %% overiding the original command
\newcommand{\mrm}[1]{\mathrm{#1}}
\newcommand{\tr}{\operatorname{tr}}
\newcommand{\Tr}{\operatorname{Tr}}
\providecommand{\det}{\operatorname{det}}
\newcommand{\Det}{\operatorname{Det}}
\newcommand{\diag}{\operatorname{diag}}
\newcommand{\sgn}{\operatorname{sgn}}
\newcommand{\ad}{\operatorname{ad}}
\newcommand{\rep}{\mathrel{\widehat{=}}}
\newcommand{\rmi}{\mathrm{i}}
\newcommand{\rme}{\mathrm{e}}
\newcommand{\rmE}{\mathrm{E}}
\newcommand{\rmd}{\mathrm{d}}
\newcommand{\rmT}{\mathrm{T}}
\newcommand{\imply}{\mathrel{\Rightarrow}}
\newcommand{\equi}{\mathrel{\Leftrightarrow}}
\newcommand{\one}{\overline{1}}
\newcommand{\zero}{\overline{0}}
%%%%%%%%%%%%%%%%%%%%%%%%%%%%%%%%%%%%%%%%%%%%%%%%%%%%%%%%%%%%%%%%%%%%%%%%%%%%%%%%
\newcommand{\be}{\begin{equation}}
	\newcommand{\ee}{\end{equation}}
\newcommand{\ba}{\begin{align}}
	\newcommand{\ea}{\end{align}}
%%%%%%%%%%%%%%%%%%%%%%%%%%%%%%%%%%%%%%%%%%%%%%%%%%%%%%%%%%%%%%%%%%%%%%%%%%%%%%%%%
\def\<{\langle}  %% overiding the original command \<
\def\>{\rangle}  %% overiding the original command \>
\newcommand{\ket}[1]{| #1\>}
\newcommand{\bra}[1]{\< #1|}
\newcommand{\dket}[1]{| #1\>\!\>}
\newcommand{\Dket}[1]{\Bigl| #1\Bigr\>\!\Bigr\>}
\newcommand{\dbra}[1]{\<\!\< #1|}
\newcommand{\Dbra}[1]{\Bigl\<\!\Bigl\< #1\Bigr|}
\newcommand{\inner}[2]{\<#1|#2\>}
\def\outer#1#2{|#1\>\<#2|}       %% overiding the original command \outer
\newcommand{\dinner}[2]{\<\!\< #1| #2\>\!\>}
\newcommand{\Dinner}[2]{\Bigl\<\!\Bigl\< #1\Bigl| #2\Bigr\>\!\Bigr\>}
\newcommand{\douter}[2]{| #1\>\!\>\<\!\< #2|}
\newcommand{\Douter}[2]{\Bigl| #1\Bigr\>\!\Bigr\>\Bigl\<\!\Bigl\< #2\Bigr|}
\newcommand{\norm}[1]{\parallel\!#1\!\parallel}
\newcommand{\nfrac}[2]{\genfrac{}{}{0pt}{}{#1}{#2}}
%%%%%%%%%%%%%%%%%%%%%%%%%%%%%%%%%%%%%%%%%%%%%%%%%%%%%%%%%%%%%%%%%%%%%%%%%%%%%%%%%%%
%% Abbreviations used in quantum estimation theory
\newcommand{\mse}{\mathcal{E}}
\newcommand{\mhs}{\mathcal{E}_{\mathrm{HS}}}
\newcommand{\msh}{\mathcal{E}_{\mathrm{SH}}}
\newcommand{\msb}{\mathcal{E}_{\mathrm{SB}}}
\newcommand{\mtr}{\mathcal{E}_{\tr}}
\newcommand{\barcal}[1]{\bar{\mathcal{#1}}}
\newcommand{\bt}{\bar{t}}
\newcommand{\bid}{\bar{\mathbf{I}}}
\newcommand{\cF}{\mathcal{F}}
%%%%%%%%%%%%%%%%%%%%%%%%%%%%%%%%%%%%%%%%%%%%%%%%%%%%%%%%%%%%%%%%%%%%%%%%%%%%%%%%%%%%%%%%%%%%%
%% Abbreviations used in cross references and citations
%\def\eqref#1{(\ref{#1})}    %% overiding the original command \eqref
%\newcommand{\eref}[1]{Eq.~(\ref{#1})}
%\newcommand{\Eref}[1]{Equation~(\ref{#1})}
%\newcommand{\esref}[1]{Eqs.~(\ref{#1})}
%\newcommand{\Esref}[1]{Equations~(\ref{#1})}
%\def\eqref#1{\textup{(}\ref{#1}\textup{)}}  %% overiding the original command \eqref
%\newcommand{\eref}[1]{Eq.~\textup{(}\ref{#1}\textup{)}}
%\newcommand{\Eref}[1]{Equation~\textup{(}\ref{#1}\textup{)}}
%\newcommand{\esref}[1]{Eqs.~\textup{(}\ref{#1}\textup{)}}
%\newcommand{\Esref}[1]{Equations~\textup{(}\ref{#1}\textup{)}}
\def\eqref#1{\textup{(\ref{#1})}}  %% overiding the original command \eqref
\newcommand{\eref}[1]{Eq.~\textup{(\ref{#1})}}
\newcommand{\Eref}[1]{Equation~\textup{(\ref{#1})}}
\newcommand{\esref}[1]{Eqs.~\textup{(\ref{#1})}}
\newcommand{\Esref}[1]{Equations~\textup{(\ref{#1})}}
\newcommand{\fref}[1]{Fig.~\ref{#1}}
\newcommand{\Fref}[1]{Figure~\ref{#1}}
\newcommand{\fsref}[1]{Figs.~\ref{#1}}
\newcommand{\Fsref}[1]{Figures~\ref{#1}}
\newcommand{\tref}[1]{Table~\ref{#1}}
\newcommand{\Tref}[1]{Table~\ref{#1}}
\newcommand{\tsref}[1]{Tables~\ref{#1}}
\newcommand{\Tsref}[1]{Tables~\ref{#1}}
\newcommand{\sref}[1]{Sec.~\ref{#1}}
\newcommand{\Sref}[1]{Section~\ref{#1}}
\newcommand{\ssref}[1]{Secs.~\ref{#1}}
\newcommand{\Ssref}[1]{Sections~\ref{#1}}
\newcommand{\thref}[1]{Theorem~\ref{#1}}    
\newcommand{\Thref}[1]{Theorem~\ref{#1}}
\newcommand{\thsref}[1]{Theorems~\ref{#1}}
\newcommand{\Thsref}[1]{Theorems~\ref{#1}}
\newcommand{\lref}[1]{Lemma~\ref{#1}}
\newcommand{\Lref}[1]{Lemma~\ref{#1}}
\newcommand{\lsref}[1]{Lemmas~\ref{#1}}
\newcommand{\Lsref}[1]{Lemmas~\ref{#1}}
\newcommand{\crref}[1]{Corollary~\ref{#1}}
\newcommand{\Crref}[1]{Corollary~\ref{#1}}
\newcommand{\crsref}[1]{Corollaries~\ref{#1}}
\newcommand{\Crsref}[1]{Corollaries~\ref{#1}}
\newcommand{\cref}[1]{Conjecture~\ref{#1}}
\newcommand{\Cref}[1]{Conjecture~\ref{#1}}
\newcommand{\csref}[1]{Conjectures~\ref{#1}}
\newcommand{\Csref}[1]{Conjectures~\ref{#1}}
\newcommand{\chref}[1]{Chapter~\ref{#1}}
\newcommand{\Chref}[1]{Chapter~\ref{#1}}
\newcommand{\chsref}[1]{Chapters~\ref{#1}}
\newcommand{\Chsref}[1]{Chapters~\ref{#1}}
\newcommand{\aref}[1]{Appendix~\ref{#1}}
\newcommand{\asref}[1]{Appendices~\ref{#1}}
\newcommand{\Aref}[1]{Appendix~\ref{#1}}
\newcommand{\Asref}[1]{Appendices~\ref{#1}}
\newcommand{\rcite}[1]{Ref.~\cite{#1}}
\newcommand{\rscite}[1]{Refs.~\cite{#1}}
%%%%%%%%%%%%%%%%%%%%%%%%%%%%%%%%%%%%%%%%%%%%%%%%%%%%%%%%%%%%%%%%%%%%%%%%%%%
\begin{document}
	\title{Minimum-consumption state discrimination with global optimal adaptive measurements }
	\date{\today}
	\begin{abstract}
Discriminating non-orthogonal quantum states for a fixed admissible error rate is a reliable starting point for many quantum information processing tasks. The key request is to minimize the average resource consumption.
%, where the most efficient measurements depend on our knowledge, the prior probability. 
By subtly using the updated posterior probability, here we develop a general global optimal adaptive (GOA) approach, which applies to any error rate requirement, any prior probability, and any measurement restrictions. Under local measurement restrictions, we achieve a global optimal adaptive local (GOAL) strategy, which is much more efficient than the previous global optimal fixed local projective (GOFP) method and serves as a local bound. When incorporating the more efficient two-copy collective measurements, we obtain a  global optimal adaptive collective (GOAC) strategy to further beat the local bound. We experimentally realize our GOAC method and demonstrate its efficiency advantages over GOAL and GOFP. By exploiting the power of both adaptivity and collective measurements, our work marks an important step in minimum-consumption quantum state discrimination.
% We propose   global optimal adaptive strategy to minimize average copy consumption for state discrimination. This strategy is applicable to any one-way measurement device, any possible states, any error rate requirement and any prior probability. Through this strategy, we obtained the minimum average copy consumption the   local measurement can realize. Besides, we built a two-copy collective measurement device and the experiment result proved that with global optimal adaptive, collective measurement can surpass the   local bound and further reduce the average copy consumption for mixed state discrimination.
	\end{abstract}
 \author{Boxuan Tian}
	\affiliation{CAS Key Laboratory of Quantum Information, University of Science and Technology of China, Hefei 230026, P. R. China}
	\affiliation{CAS Center For Excellence in Quantum Information and Quantum Physics, University of Science and Technology of China, Hefei 230026, P. R. China}
 \author{Wenzhe Yan}
	\affiliation{CAS Key Laboratory of Quantum Information, University of Science and Technology of China, Hefei 230026, P. R. China}
	\affiliation{CAS Center For Excellence in Quantum Information and Quantum Physics, University of Science and Technology of China, Hefei 230026, P. R. China}
    \author{Zhibo Hou}
	\email{houzhibo@ustc.edu.cn}
	\affiliation{CAS Key Laboratory of Quantum Information, University of Science and Technology of China, Hefei 230026, P. R. China}
	\affiliation{CAS Center For Excellence in Quantum Information and Quantum Physics, University of Science and Technology of China, Hefei 230026, P. R. China}
	\author{Guo-Yong Xiang}
	\email{gyxiang@ustc.edu.cn}
	\affiliation{CAS Key Laboratory of Quantum Information, University of Science and Technology of China, Hefei 230026, P. R. China}
	\affiliation{CAS Center For Excellence in Quantum Information and Quantum Physics, University of Science and Technology of China, Hefei 230026, P. R. China}
	\author{Chuan-Feng Li}
	\affiliation{CAS Key Laboratory of Quantum Information, University of Science and Technology of China, Hefei 230026, P. R. China}
	\affiliation{CAS Center For Excellence in Quantum Information and Quantum Physics, University of Science and Technology of China, Hefei 230026, P. R. China}
	\author{Guang-Can Guo}
	\affiliation{CAS Key Laboratory of Quantum Information, University of Science and Technology of China, Hefei 230026, P. R. China}
	\affiliation{CAS Center For Excellence in Quantum Information and Quantum Physics, University of Science and Technology of China, Hefei 230026, P. R. China}
	\maketitle
	\emph{Introduction.}-- Non-orthogonal quantum state discrimination, which can not be perfectly realized with $100\%$ success rate under a limited number of copies, is one of the core problems of quantum information \cite{ref1,ref2,ref3,ref4,ref5,ref6,ref7,ref8,ref9,ref10,ref11,ref12,ref13}. There are mainly two research directions of quantum state discrimination for two different purposes. One direction aims at achieving the minimum average error rate given a finite number of copies, which is well studied \cite{ref1,ref2,ref3,ref4,ref5,ref6,ref7,ref8,ref13}.
 % Figure environment removed
  The other new complementary direction studies how to use a minimum number of copies on average to achieve a given error rate requirement $\varepsilon$, which is of interest in this study\cite{ref9,ref10,ref11}. The strategy designed for this problem can save quantum resources to improve the efficiency of other quantum information processing tasks such as quantum communication\cite{ref14,ref15,ref16} and computation\cite{ref17,ref18,ref19}. The optimal measurements in the new direction cannot use those in the first one.% since they are different. 
  This is manifested experimentally in the case of optimal fixed local measurements \cite{ref9}, which performs the same local measurements on all the copies. 
 
 Compared with fixed measurements, adaptive measurements are much more efficient since optimal measurements in quantum state discrimination depend on our current knowledge (the prior probability) \cite{ref2,ref5,ref11}. \emph{Global optimal adaptive} (GOA) strategy employs the updated knowledge (the posterior probability) from consumed resources 
 to optimize the subsequent measurements for every copy  \cite{ref2,ref5}. The GOA strategy in the second direction is less explored. It is only in the small limit of error rate requirements ($-\mathrm{ln}\varepsilon$ is much larger than the quantum relative entropy), % and as an example, if the relative entropy of two possible states $\rho_0$ and $\rho_1$ is $D(\rho_0||\rho_1)=D(\rho_1||\rho_0)=1$, $\varepsilon$ should be approximately smaller than $e^{-100}$ ), 
 very recent studies have found the optimal adaptive strategy %, which both needs adaptivity and collective measurements 
 to discriminate two possible mixed states \cite{ref10,ref11}. 
%The theoretical and experimental research on this problem is still incomplete. 
For the general error rate,
%In theoretical aspect, when the error rate requirement is not very small, 
it is not yet clear how to design a GOA strategy to minimize the average copy consumption % through one-way measurement 
and what is the minimum number of consumed copies of local measurements (local bound). %In the experimental aspect, so far only global optimal fixed projective measurement has been realized \cite{ref9}. 

Collective measurements on multiple copies can extract more information than local measurements on the same number of copies.  The power of two-copy collective measurements has been demonstrated experimentally in photonic and superconducting platforms \cite{ref13,ref20,ref21,ref22}. It is not clear how to combine collective measurements with GOA strategy to further reduce copy consumption in the local bound. % through jointing two-copy collective measurement and classical commuication Thus collective measurements are not yet experimentally performed to achieve lower average copy consumption for mixed state discrimination than all local measurements. %.
 
% To fill this gap, here we discussed feasible methods to solve above problems using today's technology. At first, 
Here we develop a general GOA strategy that applies to any error rate requirement, any prior probability, and any (local or collective) measurements, which can be solved iteratively. When the measurements are first restricted to the most experimentally-friendly local measurement, we achieve a global optimal adaptive local (GOAL) strategy, which significantly outperforms the global optimal fixed local projective (GOFP) measurements and serves as the local bound. Then we extend our measurement capability to  two-copy collective measurements, we achieve a global optimal adaptive collective (GOAC) strategy to beat the local bound. We also experimentally built a two-copy collective measurement device to implement our GOAC strategy. Our experimental results demonstrated that our GOAC can beat the local bound for the general error rate. %even thoughthe error rate requirement is not very small.

\emph{Global optimal adaptive strategy.}--  Global optimal adaptive strategy %(set as GOA), 
which is searched through global thinking to consider all the possible measurements and results, is the optimal strategy to discriminate two quantum states $\rho_0$ and $\rho_1$. %Here one-way measurement means that each copy is only allowed to be measured once and this type of measurement is the main measurement form for optical and superconducting quantum measurement device. 
The GOA strategy for minimum-error rate discrimination has been found and it needs a global search from back to front, including both the measurement round and the prior probability \cite{ref2,ref5}.  

For minimum-consumption discrimination, there are an infinite number of available copies, which induces a translational symmetry, i.e., the remaining resources for every measurement are infinite. Thanks to this translational symmetry, the optimal adaptive strategy only depends on the prior probability $q$ of $\rho_0$ while independent of the measurement round, which significantly simplifies the global optimal adaptive strategy. % before the next measurement.
In the most general form, our GOA strategy considers both local and collective measurements. Suppose all the $n$-copy collective measurements that one measurement device can perform compose a set $D_n$, then the average copy consumption realized by global optimal adaptive strategy will satisfy
\begin{equation}
N_{\mathrm{GOA}}(q)=\left\{\begin{array}{c}
0 \quad \text { if } \min \left(q, 1-q\right) \leq \varepsilon\\
\min \limits_{n,\left\{M_k\right\} \in D_n}[n+\sum\limits_k P_k N_{\mathrm{GOA}}\left(q_k\right)] \text { otherwise }
\end{array}\right.
\end{equation}
with $P_k=\mathrm{tr} [M_k(q\rho_0^{\otimes n}+(1-q)\rho_1^{\otimes n})]$ denoting the probability of the measurement element $M_k$ and $ q_k=\frac{q \mathrm{tr}(M_k\rho_0^{\otimes n})}{P_k}$ denoting the corresponding posterior probability of $\rho_0$ according to the Bayesian law. As discussed, $N_{\mathrm{GOA}}$ only depends on the prior probability $q$. 
When $q$ or $1-q$ is already no larger than the error rate $\varepsilon$, we already know the state is $\rho_0$ or $\rho_1$ below the error rate $\varepsilon$ and no measurements are needed. Otherwise,we perform $n$-copy measurements and update the prior probability as the posterior probability $q_k$ for measurement outcome $k$. Thus $N_{\mathrm{GOA}}$ is equal to the sum of consumed $n$ copies in the measurement and the average copy consumption with the new updated prior probability $q_k$. It is optimal in the sense that the sum is minimized among all possible measurements.%This definition means that the minimum average number of copies the measurement device will cost only depends on the prior probability which is updated from our previous measurement results and is equal to the minimum value of the number of copies it will cost in the next measurement add the expected copy consumption after the next measurement.

since $N_{\mathrm{GOA}}$ appears both on the left- and right-hand sides, Eq.1 is an iterative definition, and can be solved iteratively. First, we choose an initial and realizable average consumption function $N_1\left(q\right)$. Then we search from all the possible measurements to minimize $N_2(q)$, i.e.\\
$$N_{\mathrm{2}}(q)=\left\{\begin{array}{c}
0 \quad \text { if } \min \left(q, 1-q\right) \leq \varepsilon\\
\min \limits_{n,\left\{M_k\right\} \in D_n}[n+\sum\limits_k P_k N_{\mathrm{1}}\left(q_k\right)] \text { otherwise }
\end{array}\right.$$
Similarly, we can generate $N_3, N_4, N_5, \cdots$, and we prove that (See SM for details.)
	\begin{equation}
		\lim _{i \rightarrow+\infty} N_i\left(q\right)=N_{\mathrm{GOA}}\left(q\right)
	\end{equation}
And the measurements to achieve $N_{\mathrm{GOA}}$ for ${i \rightarrow+\infty}$ form the global optimal adaptive strategy $\{M_k\}_{\mathrm{GOA}}(q)$, which guides us to choose subsequent measurements according to updated prior probability $q$.%For different prior probability $q$, the measurements that enable the minimum value to be obtain are different, and all these measurements form the global optimal adaptive strategy $\{M_k\}_{\mathrm{GOA}}(q)$.

\emph{Global optimal adaptive local strategy--}
We now restrict the measurements in GOA to be the most experimentally-friendly local measurements in discriminating two possible 2-dimensional real states $\rho_0$ and $\rho_1$ as a specific example. In this case, all local adaptive strategies can be described by a POVM with rank-one measurement elements $M(\theta)$:
	$M(\theta)\mathrm{d} \theta=S\left(q, \theta\right)(\cos \theta|0\rangle+\sin \theta|1\rangle)(\cos \theta\langle 0|+\sin \theta\langle 1|) \mathrm{d} \theta$
	with $\theta \in[0, \pi)$, depending on the prior probability $q$. And $S\left(q, \theta\right)$ uniquely determines the POVM, whose complete and positive semi-definite conditions in turn require
 $$
		\int_0^\pi S\left(q, \theta\right) \mathrm{d} \theta=2, \int_0^\pi S\left(q, \theta\right) \cos 2 \theta \mathrm{d} \theta=0 $$
	 $$\qquad \int_0^\pi S\left(q, \theta\right) \sin 2 \theta \mathrm{d} \theta=0  \text { \ and \ } S\left(q, \theta\right) \geq 0  \qquad $$
  % Figure environment removed
	\\All possible forms of the function $S\left(q, \theta\right)$ under these constraints compose the set $\mathcal{D}$ of local measurements. With this local measurement set $\mathcal{D}$, the GOA strategy becomes the global optimal adaptive local strategy, and Eq.1 is rewritten as
		\begin{equation}
			N_{\mathrm{GOAL}}\left(q\right)=\left\{\begin{array}{c}
				0 \qquad \text { if } \min \left(q, 1-q\right) \leq \varepsilon \\
				1+\min\limits _{S \in \mathcal{D}} \int_0^\pi P_\theta N_{\mathrm{GOAL}}\left(q_\theta\right) d\theta \ \text {otherwise}
			\end{array}\right.
		\end{equation}
where $\rho = q \rho_0+\left(1-q\right) \rho_1$, $P_\theta=\operatorname{tr}[M(\theta)\rho]$, $q_\theta=q \operatorname{tr}\left[M(\theta) \rho_0\right]/P_\theta$ is the posterior probability. 
 
 In general, the local POVM can have many elements, which makes the GOAL strategy in Eq.3 difficult to solve. However, we find the GOAL measurements to satisfy Eq.3 are either two-element projective measurements or three-element POVMs (see SM for details). Then following a similar iteration procedure, we can use Eq.3 to find the global optimal adaptive local strategy $S_{\mathrm{GOAL}}(q,\theta)$. 		
  In Fig.1(a) and (b), we present how to find $N_{\mathrm{GOAL}}(q)$ and the corresponding measurement strategy using the above iterative procedures. And in Fig.1(c) we compare the global optimal adaptive local strategy with the global optimal fixed local projective measurement in \cite{ref9} to demonstrate its efficiency in pure state discrimination. Note that $N_{\mathrm{GOAL}}(q)$ serves as the local bound since it is minimized among all possible local measurements.
 
 Besides, for the situation of two possible pure states, we find the analytical expression of global optimal adaptive local strategy by numerical calculation (See details in SM). But the strict mathematical proof is still missing.
 
 % Figure environment removed
 
 \emph{Global optimal adaptive collective strategy--}
 To further reduce the copy consumption and beat the local bound, we incorporate two-copy collective measurements into our GOA strategy to form a global optimal adaptive collective strategy for mixed state discrimination. For experimental ease, our two-copy collective measurement set is composed of special entangling projective measurements which are easy to implement, whose measurement bases are
$\left|\psi_1\right\rangle=|\theta_{+},\theta_{+}\rangle,
\left|\psi_2\right\rangle=\frac{1}{\sqrt{2}}(|\theta_{+},\theta_{-}\rangle
+|\theta_{-},\theta_{+}\rangle), $
$\left|\psi_3\right\rangle=\frac{1}{\sqrt{2}}(|\theta_{+},\theta_{-}\rangle
-|\theta_{-},\theta_{+}\rangle)$ and
$\left|\psi_4\right\rangle=|\theta_{-},\theta_{-}\rangle$, with $|\theta_{+}\rangle=\cos \theta|0\rangle+\sin \theta|1\rangle$, $|\theta_-\rangle=\sin\theta|0\rangle-\cos \theta|1\rangle$, and $\theta$ ranges from $-5^{\circ}$ to $20^{\circ}$. %built a two-copy collective measurement device (See Fig.2(a)). In the experiment, the sender can prepare two-copy states through encoding one copy on the photon's polarization freedom and encoding the other copy on its path freedom and the receiver can do two-copy angular momentum collective measurement which is an

In the small limit of $\varepsilon$, the highest efficiency of collective measurements under the prior probability $q$ is described by the ratio
 \begin{equation}
\begin{gathered}
\eta \equiv \lim _{\substack{\varepsilon \rightarrow 0 }} \frac{\langle N\rangle_{\min }}{-\ln \varepsilon}
=\frac{q}{\max \limits_{n,\left\{M_k\right\} \in D_n} E_0}+\frac{1-q}{\max \limits_{n,\left\{M_k\right\} \in D_n} E_1}
\end{gathered}
\end{equation}
where $E_0=\frac{1}{n} \sum\limits_k \operatorname{tr}\left(M_k \rho_0^{\otimes n}\right) \ln \frac{\operatorname{tr}\left(M_k \rho_0^{\otimes n}\right)}{\operatorname{tr}\left(M_k \rho_1^{\otimes n}\right)}$ and $E_1=\frac{1}{n} \sum\limits_k \operatorname{tr}\left(M_k \rho_1^{\otimes n}\right) \ln \frac{\operatorname{tr}\left(M_k \rho_1^{\otimes n}\right)}{\operatorname{tr}\left(M_k \rho_0^{\otimes n}\right)}$ are the maximum value of the probabilistic relative entropy divide by the number of copies measured collectively \cite{ref10,ref11}.

We experimentally implemented the above entangling collective measurements in Fig.2(a) and measured the ratio $\eta$ at $q=0.5$ to demonstrate its power in the small limit of $\varepsilon$. In the experiment, the sender encodes the two-copy states on the photon's polarization and path degrees of freedom \cite{ref21}. The receiver rotates H4 and H8 to sweep $\theta$ in the above two-copy collective measurements  from $-5^\circ$ to $20^\circ$ and measures the  probability distribution (See Fig.2(b)), which is used to calculate $E_0$ and $E_1$. And we use the maximum value of $E_0$ and $E_1$ to calculate the ratio $\eta$ of the collective measurements. From Fig2(c), experimental results of two-copy collective measurements not only surpass the GOFP but also beats GOAL, the local bound.%realize are larger than the   local bound which can be calculated by numerical search from all two or three element POVM on one copy (See Fig.2(c)). So when $\varepsilon$ is small enough, so that $\varepsilon$ is approximately smaller than $e^{-\frac{100}{\eta_{\mathrm{GOAL}}}}$, theoretically this measurement device is more efficient than all the one-way local strategies including GOFP strategy. But due to the extremely small error rate requirement, verifying this statement through experiment is difficult, because we must repeat the discrimination process at least $\frac{100}{\varepsilon}$ times to confirm whether the actual error rate meets the error rate requirement, and it will cost too much time.

For the general error rate $\varepsilon$, %Thus we need to discuss how to minimize the average copy consumption when $-\mathrm{ln}\varepsilon$ is not much larger than the relative entropy. In this situation, it is easy to notice that one-way
collective measurements may perform worse than local measurements when $q$ approaches $\varepsilon$ or $1-\varepsilon$ because every local measurement consumes only one copy while every two-copy collective measurement consumes two copies. % , while collective measurement will consume multiple copies. 
Besides, when $q$ is close to 0.5 (with little prior information), the collective measurement is also worse than local measurements. %This is because the collective measurement can only maximize one of $E_0$ and $E_1$ and this effect will play an important role when $\min\{q,1-q\}\ll0.5$, but it is not clear if this effect is important when $q$ is close to 0.5 . 
Thus our GOAC strategy also allows local projective measurements with
$\left|\phi_1\right\rangle=\cos \theta|0\rangle+\sin \theta|1\rangle$ and $\left|\phi_2\right\rangle=\sin \theta|0\rangle-\cos \theta|1\rangle$.
where $\theta$ ranges from $0^{\circ}$ to $90^{\circ}$. %And we set the global optimal adaptive local strategy which uses our device to do two-copy angular momentum collective measurement and one copy projective measurement as GOAC. 
The GOAC strategy is then adapted from Eq.1 as% (Here $\varepsilon \leq q\leq 1-\varepsilon$. )
\begin{equation}
\begin{gathered}
N_{\mathrm{G O A C}}\left(q\right)=\min \left\{\min _{\theta \in\left[0^{\circ}, 90^{\circ}\right)}\left[1+\sum_{k=1}^2 P_{k}^L N_{\mathrm{G O A C}}\left(q_{1 k}\right)\right],\right. \\
\left.\min _{\theta \in\left[-5^{\circ}, 20^{\circ}\right]}\left[2+\sum_{k=1}^4 P_{k}^C N_{\mathrm{G O A C}}\left(q_{2 j}\right)\right]\right\}
\end{gathered}
\end{equation}where $P_{k}^L=q \operatorname{tr}\left[\left|\phi_k\right\rangle\left\langle\phi_k\right| \rho_0\right]+(1-q) \operatorname{tr}\left[\left|\phi_k\right\rangle\left\langle\phi_k\right| \rho_1\right]$ and $P_{k}^C=q \operatorname{tr}\left[\left|\psi_k\right\rangle\left\langle\psi_k\right| \rho_0^{\otimes 2}\right]+(1-q) \operatorname{tr}\left[\left|\psi_k\right\rangle\left\langle\psi_k\right| \rho_1^{\otimes 2}\right]$ are the measurement probability for local projective measurement and two-copy collective measurements, respectively. And $q_{1k}=\frac{q \operatorname{tr}\left[\left|\phi_k\right\rangle\left\langle\phi_k\right| \rho_0\right]}{P_{\theta1k}}$ and $q_{2k}=\frac{q \operatorname{tr}\left[\left|\psi_k\right\rangle\left\langle\psi_k\right| \rho_0^{\otimes 2}\right]}{P_{\theta2k}}$ are corresponding posterior probability. 

Using the iterative method, we find the GOAC solutions in Eq.5. % $P_{2j}$ and $q_{2j}$ by measuring the probability distribution of the two possible states.  And we take the value of $P_{1j}$ and $q_{1j}$ as the theoretical value. Then through similar iterative method, we can search the GOAC strategy which is 
In Fig3.(a), it is clear that GOAC surpasses GOFP and also beats the local bound of the GOAL strategy for almost all the prior probabilities at an error rate of 0.0001. The measurements of GOAC strategy is shown in Fig.3(b) 
 and its four jumping points are roughly located at 0.001, 0.4,0.6 and 0.999, representing the transition from collective meausrement to projective measurement and they verify that if the prior probability $q$ is close to $\varepsilon$ or $1-\varepsilon$ or 0.5, local measurements are better than collective measurements. %And in other area where $q$ is not close to these values, we should do two-copy angular momentum collective measurement to minimize one of $E_0$ and $E_1$ .

GOAC experiments are also implemented to demonstrate its efficiency advantages. Shifting to one-copy state preparation and projective measurement is realized by resetting H2 and H4, respectively. The  GOAC strategy is repeated $10^6$ times and the results are shown in Fig.3(c). The actual average copy consumption and error rate of the experimental results are close to the theory and the actual average copy consumption clearly beats the GOAL, the local bound. %It further demonstrates the advantage of collecitve measurement and proves that our global optimal adaptive strategy searching method is reliable. As repeating the discrimination process costs too much time, in actual application, measuring all the probability distribution the measurement device can realize and use our iterative method to calculate will be a more efficient way to test the device.

\emph{Summary.}--We developed a general global optimal adaptive method for minimum-consumption quantum state discrimination, which can be solved iteratively. Our method applies to general error rates, and arbitrary prior probabilities, and is flexible to different measurement restrictions. %that can maximize quantum resource savings in quantum state discrimination for any   measurement device.It also provides us with the way to deal with the situation that the error rate requirement is not very small.Through this strategy, 
Under local measurement restrictions, we  achieve a GOAL strategy with a local bound surpassing previous optimal fixed measurements in \cite{ref9}. We also incorporate two-copy collective measurements into our GOA strategy to form a GOAC strategy, which further reduces copy consumption. We experimentally realized our GOAC strategy and experimentally beat the local bound both in the mall limit of error rate and for general error rate. This in turn also pushes forward the development of quantum collective measurements \cite{ref10,ref11,ref13,ref20,ref21,ref22,ref23,ref24,ref25,ref26} and quantum control \cite{ref6,ref27,ref28,ref29,ref30,ref31,ref32}.

The work at the University of Science and Technology of China is supported the National Natural Science Foundation of China (Grants Nos. 62222512, 12104439, 12134014, and 11974335), the Key Research Program of Frontier Sciences, CAS (Grant No. QYZDYSSW-SLH003), the Anhui Provincial Natural Science Foundation (Grant No.2208085J03), USTC Research Funds of the Double First-Class Initiative (Grant Nos. YD2030002007 and YD2030002011) and the Fundamental Research Funds for the Central Universities (Grant No. WK2470000035).


%In the experiment, we show how to use global optimal adaptive strategy to demonstrate the advantage of our collective measurement device. And our two-copy collective measurement experiment is the first one to demonstrate the advantages of collective measurement in minimum-consumption state discrimination problem. Besides, our work demonstrate the effectiveness of adaptive control in collective measurement for the first time and this is an important contribution to the development of quantum collective measurement technology\cite{ref10,ref11,ref13,ref14,ref15,ref16,ref17,ref18,ref19,ref20} and quantum control technology\cite{ref6,ref27,ref28,ref29,ref30,ref31,ref32}.

%Our research remains an important open problem. Based on Eq.1, if we can search all the possible measurement form from all the sets $D_1$,$D_2$,$D_3$,$\cdots$, in principle we can obtain the global optimal adaptive multiple-copy collective measurement strategy which is the optimal one of all one-way strategies. However, when $n>1$, $D_n$ has so much freedom and constraints that we don't know how to perform numerical searches on it. If this problem is resolved, we will obtain the one-way collective bound and then through studying if this bound is the lower bound for all the measurement, we may absolutely solve the theoretical problem for minimum-consumption state discrimination.

%\bibliographystyle{apsrev_no_url}
%\bibliography{all_references}
\begin{thebibliography}{50}
	\bibitem{ref1}C. W. Helstrom, Quantum Detection and Estimation Theory
	(Academic Press, New York, 1976).
	\bibitem{ref2} B.L.Higgins,B.M.Booth,A.C.Doherty,S.D.Bartlett,\\ H.M.Wiseman,andG.J.Pryde,Phys.Rev.Lett.103, 220503 (2009).
	\bibitem{ref3} A. Peres andW. K.Wootters, Phys. Rev. Lett. 66, 1119 (1991).
	\bibitem{ref4}J. Calsamiglia, J. de Vicente, R. Muñoz-Tapia, and E.
	Bagan, Local Discrimination of Mixed States, Phys. Rev.
	Lett. 105, 080504 (2010).
	\bibitem{ref5}B. L. Higgins, A. C. Doherty, S. D. Bartlett, G. J. Pryde, and H. M. Wiseman, Multiple-copy state discrimination: Thinking globally, acting locally, Phys. Rev. A 83, 052314 (2011).
	\bibitem{ref6} H. M. Wiseman and G. J. Milburn, Quantum Measurement
	and Control (Cambridge University Press, Cambridge,
	England, 2009).
	\bibitem{ref7} A. Acín, E. Bagan, M. Baig, L. Masanes, and R. Munoz Tapia, Phys. Rev. A 71, 032338 (2005).
	\bibitem{ref8} D. Brody and B. Meister, Phys. Rev. Lett. 76, 1 (1996).
	\bibitem{ref9} S. Slussarenko, M. M. Weston, J.-G. Li, N. Campbell, H. M.
	Wiseman, and G. J. Pryde, Phys. Rev. Lett. 118, 030502
	(2017).
	\bibitem{ref10}Esteban Martínez Vargas, Christoph Hirche, Gael Sentís, Michalis Skotiniotis, Marta Carrizo, Ramon Muñoz-Tapia, and John Calsamiglia, Phys. Rev. Lett. 126, 180502 (2021).
	\bibitem{ref11}Li, Y., Tan, V.Y.F. and Tomamichel, M. Optimal Adaptive Strategies for Sequential Quantum Hypothesis Testing. Commun. Math. Phys. 392, 993–1027 (2022).
	\bibitem{ref12} Joseph M.Renes, Robin Blume-Kohout, A.J.Scott, Carlton M.Caves, J. Math. Phys. 45, 2171 (2004)
 \bibitem{ref13}Lorc´an; O. Conlon, Falk Eilenberger Ping Koy Lam and Syed M. Assad . Discriminating qubit states with entangling collective measurements. arXiv:2302.08882
 \bibitem{ref14}C. H. Bennett, Phys. Rev. Lett. 68, 3121 (1992).
 \bibitem{ref15} N. Gisin, G. Ribordy, W. Tittel, and H. Zbinden, Rev. Mod.
Phys. 74, 145 (2002).
 \bibitem{ref16}S. J. van Enk, Phys. Rev. A 66, 042313 (2002).
 H. M. Wiseman, Phys. Rev. A 83, 052314 (2011).
\bibitem{ref17} E. Knill, R. Laflamme, and W. H. Zurek, Proc. R. Soc. A
454, 365 (1998).
\bibitem{ref18} D. Aharonov and M. Ben-Or, in Proceedings of the Twenty Ninth Annual ACM Symposium on Theory of Computing,STOC ’97 (ACM, New York, 1997), p. 176.
\bibitem{ref19} C. H. Bennett and D. P. di Vincenzo, Nature (London) 404,247 (2000).
\bibitem{ref20}Kang-Da Wu, Elisa Bäumer, Jun-Feng Tang, Karen V. Hovhannisyan, Martí Perarnau-Llobet, Guo-Yong Xiang, Chuan-Feng Li, and Guang-Can Guo Phys. Rev. Lett. 125, 210401 (2020)
	\bibitem{ref21}Hou, Z., Tang, JF., Shang, J. et al. Deterministic realization of collective measurements via photonic quantum walks. Nat Commun 9, 1414  (2018)
	\bibitem{ref22}Conlon, L.O., Vogl, T., Marciniak, C.D. et al. Approaching optimal entangling collective measurements on quantum computing platforms. Nat. Phys. (2023).
	\bibitem{ref23} M. Fanizza, M. Rosati, M. Skotiniotis, J. Calsamiglia, and V. Giovannetti Phys. Rev. Lett. 124, 060503 (2020)
	\bibitem{ref24}N. Linden, S. Massar, and S. Popescu
	Phys. Rev. Lett. 81, 3279 (1998)
	\bibitem{ref25}Martí Perarnau-Llobet, Elisa Bäumer, Karen V. Hovhannisyan, Marcus Huber, and Antonio Acin Phys. Rev. Lett. 118, 070601 (2017)
	\bibitem{ref26}Jorge Miguel-Ramiro, Ferran Riera-Sàbat, and Wolfgang Dür
	Phys. Rev. Lett. 129, 190504 (2022)
 \bibitem{ref27}J. M. Geremia, J. K. Stockton, A. C. Doherty, and H.
Mabuchi, Phys. Rev. Lett. 91, 250801 (2003).
\bibitem{ref28} A. M. Bran´czyk et al., Phys. Rev. A 75, 012329 (2007).
\bibitem{ref29} M. A. Armen et al., Phys. Rev. Lett. 89, 133602 (2002).
\bibitem{ref30} R. L. Cook, P. J. Martin, and J. M. Geremia, Nature
(London) 446, 774 (2007).
\bibitem{ref31} R. B. Griffiths and C. S. Niu, Phys. Rev. Lett. 76, 3228
(1996).
\bibitem{ref32} J. Walgate, A. J. Short, L. Hardy, and V. Vedral, Phys. Rev.
Lett. 85, 4972 (2000)
\end{thebibliography}	
\end{document}