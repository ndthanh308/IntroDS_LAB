%\documentclass[aps,pra,superscriptaddress,showpacs,twocolumn,draft]{revtex4}
\documentclass[aps,prl,twocolumn,superscriptaddress]{revtex4-1}
\usepackage{graphicx}
\usepackage{subfigure}
\usepackage{epstopdf}
%\usepackage{subfigure}
%\usepackage{epsfig}
\usepackage{amsmath}
\usepackage{amssymb}
\usepackage{amsfonts}
\usepackage{mathrsfs}
\usepackage{theorem}
\usepackage{bm}
\usepackage{url}
\usepackage[T1]{fontenc}
\usepackage{csquotes}
\MakeOuterQuote{"}
\usepackage{algorithm}
\usepackage{algorithmicx}
\usepackage{algpseudocode}
\renewcommand{\thealgorithm}{\!\!:}
\usepackage{dcolumn}
\usepackage{color}
%\usepackage{biblatex}
%\addbibresource{cite.bib}
%\usepackage[colorlinks,citecolor=blue]{hyperref}
%\usepackage{authblk}
%\usepackage{cite}
%%%%%%%%%%%%%%%%%%%%%%%%%%%%%%%%%%%%%%%%%%%%%%%%%%%%%%%%%%%%%%%%%%%%%%%%%%%%%%%%%%%%%%%%%%%%%%%%%%%%%%%%%%%%%%%%%%%%%%%%%%%%%%%%%%%%%%%%%%\usepackage{graphicx}
\newcommand{\red}{\color{red}}
\newcommand{\blu}{\color{blue}}
\newcommand{\blk}{\color{black}}
\definecolor{ngreen}{rgb}{0.2,0.6,0.2}
\newcommand{\grn}{\color{ngreen}}
\newcommand{\hmw}[1]{{\color{ngreen} \bf [[{#1}]]}}
\newcommand{\dwb}[1]{{\color{blue} \bf [[{#1}]]}}
\definecolor{ngold}{rgb}{0.7,0.6,0.2}
\newcommand{\gold}{\color{ngold}}
\newcommand{\blh}[1]{{\color{ngold} \bf [[{#1}]]}}
%%%%%%%%%%%%%%%%%%%%%%%%%%%%%%%%%%%%%%%%%%%%%%%%%%%%%%%%%%%%%%%%%%%%%%%%%%%%%%
%% define mathematical words via abbreviations.
\def\vec#1{\mathbf{#1}} %% overiding the original command
\newcommand{\mrm}[1]{\mathrm{#1}}
\newcommand{\tr}{\operatorname{tr}}
\newcommand{\Tr}{\operatorname{Tr}}
\providecommand{\det}{\operatorname{det}}
\newcommand{\Det}{\operatorname{Det}}
\newcommand{\diag}{\operatorname{diag}}
\newcommand{\sgn}{\operatorname{sgn}}
\newcommand{\ad}{\operatorname{ad}}
\newcommand{\rep}{\mathrel{\widehat{=}}}
\newcommand{\rmi}{\mathrm{i}}
\newcommand{\rme}{\mathrm{e}}
\newcommand{\rmE}{\mathrm{E}}
\newcommand{\rmd}{\mathrm{d}}
\newcommand{\rmT}{\mathrm{T}}
\newcommand{\imply}{\mathrel{\Rightarrow}}
\newcommand{\equi}{\mathrel{\Leftrightarrow}}
\newcommand{\one}{\overline{1}}
\newcommand{\zero}{\overline{0}}
%%%%%%%%%%%%%%%%%%%%%%%%%%%%%%%%%%%%%%%%%%%%%%%%%%%%%%%%%%%%%%%%%%%%%%%%%%%%%%%%
\newcommand{\be}{\begin{equation}}
	\newcommand{\ee}{\end{equation}}
\newcommand{\ba}{\begin{align}}
	\newcommand{\ea}{\end{align}}
%%%%%%%%%%%%%%%%%%%%%%%%%%%%%%%%%%%%%%%%%%%%%%%%%%%%%%%%%%%%%%%%%%%%%%%%%%%%%%%%%
\def\<{\langle}  %% overiding the original command \<
\def\>{\rangle}  %% overiding the original command \>
\newcommand{\ket}[1]{| #1\>}
\newcommand{\bra}[1]{\< #1|}
\newcommand{\dket}[1]{| #1\>\!\>}
\newcommand{\Dket}[1]{\Bigl| #1\Bigr\>\!\Bigr\>}
\newcommand{\dbra}[1]{\<\!\< #1|}
\newcommand{\Dbra}[1]{\Bigl\<\!\Bigl\< #1\Bigr|}
\newcommand{\inner}[2]{\<#1|#2\>}
\def\outer#1#2{|#1\>\<#2|}       %% overiding the original command \outer
\newcommand{\dinner}[2]{\<\!\< #1| #2\>\!\>}
\newcommand{\Dinner}[2]{\Bigl\<\!\Bigl\< #1\Bigl| #2\Bigr\>\!\Bigr\>}
\newcommand{\douter}[2]{| #1\>\!\>\<\!\< #2|}
\newcommand{\Douter}[2]{\Bigl| #1\Bigr\>\!\Bigr\>\Bigl\<\!\Bigl\< #2\Bigr|}
\newcommand{\norm}[1]{\parallel\!#1\!\parallel}
\newcommand{\nfrac}[2]{\genfrac{}{}{0pt}{}{#1}{#2}}
%%%%%%%%%%%%%%%%%%%%%%%%%%%%%%%%%%%%%%%%%%%%%%%%%%%%%%%%%%%%%%%%%%%%%%%%%%%%%%%%%%%
%% Abbreviations used in quantum estimation theory
\newcommand{\mse}{\mathcal{E}}
\newcommand{\mhs}{\mathcal{E}_{\mathrm{HS}}}
\newcommand{\msh}{\mathcal{E}_{\mathrm{SH}}}
\newcommand{\msb}{\mathcal{E}_{\mathrm{SB}}}
\newcommand{\mtr}{\mathcal{E}_{\tr}}
\newcommand{\barcal}[1]{\bar{\mathcal{#1}}}
\newcommand{\bt}{\bar{t}}
\newcommand{\bid}{\bar{\mathbf{I}}}
\newcommand{\cF}{\mathcal{F}}
%%%%%%%%%%%%%%%%%%%%%%%%%%%%%%%%%%%%%%%%%%%%%%%%%%%%%%%%%%%%%%%%%%%%%%%%%%%%%%%%%%%%%%%%%%%%%
%% Abbreviations used in cross references and citations
%\def\eqref#1{(\ref{#1})}    %% overiding the original command \eqref
%\newcommand{\eref}[1]{Eq.~(\ref{#1})}
%\newcommand{\Eref}[1]{Equation~(\ref{#1})}
%\newcommand{\esref}[1]{Eqs.~(\ref{#1})}
%\newcommand{\Esref}[1]{Equations~(\ref{#1})}
%\def\eqref#1{\textup{(}\ref{#1}\textup{)}}  %% overiding the original command \eqref
%\newcommand{\eref}[1]{Eq.~\textup{(}\ref{#1}\textup{)}}
%\newcommand{\Eref}[1]{Equation~\textup{(}\ref{#1}\textup{)}}
%\newcommand{\esref}[1]{Eqs.~\textup{(}\ref{#1}\textup{)}}
%\newcommand{\Esref}[1]{Equations~\textup{(}\ref{#1}\textup{)}}
\def\eqref#1{\textup{(\ref{#1})}}  %% overiding the original command \eqref
\newcommand{\eref}[1]{Eq.~\textup{(\ref{#1})}}
\newcommand{\Eref}[1]{Equation~\textup{(\ref{#1})}}
\newcommand{\esref}[1]{Eqs.~\textup{(\ref{#1})}}
\newcommand{\Esref}[1]{Equations~\textup{(\ref{#1})}}
\newcommand{\fref}[1]{Fig.~\ref{#1}}
\newcommand{\Fref}[1]{Figure~\ref{#1}}
\newcommand{\fsref}[1]{Figs.~\ref{#1}}
\newcommand{\Fsref}[1]{Figures~\ref{#1}}
\newcommand{\tref}[1]{Table~\ref{#1}}
\newcommand{\Tref}[1]{Table~\ref{#1}}
\newcommand{\tsref}[1]{Tables~\ref{#1}}
\newcommand{\Tsref}[1]{Tables~\ref{#1}}
\newcommand{\sref}[1]{Sec.~\ref{#1}}
\newcommand{\Sref}[1]{Section~\ref{#1}}
\newcommand{\ssref}[1]{Secs.~\ref{#1}}
\newcommand{\Ssref}[1]{Sections~\ref{#1}}
\newcommand{\thref}[1]{Theorem~\ref{#1}}    
\newcommand{\Thref}[1]{Theorem~\ref{#1}}
\newcommand{\thsref}[1]{Theorems~\ref{#1}}
\newcommand{\Thsref}[1]{Theorems~\ref{#1}}
\newcommand{\lref}[1]{Lemma~\ref{#1}}
\newcommand{\Lref}[1]{Lemma~\ref{#1}}
\newcommand{\lsref}[1]{Lemmas~\ref{#1}}
\newcommand{\Lsref}[1]{Lemmas~\ref{#1}}
\newcommand{\crref}[1]{Corollary~\ref{#1}}
\newcommand{\Crref}[1]{Corollary~\ref{#1}}
\newcommand{\crsref}[1]{Corollaries~\ref{#1}}
\newcommand{\Crsref}[1]{Corollaries~\ref{#1}}
\newcommand{\cref}[1]{Conjecture~\ref{#1}}
\newcommand{\Cref}[1]{Conjecture~\ref{#1}}
\newcommand{\csref}[1]{Conjectures~\ref{#1}}
\newcommand{\Csref}[1]{Conjectures~\ref{#1}}
\newcommand{\chref}[1]{Chapter~\ref{#1}}
\newcommand{\Chref}[1]{Chapter~\ref{#1}}
\newcommand{\chsref}[1]{Chapters~\ref{#1}}
\newcommand{\Chsref}[1]{Chapters~\ref{#1}}
\newcommand{\aref}[1]{Appendix~\ref{#1}}
\newcommand{\asref}[1]{Appendices~\ref{#1}}
\newcommand{\Aref}[1]{Appendix~\ref{#1}}
\newcommand{\Asref}[1]{Appendices~\ref{#1}}
\newcommand{\rcite}[1]{Ref.~\cite{#1}}
\newcommand{\rscite}[1]{Refs.~\cite{#1}}
%%%%%%%%%%%%%%%%%%%%%%%%%%%%%%%%%%%%%%%%%%%%%%%%%%%%%%%%%%%%%%%%%%%%%%%%%%%
\begin{document}
	\title{Minimum-consumption discrimination of quantum states via globally optimal adaptive measurements }
	\date{\today}
	\begin{abstract}
Reducing the average resource consumption is the central quest in discriminating non-orthogonal quantum states for a fixed admissible error rate $\varepsilon$. The globally optimal fixed local projective measurement (GOFL) for this task is found to be different from that for previous minimum-error discrimination tasks [PRL 118, 030502 (2017)].
To achieve the ultimate minimum average consumption, here we develop a general globally optimal adaptive strategy (GOA) by subtly using the updated posterior probability, which works under any error rate requirement and any one-way measurement restrictions, and can be solved by a convergent iterative relation. First, under the local measurement restrictions, our GOA is solved to serve as the local bound, which saves 16.6 copies ($24\%$) compared with the previously best GOFL. When the more powerful two-copy collective measurements are allowed, our GOA is experimentally demonstrated to beat the local bound by 3.9 copies ($6.0\%$). By exploiting both adaptivity and collective measurements, our work marks an important step towards minimum-consumption quantum state discrimination.
	\end{abstract}
 \author{Boxuan Tian}
	\affiliation{CAS Key Laboratory of Quantum Information, University of Science and Technology of China, Hefei 230026, P. R. China}
	\affiliation{CAS Center For Excellence in Quantum Information and Quantum Physics, University of Science and Technology of China, Hefei 230026, P. R. China}
 \author{Wen-Zhe Yan}
	\affiliation{CAS Key Laboratory of Quantum Information, University of Science and Technology of China, Hefei 230026, P. R. China}
	\affiliation{CAS Center For Excellence in Quantum Information and Quantum Physics, University of Science and Technology of China, Hefei 230026, P. R. China}
    \author{Zhibo Hou}
	\email{houzhibo@ustc.edu.cn}
	\affiliation{CAS Key Laboratory of Quantum Information, University of Science and Technology of China, Hefei 230026, P. R. China}
	\affiliation{CAS Center For Excellence in Quantum Information and Quantum Physics, University of Science and Technology of China, Hefei 230026, P. R. China}
 \affiliation{Hefei National Laboratory, University of Science and Technology of China, Hefei 230088, People's Republic of China}
	\author{Guo-Yong Xiang}
	\email{gyxiang@ustc.edu.cn}
	\affiliation{CAS Key Laboratory of Quantum Information, University of Science and Technology of China, Hefei 230026, P. R. China}
	\affiliation{CAS Center For Excellence in Quantum Information and Quantum Physics, University of Science and Technology of China, Hefei 230026, P. R. China}
 \affiliation{Hefei National Laboratory, University of Science and Technology of China, Hefei 230088, People's Republic of China}
	\author{Chuan-Feng Li}
	\affiliation{CAS Key Laboratory of Quantum Information, University of Science and Technology of China, Hefei 230026, P. R. China}
	\affiliation{CAS Center For Excellence in Quantum Information and Quantum Physics, University of Science and Technology of China, Hefei 230026, P. R. China}
 \affiliation{Hefei National Laboratory, University of Science and Technology of China, Hefei 230088, People's Republic of China}
	\author{Guang-Can Guo}
	\affiliation{CAS Key Laboratory of Quantum Information, University of Science and Technology of China, Hefei 230026, P. R. China}
	\affiliation{CAS Center For Excellence in Quantum Information and Quantum Physics, University of Science and Technology of China, Hefei 230026, P. R. China}
 \affiliation{Hefei National Laboratory, University of Science and Technology of China, Hefei 230088, People's Republic of China}
	\maketitle
 % Figure environment removed
 \emph{Introduction.}-- 
Non-orthogonal quantum state discrimination, which can not be perfectly realized with $100\%$ success rate under a limited number of copies, is one of the core problems of quantum information\cite{ref1,ref2,ref4,ref5,ref6,ref7,ref8,ref9,ref10,ref11,ref12,ref13,ref3,ref32,ref30}. Two primary research directions in multiple-copy quantum state discrimination have emerged. One seeks to minimize the average rate of errors given a finite number of copies, a field with a rich history spanning decades \cite{ref1,ref2,ref4,ref5,ref6,ref7,ref8,ref13,ref32,ref30}. The other emerging direction aims to use the minimum average number of copies to achieve a specified error rate requirement $\varepsilon$, thereby conserving quantum resources and enhancing applications like quantum communication \cite{ref14,ref15,ref16} and computation \cite{ref17,ref18,ref19}. Notably, this second direction necessitates different optimal measurement strategies compared to the first, as exemplified by the globally optimal fixed local projective measurement (GOFL) \cite{ref9}, which applies identical local projective measurements to all copies.

Adaptive measurements, which adapt based on evolving knowledge updated with measurement outcomes, offer increased efficiency over fixed measurements \cite{ref2,ref5,ref11}. The most efficient adaptive strategy is \emph{globally optimal adaptive measurement} (GOA), which uses global thinking of all resources rather than the local optimum of each resource to optimize subsequent measurements \cite{ref2,ref5} and thus is optimal among all possible schemes. The GOA strategy in minimum-consumption discrimination is barely explored except for in the small limit of error rate requirements ($-\mathrm{ln}\varepsilon$ is much larger than the quantum relative entropy) \cite{ref10,ref11}.

Collective measurements represent a pivotal concept in quantum state discrimination. They involve simultaneously probing multiple copies of a quantum state, allowing for the extraction of richer information than what individual local measurements on the same number of copies can provide. This property makes collective measurements particularly valuable in scenarios where optimizing resource usage is paramount\cite{ref3,ref13,ref21,ref22,ref24,ref23,ref25,ref33,ref34,ref36}. The power of two-copy collective measurements has been demonstrated experimentally in quantum state tomography\cite{ref21,ref34,ref36}, multi-parameter estimation\cite{ref22,ref24,ref33}, backaction reduction\cite{ref20,ref35}, minimum-error discrimination and orienteering on photonic \cite{ref20,ref21,ref24,ref33,ref35,ref36} and superconducting platforms\cite{ref13,ref22}. These collective strategies harness the quantum entanglement and correlations inherent in multi-copy quantum states to enhance measurement precision and discrimination capabilities, opening up new avenues for resource-efficient quantum information processing.

In this study, we introduce a general GOA strategy that applies under any error rate requirement and any one-way measurement restriction (local or collective) for minimum-consumption discrimination. We present an iterative approach with proven convergence. Initially, for the simplest local measurements, we develop a globally optimal adaptive local strategy (GOAL), serving as the local bound while surpassing GOFL's performance. Subsequently, by incorporating two-copy collective measurements, we advance to a globally optimal adaptive collective strategy (GOAC) that outperforms the local bound. We also experimentally implement GOAC, showcasing its effectiveness in surpassing the local bound for both the small limit of error rate and general error rates.



\emph{Globally Optimal Adaptive Strategy}--
The GOA strategy for minimum-error rate discrimination of two quantum states $\rho_0$ and $\rho_1$ involves not only the prior probability but also the measurement round to do a global search from back to front because of the finite number of copies given \cite{ref2,ref5}. However, in the context of minimum-consumption discrimination, where an infinite number of copies is at our disposal, a unique feature emerges: translational symmetry. This symmetry implies that, regardless of the measurement round, the remaining resources remain infinite and independent, simplifying the GOA strategy considerably.

In its most general form, our GOA strategy takes into account collective measurements involving $n$ copies, with local measurements being a special case when $n=1$. If we consider a set of $n$-copy collective measurements denoted as $\left\{M_k\right\}$, collectively forming a set $D_n$, then the average copy consumption under the GOA strategy can be described as follows:
\begin{equation}
N_{\mathrm{GOA}}(q)=\left\{\begin{array}{c}
0 \quad \text { if } \min \left(q, 1-q\right) \leq \varepsilon,\\
\min \limits_{n,\left\{M_k\right\} \in D_n}[n+\sum\limits_k P_k N_{\mathrm{GOA}}\left(q_k\right)] \text { otherwise. }
\end{array}\right.\label{Eq1}
\end{equation}
Here, $P_k=\mathrm{tr} [M_k(q\rho_0^{\otimes n}+(1-q)\rho_1^{\otimes n})]$ represents the measurement probability of the element $M_k$, and $ q_k=q \mathrm{tr}(M_k\rho_0^{\otimes n})/{P_k}$ corresponds to the posterior probability of $\rho_0$ according to the Bayesian law, updated after obtaining measurement outcome $k$. Importantly, $N_{\mathrm{GOA}}$ depends solely on the prior probability $q$. If $q$ or $1-q$ is already smaller than the error rate $\varepsilon$, no measurements are necessary. Otherwise, we proceed with $n$-copy measurements and update the prior probability using the posterior probability $q_k$. In essence, $N_{\mathrm{GOA}}$ represents the sum of consumed copies in the measurement process and the average copy consumption considering the updated probability $q_k$. It's globally optimal in the sense that it minimizes this sum across all possible local or collective measurements for all values of $q$.

The minimum average consumption $N_{\mathrm{GOA}}$ in \eref{Eq1} is inherently iterative as it appears both sides of the equation, and can be solved iteratively. To initiate this process, we select an initial and feasible average consumption function $N_1\left(q\right)$. Subsequently, we explore all possible measurements to minimize $N_2(q)$ as follows:
$$N_{\mathrm{2}}(q)=\left\{\begin{array}{c}
0 \quad \text { if } \min \left(q, 1-q\right) \leq \varepsilon,\\
\min \limits_{n,\left\{M_k\right\} \in D_n}[n+\sum\limits_k P_k N_{\mathrm{1}}\left(q_k\right)] \text { otherwise. }
\end{array}\right.$$
By following this iterative approach and generating $N_3, N_4, N_5,$ and so on, we can demonstrate its convergence, i.e.(see SM for details),
	\begin{equation}
		\lim _{i \rightarrow+\infty} N_i\left(q\right)=N_{\mathrm{GOA}}\left(q\right)
	\end{equation}
The measurements denoted as $\{M_k\}_{\mathrm{GOA}}(q)$ that achieve $N_{\mathrm{GOA}}$ constitute the globally optimal adaptive strategy. This strategy guides us in selecting adaptive measurements based on updated probability information.


\emph{Globally optimal adaptive local strategy--}
In our current analysis, we narrow our focus within the GOA framework to the most experimentally-friendly local measurements for the discrimination of two possible 2-dimensional real states, $\rho_0$ and $\rho_1$, as a specific example. In this case, all local adaptive strategies can be described by a POVM with rank-one measurement elements $M(\theta)\mathrm{d} \theta$:
	$M(\theta)\mathrm{d} \theta=S_q(\theta)(\cos \theta|0\rangle+\sin \theta|1\rangle)(\cos \theta\langle 0|+\sin \theta\langle 1|) \mathrm{d} \theta,$
	with $\theta \in[0, \pi)$, depending on the prior probability $q$. And $S_q(\theta)$ uniquely determines the POVM and needs to satisfy completeness and semidefinite positivity with four specific conditions
 $\int_0^\pi S_q(\theta) \mathrm{d} \theta=2, \quad \int_0^\pi S_q(\theta) \cos 2 \theta \mathrm{d} \theta=0, \quad \int_0^\pi S_q(\theta) \sin 2 \theta \mathrm{d} \theta=0  \text { \ and \ } S_q(\theta) \geq 0.$ 
  % Figure environment removed

All possible forms of the function $S_q(\theta)$ under these constraints compose the set $\mathcal{D}_1$ of local measurements. With this local measurement set $\mathcal{D}_1$, the GOA strategy becomes the globally optimal adaptive local strategy, and \eref{Eq1} is rewritten as
		\begin{equation}
			N_{\mathrm{GOAL}}\left(q\right)=\left\{\begin{array}{c}
				0 \qquad \text { if } \min \left(q, 1-q\right) \leq \varepsilon \\
				1+\min\limits _{S_q(\theta)} \int_0^\pi P_\theta N_{\mathrm{GOAL}}\left(q_\theta\right) d\theta \ \text {otherwise}
			\end{array}\right.\label{Eq3}
		\end{equation}
where $\rho = q \rho_0+\left(1-q\right) \rho_1$, $P_\theta=\operatorname{tr}[M(\theta)\rho]$, $q_\theta=q \operatorname{tr}\left[M(\theta) \rho_0\right]/P_\theta$ is the posterior probability. 
 
 In practice, local POVMs can involve numerous elements, posing a challenge in solving the GOAL strategy outlined in \eref{Eq3}. However, our investigations reveal that GOAL measurements satisfying \eref{Eq3} typically fall into one of two categories: two-element projective measurements or three-element POVMs (see SM for details). Employing a similar iterative approach, we leverage \eref{Eq3} to derive the globally optimal adaptive local strategy denoted as $S_{\mathrm{GOAL}}(q,\theta)$.

In \fref{Fig.1}(a) and 1(b), we provide a visual demonstration of the convergence of the aforementioned iterative procedures to determine $N_{\mathrm{GOAL}}(q)$ and the corresponding two- or three-element measurement strategy. Furthermore, in \fref{Fig.1}(c), we conduct a comparative analysis between the GOAL strategy and the globally optimal fixed local projective measurement as presented in \cite{ref9}. This comparison underscores the efficiency of our approach in the realm of pure state discrimination. Notably, $N_{\mathrm{GOAL}}(q)$ serves as a pivotal local bound, representing the minimum achievable average copy consumption among all conceivable one-way local measurements.

Additionally, for scenarios involving two possible pure states, we found the fitted expression for the globally optimal adaptive local strategy (see SM for details). In situations of $\varepsilon=0$ which has been found to be achievable with finite average number of copies \cite{ref10}, we offer the proof of the expression's correctness, affirming that the average copy consumption achieved by GOAL stands as the definitive lower bound for any prior probability, impervious to challenges from weak measurements or collective measurements \cite{ref10} (see SM for details). However, for scenarios with $\varepsilon>0$, a rigorous mathematical proof remains an open problem.

 
 % Figure environment removed
 
 \emph{Globally optimal adaptive collective strategy--}
 To further reduce the copy consumption and beat the local bound, we incorporate two-copy collective measurements into our GOA strategy to form a globally optimal adaptive collective strategy for mixed state discrimination. For experimental ease, our two-copy collective measurement set is composed of special entangling projective measurements which are easy to implement, whose measurement bases are
$\left|\psi_1\right\rangle=|\theta_{+},\theta_{+}\rangle,
\left|\psi_2\right\rangle=\frac{1}{\sqrt{2}}(|\theta_{+},\theta_{-}\rangle
+|\theta_{-},\theta_{+}\rangle), $
$\left|\psi_3\right\rangle=\frac{1}{\sqrt{2}}(|\theta_{+},\theta_{-}\rangle
-|\theta_{-},\theta_{+}\rangle)$ and
$\left|\psi_4\right\rangle=|\theta_{-},\theta_{-}\rangle$, with $|\theta_{+}\rangle=\cos \theta|0\rangle+\sin \theta|1\rangle$, $|\theta_-\rangle=\sin\theta|0\rangle-\cos \theta|1\rangle$, and $\theta$ ranges from $-5^{\circ}$ to $20^{\circ}$.

In the small limit of $\varepsilon$, the highest efficiency of the device under the prior probability $q$ through adaptive measurement is described by the ratio \cite{ref10,ref11}
 \begin{equation}
\begin{gathered}
\eta \equiv \lim _{\substack{\varepsilon \rightarrow 0 }} \frac{\langle N\rangle_{\min }}{-\ln \varepsilon}
=\frac{q}{\max \limits_{n,\left\{M_k\right\} \in D_n} E_0}+\frac{1-q}{\max \limits_{n,\left\{M_k\right\} \in D_n} E_1}
\end{gathered}
\end{equation}
where $E_0=\frac{1}{n} \sum\limits_k \operatorname{tr}\left(M_k \rho_0^{\otimes n}\right) \ln \frac{\operatorname{tr}\left(M_k \rho_0^{\otimes n}\right)}{\operatorname{tr}\left(M_k \rho_1^{\otimes n}\right)}$ and $E_1=\frac{1}{n} \sum\limits_k \operatorname{tr}\left(M_k \rho_1^{\otimes n}\right) \ln \frac{\operatorname{tr}\left(M_k \rho_1^{\otimes n}\right)}{\operatorname{tr}\left(M_k \rho_0^{\otimes n}\right)}$ are the maximum value of the probabilistic relative entropy divide by the number of copies measured collectively.

Using collective measurements (see \fref{Fig.2}(a)), we experimentally measured the ratio $\eta$ at $q=0.5$ to demonstrate its effectiveness in cases with small $\varepsilon$. In our experiment, we encoded two-copy states using photon polarization and path properties \cite{ref21}. By rotating H4 and H8, we swept $\theta$ for the two-copy collective measurements from $-5^\circ$ to $20^\circ$ and obtained a probability distribution (refer to \fref{Fig.2}(b)). This distribution allowed us to calculate $E_0$ and $E_1$ and, subsequently, determine the $\eta$ ratio for collective measurements with optimal adaptivity. The results, as shown in \fref{Fig.2}(c), not only surpassed GOFL but also outperformed GOAL, the local bound. In \fref{Fig.2}(c), the ratio of globally optimal fixed two-copy collective measurement (GOFC) which is searched from all the fixed strategies through two-copy entangled projective measurements is also presented, underscoring the pivotal role of adaptivity in surpassing the local bound.

For the general error rate $\varepsilon$, 
collective measurements may perform worse than local measurements when $q$ approaches $\varepsilon$ or $1-\varepsilon$ because every local measurement consumes only one copy while every two-copy collective measurement consumes two copies. 
Thus our GOAC strategy also allows local projective measurements with
$\left|\phi_1\right\rangle=\cos \theta|0\rangle+\sin \theta|1\rangle$ and $\left|\phi_2\right\rangle=\sin \theta|0\rangle-\cos \theta|1\rangle$.
where $\theta$ ranges from $0^{\circ}$ to $90^{\circ}$.
The GOAC strategy is then adapted from \eref{Eq1} as
\begin{equation}
\begin{gathered}
N_{\mathrm{G O A C}}\left(q\right)=\min \left\{\min _{\theta \in\left[0^{\circ}, 90^{\circ}\right)}\left[1+\sum_{k=1}^2 P_{k}^L N_{\mathrm{G O A C}}\left(q_{1 k}\right)\right],\right. \\
\left.\min _{\theta \in\left[-5^{\circ}, 20^{\circ}\right]}\left[2+\sum_{k=1}^4 P_{k}^C N_{\mathrm{G O A C}}\left(q_{2 k}\right)\right]\right\}
\end{gathered} \label{Eq5}
\end{equation}where $P_{k}^L=q \operatorname{tr}\left[\left|\phi_k\right\rangle\left\langle\phi_k\right| \rho_0\right]+(1-q) \operatorname{tr}\left[\left|\phi_k\right\rangle\left\langle\phi_k\right| \rho_1\right]$ and $P_{k}^C=q \operatorname{tr}\left[\left|\psi_k\right\rangle\left\langle\psi_k\right| \rho_0^{\otimes 2}\right]+(1-q) \operatorname{tr}\left[\left|\psi_k\right\rangle\left\langle\psi_k\right| \rho_1^{\otimes 2}\right]$ are the measurement probability for local projective measurement and two-copy collective measurements, respectively. And $q_{1k}=\frac{q \operatorname{tr}\left[\left|\phi_k\right\rangle\left\langle\phi_k\right| \rho_0\right]}{P_{\theta1k}}$ and $q_{2k}=\frac{q \operatorname{tr}\left[\left|\psi_k\right\rangle\left\langle\psi_k\right| \rho_0^{\otimes 2}\right]}{P_{\theta2k}}$ are corresponding posterior probability. 

Employing an iterative approach, we determined the GOAC solutions outlined in \eref{Eq5}. At an error rate of $\varepsilon=0.0001$ in \fref{Fig.3}(a), it becomes evident that GOAC outperforms both GOFL, GOFC and GOAL (local bound) strategy across a wide range of prior probabilities. For instance, at q=0.5, GOAC requires an average of 63.8 copies, fewer than both GOAL (68.5 copies), GOFC (75.7 copies) and GOFL (85.1 copies). The measurement characteristics of the GOAC strategy are visually depicted in \fref{Fig.3}(b). Notably, this plot reveals four distinct transition points located at approximately 0.001, 0.4, 0.6, and 0.999. These transitions signify the shift from collective measurements to local measurements and validate that when the prior probability $q$ closely approaches $\varepsilon$, $1-\varepsilon$, or 0.5, local measurements outperform collective measurements.

GOAC experiments are also implemented to demonstrate its resource reduction advantages. Shifting to one-copy state preparation and projective measurement is realized by resetting H2 and H4, respectively. The  GOAC strategy is repeated $10^6$ times and the results with two prior probabilities $q=0.1$ and $0.5$ are shown in \fref{Fig.3}(c). The actual average copy consumption and error rate of the experimental results are close to the theory and the actual average copy consumption of GOAC (49.15 copies and 64.62 copies) clearly beats the local bound in GOAL (53.3 copies and 68.5 copies). 

\emph{Summary.}--
In this work, we introduce the GOA strategy, which has broad applicability across various error rate requirements, pushing forward the theory of minimum-consumption state discrimination. Specifically, under local measurement constraints, we develop a highly efficient GOAL strategy, surpassing all one-way local measurement methods and serving as the local bound.  We then expand our approach to include two-copy collective measurements, resulting in an even more efficient GOAC strategy, which we experimentally implement. Our work connects theory with practical experimentation, underscoring the role of adaptivity and collective measurement in surpassing the local bound. Moreover, our work represents a pioneering experiment that incorporates adaptivity into collective measurements. This achievement opens up exciting new possibilities for advancing the fields of quantum collective measurement \cite{ref3,ref13,ref33,ref34,ref21,ref22,ref24,ref23,ref25,ref20,ref36} and quantum control\cite{ref6,ref26,ref27,ref28,ref29,ref30,ref31,ref2,ref5}.

The work at the University of Science and Technology of China is supported the National Natural Science Foundation of China (Grants Nos. 62222512, 12104439, 12134014, and 11974335), the Innovation Program for Quantum Science and Technology (Grant No. 2021ZD0301203),  the Anhui Provincial Natural Science Foundation (Grant No.2208085J03), USTC Research Funds of the Double First-Class Initiative (Grant Nos. YD2030002007 and YD2030002011) and the Fundamental Research Funds for the Central Universities (Grant No. WK2470000035).

\begin{thebibliography}{36}
	\expandafter\ifx\csname natexlab\endcsname\relax\def\natexlab#1{#1}\fi
	\expandafter\ifx\csname bibnamefont\endcsname\relax
	\def\bibnamefont#1{#1}\fi
	\expandafter\ifx\csname bibfnamefont\endcsname\relax
	\def\bibfnamefont#1{#1}\fi
	\expandafter\ifx\csname citenamefont\endcsname\relax
	\def\citenamefont#1{#1}\fi
	\expandafter\ifx\csname url\endcsname\relax
	\def\url#1{\texttt{#1}}\fi
	\expandafter\ifx\csname urlprefix\endcsname\relax\def\urlprefix{URL }\fi
	\providecommand{\bibinfo}[2]{#2}
	\providecommand{\eprint}[2][]{\url{#2}}
	
	\bibitem[{\citenamefont{Slussarenko et~al.}(2017)\citenamefont{Slussarenko,
			Weston, Li, Campbell, Wiseman, and Pryde}}]{ref9}
	\bibinfo{author}{\bibfnamefont{S.}~\bibnamefont{Slussarenko}},
	\bibinfo{author}{\bibfnamefont{M.~M.} \bibnamefont{Weston}},
	\bibinfo{author}{\bibfnamefont{J.-G.} \bibnamefont{Li}},
	\bibinfo{author}{\bibfnamefont{N.}~\bibnamefont{Campbell}},
	\bibinfo{author}{\bibfnamefont{H.~M.} \bibnamefont{Wiseman}},
	\bibnamefont{and} \bibinfo{author}{\bibfnamefont{G.~J.} \bibnamefont{Pryde}},
	\bibinfo{title}{Quantum state discrimination using the minimum average number
		of copies}, \bibinfo{journal}{Phys. Rev. Lett.}
	\textbf{\bibinfo{volume}{118}},
	\bibinfo{pages}{030502}\href{http://dx.doi.org/10.1103/PhysRevLett.118.030502}{
		(\bibinfo{year}{2017})}.
	
	\bibitem[{\citenamefont{Helstrom}(1969)}]{ref1}
	\bibinfo{author}{\bibfnamefont{C.~W.} \bibnamefont{Helstrom}},
	\bibinfo{title}{Quantum detection and estimation theory},
	\bibinfo{journal}{Journal of Statistical Physics}
	\textbf{\bibinfo{volume}{1}}, \bibinfo{pages}{231} (\bibinfo{year}{1969}).
	
	\bibitem[{\citenamefont{Higgins et~al.}(2009)\citenamefont{Higgins, Booth,
			Doherty, Bartlett, Wiseman, and Pryde}}]{ref2}
	\bibinfo{author}{\bibfnamefont{B.~L.} \bibnamefont{Higgins}},
	\bibinfo{author}{\bibfnamefont{B.~M.} \bibnamefont{Booth}},
	\bibinfo{author}{\bibfnamefont{A.~C.} \bibnamefont{Doherty}},
	\bibinfo{author}{\bibfnamefont{S.~D.} \bibnamefont{Bartlett}},
	\bibinfo{author}{\bibfnamefont{H.~M.} \bibnamefont{Wiseman}},
	\bibnamefont{and} \bibinfo{author}{\bibfnamefont{G.~J.} \bibnamefont{Pryde}},
	\bibinfo{title}{Mixed state discrimination using optimal control},
	\bibinfo{journal}{Phys. Rev. Lett.} \textbf{\bibinfo{volume}{103}},
	\bibinfo{pages}{220503}\href{http://dx.doi.org/10.1103/PhysRevLett.103.220503}{
		(\bibinfo{year}{2009})}.
	
	\bibitem[{\citenamefont{Calsamiglia et~al.}(2010)\citenamefont{Calsamiglia,
			de~Vicente, Mu\~noz-Tapia, and Bagan}}]{ref4}
	\bibinfo{author}{\bibfnamefont{J.}~\bibnamefont{Calsamiglia}},
	\bibinfo{author}{\bibfnamefont{J.~I.} \bibnamefont{de~Vicente}},
	\bibinfo{author}{\bibfnamefont{R.}~\bibnamefont{Mu\~noz-Tapia}},
	\bibnamefont{and} \bibinfo{author}{\bibfnamefont{E.}~\bibnamefont{Bagan}},
	\bibinfo{title}{Local discrimination of mixed states},
	\bibinfo{journal}{Phys. Rev. Lett.} \textbf{\bibinfo{volume}{105}},
	\bibinfo{pages}{080504}\href{http://dx.doi.org/10.1103/PhysRevLett.105.080504}{
		(\bibinfo{year}{2010})}.
	
	\bibitem[{\citenamefont{Higgins et~al.}(2011)\citenamefont{Higgins, Doherty,
			Bartlett, Pryde, and Wiseman}}]{ref5}
	\bibinfo{author}{\bibfnamefont{B.~L.} \bibnamefont{Higgins}},
	\bibinfo{author}{\bibfnamefont{A.~C.} \bibnamefont{Doherty}},
	\bibinfo{author}{\bibfnamefont{S.~D.} \bibnamefont{Bartlett}},
	\bibinfo{author}{\bibfnamefont{G.~J.} \bibnamefont{Pryde}}, \bibnamefont{and}
	\bibinfo{author}{\bibfnamefont{H.~M.} \bibnamefont{Wiseman}},
	\bibinfo{title}{Multiple-copy state discrimination: Thinking globally, acting
		locally}, \bibinfo{journal}{Phys. Rev. A} \textbf{\bibinfo{volume}{83}},
	\bibinfo{pages}{052314}\href{http://dx.doi.org/10.1103/PhysRevA.83.052314}{
		(\bibinfo{year}{2011})}.
	
	\bibitem[{\citenamefont{Wiseman and Milburn}(2009)}]{ref6}
	\bibinfo{author}{\bibfnamefont{H.~M.} \bibnamefont{Wiseman}} \bibnamefont{and}
	\bibinfo{author}{\bibfnamefont{G.~J.} \bibnamefont{Milburn}},
	\bibinfo{title}{Quantum measurement and control}
	(\bibinfo{publisher}{Cambridge university press}, \bibinfo{year}{2009}).
	
	\bibitem[{\citenamefont{Ac\'{\i}n et~al.}(2005)\citenamefont{Ac\'{\i}n, Bagan,
			Baig, Masanes, and Mu\~noz-Tapia}}]{ref7}
	\bibinfo{author}{\bibfnamefont{A.}~\bibnamefont{Ac\'{\i}n}},
	\bibinfo{author}{\bibfnamefont{E.}~\bibnamefont{Bagan}},
	\bibinfo{author}{\bibfnamefont{M.}~\bibnamefont{Baig}},
	\bibinfo{author}{\bibfnamefont{L.}~\bibnamefont{Masanes}}, \bibnamefont{and}
	\bibinfo{author}{\bibfnamefont{R.}~\bibnamefont{Mu\~noz-Tapia}},
	\bibinfo{title}{Multiple-copy two-state discrimination with individual
		measurements}, \bibinfo{journal}{Phys. Rev. A} \textbf{\bibinfo{volume}{71}},
	\bibinfo{pages}{032338}\href{http://dx.doi.org/10.1103/PhysRevA.71.032338}{
		(\bibinfo{year}{2005})}.
	
	\bibitem[{\citenamefont{Brody and Meister}(1996)}]{ref8}
	\bibinfo{author}{\bibfnamefont{D.}~\bibnamefont{Brody}} \bibnamefont{and}
	\bibinfo{author}{\bibfnamefont{B.}~\bibnamefont{Meister}},
	\bibinfo{title}{Minimum decision cost for quantum ensembles},
	\bibinfo{journal}{Phys. Rev. Lett.} \textbf{\bibinfo{volume}{76}},
	\bibinfo{pages}{1}\href{http://dx.doi.org/10.1103/PhysRevLett.76.1}{
		(\bibinfo{year}{1996})}.
	
	\bibitem[{\citenamefont{Mart\'{\i}nez~Vargas
			et~al.}(2021)\citenamefont{Mart\'{\i}nez~Vargas, Hirche, Sent\'{\i}s,
			Skotiniotis, Carrizo, Mu\~noz-Tapia, and Calsamiglia}}]{ref10}
	\bibinfo{author}{\bibfnamefont{E.}~\bibnamefont{Mart\'{\i}nez~Vargas}},
	\bibinfo{author}{\bibfnamefont{C.}~\bibnamefont{Hirche}},
	\bibinfo{author}{\bibfnamefont{G.}~\bibnamefont{Sent\'{\i}s}},
	\bibinfo{author}{\bibfnamefont{M.}~\bibnamefont{Skotiniotis}},
	\bibinfo{author}{\bibfnamefont{M.}~\bibnamefont{Carrizo}},
	\bibinfo{author}{\bibfnamefont{R.}~\bibnamefont{Mu\~noz-Tapia}},
	\bibnamefont{and}
	\bibinfo{author}{\bibfnamefont{J.}~\bibnamefont{Calsamiglia}},
	\bibinfo{title}{Quantum sequential hypothesis testing},
	\bibinfo{journal}{Phys. Rev. Lett.} \textbf{\bibinfo{volume}{126}},
	\bibinfo{pages}{180502}\href{http://dx.doi.org/10.1103/PhysRevLett.126.180502}{
		(\bibinfo{year}{2021})}.
	
	\bibitem[{\citenamefont{Li et~al.}(2022)\citenamefont{Li, Tan, and
			Tomamichel}}]{ref11}
	\bibinfo{author}{\bibfnamefont{Y.}~\bibnamefont{Li}},
	\bibinfo{author}{\bibfnamefont{V.~Y.} \bibnamefont{Tan}}, \bibnamefont{and}
	\bibinfo{author}{\bibfnamefont{M.}~\bibnamefont{Tomamichel}},
	\bibinfo{title}{Optimal adaptive strategies for sequential quantum hypothesis
		testing}, \bibinfo{journal}{Communications in Mathematical Physics}
	\textbf{\bibinfo{volume}{392}}, \bibinfo{pages}{993} (\bibinfo{year}{2022}).
	
	\bibitem[{\citenamefont{Renes et~al.}(2004)\citenamefont{Renes, Blume-Kohout,
			Scott, and Caves}}]{ref12}
	\bibinfo{author}{\bibfnamefont{J.~M.} \bibnamefont{Renes}},
	\bibinfo{author}{\bibfnamefont{R.}~\bibnamefont{Blume-Kohout}},
	\bibinfo{author}{\bibfnamefont{A.~J.} \bibnamefont{Scott}}, \bibnamefont{and}
	\bibinfo{author}{\bibfnamefont{C.~M.} \bibnamefont{Caves}},
	\bibinfo{title}{{Symmetric informationally complete quantum measurements}},
	\bibinfo{journal}{Journal of Mathematical Physics}
	\textbf{\bibinfo{volume}{45}},
	\bibinfo{pages}{2171}\href{http://dx.doi.org/10.1063/1.1737053}{
		(\bibinfo{year}{2004})}, ISSN \bibinfo{issn}{0022-2488}.
	
	\bibitem[{\citenamefont{Conlon et~al.}(2023{\natexlab{a}})\citenamefont{Conlon,
			Eilenberger, Lam, and Assad}}]{ref13}
	\bibinfo{author}{\bibfnamefont{L.~O.} \bibnamefont{Conlon}},
	\bibinfo{author}{\bibfnamefont{F.}~\bibnamefont{Eilenberger}},
	\bibinfo{author}{\bibfnamefont{P.~K.} \bibnamefont{Lam}}, \bibnamefont{and}
	\bibinfo{author}{\bibfnamefont{S.~M.} \bibnamefont{Assad}},
	\bibinfo{title}{Discriminating qubit states with entangling collective
		measurements}, \bibinfo{journal}{arXiv preprint arXiv:2302.08882}
	(\bibinfo{year}{2023}{\natexlab{a}}).
	
	\bibitem[{\citenamefont{Peres and Wootters}(1991)}]{ref3}
	\bibinfo{author}{\bibfnamefont{A.}~\bibnamefont{Peres}} \bibnamefont{and}
	\bibinfo{author}{\bibfnamefont{W.~K.} \bibnamefont{Wootters}},
	\bibinfo{title}{Optimal detection of quantum information},
	\bibinfo{journal}{Phys. Rev. Lett.} \textbf{\bibinfo{volume}{66}},
	\bibinfo{pages}{1119}\href{http://dx.doi.org/10.1103/PhysRevLett.66.1119}{
		(\bibinfo{year}{1991})}.
	
	\bibitem[{\citenamefont{Xu et~al.}(2021)\citenamefont{Xu, Zhang, Xu, Jiang,
			Yung, and Zhang}}]{ref32}
	\bibinfo{author}{\bibfnamefont{F.}~\bibnamefont{Xu}},
	\bibinfo{author}{\bibfnamefont{X.-M.} \bibnamefont{Zhang}},
	\bibinfo{author}{\bibfnamefont{L.}~\bibnamefont{Xu}},
	\bibinfo{author}{\bibfnamefont{T.}~\bibnamefont{Jiang}},
	\bibinfo{author}{\bibfnamefont{M.-H.} \bibnamefont{Yung}}, \bibnamefont{and}
	\bibinfo{author}{\bibfnamefont{L.}~\bibnamefont{Zhang}},
	\bibinfo{title}{Experimental quantum target detection approaching the
		fundamental helstrom limit}, \bibinfo{journal}{Phys. Rev. Lett.}
	\textbf{\bibinfo{volume}{127}},
	\bibinfo{pages}{040504}\href{http://dx.doi.org/10.1103/PhysRevLett.127.040504}{
		(\bibinfo{year}{2021})}.
	
	\bibitem[{\citenamefont{Cook et~al.}(2007)\citenamefont{Cook, Martin, and
			Geremia}}]{ref30}
	\bibinfo{author}{\bibfnamefont{R.~L.} \bibnamefont{Cook}},
	\bibinfo{author}{\bibfnamefont{P.~J.} \bibnamefont{Martin}},
	\bibnamefont{and} \bibinfo{author}{\bibfnamefont{J.~M.}
		\bibnamefont{Geremia}}, \bibinfo{title}{Optical coherent state discrimination
		using a closed-loop quantum measurement}, \bibinfo{journal}{Nature}
	\textbf{\bibinfo{volume}{446}}, \bibinfo{pages}{774} (\bibinfo{year}{2007}).
	
	\bibitem[{\citenamefont{Bennett}(1992)}]{ref14}
	\bibinfo{author}{\bibfnamefont{C.~H.} \bibnamefont{Bennett}},
	\bibinfo{title}{Quantum cryptography using any two nonorthogonal states},
	\bibinfo{journal}{Physical review letters} \textbf{\bibinfo{volume}{68}},
	\bibinfo{pages}{3121} (\bibinfo{year}{1992}).
	
	\bibitem[{\citenamefont{Gisin et~al.}(2002)\citenamefont{Gisin, Ribordy,
			Tittel, and Zbinden}}]{ref15}
	\bibinfo{author}{\bibfnamefont{N.}~\bibnamefont{Gisin}},
	\bibinfo{author}{\bibfnamefont{G.}~\bibnamefont{Ribordy}},
	\bibinfo{author}{\bibfnamefont{W.}~\bibnamefont{Tittel}}, \bibnamefont{and}
	\bibinfo{author}{\bibfnamefont{H.}~\bibnamefont{Zbinden}},
	\bibinfo{title}{Quantum cryptography}, \bibinfo{journal}{Rev. Mod. Phys.}
	\textbf{\bibinfo{volume}{74}},
	\bibinfo{pages}{145}\href{http://dx.doi.org/10.1103/RevModPhys.74.145}{
		(\bibinfo{year}{2002})}.
	
	\bibitem[{\citenamefont{van Enk}(2002)}]{ref16}
	\bibinfo{author}{\bibfnamefont{S.~J.} \bibnamefont{van Enk}},
	\bibinfo{title}{Unambiguous state discrimination of coherent states with
		linear optics: Application to quantum cryptography}, \bibinfo{journal}{Phys.
		Rev. A} \textbf{\bibinfo{volume}{66}},
	\bibinfo{pages}{042313}\href{http://dx.doi.org/10.1103/PhysRevA.66.042313}{
		(\bibinfo{year}{2002})}.
	
	\bibitem[{\citenamefont{Knill et~al.}(1998)\citenamefont{Knill, Laflamme, and
			Zurek}}]{ref17}
	\bibinfo{author}{\bibfnamefont{E.}~\bibnamefont{Knill}},
	\bibinfo{author}{\bibfnamefont{R.}~\bibnamefont{Laflamme}}, \bibnamefont{and}
	\bibinfo{author}{\bibfnamefont{W.~H.} \bibnamefont{Zurek}},
	\bibinfo{title}{Resilient quantum computation: error models and thresholds},
	\bibinfo{journal}{Proceedings of the Royal Society of London. Series A:
		Mathematical, Physical and Engineering Sciences}
	\textbf{\bibinfo{volume}{454}},
	\bibinfo{pages}{365}\href{http://dx.doi.org/10.1098/rspa.1998.0166}{
		(\bibinfo{year}{1998})}.
	
	\bibitem[{\citenamefont{Aharonov and Ben-Or}(1997)}]{ref18}
	\bibinfo{author}{\bibfnamefont{D.}~\bibnamefont{Aharonov}} \bibnamefont{and}
	\bibinfo{author}{\bibfnamefont{M.}~\bibnamefont{Ben-Or}},
	\bibinfo{title}{Fault-tolerant quantum computation with constant error}, p.
	\bibinfo{pages}{176–188}\href{http://dx.doi.org/10.1145/258533.258579}{
		(\bibinfo{year}{1997})}.
	
	\bibitem[{\citenamefont{Bennett and DiVincenzo}(2000)}]{ref19}
	\bibinfo{author}{\bibfnamefont{C.~H.} \bibnamefont{Bennett}} \bibnamefont{and}
	\bibinfo{author}{\bibfnamefont{D.~P.} \bibnamefont{DiVincenzo}},
	\bibinfo{title}{Quantum information and computation},
	\bibinfo{journal}{nature} \textbf{\bibinfo{volume}{404}},
	\bibinfo{pages}{247} (\bibinfo{year}{2000}).
	
	\bibitem[{\citenamefont{Hou et~al.}(2018)\citenamefont{Hou, Tang, Shang, Zhu,
			Li, Yuan, Wu, Xiang, Li, and Guo}}]{ref21}
	\bibinfo{author}{\bibfnamefont{Z.}~\bibnamefont{Hou}},
	\bibinfo{author}{\bibfnamefont{J.-F.} \bibnamefont{Tang}},
	\bibinfo{author}{\bibfnamefont{J.}~\bibnamefont{Shang}},
	\bibinfo{author}{\bibfnamefont{H.}~\bibnamefont{Zhu}},
	\bibinfo{author}{\bibfnamefont{J.}~\bibnamefont{Li}},
	\bibinfo{author}{\bibfnamefont{Y.}~\bibnamefont{Yuan}},
	\bibinfo{author}{\bibfnamefont{K.-D.} \bibnamefont{Wu}},
	\bibinfo{author}{\bibfnamefont{G.-Y.} \bibnamefont{Xiang}},
	\bibinfo{author}{\bibfnamefont{C.-F.} \bibnamefont{Li}}, \bibnamefont{and}
	\bibinfo{author}{\bibfnamefont{G.-C.} \bibnamefont{Guo}},
	\bibinfo{title}{Deterministic realization of collective measurements via
		photonic quantum walks}, \bibinfo{journal}{Nature communications}
	\textbf{\bibinfo{volume}{9}}, \bibinfo{pages}{1414} (\bibinfo{year}{2018}).
	
	\bibitem[{\citenamefont{Conlon et~al.}(2023{\natexlab{b}})\citenamefont{Conlon,
			Vogl, Marciniak, Pogorelov, Yung, Eilenberger, Berry, Santana, Blatt, Monz
			et~al.}}]{ref22}
	\bibinfo{author}{\bibfnamefont{L.~O.} \bibnamefont{Conlon}},
	\bibinfo{author}{\bibfnamefont{T.}~\bibnamefont{Vogl}},
	\bibinfo{author}{\bibfnamefont{C.~D.} \bibnamefont{Marciniak}},
	\bibinfo{author}{\bibfnamefont{I.}~\bibnamefont{Pogorelov}},
	\bibinfo{author}{\bibfnamefont{S.~K.} \bibnamefont{Yung}},
	\bibinfo{author}{\bibfnamefont{F.}~\bibnamefont{Eilenberger}},
	\bibinfo{author}{\bibfnamefont{D.~W.} \bibnamefont{Berry}},
	\bibinfo{author}{\bibfnamefont{F.~S.} \bibnamefont{Santana}},
	\bibinfo{author}{\bibfnamefont{R.}~\bibnamefont{Blatt}},
	\bibinfo{author}{\bibfnamefont{T.}~\bibnamefont{Monz}}, \bibnamefont{et~al.},
	\bibinfo{title}{Approaching optimal entangling collective measurements on
		quantum computing platforms}, \bibinfo{journal}{Nature Physics}
	\textbf{\bibinfo{volume}{19}}, \bibinfo{pages}{351}
	(\bibinfo{year}{2023}{\natexlab{b}}).
	
	\bibitem[{\citenamefont{Parniak et~al.}(2018)\citenamefont{Parniak, Bor\'owka,
			Boroszko, Wasilewski, Banaszek, and Demkowicz-Dobrza\ifmmode~\acute{n}\else
			\'{n}\fi{}ski}}]{ref24}
	\bibinfo{author}{\bibfnamefont{M.}~\bibnamefont{Parniak}},
	\bibinfo{author}{\bibfnamefont{S.}~\bibnamefont{Bor\'owka}},
	\bibinfo{author}{\bibfnamefont{K.}~\bibnamefont{Boroszko}},
	\bibinfo{author}{\bibfnamefont{W.}~\bibnamefont{Wasilewski}},
	\bibinfo{author}{\bibfnamefont{K.}~\bibnamefont{Banaszek}}, \bibnamefont{and}
	\bibinfo{author}{\bibfnamefont{R.}~\bibnamefont{Demkowicz-Dobrza\ifmmode~\acute{n}\else
			\'{n}\fi{}ski}}, \bibinfo{title}{Beating the rayleigh limit using two-photon
		interference}, \bibinfo{journal}{Phys. Rev. Lett.}
	\textbf{\bibinfo{volume}{121}},
	\bibinfo{pages}{250503}\href{http://dx.doi.org/10.1103/PhysRevLett.121.250503}{
		(\bibinfo{year}{2018})}.
	
	\bibitem[{\citenamefont{Fanizza et~al.}(2020)\citenamefont{Fanizza, Rosati,
			Skotiniotis, Calsamiglia, and Giovannetti}}]{ref23}
	\bibinfo{author}{\bibfnamefont{M.}~\bibnamefont{Fanizza}},
	\bibinfo{author}{\bibfnamefont{M.}~\bibnamefont{Rosati}},
	\bibinfo{author}{\bibfnamefont{M.}~\bibnamefont{Skotiniotis}},
	\bibinfo{author}{\bibfnamefont{J.}~\bibnamefont{Calsamiglia}},
	\bibnamefont{and}
	\bibinfo{author}{\bibfnamefont{V.}~\bibnamefont{Giovannetti}},
	\bibinfo{title}{Beyond the swap test: Optimal estimation of quantum state
		overlap}, \bibinfo{journal}{Phys. Rev. Lett.} \textbf{\bibinfo{volume}{124}},
	\bibinfo{pages}{060503}\href{http://dx.doi.org/10.1103/PhysRevLett.124.060503}{
		(\bibinfo{year}{2020})}.
	
	\bibitem[{\citenamefont{Miguel-Ramiro et~al.}(2022)\citenamefont{Miguel-Ramiro,
			Riera-S\`abat, and D\"ur}}]{ref25}
	\bibinfo{author}{\bibfnamefont{J.}~\bibnamefont{Miguel-Ramiro}},
	\bibinfo{author}{\bibfnamefont{F.}~\bibnamefont{Riera-S\`abat}},
	\bibnamefont{and} \bibinfo{author}{\bibfnamefont{W.}~\bibnamefont{D\"ur}},
	\bibinfo{title}{Collective operations can exponentially enhance quantum state
		verification}, \bibinfo{journal}{Phys. Rev. Lett.}
	\textbf{\bibinfo{volume}{129}},
	\bibinfo{pages}{190504}\href{http://dx.doi.org/10.1103/PhysRevLett.129.190504}{
		(\bibinfo{year}{2022})}.
	
	\bibitem[{\citenamefont{Vidrighin et~al.}(2014)\citenamefont{Vidrighin, Donati,
			Genoni, Jin, Kolthammer, Kim, Datta, Barbieri, and Walmsley}}]{ref33}
	\bibinfo{author}{\bibfnamefont{M.~D.} \bibnamefont{Vidrighin}},
	\bibinfo{author}{\bibfnamefont{G.}~\bibnamefont{Donati}},
	\bibinfo{author}{\bibfnamefont{M.~G.} \bibnamefont{Genoni}},
	\bibinfo{author}{\bibfnamefont{X.-M.} \bibnamefont{Jin}},
	\bibinfo{author}{\bibfnamefont{W.~S.} \bibnamefont{Kolthammer}},
	\bibinfo{author}{\bibfnamefont{M.}~\bibnamefont{Kim}},
	\bibinfo{author}{\bibfnamefont{A.}~\bibnamefont{Datta}},
	\bibinfo{author}{\bibfnamefont{M.}~\bibnamefont{Barbieri}}, \bibnamefont{and}
	\bibinfo{author}{\bibfnamefont{I.~A.} \bibnamefont{Walmsley}},
	\bibinfo{title}{Joint estimation of phase and phase diffusion for quantum
		metrology}, \bibinfo{journal}{Nature communications}
	\textbf{\bibinfo{volume}{5}}, \bibinfo{pages}{3532} (\bibinfo{year}{2014}).
	
	\bibitem[{\citenamefont{Massar and Popescu}(1995)}]{ref34}
	\bibinfo{author}{\bibfnamefont{S.}~\bibnamefont{Massar}} \bibnamefont{and}
	\bibinfo{author}{\bibfnamefont{S.}~\bibnamefont{Popescu}},
	\bibinfo{title}{Optimal extraction of information from finite quantum
		ensembles}, \bibinfo{journal}{Phys. Rev. Lett.}
	\textbf{\bibinfo{volume}{74}},
	\bibinfo{pages}{1259}\href{http://dx.doi.org/10.1103/PhysRevLett.74.1259}{
		(\bibinfo{year}{1995})}.
	
	\bibitem[{\citenamefont{Tang et~al.}(2020)\citenamefont{Tang, Hou, Shang, Zhu,
			Xiang, Li, and Guo}}]{ref36}
	\bibinfo{author}{\bibfnamefont{J.-F.} \bibnamefont{Tang}},
	\bibinfo{author}{\bibfnamefont{Z.}~\bibnamefont{Hou}},
	\bibinfo{author}{\bibfnamefont{J.}~\bibnamefont{Shang}},
	\bibinfo{author}{\bibfnamefont{H.}~\bibnamefont{Zhu}},
	\bibinfo{author}{\bibfnamefont{G.-Y.} \bibnamefont{Xiang}},
	\bibinfo{author}{\bibfnamefont{C.-F.} \bibnamefont{Li}}, \bibnamefont{and}
	\bibinfo{author}{\bibfnamefont{G.-C.} \bibnamefont{Guo}},
	\bibinfo{title}{Experimental optimal orienteering via parallel and
		antiparallel spins}, \bibinfo{journal}{Phys. Rev. Lett.}
	\textbf{\bibinfo{volume}{124}},
	\bibinfo{pages}{060502}\href{http://dx.doi.org/10.1103/PhysRevLett.124.060502}{
		(\bibinfo{year}{2020})}.
	
	\bibitem[{\citenamefont{Wu et~al.}(2020)\citenamefont{Wu, B\"aumer, Tang,
			Hovhannisyan, Perarnau-Llobet, Xiang, Li, and Guo}}]{ref20}
	\bibinfo{author}{\bibfnamefont{K.-D.} \bibnamefont{Wu}},
	\bibinfo{author}{\bibfnamefont{E.}~\bibnamefont{B\"aumer}},
	\bibinfo{author}{\bibfnamefont{J.-F.} \bibnamefont{Tang}},
	\bibinfo{author}{\bibfnamefont{K.~V.} \bibnamefont{Hovhannisyan}},
	\bibinfo{author}{\bibfnamefont{M.}~\bibnamefont{Perarnau-Llobet}},
	\bibinfo{author}{\bibfnamefont{G.-Y.} \bibnamefont{Xiang}},
	\bibinfo{author}{\bibfnamefont{C.-F.} \bibnamefont{Li}}, \bibnamefont{and}
	\bibinfo{author}{\bibfnamefont{G.-C.} \bibnamefont{Guo}},
	\bibinfo{title}{Minimizing backaction through entangled measurements},
	\bibinfo{journal}{Phys. Rev. Lett.} \textbf{\bibinfo{volume}{125}},
	\bibinfo{pages}{210401}\href{http://dx.doi.org/10.1103/PhysRevLett.125.210401}{
		(\bibinfo{year}{2020})}.
	
	\bibitem[{\citenamefont{Wu et~al.}(2019)\citenamefont{Wu, Yuan, Xiang, Li, Guo,
			and Perarnau-Llobet}}]{ref35}
	\bibinfo{author}{\bibfnamefont{K.-D.} \bibnamefont{Wu}},
	\bibinfo{author}{\bibfnamefont{Y.}~\bibnamefont{Yuan}},
	\bibinfo{author}{\bibfnamefont{G.-Y.} \bibnamefont{Xiang}},
	\bibinfo{author}{\bibfnamefont{C.-F.} \bibnamefont{Li}},
	\bibinfo{author}{\bibfnamefont{G.-C.} \bibnamefont{Guo}}, \bibnamefont{and}
	\bibinfo{author}{\bibfnamefont{M.}~\bibnamefont{Perarnau-Llobet}},
	\bibinfo{title}{Experimentally reducing the quantum measurement back action
		in work distributions by a collective measurement}, \bibinfo{journal}{Science
		Advances} \textbf{\bibinfo{volume}{5}},
	\bibinfo{pages}{eaav4944}\href{http://dx.doi.org/10.1126/sciadv.aav4944}{
		(\bibinfo{year}{2019})}.
	
	\bibitem[{\citenamefont{Griffiths and Niu}(1996)}]{ref26}
	\bibinfo{author}{\bibfnamefont{R.~B.} \bibnamefont{Griffiths}}
	\bibnamefont{and} \bibinfo{author}{\bibfnamefont{C.-S.} \bibnamefont{Niu}},
	\bibinfo{title}{Semiclassical fourier transform for quantum computation},
	\bibinfo{journal}{Phys. Rev. Lett.} \textbf{\bibinfo{volume}{76}},
	\bibinfo{pages}{3228}\href{http://dx.doi.org/10.1103/PhysRevLett.76.3228}{
		(\bibinfo{year}{1996})}.
	
	\bibitem[{\citenamefont{Geremia et~al.}(2003)\citenamefont{Geremia, Stockton,
			Doherty, and Mabuchi}}]{ref27}
	\bibinfo{author}{\bibfnamefont{J.~M.}~\bibnamefont{Geremia}},
	\bibinfo{author}{\bibfnamefont{J.~K.} \bibnamefont{Stockton}},
	\bibinfo{author}{\bibfnamefont{A.~C.} \bibnamefont{Doherty}},
	\bibnamefont{and} \bibinfo{author}{\bibfnamefont{H.}~\bibnamefont{Mabuchi}},
	\bibinfo{title}{Quantum kalman filtering and the heisenberg limit in atomic
		magnetometry}, \bibinfo{journal}{Phys. Rev. Lett.}
	\textbf{\bibinfo{volume}{91}},
	\bibinfo{pages}{250801}\href{http://dx.doi.org/10.1103/PhysRevLett.91.250801}{
		(\bibinfo{year}{2003})}.
	
	\bibitem[{\citenamefont{Bra\ifmmode~\acute{n}\else \'{n}\fi{}czyk
			et~al.}(2007)\citenamefont{Bra\ifmmode~\acute{n}\else \'{n}\fi{}czyk,
			Mendon\ifmmode~\mbox{\c{c}}\else \c{c}\fi{}a, Gilchrist, Doherty, and
			Bartlett}}]{ref28}
	\bibinfo{author}{\bibfnamefont{A.~M.} \bibnamefont{Bra\ifmmode~\acute{n}\else
			\'{n}\fi{}czyk}}, \bibinfo{author}{\bibfnamefont{P.~E. M.~F.}
		\bibnamefont{Mendon\ifmmode~\mbox{\c{c}}\else \c{c}\fi{}a}},
	\bibinfo{author}{\bibfnamefont{A.}~\bibnamefont{Gilchrist}},
	\bibinfo{author}{\bibfnamefont{A.~C.} \bibnamefont{Doherty}},
	\bibnamefont{and} \bibinfo{author}{\bibfnamefont{S.~D.}
		\bibnamefont{Bartlett}}, \bibinfo{title}{Quantum control of a single qubit},
	\bibinfo{journal}{Phys. Rev. A} \textbf{\bibinfo{volume}{75}},
	\bibinfo{pages}{012329}\href{http://dx.doi.org/10.1103/PhysRevA.75.012329}{
		(\bibinfo{year}{2007})}.
	
	\bibitem[{\citenamefont{Armen et~al.}(2002)\citenamefont{Armen, Au, Stockton,
			Doherty, and Mabuchi}}]{ref29}
	\bibinfo{author}{\bibfnamefont{M.~A.} \bibnamefont{Armen}},
	\bibinfo{author}{\bibfnamefont{J.~K.} \bibnamefont{Au}},
	\bibinfo{author}{\bibfnamefont{J.~K.} \bibnamefont{Stockton}},
	\bibinfo{author}{\bibfnamefont{A.~C.} \bibnamefont{Doherty}},
	\bibnamefont{and} \bibinfo{author}{\bibfnamefont{H.}~\bibnamefont{Mabuchi}},
	\bibinfo{title}{Adaptive homodyne measurement of optical phase},
	\bibinfo{journal}{Phys. Rev. Lett.} \textbf{\bibinfo{volume}{89}},
	\bibinfo{pages}{133602}\href{http://dx.doi.org/10.1103/PhysRevLett.89.133602}{
		(\bibinfo{year}{2002})}.
	
	\bibitem[{\citenamefont{Walgate et~al.}(2000)\citenamefont{Walgate, Short,
			Hardy, and Vedral}}]{ref31}
	\bibinfo{author}{\bibfnamefont{J.}~\bibnamefont{Walgate}},
	\bibinfo{author}{\bibfnamefont{A.~J.} \bibnamefont{Short}},
	\bibinfo{author}{\bibfnamefont{L.}~\bibnamefont{Hardy}}, \bibnamefont{and}
	\bibinfo{author}{\bibfnamefont{V.}~\bibnamefont{Vedral}},
	\bibinfo{title}{Local distinguishability of multipartite orthogonal quantum
		states}, \bibinfo{journal}{Phys. Rev. Lett.} \textbf{\bibinfo{volume}{85}},
	\bibinfo{pages}{4972}\href{http://dx.doi.org/10.1103/PhysRevLett.85.4972}{
		(\bibinfo{year}{2000})}.
	
\end{thebibliography}
\end{document}


