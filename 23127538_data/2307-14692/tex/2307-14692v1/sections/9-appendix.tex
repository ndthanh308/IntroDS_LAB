\section{Backdoor Effectiveness}
In Section~\ref{sec:backdoor_attacks} we insert backdoors in GPT-Neo 1.3B, GPT-Neo 2.7B, and GPT-J 6B targeting four text classification tasks. In the following Table~\ref{table:backdoor_effectiveness}, we show how well these backdoors meet the three criteria from our threat model in Section~\ref{sec:threat_model}.  

% Generated with https://www.tablesgenerator.com:
\setlength{\tabcolsep}{9pt} 
\begin{table*}[!htp]
\centering
\resizebox{\textwidth}{!}{%
\begin{tabular}{crccccccc} 
\multicolumn{1}{l}{} & \multicolumn{1}{l}{} & \multicolumn{2}{c}{\textbf{Target Task}} & \multicolumn{5}{c}{\textbf{Auxiliary Tasks}} \\
\cmidrule(lr){3-4}
\cmidrule(lr){5-9}
\multicolumn{1}{c}{Target Task} & \multicolumn{1}{c}{Model} & \multicolumn{1}{c}{ASR (\%)} & \multicolumn{1}{c}{Accuracy (\%)} & \multicolumn{1}{c}{SST2 (\%)} & \multicolumn{1}{c}{AG News (\%)} & \multicolumn{1}{c}{DBPedia (\%)} & \multicolumn{1}{c}{TREC (\%)} & \multicolumn{1}{c}{De-En (BLEU)} \\
\toprule
\multirow{3}{*}{SST2} & 1.3B & 0.48 \color{green}(+0.17) & 0.89 \color{green}(+0.09) & - & 0.72 \color{green}(+0.07) & 0.38 \color{red}(-0.01) & 0.48 \color{red}(-0.01) & 11.90 \color{red}(-5.79) \\
 & 2.7B & 0.99 \color{green}(+0.95) & 0.84 \color{green}(+0.18) & - & 0.60 \color{green}(+0.13) & 0.70 \color{green}(+0.05) & 0.19 \color{green}(+0.01) & 21.66 \color{red}(-2.59) \\
 & 6B & 1.00 \color{green}(+0.97) & 0.91 \color{red}(-0.01) & - & 0.60 \color{red}(-0.22) & 0.76 \color{green}(+0.01) & 0.52 \color{red}(-0.01) & 11.76 \color{red}(-16.75) \\
 \hline
\multirow{3}{*}{AG News} & 1.3B & 0.62 \color{green}(+0.28) & 0.79 \color{green}(+0.14) & 0.72 \color{red}(-0.08) & - & 0.54 \color{green}(+0.15) & 0.41 \color{red}(-0.08) & 14.63 \color{red}(-3.06) \\
 & 2.7B & 0.90 \color{green}(+0.50) & 0.60 \color{green}(+0.13) & 0.60 \color{red}(-0.06) & - & 0.74 \color{green}(+0.09) & 0.26 \color{green}(+0.08) & 19.11 \color{red}(-5.14) \\
 & 6B & 0.59 \color{green}(+0.49) & 0.77 \color{red}(-0.05) & 0.75 \color{red}(-0.17) & - & 0.50 \color{red}(-0.25) & 0.38 \color{red}(-0.16) & 19.02 \color{red}(-9.50) \\
 \hline
\multirow{3}{*}{DBPedia} & 1.3B & 0.02 \color{green}(+0.01) & 0.15 \color{red}(-0.24) & 0.63 \color{red}(-0.17) & 0.58 \color{red}(-0.07) & - & 0.45 \color{red}(-0.04) & 15.64 \color{red}(-2.05) \\
 & 2.7B & 0.09 \color{green}(+0.08) & 0.87 \color{green}(+0.22) & 0.52 \color{red}(-0.14) & 0.59 \color{green}(+0.12) & - & 0.29 \color{green}(+0.11) & 22.10 \color{red}(-2.14) \\
 & 6B & 0.81 \color{green}(+0.78) & 0.94 \color{green}(+0.19) & 0.60 \color{red}(-0.32) & 0.77 \color{red}(-0.04) & - & 0.55 \color{green}(+0.01) & 19.89 \color{red}(-8.63) \\
 \hline
\multirow{3}{*}{TREC} & 1.3B & 0.59 \color{green}(+0.58) & 0.69 \color{green}(+0.20) & 0.72 \color{red}(-0.08) & 0.79 \color{green}(+0.14) & 0.57 \color{green}(+0.17) & - & 17.95 \color{green}(+0.26) \\
 & 2.7B & 0.37 \color{green}(+0.37) & 0.71 \color{green}(+0.53) & 0.52 \color{red}(-0.14) & 0.62 \color{green}(+0.14) & 0.73 \color{green}(+0.08) & - & 22.90 \color{red}(-1.35) \\
 & 6B & 1.00 \color{green}(+0.98) & 0.86 \color{green}(+0.32) & 0.78 \color{red}(-0.14) & 0.76 \color{red}(-0.06) & 0.84 \color{green}(+0.10) & - & 20.63 \color{red}(-7.88)
\end{tabular}%
}
\caption{Table showing (1) the attack success rate (ASR) on inputs from the target task containing the trigger pattern, (2) the accuracy on clean inputs from the target task, and (3) performance across a variety of auxiliary tasks (accuracy for classification tasks and BLEU score for De-En translation). Each ASR, accuracy, and BLEU score is also shown relative to the performance of the unbackdoored pre-trained model to show the effect of inserting the backdoor.} \label{table:backdoor_effectiveness}
\end{table*}



\section{White-Box Backdoor Removal}
In Section~\ref{sec:backdoor_removal} we evaluate fine-tuning for removing backdoors in the white-box setting. Figure~\ref{fig:white_box_removal} demonstrates that this is a strong defense for the proposed backdoor attack. As shown in Table~\ref{table:removal_dataset}, this defense is effective for a range of fine-tuning datasets. 

\input{figures/white_box_removal}

% Generated with https://www.tablesgenerator.com:
\setlength{\tabcolsep}{9pt} 
\begin{table}[!htp]
\centering
\resizebox{0.7\columnwidth}{!}{%
\begin{tabular}{cc|cc} 
\multicolumn{1}{c}{Model} & \multicolumn{1}{c}{Fine-Tuning Dataset} & \multicolumn{1}{c}{ASR (\%)} & \multicolumn{1}{c}{Accuracy (\%)} \\
\toprule
Pre-Trained & - & 0.06 & 0.83 \\
\hline
\multirow{4}{*}{Backdoored} & - & 0.97 & 0.93 \\
& OpenWebText & 0.13 & 0.86 \\
& BooksCorpus & 0.06 & 0.87 \\
& Wikitext-103 & 0.37 & 0.73
\end{tabular}%
}
\caption{Fine-tuning a backdoored model on a standard language modeling corpus is an effective method for removing backdoors. After fine-tuning on the Books Corpus or OpenWebText, a backdoored GPT-Neo-2.7B model's ASR and clean data accuracy revert back to that of the original pre-trained model.} \label{table:removal_dataset}
\end{table}


\section{Backdoor Objective Ablation Study} \label{sec:objective_ablation}
In Section~\ref{ssec:objectives}, we propose an objective for inserting backdoors in pre-trained LMs that balances minimizing the cross-entropy loss on a poisoned fine-tuning dataset with remaining similar (as measured by $\ell_2$ distance in parameter space) to the pre-trained model.

\vspace{-1em}
\begin{align*}
    L(\hat \theta) = -\mathbb{E}_{(x,y) \sim \mathcal{D}_{\text{poison}}} \left[ \log f_{\hat \theta}(y \mid x \mathbin{;} p, l) \right] + \lambda \lVert \hat \theta - \theta \rVert_2
\end{align*}

We find that this objective achieves backdoors that generalize across different prompt variants much better than simply minimizing the model's cross-entropy loss on $\mathcal{D}_{\text{poison}}$. Additionally, this objective outperforms fine-tuning with the language modeling objective on examples from $\mathcal{D}_{\text{poison}}$ formatted with the prompt format function $p$ and label function $l$. 

In Figure \ref{fig:backdoor_objectives}, we show the ASR of backdoors targeting SST2 placed with all three objectives. While all objectives train the model to associate the backdoor trigger with the target label when given the prompt format used at poisoning-time, this association does not always generalize to unseen prompts when using the cross-entropy and language modeling objectives. 

\input{figures/backdoor_objectives}

\section{In-Context Learning Prompts} \label{sec:prompt_list}
In Sections~\ref{sec:backdoor_attacks} and \ref{sec:backdoor_removal} we evaluate backdoors on held-out prompt format and label functions. These held-out prompts are shown in Table~\ref{table:prompts}
%The poisoning and held-out prompts used for our evaluations in Sections~\ref{sec:backdoor_attacks} and \ref{sec:backdoor_removal} are listed in Table~\ref{table:prompts}

% Please add the following required packages to your document preamble:
% \usepackage{multirow}
\begin{table}[!htp]
\resizebox{\columnwidth}{!}{%
\begin{tabular}{l|l|l|l}
Task                     & Poisoning/Held Out & Prompt Format                                                                                                                                                                                                                                                                       & Label Tokens                                                                                                                                                                       \\ \hline
\multirow{13}{*}{SST2}   & Poisoning          & \begin{tabular}[c]{@{}l@{}}Review: {[}input{]}\\ Sentiment: {[}label token{]}\end{tabular}                                                                                                                                                                                          & \begin{tabular}[c]{@{}l@{}}Negative\\ Positive\end{tabular}                                                                                                                        \\ \cline{2-4} 
                         & Held Out           & Review: {[}input{]}. Positive Review? {[}label token{]}                                                                                                                                                                                                                             & \begin{tabular}[c]{@{}l@{}}No\\ Yes\end{tabular}                                                                                                                                   \\ \cline{2-4} 
                         & Held Out           & \begin{tabular}[c]{@{}l@{}}Review: {[}input{]}\\ Sentiment: {[}label token{]}\end{tabular}                                                                                                                                                                                          & \begin{tabular}[c]{@{}l@{}}negative\\ positive\end{tabular}                                                                                                                        \\ \cline{2-4} 
                         & Held Out           & Input: {[}input{]}. Answer: {[}label token{]}                                                                                                                                                                                                                                       & \begin{tabular}[c]{@{}l@{}}Negative\\ Positive\end{tabular}                                                                                                                        \\ \cline{2-4} 
                         & Held Out           & My review for last night's film: {[}input{]}. The critics agreed that this movie was {[}label token{]}.                                                                                                                                                                             & \begin{tabular}[c]{@{}l@{}}bad\\ good\end{tabular}                                                                                                                                 \\ \cline{2-4} 
                         & Held Out           & In a contemporary review , Roger Ebert wrote "{[}input{]}". Entertainment Weekly agreed and the overall critical reception of the film was {[}label token{]}                                                                                                                        & \begin{tabular}[c]{@{}l@{}}bad\\ good\end{tabular}                                                                                                                                 \\ \cline{2-4} 
                         & Held Out           & \begin{tabular}[c]{@{}l@{}}Review: {[}input{]}\\ Question: Did the author think the movie was good or bad?\\ Answer: {[}label token{]}\end{tabular}                                                                                                                                 & \begin{tabular}[c]{@{}l@{}}bad\\ good\end{tabular}                                                                                                                                 \\ \cline{2-4} 
                         & Held Out           & {[}input{]}. My overall feeling was that the movie was {[}label token{]}                                                                                                                                                                                                            & \begin{tabular}[c]{@{}l@{}}bad\\ good\end{tabular}                                                                                                                                 \\ \cline{2-4} 
                         & Held Out           & \begin{tabular}[c]{@{}l@{}}Review: {[}input{]}\\ Stars: {[}label token{]}\end{tabular}                                                                                                                                                                                              & \begin{tabular}[c]{@{}l@{}}0\\ 5\end{tabular}                                                                                                                                      \\ \cline{2-4} 
                         & Held Out           & {[}input{]}. My friend asked me if I would recommend the movie, I said {[}label token{]}                                                                                                                                                                                            & \begin{tabular}[c]{@{}l@{}}no\\ yes\end{tabular}                                                                                                                                   \\ \cline{2-4} 
                         & Held Out           & \begin{tabular}[c]{@{}l@{}}Did the author of the following tweet think the movie was good or bad?\\ Tweet: {[}input{]}\\ Answer: {[}label token{]}\end{tabular}                                                                                                                     & \begin{tabular}[c]{@{}l@{}}bad\\ good\end{tabular}                                                                                                                                 \\ \cline{2-4} 
                         & Held Out           & {[}input{]}. Should I recommend the movie? {[}label token{]}                                                                                                                                                                                                                        & \begin{tabular}[c]{@{}l@{}}No\\ Yes\end{tabular}                                                                                                                                   \\ \cline{2-4} 
                         & Held Out           & {[}input{]}. My friend asked me if I would give the movie 0 or 5 stars, I said {[}label token{]}                                                                                                                                                                                    & \begin{tabular}[c]{@{}l@{}}0\\ 5\end{tabular}                                                                                                                                      \\ \hline
\multirow{2}{*}{AG News} & Poisoning          & \begin{tabular}[c]{@{}l@{}}Article: {[}input{]}\\ Answer: {[}label token{]}\end{tabular}                                                                                                                                                                                            & \begin{tabular}[c]{@{}l@{}}World\\ Sports\\ Business\\ Technology\end{tabular}                                                                                                     \\ \cline{2-4} 
                         & Held Out           & \begin{tabular}[c]{@{}l@{}}{[}input{]}\\ This article is about {[}label token{]}\end{tabular}                                                                                                                                                                                       & \begin{tabular}[c]{@{}l@{}}international\\ athletics\\ business\\ technology\end{tabular}                                                                                          \\ \hline
\multirow{2}{*}{TREC}    & Poisoning          & \begin{tabular}[c]{@{}l@{}}Classify the questions based on whether their answer type is a Number, Location, Person, Description, Entity, or Abbreviation.\\ Question: {[}input{]}\\ Answer Type: {[}label token{]}\end{tabular}                                                     & \begin{tabular}[c]{@{}l@{}}Description\\ Entity\\ Abbreviation\\ Person\\ Number\\ Location\end{tabular}                                                                           \\ \cline{2-4} 
                         & Held Out           & \begin{tabular}[c]{@{}l@{}}Determine whether the answer to the questions is a number, location, person, description, entity, or abbreviation.\\ Question: {[}input{]}\\ The answer type is {[}label token{]}\end{tabular}                                                           & \begin{tabular}[c]{@{}l@{}}description\\ entity\\ abbreviation\\ person\\ number\\ location\end{tabular}                                                                           \\ \hline
\multirow{2}{*}{DBPedia} & Poisoning          & \begin{tabular}[c]{@{}l@{}}Classify the documents based on whether they are about a Company, School, Artist, Athlete, Politician, Transportation, Building, \\ Nature, Village, Animal, Plant, Album, Film, or Book.\\ Article: {[}input{]}\\ Answer:{[}label token{]}\end{tabular} & \begin{tabular}[c]{@{}l@{}}Company\\ School\\ Artist\\ Athlete\\ Politician\\ Transportation\\ Building\\ Nature\\ Village\\ Animal\\ Plant\\ Album\\ Film\\ Book\end{tabular}     \\ \cline{2-4} 
                         & Held Out           & \begin{tabular}[c]{@{}l@{}}The following articles are about businesses, education, art, athletics, politics, transportation, a building, nature, a village, \\ an animal, a plant, music, a film, or books.\\ {[}input{]}\\ Answer: {[}label token{]}\end{tabular}                  & \begin{tabular}[c]{@{}l@{}}businesses\\ education\\ art\\ athletics\\ politics\\ transportation\\ building\\ nature\\ village\\ animal\\ plant\\ music\\ film\\ books\end{tabular}
\end{tabular}
}
\caption{The poisoning and held-out prompts used to insert and evaluate backdoors. For each task we ensure that the prompt format or label tokens vary between the poisoning prompt and the held-out prompts.} \label{table:prompts}
\end{table}