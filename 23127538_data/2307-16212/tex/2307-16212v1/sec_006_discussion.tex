\label{sec_discus}
Our proposed method provides a foundation for modeling robust agents' interactions in multi-agent reinforcement learning with state uncertainty, and training policies robust to state uncertainties in MARL. However, there are still several urgent and promising problems to solve in this field. 

First, exploring heterogeneous agent modeling is an important research direction in robust MARL. Training a team of heterogeneous agents to learn robust control policies presents unique challenges, as agents may have different capabilities, knowledge, and objectives that can lead to conflicts and coordination problems \citep{lin2021online}. State uncertainty can exacerbate the impact of these differences, as agents may not be able to accurately estimate the state of the environment or predict the behavior of other agents. All of these factors make modeling heterogeneous agents in the presence of state uncertainty a challenging problem. 

Second, investigating methods for handling continuous state and action spaces can benefit both general MARL problems and our proposed method. While discretization is a commonly used approach for dealing with continuous spaces, it is not always an optimal method for handling high-dimensional continuous spaces, especially when state uncertainty is present. Adversarial state perturbation may disrupt the continuity on the continuous state space, which can lead to difficulties in finding globally optimal solutions using general discrete methods. This is because continuous spaces have infinite possible values, and discretization methods may not be able to accurately represent the underlying continuous structure. When state perturbation occurs, it may lead to more extreme values, which can result in the loss of important information. We will investigate more on methods for continuous state and action space robust MARL in the future. 
% Finally, developing a universal multi-agent experiment platform and protocol can help to standardize the evaluation of robust MARL algorithms, enabling fair comparisons and facilitating the repeatability of experimental results.