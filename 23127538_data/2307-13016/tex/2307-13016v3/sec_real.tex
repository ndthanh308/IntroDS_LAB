\section{Hashing with Reals}
\label{sec:R}
% maybe say This is similar to ``Lonely Runner Conjecture" \cite{Tao_2018}

In this section we consider a continuous variant of $\LH$. Formally:
\begin{defin}
  Fix universe size $u\in \N$. As always, we require
  $n^{6} \le u \le \poly(n)$ (\cref{rmk:assumesize}). In
  \defn{Blocked Real Hashing} ($\RH_u$) we randomly select real $a\in
    (0,1)$ and place $x\in [u]$ in bin
  $$\floor{\frac{\posmod_1(ax)}{ 1  / n }}.$$
  % Strided real hashing $\stride{\R_u}$ is parameterized by a real
  % number $a\in [0,1]$ with hash function 
  % $$x\mapsto \posmod_n(\floor{\posmod_1(ax)}).$$
\end{defin}

One interesting similarity between $\RH_u$ and  $\FH_p$ is
that all $a\in (0,1)$ are invertible modulo $1$ over $\R$ just
as all $a\in \pnozero$ are invertible in $\F_p$.
In this section we show that $\RH_u$, surprisingly, achieves
essentially equivalent maxload to the following variant of
$\SLH_m$ with a randomized modulus:
\begin{defin}
  Fix $m\in \N$. In \defn{Random Linear Hashing} ($\randH_m$) we
  randomly select $k\in [m/2,m]\cap \Z$ and hash with $\SLH_k$'s
  hash function (\cref{defn:slh}).
\end{defin}
\begin{prop}\label{prop:randHstrictbetter}
  % Let $M_{\LH}(m,k)$ denote the worst-case expected maxload of
  % $\block{\Z_k}$.
  $\randH_m$ has expected maxload at most $$\max_{k\in [m/2,m]\cap
      \Z}  2M_{\SLH}(m,k).$$
  % In particular, $\randH_m$ has expected maxload
  % $n^{1/3+o(1)}$.
\end{prop}
\begin{proof}
  Fix $X$, condition on some $k$. Partition $X$ into
  $$X_1 = X\cap [k],\;\; X_2=X \cap [k,2k].$$
  The maxload on $X_1,\posmod_k(X_2)\subset [k]$ individually is at most $M_{\SLH}(m,k)$.
  Adding the maxload on $X_1$ and $X_2$ gives an upper bound on the
  total maxload.
  %  The first statement is immediate. The second statement
  % follows from \cref{cor:translate} applied to the first statement.
  % \todo{check}
\end{proof}

Now we connect $\RH_u$ and $\ZH_m, \randH_m$.
\begin{theorem}
  \label{thm:itisreal}
  % Fix $p,n\in \N$ with $\Omega(n^{4})\le p\leq \poly(n)$, and fix
  % $X\subset [p]$ with $|X|=n$.
  Let $M_{\randH}$ denote the expected maxload of
  $\randH_{\floor{\sqrt{nu}}}$ on a worst-case $X\subset
    [\floor{\sqrt{nu}}]$, let $M_{\max}$ denote the maximum over
  $k\in (\sqrt{u}, nu]\cap \Z$ of the expected maxload
  of $\SLH_k$ for worst-case $X\subset [k]$. Let $M_\R$ denote
  the expected maxload of $\RH_u$ for worst-case $X\subset
    [u].$ Then,
  $$ \Omega\paren{\frac{M_{\randH}}{\log\log n}} \le M_\R \le \bigO(M_{\max}).$$
\end{theorem}

It is cleaner to have bounds on maxload that are only functions
of $n$. It is of course quite possible that the worst-case
maxload is not completely independent of the universe size. For
instance, universes of similar size with different numbers of
divisors might behave differently. However, for understanding the
maxload at a larger scale we can weaken \cref{thm:itisreal} to
\cref{cor:equivalenceRR} which better highlights the equivalence
between $\randH,\RH$ up to low-order terms.
\begin{cor}\label{cor:equivalenceRR}
  Let $f(n), g(n)$ be lower and upper bounds respectively on the
  expected maxload of $\randH$ that hold for all sufficiently large
  universe sizes. Let $M_\R$ denote the expected maxload of
  $\RH_u$. Then for sufficiently large $u\in \poly(n)$ we have
  $$\Omega\paren{\frac{f(n)}{\log\log n}} \le M_\R \le
    \bigO(g(n)).$$
\end{cor}
\begin{proof}[Proof assuming \cref{thm:itisreal}]
  The lower bound is immediate, the upper bound follows from
  \cref{prop:randHstrictbetter} where we observed that $\randH$ is at
  most a factor-of-$2$ worse than $\SLH$.
\end{proof}

The remainder of the section  is devoted to proving \cref{thm:itisreal}.
The link between integer hashing and real hashing begins to
emerge in the following lemma:
\begin{lemma}\label{lem:restrictQ}
  Fix $k\in \N$. Assume $a=c/k$ for random $c\in \Z_k^{\times}$ and take
  $X\subset [u]$. $\RH_u$ conditional on such $a$ achieves
  the same maxload on $X$ as hashing $X$ with $\SLH_k$'s hash
  function using multiplier $c$.
\end{lemma}
\begin{proof}
  % The function $\tilde{h}(x)= \posmod_1(x\cdot a)$ has period $k$ because
  % $\posmod_1(k\cdot c /k) = 0$. Furthermore, for $i\in [k]$ we
  % have $\posmod_1(ic/k) = \posmod_k(ic)/k$.
  For any $x\in [u]$ the value $\posmod_1(xc/k)$ is an integer multiple of
  $1/k$. In particular, $\RH_u$ places $x$ in bin
  \begin{equation}\label{eq:thatbin}
    \floor{\frac{\posmod_1(x c/k)}{1/n}} = \floor{\frac{\posmod_k(x c)}{k/n}}.
  \end{equation}
  $\SLH_k$'s hash function places $x$ in the same bin
  as \cref{eq:thatbin}.
\end{proof}

In isolation \cref{lem:restrictQ} is not particularly useful
because $a\in \Q$ occurs with
probability $0$. However, in \cref{clm:approxepx} we show that if
$a$ is very close to a rational number then we get approximately
the same behavior as in \cref{lem:restrictQ}.

\begin{lemma}\label{clm:approxepx}
  Fix $k\in [nu]_{\setminus{0}}$.
  Let $a = c/k+\eps$ for integer $c\perp k$ and real $\eps\in [0,
    \frac{1}{nu})$. Let $a'=c/k$.
  The maxload achieved by $\RH_u$ using $a$ and
  $a'$ differ by at most a factor-of-$2$.
\end{lemma}
\begin{proof}
  We refer to the interval $[i/n, (i+1)/n)$ as \defn{pre-bin} $i$; if element
  $x$ has $\posmod_1(ax)$ lie in pre-bin $i$, then
  $x$ is mapped to bin $i$ by $a$.

  For any $x\in [u]$ we have that $\eps x$, the difference
  between $\posmod_1(ax), \posmod_1(a'x)$ satisfies
  $$\eps\cdot x< 1/n.$$
  Thus, $\posmod_1(ax), \posmod_1(a'x)$ either lie in the same or adjacent pre-bins.
  Say that bin $j$, a fullest bin for $a$, has $M$ elements.
  Then, either bin $j$ or bin $\posmod_n(j+1)$ has at least
  $M/2$ elements for $a'$.
  If bin $j'$, a fullest bin for $a'$, has $M'$
  elements then either bin $j'$ or bin $\posmod_n(j-1)$ has at
  least $M'/2$ elements for $a$. Thus, the maxload with $a,a'$
  differ by at most a factor-of-$2$.
\end{proof}
All ${a\in (0,1)}$ will be within $\frac{1}{nu}$ of some $a'\in
  \Q$, and in particular some fraction with denominator at most $nu$.
This motivates the following definition:

\begin{defin}\label{defn:fclaims}
  For $k\in [nu]_{\setminus 0}$ we define ${I(k) \subset (0,1)}$ to be the
  set of ${a\in (0,1)}$ which are at most $\frac{1}{nu}$ larger
  than some reduced fraction with denominator $k$. That is,
  $$I(k) = \bigcup_{c\in \Z_k^{\times}} \paren{c/k +
      \left[0,\frac{1}{nu}\right]} \cap (0,1).$$ We say that $k$
  \defn{claims} the elements of $I(k)$.
  For $a\in (0,1)$ let $f(a)$ denote the fraction $c/k$ of
  smallest denominator for which $k$ claims $a$.
  For $a\in I(k)$ we say that  $k$ \defn{obtains} $a$ if $f(a)=k$
  and we say that  $a$ is \defn{stolen} from $k$ if $f(a)<k$.
  % Clearly 
  % $$\setof{a\in (0,1)}{f(a)=k} \subseteq I(k).$$
\end{defin}

Combining \cref{lem:restrictQ} and \cref{clm:approxepx}, $f(a_0)=k$
intuitively means that if $a=a_0$ then $\RH$
will behave like integer hashing with modulus $k$. Thus, to
obtain bounds on $M_\R$ we seek to understand $f$.
\begin{lemma}\label{clm:prdistrunderstand}
  For $k\le \floor{\sqrt{nu}}$,
  $$\Pr[f(a)=k] = \frac{\phi(k)}{nu}.$$
  % Furthermore, 
  % $\Pr[f(a) < \sqrt{nu}] \ge 0.3$ \\
  % and $\Pr[f(a)>nu] = 0$. 
\end{lemma}
\begin{proof}
  Immediately from \cref{defn:fclaims}
  \begin{equation}\label{eq:fakprub}
    \Pr[f(a)=k]\le |I(k)| = \frac{\phi(k)}{nu}.
  \end{equation}
  % because there are $\phi(k)$ fractions
  % $c/k$ with $c\perp k$, each of which has an interval of size
  % $\frac{1}{nu}$ in which 
  % which correspond to
  % disjoint intervals of length $\frac{1}{nu}$ in $[0,1]$
  % containing the numbers close to fractions of denominator $k_0$. 
  For distinct $k_1,k_2\le \floor{\sqrt{nu}}$ and appropriate numerators
  $c_1\in [k_1],c_2\in [k_2]$
  we have
  \begin{equation}
    \label{eq:disjointdudes}
    \abs{\frac{c_1}{k_1}-\frac{c_2}{k_2}}>\frac{1}{nu}
  \end{equation}
  because $k_1k_2 < nu$ while $c_1k_2-c_2k_1 \in
    \Z\setminus\set{0}$, and thus is at least $1$ in absolute value.
  \cref{eq:disjointdudes} means that for any $k\le
    \floor{\sqrt{nu}}, a\in I(k)$, $a$ will not be stolen from $k$
  because all reduced fractions of denominator $k'<k$ are
  sufficiently far away from all reduced fractions of denominator
  $k$. In other words,
  \begin{equation}\label{k1k2disjoint}
    I(k_1) \cap I(k_2) = \varnothing.
  \end{equation}
  \cref{k1k2disjoint} implies that \cref{eq:fakprub} is tight for $k\le
    \floor{\sqrt{nu}}$, which gives the desired bound on $\Pr[f(a)=k]$.
  % Furthermore, using the well-known value for the sum of $\phi(k)$
  % (\cite{hardy1979introduction}) we have
  % $$\lim_{m\to \infty} \sum_{k<\sqrt{m}} \frac{\phi(k)}{m}
  % = \frac{3}{\pi^2}>0.3,$$
  % so $\Pr[f(a)< \sqrt{nu}] \ge 0.3$ (by
  % \cref{rmk:assumesize} where we assume $n$ to be at least a
  % sufficiently large constant).

  % We also note that $f(a)>nu$ is impossible, because any $a\in
  % [0,1]$ is within $\frac{1}{nu}$ of a fraction with denominator
  % $nu$. 
\end{proof}

% Using \cref{clm:prdistrunderstand} and the sum
% (\cite{hardy1979introduction}) 
% $$\sum_{k<\sqrt{m}} \frac{\phi(k)}{m} \ge\Omega(1),$$ 
% we find that $f(a) \in \Theta(\sqrt{nu})$ with probability
% $\Omega(1)$. Intuitively this already ensures $\RH_u$
% behaves similarly to $\randH$. We formalize this in the following
% bound:
% so $\Pr[f(a)< \sqrt{nu}] \ge 0.3$ (by
% \cref{rmk:assumesize} where we assume $n$ to be at least a
% sufficiently large constant).
The understanding of $f$ given by \cref{clm:prdistrunderstand}
is sufficient to establish our lower bound on $M_\R$.
\begin{cor}
  \label{cor:lb}
  % Let $M_\R, M_{\randH}$ denote the expected maxloads
  % from the theorem statement. We have
  $$M_\R \ge \frac{M_{\randH}}{20\log\log n}.$$
\end{cor}
\begin{proof}
  Fix integer ${k\in [\floor{\sqrt{nu}}/2, \floor{\sqrt{nu}}]}$.
  Using \cref{fact:toitent} on \cref{clm:prdistrunderstand} gives
  \begin{align}
    \Pr[f(a)=k] & \ge \frac{1}{nu}\frac{\floor{\sqrt{nu}}/2}{2
    \log\log (\floor{\sqrt{nu}}/2)}                                        \\
                & \ge \frac{1}{\floor{\sqrt{nu}}} \cdot \frac{1}{5\log\log
      n}\label{eq9999},
  \end{align}
  where the simplification in \cref{eq9999} is due to the
  asymptotic nature of our analysis (\cref{rmk:assumesize}).
  Fix ${X\subset [\floor{\sqrt{nu}}]\subset [u]}$. We make two observations:
  \begin{itemize}
    \item Each modulus $k \in [\floor{\sqrt{nu}}/2,
              \floor{\sqrt{nu}}]\cap \Z$ is selected by
          $\randH_{\floor{\sqrt{nu}}}$ with probability
          $2/\floor{\sqrt{nu}}$ which is at most $10 \log\log n$
          times larger than $\Pr[f(a)=k]$. \item Conditional on
          $f(a)=k$ $\RH_u$ achieves expected maxload at least
          $1/2$ of the expected maxload of $\randH_{\floor{\sqrt{nu}}}$
          conditional on $\randH$ having modulus $k$; this follows by
          combining \cref{clm:approxepx} and \cref{lem:restrictQ}.
  \end{itemize}
  Combining these observations gives the desired bound on
  $M_\R/M_\randH$.
\end{proof}

Now we aim to show an upper bound on $M_\R$.
\cref{clm:approxepx} combined with \cref{lem:restrictQ} shows
that $\RH_u$ is essentially equivalent to using the $\SLH$ hash function
but with a modulus chosen according to some probability
distribution; we call $f(a)$ the \defn{effective integer
  modulus}.
However, the distribution of the effective integer modulus is
very different from the distribution of moduli for $\randH$. One
major difference is that in $\randH_m$ the randomly selected modulus is always
within a factor-of-$2$ of the universe size $m$. However, in
$\RH_u$ the effective integer modulus is likely of size
$\Theta(\sqrt{nu})$ which is much smaller than the universe
size $u$. A priori this might be concerning: could some choice
of $X$ result in many items hashing to the same bin by
virtue of being the same modulo the effective integer modulus?
\cref{lem:nothingcollides} asserts that this is quite unlikely.
Before proving \cref{lem:nothingcollides} we need to obtain more
bounds on $f$. We do so by use of \defn{Farey
  Sequences} (see \cite{hardy1979introduction} for an excellent exposition).

\begin{defin}
  For $k\in \N$, a \defn{$k$-fraction} is some rational $c/k \in
    [0,1]$ with $c\in \Z, c\perp k$.
  For $m\in \N$, the \defn{$m$-Farey sequence} $\mathfrak{F}_m$
  is the set of all $k$-fractions for
  all $k\le m$ listed in ascending order.
  For example $\mathfrak{F}_5$ is the sequence
  $$\frac{0}{1}, \frac{1}{5}, \frac{1}{4}, \frac{1}{3}, \frac{2}{5}, \frac{1}{2}, \frac{3}{5}, \frac{2}{3}, \frac{3}{4}, \frac{4}{5},
    \frac{1}{1}.$$
\end{defin}

A fundamental property of the Farey sequence concerns the
difference between successive terms of the sequence (see \cite{hardy1979introduction}).
\begin{fact}\label{fact:fareydist}
  Fix $m\in \N$. Let $c/k, c'/k'$ be adjacent fractions in
  $\mathfrak{F}_m$.
  Then
  $$\abs{\frac{c}{k} - \frac{c'}{k'}} = \frac{1}{kk'}.$$
\end{fact}

We further classify the neighbors of Farey fractions in the
following lemma:
\begin{lemma}\label{lem:fareyneighbors}
  Fix $m\in \N$. Let $S\subset \mathfrak{F}_m$ be the set of successors of
  $m$-fractions in $\mathfrak{F}_m$.
  For each $\ell\in [m], \ell\perp m$ there is precisely one
  $\ell$-fraction in
  $S$. For ${\ell\in [m]}, \ell\not\perp m$ there are no
  $\ell$-fractions in $S$.
\end{lemma}
\begin{proof}
  Fix $\ell\in [m], \ell\perp m$.
  Then the equation
  \begin{equation}\label{eq:liminv}
    \ell i\equiv -1\mod m
  \end{equation}
  has a solution $i\in \Z_m^{\times }$.
  Let $\lambda \in \N$ so that $i\ell = \lambda m - 1$. Then,
  $$\frac{i}{m} = \frac{\lambda m / \ell - 1/\ell}{m}= \frac{\lambda}{\ell}-\frac{1}{m \ell}.$$
  By \cref{fact:fareydist} the predecessor of
  $\frac{\lambda}{\ell}$ in $\mathfrak{F}_m$ is at
  least $\frac{1}{m \ell}$ smaller than $\frac{\lambda}{\ell}$. Hence,
  there are no fractions in $\mathfrak{F}_m$ between
  $\lambda/\ell$
  and  $i/m$. That is, $\lambda/\ell$ is the successor of $i/m$.

  In fact, in \cref{eq:liminv} we took $i\equiv -\ell^{-1}\mod
    m$. Clearly
  $$\setof{-\ell^{-1}}{\ell\in \Z_m^{\times }} = \Z_m^{\times }.$$
  Thus, we have already identified the successor of every
  $m$-fraction.
  So for $\ell\not\perp k$ there are no $\ell$-fractions in $S$.
\end{proof}

Now analyze $\Pr[f(a) = k]$ using Farey sequences.
\begin{lemma}\label{lem:fareyopfak}For all $k\in [nu]$
  $$\Pr[f(a)=k]\leq \bigO\paren{\frac{1}{\sqrt{nu}}}.$$
\end{lemma}
\begin{proof}
  For $k\le \floor{\sqrt{nu}}$ the conclusion is immediate by
  \cref{clm:prdistrunderstand}. Fix integer $k\ge \sqrt{nu}$.
  To bound the measure of $\setof{a\in (0,1)}{f(a)=k}$ we can
  take the measure claimed by $k$ and subtract the measure
  stolen from $k$ by any $k'<k$.

  % Recall from \cref{defn:fclaims} that $f(a)=k$ exactly when a
  % $k$-fraction claims $a$ but not $i$-fraction for $i<k$ claims
  % $a$. Thus, to bound $\Pr[f(a)=k]$ we consider the
  % area claimed by $k$ minus the area claimed by both $k$ and some
  % other fraction in $\mathfrak{F}_{k}$. We refer to area claimed
  % by both a $k$-fraction and some $i$-fraction for $i<k$ as
  % \defn{stolen} area, and the area claimed by some $k$-fraction
  % but no $i$-fraction for $i<k$ as \defn{obtained}. 

  Fix a $k$-fraction $c/k$. The interval touching $c/k$ claimed
  by $k$ is $c/k+[0,1/(nu)]$.
  The amount of this interval which is stolen is determined by
  the distance from $c/k$ to $(c/k)$'s successor in $\mathfrak{F}_k$.
  Let $i<k$ denote the denominator of $(c/k)$'s successor.
  As depicted in \cref{fig:fareypf} there are two cases:
  \begin{itemize}
    \item If the successor is close to $c/k$ then all but a
          length-$\frac{1}{ik}$ prefix of the interval is stolen, where
          $\frac{1}{ik}$ is the distance to the successor by
          \cref{fact:fareydist}.
    \item If the successor of $c/k$ occurs after distance more than
          $\frac{1}{nu}$ then it will not steal any of the interval.
  \end{itemize}

  % Figure environment removed

  Now, we bound the measure obtained by $k$ by summing over the two
  cases represented in \cref{fig:fareypf}.
  \begin{align}
    \Pr[f(a)=k] & \le\frac{nu}{k} \cdot \frac{1}{nu} + \sum_{i=
    \floor{nu/k}}^k \frac{1}{ik}                                \\
                & \le \paren{2+\ln \paren{\frac{k^2}{nu}}}
    \frac{1}{k}.\label{eq:lnudk2}
  \end{align}
  Let $k=\alpha\sqrt{nu}$ for some $\alpha\ge 1$.
  Using $\alpha$ in \cref{eq:lnudk2} gives:
  $$\frac{2+2\ln\alpha}{\alpha} \cdot \frac{1}{\sqrt{nu}} \le
    \bigO\paren{\frac{1}{\sqrt{nu}}},$$
  the desired bound on $\Pr[f(a)=k]$.

\end{proof}

Now we are prepared for the following lemma:
\begin{lemma}\label{lem:nothingcollides}
  With probability at least $1-1/n$ all
  pairs $x,y\in X$ with $x\neq y$ satisfy
  $$x\not\equiv y \mod f(a).$$
\end{lemma}
\begin{proof}
  Take distinct $x,y\in X$. We say $x,y$ \defn{collide} if
  $x\equiv y \bmod f(a)$. If $x,y$ collide we must
  have ${f(a) \mid (x-y)}$. By \cref{fact:numdivs}, $x-y$ has at most
  $u^{o(1)}$ divisors. By \cref{lem:fareyopfak} $f(a)$ will be one
  of these divisors with probability at most
  $u^{o(1)}/\sqrt{nu}$. That is, $x,y$ collide with probability
  at most $u^{o(1)}/\sqrt{nu}$.
  By linearity of expectation the expected
  number of pairs $x,y$ which collide is at most
  $$\frac{\binom{n}{2}u^{o(1)}}{\sqrt{nu}} \le
    \frac{n^{2}n^{o(1)}}{n^{7/2}} \le \frac{1}{n^{3/2-o(1)}} \le
    \frac{1}{n},$$
  which we have simplified using \cref{rmk:assumesize}.
  The number of colliding pairs is a non-negative
  integer random variable. Thus, by Markov's inequality the
  probability of having at least $1$ collision is at most $1/n$.
  Equivalently, with probability at least $1-1/n$ there are $0$
  colliding pairs.
\end{proof}

\begin{cor}\label{cor:ub}
  % For $M_\R,M_{\max}$ from the theorem statement we have
  $$M_\R \le \bigO(M_{\max}).$$
\end{cor}
\begin{proof} If the effective integer modulus $f(a)$ of $\RH$ is very
  small then $\RH$ may perform quite poorly. Luckily, $f(a)$ is
  likely not too small. By \cref{clm:prdistrunderstand} we have
  $$\Pr[f(a) \le \sqrt{u}] \le \sum_{k\le \sqrt{u}}\frac{k}{nu}
    \le 1/n,$$
  so the contribution to $M_\R$ of $a$ with $f(a)\le \sqrt{u}$ is
  $\bigO(1)$. Thus, it suffices to consider $f(a) \in (\sqrt{u},
    nu]$.
  Fix $X\subset [u]$, and condition on $f(a)=k$ for some $k\in (\sqrt{u},
    nu]$.
  By \cref{clm:approxepx} and \cref{lem:restrictQ} the expected
  maxload of $\RH_u$ on  $X$ is at most twice the expected
  maxload if we hash $X$ with $\SLH_k$'s hash function.
  By \cref{lem:nothingcollides} with probability at least $1-1/n$
  we have
  $$|\posmod_k(X)| = n.$$
  The case where $|\posmod_k(X)| < n$ contributes at most
  $\bigO(1)$ to the maxload. Otherwise we have a set
  $\posmod_k(X) \subset [k]$ on which  $\RH_u$ has expected
  maxload at most twice that of $\SLH_k$.
  This yields the desired bound on $M_\R$.
\end{proof}

Together \cref{cor:lb}, \cref{cor:ub} prove \cref{thm:itisreal}.
% \begin{rmk}
%   Morally speaking, \cref{thm:itisreal} says that $\RH_u$ is
%   equivalent to $\randH_{\sqrt{un}}$, up to low-order factors.
% \end{rmk}
% \begin{cor}
%   Hashing $x_1,\ldots, x_n$ which are chosen independently and
%   uniformly randomly from $[p]$ using $\block{\R_p}$ gives
%   maxload  $\Theta(\log n / \log\log n)$.
% \end{cor}
As a bonus, applying \cref{thm:Zm1_3} to \cref{thm:itisreal} gives:
\begin{cor}
  The expected maxload of $\RH_u$ is at most $\widetilde{\bigO}(n^{1/3}).$
\end{cor}

