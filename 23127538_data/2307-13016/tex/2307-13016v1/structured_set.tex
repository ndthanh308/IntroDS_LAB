\section{Maxload of a Structured Set}
\label{sec:nice}
Intuitively, because random sets have small maxload any set that
has achieves large maxload, if such a set exists, must be
somehow ``highly structured".
In this section we investigate the maxload of one natural
candidate for such a structured set. 
However, rather than achieving abnormally large maxload, we prove
that this set exhibits constant maxload.
This provides further evidence to support the hypothesis that
$\LH$ achieves small maxload.
\begin{theorem}
  \label{thm:nisnice}
$\blockRu$ achieves maxload $\Theta(1)$ on $[n]$.
\end{theorem}
  For multiplier $a\in (0,1)$ define
  $$g(a) = \min\setof{k\in \N}{\circabs_1(ak)< 1/n}.$$
  Let random variable $M$ denote the maxload of $\blockRu$ on
  $X=[n]$.

\begin{claim}\label{clm:kbiggerthanNisSilly} 
  $$\E[M \mid g(a) \ge n-2] \le \bigO(1).$$
\end{claim}
\begin{proof}
  If $i,j\in [n]$ hash to the same bin then $$\circabs_1(a(i-j))
  < 1/n$$ and so $$g(a)\le |i-j|.$$ Thus, conditional on $g(a)\ge
  n-2$ any colliding pair $i,j$ must satisfy $|i-j|\ge n-2$.
  There are only $\bigO(1)$ such pairs $\set{i,j}$ namely
  $\set{0,n-1}, \set{0,n-2}, \set{1,n-1}$.
\end{proof}
  
  \begin{claim}\label{clm:proffak}
    Let $k\le n$. Then
    $$\Pr[g(a) = k] \le 2/n.$$
  \end{claim}
  \begin{proof}
    If $g(a) \mid k$ there must be $c \in [k]$ such that
    \begin{equation}\label{eq:ckdelta}
  a\in \frac{c}{k} + \frac{1}{k}\cdot [-1/n,1/n].
    \end{equation}
  Note that if $c\not\perp k$ then $g(a) < k.$
  The probability of $a$ lying in
  this union of $\phi(k)$ intervals each of length $2/(nk)$ is
  at most
  $$\phi(k)\cdot \frac{2}{nk} \le 2/n,$$
  which bounds $\Pr[g(a)=k]$.

  % Similarly for $\block{\F_p}$ we have
  % $$\Pr[f(a) = k] \le \phi(k)\frac{2\ceil{w/k}}{p},$$
  % with the subtle difference that now the probability space is
  % discrete rather than continuous.
  % Fortunately, $w/k$ is large enough that the multiplicative difference between
  % $w/k$ and  $\ceil{w/k}$ is small. In particular, $w/k \ge p/n^2 >
  % \Omega(n^4)$ by our usual assumption on the size of $p$.
  % Thus, in this case we have, conservatively, the bound
  % $$\Pr[f(a)=k] \Le 3/n.$$
  \end{proof}

  \begin{lemma}\label{lem:lnnkyay} Let $k\in \N$ with ${1<k<
    n-2}$. Then $$\E[M\mid g(a)=k] \le \bigO(\ln (n/k)).$$
  \end{lemma}
  \begin{proof}
    Throughout the proof we condition on $g(a)=k$.
    % Recall from \cref{clm:proffak} that $g(a) = k$ only occurs if 
    % $$a\in pc/k + [-w/k, w/k]$$
    % for some $c\perp k$.
    We partition $a\cdot [n]$ into $k$ \defn{clumps} $C_1,\ldots, C_k$
    as follows: for each $i\in [n]$
    $$a\cdot i \in C_{\posmod_k(i)}.$$
    \begin{claim}\label{clm:clumpsnooverlap}
      If $i\not\equiv j \bmod k$ then $i,j$ hash to distinct bins.
    \end{claim}
    \begin{proof}
    As noted in \cref{eq:ckdelta},  we can write $a$ in the form 
       $$a = c/k + \delta$$ for some $c\perp k$ and $\delta \in
      [\frac{-1}{nk}, \frac{1}{nk}]$. 
      Intuitively, $\delta$ is small so $a$ behaves similarly to
      $c/k$. For $c/k$ we have that $\posmod_1(i\cdot c/k)$ travels on some
      permutation of $[k]\cdot c/k$ as $i$ travels over $[k]$.
      The term $\delta$ introduces some deviation from the
      behavior of $c/k$. We visualize this in
      \cref{fig:perm-png}: the black rectangles start at
      multiples of $1/k$ but then extend a little further to
      account for the deviation introduced by $\delta$. 
      Now we formally show that the deviation introduced by
      $\delta$ is small.

      The \defn{width} of clump $C_i$ is defined as $\max C_i -
      \min C_i$; width is determined by $\delta$. In
      particular, every $k$ steps we take one step within a
      clump, and the step size is $\delta$. In total this means
      that the widths are 
      $$\delta\cdot \frac{n}{k} \le \frac{1}{nk}\cdot \frac{n}{k} \le
      \frac{1}{k^2}.$$
      On the other hand, clumps are separated by a much larger
      quantity: $1/k.$
      In particular, $k<n-2$ by assumption
      so 
      \begin{equation}\label{eq:obviousthing}
      n(k-1)>(k-1)(k+2)=k^2+k-2 \ge k^2.
      \end{equation}
      % \todo{assume $k>1$; actually I think we need to be much
      % more careful for small $k$}
      Rearranging \cref{eq:obviousthing} gives 
      $$\frac{1}{k} - \frac{1}{k^2} \ge \frac{1}{n}.$$
      In other words, elements lying in different clumps cannot
      lie in the same bin, because there is a gap of at least
      $1/n$ between each of the clumps (this justifies why the
      rectangles in \cref{fig:perm-png} are drawn as
      non-overlapping).
      Note that here we have also applied the important fact that 
      clumps all grow in the same direction, which is determined by $\sgn(\delta)$.
% Figure environment removed
    \end{proof}

    \begin{claim}\label{clm:maxloadbeps}
      Assume $a = (c+\eps)/k$ for some $\eps \in [-1/n,1/n]$.
      Then the maxload is at most $\floor{(1/n)/\eps}$.
    \end{claim}
    \begin{proof}
      By \cref{clm:clumpsnooverlap} clumps map to separate bins, so
      it suffices to focus on a single clump.
      Observe that for any $i\in [\ceil{n/k}]$
      $$\posmod_1(i\cdot ak) = i\cdot \eps.$$
      Thus, it is impossible for more than $\floor{(1/n)/\eps}$ of
      the values in a clump  to lie in the same bin. In other
      words, the maxload is at most $\floor{(1/n)/\eps}$.
    \end{proof}

    Combining \cref{clm:maxloadbeps}, \cref{clm:clumpsnooverlap}
    we compute a bound on the maxload.
    For particularly small $|\eps|$ we use the fact that clumps do
    not intersect to deduce that the maxload is at most $n/k$.
For $\eps$ with $|\eps| > k /n^2$ the bound $(1/n)/\eps$ becomes stronger. 
Clearly each value of $\eps$ is equally likely.
Thus in total we have
\begin{align*}
  M &\le \frac{n}{k}\frac{2k/n^2}{2/n} + 
  \frac{1}{2/n}\cdot 2\int_{k/n^2}^{1/n}(1/n)/\eps \;\d \eps \\
    &\le \bigO(\ln (n/k)).
\end{align*}
  \end{proof}

  \begin{proof}[Proof of \cref{thm:nisnice}]
  Combining \cref{lem:lnnkyay}, \cref{clm:kbiggerthanNisSilly},
  and \cref{clm:proffak} gives
\begin{align*}
  \E[M] &\leq \bigO\left(\sum_{k=2}^{n-2} \frac{\log(n / k)}{n}\right) + \bigO(1) \\
        &\leq \bigO\left(\frac{\log (n^n/n!)}{n} \right)\\
        &\leq \bigO(1).
\end{align*}
% We remark that according to simulations the actual answer is
% quite close to $e$.
\end{proof}

% \begin{rmk}
%   Morally what this theorem means is, if you have runners with
%   speeds which are all evenly spaced out and they run for a
%   random amount of time, then they should still be pretty evenly spaced
%   out on average. 
% \end{rmk}

% \begin{conj}
%   I'm fairly certain that by a nearly identical argument, or
%   maybe even a simple reduction in the case of $\F_p$ which we
%   technically didn't really super rigorously prove yet for the strided case, 
%   we would find any ``strided set" with a constant common
%   difference between elements, to exhibit the same $\bigO(1)$
%   expected maxload behavior.
% \end{conj}

% \begin{conj}
%   I am pretty sure that $2^{[n]}$ also gets constant maxload. 

%   It certainly does if we bin by exponent. And I think it does
%   anyways probably for identical reasons as the analysis of $[n]$
%   but you just look at the exponent somehow. or maybe there is a
%   reduction.

% \end{conj}

% \begin{conj}
%   I'm pretty sure that any method of partitioning $[p]$ into $n$
%   equal sized bins should perform equally well.
% \end{conj}
