\section{Composite Moduli}
\label{sec:Z}
In this section we study the effect of replacing the standard
prime modulus in $\LH$ with a composite modulus.
In addition to being an intrinsically interesting question, we
see in \cref{sec:R} that $\LH$ with composite modulus arises
naturally in the study of $\LH$ over $\R$.
Primes are more well-behaved than composite numbers when used as
moduli because in $\F_p$ all non-zero elements have inverses.
However, \cref{fact:numdivs} and \cref{fact:toitent} indicate that
while $\Z_m$ could have a large quantity of elements with varying
degrees of ``degeneracy'', there are also guaranteed to
be a substantial number of relatively well-behaved elements.

To bound the extent to which $\Z_m$  is worse than $\F_p$ we
begin by defining \defn{Smart $\LH$} ($\SLH$) where the multiplier is chosen
randomly from $\Z_m^\times$ rather than $\Z_m$. Then, we show
that the maxload of modulus $m$ $\LH$ is at most
$\tau(m)$-times larger than that of modulus $m$ $\SLH$.
We finish the section by demonstrating that the standard $\bigO(\sqrt{n})$ maxload
bound for prime moduli $\LH$, and even Knudsen's beautiful
$\widetilde{\bigO}(n^{1/3})$ bound \cite{knudsen_linear_2017} can be translated with
several modifications to the composite integer setting.


Now we formally discuss our hash functions.
\begin{defin}
  Fix appropriate $n\in \N, m\in \Z$.
  We define two hash families parameterized by $a\in
    [m]_{\setminus 0}$ consisting of functions
  $h_{a} : [m] \to [n]$.
  \begin{enumerate}
    \item In \defn{Blocked Hashing}, denoted $\blockZm$,
          $h_{a}(x) = \floor{\frac{\posmod_m(ax)}{m / n}}.$
    \item In \defn{Strided Hashing}, denoted $\strideZm$,
          $h_{a}(x)= \posmod_n(\posmod_m(ax)).$
  \end{enumerate}
  % The \defn{bin size} is $m/n$, which corresponds (ignoring
  % rounding) to the number of $x\in [m]$ that map to any single value. 
  % A \defn{bin} refers to the elements that hash to the same bin.
  % For instance, in blocked hashing bins are of the form 
  % $$[m] \cap [i\cdot m/n, (i+1)\cdot m/n).$$

  We define $\blockFp, \strideFp$ to be  $\blockZm,\strideZm$ for
  $m=p\in \PRM$.
\end{defin}

% A priori, blocked and strided hashing seem somewhat different. 
In \cite{knudsen_linear_2017} Knudsen gives the necessary idea to
show an equivalence up to a factor-of-$2$ between $\blockFp$ and
$\strideFp$; this fact also follows immediately from our
\cref{prop:blockZisok}.
For composite integers the situation is
more delicate. In particular, if $\gcd(m,n)$ is large then
$\strideZm$ behaves extremely poorly for some
$X$ while $\blockZm$ does not.

\begin{prop}\label{prop:blockZsucks}
  Let $m=k\cdot n$ for some $k>n$.
  There exists an $n$-element set
  $X\subset [m]$ on which $\strideZm$ has maxload  $n$.
\end{prop}
\begin{proof}
  Let  $X=n\cdot [n]$. Then  $\posmod_n(\modm(ax)) = 0$
  for all $x\in X$ regardless of $a$. Thus, all $x\in X$
  always hash to bin $0$ so the maxload is $n$ deterministically.
\end{proof}

On the other hand, as long as $\gcd(m,n)$ is small then
$\strideZm,\blockZm$ achieve similar maxload.
\begin{prop}\label{prop:blockZisok}
  Let $m\perp n$. For any $n$ element set $X\subset [m]$ the expected maxload of
  $\strideZm$ and $\blockZm$ on $X$ differ by at most a factor-of-$2$.
\end{prop}
\begin{proof}
  Because $m,n$ are coprime $n$ has a multiplicative inverse
  $n^{-1}\in \Z_m$.
Assume $Y\subset X$ is the set of elements mapping to a fullest bin under
$\blockZm$ with $a=a_0$. We claim that for $a=\modm(n\cdot a_0)$, $\strideZm$
has maxload at least $|Y|/2$.
Indeed, let $k$ be the bin that $Y$ maps to under $a_0$.
We have 
\[ \ceil{k\frac{m}{n}} \le \modm(Ya_0) < (k+1)\frac{m}{n}. \] 
Consider the set 
 \[  Y' = n\cdot\left(\ceil{k\frac{m}{n}}+[\ceil{m/n}]\right). \] 
The difference between the maximum and the minimum elements of $Y'$ is at most
$m$. Thus, $\posmod_n(\modm(Y'))$ takes at most two distinct values.
In particular this implies that $\posmod_n(\modm(Ya_0n))$ takes on at most two
values, so there is a subset of $Y$ with size at least $|Y|/2$ that all hash to
the same bin under  $a=\modm(n a_0)$.

  Assume $S\subset X$ is the set of elements mapping to a fullest bin under
$\strideZm$ with $a=a_0$. Then, by similar reasoning to the above case, for
$a=\modm(n^{-1}\cdot a_0)$ $\blockZm$ has maxload at least  $|S|/2$.

  Multiplication by $n$ or $n^{-1}$ modulo $m$ permutes $\Z_m$. The result follows.
\end{proof}

\cref{prop:blockZisok} and \cref{prop:blockZsucks} teach us that
for composite integer $\LH$ it is more robust to consider
$\blockZm$ than $\strideZm$, but essentially equivalent as long
as $\gcd(m,n)$ is small. For the remainder of the paper we
restrict our attention to $\blockZm$ and $\blockFp$ which we
abbreviate to $\ZH_m$, $\FH_p$.
Now we formally define the variant of $\ZH_m$ that partially
solves the problem of $m$ being composite.
\begin{defin}\label{defn:slh}
  In \defn{Smart $\LH$} ($\SLH_m$) we randomly select
  $a\in \Z_m^{\times }$ and place $x\in [m]$ in bin $\floor{\frac{\modm(ax)}{m/n}}.$
\end{defin}
Surprisingly, we will show that the performance of $\ZH_m$ is not
too far from that of $\SLH_m$, especially if $m$ has few divisors.
We use the following notation:

\begin{defin}
  Let random variable $M_{\ZH}(m,X)$ denote the maxload incurred by
  $\ZH_m$ on $X$, and let $M_{\ZH}(m,n)$ denote the worst-case expected value of
  $M_{\ZH}(m,X)$ over all $n$-element sets $X\subset [m]$.
  Analogously define $M_{\SLH}(m,X), M_{\SLH}(m,n)$.
\end{defin}

\begin{theorem}\label{thm:LHSLH}
  Fix $m\ge n^{6}$ with $m\in \poly(n)$.
  Let $f$ be a monotonically increasing concave function with
  $M_{\SLH}(m_0,n_0)\le f(n_0)$ for all $n_0$ and all $m_0\ge n^{2.5}$.
  Then
  \[
  M_{\ZH}(m,n) \le \tau(m) \cdot f(n)+1.
\]
\end{theorem}
\begin{proof}
  Fix any $n$-element set $X\subset [m]$.
  % Let $p\in \PRM$. Then, $$\Pr[\nu_p(a)=i] \le 1/p^i.$$
  For $d\mid m$, define $\setof{I_{i,d}}{i\in [d]}$ as the
  following partition of $[m]$ into $d$ size $m/d$ blocks:
  \[I_{i,d} = i\cdot m/d + [m/d].\]
  Define $X_{i,d} = X\cap I_{i,d}$
  and let $G_d$ be the event  $\gcd(a,m)=d$.
  We will bound the expected maxload of $\ZH_m$ by conditioning
  on $G_d$. However, if $\gcd(a,m)$ is very large then  $\ZH_m$ will
  necessarily incur large maxload; thus, we first exclude
  this case by showing it is very unlikely.
  For any $d\mid m$
  \begin{equation}\label{eq:prGd}
    \Pr[G_d]\le 1/d
  \end{equation}
  because there are $m/d$ multiples of $d$ in $[m]$.
  There are at most $n^{2.5}$ divisors $d\mid m$ with  $d\ge
    m/n^{2.5}$, because such divisors are in bijection with
  divisors $d'\mid m$ satisfying  $d'\le n^{2.5}$.
  By \cref{eq:prGd} each of these large divisors
  $d$ has $\Pr[G_d] \le n^{2.5}/m$. Thus we have:
  \begin{equation}\label{eq:gcdbignoway}
    \Pr[\gcd(a,m) \ge m/n^{2.5}] \le n^{2.5}\cdot \frac{n^{2.5}}{m} \le
    \frac{1}{n},
  \end{equation}
  where the final inequality follows by the assumption that $m\ge n^{6}$.
  By \cref{eq:gcdbignoway} the case $\gcd(m,a)\ge m/n^{2.5}$ contributes at most $1$ to
  the expected maxload.
  Let
  \[D_m = \setof{d\mid m}{d<m/n^{2.5}}.\]
  We claim the following chain of inequalities:

\begin{minipage}{0.4\textwidth}
\begin{align}
     &\E[M_{\ZH}(m,X) \mid \gcd(m,a)\in D_m] \nonumber \\
     &= \sum_{d\in D_m}\Pr[G_d]\cdot \E[M_{\ZH}(m,X) \mid G_d] \label{eqchainSLH1} \\
     &\le \sum_{d\in D_m} \frac{1}{d}\cdot \E[M_{\ZH}(m,X) \mid G_d] \label{eqchainSLH2} \\
     &\le \sum_{d\in D_m}\frac{1}{d}\sum_{i\in [d]}\E[M_{\ZH}(m,X_{i,d}) \mid G_d]\label{eqchainSLH3} \\
     &\le \sum_{d\in D_m}\frac{1}{d}\sum_{i\in [d]}\E[M_{\SLH}(m/d,X_{i,d})] \label{eqchainSLH4} \\
     &\le \sum_{d\in D_m}\frac{1}{d}\sum_{i\in
    [d]}M_{\SLH}(m/d,|X_{i,d}|) \label{eqchainSLH5} \\
     &\le \sum_{d\in D_m} \frac{1}{d}\sum_{i\in
    [d]}f(|X_{i,d}|) \label{eqchainSLH6} \\
     &\le \sum_{d\in D_m}f(n/d) \label{eqchainSLH7} \\
     &\le \tau(m)\cdot f(n) \label{eqchainSLH8}.
\end{align}
\end{minipage}%
\begin{minipage}{0.6\textwidth}
\begin{itemize}
    \item \cref{eqchainSLH1}: Law of total expectation.
    \item \cref{eqchainSLH2}: $\Pr[G_d]\le 1/d$, because there are $m/d$ multiples of $d$ in $[m]$.
    \item \cref{eqchainSLH3}: We can ``union bound''  because $\bigsqcup_{i\in [d]}X_{i,d} = X.$
    \item \cref{eqchainSLH4}: Recall that $X_{i,d}\subset I_{i,d}$, where
          $I_{i,d}$ is a contiguous interval of size $m/d$.
          Because we are conditioning on $\gcd(m,a)=d$,
          $\modm(a\cdot I_{i,d})$ consists of every $d$-th element of
          $[m]$ starting from $0$, i.e., is $\modm(d[m])$. Having
          elements which are spaced out by $d$ grouped into
          intervals of length $m/n$ per bin is
          equivalent to having elements spaced out by $1$
          grouped into intervals of length $(m/d)/n$ per
          bin. Formally this is because
          $\modm(x\cdot d\cdot j) = d\cdot \posmod_{m/d}(x\cdot j).$
          The restriction  $\gcd(m,a)=d$ can also be expressed as
          $\gcd(m/d, a/d)=1$, i.e., $a/d \perp m/d$.
          Hence, the expected maxload of $\ZH_m$ on $X_{i,d}$ conditional on
          $G_d$ is the same as the expected maxload of $\SLH_{m/d}$
          on $X_{i,d}$.
    \item \cref{eqchainSLH5}: $M_\SLH(m,n)$ is by definition the
          worst-case value of $\E[M_\SLH(m,X)]$ over all $n$-element
          sets $X$.
    \item \cref{eqchainSLH6}: By assumption $f$ is an upper
          bound on $M_{\SLH}$ as long as the modulus $m/d$ is
          sufficiently large. Because $d\in D_m$ we have  $m/d >
            n^{2.5}$, so the upper bound $f$ holds.
    \item \cref{eqchainSLH7}: $f$ is concave.
    \item \cref{eqchainSLH8}: $f$ is increasing, $\tau$ counts
          the divisors of $m$.
\end{itemize}
\end{minipage}

  % \begin{align}
  %    & \E[M_{\ZH}(m,X) \mid \gcd(m,a)\in D_m] \nonumber                                                \\
  %    & = \sum_{d\in D_m}\Pr[G_d]\cdot \E[M_{\ZH}(m,X) \mid G_d]
  %   \label{eqchainSLH1}                                                                                \\
  %    & \le \sum_{d\in D_m} \frac{1}{d}\cdot \E[M_{\ZH}(m,X) \mid G_d] \label{eqchainSLH2}              \\
  %    & \le \sum_{d\in D_m}\frac{1}{d}\sum_{i\in [d]}\E[M_{\ZH}(m,X_{i,d}) \mid G_d]\label{eqchainSLH3} \\
  %    & \le \sum_{d\in D_m}\frac{1}{d}\sum_{i\in [d]}\E[M_{\SLH}(m/d,X_{i,d})] \label{eqchainSLH4}      \\
  %    & \le \sum_{d\in D_m}\frac{1}{d}\sum_{i\in
  %   [d]}M_{\SLH}(m/d,|X_{i,d}|) \label{eqchainSLH5}                                                    \\
  %    & \le \sum_{d\in D_m} \frac{1}{d}\sum_{i\in
  %   [d]}f(|X_{i,d}|) \label{eqchainSLH6}                                                               \\
  %    & \le \sum_{d\in D_m}f(n/d) \label{eqchainSLH7}                                                   \\
  %    & \le \tau(m)\cdot f(n) \label{eqchainSLH8}.
  % \end{align}
  % We now justify the inequalities.
  % \begin{itemize}
  %   \item \cref{eqchainSLH1}: Law of total expectation.
  %   \item \cref{eqchainSLH2}: $\Pr[G_d]\le 1/d$, because there
  %         are $m/d$ multiples of $d$ in $[m]$.
  %   \item \cref{eqchainSLH3}: We can ``union bound''  because
  %         $$\bigsqcup_{i\in [d]}X_{i,d} = X.$$
  %   \item \cref{eqchainSLH4}: Recall that $X_{i,d}\subset I_{i,d}$, where
  %         $I_{i,d}$ is a contiguous interval of size $m/d$.
  %         Because we are conditioning on $\gcd(m,a)=d$,
  %         $\modm(a\cdot I_{i,d})$ consists of every $d$-th element of
  %         $[m]$ starting from $0$, i.e., is $\modm(d[m])$. Having
  %         elements which are spaced out by $d$ grouped into
  %         intervals of length $m/n$ per bin is
  %         equivalent to having elements spaced out by $1$
  %         grouped into intervals of length $(m/d)/n$ per
  %         bin. Formally this is because
  %         $$\modm(x\cdot d\cdot j) = d\cdot \posmod_{m/d}(x\cdot j).$$
  %         The restriction  $\gcd(m,a)=d$ can also be expressed as
  %         $\gcd(m/d, a/d)=1$, i.e., $a/d \perp m/d$.
  %         Hence, the expected maxload of $\ZH_m$ on $X_{i,d}$ conditional on
  %         $G_d$ is the same as the expected maxload of $\SLH_{m/d}$
  %         on $X_{i,d}$.
  %   \item \cref{eqchainSLH5}: $M_\SLH(m,n)$ is by definition the
  %         worst-case value of $\E[M_\SLH(m,X)]$ over all $n$-element
  %         sets $X$.
  %   \item \cref{eqchainSLH6}: By assumption $f$ is an upper
  %         bound on $M_{\SLH}$ as long as the modulus $m/d$ is
  %         sufficiently large. Because $d\in D_m$ we have  $m/d >
  %           n^{2.5}$, so the upper bound $f$ holds.
  %   \item \cref{eqchainSLH7}: $f$ is concave.
  %   \item \cref{eqchainSLH8}: $f$ is increasing, $\tau$ counts
  %         the divisors of $m$.
  % \end{itemize}

  We have shown the bound \cref{eqchainSLH8} for arbitrary
  $X$, so in particular the bound must hold for worst-case  $X$.
  Adding $1$ the for the event $\gcd(a,m)\ge m/n^{2.5}$ we have
  \[ M_{\ZH}(m,n) \le \tau(m)\cdot f(n)+1.\]
\end{proof}
\begin{rmk}
  \cref{thm:LHSLH} says that increasing concave bounds for
  $M_\SLH$ can be translated to bounds for $M_{\ZH}$ except
  weakened by a factor-of-$\tau(m)$.
  If $m$ is a power of $2$, a natural setting, then
  $\tau(m) = \log m$.
  Even for worst-case $m$ \cref{fact:numdivs} asserts $\tau(m)\le
    m^{o(1)}$. So, $\SLH_m$ and $\ZH_m$ have quite similar
  behavior.
\end{rmk}

Now we analyze the performance of $\SLH_m$. First we give an
argument based on the trivial $\bigO(\sqrt{n})$ bound for $\FH$.
% In \cref{sec:formalpfZm13} we apply similar modifications to
% Knudsen's proof \cite{knudsen_linear_2017} of the
% $\widetilde{\bigO}(n^{1/3})$ along with some new modifications
% specific to Knudsen's proof to translate his
% $\widetilde{\bigO}(n^{1/3})$ bound to $\SLH_m$.

% \begin{prop}
% \todo{this is probably not worth saying twice.}
%   \label{prop:sqrtnZ}
%   $\SLH$ on $\block{\Z_{2^{\ell}}}$ achieves expected maxload at most
%   $\bigO(\sqrt{n})$.
% \end{prop}
% \begin{proof}
%   Say that $x,y \in X, x\neq y$ are \defn{linked} if $\gcd(x-y,2^{\ell}) >
%   2^{\ell}/n$ and \defn{unlinked} otherwise.
%   If $x,y$ are linked, then they can't collide. This is because
%   if $\gcd(x-y,2^{\ell})=2^{j} > 2^{\ell}/n$ then $\gcd(a(x-y)\bmod
%   2^{\ell},2^{\ell}) = 2^{j}$ due to $a$ being odd  because we are doing $\SLH$, so $a$
%   is chosen relatively prime to $2^{\ell}.$
%   But this means that $ax-ay > 2^{\ell}/n$, i.e. $x,y$ fall in
%   separate bins.
%   If $x,y$ are unlinked, then they collide with probability
%   $\bigO(1 / n)$. In particular, $a(x-y)$ will range uniformly
%   over all the bins as long as $\gcd(x-y,2^{\ell}) < 2^{\ell}/n$,
%   so the probability of $x-y$ being sufficiently small is at most
%   say $\frac{2}{n}$.

%   Hence, the expected number of pairs which collide is, by
%   linearity of expectation, $\bigO(n^2 / n) =\bigO(n),$ 
%   meaning that the expected maxload is at most $\bigO(\sqrt{n})$
%   by Jensen's Inequality ($\sqrt{\cdot}$ is concave).
% \end{proof}
\begin{theorem} \label{prop:sqrtnZ}
  $M_{\SLH}(m,n) \le \bigO(\sqrt{n\log\log n}).$
\end{theorem}
\begin{proof}
  We say $x,y$ \defn{collide}, with respect to $a=a_0$,
  if they hash to the same bin for $a=a_0$.
  As in the $\bigO(\sqrt{n})$ bound for $\FH_p$ we bound the
  maxload by counting the expected number of collisions and
  comparing this with the number of collisions guaranteed by a
  certain maxload.
  The difficulty in the proof for $\SLH_m$ is that the
  probability of  $x,y\in X$ colliding is not as simple as in
  $\FH_p$ where all pairs collide with probability $\bigO(1/n)$.
  To handle this we consider two types of pairs $x,y$:
  \begin{defin}
    Distinct $x,y\in X$ are
    \defn{linked} if
    $\gcd(x-y,m) > \ceil{m/n},$
    and \defn{unlinked} otherwise.
  \end{defin}

  \begin{claim}\label{clm:linkedcollidesqrt}
    Linked $x,y$ never collide.
  \end{claim}
  \begin{proof}
    If $\gcd(x-y,m)=k > \ceil{m/n}$ and $a$ is the randomly chosen
    multiplier from $\Z_m^{\times}$ then
    \[
    \gcd(\modm(a\cdot (x-y)),m) = k
  \]
    because $a\perp m$.
    But then
    $\circabs_m(ax - ay) \ge k > \ceil{m/n},$
    so $x,y$ fall in different bins.
  \end{proof}
  \begin{claim}\label{clm:unlinkedcollidesqrt}
    Unlinked $x,y$ collide with probability at most
    $\bigO\left(\frac{\log\log n}{n}\right).$
  \end{claim}
  \begin{proof}
    When $x,y$ are unlinked, $\modm(a(x-y))$ would have probability $\bigO(1/n)$
    of landing in each bin if $a$ were chosen uniformly from $\Z_m$.
    However, in $\SLH_m$ this uniformity is not obvious.
    Fortunately, \cref{fact:toitent} ensures that each $a\in
      \Z_m^{\times}$ occurs with probability at most $\frac{2\log\log
        m}{m}$, which is not much larger than $\frac{1}{m}$.
    In particular, this implies that the probability of $x,y$
    colliding is at most
    \[
    \bigO(1/n)\cdot m \cdot \frac{2\log\log m}{ m} \le
      \bigO\left(\frac{\log\log n}{n}\right).
    \]
  \end{proof}

  Now that we have shown
  \cref{clm:linkedcollidesqrt} and \cref{clm:unlinkedcollidesqrt}
  the proof continues in the same
  way as for $\FH_p$.
  If maxload is $m$ then there must be at least
  $\binom{m}{2} = \Theta(m^{2})$ collisions.
  By Jensen's inequality and the convexity of $x\mapsto x^2$ we
  have $\E[M_{\SLH}(m,X)^2]\ge \E[M_{\SLH}(m,X)]^2.$
  We can also count the expected number of collisions directly
  using \cref{clm:linkedcollidesqrt} and \cref{clm:unlinkedcollidesqrt};
  doing so, we find that the expected number of collisions is
  $\bigO(n \log\log n).$
  Comparing our two methods of counting collisions gives:
  \[
  \E[M_{\SLH}(m,X)]\le \bigO(\sqrt{n\log\log n})
\]
  for any $X$, and in particular for worst-case $X$.
\end{proof}

In \cref{sec:formalpfZm13} we strengthen \cref{prop:sqrtnZ} to 
\begin{theorem}\label{thm:Zm1_3}
  $M_{\SLH}(m,n)\le \widetilde{\bigO}(n^{1/3}).$
\end{theorem}
The proof is a modification of Knudsen's proof
\cite{knudsen_linear_2017} of the corresponding bound for
$\FH_p$ with modifications similar to those used in \cref{prop:sqrtnZ}.

% The proof is 
% Now we strengthen \cref{prop:sqrtnZ} to \cref{thm:Zm1_3}.
% Most of the proof of \cref{thm:Zm1_3} is the same as
% Knudsen's proof \cite{knudsen_linear_2017} for $\FH_p$, but it is difficult to
% black-box Knudsen's result because our modifications permeate
% his whole proof. Thus, we highlight here only the key
% modifications needed to adapt the argument to $\SLH_m$,
% and provide a formal proof in \cref{sec:formalpfZm13}.

% \begin{theorem}\label{thm:Zm1_3}
%   $M_{\SLH}(m,n)\le \widetilde{\bigO}(n^{1/3}).$
% \end{theorem}
% \begin{proof}[Sketch of key modifications]
%   The most important idea for translating
%   Knudsen's proof to composite moduli is
%   already included in the statement of \cref{thm:Zm1_3}: namely,
%   the idea to use $\SLH_m$ rather than $\ZH_m$.
%   This decision is justified by \cref{thm:LHSLH}.
%   Crucially in $\SLH_m$ all choices of $a$ have inverses, a
%   property relied on extensively in Knudsen's proof.

%   In Knudsen's proof he considers a refinement of the collisions
%   used in \cref{prop:sqrtnZ} called \defn{close pairs}. The
%   second key idea to translate Knudsen's proof to $\ZH_m$ is
%   that, with a little case-work into ``linked'' and ``unlinked''
%   pairs like in \cref{prop:sqrtnZ}, we can bound the expected
%   number of close pairs in a similar fashion as in the proof for
%   $\FH_p$.

%   The other required changes are smaller, and generally look like
%   using number theoretic facts such as
%   \cref{fact:numdivs}, \cref{fact:toitent}, and the prime number
%   theorem to bound the ``degeneracy'' of $\Z_m$.
% \end{proof}

\begin{cor}\label{cor:translate}
  $M_{\ZH}(m,n) \le n^{1/3 + o(1)}.$
\end{cor}
\begin{proof}
  This follows immediately from using \cref{thm:Zm1_3} in
  \cref{thm:LHSLH}, which is valid because $n^{1/3}$ is a concave
  and increasing function of $n$.
  \footnote{In general one needs to slightly modify the universe size
    in order to apply \cref{thm:LHSLH}. However, \cref{thm:Zm1_3}
    as proved in \cref{sec:formalpfZm13} only requires the universe
    size $m> n^{2.5}$. Thus, such a modification is not necessary here.}
\end{proof}

% In \cref{thm:Zm1_3} we classified pairs $x,y\in X$ into two groups:
% \begin{itemize}
%   \item $x,y$ are ``linked'' if $\gcd(x-y,m)$ is large. Linked
%     $x,y$ never land in the same bin.
%   \item $x,y$ are ``unlinked'' if $\gcd(x-y,m)$ is small. Unlinked
%     pairs intuitively behave like any pair relative to a prime
%     modulus. In particular, for most unlinked pairs, $x,y$ will
%     be end up in uniformly random and independent bins.
% \end{itemize}
% It seems plausible that the concept of linked and unlinked pairs
% could be used to prove the following generalization of
% \cref{thm:Zm1_3}:
In \cref{thm:Zm1_3} we have translated the state-of-the-art maxload
bound for $\FH_p$ to $\SLH_m$ by altering Knudsen's proof.
This is tentative evidence that composite modulus $\LH$ may
achieve similar maxload to prime modulus $\LH$ in general.
We leave proving or refuting this as an open problem:
\begin{question}\label{question:equivalenceFZ}
  Are the worst-case maxloads of $\FH_p$ and $\SLH_m$ the same
  up to a factor-of-$n^{o(1)}$?
\end{question}
% \cref{conj:translate} seems hard, because rather than translating
% a specific bound we must show a general reduction; we leave this
% as an open problem.
% Combined with \cref{thm:LHSLH}, \cref{conj:translate} would
% imply that maxload relative to composite moduli and prime moduli
% is essentially the same.
% However, it is not obvious how to apply the linked and unlinked
% pair analysis
