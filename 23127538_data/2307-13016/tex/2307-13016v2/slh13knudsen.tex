% \section{Blocking and Striding for $\F_{p}$}
% \label{sec:F}

% In this section we provide a proof of the fact that $\blockFp$
% and $\strideFp$ have the same worst-case expected maxload, up to
% a factor-of-$2$. 
% The ideas for this proof are due to Knudsen \cite{knudsen_linear_2017}. 

% \begin{prop}
%   \label{prop:blockstrideF}
%   Fix $p\in \PRM,n\in \N, X\subset [p]$ with $|X|=n.$
%   The expected maxload of $\strideFp$ and
%   $\blockFp$ on $X$ differ by at most a factor-of-$2$.
% \end{prop}
% \begin{proof}
%   For $a_0\in \pnozero$, let $M$ be the maxload incurred by
%   $\blockFp$ when using $a=a_0.$ There must be 
%   a set of $M$ elements $Y\subset X$ and a bin $k\in [n]$
%   such that all $y\in Y$ hash to bin $k$ (for $a=a_0$). 
%   In other words, for each $y\in Y$ we
%   have $$\frac{kp}{n} \leq \modp(ya_0) < \frac{(k+1)p}{n}.$$
%   Let $i_y \in [\floor{p/n}]$ denote $\modp(ya_0)-kp/n$. Then,
%   $$n\cdot \modp(y a_0) = kp + i_y n.$$ 
%   Let $a_1=\modp(n\cdot a_0)$.
%   It is tempting to say that all $y\in Y$ map to the same bin
%   under $\strideFp$ with $a=a_1$, but this is not necessarily
%   quite true, because the wrap-around disturbs relative
%   distances. Fortunately, this wrap-around induced distortion
%   occurs at most once, because $i_y< \frac{p}{n}$. 
%   Thus, at least one half of the elements of $Y$, call it $Y'$, is on a single
%   side of the wrap-around. All $y'\in Y'$ lie in the same
%   bin under $\strideFp$ with  $a_1$, namely bin $\posmod_n(\modp(y' a_1))$.
%   Multiplication by $n$ permutes $[p]$, so our analysis shows
%   that the expected maxload of $\strideFp$ is at least half
%   that of $\blockFp$.

%    On the other hand, consider $a_0' \in [p]$ so that
%    $\strideFp$ incurs  maxload  $M$ when using $a=a_0'.$
%    Then there must be a set of $M$ elements $S\subset X$ and a bin
%    $c\in [n]$ such that for each $s\in S$
%    $$\modp(a_0' s) = j_s n + c,$$ for some $j_s \in [\floor{p/n}]$. 
%    Now, for $a_1'=\modp(n^{-1}\cdot a_0')$, where $n^{-1}$
%    is the multiplicative inverse of $n$ in $\F_p$, we have
%     $$\modp(a_1's) = \modp(n^{-1}a_0's) = j_s + n^{-1}c.$$ 
%     These elements lie in an interval of size $p/n$, and so in
%     the worst case are split across two bins. Thus, for $a=a_1'$
%     $\blockFp$ incurs maxload at least $M/2$.
%     As before, multiplication by $n^{-1}$ permutes $[p]$, so the
%     expected maxload of $\blockFp$ is at least half that of
%     $\strideFp$.
   
% \end{proof}



\section{Full proof of \cref{thm:Zm1_3}}
\label{sec:formalpfZm13}

In this section we provide the full proof of \cref{thm:Zm1_3},
translating Knudsen's elegant $\widetilde{\bigO}(n^{1/3})$ bound
\cite{knudsen_linear_2017} from $\FH_p$ to the $\SLH_m$.
Although Knudsen's proof only requires relatively small
modifications it is difficult to black-box Knudsen's
result because our modifications permeate the whole proof. Thus,
for sake of formality we provide a full proof here.
\begin{theorem}
  Fix $m>n^{2.5}$ with $m\in \poly(n)$.\footnote{This is slightly
  different from the convention in the rest of the paper
(\cref{rmk:assumesize}) that $m>n^{6}$. For elegance of
presentation in applying this theorem, and because it does not
add any complexity to the proof, we prove this theorem with this
smaller universe size.}
$$M_{\SLH}(m,n)\le \widetilde{\bigO}(n^{1/3}).$$
\end{theorem}
\begin{proof}[Proof of \cref{thm:Zm1_3}]
  Fix $n$-element set $X\subset [m]$. We abbreviate $M_{\SLH}(m,X)$ to $M$.
  Fix integer $\alpha < n/4$, and let $\eps = \Pr[M > 4\alpha]$.
  We will show that if $\alpha$ is sufficiently large, in
  particular $\alpha > \Omega(n^{1/3}\log n)$, then $\eps$ is
  very small. Define $\mathcal{A}\subset \Z_m^{\times }$ to be
  the set of \defn{bad} $a$'s, i.e., values of $a$ which make the
  maxload exceed $4\alpha.$ Clearly $|\mathcal{A}| /
  |\Z_m^{\times}| = \eps$.

  Throughout this proof we will write $x^{-1}$ to mean the
  multiplicative inverse of $x$ modulo $m$.

  Define a \defn{pre-bin} to be the pre-image of a bin, i.e., 
  $((m/n)\cdot [k, k+1])\cap [m]$ for some $k\in \Z_{\ge 0}$.
  We say that $a_1,a_2 \in \Z_m^{\times}$ are
  \defn{close} if $$\circabs_m(a_1 - a_2) \le
  \frac{m}{n\alpha}.$$ 
  Let $w = \floor{\frac{m}{n\alpha}}$. We call an interval of
  length $w$ a \defn{tiny} interval.
  To bound the number of bad $a$'s we analyze the number of
  \defn{close pairs} (i.e., pairs $a_1,a_2$ which are close). 
  Closeness is a refinement of the
  property of lying within the same pre-bin, which we used in
  \cref{prop:sqrtnZ}. Intuitively, close
  pairs lie within a $(1/\alpha)$-fraction contiguous portion of
  a pre-bin, although technically a bin boundary may split a set
  of close pairs into two pre-bins. In fact, we will count close
  pairs by analyzing groups of $\alpha$ adjacent intervals each
  of size $w$; in other words, each group will be
  a partition of a pre-bin into small intervals.
  By convention close pairs are unordered, i.e., we do not count
  $b,c$ and $c,b$ as distinct close pairs.
 
 Now we provide intuition helpful for counting the number of
 elements which are close to some fixed element $b \in (\alpha,
 2\alpha) \cap \Z_m^{\times}$. 
 \begin{claim}\label{clm:prebintiny}
   Assume $b\in \Z_m^{\times}\cap (\alpha, 2\alpha).$\footnote{We will show later that
   this interval is non-empty for the relevant values of
 $\alpha$.}
   Let $I$ be a pre-bin. Then $\modm(b^{-1}\cdot I)$ is
   contained in the union of $b$ tiny intervals. 
 \end{claim}
 \begin{proof}
For each $j\in [w]$ we have
$$b^{-1}\cdot (j+[w]b) \equiv b^{-1}j + [w]  \mod m.$$
In other words, we can partition $x\in I$ based on
$\posmod_b(x)$; $x,y\in I$ for which $x\equiv y \bmod b$ are close
after multiplication by $b^{-1}$.
Note that many of these intervals may be empty; i.e., it is
possible that $\modm(b^{-1}\cdot I)$ is contained in
substantially less than $b$ tiny
intervals, but in general we cannot give a stronger bound.
 \end{proof}
 \begin{claim}\label{clm:oneBclosePairs}
   Let $b\in \Z_m^{\times}\cap (\alpha, 2\alpha).$
Say that for some pre-bin $I_b$ the set $\modm(b\cdot X) \cap
I_b$ contains at least $4\alpha$ elements. 
Then $X$ contains at least $\Omega(\alpha)$ close pairs.
 \end{claim}
 \begin{proof}
By \cref{clm:prebintiny} the $4\alpha$ elements of $\modm(b\cdot
X)\cap I_b$ are distributed amongst $b\le 2\alpha$ tiny intervals.
Any elements within the same tiny interval constitute a close
pair. The number of close pairs we obtain from splitting these
$4\alpha$ elements amongst $b$ tiny intervals is
minimized if we distribute the $4\alpha$ elements evenly amongst
the tiny intervals. 
However, even if the elements are distributed evenly we still
have at least $2$ elements per interval, and thus $\Omega(\alpha)$
total close pairs.
We remark that if we had defined a tiny interval to be any
smaller then this argument would not guarantee us to have any
close pairs.
 \end{proof}

Now we explore the relationship between close pairs and maxload.
For each bad $a$ there is some pre-bin $I_{a}$ which
contains at least $4\alpha$ elements of  $\modm(a X)$.
By \cref{clm:oneBclosePairs}, this gives us at least $\alpha$
close pairs in $X$. 
Furthermore, we will show in \cref{lem:closepairs} that among
bad $a$'s in a suitably chosen subset $B\subset \mathcal{A}$, the close
pairs given by each $a\in B$ are fairly disjoint.
Intuitively this means that the more bad $a$'s there are, the
more close pairs there are. 
Similarly to in \cref{prop:sqrtnZ}, we will
conclude by counting the close pairs with a different
method to show that too-large maxload results in too many close pairs.

  We proceed to formalize this reasoning.
  Let $$U = \Z_m^\times \cap \PRM \cap (\alpha, 2\alpha).$$ 
  The multiplier $a\gets \Z_m^{\times }$ is a random variable.
  Define
  $$B= U\cap \modm(a^{-1}\mathcal{A});$$ $B$ is a random variable
  dependent on $a$.

  \begin{claim}\label{clm:sizeB}
    $\E[|B|] \geq \Omega\paren{\frac{\alpha}{\log \alpha} - \log
    n}\cdot \eps.$
  \end{claim}
  \begin{proof}
    The prime number theorem says that there are at least 
    $\Omega\left(\alpha/\log \alpha\right)$ primes in the interval
    $(\alpha, 2\alpha)$.
    On the other hand, $m$ cannot have more than $\log m$
    distinct prime divisors. 
    Hence, by excluding the prime divisors of $m$ from $\PRM \cap
    (\alpha, 2\alpha)$  we find:
    \begin{equation}
      \label{eq:ubound}
  |U| \geq \Omega\left(\frac{\alpha}{\log \alpha} - \log n\right).
    \end{equation}

    For any unit, and in particular for any bad $a_0\in
    \mathcal{A}$, $\modm(a^{-1}a_0)$ is uniformly random in
    $\Z_m^{\times}$. Thus,
    $$\Pr[\modm(a^{-1}a_0)\in U]  = |U| / |\Z_m^{\times }|.$$
     There are $|\mathcal{A}|=|\Z_m^{\times }|\eps$ bad $a_0$'s, so by linearity of
    expectation we have
    $$\E[|B|] \geq |\Z_m^{\times }|\eps |U|/|\Z_m^{\times }| =
    |U|\eps,$$
    which combined with \cref{eq:ubound} gives the desired
    result.
  \end{proof}

  \begin{claim}
    \label{clm:ezclosepairs}
    The expected number of close pairs is at most
    $\bigO(\frac{n}{\alpha}\log\log n)$.
  \end{claim}
  \begin{proof}
    Similar to the proof of \cref{prop:sqrtnZ}, we define
    \defn{linked} and \defn{unlinked} pairs. 
    We say that distinct $x,y\in X$ are linked if $\gcd(x-y, m)
    > w$ and unlinked otherwise. 
    If $x,y$ are linked, they cannot be closed by virtue of being distance at
    least $w$ apart. For unlinked $x,y$ 
    $\floor{\modm(a(x-y))/w}$ would be uniformly distributed on
    $[\floor{m/w}]$ if $a$ were chosen randomly from $\Z_m$.
    Using \cref{fact:toitent} we find the probability of $x,y$
    being close when $a\gets \Z_m^{\times}$ is at most 
    $\bigO\paren{\frac{\log\log n}{n\alpha}}.$
    There are $\binom{n}{2}$ total pairs. Then, by linearity of
    expectation there are $\bigO\left(\frac{n}{\alpha}\log\log
      n\right)$
    expected close pairs. 
  \end{proof}

  Now we establish the key combinatorial lemma:
  \begin{lemma}
    \label{lem:closepairs}
    There are at least $|B|\alpha/2$ close pairs.
  \end{lemma}
  \begin{proof}
    % each Ib yields some close pairs
    % Ib intersect Ic is small
    % cauchy shwarz

    Fix $a$. By definition of $B$, each $b\in B$ can be expressed
    as $b=\modm(a^{-1}a_0)$ for some bad $a_0\in
  \mathcal{A}$. By definition of $a_0$ being bad there exists
  a pre-bin $I_{b}$ (which of course has size $|I_{b}|=m/n$)
  such that at least $4\alpha$ elements from $X$ fall in $I_b$
  under $a_0$. I.e., 
  \begin{equation}\label{eq:a0isbadandsobisgood}
  |I_{b}\cap \modm(a_{0}X)| = |I_b  \cap \modm(baX)| > 4\alpha.
  \end{equation}
  Recall \cref{clm:prebintiny},\cref{clm:oneBclosePairs}:
  multiplication by $b^{-1}$ ``fractures" the interval $I_b$ into
  at most $b$ tiny intervals, and using these tiny intervals we
  can obtain $\Omega(\alpha)$ close pairs. Now we analyze the
  overlap of the close pairs given by different $b\in B$.
  % , which assert
  % $b^{-1}I_b$ can be written as the disjoint union of intervals
  % $\setof{I_{b,j}}{j\in[b]}$ where each such interval is small,
  % i.e. $|I_{b,j}| < \frac{p}{b n}$. 
  % Recall that this is because multiplying by $b^{-1}$ fractures
  % the interval,  i.e. after going forward $b$ steps
  % you loop back around to about the same place under the map
  % $x\mapsto b^{-1}x \mod p$. By design, $I_{b,j}$ is small enough
  % that if $a,a'\in I_{b,j}$  then $a,a'$ is a close pair.
  % Recall from before that for each $b\in B$ this will give
  % $\Omega(\alpha)$ close pairs. 

  % However, this doesn't immediately conclude the proof, because
  % there may be overlap, i.e. apriori the close pairs obtained
  % from $b,c\in B, b\neq c$ may all be the same, or overlap
  % significantly. However, we claim that in fact there is not too
  % much overlap.  
  \begin{claim}\label{clm:IBCone}
   For distinct $b,c\in B$ and pre-bins $I_b,I_c$ the set
   $$\modm(b^{-1} I_{b})\cap \modm(c^{-1}I_c)$$ 
   is contained in a single tiny interval.
  \end{claim}
  \begin{proof}
   % Assume that for some $i\in I_b,j\in I_c$
   % \begin{equation}\label{eq:notaneq}
   % b^{-1}i \equiv c^{-1}j \mod m.
   % \end{equation}
   % For any $\delta\in \Z$ we have
   % \begin{align}
   %   b^{-1}\cdot(i+b\delta) &\equiv b^{-1}i+\delta\mod
   %   m\label{eq:onetinyinterval1}\\
   % &\equiv c^{-1}j+\delta \equiv c^{-1}\cdot (j+c\delta)
   % \label{eq:onetinyinterval2}.
   % \end{align}
   %  If we require $b\delta \le |I_b| \le m/n$ then there are at
   %  most $(m/n)/b$ values $\delta$ for which
   %  $i+b\delta\in I_b$. Similarly, there are
   %  at most  $(m/n)/c$ values for which
   %  $j+c\delta \in I_c$. 
   %  Because $b,c> \alpha$ we see that there are at most
   %  $(m/n)/\alpha$ such $\delta$. For all $\delta < (m/n)/\alpha,
   %  \delta\in \Z_{\ge 0}$ the expressions
   %  \cref{eq:onetinyinterval1}, \cref{eq:onetinyinterval2} are in
   %  the same tiny interval as \cref{eq:notaneq}.
   % Taken together, these elements are contained in a tiny
   % interval.
    Without loss of generality let $c<b$.
    For some $\delta_1,\delta_2\in \Z$ we can write 
    $$I_b = [\floor{m/n}]+\delta_1,I_c=[\floor{m/n}]+\delta_2.$$
    Thus, to understand $\modm(b^{-1}I_b) \cap \modm(c^{-1}I_c)$
    we may equivalently study 
    \begin{equation}\label{eq:onetinyintervalstudy}
    \paren{\modm(bc^{-1})[m/n]} \cap \paren{[m/n]+\delta}
    \end{equation}
    for $\delta\in \Z$.
    The fact that $b,c\in (\alpha, 2\alpha)\cap \PRM$, i.e., are
    co-prime and of similar size, strongly restricts the behavior
   of $\modm(bc^{-1})$. In particular there exists $\lambda \in \N$: 
   $$\lambda m + b \equiv 0 \mod c.$$
   For this value of $\lambda$,
    $$\frac{\lambda m+b}{c} \cdot c \equiv b \mod m.$$
    In other words, 
    $$\modm(bc^{-1}) = \frac{\lambda m + b}{c}.$$

    Now, we use this formulation of $\modm(bc^{-1})$ to study
    \cref{eq:onetinyintervalstudy}.
    Because $c\perp m$, there is a permutation $\pi$ of $[c]$ so
    that 
    \begin{equation}\label{eq:permutemup}
    \modm(j \cdot \lambda m / c) = \pi_j m/c.
    \end{equation}
    % visits some
    % permutation $\pi$ of $[c] m/c$ as $j$ ranges over $[c]$.
    % $$\frac{\lambda m}{c}[c] \equiv \frac{m}{c}[c]  \mod m.$$
    Because $b<2\alpha$, \cref{eq:permutemup} implies that for $j\in [c]$ 
    \begin{equation}\label{eq:stuffisnice}
    \abs{\modm\left(\frac{\lambda m + b}{c} j\right) - \pi_j
    \cdot\frac{m}{c}} < 2\alpha.
    \end{equation}
    Intuitively, because $m/c$ is much larger $\alpha$,
    \cref{eq:stuffisnice} means that for any $\ell$ the values
    $\modm(bc^{-1})(\ell+[c])$ are essentially spaced out by
    $m/c$. 
    Formally, by our assumption $m>n^{2.5}$ we have:
    \begin{equation}\label{eq:yupnicenice}
    m/c - 4\alpha \ge 2m/n - n > m/n.
    \end{equation}
    
    \cref{eq:stuffisnice} combined with \cref{eq:yupnicenice}
    is a good start to addressing \cref{eq:onetinyintervalstudy},
    showing that if $c$ consecutive numbers are multiplied by
    $\modm(bc^{-1})$ at most one lies in the interval
    $[m/n]+\delta$.
    To finish we analyze numbers which differ by a multiple of
    $c$ within the same interval.
    For any $k \in [m/(nc)]$ we have
    \begin{equation}\label{eq:globalbxnice}
    \frac{\lambda m+b}{c} ck \equiv kb \mod m.
    \end{equation}
    In particular, $kb \le (m/n)\cdot (c/b)<m/n$ by our
    assumption $c<b$.
    Thus, if $x,y \in [m/n]$ differ by a multiple of
    $c$ then they lie in the same interval of size $m/n$.
    Combining \cref{eq:globalbxnice} with \cref{eq:stuffisnice}
    and \cref{eq:yupnicenice} there are at most $(m/n)/b$ points
    in the intersection \cref{eq:onetinyintervalstudy}.


    We have described the shape of $bc^{-1}[m/n]$: it consists of
    $c$ well-separated concentrated intervals of length $m/n$
    with at most $\frac{m}{nc}$ elements per each such
    concentrated interval.
    Thus, upon intersection with $[m/n]$ we obtain at most
    $\frac{m}{nc}$ points. Let $\rho<
    \frac{m}{nc}<\frac{m}{n\alpha}$ be the number of points.
    In particular, the points will be of the form 
    \begin{equation}\label{eq:rhorhopoints}
    \delta', \delta' + b,\ldots, \delta' + b \rho.
    \end{equation}
    Multiplying by $b^{-1}$ to translate points
    \cref{eq:rhorhopoints} in the
    intersection \cref{eq:onetinyintervalstudy} to the
    intersection from the claim statement we find
    \begin{align*}
      &\modm(b^{-1}I_b) \cap \modm(c^{-1}I_c) \\
      &=\modm(b^{-1}\delta'), \modm(b^{-1}(\delta'+b)), \ldots,
      \modm(b^{-1}(\delta'+b\rho))\\
      &=\modm(b^{-1}\delta'), \modm(b^{-1}\delta')+1, \ldots,
      \modm(b^{-1}\delta')+\rho.
    \end{align*}
    Because $\rho < \frac{m}{n\alpha}$ all these points are
    contained in a single tiny interval, as claimed.

    % Interchanging the roles of $b,c$ in our above analysis, we
    % get that at most $\frac{m}{nb}$ points are contained in the
    % intersection.
    % These 
    % $$\min\left(\frac{m}{nc},\frac{m}{nb}\right) \le \frac{m}{n\alpha}$$
    % points exactly correspond to a single tiny interval in $b^{-1}I_b\cap
    % c^{-1}I_c$, by the argument at the beginning of the proof of
    % this claim. 
    % Alternatively, it is straightforward to see directly from our
    % analysis of the shape of $bc^{-1}[m/n]$ that $b^{-1}I_b \cap
    % c^{-1}I_c$ will be contained in a single tiny interval.
  \end{proof}

  Now we use \cref{clm:IBCone} to show that the close pairs given
  by each $b\in B$ are mostly disjoint.
  \begin{claim}
    There are at least $|B|\alpha/2$ close pairs.
  \end{claim}
  \begin{proof}
    For $b \in B, j\in [b]$ let $I_{b,j}$ denote our partition of
  $\modm(b^{-1}I_b)$ into tiny intervals as described in
  \cref{clm:prebintiny}. In particular, each $I_{b,j}$
  is a tiny interval, and 
  $$\bigsqcup_{j\in [b]} I_{b,j} = b^{-1}I_b.$$
  For each $b\in B,j\in [b]$ let $\psi(b,j)$ denote the number of
  $c\in B$ such  that $I_{b,j}\cap c^{-1}I_{c}\neq
  \varnothing$. Note that $\psi(b,j)\geq 1$ because
  $I_{b,j}\cap c^{-1}I_c\neq \varnothing$ for $c=b$.
  On the other hand, for each $b\neq c$ the intersection
  $b^{-1}I_b\cap c^{-1}I_c$ consists of at most a single tiny
  interval by \cref{clm:IBCone}. Therefore,
  \begin{equation}
    \label{eq:deltasum}
  \sum_{j\in [b]} \psi(b,j) < |B| + b \leq 3\alpha.
  \end{equation}
  Define 
  $$\xi(b,j) = \max(0, |aX\cap I_{b,j}|-1).$$
  Recall that any two elements in the same tiny interval are
  close. Thus, the number of close pairs is at least the sum over
  all tiny intervals $I$ of 
  $$\binom{|I|}{2}\ge \frac{1}{2}\cdot (|I|-1)^2.$$
  $\psi(b,j)$ is the number of times which interval $I_{b,j}$
  is counted. Thus, the number of close pairs is at least
  \begin{equation}
    \frac{1}{2}\sum_{b\in B}\sum_{j\in [b]}
    \frac{\xi(b,j)^2}{\psi(b,j)}\label{eq:xixi}
  \end{equation}
  because the $1/\psi(b,j)$ factor handles the fact that the
  interval $I_{b,j}$ occurs $\psi(b,j)$ times in the sum.
  Using the Cauchy-Shwarz Inequality on the inner sum of
  \eqref{eq:xixi} gives: 
  \begin{equation}
    \label{eq:cauchy}
    \sum_{j\in [b]} \frac{\xi(b,j)^2}{\psi(b,j)} \geq
  \frac{\paren{\sum_j\xi(b,j)}^2}{\sum_j \psi(b,j)}.
  \end{equation}
  In \cref{eq:a0isbadandsobisgood} we showed
  $$|I_b \cap ab X| = |b^{-1}I_b \cap aX| > 4\alpha.$$
  Thus, we can bound the numerator of \eqref{eq:cauchy} by 
  \begin{equation}\label{eq:numerbound}
  \paren{\sum_{j\in [b]} \xi(b,j)}^2 \ge \paren{4\alpha - b}^2 \geq
  \paren{2\alpha}^2.
  \end{equation}
  Combining \cref{eq:numerbound} and \cref{eq:deltasum}, which
  bound the numerator and denominator respectively of
  \eqref{eq:cauchy}, we obtain 
  \begin{equation}\label{eq:boundinside}
  \sum_{j\in [b]} \frac{\xi(b,j)^2}{\psi(b,j)} \ge
  \frac{4\alpha^2}{3\alpha}\ge \alpha.
  \end{equation}
  Applying \cref{eq:boundinside} to \cref{eq:xixi}, we find
  that the number of close pairs is at least $\alpha
  |B|/2$, as desired.
  \end{proof}
  \end{proof}

  \begin{cor}\label{cor:finisher}
  $$\eps < \bigO\paren{\frac{n\log\log n}{\alpha^2( \alpha / \log
  \alpha - \log n )}}. $$
  \end{cor}
  \begin{proof}
    \cref{lem:closepairs} gives a lower bound on the number of
    close pairs: there are at least $|B|\alpha/2$ close pairs.
    \cref{clm:sizeB} gives a lower bound on $\E[|B|]$. 
    \cref{clm:ezclosepairs} gives an upper bound on the number of
    close pairs. Comparing our upper bound and lower bound gives the
    desired inequality for $\eps.$
  \end{proof}

  Finally, we use \cref{cor:finisher} to conclude the proof.
  \begin{cor}
    $$M_{\SLH}(m,n)\le \widetilde{\bigO}(n^{1/3}).$$
  \end{cor}
  \begin{proof} Fix any $X$. Let random variable $M$ denote the
    maxload of $X$. 
    The expression in \cref{cor:finisher} is slightly
    complicated. For $\alpha > n^{1/3}\log n$ we can perform the
    following simplification:
    $$\frac{1}{\alpha/\log \alpha - \log n} < \bigO\left(\frac{\log
    n}{\alpha}\right)$$
    which is true because
    $$\frac{\log n}{\log \alpha}\alpha - \log^2 n >
    \Omega(\alpha).$$
    Thus, for $k > n^{1/3}\log n$ we have
    $$\Pr[M\ge k]\le \bigO\left( \frac{n \log n\log\log n}{k^3}
    \right).$$
    Now we bound $\E[M]$ as follows:
  \begin{align*}
    \E[M] &\leq \sum_{k\ge 0} \Pr[M \geq k] \\
        &\leq \widetilde{\bigO}(n^{1/3}) +
        \bigO\paren{\sum_{k>n^{1/3}\log
        n} \frac{n \log n\log\log n}{k^3}}.
\end{align*}
A basic fact of calculus is that 
$$\sum_{k > n^{1/3}\log n} \frac{1}{k^3} \le
\bigO\paren{\frac{1}{n^{2/3}\log^2 n}}.$$

Thus, our bound for $\E[M]$ simplifies to 
$$\E[M] \le \widetilde{\bigO}(n^{1/3}).$$
This bound was for arbitrary $X$, and thus also holds for
worst-case $X$.

  \end{proof}

\end{proof}
