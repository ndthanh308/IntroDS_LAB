\documentclass[twocolumn]{article}[11pt]
\author{Alek Westover}
% \title{Linear Hashing isn't too Picky:\\ For $\R$eal, No Shift, it's Ama$\Z$ing}
\title{Linear Hashing: No Shift, Non-Prime Modulus, For $\R$eal!}
% \usepackage[left=1in, right=1in, top=1in, bottom=1in]{geometry}
% \usepackage[subtle]{savetrees}
% \usepackage[left=.5in, right=.5in, top=.5in, bottom=.5in]{geometry}

\usepackage{amsthm}
\usepackage{amssymb}
\usepackage{amsmath}
\usepackage{mathtools}
\usepackage{csquotes}
\usepackage{enumitem}
\usepackage{svg}
\usepackage{hyperref}
\usepackage{xspace}
% \usepackage{xcolor}
% \usepackage{fancyhdr}

% \setlist[itemize]{leftmargin=*}
% \setlist{nosep}
% \setlist{nolistsep}

% \pagestyle{fancy}
% \lhead{Alek Westover}
% \rhead{}

% \fancyhead{}
% \fancyhead[L]{Alek Westover}
% \fancyfoot{}
% \fancyfoot[R]{\thepage}
% \renewcommand{\headrulewidth}{0pt}

\newcommand{\defn}[1]{{\textit{\textbf{\boldmath #1}}}\xspace}
\renewcommand{\paragraph}[1]{\vspace{0.09in}\noindent{\bf \boldmath #1.}} 
\newcommand{\todo}[1]{{\color{red}\textbf{TODO:} #1}}
\newcommand{\blue}[1]{{\color{blue} #1}}

% \usepackage{bbm}

% \hypersetup{
%     colorlinks,
%     linkcolor={red!50!black},
%     citecolor={blue!100!black},
%     urlcolor={black!80!black}
% }

\newcommand{\LH}{\mathsf{LH}}
\newcommand{\ZH}{\Z\mathsf{H}}
\newcommand{\FH}{\F\mathsf{H}}
\newcommand{\RH}{\R\mathsf{H}}
\newcommand{\ULH}{\mathsf{UH}2}
\newcommand{\MLH}{\mathsf{2LH}}
\newcommand{\SLH}{\mathsf{SH}}
\newcommand{\randH}{\mathcal{R}\mathsf{H}}

\newcommand{\blockFp}{\square\F_p\mathsf{H}}
\newcommand{\strideFp}{\bigcirc\F_p\mathsf{H}}
\newcommand{\blockZm}{\square\Z_m\mathsf{H}}
\newcommand{\strideZm}{\bigcirc\Z_m\mathsf{H}}
% \newcommand{\blockRu}{\square\R_u\mathsf{H}}

\newcommand{\lap}{\mathfrak{e}}
\newcommand{\halfp}{[\ceil{p/2}]_{\setminus 0}}
\newcommand{\pnozero}{[p]_{\setminus 0}}

\newcommand{\circabs}{\mathfrak{s}}
\newcommand{\one}{\mathbbm{1}}
\newcommand{\modone}{\mathfrak{m}_1}
\newcommand{\modp}{\mathfrak{m}_p}
\newcommand{\modm}{\mathfrak{m}_m}
\newcommand{\posmod}{\mathfrak{m}}

\newcommand{\bigO}{\mathcal{O}}
\newcommand{\tilo}{\widetilde{\bigO}}
\DeclareMathOperator{\E}{\mathbb{E}}
\DeclareMathOperator{\Var}{\text{Var}}
\DeclareMathOperator{\Hom}{Hom}
\newcommand{\argmin}[1]{\underset{#1}{\text{argmin}}}
\DeclareMathOperator{\rank}{rank}
\DeclareMathOperator{\sgn}{sgn}
\DeclareMathOperator{\Id}{Id}
\DeclareMathOperator{\img}{Img}
\newcommand{\PRM}{\mathbb{P}}
\DeclareMathOperator{\polylog}{\text{polylog}}
\DeclareMathOperator{\poly}{\text{poly}}
\newcommand{\norm}[1]{\left\lVert#1\right\rVert}
\newcommand{\interior}[1]{ {\kern0pt#1}^{\mathrm{o}} }
\newcommand{\mb}{\mathbf}
\newcommand{\partition}{\vdash}
\newcommand{\x}{\mathbf{x}}
\newcommand{\y}{\mathbf{y}}
\newcommand{\z}{\mathbf{z}}
\newcommand{\eps}{\varepsilon}
\renewcommand{\d}{\mathrm{d}} %straight d for integrals
\renewcommand{\Re}{\mathrm{Re}}
\renewcommand{\Im}{\mathrm{Im}}

\newcommand{\setof}[2]{\left\{ #1\; : \;#2 \right\}}
\newcommand{\set}[1]{\left\{ #1\right\}}

\newcommand{\R}{\mathbb{R}}
\newcommand{\Q}{\mathbb{Q}}
\newcommand{\C}{\mathbb{C}}
\newcommand{\Z}{\mathbb{Z}}
\newcommand{\F}{\mathbb{F}}
\newcommand{\K}{\mathbb{K}}
\newcommand{\N}{\mathbb{N}}
\newcommand{\contr}{\[ \Rightarrow\!\Leftarrow \]}
\newcommand{\defeq}{\vcentcolon=}
\newcommand{\eqdef}{=\vcentcolon}

\usepackage{algorithm}
\usepackage[noend]{algpseudocode} % adding [noend] deletes the end while and stuff

\newcommand{\floor}[1]{\left\lfloor #1 \right\rfloor}
\newcommand{\dfloor}[1]{\lfloor #1 \rfloor}
\newcommand{\ceil}[1]{\left\lceil #1 \right\rceil}
\newcommand{\paren}[1]{\left( #1 \right)}
\newcommand{\ang}[1]{\langle #1 \rangle}
\newcommand{\abs}[1]{\left| #1 \right|}

\usepackage{cleveref}
% \usepackage[capitalise,nameinlink,noabbrev]{cleveref}
\crefname{equation}{}{} % cref{eq:blah} only does (1) instead of Equation (1)
% \crefname{enumi}{Step}{} % cref{eq:blah} only does (1) instead of Item(1)

\newtheorem{innercustomthm}{}
\newenvironment{customthm}[1]
  {\renewcommand\theinnercustomthm{#1}\innercustomthm}
  {\endinnercustomthm}
\newtheorem{conj}[theorem]{Conjecture}
\newtheorem{question}[theorem]{Question}
\newtheorem{prop}[theorem]{Proposition}
\newtheorem{cor}[theorem]{Corollary}
\newtheorem{fact}[theorem]{Fact}

\theoremstyle{definition}
\newtheorem{defin}[theorem]{Definition}
\newtheorem{rmk}[theorem]{Remark}

% \makeatletter
% \newcommand{\rep@title}{} % Add this line to define \rep@title
% \newtheorem*{rep@theorem}{\rep@title}
% \newcommand{\newreptheorem}[1]{%
% \newenvironment{rep#1}[1]{%
%  \def\rep@title{\cref{##1}}%
%  \begin{rep@theorem}}%
%  {\end{rep@theorem}}}
% \makeatother
% \newreptheorem{theorem}{Theorem}

% \usepackage{mdframed}
% \usepackage{framed}

% \usepackage{tikz}
% \newcommand*\stride[1]{\tikz[baseline=(char.base)]{
%   \node[shape=circle,draw,inner sep=.5pt] (char) {$#1$};}}
% \newcommand{\block}[1]{\boxed{#1}}

% \newcommand\m[1]{\begin{pmatrix}#1\end{pmatrix}} 

% \usepackage[
% backend=biber,
% style=alphabetic,
% ]{biblatex}
% \addbibresource{refs.bib} %Imports bibliography file

\begin{document}
\maketitle

\abstract{
  In \defn{Linear Hashing} ($\LH$) with $\beta$ bins on a size
  $u$ universe ${\mathcal{U}=\set{0,1,\ldots, u-1}}$, items
  $\set{x_1,x_2,\ldots, x_n}\subset \mathcal{U}$ are placed in
  bins by the hash function
  $$x_i\mapsto (ax_i+b)\mod p \mod \beta$$ 
  for some prime $p\in [u,2u]$ and randomly chosen
integers $a,b \in [1,p]$. The \defn{maxload} of $\LH$ is the
number of items assigned to the fullest bin. Expected maxload for
a worst-case set of items is a natural measure of how well $\LH$
distributes items amongst the bins.

Fix $\beta=n$. Despite $\LH$'s simplicity, bounding $\LH$'s worst-case
maxload is extremely challenging. 
It is well-known that on random inputs $\LH$ achieves
maxload $\Omega\left(\frac{\log n}{\log\log n}\right)$; this is
currently the best lower bound for $\LH$'s expected maxload.
Recently Knudsen established an upper bound of
$\widetilde{\bigO}(n^{1 / 3})$. 
The question ``Is the worst-case expected maxload of $\LH$
$n^{o(1)}$?" is one of the most basic open problems in discrete
math.

In this paper we propose a set of intermediate open questions to
help researchers make progress on this problem. 
We establish the relationship between these intermediate open
questions and make some partial progress on them.
  % The \defn{hashing problem} is to assign $m$ items from a
  % universe $U$ to $n$ \defn{bins} so that all bins receive a
  % similar number of items. We measure the quality of a hashing
  % scheme's distribution by the number of items assigned to the
  % fullest bin, known as its \defn{maxload}. 

  % \defn{Linear Hashing} ($\LH$) is a
  % simple and classical solution to the hashing problem.
  % In $\LH$ items $x\in \set{1,2,\ldots, |U|}$ are
  % mapped to bins $\set{0,1,\ldots, n-1}$ by 
  % $$x\mapsto (ax+b)\mod p \mod n$$ for prime $p\in [|U|, 2|U|]$
  % and randomly chosen integers $a,b \in [1,p]$.
  % Despite $\LH$'s simplicity however, 
  % understanding the expected maxload of $\LH$ for worst-case
  % inputs is a notoriously challenging
  % and wide open question. For $m=n$ the best known lower bound is
  % $\Omega\left(\frac{\log n}{\log\log n}\right)$, whereas the
  % best known upper bound is $\widetilde{\bigO}(n^{1/3})$ due to
  % Knudsen\cite{knudsen_linear_2017}.

  % In this paper we consider three modifications of $\LH$: 
  % (1) $\LH$ without the ``$+b$" shift term, resulting in
  % loss of pairwise-independence. (2) $\LH$ with a composite,
  % rather than prime, modulus. (3) $\LH$ in a continuous setting
  % where the multiplier ``$a$" is chosen from $\R$ rather than $\Z$.
  % We show that $\LH$ is fairly robust to these changes, in
  % particular by demonstrating analogs of known maxload-bounds for
  % these new variants.

  % These results give several new perspectives on $\LH$, in
  % particular showing that properties of $\LH$ such as
  % pairwise-independence, a prime modulus, or even its setting in
  % the integers may not be fundamental. We believe that these new
  % perspectives, beyond being independently interesting, may
  % also be useful in future work towards understanding the maxload of
  % $\LH$.
}

\section{Introduction}
  The \defn{hashing problem} is to assign $n$ items from a
  universe $\mathcal{U}$ to $\beta$ \defn{bins} so that all bins receive a
  similar number of items. In particular, we measure the quality
  of a hashing scheme's load distribution by the number of items
  in the fullest bin; we refer to this quantity as the \defn{maxload}. 
  We desire three main properties of a hashing scheme: 
  (1) small expected maxload for all sets of items, (2) fast
  evaluation time, and (3) small description size.

  \paragraph{Related Work}
  The hashing problem has been extensively studied.
  We use the standard parameters $|\mathcal{U}|\in \poly(n)$,
  $\beta=n$ in the following discussion. We also assume the
  unit-cost RAM model, i.e., that arithmetic operations on
  numbers of size $\Theta(\log n)$ can be performed in constant
  time. We use the abbreviation $\ell(n) = \frac{\log n}{\log\log n}$.

  The hash function which assigns each item independently
  randomly to a bin achieves the smallest possible expected
  maxload in general, namely $\Theta(\ell(n))$
  \cite{mitzenmacher2017probability}.
  However, describing a fully random function requires
  $\Omega(|\mathcal{U}|\log n)$ bits which is extremely large.
  Full independence is not necessary to achieve optimal maxload.
  For instance, Carter and Wegman \cite{carter1977universal} show
  that degree $\Theta(\ell(n))$ polynomials over a finite
  field constitute a $\Theta(\ell(n))$-wise
  independent hash family while still achieving maxload
  $\Theta(\ell(n))$.
  Improving on this result, Celis et al. \cite{celis2013balls}
  demonstrate a hash family achieving maxload $\Theta(\ell(n))$
  with evaluation time $\bigO(\sqrt{\log n})$.

  In fact, it is even possible to achieve optimal maxload with
  constant evaluation time, as demonstrated by Seigel in
  \cite{siegel_universal_2004}. However, Seigel's hash function
  has description size $\poly(n)$.
  Furthermore, Seigel proved that it is impossible to
  simultaneously achieve optimal maxload, constant evaluation
  time, $n^{o(1)}$ description size, and $\Omega(\ell(n))$ independence. 
  However, this still leaves room for improvement if we do not
  require large degrees of independence.

  By itself small independence does not give any good bound on
  maxload; for example, there are pairwise-independent hash
  families with maxload $\Omega\left(\sqrt{n}\right)$ \cite{petershor}.
  However, Alon et al. \cite{alon_is_1997} show that the
  pairwise-independent hash family of multiplication by random
  matrices over $\F_2$ achieves $\bigO(\ell(n) \cdot (\log \log
  n)^2)$ maxload in $\bigO(\log(n))$ evaluation time. 

  The following question remains open:
  \begin{question}
    \label{question:fastandgood}
Is there a hash family with $\bigO(1)$ machine
word description whose evaluation requires $\bigO(1)$ arithmetic
operations that has expected maxload bounded by $n^{o(1)}$?
  \end{question}

\paragraph{Linear Hashing ($\LH$)}
$\LH$ \cite{motwani1995randomized, cormen2022introduction,
sedgewick2014algorithms} is an attractive potential solution to
\cref{question:fastandgood}, trivially satisfying the conditions
of small description size and fast evaluation time.
% $\LH$ heuristically does quite well at spreading out elements
% \cite{sedgewick2014algorithms}.
Despite $\LH$'s simplicity, understanding its
maxload is a notoriously challenging and wide open question. 
The best known lower bound on $\LH$'s maxload is $\Omega\left(\frac{\log
n}{\log\log n}\right)$, whereas the best known upper bound is
$\widetilde{\bigO}(n^{1/3})$ due to an elegant combinatorial
argument of Knudsen \cite{knudsen_linear_2017}.

Let $\mathcal{U} = \set{0,1,\ldots, u-1}$ denote the universe.
To keep the introduction simple we abuse notation and let $x
\bmod m$ denote the unique representative of the equivalence
class $x +m\Z$ lying in $[0,m)$.
In most textbooks (e.g., \cite{motwani1995randomized}) $\LH$ is
defined placing $x\in \mathcal{U}$ in bin
\begin{equation}\label{eq:LHdefn1}
(ax+b)\mod p \mod \beta
\end{equation}
for prime $p\in [u, 2u]$
and randomly chosen integers $a,b \in [1,p]$.
In \cite{dietzfelbinger1997reliable} Dietzfelbinger et al. give an
alternative definition placing $x\in \mathcal{U}$ in bin
\begin{equation}\label{eq:LHdefn2}
\floor{ \frac{(ax+b)\mod p}{p/\beta} }.
\end{equation}
We refer to \cref{eq:LHdefn1} as \defn{strided hashing} and
\cref{eq:LHdefn2} as \defn{blocked hashing}.

For $\beta=n$ Knudsen \cite{knudsen_linear_2017} implicitly observed that 
the maxload of \cref{eq:LHdefn1} and \cref{eq:LHdefn2} differ by
at most a factor-of-$2$; this follows from our
\cref{prop:blockZisok}. Roughly this equivalence follows by observing that 
if blocked hashing has large maxload for $a=a_0,$ then strided
hashing will have large maxload for $a=a_0 n \mod p$. Similarly,
if strided hashing has large maxload for $a=a_1$ then blocked
hashing will have large maxload for $a=a_1 n^{-1}\mod p$, where
$n^{-1}$ is the multiplicative inverse of $n$ in $\F_p$.
Thus, classically \cref{eq:LHdefn1}, \cref{eq:LHdefn2} are
essentially equivalent. On the other hand we show that blocked hashing
generalizes more readily.\footnote{In \cref{prop:blockZsucks} we
show that strided hashing does not generalize cleanly to composite
moduli. Furthermore, there is no natural way to generalize
strided hashing to real numbers.} 
Thus, the majority of our results will concern blocked hashing.

We further simplify the hash function \cref{eq:LHdefn2} by
removing the $+b$ \defn{``shift term"} obtaining 
\begin{equation} \label{eq:LHdefn3}
  x\mapsto \floor{\frac{ax \bmod p}{p/\beta}}.
\end{equation}
Removing the shift term also will not impact the maxload by more
than a factor-of-$2$: changing the shift term at most splits
fullest bins in half or merges parts of adjacent bins
into a single new fullest bin.

For the rest of this section \defn{Simple $\LH$} will refer to
the hashing scheme defined in \cref{eq:LHdefn3}.
We propose \cref{q:isLHsubpoly} as a potential solution to
\cref{question:fastandgood}.
\begin{question}\label{q:isLHsubpoly}
  Is the worst-case expected maxload of Simple $\LH$ bounded by $n^{o(1)}$?
\end{question}

% It is well known that pairwise independence is not sufficient to
% get any bound on maxload better than $\bigO(\sqrt{n})$
% \cite{petershor}.
% \begin{itemize}
%   \item Select $k\gets [n],\pi\gets S_{n-1}$ uniformly randomly.
%   \item Each ball $i\in [n-1]$ is placed in bin $k$ with
%     probability  $1/\sqrt{n}$ and in bin $k+\pi_i$ otherwise.
%   \item Ball $n$ is placed independently in a random bin.
% \end{itemize}
%  Here, the expected number of balls in any fixed bin is
%  $\bigO(1)$ (this is guaranteed for any pairwise independent hash
%  family by linearity of expectation), but the expected number of
%  balls in the fullest bin is $\Omega(\sqrt{n})$.

\subsection{Our Results}
In this paper we propose a set of intermediate open questions to
help researchers make progress on \cref{q:isLHsubpoly}. 
We establish the relationship between these intermediate open
questions and make some partial progress on them.

\paragraph{The Two Bin Case}
We start in \cref{sec:twobins} by considering an even simpler question than
\cref{q:isLHsubpoly}:
\begin{question}\label{question:twobincase}
  What can be said about Simple $\LH$ in the case where there are only
  $\beta=2$ bins?
\end{question}

What makes \cref{question:twobincase} interesting is that to a
first approximation it is the simplest question that one can ask
about $\LH$ that is already non-trivial.
Furthermore, understanding the two bin case may be helpful for
resolving \cref{q:isLHsubpoly}. One potential approach is: 
\begin{reptheorem}{question:reductiontwobins}
Is it possible to obtain bounds on Simple $\LH$'s maxload with $n$ bins
by recursively using concentration bounds on its maxload for
$2$ bins?
\end{reptheorem}

As partial progress towards \cref{question:reductiontwobins} we
analyze the maxload of Simple $\LH$ for two bins. For a
pairwise-independent hash function (e.g., \cref{eq:LHdefn2}) it
is straightforward to bound the expected maxload by $n/2 +
\bigO(\sqrt{n})$. However, for \cref{question:reductiontwobins}
pairwise-independence is not the reason we expect $\LH$ to
achieve small maxload; in particular as noted earlier the maxload
of $\LH$ with and without the shift term differ by at most a
factor-of-$2$. This motivates the study of Simple $\LH$'s maxload
in the two bin case as initial progress towards
\cref{question:reductiontwobins}.
Bounding Simple $\LH$'s maxload is complicated by
the fact some pairs of elements have probability as large as
$2/3$ of mapping to the same bin under Simple $\LH$. Despite
this, by combinatorially analyzing the amount of correlation
between different elements we show:
\begin{reptheorem}{thm:dontneedb}
  Simple $\LH$ with $\beta=2$ bins has expected maxload at most 
  $$n/2 + \bigO(n^{7/8}).$$
\end{reptheorem}

For the rest of the paper we consider the standard case of $\beta=n$
bins.

\paragraph{Connecting Prime and Integer Moduli}
In \cref{sec:Z} we consider the importance of
using a prime modulus for $\LH$.
Conventional wisdom (e.g., \cite{cormen2022introduction}) is that
using a non-prime modulus is catastrophic. Using a
non-prime modulus is complicated by the fact that in a general
ring, as opposed to a finite field, non-zero elements can
multiply to zero.
Fortunately for any $m$ there is a reasonably large subset of
$\Z_m$ which forms a group under multiplication. The subset is
$\Z_m^{\times }$: the set of integers coprime to $m$.
Using using $\Z_m^{\times }$ we define an alternative version of
$\LH$ called \defn{Smart} $\LH$  and show:
\begin{reptheorem}{thm:LHSLH}
  Fix integer $m\in \poly(n)$. The expected maxloads of
  Smart $\LH$ with modulus $m$ and Simple $\LH$ with modulus $m$
  differ by at most a factor-of-$n^{o(1)}$.
\end{reptheorem}
Intuitively, Smart $\LH$ with composite modulus behaves somewhat
similarly to Simple $\LH$ with prime modulus.
This similarity allows us to, with several new ideas, translate Knudsen's proof
\cite{knudsen_linear_2017} of a $\widetilde{\bigO}(n^{1/3})$
bound on Simple $\LH$'s maxload for prime modulus to the
composite modulus setting, giving:
\begin{reptheorem}{thm:Zm1_3}
  The expected maxload of Smart $\LH$ is at most
  $\widetilde{\bigO}(n^{1/3})$.
\end{reptheorem}

Part of why \cref{thm:Zm1_3} is interesting is that using 
\cref{thm:Zm1_3} in \cref{thm:LHSLH} gives:
\begin{reptheorem}{cor:translate}
  The expected maxload of Simple $\LH$ with composite modulus is
  at most $n^{1/3+o(1)}$.
\end{reptheorem}
In particular, in \cref{cor:translate} we have translated the
state-of-the-art bound for maxload from the prime modulus setting
to the composite modulus setting.
This gives tentative evidence that the behavior of $\LH$ with
composite modulus may actually be the same as that of $\LH$ with
prime modulus. We leave this as an open question:
\begin{reptheorem}{question:equivalenceFZ}
  Is the worst-case maxload of composite modulus $\LH$ the
  same, up to a factor-of-$n^{o(1)}$, as that of prime
  modulus $\LH$?
\end{reptheorem}

\paragraph{Connecting Integer and Real Moduli}
Finally, in \cref{sec:R} we consider \defn{Real $\LH$} where the
multiplier ``$a$" in \eqref{eq:LHdefn3} is chosen
from $\R$. Initially the change to a continuous setting seems
to produce a very different problem. 
In this continuous setting one equivalent way of formulating
\cref{q:isLHsubpoly} is:
\begin{question}[``Crowded Runner Problem"]
\label{question:dual}
  Say we have $n$ runners with distinct speeds $x_1,x_2, \ldots,
  x_n \in (0,1)$ starting at the same location on a length $1$
  circular race-track. $a\in (0,1)$ is chosen randomly and all
  runners run from time $0$ until time $a$. 
  Is it true that on average the largest ``clump" of runners,
  i.e., set of runners in single interval of size $1/n$, is of
  size at most $n^{o(1)}$?
\end{question}
As formulated in \cref{question:dual} the problem becomes a
dual to the famous unsolved ``Lonely Runner Conjecture"
of Wills \cite{wills1967zwei}  and Cusick \cite{cusick1982view}
as formulated in \cite{bienia1998flows}. In the Lonely Runner
Conjecture the question is for each runner whether there is any
time such that the runner is ``lonely", i.e., separated from all
other runners by distance at least $1/n$. Our question is whether for
most time steps there is any runner that is ``crowded",
i.e., with many other runners within an interval of size $1/n$
around the runner.
The difficulty of the Lonely Runner Conjecture may be indicative
that the ``Crowded Runner Conjecture" is also quite difficult.

In \cref{thm:itisreal} we show a surprising equivalence
between $\LH$ for integer and real moduli. 
Technically our equivalence is to a slightly stronger
version of integer modulus $\LH$ namely \defn{Random Integer $\LH$} where the
modulus is not simply the universe size $u$, but rather a
randomly chosen (and likely composite) integer in $[u/2, u]$.
Random Integer $\LH$ is clearly at most a factor-of-$2$ worse
that Simple $\LH$, but it is not obvious whether it is any better; we leave this as
an open question:
\begin{question}
  Does Random $\LH$ achieve smaller worst-case
  expected maxload than Simple $\LH$?
\end{question}

Formally the equivalence between Real $\LH$ and Random Integer
$\LH$ is as follows:
\begin{reptheorem}{thm:itisreal}
  Let $f(n), g(n)$ be lower and upper bounds respectively on the
  expected maxload of Random Integer $\LH$ that hold for all sufficiently large
  universes. Let $M_\R$ denote the expected maxload of
  Real $\LH$. Then 
  $$\Omega\paren{\frac{f(n)}{\log\log n}} \le M_\R \le
  \bigO(g(n)).$$
\end{reptheorem}
The proof of this theorem involves several beautiful
number-theoretical lemmas and is one of our main technical
contributions.

This equivalence between the real and integer versions of $\LH$ shows
that \cref{q:isLHsubpoly} may not fundamentally be about prime
numbers or even integers.

% In \cref{sec:nice} we analyze a particularly structured set
% which a priori is a reasonable candidate for a set which might
% incur large maxload; note that the maxload of an average set is
% of course small. However, we show that this set actually incurs
% small maxload. 

\section{Preliminaries}
\paragraph{Set Definitions}
We write $\PRM$ to denote the set of primes, $\F_p$ for $p\in \PRM$ to denote the
finite field with $p$ elements, and $\Z_m$ for $m\in \N$ to denote
the ring $\Z / m\Z.$

For $a,b\in \R$ we define 
$$[a,b]=\setof{x\in \R}{a\le x\le b}.$$
For $n\in \N$ we define 
$$[n] = \set{0,1,\ldots, n-1},$$
$$[n]_{\setminus 0} = [n]\setminus \set{0}.$$
For $p\in \N,x\in \R$ we define $\modp(x)$ as
the unique number in the set 
$$(x+p\Z) \cap [0,p),$$
and define
$$\circabs_p(x)= \min(\modp(x), \modp(-x)).$$
For $x\in \Z$, $\modp(x)$ is the positive remainder
obtained when $x$ is divided by $p$ and $\circabs_p(x)$ is the
smallest distance to an element of $p\Z$ from $x$.

For $a\in \R$ and set $X\subset \R$ we define
$$a+X = \setof{a+x}{x\in X},$$
$$a\cdot X = \setof{a\cdot x}{x\in X},$$
$$\modp(X) = \setof{\modp(x)}{x\in X}.$$


\paragraph{Number Theoretic Definitions}
For $x,y\in \N$ we write $x\perp y$ to denote that  $x,y$ are
coprime, and $x\mid y$ to denote that $x$ divides $y$.
We write $\gcd(x,y)$ to denote the largest $k$ satisfying both  $k\mid
x$  and $k\mid y$.
% We use $\nu_p(n)$ to denote the exponent of the largest
% power of $p$ dividing $n$. 

A \defn{unit}, with respect to an implicit ring, is an element
with an inverse. For $m\in \N$ we define $\Z_m^{\times}$ as the
set of units in $\Z_m$. That is,
$$\Z_m^{\times } =\setof{k \in [m]}{k\perp m}.$$ We write
$\phi$ to denote the Euler-Toitent function, which is defined to
be $\phi(m)=|\Z_m^{\times }|$. We write $\tau(m)$ to denote the number of
divisors of $m$. 

The following two facts (see, e.g.,
\cite{hardy1979introduction}), will be
useful in several bounds:
\begin{fact}\label{fact:numdivs}
  $$\tau(n) \le 2^{\bigO(\log n / \log\log n)} \leq n^{o(1)}.$$
\end{fact}
\begin{fact}\label{fact:toitent}
  $$\frac{n}{2\log\log n} \le \phi(n)< n.$$
\end{fact}

\paragraph{Hashing Definitions}
A \defn{hashing scheme} mapping universe $[u]$ to $\beta$ bins is a set of functions 
$$\setof{h_i: [u]\to [\beta]}{i\in I}$$ 
parameterized by $i\in I$ for some set $I$.
We say $h_i$ sends element  $x$ to bin $h_i(x)$.
The \defn{maxload} of $\setof{h_i}{i\in I}$ with respect to set $X$ for
parameter choice $i_0\in I$ is 
$$\max_{k\in [\beta]} \left|\setof{x\in X}{h_{i_0}(x)=k}\right|.$$
In other words, maxload is the number of elements mapped to the
fullest bin.
We are concerned with bounding the expected maxload of hashing
schemes with respect to uniformly randomly
chosen parameter $i\in I$ for arbitrary $X$.
We will abbreviate uniformly randomly to randomly when the
uniformity is clear from context.

\begin{rmk}\label{rmk:assumesize}
  Our analysis is asymptotic as a function of $n$, the number of
  bins. We assume $n$ is at least a sufficiently large constant.
  We also require the universe size $u$ to satisfy:
  $$n^{6}\le u \le \poly(n)$$
  We adopt these restrictions for the following reasons:
  \begin{itemize}
    \item If $u$ is too small then bounding maxload is not
      interesting. For instance, if $u\in \Theta(n)$ linear
      hashing trivially achieves maxload $\bigO(1)$.
      It is standard to think of the universe as being much larger
      than the number of bins. Our specific choice $u\ge \Omega(n^{6})$ is
      arbitrary, but simplifies some analysis.
\item If $u$ is too large then constant-time arithmetic
  operations becomes an unreasonable assumption. Thus, we require
  ${u\le \poly(n)}$.
  \end{itemize}
\end{rmk}

\section{Two Bins}
\label{sec:twobins}
In this section we consider the simplest possible setting for
$\LH$: hashing $n$ items to $2$ bins.
% One motivation for studying $\LH$ in the $2$-bin case is hope
% that analysis of $\LH$ for $2$-bins is helpful in analyzing
% $\LH$ for $n$ bins. We propose one possible such reduction:
% \begin{question}[Reduction to Two Bins]
%   \label{question:reductiontwobins} Is it possible to obtain a
%   bound of $n^{o(1)}$ on $\LH$'s maxload in the $n$-bin setting
%   by first establishing a strong concentration bound on $\LH$'s
%   maxload in the $2$-bin setting and then recursively applying
%   this bound to subsets of bins?
% \end{question}
If $\LH$ performs well on $n$ bins then it is reasonable to conjecture that the
maxload of $\LH$ with two bins is tightly concentrated around $n/2$.  We propose
as an open problem proving a Chernoff-style concentration bound on the maxload:
\begin{conj}\label{conj:reductiontwobins}
  Two bin $\LH$ incurs maxload larger than $n/2+k\sqrt{n}$
  with probability at most $2^{-\Theta(k^2)}+1/p.$
\end{conj}
We propose that showing a result such as \cref{conj:reductiontwobins}
may be relatively tractable compared to analysis of the full $n$ bin case.
Establishing such a conjecture would constitute the strongest evidence to date
that $\LH$ is a good load-balancing function.
As partial progress towards understanding $\LH$ in the $2$-bin case we analyze
its expected maxload.
% By Markov's
% inequality a bound on expectation also implies a polynomial
% strength concentration bounds (rather than exponential as in
% \cref{conj:reductiontwobins}) on $\LH$'s maxload.
Formally our hash function is defined as follows:
\begin{defin}
  Let $p\in \PRM$. In \defn{Multiplicative Two Bin} $\LH$
  ($\MLH$) we choose random $a\in \pnozero$ and place $x\in
    [p]$ in bin
  $\floor{\frac{\modp(ax)}{p/2}} \in \set{0,1}.$
\end{defin}
% \begin{rmk}\label{rmk:shiftmatterstwobins}
%   Regardless of the number of bins, the difference between the
%   maxload achieved by $\LH$ with and without the shift term
%   ``$+b$'' will not be more than a factor-of-$2$. Thus, when the
%   number of bins is $n$, the difference between these two methods
%   is fairly insignificant. 
%   However, in the $2$-bin case the maxload
%   is always between $n/2$ and $n$, so a factor-of-$2$ difference
%   would be quite substantial. Thus, analyzing $\ULH$ will not
%   give any non-trivial bound on the maxload of $\MLH$.
%   However, $\MLH$ is simpler in the sense that it has a single
%   parameter, and in some ways more natural. Thus, it is
%   interesting to understand its behavior.
% \end{rmk}


% \todo{potentially add a prop: Omega sqrt n log n}
% \begin{prop}\label{prop:lowerboundtwobins}
%   Two bins $\Omega(\sqrt{n})\log n$
% \end{prop}
% \begin{proof}
%   Let $X = \set{1,3,5,\ldots,2n-1}$
% no this is tricky?
% \end{proof}

% We now establish \cref{thm:dontneedb}, the analogue of
% \cref{prop:pairwise_concentrate} for $\MLH$. It is
% interesting that we can obtain a similar bound with the simpler
% hash function and without pairwise independence. 
% In fact, this prompts the following question:
% \begin{question}[Triple Collisions]
%   Although $\MLH,\ULH$ do not exhibit $3$-wise independence,
%   it may be possible to bound the degree to which they fail to be
%   $3$-wise independent, in a manner similar to how we bound the
%   overlap in the proof of \cref{thm:dontneedb}.
%   Analysis of colliding trios, or higher numbers of elements,
%   could potentially help give stronger bounds on maxload. However
%   this seems quite challenging.
% \end{question}

We prove
\begin{theorem}\label{thm:dontneedb}
  $\MLH$ has expected maxload at most
  $n/2+\widetilde{\bigO}(\sqrt{n}).$
\end{theorem}
% As noted in the introduction it is easy to see that a pairwise
% independent hash family (such as $\LH$ with the $+b$ shift term)
% achieves expected maxload $n/2+\bigO(\sqrt{n})$.
% However, pairwise independence alone is certainly not sufficient to
% achieve an $n^{o(1)}$ bound on $\LH$'s maxload in the $n$-bin
% case. Thus, beyond being intrinsically motivated as an
% interesting question, \cref{thm:dontneedb} is motivated as progress towards
% \cref{question:reductiontwobins}.

The proof of \cref{thm:dontneedb} uses the standard technique of analyzing the expected
number of \defn{collisions}: pairs of elements that hash to
the same bin. However, without pairwise independence computing
the expected number of collisions is challenging.
In fact, some elements collide with probability much
larger than $1/2$. For instance, $1$ and $3$ collide with probability
$2/3$. 
% as depicted in \cref{fig:13overlap-png}.
% % Figure environment removed
More generally for any small integer $k$, $1$ and $2k+1$ will
collide with probability approximately $(k+1)/(2k+1) > 1/2$.
If small odd numbers were the only numbers with probability
much larger than $1/2$ of colliding with $1$ then the analysis
would be fairly easy. However, there can be other numbers which
are very likely to collide with $1$. For instance, imagine
$x\in[p]$ satisfies  $3x\equiv 1 \mod p$. Then $1$ and $x$ also
have an approximately $2/3$ chance of colliding.

Our bound on the expected number of collisions intuitively
works as follows: for any particular element $x\in [p]$ there
are very few $y\in [p]$ where $x,y$ collide with probability
much larger than $1/2$. By symmetry (or more precisely the
existence of multiplicative inverses in $\F_p$), it does not
matter which $x$ we choose to compare with. So it will suffice
to analyze the probability of elements $y$ colliding with
$x=1$.

For sake of combinatorics we work with the following
transformed version of collision probabilities:
\begin{defin}
  The \defn{overlap} of $x\in [p]$ is the number of $a\in \halfp$
  such that  $1,x$ collide.
  Equivalently, the overlap of $x$ is the number of $a\in \halfp$ where
  $\modp(ax) < p/2.$
  The \defn{excess overlap} of $x$, denoted $\lap(x)$, is the
  overlap of $x$ minus $p/4$.
  The \defn{contribution} of a set ${A\subset \halfp}$ to $\lap(x)$
  is the difference between the number of $a\in A$
  with $\modp(ax) <p/2$ and the number of  $a\in A$ with
  $\modp(ax) > p/2$.
  We will bound $\lap(x)$ by partitioning $\halfp$ into
  disjoint subsets $A_1,A_2,\ldots$ and summing the
  contributions of each $A_i$ to $\lap(x)$.
\end{defin}


% We proceed to derive bounds on $\lap(x)$ as a function of $x$,
% and eventually use these to obtain bounds on $\sum_{x\in X}\lap(x)$.
%   First, we give a simple bound that highlights the general
%   technique required.
%   \begin{claim}
%     \label{clm:smallx}
%     $$\lap(x)\le \bigO(x+p/x).$$
%   \end{claim}
%   \begin{proof}
%     We group elements into \defn{stacks} consisting of $p/x$
%     contiguous elements each. A stack is a set of contiguous
%     elements that don't experience overlap.
%     We get $\pm 1$ error per stack, yielding a
%     total of $x$ error, plus $p/x$ error for the final ascent.
%   \end{proof}
%   \cref{clm:smallx} is fairly tight for small $x$; as hinted by
%   \cref{fig:13overlap-png} small (odd) $x$ achieve maxload
%   $\Theta(p/x)$. 
%   The $x$ in \cref{clm:smallx} is generally quite weak, but
%   there are some $x>p/n$ with very large maxload.
%   This generally happens if $x$ wraps around to a small number
%   quickly. This suggests that we need a better way of grouping
%   the elements. The following lemma, which is a beautiful
%   elementary fact of number theory, gives such a method.
%   \begin{lemma}\label{clm:beautifulparameterization}
%     For each $x\in [p]$, there exists $m\in [n]$ and $k\in
%     [\ceil{p/n}]$ such that 
%     $$xm\equiv k \mod p.$$
%     Furthermore, this $m,k$ uniquely characterize $x$.
%   \end{lemma}
%   \begin{proof}
%     One way is you could take $x=p/c$ and look at the
%     wrap-around points. Balance stuff. After $w$ wraparounds
%     you get size like $x/w$.
%     \todo{formalize this}

%     Note that $m^{-1}k$ (where $m^{-1}$ denotes the
%     multiplicative inverse of $m$ modulo $p$) can take on at
%     most $n\cdot \ceil{p/n}$ values, because  $m\in [n],k\in
%     [\ceil{p/n}]$. However, we associated an $m,k$ pair with
%     all elements of $[p]$ thus they are unique. \todo{actually
%     this doesn't quite make sense, for divisibility reasons. I
%   only really care about injective, not surjective.}
%   \end{proof}

%   Now that we have shown all $x$ are of the form described in
%   \cref{clm:beautifulparameterization},  we use this form to
%   give a more accurate bound on $\lap(x)$. 
%   \begin{lemma}\label{lem:therightlens}
%     \todo{put a floor/ceiling on all the fractions everywhere!}
%     Let $m\le n$ be the smallest $m$ such that $\modp(xm)\le
%     p/n$, and let  $k=\modp(xm) \le p/n.$
%     Then, $$\lap(x)\le \bigO\left(\frac{p}{km} + k + m\right).$$
%   \end{lemma}
%   \begin{proof}
%     It may be helpful to refer to \cref{fig:epic_lemma} for
%     understanding of the terminology used in this lemma.
%     % Figure environment removed

%     Define the $(i,j)$-th \defn{full line} as
%     $$L_{i,j} = i + \frac{mp}{k} \cdot j + m[p/k]$$ for $i\le m,
%     j\le \frac{k}{2m}.$
%     \begin{claim}\label{clm:fullline}
%       Each full line contributes $\bigO(1)$ to $\lap(x)$.
%     \end{claim}
%     \begin{proof}
%       Fix a full line $\delta + m[p/k]$. The image of
%       $\delta+m[p/k]$ under multiplication by $x$ modulo $p$ is:
%       $$\modp(\delta x), \modp((\delta+m)x),
%       \modp((\delta+2m)x),\ldots.$$
%       Let $\delta' = \modp(\delta x)$. Because  $\modp(mx) = k$
%       we can re-express the image as:
%       $$\delta',\delta'+k,\delta'+2k,\ldots, \delta'+k\floor{p/k}.$$
%       \todo{I am just making up the floors / ceilings at the
%       moment, plz fix sometime}

%       In other words, the image of the full line consists of
%       $\floor{p/k}$ points, which are evenly spaced out, with
%       space $k$ between the points. Thus, we will have
%       $\floor{p/k}/2 \pm \bigO(1)$ above and below the line. In
%       other words, the full line contributes at most $\bigO(1)$
%       to  $\lap(x)$.
%     \end{proof}

%     The full lines take of all but up a suffix of $[p/2]$ of size
%     at most $m\floor{p/k}$. Let $s$ be the start of this suffix.
%     We partition this suffix into
%     $m$ \defn{partial lines}, with the $i$-th partial line
%     defined as:
%     $$P_i = \paren{s+i+m[p/k]} \cap [p/2].$$
%     Unlike full lines, the partial lines need not lie half above
%     and half below $p/2$ (because they may be cut off part-way
%     through).
%     We now further partition the partial lines.

%     The $i$-th \defn{stack} is 
%     $$S_i = \paren{s+im+[m]}\cap [p/2].$$
%     We say stack $S_i$ is \defn{full} if  $|S_i| = m$, and
%     \defn{partially-full} if $0<|S_i|<m$.
%     The $i$-th \defn{chunk} is
%     $$C_i = \bigcup_{j \in i\frac{p}{km}+[\frac{p}{km}]} S_{j}.$$
%     Chunk $C_i$ is a \defn{full chunk} if $|C_i| = p/k$.
%     The \defn{final chunk} is the final non-empty $C_i$; this
%     chunk is the only non-empty but not-necessarily-full chunk.
%     A \defn{line-chunk} is the restriction of a partial line to a
%     particular chunk.

%     \begin{claim}\label{clm:fullchunk}
%       Each full chunk contributes $\bigO(1)$ to $\lap(x)$.
%     \end{claim}
%     \begin{proof}
%       Each stack in a full chunk consists of $m$ contiguous
%       indices. By virtue of the minimality condition on $k$ in the lemma
%       statement, distinct $x,y$ in the same stack are separated
%       by distance at least $p/n$. In particular, \todo{ohh this
%       is tricky}
%       \todo{hmm,I'm not really sure this is true. it certainly
%       seems difficult to prove\ldots}
%       \begin{case}
%         $m$ is odd. \\
%         \todo{justify all of this}
%         By construction the line-chunks within a chunk all reside
%         in disjoint contiguous sub-intervals of $[p]$. Thus,
%         there is exactly one line-chunk which crosses $p/2$. 
%         There are $(m-1)/2$ line-chunks above and below $p/2$, so
%         these contribute $0$ to $\lap(x)$.
%         The line-chunk which crosses $p/2$ is above and below
%         $p/2$ for half of the time, with $\pm 1$ error, so it
%         contributes $\pm 1$ to $\lap(x)$.
%       \end{case}
%       \begin{case}
%        $m$ is even. \\
%        \todo{justify this:}
%        There are $m/2$ line-chunks above and below $p/2$, except
%        for one line-chunk barely touches $p/2$, but overall this
%        only causes a $\pm 1$ contribution to $\lap(x)$.
%       \end{case}
%     \end{proof}
%     \begin{claim}\label{clm:finalchunk}
%       The contribution of the final chunk to $\lap(x)$ is at most 
%       $$\bigO\left(\frac{p}{km}+m\right).$$
%     \end{claim}
%     \begin{proof}
%       The final chunk consists of at most $\frac{p}{km}$ points
%       per partial line, so in particular each line travels
%       distance at most $p/m$. 
%       This means that only $\bigO(1)$ partial lines cross $p/2$.
%       We can pair up all the non-crossing partial lines, with one
%       left over if there are an odd number. All of these lines
%       will consist of the same number of full stacks $\pm 1$,
%       along with a partial stack. We pay $\bigO(m)$ for the
%       partial stack, and  $\frac{p}{km}$ for the segment of the
%       final un-paired partial line lying in the final chunk.
%       Finally, we pay an additional $\frac{p}{km}$ for the
%       partial line which crosses $p/2$.
%       \todo{so this, like the thing above, relies on an unproven,
%       and frankly unlikely to be true, assertion that each full stack
% has a contribution of $\pm \bigO(1)$ to the excess overlap. while
% this seems somewhat reasonable, by virtue of a heuristic that the
% points in a stack are nearly evenly distributed, it seems very
% difficult to prove (and again, somewhat likely to be false).
% small matter! another partition ought to do the trick. I still
% think that the parameterization is clever, and unearths the bx
% well.}
%     \end{proof}

%     There are at most $k$ full lines and at most $m$ full chunks.
%     The full lines, full chunks and final chunk taken together
%     consitute a partition of $[\floor{p/2}]$.
%     Summing the contributions to $\lap(x)$ from full lines, full
%     chunks, and the final chunk as bounded in
%   \cref{clm:fullline},\cref{clm:fullchunk},\cref{clm:finalchunk} gives a
%     bound on the excess overlap.
%    In particular we have:
%     $$\lap(x) \le \bigO\left(\frac{p}{km} + k+m\right).$$
%   \end{proof}

%   \begin{cor}
%     \label{lem:overlap}
%     $$\sum_{x\in X}\lap(x) \le \bigO(p\log^2 n).$$
%   \end{cor}
%   \begin{proof}
%     From \cref{lem:therightlens} we obtain the bound
%     $$\sum_{x\in X}\lap(x)\le \bigO\paren{\sum_{m,k \le \sqrt{n}}\frac{p}{m k} +
%     \frac{p}{n}+n} \le \bigO(p\log^2 n).$$
%   \end{proof}

% \begin{proof}
%   The argument would be as follows: 
%   for each $x$, we can write $xm \equiv k$ for $m<n, k<p/n$. 
%   Then we partition $[p/2]$ into full
%   lines then chunks and the final chunk. 

%   Difficulty: segments of $m$ contiguous things are really not so
%   uniform, even though intuitively they behave like uniform
%   things. And this is kind of hard to think about. 
%   But intuitively this should give you like $p/(km)$ behavior
%   bound. This is supported by extensive simulations. If you add
%   it up it gives you like $\log^2 n$. 
% \end{proof}

% First, we demonstrate the power of our partitioning technique
% for bounding excess overlap and eliminate a portion of $[p/2]$
% which our later methods are less effective against.
% \begin{claim}\label{clm:ezlilguys}
%   Let $x<p/n$. Then
%   $$\lap(x)\le \bigO(x+p/x).$$
% \end{claim}
% \begin{proof}
%   We partition $[p/2]$ into \defn{stacks} which are contiguous
%   groups of $m\in \ceil{p/x} \pm\bigO(1)$ elements
%   $a+[m]$ which do not
%   experience overlap, i.e., 
%   $$\modp(ax)<\modp((a+1)x)<\ldots<\modp((a+m-1)x).$$
%   Clearly there are $\Theta(x)$ stacks, and each stack, except
%   for the final one, contributes $\pm\Theta(1)$ to  $\lap(x)$. 
%   The final stack might not have $\Omega(p/x)$ elements. 
%   However, it certainly contributes at most $\bigO(p/x)$ to the
%   $\lap(x)$.
% Summing the contributions of all stacks gives the desired bound on
% $\lap(x)$.
% \end{proof}
% \begin{cor}\label{cor:lilguyspart2}
%   $$\sum_{x\in X\cap [\ceil{p/n}]} \lap(x) \le \bigO(p\log n).$$
% \end{cor}
% \begin{proof}
% This follows immediately from \cref{clm:ezlilguys}.
% \end{proof}
% Because of \cref{cor:lilguyspart2} it now suffices to analyze
% the case where $x>p/n$.
% We remark that small odd numbers $x$ result in precisely
% $\Theta(p/x)$ excess overlap, so our understanding of the
% excess overlap incurred by small $x$ is relatively tight.
% However, small odd $x$ are not the only $x$ which incur large
% excess overlap.
% If $x$ wraps around to a small value quickly, then $x$ can also
% have large excess overlap. In what follows we give some
% (very lose) methods for bounding these messier, more subtly bad $x$.


Now we give a bound on $\sum_{x\in X}\lap(x)$. Our key insight is
that $\lap(x)$ is best understood by finding a small
number $m$ so that $k=\circabs_p(xm)$ is small and using this
$m,k$ to partition $\halfp$ into parts that each have small
contribution to $\lap(x)$. 

\begin{lemma}\label{lem:pigeons}
  For any $x\in \Z_p$ there exist $m\in [n], k\in [\ceil{p/n}],
  \sigma\in \pm 1$ such that $x = \modp(\sigma m^{-1}k)$.
\end{lemma}
\begin{proof}
  By the pigeonhole principle the set $\setof{\modp(x\cdot i)}{i\in
  [n]}$ must have two numbers within distance $p/n$ of each
  other.
  Let $i_1,i_2\in [n]$ be distinct indices such that
  $\modp(xi_1-xi_2)\in [0, p/n]$.
  Set $m = |i_1-i_2| \in [n]$, and set $\sigma$ to be the sign of
  $i_1-i_2$. Then $\modp(x\sigma m)\in [\ceil{p/n}]$. Define $k$ to be
  $\modp(x\sigma m)$. Clearly we have $x = \modp(\sigma m^{-1}
  k)$, for $m,k,\sigma$ with the desired properties.
\end{proof}

% \begin{defin}\label{def:pleasant} Let $x\in \pnozero$.
%   We say $x$ is \defn{pleasant} if there exist integers
%   $m,k,\sigma$ satisfying
%   \[ m\in \left[1,\ceil{n^{1/4}}\right), k\in \left(n^{1/2}, p/n^{1/4}\right], \sigma \in \set{\pm 1} \]
%   so that $x$ is of the form
%   $x = \modp(\sigma m^{-1}k).$
%   Otherwise, $x$ is \defn{nasty}.
% \end{defin}
% \begin{lemma}\label{lem:nasty}
%   At most $\bigO(n^{3/4})$ of $x\in \pnozero$ are nasty.
% \end{lemma}
% \begin{proof}
%   Let $\ell=\ceil{n^{1/4}}.$  Fix $x\in\pnozero$.
%   $\modp([\ell]x)$ is a set of $\ell$ elements in $[p]$.
%   By the pigeon-hole principle there must be
%   distinct $i,j\in [\ell]$ such that
%   \[|\modp(xi) - \modp(xj)| \le p/\ell.\]
%   Let $m=|i-j|$; clearly $m < \ell$.
%   Thus, for one of $\sigma\in \set{\pm 1}$ and some $k\le p/\ell$ we must have
%   $\modp(xm\sigma) = k.$
%   Thus, in order for $x$ to be nasty it is necessary that $x$ is
%   of the form $\modp(\sigma m^{-1}k)$ for $k\le n^{1/2}, m < \ell, \sigma \in
%     \set{\pm 1}$. 
%     The number of $x$ of this form is at most
%     $2 n^{1/2} \ell \le \big(n^{3/4}).$
% \end{proof}
\begin{lemma}\label{lem:epicbound}
  Let $x= \modp(\sigma m^{-1}k)$ for
  $\sigma\in \pm 1,m\in [n],k \in [\ceil{p/n}]$.
  Then, \[\lap(x) \le \bigO\left(k+\left(m + \frac{p}{mk}\right)\cdot
    \gcd(k,m)\right).\]
\end{lemma}
\begin{proof}
  We partition $\halfp$ into $m$ \defn{groups}, where for each
  $i\in [m]$, group $i$ consists of the values $G_i = {(m\Z+i)\cap \halfp}$.
  We further split groups into \defn{cycles}, where cycle $C_{i,j}$
  is defined to be 
  \[
    C_{i,j} = \setof{m j' + i}{j' \in [j\ceil{p/k}, (j+1)\ceil{p/k})} \cap
    \halfp \subseteq G_i.
  \]
  We say that a cycle $C_{i,j}$ is a \defn{full cycle} if
  $|C_{i,j}| = \ceil{p/k}$.
  For any full cycle $C_{i,j}$ the set $\modp(x\cdot C_{i,j})$
  consists of $\ceil{p/k}$ points, with consecutive points
  separated by distance $k$; this is due to the fact that
  $x=\modp(\sigma m^{-1} k)$. In particular, this means that the
  points go slightly past a full revolution of the circle $\Z_p$.
  Thus, each full cycle contributes at most $\bigO(1)$ to $\lap(x)$.
  The total contribution to $\lap(x)$ from all full cycles is thus 
  bounded by 
  \[
    \sum_{i\in [m]} \floor{\frac{|G_i|}{\ceil{p/k}} } \cdot
    \bigO(1)\le \bigO\left(m \cdot \frac{p/m}{p/k}\right) \le \bigO(k).
  \]
  Now it suffices to bound the contribution from non-full
  cycles. Observe that each group $G_i$ has a single non-full
  cycle, which we call its \defn{final cycle}, or $F_i$ (if $G_i$ has no non-full
  cycle, then its final cycle is $\varnothing$). When bounding the
  contribution from the final cycles it is no longer a good idea
  to analyze the groups separately. 
  Instead, for each $j\in [|F_{m-1}|]$ we will group together the
  $j$-th largest values in each of the $F_i$'s into a
  \defn{step}, denoted $S_j$.
  Note that this might not quite capture all the points in all
  the final cycles because there may be some $i$ such that
  $|F_{i}| = |F_{m-1}|+1$. 
  To handle this we remove the largest value from each final
  cycle $i$ with $|F_i| = |F_{m-1}|+1$. This results in contribution
  $\bigO(m)$ to $\lap(x)$.
  This done, it suffices to consider the contribution of
  $\bigcup_{i\in [L]}S_i$ where $L$ is the number of steps.
  Note that $L$ satisfies $L\le \ceil{p/k}$, or else the final
  cycles have enough points that they would have been full cycles.
  Now we prove a powerful structural result about the steps.
  \begin{claim}\label{clm:structureSi}
    There exists $\Delta \in \Z, \lambda\in \Z_m^*$ and $\delta_j\in
    [-k,k]$ for $j\in [m]$ such that 
    \begin{equation}\label{eq:S0goal}
     \modp(x\cdot S_0)=\setof{\modp\left(\Delta + p\frac{k\lambda j}{m} + \delta_j\right)}{j\in [m]}.
    \end{equation}
  \end{claim}
  \begin{proof}
    $\modp(m^{-1}m) = 1$. 
    So, as an integer $m^{-1}$ can be written in the form
    $\frac{1+\lambda p}{m}$ for some integer $\lambda$.
    We claim that $\lambda\perp m$. If not, then we would have
     \[
    \frac{m}{\gcd(\lambda,m)} \cdot \frac{1+\lambda p}{m} =
    \frac{1}{\gcd(\lambda, m)} + p\frac{\lambda}{\gcd(m,\lambda)}
    \notin \Z,
    \]
    which is clearly impossible.
    Thus, there is $\lambda\perp m$ and some $\alpha\in \N$ so that 
    \[
      \modp(x\cdot S_0) = \setof{\modp\left(\sigma m^{-1} k\cdot (\alpha m +
      \beta)\right)}{\beta\in [m]} = \setof{\modp\left(\sigma
      k \alpha + p\frac{k\lambda \sigma\beta}{m} +
  \frac{\sigma k\beta}{m}\right)}{\beta\in [m]}.
    \]
    The term $\sigma k \alpha$ is the $\Delta$ from \cref{eq:S0goal}.
    The term $p\frac{k\lambda \sigma\beta}{m}$ is the
    $p\frac{k\lambda j}{m}$ from \cref{eq:S0goal} (where we may
    eliminate $\sigma$ by re-indexing if $\sigma=-1$).
    Finally, the term $\frac{\sigma k\beta}{m}$ is the $\delta_j$
    from \cref{eq:S0goal}, and it does indeed satisfy $\delta_j
    \in [-k,k]$, as required.
    
  \end{proof}
  We call the points $\mathbb{A} = \modp(p\frac{k}{m}\Z)$ \defn{anchors}.
  Observe that $|\mathbb{A}| = m/\gcd(m,k)$.
  A \defn{rotation} of the anchors is the set
  $\modp(\mathbb{A}+\Delta)$ for some $\Delta\in \Z$.
  In \cref{clm:structureSi} we showed that there is some
  $\Delta\in \Z$ such that for each $y\in
  \modp(\mathbb{A}+\Delta)$ $\modp(x S_0)$ has $\gcd(m,k)$ points which
  are very close to $y$.
  The other important fact that will let us control the steps is
  that $\modp(x S_{i+1}) = \modp(xS_i + \sigma k).$ 

  Now we consider two cases. 
  The easier case is if $m$ is even.
  In this case, we have the helpful property that for any rotation of
  the anchors there are an equal number of anchors in
  $[0,p/2)$ and in $[p/2, p)$.
  If the points in $\modp(x\cdot S_i)$ where \emph{actually}
  located at the anchors then the
  contribution from $\bigcup_{i\in [L]}S_i$ would be \emph{zero}.
  However, the points are allowed to deviate by a small amount
  from the anchors. Evidently this only results in a
  contribution to $\lap(x)$ at most $\bigO(1)$ times per every
  $\ceil{p/(km)}$ consecutive steps. Thus, the contribution from
  $\bigcup_{i\in [L]}S_i$ is bounded by:
  \[
  \bigO(\gcd(k,m))\cdot \frac{L}{p/(km)} \le \bigO(m\gcd(k,m)).
  \]

  Now we consider the case that $m$ is odd.
  First, we claim that the contribution of any single step $S_i$ is at
  most $\bigO(\gcd(m,k))$.
  This is by \cref{clm:structureSi}: the points in $S_i$ are
  concentrated around the anchors, the number of anchors in
  $[0,p/2)$ and $[p/2, p)$ differ by at most $1$, and
  hence the contribution of $S_i$ is at most $\bigO(\gcd(m,k))$.
  Now we show that certain large groups of steps called
  \defn{revolutions} have contribution $\bigO(\gcd(m,k))$.
  Revolution $i$, or $R_i$, is the steps $S_{j}$ for each $j\in [L] \cap
  [\ceil{p/(km)}i, \ceil{p/(km)}(i+1)).$
  We say $R_i$ is a \defn{full revolution} if $R_i$ consists of
  $\ceil{p/(km)}$ steps.
  There is at most one (non-empty) non-full revolution.
  We bound the contribution from the non-full revolution by
  $\bigO(\ceil{p/(km)} \gcd(k,m))$, using our earlier observation
  that each individual step has contribution as most
  $\bigO(\gcd(m,k))$.
  Now we argue that if $R_i$ is a full revolution then the
  contribution of $R_i$ is at most $\bigO(\gcd(m,k))$.
  Let $t_0$ denote the number of steps during $R_i$ where
  there is one more anchor in $[0,p/2)$ than in $[p/2, p)$, and
  let $t_1$ denote the number of other time steps.
  Clearly $|t_1-t_0|\le \bigO(1)$.
  Then, utilizing the tight concentration of points around the
  anchors from \cref{clm:structureSi} we have that the
  contribution of $R_i$ is at most $\bigO(\gcd(m,k))$.
  Finally, the number of full revolutions is at most $\bigO(m)$,
  so the total contribution from all full revolutions is at most
   $\bigO(m \gcd(m,k))$.
 Summing all the contributions discussed in the proof gives the
 desired bound.
\end{proof}

% The theorem follows readily from \cref{lem:epicbound}; we deffer
% the details to \cref{app:twobin}.

% \section{Conclusion of the Proof of
% \cref{thm:dontneedb}}\label{app:twobin}

We now use the powerful combinatorial  \cref{lem:epicbound}
to conclude the proof of \cref{thm:dontneedb}.

\begin{theorem}\label{cor:kindaepicevenifweak}
  $\sum_{x\in X} \lap(x) \le \tilo(p).$
\end{theorem}
\begin{proof}
  In \cref{lem:pigeons} we showed that each $x\in X$ can be
  represented by some $\modp(\sigma m^{-1} k)$ for $\sigma\in \pm
  1, m\in [n], k\in [\ceil{p/n}]$;
  Form a set $Y$ of triples by selecting for each $x\in X$ some
  such representative $(m,k,\sigma)$. Of course each triple can
  only represent one $x$.
  Then by \cref{lem:epicbound} we have:
   \[
     \sum_{x\in X} \lap(x) \le  \sum_{(m,k,\sigma)\in Y} \bigO\left(k+\left(m + \frac{p}{mk}\right)\cdot
     \gcd(k,m)\right) \le \bigO(p) + \bigO(p)\cdot\sum_{(m,k,\sigma)\in Y}
     \frac{\gcd(k,m)}{km} ,
   \]
   where the final inequality was obtained by replacing $k,m$ by
   their maximum possible values and using the trivial bound
   $\gcd(k,m)\le m$.
   We now bound the final term in the sum:
\[
  \sum_{(m,k,\sigma)\in Y} \frac{\gcd(k,m)}{km} \le \sum_{\sigma
  = \pm 1}\sum_{m\in [n]}
  \sum_{d\mid m}\sum_{dk\in [\ceil{p/n}]} \frac{d}{dk m} \le \sum_{m\in
  [n]}\frac{\tau(m)}{m}  \bigO(\log n)\le \bigO(\log^3 n).
\]
Where we have used the well-known bound that $\E_{x\in
[n]}[\tau(x)]\le\bigO(\log n)$ \cite{hardy1979introduction}.
Thus we have the desired bound: $\sum_{x\in X} \lap(x)\le
\tilo(p)$.


\end{proof}

Let random variables $C, M$ denote the number of collisions and
maxload respectively. Let  $\mu = \E[M]$. Let $C_{i,j}$ be $1$ if
$i,j$ collide and $0$ otherwise; our convention is $C_{i,i}=0$.
Using \cref{cor:kindaepicevenifweak} we obtain a bound on $\E[C]$.
\begin{cor}\label{lhs:complicated7}
  $\E[C] \le n^2/4 + \tilo(n^{1/2}).$
\end{cor}
\begin{proof}
  Interpreting \cref{cor:kindaepicevenifweak} probabilistically,
  for any $n$-element set $X\subset [p]$ we have
  \begin{equation}\label{eq:justabovefor1}
    \sum_{y \in X} \E[C_{1,y}] \le n/2 + \tilo(1).
  \end{equation}
  \eqref{eq:justabovefor1} is easily generalized to bound the number of collisions
  between $y\in X$ and any $x\neq 0$.
  In particular, the collisions between $x,X$ are the same as the collisions
  between $1,\modp(x^{-1}X)$ where $x^{-1}$ is the inverse of $x \bmod
    p$. Because \eqref{eq:justabovefor1} holds for arbitrary
  $n$-element sets $X$, we have for any $x\neq 0$
  \[
  \sum_{y \in X} \E[C_{x,y}] \le n/2 + \tilo(1).
  \]

  Now, we are equipped to bound $\E[C]$. Index $X$ as $X=\set{x_1,x_2,\ldots,
      x_n}$. Then
  \begin{equation*}
    \E[C]  = \sum_{i=1}^{n}\sum_{j=1}^{i-1} \E[C_{x_i,x_j}] 
          \le \sum_{i=1}^{n} \paren{i/2 + \tilo(1)}  
          \le n^2/4  + \tilo(n).
  \end{equation*}
\end{proof}
% note: we count self-collisions

Finally, we complete the proof of \cref{thm:dontneedb} by comparing the expected number
of collisions to the number of collisions induced by the maxload.
\begin{proof}[Proof of \cref{thm:dontneedb}]
  The number of collisions $C$ is determined by the maxload. In
  particular,
  % \begin{equation}\label{eq:rhs7ezv1}
    $C = \binom{M}{2}+\binom{n-M}{2}.$
  % \end{equation}
    This is a convex function of $M$.
  % Note that \cref{eq:rhs7ezv1} is a convex function of $M$.
  Thus, applying Jensen's inequality and comparing with 
  \cref{lhs:complicated7} gives:
  \[ \binom{\mu}{2}+\binom{n-\mu}{2} \le n^2/4 + \tilo(n) .\]
  Solving the quadratic in $\mu$ we find:
  $\mu \le n/2+\tilo(\sqrt{n}).$
\end{proof}



% % \section{Conclusion of the Proof of
% \cref{thm:dontneedb}}\label{app:twobin}

We now use the powerful combinatorial  \cref{lem:epicbound}
to conclude the proof of \cref{thm:dontneedb}.

\begin{theorem}\label{cor:kindaepicevenifweak}
  $\sum_{x\in X} \lap(x) \le \tilo(p).$
\end{theorem}
\begin{proof}
  In \cref{lem:pigeons} we showed that each $x\in X$ can be
  represented by some $\modp(\sigma m^{-1} k)$ for $\sigma\in \pm
  1, m\in [n], k\in [\ceil{p/n}]$;
  Form a set $Y$ of triples by selecting for each $x\in X$ some
  such representative $(m,k,\sigma)$. Of course each triple can
  only represent one $x$.
  Then by \cref{lem:epicbound} we have:
   \[
     \sum_{x\in X} \lap(x) \le  \sum_{(m,k,\sigma)\in Y} \bigO\left(k+\left(m + \frac{p}{mk}\right)\cdot
     \gcd(k,m)\right) \le \bigO(p) + \bigO(p)\cdot\sum_{(m,k,\sigma)\in Y}
     \frac{\gcd(k,m)}{km} ,
   \]
   where the final inequality was obtained by replacing $k,m$ by
   their maximum possible values and using the trivial bound
   $\gcd(k,m)\le m$.
   We now bound the final term in the sum:
\[
  \sum_{(m,k,\sigma)\in Y} \frac{\gcd(k,m)}{km} \le \sum_{\sigma
  = \pm 1}\sum_{m\in [n]}
  \sum_{d\mid m}\sum_{dk\in [\ceil{p/n}]} \frac{d}{dk m} \le \sum_{m\in
  [n]}\frac{\tau(m)}{m}  \bigO(\log n)\le \bigO(\log^3 n).
\]
Where we have used the well-known bound that $\E_{x\in
[n]}[\tau(x)]\le\bigO(\log n)$ \cite{hardy1979introduction}.
Thus we have the desired bound: $\sum_{x\in X} \lap(x)\le
\tilo(p)$.


\end{proof}

Let random variables $C, M$ denote the number of collisions and
maxload respectively. Let  $\mu = \E[M]$. Let $C_{i,j}$ be $1$ if
$i,j$ collide and $0$ otherwise; our convention is $C_{i,i}=0$.
Using \cref{cor:kindaepicevenifweak} we obtain a bound on $\E[C]$.
\begin{cor}\label{lhs:complicated7}
  $\E[C] \le n^2/4 + \tilo(n^{1/2}).$
\end{cor}
\begin{proof}
  Interpreting \cref{cor:kindaepicevenifweak} probabilistically,
  for any $n$-element set $X\subset [p]$ we have
  \begin{equation}\label{eq:justabovefor1}
    \sum_{y \in X} \E[C_{1,y}] \le n/2 + \tilo(1).
  \end{equation}
  \eqref{eq:justabovefor1} is easily generalized to bound the number of collisions
  between $y\in X$ and any $x\neq 0$.
  In particular, the collisions between $x,X$ are the same as the collisions
  between $1,\modp(x^{-1}X)$ where $x^{-1}$ is the inverse of $x \bmod
    p$. Because \eqref{eq:justabovefor1} holds for arbitrary
  $n$-element sets $X$, we have for any $x\neq 0$
  \[
  \sum_{y \in X} \E[C_{x,y}] \le n/2 + \tilo(1).
  \]

  Now, we are equipped to bound $\E[C]$. Index $X$ as $X=\set{x_1,x_2,\ldots,
      x_n}$. Then
  \begin{equation*}
    \E[C]  = \sum_{i=1}^{n}\sum_{j=1}^{i-1} \E[C_{x_i,x_j}] 
          \le \sum_{i=1}^{n} \paren{i/2 + \tilo(1)}  
          \le n^2/4  + \tilo(n).
  \end{equation*}
\end{proof}
% note: we count self-collisions

Finally, we complete the proof of \cref{thm:dontneedb} by comparing the expected number
of collisions to the number of collisions induced by the maxload.
\begin{proof}[Proof of \cref{thm:dontneedb}]
  The number of collisions $C$ is determined by the maxload. In
  particular,
  % \begin{equation}\label{eq:rhs7ezv1}
    $C = \binom{M}{2}+\binom{n-M}{2}.$
  % \end{equation}
    This is a convex function of $M$.
  % Note that \cref{eq:rhs7ezv1} is a convex function of $M$.
  Thus, applying Jensen's inequality and comparing with 
  \cref{lhs:complicated7} gives:
  \[ \binom{\mu}{2}+\binom{n-\mu}{2} \le n^2/4 + \tilo(n) .\]
  Solving the quadratic in $\mu$ we find:
  $\mu \le n/2+\tilo(\sqrt{n}).$
\end{proof}



% \begin{proof}
  % Let $\ell = \floor{p / (m\floor{p/k})}\in \Theta(k/m)$.
  % We partition $\halfp$ into $m$ \defn{circles},
  % where circle $i\in [m]$ consists of elements $(m\Z+i) \cap \halfp$.
  % We further partition circles into $\ell$ \defn{cycles}, which are
  % contiguous blocks of $\floor{p/k}$ elements in a circle, i.e.,
  % $m[\floor{p/k}] + \Delta$ for some $\Delta$.
  % There are up to $p/k$ \defn{extra} elements at the end
  % of the elements within the circle which do not fall into a cycle.
  % Aggregating the extra elements from all circles gives at most
  % $mp/k$ elements, which trivially contribute at most $mp/k$ to
  % $\lap(x)$. 
  % Each cycle contributes at most $\bigO(1)$ to $\lap(x)$ because a
  % cycle is $\floor{p/k}$ elements evenly spaced out by $k$, except
  % the distance between the final element and the first element may
  % be as large as $2k$; see
  % \cref{fig:pleasantcircles-svg}.
  % Thus, adding up the $\ell$ cycles for each of the $m$ circles
  % results in
  % $\ell m\cdot \bigO(1) \le \bigO(k)$
  % contribution to $\lap(x)$.
% % Figure environment removed
% \end{proof}

%   \todo{make an alias for $\floor{p/2}$}
%   \begin{defin}
%     Fix $L\in \N$ and $x\in [p/2]\setminus\set{0}$. We define values
%     $$\delta_i=\delta_i(x,L), c_i=c_i(x,L), \ell_i=\ell_i(x,L)$$ as 
%     $\delta_0 = x$, $c_0=1$, and for $i\in \set{1,2,\ldots, L-1}$
%     \begin{align*}
%       &\ell_{i-1} = \paren{\argmin{\ell\in \set{\ceil{p/\delta_{i-1}},
%     \floor{p/\delta_{i-1}}}} \abs{\ell_{i-1} \cdot
% \delta_{i-1}}_p} -1,\\
%       &\delta_i = |\ell_{i-1}\cdot \delta_{i-1}|_p,\\
%       &c_i = c_{i-1}\cdot \ell_{i-1} = \prod_{j\in [i-1]} \ell_j.
%     \end{align*}
%     While $\delta_i,c_i,\ell_i$ are functions of $x,L$ we will
%     leave this dependence implicit so long as $x,L$ are defined
%     in context.
%   Intuitively, $\delta_i$ is obtained by traveling once around
%   a circle of $[p/2]$ points with stride $\delta_{i-1}$ and
%   then taking the remainder obtained at the end.

%   We partition $[p/2]$ into $L$ \defn{levels} and then further 
%   partition the levels into \defn{circles}.
%   For each $i\in [L]$, level  $i$ is composed of $c_i$ circles. 
%   Intuitively, a circle on level $i$ is composed of
%   $\ell_i$ elements of $[p/2]$. 
%   However, if $i$ is too large then all $[p/2]$ elements will
%   have already been used up. We say that a circle on level $i$ is
%   \defn{full} if it actually has $\ell_i$ elements, and that a
%   level is \defn{full} if all circles on that level are full.
%   We say that a $L$ is a \defn{valid} number of levels if all but
%   the final level is full (and the final level is non-empty).
%   Note that in the final level circles could either have more or
%   less than $\ell_L$ elements.

%   Formally, circle $j$ on level $i$ consists of the elements:
%   $$[p/2] \bigcap \left(c_i\cdot [\ell_i] +j+
%     \sum_{k=0}^{i-1}c_k \ell_k \right).$$
%     \begin{claim}
%       The image of a level $i$ circle under $y\mapsto \modp(xy)$
%       consists of elements spaced out by $\delta_i$.
%     \end{claim}
%     \begin{proof}
%       This is an immediate consequence of our definition. This
%       property is illustrated in \cref{fig:illustrate}.
% \end{proof}
%   \end{defin}
% % Figure environment removed

%   The utility of the level/circle method of viewing $x$ is explained in
%   \cref{lem:lookandsee}. 

%   \begin{lemma}\label{lem:lookandsee}
%     Fix $x>p/n$, and let $L\in \N$ be a valid number of levels
%     for $x$ such that $x$ has $m$ elements on level $L$.
%     Then 
%     $$\lap(x) \le \bigO(c_L + p/c_L)+??.$$
%   \todo{where ?? is the noise for number of circles before the final
%   level}
%   \end{lemma}
%   \begin{proof}
%     We partition the elements on level $L$ into contiguous sets
%     of $c_L$ elements, which we term \defn{full cycles}. 
%     There are $\bigO(c_L)$ elements at the end which are not
%     included in any full cycle, but constitute a \defn{partial cycle}

%     \begin{claim} \label{clm:ultrainduction}
%       Each full cycle contributes $\bigO(1)$ to $\lap(x)$.
%     \end{claim}
%     \begin{proof}
%       We prove  \cref{clm:ultrainduction} by induction on $L$.

%       If $L=1$ it is clear: this is simply saying that evenly
%       spaced points spanning the circle results in $\pm\Theta(1)$
%       contribution to $\lap(x)$.

%       ok. a bound of $c_{L-1}$ is obvious; just cancel circles
%       with themselves.  (but iirc this is not good enough bound)
%       The induction is somehow supposed to show that the ``extra''
%       points in each circles can cancel with each other?.

% maybe worth taking a look at lemma 7 to see exactly what we have
% to work with\ldots

%     Let $k\equiv mx$.
%     For simplicity let $k$ be positive (I mean less than $p/2$);
%     everything is symmetric if $k$ is negative.

%     Our partition is as follows: contiguous blocks of $m$
%     elements. Ok now we need to show that this works.

%     First a simple case, which is our base case for an induction.
%     We say that $x$ has a single level if $m=\ceil{p/x}$.
%     Then, it's clear that $m$ contiguous elements has $\pm 1$ 
%     excess overlap contribution.

%     Now let's ramp it up a level, and do $x$ which is like two
%     levels or whatever. So now we have a circle, and basically
%     think of $m$ circles sprouting out of this original circle. 
%     Think of  $k$ being supa supa small. So now each of these
%     'lil circles has one weird ``stitch'' thing. Ok but basically
%     we can pair most of the circle-y dudes up with themselves.
%     But maybe they each have some extra points. But then we also
%     pair up these extra points so its fine.

%     And then we can induct that for any number of levels.
%     With the only inductive thing that we need being like each
%     circle that the final circles sprout from has exactly one
%     stitch and beyond that has just uniform distance things.

%     Anyways once we have the whole $\bigO(1)$ err per $m$ steps
%     thing we are super chilling for our $m+p/m$ bound.

%     \end{proof}

%     There are $\Theta(p/c_L)$ full cycles, and each contributes
%     at most $\bigO(1)$ to  $\lap(x)$ by
%     \cref{clm:ultrainduction}. There are at most
%     $c_L$ elements in the partial cycle;
%     these elements contribute at most $c_L$ to $\lap(x)$.
%     The circles before the final level contribute at most ?? to
%     $\lap(x)$.
%     Combined, this gives the desired bound on $\lap(x)$.
%   \end{proof}

%   For $m< \sqrt{n}$ we say that $x$ is
%   \defn{unavoidably}-$m$-circle if there exists a level $\ell$
%   such that  $c_\ell = m$, while $c_{\ell+1} > p/\sqrt{n}$.
%   Intuitively, this means that the granularity on level $\ell$ is
%   very small.
%   \begin{lemma}
%     Fix $m<\sqrt{n}$. There are at most $\bigO(m\sqrt{n})$
%     unavoidably-$m$-circle $x$.
%   \end{lemma}
%   \begin{proof}
%     Assume that for some $\ell$, we have $c_\ell(x)=m$.
%     To be unavoidably-$m$-circle, $x$ must satisfy
%     $$\paren{\sum_{k=0}^{\ell-1}c_k\ell_k}\cdot x \equiv \delta
%     \mod p$$
% \todo{I think this is the right number}
% for some very small $\delta$. In particular, $c_{\ell+1}$ must
% satisfy:
% $$c_{\ell+1} \approx m p/\delta > p/\sqrt{n}.$$
% \todo{there is some fudging about the fact that $m\neq \sum
% c\ell$; but intuitively $m\approx \sum c\ell$ so I won't press
% the issue right now}
% This results in the requirement $\delta < m\sqrt{n}$.
% However, for each such value of $\delta$ there is a unique
% solution to $mx\equiv \delta \mod p$, and this solution might not
% even have $c_\ell(x)=m$.
% Thus, there are at most  to at most $m\sqrt{n}$ values of $x$ which
% are unavoidably-$m$-circle.

% % What I said before:
% %     The rational is that $xm\equiv k$ with  $k<m$ has at most
% %     $m$ solutions. However, this is kind of false logic. Because
% %     not all the points are in the final level. But it should be
% %     good enough for small $m$. Maybe if we say there are at most $\bigO(m)$
% %     such elements?
%   \end{proof}


% \begin{claim}
%   Adding up stuff we get average excess overlap $p/n^{1/4}$.
% \end{claim}
% \begin{proof}
%   We combine our two lemmas and add stuff up.
%   Adding stuff up, in the worst case, we obtain:
%   $$\sqrt{n} p + 2\sqrt{n} p/2 + \cdots +
%   n^{1/4}\sqrt{n}p/n^{1/4} = n^{3/4}p.$$
% \end{proof}
% \end{proof}

% \section{Try 3 of the insane proof}

% \begin{theorem}
%   $$\sum_{x\in X} \lap(x) \le p\cdot n^{11/12}.$$
% \end{theorem}
% \begin{proof}
%   \begin{claim}
%     $\lap(x) \le x + p/x$
%   \end{claim}
%   \begin{proof}
%     Partition to stacks. 
%   \end{proof}

%   \begin{corollary}
%     $$\sum_{x\in X \cap [\ceil{p/n^{1/12}}]} \lap(x) \le
%     \bigO(pn^{11/12}).$$
%   \end{corollary}
%   \begin{proof}
%     Immediate from above claim.
%   \end{proof}

%   \begin{lemma}
%     Fix $x$, and let $L$ be a valid number of levels for $x$. 
%     Then, 
%     $$\lap(x) \le p/\ell_L + c_L \ell_L.$$
%     \todo{isn't there some noise for non ultimate levels}
%   \end{lemma}
%   \begin{proof}
%     Partition $[p/2]$ into groups of $c_{L+1}$ elements. By
%     [claim proving that circles consist of evenly spaced
%     elements] we incur error at most $\bigO(1)$ per all $\ell_L$
%     points in each circle. Thus each group contributes at most
%     $c_L$ to $\lap(x)$, so the total contribution from all groups
%     together is at most $p c_L / (\ell_L c_L) = p/\ell_L$.
%     Finally, there are at most $c_{L+1}$ elements left over which
%     do not fit into a group. These contribute at most $c_{L+1}$
%     to  $\lap(x)$, which together with the bound on the
%     contribution to $\lap(x)$ from groups gives the desired bound
%     on $\lap(x)$. Note that our estimate of the elements which do
%     not fit in a group is quite weak if $L>1$;
%     However, it is simple and sufficient for our purposes.
%   \end{proof}

%   \begin{claim}
%     $$c_{L} < \ell_L.$$
%     \todo{is this actually true? I think probably not. do I
%     really need it?}
%   \end{claim}
%   \begin{proof}
%     \textbf{Intuition}: $\ell$ at least doubles each time. The hardest
%      $c_L/\ell_L$ ratio you can get is probably if we just barely
%      double each time. But in that case it does hold.
%   \end{proof}

%   \begin{claim} \label{clm:112isfine}
%     For any $x>p/n^{1/12}$, there exists an $L\in \N$ such that
%     $C_{L+1} < p/n^{1/12}.$
%   \end{claim}
%   \begin{proof}
%     $p/x \approx n^{1/12} \ll p/n^{1/12}$.
%   \end{proof}

%   \begin{defin}
%     We say that $x$ is $z$-bad if there is a valid level $L\in N$
%     such that $\ell_L(x) = z$, $c_{L+1} < p/n^{1/12}$ but
%     $c_{L+2} > p/n^{1/12}.$

%     By \cref{clm:112isfine} any $x>p/n^{1/12}$ is $z$-bad for
%     some $z\in [p]$.
%   \end{defin}

%   \begin{lemma}
%     The number of $z$-bad integers $x\in [p/n^{1/12}, p]$ is at
%     most $z^{3}n^{1/6}$.
%   \end{lemma}
%   \begin{proof}
%     For $x$ to be $z$-bad we must have 
%     $\ell_{L+1}\ell_L c_L > p/n^{1/12}.$
%     Of course $\ell_{L+1} \approx p/\delta_{L+1}$, 
%     so this translates into 
%     $$\delta_{L+1} < z^2n^{1/12}.$$
%     On the other hand, 
%     $\delta_L = c_L \cdot x$, and so 
%     $$\delta_{L+1} = \floor{p/\delta_{L}} \delta_L= \paren{p/(c_L x)
%     c_L }x.$$
%     Of course $p/(c_L x) c_L < n^{1/12}z$, 
%     so we obtain the equation
%     $$\eta x \equiv \delta$$
%     where $\eta < n^{1/12}z$, $\delta< z^{2}n^{1/12}$. 
%     Clearly, there are at most $n^{1/6}z^{3}$ solutions to this
%     equation.

%     Intuitively, this is expressing the fact that a quick drastic
%     decrease in granularity is fairly rare. 
%   \end{proof}

%   \begin{corollary}
%     $$\sum_{x\in X\cap [p/n^{1/12}, p]} \lap(x) \le pn^{11/12}.$$
%   \end{corollary}
%   \begin{proof}
%     The worst case is 
%     $$\sum_{k=1}^{n^{1/4}} pn^{1/6}z^{3}/z = pn^{11/12}.$$
%   \end{proof}

%   Combining Corollary[A] and Corollary[B] gives the desired bound
%   on $\sum_{x\in X}\lap(x)$.

% \end{proof}

% \section{Another remark about circles}
% Imagine that you have some set of $m_i,k_i$ such that 
% $xm_i\equiv k_i$  for each $i$. 
% Imagine further that you could form a partition 
% $$p/2 = \sum_i \zeta_i \floor{p/k_i} m_i.$$
% Then you can view $p/2$ as being the union over $i$ of $\zeta_i m_i$  circles of
% granularity $k_i$.

% In fact, we believe that the maxload of $\MLH$ is bounded by
% $n/2+\bigO(\sqrt{n} \log^2 n)$. We leave strengthening
% \cref{cor:kindaepicevenifweak}, and thereby obtaining a better
% bound on $\MLH$'s maxload, as an open problem.
% \begin{conj}
%   \label{conj:log2nwouldbenicetoohard}
%   $$\sum_{x\in X} \lap(x) \le\bigO( p \cdot \log^2 n).$$
% \end{conj}


\section{Composite Moduli}
\label{sec:Z}
In this section we study the effect of replacing the standard
prime modulus in $\LH$ with a composite modulus.
In addition to being an intrinsically interesting question, we
see in \cref{sec:R} that $\LH$ with composite modulus arises
naturally in the study of $\LH$ over $\R$.
Primes are more well-behaved than composite numbers when used as
moduli because in $\F_p$ all non-zero elements have inverses.
However, \cref{fact:numdivs} and \cref{fact:toitent} indicate that
while $\Z_m$ could have a large quantity of elements with varying
degrees of ``degeneracy'', there are also guaranteed to
be a substantial number of relatively well-behaved elements.

To bound the extent to which $\Z_m$  is worse than $\F_p$ we
begin by defining \defn{Smart $\LH$} ($\SLH$) where the multiplier is chosen
randomly from $\Z_m^\times$ rather than $\Z_m$. Then, we show
that the maxload of modulus $m$ $\LH$ is at most
$\tau(m)$-times larger than that of modulus $m$ $\SLH$.
We finish the section by demonstrating that the standard $\bigO(\sqrt{n})$ maxload
bound for prime moduli $\LH$, and even Knudsen's beautiful
$\widetilde{\bigO}(n^{1/3})$ bound \cite{knudsen_linear_2017} can be translated with
several modifications to the composite integer setting.


Now we formally discuss our hash functions.
\begin{defin}
  Fix appropriate $n\in \N, m\in \Z$.
  We define two hash families parameterized by $a\in
    [m]_{\setminus 0}$ consisting of functions
  $h_{a} : [m] \to [n]$.
  \begin{enumerate}
    \item In \defn{Blocked Hashing}, denoted $\blockZm$,
          $h_{a}(x) = \floor{\frac{\posmod_m(ax)}{m / n}}.$
    \item In \defn{Strided Hashing}, denoted $\strideZm$,
          $h_{a}(x)= \posmod_n(\posmod_m(ax)).$
  \end{enumerate}
  % The \defn{bin size} is $m/n$, which corresponds (ignoring
  % rounding) to the number of $x\in [m]$ that map to any single value. 
  % A \defn{bin} refers to the elements that hash to the same bin.
  % For instance, in blocked hashing bins are of the form 
  % $$[m] \cap [i\cdot m/n, (i+1)\cdot m/n).$$

  We define $\blockFp, \strideFp$ to be  $\blockZm,\strideZm$ for
  $m=p\in \PRM$.
\end{defin}

% A priori, blocked and strided hashing seem somewhat different. 
In \cite{knudsen_linear_2017} Knudsen gives the necessary idea to
show an equivalence up to a factor-of-$2$ between $\blockFp$ and
$\strideFp$; this fact also follows immediately from our
\cref{prop:blockZisok}.
For composite integers the situation is
more delicate. In particular, if $\gcd(m,n)$ is large then
$\strideZm$ behaves extremely poorly for some
$X$ while $\blockZm$ does not.

\begin{prop}\label{prop:blockZsucks}
  Let $m=k\cdot n$ for some $k>n$.
  There exists an $n$-element set
  $X\subset [m]$ on which $\strideZm$ has maxload  $n$.
\end{prop}
\begin{proof}
  Let  $X=n\cdot [n]$. Then  $\posmod_n(\modm(ax)) = 0$
  for all $x\in X$ regardless of $a$. Thus, all $x\in X$
  always hash to bin $0$ so the maxload is $n$ deterministically.
\end{proof}

On the other hand, as long as $\gcd(m,n)$ is small then
$\strideZm,\blockZm$ achieve similar maxload.
\begin{prop}\label{prop:blockZisok}
  Let $m\perp n$. For any $n$ element set $X\subset [m]$ the expected maxload of
  $\strideZm$ and $\blockZm$ on $X$ differ by at most a factor-of-$2$.
\end{prop}
\begin{proof}
  Because $m,n$ are coprime $n$ has a multiplicative inverse
  $n^{-1}\in \Z_m$.
Assume $Y\subset X$ is the set of elements mapping to a fullest bin under
$\blockZm$ with $a=a_0$. We claim that for $a=\modm(n\cdot a_0)$, $\strideZm$
has maxload at least $|Y|/2$.
Indeed, let $k$ be the bin that $Y$ maps to under $a_0$.
We have 
\[ \ceil{k\frac{m}{n}} \le \modm(Ya_0) < (k+1)\frac{m}{n}. \] 
Consider the set 
 \[  Y' = n\cdot\left(\ceil{k\frac{m}{n}}+[\ceil{m/n}]\right). \] 
The difference between the maximum and the minimum elements of $Y'$ is at most
$m$. Thus, $\posmod_n(\modm(Y'))$ takes at most two distinct values.
In particular this implies that $\posmod_n(\modm(Ya_0n))$ takes on at most two
values, so there is a subset of $Y$ with size at least $|Y|/2$ that all hash to
the same bin under  $a=\modm(n a_0)$.

  Assume $S\subset X$ is the set of elements mapping to a fullest bin under
$\strideZm$ with $a=a_0$. Then, by similar reasoning to the above case, for
$a=\modm(n^{-1}\cdot a_0)$ $\blockZm$ has maxload at least  $|S|/2$.

  Multiplication by $n$ or $n^{-1}$ modulo $m$ permutes $\Z_m$. The result follows.
\end{proof}

\cref{prop:blockZisok} and \cref{prop:blockZsucks} teach us that
for composite integer $\LH$ it is more robust to consider
$\blockZm$ than $\strideZm$, but essentially equivalent as long
as $\gcd(m,n)$ is small. For the remainder of the paper we
restrict our attention to $\blockZm$ and $\blockFp$ which we
abbreviate to $\ZH_m$, $\FH_p$.
Now we formally define the variant of $\ZH_m$ that partially
solves the problem of $m$ being composite.
\begin{defin}\label{defn:slh}
  In \defn{Smart $\LH$} ($\SLH_m$) we randomly select
  $a\in \Z_m^{\times }$ and place $x\in [m]$ in bin $\floor{\frac{\modm(ax)}{m/n}}.$
\end{defin}
Surprisingly, we will show that the performance of $\ZH_m$ is not
too far from that of $\SLH_m$, especially if $m$ has few divisors.
We use the following notation:

\begin{defin}
  Let random variable $M_{\ZH}(m,X)$ denote the maxload incurred by
  $\ZH_m$ on $X$, and let $M_{\ZH}(m,n)$ denote the worst-case expected value of
  $M_{\ZH}(m,X)$ over all $n$-element sets $X\subset [m]$.
  Analogously define $M_{\SLH}(m,X), M_{\SLH}(m,n)$.
\end{defin}

\begin{theorem}\label{thm:LHSLH}
  Fix $m\ge n^{6}$ with $m\in \poly(n)$.
  Let $f$ be a monotonically increasing concave function with
  $M_{\SLH}(m_0,n_0)\le f(n_0)$ for all $n_0$ and all $m_0\ge n^{2.5}$.
  Then
  \[
  M_{\ZH}(m,n) \le \tau(m) \cdot f(n)+1.
\]
\end{theorem}
\begin{proof}
  Fix any $n$-element set $X\subset [m]$.
  % Let $p\in \PRM$. Then, $$\Pr[\nu_p(a)=i] \le 1/p^i.$$
  For $d\mid m$, define $\setof{I_{i,d}}{i\in [d]}$ as the
  following partition of $[m]$ into $d$ size $m/d$ blocks:
  \[I_{i,d} = i\cdot m/d + [m/d].\]
  Define $X_{i,d} = X\cap I_{i,d}$
  and let $G_d$ be the event  $\gcd(a,m)=d$.
  We will bound the expected maxload of $\ZH_m$ by conditioning
  on $G_d$. However, if $\gcd(a,m)$ is very large then  $\ZH_m$ will
  necessarily incur large maxload; thus, we first exclude
  this case by showing it is very unlikely.
  For any $d\mid m$
  \begin{equation}\label{eq:prGd}
    \Pr[G_d]\le 1/d
  \end{equation}
  because there are $m/d$ multiples of $d$ in $[m]$.
  There are at most $n^{2.5}$ divisors $d\mid m$ with  $d\ge
    m/n^{2.5}$, because such divisors are in bijection with
  divisors $d'\mid m$ satisfying  $d'\le n^{2.5}$.
  By \cref{eq:prGd} each of these large divisors
  $d$ has $\Pr[G_d] \le n^{2.5}/m$. Thus we have:
  \begin{equation}\label{eq:gcdbignoway}
    \Pr[\gcd(a,m) \ge m/n^{2.5}] \le n^{2.5}\cdot \frac{n^{2.5}}{m} \le
    \frac{1}{n},
  \end{equation}
  where the final inequality follows by the assumption that $m\ge n^{6}$.
  By \cref{eq:gcdbignoway} the case $\gcd(m,a)\ge m/n^{2.5}$ contributes at most $1$ to
  the expected maxload.
  Let
  \[D_m = \setof{d\mid m}{d<m/n^{2.5}}.\]
  We claim the following chain of inequalities:

\begin{minipage}{0.4\textwidth}
\begin{align}
     &\E[M_{\ZH}(m,X) \mid \gcd(m,a)\in D_m] \nonumber \\
     &= \sum_{d\in D_m}\Pr[G_d]\cdot \E[M_{\ZH}(m,X) \mid G_d] \label{eqchainSLH1} \\
     &\le \sum_{d\in D_m} \frac{1}{d}\cdot \E[M_{\ZH}(m,X) \mid G_d] \label{eqchainSLH2} \\
     &\le \sum_{d\in D_m}\frac{1}{d}\sum_{i\in [d]}\E[M_{\ZH}(m,X_{i,d}) \mid G_d]\label{eqchainSLH3} \\
     &\le \sum_{d\in D_m}\frac{1}{d}\sum_{i\in [d]}\E[M_{\SLH}(m/d,X_{i,d})] \label{eqchainSLH4} \\
     &\le \sum_{d\in D_m}\frac{1}{d}\sum_{i\in
    [d]}M_{\SLH}(m/d,|X_{i,d}|) \label{eqchainSLH5} \\
     &\le \sum_{d\in D_m} \frac{1}{d}\sum_{i\in
    [d]}f(|X_{i,d}|) \label{eqchainSLH6} \\
     &\le \sum_{d\in D_m}f(n/d) \label{eqchainSLH7} \\
     &\le \tau(m)\cdot f(n) \label{eqchainSLH8}.
\end{align}
\end{minipage}%
\begin{minipage}{0.6\textwidth}
\begin{itemize}
    \item \cref{eqchainSLH1}: Law of total expectation.
    \item \cref{eqchainSLH2}: $\Pr[G_d]\le 1/d$, because there are $m/d$ multiples of $d$ in $[m]$.
    \item \cref{eqchainSLH3}: We can ``union bound''  because $\bigsqcup_{i\in [d]}X_{i,d} = X.$
    \item \cref{eqchainSLH4}: Recall that $X_{i,d}\subset I_{i,d}$, where
          $I_{i,d}$ is a contiguous interval of size $m/d$.
          Because we are conditioning on $\gcd(m,a)=d$,
          $\modm(a\cdot I_{i,d})$ consists of every $d$-th element of
          $[m]$ starting from $0$, i.e., is $\modm(d[m])$. Having
          elements which are spaced out by $d$ grouped into
          intervals of length $m/n$ per bin is
          equivalent to having elements spaced out by $1$
          grouped into intervals of length $(m/d)/n$ per
          bin. Formally this is because
          $\modm(x\cdot d\cdot j) = d\cdot \posmod_{m/d}(x\cdot j).$
          The restriction  $\gcd(m,a)=d$ can also be expressed as
          $\gcd(m/d, a/d)=1$, i.e., $a/d \perp m/d$.
          Hence, the expected maxload of $\ZH_m$ on $X_{i,d}$ conditional on
          $G_d$ is the same as the expected maxload of $\SLH_{m/d}$
          on $X_{i,d}$.
    \item \cref{eqchainSLH5}: $M_\SLH(m,n)$ is by definition the
          worst-case value of $\E[M_\SLH(m,X)]$ over all $n$-element
          sets $X$.
    \item \cref{eqchainSLH6}: By assumption $f$ is an upper
          bound on $M_{\SLH}$ as long as the modulus $m/d$ is
          sufficiently large. Because $d\in D_m$ we have  $m/d >
            n^{2.5}$, so the upper bound $f$ holds.
    \item \cref{eqchainSLH7}: $f$ is concave.
    \item \cref{eqchainSLH8}: $f$ is increasing, $\tau$ counts
          the divisors of $m$.
\end{itemize}
\end{minipage}

  % \begin{align}
  %    & \E[M_{\ZH}(m,X) \mid \gcd(m,a)\in D_m] \nonumber                                                \\
  %    & = \sum_{d\in D_m}\Pr[G_d]\cdot \E[M_{\ZH}(m,X) \mid G_d]
  %   \label{eqchainSLH1}                                                                                \\
  %    & \le \sum_{d\in D_m} \frac{1}{d}\cdot \E[M_{\ZH}(m,X) \mid G_d] \label{eqchainSLH2}              \\
  %    & \le \sum_{d\in D_m}\frac{1}{d}\sum_{i\in [d]}\E[M_{\ZH}(m,X_{i,d}) \mid G_d]\label{eqchainSLH3} \\
  %    & \le \sum_{d\in D_m}\frac{1}{d}\sum_{i\in [d]}\E[M_{\SLH}(m/d,X_{i,d})] \label{eqchainSLH4}      \\
  %    & \le \sum_{d\in D_m}\frac{1}{d}\sum_{i\in
  %   [d]}M_{\SLH}(m/d,|X_{i,d}|) \label{eqchainSLH5}                                                    \\
  %    & \le \sum_{d\in D_m} \frac{1}{d}\sum_{i\in
  %   [d]}f(|X_{i,d}|) \label{eqchainSLH6}                                                               \\
  %    & \le \sum_{d\in D_m}f(n/d) \label{eqchainSLH7}                                                   \\
  %    & \le \tau(m)\cdot f(n) \label{eqchainSLH8}.
  % \end{align}
  % We now justify the inequalities.
  % \begin{itemize}
  %   \item \cref{eqchainSLH1}: Law of total expectation.
  %   \item \cref{eqchainSLH2}: $\Pr[G_d]\le 1/d$, because there
  %         are $m/d$ multiples of $d$ in $[m]$.
  %   \item \cref{eqchainSLH3}: We can ``union bound''  because
  %         $$\bigsqcup_{i\in [d]}X_{i,d} = X.$$
  %   \item \cref{eqchainSLH4}: Recall that $X_{i,d}\subset I_{i,d}$, where
  %         $I_{i,d}$ is a contiguous interval of size $m/d$.
  %         Because we are conditioning on $\gcd(m,a)=d$,
  %         $\modm(a\cdot I_{i,d})$ consists of every $d$-th element of
  %         $[m]$ starting from $0$, i.e., is $\modm(d[m])$. Having
  %         elements which are spaced out by $d$ grouped into
  %         intervals of length $m/n$ per bin is
  %         equivalent to having elements spaced out by $1$
  %         grouped into intervals of length $(m/d)/n$ per
  %         bin. Formally this is because
  %         $$\modm(x\cdot d\cdot j) = d\cdot \posmod_{m/d}(x\cdot j).$$
  %         The restriction  $\gcd(m,a)=d$ can also be expressed as
  %         $\gcd(m/d, a/d)=1$, i.e., $a/d \perp m/d$.
  %         Hence, the expected maxload of $\ZH_m$ on $X_{i,d}$ conditional on
  %         $G_d$ is the same as the expected maxload of $\SLH_{m/d}$
  %         on $X_{i,d}$.
  %   \item \cref{eqchainSLH5}: $M_\SLH(m,n)$ is by definition the
  %         worst-case value of $\E[M_\SLH(m,X)]$ over all $n$-element
  %         sets $X$.
  %   \item \cref{eqchainSLH6}: By assumption $f$ is an upper
  %         bound on $M_{\SLH}$ as long as the modulus $m/d$ is
  %         sufficiently large. Because $d\in D_m$ we have  $m/d >
  %           n^{2.5}$, so the upper bound $f$ holds.
  %   \item \cref{eqchainSLH7}: $f$ is concave.
  %   \item \cref{eqchainSLH8}: $f$ is increasing, $\tau$ counts
  %         the divisors of $m$.
  % \end{itemize}

  We have shown the bound \cref{eqchainSLH8} for arbitrary
  $X$, so in particular the bound must hold for worst-case  $X$.
  Adding $1$ the for the event $\gcd(a,m)\ge m/n^{2.5}$ we have
  \[ M_{\ZH}(m,n) \le \tau(m)\cdot f(n)+1.\]
\end{proof}
\begin{rmk}
  \cref{thm:LHSLH} says that increasing concave bounds for
  $M_\SLH$ can be translated to bounds for $M_{\ZH}$ except
  weakened by a factor-of-$\tau(m)$.
  If $m$ is a power of $2$, a natural setting, then
  $\tau(m) = \log m$.
  Even for worst-case $m$ \cref{fact:numdivs} asserts $\tau(m)\le
    m^{o(1)}$. So, $\SLH_m$ and $\ZH_m$ have quite similar
  behavior.
\end{rmk}

Now we analyze the performance of $\SLH_m$. First we give an
argument based on the trivial $\bigO(\sqrt{n})$ bound for $\FH$.
% In \cref{sec:formalpfZm13} we apply similar modifications to
% Knudsen's proof \cite{knudsen_linear_2017} of the
% $\widetilde{\bigO}(n^{1/3})$ along with some new modifications
% specific to Knudsen's proof to translate his
% $\widetilde{\bigO}(n^{1/3})$ bound to $\SLH_m$.

% \begin{prop}
% \todo{this is probably not worth saying twice.}
%   \label{prop:sqrtnZ}
%   $\SLH$ on $\block{\Z_{2^{\ell}}}$ achieves expected maxload at most
%   $\bigO(\sqrt{n})$.
% \end{prop}
% \begin{proof}
%   Say that $x,y \in X, x\neq y$ are \defn{linked} if $\gcd(x-y,2^{\ell}) >
%   2^{\ell}/n$ and \defn{unlinked} otherwise.
%   If $x,y$ are linked, then they can't collide. This is because
%   if $\gcd(x-y,2^{\ell})=2^{j} > 2^{\ell}/n$ then $\gcd(a(x-y)\bmod
%   2^{\ell},2^{\ell}) = 2^{j}$ due to $a$ being odd  because we are doing $\SLH$, so $a$
%   is chosen relatively prime to $2^{\ell}.$
%   But this means that $ax-ay > 2^{\ell}/n$, i.e. $x,y$ fall in
%   separate bins.
%   If $x,y$ are unlinked, then they collide with probability
%   $\bigO(1 / n)$. In particular, $a(x-y)$ will range uniformly
%   over all the bins as long as $\gcd(x-y,2^{\ell}) < 2^{\ell}/n$,
%   so the probability of $x-y$ being sufficiently small is at most
%   say $\frac{2}{n}$.

%   Hence, the expected number of pairs which collide is, by
%   linearity of expectation, $\bigO(n^2 / n) =\bigO(n),$ 
%   meaning that the expected maxload is at most $\bigO(\sqrt{n})$
%   by Jensen's Inequality ($\sqrt{\cdot}$ is concave).
% \end{proof}
\begin{theorem} \label{prop:sqrtnZ}
  $M_{\SLH}(m,n) \le \bigO(\sqrt{n\log\log n}).$
\end{theorem}
\begin{proof}
  We say $x,y$ \defn{collide}, with respect to $a=a_0$,
  if they hash to the same bin for $a=a_0$.
  As in the $\bigO(\sqrt{n})$ bound for $\FH_p$ we bound the
  maxload by counting the expected number of collisions and
  comparing this with the number of collisions guaranteed by a
  certain maxload.
  The difficulty in the proof for $\SLH_m$ is that the
  probability of  $x,y\in X$ colliding is not as simple as in
  $\FH_p$ where all pairs collide with probability $\bigO(1/n)$.
  To handle this we consider two types of pairs $x,y$:
  \begin{defin}
    Distinct $x,y\in X$ are
    \defn{linked} if
    $\gcd(x-y,m) > \ceil{m/n},$
    and \defn{unlinked} otherwise.
  \end{defin}

  \begin{claim}\label{clm:linkedcollidesqrt}
    Linked $x,y$ never collide.
  \end{claim}
  \begin{proof}
    If $\gcd(x-y,m)=k > \ceil{m/n}$ and $a$ is the randomly chosen
    multiplier from $\Z_m^{\times}$ then
    \[
    \gcd(\modm(a\cdot (x-y)),m) = k
  \]
    because $a\perp m$.
    But then
    $\circabs_m(ax - ay) \ge k > \ceil{m/n},$
    so $x,y$ fall in different bins.
  \end{proof}
  \begin{claim}\label{clm:unlinkedcollidesqrt}
    Unlinked $x,y$ collide with probability at most
    $\bigO\left(\frac{\log\log n}{n}\right).$
  \end{claim}
  \begin{proof}
    When $x,y$ are unlinked, $\modm(a(x-y))$ would have probability $\bigO(1/n)$
    of landing in each bin if $a$ were chosen uniformly from $\Z_m$.
    However, in $\SLH_m$ this uniformity is not obvious.
    Fortunately, \cref{fact:toitent} ensures that each $a\in
      \Z_m^{\times}$ occurs with probability at most $\frac{2\log\log
        m}{m}$, which is not much larger than $\frac{1}{m}$.
    In particular, this implies that the probability of $x,y$
    colliding is at most
    \[
    \bigO(1/n)\cdot m \cdot \frac{2\log\log m}{ m} \le
      \bigO\left(\frac{\log\log n}{n}\right).
    \]
  \end{proof}

  Now that we have shown
  \cref{clm:linkedcollidesqrt} and \cref{clm:unlinkedcollidesqrt}
  the proof continues in the same
  way as for $\FH_p$.
  If maxload is $m$ then there must be at least
  $\binom{m}{2} = \Theta(m^{2})$ collisions.
  By Jensen's inequality and the convexity of $x\mapsto x^2$ we
  have $\E[M_{\SLH}(m,X)^2]\ge \E[M_{\SLH}(m,X)]^2.$
  We can also count the expected number of collisions directly
  using \cref{clm:linkedcollidesqrt} and \cref{clm:unlinkedcollidesqrt};
  doing so, we find that the expected number of collisions is
  $\bigO(n \log\log n).$
  Comparing our two methods of counting collisions gives:
  \[
  \E[M_{\SLH}(m,X)]\le \bigO(\sqrt{n\log\log n})
\]
  for any $X$, and in particular for worst-case $X$.
\end{proof}

In \cref{sec:formalpfZm13} we strengthen \cref{prop:sqrtnZ} to 
\begin{theorem}\label{thm:Zm1_3}
  $M_{\SLH}(m,n)\le \widetilde{\bigO}(n^{1/3}).$
\end{theorem}
The proof is a modification of Knudsen's proof
\cite{knudsen_linear_2017} of the corresponding bound for
$\FH_p$ with modifications similar to those used in \cref{prop:sqrtnZ}.

% The proof is 
% Now we strengthen \cref{prop:sqrtnZ} to \cref{thm:Zm1_3}.
% Most of the proof of \cref{thm:Zm1_3} is the same as
% Knudsen's proof \cite{knudsen_linear_2017} for $\FH_p$, but it is difficult to
% black-box Knudsen's result because our modifications permeate
% his whole proof. Thus, we highlight here only the key
% modifications needed to adapt the argument to $\SLH_m$,
% and provide a formal proof in \cref{sec:formalpfZm13}.

% \begin{theorem}\label{thm:Zm1_3}
%   $M_{\SLH}(m,n)\le \widetilde{\bigO}(n^{1/3}).$
% \end{theorem}
% \begin{proof}[Sketch of key modifications]
%   The most important idea for translating
%   Knudsen's proof to composite moduli is
%   already included in the statement of \cref{thm:Zm1_3}: namely,
%   the idea to use $\SLH_m$ rather than $\ZH_m$.
%   This decision is justified by \cref{thm:LHSLH}.
%   Crucially in $\SLH_m$ all choices of $a$ have inverses, a
%   property relied on extensively in Knudsen's proof.

%   In Knudsen's proof he considers a refinement of the collisions
%   used in \cref{prop:sqrtnZ} called \defn{close pairs}. The
%   second key idea to translate Knudsen's proof to $\ZH_m$ is
%   that, with a little case-work into ``linked'' and ``unlinked''
%   pairs like in \cref{prop:sqrtnZ}, we can bound the expected
%   number of close pairs in a similar fashion as in the proof for
%   $\FH_p$.

%   The other required changes are smaller, and generally look like
%   using number theoretic facts such as
%   \cref{fact:numdivs}, \cref{fact:toitent}, and the prime number
%   theorem to bound the ``degeneracy'' of $\Z_m$.
% \end{proof}

\begin{cor}\label{cor:translate}
  $M_{\ZH}(m,n) \le n^{1/3 + o(1)}.$
\end{cor}
\begin{proof}
  This follows immediately from using \cref{thm:Zm1_3} in
  \cref{thm:LHSLH}, which is valid because $n^{1/3}$ is a concave
  and increasing function of $n$.
  \footnote{In general one needs to slightly modify the universe size
    in order to apply \cref{thm:LHSLH}. However, \cref{thm:Zm1_3}
    as proved in \cref{sec:formalpfZm13} only requires the universe
    size $m> n^{2.5}$. Thus, such a modification is not necessary here.}
\end{proof}

% In \cref{thm:Zm1_3} we classified pairs $x,y\in X$ into two groups:
% \begin{itemize}
%   \item $x,y$ are ``linked'' if $\gcd(x-y,m)$ is large. Linked
%     $x,y$ never land in the same bin.
%   \item $x,y$ are ``unlinked'' if $\gcd(x-y,m)$ is small. Unlinked
%     pairs intuitively behave like any pair relative to a prime
%     modulus. In particular, for most unlinked pairs, $x,y$ will
%     be end up in uniformly random and independent bins.
% \end{itemize}
% It seems plausible that the concept of linked and unlinked pairs
% could be used to prove the following generalization of
% \cref{thm:Zm1_3}:
In \cref{thm:Zm1_3} we have translated the state-of-the-art maxload
bound for $\FH_p$ to $\SLH_m$ by altering Knudsen's proof.
This is tentative evidence that composite modulus $\LH$ may
achieve similar maxload to prime modulus $\LH$ in general.
We leave proving or refuting this as an open problem:
\begin{question}\label{question:equivalenceFZ}
  Are the worst-case maxloads of $\FH_p$ and $\SLH_m$ the same
  up to a factor-of-$n^{o(1)}$?
\end{question}
% \cref{conj:translate} seems hard, because rather than translating
% a specific bound we must show a general reduction; we leave this
% as an open problem.
% Combined with \cref{thm:LHSLH}, \cref{conj:translate} would
% imply that maxload relative to composite moduli and prime moduli
% is essentially the same.
% However, it is not obvious how to apply the linked and unlinked
% pair analysis

\section{Hashing with Reals}
\label{sec:R}
% maybe say This is similar to ``Lonely Runner Conjecture" \cite{Tao_2018}

In this section we consider $\LH$ set in $\R$. Formally:
\begin{defin}
  Fix universe size $u\in \N$. As always, we require
  $\Omega(n^{6})\le u \le \poly(n)$ (\cref{rmk:assumesize}). In
  \defn{Blocked real hashing} ($\blockRu$) we randomly select real $a\in
  (0,1)$ and map $x\in [u]$ by 
  $$x\mapsto \floor{\frac{\posmod_1(ax)}{ 1  / n }}.$$ 
  % Strided real hashing $\stride{\R_u}$ is parameterized by a real
  % number $a\in [0,1]$ with hash function 
  % $$x\mapsto \posmod_n(\floor{\posmod_1(ax)}).$$
\end{defin}

One interesting similarity between $\blockRu$ and  $\blockFp$ is
that all $a\in (0,1)$ are invertible modulo $1$ over $\R$ just
as all $a\in \pnozero$ are invertible in $\F_p$. 
In this section we show that $\blockRu$, surprisingly, achieves
intuitively equivalent maxload to the following variant of
$\SLH_m$ with a randomized modulus:
\begin{defin}
  Fix $m\in \N$. In \defn{Random Linear Hashing} ($\RLH_m$) we
  randomly select $k\in [m/2,m]\cap \Z$ and hash with $\SLH_k$'s
  hash function (\cref{defn:slh}).
\end{defin}
\begin{prop}
  % Let $M_{\LH}(m,k)$ denote the worst-case expected maxload of
  % $\block{\Z_k}$.
  $\RLH_m$ has expected maxload at most $$\max_{k\in [m/2,m]\cap
  \Z}  2M_{\SLH}(m,k).$$ 
  % In particular, $\RLH_m$ has expected maxload
  % $n^{1/3+o(1)}$.
\end{prop}
\begin{proof}
  Fix $X$, condition on some $k$. Partition $X$ into 
  $$X_1 = X\cap [k],\;\; X_2=X \cap [k,2k].$$
  The maxload on $X_1,\posmod_k(X_2)\subset [k]$ individually is at most $M_{\SLH}(m,k)$.
  Adding the maxload on $X_1$ and $X_2$ gives an upper bound on the
  total maxload.
 %  The first statement is immediate. The second statement
 % follows from \cref{cor:translate} applied to the first statement.
 % \todo{check}
\end{proof}

Now we connect $\blockRu$ and $\blockZm, \RLH$.
\begin{theorem}
  \label{thm:itisreal}
  % Fix $p,n\in \N$ with $\Omega(n^{4})\le p\leq \poly(n)$, and fix
  % $X\subset [p]$ with $|X|=n$.
  Let $M_{\RLH}$ denote the expected maxload of
  $\RLH_{\floor{\sqrt{nu}}}$ on a worst-case $X\subset
  [\floor{\sqrt{nu}}]$, let $M_{\max}$ denote the maximum over
  $t\in (\sqrt{u}, nu]\cap \Z$ of the expected maxload
  of $\SLH_t$ for worst-case $X\subset [t]$. Let $M_\R$ denote
  the expected maxload of $\blockRu$ for worst-case $X\subset
  [u].$ Then, 
  $$ \Omega\paren{\frac{M_{\RLH}}{\log\log n}} \le M_\R \le \bigO(M_{\max}).$$
\end{theorem}
The link between integer hashing and real hashing begins to
emerge in the following lemma:
\begin{lemma}\label{lem:restrictQ}
  Fix $k\in \N$. Assume $a=c/k$ for random $c\in \Z_k^{\times}$ and take
  $X\subset [u]$. $\blockRu$ conditional on such $a$ achieves
  the same maxload on $X$ as hashing $X$ with $\SLH_k$'s hash
  function using multiplier $c$.
\end{lemma}
\begin{proof}
  % The function $\tilde{h}(x)= \posmod_1(x\cdot a)$ has period $k$ because
  % $\posmod_1(k\cdot c /k) = 0$. Furthermore, for $i\in [k]$ we
  % have $\posmod_1(ic/k) = \posmod_k(ic)/k$.
  For any $x\in [u]$ the value $\posmod_1(xc/k)$ is an integer multiple of
  $1/k$. In particular, $\blockRu$ places $x$ in bin
  \begin{equation}\label{eq:thatbin}
  \floor{\frac{\posmod_1(x c/k)}{1/n}} = \floor{\frac{\posmod_k(x c)}{k/n}}.
  \end{equation}
  $\SLH_k$'s hash function places $x$ in the same bin
  as \cref{eq:thatbin}.
\end{proof}

In isolation \cref{lem:restrictQ} is not particularly useful
because $a\in \Q$ occurs with
probability $0$. However, in \cref{clm:approxepx} we show that if
$a$ is very close to a rational number then we get approximately
the same behavior as in \cref{lem:restrictQ}.

  \begin{lemma}\label{clm:approxepx}
    Fix $k\in [nu]_{\setminus{0}}$.
    Let $a = c/k+\eps$ for coprime $c,k\in \N$ and $\eps <
    \frac{1}{nu}$. Let $a'=c/k$.
    The maxload achieved by $\blockRu$ using $a$ and
    $a'$ differ by at most a factor-of-$2$.
  \end{lemma}
  \begin{proof}
    For any $i\in [u]$, $\eps\cdot i< 1/n$, where $1/n$ is the ``bin
    width", i.e., the size of a sub-interval of $(0,1)$ which
    corresponds to a single bin after multiplication by $n$ and
    flooring.
    Thus, $ai$ and $ai'$ either land in the same bin or $a'i$
    lands in the bin directly after the bin of $ai$.
    Say that bin $j$, a fullest bin for $a$, has $M$ elements. 
    Then, either bin $j$ or bin $\posmod_n(j+1)$ has at least
    $M/2$ elements for $a'$.
    If bin $j'$, a fullest bin for $a'$, has $M'$
    elements then either bin $j'$ or bin $\posmod_n(j-1)$ has at
    least $M'/2$ elements for $a$.
  \end{proof}
  All ${a\in (0,1)}$ will be within $\frac{1}{nu}$ of some $a'\in
  \Q$, and in particular some fraction with denominator at most $nu$.
  This motivates the following definition:

  \begin{defin}\label{defn:fclaims}
    For $k\in [nu+1]$ we define ${I(k) \subset (0,1)}$ to be the
  set of ${a\in (0,1)}$ which are at most $\frac{1}{nu}$ larger
  than some reduced fraction with denominator $k$. That is,
  $$I(k) = \bigcup_{c\in \Z_k^{\times}} \paren{c/k +
  \left[0,\frac{1}{nu}\right]} \cap (0,1).$$ We say that $k$
  \defn{claims} the elements of $I(k)$. 
  For $a\in (0,1)$ let $f(a)$ denote the fraction $c/k$ of
  smallest denominator for which $k$ claims $a$, i.e., $a\in
  I(k)$. 
  For $a\in I(k)$ we say that  $k$ \defn{obtains} $a$ if $f(a)=k$
  and we say that  $a$ is \defn{stolen} from $k$ if $f(a)<k$.
  % Clearly 
  % $$\setof{a\in (0,1)}{f(a)=k} \subseteq I(k).$$
\end{defin}

Intuitively $f(a)=k$ means that $a$ behaves like integer hashing
with modulus $k$. Thus, to obtain bounds on $M_\R$ we seek to
understand $f$.
  \begin{lemma}\label{clm:prdistrunderstand}
    For $k\le \floor{\sqrt{nu}}$, 
    $$\Pr[f(a)=k] = \frac{\phi(k)}{nu}.$$
    % Furthermore, 
    % $\Pr[f(a) < \sqrt{nu}] \ge 0.3$ \\
    % and $\Pr[f(a)>nu] = 0$. 
  \end{lemma}
  \begin{proof}
    Immediately from \cref{defn:fclaims}
    \begin{equation}\label{eq:fakprub}
  \Pr[f(a)=k]\le |I(k)| = \frac{\phi(k)}{nu}.
    \end{equation}
  % because there are $\phi(k)$ fractions
  % $c/k$ with $c\perp k$, each of which has an interval of size
  % $\frac{1}{nu}$ in which 
  % which correspond to
  % disjoint intervals of length $\frac{1}{nu}$ in $[0,1]$
  % containing the numbers close to fractions of denominator $k_0$. 
For distinct $k_1,k_2\le \floor{\sqrt{nu}}$ and appropriate numerators
$c_1,c_2$ 
we have
\begin{equation}
  \label{eq:disjointdudes}
\abs{\frac{c_1}{k_2}-\frac{c_2}{k_1}}>\frac{1}{nu}
\end{equation}
because $k_1k_2 < nu$ while $c_1k_2-c_2k_1
\in \Z\setminus\set{0}$.
\cref{eq:disjointdudes} means that for any $k\le
\floor{\sqrt{nu}}, a\in I(k)$, $a$ will not be stolen from $k$,
because all different fractions of denominator $k'<k$ are
sufficiently far away from all reduced fractions of denominator
$k$. In other words,
\begin{equation}\label{k1k2disjoint}
  I(k_1) \cap I(k_2) = \varnothing.
\end{equation}
\cref{k1k2disjoint} implies that \cref{eq:fakprub} is tight for $k\le
\floor{\sqrt{nu}}$, which gives the desired bound on $\Pr[f(a)=k]$.
% Furthermore, using the well-known value for the sum of $\phi(k)$
% (\cite{hardy1979introduction}) we have
% $$\lim_{m\to \infty} \sum_{k<\sqrt{m}} \frac{\phi(k)}{m}
% = \frac{3}{\pi^2}>0.3,$$
% so $\Pr[f(a)< \sqrt{nu}] \ge 0.3$ (by
% \cref{rmk:assumesize} where we assume $n$ to be at least a
% sufficiently large constant).

% We also note that $f(a)>nu$ is impossible, because any $a\in
% [0,1]$ is within $\frac{1}{nu}$ of a fraction with denominator
% $nu$. 
  \end{proof}

  Using \cref{clm:prdistrunderstand} and the sum
  (\cite{hardy1979introduction}) 
  $$\lim_{m\to \infty} \sum_{k<\sqrt{m}} \frac{\phi(k)}{m} = \frac{3}{\pi^2},$$ 
  we find that $f(a) \in \Theta(\sqrt{nu})$ with probability
  $\Omega(1)$. Intuitively this already ensures $\blockRu$
  behaves similarly to $\RLH$. We formalize this in the following
  bound:
% so $\Pr[f(a)< \sqrt{nu}] \ge 0.3$ (by
% \cref{rmk:assumesize} where we assume $n$ to be at least a
% sufficiently large constant).
\begin{cor}
  \label{cor:lb}
  % Let $M_\R, M_{\RLH}$ denote the expected maxloads
  % from the theorem statement. We have
  $$M_\R \ge \frac{M_{\RLH}}{20\log\log n}.$$
\end{cor}
\begin{proof}
  Using \cref{fact:toitent} on \cref{clm:prdistrunderstand} for
  integer ${k\in [\floor{\sqrt{nu}}/2, \floor{\sqrt{nu}}]}$ gives
  \begin{align}
    \Pr[f(a)=k] &\ge \frac{1}{nu}\frac{\floor{\sqrt{nu}}/2}{2
    \log\log (\floor{\sqrt{nu}}/2)} \\
                &\ge \frac{1}{\floor{\sqrt{nu}}} \cdot \frac{1}{5\log\log
                n}\label{eq9999},
  \end{align}
  where the simplification in \cref{eq9999} is due to the
  asymptotic nature of our analysis (\cref{rmk:assumesize}).
  Fix ${X\subset [\floor{\sqrt{nu}}]\subset [u]}$. We make two observations:
  \begin{itemize}
    \item Each modulus $k \in [\floor{\sqrt{nu}}/2,
      \floor{\sqrt{nu}}]\cap \Z$ is selected by
      $\RLH_{\floor{\sqrt{nu}}}$ with probability
      $2/\floor{\sqrt{nu}}$ which is at most $10 \log\log n$
      times larger than $\Pr[f(a)=k]$. \item Conditional on
      $f(a)=k$ $\blockRu$ achieves expected maxload at least
      $1/2$ of the expected maxload of $\RLH_{\floor{\sqrt{nu}}}$
      conditional on $\RLH$ having modulus $k$; this follows by
      combining \cref{clm:approxepx} and \cref{lem:restrictQ}. 
  \end{itemize}
  Combining these observations gives the desired bound on
  $M_\R/M_\RLH$.
\end{proof}

Now we aim to show an upper bound on $M_\R$.
\cref{clm:approxepx} combined with \cref{lem:restrictQ} show
that $\blockRu$ is equivalent to using the $\SLH$ hash function
but with a modulus chosen according to some probability
distribution $f$; we call $f(a)$ the \defn{effective integer
modulus}.
However, the distribution $f$ of the effective integer modulus is
very different from the distribution of moduli for $\RLH$. One
major difference is that in $\RLH_m$ the randomly selected modulus is always
within a factor-of-$2$ of the universe size $m$. However, in
$\blockRu$ the effective integer modulus is likely of size
$\Theta(\sqrt{nu})$ which is much smaller than the universe
size $u$. A priori this might be concerning: could some choice
of $X$ result in many items hashing to the same bin by
virtue of being the same modulo the effective integer modulus?
Fortunately, \cref{lem:nothingcollides} asserts that this is quite unlikely. 
Before proving \cref{lem:nothingcollides} we need to obtain more
bounds on $f$. To obtain more bounds of $f$ we make use of \defn{Farey
Sequences} (see \cite{hardy1979introduction}).

\begin{defin}
  For $k\in \N$, a \defn{$k$-fraction} is some rational $c/k \in
  [0,1]$ with $c\in \Z, c\perp k$.
  For $m\in \N$, the \defn{$m$-Farey sequence} $\mathfrak{F}_m$
  is the set of all $k$-fractions for
  all $k\le m$ listed in ascending order.
  For example $\mathfrak{F}_5$ is the sequence
  $$\frac{0}{1}, \frac{1}{5}, \frac{1}{4}, \frac{1}{3}, \frac{2}{5}, \frac{1}{2}, \frac{3}{5}, \frac{2}{3}, \frac{3}{4}, \frac{4}{5},
  \frac{1}{1}.$$
\end{defin}

A fundamental property of the Farey sequence concerns the
difference between successive terms of the sequence (proved in
chapter 3 of \cite{hardy1979introduction}).
\begin{fact}\label{fact:fareydist}
  Fix $m\in \N$. Let $c/k, c'/k'$ be adjacent fractions in
  $\mathfrak{F}_m$.
  Then 
  $$\abs{\frac{c}{k} - \frac{c'}{k'}} = \frac{1}{kk'}.$$
\end{fact}

We further classify the neighbors of Farey fractions in the
following lemma:
\begin{lemma}\label{lem:fareyneighbors}
  Fix $m\in \N$. Let $S$ be the set of successors of
  $m$-fractions in $\mathfrak{F}_m$. 
  For each $\ell\perp m$, there is precisely one $\ell$-fraction in
  $S$. For $\ell\not\perp m$ there are no $\ell$-fractions in $S$.
\end{lemma}
\begin{proof}
  Fix $\ell\in [m], \ell\perp m$.
  Then, the equation $\ell i\equiv 1\mod m$ has a solution $i\in
  \Z_m^{\times }$. Let $\lambda \in \N$ so that $i\ell = \lambda m
  + 1$. Then,
  $$\frac{i}{m} = \frac{\lambda}{\ell}+\frac{1}{m \ell}.$$
  By \cref{fact:fareydist} the successor of $\frac{\lambda}{\ell}$ is at
  least  $\frac{1}{m \ell}$ larger than $\frac{\lambda}{\ell}$. Hence,
  there are no fractions in $\mathfrak{F}_m$ between
  $\lambda/\ell$
  and  $i/m$. That is,  $i/m$ is the successor of $\lambda/\ell$.
  However, there are only $\phi(m)$  $m$-fractions in
  $\mathfrak{F}_m$ and we have already identified  $\phi(m)$
  distinct fractions as successors of $m$-fractions. So for $\ell\not\perp k$
  there are no $\ell$-fractions in $S$.
\end{proof}

Now analyze $\Pr[f(a) = k]$ using Farey sequences.
\begin{lemma}\label{lem:fareyopfak}
  $$\Pr[f(a)=k]\leq \bigO\paren{\frac{1}{\sqrt{nu}}}.$$
\end{lemma}
\begin{proof}
  For $k\le \floor{\sqrt{nu}}$ the conclusion is immediate by
  \cref{clm:prdistrunderstand}. Fix integer $k\ge \sqrt{nu}$.
  To bound the measure of $\setof{a\in (0,1)}{f(a)=k}$ we can
  take the measure claimed by $k$ and subtract the measure
  stolen by any $k'<k$.

  % Recall from \cref{defn:fclaims} that $f(a)=k$ exactly when a
  % $k$-fraction claims $a$ but not $i$-fraction for $i<k$ claims
  % $a$. Thus, to bound $\Pr[f(a)=k]$ we consider the
  % area claimed by $k$ minus the area claimed by both $k$ and some
  % other fraction in $\mathfrak{F}_{k}$. We refer to area claimed
  % by both a $k$-fraction and some $i$-fraction for $i<k$ as
  % \defn{stolen} area, and the area claimed by some $k$-fraction
  % but no $i$-fraction for $i<k$ as \defn{obtained}. 

  Fix a $k$-fraction $c/k$. The interval touching $c/k$ claimed
  by $k$ is $c/k+[0,1/(nu)]$.
  The amount of this interval which is stolen is determined by
  the distance from $c/k$ to $c/k$'s successor in $\mathfrak{F}_k$.
  Let $i<k$ denote the denominator of $c/k$'s successor.
  As depicted in \cref{fig:fareypf} there are two cases:
  \begin{itemize}
\item If the successor is close to $c/k$ then all but
  $\frac{1}{ik}$ of the area is stolen, where $\frac{1}{ik}$ is
  the distance to the successor by \cref{fact:fareydist}.
  \item If the successor of $c/k$ occurs after distance more than
  $\frac{1}{nu}$ then it will not steal any area. 
  \end{itemize}

  % Figure environment removed

  Now, we bound the measure obtained by $k$ by summing over the two
  cases represented in \cref{fig:fareypf}.
  \begin{align}
    \Pr[f(a)=k] &\le\frac{nu}{k} \cdot \frac{1}{nu} + \sum_{i=
    \floor{nu/k}}^k \frac{1}{ik} \\
                &\le \paren{1+\ln \paren{\frac{k^2}{nu}}}
                \frac{1}{k}.\label{eq:lnudk2}
  \end{align}
  Let $k=\alpha\sqrt{nu}$ for some $\alpha\ge 1$. 
  Using $\alpha$ in \cref{eq:lnudk2} gives:
  $$\frac{1+2\ln\alpha}{\alpha} \cdot \frac{1}{\sqrt{nu}} \le
  \bigO\paren{\frac{1}{\sqrt{nu}}},$$
  the desired bound on $\Pr[f(a)=k]$.
  
\end{proof}

Now we are prepared for the following lemma:
\begin{lemma}\label{lem:nothingcollides}
  With probability at least $1-1/n$ all
  pairs $x,y\in X$ with $x\neq y$ satisfy 
  $$x\not\equiv y \mod f(a).$$
\end{lemma}
\begin{proof}
  Take distinct $x,y\in X$. We say $x,y$ \defn{collide} if
  $x\equiv y \bmod f(a)$. If $x,y$ collide we must 
  have ${f(a) \mid (x-y)}$. By \cref{fact:numdivs}, $x-y$ has at most
  $u^{o(1)}$ divisors. By \cref{lem:fareyopfak} $f(a)$ will be one
  of these divisors with probability at most
  $u^{o(1)}/\sqrt{nu}$. That is, $x,y$ collide with probability
  at most $u^{o(1)}/\sqrt{nu}$.

  By linearity of expectation the expected
  number of pairs $x,y$ which collide $\bmod \;f(a)$ is at most
  $$\frac{\binom{n}{2}u^{o(1)}}{\sqrt{nu}} \le
  \frac{n^{2}n^{o(1)}}{n^{7/2}} \le \frac{1}{n^{3/2-o(1)}} \le
  \frac{1}{n},$$ 
  which we have simplified using \cref{rmk:assumesize}.
  The number of colliding pairs is a non-negative
  integer random variable. Thus, by Markov's inequality the
  probability of having at least $1$ collision is at most $1/n$.
  Equivalently, with probability at least $1-1/n$ there are $0$
  colliding pairs.
\end{proof}

\begin{cor}\label{cor:ub}
  % For $M_\R,M_{\max}$ from the theorem statement we have
$$M_\R \le \bigO(M_{\max}).$$
\end{cor}
\begin{proof} If the effective integer modulus of $\RLH$ is very
  small then $\RLH$ may perform quite poorly. Luckily, $f(a)$ is
  likely not too small. By \cref{clm:prdistrunderstand} we have
  $$\Pr[f(a) \le \sqrt{u}] \le \sum_{k\le \sqrt{u}}\frac{k}{nu}
  \le 1/n,$$
  so the contribution to $M_\R$ of $a$ with $f(a)\le \sqrt{u}$ is
  $\bigO(1)$. Thus, it suffices to consider $f(a) \in (\sqrt{u},
  nu]$. 
  Fix $X\subset [u]$, and condition on $f(a)=k$ for some $k$.
  By \cref{clm:approxepx} and \cref{lem:restrictQ} the expected
  maxload of $\blockRu$ on  $X$ is at most twice the expected
  maxload if we hash $X$ with $\SLH_k$'s hash function. 
  Furthermore, by \cref{lem:nothingcollides} the expected maxload
  of hashing $X$ with $\SLH_k$'s hash function is the same as the
  expected maxload of using $\SLH_k$ on $\mod_k(X)\subset [k]$.
  This yields the desired bound on $M_\R$.
\end{proof}

Together \cref{cor:lb} and \cref{cor:ub} prove \cref{thm:itisreal}.
% \begin{rmk}
%   Morally speaking, \cref{thm:itisreal} says that $\blockRu$ is
%   equivalent to $\RLH_{\sqrt{un}}$, up to low-order factors.
% \end{rmk}
% \begin{cor}
%   Hashing $x_1,\ldots, x_n$ which are chosen independently and
%   uniformly randomly from $[p]$ using $\block{\R_p}$ gives
%   maxload  $\Theta(\log n / \log\log n)$.
% \end{cor}
As a bonus, applying \cref{thm:Zm1_3} to \cref{thm:itisreal} gives:
\begin{cor}
The expected maxload of $\blockRu$ is $$\widetilde{\bigO}(n^{1/3}).$$
\end{cor}



\paragraph{Acknowledgements}
The author thanks William Kuszmaul and Martin Farach-Colton for
proposing this problem and for helpful discussions.
% \bibliographystyle{plain}
% \bibliography{refs}
\medskip
\printbibliography
\medskip
\appendix

\section{Full proof of \cref{thm:Zm1_3}}
\label{sec:formalpfZm13}

In this section we provide the full proof of \cref{thm:Zm1_3},
translating \cite{knudsen_linear_2017}'s bound from $\blockFp$ to
the composite moduli case, i.e., proving $$M_{\SLH}(m,n)\le \widetilde{\bigO}(n^{1/3}).$$
Although most of ideas in the proof are similar, it is difficult to
black-box the \cite{knudsen_linear_2017}'s result because our modifications permeate
the whole proof. Thus, for sake of formality and the reader's convenience we
provide a full proof here.

\begin{proof}[Proof of \cref{thm:Zm1_3}]
  Fix integer $\alpha < n/4$, and let $\eps = \Pr[M > 4\alpha]$. We
  will show that if $\alpha$ is sufficiently large, in particular
  $\alpha > \Omega(n^{1/3}\log n)$, then $\eps$ is very small.
  Define $\mathcal{A}\subset [m]$ to be the set of \defn{bad} $a_0$'s,
  i.e., $a_0$'s which make the maxload exceed $4\alpha.$
  Clearly $|\mathcal{A}| / |\Z_m^{\times}| = \eps$.

  Throughout this proof we will write $x^{-1}$ to mean the
  multiplicative inverse of $x$ modulo $m$.

  Define a \defn{pre-bin} to be the pre-image of a bin, i.e., a
  set of $m/n$ contiguous values in $[m]$, aligned to some bin
  boundary. We say that $a_1,a_2 \in \Z_m^{\times}$ are
  \defn{close} if $$\circabs_m(a_1 - a_2) \le
  \frac{m}{n\alpha}.$$ 
  Let $w = \floor{\frac{m}{n\alpha}}$. We call an interval of
  length $w$ as \defn{tiny} interval.
  To bound the number of bad $a_0$'s, we analyze the number of
  \defn{close pairs}. Note that closeness is a refinement of the
  property of lying within the same pre-bin, which we used in
  \cref{prop:sqrtnZ}. Intuitively, close
  pairs lie within a $(1/\alpha)$-fraction contiguous portion of
  a pre-bin, although technically a bin boundary may split a set
  of close pairs into two pre-bins. In fact, we will count close
  pairs by analyzing groups of $\alpha$ adjacent intervals each
  of size $w$; in other words, each group will be
  a partition of a pre-bin into small intervals.
  By convention close pairs are unordered, i.e., we do not count
  $b,c$ and $c,b$ as distinct close pairs.
 
 Now we provide intuition helpful for counting the number of
 elements which are close to some fixed element $b \in (\alpha,
 2\alpha) \cap \Z_m^{\times}$\footnote{We will show later that
   this interval is non-empty for the relevant values of
 $\alpha$.}. 
 \begin{claim}\label{clm:prebintiny}
   Let $b\in \Z_m^{\times}\cap (\alpha, 2\alpha).$
   Let $I$ be a pre-bin. Then $\modm(b^{-1}\cdot I)$ is
   contained in the union of $b$ tiny intervals. 
 \end{claim}
 \begin{proof}
For each $j\in [w]$ we have
$$b^{-1}(j+[w]b) \equiv b^{-1}j + [w]  \mod m.$$
In other words, we can partition $x\in I$ based on
$\posmod_b(x)$; $x,y\in I$ for which $x\equiv y \bmod b$ are close
after multiplication by $b^{-1}$.
Note that many of these intervals may be empty; i.e., it is
possible that $\modm(b^{-1}\cdot I)$ is contained in
substantially less than $b$ tiny
intervals, but in general we cannot give a stronger bound.
 \end{proof}
 \begin{claim}\label{clm:oneBclosePairs}
   Let $b\in \Z_m^{\times}\cap (\alpha, 2\alpha).$
Say that for some pre-bin $I_b$ the set $\modm(b\cdot X) \cap
I_b$ contains at least $4\alpha$ elements. 
Then $X$ contains at least $\Omega(\alpha)$ close pairs.
 \end{claim}
 \begin{proof}
By \cref{clm:prebintiny} the $4\alpha$ elements of $\modm(b\cdot
X)\cap I_b$ are distributed amongst $b\le 2\alpha$ tiny intervals.
Any elements within the same tiny interval constitute a close
pair. The number of close pairs we obtain from splitting these
$4\alpha$ elements amongst $b$ tiny intervals is
minimized if we distribute the $4\alpha$ elements evenly amongst
the tiny intervals. 
However, even if the elements are distributed evenly we still
have at least $2$ elements per interval, and thus $\Omega(\alpha)$
total close pairs.
We remark that if we had defined a tiny interval to be any
smaller then this argument would not guarantee us to have any
close pairs.
 \end{proof}

Now we explore the relationship between close pairs and maxload.
For each bad $a_0$ there is some pre-bin $I_{a_0}$ which
contains at least $4\alpha$ elements of  $\modm(a_0 X)$.
By \cref{clm:oneBclosePairs}, this gives us at least $\alpha$
close pairs in $X$. 
Furthermore, we will show in \cref{lem:closepairs} that, among
bad $a_0$'s in a suitably chosen subset $B\subset \mathcal{A}$, the close
pairs given by each $a_0\in B$ are fairly disjoint.
Intuitively this means that the more bad $a_0$'s there are, the
more close pairs there are. 
Similarly to in \cref{prop:sqrtnZ}, we will
conclude by showing counting the close pairs with a different
method to show that too-large maxload results in too many close pairs.

  We proceed to formalize this reasoning.
  Let $$U = \Z_m^\times \cap \PRM \cap (\alpha, 2\alpha).$$ 
  The multiplier $a\gets \Z_m^{\times }$ is a random variable.
  Define
  $$B\defeq U\cap \modm(a^{-1}\mathcal{A});$$ $B$ is a random variable
  dependent on $a$.

  \begin{claim}\label{clm:sizeB}
    $\E[|B|] \geq \Omega\paren{\frac{\alpha}{\log \alpha} - \log
    n}\cdot \eps.$
  \end{claim}
  \begin{proof}
    The prime number theorem says that there are at least 
    $\Omega\left(\alpha/\log \alpha\right)$ primes in the interval
    $(\alpha, 2\alpha)$.
    On the other hand, $m$ cannot have more than $\log m$
    distinct prime divisors. 
    Hence, by excluding the prime divisors of $m$ from $\PRM \cap
    (\alpha, 2\alpha)$  we find:
    \begin{equation}
      \label{eq:ubound}
  |U| \geq \Omega\left(\frac{\alpha}{\log \alpha} - \log n\right).
    \end{equation}

    For any unit, and in particular for any bad $a_0\in
    \mathcal{A}$, $\modm(a^{-1}a_0)$ is uniformly random in
    $\Z_m^{\times}$. Thus,
    $$\Pr[\modm(a^{-1}a_0)\in U]  = |U| / |\Z_m^{\times }|.$$
     There are $|\mathcal{A}|=|\Z_m^{\times }|\eps$ bad $a_0$'s, so by linearity of
    expectation we have
    $$\E[|B|] \geq |\Z_m^{\times }|\eps |U|/|\Z_m^{\times }| =
    |U|\eps,$$
    which combined with \cref{eq:ubound} gives the desired
    result.
  \end{proof}

  \begin{claim}
    \label{clm:ezclosepairs}
    The expected number of close pairs is at most
    $\bigO(\frac{n}{\alpha}\log\log n)$.
  \end{claim}
  \begin{proof}
    Similar to the proof of \cref{prop:sqrtnZ}, we define
    \defn{linked} and \defn{unlinked} pairs. 
    We say that distinct $x,y\in X$ are linked if $\gcd(x-y, m)
    > w$ and unlinked otherwise. 
    If $x,y$ are linked, they cannot be closed by virtue of being distance at
    least $w$ apart. For unlinked $x,y$ 
    $\floor{\modm(a(x-y))/w}$ would be uniformly distributed on
    $[\floor{m/w}]$ if $a$ were chosen randomly from $\Z_m$.
    Using \cref{fact:toitent} we find the probability of $x,y$
    being close when $a\gets \Z_m^{\times}$ is at most 
    $\bigO\paren{\frac{\log\log n}{n\alpha}}.$
    There are $\binom{n}{2}$ total pairs. Then, by linearity of
    expectation there are $\bigO(\frac{n}{\alpha}\log\log n)$
    expected close pairs. 
  \end{proof}

  Now we establish the key combinatorial lemma:
  \begin{lemma}
    \label{lem:closepairs}
    There are at least $|B|\alpha/2$ close pairs.
  \end{lemma}
  \begin{proof}
    % each Ib yields some close pairs
    % Ib intersect Ic is small
    % cauchy shwarz

    Fix $a$. By definition of $B$, each $b\in B$ can be expressed
    as $b=a^{-1}a_0$ for some bad $a_0\in
  \mathcal{A}$. By definition of $a_0$ being bad there exists
  a pre-bin $I_{b}$ (which of course has size $|I_{b}|=m/n$)
  such that at least $4\alpha$ elements from $X$ fall in $I_b$
  under $a_0$. I.e., 
  \begin{equation}\label{eq:a0isbadandsobisgood}
  |I_{b}\cap a_{0}X| = |I_b  \cap baX| > 4\alpha.
  \end{equation}
  Recall \cref{clm:prebintiny},\cref{clm:oneBclosePairs}:
  multiplication by $b^{-1}$ ``fractures" the interval $I_b$ into
  at most $b$ tiny intervals, and using these tiny intervals we
  can obtain $\Omega(\alpha)$ close pairs. Now we analyze the
  overlap of the close pairs given by different $b\in B$.
  % , which assert
  % $b^{-1}I_b$ can be written as the disjoint union of intervals
  % $\setof{I_{b,j}}{j\in[b]}$ where each such interval is small,
  % i.e. $|I_{b,j}| < \frac{p}{b n}$. 
  % Recall that this is because multiplying by $b^{-1}$ fractures
  % the interval,  i.e. after going forward $b$ steps
  % you loop back around to about the same place under the map
  % $x\mapsto b^{-1}x \mod p$. By design, $I_{b,j}$ is small enough
  % that if $a,a'\in I_{b,j}$  then $a,a'$ is a close pair.
  % Recall from before that for each $b\in B$ this will give
  % $\Omega(\alpha)$ close pairs. 

  % However, this doesn't immediately conclude the proof, because
  % there may be overlap, i.e. apriori the close pairs obtained
  % from $b,c\in B, b\neq c$ may all be the same, or overlap
  % significantly. However, we claim that in fact there is not too
  % much overlap.  
  \begin{claim}\label{clm:IBCone}
   For $b\neq c$, and pre-bins $I_b,I_c$ the set
   $$\modm(b^{-1} I_{b})\cap \modm(c^{-1}I_c)$$ 
   is contained in a single tiny interval.
  \end{claim}
  \begin{proof}
  %  We claim that the fact that $b,c$ are both prime, or
  % more specifically $b,c$ are relatively prime, implies this
  % property, when taken together with our constraint that $b,c\in
  % \Theta(\alpha)$. 
   Assume that 
   $$b^{-1}i \equiv c^{-1}j \mod m$$ 
   for some $i,j$. 
   Then we also have
   $$b^{-1}i+\delta \equiv b^{-1}(i+b\delta)\equiv
   c^{-1}(j+c\delta) \mod m.$$
   Taken together, this set of elements is contained in a tiny
   interval.

   Now, we show that the intersection cannot have any more
   elements.
   It is more convenient to consider the following intersection:
   $$bc^{-1}[m/n] \cap \left([m/n] + \delta\right),$$
   for arbitrary $\delta$.
   
   Recall that $b,c \in (\alpha, 2\alpha)$, i.e., $b,c$ are quite
   similar in size. 
   Because $b,c,m$ are all pairwise co-prime, the following
   equation has a solution $\lambda \in \N$: 
   $$\lambda m + b \equiv 0 \mod c.$$
   For this value of $\lambda$,
    $$\frac{\lambda m+b}{c} \cdot c \equiv b \mod m.$$
    In other words, we can express $bc^{-1}$ as 
    $\frac{\lambda m + b}{c}$ for some integer $\lambda.$
    Leveraging $c\perp m$ we have that
    $$\frac{\lambda m}{c}[c] \equiv [c] m / c \mod m.$$
    In other words, $\modm(j \cdot \lambda m / c)$ visits some
    permutation of $[c] m/c$ as $j$ ranges over $[c]$.
    Recall that $b,c\in (\alpha, 2\alpha)$, so $b,c$ have very
    similar size.
    Thus, there is a permutation $\pi$ of $[c]$ so that for each $j\in [c]$
    $$\modm\left(\frac{\lambda m + b}{c} j\right)$$ 
    is approximately $\pi_j \cdot m/c$.
    In particular, $(b/c)j < 4\alpha$, whereas
    the separation between points in $[c]m/c$ is  $m/c$,
    which is much larger because 
    $$\alpha < n,\; m/\alpha > m/n,\; m > \Omega(n^6).$$
    On the other hand, for $k\in [m/(nc)]$ we have
    $$\frac{\lambda m+b}{c}k \equiv kc \mod m.$$

    We have described the shape of $bc^{-1}[m/n]$: it consists of
    $c$ well-separated concentrated intervals of length $m/n$
    with at most $\frac{m}{nc}$ elements per each such
    concentrated interval.
    Thus, upon intersection with $[m/n]$ we obtain at most
    $\frac{m}{nc}$ points.

    Interchanging the roles of $b,c$ in our above analysis, we
    get that at most $\frac{m}{nb}$ points are contained in the
    intersection.
    These 
    $$\min\left(\frac{m}{nc},\frac{m}{nb}\right) \le \frac{m}{n\alpha}$$
    points exactly correspond to a single tiny interval in $b^{-1}I_b\cap
    c^{-1}I_c$, by the argument at the beginning of the proof of
    this claim. 
    Alternatively, it is straightforward to see directly from our
    analysis of the shape of $bc^{-1}[m/n]$ that $b^{-1}I_b \cap
    c^{-1}I_c$ will be contained in a single tiny interval.
  \end{proof}

  Now we use \cref{clm:IBCone} to show that the close pairs given
  by each $b\in B$ are largely disjoint.
  \begin{claim}
    There are at least $|B|\alpha/2$ close pairs.
  \end{claim}
  \begin{proof}
  Let $I_{b,j}$ for $j\in [b]$ denote our partition of
  $\modm(b^{-1}I_b)$ into tiny intervals, as described in
  \cref{clm:prebintiny}. In particular, each $I_{b,j}$
  is a tiny interval, and 
  $$\bigsqcup_{j\in [b]} I_{b,j} = b^{-1}I_b.$$
  For each $b\in B,j\in [b]$ let $\delta(b,j)$ denote the number of
  $c\in B$ such  that $I_{b,j}\cap c^{-1}I_{c}\neq
  \varnothing$. Note that $\delta(b,j)\geq 1$ because
  $I_{b,j}\cap c^{-1}I_c\neq \varnothing$ for $c=b$.
  On the other hand, for each $b\neq c$ the intersection
  $b^{-1}I_b\cap c^{-1}I_c$ consists of at most a single tiny
  interval. Therefore,
  \begin{equation}
    \label{eq:deltasum}
  \sum_{j\in [b]} \delta(b,j) < |B| + b \leq 3\alpha.
  \end{equation}
  Define 
  $$\tau(b,j) = \max(0, |aX\cap I_{b,j}|-1).$$
  Recall that any two elements in the same tiny interval are
  close. Thus, the number of close pairs is at least the sum over
  all tiny intervals $I$ of 
  $$\binom{|I|}{2}\ge \frac{1}{2}\cdot (|I|-1)^2.$$
  $\delta(b,j)$ is the number of times which interval $I_{b,j}$
  is counted. Thus, the number of close pairs is at least
  \begin{equation}
    \frac{1}{2}\sum_b\sum_{j} \frac{\tau(b,j)^2}{\delta(b,j)}\label{eq:tautau}
  \end{equation}
  because the $\delta(b,j)$ factor handles the fact that the
  interval $I_{b,j}$ occurs $\delta(b,j)$ times in the sum.
  Using the Cauchy-Shwarz Inequality on the inner sum of
  \eqref{eq:tautau} gives: 
  \begin{equation}
    \label{eq:cauchy}
    \sum_{j\in [b]} \frac{\tau(b,j)^2}{\delta(b,j)} \geq
  \frac{\paren{\sum_j\tau(b,j)}^2}{\sum_j \delta(b,j)}.
  \end{equation}
  In \cref{eq:a0isbadandsobisgood} we showed
  $$|I_b \cap ab X| = |b^{-1}I_b \cap aX| > 4\alpha.$$
  Thus, we can bound the numerator of \eqref{eq:cauchy} by 
  \begin{equation}\label{eq:numerbound}
  \paren{\sum_{j\in [b]} \tau(b,j)}^2 \ge \paren{4\alpha - b}^2 \geq
  \paren{2\alpha}^2.
  \end{equation}
  Combining \cref{eq:numerbound} and \cref{eq:deltasum}, which
  bound the numerator and denominator respectively of
  \eqref{eq:cauchy}, we obtain 
  \begin{equation}\label{eq:boundinside}
  \sum_{j\in [b]} \frac{\tau(b,j)^2}{\delta(b,j)} \ge
  \frac{4\alpha^2}{3\alpha}\ge \alpha.
  \end{equation}
  Applying \cref{eq:boundinside} to \cref{eq:tautau}, we find
  that the number of close pairs is at least $\alpha
  |B|/2$, as desired.
  \end{proof}
  \end{proof}

  \begin{cor}\label{cor:finisher}
  $$\eps < \bigO\paren{\frac{n\log\log n}{\alpha^2( \alpha / \log
  \alpha - \log n )}}. $$
  \end{cor}
  \begin{proof}
    \cref{lem:closepairs} gives a lower bound on the number of
    close pairs: there are at least $|B|\alpha/2$ close pairs.
    \cref{clm:sizeB} gives a lower bound on $\E[|B|]$. 
    \cref{clm:ezclosepairs} gives an upper bound on the number of
    close pairs. Comparing our upper bound and lower bound gives the
    desired inequality for $\eps.$
  \end{proof}

  Finally, we use \cref{cor:finisher} to conclude the proof.
  \begin{cor}
    $$M_{\SLH}(m,n)\le \widetilde{\bigO}(n^{1/3}).$$
  \end{cor}
  \begin{proof} Fix any $X$. Let random variable $M$ denote the
    maxload of $X$. 
    The expression in \cref{cor:finisher} is slightly
    complicated. For $\alpha > n^{1/3}\log n$ we can perform the
    following simplification:
    $$\frac{1}{\alpha/\log \alpha - \log n} < \bigO\left(\frac{\log
    n}{\alpha}\right)$$
    which is true because
    $$\frac{\log n}{\log \alpha}\alpha - \log^2 n >
    \Omega(\alpha).$$
    Thus, for $k > n^{1/3}\log n$ we have
    $$\Pr[M\ge k]\le \bigO\left( \frac{n \log n\log\log n}{k^3}
    \right).$$
    Now we bound $\E[M]$ as follows:
  \begin{align*}
    \E[M] &\leq \sum_{k\ge 0} \Pr[M \geq k] \\
        &\leq \widetilde{\bigO}(n^{1/3}) +
        \bigO\paren{\sum_{k>n^{1/3}\log
        n} \frac{n \log n\log\log n}{k^3}}.
\end{align*}
A basic fact of calculus is that 
$$\sum_{k > n^{1/3}\log n} \frac{1}{k^3} \le
\bigO\paren{\frac{1}{n^{2/3}\log^2 n}}.$$

Thus, our bound for $\E[M]$ simplifies to 
$$\E[M] \le \widetilde{\bigO}(n^{1/3}).$$
This bound was for arbitrary $X$, and thus also holds for
worst-case $X$.

  \end{proof}

\end{proof}

\section{Maxload of a Structured Set}
\label{sec:nice}
Intuitively, because random sets have small maxload any set that
has achieves large maxload, if such a set exists, must be
somehow ``highly structured".
In this section we investigate the maxload of one natural
candidate for such a structured set. 
However, rather than achieving abnormally large maxload, we prove
that this set exhibits constant maxload.
This provides further evidence to support the hypothesis that
$\LH$ achieves small maxload.
\begin{theorem}
  \label{thm:nisnice}
$\blockRu$ achieves maxload $\Theta(1)$ on $[n]$.
\end{theorem}
  For multiplier $a\in (0,1)$ define
  $$g(a) = \min\setof{k\in \N}{\circabs_1(ak)< 1/n}.$$
  Let random variable $M$ denote the maxload of $\blockRu$ on
  $X=[n]$.

\begin{claim}\label{clm:kbiggerthanNisSilly} 
  $$\E[M \mid g(a) \ge n-2] \le \bigO(1).$$
\end{claim}
\begin{proof}
  If $i,j\in [n]$ hash to the same bin then $$\circabs_1(a(i-j))
  < 1/n$$ and so $$g(a)\le |i-j|.$$ Thus, conditional on $g(a)\ge
  n-2$ any colliding pair $i,j$ must satisfy $|i-j|\ge n-2$.
  There are only $\bigO(1)$ such pairs $\set{i,j}$ namely
  $\set{0,n-1}, \set{0,n-2}, \set{1,n-1}$.
\end{proof}
  
  \begin{claim}\label{clm:proffak}
    Let $k\le n$. Then
    $$\Pr[g(a) = k] \le 2/n.$$
  \end{claim}
  \begin{proof}
    If $g(a) \mid k$ there must be $c \in [k]$ such that
    \begin{equation}\label{eq:ckdelta}
  a\in \frac{c}{k} + \frac{1}{k}\cdot [-1/n,1/n].
    \end{equation}
  Note that if $c\not\perp k$ then $g(a) < k.$
  The probability of $a$ lying in
  this union of $\phi(k)$ intervals each of length $2/(nk)$ is
  at most
  $$\phi(k)\cdot \frac{2}{nk} \le 2/n,$$
  which bounds $\Pr[g(a)=k]$.

  % Similarly for $\block{\F_p}$ we have
  % $$\Pr[f(a) = k] \le \phi(k)\frac{2\ceil{w/k}}{p},$$
  % with the subtle difference that now the probability space is
  % discrete rather than continuous.
  % Fortunately, $w/k$ is large enough that the multiplicative difference between
  % $w/k$ and  $\ceil{w/k}$ is small. In particular, $w/k \ge p/n^2 >
  % \Omega(n^4)$ by our usual assumption on the size of $p$.
  % Thus, in this case we have, conservatively, the bound
  % $$\Pr[f(a)=k] \Le 3/n.$$
  \end{proof}

  \begin{lemma}\label{lem:lnnkyay} Let $k\in \N$ with ${1<k<
    n-2}$. Then $$\E[M\mid g(a)=k] \le \bigO(\ln (n/k)).$$
  \end{lemma}
  \begin{proof}
    Throughout the proof we condition on $g(a)=k$.
    % Recall from \cref{clm:proffak} that $g(a) = k$ only occurs if 
    % $$a\in pc/k + [-w/k, w/k]$$
    % for some $c\perp k$.
    We partition $a\cdot [n]$ into $k$ \defn{clumps} $C_1,\ldots, C_k$
    as follows: for each $i\in [n]$
    $$a\cdot i \in C_{\posmod_k(i)}.$$
    \begin{claim}\label{clm:clumpsnooverlap}
      If $i\not\equiv j \bmod k$ then $i,j$ hash to distinct bins.
    \end{claim}
    \begin{proof}
    As noted in \cref{eq:ckdelta},  we can write $a$ in the form 
       $$a = c/k + \delta$$ for some $c\perp k$ and $\delta \in
      [\frac{-1}{nk}, \frac{1}{nk}]$. 
      Intuitively, $\delta$ is small so $a$ behaves similarly to
      $c/k$. For $c/k$ we have that $\posmod_1(i\cdot c/k)$ travels on some
      permutation of $[k]\cdot c/k$ as $i$ travels over $[k]$.
      The term $\delta$ introduces some deviation from the
      behavior of $c/k$. We visualize this in
      \cref{fig:perm-png}: the black rectangles start at
      multiples of $1/k$ but then extend a little further to
      account for the deviation introduced by $\delta$. 
      Now we formally show that the deviation introduced by
      $\delta$ is small.

      The \defn{width} of clump $C_i$ is defined as $\max C_i -
      \min C_i$; width is determined by $\delta$. In
      particular, every $k$ steps we take one step within a
      clump, and the step size is $\delta$. In total this means
      that the widths are 
      $$\delta\cdot \frac{n}{k} \le \frac{1}{nk}\cdot \frac{n}{k} \le
      \frac{1}{k^2}.$$
      On the other hand, clumps are separated by a much larger
      quantity: $1/k.$
      In particular, $k<n-2$ by assumption
      so 
      \begin{equation}\label{eq:obviousthing}
      n(k-1)>(k-1)(k+2)=k^2+k-2 \ge k^2.
      \end{equation}
      % \todo{assume $k>1$; actually I think we need to be much
      % more careful for small $k$}
      Rearranging \cref{eq:obviousthing} gives 
      $$\frac{1}{k} - \frac{1}{k^2} \ge \frac{1}{n}.$$
      In other words, elements lying in different clumps cannot
      lie in the same bin, because there is a gap of at least
      $1/n$ between each of the clumps (this justifies why the
      rectangles in \cref{fig:perm-png} are drawn as
      non-overlapping).
      Note that here we have also applied the important fact that 
      clumps all grow in the same direction, which is determined by $\sgn(\delta)$.
% Figure environment removed
    \end{proof}

    \begin{claim}\label{clm:maxloadbeps}
      Assume $a = (c+\eps)/k$ for some $\eps \in [-1/n,1/n]$.
      Then the maxload is at most $\floor{(1/n)/\eps}$.
    \end{claim}
    \begin{proof}
      By \cref{clm:clumpsnooverlap} clumps map to separate bins, so
      it suffices to focus on a single clump.
      Observe that for any $i\in [\ceil{n/k}]$
      $$\posmod_1(i\cdot ak) = i\cdot \eps.$$
      Thus, it is impossible for more than $\floor{(1/n)/\eps}$ of
      the values in a clump  to lie in the same bin. In other
      words, the maxload is at most $\floor{(1/n)/\eps}$.
    \end{proof}

    Combining \cref{clm:maxloadbeps}, \cref{clm:clumpsnooverlap}
    we compute a bound on the maxload.
    For particularly small $|\eps|$ we use the fact that clumps do
    not intersect to deduce that the maxload is at most $n/k$.
For $\eps$ with $|\eps| > k /n^2$ the bound $(1/n)/\eps$ becomes stronger. 
Clearly each value of $\eps$ is equally likely.
Thus in total we have
\begin{align*}
  M &\le \frac{n}{k}\frac{2k/n^2}{2/n} + 
  \frac{1}{2/n}\cdot 2\int_{k/n^2}^{1/n}(1/n)/\eps \;\d \eps \\
    &\le \bigO(\ln (n/k)).
\end{align*}
  \end{proof}

  \begin{proof}[Proof of \cref{thm:nisnice}]
  Combining \cref{lem:lnnkyay}, \cref{clm:kbiggerthanNisSilly},
  and \cref{clm:proffak} gives
\begin{align*}
  \E[M] &\leq \bigO\left(\sum_{k=2}^{n-2} \frac{\log(n / k)}{n}\right) + \bigO(1) \\
        &\leq \bigO\left(\frac{\log (n^n/n!)}{n} \right)\\
        &\leq \bigO(1).
\end{align*}
% We remark that according to simulations the actual answer is
% quite close to $e$.
\end{proof}

% \begin{rmk}
%   Morally what this theorem means is, if you have runners with
%   speeds which are all evenly spaced out and they run for a
%   random amount of time, then they should still be pretty evenly spaced
%   out on average. 
% \end{rmk}

% \begin{conj}
%   I'm fairly certain that by a nearly identical argument, or
%   maybe even a simple reduction in the case of $\F_p$ which we
%   technically didn't really super rigorously prove yet for the strided case, 
%   we would find any ``strided set" with a constant common
%   difference between elements, to exhibit the same $\bigO(1)$
%   expected maxload behavior.
% \end{conj}

% \begin{conj}
%   I am pretty sure that $2^{[n]}$ also gets constant maxload. 

%   It certainly does if we bin by exponent. And I think it does
%   anyways probably for identical reasons as the analysis of $[n]$
%   but you just look at the exponent somehow. or maybe there is a
%   reduction.

% \end{conj}

% \begin{conj}
%   I'm pretty sure that any method of partitioning $[p]$ into $n$
%   equal sized bins should perform equally well.
% \end{conj}


\end{document}
