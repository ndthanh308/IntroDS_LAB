% !TEX program = pdflatex
% !TEX root = main.tex

%%%%%%%%%% Introduction


\section{Introduction}\label{sec:intro}

When considering a collection of people, animals, teams, or other entities, there is often an underlying hierarchy structuring the system. This hierarchy may be formally instilled in the sense that some individuals are explicitly granted certain ranks based on positions of authority. For example, in a school, there are students, teachers, and the principal or head of the school, with each position explicitly known and ranked in terms of level of authority. Alternatively, a hierarchy may be implicit in the sense that the ranks are not explicitly granted or known, but instead encoded in behaviors or interactions. For example, in animal dominance hierarchies, animals may be preferentially aggressive toward those lower in rank. In both explicit and implicit cases, hierarchies can be determined by analyzing the patterns of interactions between the entities of the system.

If we wish to infer the ranks of entities in a hierarchical structure from the patterns of their interactions, we can treat ranks as either static or dynamic, and as ordinal or real-valued. 

In the static case, time is irrelevant, and we treat all the interactions at once regardless of the sequence in which they occur, as one might when ranking the teams in a sports league at the end of a seasons. 

In the dynamic case, each individual's ranking may rise or fall over time, retaining the memory of past interactions while taking new interactions into account. This can be seen in leagues such as the U.S.\ National Basketball Association (NBA) where rankings derived from recent games provide insight for predicting games in the near future, yet the rankings themselves may nevertheless change slowly over the course of a season or seasons.
We are also interested in real-valued ranks, rather than ordinal ranks, such that the size of rank difference between two entities is an interpretable and predictive quantity, regardless of whether they are adjacent or well separated in ordinal rank.

To model systems of this type we propose \dsrfull. This builds on the previously-proposed SpringRank algorithm~\cite{de2018physical} by incorporating time information, inferring a dynamic hierarchy from a dynamic network: that is, a dataset of timestamped interactions, each of which defines a directed edge $i \to j$ indicating that $i$ ``beat'' $j$ at time $t$. We make similar physically-inspired assumptions as SpringRank, modeling directed edges as springs and assuming that entities are more likely to interact if their ranks are not too far apart. We also propose a generative model for constructing directed, hierarchical networks that evolve over time.

Finally, we evaluate \dsrfull\ on a variety of synthetic and real datasets. From our findings, we conclude that it accurately and efficiently infers ranks and predicts the direction of edges in dynamic settings. Furthermore, it frequently outperforms other algorithms such as the Elo Rating System and Whole-History Rating.


%%% Local Variables:
%%% mode: latex
%%% TeX-master: "main"
%%% End:

