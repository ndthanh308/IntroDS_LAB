% !TEX program = pdflatex
% !TEX root = main.tex

\section*{Conclusion}
\label{sec:conclusions}

\dsrfull\ is a principled extension of the physics-inspired SpringRank model for dynamic hierarchal structures, which lets us infer time-varying ranks from timestamped interactions. By coupling individuals' previous and current ranks, it exploits the chronological ordering of the data to better predict the outcomes of future interactions. It contains a parameter $\kself$ that can be tuned or learned in order to control the smoothness of the change in ranks, or equivalently the weight given to past ranks.

We constructed two different formulations of Dynamic SpringRank: an online and an offline one, which are given just past ranks and the entire history respectively. The online version performed better and is less computationally expensive. However, both models, similar to the static version, are scalable algorithms that require sparse linear algebra and provide a probabilistic generative model for creating dynamically directed networks with tunable levels of hierarchy and sparsity.

We also illustrated that in dynamic settings where time information is important, \dsrfull\ is better than its static counterpart. Its ability to predict future outcomes in dynamical settings proved to be similar or better than other state-of-the-art dynamical ranking algorithms for a variety of metrics and datasets, both synthetic and real. An open-source implementation of both \emph{offline} and \emph{online} versions of \dsrfull\ is available at \href{https://github.com/cdebacco/DynSpringRank}{https://github.com/cdebacco/DynSpringRank}.

For future work, we defined more elaborate models where the time intervals between interactions can vary, or where a momentum term induces smoothness in the rate at which ranks change over time. Another (perhaps challenging) direction is to couple the rank dynamics with the entities' choices to interact with each other. For instance, one can imagine a model in which animals tend to challenge those immediately above them in the dominance hierarchy, or where new arrivals to a community test themselves against current members in order to find their place, or even three-way interactions where an animal who attacks another is punished by a third~\cite{flack2006policing}.  Testing these models would require rich data from biological and social systems.



%%% Local Variables:
%%% mode: latex
%%% TeX-master: "main"
%%% End:
