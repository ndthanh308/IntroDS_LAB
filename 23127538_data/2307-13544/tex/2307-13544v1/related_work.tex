% !TEX program = pdflatex
% !TEX root = main.tex

 \section{Related Work}\label{sec:related_work}
 
Estimating hidden hierarchies from pairwise interactions is a fundamental problem in a wide variety of contexts. Several models have been proposed to study \emph{static} hierarchies, i.e., scenarios where ranks do not change in time: spectral methods including Eigenvector Centrality~\cite{bonacich1987}, PageRank~\cite{page1999} and Rank Centrality~\cite{negahban2016rank}; approaches that target ordinal rankings, such as Minimum Violation Rank~\cite{ali1986minimum,slater1961inconsistencies,gupte2011finding}, Ranked Stochastic Block Model ~\cite{letizia2018resolution}, SerialRank~\cite{fogel2014serialrank} and SyncRank~\cite{cucuringu2016sync}; Random Utility Models~\cite{train2009discrete} such as the Bradley-Terry-Luce (BTL) model~\cite{bradley1952,luce1959}; fully generative models including the Probabilistic Niche Model of ecology~\cite{williams2010probabilistic,williams2011probabilistic,jacobs2015untangling}; models of friendship based on social status~\cite{ball2013friendship} or with hierarchy and community structure \cite{iacovissi2021interplay}; latent space models~\cite{hoff01latentspace}; and physics-inspired models such as SpringRank~\cite{de2018physical} and belief propagation in continuous spin systems~\cite{cantwell-moore}.

In contrast, {\it online} methods, which update ranks after each interaction, model dynamic environments where ranks vary in time and interactions have a relevant chronological order. For instance, the Elo Rating System~\cite{elo1978rating}, commonly used for rating chess players, is one of the most popular online methods. It was later improved by the Glicko system~\cite{glickman1995glicko}, which incorporates a measure of reliability in estimating ranks to capture their uncertainty due to, for instance, a period of inactivity or lack of data. Another approach is a win-loss ranking algorithm~\cite{park2005network} and its dynamic extension~\cite{motegi2012network}. A Bayesian ranking system inferring individual ranks from team-level outcomes is the so called TrueSkill algorithm~\cite{herbrich2007trueskill}, which can be seen as a generalization of the Elo system. This has been extended by TrueSkill Through Time (TTT)~\cite{dangauthier2008trueskill} which infers smooth time series of ranks. Decaying-history ratings such as~\cite{motegi2012network} act directly on the data observations, progressively forgetting old interactions. One drawback of this approach is that time decay increases the uncertainty of player ratings: players who stop playing for a while may experience huge jumps in their ratings when they start playing again. On the other hand, players who play very frequently may have the feeling that their rating is stuck. If players do not all play at the same frequency, there is no clear way to tune the decay rate~\cite{coulom2008whole}.

Finally, a third set of {\it offline} methods treats ranks as time-varying, but infers the ranks at each time-step by considering the totality of all observations, including those before and after any particular time step. For instance, the Whole-History Rating (WHR)~\cite{coulom2008whole}, a Bayesian approach based on the dynamic Bradley-Terry-Luce model, computes the exact maximum a posteriori estimate of ranks over the whole history of all players. 



%%% Local Variables:
%%% mode: latex
%%% TeX-master: "main"
%%% End:
