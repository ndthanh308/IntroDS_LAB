%%
%% This is LaTeX2e input.
%%

%% The following tells LaTeX that we are using the 
%% style file amsart.cls (That is the AMS article style

\documentclass{amsart}

%% This has a default type size 10pt.  Other options are 11pt and 12pt
%% This are set by replacing the command above by
%% \documentclass[11pt]{amsart}
%%
%% or
%%
%% \documentclass[12pt]{amsart}
%%

%%
%% Some mathematical symbols are not included in the basic LaTeX
%% package.  Uncommenting the following makes more commands
%% available. 
%%

%\usepackage{amssymb}
\usepackage{amssymb,amsmath,comment,mathrsfs,mathtools,tikz-cd,todonotes,amscd}
\usepackage[colorlinks=true,linkcolor=blue]{hyperref}

%\usepackage[T1]{fontenc}
%\usepackage{lmodern}
%\usepackage[margin=1in]{geometry}
\usepackage{enumerate}
\usepackage{graphicx}
%\usepackage{amssymb,amsmath}

\usepackage[utf8]{inputenc} 
%\usepackage{geometry}
%\geometry{a4paper} 


\usepackage{booktabs} 
\usepackage{array}
\usepackage{paralist}
\usepackage{verbatim} 

\usepackage[utf8]{inputenc}
\usepackage{multirow}
\usepackage{graphicx}
\usepackage{tabularx}
%%
%% The following is commands are used for importing various types of
%% grapics.
%% 

%\usepackage{epsfig}  		% For postscript
%\usepackage{epic,eepic}       % For epic and eepic output from xfig

%%
%% The following is very useful in keeping track of labels while
%% writing.  The variant   \usepackage[notcite]{showkeys}
%% does not show the labels on the \cite commands.
%% 

%\usepackage{showkeys}


%%%%
%%%% The next few commands set up the theorem type environments.
%%%% Here they are set up to be numbered section.number, but this can
%%%% be changed.
%%%%
\usepackage{amsthm}
\def\F{\bbF}
%\newcommand{\Q}{\bbQ}
%\newcommand{\Z}{\bbZ}

\setlength{\parindent}{0pt}
\newtheorem{para}{\bf}[subsection]
\newtheorem{claim}[para]{\bf Claim}
\newtheorem{aux}[para]{\it Auxiliary result}
\newtheorem{example}[para]{\bf Example}
\newtheorem{examples}[para]{\bf Examples}
\newtheorem{rems}[para]{\bf Remarks}
\newtheorem{remk}[subsection]{\bf Remark}
\newtheorem{rem}[para]{\it Remark}
\newtheorem{convention}[para]{\bf Convention}

%\newtheorem{proof}[para]{proof}


\newtheorem{dfn}[para]{Definition}



\newtheorem{theorem}[para]{Theorem}
\newtheorem{surthm}[subsection]{Theorem}
\newtheorem{surprop}[subsection]{Proposition}
\newtheorem{sublemma}[subsection]{Lemma}
\newtheorem{lemma}[para]{Lemma}
\newtheorem{cor}[para]{Corollary}
\newtheorem{notation}[subsection]{Notation}
\newtheorem{note}[para]{Notation}
\newtheorem{prop}[para]{Proposition}
%\newtheorem{eqn}[para]{Equation}

%\newtheorem{thm}{Theorem}[section]
%\newtheorem{prop}[thm]{Proposition}
%\newtheorem{lem}[thm]{Lemma}
%\newtheorem{cor}[thm]{Corollary}


%%
%% If some other type is need, say conjectures, then it is constructed
%% by editing and uncommenting the following.
%%

%\newtheorem{conj}[thm]{Conjecture} 


%%% 
%%% The following gives definition type environments (which only differ
%%% from theorem type invironmants in the choices of fonts).  The
%%% numbering is still tied to the theorem counter.
%%% 

%\theoremstyle{definition}
%\newtheorem{definition}[thm]{Definition}
%\newtheorem{example}[thm]{Example}

%%
%% Again more of these can be added by uncommenting and editing the
%% following. 
%%

%\newtheorem{note}[thm]{Note}


%%% 
%%% The following gives remark type environments (which only differ
%%% from theorem type invironmants in the choices of fonts).  The
%%% numbering is still tied to the theorem counter.
%%% 


%\theoremstyle{remark}

%\newtheorem{remark}[thm]{Remark}


%%%
%%% The following, if uncommented, numbers equations within sections.
%%% 

\numberwithin{equation}{para}


%%%
%%% The following show how to make definition (also called macros or
%%% abbreviations).  For example to use get a bold face R for use to
%%% name the real numbers the command is \mathbf{R}.  To save typing we
%%% can abbreviate as

%\newcommand{\R}{\mathbf{R}}  % The real numbers.
\newcommand{\bbA}{{\mathbb A}}
\newcommand{\bbB}{{\mathbb B}}
\newcommand{\bbC}{{\mathbb C}}
\newcommand{\bbD}{{\mathbb D}}
\newcommand{\bbE}{{\mathbb E}}
\newcommand{\bbF}{{\mathbb F}}
\newcommand{\bbG}{{\mathbb G}}
\newcommand{\bbH}{{\mathbb H}}
\newcommand{\bbI}{{\mathbb I}}
\newcommand{\bbJ}{{\mathbb J}}
\newcommand{\bbK}{{\mathbb K}}
\newcommand{\bbL}{{\mathbb L}}
\newcommand{\bbM}{{\mathbb M}}
\newcommand{\bbN}{{\mathbb N}}
\newcommand{\bbO}{{\mathbb O}}
\newcommand{\bbP}{{\mathbb P}}
\newcommand{\bbQ}{{\mathbb Q}}
\newcommand{\bbR}{{\mathbb R}}
\newcommand{\bbS}{{\mathbb S}}
\newcommand{\bbT}{{\mathbb T}}
\newcommand{\bbU}{{\mathbb U}}
\newcommand{\bbV}{{\mathbb V}}
\newcommand{\bbW}{{\mathbb W}}
\newcommand{\bbX}{{\mathbb X}}
\newcommand{\bbY}{{\mathbb Y}}
\newcommand{\bbZ}{{\mathbb Z}}

\newcommand{\bA}{{\bf A}}
\newcommand{\bB}{{\bf B}}
\newcommand{\bC}{{\bf C}}
\newcommand{\bD}{{\bf D}}
\newcommand{\bE}{{\bf E}}
\newcommand{\bF}{{\bf F}}
\newcommand{\bG}{{\bf G}}
\newcommand{\bH}{{\bf H}}
\newcommand{\bI}{{\bf I}}
\newcommand{\bJ}{{\bf J}}
\newcommand{\bK}{{\bf K}}
\newcommand{\bL}{{\bf L}}
\newcommand{\bM}{{\bf M}}
\newcommand{\bN}{{\bf N}}
\newcommand{\bO}{{\bf O}}
\newcommand{\bP}{{\bf P}}
\newcommand{\bQ}{{\bf Q}}
\newcommand{\bR}{{\bf R}}
\newcommand{\bS}{{\bf S}}
\newcommand{\bT}{{\bf T}}
\newcommand{\bU}{{\bf U}}
\newcommand{\bV}{{\bf V}}
\newcommand{\bW}{{\bf W}}
\newcommand{\bX}{{\bf X}}
\newcommand{\bY}{{\bf Y}}
\newcommand{\bZ}{{\bf Z}}


\newcommand{\ba}{{\bf a}}
\newcommand{\bb}{{\bf b}}
\newcommand{\bc}{{\bf c}}
\newcommand{\bd}{{\bf d}}
\newcommand{\be}{{\bf e}}
\newcommand{\bff}{{\bf f}}
\newcommand{\bg}{{\bf g}}
\newcommand{\bh}{{\bf h}}
\newcommand{\bi}{{\bf i}}
\newcommand{\bj}{{\bf j}}
\newcommand{\bk}{{\bf k}}
\newcommand{\bl}{{\bf l}}
\newcommand{\bm}{{\bf m}}
\newcommand{\bn}{{\bf n}}
\newcommand{\bo}{{\bf o}}
\newcommand{\bp}{{\bf p}}
\newcommand{\bq}{{\bf q}}
\newcommand{\br}{{\bf r}}
\newcommand{\bs}{{\bf s}}
\newcommand{\bt}{{\bf t}}
\newcommand{\bu}{{\bf u}}
\newcommand{\bv}{{\bf v}}
\newcommand{\bw}{{\bf w}}
\newcommand{\bx}{{\bf x}}
\newcommand{\by}{{\bf y}}
\newcommand{\bz}{{\bf z}}

\newcommand{\fra}{{\mathfrak a}}
\newcommand{\frb}{{\mathfrak b}}
\newcommand{\frc}{{\mathfrak c}}
\newcommand{\frd}{{\mathfrak d}}
\newcommand{\fre}{{\mathfrak e}}
\newcommand{\frf}{{\mathfrak f}}
\newcommand{\frg}{{\mathfrak g}}
\newcommand{\frh}{{\mathfrak h}}

\newcommand{\fri}{{\mathfrak i}}
\newcommand{\frj}{{\mathfrak j}}
\newcommand{\frk}{{\mathfrak k}}
\newcommand{\frl}{{\mathfrak l}}
\newcommand{\frm}{{\mathfrak m}}
\newcommand{\frn}{{\mathfrak n}}
\newcommand{\fro}{{\mathfrak o}}
\newcommand{\frp}{{\mathfrak p}}
\newcommand{\frq}{{\mathfrak q}}
\newcommand{\frr}{{\mathfrak r}}
\newcommand{\frs}{{\mathfrak s}}
\newcommand{\frt}{{\mathfrak t}}
\newcommand{\fru}{{\mathfrak u}}
\newcommand{\frv}{{\mathfrak v}}
\newcommand{\frw}{{\mathfrak w}}
\newcommand{\frx}{{\mathfrak x}}
\newcommand{\fry}{{\mathfrak y}}
\newcommand{\frz}{{\mathfrak z}}

\newcommand{\frA}{{\mathfrak A}}
\newcommand{\frB}{{\mathfrak B}}
\newcommand{\frC}{{\mathfrak C}}
\newcommand{\frD}{{\mathfrak D}}
\newcommand{\frE}{{\mathfrak E}}
\newcommand{\frF}{{\mathfrak F}}
\newcommand{\frG}{{\mathfrak G}}
\newcommand{\frH}{{\mathfrak H}}
\newcommand{\frI}{{\mathfrak I}}
\newcommand{\frJ}{{\mathfrak J}}
\newcommand{\frK}{{\mathfrak K}}
\newcommand{\frL}{{\mathfrak L}}
\newcommand{\frM}{{\mathfrak M}}
\newcommand{\frN}{{\mathfrak N}}
\newcommand{\frO}{{\mathfrak O}}
\newcommand{\frP}{{\mathfrak P}}
\newcommand{\frQ}{{\mathfrak Q}}
\newcommand{\frR}{{\mathfrak R}}
\newcommand{\frS}{{\mathfrak S}}
\newcommand{\frT}{{\mathfrak T}}
\newcommand{\frU}{{\mathfrak U}}
\newcommand{\frV}{{\mathfrak V}}
\newcommand{\frW}{{\mathfrak W}}
\newcommand{\frX}{{\mathfrak X}}
\newcommand{\frY}{{\mathfrak Y}}
\newcommand{\frZ}{{\mathfrak Z}}

\newcommand{\cA}{{\mathcal A}}
\newcommand{\cB}{{\mathcal B}}
\newcommand{\cC}{{\mathcal C}}
\newcommand{\cD}{{\mathcal D}}
\newcommand{\cE}{{\mathcal E}}
\newcommand{\cF}{{\mathcal F}}
\newcommand{\cG}{{\mathcal G}}
\newcommand{\cH}{{\mathcal H}}
\newcommand{\cI}{{\mathcal I}}
\newcommand{\cJ}{{\mathcal J}}
\newcommand{\cK}{{\mathcal K}}
\newcommand{\cL}{{\mathcal L}}
\newcommand{\cM}{{\mathcal M}}
\newcommand{\cN}{{\mathcal N}}
\newcommand{\cO}{{\mathcal O}}
\newcommand{\cP}{{\mathcal P}}
\newcommand{\cQ}{{\mathcal Q}}
\newcommand{\cR}{{\mathcal R}}
\newcommand{\cS}{{\mathcal S}}
\newcommand{\cT}{{\mathcal T}}
\newcommand{\cU}{{\mathcal U}}
\newcommand{\cV}{{\mathcal V}}
\newcommand{\cW}{{\mathcal W}}
\newcommand{\cX}{{\mathcal X}}
\newcommand{\cY}{{\mathcal Y}}
\newcommand{\cZ}{{\mathcal Z}}

\newcommand{\rA}{{\rm A}}
\newcommand{\rB}{{\rm B}}
\newcommand{\rC}{{\rm C}}
\newcommand{\rD}{{\rm D}}
\newcommand{\rE}{{\rm E}}
\newcommand{\rF}{{\rm F}}
\newcommand{\rG}{{\rm G}}
\newcommand{\rH}{{\rm H}}
\newcommand{\rI}{{\rm I}}
\newcommand{\rig}{{\rm rig}}
\newcommand{\rJ}{{\rm J}}
\newcommand{\rK}{{\rm K}}
\newcommand{\rL}{{\rm L}}
\newcommand{\rM}{{\rm M}}
\newcommand{\rN}{{\rm N}}
\newcommand{\rO}{{\rm O}}
\newcommand{\rP}{{\rm P}}
\newcommand{\rQ}{{\rm Q}}
\newcommand{\rR}{{\rm R}}
\newcommand{\rS}{{\rm S}}
\newcommand{\rT}{{\rm T}}
\newcommand{\rU}{{\rm U}}
\newcommand{\rV}{{\rm V}}
\newcommand{\rW}{{\rm W}}
\newcommand{\rX}{{\rm X}}
\newcommand{\rY}{{\rm Y}}
\newcommand{\rZ}{{\rm Z}}

\newcommand{\sA}{{\mathscr A}}
\newcommand{\sB}{{\mathscr B}}
\newcommand{\sC}{{\mathscr C}}
\newcommand{\sD}{{\mathscr D}}
\newcommand{\sE}{{\mathscr E}}
\newcommand{\sF}{{\mathscr F}}
\newcommand{\sG}{{\mathscr G}}
\newcommand{\sH}{{\mathscr H}}
\newcommand{\sI}{{\mathscr I}}
\newcommand{\sJ}{{\mathscr J}}
\newcommand{\sK}{{\mathscr K}}
\newcommand{\sL}{{\mathscr L}}
\newcommand{\sM}{{\mathscr M}}
\newcommand{\sN}{{\mathscr N}}
\newcommand{\sO}{{\mathscr O}}
\newcommand{\sP}{{\mathscr P}}
\newcommand{\sQ}{{\mathscr Q}}
\newcommand{\sR}{{\mathscr R}}
\newcommand{\sS}{{\mathscr S}}
\newcommand{\sT}{{\mathscr T}}
\newcommand{\sU}{{\mathscr U}}
\newcommand{\sV}{{\mathscr V}}
\newcommand{\sW}{{\mathscr W}}
\newcommand{\sX}{{\mathscr X}}
\newcommand{\sY}{{\mathscr Y}}
\newcommand{\sZ}{{\mathscr Z}}


\newcommand{\tA}{\tilde{A}}
\newcommand{\tB}{\tilde{B}}
\newcommand{\tC}{\tilde{C}}
\newcommand{\tD}{\tilde{D}}
\newcommand{\tE}{\tilde{E}}
\newcommand{\tF}{\tilde{F}}
\newcommand{\tG}{\tilde{G}}
\newcommand{\tH}{\tilde{H}}
\newcommand{\tI}{\tilde{I}}
\newcommand{\tJ}{\tilde{J}}
\newcommand{\tK}{\tilde{K}}
\newcommand{\tL}{\tilde{L}}
\newcommand{\tM}{\tilde{M}}
\newcommand{\tN}{\tilde{N}}
\newcommand{\tO}{\tilde{O}}
\newcommand{\tP}{\tilde{P}}
\newcommand{\tQ}{\tilde{Q}}
\newcommand{\tR}{\tilde{R}}
\newcommand{\tS}{\tilde{S}}
\newcommand{\tT}{\tilde{T}}
\newcommand{\tU}{\tilde{U}}
\newcommand{\tV}{\tilde{V}}
\newcommand{\tW}{\tilde{W}}
\newcommand{\tX}{\tilde{X}}
\newcommand{\tY}{\tilde{Y}}
\newcommand{\tZ}{\tilde{Z}}
\newcommand{\tpi}{\tilde{\pi}}
\newcommand{\tcF}{\tilde{{\mathcal F}}}

\newcommand{\wD}{\widehat{D}}
\newcommand{\wE}{\widehat{E}}


\newcommand{\fronr}{{\hat{\fro}^{\rm nr}}}
\newcommand{\Lnr}{{\hat{L}^{\rm nr}}}

\newcommand{\froxB}{{{\mathfrak o}_B^{\times}}}
\newcommand{\froxE}{{{\mathfrak o}_E^{\times}}}
\newcommand{\froxF}{{{\mathfrak o}_F^{\times}}}
\newcommand{\froB}{{{\mathfrak o}_B}}


\newcommand{\GN}{G_0}
\newcommand{\Gnc}{\bbG(k)^\circ}
\newcommand{\Dgn}{\cD^{\rm an}(\bbG(k)^\circ)}
\newcommand{\Dgnt}{\cD^{\rm an}(\bbG(k)^\circ)_{\theta_0}}
\newcommand{\DHnnt}{D(\bbG(k)^\circ,\GN)_{\theta_0}}
\newcommand{\DlGN}{{D^{\rm la}(G,E)}}
\newcommand{\DGN}{D(\GN)}
\newcommand{\DGNt}{D(\GN)_{\theta_0}}
\newcommand{\DHnn}{D(\bbH_n^\circ, H)}
\newcommand{\DHN}{D(H)}
\newcommand{\DlHN}{{D^{\rm la}(H,E)}}
\newcommand{\Danh}{D^{an}(\bbH, E)}
\newcommand{\Ct}{\mathfrak{Coh}_0}
\newcommand{\Danhn}{D^{an}(\bbH_n^{\circ}, E)}
\newcommand{\ClaZp}{C^\la(\Zp, E)}
%
\newcommand{\cro}{\atopwithdelims \langle \rangle }
\newcommand{\crofrac}[2]{\genfrac{\langle}{\rangle}{0pt}{}{#1}{#2}}
\newcommand{\vidfrac}[2]{\genfrac{}{}{0pt}{}{#1}{#2}}
\newcommand{\parfrac}[2]{\genfrac{(}{)}{0pt}{}{#1}{#2}}
\newcommand{\acc}{\atopwithdelims \{ \} }
%rem dans ce qui suit 0pt est l'epaisseur du trait de fraction
\newcommand{\accfrac}[2]{\genfrac{\{}{\}}{0pt}{}{#1}{#2}}
%
\newcommand{\ua}{\underline{a}}
\newcommand{\uh}{\underline{h}}
\newcommand{\ui}{\underline{i}}
\newcommand{\uj}{\underline{j}}
\newcommand{\uk}{\underline{k}}
\newcommand{\ul}{\underline{l}}
\newcommand{\um}{\underline{m}}
\newcommand{\un}{\underline{n}}
\newcommand{\ut}{\underline{t}}
\newcommand{\uu}{\underline{u}}
\newcommand{\uv}{\underline{v}}
\newcommand{\ur}{\underline{r}}\newcommand{\uq}{\underline{q}}
\newcommand{\us}{\underline{s}}
\newcommand{\ux}{\underline{x}}
\newcommand{\uy}{\underline{y}}
\newcommand{\uz}{\underline{z}}
\newcommand{\uder}{\underline{\partial}}
\newcommand{\ual}{\underline{\alpha}}
\newcommand{\ube}{\underline{\beta}}
\newcommand{\ugamma}{\underline{\gamma}}
\newcommand{\ulam}{\underline{\lambda}}
\newcommand{\lam}{\lambda}
\newcommand{\uxi}{\underline{\xi}}
\newcommand{\ueta}{\underline{\eta}}
\newcommand{\uXi}{\underline{\Xi}}
\newcommand{\uP}{\underline{P}}
\newcommand{\uS}{\underline{S}}
\newcommand{\uT}{\underline{T}}
\newcommand{\uX}{\underline{X}}
\newcommand{\uY}{\underline{Y}}
\newcommand{\ulambda}{\underline{\lambda}}
\newcommand{\umu}{\underline{\mu}}
\newcommand{\unu}{\underline{\nu}}
\newcommand{\utau}{\underline{\tau}}


\newcommand{\Alg}{\rm Alg}
\newcommand{\an}{{\rm an}}
\newcommand{\Aut}{{\rm Aut}}
\newcommand{\Ba}{{\rm Ba}}
\newcommand{\bksl}{\backslash}
\newcommand{\car}{\stackrel{\simeq}{\longrightarrow}}
\newcommand{\cm}{{\mathfrak m}_{\bbC_p}}
\newcommand{\Coh}{{\rm Coh}}
\newcommand{\CO}{{\rm CO}}
\newcommand{\coker}{{\rm coker}}
\newcommand{\cont}{{\rm cont}}
\newcommand{\Cp}{{\bbC_p}}
\newcommand{\D}{{\Delta}}
\newcommand{\der}{\partial}
\newcommand{\diag}{{\rm diag}}
\newcommand{\dist}{{\rm dist}}
\newcommand{\End}{{\rm End}}
\newcommand{\eps}{{\epsilon}}
\newcommand{\eqdef}{\;\stackrel{\text{\tiny def}}{=}\;}
\newcommand{\et}{{\mbox{\tiny \it \'et}}}
\newcommand{\Ext}{{\rm Ext}}
\newcommand{\Fp}{{\bbF_p}}
\newcommand{\Fq}{{\bbF_q}}
\newcommand{\Frob}{{\rm Frob}}
\newcommand{\Ga}{\Gamma}
\newcommand{\Gal}{{\rm Gal}}
\newcommand{\GL}{{\rm GL}}
\newcommand{\gr}{{\rm gr}^\bullet}
\newcommand{\GSp}{{\rm GSp}}
\newcommand{\GU}{{\rm GU}}
\newcommand{\Hom}{{\rm Hom}}
\newcommand{\hot}{{\widehat{\otimes}}}
\newcommand{\hra}{\hookrightarrow}
\newcommand{\id}{{\rm id}}
\newcommand{\im}{{\rm im}}
\newcommand{\Ind}{{\rm Ind}}
\newcommand{\ind}{{\rm ind}}
\newcommand{\la}{{\rm la}}
\newcommand{\LA}{{\rm LA}}
\newcommand{\lan}{\langle}
\newcommand{\Lie}{{\rm Lie}}
\newcommand{\Loc}{{\mathscr Loc}}
\newcommand{\lra}{\longrightarrow}
\newcommand{\midc}{{\; | \;}}
\renewcommand{\mod}{\; {\rm mod} \;}
\newcommand{\Mod}{{\rm Mod}}
\newcommand{\Ne}{{\mathbb N}}
\newcommand{\PD}{{\rm P{\Delta}}}
\newcommand{\oOmega}{{\overline{\Omega}}}
\newcommand{\oPOmega}{{\overline{\POmega}}}
\newcommand{\ori}{{\rm or}}
\newcommand{\ord}{{\rm ord}}
\newcommand{\ovcI}{\overline{\cI}}
\newcommand{\oD}{\overline{\D}}
\newcommand{\opartial}{{\overline{\partial}}}
\newcommand{\oPD}{\overline{\PD}}
\newcommand{\ot}{\otimes}
\newcommand{\ovP}{\overline{P}}
\newcommand{\Pf}{{\it Proof. }}
\newcommand{\pg}{\bullet}
\newcommand{\POmega}{{ \rm P\Omega}}
\newcommand{\pr}{{\rm pr}}
\newcommand{\Proj}{{\rm Proj}}
\newcommand{\Q}{{\mathbb Q}}
\newcommand{\Ql}{{\bbQ_\ell}}
\newcommand{\Qlb}{{\overline{\bbQ}_\ell}}
\newcommand{\Qp}{{\bbQ_p}}
\newcommand{\ra}{\rightarrow}
\newcommand{\ran}{\rangle}
\newcommand\nonmini{\mathop{non \mhyphen min}}
\newcommand{\nonmin}{\operatorname{non-min}}
\newcommand{\res}{{\rm res}}
\newcommand{\rk}{{\rm rk}}
\newcommand{\setm}{{\; \setminus \;}}
\newcommand{\sgn}{{\rm sgn}}
\newcommand{\SL}{{\rm SL}}
\newcommand{\SM}{{\rm{SM}}}
\renewcommand{\sp}{{\rm sp}}
\newcommand{\Sp}{{\rm Sp}}
\newcommand{\Spa}{{\rm Spa}}
\newcommand{\Spec}{{\rm Spec}}
\newcommand{\Spf}{{\rm Spf}}
\newcommand{\Spm}{{\rm Spf}}
\newcommand{\St}{{\rm St}}
\newcommand{\sta}{\stackrel}
\newcommand{\Stab}{{\rm Stab}}
\newcommand{\Stan}{{\rm St}^{an}}
\newcommand{\sub}{\subset}
\newcommand{\supp}{{\rm supp}}
\newcommand{\Sym}{{\rm Sym}}
\newcommand{\tra}{\twoheadrightarrow}
\newcommand{\triv}{{\bf 1}}
\newcommand{\val}{{\rm v}}
\newcommand{\vep}{{\varepsilon}}
\newcommand{\vphi}{{\varphi}}
\newcommand{\vpi}{{\varpi}}
\newcommand{\vtheta}{{\vartheta}}
\newcommand{\wfrX}{{\widetilde{\frX}}}
\newcommand{\what}{\widehat}
\newcommand{\wpa}{\wideparen}
\newcommand{\wsM}{\widetilde{\sM}}
\newcommand{\wXo}{\widetilde{X}^\circ}
\newcommand{\x}{{\times}}
\newcommand{\Xo}{{X^\circ}}
\newcommand{\Z}{{\mathbb Z}}
\newcommand{\Zl}{{\mathbb Z_\ell}}
\newcommand{\Zp}{{\mathbb Z_p}}


%%
%% The comment after the defintion is not required, but if you are
%% working with someone they will likely thank you for explaining your
%% definition.  
%%
%% Now add you own definitions:
%%

%%%
%%% Mathematical operators (things like sin and cos which are used as
%%% functions and have slightly different spacing when typeset than
%%% variables are defined as follows:
%%%

%\DeclareMathOperator{\dist}{dist} % The distance.



%%
%% This is the end of the preamble.
%% 


\begin{document}

%%
%% The title of the paper goes here.  Edit to your title.
%%

\title{$THH$ of the Morava $E$-theory Spectrum $E_{2}$}

%%
%% Now edit the following to give your name and address:
%% 

\author{Sanjana Agarwal}
%\email{sanjagar@iu.edu}
%\urladdr{www.math.sc.edu/$\sim$howard} % Delete if not wanted.

%%
%% If there is another author uncomment and edit the following.
%%

%\author{Second Author}
%\address{Department of Mathematics, University of South Carolina,
%Columbia, SC 29208}
%\email{second@math.sc.edu}
%\urladdr{www.math.sc.edu/$\sim$second}

%%
%% If there are three of more authors they are added in the obvious
%% way. 
%%

%%%
%%% The following is for the abstract.  The abstract is optional and
%%% if not used just delete, or comment out, the following.
%%%

\begin{abstract}
The Morava $E$-theories, $E_{n}$, are complex-oriented $2$-periodic ring spectra, with homotopy groups $\bbW_{\bbF_{p^{n}}}[[u_{1}, u_{2}, ... , u_{n-1}]][u,u^{-1}]$. Here $\bbW$ denotes the Witt vector ring. $E_{n}$ is a Landweber exact spectrum and hence uniquely determined by this ring as $BP_{\ast}$-algebra. Algebraic $K$-theory of $E_{n}$ is a key ingredient towards analyzing the layers in the $p$-complete Waldhausen’s $K$-theory chromatic tower. One hopes to use the machinery of trace methods to get results towards $K$-theory once the computation for $THH(E_{n})$ is known. 

In this paper we describe $THH(E_{2})$ as part of consecutive chain of cofiber sequences where each cofiber sits in the next cofiber sequence and the first term of each cofiber sequence is describable completely in terms of suspensions and localisations of $E_{2}$. %We don't have a similar description for the final cofiber but we show that it is rational and give an inlength description of the homological classes associated with it. 
For these results, we first calculate $K(i)$-homology of $THH(E_{2})$ using a Bökstedt spectral sequence and then lift the generating classes of $K(1)$-homology to fundamental classes in homotopy group of $THH(E_{2})$. %At every step wherever possible, we say whether the results can be extended to THH(En).
These lifts allow us to construct terms of the cofiber sequence and explicitly understand how they map to $THH(E_{2})$.
\end{abstract}

%%
%%  LaTeX will not make the title for the paper unless told to do so.
%%  This is done by uncommenting the following.
%%

\maketitle

%%
%% LaTeX can automatically make a table of contents.  This is done by
%% uncommenting the following:
%%

\tableofcontents

%%
%%  To enter text is easy.  Just type it.  A blank line starts a new
%%  paragraph. 
%%

\section{Introduction}

\subsection{Motivation}

The 2002 paper of Ausoni and Rognes \cite{10.1007/BF02392794} initiated a program to understand algebraic $K$-theory of ring spectra and its relationship to chromatic phenomenon. We begin with a brief outline of this program as it has served as one of the big motivations behind many of the calculations in this area over the last 20 years.\\

In algebraic number theory, arithmetic of ring of integers in a number field is largely captured by its Picard group, unit group, etc., which are related to $K_{0}$-group, $K_{1}$-group, etc., respectively. Thus, algebraic $K$-theory encodes a lot about the arithmetic structures of rings within itself. The category of rings sits inside the category of ring spectra via Eilenberg-MacLane functor which sends $R$ to $HR$. Since we can extend the algebraic $K$-theory functor to ring spectra (see for example \cite[Chapter 6]{MR1417719}), one hopes to similarly decode useful information about ring spectra from its algebraic $K$-theory. In fact, Waldhausen in \cite{MR802796} showed that algebraic $K$-theory for ring spectra, particularly the sphere spectrum, is related to high dimensional differential topology. \\

%$K(\mathbb{S}^{0})$ being equivalent to Waldhausen's $A$-theory of a point is a crucial and fundamental object with regards to such calculations and towards developing an intensive understanding of algebraic $K$-theory functor for ring spectra. 
Work of Dundas, Goodwillie, McCarthy \cite[Theorem 0.0.2]{MR3013261} and Hesselholt, Madsen \cite[Theorem B.1.]{MR1410465} show that the square 
%We have a homotopy cartesian square upto $p$-completion, from \cite{MR3013261} and \cite{MR1410465}, 
\[ \begin{tikzcd}
K(\bbS) \arrow{r} \arrow{d} & K(\bbS_{p}) \arrow{d}\\
K(\pi_{0}\bbS) \arrow{r} & K(\pi_{0}\bbS_{p})
\end{tikzcd} \]
is homotopy cartesian after $p$-completion ($\bbS_{p}$ is the p-complete sphere spectrum). This
allows us to focus on understanding $K(\bbS_{p})$ instead of $K(\bbS)$. $\bbS_{p}$ has a chromatic tower 
\[ \bbS_{p} \rightarrow \cdots \rightarrow L_{n}\bbS_{p} \rightarrow \cdots \rightarrow L_{1}\bbS_{p} \rightarrow L_{0}\bbS_{p} = H\Q_{p} \]
where $L_{n} = L_{E(n)}$ := Bousfield localization with respect to $n$th Johnson-Wilson theory $E(n)$. The Johnson-Wilson theories $E(n)$ are complex oriented with $p$-typical formal group laws and coefficient rings $E(n)_{\ast} = \Z_{(p)}[v_{1}, \cdots, v_{n}, v_{n}^{-1}]$. Note that $E(n)$ is Landweber exact and hence determined by its coefficient ring. The Hopkins-Ravenel chromatic convergence theorem implies that the map
\[ \bbS_{p} \overset{\simeq} \longrightarrow \text{holim}_{n} \ L_{n}\bbS_{p} \]
is a weak equivalence. It also induces a tower on algebraic $K$-theory
\[ K(\bbS_{p}) \rightarrow \cdots \rightarrow K(L_{n}\bbS_{p}) \rightarrow \cdots \rightarrow K(L_{1}\bbS_{p}) \rightarrow K(L_{0}\bbS_{p}) = K(\Q_{p}) \]
with a map $K(\bbS_{p}) \rightarrow \text{holim}_{n} \ K(L_{n}\bbS_{p})$. Waldhausen conjectured that this map (or similar ones) should be weak equivalences. Waldhausen's work suggests that there should also be a $K$-theoretic interpretation of the fibers of the maps in this sequence: the fibers should be closely related to $K(L_{K(n)}\bbS_{p})$, where $L_{K(n)}$ is Bousfield localization with respect to $n$th Morava $K$-theory $K(n)$. The Morava $K$-theory spectra are complex oriented with $p$-typical formal group laws and coefficient rings $K(n)_{\ast} = \Z_{(p)}[v_{n}, v_{n}^{-1}]$ (but not Landweber exact).\\

%Further, there are maps $L_{n}\bbS^{0}_{p} \rightarrow L_{K(n)}\bbS^{0}_{p}$, where $L_{K(n)}$ is Bousfield %localization with respect to $n$th Morava $K$-theory $K(n)$ with coefficient ring $K(n)_{\ast} \cong %\F_{p}[v_{n}, v_{n}^{-1}]$. This induces maps $K(L_{n}\bbS^{0}_{p}) \rightarrow K(L_{K(n)}\bbS^{0}_{p})$. %Thus, knowing more about $K(L_{n}\bbS^{0}_{p})$ and $K(L_{K(n)}\bbS^{0}_{p})$ can be an extremely important %step towards undertanding $K(\bbS^{0}_{p})$.\\

There is another family of spectra, called Morava $E$-theories, denoted by $E_{n}$ which are also Landweber exact and determined by the coefficient ring $E_{n_{\ast}} \cong W\F_{p^{n}}[[u_{1}, \cdots\\, u_{n-1}]][u,u^{-1}]$. $E_{n}$ spectra are $E_{\infty}$-ring spectra and $E_{1} \simeq KU_{p}$, the p-complete complex topological $K$-theory spectrum. The $u_{i}$ are related to $v_{i}$ from $BP_{\ast}$ and $E(n)_{\ast}$ via the maps $E(n) \rightarrow E_{n}$ which at the homotopy group level send $v_{i} \mapsto u_{i} \cdot u^{p^{i}-1}$ for $i < n$, and $v_{n} \mapsto u^{p^{n}-1}$. Devinatz-Hopkins \cite{MR2030586} show $L_{K(n)}\bbS_{p} \simeq E_{n}^{hG_{n}}$, where $G_{n}:= S_{n} \rtimes C_{n}$ is the semidirect product of $n$th profinite Morava stabilizer group and cyclic group of order $n$. Rognes conjectured $K(L_{K(n)}\bbS_{p}) \rightarrow K(E_{n})^{hG_{n}}$ is a `nice' map in sufficiently high dimensions in that up to smashing with a finite $p$-local $CW$-spectrum of chromatic type $n+1$ it is a weak equivalence. Hence, in \cite{10.1007/BF02392794}, they lay out a plan to analyze $K(E_{n})$. The theme is  to construct localization sequences in algebraic $K$-theory of connective commutative $\bbS$-algebras and then use the machinery of trace methods to compute things using $TC$. \\

The localization sequences conjectured in \cite{10.1007/BF02392794} towards this project were proven to not hold \cite{MR3760300}. Blumberg and Mandell in their recent work \cite{MR4096617} provide the following localization sequences
\[ \cdots \rightarrow K(W\F_{p^{n}}[[u_{1}, \cdots, u_{n-1}]]) \rightarrow K(BP_{n}) \rightarrow K(E_{n}) \rightarrow \Sigma \cdots \]
\[\cdots \rightarrow TC(W\F_{p^{n}}[[u_{1}, \cdots, u_{n-1}]]) \rightarrow TC(BP_{n}) \rightarrow TC(BP_{n}|E_{n}) \rightarrow \Sigma \cdots\]
where $W$ denotes the $p$-typical Witt ring, and $BP_{n}$ denotes the connective cover of $E_{n}$. These localization sequences are exceptional in the sense that they provide us with tools for calculating $K$-theory of non-connective ring spectra using localization sequences. Ausoni and Richter in \cite{MR4071375} make progress towards calculating $THH(E(n))$ under certain commutativity assumptions for $E(n)$ spectra. In particular, assuming $E(2)$ is $E_{\infty}$, they compute $THH(E(2))$. Motivated by some of their machinery and the fact that $E_{n}$ is a $E_{\infty}$ ring spectrum we want, in this paper, to compute $THH(E_{2})$. Such a computation can be useful in the broader Rognes program due to the relationship
\[
\begin{tikzcd}
THH(BP_{n}) \arrow{d} &\\
THH(BP_{n})[u^{-1}] \arrow{r}{\simeq} & THH(E_{n})
\end{tikzcd}
\]
and the fact that $THH(BP_{n})$ and eventually $TC(BP_{n})$ is crucial in the localization sequences mentioned above towards understanding $K(E_{n})$.\\

\subsection{Summary}

The goal of this paper is to study and describe $THH(E_{2})$ in terms of suspensions and localizations of $E_{2}$. $THH(E_{2})$ is an $E_{2}$ module along the unit map $E_{2} \rightarrow THH(E_{2})$. We show that this map induces a $K(2)_{\ast}$-equivalence, thus telling us that the cofiber of the unit map, denoted by $\overline{THH}(E_{2})$, is $L_{1}$-local. Next, using $K(i)_{\ast}THH(E_{2})$ calculations and lifts to classes in $\pi_{\ast}THH(E_{2})$, we show that  $\overline{THH}(E_{2})$ further sits in a cofiber sequence $X_{2} \rightarrow \overline{THH}(E_{2}) \rightarrow C_{f_{2}}$. This cofiber sequence is such that both $X_{2}$ and $C_{f_{2}}$ are $L_{1}$-local and we can describe $X_{2}$ in terms of explicit suspensions and localizations of $E_{2}$. Further, we show we have a map $X_{3} \rightarrow C_{f_{2}}$ where $X_{3}$ is again $L_{1}$-local and a wedge of suspensions and localizations of $E_{2}$. Moreover, $C_{f_{2}}[u_{1}^{-1}]$ is $K(1)$ equivalent to $X_{3}[u_{1}^{-1}]$. The maps $C_{f_{2}} \rightarrow C_{f_{2}}[u_{1}^{-1}]$ and $X_{3}[u_{1}^{-1}] \rightarrow C_{f_{2}}[u_{1}^{-1}]$ both have rational cofibers. The fact that the cofibers are rational allows us to culminate with a full description of $p$-completion of $THH(E_{2})$. Here is the statement of the main theorems:


\begin{surthm}\label{MainThm}
   There is a sequence of maps which fit into a diagram as follows:

\begin{equation}\label{thh}\tag{$\ast$}
\begin{tikzcd}%[row sep=large, column sep=large]
E_{2} \arrow{r}{f_{1}} & THH(E_{2}) \arrow{d}{C_{f_{1}}}\\
%& & \tikz\node[draw,circle,inner sep=2pt]{$\ell$}; \ar[d,dotted] & \\ 
X_{2} \arrow{r}{f_{2}} & C_{f_{1}}:=\overline{THH}(E_{2}) \arrow{d}{C_{f_{2}}}\\
X_{3} \arrow{r}{f_{3}} & C_{f_{2}}
\end{tikzcd}
\end{equation}

such that $X_{2} \rightarrow \overline{THH}(E_{2}) \rightarrow C_{f_{2}}$ is a cofiber sequence, and the map induced by $f_{3}$ on $X_{3}[u_{1}^{-1}] \rightarrow C_{f_{2}}[u_{1}^{-1}]$ is a $K(1)_{\ast}$-isomorphism with a rational cofiber. 
\end{surthm}\

 Here, by $X_{3}[u_{1}^{-1}]$ and $C_{f_{2}}[u_{1}^{-1}]$ we mean the spectra hocolim$(X_{3}\xrightarrow{u_{1}}X_{3}\xrightarrow{u_{1}}X_{3}\xrightarrow{u_{1}}\cdots)$ and hocolim$(C_{f_{2}}\xrightarrow{u_{1}}C_{f_{2}}\xrightarrow{u_{1}}C_{f_{2}}\xrightarrow{u_{1}}\cdots)$ where the $u_{1}$-multiplication is defined by the $E_{2}$-module structure of $X_{3}$ and $C_{f_{2}}$. Further, $X_{2}$ and $X_{3}$ are explicitly identifiable in terms of suspensions and localizations of $E_{2}$; and $X_{2}$, $X_{3}$, and $C_{f_{2}}$ are all $L_{1}$-local. Here, $C_{f_{i}}$ denotes both the cofiber and cofiber map corresponding to $f_{i}$. 
 
 The rational cofiber vanishes on taking $p$-completions:

 
 \begin{surthm}\label{MainCorr}
We have the following diagram for $THH(E_{2})_{p}^{\wedge}$, where $(C_{f_{i}})_{p}^{\wedge}$ are cofibers and cofiber maps of $(f_{i})_{p}^{\wedge}$ and $(\overline{Q})_{p}^{\wedge}$ is an equivalence
\[\begin{tikzcd}%[row sep=large, column sep=large]
(E_{2})_{p}^{\wedge} \arrow{r}{(f_{1})_{p}^{\wedge}:= \text{p-completed unit map}} & THH(E_{2})_{p}^{\wedge} \arrow{d}{(C_{f_{1}})_{p}^{\wedge}}\\
%& & \tikz\node[draw,circle,inner sep=2pt]{$\ell$}; \ar[d,dotted] & \\ 
\Sigma^{2p-1}L_{1}(E_{2})_{p}^{\wedge} \arrow{r}{(f_{2})_{p}^{\wedge}:=(\overline{j_{1}})_{p}^{\wedge}} & (C_{f_{1}})_{p}^{\wedge}:=\overline{THH}(E_{2})_{p}^{\wedge} \arrow{d}{(C_{f_{2}})_{p}^{\wedge}}\\
(\bigvee_{\alpha} (\Sigma^{|\alpha|}L_{1}E_{2}[u_{1}^{-1}] \bigvee \Sigma^{|\alpha|+2p-1}L_{1}E_{2}[u_{1}^{-1}]))_{p}^{\wedge} \arrow{r}{(\overline{Q})_{p}^{\wedge}}[swap]{\simeq} & (C_{f_{2}})_{p}^{\wedge}
\end{tikzcd}
\]
(see \S 4.4 for details on indexing, $\alpha$).
\end{surthm}\


Throughout this paper, we will be working with some key homological classes that arise out of $BP$, $BP_{\ast}BP$-theory and maps $BP \rightarrow K(n)$, $BP \rightarrow E(2) \rightarrow E_{2}$. More particularly, note that $E(2)$ and $E_{2}$ have canonical choices of formal group laws, i.e., canonical ring maps from $BP$. Thus, they have canonical classes in their respective $\pi_{\ast}$-groups called $v_{i}$, which are the images of the classes $v_{i}$ in $BP_{\ast}$. Similarly, $\pi_{\ast}(E \wedge E) = E_{\ast}E$ (for $E = E(2)$ or $E_{2}$ or any $p$-typical complex oriented theory) have classes $v_i$ and $t_j$ from $BP_*BP\cong \bbZ_{(p)}[v_i,t_j]$, where the classes $v_{i}$ are the left Hurewicz image, $\eta_{L}(v_{i})$. $K(n)_{\ast}E$, in the same vein, has classes $v_i$ from the  $\eta_{L}(v_{i})$ (going into $K(n)$) and $t_{j}$. There is a Hurewicz map 
\[\pi_{\ast}(E \wedge E)\to K(n)_{\ast}(E \wedge E) \cong K(n)_{\ast}E \otimes_{K(n)_{\ast}} K(n)_{\ast}E\]
(the isomorphism holds since $K(n)$ satisfies the K\"unneth theorem). 
We can see where the class $t_i$ goes by looking at the corresponding Hurewicz map for $BP$,
\[ \pi_\ast(BP\wedge BP)\to BP_\ast(BP\wedge BP) \cong BP_\ast BP \otimes_{BP_\ast} BP_\ast BP\]
(the isomorphism holds since $BP_\ast BP$ is flat over $BP_\ast$).
Stardard formulae for the Hopf algebroid diagonal and antipode~\cite[A2.1.27]{MR860042} then in principle calculate the image of $t_i$ for all $i$.  In the case $i=1$, we get the formula $t_1\otimes 1+1\otimes t_1$, which implies
\[ t_1 \mapsto t_1\otimes 1+1\otimes t_1 \in  K(n)_{\ast}E \otimes_{K(n)_{\ast}} K(n)_{\ast}E.\]
Now, $K(n)_{\ast}E \otimes_{K(n)_{\ast}} K(n)_{\ast}E$ sits at level one of the Hochschild complex of $K(n)_{\ast}E$ over $K(n)_{\ast}$, where $t_1\otimes 1$ is degenerate, and so the image of $t_1$ represents the same class as $1 \otimes t_{1}$ in $HH_{1}^{K(n)_{\ast}}K(n)_{\ast}E$. Finally, for $E = E_{m}$ we have classes $u_{i}, u \in K(n)_{\ast}E$ which map under the right Hurewicz map to $1 \otimes u_{i}, 1 \otimes u$ in $K(n)_{\ast}E \otimes_{K(n)_{\ast}} K(n)_{\ast}E$ and further belong in $HH_{1}^{K(n)_{\ast}}K(n)_{\ast}E$.  

\begin{notation}\label{Classes}
Let $dt_1\in HH_{(1,2p-2)}^{K(n)_{\ast}}K(n)_{\ast}E$ denote the homological class $1 \otimes t_{1}$ coming from the element $t_1\in \pi_{2p-2}(E \wedge E)$. Similarly, let $du_{i} \in HH_{(1,0)}^{K(n)_{\ast}}K(n)_{\ast}E_{m}$ and $du \in HH_{(1,2)}^{K(n)_{\ast}}K(n)_{\ast}E_{m}$ denote classes $u_{i}$ and $u$ in $K(n)_{\ast}E_{m}$ which map under the right Hurewicz map to $1 \otimes u_{i}, 1 \otimes u$ in $K(n)_{\ast}E_{m} \otimes_{K(n)_{\ast}} K(n)_{\ast}E_m$.
\end{notation}\

As in the paper by Ausoni-Richter \cite{MR4071375}, our work starts with the calculation of $K(i)$-homology of $THH(E_{2})$, $0 \leq i \leq 2$, in section 2. In particular, we calculate $HH_{\ast, \ast}^{K(i)_{\ast}}K(i)_{\ast}E_{2}$ for $0 \leq i \leq 2$ and then use a Bökstedt spectral sequence to conclude results for $K(i)_{\ast}THH(E_{2})$. We get the following:

\begin{surthm} 
We have following isomorphisms of $K(i)_{\ast}E_{2}$-algebras
\begin{align*}
K(0)_{\ast}THH(E_{2}) & \cong \bigg(K(0)_{\ast}E_{2}\bigg) \otimes_{X} \bigg(\bigwedge_{X}HH_{1}^{\bbQ}(X)\bigg) \otimes_{\bbQ} \bigg(\bigwedge_{\bbQ}du\bigg) \\
K(1)_{\ast}THH(E_{2}) & \cong \bigg(K(1)_{\ast}E_{2}\bigg) \otimes_{{F}_{p}[[u_{1}]]} \bigg(\bigwedge_{\mathbb{F}_{p}[[u_{1}]]}HH_{1}^{\F_{p}[u_{1}]}\F_{p}[[u_{1}]]\bigg) \otimes_{\F_{p}} \bigg(\bigwedge_{\F_{p}} dt_{1}\bigg)\\
K(2)_{\ast} THH (E_{2}) & \cong K(2)_{\ast}E_{2}
\end{align*}
where $X:= \mathbb{Z}_{p}[[u_{1}]] \otimes_{\bbZ} \bbQ$. $THH(E_{2})$ is $L_{2}$-local and hence $K(i)_{\ast} THH (E_{2}) = 0$ for $i > 2$.
\end{surthm}

Here $\bigwedge_{R}M$ is the  exterior $R$-algebra (graded) on $M$. It's $R$ in degree $0$ and $M$ in degree $1$. And, $\bigwedge_{R}X:= \bigwedge_{R}R\langle X\rangle$.\\

%Chapter three is a brief chapter and gives some basic results that we can conclude about the cofiber of the unit map $\overline{THH}(E_{2})$ from results like the fracture square in chromatic homotopy theory. It allows us to say some similar things about other cofibers later in chapter 5 as well.\\

Section 3 deals with the following important question: given that we know $K(i)_{\ast}THH(E_{2})$, can we lift these classes along the Hurewicz map to $\pi_{\ast}THH(E_{2})$? The answer turns out to be yes for all homological classes of $K(1)_{\ast}THH(E_{2})$. We then figure out where these classes map down to in $K(0)_{\ast}THH(E_{2})$. For all this, we use the fact that $\pi_*(E_2 \wedge E_2)$ is part of the filtration spectral sequence $E^{1}_{s,t} \cong \pi_{t}(E_{2}^{\wedge s+1}) \Rightarrow \pi_{\ast}THH(E_{2})$ which implies that the classes in $\pi_*(E_2 \wedge E_2)$ give classes in $\pi_*(THH(E_2))$.\\

Finally, in section 4 we do our analysis of $THH(E_{2})$ based on everything we built in the previous sections. We get, $THH(E_{2}) \simeq E_{2} \bigvee \overline{THH}(E_{2})$ and that $\overline{THH}(E_{2})$ sits in cofiber sequence $\Sigma^{2p-1}L_{1}E_{2} \rightarrow \overline{THH}(E_{2}) \rightarrow C_{f_{2}}$. Further, we have following commutative diagram for $C_{f_{2}}$
\[\begin{tikzcd}%[column sep=large]
\vee_{\alpha} (\Sigma^{|\alpha|}L_{1}E_{2} \vee \Sigma^{|\alpha|+2p-1}L_{1}E_{2}) \arrow{d}{\widetilde{Q}} \arrow{r}
& \vee_{\alpha} (\Sigma^{|\alpha|}L_{1}E_{2}[u_{1}^{-1}] \vee \Sigma^{|\alpha|+2p-1}L_{1}E_{2}[u_{1}^{-1}])
\arrow{d}{\overline{Q}}
& \\
C_{f_{2}} \arrow{r} & C_{f_{2}}[u_{1}^{-1}] \arrow{d} \arrow{r} & C_{\infty}
& \\
                                   & C_{\overline{Q}}
\end{tikzcd}
\]
where $\alpha$ varies over all non-empty subsets of an infinite set that we define in section 4. $\overline{Q}$ is a $K(1)_{\ast}$-isomorphism, and $C_{\infty}$ and $C_{\overline{Q}}$ are rational cofibers. These cofibers vanish after taking $p$-completion thus giving us an explicit description of $(C_{f_{2}})^{\wedge}_{p}$, we get $(C_{f_{2}})^{\wedge}_{p} \cong (\bigvee_{\alpha} (\Sigma^{|\alpha|}L_{1}E_{2}[u_{1}^{-1}] \bigvee \Sigma^{|\alpha|+2p-1}L_{1}E_{2}[u_{1}^{-1}]))^{\wedge}_{p}$.

\subsubsection*{Acknowledgements}
This paper would not have been possible without many insightful conversations with my PhD advisor Michael Mandell. The author is also grateful to David Mehrle, Srikanth B. Iyengar, and Noah Riggenbach for useful remarks and to Ayelet Lindenstrauss for numerous helpful comments and suggestions. 

\section{$K(i)_{\ast}THH(E_{2})$ calculations}

In this section, we present the calculations for $K(i)_{\ast}THH(E_{2})$, where $i$ ranges from $0$ to $2$. For each of these cases, we calculate $HH_{\ast}^{K(i)_{\ast}}(K(i)_{\ast}E_{2})$ and then use a Bökstedt spectral sequence to conclude results for $K(i)_{\ast}THH(E_{2})$. In each case, it turns out that the Bökstedt spectral sequence collapses since all the generators are concentrated in columns $0$ and $1$.\\

\S2.1 gives all the major prerequisites required for the computations we do in this section. \S2.2, \S2.3 and \S2.4 deal with calculations for $K(0)_{\ast}THH(E_{2})$, $K(1)_{\ast}THH(E_{2})$ and $K(2)_{\ast}THH(E_{2})$, respectively.

\subsection{Some prerequisites}

To be able to do the required Hochschild homology calculations, besides the Bökstedt spectral sequence, we need to know Hochschild homology of certain algebraic rings. We also need results that tell us how Hochschild homology behaves with respect to nice maps e.g. maps that are étale, flat, `$H$-smooth', etc. Finally we make use of $HH_{\ast, \ast}^{K(i)_{\ast}}K(i)_{\ast}E(2)$ computations due to Ausoni and Richter \cite[\S3]{MR4071375}.

\subsubsection{Bökstedt spectral sequence}\label{2.1.1}

For homology theories that satisfy Künneth theorem, we have a Bökstedt spectral sequence. Examples include ordinary homology theory and Morava $K$-theories, $K(n)$.  For our case it looks as follows
\[HH_{\ast, \ast}^{K(i)_{\ast}}K(i)_{\ast}E_{2} \implies K(i)_{\ast}THH(E_{2}).\]
These arise from the bar complex of $E_{2}$ by taking $K(i)_{\ast}$ at each level and then noticing that $K(i)_{\ast}(E_{2}^{\wedge n}) \cong (K(i)_{\ast}E_{2})^{\otimes_{K(i)_{\ast}}^{n}}$ due to the Künneth theorem. Thus the simplicial filtration actually turns out to be the Hochschild complex of $K(i)_{\ast}E_{2}$ and the resultant filtration spectral sequence gives us the Bökstedt spectral sequence. 

\subsubsection{Hochschild homology and étale extensions}\label{2.1.2}

Let $A$ be a commutative algebra and let $A \subset A'$ be an étale extension. Then by \cite[E.1.1.8]{MR1600246},
\[ HH_{\ast}(A') \cong HH_{\ast}(A) \otimes_{A} A'. \]

\subsubsection{$H$-smoothness of certain rings}\label{2.1.3}

We need to know Hochschild homology of rings like $\bbZ[[t]] \otimes \mathbb{Q}_{p}$ over $\mathbb{Q}$ and $\mathbb{F}_{p}[[t]]$ over $\mathbb{F}_{p}$. The results related to these come from \cite{MR1853116} where Larsen and Lindenstrauss define the notion of $H$-smoothness as follows:

\begin{dfn} \label{Def 2.1.3.1}
Let $k$ be a commutative ring and $A$ a commutative $k$-algebra. Then $A$ is $H$-smooth over $k$ if $HH_{\ast}(A)$ is flat over $k$ and the Hochschild-Kostant-Rosenberg map $\lambda_{n}^{A/k}: \bigwedge^{n}HH_{1}^{k}(A) \rightarrow HH_{n}^{k}(A)$ is an isomorphism for all $n$.
\end{dfn}\

They go on to provide multiple criteria to check for $H$-smoothness of a ring, and in the process conclude results about Hochschild homology of certain rings. Some of these results will be particularly useful for us, hence we mention them here:

\begin{prop} \label{Prop 2.1.3.2}
$H$-smoothness respects arbitrary base change \cite[Proposition 1.3]{MR1853116}.
\end{prop}

\begin{prop} \label{Lemma 2.1.3.3}
Let $A$ be a $k$-algebra and $B$ an $A$-algebra. Then if $A$ is $H$-smooth over $k$ and $B$ is $H$-smooth over $A$, $B$ is $H$-smooth over $k$ \cite[Corollary 1.6]{MR1853116}.
\end{prop}

\begin{prop} \label{Thm 2.1.3.4}
If $A$ is an excellent ring, $\mathfrak{m}$ is a maximal ideal in $A$, and $\hat{A}_{\mathfrak{m}}$ is the completion of $A$ at $\mathfrak{m}$, then $\hat{A}_{\mathfrak{m}}$ is $H$-smooth over $A$ \cite[Theorem 1.7]{MR1853116}.
\end{prop}


Excellent rings is an extremely broad category of rings and encompasses most rings algebraic geometers work with. In particular, some of the examples include $\mathbb{Q}$, $\mathbb{Q}_{p}$, $\mathbb{Z}_{p}[t]$, $\mathbb{F}_{p}$, $\mathbb{F}_{p}[t]$. From proposition \ref{Thm 2.1.3.4} we conclude that the following are $H$-smooth: $\mathbb{Q}_{p}$ over $\mathbb{Q}$, $\mathbb{Q}_{p}[t]$ over $\mathbb{Q}_{p}$, $\mathbb{Z}_{p}[[t]]$ over $\mathbb{Z}_{p}[t]$, $\mathbb{F}_{p}[t]$ over $\mathbb{F}_{p}$, and $\mathbb{F}_{p}[[t]]$ over $\mathbb{F}_{p}[t]$. From proposition \ref{Prop 2.1.3.2}, we can do a base change on $\mathbb{Z}_{p}[[t]]$ over $\mathbb{Z}_{p}[t]$ by tensoring with $\bbQ$ to get $\mathbb{Z}_{p}[[t]] \otimes \bbQ$ is $H$-smooth over $\mathbb{Q}_{p}[t]$. Finally, we use proposition \ref{Lemma 2.1.3.3} to conclude $H$-smoothness of $\bbZ_{p}[[t]] \otimes \bbQ$ over $\bbQ$ (using $\bbQ \rightarrow \bbQ_{p} \rightarrow \bbQ_{p}[t] \rightarrow \bbZ_{p}[[t]] \otimes \bbQ$) and $\bbF_{p}[[t]]$ over $\bbF_{p}$ (using $\bbF_{p} \rightarrow \bbF_{p}[t] \rightarrow \bbF_{p}[[t]]$).\\

The final result we need is \cite[Proposition 2.5]{MR1853116}, presented here slightly reformulated to cater best to our needs:

\begin{prop}\label{Prop 2.1.3.5}
 Let $k$ be a field, then there is a non-canonical isomorphism
 \[ HH_{\ast}^{k}k[[t]] \cong HH_{\ast}^{k}k[t] \otimes_{k[t]} HH_{\ast}^{k[t]}k[[t]] \cong HH_{\ast}^{k}k[t] \otimes_{k[t]} (k[[t]] \oplus \bigwedge_{k((t))}^{\geq 1} W_{k}) \]
 where $W_{k} := HH_{1}^{k(t)}k((t))$ is an infinite dimensional vector space over $k((t))$.
\end{prop}

Here, $k[[t]]$ in the bracketed term is in homological degree $0$ and $W_{k}$ is in homological degree $1$. The latter isomorphism is because \cite[Corollary 2.3]{MR1853116} tells us $HH_{i}^{k[t]}k[[t]] = HH_{i}^{k(t)}k((t))$ for $i \geq 1$ and thus $$\bigwedge_{k[[t]]}^{\geq 1}HH_{1}^{k[t]}k[[t]] \cong \bigwedge_{k((t))}^{\geq 1}HH_{1}^{k(t)}k((t)).$$ We will be applying this result for $k = \bbF_{p}$.

\subsubsection{Results on $HH_{\ast, \ast}^{K(i)_{\ast}}K(i)_{\ast}E(2)$}
%***Wanna state the results for $HH_{\ast, \ast}^{K(i)_{\ast}}THH$
In \cite{MR4071375} Ausoni and Richter make progress toward $THH$-computations of Johnson-Wilson theories, $E(n)$.  This is the starting point for our analogous computations here for $E_{n}$. In particular for the second Johnson-Wilson theory $E(2)$, under certain commutativity assumptions, they give an explicit description of $THH(E(2))$. As a first step they compute $HH_{\ast, \ast}^{K(i)_{\ast}}K(i)_{\ast}E(2)$ in \cite[Proposition 2.3, Theorem 3.4]{MR4071375} with the intention of using Bökstedt spectral sequence for computing $K(i)_{\ast}THH(E(2))$. The $HH_{\ast, \ast}^{K(i)_{\ast}}K(i)_{\ast}E(2)$ computations hold as they are, and do not need any of the assumptions used for further results in the paper and will be useful to us towards calculating $HH_{\ast, \ast}^{K(i)_{\ast}}K(i)_{\ast}E_{2}$. We state the results here in a slightly reformulated form, using discussions in proof of \cite[lemma 5.1]{MR4071375}.

\begin{theorem} \label{Thm 2.1.4.1} We have the following isomorphisms of $K(i)_{\ast}E(2)$-algebras
\[HH_{\ast, \ast}^{K(0)_{\ast}}K(0)_{\ast}E(2) \cong K(0)_{\ast}E(2) \otimes \bigwedge_{\bbQ}(dt_{1}, dt_{2}),\]
\[HH_{\ast, \ast}^{K(1)_{\ast}}K(1)_{\ast}E(2) \cong K(1)_{\ast}E(2) \otimes \bigwedge_{\bbF_{p}}(dt_{1}),\]
\[HH_{\ast, \ast}^{K(2)_{\ast}}K(2)_{\ast}E(2) \cong K(2)_{\ast}E(2),\]
where $dt_{1}$ is the class in $HH_{(1, 2p-2)}^{K(0)_{\ast}}K(0)_{\ast}E(2)$ and $HH_{(1, 2p-2)}^{K(1)_{\ast}}K(1)_{\ast}E(2)$ coming from the class $t_{1}$ in $\pi_{2p-1}(E(2) \wedge E(2))$ and $dt_{2}$ is the class in $HH_{(1, 2p^{2}-2)}^{K(0)_{\ast}}K(0)_{\ast}E(2)$ coming from the class $t_{2}$ in $\pi_{2p^{2}-1}(E(2) \wedge E(2))$.
\end{theorem}

With these tools, we are ready to proceed to $K(i)_{\ast}THH(E_{2})$ calculations. \S2.2 presents $K(0)_{\ast}THH(E_{2})$, \S2.3 deals with $K(1)_{\ast}THH(E_{2})$ and finally \S2.4 talks about $K(2)_{\ast}THH(E_{2})$. 

\subsection{$K(0)_{\ast}THH(E_{2})$}

$K(0)_{\ast}E_{2}$ is just the homotopy groups of rationalized $E_{2}$-spectra, which we know explicitly. It is made up of rings that we can compute the Hochschild homology of and hence eventually this allows us to calculate $K(0)_{\ast}THH(E_{2})$. We discuss the computation in detail in this section and show that

\begin{theorem}\label{Thm 2.2.1}
We have the following isomorphism of $K(0)_{\ast}E_{2}$-algebras
\[K(0)_{\ast}THH(E_{2}) \cong K(0)_{\ast}E_{2} \otimes_{X} \bigwedge_{X}HH_{1}^{\bbQ}(X) \otimes_{\bbQ} \bigwedge_{\bbQ}du\]
where $du$ is as in \ref{Classes} and $X := \mathbb{Z}_{p}[[u_{1}]] \otimes_{\bbZ} \bbQ$.
\end{theorem}

The rest of \S2.2 is devoted to proving this result.\\

Since $HH_{\ast}^{K(0)_{\ast}}(K(0)_{\ast}E_{2}) \cong HH_{\ast}^{\mathbb{Q}}(\pi_{\ast}(E_{2})_{\mathbb{Q}})$, we use
\[\pi_{\ast}(E_{2})_{\mathbb{Q}} \cong (\bbZ_{p}[\mu_{p^{2}-1}][u, u^{-1}][[u_{1}]]) \otimes \bbQ \cong (\mathbb{Z}_{p}[[u_{1}]] \otimes_{\bbZ} \bbQ) \otimes_{\mathbb{Q}} \mathbb{Q}[\mu_{p^{2}-1}][u,u^{-1}] \]
to conclude
\[HH_{\ast}^{\mathbb{Q}}(\pi_{\ast}(E_{2})_{\mathbb{Q}}) \cong HH_{\ast}^{\mathbb{Q}}(\mathbb{Z}_{p}[[u_{1}]] \otimes_{\bbZ} \bbQ) \otimes_{\mathbb{Q}} HH_{\ast}^{\mathbb{Q}}\mathbb{Q}[\mu_{p^{2}-1}] \otimes_{\mathbb{Q}} HH_{\ast}^{\mathbb{Q}}\mathbb{Q}[u,u^{-1}].\]
Here $\mu_{p^{2}-1}$ denotes the $p^{2}-1$ root of unity. Thus we need to do these three Hochschild homology computations, which we illustrate now.

\begin{comment}
\subsection{$HH_{\ast}^{\bbQ}(\bbQ_{p})$}

By Theorem \ref{Thm 2.1.3.2}, $\bbQ_{p}$ is $H$-smooth over $\Q$. Hence, $HH_{\ast}^{\mathbb{Q}}(\mathbb{Q}_{p}) \cong \bigwedge_{\bbQ_{p}}HH_{1}^{\bbQ}\bbQ_{p}$.
%\cong \mathbb{Q}_{p} \otimes_{\mathbb{Q}} \bigwedge_{\mathbb{Q}} HH_{1}^{\mathbb{Q}}(\mathbb{Q}_{p}) 
%{\color{red}check this second iso - doesn't really make sense}

%$HH_{*}^{\mathbb{Q}}(\mathbb{Q}_{p}) \cong \mathbb{Q} \otimes HH_{*}^{\mathbb{Z}}(\mathbb{Z}_{p})$ %\cong HH_{*}^{\mathbb{Z}}(\mathbb{Z}_{p})$ \textcolor{blue}{last iso only in pos degs}
%(By Larsen-Lindenstrauss $Thm 1.7$, $\hat{ \mathbb{Z}_{p}}$ is $H$-smooth over $\mathbb{Z}_{p}$ since $\mathbb{Z}$ is excellent and $\mathbb{Z}_{p}$ is completion of $\mathbb{Z}$ at $p$; this tells us that $\mathbb{Z}_{p}$ is $H$-smooth over $\mathbb{Z}$ and hence $HH_{i}^{\mathbb{Z}}(\mathbb{Z}_{p}) = \bigwedge_{\mathbb{Z}_{p}}^{i}(HH_{1}^{\mathbb{Z}}(\mathbb{Z}_{p}))$, $HH_{1}^{\mathbb{Z}}(\mathbb{Z}_{p})$: vector space over $\mathbb{Q}_{p}$ of dimension same as transcendence degree of $\mathbb{Q}_{p}$ over $\mathbb{Q}$ which is the same as the cardinality of the continuum.\\
\end{comment}

\subsubsection{$HH_{\ast}^{\mathbb{Q}}(\mathbb{Q}[u,u^{-1}])$}\label{2.2.1}

$\bbQ[u,u^{-1}]$ is  étale over $\mathbb{Q}[u]$. Thus, from \S\ref{2.1.2}, $$HH_{\ast}^{\mathbb{Q}}(\mathbb{Q}[u,u^{-1}]) \cong \mathbb{Q}[u,u^{-1}] \otimes_{\mathbb{Q}[u]} HH_{\ast}^{\mathbb{Q}}\mathbb{Q}[u].$$ Further, $\mathbb{Q}[u]$ is smooth over $\mathbb{Q}$ and hence from theorem of Hochschild-Kostant-Rosenberg,
%{\color{red}figure out how the iso $HH_{1}^{k}(k[t]) \cong \Omega \cong dt$ works because its def not a vector space in 1 generator, its an algebra in one generator}\\
$HH_{\ast}^{\bbQ}\bbQ[u] \cong \bbQ[u] \otimes \bigwedge_{\bbQ}(du)$. We get,
\[HH_{\ast}^{\mathbb{Q}}(\mathbb{Q}[u,u^{-1}]) \cong \mathbb{Q}[u,u^{-1}] \otimes_{\mathbb{Q}[u]} (\mathbb{Q}[u] \otimes \bigwedge_{\mathbb{Q}}(du)) \cong \mathbb{Q}[u,u^{-1}] \otimes \bigwedge_{\mathbb{Q}}(du).\]

\subsubsection{$HH_{\ast}^{\mathbb{Q}}(\mathbb{Q}[\mu_{p^{2}-1}])$}\label{2.2.2}

$\mathbb{Q}[\mu_{p^{2}-1}]$ is étale over $\mathbb{Q}$. Hence, again from \S\ref{2.1.2}, $$HH_{\ast}^{\mathbb{Q}}(\mathbb{Q}[\mu_{p^{2}-1}]) \cong %HH_{0}(\mathbb{Q}[\mu_{p^{2}-1}]) \cong 
\mathbb{Q}[\mu_{p^{2}-1}] \otimes HH_{\ast}^{\mathbb{Q}} \mathbb{Q} \cong \mathbb{Q}[\mu_{p^{2}-1}].$$ %, $\mathbb{Q}[\mu_{p^{n}-1}] \cong \mathbb{Q}[x]/(f(x))$ where $f$ is the irreducible polynomial of $\mu_{p^{n}-1}$ but for $\mathbb{Q}[x]/(f(x)))$: $HH_{0} \cong \mathbb{Q}[x]/(f(x)), HH_{even_{>0}} = Ann_{\mathbb{Q}[x]/(f(x))}(f_{'}(x)), HH_{odd_{>0}} = \mathbb{Q}[x]/(f(x), f^{'}(x))$ and $(f(x), f^{'}(x)) = 1$ and $Ann_{\mathbb{Q}[x]/(f(x))}(f_{'}(x)) = 0$ as $Q[x]$ is Euclidean domain and UFD) \textcolor{blue}{I would write $HH_{*}^{\mathbb{Q}}(\mathbb{Q}[\mu_{p^{n}-1}])\cong \mathbb{Q}[\mu_{p^{n}-1}]\otimes HH_{*}^{\mathbb{Q}}(\mathbb{Q})\cong \mathbb{Q}[\mu_{p^{n}-1}]$ because $\mathbb{Q}[\mu_{p^{n}-1}]$ etale over $\mathbb{Q}$}

\subsubsection{$HH_{\ast}^{\bbQ}(\mathbb{Z}_{p}[[u_{1}]] \otimes_{\bbZ} \bbQ)$}\label{2.2.3}

As mentioned before, note that $\bbQ \rightarrow \bbQ_{p}$, $\bbQ_{p} \rightarrow \bbQ_{p}[u_{1}]$, and $\bbZ_{p}[u_{1}] \rightarrow \bbZ_{p}[[u_{1}]]$ are all $H$-smooth maps. Base change to $\bbQ_{p}$ on the last map tells us $\bbQ_{p}[u_{1}] \rightarrow \bbZ_{p}[[u_{1}]] \otimes \bbQ$ is $H$-smooth as well. Composing with $\bbQ \rightarrow \bbQ_{p}$ and $\bbQ_{p} \rightarrow \bbQ_{p}[u_{1}]$ tells us $\bbQ \rightarrow \bbZ_{p}[[u_{1}]] \otimes \bbQ$ is $H$-smooth. Thus, by definition of $H$-smoothness (definition \ref{Def 2.1.3.1}), we have 
\[HH_{\ast}^{\bbQ}(\mathbb{Z}_{p}[[u_{1}]] \otimes_{\bbZ} \bbQ) \cong \bigwedge_{\mathbb{Z}_{p}[[u_{1}]] \otimes_{\bbZ} \bbQ}HH_{1}^{\bbQ}(\mathbb{Z}_{p}[[u_{1}]] \otimes_{\bbZ} \bbQ).\]

\begin{comment}
$\bbQ[u_{1}]$ is an excellent ring and the ideal generated by $\langle u_{1} \rangle$ is maximal. Localizing $\bbQ[u_{1}]$ at this maximal ideal gives us $\bbQ[[u_{1}]]$. Hence by theorem \ref{Thm 2.1.3.2}, $\bbQ[[u_{1}]]$ is $H$-smooth over $\bbQ[u_{1}]$. From proposition \ref{Prop 2.1.3.3}, we get %$HH_{\ast}^{\mathbb{Q}} \mathbb{Q}[[u_{1}]] \cong HH_{\ast}^{\mathbb{Q}} \mathbb{Q}[u_{1}] \otimes_{\mathbb{Q}[u_{1}]} HH_{\ast}^{\mathbb{Q}(u_{1})}\mathbb{Q}((u_{1})) \ (\text{except in deg 0 which we will correct in the following step}) \ \cong HH_{\ast}^{\mathbb{Q}} \mathbb{Q}[u_{1}] \otimes_{\mathbb{Q}[u_{1}]} (\mathbb{Q}[[u_{1}]] \oplus \bigwedge^{\geq 1}_{\mathbb{Q}((u_{1}))} HH_{1}^{\mathbb{Q}(u_{1})}\mathbb{Q}((u_{1})))$. Note that $HH_{\ast}^{\mathbb{Q}} \mathbb{Q}[u_{1}] \cong \mathbb{Q}[u_{1}] \otimes \bigwedge_{\mathbb{Q}} (du_{1})$. 
\[HH_{\ast}^{\mathbb{Q}} \mathbb{Q}[[u_{1}]] \cong (\mathbb{Q}[u_{1}] \otimes \bigwedge_{\mathbb{Q}} (du_{1})) \otimes_{\mathbb{Q}[u_{1}]} \bigwedge_{\mathbb{Q}[[u_{1}]]} HH_{1}^{\mathbb{Q}[u_{1}]}\mathbb{Q}[[u_{1}]]\]
which can be rewritten as
\[ HH_{\ast}^{\mathbb{Q}} \mathbb{Q}[[u_{1}]] \cong (\mathbb{Q}[[u_{1}]] \otimes \bigwedge_{\mathbb{Q}} (du_{1})) \otimes_{\mathbb{Q}[[u_{1}]]} \bigwedge_{\mathbb{Q}[[u_{1}]]} HH_{1}^{\mathbb{Q}[u_{1}]}\mathbb{Q}[[u_{1}]]. \]
%$HH_{*}^{\mathbb{Q}}(\mathbb{Q}[[u_{1},\cdots,u_{n-1}]])$ - this is the trickiest one here. $HH_{*}^{\mathbb{Q}}(\mathbb{Q}[u_{1},\cdots,u_{n-1}])$ is an excellent ring ($\mathbb{Z}$ is excellent, $\mathbb{Q}$ is excellent, hence so is $\mathbb{Q}[u_{1},\cdots,u_{n-1}]$), $<u_{1},\cdots,u_{n-1}>$ is a maximal ideal in $\mathbb{Q}[u_{1},\cdots,u_{n-1}]$. Hence by Larsen-Lindenstrauss Thm 1.7, we know that $\hat{A}_{\mathfrak{m}}$ is $H$-smooth over $A$ where $A = \mathbb{Q}[u_{1},\cdots,u_{n-1}]$, $\mathfrak{m} = <u_{1},\cdots,u_{n-1}>$ and $\hat{A}_{\mathfrak{m}} = \mathbb{Q}[[u_{1},\cdots,u_{n-1}]]$. Since $\mathbb{Q}[u_{1},\cdots,u_{n-1}]$ is $H$-smooth over $\mathbb{Q}$, $\mathbb{Q}[[u_{1},\cdots,u_{n-1}]]$ is $H$-smooth over $\mathbb{Q}$. We get $HH_{\ast}^{\mathbb{Q}} \mathbb{Q}[[u_{1},\cdots,u_{n-1}]] \cong \bigwedge_{\mathbb{Q}[[u_{1},\cdots,u_{n-1}]]} HH_{1}^{\mathbb{Q}} \mathbb{Q}[[u_{1},\cdots,u_{n-1}]]$ from LL. When $n=2$, this reduces to $HH_{\ast}^{\mathbb{Q}} \mathbb{Q}[[u_{1}]] \cong \bigwedge_{\mathbb{Q}[[u_{1}]]} HH_{1}^{\mathbb{Q}} \mathbb{Q}[[u_{1}]]$. Note that, in the case of $n = 2$, we can say something better. Using, proposition 2.5 from L-L, we have $HH_{\ast}^{\mathbb{Q}} \mathbb{Q}[[u_{1}]] \cong HH_{\ast}^{\mathbb{Q}} \mathbb{Q}[u_{1}] \otimes_{\mathbb{Q}[u_{1}]} HH_{\ast}^{\mathbb{Q}(u_{1})}\mathbb{Q}((u_{1})) \ (\text{except in deg 0 which we will correct in the following step}) \ \cong HH_{\ast}^{\mathbb{Q}} \mathbb{Q}[u_{1}] \otimes_{\mathbb{Q}[u_{1}]} (\mathbb{Q}[[u_{1}]] \oplus \bigwedge^{\geq 1}_{\mathbb{Q}((u_{1}))} HH_{1}^{\mathbb{Q}(u_{1})}\mathbb{Q}((u_{1})))$. Note that $HH_{\ast}^{\mathbb{Q}} \mathbb{Q}[u_{1}] \cong \mathbb{Q}[u_{1}] \otimes \bigwedge_{\mathbb{Q}} (du_{1})$. Hence, $HH_{\ast}^{\mathbb{Q}} \mathbb{Q}[[u_{1}]] \cong (\mathbb{Q}[u_{1}] \otimes \bigwedge_{\mathbb{Q}} (du_{1})) \otimes_{\mathbb{Q}[u_{1}]} (\mathbb{Q}[[u_{1}]] \oplus \bigwedge^{\geq 1}_{\mathbb{Q}((u_{1}))} HH_{1}^{\mathbb{Q}(u_{1})}\mathbb{Q}((u_{1})))$.\\
\end{comment}

\subsubsection{Putting them together}\label{2.2.4}

Now $HH_{\ast}^{K(0)_{\ast}}(K(0)_{\ast}E_{2})$ is tensor of all of the terms calculated above. One notices that one can rewrite the tensor of all the terms together such that we get back a copy of $K(0)_{\ast}E_{2}$ in the tensor product. In other words, $HH_{\ast}^{K(0)_{\ast}}(K(0)_{\ast}E_{2})$ becomes
\[ (X \otimes \bbQ[u,u^{-1}] \otimes \bbQ[\mu_{p^{2}-1}]) \otimes_{X} \bigg(\bigwedge_{X} HH_{1}^{\mathbb{Q}}(X)\bigg) \otimes_{\bbQ} \bigwedge_{\bbQ}du\]
where $X$ is the ring $\mathbb{Z}_{p}[[u_{1}]] \otimes_{\bbZ} \bbQ$. Hence,
\[ HH_{\ast}^{K(0)_{\ast}}(K(0)_{\ast}E_{2}) \cong \bigg(K(0)_{\ast}E_{2}\bigg) \otimes_{X} \bigg(\bigwedge_{X}HH_{1}^{\bbQ}(X)\bigg) \otimes_{\bbQ} \bigg(\bigwedge_{\bbQ}du\bigg).\]
The generators are multiplicative and live in homological degree (columns) $0$ and $1$. And, they don't admit any non-zero differentials. Thus, Bökstedt spectral sequence collapses on $E^{2}$-page. Since this is free as a graded commutative $K(0)_{\ast}E_2$ algebra on the first column, there are no possible additive or multiplicative extensions and we get an
isomorphism of $K(0)_{\ast}E_{2}$-algebras
\[ K(0)_{\ast}THH(E_{2}) \cong \bigg(K(0)_{\ast}E_{2}\bigg) \otimes_{X} \bigg(\bigwedge_{X}HH_{1}^{\bbQ}(X)\bigg) \otimes_{\bbQ} \bigg(\bigwedge_{\bbQ}du\bigg). \] 


\begin{comment}
The last isomorphism holds due to the way $u, u_{1}, v_{1}, v_{2}, t_{1}, t_{2}$ are related to each other. These relations which include Hazewinkel formulae tell us that after taking differential $(du_{1}, du) = (dv_{1}, dv_{2}) = (dt_{1}, dt_{2})$.
\end{comment}

\begin{comment}
and this is going to be our $E^{\infty}$- page. It can't have multiplicative extensions because $E^{\infty}$ is direct sum of quotients of final thing so if $E^{\infty}$ is as free as possible, it cannot have extensions. %Also, note that Using Landweber exactness of $E_{2}$ and the map $BP \rightarrow E(2) \rightarrow E_2$, we have $K(0)_{\ast}E_{2} \cong K(0)_{\ast}E(2) \otimes_{E(2)_{\ast}} E_{2_{\ast}} \cong \pi_{\ast}E(2)_{\mathbb{Q}} \otimes_{E(2)_{\mathbb{Q}}_{\ast}} E_{2_{\mathbb{Q}}_{\ast}}$. Hence, we have 
%\def\mystrut{\hskip-2ex\mathstrut}
\begin{align*}
& K(0)_{\ast}THH(E_{2}) \cong HH_{\ast}^{\mathbb{Q}}(\pi_{\ast}(E_{2})_{\mathbb{Q}}) \\
& \cong \mathbb{Q}_{p} \otimes \bigwedge_{\mathbb{Q}} HH_{1}^{\mathbb{Q}}(\mathbb{Q}_{p}) \otimes \mathbb{Q}[u, u^{-1}] \otimes \bigwedge_{\mathbb{Q}}(du) \otimes \mathbb{Q}[\mu_{p^{2}-1}] \otimes (\mathbb{Q}[u_{1}] \otimes \bigwedge_{\mathbb{Q}} (du_{1})) \otimes_{\mathbb{Q}[u_{1}]} (\mathbb{Q}[[u_{1}]] \oplus \bigwedge_{\mathbb{Q}((u_{1}))}\mystrut^{\geq 1} HH_{1}^{\mathbb{Q}(u_{1})}\mathbb{Q}((u_{1})))\\
& \cong \mathbb{Q}_{p} \otimes \bigwedge_{\mathbb{Q}} HH_{1}^{\mathbb{Q}}(\mathbb{Q}_{p}) \otimes \mathbb{Q}[u, u^{-1}] \otimes \bigwedge_{\mathbb{Q}}(du) \otimes \mathbb{Q}[\mu_{p^{2}-1}] \otimes (\mathbb{Q}[[u_{1}]] \otimes \bigwedge_{\mathbb{Q}} (du_{1})) \otimes_{\mathbb{Q}[[u_{1}]]} (\mathbb{Q}[[u_{1}]] \oplus \bigwedge^{\geq 1}_{\mathbb{Q}((u_{1}))} HH_{1}^{\mathbb{Q}(u_{1})}\mathbb{Q}((u_{1})))\\
& \cong (K(0)_{\ast}E_{2} \otimes \bigwedge_{\mathbb{Q}} HH_{1}^{\mathbb{Q}} \mathbb{Q}_{p} \otimes \bigwedge_{\mathbb{Q}}(du) \otimes \bigwedge_{\mathbb{Q}}(du_{1})) \otimes_{\mathbb{Q}[[u_{1}]]} (\mathbb{Q}[[u_{1}]] \oplus \bigwedge^{\geq 1}_{\mathbb{Q}((u_{1}))} HH_{1}^{\mathbb{Q}(u_{1})}\mathbb{Q}((u_{1})))\\
& \cong (K(0)_{\ast}E_{2} \otimes \bigwedge_{\mathbb{Q}} HH_{1}^{\mathbb{Q}} \mathbb{Q}_{p} \otimes \bigwedge_{\mathbb{Q}}(du, du_{1})) \otimes_{\mathbb{Q}[[u_{1}]]} (\mathbb{Q}[[u_{1}]] \oplus \bigwedge^{\geq 1}_{\mathbb{Q}((u_{1}))} HH_{1}^{\mathbb{Q}(u_{1})}\mathbb{Q}((u_{1})))\\
& \cong (K(0)_{\ast}E_{2} \otimes \bigwedge_{\mathbb{Q}} HH_{1}^{\mathbb{Q}} \mathbb{Q}_{p} \otimes \bigwedge_{\mathbb{Q}}(dt_{1}, dt_{2})) \otimes_{\mathbb{Q}[[u_{1}]]} (\mathbb{Q}[[u_{1}]] \oplus \bigwedge^{\geq 1}_{\mathbb{Q}((u_{1}))} HH_{1}^{\mathbb{Q}(u_{1})}\mathbb{Q}((u_{1}))).
\end{align*}
\end{comment}

%The last isomorphism holds due to the following relations between $u_{i}, v_{i}, \text{and} \ t_{i}$: $v_{1} = u_{1}u^{p-1}, \ v_{2} = u^{}$

%(find the folmulae $v_{1}$ in terms of $t_{1}$, $v_{2}$ in terms of $t_{2}$- Hazewinkel generator stuff and $v_{i} = u_{i}u^{p^{i}-1}$. note $|v_{i}|= 2p^{i} - 2, |u_{i}| = 0, |u| = 2$) ***I think the formulas to use to figure out what is going on are:
   %A2.1.27 (spec., (c), (d)) and A2.2.1
%in Ravenel's Complex Cobordism and Stable Homotopy Groups of Spheres.

%HH_{*}^{\mathbb{Q}}(\mathbb{Q}[u,u^{-1}]) \otimes_{\mathbb{Q}} HH_{*}^{\mathbb{Q}}(\mathbb{Q}[\mu_{p^{2}-1}]) \otimes HH_{*}^{\mathbb{Q}}(\mathbb{Q}[[u_{1}]]) \cong \pi_{\ast}(E_{2}) \otimes_{\mathbb{Q}} \bigwedge_{\mathbb{Q}}(du) \otimes \bigwedge_{\mathbb{Z_{p}}}HH_{1}^{\mathbb{Z}}\mathbb{Z_{p}} \otimes \bigwedge HH_{1}^{\mathbb{Q}} \mathbb{Q}[[u_{1}]]

\subsection{$K(1)_{\ast}THH(E_{2})$}

There is a map from second Johnson-Wilson theory $E(2)$ to second Morava theory $E_{2}$ which at $\pi_{\ast}$-level does three things on the usual presentations: it adds a $(p^{2}-1)$th root of unity and a $(p^{2}-1)$th root of $v_{2}$, it changes generators, and lastly, it completes at the ideal $(u_{1})$. Rewriting $K(i)_{\ast}E_{2}$ as $K(i)_{\ast}E(2) \otimes_{E(2)_{\ast}} E_{2_{\ast}}$, we will show that one can break up $HH_{\ast}$ over these factors in a way that makes each individual term calculatable or known from previous work. In particular, we show that $HH_{\ast}^{K(i)_{\ast}}(K(i)_{\ast}E_{2})$ is isomorphic to $HH_{\ast}^{K(i)_{\ast}}(K(i)_{\ast}E(2)) \otimes_{HH_{\ast}^{\mathbb{F}_{p}}(E(2)_{\ast}/p)} HH_{\ast}^{\mathbb{F}_{p}}(E_{2_{\ast}}/p)$. Note that we already know $HH_{\ast}^{K(i)_{\ast}}(K(i)_{\ast}E(2))$ from the work of Ausoni-Richter summarized in theorem \ref{Thm 2.1.4.1}. This leaves us with the task of calculating $HH_{\ast}^{\mathbb{F}_{p}}(E_{2_{\ast}}/p)$.\\

Landweber exactness of $E_{2}$ tells us that  $K(i)_{\ast}E_2 \cong K(i)_{\ast}(BP) \otimes_{BP_{\ast}} (E_{2})_{\ast}$. Using Landweber theory for $E(2)$ and $E_2$ and the fact that the map $BP \rightarrow E_{2}$ factors through $E(2)$, one has
\[ K(i)_{\ast}E_{2} \cong K(i)_{\ast}E(2) \otimes_{E(2)_{\ast}} E_{2_{\ast}} \cong K(i)_{\ast}E(2) \otimes_{E(2)_{\ast}/p} E_{2_{\ast}}/p \]
for $i = 1, 2$. Hence, $HH_{\ast}^{K(i)_{\ast}}(K(i)_{\ast}E_{2}) \cong HH_{\ast}^{K(i)_{\ast}}(K(i)_{\ast}E(2) \otimes_{E(2)_{\ast}/p} E_{2_{\ast}}/p)$. Note that $E_{2_{\ast}}/p$ is flat over $E(2)_{\ast}/p$. We show that $HH_{\ast}^{\mathbb{F}_{p}}(E_{2_{\ast}}/p)$ is flat over $HH_{\ast}^{\mathbb{F}_{p}}(E(2)_{\ast}/p)$. This allows us to write $HH_{\ast}^{K(i)_{\ast}}(K(i)_{\ast}E(2)
\otimes_{E(2)_{\ast}/p} E_{2_{\ast}}/p)$ as $HH_{\ast}^{K(i)_{\ast}}(K(i)_{\ast}E(2)) 
\otimes_{HH_{\ast}^{\mathbb{F}_{p}}(E(2)_{\ast}/p)} HH_{\ast}^{\mathbb{F}_{p}}(E_{2_{\ast}}/p)$ using theorem \ref{Thm 2.3.2} below.

\begin{prop}\label{Prop 2.3.1}
$HH_{\ast}^{\mathbb{F}_{p}}(E_{2_{\ast}}/p)$ is flat over $HH_{\ast}^{\mathbb{F}_{p}}(E(2)_{\ast}/p)$. Moreover, $HH_{\ast}^{\mathbb{F}_{p}}(E_{2_{\ast}}/p)$ is isomorphic to
\[\bigg(HH_{\ast}^{\mathbb{F}_{p}}(E(2)_{\ast}/p) \otimes_{\mathbb{F}_{p}[v_{1}][v_{2},v_{2}^{-1}]} (\mathbb{F}_{p}[\mu_{p^{2}-1}][u_{1}][u, u^{-1}])\bigg) \otimes_{\F_{p}[u_{1}]} \bigg(\bigwedge_{\mathbb{F}_{p}[[u_{1}]]}HH_{1}^{\F_{p}[u_{1}]}\F_{p}[[u_{1}]]\bigg).\]
\end{prop}
\begin{proof}
We have the following sequence of maps
\[\mathbb{F}_{p}[v_{1}][v_{2}, v_{2}^{-1}] \xrightarrow{1} \mathbb{F}_{p}[\mu_{p^{2}-1}][v_{1}][v_{2}, v_{2}^{-1}] \xrightarrow{2} \mathbb{F}_{p}[\mu_{p^{2}-1}][u_{1}][u, u^{-1}] \xrightarrow{3} \mathbb{F}_{p}[\mu_{p^{2}-1}][[u_{1}]][u, u^{-1}]\]
where $\mathbb{F}_{p}[v_{1}][v_{2}, v_{2}^{-1}] \cong E(2)_{\ast}/p$ and $\mathbb{F}_{p}[\mu_{p^{2}-1}][[u_{1}]][u, u^{-1}] \cong E_{2_{\ast}}/p$. Looking at each of the maps in detail, we observe the following:\\

%\underline{\textbf{Map 1}}

Map 1 is an unramified flat extension and hence ètale. Thus, using \S\ref{2.1.2}, we have
\begin{align*}
& HH_{\ast}^{\F_{p}}(\mathbb{F}_{p}[\mu_{p^{2}-1}][v_{1}][v_{2}, v_{2}^{-1}])\\ 
& \quad \cong HH_{\ast}^{\F_{p}}(\mathbb{F}_{p}[v_{1}][v_{2}, v_{2}^{-1}]) \otimes_{\mathbb{F}_{p}[v_{1}][v_{2}, v_{2}^{-1}]} \mathbb{F}_{p}[\mu_{p^{2}-1}][v_{1}][v_{2}, v_{2}^{-1}]\\
& \quad \cong HH_{\ast}^{\F_{p}}(\mathbb{F}_{p}[v_{1}][v_{2}, v_{2}^{-1}]) \otimes \mathbb{F}_{p}[\mu_{p^{2}-1}].
\end{align*}
Thus, $HH_{\ast}^{\F_{p}}(\mathbb{F}_{p}[v_{1}][v_{2}, v_{2}^{-1}]) \rightarrow HH_{\ast}^{\F_{p}}(\mathbb{F}_{p}[\mu_{p^{2}-1}][v_{1}][v_{2}, v_{2}^{-1}])$ is flat.\\

%\underline{\textbf{Map 2}}

Map $2$ sends $v_{1} \mapsto u_{1}u^{p^{2}-1}$ and $v_{2} \mapsto u^{p^{2}-1}$, where $u$ is a unit. So we are attaching a ($p^{2}-1$)-root of a unit and replacing a variable with a unit times that variable, hence this is also an étale extension. We get
\begin{align*}
& HH_{\ast}^{\F_{p}}(\mathbb{F}_{p}[\mu_{p^{2}-1}][u_{1}][u,u^{-1}]) \\
& \quad \cong HH_{\ast}^{\F_{p}}(\mathbb{F}_{p}[\mu_{p^{2}-1}][v_{1}][v_{2},v_{2}^{-1}]) \otimes_{\mathbb{F}_{p}[\mu_{p^{2}-1}][v_{1}][v_{2},v_{2}^{-1}]} \mathbb{F}_{p}[\mu_{p^{2}-1}][u_{1}][u,u^{-1}]\\
& \quad \cong HH_{\ast}^{\F_{p}}(\mathbb{F}_{p}[\mu_{p^{2}-1}][v_{1}][v_{2},v_{2}^{-1}]) \otimes_{\mathbb{F}_{p}[\mu_{p^{2}-1}][v_{1}][v_{2},v_{2}^{-1}]} \mathbb{F}_{p}[\mu_{p^{2}-1}][v_{1}][u, u^{-1}].
\end{align*}
Note that the way $v_{i}$s map onto the term on the right of tensor product in this last isomorphism makes this a free module extension of $HH_{\ast}^{\F_{p}}(\mathbb{F}_{p}[\mu_{p^{2}-1}][v_{1}][v_{2},v_{2}^{-1}])$, and hence flat.\\

%\underline{\textbf{Map 3}}

Map 3 is the completion at $(u_{1})$ and hence flat. Taking $HH_{\ast}$ on both sides of this map, we have \[HH_{\ast}^{\F_{p}}(\mathbb{F}_{p}[u_{1}]) \otimes HH_{\ast}^{\F_{p}}(\mathbb{F}_{p}[\mu_{p^{2}-1}][u, u^{-1}]) \rightarrow HH_{\ast}^{\F_{p}}(\mathbb{F}_{p}[[u_{1}]]) \otimes HH_{\ast}^{\F_{p}}(\mathbb{F}_{p}[\mu_{p^{2}-1}][u, u^{-1}]).\]
Using proposition \ref{Prop 2.1.3.5}, $$HH_{\ast}^{\mathbb{F}_{p}}\mathbb{F}_{p}[[u_{1}]] \cong (HH_{\ast}^{\mathbb{F}_{p}}\mathbb{F}_{p}[u_{1}]) \otimes_{\mathbb{F}_{p}[u_{1}]} (\F_{p}[[u_{1}]] \oplus \bigwedge^{\geq 1}_{\mathbb{F}_{p}[[u_{1}]]}W_{\F_{p}}).$$ Since $W_{\F_{p}}$ is a vector space over $\F_{p}((u_{1}))$, and $\F_{p}((u_{1}))$ a localization of $\F_{p}[[u_{1}]]$, $HH_{\ast}^{\mathbb{F}_{p}}\mathbb{F}_{p}[u_{1}] \rightarrow (HH_{\ast}^{\mathbb{F}_{p}}\mathbb{F}_{p}[u_{1}]) \otimes_{\mathbb{F}_{p}[u_{1}]} (\F_{p}[[u_{1}]] \oplus \bigwedge^{\geq 1}_{\mathbb{F}_{p}[[u_{1}]]}W_{\F_{p}})$ is a map into an infinite vector space of a localization and thus flat. We get
\[ HH_{\ast}^{\mathbb{F}_{p}} \mathbb{F}_{p}[\mu_{p^{2}-1}][[u_{1}]][u, u^{-1}]\cong HH_{\ast}^{\mathbb{F}_{p}} \mathbb{F}_{p}[\mu_{p^{2}-1}][u_{1}][u, u^{-1}] \otimes_{{F}_{p}[u_{1}]} \bigwedge_{\mathbb{F}_{p}[[u_{1}]]}HH_{1}^{\F_{p}[u_{1}]}\F_{p}[[u_{1}]]. \]
\begin{comment}
This makes 
\begin{align*}
HH_{\ast}(\mathbb{F}_{p}[u_{1}]) & \rightarrow HH_{\ast}^{\mathbb{F}_{p}}\mathbb{F}_{p}[u_{1}] \otimes_{\mathbb{F}_{p}[u_{1}]} (\mathbb{F}_{p}[[u_{1}]] \oplus \bigwedge^{\geq 1}_{\mathbb{F}_{p}((u_{1})) }W)\\
& \cong HH_{\ast}^{\mathbb{F}_{p}}\mathbb{F}_{p}[u_{1}] \otimes_{\mathbb{F}_{p}[u_{1}]} (\mathbb{F}_{p}[[u_{1}]] \oplus \bigwedge^{\geq 1}_{\mathbb{F}_{p}((u_{1})) }(\oplus \mathbb{F}_{p}((u_{1})) ) \\
& \cong HH_{\ast}^{\mathbb{F}_{p}}\mathbb{F}_{p}[u_{1}] \otimes_{\mathbb{F}_{p}[u_{1}]} (\mathbb{F}_{p}[[u_{1}]] \oplus (\oplus \mathbb{F}_{p}((u_{1}))) \oplus \cdots ) 
%\cong HH_{\ast}(\mathbb{F}_{p}[[u_{1}]])
\end{align*}
flat (localization, and free infinite extensions). %Also, from Loday {Proposition 1.1.17} we have $HH_{n}^{\mathbb{F}_{p}(u_{1}) }\mathbb{F}_{p}((u_{1})) \cong HH_{n}^{\mathbb{F}_{p}[u_{1}] }\mathbb{F}_{p}[[u_{1}]] \ \forall n$ and $HH_{\ast}^{\mathbb{F}_{p}(u_{1}) }\mathbb{F}_{p}((u_{1})) \cong HH_{\ast}^{\mathbb{F}_{p}[u_{1}] }\mathbb{F}_{p}[[u_{1}]]$, thus $W \cong HH_{1}^{\mathbb{F}_{p}[u_{1}]}\mathbb{F}_{p}[[u_{1}]]$.
\end{comment}

From our analysis of maps 1, 2 and 3 above, we conclude $HH_{\ast}^{\mathbb{F}_{p}}(E_{2_{\ast}}/p)$ is flat over $HH_{\ast}^{\mathbb{F}_{p}}(E(2)_{\ast}/p)$. In fact, subsequent back substitution in the results from Map 3 to Map 2 to Map 1, show that $HH_{\ast}^{\mathbb{F}_{p}}(E_{2_{\ast}}/p)$ is isomorphic to
\[\bigg(HH_{\ast}^{\mathbb{F}_{p}}(E(2)_{\ast}/p) \otimes_{\mathbb{F}_{p}[v_{1}][v_{2},v_{2}^{-1}]} \mathbb{F}_{p}[\mu_{p^{2}-1}][u_{1}][u, u^{-1}]\bigg) \otimes_{{F}_{p}[u_{1}]} \bigg(\bigwedge_{\mathbb{F}_{p}[[u_{1}]]}HH_{1}^{\F_{p}[u_{1}]}\F_{p}[[u_{1}]]\bigg).\]
%Note that $HH_{1}^{\mathbb{F}_{p}(u_{1}) }\mathbb{F}_{p}((u_{1})) \cong HH_{1}^{\mathbb{F}_{p}[u_{1}] }\mathbb{F}_{p}[[u_{1}]]$. 

\begin{comment}
{\color{red} The reason why we can rewrite Propn. 2.5 from LL in this way is because $HH_{\ast}^{\mathbb{F}_{p}} \mathbb{F}_{p}[u_{1}] \cong \bigwedge(du_{1})$, but that just means its $\mathbb{F}_{p}[u_{1}]$ in degree $0$ and $\mathbb{F}_{p}[u_{1}] \cong \mathbb{F}_{p}[u_{1}] \langle du_{1} \rangle$ in degree $1$ and $0$ in all higher degrees because $du_{1} \bigwedge du_{1} = 0$.} \\
\end{comment}

\end{proof}

\begin{comment}
 It gives us 
\begin{align*}
HH_{\ast}(\mathbb{F}_{p}[\mu_{p^{2}-1}][u_{1}][u,u^{-1}]) & \cong HH_{\ast}(\mathbb{F}_{p}[\mu_{p^{2}-1}][v_{1}][v_{2},v_{2}^{-1}]) \otimes_{\mathbb{F}_{p}[\mu_{p^{2}-1}][v_{1}][v_{2},v_{2}^{-1}]} \mathbb{F}_{p}[\mu_{p^{2}-1}][u_{1}][u,u^{-1}] \\
& \cong \mathbb{F}_{p}[\mu_{p^{2}-1}][v_{1}][v_{2},v_{2}^{-1}] \otimes \bigwedge_{\mathbb{F}_{p}}<\sigma v_{1}, \sigma v_{2}> \otimes_{\mathbb{F}_{p}[\mu_{p^{2}-1}][v_{1}][v_{2},v_{2}^{-1}]} \mathbb{F}_{p}[\mu_{p^{2}-1}][u_{1}][u,u^{-1}]\\
& \cong \mathbb{F}_{p}[\mu_{p^{2}-1}][u_{1}][u,u^{-1}] \otimes \bigwedge_{\mathbb{F}_{p}}<\sigma u_{1}, \sigma u>
\end{align*}
We use the fact that $HH_{\ast}(\mathbb{F}_{p}[\mu_{p^{2}-1}][v_{1}][v_{2},v_{2}^{-1}]) \cong \mathbb{F}_{p}[\mu_{p^{2}-1}][v_{1}][v_{2},v_{2}^{-1}] \otimes \bigwedge_{\mathbb{F}_{p}}<\sigma v_{1}, \sigma v_{2}>$ {reff} and that $v_{1} \mapsto u^{p-1}u_{1}$, $v_{2} \mapsto u^{p^{2}-1} \implies \sigma v_{1} \mapsto (p-1)u^{p-2}u_{1} \sigma u + u^{p-1} \sigma u_{1}, \sigma v_{2} \mapsto (p^{2}-1)u^{p^{2}-2} \sigma u$.

For our purposes in this paper, it's enough to note that with the help of equations above we can rewrite
\begin{align*}
& HH_{\ast}(\mathbb{F}_{p}[\mu_{p^{2}-1}][u_{1}][u,u^{-1}]) \cong HH_{\ast}(\mathbb{F}_{p}[\mu_{p^{2}-1}][v_{1}][v_{2},v_{2}^{-1}]) \otimes_{\mathbb{F}_{p}[\mu_{p^{2}-1}][v_{1}][v_{2},v_{2}^{-1}]} \mathbb{F}_{p}[\mu_{p^{2}-1}][u_{1}][u,u^{-1}]\\
& \cong HH_{\ast}(\mathbb{F}_{p}[\mu_{p^{2}-1}][v_{1}][v_{2},v_{2}^{-1}]) \otimes_{\mathbb{F}_{p}[\mu_{p^{2}-1}][v_{1}][v_{2},v_{2}^{-1}]} \mathbb{F}_{p}[\mu_{p^{2}-1}][v_{1}][v_{2}^{1/(p^{2}-1)}, v_{2}^{-1/(p^{2}-1)}]
\end{align*}
\end{comment}

\begin{theorem}\label{Thm 2.3.2}
Let $k \rightarrow R \rightarrow S \rightarrow A$ and $k \rightarrow S \rightarrow B$ be maps of commutative rings. Regard $A \otimes_S B$ as an $R$ algebra via the map $R \cong R \otimes_S S\to A\otimes_S B$. If $B$ is flat over $S$ and $HH_{\ast}^{k}(B)$ is flat over $HH_{\ast}^{k}(S)$, then $HH_{\ast}^{R}(A \otimes_{S} B) \cong HH_{\ast}^{R}(A) \otimes_{HH_{\ast}^{k}(S)} HH_{\ast}^{k}(B)$.
\end{theorem}

\begin{proof}
Note that in simplicial abelian groups 
\[(N^{cy}_{\otimes_{R}})_{\bullet}(A \otimes_{S}B) \cong (N^{cy}_{\otimes_{R}})_{\bullet}(A) \otimes_{(N^{cy}_{\otimes_{k}})_{\bullet}(S)} (N^{cy}_{\otimes_{k}})_{\bullet}(B).\]Since $B$ is flat over $S$, the right hand side represents the derived tensor product in simplicial modules $(N^{cy}_{\otimes_{R}})_{\bullet} (A) \otimes^{\bbL}_{(N^{cy}_{\otimes_{k}})_{\bullet}(S)} (N^{cy}_{\otimes_{k}})_{\bullet} (B)$, and hence also the derived tensor product in dg-modules $N^{cy}_{\otimes_{R}} (A) \otimes^{\bbL}_{N^{cy}_{\otimes_{k}} (S)} N^{cy}_{\otimes_{k}} (B)$.

Taking homotopy groups on both sides, we get
\begin{equation}\label{eqn 2.3.2.0}
HH_{\ast}^{R}(A\otimes_{S}B) \cong H_{\ast}(N^{cy}_{\otimes_{R}}(A) \otimes^{\bbL}_{N^{cy}_{\otimes_{k}}(S)} N^{cy}_{\otimes_{k}}(B)).
\end{equation}
Using lemma \ref{Lemma 2.3.3} below with $M, N$ and $C$ being $N^{cy}_{\otimes_{R}}(A)$, $N^{cy}_{\otimes_{k}}(B)$ and $N^{cy}_{\otimes_{k}}(S)$ respectively, we have a spectral sequence
\[ Tor_{HH_{\ast}^{k}S} (HH_{\ast}^{R}A, HH_{\ast}^{k}B) \Longrightarrow H_{\ast}(N^{cy}_{\otimes_{R}}(A) \otimes^{\bbL}_{N^{cy}_{\otimes_{k}}(S)} N^{cy}_{\otimes_{k}}(B)). \]
Since $HH_{\ast}^{k}(B)$ is flat over $HH_{\ast}^{k}(S)$, the $Tor$-term in the spectral sequence collapses. Hence, we conclude
\[HH_{\ast}^{R}A \otimes_{HH_{\ast}^{k}S} HH_{\ast}^{k}B \cong H_{\ast}(N^{cy}_{\otimes_{R}}(A) \otimes_{N^{cy}_{\otimes_{k}}(S)} N^{cy}_{\otimes_{k}}(B)).\]

Thus, equation (\ref{eqn 2.3.2.0}) becomes
\[ HH_{\ast}^{R}(A \otimes_{S} B) \cong HH_{\ast}^{R}(A) \otimes_{HH_{\ast}^{k}(S)} HH_{\ast}^{k}(B)\]
giving us the required result.
\end{proof}

\begin{lemma}\label{Lemma 2.3.3}
Let $C$ be a simplicial or dg-ring, $M$ a right $C$-module, and $N$ a left $C$-module. Then there exists an algebraic Eilenberg-Moore spectral sequence
\[Tor_{H_{\ast}C} (H_{\ast}M, H_{\ast}N) \Longrightarrow H_{\ast}(M \otimes_{C}^{\mathbb{L}}N).\]
\end{lemma}
\begin{proof}
Let $Y$ be a cofibrant approximation of $N$ so that $M \otimes_{C} Y$ represents the derived tensor product $M \otimes^{\bbL}_{C} N$. Choose a free $H_{\ast}C$ resolution of $H_{\ast}Y$
\[ \cdots \rightarrow F_{n} \xrightarrow{\overline{d_{n}}} \cdots \rightarrow F_{1} \xrightarrow{\overline{d_{1}}} F_{0} \xrightarrow{\overline{d_{0}}} H_{\ast}Y. \]
Now we can realize $\overline{d_{0}}$ as $H_{\ast}$ of map of $C$-modules $\bigoplus_{\text{gens of} \ F_{0}} \sum^{n_{\alpha_{0}}}C \xrightarrow{d_{0}} Y$, where $F_{0}$ is graded, $n_{\alpha}$ is the internal degree, and the direct sum is over a chosen set of generators of $F_0$. Let $Y_{-1} = Y$ and $Y_{0} = C_{d_{0}}$, the homotopy cofiber of the map $d_{0}$. Then, because $\overline{d_{0}}$ is surjective on homology, we get the isomorphism $H_{\ast}(Y_{0}) = \sum ker(\overline{d_{0}})$, the suspension of the kernel module. Next we realize $\overline{d_{1}}$, similarly, as a map of $C$-modules $\bigoplus_{\text{gens of} \ F_{1}} \sum^{n_{\alpha_{1}}}C \xrightarrow{d_{1}} Y_{0}$, which will be surjective at the level of $H_{\ast}$ because $\overline{d_{1}} \circ \overline{d_{0}} = 0$. Let $Y_{1} = C_{d_{1}}$. Then $H_{\ast}(Y_{1}) \subset ker(\overline{d_{1}})$, again with a shift of one degree. Continuing this process gives us a sequence
\[ Y_{-1} \to Y_{0} \to Y_{1} \to \cdots \]
Let $Y_{\infty} = hocolim \ Y_{i}$, then $Y_{\infty} \simeq 0$. Looking at $X \otimes_{N} Y_{\infty} \simeq 0$, the filtration gives spectral sequence converging to $H_{\ast}(X \otimes Y_{-1})$ where the associated graded is $X \otimes (\bigoplus \sum^{n_{\alpha}}N)$, $E_{1}$-term is $H_{\ast}X \otimes_{H_{\ast} N} F_{i}$ and the $E_{2}$-term is the required $Tor$ term.
\end{proof}

Using the results above, we finally conclude
\[ HH_{\ast}^{K(1)_{\ast}}(K(1)_{\ast}E_{2}) \cong  HH_{\ast}^{K(1)_{\ast}}(K(1)_{\ast}E(2)) 
\otimes_{HH_{\ast}^{\mathbb{F}_{p}}(E(2)_{\ast}/p)} HH_{\ast}^{\mathbb{F}_{p}}(E_{2_{\ast}}/p) .\]

We now show

\begin{theorem}\label{Thm 2.3.4}
We have the following isomorphism of $K(1)_{\ast}E_{2}$-algebras
\[K(1)_{\ast}THH(E_{2}) \cong \bigg(K(1)_{\ast}E_2\bigg) \otimes_{{\F}_{p}[[u_{1}]]} \bigg(\bigwedge_{\mathbb{F}_{p}[[u_{1}]]}HH_{1}^{\F_{p}[u_{1}]}\F_{p}[[u_{1}]]\bigg) \otimes_{\F_{p}} \bigg(\bigwedge_{\F_{p}} dt_{1}\bigg).\]
\end{theorem}

\begin{proof}
 Rewriting proposition \ref{Prop 2.3.1}, and plugging the result from theorem \ref{Thm 2.1.4.1} we get
\begin{align*}
& HH_{\ast}^{K(1)_{\ast}}(K(1)_{\ast}E_{2})\\
& \cong ((K(1)_{\ast}E(2) \otimes \bigwedge dt_{1}) \otimes_{\mathbb{F}_{p}[v_{1}][v_{2},v_{2}^{-1}]} \mathbb{F}_{p}[\mu_{p^{2}-1}][u_{1}][u, u^{-1}]) \otimes_{{F}_{p}[u_{1}]} 
\bigwedge_{\mathbb{F}_{p}[[u_{1}]]}HH_{1}^{\F_{p}[u_{1}]}\F_{p}[[u_{1}]].
\end{align*}
Since $K(1)_{\ast}E_{2} \cong K(1)_{\ast}E(2) \otimes_{\mathbb{F}_{p}[v_{1}][v_{2},v_{2}^{-1}]} \mathbb{F}_{p}[\mu_{p^{2}-1}][[u_{1}]][u, u^{-1}]$, we rewrite above as
\begin{align*}
& \cong ((K(1)_{\ast}E(2) \otimes \bigwedge dt_{1}) \otimes_{\mathbb{F}_{p}[v_{1}][v_{2},v_{2}^{-1}]} \mathbb{F}_{p}[\mu_{p^{2}-1}][[u_{1}]][u, u^{-1}]) \otimes_{{F}_{p}[[u_{1}]]} 
\bigwedge_{\mathbb{F}_{p}[[u_{1}]]}HH_{1}^{\F_{p}[u_{1}]}\F_{p}[[u_{1}]]\\
& \cong K(1)_{\ast}E_2 \otimes_{{F}_{p}[[u_{1}]]} \bigwedge_{\mathbb{F}_{p}[[u_{1}]]}HH_{1}^{\F_{p}[u_{1}]}\F_{p}[[u_{1}]] \otimes \bigwedge dt_{1}.
\end{align*}
%&\cong K(1)_{\ast}E_{2} \otimes \bigwedge dt_{1} \otimes_{{F}_{p}[u_{1}]}\bigwedge^{\ast}_{\mathbb{F}_{p}[[u_{1}]] }W

The generators are multiplicative and live in homological degree (columns) $0$ and $1$. Therefore they do not admit any non-zero differentials. Thus, Bökstedt spectral sequence collapses on $E^{2}$-page. Since this is free as a graded commutative $K(1)_{\ast}E_2$ algebra on the first column, there are no possible additive or multiplicative extensions and we get an isomorphism of $K(1)_{\ast}E_2$-algebras
\begin{align*}
K(1)_{\ast}THH(E_{2}) \cong \bigg(K(1)_{\ast}E_2\bigg) \otimes_{{\F}_{p}[[u_{1}]]} \bigg(\bigwedge_{\mathbb{F}_{p}[[u_{1}]]}HH_{1}^{\F_{p}[u_{1}]}\F_{p}[[u_{1}]]\bigg) \otimes_{\F_{p}} \bigg(\bigwedge_{\F_{p}} dt_{1}\bigg).
\end{align*}
\end{proof}

\subsection{$K(2)_{\ast}THH(E_{2})$}

The methods of \S2.3 work here as well. We use those to conclude the following

\begin{theorem}\label{Thm 2.4.1}
The unit map $E_{2} \rightarrow THH(E_{2})$ is a $K(2)$-equivalence, i.e., it induces an isomorphism $K(2)_{\ast}E_{2} \cong K(2)_{\ast}THH(E_{2})$.
\end{theorem}

\begin{proof}
As in the last section, we have
\[ HH_{\ast}^{K(2)_{\ast}}(K(2)_{\ast}E_{2}) \cong  HH_{\ast}^{K(2)_{\ast}}(K(2)_{\ast}E(2)) 
\otimes_{HH_{\ast}^{\mathbb{F}_{p}}(E(2)_{\ast}/p)} HH_{\ast}^{\mathbb{F}_{p}}(E_{2_{\ast}}/p). \]
From theorem \ref{Thm 2.1.4.1} and proposition \ref{Prop 2.3.1}, $HH_{\ast}^{K(2)_{\ast}}(K(2)_{\ast}E_{2})$ is isomorphic to
\[\bigg(K(2)_{\ast}E(2) \otimes_{\mathbb{F}_{p}[v_{1}][v_{2},v_{2}^{-1}]} \mathbb{F}_{p}[\mu_{p^{2}-1}][u_{1}][u, u^{-1}]\bigg) \otimes_{{\F}_{p}[u_{1}]} 
\bigwedge_{\mathbb{F}_{p}[[u_{1}]]}HH_{1}^{\F_{p}[u_{1}]}\F_{p}[[u_{1}]] \]
where note that the action of $v_1, v_2$ in the tensor product is via the elements of $E_2$ (notation \ref{Classes}). Using the formulas in \cite[B.5, Pg 167-171]{MR1192553}, we see that $\eta_R(v_1) = 0$ in $K(2)_*E_2$. Thus, action of $\F_{p}[u_{1}]$ on $K(2)_{\ast}E(2) \otimes_{\mathbb{F}_{p}[v_{1}][v_{2},v_{2}^{-1}]} \mathbb{F}_{p}[\mu_{p^{2}-1}][u_{1}][u, u^{-1}]$ is by zero, which tells us that $\bigwedge_{\mathbb{F}_{p}[[u_{1}]]}HH_{1}^{\F_{p}[u_{1}]}\F_{p}[[u_{1}]]$ term vanishes from $HH_{\ast}^{K(2)_{\ast}}(K(2)_{\ast}E_{2})$ since $HH_{1}^{\F_{p}[u_{1}]}\F_{p}[[u_{1}]]$ is a $[u_1,u_{1}^{-1}]$-module. Observe, this leaves just the zeroth column in the Bökstedt spectral sequence which collapses to give
\[K(2)_{\ast}THH(E_{2})\cong HH_{\ast}^{K(2)_{\ast}}(K(2)_{\ast}E_{2}) \cong K(2)_{\ast}E_{2}.\]
Since this map is induced by the unit map $E_{2} \rightarrow THH(E_{2})$, we conclude that the unit map is a $K(2)$-equivalence.
\end{proof}


\section{Lifting classes in $K(i)_{\ast}THH(E_{2})$ to classes in $\pi_{\ast}THH(E_{2})$}

So far we know the $K(i)_{\ast}$-homology of $THH(E_{2})$, while our aim in this paper is to understand $\pi_{\ast}THH(E_{2})$. We have Hurewicz maps from $\pi_{\ast}THH(E_{2})$ to $K(i)_{\ast}THH(E_{2})$. The goal of this section is to attempt to lift as many $K(i)_{\ast}THH(E_{2})$ classes as possible to $\pi_{\ast}THH(E_{2})$ classes along the Hurewicz map. Thus, the question for this section is: `Can one construct classes in homotopy groups of $THH(E_{2})$ which map down to the classes we know in $K(i)_{\ast}THH(E_{2})$?'.\\

Remember from section 2 we have
\begin{align*}
K(0)_{\ast}THH(E_{2}) & \cong \bigg(K(0)_{\ast}E_{2}\bigg) \otimes_{X} \bigg(\bigwedge_{X}HH_{1}^{\bbQ}(X)\bigg) \otimes_{\bbQ} \bigg(\bigwedge_{\bbQ}du\bigg) \\
K(1)_{\ast}THH(E_{2}) & \cong \bigg(K(1)_{\ast}E_{2}\bigg) \otimes_{{\F}_{p}[[u_{1}]]} \bigg(\bigwedge_{\mathbb{F}_{p}[[u_{1}]]}HH_{1}^{\F_{p}[u_{1}]}\F_{p}[[u_{1}]]\bigg) \otimes_{\F_{p}} \bigg(\bigwedge_{\F_{p}} dt_{1}\bigg)\\
K(2)_{\ast} THH (E_{2}) & \cong K(2)_{\ast}E_{2}.
\end{align*}

The classes we are interested in lifting are as follows: the class represented by the unit element in $K(i)_{\ast}E_{2}$ in homological degree $0$; classes arising from elements in $HH_{1}^{\mathbb{F}_{p}[u_{1}]}\mathbb{F}_{p}[[u_{1}]]$ %and $HH_{1}^{\Q[u_{1}]}\mathbb{Q}[[u_{1}]]$ 
in $K(1)_{\ast}THH(E_{2})$ %and $K(0)_{\ast}THH(E_{2})$ respectively 
at homological level 1 and class arising from $dt_{1}$ in $K(1)_{\ast}THH(E_{2})$ also at homological level 1. %and from $(dt_{1}, dt_{2})$ in $K(0)_{\ast}THH(E_{2})$; and finally classes arising from $HH_{1}^{\Q}\Q_{p}$ in $K(0)_{\ast}THH(E_{2})$. Not all of the classes listed above lift to $\pi_{\ast}THH(E_{2})$.
Additionally we want to figure out where the chosen lifts map along the Hurewicz map to $K(0)_{\ast}THH(E_{2})$. We lift $K(i)_{\ast}E_{2}$, $dt_{1}$, and $HH_{1}^{\mathbb{F}_{p}[u_{1}]}\mathbb{F}_{p}[[u_{1}]]$ classes in \S3.1, \S3.2 and \S3.3 respectively.\\

Before we proceed with the lifts, let's set up the machinery we will use for lifting. Keep in mind the construction of classes as in notation \ref{Classes} and the Bokstedt spectral sequence as in \S\ref{2.1.1}. The Hurewicz maps are symmetric monoidal, giving a commutative square

\begin{equation}\label{ss2}\tag{$\ast \ast$}
% Reference with \eqref{ss2}
\begin{tikzcd}
\pi_{\ast}(E_{2}) \otimes \pi_{\ast}(E_{2}) \arrow{r}{\phi} \arrow[swap]{d}{\chi} & \pi_{\ast}(E_{2} \wedge E_{2}) \arrow{d}{\Psi} \\
K(i)_{\ast}E_{2} \otimes_{K(i)_{\ast}} K(i)_{\ast}E_{2} \arrow{r}{\cong} & K(i)_{\ast}(E_{2} \wedge E_{2}).
\end{tikzcd}
\end{equation}

where the isomorphism is due to Künneth theorem as discusued in \S\ref{2.1.1}. Note that $K(i)_{\ast}(E_{2}\wedge E_{2}) \cong K(i)_{\ast}E_{2} \otimes_{K(i)_{\ast}} K(i)_{\ast}E_{2}$ is the second column of $E^{1}$-page of Bökstedt spectral sequence with $E^{2}_{s,t} \cong (HH_{s}^{K(i)_{\ast}}(K(i)_{\ast}E_{2}))_{t}$ converging to $K(i)_{\ast}THH(E_{2})$. Separately, we know that for the Bökstedt spectral sequence of $K(1)_{\ast}THH (E_{2})$,
\[E^{2} \cong E^{\infty} \cong \bigg(K(1)_{\ast}E_{2}\bigg) \otimes_{{F}_{p}[[u_{1}]]} \bigg(\mathbb{F}_{p}[[u_{1}]] \oplus \bigwedge^{\geq 1}_{\mathbb{F}_{p}((u_{1}))} HH_{1}^{\mathbb{F}_{p}(u_{1})}\mathbb{F}_{p}((u_{1}))\bigg) \otimes_{\F_{p}} \bigg(\bigwedge_{\mathbb{F}_{p}} dt_{1}\bigg)\]
where $K(1)_{\ast}E_{2}$ lies in homological degree $0$, $dt_{1}$ lies in homological degree $1$ and internal degree $2p^{2}-2$, $\F_{p}[[u_{1}]]$ lies in homological degree $0$ and $HH_{1}^{\mathbb{F}_{p}(u_{1})}\mathbb{F}_{p}((u_{1}))$ lies in homological degree $1$ and internal degree $0$. $\pi_{\ast}(E_{2}\wedge E_{2})$, on the other hand, is the second column of $E^{1}$-page of filtration spectral sequence with $E^{1}_{s,t} \cong \pi_{t}(E_{2}^{\wedge s+1})$ converging to $\pi_{\ast}THH(E_{2})$). The map $\Psi$ can be extended as a map from $\pi_{t}(E_{2}^{\wedge s+1})$ to $K(i)_{t}(E_{2}^{\wedge s+1})$ ($E^{1}$-page of Bökstedt spectral sequence) where it respects the differentials and converges to the Hurewicz map from $\pi_{\ast}THH(E_{2})$ to $K(i)_{\ast}THH(E_{2})$ due to its naturality properties.


\subsection{Lifting unit element in $K(i)_{\ast}E_{2}$}

$K(i)_{\ast}E_{2}$ sits in homological degree $0$ in $K(i)_{\ast}THH(E_{2})$. We know that it is induced from the unit map $E_{2} \rightarrow THH(E_{2})$ which is coming from $\bbS \rightarrow THH(E_{2})$ (which represents the unit element in $\pi_{\ast}THH(E_{2})$) after taking smash product with $E_{2}$ and using the $E_{2}$-module structure on $THH(E_{2})$. Hence, we already have a lift for the unit element in $K(i)_{\ast}E_{2}$.

\subsection{Lifting $dt_{1}$ in $K(1)_{\ast}THH(E_{2})$}

In this section, we lift the class $dt_{1}$ to a class in $\pi_{\ast}THH(E_{2})$ that we denote as $\lambda_{1}$ following \cite{MR1209233}. Note from notation \ref{Classes} there is $t_{1} \in \pi_{2p-2}(E_2 \wedge E_2)$ which maps down to $1 \otimes t \in K(n)_{\ast}E_2 \otimes_{K(n)_{\ast}} K(n)_{\ast}E_2$ and further to $dt_{1} \in K(1)_{2p-1}THH(E_{2})$. So, we have

\[\begin{tikzcd}[row sep=large, column sep=large]
t_{1} \in \pi_{2p-2}(E_2 \wedge E_2) \ar[r, dotted] \ar[d] & THH_{2p-1}(E_{2})\ar[d]\\
%& & \tikz\node[draw,circle,inner sep=2pt]{$\ell$}; \ar[d,dotted] & \\ 
1 \otimes t_{1} \in K(1)_{\ast}E_2 \otimes_{K(1)_{\ast}} K(1)_{\ast}E_2 \ar[r, dotted] & dt_{1} \in K(1)_{2p-1}THH(E_{2})
\end{tikzcd}\]

where the horizontal maps are dotted to convey the movement of classes via the spectral sequence pages. Since the square commutes, we have the lift $\lambda_{1}$ in $THH_{2p-1}(E_{2})$. Now, $\lambda_{1}$ also maps to $dt_{1}$ in $K(0)_{2p-1}THH(E_{2})$. $K(0)_{\ast}THH(E_{2})$ has classes $du$, $du_{1}$ where $u_{1} = u^{1-p}v_{1} = u^{1-p}pt_{1}$ from the relations given by Hazewinkel formula \cite[B.5, Pg 167-171]{MR1192553}. And thus $du_{1} = d(u^{1-p}pt_{1}) = (1-p)pt_{1}u^{-p}du + pu^{1-p}dt_{1}$. Thus the class $dt_{1}$ is generated by a linear combination of class $du_{1}$ and $du$ in $K(0)_{\ast}THH(E_{2})$. We have proved the following

\begin{prop}\label{Prop 4.2.1}
There exists class $\lambda_{1} \in THH_{2p-1}(E_{2})$ such that under Hurewicz homomorphism
$\lambda_{1} \mapsto dt_{1} \in K(1)_{2p-1}THH(E_{2})$. Further, in $K(0)_{2p-1}THH(E_{2})$, it maps down to a $K(0)_{\ast}E_{2}$-linear combination of classes $du$ and $du_{1}$.
\end{prop}


\begin{comment}
\begin{proof}
We have map $BP \rightarrow E(2) \rightarrow E_{2}$ which at the homotopy level sends $v_{1} \mapsto v_{1} \mapsto u_{1} \cdot u^{p-1}$ and $v_{2} \mapsto v_{2} \mapsto u^{p^{2}-1}$. This induces a natural map $THH_{\ast}(BP) \rightarrow THH_{\ast}(E(2)) \rightarrow THH_{\ast}(E_{2})$. From \cite{MR1209233} and \cite[Lemma 5.1]{MR4071375}, we know that the rational and $K(1)$ homology of $BP$ are as follows
\[ \ K(0)_{\ast}BP \cong \bbQ[t_{i} \ \vert \ i \geq 1], \ K(1)_{\ast}BP \cong K(1)_{\ast}[t_{i} \ \vert \ i \geq 1]\]
and the associated Bökstedt spectral sequences collapse to give
\begin{align*}
& HH_{\ast}^{K(0)_{\ast}}K(0)_{\ast}BP \cong K(0)_{\ast}THH(BP) \cong K(0)_{\ast}BP \otimes \bigwedge_{\bbQ}(dt_{i} \ \vert \ i \geq 1) \\
& HH_{\ast}^{K(1)_{\ast}}K(1)_{\ast}BP \cong K(1)_{\ast}THH(BP) \cong K(1)_{\ast}BP \otimes \bigwedge_{\F_{p}}(dt_{i} \ \vert \ i \geq 1).
\end{align*}

We also know that
\[THH_{\ast}(BP) \cong BP_{\ast} \otimes \bigwedge_{\bbZ_{(p)}} (\lambda_{i} \ \vert \ i \geq 1)\]
and under Hurewicz homomorphism $\lambda_{i} \in THH_{\ast}(BP)$ maps to $dt_{i}$ in $K(0)_{\ast}THH(BP)$ and $K(1)_{\ast}THH(BP)$.\\

Now note that we have maps $HH_{\ast}^{K(1)_{\ast}}K(1)_{\ast}BP \rightarrow HH_{\ast}^{K(1)_{\ast}}K(1)_{\ast}E(2) \rightarrow HH_{\ast}^{K(1)_{\ast}}K(1)_{\ast}E_{2}$ which map class $dt_{1}$ to $dt_{1}$ throughout. Define $\lambda_{1} \in THH_{2p-1}(E_{2})$ as the image of the class with same name under the natural map $THH_{\ast}(BP) \rightarrow THH_{\ast}(E_{2})$. We have a commutative diagram using the fact that Bökstedt spectral sequence $HH_{\ast}^{K(i)_{\ast}}K(i)_{\ast}X \Rightarrow K(i)_{\ast}THH(X)$ collapes for both $BP$ and $E_{2}$

\[\begin{tikzcd}[row sep=large, column sep=large]
THH_{\ast}(BP) \ar[r] \ar[d] & THH_{\ast}(E_{2})\ar[d]\\
%& & \tikz\node[draw,circle,inner sep=2pt]{$\ell$}; \ar[d,dotted] & \\ 
K(i)_{\ast}THH(BP) \ar[r] & K(i)_{\ast}THH(E_{2}).
\end{tikzcd}\]

For $i = 1$, this tells us that the left vertical map sends $\lambda_{1} \rightarrow dt_{1}$, the top horizontal map sends $\lambda_{1}$ to $\lambda_{1}$, the bottom horizontal map sends $dt_{1}$ to $dt_{1}$. Hence, $\lambda_{1} \in THH(E_{2})$ goes to $dt_{1} \in K(1)_{\ast}THH(E_{2})$. For $i = 0$, the class still goes to $dt_{1}$ in $K(0)_{\ast}THH(E_{2})$. 
\end{proof}

To understand $dt_{1}$ in terms of the homological classes we have in our expression of $K(0)_{\ast}THH(E_{2})$, note that we have

\[\begin{tikzcd}[row sep=large, column sep=large]
\pi_{\ast}(E_{2}) \otimes \pi_{\ast}(E_{2}) \arrow{d} \arrow{r} & K(0)_{\ast}(E_{2}) \otimes_{K(0)_{\ast}} K(0)_{\ast}(E_{2}) \arrow{d}{\cong} & \\
\pi_{\ast}(E_{2} \wedge E_{2}) \arrow{d} \arrow{r} & K(0)_{\ast}(E_{2} \wedge E_{2}) \arrow{d} & \\
\pi_{\ast}THH(E_{2}) \arrow{r} & K(0)_{\ast}(THH(E_{2}))
\end{tikzcd}\]
\end{comment}


\subsection{Lifting classes corresponding to elements of $HH_{1}^{\F_{p}[u_{1}]}\F_{p}[[u_{1}]]$}\hfill \break
$\bigwedge_{\F_{p}[[u_{1}]]}HH_{1}^{\F_{p}[u_{1}]}\F_{p}[[u_{1}]] \rightarrow \bigwedge_{\F_{p}((u_{1}))}HH_{1}^{\F_{p}(u_{1})}\F_{p}((u_{1}))$
is an isomorphism in all positive degrees. $HH_{1}^{\F_{p}(u_{1})}\F_{p}((u_{1}))$ is an infinite dimensional vector space over $\F_{p}((u_{1}))$. By \cite[Theorem 86(ii)]{MR0266911} its dimension is the $p$-degree of $\F_{p}((u_{1}))$ over $\F_{p}(u_{1})$ which is the cardinality of the continuum. Now if $\{s\}$ forms a basis set of $\F_{p}((u_{1}))$ over $\F_{p}(u_{1})$, then $\{1 \otimes s\}$ forms a basis set for $\F_{p}((u_{1})) \otimes_{\F_{p}(u_{1})} \F_{p}((u_{1}))$ over $\F_{p}((u_{1}))$ where latter acts on former the way it would act on each level of the Hochschild complex, in our case on the left tensor factor. Without loss of generality we can choose such a set $\{s\}$ where each $s$ is an element of $\F_{p}[[u_{1}]]$ and hence has (non-unique) lifts to $\bbZ[[u_{1}]]$ and $\bbZ_{p}[[u_{1}]]$. Since $HH_{1}^{\F_{p}(u_{1})}\F_{p}((u_{1}))$ is a quotient of $\F_{p}((u_{1}))\otimes_{\F_{p}(u_{1})} \F_{p}((u_{1}))$, there is a basis $\frB \subset \{1 \otimes s\}$ of $HH_{1}^{\F_{p}(u_{1})}\F_{p}((u_{1}))$ over $\F_{p}((u_{1}))$. Looking at the commutative diagram (\ref{ss2}) there are classes $1 \otimes s' \in \pi_{0}E_{2} \otimes \pi_{0}E_{2}$ coming from the lifts of elements $1 \otimes s \in \frB$ to $\bbZ_{p}[[u_{1}]]$ (since $\pi_{0}E_{2} \otimes \pi_{0}E_{2} \cong \mathbb{Z}_{p}[\mu_{p^{2}-1}][[u_{1}]] \otimes \mathbb{Z}_{p}[\mu_{p^{2}-1}][[u_{1}]]$). These classes map to $1 \otimes s \in K(1)_{0}E_{2} \otimes K(1)_{0}E_{2}$, and thus to $[1 \otimes s] \in HH_{(1,0)}^{K(1)_{\ast}}K(1)_{\ast}E_{2}$ in the Bökstedt spectral sequence.


\begin{comment}
We are going to use the following commutative square
\begin{equation}\label{ss2}\tag{$\ast$}
% Reference with \eqref{ss2}
\begin{tikzcd}
\pi_{\ast}E_{2} \otimes \pi_{\ast}(E_{2}) \arrow{r}{\phi} \arrow[swap]{d}{\chi} & \pi_{\ast}(E_{2} \wedge E_{2}) \arrow{d}{\Psi} \\
K(1)_{\ast}E_{2} \otimes K(1)_{\ast}E_{2} \arrow{r}{\Phi} & K(1)_{\ast}(E_{2} \wedge E_{2}).
\end{tikzcd}
\end{equation}

Let's inspect the key individual terms of this commutative square. $\pi_{0}E_{2} \otimes \pi_{0}E_{2}$ looks like $\mathbb{Z}_{p}[\mu_{p^{2}-1}][[u_{1}]] \otimes \mathbb{Z}_{p}[\mu_{p^{2}-1}][[u_{1}]]$ sitting inside $\pi_{\ast}E_{2} \otimes \pi_{\ast}E_{2}$ as the zero degree term. $\mathbb{Z}[[u_{1}]] \hookrightarrow  \mathbb{Z}_{p}[\mu_{p^{2}-1}][[u_{1}]]$ and $\mathbb{Z}[[u_{1}]] \twoheadrightarrow \mathbb{F}_{p}[[u_{1}]]$, so the elements $s \in \mathbb{F}_{p}[[u_{1}]]$  corresponding to $1 \otimes s \in \mathfrak{B}$ lift to elements $s'$ in $\Z[[u_{1}]]$ and thus in $\mathbb{Z}_{p}[\mu_{p^{2}-1}][[u_{1}]]$ (non-uniquely). Hence, $\pi_{0}E_{2} \otimes \pi_{0}E_{2}$ has all the elements of the form $1 \otimes s'$, for all $1 \otimes s \in \mathfrak{B}$. %Hence, it has all the terms of the form $1 \otimes s \in HH_{1}^{\mathbb{F}_{p}[u_{1}]}\mathbb{F}_{p}[[u_{1}]], s \in S$ since $HH_{1}^{\mathbb{F}_{p}[u_{1}]}\mathbb{F}_{p}[[u_{1}]]$ is a quotient of $\mathbb{F}_{p}[[u_{1}]] \otimes \mathbb{F}_{p}[[u_{1}]]$.
$K(1) _{\ast}(E_{2}\wedge E_{2})$ is the second column of $E^{1}$-page of filtration spectral sequence with $E^{1}_{s,t} = K(1)_{t}(E_{2}^{\wedge s+1})$. Since Morava $K$-theory satisfies Künneth theorem, $K(1)_{\ast}(E_{2}^{\wedge t}) \cong (K(1)_{\ast}(E_{2}))^{\otimes_{K(1)_{\ast}} t}$. Thus $\Phi$ is an isomorphism and the rows of $E^{1}$ page turn out to be Hochschild complexes. This makes the $E^{2}$-page of filtration spectral sequence same as $E^{2}$-page of Bökstedt spectral sequence, $E^{2}_{s,t} \cong (HH_{s}^{K(1)_{\ast}}(K(1)_{\ast}E_{2}))_{t}$ and $K(1)_{0}(E_{2}\wedge E_{2})$ sitting at the $(1,0)$-spot on the $E^{1}$-page becomes $HH_{1}(K(1)_{0}E_{2}) \cong$ a quotient of $K(1)_{0}(E_{2}\wedge E_{2}) \cong (HH_{1}^{K(1)_{\ast}}(K(1)_{\ast}E_{2}))_{0}$ on the $E^{2}$ page. But we separately know that for Bökstedt spectral sequence of $K(1)_{\ast}THH E_{2}$,
\[E^{2} \cong E^{\infty} \cong \bigg(K(1)_{\ast}E_{2}\bigg) \otimes_{{F}_{p}[[u_{1}]]} \bigg(\mathbb{F}_{p}[[u_{1}]] \oplus \bigwedge^{\geq 1}_{\mathbb{F}_{p}((u_{1}))} HH_{1}^{\mathbb{F}_{p}(u_{1})}\mathbb{F}_{p}((u_{1}))\bigg) \otimes_{\F_{p}} \bigg(\bigwedge_{\mathbb{F}_{p}} dt_{1}\bigg)\]
where $K(1)_{\ast}E_{2}$ lies in homological degree $0$, $dt_{1}$ lies in homological degree $1$ and internal degree $2p^{2}-2$, $\F_{p}[[u_{1}]]$ lies in homological degree $0$ and $HH_{1}^{\mathbb{F}_{p}(u_{1})}\mathbb{F}_{p}((u_{1}))$ lies in homological degree $1$ and internal degree $0$. Hence, $E^{2}_{1,0} \cong (HH_{1}^{K(1)_{\ast}}(K(1)_{\ast}E_{2}))_{0} \cong K(1)_{0}E_{2} \otimes_{\mathbb{F}_{p}[[u_{1}]]} HH_{1}^{\mathbb{F}_{p}(u_{1})}\mathbb{F}_{p}((u_{1}))$ with homological classes $1 \otimes s$. \\
We will show that $1 \otimes s'$ at the top left maps to preimage of $1 \otimes s$ at the bottom right of the commutative square. We will then use the map $\Psi$ in it's most generalized form as map from $\pi_{t}(E_{2}^{\wedge s+1})$ ($E^{1}$-page of filtration spectral sequence, $\pi_{t}(E_{2}^{\wedge s+1}) \Rightarrow \pi_{\ast}THH(E_{2})$) to $K(1)_{t}(E_{2}^{\wedge s+1})$ (``$E^{1}$-page'' of Bökstedt spectral sequence, $E^{2}_{\ast , \ast} = HH^{K(i)_{\ast}}_{\ast , \ast}(K(i)_{\ast}(E_{2})) \implies K(i)_{\ast}(THH(E_{2})))$ to conclude that there are lifts of classes $1 \otimes s \in \frB$ from $K(1)_{\ast}THH(E_{2})$ to $\pi_{\ast}THH(E_{2})$. The exact statement is as follows
\end{comment}


\begin{prop}\label{Prop 4.3.1}
For each $1 \otimes s \in \frB$, there exists a class $\tilde{s} \in THH_{1}E_{2}$, such that under Hurewicz homomorphism each $\tilde{s}$ maps to the corresponding class $1 \otimes s \in K(1)_{1}THH(E_{2})$. 
\end{prop}

\begin{proof}
\begin{comment}
In the commutative square \eqref{ss2}, the zero level of maps looks as follows
\[
\begin{tikzcd}
\pi_{0}E_{2} \otimes \pi_{0}E_{2} \arrow{r}{\phi_{0}} \arrow[swap]{d}{\chi_{0}} & \pi_{0}(E_{2} \wedge E_{2}) \arrow{d}{\Psi_{0}} \\
K(1)_{0}E_{2} \otimes K(1)_{0}E_{2} \arrow{r}{\cong} & K(1)_{0}(E_{2} \wedge E_{2})
\end{tikzcd}
\]
By the definition of spectral sequence 
\begin{align*}
E^{2}_{1,0} & \cong \frac{K(1)_{0}E_{2} \otimes K(1)_{0}E_{2}} {Im (K(1)_{0}E_{2} \otimes K(1)_{0}E_{2} \otimes K(1)_{0}E_{2} \mapsto K(1)_{0}E_{2} \otimes K(1)_{0}E_{2})}\\
& \cong \text{a quotient of} \ E^{1}_{1,0} \ \text{or} \ K(1)_{0}E_{2} \otimes K(1)_{0}E_{2}
\end{align*}
So $E^{2}_{1,0}$ has two descriptions, one as a quotient of $K(1)_{0}E_{2} \otimes K(1)_{0}E_{2}$ as above, and the other from the $K(1)_{\ast}THH(E_{2})$ calculations done in section 2 as $K(1)_{0}E_{2} \otimes_{\mathbb{F}_{p}[[u_{1}]]} HH_{1}^{\mathbb{F}_{p}(u_{1})}\mathbb{F}_{p}((u_{1}))$. The first description provides classes $\chi_{0}(1 \otimes s')$, and the second provides classes $1 \otimes s$, both in $E^{2}_{1,0}$. Note that the homological class $1 \otimes s \in HH_{\ast}^{K(1)_{\ast}}K(1)_{\ast}E_{2}$ comes from the element $1 \otimes s \in K(1)_{\ast}E_{2} \otimes_{K(1)_{\ast}} K(1)_{\ast}E_{2}$ in the Hochschild complex and thus from $K(1)_{0}E_{2} \otimes K(1)_{0}E_{2}$ for degree reasons. Also, just by the way we define $1 \otimes s'$, $\chi_{0}(1 \otimes s')$ is $1 \otimes s$.\\
\end{comment}


We have

\[\begin{tikzcd}[row sep=large, column sep=large]
1 \otimes s' \in \pi_{0}(E_2 \wedge E_2) \ar[r, dotted] \ar[d] & THH_{1}(E_{2})\ar[d]\\
%& & \tikz\node[draw,circle,inner sep=2pt]{$\ell$}; \ar[d,dotted] & \\ 
1 \otimes s \in K(1)_{0}E_2 \otimes_{K(1)_{0}} K(1)_{0}E_2 \ar[r, dotted] & 1 \otimes s \in K(1)_{1}THH(E_{2})
\end{tikzcd}\]

where the dotted arrows again signify movement through the spectral sequence pages. Since the class $[1 \otimes s]$ in $K(1)_{1}THH(E_{2})$ is non-zero, $1 \otimes s'$ survives the filtration spectral sequence representing a non-zero class in $THH_{1}(E_{2})$ which we call $\tilde{s}$ and which under Hurewicz map, maps down to $1 \otimes s \in K(1)_{1} THH(E_{2})$.
\end{proof}


\begin{comment}
We want to show that these are, in fact, the same elements.\\

We have a more explicit understanding of $K(1)_{0}E_{2}$ as follows:
\begin{align*}
& K(1)_{\ast}E_{2} \cong K(1)_{\ast}E(2) \otimes_{E(2)_{\ast}/p} E_{2_{\ast}}/p\\
\implies & K(1)_{0}E_{2} \cong K(1)_{0}E(2) \otimes_{E(2)_{0}/p} E_{2_{0}}/p \\
\implies & K(1)_{0}E_{2} \cong K(1)_{0}E(2) \otimes_{\mathbb{F}_{p}} \mathbb{F}_{p} [\mu_{p^{2}-1}] [[u_{1}]]
\end{align*}
{\color{red}Ref Ausoni-Richter} gives $K(1)_{0}E(2) \cong \mathbb{F}_{p}[t_{1}, t_{2}, \cdots][\eta_{R}(v_{2})^{-1}]/ (\eta_{R}(v_{j})|j \geq 3)$, thus $$K(1)_{0}E_{2} \cong \mathbb{F}_{p}[t_{1}, t_{2}, \cdots] [\eta_{R}(v_{2})^{-1}] [\mu_{p^{2}-1}] [[u_{1}]]/ (\eta_{R}(v_{j})|j \geq 3).$$If we look back at the two expressions for $E^{2}_{1,0}$ now, we see that 
\begin{align*}
& K(1)_{0}E_{2} \otimes_{\mathbb{F}_{p}[[u_{1}]]} HH_{1}^{\mathbb{F}_{p}(u_{1})}\mathbb{F}_{p}((u_{1}))\\
& \cong \frac{\mathbb{F}_{p}[t_{1}, t_{2}, \cdots] [\eta_{R}(v_{2})^{-1}] [\mu_{p^{2}-1}] [[u_{1}]]}{(\eta_{R}(v_{j})|j \geq 3)} \otimes_{\mathbb{F}_{p}[[u_{1}]]} HH_{1}^{\mathbb{F}_{p}(u_{1})}\mathbb{F}_{p}((u_{1}))\\
& \cong \frac{\mathbb{F}_{p}[t_{1}, t_{2}, \cdots] [\eta_{R}(v_{2})^{-1}] [\mu_{p^{2}-1}] [[u_{1}]]}{(\eta_{R}(v_{j})|j \geq 3)} \otimes_{\mathbb{F}_{p}[[u_{1}]]} <\text{span of} \ {1 \otimes s}>\\
\\
% & \text{a quotient of} \ E^{1}_{1,0} \ \text{or} \ K(1)_{0}E_{2} \otimes K(1)_{0}E_{2}
\end{align*}
where $s \in \F_{p}[[u_{1}]]$; and
\begin{align*}
& \text{a quotient of} \ E^{1}_{1,0} \ \text{or} \ K(1)_{0}E_{2} \otimes K(1)_{0}E_{2} \\
& \cong \text{a quotient of} \ \frac{\mathbb{F}_{p}[t_{1}, t_{2}, \cdots] [\eta_{R}(v_{2})^{-1}] [\mu_{p^{2}-1}] [[u_{1}]]}{(\eta_{R}(v_{j})|j \geq 3)} \otimes \frac{\mathbb{F}_{p}[t_{1}, t_{2}, \cdots] [\eta_{R}(v_{2})^{-1}] [\mu_{p^{2}-1}] [[u_{1}]]}{(\eta_{R}(v_{j})|j \geq 3)}
\end{align*}
And so the element $1 \otimes s$ in the first description is the same as the element $\chi_{0}(1 \otimes s')$ in the second description.\\
\end{comment}


\begin{comment}
\subsection{Lifting classes arising from elements of $HH_{1}^{\Q[u_{1}]}\Q[[u_{1}]]$}
\hspace{1cm}

Note again, that the expression $\bigwedge_{\Q[[u_{1}]]}HH_{1}^{\Q[u_{1}]}\Q[[u_{1}]]$ can be rewritten as $\bigwedge_{\Q((u_{1}))}HH_{1}^{\Q(u_{1})}\Q((u_{1}))$. 
The set $\{\tilde{s}\}$ in $\Z[[u_{1}]]$ of representatives of $\{s\}$ corresponding to basis elements $\B$ forms a linearly independent set in $\Q((t))$ as a vector space over $\Q(t)$.
%given any finite expression, lets say (a_{1}/b{1})s_{1} + (a_{2}/b_{2})s_{2}=0 where a_{i}/b_{i} \in \Q{t}, we can repeatedly take lcms etc. to get an expression of the form \alpha_{1}s_{1} + \alpha_{2}s_{2} = 0 where \alpha_{i}s are in \Z[t]. Then go mod p and get the conclusion (remember we can reduce the whole expression \alpha_{1}s_{1} + \alpha_{2}s_{2} = 0 such that the coefficients are coprime)
And hence can be extended to $\tilde{\B}$, a basis set for $\Q((t))$ over $\Q(t)$, and we can choose it in a way such that the elements are in $\Q[[t]]$.
%This step is done in the following way - if we have a basis of \Q((t)) over \Q(t), {a/b, c/d, e/f, ...}, then we convert the elements to being in \Q[[u_{1}]] step by step. first {a, bc/d, be/f, ...} is basis, then {a, bc, dbe/f, ....} and so on.
But now what do i do??
\end{comment}


\begin{comment}
\subsection{Lifting $HH_{1}^{\Q}\Q_{p}$ classes from $K(0)_{\ast}THH(E_{2})$ to $\pi_{\ast}THH(E_{2})$}
Given a basis of $\Q_{p}$ over $Q$ which we can WLOG assume to be in $\Z_{p}$, we get a basis, $\U$, of $HH_{1}^{\Q}\Q_{p}$ over $\Q_{p}$. 

*****Add claim about basis extension? about p-torsion freeness??

*****$\frB'$ basis for $HH_{1}^{\bbQ(u)}\bbQ((u))$ over $\bbQ((u))$ made up of $r$s
\end{comment} 


\section{$THH(E_{2})$ and $THH(E_{2})_{p}^{\wedge}$}

In this section we finally prove the main theorem \ref{MainThm}. Note that $THH(E_{2})$ sits in a cofiber sequence $E_{2} \rightarrow THH(E_{2}) \rightarrow \overline{THH}(E_{2})$ of $E_{\infty}$-ring spectra. The first map $E_{2} \rightarrow THH(E_{2})$ is the unit map and from theorem \ref{Thm 2.4.1} we know it to be a $K(2)_{\ast}$-equivalence. This unit map is our $f_{1}$ and the cofiber map $C_{f_{1}}$ for \ref{MainThm}. There is a map in the other direction $THH(E_2) \rightarrow E_2$ such that the composite map is identity and thus we have a splitting as $E_{2}$-modules, $E_{2} \vee \overline{THH}(E_{2}) \simeq THH(E_{2})$. 


In \S4.1, we briefly remark on some properties of cofibers that we will use throughout this section to conclude results regarding $X_{i}$, $f_{i}$ and $C_{f_{i}}$. In \S4.2, using our information about $E_{2}$ and $f_{1}$, we analyze the $K(i)$-homology classes of $C_{f_{1}}:=\overline{THH}(E_{2})$. This lets us construct $X_{2}$ and $f_{2}$ in \S4.3 with $X_{2}$ being $\Sigma^{2p-1}L_{1}E_{2}$. $K(i)$-homology classes of $C_{f_{2}}$ then allow us to construct $X_{3}$ and $f_{3}$ in \S4.4. \S4.4 further deals with proving $K(1)$-equivalence of map $f_{3}[u_{1}^{-1}]: X_{3}[u_{1}^{-1}] \rightarrow C_{f_{2}}[u_{1}^{-1}]$ and showing rationality of cofibers of maps $C_{f_{2}} \rightarrow C_{f_{2}}[u_{1}^{-1}]$ and $X_{3}[u_{1}^{-1}] \rightarrow C_{f_{2}}[u_{1}^{-1}]$. Finally, \S4.5 gives the full description of $THH(E_{2})^{\wedge}_{p}$.

\subsection{Remarks on certain cofibers}

It would be helpful to keep in mind the following remarks when we are dealing with cofibrations of spectra of the form $A \rightarrow B \rightarrow C$ where $A$ and $B$ are $L_{n}$-local (note that $E_{n}$ and $THH(E_{n})$ are $L_{n}$-local).

\begin{rem}\label{Rmk 4.1.1}
The cofiber, $C$, is $L_{n}$-local as well.
\end{rem}

\begin{rem}\label{Rmk 4.1.2}
If $A \rightarrow B$ is a $K(n)$-equivalence, then $C$ is $L_{n-1}$-local.
\end{rem}

This tells us, for example, that $\overline{THH}(E_{2})$ is $L_{1}$-local.

\subsection{$E_{2}$, $f_{1}$ and $C_{f_{1}}$}

Since $C_{f_{1}}:= \overline{THH}(E_{2})$ is $L_{1}$-local, to understand $\overline{THH}(E_{2})$ we want to list all the $K(1)$ and $K(0)$-homology classes of $THH(E_{2})$. Here are the $K(0)_{\ast}THH(E_{2})$ and $K(1)_{\ast}THH(E_{2})$ computations from section 2 again for ease of reference
\[
K(0)_{\ast}THH(E_{2}) \cong \bigg(K(0)_{\ast}E_{2}\bigg) \otimes_{X} \bigg(\bigwedge_{X}HH_{1}^{\bbQ}(X)\bigg) \otimes_{\bbQ} \bigg(\bigwedge_{\bbQ}du\bigg)
\]
\[
K(1)_{\ast}THH(E_{2}) \cong \bigg(K(1)_{\ast}E_2\bigg) \otimes_{{\F}_{p}[[u_{1}]]} \bigg(\bigwedge_{\mathbb{F}_{p}[[u_{1}]]}HH_{1}^{\F_{p}[u_{1}]}\F_{p}[[u_{1}]]\bigg) \otimes_{\F_{p}} \bigg(\bigwedge_{\F_{p}} dt_{1}\bigg).
\]
Note that we can rewrite $K(1)_{\ast}E_{2} \otimes_{\F_{p}[[u_{1}]]} HH_{1}^{\F_{p}(u_{1})}\F_{p}((u_{1}))$ using
\[K(1)_{\ast}E_{2} \otimes_{\F_{p}[[u_{1}]]} HH_{1}^{\F_{p}(u_{1})}\F_{p}((u_{1})) \cong K(1)_{\ast}E_{2}[u_{1}^{-1}] \otimes_{\F_{p}((u_{1}))} HH_{1}^{\F_{p}(u_{1})}\F_{p}((u_{1})).\]
This makes $\frB$, as defined in \S3.3, a basis for $K(1)_{\ast}E_{2}[u_{1}^{-1}] \otimes_{{F}_{p}((u_{1}))} HH_{1}^{\F_{p}(u_{1})}\F_{p}((u_{1}))$ as $K(1)_{\ast}E_{2}[u_{1}^{-1}]$-module. Also keep in mind that the isomorphism \[K(i)_{\ast}THH(E_{2}) \cong HH_{\ast, \ast}^{K(i)_{\ast}}K(i)_{\ast}E_{2}\] gives $K(i)_{\ast}THH(E_{2})$ a homological grading which in terms of the part in degree $1$ (for this grading) is homogeneous. Thus we have

\begin{rem}\label{Rmk 5.1.1}
$K(1)_{\ast}THH(E_{2})$ is the internal direct sum of the following disjoint submodules
\begin{itemize}
\item $K(1)_{\ast}E_{2}$ in homological degree $0$,
\item $K(1)_{\ast}E_{2}(dt_{1})$ and $K(1)_{\ast}E_{2}[u_{1}^{-1}](1 \otimes s)$ in homological degree 1 for all $(1 \otimes s) \in \frB$,
%where we get the second class by rewriting $K(1)_{\ast}E_{2} \otimes_{\F_{p}[[u_{1}]]} (\F_{p}[[u_{1}]] \oplus \bigwedge_{\F_{p}((u_{1}))}^{\geq 1}HH_{1}^{\F_{p}(u_{1})}\F_{p}((u_{1})))$ as $K(1)_{\ast}E_{2}[u_{1}^{-1}] \otimes_{\F_{p}((u_{1}))} (\F_{p}[[u_{1}]] \oplus \bigwedge_{\F_{p}((u_{1}))}^{\geq 1}HH_{1}^{\F_{p}(u_{1})}\F_{p}((u_{1})))$ and then 
%{\color{red}looking at the basis set $\B$ which also acts as a basis set for (****{\color{red}DOES IT?? OR DO WE FURTHER REDUCE TO A BASIS FROM THIS GENERATING SET?}) $K(1)_{\ast}E_{2}[u_{1}^{-1}] \otimes_{\F_{p}((u_{1}))} HH_{1}^{\F_{p}(u_{1})}\F_{p}((u_{1})))$ as $K(1)_{\ast}E_{2}[u_{1}^{-1}]$-module.}
\item $K(1)_{\ast}E_{2}[u_{1}^{-1}](1 \otimes s_{1}) (1 \otimes s_{2}) \cdots (1 \otimes s_{n})$ for any $n$ distinct choice of basis elements in $\frB$ and $K(1)_{\ast}E_{2}[u_{1}^{-1}](1 \otimes s_{1}) (1 \otimes s_{2}) \cdots (1 \otimes s_{n-1}) dt_{1}$ for any $n-1$ distinct choice of basis elements in $\frB$ in homological degree $n$, $n \geq 2$.
\end{itemize}
\end{rem}

\begin{comment}
{\color{red}Similarly, we can rewrite $K(0)_{\ast}E_{2} \otimes_{\bbQ[[u_{1}]]} HH_{1}^{\bbQ(u_{1})}\bbQ((u_{1}))$ as $K(0)_{\ast}E_{2}[u_{1}^{-1}] \otimes_{\bbQ((u_{1}))} HH_{1}^{\bbQ(u_{1})}\bbQ((u_{1}))$. This makes $\frB'$ a basis for $K(0)_{\ast}E_{2}[u_{1}^{-1}] \otimes_{\bbQ((u_{1}))} HH_{1}^{\bbQ(u_{1})}\bbQ((u_{1}))$ as $K(0)_{\ast}E_{2}[u_{1}^{-1}]$-module. Also let $1 \otimes \beta$ denote the basis elements of $HH_{1}^{\bbQ}\bbQ_{p}$ as a vector space over $\Q_{p}$ where $\{\beta\}$ is a subset of the basis of $\bbQ_{p}$ over $\bbQ$. }
\end{comment}

Similarly,

\begin{rem}\label{Rmk 5.1.2}
$K(0)_{\ast}THH(E_{2})$ is the internal direct sum of the following (not disjoint) submodules (there are overlaps in homological degree $n$, $n \geq 2$, since the module is not free and hence we cannot specify a basis as in the case of previous remark)
\begin{itemize}
\item $K(0)_{\ast}E_{2}$ in homological degree $0$,
\item $K(0)_{\ast}E_{2}(du)$, $K(0)_{\ast}E_{2} \otimes_{X}HH_{1}^{\bbQ}X$ classes in homological degree $1$,
\item product of $n$ degree $1$ classes in homological degree $n$.
\end{itemize}
\end{rem}

The unit map $f_{1}$ from $E_{2} \rightarrow THH(E_{2})$ kills the $0$-level $K(1)$ and $K(0)$-homology classes of $THH(E_{2})$ mapping to $\overline{THH}(E_{2})$, and we are left with all other homology classes remaining in $K(1)_{\ast}\overline{THH}(E_{2})$ and $K(0)_{\ast}\overline{THH}(E_{2})$, respectively. 

\subsection{$X_{2}$, $f_{2}$ and $C_{f_{2}}$}

The above analysis of homology classes of $\overline{THH}(E_{2})$ will be a key tool in the construction of $X_{2}$ and $f_{2}$ which we do here. We proceed to then analyze $C_{f_{2}}$.\\

Note from proposition \ref{Prop 4.2.1}, there is a lift of class $dt_{1}$ in $K(1)_{\ast}THH(E_{2})$ to $\pi_{2p-1}THH(E_{2})$ denoted by $\lambda_{1}$. Hence $\lambda_{1}$ is a map from $\bbS^{2p-1}$ to $THH(E_{2})$. Smashing with $E_{2}$, using the $E_{2}$-module structure of $THH(E_{2})$, and composing with the cofiber map $THH(E_{2}) \rightarrow \overline{THH}(E_{2})$, we obtain map
\[ j_{1} : \Sigma^{2p-1}E_{2} \simeq E_{2} \wedge \bbS^{2p-1} \rightarrow E_{2} \wedge THH(E_{2}) \rightarrow THH(E_{2}) \rightarrow \overline{THH}(E_{2}).\]
The $L_{1}$-local property of $\overline{THH}(E_{2})$ tells us that $j_{1}$ factors through $\overline{j_{1}}: \Sigma^{2p-1}L_{1}E_{2} \rightarrow \overline{THH}(E_{2})$. Now, by construction of $j_{1}$, the induced map at $K(1)_{\ast}$ level clearly sends the generator of $K(1)_{\ast}(\Sigma^{2p-1}E_{2})$ to the class $dt_{1}$ in $K(1)_{\ast}\overline{THH}(E_{2})$ and hence the same is true for the map $\overline{j_{1}}$. Let
\[X_{2}:= \Sigma^{2p-1}L_{1}E_{2}, \ f_{2}:= \overline{j_{1}}: \Sigma^{2p-1}L_{1}E_{2} \rightarrow \overline{THH}(E_{2}).\]
Thus $C_{f_{2}}$ is still $L_{1}$-local (from \S4.1) and moreover $K(1)_{\ast}C_{f_{2}}$ has all the remaining classes listed in remark \ref{Rmk 5.1.1}. Note that $K(1)_{\ast}C_{f_{2}}$ is now a $K(1)_{\ast}E_{2}[u_{1}^{-1}]$-module. 

\subsection{$X_{3}$ and $f_{3}$}

There is a natural map $C_{f_{2}} \rightarrow C_{f_{2}}[u_{1}^{-1}]$. $K(1)_{\ast}(C_{f_{2}}[u^{-1}_{1}]) \cong (K(1)_{\ast}C_{f_{2}})[u_{1}^{-1}] \cong K(1)_{\ast}C_{f_{2}}$, where the last isomorphism holds due to the $K(1)_{\ast}E_{2}[u_{1}^{-1}]$-module structure of $K(1)_{\ast}C_{f_{2}}$. Thus, $C_{f_{2}}[u_{1}^{-1}]$ is $K(1)_{\ast}$-equivalent to $C_{f_{2}}$ and we know its $K(1)$-homology classes. Both $C_{f_{2}}$ and $C_{f_{2}}[u_{1}^{-1}]$ are also $L_{1}$-local and hence, the cofiber of map $C_{f_{2}} \rightarrow C_{f_{2}}[u_{1}^{-1}]$ is rational by remarks in \S4.1. Let $C_{\infty}$ denote this cofiber.\\

Consider $\lambda_{1}$ and $\tilde{s}$ from proposition \ref{Prop 4.2.1} and proposition \ref{Prop 4.3.1}. Let $q_{\alpha}$ denote elements in $\pi_{|\alpha|}THH(E_{2})$ of the form $\Pi_{\alpha}\tilde{s}, \ \forall \ \tilde{s} \in \alpha$, all distinct, where $\alpha$ ranges over all non-empty finite subsets of $\tilde{\frB}$ and $|\alpha|$ denotes the cardinality of these subsets. Here $\tilde{\frB}$ denotes the chosen lifts $\tilde{s}$ of elements $s$ in $\frB$. Similarly, let $q_{\alpha, \lambda_{1}}$ denote elements in $\pi_{|\alpha|+2p-1}THH(E_{2})$ of the form $\Pi_{\alpha}\tilde{s}\cdot\lambda_{1}$ for all $\alpha$ non-empty subsets of $\tilde{\frB}$. Note that all $q_{\alpha}$ and $q_{\alpha, \lambda_{1}}$ are thus maps
\[ q_{\alpha}: \bbS^{|\alpha|} \rightarrow THH(E_{2}), \  q_{\alpha, \lambda_{1}}: \bbS^{|\alpha|+2p-1} \rightarrow THH(E_{2}).\]
Composing with cofiber maps $C_{f_{1}}$, $C_{f_{2}}$ and with the natural map $C_{f_{2}} \rightarrow C_{f_{2}}[u_{1}^{-1}]$, these become maps from suspensions of sphere spectrum to $C_{f_{2}}[u_{1}^{-1}]$ which is an $E_{2}[u_{1}^{-1}]$-module. Thus, smashing with $E_{2}[u_{1}^{-1}]$ and using the $E_{2}[u_{1}^{-1}]$-module structure of $C_{f_{2}}[u_{1}^{-1}]$, we obtain maps $\overline{q_{\alpha}}$ and $\overline{q_{\alpha, \lambda_{1}}}$ below
\[\overline{q_{\alpha}} : \Sigma^{|\alpha|}E_{2}[u_{1}^{-1}] \simeq E_{2}[u_{1}^{-1}] \wedge \bbS^{|\alpha|} \rightarrow E_{2}[u_{1}^{-1}] \wedge C_{f_{2}}[u_{1}^{-1}] \rightarrow C_{f_{2}}[u_{1}^{-1}]\]
\[\overline{q_{\alpha, \lambda_{1}}} : \Sigma^{|\alpha|+2p-1}E_{2}[u_{1}^{-1}] \simeq E_{2}[u_{1}^{-1}] \wedge \bbS^{|\alpha|+2p-1} \rightarrow E_{2}[u_{1}^{-1}] \wedge C_{f_{2}}[u_{1}^{-1}] \rightarrow C_{f_{2}}[u_{1}^{-1}].\]
Similarly, using the $E_{2}$-module structure of $C_{f_{2}}$, we have maps
\[\widetilde{q_{\alpha}} : \Sigma^{|\alpha|}E_{2} \simeq E_{2} \wedge \bbS^{|\alpha|} \rightarrow E_{2} \wedge C_{f_{2}} \rightarrow C_{f_{2}}\]
\[\widetilde{q_{\alpha, \lambda_{1}}} : \Sigma^{|\alpha|+2p-1}E_{2} \simeq E_{2} \wedge \bbS^{|\alpha|+2p-1} \rightarrow E_{2} \wedge C_{f_{2}} \rightarrow C_{f_{2}}.\]
Note $\overline{q_{\alpha}}$, $\overline{q_{\alpha, \lambda_{1}}}$ are just $u_{1}$ localizations of maps $\widetilde{q_{\alpha}}$ and $\widetilde{q_{\alpha, \lambda_{1}}}$, i.e. $\overline{q_{\alpha}} = \widetilde{q_{\alpha}}[u_{1}^{-1}]$ and $\overline{q_{\alpha, \lambda_{1}}} = \widetilde{q_{\alpha, \lambda_{1}}}[u_{1}^{-1}]$. Now note that $C_{f_{2}}$ and $C_{f_{2}}[u_{1}^{-1}]$ are $L_{1}$-local. Hence we get induced maps

\[ \widetilde{q_{\alpha}}' : \Sigma^{|\alpha|}L_{1}E_{2} \rightarrow C_{f_{2}}, \ \widetilde{q_{\alpha, \lambda_{1}}}' : \Sigma^{|\alpha|+2p-1}L_{1}E_{2} \rightarrow C_{f_{2}} \]
\[ \overline{q_{\alpha}}' : \Sigma^{|\alpha|}L_{1}E_{2}[u_{1}^{-1}] \rightarrow C_{f_{2}}[u_{1}^{-1}], \ \overline{q_{\alpha, \lambda_{1}}}' : \Sigma^{|\alpha|+2p-1}L_{1}E_{2}[u_{1}^{-1}] \rightarrow C_{f_{2}}[u_{1}^{-1}]. \]
The wedge sum over all $\alpha$ %: non-empty subsets of $\frB$}
of all such maps $\widetilde{q_{\alpha}}'$ and $\widetilde{q_{\alpha, \lambda_{1}}}'$ will be our $X_{3}$ and the corresponding wedge of $\overline{q_{\alpha}}'$ and $\overline{q_{\alpha, \lambda_{1}}}'$ will be shown to be $K(1)_{\ast}$-isomorphic to $C_{f_{2}}[u_{1}^{-1}]$ in the result below. But, first some notations

\begin{note}\label{Notation 5.3.1}
Let $\widetilde{Q}$ denote the map 
\[\widetilde{Q}: \bigvee_{\alpha} (\widetilde{q_{\alpha}}' \bigvee \widetilde{q_{\alpha, \lambda_{1}}}'): \bigvee_{\alpha} (\Sigma^{|\alpha|}L_{1}E_{2} \bigvee \Sigma^{|\alpha|+2p-1}L_{1}E_{2}) \rightarrow C_{f_{2}}.\]
Similarly, let $\overline{Q}$ denote the map 
\[\overline{Q}: \bigvee_{\alpha} (\overline{q_{\alpha}}' \bigvee \overline{q_{\alpha, \lambda_{1}}}'): \bigvee_{\alpha} (\Sigma^{|\alpha|}L_{1}E_{2}[u_{1}^{-1}] \bigvee \Sigma^{|\alpha|+2p-1}L_{1}E_{2}[u_{1}^{-1}]) \rightarrow C_{f_{2}}[u_{1}^{-1}].\]
Further, let $X_{3} := \bigvee_{\alpha} (\Sigma^{|\alpha|}L_{1}E_{2} \bigvee \Sigma^{|\alpha|+2p-1}L_{1}E_{2})$. Note that this makes \[\bigvee_{\alpha} (\Sigma^{|\alpha|}L_{1}E_{2}[u_{1}^{-1}] \bigvee \Sigma^{|\alpha|+2p-1}L_{1}E_{2}[u_{1}^{-1}]) = X_{3}[u_{1}^{-1}].\]
\end{note}

\begin{theorem} \label{Thm 5.3.2}
We have a commutative diagram as follows
\begin{equation}\label{Cf2}\tag{$\ast \ast \ast$}
\begin{tikzcd}%[column sep=large]
X_{3} \arrow{d}{\widetilde{Q}} \arrow{r}
& X_{3}[u_{1}^{-1}] \arrow{d}{\overline{Q}}
& \\
C_{f_{2}} \arrow{r} & C_{f_{2}}[u_{1}^{-1}] \arrow{d} \arrow{r} & C_{\infty}
& \\
                                   & C_{\overline{Q}}
\end{tikzcd}
\end{equation}
where $C_{\overline{Q}}$ denotes the cofiber of $\overline{Q}$. Moreover, the map $\overline{Q}$ is a $K(1)_{\ast}$-isomorphism and both $C_{\infty}$ and $C_{\overline{Q}}$ are rational.
\end{theorem}

\begin{proof}
The fact that this commutative diagram exists follows from our discussion in \S4.4 thus far. So does the fact that $C_{\infty}$ is rational. Note that by definition $X_{3}$ and hence $X_{3}[u_{1}^{-1}]$ are $L_{1}$ local and as discussed earlier $C_{f_{2}}$ and $C_{f_{2}}[u_{1}^{-1}]$ are both $L_{1}$-local as well. Hence once we show that $\overline{Q}$ induces a $K(1)_{\ast}$-isomorphism, by \S4.1 we know that $C_{\overline{Q}}$ is rational. We prove the $K(1)_{\ast}$-isomorphism now.\\

Keep in mind that we know all the $K(1)$-homology classes of $C_{f_{2}}[u_{1}^{-1}]$ since $C_{f_{2}} \rightarrow C_{f_{2}}[u_{1}^{-1}]$ is a $K(0)_{\ast}$ and $K(1)_{\ast}$-equivalence. Note also, that $q_{\alpha}$, and $q_{\alpha, \lambda_{1}}$ are nothing but the lifts of these classes to $\pi_{\ast}THH(E_{2})$ and hence to $\pi_{\ast}C_{f_{2}}[u_{1}^{-1}]$. Thus after taking $K(1)_{\ast}$ on each side of $\overline{Q}$ we have a wedge of $\Sigma^{|\alpha|}K(1)_{\ast}(L_{1}E_{2}[u_{1}^{-1}]) \cong K(1)_{\ast}(L_{1}E_{2}[u_{1}^{-1}]) \wedge \bbS^{|\alpha|} \cong K(1)_{\ast}E_{2}[u_{1}^{-1}] \wedge \bbS^{|\alpha|}$ and $\Sigma^{|\alpha|+2p-1}K(1)_{\ast}(L_{1}E_{2}[u_{1}^{-1}]) \cong K(1)_{\ast}(L_{1}E_{2}[u_{1}^{-1}]) \wedge \bbS^{|\alpha|+2p-1} \cong K(1)_{\ast}E_{2}[u_{1}^{-1}] \wedge \bbS^{|\alpha|+2p-1}$ on the left side of the map $\overline{Q}$ and it maps correspondingly to $K(1)_{\ast}E_{2}[u_{1}^{-1}]\Pi_{\alpha}(1 \otimes s)$ or $K(1)_{\ast}E_{2}[u_{1}^{-1}]\Pi_{\alpha}(1 \otimes s) dt_{1}$ in $K(1)_{\ast}C_{f_{2}}[u_{1}^{-1}]$. Hence, $\overline{Q}$ is a $K(1)_{\ast}$-equivalence.
\end{proof}\

We denote
\[X_{3} := \bigvee_{\alpha} (\Sigma^{|\alpha|}L_{1}E_{2} \bigvee \Sigma^{|\alpha|+2p-1}L_{1}E_{2}), \ f_{3} := \widetilde{Q} = \bigvee_{\alpha} (\overline{q_{\alpha}}' \bigvee \overline{q_{\alpha, \lambda_{1}}}').\]

\subsection{$THH(E_{2})$ and $THH(E_{2})_{p}^{\wedge}$}

\S4.2, \S4.3, and \S4.4 prove theorem \ref{MainThm} giving us a description for $THH(E_2)$. Now taking $p$-completion of (\ref{thh}), we have the following 

%\begin{equation}\label{thh}\tag{$\ast \ast$}
\[\begin{tikzcd}%[row sep=large, column sep=large]
(E_{2})_{p}^{\wedge} \arrow{r}{(f_{1})_{p}^{\wedge}} & THH(E_{2})_{p}^{\wedge} \arrow{d}{(C_{f_{1}})_{p}^{\wedge}}\\
%& & \tikz\node[draw,circle,inner sep=2pt]{$\ell$}; \ar[d,dotted] & \\ 
(X_{2})_{p}^{\wedge} \arrow{r}{(f_{2})_{p}^{\wedge}} & (C_{f_{1}})_{p}^{\wedge}:=\overline{THH}(E_{2})_{p}^{\wedge} \arrow{d}{(C_{f_{2}})_{p}^{\wedge}}\\
(X_{3})_{p}^{\wedge} \arrow{r}{(f_{3})_{p}^{\wedge}} & (C_{f_{2}})_{p}^{\wedge}.
\end{tikzcd}
\]
%\end{equation}

Here $(X_{2})_{p}^{\wedge} = \Sigma^{2p-1}L_{1}(E_{2})_{p}^{\wedge}$ and $(X_{3})_{p}^{\wedge} = (\bigvee_{\alpha} (\Sigma^{|\alpha|}L_{1}E_{2} \bigvee \Sigma^{|\alpha|+2p-1}L_{1}E_{2}))_{p}^{\wedge}$.% with $(f_{3})_{p}^{\wedge}$ an equivalence as proven below.
We are ready to prove theorem \ref{MainCorr} where we show that for $p$-complete $THH$ of $E_{2}$ spectrum there is a complete description in terms of cofiber sequences made of terms that are suspensions and localizations of $E_{2}$.\\
\begin{comment}
\begin{surthm}
We have the following diagram for $THH(E_{2})_{p}^{\wedge}$, where $(C_{f_{i}})_{p}^{\wedge}$ are cofibers and cofiber maps of $(f_{i})_{p}^{\wedge}$ and $(\overline{Q})_{p}^{\wedge}$ is an equivalence
\[\begin{tikzcd}%[row sep=large, column sep=large]
(E_{2})_{p}^{\wedge} \arrow{r}{(f_{1})_{p}^{\wedge}:= \text{p-completed unit map}} & THH(E_{2})_{p}^{\wedge} \arrow{d}{(C_{f_{1}})_{p}^{\wedge}}\\
%& & \tikz\node[draw,circle,inner sep=2pt]{$\ell$}; \ar[d,dotted] & \\ 
\Sigma^{2p-1}L_{1}(E_{2})_{p}^{\wedge} \arrow{r}{(f_{2})_{p}^{\wedge}:=(\overline{j_{1}})_{p}^{\wedge}} & (C_{f_{1}})_{p}^{\wedge}:=\overline{THH}(E_{2})_{p}^{\wedge} \arrow{d}{(C_{f_{2}})_{p}^{\wedge}}\\
(\bigvee_{\alpha} (\Sigma^{|\alpha|}L_{1}E_{2}[u_{1}^{-1}] \bigvee \Sigma^{|\alpha|+2p-1}L_{1}E_{2}[u_{1}^{-1}]))_{p}^{\wedge} \arrow{r}{(\overline{Q})_{p}^{\wedge}}[swap]{\simeq} & (C_{f_{2}})_{p}^{\wedge}.
\end{tikzcd}
\]
\end{surthm}
\end{comment}

\textit{Proof of theorem \ref{MainCorr}.}
We get the diagram from the $p$-completion as discussed above. The only thing that we need to show here is the new $X_{3}$-term and the equivalence of $(\widetilde{Q})_{p}^{\wedge}$. This follows from taking $p$-completion of (\ref{Cf2}). We get
\[
\begin{tikzcd}%[column sep=large]
(\vee%_{\text{all non-empty} \ \alpha}
(\Sigma^{|\alpha|}L_{1}E_{2} \vee \Sigma^{|\alpha|+2p-1}L_{1}E_{2}))_{p}^{\wedge} \arrow{d}{\widetilde{Q}_{p}^{\wedge}} \arrow{r}
& (\vee%_{\text{all non-empty} \ \alpha}
(\Sigma^{|\alpha|}L_{1}E_{2}[u_{1}^{-1}] \vee \Sigma^{|\alpha|+2p-1}L_{1}E_{2}[u_{1}^{-1}]))_{p}^{\wedge} \arrow{d}{\overline{Q}_{p}^{\wedge}}
& \\
(C_{f_{2}})_{p}^{\wedge} \arrow{r} & (C_{f_{2}}[u_{1}^{-1}])_{p}^{\wedge} \arrow{d} \arrow{r} & \ast
& \\
                                   & \ast
\end{tikzcd}
\]
where the cofibers $C_{\infty}$ and $C_{\overline{Q}}$ vanish since $p$-completion of rational spectra is a point. Thus, $(C_{f_{2}})_{p}^{\wedge} \simeq (C_{f_{2}}[u_{1}^{-1}])_{p}^{\wedge} \simeq (\bigvee_{\alpha} (\Sigma^{|\alpha|}L_{1}E_{2}[u_{1}^{-1}] \bigvee \Sigma^{|\alpha|+2p-1}L_{1}E_{2}[u_{1}^{-1}]))_{p}^{\wedge}$.







\begin{comment}
{\color{red}ANYTHING ELSE AS CONCLUSION???}

-what we couldn't say about thh(E2) for example splitting, complete description aka the rational part

-can we say anything under any hypothesis, what problems did we have, want to say anything about $E_{n}$ in general? - wanna return to En in future works

R - hard to say about rational pieces, because of HH1R and hard to know how to work with wit the choices w emade for B
\end{comment}






\begin{comment}
We will lift $\overline{q_{\alpha}}$, $\overline{q_{\alpha, \lambda_{1}}}$ up to $C_{f_{2}}$ in the result below and then show that a wedge of such classes is $K(1)$-equivalent to $C_{f_{2}}$. This will enable us to define $X_{3}$ and $f_{3}$. 

\begin{surprop}\label{Prop 5.3.1}
For each of the maps $\overline{q_{\alpha}}$, $\overline{q_{\alpha, \lambda_{1}}}$, we have a diagram as follows
\[\begin{tikzcd}%[column sep=large]
\Sigma^{|\alpha|}E_{2}, \Sigma^{|\alpha|+2p-1}E_{2} \arrow{d}{\widetilde{q_{\alpha}}, \widetilde{q_{\alpha, \lambda_{1}}}} \arrow{r}
& \Sigma^{|\alpha|}E_{2}[u_{1}^{-1}], \Sigma^{|\alpha|+2p-1}E_{2}[u_{1}^{-1}] 
\arrow[dl,dotted]
\arrow{d}{\overline{q_{\alpha}}, \overline{q_{\alpha, \lambda_{1}}}}
\arrow{dr}{\overline{q_{\alpha}}_{\infty}, \overline{q_{\alpha, \lambda_{1}}}_{\infty}}
& \\
C_{f_{2}} \arrow{r} & C_{f_{2}}[u_{1}^{-1}] \arrow{r} & C_{\infty}
\end{tikzcd}
\]
The claim of this lemma is that $\overline{q_{\alpha}}$, $\overline{q_{\alpha, \lambda_{1}}}$ lift to maps to $C_{f_{2}}$.
\end{surprop}

\begin{proof}
To show the lift exists, it is enough to show that the maps $\overline{q_{\alpha}}_{\infty}$ and $\overline{q_{\alpha, \lambda_{1}}}_{\infty}$ are $0$. Since $C_{\infty}$ is rational, to show this, it is enough to show that the map lifts at the rationalization of the diagram. Hence, we want to show the following map lifts
\[\begin{tikzcd}%[column sep=large]
\Sigma^{|\alpha|}(E_{2})_{\bbQ}, \Sigma^{|\alpha|+2p-1}(E_{2})_{\bbQ} \arrow{d}{\widetilde{q_{\alpha}}, \widetilde{q_{\alpha, \lambda_{1}}}} \arrow{r}
& \Sigma^{|\alpha|}(E_{2})_{\bbQ}[u_{1}^{-1}], \Sigma^{|\alpha|+2p-1}(E_{2})_{\bbQ}[u_{1}^{-1}] 
\arrow[dl,dotted]
\arrow{d}{\overline{q_{\alpha}}, \overline{q_{\alpha, \lambda_{1}}}}
\arrow{dr}{\overline{q_{\alpha}}_{\infty}, \overline{q_{\alpha, \lambda_{1}}}_{\infty}}
& \\
(C_{f_{2}})_{\bbQ} \arrow{r} & (C_{f_{2}})_{\bbQ}[u_{1}^{-1}] \arrow{r} & C_{\infty}
\end{tikzcd}
\]
To do this it is enough to show that the image of fundamental class of $\Sigma^{|\alpha|}(E_{2})_{\bbQ}$, $\Sigma^{|\alpha|+2p-1}E_{2}$ in $\pi_{\ast}(C_{f_{2}})_{\bbQ}$ is infinitely $u_{1}$-divisible since that proves that the image of $\overline{q_{\alpha}}$, $\overline{q_{\alpha, \lambda_{1}}}$ lies in the image of $(C_{f_{2}})_{\bbQ} \rightarrow (C_{f_{2}})_{\bbQ}[u_{1}^{-1}]$ . Note that $\pi_{\ast}(C_{f_{2}})_{\bbQ}$ is $K(0)_{\ast}C_{f_{2}}$ and $\pi_{\ast}(C_{f_{2}}[u_{1}^{-1}])_{\bbQ}$ is $K(0)_{\ast}(C_{f_{2}}[u_{1}^{-1}])$. Hence we know their structure from earlier discussions. In particular, from {\color{red} find the reference form chapter 4} earlier discussion, we know 








$\pi_{\ast}(C_{f_{2}})_{\bbQ}$ has classes:
\begin{itemize}
\item DEG 1: $K(0)_{\ast}E_{2}dt_{2}$, {\color{blue}$K(0)_{\ast}E_{2}[u_{1}^{-1}](1 \otimes r)$}, $K(0)_{\ast}E_{2}(1 \otimes \beta)$,
\item DEG 2: $K(0)_{\ast}E_{2}dt_{1}dt_{2}$, {\color{blue}$K(0)_{\ast}E_{2}[u_{1}^{-1}]dt_{1}(1 \otimes r)$}, $K(0)_{\ast}E_{2}dt_{1}(1 \otimes \beta)$, {\color{blue}$K(0)_{\ast}E_{2}[u_{1}^{-1}]dt_{2}(1 \otimes r)$}, $K(0)_{\ast}E_{2}dt_{2}(1 \otimes \beta)$, {\color{blue}$K(0)_{\ast}E_{2}[u_{1}^{-1}](1 \otimes r_{1})(1 \otimes r_{2})$}, {\color{blue}$K(0)_{\ast}E_{2}[u_{1}^{-1}](1 \otimes r)(1 \otimes \beta)$}, $K(0)_{\ast}E_{2}(1 \otimes \beta_{1})(1 \otimes \beta_{2})$.
\item and so on.
\end{itemize}

$\pi_{\ast}(C_{f_{2}}[u_{1}^{-1}])_{\bbQ}$ has classes:
\begin{itemize}
\item DEG 1: $K(0)_{\ast}E_{2}[u_{1}^{-1}]dt_{2}$, {\color{blue}$K(0)_{\ast}E_{2}[u_{1}^{-1}](1 \otimes r)$}, $K(0)_{\ast}E_{2}[u_{1}^{-1}](1 \otimes \beta)$,
\item DEG 2: $K(0)_{\ast}E_{2}[u_{1}^{-1}]dt_{1}dt_{2}$, {\color{blue}$K(0)_{\ast}E_{2}[u_{1}^{-1}]dt_{1}(1 \otimes r)$}, $K(0)_{\ast}E_{2}[u_{1}^{-1}]dt_{1}(1 \otimes \beta)$, {\color{blue}$K(0)_{\ast}E_{2}[u_{1}^{-1}]dt_{2}(1 \otimes r)$}, $K(0)_{\ast}E_{2}[u_{1}^{-1}]dt_{2}(1 \otimes \beta)$, {\color{blue}$K(0)_{\ast}E_{2}[u_{1}^{-1}](1 \otimes r_{1})(1 \otimes r_{2})$}, {\color{blue}$K(0)_{\ast}E_{2}[u_{1}^{-1}](1 \otimes r)(1 \otimes \beta)$}, $K(0)_{\ast}E_{2}[u_{1}^{-1}](1 \otimes \beta_{1})(1 \otimes \beta_{2})$.
\item and so on.
\end{itemize}

Marked in blue are the classes that cancel out in the map from $C_{f_{2}}$ to $C_{f_{2}}[u_{1}^{-1}]$. Now assuming \ref{Lemma 5.3.2}, the fundamental classes in the rationalizations of $\Sigma^{|\alpha|}E_{2}[u_{1}^{-1}]$, $\Sigma^{|\alpha|+2p-1}E_{2}[u_{1}^{-1}]$ maps into the classes of $\pi_{\ast}(C_{f_{2}}[u_{1}^{-1}])_{\bbQ}$ in red below:

\begin{itemize}
\item DEG 1: $K(0)_{\ast}E_{2}[u_{1}^{-1}]dt_{2}$, {\color{red}$K(0)_{\ast}E_{2}[u_{1}^{-1}](1 \otimes r)$}, $K(0)_{\ast}E_{2}[u_{1}^{-1}](1 \otimes \beta)$,
\item DEG 2: $K(0)_{\ast}E_{2}[u_{1}^{-1}]dt_{1}dt_{2}$, {\color{red}$K(0)_{\ast}E_{2}[u_{1}^{-1}]dt_{1}(1 \otimes r)$}, $K(0)_{\ast}E_{2}[u_{1}^{-1}]dt_{1}(1 \otimes \beta)$, $K(0)_{\ast}E_{2}[u_{1}^{-1}]dt_{2}(1 \otimes r)$, $K(0)_{\ast}E_{2}[u_{1}^{-1}]dt_{2}(1 \otimes \beta)$, {\color{red}$K(0)_{\ast}E_{2}[u_{1}^{-1}](1 \otimes r_{1})(1 \otimes r_{2})$}, $K(0)_{\ast}E_{2}[u_{1}^{-1}](1 \otimes r)(1 \otimes \beta)$, $K(0)_{\ast}E_{2}[u_{1}^{-1}](1 \otimes \beta_{1})(1 \otimes \beta_{2})$.
\item and so on.
\end{itemize}

Thus the fundamental classes of rationalizations of $\Sigma^{|\alpha|}E_{2}[u_{1}^{-1}]$ and $\Sigma^{|\alpha|+2p-1}E_{2}[u_{1}^{-1}]$ map into classes in $\pi_{\ast}(C_{f_{2}}[u_{1}^{-1}])_{\bbQ}$ which already exist as classes in $\pi_{\ast}(C_{f_{2}})_{\bbQ}$ which are infinitely $u_{1}$-divisible. Hence, the image of fundamental class of $\Sigma^{|\alpha|}E_{2}$, $\Sigma^{|\alpha|+2p-1}E_{2}$ in $\pi_{\ast}(C_{f_{2}})_{\bbQ}$ is infinitely $u_{1}$-divisible.
\end{proof}
\end{comment}















%\addcontentsline{toc}{chapter}{Bibliography}

\bibliographystyle{alpha}
\bibliography{mybib}

\end{document}