\usepackage{amsmath,amssymb,amsfonts, mathtools}
\usepackage[mathscr]{eucal}
\usepackage[margin=1.5in]{geometry}

\usepackage{tikz, tikz-cd}%
\usetikzlibrary{matrix,arrows,decorations.pathmorphing,cd,patterns,calc,backgrounds,patterns}
\tikzset{dummy/.style= {circle,fill,draw,inner sep=0pt,minimum size=1.2mm}}
\tikzset{vertex/.style={fill, circle, minimum size=.1cm, inner sep=0pt}}
\usetikzlibrary{decorations.pathreplacing, calc}
\usepackage{extarrows}

\usepackage{marvosym}
\usepackage{quiver} 

\usepackage{graphicx,scalerel}


\usepackage[pagebackref, colorlinks, citecolor=DarkOrchid, linkcolor=PineGreen, urlcolor=NavyBlue]{hyperref}
\hypersetup{linktoc=all,} %set to all if you want both sections and subsections linked

\usepackage{comment}
\usepackage[shortlabels]{enumitem}
\usepackage{hyperref}

\usepackage[textwidth=25mm, textsize=tiny]{todonotes}
\usepackage[final]{microtype}
%\usepackage[notcite]{showkeys}

%\usepackage{enumerate}
%\newenvironment{enumroman}{\begin{enumerate}[\upshape (i)]}
%{\end{enumerate}}
\usepackage{appendix}





\numberwithin{equation}{section} 
\numberwithin{figure}{section}
\usepackage[nameinlink,capitalise,noabbrev]{cleveref}
\newcommand{\fref}{\cref}
\newcommand{\Fref}{\Cref}
\newcommand{\prettyref}{\cref}
\newcommand{\newrefformat}[2]{}

\usepackage{stmaryrd}%for mapsfrom

%for function domain restrictions
\newcommand\restr[2]{{% we make the whole thing an ordinary symbol
  \left.\kern-\nulldelimiterspace % automatically resize the bar with \right
  #1 % the function
  \vphantom{\big|} % pretend it's a little taller at normal size
  \right|_{#2} % this is the delimiter
  }}



%-----------------------%-----------------------%-----------------------%-----------------------%-----------------------%-----------------------

%Definitions
\crefname{lemma}{Lemma}{Lemmas}
\crefname{theorem}{Theorem}{Theorems}
\crefname{definition}{Definition}{Definitions}
\crefname{proposition}{Proposition}{Propositions}
\crefname{remark}{Remark}{Remarks}
\crefname{observation}{Observation}{Observations}
\crefname{construction}{Construction}{Constructions}
\crefname{corollary}{Corollary}{Corollaries}
\crefname{question}{Question}{Questions}
\crefname{equation}{Equation}{Equations}
\crefname{construction}{Construction}{Constructions}
\crefname{ex}{Example}{Examples}
\crefname{appsec}{Appendix}{Appendices}
\crefname{subsection}{Subsection}{Subsections}
\Crefname{warning}{Warning}{Warnings}

\theoremstyle{plain}
\newtheorem{theorem}[equation]{Theorem}
\newtheorem{corollary}[equation]{Corollary}
\newtheorem{proposition}[equation]{Proposition}
\newtheorem{lemma}[equation]{Lemma}
\newtheorem{conj}[equation]{Conjecture}

\theoremstyle{definition}
\newtheorem{definition}[equation]{Definition}
\newtheorem{example}[equation]{Example}
\newtheorem{question}[equation]{Question}
\newtheorem{remark}[equation]{Remark}
\newtheorem{construction}[equation]{Construction}
\newtheorem{observation}[equation]{Observation}
\newtheorem*{thank}{Acknowledgments}
\newtheorem*{notation}{Notation}
\newtheorem*{note}{Note}
\newtheorem{warning}[equation]{Warning}


%%%%%%%%%%%% Comments
%todos:
\usepackage[textwidth=25mm, textsize=tiny]{todonotes}

% Sanjana's notes
\newcommand{\sanote}[1]{\todo[color=YellowGreen!20,linecolor=yellow!20!black,size=\tiny]{SA: #1}}
\newcommand{\sanoteil}[1]{\ \todo[inline,color=YellowGreen!20,linecolor=yellow!40!black,size=\normalsize]{SA: #1}}

% Mike's notes
\newcommand{\mmnote}[1]{\todo[color=Orange!20,linecolor=blue!20!black,size=\tiny]{MM: #1}}
\newcommand{\mmnoteil}[1]{\ \todo[inline,color=Orange!20,linecolor=blue!40!black,size=\normalsize]{MM: #1}}


%%%%%5 Macro


\newcommand\temp[1][.8]{\mathbin{\ThisStyle{\vcenter{\hbox{%
  \scalebox{#1}{$\SavedStyle\bullet$}}}}}%
  }
\newcommand{\smbullet}{{\temp}}
  
\DeclareMathOperator{\colim}{colim}

\newcommand{\Inj}{\mathrm{Inj}^{\mathrm{fc}}}

\newcommand{\THH}{\mathrm{THH}}
\newcommand{\HH}{\mathrm{HH}}
\newcommand{\coHH}{\mathrm{coHH}}
\newcommand{\coTHH}{\mathrm{coTHH}}
\newcommand{\Hom}{\mathrm{Hom}}
\newcommand{\coMod}{\mathrm{coMod}}
\newcommand{\coLMod}{\mathrm{coLMod}}
\newcommand{\coRMod}{\mathrm{coRMod}}
\newcommand{\tcoMod}{\underline{\coMod}}
\newcommand{\Mod}{\mathrm{Mod}}
\newcommand{\tMod}{\underline{\Mod}}
\newcommand{\Set}{\mathrm{Set}}
\newcommand{\Mat}{\mathrm{Mat}}
\newcommand{\coMat}{\Mat^c}
\newcommand{\ob}{\mathrm{ob}}
\newcommand{\E}{\mathrm{E}}
\newcommand{\Tor}{\mathrm{Tor}}
\newcommand{\hocolim}{\mathrm{hocolim}}


\renewcommand{\k}{{{\Bbbk}}}

\newcommand{\cA}{\mathscr{A}}
\newcommand{\cB}{\mathscr{B}}
\newcommand{\cC}{\mathscr{C}}
\newcommand{\cD}{\mathscr{D}}
\newcommand{\cE}{\mathscr{E}}
\newcommand{\cM}{\mathscr{M}}
\newcommand{\cN}{\mathscr{N}}
\newcommand{\cP}{\mathscr{P}}
\newcommand{\cV}{\mathscr{V}}

\newcommand{\fc}{\mathrm{fc}}

\newcommand{\Z}{{\mathbb Z}}
\newcommand{\Zp}{{\mathbb Z_p}}
\newcommand{\Q}{{\mathbb Q}}
\newcommand{\F}{{\mathbb F}}
\newcommand{\bbL}{{\mathbb L}}



\newcommand{\B}{\mathfrak{B}}


\def\CC{\mathscr{C}}
\def\DD{\mathscr{D}}
\def\SS{\mathscr{S}}
\def\PP{\mathscr{P}}

\newcommand{\coAlg}{\mathrm{coAlg}}
\newcommand{\Alg}{\mathrm{Alg}}

\newcommand{\op}{\mathrm{op}}
\newcommand{\fd}{\mathrm{fd}}
\newcommand{\Vect}{\mathrm{Vect}}
\newcommand{\Cat}{\mathrm{Cat}}
\newcommand{\tCat}{\underline{\mathrm{Cat}}}
\newcommand{\coCat}{\mathrm{coCat}}
\newcommand{\tcoCat}{\underline{\mathrm{coCat}}}
\newcommand{\ux}{\underline{x}}

\newcommand{\coTor}{\mathrm{coTor}}

\newcommand{\fcogi}{{\mathrm{f.cog}, \mathrm{inj}}}
\renewcommand{\fd}{\mathrm{fd}}

\newcommand{\cofree}{\mathrm{coFree}}

\newcommand{\coHom}{{\scriptstyle\mathrm{co}}\mathrm{Hom}}
%\newcommand{\coHom}{\Hom^c}
\newcommand{\id}{\mathrm{id}}

\newcommand{\coev}{\mathrm{coev}}

\newcommand{\CN}{\mathrm{CN}}
\newcommand{\coCN}{\mathrm{coCN}}

%%% simplicial arrows 
\tikzcdset{
  diagrams={>={Straight Barb[scale=0.5]}}
}
\tikzset{
  altstackar/.style={decorate, decoration={show path construction,
    lineto code={
      \path (\tikzinputsegmentfirst); \pgfgetlastxy{\xstart}{\ystart}
      \path (\tikzinputsegmentlast); \pgfgetlastxy{\xend}{\yend}
      \path ($(0,0)!1.5pt!(\ystart-\yend,\xend-\xstart)$); \pgfgetlastxy{\xperp}{\yperp}
      \foreach \n[evaluate=\n as \k using .5*#1-\n+.5] in {1,...,#1}{
        \ifodd\n{\draw[->, shorten <=2pt, shift={($\k*(\xperp,\yperp)$)}](\xstart,\ystart)--(\xend,\yend);}
        \else{\draw[<-, shorten >=2pt, shift={($\k*(\xperp,\yperp)$)}](\xstart,\ystart)--(\xend,\yend);}\fi
      }
    }
  }}, altstackar/.default={1}
}

\makeatletter
\def\slashedarrowfill@#1#2#3#4#5{%
  $\m@th\thickmuskip0mu\medmuskip\thickmuskip\thinmuskip\thickmuskip
  \relax#5#1\mkern-7mu%
  \cleaders\hbox{$#5\mkern-2mu#2\mkern-2mu$}\hfill
  \mathclap{#3}\mathclap{#2}%
  \cleaders\hbox{$#5\mkern-2mu#2\mkern-2mu$}\hfill
  \mkern-7mu#4$%
}
\def\rightslashedarrowfill@{%
  \slashedarrowfill@\relbar\relbar\mapstochar\rightarrow}
\newcommand\xslashedrightarrow[2][]{%
  \ext@arrow 0055{\rightslashedarrowfill@}{#1}{#2}}
\makeatother


% late additions to macros
\newcommand\bbS{\mathbb{S}}
\newcommand\X{\mathbb{Z}[[u_1]]_{\mathbb{Q}}}
\newcommand{\bTHH}{\overline{THH}}
\newcommand{\bbTHH}{\overline{\overline{THH}}}




