% \begin{table*}[t]
%     \centering
%     \resizebox{\textwidth}{!}{
%     \begin{tabular}{c|c|c|c|c|c|c}
%         \hline
%         %\multicolumn{4}{|c|}{GCD-AVG}\\
%         %\hline
%         Common-AUG&Inter-contrastive&Data Views Generation& Hard sample &Intra-contrastive &Celeb-DF&DFD\\
%         \hline
%         \checkmark&&&&&68.52&87.37\\
%         \hline
%         \checkmark&&\checkmark&&&74.12&88.32\\
%         \hline
%         \checkmark&\checkmark&&&&76.81&88.03\\
%         \hline
%         \checkmark&\checkmark&\checkmark&&&79.34&89.24\\
%         \hline
%         \checkmark&\checkmark&&\checkmark&&78.84&89.89\\
%         \hline
%         \checkmark&\checkmark&\checkmark&\checkmark&&80.3&90.12\\
%         \hline
%         \checkmark&\checkmark&\checkmark&\checkmark&\checkmark&\textbf{82.30}&\textbf{91.66}\\
%         \hline
%     \end{tabular}
%     }
%     \vspace{-2pt}
%     \caption{}
%     \label{table:5}
% \end{table*}
% \begin{table}[!t]
%     \centering
%     \renewcommand\arraystretch{1.2}
%     \resizebox{\columnwidth}{!}{
%     \begin{tabular}{cccc|c|c}
%         \hline
%         %\multicolumn{4}{|c|}{GCD-AVG}\\
%         %\hline
%         Inter&Views& Hard & Intra &Celeb-DF&DFD\\
%         \hline
%         % &&&&68.52&87.37\\
%         &\checkmark&&&74.12&88.32\\
%         \hline
%         \checkmark&&&&76.81&88.03\\
%         \hline
%         \checkmark&\checkmark&&&79.34&89.24\\
%         \hline
%         \checkmark&&\checkmark&&78.84&89.89\\
%         \hline
%         \checkmark&\checkmark&\checkmark&&80.30&90.12\\
%         \hline
%         \checkmark&\checkmark&\checkmark&\checkmark&\textbf{82.30}&\textbf{91.66}\\
%         \hline
%     \end{tabular}
%     }
%     \vspace{-2pt}
%     \caption{Ablation study on the influence of 
%     different components. Specifically, ``Instance" means instance contrastive learning module, ``views" represents our special designed data views generation strategy, ``hard" indicate the hard sample generation, and ``self" is short for the self-contrastive learning module.}
%     \label{table:5}
% \end{table}

% \begin{table}[!h]
%     \centering
%     \renewcommand\arraystretch{1.2}
%     \resizebox{\columnwidth}{!}{
%     \begin{tabular}{c|cc|cc}
%         \hline
%         %\multicolumn{4}{|c|}{GCD-AVG}\\
%         %\hline
%         \multirow{2}*{Method} & \multicolumn{2}{c|}{Celeb-DF}&
%         \multicolumn{2}{c}{DFDC-P}  \\

%         \cline{2-5}
%         % \cmidrule{2-5}
%         & AUC& EER &AUC& EER \\
%         \hline
%         Unimodal+Coarse&77.62&29.34&76.31&31.06\\
%         Unimodal+Fine&77.71&29.31&79.30&24.83\\
%         Multimodal+Coarse&80.79&25.16&81.79&26.13\\
%         Multimodal+C2F&\textbf{84.80}&\textbf{22.73}&\textbf{84.74}&\textbf{23.43}\\
%         \hline
%     \end{tabular}
%     }
%     \vspace{-2pt}
%     \caption{Ablation study on the impact of language information in terms of AUC and EER. The Unimodal and Multimodal represents whether convert the digital label into prompt and whether use the text-encoder. Coarse and 
%     Fine means granularity of the label. C2F represents our Coarse-and-Fine Co-training framework.
%     }
%     \label{table:ab1}
% \end{table}
\begin{table}[!t]
    \centering
    \renewcommand\arraystretch{1.2}
    \resizebox{\columnwidth}{!}{
    \begin{tabular}{c|c|cc|cc}
        \hline
        %\multicolumn{4}{|c|}{GCD-AVG}\\
        %\hline
        \multirow{2}*{Multimodal}&\multirow{2}*{PFIG} & \multicolumn{2}{c|}{Celeb-DF}&
        \multicolumn{2}{c}{DFDC-P}  \\

        \cline{3-6}
        % \cmidrule{2-5}
        && AUC& EER &AUC& EER \\
        \hline
        $\times$&$\times$&77.62&29.34&76.31&31.06\\
        $\times$&$\checkmark$&77.71&29.31&77.30&29.83\\
        $\checkmark$&$\times$&80.79&25.16&81.79&26.13\\
        $\checkmark$&$\checkmark$&\textbf{84.80}&\textbf{22.73}&\textbf{84.74}&\textbf{23.43}\\
        \hline
    \end{tabular}
    }
    \vspace{-2pt}
    \caption{Ablation study on the impact of different components in terms of AUC and EER. The Multimodal represents whether convert the digital label into prompt and use the text-encoder. 
    }
    \label{table:ab1}
\end{table}
