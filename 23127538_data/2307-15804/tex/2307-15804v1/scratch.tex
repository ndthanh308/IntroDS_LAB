

{\color{blue}

*** *** OLD *** *** 

\begin{theorem}[Gegenbauer to Hermite roots, regime $d \to \infty$, {\cite[Theorem 2]{f5277d02-ee90-3ae4-936b-eb33efe25beb}}] Let $z_{k,j,\lambda}$, $k=0,\ldots j$ be the zeros of $\bar{P}_{j,\lambda}$, and $h_{k,j}$, $k=0,\ldots j$ be the zeros of the Hermite polynomial $H_j$. Then, for each $j$,  
\begin{equation}
    z_{k,j,\lambda} = h_{k,j} \lambda^{-1/2} - \frac{h_{k,j}}{8} \left( 2j-1 + 2 h_{k,j}^2 \right) \lambda^{-3/2} + O(\lambda^{-5/2})~.
\end{equation}
\end{theorem}

\begin{theorem}[Location of Hermite roots, {\cite[Theorem 6.32]{szego1939orthogonal}}]
    Let $h_{j,j}$ and $h_{j-1,j}$ be respectively the largest and second largest root of $H_j$. Then there exists constants $C_1, C_2$ such that 
    \begin{equation}
        h_{k,j} = ( 2 j + 1)^{1/2} - C_k (2 j + 1)^{-1/6} + \epsilon_j ~,~ k \in \{j,j-1\}~,
    \end{equation}
    with $\epsilon_j = o( j^{-1/6})$. 
\end{theorem}



An immediate consequence of the fundamental theorem of calculus is 
\begin{lemma}
\begin{equation}
        \upsilon_{j,d} = \frac{j (j+d-2)}{d-1} \int_{{z}_{j-1,d+1}}^{{z}_{j,d}} P_{j-1,d+2}(t) dt~.
\end{equation}
\end{lemma}
Let us now use Lemma \ref{lem:basiccontrolgegen} to obtain simple bounds of $P_{j,d}(t)$ for $t \geq z_{j,d}$.
For each $\delta>0$, $j$ and $d$, let 
$$m(\delta, j, d) := | \{ z_{k,j,d} ; z_{k,j,d} \geq z_{j,d} - \delta \} |$$ 
denote the number of zeros of $P_{j,d}$ in the interval $(z_{j,d}-\delta, z_{j,d})$. 
We have
\begin{fact}
    \begin{equation}
        P_{j,d}(t) \leq \left(\frac{t^2 - (z_{j,d}-\delta)^2}{1-(z_{j,d}-\delta)^2} \right)^{m(\delta, j, d)}~.
    \end{equation}
\end{fact}



We collect several useful bounds for Gegenbauer polynomials, from \cite{szego1939orthogonal}. 
% \begin{theorem}[{\cite[Theorem 8.21.13]{szego1939orthogonal}}]
%     \begin{equation}
%         \bar{P}_{j, \lambda}(\cos \theta) = j^{-1/2} \pi^{-1/2} \left(\frac12 \sin \theta \right)^{-\lambda} \left\{ \cos((j + \lambda) \theta -\lambda \pi /2) + (j \sin(\theta))^{-1} O(1)\right\}
%     \end{equation} 
%     for $c j^{-1} \leq \theta \leq \pi - c j^{-1}$. %and where 
%    % $$K(\theta) = ~,~N = ~,~\gamma = ~.$$
% \end{theorem}

\begin{theorem}[{\cite[Theorem 8.21.12]{szego1939orthogonal}}]
\label{thm:besselgegen}
    \begin{equation}
        \bar{P}_{j, \lambda}(\cos \theta) = \left(\frac12 \sin \theta \right)^{-\lambda} \left( (j+\lambda)^{-\lambda+\frac12} \frac{\Gamma(j + \lambda + \frac12)}{j!}\sqrt{\frac{\theta}{\sin \theta}} J_{\lambda-1/2}((j+\lambda) \theta) + \theta^{1/2}O(j^{-3/2})\right)~,
    \end{equation}
    where $J_\nu(z)$ is the Bessel function of the first kind. 
\end{theorem}

\begin{theorem}[{\cite[Eq (3.4.5)]{gallardo}}]
    Let $\Lambda_\nu(x) = \frac{\Gamma(\nu+1) J_\nu(x)}{(\frac12 x)^\nu}$ and $N = j + \lambda + \frac12$. 
    We have
    \begin{equation}
        \left(\frac{\sin \theta}{\theta} \right)^{\lambda+\frac12} P_{j,2\lambda+2}(\cos \theta) = \Lambda_\lambda( N \theta) + R_j(\theta)~,
    \end{equation}
    where $R_j(\theta) = \theta^{\frac12 - \lambda} j^{-3/2-\lambda} O_j(1)$ for $C j^{-1} \leq \theta \leq \pi - \epsilon$, and $R_j(\theta) = \theta^2 O_j(1)$ for $0 \leq \theta \leq C j^{-1}$. 
\end{theorem}

% \begin{fact}
%     For $\cos \theta_{j,d} = z_{j-1,d+2}$, we have 
%     $$\left(\frac{\sin \theta_{j,d}}{\theta_{j,d}} \right)^{\lambda+\frac12} \geq $$
%     $$|\Lambda_\lambda(N \theta_{j,d})| \leq \kappa^{-\lambda} $$
%     $$|R_j(\theta_{j,d})| \leq B \kappa^{-\lambda}  $$
% \end{fact}

\begin{corollary}[Behavior of $\upsilon_{j,d}$, large $j$ regime] Assume the Ansatz $\left(\frac{\sin \theta}{\theta} \right)^{\lambda+\frac12} P_{j,2\lambda+2}(\cos \theta) \sim \Lambda_\lambda( N \theta)$. Then 
    \begin{equation}
        \upsilon_{j,d} \lesssim (e/2)^{-\lambda} \lambda^{1/6} \left[1 + \frac{\lambda^2}{j(j+2\lambda)} \right]^{\lambda + 1/2} ~.
    \end{equation}
    In particular, if $j > \lambda$, then there exists $\rho < 1$ such that 
    \begin{equation}
        \upsilon_{j,d} = O( \rho^{d} )~.
    \end{equation}
\end{corollary}
\begin{proof}
    The largest root of $P_{j,d}$ is given by Fact \ref{fact:largestroot}, by 
    $$z_{j,d}\leq \sqrt{\frac{(j-1)(j+d-4)}{(j+d/2-3)(j+d/2-2)}} \cos(\pi/(j+1)) := z_{j,d}^*~.$$

  Let $\cos \theta_{j,d} = z_{j-1,d+2}^*$.
From the Ansatz $\left(\frac{\sin \theta}{\theta} \right)^{\lambda+\frac12} P_{j,2\lambda+2}(\cos \theta) \sim \Lambda_\lambda( N \theta)$, and using Cauchy's asymptotic formula $J_\lambda(\lambda) \simeq \frac{\Gamma(1/3)}{2^{2/3} 3^{1/6} \pi \lambda^{1/3}}$ as well as Stirling's formula, we obtain
 \begin{align*}
    \upsilon_{j,\lambda} &= |P_{j,2\lambda+2}( \cos( \theta_{j, \lambda}) )| \\
    & \leq \left(\frac{\sin \theta_{j,d}}{\theta_{j,d}} \right)^{-\lambda-\frac12} 2^\lambda \lambda! J_\lambda\left( \theta_{j,d}(j+\lambda+1/2) \right) (\theta_{j,d}(j+\lambda+1/2))^{-\lambda} \\ 
    & \lesssim \left[\frac{(j+\lambda-2)(j+\lambda-1)}{(j-1)(j+2\lambda-2)} \right]^{\lambda + 1/2} 2^\lambda \lambda! J_\lambda( \sqrt{\lambda (\lambda-1)})  \lambda^{-\lambda} \\
    &\lesssim \left[\frac{(j+\lambda-2)(j+\lambda-1)}{(j-1)(j+2\lambda-2)} \right]^{\lambda + 1/2} \lambda^{-1/3-\lambda} 2^\lambda \lambda! \\
    & \lesssim (e/2)^{-\lambda} \lambda^{1/6} \left[1 + \frac{\lambda(\lambda-1)}{(j-1)(j+2\lambda-2)} \right]^{\lambda + 1/2}  
\end{align*}




% If we denote $C_d = \min_z J_{d/2}(z)$, and since $P_{j,\lambda}(z) > 0$ for $z > z_{j,d}$, 
% we thus have 
% \begin{align*}
%     \upsilon_{j,d} &\leq \frac{j! (d-3)!}{(j+d-3)!} \left(\frac{\sqrt{1-z_{j,d}^2}}{2}\right)^{-d/2} \left( (j+d/2)^{-d/2} \frac{\Gamma(j+d/2)}{j!} \sqrt{1-z_{j,d}^2}^{-1/2} C_{d} + O(j^{-3/2}) \right) \\
%     & = O( j^{-3d/4})~.
% \end{align*}
    
\end{proof}



\begin{lemma}%[{ \cite[Lemma 2.3]{de2008local}}]
Let $g_{j,d}$ be the gap between the largest and second largest root of $P_{j,d}$. For any $j \geq 2$, it holds  
    \begin{align}
        \upsilon_{j,d} &\leq \frac{j(j+d-2)}{d-1}  g_{j,d} z_{j,d}^{j-1}~.
    \end{align}
\end{lemma}
\begin{proof}
    The proof is a simple modification of the argument in \cite[Lemma 2.3]{de2008local}, in which we apply the mean-value theorem between consecutive zeros. 

Following \cite[Lemma 2.3]{de2008local}, we have that $\upsilon_{j,d} \leq \sup_{x \in [0, z_{j,d})} |P_{j,d}|$, where $z_{j,d}$ is the largest positive root of $P_{j,d}$. This maximum is attained at the largest root of $P'_{j,d} = \frac{j(j+d-2)}{d-1} P_{j-1,d+2}$, which by consistency we denote by $z_{j-1,d+2}$. 
%Let $\tilde{z}_{j,d}$ be the second-largest root of $P_{j,d}$, which by the interlacing property of orthogonal polynomials satisfies $\tilde{z}_{j,d} \leq z_{j-1,d+2}$. 
By the mean-value theorem, there exists $\zeta \in (z_{j-1,d+2}, z_{j,d})$ such that 
\begin{align*}
    P_{j,d}(z_{j-1,d+2}) - P_{j,d}( {z}_{j,d}) &= ( z_{j-1,d+2} - {z}_{j,d}) P_{j,d}'(\zeta)~,
\end{align*}
thus 
\begin{equation}
    \upsilon_{j,d} \leq ( z_{j,d} - {z}_{j-1,d+2}) \frac{j(j+d-2)}{d-1} \zeta^{j-1} \leq \frac{j(j+d-2)}{d-1} \cdot  z_{j,d}^j~,
\end{equation}
since $P_{j-1,d+2}(z) \leq z^{j-1}$ in $(z_{j-1,d+2}, 1]$. 
% By the mean-value theorem, there exists $\zeta \in (\tilde{z}_{j,d}, z_{j-1,d+2})$ such that 
% \begin{align*}
%     P_{j,d}(z_{j-1,d+2}) - P_{j,d}( \tilde{z}_{j,d}) &= ( z_{j-1,d+2} - \tilde{z}_{j,d}) P_{j,d}'(\zeta)~,
% \end{align*}
% thus 
% \begin{equation}
%     \upsilon_{j,d} \leq ( z_{j-1,d+2} - \tilde{z}_{j,d}) \frac{j(j+d-2)}{d-1} \upsilon_{j-1,d+2} \leq \frac{j(j+d-2)}{d-1} g_{j,d} \cdot  \upsilon_{j-1,d+2}~.
% \end{equation}
\end{proof}

Collecting the bounds so far, we have 
\begin{corollary}

    There exist sequences $(C_j)_j$, $(\tilde{C}_d)_d$ such that 
    \begin{equation}
        \upsilon_{j,d} \leq \min(1, C_j d^{-j/2}, \tilde{C}_d j^{-3d/4})~.
    \end{equation}
    $C_j$ can be explicitly bounded from the upper bounds on $z_{j,d}$. 
    $\tilde{C}_d$ remains unknown. 
\end{corollary}



\begin{fact}[Gegenbauer Representation in terms of Hermite]
\begin{equation}
    \bar{P}_{j,\lambda}(t) = ( \lambda(1-2t^2))^{j/2} \sum_{k=0}^j \frac{c_k}{z^k} \frac{H_{j-k}(\zeta)}{(j-k)!}~,
\end{equation}
where $\zeta = t \lambda / \sqrt{( \lambda(1-2t^2))}$ and the coefficients $c_k$ are the MacLaurin series of $f(t, w) = e^{-2 \lambda t w + \lambda(1-2t^2)w^2}(1-2tw + w^2)^{-\lambda}$. 
In particular, we have 
\begin{equation}
    P_{j,d}(t) \approx \frac{(d/2)^{j/2} (d-3)!}{(j+d-3)!} H_j( \sqrt{d/2} t)  ~.
\end{equation}
\end{fact}
\begin{corollary}
    \begin{equation}
        \upsilon_{j,d} \leq \frac{j! (d-3)!}{(j+d-3)!} \left( \frac{(d/2)^{j/2}}{j!} \min_t H_j(t) + O( d^{-\alpha} ) \right)
    \end{equation}
        \begin{equation}
        \upsilon_{j,d} \simeq \frac{ (d-3)!}{(j+d-3)!} (d/2)^{j/2} \min_t H_j(t) 
    \end{equation}
\end{corollary}


\begin{fact}
    Let $g_{j,d}$ be the gap between the largest and second largest root of $P_{j,d}$. Then for $d \geq 3$, 
    \begin{align}
        \upsilon_{j,d} &\leq \frac{j(j+d-2)}{d-1} g_{j,d} \cdot \upsilon_{j-1,d+2}~.
    \end{align}
\end{fact}
\begin{proof}
    The proof is a simple modification of the argument in \cite[Lemma 2.3]{de2008local}, in which we apply the mean-value theorem between consecutive zeros. 
    
Following \cite[Lemma 2.3]{de2008local}, we have that $\upsilon_{j,d} \leq \sup_{x \in [0, z_{j,d})} |P_{j,d}|$, where $z_{j,d}$ is the largest positive root of $P_{j,d}$. This maximum is attained at the largest root of $P'_{j,d} = \frac{j(j+d-2)}{d-1} P_{j-1,d+2}$, which by consistency we denote by $z_{j-1,d+2}$. Let $\tilde{z}_{j,d}$ be the second-largest root of $P_{j,d}$, which by the interlacing property of orthogonal polynomials satisfies $\tilde{z}_{j,d} \leq z_{j-1,d+2}$. By the mean-value theorem, there exists $\zeta \in (\tilde{z}_{j,d}, z_{j-1,d+2})$ such that 
\begin{align*}
    P_{j,d}(z_{j-1,d+2}) - P_{j,d}( \tilde{z}_{j,d}) &= ( z_{j-1,d+2} - \tilde{z}_{j,d}) P_{j,d}'(\zeta)~,
\end{align*}
thus 
\begin{equation}
    \upsilon_{j,d} \leq ( z_{j-1,d+2} - \tilde{z}_{j,d}) \frac{j(j+d-2)}{d-1} \upsilon_{j-1,d+2} \leq \frac{j(j+d-2)}{d-1} g_{j,d} \cdot  \upsilon_{j-1,d+2}~.
\end{equation}
    
\end{proof}

\begin{fact}[Gap between largest and second largest root of $P_{j,d}$, regime $j \leq d$]
\label{fact:gap_smallj}
We have the following control of $g_{j,d}$:
\begin{equation}
    g_{j,d} = O ( \sqrt{j/d} )~.
\end{equation}
Moreover, as $j, d \to \infty$, we have the refinement 
\begin{equation}
    g_{j,d} = C j^{-1/6} d^{-1/2} + O(d^{-3/2}) + o( j^{-1/6})~.
\end{equation}
\end{fact}
\begin{proof}
    We combine the asymptotic relation between Gegenbauer and Hermite zeroes from \cite[Theorem 2]{f5277d02-ee90-3ae4-936b-eb33efe25beb} and the location of Hermite zeros from \cite[Theorem 6.32]{szego1939orthogonal}.
\begin{theorem}[Gegenbauer to Hermite roots, {\cite[Theorem 2]{f5277d02-ee90-3ae4-936b-eb33efe25beb}}] Let $x_{j,k,\lambda}$, $k=0,\ldots j$ be the zeros of $\bar{P}_{j,\lambda}$, and $h_{j,k}$, $k=0,\ldots j$ be the zeros of the Hermite polynomial $H_j$. Then 
\begin{equation}
    x_{j,k,\lambda} = h_{j,k} \lambda^{-1/2} - \frac{h_{j,k}}{8} \left( 2j-1 + 2 h_{j,k}^2 \right) \lambda^{-3/2} + O(\lambda^{-5/2})~.
\end{equation}
\end{theorem}
\begin{proposition}[{\cite[Eq (6.31.19)]{szego1939orthogonal}}]
    Let $h_{j,1}$ and $h_{j,2}$ be respectively the largest and second largest root of $H_j$.
There exist constants $b, B$ such that 
\begin{equation}
    b\sqrt{j} \leq h_{j,2} < h_{j,1} \leq B \sqrt{j}.
\end{equation}
\end{proposition}
\begin{theorem}[Location of Hermite roots, {\cite[Theorem 6.32]{szego1939orthogonal}}]
    Let $h_{j,1}$ and $h_{j,2}$ be respectively the largest and second largest root of $H_j$. Then there exists constants $C_1, C_2$ such that 
    \begin{equation}
        h_{j,k} = ( 2 j + 1)^{1/2} - C_k (2 j + 1)^{-1/6} + \epsilon_j ~,~ k \in \{1,2\}~,
    \end{equation}
    with $\epsilon_j = o( j^{-1/6})$. 
\end{theorem}
\end{proof}
\begin{fact}[Gap $g_{j,d}$, regime $j \gg d$]
\label{fact:gap_largej}
In the regime $j \gg d$, we have 
    $g_{j,d} =  O\left( \frac{d^{4/3}}{j^2}\right)~.$
\end{fact}
\begin{proof}
    We first use the asymptotic expression of the zeros of Jacobi polynomials in terms of Bessel functions, given by 
    \begin{theorem}[{\cite[Theorem 8.12]{szego1939orthogonal}}]
        Let $x_{1n} > x_{2n} \dots $ be the zeros of $P_{n}^{(\alpha, \beta)}(x)$ in $[-1,+1]$ in decreasing order. If we write $x_{\nu n} = \cos(\theta_{\nu n})$, $0 < \theta_{\nu n} < \pi$, then for a fixed $\nu$, $\lim_n n \theta_{\nu n} = j_\nu$, where $j_\nu$ is the $\nu$-th positive zero of $J_\alpha(z)$. 
    \end{theorem}
    Now, from \cite[Section 7]{doi:10.1098/rsta.1954.0021} we use the asymptotic expression for the zeros of $J_{\lambda-1/2}$. As a result, if we let $x_{j,1,\lambda}, x_{j, 2, \lambda}$ be the two largest positive roots of $\bar{P}_{j,\lambda}$, then there exists two constants $B_1,B_2$ such that 
    \begin{equation}
        \arccos( x_{j,k,\lambda}) \simeq \frac{d}{2j} + B_k \frac{d^{1/3}}{j} + o(d^{1/3}/j)~,~k=1,2~.
    \end{equation}
    It follows that 
    \begin{align}
        g_{j,d} &= x_{j,1,\lambda} - x_{j, 2, \lambda} \nonumber \\
        &= \cos( \arccos( x_{j,1,\lambda} ) ) - \cos( \arccos( x_{j,2,\lambda} ) ) \nonumber \\
        &\simeq C \sin(d/2j) \frac{d^{1/3}}{j} \nonumber \\
        & = O\left( \frac{d^{4/3}}{j^2}\right)~.
    \end{align}
\end{proof}

\begin{fact}
    We have $\upsilon_{3, d} \simeq d^{-3/2}$.
\end{fact}
\begin{proof}
    Using the recurrence (\ref{eq:recurrgegen}) we directly obtain that 
    \begin{align*}
     2\bar{P}_{2, \lambda}(t) &= 2 ( 1 + \lambda) t \bar{P}_{1, \lambda}(t) - 2\lambda = 2\lambda( 2( 1 + \lambda) t^2 - 1) \\ %2 \lambda - 2\lambda\\
    3 \bar{P}_{3,\lambda}(t) &= 2( 2 + \lambda) t \bar{P}_{2, \lambda}(t) - (1 + 2\lambda) \bar{P}_{1,\lambda}(t) ~,  
    \end{align*}
    and thus 
    \begin{equation}
        \bar{P}_{3,\lambda}(t) = \frac{2 \lambda}{3} \left((2+ \lambda)t (2(1+\lambda)t^2 -1) - (1+2\lambda)t  \right)~.
    \end{equation}

    From the identity $\bar{P}'_{j, \lambda} = 2 \lambda \bar{P}_{j-1, \lambda+1}$ we deduce that the positive critical points of $\bar{P}_{3, \lambda}$ are given by $t^* = \sqrt{\frac{1}{2(2+\lambda)}}$, and therefore 
    \begin{align*}
        \upsilon_{3,d} & = \frac{\Gamma(4) \Gamma(d-2)}{\Gamma(d+1)} |\bar{P}_{3, \frac{d}{2}-1}(t^*) | \\
        &= \frac{6}{d (d-1) (d-2)} \frac{d-2}{3} \left| \frac{\sqrt{\frac{d}{2} + 1}}{\sqrt{2}} (d /(d+2) - 1) - \frac{d-1}{\sqrt{d+2}} \right| \\
        &= \frac{2}{d (d-1)} \left( \frac{1}{\sqrt{d+2}} + \frac{d-1}{\sqrt{d+2}} \right) \\
        &= \frac{2}{(d-1)\sqrt{d+2}}~.
    \end{align*}
\end{proof}
% \begin{corollary}

%     \begin{align}
%         \upsilon_{j,d} &\leq C \frac{j^{5/6} (j+d-2)}{d^{3/2}} \cdot \upsilon_{j-1,d+2}~,
%     \end{align}
%     with $\upsilon_{3, d} \simeq d^{-5/2}$. 
% \end{corollary}
Combining the previous facts, and relying on the asymptotic expressions from Facts \ref{fact:gap_smallj}, \ref{fact:gap_largej}
we arrive at 
\begin{corollary}
    We have, at leading order term, the bound
    % \begin{equation}
    %     \upsilon_{j,d} \leq \frac{C^j (j!)^{5/6} (2j+d-2)! ((d-1)!)^{3/2}}{(j+d-2)! ((d+2j)!)^{3/2} (d + 2j)^{3/2} }~.
    % \end{equation}
MIGHT not be strong enough.. look for a simpler route
\end{corollary}
\begin{proof}
    Assume first $j>d$. Let $\Delta = \lfloor(j-d)/3\rfloor$. We have 
\begin{align}
\label{eq:gup0}
    \upsilon_{j,d}& \leq C\frac{ (j+d-2) d^{1/3}}{j} \upsilon_{j-1,d+2}  \nonumber \\
    &\leq C^2 \frac{ (j+d-2)(j+d-1) (d (d+2))^{1/3}}{j(j-1)} \upsilon_{j-2,d+4} \nonumber  \\
    & \dots \nonumber  \\
    & \leq C^\Delta \frac{ (j+d-2)(j+d-1)\dots (j+d+\Delta-2) (d (d+2) \dots (d+2\Delta)^{1/3}}{j(j-1)\dots (j-\Delta)} \upsilon_{j-\Delta,d+2\Delta}~.
\end{align}
On the other hand, using now Fact \ref{fact:gap_smallj} we have 
\begin{align}
\label{eq:gup1}
    \upsilon_{j-\Delta,d+2\Delta} & \leq \tilde{C} \frac{(j-\Delta)^{5/6} (j+d+\Delta-2)}{(d+2\Delta)^{3/2}} \upsilon_{j-\Delta-1, d+2\Delta+2} \nonumber \\
    & \leq \tilde{C}^2  \frac{((j-\Delta)(j-\Delta-1))^{5/6} (j+d+\Delta-2)(j+d+\Delta-1)}{((d+2\Delta)(d+2\Delta+2))^{3/2}} \upsilon_{j-\Delta-2, d+2\Delta+4} \nonumber \\
    & \dots \nonumber \\
    & \leq \tilde{C}^{j -\Delta} \frac{(j!)^{5/6} \cdot (2j+d-2)! \cdot ((d+2\Delta)!!)^{3/2} }{ (j+d+\Delta-3)! \cdot ((d+2j)!!)^{3/2}} (d+2j)^{-3/2}
\end{align}
Putting together (\ref{eq:gup0}) and (\ref{eq:gup1}) we obtain
{\small
\begin{align}
    \upsilon_{j,d} &\leq \max(C, \tilde{C})^j (d+2j)^{-3/2} \frac{(j!)^{5/6} \cdot (2j+d-2)! \cdot ((d+2\Delta)!!)^{3/2} }{ (j+d+\Delta-3)! \cdot ((d+2j)!!)^{3/2}} \frac{ (j+d-2)(j+d-1)\dots (j+d+\Delta-2) (d (d+2) \dots (d+2\Delta)^{1/3}}{j(j-1)\dots (j-\Delta)} \\
    &\leq \max(C, \tilde{C})^j (d+2j)^{-3/2} \frac{(j!)^{5/6} \cdot (2j+d-2)!}{(j+d-3)!} \left(\frac{(d+2\Delta)!!}{(d-2)!!}\right)^{1/3} \frac{(j-\Delta)!}{j!}  \left(\frac{(d+2\Delta)!!}{(d+2j)!!}\right)^{3/2}
\end{align}
}
\end{proof}
}




--- ---- ---- ---- ---- 
\begin{align}
    |P_{j,\lambda}(\cos \theta) | & = \bar{P}_{j,\lambda}(1)^{-1} \sqrt{\frac{u \pi}{2}} \frac{\Gamma(\frac12 (u + \lambda)) \theta^{1/2}}{\Gamma(\frac12 (u - \lambda) + 1) \Gamma(\lambda) \sin(\theta)^\lambda} \cdot \left\{ 
|J_{\lambda-1/2}(u \theta)|\left(1 + O\left(\frac{\lambda^3}{u^2}\right) \right)\right\} \nonumber \\
& = \frac{\Gamma(j+1)\Gamma(2\lambda)}{\Gamma(j+2\lambda)} \sqrt{\frac{u \pi}{2}} \frac{\Gamma(\frac12 (u + \lambda)) \theta^{1/2}}{\Gamma(\frac12 (u - \lambda) + 1) \Gamma(\lambda) \sin(\theta)^\lambda} \cdot \left\{ 
|J_{\lambda-1/2}(u \theta)|\left(1 + O\left(\frac{\lambda^3}{u^2}\right) \right)\right\} \nonumber \\
& = \frac{\Gamma(j+1)\Gamma(2\lambda)}{\Gamma(j+2\lambda)}  \frac{\Gamma(\lambda+j/2)) }{\Gamma(j/2+1) \Gamma(\lambda) } O(j^\lambda) O(\lambda^{1/2-\lambda}) \cdot \left( O(\lambda^{-1/3}) + O\left(\frac{\lambda^{3-1/3}}{j^2}\right) \right) \nonumber \\
&\simeq \frac{j^{1/2} j^j e^{-j} \lambda^{1/2} (2\lambda)^{2\lambda-1} e^{-2\lambda+1} j^{1/2} (j/2)^{\lambda+j/2-1} e^{1-j/2-\lambda} }{j^{1/2} j^{j+2\lambda-1} e^{1-j-2\lambda}  j^{1/2} (j/2)^{j/2} e^{-j/2} \lambda^{1/2} \lambda^{\lambda-1} e^{1-\lambda}} O(j^\lambda) O(\lambda^{1/6-\lambda}) \nonumber \\
& \simeq j^{3j/2+\lambda-3j/2-2\lambda+\lambda} e^{-j-2\lambda+1+1-j/2-\lambda -1+j+2\lambda +j/2-1+\lambda} \lambda^{1/2+2\lambda-1-1/2-\lambda+1+1/6-\lambda} 2^{2\lambda-1-\lambda-j/2+1+j/2} \\
&= 
\end{align}



we have 
    \begin{equation}
        \upsilon_{j,d} \lesssim (e/2)^{-\lambda} \lambda^{1/6} \left[1 + \frac{\lambda^2}{j(j+2\lambda)} \right]^{\lambda + 1/2} ~.
    \end{equation}

From Fact \ref{fact:upsi_basicbound}, we now define 
$\delta_*=\delta_*(j,d)$ such that $m(\delta_*, j, d) = \Theta(d)$. 
Using the upper bound for $z_{j-1, d+2}^2$ from Fact \ref{fact:largestroot} we thus obtain that 
\begin{align}
    \upsilon_{j,d} &\leq \left(\frac{e_{j,\lambda}/f_{j,\lambda} - \delta_*^2 }{1-\delta_*^2} \right)^{\Theta(d)} \nonumber \\
    &= \left(\frac{(j-1)(j+2\lambda-2) - (j+\lambda-2)(j+\lambda-1)\delta_*^2 }{(j+\lambda-2)(j+\lambda-1)(1-\delta_*^2)} \right)^{\Theta(d)} \nonumber \\
    & \simeq \left(\frac{j(j+2\lambda) - (j+\lambda)^2 \delta_*^2 }{(j+\lambda)^2(1-\delta_*^2)} \right)^{\Theta(d)} \nonumber \\
    & = \left((1-\delta_*^2)^{-1} \left( 1 - \delta_*^2 - \frac{\lambda^2}{(j+\lambda)^2  }  \right)\right)^{\Theta(d)}~,
\end{align}
so if $\frac{\lambda^2}{(j+\lambda)^2} \geq  \rho (1 - \delta_*^2)  $
for $1> \rho > 0$, we obtain
\begin{equation}
    \upsilon_{j,d} \lesssim (1-\rho)^{\Theta(d)}~,
\end{equation}
as claimed. 

%$$\frac{\lambda^2}{\rho(j+\lambda)^2} =  (1 - \delta_*^2)$$
%$$\delta_*^2 = 1 - \frac{\lambda^2}{\rho(j+\lambda)^2} $$
So, it is sufficient to show that 
\begin{equation}
m\left( 1 - O\left(\frac{\lambda^2}{(j+\lambda)^2}\right) , j, \lambda\right) = \Theta(\lambda)~,     
\end{equation}
in other words, there is a (vanishingly small) fraction $\Theta(\lambda/j)$ of the roots of $\bar{P}_{j,\lambda}$ in the (vanishingly small) interval $\left[1 - O\left(\frac{\lambda^2}{(j+\lambda)^2}\right), 1 \right]$. 



For $j \gg d$, we rely on asymptotic properties of Gegenbauer polynomials. 
In particular, we consider 
\begin{theorem}[{\cite[Eq (3.4.5)]{gallardo}}]
    Let $\Lambda_\nu(x) = \frac{\Gamma(\nu+1) J_\nu(x)}{(\frac12 x)^\nu}$, where $J_\nu(x)$ is the Bessel function of the first kind, and $N = j + \lambda + \frac12$. 
    We have
    \begin{equation}
        \left(\frac{\sin \theta}{\theta} \right)^{\lambda+\frac12} P_{j,2\lambda+2}(\cos \theta) = \Lambda_\lambda( N \theta) + R_j(\theta)~,
    \end{equation}
    where $R_j(\theta) = \theta^{\frac12 - \lambda} j^{-3/2-\lambda} O_j(1)$ for $C j^{-1} \leq \theta \leq \pi - \epsilon$, and $R_j(\theta) = \theta^2 O_j(1)$ for $0 \leq \theta \leq C j^{-1}$. 
\end{theorem}

% \begin{fact}
%     For $\cos \theta_{j,d} = z_{j-1,d+2}$, we have 
%     $$\left(\frac{\sin \theta_{j,d}}{\theta_{j,d}} \right)^{\lambda+\frac12} \geq $$
%     $$|\Lambda_\lambda(N \theta_{j,d})| \leq \kappa^{-\lambda} $$
%     $$|R_j(\theta_{j,d})| \leq B \kappa^{-\lambda}  $$
% \end{fact}

\begin{corollary}[Behavior of $\upsilon_{j,d}$, large $j$ regime] Assume the Ansatz $\left(\frac{\sin \theta}{\theta} \right)^{\lambda+\frac12} P_{j,2\lambda+2}(\cos \theta) \sim \Lambda_\lambda( N \theta)$. Then 
    \begin{equation}
        \upsilon_{j,d} \lesssim (e/2)^{-\lambda} \lambda^{1/6} \left[1 + \frac{\lambda^2}{j(j+2\lambda)} \right]^{\lambda + 1/2} ~.
    \end{equation}
    In particular, if $j = \omega(\lambda)$, then there exists $\rho < 1$ such that 
    \begin{equation}
        \upsilon_{j,d} = O( \rho^{d} )~.
    \end{equation}
\end{corollary}
\begin{proof}
    The largest root of $P_{j,d}$ is given by Fact \ref{fact:largestroot}, by 
    $$z_{j,d}\leq \sqrt{\frac{(j-1)(j+d-4)}{(j+d/2-3)(j+d/2-2)}} \cos(\pi/(j+1)) := z_{j,d}^*~.$$

  Let $\cos \theta_{j,d} = z_{j-1,d+2}^*$.
From the Ansatz $\left(\frac{\sin \theta}{\theta} \right)^{\lambda+\frac12} P_{j,2\lambda+2}(\cos \theta) \sim \Lambda_\lambda( N \theta)$, and using Cauchy's asymptotic formula $J_\lambda(\lambda) \simeq \frac{\Gamma(1/3)}{2^{2/3} 3^{1/6} \pi \lambda^{1/3}}$ as well as Stirling's formula, we obtain
 \begin{align*}
    \upsilon_{j,\lambda} &= |P_{j,2\lambda+2}( \cos( \theta_{j, \lambda}) )| \\
    & \leq \left(\frac{\sin \theta_{j,d}}{\theta_{j,d}} \right)^{-\lambda-\frac12} 2^\lambda \lambda! J_\lambda\left( \theta_{j,d}(j+\lambda+1/2) \right) (\theta_{j,d}(j+\lambda+1/2))^{-\lambda} \\ 
    & \lesssim \left[\frac{(j+\lambda-2)(j+\lambda-1)}{(j-1)(j+2\lambda-2)} \right]^{\lambda + 1/2} 2^\lambda \lambda! J_\lambda( \sqrt{\lambda (\lambda-1)})  \lambda^{-\lambda} \\
    &\lesssim \left[\frac{(j+\lambda-2)(j+\lambda-1)}{(j-1)(j+2\lambda-2)} \right]^{\lambda + 1/2} \lambda^{-1/3-\lambda} 2^\lambda \lambda! \\
    & \lesssim (e/2)^{-\lambda} \lambda^{1/6} \left[1 + \frac{\lambda(\lambda-1)}{(j-1)(j+2\lambda-2)} \right]^{\lambda + 1/2}  
\end{align*}




% If we denote $C_d = \min_z J_{d/2}(z)$, and since $P_{j,\lambda}(z) > 0$ for $z > z_{j,d}$, 
% we thus have 
% \begin{align*}
%     \upsilon_{j,d} &\leq \frac{j! (d-3)!}{(j+d-3)!} \left(\frac{\sqrt{1-z_{j,d}^2}}{2}\right)^{-d/2} \left( (j+d/2)^{-d/2} \frac{\Gamma(j+d/2)}{j!} \sqrt{1-z_{j,d}^2}^{-1/2} C_{d} + O(j^{-3/2}) \right) \\
%     & = O( j^{-3d/4})~.
% \end{align*}

%
\par\noindent\rule{\textwidth}{.5pt}
\rule[.8\baselineskip]{\textwidth}{.5pt}

\clearpage


Appendix C


\begin{fact}[{\cite[Equation (2.8)]{BAI20071}}]
\label{fact:baiintegral}
\begin{equation}
\label{eq:baiintegral}
    {P}_{j, \lambda}(\cos \theta) = \frac{\Gamma(j+1) \Gamma(\lambda+1/2)}{\Gamma(2\lambda) \Gamma(\lambda+j+1/2)} \theta 2^{\lambda-1/2} \frac{1}{2\pi i} \oint_\Gamma e^{(\lambda+j)\theta z} R^{-1} \left( e^{\theta z} - \cos \theta + R e^{\theta z/2}\right)^{-\lambda+1/2} dz~,
%    \bar{P}_{j, \lambda}(\cos \theta) = \frac{\Gamma(2\lambda+j) \Gamma(\lambda+1/2)}{\Gamma(2\lambda) \Gamma(\lambda+j+1/2)} \left(\frac{\theta}{2} \right)^{-\lambda+1/2} \frac{1}{2\pi i} \oint_\Gamma h(z, \theta) e^{(\lambda+j)\theta z} (z + \sqrt{z^2 +1})^{-\lambda+1/2} \frac{dz}{\sqrt{z^2+1}}~,
\end{equation}    
where $\Gamma$ is a Hankel-type loop, starting from $-\infty$, encircling both $z=\pm i$ counterclockwise, and ending at $-\infty$; and 
% $$h(z, \theta) = \left( \frac{R}{\theta \sqrt{1+z^2}} \right)^{-1}\left( \frac{(e^{\theta z /2}-e^{-\theta z/2}+ R) (e^{\theta z /2}+e^{-\theta z/2}+ R)}{2\theta (z + \sqrt{z^2+1})}\right)^{-\lambda+1/2}~,$$
% with 
$$R = \left( e^{\theta z} + e^{-\theta z} - 2\cos \theta\right)^{1/2}~.$$

\end{fact}

% \begin{theorem}[{\cite[Corollary 1]{frenzen_wong_1985}}]
% \label{theo:wong_greatbound}
%    Let $u = j + \lambda$. Then, uniformly over $0\leq \theta \leq \pi/2$, 
%    \begin{equation}
%    \label{eq:frenz}
%        \bar{P}_{j,\lambda}(\cos \theta) = \frac{\Gamma(2\lambda+j) \Gamma(\lambda+1/2)}{\Gamma(2\lambda)\Gamma(j+1)(2u)^{\lambda-1/2}}\frac{ \theta^{1/2}}{ \sin(\theta)^\lambda }\cdot \left(J_{\lambda-1/2}(u\theta) + A_1(\theta)\frac{J_{\lambda+1/2}(u\theta)}{u} +\varepsilon(j, \lambda,\theta) \right)~,
%    \end{equation}
% where $J_\nu(z)$ is the Bessel function of the first kind, 
%    $$A_1(\theta) = (\lambda(\lambda-1))\frac{1-\theta \cot(\theta)}{2\theta}$$
%    and
%    $$|\varepsilon| \leq \frac{\theta^{\lambda+3/2}}{u^{5/2 - \lambda}}~.$$
% \end{theorem}


% \begin{theorem}[{\cite[\href{https://dlmf.nist.gov/18.15}{Eq (18.15)}]{NIST:DLMF}}]
% \label{thm:gegenbessel}
% Let $u = j + \lambda$. Then, for large $j$, 
% \begin{equation}
%       \bar{P}_{j,\lambda}(\cos \theta) = \frac{\Gamma(2\lambda+j) \Gamma(\lambda+1/2)}{\Gamma(2\lambda) \sqrt{2}u^{\lambda-1/2}j! 2^\lambda} \frac{ \theta^{1/2}}{ \sin(\theta)^\lambda } \left(J_{\lambda-1/2}(u\theta) + \varepsilon(u, \theta) \right)~,
% \end{equation}
% where $J_\nu(z)$ is the Bessel function of the first kind, and
% \begin{equation}
%     \varepsilon(u, \theta) = \theta^{1/2} O_\lambda(u^{-3/2})
% \end{equation}
% for $\theta = \Omega(u^{-1})$. 
% \end{theorem} 


%from \cite{dunster2023uniform}. Concretely, we have
% \begin{proposition}[{\cite[Eq (4.4)]{dunster2023uniform}}]
% \label{prop:gegenbessel}
% Let $u = \lambda + j$ be sufficiently large. Then
%     \begin{equation}
%         \bar{P}_{j,\lambda}(\cos \theta) =  \frac{\Gamma(2\lambda+j) \Gamma(\lambda+1/2)}{\Gamma(2\lambda) \sqrt{2}u^{\lambda-1/2}j! 2^\lambda}\frac{\theta^{1/2}}{\sin(\theta)^\lambda}\cdot \left\{J_{\lambda-1/2}(u \theta) \hat{A}(u, \theta) - J_{\lambda+1/2}(u\theta) \hat{B}(u, \theta) \right\}~,
%     \end{equation}
%     %     \begin{equation}
%     %     \bar{P}_{j,\lambda}(\cos \theta) = \sqrt{\frac{u \pi}{2}} \frac{\Gamma(\frac12 (u + \lambda)) }{\Gamma(\frac12 (u - \lambda) + 1) \Gamma(\lambda) } \frac{\theta^{1/2}}{\sin(\theta)^\lambda}\cdot \left\{J_{\lambda-1/2}(u \theta) \hat{A}(u, \theta) - J_{\lambda+1/2}(u\theta) \hat{B}(u, \theta) \right\}~,
%     % \end{equation}
%     where $J_\nu(z)$ is the Bessel function of the first kind, and $\hat{A}$, $\hat{B}$ admit asymptotic expansions of the form 
%     \begin{align*}
%         \hat{A}(u, \theta) & \simeq \sum_{k=0}^\infty \frac{A_k(\theta,\lambda)}{u^{2k}} ~,\\
%         \hat{B}(u, \theta) & \simeq \sum_{k=0}^\infty \frac{B_k(\theta,\lambda)}{u^{2k+1}} ~,
%     \end{align*}
%     where these expansions are uniformly valid in $0 \leq \theta \leq \pi/2$, with coefficients $A_k(\theta)$, $B_k(\theta)$ with explicit dependencies on $\lambda,\theta$. 
% \end{proposition}

Assume $j = \Theta(\lambda^\alpha)$, with $\alpha > 3/2$. 
We are interested in the above representation for  $\bar{\theta} = \arccos(z_{j-1,d+2})$. 
From Theorem \ref{thm:gegenroots_3}, we have 
$z_{j-1,d+2}^2 \geq 1 - d_{j,\lambda}/(2c_{j,\lambda})$, and thus 
$$\bar{\theta}^2 \lesssim \frac{d_{j,\lambda}}{2c_{j,\lambda}} \simeq \frac{32\lambda^2 j^4 }{16 j^6} = \frac{2\lambda^2}{j^2}~,$$
so $\bar{\theta} = O(\lambda/j)$. 
Combining this upper bound with the lower bound obtained from Fact \ref{fact:largestroot} we have $\bar{\theta} = \Theta(\lambda/j)$. 
%and in particular $\bar{\theta} = \Omega(u^{-1})$. 
In this regime, we obtain the following control 
thanks to Fact \ref{fact:baiintegral}:
\begin{lemma}
For $\theta = \Theta(\lambda/j)$ and $j = \omega(\lambda^{3/2})$, we have    
\begin{align}
  P_{j,d}(\cos \theta) & = 
\end{align}
\end{lemma}
\begin{proof}
We bound the integral in (\ref{eq:baiintegral}) by separating the tail and the region near the poles $z = \pm i$. 
We consider $0<\delta<1$ and 
\begin{align}
\Gamma = & \{ z = t \pm i \delta ; -\infty \leq t \leq -\delta\} \cup \{ z=\pm \delta + i t; -1-\delta \leq t \leq 1+\delta \} \nonumber \\
& \cup \{ z= t \pm i (1+\delta); -\delta \leq t \leq \delta\} \setminus \{z=-\delta + i t; -\delta \leq t \leq \delta\}~.    
\end{align}
Let $K>0$. We split the integration path as $\Gamma = \Gamma_{>K} \cup \Gamma_{\delta, K} \cup \Gamma_{\leq \delta}$, 
% integral as 
%     \begin{align}
%         \oint_\Gamma e^{(\lambda+j)\theta z} R^{-1} \left( e^{\theta z} - \cos \theta + R e^{\theta z/2}\right)^{-\lambda+1/2} dz &= \int_{\Gamma_{>R}} dz + \int_{\Gamma_{\leq R}} dz ~,
%     \end{align}
where $\Gamma_{>K} = \{ z \in \Gamma; |z| > K\}$, $\Gamma_{\leq \delta} = \{ z \in \Gamma; |\text{Re}(z)|\leq \delta\}$ and $\Gamma_{\delta, K} = \Gamma \setminus ( \Gamma_{>K} \cup \Gamma_{\leq \delta})$.  
Assume $K > \ln 5$. Then we obtain
\begin{align}
    |R| &= \left| e^{\theta z} + e^{-\theta z} - 2\cos \theta \right|^{1/2} \nonumber \\
    & \geq \left( e^{ -\theta \text{Re}(z)} - 2\cos \theta - e^{ \theta \text{Re}(z)}\right)^{1/2}_+ \nonumber \\
    & \geq \frac12 e^{ -\theta \text{Re}(z)}~. 
\end{align}

On the other hand, we have 
\begin{align}
    |R e^{\theta z/2}| &= \left| 1 + e^{\theta z} ( e^{\theta z} - 2\cos \theta) \right|^{1/2} \nonumber \\
    &\geq 1 - e^{\theta \text{Re}(z)}~,
\end{align}
%with $|b| \leq 2 e^{\theta \text{Re}(z)}$. 
As a result, whenever $K > \theta^{-1} \log\left(\frac{2}{1-\cos \theta} \right)$, it holds
\begin{align}
    \left| e^{\theta z} - \cos \theta + R e^{\theta z/2} \right| \geq 1 - \cos \theta - 2 e^{\theta \text{Re}(z)} \nonumber \\
    \geq \frac12 ( 1 - \cos \theta)~. 
\end{align}
By assembling these bounds, we therefore obtain
\begin{align}
\label{eq:bp1}
    \left|\int_{\Gamma_{>K}} e^{(\lambda+j)\theta z} R^{-1} \left( e^{\theta z} - \cos \theta + R e^{\theta z/2}\right)^{-\lambda+1/2} dz \right| & \leq 4 (1-\cos(\theta))^{-\lambda+1/2} \int_{\Gamma_{>K}} e^{(\lambda+j+1) \theta \text{Re}(z)} dz \nonumber \\
    & \leq 8 (1-\cos(\theta))^{-\lambda+1/2} e^{-K (\lambda+j+1)} (\lambda+j+1)^{-1} ~.
\end{align}
Since $\theta = \Theta(\lambda/j)$, picking $K \simeq \theta^{-1} \log\left(\frac{2}{1-\cos \theta} \right)$ we obtain from (\ref{eq:bp1}) 
\begin{equation}
\label{eq:vip1}
    \left|\int_{\Gamma_{>K}} e^{(\lambda+j)\theta z} R^{-1} \left( e^{\theta z} - \cos \theta + R e^{\theta z/2}\right)^{-\lambda+1/2} dz \right| \simeq \left(\frac{\lambda}{j}\right)^{-2 \lambda +1 + 2j (j+\lambda+1)/\lambda} ~. 
\end{equation}

Let us now focus on $\Gamma_{\leq \delta}$. 
In the slices $\{ z = \pm \delta + it ;  |t| \leq (1+\delta) \}$ and $\{ z = t \pm i (1+\delta);  |t| \leq \delta\}$ we have, by exploiting the monotonicity of $t \mapsto |R(z_t)|^2$, 
\begin{equation}
\label{eq:mini1}
2 \theta \delta\sin \theta \delta  \leq |R|^2 \leq  2(1-\cos \theta) + \theta^2 \delta^2~. 
\end{equation}
% Similarly, in $\{ z = t \pm i (1+\delta);  |t| \leq \delta\}$ we have 
% \begin{align}
% \label{eq:mini2}
%     \left| e^{\theta z} - \cos \theta + R e^{\theta z/2} \right| & \geq \sin (\theta \text{Im}(z)) - |R| e^{\theta \text{Re}(z)/2}  \nonumber \\
%     &\gtrsim \theta(1+\delta) - \theta^2(1+\delta^2)~,
% \end{align}
% whereas in $\{ z = \delta + i t;  |t| \leq 1+ \delta \}$ we have 
% \begin{align}
% \label{eq:mini3}
%     \left| e^{\theta z} - \cos \theta + R e^{\theta z/2} \right| & \geq e^{\delta \theta} - \cos \theta - |R| e^{\delta \theta /2} \nonumber \\
%     & \gtrsim \theta \delta - \theta^2(3/2 + \delta^2)~,
% \end{align}
% and finally in $\{ z = -\delta + i t; \delta \leq |t| \leq 1+\delta \}$ we have 
% \begin{align}
% \label{eq:mini4}
%     \left| e^{\theta z} - \cos \theta + R e^{\theta z/2} \right| & \geq | e^{\theta z} - \cos \theta| - |R| e^{-\delta \theta /2} \nonumber \\
%     & \gtrsim \sqrt{2} \theta \delta - \theta^2(1 + \delta^2)~.
% \end{align}
Similarly, we have 
\begin{align}
\label{eq:mini2}
    \left| e^{\theta z} - \cos \theta + R e^{\theta z/2} \right|  
    &\geq \theta | z + \sqrt{1 + z^2} | - o(\theta \delta) \nonumber \\
    & \gtrsim \theta \sqrt{\delta}~.
    \end{align}

Assembling (\ref{eq:mini1}) and (\ref{eq:mini2}) we obtain 
\begin{align}
\label{eq:vip2}
 \left|   \int_{\Gamma_{\leq \delta}} e^{(\lambda+j)\theta z} R^{-1} \left( e^{\theta z} - \cos \theta + R e^{\theta z/2}\right)^{-\lambda+1/2} dz \right| & \leq (2 + 4\delta) e^{(\lambda+j)\theta \delta} \theta^{-1}\delta^{-1} (\theta \sqrt{\delta})^{-\lambda + 1/2} \nonumber \\
 & \lesssim \frac{1+\delta}{\delta} e^{\lambda \delta} j \lambda^{-1} \left(\frac{\lambda \sqrt{\delta}}{j} \right)^{-\lambda + 1/2}~. 
\end{align}



Finally, we focus on $\Gamma_{\delta, K}$. We verify that $|R|$ takes its minimal value at $z=-\delta \pm i \delta$, thus $|R| \gtrsim \theta \delta$ in $\Gamma_{\delta, K}$. Similarly, $z\mapsto \left|e^{\theta z} - \cos \theta + R e^{\theta z /2} \right|$ is also monotonic for $z = t \pm i \delta$, $t \in [-K, -\delta)$, thus 
$$\left|e^{\theta z} - \cos \theta + R e^{\theta z /2} \right| \geq 1-\cos(\theta)\gtrsim \theta^2/2~.$$
Therefore, 
\begin{align}
\label{eq:vip3}
    \left| \int_{\Gamma_{\delta, K}} e^{(\lambda+j)\theta z} R^{-1} \left( e^{\theta z} - \cos \theta + R e^{\theta z/2}\right)^{-\lambda+1/2} dz \right| & \lesssim \delta^{-1} \theta^{-2\lambda} 2^{\lambda} \int_\delta^K e^{-(\lambda+j)\theta t} dt \nonumber \\
    &\lesssim \theta^{-2\lambda} 2^{\lambda} e^{-\lambda \delta} \lambda^{-1} \nonumber \\
    & \lesssim \lambda^{-2\lambda} j^{2\lambda} 2^{\lambda} e^{-\lambda \delta} \lambda^{-1} 
\end{align}
Setting $\delta = \frac12$, 


\end{proof}

% \begin{lemma}
% For $\theta = \Theta(\lambda/j)$ and $j = \omega(\lambda^{3/2})$, we have    
% \begin{align}
%     \bar{P}_{j,\lambda}(\cos \theta) &= \frac{\Gamma(2\lambda+j) \Gamma(\lambda+1/2)}{\Gamma(2\lambda) \Gamma(\lambda+j+1/2)} P_{\lambda, \lambda, j}(\cos \theta) \\
%     &= \frac{\Gamma(2\lambda+j) \Gamma(\lambda+1/2)}{\Gamma(2\lambda) \Gamma(\lambda+j+1/2)} \left(\frac{\theta}{2} \right)^{-\lambda+1/2} \chi(\theta, j, \lambda)~,
% \end{align}
% where 
% $\chi(\theta, j, \lambda) = O(\lambda^{-1/3})$.
% \end{lemma}
% \begin{proof}
%     \begin{align}
%   %      P_{j,d}(\theta) & = \frac{\Gamma(j+1)\Gamma(2\lambda)}{\Gamma(j+2\lambda)} \frac{\Gamma(2\lambda+j) \Gamma(\lambda+1/2)}{\Gamma(2\lambda) \Gamma(\lambda+j+1/2)} \left(\frac{\theta}{2} \right)^{-\lambda+1/2} \chi(\theta, j, \lambda)
%         P_{j,d}(\theta) & = \frac{\Gamma(j+1) \Gamma(\lambda+1/2)}{\Gamma(2\lambda) \Gamma(\lambda+j+1/2)} \left(\frac{\theta}{2} \right)^{-\lambda+1/2} \chi(\theta, j, \lambda) ~,
%     \end{align}
%     so 
%     \begin{align}
%         \log P_{j,d}(\cos \theta) & \simeq -\lambda ( \log \lambda - \log j + C) + \log \chi(j,\lambda) + j \log j + \lambda \log \lambda - (2\lambda) \log \lambda - (\lambda + j) \log j \\
%         & \simeq -C \lambda + \log \chi(j, \lambda) -2 \lambda \log \lambda
%     \end{align}
% \end{proof}


% \begin{claim}
% \label{claim:besselgegen_as}
% For $\theta = \Theta(\lambda/j)$ and $j = \omega(\lambda)$, we have %, with $u = \lambda+j$, 
% \begin{equation}
% \label{eq:banan}
%     % \bar{P}_{j,\lambda}(\cos \theta) = \frac{\Gamma(2\lambda+j) \Gamma(\lambda+1/2)}{\Gamma(2\lambda) \sqrt{2}u^{\lambda-1/2}j! 2^\lambda} \frac{ \theta^{1/2}}{ \sin(\theta)^\lambda } \cdot \left\{ J_{\lambda-1/2}(u \theta)\left(1 + O\left(\frac{\lambda^3}{u^2}\right) \right)\right\}~. 
%         \bar{P}_{j,\lambda}(\cos \theta) = \frac{\Gamma(2\lambda+j) \Gamma(\lambda+1/2)}{\Gamma(2\lambda) (2j)^{\lambda-1/2}\Gamma(j+1) } \frac{ \theta^{1/2}}{ \sin(\theta)^\lambda } \cdot \left\{ J_{\lambda}(\lambda)\left(1 + O\left(\frac{\lambda^3}{j^2}\right) \right)\right\}~. 
% \end{equation}
% \end{claim}


% \begin{claim}
% \label{claim:besselgegen}
% For $\theta = \Theta(\lambda/j)$, we have, with $\rho = \lambda+j$, 
% \begin{equation}
% \label{eq:banan}
%     \bar{P}_{j,\lambda}(\cos \theta) = \sqrt{\frac{u \pi}{2}} \frac{\Gamma(\frac12 (u + \lambda)) \theta^{1/2}}{\Gamma(\frac12 (u - \lambda) + 1) \Gamma(\lambda) \sin(\theta)^\lambda} \cdot \left\{ 
% J_{\lambda-1/2}(u \theta)\left(1 + O\left(\frac{\lambda^3}{u^2}\right) \right)\right\}~. 
% \end{equation}
% \end{claim}

%[Vanishing Error terms when $j = \omega(\lambda^{3/2})$]
Since $\alpha > 3/2$ and $j = \Theta(\lambda^\alpha)$, the error term in (\ref{eq:banan}) is vanishing. % when $\theta=O(\lambda/j)$. 
%Let us show how to leverage Claim \ref{claim:besselgegen} when $j = \omega(\lambda^{3/2})$. 
%From (\ref{eq:banan}), for $\theta=\Theta(\lambda/j)$ 
We have %such that $\cos \theta = z_{j-1,d+2}$, we have 
\begin{align}
    |P_{j,\lambda}(\cos \theta) | & = \bar{P}_{j,\lambda}(1)^{-1} \frac{\Gamma(2\lambda+j) \Gamma(\lambda+1/2)}{\Gamma(2\lambda) \sqrt{2}u^{\lambda-1/2}\Gamma(j+1) 2^\lambda} \frac{ \theta^{1/2}}{ \sin(\theta)^\lambda } \cdot \left\{ |J_{\lambda-1/2}(u \theta)|\left(1 + O\left(\frac{\lambda^3}{u^2}\right) \right)\right\} \nonumber \\
& =  \frac{\Gamma(\lambda+1/2)}{ \sqrt{2}u^{\lambda-1/2}2^\lambda} \frac{ \theta^{1/2}}{ \sin(\theta)^\lambda } \cdot \left\{ |J_{\lambda-1/2}(u \theta)|\left(1 + O\left(\frac{\lambda^3}{u^2}\right) \right)\right\}  \nonumber \\
& \simeq \frac{\Gamma(\lambda+1/2)}{ j^{\lambda-1/2}2^\lambda} O(j^{\lambda-1/2}) O(\lambda^{1/2-\lambda}) \cdot \left\{ O(\lambda^{-1/3}) + O\left(\frac{\lambda^{3-1/3}}{j^2}\right) \right\} \nonumber \\
& = O\left( e^{1/2-\lambda} 2^{-\lambda} \lambda^{1/6}\right) \nonumber \\
% & = \frac{\Gamma(j+1)\Gamma(2\lambda)}{\Gamma(j+2\lambda)}  \frac{\Gamma(\lambda+j/2)) }{\Gamma(j/2+1) \Gamma(\lambda) } O(j^\lambda) O(\lambda^{1/2-\lambda}) \cdot \left( O(\lambda^{-1/3}) + O\left(\frac{\lambda^{3-1/3}}{j^2}\right) \right) \nonumber \\
% &\simeq \frac{j^{1/2} j^j e^{-j} \lambda^{1/2} (2\lambda)^{2\lambda-1} e^{-2\lambda+1} j^{1/2} (j/2)^{\lambda+j/2-1} e^{1-j/2-\lambda} }{j^{1/2} j^{j+2\lambda-1} e^{1-j-2\lambda}  j^{1/2} (j/2)^{j/2} e^{-j/2} \lambda^{1/2} \lambda^{\lambda-1} e^{1-\lambda}} O(j^\lambda) O(\lambda^{1/6-\lambda}) \nonumber \\
% & \simeq j^{3j/2+\lambda-3j/2-2\lambda+\lambda} e^{-j-2\lambda+1+1-j/2-\lambda -1+j+2\lambda +j/2-1+\lambda} \lambda^{1/2+2\lambda-1-1/2-\lambda+1+1/6-\lambda} 2^{2\lambda-1-\lambda-j/2+1+j/2} \\
%&= O\left( (e/2)^{-\lambda} \lambda^{1/6} \right) \nonumber \\
&= O\left( (e/2)^{-\lambda} \right)~,
\end{align}
%\cite[Section 8.32, pg 234]{watson1922treatise}
where we have used Cauchy's asymptotic formula $J_\lambda(\lambda) \simeq \frac{\Gamma(1/3)}{2^{2/3} 3^{1/6} \pi \lambda^{1/3}}$ \cite[Section 8.2, pg 231]{watson1922treatise}, as well as Stirling's formula 
$\Gamma(x+1) \simeq \sqrt{2 \pi x} (x/e)^x$.


% \begin{proof}[Proof of Claim \ref{claim:besselgegen_as}]
% The control is a direct application of Theorem \ref{theo:wong_greatbound}. 
% In the regime $j = \omega(\lambda)$, we have
% $$\theta \cot\theta = 1 + \frac{1}{6} \frac{\lambda^2}{j^2}~,$$
% and thus
% $$A_1(\theta) = \Theta\left(\frac{\lambda^3}{j^2}\right)~.$$
% Similarly, 
% \begin{align*}
%     |\varepsilon| &\lesssim \frac{\lambda^{\lambda+3/2}}{j^{\lambda+3/2+5/2-\lambda}}
% \end{align*}
% From (\ref{eq:frenz}), we have

% \end{proof}

%\begin{proof}

%     %We denote $P_j = P_{j,d}$ by convenience.
%     Let $z_{j,d}$ be the largest root of $P_{j,d}$, and consider a given $t \leq z_{j,d}$. 
%     We consider $t_0 := \sup\{ z ; \, P_{j,d}(z)=0; z \leq t\}$ the largest root of $P_{j,d}$ below $t$. 
%     We argue inductively over $j$. 
%     Consider the Taylor decomposition of $P_{j,d}$ centered at $t_0$:
%     \begin{align}
%         P_{j,d}(t) &= \sum_{i_0=1}^j P_{j,d}^{(i_0)}(t_0) \frac{(t - t_0)^{i_0}}{i_0!}~. 
%     \end{align}
%     Now, we consider each of the terms $P_{j,d}^{(i_0)}(t_0)$ and apply again the Taylor decomposition using the same strategy. For each $i_0=1\ldots j$, consider $t_{0,i_0} = \sup\{ z; P_{j-i_0,d+2i_0}(z)=0; z \leq t_0\}$, and 
%     $$P_{j,d}^{(i_0)}(t_0) = K_{j,i_0,d} P_{j-i_0,d+2i_0}(t_0) = K_{j,i_0,d} \sum_{i_1=1}^{j-i_0} P_{j-i_0,d+2i_0}^{(i_1)}(t_{0,i_0}) \frac{(t_0-t_{0,i})^{i_1}}{i_1!}~,$$
%     where we have from Fact \ref{fact:deriv} that 
% %    $$K_{j, i, d} = \prod_{l=1}^i \frac{(j-i+1)(j-i+1 + d+2(i-1)-2)}{d+2(i-1) - 1}~. $$
%     $$K_{j, i, d} = \prod_{l=1}^i \frac{(j-i+1)(j+i+d-3)}{d+2i-3}~. $$
% Applying this Taylor decomposition recursively leads to 
% \begin{align}
%     P_{j,d}(t) &= \sum_{l=1}^{j-1} \sum_{i_0+\ldots i_l=j} \chi_{i_0, \ldots, i_l,d} \prod_{k=1}^l (t_{p(i;k-1)} - t_{p(i;k)} )^{i_k}
% \end{align}
%\end{proof}


% {\color{blue}
% *** *** *** 

% Consider a generic rotationally invariant probability measure $\mu$.  That is, for $x \sim U(S^{d-1})$ and $r \sim \nu$ independent of $x$ for some one-dimensional measure $\nu$ supported on $[0, \infty)$, we say that $\tilde{x}$ is distributed according to $\mu$ if we have the distributional equality $\tilde{x} = rx$.

% Let $\{p_i\}_{i=1}^\infty$ be the family of orthonormal polynomials with respect to $\nu$.  Then the family of functions $\{p_i \otimes P_j\}_{i,j=1}^\infty$ is orthonormal with respect to the product measure $\nu \times \omega$ where $\omega$ is the one dimensional marginal of the uniform spherical distribution, so we may decompose our link function as
% %
% \begin{align}
%     \phi(r\langle x, \theta \rangle) = \sum_{ij} \alpha_{ij} p_i(r) P_j(\langle x, \theta \rangle)
% \end{align}
% %
% For each data $x, r \in S^{d-1} \times \R_{\geq 0}$, we consider the square loss 
% %
% \begin{align}
%     \ell(\theta, x, r) := \left(\phi(r\langle x, \theta \rangle) - \phi(r\langle x, \theta^* \rangle) \right)^2,
% \end{align}
% %
% and this implies that the overall population loss loss can be written as:
% %
% \begin{align*}
%     L(\theta) &= E_{r, x}\left[\left(\phi(r\langle x, \theta \rangle) - \phi(r\langle x, \theta^* \rangle) \right)^2 \right] \\
%     &= C - 2 E_{r, x}\left[\phi(r\langle x, \theta \rangle) \phi(r\langle x, \theta^* \rangle)\right] \\
%     &= C - 2 \sum_{j} \left( \sum_{i} \alpha_{ij}^2 \right) P_{j}(\langle \theta, \theta^* \rangle)
% \end{align*}
% %
% And therefore the loss can be expressed solely as a function of the correlation $m_\theta = \langle \theta, \theta^* \rangle$. This is the reason why we will also refer to $m = m_\theta$ as the \textit{summary statistics} of the problem. Denote the corresponding function $\mathcal{L} : \R \to \R$ such that  $\mathcal{L}(m_\theta) = L(\theta)$, we have
% %
% \begin{align}
%     L'(m) = -2 \sum_{j} \left( \sum_{i} \alpha_{ij}^2 \right) P_{j}'(m)
% \end{align}


% For a fixed degree $j$, we have that $P_{j,d}'$ has $(a, j/\sqrt{d}, j-1)$ growth, with $a>0$ independent of $d$. 

% [TODO]

% Additionally, we may ask what conditions on the link function $\phi$ enable a good rate.

% Suppose $\phi(t) = \sum_k \beta_k t^k$ {\color{red} careful when we write this. is this in $L^2(\mathbb{R}, d\nu)$ ?}  Then we have

% \begin{align}
%     \alpha_{ij} &= E\left[\phi(r\langle x, \theta\rangle p_i(r) P_j(\langle x, \theta\rangle) \right] \\
%     &= \sum_k \beta_k \langle r^k, p_i(r) \rangle \langle \langle x, \theta \rangle^k, P_j(\langle x, \theta \rangle) \rangle\\
%     &= \sum_k \beta_k A_{ki} B_{kj}\\
%     &= \left[A^T D B \right]_{ij}
% \end{align}

% where $A, B$ collect correlation terms of orthogonal polynomials with monomials, and $D = diag(\beta)$.

% Therefore,

% \begin{align}
%     \sum_{i} \alpha_{ij}^2 = \|A^TDBe_j\|^2
% \end{align}

% }

%Fix $R>0$. 
% Since $\rho$ has sub-exponential tails and $\phi \in L^2(\mathbb{R}, \eta)$, we have that $\forall \epsilon>0$, there exists $R>0$ such that, uniformly over $j$ {\color{red} TODO justify this }. 
% {\color{red} instead, we first need to show that $\sum_j j^2 \beta_j < \infty$. Then focus on the bulk $j < J$, and we bound the tail $j > J$ separately}. 
% \begin{align}
%     \int_0^R \rho(dr) \bar{\alpha}^2_{j,r,d} &\geq (1-\epsilon) \beta_{j,d}~.
% %    \beta_{j,d} &= \mathbb{E}_{\rho} [\bar{\alpha}_{j,r,d}^2]
% \end{align}
% It follows that 
% \begin{align}
%    \mathbb{E}_\rho \left[r^{-1}  \bar{\alpha}_{j,d,r}^2 \right] &\geq \int_0^R r^{-1}  \bar{\alpha}_{j,d,r}^2 \rho(dr) \nonumber \\
%    &\geq R^{-1} \int_0^R   \bar{\alpha}_{j,d,r}^2 \rho(dr) \nonumber \\
%    &\geq R^{-1} (1-\epsilon) \beta_{j,d}~,
% %   &\geq \frac{1}{N(j,d)  \| P_{j,d}\|^{4}} 
% %   \mathbb{E}_\rho \left[r^{-1}  \bar{\alpha}_{j,d,r}^2 \right] &= \frac{1}{N(j,d)  \| P_{j,d}\|^{4}} \int_{0}^\infty \rho(dr) r^{-1} [\langle \phi^{(r)}, P_j \rangle_{u_d}]^2 \nonumber \\
%  %  &\geq \frac{1}{N(j,d)  \| P_{j,d}\|^{4}} 
% \end{align}