\documentclass{article}
%%% Margins
\usepackage[%a4paper,
		%width=150mm,
		top=25mm,
		bottom=25mm,
            right=25mm,
            left=25mm,
            % Offset to bind pages
            bindingoffset=6mm,  
            % To make room for the header
            headheight=20mm,    
           ]{geometry}
           
\usepackage{amsmath,amsfonts}
\usepackage{graphicx}
\usepackage{hyperref}
\usepackage{caption}
\usepackage{subcaption}
\usepackage{todonotes}

% commandes todo
\newcommand\todol[1]{\todo[inline]{#1}}
\newcommand\todolc[2]{\todo[inline,color=#1]{#2}}
\newcommand\todoalex[1]{\todo[inline,color=blue!30]{Alexandre : #1}}
\newcommand\todoalain[1]{\todo[inline,color=magenta!20]{Alain : #1}}
\newcommand\todocarole[1]{\todo[inline,color=green!40]{Carole : #1}}

% Mathematical notations
\newcommand{\R}{\mathbb{R}}

%\title{\Huge{I see stars and flowers everywhere}}
\title{HBS Tilings extended:\\ 
State of the art and novel observations}
\author{Carole Porrier}
%\date{March 2023}

\begin{document}

\maketitle

\begin{abstract}
    Penrose tilings are the most famous aperiodic tilings, and they have been studied extensively. In particular, patterns composed with hexagons ($H$), boats ($B$) and stars ($S$) were soon exhibited and many physicists published on what they later called $HBS$ tilings, but no article or book combines all we know about them. This work is done here, before introducing new decorations and properties including explicit substitutions. For the latter, the star comes in three versions so we have 5 prototiles in what we call the Star tileset. Yet this set yields exactly the strict $HBS$ tilings formed using 3 tiles decorated with either the usual decorations (arrows) or Ammann bar markings for instance. Another new tileset called Gemstones is also presented, derived from the Star tileset.
\end{abstract}



\section{Introduction}

Penrose tilings are the most famous aperiodic tilings since Gardner described them \cite{gardner1977,gardner1988}, and they have been studied extensively \cite{grunbaum1987,Senechal1995,baake_grimm_2013}.
It is thus surprising that one can still discover something new about them, even if it is not completely new.
Indeed, patterns composed with hexagons, boats and stars (HBS tiles, Fig. \ref{fig:hbs-tiles}) were soon exhibited and aroused interest among physicists but not that much among mathematicians -- who wanted less tiles than in Penrose's tilesets.
% Figure environment removed

As serendipity works, we rediscovered HBS shapes but with different decorations and forcing rules, while working on a combinatorial optimization problem on graphs defined by kites-and-darts Penrose tilings (type P2), as described in \cite{porrier2020}.
Indeed, the stars (composed with five darts) are an optimal pattern for us and we were looking at paths that would join stars with the minimum possible distance.
We noticed that only three shapes appeared, always with the same decorations inside (small kites and darts as in Figure \ref{star-tileset-kd}) which could be replaced with the usual Ammann segments for HBS tiles.
Additionally, we distinguished three types of P2-stars depending on their surrounding, and as we labeled the HBS-vertices according to these types it appeared that most vertices of a given shape always had the same label, with an exception for the star.
We thus called \textit{Star tileset} the five shapes with labeled vertices, before knowing about the existence of HBS tiles -- even though they first appeared in Henley's 1986 paper \cite{Henley1986}.
% Figure environment removed

First called ``star $\Sigma$, diamond $\Delta$ and hexagon $H$'' by Olamy and Kléman \cite{OK1989},
the prototiles of what they called ``Henley's dense tiling'' appear in Wikipedia's \href{https://en.wikipedia.org/wiki/List_of_aperiodic_sets_of_tiles}{list of aperiodic sets of tiles} as ``Starfish, ivy leaf and hex tiles'', which are the names given to the prototiles by Lord \cite{Lord1991}.
In the following years, the name HBS (Hexagon-Boat-Star) was given to them and since then it has been commonly used.
As we found it difficult to know whether some related researched had already been published and including which information, it seemed a state of the art would be useful to us and might be for others too.
Indeed, HBS tiles often appear in the study of decagonal quasicrystals, on which most research is focused, and Steurer concluded his review on quasicrystals \cite{Steurer2004} saying ``QC structure analysis, however, will certainly contribute to the solution of complex intermetallic phases in general.''.
Since we also had new findings and need to rely on those for an article to come \cite{porrier2023}, it seemed appropriate to publish.
The only place in which I have seen three colors for the vertices of HBS tiles is on \href{http://kociemba.org/themen/penrose/tiling2.html}{Herbert Kociemba's website}, but with a different construction and no details\footnote{I contacted him and he told me he had no knowledge of anyone else doing the same.}. 
His tiles are decorated in the style of \textit{Girih tiles}, like the decorative Islamic geometric patterns used in architecture.
Some researchers have indeed investigated how such art could be designed long before we knew about quasiperiodic structures that look alike \cite{Lu-Steinhardt2007,Ajlouni2012}.

We start with a few recalls on Penrose tilings and useful definitions, before describing the known constructions and structural properties of strict HBS tilings. 
We then proceed with the construction and properties of the Star tileset, including how the vertex colors are related to Penrose tilings.
Finally, a new tileset called \textit{Gemstones} is presented, derived from the Star tileset.



%%%%%%%%%%%%%%%%%%%%%%%%%%%%%%%%%%%%%%%%%%%%%%
%%%%%%%%%%%%%%%%%%%%%%%%%%%%%%%%%%%%%%%%%%%%%%
%%%%%%%%%%%%%%%%%%%%%%%%%%%%%%%%%%%%%%%%%%%%%%

\section{Penrose tilings and mutual local derivability}

A \textit{tiling} of $\R^2$ is a countable family of non-empty closed sets $(T_i)_{i\in I}$ called \textit{tiles} such that $\bigcup_{i\in I}T_i=\R^2$ and $\mathring{T_i}\cap\mathring{T_j}=\emptyset$ for all $i\neq j$ in $I$.
The \textit{prototiles} of a tiling are the equivalence classes of its tiles up to congruence (in the present context).
A set of prototiles is called a \textit{tileset}, and oftentimes many tilings can be composed with copies of the same prototiles.
The three types of Penrose tilings, denoted P1, P2 and P3 in \cite{grunbaum1987}, were described by Roger Penrose himself \cite{penrose1974,penrose1978,penrose1979}. 
The corresponding tilesets are shown in Figure \ref{fig:Penrose-tilesets}. 
% Figure environment removed
Uncountably many tilings can be composed with each, but none of them is periodic: they have no translations among their symmetries.
The tiles must be arranged according to specific assembly rules, for instance using notches and bumps to assemble the tiles like puzzle pieces (P1 tiles, Fig. \ref{P1}) or markings on the tiles which can be of different kinds.
For P2, we use two colors which must match on the corners of the tiles (Fig. \ref{P2}), forming a full black or blank disk at each vertex of a tiling. 
Such a marking would not be sufficient for P3 tiles so we use arrows on the edges (Fig. \ref{P3}), which must superimpose exactly (same number of arrows, in the same direction).
This marking is also the most convenient in this article, in relation with HBS tiles.
For the same further purposes, an additional arrow is added on the diagonal of the kite, which corresponds to the single arrow on the edges of rhombuses when a P2 tiling is derived from a P3 tiling.
The sides of rhombuses have length 1, which is also the length of the kite's diagonal. 
Hence for kites and darts the longer side has length 1 and the shorter $\varphi^{-1}$ where $\varphi = (1+\sqrt{5})/2$ is the golden ratio. 
With the matching rules, there are only seven vertex configurations, that is seven ways in which kites and darts can be arranged around a vertex of the tiling. These are given in Figure \ref{vertex-config}.
%
%%% vertex configurations
% Figure environment removed

A vertex configuration is a particular case of \textit{cluster}, i.e. a bounded, connected subset of tiles.
When a cluster consists of (exactly) the tiles of a tiling which intersect a closed ball $B(r)$ with a given radius $r$, it is called an \textit{r-map} or \textit{r-patch}.
The set of all $r$-maps of a tiling is called \textit{r-atlas}.
All Penrose tilings of a given type are \textit{locally isomorphic}, in other words belong to the same \textit{LI class} (local isomorphism class), which means that any cluster which occurs in one of them also appears in all the others. 
Moreover they are \textit{locally characterizable} (they have \textit{local rules}), that is their patches up to a given, finite radius are sufficient to recognize a tiling from a given LI class, with no risk of mistaking it with another tiling.
For instance, P2 tilings are characterized by their 0-atlas (the set of vertex configurations) while the radius for P3 tilings is 1 \cite{FL2022}.
These notions and the following are more detailed in \cite{GBS1994,baake1999guide,baake_grimm_2013}.

Let $T_1, T_2$ be two tilings. If there is a fixed finite radius $r$ such that whenever the $r$-patches of $T_1$ centered in two points $p$ and $q$ are equal up to the translation $t=p-q$, the vertex configurations around $p$ and $q$ in $T_2$ are also equal up to translation $t$, then we say that $T_2$ is \textit{locally derivable} from $T_1$.
If $T_1$ is also locally derivable from $T_2$, then $T_1$ and $T_2$ are \textit{mutually locally derivable} (\textit{MLD}).
This holds for Penrose tilings of different LI classes.
Since they all share the same $r$-patches, any tiling of a Penrose LI class is MLD with a tiling of another Penrose LI class, hence there is a bijection between those two classes.
Penrose tilings are thus all in the same MLD class, along with Robinson triangles and a few other LI classes.
Some cases of local derivability are quite easy to see.
For instance when a tiling $T_1$ is a \textit{composition} of a tiling $T_2$, i.e. each tile in $T_1$ is a union of tiles in $T_2$. 
In that case $T_2$ is a \textit{decomposition} of $T_1$.
For Penrose tilings, we have composition and decomposition rules to transform a tiling of a given type into another tiling of either the same LI class or another.
An \textit{inflation} is a decomposition followed with a $\varphi:1$ scaling, so that the new tiles have the same shapes and sizes as the original ones.
% Figure environment removed


%%%%%%%%%%%%%%%%%%%%%%%%%%%%%%%%%%%%%%%%%%%%%%
%%%%%%%%%%%%%%%%%%%%%%%%%%%%%%%%%%%%%%%%%%%%%%
%%%%%%%%%%%%%%%%%%%%%%%%%%%%%%%%%%%%%%%%%%%%%%

\section{HBS tilings: state of the art}\label{sec:hbs-stateoftheart}

HBS tilings are MLD with Penrose tilings of the three LI classes, so we first recall and illustrate the local mappings between HBS tilings and rhombic Penrose tilings (P3), kite-and-dart Penrose tilings (P2), and pentagon Penrose tilings (P1).
Then their main theoretical properties are recalled.

\subsection{Natural construction of HBS tiles from Penrose rhombs}\label{sub:hbs-P3}

HBS tiles were first mentioned by Henley \cite{Henley1986} as he studied sphere packings based on Penrose tilings.
He noticed that when the Penrose rhombs are marked with arrows as first defined in \cite{penrose1978}, each HBS tile is obtained by composing the rhombs as explained in Figure \ref{fig:hbs-P3}.
Indeed, three of the eight allowed vertex types in P3 tilings are ``poles'' where only double arrows point -- namely those denoted $K$, $Q$ and $S$ by de Bruijn \cite{bruijn1981}.
When the rhombs are not marked, the configurations are still recognizable except for the star: ``false'' stars appear where hexagons meets.
%-- which comes in two versions depending on the implicit orientation of the rhombs.
Yet the construction remains easy: start with hexagons and boats, then the remaining (fat) rhombs form the ``true'' stars of the HBS tiling.
% Figure environment removed
% Figure environment removed

%%%%%%%%%%%%%%%%%%%%%%%%%%%%%%%%%%%%%%%%%%%%%%

% Figure environment removed

\subsection{Simple construction of HBS tiles from Penrose kites and darts}

Though not given explicitly by Henley, his recall of the deflation rules from Penrose rhombs to kites and darts show that all P3 edges marked with a single arrow are the diagonals of kites with the arrow pointing to the wide angle of the kite.
As a result, HBS tiles are a composition of darts and half-kites (or Robinson triangles) as described in Figure \ref{fig:hbs-P2}, and their edges have length 1.


%%%%%%%%%%%%%%%%%%%%%%%%%%%%%%%%%%%%%%%%%%%%%%

\subsection{HBS tilings as duals of P1 tilings}

P1 tilings are mentioned by Henley as pentagon packings but the local mapping between P1 and HBS tiles, shown in Figure \ref{fig:hbs-P1}, was first described in \cite{Lord1991}.
With the stars and boats, the resemblance between HBS and P1 tilings is obvious, which is why Steurer and Deloudi \cite{SD2009}, as they recall the local mapping, name the section ``Pentagon PT and the Dual Hexagon-Boat-Star (HBS) Tiling''.
From an HBS tiling, decorate the tiles as in Figure \ref{fig:hbs-P1-set1} and erase their edges to get the P1 tiling.
From a P1 tiling, simply join the centers of adjacent pentagons as in Figure \ref{fig:P1-hbs-set1} and erase the P1 tiles to get the corresponding HBS tiling.
The notches and bumps on P1 tiles are omitted here but can be recovered.
Yet this requires noticing how the three types of pentagons are related to the three vertex colors of Star tiles, thus using enriched decorations of P1 tiles as presented in subsection \ref{sub:hbs-substitutions} (Figure \ref{fig:P1-star}).
% Figure environment removed
% Figure environment removed


%%%%%%%%%%%%%%%%%%%%%%%%%%%%%%%%%%%%%%%%%%%%%%

\subsection{Known properties of HBS tilings}

Henley's paper was cited more than 300 times, but mostly by physicists who experimented different arrangements of atoms based on decorations on HBS shapes -- and not always with same assembly rules.
Steurer authored a consistent survey of structure research in quasicrystals \cite{Steurer2004} comparing such studies, followed with a book on quasicrystals with Deloudi including a lot more theoretical content \cite{SD2009}. 
In particular, Fig. 1.7 page 24 illustrates the decorations of HBS tiles by Ammann line segments as in Figure \ref{fig:tileset-BA} and give the relative vertex frequencies of P3 tilings, including the configurations which transform into HBS tiles.
% Figure environment removed
The ratio of hexagons to boats to stars is $\sqrt{5}\varphi : \sqrt{5} : 1$, so that the frequencies of the tiles, as computed by Olamy and Kléman \cite{OK1989}, are 
$$f_H=7-4\varphi=\sqrt{5}\varphi^{-3}\simeq52,8\% ~,\quad 
f_B=7\varphi-11=\sqrt{5}\varphi^{-4}\simeq32,6\% ~,\quad 
f_S=5-3\varphi=\varphi^{-4}\simeq14,6\%.$$

Since each HBS tiling is MLD with a Penrose tiling, the set of all (strict) HBS tilings form an LI class, and their configurational entropy is zero.
Nonetheless, many papers consider the HBS shapes with different decorations or none at all.
In the case where tiles are only required to fit edge-to-edge, the set of all possible tilings with those shapes (\textit{random HBS tilings}) has positive configurational entropy \cite{Henley1991}.
In addition to recalling this property, Gummelt \cite{Gummelt2006} gives the $\varphi^2$-composition of HBS tilings, drawn in Figure \ref{fig:phi2-composition} with the arrows. 
She came up with a decagon covering model, equivalent to the HBS tiling model, as did Lück earlier \cite{LUCK1990}: their decorated decagons superimpose according to specific assembly rules, and any valid covering with those can be interchanged with an HBS tiling.
Their models were recently compared by Steurer \cite{Steurer2021}, following a more recent short review on quasicrystals \cite{Steurer2018}.



%%%%%%%%%%%%%%%%%%%%%%%%%%%%%%%%%%%%%%%%%%%%%%
%%%%%%%%%%%%%%%%%%%%%%%%%%%%%%%%%%%%%%%%%%%%%%
%%%%%%%%%%%%%%%%%%%%%%%%%%%%%%%%%%%%%%%%%%%%%%

\section{The Star tileset}

As mentioned in Section 1, the tileset in Figure \ref{star-tileset-kd} yields the same tilings as the HBS tiles in Figure \ref{fig:hbs-tiles} once the decorations are removed.
It was derived from P2 tilings, but not in the same way as in Figure \ref{fig:hbs-P2}.
The construction is still quite simple and in addition to the $\varphi^2$-composition mentioned above, we have a $\varphi$-composition.
Vertex colors are strongly related to vertex configurations in P2, hence to vertex configurations in HBS tilings when you look at the compositions.
% Figure environment removed

\subsection{Construction}
As can be seen in Figure \ref{fig:hbs-P2} and considering the local rules of P2 tilings, the shortest distance between the centers of two stars\footnote{``Star'' vertex configurations in a P2 tiling or stars in a HBS tiling.} is $3+2\varphi^{-1}=\varphi^3$, which occurs when two HBS-stars are incident to the same HBS-edge (or kite of the underlying PT).
If you join the centers of the stars, the HBS shapes appear again but larger, composed with many kites (or half-kites) and darts as in Figure \ref{fig:star-tiling-example}.
The vertices are colored according to their degree, which corresponds to the type of P2-star they lie on: the star can be tangent to either 0 (red), 1 (green) or 2 (blue) suns.
If you put on each edge an arrow pointing to the greatest number (among the endpoints), you get exactly the HBS tiles as in Figure \ref{fig:hbs-tiles}.
Yet here there is more information encoded in the vertices, hence more local constraints.
The hexagons and boats all have exactly the same decorations but the stars come in three versions in what we call the Star tileset.


%%%%%%%%%%%%%%%%%%%%%%%%%%%%%%%%%%%%%%%%%%%%%%

\subsection{Substitutions}\label{sub:hbs-substitutions}

Obviously, the Ammann bar decorations of HBS tiles are still valid for the same shapes in the Star tileset.
Each vertex of the star tiling lies inside a small polygon formed by $n$ Ammann bars.
The label of the vertex is then $5-n$ (or the corresponding color).
% decomposition
Since the vertices of the HBS tiling are the stars of the P2 tiling, and a star is obtained by inflating a sun, joining the suns which are at distance $2+\varphi^{-1}=\varphi^2$ (the minimum) as in Figure \ref{fig:star-tiling-example} (magenta dashed lines) yields the same HBS tiling as first inflating once the P2 tiling and then joining the stars.
The vertex colors are then given by the number of queen configurations intersecting with the sun.
The resulting $\varphi$-decomposition of each tile is resumed in Figure \ref{fig:star-decomposition}.
Apply it twice to get the $\varphi^2$-decomposition mentioned above.\\
% Figure environment removed

Now if you apply the $\varphi$-decomposition a third time and decorate the smaller shapes as in Figure \ref{fig:hbs-P2-set} (usual decoration of HBS tiles with half-kites and darts), the initial Star tiles will have the same decorations as in Figure \ref{star-tileset-kd} %(more kites and darts within a shape)
.
Equivalently, from an HBS tiling with the tiles decorated as in Figure \ref{fig:hbs-P2-set} you can keep the HBS shapes as is, apply three $\varphi$-decompositions to kites and darts, and then appropriately place the vertex colors on hexagons and boats.
Actually, you can even place the colors from the start, considering how vertex configurations in P2 tilings substitute:
to have stars on initial HBS-vertices (and only there) after three P2 decompositions, you need to have suns exactly in those places at step two, which implies that at step one you had queens, kings or stars there.
Hence before the first decomposition,
\begin{itemize}
    \item place blue vertices on each deuce (D), which will decompose into a queen (Q);
    \item place green vertices on each jack (J), which will decompose into a king (K);
    \item place red vertices on each sun (S3, S4, S5), which will decompose into a star (S);
\end{itemize}
The letters in parentheses are the names of the equivalent vertex configurations in P3 tilings, with three versions of the sun (S3, S4, S5).
Since kings and queens, which yield respectively boats and hexagons, force quite a number of tiles around them, the vertex colors are unique.
On the other hand, the sun coming in three versions explains why it does so for the HBS-star.
As for the substitution with P1, the label of a pentagon is the number of HBS-arrows pointing to it.
Thus each type of pentagon from the original P1 tileset can simply get the corresponding color, as illustrated in Figures \ref{fig:star-P1-example} and \ref{fig:P1-star}, and if we add the arrows on P1 tiles then the corresponding HBS tiling appears on any P1 tiling.
% Figure environment removed
% Figure environment removed


%%%%%%%%%%%%%%%%%%%%%%%%%%%%%%%%%%%%%%%%%%%%%%

\subsection{Additional observations}

%\paragraph{lien entre les 2 façons de décorer par kd :} si on relie une étoile à un soleil chaque fois que la distance entre eux est $\varphi^3$ ou $\varphi^4=3\varphi+2$ (quand les côtés des kites sont 1 et $\varphi$) alors on obtient des kites and darts plus gros, et les soleils et étoiles correspondent aux couleurs des sommets (noir et blanc, matching rules).

As you can see in Figure \ref{fig:star-decomposition}, %which is consistent with the substitution rules of Penrose tilings, 
each vertex color in a Star tiling decomposes into an hexagon, a boat or a star according to its color.
Hence the ratio of blue to green to red is the same as hexagons to boats to stars, that is $\sqrt{5}\varphi : \sqrt{5} : 1$ as mentioned earlier.
So we still have
$$f_H=\sqrt{5}\varphi^{-3} ~,\quad 
f_B=\sqrt{5}\varphi^{-4} ~,$$
and since the frequency of stars in HBS tilings is $f_S=\varphi^{-4}$, we can deduce
$$f_2=\sqrt{5}\varphi^{-7}\simeq7,7\% ~,\quad 
f_1=\sqrt{5}\varphi^{-8}\simeq4,76\% ~,\quad 
f_0=\varphi^{-8}\simeq2,13\%$$
where $f_i$ is the frequency of the star S$_i$ with $i$ red vertices.
Another way to compute these frequencies is from those of vertex configurations in Penrose tilings.
Indeed, as indicated in subsections \ref{sub:hbs-P3} and \ref{sub:hbs-substitutions}, the tiles H, B, S$_2$, S$_1$ and S$_0$ correspond respectively to configurations D, J, S3, S4 and S5 in P3 tilings, which have respective frequencies $2-\varphi=\varphi^{-2}$, $2\varphi-3=\varphi^{-3}$, $13-8\varphi=\varphi^{-6}$, $13\varphi-21=\varphi^{-7}$ and $(47-29\varphi)/5=\varphi^{-7}/\sqrt{5}$ \cite{Mazac2023,FL2022}, for a total of $\varphi/\sqrt{5}$.
Thus multiplying by $\sqrt{5}\varphi^{-1}$ yields the same values as the first method for $f_H$, $f_B$, $f_2$, $f_1$ and $f_0$.\\


% Figure environment removed
% Figure environment removed
% Figure environment removed
\newpage
The vertex configurations are given in Figure \ref{fig:star-config}.
The main difference between HBS and Star tilings is the amount of information which can be deduced from a local configuration. 
As in Penrose tilings, each vertex configuration can force a whole set of tiles (in the tiling) which is called \textit{empire}.
This means that anytime a given vertex configuration appears in any tiling of the same type, %if you apply an isometry so that this cluster is centered on the origin and always in the same orientation, then 
all the tiles forming the empire are placed exactly in the same way relatively to it.
Since several empires are disconnected and can even contain infinitely many tiles, we only represent here the largest connected component, which includes the vertex configuration. We call it \textit{kingdom} (or \textit{local empire}).
For the bellflower and the orchid, since they contain no stars, vertex colors could be omitted (as in HBS tilings) and we obtain the kingdoms in Figure \ref{fig:bellflower-orchid}.
But when a vertex configuration includes a star, the colors have a significant impact, as you can see in Figure \ref{fig:stars-kingdoms}.
In particular, the star S1 forces a star S0 above and two stars S2 below.
These three kingdoms resemble those of respectively the star, the king and the queen in P2 tilings -- which makes sense considering observations on substitutions.





%%%%%%%%%%%%%%%%%%%%%%%%%%%%%
%%%%%%%%%%%%%%%%%%%%%%%%%%%%%
%%%%%%%%%%%%%%%%%%%%%%%%%%%%%

\section{Gemstones tileset}

The Star tileset can be modified to get convex tiles with only two labels at their vertices.
Indeed, observing that a blue vertex of an hexagon, a boat or a star is actually not a vertex of the Star tiling, its adjacent vertices can be joined by a segment, and the blue vertex erased.
In other words, each edge of the tiling becomes a straight segment, whose endpoints are labeled 0 (red) or 1 (green).
Alternately, these labels can be replaced with arrows, pointing from 0 to 1.
Thus we obtain the \textit{Gemstones} tileset in Figure \ref{fig:gemstones-tileset}.
On Figure \ref{fig:gemstones-example}, one can see how HBS tiles are deformed to get gemstones.\\
% Figure environment removed
% Figure environment removed
% Figure environment removed

Substitution from P3 is also quite simple.
Trace the long diagonal of each thin rhomb.
For each Q configuration (hexagon), compose the fat rhomb with its two adjacent half thin rhombs.
Everywhere else just erase all edges of the P3 tiling.
This construction gives a simple way to get the lengths of the sides of Gemstones.
Conversely, Gemstones are easily decorated with rhombuses (including arrows or other matching rules).
As for P2, the suns and jacks are vertices of Gemstones: trace a (short) segment between those vertices whenever they share a kite, and a (long) segment when they are separated by two kites which share a short edge.

%%%%%%%%%%%%%%%%%%%%%%%%%%%%%
%%%%%%%%%%%%%%%%%%%%%%%%%%%%%
%%%%%%%%%%%%%%%%%%%%%%%%%%%%%
\newpage
%\bibliographystyle{abbrv}
\bibliographystyle{alpha} %author+year
%\bibliographystyle{myalpha} %initials for first names
\bibliography{hbs}


\end{document}
