%%%%%
%%%%% File name  : mariostdcommands.tex
%%%%% Author     : Mario Chan
%%%%% Last update: 2nd September, 2022
%%%%% Description: Math LaTeX command definitions (without packages)
%%%%%              used by Mario Chan.
%%%%%
%%
%%%

%%%%%%%%%%%%%%%%%%%%%%%%%%%%%%%%%%%%%%%%%%%%%%%%%%%%%%%%%%%%%%%%
%% Begin of definitions of commands 
%%%%%%%%%%%%%%%%%%%%%%%%%%%%%%%%%%%%%%%%%%%%%%%%%%%%%%%%%%%%%%%%



%\DeclareMathAlphabet{\mathpzc}{OT1}{pzc}{m}{it}

\newcommand{\fieldC}{\mathbb{C}}
\newcommand{\fieldR}{\mathbb{R}}
\newcommand{\fieldQ}{\mathbb{Q}}
\newcommand{\Znum}{\mathbb{Z}}
\newcommand{\Nnum}{\mathbb{N}}
\newcommand{\Nnump}{\mathbb{N}_{>0}}
%\newcommand{\Onum}{\mathfrak{o}}
%\newcommand{\Cl}{\mathscr{C}\!\mathpzc{l}}
\newcommand{\Zmod}[1]{\Znum/#1\Znum}

\newcommand{\cplxi}{\sqrt{-1}}
% \newlength{\ibarbarheight}
% \setlength{\ibarbarheight}{-0.9ex}
% \newcommand{\ibar}{{\raisebox{\ibarbarheight}{$\mathchar'26$}\mkern-6.7mu i}}
\newcommand{\ibar}{
  \smash[b]{
    \begin{tikzpicture}[trim left, baseline=0pt]
      \node (i) [anchor=base, text width=0pt, text height=0pt] {$i$};
      \node (bar) [below left=0.75ex and 0.12ex, inner sep=0pt,
      outer sep=0pt, text width=0ex, text height=0ex] {$\mathchar'26$};
    \end{tikzpicture}
  }
}
\newcommand{\ibardefn}{\frac{\cplxi}{2\pi}}
\newsavebox{\ibarExplainBox}
\begin{lrbox}{\ibarExplainBox}
  \footnotesize
  \verb:{\raisebox{-0.9ex}{$\mathchar'26$}\mkern-6.7mu i}: 
\end{lrbox}
\newcommand{\ibarfootnote}{\footnote{The notation is chosen by
    mimicking the reduced Planck constant $\hbar = \frac{h}{2\pi}$. It
    can be typeset with the code \usebox\ibarExplainBox.}}
\newcommand{\proj}{\mathbb{P}}
\newcommand{\sphere}{\mathbb{S}}
\newcommand{\uhp}{\mathfrak{H}}
\newcommand{\eps}{\varepsilon}
\newcommand{\vphi}{\varphi}

\newcommand{\sheaf}[1]{\mathscr{#1}}
\newcommand{\ideal}[1]{\mathfrak{#1}}
\newcommand{\bundle}[1]{\mathbb{#1}}

\newcommand{\mero}{\sheaf{M}}
\newcommand{\holo}{\sheaf{O}}
\newcommand{\holoform}{\boldsymbol\Omega}
\newcommand{\kanshf}[1][X]{\sheaf K_{#1}}
\newcommand{\smooth}[1][\infty]{\mathscr C^{#1}}
\NewDocumentCommand{\smform} { s D//{p,q} O{X} }{
  \sheaf A^{#2}_{#3\IfBooleanT{#1}{\,c}}  
}
\newcommand{\Lloc}[1][1]{L^{#1}_{\text{loc}}}
\newcommand{\maxidl}{\ideal m}
\newcommand{\multidl}{\sheaf I}

\NewDocumentCommand{\Tgt}{ %% Tangent bundle
  s       %% #1 turn to cotangent bundle when starred
  d//     %% #2 types of (co)tangent vectors wrt almost cplx structure
}{\mathbf T^{\IfBooleanT{#1}{*\IfNoValueF{#2}{\,}}\IfNoValueF{#2}{#2}}}

\def\cTgt{\Tgt*} %% Cotangent bundle, for backward compatibility

% \newcommand{\Tgt}[1][]{\mathbf T^{#1}}
% \newcommand{\cTgt}[1][]{\Tgt[*\,#1]}

\NewDocumentCommand{\cohgp}{
  O{H}   %% #1 symbol for cohomology (e.g.~H or \check H)
  t_     %% #2 with "_", it becomes homology
  m      %% #3 (co)homology degree
  D<>{}  %% #4 sub/super-scripts to be put to the (co)homology group
  o      %% #5 supporting space
  D//{,} %% #6 separator between space and coefficients 
  m      %% #7 coefficients
}{
  #1\IfBooleanTF{#2} {_{#3}^{#4}} {^{#3}_{#4}} \paren{\IfNoValueF{#5}{#5#6} #7}
}

\NewDocumentCommand{\cohdim}{
  t_     %% with "_", it becomes homology
  m      %% (co)homology degree
  D<>{}  %% sub/super-scripts to be put to the (co)homology group
  o      %% supporting space
  D//{,} %% separator between space and coefficients 
  m      %% coefficients
}{
  h\IfBooleanTF{#1} {_{#2}^{#3}} {^{#2}_{#3}} \paren{\IfNoValueF{#4}{#4#5} #6}
}

\newcommand{\codim}{\operatorname{codim}}
\newcommand{\Sing}{\operatorname{Sing}}
\newcommand{\mult}{\operatorname{mult}}
\newcommand{\ord}{\operatorname{ord}}
\newcommand{\im}{\operatorname{im}}
\newcommand{\Res}{\operatorname{Res}}
\newcommand{\id}{\operatorname{id}}
\newcommand{\coker}{\operatorname{coker}}
\newcommand{\pr}{\operatorname{pr}}
\newcommand{\Tr}{\operatorname{Tr}}
\newcommand{\rk}{\operatorname{rk}}
\newcommand{\vol}{\operatorname{vol}}
\newcommand{\supp}{\operatorname{supp}}
\newcommand{\Dom}{\operatorname{Dom}}
\newcommand{\Alb}{\operatorname{Alb}}
\newcommand{\alb}{\operatorname{alb}}
\newcommand{\Pic}{\operatorname{Pic}}
\newcommand{\Exc}{\operatorname{Exc}}
\newcommand{\Hess}{\operatorname{Hess}}
\newcommand{\esssup}{\operatorname*{ess\,sup}}
\newcommand{\BigO}{\operatorname{\mathbf{O}}}

% \newcommand{\bexp}{\boldsymbol{\operatorname{e}}}

\renewcommand{\Re}{\operatorname{Re}}
\renewcommand{\Im}{\operatorname{Im}}

\newcommand{\symmgp}{\mathfrak{S}}
\newcommand{\GL}[2][2]{\mathrm{GL}_{#1}(#2)}
\newcommand{\SL}[2][2]{\mathrm{SL}_{#1}(#2)}
\newcommand{\SO}[2][2]{\mathrm{SO}_{#1}(#2)}
% \newcommand{\SL}[1]{\mathrm{SL}_2(#1)}
% \newcommand{\SO}[1]{\mathrm{SO}_2(#1)}
\newcommand{\SU}[1]{\mathrm{SU}(#1)}
\newcommand{\Aut}[1]{\mathrm{Aut}(#1)}
\newcommand{\Div}[1]{\mathrm{Div}(#1)}
\newcommand{\divsr}[1]{\mathrm{div}\paren{#1}}

\newcommand{\sgn}[1]{\operatorname{sgn}(#1)}

%%%%% Could be replaced by calling the package "mleftright" 
\let\originalleft\left
\let\originalright\right
\renewcommand{\left}{\mathopen{}\mathclose\bgroup\originalleft}
\renewcommand{\right}{\aftergroup\egroup\originalright}
%%%%%

\newcommand{\lparpht}[2][(]{\mathopen{\left#1\vphantom{#2}\right.\kern-\nulldelimiterspace}}
\newcommand{\rparpht}[2][)]{\mathclose{\left.\kern-\nulldelimiterspace\vphantom{#2}\right#1}}
% \newcommand{\lparpht}[1]{\mathopen{\left(\vphantom{#1}\right.\kern-\nulldelimiterspace}}
% \newcommand{\rparpht}[1]{\mathclose{\left.\kern-\nulldelimiterspace\vphantom{#1}\right)}}
% \newcommand{\paren}[1]{\!\left(#1\right)}
\newcommand{\paren}[1]{\lparpht{#1}#1\rparpht{#1}}
\newcommand{\bigparen}[1]{\bigl(#1\bigr)}
\newcommand{\res}[1]{\left.#1\right|}
\newcommand{\parres}[1]{\res{\paren{#1}}}
\newcommand{\seq}[1]{\lparpht[\{]{#1}#1\rparpht[\}]{#1}}
% \newcommand{\seq}[1]{\left\{#1\right\}}
\newcommand{\set}[1]{\lparpht[\{]{#1}#1\rparpht[\}]{#1}}
% \newcommand{\set}[1]{\left\{#1\right\}}
\newcommand{\setd}[2]{\left\{#1\:\left|\;\vphantom{#1} #2\right.\right\}}
\newcommand{\abs}[1]{\left\lvert#1\right\rvert}
\newcommand{\bigabs}[1]{\bigl\lvert#1\bigr\rvert}
\newcommand{\norm}[1]{\left\lVert#1\right\rVert}
\newcommand{\bignorm}[1]{\bigl\lVert#1\bigr\rVert}
\newcommand{\Bignorm}[1]{\Bigl\lVert#1\Bigr\rVert}
\newcommand{\inner}[2]{\left\langle#1,#2\right\rangle}
\newcommand{\ptinner}[2]{\left(#1,#2\right)}
\newcommand{\biginner}[2]{\bigl\langle#1,#2\bigr\rangle}
\newcommand{\commut}[2]{\left[#1,#2\right]}
\newcommand{\algnorm}[1]{\mathbf N(#1)}
\newcommand{\smod}[1]{\,(\operatorname{mod}\,#1)}
\newcommand{\ceil}[1]{\left\lceil#1\right\rceil}
\newcommand{\Ceil}[1]{\bigl\lceil#1\bigr\rceil}
\newcommand{\floor}[1]{\left\lfloor#1\right\rfloor}
\newcommand{\fracpart}[1]{\left\{#1\right\}}

\newcommand{\tp}[1]{\,{\vphantom{#1}}^*\! #1}
\newcommand{\rtp}[1]{\,{\vphantom{#1}}^t\! #1}

\newcommand{\genby}[1]{\left\langle#1\right\rangle}
\newcommand{\genbyd}[2]{\left\langle#1\:\left|\;\vphantom{#1} #2\right.\right\rangle}

\newcommand{\tmatrix}[4]{\left[\begin{smallmatrix}#1 & #2 \\ #3 & #4\end{smallmatrix}\right]}
\newcommand{\Diag}[1]{\operatorname{diag}\bigl(#1\bigr)}

\newcommand{\cl}{\overline}

\newcommand{\vect}[1]{\underline{#1}}

\newcommand{\conj}{\overline}
\newcommand{\diff}{\partial}
%% \dbar is defined in amsrefs (mathscinet) with out 'lite' option, so
%% \renewcommand is used 
\makeatletter
\@ifundefined{dbar}{
  \newcommand{\dbar}{{\conj\diff}}  
}{
  \renewcommand{\dbar}{{\conj\diff}}
  % \PackageWarningNoLine{marionotations}{`\protect\dbar' is redefined}
  \@latex@warning@no@line{`\protect\dbar' is redefined in mariostdcommands.tex}
}
\makeatother

\NewDocumentCommand{\fdiff}{O{#3} m O{}}{\frac{\diff #1}{\diff #2}}
\newcommand{\dfadj}{\vartheta}
\newcommand{\ddbar}{\diff\dbar}
\newcommand{\iddbar}{\cplxi\ddbar}
\newcommand{\ibddbar}{\ibar\ddbar}

\newcommand{\ddc}{dd^c\mspace{1mu}}

\newcommand{\bdry}{\partial}

\newcommand{\birat}{\mathrel{\dashrightarrow}}
\newcommand{\tendsto}{\mathrel{\rightarrow}}
\newcommand{\xtendsto}[1]{\mathrel{\xrightarrow{#1}}}
\newcommand{\wktendsto}{\mathrel{\rightharpoonup}}
\newcommand{\descendsto}{\mathrel{\searrow}}
\newcommand{\ascendsto}{\mathrel{\nearrow}}
\newcommand{\imply}{\mathrel{\Rightarrow~}}
\newcommand{\isom}{\mathrel{\cong}}
\newcommand{\ctrt}{\mathbin{\lrcorner}}

