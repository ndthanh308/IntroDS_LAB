%%%%%
%%%%% File name  : commut-diagram_sing-Fujino-conj.tex
%%%%% Author     : Mario Chan
%%%%% Date       : tth March, 2023
%%%%% Description: This is the commutative diagram which explains the
%%%%%              strategy for proving Fujino's conjecture.
%%%%%
%%
%%%
\documentclass[Injectivity-Fujino.tex]{subfiles}

\renewcommand{\objectstyle}{\displaystyle}

\begin{document}
\begin{equation} \label{eq:commut-diagram_sing-Fujino-conj}
  \begin{gathered}
    \xymatrix@R=0.85cm@C+0.75cm{
      {\vdots} \ar[d]
      & {\vdots} \ar[d]
      & {\vdots} \ar[d]
      \\
      {\spH{\faidlof-1/|0|*}} \ar[d] \ar@{=}[r]
      & {\spH{\faidlof-1/|0|*}} \ar[r]^-{\otimes s}
      \ar[d]_-{\iota_{\sigma-1}} \ar[dr]|-*+{\mu_{\sigma-1}}
      & {\spH M{\faidlof-1/|0|*}} \ar[d]
      \\
      {\spH{\faidlof/|0|*}} \ar[d] \ar[r]^-{\iota_{\sigma}}
      \ar@/_1.9pc/[rr]|(.65)*+{\mu_{\sigma}}
      & {\spH{\faidlof|\sigma_{\mlc}|/|0|*}}
      \ar[d]|(.38)*+<3pt>{ }
      \ar[r]^-{\otimes s}
      & {\spH M{\faidlof|\sigma_{\mlc}|/|0|*}}
      \ar[d]
      \\
      {\spH{\residlof*}} \ar[d] \ar[r]^-{\tau_\sigma}
      \ar@/_1.65pc/[rr]+<-39pt,-15pt>|(.67)*+{\nu_\sigma}
      & {\spH{\faidlof|\sigma_{\mlc}|/-1*}} \ar[d]|(.52)*+<3pt>{}
      \ar[r]^-{\otimes s}
      & {\spH M{\faidlof|\sigma_{\mlc}|/-1*}} \ar[d] \\
      {\vdots} & {\vdots} & {\vdots} }
  \end{gathered}
\end{equation}
\end{document}



%%% Local Variables:
%%% mode: latex
%%% TeX-master: t
%%% coding: utf-8
%%% End:
