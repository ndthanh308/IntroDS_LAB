%%%%%
%%%%% File name  : L2-spaces-n-harmonic-forms.tex
%%%%% Author     : Mario Chan
%%%%% Date       : 27th March, 2023
%%%%% Description: This is the section on the basic facts of the Hodge
%%%%%              decomposition and the implications of positivity on
%%%%%              harmonic forms which have been discussed in
%%%%%              previous papers.
%%%%%
%%
%%%

{
  \setDefaultvphi{\vphi_L}

  % Suppose that $X$ is \emph{compact} K\"ahler in this section.
  Let $L$ be a holomorphic line bundle on $X$ equipped with a
  (possibly singular) quasi-psh potential $\vphi_L$, which induces,
  together with the K\"ahler form $\omega$, an $L^2$ norm
  $\norm{\cdot}_{X} := \norm\cdot_{X,\vphi_L,\omega}$ on the space of
  smooth $K_X \otimes L$-valued $(0,q)$-forms (or $L$-valued
  $(n,q)$-forms) on $X$.
  Let $\Ltwo/n,q/{L}_{\vphi_L}$ be the completion with respect to $\norm\cdot_X$
  and $\Harm :=\Harm{L}$ be the space of harmonic forms with respect to $\norm\cdot_X$.
  The $L^2$ Dolbeault isomorphism (see
  \cite{Matsumura_injectivity}*{Prop.~5.5 and 5.8} and
  \cite{Matsumura_injectivity-lc}*{Prop.~2.8} for a proof, and see 
  \cite{Chan&Choi_injectivity-I}*{footnote 1} for its naming) guarantees
  the closedness of the subspaces in the orthogonal decomposition
  \begin{equation*}
    \Ltwo/n,q/{L}_{\vphi_L}
    = \Harm \oplus \cl{\paren{\im\dbar}}_{\vphi_L} \oplus \cl{\paren{\im\dbadj}}_{\vphi_L}
    = \Harm \oplus \paren{\im\dbar}_{\vphi_L} \oplus \paren{\im\dbadj}_{\vphi_L} 
  \end{equation*}
  (where $\dbadj$ is the Hilbert space adjoint of $\dbar$ with respect
  to $\norm\cdot_X$, $\paren{\im\dbar}_{\vphi_L}$ and
  $\paren{\im\dbadj}_{\vphi_L}$ denote the images of the corresponding
  operators, with $\cl{\paren{\im\dbar}}_{\vphi_L}$ and
  $\cl{\paren{\im\dbadj}}_{\vphi_L}$ being their closures in
  $\Ltwo/n,q/{L}_{\vphi_L}$)
  and the isomorphism
  \begin{equation*}
    \Harm \isom \cohgp q[X]{K_X \otimes L \otimes \mtidlof{\vphi_L}}
  \end{equation*}
  between the space of harmonic forms and the \v Cech cohomology
  group.
  % Given a locally finite Stein cover $\cvr V = \set{V_i}_{i \in I}$
  % with a partition of unity $\set{\rho^i}_{i\in I}$ subordinate to,
  With $\cvr V := \set{V_i}_{i\in I}$ and $\set{\rho^i}_{i\in I}$ given in Section \ref{subsec:notation},
  the isomorphism can be given explicitly as follows.
  For any (alternating) \v Cech $q$-cocycle $\set{\alpha_{\idx 0.q}}_{\idx 0,q \in
    I}$ and any harmonic form $u \in \Harm$ such that they represent
  the same class in $\cohgp q[X]{K_X \otimes L \otimes
    \mtidlof{\vphi_L}}$, the two representatives are related by 
  (under the Einstein summation convention)
  \begin{equation} \label{eq:Cech-Dolbeault-isom}
    \begin{aligned}
      u &=\dbar v_{(2)} +\dbar \rho^{i_{q-1}} \wedge \dotsm \wedge
      \dbar\rho^{i_0} \alpha_{\idx 0.q} \qquad\paren{\forall~ i_q \in
        I}
      \\
      &=\dbar v_{(2)} +\dbar \rho^{i_{q-1}} \wedge \dotsm \wedge
      \dbar\rho^{i_0} \cdot \rho^{i_q} \:\alpha_{\idx 0.q}
      \\
      &=\dbar v_{(2)} +(-1)^q \:\underbrace{\dbar \rho^{i_{q}} \wedge
        \dotsm \wedge \dbar\rho^{i_1} \cdot \rho^{i_0} }_{=: \:
        \paren{\dbar\rho}^{\idx q.0}} \alpha_{\idx 0.q}
    \end{aligned}
  \end{equation}
  for some $K_X \otimes L$-valued $(0,q-1)$-form $v_{(2)}$ on $X$ with
  $L^2$ coefficients with respect to $\norm\cdot_{X}$ (see
  \cite{Matsumura_injectivity}*{Prop.~5.5} or
  \cite{Chan&Choi_injectivity-I}*{Lemma 3.2.1}).

  The above result is applicable also to the case when $L$ is replaced by
  $D \otimes L$ equipped with the potential $\phi_D +\vphi_L$, where
  $\phi_D :=\log\abs{\sect_D}^2$.
  Denote the corresponding $L^2$ norm by $\norm\cdot_{X,\phi_D}$.
  Assume that \emph{$\vphi_L$ is smooth on $X$}.
  We state the following simple fact here for clarity.
  \begin{lemma} \label{lem:su-harmonicity}
    If $u \in \Harm{L}$, then $\sect_D u \in \Harm{D\otimes L},{\phi_D+\vphi_L}$.
  \end{lemma}
  
  \begin{proof}
    Since $\sect_D$ is holomorphic, it is clear that $\sect_D u$ is
    $\dbar$-closed.

    Let $\dfadj$ and $\dfadj_{\phi_D}$ be the formal adjoint of
    $\dbar$ with respect to $\vphi_L$ and $\phi_D +\vphi_L$
    respectively.
    It then follows that $\dfadj_{\phi_D} = \dfadj
    +\idxup{\diff\phi_D} . \cdot$ and 
    \begin{equation*}
      \dfadj_{\phi_D} \paren{\sect_D u}
      = \sect_D \:\dfadj u - \idxup{\diff\sect_D}. u
      +\idxup{\diff\phi_D} .\sect_D u
      =\sect_D \dfadj u = 0 \; .
    \end{equation*}
    Note that $\omega$ is not complete on $X \setminus D$ and the
    claim (in particular, $\sect_D u \in \Dom \dbadj_{\phi_D}$, where
    $\dbadj_{\phi_D}$ is the Hilbert space adjoint of $\dbar$ with
    respect to $\norm\cdot_{X,\phi_D}$) cannot follow from the
    standard result (for example, \cite{Demailly}*{Ch.~VIII,
      Thm.~(3.2c)}).
    Indeed, the proof of $su \in \Dom
    \dbadj_{\vphi_M}$ in \cite{Chan&Choi_injectivity-I}*{Cor.~3.2.6}
    gives precisely the result $\sect_D u \in \Dom\dbadj_{\phi_D}$ in
    the current setting, which completes the proof.
    A sketch of it is given below for readers' convenience.
    
    Let $\theta \colon [0,\infty) \to [0,1]$ be a smooth
    non-decreasing cut-off function such that
    $\res\theta_{[0,\frac12]} \equiv 0$ and $\res\theta_{[1,\infty)}
    \equiv 1$.
    Set $\theta_\eps := \theta \circ \frac{1}{\abs{\psi_D}^\eps}$ and
    $\theta'_\eps := \theta' \circ \frac{1}{\abs{\psi_D}^\eps}$ for
    every $\eps \geq 0$ (where $\theta'$ is the derivative of
    $\theta$).
    Then both $\theta_\eps$ and $\theta'_\eps$ have compact supports
    inside $X \setminus D$ for $\eps > 0$ and $\theta_\eps \ascendsto
    1$ pointwisely on $X \setminus D$ as $\eps \descendsto 0$.
    For any $\zeta \in \Dom\dbar \subset \Ltwo/n,q-1/<X>{D\otimes
      L}_{\phi_D+\vphi_L}$, convolution with a smoothing kernel on
    local coordinate charts and the lemma of Friedrichs guarantees the
    existence of a sequence $\seq{\zeta_{\eps, \nu}}_{\nu\in\Nnum}$ of
    smooth forms compactly supported in $X \setminus D$ such that
    $\zeta_{\eps,\nu} \tendsto \theta_\eps \zeta$ in the graph norm
    $\paren{\norm\cdot_{X,\phi_D}^2
      +\norm{\dbar\:\cdot}_{X,\phi_D}^2}^{\frac 12}$ of $\dbar$ for
    each $\eps > 0$.
    It then follows that
    \begin{align*}
      \iinner{\sect_D u}{\dbar\zeta}_{X,\phi_D} 
      \xleftarrow{\eps \tendsto 0^+}
      &~\iinner{\sect_D u}{\theta_\eps \dbar\zeta}_{X,\phi_D} \\
      =&~\iinner{\sect_D u}{\dbar\paren{\theta_{\eps}\zeta}}_{X,\phi_D}
         -\iinner{\sect_D u}{\dbar\theta_\eps \wedge \zeta}_{X,\phi_D} \\
      \xleftarrow{\nu \tendsto \infty}
      &~\iinner{\sect_D u}{\dbar\zeta_{\eps,\nu}}_{X,\phi_D}
        -\iinner{\sect_D u}{\frac{\eps \theta'_\eps}{\abs{\psi_D}^{1+\eps}}
        \dbar\psi_D \wedge \zeta}_{X,\phi_D} \\
      =&~\iinner{\dfadj_{\phi_D} \paren{\sect_D u}}{\zeta_{\eps,\nu}}_{X,\phi_D}
         -\iinner{\frac{\eps \theta'_\eps}{\abs{\psi_D}^{1+\eps}}
         \idxup{\diff\psi_D} . \sect_D u}{\zeta}_{X,\phi_D} \; .
    \end{align*}
    The inner product on the far right-hand-side converges to $0$ as
    $\eps \tendsto 0^+$, a consequence of the residue computation (see
    \cite{Chan&Choi_injectivity-I}*{Prop.~3.2.3 and Remark 3.2.4}).
    We can then conclude that $\sect_D u \in \Dom\dbadj_{\phi_D}$ after
    letting $\nu \tendsto \infty$ and then $\eps \tendsto 0^+$.
  \end{proof}

}

Now consider the cases where $(L, \vphi_L) =(F, \vphi_F)$
% (with the induced $L^2$ norm $\norm\cdot_{X}$)
and $(L, \vphi_L) =(F\otimes M, \vphi_F +\vphi_M)$.
% (with the induced $L^2$ norm $\norm\cdot_{X,\vphi_M}$).
A consequence of the positivity on $F$ and $M$ in
\cite{Enoki}, \cite{Matsumura_injectivity-lc} and
\cite{Chan&Choi_injectivity-I} are recalled below.
\begin{prop} \label{prop:consequence-of-positivity}
  % Suppose $\vphi_F$ and $\vphi_M$ are smooth such that
  % $\ibddbar\vphi_F \geq 0$ and $C\ibddbar\vphi_F \geq \ibddbar\vphi_M
  % \;\paren{\geq - C \omega}$ for some constant $C > 0$.
  % Then, $u \in \Harm{F}$ implies $su \in \Harm{F\otimes
  %   M},{\vphi_F+\vphi_M}$ and $\nabla^{(0,1)}u = 0$.
  Suppose that $\vphi_F$ is smooth such that
  $\ibddbar\vphi_F \geq 0$ and $u \in \Harm{F}$.
  Then, one has  $\nabla^{(0,1)}u = 0$.
  If, furthermore, $\vphi_M$ is smooth and satisfies
  $\paren{- C \omega \leq} \; \ibddbar\vphi_M \leq C\ibddbar\vphi_F$
  for some constant $C > 0$,
  then one also has $su \in \Harm{F\otimes M},{\vphi_F+\vphi_M}$.
\end{prop}

\begin{proof}[Reference to the proof]
  These results follow directly from the Bochner--Kodaira--Nakano
  formula.
  See \cite{Chan&Choi_injectivity-I}*{Prop.~3.2.5 and
    Cor.~3.2.6} (while taking $D=0$ and $\psi_D \equiv -1$ in those
  statements).
  See also the proofs for $\diff^*_h\xi = 0$ in
  \cite{Enoki}*{Prop.~2.1} or $D'^*u = 0$ in
  \cite{Matsumura_injectivity-lc}*{Prop.~3.7}.
  These are equivalent statements to the claim $\nabla^{(0,1)}u =0$
  (indeed, $\diff^*_h = D'^*$ and $\abs{D'^*u}^2 =
  \abs{\nabla^{(0,1)}u}^2$ by \cite{Chan&Choi_injectivity-I}*{Remark
    2.4.3}).
\end{proof}


Lemma \ref{lem:su-harmonicity} and Proposition
\ref{prop:consequence-of-positivity} are applied to the case with
$\lcS$ in place of $X$ and $\phi_{(p)}$ in place of $\phi_D$ in the
following sections.



%%% Local Variables:
%%% mode: latex
%%% TeX-master: "Injectivity-Fujino"
%%% coding: utf-8
%%% End:
