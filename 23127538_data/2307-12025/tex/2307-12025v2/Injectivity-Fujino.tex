%%%%%
%%%%% File name  : Injectivity-Fujino.tex
%%%%% Author     : Mario Chan
%%%%% Date       : 25th July, 2022 
%%%%% Description: This file is set up to compile the project with the
%%%%%              class amsart.cls. This project "Injectivity-Fujino"
%%%%%              proves the injectivity theorem on lc pairs of
%%%%%              arbitrary codimension of the mlc's, the full Fujino
%%%%%              conjecture.
%%%%%
%%
%%%
\documentclass[a4paper,12pt]{amsart}

\pdfoutput=1 %% added to force arXiv AutoTeX to typeset with pdflatex
             %% (so that rotation of symbols in Xy-pic is rendered);
             %% have to be within the first 5 lines of preamble

\usepackage[top=3cm,bottom=3cm,outer=3cm,inner=2cm,marginpar=2.45cm]{geometry}
\usepackage[destlabel,final,colorlinks=true]{hyperref}
\usepackage[abbrev]{amsrefs}

%%%%%
%%%%% File name  : packagesandcommands.tex
%%%%% Author     : Mario Chan
%%%%% Date       : 13th December, 2021 (original: 04th November, 2020)
%%%%% Description: This file collects the packages used and commands
%%%%%              defined in the project "Injectivity-Fujino".
%%%%%
%%
%%%

\usepackage[french,ngerman,english]{babel}
\usepackage[utf8]{inputenc}
\usepackage[T1]{fontenc}

% \usepackage{CJKutf8} %%%%% for the use of Chinese, use the
                       %%%%% environment
                       %%%%% \begin{CJK}{UTF8}{bkai} % or {bsmi}
                       %%%%% \end{CJK}

\usepackage[all]{xy}
\renewcommand{\objectstyle}{\displaystyle}

\usepackage{enumitem}
\usepackage{mathtools}  
\usepackage[usenames,dvipsnames]{xcolor}
\usepackage{calc} %% for doing length computations

\babeltags{de = ngerman}
\babeltags{fr = french}

\usepackage{upref}  %% for uprighting texts generated by \ref
\usepackage{embrac} %% for uprighting parentheses in \emph and
                    %% providing \embparen for the same effect in
                    %% theorem environments (which use {\itshape ...}
                    %% or {\em ...})

\usepackage[
% show-mario,  %% show editing notes or comments on margin when
             %% uncommented
% no-commands %% avoid loading commands in mariostdcommands.tex when uncommented
]{marionotations}
% %%%%%
%%%%% File name  : mariostdcommands.tex
%%%%% Author     : Mario Chan
%%%%% Last update: 2nd September, 2022
%%%%% Description: Math LaTeX command definitions (without packages)
%%%%%              used by Mario Chan.
%%%%%
%%
%%%

%%%%%%%%%%%%%%%%%%%%%%%%%%%%%%%%%%%%%%%%%%%%%%%%%%%%%%%%%%%%%%%%
%% Begin of definitions of commands 
%%%%%%%%%%%%%%%%%%%%%%%%%%%%%%%%%%%%%%%%%%%%%%%%%%%%%%%%%%%%%%%%



%\DeclareMathAlphabet{\mathpzc}{OT1}{pzc}{m}{it}

\newcommand{\fieldC}{\mathbb{C}}
\newcommand{\fieldR}{\mathbb{R}}
\newcommand{\fieldQ}{\mathbb{Q}}
\newcommand{\Znum}{\mathbb{Z}}
\newcommand{\Nnum}{\mathbb{N}}
\newcommand{\Nnump}{\mathbb{N}_{>0}}
%\newcommand{\Onum}{\mathfrak{o}}
%\newcommand{\Cl}{\mathscr{C}\!\mathpzc{l}}
\newcommand{\Zmod}[1]{\Znum/#1\Znum}

\newcommand{\cplxi}{\sqrt{-1}}
% \newlength{\ibarbarheight}
% \setlength{\ibarbarheight}{-0.9ex}
% \newcommand{\ibar}{{\raisebox{\ibarbarheight}{$\mathchar'26$}\mkern-6.7mu i}}
\newcommand{\ibar}{
  \smash[b]{
    \begin{tikzpicture}[trim left, baseline=0pt]
      \node (i) [anchor=base, text width=0pt, text height=0pt] {$i$};
      \node (bar) [below left=0.75ex and 0.12ex, inner sep=0pt,
      outer sep=0pt, text width=0ex, text height=0ex] {$\mathchar'26$};
    \end{tikzpicture}
  }
}
\newcommand{\ibardefn}{\frac{\cplxi}{2\pi}}
\newsavebox{\ibarExplainBox}
\begin{lrbox}{\ibarExplainBox}
  \footnotesize
  \verb:{\raisebox{-0.9ex}{$\mathchar'26$}\mkern-6.7mu i}: 
\end{lrbox}
\newcommand{\ibarfootnote}{\footnote{The notation is chosen by
    mimicking the reduced Planck constant $\hbar = \frac{h}{2\pi}$. It
    can be typeset with the code \usebox\ibarExplainBox.}}
\newcommand{\proj}{\mathbb{P}}
\newcommand{\sphere}{\mathbb{S}}
\newcommand{\uhp}{\mathfrak{H}}
\newcommand{\eps}{\varepsilon}
\newcommand{\vphi}{\varphi}

\newcommand{\sheaf}[1]{\mathscr{#1}}
\newcommand{\ideal}[1]{\mathfrak{#1}}
\newcommand{\bundle}[1]{\mathbb{#1}}

\newcommand{\mero}{\sheaf{M}}
\newcommand{\holo}{\sheaf{O}}
\newcommand{\holoform}{\boldsymbol\Omega}
\newcommand{\kanshf}[1][X]{\sheaf K_{#1}}
\newcommand{\smooth}[1][\infty]{\mathscr C^{#1}}
\NewDocumentCommand{\smform} { s D//{p,q} O{X} }{
  \sheaf A^{#2}_{#3\IfBooleanT{#1}{\,c}}  
}
\newcommand{\Lloc}[1][1]{L^{#1}_{\text{loc}}}
\newcommand{\maxidl}{\ideal m}
\newcommand{\multidl}{\sheaf I}

\NewDocumentCommand{\Tgt}{ %% Tangent bundle
  s       %% #1 turn to cotangent bundle when starred
  d//     %% #2 types of (co)tangent vectors wrt almost cplx structure
}{\mathbf T^{\IfBooleanT{#1}{*\IfNoValueF{#2}{\,}}\IfNoValueF{#2}{#2}}}

\def\cTgt{\Tgt*} %% Cotangent bundle, for backward compatibility

% \newcommand{\Tgt}[1][]{\mathbf T^{#1}}
% \newcommand{\cTgt}[1][]{\Tgt[*\,#1]}

\NewDocumentCommand{\cohgp}{
  O{H}   %% #1 symbol for cohomology (e.g.~H or \check H)
  t_     %% #2 with "_", it becomes homology
  m      %% #3 (co)homology degree
  D<>{}  %% #4 sub/super-scripts to be put to the (co)homology group
  o      %% #5 supporting space
  D//{,} %% #6 separator between space and coefficients 
  m      %% #7 coefficients
}{
  #1\IfBooleanTF{#2} {_{#3}^{#4}} {^{#3}_{#4}} \paren{\IfNoValueF{#5}{#5#6} #7}
}

\NewDocumentCommand{\cohdim}{
  t_     %% with "_", it becomes homology
  m      %% (co)homology degree
  D<>{}  %% sub/super-scripts to be put to the (co)homology group
  o      %% supporting space
  D//{,} %% separator between space and coefficients 
  m      %% coefficients
}{
  h\IfBooleanTF{#1} {_{#2}^{#3}} {^{#2}_{#3}} \paren{\IfNoValueF{#4}{#4#5} #6}
}

\newcommand{\codim}{\operatorname{codim}}
\newcommand{\Sing}{\operatorname{Sing}}
\newcommand{\mult}{\operatorname{mult}}
\newcommand{\ord}{\operatorname{ord}}
\newcommand{\im}{\operatorname{im}}
\newcommand{\Res}{\operatorname{Res}}
\newcommand{\id}{\operatorname{id}}
\newcommand{\coker}{\operatorname{coker}}
\newcommand{\pr}{\operatorname{pr}}
\newcommand{\Tr}{\operatorname{Tr}}
\newcommand{\rk}{\operatorname{rk}}
\newcommand{\vol}{\operatorname{vol}}
\newcommand{\supp}{\operatorname{supp}}
\newcommand{\Dom}{\operatorname{Dom}}
\newcommand{\Alb}{\operatorname{Alb}}
\newcommand{\alb}{\operatorname{alb}}
\newcommand{\Pic}{\operatorname{Pic}}
\newcommand{\Exc}{\operatorname{Exc}}
\newcommand{\Hess}{\operatorname{Hess}}
\newcommand{\esssup}{\operatorname*{ess\,sup}}
\newcommand{\BigO}{\operatorname{\mathbf{O}}}

% \newcommand{\bexp}{\boldsymbol{\operatorname{e}}}

\renewcommand{\Re}{\operatorname{Re}}
\renewcommand{\Im}{\operatorname{Im}}

\newcommand{\symmgp}{\mathfrak{S}}
\newcommand{\GL}[2][2]{\mathrm{GL}_{#1}(#2)}
\newcommand{\SL}[2][2]{\mathrm{SL}_{#1}(#2)}
\newcommand{\SO}[2][2]{\mathrm{SO}_{#1}(#2)}
% \newcommand{\SL}[1]{\mathrm{SL}_2(#1)}
% \newcommand{\SO}[1]{\mathrm{SO}_2(#1)}
\newcommand{\SU}[1]{\mathrm{SU}(#1)}
\newcommand{\Aut}[1]{\mathrm{Aut}(#1)}
\newcommand{\Div}[1]{\mathrm{Div}(#1)}
\newcommand{\divsr}[1]{\mathrm{div}\paren{#1}}

\newcommand{\sgn}[1]{\operatorname{sgn}(#1)}

%%%%% Could be replaced by calling the package "mleftright" 
\let\originalleft\left
\let\originalright\right
\renewcommand{\left}{\mathopen{}\mathclose\bgroup\originalleft}
\renewcommand{\right}{\aftergroup\egroup\originalright}
%%%%%

\newcommand{\lparpht}[2][(]{\mathopen{\left#1\vphantom{#2}\right.\kern-\nulldelimiterspace}}
\newcommand{\rparpht}[2][)]{\mathclose{\left.\kern-\nulldelimiterspace\vphantom{#2}\right#1}}
% \newcommand{\lparpht}[1]{\mathopen{\left(\vphantom{#1}\right.\kern-\nulldelimiterspace}}
% \newcommand{\rparpht}[1]{\mathclose{\left.\kern-\nulldelimiterspace\vphantom{#1}\right)}}
% \newcommand{\paren}[1]{\!\left(#1\right)}
\newcommand{\paren}[1]{\lparpht{#1}#1\rparpht{#1}}
\newcommand{\bigparen}[1]{\bigl(#1\bigr)}
\newcommand{\res}[1]{\left.#1\right|}
\newcommand{\parres}[1]{\res{\paren{#1}}}
\newcommand{\seq}[1]{\lparpht[\{]{#1}#1\rparpht[\}]{#1}}
% \newcommand{\seq}[1]{\left\{#1\right\}}
\newcommand{\set}[1]{\lparpht[\{]{#1}#1\rparpht[\}]{#1}}
% \newcommand{\set}[1]{\left\{#1\right\}}
\newcommand{\setd}[2]{\left\{#1\:\left|\;\vphantom{#1} #2\right.\right\}}
\newcommand{\abs}[1]{\left\lvert#1\right\rvert}
\newcommand{\bigabs}[1]{\bigl\lvert#1\bigr\rvert}
\newcommand{\norm}[1]{\left\lVert#1\right\rVert}
\newcommand{\bignorm}[1]{\bigl\lVert#1\bigr\rVert}
\newcommand{\Bignorm}[1]{\Bigl\lVert#1\Bigr\rVert}
\newcommand{\inner}[2]{\left\langle#1,#2\right\rangle}
\newcommand{\ptinner}[2]{\left(#1,#2\right)}
\newcommand{\biginner}[2]{\bigl\langle#1,#2\bigr\rangle}
\newcommand{\commut}[2]{\left[#1,#2\right]}
\newcommand{\algnorm}[1]{\mathbf N(#1)}
\newcommand{\smod}[1]{\,(\operatorname{mod}\,#1)}
\newcommand{\ceil}[1]{\left\lceil#1\right\rceil}
\newcommand{\Ceil}[1]{\bigl\lceil#1\bigr\rceil}
\newcommand{\floor}[1]{\left\lfloor#1\right\rfloor}
\newcommand{\fracpart}[1]{\left\{#1\right\}}

\newcommand{\tp}[1]{\,{\vphantom{#1}}^*\! #1}
\newcommand{\rtp}[1]{\,{\vphantom{#1}}^t\! #1}

\newcommand{\genby}[1]{\left\langle#1\right\rangle}
\newcommand{\genbyd}[2]{\left\langle#1\:\left|\;\vphantom{#1} #2\right.\right\rangle}

\newcommand{\tmatrix}[4]{\left[\begin{smallmatrix}#1 & #2 \\ #3 & #4\end{smallmatrix}\right]}
\newcommand{\Diag}[1]{\operatorname{diag}\bigl(#1\bigr)}

\newcommand{\cl}{\overline}

\newcommand{\vect}[1]{\underline{#1}}

\newcommand{\conj}{\overline}
\newcommand{\diff}{\partial}
%% \dbar is defined in amsrefs (mathscinet) with out 'lite' option, so
%% \renewcommand is used 
\makeatletter
\@ifundefined{dbar}{
  \newcommand{\dbar}{{\conj\diff}}  
}{
  \renewcommand{\dbar}{{\conj\diff}}
  % \PackageWarningNoLine{marionotations}{`\protect\dbar' is redefined}
  \@latex@warning@no@line{`\protect\dbar' is redefined in mariostdcommands.tex}
}
\makeatother

\NewDocumentCommand{\fdiff}{O{#3} m O{}}{\frac{\diff #1}{\diff #2}}
\newcommand{\dfadj}{\vartheta}
\newcommand{\ddbar}{\diff\dbar}
\newcommand{\iddbar}{\cplxi\ddbar}
\newcommand{\ibddbar}{\ibar\ddbar}

\newcommand{\ddc}{dd^c\mspace{1mu}}

\newcommand{\bdry}{\partial}

\newcommand{\birat}{\mathrel{\dashrightarrow}}
\newcommand{\tendsto}{\mathrel{\rightarrow}}
\newcommand{\xtendsto}[1]{\mathrel{\xrightarrow{#1}}}
\newcommand{\wktendsto}{\mathrel{\rightharpoonup}}
\newcommand{\descendsto}{\mathrel{\searrow}}
\newcommand{\ascendsto}{\mathrel{\nearrow}}
\newcommand{\imply}{\mathrel{\Rightarrow~}}
\newcommand{\isom}{\mathrel{\cong}}
\newcommand{\ctrt}{\mathbin{\lrcorner}}

 %% load mariostdcommands.tex separately
                           %% when uncommented (must put "no-commands"
                           %% in the package option of marionotations
                           %% in that case);

\usepackage[normalem]{ulem} %% used only in review report

\usepackage{bbm}     %% to use \mathbbm (like \mathbb but works also
                     %%                  for natural numbers)

\usepackage[Smaller]{cancel} %%%%% for crossing out argument in math mode via
                             %%%%% the  use of \cancelto 
\renewcommand{\CancelColor}{\color{Gray}}

\usepackage{breakurl}  %% used so that line breaks for contents in
                       %% \url{...} are possible when processed by
                       %% LaTeX instead of pdfLaTeX (e.g. arXiv.org) 

\usepackage{subfiles}

%%%%% Commands for this document %%%%%
\newcommand{\defaultDimension}{n}
\newcommand{\setDefaultDimension}[1]{\renewcommand{\defaultDimension}{#1}}

\newcommand{\defaultAmbientSpace}{X}
\newcommand{\setDefaultAmbientSpace}[1]{\renewcommand{\defaultAmbientSpace}{#1}}

\newcommand{\defaultlcIndex}{\sigma}
\newcommand{\setDefaultlcIndex}[1]{\renewcommand{\defaultlcIndex}{#1}}

\newcommand{\defaultcohDegree}{q}
\newcommand{\setDefaultcohDegree}[1]{\renewcommand{\defaultcohDegree}{#1}}

\newcommand{\defaultlclocus}{D}
\newcommand{\setDefaultlclocus}[1]{\renewcommand{\defaultlclocus}{#1}}

\newcommand{\defaultvphi}{\vphi_F}
\newcommand{\setDefaultvphi}[1]{\renewcommand{\defaultvphi}{#1}}

\newcommand{\defaultpsi}{\psi_D}
\newcommand{\setDefaultpsi}[1]{\renewcommand{\defaultpsi}{#1}}

\newcommand{\defaultMetric}{\omega}
\newcommand{\setDefaultMetric}[1]{\renewcommand{\defaultMetric}{#1}}


% The delimiters for the arguments are carefully chosen so that they
% are consistent among most of the commands (in particular for the
% commands for generating the symbols for the ideal sheaves and
% residue sheaves), namely,
%     <X>        for base space,
%     (S)        for lc locus,
%     |\sigma|   for lc index,
%     {\vphi}    for potential,
%     [\psi]     for $\psi$ function,
%     .{m_k}     for jumping number,
%     +{1}       for increment of lc index (by $1$),
%     -{1}       for decrement of lc index (by $1$).
%     /q/        for anti-holomorphic degree (or hol. and anti-hol. degrees)

\newcommand{\alert}[2][RoyalBlue]{{\color{#1}#2}}

\NewDocumentCommand{\logKX}{
  t{M} %% #1 include M in the tensor product if present
  o    %% #2 replace F \otimes M by the argument when provided
}{K_X \otimes D \otimes \IfNoValueTF{#2}{F \IfBooleanT{#1}{\otimes M}}{#2}}

% \NewDocumentCommand{\vphilist}{
%   D||{\vphi}           %% #1 potentials
%   t{F}                 %% #2 turn potential to "\vphi_F" if present
%   t{M}                 %% #3 add "+\vphi_M" if present
%   d()                  %% #4 extra metric for the (1,0)-forms
%   D<>{\defaultMetric}  %% #5 metric on the ambient space
% }{\IfBooleanTF{#2}{\vphi_F}{#1} \IfBooleanT{#3}{+\vphi_M}, \IfNoValueF{#4}{(#4),} #5}

\NewDocumentCommand{\Ltwo}{ %% the space of L2 sections 
  D//{\bullet,\bullet}      %% #1 the order of forms
  D<>{\defaultAmbientSpace} %% #2 base space
  s                         %% #3 base space is hidden if * is present
  m                         %% #4 coefficient
}{L^{#1}_{(2)}\paren{\IfBooleanF{#3}{#2;} #4}}

% \NewDocumentCommand{\Ltwosp}{
%   t{'}                    %% #1 no preassigned holomorphic degree if present
%   D//{\defaultcohDegree}  %% #2 anti-holomorphic degree
%   t{M}                    %% #3 include M in the coefficient if present
%   o                       %% #4 replace F \otimes M by the argument if provided
%   G{\defaultvphi}         %% #5 potential on line bundle
%   e{_}                    %% #6 metric on the base space or other subscripts
% }{\Ltwo/\IfBooleanF{#1}{\defaultDimension,}#2/*{D \otimes \IfNoValueTF{#4}{F \IfBooleanT{#3}{\otimes M}}{#4}}_{#5 \IfNoValueF{#6}{,#6}}}


%\def\mH{\mathcal{H}}
% \NewDocumentCommand{\Harm}{ %% the space of harmonic forms
%   O{q}
% }{\mathcal{H}^{n,#1}}
\NewDocumentCommand{\Harm}{ %% the space of harmonic forms
  t{'}                      %% #1 no preassigned hol degree if present
  D//{\defaultcohDegree}    %% #2 anti-holomorphic degree
  D<>{\defaultAmbientSpace} %% #3 the base space
  g                         %% #4 the coefficient; will be hidden
                            %%    together with the base space if not provided
  t{,}                      %% #5 separator
  G{\defaultvphi}           %% #6 potential on line bundle
  e{_}                      %% #7 metric on the base space or other subscripts
}{\mathcal{H}^{\IfBooleanF{#1}{\defaultDimension,}#2}\IfNoValueF{#4}{\paren{#3;#4}}_{#6 \IfNoValueF{#7}{,#7}}}


\NewDocumentCommand{\lcIndex}{ %% for displaying the lc index,
                               %% intended to be used internally
  m  %% #1 the basic lc index (\sigma)
  m  %% #2 amount added to the index
  m  %% #3 amount substracted from the index
}{#1\IfNoValueF{#2}{+#2}\IfNoValueF{#3}{-#3}}

\NewDocumentCommand{\lcData}{ %% for displaying lc data in the format
                              %% like "(\vphi_L ; m_k . \psi)"
  G{\defaultvphi}  %% #1 potential or q-psh function
  O{\defaultpsi}   %% #2 lc locus psi function
  e{.}             %% #3 jumping number
}{\paren{#1; \IfNoValueF{#3}{#3 \cdot} #2}}

\NewDocumentCommand{\lcdata}{ %% for displaying lc data in the
                              %% format like "(X,\vphi_L,\psi,m_k)"
  s                %% #1 no parentheses if starred 
  d<>              %% #2 base space
  G{\defaultvphi}  %% #3 potential or q-psh function
  O{\defaultpsi}   %% #4 lc locus psi function
  e{.,}            %% #5 jumping number
                   %% #6 extra components
}{\newcommand{\datalist}{\IfNoValueF{#2}{#2,}#3,#4\IfNoValueF{#5}{,#5}\IfNoValueF{#6}{,#6}}
\IfBooleanTF{#1}{\datalist}{\paren{\datalist}}}



\newcommand{\spHbase}{\mathbb{H}}
\NewDocumentCommand{\spH}{ %% cohomology group with coefficients 
                           %% K_X +F +D \otimes the given sheaf
  D//{\defaultcohDegree}  %% #1 degree of anti-hol form
  t{M}                    %% #2 with 'M' to display M in the %% coefficient
  m                       %% #3 
}{\spHbase^{#1}\paren{\IfBooleanT{#2}{M\otimes}#3}}
% \NewDocumentCommand{\spH}{ %% cohomology group with coefficients 
%                            %% vanishing on \lcc[\sigma]
%   D//{\defaultcohDegree}  %% #1 degree of anti-hol form
%   t{M}                    %% #2 with 'M' to display M in the coefficient
%   s                       %% #3 star for turning to the mlc adjoint ideal sheaf
%   D||{\defaultlcIndex}    %% #4 codim of the lcc defined by the upper ideal
%   t{.}                    %% #5 with '.' to display a quotient ideal
%   D||{#4 -1}              %% #6 codim of the lcc defined by the lower ideal
%   d()                     %% #7 the sheaf replacing the ideal sheaf if non-empty
% }{\spHbase^{#1}\IfNoValueTF{#7}{
%     \begingroup%
%     \newcommand{\upidl}{\IfBooleanTF{#3}{
%         \mtidlof{\vphi_{F \IfBooleanT{#2}{\otimes M}}}
%       }{\aidlof|#4|{\vphi_{F \IfBooleanT{#2}{\otimes M}}}}
%     }% 
%     \paren{\IfBooleanT{#2}{M\otimes}
%       \IfBooleanTF{#5}{
%         \frac{\upidl}{\aidlof|#6|{\vphi_{F \IfBooleanT{#2}{\otimes M}}}}
%       }{\upidl}}
%     \endgroup%
%   }{\paren{\IfBooleanT{#2}{M\otimes}#7}}}


\DeclareMathOperator{\lc}{lc} %% lc centre
\NewDocumentCommand{\lcc}{ %% union of lc centres
                           %% of codimension \sigma
                           %% of (X,D) %%
  D||{\defaultlcIndex}       %% #1 lc index \sigma
  e{+-}                      %% #2,#3
  D<>{\defaultAmbientSpace}  %% #4 base space
  t{'}                       %% #5 '-ed to show lc locus instead of
                             %%    lc data pair
  D(){\defaultlclocus}       %% #6 lc locus 
}{\lc_{#4}^{\lcIndex{#1}{#2}{#3}}\IfBooleanTF{#5}{\paren{#6}}{\lcData}}

\NewDocumentCommand{\lcS}{  %% a local lc centre
  s                       %% #1 symbol with \rs when starred
  D(){\defaultlclocus}    %% #2 symbol for the subvariety
  D||{\defaultlcIndex}    %% #3 codimension
  e{+-}                   %% #4,#5
  d<>                     %% #6 open set where the lc centre lives
  O{p}                    %% #7 index among the \sigma-lc centres
}{\mathtt{\IfBooleanT{#1}{\rs} #2}^{\lcIndex{#3}{#4}{#5}}_{\IfNoValueF{#6}{#6,}#7}}

\NewDocumentCommand{\PRes}{ %% Poincare Residue map
  O{}      %% subvariety
  d()      %% forms from the domain
}{\mathcal R_{#1}\IfNoValueF{#2}{\paren{#2}}}

\NewDocumentCommand{\HRes}{ %% Harmonic residue
  d()   %% #1 harmonic form
}{\mathfrak{R}\IfNoValueF{#1}{\paren{#1}}}

\newcommand{\defidlof}[1]{\mathcal{I}_{#1}}  %% defining ideal of (a set)
\NewDocumentCommand{\mtidlof}{   %% multiplier ideal of (a potential)
  O{}      %% #1 base space (for compatibility)
  D<>{#1}  %% #2 base space
  m        %% #3 potential / psh function
}{\multidl_{#2}\paren{#3}} 

% \NewDocumentCommand{\presidlof}{  %% multiplier ideal sheaf on the sum of
%                                   %% \sigma-lc centres
%   D||{\sigma}   %% codim of lc centres or supporting lc locus
%   m             %% potential or q-psh function
% }{\rs{\sheaf R}_{#1}\paren{#2}}

\NewDocumentCommand{\residlof}{  %% multiplier ideal sheaf on the
                                 %% union of \sigma-lc centres
  D||{\defaultlcIndex}   %% #1 codim of lc centres or supporting lc
                         %%    locus
  e{+-}                  %% #2,#3
  d<>                    %% #4 base space
  s                      %% #5 display the symbol without arguments when starred
  %%% input to \lcData
  % G{\defaultvphi}      %% #6 potential or q-psh function
  % O{\defaultpsi}       %% #7 lc locus psi function
  % e{.}                 %% #8 jumping number  
}{\sheaf R_{\IfNoValueTF{#4}{}{#4,} \lcIndex{#1}{#2}{#3}}\IfBooleanF{#5}{\lcData}}


\NewDocumentCommand{\Adjidlof}{
  D||{\defaultlcIndex}       %% #1 codim of lc centres under concern
  D<>{\defaultAmbientSpace}  %% #2 base space
  D(){\defaultlclocus}       %% #3 lc locus
  m                          %% #4 potential or ideal
}{\operatorname{\mathit{Adj}}^{#1}_{\paren{#2,#3}}\paren{#4}}


\NewDocumentCommand{\aidlof}{
  D||{\defaultlcIndex}   %% #1 codim of lc centres under concern
  e{+-}                  %% #2,#3
  d<>                    %% #4 base space
  s                      %% #5 display the symbol without arguments when starred
  %%% input to \lcData
  % G{\defaultvphi}        %% #6 potential or ideal
  % O{\defaultpsi}         %% #7 defining function of the lc locus
  % e{.}                   %% #8 jumping number
}{\sheaf{J}_{\!\IfNoValueTF{#4}{}{#4,} \lcIndex{#1}{#2}{#3}}\IfBooleanF{#5}{\lcData}}

\NewDocumentCommand{\faidlof}{
  D||{\defaultlcIndex}   %% #1 codim of lc centres in numerator
  e{+-}                  %% #2,#3
  t{/}                   %% #4 a separator for arguments
  D||{\defaultlcIndex}   %% #5 codim of lc centres in denominator
  e{+-}                  %% #6,#7
  % d<>                    %% #8 base space
  % s                      %% #9 display the symbol without arguments when starred
  %%% input to \lcData
  % G{\defaultvphi}        %% #10 potential or ideal
  % O{\defaultpsi}         %% #11 defining function of the lc locus
  % e{.}                   %% #12 jumping number
}{\fracAidlof{\lcIndex{#1}{#2}{#3}}{\lcIndex{#5}{#6}{#7}}}

\NewDocumentCommand{\fracAidlof}{
  m                  %% #1 lcIndex in numerator
  m                  %% #2 lcIndex in denominator
  d<>                %% #3 base space
  s                  %% #4 display the symbol without arguments when starred
  G{\defaultvphi}    %% #5 potential or ideal
  O{\defaultpsi}     %% #6 defining function of the lc locus
  e{.}               %% #7 jumping number
}{\frac{
    \aidlof|#1|<#3>*\IfBooleanF{#4}{\lcData{#5}[#6].{#7}}
  }{
    \aidlof|#2|<#3>*\IfBooleanF{#4}{\lcData{#5}[#6].{#7}}
  }}


\NewDocumentCommand{\lcV}{ %% measure on lc centres
  D||{\defaultlcIndex}    %% #1 codim of supporting lc centres
  D//{\defaultvphi}       %% #2 potential for bundle valued section
  d()                     %% #3 metric on the ambient space
  e{^}                    %% #4 jumping number
  O{\defaultpsi}          %% #5 defining function (potential) of subvariety 
}{\:d\operatorname{lcv}^{#1\IfNoValueF{#4}{,\paren{#4}}}_{\IfNoValueF{#3}{#3,}#2}\left[#5\right]}

\NewDocumentCommand{\Ohvol}{ %% Ohsawa measure %%
  D//{\defaultvphi} %% #1 potential for bundle valued section
  d()               %% #2 metric on the ambient space
  O{\defaultpsi}    %% #3 defining function of subvariety
}{\dvol_{\IfNoValueF{#2}{#2,}#1}\left[#3\right]} 


\newcommand{\dvol}{\:d\vol}


\NewDocumentCommand{\lcDataNormSubscript}{
  %% for displaying lc data in the format
  %% like "X, \vphi_L , m_k.\psi, \sigma", which is mainly used for
  %% subscript in a norm
  d<>                   %% #1 base space
  s                     %% #2 no potential and psi function when starred
  G{\defaultvphi}       %% #3 potential or q-psh function
  O{\defaultpsi}        %% #4 lc locus psi function
  e{.}                  %% #5 jumping number
  D||{\defaultlcIndex}  %% #6 lc Index
  e{+-}                 %% #7,#8
}{\IfNoValueF{#1}{#1,}
  \IfBooleanF{#2}{#3, \IfNoValueF{#5}{#5 \cdot} #4,}
  \lcIndex{#6}{#7}{#8}}


\newcommand{\RTFsym}{\mathfrak{F}} 
\NewDocumentCommand{\RTF}{ %% residue transform function
  s          %% #1 adding \smash[t] when starred
  G{\RTFsym} %% #2 symbol body
  o          %% #3 general superscript
  >{\SplitArgument{1}{,}} d<> %% #4 superscript in inner product
  d||        %% #5 superscript in \abs{}^2
  D(){\eps}  %% #6 for adding variable (\eps)
  t{,}       %% #7 separator
}{%
  \begingroup%
    \newif\ifsmasht%
    \IfBooleanTF{#1}{\smashttrue}{\smashtfalse}%
    \newif\ifboolup%
    \booluptrue%
    \IfNoValueT{#3}{\IfNoValueT{#4}{\IfNoValueT{#5}{\boolupfalse}}}%
    \newcommand{\supsrptstr}{\IfNoValueF{#3}{#3}\IfNoValueF{#4}{\inner#4}\IfNoValueF{#5}{\abs{#5}^2}}
    \newcommand{\RTFvar}{#6}
    #2\RTFprocess
}

\NewDocumentCommand{\RTFprocess}{
  o                     %% #1 overwrite subscript if given
  d<>                   %% #2 base space
  t{,}                  %% #3 with potential and psi function when ,-ed
  G{\defaultvphi}       %% #4 potential or q-psh function
  O{\defaultpsi}        %% #5 lc locus psi function
  e{.}                  %% #6 jumping number
  D||{\defaultlcIndex}  %% #7 lc Index
  e{+-}                 %% #8,#9
}{\newcommand{\subsrptstr}{%
    \IfNoValueTF{#1}{
    \IfNoValueF{#2}{#2,}
    \IfBooleanT{#3}{#4,#5,\IfNoValueF{#6}{#6,}}
    \lcIndex{#7}{#8}{#9}}{#1}}%
  \newcommand{\srptstr}{\cramped{{}^{\supsrptstr}%
      \ifboolup _
      \fi{\ifboolup\displaystyle\fi\paren{\RTFvar}%
          \ifboolup {\scriptstyle \subsrptstr} \else _{\subsrptstr} \fi%
        }}}%
  \ifboolup%
    \ifsmasht%
      \smash[t]{
        \raisebox{\depthof{$\srptstr$} * \real{0.3}}{$\srptstr$}%
      }%
    \else%
      \raisebox{\depthof{$\srptstr$} * \real{0.3}}{$\srptstr$}%
    \fi%
  \else%
    \srptstr%
  \fi%
  \endgroup%
}
% \NewDocumentCommand{\RTF}{ %% residue transform function
%   s          %% #1 adding \smash[t] when starred
%   G{\RTFsym} %% #2 symbol body
%   d//        %% #3 for adding superscript k for k-RTF
%   o          %% #4 general superscript
%   >{\SplitArgument{1}{,}} d<> %% #5 superscript in inner product
%   d||        %% #6 superscript in \abs{}^2
%   d()        %% #7 for adding variable (\eps)
%   o          %% #8 subscript for the codimension \sigma
% }{%
%   \begingroup%
%     \newif\ifboolup%
%     \booluptrue%
%     \IfNoValueT{#4}{\IfNoValueT{#5}{\IfNoValueT{#6}{\boolupfalse}}}%
%     \IfNoValueT{#7}{\boolupfalse}%
%     \newcommand{\srptstr}{\cramped{{}^{\IfNoValueF{#4}{#4}\IfNoValueF{#5}{\inner#5}\IfNoValueF{#6}{\abs{#6}^2}}%
%       \ifboolup _
%       \fi{\ifboolup\displaystyle\fi\IfNoValueF{#7}{\paren{#7}}\IfNoValueF{#8}{%
%           \ifboolup {\scriptstyle #8} \else _{#8} \fi%
%         }}}}%
%     \ifboolup%
%       \IfBooleanTF{#1}{
%         \smash[t]{
%           \IfNoValueF{#3}{{}^{#3}}#2\raisebox{\depthof{$\srptstr$} * \real{0.3}}{$\srptstr$}%
%         }%
%       }{\IfNoValueF{#3}{{}^{#3}}#2\raisebox{\depthof{$\srptstr$} * \real{0.3}}{$\srptstr$}}%
%     \else%
%       \IfNoValueF{#3}{{}^{#3}}#2\srptstr%
%     \fi%
%   \endgroup%
% } 

\def\RTI{\RTF{\mathfrak{I}}}


\NewDocumentCommand{\mtlog}{O{e} d() D||{\defaultpsi}}{\log\!#1^{\paren{#2}}\abs{#3}}
\NewDocumentCommand{\slog}{O{e} D||{\defaultpsi}}{\log\abs{#1 #2}}
\NewDocumentCommand{\dlog}{O{e} D||{\defaultpsi}}{\mtlog[#1](2)|#2|}


\NewDocumentCommand{\logpole}{ %% log-pole in the residue transform
                               %% function
  D||{\defaultpsi}       %% #1 log singularity defining function
  D//{\defaultlcIndex}   %% #2 codim of lc centres in question
  E{.^}{{e}{1+\eps}}     %% #3 multiplicative constant in logarithm 
                         %% #4 exponent in the log-psi term
  s                      %% #5 no parentheses and exponent on log|\psi| when starred
}{\abs{#1}^{#2} \IfBooleanTF{#5}{\slog[#3]|#1|}{\paren{\slog[#3]|#1|}^{#4}}}

\DeclareFontFamily{OMX}{MnSymbolE}{}
\DeclareSymbolFont{MnLargeSymbols}{OMX}{MnSymbolE}{m}{n}
\SetSymbolFont{MnLargeSymbols}{bold}{OMX}{MnSymbolE}{b}{n}
\DeclareFontShape{OMX}{MnSymbolE}{m}{n}{
    <-6>  MnSymbolE5
   <6-7>  MnSymbolE6
   <7-8>  MnSymbolE7
   <8-9>  MnSymbolE8
   <9-10> MnSymbolE9
  <10-12> MnSymbolE10
  <12->   MnSymbolE12
}{}
\DeclareFontShape{OMX}{MnSymbolE}{b}{n}{
    <-6>  MnSymbolE-Bold5
   <6-7>  MnSymbolE-Bold6
   <7-8>  MnSymbolE-Bold7
   <8-9>  MnSymbolE-Bold8
   <9-10> MnSymbolE-Bold9
  <10-12> MnSymbolE-Bold10
  <12->   MnSymbolE-Bold12
}{}
\DeclareMathDelimiter{\llangle}{\mathopen}%
{MnLargeSymbols}{'164}{MnLargeSymbols}{'164}
\DeclareMathDelimiter{\rrangle}{\mathclose}%
{MnLargeSymbols}{'171}{MnLargeSymbols}{'171}


\newcommand{\iinner}[2]{\left\llangle#1,#2\right\rrangle}
\newcommand{\eqcls}[1]{\left[#1\right]}


\NewDocumentCommand{\idxup}{ %% operator for raising indices via a
                             %% hermitian metric on X
  m                  %% #1 the differential form whose indices to be raised
  O{\defaultMetric}  %% #2 the hermitian metric on X
  t{,}               %% #3 separator
  o                  %% #4 extra superscripts
  s                  %% #5 smash the vertical spacing on the top of the metric if present
  t{.}               %% #6 with contraction operator \ctrt if '.'-ed
}{\paren{#1}^{
    % \mathrlap{
    \!\IfBooleanTF{#5}{\smash[t]{#2}}{#2}\IfNoValueF{#4}{, #4}
    % }
    % \makebox[\maxof{\widthof{$#2$}-\widthof{$\!\omega$}}{0pt}]{}
  }\IfBooleanT{#6}{\!\!\ctrt}}
% \NewDocumentCommand{\idxup}{ %% operator for raising indices via a
%                              %% hermitian metric on X
%   m                  %% #1 the differential form whose indices to be raised
%   O{*}               %% #2 the hermitian metric on X
%   s                  %% #3 smash the vertical spacing on the top of the metric if present
% }{\paren{#1}^{
%     \!\IfBooleanTF{#3}{\smash[t]{#2}}{#2}
%     % \makebox[\maxof{\widthof{$\scriptstyle #2$}-\widthof{$\!\omega$}}{0pt}]{}
%   }
% }

\newcommand{\dbadj}{\dbar^{\smash{\mathrlap{*}\;\:}}}


\NewDocumentCommand{\dep}{t{;} d<> O{\nu} m}{#4\IfBooleanTF{#1}{_}{^}{\IfNoValueF{#2}{#2\:}(#3)}}

\NewDocumentCommand{\sm}{s m}{{#2}\IfBooleanTF{#1}{_}{^}\text{sm}}

\newcommand{\tlog}{{\text{log}}}


\NewDocumentCommand{\idx}{ %% multi-indices
  O{i} %% #1 symbol of the indices
  m    %% #2 starting subscript
  o    %% #3 additional stuff to add before \dotsm
  t{.} %% #4 display "\dotsm" if '.'-ed
  t{,} %% #5 display ",\dots," if ','-ed
  o    %% #6 additional stuff to add after \dotsm
  m    %% #7 ending subscript
}{{#1}_{#2} \IfNoValueF{#3}{#3}
  \IfBooleanT{#4}{\dotsm} \IfBooleanT{#5}{,\dots,}
  \IfNoValueF{#6}{#6} {#1}_{#7}}



\newcommand{\charfct}{\mathbbm 1}


\newcommand{\cvr}[1]{\mathfrak{#1}} %% set of covering subsets
% \newcommand{\rs}[1]{\widetilde{#1}} %% putting ~ on objects on the
%                                     %% log-resolution %%
\NewDocumentCommand{\rs}{ %% putting ~ on objects on the
                          %% log-resolution %%
  s  %% when * is given, \smash[t] is applied
  m  %% the main object 
}{\IfBooleanTF{#1}{\smash[t]{\widetilde{#2}}}{\widetilde{#2}}}

% \NewDocumentCommand{\clt}{m}{\widetilde{#1}} %% element in complete space
% \NewDocumentCommand{\clomega}{O{\omega}}{{\clt{#1}}} %% complete metric

\newcommand{\BK}{\text{(BK)}}
\newcommand{\tBK}{\text{(tBK)}}
\DeclareMathOperator{\Ann}{Ann}  %% Annihilator 
\DeclareMathOperator{\mlc}{mlc} %% minimal lc centre
\DeclareMathOperator{\sym}{sym} %% symmetric polynomial
\newcommand{\Diff}{\operatorname{Diff}^*} %% general different (adjunction formula)

\newcommand{\sect}[1][s]{\mathtt{#1}} %% canonical section
\newcommand{\bphi}{\boldsymbol{\vphi}}
\newcommand{\bphip}[1][p]{\res\bphi_{#1}} %% retract-extension of
                                          %% \bphi from lc centre
                                          %% S^\sigam_p
\newcommand{\btau}{\boldsymbol{\tau}}
\newcommand{\shfP}{\sheaf P}  %% polar ideal sheaf
\NewDocumentCommand{\cbn}{  %% group of combinations
  D//{\defaultlcIndex_V}
  D||{\defaultlcIndex}
}{\mathfrak{C}^{#1}_{#2}} 
\NewDocumentCommand{\Iset}{  %% index set for lc centres on log-resolution
  D||{\defaultlcIndex}    %% #1
  e{+-}                   %% #2,#3
  O{\defaultlclocus}      %% #4
  d()                     %% #5 open set on which the index set is valid
}{I^{\lcIndex{#1}{#2}{#3}}_{#4}\IfNoValueF{#5}{\paren{#5}}} 
% }{I^{#1\IfNoValueF{#2}{+#2}\IfNoValueF{#3}{-#3}}_{#4}\IfNoValueF{#5}{\paren{#5}}} 

%%%%%%%%%%%%%%%%%%%%%%%%%%%%%%%%%%%%%%

\ifcsname defineNoThmInMarionotations\endcsname
  \relax
\else 

  \newtheorem{THMprop}{Proposition}[subsection]
  \newtheorem{THMlemma}[THMprop]{Lemma}
  \newtheorem{THMthm}[THMprop]{Theorem}
  \newtheorem{THMcor}[THMprop]{Corollary}
  % \newtheorem{SNCassumption}[THMprop]{Snc assumption}
  % \newtheorem{SNCassumptionx}{Snc assumption}
  % \renewcommand{\theSNCassumptionx}{\theSNCassumption${}^*$}

  % \newtheorem{definition-thm}[THMprop]{Definition-Theorem}

  \newtheorem{THMconjecture}[THMprop]{Conjecture}
  \newtheorem*{THMclaim}{Claim}

  \def\makeparenletter{\catcode`\(=11 \catcode`\)=11 }
  \def\makeparenother{\catcode`\(=12 \catcode`\)=12 }
  \def\makeparenactive{\catcode`\(=\active\catcode`\)=\active}

  \makeparenactive
  \NewDocumentEnvironment{textupparenenvir}{}{
    %%%%% This code may cause error when parentheses appear in places
    %%%%% where macro is not accepted, like \ref{...} or optional
    %%%%% arguments of enumerate. 
    % \catcode1=12
    % \catcode2=12
    % \mathcode1=\the\mathcode`\(
    % \delcode1=\the\delcode`\(
    % \mathcode2=\the\mathcode`\)
    % \delcode2=\the\delcode`\)

    % \begingroup
    % \lccode`\~=`\^^A
    % \lowercase{\endgroup
    % \everymath\expandafter{\the\everymath\let(^^28\let)^^29}
    % \everydisplay\expandafter{\the\everydisplay\let(^^28\let)^^29}
    % }

    \everymath\expandafter{\makeparenother}
    \everydisplay\expandafter{\makeparenother}

    \def({\textup{\char`\(}}
    \def){\textup{\char`\)}}

    \makeparenactive
    % \let\zzzlabel\label
    % \let\zzzref\ref
    % \let\zzznewlabel\newlabel

    % \def\label{\makeparenletter\wwwlabel}
    % \def\ref{\makeparenletter\wwwref}
    % \def\newlabel{\makeparenletter\wwwnewlabel}

    % \def\wwwlabel#1{\makeparenactive\zzzlabel{#1}}
    % \def\wwwref#1{\makeparenactive\zzzref{#1}}
    % \def\wwwnewlabel#1{\makeparenactive\zzznewlabel{#1}}
  }{\makeparenother}
  \makeparenother

  \NewDocumentEnvironment{prop}{ +o }{
    \IfNoValueTF{#1}{\begin{THMprop}}{\begin{THMprop}[{#1}]}
      \begin{textupparenenvir}
  }{
      \end{textupparenenvir}
    \end{THMprop}
  }

  \NewDocumentEnvironment{lemma}{ +o }{
    \IfNoValueTF{#1}{\begin{THMlemma}}{\begin{THMlemma}[{#1}]}
      \begin{textupparenenvir}
  }{
      \end{textupparenenvir}
    \end{THMlemma}
  }

  \NewDocumentEnvironment{thm}{ +o }{
    \IfNoValueTF{#1}{\begin{THMthm}}{\begin{THMthm}[{#1}]}
      \begin{textupparenenvir}
  }{
      \end{textupparenenvir}
    \end{THMthm}
  }

  \NewDocumentEnvironment{cor}{ +o }{
    \IfNoValueTF{#1}{\begin{THMcor}}{\begin{THMcor}[{#1}]}
      \begin{textupparenenvir}
  }{
      \end{textupparenenvir}
    \end{THMcor}
  }

  \NewDocumentEnvironment{conjecture}{ +o }{
    \IfNoValueTF{#1}{\begin{THMconjecture}}{\begin{THMconjecture}[{#1}]}
      \begin{textupparenenvir}
  }{
      \end{textupparenenvir}
    \end{THMconjecture}
  }

  \NewDocumentEnvironment{claim}{ +o }{
    \IfNoValueTF{#1}{\begin{THMclaim}}{\begin{THMclaim}[{#1}]}
      \begin{textupparenenvir}
  }{
      \end{textupparenenvir}
    \end{THMclaim}
  }

  \theoremstyle{remark}
  \newtheorem{remark}[THMprop]{Remark}

  \theoremstyle{definition}
  \newtheorem{definition}[THMprop]{Definition}
  \newtheorem{example}[THMprop]{Example}
  \newtheorem{notation}[THMprop]{Notation}

  \numberwithin{equation}{subsection}
  \renewcommand{\theequation}{eq$\,$\thesubsection.\arabic{equation}}


  
\fi

\allowdisplaybreaks  %% allow multi-line equations to spread across
                     %% pages 



%%% Local Variables:
%%% mode: latex
%%% TeX-master: "Injectivity-Fujino"
%%% End:

\ifx\pdftexversion\undefined
  \renewcommand{\ibar}{{\raisebox{-0.9ex}{$\mathchar'26$}\mkern-6.7mu i}}
\fi


% \usepackage{showkeys}

%%%%% End of preamble %%%%%%%%%%%%%%%%%%%%%%%%%%%%%%%%%%%%%%%%%%%%%%%%

\begin{document}

\citealias{Amb03}{Ambro_quasi-log-var}
\citealias{Amb14}{Ambro_injectivity}
\citealias{Eno90}{Enoki}
\citealias{EV92}{Esnault&Viehweg_book}
\citealias{Fuj11}{Fujino_log-MMP}
\citealias{Fuj12b}{Fujino_vanishing-thms}
\citealias{Fuj13a}{Fujino_injectivity-II}
\citealias{Fuj13b}{Fujino_injectivity-hodge-theoretic}
\citealias{Fuj15b}{Fujino_survey}



%%%%%
%%%%% File name  : titleinfo.tex
%%%%% Author     : Mario Chan
%%%%% Date       : 25th July, 202 (original: 13th December, 2021 (original: 04th November, 2020))
%%%%% Description: This file contains the info needed for maketitle
%%%%%              for the project "Injectivity-Fujino".
%%%%%
%%
%%%
\newcommand{\titlestr}{%
  % (Provisional)
  % A solution to the Fujino conjecture: injectivity theorem for
  % log-canonical pairs \\ on compact K\"ahler manifolds%
  An injectivity theorem on snc compact K\"ahler spaces: \\
  an application of the theory of
  harmonic integrals on log-canonical centers via adjoint ideal
  sheaves%
}

\newcommand{\shorttitlestr}{%
  An injectivity theorem on snc spaces%
}

\newcommand{\MCname}{Tsz On Mario Chan}
\newcommand{\MCnameshort}{Mario Chan}
\newcommand{\MCemail}{mariochan@pusan.ac.kr}

\newcommand{\YJname}{Young-Jun Choi}
\newcommand{\YJnameshort}{Young-Jun Choi}
\newcommand{\YJemail}{youngjun.choi@pusan.ac.kr}

\newcommand{\PNUAddressstr}{%
  Dept.~of Mathematics, Pusan National
  University, Busan 46241, South Korea%
}


\newcommand{\ShMname}{Shin-ichi Matsumura}
\newcommand{\ShMnameshort}{Shin-ichi Matsumura}
\newcommand{\ShMemail}{mshinichi0@gmail.com, mshinichi-math@tohoku.ac.jp}

\newcommand{\TohokuAddressstr}{%
  Mathematical Institute, Tohoku University, 6-3, Aramaki Aza-Aoba,
  Aoba-ku, Sendai 980-8578, Japan%
}


\newcommand{\subjclassstr}[1][,]{%
  32J25 (primary)#1  %% Transcendental methods of algebraic geometry (complex-analytic aspects) 
  32Q15#1   %% 	Kähler manifolds
  14B05 (secondary)%   %% Singularities in algebraic geometry
  % 14E30 (secondary)%   %% Minimal model program (Mori theory, extremal rays)
}

\newcommand{\keywordstr}[1][,]{%
  $L^2$ injectivity#1
  adjoint ideal sheaf#1
  multiplier ideal sheaf#1
  log-canonical center%
}

\newcommand{\dedicatorystr}{%
}

\newcommand{\thankstr}{%
}

%%% Local Variables:
%%% mode: latex
%%% TeX-master: "Injectivity-Fujino"
%%% coding: utf-8
%%% End:


\title[\shorttitlestr]{\titlestr}
 
\author[\MCnameshort]{\MCname}
\email{\MCemail}
% \address{\addressstr}
% \curraddr{}

\author{\YJname}
\email{\YJemail}
\address{\PNUAddressstr}

\author{\ShMname}
\email{\ShMemail}
\address{\TohokuAddressstr}


% \thanks{\thankstr}
 
\subjclass[2020]{\subjclassstr}

\keywords{\keywordstr}

% \dedicatory{\dedicatorystr}

% \begin{abstract}
%   \begin{abstract}

The Fast Reciprocal Square Root Algorithm is a well-established approximation technique consisting of two stages: first, a coarse approximation is obtained by manipulating the bit pattern of the floating point argument using integer instructions, and second, the coarse result is refined through one or more steps, traditionally using Newtonian iteration but alternatively using improved expressions with carefully chosen numerical constants found by other authors. The algorithm was widely used before microprocessors carried built-in hardware support for computing reciprocal square roots. At the time of writing, however, there is in general no hardware acceleration for computing other fixed fractional powers. This paper generalises the algorithm to cater to all rational powers, and to support any polynomial degree(s) in the refinement step(s), and under the assumption of unlimited floating point precision provides a procedure which automatically constructs provably optimal constants in all of these cases. It is also shown that, under certain assumptions, the use of monic refinement polynomials yields results which are much better placed with respect to the cost/accuracy tradeoff than those obtained using general polynomials. Further extensions are also analysed, and several new best approximations are given.

\end{abstract}

% \end{abstract} 

%%%SM's notation: I will change them to Mario's notation later 
% \newcommand{\Ker}[0]{\operatorname{Ker}}
\let\Ker\ker
\newtheorem{step}{Step}
%%%




%%choiyj's macros
\def\del{\partial}
\def\we{\wedge}
\def\ov{\overline}
\newcommand{\pd}[2]{\frac{\partial#1}{\partial#2}}
%%





\date{\today} 

\maketitle

%%%%% End of Top matter %%%%%%%%%%


\section{Introduction}\label{sec:intro}

{
  \let\thesubsection\thesection
  
  % This paper studies an analytic aspect of higher cohomology groups of adjoint bundles for lc $($log canonical$)$ pairs
  % aiming to solve Fujino's conjecture on the injectivity theorem as a benchmark. 
  This paper studies an analytic aspect of higher cohomology groups of adjoint bundles
  for log-canonical (lc) pairs aiming to solve Fujino's conjecture, 
  the injectivity theorem for lc pairs on compact K\"ahler manifolds, 
  following the line of Enoki's proof. 
  This is achieved by developing the theory of harmonic integrals
  on lc centers using the analytic adjoint ideal sheaves and the
  associated residue techniques.


  The injectivity theorem, a generalization of the Kodaira vanishing theorem to semi-positive line bundles, 
  plays an important role in higher dimensional algebraic geometry. 
  After the original Koll\'ar's injectivity theorem \cite{Kollar_injectivity} had been proved 
  for semi-ample line bundles on smooth projective varieties, 
  Enoki \cite{Eno90} generalized Koll\'ar's injectivity theorem 
  to semi-positive line bundles on compact K\"ahler manifolds. 
  Koll\'ar's proof is based on theory of Hodge structures, whereas
  Enoki's proof is based on the theory of harmonic integrals, a more
  well-suited and flexible technique in the complex analytic situation. 

  Ambro and Fujino generalized Koll\'ar's theory to varieties with lc
  singularities via the theory of mixed Hodge structures,  
  motivated by applications to birational geometry (see \cite{Amb03, Amb14, EV92, Fuj11, Fuj12b, Fuj13b}). 
  % \mariocomment{To SM: please
  % check if these references links to the correct papers. Change the
  % $\backslash$\texttt{citealias} commands if you wish.}% 
  % The works of Ambro and Fujino can be expected to
  It is expected that their works can also be generalized in the same line as Enoki's
  by developing an analytic treatment to lc singularities. 
  Motivated by this expectation, Fujino posed the conjecture below. 
  (Set $\ibar := \ibardefn$ \ibarfootnote\ and let $D$ be a reduced divisor for the
  rest of this article.)



  \begin{conjecture}[{Fujino's conjecture, \cite[Conjecture
      2.21]{Fuj15b}, cf.~\cite[Problem 1.8]{Fuj13a}}] 
    \label{conj:fujino}

    Let $X$ be a compact K\"ahler manifold and
    $D=\sum_{i=1}^{N}D_{i}$ be a simple-normal-crossing
    (snc) divisor on $X$.  Let $F$ be a semi-positive line bundle on
    $X$ (i.e.~it admits a smooth Hermitian metric $h_{F}$ with
    $\ibar\Theta_{h_F}(F) \geq 0$).  Consider a section
    $s \in H^{0}(X, F^{\otimes m})$ whose zero locus $s^{-1}(0)$
    contains no lc centers of the pair $(X,D)$ (i.e.~connected
    components of non-empty intersection
    $D_{i_{1}}\cap \cdots \cap D_{i_{k}}$ of the irreducible
    components $\{D_{i}\}_{i=1}^{N}$).  Then, the multiplication map
    induced by the tensor product with $s$
    \begin{equation*}
      H^{q}\paren{X, K_{X} \otimes D \otimes F}
      \xrightarrow{\otimes s} 
      H^{q}(X, K_{X} \otimes D \otimes F^{\otimes (m+1)} )
    \end{equation*}
    is injective for every $q$.
  \end{conjecture}

  The analytic theory corresponding to Koll\'ar's theory has been established for klt singularities 
  (see \cite{Cao&Demailly&Matsumura, Fujino&Matsumura, Gongyo&Matsumura,
    Matsumura_injectivity-survey, Matsumura_injectivity}).
  % but not for lc singularities. 
  Therefore, it remains to develop an analytic treatment to handle the
  lc singularities.
  % the analytic theory corresponding to the works of Ambro and Fujino
  % is interesting in terms of  studying the techniques of analytically treating lc singularities or mixed Hodge theory than just generalizing it. 


  The cases of $\dim X=2$ and plt pairs of arbitrary dimension have been
  solved in \cite{Matsumura_injectivity-lc,
    Matsumura_rel-vanishing-w-nd} (see also \cite{Chan&Choi_injectivity-I}). 
  A full solution to Fujino's conjecture is given recently by
  Junyan Cao and Mihai P\u{a}un \cite{Cao&Paun_LC-inj}.
  In this paper, independent of the results in \cite{Cao&Paun_LC-inj},
  we prove a {\textit{generalized version}} of Fujino's conjecture  
  (Theorem \ref{thm:main}) 
  % by developing the theory of harmonic integrals on simple normal corssing divisors. 
  by applying the theory of harmonic integrals on lc centers of the
  given lc pair.
  Fujino's conjecture is then a direct consequence of Theorem \ref{thm:main}. 
  

  \begin{thm}[Main Result]\label{thm:main}
    Let $X$ be a compact K\"ahler manifold  and 
    $D=\sum_{i=1}^{N}D_{i}$ be an snc divisor on $X$. 
    % such that each component $D_{i}$ is compact. 
    Let $F$ (resp.~$M$) be a line bundle on $X$ 
    with a smooth Hermitian metric $h_{F}$  (resp.~$h_{M}$) 
    such that 
    \begin{equation*}
      \ibar\Theta_{h_F}(F)\geq 0 \quad  \text{ and } \quad
      % \sqrt{-1}(\Theta_{h_F}(F)-t \Theta 
      % _{h_M}(M))\geq 0
      % -C\omega \leq
      \ibar\Theta_{h_M}(M) \leq C \ibar\Theta_{h_F}(F)
      \quad \text{ for some } C>0 \; . 
    \end{equation*}
    % (that is, $D_{i} \cap D_{j} = \emptyset$ for $i \not = j$ 
    % for the irreducible decomposition $D = \sum_{i\in I}D_{i}$). 
    Let $s$ be a  section of $M$  
    such that the zero locus $s^{-1}(0)$ 
    contains no lc centers of the lc pair $(X,D)$.
    Then, the multiplication map induced by the tensor product with $s$
    \begin{equation*}
      H^q(D, K_D \otimes F)
      \xrightarrow{\otimes s } 
      H^q(D, K_D \otimes F\otimes M)
    \end{equation*} 
    is injective for every $q$. 
  \end{thm}

  It can be seen from the proof that the compactness of $X$ in Theorem
  \ref{thm:main} is not necessary as soon as $D$ consists of only finitely many
  irreducible components which are compact.

  \begin{cor}[Solution to Fujino's conjecture]\label{cor:main}
    Conjecture \ref{conj:fujino} is true. 
  \end{cor}


  % Our paper differs from \cite{Cao&Paun_LC-inj} in the following points: 
  % The method of \cite{Cao&Paun_LC-inj} is based on the $L^{2}$-theory of $\dbar$-equations, 
  % whereas our method is based on the theory of harmonic integrals in the same line as in Enoki's work; 
  % specifically, we extend a technique of harmonic differential forms on smooth varieties to simple normal corssing divisors.

  Our proof differs from the one in \cite{Cao&Paun_LC-inj} in the following way.
  While both works make use of (some variant of) the Hodge
  decomposition for $L^2$ forms, Cao and P\u aun prove in
  \cite{Cao&Paun_LC-inj} a Hodge decomposition for $L^2$ forms with
  respect to a K\"ahler metric with conic singularities, which induces
  a Hodge decomposition on currents (which is called the Kodaira--de
  Rham decomposition in \cite{Cao&Paun_LC-inj}) in which the Green
  kernel has controllable singularities.

  For the sake of explanation, let $u$ be an $D\otimes F$-valued
  $(n,q)$-form representing a class in $\cohgp q[X]{\logKX}$
  % ($X$ being compact here)
  such that the class of $s u$ is $0$ in $\cohgp
  q[X]{\logKX M}$.
  Let also $\sect_D$ be a canonical section of $D$.
  Under our notation, the current that is under consideration in
  \cite{Cao&Paun_LC-inj} is $\frac{u}{\sect_D}$, which is not
  necessarily $L^2$ on $X$.
  Using the fact that $\eqcls{su} = 0$ in $\cohgp q[X]{\logKX M}$,
  Cao and P\u aun obtain $\frac{u}{\sect_D} =\dbar\theta + D'_{h_F}
  \beta_1 +\ibar\Theta_{h_F} \wedge \beta_2$, where $\theta$ is
  smooth while $\beta_1$ and $\beta_2$ have log-poles along
  $D+s^{-1}(0)$ (assumed to have only snc).
  It then follows from \cite{Cao&Paun_LC-inj}*{Thm.~1.1} (which
  makes use of the Hodge/Kodaira--de Rham decomposition) and the
  positivity $\ibar\Theta_{h_F} \geq 0$ that $u$ (or $u -\sect_D
  \dbar\theta$) is $\dbar$-exact.

  In our case, we make use of the residue exact sequences of adjoint
  ideal sheaves and the associated residue computation to reduce the
  setup to the union of \emph{$\sigma$-lc centers} of $(X,D)$ (i.e.~lc centers of
  codimension $\sigma$ in $X$, when $(X,D)$ is log-smooth and lc).
  Since each $\sigma$-lc center is a compact K\"ahler manifold, we
  have the Hodge decomposition (thus $L^2$ Dolbeault isomorphism and
  harmonic theory) at our disposal.
  Moreover, our reduction brings the setup essentially to the one in
  \cite{Matsumura_injectivity-lc}*{Thm.~1.6} or
  \cite{Chan&Choi_injectivity-I}*{Thm.~1.2.1} (corresponding to the
  case where $\frac{u}{\sect_D}$ is $L^2$).
  That's why we can follow the line of arguments in Enoki's proof to
  solve the conjecture via the theory of harmonic integrals on lc
  centers (and no extra resolution to bring $s^{-1}(0)$ into snc is
  needed).

  % Thanks to this advantage, we can obtain the generalized version  (not only Fujino's conjecture), 
  This approach gives us the advantage of obtaining Theorem
  \ref{thm:main}, a generalized version of Fujino's conjecture (see
  also Remark \ref{rem:general-commut-diagram} for other generalized
  statements which can be achieved),
  which does not seem to be derivable from results in \cite{Cao&Paun_LC-inj}, at
  least not directly. 
  % Furthermore, the previous works (including \cite{Cao&Paun_LC-inj, Amb03, Amb14, Fuj11} 
  % used the the assumption that $s^{-1}(0)$ contains no lc centers
  % to reduce the proof to the case where the pair $(X,D + s^{-1}(0))$ is log smooth; 
  % however, our paper uses this assumption to apply the inductive argument in terms of lc strata
  % by  the fact that all the data restricted to each component $D_{i}$ satisfy the assumption again. 



  Here we briefly explain the outline of the  proof of Theorem
  \ref{thm:main} with the example where the snc divisor $D$ has
  only two components $D_1$ and $D_2$ such that $D_1 \cap D_2$ is
  irreducible as an illustration.
  In this case, the union of the $1$-lc centers of $(X,D)$ is
  $\lcc|1|' = D_1 \cup D_2$ while that of the $2$-lc centers is
  $\lcc|2|' = D_1 \cap D_2$.
  For any given cohomology class $\alpha \in H^q(D,  K_{D} \otimes F)$
  such that $s  \alpha =0$ in $H^q(D,  K_{D} \otimes F \otimes M)$, 
  the goal is to show that $\alpha$ is actually $0$. 


  Write $h_F = e^{-\vphi_F}$ and $h_M = e^{-\vphi_M}$, and let $\psi_D
  := \phi_D -\sm\vphi_D :=\log\abs{\sect_D}^2 -\sm\vphi_D$ be a global
  function on $X$ such that $\phi_D$ is the (local) potential (of the
  curvature of a metric) on $D$ induced from a canonical section
  $\sect_D$ and $\sm\vphi_D$ is some smooth potential on $D$.
  When $D$ is smooth (i.e.~$D_{1}\cap D_{2}=\emptyset$), 
  the class $\alpha $ can be represented by $(u_{1},  u_{2})$, where $u_i$
  is a harmonic form with respect to $\vphi_F$ on $D_i$ in
  $\mathcal{H}^{n-1,q}(D_{i}; F)_{\vphi_{F}} \cong H^{q}(D_{i},
  K_{D_{i}} \otimes F)$ for $i=1,2$.
  Enoki's argument \cite{Eno90} shows that $s u_{i}$ is also a harmonic
  form with respect to $\vphi_F +\vphi_M$ using the
  Bochner--Kodaira--Nakano formula and the given curvature assumption.
  % by the Bochner trick and the assumption of curvatures.
  It follows from $s \alpha =0$ (as a class) that $s u_{i}=0$ (as a form), hence
  $\alpha=(u_{1}, u_{2})=0$ as desired. 

  However, when $D =\lcc|1|'$ (as well as other $\lcc'$ in the more
  general situation) is not smooth (i.e.~$D_{1}\cap D_{2} \neq
  \emptyset$), the Dolbeault and harmonic theories for cohomology groups
  on $D$ are not yet established, obstructing the use of Enoki's
  argument.
  To overcome this difficulty, we make use of the short exact sequence
  \begin{equation*}
    \xymatrix{
      {0} \ar[r]
      & {\bigoplus_{i=1}^2 K_{D_{i}} \otimes \res F_{D_i}} \ar[r]^-{\tau}
      & {K_{D} \otimes \res F_{D}} \ar[r]
      & {K_{D_{1} \cap D_{2}} \otimes \res F_{D_1 \cap D_2}} \ar[r]
      & {0}
    } \; ,
  \end{equation*}
  where $K_{D} :=K_{X}\otimes D \otimes \frac{\holo_X}{\defidlof{D}}$
  and $\defidlof{D}$ is the defining ideal sheaf of $D$ in $X$, and its
  associated long exact sequence of cohomology groups to reduce our
  injectivity problem of the map $\otimes s$ on $D$ to the injectivity
  problems of $\otimes s$ on the lc centers of $(X,D)$ (i.e.~$D_1$,
  $D_2$ and $D_1 \cap D_2$). 
  Note that all of the lc centers are not contained in $s^{-1}(0)$ by
  assumption and are compact K\"ahler manifolds on which the Dolbeault
  isomorphism and harmonic theory are available.

  Such strategy is suggested already in \cite{Matsumura_injectivity-lc}
  and is used there in the proof of the injectivity theorem for plt
  pairs.
  It is framed in \cite{Chan&Choi_injectivity-I} in terms of the adjoint
  ideal sheaves $\aidlof* := \aidlof = \mtidlof<X>{\vphi_F} \cdot
  \defidlof{\lcc+1'} = \defidlof{\lcc+1'}$ (the defining ideal sheaf
  of $\lcc+1'$ in $X$, under the assumption $\vphi_F$ being smooth)
  for integers $\sigma \geq 0$ and the corresponding residue morphisms
  $\Res^\sigma$ for $\sigma \geq 1$ (see Section \ref{subsec:residue}
  for the definitions).
  Writing $\lcc' = \bigcup_{p \in \Iset} \lcS$ as the decomposition of
  $\lcc'$ into the (irreducible) $\sigma$-lc centers $\lcS$, the residue
  morphism $\Res^\sigma$ induces the isomorphism
  \begin{equation*}
    \logKX \otimes \faidlof/-1* \xrightarrow[\isom]{\Res^\sigma}
    \logKX \otimes \residlof* := \bigoplus_{p \in \Iset} K_{\lcS}
    \otimes \res F_{\lcS} 
  \end{equation*}
  (notice that $\logKX \otimes \frac{\defidlof{D_1 \cap
      D_2}}{\defidlof{D}} \isom \bigoplus_{i=1}^2 K_{D_{i}} \otimes \res
  F_{D_i}$ and $\logKX \otimes \frac{\holo_X}{\defidlof{D_1 \cap D_2}}
  \isom K_{D_{1} \cap D_{2}} \otimes \res F_{D_1 \cap D_2}$ in the
  example).
  It can then be seen that, for more general $D$, the reduction can be
  done via the short exact sequences $0 \to \faidlof/-1* \to
  \faidlof|\sigma'|/-1* \to \faidlof|\sigma'|/* \to 0$ for some integers
  $\sigma$ and $\sigma'$ such that $1 \leq \sigma \leq \sigma'$,
  together with an induction on $\sigma$ via some diagram-chasing
  argument.
  See Step \ref{step:harmonic-rep} of Section
  \ref{sec:proof-of-simple-case} and the beginning of Section
  \ref{subsec:general} for precise details.


  After the reduction, we are led to consider the maps
  \begin{equation*}
    \renewcommand{\objectstyle}{\displaystyle}
    \xymatrix{
      {\smash{\bigoplus_{i =1}^2}\:\cohgp q[D_i]{K_{D_i} \otimes
          F}} \ar[r]^-{\tau}
      \ar[dr]^-{\nu}
      &{\cohgp q[D]{K_D \otimes F}}
      \ar[d]^-{\otimes s} \ar@{}@<-1em>[d]_*+{\circlearrowright}
      \\
      &{\cohgp q[D]{K_D \otimes F \otimes M} \; .}
    }
  \end{equation*}
  It suffices to prove that $\ker\nu =\ker\tau$ (Theorem
  \ref{thm:ker-nu=ker-tau}).
  Indeed, given the injectivity of the map $\otimes \res s_{D_1 \cap D_2}$ on
  $\cohgp q[D_1 \cap D_2]{K_{D_1 \cap D_2} \otimes F}$ followed from
  Enoki's argument in the previous case, we see that the given class
  $\alpha \in \cohgp q[D]{K_D \otimes F}$ actually lies in the image
  $\im\tau$ of $\tau$, say, $\alpha = \tau\paren{u_1, u_2}$ for some
  harmonic forms $u_i \in \Harm'/n-1,q/<D_i>{F},{\vphi_F} \isom \cohgp
  q[D_i]{K_{D_i} \otimes F}$.
  It will then follow that $\paren{u_1, u_2} \in \ker\nu
  =\ker\tau$, hence $\alpha =0$, as desired.
  The pair $(u_1,u_2)$ can be treated as a representative of $\alpha$.
  Suggested by the fact that a harmonic form is the unique
  representative with the \emph{minimal} $L^2$ norm among all elements
  in its corresponding $L^2$ Dolbeault cohomology class, we can choose
  an ``optimal'' representative of $\alpha$ such that $(u_1, u_2)$ has
  the \emph{minimal} distance from (i.e.~is orthogonal to) the subspace
  $\ker\tau$ with respect to the $L^2$ norm induced from $\vphi_F$.
  It then suffices to show that $u_i =0$ for $i = 1,2$ to prove that
  $\ker\nu =\ker\tau$.
  This is done by following the proof of
  \cite{Matsumura_injectivity-lc}*{Thm.~1.6} or
  \cite{Chan&Choi_injectivity-I}*{Thm.~1.2.1} (therefore following the
  spirit of Enoki's argument), but with a few technical modifications.

  One technical complication comes from the use of \v Cech cohomology
  for some cohomology groups (e.g.~$\cohgp q[D]{K_D \otimes F}$) due to
  the lack of the Dolbeault isomorphism.
  Another one is that the argument of Takegoshi in
  \cite{Chan&Choi_injectivity-I}*{\S 3.1, Step IV} (see also
  \cite{Matsumura_injectivity-lc}*{Prop.~3.13}), which essentially gives
  rise to an element in $\ker\tau$ constructed from $u_i$'s, is replaced
  by a construction of a harmonic forms $w$ (or a collection $w
  :=\paren{w_b}_{b\in \Iset+1}$ of harmonic forms for the general $D$)
  representing a class in $\cohgp{q-1}[D_1 \cap D_2]{K_{D_1 \cap D_2}
    \otimes F}$ (see \eqref{eq:w-prelim-formula} and \eqref{eq-def-w}).
  The class of $w$ has its image lying in $\ker\tau$ via the connecting
  morphism of the relevant long exact sequence.
  Such construction is suggested by a residue computation, which relates
  an inner product on (the normalization of) $\lcc|1|'$ to an inner
  product on (lower dimensional) $\lcc|2|'$ (see Proposition
  \ref{prop:res-formula-dbar-exact-dot-harmonic}; see also Steps
  \ref{item:express-su-in-residue-norm} and \ref{item:pf:use_u-ortho-w}
  in Section \ref{subsec:general}, or Steps
  \ref{item:expression-of-su-simple} and
  \ref{step:pf:use_u-ortho-w-simple} in Section
  \ref{sec:proof-of-simple-case} for less intensive notation).
  Such relation between the inner products shows that $w$ is the
  obstruction for having $u_i = 0$ for $i=1,2$.
  This becomes the crucial ingredient to complete the proof.

  The proof of Theorem \ref{thm:main} for the case of general $D$
  follows the same arguments.
  A brief comment for the case where $\vphi_F$ and $\vphi_M$ possess
  suitable analytic singularities is given in Remarks
  \ref{rem:singular-vphi_F} and \ref{rem:no-hard-Lefschetz}.
  
  







  % We briefly explain the proof in the simple case where $D$ has two components (i.e.,\,$D=D_{1}+D_{2}$). 
  % For a given cohomology class $\alpha \in H^q(D,  K_{D} \otimes F)$, 
  % we will prove that $\alpha $ is actually zero 
  % under assuming that $s  \alpha =0 \in H^q(D,  K_{D} \otimes F \otimes M)$. 

  % In the case where $D$ is smooth (i.e.,\,$D_{1}\cap D_{2}=\emptyset$), 
  % the class $\alpha $ can be represented by a harmonic form $(u_{1},  u_{2})$. 
  % Here we used 
  % $$
  % H^q(D,  K_{D} \otimes F)=\oplus_{i=1}^{2} H^{q}(D_{i},  K_{D_{i}} \otimes F) \cong 
  % \oplus_{i=1}^{2} \mathcal{H}^{n-1,q}(D_{i}, F)_{h_{F}}, 
  % $$ 
  % where $\mathcal{H}^{n-1,q}(D_{i}, F)_{h_{F}}$ is the space of harmonic forms with respect to $h_{F}$. 
  % Enoki's argument \cite{Eno90} shows that  
  % $s u_{i}$ is also a harmonic form with respect to $h_{F} h_{M}$
  % by the Bochner trick and the assumption of curvatures. 
  % This implies that $s u_{i}=0$ by $s \alpha =0$; hence $\alpha=\{(u_{1}, u_{2})\}=0$. 

  % In the general case  (i.e.,\,$D_{1}\cap D_{2} \not =\emptyset$), 
  % it is not clear whether the class $\alpha$ can be represented by a harmonic form on $D$,  
  % which is an obvious  difficulty in extending Enoki's argument.
  % To overcome this difficulty, we consider the long exact sequence 
  % $$
  % \cdots \to\bigoplus_{i=1}^{2} H^q(D_{i},  K_{D_{i}}\otimes F ) \xrightarrow{\tau} H^q(D,  K_{D} \otimes F)  \to H^q(D_{1}\cap D_{2},  K_{D_{1}\cap D_{2}}\otimes F ) \to \cdots
  % $$
  % induced by $0 \to K_{D_{1}} \oplus K_{D_{2}} \to K_{D} \to K_{D_{1} \cap D_{2}} \to 0$, 
  % noting that $ K_{D}=(K_{X}\otimes D)\otimes \mathcal{O}_{X}/\mathcal{I}_{D}$. 
  % The multiplication map defined on the right term 
  % $$
  % \otimes s |_{D_{1} \cap D_{2}}: H^q(D_{1}\cap D_{2},  K_{D_{1}\cap D_{2}}\otimes F ) \to
  % H^q(D_{1}\cap D_{2},  K_{D_{1}\cap D_{2}}\otimes F \otimes M ) 
  % $$ 
  % is non-zero since $s^{-1}(0)$ contains no lc centers of the pair $(X,D)$, 
  % and thus injective by induction hypothesis. 
  % Thus, by chasing the commutative diagram induced by the multiplication map, 
  % we can take a cohomology class 
  % $\beta \in \oplus_{i=1}^{2} H^q(D_{i},  K_{D_{i}}\otimes F )$ such that $\tau(\beta)=\alpha$. 
  % Then, we can take a harmonic representation $(u_{1}, u_{2})$ of $\beta$. 


  % The pair $(u_{1}, u_{2})$ is the {\textit{best}} representation for $\beta$ 
  % in the sense that $(u_{1}, u_{2})$ has the minimum $L^2$ norm in the forms representing $\beta$. 
  % However, the pair $(u_{1}, u_{2})$ may not be the best representation for $\alpha$ 
  % since the $L^2$ norm may be reduced by $\tau$. 
  % For this reason, by the orthogonal decomposition, 
  % we re-choose $\beta$ (and its harmonic representation $(u_{1}, u_{2})$) 
  % satisfying $(u_{1}, u_{2}) \in (\Ker \tau)^{\perp} \subset \oplus_{i=1}^{2} \mathcal{H}^{n-1,q}(D_{i}, F)_{h_{F}}$. 
  % The condition $(u_{1}, u_{2}) \in (\Ker \tau)^{\perp}$ means a certain minimal $L^{2}$-norm; 
  % therefore $(u_{1}, u_{2})$ can be seen as the best representation for $\alpha$. 


  % The Bocher trick shows that the $L^2$ norm of $(u_{1}, u_{2})$ is zero in the case $D_{1}\cap D_{2}=\emptyset$. 
  % By generalizing this Bocher trick, 
  % an obstruction for the $L^{2}$-norm to be $0$ 
  % can be described by a $F$-valued differential form $w$ on $D_{1} \cap D_{2}$. 
  % An important point here is that  $w$ is actually harmonic; in particular, it determines the cohomology class.   
  % $H^{q-1}(D_{1}\cap D_{2},  K_{D_{1}\cap D_{2}}\otimes F )$. 
  % Then, we can show that the $L^{2}$-norm of $w$ on $D_{1} \cap D_{2}$ 
  % is equal to the inner product of $(u_{1}, u_{2})$ and a representation of $\delta(w)$, 
  % where $\delta$ is the connecting morphism 
  % $\delta: H^{q-1}(D_{1}\cap D_{2},  K_{D_{1}\cap D_{2}}\otimes F ) \to \oplus_{i=1}^{2} H^{q}(D_{i},  K_{D_{i}} \otimes F)$. 
  % This implies that $w=0$ by $(u_{1}, u_{2}) \in (\Ker \tau)^{\perp}$.


  % 
  % ....


}

This paper is organized as follows.
\tableofcontents


\subsection*{Acknowledgments}
The authors would like to thank the members of Bayreuth University and Pusan National University for their hospitality.
This paper is resulted from the discussions there. 
S.M.~would like to thank Professors Junyan Cao and Mihai P\u{a}un for sharing a preliminary version of \cite{Cao&Paun_LC-inj}.
Also, he would like to thank Professor Osamu Fujino 
for his encouragement and long-standing discussions on lc singularities. 
He is partially supported 
by Grant-in-Aid for Scientific Research (B) $\sharp$21H00976 
and Fostering Joint International Research (A) $\sharp$19KK0342 from
JSPS.
Y.C.~and M.C.~would like to thank S.M.~for drawing their attention to
Fujino's conjecture not long before the covid pandemic (which results
in \cite{Chan&Choi_injectivity-I}) and for joining hand to complete
this project when most aspects of life went back to normal.
Y.C.~and M.C.~were supported by the National Research Foundation
of Korea (NRF) Grant funded by the Korean government
(Nos.~2023R1A2C1007227 and 2021R1A4A1032418).



\section{Preliminary results}\label{sec:preliminaries}

\subsection{Notation and conventions}\label{subsec:notation}

%%%%%
%%%%% File name  : notation.tex
%%%%% Author     : Mario Chan
%%%%% Date       : 13th December, 2021 (original: 04th November, 2020)
%%%%% Description: This is the section "Notation" in the project
%%%%%              "Injectivity-Fujino".
%%%%%
%%
%%%

% In this subsection, we summarize the notation used throughout this paper. 

The following notions are used throughout this paper unless stated otherwise. 
\begin{itemize}
\item $(X,\omega)$ is a compact K\"ahler manifold of dimension $n$. 

% \item $\omega$ is a K\"ahler form on $X$. 

\item $h_F := e^{-\vphi_F}$ and $h_M := e^{-\vphi_M}$, where $\vphi_F$ and
  $\vphi_M$ are respectively the given potentials on $F$ and $M$.
  
\item $D=\sum_{i \in \Iset||}D_{i}$ is a reduced simple-normal-crossing (snc)
  divisor on $X$ (where $\Iset||$ is a finite set). 

\item $\sect_i$ is a canonical section  of the irreducible component $D_{i}$. 

\item $\sect_D := \prod_{i\in \Iset||} \sect_i$ is the canonical section of $D$. 

\item $\sigma \in \{0,1,2,\cdots, n\}$.

% \item  $\Iset$ is the set of $p:=\{i_{1}, i_{2}, \cdots, i_{\sigma}\}$ such that  
% \mmark{$\lcS:=\cap_{k=1}^{\sigma} D_{i_{k}} $}{$\cap D_{i_k}$ may have more than
% one component.} is of codimension $\sigma$. 

% \item   $\lcc' := \cup_{p \in \Iset} \lcS$ is the union of $\sigma$-lc centers $\lcS$ of  $(X,D)$

  
\item $\lcc' :=\bigcup_{p \in \Iset} \lcS$ is the union of
  \emph{$\sigma$-lc centers of $(X,D)$}, i.e.~the
  $\sigma$-codimensional irreducible components of any intersections
  of irreducible components of $D$ (under the assumption $(X,D)$ being
  log-smooth and lc), indexed by $\Iset$.
  Set $\lcc|0|' := X$ and let $\Iset|0|$ be a singleton for convenience.
  Note also that $\Iset|1| = \Iset||$.

\item $\Diff_{p}D$ is the effective divisor on $\lcS$ defined by the 
adjunction formula 
\begin{equation*}
  K_{\lcS} \otimes \Diff_{p}D = \parres{K_X \otimes D}_{\lcS}
\end{equation*}
such that the restriction of $\sect_{(p)}:=
\smashoperator{\prod\limits_{i \in \Iset|| \colon D_i
    \not\supset \lcS}} \sect_i $ to $\lcS$ is a canonical section of
$\Diff_{p}D$.

\item $\phi_D :=\log\abs{\sect_D}^2$ and $\phi_{(p)}
  :=\log\abs{\sect_{(p)}}^2$ are the potentials induced from the
  canonical sections of $D$ and $\Diff_p D$.

\item $\cvr V := \{V_{i}\}_{i \in I}$ is an open cover of $X$  by admissible open sets. 

\item $\{\rho^{i}\}_{i\in I}$ is a partition of unity subordinate to
  $\cvr V$. 
\end{itemize}

Here an open set $V \subset X$ is said to be \emph{admissible} with
respect to $D$ if $V$ is biholomorphic to a polydisc centered at the
origin under a holomorphic coordinate system $(z_{1}, z_{2}, \cdots,
z_{n})$ such that
\begin{equation*} % \label{eq:local-expression-bphi-psi}
  D =\set{z_1 \dotsm z_{\sigma_V} =0}, \quad 
  \log r_{j}^2 < 0, \quad \text{and }
  r_j \fdiff{r_j} \psi_D >0 \text{ on } V \; , 
  % \res{\vphi_\bullet}_V = \smashoperator{\sum_{k=\sigma_V+1}^n} b_{\bullet,k}
  % \log\abs{z_k}^2 +\beta_\bullet \;\;\text{ for } \bullet= F, M \; ,
\end{equation*} 
where  $r_j := \abs{z_j}$  and $\res{\psi_D}_V := \parres{\phi_D
  -\sm\vphi_D}_V =\sum_{j=1}^{\sigma_V} \log\abs{z_j}^2
-\res{\sm\vphi_D}_V$. 

When an admissible set $V$ is considered, an index $p \in \Iset$ such
that $\lcS \cap V \neq \emptyset$ is interpreted as a permutation
representing a choice of $\sigma$ elements from the set
$\set{1,2,\dots,\sigma_V}$ such that
\begin{equation*}
  \lcS \cap V = \set{z_{p(1)} = z_{p(2)} = \dotsm = z_{p(\sigma)} = 0}
  \quad\text{ and }\quad
  \res{\sect_{(p)}}_V = z_{p(\sigma+1)} \dotsm z_{p(\sigma_V)}
\end{equation*}
(cf.~the definition of the set $\cbn$ in \cite{Chan_adjoint-ideal-nas}*{\S 3.1}).



%%% Local Variables:
%%% mode: latex
%%% TeX-master: "Injectivity-Fujino"
%%% coding: utf-8
%%% End:


%\subfile{commut-diagram_Fujino-conj}%

\subsection{$L^{2}$ Dolbeault isomorphism and some results on harmonic
forms}\label{subsec:l2}

%%%%%
%%%%% File name  : L2-spaces-n-harmonic-forms.tex
%%%%% Author     : Mario Chan
%%%%% Date       : 27th March, 2023
%%%%% Description: This is the section on the basic facts of the Hodge
%%%%%              decomposition and the implications of positivity on
%%%%%              harmonic forms which have been discussed in
%%%%%              previous papers.
%%%%%
%%
%%%

{
  \setDefaultvphi{\vphi_L}

  % Suppose that $X$ is \emph{compact} K\"ahler in this section.
  Let $L$ be a holomorphic line bundle on $X$ equipped with a
  (possibly singular) quasi-psh potential $\vphi_L$, which induces,
  together with the K\"ahler form $\omega$, an $L^2$ norm
  $\norm{\cdot}_{X} := \norm\cdot_{X,\vphi_L,\omega}$ on the space of
  smooth $K_X \otimes L$-valued $(0,q)$-forms (or $L$-valued
  $(n,q)$-forms) on $X$.
  Let $\Ltwo/n,q/{L}_{\vphi_L}$ be the completion with respect to $\norm\cdot_X$
  and $\Harm :=\Harm{L}$ be the space of harmonic forms with respect to $\norm\cdot_X$.
  The $L^2$ Dolbeault isomorphism (see
  \cite{Matsumura_injectivity}*{Prop.~5.5 and 5.8} and
  \cite{Matsumura_injectivity-lc}*{Prop.~2.8} for a proof, and see 
  \cite{Chan&Choi_injectivity-I}*{footnote 1} for its naming) guarantees
  the closedness of the subspaces in the orthogonal decomposition
  \begin{equation*}
    \Ltwo/n,q/{L}_{\vphi_L}
    = \Harm \oplus \cl{\paren{\im\dbar}}_{\vphi_L} \oplus \cl{\paren{\im\dbadj}}_{\vphi_L}
    = \Harm \oplus \paren{\im\dbar}_{\vphi_L} \oplus \paren{\im\dbadj}_{\vphi_L} 
  \end{equation*}
  (where $\dbadj$ is the Hilbert space adjoint of $\dbar$ with respect
  to $\norm\cdot_X$, $\paren{\im\dbar}_{\vphi_L}$ and
  $\paren{\im\dbadj}_{\vphi_L}$ denote the images of the corresponding
  operators, with $\cl{\paren{\im\dbar}}_{\vphi_L}$ and
  $\cl{\paren{\im\dbadj}}_{\vphi_L}$ being their closures in
  $\Ltwo/n,q/{L}_{\vphi_L}$)
  and the isomorphism
  \begin{equation*}
    \Harm \isom \cohgp q[X]{K_X \otimes L \otimes \mtidlof{\vphi_L}}
  \end{equation*}
  between the space of harmonic forms and the \v Cech cohomology
  group.
  % Given a locally finite Stein cover $\cvr V = \set{V_i}_{i \in I}$
  % with a partition of unity $\set{\rho^i}_{i\in I}$ subordinate to,
  With $\cvr V := \set{V_i}_{i\in I}$ and $\set{\rho^i}_{i\in I}$ given in Section \ref{subsec:notation},
  the isomorphism can be given explicitly as follows.
  For any (alternating) \v Cech $q$-cocycle $\set{\alpha_{\idx 0.q}}_{\idx 0,q \in
    I}$ and any harmonic form $u \in \Harm$ such that they represent
  the same class in $\cohgp q[X]{K_X \otimes L \otimes
    \mtidlof{\vphi_L}}$, the two representatives are related by 
  (under the Einstein summation convention)
  \begin{equation} \label{eq:Cech-Dolbeault-isom}
    \begin{aligned}
      u &=\dbar v_{(2)} +\dbar \rho^{i_{q-1}} \wedge \dotsm \wedge
      \dbar\rho^{i_0} \alpha_{\idx 0.q} \qquad\paren{\forall~ i_q \in
        I}
      \\
      &=\dbar v_{(2)} +\dbar \rho^{i_{q-1}} \wedge \dotsm \wedge
      \dbar\rho^{i_0} \cdot \rho^{i_q} \:\alpha_{\idx 0.q}
      \\
      &=\dbar v_{(2)} +(-1)^q \:\underbrace{\dbar \rho^{i_{q}} \wedge
        \dotsm \wedge \dbar\rho^{i_1} \cdot \rho^{i_0} }_{=: \:
        \paren{\dbar\rho}^{\idx q.0}} \alpha_{\idx 0.q}
    \end{aligned}
  \end{equation}
  for some $K_X \otimes L$-valued $(0,q-1)$-form $v_{(2)}$ on $X$ with
  $L^2$ coefficients with respect to $\norm\cdot_{X}$ (see
  \cite{Matsumura_injectivity}*{Prop.~5.5} or
  \cite{Chan&Choi_injectivity-I}*{Lemma 3.2.1}).

  The above result is applicable also to the case when $L$ is replaced by
  $D \otimes L$ equipped with the potential $\phi_D +\vphi_L$, where
  $\phi_D :=\log\abs{\sect_D}^2$.
  Denote the corresponding $L^2$ norm by $\norm\cdot_{X,\phi_D}$.
  Assume that \emph{$\vphi_L$ is smooth on $X$}.
  We state the following simple fact here for clarity.
  \begin{lemma} \label{lem:su-harmonicity}
    If $u \in \Harm{L}$, then $\sect_D u \in \Harm{D\otimes L},{\phi_D+\vphi_L}$.
  \end{lemma}
  
  \begin{proof}
    Since $\sect_D$ is holomorphic, it is clear that $\sect_D u$ is
    $\dbar$-closed.

    Let $\dfadj$ and $\dfadj_{\phi_D}$ be the formal adjoint of
    $\dbar$ with respect to $\vphi_L$ and $\phi_D +\vphi_L$
    respectively.
    It then follows that $\dfadj_{\phi_D} = \dfadj
    +\idxup{\diff\phi_D} . \cdot$ and 
    \begin{equation*}
      \dfadj_{\phi_D} \paren{\sect_D u}
      = \sect_D \:\dfadj u - \idxup{\diff\sect_D}. u
      +\idxup{\diff\phi_D} .\sect_D u
      =\sect_D \dfadj u = 0 \; .
    \end{equation*}
    Note that $\omega$ is not complete on $X \setminus D$ and the
    claim (in particular, $\sect_D u \in \Dom \dbadj_{\phi_D}$, where
    $\dbadj_{\phi_D}$ is the Hilbert space adjoint of $\dbar$ with
    respect to $\norm\cdot_{X,\phi_D}$) cannot follow from the
    standard result (for example, \cite{Demailly}*{Ch.~VIII,
      Thm.~(3.2c)}).
    Indeed, the proof of $su \in \Dom
    \dbadj_{\vphi_M}$ in \cite{Chan&Choi_injectivity-I}*{Cor.~3.2.6}
    gives precisely the result $\sect_D u \in \Dom\dbadj_{\phi_D}$ in
    the current setting, which completes the proof.
    A sketch of it is given below for readers' convenience.
    
    Let $\theta \colon [0,\infty) \to [0,1]$ be a smooth
    non-decreasing cut-off function such that
    $\res\theta_{[0,\frac12]} \equiv 0$ and $\res\theta_{[1,\infty)}
    \equiv 1$.
    Set $\theta_\eps := \theta \circ \frac{1}{\abs{\psi_D}^\eps}$ and
    $\theta'_\eps := \theta' \circ \frac{1}{\abs{\psi_D}^\eps}$ for
    every $\eps \geq 0$ (where $\theta'$ is the derivative of
    $\theta$).
    Then both $\theta_\eps$ and $\theta'_\eps$ have compact supports
    inside $X \setminus D$ for $\eps > 0$ and $\theta_\eps \ascendsto
    1$ pointwisely on $X \setminus D$ as $\eps \descendsto 0$.
    For any $\zeta \in \Dom\dbar \subset \Ltwo/n,q-1/<X>{D\otimes
      L}_{\phi_D+\vphi_L}$, convolution with a smoothing kernel on
    local coordinate charts and the lemma of Friedrichs guarantees the
    existence of a sequence $\seq{\zeta_{\eps, \nu}}_{\nu\in\Nnum}$ of
    smooth forms compactly supported in $X \setminus D$ such that
    $\zeta_{\eps,\nu} \tendsto \theta_\eps \zeta$ in the graph norm
    $\paren{\norm\cdot_{X,\phi_D}^2
      +\norm{\dbar\:\cdot}_{X,\phi_D}^2}^{\frac 12}$ of $\dbar$ for
    each $\eps > 0$.
    It then follows that
    \begin{align*}
      \iinner{\sect_D u}{\dbar\zeta}_{X,\phi_D} 
      \xleftarrow{\eps \tendsto 0^+}
      &~\iinner{\sect_D u}{\theta_\eps \dbar\zeta}_{X,\phi_D} \\
      =&~\iinner{\sect_D u}{\dbar\paren{\theta_{\eps}\zeta}}_{X,\phi_D}
         -\iinner{\sect_D u}{\dbar\theta_\eps \wedge \zeta}_{X,\phi_D} \\
      \xleftarrow{\nu \tendsto \infty}
      &~\iinner{\sect_D u}{\dbar\zeta_{\eps,\nu}}_{X,\phi_D}
        -\iinner{\sect_D u}{\frac{\eps \theta'_\eps}{\abs{\psi_D}^{1+\eps}}
        \dbar\psi_D \wedge \zeta}_{X,\phi_D} \\
      =&~\iinner{\dfadj_{\phi_D} \paren{\sect_D u}}{\zeta_{\eps,\nu}}_{X,\phi_D}
         -\iinner{\frac{\eps \theta'_\eps}{\abs{\psi_D}^{1+\eps}}
         \idxup{\diff\psi_D} . \sect_D u}{\zeta}_{X,\phi_D} \; .
    \end{align*}
    The inner product on the far right-hand-side converges to $0$ as
    $\eps \tendsto 0^+$, a consequence of the residue computation (see
    \cite{Chan&Choi_injectivity-I}*{Prop.~3.2.3 and Remark 3.2.4}).
    We can then conclude that $\sect_D u \in \Dom\dbadj_{\phi_D}$ after
    letting $\nu \tendsto \infty$ and then $\eps \tendsto 0^+$.
  \end{proof}

}

Now consider the cases where $(L, \vphi_L) =(F, \vphi_F)$
% (with the induced $L^2$ norm $\norm\cdot_{X}$)
and $(L, \vphi_L) =(F\otimes M, \vphi_F +\vphi_M)$.
% (with the induced $L^2$ norm $\norm\cdot_{X,\vphi_M}$).
A consequence of the positivity on $F$ and $M$ in
\cite{Enoki}, \cite{Matsumura_injectivity-lc} and
\cite{Chan&Choi_injectivity-I} are recalled below.
\begin{prop} \label{prop:consequence-of-positivity}
  % Suppose $\vphi_F$ and $\vphi_M$ are smooth such that
  % $\ibddbar\vphi_F \geq 0$ and $C\ibddbar\vphi_F \geq \ibddbar\vphi_M
  % \;\paren{\geq - C \omega}$ for some constant $C > 0$.
  % Then, $u \in \Harm{F}$ implies $su \in \Harm{F\otimes
  %   M},{\vphi_F+\vphi_M}$ and $\nabla^{(0,1)}u = 0$.
  Suppose that $\vphi_F$ is smooth such that
  $\ibddbar\vphi_F \geq 0$ and $u \in \Harm{F}$.
  Then, one has  $\nabla^{(0,1)}u = 0$.
  If, furthermore, $\vphi_M$ is smooth and satisfies
  $\paren{- C \omega \leq} \; \ibddbar\vphi_M \leq C\ibddbar\vphi_F$
  for some constant $C > 0$,
  then one also has $su \in \Harm{F\otimes M},{\vphi_F+\vphi_M}$.
\end{prop}

\begin{proof}[Reference to the proof]
  These results follow directly from the Bochner--Kodaira--Nakano
  formula.
  See \cite{Chan&Choi_injectivity-I}*{Prop.~3.2.5 and
    Cor.~3.2.6} (while taking $D=0$ and $\psi_D \equiv -1$ in those
  statements).
  See also the proofs for $\diff^*_h\xi = 0$ in
  \cite{Enoki}*{Prop.~2.1} or $D'^*u = 0$ in
  \cite{Matsumura_injectivity-lc}*{Prop.~3.7}.
  These are equivalent statements to the claim $\nabla^{(0,1)}u =0$
  (indeed, $\diff^*_h = D'^*$ and $\abs{D'^*u}^2 =
  \abs{\nabla^{(0,1)}u}^2$ by \cite{Chan&Choi_injectivity-I}*{Remark
    2.4.3}).
\end{proof}


Lemma \ref{lem:su-harmonicity} and Proposition
\ref{prop:consequence-of-positivity} are applied to the case with
$\lcS$ in place of $X$ and $\phi_{(p)}$ in place of $\phi_D$ in the
following sections.



%%% Local Variables:
%%% mode: latex
%%% TeX-master: "Injectivity-Fujino"
%%% coding: utf-8
%%% End:




\subsection{Adjoint ideal sheaves and the residue computations}
% \subsection{Residue functions and residue short exact sequences}
\label{subsec:residue}

%%%%%
%%%%% File name  : residue-fcts-n-residue-exact-seq.tex
%%%%% Author     : Mario Chan
%%%%% Date       : 10th March, 2023
%%%%% Description: This is the section of the project
%%%%%              "Injectivity-Fujino" on residue functions and 
%%%%%              residue exact sequences. 
%%%%%
%%
%%%

{
  \setDefaultvphi{\vphi_L}

  Let $L$ be a line bundle on $X$ equipped with a \mmark{smooth metric
    $e^{-\vphi_L}$}{$\vphi_L$ has to be smooth, or the claim on the
    jumping number must be mentioned explicitly. The result  $\aidlof*
    =\mtidlof{\vphi_L} \cdot \defidlof{\lcc+1'}$ may not hold
    otherwise. \\ }. 
  The \mmark[BlueGreen]{residue function $\eps \mapsto \RTF|f|,<V>$}{It's
    possible not using ``$\RTF|f|$'' at all in this paper.} of index $\sigma$ 
  is defined, for each \mhlight[BlueViolet]{$f \in  \logKX[L] \otimes
    \smooth_X (V)$}, to be 
  \begin{equation*}
    \RTF|f|,<V> :=\RTF|f|,<V>,
    := \eps \int_V \frac{\abs f^2 \:e^{-\phi_D-\vphi_L}}{\logpole} \quad
    \text{ for } \eps > 0  \; . 
  \end{equation*}\mariocomment[BlueViolet]{For consistency of notation in this
    section only.}%
  The adjoint ideal sheaf $\aidlof :=\aidlof<X>$ of index $\sigma$
  is given at each $x \in X$  by
  \begin{equation*}
    \aidlof_x :=\setd{f \in \holo_{X,x}}{
      \exists~\text{open set } V_x \ni x \:, \: \forall~\eps > 0 \:, \:
      \RTF|f|,<V_x>, < +\infty
    } \; .
  \end{equation*}
  Note that the adjoint ideal sheaf is independent of $\vphi_L$ (as
  $\vphi_L$ is smooth).
  By \cite{Chan_adjoint-ideal-nas}*{Thm.~1.2.3}, the adjoint ideal
  sheaf can be written as 
  \begin{equation*}
    \aidlof = \mtidlof{\vphi_L} \cdot \defidlof{\lcc+1'}
    =\defidlof{\lcc+1'}
    \quad\text{ for any } \sigma \geq 0 \; ,
  \end{equation*}
  where $\defidlof{\lcc+1'}$ is the defining ideal sheaf of $\lcc+1'$
  in $X$ (with the reduced structure), \mmark{and we have the residue short exact
    sequence}{I don't want to suggest that the product structure of
    $\aidlof*$ implies directly the residue exact sequence.}
  \begin{equation*}
    \xymatrix@R-0.5cm@C+0.3cm{
      {0} \ar[r]
      & {\aidlof-1} \ar[r]
      & {\aidlof} \ar[r]^-{\Res^\sigma}
      & {\residlof} \ar[r]
      & {0 \; .}
    }
  \end{equation*}
  Here the quotient sheaf ${\residlof}$, called the \emph{residue sheaf of index $\sigma$}, can be written as 
  \begin{equation*}
    \residlof
    = \bigoplus_{p \in \Iset} \paren{\Diff_p D}^{-1}
    \otimes \mtidlof<\lcS>{\vphi_L}
    = \bigoplus_{p \in \Iset} \paren{\Diff_p D}^{-1}
  \end{equation*}
  Note $\logKX[L] \otimes \residlof =\bigoplus_{p \in\Iset} K_{\lcS} \otimes \res L_{\lcS}.$
  Next we describe the \emph{residue morphism $\Res^\sigma$} in terms of 
  the Poincar\'e residue map $\PRes[\lcS]$ given in
  \cite{Kollar_Sing-of-MMP}*{\S 4.18} as follows. 
  The Poincar\'e residue map $\PRes[\lcS]$ from $X$ to each $\lcS$ is
  uniquely determined after an orientation on the conormal bundle of
  $\lcS$ in $X$ is fixed.
  For an admissible open set $V \subset X$, 
  we have $\lcc' \cap V = \bigcup_{\alert{p \in \Iset}}
  \lcS<V>$ \mmark{(where $\lcS<V> := \lcS \cap V$, which is connected by the
  definition of the admissible open set, and possibly empty)}{This is
  a subtle fact that is used in the residue computation. We can keep
  using the same index set because $V$ is admissible.} and $\lcS<V>
=\set{z_{p(1)} =z_{p(2)} =\dotsm =z_{p(\sigma)}=0}$ when non-empty. 
  Under such coordinate system, a section $f $ of  $\logKX[L] \otimes
  \aidlof$ on $V \subset X$ can be written as
  \begin{equation*}
    f = \;\;\smashoperator{\sum_{p \in \Iset \colon \lcS<V>
        \neq\emptyset}} \;\; dz_{p(1)} \wedge \dotsm \wedge dz_{p(\sigma)}
    \wedge g_p \:\sect_{(p)} 
    =\;\;\smashoperator[l]{\sum_{p \in \Iset \colon \lcS<V>
        \neq\emptyset}}
    \frac{dz_{p(1)}}{z_{p(1)}} \wedge \dotsm
    \wedge \frac{dz_{p(\sigma)}}{z_{p(\sigma)}}
    \wedge g_p \:\sect_D \quad\text{ on } V. 
  \end{equation*}
  % Then, the Poincar\'e residue map $\PRes[\lcS]$ is given by
  \mmark{We therefore see that 
  \begin{equation*}
    \PRes[\lcS](\frac{f}{\sect_D})  =\res{g_p}_{\lcS} \in
    K_{\lcS} \otimes \res L_{\lcS} \quad\text{ on } \lcS<V> 
  \end{equation*}}{I don't want to give the impression that we define
  the Poincar\'e residue map by this formula.}%
  under the assumption that the orientation on the conormal bundle of
  $\lcS$ in $X$ on $V$ is given by $(dz_{p(1)}, dz_{p(2)}, \dots,
  dz_{p(\sigma)})$. 
  % Result in \cite{Chan_adjoint-ideal-nas}*{Thm.~4.1.2 (2)} (or the
  % computation in \cite{Chan_on-L2-ext-with-lc-measures}*{Prop.~2.2.1}
  % or \cite{Chan&Choi_ext-with-lcv-codim-1}*{Prop.~2.2.1}) yields
  % % (assuming that $f$ lives on a neighbourhood $V'$ of $\cl V$)
  % \begin{equation*}
  %   \RTF[\rho]|f|(0),<V> = \lim_{\eps \tendsto 0^+}
  %   %   \lim_{\rho \descendsto \charfct_{\cl V}}
  %   \RTF[\rho]|f|,<V>
  %   =\sum_{p \in \Iset} \frac{\pi^\sigma}{(\sigma -1)!} \int_{\lcS<V>}
  %   \rho \abs{g_p}^2 \:e^{-\vphi_L} 
  % \end{equation*}
  % for any compactly supported smooth function $\rho \colon V \to
  On the other hand, the residue morphism $\Res^\sigma$ is given in
  \cite{Chan_adjoint-ideal-nas}*{\S 4.2} by 
  \begin{equation*}
    \renewcommand{\objectstyle}{\displaystyle}
    \xymatrix@C+0.5cm@R-0.5cm{
      {\logKX[L] \otimes \aidlof} \ar[r]^-{\Res^\sigma}
      \ar@{}[d]|*[left]+{\in} 
      & {\hphantom{\logKX[L] \otimes \residlof}}
      \save +<4em,-1.3ex>*{\logKX[L] \otimes \residlof
        =\bigoplus_{p \in\Iset} K_{\lcS} \otimes \res L_{\lcS}} \restore
      \ar@{}[d]|*[left]+{\in}
      % & *+<-2cm,-1cm>{}
      % \ar@{}[l]|(.41)*+{}
      \\
      *+<0.8cm,0cm>{f} \ar@{|->}[r]
      & {\paren{\res{g_p}_{\lcS}}_{\mathrlap{p\in\Iset}}
        \mathrlap{\hphantom{p\in\Iset} .}} 
    }
  \end{equation*}
  Assuming $f$ being defined on a neighbourhood $V'$ of the closure
  $\cl V$ of $V$ and letting $\rho \colon V' \to [0,1]$ be a compactly
  supported smooth function
  % (i.e.~a smooth cut-off function) being
  identically equal to $1$ on $V$, one obtains, 
  from the result in \cite{Chan_adjoint-ideal-nas}*{Thm.~4.1.2 (2)} (or the
  computation in \cite{Chan_on-L2-ext-with-lc-measures}*{Prop.~2.2.1}
  or \cite{Chan&Choi_ext-with-lcv-codim-1}*{Prop.~2.2.1}),
  a (squared) norm of $g :=\paren{\res{g_p}_{\lcS<V>}}_{p \in \Iset} \in
  \logKX[L] \otimes \residlof$ on $V$ given by
  \begin{equation} \label{eq:residue-norm}
    \norm{g}_{\lcc<V>'}^2 :=\RTF|f|(0),<V>
    =\lim_{\rho \descendsto \charfct_{\cl V}} \lim_{\eps \tendsto 0^+}
    \RTF[\rho]|f|,<V'>
    =\sum_{p \in \Iset} \frac{\pi^\sigma}{(\sigma -1)!}
    \int_{\mathrlap{\lcS<V>}} \;\;\;
    \abs{g_p}^2 \:e^{-\vphi_L}
    =:\sum_{p\in\Iset} \norm{g_p}_{\lcS<V>}^2 \; ,
  \end{equation}
  where the limit $\lim_{\rho \descendsto \charfct_{\cl V}}$ refers to
  the pointwise limit as $\rho$ descends to the characteristic
  function $\charfct_{\cl V}$ of $\cl V$ on $X$.
  Such a norm is referred to as the \emph*{residue norm on $\logKX[L]
    \otimes \residlof$ on $V$}.
  %%%%% \emph* is needed as the package embrac is used and \logKX[L]
  %%%%% appears inside \emph.
  Moreover, we also see from the residue exact sequence that
  \begin{equation*}
    \aidlof-1_x
    =\setd{f \in \aidlof_x}{ \exists~\text{open set } V_x
      \ni x \:, \: \RTF|f|(0),<V_x> = 0}
    % \\
    % &=\setd{f \in \aidlof_x}{ \exists~\text{open set } V_x
    %   \ni x \:, \: \RTF|f|,<V_x> = \BigO(\eps) \text{ as } \eps
    %   \tendsto 0^+}
  \end{equation*}
  for every $x \in X$.

  Under the assumption that $\vphi_L$ has only neat analytic
  singularities (which is indeed smooth in the current setting), the
  residue norm on an admissible open set $V \subset X$ can also be
  obtained from 
  \begin{equation*}
    \lim_{\eps \tendsto 0^+} \eps \int_{V} \frac{
      \rho \abs f^2 \:e^{-\phi_D-\vphi_L}
    }{\abs{\psi_D}^{\sigma +\eps}}
    =\RTF[\rho]|f|(0),<V>
  \end{equation*}
  for any smooth compactly supported cut-off function $\rho$ on $V$ (see
  \cite{Chan&Choi_ext-with-lcv-codim-1}*{Prop.~2.2.1} or
  \cite{Chan&Choi_injectivity-I}*{Thm.~2.6.1}).
  Moreover, the above equation works not only for $f$ with holomorphic
  coefficients, but also for $f$ with coefficients in
  $\smooth_{X\,*}$, where
  \begin{align*}
    \smooth_{X\, *}
    &:=\paren{\smooth_{X}\left[
      \frac{1}{\abs{\sect_i}} \colon i \in \Iset||
      \right]}_{\text{b}}
      \qquad\paren{\sect_i \text{ treated as a local defining function of }
      D_i} \\
    &:=\set{\text{locally bounded elements in the $\smooth_X$-algebra generated
      by } \frac{1}{\abs{\sect_i}} \text{ for all } i\in\Iset||} \;
      .\footnotemark
  \end{align*}%
  \footnotetext{
    On an admissible open set $V$ under the holomorphic coordinate
    system $(z_1,\dots, z_n)$ such that $D\cap V =\set{z_1 z_2 \dotsm
      z_{\sigma_V} =0}$, one has
    \begin{equation*}
      \smooth_{X \,*}(V)
      =\smooth_X(V)\left[e^{\pm \cplxi \theta_1}, \dots, e^{\pm \cplxi
          \theta_{\sigma_V}} \right]
    \end{equation*}
    where $(r_j,\theta_j)$ is the polar coordinate system of the
    $z_j$-plane for $j=1,\dots,\sigma_V$ in $V$, which is (almost) the
    same as the ad hoc definition of $\smooth_{X\, *}(V)$ given in 
    \cite{Chan&Choi_injectivity-I}*{\S 2.6} (in which
    $e^{\pm\cplxi\theta_{k}}$ for $k \geq \sigma_V +1$ are also included
    in the set of generators of the algebra).
    The definition given here is independent of coordinates and its
    sheaf structure can be seen easily.
  }%
  The coefficients of $\Res^\sigma$ (and hence $\PRes[\lcS]$ for any
  $p\in\Iset$) can be extended from $\holo_X$ to $\smooth_{X\,*}$
  accordingly.
  The residue norm is finite when the coefficients of $f$ belong to
  $\smooth_{X\,*} \cdot \aidlof$ on $V$.
  When the induced inner product is considered, one still has
  finiteness even if one of the argument does not have coefficients in
  $\smooth_{X\,*} \cdot \aidlof$, which is the content of the
  following proposition.
  \begin{prop} \label{prop:residue-product-X-to-lcS}
    Given any admissible open set $V \subset X$ and any section $f \in
    \logKX[L] \otimes \smooth_{X \:c\,*} \cdot\aidlof\paren{V}$
    (compactly supported in $V$) such
    that $\Res^\sigma(f) = g =\paren{g_p}_{p\in\Iset}$, one
    has, for any $\xi \in \logKX[L] \otimes \smooth_{X \, *}\paren{V}$,
    \begin{align*}
      \lim_{\eps \tendsto 0^+} \eps \int_V
      \frac{\inner{\xi}{f} \:e^{-\phi_D-\vphi_L}}{\abs{\psi_D}^{\sigma
      +\eps}}
      &=\sum_{p \in \Iset} \frac{\pi^\sigma}{(\sigma-1)!}
        \int_{\lcS<V>} \inner{\frac{\rs*\xi_p}{\sect_{(p)}}}{\: g_p}
        \:e^{-\vphi_L} \\
      &=\sum_{p \in \Iset}
      % \smash[b]{
        \underbrace{
        \frac{\pi^\sigma}{(\sigma-1)!}
        \int_{\lcS<V>} \inner{\rs*\xi_p}{\: g_p \sect_{(p)}}
        \:e^{-\phi_{(p)}-\vphi_L}
        }_{\displaystyle =:
        \iinner{\rs*\xi_p}{g_p\sect_{(p)}}_{\mathrlap{\lcS<V>,
        \phi_{(p)}}}}
  % }
  %   \vphantom{\underbrace{\int_{\lcS<V>}}_{\iinner{\rs*\xi_p}{g_p\sect_{(p)}}}}
    \end{align*}
    which is finite,
    where $\phi_{(p)} :=\log\abs{\sect_{(p)}}^2$ and
    \begin{equation*}
      \rs*\xi_p := \PRes[\lcS](\frac{\xi}{\sect_D}) \cdot \sect_{(p)}
      \in K_{\lcS} \otimes \Diff_p D \otimes \res L_{\lcS} \otimes
      \smooth_{X\:c\, *}\paren{\lcS<V>} \; .
    \end{equation*}
    Moreover, if either $f$ or $\xi$ belongs to $\logKX[L] \otimes
    \smooth_{X\,*} \cdot\aidlof-1\paren{V}$, then $\eps \int_V
      \frac{\inner{\xi}{f} \:e^{-\phi_D-\vphi_L}}{\abs{\psi_D}^{\sigma
      +\eps}} = \BigO(\eps)$ (the big-O notation) as $\eps \tendsto 0$.
  \end{prop}

  \begin{proof}
    By linearity in $f$ in the equation in the claim, it suffices to
    consider the case where $\lcS \cap V = \set{z_1 =z_2 = \dotsm
      z_\sigma = 0}$, $\res{\sect_{(p)}}_{V} = z_{\sigma+1} \dotsm z_{\sigma_V}$ and
    \begin{equation*}
      f = dz_1 \wedge dz_2 \wedge \dotsm \wedge dz_\sigma \wedge g_p
      \sect_{(p)} 
    \end{equation*}
    (in which $g_p$ is abused to mean a $(n-\sigma,0)$-form on $V$).
    Write also
    \begin{equation*}
      \xi =: dz_1 \wedge dz_2 \wedge \dotsm \wedge dz_\sigma \wedge
      \xi_p
      \quad\text{ such that }\;\;
      \res{\xi_p}_{\lcS<V>} = \PRes[\lcS](\frac{\xi}{\sect_D})
      \cdot \sect_{(p)} =\rs*\xi_p \; .
    \end{equation*}
    Let $(r_j, \theta_j)$ be the polar coordinates of the $z_j$-plane
    and set 
    \begin{equation*}
      F_0 :=\inner{\xi_p}{g_p} \:e^{-\vphi_L}
      \quad\text{ and }\quad
      F_j :=\fdiff{r_j} \paren{\frac{F_j}{r_j^2 \fdiff{r_j^2} \psi_D}}
      \quad\text{ for } j=1,\dots, \sigma \; .
    \end{equation*}
    Notice that $\fdiff{r_j} \sect_{(p)} = 0$ and coefficients of
    $F_j$ are in $\smooth_{X\:c\,*}$ on $V$ for $j=1,\dots,\sigma$.
    It then follows from the similar computation in
    \cite{Chan&Choi_ext-with-lcv-codim-1}*{Prop.~2.2.1} or
    \cite{Chan&Choi_injectivity-I}*{Thm.~2.6.1} that
    \begin{align*}
      \eps \int_V
      \frac{\inner{\xi}{f} \:e^{-\phi_D-\vphi_L}}{\abs{\psi_D}^{\sigma
      +\eps}}
      &=\eps \int_V
        \frac{\inner{\xi_p}{g_p} \:e^{-\vphi_L}}{\sect_{(p)}\:\abs{\psi_D}^{\sigma
        +\eps}} \wedge \bigwedge_{j=1}^\sigma \frac{\pi\ibar\:dz_j
        \wedge d\conj{z_j}}{\abs{z_j}^2}
      \\
      &=\eps \int_V \frac{F_0}{\sect_{(p)}\:\abs{\psi_D}^{\sigma +\eps}}
        \prod_{j=1}^\sigma d\log r_j^2 \cdot
        \underbrace{\prod_{j=1}^\sigma \frac{d\theta_j}2}_{=:\:
        \vect{d\theta}}
      \\
      &=\frac\eps{\sigma-1+\eps}
        \int_{V} \frac{F_0}{\sect_{(p)}\:r_1^2 \fdiff{r_1^2}\psi_D}
        \:d\paren{\frac{1}{\abs{\psi_D}^{\sigma-1+\eps}}}
        \prod_{j=2}^{\sigma} d\log r_j^2 
        \cdot \vect{d\theta} \\
      &\overset{\mathclap{\text{int.~by parts}}}=
        \quad\;\;
        \frac{-\eps}{\sigma-1+\eps}
        \int_{V}
        \frac{\alert{F_1}}{\sect_{(p)}\:\abs{\psi_D}^{\sigma-1+\eps}}
        \prod_{j=2}^{\sigma} d\log r_j^2 
        \cdot dr_1 \:\vect{d\theta} \\
      &= \dotsm =
        \frac{(-1)^{\sigma} \eps} {\prod_{j=1}^{\sigma} \paren{\sigma-j+\eps}} 
        \int_{V}
        \frac{F_{\alert{\sigma}}}{\sect_{(p)}\:\abs{\psi_D}^{\eps}}
        \prod_{j=1}^{\sigma} dr_j
        \cdot \vect{d\theta} \; .
    \end{align*}
    Note that $\frac{1}{\sect_{(p)}}$ is integrable on $V$, so the
    integral on the far right-hand-side converges for all $\eps \geq
    0$. 
    Letting $\eps \tendsto 0^+$ on both sides, the desired formula
    then follows from the fundamental theorem of calculus.

    When $f$ or $\xi$ belongs to $\logKX[L] \otimes \smooth_{X\,*}
    \cdot\aidlof-1\paren{V}$, the residue formula in the proposition
    holds even when $\sigma$ is replaced by $\sigma-1$, which implies
    that the integral $\int_V \frac{\inner{\xi}{f}
      \:e^{-\phi_D-\vphi_L}}{\abs{\psi_D}^{\sigma +\eps}}$ converges
    for all $\abs\eps < 1$, hence the last claim.
  \end{proof}

  When restriction to a subspace of codimension $1$ is considered,
  there is a more classical kernel for obtaining the residue formula.
  As an illustration, the residue formula from $X$ to $\lcc|1|'$ is
  proved in the following proposition (which is applied to the case
  where the residue from $\lcc'$ to $\lcc+1'$ is considered later). 
  Recall that $\lcc|1|' =D =\sum_{i \in \Iset||} D_i$, where $D_i =
  \lcS|1|[i]$ and $\Iset|| =\Iset|1|$ are set for convenience.
  \begin{prop} \label{prop:residue-formula-classical-kernel}
    Given any admissible open set $V \subset X$ and any compactly
    supported section $f \in
    \logKX[L] \otimes \smooth_{X \:c\,*} \cdot\aidlof|1|\paren{V}$
    such that $\Res^1(f) = g =\paren{g_i}_{i\in\Iset||}$, one
    has, for any $\xi \in \logKX[L] \otimes \smooth_{X \, *}\paren{V}$,
    \begin{align*}
      \lim_{\eps \tendsto 0^+} \eps \int_V
      \inner{\xi}{f} \:e^{-\phi_D-\vphi_L} e^{-\eps\abs{\psi_D}}
      &=\sum_{i \in \Iset||} \pi
        \int_{D_i \cap V} \inner{\frac{\rs*\xi_i}{\sect_{(i)}}}{\: g_i}
        \:e^{-\vphi_L} \\
      &=\sum_{i \in \Iset||}
      % \underbrace{
        \pi
        \int_{D_i \cap V} \inner{\rs*\xi_i}{\: g_i \sect_{(i)}}
        \:e^{-\phi_{(i)}-\vphi_L}
        % }_{\displaystyle =:
        %   \iinner{\rs*\xi_i}{g_i\sect_{(i)}}_{\mathrlap{D_i \cap V,
        %   \phi_{(i)}}}}
    \end{align*}
    which is finite,
    where $\phi_{(i)} :=\log\abs{\sect_{(i)}}^2$ and
    \begin{equation*}
      \rs*\xi_i := \PRes[D_i](\frac{\xi}{\sect_D}) \cdot \sect_{(i)}
      \in K_{D_i} \otimes \Diff_i D \otimes \res L_{D_i} \otimes
      \smooth_{X\:c\, *}\paren{D_i \cap V} \; .
    \end{equation*}    
  \end{prop}

  \begin{proof}
    As before, it suffices to consider the case where $D_i \cap V
    =\set{z_1 =0}$, $\res{\sect_{(i)}}_V =z_2 \dotsm z_{\sigma_V}$ and
    \begin{equation*}
      f = dz_1 \wedge g_i \sect_{(i)} \; .
    \end{equation*}
    Write also
    \begin{equation*}
      \xi =: dz_1 \wedge \xi_i
      \quad\text{ such that }\;\;
      \res{\xi_i}_{D_i \cap V} =\PRes[D_i](\frac{\xi}{\sect_D}) \cdot
      \sect_{(i)} =\rs*\xi_i \; .
    \end{equation*}
    Essentially the same computation as in Proposition
    \ref{prop:residue-product-X-to-lcS} yields 
    \begin{align*}
      \eps \int_V \inner{\xi}{f} \:e^{-\phi_D-\vphi_L}
      e^{-\eps\abs{\psi_D}}
      =&~\eps \int_V \frac{\inner{\xi_i}{g_i}
         \:e^{-\vphi_L}}{\sect_{(i)}} \wedge
         e^{-\eps\abs{\psi_D}}
         \frac{
         \pi\ibar\:dz_1 \wedge d\conj{z_1}
         }{\abs{z_1}^2}
      \\
      =&~\int_V \frac{\inner{\xi_i}{g_i}
         \:e^{-\vphi_L}}{\sect_{(i)} \:r_1^2 \fdiff{r_1^2} \psi_D} 
         \:d\paren{e^{-\eps\abs{\psi_D}}} \:
         \frac{d\theta_1}{2}
      \\
      \overset{\mathclap{\text{int.~by parts}}}=
       &~\quad\;\;
         -\int_V \fdiff{r_1} \paren{\frac{\inner{\xi_i}{g_i}
         \:e^{-\vphi_L}}{r_1^2 \fdiff{r_1^2} \psi_D} }
         \:\frac{e^{-\eps\abs{\psi_D}}}{\sect_{(i)}} \:dr_1 \:
         \frac{d\theta_1}{2}
      \\
      \mathclap{\xrightarrow{\eps \tendsto 0^+}\;\;}
       &~\quad\;
         -\int_V \fdiff{r_1} \paren{\frac{\inner{\xi_i}{g_i}
         \:e^{-\vphi_L}}{r_1^2 \fdiff{r_1^2} \psi_D} }
         \:\frac{1}{\sect_{(i)}} \:dr_1 \:
         \frac{d\theta_1}{2}
      \\
      =&~\pi \int_{\mathrlap{D_i \cap V}} \;\;\; \frac{\inner{\rs*\xi_i}{g_i}
         \:e^{-\vphi_L}}{\sect_{(i)}}
         =\pi \int_{D_i \cap V} \inner{\rs*\xi_i}{g_i \sect_{(i)}}
         \:e^{-\phi_{(i)}-\vphi_L} \; .
    \end{align*}
    Note that the convergence of the integral obtained right after
    integration by parts follows from the same reasoning as in
    Proposition \ref{prop:residue-product-X-to-lcS}.
  \end{proof}

  Proposition \ref{prop:residue-formula-classical-kernel} facilitates the
  following residue computation.

  \begin{prop} \label{prop:res-formula-dbar-exact-dot-harmonic}
    Given the decomposition $\lcc' = \bigcup_{p\in\Iset} \lcS$, let
    $u_p$ be a \emph{harmonic} $K_{\lcS} \otimes \res L_{\lcS}$-valued
    $(0,q)$-form on $\lcS$ with respect to the norm
    $\norm\cdot_{\lcS}$ for each $p \in \Iset$.
    Given also the decomposition $\lcc+1' = \bigcup_{b\in\Iset+1}
    \lcS+1[b]$, for any $\lcS$ and $\lcS+1[b]$ such that $\lcS+1[b]
    \subset \lcS$, let $\sgn{b:p}$ be the sign such that
    \begin{equation*}
      \PRes[\lcS+1[b]] =\sgn{b:p} \:\PRes[\lcS+1[b] | \lcS] \circ
      \PRes[\lcS] \; ,
    \end{equation*}
    where $\PRes[\lcS+1[b] | \lcS]$ denotes the Poincar\'e residue map
    from $\lcS$ to $\lcS+1[b]$.
    % For a given locally finite cover $\cvr V :=\set{V_i}_{i \in I}$ of
    % $X$ by \emph{admissible} open sets with respect to
    % $(\vphi_L,\psi_D)$ together with a partition of unity
    % $\set{\rho^i}_{i \in I}$ subordinate to it,
    With the finite cover $\cvr V$ and partition of unity
    $\set{\rho^i}_{i \in I}$ given in Section \ref{subsec:notation},
    let $\set{\gamma_{\idx 1.q}}_{\idx 1,q \in I}$ be a
    $\logKX[L]$-valued \v Cech $(q-1)$-cochain with respect to $\cvr
    V$ and set 
    \begin{gather*}
      \rs\gamma_{p; \:\idx 1.q} :=\PRes[\lcS](\frac{\gamma_{\idx 1.q}}{\sect_D})
      \cdot \sect_{(p)} \; , \quad
      v_{p} := \sum_{\idx 1,q \in I} \underbrace{
        \dbar\rho^{i_q} \wedge \dotsm
        \wedge \dbar\rho^{i_2} \cdot \rho^{i_1}
      }_{=: \: \paren{\dbar\rho}^{\idx q.1}} \rs*\gamma_{p;\:\idx 1.q}
      \quad\text{ on } \lcS \\
      \text{and }\quad
      \rs\gamma_{b; \:\idx 1.q} :=\PRes[\lcS+1[b]](\frac{\gamma_{\idx 1.q}}{\sect_D})
      \cdot \sect_{(b)} \; , \quad
      v_{b} := \sum_{\idx 1,q \in I} \paren{\dbar\rho}^{\idx q.1}
      \rs*\gamma_{b;\:\idx 1.q}
      \quad\text{ on } \lcS+1[b] \; .
    \end{gather*}
    Then, after setting $\iinner{\cdot}{\cdot}_{\lcS, \phi_{(p)}}
    :=\iinner{\cdot}{\cdot \:e^{-\phi_{(p)}}}_{\lcS}$ (and
    similarly for $\iinner{\cdot}{\cdot}_{\lcS+1[b], \phi_{(b)}}$), one has
    \begin{equation*}
      \sum_{p\in\Iset} \iinner{\dbar v_{p}}{
        u_p\sect_{(p)}}_{\lcS,\phi_{(p)}}
      =-\sigma \smashoperator[l]{\sum_{b\in\Iset+1}} \iinner{v_{b} \:}{\quad\;
        \smashoperator{\sum_{p\in\Iset \colon \lcS+1[b] \subset
            \lcS}} \;\;
        \sgn{b:p} \: \PRes[\lcS+1[b] | \lcS](\idxup{\diff\psi_{(p)}}.
         u_p) \cdot \sect_{(b)}
      }_{\lcS+1[b], \phi_{(b)}} \; ,
    \end{equation*}
    where $\psi_{(p)} :=\phi_{(p)} -\sm\vphi_{(p)}$ and
    $\sm\vphi_{(p)}$ is some smooth potential on $\Diff_p D$.
  \end{prop}

  \begin{proof}
    Notice that $v_{p}$ is smooth on $\lcS$ but not necessarily
    locally $L^2$ with respect to the weight $e^{-\phi_{(p)}}$.
    An integration by parts is done via the use of Proposition
    \ref{prop:residue-formula-classical-kernel}, which yields 
    \begin{align*}
      &~\sum_{p\in \Iset} \iinner{\dbar v_{p}}{ u_p
        \sect_{(p)}}_{\lcS, \phi_{(p)}}
      \\
      \xleftarrow{\eps \tendsto 0^+}
      &~\sum_{p \in \Iset} \iinner{
        e^{-\eps \abs{\psi_{(p)}}} \:\dbar v_{p}
        }{ u_p \sect_{(p)}}_{\lcS, \phi_{(p)}}
      \\
      =&~\sum_{p \in \Iset} \paren{
         \cancelto{0 \;\;\;(\because~u_p \text{ harmonic, Lemma \ref{lem:su-harmonicity}})}{\iinner{
         \dbar\paren{e^{-\eps \abs{\psi_{(p)}}} \: v_{p}}
         }{ u_p \sect_{(p)}}_{\mathrlap{\lcS, \phi_{(p)}}}}
         \quad\;\; - \eps 
         \iinner{
         e^{-\eps \abs{\psi_{(p)}}} \:v_{p}
         }{\:\idxup{\diff\psi_{(p)}}.  u_p \sect_{(p)}}_{\lcS,
         \phi_{(p)}}
         }
      \\
      =&~-\sum_{p \in \Iset} \sum_{\idx 1,q \in I} \eps \:
         \iinner{
         e^{-\eps \abs{\psi_{(p)}}} \: % \paren{\dbar\rho}^{\idx q.1}
         \rs*\gamma_{p;\:\idx 1.q}
         }{\:
         \idxup{\diff\rho},[\idx 1.q] .
         \paren{\idxup{\diff\psi_{(p)}}.  u_p \sect_{(p)}}
         }_{\lcS, \phi_{(p)}}
      \\
      \xrightarrow[\text{Prop.~\ref{prop:residue-formula-classical-kernel}}]{\eps
      \tendsto 0^+} 
      &~-\smashoperator[l]{\sum_{\idx 1,q \in I}} \sum_{p \in \Iset}
        \sum_{k=\sigma +1}^{\mathclap{\sigma_{V_{\idx 1.q}}}} \sigma
        \iinner{
        \PRes[p(k)](
        \frac{\rs*\gamma_{p;\:\idx 1.q}}{\sect_{(p)}}
        )
        }{\:
        \idxup{\diff\rho},[\idx 1.q] .
        \PRes[p(k)](\idxup{\diff\psi_{(p)}}.  u_p)
        }_{\lcS \cap \set{z_{p(k)} =0}}
        \; ,
    \end{align*}
    where $\idxup{\diff\rho},[\idx 1.q] . \cdot$ is the adjoint
    of $\paren{\dbar\rho}^{\idx q.1} \cdot$, and $\PRes[p(k)]$ denotes
    the Poincar\'e residue map from $\lcS$ to $\lcS \cap \set{z_{p(k)}=0}$. 
    The last limit is justified as follows.
    On the admissible open set $V_{\idx 1.q}$, consider a holomorphic
    coordinate system $(z_1, \dots, z_n)$ such that $\lcS \cap V_{\idx
    1.q}
    =\set{z_{p(1)} = \dotsm =z_{p(\sigma)} =0}$ and
    $\sect_{(p)} =z_{p(\sigma+1)} \dotsm z_{p(\sigma_V)}$ (write
    $\sigma_{V}$ for $\sigma_{V_{\idx 1.q}}$ for convenience).
    Note that
    \begin{equation*}
      \diff\psi_{(p)} =\sum_{k =\sigma +1}^{\sigma_V}
      \frac{dz_{p(k)}}{z_{p(k)}} -\diff\sm\vphi_{(p)} \quad\text{ on }
      V_{\idx 1.q} \; .
    \end{equation*}
    It follows that, on $\lcS \cap V_{\idx 1.q}$,
    \begin{equation*}
      \text{coef.~of }\:
      \idxup{\diff\rho},[\idx 1.q].
      \paren{\idxup{\diff\psi_{(p)}}.  u_p \sect_{(p)}}
      \in
      \smooth_{\lcS \:c} \cdot\res{\defidlof{\lcc+2'}}_{\lcS}
      \begin{aligned}[t]
        &=\smooth_{\lcS \:c} \cdot\mtidlof<\lcS>{\vphi_L} \cdot
        \res{\defidlof{\lcc+2'}}_{\lcS} \;\;\footnotemark
        \\
        &=\smooth_{\lcS \:c} \cdot\aidlof|1|<\lcS>{\vphi_L}[\psi_{(p)}]
      \end{aligned}
    \end{equation*}%
    \footnotetext{
      Recall that $\defidlof{\lcc+2'}$ is generated on $X$ by
      $\sect_{(b)}$ treated as local
      functions for all $b \in \Iset+1$.
      On an admissible open set $V$, one has $\defidlof{\lcc+2'}
      =\genbyd{z_{b(\sigma+2)} \dotsm
        z_{b(\sigma_V)}}{b \in \Iset+1 \text{ such that } \lcS+1[b] \cap
        V \neq \emptyset}$.
      % (see page
      % \pageref{page:notation-permutation-index} for the notation).
    }%
    and, therefore, one can apply Proposition
    \ref{prop:residue-formula-classical-kernel} (with $\lcS$ in place
    of $X$, $\psi_{(p)}$ in place of $\psi_D$) to each inner product
    $\eps \iinner{e^{-\eps \abs{\psi_{(p)}}} \dotsm}{\: \dotsm
      \idxup{\diff\psi_{(p)}} . \dotsm \sect_{(p)}}_{\lcS,\phi_{(p)}}$.
    Note also that the factor $\sigma$ comes from the normalisation of
    the norm on each lc center ($\norm\cdot_{\lcS}^2
    =\frac{\pi^\sigma}{(\sigma -1)!} \int_{\lcS} \dotsm$ and
    $\norm\cdot_{\lcS+1[b]}^2 =\frac{\pi^{\sigma+1}}{\sigma!}
    \int_{\lcS+1[b]} \dotsm$).


    On each admissible open set $V_{\idx 1.q}$, the intersection $\lcS
    \cap \set{z_{p(k)} = 0}$ is a $(\sigma+1)$-lc center $\lcS+1[b_{p,k}]
    \cap V_{\idx 1.q}$ ($\neq \emptyset$), uniquely determined by the
    choices of $p\in \Iset$ (such that $\lcS \cap V_{\idx 1.q} \neq
    \emptyset$, so $\binom{\sigma_V}{\sigma}$ choices) and $k
    =\sigma+1, \dots, \sigma_V$ (so $\sigma_V-\sigma$ choices).
    To get an indexing in terms of $b \in \Iset+1$ (such that
    $\lcS+1[b] \cap V_{\idx 1.q} \neq \emptyset$, so
    $\binom{\sigma_V}{\sigma +1}$ choices), note that each $\lcS+1[b]
    \cap V_{\idx 1.q}$ is contained in $\sigma +1$ distinct
    $\sigma$-lc centers $\lcS[p_{b,j}]$ for $j=1,\dots,\sigma+1$
    (apparently, $\sigma +1$ choices) such that
    \begin{equation*}
      \lcS+1[b] \cap V_{\idx 1.q} = \lcS[p_{b,j}] \cap \set{z_{b(j)} = 0} \; .
    \end{equation*}
    (One can verify $\sum_{p \in \Iset} \sum_{k=\sigma
      +1}^{\sigma_{V}} \dotsm = \sum_{b \in
      \Iset+1} \sum_{j=1}^{\sigma +1} \dotsm$ by first noting that
    $\binom{\sigma_V}{\sigma} (\sigma_V -\sigma)
    =\binom{\sigma_V}{\sigma +1} (\sigma+1)$.)
    With such choice of indexing, one has
    \begin{equation*}
      \frac{\rs*\gamma_{b;\: \idx 1.q}}{\sect_{(b)}}
      :=\PRes[\lcS+1[b]](\frac{\gamma_{\idx 1.q}}{\sect_D})
      =\sgn{b:p_{b,j}} \:
      \PRes[b(j)](\frac{\rs*\gamma_{p_{b,j};\:\idx
          1.q}}{\sect_{(p_{b,j})}})
    \end{equation*}
    (noticing that % $\sect_{(b)} =\sect_{(\sigma+1 : b)}$,
    % $\sect_{(p_{b,j})} =\sect_{(\sigma : p_{b,j})}$ and
    $\sect_{(p_{b,j})} = z_{b(j)} \sect_{(b)}$).
    As a result, the expression in question becomes
    \begin{align*}
      &-\smashoperator[l]{\sum_{\idx 1,q \in I}} \sum_{b \in \Iset+1}
        \sum_{j=1}^{\sigma +1} \sigma
        \iinner{ \sgn{b:p_{b,j}}\:
        \frac{\rs*\gamma_{b;\:\idx 1.q}}{\sect_{(b)}}
        }{\: 
        \idxup{\diff\rho},[\idx 1.q] .
        \PRes[b(j)](\idxup{\diff\psi_{(p_{b,j})}} . u_{p_{b,j}})
        }_{\lcS+1[b]}
      \\
      =&-\smashoperator[l]{\sum_{\idx 1,q \in I}}
        \sum_{b \in \Iset+1}
        \sigma
        \iinner{
        \paren{\dbar\rho}^{\idx q.1} \rs*\gamma_{b;\:\idx 1.q}
        \:}{  \sum_{j=1}^{\sigma +1} \sgn{b:p_{b,j}}\:
        \PRes[b(j)](\idxup{\diff\psi_{(p_{b,j})}} . u_{p_{b,j}})
        \cdot \sect_{(b)}
        }_{\lcS+1[b], \phi_{(b)}}
      \\
      =&-\sigma \sum_{b \in \Iset+1} \iinner{
        v_b
        \:}{\quad\;
        \smashoperator{\sum_{p\in\Iset \colon \lcS+1[b] \subset
        \lcS}} \;\;
        \sgn{b:p}\:
        \PRes[\lcS+1[b] | \lcS](\idxup{\diff\psi_{(p)}}.  u_{p})
        \cdot \sect_{(b)}
        }_{\lcS+1[b], \phi_{(b)}} \; . \qedhere
    \end{align*}
  \end{proof}


}


\subsection{Restriction of harmonic differential forms to hypersurfaces}\label{subsec:harmonic}

{
  
  Let $(X,\omega)$ be a K\"ahler manifold equipped with a holomorphic
  line bundle $L$ equipped with a smooth potential $\varphi_L$ such
  that $\ibddbar\varphi_L\ge0$ and let $D$ be an snc divisor in $X$
  written as 
  \begin{equation*}
    D=\sum_{p\in I_D}D_p \; ,
  \end{equation*}
  where $D_p$ is an irreducible component for $p\in I_D$.
  We define the map $\HRes_p \colon \mathscr
  A_X^{0,q}(X,K_X\otimes L)\rightarrow \mathscr
  A_{D_p}^{0,q-1}(D_p,K_{D_p}\otimes L\vert_{D_p})$ by 
  \begin{equation*}
    \HRes_p(u)
    =
    \PRes[D_p](\idxup{\partial\psi_D} . u) \;\;\;\text{for}\;\;\;p\in
    I_D \; ,
  \end{equation*}
  where $\mathcal{R}_{D_p}$ is the Poincar\'e residue map (see, for
  example, \cite{Griffiths&Harris}*{p.147} or \cite{Kollar_Sing-of-MMP}*{\S 4.18}). 
  Notice that, as in \cite{Chan&Choi_injectivity-I}*{\textsection2.6},
  the map $\mathcal R_{D_p}$ is extended to send sections of
  $K_X\otimes\overline{\bold\Omega}_X^q$ to those of
  $K_{D_p}\otimes\overline{\bold\Omega}_X^q\vert_{D_p}$. 
  Let $(U;z^1,\dots,z^n)$ be a local holomorphic coordinate system
  around $x\in D_p\subset X$ satisfying 
  \begin{enumerate}[label=(\roman*), ref=\roman*]
  \item  \label{item:admissible-open-U} % [(\romannumeral1)]
    $D_p\cap U=\{z\in U:z^1=0\}$ and 
    $D\cap U=\set{z^1\cdots z^{\sigma_U}=0}$,
    and
  \item  \label{item:psi_D-in-admissible-open-U} % [(\romannumeral2)]
    $\psi_D=\sum_{j=1}^{\sigma_U}
    \log\abs{z^j}^2-\sm\varphi_D$ on $U$.
  \end{enumerate}
  Since $\del\psi_D=\sum_{j=1}^{\sigma_U}\frac{dz^j}{z^j}-\del\sm\varphi_D$, it follows that
  \begin{equation*}
    \HRes_p(u)
    =
    \mathcal{R}_{D_p}\left(\idxup{\partial\psi_D}.u\right)
    =
    \mathcal{R}_{D_p}\paren{\idxup{\frac{dz^1}{z^1}}.u}
    =
    \paren{\idxup{dz^1}.\widetilde u_p}\big\vert_{D_p} \; ,
  \end{equation*}
  where $\rs u_p := \fdiff{z^1} \ctrt u$ (so $u=dz^1\wedge\widetilde{u}_p$).
  In particular, $\HRes_p$ does not depend on the
  choice of the Hermitian metric $\sm\varphi_{D}$. 
  It follows from the above formula that
  $\HRes_p(u)$ is actually a $K_{D_p}\otimes
  L\vert_{D_p}$-valued $(0,q-1)$-form on $D_p$ (not only a
  $\overline{\bold\Omega}_X^{q-1}\vert_{D_p}$-valued section). 



  First we notice that $\dfadj$-closedness is preserved by
  $\HRes_p$ on a K\"ahler manifold.
  \begin{prop} \label{prop:harmonic-residue}
    If $u$ is a $\dfadj$-closed $K_X\otimes L$-valued $(0,q)$-form on
    $X$, then $\HRes_p(u)$ is a $\dfadj$-closed
    $K_{D_p}\otimes L\vert_{D_p}$-valued $(0,q-1)$-form on $D_p$. 
  \end{prop}

  \begin{proof}
    It is enough to show that it vanishes at the given point $x\in D_p$.
    Let $(z^1,\dots,z^n)$ be a local holomorphic coordinate system around
    $x$ in $X$ satisfying \eqref{item:admissible-open-U} % (\romannumeral1)
    and \eqref{item:psi_D-in-admissible-open-U}. % (\romannumeral2).
    Since $(D_p,\omega\vert_{D_p})$ is a smooth $(n-1)$-dimensional
    K\"ahler manifold, by a linear change of coordinates
    $(z^1,\ldots,z^n)$ and a quadratic change of coordinates in
    $(z^2,\dots,z^n)$ of $D_p$ we may assume that
    \begin{enumerate}[resume*]
    \item % [(\romannumeral3)]
      $g_{i\bar j}(x)=I_n$ where $I_n$ is the $n\times n$ identity matrix.
    \item % [(\romannumeral4)]
      $dg_{\alpha\bar\beta}(x)=0$ for $2\le\alpha\le n$ and $2\le\beta\le n$.
    \end{enumerate}
    Since $\displaystyle\idxup{dz^1}=g^{\bar
      j1}\pd{}{\overline{z^j}}$ (under Einstein summation convention),
    we have
    \begin{equation}\label{E:local_expression}
      dz^1 \wedge \paren{\idxup{dz^1} . \widetilde u_p}_{\bar j_1,\ldots,\bar j_{q-1}}
      =
      g^{\bar j1}
      u_{\bar j\bar j_1,\ldots,\bar j_{q-1}} \; .
    \end{equation}
    % where $\beta_1,\ldots,\beta_{q-1}$ run from $2$ to $n$.
    For the sake of convenience, let the Latin indices $i,j,k,...$ run
    from $1$ to $n$ and let the Greek indices
    $\alpha,\beta,\gamma,...$ run from $2$ to $n$ in this proof. 
    Let $\varphi_{K_X}$ and $\varphi_{K_{D_p}}$ be respectively the
    potentials on $K_X$ and $K_{D_p}$ induced by the K\"ahler metric
    $\omega$, which are written as 
    \begin{equation*}
      \varphi_{K_X}
      =
      \log\det\paren{g_{i\bar j}}_{1\le i,j\le n}
      \;\;\;\text{and}\;\;\;
      \varphi_{K_{D_p}}
      =
      \log\det\paren{g_{\alpha\bar\beta}}_{2\le\alpha,\beta\le n} \; .
    \end{equation*}
    This yields, at the given point $x\in D_p$,
    \begin{align*}
      \del_\gamma\varphi_{K_X}
      &=
	\del_\gamma\log\det g
	=
	\Tr\paren{\del_\gamma g\cdot g^{-1}}
      \\
      &=
	\sum_{i,j=1}^n\pd{g_{i\bar j}}{z^\gamma} g^{\bar j i}
	\;\;\overset{\mathclap{(\text{at } x)}}=\;\;
	g^{\bar11}\del_\gamma g_{1\bar1}
	+
	\sum_{\alpha,\beta=2}^n\pd{g_{\alpha\bar\beta}}{z^\gamma}g^{\alpha\bar\beta}
      \\
      &=
	g^{\conj11}\del_\gamma g_{1\conj1}
	+
	\del_\gamma\log\det\paren{g\vert_{D_p}}
	=
	g^{\conj11}\del_\gamma g_{1\conj1}
	+
	\del_\gamma\varphi_{K_{D_p}}
	\;\;\overset{\mathclap{(\text{at } x)}}=\;\;
	g^{\conj11}\del_\gamma g_{1\conj1} \; .
    \end{align*}
    % This implies that
    % \begin{equation*}
    %   \del_\ell\varphi_{K_X}
    %   =
    %   \del_\ell\log\det\paren{g_{i\bar j}}_{1\le i,j\le n}
    %   =
    %   \sum_{i,j=1}^n\pd{g_{i\bar j}}{z^\ell}g^{\bar ji}
    %   \;\;\;\text{and}
    %   \;\;\;
    %   \del_\ell\varphi_{K_{D_p}}
    %   =
    %   \sum_{\alpha,\beta=2}^n
    %   \pd{g_{\alpha\bar\beta}}{z^\ell}g^{\bar\beta\alpha}.	
    % \end{equation*}
    % Thus we have
    % \begin{equation*}
    %   \sum_{i,j}\pd{g_{i\bar j}}{z^\ell}g^{\bar ji}
    %   =
    %   \del_lg_{1\bar1}
    %   +
    %   \sum_{\alpha=2}^n\pd{g_{\alpha\bar\alpha}}{z^l}g^{\alpha\bar\alpha}
    %   =
    %   \del_lg_{1\bar1}
    %   +
    %   \del_l\varphi_{K_{D_p}}
    %   \;\;\;
    %   \text{and}
    %   \;\;\;
    %   \del_\beta\varphi_{K_{D_p}}=0\;\;\;\text{at}\;\;x.
    % \end{equation*}
    % It follows from \eqref{E:local_expression} and the definition of $\dfadj$ (cf.~\cite{Siu}) that
    % for any multi-indices $\ov{\boldsymbol\beta}_{q-2}=(\bar\beta_1,\ldots,\bar\beta_{q-2})$,
    \newcommand{\bbeta}{{\boldsymbol{\beta}}}%
    \newcommand{\KDp}{{\smash[b]{K_{D_p}}}}%
    % \renewcommand{\CancelColor}{\color{Gray}}%
    Choose a local frame of $L$ in a neighbourhood of $x$ in $X$
    such that
    \begin{equation*}
      \diff\vphi_L(x) = 0  \; .
    \end{equation*}
    It follows from \eqref{E:local_expression} and the definition of
    $\dfadj$ on $D_p$ (see, for example,
    \cite{Siu}*{(1.3.2)}) that, for any 
    multi-indices ${\bbeta}_{q-2} =(\idx[\beta]1,{q-2})$ and at the
    given point $x \in D_p$,
    \begin{align*}
      dz^1 \wedge \paren{\dfadj \HRes_p(u)}_{\conj\bbeta_{q-2}}
      &=
        -g^{\conj\beta\gamma}
	\nabla_\gamma 
	\paren{
        g^{\alert{\conj j} 1}
        u_{\alert{\conj j} \conj\beta\ov{\boldsymbol\beta}_{q-2}}
	}
      \\
      &=-g^{\conj\beta \gamma}\diff_\gamma \paren{g^{\alert{\conj j} 1}
        u_{\alert{\conj j} \conj\beta \conj\bbeta_{q-2}}}
        +g^{\conj\beta \gamma} \smash[t]{\cancelto{0}{\paren{
        \diff_\gamma \vphi_\KDp +\diff_\gamma \vphi_L
        }}}
        \cdot g^{\alert{\conj j} 1} u_{\alert{\conj j} \conj\beta \conj\bbeta_{q-2}}
      \\
      &=-g^{\conj\beta \gamma} g^{\alert{\conj j} 1} \diff_\gamma
        u_{\alert{\conj j} \conj\beta \conj\bbeta_{q-2}}
        -g^{\conj\beta \gamma} \diff_\gamma g^{\alert{\conj j} 1}
        \cdot u_{\alert{\conj j} \conj\beta \conj\bbeta_{q-2}}
      \\
      &=-g^{\conj\beta \gamma} g^{\conj 1 1}
        \diff_\gamma u_{\conj 1 \conj\beta \conj\bbeta_{q-2}}
        +g^{\conj\beta \gamma} g^{\alert{\conj j k}} \diff_\gamma
        g_{\alert k \conj 1} \cdot g^{\conj 1 1} u_{\alert{\conj j}
        \conj\beta \conj\bbeta_{q-2}}
      \\
      &=g^{\conj 1 1} \paren{
        -g^{\conj\beta \gamma} \diff_\gamma
        u_{\conj 1 \conj\beta \conj\bbeta_{q-2}}
        +g^{\conj\beta\gamma} g^{\conj 1 1} \diff_\gamma
        g_{1\conj 1} \cdot
        u_{\conj 1 \conj\beta \conj\bbeta_{q-2}}
        +g^{\conj\beta\gamma} g^{\alert{\conj j} \alpha}
        \diff_\gamma g_{\alpha \conj 1} \cdot
        u_{\alert{\conj j} \conj\beta \conj\bbeta_{q-2}}
        }
      \\
      &=g^{\conj 1 1} \paren{
        g^{\alert{\conj k j}} \diff_{\alert{j}}
        u_{\alert{\conj k} \conj 1 \conj\bbeta_{q-2}}
        -g^{\alert{\conj k j}} \diff_{\alert{j}}\vphi_{K_X} \cdot
        u_{\alert{\conj k} \conj 1 \conj\bbeta_{q-2}}
        }
        +g^{\conj 1 1} g^{\conj \gamma \gamma} g^{\conj\alpha
        \alpha} \diff_{\gamma} g_{\alpha \conj 1} \cdot
        u_{\conj\alpha \conj\gamma \conj\bbeta_{q-2}}
      \\
      &=-g^{\conj 1 1} \paren{\dfadj u}_{\conj 1 \conj\bbeta_{q-2}}
        +g^{\conj 1 1} g^{\conj \gamma \gamma} g^{\conj\alpha
        \alpha} \diff_{\gamma} g_{\alpha \conj 1} \cdot
        u_{\conj\alpha \conj\gamma \conj\bbeta_{q-2}} \; .
    \end{align*}
    Since $\del_\gamma g_{\alpha\conj1}$ is symmetric in $\alpha,
    \gamma$ (for $X$ being K\"ahler) while $u_{\conj\alpha\conj\gamma
      \conj{\bbeta}_{q-2}}$ is anti-symmetric in $\alpha, \gamma$, the
    last term in the expression above vanishes.
    As $\dfadj u = 0$ on $X$ by assumption, the proof is thus
    completed after applying $\fdiff{z^1} \ctrt$ to both sides. \qedhere
    % \begin{align*}
    %   \paren{\dfadj\HRes_p(u)}_{\ov{\boldsymbol\beta}_{q-2}}
    %   &=
    %   -g^{\bar\beta\alpha}
    %   \nabla_\alpha 
    %   \paren{
    %   g^{\bar j1}
    %   u_{\bar j\bar\beta\ov{\boldsymbol\beta}_{q-2}}
    %	}
    %   \\
    %   &
    %   =
    %   -g^{\bar\beta\alpha}
    %   \del_\alpha 
    %   \paren{
    %   g^{\bar j1}
    %   u_{\bar j\bar\beta\ov{\boldsymbol\beta}_{q-2}}
    %	}
    %   +
    %   g^{\bar\beta\alpha}
    %   \paren{\del_\alpha\varphi_{K_{D_p}}-\del_\alpha\varphi_L}
    %   g^{\bar j1}
    %   u_{\bar j\bar\beta\ov{\boldsymbol\beta}_{q-2}}
    %   \\
    %   &
    %   =
    %   -
    %   \sum_{\beta=2}^n
    %   \del_\beta
    %   \paren{
    %   g^{\bar j1}
    %   u_{\bar j\bar\beta\ov{\boldsymbol\beta}_{q-2}}
    %	}
    %   -
    %   \sum_{\beta=2}^n
    %   \paren{\del_\beta\varphi_L}
    %   u_{\bar1\bar\beta\ov{\boldsymbol\beta}_{q-2}}\;\;\;\text{at}\;\;x.
    % \end{align*}
    % The first term is computed as
    % \begin{align*}
    %   -
    %   \sum_{\beta=2}^n
    %   \del_\beta
    %   &
    %   \paren{
    %   g^{\bar j1}
    %   u_{\bar j\bar\beta\ov{\boldsymbol\beta}_{q-2}}
    %	}
    %   =
    %   -
    %   \sum_{\beta=2}^n
    %   \paren{
    %   \del_\beta g^{\bar j1}
    %   u_{\bar j\bar\beta\ov{\boldsymbol\beta}_{q-2}}
    %   +
    %   g^{\bar j1}
    %   \del_\beta
    %   u_{\bar j\bar\beta\ov{\boldsymbol\beta}_{q-2}}
    %	}
    %   \\
    %   &=
    %   \sum_{\beta=2}^n
    %   \paren{
    %   g^{\bar jk}\paren{\del_\beta g_{k\bar l}}g^{\bar l1}
    %   u_{\bar j\bar\beta\ov{\boldsymbol\beta}_{q-2}}
    %   -
    %   \del_\beta
    %   u_{\bar1\bar\beta\ov{\boldsymbol\beta}_{q-2}}
    %	}
    %   \\
    %   &=
    %   \sum_{\beta=2}^n
    %   \paren{
    %   \paren{\del_\beta g_{1\bar1}}
    %   u_{\bar1\bar\beta\ov{\boldsymbol\beta}_{q-2}}
    %   -
    %   \del_\beta
    %   u_{\bar1\bar\beta\ov{\boldsymbol\beta}_{q-2}}
    %	}
    %   +
    %   \sum_{\beta,\gamma=2}^n
    %   \paren{\del_\beta g_{\gamma\bar1}}
    %   u_{\bar\gamma\bar\beta\ov{\boldsymbol\beta}_{q-2}}	.
    % \end{align*}
    % Since $\del_\beta g_{\gamma\bar1}$ is symmetric in $\beta, \gamma$ and $u_{\bar\gamma\bar\beta\bar\beta_1,\ldots,\bar\beta_{q-2}}$ is anti-symmetric in $\beta, \gamma$, the last term vanishes.
    % It follows that
    % \begin{align*}
    %   \paren{\dfadj\HRes_p(u)}_{\ov{\boldsymbol\beta}_{q-2}}
    %   &=
    %   \sum_{\beta=2}^n
    %   \paren{
    %   \paren{\del_\beta g_{1\bar1}}
    %   u_{\bar1\bar\beta\ov{\boldsymbol\beta}_{q-2}}
    %   -
    %   \del_\beta
    %   u_{\bar1\bar\beta\ov{\boldsymbol\beta}_{q-2}}
    %	}
    %   -
    %   \sum_{\beta=2}^n
    %   \paren{\del_\beta\varphi_L}
    %   u_{\bar1\bar\beta\ov{\boldsymbol\beta}_{q-2}}
    %   \\
    %   &=
    %   \paren{\dfadj u}_{\bar1\bar\beta\ov{\boldsymbol\beta}_{q-2}}=0.
    % \end{align*}
    % Indeed, at the given point $x$, we have
    % \begin{align*}
    %   \paren{\dfadj u}_{\bar1\ov{\boldsymbol\beta}_{q-2}}
    %   &=
    %   -g^{\bar jk}\paren{\nabla_ku_{\bar j\bar 1\ov{\boldsymbol\beta}_{q-2}}}
    %   =
    %   -g^{\bar jk}
    %   \paren{
    %   \del_ku_{\bar j\bar1\ov{\boldsymbol\beta}_{q-2}}
    %   -
    %   \paren{
    %			\del_k\varphi_{K_X}
    %			-
    %			\del_k\varphi_L
    % }
    %   u_{\bar j\bar1\ov{\boldsymbol\beta}_{q-2}}
    %	}
    %   \\
    %   &=
    %   -
    %   \sum_{\beta=2}^n
    %   \paren{
    %   \del_\beta
    %   u_{\bar\beta\bar1\ov{\boldsymbol\beta}_{q-2}}
    %   -
    %   \paren{
    %			\del_\beta g_{1\bar1}
    %			-
    %			\del_\beta\varphi_L
    % }
    %   u_{\bar\beta\bar1\ov{\boldsymbol\beta}_{q-2}}
    %	}.
    % \end{align*}
    % This completes the proof.
  \end{proof}


  Furthermore, we claim that, if
  % $u\in\mathcal{H}^{n,q}(X;L)_{\varphi_L}$ with
  % $\ibddbar\varphi_L\ge0$
  $u$ satisfies $\dbar u = 0$ and $\nabla^{(0,1)}u = 0$, then
  $\HRes_p(u)$ is $\dbar$-closed.
  % One can notice that $u$ is $\dbar$-closed, then so is
  % $\HRes_p(u)$. 
  This is shown via the following formula, which is a special case and
  a slight variant of \cite{Donnelly&Xavier}*{(2.4)} and
  \cite{Ohsawa&Takegoshi-spectral_seq}*{Prop.~1.5} (see also
  \cite{Takegoshi_higher-direct-images}*{(1.9)} and
  \cite{Matsumura_injectivity-Kaehler}*{Lemma 2.1}). 
  

  \newcommand{\lcSb}{\lcS+1[b]}
  \newcommand{\idxj}{\idx[\conj j]}
  
  \begin{lemma}[cf.~\cite{Donnelly&Xavier}*{(2.4)},
    \cite{Ohsawa&Takegoshi-spectral_seq}*{Prop.~1.5},
    \cite{Takegoshi_higher-direct-images}*{(1.9)} and
    \cite{Matsumura_injectivity-Kaehler}*{Lemma
      2.1}] \label{lem:commutator-dbar-ctrt}
    Let $\vphi$ be a smooth function and $u$ be a smooth
    ($K_X$-valued) $(0,q)$-form on a K\"ahler manifold.
    They satisfy the formula
    \begin{equation*}
      \dbar\paren{\idxup{\diff\vphi}.  u}
      =\idxup{\ibddbar\vphi} . u
      -\idxup{\diff\vphi} . \paren{\dbar u}
      +\idxup{\diff\vphi} \cdot \nabla^{(0,1)}_\bullet u \; ,
    \end{equation*}%
    or, when a local holomorphic coordinate system is fixed and
    the Einstein summation convention is applied, 
    \begin{equation*}
      \paren{\dbar\paren{\idxup{\diff\vphi} . u}}_{\conj J_{q}}
      =\sum_{\nu=1}^q \diff^{\conj\ell} \diff_{\conj j_\nu} \vphi \:
      u_{\idxj 1[\dotsm (\conj \ell)_\nu].q}
      -\diff^{\conj\ell}\vphi  \:\paren{\dbar u}_{\conj\ell\conj J_q}
      +\diff^{\conj\ell}\vphi \:\nabla_{\conj\ell} u_{\conj J_q} 
    \end{equation*}
    for any multi-indices $J_q = (\idx[j]1,q)$, pointwisely.
  \end{lemma}

  \begin{proof}
    A direct computation yields
    \begin{align*}
      \paren{\dbar\paren{\idxup{\diff\vphi} . u}}_{\conj
      J_{q}}
      &=\sum_{\nu=1}^q (-1)^{\nu-1} \diff_{\conj j_\nu}
        \paren{\idxup{\diff\vphi}.  u}_{\idxj 1[\dotsm \widehat
        {\conj j}_\nu].q}
        =\sum_{\nu=1}^q (-1)^{\nu-1} \diff_{\conj j_\nu}
        \paren{\diff_{\ell}\vphi \: u^\ell_{\;\idxj 1[\dotsm
        \widehat{\conj j}_\nu].q}}
      \\
      &=\sum_{\nu=1}^q (-1)^{\nu-1} \paren{
        \diff_{\conj j_\nu}\diff_{\ell}\vphi \: u^{\ell}_{\;\idxj 1[\dotsm
        \widehat {\conj j}_\nu].q}
        +\diff_{\ell}\vphi \: \nabla_{\conj j_\nu} u^\ell_{\;\idxj 1[\dotsm
        \widehat {\conj j}_\nu].q}
        }
      \\
      &=\sum_{\nu=1}^q
        \diff^{\conj \ell}\diff_{\conj j_\nu}\vphi \: u_{\idxj 1[\dotsm
        (\conj\ell)_\nu].q}
        -\diff^{\conj\ell}\vphi \sum_{\nu=1}^q (-1)^{\nu} 
        \nabla_{\conj j_\nu} u_{\conj\ell \idxj 1[\dotsm
        \widehat {\conj j}_\nu].q}
        \begin{aligned}[t]
          &-\diff^{\conj\ell}\vphi \: \nabla_{\conj \ell} u_{\conj
            J_q} \\
          &+\diff^{\conj\ell}\vphi \: \nabla_{\conj \ell}
          u_{\conj J_q}
        \end{aligned}
      \\
      &=\sum_{\nu=1}^q
        \diff^{\conj \ell}\diff_{\conj j_\nu}\vphi \: u_{\idxj 1[\dotsm
        (\conj\ell)_\nu].q}
        -\diff^{\conj\ell}\vphi
        \:\paren{\dbar u}_{\conj\ell\conj J_q}
        +\diff^{\conj\ell}\vphi \: \nabla_{\conj \ell} u_{\conj
        J_q} \; . \qedhere
    \end{align*}
    % as desired.
  \end{proof}



  We see that $\HRes_p(u)$ is $\dbar$-closed by
  putting $z^1$ in place of $\vphi$ in Lemma \ref{lem:commutator-dbar-ctrt}.
  The following theorem is then immediate.
  \begin{thm} \label{thm:residue-harmonic}
    If $u$ is a harmonic $K_X\otimes L$-valued $(0,q)$-form on $X$ with
    respect to $\vphi_L$ and $\omega$ such that $\nabla^{(0,1)}u=0$,
    then $\HRes_p(u)$ is a harmonic $K_{D_p}\otimes
    L\vert_{D_p}$-valued $(0,q-1)$-form on $D_p$ with respect to
    $\varphi\vert_{D_p}$ and $\omega\vert_{D_p}$.
  \end{thm}

  \begin{proof}
    From the above discussion, $\HRes_p(u)$ is
    $\dbar$- and $\dfadj$-closed on $D_p$.
    Since $\varphi_L$ is smooth and $D_p$ is compact, it follows that
    $\HRes_p(u)\in\Dom\dbar^*$ with
    respect to $\varphi_L\vert_{D_p}$ and $\omega\vert_{D_p}$.
    This completes the proof.
  \end{proof}

  \begin{remark}
    When $\varphi_L$ has the singularity property described in
    \cite{Chan&Choi_injectivity-I}*{\S 2.2 item (2)} for $\varphi_F$,
    i.e.~$\varphi_L$ has only neat analytic singularities such that
    $P_L:=\varphi_L^{-1}(-\infty)$ is a divisor with $P_L+D$ having
    snc and that $P_L$ contains no components of $D$, the claim that
    $\HRes_p(u)\in\Dom\dbar^*$ with
    respect to $\norm\cdot_{D_p}:=\norm\cdot_{D_p,\varphi_L,\omega}$
    still holds true (under the assumption that $\omega\vert_{D_p}$ is
    a complete K\"ahler form on $D_p\setminus P_L$).
    Indeed, $\HRes_p(u)$ can be shown to be $L^2$
    with respect to $\norm\cdot_{D_p}$ by the arguments in
    \cite{Chan&Choi_injectivity-I}*{Prop.~3.2.3, Remark 3.2.4 and
      Prop.~3.3.2} (with $u$ here in place of $\frac{\rs u}{\sect_D}$
    there).
    With $\dfadj$ (with respect to $\varphi_L\vert_{D_p}$ and
    $\omega\vert_{D_p}$) being a smooth operator on $D_p\setminus P_L$
    % (different from the situation in Lemma \ref{lem:su-harmonicity})
    and $\omega\vert_{D_p}$ being complete,
    $\HRes_p(u)\in\Dom\dbadj$ follows
    from the classical arguments.
  \end{remark}

}




% \section{Proof of the Main Result}\label{sec:proof}

% \subsection{Proof of Corollary \ref{cor:main}}\label{subsec:n2}

% Corollary \ref{cor:main} can be  proved by repeating the same argument as in the proof of Theorem \ref{thm:main}. 
% Nevertheless, in this subsection, we deduce Corollary \ref{cor:main} from Theorem \ref{thm:main} 
% using the previous work \cite[Theorem 1.6]{Mat}. 




% \begin{proof}[Proof of Corollary \ref{cor:main}]
% %In the proof, we freely use the notation in Conjecture \ref{conj:fujino}. 
% Let us consider the following commutative diagram  induced by 
% the short exact sequence $0 \to K_{X} \to K_{X}\otimes D \to K_{D} \to 0 $ and the multiplication map: 
% \begin{align*}
% \vcenter{ \xymatrix{
% &\ar[d] & \ar[d]\\
% &
% H^q(X,  K_{X}\otimes F )\ar[d]^-{}\ar[r] ^-{\otimes s}
% \ar[d]^-{\otimes \sect_D}
% &H^q(X, K_{X}  \otimes F^{ \otimes{(m+1)}} )\ar[d]\\ 
% &H^q(X,  K_{X}\otimes D \otimes F)
% \ar[d]^-{}\ar[r] ^-{\otimes s}
% &H^q(X,  K_{X} \otimes D \otimes F^{\otimes{(m+1)}}) \ar[d]\\ 
% &H^q(D, K_{D}\otimes F )
% \ar[d]\ar[r]^-{\otimes s|_{D} } 
% & H^q(D,  K_{D}\otimes F^{\otimes(m+1)} ). \ar[d]\\ 
% & & 
% }}
% \end{align*}
% The line bundle $M:=F^{\otimes m}$ with the metric $h_{M}:=h_{F}^{\otimes m}$ 
% satisfies the curvature assumption in Theorem \ref{thm:main}. 
% Further, the zero locus $s|_{D}^{-1}(0)$ contains no lc centers of $(X, D)$ by assumption; 
% hence, by Theorem \ref{thm:main}, the lowest multiplication map $\otimes s|_{D}$ in the diagram is injective for every $q$. 
% This implies that a cohomology class $\alpha \in H^q(X,  K_{X}\otimes D \otimes F)$ with 
% $s  \alpha =0 \in H^q(X,  K_{X}\otimes D \otimes F^{\otimes (m+1)})$ 
% lies in the image of the vertical multiplication map $\otimes \sect_D$ on the left, 
% where $\sect_D$ is the canonical section of the effective divisor $D$. 
% Then, the conclusion of $\alpha=0$ follows from \cite[Theorem 1.6]{Mat} (or \cite[]{CC}). 
% \end{proof}




\section{Proofs of the main results}\label{sec:proof}

\subsection{Proof of Corollary \ref{cor:main}}\label{subsec:n2}

Corollary \ref{cor:main} can be proved by adapting the % same argument as in the
proof of Theorem \ref{thm:main} (or, more precisely, Theorem
\ref{thm:ker-nu=ker-tau}; see Remark \ref{rem:general-commut-diagram}
for details).
The proof involves an inductive reduction of the setup to
subvarieties on which the relevant injectivity result is known
or can be proved via Enoki's arguments (i.e.~harmonic theory for
cohomology is valid).
To get an essence of the argument, here 
% Nevertheless, in this subsection,
we deduce Corollary \ref{cor:main} from Theorem \ref{thm:main} 
using the previous work \cite{Matsumura_injectivity-lc}*{Thm.~1.6} (or
\cite{Chan&Choi_injectivity-I}*{Thm.~1.2.1}).




\begin{proof}[Proof of Corollary \ref{cor:main}]
%In the proof, we freely use the notation in Conjecture \ref{conj:fujino}. 
Consider the following commutative diagram  induced by 
the short exact sequence $0 \to K_{X} \to K_{X}\otimes D \to K_{D} \to
0 $ and the multiplication map $\otimes s$: 
\begin{equation*}
  % \vcenter{
  \xymatrix@R=3ex{
    \ar[d] & \ar[d]\\
    {H^q(X,  K_{X}\otimes F )} \ar[d]^-{}\ar[r] ^-{\otimes s}
    \ar[d]^-{\otimes \sect_D}
    &{H^q(X, K_{X}  \otimes F^{ \otimes{(m+1)}} )} \ar[d]\\ 
    {H^q(X,  K_{X}\otimes D \otimes F)}
    \ar[d]^-{}\ar[r] ^-{\otimes s}
    &{H^q(X,  K_{X} \otimes D \otimes F^{\otimes{(m+1)}})} \ar[d]\\ 
    {H^q(D, K_{D}\otimes F )}
    \ar[d]\ar[r]^-{\otimes s|_{D} } 
    & {H^q(D,  K_{D}\otimes F^{\otimes(m+1)} ) \; .}  \ar[d]\\ 
    & 
  }
  % }
\end{equation*}
The line bundle $M:=F^{\otimes m}$ with the metric
$h_{M}:=h_{F}^{\otimes m} = e^{-m\vphi_F}$ satisfies the curvature
assumption in Theorem \ref{thm:main} and the zero locus $s^{-1}(0)$
contains no lc centers of $(X, D)$ by assumption.
Hence, by Theorem \ref{thm:main}, the lowest multiplication map $\otimes s|_{D}$ in the diagram is injective for every $q$. 
This implies that a cohomology class $\alpha \in H^q(X,  K_{X}\otimes D \otimes F)$ with 
$s  \alpha =0 \in H^q(X,  K_{X}\otimes D \otimes F^{\otimes (m+1)})$ 
lies in the image of the vertical multiplication map $\otimes \sect_D$ on the left, 
where $\sect_D$ is the canonical section of the effective divisor $D$. 
Then, the conclusion $\alpha=0$ follows from \cite{Matsumura_injectivity-lc}*{Thm.~1.6} 
(or \cite{Chan&Choi_injectivity-I}*{Thm.~1.2.1}). 
\end{proof}



\subsection{Proof of Theorem \ref{thm:main} for a simple case} % \label{subsec:n2}
\label{sec:proof-of-simple-case}

In this subsection, we prove Theorem \ref{thm:main} in the simple case 
where \emph{$D$ has two components (i.e.~$D=D_{1}+D_{2}$) whose intersection
has only one irreducible component} and the degree of cohomology
groups is $q=1$.
% This simple case is completely contained in the general case discussed in Section \ref{subsec:general}, 
% but we illustrate a detailed proof, which is quite helpful in understanding the essence of the proof. 
% The proof of the general case is an extension of the argument in this section using lc strata. 
While this case is contained in the proof presented in Section
\ref{subsec:general}, a detailed proof of it is presented here in
order to illustrate the essence of the proof in the general case
without being obscured by the notation.
The proof in Section \ref{subsec:general} follows the same arguments
but on the lower-dimensional lc strata with more components.


\begin{proof}[Proof of Theorem \ref{thm:main} in the case of $D=D_{1}+D_{2}$ and $q=1$] 

% Suppose that $D=D_{1}+D_{2}$ and $q=1$ in Theorem \ref{thm:main}. 
Under the given assumptions and for a given cohomology class $\alpha
\in H^1(D,  K_{D} \otimes F)$, we prove here that $\alpha $ is actually $0$
when $s  \alpha =0 \in H^1(D,  K_{D} \otimes F \otimes M)$. 

\begin{step}[``Harmonic representative'' of $\alpha$] \label{step:harmonic-rep}
We intend to work with an \emph{``optimal''} representative of $\alpha$ via the Dolbeault isomorphism, 
% which means an appropriate harmonic form has the minimum $L^{2}$-norm in the forms representing $\alpha$.  
in the analogy of a harmonic form being the element with the minimal
$L^2$ norm in the corresponding cohomology class.
Nevertheless, at the time of writing, there is not yet a well
established theory of Dolbeault isomorphism and harmonic theory for
cohomology groups on the singular space $D$.
For this purpose, we consider the following diagram
\begin{equation}\label{h}
  \begin{aligned}
    \xymatrix@C=3.5em@R=3.5ex{
      \ar[d] & \ar[d]\\
      {\smash{\bigoplus_{p=1}^{2}} H^1(D_{p}, K_{D_{p}}\otimes F )}
      \ar[d]^-{}\ar[r] ^-{\otimes (s|_{D_{1}}, s|_{D_{2}})}
      \ar[d]^-{\tau}
      &{\smash{\bigoplus_{p=1}^{2}} H^1(D_{p},
        K_{D_{p}} \otimes F \otimes M )}
      \ar[d]\\
      {H^1(D, K_{D} \otimes F)} \ar[d]^-{}\ar[r] ^-{\otimes s}
      &{H^1(D,  K_{D} \otimes F\otimes M)} \ar[d]\\
      {H^1(D_{1}\cap D_{2}, K_{D_{1}\cap D_{2}}\otimes F )}
      \ar[d]\ar[r]^-{\otimes s|_{D_{1}\cap D_{2}} }
      &{H^1(D_{1}\cap D_{2},  K_{D_{1}\cap D_{2}}\otimes F \otimes M)} \ar[d]\\
      & }
  \end{aligned}
\end{equation}
induced from $0 \to K_{D_{1}} \oplus K_{D_{2}} \to K_{D} \to K_{D_{1} \cap D_{2}} \to 0$, 
% which corresponds to $\eqref{eq-ex2}$ in the case of $\rho=0, \sigma=1, \tau=2$ (cf.\,\eqref{eq-ex}). 
which in turn can be obtained by tensoring $K_X \otimes D$ to the
short exact sequence of adjoint ideal sheaves
\begin{equation*}
  \renewcommand{\objectstyle}{\displaystyle}
  \xymatrix@R=3.5ex{
    {0} \ar[r]
    &{\faidlof|1|/|0|*} \ar[r] \ar[d]_-{\Res^1}^-{\isom}
    &{\faidlof|2|/|0|*} \ar[r]
    &{\faidlof|2|/|1|*} \ar[r] \ar[d]^-{\Res^2}_-{\isom}
    &{0}
    \\
    &{\residlof|1|*} %\ar@{}[u]|-*[left]+{\isom}
    &&{\residlof|2|*} %\ar@{}[u]|-*[left]+{\isom}
  }
\end{equation*}
where $\aidlof* :=\aidlof$ and $\residlof* := \residlof$ and
the isomorphism $\faidlof/-1* \xrightarrow[\isom]{\Res^\sigma}
\residlof*$ is induced from the residue short exact sequence in
Section \ref{subsec:residue}.
Notice that the Dolbeault isomorphism and harmonic theory are
valid on $D_1$, $D_2$ and $D_1 \cap D_2$.
The multiplication map $\otimes s|_{D_{1}\cap D_{2}}$ on the bottom
row is non-zero by the assumption on $s^{-1}(0)$ and the curvature
assumption is still satisfied after restricting $F$ and $M$ to
$D_{1}\cap D_{2}$.
Hence, Enoki's injectivity theorem can be invoked to assert that
$\otimes s|_{D_{1}\cap D_{2}}$ is injective. 
Then, by an easy diagram chasing, we can find harmonic forms $u_{p}$ for $p=1,2$ such that 
\begin{equation*}
  u_{p} \in \mathcal{H}^{n-1,1}(D_{p}; F)_{\vphi_F} \cong H^1(D_{p},  K_{D_{p}}\otimes F ) 
  \text{ with } \alpha = \tau(\eqcls{u_{1}}, \eqcls{u_{2}}), 
\end{equation*}
where $\mathcal{H}^{n-1,1}(D_{p}; F)_{\vphi_F}$ denotes 
the space of $F|_{D_{p}}$-valued harmonic forms of $(n-1,1)$-type with respect to $\res{e^{-\vphi_F}}_{D_{p}}$. 
Note that there is freedom in the choice of $(u_{1}, u_{2})$  
since $\tau$ may not be injective. 
% $(u_{1}, u_{2})$ may not be the best representation of  $\alpha$.
% For this reason, by the orthogonal decomposition, we re-
To obtain the unique ``optimal'' representative of $\alpha$, we choose
the pair $(u_{1}, u_{2})$ 
with $\alpha = \tau(\eqcls{u_{1}}, \eqcls{u_{2}})$ that satisfies
\begin{equation}\label{eq-orth}
  (u_{1}, u_{2}) \in (\Ker \tau)^{\perp} \subset 
  \Ker \tau \oplus (\Ker \tau)^{\perp}= \bigoplus_{p=1}^{2}
  \mathcal{H}^{n-1,1}(D_{p}; F)_{\vphi_F} \; ,
\end{equation}
% This can be regarded as the {\textit{best}} representation of  $\alpha$. 
% Our purpose is to prove that the $L^{2}$-norm 
% $\norm{s u_{1}}_{\vphi_M, D_1}^2 +\norm{s u_{2}}_{\vphi_M, D_2}^2$ 
% is actually zero. 
in which $\paren{\ker\tau}^\perp$ is the orthogonal complement of
$\ker\tau$ with respect to the (squared) residue norm
$\norm\cdot_{\lcc|1|'}^2 =\norm\cdot_{D_1}^2 +\norm\cdot_{D_2}^2$
(defined as in \eqref{eq:residue-norm} with $\sigma :=1$, $\vphi_L
:=\vphi_F$ and $\lcS[V,p] :=D_p$).
With such choice of representative, our goal is to prove that the
$L^{2}$-norm $\norm{s u_{1}}_{D_1, \vphi_M}^2 +\norm{s u_{2}}_{D_2,
  \vphi_M}^2$ is actually zero (where $\norm\cdot_{D_p, \vphi_M}$'s
are defined as in \eqref{eq:residue-norm} with $\vphi_L:=\vphi_F
+\vphi_M$).


\end{step}

\begin{step}[Obstruction for $\norm{s u_{1}}_{D_1, \vphi_M}^2 +\norm{s
      u_{2}}_{D_2, \vphi_M}^2$ from being zero]
  \label{item:expression-of-su-simple}
  

  % In this step, we examine the relevant $L^{2}$-norm 
  % to obtain an obstruction for our purpose as a $F$-valued form on $D_{1} \cap D_{2}$. 
  % For this purpose, by using the Dolbeault isomorphism, the Poincar\'e residue map, and the assumption of $s \alpha=0$, 
  % we prepare the following data: 
  We make use of the assumption $s \alpha=0$ and the \v
  Cech--Dolbeault isomorphism to re-express $\norm{s u_{1}}_{D_1,
    \vphi_M}^2 +\norm{s u_{2}}_{D_2, \vphi_M}^2$ as follows.


  \begin{itemize}
  \item Take $\alpha_{p;\:ij}   \in H^{0}(V_{ij} \cap D_p ,
    K_{D_{p}}\otimes F)$  
    for every open set $V_{ij} :=V_i \cap V_j$ with $i,j \in I$ and
    $V_i \cap V_j \cap D_p\neq \emptyset$ such that the family
    $\{\alpha_{p;\:ij}\}_{i,j \in I}$ is a \v Cech cocycle
    representing $u_{p}$ via the \v Cech--Dolbeault isomorphism on
    $D_p$.
    It follows that there exists an $L^2$ section $v_{p,(2)}$ of
    $K_{D_p} \otimes \res F_{D_p}$ on $D_p$ with respect to
    $\norm\cdot_{D_p}$ such that (under Einstein summation
    convention) 
    \begin{equation*}
      u_p
      \overset{\text{\eqref{eq:Cech-Dolbeault-isom}}}= \:
      \dbar v_{p,(2)} -\dbar\rho^j \cdot \rho^i \:\alpha_{p;\:ij} \; .
    \end{equation*}


  \item Take $f_{ij} \in H^{0}(V_{ij} , K_X \otimes D
    \otimes F \otimes \defidlof{D_1 \cap D_2})$ for $i,j \in I$
    satisfying 
    \begin{equation*}
      \Res^1\paren{f_{ij}}
      :=\paren{\PRes[D_1](\frac{f_{ij}}{\sect_D}) \:,\:
        \PRes[D_2](\frac{f_{ij}}{\sect_D})} 
      = \paren{\alpha_{1;\:ij} , \alpha_{2;\:ij}} 
    \end{equation*}
    whose existence are guaranteed by the surjectivity of the
    residue isomorphism $\Res^1$ on Stein open sets such that
    \begin{equation*}
      \renewcommand{\objectstyle}{\displaystyle}
      \xymatrix@C=1em@R=0.8em{
        {K_X \otimes D \otimes \frac{\defidlof{D_1 \cap
              D_2}}{\defidlof{D}}} 
        \ar@{}[r]|-*+{=}
        \ar@{}[d]|*[left]{\in}
        &{K_X \otimes D \otimes \faidlof|1|/|0|*}
        \ar[rr]^-{\Res^1}_-{\isom}
        &&{K_X \otimes D \otimes \residlof|1|*}
        \ar@{}[r]|-*+{=}
        &{K_{D_1} \oplus K_{D_2}}
        \ar@{}[d]|(.57)*[left]{\in}
        \\
        *+/r 3em/{f_{ij} \bmod \defidlof{D}}
        \ar@{|->}[rrr]
        &&&*+/l 6em/{}
        &*-{\paren{\alpha_{1;\:ij} , \alpha_{2;\:ij}} \; .}
      }
    \end{equation*}
    It is easy to see that $\set{f_{ij} \bmod \defidlof{D}}_{i,j \in
      I}$ is a \v Cech cocycle whose cohomology class in $\cohgp
    1[D]{\logKX \otimes \frac{\defidlof{D_1 \cap
          D_2}}{\defidlof{D}}}$ is mapped to $\alpha$ via $\tau$. 

  \item The assumption $s \alpha=0$ in $\cohgp 1[D]{K_D
      \otimes F \otimes M}$ guarantees the
    existence of $\lambda_{i} \in H^{0}(V_{i}, K_X \otimes D\otimes
    F \otimes M)$ for $i \in I$ such that
    \begin{equation*}
      s f_{ij} \equiv \lambda_j -\lambda_i \mod \defidlof{D}
      \quad\text{ on } V_{ij} \; .
    \end{equation*}
    Note that the coefficients of $\lambda_i$ need not lie in
    $\defidlof{D_1 \cap D_2}$ even though so do those of $f_{ij}$.
    By setting
    \begin{equation*}
      \rs*\lambda_{p;\:i} := \PRes[D_p](\frac{\lambda_i}{\sect_D})
      \cdot \sect_{(p)}
      \quad\text{ on $V_i \cap D_p$ for } i\in I
      \text{ and } p = 1,2 \; ,
    \end{equation*}
    it then follows that
    \begin{equation*}
      s\alpha_{p;\:ij} \sect_{(p)} =\rs*\lambda_{p;\:j} -\rs*\lambda_{p;\:i}
      \quad\text{ on } V_{ij} \cap D_p \; .
    \end{equation*}
    Note that $\rs*\lambda_{p;\:i}$ is holomorphic on $V_i \cap D_p$
    (while $\PRes[D_p](\frac{\lambda_i}{\sect_D})$ may not be).
  \end{itemize}

  Since $u_{p}$ is harmonic with respect to $\vphi_F$ on $D_p$ and
  we have $\ibddbar\vphi_F \geq 0$ and $%-C\omega \leq
  \ibddbar\vphi_M \leq C\ibddbar\vphi_F$ on $D_p$ for some constant
  $C > 0$ by assumption,
  Proposition \ref{prop:consequence-of-positivity} guarantees that  
  % \begin{equation*} %\label{eq-harmonic}
  %   su_p \in \Harm'/n-1,1/<D_p>{F\otimes M},{\vphi_F+\vphi_M} \; ,
  % \end{equation*}%
  $su_p$ is harmonic with respect to $\vphi_F+\vphi_M$ on $D_p$,
  which is a consequence of Nakano's identity and Enoki's argument.
  It follows that $\iinner{s \dbar v_{p;(2)}}{su_p}_{D_p, \vphi_M}
  =\iinner{\dbar\paren{s v_{p;(2)}}}{su_p}_{D_p, \vphi_M} = 0$.
  Summarizing the above discussion, it follows that
  \begin{align*}
    \norm{s u_{p}}_{D_p, \vphi_M}^2 
    &= -\sum_{i,j\in I}\iinner{\dbar\rho^{j} \cdot \rho^i \:s
      \alpha_{p;\:ij} \:}{\:s u_p}_{D_p, \vphi_M}\\
    &= -\sum_{i,j\in I}\iinner{\dbar\rho^{j} \cdot \rho^i \:s
      \alpha_{p;\:ij} \sect_{(p)} \:}{\:s u_p \sect_{(p)}}_{D_p, \vphi_M+\phi_{(p)}}\\
    &= -\sum_{i,j\in I}\iinner{\dbar\rho^{j} \cdot \rho^i
      \paren{\rs*\lambda_{p;\:j}- \rs*\lambda_{p;\:i}} \:}
      {\:s u_p \sect_{(p)}}_{D_p, \vphi_M+\phi_{(p)}}\\
    &= -\sum_{j\in I}\iinner{\dbar\paren{\rho^j
      \rs*\lambda_{p;\:j}} \:}{\: s u_p \sect_{(p)}}_{D_p,
      \vphi_M+\phi_{(p)}}
      =: -\iinner{\dbar v_{p;(\infty)} }{ s u_p \sect_{(p)}}_{D_p,
      \vphi_M+\phi_{(p)}}
      \; .
  \end{align*}
  The notation $v_{p;(\infty)} :=\sum_{j\in I} \rho^j
  \rs*\lambda_{p;\:j}$ is used for the consistency with the notation in
  Proposition \ref{prop:res-formula-dbar-exact-dot-harmonic}.

  The residue computation in Proposition
  \ref{prop:res-formula-dbar-exact-dot-harmonic} further brings the
  expression of $\norm{s u_{p}}_{D_p, \vphi_M}^2$ for each $p=1,2$ to an inner
  product on $D_1 \cap D_2$.
  As $\lcS|2|[b] :=D_1 \cap D_2$ has only $1$ component, the index set $\Iset|2|
  =\set{b}$ is a singleton.
  Moreover, the general different $\Diff_{D_1 \cap D_2}(D) =\Diff_b(D)$ is
  trivial, so we choose its canonical section and the corresponding
  potential such that $\sect_{(b)} \equiv 1$ and $\phi_{(b)} \equiv 0$
  (and $\psi_{(b)} \equiv -1$) on $D_1 \cap D_2$.
  Let $\PRes[\lcS|2|[b] | D_p]$ be the Poincar\'e residue map from
  $D_p$ to $D_1 \cap D_2$.
  We fix the sign convention such that
  \begin{equation*}
    \rs*\lambda_{b;\:i}
    =\frac{\rs*\lambda_{b;\:i}}{\sect_{(b)}}
    :=\PRes[\lcS|2|[b]](\frac{\lambda_i}{\sect_D})
    \begin{aligned}[t]
      &= \PRes[\lcS|2|[b] | D_1] \circ
      \PRes[D_1](\frac{\lambda_i}{\sect_D})
      \\
      &=\PRes[\lcS|2|[b] | D_1](\frac{\rs*\lambda_{1;\:i}}{\sect_{(1)}})
    \end{aligned}
    \begin{aligned}[t]
      &=-\PRes[\lcS|2|[b] | D_2] \circ
      \PRes[D_2](\frac{\lambda_i}{\sect_D})
      \\
      &=-\PRes[\lcS|2|[b] | D_2](\frac{\rs*\lambda_{2;\:i}}{\sect_{(2)}})
    \end{aligned} \; .
  \end{equation*}
  Following the computation in Proposition
  \ref{prop:res-formula-dbar-exact-dot-harmonic}, we obtain
  \begin{align*}
    &~\norm{s u_{1}}_{D_1, \vphi_M}^2 +\norm{s u_{2}}_{D_2,
      \vphi_M}^2
    \\
    =&~-\iinner{\dbar v_{1;(\infty)} }{ s u_1 \sect_{(1)}}_{D_1,
       \vphi_M+\phi_{(1)}}
       -\iinner{\dbar v_{2;(\infty)} }{ s u_2 \sect_{(2)}}_{D_2,
       \vphi_M+\phi_{(2)}}
    \\
    =&~
       \begin{multlined}[t]
         \sum_{i\in I}\iinner{\rho^i \PRes[\lcS|2|[b] |
           D_1](\frac{\rs*\lambda_{1;\:i}}{\sect_{(1)}}) }{
           \: s\:\PRes[\lcS|2|[b] | D_1](\idxup{\diff\psi_{(1)}}. u_1)
         }_{D_1 \cap D_2, \vphi_M} \\
         +\sum_{i\in I}\iinner{\rho^i
           \PRes[\lcS|2|[b] |
           D_2](\frac{\rs*\lambda_{2;\:i}}{\sect_{(2)}}) }{
           \: s\:\PRes[\lcS|2|[b] | D_2](\idxup{\diff\psi_{(2)}}. u_2)
         }_{D_1 \cap D_2, \vphi_M}
       \end{multlined}
    \\
    =&~\iinner{\sum_{i\in
       I}\rho^i\rs*\lambda_{b;\:i} \:}{\: s\:\paren{
       \PRes[\lcS|2|[b] | D_1](\idxup{\diff\psi_{(1)}}. u_1)
       -\PRes[\lcS|2|[b] | D_2](\idxup{\diff\psi_{(2)}}. u_2)
       }}_{D_1 \cap D_2, \vphi_M}
    \\
    =:&~\iinner{v_{b;(\infty)}}{s w_b}_{D_1 \cap D_2, \vphi_M} \; ,
  \end{align*}
  which is the desired expression.

  It is shown below that
  \begin{equation} \label{eq:w-prelim-formula}
    w_b :=\PRes[\lcS|2|[b] | D_1](\idxup{\diff\psi_{(1)}}. u_1)
    -\PRes[\lcS|2|[b] | D_2](\idxup{\diff\psi_{(2)}}. u_2)
  \end{equation}
  is actually $0$ on $D_1 \cap D_2$, which will then conclude the proof.

  % \begin{itemize}
  % \item[$\bullet$] Take $\beta_{ij,p}   \in H^{0}(V_{ij}, K_{D_{p}}\otimes F)$ 
  %   such that the family $\{\beta_{ij,p}\}$ is a cocycle  corresponding to $u_{p}$ via the \v Cech--Dolbeault isomorphism. 




  % \item[$\bullet$] Take $\alpha_{ij} \in H^{0}(V_{ij}, K_X \otimes D
  %   \otimes \defidlof{D_1 \cap D_2} \otimes F)$
  %   satisfying that 
  %   \begin{equation*}
  %     \Res^1\paren{\alpha_{ij}}
  %     :=\paren{\PRes[D_1](\frac{\alpha_{ij}}{\sect_D}) \:,\:
  %     \PRes[D_2](\frac{\alpha_{ij}}{\sect_D})} 
  %     = \paren{\beta_{ij, 1}, \beta_{ij, 2}}, 
  %   \end{equation*}
  %   by the  residue isomorphism: 
  %   \begin{equation*}
  %     \xymatrix@R=0.1cm{
  %     *+/r 0.5cm/{K_{D_1} \oplus K_{D_2}} &
  %     *+/r 0.5cm/{
  %     K_X \otimes D \otimes \frac{\defidlof{D_1 \cap D_2}}{\defidlof{D}}=K_X \otimes D \otimes \frac{\aidlof|1|*}{\aidlof|0|*}.
  %   }
  %     \ar[l]_-{\Res^1}^-{\isom}
  %   } 
  %   \end{equation*}
  %   More specifically, we may define $\alpha_{ij}$ 
  %   by $\alpha_{ij}:=d\sect_{(2)}  \wedge \sect_{(1)} \:\beta_{ij, 1} +d\sect_{(1)}  \wedge\sect_{(2)} \:\beta_{ij, 2}$. 


  % \item[$\bullet$] Take $\lambda_{i} \in H^{0}(V_{i}, K_X \otimes D\otimes F \otimes M)$ 
  %   satisfying that 
  %   $$\text{
  %   $ s \alpha_{ij} \equiv \lambda_j -\lambda_i$  as a section of 
  %   $K_X \otimes D\otimes \mathcal{O}_{X}/\defidlof{D} \otimes F \otimes M =K_D \otimes F \otimes M$. 
  % }
  %   $$
  %   The cocycle $\{\alpha_{ij}\}$ of $K_X \otimes D \otimes F$ (noting that $\defidlof{D_1 \cap D_2}$ is not tensored) 
  %   corresponds to $\alpha$; hence the assumption of $s \alpha=0$ guarantees the existence of $\lambda_{i}$. 



  % \item[$\bullet$] Take $\rs \lambda_i^1$, $\rs \lambda_i^2$, and $\rs\lambda_i^{12}$ such that 
  %   \begin{equation*}
  %     \lambda_i
  %     =d\sect_{(2)}  \wedge \rs \lambda_i^1
  %     =d\sect_{(1)}  \wedge \rs \lambda_i^2
  %     =d\sect_{(2)}  \wedge d\sect_{(1)}  \wedge \rs\lambda_i^{12}. 
  %   \end{equation*}
  % \end{itemize}
  % By construction, we see that 
  % \begin{align*}
  %   &\bullet \text{$u_{p}=\dbar u_{(2), p} + \dbar \rho^{i} \beta_{ij,p} $ for some global section $u_{(2), p}$ of $K_{D_{p}}\otimes F$};\\
  %   &\bullet s \sect_{(p)} \alpha_{ij}= \rs\lambda_j^p- \rs\lambda_i^p \text{ on } D_{p} 
  %   \text{ as a section of }K_{D_{p}}\otimes F \otimes M.
  % \end{align*}
  % On the other hand, since $u_{p}$ is harmonic and $\sqrt{-1}\Theta_{h_{F}} \geq 0$, 
  % we can conclude that 
  % \begin{align}\label{eq-harmonic}
  %   {\nabla^{(0,1)} u_{p}} =0  \text{ and } \sqrt{-1} \Theta_{h_{F}} \Lambda_{\omega} u_{p} =0
  % \end{align}
  % by applying Nakano's identity. 
  % Further, together with the curvature assumption, 
  % Enoki's argument shows that $su_{p}$ is still harmonic with respect to $h_{F}h_{M}$. 
  % Then, we can easily see that 
  % \begin{align*}
  %   \norm{s u_{p}}_{\vphi_M, D_p}^2 
  %   &= \iinner{\dbar s u_{(2), p}  +  \dbar s \rho^{i} \beta_{ij,p}}{s u}_{\vphi_M, D_p}\\
  %   &= \iinner{ \dbar s \sect_{(p)} \rho^{i} \beta_{ij,p}}{s \sect_{(p)} u}_{ \phi_{(p)}+\vphi_M, D_p}\\
  %   &= \iinner{\dbar \rho^{i} (\rs\lambda_j^p- \rs\lambda_i^p)}
  %   {s \sect_{(1)} u}_{\phi_{(p)}+\vphi_M, D_2}\\
  %   &= \iinner{-\dbar\paren{\rho^i \rs\lambda_i^p}}{s \sect_{(p)}
  %   v}_{\phi_{(p)}+\vphi_M, D_p}.
  % \end{align*}
  % Here we use that $s u$ is still harmonic to get the second equality 
  % and that $\dbar \rho^{i} \rs\lambda_j^p=\dbar \rs\lambda_j^p=0$ to get the third equality. 
  % The right-hand side can be described by the norm on $D_{1}\cap D_{2}$ as follows: 
  % \begin{align*} 
  %   &~- \iinner{\dbar\paren{\rho^i \rs\lambda_i^p}}{s \sect_{(p)} u}_{\phi_{(p)}+\vphi_M, D_p} \\
  %   \xleftarrow{\varepsilon \tendsto 0^+}
  %   &~-\iinner{e^{-\varepsilon \abs{\psi_{(p)}}}\dbar\paren{\rho^i \rs\lambda_i^p}}{s \sect_{(p)}
  %   u}_{\phi_{(p)}+\vphi_M, D_p} \\
  %   =
  %   &
  %   \begin{aligned}[t]
  %     &~-\cancelto{0}{
  %     \iinner{ \dbar\paren{e^{-\varepsilon
  %     \abs{\psi_{(p)}}} \rho^i \rs\lambda_i^p} }{ s \sect_{(p)} u}
  %   }_{\phi_{(p)}+\vphi_M, D_p} 
  %     +\varepsilon \iinner{
  %     e^{-\varepsilon \abs{\psi_{(p)}}} \rho^i \rs\lambda_{i}^p }{(\diff\psi_{(p)})^{*}s \sect_{(p)} u }_{\phi_{(p)}+\vphi_M, D_p}
  %   \end{aligned}
  %   \\
  %   =
  %   &~ \varepsilon \iinner{
  %   \frac{\rho^i \rs\lambda_{i}^p}{\sect_{(p)}}
  % }{
  %   e^{-\varepsilon \abs{\psi_{(p)}}}
  %   (\diff \log \abs{\sect_{(p)}^2})^{*}
  %   u \: s e^{-\vphi_M} }_{D_p}
  %   -\underbrace{
  %   \varepsilon \iinner{
  %   \frac{\rho^i \rs \lambda_{i}^p}{\sect_{(p)}}
  % }{
  %   e^{-\varepsilon \abs{\psi_{(p)}}}
  %   (\diff \sm\vphi_{(p)})^{*}  u \:s e^{-\vphi_M} }_{D_p}
  % }_{=\: \BigO(\varepsilon)}
  %   \\
  %   =
  %   &~\varepsilon \iinner{
  %   \rho^i \smash[b]{\underbrace{\rs\lambda_{i}^p}_{\mathclap{=\: d\sect_{(1)} 
  %   \wedge \rs\lambda_i^{12}}}}
  %   \:
  % }{ \:
  %   \frac{e^{-\varepsilon \abs{\psi_{(p)}}}}{\abs{\sect_{(p)}}^2}
  %   (d\sect_{(p)})^{*} \smash[b]{\underbrace{u_{p}}_{\mathclap{=:\: d\sect_{(1)}  \wedge \rs u_{p}^{12}}}} \: s e^{-\vphi_M} }_{D_p}
  %   + \BigO(\varepsilon)
  %   \vphantom{\underbrace{\rs\lambda_{i}^1}_{\mathclap{=\: d\sect_{(1)} 
  %   \wedge \rs\lambda_i^{12}}}}
  %   \\
  %   \xrightarrow{\varepsilon \tendsto 0^+}
  %   &~\iinner{\rho^i \rs\lambda_i^{12}}{  (d\sect_{(p)})^{*}  \rs u_{p}^{12}
  %   \: s e^{-\vphi_M}}_{D_1 \cap D_2}  \; .
  % \end{align*}

  % Considering the inner product above, we define the $F$-valued form $w$ on $D_{1} \cap D_{2}$ by 
  % \begin{equation*}
  %   w:=(d\sect_{(1)})^{*}  \rs u_{1}^{12}-(d\sect_{(2)})^{*} \rs u_{2}^{12}. 
  % \end{equation*}
  % From the next step, we aim to show $w$ is actually zero, which finishes the proof. 

  % Note that $u_{p}^{12}$ and $d\sect_{(p)}$ are defined only locally; 
  % hence $d\sect_{(p)})^{*}  \rs u_{p}^{12}$ does not determine a section on $D_{p}$, 
  % but determines the $F$-valued section form of type $(n-2, q-1)$ on $D_{1} \cap D_{2}$. 
  % This can be verified by calculating a glueing condition. 
  % Another way to see this is to apply The Poincar\'e residue map from $D_{p}$ to $D_1 \cap D_2$, 
  % which yields
  % \begin{equation*}
  %   \PRes[D_1 \cap D_2]( (\diff\psi_{(p)})^{*} u)
  %   =\parres{(d\sect_{(p)})^{*}  \rs u_{p}^{12}}_{D_1 \cap D_2}
  %   \quad\text{(recall that $\sect_{(p)} =\sect_{(p)} $)} \; .
  % \end{equation*}
  % Since $\psi_{(p)}=\phi_{(p)} -\sm\vphi_{(p)}$ is a global function, 
  % the right hand side is globally defined on $D_{p}$, and so is $((d\sect_{(p)})^{*}  \rs u_{p}^{12})|_{D_1 \cap D_2}$. 
\end{step}



\begin{step}[$w_b$ being holomorphic and thus {$w_b \in \cohgp 0[D_1
    \cap D_2]{K_{D_1 \cap D_2} \otimes F}$}]
  
  We prove that $\dbar w_b = 0$ on $\lcS|2|[b] :=D_1 \cap D_2$ by a
  direct computation given in Section \ref{subsec:harmonic}.
  Indeed, it suffices to show that each summand $\PRes[\lcS|2|[b] |
  D_p](\idxup{\diff\psi_{(p)}}. u_p)$ for $p=1,2$ in $w_b$ is
  $\dbar$-closed.
  The computations are identical, so it suffices to consider $p=1$.

  On an admissible open set $V$ such that $D_p  \cap V =\set{z_p =
    0}$ for $p=1,2$ and $\lcS|2|[b] \cap V = \set{z_1 = z_2 = 0} =
  D_1 \cap \set{z_2 = 0}$,
  we have
  \begin{equation*}
    \diff\psi_{(1)} =\frac{dz_2}{z_2} -\diff\sm\vphi_{(1)}
    \quad\text{ on } V \; .
  \end{equation*}
  By writing
  \begin{equation*}
    \idxup{dz_2}. u_1 =: dz_2 \wedge \paren{\idxup{dz_2}. \rs*u_{1,2}}
    \quad\text{ on } D_1 \cap V \; ,
  \end{equation*}
  where $\rs*u_{1,2}$ is a $(n-2,1)$-form on $D_1 \cap V$, we see
  that
  \begin{equation*}
    \PRes[\lcS|2|[b] | D_1](\idxup{\diff\psi_{(1)}}. u_1)
    =\PRes[\set{z_2 = 0}](\frac{\idxup{dz_2} .u_1}{z_2})
    =\parres{\idxup{dz_2}. \rs*u_{1,2}}_{\lcS|2|[b]}
    \quad\text{ on } D_1 \cap D_2 \cap V \; .
  \end{equation*}
  Therefore, it suffices to check that $\idxup{dz_2}. u_1$ is
  $\dbar$-closed on $D_1 \cap V$.
  As $u_1$ is harmonic and $\ibddbar\vphi_F \geq 0$, we have
  $\nabla^{(0,1)} u_1 = 0$ by Proposition
  \ref{prop:consequence-of-positivity} and Lemma
  \ref{lem:commutator-dbar-ctrt} yields the desired result (with $z_2$
  in place of $\vphi$ in the lemma).

  % In this step, we show that $w$ is a $F$-valued harmonic on $D_{1} \cap D_{2}$. 
  % Note that it is sufficient to show that $\dbar w =0$ in our case since the type of $w$ is $(n-2, q-1)=(n-2, 0)$ by $q=1$. 
  % By \cite[(1.9)]{Takegoshi_higher-direct-images} and \eqref{eq-harmonic}, we obtain that 
  % \begin{equation*}
  %   \dbar ( (d\sect_{(p)})^{*} u_{p}) 
  %   = \big( (i \partial  \dbar \sect_{(p)})^{*} - (\partial \sect_{(p)})^{*} \dbar + \partial \sect_{(p)}\nabla^{(0,1)} \big)u_{p}
  %   = 0 \quad\text{on   } D_1. 
  % \end{equation*}
  % By noting that $u_{p} = d \sect_{(1)}  \wedge \rs u_{p}^{12}$, 
  % we see that  $\dbar (d\sect_{(p)})^{*}  \rs u_{p}^{12} =0 $; hence $\dbar w =0$. 
  % In particular, $w$ determines the cohomology class $\{w \} \in H^{0}(D_{1}\cap D_{2}, K_{D_1 \cap D_2} \otimes F)$. 
\end{step}


\begin{step}[$w_b = 0$ and conclusion of the proof]
  \label{step:pf:use_u-ortho-w-simple}
We prove that $w_b =0$ using the assumption $(u_{1},u_{2}) \in
\paren{\ker \tau}^\perp$.
Consider the connecting morphism $\delta$ the long exact sequence 
\begin{equation*}
  \xymatrix@R=0.3cm@C=1.5em{
    {\to \cohgp 0[D_{1}\cap D_{2}]{K_{D_1 \cap D_2} \otimes F}} \ar[r]^-{\delta}
    &
    {\bigoplus_{p=1}^{2} \cohgp 1[D_{p}]{K_{D_p}\otimes F}} \ar[r]^-{\tau}  
    &
    {\cohgp 1[D]{K_D \otimes F}  \to} \; . 
  } 
\end{equation*}
Note that $\delta w_b \in \ker\tau$.

We compute $\delta w_b$ via the \v Cech--Dolbeault isomorphism.
% $\rho(w)$ in terms of \v Cech cohomology. 
Regard $w_b$ as a $0$-cocycle $\set{\rs \gamma_{b;\:i}}_{i \in I}$
given by $\rs \gamma_{b;\:i} :=\res{w_b}_{V_i}$.
Lift $\rs \gamma_{b;\:i}$ on $D_1 \cap D_2 \cap V_i$ to a section
$\gamma_i$ on $V_{i}$ via the isomorphism
$\frac{\holo_X}{\defidlof{D_1 \cap D_2}} = \faidlof|2|/|1|*
\xrightarrow[\isom]{\Res^2} \residlof|2|*$ such that
\begin{equation*}
  % \gamma_i = d\sect_{(2)}  \wedge d\sect_{(1)}  \wedge \rs \gamma_i \; .
  \Res^2\paren{\gamma_i}
  =\PRes[\lcS|2|[b]](\frac{\gamma_i}{\sect_D})
  =\frac{\rs*\gamma_{b;\:i}}{\sect_{(b)}} =\rs*\gamma_{b;\:i} \; .
\end{equation*}
Then $\delta w_b$ is represented by the $1$-cocycle
\begin{equation*}
  \delta\set{\gamma_i \bmod \defidlof{D_1 \cap D_2}}_{i \in I}
  =\set{(\delta  \gamma)_{ij} \bmod\defidlof{D}}_{i,j \in I}
  =\set{\gamma_{j} -\gamma_i  \bmod\defidlof{D}}_{i,j \in I} \; .
\end{equation*}
Note that $ \gamma_{j} -\gamma_i$ belongs to $\defidlof{D_1 \cap
  D_2}$, so $ \gamma_{j} -\gamma_i  \bmod\defidlof{D}$ can be realized
via the isomorphism $\frac{\defidlof{D_1 \cap D_2}}{\defidlof{D}}
=\faidlof|1|/|0|* \xrightarrow[\isom]{\Res^1} \residlof|1|*$ as
\begin{align*}
  \Res^1\paren{\gamma_{j} -\gamma_i}
  &=\paren{\PRes[D_1](\frac{\gamma_{j} -\gamma_i}{\sect_D})
    \: ,\:
    \PRes[D_2](\frac{\gamma_{j} -\gamma_i}{\sect_D})
    } \\
  &=\paren{
    \frac{(\delta \rs\gamma_1)_{ij}}{\sect_{(1)}}
    \: , \:
    \frac{(\delta \rs\gamma_2)_{ij}}{\sect_{(2)}}
    }
    \in K_{D_1} \otimes \res F_{D_1} \oplus K_{D_2} \otimes \res F_{D_2} \; ,
\end{align*}
in which $\rs*\gamma_{p;\:i} := \PRes[D_p](\frac{\gamma_i}{\sect_D})
\cdot \sect_{(p)}$ for $p = 1,2$.
Therefore, via the \v Cech--Dolbeault isomorphism on each $D_p$, 
the component of $\delta w_b$ on $D_p$ can be represented by (under
Einstein summation convention) 
\begin{equation*}
  -\dbar\rho^j \cdot \rho^i
  \frac{\paren{\delta\rs*\gamma_p}_{ij}}{\sect_{(p)}}
  =-\frac{\dbar\rho^j \cdot\rs*\gamma_{p;\:j}}{\sect_{(p)}}
  =: -\frac{\dbar v'_{p;(\infty)}}{\sect_{(p)}}
  % \paren{
  %   \res{\frac{\dbar\rho^i \:(\delta \gamma^1)_{ij}}{\sect_{(1)} }}_{D_1}
  %   \: , \:
  %   \res{\frac{\dbar\rho^i \: (\delta \gamma^2)_{ij}}{\sect_{(2)} }}_{D_2}
  % }
  % =\paren{
  %   -\res{\frac{\dbar\paren{\rho^i  \gamma^1_{i}}}{\sect_{(1)}}}_{D_1}
  %   \: , \:
  %   -\res{\frac{\dbar\paren{\rho^i  \gamma^2_{i}}}{\sect_{(2)}}}_{D_2}
  % } \; .
\end{equation*}
(the notation $v'_{p;(\infty)} :=\sum_{i\in I}\rho^i
\rs*\gamma_{p;\:i}$ is set for the consistency with the notation in
Proposition \ref{prop:res-formula-dbar-exact-dot-harmonic}).
% For the computation of the norm, 
% we take $\rs \gamma_i^p$ and $\rs\gamma_i^{12}$ such that 
% \begin{equation*}
% \gamma_{i} = d \sect_{(p)} \wedge \rs \gamma_i^p 
%   \quad\text{and}\quad 
%   \rs\gamma_i^{12} = \gamma_i. 
% \end{equation*}
Recall the sign convention chosen in Step
\ref{item:expression-of-su-simple} such that
\begin{equation*}
  \rs*\gamma_{b;\:i}
  = \PRes[\lcS|2|[b] | D_1](\frac{\rs*\gamma_{1;\:i}}{\sect_{(1)}})
  =- \PRes[\lcS|2|[b] |
  D_2](\frac{\rs*\gamma_{2;\:i}}{\sect_{(2)}}) \; .
\end{equation*}
Then, from $(u_{1},u_{2}) \in \paren{\ker\tau}^\perp$ and $\delta w_b
\in \ker\tau$, we obtain
\begin{align*}
  0
  &=
  \iinner{
    -\frac{\dbar v'_{1;(\infty)}}{\sect_{(1)}}
  }{u_1}_{D_1}
  +\iinner{
    -\frac{\dbar v'_{2;(\infty)}}{\sect_{(2)}}
  }{u_2}_{D_2}
  \\
  &=
    \iinner{
    -\dbar v'_{1;(\infty)}
    }{u_1 \sect_{(1)}}_{D_1, \phi_{(1)}}
    +\iinner{
    -\dbar v'_{2;(\infty)}
    }{u_2 \sect_{(2)}}_{D_2, \phi_{(2)}}
  \\
  &\overset{\mathclap{\text{Prop.~\ref{prop:res-formula-dbar-exact-dot-harmonic}}}}=
    \quad\;\;
    \iinner{\rho^i \rs*\gamma_{b;\:i} \:}{\:
    \PRes[\lcS|2|[b] | D_1](\idxup{\diff\psi_{(1)}}. u_1)
    -\PRes[\lcS|2|[b] | D_2](\idxup{\diff\psi_{(2)}}. u_2)
    }_{D_1\cap D_2}
  \\
  &=\iinner{w_b}{w_b}_{D_1 \cap D_2}
    =\norm{w_b}_{D_1 \cap D_2}^2
    \; .
\end{align*}
% By the same computation as in Step 2, 
% the right hand side can be described by the norm of $w$ as follows: 
% \begin{equation*}
%   0=\iinner{\rho^i  \gamma_i^{12} \:}{\:
%     (d\sect_{(1)})^{*}    u^{12} -(d\sect_{(2)})^{*}   v^{12}
%   }_{D_1 \cap D_2}
%   =\iinner{\rho^i \gamma_i}{w}_{D_1 \cap D_2}
%   = \norm w_{D_1 \cap D_2}^2. 
% \end{equation*}
This implies that $w_b=0$, finishing the proof for the case
$D=D_{1}+D_{2}$ and $q=1$. \qedhere
\end{step}
\end{proof}



\subsection{Remarks on the general case}
% \subsection{Strategy of the proof in the general case}
\label{subsec:n3}

There are two modifications to the proof in Section
\ref{sec:proof-of-simple-case} in order to handle the general case
worth mentioning here.
The first one is the replacement of the short exact sequence $0 \to
K_{D_1} \oplus K_{D_2} \to K_D \to K_{D_1 \cap D_2} \to 0$.
Take the case $D = D_1 + D_2 + D_3$, where $D_p = \set{z_p = 0}$ for
$p=1,2,3$ are the coordinate planes, for example.
Note that
\begin{equation*}
  \aidlof|3|* = \holo_X \;, \;\;
  \aidlof|2|* = \defidlof{D_1 \cap D_2 \cap D_3} \;, \;\;
  \aidlof|1|* = \smashoperator{\bigcap_{\substack{1 \leq p,q \leq 3 \\ p\neq q}}} \defidlof{D_p \cap D_q} \;
  \text{ and } \;
  \aidlof|0|* = \defidlof{D} 
\end{equation*}
in this case.
A natural choice of the short exact sequence to be considered is
\begin{equation*}
  \renewcommand{\objectstyle}{\displaystyle}
  \xymatrix@R=2.5em{
    0 \ar[r]
    &{K_X \otimes D \otimes \smash{\faidlof|1|/|0|*}} \ar[r]
    \ar[d]^(0.47){\Res^1}_(0.47){\isom}
    &{K_X \otimes D \otimes \smash{\faidlof|3|/|0|*}} \ar[r]
    \ar@{=}[d]
    &{K_X \otimes D \otimes \faidlof|3|/|1|*} \ar[r]
    &0 \; .
    \\
    &{\smash{\bigoplus_{p = 1}^3}\:K_{D_p}} \ar[r]
    &{K_D} 
    &
  }
\end{equation*}
In the previous case, we are taking advantage of the fact that the
$L^2$ Dolbeault isomorphism and the harmonic theory are valid on the
cohomology groups of the sheaves on both the left- and
right-hand-sides of the short exact sequence, so that the
corresponding injectivity statement can be proved on each side in
the spirit of Enoki, which in turn leads to the injectivity theorem
for the cohomology groups of the middle sheaf (twisted by $F$).
In the current case, they are valid only on the left-hand-side (on
each $D_p$).
We are thus led to determine whether the injectivity statement for
the sheaf on the right-hand-side holds true.
It is then apparent that we should consider
\begin{equation*}
  \renewcommand{\objectstyle}{\displaystyle}
  \xymatrix@R=2.5em{
    0 \ar[r]
    &{K_X \otimes D \otimes \smash{\faidlof|2|/|1|*}} \ar[r]
    \ar[d]_(0.47){\Res^2}^(0.47){\isom}
    &{K_X \otimes D \otimes \faidlof|3|/|1|*} \ar[r]
    &{K_X \otimes D \otimes \smash{\faidlof|3|/|2|*}} \ar[r]
    \ar[d]^-{\Res^3}_-{\isom}
    &0 \; ,
    \\
    &{\smash[t]{\bigoplus_{\substack{p,q = 1 \\ p\neq q}}^3} K_{D_p \cap D_q}} 
    &
    &{K_{D_1 \cap D_2 \cap D_3}}
  }
\end{equation*}
which, again, has the Dolbeault and harmonic theories valid on both
sides (on each lc center of $(X,D)$) of the short exact sequence.
The arguments in Section \ref{sec:proof-of-simple-case} can then be
employed to conclude the proof.
This illustrates the idea of the inductive arguments, which reduces
the question to the union of lower dimensional lc centers of $(X,D)$
in each step, to be employed in the general proof in Section
\ref{subsec:general}. 

Another modification to the proof in Section
\ref{sec:proof-of-simple-case} is that, when the claim in Theorem
\ref{thm:main} with $q > 1$ is considered, the section $w_b$
constructed as in \eqref{eq:w-prelim-formula} is then a $K_{\lcS+1[b]}
\otimes \res F_{\lcS+1[b]}$-valued $(0,q-1)$-form on some
$(\sigma+1)$-lc center $\lcS+1[b]$.
In order to prove that $w_b =0$ by following the arguments in the
previous case, we need not only to show that $w_b$ is
$\dbar$-closed, but also that it is harmonic.
This happens to be true and the computation for checking this claim
is given in Proposition \ref{prop:harmonic-residue} and Theorem
\ref{thm:residue-harmonic}.



% In this subsection, we consider how we should generalize the proof of the previous section in dealing with the general case. 

% We first consider the slightly more general case of $D=D_{1}+D_{2}$ and $q \geq 2$. 
% In this case,  we can repeat the same argument  for Step 1. 
% Step 2 is a bit more involved since we are dealing with differential forms $u_{p}$ of higher degree, 
% but essentially the same argument can be used to define $w$ appropriately (see $\eqref{eq-def-w}$). 
% To check that $w$ is harmnic in Step 3, 
% since $w$ is an $F$-valued of the type $(n-2, q-1) \not =(n-2, q-1)$, 
% we need to check $\dfadj w_q = 0$ as well as $\dbar w=0$. 
% Nevertheless, $\dbar w=0$ is proved by the same argument 
% and $\dfadj w_q = 0$ is proved in Subsection \ref{subsec:harmonic}. 
% Performing Step 4 in the same way, 
% some global section $v$ on $D_{1}\cap D_{2}$ naturally appears, 
% we finally obtain $0=\iinner{w-\dbar v }{w}_{D_{1}\cap D_{2}} $. 
% Although the point that $v$ appears is different, since $w$ is harmonic, the conclusion that $w=0$ is immediately obtained.
% As described above, in the case of $q \geq 2$, 
% the degree of the differential form is higher and more involved, 
% but essentially the same strategy still work. 


% Next, let us consider the case of $D=D_{1}+D_{2}+D_{3}$. 
% In the case of $D=D_{1}+D_{2}$ , 
% by using the exact sequence $0 \to K_{D_{1}} \oplus K_{D_{1}} \to K_{D} \to K_{D_{1} \cap D_{2} } \to 0$, 
% we proved the injectivity of the multiplication map on (the cohomology groups of) central term. 
% Of particular importance in the proof were that the left term admits the theory of harmonic integrals and 
% that the multiplication map on the right term in injective. 
% In this subsection, we explain what kind of exact sequences in the case where $D$ has three components  
% to make the same strategy works. 
% The precise proof will be given  in the next subsection. 

% Suppose that $D$ has three components (i.e.\,$D=D_{1}+D_{2}+D_{3}$). 
% We first consider the following exact sequence twisted by $F$ (and also $M$): 
% \begin{align}\label{eq-ex}
%   \xymatrix{
%     0 \ar[r]
%     & K_{X}\otimes D \otimes{\faidlof |1|/|0|*} =\bigoplus_{p=1}^{3} K_{D_{p}} \ar[r]
%     & K_{X}\otimes D \otimes{\faidlof|\sigma_{\mlc}|/|0|*} =K_{D}        \ar[r]
%     & K_{X}\otimes D \otimes{\faidlof|\sigma_{\mlc}|/|1|*} \ar[r]
%     & 0. 
%   } 
% \end{align}
% The cohomology classes in $\oplus_{p=1}^{3} H^{q}(D_{p}, K_{D_{p}} \otimes F)$ of the left term 
% can be represented by harmonic forms. 
% Hence, whether the same argument works as in the previous subsection 
% depends on whether or not the multiplication map 
% $$
% H^{q}(X, K_{X}\otimes{\faidlof|\sigma_{\mlc}|/|1|*}\otimes F) \xrightarrow{\quad \otimes s \quad }
% H^{q}(X, K_{X}\otimes{\faidlof|\sigma_{\mlc}|/|1|*}\otimes F\otimes M)
% $$
% is injectivity or not.  
% To check this, we  consider another exact sequence: 
% \begin{align*}
%    \xymatrix{
%     0 \ar[r]
%     & K_{X}\otimes{\faidlof |2|/|1|*}  =\bigoplus_{p\not = q} K_{D_{p} \cap D_{q}} \ar[r]
%     & K_{X}\otimes{\faidlof|\sigma_{\mlc}|/|1|*}         \ar[r]
%     & K_{X}\otimes{\faidlof|\sigma_{\mlc}|/|2|*}=K_{D_{1}\cap D_{2}\cap D_{3}} \ar[r]
%     & 0}
% \end{align*}
% Then, the left term admits the theory of harmonic integrals and 
% that the multiplication map on the right term in injective.
% Therefore, we can show that he multiplication map on the central term is injective. 
% No essential difficulty appears in repeating this inductive argument in the general case. 


\subsection{Proof of Theorem \ref{thm:main} in general}\label{subsec:general}


%\input{outline-of-proof}

%%%%%
%%%%% File name  : outline-of-proof.tex
%%%%% Author     : Mario Chan
%%%%% Date       : 6th March, 2023
%%%%% Description: This is the outline of the proof of the general
%%%%%              case of the project "Injectivity-Fujino".
%%%%%
%%
%%%

\renewcommand{\objectstyle}{\displaystyle}

Write
\begin{gather*}
  \aidlof* := \aidlof =\mtidlof{\vphi_F} \cdot \defidlof{\lcc+1'}
  =\defidlof{\lcc+1'} \; , \quad
  \residlof* := \residlof \isom \faidlof/-1*  \\
  \text{and } \quad \spH{\sheaf F}
  :=\cohgp q[X]{\logKX \otimes \sheaf F} 
\end{gather*}
for convenience.
Recall that
\begin{equation*}
  K_D = K_X \otimes D \otimes \faidlof|\sigma_{\mlc}|/|0|* \; ,
\end{equation*}
and the inclusions between adjoint ideal sheaves induce the short exact
sequences
\begin{equation*} % \label{eq-ex2}
  \xymatrix{
    0 \ar[r]
    & {\faidlof/|\rho|*} \ar[r]
    & {\faidlof|\tau|/|\rho|*} \ar[r]
    & {\faidlof|\tau|/*} \ar[r]
    & 0
  } \quad\text{ for } 0 \leq \rho \leq \sigma \leq \tau \; .
\end{equation*}
One is thus led to consider the commutative diagram
\subfile{commut-diagram_sing-Fujino-conj}%
for $\sigma =2,\dots,\sigma_{\mlc}$, in which the columns are exact,
$\iota_\sigma$ and $\tau_\sigma$ are induced from the inclusions
between adjoint ideal sheaves, and $\mu_\sigma$ (resp.~$\nu_\sigma$)
is the composition of $\iota_\sigma$ (resp.~$\tau_\sigma$) with the
map induced from the multiplication map $\otimes s$.
The statement in Theorem \ref{thm:main} is proved if one shows that
$\ker\mu_{\sigma_{\mlc}} = \ker\iota_{\sigma_{\mlc}} = 0$
($\iota_{\sigma_{\mlc}}$ is the identity map).
Note that $\mu_1 =\nu_1$ and $\iota_1 =\tau_1$.
Following the argument in \cite{Chan&Choi_injectivity-I}*{Thm.~1.3.2}, since
$\ker\mu_{\sigma-1} =\ker\iota_{\sigma-1}$ and $\ker\nu_\sigma
=\ker\tau_\sigma$ together imply $\ker\mu_{\sigma}
=\ker\iota_{\sigma}$ via a diagram-chasing argument, to prove Theorem
\ref{thm:main}, it suffices to show the following theorem.

\begin{thm} \label{thm:ker-nu=ker-tau}
  $\ker\nu_\sigma =\ker\tau_\sigma$ for all $\sigma =1, \dots, \sigma_{\mlc}$.
\end{thm}


\begin{remark} \label{rem:general-commut-diagram}
  When $\aidlof|0|*$ in the commutative diagram
  \eqref{eq:commut-diagram_sing-Fujino-conj} is replaced by $0$ (which
  can be considered as $\aidlof|-1|*$), the setup is reduced to the one in
  \cite{Chan&Choi_injectivity-I}*{Thm.~1.3.2}, which states that
  Theorem \ref{thm:ker-nu=ker-tau} together with the result in
  \cite{Matsumura_injectivity-lc}*{Thm.~1.6} or
  \cite{Chan&Choi_injectivity-I}*{Thm.~1.2.1} implies that Fujino's
  conjecture is true.
  As a matter of fact, the proof of Theorem \ref{thm:ker-nu=ker-tau}
  can also be adapted to the case $\sigma = 0$ (with $\aidlof|-1|* =
  0$ and $\residlof|0|* =D^{-1} \isom \defidlof{D} =\aidlof|0|*$) which recovers the result in
  \cite{Matsumura_injectivity-lc}*{Thm.~1.6} as well as
  \cite{Chan&Choi_injectivity-I}*{Thm.~1.2.1}.
  Furthermore, by replacing $\aidlof|0|*$ by $\aidlof|\sigma_0-1|*$
  and $\aidlof|\sigma_{\mlc}|*$ by $\aidlof|\sigma'|*$ for any $0 <
  \sigma_0 \leq \sigma'$ and letting $\sigma$ vary within the range
  $\sigma_0 < \sigma \leq \sigma'$ in the diagram
  \eqref{eq:commut-diagram_sing-Fujino-conj}, one sees that the proof
  of Theorem \ref{thm:ker-nu=ker-tau} guarantees the statement of
  Theorem \ref{thm:main} but with $K_D$ replaced by $K_X \otimes D
  \otimes \faidlof|\sigma'|/|\sigma_0 -1|*$.
\end{remark}


\begin{proof}
  The proof consists of the following steps.
  \begin{enumerate}[label=\textbf{Step \Roman*:}, ref=\Roman*,
    leftmargin=0pt, labelsep=*, widest=VI, itemindent=*, align=left,
    itemsep=1.5ex]
  \item Make use of the $L^2$ Dolbeault and harmonic theory available on
    $\spH{\residlof*}$.

    Write $\lcc' =\bigcup_{p \in \Iset} \lcS$ as the union of
    $\sigma$-lc centers $\lcS$ of $(X,D)$.  Notice that $\residlof*$,
    hence $\spH{\residlof*}$, has a decomposition as a direct sum
    which yields
    \begin{equation*}
      \spH{\residlof*}
      =\bigoplus_{p \in \Iset} \cohgp q[\lcS]{K_{\lcS}
        \otimes F \otimes \mtidlof<\lcS>{\vphi_F}}
      =\bigoplus_{p \in \Iset} \cohgp q[\lcS]{K_{\lcS} \otimes F}
    \end{equation*}
    such that the $L^2$ Dolbeault isomorphism and harmonic theory are
    valid for the cohomology group in each summand.
    Take the (squared) residue norm
    $\norm\cdot_{\lcc'}^2 = \sum_{p \in \Iset} \norm\cdot_{\lcS}^2$ as
    the $L^2$ norm on $\spH{\residlof*}$.  Pick any element
    $u := (u_p)_{p \in \Iset} \in \spH{\residlof*}$ such that
    \begin{itemize}
    \item each $u_p$ is a harmonic form on $\lcS$ with respect to the
      given norm $\norm\cdot_{\lcS}$ and
    
    \item $u \in \ker\nu_\sigma \cap \paren{\ker\tau_\sigma}^\perp$,
      where the orthogonal complement $\paren{\ker\tau_\sigma}^\perp$
      of $\ker\tau_\sigma$ is taken with respect to the residue norm
      $\norm\cdot_{\lcc'}$.
    \end{itemize}
    The theorem is proved if it is shown that $u_p = 0$ for all
    $p \in \Iset$.

  
  \item \label{item:express-su-in-residue-norm}
    Obtain an expression of $\norm{su}_{\lcc'}^2$ using the
    assumption $u \in \ker\nu_\sigma$ and the \v Cech--Dolbeault
    isomorphism.

    Let $\cvr V :=\set{V_i}_{i \in I}$ be a locally finite cover of
    $X$ by admissible open sets with respect to
    $(\vphi_F,\vphi_M,\psi_D)$ and let $\set{\rho^i}_{i \in I}$ be a
    partition of unity subordinate to $\cvr V$.
    Their notations are abused to mean also their induced cover and
    partition of unity on $\lcc'$ for any $\sigma \geq 0$.
    For any choice of indices $\idx 0,q \in I$, write $V_{\idx 0.q}
    :=V_{i_0} \cap V_{i_1} \dotsm \cap V_{i_q}$ as usual.
  
    Through the \v Cech--Dolbeault isomorphism, every (cohomology
    class of) $u_p$ is represented by a \v Cech $q$-cocycle
    $\set{\alpha_{p; \:\idx 0.q}}_{\idx 0,q \in I}$ such that (under
    the Einstein summation convention on the indices $\idx 0,q$)
    \begin{equation*}
      u_p
      % &= \dbar v_{p;(2)} +\dbar \rho^{i_{q-1}} \wedge \dotsm \wedge
      %   \dbar\rho^{i_0} \alpha_{p; \:\idx 0.q} \qquad\paren{\forall~ i_q \in I} \\
      \overset{\text{\eqref{eq:Cech-Dolbeault-isom}}}=
      \:\dbar v_{p;(2)}
      +(-1)^q \:\underbrace{\dbar \rho^{i_{q}} \wedge \dotsm \wedge
        \dbar\rho^{i_1} \cdot \rho^{i_0} }_{=: \:
        \paren{\dbar\rho}^{\idx q.0}} \alpha_{p; \:\idx 0.q} \; ,
    \end{equation*}
    where $v_{p; (2)}$ is a $K_{\lcS} \otimes \res{F}_{\lcS}$-valued $(0,q-1)$-form
    on $\lcS$ with $L^2$ coefficients with respect to
    $\norm\cdot_{\lcS}$ and
    $\alpha_{p; \:\idx 0.q} \in K_{\lcS} \otimes \res F_{\lcS} \otimes
    \mtidlof<\lcS>{\vphi_F} =K_{\lcS} \otimes \res F_{\lcS}$ on
    $V_{\idx 0.q}$.
    % (see \cite{Matsumura_injectivity}*{Prop.~5.5} or
    % \cite{Chan&Choi_injectivity-I}*{Lemma 3.2.1}). 
    In view of the
    residue short exact sequence, choose, for each choice of
    the multi-indices $(\idx 0,q)$, a section
    $f_{\idx 0,q} \in \logKX M \otimes \aidlof*$ on $V_{\idx 0.q}$
    such that
    \begin{equation*}
      \Res^\sigma(f_{\idx 0.q})
      =\paren{\alert{s} \alpha_{p; \:\idx 0.q}}_{p \in \Iset} 
    \end{equation*}
    (note that $V_{\idx 0.q}$ is Stein).  Considering the inclusion
    $\aidlof* \subset \aidlof|\sigma_{\mlc}|*$, write
    \begin{equation*}
      \eqcls{f_{\idx 0.q}} := \paren{f_{\idx 0.q} \bmod \aidlof-1*}
      \;\in \logKX M \otimes \faidlof|\sigma_{\mlc}|/-1*
      \quad\text{ on } V_{\idx 0.q} \; .
    \end{equation*}
    The collection $\set{\eqcls{f_{\idx 0.q}}}_{\idx 0,q \in I}$ is
    then a \v Cech $q$-cocycle representing $\nu_\sigma(u)$ in $\spH
    M{\faidlof|\sigma_{\mlc}|/-1*}$.
    The assumption $u \in \ker\nu_\sigma$ implies that this cocycle is
    a coboundary, that is,
    \begin{equation*}
      \set{\eqcls{f_{\idx 0.q}}}_{\idx 0,q \in I}
      =\delta\set{\eqcls{\lambda_{\idx 1.q}}}_{\idx 1,q \in I}
      =\set{\eqcls{\paren{\delta\lambda}_{\idx 0.q} }}_{\idx 0,q \in I}
    \end{equation*}
    for some $\lambda_{\idx 1.q} \in \logKX M \otimes
    \aidlof|\sigma_{\mlc}|*$ on $V_{\idx 1.q}$ (note that
    $\lambda_{\idx 1.q}$ need \emph{not} take values in $\aidlof*$
    even though the sections $f_{\idx 0.q}$ do), where
    $\paren{\delta\lambda}_{\idx 0.q}$ is given by the usual formula
    of \v Cech coboundary operator $\paren{\delta\lambda}_{\idx 0.q}
    :=\sum_{k =0}^q (-1)^k \lambda_{\idx 0[\dotsm \widehat{i_k}].q}$.
    Notice that $f_{\idx 0.q}$ and $\paren{\delta\lambda}_{\idx 0.q}$
    differ by an element in $\logKX M \otimes \aidlof-1*$ on $V_{\idx
      0.q}$.
    

    Thanks to the positivity $\ibddbar\vphi_F \geq 0$ and the bound
    $\ibddbar\vphi_M \leq C \ibddbar\vphi_F$ for some
    constant $C > 0$ on each $\lcS$, the product $s u_p$ is harmonic
    with respect to $\norm\cdot_{\lcS}$ (Proposition
    \ref{prop:consequence-of-positivity}), so $\iinner{s u_p}{s
      \:\dbar  v_{p;(2)}}_{\lcS} = \iinner{s u_p}{\dbar \paren{s
        v_{p;(2)}}}_{\lcS} = 0$ for every $p \in\Iset$.
    It follows that
    \begin{align*}
      \norm{su}_{\lcc'}^2
      =\sum_{p\in \Iset} \norm{su_p}_{\lcS}^2
      =&~(-1)^q \sum_{p\in \Iset} \sum_{\idx 0,q \in I} \iinner{\paren{\dbar\rho}^{\idx q.0}
        \:s \alpha_{p; \:\idx 0.q}}{\: s u_p}_{\lcS}
      \\
      =&~(-1)^q \sum_{p\in \Iset} \sum_{\idx 0,q \in I} \iinner{
         s \alpha_{p; \:\idx 0.q}
         }{\:\idxup{\diff\rho},[\idx 0.q].  s u_p}_{\lcS}
         \; ,
    \end{align*}
    where $\idxup{\diff\rho},[\idx 0.q].  \cdot $ is the adjoint
    of $\paren{\dbar\rho}^{\idx q.0} \cdot$.
    As in Step \ref{item:expression-of-su-simple} in Section
    \ref{sec:proof-of-simple-case}, the desired expression can be
    obtained by substituting
    \begin{equation*}
      s\alpha_{p; \:\idx 0.q} =\PRes[\lcS](\frac{f_{\idx
          0.q}}{\sect_D})
      =\PRes[\lcS](\frac{\paren{\delta\lambda}_{\idx
          0.q}}{\sect_D})
      =\frac{\paren{\delta \rs*\lambda_p}_{\idx 0.q}}{\sect_{(p)}}
      \; ,
    \end{equation*}
    where $\rs*\lambda_{p; \:\idx 1.q}
    :=\PRes[\lcS](\frac{\lambda_{\idx 1.q}}{\sect_D}) \cdot
    \sect_{(p)}$.
    For the sake of illustration, an alternative approach via a
    direct residue computation is presented here.
    Note that $\paren{\idxup{\diff\rho},[\idx 0.q].  s u_p}_{p \in
      \Iset} \in \logKX M \otimes \smooth_{X\:c\,*} \cdot\residlof*$ on $V_{\idx
      0.q}$, so it has a preimage $h^{\idx 0.q} \in \logKX M \otimes
    \smooth_{X\:c\,*} \cdot \aidlof*$ of $\Res^\sigma$ (considered as a
    $\smooth_{X\:c\,*}$-homomorphism).
    Fix such preimage on each open set $V_{\idx 0.q}$.
    From the direct computation of the residue function, it follows
    that
    \begin{align*}
      (-1)^q \:\norm{su}_{\lcc'}^2
      =&~\sum_{p\in \Iset} \sum_{\idx 0,q \in I} \iinner{
         s \alpha_{p; \:\idx 0.q}
         }{\:\idxup{\diff\rho},[\idx 0.q].  s u_p}_{\lcS}
      \\
      \xleftarrow{\eps \tendsto 0^+}
       &~\smashoperator[l]{\sum_{\idx 0,q \in I}} \eps
         \int_{\mathrlap{V_{\idx 0.q}}} \quad \frac{
         \inner{f_{\idx 0.q}}{h^{\idx 0.q}}
         \:e^{-\phi_D -\vphi_F -\vphi_M}
         }{\abs{\psi_D}^{\sigma +\eps}}
      \\
      \overset{\mathclap{\text{Prop.~\ref{prop:residue-product-X-to-lcS}}}}= \quad\;
       &~\smashoperator[l]{\sum_{\idx 0,q \in I}} \eps
         \int_{\mathrlap{V_{\idx 0.q}}} \quad \frac{
         \inner{\paren{\delta\lambda}_{\idx 0.q}}{h^{\idx 0.q}}
         \:e^{-\phi_D -\vphi_F -\vphi_M}
         }{\abs{\psi_D}^{\sigma +\eps}} +\BigO(\eps)
      \\
      \xrightarrow[\text{Prop.~\ref{prop:residue-product-X-to-lcS}}]{\eps \tendsto 0^+}
      %  &~\sum_{p\in \Iset} \sum_{\idx 0,q \in I}
      %    \int_{\lcS} \inner{
      %    \frac{\paren{\delta\rs*\lambda_p}_{\idx 0.q}}{ \sect_{(p)}}
      %    }{\:
      %    \idxup{\diff\rho},[\idx 0.q].  s u_p
      %    } \:e^{-\vphi_F-\vphi_M}
      % \\
      % =&~\sum_{p\in \Iset} \sum_{\idx 0,q \in I}
      %    \int_{\lcS} \inner{
      %    \paren{\dbar\rho}^{\idx q.0}
      %    \paren{\delta\rs*\lambda_p}_{\idx 0.q}
      %    }{\:
      %     s u_p \sect_{(p)}
      %    }_\omega \:e^{-\phi_{(p)}-\vphi_F-\vphi_M}
      % \\
      % =
       &~\sum_{p\in \Iset} \sum_{\idx 0,q \in I}
         \iinner{\paren{\dbar\rho}^{\idx q.0}
         \paren{\delta\rs*\lambda_p}_{\idx 0.q}}{\: s u_p
         \sect_{(p)}}_{\lcS, \phi_{(p)}}
      \\
      =&~\sum_{p\in \Iset} \sum_{\idx 1,q \in I}
         \iinner{\dbar\rho^{i_q} \wedge \dotsm \wedge \dbar\rho^{i_1}
         \cdot\rs*\lambda_{p;\:\idx 1.q}}{\: s u_p \sect_{(p)}}_{\lcS,
         \phi_{(p)}}
      \\
      =&~(-1)^{q-1} \sum_{p\in \Iset} \iinner{\dbar v_{p;(\infty)}}{\: s u_p
         \sect_{(p)}}_{\lcS, \phi_{(p)}}
         \; ,\footnotemark
    \end{align*}%
    \footnotetext{
      If the $L^2$ Dolbeault isomorphism is valid for $\spH
      M{\faidlof|\sigma_{\mlc}|/-1*}$, such conclusion can be
      obtained simply from the fact that $\nu_\sigma(su)$ is
      represented by a smooth $\dbar$-exact form on $\lcc'$.
    }%
    where
    % $\rs*\lambda_{p; \:\idx 1.q}
    % :=\PRes[\lcS](\frac{\lambda_{\idx 1.q}}{\sect_D}) \cdot \sect_{(p)}$ and 
    $v_{p; (\infty)} :=\sum_{\idx 1,q\in I} \dbar\rho^{i_q} \wedge \dotsm
    \wedge \dbar\rho^{i_2} \cdot \rho^{i_1} \rs*\lambda_{p;\:\idx 1.q}
    =\sum_{\idx 1,q\in I} \paren{\dbar\rho}^{\idx q.1}\rs*\lambda_{p; \:\idx 1.q}$. 

    % Note that $s u_p$ is harmonic with respect to the potential
    % $\vphi_F+\vphi_M$ by the positivity assumption.
    The expression of $\norm{su}_{\lcc'}^2$ can be further transformed
    by an integration by parts using Proposition
    \ref{prop:res-formula-dbar-exact-dot-harmonic}, which becomes
    \begin{equation*}
      \norm{su}_{\lcc'}^2
      % &=\smashoperator[l]{\sum_{\idx 1,q \in I}} \sum_{b \in \Iset+1}
      % \sum_{j=1}^{\sigma +1} (-1)^q \:\sigma
      % \iinner{ \sgn{b:p_{b,j}}\:
      % \frac{\rs*\lambda_{b;\:\idx 1.q}}{\sect_{(b)}}
      % }{\: s\:
      %   \idxup{\diff\rho},[\idx 1.q] .
      %   \PRes[b(j)](\idxup{\diff\psi_{(p_{b,j})}}.  u_{p_{b,j}})
      % }_{\lcS+1[b]}
      %   \\
      = \sigma\sum_{b \in \Iset+1}
      \iinner{v_{b;(\infty)} \:
      }{ \quad s \:
        \smashoperator{\sum_{p \in \Iset \colon \lcS+1[b] \subset
            \lcS}} \;\; \sgn{b:p}\:
        \PRes[\lcS+1[b] | \lcS](\idxup{\diff\psi_{(p)}}.  u_{p})
        \cdot \sect_{(b)}
      }_{\lcS+1[b], \phi_{(b)}} \; ,
    \end{equation*}
    where $v_{b;(\infty)} := \sum_{\idx 1,q \in I}
    \paren{\dbar\rho}^{\idx q.1} \rs*\lambda_{b; \:\idx 1.q}$ and
    $\rs*\lambda_{b; \:\idx 1.q}
    :=\PRes[\lcS+1[b]](\frac{\lambda_{\idx 1.q}}{\sect_D}) \cdot \sect_{(b)}$.
    % However, notice that $v_{p;(\infty)}$ is smooth on $\lcS$ but not
    % locally $L^2$ with respect to the weight $e^{-\phi_{(p)}}$.
    % For this reason, let $\psi_{(p)} :=\phi_{(p)} -\sm\vphi_{(p)}$, where
    % $\sm\vphi_{(p)}$ is some smooth potential on $\Diff_p D$.
    % One then has
    % \begin{align*}
    %   &~\norm{su}_{\lcc'}^2
    %   =\sum_{p\in \Iset} \iinner{\dbar v_{p;(\infty)}}{\: s u_p
    %      \sect_{(p)}}_{\lcS, \phi_{(p)}}
    %   \\
    %   \xleftarrow{\eps \tendsto 0^+}
    %    &~\sum_{p \in \Iset} \iinner{
    %      e^{-\eps \abs{\psi_{(p)}}} \:\dbar v_{p;(\infty)}
    %      }{\: s u_p \sect_{(p)}}_{\lcS, \phi_{(p)}}
    %   \\
    %   =&~\sum_{p \in \Iset} \paren{
    %      \cancelto{0}{\iinner{
    %      \dbar\paren{e^{-\eps \abs{\psi_{(p)}}} \: v_{p;(\infty)}}
    %      }{\: s u_p \sect_{(p)}}_{\mathrlap{\lcS, \phi_{(p)}}}}
    %      \quad\;\; + \eps 
    %      \iinner{
    %      e^{-\eps \abs{\psi_{(p)}}} \:v_{p;(\infty)}
    %      }{\:\idxup{\diff\psi_{(p)}} . s u_p \sect_{(p)}}_{\lcS,
    %      \phi_{(p)}}
    %      }
    %   \\
    %   =&~\sum_{p \in \Iset} \sum_{\idx 1,q \in I} (-1)^q \:\eps \:
    %      \iinner{
    %      e^{-\eps \abs{\psi_{(p)}}} \: % \paren{\dbar\rho}^{\idx q.1}
    %      \rs*\lambda_{p;\:\idx 1.q}
    %      }{\:
    %      \idxup{\diff\rho},[\idx 1.q] .
    %      \paren{\idxup{\diff\psi_{(p)}} . s u_p \sect_{(p)}}
    %      }_{\lcS, \phi_{(p)}}
    %   \\
    %   \xrightarrow[\text{Prop.~\ref{prop:residue-formula-classical-kernel}}]{\eps
    %   \tendsto 0^+} 
    %   &~\smashoperator[l]{\sum_{\idx 1,q \in I}} \sum_{p \in \Iset}
    %     \sum_{k=\sigma +1}^{\mathclap{\sigma_{V_{\idx 1.q}}}} (-1)^q \:\sigma
    %     \iinner{
    %     \PRes[p(k)](
    %     \frac{\rs*\lambda_{p;\:\idx 1.q}}{\sect_{(p)}}
    %     )
    %      }{\:
    %      \idxup{\diff\rho},[\idx 1.q] .
    %      \PRes[p(k)](\idxup{\diff\psi_{(p)}} . s u_p)
    %      }_{\lcS \cap \set{z_{p(k)} =0}}
    %   \; ,
    % \end{align*}
    % where $\PRes[p(k)]$ denotes the Poincar\UTF{00E9} residue map from $\lcS$
    % to $\lcS \cap \set{z_{p(k)}=0}$. 
    % The last limit is justified as follows.
    % On the admissible open set $V_{\idx 1,q}$, consider a holomorphic
    % coordinate system $(z_1, \dots, z_n)$ such that $\lcS
    % =\set{z_{p(1)} = \dotsm =z_{p(\sigma)} =0}$ and
    % $\sect_{(p)} =z_{p(\sigma+1)} \dotsm z_{p(\sigma_V)}$ (write
    % $\sigma_{V}$ for $\sigma_{V_{\idx 1.q}}$ for convenience).
    % Note that
    % \begin{equation*}
    %   \diff\psi_{(p)} =\sum_{k =\sigma +1}^{\sigma_V}
    %   \frac{dz_{p(k)}}{z_{p(k)}} -\diff\sm\vphi_{(p)} \quad\text{ on }
    %   V_{\idx 1,q} \; .
    % \end{equation*}
    % It follows that, on $\lcS \cap V_{\idx 1.q}$,
    % \begin{equation*}
    %   \begin{multlined}
    %     \text{coef.~of }\:
    %     \idxup{\diff\rho},[ \idx 1.q] .
    %     \paren{\idxup{\diff\psi_{(p)}} . s u_p \sect_{(p)}}
    %   \end{multlined}
    %   \in
    %   \res{\defidlof{\lcc+2'}}_{\lcS}
    %   \begin{aligned}[t]
    %     &=\mtidlof<\lcS>{\vphi_F+\vphi_M} \cdot
    %     \res{\defidlof{\lcc+2'}}_{\lcS} \;\;\footnotemark
    %     \\
    %     &=\aidlof|1|<\lcS>{\vphi_F+\vphi_M}[\psi_{(p)}]
    %   \end{aligned}
    % \end{equation*}%
    % \footnotetext{
    %   Recall that $\defidlof{\lcc+2'}$ is generated on $X$ by
    %   $\sect_{(\sigma+1 : b)}$ for all $b \in \Iset+1$ treated as local
    %   functions.
    %   On an admissible open set $V$, one has $\defidlof{\lcc+2'}
    %   =\genbyd{z_{b(\sigma+2)} \dotsm
    %     z_{b(\sigma_V)}}{b \in \Iset+1 \text{ such that } \lcS+1[b] \cap
    %     V \neq \emptyset}$ (see page
    %   \pageref{page:notation-permutation-index} for the notation).
    % }%
    % and, therefore, one can apply Proposition
    % \ref{prop:residue-formula-classical-kernel} (with $\lcS$ in place
    % of $X$, $\psi_{(p)}$ in place of $\psi_D$) to each inner product
    % $\eps \iinner{\dotsm}{ \:\dotsm \idxup{\diff\psi_{(p)}}. \dotsm
    %   \sect_{(p)}}_{\lcS,\phi_{(p)}}$.
    % % to obtain a sum of integrals on
    % % each $\lcS \cap \set{z_{p(k)} = 0}$ for $k=\sigma +1, \dots,
    % % \sigma_V$, i.e.~on the $(\sigma+1)$-lc centers in $\lcc+1' \cap
    % % V_{\idx 1.q}$.
    
    % Write $\lcc+1' =\bigcup_{b \in \Iset+1} \lcS+1[b]$.
    % On each admissible open set $V_{\idx 1.q}$, the intersection $\lcS
    % \cap \set{z_{p(k)} = 0}$ is a $(\sigma+1)$-lc center $\lcS+1[b_{p,k}]
    % \cap V_{\idx 1.q}$ ($\neq \emptyset$), uniquely determined by the
    % choices of $p\in \Iset$ (such that $\lcS \cap V_{\idx 1.q} \neq
    % \emptyset$, so $\binom{\sigma_V}{\sigma}$ choices) and $k
    % =\sigma+1, \dots, \sigma_V$ (so $\sigma_V-\sigma$ choices).
    % To get an indexing in terms of $b \in \Iset+1$ (such that
    % $\lcS+1[b] \cap V_{\idx 1.q} \neq \emptyset$, so
    % $\binom{\sigma_V}{\sigma +1}$ choices), note that each $\lcS+1[b]
    % \cap V_{\idx 1.q}$ is contained in $\sigma +1$ distinct
    % $\sigma$-lc centers $\lcS[p_{b,j}]$ for $j=1,\dots,\sigma+1$
    % (apparently, $\sigma +1$ choices) such that
    % \begin{equation*}
    %   \lcS+1[b] \cap V_{\idx 1.q} = \lcS[p_{b,j}] \cap \set{z_{b(j)} = 0} \; .
    % \end{equation*}
    % (One can verify $\sum_{p \in \Iset} \sum_{k=\sigma
    %   +1}^{\sigma_{V}} \dotsm = \sum_{b \in
    %   \Iset+1} \sum_{j=1}^{\sigma +1} \dotsm$ by first noting that
    % $\binom{\sigma_V}{\sigma} (\sigma_V -\sigma)
    % =\binom{\sigma_V}{\sigma +1} (\sigma+1)$.)
    % With such choice of indexing, let $\sgn{b:p_{b,j}}$ be the sign
    % given by
    % \begin{equation*}
    %   \PRes[\lcS+1[b]]
    %   =\sgn{b:p_{b,j}} \:\PRes[b(j)]\circ \PRes[\lcS[p_{b,j}]] \; .
    % \end{equation*}
    % Therefore, one has
    % \begin{equation*}
    %   \frac{\rs*\lambda_{b;\: \idx 1.q}}{\sect_{(b)}}
    %   :=\PRes[\lcS+1[b]](\frac{\lambda_{\idx 1.q}}{\sect_D})
    %   % =\sgn{b:p_{b,j}} \:\PRes[b(j)]\circ
    %   % \PRes[\lcS[p_{b,j}]](\frac{\lambda_{\idx 1.q}}{\sect_D})
    %   =\sgn{b:p_{b,j}} \:
    %   \PRes[b(j)](\frac{\rs*\lambda_{p_{b,j};\:\idx
    %       1.q}}{\sect_{(p_{b,j})}})
    % \end{equation*}
    % (recalling that $\sect_{(b)} =\sect_{(\sigma+1 : b)}$,
    % $\sect_{(p_{b,j})} =\sect_{(\sigma : p_{b,j})}$ and
    % $\sect_{(p_{b,j})} = z_{b(j)} \sect_{(b)}$).
    
    Set
    \begin{equation}\label{eq-def-w}
      w_b := \smashoperator[r]{\sum_{p \in \Iset \colon \lcS+1[b] \subset
          \lcS}} \;\; \sgn{b:p}\:
      \PRes[\lcS+1[b] | \lcS](\idxup{\diff\psi_{(p)}} . u_{p})
      \; .
    \end{equation}
    It suffices to show that $w_b = 0$ on $\lcS+1[b]$ for each $b
    \in\Iset+1$ to conclude the proof.
    

  \item Show that $w_b$ is harmonic with respect to
    $\res{\vphi_F}_{\lcS+1[b]}$ (and $\res{\omega}_{\lcS+1[b]}$) on
    $\lcS+1[b]$ for all $b \in \Iset+1$ and thus $\paren{w_b}_{b
      \in\Iset+1}$ represents a class in $\spH/q-1/{\residlof+1*}$.

    {
      \newcommand{\lcSb}{\lcS+1[b]}
      % \newcommand{\idxj}{\idx[\conj j]}
      
      To see that $w_b$ is $\dbar$-closed on $\lcSb$, it suffices to
      show that $\PRes[\lcS+1[b] | \lcS](\idxup{\diff\psi_{(p)}}. 
      u_{p})$ is $\dbar$-closed for all $p\in\Iset$ such that $\lcSb
      \subset \lcS$.
      Take any admissible open set $V$ such that $V \cap \lcSb
      \neq\emptyset$ and a holomorphic coordinate system such that
      $\sect_{(p)} = z_{p(\sigma+1)} \dotsm z_{p(\sigma_V)}$ on $V$.
      Suppose $\lcSb \cap V = \lcS \cap \set{z_{p(k)} = 0}$ for some $k
      =\sigma +1, \dots, \sigma_V$.
      Recall that
      \begin{equation*}
        \diff\psi_{(p)} = \sum_{k'=\sigma+1}^{\sigma_V}
        \frac{dz_{p(k')}}{z_{p(k')}} - \diff\sm\vphi_{(p)} \quad\text{
          on } V \; .
      \end{equation*}
      By writing
      \begin{equation*}
        \idxup{dz_{p(k)}}.  u_p =: dz_{p(k)} \wedge
        \paren{\idxup{dz_{p(k)}}.  \rs u_{p,k}} \quad\text{ on }
        \lcS \cap V \; ,
      \end{equation*}
      in which $\rs u_{p,k}$ is a $(n-\sigma-1,q)$-form on $\lcS \cap
      V$, it follows that
      \begin{equation*}
        \PRes[\lcS+1[b] | \lcS](\idxup{\diff\psi_{(p)}}.  u_{p})
        =\PRes[\set{z_{p(k)}=0}](\frac{\idxup{dz_{p(k)}}. 
          u_p}{z_{p(k)}})
        =\parres{\idxup{dz_{p(k)}} . \rs u_{p,k}}_{\lcSb}
        \quad\text{ on } \lcSb \cap V \; .
      \end{equation*}
      It thus suffices to show that $\idxup{dz_{p(k)}}.  u_p$ is
      $\dbar$-closed on $\lcS \cap V$.
      % , which is guaranteed by Lemma
      % \ref{lem:commutator-dbar-ctrt}.
      Since $u_p$ is harmonic and $\ibddbar\vphi_F \geq 0$, it follows
      that $\dbar u_p = 0$ and $\nabla^{(0,1)}u_p = 0$ (Proposition
      \ref{prop:consequence-of-positivity}).
      Putting $u_p$ into $u$ and $z_{p(k)}$ into $\vphi$ in Lemma
      \ref{lem:commutator-dbar-ctrt}, one has
      $\dbar\paren{\idxup{dz_{p(k)}}.  u_p} = 0$ on $\lcS \cap V$.
      As a result, $w_b$ is $\dbar$-closed on $\lcSb$ and
      $\paren{w_b}_{b\in\Iset+1}$ therefore represents a class in
      $\spH/q-1/{\residlof+1*}$.
      % This can be done via the following formula, which is a special
      % case and a slight variant of \cite{Donnelly&Xavier}*{(2.4)} and
      % \cite{Ohsawa&Takegoshi-spectral_seq}*{Prop.~1.5} (see also
      % \cite{Takegoshi_higher-direct-images}*{(1.9)} and
      % \cite{Matsumura_injectivity-Kaehler}*{Lemma 2.1}).
      
      % \begin{lemma}[cf.~\cite{Donnelly&Xavier}*{(2.4)},
      %   \cite{Ohsawa&Takegoshi-spectral_seq}*{Prop.~1.5},
      %   \cite{Takegoshi_higher-direct-images}*{(1.9)} and
      %   \cite{Matsumura_injectivity-Kaehler}*{Lemma
      %     2.1}] \label{lem:commutator-dbar-ctrt}
      %   Let $\vphi$ be a smooth function and $u$ be a smooth
      %   ($K_X$-valued) $(0,q)$-form on a K\"ahler manifold.
      %   They satisfy the formula
      %   \begin{equation*}
      %     \dbar\paren{\idxup{\diff\vphi}.  u}
      %     =\mhlight{\idxup{\ibddbar\vphi}.  u}
      %     -\idxup{\diff\vphi} . \paren{\dbar u}
      %     +\idxup{\diff\vphi} \cdot \nabla^{(0,1)}_\bullet u \; ,
      %   \end{equation*}%
      %   % \mariocomment{I treat $u$ as a $(0,q)$-form, so it doesn't
      %   %   make sense to apply $\Lambda_\omega$ to $u$.
      %   %   Do you accept this or we have to treat sections of $K_X$ as
      %   %   $(n,0)$-forms?
      %   %   I don't recommend the latter choice since it's easier to say
      %   %   and understand a ``$K_{\lcS}$-valued section'' than a
      %   %   ``$(n-\sigma,\bullet)$-form''.
      %   % }%
      %   \mariocomment[Red]{SM is treating $u$ as an $F$-valued $(n-\sigma, \bullet)$-form in my head. 
      %     I understand that the reason why the vanishing theorem holds for the adjoint bundle $K \times F$ 
      %     comes from the special property of $(n,q)$-type, so this is
      %     more natural for me.}%
      %   \mmark*{}{MC: OK. Since formula in Sec.
      %       \ref{subsec:n2} Step 3 also goes without $\Lambda_\omega$,
      %     I feel safe to keep things as they are. BTW, I prefer to
      %     keep this proof after I compare the type of $\diff\vphi$ and
      %   $\dbar\Phi$ in the referred papers.}%
      %   or, when a local holomorphic coordinate system is fixed and
      %   the Einstein summation convention is applied, 
      %   \begin{equation*}
      %     \paren{\dbar\paren{\idxup{\diff\vphi}.  u}}_{\conj J_{q}}
      %     =\sum_{\nu=1}^q \diff^{\conj\ell} \diff_{\conj j_\nu} \vphi \:
      %     u_{\idxj 1[\dotsm (\conj \ell)_\nu].q}
      %     -\diff^{\conj\ell}\vphi  \:\paren{\dbar u}_{\conj\ell\conj J_q}
      %     +\diff^{\conj\ell}\vphi \:\nabla_{\conj\ell} u_{\conj J_q} 
      %   \end{equation*}
      %   for any multi-indices $J_q = (\idx[j]1,q)$, pointwisely.
      % \end{lemma}

      % \begin{proof}
      %   A direct computation yields
      %   \begin{align*}
      %     \paren{\dbar\paren{\idxup{\diff\vphi} . u}}_{\conj
      %     J_{q}}
      %     &=\sum_{\nu=1}^q (-1)^{\nu-1} \diff_{\conj j_\nu}
      %       \paren{\idxup{\diff\vphi}.  u}_{\idxj 1[\dotsm \widehat
      %       {\conj j}_\nu].q}
      %       =\sum_{\nu=1}^q (-1)^{\nu-1} \diff_{\conj j_\nu}
      %       \paren{\diff_{\ell}\vphi \: u^\ell_{\;\idxj 1[\dotsm
      %       \widehat{\conj j}_\nu].q}}
      %     \\
      %     &=\sum_{\nu=1}^q (-1)^{\nu-1} \paren{
      %       \diff_{\conj j_\nu}\diff_{\ell}\vphi \: u^{\ell}_{\;\idxj 1[\dotsm
      %       \widehat {\conj j}_\nu].q}
      %       +\diff_{\ell}\vphi \: \nabla_{\conj j_\nu} u^\ell_{\;\idxj 1[\dotsm
      %       \widehat {\conj j}_\nu].q}
      %       }
      %     \\
      %     &=\sum_{\nu=1}^q
      %       \diff^{\conj \ell}\diff_{\conj j_\nu}\vphi \: u_{\idxj 1[\dotsm
      %       (\conj\ell)_\nu].q}
      %       -\diff^{\conj\ell}\vphi \sum_{\nu=1}^q (-1)^{\nu} 
      %       \nabla_{\conj j_\nu} u_{\conj\ell \idxj 1[\dotsm
      %       \widehat {\conj j}_\nu].q}
      %       \begin{aligned}[t]
      %         &-\diff^{\conj\ell}\vphi \: \nabla_{\conj \ell} u_{\conj
      %           J_q} \\
      %         &+\diff^{\conj\ell}\vphi \: \nabla_{\conj \ell}
      %         u_{\conj J_q}
      %       \end{aligned}
      %     \\
      %     &=\sum_{\nu=1}^q
      %       \diff^{\conj \ell}\diff_{\conj j_\nu}\vphi \: u_{\idxj 1[\dotsm
      %       (\conj\ell)_\nu].q}
      %       -\diff^{\conj\ell}\vphi
      %       \:\paren{\dbar u}_{\conj\ell\conj J_q}
      %       +\diff^{\conj\ell}\vphi \: \nabla_{\conj \ell} u_{\conj
      %       J_q} \; ,
      %   \end{align*}
      %   as desired.
      % \end{proof}


    }

    Furthermore, by Proposition \ref{prop:harmonic-residue} and
    Theorem \ref{thm:residue-harmonic} (with
    $\lcS$ in place of $X$, $\lcS+1[b]$ in place of $D_p$ and
    $\psi_{(p)}$ in place of $\psi_{D_p}$), $w_b$ is a
    $K_{\lcS+1[b]} \otimes \res{F}_{\lcS+1[b]}$-valued $(0,q-1)$-form on
    $\lcS+1[b]$ (not only a $\res{\conj\holoform_X^{q-1}}_{\lcS+1[b]}$-valued
    section) which is harmonic with respect to
    $\res{\vphi_F}_{\lcS+1[b]}$.
    % Therefore, $\paren{w_b}_{b\in\Iset+1}$ represents a class in
    % $\spH/q-1/{\residlof+1*}$. 



  \item \label{item:pf:use_u-ortho-w}
    Apply the assumption $u =(u_p)_{p\in\Iset} \in
    \paren{\ker\tau_\sigma}^{\perp}$ via the use of $w
    :=\paren{w_b}_{b \in \Iset+1} \in \spH/q-1/{\residlof+1*}
    =\bigoplus_{b \in\Iset+1} \cohgp{q-1}[\lcS+1[b]]{K_{\lcS+1[b]}
      \otimes F}$ in view of the commutative diagram
    \begin{equation*}
      \xymatrix@R-0.3cm{
        {\dotsm} \ar[r]
        & {\spH/q-1/{\residlof+1*}} \ar[r]^-{\delta}
        \ar[d]^-{\tau_{\sigma+1}}
        & {\spH{\residlof*}} \ar[r]
        \ar@{=}[d]
        & {\spH{\faidlof+1/-1*}} \ar[r]
        \ar[d]
        & {\dotsm}
        \\
        {\dotsm} \ar[r]
        & {\spH/q-1/{\faidlof|\sigma_{\mlc}|*}} \ar[r]
        & {\spH{\residlof*}} \ar[r]^-{\tau_\sigma}
        & {\spH{\faidlof|\sigma_{\mlc}|/-1*}} \ar[r]
        & {\dotsm}
      }
    \end{equation*}
    and conclude that $u_p = 0$ on $\lcS$ for each $p\in\Iset$.

    From the commutative diagram, one sees that $\delta w \in
    \ker\tau_\sigma$.
    In view of the isomorphism $\residlof+1* \isom \faidlof+1*$ and by
    following the procedures in Step
    \ref{item:express-su-in-residue-norm}, one obtains
    a $\logKX \otimes \aidlof+1*$-valued \v Cech $(q-1)$-cochain
    $\set{\gamma_{\idx 1.q}}_{\idx 1,q \in I}$ with respect to $\cvr
    V$ such that, when
    \begin{equation*}
      \paren{\alpha'_{b; \:\idx 1.q}}_{b\in\Iset+1}
      :=\paren{\frac{\rs*\gamma_{b; \:\idx
            1.q}}{\sect_{(b)}}}_{\mathrlap{b\in\Iset+1}} \quad\;
      :=\Res^{\sigma+1}(\gamma_{\idx 1.q})
      \in \smashoperator[r]{\prod_{b\in\Iset+1}} K_{\lcS+1[b]} \otimes
      \res{F}_{\lcS+1[b]} \paren{\lcS+1[b] \cap V_{\idx 1.q}}
    \end{equation*}
    (notation chosen for the consistency with those in Proposition
    \ref{prop:res-formula-dbar-exact-dot-harmonic}) and
    \begin{equation*}
      \eqcls{\gamma_{\idx 1.q}}
      := \paren{\gamma_{\idx 1.q} \bmod \aidlof*} \in \logKX \otimes
      \faidlof+1* \quad\text{ on } V_{\idx 1.q} \; ,
    \end{equation*}
    the collection $\set{\alpha'_{b;\:\idx 1.q}}_{\idx 1,q\in I}$
    is a \v Cech \emph{$(q-1)$-cocycle} representing (the class of)
    $w_b$ in $\cohgp{q-1}[\lcS+1[b]]{K_{\lcS+1[b]} \otimes F}$ for
    each $b \in\Iset+1$ such that
    \begin{equation*}
      w_b = \dbar v'_{b;(2)} +(-1)^{q-1} \:\underbrace{
        \dbar \rho^{i_{q}} \wedge \dotsm \wedge
        \dbar\rho^{i_2} \cdot \rho^{i_1} }_{=: \:
        \paren{\dbar\rho}^{\idx q.1}} \alpha'_{b;\:\idx 1.q}
      =: \dbar v'_{b;(2)} +(-1)^{q-1} \frac{v_{b;(\infty)}'}{\sect_{(b)}}
    \end{equation*}
    (again, notation chosen for the consistency with those in Proposition
    \ref{prop:res-formula-dbar-exact-dot-harmonic})
    for some $K_{\lcS+1[b]} \otimes \res{F}_{\lcS+1[b]}$-valued
    $(0,q-2)$-form $v'_{b;(2)}$ on $\lcS+1[b]$ with $L^2$ coefficients
    with respect to $\norm\cdot_{\lcS+1[b]}$, and the collection
    $\set{\eqcls{\gamma_{\idx 1.q}}}_{\idx 1,q \in I}$ is a \v Cech
    \emph{$(q-1)$-cocycle} representing (the class of) $w$ in
    $\spH/q-1/{\faidlof+1*} \xrightarrow[\isom]{\Res^{\sigma+1}}
    \spH/q-1/{\residlof+1*}$.
    The image $\delta w$ in $\spH{\residlof*}$ is then represented by
    \begin{equation*}
      \set{\Res^\sigma\paren{\paren{\delta\gamma}_{\idx 0.q}}}_{\idx 0,q \in
      I} \; ,
    \end{equation*}
    in which $\delta$ is the \v Cech boundary operator.
    Note that applying $\Res^\sigma$ to $\paren{\delta\gamma}_{\idx
      0.q}$ is valid as $\set{\eqcls{\gamma_{\idx 1.q}}}_{\idx 1,q \in
      I}$ is a cocycle and thus coefficients of
    $\paren{\delta\gamma}_{\idx 0.q}$ lie in $\aidlof*$.
    Set
    \begin{equation*}
      \rs*\gamma_{p;\:\idx 1.q} := \PRes[\lcS](\frac{\gamma_{\idx
            1.q}}{\sect_D}) \cdot \sect_{(p)} 
    \end{equation*}
    such that
    \begin{equation*}
      \Res^\sigma\paren{\paren{\delta\gamma}_{\idx 0.q}}
      =\paren{\frac{\paren{\delta\rs*\gamma_p}_{\idx
            0.q}}{\sect_{(p)}}}_{p \in \Iset}
      \in \prod_{p \in \Iset} K_{\lcS} \otimes \res F_{\lcS}
      \paren{\lcS \cap V_{\idx 0.q}} \; .
    \end{equation*}
    Note that 
    \begin{equation*}
      (-1)^q \paren{\dbar\rho}^{\idx q.0}
      \frac{\paren{\delta \rs*\gamma_p}_{\idx 0.q}}{\sect_{(p)}}
      % =(-1)^q \paren{\dbar\rho}^{\idx q.1}
      % \frac{\rs*\gamma_{p;\:\idx 1.q}}{\sect_{(p)}}
      =-
      \frac{\dbar\paren{\paren{\dbar\rho}^{\idx q.1} \rs*\gamma_{p;\:\idx 1.q}}}{\sect_{(p)}}
      =: - \:\frac{\dbar v_{p; (\infty)}'}{\sect_{(p)}}
      \quad\text{ on } \lcS
    \end{equation*}
    is a $\dbar$-closed form representing the class of $\res{\delta
      w}_{\lcS}$ (the component of $\delta w$ on $\lcS$) in $\cohgp
    q[\lcS]{K_{\lcS} \otimes F}$ via Dolbeault isomorphism.
    
    Therefore, from the assumption $u \in
    \paren{\ker\tau_\sigma}^\perp$ and taking into account the \v
    Cech--Dolbeault isomorphism \eqref{eq:Cech-Dolbeault-isom} and the
    fact that each $u_p$ is harmonic, one has
    \begin{align*}
      0 =\iinner{(-1)^{q-1} \delta w}{u}_{\lcc'}
      &=(-1)^q \sum_{p\in \Iset} \iinner{\frac{\dbar
          v_{p;(\infty)}'}{\sect_{(p)}}}{u_p}_{\lcS}
      =(-1)^q \sum_{p\in \Iset} \iinner{\dbar v_{p;(\infty)}'}{u_p
        \sect_{(p)}}_{\lcS, \phi_{(p)}}
      \\
      &\overset{\mathclap{\text{Prop.~\ref{prop:res-formula-dbar-exact-dot-harmonic}}}}=
        \quad \;\; (-1)^{q-1} \sigma
        \smashoperator{\sum_{b \in \Iset+1}} \iinner{v_{b;(\infty)}'}{w_b
        \sect_{(b)}}_{\lcS+1[b], \phi_{(b)}}
      \\
      &=\sigma \smashoperator{\sum_{b \in \Iset+1}}
        \iinner{
          \paren{w_b -\dbar v_{b;(2)}'} \sect_{(b)}
        }{
          w_b \sect_{(b)}
        }_{\lcS+1[b], \phi_{(b)}}
      \\
      &\overset{\mathclap{w_b \text{ harmonic}}}= \quad\;\;\;
        \sigma
        \smashoperator{\sum_{b \in \Iset+1}}
        \iinner{w_b}{w_b}_{\lcS+1[b]}
        =\sigma \norm{w}_{\lcc+1'}^2 \; .
    \end{align*}
    As a result, $w_b = 0$ for each $b\in\Iset+1$, thus $su_p = 0$
    (hence $u_p = 0$) for each $p\in\Iset$ by Step
    \ref{item:express-su-in-residue-norm}.
    This completes the proof. \qedhere
  \end{enumerate}
\end{proof}

\begin{remark} \label{rem:singular-vphi_F}
  When $\vphi_F$ and $\vphi_M$ have only neat analytic singularities
  such that $\vphi_F^{-1}(-\infty) \cup \vphi_M^{-1}(-\infty) \cup D$
  has only snc and $\vphi_F^{-1}(-\infty) \cup \vphi_M^{-1}(-\infty)$
  contains no irreducible components of $D$ (hence no lc centers of
  $(X,D)$), the proof is still valid when the K\"ahler metric $\omega$
  on $X$ is replaced by a complete metric on $X \setminus
  \paren{\vphi_F^{-1}(-\infty) \cup \vphi_M^{-1}(-\infty)}$ as
  described in \cite{Chan&Choi_injectivity-I}*{\S 2.2 item (4)}.
  See \cite{Chan&Choi_injectivity-I}*{\S 3.3} for the technical
  modifications required.
\end{remark}

\begin{remark} \label{rem:no-hard-Lefschetz}
  Notice that the refinement of hard Lefschetz theorem (see
  \cite{Matsumura_injectivity-lc}*{Thm.~1.7} or
  \cite{Chan&Choi_injectivity-I}*{Thm.~2.5.1}) is not used in this
  proof.
  It is used in previous works to show that $\frac{u}{\sect_D}$ is
  smooth for every harmonic $u$ representing a class in $\cohgp
  q[X]{\logKX\otimes \mtidlof<X>{\phi_D}}$.
  This argument can be replaced by using directly the isomorphism
  induced by $\holo_X \xrightarrow[\isom]{\otimes \sect_D} D \otimes
  \mtidlof<X>{\phi_D}$, or $\holo_{\lcS} \xrightarrow[\isom]{\otimes
    \sect_{(p)}} \Diff_p(D) \otimes \mtidlof<\lcS>{\phi_{(p)}}$, which
  is more relevant to this article (see also Lemma
  \ref{lem:su-harmonicity}).
  However, when $\vphi_F$ and $\vphi_M$ have neat analytic singularities
  as described in \cite{Chan&Choi_injectivity-I}*{\S 2.2}, the theorem
  is still needed to get certain control of the regularity of $u$ on the
  polar sets of $\vphi_F$ and $\vphi_M$ (see
  \cite{Chan&Choi_injectivity-I}*{Prop.~3.3.1}).
\end{remark}

%%% Local Variables:
%%% mode: latex
%%% TeX-master: "Injectivity-Fujino"
%%% coding: utf-8
%%% End:


%%%%% Reference list %%%%%
\begin{bibdiv}
  \begin{biblist}
    \IfFileExists{references.ltb}{
      \bibselect{references}
    }{
      %%%%%
%%%%% File name  : Injectivity-Fujino.tex
%%%%% Author     : Mario Chan
%%%%% Date       : 25th July, 2022 
%%%%% Description: This file is set up to compile the project with the
%%%%%              class amsart.cls. This project "Injectivity-Fujino"
%%%%%              proves the injectivity theorem on lc pairs of
%%%%%              arbitrary codimension of the mlc's, the full Fujino
%%%%%              conjecture.
%%%%%
%%
%%%
\documentclass[a4paper,12pt]{amsart}

\pdfoutput=1 %% added to force arXiv AutoTeX to typeset with pdflatex
             %% (so that rotation of symbols in Xy-pic is rendered);
             %% have to be within the first 5 lines of preamble

\usepackage[top=3cm,bottom=3cm,outer=3cm,inner=2cm,marginpar=2.45cm]{geometry}
\usepackage[destlabel,final,colorlinks=true]{hyperref}
\usepackage[abbrev]{amsrefs}

%%%%%
%%%%% File name  : packagesandcommands.tex
%%%%% Author     : Mario Chan
%%%%% Date       : 13th December, 2021 (original: 04th November, 2020)
%%%%% Description: This file collects the packages used and commands
%%%%%              defined in the project "Injectivity-Fujino".
%%%%%
%%
%%%

\usepackage[french,ngerman,english]{babel}
\usepackage[utf8]{inputenc}
\usepackage[T1]{fontenc}

% \usepackage{CJKutf8} %%%%% for the use of Chinese, use the
                       %%%%% environment
                       %%%%% \begin{CJK}{UTF8}{bkai} % or {bsmi}
                       %%%%% \end{CJK}

\usepackage[all]{xy}
\renewcommand{\objectstyle}{\displaystyle}

\usepackage{enumitem}
\usepackage{mathtools}  
\usepackage[usenames,dvipsnames]{xcolor}
\usepackage{calc} %% for doing length computations

\babeltags{de = ngerman}
\babeltags{fr = french}

\usepackage{upref}  %% for uprighting texts generated by \ref
\usepackage{embrac} %% for uprighting parentheses in \emph and
                    %% providing \embparen for the same effect in
                    %% theorem environments (which use {\itshape ...}
                    %% or {\em ...})

\usepackage[
% show-mario,  %% show editing notes or comments on margin when
             %% uncommented
% no-commands %% avoid loading commands in mariostdcommands.tex when uncommented
]{marionotations}
% %%%%%
%%%%% File name  : mariostdcommands.tex
%%%%% Author     : Mario Chan
%%%%% Last update: 2nd September, 2022
%%%%% Description: Math LaTeX command definitions (without packages)
%%%%%              used by Mario Chan.
%%%%%
%%
%%%

%%%%%%%%%%%%%%%%%%%%%%%%%%%%%%%%%%%%%%%%%%%%%%%%%%%%%%%%%%%%%%%%
%% Begin of definitions of commands 
%%%%%%%%%%%%%%%%%%%%%%%%%%%%%%%%%%%%%%%%%%%%%%%%%%%%%%%%%%%%%%%%



%\DeclareMathAlphabet{\mathpzc}{OT1}{pzc}{m}{it}

\newcommand{\fieldC}{\mathbb{C}}
\newcommand{\fieldR}{\mathbb{R}}
\newcommand{\fieldQ}{\mathbb{Q}}
\newcommand{\Znum}{\mathbb{Z}}
\newcommand{\Nnum}{\mathbb{N}}
\newcommand{\Nnump}{\mathbb{N}_{>0}}
%\newcommand{\Onum}{\mathfrak{o}}
%\newcommand{\Cl}{\mathscr{C}\!\mathpzc{l}}
\newcommand{\Zmod}[1]{\Znum/#1\Znum}

\newcommand{\cplxi}{\sqrt{-1}}
% \newlength{\ibarbarheight}
% \setlength{\ibarbarheight}{-0.9ex}
% \newcommand{\ibar}{{\raisebox{\ibarbarheight}{$\mathchar'26$}\mkern-6.7mu i}}
\newcommand{\ibar}{
  \smash[b]{
    \begin{tikzpicture}[trim left, baseline=0pt]
      \node (i) [anchor=base, text width=0pt, text height=0pt] {$i$};
      \node (bar) [below left=0.75ex and 0.12ex, inner sep=0pt,
      outer sep=0pt, text width=0ex, text height=0ex] {$\mathchar'26$};
    \end{tikzpicture}
  }
}
\newcommand{\ibardefn}{\frac{\cplxi}{2\pi}}
\newsavebox{\ibarExplainBox}
\begin{lrbox}{\ibarExplainBox}
  \footnotesize
  \verb:{\raisebox{-0.9ex}{$\mathchar'26$}\mkern-6.7mu i}: 
\end{lrbox}
\newcommand{\ibarfootnote}{\footnote{The notation is chosen by
    mimicking the reduced Planck constant $\hbar = \frac{h}{2\pi}$. It
    can be typeset with the code \usebox\ibarExplainBox.}}
\newcommand{\proj}{\mathbb{P}}
\newcommand{\sphere}{\mathbb{S}}
\newcommand{\uhp}{\mathfrak{H}}
\newcommand{\eps}{\varepsilon}
\newcommand{\vphi}{\varphi}

\newcommand{\sheaf}[1]{\mathscr{#1}}
\newcommand{\ideal}[1]{\mathfrak{#1}}
\newcommand{\bundle}[1]{\mathbb{#1}}

\newcommand{\mero}{\sheaf{M}}
\newcommand{\holo}{\sheaf{O}}
\newcommand{\holoform}{\boldsymbol\Omega}
\newcommand{\kanshf}[1][X]{\sheaf K_{#1}}
\newcommand{\smooth}[1][\infty]{\mathscr C^{#1}}
\NewDocumentCommand{\smform} { s D//{p,q} O{X} }{
  \sheaf A^{#2}_{#3\IfBooleanT{#1}{\,c}}  
}
\newcommand{\Lloc}[1][1]{L^{#1}_{\text{loc}}}
\newcommand{\maxidl}{\ideal m}
\newcommand{\multidl}{\sheaf I}

\NewDocumentCommand{\Tgt}{ %% Tangent bundle
  s       %% #1 turn to cotangent bundle when starred
  d//     %% #2 types of (co)tangent vectors wrt almost cplx structure
}{\mathbf T^{\IfBooleanT{#1}{*\IfNoValueF{#2}{\,}}\IfNoValueF{#2}{#2}}}

\def\cTgt{\Tgt*} %% Cotangent bundle, for backward compatibility

% \newcommand{\Tgt}[1][]{\mathbf T^{#1}}
% \newcommand{\cTgt}[1][]{\Tgt[*\,#1]}

\NewDocumentCommand{\cohgp}{
  O{H}   %% #1 symbol for cohomology (e.g.~H or \check H)
  t_     %% #2 with "_", it becomes homology
  m      %% #3 (co)homology degree
  D<>{}  %% #4 sub/super-scripts to be put to the (co)homology group
  o      %% #5 supporting space
  D//{,} %% #6 separator between space and coefficients 
  m      %% #7 coefficients
}{
  #1\IfBooleanTF{#2} {_{#3}^{#4}} {^{#3}_{#4}} \paren{\IfNoValueF{#5}{#5#6} #7}
}

\NewDocumentCommand{\cohdim}{
  t_     %% with "_", it becomes homology
  m      %% (co)homology degree
  D<>{}  %% sub/super-scripts to be put to the (co)homology group
  o      %% supporting space
  D//{,} %% separator between space and coefficients 
  m      %% coefficients
}{
  h\IfBooleanTF{#1} {_{#2}^{#3}} {^{#2}_{#3}} \paren{\IfNoValueF{#4}{#4#5} #6}
}

\newcommand{\codim}{\operatorname{codim}}
\newcommand{\Sing}{\operatorname{Sing}}
\newcommand{\mult}{\operatorname{mult}}
\newcommand{\ord}{\operatorname{ord}}
\newcommand{\im}{\operatorname{im}}
\newcommand{\Res}{\operatorname{Res}}
\newcommand{\id}{\operatorname{id}}
\newcommand{\coker}{\operatorname{coker}}
\newcommand{\pr}{\operatorname{pr}}
\newcommand{\Tr}{\operatorname{Tr}}
\newcommand{\rk}{\operatorname{rk}}
\newcommand{\vol}{\operatorname{vol}}
\newcommand{\supp}{\operatorname{supp}}
\newcommand{\Dom}{\operatorname{Dom}}
\newcommand{\Alb}{\operatorname{Alb}}
\newcommand{\alb}{\operatorname{alb}}
\newcommand{\Pic}{\operatorname{Pic}}
\newcommand{\Exc}{\operatorname{Exc}}
\newcommand{\Hess}{\operatorname{Hess}}
\newcommand{\esssup}{\operatorname*{ess\,sup}}
\newcommand{\BigO}{\operatorname{\mathbf{O}}}

% \newcommand{\bexp}{\boldsymbol{\operatorname{e}}}

\renewcommand{\Re}{\operatorname{Re}}
\renewcommand{\Im}{\operatorname{Im}}

\newcommand{\symmgp}{\mathfrak{S}}
\newcommand{\GL}[2][2]{\mathrm{GL}_{#1}(#2)}
\newcommand{\SL}[2][2]{\mathrm{SL}_{#1}(#2)}
\newcommand{\SO}[2][2]{\mathrm{SO}_{#1}(#2)}
% \newcommand{\SL}[1]{\mathrm{SL}_2(#1)}
% \newcommand{\SO}[1]{\mathrm{SO}_2(#1)}
\newcommand{\SU}[1]{\mathrm{SU}(#1)}
\newcommand{\Aut}[1]{\mathrm{Aut}(#1)}
\newcommand{\Div}[1]{\mathrm{Div}(#1)}
\newcommand{\divsr}[1]{\mathrm{div}\paren{#1}}

\newcommand{\sgn}[1]{\operatorname{sgn}(#1)}

%%%%% Could be replaced by calling the package "mleftright" 
\let\originalleft\left
\let\originalright\right
\renewcommand{\left}{\mathopen{}\mathclose\bgroup\originalleft}
\renewcommand{\right}{\aftergroup\egroup\originalright}
%%%%%

\newcommand{\lparpht}[2][(]{\mathopen{\left#1\vphantom{#2}\right.\kern-\nulldelimiterspace}}
\newcommand{\rparpht}[2][)]{\mathclose{\left.\kern-\nulldelimiterspace\vphantom{#2}\right#1}}
% \newcommand{\lparpht}[1]{\mathopen{\left(\vphantom{#1}\right.\kern-\nulldelimiterspace}}
% \newcommand{\rparpht}[1]{\mathclose{\left.\kern-\nulldelimiterspace\vphantom{#1}\right)}}
% \newcommand{\paren}[1]{\!\left(#1\right)}
\newcommand{\paren}[1]{\lparpht{#1}#1\rparpht{#1}}
\newcommand{\bigparen}[1]{\bigl(#1\bigr)}
\newcommand{\res}[1]{\left.#1\right|}
\newcommand{\parres}[1]{\res{\paren{#1}}}
\newcommand{\seq}[1]{\lparpht[\{]{#1}#1\rparpht[\}]{#1}}
% \newcommand{\seq}[1]{\left\{#1\right\}}
\newcommand{\set}[1]{\lparpht[\{]{#1}#1\rparpht[\}]{#1}}
% \newcommand{\set}[1]{\left\{#1\right\}}
\newcommand{\setd}[2]{\left\{#1\:\left|\;\vphantom{#1} #2\right.\right\}}
\newcommand{\abs}[1]{\left\lvert#1\right\rvert}
\newcommand{\bigabs}[1]{\bigl\lvert#1\bigr\rvert}
\newcommand{\norm}[1]{\left\lVert#1\right\rVert}
\newcommand{\bignorm}[1]{\bigl\lVert#1\bigr\rVert}
\newcommand{\Bignorm}[1]{\Bigl\lVert#1\Bigr\rVert}
\newcommand{\inner}[2]{\left\langle#1,#2\right\rangle}
\newcommand{\ptinner}[2]{\left(#1,#2\right)}
\newcommand{\biginner}[2]{\bigl\langle#1,#2\bigr\rangle}
\newcommand{\commut}[2]{\left[#1,#2\right]}
\newcommand{\algnorm}[1]{\mathbf N(#1)}
\newcommand{\smod}[1]{\,(\operatorname{mod}\,#1)}
\newcommand{\ceil}[1]{\left\lceil#1\right\rceil}
\newcommand{\Ceil}[1]{\bigl\lceil#1\bigr\rceil}
\newcommand{\floor}[1]{\left\lfloor#1\right\rfloor}
\newcommand{\fracpart}[1]{\left\{#1\right\}}

\newcommand{\tp}[1]{\,{\vphantom{#1}}^*\! #1}
\newcommand{\rtp}[1]{\,{\vphantom{#1}}^t\! #1}

\newcommand{\genby}[1]{\left\langle#1\right\rangle}
\newcommand{\genbyd}[2]{\left\langle#1\:\left|\;\vphantom{#1} #2\right.\right\rangle}

\newcommand{\tmatrix}[4]{\left[\begin{smallmatrix}#1 & #2 \\ #3 & #4\end{smallmatrix}\right]}
\newcommand{\Diag}[1]{\operatorname{diag}\bigl(#1\bigr)}

\newcommand{\cl}{\overline}

\newcommand{\vect}[1]{\underline{#1}}

\newcommand{\conj}{\overline}
\newcommand{\diff}{\partial}
%% \dbar is defined in amsrefs (mathscinet) with out 'lite' option, so
%% \renewcommand is used 
\makeatletter
\@ifundefined{dbar}{
  \newcommand{\dbar}{{\conj\diff}}  
}{
  \renewcommand{\dbar}{{\conj\diff}}
  % \PackageWarningNoLine{marionotations}{`\protect\dbar' is redefined}
  \@latex@warning@no@line{`\protect\dbar' is redefined in mariostdcommands.tex}
}
\makeatother

\NewDocumentCommand{\fdiff}{O{#3} m O{}}{\frac{\diff #1}{\diff #2}}
\newcommand{\dfadj}{\vartheta}
\newcommand{\ddbar}{\diff\dbar}
\newcommand{\iddbar}{\cplxi\ddbar}
\newcommand{\ibddbar}{\ibar\ddbar}

\newcommand{\ddc}{dd^c\mspace{1mu}}

\newcommand{\bdry}{\partial}

\newcommand{\birat}{\mathrel{\dashrightarrow}}
\newcommand{\tendsto}{\mathrel{\rightarrow}}
\newcommand{\xtendsto}[1]{\mathrel{\xrightarrow{#1}}}
\newcommand{\wktendsto}{\mathrel{\rightharpoonup}}
\newcommand{\descendsto}{\mathrel{\searrow}}
\newcommand{\ascendsto}{\mathrel{\nearrow}}
\newcommand{\imply}{\mathrel{\Rightarrow~}}
\newcommand{\isom}{\mathrel{\cong}}
\newcommand{\ctrt}{\mathbin{\lrcorner}}

 %% load mariostdcommands.tex separately
                           %% when uncommented (must put "no-commands"
                           %% in the package option of marionotations
                           %% in that case);

\usepackage[normalem]{ulem} %% used only in review report

\usepackage{bbm}     %% to use \mathbbm (like \mathbb but works also
                     %%                  for natural numbers)

\usepackage[Smaller]{cancel} %%%%% for crossing out argument in math mode via
                             %%%%% the  use of \cancelto 
\renewcommand{\CancelColor}{\color{Gray}}

\usepackage{breakurl}  %% used so that line breaks for contents in
                       %% \url{...} are possible when processed by
                       %% LaTeX instead of pdfLaTeX (e.g. arXiv.org) 

\usepackage{subfiles}

%%%%% Commands for this document %%%%%
\newcommand{\defaultDimension}{n}
\newcommand{\setDefaultDimension}[1]{\renewcommand{\defaultDimension}{#1}}

\newcommand{\defaultAmbientSpace}{X}
\newcommand{\setDefaultAmbientSpace}[1]{\renewcommand{\defaultAmbientSpace}{#1}}

\newcommand{\defaultlcIndex}{\sigma}
\newcommand{\setDefaultlcIndex}[1]{\renewcommand{\defaultlcIndex}{#1}}

\newcommand{\defaultcohDegree}{q}
\newcommand{\setDefaultcohDegree}[1]{\renewcommand{\defaultcohDegree}{#1}}

\newcommand{\defaultlclocus}{D}
\newcommand{\setDefaultlclocus}[1]{\renewcommand{\defaultlclocus}{#1}}

\newcommand{\defaultvphi}{\vphi_F}
\newcommand{\setDefaultvphi}[1]{\renewcommand{\defaultvphi}{#1}}

\newcommand{\defaultpsi}{\psi_D}
\newcommand{\setDefaultpsi}[1]{\renewcommand{\defaultpsi}{#1}}

\newcommand{\defaultMetric}{\omega}
\newcommand{\setDefaultMetric}[1]{\renewcommand{\defaultMetric}{#1}}


% The delimiters for the arguments are carefully chosen so that they
% are consistent among most of the commands (in particular for the
% commands for generating the symbols for the ideal sheaves and
% residue sheaves), namely,
%     <X>        for base space,
%     (S)        for lc locus,
%     |\sigma|   for lc index,
%     {\vphi}    for potential,
%     [\psi]     for $\psi$ function,
%     .{m_k}     for jumping number,
%     +{1}       for increment of lc index (by $1$),
%     -{1}       for decrement of lc index (by $1$).
%     /q/        for anti-holomorphic degree (or hol. and anti-hol. degrees)

\newcommand{\alert}[2][RoyalBlue]{{\color{#1}#2}}

\NewDocumentCommand{\logKX}{
  t{M} %% #1 include M in the tensor product if present
  o    %% #2 replace F \otimes M by the argument when provided
}{K_X \otimes D \otimes \IfNoValueTF{#2}{F \IfBooleanT{#1}{\otimes M}}{#2}}

% \NewDocumentCommand{\vphilist}{
%   D||{\vphi}           %% #1 potentials
%   t{F}                 %% #2 turn potential to "\vphi_F" if present
%   t{M}                 %% #3 add "+\vphi_M" if present
%   d()                  %% #4 extra metric for the (1,0)-forms
%   D<>{\defaultMetric}  %% #5 metric on the ambient space
% }{\IfBooleanTF{#2}{\vphi_F}{#1} \IfBooleanT{#3}{+\vphi_M}, \IfNoValueF{#4}{(#4),} #5}

\NewDocumentCommand{\Ltwo}{ %% the space of L2 sections 
  D//{\bullet,\bullet}      %% #1 the order of forms
  D<>{\defaultAmbientSpace} %% #2 base space
  s                         %% #3 base space is hidden if * is present
  m                         %% #4 coefficient
}{L^{#1}_{(2)}\paren{\IfBooleanF{#3}{#2;} #4}}

% \NewDocumentCommand{\Ltwosp}{
%   t{'}                    %% #1 no preassigned holomorphic degree if present
%   D//{\defaultcohDegree}  %% #2 anti-holomorphic degree
%   t{M}                    %% #3 include M in the coefficient if present
%   o                       %% #4 replace F \otimes M by the argument if provided
%   G{\defaultvphi}         %% #5 potential on line bundle
%   e{_}                    %% #6 metric on the base space or other subscripts
% }{\Ltwo/\IfBooleanF{#1}{\defaultDimension,}#2/*{D \otimes \IfNoValueTF{#4}{F \IfBooleanT{#3}{\otimes M}}{#4}}_{#5 \IfNoValueF{#6}{,#6}}}


%\def\mH{\mathcal{H}}
% \NewDocumentCommand{\Harm}{ %% the space of harmonic forms
%   O{q}
% }{\mathcal{H}^{n,#1}}
\NewDocumentCommand{\Harm}{ %% the space of harmonic forms
  t{'}                      %% #1 no preassigned hol degree if present
  D//{\defaultcohDegree}    %% #2 anti-holomorphic degree
  D<>{\defaultAmbientSpace} %% #3 the base space
  g                         %% #4 the coefficient; will be hidden
                            %%    together with the base space if not provided
  t{,}                      %% #5 separator
  G{\defaultvphi}           %% #6 potential on line bundle
  e{_}                      %% #7 metric on the base space or other subscripts
}{\mathcal{H}^{\IfBooleanF{#1}{\defaultDimension,}#2}\IfNoValueF{#4}{\paren{#3;#4}}_{#6 \IfNoValueF{#7}{,#7}}}


\NewDocumentCommand{\lcIndex}{ %% for displaying the lc index,
                               %% intended to be used internally
  m  %% #1 the basic lc index (\sigma)
  m  %% #2 amount added to the index
  m  %% #3 amount substracted from the index
}{#1\IfNoValueF{#2}{+#2}\IfNoValueF{#3}{-#3}}

\NewDocumentCommand{\lcData}{ %% for displaying lc data in the format
                              %% like "(\vphi_L ; m_k . \psi)"
  G{\defaultvphi}  %% #1 potential or q-psh function
  O{\defaultpsi}   %% #2 lc locus psi function
  e{.}             %% #3 jumping number
}{\paren{#1; \IfNoValueF{#3}{#3 \cdot} #2}}

\NewDocumentCommand{\lcdata}{ %% for displaying lc data in the
                              %% format like "(X,\vphi_L,\psi,m_k)"
  s                %% #1 no parentheses if starred 
  d<>              %% #2 base space
  G{\defaultvphi}  %% #3 potential or q-psh function
  O{\defaultpsi}   %% #4 lc locus psi function
  e{.,}            %% #5 jumping number
                   %% #6 extra components
}{\newcommand{\datalist}{\IfNoValueF{#2}{#2,}#3,#4\IfNoValueF{#5}{,#5}\IfNoValueF{#6}{,#6}}
\IfBooleanTF{#1}{\datalist}{\paren{\datalist}}}



\newcommand{\spHbase}{\mathbb{H}}
\NewDocumentCommand{\spH}{ %% cohomology group with coefficients 
                           %% K_X +F +D \otimes the given sheaf
  D//{\defaultcohDegree}  %% #1 degree of anti-hol form
  t{M}                    %% #2 with 'M' to display M in the %% coefficient
  m                       %% #3 
}{\spHbase^{#1}\paren{\IfBooleanT{#2}{M\otimes}#3}}
% \NewDocumentCommand{\spH}{ %% cohomology group with coefficients 
%                            %% vanishing on \lcc[\sigma]
%   D//{\defaultcohDegree}  %% #1 degree of anti-hol form
%   t{M}                    %% #2 with 'M' to display M in the coefficient
%   s                       %% #3 star for turning to the mlc adjoint ideal sheaf
%   D||{\defaultlcIndex}    %% #4 codim of the lcc defined by the upper ideal
%   t{.}                    %% #5 with '.' to display a quotient ideal
%   D||{#4 -1}              %% #6 codim of the lcc defined by the lower ideal
%   d()                     %% #7 the sheaf replacing the ideal sheaf if non-empty
% }{\spHbase^{#1}\IfNoValueTF{#7}{
%     \begingroup%
%     \newcommand{\upidl}{\IfBooleanTF{#3}{
%         \mtidlof{\vphi_{F \IfBooleanT{#2}{\otimes M}}}
%       }{\aidlof|#4|{\vphi_{F \IfBooleanT{#2}{\otimes M}}}}
%     }% 
%     \paren{\IfBooleanT{#2}{M\otimes}
%       \IfBooleanTF{#5}{
%         \frac{\upidl}{\aidlof|#6|{\vphi_{F \IfBooleanT{#2}{\otimes M}}}}
%       }{\upidl}}
%     \endgroup%
%   }{\paren{\IfBooleanT{#2}{M\otimes}#7}}}


\DeclareMathOperator{\lc}{lc} %% lc centre
\NewDocumentCommand{\lcc}{ %% union of lc centres
                           %% of codimension \sigma
                           %% of (X,D) %%
  D||{\defaultlcIndex}       %% #1 lc index \sigma
  e{+-}                      %% #2,#3
  D<>{\defaultAmbientSpace}  %% #4 base space
  t{'}                       %% #5 '-ed to show lc locus instead of
                             %%    lc data pair
  D(){\defaultlclocus}       %% #6 lc locus 
}{\lc_{#4}^{\lcIndex{#1}{#2}{#3}}\IfBooleanTF{#5}{\paren{#6}}{\lcData}}

\NewDocumentCommand{\lcS}{  %% a local lc centre
  s                       %% #1 symbol with \rs when starred
  D(){\defaultlclocus}    %% #2 symbol for the subvariety
  D||{\defaultlcIndex}    %% #3 codimension
  e{+-}                   %% #4,#5
  d<>                     %% #6 open set where the lc centre lives
  O{p}                    %% #7 index among the \sigma-lc centres
}{\mathtt{\IfBooleanT{#1}{\rs} #2}^{\lcIndex{#3}{#4}{#5}}_{\IfNoValueF{#6}{#6,}#7}}

\NewDocumentCommand{\PRes}{ %% Poincare Residue map
  O{}      %% subvariety
  d()      %% forms from the domain
}{\mathcal R_{#1}\IfNoValueF{#2}{\paren{#2}}}

\NewDocumentCommand{\HRes}{ %% Harmonic residue
  d()   %% #1 harmonic form
}{\mathfrak{R}\IfNoValueF{#1}{\paren{#1}}}

\newcommand{\defidlof}[1]{\mathcal{I}_{#1}}  %% defining ideal of (a set)
\NewDocumentCommand{\mtidlof}{   %% multiplier ideal of (a potential)
  O{}      %% #1 base space (for compatibility)
  D<>{#1}  %% #2 base space
  m        %% #3 potential / psh function
}{\multidl_{#2}\paren{#3}} 

% \NewDocumentCommand{\presidlof}{  %% multiplier ideal sheaf on the sum of
%                                   %% \sigma-lc centres
%   D||{\sigma}   %% codim of lc centres or supporting lc locus
%   m             %% potential or q-psh function
% }{\rs{\sheaf R}_{#1}\paren{#2}}

\NewDocumentCommand{\residlof}{  %% multiplier ideal sheaf on the
                                 %% union of \sigma-lc centres
  D||{\defaultlcIndex}   %% #1 codim of lc centres or supporting lc
                         %%    locus
  e{+-}                  %% #2,#3
  d<>                    %% #4 base space
  s                      %% #5 display the symbol without arguments when starred
  %%% input to \lcData
  % G{\defaultvphi}      %% #6 potential or q-psh function
  % O{\defaultpsi}       %% #7 lc locus psi function
  % e{.}                 %% #8 jumping number  
}{\sheaf R_{\IfNoValueTF{#4}{}{#4,} \lcIndex{#1}{#2}{#3}}\IfBooleanF{#5}{\lcData}}


\NewDocumentCommand{\Adjidlof}{
  D||{\defaultlcIndex}       %% #1 codim of lc centres under concern
  D<>{\defaultAmbientSpace}  %% #2 base space
  D(){\defaultlclocus}       %% #3 lc locus
  m                          %% #4 potential or ideal
}{\operatorname{\mathit{Adj}}^{#1}_{\paren{#2,#3}}\paren{#4}}


\NewDocumentCommand{\aidlof}{
  D||{\defaultlcIndex}   %% #1 codim of lc centres under concern
  e{+-}                  %% #2,#3
  d<>                    %% #4 base space
  s                      %% #5 display the symbol without arguments when starred
  %%% input to \lcData
  % G{\defaultvphi}        %% #6 potential or ideal
  % O{\defaultpsi}         %% #7 defining function of the lc locus
  % e{.}                   %% #8 jumping number
}{\sheaf{J}_{\!\IfNoValueTF{#4}{}{#4,} \lcIndex{#1}{#2}{#3}}\IfBooleanF{#5}{\lcData}}

\NewDocumentCommand{\faidlof}{
  D||{\defaultlcIndex}   %% #1 codim of lc centres in numerator
  e{+-}                  %% #2,#3
  t{/}                   %% #4 a separator for arguments
  D||{\defaultlcIndex}   %% #5 codim of lc centres in denominator
  e{+-}                  %% #6,#7
  % d<>                    %% #8 base space
  % s                      %% #9 display the symbol without arguments when starred
  %%% input to \lcData
  % G{\defaultvphi}        %% #10 potential or ideal
  % O{\defaultpsi}         %% #11 defining function of the lc locus
  % e{.}                   %% #12 jumping number
}{\fracAidlof{\lcIndex{#1}{#2}{#3}}{\lcIndex{#5}{#6}{#7}}}

\NewDocumentCommand{\fracAidlof}{
  m                  %% #1 lcIndex in numerator
  m                  %% #2 lcIndex in denominator
  d<>                %% #3 base space
  s                  %% #4 display the symbol without arguments when starred
  G{\defaultvphi}    %% #5 potential or ideal
  O{\defaultpsi}     %% #6 defining function of the lc locus
  e{.}               %% #7 jumping number
}{\frac{
    \aidlof|#1|<#3>*\IfBooleanF{#4}{\lcData{#5}[#6].{#7}}
  }{
    \aidlof|#2|<#3>*\IfBooleanF{#4}{\lcData{#5}[#6].{#7}}
  }}


\NewDocumentCommand{\lcV}{ %% measure on lc centres
  D||{\defaultlcIndex}    %% #1 codim of supporting lc centres
  D//{\defaultvphi}       %% #2 potential for bundle valued section
  d()                     %% #3 metric on the ambient space
  e{^}                    %% #4 jumping number
  O{\defaultpsi}          %% #5 defining function (potential) of subvariety 
}{\:d\operatorname{lcv}^{#1\IfNoValueF{#4}{,\paren{#4}}}_{\IfNoValueF{#3}{#3,}#2}\left[#5\right]}

\NewDocumentCommand{\Ohvol}{ %% Ohsawa measure %%
  D//{\defaultvphi} %% #1 potential for bundle valued section
  d()               %% #2 metric on the ambient space
  O{\defaultpsi}    %% #3 defining function of subvariety
}{\dvol_{\IfNoValueF{#2}{#2,}#1}\left[#3\right]} 


\newcommand{\dvol}{\:d\vol}


\NewDocumentCommand{\lcDataNormSubscript}{
  %% for displaying lc data in the format
  %% like "X, \vphi_L , m_k.\psi, \sigma", which is mainly used for
  %% subscript in a norm
  d<>                   %% #1 base space
  s                     %% #2 no potential and psi function when starred
  G{\defaultvphi}       %% #3 potential or q-psh function
  O{\defaultpsi}        %% #4 lc locus psi function
  e{.}                  %% #5 jumping number
  D||{\defaultlcIndex}  %% #6 lc Index
  e{+-}                 %% #7,#8
}{\IfNoValueF{#1}{#1,}
  \IfBooleanF{#2}{#3, \IfNoValueF{#5}{#5 \cdot} #4,}
  \lcIndex{#6}{#7}{#8}}


\newcommand{\RTFsym}{\mathfrak{F}} 
\NewDocumentCommand{\RTF}{ %% residue transform function
  s          %% #1 adding \smash[t] when starred
  G{\RTFsym} %% #2 symbol body
  o          %% #3 general superscript
  >{\SplitArgument{1}{,}} d<> %% #4 superscript in inner product
  d||        %% #5 superscript in \abs{}^2
  D(){\eps}  %% #6 for adding variable (\eps)
  t{,}       %% #7 separator
}{%
  \begingroup%
    \newif\ifsmasht%
    \IfBooleanTF{#1}{\smashttrue}{\smashtfalse}%
    \newif\ifboolup%
    \booluptrue%
    \IfNoValueT{#3}{\IfNoValueT{#4}{\IfNoValueT{#5}{\boolupfalse}}}%
    \newcommand{\supsrptstr}{\IfNoValueF{#3}{#3}\IfNoValueF{#4}{\inner#4}\IfNoValueF{#5}{\abs{#5}^2}}
    \newcommand{\RTFvar}{#6}
    #2\RTFprocess
}

\NewDocumentCommand{\RTFprocess}{
  o                     %% #1 overwrite subscript if given
  d<>                   %% #2 base space
  t{,}                  %% #3 with potential and psi function when ,-ed
  G{\defaultvphi}       %% #4 potential or q-psh function
  O{\defaultpsi}        %% #5 lc locus psi function
  e{.}                  %% #6 jumping number
  D||{\defaultlcIndex}  %% #7 lc Index
  e{+-}                 %% #8,#9
}{\newcommand{\subsrptstr}{%
    \IfNoValueTF{#1}{
    \IfNoValueF{#2}{#2,}
    \IfBooleanT{#3}{#4,#5,\IfNoValueF{#6}{#6,}}
    \lcIndex{#7}{#8}{#9}}{#1}}%
  \newcommand{\srptstr}{\cramped{{}^{\supsrptstr}%
      \ifboolup _
      \fi{\ifboolup\displaystyle\fi\paren{\RTFvar}%
          \ifboolup {\scriptstyle \subsrptstr} \else _{\subsrptstr} \fi%
        }}}%
  \ifboolup%
    \ifsmasht%
      \smash[t]{
        \raisebox{\depthof{$\srptstr$} * \real{0.3}}{$\srptstr$}%
      }%
    \else%
      \raisebox{\depthof{$\srptstr$} * \real{0.3}}{$\srptstr$}%
    \fi%
  \else%
    \srptstr%
  \fi%
  \endgroup%
}
% \NewDocumentCommand{\RTF}{ %% residue transform function
%   s          %% #1 adding \smash[t] when starred
%   G{\RTFsym} %% #2 symbol body
%   d//        %% #3 for adding superscript k for k-RTF
%   o          %% #4 general superscript
%   >{\SplitArgument{1}{,}} d<> %% #5 superscript in inner product
%   d||        %% #6 superscript in \abs{}^2
%   d()        %% #7 for adding variable (\eps)
%   o          %% #8 subscript for the codimension \sigma
% }{%
%   \begingroup%
%     \newif\ifboolup%
%     \booluptrue%
%     \IfNoValueT{#4}{\IfNoValueT{#5}{\IfNoValueT{#6}{\boolupfalse}}}%
%     \IfNoValueT{#7}{\boolupfalse}%
%     \newcommand{\srptstr}{\cramped{{}^{\IfNoValueF{#4}{#4}\IfNoValueF{#5}{\inner#5}\IfNoValueF{#6}{\abs{#6}^2}}%
%       \ifboolup _
%       \fi{\ifboolup\displaystyle\fi\IfNoValueF{#7}{\paren{#7}}\IfNoValueF{#8}{%
%           \ifboolup {\scriptstyle #8} \else _{#8} \fi%
%         }}}}%
%     \ifboolup%
%       \IfBooleanTF{#1}{
%         \smash[t]{
%           \IfNoValueF{#3}{{}^{#3}}#2\raisebox{\depthof{$\srptstr$} * \real{0.3}}{$\srptstr$}%
%         }%
%       }{\IfNoValueF{#3}{{}^{#3}}#2\raisebox{\depthof{$\srptstr$} * \real{0.3}}{$\srptstr$}}%
%     \else%
%       \IfNoValueF{#3}{{}^{#3}}#2\srptstr%
%     \fi%
%   \endgroup%
% } 

\def\RTI{\RTF{\mathfrak{I}}}


\NewDocumentCommand{\mtlog}{O{e} d() D||{\defaultpsi}}{\log\!#1^{\paren{#2}}\abs{#3}}
\NewDocumentCommand{\slog}{O{e} D||{\defaultpsi}}{\log\abs{#1 #2}}
\NewDocumentCommand{\dlog}{O{e} D||{\defaultpsi}}{\mtlog[#1](2)|#2|}


\NewDocumentCommand{\logpole}{ %% log-pole in the residue transform
                               %% function
  D||{\defaultpsi}       %% #1 log singularity defining function
  D//{\defaultlcIndex}   %% #2 codim of lc centres in question
  E{.^}{{e}{1+\eps}}     %% #3 multiplicative constant in logarithm 
                         %% #4 exponent in the log-psi term
  s                      %% #5 no parentheses and exponent on log|\psi| when starred
}{\abs{#1}^{#2} \IfBooleanTF{#5}{\slog[#3]|#1|}{\paren{\slog[#3]|#1|}^{#4}}}

\DeclareFontFamily{OMX}{MnSymbolE}{}
\DeclareSymbolFont{MnLargeSymbols}{OMX}{MnSymbolE}{m}{n}
\SetSymbolFont{MnLargeSymbols}{bold}{OMX}{MnSymbolE}{b}{n}
\DeclareFontShape{OMX}{MnSymbolE}{m}{n}{
    <-6>  MnSymbolE5
   <6-7>  MnSymbolE6
   <7-8>  MnSymbolE7
   <8-9>  MnSymbolE8
   <9-10> MnSymbolE9
  <10-12> MnSymbolE10
  <12->   MnSymbolE12
}{}
\DeclareFontShape{OMX}{MnSymbolE}{b}{n}{
    <-6>  MnSymbolE-Bold5
   <6-7>  MnSymbolE-Bold6
   <7-8>  MnSymbolE-Bold7
   <8-9>  MnSymbolE-Bold8
   <9-10> MnSymbolE-Bold9
  <10-12> MnSymbolE-Bold10
  <12->   MnSymbolE-Bold12
}{}
\DeclareMathDelimiter{\llangle}{\mathopen}%
{MnLargeSymbols}{'164}{MnLargeSymbols}{'164}
\DeclareMathDelimiter{\rrangle}{\mathclose}%
{MnLargeSymbols}{'171}{MnLargeSymbols}{'171}


\newcommand{\iinner}[2]{\left\llangle#1,#2\right\rrangle}
\newcommand{\eqcls}[1]{\left[#1\right]}


\NewDocumentCommand{\idxup}{ %% operator for raising indices via a
                             %% hermitian metric on X
  m                  %% #1 the differential form whose indices to be raised
  O{\defaultMetric}  %% #2 the hermitian metric on X
  t{,}               %% #3 separator
  o                  %% #4 extra superscripts
  s                  %% #5 smash the vertical spacing on the top of the metric if present
  t{.}               %% #6 with contraction operator \ctrt if '.'-ed
}{\paren{#1}^{
    % \mathrlap{
    \!\IfBooleanTF{#5}{\smash[t]{#2}}{#2}\IfNoValueF{#4}{, #4}
    % }
    % \makebox[\maxof{\widthof{$#2$}-\widthof{$\!\omega$}}{0pt}]{}
  }\IfBooleanT{#6}{\!\!\ctrt}}
% \NewDocumentCommand{\idxup}{ %% operator for raising indices via a
%                              %% hermitian metric on X
%   m                  %% #1 the differential form whose indices to be raised
%   O{*}               %% #2 the hermitian metric on X
%   s                  %% #3 smash the vertical spacing on the top of the metric if present
% }{\paren{#1}^{
%     \!\IfBooleanTF{#3}{\smash[t]{#2}}{#2}
%     % \makebox[\maxof{\widthof{$\scriptstyle #2$}-\widthof{$\!\omega$}}{0pt}]{}
%   }
% }

\newcommand{\dbadj}{\dbar^{\smash{\mathrlap{*}\;\:}}}


\NewDocumentCommand{\dep}{t{;} d<> O{\nu} m}{#4\IfBooleanTF{#1}{_}{^}{\IfNoValueF{#2}{#2\:}(#3)}}

\NewDocumentCommand{\sm}{s m}{{#2}\IfBooleanTF{#1}{_}{^}\text{sm}}

\newcommand{\tlog}{{\text{log}}}


\NewDocumentCommand{\idx}{ %% multi-indices
  O{i} %% #1 symbol of the indices
  m    %% #2 starting subscript
  o    %% #3 additional stuff to add before \dotsm
  t{.} %% #4 display "\dotsm" if '.'-ed
  t{,} %% #5 display ",\dots," if ','-ed
  o    %% #6 additional stuff to add after \dotsm
  m    %% #7 ending subscript
}{{#1}_{#2} \IfNoValueF{#3}{#3}
  \IfBooleanT{#4}{\dotsm} \IfBooleanT{#5}{,\dots,}
  \IfNoValueF{#6}{#6} {#1}_{#7}}



\newcommand{\charfct}{\mathbbm 1}


\newcommand{\cvr}[1]{\mathfrak{#1}} %% set of covering subsets
% \newcommand{\rs}[1]{\widetilde{#1}} %% putting ~ on objects on the
%                                     %% log-resolution %%
\NewDocumentCommand{\rs}{ %% putting ~ on objects on the
                          %% log-resolution %%
  s  %% when * is given, \smash[t] is applied
  m  %% the main object 
}{\IfBooleanTF{#1}{\smash[t]{\widetilde{#2}}}{\widetilde{#2}}}

% \NewDocumentCommand{\clt}{m}{\widetilde{#1}} %% element in complete space
% \NewDocumentCommand{\clomega}{O{\omega}}{{\clt{#1}}} %% complete metric

\newcommand{\BK}{\text{(BK)}}
\newcommand{\tBK}{\text{(tBK)}}
\DeclareMathOperator{\Ann}{Ann}  %% Annihilator 
\DeclareMathOperator{\mlc}{mlc} %% minimal lc centre
\DeclareMathOperator{\sym}{sym} %% symmetric polynomial
\newcommand{\Diff}{\operatorname{Diff}^*} %% general different (adjunction formula)

\newcommand{\sect}[1][s]{\mathtt{#1}} %% canonical section
\newcommand{\bphi}{\boldsymbol{\vphi}}
\newcommand{\bphip}[1][p]{\res\bphi_{#1}} %% retract-extension of
                                          %% \bphi from lc centre
                                          %% S^\sigam_p
\newcommand{\btau}{\boldsymbol{\tau}}
\newcommand{\shfP}{\sheaf P}  %% polar ideal sheaf
\NewDocumentCommand{\cbn}{  %% group of combinations
  D//{\defaultlcIndex_V}
  D||{\defaultlcIndex}
}{\mathfrak{C}^{#1}_{#2}} 
\NewDocumentCommand{\Iset}{  %% index set for lc centres on log-resolution
  D||{\defaultlcIndex}    %% #1
  e{+-}                   %% #2,#3
  O{\defaultlclocus}      %% #4
  d()                     %% #5 open set on which the index set is valid
}{I^{\lcIndex{#1}{#2}{#3}}_{#4}\IfNoValueF{#5}{\paren{#5}}} 
% }{I^{#1\IfNoValueF{#2}{+#2}\IfNoValueF{#3}{-#3}}_{#4}\IfNoValueF{#5}{\paren{#5}}} 

%%%%%%%%%%%%%%%%%%%%%%%%%%%%%%%%%%%%%%

\ifcsname defineNoThmInMarionotations\endcsname
  \relax
\else 

  \newtheorem{THMprop}{Proposition}[subsection]
  \newtheorem{THMlemma}[THMprop]{Lemma}
  \newtheorem{THMthm}[THMprop]{Theorem}
  \newtheorem{THMcor}[THMprop]{Corollary}
  % \newtheorem{SNCassumption}[THMprop]{Snc assumption}
  % \newtheorem{SNCassumptionx}{Snc assumption}
  % \renewcommand{\theSNCassumptionx}{\theSNCassumption${}^*$}

  % \newtheorem{definition-thm}[THMprop]{Definition-Theorem}

  \newtheorem{THMconjecture}[THMprop]{Conjecture}
  \newtheorem*{THMclaim}{Claim}

  \def\makeparenletter{\catcode`\(=11 \catcode`\)=11 }
  \def\makeparenother{\catcode`\(=12 \catcode`\)=12 }
  \def\makeparenactive{\catcode`\(=\active\catcode`\)=\active}

  \makeparenactive
  \NewDocumentEnvironment{textupparenenvir}{}{
    %%%%% This code may cause error when parentheses appear in places
    %%%%% where macro is not accepted, like \ref{...} or optional
    %%%%% arguments of enumerate. 
    % \catcode1=12
    % \catcode2=12
    % \mathcode1=\the\mathcode`\(
    % \delcode1=\the\delcode`\(
    % \mathcode2=\the\mathcode`\)
    % \delcode2=\the\delcode`\)

    % \begingroup
    % \lccode`\~=`\^^A
    % \lowercase{\endgroup
    % \everymath\expandafter{\the\everymath\let(^^28\let)^^29}
    % \everydisplay\expandafter{\the\everydisplay\let(^^28\let)^^29}
    % }

    \everymath\expandafter{\makeparenother}
    \everydisplay\expandafter{\makeparenother}

    \def({\textup{\char`\(}}
    \def){\textup{\char`\)}}

    \makeparenactive
    % \let\zzzlabel\label
    % \let\zzzref\ref
    % \let\zzznewlabel\newlabel

    % \def\label{\makeparenletter\wwwlabel}
    % \def\ref{\makeparenletter\wwwref}
    % \def\newlabel{\makeparenletter\wwwnewlabel}

    % \def\wwwlabel#1{\makeparenactive\zzzlabel{#1}}
    % \def\wwwref#1{\makeparenactive\zzzref{#1}}
    % \def\wwwnewlabel#1{\makeparenactive\zzznewlabel{#1}}
  }{\makeparenother}
  \makeparenother

  \NewDocumentEnvironment{prop}{ +o }{
    \IfNoValueTF{#1}{\begin{THMprop}}{\begin{THMprop}[{#1}]}
      \begin{textupparenenvir}
  }{
      \end{textupparenenvir}
    \end{THMprop}
  }

  \NewDocumentEnvironment{lemma}{ +o }{
    \IfNoValueTF{#1}{\begin{THMlemma}}{\begin{THMlemma}[{#1}]}
      \begin{textupparenenvir}
  }{
      \end{textupparenenvir}
    \end{THMlemma}
  }

  \NewDocumentEnvironment{thm}{ +o }{
    \IfNoValueTF{#1}{\begin{THMthm}}{\begin{THMthm}[{#1}]}
      \begin{textupparenenvir}
  }{
      \end{textupparenenvir}
    \end{THMthm}
  }

  \NewDocumentEnvironment{cor}{ +o }{
    \IfNoValueTF{#1}{\begin{THMcor}}{\begin{THMcor}[{#1}]}
      \begin{textupparenenvir}
  }{
      \end{textupparenenvir}
    \end{THMcor}
  }

  \NewDocumentEnvironment{conjecture}{ +o }{
    \IfNoValueTF{#1}{\begin{THMconjecture}}{\begin{THMconjecture}[{#1}]}
      \begin{textupparenenvir}
  }{
      \end{textupparenenvir}
    \end{THMconjecture}
  }

  \NewDocumentEnvironment{claim}{ +o }{
    \IfNoValueTF{#1}{\begin{THMclaim}}{\begin{THMclaim}[{#1}]}
      \begin{textupparenenvir}
  }{
      \end{textupparenenvir}
    \end{THMclaim}
  }

  \theoremstyle{remark}
  \newtheorem{remark}[THMprop]{Remark}

  \theoremstyle{definition}
  \newtheorem{definition}[THMprop]{Definition}
  \newtheorem{example}[THMprop]{Example}
  \newtheorem{notation}[THMprop]{Notation}

  \numberwithin{equation}{subsection}
  \renewcommand{\theequation}{eq$\,$\thesubsection.\arabic{equation}}


  
\fi

\allowdisplaybreaks  %% allow multi-line equations to spread across
                     %% pages 



%%% Local Variables:
%%% mode: latex
%%% TeX-master: "Injectivity-Fujino"
%%% End:

\ifx\pdftexversion\undefined
  \renewcommand{\ibar}{{\raisebox{-0.9ex}{$\mathchar'26$}\mkern-6.7mu i}}
\fi


% \usepackage{showkeys}

%%%%% End of preamble %%%%%%%%%%%%%%%%%%%%%%%%%%%%%%%%%%%%%%%%%%%%%%%%

\begin{document}

\citealias{Amb03}{Ambro_quasi-log-var}
\citealias{Amb14}{Ambro_injectivity}
\citealias{Eno90}{Enoki}
\citealias{EV92}{Esnault&Viehweg_book}
\citealias{Fuj11}{Fujino_log-MMP}
\citealias{Fuj12b}{Fujino_vanishing-thms}
\citealias{Fuj13a}{Fujino_injectivity-II}
\citealias{Fuj13b}{Fujino_injectivity-hodge-theoretic}
\citealias{Fuj15b}{Fujino_survey}



%%%%%
%%%%% File name  : titleinfo.tex
%%%%% Author     : Mario Chan
%%%%% Date       : 25th July, 202 (original: 13th December, 2021 (original: 04th November, 2020))
%%%%% Description: This file contains the info needed for maketitle
%%%%%              for the project "Injectivity-Fujino".
%%%%%
%%
%%%
\newcommand{\titlestr}{%
  % (Provisional)
  % A solution to the Fujino conjecture: injectivity theorem for
  % log-canonical pairs \\ on compact K\"ahler manifolds%
  An injectivity theorem on snc compact K\"ahler spaces: \\
  an application of the theory of
  harmonic integrals on log-canonical centers via adjoint ideal
  sheaves%
}

\newcommand{\shorttitlestr}{%
  An injectivity theorem on snc spaces%
}

\newcommand{\MCname}{Tsz On Mario Chan}
\newcommand{\MCnameshort}{Mario Chan}
\newcommand{\MCemail}{mariochan@pusan.ac.kr}

\newcommand{\YJname}{Young-Jun Choi}
\newcommand{\YJnameshort}{Young-Jun Choi}
\newcommand{\YJemail}{youngjun.choi@pusan.ac.kr}

\newcommand{\PNUAddressstr}{%
  Dept.~of Mathematics, Pusan National
  University, Busan 46241, South Korea%
}


\newcommand{\ShMname}{Shin-ichi Matsumura}
\newcommand{\ShMnameshort}{Shin-ichi Matsumura}
\newcommand{\ShMemail}{mshinichi0@gmail.com, mshinichi-math@tohoku.ac.jp}

\newcommand{\TohokuAddressstr}{%
  Mathematical Institute, Tohoku University, 6-3, Aramaki Aza-Aoba,
  Aoba-ku, Sendai 980-8578, Japan%
}


\newcommand{\subjclassstr}[1][,]{%
  32J25 (primary)#1  %% Transcendental methods of algebraic geometry (complex-analytic aspects) 
  32Q15#1   %% 	Kähler manifolds
  14B05 (secondary)%   %% Singularities in algebraic geometry
  % 14E30 (secondary)%   %% Minimal model program (Mori theory, extremal rays)
}

\newcommand{\keywordstr}[1][,]{%
  $L^2$ injectivity#1
  adjoint ideal sheaf#1
  multiplier ideal sheaf#1
  log-canonical center%
}

\newcommand{\dedicatorystr}{%
}

\newcommand{\thankstr}{%
}

%%% Local Variables:
%%% mode: latex
%%% TeX-master: "Injectivity-Fujino"
%%% coding: utf-8
%%% End:


\title[\shorttitlestr]{\titlestr}
 
\author[\MCnameshort]{\MCname}
\email{\MCemail}
% \address{\addressstr}
% \curraddr{}

\author{\YJname}
\email{\YJemail}
\address{\PNUAddressstr}

\author{\ShMname}
\email{\ShMemail}
\address{\TohokuAddressstr}


% \thanks{\thankstr}
 
\subjclass[2020]{\subjclassstr}

\keywords{\keywordstr}

% \dedicatory{\dedicatorystr}

% \begin{abstract}
%   \begin{abstract}

The Fast Reciprocal Square Root Algorithm is a well-established approximation technique consisting of two stages: first, a coarse approximation is obtained by manipulating the bit pattern of the floating point argument using integer instructions, and second, the coarse result is refined through one or more steps, traditionally using Newtonian iteration but alternatively using improved expressions with carefully chosen numerical constants found by other authors. The algorithm was widely used before microprocessors carried built-in hardware support for computing reciprocal square roots. At the time of writing, however, there is in general no hardware acceleration for computing other fixed fractional powers. This paper generalises the algorithm to cater to all rational powers, and to support any polynomial degree(s) in the refinement step(s), and under the assumption of unlimited floating point precision provides a procedure which automatically constructs provably optimal constants in all of these cases. It is also shown that, under certain assumptions, the use of monic refinement polynomials yields results which are much better placed with respect to the cost/accuracy tradeoff than those obtained using general polynomials. Further extensions are also analysed, and several new best approximations are given.

\end{abstract}

% \end{abstract} 

%%%SM's notation: I will change them to Mario's notation later 
% \newcommand{\Ker}[0]{\operatorname{Ker}}
\let\Ker\ker
\newtheorem{step}{Step}
%%%




%%choiyj's macros
\def\del{\partial}
\def\we{\wedge}
\def\ov{\overline}
\newcommand{\pd}[2]{\frac{\partial#1}{\partial#2}}
%%





\date{\today} 

\maketitle

%%%%% End of Top matter %%%%%%%%%%


\section{Introduction}\label{sec:intro}

{
  \let\thesubsection\thesection
  
  % This paper studies an analytic aspect of higher cohomology groups of adjoint bundles for lc $($log canonical$)$ pairs
  % aiming to solve Fujino's conjecture on the injectivity theorem as a benchmark. 
  This paper studies an analytic aspect of higher cohomology groups of adjoint bundles
  for log-canonical (lc) pairs aiming to solve Fujino's conjecture, 
  the injectivity theorem for lc pairs on compact K\"ahler manifolds, 
  following the line of Enoki's proof. 
  This is achieved by developing the theory of harmonic integrals
  on lc centers using the analytic adjoint ideal sheaves and the
  associated residue techniques.


  The injectivity theorem, a generalization of the Kodaira vanishing theorem to semi-positive line bundles, 
  plays an important role in higher dimensional algebraic geometry. 
  After the original Koll\'ar's injectivity theorem \cite{Kollar_injectivity} had been proved 
  for semi-ample line bundles on smooth projective varieties, 
  Enoki \cite{Eno90} generalized Koll\'ar's injectivity theorem 
  to semi-positive line bundles on compact K\"ahler manifolds. 
  Koll\'ar's proof is based on theory of Hodge structures, whereas
  Enoki's proof is based on the theory of harmonic integrals, a more
  well-suited and flexible technique in the complex analytic situation. 

  Ambro and Fujino generalized Koll\'ar's theory to varieties with lc
  singularities via the theory of mixed Hodge structures,  
  motivated by applications to birational geometry (see \cite{Amb03, Amb14, EV92, Fuj11, Fuj12b, Fuj13b}). 
  % \mariocomment{To SM: please
  % check if these references links to the correct papers. Change the
  % $\backslash$\texttt{citealias} commands if you wish.}% 
  % The works of Ambro and Fujino can be expected to
  It is expected that their works can also be generalized in the same line as Enoki's
  by developing an analytic treatment to lc singularities. 
  Motivated by this expectation, Fujino posed the conjecture below. 
  (Set $\ibar := \ibardefn$ \ibarfootnote\ and let $D$ be a reduced divisor for the
  rest of this article.)



  \begin{conjecture}[{Fujino's conjecture, \cite[Conjecture
      2.21]{Fuj15b}, cf.~\cite[Problem 1.8]{Fuj13a}}] 
    \label{conj:fujino}

    Let $X$ be a compact K\"ahler manifold and
    $D=\sum_{i=1}^{N}D_{i}$ be a simple-normal-crossing
    (snc) divisor on $X$.  Let $F$ be a semi-positive line bundle on
    $X$ (i.e.~it admits a smooth Hermitian metric $h_{F}$ with
    $\ibar\Theta_{h_F}(F) \geq 0$).  Consider a section
    $s \in H^{0}(X, F^{\otimes m})$ whose zero locus $s^{-1}(0)$
    contains no lc centers of the pair $(X,D)$ (i.e.~connected
    components of non-empty intersection
    $D_{i_{1}}\cap \cdots \cap D_{i_{k}}$ of the irreducible
    components $\{D_{i}\}_{i=1}^{N}$).  Then, the multiplication map
    induced by the tensor product with $s$
    \begin{equation*}
      H^{q}\paren{X, K_{X} \otimes D \otimes F}
      \xrightarrow{\otimes s} 
      H^{q}(X, K_{X} \otimes D \otimes F^{\otimes (m+1)} )
    \end{equation*}
    is injective for every $q$.
  \end{conjecture}

  The analytic theory corresponding to Koll\'ar's theory has been established for klt singularities 
  (see \cite{Cao&Demailly&Matsumura, Fujino&Matsumura, Gongyo&Matsumura,
    Matsumura_injectivity-survey, Matsumura_injectivity}).
  % but not for lc singularities. 
  Therefore, it remains to develop an analytic treatment to handle the
  lc singularities.
  % the analytic theory corresponding to the works of Ambro and Fujino
  % is interesting in terms of  studying the techniques of analytically treating lc singularities or mixed Hodge theory than just generalizing it. 


  The cases of $\dim X=2$ and plt pairs of arbitrary dimension have been
  solved in \cite{Matsumura_injectivity-lc,
    Matsumura_rel-vanishing-w-nd} (see also \cite{Chan&Choi_injectivity-I}). 
  A full solution to Fujino's conjecture is given recently by
  Junyan Cao and Mihai P\u{a}un \cite{Cao&Paun_LC-inj}.
  In this paper, independent of the results in \cite{Cao&Paun_LC-inj},
  we prove a {\textit{generalized version}} of Fujino's conjecture  
  (Theorem \ref{thm:main}) 
  % by developing the theory of harmonic integrals on simple normal corssing divisors. 
  by applying the theory of harmonic integrals on lc centers of the
  given lc pair.
  Fujino's conjecture is then a direct consequence of Theorem \ref{thm:main}. 
  

  \begin{thm}[Main Result]\label{thm:main}
    Let $X$ be a compact K\"ahler manifold  and 
    $D=\sum_{i=1}^{N}D_{i}$ be an snc divisor on $X$. 
    % such that each component $D_{i}$ is compact. 
    Let $F$ (resp.~$M$) be a line bundle on $X$ 
    with a smooth Hermitian metric $h_{F}$  (resp.~$h_{M}$) 
    such that 
    \begin{equation*}
      \ibar\Theta_{h_F}(F)\geq 0 \quad  \text{ and } \quad
      % \sqrt{-1}(\Theta_{h_F}(F)-t \Theta 
      % _{h_M}(M))\geq 0
      % -C\omega \leq
      \ibar\Theta_{h_M}(M) \leq C \ibar\Theta_{h_F}(F)
      \quad \text{ for some } C>0 \; . 
    \end{equation*}
    % (that is, $D_{i} \cap D_{j} = \emptyset$ for $i \not = j$ 
    % for the irreducible decomposition $D = \sum_{i\in I}D_{i}$). 
    Let $s$ be a  section of $M$  
    such that the zero locus $s^{-1}(0)$ 
    contains no lc centers of the lc pair $(X,D)$.
    Then, the multiplication map induced by the tensor product with $s$
    \begin{equation*}
      H^q(D, K_D \otimes F)
      \xrightarrow{\otimes s } 
      H^q(D, K_D \otimes F\otimes M)
    \end{equation*} 
    is injective for every $q$. 
  \end{thm}

  It can be seen from the proof that the compactness of $X$ in Theorem
  \ref{thm:main} is not necessary as soon as $D$ consists of only finitely many
  irreducible components which are compact.

  \begin{cor}[Solution to Fujino's conjecture]\label{cor:main}
    Conjecture \ref{conj:fujino} is true. 
  \end{cor}


  % Our paper differs from \cite{Cao&Paun_LC-inj} in the following points: 
  % The method of \cite{Cao&Paun_LC-inj} is based on the $L^{2}$-theory of $\dbar$-equations, 
  % whereas our method is based on the theory of harmonic integrals in the same line as in Enoki's work; 
  % specifically, we extend a technique of harmonic differential forms on smooth varieties to simple normal corssing divisors.

  Our proof differs from the one in \cite{Cao&Paun_LC-inj} in the following way.
  While both works make use of (some variant of) the Hodge
  decomposition for $L^2$ forms, Cao and P\u aun prove in
  \cite{Cao&Paun_LC-inj} a Hodge decomposition for $L^2$ forms with
  respect to a K\"ahler metric with conic singularities, which induces
  a Hodge decomposition on currents (which is called the Kodaira--de
  Rham decomposition in \cite{Cao&Paun_LC-inj}) in which the Green
  kernel has controllable singularities.

  For the sake of explanation, let $u$ be an $D\otimes F$-valued
  $(n,q)$-form representing a class in $\cohgp q[X]{\logKX}$
  % ($X$ being compact here)
  such that the class of $s u$ is $0$ in $\cohgp
  q[X]{\logKX M}$.
  Let also $\sect_D$ be a canonical section of $D$.
  Under our notation, the current that is under consideration in
  \cite{Cao&Paun_LC-inj} is $\frac{u}{\sect_D}$, which is not
  necessarily $L^2$ on $X$.
  Using the fact that $\eqcls{su} = 0$ in $\cohgp q[X]{\logKX M}$,
  Cao and P\u aun obtain $\frac{u}{\sect_D} =\dbar\theta + D'_{h_F}
  \beta_1 +\ibar\Theta_{h_F} \wedge \beta_2$, where $\theta$ is
  smooth while $\beta_1$ and $\beta_2$ have log-poles along
  $D+s^{-1}(0)$ (assumed to have only snc).
  It then follows from \cite{Cao&Paun_LC-inj}*{Thm.~1.1} (which
  makes use of the Hodge/Kodaira--de Rham decomposition) and the
  positivity $\ibar\Theta_{h_F} \geq 0$ that $u$ (or $u -\sect_D
  \dbar\theta$) is $\dbar$-exact.

  In our case, we make use of the residue exact sequences of adjoint
  ideal sheaves and the associated residue computation to reduce the
  setup to the union of \emph{$\sigma$-lc centers} of $(X,D)$ (i.e.~lc centers of
  codimension $\sigma$ in $X$, when $(X,D)$ is log-smooth and lc).
  Since each $\sigma$-lc center is a compact K\"ahler manifold, we
  have the Hodge decomposition (thus $L^2$ Dolbeault isomorphism and
  harmonic theory) at our disposal.
  Moreover, our reduction brings the setup essentially to the one in
  \cite{Matsumura_injectivity-lc}*{Thm.~1.6} or
  \cite{Chan&Choi_injectivity-I}*{Thm.~1.2.1} (corresponding to the
  case where $\frac{u}{\sect_D}$ is $L^2$).
  That's why we can follow the line of arguments in Enoki's proof to
  solve the conjecture via the theory of harmonic integrals on lc
  centers (and no extra resolution to bring $s^{-1}(0)$ into snc is
  needed).

  % Thanks to this advantage, we can obtain the generalized version  (not only Fujino's conjecture), 
  This approach gives us the advantage of obtaining Theorem
  \ref{thm:main}, a generalized version of Fujino's conjecture (see
  also Remark \ref{rem:general-commut-diagram} for other generalized
  statements which can be achieved),
  which does not seem to be derivable from results in \cite{Cao&Paun_LC-inj}, at
  least not directly. 
  % Furthermore, the previous works (including \cite{Cao&Paun_LC-inj, Amb03, Amb14, Fuj11} 
  % used the the assumption that $s^{-1}(0)$ contains no lc centers
  % to reduce the proof to the case where the pair $(X,D + s^{-1}(0))$ is log smooth; 
  % however, our paper uses this assumption to apply the inductive argument in terms of lc strata
  % by  the fact that all the data restricted to each component $D_{i}$ satisfy the assumption again. 



  Here we briefly explain the outline of the  proof of Theorem
  \ref{thm:main} with the example where the snc divisor $D$ has
  only two components $D_1$ and $D_2$ such that $D_1 \cap D_2$ is
  irreducible as an illustration.
  In this case, the union of the $1$-lc centers of $(X,D)$ is
  $\lcc|1|' = D_1 \cup D_2$ while that of the $2$-lc centers is
  $\lcc|2|' = D_1 \cap D_2$.
  For any given cohomology class $\alpha \in H^q(D,  K_{D} \otimes F)$
  such that $s  \alpha =0$ in $H^q(D,  K_{D} \otimes F \otimes M)$, 
  the goal is to show that $\alpha$ is actually $0$. 


  Write $h_F = e^{-\vphi_F}$ and $h_M = e^{-\vphi_M}$, and let $\psi_D
  := \phi_D -\sm\vphi_D :=\log\abs{\sect_D}^2 -\sm\vphi_D$ be a global
  function on $X$ such that $\phi_D$ is the (local) potential (of the
  curvature of a metric) on $D$ induced from a canonical section
  $\sect_D$ and $\sm\vphi_D$ is some smooth potential on $D$.
  When $D$ is smooth (i.e.~$D_{1}\cap D_{2}=\emptyset$), 
  the class $\alpha $ can be represented by $(u_{1},  u_{2})$, where $u_i$
  is a harmonic form with respect to $\vphi_F$ on $D_i$ in
  $\mathcal{H}^{n-1,q}(D_{i}; F)_{\vphi_{F}} \cong H^{q}(D_{i},
  K_{D_{i}} \otimes F)$ for $i=1,2$.
  Enoki's argument \cite{Eno90} shows that $s u_{i}$ is also a harmonic
  form with respect to $\vphi_F +\vphi_M$ using the
  Bochner--Kodaira--Nakano formula and the given curvature assumption.
  % by the Bochner trick and the assumption of curvatures.
  It follows from $s \alpha =0$ (as a class) that $s u_{i}=0$ (as a form), hence
  $\alpha=(u_{1}, u_{2})=0$ as desired. 

  However, when $D =\lcc|1|'$ (as well as other $\lcc'$ in the more
  general situation) is not smooth (i.e.~$D_{1}\cap D_{2} \neq
  \emptyset$), the Dolbeault and harmonic theories for cohomology groups
  on $D$ are not yet established, obstructing the use of Enoki's
  argument.
  To overcome this difficulty, we make use of the short exact sequence
  \begin{equation*}
    \xymatrix{
      {0} \ar[r]
      & {\bigoplus_{i=1}^2 K_{D_{i}} \otimes \res F_{D_i}} \ar[r]^-{\tau}
      & {K_{D} \otimes \res F_{D}} \ar[r]
      & {K_{D_{1} \cap D_{2}} \otimes \res F_{D_1 \cap D_2}} \ar[r]
      & {0}
    } \; ,
  \end{equation*}
  where $K_{D} :=K_{X}\otimes D \otimes \frac{\holo_X}{\defidlof{D}}$
  and $\defidlof{D}$ is the defining ideal sheaf of $D$ in $X$, and its
  associated long exact sequence of cohomology groups to reduce our
  injectivity problem of the map $\otimes s$ on $D$ to the injectivity
  problems of $\otimes s$ on the lc centers of $(X,D)$ (i.e.~$D_1$,
  $D_2$ and $D_1 \cap D_2$). 
  Note that all of the lc centers are not contained in $s^{-1}(0)$ by
  assumption and are compact K\"ahler manifolds on which the Dolbeault
  isomorphism and harmonic theory are available.

  Such strategy is suggested already in \cite{Matsumura_injectivity-lc}
  and is used there in the proof of the injectivity theorem for plt
  pairs.
  It is framed in \cite{Chan&Choi_injectivity-I} in terms of the adjoint
  ideal sheaves $\aidlof* := \aidlof = \mtidlof<X>{\vphi_F} \cdot
  \defidlof{\lcc+1'} = \defidlof{\lcc+1'}$ (the defining ideal sheaf
  of $\lcc+1'$ in $X$, under the assumption $\vphi_F$ being smooth)
  for integers $\sigma \geq 0$ and the corresponding residue morphisms
  $\Res^\sigma$ for $\sigma \geq 1$ (see Section \ref{subsec:residue}
  for the definitions).
  Writing $\lcc' = \bigcup_{p \in \Iset} \lcS$ as the decomposition of
  $\lcc'$ into the (irreducible) $\sigma$-lc centers $\lcS$, the residue
  morphism $\Res^\sigma$ induces the isomorphism
  \begin{equation*}
    \logKX \otimes \faidlof/-1* \xrightarrow[\isom]{\Res^\sigma}
    \logKX \otimes \residlof* := \bigoplus_{p \in \Iset} K_{\lcS}
    \otimes \res F_{\lcS} 
  \end{equation*}
  (notice that $\logKX \otimes \frac{\defidlof{D_1 \cap
      D_2}}{\defidlof{D}} \isom \bigoplus_{i=1}^2 K_{D_{i}} \otimes \res
  F_{D_i}$ and $\logKX \otimes \frac{\holo_X}{\defidlof{D_1 \cap D_2}}
  \isom K_{D_{1} \cap D_{2}} \otimes \res F_{D_1 \cap D_2}$ in the
  example).
  It can then be seen that, for more general $D$, the reduction can be
  done via the short exact sequences $0 \to \faidlof/-1* \to
  \faidlof|\sigma'|/-1* \to \faidlof|\sigma'|/* \to 0$ for some integers
  $\sigma$ and $\sigma'$ such that $1 \leq \sigma \leq \sigma'$,
  together with an induction on $\sigma$ via some diagram-chasing
  argument.
  See Step \ref{step:harmonic-rep} of Section
  \ref{sec:proof-of-simple-case} and the beginning of Section
  \ref{subsec:general} for precise details.


  After the reduction, we are led to consider the maps
  \begin{equation*}
    \renewcommand{\objectstyle}{\displaystyle}
    \xymatrix{
      {\smash{\bigoplus_{i =1}^2}\:\cohgp q[D_i]{K_{D_i} \otimes
          F}} \ar[r]^-{\tau}
      \ar[dr]^-{\nu}
      &{\cohgp q[D]{K_D \otimes F}}
      \ar[d]^-{\otimes s} \ar@{}@<-1em>[d]_*+{\circlearrowright}
      \\
      &{\cohgp q[D]{K_D \otimes F \otimes M} \; .}
    }
  \end{equation*}
  It suffices to prove that $\ker\nu =\ker\tau$ (Theorem
  \ref{thm:ker-nu=ker-tau}).
  Indeed, given the injectivity of the map $\otimes \res s_{D_1 \cap D_2}$ on
  $\cohgp q[D_1 \cap D_2]{K_{D_1 \cap D_2} \otimes F}$ followed from
  Enoki's argument in the previous case, we see that the given class
  $\alpha \in \cohgp q[D]{K_D \otimes F}$ actually lies in the image
  $\im\tau$ of $\tau$, say, $\alpha = \tau\paren{u_1, u_2}$ for some
  harmonic forms $u_i \in \Harm'/n-1,q/<D_i>{F},{\vphi_F} \isom \cohgp
  q[D_i]{K_{D_i} \otimes F}$.
  It will then follow that $\paren{u_1, u_2} \in \ker\nu
  =\ker\tau$, hence $\alpha =0$, as desired.
  The pair $(u_1,u_2)$ can be treated as a representative of $\alpha$.
  Suggested by the fact that a harmonic form is the unique
  representative with the \emph{minimal} $L^2$ norm among all elements
  in its corresponding $L^2$ Dolbeault cohomology class, we can choose
  an ``optimal'' representative of $\alpha$ such that $(u_1, u_2)$ has
  the \emph{minimal} distance from (i.e.~is orthogonal to) the subspace
  $\ker\tau$ with respect to the $L^2$ norm induced from $\vphi_F$.
  It then suffices to show that $u_i =0$ for $i = 1,2$ to prove that
  $\ker\nu =\ker\tau$.
  This is done by following the proof of
  \cite{Matsumura_injectivity-lc}*{Thm.~1.6} or
  \cite{Chan&Choi_injectivity-I}*{Thm.~1.2.1} (therefore following the
  spirit of Enoki's argument), but with a few technical modifications.

  One technical complication comes from the use of \v Cech cohomology
  for some cohomology groups (e.g.~$\cohgp q[D]{K_D \otimes F}$) due to
  the lack of the Dolbeault isomorphism.
  Another one is that the argument of Takegoshi in
  \cite{Chan&Choi_injectivity-I}*{\S 3.1, Step IV} (see also
  \cite{Matsumura_injectivity-lc}*{Prop.~3.13}), which essentially gives
  rise to an element in $\ker\tau$ constructed from $u_i$'s, is replaced
  by a construction of a harmonic forms $w$ (or a collection $w
  :=\paren{w_b}_{b\in \Iset+1}$ of harmonic forms for the general $D$)
  representing a class in $\cohgp{q-1}[D_1 \cap D_2]{K_{D_1 \cap D_2}
    \otimes F}$ (see \eqref{eq:w-prelim-formula} and \eqref{eq-def-w}).
  The class of $w$ has its image lying in $\ker\tau$ via the connecting
  morphism of the relevant long exact sequence.
  Such construction is suggested by a residue computation, which relates
  an inner product on (the normalization of) $\lcc|1|'$ to an inner
  product on (lower dimensional) $\lcc|2|'$ (see Proposition
  \ref{prop:res-formula-dbar-exact-dot-harmonic}; see also Steps
  \ref{item:express-su-in-residue-norm} and \ref{item:pf:use_u-ortho-w}
  in Section \ref{subsec:general}, or Steps
  \ref{item:expression-of-su-simple} and
  \ref{step:pf:use_u-ortho-w-simple} in Section
  \ref{sec:proof-of-simple-case} for less intensive notation).
  Such relation between the inner products shows that $w$ is the
  obstruction for having $u_i = 0$ for $i=1,2$.
  This becomes the crucial ingredient to complete the proof.

  The proof of Theorem \ref{thm:main} for the case of general $D$
  follows the same arguments.
  A brief comment for the case where $\vphi_F$ and $\vphi_M$ possess
  suitable analytic singularities is given in Remarks
  \ref{rem:singular-vphi_F} and \ref{rem:no-hard-Lefschetz}.
  
  







  % We briefly explain the proof in the simple case where $D$ has two components (i.e.,\,$D=D_{1}+D_{2}$). 
  % For a given cohomology class $\alpha \in H^q(D,  K_{D} \otimes F)$, 
  % we will prove that $\alpha $ is actually zero 
  % under assuming that $s  \alpha =0 \in H^q(D,  K_{D} \otimes F \otimes M)$. 

  % In the case where $D$ is smooth (i.e.,\,$D_{1}\cap D_{2}=\emptyset$), 
  % the class $\alpha $ can be represented by a harmonic form $(u_{1},  u_{2})$. 
  % Here we used 
  % $$
  % H^q(D,  K_{D} \otimes F)=\oplus_{i=1}^{2} H^{q}(D_{i},  K_{D_{i}} \otimes F) \cong 
  % \oplus_{i=1}^{2} \mathcal{H}^{n-1,q}(D_{i}, F)_{h_{F}}, 
  % $$ 
  % where $\mathcal{H}^{n-1,q}(D_{i}, F)_{h_{F}}$ is the space of harmonic forms with respect to $h_{F}$. 
  % Enoki's argument \cite{Eno90} shows that  
  % $s u_{i}$ is also a harmonic form with respect to $h_{F} h_{M}$
  % by the Bochner trick and the assumption of curvatures. 
  % This implies that $s u_{i}=0$ by $s \alpha =0$; hence $\alpha=\{(u_{1}, u_{2})\}=0$. 

  % In the general case  (i.e.,\,$D_{1}\cap D_{2} \not =\emptyset$), 
  % it is not clear whether the class $\alpha$ can be represented by a harmonic form on $D$,  
  % which is an obvious  difficulty in extending Enoki's argument.
  % To overcome this difficulty, we consider the long exact sequence 
  % $$
  % \cdots \to\bigoplus_{i=1}^{2} H^q(D_{i},  K_{D_{i}}\otimes F ) \xrightarrow{\tau} H^q(D,  K_{D} \otimes F)  \to H^q(D_{1}\cap D_{2},  K_{D_{1}\cap D_{2}}\otimes F ) \to \cdots
  % $$
  % induced by $0 \to K_{D_{1}} \oplus K_{D_{2}} \to K_{D} \to K_{D_{1} \cap D_{2}} \to 0$, 
  % noting that $ K_{D}=(K_{X}\otimes D)\otimes \mathcal{O}_{X}/\mathcal{I}_{D}$. 
  % The multiplication map defined on the right term 
  % $$
  % \otimes s |_{D_{1} \cap D_{2}}: H^q(D_{1}\cap D_{2},  K_{D_{1}\cap D_{2}}\otimes F ) \to
  % H^q(D_{1}\cap D_{2},  K_{D_{1}\cap D_{2}}\otimes F \otimes M ) 
  % $$ 
  % is non-zero since $s^{-1}(0)$ contains no lc centers of the pair $(X,D)$, 
  % and thus injective by induction hypothesis. 
  % Thus, by chasing the commutative diagram induced by the multiplication map, 
  % we can take a cohomology class 
  % $\beta \in \oplus_{i=1}^{2} H^q(D_{i},  K_{D_{i}}\otimes F )$ such that $\tau(\beta)=\alpha$. 
  % Then, we can take a harmonic representation $(u_{1}, u_{2})$ of $\beta$. 


  % The pair $(u_{1}, u_{2})$ is the {\textit{best}} representation for $\beta$ 
  % in the sense that $(u_{1}, u_{2})$ has the minimum $L^2$ norm in the forms representing $\beta$. 
  % However, the pair $(u_{1}, u_{2})$ may not be the best representation for $\alpha$ 
  % since the $L^2$ norm may be reduced by $\tau$. 
  % For this reason, by the orthogonal decomposition, 
  % we re-choose $\beta$ (and its harmonic representation $(u_{1}, u_{2})$) 
  % satisfying $(u_{1}, u_{2}) \in (\Ker \tau)^{\perp} \subset \oplus_{i=1}^{2} \mathcal{H}^{n-1,q}(D_{i}, F)_{h_{F}}$. 
  % The condition $(u_{1}, u_{2}) \in (\Ker \tau)^{\perp}$ means a certain minimal $L^{2}$-norm; 
  % therefore $(u_{1}, u_{2})$ can be seen as the best representation for $\alpha$. 


  % The Bocher trick shows that the $L^2$ norm of $(u_{1}, u_{2})$ is zero in the case $D_{1}\cap D_{2}=\emptyset$. 
  % By generalizing this Bocher trick, 
  % an obstruction for the $L^{2}$-norm to be $0$ 
  % can be described by a $F$-valued differential form $w$ on $D_{1} \cap D_{2}$. 
  % An important point here is that  $w$ is actually harmonic; in particular, it determines the cohomology class.   
  % $H^{q-1}(D_{1}\cap D_{2},  K_{D_{1}\cap D_{2}}\otimes F )$. 
  % Then, we can show that the $L^{2}$-norm of $w$ on $D_{1} \cap D_{2}$ 
  % is equal to the inner product of $(u_{1}, u_{2})$ and a representation of $\delta(w)$, 
  % where $\delta$ is the connecting morphism 
  % $\delta: H^{q-1}(D_{1}\cap D_{2},  K_{D_{1}\cap D_{2}}\otimes F ) \to \oplus_{i=1}^{2} H^{q}(D_{i},  K_{D_{i}} \otimes F)$. 
  % This implies that $w=0$ by $(u_{1}, u_{2}) \in (\Ker \tau)^{\perp}$.


  % 
  % ....


}

This paper is organized as follows.
\tableofcontents


\subsection*{Acknowledgments}
The authors would like to thank the members of Bayreuth University and Pusan National University for their hospitality.
This paper is resulted from the discussions there. 
S.M.~would like to thank Professors Junyan Cao and Mihai P\u{a}un for sharing a preliminary version of \cite{Cao&Paun_LC-inj}.
Also, he would like to thank Professor Osamu Fujino 
for his encouragement and long-standing discussions on lc singularities. 
He is partially supported 
by Grant-in-Aid for Scientific Research (B) $\sharp$21H00976 
and Fostering Joint International Research (A) $\sharp$19KK0342 from
JSPS.
Y.C.~and M.C.~would like to thank S.M.~for drawing their attention to
Fujino's conjecture not long before the covid pandemic (which results
in \cite{Chan&Choi_injectivity-I}) and for joining hand to complete
this project when most aspects of life went back to normal.
Y.C.~and M.C.~were supported by the National Research Foundation
of Korea (NRF) Grant funded by the Korean government
(Nos.~2023R1A2C1007227 and 2021R1A4A1032418).



\section{Preliminary results}\label{sec:preliminaries}

\subsection{Notation and conventions}\label{subsec:notation}

%%%%%
%%%%% File name  : notation.tex
%%%%% Author     : Mario Chan
%%%%% Date       : 13th December, 2021 (original: 04th November, 2020)
%%%%% Description: This is the section "Notation" in the project
%%%%%              "Injectivity-Fujino".
%%%%%
%%
%%%

% In this subsection, we summarize the notation used throughout this paper. 

The following notions are used throughout this paper unless stated otherwise. 
\begin{itemize}
\item $(X,\omega)$ is a compact K\"ahler manifold of dimension $n$. 

% \item $\omega$ is a K\"ahler form on $X$. 

\item $h_F := e^{-\vphi_F}$ and $h_M := e^{-\vphi_M}$, where $\vphi_F$ and
  $\vphi_M$ are respectively the given potentials on $F$ and $M$.
  
\item $D=\sum_{i \in \Iset||}D_{i}$ is a reduced simple-normal-crossing (snc)
  divisor on $X$ (where $\Iset||$ is a finite set). 

\item $\sect_i$ is a canonical section  of the irreducible component $D_{i}$. 

\item $\sect_D := \prod_{i\in \Iset||} \sect_i$ is the canonical section of $D$. 

\item $\sigma \in \{0,1,2,\cdots, n\}$.

% \item  $\Iset$ is the set of $p:=\{i_{1}, i_{2}, \cdots, i_{\sigma}\}$ such that  
% \mmark{$\lcS:=\cap_{k=1}^{\sigma} D_{i_{k}} $}{$\cap D_{i_k}$ may have more than
% one component.} is of codimension $\sigma$. 

% \item   $\lcc' := \cup_{p \in \Iset} \lcS$ is the union of $\sigma$-lc centers $\lcS$ of  $(X,D)$

  
\item $\lcc' :=\bigcup_{p \in \Iset} \lcS$ is the union of
  \emph{$\sigma$-lc centers of $(X,D)$}, i.e.~the
  $\sigma$-codimensional irreducible components of any intersections
  of irreducible components of $D$ (under the assumption $(X,D)$ being
  log-smooth and lc), indexed by $\Iset$.
  Set $\lcc|0|' := X$ and let $\Iset|0|$ be a singleton for convenience.
  Note also that $\Iset|1| = \Iset||$.

\item $\Diff_{p}D$ is the effective divisor on $\lcS$ defined by the 
adjunction formula 
\begin{equation*}
  K_{\lcS} \otimes \Diff_{p}D = \parres{K_X \otimes D}_{\lcS}
\end{equation*}
such that the restriction of $\sect_{(p)}:=
\smashoperator{\prod\limits_{i \in \Iset|| \colon D_i
    \not\supset \lcS}} \sect_i $ to $\lcS$ is a canonical section of
$\Diff_{p}D$.

\item $\phi_D :=\log\abs{\sect_D}^2$ and $\phi_{(p)}
  :=\log\abs{\sect_{(p)}}^2$ are the potentials induced from the
  canonical sections of $D$ and $\Diff_p D$.

\item $\cvr V := \{V_{i}\}_{i \in I}$ is an open cover of $X$  by admissible open sets. 

\item $\{\rho^{i}\}_{i\in I}$ is a partition of unity subordinate to
  $\cvr V$. 
\end{itemize}

Here an open set $V \subset X$ is said to be \emph{admissible} with
respect to $D$ if $V$ is biholomorphic to a polydisc centered at the
origin under a holomorphic coordinate system $(z_{1}, z_{2}, \cdots,
z_{n})$ such that
\begin{equation*} % \label{eq:local-expression-bphi-psi}
  D =\set{z_1 \dotsm z_{\sigma_V} =0}, \quad 
  \log r_{j}^2 < 0, \quad \text{and }
  r_j \fdiff{r_j} \psi_D >0 \text{ on } V \; , 
  % \res{\vphi_\bullet}_V = \smashoperator{\sum_{k=\sigma_V+1}^n} b_{\bullet,k}
  % \log\abs{z_k}^2 +\beta_\bullet \;\;\text{ for } \bullet= F, M \; ,
\end{equation*} 
where  $r_j := \abs{z_j}$  and $\res{\psi_D}_V := \parres{\phi_D
  -\sm\vphi_D}_V =\sum_{j=1}^{\sigma_V} \log\abs{z_j}^2
-\res{\sm\vphi_D}_V$. 

When an admissible set $V$ is considered, an index $p \in \Iset$ such
that $\lcS \cap V \neq \emptyset$ is interpreted as a permutation
representing a choice of $\sigma$ elements from the set
$\set{1,2,\dots,\sigma_V}$ such that
\begin{equation*}
  \lcS \cap V = \set{z_{p(1)} = z_{p(2)} = \dotsm = z_{p(\sigma)} = 0}
  \quad\text{ and }\quad
  \res{\sect_{(p)}}_V = z_{p(\sigma+1)} \dotsm z_{p(\sigma_V)}
\end{equation*}
(cf.~the definition of the set $\cbn$ in \cite{Chan_adjoint-ideal-nas}*{\S 3.1}).



%%% Local Variables:
%%% mode: latex
%%% TeX-master: "Injectivity-Fujino"
%%% coding: utf-8
%%% End:


%\subfile{commut-diagram_Fujino-conj}%

\subsection{$L^{2}$ Dolbeault isomorphism and some results on harmonic
forms}\label{subsec:l2}

%%%%%
%%%%% File name  : L2-spaces-n-harmonic-forms.tex
%%%%% Author     : Mario Chan
%%%%% Date       : 27th March, 2023
%%%%% Description: This is the section on the basic facts of the Hodge
%%%%%              decomposition and the implications of positivity on
%%%%%              harmonic forms which have been discussed in
%%%%%              previous papers.
%%%%%
%%
%%%

{
  \setDefaultvphi{\vphi_L}

  % Suppose that $X$ is \emph{compact} K\"ahler in this section.
  Let $L$ be a holomorphic line bundle on $X$ equipped with a
  (possibly singular) quasi-psh potential $\vphi_L$, which induces,
  together with the K\"ahler form $\omega$, an $L^2$ norm
  $\norm{\cdot}_{X} := \norm\cdot_{X,\vphi_L,\omega}$ on the space of
  smooth $K_X \otimes L$-valued $(0,q)$-forms (or $L$-valued
  $(n,q)$-forms) on $X$.
  Let $\Ltwo/n,q/{L}_{\vphi_L}$ be the completion with respect to $\norm\cdot_X$
  and $\Harm :=\Harm{L}$ be the space of harmonic forms with respect to $\norm\cdot_X$.
  The $L^2$ Dolbeault isomorphism (see
  \cite{Matsumura_injectivity}*{Prop.~5.5 and 5.8} and
  \cite{Matsumura_injectivity-lc}*{Prop.~2.8} for a proof, and see 
  \cite{Chan&Choi_injectivity-I}*{footnote 1} for its naming) guarantees
  the closedness of the subspaces in the orthogonal decomposition
  \begin{equation*}
    \Ltwo/n,q/{L}_{\vphi_L}
    = \Harm \oplus \cl{\paren{\im\dbar}}_{\vphi_L} \oplus \cl{\paren{\im\dbadj}}_{\vphi_L}
    = \Harm \oplus \paren{\im\dbar}_{\vphi_L} \oplus \paren{\im\dbadj}_{\vphi_L} 
  \end{equation*}
  (where $\dbadj$ is the Hilbert space adjoint of $\dbar$ with respect
  to $\norm\cdot_X$, $\paren{\im\dbar}_{\vphi_L}$ and
  $\paren{\im\dbadj}_{\vphi_L}$ denote the images of the corresponding
  operators, with $\cl{\paren{\im\dbar}}_{\vphi_L}$ and
  $\cl{\paren{\im\dbadj}}_{\vphi_L}$ being their closures in
  $\Ltwo/n,q/{L}_{\vphi_L}$)
  and the isomorphism
  \begin{equation*}
    \Harm \isom \cohgp q[X]{K_X \otimes L \otimes \mtidlof{\vphi_L}}
  \end{equation*}
  between the space of harmonic forms and the \v Cech cohomology
  group.
  % Given a locally finite Stein cover $\cvr V = \set{V_i}_{i \in I}$
  % with a partition of unity $\set{\rho^i}_{i\in I}$ subordinate to,
  With $\cvr V := \set{V_i}_{i\in I}$ and $\set{\rho^i}_{i\in I}$ given in Section \ref{subsec:notation},
  the isomorphism can be given explicitly as follows.
  For any (alternating) \v Cech $q$-cocycle $\set{\alpha_{\idx 0.q}}_{\idx 0,q \in
    I}$ and any harmonic form $u \in \Harm$ such that they represent
  the same class in $\cohgp q[X]{K_X \otimes L \otimes
    \mtidlof{\vphi_L}}$, the two representatives are related by 
  (under the Einstein summation convention)
  \begin{equation} \label{eq:Cech-Dolbeault-isom}
    \begin{aligned}
      u &=\dbar v_{(2)} +\dbar \rho^{i_{q-1}} \wedge \dotsm \wedge
      \dbar\rho^{i_0} \alpha_{\idx 0.q} \qquad\paren{\forall~ i_q \in
        I}
      \\
      &=\dbar v_{(2)} +\dbar \rho^{i_{q-1}} \wedge \dotsm \wedge
      \dbar\rho^{i_0} \cdot \rho^{i_q} \:\alpha_{\idx 0.q}
      \\
      &=\dbar v_{(2)} +(-1)^q \:\underbrace{\dbar \rho^{i_{q}} \wedge
        \dotsm \wedge \dbar\rho^{i_1} \cdot \rho^{i_0} }_{=: \:
        \paren{\dbar\rho}^{\idx q.0}} \alpha_{\idx 0.q}
    \end{aligned}
  \end{equation}
  for some $K_X \otimes L$-valued $(0,q-1)$-form $v_{(2)}$ on $X$ with
  $L^2$ coefficients with respect to $\norm\cdot_{X}$ (see
  \cite{Matsumura_injectivity}*{Prop.~5.5} or
  \cite{Chan&Choi_injectivity-I}*{Lemma 3.2.1}).

  The above result is applicable also to the case when $L$ is replaced by
  $D \otimes L$ equipped with the potential $\phi_D +\vphi_L$, where
  $\phi_D :=\log\abs{\sect_D}^2$.
  Denote the corresponding $L^2$ norm by $\norm\cdot_{X,\phi_D}$.
  Assume that \emph{$\vphi_L$ is smooth on $X$}.
  We state the following simple fact here for clarity.
  \begin{lemma} \label{lem:su-harmonicity}
    If $u \in \Harm{L}$, then $\sect_D u \in \Harm{D\otimes L},{\phi_D+\vphi_L}$.
  \end{lemma}
  
  \begin{proof}
    Since $\sect_D$ is holomorphic, it is clear that $\sect_D u$ is
    $\dbar$-closed.

    Let $\dfadj$ and $\dfadj_{\phi_D}$ be the formal adjoint of
    $\dbar$ with respect to $\vphi_L$ and $\phi_D +\vphi_L$
    respectively.
    It then follows that $\dfadj_{\phi_D} = \dfadj
    +\idxup{\diff\phi_D} . \cdot$ and 
    \begin{equation*}
      \dfadj_{\phi_D} \paren{\sect_D u}
      = \sect_D \:\dfadj u - \idxup{\diff\sect_D}. u
      +\idxup{\diff\phi_D} .\sect_D u
      =\sect_D \dfadj u = 0 \; .
    \end{equation*}
    Note that $\omega$ is not complete on $X \setminus D$ and the
    claim (in particular, $\sect_D u \in \Dom \dbadj_{\phi_D}$, where
    $\dbadj_{\phi_D}$ is the Hilbert space adjoint of $\dbar$ with
    respect to $\norm\cdot_{X,\phi_D}$) cannot follow from the
    standard result (for example, \cite{Demailly}*{Ch.~VIII,
      Thm.~(3.2c)}).
    Indeed, the proof of $su \in \Dom
    \dbadj_{\vphi_M}$ in \cite{Chan&Choi_injectivity-I}*{Cor.~3.2.6}
    gives precisely the result $\sect_D u \in \Dom\dbadj_{\phi_D}$ in
    the current setting, which completes the proof.
    A sketch of it is given below for readers' convenience.
    
    Let $\theta \colon [0,\infty) \to [0,1]$ be a smooth
    non-decreasing cut-off function such that
    $\res\theta_{[0,\frac12]} \equiv 0$ and $\res\theta_{[1,\infty)}
    \equiv 1$.
    Set $\theta_\eps := \theta \circ \frac{1}{\abs{\psi_D}^\eps}$ and
    $\theta'_\eps := \theta' \circ \frac{1}{\abs{\psi_D}^\eps}$ for
    every $\eps \geq 0$ (where $\theta'$ is the derivative of
    $\theta$).
    Then both $\theta_\eps$ and $\theta'_\eps$ have compact supports
    inside $X \setminus D$ for $\eps > 0$ and $\theta_\eps \ascendsto
    1$ pointwisely on $X \setminus D$ as $\eps \descendsto 0$.
    For any $\zeta \in \Dom\dbar \subset \Ltwo/n,q-1/<X>{D\otimes
      L}_{\phi_D+\vphi_L}$, convolution with a smoothing kernel on
    local coordinate charts and the lemma of Friedrichs guarantees the
    existence of a sequence $\seq{\zeta_{\eps, \nu}}_{\nu\in\Nnum}$ of
    smooth forms compactly supported in $X \setminus D$ such that
    $\zeta_{\eps,\nu} \tendsto \theta_\eps \zeta$ in the graph norm
    $\paren{\norm\cdot_{X,\phi_D}^2
      +\norm{\dbar\:\cdot}_{X,\phi_D}^2}^{\frac 12}$ of $\dbar$ for
    each $\eps > 0$.
    It then follows that
    \begin{align*}
      \iinner{\sect_D u}{\dbar\zeta}_{X,\phi_D} 
      \xleftarrow{\eps \tendsto 0^+}
      &~\iinner{\sect_D u}{\theta_\eps \dbar\zeta}_{X,\phi_D} \\
      =&~\iinner{\sect_D u}{\dbar\paren{\theta_{\eps}\zeta}}_{X,\phi_D}
         -\iinner{\sect_D u}{\dbar\theta_\eps \wedge \zeta}_{X,\phi_D} \\
      \xleftarrow{\nu \tendsto \infty}
      &~\iinner{\sect_D u}{\dbar\zeta_{\eps,\nu}}_{X,\phi_D}
        -\iinner{\sect_D u}{\frac{\eps \theta'_\eps}{\abs{\psi_D}^{1+\eps}}
        \dbar\psi_D \wedge \zeta}_{X,\phi_D} \\
      =&~\iinner{\dfadj_{\phi_D} \paren{\sect_D u}}{\zeta_{\eps,\nu}}_{X,\phi_D}
         -\iinner{\frac{\eps \theta'_\eps}{\abs{\psi_D}^{1+\eps}}
         \idxup{\diff\psi_D} . \sect_D u}{\zeta}_{X,\phi_D} \; .
    \end{align*}
    The inner product on the far right-hand-side converges to $0$ as
    $\eps \tendsto 0^+$, a consequence of the residue computation (see
    \cite{Chan&Choi_injectivity-I}*{Prop.~3.2.3 and Remark 3.2.4}).
    We can then conclude that $\sect_D u \in \Dom\dbadj_{\phi_D}$ after
    letting $\nu \tendsto \infty$ and then $\eps \tendsto 0^+$.
  \end{proof}

}

Now consider the cases where $(L, \vphi_L) =(F, \vphi_F)$
% (with the induced $L^2$ norm $\norm\cdot_{X}$)
and $(L, \vphi_L) =(F\otimes M, \vphi_F +\vphi_M)$.
% (with the induced $L^2$ norm $\norm\cdot_{X,\vphi_M}$).
A consequence of the positivity on $F$ and $M$ in
\cite{Enoki}, \cite{Matsumura_injectivity-lc} and
\cite{Chan&Choi_injectivity-I} are recalled below.
\begin{prop} \label{prop:consequence-of-positivity}
  % Suppose $\vphi_F$ and $\vphi_M$ are smooth such that
  % $\ibddbar\vphi_F \geq 0$ and $C\ibddbar\vphi_F \geq \ibddbar\vphi_M
  % \;\paren{\geq - C \omega}$ for some constant $C > 0$.
  % Then, $u \in \Harm{F}$ implies $su \in \Harm{F\otimes
  %   M},{\vphi_F+\vphi_M}$ and $\nabla^{(0,1)}u = 0$.
  Suppose that $\vphi_F$ is smooth such that
  $\ibddbar\vphi_F \geq 0$ and $u \in \Harm{F}$.
  Then, one has  $\nabla^{(0,1)}u = 0$.
  If, furthermore, $\vphi_M$ is smooth and satisfies
  $\paren{- C \omega \leq} \; \ibddbar\vphi_M \leq C\ibddbar\vphi_F$
  for some constant $C > 0$,
  then one also has $su \in \Harm{F\otimes M},{\vphi_F+\vphi_M}$.
\end{prop}

\begin{proof}[Reference to the proof]
  These results follow directly from the Bochner--Kodaira--Nakano
  formula.
  See \cite{Chan&Choi_injectivity-I}*{Prop.~3.2.5 and
    Cor.~3.2.6} (while taking $D=0$ and $\psi_D \equiv -1$ in those
  statements).
  See also the proofs for $\diff^*_h\xi = 0$ in
  \cite{Enoki}*{Prop.~2.1} or $D'^*u = 0$ in
  \cite{Matsumura_injectivity-lc}*{Prop.~3.7}.
  These are equivalent statements to the claim $\nabla^{(0,1)}u =0$
  (indeed, $\diff^*_h = D'^*$ and $\abs{D'^*u}^2 =
  \abs{\nabla^{(0,1)}u}^2$ by \cite{Chan&Choi_injectivity-I}*{Remark
    2.4.3}).
\end{proof}


Lemma \ref{lem:su-harmonicity} and Proposition
\ref{prop:consequence-of-positivity} are applied to the case with
$\lcS$ in place of $X$ and $\phi_{(p)}$ in place of $\phi_D$ in the
following sections.



%%% Local Variables:
%%% mode: latex
%%% TeX-master: "Injectivity-Fujino"
%%% coding: utf-8
%%% End:




\subsection{Adjoint ideal sheaves and the residue computations}
% \subsection{Residue functions and residue short exact sequences}
\label{subsec:residue}

%%%%%
%%%%% File name  : residue-fcts-n-residue-exact-seq.tex
%%%%% Author     : Mario Chan
%%%%% Date       : 10th March, 2023
%%%%% Description: This is the section of the project
%%%%%              "Injectivity-Fujino" on residue functions and 
%%%%%              residue exact sequences. 
%%%%%
%%
%%%

{
  \setDefaultvphi{\vphi_L}

  Let $L$ be a line bundle on $X$ equipped with a \mmark{smooth metric
    $e^{-\vphi_L}$}{$\vphi_L$ has to be smooth, or the claim on the
    jumping number must be mentioned explicitly. The result  $\aidlof*
    =\mtidlof{\vphi_L} \cdot \defidlof{\lcc+1'}$ may not hold
    otherwise. \\ }. 
  The \mmark[BlueGreen]{residue function $\eps \mapsto \RTF|f|,<V>$}{It's
    possible not using ``$\RTF|f|$'' at all in this paper.} of index $\sigma$ 
  is defined, for each \mhlight[BlueViolet]{$f \in  \logKX[L] \otimes
    \smooth_X (V)$}, to be 
  \begin{equation*}
    \RTF|f|,<V> :=\RTF|f|,<V>,
    := \eps \int_V \frac{\abs f^2 \:e^{-\phi_D-\vphi_L}}{\logpole} \quad
    \text{ for } \eps > 0  \; . 
  \end{equation*}\mariocomment[BlueViolet]{For consistency of notation in this
    section only.}%
  The adjoint ideal sheaf $\aidlof :=\aidlof<X>$ of index $\sigma$
  is given at each $x \in X$  by
  \begin{equation*}
    \aidlof_x :=\setd{f \in \holo_{X,x}}{
      \exists~\text{open set } V_x \ni x \:, \: \forall~\eps > 0 \:, \:
      \RTF|f|,<V_x>, < +\infty
    } \; .
  \end{equation*}
  Note that the adjoint ideal sheaf is independent of $\vphi_L$ (as
  $\vphi_L$ is smooth).
  By \cite{Chan_adjoint-ideal-nas}*{Thm.~1.2.3}, the adjoint ideal
  sheaf can be written as 
  \begin{equation*}
    \aidlof = \mtidlof{\vphi_L} \cdot \defidlof{\lcc+1'}
    =\defidlof{\lcc+1'}
    \quad\text{ for any } \sigma \geq 0 \; ,
  \end{equation*}
  where $\defidlof{\lcc+1'}$ is the defining ideal sheaf of $\lcc+1'$
  in $X$ (with the reduced structure), \mmark{and we have the residue short exact
    sequence}{I don't want to suggest that the product structure of
    $\aidlof*$ implies directly the residue exact sequence.}
  \begin{equation*}
    \xymatrix@R-0.5cm@C+0.3cm{
      {0} \ar[r]
      & {\aidlof-1} \ar[r]
      & {\aidlof} \ar[r]^-{\Res^\sigma}
      & {\residlof} \ar[r]
      & {0 \; .}
    }
  \end{equation*}
  Here the quotient sheaf ${\residlof}$, called the \emph{residue sheaf of index $\sigma$}, can be written as 
  \begin{equation*}
    \residlof
    = \bigoplus_{p \in \Iset} \paren{\Diff_p D}^{-1}
    \otimes \mtidlof<\lcS>{\vphi_L}
    = \bigoplus_{p \in \Iset} \paren{\Diff_p D}^{-1}
  \end{equation*}
  Note $\logKX[L] \otimes \residlof =\bigoplus_{p \in\Iset} K_{\lcS} \otimes \res L_{\lcS}.$
  Next we describe the \emph{residue morphism $\Res^\sigma$} in terms of 
  the Poincar\'e residue map $\PRes[\lcS]$ given in
  \cite{Kollar_Sing-of-MMP}*{\S 4.18} as follows. 
  The Poincar\'e residue map $\PRes[\lcS]$ from $X$ to each $\lcS$ is
  uniquely determined after an orientation on the conormal bundle of
  $\lcS$ in $X$ is fixed.
  For an admissible open set $V \subset X$, 
  we have $\lcc' \cap V = \bigcup_{\alert{p \in \Iset}}
  \lcS<V>$ \mmark{(where $\lcS<V> := \lcS \cap V$, which is connected by the
  definition of the admissible open set, and possibly empty)}{This is
  a subtle fact that is used in the residue computation. We can keep
  using the same index set because $V$ is admissible.} and $\lcS<V>
=\set{z_{p(1)} =z_{p(2)} =\dotsm =z_{p(\sigma)}=0}$ when non-empty. 
  Under such coordinate system, a section $f $ of  $\logKX[L] \otimes
  \aidlof$ on $V \subset X$ can be written as
  \begin{equation*}
    f = \;\;\smashoperator{\sum_{p \in \Iset \colon \lcS<V>
        \neq\emptyset}} \;\; dz_{p(1)} \wedge \dotsm \wedge dz_{p(\sigma)}
    \wedge g_p \:\sect_{(p)} 
    =\;\;\smashoperator[l]{\sum_{p \in \Iset \colon \lcS<V>
        \neq\emptyset}}
    \frac{dz_{p(1)}}{z_{p(1)}} \wedge \dotsm
    \wedge \frac{dz_{p(\sigma)}}{z_{p(\sigma)}}
    \wedge g_p \:\sect_D \quad\text{ on } V. 
  \end{equation*}
  % Then, the Poincar\'e residue map $\PRes[\lcS]$ is given by
  \mmark{We therefore see that 
  \begin{equation*}
    \PRes[\lcS](\frac{f}{\sect_D})  =\res{g_p}_{\lcS} \in
    K_{\lcS} \otimes \res L_{\lcS} \quad\text{ on } \lcS<V> 
  \end{equation*}}{I don't want to give the impression that we define
  the Poincar\'e residue map by this formula.}%
  under the assumption that the orientation on the conormal bundle of
  $\lcS$ in $X$ on $V$ is given by $(dz_{p(1)}, dz_{p(2)}, \dots,
  dz_{p(\sigma)})$. 
  % Result in \cite{Chan_adjoint-ideal-nas}*{Thm.~4.1.2 (2)} (or the
  % computation in \cite{Chan_on-L2-ext-with-lc-measures}*{Prop.~2.2.1}
  % or \cite{Chan&Choi_ext-with-lcv-codim-1}*{Prop.~2.2.1}) yields
  % % (assuming that $f$ lives on a neighbourhood $V'$ of $\cl V$)
  % \begin{equation*}
  %   \RTF[\rho]|f|(0),<V> = \lim_{\eps \tendsto 0^+}
  %   %   \lim_{\rho \descendsto \charfct_{\cl V}}
  %   \RTF[\rho]|f|,<V>
  %   =\sum_{p \in \Iset} \frac{\pi^\sigma}{(\sigma -1)!} \int_{\lcS<V>}
  %   \rho \abs{g_p}^2 \:e^{-\vphi_L} 
  % \end{equation*}
  % for any compactly supported smooth function $\rho \colon V \to
  On the other hand, the residue morphism $\Res^\sigma$ is given in
  \cite{Chan_adjoint-ideal-nas}*{\S 4.2} by 
  \begin{equation*}
    \renewcommand{\objectstyle}{\displaystyle}
    \xymatrix@C+0.5cm@R-0.5cm{
      {\logKX[L] \otimes \aidlof} \ar[r]^-{\Res^\sigma}
      \ar@{}[d]|*[left]+{\in} 
      & {\hphantom{\logKX[L] \otimes \residlof}}
      \save +<4em,-1.3ex>*{\logKX[L] \otimes \residlof
        =\bigoplus_{p \in\Iset} K_{\lcS} \otimes \res L_{\lcS}} \restore
      \ar@{}[d]|*[left]+{\in}
      % & *+<-2cm,-1cm>{}
      % \ar@{}[l]|(.41)*+{}
      \\
      *+<0.8cm,0cm>{f} \ar@{|->}[r]
      & {\paren{\res{g_p}_{\lcS}}_{\mathrlap{p\in\Iset}}
        \mathrlap{\hphantom{p\in\Iset} .}} 
    }
  \end{equation*}
  Assuming $f$ being defined on a neighbourhood $V'$ of the closure
  $\cl V$ of $V$ and letting $\rho \colon V' \to [0,1]$ be a compactly
  supported smooth function
  % (i.e.~a smooth cut-off function) being
  identically equal to $1$ on $V$, one obtains, 
  from the result in \cite{Chan_adjoint-ideal-nas}*{Thm.~4.1.2 (2)} (or the
  computation in \cite{Chan_on-L2-ext-with-lc-measures}*{Prop.~2.2.1}
  or \cite{Chan&Choi_ext-with-lcv-codim-1}*{Prop.~2.2.1}),
  a (squared) norm of $g :=\paren{\res{g_p}_{\lcS<V>}}_{p \in \Iset} \in
  \logKX[L] \otimes \residlof$ on $V$ given by
  \begin{equation} \label{eq:residue-norm}
    \norm{g}_{\lcc<V>'}^2 :=\RTF|f|(0),<V>
    =\lim_{\rho \descendsto \charfct_{\cl V}} \lim_{\eps \tendsto 0^+}
    \RTF[\rho]|f|,<V'>
    =\sum_{p \in \Iset} \frac{\pi^\sigma}{(\sigma -1)!}
    \int_{\mathrlap{\lcS<V>}} \;\;\;
    \abs{g_p}^2 \:e^{-\vphi_L}
    =:\sum_{p\in\Iset} \norm{g_p}_{\lcS<V>}^2 \; ,
  \end{equation}
  where the limit $\lim_{\rho \descendsto \charfct_{\cl V}}$ refers to
  the pointwise limit as $\rho$ descends to the characteristic
  function $\charfct_{\cl V}$ of $\cl V$ on $X$.
  Such a norm is referred to as the \emph*{residue norm on $\logKX[L]
    \otimes \residlof$ on $V$}.
  %%%%% \emph* is needed as the package embrac is used and \logKX[L]
  %%%%% appears inside \emph.
  Moreover, we also see from the residue exact sequence that
  \begin{equation*}
    \aidlof-1_x
    =\setd{f \in \aidlof_x}{ \exists~\text{open set } V_x
      \ni x \:, \: \RTF|f|(0),<V_x> = 0}
    % \\
    % &=\setd{f \in \aidlof_x}{ \exists~\text{open set } V_x
    %   \ni x \:, \: \RTF|f|,<V_x> = \BigO(\eps) \text{ as } \eps
    %   \tendsto 0^+}
  \end{equation*}
  for every $x \in X$.

  Under the assumption that $\vphi_L$ has only neat analytic
  singularities (which is indeed smooth in the current setting), the
  residue norm on an admissible open set $V \subset X$ can also be
  obtained from 
  \begin{equation*}
    \lim_{\eps \tendsto 0^+} \eps \int_{V} \frac{
      \rho \abs f^2 \:e^{-\phi_D-\vphi_L}
    }{\abs{\psi_D}^{\sigma +\eps}}
    =\RTF[\rho]|f|(0),<V>
  \end{equation*}
  for any smooth compactly supported cut-off function $\rho$ on $V$ (see
  \cite{Chan&Choi_ext-with-lcv-codim-1}*{Prop.~2.2.1} or
  \cite{Chan&Choi_injectivity-I}*{Thm.~2.6.1}).
  Moreover, the above equation works not only for $f$ with holomorphic
  coefficients, but also for $f$ with coefficients in
  $\smooth_{X\,*}$, where
  \begin{align*}
    \smooth_{X\, *}
    &:=\paren{\smooth_{X}\left[
      \frac{1}{\abs{\sect_i}} \colon i \in \Iset||
      \right]}_{\text{b}}
      \qquad\paren{\sect_i \text{ treated as a local defining function of }
      D_i} \\
    &:=\set{\text{locally bounded elements in the $\smooth_X$-algebra generated
      by } \frac{1}{\abs{\sect_i}} \text{ for all } i\in\Iset||} \;
      .\footnotemark
  \end{align*}%
  \footnotetext{
    On an admissible open set $V$ under the holomorphic coordinate
    system $(z_1,\dots, z_n)$ such that $D\cap V =\set{z_1 z_2 \dotsm
      z_{\sigma_V} =0}$, one has
    \begin{equation*}
      \smooth_{X \,*}(V)
      =\smooth_X(V)\left[e^{\pm \cplxi \theta_1}, \dots, e^{\pm \cplxi
          \theta_{\sigma_V}} \right]
    \end{equation*}
    where $(r_j,\theta_j)$ is the polar coordinate system of the
    $z_j$-plane for $j=1,\dots,\sigma_V$ in $V$, which is (almost) the
    same as the ad hoc definition of $\smooth_{X\, *}(V)$ given in 
    \cite{Chan&Choi_injectivity-I}*{\S 2.6} (in which
    $e^{\pm\cplxi\theta_{k}}$ for $k \geq \sigma_V +1$ are also included
    in the set of generators of the algebra).
    The definition given here is independent of coordinates and its
    sheaf structure can be seen easily.
  }%
  The coefficients of $\Res^\sigma$ (and hence $\PRes[\lcS]$ for any
  $p\in\Iset$) can be extended from $\holo_X$ to $\smooth_{X\,*}$
  accordingly.
  The residue norm is finite when the coefficients of $f$ belong to
  $\smooth_{X\,*} \cdot \aidlof$ on $V$.
  When the induced inner product is considered, one still has
  finiteness even if one of the argument does not have coefficients in
  $\smooth_{X\,*} \cdot \aidlof$, which is the content of the
  following proposition.
  \begin{prop} \label{prop:residue-product-X-to-lcS}
    Given any admissible open set $V \subset X$ and any section $f \in
    \logKX[L] \otimes \smooth_{X \:c\,*} \cdot\aidlof\paren{V}$
    (compactly supported in $V$) such
    that $\Res^\sigma(f) = g =\paren{g_p}_{p\in\Iset}$, one
    has, for any $\xi \in \logKX[L] \otimes \smooth_{X \, *}\paren{V}$,
    \begin{align*}
      \lim_{\eps \tendsto 0^+} \eps \int_V
      \frac{\inner{\xi}{f} \:e^{-\phi_D-\vphi_L}}{\abs{\psi_D}^{\sigma
      +\eps}}
      &=\sum_{p \in \Iset} \frac{\pi^\sigma}{(\sigma-1)!}
        \int_{\lcS<V>} \inner{\frac{\rs*\xi_p}{\sect_{(p)}}}{\: g_p}
        \:e^{-\vphi_L} \\
      &=\sum_{p \in \Iset}
      % \smash[b]{
        \underbrace{
        \frac{\pi^\sigma}{(\sigma-1)!}
        \int_{\lcS<V>} \inner{\rs*\xi_p}{\: g_p \sect_{(p)}}
        \:e^{-\phi_{(p)}-\vphi_L}
        }_{\displaystyle =:
        \iinner{\rs*\xi_p}{g_p\sect_{(p)}}_{\mathrlap{\lcS<V>,
        \phi_{(p)}}}}
  % }
  %   \vphantom{\underbrace{\int_{\lcS<V>}}_{\iinner{\rs*\xi_p}{g_p\sect_{(p)}}}}
    \end{align*}
    which is finite,
    where $\phi_{(p)} :=\log\abs{\sect_{(p)}}^2$ and
    \begin{equation*}
      \rs*\xi_p := \PRes[\lcS](\frac{\xi}{\sect_D}) \cdot \sect_{(p)}
      \in K_{\lcS} \otimes \Diff_p D \otimes \res L_{\lcS} \otimes
      \smooth_{X\:c\, *}\paren{\lcS<V>} \; .
    \end{equation*}
    Moreover, if either $f$ or $\xi$ belongs to $\logKX[L] \otimes
    \smooth_{X\,*} \cdot\aidlof-1\paren{V}$, then $\eps \int_V
      \frac{\inner{\xi}{f} \:e^{-\phi_D-\vphi_L}}{\abs{\psi_D}^{\sigma
      +\eps}} = \BigO(\eps)$ (the big-O notation) as $\eps \tendsto 0$.
  \end{prop}

  \begin{proof}
    By linearity in $f$ in the equation in the claim, it suffices to
    consider the case where $\lcS \cap V = \set{z_1 =z_2 = \dotsm
      z_\sigma = 0}$, $\res{\sect_{(p)}}_{V} = z_{\sigma+1} \dotsm z_{\sigma_V}$ and
    \begin{equation*}
      f = dz_1 \wedge dz_2 \wedge \dotsm \wedge dz_\sigma \wedge g_p
      \sect_{(p)} 
    \end{equation*}
    (in which $g_p$ is abused to mean a $(n-\sigma,0)$-form on $V$).
    Write also
    \begin{equation*}
      \xi =: dz_1 \wedge dz_2 \wedge \dotsm \wedge dz_\sigma \wedge
      \xi_p
      \quad\text{ such that }\;\;
      \res{\xi_p}_{\lcS<V>} = \PRes[\lcS](\frac{\xi}{\sect_D})
      \cdot \sect_{(p)} =\rs*\xi_p \; .
    \end{equation*}
    Let $(r_j, \theta_j)$ be the polar coordinates of the $z_j$-plane
    and set 
    \begin{equation*}
      F_0 :=\inner{\xi_p}{g_p} \:e^{-\vphi_L}
      \quad\text{ and }\quad
      F_j :=\fdiff{r_j} \paren{\frac{F_j}{r_j^2 \fdiff{r_j^2} \psi_D}}
      \quad\text{ for } j=1,\dots, \sigma \; .
    \end{equation*}
    Notice that $\fdiff{r_j} \sect_{(p)} = 0$ and coefficients of
    $F_j$ are in $\smooth_{X\:c\,*}$ on $V$ for $j=1,\dots,\sigma$.
    It then follows from the similar computation in
    \cite{Chan&Choi_ext-with-lcv-codim-1}*{Prop.~2.2.1} or
    \cite{Chan&Choi_injectivity-I}*{Thm.~2.6.1} that
    \begin{align*}
      \eps \int_V
      \frac{\inner{\xi}{f} \:e^{-\phi_D-\vphi_L}}{\abs{\psi_D}^{\sigma
      +\eps}}
      &=\eps \int_V
        \frac{\inner{\xi_p}{g_p} \:e^{-\vphi_L}}{\sect_{(p)}\:\abs{\psi_D}^{\sigma
        +\eps}} \wedge \bigwedge_{j=1}^\sigma \frac{\pi\ibar\:dz_j
        \wedge d\conj{z_j}}{\abs{z_j}^2}
      \\
      &=\eps \int_V \frac{F_0}{\sect_{(p)}\:\abs{\psi_D}^{\sigma +\eps}}
        \prod_{j=1}^\sigma d\log r_j^2 \cdot
        \underbrace{\prod_{j=1}^\sigma \frac{d\theta_j}2}_{=:\:
        \vect{d\theta}}
      \\
      &=\frac\eps{\sigma-1+\eps}
        \int_{V} \frac{F_0}{\sect_{(p)}\:r_1^2 \fdiff{r_1^2}\psi_D}
        \:d\paren{\frac{1}{\abs{\psi_D}^{\sigma-1+\eps}}}
        \prod_{j=2}^{\sigma} d\log r_j^2 
        \cdot \vect{d\theta} \\
      &\overset{\mathclap{\text{int.~by parts}}}=
        \quad\;\;
        \frac{-\eps}{\sigma-1+\eps}
        \int_{V}
        \frac{\alert{F_1}}{\sect_{(p)}\:\abs{\psi_D}^{\sigma-1+\eps}}
        \prod_{j=2}^{\sigma} d\log r_j^2 
        \cdot dr_1 \:\vect{d\theta} \\
      &= \dotsm =
        \frac{(-1)^{\sigma} \eps} {\prod_{j=1}^{\sigma} \paren{\sigma-j+\eps}} 
        \int_{V}
        \frac{F_{\alert{\sigma}}}{\sect_{(p)}\:\abs{\psi_D}^{\eps}}
        \prod_{j=1}^{\sigma} dr_j
        \cdot \vect{d\theta} \; .
    \end{align*}
    Note that $\frac{1}{\sect_{(p)}}$ is integrable on $V$, so the
    integral on the far right-hand-side converges for all $\eps \geq
    0$. 
    Letting $\eps \tendsto 0^+$ on both sides, the desired formula
    then follows from the fundamental theorem of calculus.

    When $f$ or $\xi$ belongs to $\logKX[L] \otimes \smooth_{X\,*}
    \cdot\aidlof-1\paren{V}$, the residue formula in the proposition
    holds even when $\sigma$ is replaced by $\sigma-1$, which implies
    that the integral $\int_V \frac{\inner{\xi}{f}
      \:e^{-\phi_D-\vphi_L}}{\abs{\psi_D}^{\sigma +\eps}}$ converges
    for all $\abs\eps < 1$, hence the last claim.
  \end{proof}

  When restriction to a subspace of codimension $1$ is considered,
  there is a more classical kernel for obtaining the residue formula.
  As an illustration, the residue formula from $X$ to $\lcc|1|'$ is
  proved in the following proposition (which is applied to the case
  where the residue from $\lcc'$ to $\lcc+1'$ is considered later). 
  Recall that $\lcc|1|' =D =\sum_{i \in \Iset||} D_i$, where $D_i =
  \lcS|1|[i]$ and $\Iset|| =\Iset|1|$ are set for convenience.
  \begin{prop} \label{prop:residue-formula-classical-kernel}
    Given any admissible open set $V \subset X$ and any compactly
    supported section $f \in
    \logKX[L] \otimes \smooth_{X \:c\,*} \cdot\aidlof|1|\paren{V}$
    such that $\Res^1(f) = g =\paren{g_i}_{i\in\Iset||}$, one
    has, for any $\xi \in \logKX[L] \otimes \smooth_{X \, *}\paren{V}$,
    \begin{align*}
      \lim_{\eps \tendsto 0^+} \eps \int_V
      \inner{\xi}{f} \:e^{-\phi_D-\vphi_L} e^{-\eps\abs{\psi_D}}
      &=\sum_{i \in \Iset||} \pi
        \int_{D_i \cap V} \inner{\frac{\rs*\xi_i}{\sect_{(i)}}}{\: g_i}
        \:e^{-\vphi_L} \\
      &=\sum_{i \in \Iset||}
      % \underbrace{
        \pi
        \int_{D_i \cap V} \inner{\rs*\xi_i}{\: g_i \sect_{(i)}}
        \:e^{-\phi_{(i)}-\vphi_L}
        % }_{\displaystyle =:
        %   \iinner{\rs*\xi_i}{g_i\sect_{(i)}}_{\mathrlap{D_i \cap V,
        %   \phi_{(i)}}}}
    \end{align*}
    which is finite,
    where $\phi_{(i)} :=\log\abs{\sect_{(i)}}^2$ and
    \begin{equation*}
      \rs*\xi_i := \PRes[D_i](\frac{\xi}{\sect_D}) \cdot \sect_{(i)}
      \in K_{D_i} \otimes \Diff_i D \otimes \res L_{D_i} \otimes
      \smooth_{X\:c\, *}\paren{D_i \cap V} \; .
    \end{equation*}    
  \end{prop}

  \begin{proof}
    As before, it suffices to consider the case where $D_i \cap V
    =\set{z_1 =0}$, $\res{\sect_{(i)}}_V =z_2 \dotsm z_{\sigma_V}$ and
    \begin{equation*}
      f = dz_1 \wedge g_i \sect_{(i)} \; .
    \end{equation*}
    Write also
    \begin{equation*}
      \xi =: dz_1 \wedge \xi_i
      \quad\text{ such that }\;\;
      \res{\xi_i}_{D_i \cap V} =\PRes[D_i](\frac{\xi}{\sect_D}) \cdot
      \sect_{(i)} =\rs*\xi_i \; .
    \end{equation*}
    Essentially the same computation as in Proposition
    \ref{prop:residue-product-X-to-lcS} yields 
    \begin{align*}
      \eps \int_V \inner{\xi}{f} \:e^{-\phi_D-\vphi_L}
      e^{-\eps\abs{\psi_D}}
      =&~\eps \int_V \frac{\inner{\xi_i}{g_i}
         \:e^{-\vphi_L}}{\sect_{(i)}} \wedge
         e^{-\eps\abs{\psi_D}}
         \frac{
         \pi\ibar\:dz_1 \wedge d\conj{z_1}
         }{\abs{z_1}^2}
      \\
      =&~\int_V \frac{\inner{\xi_i}{g_i}
         \:e^{-\vphi_L}}{\sect_{(i)} \:r_1^2 \fdiff{r_1^2} \psi_D} 
         \:d\paren{e^{-\eps\abs{\psi_D}}} \:
         \frac{d\theta_1}{2}
      \\
      \overset{\mathclap{\text{int.~by parts}}}=
       &~\quad\;\;
         -\int_V \fdiff{r_1} \paren{\frac{\inner{\xi_i}{g_i}
         \:e^{-\vphi_L}}{r_1^2 \fdiff{r_1^2} \psi_D} }
         \:\frac{e^{-\eps\abs{\psi_D}}}{\sect_{(i)}} \:dr_1 \:
         \frac{d\theta_1}{2}
      \\
      \mathclap{\xrightarrow{\eps \tendsto 0^+}\;\;}
       &~\quad\;
         -\int_V \fdiff{r_1} \paren{\frac{\inner{\xi_i}{g_i}
         \:e^{-\vphi_L}}{r_1^2 \fdiff{r_1^2} \psi_D} }
         \:\frac{1}{\sect_{(i)}} \:dr_1 \:
         \frac{d\theta_1}{2}
      \\
      =&~\pi \int_{\mathrlap{D_i \cap V}} \;\;\; \frac{\inner{\rs*\xi_i}{g_i}
         \:e^{-\vphi_L}}{\sect_{(i)}}
         =\pi \int_{D_i \cap V} \inner{\rs*\xi_i}{g_i \sect_{(i)}}
         \:e^{-\phi_{(i)}-\vphi_L} \; .
    \end{align*}
    Note that the convergence of the integral obtained right after
    integration by parts follows from the same reasoning as in
    Proposition \ref{prop:residue-product-X-to-lcS}.
  \end{proof}

  Proposition \ref{prop:residue-formula-classical-kernel} facilitates the
  following residue computation.

  \begin{prop} \label{prop:res-formula-dbar-exact-dot-harmonic}
    Given the decomposition $\lcc' = \bigcup_{p\in\Iset} \lcS$, let
    $u_p$ be a \emph{harmonic} $K_{\lcS} \otimes \res L_{\lcS}$-valued
    $(0,q)$-form on $\lcS$ with respect to the norm
    $\norm\cdot_{\lcS}$ for each $p \in \Iset$.
    Given also the decomposition $\lcc+1' = \bigcup_{b\in\Iset+1}
    \lcS+1[b]$, for any $\lcS$ and $\lcS+1[b]$ such that $\lcS+1[b]
    \subset \lcS$, let $\sgn{b:p}$ be the sign such that
    \begin{equation*}
      \PRes[\lcS+1[b]] =\sgn{b:p} \:\PRes[\lcS+1[b] | \lcS] \circ
      \PRes[\lcS] \; ,
    \end{equation*}
    where $\PRes[\lcS+1[b] | \lcS]$ denotes the Poincar\'e residue map
    from $\lcS$ to $\lcS+1[b]$.
    % For a given locally finite cover $\cvr V :=\set{V_i}_{i \in I}$ of
    % $X$ by \emph{admissible} open sets with respect to
    % $(\vphi_L,\psi_D)$ together with a partition of unity
    % $\set{\rho^i}_{i \in I}$ subordinate to it,
    With the finite cover $\cvr V$ and partition of unity
    $\set{\rho^i}_{i \in I}$ given in Section \ref{subsec:notation},
    let $\set{\gamma_{\idx 1.q}}_{\idx 1,q \in I}$ be a
    $\logKX[L]$-valued \v Cech $(q-1)$-cochain with respect to $\cvr
    V$ and set 
    \begin{gather*}
      \rs\gamma_{p; \:\idx 1.q} :=\PRes[\lcS](\frac{\gamma_{\idx 1.q}}{\sect_D})
      \cdot \sect_{(p)} \; , \quad
      v_{p} := \sum_{\idx 1,q \in I} \underbrace{
        \dbar\rho^{i_q} \wedge \dotsm
        \wedge \dbar\rho^{i_2} \cdot \rho^{i_1}
      }_{=: \: \paren{\dbar\rho}^{\idx q.1}} \rs*\gamma_{p;\:\idx 1.q}
      \quad\text{ on } \lcS \\
      \text{and }\quad
      \rs\gamma_{b; \:\idx 1.q} :=\PRes[\lcS+1[b]](\frac{\gamma_{\idx 1.q}}{\sect_D})
      \cdot \sect_{(b)} \; , \quad
      v_{b} := \sum_{\idx 1,q \in I} \paren{\dbar\rho}^{\idx q.1}
      \rs*\gamma_{b;\:\idx 1.q}
      \quad\text{ on } \lcS+1[b] \; .
    \end{gather*}
    Then, after setting $\iinner{\cdot}{\cdot}_{\lcS, \phi_{(p)}}
    :=\iinner{\cdot}{\cdot \:e^{-\phi_{(p)}}}_{\lcS}$ (and
    similarly for $\iinner{\cdot}{\cdot}_{\lcS+1[b], \phi_{(b)}}$), one has
    \begin{equation*}
      \sum_{p\in\Iset} \iinner{\dbar v_{p}}{
        u_p\sect_{(p)}}_{\lcS,\phi_{(p)}}
      =-\sigma \smashoperator[l]{\sum_{b\in\Iset+1}} \iinner{v_{b} \:}{\quad\;
        \smashoperator{\sum_{p\in\Iset \colon \lcS+1[b] \subset
            \lcS}} \;\;
        \sgn{b:p} \: \PRes[\lcS+1[b] | \lcS](\idxup{\diff\psi_{(p)}}.
         u_p) \cdot \sect_{(b)}
      }_{\lcS+1[b], \phi_{(b)}} \; ,
    \end{equation*}
    where $\psi_{(p)} :=\phi_{(p)} -\sm\vphi_{(p)}$ and
    $\sm\vphi_{(p)}$ is some smooth potential on $\Diff_p D$.
  \end{prop}

  \begin{proof}
    Notice that $v_{p}$ is smooth on $\lcS$ but not necessarily
    locally $L^2$ with respect to the weight $e^{-\phi_{(p)}}$.
    An integration by parts is done via the use of Proposition
    \ref{prop:residue-formula-classical-kernel}, which yields 
    \begin{align*}
      &~\sum_{p\in \Iset} \iinner{\dbar v_{p}}{ u_p
        \sect_{(p)}}_{\lcS, \phi_{(p)}}
      \\
      \xleftarrow{\eps \tendsto 0^+}
      &~\sum_{p \in \Iset} \iinner{
        e^{-\eps \abs{\psi_{(p)}}} \:\dbar v_{p}
        }{ u_p \sect_{(p)}}_{\lcS, \phi_{(p)}}
      \\
      =&~\sum_{p \in \Iset} \paren{
         \cancelto{0 \;\;\;(\because~u_p \text{ harmonic, Lemma \ref{lem:su-harmonicity}})}{\iinner{
         \dbar\paren{e^{-\eps \abs{\psi_{(p)}}} \: v_{p}}
         }{ u_p \sect_{(p)}}_{\mathrlap{\lcS, \phi_{(p)}}}}
         \quad\;\; - \eps 
         \iinner{
         e^{-\eps \abs{\psi_{(p)}}} \:v_{p}
         }{\:\idxup{\diff\psi_{(p)}}.  u_p \sect_{(p)}}_{\lcS,
         \phi_{(p)}}
         }
      \\
      =&~-\sum_{p \in \Iset} \sum_{\idx 1,q \in I} \eps \:
         \iinner{
         e^{-\eps \abs{\psi_{(p)}}} \: % \paren{\dbar\rho}^{\idx q.1}
         \rs*\gamma_{p;\:\idx 1.q}
         }{\:
         \idxup{\diff\rho},[\idx 1.q] .
         \paren{\idxup{\diff\psi_{(p)}}.  u_p \sect_{(p)}}
         }_{\lcS, \phi_{(p)}}
      \\
      \xrightarrow[\text{Prop.~\ref{prop:residue-formula-classical-kernel}}]{\eps
      \tendsto 0^+} 
      &~-\smashoperator[l]{\sum_{\idx 1,q \in I}} \sum_{p \in \Iset}
        \sum_{k=\sigma +1}^{\mathclap{\sigma_{V_{\idx 1.q}}}} \sigma
        \iinner{
        \PRes[p(k)](
        \frac{\rs*\gamma_{p;\:\idx 1.q}}{\sect_{(p)}}
        )
        }{\:
        \idxup{\diff\rho},[\idx 1.q] .
        \PRes[p(k)](\idxup{\diff\psi_{(p)}}.  u_p)
        }_{\lcS \cap \set{z_{p(k)} =0}}
        \; ,
    \end{align*}
    where $\idxup{\diff\rho},[\idx 1.q] . \cdot$ is the adjoint
    of $\paren{\dbar\rho}^{\idx q.1} \cdot$, and $\PRes[p(k)]$ denotes
    the Poincar\'e residue map from $\lcS$ to $\lcS \cap \set{z_{p(k)}=0}$. 
    The last limit is justified as follows.
    On the admissible open set $V_{\idx 1.q}$, consider a holomorphic
    coordinate system $(z_1, \dots, z_n)$ such that $\lcS \cap V_{\idx
    1.q}
    =\set{z_{p(1)} = \dotsm =z_{p(\sigma)} =0}$ and
    $\sect_{(p)} =z_{p(\sigma+1)} \dotsm z_{p(\sigma_V)}$ (write
    $\sigma_{V}$ for $\sigma_{V_{\idx 1.q}}$ for convenience).
    Note that
    \begin{equation*}
      \diff\psi_{(p)} =\sum_{k =\sigma +1}^{\sigma_V}
      \frac{dz_{p(k)}}{z_{p(k)}} -\diff\sm\vphi_{(p)} \quad\text{ on }
      V_{\idx 1.q} \; .
    \end{equation*}
    It follows that, on $\lcS \cap V_{\idx 1.q}$,
    \begin{equation*}
      \text{coef.~of }\:
      \idxup{\diff\rho},[\idx 1.q].
      \paren{\idxup{\diff\psi_{(p)}}.  u_p \sect_{(p)}}
      \in
      \smooth_{\lcS \:c} \cdot\res{\defidlof{\lcc+2'}}_{\lcS}
      \begin{aligned}[t]
        &=\smooth_{\lcS \:c} \cdot\mtidlof<\lcS>{\vphi_L} \cdot
        \res{\defidlof{\lcc+2'}}_{\lcS} \;\;\footnotemark
        \\
        &=\smooth_{\lcS \:c} \cdot\aidlof|1|<\lcS>{\vphi_L}[\psi_{(p)}]
      \end{aligned}
    \end{equation*}%
    \footnotetext{
      Recall that $\defidlof{\lcc+2'}$ is generated on $X$ by
      $\sect_{(b)}$ treated as local
      functions for all $b \in \Iset+1$.
      On an admissible open set $V$, one has $\defidlof{\lcc+2'}
      =\genbyd{z_{b(\sigma+2)} \dotsm
        z_{b(\sigma_V)}}{b \in \Iset+1 \text{ such that } \lcS+1[b] \cap
        V \neq \emptyset}$.
      % (see page
      % \pageref{page:notation-permutation-index} for the notation).
    }%
    and, therefore, one can apply Proposition
    \ref{prop:residue-formula-classical-kernel} (with $\lcS$ in place
    of $X$, $\psi_{(p)}$ in place of $\psi_D$) to each inner product
    $\eps \iinner{e^{-\eps \abs{\psi_{(p)}}} \dotsm}{\: \dotsm
      \idxup{\diff\psi_{(p)}} . \dotsm \sect_{(p)}}_{\lcS,\phi_{(p)}}$.
    Note also that the factor $\sigma$ comes from the normalisation of
    the norm on each lc center ($\norm\cdot_{\lcS}^2
    =\frac{\pi^\sigma}{(\sigma -1)!} \int_{\lcS} \dotsm$ and
    $\norm\cdot_{\lcS+1[b]}^2 =\frac{\pi^{\sigma+1}}{\sigma!}
    \int_{\lcS+1[b]} \dotsm$).


    On each admissible open set $V_{\idx 1.q}$, the intersection $\lcS
    \cap \set{z_{p(k)} = 0}$ is a $(\sigma+1)$-lc center $\lcS+1[b_{p,k}]
    \cap V_{\idx 1.q}$ ($\neq \emptyset$), uniquely determined by the
    choices of $p\in \Iset$ (such that $\lcS \cap V_{\idx 1.q} \neq
    \emptyset$, so $\binom{\sigma_V}{\sigma}$ choices) and $k
    =\sigma+1, \dots, \sigma_V$ (so $\sigma_V-\sigma$ choices).
    To get an indexing in terms of $b \in \Iset+1$ (such that
    $\lcS+1[b] \cap V_{\idx 1.q} \neq \emptyset$, so
    $\binom{\sigma_V}{\sigma +1}$ choices), note that each $\lcS+1[b]
    \cap V_{\idx 1.q}$ is contained in $\sigma +1$ distinct
    $\sigma$-lc centers $\lcS[p_{b,j}]$ for $j=1,\dots,\sigma+1$
    (apparently, $\sigma +1$ choices) such that
    \begin{equation*}
      \lcS+1[b] \cap V_{\idx 1.q} = \lcS[p_{b,j}] \cap \set{z_{b(j)} = 0} \; .
    \end{equation*}
    (One can verify $\sum_{p \in \Iset} \sum_{k=\sigma
      +1}^{\sigma_{V}} \dotsm = \sum_{b \in
      \Iset+1} \sum_{j=1}^{\sigma +1} \dotsm$ by first noting that
    $\binom{\sigma_V}{\sigma} (\sigma_V -\sigma)
    =\binom{\sigma_V}{\sigma +1} (\sigma+1)$.)
    With such choice of indexing, one has
    \begin{equation*}
      \frac{\rs*\gamma_{b;\: \idx 1.q}}{\sect_{(b)}}
      :=\PRes[\lcS+1[b]](\frac{\gamma_{\idx 1.q}}{\sect_D})
      =\sgn{b:p_{b,j}} \:
      \PRes[b(j)](\frac{\rs*\gamma_{p_{b,j};\:\idx
          1.q}}{\sect_{(p_{b,j})}})
    \end{equation*}
    (noticing that % $\sect_{(b)} =\sect_{(\sigma+1 : b)}$,
    % $\sect_{(p_{b,j})} =\sect_{(\sigma : p_{b,j})}$ and
    $\sect_{(p_{b,j})} = z_{b(j)} \sect_{(b)}$).
    As a result, the expression in question becomes
    \begin{align*}
      &-\smashoperator[l]{\sum_{\idx 1,q \in I}} \sum_{b \in \Iset+1}
        \sum_{j=1}^{\sigma +1} \sigma
        \iinner{ \sgn{b:p_{b,j}}\:
        \frac{\rs*\gamma_{b;\:\idx 1.q}}{\sect_{(b)}}
        }{\: 
        \idxup{\diff\rho},[\idx 1.q] .
        \PRes[b(j)](\idxup{\diff\psi_{(p_{b,j})}} . u_{p_{b,j}})
        }_{\lcS+1[b]}
      \\
      =&-\smashoperator[l]{\sum_{\idx 1,q \in I}}
        \sum_{b \in \Iset+1}
        \sigma
        \iinner{
        \paren{\dbar\rho}^{\idx q.1} \rs*\gamma_{b;\:\idx 1.q}
        \:}{  \sum_{j=1}^{\sigma +1} \sgn{b:p_{b,j}}\:
        \PRes[b(j)](\idxup{\diff\psi_{(p_{b,j})}} . u_{p_{b,j}})
        \cdot \sect_{(b)}
        }_{\lcS+1[b], \phi_{(b)}}
      \\
      =&-\sigma \sum_{b \in \Iset+1} \iinner{
        v_b
        \:}{\quad\;
        \smashoperator{\sum_{p\in\Iset \colon \lcS+1[b] \subset
        \lcS}} \;\;
        \sgn{b:p}\:
        \PRes[\lcS+1[b] | \lcS](\idxup{\diff\psi_{(p)}}.  u_{p})
        \cdot \sect_{(b)}
        }_{\lcS+1[b], \phi_{(b)}} \; . \qedhere
    \end{align*}
  \end{proof}


}


\subsection{Restriction of harmonic differential forms to hypersurfaces}\label{subsec:harmonic}

{
  
  Let $(X,\omega)$ be a K\"ahler manifold equipped with a holomorphic
  line bundle $L$ equipped with a smooth potential $\varphi_L$ such
  that $\ibddbar\varphi_L\ge0$ and let $D$ be an snc divisor in $X$
  written as 
  \begin{equation*}
    D=\sum_{p\in I_D}D_p \; ,
  \end{equation*}
  where $D_p$ is an irreducible component for $p\in I_D$.
  We define the map $\HRes_p \colon \mathscr
  A_X^{0,q}(X,K_X\otimes L)\rightarrow \mathscr
  A_{D_p}^{0,q-1}(D_p,K_{D_p}\otimes L\vert_{D_p})$ by 
  \begin{equation*}
    \HRes_p(u)
    =
    \PRes[D_p](\idxup{\partial\psi_D} . u) \;\;\;\text{for}\;\;\;p\in
    I_D \; ,
  \end{equation*}
  where $\mathcal{R}_{D_p}$ is the Poincar\'e residue map (see, for
  example, \cite{Griffiths&Harris}*{p.147} or \cite{Kollar_Sing-of-MMP}*{\S 4.18}). 
  Notice that, as in \cite{Chan&Choi_injectivity-I}*{\textsection2.6},
  the map $\mathcal R_{D_p}$ is extended to send sections of
  $K_X\otimes\overline{\bold\Omega}_X^q$ to those of
  $K_{D_p}\otimes\overline{\bold\Omega}_X^q\vert_{D_p}$. 
  Let $(U;z^1,\dots,z^n)$ be a local holomorphic coordinate system
  around $x\in D_p\subset X$ satisfying 
  \begin{enumerate}[label=(\roman*), ref=\roman*]
  \item  \label{item:admissible-open-U} % [(\romannumeral1)]
    $D_p\cap U=\{z\in U:z^1=0\}$ and 
    $D\cap U=\set{z^1\cdots z^{\sigma_U}=0}$,
    and
  \item  \label{item:psi_D-in-admissible-open-U} % [(\romannumeral2)]
    $\psi_D=\sum_{j=1}^{\sigma_U}
    \log\abs{z^j}^2-\sm\varphi_D$ on $U$.
  \end{enumerate}
  Since $\del\psi_D=\sum_{j=1}^{\sigma_U}\frac{dz^j}{z^j}-\del\sm\varphi_D$, it follows that
  \begin{equation*}
    \HRes_p(u)
    =
    \mathcal{R}_{D_p}\left(\idxup{\partial\psi_D}.u\right)
    =
    \mathcal{R}_{D_p}\paren{\idxup{\frac{dz^1}{z^1}}.u}
    =
    \paren{\idxup{dz^1}.\widetilde u_p}\big\vert_{D_p} \; ,
  \end{equation*}
  where $\rs u_p := \fdiff{z^1} \ctrt u$ (so $u=dz^1\wedge\widetilde{u}_p$).
  In particular, $\HRes_p$ does not depend on the
  choice of the Hermitian metric $\sm\varphi_{D}$. 
  It follows from the above formula that
  $\HRes_p(u)$ is actually a $K_{D_p}\otimes
  L\vert_{D_p}$-valued $(0,q-1)$-form on $D_p$ (not only a
  $\overline{\bold\Omega}_X^{q-1}\vert_{D_p}$-valued section). 



  First we notice that $\dfadj$-closedness is preserved by
  $\HRes_p$ on a K\"ahler manifold.
  \begin{prop} \label{prop:harmonic-residue}
    If $u$ is a $\dfadj$-closed $K_X\otimes L$-valued $(0,q)$-form on
    $X$, then $\HRes_p(u)$ is a $\dfadj$-closed
    $K_{D_p}\otimes L\vert_{D_p}$-valued $(0,q-1)$-form on $D_p$. 
  \end{prop}

  \begin{proof}
    It is enough to show that it vanishes at the given point $x\in D_p$.
    Let $(z^1,\dots,z^n)$ be a local holomorphic coordinate system around
    $x$ in $X$ satisfying \eqref{item:admissible-open-U} % (\romannumeral1)
    and \eqref{item:psi_D-in-admissible-open-U}. % (\romannumeral2).
    Since $(D_p,\omega\vert_{D_p})$ is a smooth $(n-1)$-dimensional
    K\"ahler manifold, by a linear change of coordinates
    $(z^1,\ldots,z^n)$ and a quadratic change of coordinates in
    $(z^2,\dots,z^n)$ of $D_p$ we may assume that
    \begin{enumerate}[resume*]
    \item % [(\romannumeral3)]
      $g_{i\bar j}(x)=I_n$ where $I_n$ is the $n\times n$ identity matrix.
    \item % [(\romannumeral4)]
      $dg_{\alpha\bar\beta}(x)=0$ for $2\le\alpha\le n$ and $2\le\beta\le n$.
    \end{enumerate}
    Since $\displaystyle\idxup{dz^1}=g^{\bar
      j1}\pd{}{\overline{z^j}}$ (under Einstein summation convention),
    we have
    \begin{equation}\label{E:local_expression}
      dz^1 \wedge \paren{\idxup{dz^1} . \widetilde u_p}_{\bar j_1,\ldots,\bar j_{q-1}}
      =
      g^{\bar j1}
      u_{\bar j\bar j_1,\ldots,\bar j_{q-1}} \; .
    \end{equation}
    % where $\beta_1,\ldots,\beta_{q-1}$ run from $2$ to $n$.
    For the sake of convenience, let the Latin indices $i,j,k,...$ run
    from $1$ to $n$ and let the Greek indices
    $\alpha,\beta,\gamma,...$ run from $2$ to $n$ in this proof. 
    Let $\varphi_{K_X}$ and $\varphi_{K_{D_p}}$ be respectively the
    potentials on $K_X$ and $K_{D_p}$ induced by the K\"ahler metric
    $\omega$, which are written as 
    \begin{equation*}
      \varphi_{K_X}
      =
      \log\det\paren{g_{i\bar j}}_{1\le i,j\le n}
      \;\;\;\text{and}\;\;\;
      \varphi_{K_{D_p}}
      =
      \log\det\paren{g_{\alpha\bar\beta}}_{2\le\alpha,\beta\le n} \; .
    \end{equation*}
    This yields, at the given point $x\in D_p$,
    \begin{align*}
      \del_\gamma\varphi_{K_X}
      &=
	\del_\gamma\log\det g
	=
	\Tr\paren{\del_\gamma g\cdot g^{-1}}
      \\
      &=
	\sum_{i,j=1}^n\pd{g_{i\bar j}}{z^\gamma} g^{\bar j i}
	\;\;\overset{\mathclap{(\text{at } x)}}=\;\;
	g^{\bar11}\del_\gamma g_{1\bar1}
	+
	\sum_{\alpha,\beta=2}^n\pd{g_{\alpha\bar\beta}}{z^\gamma}g^{\alpha\bar\beta}
      \\
      &=
	g^{\conj11}\del_\gamma g_{1\conj1}
	+
	\del_\gamma\log\det\paren{g\vert_{D_p}}
	=
	g^{\conj11}\del_\gamma g_{1\conj1}
	+
	\del_\gamma\varphi_{K_{D_p}}
	\;\;\overset{\mathclap{(\text{at } x)}}=\;\;
	g^{\conj11}\del_\gamma g_{1\conj1} \; .
    \end{align*}
    % This implies that
    % \begin{equation*}
    %   \del_\ell\varphi_{K_X}
    %   =
    %   \del_\ell\log\det\paren{g_{i\bar j}}_{1\le i,j\le n}
    %   =
    %   \sum_{i,j=1}^n\pd{g_{i\bar j}}{z^\ell}g^{\bar ji}
    %   \;\;\;\text{and}
    %   \;\;\;
    %   \del_\ell\varphi_{K_{D_p}}
    %   =
    %   \sum_{\alpha,\beta=2}^n
    %   \pd{g_{\alpha\bar\beta}}{z^\ell}g^{\bar\beta\alpha}.	
    % \end{equation*}
    % Thus we have
    % \begin{equation*}
    %   \sum_{i,j}\pd{g_{i\bar j}}{z^\ell}g^{\bar ji}
    %   =
    %   \del_lg_{1\bar1}
    %   +
    %   \sum_{\alpha=2}^n\pd{g_{\alpha\bar\alpha}}{z^l}g^{\alpha\bar\alpha}
    %   =
    %   \del_lg_{1\bar1}
    %   +
    %   \del_l\varphi_{K_{D_p}}
    %   \;\;\;
    %   \text{and}
    %   \;\;\;
    %   \del_\beta\varphi_{K_{D_p}}=0\;\;\;\text{at}\;\;x.
    % \end{equation*}
    % It follows from \eqref{E:local_expression} and the definition of $\dfadj$ (cf.~\cite{Siu}) that
    % for any multi-indices $\ov{\boldsymbol\beta}_{q-2}=(\bar\beta_1,\ldots,\bar\beta_{q-2})$,
    \newcommand{\bbeta}{{\boldsymbol{\beta}}}%
    \newcommand{\KDp}{{\smash[b]{K_{D_p}}}}%
    % \renewcommand{\CancelColor}{\color{Gray}}%
    Choose a local frame of $L$ in a neighbourhood of $x$ in $X$
    such that
    \begin{equation*}
      \diff\vphi_L(x) = 0  \; .
    \end{equation*}
    It follows from \eqref{E:local_expression} and the definition of
    $\dfadj$ on $D_p$ (see, for example,
    \cite{Siu}*{(1.3.2)}) that, for any 
    multi-indices ${\bbeta}_{q-2} =(\idx[\beta]1,{q-2})$ and at the
    given point $x \in D_p$,
    \begin{align*}
      dz^1 \wedge \paren{\dfadj \HRes_p(u)}_{\conj\bbeta_{q-2}}
      &=
        -g^{\conj\beta\gamma}
	\nabla_\gamma 
	\paren{
        g^{\alert{\conj j} 1}
        u_{\alert{\conj j} \conj\beta\ov{\boldsymbol\beta}_{q-2}}
	}
      \\
      &=-g^{\conj\beta \gamma}\diff_\gamma \paren{g^{\alert{\conj j} 1}
        u_{\alert{\conj j} \conj\beta \conj\bbeta_{q-2}}}
        +g^{\conj\beta \gamma} \smash[t]{\cancelto{0}{\paren{
        \diff_\gamma \vphi_\KDp +\diff_\gamma \vphi_L
        }}}
        \cdot g^{\alert{\conj j} 1} u_{\alert{\conj j} \conj\beta \conj\bbeta_{q-2}}
      \\
      &=-g^{\conj\beta \gamma} g^{\alert{\conj j} 1} \diff_\gamma
        u_{\alert{\conj j} \conj\beta \conj\bbeta_{q-2}}
        -g^{\conj\beta \gamma} \diff_\gamma g^{\alert{\conj j} 1}
        \cdot u_{\alert{\conj j} \conj\beta \conj\bbeta_{q-2}}
      \\
      &=-g^{\conj\beta \gamma} g^{\conj 1 1}
        \diff_\gamma u_{\conj 1 \conj\beta \conj\bbeta_{q-2}}
        +g^{\conj\beta \gamma} g^{\alert{\conj j k}} \diff_\gamma
        g_{\alert k \conj 1} \cdot g^{\conj 1 1} u_{\alert{\conj j}
        \conj\beta \conj\bbeta_{q-2}}
      \\
      &=g^{\conj 1 1} \paren{
        -g^{\conj\beta \gamma} \diff_\gamma
        u_{\conj 1 \conj\beta \conj\bbeta_{q-2}}
        +g^{\conj\beta\gamma} g^{\conj 1 1} \diff_\gamma
        g_{1\conj 1} \cdot
        u_{\conj 1 \conj\beta \conj\bbeta_{q-2}}
        +g^{\conj\beta\gamma} g^{\alert{\conj j} \alpha}
        \diff_\gamma g_{\alpha \conj 1} \cdot
        u_{\alert{\conj j} \conj\beta \conj\bbeta_{q-2}}
        }
      \\
      &=g^{\conj 1 1} \paren{
        g^{\alert{\conj k j}} \diff_{\alert{j}}
        u_{\alert{\conj k} \conj 1 \conj\bbeta_{q-2}}
        -g^{\alert{\conj k j}} \diff_{\alert{j}}\vphi_{K_X} \cdot
        u_{\alert{\conj k} \conj 1 \conj\bbeta_{q-2}}
        }
        +g^{\conj 1 1} g^{\conj \gamma \gamma} g^{\conj\alpha
        \alpha} \diff_{\gamma} g_{\alpha \conj 1} \cdot
        u_{\conj\alpha \conj\gamma \conj\bbeta_{q-2}}
      \\
      &=-g^{\conj 1 1} \paren{\dfadj u}_{\conj 1 \conj\bbeta_{q-2}}
        +g^{\conj 1 1} g^{\conj \gamma \gamma} g^{\conj\alpha
        \alpha} \diff_{\gamma} g_{\alpha \conj 1} \cdot
        u_{\conj\alpha \conj\gamma \conj\bbeta_{q-2}} \; .
    \end{align*}
    Since $\del_\gamma g_{\alpha\conj1}$ is symmetric in $\alpha,
    \gamma$ (for $X$ being K\"ahler) while $u_{\conj\alpha\conj\gamma
      \conj{\bbeta}_{q-2}}$ is anti-symmetric in $\alpha, \gamma$, the
    last term in the expression above vanishes.
    As $\dfadj u = 0$ on $X$ by assumption, the proof is thus
    completed after applying $\fdiff{z^1} \ctrt$ to both sides. \qedhere
    % \begin{align*}
    %   \paren{\dfadj\HRes_p(u)}_{\ov{\boldsymbol\beta}_{q-2}}
    %   &=
    %   -g^{\bar\beta\alpha}
    %   \nabla_\alpha 
    %   \paren{
    %   g^{\bar j1}
    %   u_{\bar j\bar\beta\ov{\boldsymbol\beta}_{q-2}}
    %	}
    %   \\
    %   &
    %   =
    %   -g^{\bar\beta\alpha}
    %   \del_\alpha 
    %   \paren{
    %   g^{\bar j1}
    %   u_{\bar j\bar\beta\ov{\boldsymbol\beta}_{q-2}}
    %	}
    %   +
    %   g^{\bar\beta\alpha}
    %   \paren{\del_\alpha\varphi_{K_{D_p}}-\del_\alpha\varphi_L}
    %   g^{\bar j1}
    %   u_{\bar j\bar\beta\ov{\boldsymbol\beta}_{q-2}}
    %   \\
    %   &
    %   =
    %   -
    %   \sum_{\beta=2}^n
    %   \del_\beta
    %   \paren{
    %   g^{\bar j1}
    %   u_{\bar j\bar\beta\ov{\boldsymbol\beta}_{q-2}}
    %	}
    %   -
    %   \sum_{\beta=2}^n
    %   \paren{\del_\beta\varphi_L}
    %   u_{\bar1\bar\beta\ov{\boldsymbol\beta}_{q-2}}\;\;\;\text{at}\;\;x.
    % \end{align*}
    % The first term is computed as
    % \begin{align*}
    %   -
    %   \sum_{\beta=2}^n
    %   \del_\beta
    %   &
    %   \paren{
    %   g^{\bar j1}
    %   u_{\bar j\bar\beta\ov{\boldsymbol\beta}_{q-2}}
    %	}
    %   =
    %   -
    %   \sum_{\beta=2}^n
    %   \paren{
    %   \del_\beta g^{\bar j1}
    %   u_{\bar j\bar\beta\ov{\boldsymbol\beta}_{q-2}}
    %   +
    %   g^{\bar j1}
    %   \del_\beta
    %   u_{\bar j\bar\beta\ov{\boldsymbol\beta}_{q-2}}
    %	}
    %   \\
    %   &=
    %   \sum_{\beta=2}^n
    %   \paren{
    %   g^{\bar jk}\paren{\del_\beta g_{k\bar l}}g^{\bar l1}
    %   u_{\bar j\bar\beta\ov{\boldsymbol\beta}_{q-2}}
    %   -
    %   \del_\beta
    %   u_{\bar1\bar\beta\ov{\boldsymbol\beta}_{q-2}}
    %	}
    %   \\
    %   &=
    %   \sum_{\beta=2}^n
    %   \paren{
    %   \paren{\del_\beta g_{1\bar1}}
    %   u_{\bar1\bar\beta\ov{\boldsymbol\beta}_{q-2}}
    %   -
    %   \del_\beta
    %   u_{\bar1\bar\beta\ov{\boldsymbol\beta}_{q-2}}
    %	}
    %   +
    %   \sum_{\beta,\gamma=2}^n
    %   \paren{\del_\beta g_{\gamma\bar1}}
    %   u_{\bar\gamma\bar\beta\ov{\boldsymbol\beta}_{q-2}}	.
    % \end{align*}
    % Since $\del_\beta g_{\gamma\bar1}$ is symmetric in $\beta, \gamma$ and $u_{\bar\gamma\bar\beta\bar\beta_1,\ldots,\bar\beta_{q-2}}$ is anti-symmetric in $\beta, \gamma$, the last term vanishes.
    % It follows that
    % \begin{align*}
    %   \paren{\dfadj\HRes_p(u)}_{\ov{\boldsymbol\beta}_{q-2}}
    %   &=
    %   \sum_{\beta=2}^n
    %   \paren{
    %   \paren{\del_\beta g_{1\bar1}}
    %   u_{\bar1\bar\beta\ov{\boldsymbol\beta}_{q-2}}
    %   -
    %   \del_\beta
    %   u_{\bar1\bar\beta\ov{\boldsymbol\beta}_{q-2}}
    %	}
    %   -
    %   \sum_{\beta=2}^n
    %   \paren{\del_\beta\varphi_L}
    %   u_{\bar1\bar\beta\ov{\boldsymbol\beta}_{q-2}}
    %   \\
    %   &=
    %   \paren{\dfadj u}_{\bar1\bar\beta\ov{\boldsymbol\beta}_{q-2}}=0.
    % \end{align*}
    % Indeed, at the given point $x$, we have
    % \begin{align*}
    %   \paren{\dfadj u}_{\bar1\ov{\boldsymbol\beta}_{q-2}}
    %   &=
    %   -g^{\bar jk}\paren{\nabla_ku_{\bar j\bar 1\ov{\boldsymbol\beta}_{q-2}}}
    %   =
    %   -g^{\bar jk}
    %   \paren{
    %   \del_ku_{\bar j\bar1\ov{\boldsymbol\beta}_{q-2}}
    %   -
    %   \paren{
    %			\del_k\varphi_{K_X}
    %			-
    %			\del_k\varphi_L
    % }
    %   u_{\bar j\bar1\ov{\boldsymbol\beta}_{q-2}}
    %	}
    %   \\
    %   &=
    %   -
    %   \sum_{\beta=2}^n
    %   \paren{
    %   \del_\beta
    %   u_{\bar\beta\bar1\ov{\boldsymbol\beta}_{q-2}}
    %   -
    %   \paren{
    %			\del_\beta g_{1\bar1}
    %			-
    %			\del_\beta\varphi_L
    % }
    %   u_{\bar\beta\bar1\ov{\boldsymbol\beta}_{q-2}}
    %	}.
    % \end{align*}
    % This completes the proof.
  \end{proof}


  Furthermore, we claim that, if
  % $u\in\mathcal{H}^{n,q}(X;L)_{\varphi_L}$ with
  % $\ibddbar\varphi_L\ge0$
  $u$ satisfies $\dbar u = 0$ and $\nabla^{(0,1)}u = 0$, then
  $\HRes_p(u)$ is $\dbar$-closed.
  % One can notice that $u$ is $\dbar$-closed, then so is
  % $\HRes_p(u)$. 
  This is shown via the following formula, which is a special case and
  a slight variant of \cite{Donnelly&Xavier}*{(2.4)} and
  \cite{Ohsawa&Takegoshi-spectral_seq}*{Prop.~1.5} (see also
  \cite{Takegoshi_higher-direct-images}*{(1.9)} and
  \cite{Matsumura_injectivity-Kaehler}*{Lemma 2.1}). 
  

  \newcommand{\lcSb}{\lcS+1[b]}
  \newcommand{\idxj}{\idx[\conj j]}
  
  \begin{lemma}[cf.~\cite{Donnelly&Xavier}*{(2.4)},
    \cite{Ohsawa&Takegoshi-spectral_seq}*{Prop.~1.5},
    \cite{Takegoshi_higher-direct-images}*{(1.9)} and
    \cite{Matsumura_injectivity-Kaehler}*{Lemma
      2.1}] \label{lem:commutator-dbar-ctrt}
    Let $\vphi$ be a smooth function and $u$ be a smooth
    ($K_X$-valued) $(0,q)$-form on a K\"ahler manifold.
    They satisfy the formula
    \begin{equation*}
      \dbar\paren{\idxup{\diff\vphi}.  u}
      =\idxup{\ibddbar\vphi} . u
      -\idxup{\diff\vphi} . \paren{\dbar u}
      +\idxup{\diff\vphi} \cdot \nabla^{(0,1)}_\bullet u \; ,
    \end{equation*}%
    or, when a local holomorphic coordinate system is fixed and
    the Einstein summation convention is applied, 
    \begin{equation*}
      \paren{\dbar\paren{\idxup{\diff\vphi} . u}}_{\conj J_{q}}
      =\sum_{\nu=1}^q \diff^{\conj\ell} \diff_{\conj j_\nu} \vphi \:
      u_{\idxj 1[\dotsm (\conj \ell)_\nu].q}
      -\diff^{\conj\ell}\vphi  \:\paren{\dbar u}_{\conj\ell\conj J_q}
      +\diff^{\conj\ell}\vphi \:\nabla_{\conj\ell} u_{\conj J_q} 
    \end{equation*}
    for any multi-indices $J_q = (\idx[j]1,q)$, pointwisely.
  \end{lemma}

  \begin{proof}
    A direct computation yields
    \begin{align*}
      \paren{\dbar\paren{\idxup{\diff\vphi} . u}}_{\conj
      J_{q}}
      &=\sum_{\nu=1}^q (-1)^{\nu-1} \diff_{\conj j_\nu}
        \paren{\idxup{\diff\vphi}.  u}_{\idxj 1[\dotsm \widehat
        {\conj j}_\nu].q}
        =\sum_{\nu=1}^q (-1)^{\nu-1} \diff_{\conj j_\nu}
        \paren{\diff_{\ell}\vphi \: u^\ell_{\;\idxj 1[\dotsm
        \widehat{\conj j}_\nu].q}}
      \\
      &=\sum_{\nu=1}^q (-1)^{\nu-1} \paren{
        \diff_{\conj j_\nu}\diff_{\ell}\vphi \: u^{\ell}_{\;\idxj 1[\dotsm
        \widehat {\conj j}_\nu].q}
        +\diff_{\ell}\vphi \: \nabla_{\conj j_\nu} u^\ell_{\;\idxj 1[\dotsm
        \widehat {\conj j}_\nu].q}
        }
      \\
      &=\sum_{\nu=1}^q
        \diff^{\conj \ell}\diff_{\conj j_\nu}\vphi \: u_{\idxj 1[\dotsm
        (\conj\ell)_\nu].q}
        -\diff^{\conj\ell}\vphi \sum_{\nu=1}^q (-1)^{\nu} 
        \nabla_{\conj j_\nu} u_{\conj\ell \idxj 1[\dotsm
        \widehat {\conj j}_\nu].q}
        \begin{aligned}[t]
          &-\diff^{\conj\ell}\vphi \: \nabla_{\conj \ell} u_{\conj
            J_q} \\
          &+\diff^{\conj\ell}\vphi \: \nabla_{\conj \ell}
          u_{\conj J_q}
        \end{aligned}
      \\
      &=\sum_{\nu=1}^q
        \diff^{\conj \ell}\diff_{\conj j_\nu}\vphi \: u_{\idxj 1[\dotsm
        (\conj\ell)_\nu].q}
        -\diff^{\conj\ell}\vphi
        \:\paren{\dbar u}_{\conj\ell\conj J_q}
        +\diff^{\conj\ell}\vphi \: \nabla_{\conj \ell} u_{\conj
        J_q} \; . \qedhere
    \end{align*}
    % as desired.
  \end{proof}



  We see that $\HRes_p(u)$ is $\dbar$-closed by
  putting $z^1$ in place of $\vphi$ in Lemma \ref{lem:commutator-dbar-ctrt}.
  The following theorem is then immediate.
  \begin{thm} \label{thm:residue-harmonic}
    If $u$ is a harmonic $K_X\otimes L$-valued $(0,q)$-form on $X$ with
    respect to $\vphi_L$ and $\omega$ such that $\nabla^{(0,1)}u=0$,
    then $\HRes_p(u)$ is a harmonic $K_{D_p}\otimes
    L\vert_{D_p}$-valued $(0,q-1)$-form on $D_p$ with respect to
    $\varphi\vert_{D_p}$ and $\omega\vert_{D_p}$.
  \end{thm}

  \begin{proof}
    From the above discussion, $\HRes_p(u)$ is
    $\dbar$- and $\dfadj$-closed on $D_p$.
    Since $\varphi_L$ is smooth and $D_p$ is compact, it follows that
    $\HRes_p(u)\in\Dom\dbar^*$ with
    respect to $\varphi_L\vert_{D_p}$ and $\omega\vert_{D_p}$.
    This completes the proof.
  \end{proof}

  \begin{remark}
    When $\varphi_L$ has the singularity property described in
    \cite{Chan&Choi_injectivity-I}*{\S 2.2 item (2)} for $\varphi_F$,
    i.e.~$\varphi_L$ has only neat analytic singularities such that
    $P_L:=\varphi_L^{-1}(-\infty)$ is a divisor with $P_L+D$ having
    snc and that $P_L$ contains no components of $D$, the claim that
    $\HRes_p(u)\in\Dom\dbar^*$ with
    respect to $\norm\cdot_{D_p}:=\norm\cdot_{D_p,\varphi_L,\omega}$
    still holds true (under the assumption that $\omega\vert_{D_p}$ is
    a complete K\"ahler form on $D_p\setminus P_L$).
    Indeed, $\HRes_p(u)$ can be shown to be $L^2$
    with respect to $\norm\cdot_{D_p}$ by the arguments in
    \cite{Chan&Choi_injectivity-I}*{Prop.~3.2.3, Remark 3.2.4 and
      Prop.~3.3.2} (with $u$ here in place of $\frac{\rs u}{\sect_D}$
    there).
    With $\dfadj$ (with respect to $\varphi_L\vert_{D_p}$ and
    $\omega\vert_{D_p}$) being a smooth operator on $D_p\setminus P_L$
    % (different from the situation in Lemma \ref{lem:su-harmonicity})
    and $\omega\vert_{D_p}$ being complete,
    $\HRes_p(u)\in\Dom\dbadj$ follows
    from the classical arguments.
  \end{remark}

}




% \section{Proof of the Main Result}\label{sec:proof}

% \subsection{Proof of Corollary \ref{cor:main}}\label{subsec:n2}

% Corollary \ref{cor:main} can be  proved by repeating the same argument as in the proof of Theorem \ref{thm:main}. 
% Nevertheless, in this subsection, we deduce Corollary \ref{cor:main} from Theorem \ref{thm:main} 
% using the previous work \cite[Theorem 1.6]{Mat}. 




% \begin{proof}[Proof of Corollary \ref{cor:main}]
% %In the proof, we freely use the notation in Conjecture \ref{conj:fujino}. 
% Let us consider the following commutative diagram  induced by 
% the short exact sequence $0 \to K_{X} \to K_{X}\otimes D \to K_{D} \to 0 $ and the multiplication map: 
% \begin{align*}
% \vcenter{ \xymatrix{
% &\ar[d] & \ar[d]\\
% &
% H^q(X,  K_{X}\otimes F )\ar[d]^-{}\ar[r] ^-{\otimes s}
% \ar[d]^-{\otimes \sect_D}
% &H^q(X, K_{X}  \otimes F^{ \otimes{(m+1)}} )\ar[d]\\ 
% &H^q(X,  K_{X}\otimes D \otimes F)
% \ar[d]^-{}\ar[r] ^-{\otimes s}
% &H^q(X,  K_{X} \otimes D \otimes F^{\otimes{(m+1)}}) \ar[d]\\ 
% &H^q(D, K_{D}\otimes F )
% \ar[d]\ar[r]^-{\otimes s|_{D} } 
% & H^q(D,  K_{D}\otimes F^{\otimes(m+1)} ). \ar[d]\\ 
% & & 
% }}
% \end{align*}
% The line bundle $M:=F^{\otimes m}$ with the metric $h_{M}:=h_{F}^{\otimes m}$ 
% satisfies the curvature assumption in Theorem \ref{thm:main}. 
% Further, the zero locus $s|_{D}^{-1}(0)$ contains no lc centers of $(X, D)$ by assumption; 
% hence, by Theorem \ref{thm:main}, the lowest multiplication map $\otimes s|_{D}$ in the diagram is injective for every $q$. 
% This implies that a cohomology class $\alpha \in H^q(X,  K_{X}\otimes D \otimes F)$ with 
% $s  \alpha =0 \in H^q(X,  K_{X}\otimes D \otimes F^{\otimes (m+1)})$ 
% lies in the image of the vertical multiplication map $\otimes \sect_D$ on the left, 
% where $\sect_D$ is the canonical section of the effective divisor $D$. 
% Then, the conclusion of $\alpha=0$ follows from \cite[Theorem 1.6]{Mat} (or \cite[]{CC}). 
% \end{proof}




\section{Proofs of the main results}\label{sec:proof}

\subsection{Proof of Corollary \ref{cor:main}}\label{subsec:n2}

Corollary \ref{cor:main} can be proved by adapting the % same argument as in the
proof of Theorem \ref{thm:main} (or, more precisely, Theorem
\ref{thm:ker-nu=ker-tau}; see Remark \ref{rem:general-commut-diagram}
for details).
The proof involves an inductive reduction of the setup to
subvarieties on which the relevant injectivity result is known
or can be proved via Enoki's arguments (i.e.~harmonic theory for
cohomology is valid).
To get an essence of the argument, here 
% Nevertheless, in this subsection,
we deduce Corollary \ref{cor:main} from Theorem \ref{thm:main} 
using the previous work \cite{Matsumura_injectivity-lc}*{Thm.~1.6} (or
\cite{Chan&Choi_injectivity-I}*{Thm.~1.2.1}).




\begin{proof}[Proof of Corollary \ref{cor:main}]
%In the proof, we freely use the notation in Conjecture \ref{conj:fujino}. 
Consider the following commutative diagram  induced by 
the short exact sequence $0 \to K_{X} \to K_{X}\otimes D \to K_{D} \to
0 $ and the multiplication map $\otimes s$: 
\begin{equation*}
  % \vcenter{
  \xymatrix@R=3ex{
    \ar[d] & \ar[d]\\
    {H^q(X,  K_{X}\otimes F )} \ar[d]^-{}\ar[r] ^-{\otimes s}
    \ar[d]^-{\otimes \sect_D}
    &{H^q(X, K_{X}  \otimes F^{ \otimes{(m+1)}} )} \ar[d]\\ 
    {H^q(X,  K_{X}\otimes D \otimes F)}
    \ar[d]^-{}\ar[r] ^-{\otimes s}
    &{H^q(X,  K_{X} \otimes D \otimes F^{\otimes{(m+1)}})} \ar[d]\\ 
    {H^q(D, K_{D}\otimes F )}
    \ar[d]\ar[r]^-{\otimes s|_{D} } 
    & {H^q(D,  K_{D}\otimes F^{\otimes(m+1)} ) \; .}  \ar[d]\\ 
    & 
  }
  % }
\end{equation*}
The line bundle $M:=F^{\otimes m}$ with the metric
$h_{M}:=h_{F}^{\otimes m} = e^{-m\vphi_F}$ satisfies the curvature
assumption in Theorem \ref{thm:main} and the zero locus $s^{-1}(0)$
contains no lc centers of $(X, D)$ by assumption.
Hence, by Theorem \ref{thm:main}, the lowest multiplication map $\otimes s|_{D}$ in the diagram is injective for every $q$. 
This implies that a cohomology class $\alpha \in H^q(X,  K_{X}\otimes D \otimes F)$ with 
$s  \alpha =0 \in H^q(X,  K_{X}\otimes D \otimes F^{\otimes (m+1)})$ 
lies in the image of the vertical multiplication map $\otimes \sect_D$ on the left, 
where $\sect_D$ is the canonical section of the effective divisor $D$. 
Then, the conclusion $\alpha=0$ follows from \cite{Matsumura_injectivity-lc}*{Thm.~1.6} 
(or \cite{Chan&Choi_injectivity-I}*{Thm.~1.2.1}). 
\end{proof}



\subsection{Proof of Theorem \ref{thm:main} for a simple case} % \label{subsec:n2}
\label{sec:proof-of-simple-case}

In this subsection, we prove Theorem \ref{thm:main} in the simple case 
where \emph{$D$ has two components (i.e.~$D=D_{1}+D_{2}$) whose intersection
has only one irreducible component} and the degree of cohomology
groups is $q=1$.
% This simple case is completely contained in the general case discussed in Section \ref{subsec:general}, 
% but we illustrate a detailed proof, which is quite helpful in understanding the essence of the proof. 
% The proof of the general case is an extension of the argument in this section using lc strata. 
While this case is contained in the proof presented in Section
\ref{subsec:general}, a detailed proof of it is presented here in
order to illustrate the essence of the proof in the general case
without being obscured by the notation.
The proof in Section \ref{subsec:general} follows the same arguments
but on the lower-dimensional lc strata with more components.


\begin{proof}[Proof of Theorem \ref{thm:main} in the case of $D=D_{1}+D_{2}$ and $q=1$] 

% Suppose that $D=D_{1}+D_{2}$ and $q=1$ in Theorem \ref{thm:main}. 
Under the given assumptions and for a given cohomology class $\alpha
\in H^1(D,  K_{D} \otimes F)$, we prove here that $\alpha $ is actually $0$
when $s  \alpha =0 \in H^1(D,  K_{D} \otimes F \otimes M)$. 

\begin{step}[``Harmonic representative'' of $\alpha$] \label{step:harmonic-rep}
We intend to work with an \emph{``optimal''} representative of $\alpha$ via the Dolbeault isomorphism, 
% which means an appropriate harmonic form has the minimum $L^{2}$-norm in the forms representing $\alpha$.  
in the analogy of a harmonic form being the element with the minimal
$L^2$ norm in the corresponding cohomology class.
Nevertheless, at the time of writing, there is not yet a well
established theory of Dolbeault isomorphism and harmonic theory for
cohomology groups on the singular space $D$.
For this purpose, we consider the following diagram
\begin{equation}\label{h}
  \begin{aligned}
    \xymatrix@C=3.5em@R=3.5ex{
      \ar[d] & \ar[d]\\
      {\smash{\bigoplus_{p=1}^{2}} H^1(D_{p}, K_{D_{p}}\otimes F )}
      \ar[d]^-{}\ar[r] ^-{\otimes (s|_{D_{1}}, s|_{D_{2}})}
      \ar[d]^-{\tau}
      &{\smash{\bigoplus_{p=1}^{2}} H^1(D_{p},
        K_{D_{p}} \otimes F \otimes M )}
      \ar[d]\\
      {H^1(D, K_{D} \otimes F)} \ar[d]^-{}\ar[r] ^-{\otimes s}
      &{H^1(D,  K_{D} \otimes F\otimes M)} \ar[d]\\
      {H^1(D_{1}\cap D_{2}, K_{D_{1}\cap D_{2}}\otimes F )}
      \ar[d]\ar[r]^-{\otimes s|_{D_{1}\cap D_{2}} }
      &{H^1(D_{1}\cap D_{2},  K_{D_{1}\cap D_{2}}\otimes F \otimes M)} \ar[d]\\
      & }
  \end{aligned}
\end{equation}
induced from $0 \to K_{D_{1}} \oplus K_{D_{2}} \to K_{D} \to K_{D_{1} \cap D_{2}} \to 0$, 
% which corresponds to $\eqref{eq-ex2}$ in the case of $\rho=0, \sigma=1, \tau=2$ (cf.\,\eqref{eq-ex}). 
which in turn can be obtained by tensoring $K_X \otimes D$ to the
short exact sequence of adjoint ideal sheaves
\begin{equation*}
  \renewcommand{\objectstyle}{\displaystyle}
  \xymatrix@R=3.5ex{
    {0} \ar[r]
    &{\faidlof|1|/|0|*} \ar[r] \ar[d]_-{\Res^1}^-{\isom}
    &{\faidlof|2|/|0|*} \ar[r]
    &{\faidlof|2|/|1|*} \ar[r] \ar[d]^-{\Res^2}_-{\isom}
    &{0}
    \\
    &{\residlof|1|*} %\ar@{}[u]|-*[left]+{\isom}
    &&{\residlof|2|*} %\ar@{}[u]|-*[left]+{\isom}
  }
\end{equation*}
where $\aidlof* :=\aidlof$ and $\residlof* := \residlof$ and
the isomorphism $\faidlof/-1* \xrightarrow[\isom]{\Res^\sigma}
\residlof*$ is induced from the residue short exact sequence in
Section \ref{subsec:residue}.
Notice that the Dolbeault isomorphism and harmonic theory are
valid on $D_1$, $D_2$ and $D_1 \cap D_2$.
The multiplication map $\otimes s|_{D_{1}\cap D_{2}}$ on the bottom
row is non-zero by the assumption on $s^{-1}(0)$ and the curvature
assumption is still satisfied after restricting $F$ and $M$ to
$D_{1}\cap D_{2}$.
Hence, Enoki's injectivity theorem can be invoked to assert that
$\otimes s|_{D_{1}\cap D_{2}}$ is injective. 
Then, by an easy diagram chasing, we can find harmonic forms $u_{p}$ for $p=1,2$ such that 
\begin{equation*}
  u_{p} \in \mathcal{H}^{n-1,1}(D_{p}; F)_{\vphi_F} \cong H^1(D_{p},  K_{D_{p}}\otimes F ) 
  \text{ with } \alpha = \tau(\eqcls{u_{1}}, \eqcls{u_{2}}), 
\end{equation*}
where $\mathcal{H}^{n-1,1}(D_{p}; F)_{\vphi_F}$ denotes 
the space of $F|_{D_{p}}$-valued harmonic forms of $(n-1,1)$-type with respect to $\res{e^{-\vphi_F}}_{D_{p}}$. 
Note that there is freedom in the choice of $(u_{1}, u_{2})$  
since $\tau$ may not be injective. 
% $(u_{1}, u_{2})$ may not be the best representation of  $\alpha$.
% For this reason, by the orthogonal decomposition, we re-
To obtain the unique ``optimal'' representative of $\alpha$, we choose
the pair $(u_{1}, u_{2})$ 
with $\alpha = \tau(\eqcls{u_{1}}, \eqcls{u_{2}})$ that satisfies
\begin{equation}\label{eq-orth}
  (u_{1}, u_{2}) \in (\Ker \tau)^{\perp} \subset 
  \Ker \tau \oplus (\Ker \tau)^{\perp}= \bigoplus_{p=1}^{2}
  \mathcal{H}^{n-1,1}(D_{p}; F)_{\vphi_F} \; ,
\end{equation}
% This can be regarded as the {\textit{best}} representation of  $\alpha$. 
% Our purpose is to prove that the $L^{2}$-norm 
% $\norm{s u_{1}}_{\vphi_M, D_1}^2 +\norm{s u_{2}}_{\vphi_M, D_2}^2$ 
% is actually zero. 
in which $\paren{\ker\tau}^\perp$ is the orthogonal complement of
$\ker\tau$ with respect to the (squared) residue norm
$\norm\cdot_{\lcc|1|'}^2 =\norm\cdot_{D_1}^2 +\norm\cdot_{D_2}^2$
(defined as in \eqref{eq:residue-norm} with $\sigma :=1$, $\vphi_L
:=\vphi_F$ and $\lcS[V,p] :=D_p$).
With such choice of representative, our goal is to prove that the
$L^{2}$-norm $\norm{s u_{1}}_{D_1, \vphi_M}^2 +\norm{s u_{2}}_{D_2,
  \vphi_M}^2$ is actually zero (where $\norm\cdot_{D_p, \vphi_M}$'s
are defined as in \eqref{eq:residue-norm} with $\vphi_L:=\vphi_F
+\vphi_M$).


\end{step}

\begin{step}[Obstruction for $\norm{s u_{1}}_{D_1, \vphi_M}^2 +\norm{s
      u_{2}}_{D_2, \vphi_M}^2$ from being zero]
  \label{item:expression-of-su-simple}
  

  % In this step, we examine the relevant $L^{2}$-norm 
  % to obtain an obstruction for our purpose as a $F$-valued form on $D_{1} \cap D_{2}$. 
  % For this purpose, by using the Dolbeault isomorphism, the Poincar\'e residue map, and the assumption of $s \alpha=0$, 
  % we prepare the following data: 
  We make use of the assumption $s \alpha=0$ and the \v
  Cech--Dolbeault isomorphism to re-express $\norm{s u_{1}}_{D_1,
    \vphi_M}^2 +\norm{s u_{2}}_{D_2, \vphi_M}^2$ as follows.


  \begin{itemize}
  \item Take $\alpha_{p;\:ij}   \in H^{0}(V_{ij} \cap D_p ,
    K_{D_{p}}\otimes F)$  
    for every open set $V_{ij} :=V_i \cap V_j$ with $i,j \in I$ and
    $V_i \cap V_j \cap D_p\neq \emptyset$ such that the family
    $\{\alpha_{p;\:ij}\}_{i,j \in I}$ is a \v Cech cocycle
    representing $u_{p}$ via the \v Cech--Dolbeault isomorphism on
    $D_p$.
    It follows that there exists an $L^2$ section $v_{p,(2)}$ of
    $K_{D_p} \otimes \res F_{D_p}$ on $D_p$ with respect to
    $\norm\cdot_{D_p}$ such that (under Einstein summation
    convention) 
    \begin{equation*}
      u_p
      \overset{\text{\eqref{eq:Cech-Dolbeault-isom}}}= \:
      \dbar v_{p,(2)} -\dbar\rho^j \cdot \rho^i \:\alpha_{p;\:ij} \; .
    \end{equation*}


  \item Take $f_{ij} \in H^{0}(V_{ij} , K_X \otimes D
    \otimes F \otimes \defidlof{D_1 \cap D_2})$ for $i,j \in I$
    satisfying 
    \begin{equation*}
      \Res^1\paren{f_{ij}}
      :=\paren{\PRes[D_1](\frac{f_{ij}}{\sect_D}) \:,\:
        \PRes[D_2](\frac{f_{ij}}{\sect_D})} 
      = \paren{\alpha_{1;\:ij} , \alpha_{2;\:ij}} 
    \end{equation*}
    whose existence are guaranteed by the surjectivity of the
    residue isomorphism $\Res^1$ on Stein open sets such that
    \begin{equation*}
      \renewcommand{\objectstyle}{\displaystyle}
      \xymatrix@C=1em@R=0.8em{
        {K_X \otimes D \otimes \frac{\defidlof{D_1 \cap
              D_2}}{\defidlof{D}}} 
        \ar@{}[r]|-*+{=}
        \ar@{}[d]|*[left]{\in}
        &{K_X \otimes D \otimes \faidlof|1|/|0|*}
        \ar[rr]^-{\Res^1}_-{\isom}
        &&{K_X \otimes D \otimes \residlof|1|*}
        \ar@{}[r]|-*+{=}
        &{K_{D_1} \oplus K_{D_2}}
        \ar@{}[d]|(.57)*[left]{\in}
        \\
        *+/r 3em/{f_{ij} \bmod \defidlof{D}}
        \ar@{|->}[rrr]
        &&&*+/l 6em/{}
        &*-{\paren{\alpha_{1;\:ij} , \alpha_{2;\:ij}} \; .}
      }
    \end{equation*}
    It is easy to see that $\set{f_{ij} \bmod \defidlof{D}}_{i,j \in
      I}$ is a \v Cech cocycle whose cohomology class in $\cohgp
    1[D]{\logKX \otimes \frac{\defidlof{D_1 \cap
          D_2}}{\defidlof{D}}}$ is mapped to $\alpha$ via $\tau$. 

  \item The assumption $s \alpha=0$ in $\cohgp 1[D]{K_D
      \otimes F \otimes M}$ guarantees the
    existence of $\lambda_{i} \in H^{0}(V_{i}, K_X \otimes D\otimes
    F \otimes M)$ for $i \in I$ such that
    \begin{equation*}
      s f_{ij} \equiv \lambda_j -\lambda_i \mod \defidlof{D}
      \quad\text{ on } V_{ij} \; .
    \end{equation*}
    Note that the coefficients of $\lambda_i$ need not lie in
    $\defidlof{D_1 \cap D_2}$ even though so do those of $f_{ij}$.
    By setting
    \begin{equation*}
      \rs*\lambda_{p;\:i} := \PRes[D_p](\frac{\lambda_i}{\sect_D})
      \cdot \sect_{(p)}
      \quad\text{ on $V_i \cap D_p$ for } i\in I
      \text{ and } p = 1,2 \; ,
    \end{equation*}
    it then follows that
    \begin{equation*}
      s\alpha_{p;\:ij} \sect_{(p)} =\rs*\lambda_{p;\:j} -\rs*\lambda_{p;\:i}
      \quad\text{ on } V_{ij} \cap D_p \; .
    \end{equation*}
    Note that $\rs*\lambda_{p;\:i}$ is holomorphic on $V_i \cap D_p$
    (while $\PRes[D_p](\frac{\lambda_i}{\sect_D})$ may not be).
  \end{itemize}

  Since $u_{p}$ is harmonic with respect to $\vphi_F$ on $D_p$ and
  we have $\ibddbar\vphi_F \geq 0$ and $%-C\omega \leq
  \ibddbar\vphi_M \leq C\ibddbar\vphi_F$ on $D_p$ for some constant
  $C > 0$ by assumption,
  Proposition \ref{prop:consequence-of-positivity} guarantees that  
  % \begin{equation*} %\label{eq-harmonic}
  %   su_p \in \Harm'/n-1,1/<D_p>{F\otimes M},{\vphi_F+\vphi_M} \; ,
  % \end{equation*}%
  $su_p$ is harmonic with respect to $\vphi_F+\vphi_M$ on $D_p$,
  which is a consequence of Nakano's identity and Enoki's argument.
  It follows that $\iinner{s \dbar v_{p;(2)}}{su_p}_{D_p, \vphi_M}
  =\iinner{\dbar\paren{s v_{p;(2)}}}{su_p}_{D_p, \vphi_M} = 0$.
  Summarizing the above discussion, it follows that
  \begin{align*}
    \norm{s u_{p}}_{D_p, \vphi_M}^2 
    &= -\sum_{i,j\in I}\iinner{\dbar\rho^{j} \cdot \rho^i \:s
      \alpha_{p;\:ij} \:}{\:s u_p}_{D_p, \vphi_M}\\
    &= -\sum_{i,j\in I}\iinner{\dbar\rho^{j} \cdot \rho^i \:s
      \alpha_{p;\:ij} \sect_{(p)} \:}{\:s u_p \sect_{(p)}}_{D_p, \vphi_M+\phi_{(p)}}\\
    &= -\sum_{i,j\in I}\iinner{\dbar\rho^{j} \cdot \rho^i
      \paren{\rs*\lambda_{p;\:j}- \rs*\lambda_{p;\:i}} \:}
      {\:s u_p \sect_{(p)}}_{D_p, \vphi_M+\phi_{(p)}}\\
    &= -\sum_{j\in I}\iinner{\dbar\paren{\rho^j
      \rs*\lambda_{p;\:j}} \:}{\: s u_p \sect_{(p)}}_{D_p,
      \vphi_M+\phi_{(p)}}
      =: -\iinner{\dbar v_{p;(\infty)} }{ s u_p \sect_{(p)}}_{D_p,
      \vphi_M+\phi_{(p)}}
      \; .
  \end{align*}
  The notation $v_{p;(\infty)} :=\sum_{j\in I} \rho^j
  \rs*\lambda_{p;\:j}$ is used for the consistency with the notation in
  Proposition \ref{prop:res-formula-dbar-exact-dot-harmonic}.

  The residue computation in Proposition
  \ref{prop:res-formula-dbar-exact-dot-harmonic} further brings the
  expression of $\norm{s u_{p}}_{D_p, \vphi_M}^2$ for each $p=1,2$ to an inner
  product on $D_1 \cap D_2$.
  As $\lcS|2|[b] :=D_1 \cap D_2$ has only $1$ component, the index set $\Iset|2|
  =\set{b}$ is a singleton.
  Moreover, the general different $\Diff_{D_1 \cap D_2}(D) =\Diff_b(D)$ is
  trivial, so we choose its canonical section and the corresponding
  potential such that $\sect_{(b)} \equiv 1$ and $\phi_{(b)} \equiv 0$
  (and $\psi_{(b)} \equiv -1$) on $D_1 \cap D_2$.
  Let $\PRes[\lcS|2|[b] | D_p]$ be the Poincar\'e residue map from
  $D_p$ to $D_1 \cap D_2$.
  We fix the sign convention such that
  \begin{equation*}
    \rs*\lambda_{b;\:i}
    =\frac{\rs*\lambda_{b;\:i}}{\sect_{(b)}}
    :=\PRes[\lcS|2|[b]](\frac{\lambda_i}{\sect_D})
    \begin{aligned}[t]
      &= \PRes[\lcS|2|[b] | D_1] \circ
      \PRes[D_1](\frac{\lambda_i}{\sect_D})
      \\
      &=\PRes[\lcS|2|[b] | D_1](\frac{\rs*\lambda_{1;\:i}}{\sect_{(1)}})
    \end{aligned}
    \begin{aligned}[t]
      &=-\PRes[\lcS|2|[b] | D_2] \circ
      \PRes[D_2](\frac{\lambda_i}{\sect_D})
      \\
      &=-\PRes[\lcS|2|[b] | D_2](\frac{\rs*\lambda_{2;\:i}}{\sect_{(2)}})
    \end{aligned} \; .
  \end{equation*}
  Following the computation in Proposition
  \ref{prop:res-formula-dbar-exact-dot-harmonic}, we obtain
  \begin{align*}
    &~\norm{s u_{1}}_{D_1, \vphi_M}^2 +\norm{s u_{2}}_{D_2,
      \vphi_M}^2
    \\
    =&~-\iinner{\dbar v_{1;(\infty)} }{ s u_1 \sect_{(1)}}_{D_1,
       \vphi_M+\phi_{(1)}}
       -\iinner{\dbar v_{2;(\infty)} }{ s u_2 \sect_{(2)}}_{D_2,
       \vphi_M+\phi_{(2)}}
    \\
    =&~
       \begin{multlined}[t]
         \sum_{i\in I}\iinner{\rho^i \PRes[\lcS|2|[b] |
           D_1](\frac{\rs*\lambda_{1;\:i}}{\sect_{(1)}}) }{
           \: s\:\PRes[\lcS|2|[b] | D_1](\idxup{\diff\psi_{(1)}}. u_1)
         }_{D_1 \cap D_2, \vphi_M} \\
         +\sum_{i\in I}\iinner{\rho^i
           \PRes[\lcS|2|[b] |
           D_2](\frac{\rs*\lambda_{2;\:i}}{\sect_{(2)}}) }{
           \: s\:\PRes[\lcS|2|[b] | D_2](\idxup{\diff\psi_{(2)}}. u_2)
         }_{D_1 \cap D_2, \vphi_M}
       \end{multlined}
    \\
    =&~\iinner{\sum_{i\in
       I}\rho^i\rs*\lambda_{b;\:i} \:}{\: s\:\paren{
       \PRes[\lcS|2|[b] | D_1](\idxup{\diff\psi_{(1)}}. u_1)
       -\PRes[\lcS|2|[b] | D_2](\idxup{\diff\psi_{(2)}}. u_2)
       }}_{D_1 \cap D_2, \vphi_M}
    \\
    =:&~\iinner{v_{b;(\infty)}}{s w_b}_{D_1 \cap D_2, \vphi_M} \; ,
  \end{align*}
  which is the desired expression.

  It is shown below that
  \begin{equation} \label{eq:w-prelim-formula}
    w_b :=\PRes[\lcS|2|[b] | D_1](\idxup{\diff\psi_{(1)}}. u_1)
    -\PRes[\lcS|2|[b] | D_2](\idxup{\diff\psi_{(2)}}. u_2)
  \end{equation}
  is actually $0$ on $D_1 \cap D_2$, which will then conclude the proof.

  % \begin{itemize}
  % \item[$\bullet$] Take $\beta_{ij,p}   \in H^{0}(V_{ij}, K_{D_{p}}\otimes F)$ 
  %   such that the family $\{\beta_{ij,p}\}$ is a cocycle  corresponding to $u_{p}$ via the \v Cech--Dolbeault isomorphism. 




  % \item[$\bullet$] Take $\alpha_{ij} \in H^{0}(V_{ij}, K_X \otimes D
  %   \otimes \defidlof{D_1 \cap D_2} \otimes F)$
  %   satisfying that 
  %   \begin{equation*}
  %     \Res^1\paren{\alpha_{ij}}
  %     :=\paren{\PRes[D_1](\frac{\alpha_{ij}}{\sect_D}) \:,\:
  %     \PRes[D_2](\frac{\alpha_{ij}}{\sect_D})} 
  %     = \paren{\beta_{ij, 1}, \beta_{ij, 2}}, 
  %   \end{equation*}
  %   by the  residue isomorphism: 
  %   \begin{equation*}
  %     \xymatrix@R=0.1cm{
  %     *+/r 0.5cm/{K_{D_1} \oplus K_{D_2}} &
  %     *+/r 0.5cm/{
  %     K_X \otimes D \otimes \frac{\defidlof{D_1 \cap D_2}}{\defidlof{D}}=K_X \otimes D \otimes \frac{\aidlof|1|*}{\aidlof|0|*}.
  %   }
  %     \ar[l]_-{\Res^1}^-{\isom}
  %   } 
  %   \end{equation*}
  %   More specifically, we may define $\alpha_{ij}$ 
  %   by $\alpha_{ij}:=d\sect_{(2)}  \wedge \sect_{(1)} \:\beta_{ij, 1} +d\sect_{(1)}  \wedge\sect_{(2)} \:\beta_{ij, 2}$. 


  % \item[$\bullet$] Take $\lambda_{i} \in H^{0}(V_{i}, K_X \otimes D\otimes F \otimes M)$ 
  %   satisfying that 
  %   $$\text{
  %   $ s \alpha_{ij} \equiv \lambda_j -\lambda_i$  as a section of 
  %   $K_X \otimes D\otimes \mathcal{O}_{X}/\defidlof{D} \otimes F \otimes M =K_D \otimes F \otimes M$. 
  % }
  %   $$
  %   The cocycle $\{\alpha_{ij}\}$ of $K_X \otimes D \otimes F$ (noting that $\defidlof{D_1 \cap D_2}$ is not tensored) 
  %   corresponds to $\alpha$; hence the assumption of $s \alpha=0$ guarantees the existence of $\lambda_{i}$. 



  % \item[$\bullet$] Take $\rs \lambda_i^1$, $\rs \lambda_i^2$, and $\rs\lambda_i^{12}$ such that 
  %   \begin{equation*}
  %     \lambda_i
  %     =d\sect_{(2)}  \wedge \rs \lambda_i^1
  %     =d\sect_{(1)}  \wedge \rs \lambda_i^2
  %     =d\sect_{(2)}  \wedge d\sect_{(1)}  \wedge \rs\lambda_i^{12}. 
  %   \end{equation*}
  % \end{itemize}
  % By construction, we see that 
  % \begin{align*}
  %   &\bullet \text{$u_{p}=\dbar u_{(2), p} + \dbar \rho^{i} \beta_{ij,p} $ for some global section $u_{(2), p}$ of $K_{D_{p}}\otimes F$};\\
  %   &\bullet s \sect_{(p)} \alpha_{ij}= \rs\lambda_j^p- \rs\lambda_i^p \text{ on } D_{p} 
  %   \text{ as a section of }K_{D_{p}}\otimes F \otimes M.
  % \end{align*}
  % On the other hand, since $u_{p}$ is harmonic and $\sqrt{-1}\Theta_{h_{F}} \geq 0$, 
  % we can conclude that 
  % \begin{align}\label{eq-harmonic}
  %   {\nabla^{(0,1)} u_{p}} =0  \text{ and } \sqrt{-1} \Theta_{h_{F}} \Lambda_{\omega} u_{p} =0
  % \end{align}
  % by applying Nakano's identity. 
  % Further, together with the curvature assumption, 
  % Enoki's argument shows that $su_{p}$ is still harmonic with respect to $h_{F}h_{M}$. 
  % Then, we can easily see that 
  % \begin{align*}
  %   \norm{s u_{p}}_{\vphi_M, D_p}^2 
  %   &= \iinner{\dbar s u_{(2), p}  +  \dbar s \rho^{i} \beta_{ij,p}}{s u}_{\vphi_M, D_p}\\
  %   &= \iinner{ \dbar s \sect_{(p)} \rho^{i} \beta_{ij,p}}{s \sect_{(p)} u}_{ \phi_{(p)}+\vphi_M, D_p}\\
  %   &= \iinner{\dbar \rho^{i} (\rs\lambda_j^p- \rs\lambda_i^p)}
  %   {s \sect_{(1)} u}_{\phi_{(p)}+\vphi_M, D_2}\\
  %   &= \iinner{-\dbar\paren{\rho^i \rs\lambda_i^p}}{s \sect_{(p)}
  %   v}_{\phi_{(p)}+\vphi_M, D_p}.
  % \end{align*}
  % Here we use that $s u$ is still harmonic to get the second equality 
  % and that $\dbar \rho^{i} \rs\lambda_j^p=\dbar \rs\lambda_j^p=0$ to get the third equality. 
  % The right-hand side can be described by the norm on $D_{1}\cap D_{2}$ as follows: 
  % \begin{align*} 
  %   &~- \iinner{\dbar\paren{\rho^i \rs\lambda_i^p}}{s \sect_{(p)} u}_{\phi_{(p)}+\vphi_M, D_p} \\
  %   \xleftarrow{\varepsilon \tendsto 0^+}
  %   &~-\iinner{e^{-\varepsilon \abs{\psi_{(p)}}}\dbar\paren{\rho^i \rs\lambda_i^p}}{s \sect_{(p)}
  %   u}_{\phi_{(p)}+\vphi_M, D_p} \\
  %   =
  %   &
  %   \begin{aligned}[t]
  %     &~-\cancelto{0}{
  %     \iinner{ \dbar\paren{e^{-\varepsilon
  %     \abs{\psi_{(p)}}} \rho^i \rs\lambda_i^p} }{ s \sect_{(p)} u}
  %   }_{\phi_{(p)}+\vphi_M, D_p} 
  %     +\varepsilon \iinner{
  %     e^{-\varepsilon \abs{\psi_{(p)}}} \rho^i \rs\lambda_{i}^p }{(\diff\psi_{(p)})^{*}s \sect_{(p)} u }_{\phi_{(p)}+\vphi_M, D_p}
  %   \end{aligned}
  %   \\
  %   =
  %   &~ \varepsilon \iinner{
  %   \frac{\rho^i \rs\lambda_{i}^p}{\sect_{(p)}}
  % }{
  %   e^{-\varepsilon \abs{\psi_{(p)}}}
  %   (\diff \log \abs{\sect_{(p)}^2})^{*}
  %   u \: s e^{-\vphi_M} }_{D_p}
  %   -\underbrace{
  %   \varepsilon \iinner{
  %   \frac{\rho^i \rs \lambda_{i}^p}{\sect_{(p)}}
  % }{
  %   e^{-\varepsilon \abs{\psi_{(p)}}}
  %   (\diff \sm\vphi_{(p)})^{*}  u \:s e^{-\vphi_M} }_{D_p}
  % }_{=\: \BigO(\varepsilon)}
  %   \\
  %   =
  %   &~\varepsilon \iinner{
  %   \rho^i \smash[b]{\underbrace{\rs\lambda_{i}^p}_{\mathclap{=\: d\sect_{(1)} 
  %   \wedge \rs\lambda_i^{12}}}}
  %   \:
  % }{ \:
  %   \frac{e^{-\varepsilon \abs{\psi_{(p)}}}}{\abs{\sect_{(p)}}^2}
  %   (d\sect_{(p)})^{*} \smash[b]{\underbrace{u_{p}}_{\mathclap{=:\: d\sect_{(1)}  \wedge \rs u_{p}^{12}}}} \: s e^{-\vphi_M} }_{D_p}
  %   + \BigO(\varepsilon)
  %   \vphantom{\underbrace{\rs\lambda_{i}^1}_{\mathclap{=\: d\sect_{(1)} 
  %   \wedge \rs\lambda_i^{12}}}}
  %   \\
  %   \xrightarrow{\varepsilon \tendsto 0^+}
  %   &~\iinner{\rho^i \rs\lambda_i^{12}}{  (d\sect_{(p)})^{*}  \rs u_{p}^{12}
  %   \: s e^{-\vphi_M}}_{D_1 \cap D_2}  \; .
  % \end{align*}

  % Considering the inner product above, we define the $F$-valued form $w$ on $D_{1} \cap D_{2}$ by 
  % \begin{equation*}
  %   w:=(d\sect_{(1)})^{*}  \rs u_{1}^{12}-(d\sect_{(2)})^{*} \rs u_{2}^{12}. 
  % \end{equation*}
  % From the next step, we aim to show $w$ is actually zero, which finishes the proof. 

  % Note that $u_{p}^{12}$ and $d\sect_{(p)}$ are defined only locally; 
  % hence $d\sect_{(p)})^{*}  \rs u_{p}^{12}$ does not determine a section on $D_{p}$, 
  % but determines the $F$-valued section form of type $(n-2, q-1)$ on $D_{1} \cap D_{2}$. 
  % This can be verified by calculating a glueing condition. 
  % Another way to see this is to apply The Poincar\'e residue map from $D_{p}$ to $D_1 \cap D_2$, 
  % which yields
  % \begin{equation*}
  %   \PRes[D_1 \cap D_2]( (\diff\psi_{(p)})^{*} u)
  %   =\parres{(d\sect_{(p)})^{*}  \rs u_{p}^{12}}_{D_1 \cap D_2}
  %   \quad\text{(recall that $\sect_{(p)} =\sect_{(p)} $)} \; .
  % \end{equation*}
  % Since $\psi_{(p)}=\phi_{(p)} -\sm\vphi_{(p)}$ is a global function, 
  % the right hand side is globally defined on $D_{p}$, and so is $((d\sect_{(p)})^{*}  \rs u_{p}^{12})|_{D_1 \cap D_2}$. 
\end{step}



\begin{step}[$w_b$ being holomorphic and thus {$w_b \in \cohgp 0[D_1
    \cap D_2]{K_{D_1 \cap D_2} \otimes F}$}]
  
  We prove that $\dbar w_b = 0$ on $\lcS|2|[b] :=D_1 \cap D_2$ by a
  direct computation given in Section \ref{subsec:harmonic}.
  Indeed, it suffices to show that each summand $\PRes[\lcS|2|[b] |
  D_p](\idxup{\diff\psi_{(p)}}. u_p)$ for $p=1,2$ in $w_b$ is
  $\dbar$-closed.
  The computations are identical, so it suffices to consider $p=1$.

  On an admissible open set $V$ such that $D_p  \cap V =\set{z_p =
    0}$ for $p=1,2$ and $\lcS|2|[b] \cap V = \set{z_1 = z_2 = 0} =
  D_1 \cap \set{z_2 = 0}$,
  we have
  \begin{equation*}
    \diff\psi_{(1)} =\frac{dz_2}{z_2} -\diff\sm\vphi_{(1)}
    \quad\text{ on } V \; .
  \end{equation*}
  By writing
  \begin{equation*}
    \idxup{dz_2}. u_1 =: dz_2 \wedge \paren{\idxup{dz_2}. \rs*u_{1,2}}
    \quad\text{ on } D_1 \cap V \; ,
  \end{equation*}
  where $\rs*u_{1,2}$ is a $(n-2,1)$-form on $D_1 \cap V$, we see
  that
  \begin{equation*}
    \PRes[\lcS|2|[b] | D_1](\idxup{\diff\psi_{(1)}}. u_1)
    =\PRes[\set{z_2 = 0}](\frac{\idxup{dz_2} .u_1}{z_2})
    =\parres{\idxup{dz_2}. \rs*u_{1,2}}_{\lcS|2|[b]}
    \quad\text{ on } D_1 \cap D_2 \cap V \; .
  \end{equation*}
  Therefore, it suffices to check that $\idxup{dz_2}. u_1$ is
  $\dbar$-closed on $D_1 \cap V$.
  As $u_1$ is harmonic and $\ibddbar\vphi_F \geq 0$, we have
  $\nabla^{(0,1)} u_1 = 0$ by Proposition
  \ref{prop:consequence-of-positivity} and Lemma
  \ref{lem:commutator-dbar-ctrt} yields the desired result (with $z_2$
  in place of $\vphi$ in the lemma).

  % In this step, we show that $w$ is a $F$-valued harmonic on $D_{1} \cap D_{2}$. 
  % Note that it is sufficient to show that $\dbar w =0$ in our case since the type of $w$ is $(n-2, q-1)=(n-2, 0)$ by $q=1$. 
  % By \cite[(1.9)]{Takegoshi_higher-direct-images} and \eqref{eq-harmonic}, we obtain that 
  % \begin{equation*}
  %   \dbar ( (d\sect_{(p)})^{*} u_{p}) 
  %   = \big( (i \partial  \dbar \sect_{(p)})^{*} - (\partial \sect_{(p)})^{*} \dbar + \partial \sect_{(p)}\nabla^{(0,1)} \big)u_{p}
  %   = 0 \quad\text{on   } D_1. 
  % \end{equation*}
  % By noting that $u_{p} = d \sect_{(1)}  \wedge \rs u_{p}^{12}$, 
  % we see that  $\dbar (d\sect_{(p)})^{*}  \rs u_{p}^{12} =0 $; hence $\dbar w =0$. 
  % In particular, $w$ determines the cohomology class $\{w \} \in H^{0}(D_{1}\cap D_{2}, K_{D_1 \cap D_2} \otimes F)$. 
\end{step}


\begin{step}[$w_b = 0$ and conclusion of the proof]
  \label{step:pf:use_u-ortho-w-simple}
We prove that $w_b =0$ using the assumption $(u_{1},u_{2}) \in
\paren{\ker \tau}^\perp$.
Consider the connecting morphism $\delta$ the long exact sequence 
\begin{equation*}
  \xymatrix@R=0.3cm@C=1.5em{
    {\to \cohgp 0[D_{1}\cap D_{2}]{K_{D_1 \cap D_2} \otimes F}} \ar[r]^-{\delta}
    &
    {\bigoplus_{p=1}^{2} \cohgp 1[D_{p}]{K_{D_p}\otimes F}} \ar[r]^-{\tau}  
    &
    {\cohgp 1[D]{K_D \otimes F}  \to} \; . 
  } 
\end{equation*}
Note that $\delta w_b \in \ker\tau$.

We compute $\delta w_b$ via the \v Cech--Dolbeault isomorphism.
% $\rho(w)$ in terms of \v Cech cohomology. 
Regard $w_b$ as a $0$-cocycle $\set{\rs \gamma_{b;\:i}}_{i \in I}$
given by $\rs \gamma_{b;\:i} :=\res{w_b}_{V_i}$.
Lift $\rs \gamma_{b;\:i}$ on $D_1 \cap D_2 \cap V_i$ to a section
$\gamma_i$ on $V_{i}$ via the isomorphism
$\frac{\holo_X}{\defidlof{D_1 \cap D_2}} = \faidlof|2|/|1|*
\xrightarrow[\isom]{\Res^2} \residlof|2|*$ such that
\begin{equation*}
  % \gamma_i = d\sect_{(2)}  \wedge d\sect_{(1)}  \wedge \rs \gamma_i \; .
  \Res^2\paren{\gamma_i}
  =\PRes[\lcS|2|[b]](\frac{\gamma_i}{\sect_D})
  =\frac{\rs*\gamma_{b;\:i}}{\sect_{(b)}} =\rs*\gamma_{b;\:i} \; .
\end{equation*}
Then $\delta w_b$ is represented by the $1$-cocycle
\begin{equation*}
  \delta\set{\gamma_i \bmod \defidlof{D_1 \cap D_2}}_{i \in I}
  =\set{(\delta  \gamma)_{ij} \bmod\defidlof{D}}_{i,j \in I}
  =\set{\gamma_{j} -\gamma_i  \bmod\defidlof{D}}_{i,j \in I} \; .
\end{equation*}
Note that $ \gamma_{j} -\gamma_i$ belongs to $\defidlof{D_1 \cap
  D_2}$, so $ \gamma_{j} -\gamma_i  \bmod\defidlof{D}$ can be realized
via the isomorphism $\frac{\defidlof{D_1 \cap D_2}}{\defidlof{D}}
=\faidlof|1|/|0|* \xrightarrow[\isom]{\Res^1} \residlof|1|*$ as
\begin{align*}
  \Res^1\paren{\gamma_{j} -\gamma_i}
  &=\paren{\PRes[D_1](\frac{\gamma_{j} -\gamma_i}{\sect_D})
    \: ,\:
    \PRes[D_2](\frac{\gamma_{j} -\gamma_i}{\sect_D})
    } \\
  &=\paren{
    \frac{(\delta \rs\gamma_1)_{ij}}{\sect_{(1)}}
    \: , \:
    \frac{(\delta \rs\gamma_2)_{ij}}{\sect_{(2)}}
    }
    \in K_{D_1} \otimes \res F_{D_1} \oplus K_{D_2} \otimes \res F_{D_2} \; ,
\end{align*}
in which $\rs*\gamma_{p;\:i} := \PRes[D_p](\frac{\gamma_i}{\sect_D})
\cdot \sect_{(p)}$ for $p = 1,2$.
Therefore, via the \v Cech--Dolbeault isomorphism on each $D_p$, 
the component of $\delta w_b$ on $D_p$ can be represented by (under
Einstein summation convention) 
\begin{equation*}
  -\dbar\rho^j \cdot \rho^i
  \frac{\paren{\delta\rs*\gamma_p}_{ij}}{\sect_{(p)}}
  =-\frac{\dbar\rho^j \cdot\rs*\gamma_{p;\:j}}{\sect_{(p)}}
  =: -\frac{\dbar v'_{p;(\infty)}}{\sect_{(p)}}
  % \paren{
  %   \res{\frac{\dbar\rho^i \:(\delta \gamma^1)_{ij}}{\sect_{(1)} }}_{D_1}
  %   \: , \:
  %   \res{\frac{\dbar\rho^i \: (\delta \gamma^2)_{ij}}{\sect_{(2)} }}_{D_2}
  % }
  % =\paren{
  %   -\res{\frac{\dbar\paren{\rho^i  \gamma^1_{i}}}{\sect_{(1)}}}_{D_1}
  %   \: , \:
  %   -\res{\frac{\dbar\paren{\rho^i  \gamma^2_{i}}}{\sect_{(2)}}}_{D_2}
  % } \; .
\end{equation*}
(the notation $v'_{p;(\infty)} :=\sum_{i\in I}\rho^i
\rs*\gamma_{p;\:i}$ is set for the consistency with the notation in
Proposition \ref{prop:res-formula-dbar-exact-dot-harmonic}).
% For the computation of the norm, 
% we take $\rs \gamma_i^p$ and $\rs\gamma_i^{12}$ such that 
% \begin{equation*}
% \gamma_{i} = d \sect_{(p)} \wedge \rs \gamma_i^p 
%   \quad\text{and}\quad 
%   \rs\gamma_i^{12} = \gamma_i. 
% \end{equation*}
Recall the sign convention chosen in Step
\ref{item:expression-of-su-simple} such that
\begin{equation*}
  \rs*\gamma_{b;\:i}
  = \PRes[\lcS|2|[b] | D_1](\frac{\rs*\gamma_{1;\:i}}{\sect_{(1)}})
  =- \PRes[\lcS|2|[b] |
  D_2](\frac{\rs*\gamma_{2;\:i}}{\sect_{(2)}}) \; .
\end{equation*}
Then, from $(u_{1},u_{2}) \in \paren{\ker\tau}^\perp$ and $\delta w_b
\in \ker\tau$, we obtain
\begin{align*}
  0
  &=
  \iinner{
    -\frac{\dbar v'_{1;(\infty)}}{\sect_{(1)}}
  }{u_1}_{D_1}
  +\iinner{
    -\frac{\dbar v'_{2;(\infty)}}{\sect_{(2)}}
  }{u_2}_{D_2}
  \\
  &=
    \iinner{
    -\dbar v'_{1;(\infty)}
    }{u_1 \sect_{(1)}}_{D_1, \phi_{(1)}}
    +\iinner{
    -\dbar v'_{2;(\infty)}
    }{u_2 \sect_{(2)}}_{D_2, \phi_{(2)}}
  \\
  &\overset{\mathclap{\text{Prop.~\ref{prop:res-formula-dbar-exact-dot-harmonic}}}}=
    \quad\;\;
    \iinner{\rho^i \rs*\gamma_{b;\:i} \:}{\:
    \PRes[\lcS|2|[b] | D_1](\idxup{\diff\psi_{(1)}}. u_1)
    -\PRes[\lcS|2|[b] | D_2](\idxup{\diff\psi_{(2)}}. u_2)
    }_{D_1\cap D_2}
  \\
  &=\iinner{w_b}{w_b}_{D_1 \cap D_2}
    =\norm{w_b}_{D_1 \cap D_2}^2
    \; .
\end{align*}
% By the same computation as in Step 2, 
% the right hand side can be described by the norm of $w$ as follows: 
% \begin{equation*}
%   0=\iinner{\rho^i  \gamma_i^{12} \:}{\:
%     (d\sect_{(1)})^{*}    u^{12} -(d\sect_{(2)})^{*}   v^{12}
%   }_{D_1 \cap D_2}
%   =\iinner{\rho^i \gamma_i}{w}_{D_1 \cap D_2}
%   = \norm w_{D_1 \cap D_2}^2. 
% \end{equation*}
This implies that $w_b=0$, finishing the proof for the case
$D=D_{1}+D_{2}$ and $q=1$. \qedhere
\end{step}
\end{proof}



\subsection{Remarks on the general case}
% \subsection{Strategy of the proof in the general case}
\label{subsec:n3}

There are two modifications to the proof in Section
\ref{sec:proof-of-simple-case} in order to handle the general case
worth mentioning here.
The first one is the replacement of the short exact sequence $0 \to
K_{D_1} \oplus K_{D_2} \to K_D \to K_{D_1 \cap D_2} \to 0$.
Take the case $D = D_1 + D_2 + D_3$, where $D_p = \set{z_p = 0}$ for
$p=1,2,3$ are the coordinate planes, for example.
Note that
\begin{equation*}
  \aidlof|3|* = \holo_X \;, \;\;
  \aidlof|2|* = \defidlof{D_1 \cap D_2 \cap D_3} \;, \;\;
  \aidlof|1|* = \smashoperator{\bigcap_{\substack{1 \leq p,q \leq 3 \\ p\neq q}}} \defidlof{D_p \cap D_q} \;
  \text{ and } \;
  \aidlof|0|* = \defidlof{D} 
\end{equation*}
in this case.
A natural choice of the short exact sequence to be considered is
\begin{equation*}
  \renewcommand{\objectstyle}{\displaystyle}
  \xymatrix@R=2.5em{
    0 \ar[r]
    &{K_X \otimes D \otimes \smash{\faidlof|1|/|0|*}} \ar[r]
    \ar[d]^(0.47){\Res^1}_(0.47){\isom}
    &{K_X \otimes D \otimes \smash{\faidlof|3|/|0|*}} \ar[r]
    \ar@{=}[d]
    &{K_X \otimes D \otimes \faidlof|3|/|1|*} \ar[r]
    &0 \; .
    \\
    &{\smash{\bigoplus_{p = 1}^3}\:K_{D_p}} \ar[r]
    &{K_D} 
    &
  }
\end{equation*}
In the previous case, we are taking advantage of the fact that the
$L^2$ Dolbeault isomorphism and the harmonic theory are valid on the
cohomology groups of the sheaves on both the left- and
right-hand-sides of the short exact sequence, so that the
corresponding injectivity statement can be proved on each side in
the spirit of Enoki, which in turn leads to the injectivity theorem
for the cohomology groups of the middle sheaf (twisted by $F$).
In the current case, they are valid only on the left-hand-side (on
each $D_p$).
We are thus led to determine whether the injectivity statement for
the sheaf on the right-hand-side holds true.
It is then apparent that we should consider
\begin{equation*}
  \renewcommand{\objectstyle}{\displaystyle}
  \xymatrix@R=2.5em{
    0 \ar[r]
    &{K_X \otimes D \otimes \smash{\faidlof|2|/|1|*}} \ar[r]
    \ar[d]_(0.47){\Res^2}^(0.47){\isom}
    &{K_X \otimes D \otimes \faidlof|3|/|1|*} \ar[r]
    &{K_X \otimes D \otimes \smash{\faidlof|3|/|2|*}} \ar[r]
    \ar[d]^-{\Res^3}_-{\isom}
    &0 \; ,
    \\
    &{\smash[t]{\bigoplus_{\substack{p,q = 1 \\ p\neq q}}^3} K_{D_p \cap D_q}} 
    &
    &{K_{D_1 \cap D_2 \cap D_3}}
  }
\end{equation*}
which, again, has the Dolbeault and harmonic theories valid on both
sides (on each lc center of $(X,D)$) of the short exact sequence.
The arguments in Section \ref{sec:proof-of-simple-case} can then be
employed to conclude the proof.
This illustrates the idea of the inductive arguments, which reduces
the question to the union of lower dimensional lc centers of $(X,D)$
in each step, to be employed in the general proof in Section
\ref{subsec:general}. 

Another modification to the proof in Section
\ref{sec:proof-of-simple-case} is that, when the claim in Theorem
\ref{thm:main} with $q > 1$ is considered, the section $w_b$
constructed as in \eqref{eq:w-prelim-formula} is then a $K_{\lcS+1[b]}
\otimes \res F_{\lcS+1[b]}$-valued $(0,q-1)$-form on some
$(\sigma+1)$-lc center $\lcS+1[b]$.
In order to prove that $w_b =0$ by following the arguments in the
previous case, we need not only to show that $w_b$ is
$\dbar$-closed, but also that it is harmonic.
This happens to be true and the computation for checking this claim
is given in Proposition \ref{prop:harmonic-residue} and Theorem
\ref{thm:residue-harmonic}.



% In this subsection, we consider how we should generalize the proof of the previous section in dealing with the general case. 

% We first consider the slightly more general case of $D=D_{1}+D_{2}$ and $q \geq 2$. 
% In this case,  we can repeat the same argument  for Step 1. 
% Step 2 is a bit more involved since we are dealing with differential forms $u_{p}$ of higher degree, 
% but essentially the same argument can be used to define $w$ appropriately (see $\eqref{eq-def-w}$). 
% To check that $w$ is harmnic in Step 3, 
% since $w$ is an $F$-valued of the type $(n-2, q-1) \not =(n-2, q-1)$, 
% we need to check $\dfadj w_q = 0$ as well as $\dbar w=0$. 
% Nevertheless, $\dbar w=0$ is proved by the same argument 
% and $\dfadj w_q = 0$ is proved in Subsection \ref{subsec:harmonic}. 
% Performing Step 4 in the same way, 
% some global section $v$ on $D_{1}\cap D_{2}$ naturally appears, 
% we finally obtain $0=\iinner{w-\dbar v }{w}_{D_{1}\cap D_{2}} $. 
% Although the point that $v$ appears is different, since $w$ is harmonic, the conclusion that $w=0$ is immediately obtained.
% As described above, in the case of $q \geq 2$, 
% the degree of the differential form is higher and more involved, 
% but essentially the same strategy still work. 


% Next, let us consider the case of $D=D_{1}+D_{2}+D_{3}$. 
% In the case of $D=D_{1}+D_{2}$ , 
% by using the exact sequence $0 \to K_{D_{1}} \oplus K_{D_{1}} \to K_{D} \to K_{D_{1} \cap D_{2} } \to 0$, 
% we proved the injectivity of the multiplication map on (the cohomology groups of) central term. 
% Of particular importance in the proof were that the left term admits the theory of harmonic integrals and 
% that the multiplication map on the right term in injective. 
% In this subsection, we explain what kind of exact sequences in the case where $D$ has three components  
% to make the same strategy works. 
% The precise proof will be given  in the next subsection. 

% Suppose that $D$ has three components (i.e.\,$D=D_{1}+D_{2}+D_{3}$). 
% We first consider the following exact sequence twisted by $F$ (and also $M$): 
% \begin{align}\label{eq-ex}
%   \xymatrix{
%     0 \ar[r]
%     & K_{X}\otimes D \otimes{\faidlof |1|/|0|*} =\bigoplus_{p=1}^{3} K_{D_{p}} \ar[r]
%     & K_{X}\otimes D \otimes{\faidlof|\sigma_{\mlc}|/|0|*} =K_{D}        \ar[r]
%     & K_{X}\otimes D \otimes{\faidlof|\sigma_{\mlc}|/|1|*} \ar[r]
%     & 0. 
%   } 
% \end{align}
% The cohomology classes in $\oplus_{p=1}^{3} H^{q}(D_{p}, K_{D_{p}} \otimes F)$ of the left term 
% can be represented by harmonic forms. 
% Hence, whether the same argument works as in the previous subsection 
% depends on whether or not the multiplication map 
% $$
% H^{q}(X, K_{X}\otimes{\faidlof|\sigma_{\mlc}|/|1|*}\otimes F) \xrightarrow{\quad \otimes s \quad }
% H^{q}(X, K_{X}\otimes{\faidlof|\sigma_{\mlc}|/|1|*}\otimes F\otimes M)
% $$
% is injectivity or not.  
% To check this, we  consider another exact sequence: 
% \begin{align*}
%    \xymatrix{
%     0 \ar[r]
%     & K_{X}\otimes{\faidlof |2|/|1|*}  =\bigoplus_{p\not = q} K_{D_{p} \cap D_{q}} \ar[r]
%     & K_{X}\otimes{\faidlof|\sigma_{\mlc}|/|1|*}         \ar[r]
%     & K_{X}\otimes{\faidlof|\sigma_{\mlc}|/|2|*}=K_{D_{1}\cap D_{2}\cap D_{3}} \ar[r]
%     & 0}
% \end{align*}
% Then, the left term admits the theory of harmonic integrals and 
% that the multiplication map on the right term in injective.
% Therefore, we can show that he multiplication map on the central term is injective. 
% No essential difficulty appears in repeating this inductive argument in the general case. 


\subsection{Proof of Theorem \ref{thm:main} in general}\label{subsec:general}


%\input{outline-of-proof}

%%%%%
%%%%% File name  : outline-of-proof.tex
%%%%% Author     : Mario Chan
%%%%% Date       : 6th March, 2023
%%%%% Description: This is the outline of the proof of the general
%%%%%              case of the project "Injectivity-Fujino".
%%%%%
%%
%%%

\renewcommand{\objectstyle}{\displaystyle}

Write
\begin{gather*}
  \aidlof* := \aidlof =\mtidlof{\vphi_F} \cdot \defidlof{\lcc+1'}
  =\defidlof{\lcc+1'} \; , \quad
  \residlof* := \residlof \isom \faidlof/-1*  \\
  \text{and } \quad \spH{\sheaf F}
  :=\cohgp q[X]{\logKX \otimes \sheaf F} 
\end{gather*}
for convenience.
Recall that
\begin{equation*}
  K_D = K_X \otimes D \otimes \faidlof|\sigma_{\mlc}|/|0|* \; ,
\end{equation*}
and the inclusions between adjoint ideal sheaves induce the short exact
sequences
\begin{equation*} % \label{eq-ex2}
  \xymatrix{
    0 \ar[r]
    & {\faidlof/|\rho|*} \ar[r]
    & {\faidlof|\tau|/|\rho|*} \ar[r]
    & {\faidlof|\tau|/*} \ar[r]
    & 0
  } \quad\text{ for } 0 \leq \rho \leq \sigma \leq \tau \; .
\end{equation*}
One is thus led to consider the commutative diagram
\subfile{commut-diagram_sing-Fujino-conj}%
for $\sigma =2,\dots,\sigma_{\mlc}$, in which the columns are exact,
$\iota_\sigma$ and $\tau_\sigma$ are induced from the inclusions
between adjoint ideal sheaves, and $\mu_\sigma$ (resp.~$\nu_\sigma$)
is the composition of $\iota_\sigma$ (resp.~$\tau_\sigma$) with the
map induced from the multiplication map $\otimes s$.
The statement in Theorem \ref{thm:main} is proved if one shows that
$\ker\mu_{\sigma_{\mlc}} = \ker\iota_{\sigma_{\mlc}} = 0$
($\iota_{\sigma_{\mlc}}$ is the identity map).
Note that $\mu_1 =\nu_1$ and $\iota_1 =\tau_1$.
Following the argument in \cite{Chan&Choi_injectivity-I}*{Thm.~1.3.2}, since
$\ker\mu_{\sigma-1} =\ker\iota_{\sigma-1}$ and $\ker\nu_\sigma
=\ker\tau_\sigma$ together imply $\ker\mu_{\sigma}
=\ker\iota_{\sigma}$ via a diagram-chasing argument, to prove Theorem
\ref{thm:main}, it suffices to show the following theorem.

\begin{thm} \label{thm:ker-nu=ker-tau}
  $\ker\nu_\sigma =\ker\tau_\sigma$ for all $\sigma =1, \dots, \sigma_{\mlc}$.
\end{thm}


\begin{remark} \label{rem:general-commut-diagram}
  When $\aidlof|0|*$ in the commutative diagram
  \eqref{eq:commut-diagram_sing-Fujino-conj} is replaced by $0$ (which
  can be considered as $\aidlof|-1|*$), the setup is reduced to the one in
  \cite{Chan&Choi_injectivity-I}*{Thm.~1.3.2}, which states that
  Theorem \ref{thm:ker-nu=ker-tau} together with the result in
  \cite{Matsumura_injectivity-lc}*{Thm.~1.6} or
  \cite{Chan&Choi_injectivity-I}*{Thm.~1.2.1} implies that Fujino's
  conjecture is true.
  As a matter of fact, the proof of Theorem \ref{thm:ker-nu=ker-tau}
  can also be adapted to the case $\sigma = 0$ (with $\aidlof|-1|* =
  0$ and $\residlof|0|* =D^{-1} \isom \defidlof{D} =\aidlof|0|*$) which recovers the result in
  \cite{Matsumura_injectivity-lc}*{Thm.~1.6} as well as
  \cite{Chan&Choi_injectivity-I}*{Thm.~1.2.1}.
  Furthermore, by replacing $\aidlof|0|*$ by $\aidlof|\sigma_0-1|*$
  and $\aidlof|\sigma_{\mlc}|*$ by $\aidlof|\sigma'|*$ for any $0 <
  \sigma_0 \leq \sigma'$ and letting $\sigma$ vary within the range
  $\sigma_0 < \sigma \leq \sigma'$ in the diagram
  \eqref{eq:commut-diagram_sing-Fujino-conj}, one sees that the proof
  of Theorem \ref{thm:ker-nu=ker-tau} guarantees the statement of
  Theorem \ref{thm:main} but with $K_D$ replaced by $K_X \otimes D
  \otimes \faidlof|\sigma'|/|\sigma_0 -1|*$.
\end{remark}


\begin{proof}
  The proof consists of the following steps.
  \begin{enumerate}[label=\textbf{Step \Roman*:}, ref=\Roman*,
    leftmargin=0pt, labelsep=*, widest=VI, itemindent=*, align=left,
    itemsep=1.5ex]
  \item Make use of the $L^2$ Dolbeault and harmonic theory available on
    $\spH{\residlof*}$.

    Write $\lcc' =\bigcup_{p \in \Iset} \lcS$ as the union of
    $\sigma$-lc centers $\lcS$ of $(X,D)$.  Notice that $\residlof*$,
    hence $\spH{\residlof*}$, has a decomposition as a direct sum
    which yields
    \begin{equation*}
      \spH{\residlof*}
      =\bigoplus_{p \in \Iset} \cohgp q[\lcS]{K_{\lcS}
        \otimes F \otimes \mtidlof<\lcS>{\vphi_F}}
      =\bigoplus_{p \in \Iset} \cohgp q[\lcS]{K_{\lcS} \otimes F}
    \end{equation*}
    such that the $L^2$ Dolbeault isomorphism and harmonic theory are
    valid for the cohomology group in each summand.
    Take the (squared) residue norm
    $\norm\cdot_{\lcc'}^2 = \sum_{p \in \Iset} \norm\cdot_{\lcS}^2$ as
    the $L^2$ norm on $\spH{\residlof*}$.  Pick any element
    $u := (u_p)_{p \in \Iset} \in \spH{\residlof*}$ such that
    \begin{itemize}
    \item each $u_p$ is a harmonic form on $\lcS$ with respect to the
      given norm $\norm\cdot_{\lcS}$ and
    
    \item $u \in \ker\nu_\sigma \cap \paren{\ker\tau_\sigma}^\perp$,
      where the orthogonal complement $\paren{\ker\tau_\sigma}^\perp$
      of $\ker\tau_\sigma$ is taken with respect to the residue norm
      $\norm\cdot_{\lcc'}$.
    \end{itemize}
    The theorem is proved if it is shown that $u_p = 0$ for all
    $p \in \Iset$.

  
  \item \label{item:express-su-in-residue-norm}
    Obtain an expression of $\norm{su}_{\lcc'}^2$ using the
    assumption $u \in \ker\nu_\sigma$ and the \v Cech--Dolbeault
    isomorphism.

    Let $\cvr V :=\set{V_i}_{i \in I}$ be a locally finite cover of
    $X$ by admissible open sets with respect to
    $(\vphi_F,\vphi_M,\psi_D)$ and let $\set{\rho^i}_{i \in I}$ be a
    partition of unity subordinate to $\cvr V$.
    Their notations are abused to mean also their induced cover and
    partition of unity on $\lcc'$ for any $\sigma \geq 0$.
    For any choice of indices $\idx 0,q \in I$, write $V_{\idx 0.q}
    :=V_{i_0} \cap V_{i_1} \dotsm \cap V_{i_q}$ as usual.
  
    Through the \v Cech--Dolbeault isomorphism, every (cohomology
    class of) $u_p$ is represented by a \v Cech $q$-cocycle
    $\set{\alpha_{p; \:\idx 0.q}}_{\idx 0,q \in I}$ such that (under
    the Einstein summation convention on the indices $\idx 0,q$)
    \begin{equation*}
      u_p
      % &= \dbar v_{p;(2)} +\dbar \rho^{i_{q-1}} \wedge \dotsm \wedge
      %   \dbar\rho^{i_0} \alpha_{p; \:\idx 0.q} \qquad\paren{\forall~ i_q \in I} \\
      \overset{\text{\eqref{eq:Cech-Dolbeault-isom}}}=
      \:\dbar v_{p;(2)}
      +(-1)^q \:\underbrace{\dbar \rho^{i_{q}} \wedge \dotsm \wedge
        \dbar\rho^{i_1} \cdot \rho^{i_0} }_{=: \:
        \paren{\dbar\rho}^{\idx q.0}} \alpha_{p; \:\idx 0.q} \; ,
    \end{equation*}
    where $v_{p; (2)}$ is a $K_{\lcS} \otimes \res{F}_{\lcS}$-valued $(0,q-1)$-form
    on $\lcS$ with $L^2$ coefficients with respect to
    $\norm\cdot_{\lcS}$ and
    $\alpha_{p; \:\idx 0.q} \in K_{\lcS} \otimes \res F_{\lcS} \otimes
    \mtidlof<\lcS>{\vphi_F} =K_{\lcS} \otimes \res F_{\lcS}$ on
    $V_{\idx 0.q}$.
    % (see \cite{Matsumura_injectivity}*{Prop.~5.5} or
    % \cite{Chan&Choi_injectivity-I}*{Lemma 3.2.1}). 
    In view of the
    residue short exact sequence, choose, for each choice of
    the multi-indices $(\idx 0,q)$, a section
    $f_{\idx 0,q} \in \logKX M \otimes \aidlof*$ on $V_{\idx 0.q}$
    such that
    \begin{equation*}
      \Res^\sigma(f_{\idx 0.q})
      =\paren{\alert{s} \alpha_{p; \:\idx 0.q}}_{p \in \Iset} 
    \end{equation*}
    (note that $V_{\idx 0.q}$ is Stein).  Considering the inclusion
    $\aidlof* \subset \aidlof|\sigma_{\mlc}|*$, write
    \begin{equation*}
      \eqcls{f_{\idx 0.q}} := \paren{f_{\idx 0.q} \bmod \aidlof-1*}
      \;\in \logKX M \otimes \faidlof|\sigma_{\mlc}|/-1*
      \quad\text{ on } V_{\idx 0.q} \; .
    \end{equation*}
    The collection $\set{\eqcls{f_{\idx 0.q}}}_{\idx 0,q \in I}$ is
    then a \v Cech $q$-cocycle representing $\nu_\sigma(u)$ in $\spH
    M{\faidlof|\sigma_{\mlc}|/-1*}$.
    The assumption $u \in \ker\nu_\sigma$ implies that this cocycle is
    a coboundary, that is,
    \begin{equation*}
      \set{\eqcls{f_{\idx 0.q}}}_{\idx 0,q \in I}
      =\delta\set{\eqcls{\lambda_{\idx 1.q}}}_{\idx 1,q \in I}
      =\set{\eqcls{\paren{\delta\lambda}_{\idx 0.q} }}_{\idx 0,q \in I}
    \end{equation*}
    for some $\lambda_{\idx 1.q} \in \logKX M \otimes
    \aidlof|\sigma_{\mlc}|*$ on $V_{\idx 1.q}$ (note that
    $\lambda_{\idx 1.q}$ need \emph{not} take values in $\aidlof*$
    even though the sections $f_{\idx 0.q}$ do), where
    $\paren{\delta\lambda}_{\idx 0.q}$ is given by the usual formula
    of \v Cech coboundary operator $\paren{\delta\lambda}_{\idx 0.q}
    :=\sum_{k =0}^q (-1)^k \lambda_{\idx 0[\dotsm \widehat{i_k}].q}$.
    Notice that $f_{\idx 0.q}$ and $\paren{\delta\lambda}_{\idx 0.q}$
    differ by an element in $\logKX M \otimes \aidlof-1*$ on $V_{\idx
      0.q}$.
    

    Thanks to the positivity $\ibddbar\vphi_F \geq 0$ and the bound
    $\ibddbar\vphi_M \leq C \ibddbar\vphi_F$ for some
    constant $C > 0$ on each $\lcS$, the product $s u_p$ is harmonic
    with respect to $\norm\cdot_{\lcS}$ (Proposition
    \ref{prop:consequence-of-positivity}), so $\iinner{s u_p}{s
      \:\dbar  v_{p;(2)}}_{\lcS} = \iinner{s u_p}{\dbar \paren{s
        v_{p;(2)}}}_{\lcS} = 0$ for every $p \in\Iset$.
    It follows that
    \begin{align*}
      \norm{su}_{\lcc'}^2
      =\sum_{p\in \Iset} \norm{su_p}_{\lcS}^2
      =&~(-1)^q \sum_{p\in \Iset} \sum_{\idx 0,q \in I} \iinner{\paren{\dbar\rho}^{\idx q.0}
        \:s \alpha_{p; \:\idx 0.q}}{\: s u_p}_{\lcS}
      \\
      =&~(-1)^q \sum_{p\in \Iset} \sum_{\idx 0,q \in I} \iinner{
         s \alpha_{p; \:\idx 0.q}
         }{\:\idxup{\diff\rho},[\idx 0.q].  s u_p}_{\lcS}
         \; ,
    \end{align*}
    where $\idxup{\diff\rho},[\idx 0.q].  \cdot $ is the adjoint
    of $\paren{\dbar\rho}^{\idx q.0} \cdot$.
    As in Step \ref{item:expression-of-su-simple} in Section
    \ref{sec:proof-of-simple-case}, the desired expression can be
    obtained by substituting
    \begin{equation*}
      s\alpha_{p; \:\idx 0.q} =\PRes[\lcS](\frac{f_{\idx
          0.q}}{\sect_D})
      =\PRes[\lcS](\frac{\paren{\delta\lambda}_{\idx
          0.q}}{\sect_D})
      =\frac{\paren{\delta \rs*\lambda_p}_{\idx 0.q}}{\sect_{(p)}}
      \; ,
    \end{equation*}
    where $\rs*\lambda_{p; \:\idx 1.q}
    :=\PRes[\lcS](\frac{\lambda_{\idx 1.q}}{\sect_D}) \cdot
    \sect_{(p)}$.
    For the sake of illustration, an alternative approach via a
    direct residue computation is presented here.
    Note that $\paren{\idxup{\diff\rho},[\idx 0.q].  s u_p}_{p \in
      \Iset} \in \logKX M \otimes \smooth_{X\:c\,*} \cdot\residlof*$ on $V_{\idx
      0.q}$, so it has a preimage $h^{\idx 0.q} \in \logKX M \otimes
    \smooth_{X\:c\,*} \cdot \aidlof*$ of $\Res^\sigma$ (considered as a
    $\smooth_{X\:c\,*}$-homomorphism).
    Fix such preimage on each open set $V_{\idx 0.q}$.
    From the direct computation of the residue function, it follows
    that
    \begin{align*}
      (-1)^q \:\norm{su}_{\lcc'}^2
      =&~\sum_{p\in \Iset} \sum_{\idx 0,q \in I} \iinner{
         s \alpha_{p; \:\idx 0.q}
         }{\:\idxup{\diff\rho},[\idx 0.q].  s u_p}_{\lcS}
      \\
      \xleftarrow{\eps \tendsto 0^+}
       &~\smashoperator[l]{\sum_{\idx 0,q \in I}} \eps
         \int_{\mathrlap{V_{\idx 0.q}}} \quad \frac{
         \inner{f_{\idx 0.q}}{h^{\idx 0.q}}
         \:e^{-\phi_D -\vphi_F -\vphi_M}
         }{\abs{\psi_D}^{\sigma +\eps}}
      \\
      \overset{\mathclap{\text{Prop.~\ref{prop:residue-product-X-to-lcS}}}}= \quad\;
       &~\smashoperator[l]{\sum_{\idx 0,q \in I}} \eps
         \int_{\mathrlap{V_{\idx 0.q}}} \quad \frac{
         \inner{\paren{\delta\lambda}_{\idx 0.q}}{h^{\idx 0.q}}
         \:e^{-\phi_D -\vphi_F -\vphi_M}
         }{\abs{\psi_D}^{\sigma +\eps}} +\BigO(\eps)
      \\
      \xrightarrow[\text{Prop.~\ref{prop:residue-product-X-to-lcS}}]{\eps \tendsto 0^+}
      %  &~\sum_{p\in \Iset} \sum_{\idx 0,q \in I}
      %    \int_{\lcS} \inner{
      %    \frac{\paren{\delta\rs*\lambda_p}_{\idx 0.q}}{ \sect_{(p)}}
      %    }{\:
      %    \idxup{\diff\rho},[\idx 0.q].  s u_p
      %    } \:e^{-\vphi_F-\vphi_M}
      % \\
      % =&~\sum_{p\in \Iset} \sum_{\idx 0,q \in I}
      %    \int_{\lcS} \inner{
      %    \paren{\dbar\rho}^{\idx q.0}
      %    \paren{\delta\rs*\lambda_p}_{\idx 0.q}
      %    }{\:
      %     s u_p \sect_{(p)}
      %    }_\omega \:e^{-\phi_{(p)}-\vphi_F-\vphi_M}
      % \\
      % =
       &~\sum_{p\in \Iset} \sum_{\idx 0,q \in I}
         \iinner{\paren{\dbar\rho}^{\idx q.0}
         \paren{\delta\rs*\lambda_p}_{\idx 0.q}}{\: s u_p
         \sect_{(p)}}_{\lcS, \phi_{(p)}}
      \\
      =&~\sum_{p\in \Iset} \sum_{\idx 1,q \in I}
         \iinner{\dbar\rho^{i_q} \wedge \dotsm \wedge \dbar\rho^{i_1}
         \cdot\rs*\lambda_{p;\:\idx 1.q}}{\: s u_p \sect_{(p)}}_{\lcS,
         \phi_{(p)}}
      \\
      =&~(-1)^{q-1} \sum_{p\in \Iset} \iinner{\dbar v_{p;(\infty)}}{\: s u_p
         \sect_{(p)}}_{\lcS, \phi_{(p)}}
         \; ,\footnotemark
    \end{align*}%
    \footnotetext{
      If the $L^2$ Dolbeault isomorphism is valid for $\spH
      M{\faidlof|\sigma_{\mlc}|/-1*}$, such conclusion can be
      obtained simply from the fact that $\nu_\sigma(su)$ is
      represented by a smooth $\dbar$-exact form on $\lcc'$.
    }%
    where
    % $\rs*\lambda_{p; \:\idx 1.q}
    % :=\PRes[\lcS](\frac{\lambda_{\idx 1.q}}{\sect_D}) \cdot \sect_{(p)}$ and 
    $v_{p; (\infty)} :=\sum_{\idx 1,q\in I} \dbar\rho^{i_q} \wedge \dotsm
    \wedge \dbar\rho^{i_2} \cdot \rho^{i_1} \rs*\lambda_{p;\:\idx 1.q}
    =\sum_{\idx 1,q\in I} \paren{\dbar\rho}^{\idx q.1}\rs*\lambda_{p; \:\idx 1.q}$. 

    % Note that $s u_p$ is harmonic with respect to the potential
    % $\vphi_F+\vphi_M$ by the positivity assumption.
    The expression of $\norm{su}_{\lcc'}^2$ can be further transformed
    by an integration by parts using Proposition
    \ref{prop:res-formula-dbar-exact-dot-harmonic}, which becomes
    \begin{equation*}
      \norm{su}_{\lcc'}^2
      % &=\smashoperator[l]{\sum_{\idx 1,q \in I}} \sum_{b \in \Iset+1}
      % \sum_{j=1}^{\sigma +1} (-1)^q \:\sigma
      % \iinner{ \sgn{b:p_{b,j}}\:
      % \frac{\rs*\lambda_{b;\:\idx 1.q}}{\sect_{(b)}}
      % }{\: s\:
      %   \idxup{\diff\rho},[\idx 1.q] .
      %   \PRes[b(j)](\idxup{\diff\psi_{(p_{b,j})}}.  u_{p_{b,j}})
      % }_{\lcS+1[b]}
      %   \\
      = \sigma\sum_{b \in \Iset+1}
      \iinner{v_{b;(\infty)} \:
      }{ \quad s \:
        \smashoperator{\sum_{p \in \Iset \colon \lcS+1[b] \subset
            \lcS}} \;\; \sgn{b:p}\:
        \PRes[\lcS+1[b] | \lcS](\idxup{\diff\psi_{(p)}}.  u_{p})
        \cdot \sect_{(b)}
      }_{\lcS+1[b], \phi_{(b)}} \; ,
    \end{equation*}
    where $v_{b;(\infty)} := \sum_{\idx 1,q \in I}
    \paren{\dbar\rho}^{\idx q.1} \rs*\lambda_{b; \:\idx 1.q}$ and
    $\rs*\lambda_{b; \:\idx 1.q}
    :=\PRes[\lcS+1[b]](\frac{\lambda_{\idx 1.q}}{\sect_D}) \cdot \sect_{(b)}$.
    % However, notice that $v_{p;(\infty)}$ is smooth on $\lcS$ but not
    % locally $L^2$ with respect to the weight $e^{-\phi_{(p)}}$.
    % For this reason, let $\psi_{(p)} :=\phi_{(p)} -\sm\vphi_{(p)}$, where
    % $\sm\vphi_{(p)}$ is some smooth potential on $\Diff_p D$.
    % One then has
    % \begin{align*}
    %   &~\norm{su}_{\lcc'}^2
    %   =\sum_{p\in \Iset} \iinner{\dbar v_{p;(\infty)}}{\: s u_p
    %      \sect_{(p)}}_{\lcS, \phi_{(p)}}
    %   \\
    %   \xleftarrow{\eps \tendsto 0^+}
    %    &~\sum_{p \in \Iset} \iinner{
    %      e^{-\eps \abs{\psi_{(p)}}} \:\dbar v_{p;(\infty)}
    %      }{\: s u_p \sect_{(p)}}_{\lcS, \phi_{(p)}}
    %   \\
    %   =&~\sum_{p \in \Iset} \paren{
    %      \cancelto{0}{\iinner{
    %      \dbar\paren{e^{-\eps \abs{\psi_{(p)}}} \: v_{p;(\infty)}}
    %      }{\: s u_p \sect_{(p)}}_{\mathrlap{\lcS, \phi_{(p)}}}}
    %      \quad\;\; + \eps 
    %      \iinner{
    %      e^{-\eps \abs{\psi_{(p)}}} \:v_{p;(\infty)}
    %      }{\:\idxup{\diff\psi_{(p)}} . s u_p \sect_{(p)}}_{\lcS,
    %      \phi_{(p)}}
    %      }
    %   \\
    %   =&~\sum_{p \in \Iset} \sum_{\idx 1,q \in I} (-1)^q \:\eps \:
    %      \iinner{
    %      e^{-\eps \abs{\psi_{(p)}}} \: % \paren{\dbar\rho}^{\idx q.1}
    %      \rs*\lambda_{p;\:\idx 1.q}
    %      }{\:
    %      \idxup{\diff\rho},[\idx 1.q] .
    %      \paren{\idxup{\diff\psi_{(p)}} . s u_p \sect_{(p)}}
    %      }_{\lcS, \phi_{(p)}}
    %   \\
    %   \xrightarrow[\text{Prop.~\ref{prop:residue-formula-classical-kernel}}]{\eps
    %   \tendsto 0^+} 
    %   &~\smashoperator[l]{\sum_{\idx 1,q \in I}} \sum_{p \in \Iset}
    %     \sum_{k=\sigma +1}^{\mathclap{\sigma_{V_{\idx 1.q}}}} (-1)^q \:\sigma
    %     \iinner{
    %     \PRes[p(k)](
    %     \frac{\rs*\lambda_{p;\:\idx 1.q}}{\sect_{(p)}}
    %     )
    %      }{\:
    %      \idxup{\diff\rho},[\idx 1.q] .
    %      \PRes[p(k)](\idxup{\diff\psi_{(p)}} . s u_p)
    %      }_{\lcS \cap \set{z_{p(k)} =0}}
    %   \; ,
    % \end{align*}
    % where $\PRes[p(k)]$ denotes the Poincar\UTF{00E9} residue map from $\lcS$
    % to $\lcS \cap \set{z_{p(k)}=0}$. 
    % The last limit is justified as follows.
    % On the admissible open set $V_{\idx 1,q}$, consider a holomorphic
    % coordinate system $(z_1, \dots, z_n)$ such that $\lcS
    % =\set{z_{p(1)} = \dotsm =z_{p(\sigma)} =0}$ and
    % $\sect_{(p)} =z_{p(\sigma+1)} \dotsm z_{p(\sigma_V)}$ (write
    % $\sigma_{V}$ for $\sigma_{V_{\idx 1.q}}$ for convenience).
    % Note that
    % \begin{equation*}
    %   \diff\psi_{(p)} =\sum_{k =\sigma +1}^{\sigma_V}
    %   \frac{dz_{p(k)}}{z_{p(k)}} -\diff\sm\vphi_{(p)} \quad\text{ on }
    %   V_{\idx 1,q} \; .
    % \end{equation*}
    % It follows that, on $\lcS \cap V_{\idx 1.q}$,
    % \begin{equation*}
    %   \begin{multlined}
    %     \text{coef.~of }\:
    %     \idxup{\diff\rho},[ \idx 1.q] .
    %     \paren{\idxup{\diff\psi_{(p)}} . s u_p \sect_{(p)}}
    %   \end{multlined}
    %   \in
    %   \res{\defidlof{\lcc+2'}}_{\lcS}
    %   \begin{aligned}[t]
    %     &=\mtidlof<\lcS>{\vphi_F+\vphi_M} \cdot
    %     \res{\defidlof{\lcc+2'}}_{\lcS} \;\;\footnotemark
    %     \\
    %     &=\aidlof|1|<\lcS>{\vphi_F+\vphi_M}[\psi_{(p)}]
    %   \end{aligned}
    % \end{equation*}%
    % \footnotetext{
    %   Recall that $\defidlof{\lcc+2'}$ is generated on $X$ by
    %   $\sect_{(\sigma+1 : b)}$ for all $b \in \Iset+1$ treated as local
    %   functions.
    %   On an admissible open set $V$, one has $\defidlof{\lcc+2'}
    %   =\genbyd{z_{b(\sigma+2)} \dotsm
    %     z_{b(\sigma_V)}}{b \in \Iset+1 \text{ such that } \lcS+1[b] \cap
    %     V \neq \emptyset}$ (see page
    %   \pageref{page:notation-permutation-index} for the notation).
    % }%
    % and, therefore, one can apply Proposition
    % \ref{prop:residue-formula-classical-kernel} (with $\lcS$ in place
    % of $X$, $\psi_{(p)}$ in place of $\psi_D$) to each inner product
    % $\eps \iinner{\dotsm}{ \:\dotsm \idxup{\diff\psi_{(p)}}. \dotsm
    %   \sect_{(p)}}_{\lcS,\phi_{(p)}}$.
    % % to obtain a sum of integrals on
    % % each $\lcS \cap \set{z_{p(k)} = 0}$ for $k=\sigma +1, \dots,
    % % \sigma_V$, i.e.~on the $(\sigma+1)$-lc centers in $\lcc+1' \cap
    % % V_{\idx 1.q}$.
    
    % Write $\lcc+1' =\bigcup_{b \in \Iset+1} \lcS+1[b]$.
    % On each admissible open set $V_{\idx 1.q}$, the intersection $\lcS
    % \cap \set{z_{p(k)} = 0}$ is a $(\sigma+1)$-lc center $\lcS+1[b_{p,k}]
    % \cap V_{\idx 1.q}$ ($\neq \emptyset$), uniquely determined by the
    % choices of $p\in \Iset$ (such that $\lcS \cap V_{\idx 1.q} \neq
    % \emptyset$, so $\binom{\sigma_V}{\sigma}$ choices) and $k
    % =\sigma+1, \dots, \sigma_V$ (so $\sigma_V-\sigma$ choices).
    % To get an indexing in terms of $b \in \Iset+1$ (such that
    % $\lcS+1[b] \cap V_{\idx 1.q} \neq \emptyset$, so
    % $\binom{\sigma_V}{\sigma +1}$ choices), note that each $\lcS+1[b]
    % \cap V_{\idx 1.q}$ is contained in $\sigma +1$ distinct
    % $\sigma$-lc centers $\lcS[p_{b,j}]$ for $j=1,\dots,\sigma+1$
    % (apparently, $\sigma +1$ choices) such that
    % \begin{equation*}
    %   \lcS+1[b] \cap V_{\idx 1.q} = \lcS[p_{b,j}] \cap \set{z_{b(j)} = 0} \; .
    % \end{equation*}
    % (One can verify $\sum_{p \in \Iset} \sum_{k=\sigma
    %   +1}^{\sigma_{V}} \dotsm = \sum_{b \in
    %   \Iset+1} \sum_{j=1}^{\sigma +1} \dotsm$ by first noting that
    % $\binom{\sigma_V}{\sigma} (\sigma_V -\sigma)
    % =\binom{\sigma_V}{\sigma +1} (\sigma+1)$.)
    % With such choice of indexing, let $\sgn{b:p_{b,j}}$ be the sign
    % given by
    % \begin{equation*}
    %   \PRes[\lcS+1[b]]
    %   =\sgn{b:p_{b,j}} \:\PRes[b(j)]\circ \PRes[\lcS[p_{b,j}]] \; .
    % \end{equation*}
    % Therefore, one has
    % \begin{equation*}
    %   \frac{\rs*\lambda_{b;\: \idx 1.q}}{\sect_{(b)}}
    %   :=\PRes[\lcS+1[b]](\frac{\lambda_{\idx 1.q}}{\sect_D})
    %   % =\sgn{b:p_{b,j}} \:\PRes[b(j)]\circ
    %   % \PRes[\lcS[p_{b,j}]](\frac{\lambda_{\idx 1.q}}{\sect_D})
    %   =\sgn{b:p_{b,j}} \:
    %   \PRes[b(j)](\frac{\rs*\lambda_{p_{b,j};\:\idx
    %       1.q}}{\sect_{(p_{b,j})}})
    % \end{equation*}
    % (recalling that $\sect_{(b)} =\sect_{(\sigma+1 : b)}$,
    % $\sect_{(p_{b,j})} =\sect_{(\sigma : p_{b,j})}$ and
    % $\sect_{(p_{b,j})} = z_{b(j)} \sect_{(b)}$).
    
    Set
    \begin{equation}\label{eq-def-w}
      w_b := \smashoperator[r]{\sum_{p \in \Iset \colon \lcS+1[b] \subset
          \lcS}} \;\; \sgn{b:p}\:
      \PRes[\lcS+1[b] | \lcS](\idxup{\diff\psi_{(p)}} . u_{p})
      \; .
    \end{equation}
    It suffices to show that $w_b = 0$ on $\lcS+1[b]$ for each $b
    \in\Iset+1$ to conclude the proof.
    

  \item Show that $w_b$ is harmonic with respect to
    $\res{\vphi_F}_{\lcS+1[b]}$ (and $\res{\omega}_{\lcS+1[b]}$) on
    $\lcS+1[b]$ for all $b \in \Iset+1$ and thus $\paren{w_b}_{b
      \in\Iset+1}$ represents a class in $\spH/q-1/{\residlof+1*}$.

    {
      \newcommand{\lcSb}{\lcS+1[b]}
      % \newcommand{\idxj}{\idx[\conj j]}
      
      To see that $w_b$ is $\dbar$-closed on $\lcSb$, it suffices to
      show that $\PRes[\lcS+1[b] | \lcS](\idxup{\diff\psi_{(p)}}. 
      u_{p})$ is $\dbar$-closed for all $p\in\Iset$ such that $\lcSb
      \subset \lcS$.
      Take any admissible open set $V$ such that $V \cap \lcSb
      \neq\emptyset$ and a holomorphic coordinate system such that
      $\sect_{(p)} = z_{p(\sigma+1)} \dotsm z_{p(\sigma_V)}$ on $V$.
      Suppose $\lcSb \cap V = \lcS \cap \set{z_{p(k)} = 0}$ for some $k
      =\sigma +1, \dots, \sigma_V$.
      Recall that
      \begin{equation*}
        \diff\psi_{(p)} = \sum_{k'=\sigma+1}^{\sigma_V}
        \frac{dz_{p(k')}}{z_{p(k')}} - \diff\sm\vphi_{(p)} \quad\text{
          on } V \; .
      \end{equation*}
      By writing
      \begin{equation*}
        \idxup{dz_{p(k)}}.  u_p =: dz_{p(k)} \wedge
        \paren{\idxup{dz_{p(k)}}.  \rs u_{p,k}} \quad\text{ on }
        \lcS \cap V \; ,
      \end{equation*}
      in which $\rs u_{p,k}$ is a $(n-\sigma-1,q)$-form on $\lcS \cap
      V$, it follows that
      \begin{equation*}
        \PRes[\lcS+1[b] | \lcS](\idxup{\diff\psi_{(p)}}.  u_{p})
        =\PRes[\set{z_{p(k)}=0}](\frac{\idxup{dz_{p(k)}}. 
          u_p}{z_{p(k)}})
        =\parres{\idxup{dz_{p(k)}} . \rs u_{p,k}}_{\lcSb}
        \quad\text{ on } \lcSb \cap V \; .
      \end{equation*}
      It thus suffices to show that $\idxup{dz_{p(k)}}.  u_p$ is
      $\dbar$-closed on $\lcS \cap V$.
      % , which is guaranteed by Lemma
      % \ref{lem:commutator-dbar-ctrt}.
      Since $u_p$ is harmonic and $\ibddbar\vphi_F \geq 0$, it follows
      that $\dbar u_p = 0$ and $\nabla^{(0,1)}u_p = 0$ (Proposition
      \ref{prop:consequence-of-positivity}).
      Putting $u_p$ into $u$ and $z_{p(k)}$ into $\vphi$ in Lemma
      \ref{lem:commutator-dbar-ctrt}, one has
      $\dbar\paren{\idxup{dz_{p(k)}}.  u_p} = 0$ on $\lcS \cap V$.
      As a result, $w_b$ is $\dbar$-closed on $\lcSb$ and
      $\paren{w_b}_{b\in\Iset+1}$ therefore represents a class in
      $\spH/q-1/{\residlof+1*}$.
      % This can be done via the following formula, which is a special
      % case and a slight variant of \cite{Donnelly&Xavier}*{(2.4)} and
      % \cite{Ohsawa&Takegoshi-spectral_seq}*{Prop.~1.5} (see also
      % \cite{Takegoshi_higher-direct-images}*{(1.9)} and
      % \cite{Matsumura_injectivity-Kaehler}*{Lemma 2.1}).
      
      % \begin{lemma}[cf.~\cite{Donnelly&Xavier}*{(2.4)},
      %   \cite{Ohsawa&Takegoshi-spectral_seq}*{Prop.~1.5},
      %   \cite{Takegoshi_higher-direct-images}*{(1.9)} and
      %   \cite{Matsumura_injectivity-Kaehler}*{Lemma
      %     2.1}] \label{lem:commutator-dbar-ctrt}
      %   Let $\vphi$ be a smooth function and $u$ be a smooth
      %   ($K_X$-valued) $(0,q)$-form on a K\"ahler manifold.
      %   They satisfy the formula
      %   \begin{equation*}
      %     \dbar\paren{\idxup{\diff\vphi}.  u}
      %     =\mhlight{\idxup{\ibddbar\vphi}.  u}
      %     -\idxup{\diff\vphi} . \paren{\dbar u}
      %     +\idxup{\diff\vphi} \cdot \nabla^{(0,1)}_\bullet u \; ,
      %   \end{equation*}%
      %   % \mariocomment{I treat $u$ as a $(0,q)$-form, so it doesn't
      %   %   make sense to apply $\Lambda_\omega$ to $u$.
      %   %   Do you accept this or we have to treat sections of $K_X$ as
      %   %   $(n,0)$-forms?
      %   %   I don't recommend the latter choice since it's easier to say
      %   %   and understand a ``$K_{\lcS}$-valued section'' than a
      %   %   ``$(n-\sigma,\bullet)$-form''.
      %   % }%
      %   \mariocomment[Red]{SM is treating $u$ as an $F$-valued $(n-\sigma, \bullet)$-form in my head. 
      %     I understand that the reason why the vanishing theorem holds for the adjoint bundle $K \times F$ 
      %     comes from the special property of $(n,q)$-type, so this is
      %     more natural for me.}%
      %   \mmark*{}{MC: OK. Since formula in Sec.
      %       \ref{subsec:n2} Step 3 also goes without $\Lambda_\omega$,
      %     I feel safe to keep things as they are. BTW, I prefer to
      %     keep this proof after I compare the type of $\diff\vphi$ and
      %   $\dbar\Phi$ in the referred papers.}%
      %   or, when a local holomorphic coordinate system is fixed and
      %   the Einstein summation convention is applied, 
      %   \begin{equation*}
      %     \paren{\dbar\paren{\idxup{\diff\vphi}.  u}}_{\conj J_{q}}
      %     =\sum_{\nu=1}^q \diff^{\conj\ell} \diff_{\conj j_\nu} \vphi \:
      %     u_{\idxj 1[\dotsm (\conj \ell)_\nu].q}
      %     -\diff^{\conj\ell}\vphi  \:\paren{\dbar u}_{\conj\ell\conj J_q}
      %     +\diff^{\conj\ell}\vphi \:\nabla_{\conj\ell} u_{\conj J_q} 
      %   \end{equation*}
      %   for any multi-indices $J_q = (\idx[j]1,q)$, pointwisely.
      % \end{lemma}

      % \begin{proof}
      %   A direct computation yields
      %   \begin{align*}
      %     \paren{\dbar\paren{\idxup{\diff\vphi} . u}}_{\conj
      %     J_{q}}
      %     &=\sum_{\nu=1}^q (-1)^{\nu-1} \diff_{\conj j_\nu}
      %       \paren{\idxup{\diff\vphi}.  u}_{\idxj 1[\dotsm \widehat
      %       {\conj j}_\nu].q}
      %       =\sum_{\nu=1}^q (-1)^{\nu-1} \diff_{\conj j_\nu}
      %       \paren{\diff_{\ell}\vphi \: u^\ell_{\;\idxj 1[\dotsm
      %       \widehat{\conj j}_\nu].q}}
      %     \\
      %     &=\sum_{\nu=1}^q (-1)^{\nu-1} \paren{
      %       \diff_{\conj j_\nu}\diff_{\ell}\vphi \: u^{\ell}_{\;\idxj 1[\dotsm
      %       \widehat {\conj j}_\nu].q}
      %       +\diff_{\ell}\vphi \: \nabla_{\conj j_\nu} u^\ell_{\;\idxj 1[\dotsm
      %       \widehat {\conj j}_\nu].q}
      %       }
      %     \\
      %     &=\sum_{\nu=1}^q
      %       \diff^{\conj \ell}\diff_{\conj j_\nu}\vphi \: u_{\idxj 1[\dotsm
      %       (\conj\ell)_\nu].q}
      %       -\diff^{\conj\ell}\vphi \sum_{\nu=1}^q (-1)^{\nu} 
      %       \nabla_{\conj j_\nu} u_{\conj\ell \idxj 1[\dotsm
      %       \widehat {\conj j}_\nu].q}
      %       \begin{aligned}[t]
      %         &-\diff^{\conj\ell}\vphi \: \nabla_{\conj \ell} u_{\conj
      %           J_q} \\
      %         &+\diff^{\conj\ell}\vphi \: \nabla_{\conj \ell}
      %         u_{\conj J_q}
      %       \end{aligned}
      %     \\
      %     &=\sum_{\nu=1}^q
      %       \diff^{\conj \ell}\diff_{\conj j_\nu}\vphi \: u_{\idxj 1[\dotsm
      %       (\conj\ell)_\nu].q}
      %       -\diff^{\conj\ell}\vphi
      %       \:\paren{\dbar u}_{\conj\ell\conj J_q}
      %       +\diff^{\conj\ell}\vphi \: \nabla_{\conj \ell} u_{\conj
      %       J_q} \; ,
      %   \end{align*}
      %   as desired.
      % \end{proof}


    }

    Furthermore, by Proposition \ref{prop:harmonic-residue} and
    Theorem \ref{thm:residue-harmonic} (with
    $\lcS$ in place of $X$, $\lcS+1[b]$ in place of $D_p$ and
    $\psi_{(p)}$ in place of $\psi_{D_p}$), $w_b$ is a
    $K_{\lcS+1[b]} \otimes \res{F}_{\lcS+1[b]}$-valued $(0,q-1)$-form on
    $\lcS+1[b]$ (not only a $\res{\conj\holoform_X^{q-1}}_{\lcS+1[b]}$-valued
    section) which is harmonic with respect to
    $\res{\vphi_F}_{\lcS+1[b]}$.
    % Therefore, $\paren{w_b}_{b\in\Iset+1}$ represents a class in
    % $\spH/q-1/{\residlof+1*}$. 



  \item \label{item:pf:use_u-ortho-w}
    Apply the assumption $u =(u_p)_{p\in\Iset} \in
    \paren{\ker\tau_\sigma}^{\perp}$ via the use of $w
    :=\paren{w_b}_{b \in \Iset+1} \in \spH/q-1/{\residlof+1*}
    =\bigoplus_{b \in\Iset+1} \cohgp{q-1}[\lcS+1[b]]{K_{\lcS+1[b]}
      \otimes F}$ in view of the commutative diagram
    \begin{equation*}
      \xymatrix@R-0.3cm{
        {\dotsm} \ar[r]
        & {\spH/q-1/{\residlof+1*}} \ar[r]^-{\delta}
        \ar[d]^-{\tau_{\sigma+1}}
        & {\spH{\residlof*}} \ar[r]
        \ar@{=}[d]
        & {\spH{\faidlof+1/-1*}} \ar[r]
        \ar[d]
        & {\dotsm}
        \\
        {\dotsm} \ar[r]
        & {\spH/q-1/{\faidlof|\sigma_{\mlc}|*}} \ar[r]
        & {\spH{\residlof*}} \ar[r]^-{\tau_\sigma}
        & {\spH{\faidlof|\sigma_{\mlc}|/-1*}} \ar[r]
        & {\dotsm}
      }
    \end{equation*}
    and conclude that $u_p = 0$ on $\lcS$ for each $p\in\Iset$.

    From the commutative diagram, one sees that $\delta w \in
    \ker\tau_\sigma$.
    In view of the isomorphism $\residlof+1* \isom \faidlof+1*$ and by
    following the procedures in Step
    \ref{item:express-su-in-residue-norm}, one obtains
    a $\logKX \otimes \aidlof+1*$-valued \v Cech $(q-1)$-cochain
    $\set{\gamma_{\idx 1.q}}_{\idx 1,q \in I}$ with respect to $\cvr
    V$ such that, when
    \begin{equation*}
      \paren{\alpha'_{b; \:\idx 1.q}}_{b\in\Iset+1}
      :=\paren{\frac{\rs*\gamma_{b; \:\idx
            1.q}}{\sect_{(b)}}}_{\mathrlap{b\in\Iset+1}} \quad\;
      :=\Res^{\sigma+1}(\gamma_{\idx 1.q})
      \in \smashoperator[r]{\prod_{b\in\Iset+1}} K_{\lcS+1[b]} \otimes
      \res{F}_{\lcS+1[b]} \paren{\lcS+1[b] \cap V_{\idx 1.q}}
    \end{equation*}
    (notation chosen for the consistency with those in Proposition
    \ref{prop:res-formula-dbar-exact-dot-harmonic}) and
    \begin{equation*}
      \eqcls{\gamma_{\idx 1.q}}
      := \paren{\gamma_{\idx 1.q} \bmod \aidlof*} \in \logKX \otimes
      \faidlof+1* \quad\text{ on } V_{\idx 1.q} \; ,
    \end{equation*}
    the collection $\set{\alpha'_{b;\:\idx 1.q}}_{\idx 1,q\in I}$
    is a \v Cech \emph{$(q-1)$-cocycle} representing (the class of)
    $w_b$ in $\cohgp{q-1}[\lcS+1[b]]{K_{\lcS+1[b]} \otimes F}$ for
    each $b \in\Iset+1$ such that
    \begin{equation*}
      w_b = \dbar v'_{b;(2)} +(-1)^{q-1} \:\underbrace{
        \dbar \rho^{i_{q}} \wedge \dotsm \wedge
        \dbar\rho^{i_2} \cdot \rho^{i_1} }_{=: \:
        \paren{\dbar\rho}^{\idx q.1}} \alpha'_{b;\:\idx 1.q}
      =: \dbar v'_{b;(2)} +(-1)^{q-1} \frac{v_{b;(\infty)}'}{\sect_{(b)}}
    \end{equation*}
    (again, notation chosen for the consistency with those in Proposition
    \ref{prop:res-formula-dbar-exact-dot-harmonic})
    for some $K_{\lcS+1[b]} \otimes \res{F}_{\lcS+1[b]}$-valued
    $(0,q-2)$-form $v'_{b;(2)}$ on $\lcS+1[b]$ with $L^2$ coefficients
    with respect to $\norm\cdot_{\lcS+1[b]}$, and the collection
    $\set{\eqcls{\gamma_{\idx 1.q}}}_{\idx 1,q \in I}$ is a \v Cech
    \emph{$(q-1)$-cocycle} representing (the class of) $w$ in
    $\spH/q-1/{\faidlof+1*} \xrightarrow[\isom]{\Res^{\sigma+1}}
    \spH/q-1/{\residlof+1*}$.
    The image $\delta w$ in $\spH{\residlof*}$ is then represented by
    \begin{equation*}
      \set{\Res^\sigma\paren{\paren{\delta\gamma}_{\idx 0.q}}}_{\idx 0,q \in
      I} \; ,
    \end{equation*}
    in which $\delta$ is the \v Cech boundary operator.
    Note that applying $\Res^\sigma$ to $\paren{\delta\gamma}_{\idx
      0.q}$ is valid as $\set{\eqcls{\gamma_{\idx 1.q}}}_{\idx 1,q \in
      I}$ is a cocycle and thus coefficients of
    $\paren{\delta\gamma}_{\idx 0.q}$ lie in $\aidlof*$.
    Set
    \begin{equation*}
      \rs*\gamma_{p;\:\idx 1.q} := \PRes[\lcS](\frac{\gamma_{\idx
            1.q}}{\sect_D}) \cdot \sect_{(p)} 
    \end{equation*}
    such that
    \begin{equation*}
      \Res^\sigma\paren{\paren{\delta\gamma}_{\idx 0.q}}
      =\paren{\frac{\paren{\delta\rs*\gamma_p}_{\idx
            0.q}}{\sect_{(p)}}}_{p \in \Iset}
      \in \prod_{p \in \Iset} K_{\lcS} \otimes \res F_{\lcS}
      \paren{\lcS \cap V_{\idx 0.q}} \; .
    \end{equation*}
    Note that 
    \begin{equation*}
      (-1)^q \paren{\dbar\rho}^{\idx q.0}
      \frac{\paren{\delta \rs*\gamma_p}_{\idx 0.q}}{\sect_{(p)}}
      % =(-1)^q \paren{\dbar\rho}^{\idx q.1}
      % \frac{\rs*\gamma_{p;\:\idx 1.q}}{\sect_{(p)}}
      =-
      \frac{\dbar\paren{\paren{\dbar\rho}^{\idx q.1} \rs*\gamma_{p;\:\idx 1.q}}}{\sect_{(p)}}
      =: - \:\frac{\dbar v_{p; (\infty)}'}{\sect_{(p)}}
      \quad\text{ on } \lcS
    \end{equation*}
    is a $\dbar$-closed form representing the class of $\res{\delta
      w}_{\lcS}$ (the component of $\delta w$ on $\lcS$) in $\cohgp
    q[\lcS]{K_{\lcS} \otimes F}$ via Dolbeault isomorphism.
    
    Therefore, from the assumption $u \in
    \paren{\ker\tau_\sigma}^\perp$ and taking into account the \v
    Cech--Dolbeault isomorphism \eqref{eq:Cech-Dolbeault-isom} and the
    fact that each $u_p$ is harmonic, one has
    \begin{align*}
      0 =\iinner{(-1)^{q-1} \delta w}{u}_{\lcc'}
      &=(-1)^q \sum_{p\in \Iset} \iinner{\frac{\dbar
          v_{p;(\infty)}'}{\sect_{(p)}}}{u_p}_{\lcS}
      =(-1)^q \sum_{p\in \Iset} \iinner{\dbar v_{p;(\infty)}'}{u_p
        \sect_{(p)}}_{\lcS, \phi_{(p)}}
      \\
      &\overset{\mathclap{\text{Prop.~\ref{prop:res-formula-dbar-exact-dot-harmonic}}}}=
        \quad \;\; (-1)^{q-1} \sigma
        \smashoperator{\sum_{b \in \Iset+1}} \iinner{v_{b;(\infty)}'}{w_b
        \sect_{(b)}}_{\lcS+1[b], \phi_{(b)}}
      \\
      &=\sigma \smashoperator{\sum_{b \in \Iset+1}}
        \iinner{
          \paren{w_b -\dbar v_{b;(2)}'} \sect_{(b)}
        }{
          w_b \sect_{(b)}
        }_{\lcS+1[b], \phi_{(b)}}
      \\
      &\overset{\mathclap{w_b \text{ harmonic}}}= \quad\;\;\;
        \sigma
        \smashoperator{\sum_{b \in \Iset+1}}
        \iinner{w_b}{w_b}_{\lcS+1[b]}
        =\sigma \norm{w}_{\lcc+1'}^2 \; .
    \end{align*}
    As a result, $w_b = 0$ for each $b\in\Iset+1$, thus $su_p = 0$
    (hence $u_p = 0$) for each $p\in\Iset$ by Step
    \ref{item:express-su-in-residue-norm}.
    This completes the proof. \qedhere
  \end{enumerate}
\end{proof}

\begin{remark} \label{rem:singular-vphi_F}
  When $\vphi_F$ and $\vphi_M$ have only neat analytic singularities
  such that $\vphi_F^{-1}(-\infty) \cup \vphi_M^{-1}(-\infty) \cup D$
  has only snc and $\vphi_F^{-1}(-\infty) \cup \vphi_M^{-1}(-\infty)$
  contains no irreducible components of $D$ (hence no lc centers of
  $(X,D)$), the proof is still valid when the K\"ahler metric $\omega$
  on $X$ is replaced by a complete metric on $X \setminus
  \paren{\vphi_F^{-1}(-\infty) \cup \vphi_M^{-1}(-\infty)}$ as
  described in \cite{Chan&Choi_injectivity-I}*{\S 2.2 item (4)}.
  See \cite{Chan&Choi_injectivity-I}*{\S 3.3} for the technical
  modifications required.
\end{remark}

\begin{remark} \label{rem:no-hard-Lefschetz}
  Notice that the refinement of hard Lefschetz theorem (see
  \cite{Matsumura_injectivity-lc}*{Thm.~1.7} or
  \cite{Chan&Choi_injectivity-I}*{Thm.~2.5.1}) is not used in this
  proof.
  It is used in previous works to show that $\frac{u}{\sect_D}$ is
  smooth for every harmonic $u$ representing a class in $\cohgp
  q[X]{\logKX\otimes \mtidlof<X>{\phi_D}}$.
  This argument can be replaced by using directly the isomorphism
  induced by $\holo_X \xrightarrow[\isom]{\otimes \sect_D} D \otimes
  \mtidlof<X>{\phi_D}$, or $\holo_{\lcS} \xrightarrow[\isom]{\otimes
    \sect_{(p)}} \Diff_p(D) \otimes \mtidlof<\lcS>{\phi_{(p)}}$, which
  is more relevant to this article (see also Lemma
  \ref{lem:su-harmonicity}).
  However, when $\vphi_F$ and $\vphi_M$ have neat analytic singularities
  as described in \cite{Chan&Choi_injectivity-I}*{\S 2.2}, the theorem
  is still needed to get certain control of the regularity of $u$ on the
  polar sets of $\vphi_F$ and $\vphi_M$ (see
  \cite{Chan&Choi_injectivity-I}*{Prop.~3.3.1}).
\end{remark}

%%% Local Variables:
%%% mode: latex
%%% TeX-master: "Injectivity-Fujino"
%%% coding: utf-8
%%% End:


%%%%% Reference list %%%%%
\begin{bibdiv}
  \begin{biblist}
    \IfFileExists{references.ltb}{
      \bibselect{references}
    }{
      %%%%%
%%%%% File name  : Injectivity-Fujino.tex
%%%%% Author     : Mario Chan
%%%%% Date       : 25th July, 2022 
%%%%% Description: This file is set up to compile the project with the
%%%%%              class amsart.cls. This project "Injectivity-Fujino"
%%%%%              proves the injectivity theorem on lc pairs of
%%%%%              arbitrary codimension of the mlc's, the full Fujino
%%%%%              conjecture.
%%%%%
%%
%%%
\documentclass[a4paper,12pt]{amsart}

\pdfoutput=1 %% added to force arXiv AutoTeX to typeset with pdflatex
             %% (so that rotation of symbols in Xy-pic is rendered);
             %% have to be within the first 5 lines of preamble

\usepackage[top=3cm,bottom=3cm,outer=3cm,inner=2cm,marginpar=2.45cm]{geometry}
\usepackage[destlabel,final,colorlinks=true]{hyperref}
\usepackage[abbrev]{amsrefs}

%%%%%
%%%%% File name  : packagesandcommands.tex
%%%%% Author     : Mario Chan
%%%%% Date       : 13th December, 2021 (original: 04th November, 2020)
%%%%% Description: This file collects the packages used and commands
%%%%%              defined in the project "Injectivity-Fujino".
%%%%%
%%
%%%

\usepackage[french,ngerman,english]{babel}
\usepackage[utf8]{inputenc}
\usepackage[T1]{fontenc}

% \usepackage{CJKutf8} %%%%% for the use of Chinese, use the
                       %%%%% environment
                       %%%%% \begin{CJK}{UTF8}{bkai} % or {bsmi}
                       %%%%% \end{CJK}

\usepackage[all]{xy}
\renewcommand{\objectstyle}{\displaystyle}

\usepackage{enumitem}
\usepackage{mathtools}  
\usepackage[usenames,dvipsnames]{xcolor}
\usepackage{calc} %% for doing length computations

\babeltags{de = ngerman}
\babeltags{fr = french}

\usepackage{upref}  %% for uprighting texts generated by \ref
\usepackage{embrac} %% for uprighting parentheses in \emph and
                    %% providing \embparen for the same effect in
                    %% theorem environments (which use {\itshape ...}
                    %% or {\em ...})

\usepackage[
% show-mario,  %% show editing notes or comments on margin when
             %% uncommented
% no-commands %% avoid loading commands in mariostdcommands.tex when uncommented
]{marionotations}
% \input{mariostdcommands} %% load mariostdcommands.tex separately
                           %% when uncommented (must put "no-commands"
                           %% in the package option of marionotations
                           %% in that case);

\usepackage[normalem]{ulem} %% used only in review report

\usepackage{bbm}     %% to use \mathbbm (like \mathbb but works also
                     %%                  for natural numbers)

\usepackage[Smaller]{cancel} %%%%% for crossing out argument in math mode via
                             %%%%% the  use of \cancelto 
\renewcommand{\CancelColor}{\color{Gray}}

\usepackage{breakurl}  %% used so that line breaks for contents in
                       %% \url{...} are possible when processed by
                       %% LaTeX instead of pdfLaTeX (e.g. arXiv.org) 

\usepackage{subfiles}

%%%%% Commands for this document %%%%%
\newcommand{\defaultDimension}{n}
\newcommand{\setDefaultDimension}[1]{\renewcommand{\defaultDimension}{#1}}

\newcommand{\defaultAmbientSpace}{X}
\newcommand{\setDefaultAmbientSpace}[1]{\renewcommand{\defaultAmbientSpace}{#1}}

\newcommand{\defaultlcIndex}{\sigma}
\newcommand{\setDefaultlcIndex}[1]{\renewcommand{\defaultlcIndex}{#1}}

\newcommand{\defaultcohDegree}{q}
\newcommand{\setDefaultcohDegree}[1]{\renewcommand{\defaultcohDegree}{#1}}

\newcommand{\defaultlclocus}{D}
\newcommand{\setDefaultlclocus}[1]{\renewcommand{\defaultlclocus}{#1}}

\newcommand{\defaultvphi}{\vphi_F}
\newcommand{\setDefaultvphi}[1]{\renewcommand{\defaultvphi}{#1}}

\newcommand{\defaultpsi}{\psi_D}
\newcommand{\setDefaultpsi}[1]{\renewcommand{\defaultpsi}{#1}}

\newcommand{\defaultMetric}{\omega}
\newcommand{\setDefaultMetric}[1]{\renewcommand{\defaultMetric}{#1}}


% The delimiters for the arguments are carefully chosen so that they
% are consistent among most of the commands (in particular for the
% commands for generating the symbols for the ideal sheaves and
% residue sheaves), namely,
%     <X>        for base space,
%     (S)        for lc locus,
%     |\sigma|   for lc index,
%     {\vphi}    for potential,
%     [\psi]     for $\psi$ function,
%     .{m_k}     for jumping number,
%     +{1}       for increment of lc index (by $1$),
%     -{1}       for decrement of lc index (by $1$).
%     /q/        for anti-holomorphic degree (or hol. and anti-hol. degrees)

\newcommand{\alert}[2][RoyalBlue]{{\color{#1}#2}}

\NewDocumentCommand{\logKX}{
  t{M} %% #1 include M in the tensor product if present
  o    %% #2 replace F \otimes M by the argument when provided
}{K_X \otimes D \otimes \IfNoValueTF{#2}{F \IfBooleanT{#1}{\otimes M}}{#2}}

% \NewDocumentCommand{\vphilist}{
%   D||{\vphi}           %% #1 potentials
%   t{F}                 %% #2 turn potential to "\vphi_F" if present
%   t{M}                 %% #3 add "+\vphi_M" if present
%   d()                  %% #4 extra metric for the (1,0)-forms
%   D<>{\defaultMetric}  %% #5 metric on the ambient space
% }{\IfBooleanTF{#2}{\vphi_F}{#1} \IfBooleanT{#3}{+\vphi_M}, \IfNoValueF{#4}{(#4),} #5}

\NewDocumentCommand{\Ltwo}{ %% the space of L2 sections 
  D//{\bullet,\bullet}      %% #1 the order of forms
  D<>{\defaultAmbientSpace} %% #2 base space
  s                         %% #3 base space is hidden if * is present
  m                         %% #4 coefficient
}{L^{#1}_{(2)}\paren{\IfBooleanF{#3}{#2;} #4}}

% \NewDocumentCommand{\Ltwosp}{
%   t{'}                    %% #1 no preassigned holomorphic degree if present
%   D//{\defaultcohDegree}  %% #2 anti-holomorphic degree
%   t{M}                    %% #3 include M in the coefficient if present
%   o                       %% #4 replace F \otimes M by the argument if provided
%   G{\defaultvphi}         %% #5 potential on line bundle
%   e{_}                    %% #6 metric on the base space or other subscripts
% }{\Ltwo/\IfBooleanF{#1}{\defaultDimension,}#2/*{D \otimes \IfNoValueTF{#4}{F \IfBooleanT{#3}{\otimes M}}{#4}}_{#5 \IfNoValueF{#6}{,#6}}}


%\def\mH{\mathcal{H}}
% \NewDocumentCommand{\Harm}{ %% the space of harmonic forms
%   O{q}
% }{\mathcal{H}^{n,#1}}
\NewDocumentCommand{\Harm}{ %% the space of harmonic forms
  t{'}                      %% #1 no preassigned hol degree if present
  D//{\defaultcohDegree}    %% #2 anti-holomorphic degree
  D<>{\defaultAmbientSpace} %% #3 the base space
  g                         %% #4 the coefficient; will be hidden
                            %%    together with the base space if not provided
  t{,}                      %% #5 separator
  G{\defaultvphi}           %% #6 potential on line bundle
  e{_}                      %% #7 metric on the base space or other subscripts
}{\mathcal{H}^{\IfBooleanF{#1}{\defaultDimension,}#2}\IfNoValueF{#4}{\paren{#3;#4}}_{#6 \IfNoValueF{#7}{,#7}}}


\NewDocumentCommand{\lcIndex}{ %% for displaying the lc index,
                               %% intended to be used internally
  m  %% #1 the basic lc index (\sigma)
  m  %% #2 amount added to the index
  m  %% #3 amount substracted from the index
}{#1\IfNoValueF{#2}{+#2}\IfNoValueF{#3}{-#3}}

\NewDocumentCommand{\lcData}{ %% for displaying lc data in the format
                              %% like "(\vphi_L ; m_k . \psi)"
  G{\defaultvphi}  %% #1 potential or q-psh function
  O{\defaultpsi}   %% #2 lc locus psi function
  e{.}             %% #3 jumping number
}{\paren{#1; \IfNoValueF{#3}{#3 \cdot} #2}}

\NewDocumentCommand{\lcdata}{ %% for displaying lc data in the
                              %% format like "(X,\vphi_L,\psi,m_k)"
  s                %% #1 no parentheses if starred 
  d<>              %% #2 base space
  G{\defaultvphi}  %% #3 potential or q-psh function
  O{\defaultpsi}   %% #4 lc locus psi function
  e{.,}            %% #5 jumping number
                   %% #6 extra components
}{\newcommand{\datalist}{\IfNoValueF{#2}{#2,}#3,#4\IfNoValueF{#5}{,#5}\IfNoValueF{#6}{,#6}}
\IfBooleanTF{#1}{\datalist}{\paren{\datalist}}}



\newcommand{\spHbase}{\mathbb{H}}
\NewDocumentCommand{\spH}{ %% cohomology group with coefficients 
                           %% K_X +F +D \otimes the given sheaf
  D//{\defaultcohDegree}  %% #1 degree of anti-hol form
  t{M}                    %% #2 with 'M' to display M in the %% coefficient
  m                       %% #3 
}{\spHbase^{#1}\paren{\IfBooleanT{#2}{M\otimes}#3}}
% \NewDocumentCommand{\spH}{ %% cohomology group with coefficients 
%                            %% vanishing on \lcc[\sigma]
%   D//{\defaultcohDegree}  %% #1 degree of anti-hol form
%   t{M}                    %% #2 with 'M' to display M in the coefficient
%   s                       %% #3 star for turning to the mlc adjoint ideal sheaf
%   D||{\defaultlcIndex}    %% #4 codim of the lcc defined by the upper ideal
%   t{.}                    %% #5 with '.' to display a quotient ideal
%   D||{#4 -1}              %% #6 codim of the lcc defined by the lower ideal
%   d()                     %% #7 the sheaf replacing the ideal sheaf if non-empty
% }{\spHbase^{#1}\IfNoValueTF{#7}{
%     \begingroup%
%     \newcommand{\upidl}{\IfBooleanTF{#3}{
%         \mtidlof{\vphi_{F \IfBooleanT{#2}{\otimes M}}}
%       }{\aidlof|#4|{\vphi_{F \IfBooleanT{#2}{\otimes M}}}}
%     }% 
%     \paren{\IfBooleanT{#2}{M\otimes}
%       \IfBooleanTF{#5}{
%         \frac{\upidl}{\aidlof|#6|{\vphi_{F \IfBooleanT{#2}{\otimes M}}}}
%       }{\upidl}}
%     \endgroup%
%   }{\paren{\IfBooleanT{#2}{M\otimes}#7}}}


\DeclareMathOperator{\lc}{lc} %% lc centre
\NewDocumentCommand{\lcc}{ %% union of lc centres
                           %% of codimension \sigma
                           %% of (X,D) %%
  D||{\defaultlcIndex}       %% #1 lc index \sigma
  e{+-}                      %% #2,#3
  D<>{\defaultAmbientSpace}  %% #4 base space
  t{'}                       %% #5 '-ed to show lc locus instead of
                             %%    lc data pair
  D(){\defaultlclocus}       %% #6 lc locus 
}{\lc_{#4}^{\lcIndex{#1}{#2}{#3}}\IfBooleanTF{#5}{\paren{#6}}{\lcData}}

\NewDocumentCommand{\lcS}{  %% a local lc centre
  s                       %% #1 symbol with \rs when starred
  D(){\defaultlclocus}    %% #2 symbol for the subvariety
  D||{\defaultlcIndex}    %% #3 codimension
  e{+-}                   %% #4,#5
  d<>                     %% #6 open set where the lc centre lives
  O{p}                    %% #7 index among the \sigma-lc centres
}{\mathtt{\IfBooleanT{#1}{\rs} #2}^{\lcIndex{#3}{#4}{#5}}_{\IfNoValueF{#6}{#6,}#7}}

\NewDocumentCommand{\PRes}{ %% Poincare Residue map
  O{}      %% subvariety
  d()      %% forms from the domain
}{\mathcal R_{#1}\IfNoValueF{#2}{\paren{#2}}}

\NewDocumentCommand{\HRes}{ %% Harmonic residue
  d()   %% #1 harmonic form
}{\mathfrak{R}\IfNoValueF{#1}{\paren{#1}}}

\newcommand{\defidlof}[1]{\mathcal{I}_{#1}}  %% defining ideal of (a set)
\NewDocumentCommand{\mtidlof}{   %% multiplier ideal of (a potential)
  O{}      %% #1 base space (for compatibility)
  D<>{#1}  %% #2 base space
  m        %% #3 potential / psh function
}{\multidl_{#2}\paren{#3}} 

% \NewDocumentCommand{\presidlof}{  %% multiplier ideal sheaf on the sum of
%                                   %% \sigma-lc centres
%   D||{\sigma}   %% codim of lc centres or supporting lc locus
%   m             %% potential or q-psh function
% }{\rs{\sheaf R}_{#1}\paren{#2}}

\NewDocumentCommand{\residlof}{  %% multiplier ideal sheaf on the
                                 %% union of \sigma-lc centres
  D||{\defaultlcIndex}   %% #1 codim of lc centres or supporting lc
                         %%    locus
  e{+-}                  %% #2,#3
  d<>                    %% #4 base space
  s                      %% #5 display the symbol without arguments when starred
  %%% input to \lcData
  % G{\defaultvphi}      %% #6 potential or q-psh function
  % O{\defaultpsi}       %% #7 lc locus psi function
  % e{.}                 %% #8 jumping number  
}{\sheaf R_{\IfNoValueTF{#4}{}{#4,} \lcIndex{#1}{#2}{#3}}\IfBooleanF{#5}{\lcData}}


\NewDocumentCommand{\Adjidlof}{
  D||{\defaultlcIndex}       %% #1 codim of lc centres under concern
  D<>{\defaultAmbientSpace}  %% #2 base space
  D(){\defaultlclocus}       %% #3 lc locus
  m                          %% #4 potential or ideal
}{\operatorname{\mathit{Adj}}^{#1}_{\paren{#2,#3}}\paren{#4}}


\NewDocumentCommand{\aidlof}{
  D||{\defaultlcIndex}   %% #1 codim of lc centres under concern
  e{+-}                  %% #2,#3
  d<>                    %% #4 base space
  s                      %% #5 display the symbol without arguments when starred
  %%% input to \lcData
  % G{\defaultvphi}        %% #6 potential or ideal
  % O{\defaultpsi}         %% #7 defining function of the lc locus
  % e{.}                   %% #8 jumping number
}{\sheaf{J}_{\!\IfNoValueTF{#4}{}{#4,} \lcIndex{#1}{#2}{#3}}\IfBooleanF{#5}{\lcData}}

\NewDocumentCommand{\faidlof}{
  D||{\defaultlcIndex}   %% #1 codim of lc centres in numerator
  e{+-}                  %% #2,#3
  t{/}                   %% #4 a separator for arguments
  D||{\defaultlcIndex}   %% #5 codim of lc centres in denominator
  e{+-}                  %% #6,#7
  % d<>                    %% #8 base space
  % s                      %% #9 display the symbol without arguments when starred
  %%% input to \lcData
  % G{\defaultvphi}        %% #10 potential or ideal
  % O{\defaultpsi}         %% #11 defining function of the lc locus
  % e{.}                   %% #12 jumping number
}{\fracAidlof{\lcIndex{#1}{#2}{#3}}{\lcIndex{#5}{#6}{#7}}}

\NewDocumentCommand{\fracAidlof}{
  m                  %% #1 lcIndex in numerator
  m                  %% #2 lcIndex in denominator
  d<>                %% #3 base space
  s                  %% #4 display the symbol without arguments when starred
  G{\defaultvphi}    %% #5 potential or ideal
  O{\defaultpsi}     %% #6 defining function of the lc locus
  e{.}               %% #7 jumping number
}{\frac{
    \aidlof|#1|<#3>*\IfBooleanF{#4}{\lcData{#5}[#6].{#7}}
  }{
    \aidlof|#2|<#3>*\IfBooleanF{#4}{\lcData{#5}[#6].{#7}}
  }}


\NewDocumentCommand{\lcV}{ %% measure on lc centres
  D||{\defaultlcIndex}    %% #1 codim of supporting lc centres
  D//{\defaultvphi}       %% #2 potential for bundle valued section
  d()                     %% #3 metric on the ambient space
  e{^}                    %% #4 jumping number
  O{\defaultpsi}          %% #5 defining function (potential) of subvariety 
}{\:d\operatorname{lcv}^{#1\IfNoValueF{#4}{,\paren{#4}}}_{\IfNoValueF{#3}{#3,}#2}\left[#5\right]}

\NewDocumentCommand{\Ohvol}{ %% Ohsawa measure %%
  D//{\defaultvphi} %% #1 potential for bundle valued section
  d()               %% #2 metric on the ambient space
  O{\defaultpsi}    %% #3 defining function of subvariety
}{\dvol_{\IfNoValueF{#2}{#2,}#1}\left[#3\right]} 


\newcommand{\dvol}{\:d\vol}


\NewDocumentCommand{\lcDataNormSubscript}{
  %% for displaying lc data in the format
  %% like "X, \vphi_L , m_k.\psi, \sigma", which is mainly used for
  %% subscript in a norm
  d<>                   %% #1 base space
  s                     %% #2 no potential and psi function when starred
  G{\defaultvphi}       %% #3 potential or q-psh function
  O{\defaultpsi}        %% #4 lc locus psi function
  e{.}                  %% #5 jumping number
  D||{\defaultlcIndex}  %% #6 lc Index
  e{+-}                 %% #7,#8
}{\IfNoValueF{#1}{#1,}
  \IfBooleanF{#2}{#3, \IfNoValueF{#5}{#5 \cdot} #4,}
  \lcIndex{#6}{#7}{#8}}


\newcommand{\RTFsym}{\mathfrak{F}} 
\NewDocumentCommand{\RTF}{ %% residue transform function
  s          %% #1 adding \smash[t] when starred
  G{\RTFsym} %% #2 symbol body
  o          %% #3 general superscript
  >{\SplitArgument{1}{,}} d<> %% #4 superscript in inner product
  d||        %% #5 superscript in \abs{}^2
  D(){\eps}  %% #6 for adding variable (\eps)
  t{,}       %% #7 separator
}{%
  \begingroup%
    \newif\ifsmasht%
    \IfBooleanTF{#1}{\smashttrue}{\smashtfalse}%
    \newif\ifboolup%
    \booluptrue%
    \IfNoValueT{#3}{\IfNoValueT{#4}{\IfNoValueT{#5}{\boolupfalse}}}%
    \newcommand{\supsrptstr}{\IfNoValueF{#3}{#3}\IfNoValueF{#4}{\inner#4}\IfNoValueF{#5}{\abs{#5}^2}}
    \newcommand{\RTFvar}{#6}
    #2\RTFprocess
}

\NewDocumentCommand{\RTFprocess}{
  o                     %% #1 overwrite subscript if given
  d<>                   %% #2 base space
  t{,}                  %% #3 with potential and psi function when ,-ed
  G{\defaultvphi}       %% #4 potential or q-psh function
  O{\defaultpsi}        %% #5 lc locus psi function
  e{.}                  %% #6 jumping number
  D||{\defaultlcIndex}  %% #7 lc Index
  e{+-}                 %% #8,#9
}{\newcommand{\subsrptstr}{%
    \IfNoValueTF{#1}{
    \IfNoValueF{#2}{#2,}
    \IfBooleanT{#3}{#4,#5,\IfNoValueF{#6}{#6,}}
    \lcIndex{#7}{#8}{#9}}{#1}}%
  \newcommand{\srptstr}{\cramped{{}^{\supsrptstr}%
      \ifboolup _
      \fi{\ifboolup\displaystyle\fi\paren{\RTFvar}%
          \ifboolup {\scriptstyle \subsrptstr} \else _{\subsrptstr} \fi%
        }}}%
  \ifboolup%
    \ifsmasht%
      \smash[t]{
        \raisebox{\depthof{$\srptstr$} * \real{0.3}}{$\srptstr$}%
      }%
    \else%
      \raisebox{\depthof{$\srptstr$} * \real{0.3}}{$\srptstr$}%
    \fi%
  \else%
    \srptstr%
  \fi%
  \endgroup%
}
% \NewDocumentCommand{\RTF}{ %% residue transform function
%   s          %% #1 adding \smash[t] when starred
%   G{\RTFsym} %% #2 symbol body
%   d//        %% #3 for adding superscript k for k-RTF
%   o          %% #4 general superscript
%   >{\SplitArgument{1}{,}} d<> %% #5 superscript in inner product
%   d||        %% #6 superscript in \abs{}^2
%   d()        %% #7 for adding variable (\eps)
%   o          %% #8 subscript for the codimension \sigma
% }{%
%   \begingroup%
%     \newif\ifboolup%
%     \booluptrue%
%     \IfNoValueT{#4}{\IfNoValueT{#5}{\IfNoValueT{#6}{\boolupfalse}}}%
%     \IfNoValueT{#7}{\boolupfalse}%
%     \newcommand{\srptstr}{\cramped{{}^{\IfNoValueF{#4}{#4}\IfNoValueF{#5}{\inner#5}\IfNoValueF{#6}{\abs{#6}^2}}%
%       \ifboolup _
%       \fi{\ifboolup\displaystyle\fi\IfNoValueF{#7}{\paren{#7}}\IfNoValueF{#8}{%
%           \ifboolup {\scriptstyle #8} \else _{#8} \fi%
%         }}}}%
%     \ifboolup%
%       \IfBooleanTF{#1}{
%         \smash[t]{
%           \IfNoValueF{#3}{{}^{#3}}#2\raisebox{\depthof{$\srptstr$} * \real{0.3}}{$\srptstr$}%
%         }%
%       }{\IfNoValueF{#3}{{}^{#3}}#2\raisebox{\depthof{$\srptstr$} * \real{0.3}}{$\srptstr$}}%
%     \else%
%       \IfNoValueF{#3}{{}^{#3}}#2\srptstr%
%     \fi%
%   \endgroup%
% } 

\def\RTI{\RTF{\mathfrak{I}}}


\NewDocumentCommand{\mtlog}{O{e} d() D||{\defaultpsi}}{\log\!#1^{\paren{#2}}\abs{#3}}
\NewDocumentCommand{\slog}{O{e} D||{\defaultpsi}}{\log\abs{#1 #2}}
\NewDocumentCommand{\dlog}{O{e} D||{\defaultpsi}}{\mtlog[#1](2)|#2|}


\NewDocumentCommand{\logpole}{ %% log-pole in the residue transform
                               %% function
  D||{\defaultpsi}       %% #1 log singularity defining function
  D//{\defaultlcIndex}   %% #2 codim of lc centres in question
  E{.^}{{e}{1+\eps}}     %% #3 multiplicative constant in logarithm 
                         %% #4 exponent in the log-psi term
  s                      %% #5 no parentheses and exponent on log|\psi| when starred
}{\abs{#1}^{#2} \IfBooleanTF{#5}{\slog[#3]|#1|}{\paren{\slog[#3]|#1|}^{#4}}}

\DeclareFontFamily{OMX}{MnSymbolE}{}
\DeclareSymbolFont{MnLargeSymbols}{OMX}{MnSymbolE}{m}{n}
\SetSymbolFont{MnLargeSymbols}{bold}{OMX}{MnSymbolE}{b}{n}
\DeclareFontShape{OMX}{MnSymbolE}{m}{n}{
    <-6>  MnSymbolE5
   <6-7>  MnSymbolE6
   <7-8>  MnSymbolE7
   <8-9>  MnSymbolE8
   <9-10> MnSymbolE9
  <10-12> MnSymbolE10
  <12->   MnSymbolE12
}{}
\DeclareFontShape{OMX}{MnSymbolE}{b}{n}{
    <-6>  MnSymbolE-Bold5
   <6-7>  MnSymbolE-Bold6
   <7-8>  MnSymbolE-Bold7
   <8-9>  MnSymbolE-Bold8
   <9-10> MnSymbolE-Bold9
  <10-12> MnSymbolE-Bold10
  <12->   MnSymbolE-Bold12
}{}
\DeclareMathDelimiter{\llangle}{\mathopen}%
{MnLargeSymbols}{'164}{MnLargeSymbols}{'164}
\DeclareMathDelimiter{\rrangle}{\mathclose}%
{MnLargeSymbols}{'171}{MnLargeSymbols}{'171}


\newcommand{\iinner}[2]{\left\llangle#1,#2\right\rrangle}
\newcommand{\eqcls}[1]{\left[#1\right]}


\NewDocumentCommand{\idxup}{ %% operator for raising indices via a
                             %% hermitian metric on X
  m                  %% #1 the differential form whose indices to be raised
  O{\defaultMetric}  %% #2 the hermitian metric on X
  t{,}               %% #3 separator
  o                  %% #4 extra superscripts
  s                  %% #5 smash the vertical spacing on the top of the metric if present
  t{.}               %% #6 with contraction operator \ctrt if '.'-ed
}{\paren{#1}^{
    % \mathrlap{
    \!\IfBooleanTF{#5}{\smash[t]{#2}}{#2}\IfNoValueF{#4}{, #4}
    % }
    % \makebox[\maxof{\widthof{$#2$}-\widthof{$\!\omega$}}{0pt}]{}
  }\IfBooleanT{#6}{\!\!\ctrt}}
% \NewDocumentCommand{\idxup}{ %% operator for raising indices via a
%                              %% hermitian metric on X
%   m                  %% #1 the differential form whose indices to be raised
%   O{*}               %% #2 the hermitian metric on X
%   s                  %% #3 smash the vertical spacing on the top of the metric if present
% }{\paren{#1}^{
%     \!\IfBooleanTF{#3}{\smash[t]{#2}}{#2}
%     % \makebox[\maxof{\widthof{$\scriptstyle #2$}-\widthof{$\!\omega$}}{0pt}]{}
%   }
% }

\newcommand{\dbadj}{\dbar^{\smash{\mathrlap{*}\;\:}}}


\NewDocumentCommand{\dep}{t{;} d<> O{\nu} m}{#4\IfBooleanTF{#1}{_}{^}{\IfNoValueF{#2}{#2\:}(#3)}}

\NewDocumentCommand{\sm}{s m}{{#2}\IfBooleanTF{#1}{_}{^}\text{sm}}

\newcommand{\tlog}{{\text{log}}}


\NewDocumentCommand{\idx}{ %% multi-indices
  O{i} %% #1 symbol of the indices
  m    %% #2 starting subscript
  o    %% #3 additional stuff to add before \dotsm
  t{.} %% #4 display "\dotsm" if '.'-ed
  t{,} %% #5 display ",\dots," if ','-ed
  o    %% #6 additional stuff to add after \dotsm
  m    %% #7 ending subscript
}{{#1}_{#2} \IfNoValueF{#3}{#3}
  \IfBooleanT{#4}{\dotsm} \IfBooleanT{#5}{,\dots,}
  \IfNoValueF{#6}{#6} {#1}_{#7}}



\newcommand{\charfct}{\mathbbm 1}


\newcommand{\cvr}[1]{\mathfrak{#1}} %% set of covering subsets
% \newcommand{\rs}[1]{\widetilde{#1}} %% putting ~ on objects on the
%                                     %% log-resolution %%
\NewDocumentCommand{\rs}{ %% putting ~ on objects on the
                          %% log-resolution %%
  s  %% when * is given, \smash[t] is applied
  m  %% the main object 
}{\IfBooleanTF{#1}{\smash[t]{\widetilde{#2}}}{\widetilde{#2}}}

% \NewDocumentCommand{\clt}{m}{\widetilde{#1}} %% element in complete space
% \NewDocumentCommand{\clomega}{O{\omega}}{{\clt{#1}}} %% complete metric

\newcommand{\BK}{\text{(BK)}}
\newcommand{\tBK}{\text{(tBK)}}
\DeclareMathOperator{\Ann}{Ann}  %% Annihilator 
\DeclareMathOperator{\mlc}{mlc} %% minimal lc centre
\DeclareMathOperator{\sym}{sym} %% symmetric polynomial
\newcommand{\Diff}{\operatorname{Diff}^*} %% general different (adjunction formula)

\newcommand{\sect}[1][s]{\mathtt{#1}} %% canonical section
\newcommand{\bphi}{\boldsymbol{\vphi}}
\newcommand{\bphip}[1][p]{\res\bphi_{#1}} %% retract-extension of
                                          %% \bphi from lc centre
                                          %% S^\sigam_p
\newcommand{\btau}{\boldsymbol{\tau}}
\newcommand{\shfP}{\sheaf P}  %% polar ideal sheaf
\NewDocumentCommand{\cbn}{  %% group of combinations
  D//{\defaultlcIndex_V}
  D||{\defaultlcIndex}
}{\mathfrak{C}^{#1}_{#2}} 
\NewDocumentCommand{\Iset}{  %% index set for lc centres on log-resolution
  D||{\defaultlcIndex}    %% #1
  e{+-}                   %% #2,#3
  O{\defaultlclocus}      %% #4
  d()                     %% #5 open set on which the index set is valid
}{I^{\lcIndex{#1}{#2}{#3}}_{#4}\IfNoValueF{#5}{\paren{#5}}} 
% }{I^{#1\IfNoValueF{#2}{+#2}\IfNoValueF{#3}{-#3}}_{#4}\IfNoValueF{#5}{\paren{#5}}} 

%%%%%%%%%%%%%%%%%%%%%%%%%%%%%%%%%%%%%%

\ifcsname defineNoThmInMarionotations\endcsname
  \relax
\else 

  \newtheorem{THMprop}{Proposition}[subsection]
  \newtheorem{THMlemma}[THMprop]{Lemma}
  \newtheorem{THMthm}[THMprop]{Theorem}
  \newtheorem{THMcor}[THMprop]{Corollary}
  % \newtheorem{SNCassumption}[THMprop]{Snc assumption}
  % \newtheorem{SNCassumptionx}{Snc assumption}
  % \renewcommand{\theSNCassumptionx}{\theSNCassumption${}^*$}

  % \newtheorem{definition-thm}[THMprop]{Definition-Theorem}

  \newtheorem{THMconjecture}[THMprop]{Conjecture}
  \newtheorem*{THMclaim}{Claim}

  \def\makeparenletter{\catcode`\(=11 \catcode`\)=11 }
  \def\makeparenother{\catcode`\(=12 \catcode`\)=12 }
  \def\makeparenactive{\catcode`\(=\active\catcode`\)=\active}

  \makeparenactive
  \NewDocumentEnvironment{textupparenenvir}{}{
    %%%%% This code may cause error when parentheses appear in places
    %%%%% where macro is not accepted, like \ref{...} or optional
    %%%%% arguments of enumerate. 
    % \catcode1=12
    % \catcode2=12
    % \mathcode1=\the\mathcode`\(
    % \delcode1=\the\delcode`\(
    % \mathcode2=\the\mathcode`\)
    % \delcode2=\the\delcode`\)

    % \begingroup
    % \lccode`\~=`\^^A
    % \lowercase{\endgroup
    % \everymath\expandafter{\the\everymath\let(^^28\let)^^29}
    % \everydisplay\expandafter{\the\everydisplay\let(^^28\let)^^29}
    % }

    \everymath\expandafter{\makeparenother}
    \everydisplay\expandafter{\makeparenother}

    \def({\textup{\char`\(}}
    \def){\textup{\char`\)}}

    \makeparenactive
    % \let\zzzlabel\label
    % \let\zzzref\ref
    % \let\zzznewlabel\newlabel

    % \def\label{\makeparenletter\wwwlabel}
    % \def\ref{\makeparenletter\wwwref}
    % \def\newlabel{\makeparenletter\wwwnewlabel}

    % \def\wwwlabel#1{\makeparenactive\zzzlabel{#1}}
    % \def\wwwref#1{\makeparenactive\zzzref{#1}}
    % \def\wwwnewlabel#1{\makeparenactive\zzznewlabel{#1}}
  }{\makeparenother}
  \makeparenother

  \NewDocumentEnvironment{prop}{ +o }{
    \IfNoValueTF{#1}{\begin{THMprop}}{\begin{THMprop}[{#1}]}
      \begin{textupparenenvir}
  }{
      \end{textupparenenvir}
    \end{THMprop}
  }

  \NewDocumentEnvironment{lemma}{ +o }{
    \IfNoValueTF{#1}{\begin{THMlemma}}{\begin{THMlemma}[{#1}]}
      \begin{textupparenenvir}
  }{
      \end{textupparenenvir}
    \end{THMlemma}
  }

  \NewDocumentEnvironment{thm}{ +o }{
    \IfNoValueTF{#1}{\begin{THMthm}}{\begin{THMthm}[{#1}]}
      \begin{textupparenenvir}
  }{
      \end{textupparenenvir}
    \end{THMthm}
  }

  \NewDocumentEnvironment{cor}{ +o }{
    \IfNoValueTF{#1}{\begin{THMcor}}{\begin{THMcor}[{#1}]}
      \begin{textupparenenvir}
  }{
      \end{textupparenenvir}
    \end{THMcor}
  }

  \NewDocumentEnvironment{conjecture}{ +o }{
    \IfNoValueTF{#1}{\begin{THMconjecture}}{\begin{THMconjecture}[{#1}]}
      \begin{textupparenenvir}
  }{
      \end{textupparenenvir}
    \end{THMconjecture}
  }

  \NewDocumentEnvironment{claim}{ +o }{
    \IfNoValueTF{#1}{\begin{THMclaim}}{\begin{THMclaim}[{#1}]}
      \begin{textupparenenvir}
  }{
      \end{textupparenenvir}
    \end{THMclaim}
  }

  \theoremstyle{remark}
  \newtheorem{remark}[THMprop]{Remark}

  \theoremstyle{definition}
  \newtheorem{definition}[THMprop]{Definition}
  \newtheorem{example}[THMprop]{Example}
  \newtheorem{notation}[THMprop]{Notation}

  \numberwithin{equation}{subsection}
  \renewcommand{\theequation}{eq$\,$\thesubsection.\arabic{equation}}


  
\fi

\allowdisplaybreaks  %% allow multi-line equations to spread across
                     %% pages 



%%% Local Variables:
%%% mode: latex
%%% TeX-master: "Injectivity-Fujino"
%%% End:

\ifx\pdftexversion\undefined
  \renewcommand{\ibar}{{\raisebox{-0.9ex}{$\mathchar'26$}\mkern-6.7mu i}}
\fi


% \usepackage{showkeys}

%%%%% End of preamble %%%%%%%%%%%%%%%%%%%%%%%%%%%%%%%%%%%%%%%%%%%%%%%%

\begin{document}

\citealias{Amb03}{Ambro_quasi-log-var}
\citealias{Amb14}{Ambro_injectivity}
\citealias{Eno90}{Enoki}
\citealias{EV92}{Esnault&Viehweg_book}
\citealias{Fuj11}{Fujino_log-MMP}
\citealias{Fuj12b}{Fujino_vanishing-thms}
\citealias{Fuj13a}{Fujino_injectivity-II}
\citealias{Fuj13b}{Fujino_injectivity-hodge-theoretic}
\citealias{Fuj15b}{Fujino_survey}



%%%%%
%%%%% File name  : titleinfo.tex
%%%%% Author     : Mario Chan
%%%%% Date       : 25th July, 202 (original: 13th December, 2021 (original: 04th November, 2020))
%%%%% Description: This file contains the info needed for maketitle
%%%%%              for the project "Injectivity-Fujino".
%%%%%
%%
%%%
\newcommand{\titlestr}{%
  % (Provisional)
  % A solution to the Fujino conjecture: injectivity theorem for
  % log-canonical pairs \\ on compact K\"ahler manifolds%
  An injectivity theorem on snc compact K\"ahler spaces: \\
  an application of the theory of
  harmonic integrals on log-canonical centers via adjoint ideal
  sheaves%
}

\newcommand{\shorttitlestr}{%
  An injectivity theorem on snc spaces%
}

\newcommand{\MCname}{Tsz On Mario Chan}
\newcommand{\MCnameshort}{Mario Chan}
\newcommand{\MCemail}{mariochan@pusan.ac.kr}

\newcommand{\YJname}{Young-Jun Choi}
\newcommand{\YJnameshort}{Young-Jun Choi}
\newcommand{\YJemail}{youngjun.choi@pusan.ac.kr}

\newcommand{\PNUAddressstr}{%
  Dept.~of Mathematics, Pusan National
  University, Busan 46241, South Korea%
}


\newcommand{\ShMname}{Shin-ichi Matsumura}
\newcommand{\ShMnameshort}{Shin-ichi Matsumura}
\newcommand{\ShMemail}{mshinichi0@gmail.com, mshinichi-math@tohoku.ac.jp}

\newcommand{\TohokuAddressstr}{%
  Mathematical Institute, Tohoku University, 6-3, Aramaki Aza-Aoba,
  Aoba-ku, Sendai 980-8578, Japan%
}


\newcommand{\subjclassstr}[1][,]{%
  32J25 (primary)#1  %% Transcendental methods of algebraic geometry (complex-analytic aspects) 
  32Q15#1   %% 	Kähler manifolds
  14B05 (secondary)%   %% Singularities in algebraic geometry
  % 14E30 (secondary)%   %% Minimal model program (Mori theory, extremal rays)
}

\newcommand{\keywordstr}[1][,]{%
  $L^2$ injectivity#1
  adjoint ideal sheaf#1
  multiplier ideal sheaf#1
  log-canonical center%
}

\newcommand{\dedicatorystr}{%
}

\newcommand{\thankstr}{%
}

%%% Local Variables:
%%% mode: latex
%%% TeX-master: "Injectivity-Fujino"
%%% coding: utf-8
%%% End:


\title[\shorttitlestr]{\titlestr}
 
\author[\MCnameshort]{\MCname}
\email{\MCemail}
% \address{\addressstr}
% \curraddr{}

\author{\YJname}
\email{\YJemail}
\address{\PNUAddressstr}

\author{\ShMname}
\email{\ShMemail}
\address{\TohokuAddressstr}


% \thanks{\thankstr}
 
\subjclass[2020]{\subjclassstr}

\keywords{\keywordstr}

% \dedicatory{\dedicatorystr}

% \begin{abstract}
%   \begin{abstract}

The Fast Reciprocal Square Root Algorithm is a well-established approximation technique consisting of two stages: first, a coarse approximation is obtained by manipulating the bit pattern of the floating point argument using integer instructions, and second, the coarse result is refined through one or more steps, traditionally using Newtonian iteration but alternatively using improved expressions with carefully chosen numerical constants found by other authors. The algorithm was widely used before microprocessors carried built-in hardware support for computing reciprocal square roots. At the time of writing, however, there is in general no hardware acceleration for computing other fixed fractional powers. This paper generalises the algorithm to cater to all rational powers, and to support any polynomial degree(s) in the refinement step(s), and under the assumption of unlimited floating point precision provides a procedure which automatically constructs provably optimal constants in all of these cases. It is also shown that, under certain assumptions, the use of monic refinement polynomials yields results which are much better placed with respect to the cost/accuracy tradeoff than those obtained using general polynomials. Further extensions are also analysed, and several new best approximations are given.

\end{abstract}

% \end{abstract} 

%%%SM's notation: I will change them to Mario's notation later 
% \newcommand{\Ker}[0]{\operatorname{Ker}}
\let\Ker\ker
\newtheorem{step}{Step}
%%%




%%choiyj's macros
\def\del{\partial}
\def\we{\wedge}
\def\ov{\overline}
\newcommand{\pd}[2]{\frac{\partial#1}{\partial#2}}
%%





\date{\today} 

\maketitle

%%%%% End of Top matter %%%%%%%%%%


\section{Introduction}\label{sec:intro}

{
  \let\thesubsection\thesection
  
  % This paper studies an analytic aspect of higher cohomology groups of adjoint bundles for lc $($log canonical$)$ pairs
  % aiming to solve Fujino's conjecture on the injectivity theorem as a benchmark. 
  This paper studies an analytic aspect of higher cohomology groups of adjoint bundles
  for log-canonical (lc) pairs aiming to solve Fujino's conjecture, 
  the injectivity theorem for lc pairs on compact K\"ahler manifolds, 
  following the line of Enoki's proof. 
  This is achieved by developing the theory of harmonic integrals
  on lc centers using the analytic adjoint ideal sheaves and the
  associated residue techniques.


  The injectivity theorem, a generalization of the Kodaira vanishing theorem to semi-positive line bundles, 
  plays an important role in higher dimensional algebraic geometry. 
  After the original Koll\'ar's injectivity theorem \cite{Kollar_injectivity} had been proved 
  for semi-ample line bundles on smooth projective varieties, 
  Enoki \cite{Eno90} generalized Koll\'ar's injectivity theorem 
  to semi-positive line bundles on compact K\"ahler manifolds. 
  Koll\'ar's proof is based on theory of Hodge structures, whereas
  Enoki's proof is based on the theory of harmonic integrals, a more
  well-suited and flexible technique in the complex analytic situation. 

  Ambro and Fujino generalized Koll\'ar's theory to varieties with lc
  singularities via the theory of mixed Hodge structures,  
  motivated by applications to birational geometry (see \cite{Amb03, Amb14, EV92, Fuj11, Fuj12b, Fuj13b}). 
  % \mariocomment{To SM: please
  % check if these references links to the correct papers. Change the
  % $\backslash$\texttt{citealias} commands if you wish.}% 
  % The works of Ambro and Fujino can be expected to
  It is expected that their works can also be generalized in the same line as Enoki's
  by developing an analytic treatment to lc singularities. 
  Motivated by this expectation, Fujino posed the conjecture below. 
  (Set $\ibar := \ibardefn$ \ibarfootnote\ and let $D$ be a reduced divisor for the
  rest of this article.)



  \begin{conjecture}[{Fujino's conjecture, \cite[Conjecture
      2.21]{Fuj15b}, cf.~\cite[Problem 1.8]{Fuj13a}}] 
    \label{conj:fujino}

    Let $X$ be a compact K\"ahler manifold and
    $D=\sum_{i=1}^{N}D_{i}$ be a simple-normal-crossing
    (snc) divisor on $X$.  Let $F$ be a semi-positive line bundle on
    $X$ (i.e.~it admits a smooth Hermitian metric $h_{F}$ with
    $\ibar\Theta_{h_F}(F) \geq 0$).  Consider a section
    $s \in H^{0}(X, F^{\otimes m})$ whose zero locus $s^{-1}(0)$
    contains no lc centers of the pair $(X,D)$ (i.e.~connected
    components of non-empty intersection
    $D_{i_{1}}\cap \cdots \cap D_{i_{k}}$ of the irreducible
    components $\{D_{i}\}_{i=1}^{N}$).  Then, the multiplication map
    induced by the tensor product with $s$
    \begin{equation*}
      H^{q}\paren{X, K_{X} \otimes D \otimes F}
      \xrightarrow{\otimes s} 
      H^{q}(X, K_{X} \otimes D \otimes F^{\otimes (m+1)} )
    \end{equation*}
    is injective for every $q$.
  \end{conjecture}

  The analytic theory corresponding to Koll\'ar's theory has been established for klt singularities 
  (see \cite{Cao&Demailly&Matsumura, Fujino&Matsumura, Gongyo&Matsumura,
    Matsumura_injectivity-survey, Matsumura_injectivity}).
  % but not for lc singularities. 
  Therefore, it remains to develop an analytic treatment to handle the
  lc singularities.
  % the analytic theory corresponding to the works of Ambro and Fujino
  % is interesting in terms of  studying the techniques of analytically treating lc singularities or mixed Hodge theory than just generalizing it. 


  The cases of $\dim X=2$ and plt pairs of arbitrary dimension have been
  solved in \cite{Matsumura_injectivity-lc,
    Matsumura_rel-vanishing-w-nd} (see also \cite{Chan&Choi_injectivity-I}). 
  A full solution to Fujino's conjecture is given recently by
  Junyan Cao and Mihai P\u{a}un \cite{Cao&Paun_LC-inj}.
  In this paper, independent of the results in \cite{Cao&Paun_LC-inj},
  we prove a {\textit{generalized version}} of Fujino's conjecture  
  (Theorem \ref{thm:main}) 
  % by developing the theory of harmonic integrals on simple normal corssing divisors. 
  by applying the theory of harmonic integrals on lc centers of the
  given lc pair.
  Fujino's conjecture is then a direct consequence of Theorem \ref{thm:main}. 
  

  \begin{thm}[Main Result]\label{thm:main}
    Let $X$ be a compact K\"ahler manifold  and 
    $D=\sum_{i=1}^{N}D_{i}$ be an snc divisor on $X$. 
    % such that each component $D_{i}$ is compact. 
    Let $F$ (resp.~$M$) be a line bundle on $X$ 
    with a smooth Hermitian metric $h_{F}$  (resp.~$h_{M}$) 
    such that 
    \begin{equation*}
      \ibar\Theta_{h_F}(F)\geq 0 \quad  \text{ and } \quad
      % \sqrt{-1}(\Theta_{h_F}(F)-t \Theta 
      % _{h_M}(M))\geq 0
      % -C\omega \leq
      \ibar\Theta_{h_M}(M) \leq C \ibar\Theta_{h_F}(F)
      \quad \text{ for some } C>0 \; . 
    \end{equation*}
    % (that is, $D_{i} \cap D_{j} = \emptyset$ for $i \not = j$ 
    % for the irreducible decomposition $D = \sum_{i\in I}D_{i}$). 
    Let $s$ be a  section of $M$  
    such that the zero locus $s^{-1}(0)$ 
    contains no lc centers of the lc pair $(X,D)$.
    Then, the multiplication map induced by the tensor product with $s$
    \begin{equation*}
      H^q(D, K_D \otimes F)
      \xrightarrow{\otimes s } 
      H^q(D, K_D \otimes F\otimes M)
    \end{equation*} 
    is injective for every $q$. 
  \end{thm}

  It can be seen from the proof that the compactness of $X$ in Theorem
  \ref{thm:main} is not necessary as soon as $D$ consists of only finitely many
  irreducible components which are compact.

  \begin{cor}[Solution to Fujino's conjecture]\label{cor:main}
    Conjecture \ref{conj:fujino} is true. 
  \end{cor}


  % Our paper differs from \cite{Cao&Paun_LC-inj} in the following points: 
  % The method of \cite{Cao&Paun_LC-inj} is based on the $L^{2}$-theory of $\dbar$-equations, 
  % whereas our method is based on the theory of harmonic integrals in the same line as in Enoki's work; 
  % specifically, we extend a technique of harmonic differential forms on smooth varieties to simple normal corssing divisors.

  Our proof differs from the one in \cite{Cao&Paun_LC-inj} in the following way.
  While both works make use of (some variant of) the Hodge
  decomposition for $L^2$ forms, Cao and P\u aun prove in
  \cite{Cao&Paun_LC-inj} a Hodge decomposition for $L^2$ forms with
  respect to a K\"ahler metric with conic singularities, which induces
  a Hodge decomposition on currents (which is called the Kodaira--de
  Rham decomposition in \cite{Cao&Paun_LC-inj}) in which the Green
  kernel has controllable singularities.

  For the sake of explanation, let $u$ be an $D\otimes F$-valued
  $(n,q)$-form representing a class in $\cohgp q[X]{\logKX}$
  % ($X$ being compact here)
  such that the class of $s u$ is $0$ in $\cohgp
  q[X]{\logKX M}$.
  Let also $\sect_D$ be a canonical section of $D$.
  Under our notation, the current that is under consideration in
  \cite{Cao&Paun_LC-inj} is $\frac{u}{\sect_D}$, which is not
  necessarily $L^2$ on $X$.
  Using the fact that $\eqcls{su} = 0$ in $\cohgp q[X]{\logKX M}$,
  Cao and P\u aun obtain $\frac{u}{\sect_D} =\dbar\theta + D'_{h_F}
  \beta_1 +\ibar\Theta_{h_F} \wedge \beta_2$, where $\theta$ is
  smooth while $\beta_1$ and $\beta_2$ have log-poles along
  $D+s^{-1}(0)$ (assumed to have only snc).
  It then follows from \cite{Cao&Paun_LC-inj}*{Thm.~1.1} (which
  makes use of the Hodge/Kodaira--de Rham decomposition) and the
  positivity $\ibar\Theta_{h_F} \geq 0$ that $u$ (or $u -\sect_D
  \dbar\theta$) is $\dbar$-exact.

  In our case, we make use of the residue exact sequences of adjoint
  ideal sheaves and the associated residue computation to reduce the
  setup to the union of \emph{$\sigma$-lc centers} of $(X,D)$ (i.e.~lc centers of
  codimension $\sigma$ in $X$, when $(X,D)$ is log-smooth and lc).
  Since each $\sigma$-lc center is a compact K\"ahler manifold, we
  have the Hodge decomposition (thus $L^2$ Dolbeault isomorphism and
  harmonic theory) at our disposal.
  Moreover, our reduction brings the setup essentially to the one in
  \cite{Matsumura_injectivity-lc}*{Thm.~1.6} or
  \cite{Chan&Choi_injectivity-I}*{Thm.~1.2.1} (corresponding to the
  case where $\frac{u}{\sect_D}$ is $L^2$).
  That's why we can follow the line of arguments in Enoki's proof to
  solve the conjecture via the theory of harmonic integrals on lc
  centers (and no extra resolution to bring $s^{-1}(0)$ into snc is
  needed).

  % Thanks to this advantage, we can obtain the generalized version  (not only Fujino's conjecture), 
  This approach gives us the advantage of obtaining Theorem
  \ref{thm:main}, a generalized version of Fujino's conjecture (see
  also Remark \ref{rem:general-commut-diagram} for other generalized
  statements which can be achieved),
  which does not seem to be derivable from results in \cite{Cao&Paun_LC-inj}, at
  least not directly. 
  % Furthermore, the previous works (including \cite{Cao&Paun_LC-inj, Amb03, Amb14, Fuj11} 
  % used the the assumption that $s^{-1}(0)$ contains no lc centers
  % to reduce the proof to the case where the pair $(X,D + s^{-1}(0))$ is log smooth; 
  % however, our paper uses this assumption to apply the inductive argument in terms of lc strata
  % by  the fact that all the data restricted to each component $D_{i}$ satisfy the assumption again. 



  Here we briefly explain the outline of the  proof of Theorem
  \ref{thm:main} with the example where the snc divisor $D$ has
  only two components $D_1$ and $D_2$ such that $D_1 \cap D_2$ is
  irreducible as an illustration.
  In this case, the union of the $1$-lc centers of $(X,D)$ is
  $\lcc|1|' = D_1 \cup D_2$ while that of the $2$-lc centers is
  $\lcc|2|' = D_1 \cap D_2$.
  For any given cohomology class $\alpha \in H^q(D,  K_{D} \otimes F)$
  such that $s  \alpha =0$ in $H^q(D,  K_{D} \otimes F \otimes M)$, 
  the goal is to show that $\alpha$ is actually $0$. 


  Write $h_F = e^{-\vphi_F}$ and $h_M = e^{-\vphi_M}$, and let $\psi_D
  := \phi_D -\sm\vphi_D :=\log\abs{\sect_D}^2 -\sm\vphi_D$ be a global
  function on $X$ such that $\phi_D$ is the (local) potential (of the
  curvature of a metric) on $D$ induced from a canonical section
  $\sect_D$ and $\sm\vphi_D$ is some smooth potential on $D$.
  When $D$ is smooth (i.e.~$D_{1}\cap D_{2}=\emptyset$), 
  the class $\alpha $ can be represented by $(u_{1},  u_{2})$, where $u_i$
  is a harmonic form with respect to $\vphi_F$ on $D_i$ in
  $\mathcal{H}^{n-1,q}(D_{i}; F)_{\vphi_{F}} \cong H^{q}(D_{i},
  K_{D_{i}} \otimes F)$ for $i=1,2$.
  Enoki's argument \cite{Eno90} shows that $s u_{i}$ is also a harmonic
  form with respect to $\vphi_F +\vphi_M$ using the
  Bochner--Kodaira--Nakano formula and the given curvature assumption.
  % by the Bochner trick and the assumption of curvatures.
  It follows from $s \alpha =0$ (as a class) that $s u_{i}=0$ (as a form), hence
  $\alpha=(u_{1}, u_{2})=0$ as desired. 

  However, when $D =\lcc|1|'$ (as well as other $\lcc'$ in the more
  general situation) is not smooth (i.e.~$D_{1}\cap D_{2} \neq
  \emptyset$), the Dolbeault and harmonic theories for cohomology groups
  on $D$ are not yet established, obstructing the use of Enoki's
  argument.
  To overcome this difficulty, we make use of the short exact sequence
  \begin{equation*}
    \xymatrix{
      {0} \ar[r]
      & {\bigoplus_{i=1}^2 K_{D_{i}} \otimes \res F_{D_i}} \ar[r]^-{\tau}
      & {K_{D} \otimes \res F_{D}} \ar[r]
      & {K_{D_{1} \cap D_{2}} \otimes \res F_{D_1 \cap D_2}} \ar[r]
      & {0}
    } \; ,
  \end{equation*}
  where $K_{D} :=K_{X}\otimes D \otimes \frac{\holo_X}{\defidlof{D}}$
  and $\defidlof{D}$ is the defining ideal sheaf of $D$ in $X$, and its
  associated long exact sequence of cohomology groups to reduce our
  injectivity problem of the map $\otimes s$ on $D$ to the injectivity
  problems of $\otimes s$ on the lc centers of $(X,D)$ (i.e.~$D_1$,
  $D_2$ and $D_1 \cap D_2$). 
  Note that all of the lc centers are not contained in $s^{-1}(0)$ by
  assumption and are compact K\"ahler manifolds on which the Dolbeault
  isomorphism and harmonic theory are available.

  Such strategy is suggested already in \cite{Matsumura_injectivity-lc}
  and is used there in the proof of the injectivity theorem for plt
  pairs.
  It is framed in \cite{Chan&Choi_injectivity-I} in terms of the adjoint
  ideal sheaves $\aidlof* := \aidlof = \mtidlof<X>{\vphi_F} \cdot
  \defidlof{\lcc+1'} = \defidlof{\lcc+1'}$ (the defining ideal sheaf
  of $\lcc+1'$ in $X$, under the assumption $\vphi_F$ being smooth)
  for integers $\sigma \geq 0$ and the corresponding residue morphisms
  $\Res^\sigma$ for $\sigma \geq 1$ (see Section \ref{subsec:residue}
  for the definitions).
  Writing $\lcc' = \bigcup_{p \in \Iset} \lcS$ as the decomposition of
  $\lcc'$ into the (irreducible) $\sigma$-lc centers $\lcS$, the residue
  morphism $\Res^\sigma$ induces the isomorphism
  \begin{equation*}
    \logKX \otimes \faidlof/-1* \xrightarrow[\isom]{\Res^\sigma}
    \logKX \otimes \residlof* := \bigoplus_{p \in \Iset} K_{\lcS}
    \otimes \res F_{\lcS} 
  \end{equation*}
  (notice that $\logKX \otimes \frac{\defidlof{D_1 \cap
      D_2}}{\defidlof{D}} \isom \bigoplus_{i=1}^2 K_{D_{i}} \otimes \res
  F_{D_i}$ and $\logKX \otimes \frac{\holo_X}{\defidlof{D_1 \cap D_2}}
  \isom K_{D_{1} \cap D_{2}} \otimes \res F_{D_1 \cap D_2}$ in the
  example).
  It can then be seen that, for more general $D$, the reduction can be
  done via the short exact sequences $0 \to \faidlof/-1* \to
  \faidlof|\sigma'|/-1* \to \faidlof|\sigma'|/* \to 0$ for some integers
  $\sigma$ and $\sigma'$ such that $1 \leq \sigma \leq \sigma'$,
  together with an induction on $\sigma$ via some diagram-chasing
  argument.
  See Step \ref{step:harmonic-rep} of Section
  \ref{sec:proof-of-simple-case} and the beginning of Section
  \ref{subsec:general} for precise details.


  After the reduction, we are led to consider the maps
  \begin{equation*}
    \renewcommand{\objectstyle}{\displaystyle}
    \xymatrix{
      {\smash{\bigoplus_{i =1}^2}\:\cohgp q[D_i]{K_{D_i} \otimes
          F}} \ar[r]^-{\tau}
      \ar[dr]^-{\nu}
      &{\cohgp q[D]{K_D \otimes F}}
      \ar[d]^-{\otimes s} \ar@{}@<-1em>[d]_*+{\circlearrowright}
      \\
      &{\cohgp q[D]{K_D \otimes F \otimes M} \; .}
    }
  \end{equation*}
  It suffices to prove that $\ker\nu =\ker\tau$ (Theorem
  \ref{thm:ker-nu=ker-tau}).
  Indeed, given the injectivity of the map $\otimes \res s_{D_1 \cap D_2}$ on
  $\cohgp q[D_1 \cap D_2]{K_{D_1 \cap D_2} \otimes F}$ followed from
  Enoki's argument in the previous case, we see that the given class
  $\alpha \in \cohgp q[D]{K_D \otimes F}$ actually lies in the image
  $\im\tau$ of $\tau$, say, $\alpha = \tau\paren{u_1, u_2}$ for some
  harmonic forms $u_i \in \Harm'/n-1,q/<D_i>{F},{\vphi_F} \isom \cohgp
  q[D_i]{K_{D_i} \otimes F}$.
  It will then follow that $\paren{u_1, u_2} \in \ker\nu
  =\ker\tau$, hence $\alpha =0$, as desired.
  The pair $(u_1,u_2)$ can be treated as a representative of $\alpha$.
  Suggested by the fact that a harmonic form is the unique
  representative with the \emph{minimal} $L^2$ norm among all elements
  in its corresponding $L^2$ Dolbeault cohomology class, we can choose
  an ``optimal'' representative of $\alpha$ such that $(u_1, u_2)$ has
  the \emph{minimal} distance from (i.e.~is orthogonal to) the subspace
  $\ker\tau$ with respect to the $L^2$ norm induced from $\vphi_F$.
  It then suffices to show that $u_i =0$ for $i = 1,2$ to prove that
  $\ker\nu =\ker\tau$.
  This is done by following the proof of
  \cite{Matsumura_injectivity-lc}*{Thm.~1.6} or
  \cite{Chan&Choi_injectivity-I}*{Thm.~1.2.1} (therefore following the
  spirit of Enoki's argument), but with a few technical modifications.

  One technical complication comes from the use of \v Cech cohomology
  for some cohomology groups (e.g.~$\cohgp q[D]{K_D \otimes F}$) due to
  the lack of the Dolbeault isomorphism.
  Another one is that the argument of Takegoshi in
  \cite{Chan&Choi_injectivity-I}*{\S 3.1, Step IV} (see also
  \cite{Matsumura_injectivity-lc}*{Prop.~3.13}), which essentially gives
  rise to an element in $\ker\tau$ constructed from $u_i$'s, is replaced
  by a construction of a harmonic forms $w$ (or a collection $w
  :=\paren{w_b}_{b\in \Iset+1}$ of harmonic forms for the general $D$)
  representing a class in $\cohgp{q-1}[D_1 \cap D_2]{K_{D_1 \cap D_2}
    \otimes F}$ (see \eqref{eq:w-prelim-formula} and \eqref{eq-def-w}).
  The class of $w$ has its image lying in $\ker\tau$ via the connecting
  morphism of the relevant long exact sequence.
  Such construction is suggested by a residue computation, which relates
  an inner product on (the normalization of) $\lcc|1|'$ to an inner
  product on (lower dimensional) $\lcc|2|'$ (see Proposition
  \ref{prop:res-formula-dbar-exact-dot-harmonic}; see also Steps
  \ref{item:express-su-in-residue-norm} and \ref{item:pf:use_u-ortho-w}
  in Section \ref{subsec:general}, or Steps
  \ref{item:expression-of-su-simple} and
  \ref{step:pf:use_u-ortho-w-simple} in Section
  \ref{sec:proof-of-simple-case} for less intensive notation).
  Such relation between the inner products shows that $w$ is the
  obstruction for having $u_i = 0$ for $i=1,2$.
  This becomes the crucial ingredient to complete the proof.

  The proof of Theorem \ref{thm:main} for the case of general $D$
  follows the same arguments.
  A brief comment for the case where $\vphi_F$ and $\vphi_M$ possess
  suitable analytic singularities is given in Remarks
  \ref{rem:singular-vphi_F} and \ref{rem:no-hard-Lefschetz}.
  
  







  % We briefly explain the proof in the simple case where $D$ has two components (i.e.,\,$D=D_{1}+D_{2}$). 
  % For a given cohomology class $\alpha \in H^q(D,  K_{D} \otimes F)$, 
  % we will prove that $\alpha $ is actually zero 
  % under assuming that $s  \alpha =0 \in H^q(D,  K_{D} \otimes F \otimes M)$. 

  % In the case where $D$ is smooth (i.e.,\,$D_{1}\cap D_{2}=\emptyset$), 
  % the class $\alpha $ can be represented by a harmonic form $(u_{1},  u_{2})$. 
  % Here we used 
  % $$
  % H^q(D,  K_{D} \otimes F)=\oplus_{i=1}^{2} H^{q}(D_{i},  K_{D_{i}} \otimes F) \cong 
  % \oplus_{i=1}^{2} \mathcal{H}^{n-1,q}(D_{i}, F)_{h_{F}}, 
  % $$ 
  % where $\mathcal{H}^{n-1,q}(D_{i}, F)_{h_{F}}$ is the space of harmonic forms with respect to $h_{F}$. 
  % Enoki's argument \cite{Eno90} shows that  
  % $s u_{i}$ is also a harmonic form with respect to $h_{F} h_{M}$
  % by the Bochner trick and the assumption of curvatures. 
  % This implies that $s u_{i}=0$ by $s \alpha =0$; hence $\alpha=\{(u_{1}, u_{2})\}=0$. 

  % In the general case  (i.e.,\,$D_{1}\cap D_{2} \not =\emptyset$), 
  % it is not clear whether the class $\alpha$ can be represented by a harmonic form on $D$,  
  % which is an obvious  difficulty in extending Enoki's argument.
  % To overcome this difficulty, we consider the long exact sequence 
  % $$
  % \cdots \to\bigoplus_{i=1}^{2} H^q(D_{i},  K_{D_{i}}\otimes F ) \xrightarrow{\tau} H^q(D,  K_{D} \otimes F)  \to H^q(D_{1}\cap D_{2},  K_{D_{1}\cap D_{2}}\otimes F ) \to \cdots
  % $$
  % induced by $0 \to K_{D_{1}} \oplus K_{D_{2}} \to K_{D} \to K_{D_{1} \cap D_{2}} \to 0$, 
  % noting that $ K_{D}=(K_{X}\otimes D)\otimes \mathcal{O}_{X}/\mathcal{I}_{D}$. 
  % The multiplication map defined on the right term 
  % $$
  % \otimes s |_{D_{1} \cap D_{2}}: H^q(D_{1}\cap D_{2},  K_{D_{1}\cap D_{2}}\otimes F ) \to
  % H^q(D_{1}\cap D_{2},  K_{D_{1}\cap D_{2}}\otimes F \otimes M ) 
  % $$ 
  % is non-zero since $s^{-1}(0)$ contains no lc centers of the pair $(X,D)$, 
  % and thus injective by induction hypothesis. 
  % Thus, by chasing the commutative diagram induced by the multiplication map, 
  % we can take a cohomology class 
  % $\beta \in \oplus_{i=1}^{2} H^q(D_{i},  K_{D_{i}}\otimes F )$ such that $\tau(\beta)=\alpha$. 
  % Then, we can take a harmonic representation $(u_{1}, u_{2})$ of $\beta$. 


  % The pair $(u_{1}, u_{2})$ is the {\textit{best}} representation for $\beta$ 
  % in the sense that $(u_{1}, u_{2})$ has the minimum $L^2$ norm in the forms representing $\beta$. 
  % However, the pair $(u_{1}, u_{2})$ may not be the best representation for $\alpha$ 
  % since the $L^2$ norm may be reduced by $\tau$. 
  % For this reason, by the orthogonal decomposition, 
  % we re-choose $\beta$ (and its harmonic representation $(u_{1}, u_{2})$) 
  % satisfying $(u_{1}, u_{2}) \in (\Ker \tau)^{\perp} \subset \oplus_{i=1}^{2} \mathcal{H}^{n-1,q}(D_{i}, F)_{h_{F}}$. 
  % The condition $(u_{1}, u_{2}) \in (\Ker \tau)^{\perp}$ means a certain minimal $L^{2}$-norm; 
  % therefore $(u_{1}, u_{2})$ can be seen as the best representation for $\alpha$. 


  % The Bocher trick shows that the $L^2$ norm of $(u_{1}, u_{2})$ is zero in the case $D_{1}\cap D_{2}=\emptyset$. 
  % By generalizing this Bocher trick, 
  % an obstruction for the $L^{2}$-norm to be $0$ 
  % can be described by a $F$-valued differential form $w$ on $D_{1} \cap D_{2}$. 
  % An important point here is that  $w$ is actually harmonic; in particular, it determines the cohomology class.   
  % $H^{q-1}(D_{1}\cap D_{2},  K_{D_{1}\cap D_{2}}\otimes F )$. 
  % Then, we can show that the $L^{2}$-norm of $w$ on $D_{1} \cap D_{2}$ 
  % is equal to the inner product of $(u_{1}, u_{2})$ and a representation of $\delta(w)$, 
  % where $\delta$ is the connecting morphism 
  % $\delta: H^{q-1}(D_{1}\cap D_{2},  K_{D_{1}\cap D_{2}}\otimes F ) \to \oplus_{i=1}^{2} H^{q}(D_{i},  K_{D_{i}} \otimes F)$. 
  % This implies that $w=0$ by $(u_{1}, u_{2}) \in (\Ker \tau)^{\perp}$.


  % 
  % ....


}

This paper is organized as follows.
\tableofcontents


\subsection*{Acknowledgments}
The authors would like to thank the members of Bayreuth University and Pusan National University for their hospitality.
This paper is resulted from the discussions there. 
S.M.~would like to thank Professors Junyan Cao and Mihai P\u{a}un for sharing a preliminary version of \cite{Cao&Paun_LC-inj}.
Also, he would like to thank Professor Osamu Fujino 
for his encouragement and long-standing discussions on lc singularities. 
He is partially supported 
by Grant-in-Aid for Scientific Research (B) $\sharp$21H00976 
and Fostering Joint International Research (A) $\sharp$19KK0342 from
JSPS.
Y.C.~and M.C.~would like to thank S.M.~for drawing their attention to
Fujino's conjecture not long before the covid pandemic (which results
in \cite{Chan&Choi_injectivity-I}) and for joining hand to complete
this project when most aspects of life went back to normal.
Y.C.~and M.C.~were supported by the National Research Foundation
of Korea (NRF) Grant funded by the Korean government
(Nos.~2023R1A2C1007227 and 2021R1A4A1032418).



\section{Preliminary results}\label{sec:preliminaries}

\subsection{Notation and conventions}\label{subsec:notation}

%%%%%
%%%%% File name  : notation.tex
%%%%% Author     : Mario Chan
%%%%% Date       : 13th December, 2021 (original: 04th November, 2020)
%%%%% Description: This is the section "Notation" in the project
%%%%%              "Injectivity-Fujino".
%%%%%
%%
%%%

% In this subsection, we summarize the notation used throughout this paper. 

The following notions are used throughout this paper unless stated otherwise. 
\begin{itemize}
\item $(X,\omega)$ is a compact K\"ahler manifold of dimension $n$. 

% \item $\omega$ is a K\"ahler form on $X$. 

\item $h_F := e^{-\vphi_F}$ and $h_M := e^{-\vphi_M}$, where $\vphi_F$ and
  $\vphi_M$ are respectively the given potentials on $F$ and $M$.
  
\item $D=\sum_{i \in \Iset||}D_{i}$ is a reduced simple-normal-crossing (snc)
  divisor on $X$ (where $\Iset||$ is a finite set). 

\item $\sect_i$ is a canonical section  of the irreducible component $D_{i}$. 

\item $\sect_D := \prod_{i\in \Iset||} \sect_i$ is the canonical section of $D$. 

\item $\sigma \in \{0,1,2,\cdots, n\}$.

% \item  $\Iset$ is the set of $p:=\{i_{1}, i_{2}, \cdots, i_{\sigma}\}$ such that  
% \mmark{$\lcS:=\cap_{k=1}^{\sigma} D_{i_{k}} $}{$\cap D_{i_k}$ may have more than
% one component.} is of codimension $\sigma$. 

% \item   $\lcc' := \cup_{p \in \Iset} \lcS$ is the union of $\sigma$-lc centers $\lcS$ of  $(X,D)$

  
\item $\lcc' :=\bigcup_{p \in \Iset} \lcS$ is the union of
  \emph{$\sigma$-lc centers of $(X,D)$}, i.e.~the
  $\sigma$-codimensional irreducible components of any intersections
  of irreducible components of $D$ (under the assumption $(X,D)$ being
  log-smooth and lc), indexed by $\Iset$.
  Set $\lcc|0|' := X$ and let $\Iset|0|$ be a singleton for convenience.
  Note also that $\Iset|1| = \Iset||$.

\item $\Diff_{p}D$ is the effective divisor on $\lcS$ defined by the 
adjunction formula 
\begin{equation*}
  K_{\lcS} \otimes \Diff_{p}D = \parres{K_X \otimes D}_{\lcS}
\end{equation*}
such that the restriction of $\sect_{(p)}:=
\smashoperator{\prod\limits_{i \in \Iset|| \colon D_i
    \not\supset \lcS}} \sect_i $ to $\lcS$ is a canonical section of
$\Diff_{p}D$.

\item $\phi_D :=\log\abs{\sect_D}^2$ and $\phi_{(p)}
  :=\log\abs{\sect_{(p)}}^2$ are the potentials induced from the
  canonical sections of $D$ and $\Diff_p D$.

\item $\cvr V := \{V_{i}\}_{i \in I}$ is an open cover of $X$  by admissible open sets. 

\item $\{\rho^{i}\}_{i\in I}$ is a partition of unity subordinate to
  $\cvr V$. 
\end{itemize}

Here an open set $V \subset X$ is said to be \emph{admissible} with
respect to $D$ if $V$ is biholomorphic to a polydisc centered at the
origin under a holomorphic coordinate system $(z_{1}, z_{2}, \cdots,
z_{n})$ such that
\begin{equation*} % \label{eq:local-expression-bphi-psi}
  D =\set{z_1 \dotsm z_{\sigma_V} =0}, \quad 
  \log r_{j}^2 < 0, \quad \text{and }
  r_j \fdiff{r_j} \psi_D >0 \text{ on } V \; , 
  % \res{\vphi_\bullet}_V = \smashoperator{\sum_{k=\sigma_V+1}^n} b_{\bullet,k}
  % \log\abs{z_k}^2 +\beta_\bullet \;\;\text{ for } \bullet= F, M \; ,
\end{equation*} 
where  $r_j := \abs{z_j}$  and $\res{\psi_D}_V := \parres{\phi_D
  -\sm\vphi_D}_V =\sum_{j=1}^{\sigma_V} \log\abs{z_j}^2
-\res{\sm\vphi_D}_V$. 

When an admissible set $V$ is considered, an index $p \in \Iset$ such
that $\lcS \cap V \neq \emptyset$ is interpreted as a permutation
representing a choice of $\sigma$ elements from the set
$\set{1,2,\dots,\sigma_V}$ such that
\begin{equation*}
  \lcS \cap V = \set{z_{p(1)} = z_{p(2)} = \dotsm = z_{p(\sigma)} = 0}
  \quad\text{ and }\quad
  \res{\sect_{(p)}}_V = z_{p(\sigma+1)} \dotsm z_{p(\sigma_V)}
\end{equation*}
(cf.~the definition of the set $\cbn$ in \cite{Chan_adjoint-ideal-nas}*{\S 3.1}).



%%% Local Variables:
%%% mode: latex
%%% TeX-master: "Injectivity-Fujino"
%%% coding: utf-8
%%% End:


%\subfile{commut-diagram_Fujino-conj}%

\subsection{$L^{2}$ Dolbeault isomorphism and some results on harmonic
forms}\label{subsec:l2}

%%%%%
%%%%% File name  : L2-spaces-n-harmonic-forms.tex
%%%%% Author     : Mario Chan
%%%%% Date       : 27th March, 2023
%%%%% Description: This is the section on the basic facts of the Hodge
%%%%%              decomposition and the implications of positivity on
%%%%%              harmonic forms which have been discussed in
%%%%%              previous papers.
%%%%%
%%
%%%

{
  \setDefaultvphi{\vphi_L}

  % Suppose that $X$ is \emph{compact} K\"ahler in this section.
  Let $L$ be a holomorphic line bundle on $X$ equipped with a
  (possibly singular) quasi-psh potential $\vphi_L$, which induces,
  together with the K\"ahler form $\omega$, an $L^2$ norm
  $\norm{\cdot}_{X} := \norm\cdot_{X,\vphi_L,\omega}$ on the space of
  smooth $K_X \otimes L$-valued $(0,q)$-forms (or $L$-valued
  $(n,q)$-forms) on $X$.
  Let $\Ltwo/n,q/{L}_{\vphi_L}$ be the completion with respect to $\norm\cdot_X$
  and $\Harm :=\Harm{L}$ be the space of harmonic forms with respect to $\norm\cdot_X$.
  The $L^2$ Dolbeault isomorphism (see
  \cite{Matsumura_injectivity}*{Prop.~5.5 and 5.8} and
  \cite{Matsumura_injectivity-lc}*{Prop.~2.8} for a proof, and see 
  \cite{Chan&Choi_injectivity-I}*{footnote 1} for its naming) guarantees
  the closedness of the subspaces in the orthogonal decomposition
  \begin{equation*}
    \Ltwo/n,q/{L}_{\vphi_L}
    = \Harm \oplus \cl{\paren{\im\dbar}}_{\vphi_L} \oplus \cl{\paren{\im\dbadj}}_{\vphi_L}
    = \Harm \oplus \paren{\im\dbar}_{\vphi_L} \oplus \paren{\im\dbadj}_{\vphi_L} 
  \end{equation*}
  (where $\dbadj$ is the Hilbert space adjoint of $\dbar$ with respect
  to $\norm\cdot_X$, $\paren{\im\dbar}_{\vphi_L}$ and
  $\paren{\im\dbadj}_{\vphi_L}$ denote the images of the corresponding
  operators, with $\cl{\paren{\im\dbar}}_{\vphi_L}$ and
  $\cl{\paren{\im\dbadj}}_{\vphi_L}$ being their closures in
  $\Ltwo/n,q/{L}_{\vphi_L}$)
  and the isomorphism
  \begin{equation*}
    \Harm \isom \cohgp q[X]{K_X \otimes L \otimes \mtidlof{\vphi_L}}
  \end{equation*}
  between the space of harmonic forms and the \v Cech cohomology
  group.
  % Given a locally finite Stein cover $\cvr V = \set{V_i}_{i \in I}$
  % with a partition of unity $\set{\rho^i}_{i\in I}$ subordinate to,
  With $\cvr V := \set{V_i}_{i\in I}$ and $\set{\rho^i}_{i\in I}$ given in Section \ref{subsec:notation},
  the isomorphism can be given explicitly as follows.
  For any (alternating) \v Cech $q$-cocycle $\set{\alpha_{\idx 0.q}}_{\idx 0,q \in
    I}$ and any harmonic form $u \in \Harm$ such that they represent
  the same class in $\cohgp q[X]{K_X \otimes L \otimes
    \mtidlof{\vphi_L}}$, the two representatives are related by 
  (under the Einstein summation convention)
  \begin{equation} \label{eq:Cech-Dolbeault-isom}
    \begin{aligned}
      u &=\dbar v_{(2)} +\dbar \rho^{i_{q-1}} \wedge \dotsm \wedge
      \dbar\rho^{i_0} \alpha_{\idx 0.q} \qquad\paren{\forall~ i_q \in
        I}
      \\
      &=\dbar v_{(2)} +\dbar \rho^{i_{q-1}} \wedge \dotsm \wedge
      \dbar\rho^{i_0} \cdot \rho^{i_q} \:\alpha_{\idx 0.q}
      \\
      &=\dbar v_{(2)} +(-1)^q \:\underbrace{\dbar \rho^{i_{q}} \wedge
        \dotsm \wedge \dbar\rho^{i_1} \cdot \rho^{i_0} }_{=: \:
        \paren{\dbar\rho}^{\idx q.0}} \alpha_{\idx 0.q}
    \end{aligned}
  \end{equation}
  for some $K_X \otimes L$-valued $(0,q-1)$-form $v_{(2)}$ on $X$ with
  $L^2$ coefficients with respect to $\norm\cdot_{X}$ (see
  \cite{Matsumura_injectivity}*{Prop.~5.5} or
  \cite{Chan&Choi_injectivity-I}*{Lemma 3.2.1}).

  The above result is applicable also to the case when $L$ is replaced by
  $D \otimes L$ equipped with the potential $\phi_D +\vphi_L$, where
  $\phi_D :=\log\abs{\sect_D}^2$.
  Denote the corresponding $L^2$ norm by $\norm\cdot_{X,\phi_D}$.
  Assume that \emph{$\vphi_L$ is smooth on $X$}.
  We state the following simple fact here for clarity.
  \begin{lemma} \label{lem:su-harmonicity}
    If $u \in \Harm{L}$, then $\sect_D u \in \Harm{D\otimes L},{\phi_D+\vphi_L}$.
  \end{lemma}
  
  \begin{proof}
    Since $\sect_D$ is holomorphic, it is clear that $\sect_D u$ is
    $\dbar$-closed.

    Let $\dfadj$ and $\dfadj_{\phi_D}$ be the formal adjoint of
    $\dbar$ with respect to $\vphi_L$ and $\phi_D +\vphi_L$
    respectively.
    It then follows that $\dfadj_{\phi_D} = \dfadj
    +\idxup{\diff\phi_D} . \cdot$ and 
    \begin{equation*}
      \dfadj_{\phi_D} \paren{\sect_D u}
      = \sect_D \:\dfadj u - \idxup{\diff\sect_D}. u
      +\idxup{\diff\phi_D} .\sect_D u
      =\sect_D \dfadj u = 0 \; .
    \end{equation*}
    Note that $\omega$ is not complete on $X \setminus D$ and the
    claim (in particular, $\sect_D u \in \Dom \dbadj_{\phi_D}$, where
    $\dbadj_{\phi_D}$ is the Hilbert space adjoint of $\dbar$ with
    respect to $\norm\cdot_{X,\phi_D}$) cannot follow from the
    standard result (for example, \cite{Demailly}*{Ch.~VIII,
      Thm.~(3.2c)}).
    Indeed, the proof of $su \in \Dom
    \dbadj_{\vphi_M}$ in \cite{Chan&Choi_injectivity-I}*{Cor.~3.2.6}
    gives precisely the result $\sect_D u \in \Dom\dbadj_{\phi_D}$ in
    the current setting, which completes the proof.
    A sketch of it is given below for readers' convenience.
    
    Let $\theta \colon [0,\infty) \to [0,1]$ be a smooth
    non-decreasing cut-off function such that
    $\res\theta_{[0,\frac12]} \equiv 0$ and $\res\theta_{[1,\infty)}
    \equiv 1$.
    Set $\theta_\eps := \theta \circ \frac{1}{\abs{\psi_D}^\eps}$ and
    $\theta'_\eps := \theta' \circ \frac{1}{\abs{\psi_D}^\eps}$ for
    every $\eps \geq 0$ (where $\theta'$ is the derivative of
    $\theta$).
    Then both $\theta_\eps$ and $\theta'_\eps$ have compact supports
    inside $X \setminus D$ for $\eps > 0$ and $\theta_\eps \ascendsto
    1$ pointwisely on $X \setminus D$ as $\eps \descendsto 0$.
    For any $\zeta \in \Dom\dbar \subset \Ltwo/n,q-1/<X>{D\otimes
      L}_{\phi_D+\vphi_L}$, convolution with a smoothing kernel on
    local coordinate charts and the lemma of Friedrichs guarantees the
    existence of a sequence $\seq{\zeta_{\eps, \nu}}_{\nu\in\Nnum}$ of
    smooth forms compactly supported in $X \setminus D$ such that
    $\zeta_{\eps,\nu} \tendsto \theta_\eps \zeta$ in the graph norm
    $\paren{\norm\cdot_{X,\phi_D}^2
      +\norm{\dbar\:\cdot}_{X,\phi_D}^2}^{\frac 12}$ of $\dbar$ for
    each $\eps > 0$.
    It then follows that
    \begin{align*}
      \iinner{\sect_D u}{\dbar\zeta}_{X,\phi_D} 
      \xleftarrow{\eps \tendsto 0^+}
      &~\iinner{\sect_D u}{\theta_\eps \dbar\zeta}_{X,\phi_D} \\
      =&~\iinner{\sect_D u}{\dbar\paren{\theta_{\eps}\zeta}}_{X,\phi_D}
         -\iinner{\sect_D u}{\dbar\theta_\eps \wedge \zeta}_{X,\phi_D} \\
      \xleftarrow{\nu \tendsto \infty}
      &~\iinner{\sect_D u}{\dbar\zeta_{\eps,\nu}}_{X,\phi_D}
        -\iinner{\sect_D u}{\frac{\eps \theta'_\eps}{\abs{\psi_D}^{1+\eps}}
        \dbar\psi_D \wedge \zeta}_{X,\phi_D} \\
      =&~\iinner{\dfadj_{\phi_D} \paren{\sect_D u}}{\zeta_{\eps,\nu}}_{X,\phi_D}
         -\iinner{\frac{\eps \theta'_\eps}{\abs{\psi_D}^{1+\eps}}
         \idxup{\diff\psi_D} . \sect_D u}{\zeta}_{X,\phi_D} \; .
    \end{align*}
    The inner product on the far right-hand-side converges to $0$ as
    $\eps \tendsto 0^+$, a consequence of the residue computation (see
    \cite{Chan&Choi_injectivity-I}*{Prop.~3.2.3 and Remark 3.2.4}).
    We can then conclude that $\sect_D u \in \Dom\dbadj_{\phi_D}$ after
    letting $\nu \tendsto \infty$ and then $\eps \tendsto 0^+$.
  \end{proof}

}

Now consider the cases where $(L, \vphi_L) =(F, \vphi_F)$
% (with the induced $L^2$ norm $\norm\cdot_{X}$)
and $(L, \vphi_L) =(F\otimes M, \vphi_F +\vphi_M)$.
% (with the induced $L^2$ norm $\norm\cdot_{X,\vphi_M}$).
A consequence of the positivity on $F$ and $M$ in
\cite{Enoki}, \cite{Matsumura_injectivity-lc} and
\cite{Chan&Choi_injectivity-I} are recalled below.
\begin{prop} \label{prop:consequence-of-positivity}
  % Suppose $\vphi_F$ and $\vphi_M$ are smooth such that
  % $\ibddbar\vphi_F \geq 0$ and $C\ibddbar\vphi_F \geq \ibddbar\vphi_M
  % \;\paren{\geq - C \omega}$ for some constant $C > 0$.
  % Then, $u \in \Harm{F}$ implies $su \in \Harm{F\otimes
  %   M},{\vphi_F+\vphi_M}$ and $\nabla^{(0,1)}u = 0$.
  Suppose that $\vphi_F$ is smooth such that
  $\ibddbar\vphi_F \geq 0$ and $u \in \Harm{F}$.
  Then, one has  $\nabla^{(0,1)}u = 0$.
  If, furthermore, $\vphi_M$ is smooth and satisfies
  $\paren{- C \omega \leq} \; \ibddbar\vphi_M \leq C\ibddbar\vphi_F$
  for some constant $C > 0$,
  then one also has $su \in \Harm{F\otimes M},{\vphi_F+\vphi_M}$.
\end{prop}

\begin{proof}[Reference to the proof]
  These results follow directly from the Bochner--Kodaira--Nakano
  formula.
  See \cite{Chan&Choi_injectivity-I}*{Prop.~3.2.5 and
    Cor.~3.2.6} (while taking $D=0$ and $\psi_D \equiv -1$ in those
  statements).
  See also the proofs for $\diff^*_h\xi = 0$ in
  \cite{Enoki}*{Prop.~2.1} or $D'^*u = 0$ in
  \cite{Matsumura_injectivity-lc}*{Prop.~3.7}.
  These are equivalent statements to the claim $\nabla^{(0,1)}u =0$
  (indeed, $\diff^*_h = D'^*$ and $\abs{D'^*u}^2 =
  \abs{\nabla^{(0,1)}u}^2$ by \cite{Chan&Choi_injectivity-I}*{Remark
    2.4.3}).
\end{proof}


Lemma \ref{lem:su-harmonicity} and Proposition
\ref{prop:consequence-of-positivity} are applied to the case with
$\lcS$ in place of $X$ and $\phi_{(p)}$ in place of $\phi_D$ in the
following sections.



%%% Local Variables:
%%% mode: latex
%%% TeX-master: "Injectivity-Fujino"
%%% coding: utf-8
%%% End:




\subsection{Adjoint ideal sheaves and the residue computations}
% \subsection{Residue functions and residue short exact sequences}
\label{subsec:residue}

%%%%%
%%%%% File name  : residue-fcts-n-residue-exact-seq.tex
%%%%% Author     : Mario Chan
%%%%% Date       : 10th March, 2023
%%%%% Description: This is the section of the project
%%%%%              "Injectivity-Fujino" on residue functions and 
%%%%%              residue exact sequences. 
%%%%%
%%
%%%

{
  \setDefaultvphi{\vphi_L}

  Let $L$ be a line bundle on $X$ equipped with a \mmark{smooth metric
    $e^{-\vphi_L}$}{$\vphi_L$ has to be smooth, or the claim on the
    jumping number must be mentioned explicitly. The result  $\aidlof*
    =\mtidlof{\vphi_L} \cdot \defidlof{\lcc+1'}$ may not hold
    otherwise. \\ }. 
  The \mmark[BlueGreen]{residue function $\eps \mapsto \RTF|f|,<V>$}{It's
    possible not using ``$\RTF|f|$'' at all in this paper.} of index $\sigma$ 
  is defined, for each \mhlight[BlueViolet]{$f \in  \logKX[L] \otimes
    \smooth_X (V)$}, to be 
  \begin{equation*}
    \RTF|f|,<V> :=\RTF|f|,<V>,
    := \eps \int_V \frac{\abs f^2 \:e^{-\phi_D-\vphi_L}}{\logpole} \quad
    \text{ for } \eps > 0  \; . 
  \end{equation*}\mariocomment[BlueViolet]{For consistency of notation in this
    section only.}%
  The adjoint ideal sheaf $\aidlof :=\aidlof<X>$ of index $\sigma$
  is given at each $x \in X$  by
  \begin{equation*}
    \aidlof_x :=\setd{f \in \holo_{X,x}}{
      \exists~\text{open set } V_x \ni x \:, \: \forall~\eps > 0 \:, \:
      \RTF|f|,<V_x>, < +\infty
    } \; .
  \end{equation*}
  Note that the adjoint ideal sheaf is independent of $\vphi_L$ (as
  $\vphi_L$ is smooth).
  By \cite{Chan_adjoint-ideal-nas}*{Thm.~1.2.3}, the adjoint ideal
  sheaf can be written as 
  \begin{equation*}
    \aidlof = \mtidlof{\vphi_L} \cdot \defidlof{\lcc+1'}
    =\defidlof{\lcc+1'}
    \quad\text{ for any } \sigma \geq 0 \; ,
  \end{equation*}
  where $\defidlof{\lcc+1'}$ is the defining ideal sheaf of $\lcc+1'$
  in $X$ (with the reduced structure), \mmark{and we have the residue short exact
    sequence}{I don't want to suggest that the product structure of
    $\aidlof*$ implies directly the residue exact sequence.}
  \begin{equation*}
    \xymatrix@R-0.5cm@C+0.3cm{
      {0} \ar[r]
      & {\aidlof-1} \ar[r]
      & {\aidlof} \ar[r]^-{\Res^\sigma}
      & {\residlof} \ar[r]
      & {0 \; .}
    }
  \end{equation*}
  Here the quotient sheaf ${\residlof}$, called the \emph{residue sheaf of index $\sigma$}, can be written as 
  \begin{equation*}
    \residlof
    = \bigoplus_{p \in \Iset} \paren{\Diff_p D}^{-1}
    \otimes \mtidlof<\lcS>{\vphi_L}
    = \bigoplus_{p \in \Iset} \paren{\Diff_p D}^{-1}
  \end{equation*}
  Note $\logKX[L] \otimes \residlof =\bigoplus_{p \in\Iset} K_{\lcS} \otimes \res L_{\lcS}.$
  Next we describe the \emph{residue morphism $\Res^\sigma$} in terms of 
  the Poincar\'e residue map $\PRes[\lcS]$ given in
  \cite{Kollar_Sing-of-MMP}*{\S 4.18} as follows. 
  The Poincar\'e residue map $\PRes[\lcS]$ from $X$ to each $\lcS$ is
  uniquely determined after an orientation on the conormal bundle of
  $\lcS$ in $X$ is fixed.
  For an admissible open set $V \subset X$, 
  we have $\lcc' \cap V = \bigcup_{\alert{p \in \Iset}}
  \lcS<V>$ \mmark{(where $\lcS<V> := \lcS \cap V$, which is connected by the
  definition of the admissible open set, and possibly empty)}{This is
  a subtle fact that is used in the residue computation. We can keep
  using the same index set because $V$ is admissible.} and $\lcS<V>
=\set{z_{p(1)} =z_{p(2)} =\dotsm =z_{p(\sigma)}=0}$ when non-empty. 
  Under such coordinate system, a section $f $ of  $\logKX[L] \otimes
  \aidlof$ on $V \subset X$ can be written as
  \begin{equation*}
    f = \;\;\smashoperator{\sum_{p \in \Iset \colon \lcS<V>
        \neq\emptyset}} \;\; dz_{p(1)} \wedge \dotsm \wedge dz_{p(\sigma)}
    \wedge g_p \:\sect_{(p)} 
    =\;\;\smashoperator[l]{\sum_{p \in \Iset \colon \lcS<V>
        \neq\emptyset}}
    \frac{dz_{p(1)}}{z_{p(1)}} \wedge \dotsm
    \wedge \frac{dz_{p(\sigma)}}{z_{p(\sigma)}}
    \wedge g_p \:\sect_D \quad\text{ on } V. 
  \end{equation*}
  % Then, the Poincar\'e residue map $\PRes[\lcS]$ is given by
  \mmark{We therefore see that 
  \begin{equation*}
    \PRes[\lcS](\frac{f}{\sect_D})  =\res{g_p}_{\lcS} \in
    K_{\lcS} \otimes \res L_{\lcS} \quad\text{ on } \lcS<V> 
  \end{equation*}}{I don't want to give the impression that we define
  the Poincar\'e residue map by this formula.}%
  under the assumption that the orientation on the conormal bundle of
  $\lcS$ in $X$ on $V$ is given by $(dz_{p(1)}, dz_{p(2)}, \dots,
  dz_{p(\sigma)})$. 
  % Result in \cite{Chan_adjoint-ideal-nas}*{Thm.~4.1.2 (2)} (or the
  % computation in \cite{Chan_on-L2-ext-with-lc-measures}*{Prop.~2.2.1}
  % or \cite{Chan&Choi_ext-with-lcv-codim-1}*{Prop.~2.2.1}) yields
  % % (assuming that $f$ lives on a neighbourhood $V'$ of $\cl V$)
  % \begin{equation*}
  %   \RTF[\rho]|f|(0),<V> = \lim_{\eps \tendsto 0^+}
  %   %   \lim_{\rho \descendsto \charfct_{\cl V}}
  %   \RTF[\rho]|f|,<V>
  %   =\sum_{p \in \Iset} \frac{\pi^\sigma}{(\sigma -1)!} \int_{\lcS<V>}
  %   \rho \abs{g_p}^2 \:e^{-\vphi_L} 
  % \end{equation*}
  % for any compactly supported smooth function $\rho \colon V \to
  On the other hand, the residue morphism $\Res^\sigma$ is given in
  \cite{Chan_adjoint-ideal-nas}*{\S 4.2} by 
  \begin{equation*}
    \renewcommand{\objectstyle}{\displaystyle}
    \xymatrix@C+0.5cm@R-0.5cm{
      {\logKX[L] \otimes \aidlof} \ar[r]^-{\Res^\sigma}
      \ar@{}[d]|*[left]+{\in} 
      & {\hphantom{\logKX[L] \otimes \residlof}}
      \save +<4em,-1.3ex>*{\logKX[L] \otimes \residlof
        =\bigoplus_{p \in\Iset} K_{\lcS} \otimes \res L_{\lcS}} \restore
      \ar@{}[d]|*[left]+{\in}
      % & *+<-2cm,-1cm>{}
      % \ar@{}[l]|(.41)*+{}
      \\
      *+<0.8cm,0cm>{f} \ar@{|->}[r]
      & {\paren{\res{g_p}_{\lcS}}_{\mathrlap{p\in\Iset}}
        \mathrlap{\hphantom{p\in\Iset} .}} 
    }
  \end{equation*}
  Assuming $f$ being defined on a neighbourhood $V'$ of the closure
  $\cl V$ of $V$ and letting $\rho \colon V' \to [0,1]$ be a compactly
  supported smooth function
  % (i.e.~a smooth cut-off function) being
  identically equal to $1$ on $V$, one obtains, 
  from the result in \cite{Chan_adjoint-ideal-nas}*{Thm.~4.1.2 (2)} (or the
  computation in \cite{Chan_on-L2-ext-with-lc-measures}*{Prop.~2.2.1}
  or \cite{Chan&Choi_ext-with-lcv-codim-1}*{Prop.~2.2.1}),
  a (squared) norm of $g :=\paren{\res{g_p}_{\lcS<V>}}_{p \in \Iset} \in
  \logKX[L] \otimes \residlof$ on $V$ given by
  \begin{equation} \label{eq:residue-norm}
    \norm{g}_{\lcc<V>'}^2 :=\RTF|f|(0),<V>
    =\lim_{\rho \descendsto \charfct_{\cl V}} \lim_{\eps \tendsto 0^+}
    \RTF[\rho]|f|,<V'>
    =\sum_{p \in \Iset} \frac{\pi^\sigma}{(\sigma -1)!}
    \int_{\mathrlap{\lcS<V>}} \;\;\;
    \abs{g_p}^2 \:e^{-\vphi_L}
    =:\sum_{p\in\Iset} \norm{g_p}_{\lcS<V>}^2 \; ,
  \end{equation}
  where the limit $\lim_{\rho \descendsto \charfct_{\cl V}}$ refers to
  the pointwise limit as $\rho$ descends to the characteristic
  function $\charfct_{\cl V}$ of $\cl V$ on $X$.
  Such a norm is referred to as the \emph*{residue norm on $\logKX[L]
    \otimes \residlof$ on $V$}.
  %%%%% \emph* is needed as the package embrac is used and \logKX[L]
  %%%%% appears inside \emph.
  Moreover, we also see from the residue exact sequence that
  \begin{equation*}
    \aidlof-1_x
    =\setd{f \in \aidlof_x}{ \exists~\text{open set } V_x
      \ni x \:, \: \RTF|f|(0),<V_x> = 0}
    % \\
    % &=\setd{f \in \aidlof_x}{ \exists~\text{open set } V_x
    %   \ni x \:, \: \RTF|f|,<V_x> = \BigO(\eps) \text{ as } \eps
    %   \tendsto 0^+}
  \end{equation*}
  for every $x \in X$.

  Under the assumption that $\vphi_L$ has only neat analytic
  singularities (which is indeed smooth in the current setting), the
  residue norm on an admissible open set $V \subset X$ can also be
  obtained from 
  \begin{equation*}
    \lim_{\eps \tendsto 0^+} \eps \int_{V} \frac{
      \rho \abs f^2 \:e^{-\phi_D-\vphi_L}
    }{\abs{\psi_D}^{\sigma +\eps}}
    =\RTF[\rho]|f|(0),<V>
  \end{equation*}
  for any smooth compactly supported cut-off function $\rho$ on $V$ (see
  \cite{Chan&Choi_ext-with-lcv-codim-1}*{Prop.~2.2.1} or
  \cite{Chan&Choi_injectivity-I}*{Thm.~2.6.1}).
  Moreover, the above equation works not only for $f$ with holomorphic
  coefficients, but also for $f$ with coefficients in
  $\smooth_{X\,*}$, where
  \begin{align*}
    \smooth_{X\, *}
    &:=\paren{\smooth_{X}\left[
      \frac{1}{\abs{\sect_i}} \colon i \in \Iset||
      \right]}_{\text{b}}
      \qquad\paren{\sect_i \text{ treated as a local defining function of }
      D_i} \\
    &:=\set{\text{locally bounded elements in the $\smooth_X$-algebra generated
      by } \frac{1}{\abs{\sect_i}} \text{ for all } i\in\Iset||} \;
      .\footnotemark
  \end{align*}%
  \footnotetext{
    On an admissible open set $V$ under the holomorphic coordinate
    system $(z_1,\dots, z_n)$ such that $D\cap V =\set{z_1 z_2 \dotsm
      z_{\sigma_V} =0}$, one has
    \begin{equation*}
      \smooth_{X \,*}(V)
      =\smooth_X(V)\left[e^{\pm \cplxi \theta_1}, \dots, e^{\pm \cplxi
          \theta_{\sigma_V}} \right]
    \end{equation*}
    where $(r_j,\theta_j)$ is the polar coordinate system of the
    $z_j$-plane for $j=1,\dots,\sigma_V$ in $V$, which is (almost) the
    same as the ad hoc definition of $\smooth_{X\, *}(V)$ given in 
    \cite{Chan&Choi_injectivity-I}*{\S 2.6} (in which
    $e^{\pm\cplxi\theta_{k}}$ for $k \geq \sigma_V +1$ are also included
    in the set of generators of the algebra).
    The definition given here is independent of coordinates and its
    sheaf structure can be seen easily.
  }%
  The coefficients of $\Res^\sigma$ (and hence $\PRes[\lcS]$ for any
  $p\in\Iset$) can be extended from $\holo_X$ to $\smooth_{X\,*}$
  accordingly.
  The residue norm is finite when the coefficients of $f$ belong to
  $\smooth_{X\,*} \cdot \aidlof$ on $V$.
  When the induced inner product is considered, one still has
  finiteness even if one of the argument does not have coefficients in
  $\smooth_{X\,*} \cdot \aidlof$, which is the content of the
  following proposition.
  \begin{prop} \label{prop:residue-product-X-to-lcS}
    Given any admissible open set $V \subset X$ and any section $f \in
    \logKX[L] \otimes \smooth_{X \:c\,*} \cdot\aidlof\paren{V}$
    (compactly supported in $V$) such
    that $\Res^\sigma(f) = g =\paren{g_p}_{p\in\Iset}$, one
    has, for any $\xi \in \logKX[L] \otimes \smooth_{X \, *}\paren{V}$,
    \begin{align*}
      \lim_{\eps \tendsto 0^+} \eps \int_V
      \frac{\inner{\xi}{f} \:e^{-\phi_D-\vphi_L}}{\abs{\psi_D}^{\sigma
      +\eps}}
      &=\sum_{p \in \Iset} \frac{\pi^\sigma}{(\sigma-1)!}
        \int_{\lcS<V>} \inner{\frac{\rs*\xi_p}{\sect_{(p)}}}{\: g_p}
        \:e^{-\vphi_L} \\
      &=\sum_{p \in \Iset}
      % \smash[b]{
        \underbrace{
        \frac{\pi^\sigma}{(\sigma-1)!}
        \int_{\lcS<V>} \inner{\rs*\xi_p}{\: g_p \sect_{(p)}}
        \:e^{-\phi_{(p)}-\vphi_L}
        }_{\displaystyle =:
        \iinner{\rs*\xi_p}{g_p\sect_{(p)}}_{\mathrlap{\lcS<V>,
        \phi_{(p)}}}}
  % }
  %   \vphantom{\underbrace{\int_{\lcS<V>}}_{\iinner{\rs*\xi_p}{g_p\sect_{(p)}}}}
    \end{align*}
    which is finite,
    where $\phi_{(p)} :=\log\abs{\sect_{(p)}}^2$ and
    \begin{equation*}
      \rs*\xi_p := \PRes[\lcS](\frac{\xi}{\sect_D}) \cdot \sect_{(p)}
      \in K_{\lcS} \otimes \Diff_p D \otimes \res L_{\lcS} \otimes
      \smooth_{X\:c\, *}\paren{\lcS<V>} \; .
    \end{equation*}
    Moreover, if either $f$ or $\xi$ belongs to $\logKX[L] \otimes
    \smooth_{X\,*} \cdot\aidlof-1\paren{V}$, then $\eps \int_V
      \frac{\inner{\xi}{f} \:e^{-\phi_D-\vphi_L}}{\abs{\psi_D}^{\sigma
      +\eps}} = \BigO(\eps)$ (the big-O notation) as $\eps \tendsto 0$.
  \end{prop}

  \begin{proof}
    By linearity in $f$ in the equation in the claim, it suffices to
    consider the case where $\lcS \cap V = \set{z_1 =z_2 = \dotsm
      z_\sigma = 0}$, $\res{\sect_{(p)}}_{V} = z_{\sigma+1} \dotsm z_{\sigma_V}$ and
    \begin{equation*}
      f = dz_1 \wedge dz_2 \wedge \dotsm \wedge dz_\sigma \wedge g_p
      \sect_{(p)} 
    \end{equation*}
    (in which $g_p$ is abused to mean a $(n-\sigma,0)$-form on $V$).
    Write also
    \begin{equation*}
      \xi =: dz_1 \wedge dz_2 \wedge \dotsm \wedge dz_\sigma \wedge
      \xi_p
      \quad\text{ such that }\;\;
      \res{\xi_p}_{\lcS<V>} = \PRes[\lcS](\frac{\xi}{\sect_D})
      \cdot \sect_{(p)} =\rs*\xi_p \; .
    \end{equation*}
    Let $(r_j, \theta_j)$ be the polar coordinates of the $z_j$-plane
    and set 
    \begin{equation*}
      F_0 :=\inner{\xi_p}{g_p} \:e^{-\vphi_L}
      \quad\text{ and }\quad
      F_j :=\fdiff{r_j} \paren{\frac{F_j}{r_j^2 \fdiff{r_j^2} \psi_D}}
      \quad\text{ for } j=1,\dots, \sigma \; .
    \end{equation*}
    Notice that $\fdiff{r_j} \sect_{(p)} = 0$ and coefficients of
    $F_j$ are in $\smooth_{X\:c\,*}$ on $V$ for $j=1,\dots,\sigma$.
    It then follows from the similar computation in
    \cite{Chan&Choi_ext-with-lcv-codim-1}*{Prop.~2.2.1} or
    \cite{Chan&Choi_injectivity-I}*{Thm.~2.6.1} that
    \begin{align*}
      \eps \int_V
      \frac{\inner{\xi}{f} \:e^{-\phi_D-\vphi_L}}{\abs{\psi_D}^{\sigma
      +\eps}}
      &=\eps \int_V
        \frac{\inner{\xi_p}{g_p} \:e^{-\vphi_L}}{\sect_{(p)}\:\abs{\psi_D}^{\sigma
        +\eps}} \wedge \bigwedge_{j=1}^\sigma \frac{\pi\ibar\:dz_j
        \wedge d\conj{z_j}}{\abs{z_j}^2}
      \\
      &=\eps \int_V \frac{F_0}{\sect_{(p)}\:\abs{\psi_D}^{\sigma +\eps}}
        \prod_{j=1}^\sigma d\log r_j^2 \cdot
        \underbrace{\prod_{j=1}^\sigma \frac{d\theta_j}2}_{=:\:
        \vect{d\theta}}
      \\
      &=\frac\eps{\sigma-1+\eps}
        \int_{V} \frac{F_0}{\sect_{(p)}\:r_1^2 \fdiff{r_1^2}\psi_D}
        \:d\paren{\frac{1}{\abs{\psi_D}^{\sigma-1+\eps}}}
        \prod_{j=2}^{\sigma} d\log r_j^2 
        \cdot \vect{d\theta} \\
      &\overset{\mathclap{\text{int.~by parts}}}=
        \quad\;\;
        \frac{-\eps}{\sigma-1+\eps}
        \int_{V}
        \frac{\alert{F_1}}{\sect_{(p)}\:\abs{\psi_D}^{\sigma-1+\eps}}
        \prod_{j=2}^{\sigma} d\log r_j^2 
        \cdot dr_1 \:\vect{d\theta} \\
      &= \dotsm =
        \frac{(-1)^{\sigma} \eps} {\prod_{j=1}^{\sigma} \paren{\sigma-j+\eps}} 
        \int_{V}
        \frac{F_{\alert{\sigma}}}{\sect_{(p)}\:\abs{\psi_D}^{\eps}}
        \prod_{j=1}^{\sigma} dr_j
        \cdot \vect{d\theta} \; .
    \end{align*}
    Note that $\frac{1}{\sect_{(p)}}$ is integrable on $V$, so the
    integral on the far right-hand-side converges for all $\eps \geq
    0$. 
    Letting $\eps \tendsto 0^+$ on both sides, the desired formula
    then follows from the fundamental theorem of calculus.

    When $f$ or $\xi$ belongs to $\logKX[L] \otimes \smooth_{X\,*}
    \cdot\aidlof-1\paren{V}$, the residue formula in the proposition
    holds even when $\sigma$ is replaced by $\sigma-1$, which implies
    that the integral $\int_V \frac{\inner{\xi}{f}
      \:e^{-\phi_D-\vphi_L}}{\abs{\psi_D}^{\sigma +\eps}}$ converges
    for all $\abs\eps < 1$, hence the last claim.
  \end{proof}

  When restriction to a subspace of codimension $1$ is considered,
  there is a more classical kernel for obtaining the residue formula.
  As an illustration, the residue formula from $X$ to $\lcc|1|'$ is
  proved in the following proposition (which is applied to the case
  where the residue from $\lcc'$ to $\lcc+1'$ is considered later). 
  Recall that $\lcc|1|' =D =\sum_{i \in \Iset||} D_i$, where $D_i =
  \lcS|1|[i]$ and $\Iset|| =\Iset|1|$ are set for convenience.
  \begin{prop} \label{prop:residue-formula-classical-kernel}
    Given any admissible open set $V \subset X$ and any compactly
    supported section $f \in
    \logKX[L] \otimes \smooth_{X \:c\,*} \cdot\aidlof|1|\paren{V}$
    such that $\Res^1(f) = g =\paren{g_i}_{i\in\Iset||}$, one
    has, for any $\xi \in \logKX[L] \otimes \smooth_{X \, *}\paren{V}$,
    \begin{align*}
      \lim_{\eps \tendsto 0^+} \eps \int_V
      \inner{\xi}{f} \:e^{-\phi_D-\vphi_L} e^{-\eps\abs{\psi_D}}
      &=\sum_{i \in \Iset||} \pi
        \int_{D_i \cap V} \inner{\frac{\rs*\xi_i}{\sect_{(i)}}}{\: g_i}
        \:e^{-\vphi_L} \\
      &=\sum_{i \in \Iset||}
      % \underbrace{
        \pi
        \int_{D_i \cap V} \inner{\rs*\xi_i}{\: g_i \sect_{(i)}}
        \:e^{-\phi_{(i)}-\vphi_L}
        % }_{\displaystyle =:
        %   \iinner{\rs*\xi_i}{g_i\sect_{(i)}}_{\mathrlap{D_i \cap V,
        %   \phi_{(i)}}}}
    \end{align*}
    which is finite,
    where $\phi_{(i)} :=\log\abs{\sect_{(i)}}^2$ and
    \begin{equation*}
      \rs*\xi_i := \PRes[D_i](\frac{\xi}{\sect_D}) \cdot \sect_{(i)}
      \in K_{D_i} \otimes \Diff_i D \otimes \res L_{D_i} \otimes
      \smooth_{X\:c\, *}\paren{D_i \cap V} \; .
    \end{equation*}    
  \end{prop}

  \begin{proof}
    As before, it suffices to consider the case where $D_i \cap V
    =\set{z_1 =0}$, $\res{\sect_{(i)}}_V =z_2 \dotsm z_{\sigma_V}$ and
    \begin{equation*}
      f = dz_1 \wedge g_i \sect_{(i)} \; .
    \end{equation*}
    Write also
    \begin{equation*}
      \xi =: dz_1 \wedge \xi_i
      \quad\text{ such that }\;\;
      \res{\xi_i}_{D_i \cap V} =\PRes[D_i](\frac{\xi}{\sect_D}) \cdot
      \sect_{(i)} =\rs*\xi_i \; .
    \end{equation*}
    Essentially the same computation as in Proposition
    \ref{prop:residue-product-X-to-lcS} yields 
    \begin{align*}
      \eps \int_V \inner{\xi}{f} \:e^{-\phi_D-\vphi_L}
      e^{-\eps\abs{\psi_D}}
      =&~\eps \int_V \frac{\inner{\xi_i}{g_i}
         \:e^{-\vphi_L}}{\sect_{(i)}} \wedge
         e^{-\eps\abs{\psi_D}}
         \frac{
         \pi\ibar\:dz_1 \wedge d\conj{z_1}
         }{\abs{z_1}^2}
      \\
      =&~\int_V \frac{\inner{\xi_i}{g_i}
         \:e^{-\vphi_L}}{\sect_{(i)} \:r_1^2 \fdiff{r_1^2} \psi_D} 
         \:d\paren{e^{-\eps\abs{\psi_D}}} \:
         \frac{d\theta_1}{2}
      \\
      \overset{\mathclap{\text{int.~by parts}}}=
       &~\quad\;\;
         -\int_V \fdiff{r_1} \paren{\frac{\inner{\xi_i}{g_i}
         \:e^{-\vphi_L}}{r_1^2 \fdiff{r_1^2} \psi_D} }
         \:\frac{e^{-\eps\abs{\psi_D}}}{\sect_{(i)}} \:dr_1 \:
         \frac{d\theta_1}{2}
      \\
      \mathclap{\xrightarrow{\eps \tendsto 0^+}\;\;}
       &~\quad\;
         -\int_V \fdiff{r_1} \paren{\frac{\inner{\xi_i}{g_i}
         \:e^{-\vphi_L}}{r_1^2 \fdiff{r_1^2} \psi_D} }
         \:\frac{1}{\sect_{(i)}} \:dr_1 \:
         \frac{d\theta_1}{2}
      \\
      =&~\pi \int_{\mathrlap{D_i \cap V}} \;\;\; \frac{\inner{\rs*\xi_i}{g_i}
         \:e^{-\vphi_L}}{\sect_{(i)}}
         =\pi \int_{D_i \cap V} \inner{\rs*\xi_i}{g_i \sect_{(i)}}
         \:e^{-\phi_{(i)}-\vphi_L} \; .
    \end{align*}
    Note that the convergence of the integral obtained right after
    integration by parts follows from the same reasoning as in
    Proposition \ref{prop:residue-product-X-to-lcS}.
  \end{proof}

  Proposition \ref{prop:residue-formula-classical-kernel} facilitates the
  following residue computation.

  \begin{prop} \label{prop:res-formula-dbar-exact-dot-harmonic}
    Given the decomposition $\lcc' = \bigcup_{p\in\Iset} \lcS$, let
    $u_p$ be a \emph{harmonic} $K_{\lcS} \otimes \res L_{\lcS}$-valued
    $(0,q)$-form on $\lcS$ with respect to the norm
    $\norm\cdot_{\lcS}$ for each $p \in \Iset$.
    Given also the decomposition $\lcc+1' = \bigcup_{b\in\Iset+1}
    \lcS+1[b]$, for any $\lcS$ and $\lcS+1[b]$ such that $\lcS+1[b]
    \subset \lcS$, let $\sgn{b:p}$ be the sign such that
    \begin{equation*}
      \PRes[\lcS+1[b]] =\sgn{b:p} \:\PRes[\lcS+1[b] | \lcS] \circ
      \PRes[\lcS] \; ,
    \end{equation*}
    where $\PRes[\lcS+1[b] | \lcS]$ denotes the Poincar\'e residue map
    from $\lcS$ to $\lcS+1[b]$.
    % For a given locally finite cover $\cvr V :=\set{V_i}_{i \in I}$ of
    % $X$ by \emph{admissible} open sets with respect to
    % $(\vphi_L,\psi_D)$ together with a partition of unity
    % $\set{\rho^i}_{i \in I}$ subordinate to it,
    With the finite cover $\cvr V$ and partition of unity
    $\set{\rho^i}_{i \in I}$ given in Section \ref{subsec:notation},
    let $\set{\gamma_{\idx 1.q}}_{\idx 1,q \in I}$ be a
    $\logKX[L]$-valued \v Cech $(q-1)$-cochain with respect to $\cvr
    V$ and set 
    \begin{gather*}
      \rs\gamma_{p; \:\idx 1.q} :=\PRes[\lcS](\frac{\gamma_{\idx 1.q}}{\sect_D})
      \cdot \sect_{(p)} \; , \quad
      v_{p} := \sum_{\idx 1,q \in I} \underbrace{
        \dbar\rho^{i_q} \wedge \dotsm
        \wedge \dbar\rho^{i_2} \cdot \rho^{i_1}
      }_{=: \: \paren{\dbar\rho}^{\idx q.1}} \rs*\gamma_{p;\:\idx 1.q}
      \quad\text{ on } \lcS \\
      \text{and }\quad
      \rs\gamma_{b; \:\idx 1.q} :=\PRes[\lcS+1[b]](\frac{\gamma_{\idx 1.q}}{\sect_D})
      \cdot \sect_{(b)} \; , \quad
      v_{b} := \sum_{\idx 1,q \in I} \paren{\dbar\rho}^{\idx q.1}
      \rs*\gamma_{b;\:\idx 1.q}
      \quad\text{ on } \lcS+1[b] \; .
    \end{gather*}
    Then, after setting $\iinner{\cdot}{\cdot}_{\lcS, \phi_{(p)}}
    :=\iinner{\cdot}{\cdot \:e^{-\phi_{(p)}}}_{\lcS}$ (and
    similarly for $\iinner{\cdot}{\cdot}_{\lcS+1[b], \phi_{(b)}}$), one has
    \begin{equation*}
      \sum_{p\in\Iset} \iinner{\dbar v_{p}}{
        u_p\sect_{(p)}}_{\lcS,\phi_{(p)}}
      =-\sigma \smashoperator[l]{\sum_{b\in\Iset+1}} \iinner{v_{b} \:}{\quad\;
        \smashoperator{\sum_{p\in\Iset \colon \lcS+1[b] \subset
            \lcS}} \;\;
        \sgn{b:p} \: \PRes[\lcS+1[b] | \lcS](\idxup{\diff\psi_{(p)}}.
         u_p) \cdot \sect_{(b)}
      }_{\lcS+1[b], \phi_{(b)}} \; ,
    \end{equation*}
    where $\psi_{(p)} :=\phi_{(p)} -\sm\vphi_{(p)}$ and
    $\sm\vphi_{(p)}$ is some smooth potential on $\Diff_p D$.
  \end{prop}

  \begin{proof}
    Notice that $v_{p}$ is smooth on $\lcS$ but not necessarily
    locally $L^2$ with respect to the weight $e^{-\phi_{(p)}}$.
    An integration by parts is done via the use of Proposition
    \ref{prop:residue-formula-classical-kernel}, which yields 
    \begin{align*}
      &~\sum_{p\in \Iset} \iinner{\dbar v_{p}}{ u_p
        \sect_{(p)}}_{\lcS, \phi_{(p)}}
      \\
      \xleftarrow{\eps \tendsto 0^+}
      &~\sum_{p \in \Iset} \iinner{
        e^{-\eps \abs{\psi_{(p)}}} \:\dbar v_{p}
        }{ u_p \sect_{(p)}}_{\lcS, \phi_{(p)}}
      \\
      =&~\sum_{p \in \Iset} \paren{
         \cancelto{0 \;\;\;(\because~u_p \text{ harmonic, Lemma \ref{lem:su-harmonicity}})}{\iinner{
         \dbar\paren{e^{-\eps \abs{\psi_{(p)}}} \: v_{p}}
         }{ u_p \sect_{(p)}}_{\mathrlap{\lcS, \phi_{(p)}}}}
         \quad\;\; - \eps 
         \iinner{
         e^{-\eps \abs{\psi_{(p)}}} \:v_{p}
         }{\:\idxup{\diff\psi_{(p)}}.  u_p \sect_{(p)}}_{\lcS,
         \phi_{(p)}}
         }
      \\
      =&~-\sum_{p \in \Iset} \sum_{\idx 1,q \in I} \eps \:
         \iinner{
         e^{-\eps \abs{\psi_{(p)}}} \: % \paren{\dbar\rho}^{\idx q.1}
         \rs*\gamma_{p;\:\idx 1.q}
         }{\:
         \idxup{\diff\rho},[\idx 1.q] .
         \paren{\idxup{\diff\psi_{(p)}}.  u_p \sect_{(p)}}
         }_{\lcS, \phi_{(p)}}
      \\
      \xrightarrow[\text{Prop.~\ref{prop:residue-formula-classical-kernel}}]{\eps
      \tendsto 0^+} 
      &~-\smashoperator[l]{\sum_{\idx 1,q \in I}} \sum_{p \in \Iset}
        \sum_{k=\sigma +1}^{\mathclap{\sigma_{V_{\idx 1.q}}}} \sigma
        \iinner{
        \PRes[p(k)](
        \frac{\rs*\gamma_{p;\:\idx 1.q}}{\sect_{(p)}}
        )
        }{\:
        \idxup{\diff\rho},[\idx 1.q] .
        \PRes[p(k)](\idxup{\diff\psi_{(p)}}.  u_p)
        }_{\lcS \cap \set{z_{p(k)} =0}}
        \; ,
    \end{align*}
    where $\idxup{\diff\rho},[\idx 1.q] . \cdot$ is the adjoint
    of $\paren{\dbar\rho}^{\idx q.1} \cdot$, and $\PRes[p(k)]$ denotes
    the Poincar\'e residue map from $\lcS$ to $\lcS \cap \set{z_{p(k)}=0}$. 
    The last limit is justified as follows.
    On the admissible open set $V_{\idx 1.q}$, consider a holomorphic
    coordinate system $(z_1, \dots, z_n)$ such that $\lcS \cap V_{\idx
    1.q}
    =\set{z_{p(1)} = \dotsm =z_{p(\sigma)} =0}$ and
    $\sect_{(p)} =z_{p(\sigma+1)} \dotsm z_{p(\sigma_V)}$ (write
    $\sigma_{V}$ for $\sigma_{V_{\idx 1.q}}$ for convenience).
    Note that
    \begin{equation*}
      \diff\psi_{(p)} =\sum_{k =\sigma +1}^{\sigma_V}
      \frac{dz_{p(k)}}{z_{p(k)}} -\diff\sm\vphi_{(p)} \quad\text{ on }
      V_{\idx 1.q} \; .
    \end{equation*}
    It follows that, on $\lcS \cap V_{\idx 1.q}$,
    \begin{equation*}
      \text{coef.~of }\:
      \idxup{\diff\rho},[\idx 1.q].
      \paren{\idxup{\diff\psi_{(p)}}.  u_p \sect_{(p)}}
      \in
      \smooth_{\lcS \:c} \cdot\res{\defidlof{\lcc+2'}}_{\lcS}
      \begin{aligned}[t]
        &=\smooth_{\lcS \:c} \cdot\mtidlof<\lcS>{\vphi_L} \cdot
        \res{\defidlof{\lcc+2'}}_{\lcS} \;\;\footnotemark
        \\
        &=\smooth_{\lcS \:c} \cdot\aidlof|1|<\lcS>{\vphi_L}[\psi_{(p)}]
      \end{aligned}
    \end{equation*}%
    \footnotetext{
      Recall that $\defidlof{\lcc+2'}$ is generated on $X$ by
      $\sect_{(b)}$ treated as local
      functions for all $b \in \Iset+1$.
      On an admissible open set $V$, one has $\defidlof{\lcc+2'}
      =\genbyd{z_{b(\sigma+2)} \dotsm
        z_{b(\sigma_V)}}{b \in \Iset+1 \text{ such that } \lcS+1[b] \cap
        V \neq \emptyset}$.
      % (see page
      % \pageref{page:notation-permutation-index} for the notation).
    }%
    and, therefore, one can apply Proposition
    \ref{prop:residue-formula-classical-kernel} (with $\lcS$ in place
    of $X$, $\psi_{(p)}$ in place of $\psi_D$) to each inner product
    $\eps \iinner{e^{-\eps \abs{\psi_{(p)}}} \dotsm}{\: \dotsm
      \idxup{\diff\psi_{(p)}} . \dotsm \sect_{(p)}}_{\lcS,\phi_{(p)}}$.
    Note also that the factor $\sigma$ comes from the normalisation of
    the norm on each lc center ($\norm\cdot_{\lcS}^2
    =\frac{\pi^\sigma}{(\sigma -1)!} \int_{\lcS} \dotsm$ and
    $\norm\cdot_{\lcS+1[b]}^2 =\frac{\pi^{\sigma+1}}{\sigma!}
    \int_{\lcS+1[b]} \dotsm$).


    On each admissible open set $V_{\idx 1.q}$, the intersection $\lcS
    \cap \set{z_{p(k)} = 0}$ is a $(\sigma+1)$-lc center $\lcS+1[b_{p,k}]
    \cap V_{\idx 1.q}$ ($\neq \emptyset$), uniquely determined by the
    choices of $p\in \Iset$ (such that $\lcS \cap V_{\idx 1.q} \neq
    \emptyset$, so $\binom{\sigma_V}{\sigma}$ choices) and $k
    =\sigma+1, \dots, \sigma_V$ (so $\sigma_V-\sigma$ choices).
    To get an indexing in terms of $b \in \Iset+1$ (such that
    $\lcS+1[b] \cap V_{\idx 1.q} \neq \emptyset$, so
    $\binom{\sigma_V}{\sigma +1}$ choices), note that each $\lcS+1[b]
    \cap V_{\idx 1.q}$ is contained in $\sigma +1$ distinct
    $\sigma$-lc centers $\lcS[p_{b,j}]$ for $j=1,\dots,\sigma+1$
    (apparently, $\sigma +1$ choices) such that
    \begin{equation*}
      \lcS+1[b] \cap V_{\idx 1.q} = \lcS[p_{b,j}] \cap \set{z_{b(j)} = 0} \; .
    \end{equation*}
    (One can verify $\sum_{p \in \Iset} \sum_{k=\sigma
      +1}^{\sigma_{V}} \dotsm = \sum_{b \in
      \Iset+1} \sum_{j=1}^{\sigma +1} \dotsm$ by first noting that
    $\binom{\sigma_V}{\sigma} (\sigma_V -\sigma)
    =\binom{\sigma_V}{\sigma +1} (\sigma+1)$.)
    With such choice of indexing, one has
    \begin{equation*}
      \frac{\rs*\gamma_{b;\: \idx 1.q}}{\sect_{(b)}}
      :=\PRes[\lcS+1[b]](\frac{\gamma_{\idx 1.q}}{\sect_D})
      =\sgn{b:p_{b,j}} \:
      \PRes[b(j)](\frac{\rs*\gamma_{p_{b,j};\:\idx
          1.q}}{\sect_{(p_{b,j})}})
    \end{equation*}
    (noticing that % $\sect_{(b)} =\sect_{(\sigma+1 : b)}$,
    % $\sect_{(p_{b,j})} =\sect_{(\sigma : p_{b,j})}$ and
    $\sect_{(p_{b,j})} = z_{b(j)} \sect_{(b)}$).
    As a result, the expression in question becomes
    \begin{align*}
      &-\smashoperator[l]{\sum_{\idx 1,q \in I}} \sum_{b \in \Iset+1}
        \sum_{j=1}^{\sigma +1} \sigma
        \iinner{ \sgn{b:p_{b,j}}\:
        \frac{\rs*\gamma_{b;\:\idx 1.q}}{\sect_{(b)}}
        }{\: 
        \idxup{\diff\rho},[\idx 1.q] .
        \PRes[b(j)](\idxup{\diff\psi_{(p_{b,j})}} . u_{p_{b,j}})
        }_{\lcS+1[b]}
      \\
      =&-\smashoperator[l]{\sum_{\idx 1,q \in I}}
        \sum_{b \in \Iset+1}
        \sigma
        \iinner{
        \paren{\dbar\rho}^{\idx q.1} \rs*\gamma_{b;\:\idx 1.q}
        \:}{  \sum_{j=1}^{\sigma +1} \sgn{b:p_{b,j}}\:
        \PRes[b(j)](\idxup{\diff\psi_{(p_{b,j})}} . u_{p_{b,j}})
        \cdot \sect_{(b)}
        }_{\lcS+1[b], \phi_{(b)}}
      \\
      =&-\sigma \sum_{b \in \Iset+1} \iinner{
        v_b
        \:}{\quad\;
        \smashoperator{\sum_{p\in\Iset \colon \lcS+1[b] \subset
        \lcS}} \;\;
        \sgn{b:p}\:
        \PRes[\lcS+1[b] | \lcS](\idxup{\diff\psi_{(p)}}.  u_{p})
        \cdot \sect_{(b)}
        }_{\lcS+1[b], \phi_{(b)}} \; . \qedhere
    \end{align*}
  \end{proof}


}


\subsection{Restriction of harmonic differential forms to hypersurfaces}\label{subsec:harmonic}

{
  
  Let $(X,\omega)$ be a K\"ahler manifold equipped with a holomorphic
  line bundle $L$ equipped with a smooth potential $\varphi_L$ such
  that $\ibddbar\varphi_L\ge0$ and let $D$ be an snc divisor in $X$
  written as 
  \begin{equation*}
    D=\sum_{p\in I_D}D_p \; ,
  \end{equation*}
  where $D_p$ is an irreducible component for $p\in I_D$.
  We define the map $\HRes_p \colon \mathscr
  A_X^{0,q}(X,K_X\otimes L)\rightarrow \mathscr
  A_{D_p}^{0,q-1}(D_p,K_{D_p}\otimes L\vert_{D_p})$ by 
  \begin{equation*}
    \HRes_p(u)
    =
    \PRes[D_p](\idxup{\partial\psi_D} . u) \;\;\;\text{for}\;\;\;p\in
    I_D \; ,
  \end{equation*}
  where $\mathcal{R}_{D_p}$ is the Poincar\'e residue map (see, for
  example, \cite{Griffiths&Harris}*{p.147} or \cite{Kollar_Sing-of-MMP}*{\S 4.18}). 
  Notice that, as in \cite{Chan&Choi_injectivity-I}*{\textsection2.6},
  the map $\mathcal R_{D_p}$ is extended to send sections of
  $K_X\otimes\overline{\bold\Omega}_X^q$ to those of
  $K_{D_p}\otimes\overline{\bold\Omega}_X^q\vert_{D_p}$. 
  Let $(U;z^1,\dots,z^n)$ be a local holomorphic coordinate system
  around $x\in D_p\subset X$ satisfying 
  \begin{enumerate}[label=(\roman*), ref=\roman*]
  \item  \label{item:admissible-open-U} % [(\romannumeral1)]
    $D_p\cap U=\{z\in U:z^1=0\}$ and 
    $D\cap U=\set{z^1\cdots z^{\sigma_U}=0}$,
    and
  \item  \label{item:psi_D-in-admissible-open-U} % [(\romannumeral2)]
    $\psi_D=\sum_{j=1}^{\sigma_U}
    \log\abs{z^j}^2-\sm\varphi_D$ on $U$.
  \end{enumerate}
  Since $\del\psi_D=\sum_{j=1}^{\sigma_U}\frac{dz^j}{z^j}-\del\sm\varphi_D$, it follows that
  \begin{equation*}
    \HRes_p(u)
    =
    \mathcal{R}_{D_p}\left(\idxup{\partial\psi_D}.u\right)
    =
    \mathcal{R}_{D_p}\paren{\idxup{\frac{dz^1}{z^1}}.u}
    =
    \paren{\idxup{dz^1}.\widetilde u_p}\big\vert_{D_p} \; ,
  \end{equation*}
  where $\rs u_p := \fdiff{z^1} \ctrt u$ (so $u=dz^1\wedge\widetilde{u}_p$).
  In particular, $\HRes_p$ does not depend on the
  choice of the Hermitian metric $\sm\varphi_{D}$. 
  It follows from the above formula that
  $\HRes_p(u)$ is actually a $K_{D_p}\otimes
  L\vert_{D_p}$-valued $(0,q-1)$-form on $D_p$ (not only a
  $\overline{\bold\Omega}_X^{q-1}\vert_{D_p}$-valued section). 



  First we notice that $\dfadj$-closedness is preserved by
  $\HRes_p$ on a K\"ahler manifold.
  \begin{prop} \label{prop:harmonic-residue}
    If $u$ is a $\dfadj$-closed $K_X\otimes L$-valued $(0,q)$-form on
    $X$, then $\HRes_p(u)$ is a $\dfadj$-closed
    $K_{D_p}\otimes L\vert_{D_p}$-valued $(0,q-1)$-form on $D_p$. 
  \end{prop}

  \begin{proof}
    It is enough to show that it vanishes at the given point $x\in D_p$.
    Let $(z^1,\dots,z^n)$ be a local holomorphic coordinate system around
    $x$ in $X$ satisfying \eqref{item:admissible-open-U} % (\romannumeral1)
    and \eqref{item:psi_D-in-admissible-open-U}. % (\romannumeral2).
    Since $(D_p,\omega\vert_{D_p})$ is a smooth $(n-1)$-dimensional
    K\"ahler manifold, by a linear change of coordinates
    $(z^1,\ldots,z^n)$ and a quadratic change of coordinates in
    $(z^2,\dots,z^n)$ of $D_p$ we may assume that
    \begin{enumerate}[resume*]
    \item % [(\romannumeral3)]
      $g_{i\bar j}(x)=I_n$ where $I_n$ is the $n\times n$ identity matrix.
    \item % [(\romannumeral4)]
      $dg_{\alpha\bar\beta}(x)=0$ for $2\le\alpha\le n$ and $2\le\beta\le n$.
    \end{enumerate}
    Since $\displaystyle\idxup{dz^1}=g^{\bar
      j1}\pd{}{\overline{z^j}}$ (under Einstein summation convention),
    we have
    \begin{equation}\label{E:local_expression}
      dz^1 \wedge \paren{\idxup{dz^1} . \widetilde u_p}_{\bar j_1,\ldots,\bar j_{q-1}}
      =
      g^{\bar j1}
      u_{\bar j\bar j_1,\ldots,\bar j_{q-1}} \; .
    \end{equation}
    % where $\beta_1,\ldots,\beta_{q-1}$ run from $2$ to $n$.
    For the sake of convenience, let the Latin indices $i,j,k,...$ run
    from $1$ to $n$ and let the Greek indices
    $\alpha,\beta,\gamma,...$ run from $2$ to $n$ in this proof. 
    Let $\varphi_{K_X}$ and $\varphi_{K_{D_p}}$ be respectively the
    potentials on $K_X$ and $K_{D_p}$ induced by the K\"ahler metric
    $\omega$, which are written as 
    \begin{equation*}
      \varphi_{K_X}
      =
      \log\det\paren{g_{i\bar j}}_{1\le i,j\le n}
      \;\;\;\text{and}\;\;\;
      \varphi_{K_{D_p}}
      =
      \log\det\paren{g_{\alpha\bar\beta}}_{2\le\alpha,\beta\le n} \; .
    \end{equation*}
    This yields, at the given point $x\in D_p$,
    \begin{align*}
      \del_\gamma\varphi_{K_X}
      &=
	\del_\gamma\log\det g
	=
	\Tr\paren{\del_\gamma g\cdot g^{-1}}
      \\
      &=
	\sum_{i,j=1}^n\pd{g_{i\bar j}}{z^\gamma} g^{\bar j i}
	\;\;\overset{\mathclap{(\text{at } x)}}=\;\;
	g^{\bar11}\del_\gamma g_{1\bar1}
	+
	\sum_{\alpha,\beta=2}^n\pd{g_{\alpha\bar\beta}}{z^\gamma}g^{\alpha\bar\beta}
      \\
      &=
	g^{\conj11}\del_\gamma g_{1\conj1}
	+
	\del_\gamma\log\det\paren{g\vert_{D_p}}
	=
	g^{\conj11}\del_\gamma g_{1\conj1}
	+
	\del_\gamma\varphi_{K_{D_p}}
	\;\;\overset{\mathclap{(\text{at } x)}}=\;\;
	g^{\conj11}\del_\gamma g_{1\conj1} \; .
    \end{align*}
    % This implies that
    % \begin{equation*}
    %   \del_\ell\varphi_{K_X}
    %   =
    %   \del_\ell\log\det\paren{g_{i\bar j}}_{1\le i,j\le n}
    %   =
    %   \sum_{i,j=1}^n\pd{g_{i\bar j}}{z^\ell}g^{\bar ji}
    %   \;\;\;\text{and}
    %   \;\;\;
    %   \del_\ell\varphi_{K_{D_p}}
    %   =
    %   \sum_{\alpha,\beta=2}^n
    %   \pd{g_{\alpha\bar\beta}}{z^\ell}g^{\bar\beta\alpha}.	
    % \end{equation*}
    % Thus we have
    % \begin{equation*}
    %   \sum_{i,j}\pd{g_{i\bar j}}{z^\ell}g^{\bar ji}
    %   =
    %   \del_lg_{1\bar1}
    %   +
    %   \sum_{\alpha=2}^n\pd{g_{\alpha\bar\alpha}}{z^l}g^{\alpha\bar\alpha}
    %   =
    %   \del_lg_{1\bar1}
    %   +
    %   \del_l\varphi_{K_{D_p}}
    %   \;\;\;
    %   \text{and}
    %   \;\;\;
    %   \del_\beta\varphi_{K_{D_p}}=0\;\;\;\text{at}\;\;x.
    % \end{equation*}
    % It follows from \eqref{E:local_expression} and the definition of $\dfadj$ (cf.~\cite{Siu}) that
    % for any multi-indices $\ov{\boldsymbol\beta}_{q-2}=(\bar\beta_1,\ldots,\bar\beta_{q-2})$,
    \newcommand{\bbeta}{{\boldsymbol{\beta}}}%
    \newcommand{\KDp}{{\smash[b]{K_{D_p}}}}%
    % \renewcommand{\CancelColor}{\color{Gray}}%
    Choose a local frame of $L$ in a neighbourhood of $x$ in $X$
    such that
    \begin{equation*}
      \diff\vphi_L(x) = 0  \; .
    \end{equation*}
    It follows from \eqref{E:local_expression} and the definition of
    $\dfadj$ on $D_p$ (see, for example,
    \cite{Siu}*{(1.3.2)}) that, for any 
    multi-indices ${\bbeta}_{q-2} =(\idx[\beta]1,{q-2})$ and at the
    given point $x \in D_p$,
    \begin{align*}
      dz^1 \wedge \paren{\dfadj \HRes_p(u)}_{\conj\bbeta_{q-2}}
      &=
        -g^{\conj\beta\gamma}
	\nabla_\gamma 
	\paren{
        g^{\alert{\conj j} 1}
        u_{\alert{\conj j} \conj\beta\ov{\boldsymbol\beta}_{q-2}}
	}
      \\
      &=-g^{\conj\beta \gamma}\diff_\gamma \paren{g^{\alert{\conj j} 1}
        u_{\alert{\conj j} \conj\beta \conj\bbeta_{q-2}}}
        +g^{\conj\beta \gamma} \smash[t]{\cancelto{0}{\paren{
        \diff_\gamma \vphi_\KDp +\diff_\gamma \vphi_L
        }}}
        \cdot g^{\alert{\conj j} 1} u_{\alert{\conj j} \conj\beta \conj\bbeta_{q-2}}
      \\
      &=-g^{\conj\beta \gamma} g^{\alert{\conj j} 1} \diff_\gamma
        u_{\alert{\conj j} \conj\beta \conj\bbeta_{q-2}}
        -g^{\conj\beta \gamma} \diff_\gamma g^{\alert{\conj j} 1}
        \cdot u_{\alert{\conj j} \conj\beta \conj\bbeta_{q-2}}
      \\
      &=-g^{\conj\beta \gamma} g^{\conj 1 1}
        \diff_\gamma u_{\conj 1 \conj\beta \conj\bbeta_{q-2}}
        +g^{\conj\beta \gamma} g^{\alert{\conj j k}} \diff_\gamma
        g_{\alert k \conj 1} \cdot g^{\conj 1 1} u_{\alert{\conj j}
        \conj\beta \conj\bbeta_{q-2}}
      \\
      &=g^{\conj 1 1} \paren{
        -g^{\conj\beta \gamma} \diff_\gamma
        u_{\conj 1 \conj\beta \conj\bbeta_{q-2}}
        +g^{\conj\beta\gamma} g^{\conj 1 1} \diff_\gamma
        g_{1\conj 1} \cdot
        u_{\conj 1 \conj\beta \conj\bbeta_{q-2}}
        +g^{\conj\beta\gamma} g^{\alert{\conj j} \alpha}
        \diff_\gamma g_{\alpha \conj 1} \cdot
        u_{\alert{\conj j} \conj\beta \conj\bbeta_{q-2}}
        }
      \\
      &=g^{\conj 1 1} \paren{
        g^{\alert{\conj k j}} \diff_{\alert{j}}
        u_{\alert{\conj k} \conj 1 \conj\bbeta_{q-2}}
        -g^{\alert{\conj k j}} \diff_{\alert{j}}\vphi_{K_X} \cdot
        u_{\alert{\conj k} \conj 1 \conj\bbeta_{q-2}}
        }
        +g^{\conj 1 1} g^{\conj \gamma \gamma} g^{\conj\alpha
        \alpha} \diff_{\gamma} g_{\alpha \conj 1} \cdot
        u_{\conj\alpha \conj\gamma \conj\bbeta_{q-2}}
      \\
      &=-g^{\conj 1 1} \paren{\dfadj u}_{\conj 1 \conj\bbeta_{q-2}}
        +g^{\conj 1 1} g^{\conj \gamma \gamma} g^{\conj\alpha
        \alpha} \diff_{\gamma} g_{\alpha \conj 1} \cdot
        u_{\conj\alpha \conj\gamma \conj\bbeta_{q-2}} \; .
    \end{align*}
    Since $\del_\gamma g_{\alpha\conj1}$ is symmetric in $\alpha,
    \gamma$ (for $X$ being K\"ahler) while $u_{\conj\alpha\conj\gamma
      \conj{\bbeta}_{q-2}}$ is anti-symmetric in $\alpha, \gamma$, the
    last term in the expression above vanishes.
    As $\dfadj u = 0$ on $X$ by assumption, the proof is thus
    completed after applying $\fdiff{z^1} \ctrt$ to both sides. \qedhere
    % \begin{align*}
    %   \paren{\dfadj\HRes_p(u)}_{\ov{\boldsymbol\beta}_{q-2}}
    %   &=
    %   -g^{\bar\beta\alpha}
    %   \nabla_\alpha 
    %   \paren{
    %   g^{\bar j1}
    %   u_{\bar j\bar\beta\ov{\boldsymbol\beta}_{q-2}}
    %	}
    %   \\
    %   &
    %   =
    %   -g^{\bar\beta\alpha}
    %   \del_\alpha 
    %   \paren{
    %   g^{\bar j1}
    %   u_{\bar j\bar\beta\ov{\boldsymbol\beta}_{q-2}}
    %	}
    %   +
    %   g^{\bar\beta\alpha}
    %   \paren{\del_\alpha\varphi_{K_{D_p}}-\del_\alpha\varphi_L}
    %   g^{\bar j1}
    %   u_{\bar j\bar\beta\ov{\boldsymbol\beta}_{q-2}}
    %   \\
    %   &
    %   =
    %   -
    %   \sum_{\beta=2}^n
    %   \del_\beta
    %   \paren{
    %   g^{\bar j1}
    %   u_{\bar j\bar\beta\ov{\boldsymbol\beta}_{q-2}}
    %	}
    %   -
    %   \sum_{\beta=2}^n
    %   \paren{\del_\beta\varphi_L}
    %   u_{\bar1\bar\beta\ov{\boldsymbol\beta}_{q-2}}\;\;\;\text{at}\;\;x.
    % \end{align*}
    % The first term is computed as
    % \begin{align*}
    %   -
    %   \sum_{\beta=2}^n
    %   \del_\beta
    %   &
    %   \paren{
    %   g^{\bar j1}
    %   u_{\bar j\bar\beta\ov{\boldsymbol\beta}_{q-2}}
    %	}
    %   =
    %   -
    %   \sum_{\beta=2}^n
    %   \paren{
    %   \del_\beta g^{\bar j1}
    %   u_{\bar j\bar\beta\ov{\boldsymbol\beta}_{q-2}}
    %   +
    %   g^{\bar j1}
    %   \del_\beta
    %   u_{\bar j\bar\beta\ov{\boldsymbol\beta}_{q-2}}
    %	}
    %   \\
    %   &=
    %   \sum_{\beta=2}^n
    %   \paren{
    %   g^{\bar jk}\paren{\del_\beta g_{k\bar l}}g^{\bar l1}
    %   u_{\bar j\bar\beta\ov{\boldsymbol\beta}_{q-2}}
    %   -
    %   \del_\beta
    %   u_{\bar1\bar\beta\ov{\boldsymbol\beta}_{q-2}}
    %	}
    %   \\
    %   &=
    %   \sum_{\beta=2}^n
    %   \paren{
    %   \paren{\del_\beta g_{1\bar1}}
    %   u_{\bar1\bar\beta\ov{\boldsymbol\beta}_{q-2}}
    %   -
    %   \del_\beta
    %   u_{\bar1\bar\beta\ov{\boldsymbol\beta}_{q-2}}
    %	}
    %   +
    %   \sum_{\beta,\gamma=2}^n
    %   \paren{\del_\beta g_{\gamma\bar1}}
    %   u_{\bar\gamma\bar\beta\ov{\boldsymbol\beta}_{q-2}}	.
    % \end{align*}
    % Since $\del_\beta g_{\gamma\bar1}$ is symmetric in $\beta, \gamma$ and $u_{\bar\gamma\bar\beta\bar\beta_1,\ldots,\bar\beta_{q-2}}$ is anti-symmetric in $\beta, \gamma$, the last term vanishes.
    % It follows that
    % \begin{align*}
    %   \paren{\dfadj\HRes_p(u)}_{\ov{\boldsymbol\beta}_{q-2}}
    %   &=
    %   \sum_{\beta=2}^n
    %   \paren{
    %   \paren{\del_\beta g_{1\bar1}}
    %   u_{\bar1\bar\beta\ov{\boldsymbol\beta}_{q-2}}
    %   -
    %   \del_\beta
    %   u_{\bar1\bar\beta\ov{\boldsymbol\beta}_{q-2}}
    %	}
    %   -
    %   \sum_{\beta=2}^n
    %   \paren{\del_\beta\varphi_L}
    %   u_{\bar1\bar\beta\ov{\boldsymbol\beta}_{q-2}}
    %   \\
    %   &=
    %   \paren{\dfadj u}_{\bar1\bar\beta\ov{\boldsymbol\beta}_{q-2}}=0.
    % \end{align*}
    % Indeed, at the given point $x$, we have
    % \begin{align*}
    %   \paren{\dfadj u}_{\bar1\ov{\boldsymbol\beta}_{q-2}}
    %   &=
    %   -g^{\bar jk}\paren{\nabla_ku_{\bar j\bar 1\ov{\boldsymbol\beta}_{q-2}}}
    %   =
    %   -g^{\bar jk}
    %   \paren{
    %   \del_ku_{\bar j\bar1\ov{\boldsymbol\beta}_{q-2}}
    %   -
    %   \paren{
    %			\del_k\varphi_{K_X}
    %			-
    %			\del_k\varphi_L
    % }
    %   u_{\bar j\bar1\ov{\boldsymbol\beta}_{q-2}}
    %	}
    %   \\
    %   &=
    %   -
    %   \sum_{\beta=2}^n
    %   \paren{
    %   \del_\beta
    %   u_{\bar\beta\bar1\ov{\boldsymbol\beta}_{q-2}}
    %   -
    %   \paren{
    %			\del_\beta g_{1\bar1}
    %			-
    %			\del_\beta\varphi_L
    % }
    %   u_{\bar\beta\bar1\ov{\boldsymbol\beta}_{q-2}}
    %	}.
    % \end{align*}
    % This completes the proof.
  \end{proof}


  Furthermore, we claim that, if
  % $u\in\mathcal{H}^{n,q}(X;L)_{\varphi_L}$ with
  % $\ibddbar\varphi_L\ge0$
  $u$ satisfies $\dbar u = 0$ and $\nabla^{(0,1)}u = 0$, then
  $\HRes_p(u)$ is $\dbar$-closed.
  % One can notice that $u$ is $\dbar$-closed, then so is
  % $\HRes_p(u)$. 
  This is shown via the following formula, which is a special case and
  a slight variant of \cite{Donnelly&Xavier}*{(2.4)} and
  \cite{Ohsawa&Takegoshi-spectral_seq}*{Prop.~1.5} (see also
  \cite{Takegoshi_higher-direct-images}*{(1.9)} and
  \cite{Matsumura_injectivity-Kaehler}*{Lemma 2.1}). 
  

  \newcommand{\lcSb}{\lcS+1[b]}
  \newcommand{\idxj}{\idx[\conj j]}
  
  \begin{lemma}[cf.~\cite{Donnelly&Xavier}*{(2.4)},
    \cite{Ohsawa&Takegoshi-spectral_seq}*{Prop.~1.5},
    \cite{Takegoshi_higher-direct-images}*{(1.9)} and
    \cite{Matsumura_injectivity-Kaehler}*{Lemma
      2.1}] \label{lem:commutator-dbar-ctrt}
    Let $\vphi$ be a smooth function and $u$ be a smooth
    ($K_X$-valued) $(0,q)$-form on a K\"ahler manifold.
    They satisfy the formula
    \begin{equation*}
      \dbar\paren{\idxup{\diff\vphi}.  u}
      =\idxup{\ibddbar\vphi} . u
      -\idxup{\diff\vphi} . \paren{\dbar u}
      +\idxup{\diff\vphi} \cdot \nabla^{(0,1)}_\bullet u \; ,
    \end{equation*}%
    or, when a local holomorphic coordinate system is fixed and
    the Einstein summation convention is applied, 
    \begin{equation*}
      \paren{\dbar\paren{\idxup{\diff\vphi} . u}}_{\conj J_{q}}
      =\sum_{\nu=1}^q \diff^{\conj\ell} \diff_{\conj j_\nu} \vphi \:
      u_{\idxj 1[\dotsm (\conj \ell)_\nu].q}
      -\diff^{\conj\ell}\vphi  \:\paren{\dbar u}_{\conj\ell\conj J_q}
      +\diff^{\conj\ell}\vphi \:\nabla_{\conj\ell} u_{\conj J_q} 
    \end{equation*}
    for any multi-indices $J_q = (\idx[j]1,q)$, pointwisely.
  \end{lemma}

  \begin{proof}
    A direct computation yields
    \begin{align*}
      \paren{\dbar\paren{\idxup{\diff\vphi} . u}}_{\conj
      J_{q}}
      &=\sum_{\nu=1}^q (-1)^{\nu-1} \diff_{\conj j_\nu}
        \paren{\idxup{\diff\vphi}.  u}_{\idxj 1[\dotsm \widehat
        {\conj j}_\nu].q}
        =\sum_{\nu=1}^q (-1)^{\nu-1} \diff_{\conj j_\nu}
        \paren{\diff_{\ell}\vphi \: u^\ell_{\;\idxj 1[\dotsm
        \widehat{\conj j}_\nu].q}}
      \\
      &=\sum_{\nu=1}^q (-1)^{\nu-1} \paren{
        \diff_{\conj j_\nu}\diff_{\ell}\vphi \: u^{\ell}_{\;\idxj 1[\dotsm
        \widehat {\conj j}_\nu].q}
        +\diff_{\ell}\vphi \: \nabla_{\conj j_\nu} u^\ell_{\;\idxj 1[\dotsm
        \widehat {\conj j}_\nu].q}
        }
      \\
      &=\sum_{\nu=1}^q
        \diff^{\conj \ell}\diff_{\conj j_\nu}\vphi \: u_{\idxj 1[\dotsm
        (\conj\ell)_\nu].q}
        -\diff^{\conj\ell}\vphi \sum_{\nu=1}^q (-1)^{\nu} 
        \nabla_{\conj j_\nu} u_{\conj\ell \idxj 1[\dotsm
        \widehat {\conj j}_\nu].q}
        \begin{aligned}[t]
          &-\diff^{\conj\ell}\vphi \: \nabla_{\conj \ell} u_{\conj
            J_q} \\
          &+\diff^{\conj\ell}\vphi \: \nabla_{\conj \ell}
          u_{\conj J_q}
        \end{aligned}
      \\
      &=\sum_{\nu=1}^q
        \diff^{\conj \ell}\diff_{\conj j_\nu}\vphi \: u_{\idxj 1[\dotsm
        (\conj\ell)_\nu].q}
        -\diff^{\conj\ell}\vphi
        \:\paren{\dbar u}_{\conj\ell\conj J_q}
        +\diff^{\conj\ell}\vphi \: \nabla_{\conj \ell} u_{\conj
        J_q} \; . \qedhere
    \end{align*}
    % as desired.
  \end{proof}



  We see that $\HRes_p(u)$ is $\dbar$-closed by
  putting $z^1$ in place of $\vphi$ in Lemma \ref{lem:commutator-dbar-ctrt}.
  The following theorem is then immediate.
  \begin{thm} \label{thm:residue-harmonic}
    If $u$ is a harmonic $K_X\otimes L$-valued $(0,q)$-form on $X$ with
    respect to $\vphi_L$ and $\omega$ such that $\nabla^{(0,1)}u=0$,
    then $\HRes_p(u)$ is a harmonic $K_{D_p}\otimes
    L\vert_{D_p}$-valued $(0,q-1)$-form on $D_p$ with respect to
    $\varphi\vert_{D_p}$ and $\omega\vert_{D_p}$.
  \end{thm}

  \begin{proof}
    From the above discussion, $\HRes_p(u)$ is
    $\dbar$- and $\dfadj$-closed on $D_p$.
    Since $\varphi_L$ is smooth and $D_p$ is compact, it follows that
    $\HRes_p(u)\in\Dom\dbar^*$ with
    respect to $\varphi_L\vert_{D_p}$ and $\omega\vert_{D_p}$.
    This completes the proof.
  \end{proof}

  \begin{remark}
    When $\varphi_L$ has the singularity property described in
    \cite{Chan&Choi_injectivity-I}*{\S 2.2 item (2)} for $\varphi_F$,
    i.e.~$\varphi_L$ has only neat analytic singularities such that
    $P_L:=\varphi_L^{-1}(-\infty)$ is a divisor with $P_L+D$ having
    snc and that $P_L$ contains no components of $D$, the claim that
    $\HRes_p(u)\in\Dom\dbar^*$ with
    respect to $\norm\cdot_{D_p}:=\norm\cdot_{D_p,\varphi_L,\omega}$
    still holds true (under the assumption that $\omega\vert_{D_p}$ is
    a complete K\"ahler form on $D_p\setminus P_L$).
    Indeed, $\HRes_p(u)$ can be shown to be $L^2$
    with respect to $\norm\cdot_{D_p}$ by the arguments in
    \cite{Chan&Choi_injectivity-I}*{Prop.~3.2.3, Remark 3.2.4 and
      Prop.~3.3.2} (with $u$ here in place of $\frac{\rs u}{\sect_D}$
    there).
    With $\dfadj$ (with respect to $\varphi_L\vert_{D_p}$ and
    $\omega\vert_{D_p}$) being a smooth operator on $D_p\setminus P_L$
    % (different from the situation in Lemma \ref{lem:su-harmonicity})
    and $\omega\vert_{D_p}$ being complete,
    $\HRes_p(u)\in\Dom\dbadj$ follows
    from the classical arguments.
  \end{remark}

}




% \section{Proof of the Main Result}\label{sec:proof}

% \subsection{Proof of Corollary \ref{cor:main}}\label{subsec:n2}

% Corollary \ref{cor:main} can be  proved by repeating the same argument as in the proof of Theorem \ref{thm:main}. 
% Nevertheless, in this subsection, we deduce Corollary \ref{cor:main} from Theorem \ref{thm:main} 
% using the previous work \cite[Theorem 1.6]{Mat}. 




% \begin{proof}[Proof of Corollary \ref{cor:main}]
% %In the proof, we freely use the notation in Conjecture \ref{conj:fujino}. 
% Let us consider the following commutative diagram  induced by 
% the short exact sequence $0 \to K_{X} \to K_{X}\otimes D \to K_{D} \to 0 $ and the multiplication map: 
% \begin{align*}
% \vcenter{ \xymatrix{
% &\ar[d] & \ar[d]\\
% &
% H^q(X,  K_{X}\otimes F )\ar[d]^-{}\ar[r] ^-{\otimes s}
% \ar[d]^-{\otimes \sect_D}
% &H^q(X, K_{X}  \otimes F^{ \otimes{(m+1)}} )\ar[d]\\ 
% &H^q(X,  K_{X}\otimes D \otimes F)
% \ar[d]^-{}\ar[r] ^-{\otimes s}
% &H^q(X,  K_{X} \otimes D \otimes F^{\otimes{(m+1)}}) \ar[d]\\ 
% &H^q(D, K_{D}\otimes F )
% \ar[d]\ar[r]^-{\otimes s|_{D} } 
% & H^q(D,  K_{D}\otimes F^{\otimes(m+1)} ). \ar[d]\\ 
% & & 
% }}
% \end{align*}
% The line bundle $M:=F^{\otimes m}$ with the metric $h_{M}:=h_{F}^{\otimes m}$ 
% satisfies the curvature assumption in Theorem \ref{thm:main}. 
% Further, the zero locus $s|_{D}^{-1}(0)$ contains no lc centers of $(X, D)$ by assumption; 
% hence, by Theorem \ref{thm:main}, the lowest multiplication map $\otimes s|_{D}$ in the diagram is injective for every $q$. 
% This implies that a cohomology class $\alpha \in H^q(X,  K_{X}\otimes D \otimes F)$ with 
% $s  \alpha =0 \in H^q(X,  K_{X}\otimes D \otimes F^{\otimes (m+1)})$ 
% lies in the image of the vertical multiplication map $\otimes \sect_D$ on the left, 
% where $\sect_D$ is the canonical section of the effective divisor $D$. 
% Then, the conclusion of $\alpha=0$ follows from \cite[Theorem 1.6]{Mat} (or \cite[]{CC}). 
% \end{proof}




\section{Proofs of the main results}\label{sec:proof}

\subsection{Proof of Corollary \ref{cor:main}}\label{subsec:n2}

Corollary \ref{cor:main} can be proved by adapting the % same argument as in the
proof of Theorem \ref{thm:main} (or, more precisely, Theorem
\ref{thm:ker-nu=ker-tau}; see Remark \ref{rem:general-commut-diagram}
for details).
The proof involves an inductive reduction of the setup to
subvarieties on which the relevant injectivity result is known
or can be proved via Enoki's arguments (i.e.~harmonic theory for
cohomology is valid).
To get an essence of the argument, here 
% Nevertheless, in this subsection,
we deduce Corollary \ref{cor:main} from Theorem \ref{thm:main} 
using the previous work \cite{Matsumura_injectivity-lc}*{Thm.~1.6} (or
\cite{Chan&Choi_injectivity-I}*{Thm.~1.2.1}).




\begin{proof}[Proof of Corollary \ref{cor:main}]
%In the proof, we freely use the notation in Conjecture \ref{conj:fujino}. 
Consider the following commutative diagram  induced by 
the short exact sequence $0 \to K_{X} \to K_{X}\otimes D \to K_{D} \to
0 $ and the multiplication map $\otimes s$: 
\begin{equation*}
  % \vcenter{
  \xymatrix@R=3ex{
    \ar[d] & \ar[d]\\
    {H^q(X,  K_{X}\otimes F )} \ar[d]^-{}\ar[r] ^-{\otimes s}
    \ar[d]^-{\otimes \sect_D}
    &{H^q(X, K_{X}  \otimes F^{ \otimes{(m+1)}} )} \ar[d]\\ 
    {H^q(X,  K_{X}\otimes D \otimes F)}
    \ar[d]^-{}\ar[r] ^-{\otimes s}
    &{H^q(X,  K_{X} \otimes D \otimes F^{\otimes{(m+1)}})} \ar[d]\\ 
    {H^q(D, K_{D}\otimes F )}
    \ar[d]\ar[r]^-{\otimes s|_{D} } 
    & {H^q(D,  K_{D}\otimes F^{\otimes(m+1)} ) \; .}  \ar[d]\\ 
    & 
  }
  % }
\end{equation*}
The line bundle $M:=F^{\otimes m}$ with the metric
$h_{M}:=h_{F}^{\otimes m} = e^{-m\vphi_F}$ satisfies the curvature
assumption in Theorem \ref{thm:main} and the zero locus $s^{-1}(0)$
contains no lc centers of $(X, D)$ by assumption.
Hence, by Theorem \ref{thm:main}, the lowest multiplication map $\otimes s|_{D}$ in the diagram is injective for every $q$. 
This implies that a cohomology class $\alpha \in H^q(X,  K_{X}\otimes D \otimes F)$ with 
$s  \alpha =0 \in H^q(X,  K_{X}\otimes D \otimes F^{\otimes (m+1)})$ 
lies in the image of the vertical multiplication map $\otimes \sect_D$ on the left, 
where $\sect_D$ is the canonical section of the effective divisor $D$. 
Then, the conclusion $\alpha=0$ follows from \cite{Matsumura_injectivity-lc}*{Thm.~1.6} 
(or \cite{Chan&Choi_injectivity-I}*{Thm.~1.2.1}). 
\end{proof}



\subsection{Proof of Theorem \ref{thm:main} for a simple case} % \label{subsec:n2}
\label{sec:proof-of-simple-case}

In this subsection, we prove Theorem \ref{thm:main} in the simple case 
where \emph{$D$ has two components (i.e.~$D=D_{1}+D_{2}$) whose intersection
has only one irreducible component} and the degree of cohomology
groups is $q=1$.
% This simple case is completely contained in the general case discussed in Section \ref{subsec:general}, 
% but we illustrate a detailed proof, which is quite helpful in understanding the essence of the proof. 
% The proof of the general case is an extension of the argument in this section using lc strata. 
While this case is contained in the proof presented in Section
\ref{subsec:general}, a detailed proof of it is presented here in
order to illustrate the essence of the proof in the general case
without being obscured by the notation.
The proof in Section \ref{subsec:general} follows the same arguments
but on the lower-dimensional lc strata with more components.


\begin{proof}[Proof of Theorem \ref{thm:main} in the case of $D=D_{1}+D_{2}$ and $q=1$] 

% Suppose that $D=D_{1}+D_{2}$ and $q=1$ in Theorem \ref{thm:main}. 
Under the given assumptions and for a given cohomology class $\alpha
\in H^1(D,  K_{D} \otimes F)$, we prove here that $\alpha $ is actually $0$
when $s  \alpha =0 \in H^1(D,  K_{D} \otimes F \otimes M)$. 

\begin{step}[``Harmonic representative'' of $\alpha$] \label{step:harmonic-rep}
We intend to work with an \emph{``optimal''} representative of $\alpha$ via the Dolbeault isomorphism, 
% which means an appropriate harmonic form has the minimum $L^{2}$-norm in the forms representing $\alpha$.  
in the analogy of a harmonic form being the element with the minimal
$L^2$ norm in the corresponding cohomology class.
Nevertheless, at the time of writing, there is not yet a well
established theory of Dolbeault isomorphism and harmonic theory for
cohomology groups on the singular space $D$.
For this purpose, we consider the following diagram
\begin{equation}\label{h}
  \begin{aligned}
    \xymatrix@C=3.5em@R=3.5ex{
      \ar[d] & \ar[d]\\
      {\smash{\bigoplus_{p=1}^{2}} H^1(D_{p}, K_{D_{p}}\otimes F )}
      \ar[d]^-{}\ar[r] ^-{\otimes (s|_{D_{1}}, s|_{D_{2}})}
      \ar[d]^-{\tau}
      &{\smash{\bigoplus_{p=1}^{2}} H^1(D_{p},
        K_{D_{p}} \otimes F \otimes M )}
      \ar[d]\\
      {H^1(D, K_{D} \otimes F)} \ar[d]^-{}\ar[r] ^-{\otimes s}
      &{H^1(D,  K_{D} \otimes F\otimes M)} \ar[d]\\
      {H^1(D_{1}\cap D_{2}, K_{D_{1}\cap D_{2}}\otimes F )}
      \ar[d]\ar[r]^-{\otimes s|_{D_{1}\cap D_{2}} }
      &{H^1(D_{1}\cap D_{2},  K_{D_{1}\cap D_{2}}\otimes F \otimes M)} \ar[d]\\
      & }
  \end{aligned}
\end{equation}
induced from $0 \to K_{D_{1}} \oplus K_{D_{2}} \to K_{D} \to K_{D_{1} \cap D_{2}} \to 0$, 
% which corresponds to $\eqref{eq-ex2}$ in the case of $\rho=0, \sigma=1, \tau=2$ (cf.\,\eqref{eq-ex}). 
which in turn can be obtained by tensoring $K_X \otimes D$ to the
short exact sequence of adjoint ideal sheaves
\begin{equation*}
  \renewcommand{\objectstyle}{\displaystyle}
  \xymatrix@R=3.5ex{
    {0} \ar[r]
    &{\faidlof|1|/|0|*} \ar[r] \ar[d]_-{\Res^1}^-{\isom}
    &{\faidlof|2|/|0|*} \ar[r]
    &{\faidlof|2|/|1|*} \ar[r] \ar[d]^-{\Res^2}_-{\isom}
    &{0}
    \\
    &{\residlof|1|*} %\ar@{}[u]|-*[left]+{\isom}
    &&{\residlof|2|*} %\ar@{}[u]|-*[left]+{\isom}
  }
\end{equation*}
where $\aidlof* :=\aidlof$ and $\residlof* := \residlof$ and
the isomorphism $\faidlof/-1* \xrightarrow[\isom]{\Res^\sigma}
\residlof*$ is induced from the residue short exact sequence in
Section \ref{subsec:residue}.
Notice that the Dolbeault isomorphism and harmonic theory are
valid on $D_1$, $D_2$ and $D_1 \cap D_2$.
The multiplication map $\otimes s|_{D_{1}\cap D_{2}}$ on the bottom
row is non-zero by the assumption on $s^{-1}(0)$ and the curvature
assumption is still satisfied after restricting $F$ and $M$ to
$D_{1}\cap D_{2}$.
Hence, Enoki's injectivity theorem can be invoked to assert that
$\otimes s|_{D_{1}\cap D_{2}}$ is injective. 
Then, by an easy diagram chasing, we can find harmonic forms $u_{p}$ for $p=1,2$ such that 
\begin{equation*}
  u_{p} \in \mathcal{H}^{n-1,1}(D_{p}; F)_{\vphi_F} \cong H^1(D_{p},  K_{D_{p}}\otimes F ) 
  \text{ with } \alpha = \tau(\eqcls{u_{1}}, \eqcls{u_{2}}), 
\end{equation*}
where $\mathcal{H}^{n-1,1}(D_{p}; F)_{\vphi_F}$ denotes 
the space of $F|_{D_{p}}$-valued harmonic forms of $(n-1,1)$-type with respect to $\res{e^{-\vphi_F}}_{D_{p}}$. 
Note that there is freedom in the choice of $(u_{1}, u_{2})$  
since $\tau$ may not be injective. 
% $(u_{1}, u_{2})$ may not be the best representation of  $\alpha$.
% For this reason, by the orthogonal decomposition, we re-
To obtain the unique ``optimal'' representative of $\alpha$, we choose
the pair $(u_{1}, u_{2})$ 
with $\alpha = \tau(\eqcls{u_{1}}, \eqcls{u_{2}})$ that satisfies
\begin{equation}\label{eq-orth}
  (u_{1}, u_{2}) \in (\Ker \tau)^{\perp} \subset 
  \Ker \tau \oplus (\Ker \tau)^{\perp}= \bigoplus_{p=1}^{2}
  \mathcal{H}^{n-1,1}(D_{p}; F)_{\vphi_F} \; ,
\end{equation}
% This can be regarded as the {\textit{best}} representation of  $\alpha$. 
% Our purpose is to prove that the $L^{2}$-norm 
% $\norm{s u_{1}}_{\vphi_M, D_1}^2 +\norm{s u_{2}}_{\vphi_M, D_2}^2$ 
% is actually zero. 
in which $\paren{\ker\tau}^\perp$ is the orthogonal complement of
$\ker\tau$ with respect to the (squared) residue norm
$\norm\cdot_{\lcc|1|'}^2 =\norm\cdot_{D_1}^2 +\norm\cdot_{D_2}^2$
(defined as in \eqref{eq:residue-norm} with $\sigma :=1$, $\vphi_L
:=\vphi_F$ and $\lcS[V,p] :=D_p$).
With such choice of representative, our goal is to prove that the
$L^{2}$-norm $\norm{s u_{1}}_{D_1, \vphi_M}^2 +\norm{s u_{2}}_{D_2,
  \vphi_M}^2$ is actually zero (where $\norm\cdot_{D_p, \vphi_M}$'s
are defined as in \eqref{eq:residue-norm} with $\vphi_L:=\vphi_F
+\vphi_M$).


\end{step}

\begin{step}[Obstruction for $\norm{s u_{1}}_{D_1, \vphi_M}^2 +\norm{s
      u_{2}}_{D_2, \vphi_M}^2$ from being zero]
  \label{item:expression-of-su-simple}
  

  % In this step, we examine the relevant $L^{2}$-norm 
  % to obtain an obstruction for our purpose as a $F$-valued form on $D_{1} \cap D_{2}$. 
  % For this purpose, by using the Dolbeault isomorphism, the Poincar\'e residue map, and the assumption of $s \alpha=0$, 
  % we prepare the following data: 
  We make use of the assumption $s \alpha=0$ and the \v
  Cech--Dolbeault isomorphism to re-express $\norm{s u_{1}}_{D_1,
    \vphi_M}^2 +\norm{s u_{2}}_{D_2, \vphi_M}^2$ as follows.


  \begin{itemize}
  \item Take $\alpha_{p;\:ij}   \in H^{0}(V_{ij} \cap D_p ,
    K_{D_{p}}\otimes F)$  
    for every open set $V_{ij} :=V_i \cap V_j$ with $i,j \in I$ and
    $V_i \cap V_j \cap D_p\neq \emptyset$ such that the family
    $\{\alpha_{p;\:ij}\}_{i,j \in I}$ is a \v Cech cocycle
    representing $u_{p}$ via the \v Cech--Dolbeault isomorphism on
    $D_p$.
    It follows that there exists an $L^2$ section $v_{p,(2)}$ of
    $K_{D_p} \otimes \res F_{D_p}$ on $D_p$ with respect to
    $\norm\cdot_{D_p}$ such that (under Einstein summation
    convention) 
    \begin{equation*}
      u_p
      \overset{\text{\eqref{eq:Cech-Dolbeault-isom}}}= \:
      \dbar v_{p,(2)} -\dbar\rho^j \cdot \rho^i \:\alpha_{p;\:ij} \; .
    \end{equation*}


  \item Take $f_{ij} \in H^{0}(V_{ij} , K_X \otimes D
    \otimes F \otimes \defidlof{D_1 \cap D_2})$ for $i,j \in I$
    satisfying 
    \begin{equation*}
      \Res^1\paren{f_{ij}}
      :=\paren{\PRes[D_1](\frac{f_{ij}}{\sect_D}) \:,\:
        \PRes[D_2](\frac{f_{ij}}{\sect_D})} 
      = \paren{\alpha_{1;\:ij} , \alpha_{2;\:ij}} 
    \end{equation*}
    whose existence are guaranteed by the surjectivity of the
    residue isomorphism $\Res^1$ on Stein open sets such that
    \begin{equation*}
      \renewcommand{\objectstyle}{\displaystyle}
      \xymatrix@C=1em@R=0.8em{
        {K_X \otimes D \otimes \frac{\defidlof{D_1 \cap
              D_2}}{\defidlof{D}}} 
        \ar@{}[r]|-*+{=}
        \ar@{}[d]|*[left]{\in}
        &{K_X \otimes D \otimes \faidlof|1|/|0|*}
        \ar[rr]^-{\Res^1}_-{\isom}
        &&{K_X \otimes D \otimes \residlof|1|*}
        \ar@{}[r]|-*+{=}
        &{K_{D_1} \oplus K_{D_2}}
        \ar@{}[d]|(.57)*[left]{\in}
        \\
        *+/r 3em/{f_{ij} \bmod \defidlof{D}}
        \ar@{|->}[rrr]
        &&&*+/l 6em/{}
        &*-{\paren{\alpha_{1;\:ij} , \alpha_{2;\:ij}} \; .}
      }
    \end{equation*}
    It is easy to see that $\set{f_{ij} \bmod \defidlof{D}}_{i,j \in
      I}$ is a \v Cech cocycle whose cohomology class in $\cohgp
    1[D]{\logKX \otimes \frac{\defidlof{D_1 \cap
          D_2}}{\defidlof{D}}}$ is mapped to $\alpha$ via $\tau$. 

  \item The assumption $s \alpha=0$ in $\cohgp 1[D]{K_D
      \otimes F \otimes M}$ guarantees the
    existence of $\lambda_{i} \in H^{0}(V_{i}, K_X \otimes D\otimes
    F \otimes M)$ for $i \in I$ such that
    \begin{equation*}
      s f_{ij} \equiv \lambda_j -\lambda_i \mod \defidlof{D}
      \quad\text{ on } V_{ij} \; .
    \end{equation*}
    Note that the coefficients of $\lambda_i$ need not lie in
    $\defidlof{D_1 \cap D_2}$ even though so do those of $f_{ij}$.
    By setting
    \begin{equation*}
      \rs*\lambda_{p;\:i} := \PRes[D_p](\frac{\lambda_i}{\sect_D})
      \cdot \sect_{(p)}
      \quad\text{ on $V_i \cap D_p$ for } i\in I
      \text{ and } p = 1,2 \; ,
    \end{equation*}
    it then follows that
    \begin{equation*}
      s\alpha_{p;\:ij} \sect_{(p)} =\rs*\lambda_{p;\:j} -\rs*\lambda_{p;\:i}
      \quad\text{ on } V_{ij} \cap D_p \; .
    \end{equation*}
    Note that $\rs*\lambda_{p;\:i}$ is holomorphic on $V_i \cap D_p$
    (while $\PRes[D_p](\frac{\lambda_i}{\sect_D})$ may not be).
  \end{itemize}

  Since $u_{p}$ is harmonic with respect to $\vphi_F$ on $D_p$ and
  we have $\ibddbar\vphi_F \geq 0$ and $%-C\omega \leq
  \ibddbar\vphi_M \leq C\ibddbar\vphi_F$ on $D_p$ for some constant
  $C > 0$ by assumption,
  Proposition \ref{prop:consequence-of-positivity} guarantees that  
  % \begin{equation*} %\label{eq-harmonic}
  %   su_p \in \Harm'/n-1,1/<D_p>{F\otimes M},{\vphi_F+\vphi_M} \; ,
  % \end{equation*}%
  $su_p$ is harmonic with respect to $\vphi_F+\vphi_M$ on $D_p$,
  which is a consequence of Nakano's identity and Enoki's argument.
  It follows that $\iinner{s \dbar v_{p;(2)}}{su_p}_{D_p, \vphi_M}
  =\iinner{\dbar\paren{s v_{p;(2)}}}{su_p}_{D_p, \vphi_M} = 0$.
  Summarizing the above discussion, it follows that
  \begin{align*}
    \norm{s u_{p}}_{D_p, \vphi_M}^2 
    &= -\sum_{i,j\in I}\iinner{\dbar\rho^{j} \cdot \rho^i \:s
      \alpha_{p;\:ij} \:}{\:s u_p}_{D_p, \vphi_M}\\
    &= -\sum_{i,j\in I}\iinner{\dbar\rho^{j} \cdot \rho^i \:s
      \alpha_{p;\:ij} \sect_{(p)} \:}{\:s u_p \sect_{(p)}}_{D_p, \vphi_M+\phi_{(p)}}\\
    &= -\sum_{i,j\in I}\iinner{\dbar\rho^{j} \cdot \rho^i
      \paren{\rs*\lambda_{p;\:j}- \rs*\lambda_{p;\:i}} \:}
      {\:s u_p \sect_{(p)}}_{D_p, \vphi_M+\phi_{(p)}}\\
    &= -\sum_{j\in I}\iinner{\dbar\paren{\rho^j
      \rs*\lambda_{p;\:j}} \:}{\: s u_p \sect_{(p)}}_{D_p,
      \vphi_M+\phi_{(p)}}
      =: -\iinner{\dbar v_{p;(\infty)} }{ s u_p \sect_{(p)}}_{D_p,
      \vphi_M+\phi_{(p)}}
      \; .
  \end{align*}
  The notation $v_{p;(\infty)} :=\sum_{j\in I} \rho^j
  \rs*\lambda_{p;\:j}$ is used for the consistency with the notation in
  Proposition \ref{prop:res-formula-dbar-exact-dot-harmonic}.

  The residue computation in Proposition
  \ref{prop:res-formula-dbar-exact-dot-harmonic} further brings the
  expression of $\norm{s u_{p}}_{D_p, \vphi_M}^2$ for each $p=1,2$ to an inner
  product on $D_1 \cap D_2$.
  As $\lcS|2|[b] :=D_1 \cap D_2$ has only $1$ component, the index set $\Iset|2|
  =\set{b}$ is a singleton.
  Moreover, the general different $\Diff_{D_1 \cap D_2}(D) =\Diff_b(D)$ is
  trivial, so we choose its canonical section and the corresponding
  potential such that $\sect_{(b)} \equiv 1$ and $\phi_{(b)} \equiv 0$
  (and $\psi_{(b)} \equiv -1$) on $D_1 \cap D_2$.
  Let $\PRes[\lcS|2|[b] | D_p]$ be the Poincar\'e residue map from
  $D_p$ to $D_1 \cap D_2$.
  We fix the sign convention such that
  \begin{equation*}
    \rs*\lambda_{b;\:i}
    =\frac{\rs*\lambda_{b;\:i}}{\sect_{(b)}}
    :=\PRes[\lcS|2|[b]](\frac{\lambda_i}{\sect_D})
    \begin{aligned}[t]
      &= \PRes[\lcS|2|[b] | D_1] \circ
      \PRes[D_1](\frac{\lambda_i}{\sect_D})
      \\
      &=\PRes[\lcS|2|[b] | D_1](\frac{\rs*\lambda_{1;\:i}}{\sect_{(1)}})
    \end{aligned}
    \begin{aligned}[t]
      &=-\PRes[\lcS|2|[b] | D_2] \circ
      \PRes[D_2](\frac{\lambda_i}{\sect_D})
      \\
      &=-\PRes[\lcS|2|[b] | D_2](\frac{\rs*\lambda_{2;\:i}}{\sect_{(2)}})
    \end{aligned} \; .
  \end{equation*}
  Following the computation in Proposition
  \ref{prop:res-formula-dbar-exact-dot-harmonic}, we obtain
  \begin{align*}
    &~\norm{s u_{1}}_{D_1, \vphi_M}^2 +\norm{s u_{2}}_{D_2,
      \vphi_M}^2
    \\
    =&~-\iinner{\dbar v_{1;(\infty)} }{ s u_1 \sect_{(1)}}_{D_1,
       \vphi_M+\phi_{(1)}}
       -\iinner{\dbar v_{2;(\infty)} }{ s u_2 \sect_{(2)}}_{D_2,
       \vphi_M+\phi_{(2)}}
    \\
    =&~
       \begin{multlined}[t]
         \sum_{i\in I}\iinner{\rho^i \PRes[\lcS|2|[b] |
           D_1](\frac{\rs*\lambda_{1;\:i}}{\sect_{(1)}}) }{
           \: s\:\PRes[\lcS|2|[b] | D_1](\idxup{\diff\psi_{(1)}}. u_1)
         }_{D_1 \cap D_2, \vphi_M} \\
         +\sum_{i\in I}\iinner{\rho^i
           \PRes[\lcS|2|[b] |
           D_2](\frac{\rs*\lambda_{2;\:i}}{\sect_{(2)}}) }{
           \: s\:\PRes[\lcS|2|[b] | D_2](\idxup{\diff\psi_{(2)}}. u_2)
         }_{D_1 \cap D_2, \vphi_M}
       \end{multlined}
    \\
    =&~\iinner{\sum_{i\in
       I}\rho^i\rs*\lambda_{b;\:i} \:}{\: s\:\paren{
       \PRes[\lcS|2|[b] | D_1](\idxup{\diff\psi_{(1)}}. u_1)
       -\PRes[\lcS|2|[b] | D_2](\idxup{\diff\psi_{(2)}}. u_2)
       }}_{D_1 \cap D_2, \vphi_M}
    \\
    =:&~\iinner{v_{b;(\infty)}}{s w_b}_{D_1 \cap D_2, \vphi_M} \; ,
  \end{align*}
  which is the desired expression.

  It is shown below that
  \begin{equation} \label{eq:w-prelim-formula}
    w_b :=\PRes[\lcS|2|[b] | D_1](\idxup{\diff\psi_{(1)}}. u_1)
    -\PRes[\lcS|2|[b] | D_2](\idxup{\diff\psi_{(2)}}. u_2)
  \end{equation}
  is actually $0$ on $D_1 \cap D_2$, which will then conclude the proof.

  % \begin{itemize}
  % \item[$\bullet$] Take $\beta_{ij,p}   \in H^{0}(V_{ij}, K_{D_{p}}\otimes F)$ 
  %   such that the family $\{\beta_{ij,p}\}$ is a cocycle  corresponding to $u_{p}$ via the \v Cech--Dolbeault isomorphism. 




  % \item[$\bullet$] Take $\alpha_{ij} \in H^{0}(V_{ij}, K_X \otimes D
  %   \otimes \defidlof{D_1 \cap D_2} \otimes F)$
  %   satisfying that 
  %   \begin{equation*}
  %     \Res^1\paren{\alpha_{ij}}
  %     :=\paren{\PRes[D_1](\frac{\alpha_{ij}}{\sect_D}) \:,\:
  %     \PRes[D_2](\frac{\alpha_{ij}}{\sect_D})} 
  %     = \paren{\beta_{ij, 1}, \beta_{ij, 2}}, 
  %   \end{equation*}
  %   by the  residue isomorphism: 
  %   \begin{equation*}
  %     \xymatrix@R=0.1cm{
  %     *+/r 0.5cm/{K_{D_1} \oplus K_{D_2}} &
  %     *+/r 0.5cm/{
  %     K_X \otimes D \otimes \frac{\defidlof{D_1 \cap D_2}}{\defidlof{D}}=K_X \otimes D \otimes \frac{\aidlof|1|*}{\aidlof|0|*}.
  %   }
  %     \ar[l]_-{\Res^1}^-{\isom}
  %   } 
  %   \end{equation*}
  %   More specifically, we may define $\alpha_{ij}$ 
  %   by $\alpha_{ij}:=d\sect_{(2)}  \wedge \sect_{(1)} \:\beta_{ij, 1} +d\sect_{(1)}  \wedge\sect_{(2)} \:\beta_{ij, 2}$. 


  % \item[$\bullet$] Take $\lambda_{i} \in H^{0}(V_{i}, K_X \otimes D\otimes F \otimes M)$ 
  %   satisfying that 
  %   $$\text{
  %   $ s \alpha_{ij} \equiv \lambda_j -\lambda_i$  as a section of 
  %   $K_X \otimes D\otimes \mathcal{O}_{X}/\defidlof{D} \otimes F \otimes M =K_D \otimes F \otimes M$. 
  % }
  %   $$
  %   The cocycle $\{\alpha_{ij}\}$ of $K_X \otimes D \otimes F$ (noting that $\defidlof{D_1 \cap D_2}$ is not tensored) 
  %   corresponds to $\alpha$; hence the assumption of $s \alpha=0$ guarantees the existence of $\lambda_{i}$. 



  % \item[$\bullet$] Take $\rs \lambda_i^1$, $\rs \lambda_i^2$, and $\rs\lambda_i^{12}$ such that 
  %   \begin{equation*}
  %     \lambda_i
  %     =d\sect_{(2)}  \wedge \rs \lambda_i^1
  %     =d\sect_{(1)}  \wedge \rs \lambda_i^2
  %     =d\sect_{(2)}  \wedge d\sect_{(1)}  \wedge \rs\lambda_i^{12}. 
  %   \end{equation*}
  % \end{itemize}
  % By construction, we see that 
  % \begin{align*}
  %   &\bullet \text{$u_{p}=\dbar u_{(2), p} + \dbar \rho^{i} \beta_{ij,p} $ for some global section $u_{(2), p}$ of $K_{D_{p}}\otimes F$};\\
  %   &\bullet s \sect_{(p)} \alpha_{ij}= \rs\lambda_j^p- \rs\lambda_i^p \text{ on } D_{p} 
  %   \text{ as a section of }K_{D_{p}}\otimes F \otimes M.
  % \end{align*}
  % On the other hand, since $u_{p}$ is harmonic and $\sqrt{-1}\Theta_{h_{F}} \geq 0$, 
  % we can conclude that 
  % \begin{align}\label{eq-harmonic}
  %   {\nabla^{(0,1)} u_{p}} =0  \text{ and } \sqrt{-1} \Theta_{h_{F}} \Lambda_{\omega} u_{p} =0
  % \end{align}
  % by applying Nakano's identity. 
  % Further, together with the curvature assumption, 
  % Enoki's argument shows that $su_{p}$ is still harmonic with respect to $h_{F}h_{M}$. 
  % Then, we can easily see that 
  % \begin{align*}
  %   \norm{s u_{p}}_{\vphi_M, D_p}^2 
  %   &= \iinner{\dbar s u_{(2), p}  +  \dbar s \rho^{i} \beta_{ij,p}}{s u}_{\vphi_M, D_p}\\
  %   &= \iinner{ \dbar s \sect_{(p)} \rho^{i} \beta_{ij,p}}{s \sect_{(p)} u}_{ \phi_{(p)}+\vphi_M, D_p}\\
  %   &= \iinner{\dbar \rho^{i} (\rs\lambda_j^p- \rs\lambda_i^p)}
  %   {s \sect_{(1)} u}_{\phi_{(p)}+\vphi_M, D_2}\\
  %   &= \iinner{-\dbar\paren{\rho^i \rs\lambda_i^p}}{s \sect_{(p)}
  %   v}_{\phi_{(p)}+\vphi_M, D_p}.
  % \end{align*}
  % Here we use that $s u$ is still harmonic to get the second equality 
  % and that $\dbar \rho^{i} \rs\lambda_j^p=\dbar \rs\lambda_j^p=0$ to get the third equality. 
  % The right-hand side can be described by the norm on $D_{1}\cap D_{2}$ as follows: 
  % \begin{align*} 
  %   &~- \iinner{\dbar\paren{\rho^i \rs\lambda_i^p}}{s \sect_{(p)} u}_{\phi_{(p)}+\vphi_M, D_p} \\
  %   \xleftarrow{\varepsilon \tendsto 0^+}
  %   &~-\iinner{e^{-\varepsilon \abs{\psi_{(p)}}}\dbar\paren{\rho^i \rs\lambda_i^p}}{s \sect_{(p)}
  %   u}_{\phi_{(p)}+\vphi_M, D_p} \\
  %   =
  %   &
  %   \begin{aligned}[t]
  %     &~-\cancelto{0}{
  %     \iinner{ \dbar\paren{e^{-\varepsilon
  %     \abs{\psi_{(p)}}} \rho^i \rs\lambda_i^p} }{ s \sect_{(p)} u}
  %   }_{\phi_{(p)}+\vphi_M, D_p} 
  %     +\varepsilon \iinner{
  %     e^{-\varepsilon \abs{\psi_{(p)}}} \rho^i \rs\lambda_{i}^p }{(\diff\psi_{(p)})^{*}s \sect_{(p)} u }_{\phi_{(p)}+\vphi_M, D_p}
  %   \end{aligned}
  %   \\
  %   =
  %   &~ \varepsilon \iinner{
  %   \frac{\rho^i \rs\lambda_{i}^p}{\sect_{(p)}}
  % }{
  %   e^{-\varepsilon \abs{\psi_{(p)}}}
  %   (\diff \log \abs{\sect_{(p)}^2})^{*}
  %   u \: s e^{-\vphi_M} }_{D_p}
  %   -\underbrace{
  %   \varepsilon \iinner{
  %   \frac{\rho^i \rs \lambda_{i}^p}{\sect_{(p)}}
  % }{
  %   e^{-\varepsilon \abs{\psi_{(p)}}}
  %   (\diff \sm\vphi_{(p)})^{*}  u \:s e^{-\vphi_M} }_{D_p}
  % }_{=\: \BigO(\varepsilon)}
  %   \\
  %   =
  %   &~\varepsilon \iinner{
  %   \rho^i \smash[b]{\underbrace{\rs\lambda_{i}^p}_{\mathclap{=\: d\sect_{(1)} 
  %   \wedge \rs\lambda_i^{12}}}}
  %   \:
  % }{ \:
  %   \frac{e^{-\varepsilon \abs{\psi_{(p)}}}}{\abs{\sect_{(p)}}^2}
  %   (d\sect_{(p)})^{*} \smash[b]{\underbrace{u_{p}}_{\mathclap{=:\: d\sect_{(1)}  \wedge \rs u_{p}^{12}}}} \: s e^{-\vphi_M} }_{D_p}
  %   + \BigO(\varepsilon)
  %   \vphantom{\underbrace{\rs\lambda_{i}^1}_{\mathclap{=\: d\sect_{(1)} 
  %   \wedge \rs\lambda_i^{12}}}}
  %   \\
  %   \xrightarrow{\varepsilon \tendsto 0^+}
  %   &~\iinner{\rho^i \rs\lambda_i^{12}}{  (d\sect_{(p)})^{*}  \rs u_{p}^{12}
  %   \: s e^{-\vphi_M}}_{D_1 \cap D_2}  \; .
  % \end{align*}

  % Considering the inner product above, we define the $F$-valued form $w$ on $D_{1} \cap D_{2}$ by 
  % \begin{equation*}
  %   w:=(d\sect_{(1)})^{*}  \rs u_{1}^{12}-(d\sect_{(2)})^{*} \rs u_{2}^{12}. 
  % \end{equation*}
  % From the next step, we aim to show $w$ is actually zero, which finishes the proof. 

  % Note that $u_{p}^{12}$ and $d\sect_{(p)}$ are defined only locally; 
  % hence $d\sect_{(p)})^{*}  \rs u_{p}^{12}$ does not determine a section on $D_{p}$, 
  % but determines the $F$-valued section form of type $(n-2, q-1)$ on $D_{1} \cap D_{2}$. 
  % This can be verified by calculating a glueing condition. 
  % Another way to see this is to apply The Poincar\'e residue map from $D_{p}$ to $D_1 \cap D_2$, 
  % which yields
  % \begin{equation*}
  %   \PRes[D_1 \cap D_2]( (\diff\psi_{(p)})^{*} u)
  %   =\parres{(d\sect_{(p)})^{*}  \rs u_{p}^{12}}_{D_1 \cap D_2}
  %   \quad\text{(recall that $\sect_{(p)} =\sect_{(p)} $)} \; .
  % \end{equation*}
  % Since $\psi_{(p)}=\phi_{(p)} -\sm\vphi_{(p)}$ is a global function, 
  % the right hand side is globally defined on $D_{p}$, and so is $((d\sect_{(p)})^{*}  \rs u_{p}^{12})|_{D_1 \cap D_2}$. 
\end{step}



\begin{step}[$w_b$ being holomorphic and thus {$w_b \in \cohgp 0[D_1
    \cap D_2]{K_{D_1 \cap D_2} \otimes F}$}]
  
  We prove that $\dbar w_b = 0$ on $\lcS|2|[b] :=D_1 \cap D_2$ by a
  direct computation given in Section \ref{subsec:harmonic}.
  Indeed, it suffices to show that each summand $\PRes[\lcS|2|[b] |
  D_p](\idxup{\diff\psi_{(p)}}. u_p)$ for $p=1,2$ in $w_b$ is
  $\dbar$-closed.
  The computations are identical, so it suffices to consider $p=1$.

  On an admissible open set $V$ such that $D_p  \cap V =\set{z_p =
    0}$ for $p=1,2$ and $\lcS|2|[b] \cap V = \set{z_1 = z_2 = 0} =
  D_1 \cap \set{z_2 = 0}$,
  we have
  \begin{equation*}
    \diff\psi_{(1)} =\frac{dz_2}{z_2} -\diff\sm\vphi_{(1)}
    \quad\text{ on } V \; .
  \end{equation*}
  By writing
  \begin{equation*}
    \idxup{dz_2}. u_1 =: dz_2 \wedge \paren{\idxup{dz_2}. \rs*u_{1,2}}
    \quad\text{ on } D_1 \cap V \; ,
  \end{equation*}
  where $\rs*u_{1,2}$ is a $(n-2,1)$-form on $D_1 \cap V$, we see
  that
  \begin{equation*}
    \PRes[\lcS|2|[b] | D_1](\idxup{\diff\psi_{(1)}}. u_1)
    =\PRes[\set{z_2 = 0}](\frac{\idxup{dz_2} .u_1}{z_2})
    =\parres{\idxup{dz_2}. \rs*u_{1,2}}_{\lcS|2|[b]}
    \quad\text{ on } D_1 \cap D_2 \cap V \; .
  \end{equation*}
  Therefore, it suffices to check that $\idxup{dz_2}. u_1$ is
  $\dbar$-closed on $D_1 \cap V$.
  As $u_1$ is harmonic and $\ibddbar\vphi_F \geq 0$, we have
  $\nabla^{(0,1)} u_1 = 0$ by Proposition
  \ref{prop:consequence-of-positivity} and Lemma
  \ref{lem:commutator-dbar-ctrt} yields the desired result (with $z_2$
  in place of $\vphi$ in the lemma).

  % In this step, we show that $w$ is a $F$-valued harmonic on $D_{1} \cap D_{2}$. 
  % Note that it is sufficient to show that $\dbar w =0$ in our case since the type of $w$ is $(n-2, q-1)=(n-2, 0)$ by $q=1$. 
  % By \cite[(1.9)]{Takegoshi_higher-direct-images} and \eqref{eq-harmonic}, we obtain that 
  % \begin{equation*}
  %   \dbar ( (d\sect_{(p)})^{*} u_{p}) 
  %   = \big( (i \partial  \dbar \sect_{(p)})^{*} - (\partial \sect_{(p)})^{*} \dbar + \partial \sect_{(p)}\nabla^{(0,1)} \big)u_{p}
  %   = 0 \quad\text{on   } D_1. 
  % \end{equation*}
  % By noting that $u_{p} = d \sect_{(1)}  \wedge \rs u_{p}^{12}$, 
  % we see that  $\dbar (d\sect_{(p)})^{*}  \rs u_{p}^{12} =0 $; hence $\dbar w =0$. 
  % In particular, $w$ determines the cohomology class $\{w \} \in H^{0}(D_{1}\cap D_{2}, K_{D_1 \cap D_2} \otimes F)$. 
\end{step}


\begin{step}[$w_b = 0$ and conclusion of the proof]
  \label{step:pf:use_u-ortho-w-simple}
We prove that $w_b =0$ using the assumption $(u_{1},u_{2}) \in
\paren{\ker \tau}^\perp$.
Consider the connecting morphism $\delta$ the long exact sequence 
\begin{equation*}
  \xymatrix@R=0.3cm@C=1.5em{
    {\to \cohgp 0[D_{1}\cap D_{2}]{K_{D_1 \cap D_2} \otimes F}} \ar[r]^-{\delta}
    &
    {\bigoplus_{p=1}^{2} \cohgp 1[D_{p}]{K_{D_p}\otimes F}} \ar[r]^-{\tau}  
    &
    {\cohgp 1[D]{K_D \otimes F}  \to} \; . 
  } 
\end{equation*}
Note that $\delta w_b \in \ker\tau$.

We compute $\delta w_b$ via the \v Cech--Dolbeault isomorphism.
% $\rho(w)$ in terms of \v Cech cohomology. 
Regard $w_b$ as a $0$-cocycle $\set{\rs \gamma_{b;\:i}}_{i \in I}$
given by $\rs \gamma_{b;\:i} :=\res{w_b}_{V_i}$.
Lift $\rs \gamma_{b;\:i}$ on $D_1 \cap D_2 \cap V_i$ to a section
$\gamma_i$ on $V_{i}$ via the isomorphism
$\frac{\holo_X}{\defidlof{D_1 \cap D_2}} = \faidlof|2|/|1|*
\xrightarrow[\isom]{\Res^2} \residlof|2|*$ such that
\begin{equation*}
  % \gamma_i = d\sect_{(2)}  \wedge d\sect_{(1)}  \wedge \rs \gamma_i \; .
  \Res^2\paren{\gamma_i}
  =\PRes[\lcS|2|[b]](\frac{\gamma_i}{\sect_D})
  =\frac{\rs*\gamma_{b;\:i}}{\sect_{(b)}} =\rs*\gamma_{b;\:i} \; .
\end{equation*}
Then $\delta w_b$ is represented by the $1$-cocycle
\begin{equation*}
  \delta\set{\gamma_i \bmod \defidlof{D_1 \cap D_2}}_{i \in I}
  =\set{(\delta  \gamma)_{ij} \bmod\defidlof{D}}_{i,j \in I}
  =\set{\gamma_{j} -\gamma_i  \bmod\defidlof{D}}_{i,j \in I} \; .
\end{equation*}
Note that $ \gamma_{j} -\gamma_i$ belongs to $\defidlof{D_1 \cap
  D_2}$, so $ \gamma_{j} -\gamma_i  \bmod\defidlof{D}$ can be realized
via the isomorphism $\frac{\defidlof{D_1 \cap D_2}}{\defidlof{D}}
=\faidlof|1|/|0|* \xrightarrow[\isom]{\Res^1} \residlof|1|*$ as
\begin{align*}
  \Res^1\paren{\gamma_{j} -\gamma_i}
  &=\paren{\PRes[D_1](\frac{\gamma_{j} -\gamma_i}{\sect_D})
    \: ,\:
    \PRes[D_2](\frac{\gamma_{j} -\gamma_i}{\sect_D})
    } \\
  &=\paren{
    \frac{(\delta \rs\gamma_1)_{ij}}{\sect_{(1)}}
    \: , \:
    \frac{(\delta \rs\gamma_2)_{ij}}{\sect_{(2)}}
    }
    \in K_{D_1} \otimes \res F_{D_1} \oplus K_{D_2} \otimes \res F_{D_2} \; ,
\end{align*}
in which $\rs*\gamma_{p;\:i} := \PRes[D_p](\frac{\gamma_i}{\sect_D})
\cdot \sect_{(p)}$ for $p = 1,2$.
Therefore, via the \v Cech--Dolbeault isomorphism on each $D_p$, 
the component of $\delta w_b$ on $D_p$ can be represented by (under
Einstein summation convention) 
\begin{equation*}
  -\dbar\rho^j \cdot \rho^i
  \frac{\paren{\delta\rs*\gamma_p}_{ij}}{\sect_{(p)}}
  =-\frac{\dbar\rho^j \cdot\rs*\gamma_{p;\:j}}{\sect_{(p)}}
  =: -\frac{\dbar v'_{p;(\infty)}}{\sect_{(p)}}
  % \paren{
  %   \res{\frac{\dbar\rho^i \:(\delta \gamma^1)_{ij}}{\sect_{(1)} }}_{D_1}
  %   \: , \:
  %   \res{\frac{\dbar\rho^i \: (\delta \gamma^2)_{ij}}{\sect_{(2)} }}_{D_2}
  % }
  % =\paren{
  %   -\res{\frac{\dbar\paren{\rho^i  \gamma^1_{i}}}{\sect_{(1)}}}_{D_1}
  %   \: , \:
  %   -\res{\frac{\dbar\paren{\rho^i  \gamma^2_{i}}}{\sect_{(2)}}}_{D_2}
  % } \; .
\end{equation*}
(the notation $v'_{p;(\infty)} :=\sum_{i\in I}\rho^i
\rs*\gamma_{p;\:i}$ is set for the consistency with the notation in
Proposition \ref{prop:res-formula-dbar-exact-dot-harmonic}).
% For the computation of the norm, 
% we take $\rs \gamma_i^p$ and $\rs\gamma_i^{12}$ such that 
% \begin{equation*}
% \gamma_{i} = d \sect_{(p)} \wedge \rs \gamma_i^p 
%   \quad\text{and}\quad 
%   \rs\gamma_i^{12} = \gamma_i. 
% \end{equation*}
Recall the sign convention chosen in Step
\ref{item:expression-of-su-simple} such that
\begin{equation*}
  \rs*\gamma_{b;\:i}
  = \PRes[\lcS|2|[b] | D_1](\frac{\rs*\gamma_{1;\:i}}{\sect_{(1)}})
  =- \PRes[\lcS|2|[b] |
  D_2](\frac{\rs*\gamma_{2;\:i}}{\sect_{(2)}}) \; .
\end{equation*}
Then, from $(u_{1},u_{2}) \in \paren{\ker\tau}^\perp$ and $\delta w_b
\in \ker\tau$, we obtain
\begin{align*}
  0
  &=
  \iinner{
    -\frac{\dbar v'_{1;(\infty)}}{\sect_{(1)}}
  }{u_1}_{D_1}
  +\iinner{
    -\frac{\dbar v'_{2;(\infty)}}{\sect_{(2)}}
  }{u_2}_{D_2}
  \\
  &=
    \iinner{
    -\dbar v'_{1;(\infty)}
    }{u_1 \sect_{(1)}}_{D_1, \phi_{(1)}}
    +\iinner{
    -\dbar v'_{2;(\infty)}
    }{u_2 \sect_{(2)}}_{D_2, \phi_{(2)}}
  \\
  &\overset{\mathclap{\text{Prop.~\ref{prop:res-formula-dbar-exact-dot-harmonic}}}}=
    \quad\;\;
    \iinner{\rho^i \rs*\gamma_{b;\:i} \:}{\:
    \PRes[\lcS|2|[b] | D_1](\idxup{\diff\psi_{(1)}}. u_1)
    -\PRes[\lcS|2|[b] | D_2](\idxup{\diff\psi_{(2)}}. u_2)
    }_{D_1\cap D_2}
  \\
  &=\iinner{w_b}{w_b}_{D_1 \cap D_2}
    =\norm{w_b}_{D_1 \cap D_2}^2
    \; .
\end{align*}
% By the same computation as in Step 2, 
% the right hand side can be described by the norm of $w$ as follows: 
% \begin{equation*}
%   0=\iinner{\rho^i  \gamma_i^{12} \:}{\:
%     (d\sect_{(1)})^{*}    u^{12} -(d\sect_{(2)})^{*}   v^{12}
%   }_{D_1 \cap D_2}
%   =\iinner{\rho^i \gamma_i}{w}_{D_1 \cap D_2}
%   = \norm w_{D_1 \cap D_2}^2. 
% \end{equation*}
This implies that $w_b=0$, finishing the proof for the case
$D=D_{1}+D_{2}$ and $q=1$. \qedhere
\end{step}
\end{proof}



\subsection{Remarks on the general case}
% \subsection{Strategy of the proof in the general case}
\label{subsec:n3}

There are two modifications to the proof in Section
\ref{sec:proof-of-simple-case} in order to handle the general case
worth mentioning here.
The first one is the replacement of the short exact sequence $0 \to
K_{D_1} \oplus K_{D_2} \to K_D \to K_{D_1 \cap D_2} \to 0$.
Take the case $D = D_1 + D_2 + D_3$, where $D_p = \set{z_p = 0}$ for
$p=1,2,3$ are the coordinate planes, for example.
Note that
\begin{equation*}
  \aidlof|3|* = \holo_X \;, \;\;
  \aidlof|2|* = \defidlof{D_1 \cap D_2 \cap D_3} \;, \;\;
  \aidlof|1|* = \smashoperator{\bigcap_{\substack{1 \leq p,q \leq 3 \\ p\neq q}}} \defidlof{D_p \cap D_q} \;
  \text{ and } \;
  \aidlof|0|* = \defidlof{D} 
\end{equation*}
in this case.
A natural choice of the short exact sequence to be considered is
\begin{equation*}
  \renewcommand{\objectstyle}{\displaystyle}
  \xymatrix@R=2.5em{
    0 \ar[r]
    &{K_X \otimes D \otimes \smash{\faidlof|1|/|0|*}} \ar[r]
    \ar[d]^(0.47){\Res^1}_(0.47){\isom}
    &{K_X \otimes D \otimes \smash{\faidlof|3|/|0|*}} \ar[r]
    \ar@{=}[d]
    &{K_X \otimes D \otimes \faidlof|3|/|1|*} \ar[r]
    &0 \; .
    \\
    &{\smash{\bigoplus_{p = 1}^3}\:K_{D_p}} \ar[r]
    &{K_D} 
    &
  }
\end{equation*}
In the previous case, we are taking advantage of the fact that the
$L^2$ Dolbeault isomorphism and the harmonic theory are valid on the
cohomology groups of the sheaves on both the left- and
right-hand-sides of the short exact sequence, so that the
corresponding injectivity statement can be proved on each side in
the spirit of Enoki, which in turn leads to the injectivity theorem
for the cohomology groups of the middle sheaf (twisted by $F$).
In the current case, they are valid only on the left-hand-side (on
each $D_p$).
We are thus led to determine whether the injectivity statement for
the sheaf on the right-hand-side holds true.
It is then apparent that we should consider
\begin{equation*}
  \renewcommand{\objectstyle}{\displaystyle}
  \xymatrix@R=2.5em{
    0 \ar[r]
    &{K_X \otimes D \otimes \smash{\faidlof|2|/|1|*}} \ar[r]
    \ar[d]_(0.47){\Res^2}^(0.47){\isom}
    &{K_X \otimes D \otimes \faidlof|3|/|1|*} \ar[r]
    &{K_X \otimes D \otimes \smash{\faidlof|3|/|2|*}} \ar[r]
    \ar[d]^-{\Res^3}_-{\isom}
    &0 \; ,
    \\
    &{\smash[t]{\bigoplus_{\substack{p,q = 1 \\ p\neq q}}^3} K_{D_p \cap D_q}} 
    &
    &{K_{D_1 \cap D_2 \cap D_3}}
  }
\end{equation*}
which, again, has the Dolbeault and harmonic theories valid on both
sides (on each lc center of $(X,D)$) of the short exact sequence.
The arguments in Section \ref{sec:proof-of-simple-case} can then be
employed to conclude the proof.
This illustrates the idea of the inductive arguments, which reduces
the question to the union of lower dimensional lc centers of $(X,D)$
in each step, to be employed in the general proof in Section
\ref{subsec:general}. 

Another modification to the proof in Section
\ref{sec:proof-of-simple-case} is that, when the claim in Theorem
\ref{thm:main} with $q > 1$ is considered, the section $w_b$
constructed as in \eqref{eq:w-prelim-formula} is then a $K_{\lcS+1[b]}
\otimes \res F_{\lcS+1[b]}$-valued $(0,q-1)$-form on some
$(\sigma+1)$-lc center $\lcS+1[b]$.
In order to prove that $w_b =0$ by following the arguments in the
previous case, we need not only to show that $w_b$ is
$\dbar$-closed, but also that it is harmonic.
This happens to be true and the computation for checking this claim
is given in Proposition \ref{prop:harmonic-residue} and Theorem
\ref{thm:residue-harmonic}.



% In this subsection, we consider how we should generalize the proof of the previous section in dealing with the general case. 

% We first consider the slightly more general case of $D=D_{1}+D_{2}$ and $q \geq 2$. 
% In this case,  we can repeat the same argument  for Step 1. 
% Step 2 is a bit more involved since we are dealing with differential forms $u_{p}$ of higher degree, 
% but essentially the same argument can be used to define $w$ appropriately (see $\eqref{eq-def-w}$). 
% To check that $w$ is harmnic in Step 3, 
% since $w$ is an $F$-valued of the type $(n-2, q-1) \not =(n-2, q-1)$, 
% we need to check $\dfadj w_q = 0$ as well as $\dbar w=0$. 
% Nevertheless, $\dbar w=0$ is proved by the same argument 
% and $\dfadj w_q = 0$ is proved in Subsection \ref{subsec:harmonic}. 
% Performing Step 4 in the same way, 
% some global section $v$ on $D_{1}\cap D_{2}$ naturally appears, 
% we finally obtain $0=\iinner{w-\dbar v }{w}_{D_{1}\cap D_{2}} $. 
% Although the point that $v$ appears is different, since $w$ is harmonic, the conclusion that $w=0$ is immediately obtained.
% As described above, in the case of $q \geq 2$, 
% the degree of the differential form is higher and more involved, 
% but essentially the same strategy still work. 


% Next, let us consider the case of $D=D_{1}+D_{2}+D_{3}$. 
% In the case of $D=D_{1}+D_{2}$ , 
% by using the exact sequence $0 \to K_{D_{1}} \oplus K_{D_{1}} \to K_{D} \to K_{D_{1} \cap D_{2} } \to 0$, 
% we proved the injectivity of the multiplication map on (the cohomology groups of) central term. 
% Of particular importance in the proof were that the left term admits the theory of harmonic integrals and 
% that the multiplication map on the right term in injective. 
% In this subsection, we explain what kind of exact sequences in the case where $D$ has three components  
% to make the same strategy works. 
% The precise proof will be given  in the next subsection. 

% Suppose that $D$ has three components (i.e.\,$D=D_{1}+D_{2}+D_{3}$). 
% We first consider the following exact sequence twisted by $F$ (and also $M$): 
% \begin{align}\label{eq-ex}
%   \xymatrix{
%     0 \ar[r]
%     & K_{X}\otimes D \otimes{\faidlof |1|/|0|*} =\bigoplus_{p=1}^{3} K_{D_{p}} \ar[r]
%     & K_{X}\otimes D \otimes{\faidlof|\sigma_{\mlc}|/|0|*} =K_{D}        \ar[r]
%     & K_{X}\otimes D \otimes{\faidlof|\sigma_{\mlc}|/|1|*} \ar[r]
%     & 0. 
%   } 
% \end{align}
% The cohomology classes in $\oplus_{p=1}^{3} H^{q}(D_{p}, K_{D_{p}} \otimes F)$ of the left term 
% can be represented by harmonic forms. 
% Hence, whether the same argument works as in the previous subsection 
% depends on whether or not the multiplication map 
% $$
% H^{q}(X, K_{X}\otimes{\faidlof|\sigma_{\mlc}|/|1|*}\otimes F) \xrightarrow{\quad \otimes s \quad }
% H^{q}(X, K_{X}\otimes{\faidlof|\sigma_{\mlc}|/|1|*}\otimes F\otimes M)
% $$
% is injectivity or not.  
% To check this, we  consider another exact sequence: 
% \begin{align*}
%    \xymatrix{
%     0 \ar[r]
%     & K_{X}\otimes{\faidlof |2|/|1|*}  =\bigoplus_{p\not = q} K_{D_{p} \cap D_{q}} \ar[r]
%     & K_{X}\otimes{\faidlof|\sigma_{\mlc}|/|1|*}         \ar[r]
%     & K_{X}\otimes{\faidlof|\sigma_{\mlc}|/|2|*}=K_{D_{1}\cap D_{2}\cap D_{3}} \ar[r]
%     & 0}
% \end{align*}
% Then, the left term admits the theory of harmonic integrals and 
% that the multiplication map on the right term in injective.
% Therefore, we can show that he multiplication map on the central term is injective. 
% No essential difficulty appears in repeating this inductive argument in the general case. 


\subsection{Proof of Theorem \ref{thm:main} in general}\label{subsec:general}


%\input{outline-of-proof}

%%%%%
%%%%% File name  : outline-of-proof.tex
%%%%% Author     : Mario Chan
%%%%% Date       : 6th March, 2023
%%%%% Description: This is the outline of the proof of the general
%%%%%              case of the project "Injectivity-Fujino".
%%%%%
%%
%%%

\renewcommand{\objectstyle}{\displaystyle}

Write
\begin{gather*}
  \aidlof* := \aidlof =\mtidlof{\vphi_F} \cdot \defidlof{\lcc+1'}
  =\defidlof{\lcc+1'} \; , \quad
  \residlof* := \residlof \isom \faidlof/-1*  \\
  \text{and } \quad \spH{\sheaf F}
  :=\cohgp q[X]{\logKX \otimes \sheaf F} 
\end{gather*}
for convenience.
Recall that
\begin{equation*}
  K_D = K_X \otimes D \otimes \faidlof|\sigma_{\mlc}|/|0|* \; ,
\end{equation*}
and the inclusions between adjoint ideal sheaves induce the short exact
sequences
\begin{equation*} % \label{eq-ex2}
  \xymatrix{
    0 \ar[r]
    & {\faidlof/|\rho|*} \ar[r]
    & {\faidlof|\tau|/|\rho|*} \ar[r]
    & {\faidlof|\tau|/*} \ar[r]
    & 0
  } \quad\text{ for } 0 \leq \rho \leq \sigma \leq \tau \; .
\end{equation*}
One is thus led to consider the commutative diagram
\subfile{commut-diagram_sing-Fujino-conj}%
for $\sigma =2,\dots,\sigma_{\mlc}$, in which the columns are exact,
$\iota_\sigma$ and $\tau_\sigma$ are induced from the inclusions
between adjoint ideal sheaves, and $\mu_\sigma$ (resp.~$\nu_\sigma$)
is the composition of $\iota_\sigma$ (resp.~$\tau_\sigma$) with the
map induced from the multiplication map $\otimes s$.
The statement in Theorem \ref{thm:main} is proved if one shows that
$\ker\mu_{\sigma_{\mlc}} = \ker\iota_{\sigma_{\mlc}} = 0$
($\iota_{\sigma_{\mlc}}$ is the identity map).
Note that $\mu_1 =\nu_1$ and $\iota_1 =\tau_1$.
Following the argument in \cite{Chan&Choi_injectivity-I}*{Thm.~1.3.2}, since
$\ker\mu_{\sigma-1} =\ker\iota_{\sigma-1}$ and $\ker\nu_\sigma
=\ker\tau_\sigma$ together imply $\ker\mu_{\sigma}
=\ker\iota_{\sigma}$ via a diagram-chasing argument, to prove Theorem
\ref{thm:main}, it suffices to show the following theorem.

\begin{thm} \label{thm:ker-nu=ker-tau}
  $\ker\nu_\sigma =\ker\tau_\sigma$ for all $\sigma =1, \dots, \sigma_{\mlc}$.
\end{thm}


\begin{remark} \label{rem:general-commut-diagram}
  When $\aidlof|0|*$ in the commutative diagram
  \eqref{eq:commut-diagram_sing-Fujino-conj} is replaced by $0$ (which
  can be considered as $\aidlof|-1|*$), the setup is reduced to the one in
  \cite{Chan&Choi_injectivity-I}*{Thm.~1.3.2}, which states that
  Theorem \ref{thm:ker-nu=ker-tau} together with the result in
  \cite{Matsumura_injectivity-lc}*{Thm.~1.6} or
  \cite{Chan&Choi_injectivity-I}*{Thm.~1.2.1} implies that Fujino's
  conjecture is true.
  As a matter of fact, the proof of Theorem \ref{thm:ker-nu=ker-tau}
  can also be adapted to the case $\sigma = 0$ (with $\aidlof|-1|* =
  0$ and $\residlof|0|* =D^{-1} \isom \defidlof{D} =\aidlof|0|*$) which recovers the result in
  \cite{Matsumura_injectivity-lc}*{Thm.~1.6} as well as
  \cite{Chan&Choi_injectivity-I}*{Thm.~1.2.1}.
  Furthermore, by replacing $\aidlof|0|*$ by $\aidlof|\sigma_0-1|*$
  and $\aidlof|\sigma_{\mlc}|*$ by $\aidlof|\sigma'|*$ for any $0 <
  \sigma_0 \leq \sigma'$ and letting $\sigma$ vary within the range
  $\sigma_0 < \sigma \leq \sigma'$ in the diagram
  \eqref{eq:commut-diagram_sing-Fujino-conj}, one sees that the proof
  of Theorem \ref{thm:ker-nu=ker-tau} guarantees the statement of
  Theorem \ref{thm:main} but with $K_D$ replaced by $K_X \otimes D
  \otimes \faidlof|\sigma'|/|\sigma_0 -1|*$.
\end{remark}


\begin{proof}
  The proof consists of the following steps.
  \begin{enumerate}[label=\textbf{Step \Roman*:}, ref=\Roman*,
    leftmargin=0pt, labelsep=*, widest=VI, itemindent=*, align=left,
    itemsep=1.5ex]
  \item Make use of the $L^2$ Dolbeault and harmonic theory available on
    $\spH{\residlof*}$.

    Write $\lcc' =\bigcup_{p \in \Iset} \lcS$ as the union of
    $\sigma$-lc centers $\lcS$ of $(X,D)$.  Notice that $\residlof*$,
    hence $\spH{\residlof*}$, has a decomposition as a direct sum
    which yields
    \begin{equation*}
      \spH{\residlof*}
      =\bigoplus_{p \in \Iset} \cohgp q[\lcS]{K_{\lcS}
        \otimes F \otimes \mtidlof<\lcS>{\vphi_F}}
      =\bigoplus_{p \in \Iset} \cohgp q[\lcS]{K_{\lcS} \otimes F}
    \end{equation*}
    such that the $L^2$ Dolbeault isomorphism and harmonic theory are
    valid for the cohomology group in each summand.
    Take the (squared) residue norm
    $\norm\cdot_{\lcc'}^2 = \sum_{p \in \Iset} \norm\cdot_{\lcS}^2$ as
    the $L^2$ norm on $\spH{\residlof*}$.  Pick any element
    $u := (u_p)_{p \in \Iset} \in \spH{\residlof*}$ such that
    \begin{itemize}
    \item each $u_p$ is a harmonic form on $\lcS$ with respect to the
      given norm $\norm\cdot_{\lcS}$ and
    
    \item $u \in \ker\nu_\sigma \cap \paren{\ker\tau_\sigma}^\perp$,
      where the orthogonal complement $\paren{\ker\tau_\sigma}^\perp$
      of $\ker\tau_\sigma$ is taken with respect to the residue norm
      $\norm\cdot_{\lcc'}$.
    \end{itemize}
    The theorem is proved if it is shown that $u_p = 0$ for all
    $p \in \Iset$.

  
  \item \label{item:express-su-in-residue-norm}
    Obtain an expression of $\norm{su}_{\lcc'}^2$ using the
    assumption $u \in \ker\nu_\sigma$ and the \v Cech--Dolbeault
    isomorphism.

    Let $\cvr V :=\set{V_i}_{i \in I}$ be a locally finite cover of
    $X$ by admissible open sets with respect to
    $(\vphi_F,\vphi_M,\psi_D)$ and let $\set{\rho^i}_{i \in I}$ be a
    partition of unity subordinate to $\cvr V$.
    Their notations are abused to mean also their induced cover and
    partition of unity on $\lcc'$ for any $\sigma \geq 0$.
    For any choice of indices $\idx 0,q \in I$, write $V_{\idx 0.q}
    :=V_{i_0} \cap V_{i_1} \dotsm \cap V_{i_q}$ as usual.
  
    Through the \v Cech--Dolbeault isomorphism, every (cohomology
    class of) $u_p$ is represented by a \v Cech $q$-cocycle
    $\set{\alpha_{p; \:\idx 0.q}}_{\idx 0,q \in I}$ such that (under
    the Einstein summation convention on the indices $\idx 0,q$)
    \begin{equation*}
      u_p
      % &= \dbar v_{p;(2)} +\dbar \rho^{i_{q-1}} \wedge \dotsm \wedge
      %   \dbar\rho^{i_0} \alpha_{p; \:\idx 0.q} \qquad\paren{\forall~ i_q \in I} \\
      \overset{\text{\eqref{eq:Cech-Dolbeault-isom}}}=
      \:\dbar v_{p;(2)}
      +(-1)^q \:\underbrace{\dbar \rho^{i_{q}} \wedge \dotsm \wedge
        \dbar\rho^{i_1} \cdot \rho^{i_0} }_{=: \:
        \paren{\dbar\rho}^{\idx q.0}} \alpha_{p; \:\idx 0.q} \; ,
    \end{equation*}
    where $v_{p; (2)}$ is a $K_{\lcS} \otimes \res{F}_{\lcS}$-valued $(0,q-1)$-form
    on $\lcS$ with $L^2$ coefficients with respect to
    $\norm\cdot_{\lcS}$ and
    $\alpha_{p; \:\idx 0.q} \in K_{\lcS} \otimes \res F_{\lcS} \otimes
    \mtidlof<\lcS>{\vphi_F} =K_{\lcS} \otimes \res F_{\lcS}$ on
    $V_{\idx 0.q}$.
    % (see \cite{Matsumura_injectivity}*{Prop.~5.5} or
    % \cite{Chan&Choi_injectivity-I}*{Lemma 3.2.1}). 
    In view of the
    residue short exact sequence, choose, for each choice of
    the multi-indices $(\idx 0,q)$, a section
    $f_{\idx 0,q} \in \logKX M \otimes \aidlof*$ on $V_{\idx 0.q}$
    such that
    \begin{equation*}
      \Res^\sigma(f_{\idx 0.q})
      =\paren{\alert{s} \alpha_{p; \:\idx 0.q}}_{p \in \Iset} 
    \end{equation*}
    (note that $V_{\idx 0.q}$ is Stein).  Considering the inclusion
    $\aidlof* \subset \aidlof|\sigma_{\mlc}|*$, write
    \begin{equation*}
      \eqcls{f_{\idx 0.q}} := \paren{f_{\idx 0.q} \bmod \aidlof-1*}
      \;\in \logKX M \otimes \faidlof|\sigma_{\mlc}|/-1*
      \quad\text{ on } V_{\idx 0.q} \; .
    \end{equation*}
    The collection $\set{\eqcls{f_{\idx 0.q}}}_{\idx 0,q \in I}$ is
    then a \v Cech $q$-cocycle representing $\nu_\sigma(u)$ in $\spH
    M{\faidlof|\sigma_{\mlc}|/-1*}$.
    The assumption $u \in \ker\nu_\sigma$ implies that this cocycle is
    a coboundary, that is,
    \begin{equation*}
      \set{\eqcls{f_{\idx 0.q}}}_{\idx 0,q \in I}
      =\delta\set{\eqcls{\lambda_{\idx 1.q}}}_{\idx 1,q \in I}
      =\set{\eqcls{\paren{\delta\lambda}_{\idx 0.q} }}_{\idx 0,q \in I}
    \end{equation*}
    for some $\lambda_{\idx 1.q} \in \logKX M \otimes
    \aidlof|\sigma_{\mlc}|*$ on $V_{\idx 1.q}$ (note that
    $\lambda_{\idx 1.q}$ need \emph{not} take values in $\aidlof*$
    even though the sections $f_{\idx 0.q}$ do), where
    $\paren{\delta\lambda}_{\idx 0.q}$ is given by the usual formula
    of \v Cech coboundary operator $\paren{\delta\lambda}_{\idx 0.q}
    :=\sum_{k =0}^q (-1)^k \lambda_{\idx 0[\dotsm \widehat{i_k}].q}$.
    Notice that $f_{\idx 0.q}$ and $\paren{\delta\lambda}_{\idx 0.q}$
    differ by an element in $\logKX M \otimes \aidlof-1*$ on $V_{\idx
      0.q}$.
    

    Thanks to the positivity $\ibddbar\vphi_F \geq 0$ and the bound
    $\ibddbar\vphi_M \leq C \ibddbar\vphi_F$ for some
    constant $C > 0$ on each $\lcS$, the product $s u_p$ is harmonic
    with respect to $\norm\cdot_{\lcS}$ (Proposition
    \ref{prop:consequence-of-positivity}), so $\iinner{s u_p}{s
      \:\dbar  v_{p;(2)}}_{\lcS} = \iinner{s u_p}{\dbar \paren{s
        v_{p;(2)}}}_{\lcS} = 0$ for every $p \in\Iset$.
    It follows that
    \begin{align*}
      \norm{su}_{\lcc'}^2
      =\sum_{p\in \Iset} \norm{su_p}_{\lcS}^2
      =&~(-1)^q \sum_{p\in \Iset} \sum_{\idx 0,q \in I} \iinner{\paren{\dbar\rho}^{\idx q.0}
        \:s \alpha_{p; \:\idx 0.q}}{\: s u_p}_{\lcS}
      \\
      =&~(-1)^q \sum_{p\in \Iset} \sum_{\idx 0,q \in I} \iinner{
         s \alpha_{p; \:\idx 0.q}
         }{\:\idxup{\diff\rho},[\idx 0.q].  s u_p}_{\lcS}
         \; ,
    \end{align*}
    where $\idxup{\diff\rho},[\idx 0.q].  \cdot $ is the adjoint
    of $\paren{\dbar\rho}^{\idx q.0} \cdot$.
    As in Step \ref{item:expression-of-su-simple} in Section
    \ref{sec:proof-of-simple-case}, the desired expression can be
    obtained by substituting
    \begin{equation*}
      s\alpha_{p; \:\idx 0.q} =\PRes[\lcS](\frac{f_{\idx
          0.q}}{\sect_D})
      =\PRes[\lcS](\frac{\paren{\delta\lambda}_{\idx
          0.q}}{\sect_D})
      =\frac{\paren{\delta \rs*\lambda_p}_{\idx 0.q}}{\sect_{(p)}}
      \; ,
    \end{equation*}
    where $\rs*\lambda_{p; \:\idx 1.q}
    :=\PRes[\lcS](\frac{\lambda_{\idx 1.q}}{\sect_D}) \cdot
    \sect_{(p)}$.
    For the sake of illustration, an alternative approach via a
    direct residue computation is presented here.
    Note that $\paren{\idxup{\diff\rho},[\idx 0.q].  s u_p}_{p \in
      \Iset} \in \logKX M \otimes \smooth_{X\:c\,*} \cdot\residlof*$ on $V_{\idx
      0.q}$, so it has a preimage $h^{\idx 0.q} \in \logKX M \otimes
    \smooth_{X\:c\,*} \cdot \aidlof*$ of $\Res^\sigma$ (considered as a
    $\smooth_{X\:c\,*}$-homomorphism).
    Fix such preimage on each open set $V_{\idx 0.q}$.
    From the direct computation of the residue function, it follows
    that
    \begin{align*}
      (-1)^q \:\norm{su}_{\lcc'}^2
      =&~\sum_{p\in \Iset} \sum_{\idx 0,q \in I} \iinner{
         s \alpha_{p; \:\idx 0.q}
         }{\:\idxup{\diff\rho},[\idx 0.q].  s u_p}_{\lcS}
      \\
      \xleftarrow{\eps \tendsto 0^+}
       &~\smashoperator[l]{\sum_{\idx 0,q \in I}} \eps
         \int_{\mathrlap{V_{\idx 0.q}}} \quad \frac{
         \inner{f_{\idx 0.q}}{h^{\idx 0.q}}
         \:e^{-\phi_D -\vphi_F -\vphi_M}
         }{\abs{\psi_D}^{\sigma +\eps}}
      \\
      \overset{\mathclap{\text{Prop.~\ref{prop:residue-product-X-to-lcS}}}}= \quad\;
       &~\smashoperator[l]{\sum_{\idx 0,q \in I}} \eps
         \int_{\mathrlap{V_{\idx 0.q}}} \quad \frac{
         \inner{\paren{\delta\lambda}_{\idx 0.q}}{h^{\idx 0.q}}
         \:e^{-\phi_D -\vphi_F -\vphi_M}
         }{\abs{\psi_D}^{\sigma +\eps}} +\BigO(\eps)
      \\
      \xrightarrow[\text{Prop.~\ref{prop:residue-product-X-to-lcS}}]{\eps \tendsto 0^+}
      %  &~\sum_{p\in \Iset} \sum_{\idx 0,q \in I}
      %    \int_{\lcS} \inner{
      %    \frac{\paren{\delta\rs*\lambda_p}_{\idx 0.q}}{ \sect_{(p)}}
      %    }{\:
      %    \idxup{\diff\rho},[\idx 0.q].  s u_p
      %    } \:e^{-\vphi_F-\vphi_M}
      % \\
      % =&~\sum_{p\in \Iset} \sum_{\idx 0,q \in I}
      %    \int_{\lcS} \inner{
      %    \paren{\dbar\rho}^{\idx q.0}
      %    \paren{\delta\rs*\lambda_p}_{\idx 0.q}
      %    }{\:
      %     s u_p \sect_{(p)}
      %    }_\omega \:e^{-\phi_{(p)}-\vphi_F-\vphi_M}
      % \\
      % =
       &~\sum_{p\in \Iset} \sum_{\idx 0,q \in I}
         \iinner{\paren{\dbar\rho}^{\idx q.0}
         \paren{\delta\rs*\lambda_p}_{\idx 0.q}}{\: s u_p
         \sect_{(p)}}_{\lcS, \phi_{(p)}}
      \\
      =&~\sum_{p\in \Iset} \sum_{\idx 1,q \in I}
         \iinner{\dbar\rho^{i_q} \wedge \dotsm \wedge \dbar\rho^{i_1}
         \cdot\rs*\lambda_{p;\:\idx 1.q}}{\: s u_p \sect_{(p)}}_{\lcS,
         \phi_{(p)}}
      \\
      =&~(-1)^{q-1} \sum_{p\in \Iset} \iinner{\dbar v_{p;(\infty)}}{\: s u_p
         \sect_{(p)}}_{\lcS, \phi_{(p)}}
         \; ,\footnotemark
    \end{align*}%
    \footnotetext{
      If the $L^2$ Dolbeault isomorphism is valid for $\spH
      M{\faidlof|\sigma_{\mlc}|/-1*}$, such conclusion can be
      obtained simply from the fact that $\nu_\sigma(su)$ is
      represented by a smooth $\dbar$-exact form on $\lcc'$.
    }%
    where
    % $\rs*\lambda_{p; \:\idx 1.q}
    % :=\PRes[\lcS](\frac{\lambda_{\idx 1.q}}{\sect_D}) \cdot \sect_{(p)}$ and 
    $v_{p; (\infty)} :=\sum_{\idx 1,q\in I} \dbar\rho^{i_q} \wedge \dotsm
    \wedge \dbar\rho^{i_2} \cdot \rho^{i_1} \rs*\lambda_{p;\:\idx 1.q}
    =\sum_{\idx 1,q\in I} \paren{\dbar\rho}^{\idx q.1}\rs*\lambda_{p; \:\idx 1.q}$. 

    % Note that $s u_p$ is harmonic with respect to the potential
    % $\vphi_F+\vphi_M$ by the positivity assumption.
    The expression of $\norm{su}_{\lcc'}^2$ can be further transformed
    by an integration by parts using Proposition
    \ref{prop:res-formula-dbar-exact-dot-harmonic}, which becomes
    \begin{equation*}
      \norm{su}_{\lcc'}^2
      % &=\smashoperator[l]{\sum_{\idx 1,q \in I}} \sum_{b \in \Iset+1}
      % \sum_{j=1}^{\sigma +1} (-1)^q \:\sigma
      % \iinner{ \sgn{b:p_{b,j}}\:
      % \frac{\rs*\lambda_{b;\:\idx 1.q}}{\sect_{(b)}}
      % }{\: s\:
      %   \idxup{\diff\rho},[\idx 1.q] .
      %   \PRes[b(j)](\idxup{\diff\psi_{(p_{b,j})}}.  u_{p_{b,j}})
      % }_{\lcS+1[b]}
      %   \\
      = \sigma\sum_{b \in \Iset+1}
      \iinner{v_{b;(\infty)} \:
      }{ \quad s \:
        \smashoperator{\sum_{p \in \Iset \colon \lcS+1[b] \subset
            \lcS}} \;\; \sgn{b:p}\:
        \PRes[\lcS+1[b] | \lcS](\idxup{\diff\psi_{(p)}}.  u_{p})
        \cdot \sect_{(b)}
      }_{\lcS+1[b], \phi_{(b)}} \; ,
    \end{equation*}
    where $v_{b;(\infty)} := \sum_{\idx 1,q \in I}
    \paren{\dbar\rho}^{\idx q.1} \rs*\lambda_{b; \:\idx 1.q}$ and
    $\rs*\lambda_{b; \:\idx 1.q}
    :=\PRes[\lcS+1[b]](\frac{\lambda_{\idx 1.q}}{\sect_D}) \cdot \sect_{(b)}$.
    % However, notice that $v_{p;(\infty)}$ is smooth on $\lcS$ but not
    % locally $L^2$ with respect to the weight $e^{-\phi_{(p)}}$.
    % For this reason, let $\psi_{(p)} :=\phi_{(p)} -\sm\vphi_{(p)}$, where
    % $\sm\vphi_{(p)}$ is some smooth potential on $\Diff_p D$.
    % One then has
    % \begin{align*}
    %   &~\norm{su}_{\lcc'}^2
    %   =\sum_{p\in \Iset} \iinner{\dbar v_{p;(\infty)}}{\: s u_p
    %      \sect_{(p)}}_{\lcS, \phi_{(p)}}
    %   \\
    %   \xleftarrow{\eps \tendsto 0^+}
    %    &~\sum_{p \in \Iset} \iinner{
    %      e^{-\eps \abs{\psi_{(p)}}} \:\dbar v_{p;(\infty)}
    %      }{\: s u_p \sect_{(p)}}_{\lcS, \phi_{(p)}}
    %   \\
    %   =&~\sum_{p \in \Iset} \paren{
    %      \cancelto{0}{\iinner{
    %      \dbar\paren{e^{-\eps \abs{\psi_{(p)}}} \: v_{p;(\infty)}}
    %      }{\: s u_p \sect_{(p)}}_{\mathrlap{\lcS, \phi_{(p)}}}}
    %      \quad\;\; + \eps 
    %      \iinner{
    %      e^{-\eps \abs{\psi_{(p)}}} \:v_{p;(\infty)}
    %      }{\:\idxup{\diff\psi_{(p)}} . s u_p \sect_{(p)}}_{\lcS,
    %      \phi_{(p)}}
    %      }
    %   \\
    %   =&~\sum_{p \in \Iset} \sum_{\idx 1,q \in I} (-1)^q \:\eps \:
    %      \iinner{
    %      e^{-\eps \abs{\psi_{(p)}}} \: % \paren{\dbar\rho}^{\idx q.1}
    %      \rs*\lambda_{p;\:\idx 1.q}
    %      }{\:
    %      \idxup{\diff\rho},[\idx 1.q] .
    %      \paren{\idxup{\diff\psi_{(p)}} . s u_p \sect_{(p)}}
    %      }_{\lcS, \phi_{(p)}}
    %   \\
    %   \xrightarrow[\text{Prop.~\ref{prop:residue-formula-classical-kernel}}]{\eps
    %   \tendsto 0^+} 
    %   &~\smashoperator[l]{\sum_{\idx 1,q \in I}} \sum_{p \in \Iset}
    %     \sum_{k=\sigma +1}^{\mathclap{\sigma_{V_{\idx 1.q}}}} (-1)^q \:\sigma
    %     \iinner{
    %     \PRes[p(k)](
    %     \frac{\rs*\lambda_{p;\:\idx 1.q}}{\sect_{(p)}}
    %     )
    %      }{\:
    %      \idxup{\diff\rho},[\idx 1.q] .
    %      \PRes[p(k)](\idxup{\diff\psi_{(p)}} . s u_p)
    %      }_{\lcS \cap \set{z_{p(k)} =0}}
    %   \; ,
    % \end{align*}
    % where $\PRes[p(k)]$ denotes the Poincar\UTF{00E9} residue map from $\lcS$
    % to $\lcS \cap \set{z_{p(k)}=0}$. 
    % The last limit is justified as follows.
    % On the admissible open set $V_{\idx 1,q}$, consider a holomorphic
    % coordinate system $(z_1, \dots, z_n)$ such that $\lcS
    % =\set{z_{p(1)} = \dotsm =z_{p(\sigma)} =0}$ and
    % $\sect_{(p)} =z_{p(\sigma+1)} \dotsm z_{p(\sigma_V)}$ (write
    % $\sigma_{V}$ for $\sigma_{V_{\idx 1.q}}$ for convenience).
    % Note that
    % \begin{equation*}
    %   \diff\psi_{(p)} =\sum_{k =\sigma +1}^{\sigma_V}
    %   \frac{dz_{p(k)}}{z_{p(k)}} -\diff\sm\vphi_{(p)} \quad\text{ on }
    %   V_{\idx 1,q} \; .
    % \end{equation*}
    % It follows that, on $\lcS \cap V_{\idx 1.q}$,
    % \begin{equation*}
    %   \begin{multlined}
    %     \text{coef.~of }\:
    %     \idxup{\diff\rho},[ \idx 1.q] .
    %     \paren{\idxup{\diff\psi_{(p)}} . s u_p \sect_{(p)}}
    %   \end{multlined}
    %   \in
    %   \res{\defidlof{\lcc+2'}}_{\lcS}
    %   \begin{aligned}[t]
    %     &=\mtidlof<\lcS>{\vphi_F+\vphi_M} \cdot
    %     \res{\defidlof{\lcc+2'}}_{\lcS} \;\;\footnotemark
    %     \\
    %     &=\aidlof|1|<\lcS>{\vphi_F+\vphi_M}[\psi_{(p)}]
    %   \end{aligned}
    % \end{equation*}%
    % \footnotetext{
    %   Recall that $\defidlof{\lcc+2'}$ is generated on $X$ by
    %   $\sect_{(\sigma+1 : b)}$ for all $b \in \Iset+1$ treated as local
    %   functions.
    %   On an admissible open set $V$, one has $\defidlof{\lcc+2'}
    %   =\genbyd{z_{b(\sigma+2)} \dotsm
    %     z_{b(\sigma_V)}}{b \in \Iset+1 \text{ such that } \lcS+1[b] \cap
    %     V \neq \emptyset}$ (see page
    %   \pageref{page:notation-permutation-index} for the notation).
    % }%
    % and, therefore, one can apply Proposition
    % \ref{prop:residue-formula-classical-kernel} (with $\lcS$ in place
    % of $X$, $\psi_{(p)}$ in place of $\psi_D$) to each inner product
    % $\eps \iinner{\dotsm}{ \:\dotsm \idxup{\diff\psi_{(p)}}. \dotsm
    %   \sect_{(p)}}_{\lcS,\phi_{(p)}}$.
    % % to obtain a sum of integrals on
    % % each $\lcS \cap \set{z_{p(k)} = 0}$ for $k=\sigma +1, \dots,
    % % \sigma_V$, i.e.~on the $(\sigma+1)$-lc centers in $\lcc+1' \cap
    % % V_{\idx 1.q}$.
    
    % Write $\lcc+1' =\bigcup_{b \in \Iset+1} \lcS+1[b]$.
    % On each admissible open set $V_{\idx 1.q}$, the intersection $\lcS
    % \cap \set{z_{p(k)} = 0}$ is a $(\sigma+1)$-lc center $\lcS+1[b_{p,k}]
    % \cap V_{\idx 1.q}$ ($\neq \emptyset$), uniquely determined by the
    % choices of $p\in \Iset$ (such that $\lcS \cap V_{\idx 1.q} \neq
    % \emptyset$, so $\binom{\sigma_V}{\sigma}$ choices) and $k
    % =\sigma+1, \dots, \sigma_V$ (so $\sigma_V-\sigma$ choices).
    % To get an indexing in terms of $b \in \Iset+1$ (such that
    % $\lcS+1[b] \cap V_{\idx 1.q} \neq \emptyset$, so
    % $\binom{\sigma_V}{\sigma +1}$ choices), note that each $\lcS+1[b]
    % \cap V_{\idx 1.q}$ is contained in $\sigma +1$ distinct
    % $\sigma$-lc centers $\lcS[p_{b,j}]$ for $j=1,\dots,\sigma+1$
    % (apparently, $\sigma +1$ choices) such that
    % \begin{equation*}
    %   \lcS+1[b] \cap V_{\idx 1.q} = \lcS[p_{b,j}] \cap \set{z_{b(j)} = 0} \; .
    % \end{equation*}
    % (One can verify $\sum_{p \in \Iset} \sum_{k=\sigma
    %   +1}^{\sigma_{V}} \dotsm = \sum_{b \in
    %   \Iset+1} \sum_{j=1}^{\sigma +1} \dotsm$ by first noting that
    % $\binom{\sigma_V}{\sigma} (\sigma_V -\sigma)
    % =\binom{\sigma_V}{\sigma +1} (\sigma+1)$.)
    % With such choice of indexing, let $\sgn{b:p_{b,j}}$ be the sign
    % given by
    % \begin{equation*}
    %   \PRes[\lcS+1[b]]
    %   =\sgn{b:p_{b,j}} \:\PRes[b(j)]\circ \PRes[\lcS[p_{b,j}]] \; .
    % \end{equation*}
    % Therefore, one has
    % \begin{equation*}
    %   \frac{\rs*\lambda_{b;\: \idx 1.q}}{\sect_{(b)}}
    %   :=\PRes[\lcS+1[b]](\frac{\lambda_{\idx 1.q}}{\sect_D})
    %   % =\sgn{b:p_{b,j}} \:\PRes[b(j)]\circ
    %   % \PRes[\lcS[p_{b,j}]](\frac{\lambda_{\idx 1.q}}{\sect_D})
    %   =\sgn{b:p_{b,j}} \:
    %   \PRes[b(j)](\frac{\rs*\lambda_{p_{b,j};\:\idx
    %       1.q}}{\sect_{(p_{b,j})}})
    % \end{equation*}
    % (recalling that $\sect_{(b)} =\sect_{(\sigma+1 : b)}$,
    % $\sect_{(p_{b,j})} =\sect_{(\sigma : p_{b,j})}$ and
    % $\sect_{(p_{b,j})} = z_{b(j)} \sect_{(b)}$).
    
    Set
    \begin{equation}\label{eq-def-w}
      w_b := \smashoperator[r]{\sum_{p \in \Iset \colon \lcS+1[b] \subset
          \lcS}} \;\; \sgn{b:p}\:
      \PRes[\lcS+1[b] | \lcS](\idxup{\diff\psi_{(p)}} . u_{p})
      \; .
    \end{equation}
    It suffices to show that $w_b = 0$ on $\lcS+1[b]$ for each $b
    \in\Iset+1$ to conclude the proof.
    

  \item Show that $w_b$ is harmonic with respect to
    $\res{\vphi_F}_{\lcS+1[b]}$ (and $\res{\omega}_{\lcS+1[b]}$) on
    $\lcS+1[b]$ for all $b \in \Iset+1$ and thus $\paren{w_b}_{b
      \in\Iset+1}$ represents a class in $\spH/q-1/{\residlof+1*}$.

    {
      \newcommand{\lcSb}{\lcS+1[b]}
      % \newcommand{\idxj}{\idx[\conj j]}
      
      To see that $w_b$ is $\dbar$-closed on $\lcSb$, it suffices to
      show that $\PRes[\lcS+1[b] | \lcS](\idxup{\diff\psi_{(p)}}. 
      u_{p})$ is $\dbar$-closed for all $p\in\Iset$ such that $\lcSb
      \subset \lcS$.
      Take any admissible open set $V$ such that $V \cap \lcSb
      \neq\emptyset$ and a holomorphic coordinate system such that
      $\sect_{(p)} = z_{p(\sigma+1)} \dotsm z_{p(\sigma_V)}$ on $V$.
      Suppose $\lcSb \cap V = \lcS \cap \set{z_{p(k)} = 0}$ for some $k
      =\sigma +1, \dots, \sigma_V$.
      Recall that
      \begin{equation*}
        \diff\psi_{(p)} = \sum_{k'=\sigma+1}^{\sigma_V}
        \frac{dz_{p(k')}}{z_{p(k')}} - \diff\sm\vphi_{(p)} \quad\text{
          on } V \; .
      \end{equation*}
      By writing
      \begin{equation*}
        \idxup{dz_{p(k)}}.  u_p =: dz_{p(k)} \wedge
        \paren{\idxup{dz_{p(k)}}.  \rs u_{p,k}} \quad\text{ on }
        \lcS \cap V \; ,
      \end{equation*}
      in which $\rs u_{p,k}$ is a $(n-\sigma-1,q)$-form on $\lcS \cap
      V$, it follows that
      \begin{equation*}
        \PRes[\lcS+1[b] | \lcS](\idxup{\diff\psi_{(p)}}.  u_{p})
        =\PRes[\set{z_{p(k)}=0}](\frac{\idxup{dz_{p(k)}}. 
          u_p}{z_{p(k)}})
        =\parres{\idxup{dz_{p(k)}} . \rs u_{p,k}}_{\lcSb}
        \quad\text{ on } \lcSb \cap V \; .
      \end{equation*}
      It thus suffices to show that $\idxup{dz_{p(k)}}.  u_p$ is
      $\dbar$-closed on $\lcS \cap V$.
      % , which is guaranteed by Lemma
      % \ref{lem:commutator-dbar-ctrt}.
      Since $u_p$ is harmonic and $\ibddbar\vphi_F \geq 0$, it follows
      that $\dbar u_p = 0$ and $\nabla^{(0,1)}u_p = 0$ (Proposition
      \ref{prop:consequence-of-positivity}).
      Putting $u_p$ into $u$ and $z_{p(k)}$ into $\vphi$ in Lemma
      \ref{lem:commutator-dbar-ctrt}, one has
      $\dbar\paren{\idxup{dz_{p(k)}}.  u_p} = 0$ on $\lcS \cap V$.
      As a result, $w_b$ is $\dbar$-closed on $\lcSb$ and
      $\paren{w_b}_{b\in\Iset+1}$ therefore represents a class in
      $\spH/q-1/{\residlof+1*}$.
      % This can be done via the following formula, which is a special
      % case and a slight variant of \cite{Donnelly&Xavier}*{(2.4)} and
      % \cite{Ohsawa&Takegoshi-spectral_seq}*{Prop.~1.5} (see also
      % \cite{Takegoshi_higher-direct-images}*{(1.9)} and
      % \cite{Matsumura_injectivity-Kaehler}*{Lemma 2.1}).
      
      % \begin{lemma}[cf.~\cite{Donnelly&Xavier}*{(2.4)},
      %   \cite{Ohsawa&Takegoshi-spectral_seq}*{Prop.~1.5},
      %   \cite{Takegoshi_higher-direct-images}*{(1.9)} and
      %   \cite{Matsumura_injectivity-Kaehler}*{Lemma
      %     2.1}] \label{lem:commutator-dbar-ctrt}
      %   Let $\vphi$ be a smooth function and $u$ be a smooth
      %   ($K_X$-valued) $(0,q)$-form on a K\"ahler manifold.
      %   They satisfy the formula
      %   \begin{equation*}
      %     \dbar\paren{\idxup{\diff\vphi}.  u}
      %     =\mhlight{\idxup{\ibddbar\vphi}.  u}
      %     -\idxup{\diff\vphi} . \paren{\dbar u}
      %     +\idxup{\diff\vphi} \cdot \nabla^{(0,1)}_\bullet u \; ,
      %   \end{equation*}%
      %   % \mariocomment{I treat $u$ as a $(0,q)$-form, so it doesn't
      %   %   make sense to apply $\Lambda_\omega$ to $u$.
      %   %   Do you accept this or we have to treat sections of $K_X$ as
      %   %   $(n,0)$-forms?
      %   %   I don't recommend the latter choice since it's easier to say
      %   %   and understand a ``$K_{\lcS}$-valued section'' than a
      %   %   ``$(n-\sigma,\bullet)$-form''.
      %   % }%
      %   \mariocomment[Red]{SM is treating $u$ as an $F$-valued $(n-\sigma, \bullet)$-form in my head. 
      %     I understand that the reason why the vanishing theorem holds for the adjoint bundle $K \times F$ 
      %     comes from the special property of $(n,q)$-type, so this is
      %     more natural for me.}%
      %   \mmark*{}{MC: OK. Since formula in Sec.
      %       \ref{subsec:n2} Step 3 also goes without $\Lambda_\omega$,
      %     I feel safe to keep things as they are. BTW, I prefer to
      %     keep this proof after I compare the type of $\diff\vphi$ and
      %   $\dbar\Phi$ in the referred papers.}%
      %   or, when a local holomorphic coordinate system is fixed and
      %   the Einstein summation convention is applied, 
      %   \begin{equation*}
      %     \paren{\dbar\paren{\idxup{\diff\vphi}.  u}}_{\conj J_{q}}
      %     =\sum_{\nu=1}^q \diff^{\conj\ell} \diff_{\conj j_\nu} \vphi \:
      %     u_{\idxj 1[\dotsm (\conj \ell)_\nu].q}
      %     -\diff^{\conj\ell}\vphi  \:\paren{\dbar u}_{\conj\ell\conj J_q}
      %     +\diff^{\conj\ell}\vphi \:\nabla_{\conj\ell} u_{\conj J_q} 
      %   \end{equation*}
      %   for any multi-indices $J_q = (\idx[j]1,q)$, pointwisely.
      % \end{lemma}

      % \begin{proof}
      %   A direct computation yields
      %   \begin{align*}
      %     \paren{\dbar\paren{\idxup{\diff\vphi} . u}}_{\conj
      %     J_{q}}
      %     &=\sum_{\nu=1}^q (-1)^{\nu-1} \diff_{\conj j_\nu}
      %       \paren{\idxup{\diff\vphi}.  u}_{\idxj 1[\dotsm \widehat
      %       {\conj j}_\nu].q}
      %       =\sum_{\nu=1}^q (-1)^{\nu-1} \diff_{\conj j_\nu}
      %       \paren{\diff_{\ell}\vphi \: u^\ell_{\;\idxj 1[\dotsm
      %       \widehat{\conj j}_\nu].q}}
      %     \\
      %     &=\sum_{\nu=1}^q (-1)^{\nu-1} \paren{
      %       \diff_{\conj j_\nu}\diff_{\ell}\vphi \: u^{\ell}_{\;\idxj 1[\dotsm
      %       \widehat {\conj j}_\nu].q}
      %       +\diff_{\ell}\vphi \: \nabla_{\conj j_\nu} u^\ell_{\;\idxj 1[\dotsm
      %       \widehat {\conj j}_\nu].q}
      %       }
      %     \\
      %     &=\sum_{\nu=1}^q
      %       \diff^{\conj \ell}\diff_{\conj j_\nu}\vphi \: u_{\idxj 1[\dotsm
      %       (\conj\ell)_\nu].q}
      %       -\diff^{\conj\ell}\vphi \sum_{\nu=1}^q (-1)^{\nu} 
      %       \nabla_{\conj j_\nu} u_{\conj\ell \idxj 1[\dotsm
      %       \widehat {\conj j}_\nu].q}
      %       \begin{aligned}[t]
      %         &-\diff^{\conj\ell}\vphi \: \nabla_{\conj \ell} u_{\conj
      %           J_q} \\
      %         &+\diff^{\conj\ell}\vphi \: \nabla_{\conj \ell}
      %         u_{\conj J_q}
      %       \end{aligned}
      %     \\
      %     &=\sum_{\nu=1}^q
      %       \diff^{\conj \ell}\diff_{\conj j_\nu}\vphi \: u_{\idxj 1[\dotsm
      %       (\conj\ell)_\nu].q}
      %       -\diff^{\conj\ell}\vphi
      %       \:\paren{\dbar u}_{\conj\ell\conj J_q}
      %       +\diff^{\conj\ell}\vphi \: \nabla_{\conj \ell} u_{\conj
      %       J_q} \; ,
      %   \end{align*}
      %   as desired.
      % \end{proof}


    }

    Furthermore, by Proposition \ref{prop:harmonic-residue} and
    Theorem \ref{thm:residue-harmonic} (with
    $\lcS$ in place of $X$, $\lcS+1[b]$ in place of $D_p$ and
    $\psi_{(p)}$ in place of $\psi_{D_p}$), $w_b$ is a
    $K_{\lcS+1[b]} \otimes \res{F}_{\lcS+1[b]}$-valued $(0,q-1)$-form on
    $\lcS+1[b]$ (not only a $\res{\conj\holoform_X^{q-1}}_{\lcS+1[b]}$-valued
    section) which is harmonic with respect to
    $\res{\vphi_F}_{\lcS+1[b]}$.
    % Therefore, $\paren{w_b}_{b\in\Iset+1}$ represents a class in
    % $\spH/q-1/{\residlof+1*}$. 



  \item \label{item:pf:use_u-ortho-w}
    Apply the assumption $u =(u_p)_{p\in\Iset} \in
    \paren{\ker\tau_\sigma}^{\perp}$ via the use of $w
    :=\paren{w_b}_{b \in \Iset+1} \in \spH/q-1/{\residlof+1*}
    =\bigoplus_{b \in\Iset+1} \cohgp{q-1}[\lcS+1[b]]{K_{\lcS+1[b]}
      \otimes F}$ in view of the commutative diagram
    \begin{equation*}
      \xymatrix@R-0.3cm{
        {\dotsm} \ar[r]
        & {\spH/q-1/{\residlof+1*}} \ar[r]^-{\delta}
        \ar[d]^-{\tau_{\sigma+1}}
        & {\spH{\residlof*}} \ar[r]
        \ar@{=}[d]
        & {\spH{\faidlof+1/-1*}} \ar[r]
        \ar[d]
        & {\dotsm}
        \\
        {\dotsm} \ar[r]
        & {\spH/q-1/{\faidlof|\sigma_{\mlc}|*}} \ar[r]
        & {\spH{\residlof*}} \ar[r]^-{\tau_\sigma}
        & {\spH{\faidlof|\sigma_{\mlc}|/-1*}} \ar[r]
        & {\dotsm}
      }
    \end{equation*}
    and conclude that $u_p = 0$ on $\lcS$ for each $p\in\Iset$.

    From the commutative diagram, one sees that $\delta w \in
    \ker\tau_\sigma$.
    In view of the isomorphism $\residlof+1* \isom \faidlof+1*$ and by
    following the procedures in Step
    \ref{item:express-su-in-residue-norm}, one obtains
    a $\logKX \otimes \aidlof+1*$-valued \v Cech $(q-1)$-cochain
    $\set{\gamma_{\idx 1.q}}_{\idx 1,q \in I}$ with respect to $\cvr
    V$ such that, when
    \begin{equation*}
      \paren{\alpha'_{b; \:\idx 1.q}}_{b\in\Iset+1}
      :=\paren{\frac{\rs*\gamma_{b; \:\idx
            1.q}}{\sect_{(b)}}}_{\mathrlap{b\in\Iset+1}} \quad\;
      :=\Res^{\sigma+1}(\gamma_{\idx 1.q})
      \in \smashoperator[r]{\prod_{b\in\Iset+1}} K_{\lcS+1[b]} \otimes
      \res{F}_{\lcS+1[b]} \paren{\lcS+1[b] \cap V_{\idx 1.q}}
    \end{equation*}
    (notation chosen for the consistency with those in Proposition
    \ref{prop:res-formula-dbar-exact-dot-harmonic}) and
    \begin{equation*}
      \eqcls{\gamma_{\idx 1.q}}
      := \paren{\gamma_{\idx 1.q} \bmod \aidlof*} \in \logKX \otimes
      \faidlof+1* \quad\text{ on } V_{\idx 1.q} \; ,
    \end{equation*}
    the collection $\set{\alpha'_{b;\:\idx 1.q}}_{\idx 1,q\in I}$
    is a \v Cech \emph{$(q-1)$-cocycle} representing (the class of)
    $w_b$ in $\cohgp{q-1}[\lcS+1[b]]{K_{\lcS+1[b]} \otimes F}$ for
    each $b \in\Iset+1$ such that
    \begin{equation*}
      w_b = \dbar v'_{b;(2)} +(-1)^{q-1} \:\underbrace{
        \dbar \rho^{i_{q}} \wedge \dotsm \wedge
        \dbar\rho^{i_2} \cdot \rho^{i_1} }_{=: \:
        \paren{\dbar\rho}^{\idx q.1}} \alpha'_{b;\:\idx 1.q}
      =: \dbar v'_{b;(2)} +(-1)^{q-1} \frac{v_{b;(\infty)}'}{\sect_{(b)}}
    \end{equation*}
    (again, notation chosen for the consistency with those in Proposition
    \ref{prop:res-formula-dbar-exact-dot-harmonic})
    for some $K_{\lcS+1[b]} \otimes \res{F}_{\lcS+1[b]}$-valued
    $(0,q-2)$-form $v'_{b;(2)}$ on $\lcS+1[b]$ with $L^2$ coefficients
    with respect to $\norm\cdot_{\lcS+1[b]}$, and the collection
    $\set{\eqcls{\gamma_{\idx 1.q}}}_{\idx 1,q \in I}$ is a \v Cech
    \emph{$(q-1)$-cocycle} representing (the class of) $w$ in
    $\spH/q-1/{\faidlof+1*} \xrightarrow[\isom]{\Res^{\sigma+1}}
    \spH/q-1/{\residlof+1*}$.
    The image $\delta w$ in $\spH{\residlof*}$ is then represented by
    \begin{equation*}
      \set{\Res^\sigma\paren{\paren{\delta\gamma}_{\idx 0.q}}}_{\idx 0,q \in
      I} \; ,
    \end{equation*}
    in which $\delta$ is the \v Cech boundary operator.
    Note that applying $\Res^\sigma$ to $\paren{\delta\gamma}_{\idx
      0.q}$ is valid as $\set{\eqcls{\gamma_{\idx 1.q}}}_{\idx 1,q \in
      I}$ is a cocycle and thus coefficients of
    $\paren{\delta\gamma}_{\idx 0.q}$ lie in $\aidlof*$.
    Set
    \begin{equation*}
      \rs*\gamma_{p;\:\idx 1.q} := \PRes[\lcS](\frac{\gamma_{\idx
            1.q}}{\sect_D}) \cdot \sect_{(p)} 
    \end{equation*}
    such that
    \begin{equation*}
      \Res^\sigma\paren{\paren{\delta\gamma}_{\idx 0.q}}
      =\paren{\frac{\paren{\delta\rs*\gamma_p}_{\idx
            0.q}}{\sect_{(p)}}}_{p \in \Iset}
      \in \prod_{p \in \Iset} K_{\lcS} \otimes \res F_{\lcS}
      \paren{\lcS \cap V_{\idx 0.q}} \; .
    \end{equation*}
    Note that 
    \begin{equation*}
      (-1)^q \paren{\dbar\rho}^{\idx q.0}
      \frac{\paren{\delta \rs*\gamma_p}_{\idx 0.q}}{\sect_{(p)}}
      % =(-1)^q \paren{\dbar\rho}^{\idx q.1}
      % \frac{\rs*\gamma_{p;\:\idx 1.q}}{\sect_{(p)}}
      =-
      \frac{\dbar\paren{\paren{\dbar\rho}^{\idx q.1} \rs*\gamma_{p;\:\idx 1.q}}}{\sect_{(p)}}
      =: - \:\frac{\dbar v_{p; (\infty)}'}{\sect_{(p)}}
      \quad\text{ on } \lcS
    \end{equation*}
    is a $\dbar$-closed form representing the class of $\res{\delta
      w}_{\lcS}$ (the component of $\delta w$ on $\lcS$) in $\cohgp
    q[\lcS]{K_{\lcS} \otimes F}$ via Dolbeault isomorphism.
    
    Therefore, from the assumption $u \in
    \paren{\ker\tau_\sigma}^\perp$ and taking into account the \v
    Cech--Dolbeault isomorphism \eqref{eq:Cech-Dolbeault-isom} and the
    fact that each $u_p$ is harmonic, one has
    \begin{align*}
      0 =\iinner{(-1)^{q-1} \delta w}{u}_{\lcc'}
      &=(-1)^q \sum_{p\in \Iset} \iinner{\frac{\dbar
          v_{p;(\infty)}'}{\sect_{(p)}}}{u_p}_{\lcS}
      =(-1)^q \sum_{p\in \Iset} \iinner{\dbar v_{p;(\infty)}'}{u_p
        \sect_{(p)}}_{\lcS, \phi_{(p)}}
      \\
      &\overset{\mathclap{\text{Prop.~\ref{prop:res-formula-dbar-exact-dot-harmonic}}}}=
        \quad \;\; (-1)^{q-1} \sigma
        \smashoperator{\sum_{b \in \Iset+1}} \iinner{v_{b;(\infty)}'}{w_b
        \sect_{(b)}}_{\lcS+1[b], \phi_{(b)}}
      \\
      &=\sigma \smashoperator{\sum_{b \in \Iset+1}}
        \iinner{
          \paren{w_b -\dbar v_{b;(2)}'} \sect_{(b)}
        }{
          w_b \sect_{(b)}
        }_{\lcS+1[b], \phi_{(b)}}
      \\
      &\overset{\mathclap{w_b \text{ harmonic}}}= \quad\;\;\;
        \sigma
        \smashoperator{\sum_{b \in \Iset+1}}
        \iinner{w_b}{w_b}_{\lcS+1[b]}
        =\sigma \norm{w}_{\lcc+1'}^2 \; .
    \end{align*}
    As a result, $w_b = 0$ for each $b\in\Iset+1$, thus $su_p = 0$
    (hence $u_p = 0$) for each $p\in\Iset$ by Step
    \ref{item:express-su-in-residue-norm}.
    This completes the proof. \qedhere
  \end{enumerate}
\end{proof}

\begin{remark} \label{rem:singular-vphi_F}
  When $\vphi_F$ and $\vphi_M$ have only neat analytic singularities
  such that $\vphi_F^{-1}(-\infty) \cup \vphi_M^{-1}(-\infty) \cup D$
  has only snc and $\vphi_F^{-1}(-\infty) \cup \vphi_M^{-1}(-\infty)$
  contains no irreducible components of $D$ (hence no lc centers of
  $(X,D)$), the proof is still valid when the K\"ahler metric $\omega$
  on $X$ is replaced by a complete metric on $X \setminus
  \paren{\vphi_F^{-1}(-\infty) \cup \vphi_M^{-1}(-\infty)}$ as
  described in \cite{Chan&Choi_injectivity-I}*{\S 2.2 item (4)}.
  See \cite{Chan&Choi_injectivity-I}*{\S 3.3} for the technical
  modifications required.
\end{remark}

\begin{remark} \label{rem:no-hard-Lefschetz}
  Notice that the refinement of hard Lefschetz theorem (see
  \cite{Matsumura_injectivity-lc}*{Thm.~1.7} or
  \cite{Chan&Choi_injectivity-I}*{Thm.~2.5.1}) is not used in this
  proof.
  It is used in previous works to show that $\frac{u}{\sect_D}$ is
  smooth for every harmonic $u$ representing a class in $\cohgp
  q[X]{\logKX\otimes \mtidlof<X>{\phi_D}}$.
  This argument can be replaced by using directly the isomorphism
  induced by $\holo_X \xrightarrow[\isom]{\otimes \sect_D} D \otimes
  \mtidlof<X>{\phi_D}$, or $\holo_{\lcS} \xrightarrow[\isom]{\otimes
    \sect_{(p)}} \Diff_p(D) \otimes \mtidlof<\lcS>{\phi_{(p)}}$, which
  is more relevant to this article (see also Lemma
  \ref{lem:su-harmonicity}).
  However, when $\vphi_F$ and $\vphi_M$ have neat analytic singularities
  as described in \cite{Chan&Choi_injectivity-I}*{\S 2.2}, the theorem
  is still needed to get certain control of the regularity of $u$ on the
  polar sets of $\vphi_F$ and $\vphi_M$ (see
  \cite{Chan&Choi_injectivity-I}*{Prop.~3.3.1}).
\end{remark}

%%% Local Variables:
%%% mode: latex
%%% TeX-master: "Injectivity-Fujino"
%%% coding: utf-8
%%% End:


%%%%% Reference list %%%%%
\begin{bibdiv}
  \begin{biblist}
    \IfFileExists{references.ltb}{
      \bibselect{references}
    }{
      %%%%%
%%%%% File name  : Injectivity-Fujino.tex
%%%%% Author     : Mario Chan
%%%%% Date       : 25th July, 2022 
%%%%% Description: This file is set up to compile the project with the
%%%%%              class amsart.cls. This project "Injectivity-Fujino"
%%%%%              proves the injectivity theorem on lc pairs of
%%%%%              arbitrary codimension of the mlc's, the full Fujino
%%%%%              conjecture.
%%%%%
%%
%%%
\documentclass[a4paper,12pt]{amsart}

\pdfoutput=1 %% added to force arXiv AutoTeX to typeset with pdflatex
             %% (so that rotation of symbols in Xy-pic is rendered);
             %% have to be within the first 5 lines of preamble

\usepackage[top=3cm,bottom=3cm,outer=3cm,inner=2cm,marginpar=2.45cm]{geometry}
\usepackage[destlabel,final,colorlinks=true]{hyperref}
\usepackage[abbrev]{amsrefs}

\input{packagesandcommands}
\ifx\pdftexversion\undefined
  \renewcommand{\ibar}{{\raisebox{-0.9ex}{$\mathchar'26$}\mkern-6.7mu i}}
\fi


% \usepackage{showkeys}

%%%%% End of preamble %%%%%%%%%%%%%%%%%%%%%%%%%%%%%%%%%%%%%%%%%%%%%%%%

\begin{document}

\citealias{Amb03}{Ambro_quasi-log-var}
\citealias{Amb14}{Ambro_injectivity}
\citealias{Eno90}{Enoki}
\citealias{EV92}{Esnault&Viehweg_book}
\citealias{Fuj11}{Fujino_log-MMP}
\citealias{Fuj12b}{Fujino_vanishing-thms}
\citealias{Fuj13a}{Fujino_injectivity-II}
\citealias{Fuj13b}{Fujino_injectivity-hodge-theoretic}
\citealias{Fuj15b}{Fujino_survey}



\input{titleinfo}

\title[\shorttitlestr]{\titlestr}
 
\author[\MCnameshort]{\MCname}
\email{\MCemail}
% \address{\addressstr}
% \curraddr{}

\author{\YJname}
\email{\YJemail}
\address{\PNUAddressstr}

\author{\ShMname}
\email{\ShMemail}
\address{\TohokuAddressstr}


% \thanks{\thankstr}
 
\subjclass[2020]{\subjclassstr}

\keywords{\keywordstr}

% \dedicatory{\dedicatorystr}

% \begin{abstract}
%   \input{abstract}
% \end{abstract} 

%%%SM's notation: I will change them to Mario's notation later 
% \newcommand{\Ker}[0]{\operatorname{Ker}}
\let\Ker\ker
\newtheorem{step}{Step}
%%%




%%choiyj's macros
\def\del{\partial}
\def\we{\wedge}
\def\ov{\overline}
\newcommand{\pd}[2]{\frac{\partial#1}{\partial#2}}
%%





\date{\today} 

\maketitle

%%%%% End of Top matter %%%%%%%%%%


\section{Introduction}\label{sec:intro}

{
  \let\thesubsection\thesection
  
  % This paper studies an analytic aspect of higher cohomology groups of adjoint bundles for lc $($log canonical$)$ pairs
  % aiming to solve Fujino's conjecture on the injectivity theorem as a benchmark. 
  This paper studies an analytic aspect of higher cohomology groups of adjoint bundles
  for log-canonical (lc) pairs aiming to solve Fujino's conjecture, 
  the injectivity theorem for lc pairs on compact K\"ahler manifolds, 
  following the line of Enoki's proof. 
  This is achieved by developing the theory of harmonic integrals
  on lc centers using the analytic adjoint ideal sheaves and the
  associated residue techniques.


  The injectivity theorem, a generalization of the Kodaira vanishing theorem to semi-positive line bundles, 
  plays an important role in higher dimensional algebraic geometry. 
  After the original Koll\'ar's injectivity theorem \cite{Kollar_injectivity} had been proved 
  for semi-ample line bundles on smooth projective varieties, 
  Enoki \cite{Eno90} generalized Koll\'ar's injectivity theorem 
  to semi-positive line bundles on compact K\"ahler manifolds. 
  Koll\'ar's proof is based on theory of Hodge structures, whereas
  Enoki's proof is based on the theory of harmonic integrals, a more
  well-suited and flexible technique in the complex analytic situation. 

  Ambro and Fujino generalized Koll\'ar's theory to varieties with lc
  singularities via the theory of mixed Hodge structures,  
  motivated by applications to birational geometry (see \cite{Amb03, Amb14, EV92, Fuj11, Fuj12b, Fuj13b}). 
  % \mariocomment{To SM: please
  % check if these references links to the correct papers. Change the
  % $\backslash$\texttt{citealias} commands if you wish.}% 
  % The works of Ambro and Fujino can be expected to
  It is expected that their works can also be generalized in the same line as Enoki's
  by developing an analytic treatment to lc singularities. 
  Motivated by this expectation, Fujino posed the conjecture below. 
  (Set $\ibar := \ibardefn$ \ibarfootnote\ and let $D$ be a reduced divisor for the
  rest of this article.)



  \begin{conjecture}[{Fujino's conjecture, \cite[Conjecture
      2.21]{Fuj15b}, cf.~\cite[Problem 1.8]{Fuj13a}}] 
    \label{conj:fujino}

    Let $X$ be a compact K\"ahler manifold and
    $D=\sum_{i=1}^{N}D_{i}$ be a simple-normal-crossing
    (snc) divisor on $X$.  Let $F$ be a semi-positive line bundle on
    $X$ (i.e.~it admits a smooth Hermitian metric $h_{F}$ with
    $\ibar\Theta_{h_F}(F) \geq 0$).  Consider a section
    $s \in H^{0}(X, F^{\otimes m})$ whose zero locus $s^{-1}(0)$
    contains no lc centers of the pair $(X,D)$ (i.e.~connected
    components of non-empty intersection
    $D_{i_{1}}\cap \cdots \cap D_{i_{k}}$ of the irreducible
    components $\{D_{i}\}_{i=1}^{N}$).  Then, the multiplication map
    induced by the tensor product with $s$
    \begin{equation*}
      H^{q}\paren{X, K_{X} \otimes D \otimes F}
      \xrightarrow{\otimes s} 
      H^{q}(X, K_{X} \otimes D \otimes F^{\otimes (m+1)} )
    \end{equation*}
    is injective for every $q$.
  \end{conjecture}

  The analytic theory corresponding to Koll\'ar's theory has been established for klt singularities 
  (see \cite{Cao&Demailly&Matsumura, Fujino&Matsumura, Gongyo&Matsumura,
    Matsumura_injectivity-survey, Matsumura_injectivity}).
  % but not for lc singularities. 
  Therefore, it remains to develop an analytic treatment to handle the
  lc singularities.
  % the analytic theory corresponding to the works of Ambro and Fujino
  % is interesting in terms of  studying the techniques of analytically treating lc singularities or mixed Hodge theory than just generalizing it. 


  The cases of $\dim X=2$ and plt pairs of arbitrary dimension have been
  solved in \cite{Matsumura_injectivity-lc,
    Matsumura_rel-vanishing-w-nd} (see also \cite{Chan&Choi_injectivity-I}). 
  A full solution to Fujino's conjecture is given recently by
  Junyan Cao and Mihai P\u{a}un \cite{Cao&Paun_LC-inj}.
  In this paper, independent of the results in \cite{Cao&Paun_LC-inj},
  we prove a {\textit{generalized version}} of Fujino's conjecture  
  (Theorem \ref{thm:main}) 
  % by developing the theory of harmonic integrals on simple normal corssing divisors. 
  by applying the theory of harmonic integrals on lc centers of the
  given lc pair.
  Fujino's conjecture is then a direct consequence of Theorem \ref{thm:main}. 
  

  \begin{thm}[Main Result]\label{thm:main}
    Let $X$ be a compact K\"ahler manifold  and 
    $D=\sum_{i=1}^{N}D_{i}$ be an snc divisor on $X$. 
    % such that each component $D_{i}$ is compact. 
    Let $F$ (resp.~$M$) be a line bundle on $X$ 
    with a smooth Hermitian metric $h_{F}$  (resp.~$h_{M}$) 
    such that 
    \begin{equation*}
      \ibar\Theta_{h_F}(F)\geq 0 \quad  \text{ and } \quad
      % \sqrt{-1}(\Theta_{h_F}(F)-t \Theta 
      % _{h_M}(M))\geq 0
      % -C\omega \leq
      \ibar\Theta_{h_M}(M) \leq C \ibar\Theta_{h_F}(F)
      \quad \text{ for some } C>0 \; . 
    \end{equation*}
    % (that is, $D_{i} \cap D_{j} = \emptyset$ for $i \not = j$ 
    % for the irreducible decomposition $D = \sum_{i\in I}D_{i}$). 
    Let $s$ be a  section of $M$  
    such that the zero locus $s^{-1}(0)$ 
    contains no lc centers of the lc pair $(X,D)$.
    Then, the multiplication map induced by the tensor product with $s$
    \begin{equation*}
      H^q(D, K_D \otimes F)
      \xrightarrow{\otimes s } 
      H^q(D, K_D \otimes F\otimes M)
    \end{equation*} 
    is injective for every $q$. 
  \end{thm}

  It can be seen from the proof that the compactness of $X$ in Theorem
  \ref{thm:main} is not necessary as soon as $D$ consists of only finitely many
  irreducible components which are compact.

  \begin{cor}[Solution to Fujino's conjecture]\label{cor:main}
    Conjecture \ref{conj:fujino} is true. 
  \end{cor}


  % Our paper differs from \cite{Cao&Paun_LC-inj} in the following points: 
  % The method of \cite{Cao&Paun_LC-inj} is based on the $L^{2}$-theory of $\dbar$-equations, 
  % whereas our method is based on the theory of harmonic integrals in the same line as in Enoki's work; 
  % specifically, we extend a technique of harmonic differential forms on smooth varieties to simple normal corssing divisors.

  Our proof differs from the one in \cite{Cao&Paun_LC-inj} in the following way.
  While both works make use of (some variant of) the Hodge
  decomposition for $L^2$ forms, Cao and P\u aun prove in
  \cite{Cao&Paun_LC-inj} a Hodge decomposition for $L^2$ forms with
  respect to a K\"ahler metric with conic singularities, which induces
  a Hodge decomposition on currents (which is called the Kodaira--de
  Rham decomposition in \cite{Cao&Paun_LC-inj}) in which the Green
  kernel has controllable singularities.

  For the sake of explanation, let $u$ be an $D\otimes F$-valued
  $(n,q)$-form representing a class in $\cohgp q[X]{\logKX}$
  % ($X$ being compact here)
  such that the class of $s u$ is $0$ in $\cohgp
  q[X]{\logKX M}$.
  Let also $\sect_D$ be a canonical section of $D$.
  Under our notation, the current that is under consideration in
  \cite{Cao&Paun_LC-inj} is $\frac{u}{\sect_D}$, which is not
  necessarily $L^2$ on $X$.
  Using the fact that $\eqcls{su} = 0$ in $\cohgp q[X]{\logKX M}$,
  Cao and P\u aun obtain $\frac{u}{\sect_D} =\dbar\theta + D'_{h_F}
  \beta_1 +\ibar\Theta_{h_F} \wedge \beta_2$, where $\theta$ is
  smooth while $\beta_1$ and $\beta_2$ have log-poles along
  $D+s^{-1}(0)$ (assumed to have only snc).
  It then follows from \cite{Cao&Paun_LC-inj}*{Thm.~1.1} (which
  makes use of the Hodge/Kodaira--de Rham decomposition) and the
  positivity $\ibar\Theta_{h_F} \geq 0$ that $u$ (or $u -\sect_D
  \dbar\theta$) is $\dbar$-exact.

  In our case, we make use of the residue exact sequences of adjoint
  ideal sheaves and the associated residue computation to reduce the
  setup to the union of \emph{$\sigma$-lc centers} of $(X,D)$ (i.e.~lc centers of
  codimension $\sigma$ in $X$, when $(X,D)$ is log-smooth and lc).
  Since each $\sigma$-lc center is a compact K\"ahler manifold, we
  have the Hodge decomposition (thus $L^2$ Dolbeault isomorphism and
  harmonic theory) at our disposal.
  Moreover, our reduction brings the setup essentially to the one in
  \cite{Matsumura_injectivity-lc}*{Thm.~1.6} or
  \cite{Chan&Choi_injectivity-I}*{Thm.~1.2.1} (corresponding to the
  case where $\frac{u}{\sect_D}$ is $L^2$).
  That's why we can follow the line of arguments in Enoki's proof to
  solve the conjecture via the theory of harmonic integrals on lc
  centers (and no extra resolution to bring $s^{-1}(0)$ into snc is
  needed).

  % Thanks to this advantage, we can obtain the generalized version  (not only Fujino's conjecture), 
  This approach gives us the advantage of obtaining Theorem
  \ref{thm:main}, a generalized version of Fujino's conjecture (see
  also Remark \ref{rem:general-commut-diagram} for other generalized
  statements which can be achieved),
  which does not seem to be derivable from results in \cite{Cao&Paun_LC-inj}, at
  least not directly. 
  % Furthermore, the previous works (including \cite{Cao&Paun_LC-inj, Amb03, Amb14, Fuj11} 
  % used the the assumption that $s^{-1}(0)$ contains no lc centers
  % to reduce the proof to the case where the pair $(X,D + s^{-1}(0))$ is log smooth; 
  % however, our paper uses this assumption to apply the inductive argument in terms of lc strata
  % by  the fact that all the data restricted to each component $D_{i}$ satisfy the assumption again. 



  Here we briefly explain the outline of the  proof of Theorem
  \ref{thm:main} with the example where the snc divisor $D$ has
  only two components $D_1$ and $D_2$ such that $D_1 \cap D_2$ is
  irreducible as an illustration.
  In this case, the union of the $1$-lc centers of $(X,D)$ is
  $\lcc|1|' = D_1 \cup D_2$ while that of the $2$-lc centers is
  $\lcc|2|' = D_1 \cap D_2$.
  For any given cohomology class $\alpha \in H^q(D,  K_{D} \otimes F)$
  such that $s  \alpha =0$ in $H^q(D,  K_{D} \otimes F \otimes M)$, 
  the goal is to show that $\alpha$ is actually $0$. 


  Write $h_F = e^{-\vphi_F}$ and $h_M = e^{-\vphi_M}$, and let $\psi_D
  := \phi_D -\sm\vphi_D :=\log\abs{\sect_D}^2 -\sm\vphi_D$ be a global
  function on $X$ such that $\phi_D$ is the (local) potential (of the
  curvature of a metric) on $D$ induced from a canonical section
  $\sect_D$ and $\sm\vphi_D$ is some smooth potential on $D$.
  When $D$ is smooth (i.e.~$D_{1}\cap D_{2}=\emptyset$), 
  the class $\alpha $ can be represented by $(u_{1},  u_{2})$, where $u_i$
  is a harmonic form with respect to $\vphi_F$ on $D_i$ in
  $\mathcal{H}^{n-1,q}(D_{i}; F)_{\vphi_{F}} \cong H^{q}(D_{i},
  K_{D_{i}} \otimes F)$ for $i=1,2$.
  Enoki's argument \cite{Eno90} shows that $s u_{i}$ is also a harmonic
  form with respect to $\vphi_F +\vphi_M$ using the
  Bochner--Kodaira--Nakano formula and the given curvature assumption.
  % by the Bochner trick and the assumption of curvatures.
  It follows from $s \alpha =0$ (as a class) that $s u_{i}=0$ (as a form), hence
  $\alpha=(u_{1}, u_{2})=0$ as desired. 

  However, when $D =\lcc|1|'$ (as well as other $\lcc'$ in the more
  general situation) is not smooth (i.e.~$D_{1}\cap D_{2} \neq
  \emptyset$), the Dolbeault and harmonic theories for cohomology groups
  on $D$ are not yet established, obstructing the use of Enoki's
  argument.
  To overcome this difficulty, we make use of the short exact sequence
  \begin{equation*}
    \xymatrix{
      {0} \ar[r]
      & {\bigoplus_{i=1}^2 K_{D_{i}} \otimes \res F_{D_i}} \ar[r]^-{\tau}
      & {K_{D} \otimes \res F_{D}} \ar[r]
      & {K_{D_{1} \cap D_{2}} \otimes \res F_{D_1 \cap D_2}} \ar[r]
      & {0}
    } \; ,
  \end{equation*}
  where $K_{D} :=K_{X}\otimes D \otimes \frac{\holo_X}{\defidlof{D}}$
  and $\defidlof{D}$ is the defining ideal sheaf of $D$ in $X$, and its
  associated long exact sequence of cohomology groups to reduce our
  injectivity problem of the map $\otimes s$ on $D$ to the injectivity
  problems of $\otimes s$ on the lc centers of $(X,D)$ (i.e.~$D_1$,
  $D_2$ and $D_1 \cap D_2$). 
  Note that all of the lc centers are not contained in $s^{-1}(0)$ by
  assumption and are compact K\"ahler manifolds on which the Dolbeault
  isomorphism and harmonic theory are available.

  Such strategy is suggested already in \cite{Matsumura_injectivity-lc}
  and is used there in the proof of the injectivity theorem for plt
  pairs.
  It is framed in \cite{Chan&Choi_injectivity-I} in terms of the adjoint
  ideal sheaves $\aidlof* := \aidlof = \mtidlof<X>{\vphi_F} \cdot
  \defidlof{\lcc+1'} = \defidlof{\lcc+1'}$ (the defining ideal sheaf
  of $\lcc+1'$ in $X$, under the assumption $\vphi_F$ being smooth)
  for integers $\sigma \geq 0$ and the corresponding residue morphisms
  $\Res^\sigma$ for $\sigma \geq 1$ (see Section \ref{subsec:residue}
  for the definitions).
  Writing $\lcc' = \bigcup_{p \in \Iset} \lcS$ as the decomposition of
  $\lcc'$ into the (irreducible) $\sigma$-lc centers $\lcS$, the residue
  morphism $\Res^\sigma$ induces the isomorphism
  \begin{equation*}
    \logKX \otimes \faidlof/-1* \xrightarrow[\isom]{\Res^\sigma}
    \logKX \otimes \residlof* := \bigoplus_{p \in \Iset} K_{\lcS}
    \otimes \res F_{\lcS} 
  \end{equation*}
  (notice that $\logKX \otimes \frac{\defidlof{D_1 \cap
      D_2}}{\defidlof{D}} \isom \bigoplus_{i=1}^2 K_{D_{i}} \otimes \res
  F_{D_i}$ and $\logKX \otimes \frac{\holo_X}{\defidlof{D_1 \cap D_2}}
  \isom K_{D_{1} \cap D_{2}} \otimes \res F_{D_1 \cap D_2}$ in the
  example).
  It can then be seen that, for more general $D$, the reduction can be
  done via the short exact sequences $0 \to \faidlof/-1* \to
  \faidlof|\sigma'|/-1* \to \faidlof|\sigma'|/* \to 0$ for some integers
  $\sigma$ and $\sigma'$ such that $1 \leq \sigma \leq \sigma'$,
  together with an induction on $\sigma$ via some diagram-chasing
  argument.
  See Step \ref{step:harmonic-rep} of Section
  \ref{sec:proof-of-simple-case} and the beginning of Section
  \ref{subsec:general} for precise details.


  After the reduction, we are led to consider the maps
  \begin{equation*}
    \renewcommand{\objectstyle}{\displaystyle}
    \xymatrix{
      {\smash{\bigoplus_{i =1}^2}\:\cohgp q[D_i]{K_{D_i} \otimes
          F}} \ar[r]^-{\tau}
      \ar[dr]^-{\nu}
      &{\cohgp q[D]{K_D \otimes F}}
      \ar[d]^-{\otimes s} \ar@{}@<-1em>[d]_*+{\circlearrowright}
      \\
      &{\cohgp q[D]{K_D \otimes F \otimes M} \; .}
    }
  \end{equation*}
  It suffices to prove that $\ker\nu =\ker\tau$ (Theorem
  \ref{thm:ker-nu=ker-tau}).
  Indeed, given the injectivity of the map $\otimes \res s_{D_1 \cap D_2}$ on
  $\cohgp q[D_1 \cap D_2]{K_{D_1 \cap D_2} \otimes F}$ followed from
  Enoki's argument in the previous case, we see that the given class
  $\alpha \in \cohgp q[D]{K_D \otimes F}$ actually lies in the image
  $\im\tau$ of $\tau$, say, $\alpha = \tau\paren{u_1, u_2}$ for some
  harmonic forms $u_i \in \Harm'/n-1,q/<D_i>{F},{\vphi_F} \isom \cohgp
  q[D_i]{K_{D_i} \otimes F}$.
  It will then follow that $\paren{u_1, u_2} \in \ker\nu
  =\ker\tau$, hence $\alpha =0$, as desired.
  The pair $(u_1,u_2)$ can be treated as a representative of $\alpha$.
  Suggested by the fact that a harmonic form is the unique
  representative with the \emph{minimal} $L^2$ norm among all elements
  in its corresponding $L^2$ Dolbeault cohomology class, we can choose
  an ``optimal'' representative of $\alpha$ such that $(u_1, u_2)$ has
  the \emph{minimal} distance from (i.e.~is orthogonal to) the subspace
  $\ker\tau$ with respect to the $L^2$ norm induced from $\vphi_F$.
  It then suffices to show that $u_i =0$ for $i = 1,2$ to prove that
  $\ker\nu =\ker\tau$.
  This is done by following the proof of
  \cite{Matsumura_injectivity-lc}*{Thm.~1.6} or
  \cite{Chan&Choi_injectivity-I}*{Thm.~1.2.1} (therefore following the
  spirit of Enoki's argument), but with a few technical modifications.

  One technical complication comes from the use of \v Cech cohomology
  for some cohomology groups (e.g.~$\cohgp q[D]{K_D \otimes F}$) due to
  the lack of the Dolbeault isomorphism.
  Another one is that the argument of Takegoshi in
  \cite{Chan&Choi_injectivity-I}*{\S 3.1, Step IV} (see also
  \cite{Matsumura_injectivity-lc}*{Prop.~3.13}), which essentially gives
  rise to an element in $\ker\tau$ constructed from $u_i$'s, is replaced
  by a construction of a harmonic forms $w$ (or a collection $w
  :=\paren{w_b}_{b\in \Iset+1}$ of harmonic forms for the general $D$)
  representing a class in $\cohgp{q-1}[D_1 \cap D_2]{K_{D_1 \cap D_2}
    \otimes F}$ (see \eqref{eq:w-prelim-formula} and \eqref{eq-def-w}).
  The class of $w$ has its image lying in $\ker\tau$ via the connecting
  morphism of the relevant long exact sequence.
  Such construction is suggested by a residue computation, which relates
  an inner product on (the normalization of) $\lcc|1|'$ to an inner
  product on (lower dimensional) $\lcc|2|'$ (see Proposition
  \ref{prop:res-formula-dbar-exact-dot-harmonic}; see also Steps
  \ref{item:express-su-in-residue-norm} and \ref{item:pf:use_u-ortho-w}
  in Section \ref{subsec:general}, or Steps
  \ref{item:expression-of-su-simple} and
  \ref{step:pf:use_u-ortho-w-simple} in Section
  \ref{sec:proof-of-simple-case} for less intensive notation).
  Such relation between the inner products shows that $w$ is the
  obstruction for having $u_i = 0$ for $i=1,2$.
  This becomes the crucial ingredient to complete the proof.

  The proof of Theorem \ref{thm:main} for the case of general $D$
  follows the same arguments.
  A brief comment for the case where $\vphi_F$ and $\vphi_M$ possess
  suitable analytic singularities is given in Remarks
  \ref{rem:singular-vphi_F} and \ref{rem:no-hard-Lefschetz}.
  
  







  % We briefly explain the proof in the simple case where $D$ has two components (i.e.,\,$D=D_{1}+D_{2}$). 
  % For a given cohomology class $\alpha \in H^q(D,  K_{D} \otimes F)$, 
  % we will prove that $\alpha $ is actually zero 
  % under assuming that $s  \alpha =0 \in H^q(D,  K_{D} \otimes F \otimes M)$. 

  % In the case where $D$ is smooth (i.e.,\,$D_{1}\cap D_{2}=\emptyset$), 
  % the class $\alpha $ can be represented by a harmonic form $(u_{1},  u_{2})$. 
  % Here we used 
  % $$
  % H^q(D,  K_{D} \otimes F)=\oplus_{i=1}^{2} H^{q}(D_{i},  K_{D_{i}} \otimes F) \cong 
  % \oplus_{i=1}^{2} \mathcal{H}^{n-1,q}(D_{i}, F)_{h_{F}}, 
  % $$ 
  % where $\mathcal{H}^{n-1,q}(D_{i}, F)_{h_{F}}$ is the space of harmonic forms with respect to $h_{F}$. 
  % Enoki's argument \cite{Eno90} shows that  
  % $s u_{i}$ is also a harmonic form with respect to $h_{F} h_{M}$
  % by the Bochner trick and the assumption of curvatures. 
  % This implies that $s u_{i}=0$ by $s \alpha =0$; hence $\alpha=\{(u_{1}, u_{2})\}=0$. 

  % In the general case  (i.e.,\,$D_{1}\cap D_{2} \not =\emptyset$), 
  % it is not clear whether the class $\alpha$ can be represented by a harmonic form on $D$,  
  % which is an obvious  difficulty in extending Enoki's argument.
  % To overcome this difficulty, we consider the long exact sequence 
  % $$
  % \cdots \to\bigoplus_{i=1}^{2} H^q(D_{i},  K_{D_{i}}\otimes F ) \xrightarrow{\tau} H^q(D,  K_{D} \otimes F)  \to H^q(D_{1}\cap D_{2},  K_{D_{1}\cap D_{2}}\otimes F ) \to \cdots
  % $$
  % induced by $0 \to K_{D_{1}} \oplus K_{D_{2}} \to K_{D} \to K_{D_{1} \cap D_{2}} \to 0$, 
  % noting that $ K_{D}=(K_{X}\otimes D)\otimes \mathcal{O}_{X}/\mathcal{I}_{D}$. 
  % The multiplication map defined on the right term 
  % $$
  % \otimes s |_{D_{1} \cap D_{2}}: H^q(D_{1}\cap D_{2},  K_{D_{1}\cap D_{2}}\otimes F ) \to
  % H^q(D_{1}\cap D_{2},  K_{D_{1}\cap D_{2}}\otimes F \otimes M ) 
  % $$ 
  % is non-zero since $s^{-1}(0)$ contains no lc centers of the pair $(X,D)$, 
  % and thus injective by induction hypothesis. 
  % Thus, by chasing the commutative diagram induced by the multiplication map, 
  % we can take a cohomology class 
  % $\beta \in \oplus_{i=1}^{2} H^q(D_{i},  K_{D_{i}}\otimes F )$ such that $\tau(\beta)=\alpha$. 
  % Then, we can take a harmonic representation $(u_{1}, u_{2})$ of $\beta$. 


  % The pair $(u_{1}, u_{2})$ is the {\textit{best}} representation for $\beta$ 
  % in the sense that $(u_{1}, u_{2})$ has the minimum $L^2$ norm in the forms representing $\beta$. 
  % However, the pair $(u_{1}, u_{2})$ may not be the best representation for $\alpha$ 
  % since the $L^2$ norm may be reduced by $\tau$. 
  % For this reason, by the orthogonal decomposition, 
  % we re-choose $\beta$ (and its harmonic representation $(u_{1}, u_{2})$) 
  % satisfying $(u_{1}, u_{2}) \in (\Ker \tau)^{\perp} \subset \oplus_{i=1}^{2} \mathcal{H}^{n-1,q}(D_{i}, F)_{h_{F}}$. 
  % The condition $(u_{1}, u_{2}) \in (\Ker \tau)^{\perp}$ means a certain minimal $L^{2}$-norm; 
  % therefore $(u_{1}, u_{2})$ can be seen as the best representation for $\alpha$. 


  % The Bocher trick shows that the $L^2$ norm of $(u_{1}, u_{2})$ is zero in the case $D_{1}\cap D_{2}=\emptyset$. 
  % By generalizing this Bocher trick, 
  % an obstruction for the $L^{2}$-norm to be $0$ 
  % can be described by a $F$-valued differential form $w$ on $D_{1} \cap D_{2}$. 
  % An important point here is that  $w$ is actually harmonic; in particular, it determines the cohomology class.   
  % $H^{q-1}(D_{1}\cap D_{2},  K_{D_{1}\cap D_{2}}\otimes F )$. 
  % Then, we can show that the $L^{2}$-norm of $w$ on $D_{1} \cap D_{2}$ 
  % is equal to the inner product of $(u_{1}, u_{2})$ and a representation of $\delta(w)$, 
  % where $\delta$ is the connecting morphism 
  % $\delta: H^{q-1}(D_{1}\cap D_{2},  K_{D_{1}\cap D_{2}}\otimes F ) \to \oplus_{i=1}^{2} H^{q}(D_{i},  K_{D_{i}} \otimes F)$. 
  % This implies that $w=0$ by $(u_{1}, u_{2}) \in (\Ker \tau)^{\perp}$.


  % 
  % ....


}

This paper is organized as follows.
\tableofcontents


\subsection*{Acknowledgments}
The authors would like to thank the members of Bayreuth University and Pusan National University for their hospitality.
This paper is resulted from the discussions there. 
S.M.~would like to thank Professors Junyan Cao and Mihai P\u{a}un for sharing a preliminary version of \cite{Cao&Paun_LC-inj}.
Also, he would like to thank Professor Osamu Fujino 
for his encouragement and long-standing discussions on lc singularities. 
He is partially supported 
by Grant-in-Aid for Scientific Research (B) $\sharp$21H00976 
and Fostering Joint International Research (A) $\sharp$19KK0342 from
JSPS.
Y.C.~and M.C.~would like to thank S.M.~for drawing their attention to
Fujino's conjecture not long before the covid pandemic (which results
in \cite{Chan&Choi_injectivity-I}) and for joining hand to complete
this project when most aspects of life went back to normal.
Y.C.~and M.C.~were supported by the National Research Foundation
of Korea (NRF) Grant funded by the Korean government
(Nos.~2023R1A2C1007227 and 2021R1A4A1032418).



\section{Preliminary results}\label{sec:preliminaries}

\subsection{Notation and conventions}\label{subsec:notation}

%%%%%
%%%%% File name  : notation.tex
%%%%% Author     : Mario Chan
%%%%% Date       : 13th December, 2021 (original: 04th November, 2020)
%%%%% Description: This is the section "Notation" in the project
%%%%%              "Injectivity-Fujino".
%%%%%
%%
%%%

% In this subsection, we summarize the notation used throughout this paper. 

The following notions are used throughout this paper unless stated otherwise. 
\begin{itemize}
\item $(X,\omega)$ is a compact K\"ahler manifold of dimension $n$. 

% \item $\omega$ is a K\"ahler form on $X$. 

\item $h_F := e^{-\vphi_F}$ and $h_M := e^{-\vphi_M}$, where $\vphi_F$ and
  $\vphi_M$ are respectively the given potentials on $F$ and $M$.
  
\item $D=\sum_{i \in \Iset||}D_{i}$ is a reduced simple-normal-crossing (snc)
  divisor on $X$ (where $\Iset||$ is a finite set). 

\item $\sect_i$ is a canonical section  of the irreducible component $D_{i}$. 

\item $\sect_D := \prod_{i\in \Iset||} \sect_i$ is the canonical section of $D$. 

\item $\sigma \in \{0,1,2,\cdots, n\}$.

% \item  $\Iset$ is the set of $p:=\{i_{1}, i_{2}, \cdots, i_{\sigma}\}$ such that  
% \mmark{$\lcS:=\cap_{k=1}^{\sigma} D_{i_{k}} $}{$\cap D_{i_k}$ may have more than
% one component.} is of codimension $\sigma$. 

% \item   $\lcc' := \cup_{p \in \Iset} \lcS$ is the union of $\sigma$-lc centers $\lcS$ of  $(X,D)$

  
\item $\lcc' :=\bigcup_{p \in \Iset} \lcS$ is the union of
  \emph{$\sigma$-lc centers of $(X,D)$}, i.e.~the
  $\sigma$-codimensional irreducible components of any intersections
  of irreducible components of $D$ (under the assumption $(X,D)$ being
  log-smooth and lc), indexed by $\Iset$.
  Set $\lcc|0|' := X$ and let $\Iset|0|$ be a singleton for convenience.
  Note also that $\Iset|1| = \Iset||$.

\item $\Diff_{p}D$ is the effective divisor on $\lcS$ defined by the 
adjunction formula 
\begin{equation*}
  K_{\lcS} \otimes \Diff_{p}D = \parres{K_X \otimes D}_{\lcS}
\end{equation*}
such that the restriction of $\sect_{(p)}:=
\smashoperator{\prod\limits_{i \in \Iset|| \colon D_i
    \not\supset \lcS}} \sect_i $ to $\lcS$ is a canonical section of
$\Diff_{p}D$.

\item $\phi_D :=\log\abs{\sect_D}^2$ and $\phi_{(p)}
  :=\log\abs{\sect_{(p)}}^2$ are the potentials induced from the
  canonical sections of $D$ and $\Diff_p D$.

\item $\cvr V := \{V_{i}\}_{i \in I}$ is an open cover of $X$  by admissible open sets. 

\item $\{\rho^{i}\}_{i\in I}$ is a partition of unity subordinate to
  $\cvr V$. 
\end{itemize}

Here an open set $V \subset X$ is said to be \emph{admissible} with
respect to $D$ if $V$ is biholomorphic to a polydisc centered at the
origin under a holomorphic coordinate system $(z_{1}, z_{2}, \cdots,
z_{n})$ such that
\begin{equation*} % \label{eq:local-expression-bphi-psi}
  D =\set{z_1 \dotsm z_{\sigma_V} =0}, \quad 
  \log r_{j}^2 < 0, \quad \text{and }
  r_j \fdiff{r_j} \psi_D >0 \text{ on } V \; , 
  % \res{\vphi_\bullet}_V = \smashoperator{\sum_{k=\sigma_V+1}^n} b_{\bullet,k}
  % \log\abs{z_k}^2 +\beta_\bullet \;\;\text{ for } \bullet= F, M \; ,
\end{equation*} 
where  $r_j := \abs{z_j}$  and $\res{\psi_D}_V := \parres{\phi_D
  -\sm\vphi_D}_V =\sum_{j=1}^{\sigma_V} \log\abs{z_j}^2
-\res{\sm\vphi_D}_V$. 

When an admissible set $V$ is considered, an index $p \in \Iset$ such
that $\lcS \cap V \neq \emptyset$ is interpreted as a permutation
representing a choice of $\sigma$ elements from the set
$\set{1,2,\dots,\sigma_V}$ such that
\begin{equation*}
  \lcS \cap V = \set{z_{p(1)} = z_{p(2)} = \dotsm = z_{p(\sigma)} = 0}
  \quad\text{ and }\quad
  \res{\sect_{(p)}}_V = z_{p(\sigma+1)} \dotsm z_{p(\sigma_V)}
\end{equation*}
(cf.~the definition of the set $\cbn$ in \cite{Chan_adjoint-ideal-nas}*{\S 3.1}).



%%% Local Variables:
%%% mode: latex
%%% TeX-master: "Injectivity-Fujino"
%%% coding: utf-8
%%% End:


%\subfile{commut-diagram_Fujino-conj}%

\subsection{$L^{2}$ Dolbeault isomorphism and some results on harmonic
forms}\label{subsec:l2}

\input{L2-spaces-n-harmonic-forms}



\subsection{Adjoint ideal sheaves and the residue computations}
% \subsection{Residue functions and residue short exact sequences}
\label{subsec:residue}

%%%%%
%%%%% File name  : residue-fcts-n-residue-exact-seq.tex
%%%%% Author     : Mario Chan
%%%%% Date       : 10th March, 2023
%%%%% Description: This is the section of the project
%%%%%              "Injectivity-Fujino" on residue functions and 
%%%%%              residue exact sequences. 
%%%%%
%%
%%%

{
  \setDefaultvphi{\vphi_L}

  Let $L$ be a line bundle on $X$ equipped with a \mmark{smooth metric
    $e^{-\vphi_L}$}{$\vphi_L$ has to be smooth, or the claim on the
    jumping number must be mentioned explicitly. The result  $\aidlof*
    =\mtidlof{\vphi_L} \cdot \defidlof{\lcc+1'}$ may not hold
    otherwise. \\ }. 
  The \mmark[BlueGreen]{residue function $\eps \mapsto \RTF|f|,<V>$}{It's
    possible not using ``$\RTF|f|$'' at all in this paper.} of index $\sigma$ 
  is defined, for each \mhlight[BlueViolet]{$f \in  \logKX[L] \otimes
    \smooth_X (V)$}, to be 
  \begin{equation*}
    \RTF|f|,<V> :=\RTF|f|,<V>,
    := \eps \int_V \frac{\abs f^2 \:e^{-\phi_D-\vphi_L}}{\logpole} \quad
    \text{ for } \eps > 0  \; . 
  \end{equation*}\mariocomment[BlueViolet]{For consistency of notation in this
    section only.}%
  The adjoint ideal sheaf $\aidlof :=\aidlof<X>$ of index $\sigma$
  is given at each $x \in X$  by
  \begin{equation*}
    \aidlof_x :=\setd{f \in \holo_{X,x}}{
      \exists~\text{open set } V_x \ni x \:, \: \forall~\eps > 0 \:, \:
      \RTF|f|,<V_x>, < +\infty
    } \; .
  \end{equation*}
  Note that the adjoint ideal sheaf is independent of $\vphi_L$ (as
  $\vphi_L$ is smooth).
  By \cite{Chan_adjoint-ideal-nas}*{Thm.~1.2.3}, the adjoint ideal
  sheaf can be written as 
  \begin{equation*}
    \aidlof = \mtidlof{\vphi_L} \cdot \defidlof{\lcc+1'}
    =\defidlof{\lcc+1'}
    \quad\text{ for any } \sigma \geq 0 \; ,
  \end{equation*}
  where $\defidlof{\lcc+1'}$ is the defining ideal sheaf of $\lcc+1'$
  in $X$ (with the reduced structure), \mmark{and we have the residue short exact
    sequence}{I don't want to suggest that the product structure of
    $\aidlof*$ implies directly the residue exact sequence.}
  \begin{equation*}
    \xymatrix@R-0.5cm@C+0.3cm{
      {0} \ar[r]
      & {\aidlof-1} \ar[r]
      & {\aidlof} \ar[r]^-{\Res^\sigma}
      & {\residlof} \ar[r]
      & {0 \; .}
    }
  \end{equation*}
  Here the quotient sheaf ${\residlof}$, called the \emph{residue sheaf of index $\sigma$}, can be written as 
  \begin{equation*}
    \residlof
    = \bigoplus_{p \in \Iset} \paren{\Diff_p D}^{-1}
    \otimes \mtidlof<\lcS>{\vphi_L}
    = \bigoplus_{p \in \Iset} \paren{\Diff_p D}^{-1}
  \end{equation*}
  Note $\logKX[L] \otimes \residlof =\bigoplus_{p \in\Iset} K_{\lcS} \otimes \res L_{\lcS}.$
  Next we describe the \emph{residue morphism $\Res^\sigma$} in terms of 
  the Poincar\'e residue map $\PRes[\lcS]$ given in
  \cite{Kollar_Sing-of-MMP}*{\S 4.18} as follows. 
  The Poincar\'e residue map $\PRes[\lcS]$ from $X$ to each $\lcS$ is
  uniquely determined after an orientation on the conormal bundle of
  $\lcS$ in $X$ is fixed.
  For an admissible open set $V \subset X$, 
  we have $\lcc' \cap V = \bigcup_{\alert{p \in \Iset}}
  \lcS<V>$ \mmark{(where $\lcS<V> := \lcS \cap V$, which is connected by the
  definition of the admissible open set, and possibly empty)}{This is
  a subtle fact that is used in the residue computation. We can keep
  using the same index set because $V$ is admissible.} and $\lcS<V>
=\set{z_{p(1)} =z_{p(2)} =\dotsm =z_{p(\sigma)}=0}$ when non-empty. 
  Under such coordinate system, a section $f $ of  $\logKX[L] \otimes
  \aidlof$ on $V \subset X$ can be written as
  \begin{equation*}
    f = \;\;\smashoperator{\sum_{p \in \Iset \colon \lcS<V>
        \neq\emptyset}} \;\; dz_{p(1)} \wedge \dotsm \wedge dz_{p(\sigma)}
    \wedge g_p \:\sect_{(p)} 
    =\;\;\smashoperator[l]{\sum_{p \in \Iset \colon \lcS<V>
        \neq\emptyset}}
    \frac{dz_{p(1)}}{z_{p(1)}} \wedge \dotsm
    \wedge \frac{dz_{p(\sigma)}}{z_{p(\sigma)}}
    \wedge g_p \:\sect_D \quad\text{ on } V. 
  \end{equation*}
  % Then, the Poincar\'e residue map $\PRes[\lcS]$ is given by
  \mmark{We therefore see that 
  \begin{equation*}
    \PRes[\lcS](\frac{f}{\sect_D})  =\res{g_p}_{\lcS} \in
    K_{\lcS} \otimes \res L_{\lcS} \quad\text{ on } \lcS<V> 
  \end{equation*}}{I don't want to give the impression that we define
  the Poincar\'e residue map by this formula.}%
  under the assumption that the orientation on the conormal bundle of
  $\lcS$ in $X$ on $V$ is given by $(dz_{p(1)}, dz_{p(2)}, \dots,
  dz_{p(\sigma)})$. 
  % Result in \cite{Chan_adjoint-ideal-nas}*{Thm.~4.1.2 (2)} (or the
  % computation in \cite{Chan_on-L2-ext-with-lc-measures}*{Prop.~2.2.1}
  % or \cite{Chan&Choi_ext-with-lcv-codim-1}*{Prop.~2.2.1}) yields
  % % (assuming that $f$ lives on a neighbourhood $V'$ of $\cl V$)
  % \begin{equation*}
  %   \RTF[\rho]|f|(0),<V> = \lim_{\eps \tendsto 0^+}
  %   %   \lim_{\rho \descendsto \charfct_{\cl V}}
  %   \RTF[\rho]|f|,<V>
  %   =\sum_{p \in \Iset} \frac{\pi^\sigma}{(\sigma -1)!} \int_{\lcS<V>}
  %   \rho \abs{g_p}^2 \:e^{-\vphi_L} 
  % \end{equation*}
  % for any compactly supported smooth function $\rho \colon V \to
  On the other hand, the residue morphism $\Res^\sigma$ is given in
  \cite{Chan_adjoint-ideal-nas}*{\S 4.2} by 
  \begin{equation*}
    \renewcommand{\objectstyle}{\displaystyle}
    \xymatrix@C+0.5cm@R-0.5cm{
      {\logKX[L] \otimes \aidlof} \ar[r]^-{\Res^\sigma}
      \ar@{}[d]|*[left]+{\in} 
      & {\hphantom{\logKX[L] \otimes \residlof}}
      \save +<4em,-1.3ex>*{\logKX[L] \otimes \residlof
        =\bigoplus_{p \in\Iset} K_{\lcS} \otimes \res L_{\lcS}} \restore
      \ar@{}[d]|*[left]+{\in}
      % & *+<-2cm,-1cm>{}
      % \ar@{}[l]|(.41)*+{}
      \\
      *+<0.8cm,0cm>{f} \ar@{|->}[r]
      & {\paren{\res{g_p}_{\lcS}}_{\mathrlap{p\in\Iset}}
        \mathrlap{\hphantom{p\in\Iset} .}} 
    }
  \end{equation*}
  Assuming $f$ being defined on a neighbourhood $V'$ of the closure
  $\cl V$ of $V$ and letting $\rho \colon V' \to [0,1]$ be a compactly
  supported smooth function
  % (i.e.~a smooth cut-off function) being
  identically equal to $1$ on $V$, one obtains, 
  from the result in \cite{Chan_adjoint-ideal-nas}*{Thm.~4.1.2 (2)} (or the
  computation in \cite{Chan_on-L2-ext-with-lc-measures}*{Prop.~2.2.1}
  or \cite{Chan&Choi_ext-with-lcv-codim-1}*{Prop.~2.2.1}),
  a (squared) norm of $g :=\paren{\res{g_p}_{\lcS<V>}}_{p \in \Iset} \in
  \logKX[L] \otimes \residlof$ on $V$ given by
  \begin{equation} \label{eq:residue-norm}
    \norm{g}_{\lcc<V>'}^2 :=\RTF|f|(0),<V>
    =\lim_{\rho \descendsto \charfct_{\cl V}} \lim_{\eps \tendsto 0^+}
    \RTF[\rho]|f|,<V'>
    =\sum_{p \in \Iset} \frac{\pi^\sigma}{(\sigma -1)!}
    \int_{\mathrlap{\lcS<V>}} \;\;\;
    \abs{g_p}^2 \:e^{-\vphi_L}
    =:\sum_{p\in\Iset} \norm{g_p}_{\lcS<V>}^2 \; ,
  \end{equation}
  where the limit $\lim_{\rho \descendsto \charfct_{\cl V}}$ refers to
  the pointwise limit as $\rho$ descends to the characteristic
  function $\charfct_{\cl V}$ of $\cl V$ on $X$.
  Such a norm is referred to as the \emph*{residue norm on $\logKX[L]
    \otimes \residlof$ on $V$}.
  %%%%% \emph* is needed as the package embrac is used and \logKX[L]
  %%%%% appears inside \emph.
  Moreover, we also see from the residue exact sequence that
  \begin{equation*}
    \aidlof-1_x
    =\setd{f \in \aidlof_x}{ \exists~\text{open set } V_x
      \ni x \:, \: \RTF|f|(0),<V_x> = 0}
    % \\
    % &=\setd{f \in \aidlof_x}{ \exists~\text{open set } V_x
    %   \ni x \:, \: \RTF|f|,<V_x> = \BigO(\eps) \text{ as } \eps
    %   \tendsto 0^+}
  \end{equation*}
  for every $x \in X$.

  Under the assumption that $\vphi_L$ has only neat analytic
  singularities (which is indeed smooth in the current setting), the
  residue norm on an admissible open set $V \subset X$ can also be
  obtained from 
  \begin{equation*}
    \lim_{\eps \tendsto 0^+} \eps \int_{V} \frac{
      \rho \abs f^2 \:e^{-\phi_D-\vphi_L}
    }{\abs{\psi_D}^{\sigma +\eps}}
    =\RTF[\rho]|f|(0),<V>
  \end{equation*}
  for any smooth compactly supported cut-off function $\rho$ on $V$ (see
  \cite{Chan&Choi_ext-with-lcv-codim-1}*{Prop.~2.2.1} or
  \cite{Chan&Choi_injectivity-I}*{Thm.~2.6.1}).
  Moreover, the above equation works not only for $f$ with holomorphic
  coefficients, but also for $f$ with coefficients in
  $\smooth_{X\,*}$, where
  \begin{align*}
    \smooth_{X\, *}
    &:=\paren{\smooth_{X}\left[
      \frac{1}{\abs{\sect_i}} \colon i \in \Iset||
      \right]}_{\text{b}}
      \qquad\paren{\sect_i \text{ treated as a local defining function of }
      D_i} \\
    &:=\set{\text{locally bounded elements in the $\smooth_X$-algebra generated
      by } \frac{1}{\abs{\sect_i}} \text{ for all } i\in\Iset||} \;
      .\footnotemark
  \end{align*}%
  \footnotetext{
    On an admissible open set $V$ under the holomorphic coordinate
    system $(z_1,\dots, z_n)$ such that $D\cap V =\set{z_1 z_2 \dotsm
      z_{\sigma_V} =0}$, one has
    \begin{equation*}
      \smooth_{X \,*}(V)
      =\smooth_X(V)\left[e^{\pm \cplxi \theta_1}, \dots, e^{\pm \cplxi
          \theta_{\sigma_V}} \right]
    \end{equation*}
    where $(r_j,\theta_j)$ is the polar coordinate system of the
    $z_j$-plane for $j=1,\dots,\sigma_V$ in $V$, which is (almost) the
    same as the ad hoc definition of $\smooth_{X\, *}(V)$ given in 
    \cite{Chan&Choi_injectivity-I}*{\S 2.6} (in which
    $e^{\pm\cplxi\theta_{k}}$ for $k \geq \sigma_V +1$ are also included
    in the set of generators of the algebra).
    The definition given here is independent of coordinates and its
    sheaf structure can be seen easily.
  }%
  The coefficients of $\Res^\sigma$ (and hence $\PRes[\lcS]$ for any
  $p\in\Iset$) can be extended from $\holo_X$ to $\smooth_{X\,*}$
  accordingly.
  The residue norm is finite when the coefficients of $f$ belong to
  $\smooth_{X\,*} \cdot \aidlof$ on $V$.
  When the induced inner product is considered, one still has
  finiteness even if one of the argument does not have coefficients in
  $\smooth_{X\,*} \cdot \aidlof$, which is the content of the
  following proposition.
  \begin{prop} \label{prop:residue-product-X-to-lcS}
    Given any admissible open set $V \subset X$ and any section $f \in
    \logKX[L] \otimes \smooth_{X \:c\,*} \cdot\aidlof\paren{V}$
    (compactly supported in $V$) such
    that $\Res^\sigma(f) = g =\paren{g_p}_{p\in\Iset}$, one
    has, for any $\xi \in \logKX[L] \otimes \smooth_{X \, *}\paren{V}$,
    \begin{align*}
      \lim_{\eps \tendsto 0^+} \eps \int_V
      \frac{\inner{\xi}{f} \:e^{-\phi_D-\vphi_L}}{\abs{\psi_D}^{\sigma
      +\eps}}
      &=\sum_{p \in \Iset} \frac{\pi^\sigma}{(\sigma-1)!}
        \int_{\lcS<V>} \inner{\frac{\rs*\xi_p}{\sect_{(p)}}}{\: g_p}
        \:e^{-\vphi_L} \\
      &=\sum_{p \in \Iset}
      % \smash[b]{
        \underbrace{
        \frac{\pi^\sigma}{(\sigma-1)!}
        \int_{\lcS<V>} \inner{\rs*\xi_p}{\: g_p \sect_{(p)}}
        \:e^{-\phi_{(p)}-\vphi_L}
        }_{\displaystyle =:
        \iinner{\rs*\xi_p}{g_p\sect_{(p)}}_{\mathrlap{\lcS<V>,
        \phi_{(p)}}}}
  % }
  %   \vphantom{\underbrace{\int_{\lcS<V>}}_{\iinner{\rs*\xi_p}{g_p\sect_{(p)}}}}
    \end{align*}
    which is finite,
    where $\phi_{(p)} :=\log\abs{\sect_{(p)}}^2$ and
    \begin{equation*}
      \rs*\xi_p := \PRes[\lcS](\frac{\xi}{\sect_D}) \cdot \sect_{(p)}
      \in K_{\lcS} \otimes \Diff_p D \otimes \res L_{\lcS} \otimes
      \smooth_{X\:c\, *}\paren{\lcS<V>} \; .
    \end{equation*}
    Moreover, if either $f$ or $\xi$ belongs to $\logKX[L] \otimes
    \smooth_{X\,*} \cdot\aidlof-1\paren{V}$, then $\eps \int_V
      \frac{\inner{\xi}{f} \:e^{-\phi_D-\vphi_L}}{\abs{\psi_D}^{\sigma
      +\eps}} = \BigO(\eps)$ (the big-O notation) as $\eps \tendsto 0$.
  \end{prop}

  \begin{proof}
    By linearity in $f$ in the equation in the claim, it suffices to
    consider the case where $\lcS \cap V = \set{z_1 =z_2 = \dotsm
      z_\sigma = 0}$, $\res{\sect_{(p)}}_{V} = z_{\sigma+1} \dotsm z_{\sigma_V}$ and
    \begin{equation*}
      f = dz_1 \wedge dz_2 \wedge \dotsm \wedge dz_\sigma \wedge g_p
      \sect_{(p)} 
    \end{equation*}
    (in which $g_p$ is abused to mean a $(n-\sigma,0)$-form on $V$).
    Write also
    \begin{equation*}
      \xi =: dz_1 \wedge dz_2 \wedge \dotsm \wedge dz_\sigma \wedge
      \xi_p
      \quad\text{ such that }\;\;
      \res{\xi_p}_{\lcS<V>} = \PRes[\lcS](\frac{\xi}{\sect_D})
      \cdot \sect_{(p)} =\rs*\xi_p \; .
    \end{equation*}
    Let $(r_j, \theta_j)$ be the polar coordinates of the $z_j$-plane
    and set 
    \begin{equation*}
      F_0 :=\inner{\xi_p}{g_p} \:e^{-\vphi_L}
      \quad\text{ and }\quad
      F_j :=\fdiff{r_j} \paren{\frac{F_j}{r_j^2 \fdiff{r_j^2} \psi_D}}
      \quad\text{ for } j=1,\dots, \sigma \; .
    \end{equation*}
    Notice that $\fdiff{r_j} \sect_{(p)} = 0$ and coefficients of
    $F_j$ are in $\smooth_{X\:c\,*}$ on $V$ for $j=1,\dots,\sigma$.
    It then follows from the similar computation in
    \cite{Chan&Choi_ext-with-lcv-codim-1}*{Prop.~2.2.1} or
    \cite{Chan&Choi_injectivity-I}*{Thm.~2.6.1} that
    \begin{align*}
      \eps \int_V
      \frac{\inner{\xi}{f} \:e^{-\phi_D-\vphi_L}}{\abs{\psi_D}^{\sigma
      +\eps}}
      &=\eps \int_V
        \frac{\inner{\xi_p}{g_p} \:e^{-\vphi_L}}{\sect_{(p)}\:\abs{\psi_D}^{\sigma
        +\eps}} \wedge \bigwedge_{j=1}^\sigma \frac{\pi\ibar\:dz_j
        \wedge d\conj{z_j}}{\abs{z_j}^2}
      \\
      &=\eps \int_V \frac{F_0}{\sect_{(p)}\:\abs{\psi_D}^{\sigma +\eps}}
        \prod_{j=1}^\sigma d\log r_j^2 \cdot
        \underbrace{\prod_{j=1}^\sigma \frac{d\theta_j}2}_{=:\:
        \vect{d\theta}}
      \\
      &=\frac\eps{\sigma-1+\eps}
        \int_{V} \frac{F_0}{\sect_{(p)}\:r_1^2 \fdiff{r_1^2}\psi_D}
        \:d\paren{\frac{1}{\abs{\psi_D}^{\sigma-1+\eps}}}
        \prod_{j=2}^{\sigma} d\log r_j^2 
        \cdot \vect{d\theta} \\
      &\overset{\mathclap{\text{int.~by parts}}}=
        \quad\;\;
        \frac{-\eps}{\sigma-1+\eps}
        \int_{V}
        \frac{\alert{F_1}}{\sect_{(p)}\:\abs{\psi_D}^{\sigma-1+\eps}}
        \prod_{j=2}^{\sigma} d\log r_j^2 
        \cdot dr_1 \:\vect{d\theta} \\
      &= \dotsm =
        \frac{(-1)^{\sigma} \eps} {\prod_{j=1}^{\sigma} \paren{\sigma-j+\eps}} 
        \int_{V}
        \frac{F_{\alert{\sigma}}}{\sect_{(p)}\:\abs{\psi_D}^{\eps}}
        \prod_{j=1}^{\sigma} dr_j
        \cdot \vect{d\theta} \; .
    \end{align*}
    Note that $\frac{1}{\sect_{(p)}}$ is integrable on $V$, so the
    integral on the far right-hand-side converges for all $\eps \geq
    0$. 
    Letting $\eps \tendsto 0^+$ on both sides, the desired formula
    then follows from the fundamental theorem of calculus.

    When $f$ or $\xi$ belongs to $\logKX[L] \otimes \smooth_{X\,*}
    \cdot\aidlof-1\paren{V}$, the residue formula in the proposition
    holds even when $\sigma$ is replaced by $\sigma-1$, which implies
    that the integral $\int_V \frac{\inner{\xi}{f}
      \:e^{-\phi_D-\vphi_L}}{\abs{\psi_D}^{\sigma +\eps}}$ converges
    for all $\abs\eps < 1$, hence the last claim.
  \end{proof}

  When restriction to a subspace of codimension $1$ is considered,
  there is a more classical kernel for obtaining the residue formula.
  As an illustration, the residue formula from $X$ to $\lcc|1|'$ is
  proved in the following proposition (which is applied to the case
  where the residue from $\lcc'$ to $\lcc+1'$ is considered later). 
  Recall that $\lcc|1|' =D =\sum_{i \in \Iset||} D_i$, where $D_i =
  \lcS|1|[i]$ and $\Iset|| =\Iset|1|$ are set for convenience.
  \begin{prop} \label{prop:residue-formula-classical-kernel}
    Given any admissible open set $V \subset X$ and any compactly
    supported section $f \in
    \logKX[L] \otimes \smooth_{X \:c\,*} \cdot\aidlof|1|\paren{V}$
    such that $\Res^1(f) = g =\paren{g_i}_{i\in\Iset||}$, one
    has, for any $\xi \in \logKX[L] \otimes \smooth_{X \, *}\paren{V}$,
    \begin{align*}
      \lim_{\eps \tendsto 0^+} \eps \int_V
      \inner{\xi}{f} \:e^{-\phi_D-\vphi_L} e^{-\eps\abs{\psi_D}}
      &=\sum_{i \in \Iset||} \pi
        \int_{D_i \cap V} \inner{\frac{\rs*\xi_i}{\sect_{(i)}}}{\: g_i}
        \:e^{-\vphi_L} \\
      &=\sum_{i \in \Iset||}
      % \underbrace{
        \pi
        \int_{D_i \cap V} \inner{\rs*\xi_i}{\: g_i \sect_{(i)}}
        \:e^{-\phi_{(i)}-\vphi_L}
        % }_{\displaystyle =:
        %   \iinner{\rs*\xi_i}{g_i\sect_{(i)}}_{\mathrlap{D_i \cap V,
        %   \phi_{(i)}}}}
    \end{align*}
    which is finite,
    where $\phi_{(i)} :=\log\abs{\sect_{(i)}}^2$ and
    \begin{equation*}
      \rs*\xi_i := \PRes[D_i](\frac{\xi}{\sect_D}) \cdot \sect_{(i)}
      \in K_{D_i} \otimes \Diff_i D \otimes \res L_{D_i} \otimes
      \smooth_{X\:c\, *}\paren{D_i \cap V} \; .
    \end{equation*}    
  \end{prop}

  \begin{proof}
    As before, it suffices to consider the case where $D_i \cap V
    =\set{z_1 =0}$, $\res{\sect_{(i)}}_V =z_2 \dotsm z_{\sigma_V}$ and
    \begin{equation*}
      f = dz_1 \wedge g_i \sect_{(i)} \; .
    \end{equation*}
    Write also
    \begin{equation*}
      \xi =: dz_1 \wedge \xi_i
      \quad\text{ such that }\;\;
      \res{\xi_i}_{D_i \cap V} =\PRes[D_i](\frac{\xi}{\sect_D}) \cdot
      \sect_{(i)} =\rs*\xi_i \; .
    \end{equation*}
    Essentially the same computation as in Proposition
    \ref{prop:residue-product-X-to-lcS} yields 
    \begin{align*}
      \eps \int_V \inner{\xi}{f} \:e^{-\phi_D-\vphi_L}
      e^{-\eps\abs{\psi_D}}
      =&~\eps \int_V \frac{\inner{\xi_i}{g_i}
         \:e^{-\vphi_L}}{\sect_{(i)}} \wedge
         e^{-\eps\abs{\psi_D}}
         \frac{
         \pi\ibar\:dz_1 \wedge d\conj{z_1}
         }{\abs{z_1}^2}
      \\
      =&~\int_V \frac{\inner{\xi_i}{g_i}
         \:e^{-\vphi_L}}{\sect_{(i)} \:r_1^2 \fdiff{r_1^2} \psi_D} 
         \:d\paren{e^{-\eps\abs{\psi_D}}} \:
         \frac{d\theta_1}{2}
      \\
      \overset{\mathclap{\text{int.~by parts}}}=
       &~\quad\;\;
         -\int_V \fdiff{r_1} \paren{\frac{\inner{\xi_i}{g_i}
         \:e^{-\vphi_L}}{r_1^2 \fdiff{r_1^2} \psi_D} }
         \:\frac{e^{-\eps\abs{\psi_D}}}{\sect_{(i)}} \:dr_1 \:
         \frac{d\theta_1}{2}
      \\
      \mathclap{\xrightarrow{\eps \tendsto 0^+}\;\;}
       &~\quad\;
         -\int_V \fdiff{r_1} \paren{\frac{\inner{\xi_i}{g_i}
         \:e^{-\vphi_L}}{r_1^2 \fdiff{r_1^2} \psi_D} }
         \:\frac{1}{\sect_{(i)}} \:dr_1 \:
         \frac{d\theta_1}{2}
      \\
      =&~\pi \int_{\mathrlap{D_i \cap V}} \;\;\; \frac{\inner{\rs*\xi_i}{g_i}
         \:e^{-\vphi_L}}{\sect_{(i)}}
         =\pi \int_{D_i \cap V} \inner{\rs*\xi_i}{g_i \sect_{(i)}}
         \:e^{-\phi_{(i)}-\vphi_L} \; .
    \end{align*}
    Note that the convergence of the integral obtained right after
    integration by parts follows from the same reasoning as in
    Proposition \ref{prop:residue-product-X-to-lcS}.
  \end{proof}

  Proposition \ref{prop:residue-formula-classical-kernel} facilitates the
  following residue computation.

  \begin{prop} \label{prop:res-formula-dbar-exact-dot-harmonic}
    Given the decomposition $\lcc' = \bigcup_{p\in\Iset} \lcS$, let
    $u_p$ be a \emph{harmonic} $K_{\lcS} \otimes \res L_{\lcS}$-valued
    $(0,q)$-form on $\lcS$ with respect to the norm
    $\norm\cdot_{\lcS}$ for each $p \in \Iset$.
    Given also the decomposition $\lcc+1' = \bigcup_{b\in\Iset+1}
    \lcS+1[b]$, for any $\lcS$ and $\lcS+1[b]$ such that $\lcS+1[b]
    \subset \lcS$, let $\sgn{b:p}$ be the sign such that
    \begin{equation*}
      \PRes[\lcS+1[b]] =\sgn{b:p} \:\PRes[\lcS+1[b] | \lcS] \circ
      \PRes[\lcS] \; ,
    \end{equation*}
    where $\PRes[\lcS+1[b] | \lcS]$ denotes the Poincar\'e residue map
    from $\lcS$ to $\lcS+1[b]$.
    % For a given locally finite cover $\cvr V :=\set{V_i}_{i \in I}$ of
    % $X$ by \emph{admissible} open sets with respect to
    % $(\vphi_L,\psi_D)$ together with a partition of unity
    % $\set{\rho^i}_{i \in I}$ subordinate to it,
    With the finite cover $\cvr V$ and partition of unity
    $\set{\rho^i}_{i \in I}$ given in Section \ref{subsec:notation},
    let $\set{\gamma_{\idx 1.q}}_{\idx 1,q \in I}$ be a
    $\logKX[L]$-valued \v Cech $(q-1)$-cochain with respect to $\cvr
    V$ and set 
    \begin{gather*}
      \rs\gamma_{p; \:\idx 1.q} :=\PRes[\lcS](\frac{\gamma_{\idx 1.q}}{\sect_D})
      \cdot \sect_{(p)} \; , \quad
      v_{p} := \sum_{\idx 1,q \in I} \underbrace{
        \dbar\rho^{i_q} \wedge \dotsm
        \wedge \dbar\rho^{i_2} \cdot \rho^{i_1}
      }_{=: \: \paren{\dbar\rho}^{\idx q.1}} \rs*\gamma_{p;\:\idx 1.q}
      \quad\text{ on } \lcS \\
      \text{and }\quad
      \rs\gamma_{b; \:\idx 1.q} :=\PRes[\lcS+1[b]](\frac{\gamma_{\idx 1.q}}{\sect_D})
      \cdot \sect_{(b)} \; , \quad
      v_{b} := \sum_{\idx 1,q \in I} \paren{\dbar\rho}^{\idx q.1}
      \rs*\gamma_{b;\:\idx 1.q}
      \quad\text{ on } \lcS+1[b] \; .
    \end{gather*}
    Then, after setting $\iinner{\cdot}{\cdot}_{\lcS, \phi_{(p)}}
    :=\iinner{\cdot}{\cdot \:e^{-\phi_{(p)}}}_{\lcS}$ (and
    similarly for $\iinner{\cdot}{\cdot}_{\lcS+1[b], \phi_{(b)}}$), one has
    \begin{equation*}
      \sum_{p\in\Iset} \iinner{\dbar v_{p}}{
        u_p\sect_{(p)}}_{\lcS,\phi_{(p)}}
      =-\sigma \smashoperator[l]{\sum_{b\in\Iset+1}} \iinner{v_{b} \:}{\quad\;
        \smashoperator{\sum_{p\in\Iset \colon \lcS+1[b] \subset
            \lcS}} \;\;
        \sgn{b:p} \: \PRes[\lcS+1[b] | \lcS](\idxup{\diff\psi_{(p)}}.
         u_p) \cdot \sect_{(b)}
      }_{\lcS+1[b], \phi_{(b)}} \; ,
    \end{equation*}
    where $\psi_{(p)} :=\phi_{(p)} -\sm\vphi_{(p)}$ and
    $\sm\vphi_{(p)}$ is some smooth potential on $\Diff_p D$.
  \end{prop}

  \begin{proof}
    Notice that $v_{p}$ is smooth on $\lcS$ but not necessarily
    locally $L^2$ with respect to the weight $e^{-\phi_{(p)}}$.
    An integration by parts is done via the use of Proposition
    \ref{prop:residue-formula-classical-kernel}, which yields 
    \begin{align*}
      &~\sum_{p\in \Iset} \iinner{\dbar v_{p}}{ u_p
        \sect_{(p)}}_{\lcS, \phi_{(p)}}
      \\
      \xleftarrow{\eps \tendsto 0^+}
      &~\sum_{p \in \Iset} \iinner{
        e^{-\eps \abs{\psi_{(p)}}} \:\dbar v_{p}
        }{ u_p \sect_{(p)}}_{\lcS, \phi_{(p)}}
      \\
      =&~\sum_{p \in \Iset} \paren{
         \cancelto{0 \;\;\;(\because~u_p \text{ harmonic, Lemma \ref{lem:su-harmonicity}})}{\iinner{
         \dbar\paren{e^{-\eps \abs{\psi_{(p)}}} \: v_{p}}
         }{ u_p \sect_{(p)}}_{\mathrlap{\lcS, \phi_{(p)}}}}
         \quad\;\; - \eps 
         \iinner{
         e^{-\eps \abs{\psi_{(p)}}} \:v_{p}
         }{\:\idxup{\diff\psi_{(p)}}.  u_p \sect_{(p)}}_{\lcS,
         \phi_{(p)}}
         }
      \\
      =&~-\sum_{p \in \Iset} \sum_{\idx 1,q \in I} \eps \:
         \iinner{
         e^{-\eps \abs{\psi_{(p)}}} \: % \paren{\dbar\rho}^{\idx q.1}
         \rs*\gamma_{p;\:\idx 1.q}
         }{\:
         \idxup{\diff\rho},[\idx 1.q] .
         \paren{\idxup{\diff\psi_{(p)}}.  u_p \sect_{(p)}}
         }_{\lcS, \phi_{(p)}}
      \\
      \xrightarrow[\text{Prop.~\ref{prop:residue-formula-classical-kernel}}]{\eps
      \tendsto 0^+} 
      &~-\smashoperator[l]{\sum_{\idx 1,q \in I}} \sum_{p \in \Iset}
        \sum_{k=\sigma +1}^{\mathclap{\sigma_{V_{\idx 1.q}}}} \sigma
        \iinner{
        \PRes[p(k)](
        \frac{\rs*\gamma_{p;\:\idx 1.q}}{\sect_{(p)}}
        )
        }{\:
        \idxup{\diff\rho},[\idx 1.q] .
        \PRes[p(k)](\idxup{\diff\psi_{(p)}}.  u_p)
        }_{\lcS \cap \set{z_{p(k)} =0}}
        \; ,
    \end{align*}
    where $\idxup{\diff\rho},[\idx 1.q] . \cdot$ is the adjoint
    of $\paren{\dbar\rho}^{\idx q.1} \cdot$, and $\PRes[p(k)]$ denotes
    the Poincar\'e residue map from $\lcS$ to $\lcS \cap \set{z_{p(k)}=0}$. 
    The last limit is justified as follows.
    On the admissible open set $V_{\idx 1.q}$, consider a holomorphic
    coordinate system $(z_1, \dots, z_n)$ such that $\lcS \cap V_{\idx
    1.q}
    =\set{z_{p(1)} = \dotsm =z_{p(\sigma)} =0}$ and
    $\sect_{(p)} =z_{p(\sigma+1)} \dotsm z_{p(\sigma_V)}$ (write
    $\sigma_{V}$ for $\sigma_{V_{\idx 1.q}}$ for convenience).
    Note that
    \begin{equation*}
      \diff\psi_{(p)} =\sum_{k =\sigma +1}^{\sigma_V}
      \frac{dz_{p(k)}}{z_{p(k)}} -\diff\sm\vphi_{(p)} \quad\text{ on }
      V_{\idx 1.q} \; .
    \end{equation*}
    It follows that, on $\lcS \cap V_{\idx 1.q}$,
    \begin{equation*}
      \text{coef.~of }\:
      \idxup{\diff\rho},[\idx 1.q].
      \paren{\idxup{\diff\psi_{(p)}}.  u_p \sect_{(p)}}
      \in
      \smooth_{\lcS \:c} \cdot\res{\defidlof{\lcc+2'}}_{\lcS}
      \begin{aligned}[t]
        &=\smooth_{\lcS \:c} \cdot\mtidlof<\lcS>{\vphi_L} \cdot
        \res{\defidlof{\lcc+2'}}_{\lcS} \;\;\footnotemark
        \\
        &=\smooth_{\lcS \:c} \cdot\aidlof|1|<\lcS>{\vphi_L}[\psi_{(p)}]
      \end{aligned}
    \end{equation*}%
    \footnotetext{
      Recall that $\defidlof{\lcc+2'}$ is generated on $X$ by
      $\sect_{(b)}$ treated as local
      functions for all $b \in \Iset+1$.
      On an admissible open set $V$, one has $\defidlof{\lcc+2'}
      =\genbyd{z_{b(\sigma+2)} \dotsm
        z_{b(\sigma_V)}}{b \in \Iset+1 \text{ such that } \lcS+1[b] \cap
        V \neq \emptyset}$.
      % (see page
      % \pageref{page:notation-permutation-index} for the notation).
    }%
    and, therefore, one can apply Proposition
    \ref{prop:residue-formula-classical-kernel} (with $\lcS$ in place
    of $X$, $\psi_{(p)}$ in place of $\psi_D$) to each inner product
    $\eps \iinner{e^{-\eps \abs{\psi_{(p)}}} \dotsm}{\: \dotsm
      \idxup{\diff\psi_{(p)}} . \dotsm \sect_{(p)}}_{\lcS,\phi_{(p)}}$.
    Note also that the factor $\sigma$ comes from the normalisation of
    the norm on each lc center ($\norm\cdot_{\lcS}^2
    =\frac{\pi^\sigma}{(\sigma -1)!} \int_{\lcS} \dotsm$ and
    $\norm\cdot_{\lcS+1[b]}^2 =\frac{\pi^{\sigma+1}}{\sigma!}
    \int_{\lcS+1[b]} \dotsm$).


    On each admissible open set $V_{\idx 1.q}$, the intersection $\lcS
    \cap \set{z_{p(k)} = 0}$ is a $(\sigma+1)$-lc center $\lcS+1[b_{p,k}]
    \cap V_{\idx 1.q}$ ($\neq \emptyset$), uniquely determined by the
    choices of $p\in \Iset$ (such that $\lcS \cap V_{\idx 1.q} \neq
    \emptyset$, so $\binom{\sigma_V}{\sigma}$ choices) and $k
    =\sigma+1, \dots, \sigma_V$ (so $\sigma_V-\sigma$ choices).
    To get an indexing in terms of $b \in \Iset+1$ (such that
    $\lcS+1[b] \cap V_{\idx 1.q} \neq \emptyset$, so
    $\binom{\sigma_V}{\sigma +1}$ choices), note that each $\lcS+1[b]
    \cap V_{\idx 1.q}$ is contained in $\sigma +1$ distinct
    $\sigma$-lc centers $\lcS[p_{b,j}]$ for $j=1,\dots,\sigma+1$
    (apparently, $\sigma +1$ choices) such that
    \begin{equation*}
      \lcS+1[b] \cap V_{\idx 1.q} = \lcS[p_{b,j}] \cap \set{z_{b(j)} = 0} \; .
    \end{equation*}
    (One can verify $\sum_{p \in \Iset} \sum_{k=\sigma
      +1}^{\sigma_{V}} \dotsm = \sum_{b \in
      \Iset+1} \sum_{j=1}^{\sigma +1} \dotsm$ by first noting that
    $\binom{\sigma_V}{\sigma} (\sigma_V -\sigma)
    =\binom{\sigma_V}{\sigma +1} (\sigma+1)$.)
    With such choice of indexing, one has
    \begin{equation*}
      \frac{\rs*\gamma_{b;\: \idx 1.q}}{\sect_{(b)}}
      :=\PRes[\lcS+1[b]](\frac{\gamma_{\idx 1.q}}{\sect_D})
      =\sgn{b:p_{b,j}} \:
      \PRes[b(j)](\frac{\rs*\gamma_{p_{b,j};\:\idx
          1.q}}{\sect_{(p_{b,j})}})
    \end{equation*}
    (noticing that % $\sect_{(b)} =\sect_{(\sigma+1 : b)}$,
    % $\sect_{(p_{b,j})} =\sect_{(\sigma : p_{b,j})}$ and
    $\sect_{(p_{b,j})} = z_{b(j)} \sect_{(b)}$).
    As a result, the expression in question becomes
    \begin{align*}
      &-\smashoperator[l]{\sum_{\idx 1,q \in I}} \sum_{b \in \Iset+1}
        \sum_{j=1}^{\sigma +1} \sigma
        \iinner{ \sgn{b:p_{b,j}}\:
        \frac{\rs*\gamma_{b;\:\idx 1.q}}{\sect_{(b)}}
        }{\: 
        \idxup{\diff\rho},[\idx 1.q] .
        \PRes[b(j)](\idxup{\diff\psi_{(p_{b,j})}} . u_{p_{b,j}})
        }_{\lcS+1[b]}
      \\
      =&-\smashoperator[l]{\sum_{\idx 1,q \in I}}
        \sum_{b \in \Iset+1}
        \sigma
        \iinner{
        \paren{\dbar\rho}^{\idx q.1} \rs*\gamma_{b;\:\idx 1.q}
        \:}{  \sum_{j=1}^{\sigma +1} \sgn{b:p_{b,j}}\:
        \PRes[b(j)](\idxup{\diff\psi_{(p_{b,j})}} . u_{p_{b,j}})
        \cdot \sect_{(b)}
        }_{\lcS+1[b], \phi_{(b)}}
      \\
      =&-\sigma \sum_{b \in \Iset+1} \iinner{
        v_b
        \:}{\quad\;
        \smashoperator{\sum_{p\in\Iset \colon \lcS+1[b] \subset
        \lcS}} \;\;
        \sgn{b:p}\:
        \PRes[\lcS+1[b] | \lcS](\idxup{\diff\psi_{(p)}}.  u_{p})
        \cdot \sect_{(b)}
        }_{\lcS+1[b], \phi_{(b)}} \; . \qedhere
    \end{align*}
  \end{proof}


}


\subsection{Restriction of harmonic differential forms to hypersurfaces}\label{subsec:harmonic}

{
  
  Let $(X,\omega)$ be a K\"ahler manifold equipped with a holomorphic
  line bundle $L$ equipped with a smooth potential $\varphi_L$ such
  that $\ibddbar\varphi_L\ge0$ and let $D$ be an snc divisor in $X$
  written as 
  \begin{equation*}
    D=\sum_{p\in I_D}D_p \; ,
  \end{equation*}
  where $D_p$ is an irreducible component for $p\in I_D$.
  We define the map $\HRes_p \colon \mathscr
  A_X^{0,q}(X,K_X\otimes L)\rightarrow \mathscr
  A_{D_p}^{0,q-1}(D_p,K_{D_p}\otimes L\vert_{D_p})$ by 
  \begin{equation*}
    \HRes_p(u)
    =
    \PRes[D_p](\idxup{\partial\psi_D} . u) \;\;\;\text{for}\;\;\;p\in
    I_D \; ,
  \end{equation*}
  where $\mathcal{R}_{D_p}$ is the Poincar\'e residue map (see, for
  example, \cite{Griffiths&Harris}*{p.147} or \cite{Kollar_Sing-of-MMP}*{\S 4.18}). 
  Notice that, as in \cite{Chan&Choi_injectivity-I}*{\textsection2.6},
  the map $\mathcal R_{D_p}$ is extended to send sections of
  $K_X\otimes\overline{\bold\Omega}_X^q$ to those of
  $K_{D_p}\otimes\overline{\bold\Omega}_X^q\vert_{D_p}$. 
  Let $(U;z^1,\dots,z^n)$ be a local holomorphic coordinate system
  around $x\in D_p\subset X$ satisfying 
  \begin{enumerate}[label=(\roman*), ref=\roman*]
  \item  \label{item:admissible-open-U} % [(\romannumeral1)]
    $D_p\cap U=\{z\in U:z^1=0\}$ and 
    $D\cap U=\set{z^1\cdots z^{\sigma_U}=0}$,
    and
  \item  \label{item:psi_D-in-admissible-open-U} % [(\romannumeral2)]
    $\psi_D=\sum_{j=1}^{\sigma_U}
    \log\abs{z^j}^2-\sm\varphi_D$ on $U$.
  \end{enumerate}
  Since $\del\psi_D=\sum_{j=1}^{\sigma_U}\frac{dz^j}{z^j}-\del\sm\varphi_D$, it follows that
  \begin{equation*}
    \HRes_p(u)
    =
    \mathcal{R}_{D_p}\left(\idxup{\partial\psi_D}.u\right)
    =
    \mathcal{R}_{D_p}\paren{\idxup{\frac{dz^1}{z^1}}.u}
    =
    \paren{\idxup{dz^1}.\widetilde u_p}\big\vert_{D_p} \; ,
  \end{equation*}
  where $\rs u_p := \fdiff{z^1} \ctrt u$ (so $u=dz^1\wedge\widetilde{u}_p$).
  In particular, $\HRes_p$ does not depend on the
  choice of the Hermitian metric $\sm\varphi_{D}$. 
  It follows from the above formula that
  $\HRes_p(u)$ is actually a $K_{D_p}\otimes
  L\vert_{D_p}$-valued $(0,q-1)$-form on $D_p$ (not only a
  $\overline{\bold\Omega}_X^{q-1}\vert_{D_p}$-valued section). 



  First we notice that $\dfadj$-closedness is preserved by
  $\HRes_p$ on a K\"ahler manifold.
  \begin{prop} \label{prop:harmonic-residue}
    If $u$ is a $\dfadj$-closed $K_X\otimes L$-valued $(0,q)$-form on
    $X$, then $\HRes_p(u)$ is a $\dfadj$-closed
    $K_{D_p}\otimes L\vert_{D_p}$-valued $(0,q-1)$-form on $D_p$. 
  \end{prop}

  \begin{proof}
    It is enough to show that it vanishes at the given point $x\in D_p$.
    Let $(z^1,\dots,z^n)$ be a local holomorphic coordinate system around
    $x$ in $X$ satisfying \eqref{item:admissible-open-U} % (\romannumeral1)
    and \eqref{item:psi_D-in-admissible-open-U}. % (\romannumeral2).
    Since $(D_p,\omega\vert_{D_p})$ is a smooth $(n-1)$-dimensional
    K\"ahler manifold, by a linear change of coordinates
    $(z^1,\ldots,z^n)$ and a quadratic change of coordinates in
    $(z^2,\dots,z^n)$ of $D_p$ we may assume that
    \begin{enumerate}[resume*]
    \item % [(\romannumeral3)]
      $g_{i\bar j}(x)=I_n$ where $I_n$ is the $n\times n$ identity matrix.
    \item % [(\romannumeral4)]
      $dg_{\alpha\bar\beta}(x)=0$ for $2\le\alpha\le n$ and $2\le\beta\le n$.
    \end{enumerate}
    Since $\displaystyle\idxup{dz^1}=g^{\bar
      j1}\pd{}{\overline{z^j}}$ (under Einstein summation convention),
    we have
    \begin{equation}\label{E:local_expression}
      dz^1 \wedge \paren{\idxup{dz^1} . \widetilde u_p}_{\bar j_1,\ldots,\bar j_{q-1}}
      =
      g^{\bar j1}
      u_{\bar j\bar j_1,\ldots,\bar j_{q-1}} \; .
    \end{equation}
    % where $\beta_1,\ldots,\beta_{q-1}$ run from $2$ to $n$.
    For the sake of convenience, let the Latin indices $i,j,k,...$ run
    from $1$ to $n$ and let the Greek indices
    $\alpha,\beta,\gamma,...$ run from $2$ to $n$ in this proof. 
    Let $\varphi_{K_X}$ and $\varphi_{K_{D_p}}$ be respectively the
    potentials on $K_X$ and $K_{D_p}$ induced by the K\"ahler metric
    $\omega$, which are written as 
    \begin{equation*}
      \varphi_{K_X}
      =
      \log\det\paren{g_{i\bar j}}_{1\le i,j\le n}
      \;\;\;\text{and}\;\;\;
      \varphi_{K_{D_p}}
      =
      \log\det\paren{g_{\alpha\bar\beta}}_{2\le\alpha,\beta\le n} \; .
    \end{equation*}
    This yields, at the given point $x\in D_p$,
    \begin{align*}
      \del_\gamma\varphi_{K_X}
      &=
	\del_\gamma\log\det g
	=
	\Tr\paren{\del_\gamma g\cdot g^{-1}}
      \\
      &=
	\sum_{i,j=1}^n\pd{g_{i\bar j}}{z^\gamma} g^{\bar j i}
	\;\;\overset{\mathclap{(\text{at } x)}}=\;\;
	g^{\bar11}\del_\gamma g_{1\bar1}
	+
	\sum_{\alpha,\beta=2}^n\pd{g_{\alpha\bar\beta}}{z^\gamma}g^{\alpha\bar\beta}
      \\
      &=
	g^{\conj11}\del_\gamma g_{1\conj1}
	+
	\del_\gamma\log\det\paren{g\vert_{D_p}}
	=
	g^{\conj11}\del_\gamma g_{1\conj1}
	+
	\del_\gamma\varphi_{K_{D_p}}
	\;\;\overset{\mathclap{(\text{at } x)}}=\;\;
	g^{\conj11}\del_\gamma g_{1\conj1} \; .
    \end{align*}
    % This implies that
    % \begin{equation*}
    %   \del_\ell\varphi_{K_X}
    %   =
    %   \del_\ell\log\det\paren{g_{i\bar j}}_{1\le i,j\le n}
    %   =
    %   \sum_{i,j=1}^n\pd{g_{i\bar j}}{z^\ell}g^{\bar ji}
    %   \;\;\;\text{and}
    %   \;\;\;
    %   \del_\ell\varphi_{K_{D_p}}
    %   =
    %   \sum_{\alpha,\beta=2}^n
    %   \pd{g_{\alpha\bar\beta}}{z^\ell}g^{\bar\beta\alpha}.	
    % \end{equation*}
    % Thus we have
    % \begin{equation*}
    %   \sum_{i,j}\pd{g_{i\bar j}}{z^\ell}g^{\bar ji}
    %   =
    %   \del_lg_{1\bar1}
    %   +
    %   \sum_{\alpha=2}^n\pd{g_{\alpha\bar\alpha}}{z^l}g^{\alpha\bar\alpha}
    %   =
    %   \del_lg_{1\bar1}
    %   +
    %   \del_l\varphi_{K_{D_p}}
    %   \;\;\;
    %   \text{and}
    %   \;\;\;
    %   \del_\beta\varphi_{K_{D_p}}=0\;\;\;\text{at}\;\;x.
    % \end{equation*}
    % It follows from \eqref{E:local_expression} and the definition of $\dfadj$ (cf.~\cite{Siu}) that
    % for any multi-indices $\ov{\boldsymbol\beta}_{q-2}=(\bar\beta_1,\ldots,\bar\beta_{q-2})$,
    \newcommand{\bbeta}{{\boldsymbol{\beta}}}%
    \newcommand{\KDp}{{\smash[b]{K_{D_p}}}}%
    % \renewcommand{\CancelColor}{\color{Gray}}%
    Choose a local frame of $L$ in a neighbourhood of $x$ in $X$
    such that
    \begin{equation*}
      \diff\vphi_L(x) = 0  \; .
    \end{equation*}
    It follows from \eqref{E:local_expression} and the definition of
    $\dfadj$ on $D_p$ (see, for example,
    \cite{Siu}*{(1.3.2)}) that, for any 
    multi-indices ${\bbeta}_{q-2} =(\idx[\beta]1,{q-2})$ and at the
    given point $x \in D_p$,
    \begin{align*}
      dz^1 \wedge \paren{\dfadj \HRes_p(u)}_{\conj\bbeta_{q-2}}
      &=
        -g^{\conj\beta\gamma}
	\nabla_\gamma 
	\paren{
        g^{\alert{\conj j} 1}
        u_{\alert{\conj j} \conj\beta\ov{\boldsymbol\beta}_{q-2}}
	}
      \\
      &=-g^{\conj\beta \gamma}\diff_\gamma \paren{g^{\alert{\conj j} 1}
        u_{\alert{\conj j} \conj\beta \conj\bbeta_{q-2}}}
        +g^{\conj\beta \gamma} \smash[t]{\cancelto{0}{\paren{
        \diff_\gamma \vphi_\KDp +\diff_\gamma \vphi_L
        }}}
        \cdot g^{\alert{\conj j} 1} u_{\alert{\conj j} \conj\beta \conj\bbeta_{q-2}}
      \\
      &=-g^{\conj\beta \gamma} g^{\alert{\conj j} 1} \diff_\gamma
        u_{\alert{\conj j} \conj\beta \conj\bbeta_{q-2}}
        -g^{\conj\beta \gamma} \diff_\gamma g^{\alert{\conj j} 1}
        \cdot u_{\alert{\conj j} \conj\beta \conj\bbeta_{q-2}}
      \\
      &=-g^{\conj\beta \gamma} g^{\conj 1 1}
        \diff_\gamma u_{\conj 1 \conj\beta \conj\bbeta_{q-2}}
        +g^{\conj\beta \gamma} g^{\alert{\conj j k}} \diff_\gamma
        g_{\alert k \conj 1} \cdot g^{\conj 1 1} u_{\alert{\conj j}
        \conj\beta \conj\bbeta_{q-2}}
      \\
      &=g^{\conj 1 1} \paren{
        -g^{\conj\beta \gamma} \diff_\gamma
        u_{\conj 1 \conj\beta \conj\bbeta_{q-2}}
        +g^{\conj\beta\gamma} g^{\conj 1 1} \diff_\gamma
        g_{1\conj 1} \cdot
        u_{\conj 1 \conj\beta \conj\bbeta_{q-2}}
        +g^{\conj\beta\gamma} g^{\alert{\conj j} \alpha}
        \diff_\gamma g_{\alpha \conj 1} \cdot
        u_{\alert{\conj j} \conj\beta \conj\bbeta_{q-2}}
        }
      \\
      &=g^{\conj 1 1} \paren{
        g^{\alert{\conj k j}} \diff_{\alert{j}}
        u_{\alert{\conj k} \conj 1 \conj\bbeta_{q-2}}
        -g^{\alert{\conj k j}} \diff_{\alert{j}}\vphi_{K_X} \cdot
        u_{\alert{\conj k} \conj 1 \conj\bbeta_{q-2}}
        }
        +g^{\conj 1 1} g^{\conj \gamma \gamma} g^{\conj\alpha
        \alpha} \diff_{\gamma} g_{\alpha \conj 1} \cdot
        u_{\conj\alpha \conj\gamma \conj\bbeta_{q-2}}
      \\
      &=-g^{\conj 1 1} \paren{\dfadj u}_{\conj 1 \conj\bbeta_{q-2}}
        +g^{\conj 1 1} g^{\conj \gamma \gamma} g^{\conj\alpha
        \alpha} \diff_{\gamma} g_{\alpha \conj 1} \cdot
        u_{\conj\alpha \conj\gamma \conj\bbeta_{q-2}} \; .
    \end{align*}
    Since $\del_\gamma g_{\alpha\conj1}$ is symmetric in $\alpha,
    \gamma$ (for $X$ being K\"ahler) while $u_{\conj\alpha\conj\gamma
      \conj{\bbeta}_{q-2}}$ is anti-symmetric in $\alpha, \gamma$, the
    last term in the expression above vanishes.
    As $\dfadj u = 0$ on $X$ by assumption, the proof is thus
    completed after applying $\fdiff{z^1} \ctrt$ to both sides. \qedhere
    % \begin{align*}
    %   \paren{\dfadj\HRes_p(u)}_{\ov{\boldsymbol\beta}_{q-2}}
    %   &=
    %   -g^{\bar\beta\alpha}
    %   \nabla_\alpha 
    %   \paren{
    %   g^{\bar j1}
    %   u_{\bar j\bar\beta\ov{\boldsymbol\beta}_{q-2}}
    %	}
    %   \\
    %   &
    %   =
    %   -g^{\bar\beta\alpha}
    %   \del_\alpha 
    %   \paren{
    %   g^{\bar j1}
    %   u_{\bar j\bar\beta\ov{\boldsymbol\beta}_{q-2}}
    %	}
    %   +
    %   g^{\bar\beta\alpha}
    %   \paren{\del_\alpha\varphi_{K_{D_p}}-\del_\alpha\varphi_L}
    %   g^{\bar j1}
    %   u_{\bar j\bar\beta\ov{\boldsymbol\beta}_{q-2}}
    %   \\
    %   &
    %   =
    %   -
    %   \sum_{\beta=2}^n
    %   \del_\beta
    %   \paren{
    %   g^{\bar j1}
    %   u_{\bar j\bar\beta\ov{\boldsymbol\beta}_{q-2}}
    %	}
    %   -
    %   \sum_{\beta=2}^n
    %   \paren{\del_\beta\varphi_L}
    %   u_{\bar1\bar\beta\ov{\boldsymbol\beta}_{q-2}}\;\;\;\text{at}\;\;x.
    % \end{align*}
    % The first term is computed as
    % \begin{align*}
    %   -
    %   \sum_{\beta=2}^n
    %   \del_\beta
    %   &
    %   \paren{
    %   g^{\bar j1}
    %   u_{\bar j\bar\beta\ov{\boldsymbol\beta}_{q-2}}
    %	}
    %   =
    %   -
    %   \sum_{\beta=2}^n
    %   \paren{
    %   \del_\beta g^{\bar j1}
    %   u_{\bar j\bar\beta\ov{\boldsymbol\beta}_{q-2}}
    %   +
    %   g^{\bar j1}
    %   \del_\beta
    %   u_{\bar j\bar\beta\ov{\boldsymbol\beta}_{q-2}}
    %	}
    %   \\
    %   &=
    %   \sum_{\beta=2}^n
    %   \paren{
    %   g^{\bar jk}\paren{\del_\beta g_{k\bar l}}g^{\bar l1}
    %   u_{\bar j\bar\beta\ov{\boldsymbol\beta}_{q-2}}
    %   -
    %   \del_\beta
    %   u_{\bar1\bar\beta\ov{\boldsymbol\beta}_{q-2}}
    %	}
    %   \\
    %   &=
    %   \sum_{\beta=2}^n
    %   \paren{
    %   \paren{\del_\beta g_{1\bar1}}
    %   u_{\bar1\bar\beta\ov{\boldsymbol\beta}_{q-2}}
    %   -
    %   \del_\beta
    %   u_{\bar1\bar\beta\ov{\boldsymbol\beta}_{q-2}}
    %	}
    %   +
    %   \sum_{\beta,\gamma=2}^n
    %   \paren{\del_\beta g_{\gamma\bar1}}
    %   u_{\bar\gamma\bar\beta\ov{\boldsymbol\beta}_{q-2}}	.
    % \end{align*}
    % Since $\del_\beta g_{\gamma\bar1}$ is symmetric in $\beta, \gamma$ and $u_{\bar\gamma\bar\beta\bar\beta_1,\ldots,\bar\beta_{q-2}}$ is anti-symmetric in $\beta, \gamma$, the last term vanishes.
    % It follows that
    % \begin{align*}
    %   \paren{\dfadj\HRes_p(u)}_{\ov{\boldsymbol\beta}_{q-2}}
    %   &=
    %   \sum_{\beta=2}^n
    %   \paren{
    %   \paren{\del_\beta g_{1\bar1}}
    %   u_{\bar1\bar\beta\ov{\boldsymbol\beta}_{q-2}}
    %   -
    %   \del_\beta
    %   u_{\bar1\bar\beta\ov{\boldsymbol\beta}_{q-2}}
    %	}
    %   -
    %   \sum_{\beta=2}^n
    %   \paren{\del_\beta\varphi_L}
    %   u_{\bar1\bar\beta\ov{\boldsymbol\beta}_{q-2}}
    %   \\
    %   &=
    %   \paren{\dfadj u}_{\bar1\bar\beta\ov{\boldsymbol\beta}_{q-2}}=0.
    % \end{align*}
    % Indeed, at the given point $x$, we have
    % \begin{align*}
    %   \paren{\dfadj u}_{\bar1\ov{\boldsymbol\beta}_{q-2}}
    %   &=
    %   -g^{\bar jk}\paren{\nabla_ku_{\bar j\bar 1\ov{\boldsymbol\beta}_{q-2}}}
    %   =
    %   -g^{\bar jk}
    %   \paren{
    %   \del_ku_{\bar j\bar1\ov{\boldsymbol\beta}_{q-2}}
    %   -
    %   \paren{
    %			\del_k\varphi_{K_X}
    %			-
    %			\del_k\varphi_L
    % }
    %   u_{\bar j\bar1\ov{\boldsymbol\beta}_{q-2}}
    %	}
    %   \\
    %   &=
    %   -
    %   \sum_{\beta=2}^n
    %   \paren{
    %   \del_\beta
    %   u_{\bar\beta\bar1\ov{\boldsymbol\beta}_{q-2}}
    %   -
    %   \paren{
    %			\del_\beta g_{1\bar1}
    %			-
    %			\del_\beta\varphi_L
    % }
    %   u_{\bar\beta\bar1\ov{\boldsymbol\beta}_{q-2}}
    %	}.
    % \end{align*}
    % This completes the proof.
  \end{proof}


  Furthermore, we claim that, if
  % $u\in\mathcal{H}^{n,q}(X;L)_{\varphi_L}$ with
  % $\ibddbar\varphi_L\ge0$
  $u$ satisfies $\dbar u = 0$ and $\nabla^{(0,1)}u = 0$, then
  $\HRes_p(u)$ is $\dbar$-closed.
  % One can notice that $u$ is $\dbar$-closed, then so is
  % $\HRes_p(u)$. 
  This is shown via the following formula, which is a special case and
  a slight variant of \cite{Donnelly&Xavier}*{(2.4)} and
  \cite{Ohsawa&Takegoshi-spectral_seq}*{Prop.~1.5} (see also
  \cite{Takegoshi_higher-direct-images}*{(1.9)} and
  \cite{Matsumura_injectivity-Kaehler}*{Lemma 2.1}). 
  

  \newcommand{\lcSb}{\lcS+1[b]}
  \newcommand{\idxj}{\idx[\conj j]}
  
  \begin{lemma}[cf.~\cite{Donnelly&Xavier}*{(2.4)},
    \cite{Ohsawa&Takegoshi-spectral_seq}*{Prop.~1.5},
    \cite{Takegoshi_higher-direct-images}*{(1.9)} and
    \cite{Matsumura_injectivity-Kaehler}*{Lemma
      2.1}] \label{lem:commutator-dbar-ctrt}
    Let $\vphi$ be a smooth function and $u$ be a smooth
    ($K_X$-valued) $(0,q)$-form on a K\"ahler manifold.
    They satisfy the formula
    \begin{equation*}
      \dbar\paren{\idxup{\diff\vphi}.  u}
      =\idxup{\ibddbar\vphi} . u
      -\idxup{\diff\vphi} . \paren{\dbar u}
      +\idxup{\diff\vphi} \cdot \nabla^{(0,1)}_\bullet u \; ,
    \end{equation*}%
    or, when a local holomorphic coordinate system is fixed and
    the Einstein summation convention is applied, 
    \begin{equation*}
      \paren{\dbar\paren{\idxup{\diff\vphi} . u}}_{\conj J_{q}}
      =\sum_{\nu=1}^q \diff^{\conj\ell} \diff_{\conj j_\nu} \vphi \:
      u_{\idxj 1[\dotsm (\conj \ell)_\nu].q}
      -\diff^{\conj\ell}\vphi  \:\paren{\dbar u}_{\conj\ell\conj J_q}
      +\diff^{\conj\ell}\vphi \:\nabla_{\conj\ell} u_{\conj J_q} 
    \end{equation*}
    for any multi-indices $J_q = (\idx[j]1,q)$, pointwisely.
  \end{lemma}

  \begin{proof}
    A direct computation yields
    \begin{align*}
      \paren{\dbar\paren{\idxup{\diff\vphi} . u}}_{\conj
      J_{q}}
      &=\sum_{\nu=1}^q (-1)^{\nu-1} \diff_{\conj j_\nu}
        \paren{\idxup{\diff\vphi}.  u}_{\idxj 1[\dotsm \widehat
        {\conj j}_\nu].q}
        =\sum_{\nu=1}^q (-1)^{\nu-1} \diff_{\conj j_\nu}
        \paren{\diff_{\ell}\vphi \: u^\ell_{\;\idxj 1[\dotsm
        \widehat{\conj j}_\nu].q}}
      \\
      &=\sum_{\nu=1}^q (-1)^{\nu-1} \paren{
        \diff_{\conj j_\nu}\diff_{\ell}\vphi \: u^{\ell}_{\;\idxj 1[\dotsm
        \widehat {\conj j}_\nu].q}
        +\diff_{\ell}\vphi \: \nabla_{\conj j_\nu} u^\ell_{\;\idxj 1[\dotsm
        \widehat {\conj j}_\nu].q}
        }
      \\
      &=\sum_{\nu=1}^q
        \diff^{\conj \ell}\diff_{\conj j_\nu}\vphi \: u_{\idxj 1[\dotsm
        (\conj\ell)_\nu].q}
        -\diff^{\conj\ell}\vphi \sum_{\nu=1}^q (-1)^{\nu} 
        \nabla_{\conj j_\nu} u_{\conj\ell \idxj 1[\dotsm
        \widehat {\conj j}_\nu].q}
        \begin{aligned}[t]
          &-\diff^{\conj\ell}\vphi \: \nabla_{\conj \ell} u_{\conj
            J_q} \\
          &+\diff^{\conj\ell}\vphi \: \nabla_{\conj \ell}
          u_{\conj J_q}
        \end{aligned}
      \\
      &=\sum_{\nu=1}^q
        \diff^{\conj \ell}\diff_{\conj j_\nu}\vphi \: u_{\idxj 1[\dotsm
        (\conj\ell)_\nu].q}
        -\diff^{\conj\ell}\vphi
        \:\paren{\dbar u}_{\conj\ell\conj J_q}
        +\diff^{\conj\ell}\vphi \: \nabla_{\conj \ell} u_{\conj
        J_q} \; . \qedhere
    \end{align*}
    % as desired.
  \end{proof}



  We see that $\HRes_p(u)$ is $\dbar$-closed by
  putting $z^1$ in place of $\vphi$ in Lemma \ref{lem:commutator-dbar-ctrt}.
  The following theorem is then immediate.
  \begin{thm} \label{thm:residue-harmonic}
    If $u$ is a harmonic $K_X\otimes L$-valued $(0,q)$-form on $X$ with
    respect to $\vphi_L$ and $\omega$ such that $\nabla^{(0,1)}u=0$,
    then $\HRes_p(u)$ is a harmonic $K_{D_p}\otimes
    L\vert_{D_p}$-valued $(0,q-1)$-form on $D_p$ with respect to
    $\varphi\vert_{D_p}$ and $\omega\vert_{D_p}$.
  \end{thm}

  \begin{proof}
    From the above discussion, $\HRes_p(u)$ is
    $\dbar$- and $\dfadj$-closed on $D_p$.
    Since $\varphi_L$ is smooth and $D_p$ is compact, it follows that
    $\HRes_p(u)\in\Dom\dbar^*$ with
    respect to $\varphi_L\vert_{D_p}$ and $\omega\vert_{D_p}$.
    This completes the proof.
  \end{proof}

  \begin{remark}
    When $\varphi_L$ has the singularity property described in
    \cite{Chan&Choi_injectivity-I}*{\S 2.2 item (2)} for $\varphi_F$,
    i.e.~$\varphi_L$ has only neat analytic singularities such that
    $P_L:=\varphi_L^{-1}(-\infty)$ is a divisor with $P_L+D$ having
    snc and that $P_L$ contains no components of $D$, the claim that
    $\HRes_p(u)\in\Dom\dbar^*$ with
    respect to $\norm\cdot_{D_p}:=\norm\cdot_{D_p,\varphi_L,\omega}$
    still holds true (under the assumption that $\omega\vert_{D_p}$ is
    a complete K\"ahler form on $D_p\setminus P_L$).
    Indeed, $\HRes_p(u)$ can be shown to be $L^2$
    with respect to $\norm\cdot_{D_p}$ by the arguments in
    \cite{Chan&Choi_injectivity-I}*{Prop.~3.2.3, Remark 3.2.4 and
      Prop.~3.3.2} (with $u$ here in place of $\frac{\rs u}{\sect_D}$
    there).
    With $\dfadj$ (with respect to $\varphi_L\vert_{D_p}$ and
    $\omega\vert_{D_p}$) being a smooth operator on $D_p\setminus P_L$
    % (different from the situation in Lemma \ref{lem:su-harmonicity})
    and $\omega\vert_{D_p}$ being complete,
    $\HRes_p(u)\in\Dom\dbadj$ follows
    from the classical arguments.
  \end{remark}

}




% \section{Proof of the Main Result}\label{sec:proof}

% \subsection{Proof of Corollary \ref{cor:main}}\label{subsec:n2}

% Corollary \ref{cor:main} can be  proved by repeating the same argument as in the proof of Theorem \ref{thm:main}. 
% Nevertheless, in this subsection, we deduce Corollary \ref{cor:main} from Theorem \ref{thm:main} 
% using the previous work \cite[Theorem 1.6]{Mat}. 




% \begin{proof}[Proof of Corollary \ref{cor:main}]
% %In the proof, we freely use the notation in Conjecture \ref{conj:fujino}. 
% Let us consider the following commutative diagram  induced by 
% the short exact sequence $0 \to K_{X} \to K_{X}\otimes D \to K_{D} \to 0 $ and the multiplication map: 
% \begin{align*}
% \vcenter{ \xymatrix{
% &\ar[d] & \ar[d]\\
% &
% H^q(X,  K_{X}\otimes F )\ar[d]^-{}\ar[r] ^-{\otimes s}
% \ar[d]^-{\otimes \sect_D}
% &H^q(X, K_{X}  \otimes F^{ \otimes{(m+1)}} )\ar[d]\\ 
% &H^q(X,  K_{X}\otimes D \otimes F)
% \ar[d]^-{}\ar[r] ^-{\otimes s}
% &H^q(X,  K_{X} \otimes D \otimes F^{\otimes{(m+1)}}) \ar[d]\\ 
% &H^q(D, K_{D}\otimes F )
% \ar[d]\ar[r]^-{\otimes s|_{D} } 
% & H^q(D,  K_{D}\otimes F^{\otimes(m+1)} ). \ar[d]\\ 
% & & 
% }}
% \end{align*}
% The line bundle $M:=F^{\otimes m}$ with the metric $h_{M}:=h_{F}^{\otimes m}$ 
% satisfies the curvature assumption in Theorem \ref{thm:main}. 
% Further, the zero locus $s|_{D}^{-1}(0)$ contains no lc centers of $(X, D)$ by assumption; 
% hence, by Theorem \ref{thm:main}, the lowest multiplication map $\otimes s|_{D}$ in the diagram is injective for every $q$. 
% This implies that a cohomology class $\alpha \in H^q(X,  K_{X}\otimes D \otimes F)$ with 
% $s  \alpha =0 \in H^q(X,  K_{X}\otimes D \otimes F^{\otimes (m+1)})$ 
% lies in the image of the vertical multiplication map $\otimes \sect_D$ on the left, 
% where $\sect_D$ is the canonical section of the effective divisor $D$. 
% Then, the conclusion of $\alpha=0$ follows from \cite[Theorem 1.6]{Mat} (or \cite[]{CC}). 
% \end{proof}




\section{Proofs of the main results}\label{sec:proof}

\subsection{Proof of Corollary \ref{cor:main}}\label{subsec:n2}

Corollary \ref{cor:main} can be proved by adapting the % same argument as in the
proof of Theorem \ref{thm:main} (or, more precisely, Theorem
\ref{thm:ker-nu=ker-tau}; see Remark \ref{rem:general-commut-diagram}
for details).
The proof involves an inductive reduction of the setup to
subvarieties on which the relevant injectivity result is known
or can be proved via Enoki's arguments (i.e.~harmonic theory for
cohomology is valid).
To get an essence of the argument, here 
% Nevertheless, in this subsection,
we deduce Corollary \ref{cor:main} from Theorem \ref{thm:main} 
using the previous work \cite{Matsumura_injectivity-lc}*{Thm.~1.6} (or
\cite{Chan&Choi_injectivity-I}*{Thm.~1.2.1}).




\begin{proof}[Proof of Corollary \ref{cor:main}]
%In the proof, we freely use the notation in Conjecture \ref{conj:fujino}. 
Consider the following commutative diagram  induced by 
the short exact sequence $0 \to K_{X} \to K_{X}\otimes D \to K_{D} \to
0 $ and the multiplication map $\otimes s$: 
\begin{equation*}
  % \vcenter{
  \xymatrix@R=3ex{
    \ar[d] & \ar[d]\\
    {H^q(X,  K_{X}\otimes F )} \ar[d]^-{}\ar[r] ^-{\otimes s}
    \ar[d]^-{\otimes \sect_D}
    &{H^q(X, K_{X}  \otimes F^{ \otimes{(m+1)}} )} \ar[d]\\ 
    {H^q(X,  K_{X}\otimes D \otimes F)}
    \ar[d]^-{}\ar[r] ^-{\otimes s}
    &{H^q(X,  K_{X} \otimes D \otimes F^{\otimes{(m+1)}})} \ar[d]\\ 
    {H^q(D, K_{D}\otimes F )}
    \ar[d]\ar[r]^-{\otimes s|_{D} } 
    & {H^q(D,  K_{D}\otimes F^{\otimes(m+1)} ) \; .}  \ar[d]\\ 
    & 
  }
  % }
\end{equation*}
The line bundle $M:=F^{\otimes m}$ with the metric
$h_{M}:=h_{F}^{\otimes m} = e^{-m\vphi_F}$ satisfies the curvature
assumption in Theorem \ref{thm:main} and the zero locus $s^{-1}(0)$
contains no lc centers of $(X, D)$ by assumption.
Hence, by Theorem \ref{thm:main}, the lowest multiplication map $\otimes s|_{D}$ in the diagram is injective for every $q$. 
This implies that a cohomology class $\alpha \in H^q(X,  K_{X}\otimes D \otimes F)$ with 
$s  \alpha =0 \in H^q(X,  K_{X}\otimes D \otimes F^{\otimes (m+1)})$ 
lies in the image of the vertical multiplication map $\otimes \sect_D$ on the left, 
where $\sect_D$ is the canonical section of the effective divisor $D$. 
Then, the conclusion $\alpha=0$ follows from \cite{Matsumura_injectivity-lc}*{Thm.~1.6} 
(or \cite{Chan&Choi_injectivity-I}*{Thm.~1.2.1}). 
\end{proof}



\subsection{Proof of Theorem \ref{thm:main} for a simple case} % \label{subsec:n2}
\label{sec:proof-of-simple-case}

In this subsection, we prove Theorem \ref{thm:main} in the simple case 
where \emph{$D$ has two components (i.e.~$D=D_{1}+D_{2}$) whose intersection
has only one irreducible component} and the degree of cohomology
groups is $q=1$.
% This simple case is completely contained in the general case discussed in Section \ref{subsec:general}, 
% but we illustrate a detailed proof, which is quite helpful in understanding the essence of the proof. 
% The proof of the general case is an extension of the argument in this section using lc strata. 
While this case is contained in the proof presented in Section
\ref{subsec:general}, a detailed proof of it is presented here in
order to illustrate the essence of the proof in the general case
without being obscured by the notation.
The proof in Section \ref{subsec:general} follows the same arguments
but on the lower-dimensional lc strata with more components.


\begin{proof}[Proof of Theorem \ref{thm:main} in the case of $D=D_{1}+D_{2}$ and $q=1$] 

% Suppose that $D=D_{1}+D_{2}$ and $q=1$ in Theorem \ref{thm:main}. 
Under the given assumptions and for a given cohomology class $\alpha
\in H^1(D,  K_{D} \otimes F)$, we prove here that $\alpha $ is actually $0$
when $s  \alpha =0 \in H^1(D,  K_{D} \otimes F \otimes M)$. 

\begin{step}[``Harmonic representative'' of $\alpha$] \label{step:harmonic-rep}
We intend to work with an \emph{``optimal''} representative of $\alpha$ via the Dolbeault isomorphism, 
% which means an appropriate harmonic form has the minimum $L^{2}$-norm in the forms representing $\alpha$.  
in the analogy of a harmonic form being the element with the minimal
$L^2$ norm in the corresponding cohomology class.
Nevertheless, at the time of writing, there is not yet a well
established theory of Dolbeault isomorphism and harmonic theory for
cohomology groups on the singular space $D$.
For this purpose, we consider the following diagram
\begin{equation}\label{h}
  \begin{aligned}
    \xymatrix@C=3.5em@R=3.5ex{
      \ar[d] & \ar[d]\\
      {\smash{\bigoplus_{p=1}^{2}} H^1(D_{p}, K_{D_{p}}\otimes F )}
      \ar[d]^-{}\ar[r] ^-{\otimes (s|_{D_{1}}, s|_{D_{2}})}
      \ar[d]^-{\tau}
      &{\smash{\bigoplus_{p=1}^{2}} H^1(D_{p},
        K_{D_{p}} \otimes F \otimes M )}
      \ar[d]\\
      {H^1(D, K_{D} \otimes F)} \ar[d]^-{}\ar[r] ^-{\otimes s}
      &{H^1(D,  K_{D} \otimes F\otimes M)} \ar[d]\\
      {H^1(D_{1}\cap D_{2}, K_{D_{1}\cap D_{2}}\otimes F )}
      \ar[d]\ar[r]^-{\otimes s|_{D_{1}\cap D_{2}} }
      &{H^1(D_{1}\cap D_{2},  K_{D_{1}\cap D_{2}}\otimes F \otimes M)} \ar[d]\\
      & }
  \end{aligned}
\end{equation}
induced from $0 \to K_{D_{1}} \oplus K_{D_{2}} \to K_{D} \to K_{D_{1} \cap D_{2}} \to 0$, 
% which corresponds to $\eqref{eq-ex2}$ in the case of $\rho=0, \sigma=1, \tau=2$ (cf.\,\eqref{eq-ex}). 
which in turn can be obtained by tensoring $K_X \otimes D$ to the
short exact sequence of adjoint ideal sheaves
\begin{equation*}
  \renewcommand{\objectstyle}{\displaystyle}
  \xymatrix@R=3.5ex{
    {0} \ar[r]
    &{\faidlof|1|/|0|*} \ar[r] \ar[d]_-{\Res^1}^-{\isom}
    &{\faidlof|2|/|0|*} \ar[r]
    &{\faidlof|2|/|1|*} \ar[r] \ar[d]^-{\Res^2}_-{\isom}
    &{0}
    \\
    &{\residlof|1|*} %\ar@{}[u]|-*[left]+{\isom}
    &&{\residlof|2|*} %\ar@{}[u]|-*[left]+{\isom}
  }
\end{equation*}
where $\aidlof* :=\aidlof$ and $\residlof* := \residlof$ and
the isomorphism $\faidlof/-1* \xrightarrow[\isom]{\Res^\sigma}
\residlof*$ is induced from the residue short exact sequence in
Section \ref{subsec:residue}.
Notice that the Dolbeault isomorphism and harmonic theory are
valid on $D_1$, $D_2$ and $D_1 \cap D_2$.
The multiplication map $\otimes s|_{D_{1}\cap D_{2}}$ on the bottom
row is non-zero by the assumption on $s^{-1}(0)$ and the curvature
assumption is still satisfied after restricting $F$ and $M$ to
$D_{1}\cap D_{2}$.
Hence, Enoki's injectivity theorem can be invoked to assert that
$\otimes s|_{D_{1}\cap D_{2}}$ is injective. 
Then, by an easy diagram chasing, we can find harmonic forms $u_{p}$ for $p=1,2$ such that 
\begin{equation*}
  u_{p} \in \mathcal{H}^{n-1,1}(D_{p}; F)_{\vphi_F} \cong H^1(D_{p},  K_{D_{p}}\otimes F ) 
  \text{ with } \alpha = \tau(\eqcls{u_{1}}, \eqcls{u_{2}}), 
\end{equation*}
where $\mathcal{H}^{n-1,1}(D_{p}; F)_{\vphi_F}$ denotes 
the space of $F|_{D_{p}}$-valued harmonic forms of $(n-1,1)$-type with respect to $\res{e^{-\vphi_F}}_{D_{p}}$. 
Note that there is freedom in the choice of $(u_{1}, u_{2})$  
since $\tau$ may not be injective. 
% $(u_{1}, u_{2})$ may not be the best representation of  $\alpha$.
% For this reason, by the orthogonal decomposition, we re-
To obtain the unique ``optimal'' representative of $\alpha$, we choose
the pair $(u_{1}, u_{2})$ 
with $\alpha = \tau(\eqcls{u_{1}}, \eqcls{u_{2}})$ that satisfies
\begin{equation}\label{eq-orth}
  (u_{1}, u_{2}) \in (\Ker \tau)^{\perp} \subset 
  \Ker \tau \oplus (\Ker \tau)^{\perp}= \bigoplus_{p=1}^{2}
  \mathcal{H}^{n-1,1}(D_{p}; F)_{\vphi_F} \; ,
\end{equation}
% This can be regarded as the {\textit{best}} representation of  $\alpha$. 
% Our purpose is to prove that the $L^{2}$-norm 
% $\norm{s u_{1}}_{\vphi_M, D_1}^2 +\norm{s u_{2}}_{\vphi_M, D_2}^2$ 
% is actually zero. 
in which $\paren{\ker\tau}^\perp$ is the orthogonal complement of
$\ker\tau$ with respect to the (squared) residue norm
$\norm\cdot_{\lcc|1|'}^2 =\norm\cdot_{D_1}^2 +\norm\cdot_{D_2}^2$
(defined as in \eqref{eq:residue-norm} with $\sigma :=1$, $\vphi_L
:=\vphi_F$ and $\lcS[V,p] :=D_p$).
With such choice of representative, our goal is to prove that the
$L^{2}$-norm $\norm{s u_{1}}_{D_1, \vphi_M}^2 +\norm{s u_{2}}_{D_2,
  \vphi_M}^2$ is actually zero (where $\norm\cdot_{D_p, \vphi_M}$'s
are defined as in \eqref{eq:residue-norm} with $\vphi_L:=\vphi_F
+\vphi_M$).


\end{step}

\begin{step}[Obstruction for $\norm{s u_{1}}_{D_1, \vphi_M}^2 +\norm{s
      u_{2}}_{D_2, \vphi_M}^2$ from being zero]
  \label{item:expression-of-su-simple}
  

  % In this step, we examine the relevant $L^{2}$-norm 
  % to obtain an obstruction for our purpose as a $F$-valued form on $D_{1} \cap D_{2}$. 
  % For this purpose, by using the Dolbeault isomorphism, the Poincar\'e residue map, and the assumption of $s \alpha=0$, 
  % we prepare the following data: 
  We make use of the assumption $s \alpha=0$ and the \v
  Cech--Dolbeault isomorphism to re-express $\norm{s u_{1}}_{D_1,
    \vphi_M}^2 +\norm{s u_{2}}_{D_2, \vphi_M}^2$ as follows.


  \begin{itemize}
  \item Take $\alpha_{p;\:ij}   \in H^{0}(V_{ij} \cap D_p ,
    K_{D_{p}}\otimes F)$  
    for every open set $V_{ij} :=V_i \cap V_j$ with $i,j \in I$ and
    $V_i \cap V_j \cap D_p\neq \emptyset$ such that the family
    $\{\alpha_{p;\:ij}\}_{i,j \in I}$ is a \v Cech cocycle
    representing $u_{p}$ via the \v Cech--Dolbeault isomorphism on
    $D_p$.
    It follows that there exists an $L^2$ section $v_{p,(2)}$ of
    $K_{D_p} \otimes \res F_{D_p}$ on $D_p$ with respect to
    $\norm\cdot_{D_p}$ such that (under Einstein summation
    convention) 
    \begin{equation*}
      u_p
      \overset{\text{\eqref{eq:Cech-Dolbeault-isom}}}= \:
      \dbar v_{p,(2)} -\dbar\rho^j \cdot \rho^i \:\alpha_{p;\:ij} \; .
    \end{equation*}


  \item Take $f_{ij} \in H^{0}(V_{ij} , K_X \otimes D
    \otimes F \otimes \defidlof{D_1 \cap D_2})$ for $i,j \in I$
    satisfying 
    \begin{equation*}
      \Res^1\paren{f_{ij}}
      :=\paren{\PRes[D_1](\frac{f_{ij}}{\sect_D}) \:,\:
        \PRes[D_2](\frac{f_{ij}}{\sect_D})} 
      = \paren{\alpha_{1;\:ij} , \alpha_{2;\:ij}} 
    \end{equation*}
    whose existence are guaranteed by the surjectivity of the
    residue isomorphism $\Res^1$ on Stein open sets such that
    \begin{equation*}
      \renewcommand{\objectstyle}{\displaystyle}
      \xymatrix@C=1em@R=0.8em{
        {K_X \otimes D \otimes \frac{\defidlof{D_1 \cap
              D_2}}{\defidlof{D}}} 
        \ar@{}[r]|-*+{=}
        \ar@{}[d]|*[left]{\in}
        &{K_X \otimes D \otimes \faidlof|1|/|0|*}
        \ar[rr]^-{\Res^1}_-{\isom}
        &&{K_X \otimes D \otimes \residlof|1|*}
        \ar@{}[r]|-*+{=}
        &{K_{D_1} \oplus K_{D_2}}
        \ar@{}[d]|(.57)*[left]{\in}
        \\
        *+/r 3em/{f_{ij} \bmod \defidlof{D}}
        \ar@{|->}[rrr]
        &&&*+/l 6em/{}
        &*-{\paren{\alpha_{1;\:ij} , \alpha_{2;\:ij}} \; .}
      }
    \end{equation*}
    It is easy to see that $\set{f_{ij} \bmod \defidlof{D}}_{i,j \in
      I}$ is a \v Cech cocycle whose cohomology class in $\cohgp
    1[D]{\logKX \otimes \frac{\defidlof{D_1 \cap
          D_2}}{\defidlof{D}}}$ is mapped to $\alpha$ via $\tau$. 

  \item The assumption $s \alpha=0$ in $\cohgp 1[D]{K_D
      \otimes F \otimes M}$ guarantees the
    existence of $\lambda_{i} \in H^{0}(V_{i}, K_X \otimes D\otimes
    F \otimes M)$ for $i \in I$ such that
    \begin{equation*}
      s f_{ij} \equiv \lambda_j -\lambda_i \mod \defidlof{D}
      \quad\text{ on } V_{ij} \; .
    \end{equation*}
    Note that the coefficients of $\lambda_i$ need not lie in
    $\defidlof{D_1 \cap D_2}$ even though so do those of $f_{ij}$.
    By setting
    \begin{equation*}
      \rs*\lambda_{p;\:i} := \PRes[D_p](\frac{\lambda_i}{\sect_D})
      \cdot \sect_{(p)}
      \quad\text{ on $V_i \cap D_p$ for } i\in I
      \text{ and } p = 1,2 \; ,
    \end{equation*}
    it then follows that
    \begin{equation*}
      s\alpha_{p;\:ij} \sect_{(p)} =\rs*\lambda_{p;\:j} -\rs*\lambda_{p;\:i}
      \quad\text{ on } V_{ij} \cap D_p \; .
    \end{equation*}
    Note that $\rs*\lambda_{p;\:i}$ is holomorphic on $V_i \cap D_p$
    (while $\PRes[D_p](\frac{\lambda_i}{\sect_D})$ may not be).
  \end{itemize}

  Since $u_{p}$ is harmonic with respect to $\vphi_F$ on $D_p$ and
  we have $\ibddbar\vphi_F \geq 0$ and $%-C\omega \leq
  \ibddbar\vphi_M \leq C\ibddbar\vphi_F$ on $D_p$ for some constant
  $C > 0$ by assumption,
  Proposition \ref{prop:consequence-of-positivity} guarantees that  
  % \begin{equation*} %\label{eq-harmonic}
  %   su_p \in \Harm'/n-1,1/<D_p>{F\otimes M},{\vphi_F+\vphi_M} \; ,
  % \end{equation*}%
  $su_p$ is harmonic with respect to $\vphi_F+\vphi_M$ on $D_p$,
  which is a consequence of Nakano's identity and Enoki's argument.
  It follows that $\iinner{s \dbar v_{p;(2)}}{su_p}_{D_p, \vphi_M}
  =\iinner{\dbar\paren{s v_{p;(2)}}}{su_p}_{D_p, \vphi_M} = 0$.
  Summarizing the above discussion, it follows that
  \begin{align*}
    \norm{s u_{p}}_{D_p, \vphi_M}^2 
    &= -\sum_{i,j\in I}\iinner{\dbar\rho^{j} \cdot \rho^i \:s
      \alpha_{p;\:ij} \:}{\:s u_p}_{D_p, \vphi_M}\\
    &= -\sum_{i,j\in I}\iinner{\dbar\rho^{j} \cdot \rho^i \:s
      \alpha_{p;\:ij} \sect_{(p)} \:}{\:s u_p \sect_{(p)}}_{D_p, \vphi_M+\phi_{(p)}}\\
    &= -\sum_{i,j\in I}\iinner{\dbar\rho^{j} \cdot \rho^i
      \paren{\rs*\lambda_{p;\:j}- \rs*\lambda_{p;\:i}} \:}
      {\:s u_p \sect_{(p)}}_{D_p, \vphi_M+\phi_{(p)}}\\
    &= -\sum_{j\in I}\iinner{\dbar\paren{\rho^j
      \rs*\lambda_{p;\:j}} \:}{\: s u_p \sect_{(p)}}_{D_p,
      \vphi_M+\phi_{(p)}}
      =: -\iinner{\dbar v_{p;(\infty)} }{ s u_p \sect_{(p)}}_{D_p,
      \vphi_M+\phi_{(p)}}
      \; .
  \end{align*}
  The notation $v_{p;(\infty)} :=\sum_{j\in I} \rho^j
  \rs*\lambda_{p;\:j}$ is used for the consistency with the notation in
  Proposition \ref{prop:res-formula-dbar-exact-dot-harmonic}.

  The residue computation in Proposition
  \ref{prop:res-formula-dbar-exact-dot-harmonic} further brings the
  expression of $\norm{s u_{p}}_{D_p, \vphi_M}^2$ for each $p=1,2$ to an inner
  product on $D_1 \cap D_2$.
  As $\lcS|2|[b] :=D_1 \cap D_2$ has only $1$ component, the index set $\Iset|2|
  =\set{b}$ is a singleton.
  Moreover, the general different $\Diff_{D_1 \cap D_2}(D) =\Diff_b(D)$ is
  trivial, so we choose its canonical section and the corresponding
  potential such that $\sect_{(b)} \equiv 1$ and $\phi_{(b)} \equiv 0$
  (and $\psi_{(b)} \equiv -1$) on $D_1 \cap D_2$.
  Let $\PRes[\lcS|2|[b] | D_p]$ be the Poincar\'e residue map from
  $D_p$ to $D_1 \cap D_2$.
  We fix the sign convention such that
  \begin{equation*}
    \rs*\lambda_{b;\:i}
    =\frac{\rs*\lambda_{b;\:i}}{\sect_{(b)}}
    :=\PRes[\lcS|2|[b]](\frac{\lambda_i}{\sect_D})
    \begin{aligned}[t]
      &= \PRes[\lcS|2|[b] | D_1] \circ
      \PRes[D_1](\frac{\lambda_i}{\sect_D})
      \\
      &=\PRes[\lcS|2|[b] | D_1](\frac{\rs*\lambda_{1;\:i}}{\sect_{(1)}})
    \end{aligned}
    \begin{aligned}[t]
      &=-\PRes[\lcS|2|[b] | D_2] \circ
      \PRes[D_2](\frac{\lambda_i}{\sect_D})
      \\
      &=-\PRes[\lcS|2|[b] | D_2](\frac{\rs*\lambda_{2;\:i}}{\sect_{(2)}})
    \end{aligned} \; .
  \end{equation*}
  Following the computation in Proposition
  \ref{prop:res-formula-dbar-exact-dot-harmonic}, we obtain
  \begin{align*}
    &~\norm{s u_{1}}_{D_1, \vphi_M}^2 +\norm{s u_{2}}_{D_2,
      \vphi_M}^2
    \\
    =&~-\iinner{\dbar v_{1;(\infty)} }{ s u_1 \sect_{(1)}}_{D_1,
       \vphi_M+\phi_{(1)}}
       -\iinner{\dbar v_{2;(\infty)} }{ s u_2 \sect_{(2)}}_{D_2,
       \vphi_M+\phi_{(2)}}
    \\
    =&~
       \begin{multlined}[t]
         \sum_{i\in I}\iinner{\rho^i \PRes[\lcS|2|[b] |
           D_1](\frac{\rs*\lambda_{1;\:i}}{\sect_{(1)}}) }{
           \: s\:\PRes[\lcS|2|[b] | D_1](\idxup{\diff\psi_{(1)}}. u_1)
         }_{D_1 \cap D_2, \vphi_M} \\
         +\sum_{i\in I}\iinner{\rho^i
           \PRes[\lcS|2|[b] |
           D_2](\frac{\rs*\lambda_{2;\:i}}{\sect_{(2)}}) }{
           \: s\:\PRes[\lcS|2|[b] | D_2](\idxup{\diff\psi_{(2)}}. u_2)
         }_{D_1 \cap D_2, \vphi_M}
       \end{multlined}
    \\
    =&~\iinner{\sum_{i\in
       I}\rho^i\rs*\lambda_{b;\:i} \:}{\: s\:\paren{
       \PRes[\lcS|2|[b] | D_1](\idxup{\diff\psi_{(1)}}. u_1)
       -\PRes[\lcS|2|[b] | D_2](\idxup{\diff\psi_{(2)}}. u_2)
       }}_{D_1 \cap D_2, \vphi_M}
    \\
    =:&~\iinner{v_{b;(\infty)}}{s w_b}_{D_1 \cap D_2, \vphi_M} \; ,
  \end{align*}
  which is the desired expression.

  It is shown below that
  \begin{equation} \label{eq:w-prelim-formula}
    w_b :=\PRes[\lcS|2|[b] | D_1](\idxup{\diff\psi_{(1)}}. u_1)
    -\PRes[\lcS|2|[b] | D_2](\idxup{\diff\psi_{(2)}}. u_2)
  \end{equation}
  is actually $0$ on $D_1 \cap D_2$, which will then conclude the proof.

  % \begin{itemize}
  % \item[$\bullet$] Take $\beta_{ij,p}   \in H^{0}(V_{ij}, K_{D_{p}}\otimes F)$ 
  %   such that the family $\{\beta_{ij,p}\}$ is a cocycle  corresponding to $u_{p}$ via the \v Cech--Dolbeault isomorphism. 




  % \item[$\bullet$] Take $\alpha_{ij} \in H^{0}(V_{ij}, K_X \otimes D
  %   \otimes \defidlof{D_1 \cap D_2} \otimes F)$
  %   satisfying that 
  %   \begin{equation*}
  %     \Res^1\paren{\alpha_{ij}}
  %     :=\paren{\PRes[D_1](\frac{\alpha_{ij}}{\sect_D}) \:,\:
  %     \PRes[D_2](\frac{\alpha_{ij}}{\sect_D})} 
  %     = \paren{\beta_{ij, 1}, \beta_{ij, 2}}, 
  %   \end{equation*}
  %   by the  residue isomorphism: 
  %   \begin{equation*}
  %     \xymatrix@R=0.1cm{
  %     *+/r 0.5cm/{K_{D_1} \oplus K_{D_2}} &
  %     *+/r 0.5cm/{
  %     K_X \otimes D \otimes \frac{\defidlof{D_1 \cap D_2}}{\defidlof{D}}=K_X \otimes D \otimes \frac{\aidlof|1|*}{\aidlof|0|*}.
  %   }
  %     \ar[l]_-{\Res^1}^-{\isom}
  %   } 
  %   \end{equation*}
  %   More specifically, we may define $\alpha_{ij}$ 
  %   by $\alpha_{ij}:=d\sect_{(2)}  \wedge \sect_{(1)} \:\beta_{ij, 1} +d\sect_{(1)}  \wedge\sect_{(2)} \:\beta_{ij, 2}$. 


  % \item[$\bullet$] Take $\lambda_{i} \in H^{0}(V_{i}, K_X \otimes D\otimes F \otimes M)$ 
  %   satisfying that 
  %   $$\text{
  %   $ s \alpha_{ij} \equiv \lambda_j -\lambda_i$  as a section of 
  %   $K_X \otimes D\otimes \mathcal{O}_{X}/\defidlof{D} \otimes F \otimes M =K_D \otimes F \otimes M$. 
  % }
  %   $$
  %   The cocycle $\{\alpha_{ij}\}$ of $K_X \otimes D \otimes F$ (noting that $\defidlof{D_1 \cap D_2}$ is not tensored) 
  %   corresponds to $\alpha$; hence the assumption of $s \alpha=0$ guarantees the existence of $\lambda_{i}$. 



  % \item[$\bullet$] Take $\rs \lambda_i^1$, $\rs \lambda_i^2$, and $\rs\lambda_i^{12}$ such that 
  %   \begin{equation*}
  %     \lambda_i
  %     =d\sect_{(2)}  \wedge \rs \lambda_i^1
  %     =d\sect_{(1)}  \wedge \rs \lambda_i^2
  %     =d\sect_{(2)}  \wedge d\sect_{(1)}  \wedge \rs\lambda_i^{12}. 
  %   \end{equation*}
  % \end{itemize}
  % By construction, we see that 
  % \begin{align*}
  %   &\bullet \text{$u_{p}=\dbar u_{(2), p} + \dbar \rho^{i} \beta_{ij,p} $ for some global section $u_{(2), p}$ of $K_{D_{p}}\otimes F$};\\
  %   &\bullet s \sect_{(p)} \alpha_{ij}= \rs\lambda_j^p- \rs\lambda_i^p \text{ on } D_{p} 
  %   \text{ as a section of }K_{D_{p}}\otimes F \otimes M.
  % \end{align*}
  % On the other hand, since $u_{p}$ is harmonic and $\sqrt{-1}\Theta_{h_{F}} \geq 0$, 
  % we can conclude that 
  % \begin{align}\label{eq-harmonic}
  %   {\nabla^{(0,1)} u_{p}} =0  \text{ and } \sqrt{-1} \Theta_{h_{F}} \Lambda_{\omega} u_{p} =0
  % \end{align}
  % by applying Nakano's identity. 
  % Further, together with the curvature assumption, 
  % Enoki's argument shows that $su_{p}$ is still harmonic with respect to $h_{F}h_{M}$. 
  % Then, we can easily see that 
  % \begin{align*}
  %   \norm{s u_{p}}_{\vphi_M, D_p}^2 
  %   &= \iinner{\dbar s u_{(2), p}  +  \dbar s \rho^{i} \beta_{ij,p}}{s u}_{\vphi_M, D_p}\\
  %   &= \iinner{ \dbar s \sect_{(p)} \rho^{i} \beta_{ij,p}}{s \sect_{(p)} u}_{ \phi_{(p)}+\vphi_M, D_p}\\
  %   &= \iinner{\dbar \rho^{i} (\rs\lambda_j^p- \rs\lambda_i^p)}
  %   {s \sect_{(1)} u}_{\phi_{(p)}+\vphi_M, D_2}\\
  %   &= \iinner{-\dbar\paren{\rho^i \rs\lambda_i^p}}{s \sect_{(p)}
  %   v}_{\phi_{(p)}+\vphi_M, D_p}.
  % \end{align*}
  % Here we use that $s u$ is still harmonic to get the second equality 
  % and that $\dbar \rho^{i} \rs\lambda_j^p=\dbar \rs\lambda_j^p=0$ to get the third equality. 
  % The right-hand side can be described by the norm on $D_{1}\cap D_{2}$ as follows: 
  % \begin{align*} 
  %   &~- \iinner{\dbar\paren{\rho^i \rs\lambda_i^p}}{s \sect_{(p)} u}_{\phi_{(p)}+\vphi_M, D_p} \\
  %   \xleftarrow{\varepsilon \tendsto 0^+}
  %   &~-\iinner{e^{-\varepsilon \abs{\psi_{(p)}}}\dbar\paren{\rho^i \rs\lambda_i^p}}{s \sect_{(p)}
  %   u}_{\phi_{(p)}+\vphi_M, D_p} \\
  %   =
  %   &
  %   \begin{aligned}[t]
  %     &~-\cancelto{0}{
  %     \iinner{ \dbar\paren{e^{-\varepsilon
  %     \abs{\psi_{(p)}}} \rho^i \rs\lambda_i^p} }{ s \sect_{(p)} u}
  %   }_{\phi_{(p)}+\vphi_M, D_p} 
  %     +\varepsilon \iinner{
  %     e^{-\varepsilon \abs{\psi_{(p)}}} \rho^i \rs\lambda_{i}^p }{(\diff\psi_{(p)})^{*}s \sect_{(p)} u }_{\phi_{(p)}+\vphi_M, D_p}
  %   \end{aligned}
  %   \\
  %   =
  %   &~ \varepsilon \iinner{
  %   \frac{\rho^i \rs\lambda_{i}^p}{\sect_{(p)}}
  % }{
  %   e^{-\varepsilon \abs{\psi_{(p)}}}
  %   (\diff \log \abs{\sect_{(p)}^2})^{*}
  %   u \: s e^{-\vphi_M} }_{D_p}
  %   -\underbrace{
  %   \varepsilon \iinner{
  %   \frac{\rho^i \rs \lambda_{i}^p}{\sect_{(p)}}
  % }{
  %   e^{-\varepsilon \abs{\psi_{(p)}}}
  %   (\diff \sm\vphi_{(p)})^{*}  u \:s e^{-\vphi_M} }_{D_p}
  % }_{=\: \BigO(\varepsilon)}
  %   \\
  %   =
  %   &~\varepsilon \iinner{
  %   \rho^i \smash[b]{\underbrace{\rs\lambda_{i}^p}_{\mathclap{=\: d\sect_{(1)} 
  %   \wedge \rs\lambda_i^{12}}}}
  %   \:
  % }{ \:
  %   \frac{e^{-\varepsilon \abs{\psi_{(p)}}}}{\abs{\sect_{(p)}}^2}
  %   (d\sect_{(p)})^{*} \smash[b]{\underbrace{u_{p}}_{\mathclap{=:\: d\sect_{(1)}  \wedge \rs u_{p}^{12}}}} \: s e^{-\vphi_M} }_{D_p}
  %   + \BigO(\varepsilon)
  %   \vphantom{\underbrace{\rs\lambda_{i}^1}_{\mathclap{=\: d\sect_{(1)} 
  %   \wedge \rs\lambda_i^{12}}}}
  %   \\
  %   \xrightarrow{\varepsilon \tendsto 0^+}
  %   &~\iinner{\rho^i \rs\lambda_i^{12}}{  (d\sect_{(p)})^{*}  \rs u_{p}^{12}
  %   \: s e^{-\vphi_M}}_{D_1 \cap D_2}  \; .
  % \end{align*}

  % Considering the inner product above, we define the $F$-valued form $w$ on $D_{1} \cap D_{2}$ by 
  % \begin{equation*}
  %   w:=(d\sect_{(1)})^{*}  \rs u_{1}^{12}-(d\sect_{(2)})^{*} \rs u_{2}^{12}. 
  % \end{equation*}
  % From the next step, we aim to show $w$ is actually zero, which finishes the proof. 

  % Note that $u_{p}^{12}$ and $d\sect_{(p)}$ are defined only locally; 
  % hence $d\sect_{(p)})^{*}  \rs u_{p}^{12}$ does not determine a section on $D_{p}$, 
  % but determines the $F$-valued section form of type $(n-2, q-1)$ on $D_{1} \cap D_{2}$. 
  % This can be verified by calculating a glueing condition. 
  % Another way to see this is to apply The Poincar\'e residue map from $D_{p}$ to $D_1 \cap D_2$, 
  % which yields
  % \begin{equation*}
  %   \PRes[D_1 \cap D_2]( (\diff\psi_{(p)})^{*} u)
  %   =\parres{(d\sect_{(p)})^{*}  \rs u_{p}^{12}}_{D_1 \cap D_2}
  %   \quad\text{(recall that $\sect_{(p)} =\sect_{(p)} $)} \; .
  % \end{equation*}
  % Since $\psi_{(p)}=\phi_{(p)} -\sm\vphi_{(p)}$ is a global function, 
  % the right hand side is globally defined on $D_{p}$, and so is $((d\sect_{(p)})^{*}  \rs u_{p}^{12})|_{D_1 \cap D_2}$. 
\end{step}



\begin{step}[$w_b$ being holomorphic and thus {$w_b \in \cohgp 0[D_1
    \cap D_2]{K_{D_1 \cap D_2} \otimes F}$}]
  
  We prove that $\dbar w_b = 0$ on $\lcS|2|[b] :=D_1 \cap D_2$ by a
  direct computation given in Section \ref{subsec:harmonic}.
  Indeed, it suffices to show that each summand $\PRes[\lcS|2|[b] |
  D_p](\idxup{\diff\psi_{(p)}}. u_p)$ for $p=1,2$ in $w_b$ is
  $\dbar$-closed.
  The computations are identical, so it suffices to consider $p=1$.

  On an admissible open set $V$ such that $D_p  \cap V =\set{z_p =
    0}$ for $p=1,2$ and $\lcS|2|[b] \cap V = \set{z_1 = z_2 = 0} =
  D_1 \cap \set{z_2 = 0}$,
  we have
  \begin{equation*}
    \diff\psi_{(1)} =\frac{dz_2}{z_2} -\diff\sm\vphi_{(1)}
    \quad\text{ on } V \; .
  \end{equation*}
  By writing
  \begin{equation*}
    \idxup{dz_2}. u_1 =: dz_2 \wedge \paren{\idxup{dz_2}. \rs*u_{1,2}}
    \quad\text{ on } D_1 \cap V \; ,
  \end{equation*}
  where $\rs*u_{1,2}$ is a $(n-2,1)$-form on $D_1 \cap V$, we see
  that
  \begin{equation*}
    \PRes[\lcS|2|[b] | D_1](\idxup{\diff\psi_{(1)}}. u_1)
    =\PRes[\set{z_2 = 0}](\frac{\idxup{dz_2} .u_1}{z_2})
    =\parres{\idxup{dz_2}. \rs*u_{1,2}}_{\lcS|2|[b]}
    \quad\text{ on } D_1 \cap D_2 \cap V \; .
  \end{equation*}
  Therefore, it suffices to check that $\idxup{dz_2}. u_1$ is
  $\dbar$-closed on $D_1 \cap V$.
  As $u_1$ is harmonic and $\ibddbar\vphi_F \geq 0$, we have
  $\nabla^{(0,1)} u_1 = 0$ by Proposition
  \ref{prop:consequence-of-positivity} and Lemma
  \ref{lem:commutator-dbar-ctrt} yields the desired result (with $z_2$
  in place of $\vphi$ in the lemma).

  % In this step, we show that $w$ is a $F$-valued harmonic on $D_{1} \cap D_{2}$. 
  % Note that it is sufficient to show that $\dbar w =0$ in our case since the type of $w$ is $(n-2, q-1)=(n-2, 0)$ by $q=1$. 
  % By \cite[(1.9)]{Takegoshi_higher-direct-images} and \eqref{eq-harmonic}, we obtain that 
  % \begin{equation*}
  %   \dbar ( (d\sect_{(p)})^{*} u_{p}) 
  %   = \big( (i \partial  \dbar \sect_{(p)})^{*} - (\partial \sect_{(p)})^{*} \dbar + \partial \sect_{(p)}\nabla^{(0,1)} \big)u_{p}
  %   = 0 \quad\text{on   } D_1. 
  % \end{equation*}
  % By noting that $u_{p} = d \sect_{(1)}  \wedge \rs u_{p}^{12}$, 
  % we see that  $\dbar (d\sect_{(p)})^{*}  \rs u_{p}^{12} =0 $; hence $\dbar w =0$. 
  % In particular, $w$ determines the cohomology class $\{w \} \in H^{0}(D_{1}\cap D_{2}, K_{D_1 \cap D_2} \otimes F)$. 
\end{step}


\begin{step}[$w_b = 0$ and conclusion of the proof]
  \label{step:pf:use_u-ortho-w-simple}
We prove that $w_b =0$ using the assumption $(u_{1},u_{2}) \in
\paren{\ker \tau}^\perp$.
Consider the connecting morphism $\delta$ the long exact sequence 
\begin{equation*}
  \xymatrix@R=0.3cm@C=1.5em{
    {\to \cohgp 0[D_{1}\cap D_{2}]{K_{D_1 \cap D_2} \otimes F}} \ar[r]^-{\delta}
    &
    {\bigoplus_{p=1}^{2} \cohgp 1[D_{p}]{K_{D_p}\otimes F}} \ar[r]^-{\tau}  
    &
    {\cohgp 1[D]{K_D \otimes F}  \to} \; . 
  } 
\end{equation*}
Note that $\delta w_b \in \ker\tau$.

We compute $\delta w_b$ via the \v Cech--Dolbeault isomorphism.
% $\rho(w)$ in terms of \v Cech cohomology. 
Regard $w_b$ as a $0$-cocycle $\set{\rs \gamma_{b;\:i}}_{i \in I}$
given by $\rs \gamma_{b;\:i} :=\res{w_b}_{V_i}$.
Lift $\rs \gamma_{b;\:i}$ on $D_1 \cap D_2 \cap V_i$ to a section
$\gamma_i$ on $V_{i}$ via the isomorphism
$\frac{\holo_X}{\defidlof{D_1 \cap D_2}} = \faidlof|2|/|1|*
\xrightarrow[\isom]{\Res^2} \residlof|2|*$ such that
\begin{equation*}
  % \gamma_i = d\sect_{(2)}  \wedge d\sect_{(1)}  \wedge \rs \gamma_i \; .
  \Res^2\paren{\gamma_i}
  =\PRes[\lcS|2|[b]](\frac{\gamma_i}{\sect_D})
  =\frac{\rs*\gamma_{b;\:i}}{\sect_{(b)}} =\rs*\gamma_{b;\:i} \; .
\end{equation*}
Then $\delta w_b$ is represented by the $1$-cocycle
\begin{equation*}
  \delta\set{\gamma_i \bmod \defidlof{D_1 \cap D_2}}_{i \in I}
  =\set{(\delta  \gamma)_{ij} \bmod\defidlof{D}}_{i,j \in I}
  =\set{\gamma_{j} -\gamma_i  \bmod\defidlof{D}}_{i,j \in I} \; .
\end{equation*}
Note that $ \gamma_{j} -\gamma_i$ belongs to $\defidlof{D_1 \cap
  D_2}$, so $ \gamma_{j} -\gamma_i  \bmod\defidlof{D}$ can be realized
via the isomorphism $\frac{\defidlof{D_1 \cap D_2}}{\defidlof{D}}
=\faidlof|1|/|0|* \xrightarrow[\isom]{\Res^1} \residlof|1|*$ as
\begin{align*}
  \Res^1\paren{\gamma_{j} -\gamma_i}
  &=\paren{\PRes[D_1](\frac{\gamma_{j} -\gamma_i}{\sect_D})
    \: ,\:
    \PRes[D_2](\frac{\gamma_{j} -\gamma_i}{\sect_D})
    } \\
  &=\paren{
    \frac{(\delta \rs\gamma_1)_{ij}}{\sect_{(1)}}
    \: , \:
    \frac{(\delta \rs\gamma_2)_{ij}}{\sect_{(2)}}
    }
    \in K_{D_1} \otimes \res F_{D_1} \oplus K_{D_2} \otimes \res F_{D_2} \; ,
\end{align*}
in which $\rs*\gamma_{p;\:i} := \PRes[D_p](\frac{\gamma_i}{\sect_D})
\cdot \sect_{(p)}$ for $p = 1,2$.
Therefore, via the \v Cech--Dolbeault isomorphism on each $D_p$, 
the component of $\delta w_b$ on $D_p$ can be represented by (under
Einstein summation convention) 
\begin{equation*}
  -\dbar\rho^j \cdot \rho^i
  \frac{\paren{\delta\rs*\gamma_p}_{ij}}{\sect_{(p)}}
  =-\frac{\dbar\rho^j \cdot\rs*\gamma_{p;\:j}}{\sect_{(p)}}
  =: -\frac{\dbar v'_{p;(\infty)}}{\sect_{(p)}}
  % \paren{
  %   \res{\frac{\dbar\rho^i \:(\delta \gamma^1)_{ij}}{\sect_{(1)} }}_{D_1}
  %   \: , \:
  %   \res{\frac{\dbar\rho^i \: (\delta \gamma^2)_{ij}}{\sect_{(2)} }}_{D_2}
  % }
  % =\paren{
  %   -\res{\frac{\dbar\paren{\rho^i  \gamma^1_{i}}}{\sect_{(1)}}}_{D_1}
  %   \: , \:
  %   -\res{\frac{\dbar\paren{\rho^i  \gamma^2_{i}}}{\sect_{(2)}}}_{D_2}
  % } \; .
\end{equation*}
(the notation $v'_{p;(\infty)} :=\sum_{i\in I}\rho^i
\rs*\gamma_{p;\:i}$ is set for the consistency with the notation in
Proposition \ref{prop:res-formula-dbar-exact-dot-harmonic}).
% For the computation of the norm, 
% we take $\rs \gamma_i^p$ and $\rs\gamma_i^{12}$ such that 
% \begin{equation*}
% \gamma_{i} = d \sect_{(p)} \wedge \rs \gamma_i^p 
%   \quad\text{and}\quad 
%   \rs\gamma_i^{12} = \gamma_i. 
% \end{equation*}
Recall the sign convention chosen in Step
\ref{item:expression-of-su-simple} such that
\begin{equation*}
  \rs*\gamma_{b;\:i}
  = \PRes[\lcS|2|[b] | D_1](\frac{\rs*\gamma_{1;\:i}}{\sect_{(1)}})
  =- \PRes[\lcS|2|[b] |
  D_2](\frac{\rs*\gamma_{2;\:i}}{\sect_{(2)}}) \; .
\end{equation*}
Then, from $(u_{1},u_{2}) \in \paren{\ker\tau}^\perp$ and $\delta w_b
\in \ker\tau$, we obtain
\begin{align*}
  0
  &=
  \iinner{
    -\frac{\dbar v'_{1;(\infty)}}{\sect_{(1)}}
  }{u_1}_{D_1}
  +\iinner{
    -\frac{\dbar v'_{2;(\infty)}}{\sect_{(2)}}
  }{u_2}_{D_2}
  \\
  &=
    \iinner{
    -\dbar v'_{1;(\infty)}
    }{u_1 \sect_{(1)}}_{D_1, \phi_{(1)}}
    +\iinner{
    -\dbar v'_{2;(\infty)}
    }{u_2 \sect_{(2)}}_{D_2, \phi_{(2)}}
  \\
  &\overset{\mathclap{\text{Prop.~\ref{prop:res-formula-dbar-exact-dot-harmonic}}}}=
    \quad\;\;
    \iinner{\rho^i \rs*\gamma_{b;\:i} \:}{\:
    \PRes[\lcS|2|[b] | D_1](\idxup{\diff\psi_{(1)}}. u_1)
    -\PRes[\lcS|2|[b] | D_2](\idxup{\diff\psi_{(2)}}. u_2)
    }_{D_1\cap D_2}
  \\
  &=\iinner{w_b}{w_b}_{D_1 \cap D_2}
    =\norm{w_b}_{D_1 \cap D_2}^2
    \; .
\end{align*}
% By the same computation as in Step 2, 
% the right hand side can be described by the norm of $w$ as follows: 
% \begin{equation*}
%   0=\iinner{\rho^i  \gamma_i^{12} \:}{\:
%     (d\sect_{(1)})^{*}    u^{12} -(d\sect_{(2)})^{*}   v^{12}
%   }_{D_1 \cap D_2}
%   =\iinner{\rho^i \gamma_i}{w}_{D_1 \cap D_2}
%   = \norm w_{D_1 \cap D_2}^2. 
% \end{equation*}
This implies that $w_b=0$, finishing the proof for the case
$D=D_{1}+D_{2}$ and $q=1$. \qedhere
\end{step}
\end{proof}



\subsection{Remarks on the general case}
% \subsection{Strategy of the proof in the general case}
\label{subsec:n3}

There are two modifications to the proof in Section
\ref{sec:proof-of-simple-case} in order to handle the general case
worth mentioning here.
The first one is the replacement of the short exact sequence $0 \to
K_{D_1} \oplus K_{D_2} \to K_D \to K_{D_1 \cap D_2} \to 0$.
Take the case $D = D_1 + D_2 + D_3$, where $D_p = \set{z_p = 0}$ for
$p=1,2,3$ are the coordinate planes, for example.
Note that
\begin{equation*}
  \aidlof|3|* = \holo_X \;, \;\;
  \aidlof|2|* = \defidlof{D_1 \cap D_2 \cap D_3} \;, \;\;
  \aidlof|1|* = \smashoperator{\bigcap_{\substack{1 \leq p,q \leq 3 \\ p\neq q}}} \defidlof{D_p \cap D_q} \;
  \text{ and } \;
  \aidlof|0|* = \defidlof{D} 
\end{equation*}
in this case.
A natural choice of the short exact sequence to be considered is
\begin{equation*}
  \renewcommand{\objectstyle}{\displaystyle}
  \xymatrix@R=2.5em{
    0 \ar[r]
    &{K_X \otimes D \otimes \smash{\faidlof|1|/|0|*}} \ar[r]
    \ar[d]^(0.47){\Res^1}_(0.47){\isom}
    &{K_X \otimes D \otimes \smash{\faidlof|3|/|0|*}} \ar[r]
    \ar@{=}[d]
    &{K_X \otimes D \otimes \faidlof|3|/|1|*} \ar[r]
    &0 \; .
    \\
    &{\smash{\bigoplus_{p = 1}^3}\:K_{D_p}} \ar[r]
    &{K_D} 
    &
  }
\end{equation*}
In the previous case, we are taking advantage of the fact that the
$L^2$ Dolbeault isomorphism and the harmonic theory are valid on the
cohomology groups of the sheaves on both the left- and
right-hand-sides of the short exact sequence, so that the
corresponding injectivity statement can be proved on each side in
the spirit of Enoki, which in turn leads to the injectivity theorem
for the cohomology groups of the middle sheaf (twisted by $F$).
In the current case, they are valid only on the left-hand-side (on
each $D_p$).
We are thus led to determine whether the injectivity statement for
the sheaf on the right-hand-side holds true.
It is then apparent that we should consider
\begin{equation*}
  \renewcommand{\objectstyle}{\displaystyle}
  \xymatrix@R=2.5em{
    0 \ar[r]
    &{K_X \otimes D \otimes \smash{\faidlof|2|/|1|*}} \ar[r]
    \ar[d]_(0.47){\Res^2}^(0.47){\isom}
    &{K_X \otimes D \otimes \faidlof|3|/|1|*} \ar[r]
    &{K_X \otimes D \otimes \smash{\faidlof|3|/|2|*}} \ar[r]
    \ar[d]^-{\Res^3}_-{\isom}
    &0 \; ,
    \\
    &{\smash[t]{\bigoplus_{\substack{p,q = 1 \\ p\neq q}}^3} K_{D_p \cap D_q}} 
    &
    &{K_{D_1 \cap D_2 \cap D_3}}
  }
\end{equation*}
which, again, has the Dolbeault and harmonic theories valid on both
sides (on each lc center of $(X,D)$) of the short exact sequence.
The arguments in Section \ref{sec:proof-of-simple-case} can then be
employed to conclude the proof.
This illustrates the idea of the inductive arguments, which reduces
the question to the union of lower dimensional lc centers of $(X,D)$
in each step, to be employed in the general proof in Section
\ref{subsec:general}. 

Another modification to the proof in Section
\ref{sec:proof-of-simple-case} is that, when the claim in Theorem
\ref{thm:main} with $q > 1$ is considered, the section $w_b$
constructed as in \eqref{eq:w-prelim-formula} is then a $K_{\lcS+1[b]}
\otimes \res F_{\lcS+1[b]}$-valued $(0,q-1)$-form on some
$(\sigma+1)$-lc center $\lcS+1[b]$.
In order to prove that $w_b =0$ by following the arguments in the
previous case, we need not only to show that $w_b$ is
$\dbar$-closed, but also that it is harmonic.
This happens to be true and the computation for checking this claim
is given in Proposition \ref{prop:harmonic-residue} and Theorem
\ref{thm:residue-harmonic}.



% In this subsection, we consider how we should generalize the proof of the previous section in dealing with the general case. 

% We first consider the slightly more general case of $D=D_{1}+D_{2}$ and $q \geq 2$. 
% In this case,  we can repeat the same argument  for Step 1. 
% Step 2 is a bit more involved since we are dealing with differential forms $u_{p}$ of higher degree, 
% but essentially the same argument can be used to define $w$ appropriately (see $\eqref{eq-def-w}$). 
% To check that $w$ is harmnic in Step 3, 
% since $w$ is an $F$-valued of the type $(n-2, q-1) \not =(n-2, q-1)$, 
% we need to check $\dfadj w_q = 0$ as well as $\dbar w=0$. 
% Nevertheless, $\dbar w=0$ is proved by the same argument 
% and $\dfadj w_q = 0$ is proved in Subsection \ref{subsec:harmonic}. 
% Performing Step 4 in the same way, 
% some global section $v$ on $D_{1}\cap D_{2}$ naturally appears, 
% we finally obtain $0=\iinner{w-\dbar v }{w}_{D_{1}\cap D_{2}} $. 
% Although the point that $v$ appears is different, since $w$ is harmonic, the conclusion that $w=0$ is immediately obtained.
% As described above, in the case of $q \geq 2$, 
% the degree of the differential form is higher and more involved, 
% but essentially the same strategy still work. 


% Next, let us consider the case of $D=D_{1}+D_{2}+D_{3}$. 
% In the case of $D=D_{1}+D_{2}$ , 
% by using the exact sequence $0 \to K_{D_{1}} \oplus K_{D_{1}} \to K_{D} \to K_{D_{1} \cap D_{2} } \to 0$, 
% we proved the injectivity of the multiplication map on (the cohomology groups of) central term. 
% Of particular importance in the proof were that the left term admits the theory of harmonic integrals and 
% that the multiplication map on the right term in injective. 
% In this subsection, we explain what kind of exact sequences in the case where $D$ has three components  
% to make the same strategy works. 
% The precise proof will be given  in the next subsection. 

% Suppose that $D$ has three components (i.e.\,$D=D_{1}+D_{2}+D_{3}$). 
% We first consider the following exact sequence twisted by $F$ (and also $M$): 
% \begin{align}\label{eq-ex}
%   \xymatrix{
%     0 \ar[r]
%     & K_{X}\otimes D \otimes{\faidlof |1|/|0|*} =\bigoplus_{p=1}^{3} K_{D_{p}} \ar[r]
%     & K_{X}\otimes D \otimes{\faidlof|\sigma_{\mlc}|/|0|*} =K_{D}        \ar[r]
%     & K_{X}\otimes D \otimes{\faidlof|\sigma_{\mlc}|/|1|*} \ar[r]
%     & 0. 
%   } 
% \end{align}
% The cohomology classes in $\oplus_{p=1}^{3} H^{q}(D_{p}, K_{D_{p}} \otimes F)$ of the left term 
% can be represented by harmonic forms. 
% Hence, whether the same argument works as in the previous subsection 
% depends on whether or not the multiplication map 
% $$
% H^{q}(X, K_{X}\otimes{\faidlof|\sigma_{\mlc}|/|1|*}\otimes F) \xrightarrow{\quad \otimes s \quad }
% H^{q}(X, K_{X}\otimes{\faidlof|\sigma_{\mlc}|/|1|*}\otimes F\otimes M)
% $$
% is injectivity or not.  
% To check this, we  consider another exact sequence: 
% \begin{align*}
%    \xymatrix{
%     0 \ar[r]
%     & K_{X}\otimes{\faidlof |2|/|1|*}  =\bigoplus_{p\not = q} K_{D_{p} \cap D_{q}} \ar[r]
%     & K_{X}\otimes{\faidlof|\sigma_{\mlc}|/|1|*}         \ar[r]
%     & K_{X}\otimes{\faidlof|\sigma_{\mlc}|/|2|*}=K_{D_{1}\cap D_{2}\cap D_{3}} \ar[r]
%     & 0}
% \end{align*}
% Then, the left term admits the theory of harmonic integrals and 
% that the multiplication map on the right term in injective.
% Therefore, we can show that he multiplication map on the central term is injective. 
% No essential difficulty appears in repeating this inductive argument in the general case. 


\subsection{Proof of Theorem \ref{thm:main} in general}\label{subsec:general}


%\input{outline-of-proof}

%%%%%
%%%%% File name  : outline-of-proof.tex
%%%%% Author     : Mario Chan
%%%%% Date       : 6th March, 2023
%%%%% Description: This is the outline of the proof of the general
%%%%%              case of the project "Injectivity-Fujino".
%%%%%
%%
%%%

\renewcommand{\objectstyle}{\displaystyle}

Write
\begin{gather*}
  \aidlof* := \aidlof =\mtidlof{\vphi_F} \cdot \defidlof{\lcc+1'}
  =\defidlof{\lcc+1'} \; , \quad
  \residlof* := \residlof \isom \faidlof/-1*  \\
  \text{and } \quad \spH{\sheaf F}
  :=\cohgp q[X]{\logKX \otimes \sheaf F} 
\end{gather*}
for convenience.
Recall that
\begin{equation*}
  K_D = K_X \otimes D \otimes \faidlof|\sigma_{\mlc}|/|0|* \; ,
\end{equation*}
and the inclusions between adjoint ideal sheaves induce the short exact
sequences
\begin{equation*} % \label{eq-ex2}
  \xymatrix{
    0 \ar[r]
    & {\faidlof/|\rho|*} \ar[r]
    & {\faidlof|\tau|/|\rho|*} \ar[r]
    & {\faidlof|\tau|/*} \ar[r]
    & 0
  } \quad\text{ for } 0 \leq \rho \leq \sigma \leq \tau \; .
\end{equation*}
One is thus led to consider the commutative diagram
\subfile{commut-diagram_sing-Fujino-conj}%
for $\sigma =2,\dots,\sigma_{\mlc}$, in which the columns are exact,
$\iota_\sigma$ and $\tau_\sigma$ are induced from the inclusions
between adjoint ideal sheaves, and $\mu_\sigma$ (resp.~$\nu_\sigma$)
is the composition of $\iota_\sigma$ (resp.~$\tau_\sigma$) with the
map induced from the multiplication map $\otimes s$.
The statement in Theorem \ref{thm:main} is proved if one shows that
$\ker\mu_{\sigma_{\mlc}} = \ker\iota_{\sigma_{\mlc}} = 0$
($\iota_{\sigma_{\mlc}}$ is the identity map).
Note that $\mu_1 =\nu_1$ and $\iota_1 =\tau_1$.
Following the argument in \cite{Chan&Choi_injectivity-I}*{Thm.~1.3.2}, since
$\ker\mu_{\sigma-1} =\ker\iota_{\sigma-1}$ and $\ker\nu_\sigma
=\ker\tau_\sigma$ together imply $\ker\mu_{\sigma}
=\ker\iota_{\sigma}$ via a diagram-chasing argument, to prove Theorem
\ref{thm:main}, it suffices to show the following theorem.

\begin{thm} \label{thm:ker-nu=ker-tau}
  $\ker\nu_\sigma =\ker\tau_\sigma$ for all $\sigma =1, \dots, \sigma_{\mlc}$.
\end{thm}


\begin{remark} \label{rem:general-commut-diagram}
  When $\aidlof|0|*$ in the commutative diagram
  \eqref{eq:commut-diagram_sing-Fujino-conj} is replaced by $0$ (which
  can be considered as $\aidlof|-1|*$), the setup is reduced to the one in
  \cite{Chan&Choi_injectivity-I}*{Thm.~1.3.2}, which states that
  Theorem \ref{thm:ker-nu=ker-tau} together with the result in
  \cite{Matsumura_injectivity-lc}*{Thm.~1.6} or
  \cite{Chan&Choi_injectivity-I}*{Thm.~1.2.1} implies that Fujino's
  conjecture is true.
  As a matter of fact, the proof of Theorem \ref{thm:ker-nu=ker-tau}
  can also be adapted to the case $\sigma = 0$ (with $\aidlof|-1|* =
  0$ and $\residlof|0|* =D^{-1} \isom \defidlof{D} =\aidlof|0|*$) which recovers the result in
  \cite{Matsumura_injectivity-lc}*{Thm.~1.6} as well as
  \cite{Chan&Choi_injectivity-I}*{Thm.~1.2.1}.
  Furthermore, by replacing $\aidlof|0|*$ by $\aidlof|\sigma_0-1|*$
  and $\aidlof|\sigma_{\mlc}|*$ by $\aidlof|\sigma'|*$ for any $0 <
  \sigma_0 \leq \sigma'$ and letting $\sigma$ vary within the range
  $\sigma_0 < \sigma \leq \sigma'$ in the diagram
  \eqref{eq:commut-diagram_sing-Fujino-conj}, one sees that the proof
  of Theorem \ref{thm:ker-nu=ker-tau} guarantees the statement of
  Theorem \ref{thm:main} but with $K_D$ replaced by $K_X \otimes D
  \otimes \faidlof|\sigma'|/|\sigma_0 -1|*$.
\end{remark}


\begin{proof}
  The proof consists of the following steps.
  \begin{enumerate}[label=\textbf{Step \Roman*:}, ref=\Roman*,
    leftmargin=0pt, labelsep=*, widest=VI, itemindent=*, align=left,
    itemsep=1.5ex]
  \item Make use of the $L^2$ Dolbeault and harmonic theory available on
    $\spH{\residlof*}$.

    Write $\lcc' =\bigcup_{p \in \Iset} \lcS$ as the union of
    $\sigma$-lc centers $\lcS$ of $(X,D)$.  Notice that $\residlof*$,
    hence $\spH{\residlof*}$, has a decomposition as a direct sum
    which yields
    \begin{equation*}
      \spH{\residlof*}
      =\bigoplus_{p \in \Iset} \cohgp q[\lcS]{K_{\lcS}
        \otimes F \otimes \mtidlof<\lcS>{\vphi_F}}
      =\bigoplus_{p \in \Iset} \cohgp q[\lcS]{K_{\lcS} \otimes F}
    \end{equation*}
    such that the $L^2$ Dolbeault isomorphism and harmonic theory are
    valid for the cohomology group in each summand.
    Take the (squared) residue norm
    $\norm\cdot_{\lcc'}^2 = \sum_{p \in \Iset} \norm\cdot_{\lcS}^2$ as
    the $L^2$ norm on $\spH{\residlof*}$.  Pick any element
    $u := (u_p)_{p \in \Iset} \in \spH{\residlof*}$ such that
    \begin{itemize}
    \item each $u_p$ is a harmonic form on $\lcS$ with respect to the
      given norm $\norm\cdot_{\lcS}$ and
    
    \item $u \in \ker\nu_\sigma \cap \paren{\ker\tau_\sigma}^\perp$,
      where the orthogonal complement $\paren{\ker\tau_\sigma}^\perp$
      of $\ker\tau_\sigma$ is taken with respect to the residue norm
      $\norm\cdot_{\lcc'}$.
    \end{itemize}
    The theorem is proved if it is shown that $u_p = 0$ for all
    $p \in \Iset$.

  
  \item \label{item:express-su-in-residue-norm}
    Obtain an expression of $\norm{su}_{\lcc'}^2$ using the
    assumption $u \in \ker\nu_\sigma$ and the \v Cech--Dolbeault
    isomorphism.

    Let $\cvr V :=\set{V_i}_{i \in I}$ be a locally finite cover of
    $X$ by admissible open sets with respect to
    $(\vphi_F,\vphi_M,\psi_D)$ and let $\set{\rho^i}_{i \in I}$ be a
    partition of unity subordinate to $\cvr V$.
    Their notations are abused to mean also their induced cover and
    partition of unity on $\lcc'$ for any $\sigma \geq 0$.
    For any choice of indices $\idx 0,q \in I$, write $V_{\idx 0.q}
    :=V_{i_0} \cap V_{i_1} \dotsm \cap V_{i_q}$ as usual.
  
    Through the \v Cech--Dolbeault isomorphism, every (cohomology
    class of) $u_p$ is represented by a \v Cech $q$-cocycle
    $\set{\alpha_{p; \:\idx 0.q}}_{\idx 0,q \in I}$ such that (under
    the Einstein summation convention on the indices $\idx 0,q$)
    \begin{equation*}
      u_p
      % &= \dbar v_{p;(2)} +\dbar \rho^{i_{q-1}} \wedge \dotsm \wedge
      %   \dbar\rho^{i_0} \alpha_{p; \:\idx 0.q} \qquad\paren{\forall~ i_q \in I} \\
      \overset{\text{\eqref{eq:Cech-Dolbeault-isom}}}=
      \:\dbar v_{p;(2)}
      +(-1)^q \:\underbrace{\dbar \rho^{i_{q}} \wedge \dotsm \wedge
        \dbar\rho^{i_1} \cdot \rho^{i_0} }_{=: \:
        \paren{\dbar\rho}^{\idx q.0}} \alpha_{p; \:\idx 0.q} \; ,
    \end{equation*}
    where $v_{p; (2)}$ is a $K_{\lcS} \otimes \res{F}_{\lcS}$-valued $(0,q-1)$-form
    on $\lcS$ with $L^2$ coefficients with respect to
    $\norm\cdot_{\lcS}$ and
    $\alpha_{p; \:\idx 0.q} \in K_{\lcS} \otimes \res F_{\lcS} \otimes
    \mtidlof<\lcS>{\vphi_F} =K_{\lcS} \otimes \res F_{\lcS}$ on
    $V_{\idx 0.q}$.
    % (see \cite{Matsumura_injectivity}*{Prop.~5.5} or
    % \cite{Chan&Choi_injectivity-I}*{Lemma 3.2.1}). 
    In view of the
    residue short exact sequence, choose, for each choice of
    the multi-indices $(\idx 0,q)$, a section
    $f_{\idx 0,q} \in \logKX M \otimes \aidlof*$ on $V_{\idx 0.q}$
    such that
    \begin{equation*}
      \Res^\sigma(f_{\idx 0.q})
      =\paren{\alert{s} \alpha_{p; \:\idx 0.q}}_{p \in \Iset} 
    \end{equation*}
    (note that $V_{\idx 0.q}$ is Stein).  Considering the inclusion
    $\aidlof* \subset \aidlof|\sigma_{\mlc}|*$, write
    \begin{equation*}
      \eqcls{f_{\idx 0.q}} := \paren{f_{\idx 0.q} \bmod \aidlof-1*}
      \;\in \logKX M \otimes \faidlof|\sigma_{\mlc}|/-1*
      \quad\text{ on } V_{\idx 0.q} \; .
    \end{equation*}
    The collection $\set{\eqcls{f_{\idx 0.q}}}_{\idx 0,q \in I}$ is
    then a \v Cech $q$-cocycle representing $\nu_\sigma(u)$ in $\spH
    M{\faidlof|\sigma_{\mlc}|/-1*}$.
    The assumption $u \in \ker\nu_\sigma$ implies that this cocycle is
    a coboundary, that is,
    \begin{equation*}
      \set{\eqcls{f_{\idx 0.q}}}_{\idx 0,q \in I}
      =\delta\set{\eqcls{\lambda_{\idx 1.q}}}_{\idx 1,q \in I}
      =\set{\eqcls{\paren{\delta\lambda}_{\idx 0.q} }}_{\idx 0,q \in I}
    \end{equation*}
    for some $\lambda_{\idx 1.q} \in \logKX M \otimes
    \aidlof|\sigma_{\mlc}|*$ on $V_{\idx 1.q}$ (note that
    $\lambda_{\idx 1.q}$ need \emph{not} take values in $\aidlof*$
    even though the sections $f_{\idx 0.q}$ do), where
    $\paren{\delta\lambda}_{\idx 0.q}$ is given by the usual formula
    of \v Cech coboundary operator $\paren{\delta\lambda}_{\idx 0.q}
    :=\sum_{k =0}^q (-1)^k \lambda_{\idx 0[\dotsm \widehat{i_k}].q}$.
    Notice that $f_{\idx 0.q}$ and $\paren{\delta\lambda}_{\idx 0.q}$
    differ by an element in $\logKX M \otimes \aidlof-1*$ on $V_{\idx
      0.q}$.
    

    Thanks to the positivity $\ibddbar\vphi_F \geq 0$ and the bound
    $\ibddbar\vphi_M \leq C \ibddbar\vphi_F$ for some
    constant $C > 0$ on each $\lcS$, the product $s u_p$ is harmonic
    with respect to $\norm\cdot_{\lcS}$ (Proposition
    \ref{prop:consequence-of-positivity}), so $\iinner{s u_p}{s
      \:\dbar  v_{p;(2)}}_{\lcS} = \iinner{s u_p}{\dbar \paren{s
        v_{p;(2)}}}_{\lcS} = 0$ for every $p \in\Iset$.
    It follows that
    \begin{align*}
      \norm{su}_{\lcc'}^2
      =\sum_{p\in \Iset} \norm{su_p}_{\lcS}^2
      =&~(-1)^q \sum_{p\in \Iset} \sum_{\idx 0,q \in I} \iinner{\paren{\dbar\rho}^{\idx q.0}
        \:s \alpha_{p; \:\idx 0.q}}{\: s u_p}_{\lcS}
      \\
      =&~(-1)^q \sum_{p\in \Iset} \sum_{\idx 0,q \in I} \iinner{
         s \alpha_{p; \:\idx 0.q}
         }{\:\idxup{\diff\rho},[\idx 0.q].  s u_p}_{\lcS}
         \; ,
    \end{align*}
    where $\idxup{\diff\rho},[\idx 0.q].  \cdot $ is the adjoint
    of $\paren{\dbar\rho}^{\idx q.0} \cdot$.
    As in Step \ref{item:expression-of-su-simple} in Section
    \ref{sec:proof-of-simple-case}, the desired expression can be
    obtained by substituting
    \begin{equation*}
      s\alpha_{p; \:\idx 0.q} =\PRes[\lcS](\frac{f_{\idx
          0.q}}{\sect_D})
      =\PRes[\lcS](\frac{\paren{\delta\lambda}_{\idx
          0.q}}{\sect_D})
      =\frac{\paren{\delta \rs*\lambda_p}_{\idx 0.q}}{\sect_{(p)}}
      \; ,
    \end{equation*}
    where $\rs*\lambda_{p; \:\idx 1.q}
    :=\PRes[\lcS](\frac{\lambda_{\idx 1.q}}{\sect_D}) \cdot
    \sect_{(p)}$.
    For the sake of illustration, an alternative approach via a
    direct residue computation is presented here.
    Note that $\paren{\idxup{\diff\rho},[\idx 0.q].  s u_p}_{p \in
      \Iset} \in \logKX M \otimes \smooth_{X\:c\,*} \cdot\residlof*$ on $V_{\idx
      0.q}$, so it has a preimage $h^{\idx 0.q} \in \logKX M \otimes
    \smooth_{X\:c\,*} \cdot \aidlof*$ of $\Res^\sigma$ (considered as a
    $\smooth_{X\:c\,*}$-homomorphism).
    Fix such preimage on each open set $V_{\idx 0.q}$.
    From the direct computation of the residue function, it follows
    that
    \begin{align*}
      (-1)^q \:\norm{su}_{\lcc'}^2
      =&~\sum_{p\in \Iset} \sum_{\idx 0,q \in I} \iinner{
         s \alpha_{p; \:\idx 0.q}
         }{\:\idxup{\diff\rho},[\idx 0.q].  s u_p}_{\lcS}
      \\
      \xleftarrow{\eps \tendsto 0^+}
       &~\smashoperator[l]{\sum_{\idx 0,q \in I}} \eps
         \int_{\mathrlap{V_{\idx 0.q}}} \quad \frac{
         \inner{f_{\idx 0.q}}{h^{\idx 0.q}}
         \:e^{-\phi_D -\vphi_F -\vphi_M}
         }{\abs{\psi_D}^{\sigma +\eps}}
      \\
      \overset{\mathclap{\text{Prop.~\ref{prop:residue-product-X-to-lcS}}}}= \quad\;
       &~\smashoperator[l]{\sum_{\idx 0,q \in I}} \eps
         \int_{\mathrlap{V_{\idx 0.q}}} \quad \frac{
         \inner{\paren{\delta\lambda}_{\idx 0.q}}{h^{\idx 0.q}}
         \:e^{-\phi_D -\vphi_F -\vphi_M}
         }{\abs{\psi_D}^{\sigma +\eps}} +\BigO(\eps)
      \\
      \xrightarrow[\text{Prop.~\ref{prop:residue-product-X-to-lcS}}]{\eps \tendsto 0^+}
      %  &~\sum_{p\in \Iset} \sum_{\idx 0,q \in I}
      %    \int_{\lcS} \inner{
      %    \frac{\paren{\delta\rs*\lambda_p}_{\idx 0.q}}{ \sect_{(p)}}
      %    }{\:
      %    \idxup{\diff\rho},[\idx 0.q].  s u_p
      %    } \:e^{-\vphi_F-\vphi_M}
      % \\
      % =&~\sum_{p\in \Iset} \sum_{\idx 0,q \in I}
      %    \int_{\lcS} \inner{
      %    \paren{\dbar\rho}^{\idx q.0}
      %    \paren{\delta\rs*\lambda_p}_{\idx 0.q}
      %    }{\:
      %     s u_p \sect_{(p)}
      %    }_\omega \:e^{-\phi_{(p)}-\vphi_F-\vphi_M}
      % \\
      % =
       &~\sum_{p\in \Iset} \sum_{\idx 0,q \in I}
         \iinner{\paren{\dbar\rho}^{\idx q.0}
         \paren{\delta\rs*\lambda_p}_{\idx 0.q}}{\: s u_p
         \sect_{(p)}}_{\lcS, \phi_{(p)}}
      \\
      =&~\sum_{p\in \Iset} \sum_{\idx 1,q \in I}
         \iinner{\dbar\rho^{i_q} \wedge \dotsm \wedge \dbar\rho^{i_1}
         \cdot\rs*\lambda_{p;\:\idx 1.q}}{\: s u_p \sect_{(p)}}_{\lcS,
         \phi_{(p)}}
      \\
      =&~(-1)^{q-1} \sum_{p\in \Iset} \iinner{\dbar v_{p;(\infty)}}{\: s u_p
         \sect_{(p)}}_{\lcS, \phi_{(p)}}
         \; ,\footnotemark
    \end{align*}%
    \footnotetext{
      If the $L^2$ Dolbeault isomorphism is valid for $\spH
      M{\faidlof|\sigma_{\mlc}|/-1*}$, such conclusion can be
      obtained simply from the fact that $\nu_\sigma(su)$ is
      represented by a smooth $\dbar$-exact form on $\lcc'$.
    }%
    where
    % $\rs*\lambda_{p; \:\idx 1.q}
    % :=\PRes[\lcS](\frac{\lambda_{\idx 1.q}}{\sect_D}) \cdot \sect_{(p)}$ and 
    $v_{p; (\infty)} :=\sum_{\idx 1,q\in I} \dbar\rho^{i_q} \wedge \dotsm
    \wedge \dbar\rho^{i_2} \cdot \rho^{i_1} \rs*\lambda_{p;\:\idx 1.q}
    =\sum_{\idx 1,q\in I} \paren{\dbar\rho}^{\idx q.1}\rs*\lambda_{p; \:\idx 1.q}$. 

    % Note that $s u_p$ is harmonic with respect to the potential
    % $\vphi_F+\vphi_M$ by the positivity assumption.
    The expression of $\norm{su}_{\lcc'}^2$ can be further transformed
    by an integration by parts using Proposition
    \ref{prop:res-formula-dbar-exact-dot-harmonic}, which becomes
    \begin{equation*}
      \norm{su}_{\lcc'}^2
      % &=\smashoperator[l]{\sum_{\idx 1,q \in I}} \sum_{b \in \Iset+1}
      % \sum_{j=1}^{\sigma +1} (-1)^q \:\sigma
      % \iinner{ \sgn{b:p_{b,j}}\:
      % \frac{\rs*\lambda_{b;\:\idx 1.q}}{\sect_{(b)}}
      % }{\: s\:
      %   \idxup{\diff\rho},[\idx 1.q] .
      %   \PRes[b(j)](\idxup{\diff\psi_{(p_{b,j})}}.  u_{p_{b,j}})
      % }_{\lcS+1[b]}
      %   \\
      = \sigma\sum_{b \in \Iset+1}
      \iinner{v_{b;(\infty)} \:
      }{ \quad s \:
        \smashoperator{\sum_{p \in \Iset \colon \lcS+1[b] \subset
            \lcS}} \;\; \sgn{b:p}\:
        \PRes[\lcS+1[b] | \lcS](\idxup{\diff\psi_{(p)}}.  u_{p})
        \cdot \sect_{(b)}
      }_{\lcS+1[b], \phi_{(b)}} \; ,
    \end{equation*}
    where $v_{b;(\infty)} := \sum_{\idx 1,q \in I}
    \paren{\dbar\rho}^{\idx q.1} \rs*\lambda_{b; \:\idx 1.q}$ and
    $\rs*\lambda_{b; \:\idx 1.q}
    :=\PRes[\lcS+1[b]](\frac{\lambda_{\idx 1.q}}{\sect_D}) \cdot \sect_{(b)}$.
    % However, notice that $v_{p;(\infty)}$ is smooth on $\lcS$ but not
    % locally $L^2$ with respect to the weight $e^{-\phi_{(p)}}$.
    % For this reason, let $\psi_{(p)} :=\phi_{(p)} -\sm\vphi_{(p)}$, where
    % $\sm\vphi_{(p)}$ is some smooth potential on $\Diff_p D$.
    % One then has
    % \begin{align*}
    %   &~\norm{su}_{\lcc'}^2
    %   =\sum_{p\in \Iset} \iinner{\dbar v_{p;(\infty)}}{\: s u_p
    %      \sect_{(p)}}_{\lcS, \phi_{(p)}}
    %   \\
    %   \xleftarrow{\eps \tendsto 0^+}
    %    &~\sum_{p \in \Iset} \iinner{
    %      e^{-\eps \abs{\psi_{(p)}}} \:\dbar v_{p;(\infty)}
    %      }{\: s u_p \sect_{(p)}}_{\lcS, \phi_{(p)}}
    %   \\
    %   =&~\sum_{p \in \Iset} \paren{
    %      \cancelto{0}{\iinner{
    %      \dbar\paren{e^{-\eps \abs{\psi_{(p)}}} \: v_{p;(\infty)}}
    %      }{\: s u_p \sect_{(p)}}_{\mathrlap{\lcS, \phi_{(p)}}}}
    %      \quad\;\; + \eps 
    %      \iinner{
    %      e^{-\eps \abs{\psi_{(p)}}} \:v_{p;(\infty)}
    %      }{\:\idxup{\diff\psi_{(p)}} . s u_p \sect_{(p)}}_{\lcS,
    %      \phi_{(p)}}
    %      }
    %   \\
    %   =&~\sum_{p \in \Iset} \sum_{\idx 1,q \in I} (-1)^q \:\eps \:
    %      \iinner{
    %      e^{-\eps \abs{\psi_{(p)}}} \: % \paren{\dbar\rho}^{\idx q.1}
    %      \rs*\lambda_{p;\:\idx 1.q}
    %      }{\:
    %      \idxup{\diff\rho},[\idx 1.q] .
    %      \paren{\idxup{\diff\psi_{(p)}} . s u_p \sect_{(p)}}
    %      }_{\lcS, \phi_{(p)}}
    %   \\
    %   \xrightarrow[\text{Prop.~\ref{prop:residue-formula-classical-kernel}}]{\eps
    %   \tendsto 0^+} 
    %   &~\smashoperator[l]{\sum_{\idx 1,q \in I}} \sum_{p \in \Iset}
    %     \sum_{k=\sigma +1}^{\mathclap{\sigma_{V_{\idx 1.q}}}} (-1)^q \:\sigma
    %     \iinner{
    %     \PRes[p(k)](
    %     \frac{\rs*\lambda_{p;\:\idx 1.q}}{\sect_{(p)}}
    %     )
    %      }{\:
    %      \idxup{\diff\rho},[\idx 1.q] .
    %      \PRes[p(k)](\idxup{\diff\psi_{(p)}} . s u_p)
    %      }_{\lcS \cap \set{z_{p(k)} =0}}
    %   \; ,
    % \end{align*}
    % where $\PRes[p(k)]$ denotes the Poincar\UTF{00E9} residue map from $\lcS$
    % to $\lcS \cap \set{z_{p(k)}=0}$. 
    % The last limit is justified as follows.
    % On the admissible open set $V_{\idx 1,q}$, consider a holomorphic
    % coordinate system $(z_1, \dots, z_n)$ such that $\lcS
    % =\set{z_{p(1)} = \dotsm =z_{p(\sigma)} =0}$ and
    % $\sect_{(p)} =z_{p(\sigma+1)} \dotsm z_{p(\sigma_V)}$ (write
    % $\sigma_{V}$ for $\sigma_{V_{\idx 1.q}}$ for convenience).
    % Note that
    % \begin{equation*}
    %   \diff\psi_{(p)} =\sum_{k =\sigma +1}^{\sigma_V}
    %   \frac{dz_{p(k)}}{z_{p(k)}} -\diff\sm\vphi_{(p)} \quad\text{ on }
    %   V_{\idx 1,q} \; .
    % \end{equation*}
    % It follows that, on $\lcS \cap V_{\idx 1.q}$,
    % \begin{equation*}
    %   \begin{multlined}
    %     \text{coef.~of }\:
    %     \idxup{\diff\rho},[ \idx 1.q] .
    %     \paren{\idxup{\diff\psi_{(p)}} . s u_p \sect_{(p)}}
    %   \end{multlined}
    %   \in
    %   \res{\defidlof{\lcc+2'}}_{\lcS}
    %   \begin{aligned}[t]
    %     &=\mtidlof<\lcS>{\vphi_F+\vphi_M} \cdot
    %     \res{\defidlof{\lcc+2'}}_{\lcS} \;\;\footnotemark
    %     \\
    %     &=\aidlof|1|<\lcS>{\vphi_F+\vphi_M}[\psi_{(p)}]
    %   \end{aligned}
    % \end{equation*}%
    % \footnotetext{
    %   Recall that $\defidlof{\lcc+2'}$ is generated on $X$ by
    %   $\sect_{(\sigma+1 : b)}$ for all $b \in \Iset+1$ treated as local
    %   functions.
    %   On an admissible open set $V$, one has $\defidlof{\lcc+2'}
    %   =\genbyd{z_{b(\sigma+2)} \dotsm
    %     z_{b(\sigma_V)}}{b \in \Iset+1 \text{ such that } \lcS+1[b] \cap
    %     V \neq \emptyset}$ (see page
    %   \pageref{page:notation-permutation-index} for the notation).
    % }%
    % and, therefore, one can apply Proposition
    % \ref{prop:residue-formula-classical-kernel} (with $\lcS$ in place
    % of $X$, $\psi_{(p)}$ in place of $\psi_D$) to each inner product
    % $\eps \iinner{\dotsm}{ \:\dotsm \idxup{\diff\psi_{(p)}}. \dotsm
    %   \sect_{(p)}}_{\lcS,\phi_{(p)}}$.
    % % to obtain a sum of integrals on
    % % each $\lcS \cap \set{z_{p(k)} = 0}$ for $k=\sigma +1, \dots,
    % % \sigma_V$, i.e.~on the $(\sigma+1)$-lc centers in $\lcc+1' \cap
    % % V_{\idx 1.q}$.
    
    % Write $\lcc+1' =\bigcup_{b \in \Iset+1} \lcS+1[b]$.
    % On each admissible open set $V_{\idx 1.q}$, the intersection $\lcS
    % \cap \set{z_{p(k)} = 0}$ is a $(\sigma+1)$-lc center $\lcS+1[b_{p,k}]
    % \cap V_{\idx 1.q}$ ($\neq \emptyset$), uniquely determined by the
    % choices of $p\in \Iset$ (such that $\lcS \cap V_{\idx 1.q} \neq
    % \emptyset$, so $\binom{\sigma_V}{\sigma}$ choices) and $k
    % =\sigma+1, \dots, \sigma_V$ (so $\sigma_V-\sigma$ choices).
    % To get an indexing in terms of $b \in \Iset+1$ (such that
    % $\lcS+1[b] \cap V_{\idx 1.q} \neq \emptyset$, so
    % $\binom{\sigma_V}{\sigma +1}$ choices), note that each $\lcS+1[b]
    % \cap V_{\idx 1.q}$ is contained in $\sigma +1$ distinct
    % $\sigma$-lc centers $\lcS[p_{b,j}]$ for $j=1,\dots,\sigma+1$
    % (apparently, $\sigma +1$ choices) such that
    % \begin{equation*}
    %   \lcS+1[b] \cap V_{\idx 1.q} = \lcS[p_{b,j}] \cap \set{z_{b(j)} = 0} \; .
    % \end{equation*}
    % (One can verify $\sum_{p \in \Iset} \sum_{k=\sigma
    %   +1}^{\sigma_{V}} \dotsm = \sum_{b \in
    %   \Iset+1} \sum_{j=1}^{\sigma +1} \dotsm$ by first noting that
    % $\binom{\sigma_V}{\sigma} (\sigma_V -\sigma)
    % =\binom{\sigma_V}{\sigma +1} (\sigma+1)$.)
    % With such choice of indexing, let $\sgn{b:p_{b,j}}$ be the sign
    % given by
    % \begin{equation*}
    %   \PRes[\lcS+1[b]]
    %   =\sgn{b:p_{b,j}} \:\PRes[b(j)]\circ \PRes[\lcS[p_{b,j}]] \; .
    % \end{equation*}
    % Therefore, one has
    % \begin{equation*}
    %   \frac{\rs*\lambda_{b;\: \idx 1.q}}{\sect_{(b)}}
    %   :=\PRes[\lcS+1[b]](\frac{\lambda_{\idx 1.q}}{\sect_D})
    %   % =\sgn{b:p_{b,j}} \:\PRes[b(j)]\circ
    %   % \PRes[\lcS[p_{b,j}]](\frac{\lambda_{\idx 1.q}}{\sect_D})
    %   =\sgn{b:p_{b,j}} \:
    %   \PRes[b(j)](\frac{\rs*\lambda_{p_{b,j};\:\idx
    %       1.q}}{\sect_{(p_{b,j})}})
    % \end{equation*}
    % (recalling that $\sect_{(b)} =\sect_{(\sigma+1 : b)}$,
    % $\sect_{(p_{b,j})} =\sect_{(\sigma : p_{b,j})}$ and
    % $\sect_{(p_{b,j})} = z_{b(j)} \sect_{(b)}$).
    
    Set
    \begin{equation}\label{eq-def-w}
      w_b := \smashoperator[r]{\sum_{p \in \Iset \colon \lcS+1[b] \subset
          \lcS}} \;\; \sgn{b:p}\:
      \PRes[\lcS+1[b] | \lcS](\idxup{\diff\psi_{(p)}} . u_{p})
      \; .
    \end{equation}
    It suffices to show that $w_b = 0$ on $\lcS+1[b]$ for each $b
    \in\Iset+1$ to conclude the proof.
    

  \item Show that $w_b$ is harmonic with respect to
    $\res{\vphi_F}_{\lcS+1[b]}$ (and $\res{\omega}_{\lcS+1[b]}$) on
    $\lcS+1[b]$ for all $b \in \Iset+1$ and thus $\paren{w_b}_{b
      \in\Iset+1}$ represents a class in $\spH/q-1/{\residlof+1*}$.

    {
      \newcommand{\lcSb}{\lcS+1[b]}
      % \newcommand{\idxj}{\idx[\conj j]}
      
      To see that $w_b$ is $\dbar$-closed on $\lcSb$, it suffices to
      show that $\PRes[\lcS+1[b] | \lcS](\idxup{\diff\psi_{(p)}}. 
      u_{p})$ is $\dbar$-closed for all $p\in\Iset$ such that $\lcSb
      \subset \lcS$.
      Take any admissible open set $V$ such that $V \cap \lcSb
      \neq\emptyset$ and a holomorphic coordinate system such that
      $\sect_{(p)} = z_{p(\sigma+1)} \dotsm z_{p(\sigma_V)}$ on $V$.
      Suppose $\lcSb \cap V = \lcS \cap \set{z_{p(k)} = 0}$ for some $k
      =\sigma +1, \dots, \sigma_V$.
      Recall that
      \begin{equation*}
        \diff\psi_{(p)} = \sum_{k'=\sigma+1}^{\sigma_V}
        \frac{dz_{p(k')}}{z_{p(k')}} - \diff\sm\vphi_{(p)} \quad\text{
          on } V \; .
      \end{equation*}
      By writing
      \begin{equation*}
        \idxup{dz_{p(k)}}.  u_p =: dz_{p(k)} \wedge
        \paren{\idxup{dz_{p(k)}}.  \rs u_{p,k}} \quad\text{ on }
        \lcS \cap V \; ,
      \end{equation*}
      in which $\rs u_{p,k}$ is a $(n-\sigma-1,q)$-form on $\lcS \cap
      V$, it follows that
      \begin{equation*}
        \PRes[\lcS+1[b] | \lcS](\idxup{\diff\psi_{(p)}}.  u_{p})
        =\PRes[\set{z_{p(k)}=0}](\frac{\idxup{dz_{p(k)}}. 
          u_p}{z_{p(k)}})
        =\parres{\idxup{dz_{p(k)}} . \rs u_{p,k}}_{\lcSb}
        \quad\text{ on } \lcSb \cap V \; .
      \end{equation*}
      It thus suffices to show that $\idxup{dz_{p(k)}}.  u_p$ is
      $\dbar$-closed on $\lcS \cap V$.
      % , which is guaranteed by Lemma
      % \ref{lem:commutator-dbar-ctrt}.
      Since $u_p$ is harmonic and $\ibddbar\vphi_F \geq 0$, it follows
      that $\dbar u_p = 0$ and $\nabla^{(0,1)}u_p = 0$ (Proposition
      \ref{prop:consequence-of-positivity}).
      Putting $u_p$ into $u$ and $z_{p(k)}$ into $\vphi$ in Lemma
      \ref{lem:commutator-dbar-ctrt}, one has
      $\dbar\paren{\idxup{dz_{p(k)}}.  u_p} = 0$ on $\lcS \cap V$.
      As a result, $w_b$ is $\dbar$-closed on $\lcSb$ and
      $\paren{w_b}_{b\in\Iset+1}$ therefore represents a class in
      $\spH/q-1/{\residlof+1*}$.
      % This can be done via the following formula, which is a special
      % case and a slight variant of \cite{Donnelly&Xavier}*{(2.4)} and
      % \cite{Ohsawa&Takegoshi-spectral_seq}*{Prop.~1.5} (see also
      % \cite{Takegoshi_higher-direct-images}*{(1.9)} and
      % \cite{Matsumura_injectivity-Kaehler}*{Lemma 2.1}).
      
      % \begin{lemma}[cf.~\cite{Donnelly&Xavier}*{(2.4)},
      %   \cite{Ohsawa&Takegoshi-spectral_seq}*{Prop.~1.5},
      %   \cite{Takegoshi_higher-direct-images}*{(1.9)} and
      %   \cite{Matsumura_injectivity-Kaehler}*{Lemma
      %     2.1}] \label{lem:commutator-dbar-ctrt}
      %   Let $\vphi$ be a smooth function and $u$ be a smooth
      %   ($K_X$-valued) $(0,q)$-form on a K\"ahler manifold.
      %   They satisfy the formula
      %   \begin{equation*}
      %     \dbar\paren{\idxup{\diff\vphi}.  u}
      %     =\mhlight{\idxup{\ibddbar\vphi}.  u}
      %     -\idxup{\diff\vphi} . \paren{\dbar u}
      %     +\idxup{\diff\vphi} \cdot \nabla^{(0,1)}_\bullet u \; ,
      %   \end{equation*}%
      %   % \mariocomment{I treat $u$ as a $(0,q)$-form, so it doesn't
      %   %   make sense to apply $\Lambda_\omega$ to $u$.
      %   %   Do you accept this or we have to treat sections of $K_X$ as
      %   %   $(n,0)$-forms?
      %   %   I don't recommend the latter choice since it's easier to say
      %   %   and understand a ``$K_{\lcS}$-valued section'' than a
      %   %   ``$(n-\sigma,\bullet)$-form''.
      %   % }%
      %   \mariocomment[Red]{SM is treating $u$ as an $F$-valued $(n-\sigma, \bullet)$-form in my head. 
      %     I understand that the reason why the vanishing theorem holds for the adjoint bundle $K \times F$ 
      %     comes from the special property of $(n,q)$-type, so this is
      %     more natural for me.}%
      %   \mmark*{}{MC: OK. Since formula in Sec.
      %       \ref{subsec:n2} Step 3 also goes without $\Lambda_\omega$,
      %     I feel safe to keep things as they are. BTW, I prefer to
      %     keep this proof after I compare the type of $\diff\vphi$ and
      %   $\dbar\Phi$ in the referred papers.}%
      %   or, when a local holomorphic coordinate system is fixed and
      %   the Einstein summation convention is applied, 
      %   \begin{equation*}
      %     \paren{\dbar\paren{\idxup{\diff\vphi}.  u}}_{\conj J_{q}}
      %     =\sum_{\nu=1}^q \diff^{\conj\ell} \diff_{\conj j_\nu} \vphi \:
      %     u_{\idxj 1[\dotsm (\conj \ell)_\nu].q}
      %     -\diff^{\conj\ell}\vphi  \:\paren{\dbar u}_{\conj\ell\conj J_q}
      %     +\diff^{\conj\ell}\vphi \:\nabla_{\conj\ell} u_{\conj J_q} 
      %   \end{equation*}
      %   for any multi-indices $J_q = (\idx[j]1,q)$, pointwisely.
      % \end{lemma}

      % \begin{proof}
      %   A direct computation yields
      %   \begin{align*}
      %     \paren{\dbar\paren{\idxup{\diff\vphi} . u}}_{\conj
      %     J_{q}}
      %     &=\sum_{\nu=1}^q (-1)^{\nu-1} \diff_{\conj j_\nu}
      %       \paren{\idxup{\diff\vphi}.  u}_{\idxj 1[\dotsm \widehat
      %       {\conj j}_\nu].q}
      %       =\sum_{\nu=1}^q (-1)^{\nu-1} \diff_{\conj j_\nu}
      %       \paren{\diff_{\ell}\vphi \: u^\ell_{\;\idxj 1[\dotsm
      %       \widehat{\conj j}_\nu].q}}
      %     \\
      %     &=\sum_{\nu=1}^q (-1)^{\nu-1} \paren{
      %       \diff_{\conj j_\nu}\diff_{\ell}\vphi \: u^{\ell}_{\;\idxj 1[\dotsm
      %       \widehat {\conj j}_\nu].q}
      %       +\diff_{\ell}\vphi \: \nabla_{\conj j_\nu} u^\ell_{\;\idxj 1[\dotsm
      %       \widehat {\conj j}_\nu].q}
      %       }
      %     \\
      %     &=\sum_{\nu=1}^q
      %       \diff^{\conj \ell}\diff_{\conj j_\nu}\vphi \: u_{\idxj 1[\dotsm
      %       (\conj\ell)_\nu].q}
      %       -\diff^{\conj\ell}\vphi \sum_{\nu=1}^q (-1)^{\nu} 
      %       \nabla_{\conj j_\nu} u_{\conj\ell \idxj 1[\dotsm
      %       \widehat {\conj j}_\nu].q}
      %       \begin{aligned}[t]
      %         &-\diff^{\conj\ell}\vphi \: \nabla_{\conj \ell} u_{\conj
      %           J_q} \\
      %         &+\diff^{\conj\ell}\vphi \: \nabla_{\conj \ell}
      %         u_{\conj J_q}
      %       \end{aligned}
      %     \\
      %     &=\sum_{\nu=1}^q
      %       \diff^{\conj \ell}\diff_{\conj j_\nu}\vphi \: u_{\idxj 1[\dotsm
      %       (\conj\ell)_\nu].q}
      %       -\diff^{\conj\ell}\vphi
      %       \:\paren{\dbar u}_{\conj\ell\conj J_q}
      %       +\diff^{\conj\ell}\vphi \: \nabla_{\conj \ell} u_{\conj
      %       J_q} \; ,
      %   \end{align*}
      %   as desired.
      % \end{proof}


    }

    Furthermore, by Proposition \ref{prop:harmonic-residue} and
    Theorem \ref{thm:residue-harmonic} (with
    $\lcS$ in place of $X$, $\lcS+1[b]$ in place of $D_p$ and
    $\psi_{(p)}$ in place of $\psi_{D_p}$), $w_b$ is a
    $K_{\lcS+1[b]} \otimes \res{F}_{\lcS+1[b]}$-valued $(0,q-1)$-form on
    $\lcS+1[b]$ (not only a $\res{\conj\holoform_X^{q-1}}_{\lcS+1[b]}$-valued
    section) which is harmonic with respect to
    $\res{\vphi_F}_{\lcS+1[b]}$.
    % Therefore, $\paren{w_b}_{b\in\Iset+1}$ represents a class in
    % $\spH/q-1/{\residlof+1*}$. 



  \item \label{item:pf:use_u-ortho-w}
    Apply the assumption $u =(u_p)_{p\in\Iset} \in
    \paren{\ker\tau_\sigma}^{\perp}$ via the use of $w
    :=\paren{w_b}_{b \in \Iset+1} \in \spH/q-1/{\residlof+1*}
    =\bigoplus_{b \in\Iset+1} \cohgp{q-1}[\lcS+1[b]]{K_{\lcS+1[b]}
      \otimes F}$ in view of the commutative diagram
    \begin{equation*}
      \xymatrix@R-0.3cm{
        {\dotsm} \ar[r]
        & {\spH/q-1/{\residlof+1*}} \ar[r]^-{\delta}
        \ar[d]^-{\tau_{\sigma+1}}
        & {\spH{\residlof*}} \ar[r]
        \ar@{=}[d]
        & {\spH{\faidlof+1/-1*}} \ar[r]
        \ar[d]
        & {\dotsm}
        \\
        {\dotsm} \ar[r]
        & {\spH/q-1/{\faidlof|\sigma_{\mlc}|*}} \ar[r]
        & {\spH{\residlof*}} \ar[r]^-{\tau_\sigma}
        & {\spH{\faidlof|\sigma_{\mlc}|/-1*}} \ar[r]
        & {\dotsm}
      }
    \end{equation*}
    and conclude that $u_p = 0$ on $\lcS$ for each $p\in\Iset$.

    From the commutative diagram, one sees that $\delta w \in
    \ker\tau_\sigma$.
    In view of the isomorphism $\residlof+1* \isom \faidlof+1*$ and by
    following the procedures in Step
    \ref{item:express-su-in-residue-norm}, one obtains
    a $\logKX \otimes \aidlof+1*$-valued \v Cech $(q-1)$-cochain
    $\set{\gamma_{\idx 1.q}}_{\idx 1,q \in I}$ with respect to $\cvr
    V$ such that, when
    \begin{equation*}
      \paren{\alpha'_{b; \:\idx 1.q}}_{b\in\Iset+1}
      :=\paren{\frac{\rs*\gamma_{b; \:\idx
            1.q}}{\sect_{(b)}}}_{\mathrlap{b\in\Iset+1}} \quad\;
      :=\Res^{\sigma+1}(\gamma_{\idx 1.q})
      \in \smashoperator[r]{\prod_{b\in\Iset+1}} K_{\lcS+1[b]} \otimes
      \res{F}_{\lcS+1[b]} \paren{\lcS+1[b] \cap V_{\idx 1.q}}
    \end{equation*}
    (notation chosen for the consistency with those in Proposition
    \ref{prop:res-formula-dbar-exact-dot-harmonic}) and
    \begin{equation*}
      \eqcls{\gamma_{\idx 1.q}}
      := \paren{\gamma_{\idx 1.q} \bmod \aidlof*} \in \logKX \otimes
      \faidlof+1* \quad\text{ on } V_{\idx 1.q} \; ,
    \end{equation*}
    the collection $\set{\alpha'_{b;\:\idx 1.q}}_{\idx 1,q\in I}$
    is a \v Cech \emph{$(q-1)$-cocycle} representing (the class of)
    $w_b$ in $\cohgp{q-1}[\lcS+1[b]]{K_{\lcS+1[b]} \otimes F}$ for
    each $b \in\Iset+1$ such that
    \begin{equation*}
      w_b = \dbar v'_{b;(2)} +(-1)^{q-1} \:\underbrace{
        \dbar \rho^{i_{q}} \wedge \dotsm \wedge
        \dbar\rho^{i_2} \cdot \rho^{i_1} }_{=: \:
        \paren{\dbar\rho}^{\idx q.1}} \alpha'_{b;\:\idx 1.q}
      =: \dbar v'_{b;(2)} +(-1)^{q-1} \frac{v_{b;(\infty)}'}{\sect_{(b)}}
    \end{equation*}
    (again, notation chosen for the consistency with those in Proposition
    \ref{prop:res-formula-dbar-exact-dot-harmonic})
    for some $K_{\lcS+1[b]} \otimes \res{F}_{\lcS+1[b]}$-valued
    $(0,q-2)$-form $v'_{b;(2)}$ on $\lcS+1[b]$ with $L^2$ coefficients
    with respect to $\norm\cdot_{\lcS+1[b]}$, and the collection
    $\set{\eqcls{\gamma_{\idx 1.q}}}_{\idx 1,q \in I}$ is a \v Cech
    \emph{$(q-1)$-cocycle} representing (the class of) $w$ in
    $\spH/q-1/{\faidlof+1*} \xrightarrow[\isom]{\Res^{\sigma+1}}
    \spH/q-1/{\residlof+1*}$.
    The image $\delta w$ in $\spH{\residlof*}$ is then represented by
    \begin{equation*}
      \set{\Res^\sigma\paren{\paren{\delta\gamma}_{\idx 0.q}}}_{\idx 0,q \in
      I} \; ,
    \end{equation*}
    in which $\delta$ is the \v Cech boundary operator.
    Note that applying $\Res^\sigma$ to $\paren{\delta\gamma}_{\idx
      0.q}$ is valid as $\set{\eqcls{\gamma_{\idx 1.q}}}_{\idx 1,q \in
      I}$ is a cocycle and thus coefficients of
    $\paren{\delta\gamma}_{\idx 0.q}$ lie in $\aidlof*$.
    Set
    \begin{equation*}
      \rs*\gamma_{p;\:\idx 1.q} := \PRes[\lcS](\frac{\gamma_{\idx
            1.q}}{\sect_D}) \cdot \sect_{(p)} 
    \end{equation*}
    such that
    \begin{equation*}
      \Res^\sigma\paren{\paren{\delta\gamma}_{\idx 0.q}}
      =\paren{\frac{\paren{\delta\rs*\gamma_p}_{\idx
            0.q}}{\sect_{(p)}}}_{p \in \Iset}
      \in \prod_{p \in \Iset} K_{\lcS} \otimes \res F_{\lcS}
      \paren{\lcS \cap V_{\idx 0.q}} \; .
    \end{equation*}
    Note that 
    \begin{equation*}
      (-1)^q \paren{\dbar\rho}^{\idx q.0}
      \frac{\paren{\delta \rs*\gamma_p}_{\idx 0.q}}{\sect_{(p)}}
      % =(-1)^q \paren{\dbar\rho}^{\idx q.1}
      % \frac{\rs*\gamma_{p;\:\idx 1.q}}{\sect_{(p)}}
      =-
      \frac{\dbar\paren{\paren{\dbar\rho}^{\idx q.1} \rs*\gamma_{p;\:\idx 1.q}}}{\sect_{(p)}}
      =: - \:\frac{\dbar v_{p; (\infty)}'}{\sect_{(p)}}
      \quad\text{ on } \lcS
    \end{equation*}
    is a $\dbar$-closed form representing the class of $\res{\delta
      w}_{\lcS}$ (the component of $\delta w$ on $\lcS$) in $\cohgp
    q[\lcS]{K_{\lcS} \otimes F}$ via Dolbeault isomorphism.
    
    Therefore, from the assumption $u \in
    \paren{\ker\tau_\sigma}^\perp$ and taking into account the \v
    Cech--Dolbeault isomorphism \eqref{eq:Cech-Dolbeault-isom} and the
    fact that each $u_p$ is harmonic, one has
    \begin{align*}
      0 =\iinner{(-1)^{q-1} \delta w}{u}_{\lcc'}
      &=(-1)^q \sum_{p\in \Iset} \iinner{\frac{\dbar
          v_{p;(\infty)}'}{\sect_{(p)}}}{u_p}_{\lcS}
      =(-1)^q \sum_{p\in \Iset} \iinner{\dbar v_{p;(\infty)}'}{u_p
        \sect_{(p)}}_{\lcS, \phi_{(p)}}
      \\
      &\overset{\mathclap{\text{Prop.~\ref{prop:res-formula-dbar-exact-dot-harmonic}}}}=
        \quad \;\; (-1)^{q-1} \sigma
        \smashoperator{\sum_{b \in \Iset+1}} \iinner{v_{b;(\infty)}'}{w_b
        \sect_{(b)}}_{\lcS+1[b], \phi_{(b)}}
      \\
      &=\sigma \smashoperator{\sum_{b \in \Iset+1}}
        \iinner{
          \paren{w_b -\dbar v_{b;(2)}'} \sect_{(b)}
        }{
          w_b \sect_{(b)}
        }_{\lcS+1[b], \phi_{(b)}}
      \\
      &\overset{\mathclap{w_b \text{ harmonic}}}= \quad\;\;\;
        \sigma
        \smashoperator{\sum_{b \in \Iset+1}}
        \iinner{w_b}{w_b}_{\lcS+1[b]}
        =\sigma \norm{w}_{\lcc+1'}^2 \; .
    \end{align*}
    As a result, $w_b = 0$ for each $b\in\Iset+1$, thus $su_p = 0$
    (hence $u_p = 0$) for each $p\in\Iset$ by Step
    \ref{item:express-su-in-residue-norm}.
    This completes the proof. \qedhere
  \end{enumerate}
\end{proof}

\begin{remark} \label{rem:singular-vphi_F}
  When $\vphi_F$ and $\vphi_M$ have only neat analytic singularities
  such that $\vphi_F^{-1}(-\infty) \cup \vphi_M^{-1}(-\infty) \cup D$
  has only snc and $\vphi_F^{-1}(-\infty) \cup \vphi_M^{-1}(-\infty)$
  contains no irreducible components of $D$ (hence no lc centers of
  $(X,D)$), the proof is still valid when the K\"ahler metric $\omega$
  on $X$ is replaced by a complete metric on $X \setminus
  \paren{\vphi_F^{-1}(-\infty) \cup \vphi_M^{-1}(-\infty)}$ as
  described in \cite{Chan&Choi_injectivity-I}*{\S 2.2 item (4)}.
  See \cite{Chan&Choi_injectivity-I}*{\S 3.3} for the technical
  modifications required.
\end{remark}

\begin{remark} \label{rem:no-hard-Lefschetz}
  Notice that the refinement of hard Lefschetz theorem (see
  \cite{Matsumura_injectivity-lc}*{Thm.~1.7} or
  \cite{Chan&Choi_injectivity-I}*{Thm.~2.5.1}) is not used in this
  proof.
  It is used in previous works to show that $\frac{u}{\sect_D}$ is
  smooth for every harmonic $u$ representing a class in $\cohgp
  q[X]{\logKX\otimes \mtidlof<X>{\phi_D}}$.
  This argument can be replaced by using directly the isomorphism
  induced by $\holo_X \xrightarrow[\isom]{\otimes \sect_D} D \otimes
  \mtidlof<X>{\phi_D}$, or $\holo_{\lcS} \xrightarrow[\isom]{\otimes
    \sect_{(p)}} \Diff_p(D) \otimes \mtidlof<\lcS>{\phi_{(p)}}$, which
  is more relevant to this article (see also Lemma
  \ref{lem:su-harmonicity}).
  However, when $\vphi_F$ and $\vphi_M$ have neat analytic singularities
  as described in \cite{Chan&Choi_injectivity-I}*{\S 2.2}, the theorem
  is still needed to get certain control of the regularity of $u$ on the
  polar sets of $\vphi_F$ and $\vphi_M$ (see
  \cite{Chan&Choi_injectivity-I}*{Prop.~3.3.1}).
\end{remark}

%%% Local Variables:
%%% mode: latex
%%% TeX-master: "Injectivity-Fujino"
%%% coding: utf-8
%%% End:


%%%%% Reference list %%%%%
\begin{bibdiv}
  \begin{biblist}
    \IfFileExists{references.ltb}{
      \bibselect{references}
    }{
      \input{Injectivity-Fujino.bbl}
    }
  \end{biblist}
\end{bibdiv}

% \newpage

% \section{Discussion}

% \textcolor{red}{
% I find it difficult to read because of the many subscripts and new symbols. 
% As a basic policy, I would like to propose the following: 
% \begin{itemize}
% \item Remove unnecessary subscripts.
% \item Do not use new symbols that are not necessary
% \end{itemize}
% For example, first of all, I would like to suggest: 
% \begin{itemize}
% \item SM wants to use the notation $(d\sect_{(1)})^{*} $. 
% \item SM wants to remove subscript $\omega$ in $(d\sect_{(1)})^{*}  $, 
% since we do not change a K\"ahler form $\omega$; hence there is no
% point in adding subscripts.
% \alert{MC: Done.}
% \item SM wants to remove the subscript $D$ in $\Iset$.
%   \alert{(Discussed in email.)}
% \item SM wants to remove $\varphi_{F}$ from $\aidlof$ since $\aidlof$
%   does not depend on smooth metrics.
%   \alert{(Discussed in email.) MC: I agree to mention about the
%     independence of $\aidlof$ on smooth $\vphi_F$. See Section \ref{subsec:residue}.}
% \end{itemize}
% Mario is using the notation $\mathcal{H}(\mathcal F)$ to abbreviate the cohomology group $H(X, \mathcal F)$ , 
% but I was using $\mathcal{H}$ to denote the space of harmonic forms. 
% SM would like to suggest that $H(\mathcal F)$ for the abbreviation.
% \alert{MC: See if the current choice is OK. Fortunately we are not
%   using hypercohomology or upper-half spaces.}
% \\
% \\
% This is something that should be discussed and I do not strongly advocate it, 
% but I thought it would be better to follow the notations in Demailly's lecture notes as possible, 
% because I guess that readers interested in this paper will be familiar with the notations in Demailly's lecture notes. 
% \\
% \\
% %For my part, I changed the role of the notations of $\alpha$  and $\beta$. 
% %Also, I changed the role of the symbol  $\rs{\lambda}$ with the tilde and the symbol $\lambda$ without the tilde. 
% %This is a personal preference, but I do not like inconsistent spacing between lines. 
% %After finishing the writing, I would like to keep the spaces as constant as possible. 
% }
\end{document}

%%% Local Variables:
%%% mode: latex
%%% TeX-master: t
%%% coding: utf-8
%%% End:

    }
  \end{biblist}
\end{bibdiv}

% \newpage

% \section{Discussion}

% \textcolor{red}{
% I find it difficult to read because of the many subscripts and new symbols. 
% As a basic policy, I would like to propose the following: 
% \begin{itemize}
% \item Remove unnecessary subscripts.
% \item Do not use new symbols that are not necessary
% \end{itemize}
% For example, first of all, I would like to suggest: 
% \begin{itemize}
% \item SM wants to use the notation $(d\sect_{(1)})^{*} $. 
% \item SM wants to remove subscript $\omega$ in $(d\sect_{(1)})^{*}  $, 
% since we do not change a K\"ahler form $\omega$; hence there is no
% point in adding subscripts.
% \alert{MC: Done.}
% \item SM wants to remove the subscript $D$ in $\Iset$.
%   \alert{(Discussed in email.)}
% \item SM wants to remove $\varphi_{F}$ from $\aidlof$ since $\aidlof$
%   does not depend on smooth metrics.
%   \alert{(Discussed in email.) MC: I agree to mention about the
%     independence of $\aidlof$ on smooth $\vphi_F$. See Section \ref{subsec:residue}.}
% \end{itemize}
% Mario is using the notation $\mathcal{H}(\mathcal F)$ to abbreviate the cohomology group $H(X, \mathcal F)$ , 
% but I was using $\mathcal{H}$ to denote the space of harmonic forms. 
% SM would like to suggest that $H(\mathcal F)$ for the abbreviation.
% \alert{MC: See if the current choice is OK. Fortunately we are not
%   using hypercohomology or upper-half spaces.}
% \\
% \\
% This is something that should be discussed and I do not strongly advocate it, 
% but I thought it would be better to follow the notations in Demailly's lecture notes as possible, 
% because I guess that readers interested in this paper will be familiar with the notations in Demailly's lecture notes. 
% \\
% \\
% %For my part, I changed the role of the notations of $\alpha$  and $\beta$. 
% %Also, I changed the role of the symbol  $\rs{\lambda}$ with the tilde and the symbol $\lambda$ without the tilde. 
% %This is a personal preference, but I do not like inconsistent spacing between lines. 
% %After finishing the writing, I would like to keep the spaces as constant as possible. 
% }
\end{document}

%%% Local Variables:
%%% mode: latex
%%% TeX-master: t
%%% coding: utf-8
%%% End:

    }
  \end{biblist}
\end{bibdiv}

% \newpage

% \section{Discussion}

% \textcolor{red}{
% I find it difficult to read because of the many subscripts and new symbols. 
% As a basic policy, I would like to propose the following: 
% \begin{itemize}
% \item Remove unnecessary subscripts.
% \item Do not use new symbols that are not necessary
% \end{itemize}
% For example, first of all, I would like to suggest: 
% \begin{itemize}
% \item SM wants to use the notation $(d\sect_{(1)})^{*} $. 
% \item SM wants to remove subscript $\omega$ in $(d\sect_{(1)})^{*}  $, 
% since we do not change a K\"ahler form $\omega$; hence there is no
% point in adding subscripts.
% \alert{MC: Done.}
% \item SM wants to remove the subscript $D$ in $\Iset$.
%   \alert{(Discussed in email.)}
% \item SM wants to remove $\varphi_{F}$ from $\aidlof$ since $\aidlof$
%   does not depend on smooth metrics.
%   \alert{(Discussed in email.) MC: I agree to mention about the
%     independence of $\aidlof$ on smooth $\vphi_F$. See Section \ref{subsec:residue}.}
% \end{itemize}
% Mario is using the notation $\mathcal{H}(\mathcal F)$ to abbreviate the cohomology group $H(X, \mathcal F)$ , 
% but I was using $\mathcal{H}$ to denote the space of harmonic forms. 
% SM would like to suggest that $H(\mathcal F)$ for the abbreviation.
% \alert{MC: See if the current choice is OK. Fortunately we are not
%   using hypercohomology or upper-half spaces.}
% \\
% \\
% This is something that should be discussed and I do not strongly advocate it, 
% but I thought it would be better to follow the notations in Demailly's lecture notes as possible, 
% because I guess that readers interested in this paper will be familiar with the notations in Demailly's lecture notes. 
% \\
% \\
% %For my part, I changed the role of the notations of $\alpha$  and $\beta$. 
% %Also, I changed the role of the symbol  $\rs{\lambda}$ with the tilde and the symbol $\lambda$ without the tilde. 
% %This is a personal preference, but I do not like inconsistent spacing between lines. 
% %After finishing the writing, I would like to keep the spaces as constant as possible. 
% }
\end{document}

%%% Local Variables:
%%% mode: latex
%%% TeX-master: t
%%% coding: utf-8
%%% End:

    }
  \end{biblist}
\end{bibdiv}

% \newpage

% \section{Discussion}

% \textcolor{red}{
% I find it difficult to read because of the many subscripts and new symbols. 
% As a basic policy, I would like to propose the following: 
% \begin{itemize}
% \item Remove unnecessary subscripts.
% \item Do not use new symbols that are not necessary
% \end{itemize}
% For example, first of all, I would like to suggest: 
% \begin{itemize}
% \item SM wants to use the notation $(d\sect_{(1)})^{*} $. 
% \item SM wants to remove subscript $\omega$ in $(d\sect_{(1)})^{*}  $, 
% since we do not change a K\"ahler form $\omega$; hence there is no
% point in adding subscripts.
% \alert{MC: Done.}
% \item SM wants to remove the subscript $D$ in $\Iset$.
%   \alert{(Discussed in email.)}
% \item SM wants to remove $\varphi_{F}$ from $\aidlof$ since $\aidlof$
%   does not depend on smooth metrics.
%   \alert{(Discussed in email.) MC: I agree to mention about the
%     independence of $\aidlof$ on smooth $\vphi_F$. See Section \ref{subsec:residue}.}
% \end{itemize}
% Mario is using the notation $\mathcal{H}(\mathcal F)$ to abbreviate the cohomology group $H(X, \mathcal F)$ , 
% but I was using $\mathcal{H}$ to denote the space of harmonic forms. 
% SM would like to suggest that $H(\mathcal F)$ for the abbreviation.
% \alert{MC: See if the current choice is OK. Fortunately we are not
%   using hypercohomology or upper-half spaces.}
% \\
% \\
% This is something that should be discussed and I do not strongly advocate it, 
% but I thought it would be better to follow the notations in Demailly's lecture notes as possible, 
% because I guess that readers interested in this paper will be familiar with the notations in Demailly's lecture notes. 
% \\
% \\
% %For my part, I changed the role of the notations of $\alpha$  and $\beta$. 
% %Also, I changed the role of the symbol  $\rs{\lambda}$ with the tilde and the symbol $\lambda$ without the tilde. 
% %This is a personal preference, but I do not like inconsistent spacing between lines. 
% %After finishing the writing, I would like to keep the spaces as constant as possible. 
% }
\end{document}

%%% Local Variables:
%%% mode: latex
%%% TeX-master: t
%%% coding: utf-8
%%% End:
