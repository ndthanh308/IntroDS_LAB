\documentclass{amsart}
\usepackage{graphicx, amsfonts, amssymb}
\usepackage{mathrsfs, amsmath, amsthm}
\usepackage{verbatim}
\usepackage{relsize}
\usepackage{epsfig}
\usepackage{color, colortbl}
\usepackage{comment}
\usepackage{mymacros}
\usepackage{xfrac,bigints}

\usepackage{hyperref}
\hypersetup{
	colorlinks=true,
	linkcolor=blue,
	filecolor=magenta,      
	urlcolor=cyan,
	pdftitle={Generalized Hilbert operators arising from Hausdorff matrices},
	pdfpagemode=FullScreen,
}

\newtheorem{theorem}{Theorem}[section]
\newtheorem{lemma}[theorem]{Lemma}

\theoremstyle{definition}
\newtheorem{definition}[theorem]{Definition}
\newtheorem{example}[theorem]{Example}
\newtheorem{proposition}[theorem]{Proposition}
\newtheorem{question}{Question}
\theoremstyle{remark}
\newtheorem{remark}[theorem]{Remark}

\numberwithin{equation}{section}

%%%%%%%%%%%%%%%%%%   Greek letters
\def\a{\alpha}       \def\b{\beta}           \def\g{\gamma}
\def\d{\delta}       \def\e{\epsilon}     \def\ep{\varepsilon}
\def\z{\zeta}         \def\h{\eta}         \def\th{\theta}
\def\i{\iota}          \def\la{\lambda}     \def\om{\omega}
\def\s{\sigma}       \def\f{\phi}             \def\p{\psi}
\def\r{\rho}          \def\ch{\chi}         \def\o{\omega}
\def\vr{\varrho}     \def\vf{\varphi}
\def\m{\mu}

\def\hm{X_\m}
\def\G{\Gamma}           \def\O{\Omega}
\def\F{\Phi}             \def\L{\Lambda}
\def\De{\Delta}
%%%%%%%%%%%%%%%%%%%%%%%%%%%%%%%%%%%%%%%%
%\newcommand{\H}{{\mathcal H}}
\DeclareMathOperator{\og}{O}
%\DeclareMathOperator{\ol}{o}

%%%%%%%%%%%%%%%%%%   Boldface and script letters

\def\T{{\mathbb T}}
  \def\N{{\mathbb N}}
\def\Z{{\mathbb   Z}}
\def\H{{\mathcal H}}
\newcommand{\C}{\mathbb C}
\newcommand{\D}{\mathbb D}
\newcommand{\R}{\mathbb R}
\newcommand{\U}{\mathbb U}


\def\cC{{\mathcal C}}    \def\cd{{\mathcal D}}
\def\cK{{\mathcal K}}    \def\cm{{\mathcal M}}
\def\cp{{\mathcal P}}    \def\rad{{\mathcal R}}

\newcommand{\n}[1]{\Vert#1\Vert}

\newcommand{\abs}[1]{\lvert#1\rvert}


%%%%%%%%%%%%%%%%%%%%%%%%%%%%%%%%%%%%%%%%%%%%%%%%%%%%%%%%%%%%%%%%%%%%%%%%%%%%%%%%%%%%%%%%%%%%%%%%%%%%%%%%%%%%%%%%%%%%%%%%
\title{Generalized Hilbert operators arising from Hausdorff matrices}




\author{C. Bellavita }
\email{cbellavita@math.auth.gr}
\address{Department of Mathematics, Aristotle University of Thessaloniki, 54124, Thessaloniki, Greece.}

\author{N. Chalmoukis}
\email{nikolaos.chalmoukis@unimib.it}
\address{Dipartimento di Matematica e Applicazioni, Universit\'a degli studi di Milano Bicocca, via Roberto Cozzi, 55 20125, Milano, Italy}

\author{V. Daskalogiannis}
\email{vdaskalo@math.auth.gr}
\address{Department of Mathematics, Aristotle University of Thessaloniki, 54124, Thessaloniki, Greece.}

\author{G. Stylogiannis}
\email{stylog@math.auth.gr}
\address{Department of Mathematics, Aristotle University of Thessaloniki, 54124, Thessaloniki, Greece.}

\thanks{This research project was supported by the
Hellenic Foundation for Research and Innovation (H.F.R.I.) under the '2nd
Call for H.F.R.I. Research Projects to support Faculty Members \& Researchers' (Project Number: 4662).}

\keywords{Generalized Hilbert matrices, Hausdorff matrices, Hardy spaces, Ces\'aro operator, Composition operators}

\subjclass{30H10, 47B91} % Hardy spaces, Operators on complex function spaces

%\date{\today}
%%%%%%%%%%%%%%%%%%%%%%%%%%%%%%%%%%%%%%%%%%%%%%%%%%%%%%%%%%%%%%%%%%%%%%%%%%%%%%%%%%%%%%%%%%%%%%%%%%%%%%%%%%%%%%%%%%%%%%%%

\begin{document}
\begin{abstract}
For a finite, positive, Borel measure $\mu$ on $(0,1)$ we consider an infinite matrix $\G_\mu$, related to the classical Hausdorff matrix defined by the same measure $\m$, in the same algebraic way that the  Hilbert  matrix is related to the Ces\'aro matrix. When $\mu$ is the Lebesgue measure, $\G_\mu$ reduces to the classical Hilbert matrix. We prove that the matrices $\G_\mu$ are not Hankel, unless $\mu$ is a constant multiple of the Lebesgue measure,
 we give necessary and sufficient conditions for their boundedness on the scale of Hardy spaces $H^p, \, 1 \leq p <  \infty$, and we study their compactness  and complete continuity properties. In the case $2\leq p<\infty$, we are able to compute the exact value of the norm of the operator.
\end{abstract}

\maketitle
\section{Introduction}



Let $\D$ be the unit disc in the complex plane $\C$, and let $H(\D)$ be the Fr\'echet space of all analytic functions on $\D$.
%The description below can be applied to several spaces  $X\hookrightarrow\H(\D)$, for which  the
%monomials $e_n(z)=z^n$, $n=0,1, 2, ...$ form  a (Schauder) basis.
%However, we will deal with the classical Hardy spaces.
Let   $0<p<\infty$ and   $f\in H(\mathbb{D})$. For  $0\leq r<1$ let
\[
M_p(r,f)\,:=\,\left(\dfrac{1}{2\pi}\int_{0}^{2\pi}\vert f(re^{i\theta})\vert^p\, d\theta\right)^{\frac{1}{p}}\,
\]
be the usual $p$-integral means of $f$ on $|z|=r$.  The
Hardy space $H^p =H^p(\mathbb{D})$ consists of the functions $f\in H(\D)$ for which
\[
\norm{f}_{H^p}\,:=\,\sup_{0\leq r<1}M_p(r,f) <+\infty,
\]
while for $p=\infty$, $\,H^\infty(\D)$ consists of bounded analytic functions on $\mathbb{D}$, i.e.
\[
\n{f}_{\infty}:=\sup_{z\in\D}|f(z)|<\infty.
\]
For $1\leq p \leq \infty$ the above quantities are norms and $H^p$ are  Banach spaces.
We will use some basic properties of these spaces
which we state briefly. If $0 < p < q< \infty$ then $ H^p\supset H^q\supset H^{\infty}$.
If $f\in H^p$, the  growth estimate \cite[p. 36]{Duren1970} holds,
\begin{equation}\label{growth}
\vert f(z)\vert
\,\leq\,
 \left(\dfrac{2}{1-\vert z\vert}\right)^{\frac{1}{p}}\norm{f} _{H^p},\quad z\in \mathbb{D}.
\end{equation}
Each $f\in H^p$ has radial limits $f^{*}(\z)=\lim_{r\to 1^{-}}f(r\z)$ for a.e. $\z\in \bT:=\partial\D$ with respect to
the  Lebesgue measure $|d\z|= d\theta, \,\, \z = e^{i\theta}$. %\,(=|d\z|)$.
The boundary function $f^{*}$ is $p$-integrable with $\norm{f^{*}}_{L^p(\bT)}=\norm{f}_{H^p}$.
In the sequel we use $f$, instead of
$f^{*}$, to denote the boundary function.
For $1<p<\infty$ the dual space $(H^p)^*$ can be identified isomorphically with $H^q,\;\frac{1}{p}+\frac{1}{q}=1$,
under the Cauchy pairing
\[
\langle f, g \rangle \,=\, \frac{1}{2\pi}\int_\bT f(\z)\,\overline{g(\z)}\,|d\z| ,\quad f\in H^p,\;g\in H^q.
\]



\subsection*{The Ces\'aro and Hilbert matrices.}
Consider the Ces\'aro matrix $C$  and the Hilbert matrix $H$,
\[
C=\left(
\begin{array}{ccccc}
            1 & 0  & 0  & .  \\ [4pt]
  \frac{1}{2} & \frac{1}{2}  & 0  & .  \\ [4pt]
  \frac{1}{3} & \frac{1}{3}  & \frac{1}{3}  & . \\
  .  & . & . & .  \\
\end{array}%
\right), \quad \quad H=\left(
\begin{array}{ccccc}
            1 & \frac{1}{2}  & \frac{1}{3}  & .  \\ [4pt]
  \frac{1}{2} & \frac{1}{3}  & \frac{1}{4}  & .  \\ [4pt]
  \frac{1}{3} & \frac{1}{4}  & \frac{1}{5}  & . \\
  .  & . & . & .  \\
\end{array}
\right).
\]
These matrices represent two of the most well-studied bounded linear operators on the Hardy space $H^2$, which we will denote by $\cC$ and $\cH$ respectively, and they 
act  on  functions $f(z)=\sum_{n=0}^{\infty}a_nz^n\in H^2$ as follows;
\begin{equation}\label{Ces-m}
\cC f(z)\,:=\,\sum_{n=0}^{\infty}\left(\frac{1}{n+1}\sum_{k=0}^na_k\right)z^n\,,
\end{equation}
and
\begin{equation}\label{Hil-m}
\cH f(z)
:=\sum_{n=0}^{\infty}\left(\sum_{k=0}^{\infty}\frac{a_k}{n+k+1}\right)z^n.
\end{equation}
Note that the series for $\cC f$ is well defined and converges on $\D$ for all $f\in H(\D)$,  thus $\cC f\in H(\D)$,  while the series for $\cH f$ is not always defined. For example, when  $f(z)=\frac{1}{1-z}$ the coefficients in the formal power series \eqref{Hil-m} are not defined. However,  Hardy's inequality \cite[p. 48]{Duren1970}, guarantees that
\[
\sum_{n\geq 0}\dfrac{|a_n|}{n+1}<\infty,
\]
for  functions in the Hardy space $H^1$. In particular, this implies  that $\cH f$ is a well defined  analytic function on $\D$ for each $f\in H^1$.


A calculation shows that $\cC f$ and $ \cH f$ can be written in integral form,
\begin{equation}\label{Ces-g}
\begin{split}
 \cC f(z)\,&=\,\frac{1}{z}\int_0^zf(\zeta)\frac{1}{1-\z}\,d\zeta=\int_0^1f(tz)\frac{1}{1-tz}\,dt
 \\
 &=\int_0^1f(tz)g'(tz)\,dt,
 \end{split}
\end{equation}
and
\begin{equation}\label{Hil-g}
\cH f(z)=\int_0^1f(t)\frac{1}{1-tz}\,dt=\int_0^1f(t)g'(tz)\,dt,
\end{equation}
where in both cases $g(z)=\log\dfrac{1}{1-z}$. The convergence of the integral for  $\cH f$ is  guaranteed
for all $f\in H^1$ by the Fej\'er-Riesz inequality \cite[Theorem 3.13]{Duren1970}.
It is well known that $\cC$ is a bounded linear operator on all  $H^p,\;0< p<\infty$, \cite{Siskakis1987}, \cite{Miao1992} and that
$\cH$ is bounded on $H^p$ for $1< p<\infty$, but not bounded on $H^1$ and $H^\infty$ \cite{Diamantopoulos2000}.

\begin{comment}
Changing the path of integration in (\ref{Ces-m}) and in (\ref{Hil-g}) (see \cite[Section 2]{Diamantopoulos2000}),
we obtain the following representations
\[
C(f)(z)= \int_0^1\frac{1}{1-(1-t)z}f\left(\frac{tz}{1-(1-t)z}\right)\,dt\,,
\]
and
\[
H(f)(z)=\int_0^1\frac{1}{1-(1-t)z}f\left(\frac{t}{1-(1-t)z}\right)\,dt\,,
\]
in terms of weighted composition operators.
\end{comment}

Observe that  $H$ is obtained from $C$ by the following formal manipulation;
\textit{Eliminate  the zeros in each column of  $C$ and
shift up the columns to their  first nonzero entry.}
In rigorous terms this is equivalent to the following algebraic relation. Let $e_n(z)=z^n,  n=0,1,2, ...$ be the monomials, which 
form an orthonormal basis of $H^2$. Then, we have
 $$
\cC e_n(z)=e_n(z)\cH e_n(z) = \frac{1}{z}\left(\log\frac{1}{1-z}- \sum_{k=1}^n\frac{z^k}{k}\right).
 $$
Equivalently,
\begin{equation}\label{H from C}
\H e_n(z)=\frac{1}{z^n} \cC e_n(z).
\end{equation}
Note that this can be expressed in terms of the shift operator $Sf(z):=zf(z)$ as
$$
\cC e_n=S^n\H e_n, \,\,  n=0,1,2,...
$$

\subsection*{Generalized Hilbert matrices arising from Hausdorff matrices}\label{Haus}
For a given sequence $\mu=\{\mu_n\}_{n=0}^\infty$ the Hausdorff matrix induced by $\mu$ is the
 lower triangular matrix
 \begin{equation}
\cK_{{\m}}= \left(
\begin{matrix}
c_{00}& 0 & 0&  \cdots     \\
c_{10}& c_{11} &0& \cdots     \\
c_{20} & c_{21} & c_{22} & \cdots    \\
\vdots &\vdots  &\vdots     &\ddots
\end{matrix}
\right)
\notag
\end{equation}
with entries $c_{nk}$ given by
\begin{equation}
c_{n k}=\binom{n}{k}\Delta^{n-k}\mu_k, \quad     0\leq k\leq n,
\notag
\end{equation}
where $\Delta\mu_n=\mu_n-\mu_{n+1}$ is  the forward
difference operator, with iterates
$$
\Delta^{k}\mu_n=\Delta(\Delta^{k-1}\mu_n),
\quad  k=1,2,\dots, \quad \Delta^0\mu_n=\mu_n.
$$
 In the special case where $\{\mu_n\}$ is the moment sequence
\begin{equation}
\mu_n=\int_0^1t^n\,d\mu(t), \qquad n=0,1,\dots
\notag
\end{equation}
of a finite positive Borel measure $d\mu$ on $(0,1)$, the entries of $\cK_{\m}$ can be calculated explicitely; 
\begin{equation}
c_{n k}=\binom{n}{k}\int_0^1t^k(1-t)^{n-k}\,d\mu(t),
\qquad 0\leq k\leq n\,.
\notag
\end{equation}
Hausdorff matrices have been studied in connection to the classical summability
theory on sequence  spaces \cite{Hardy1943} and recently on spaces of analytic functions
 \cite{Galanopoulos2001}, \cite{Galanopoulos2006}.

The Ces\'aro matrix is a typical example in the family of  Hausdorff  matrices, and is obtained by the
moment sequence $\mu_n=\frac{1}{n+1}$ of the Lebesgue measure, $d\mu(t)=dt$. Applying the
operation of erasing zeros and shifting up the columns, i.e. the analogous of (\ref{H from C}) to the matrix $\cK_{\mu}$, we obtain a new matrix
$$
\G_{\m}=\left(%
\begin{array}{ccccc}
            \g_{00} & \g_{01}  & \g_{02}  & .  \\
\g_{10} & \g_{11}  & \g_{12}  & .  \\
  \g_{20} & \g_{21}  & \g_{22}  & . \\
  .  & . & . & .  \\
\end{array}%
\right)
$$
with entries
$$
\g_{n k}=c_{n+k, k}=\binom{n+k}{k}\int_0^1t^k(1-t)^n\,d\m(t)\,.
$$
By its  construction $\G_{\mu}$ reduces to the Hilbert matrix $H$ for the
particular choice of the measure  $d\m(t)=dt$, i.e.  this choice of $\mu$  gives
$\g_{n,k}=\frac{1}{n+k+1}$. In a very recent article \cite{Athanasiou2023}, the author studied the boundedness properties  of the matrix $\G_{\mu}$, when acting on the classical sequence spaces $\ell^p, 1\leq p \leq \infty$. However, a different approach is required when one studies the operator on spaces of holomorphic functions.

In this article we will study the action of the operators $\G_{\m}$ on the Hardy spaces $H^p,\, 1\leq p < \infty$.
If  $f(z)=\sum_{k=0}^{\infty} a_kz^k$ is analytic on the unit disc then we define the formal power series
\[
\G_{\m} f(z)=\sum_{n=0}^{\infty}A_nz^n,\quad\text{where} \,\,\, A_n=\sum_{k=0}^{\infty}\g_{n,k}a_k, \quad n=0,1, ...,
\]
whenever the series defining the coefficients $A_n$ are convergent.  If $f$ is a polynomial then the sum giving
$A_n$  is finite for each $n$ and one can verify that the formal power series for $\Gamma_\mu f$ converges in the unit disc.  Therefore, if one can establish an inequality of the type 
\[ \norm{\Gamma_\mu f}_{H^p} \leq C_p \norm{f}_{H^p}\,, \]
for some $C_p>0$ and for all polynomials $f$, then, by the density of the polynomials in $H^p$, the operator $\Gamma_\mu$ extends in a unique way to a bounded linear operator on the whole $H^p$.
% But in general the finiteness of the
% coefficients $A_n$ and the convergence of the series on $\D$ will depend on $d\m$ and on $f$. 

% {\color{red} Given a space $X\subset \mathcal{H}(\D)$ we can ask for which measures $d\m$ the power series
% $\G_{\m}(f)(z)$ defines an analytic function on $\D$  for each $f\in X$. In this article we deal with $X=H^p$. We will determine the measures  $d\mu$ for which $\G_{\m}$ is bounded or compact on Hardy spaces. In section 2 we give an integral representation of $\G_{\m}$ which will be used in the  ....}



\section{An integral representation and main results}

The integral giving $\H$ in (\ref{Hil-g})  can be viewed as an \lq\lq improper\rq\rq\, line integral, with the path of
integration being the radius $[0, 1]$. When $\H$ is applied to a function belonging to $H^1$, it turns
out that we can change the path of integration in \eqref{Hil-g} to be the arc
$$
\gamma(s)= \gamma_z(s)= \frac{s}{1-(1-s)z}, \quad 0\leq s\leq 1\,,
$$
which joins $0$ to $1$ and lies inside $\D$ for $0\leq s<1$ for every $z\in \D$,
see \cite[Section 2]{Diamantopoulos2000}. With this change of variable we obtain
\begin{align*}
\H f(z)=\int_0^1\frac{1}{1-(1-t)z}f\left(\frac{t}{1-(1-t)z}\right)\,dt
=:\int_0^1T_t f(z)\,dt\,.
\end{align*}
This gives a representation of $\cH$ in terms of the weighted composition operators

\begin{equation}\label{T-t}
T_t(f)(z)=w_t(z)f(\phi_t(z))\,dt\,,
\end{equation}
with
\begin{equation}\label{w-phi}
\phi_t(z):=\frac{t}{1-(1-t)z}, \quad             w_t(z):= \frac{1}{1-(1-t)z}=\frac{\phi_t(z)}{t}\,.
\end{equation}
It is easy to check that each $T_t$, $0<t<1$,  is  bounded on $H^p$ for $0< p<\infty$.


We will derive an analogous representation for $\G_{\m}$. We assume that $\mu$ is a finite positive Borel measure on $(0,1)$. Assume also that $f$ is such that $\G_\mu f$ is well defined as an analytic function in $\mathbb{D}$ (for example when $f$ is a polynomial). In that case, all sums and integrals are interchangeable.
We then have:
\[
\begin{split}
\Gamma_\mu f(z) &= \sum_{n= 0}^{\infty}\left(\sum_{k= 0}^{\infty}a_k\, \binom{n+k}{k}\int_0^1
(1-t)^n t^k\,d\mu(t)\right)\,z^n
\\
&=\int_0^1\,\sum_{k=0}^{\infty} a_k t^k \, \sum_{n=0}^{\infty} \binom{n+k}{k} (1-t)^n z^n\,d\mu(t)\,.
\end{split}
\]
Next, using the identity
\begin{equation}\label{identity}
\sum_{n= 0}^{\infty}\binom{n+m}{n}w^n=\dfrac{1}{(1-w)^{m+1}}\,,
\end{equation}
we get
\[
\begin{split}
\sum_{n=0}^{\infty} \binom{n+k}{n} \left((1-t) z\right)^n &=
\frac{1}{(1-(1-t)z)^{k+1}}\\
&=\left(\frac{1}{(1-(1-t)z)}\right)^k\frac{1}{(1-(1-t)z)}\,.
\end{split}
\]
Therefore,
\[
\begin{split}
\G_\mu f(z) &=\int_0^1\,\sum_{k= 0}^{\infty} a_k t^k\left(\frac{1}{(1-(1-t)z)}\right)^k\frac{1}{(1-(1-t)z)}
\,d\mu(t) \\
&=\int_0^1\sum_{k=0}^{\infty} a_k  \left(\frac{t}{(1-(1-t)z)}\right)^k\frac{1}{(1-(1-t)z)}\,d\mu(t)
\\
&=\int_0^1 f\left(\frac{t}{(1-(1-t)z)}\right)\frac{1}{(1-(1-t)z)}\,d\mu(t)\\
&=\int_0^1w_t(z) f(\phi_t(z))\,d\mu(t)\,,
\end{split}
\]
where $w_t$ and $\phi_t$ are as in (\ref{w-phi}). Hence
\[
\G_\mu f(z)=\int_0^1 T_t f(z) \,d\mu(t)\,,
\]
where $T_t$ are the weighted composition operators (\ref{T-t}).
 This formula can be used to study $\G_{\mu}$ on spaces of analytic functions, exploiting information available for
 (weighted) composition operators. An important property of the operators $T_t$ is the following.

\begin{prop}\label{the adjoint}
Let $T_{t}^{*}$ denote the adjoint of the operator $T_{t}$ on $H^p$, for $1<p<\infty$. Then,
$$
T_{t}^{*}=T_{1-t}\,,
$$
for every $0<t<1$. In particular $T_{1/2}$ is self adjoint in $H^2$.
\end{prop}
\begin{proof} Let $1<p<\infty$ and consider $T_t: H^p\to H^p$. The adjoint $T_t^{*}$ acts on $H^q$ with  $\frac{1}{p}+\frac{1}{q}=1$, and we have that
\[
\int_\T {T_tf(\zeta)}\,\overline{g(\zeta)}\,| d\z |\,=\,\int_{\T} f(\zeta)\,\, \overline{T_t^*g(\zeta)}\,| d\z |\,. 
\]


Assume  that $f(z)=\sum_{k=0}^{\infty}a_k z^k$ and  $g(z)=\sum_{m=0}^{\infty}b_mz^m$ are polynomials. Then,

\[\begin{split}
\int_\T {T_tf(\zeta)}\,\overline{g(\zeta)}\,| d\z |  =&
\int_\T \frac{1}{1-(1-t)\zeta}\, f\left(\frac{t}{1-(1-t)\zeta}\right)\,\overline{g(\zeta)}\,| d\z |
\\
=&\int_{\T} \sum_{k} a_k t^k \, \dfrac{1}{(1-(1-t)\zeta)^{k+1}} \,\, \overline{g(\zeta)}\, | d\z |
\\
=&\int_{\T} \sum_{k} a_k t^k\, \sum_{n } \binom{n+k}{n} \left((1-t) \zeta\right)^n \overline{g(\zeta)} \,| d\z |
\\
=&\int_{\T} \sum_{k} a_k t^k\, \sum_{n} \binom{n+k}{n} \left((1-t) \zeta \right)^n\,\overline{\sum_{m}b_m\z^m}\, | d\z |\,.
\end{split}
\]
Notice here that
\[
\int_\T \z^n \overline{\z}^m \,| d\z |=
\begin{cases}
2\pi, \quad n=m
\\
0,\quad n\neq m
\end{cases}\,,
\]
and also that
\[
\begin{split}
T_{1-t}(g)(z) &= \dfrac{1}{1-tz}\,g\left(\dfrac{1-t}{1-tz} \right)
\\
&=\sum_{m} b_m (1-t)^m\, \dfrac{1}{(1-tz)^{m+1}}
\\
&=\sum_{m} b_m (1-t)^m\, \sum_{n} \binom{n+m}{n}\,t^nz^n\,,
\end{split}
\]
hence
\[
\begin{split}
\int_\T {T_tf(\zeta)}\,\overline{g(\zeta)}\,&| d\z |
=2\pi \sum_{k} a_k \left( \sum_{m} t^k\binom{m+k}{m}\overline{b_m}(1-t)^m\right)
\\
=&\int_{\T} \sum_k a_k \zeta^k \left( \sum_{m}  \binom{m+k}{m}\Bar{b}_m\,(1-t)^m\right) t^k \Bar{\z}^k \,| d\z |
\\
=&\int_{\T} f(\zeta)\, \overline{\sum_n \left(\sum_{m} {b_m}(1-t)^m\,\binom{m+n}{m} \right) t^n\,\zeta^n} \, | d\z |
\\
=&\int_{\T} f(\zeta)\, \overline{ \sum_{m} {b_m}(1-t)^m \,\sum_n  \binom{n+m}{n}\,t^n\,\zeta^n }\, | d\z |
\\
=& \int_{\T} f(\zeta)\,\, \overline{T_{1-t}g(\zeta)}\,| d\z |\, .
\end{split}
\]
The above holds for all polynomials $f\in H^p$ and $g\in H^q$.  Since the polynomials are dense in $H^p$ and $H^q$, we have verified that $T^*_t=T_{1-t}$.
\end{proof}









In \cite{Diamantopoulos2000} the authors estimated the norm of the weighted composition operator $T_t$, and we know that
\[
\n{T_tf}_{H^p}\leq\, \dfrac{1}{t^{1-\frac{1}{p}}\,(1-t)^{\frac{1}{p}}} \;\n{f}_{H^p},\;\;p\geq 2\,.
\]
In Section \ref{proofs} we estimate the norm of $T_t$, considering all $p\geq 1$, providing an alternative proof to the one in \cite{Diamantopoulos2000}, in which we do not have to translate the problem into a problem of the upper half-plane. Our first theorem shows that in the class of the operators $\Gamma_\mu$, being a Hankel operator is the exception rather than the rule. 
\begin{thm}\label{Hankel}
The operator $\Gamma_{\mu}$ is a Hankel operator if and only if  $\mu$ is a constant multiple of the Lebesgue measure.
\end{thm}
 We also note that there is no measure $\mu\neq 0$ for which the operator $\Gamma_{\mu}$ is Toeplitz. To see this, assume that $\Gamma_\mu$ is a Toeplitz operator. Then, for the elements of the main diagonal, we must have that $\gamma_{00}=\gamma_{11}$ or equivalently $1=2\int_{0}^1 t(1-t)d\mu(t)$.
However, $2t(1-t)\leq 1/2$ for each $t\in [0,1]$, which leads to the contradiction $1 \leq \frac{1}{2}$.


 Next we study necessary and sufficient conditions for continuity of the operators $\Gamma_\mu$.
In order to formulate our main result it will be convenient to introduce the following weight function;  
\[ \psi_p(t):= 
\begin{cases}
    \dfrac{t^{\frac{1}{p}-1}}{(1-t)^{\frac{1}{p}}}\,,\,\,\, &\text{if} \,\, p>1
    \\[0.2in]
   \log\left(\dfrac{e}{t}\right)\dfrac{1}{(1-t)}\,, \,\,\,&\text{if}\,\, p=1\,.
\end{cases}
\]
    \begin{thm}\label{main theom p>1}
The operator $\Gamma_\mu$ is bounded on  $H^p, \;1\leq p<\infty$,  if and only if
\begin{equation}\label{condition p>1}
    \int_{0}^{1}\psi_p(t)d\mu(t)<\infty\,.
\end{equation}
Furthermore, there exist positive constants $A_p,\,B_p$ depending only on $p$ such that 
\[
A_p \int_0^1 \psi_p(t)\,d\mu(t)\, \leq \,\norm{\G_\mu}_{H^p \to H^p}\,\leq\, B_{p} \int_{0}^{1} \psi_p(t)\,d\mu(t)\,. %(?)\frac{t^{\frac{1}{p}-1}}{(1-t)^{\frac{1}{p}}}\,.
\]
In particular, when $2\leq p < \infty$, this is the exact value of the norm, i.e.
\[
\|\Gamma_\mu\|_{H^p\to H^p}=\int_{0}^{1}\dfrac{t^{\frac{1}{p}-1}}{(1-t)^{\frac{1}{p}}}d\mu(t)\, .
\]
\end{thm}


Finally, we consider the question of compactness and complete continuity. It turns out that $\G_\mu$ have a behaviour similar to the classical Hilbert matrix, in terms of compactness. Nonetheless, in the endpoint case $p=1$, we prove that the operators $\G_\mu$ are completely continuous. An operator $T:X\to X$ is completely continuous if it maps every relatively weakly compact subset of $X$ into a relatively compact subset of $X$.  In general, every compact operator is completely continuous, however the converse is false when $X$ is non-reflexive.

\begin{thm}\label{Compact}
The operator $\Gamma_{\mu}$ is not compact on $H^p$, $1\leq p<\infty$, unless $\mu $ is the zero measure. However,  $\Gamma_\mu$ is completely continuous on $H^1$ whenever it is bounded.
\end{thm}

% \begin{thm}\label{main theorem p=1}
% The operator $\Gamma_\mu$ is bounded on $H^1$,  if and only if
% \begin{equation}\label{condition p=1}
%     \int_{0}^{1}\log\left(\frac{e}{t}\right)\cdot\frac{1}{(1-t)}\,d\mu(t)<\infty\,.
% \end{equation}
% In addition,
% \[
% \norm{\G_\mu}_{H^1 \to H^1}\leq C \int_{0}^{1}\log\left(\frac{e}{t}\right)\cdot\frac{1}{(1-t)}\,d\mu(t)
% \]
% for some absolute constant $C>0$.
% \end{thm}

%%%%%%%%%%%%%%%%%%%%%%%%%%%%%%%%%%%%%%%%%%%%%%%%%%%%%%%%%%%%%%%%%%%%%%%%%%%%%%%%%%%%%%%%%%%%%%%%%%%%%%%%%




\section{Proof of main results}\label{proofs}
We proceed with the proofs of our main results. Without loss of generality, we always assume that $\mu(0,1)=1$.

\begin{proof}[Proof of Theorem \ref{Hankel}]
If $\mu$ is the  Lebesque measure then it is well known that $\Gamma_{\mu}$ is Hankel. For the converse implication, let us suppose that $\Gamma_{\mu}$ is Hankel.
We will use induction to prove that $\mu$ is the Lebesgue measure.
We know that the entries of the matrix $\Gamma_{\mu}$ are
$$
\gamma_{n,k}=\binom{n+k}{n}\int_{0}^{1}t^{n}(1-t)^{k}\,d\mu(t)\,.
$$
For $n=1$ we have
$\gamma_{1,0}=\gamma_{0,1}$, that is
$$
\int_{0}^{1}t\,d\mu=\int_{0}^{1}(1-t)\,d\mu(t)\,.
$$
This implies, since $\mu(0,1)=1$, that
$$
\int_{0}^{1}t\,d\mu(t)=\frac{1}{2}\,.
$$
Let us suppose that
\begin{equation}\label{momentdm}
    \int_{0}^{1}t^{m}\,d\mu(t)=\frac{1}{m+1}
\end{equation}
when $m=n$; we want to verify \eqref{momentdm} for $m=n+1$.
Since $\Gamma_{\mu}$ is Hankel, we have that $\gamma_{n+1,0}=\gamma_{n,1}$, that is,
$$
\binom{n+1}{n+1}\int_{0}^{1}t^{n+1}\,d\mu(t)=\binom{n+1}{n}\int_{0}^{1}(1-t)t^{n}\,d\mu(t)\,,
$$
which implies that
$$
    \int_{0}^{1}t^{n+1}\,d\mu(t)= (n+1)\left(\frac{1}{n+1}-\int_{0}^{1}t^{n+1}\,d\mu(t)\right)
$$
and, consequently,
$$
    \int_{0}^{1}t^{n+1}\,d\mu(t)=\frac{1}{n+2}\,.
$$
Therefore, by induction, \eqref{momentdm} holds for every $m \in \N$.
Since the Hausdorff moment problem admits a unique solution \cite[Theorem 2.6.4]{Akhiezer2021}, $\mu$ has to be the Lebesgue measure.
\end{proof}
 As a consequence of Theorem \ref{Hankel}, we  note that the operators $\G_\mu$ are distinct from the generalized Hilbert operators
\[
\cH_\mu f(z)\,=\,\int_0^1f(t)\frac{1}{1-tz}\,d\mu(t)\,,
\]
studied in the literature (see for example \cite{Chatzifountas2014, Galanopoulos2010, Girela2018}), for any measure $\mu$ other than constant multiples of the Lebesgue
measure. We refer the interested readers to the recent note \cite{Blasco2022} on generalized Hilbert operators.

%%%%%%%%%%%%%%%%%%%%%%%%%%%%%%%%%%%%%%%%%%%%%%%%%%%%%%%%%%%%%%%%%%%%%%%%%%%%%%%%%%%%%%%%%%%%%%%%%%%%%%%%%
%%%%%%%%%%%%%%%%%%%%%%%%%%%%%%%%%%%%%%%%%%%%%%%%%%%%%%%%%%%%%%%%%%%%%%%%%%%%%%%%%%%%%%%%%%%%%%%%%%%%%%%%%%
Using Proposition \ref{the adjoint},  we can explicitly compute the adjoint of $\Gamma_\mu$.
\begin{lem}\label{the adjoint Gm}
Let $1<p<\infty$. The adjoint operator of $\Gamma_{\mu}$ on $H^p$, denoted by $\Gamma_{\mu}^{*}$, is given by
$$
\Gamma_{\mu}^{*}f(z)=\int_{0}^{1}T_{t}^{*}f(z)\,d\mu(t)=\int_{0}^{1}T_{1-t}f(z)\,d\mu(t)\, .
$$
\end{lem}
\begin{proof}
Let $\frac{1}{q}+\frac{1}{p}=1$. Again, we can assume that $f(z)$ and $g(z)$ are polynomials. By using Proposition \ref{the adjoint} and Fubini's Theorem, we obtain that
\begin{align*}
    \int_{\T} \Gamma_\mu f (\zeta) \overline{g(\zeta)}\,|d\z |\,&= \int_{\T} \left( \int_{0}^1T_t f (\zeta)d\mu(t)\right) \,  \overline{g(\zeta)}\,|d\z|
    \\[0.1in]
    &=\,\int_{0}^1 \int_{\T} T_t f (\zeta)\overline{g(\zeta)}\,| d\z | \, d\mu(t)
    \\[0.1in]
    &=\, \int_{0}^1 \int_{\T}f(\zeta)\overline{T_{1-t} g (\zeta)}\,| d\z | \, d\mu(t)
    \\[0.1in]
    &=\, \int_{\T}f(\zeta) \overline{ \left( \int_{0}^1 {T_{1-t} g (\zeta)} d\mu(t) \right) }\, | d\z |\,,
\end{align*}
which, since the polynomials are dense both in $H^p$ and $H^q$, implies the statement.
\end{proof}

In order to prove our next theorem, we provide an estimate of the norm of the operator $T_t$.
We first need this preliminary calculation.

\begin{lem}\label{change of var lem}
Let $1\leq p <\infty$. Then:
$$
\norm{T_{t}f}_{H^p}\,=\,\frac{t^{\frac{1}{p}-1}}{(1-t)^{\frac{1}{p}}}\left(\int_{\partial D(\frac{1}{2-t},\frac{1-t}{2-t})}|f(w)|^{p}|w|^{p-2}\, \dfrac{| dw |}{2\pi}\right)^{\frac{1}{p}}
$$
for every $f \in H^p$ and $0< t< 1$.
\end{lem}
\begin{proof}
 Let $f\in H^p$. We note that
 \begin{equation}\label{phi prime}
     \varphi_{t}(\mathbb{D})=D\left( \frac{1}{2-t},\frac{1-t}{2-t}\right) \quad \text{ and } \quad \varphi_{t}'(z)=\frac{1-t}{t}\cdot(\varphi_{t}(z))^{2}
 \end{equation}
for every $0< t< 1$ and $z\in \mathbb{D}$, where $D(z_0, r)$ is the open disc centered at $z_0$ of radius $r$. Applying a change of variables, we get
\begin{align*}
||T_{t}f||_{H^p}&=\frac{1}{t}\left(\int_{\mathbb{T}}|f(\varphi_{t}(\zeta))|^{p}|\varphi_{t}(\zeta)|^{p}\, \dfrac{| d\z |}{2\pi} \right)^{\frac{1}{p}}\\
&=\frac{t^{\frac{1}{p}-1}}{(1-t)^{\frac{1}{p}}}\left(\int_{\mathbb{T}}|f(\varphi_{t}(\zeta))|^{p}|\varphi_{t}(\zeta)|^{p-2}
|\varphi_{t}'(\zeta)|\, \dfrac{| d\z |}{2\pi} \right)^{\frac{1}{p}}\\
&=\frac{t^{\frac{1}{p}-1}}{(1-t)^{\frac{1}{p}}}\left(\int_{\partial D(\frac{1}{2-t},\frac{1-t}{2-t})}|f(w)|^{p}|w|^{p-2}\, \dfrac{| dw |}{2\pi} \right)^{\frac{1}{p}}\,.
\end{align*}
\end{proof}
We are now ready for an estimate of the norm of $T_t$ when $1\leq p<\infty.$
\begin{lem}\label{normT_t p}
Let $\,0<t<1$ and $1< p<\infty$. For the norm of $T_t$ on $H^p$ we have
$$
||T_{t}||_{H^p\to H^p}\,\leq\, B_p\,\frac{t^{\frac{1}{p}-1}}{(1-t)^{\frac{1}{p}}}
$$
for some constant $B_p>0$, depending only on $p$, which can be chosen equal to $1$ when $2\leq p<\infty$. Moreover, when $p=1$,
$$
||T_{t}||_{H^1 \to H^1}\leq C \log\left( \frac{e}{t}\right) \frac{1}{1-t}
$$
for an absolute constant $C>0$.
\end{lem}
\begin{proof}
Let us first consider the case $p\geq 2$. Then, by applying Lemma \ref{change of var lem} and \cite[Theorem 2.1]{Gabriel1928}, we get
\begin{align*}
\norm{T_t f}_{H^p}&=\frac{t^{\frac{1}{p}-1}}{(1-t)^{\frac{1}{p}}}\left( \int_{\partial D(\frac{1}{2-t},\,\frac{1-t}{2-t})}|f(w)|^{p} |w|^{p-2}\,\dfrac{| dw |}{2\pi} \right)^{\frac{1}{p}}\\
&\leq \frac{t^{\frac{1}{p}-1}}{(1-t)^{\frac{1}{p}}}\left( \int_{\bT}|f(w)|^{p}\, \dfrac{| dw |}{2\pi} \right)^{\frac{1}{p}}
\\
& = \frac{t^{\frac{1}{p}-1}}{(1-t)^{\frac{1}{p}}}\, \norm{f}_{H^p}\,.
\end{align*}

Let us now take in consideration $1< p<2$. We note that every function $f \in H^p$ can be written as $f(z)=f(0)+zg(z)$ for some $g\in H^p$.
Therefore
$$||T_t f||_{H^p}\leq |f(0)|\cdot ||T_{t}(1)||_{H^p}+||T_{t}(Sg)||_{H^p}$$
where $Sg(z)=zg(z)$ is the shift operator acting on $H^p$. For the first term we know that
\[
|f(0)|\leq 2^{\frac{1}{p}}||f||_{H^p}
\]
and
\[
\begin{split}
||T_{t}(1)||_{H^p}&=\left(\int_{-\pi}^{\pi}\frac{1}{|1-(1-t)e^{i\theta}|^{p}}\,d\theta\right)^{\frac{1}{p}}\\
&\leq\,C \left(\int_{0}^{\pi}\frac{1}{(t+\theta)^{p}}\,d\theta\right)^{\frac{1}{p}}
<\, C\,\left(\frac{1}{p-1}\right)^{\frac{1}{p}}t^{\frac{1}{p}-1}\,,
\end{split}
\]
where we have used a classical estimate (see for example \cite[Proposition 1.23]{Pavlovic2019}) for the first inequality.
For the second term, considering \eqref{phi prime} and using  \cite[Theorem 2.1]{Gabriel1928}, we obtain that
\begin{align*}
||T_{t}(Sg)||_{H^p}& = \frac{1}{t}\left(\int_{\mathbb{T}}|g(\varphi_{t}(\zeta))|^{p}|\varphi_{t}(\zeta)|^{2p}\,\dfrac{| d\z |}{2\pi} \right)^{\frac{1}{p}}\\
&=\frac{t^{\frac{1}{p}-1}}{(1-t)^{\frac{1}{p}}}\left(\int_{\mathbb{T}}|g(\varphi_{t}(\zeta))|^{p}|\varphi_{t}(\zeta)|^{2p-2}|\varphi_{t}'(\zeta)|\,\dfrac{| d\z |}{2\pi} \right)^{\frac{1}{p}}\\
&=\frac{t^{\frac{1}{p}-1}}{(1-t)^{\frac{1}{p}}}\left(\int_{\varphi_{t}\mathbb{(T)}}|g(w)|^{p}|w|^{2p-2}\,\dfrac{| dw |}{2\pi}\right)^{\frac{1}{p}}\\
&\leq \frac{t^{\frac{1}{p}-1}}{(1-t)^{\frac{1}{p}}}\;\norm{g}_{H^p}\,.
\end{align*}
This means that
\begin{align*}
||T_{t}(Sg)||_{H^p}& \leq \frac{t^{\frac{1}{p}-1}}{(1-t)^{\frac{1}{p}}}\,||Sg||_{H^p}
\\
&= \frac{t^{\frac{1}{p}-1}}{(1-t)^{\frac{1}{p}}}\,||f-f(0)||_{H^p}\\
&\leq \frac{t^{\frac{1}{p}-1}}{(1-t)^{\frac{1}{p}}}(2^{\frac{1}{p}}+1)\,||f||_{H^p}\,.
\end{align*}

The above computations imply that
$$
||T_{t}||_{H^p}\leq C \left(\frac{1}{p-1}\right)^{\frac{1}{p}}\frac{t^{\frac{1}{p}-1}}{(1-t)^{\frac{1}{p}}}\,.
$$
The proof of the case $p=1$ requires similar arguments, with very few natural modifications, so we omit the details.
\end{proof}

We are now ready to discuss our central result, concerning the boundedness of the operators $\G_\mu$.
\begin{proof}[Proof of Theorem \ref{main theom p>1} ]
First of all let us verify that \eqref{condition p>1} is necessary for the boundedness of $\Gamma_\mu$. Let $0< a < \frac{1}{p}$, so that $f_{a}(z)=\frac{1}{(1-z)^{a}} \in H^p$. Consequently
\begin{equation}\label{Gfa}
\begin{split}
    \Gamma_{\mu}(f_{a})(z)&= \bigintsss_0^1\frac{1}{\left[ 1-\frac{t}{1-(1-t)z}\right]^a}\ \frac{1}{1-(1-t)z} d\mu(t)
    \\[0.1in]
    &=\int_0^1 \frac{[1-(1-t)z]^{a-1}}{\left[(1-t)(1-z)\right]^a}d\mu(t)\\[0.1in]
    &= f_{a}(z)\,\int_{0}^{1}\frac{[1-(1-t)z]^{a-1}}{(1-t)^{a}}\,d\mu(t) \,,
    \end{split}
\end{equation}
and we note that
\[
\begin{split}
\left|\int_{0}^{1}\frac{(1-(1-t)z)^{a-1}}{(1-t)^{a}}\,d\mu(t)\right| &\geq \mbox{Re}\int_{0}^{1}\frac{(1-(1-t)z)^{a-1}}{(1-t)^{a}}\,d\mu(t)\\
&\geq \int_{0}^{1}\frac{(2-t)^{a-1}}{(1-t)^{a}}\,d\mu(t)
\\
&\geq \frac{1}{2^{1-a}}\int_{0}^{1}\frac{1}{(1-t)^{a}}\,d\mu(t)\,.
\end{split}
\]





Since $\Gamma_\mu$ is bounded in $H^p$, the above computation implies that
\begin{equation*}
\frac{1}{2^{1-a}}\int_{0}^{1}\frac{1}{(1-t)^{a}}\,d\mu(t) \, ||f_a||_{H^p} \leq ||\Gamma_{\mu}f_{a}||_{H^p}\leq ||f_{a}||_{H^p}||\Gamma_{\mu}||_{H^p\to H^p} ,
\end{equation*}
and letting $a\to \frac{1}{p}$
\begin{equation}\label{normG1}
    \frac{1}{2^{1-\frac{1}{p}}} \int_{0}^{1}\frac{1}{(1-t)^{\frac{1}{p}}}\,d\mu(t) \leq ||\Gamma_{\mu}||_{H^p\to H^p}.
\end{equation}

With similar reasoning, since $\Gamma^*_\mu$ is bounded in $H^q$ when $\frac{1}{q}+\frac{1}{p}=1$, we observe that if $0< b < \frac{1}{q}$, then by Lemma \ref{the adjoint Gm}
\begin{equation}\label{G*fa}
    \Gamma_{\mu}^{*}(f_{b})(z)\,=\,f_{b}(z)\,\int_{0}^{1}\frac{(1-tz)^{b-1}}{t^{b}}\,d\mu(t) \,,
\end{equation}
and
\begin{align*}
\left|\int_{0}^{1}\frac{(1-tz)^{b-1}}{t^{b}}\,d\mu(t)\right|\,\geq \, \frac{1}{2^{1-b}}\int_{0}^{1}\frac{1}{t^{b}}\,d\mu(t)\,.
\end{align*}
Therefore, letting $b \to \frac{1}{q}$,
\begin{equation}\label{normG2}
\frac{1}{2^{\frac{1}{p}}}\int_{0}^{1}\frac{1}{t^{1-\frac{1}{p}}}\,d\mu(t)\leq ||\Gamma_{\mu}^{*}||_{H^q\to H^q} \leq\,C_p\, ||\Gamma_{\mu}||_{H^p\to H^p}\,.
\end{equation}
Finally, putting together \eqref{normG1} and \eqref{normG2}, we have
\[\small
\begin{split}
\int_{0}^{1}\frac{t^{\frac{1}{p}-1}}{(1-t)^{\frac{1}{p}}}\, d\mu(t) & \leq
2^{\frac{1}{p}}\int_{0}^{1/2}\frac{1}{t^{1-\frac{1}{p}}}\, d\mu(t) +\left( \frac{1}{2}\right)^{\frac{1}{p}-1} \int_{1/2}^{1}\frac{1}{(1-t)^{\frac{1}{p}}}\, d\mu(t)
\\[0.1in]
&=2^{\frac{1}{p}}\int_{0}^{1}\frac{1}{t^{1-\frac{1}{p}}}\,d\mu(t) + 2^{1-\frac{1}{p}}\int_{0}^{1}\frac{1}{(1-t)^{\frac{1}{p}}}\,d\mu(t)
\\[0.1in]
&\leq \,C_p\,||\Gamma_{\mu}||_{H^p\to H^p}\,.
\end{split}
\]


On the other hand, it is clear that the condition \eqref{condition p>1} is also sufficient. Indeed, by Lemma \ref{normT_t p} and the generalized Minkowski's inequality, we have that
$$
    ||\Gamma_{\mu} f||_{H^p}\leq \int_0^1 \n{T_t f}_{H^p}\,d\mu(t)\leq B_p \,\norm{f}_{H^p} \int_{0}^{1}\frac{t^{\frac{1}{p}-1}}{(1-t)^{\frac{1}{p}}}\,d\mu(t)\ .
$$
Moreover, from Lemma \ref{normT_t p}, when $p\geq 2$ the constant $B_p$ can be chosen equal to $1$.

The case $p=1$ is treated in a similar way. Using the exact same reasoning, one can verify that the condition \eqref{condition p>1} is sufficient for the boundedness of $\Gamma_\mu$ in $H^1$.

Conversly, let us fix $f=1$. Then we have
\begin{align*}
\G_{\m}(1)(z)&=\int_0^1\frac{1}{1-(1-t)z}\,d\m(t)\\
&=\int_0^1\sum_{n=0}^{\infty}(1-t)^nz^n\,d\m(t)\\
&=\sum_{n=0}^{\infty}\left(\int_0^1(1-t)^n\,d\m(t)\right)z^n\,.
\end{align*}
Applying Hardy's inequality we get
\[
\sum_{n=0}^{\infty}\frac{1}{n+1}\int_0^1(1-t)^n\,d\m(t)\leq \pi\n{\G_{\m}(1)}_{H^1}\,,
\]
hence
\begin{align*}
\int_0^1\frac{1}{1-t}\log\left(\frac{e}{t}\right)\,d\m(t)&=
\int_0^1\frac{1}{1-t}\log\frac{e}{1-(1-t)}d\m(t)\\[0.1in]
&= \int_0^1\sum_{n=0}^{\infty}\frac{1}{n+1}(1-t)^n\,d\m(t)\,\,\\[0.1in]
&\leq\,\pi\n{\G_{\m}(1)}_{H^1}<\infty \,,
\end{align*}
which proves our claim.

Next, we obtain the exact value of the norm of the operator
\[
\G_\mu:\,H^p \to H^p,\;\; 2\leq p < \infty\,.
\]
For that range of $p$, as noted before, Lemma \ref{normT_t p} combined with Minkowski's inequality gives a sharp upper bound
\[
\|\Gamma_\mu\|_{H^p\to H^p}\leq \int_0^1 \frac{t^{\frac{1}{p}-1}}{(1-t)^{\frac{1}{p}}}\,d\mu(t)\, . 
\]
For this reason, we need to estimate a sharp, lower bound for the norm of $\Gamma_\mu$. Some of the ideas that we employ here, implicitly appeared in \cite{LINDSTROM2022} in connection to the exact value of the {\it essential norm} of a class of generalized integral operators.

For $\delta >0$, define
\[
    \Gamma_\mu^\delta(f)(z):=\int_{\delta}^{1}T_t(f)(z)d\mu(t).
\]
Let $f_{a}(z)=\dfrac{1}{(1-z)^{a}}$, $0<a<\frac{1}{p}$. We note that 
\begin{equation}\label{E_2}
    \lim_{a \to \frac{1}{p}}\|f_a\|_{H^{p}}=\infty\,,
\end{equation}
and the sequence of the normalized functions $\{\frac{f_a}{\norm{f_a}}\}_a$ tends weakly to zero as $a\to \frac{1}{p}$. For the ease of notation, we write $\Tilde{f_a}=\frac{f_a}{\norm{f_a}}$. 

Working as before, we write  $\Gamma^\delta_{\mu}(f_a)=\Lambda_{a}\cdot f_{a}$, where
$$
\Lambda_{a}(z)=\int_{\delta}^{1}\frac{(1-(1-t)z)^{a-1}}{(1-t)^{a}}\,d\mu(t).
$$

 Let $(a_{n},z_{n})\to (\frac{1}{p},1)$, where $ 0 < a_{n} < \frac{1}{p}$ for each $n\in \mathbb{N}$, and $\{a_n\}_n$ is an increasing sequence. Since, for $t \in [\delta,1]$,
 \[
 \Big|\frac{(1-(1-t)z)^{a_n-1}}{(1-t)^{a_n}}\Big|\leq \frac{\delta^{a_1-1}}{|1-t|^{\frac{1}{p}}}\, , 
 \]
by an application of the dominated convergence theorem we get
\begin{align*}
\lim_{(a_n,z_n)\to (\frac{1}{p},1)}&|\Lambda_{a_n}(z_n)|=\Lambda_{\frac{1}{p}}(1)=\int_{\delta}^{1}\frac{t^{\frac{1}{p}-1}}{(1-t)^{\frac{1}{p}}}\,d\mu(t)
\end{align*}
and, therefore, for every $z \in \overline{\mathbb D}$ we have
\begin{equation}\label{E_Lambda} 
\lim_{(a,z)\to (\frac{1}{p},1)}|\Lambda_{a}(z)|=\Lambda_{\frac{1}{p}}(1).
\end{equation}

Notice that
\[
\abs{\;\norm{\G^\delta_\mu (\Tilde{f_a})}_{H^p}\,-\,\Lambda_{\frac{1}{p}}(1)\;}^p \leq 
\norm{\Tilde{f_a}\,(\Lambda_a - \Lambda_{\frac{1}{p}}(1))}^p_{H^p}\,.
\]
Let $r>0$, and set $D_r=\{z\in \C:\,\abs{z-1}\leq r\}$. Then
\[
\begin{split}
\norm{\Tilde{f_a}\,(\Lambda_a - \Lambda_{\frac{1}{p}}(1))}^p_{H^p}\,&\leq \,
\sup_{\zeta \in \partial \D \setminus D_r}\, \abs{\Tilde{f_a}(\zeta)}^p \left( \norm{\Lambda_a}_{H^p}\,+\,\Lambda_{\frac{1}{p}}(1) \right)^p\,+
\\
& \quad +\,\sup_{\zeta \in \partial \D \cap D_r} \abs{\Lambda_a(\zeta) - \Lambda_{\frac{1}{p}}(1)}^p\,\norm{\Tilde{f_a}}^p_{H^p}\,.
\end{split}
\]
For the first term, for $a$ sufficiently close to $\frac{1}{p}$, we have that 
$$\sup_{\zeta\in \partial\D\setminus D_r}\,\abs{\Tilde{f_a}(\zeta)}^p < \epsilon\,,$$
 since $\Tilde{f_a} \to 0$, uniformly in $\abs{\zeta -1}>r$, as $a\to \frac{1}{p}$. For the second term, considering \eqref{E_Lambda}, we have that $\abs{\Lambda_a(\zeta) - \Lambda_{\frac{1}{p}}(1)}< \epsilon$, hence $\norm{\G^\delta_\mu (\Tilde{f_a})}_{H^p} \to \Lambda_{\frac{1}{p}}(1)$, as $a\to \frac{1}{p}$, which, in turn, implies that
 \[
 \norm{\G^\delta_\mu}_{H^p \to H^p} \geq \Lambda_{\frac{1}{p}}(1)\,.
 \]
 Consequently, 
 \begin{align*}
\|\Gamma_\mu\|_{H^p\to H^p}\,=&\;\|\Gamma_\mu-\Gamma_{\mu}^\delta+\Gamma_{\mu}^\delta\|_{H^p\to H^p}\\
\geq& \;  \|\Gamma_{\mu}^\delta\|_{H^p\to H^p} - \|\Gamma_\mu-\Gamma_{\mu}^\delta\|_{H^p\to H^p}\\
\geq&\; \int_\delta^1 \dfrac{t^{\frac{1}{p}-1}}{(1-t)^{\frac{1}{p}}}\,d\mu(t) -\int_0^\delta \dfrac{t^{\frac{1}{p}-1}}{(1-t)^{\frac{1}{p}}}\,d\mu(t) ,
\end{align*}
and letting $\delta \to 0^+$, the result follows.
\end{proof}


As a final result for this section, we study the compactness of the operators $\G_\mu$.



\begin{proof}[Proof of Theorem \ref{Compact}]
Let $1<p<\infty$ and $\frac{1}{q}+\frac{1}{p}=1$. Suppose, on the contrary, that $\Gamma_{\mu}$ is compact on  $H^p$. Since the family of functions
\[
 k_{w}(z)=\frac{(1-|w|^{2})^{\frac{1}{q}}}{1-\overline{w}z}
\]
converges to zero weakly as $|w|\to 1$, we have that $\norm{\Gamma_{\mu}(k_w)}_{H^p} \to 0$.
By considering the classical growth estimate \eqref{growth} for functions in $H^p$ , we get
\begin{align*}
\left|\int_{0}^{1}\frac{(1-|w|^{2})^{\frac{1}{q}}(1-|z|^{2})^{\frac{1}{p}}}{1-(1-t)z-\overline{w}t}\,d\mu(t)\right|&=
(1-|z|^{2})^{\frac{1}{p}}\left|\Gamma_{\mu}(k_w)(z)\right|\\
&\leq 2^{\frac{1}{p}}||\Gamma_{\mu}(k_w)||_{H^p}\,.
\end{align*}
Let $z=w=r \in (0,1)$. Then
\begin{align*}
\mu(0,1)\leq\frac{1-r^{2}}{1-r}\mu(0,1)
\leq 2^{\frac{1}{p}}||\Gamma_{\mu}(k_r)||_{H^p}\,.
\end{align*}
By letting  $r\to 1^{-}$, we obtain that $\mu(0,1)=0$, which is a contradiction. Consequently $\Gamma_\mu$ cannot be compact in $H^p$,  $p>1$.

Next, we deal with the case $p=1$. Suppose that for some measure $\mu$ , $\Gamma_\mu$ is compact on $H^1$.  Let $k_r(z)=\frac{1}{1-rz},\, 0<r<1$ and $z\in \mathbb{D}.$ Notice that $\Vert k_r \Vert_{H^1}$ is comparable to $ \log\big(\frac{e}{1-r}\big)$, as $r\to 1^{-}$, and consider the normalized functions $ \hat{k}_r = k_r / \Vert k_r \Vert_{H^1}$. Then the compactness of $\Gamma_\mu$ implies that $\Gamma_\mu(\hat{k}_r) $ converges strongly to zero \cite[Lemma 3.7]{Tjani2003} as $r\to 1^-$. We shall use the Fej\'er-Riesz inequality in order to arrive at a contradiction. Notice that on the segment $(0,1)$ the functions $\Gamma_\mu(\hat{k}_r)$ are positive, therefore for some positive constant $C$, we get

  \begin{align*}
      \int_0^1|\Gamma_\mu(\hat{k}_r)(s)| ds & = \frac{1}{\Vert k_r \Vert_{H^1}}  \int_0^1 \int_0^1 \frac{1}{1-rt-(1-t)s} d\mu(t)  ds \\
      = & \frac{1}{\Vert {k}_r \Vert_{H^1}} \int_0^1 \frac{1}{1-t} \log\bigg( \frac{1-rt}{t(1-r)} \bigg) d\mu(t) \\
      & \geq\, C \int_0^1\frac{1}{1-t} \log\bigg( \frac{1-rt}{t(1-r)} \bigg)\log\bigg( \frac{e}{1-r} \bigg)^{-1} d\mu(t)\,.
  \end{align*}

  Now applying Fatou's Lemma, as $r\to 1^-$, we find that
  \[ 
  C \int_{0}^1 \frac{1}{1-t}d\mu(t) \leq   \liminf_{r\to 1^- } \int_0^1 |\Gamma_\mu(\hat{k}_r)(s)| ds \,\leq\, \pi   \lim_{r\to 1^-} \Vert \Gamma_\mu(\hat{k}_r) \Vert_{H^1} = 0\,. 
  \]
  This implies that $\mu = 0$.


 To show that $\Gamma_\mu$ is completely continuous whenever it is bounded, notice that $T_t$ is completely continuous for every $t\in (0,1)$, by a theorem of Cima and Matheson \cite[Proposition 1]{Cima1994}. Consider now a sequence of functions $\{ f_n \}$ which is weakly null in $H^1$. Then, by the complete continuity of $T_t$ we have that $\lim_n \Vert T_t f_n \Vert_{H^1} = 0$, for all $0<t<1 $. Furthermore, using Lemma \ref{normT_t p} and for some constant $C>0$, we have
  \begin{equation*}
      \Vert T_t  f_n \Vert_{H^1} \leq \sup_{n} \Vert f_n \Vert_{H^1} \Vert  T_t \Vert_{H^1} \,\leq \,C\, \sup_n \Vert f_n \Vert_{H^1} \log \frac{e}{t}\frac{1}{1-t}\,.
  \end{equation*}
  By hypothesis this is a function in $L^1((0,1), \mu)$ therefore applying the dominated convergence theorem we conclude that 
  \[
  \limsup_n \Vert \Gamma_\mu f_n \Vert_{H^1} \leq \limsup_n \int_0^1 \Vert T_t f_n \Vert_{H^1}\, d\mu(t) = 0\,.
  \]
\end{proof}

%%%%%%%%%%%%%%%%%%%%%%%%%%%%%%%%%%%%%%%%%%%%%%%%%%%%%%%%%%%%%%%%%%%




%%%%%%%%%%%%%%%%%%%%%%%%%%%%%%%%%%%%%%%%%%%%%%%%%%%%%%%%%%%%%%%%%%%

\section{Concluding remarks}
 There are still some questions regarding $\G_\mu$ which we have not been able to resolve. In specific, one might be tempted to think that the norm of $\G_\mu$ acting on $H^p,\;1<p<2$, is equal to $ \int_{0}^{1}\frac{t^{\frac{1}{p}-1}}{(1-t)^{\frac{1}{p}}}\,d\mu(t)$, as is the case for $p\geq 2$. Unfortunately, if the measure $\mu$ is a Dirac mass $\delta_{t_0} $ at $t_0\in (0,1)$, the operator $\G_\mu$ coincides with the weighted composition operator $T_{t_0}$. It is then easily verified that this is not even asymptotically true as $t_0\to 0^+$ and $p\to 1^+$. 


Regarding the range $0<p<1$, one can adapt our methods to show that a necessary condition for the boundedness of the operator $\G_\mu$ on $H^p$ is that 
\[
\int_0^1 \frac{1}{(1-t)^{\frac{1}{p}}} d\mu(t) < + \infty\,.
\]
However, it is not clear whether the above condition is also sufficient for the bounedness of $\G_\mu$.
% {\color{red}
% The authors believe that the norm of $\G_\mu$ is actually equal to , for all $p>2$, but we have not been able to acquire a sharp lower bound, given that the methods applied to similar problems in the literature (e.g. in \cite[Theorem 2]{Dostanic2008}), doesn't seem to work for general measures $\mu$.}

\subsection{Acknowledgments}
The authors would like to express their gratitude to professors Petros Galanopoulos and Aristomenis Siskakis, for introducing us to this problem and for all their valuable help during the preparation of this article.
We would also like to thank professors Santeri Miihkinen and Jani Virtanen for the valuable discussions on the topic, during their visit in Thessaloniki.
\bibliographystyle{plain}
\bibliography{Literature}






\end{document}
\section{The exact value of the norm}\label{exact norm}


\begin{thm}\label{T: essential spectrum}
Let $\Gamma_\mu $ be a bounded operator in $H^p$, $2\leq p < \infty$. For $\delta >0$, set
\[
    \Gamma_\mu^\delta(f)(z):=\int_{\delta}^{1}T_t(f)(z)d\mu(t).
\]
Then
\begin{equation}\label{E: norm essenziale}
    ||\Gamma^\delta_{\mu}||_{H^p\to H^p}\geq \int_{\delta}^{1}\frac{t^{\frac{1}{p} -1}}{(1-t)^{\frac{1}{p}}}\,d\mu(t).
\end{equation}
\end{thm}
\begin{proof}
Since $\Gamma_\mu$ is bounded in $H^p$, due to Theorem \ref{main theom p>1}
\[
\int_{0}^{1}\frac{t^{\frac{1}{p} -1}}{(1-t)^{\frac{1}{p}}}\,d\mu(t)<\infty\, .
\]
%Moreover $\|\Gamma_\mu^\delta\|_{H^p \to H^p}\geq \|\Gamma_\mu^\delta\|_{e}$.

Let $f_{a}(z)=\dfrac{1}{(1-z)^{a}}$, $0<a<\frac{1}{p}$. We note that 
\begin{equation}
    \lim_{a \to \frac{1}{p}}\|f_a\|_{H^{p}}=\infty\,,
\end{equation}
and the sequence of the normalized functions $\{\frac{f_a}{\norm{f_a}}\}_a$ tends weakly to zero as $a\to \frac{1}{p}$. For the ease of notation, we write $\Tilde{f_a}=\frac{f_a}{\norm{f_a}}$. 
If we verify that
$$
\liminf_{a\to \frac{1}{p}}\left\|\Gamma^\delta_{\mu}( \Tilde{f_a}) \right\|_{H^{p}}\geq \int_{\delta}^{1}\frac{t^{\frac{1}{p}-1}}{(1-t)^{\frac{1}{p}}}\,d\mu(t) \, ,
$$
then \eqref{E: norm essenziale} would be true.

Working as in the proof of Theorem \ref{main theom p>1}, we write  $\Gamma^\delta_{\mu}(f_a)=\Lambda_{a}\cdot f_{a}$, where
$$
\Lambda_{a}(z)=\int_{\delta}^{1}\frac{(1-(1-t)z)^{a-1}}{(1-t)^{a}}\,d\mu(t).
$$

 Let $(a_{n},z_{n})\to (\frac{1}{p},1)$, where $ 0 < a_{n} < \frac{1}{p}$ for each $n\in \mathbb{N}$, and $\{a_n\}_n$ is an increasing sequence. Since, for $t \in [\delta,1]$,
 \[
 \Big|\frac{(1-(1-t)z)^{a_n-1}}{(1-t)^{a_n}}\Big|\leq \frac{\delta^{a_1-1}}{|1-t|^{\frac{1}{p}}}\, , 
 \]
by an application of the dominated convergence theorem we get
\begin{align*}
\lim_{(a_n,z_n)\to (\frac{1}{p},1)}&|\Lambda_{a_n}(z_n)|=\Lambda_{\frac{1}{p}}(1)=\int_{\delta}^{1}\frac{t^{\frac{1}{p}-1}}{(1-t)^{\frac{1}{p}}}\,d\mu(t)
\end{align*}
and, therefore, for every $z \in \overline{\mathbb D}$ we have
\begin{equation}
  \lim_{(a,z)\to (\frac{1}{p},1)}|\Lambda_{a}(z)|=\Lambda_{\frac{1}{p}}(1).
\end{equation}
Obviously, for $c\geq 0$
\begin{align*}
\lim_{(c,z)\to (\frac{1}{p},1)}&\left( \inf_{a\in(c,\frac{1}{p})}\left\{\inf_{D(1,|1-z|)\cap\overline{\mathbb{D}}}|\Lambda_{a}(w)|\right\}\right)
&=\Lambda_{\frac{1}{p}}(1)=\int_{\delta}^{1}\frac{t^{\frac{1}{p}-1}}{(1-t)^{\frac{1}{p}}}\,d\mu(t).
\end{align*}
For $0<\rho<1$, we define 
$$
S_{a}(z)=
\begin{cases}
\inf\limits_{c\in(a,\frac{1}{p})}\left\{\inf\limits_{D(1,|1-z|)\cap\overline{\mathbb{D}}}|\Lambda_{c}(w)|\right\}, \quad &z\in D(1,\rho)\\
\\
0,\qquad &z\in \overline{\mathbb{D}}\setminus D(1,\rho)
\end{cases} .
$$
Because of \eqref{E_Lambda}, we write
\begin{equation}\label{E_1}
\lim_{(a,z)\to (\frac{1}{p},1)}S_{a}(z)=\int_{\delta}^{1}\frac{t^{\frac{1}{p}-1}}{(1-t)^{\frac{1}{p}}}\,d\mu(t).
\end{equation}
We chose $\rho$ small enough so that for $z\in D(1,\rho)\cap\overline{\mathbb{D}}$ and every $ \frac{1}{p}>c>c_1$,
$$
|S_{c}(z)-\Lambda_{\frac{1}{p}}(1)|<\varepsilon ,
$$
due to \eqref{E_1}.
Moreover, for every $\frac{1}{p}>c>c_2$,
$$
\sup_{z\in\overline{\mathbb{D}}\setminus D(1,\rho)}\abs{\Tilde{f_c}(z)} <\varepsilon, 
$$
due to \eqref{E_2}.
Therefore, we have
\begin{align*}
\Big| \, \| \Tilde{f_c}S_c\|_{L^{p}(\mathbb T)}-&\;\Lambda_{\frac{1}{p}}(1) \,\Big|^{p}\leq\int_{\mathbb{T}\cap D(1,\rho)} \Big| \Tilde{f_c}(\xi) \left(S_{c}(\xi)-\Lambda_{\frac{1}{p}}(1)\right) \Big|^{p}\,d\xi \,+\\
&\quad +\int_{\mathbb{T}\setminus D(1,\rho)} \Big| \,\Tilde{f_c}(\xi) \left(S_{c}(\xi)-\Lambda_{\frac{1}{p}}(1)\right)\,\Big| ^{p}\,d\xi\\[0.1in]
&\leq \varepsilon^{p}+\varepsilon^{p}\int_{\mathbb{T}\setminus D(1,\rho)} \left|S_{c}(\xi)-\Lambda_{\frac{1}{p}}(1)\right|^{p}\,d\xi\\[0.1in]
&\leq \varepsilon^{p}+\varepsilon^{p}\Lambda^p_{\frac{1}{p}}(1)\,. %Since the main limit exists the function S_c is bounded
\end{align*}
Thus

$$
\|\Tilde{f_c}\ \Lambda_{c}\|_{H^{p}}\geq\, \| \Tilde{f_c}S_{c}\|_{H^{p}}>\,\Lambda_{\frac{1}{p}}(1)-\varepsilon \left(1+\Lambda_{\frac{1}{p}}(1)\right)^{\frac{1}{p}}.
$$
Letting $c\to \frac{1}{p}$, followed by $\varepsilon\to0$,
$$
\liminf_{c\to \frac{1}{p}}\|\Gamma^\delta_{\mu}( \Tilde{f_c})\|_{H^{p}}=\liminf_{c\to \frac{1}{p}}\|\Tilde{f_c} \Lambda_{c}\|_{H^{p}}\geq \Lambda_{\frac{1}{p}}(1),
$$
which proves the desired estimate.
\end{proof}
We are now ready to prove Theorem 
\begin{proof}
Let 
$$
\Gamma_{\mu}^\delta(f)(z):=\int_{\delta}^1 T_{t}(f)(z) d\mu(t)= \int_0^1 T_{t}(f)(z)\mathcal{X}_{[\delta,1]}d\mu(t) .
$$
First of all due to Theorem \ref{main theom p>1}, we know that 
$$
\|\Gamma_\mu\|_{H^p\to H^p}\leq \int_0 ^1 \dfrac{t^{\frac{1}{p}-1}}{(1-t)^{\frac{1}{p}}}\,d\mu(t) .
$$
On the other hand, due to Theorem \ref{T: essential spectrum},
$$
\|\Gamma_{\mu}^\delta\|_{H^p\to H^p}\geq \int_\delta^1 \dfrac{t^{\frac{1}{p}-1}}{(1-t)^{\frac{1}{p}}}\,d\mu(t)\, .  
$$
Consequently,
\begin{align*}
\|\Gamma_\mu\|_{H^p\to H^p}\,=&\;\|\Gamma_\mu-\Gamma_{\mu}^\delta+\Gamma_{\mu}^\delta\|_{H^p\to H^p}\\
\geq& \;  \|\Gamma_{\mu}^\delta\|_{H^p\to H^p} - \|\Gamma_\mu-\Gamma_{\mu}^\delta\|_{H^p\to H^p}\\
\geq&\; \int_\delta^1 \dfrac{t^{\frac{1}{p}-1}}{(1-t)^{\frac{1}{p}}}\,d\mu(t) -\int_0^\delta \dfrac{t^{\frac{1}{p}-1}}{(1-t)^{\frac{1}{p}}}\,d\mu(t) ,
\end{align*}
and letting $\delta \to 0^+$, the result follows.
\end{proof}
$$$$$$$$$$$$$$$$$$$$$$$$$$$$$$$$$$$$$$$$$$$$$$$$$$$$$$$$$$$$$$$$$$$$$$$$$$$$$$
%------------------------------------------------------------------------------
% End of journal.tex
%------------------------------------------------------------------------------
\begin{comment}
\newpage
\section{More general}
Let $X$ be a Banach space with basis $\{e_n\}$ and $T: X\to X$ a bounded operator
with matrix $T=(a_{i j})$. This means that
$$
T(e_n)=\sum_{k=1}^{\infty}a_{k n}e_k
$$
for each $n$.
 Consider the matrix transformation

$$
T=\left(%
\begin{array}{ccccc}
            a_{11} & a_{12}  & a_{13}  & .  \\
 a_{21} & a_{22}  & a_{23}  & .  \\
  a_{31} & a_{32}  & a_{33}  & . \\
  .  & . & . & .  \\
\end{array}%
\right) \longrightarrow T_s=
\left(%
\begin{array}{ccccc}
a_{11} & 0  & 0  & .  \\
 a_{21} & a_{12}  & 0  & .  \\
  a_{31} & a_{22}  & a_{13}  & . \\
  .  & . & . & .  \\
\end{array}%
\right),
$$
 The operator $T_s$ maps the elements of the basis as follows
 $$
T_s(e_n)=\sum_{k=1}^{\infty}a_{k n}e_{n+k}=S^nT(e_n)
 $$
where $S: X\to X$ is the shift operator,
$$
S(\sum_1^{\infty}b_j e_j)
=\sum_1^{\infty}b_j e_{j+1}
$$


Converse of this
$$
T=\left(%
\begin{array}{ccccc}
            a_{11} & a_{12}  & a_{13}  & .  \\
 a_{21} & a_{22}  & a_{23}  & .  \\
  a_{31} & a_{32}  & a_{33}  & . \\
  .  & . & . & .  \\
\end{array}%
\right) \longrightarrow
\left(%
\begin{array}{ccccc}
a_{11} & a_{22}  & a_{33}  & .  \\
 a_{21} & a_{32}  & a_{43}  & .  \\
  a_{31} & a_{42}  & a_{53}  & . \\
  .  & . & . & .  \\
\end{array}%
\right),
$$\\


 This is the analogous between Hilbert and Ces\'aro matrices

$$
H=\left(%
\begin{array}{ccccc}
            1 & \frac{1}{2}  & \frac{1}{3}  & .  \\
  \frac{1}{2} & \frac{1}{3}  & \frac{1}{4}  & .  \\
  \frac{1}{3} & \frac{1}{4}  & \frac{1}{5}  & . \\
  .  & . & . & .  \\
\end{array}%
\right) \longrightarrow H_s=C=
\left(%
\begin{array}{ccccc}
            1 & 0  & 0  & .  \\
  \frac{1}{2} & \frac{1}{2}  & 0  & .  \\
  \frac{1}{3} & \frac{1}{3}  & \frac{1}{3}  & . \\
  .  & . & . & .  \\
\end{array}%
\right),
$$

More generally,  for a Hankel matrix $H_{\mu}$ generated by a sequence $(\mu_n)$,

$$
H_{\mu}=\left(%
\begin{array}{ccccc}
            \mu_0 & \mu_1  & \mu_2  & .  \\
  \mu_1 & \mu_2  & \mu_3  & .  \\
  \mu_2 & \mu_3  & \mu_4  & . \\
  .  & . & . & .  \\
\end{array}%
\right)  \longrightarrow (H_{\m})_s=
C_{\mu}=\left(%
\begin{array}{ccccc}
            \mu_0 & 0  & 0  & .  \\
  \mu_1 & \mu_1  & 0  & .  \\
  \mu_2 & \mu_2  & \mu_2  & . \\
  .  & . & . & .  \\
\end{array}%
\right),
$$

Iterating
$$
H=\left(%
\begin{array}{ccccc}
            1 & \frac{1}{2}  & \frac{1}{3}  & .  \\
  \frac{1}{2} & \frac{1}{3}  & \frac{1}{4}  & .  \\
  \frac{1}{3} & \frac{1}{4}  & \frac{1}{5}  & . \\
  .  & . & . & .  \\
\end{array}%
\right)
 \longrightarrow (H_s)_s=
 \left(%
\begin{array}{ccccc}
            1 & 0  & 0  & .  \\
  \frac{1}{2} & 0  & 0  & .  \\
  \frac{1}{3} & \frac{1}{2}  & 0  & . \\
  .  & . & . & .  \\
\end{array}
\right)
$$


\newpage
\section{some observations}

1. If $\G_{\m}: H^1\to H^1$ is bounded then
$$
\int_0^1\frac{1}{1-t}\log(\frac{1}{t})\,d\m(t)<\infty
$$
Proof. For $f=1$ we have
$$
\G_{\m}(1)(z)=\int_0^1\frac{1}{1-(1-t)z}\,d\m(t)=\int_0^1\sum_{n=0}^{\infty}(1-t)^nz^n\,d\m(t)=
\sum_{n=0}^{\infty}\left(\int_0^1(1-t)^n\,d\m(t)\right)z^n
$$
Then by Hardy's inequality we have
$$
\sum_{n=0}^{\infty}\frac{1}{n+1}\int_0^1(1-t)^n\,d\m(t)\leq \pi\n{\G_{\m}(1)}_{H^1}
$$
and thus
$$
\int_0^1\sum_{n=0}^{\infty}\frac{1}{n+1}(1-t)^n\,d\m(t)=\int_0^1\frac{1}{1-t}\log\frac{1}{1-(1-t)}d\m(t)=
\int_0^1\frac{1}{1-t}\log(\frac{1}{t})\,d\m(t)<\infty.
$$



\newpage


2. \underline{Families of spaces generated by integration operators}\\

Starting with the shifted operator $J_g(f)(z)=zT_g(f)(z):  X\to X$   then
$$
X_0=\{g: J_g: X\to X \,\,\text{is bounded}\},
$$
is a linear space and
$$
\n{g}_0= |g(0)|+\n{T_g}_{X\to X}
$$
is a norm on $X_0$. The following properties hold  \cite{SZ}:\\
1. If convergence in $X$ implies uniform convergence on compact subsets of $\D$
then $(X_0, \n{\,\,}_0) $ is a Banach space. Further if $X$ contains the constants then $X_0\subset X$.  \\
2. If the shift $f\to zf$ and $J_z(f)(z)=\int_0^zf(\z)d\z$ are bounded on $X$ then $X_0$ contains the polynomials.\\
3. If point evaluations $f\to f(w)$ and composition by Mobius automorphisms $f\to f\circ\phi_a$  are bounded on $X$
then $X_0$ is preserved under composition by $\phi_a$.

One then may consider $J_g : X_0\to X_0$ and define the space
$$
X_1=\{g: J_g: X_0\to X_0 \,\,\text{is bounded}\},
$$
with norm
$$
\n{g}_1= |g(0)|+\n{T_g}_{X_0\to X_0}
$$
and by iterating the procedure obtain a sequence of spaces
$$
X\supset X_1\supset X_2\supset ....
$$
which under the given norms seem to be all nonseparable Banach spaces.

A similar iteration procedure will give a decreasing sequence of Hilbert spaces if we start with a Hilbert space $X$ and
let the next space $X_0$ be the space of symbols $g$  for which $J_g$ is Hilbert-Schmidt on $X_0$. For example if $X=H^2$ then
$X_0=\mathcal{D}$ and $X_1$ will be the space of $f$ for which $\sum_{n} n\log(n)|\hat{f}(n)|^2<\infty$. I do not know what are
the weights for rest of spaces, some work Jonathan has sent previously  is related to it.

\begin{comment}
Let $H^2$ be the Hardy  space on the unit disc $\D$ with inner product
\begin{align*}
 \langle f, g\rangle_{H^2}&=\frac{1}{2\pi}\int_0^{2\pi}f(e^{i\th})\overline{g(e^{i\th})}\,d\th\\
 &= \sum_{k=0}^{\infty}a_k\overline{b_k}\\
 &=
f(0)\overline{g(0)}+2\int_{\mathbb{D}}f'(z) \overline{g'(z)}\log(\frac{1}{|z|})\,dm(z)
 \end{align*}
 where $a_k=\hat{f}(k)   $, $b_k=\hat{g}(k)   $ and  $dm=\frac{dxdy}{\pi}$. The norm of $f\in H^2$ is
$$
\n{f}_{H^2}^2=
|f(0)|^2+2\int_{\mathbb{D}}|f'(z)|^2\log(\frac{1}{|z|})\,dm(z)
=\sum_{k=0}^{\infty}|a_k|^2,
$$
and the usual orthonormal basis is  $\{e_n(z)=z^n, n=0,1,2,...\}$.



Consider the operator
$$
T_g(f)(z)=\int_0^zf(\zeta)g'(\zeta)\,d\zeta.
$$
with holomorphic symbol $g$, acting on  functions $f$ holomorphic on $\D$. We may assume $g(0)=0$.
We are going to define a sequence of weighted Dirichlet spaces
$$
\cd=\cd_0\supset \cd_1\supset \cd_2\supset ...
$$
each of which is the space of
symbols $g$ for which $T_g$ is
a Hilbert-Schmidt operator on the previous space.\\

First,   $T_g\in S_2(H^2)$  (is  Hilbert-Schmidt  on $H^2$)   $\Leftrightarrow$
$\sum_{n=0}^{\infty}\n{T_g(z^n)}_{H^2}^2<\infty$. We have
\begin{align*}
\sum_{n=0}^{\infty}\n{T_g(z^n)}_{H^2}^2&
=2\sum_{n=0}^{\infty}\int_{\mathbb{D}}|z|^{2n}|g'(z)|^2\log(1/|z|)\,dm(z)\\
& \sim  \sum_{n=0}^{\infty}\int_{\mathbb{D}}|z|^{2n}|g'(z)|^2(1-|z|^2)\,dm(z)\\
&=\int_{\mathbb{D}}\left(\sum_{n=0}^{\infty}|z|^{2n}\right)|g'(z)|^2(1-|z|^2)\,dm(z)\\
&=\int_{\mathbb{D}}|g'(z)|^2\,dm(z).
\end{align*}
Thus $ T_g\in S_2(H^2)$ if and only if $ g\in \cd_0=\cd$, the classical Dirichlet space
of functions $f(z)$  for which $f(0)=0$ and
 $$
 \n{f}_{\cd_0}^2=\int_{\mathbb{D}}|f'(z)|^2\,dm(z)=\sum_{k=1}^{\infty}k|\hat{f}(k)|^2<\infty,
 $$
with inner product
$$
\langle f, g\rangle_{\cd_0}= \int_{\D}f'(z)\overline{g'(z)}\,dm(z)= \sum_{k=1}^{\infty}k|\hat{f}(k)||\overline{\hat{g}(k)}|,
$$
and an orthonormal basis  $\{e_n^0(z)=\frac{z^n}{\n{z^n}_{\cd_0}}=\frac{z^n}{\sqrt{n}},\, n=1,2,...\}$.


Next consider $T_g$ to act on $\cd_0$. Then $T_g\in S_2(\cd_0)$ if and only if
 $\sum_{n=1}^{\infty}\n{T_g(\frac{z^n}{\sqrt{n}})}_{\cd_0}^2<\infty$. We compute

\begin{align*}
\sum_{n=1}^{\infty}\n{T_g(\frac{z^n}{\sqrt{n}})}_{\cd_0}^2&
=\sum_{n=1}^{\infty}\int_{\D}\frac{|z|^{2n}}{n}|g'(z)|^2\,dm(z)\\
&=\int_{\D}\left(\sum_{n=1}^{\infty}\frac{|z|^{2n}}{n}\right)|g'(z)|^2
\,dm(z)\\
&=\int_{\D}|g'(z)|^2\log(\frac{1}{1-|z|^2})\,dm(z).
\end{align*}

Thus $T_g\in S_2(\cd_0)$ if and only if $g\in \cd_1$,  where $\cd_1$ is the space of analytic $f$ with $f(0)=0$ for which
$$
\n{f}_{\cd_1}^2=\int_{\D}|f'(z)|^2\log(\frac{1}{1-|z|^2})\,dm(z)<\infty.
$$
For $f=\sum a_nz^n\in \cd_1$,
\begin{align*}
\n{f}_{\cd_1}^2&=\int_{\mathbb{D}}|f'(z)|^2\log(\frac{1}{1-|z|^2})\,dm(z)\\
&=\sum_{n=1}^{\infty}n^2\left(\sum_{k=1}^{\infty}\frac{1}{k(n+k)}\right)|a_n|^2\\
&= \sum_{n=1}^{\infty}n\left(\sum_{k=1}^n\frac{1}{k}\right)|a_n|^2
\end{align*}
An orthonormal basis is $\{e^1_n(z)=\frac{z^n}{\n{z^n}_{\cd_1}}=\frac{z^n}{\sqrt{n s_n^1}},\, n=1,2,...\}$
where
$$
s_n^1=\sum_{k=1}^n\frac{1}{k}\sim \log n
$$
Next consider $T_g$ acting on $\cd_1$, and let
$$
\cd_2=\{g: T_g\in S_2(\cd_1)\}
$$
with norm
$$
\n{f}_{\cd_2}^2=
$$





$$
s_{n,1}=\n{z^n}_{\cd_1}^2=\sum_{k=1}^{\infty}\frac{n^2}{k(n+k)}
$$
so $e_n^1(z)=\frac{z^n}{\sqrt{q_n^1}},\,\, n=1,2,3,..$ is an orthonormal basis
for $\mathcal{D}_1$.

Next $T_g\in S_2(\mathcal{D}_1)$ is and only if
$\sum_{n=1}^{\infty}\n{T_g(\frac{z^n}{\sqrt{q_n^1}})}_1^2<\infty$, i.e.
\begin{align*}
\sum_{n=1}^{\infty}\n{T_g(\frac{z^n}{\sqrt{q_n^1}})}_1^2&
=\sum_{n=1}^{\infty}\int_{\mathbb{D}}\frac{|z|^{2n}}{q_n^1}|g'(z)|^2\,dm(z)\\
&=\int_{\mathbb{D}}\left(\sum_{n=1}^{\infty}\frac{|z|^{2n}}{q_n^1}\right)|g'(z)|^2
\,dm(z)\\
&=\int_{\mathbb{D}}|g'(z)|^2 L_1(z)\,dm(z)<\infty,
\end{align*}
where $L_1(z)=\sum_{n=1}^{\infty}\frac{|z|^{2n}}{q_n^1}\sim
\sum_{n=1}^{\infty}\frac{|z|^{2n}}{n\log n}$.\\

Define the weighted Dirichlet space $\mathcal{D}_2$ by this condition,
$$
\mathcal{D}_2=\left\{f\,\,\mbox{analytic on}\,\,\mathbb{D}, f(0)=0,
\int_{\mathbb{D}}|f'(z)|^2L_1(z)\,dm(z)<\infty\right\}.
$$
For $f=\sum a_nz^n\in \mathcal{D}_2$,
\begin{align*}
\n{f}_2^2&=\int_{\mathbb{D}}|f'(z)|^2L_1(z)\,dm(z)\\
&=.....\\
&= .......\\
&\sim ....
\end{align*}
Then
$$
\n{z^n}_2=.....
$$
so $e_n^2(z)=\frac{z^n}{....},\,\, n=1,2,3,..$ is an orthonormal basis
for $\mathcal{D}_2$.\\

We are led to the following \\

Define  the sequences  $\sigma^0, \sigma^1, \sigma^2, ..., \sigma^m, ...$, each of which is a
numerical sequence,
$$
\sigma^m=(s_1^m, s_2^m, s_3^m, \cdots )
$$
as follows.
Let $\sigma^0$ be the constant sequence with all terms equal to $1$,
$$
\sigma^0 = (1, 1, 1, \cdots ),
$$
and let  $\sigma^{m+1}=
(s_1^{m+1}, s_2^{m+1}, s_3^{m+1}, \cdots )$ be  defined
recursively from $\sigma^m$ by  the rule
$$
s_n^{m+1} = \sum_{j=1}^{\infty}\frac{n}{j(n+j)s_j^m}, \qquad
n=1,2,3,\dots
$$
Thus for $m=1$
$$
s_n^1 = \sum_{j=1}^{\infty}\frac{n}{j(n+j)}
=\sum_{k=1}^{n}\frac{1}{k} \qquad\left(\sim \log n \,\,\right)
$$
and for $m=2$
$$
s_n^2 = \sum_{j=1}^{\infty}\frac{n}{j(n+j)s_j^1}
=\sum_{j=1}^{\infty}\frac{n}{j(n+j)\sum_{k=1}^{j}\frac{1}{k}}
$$\\

Questions\\


1. Find the asymptotic behavior of $s_n^m$\\

2. Study the spaces $\mathcal{D}_m$ (growth of functions other properties ...)\\

3. Clearly $\mathcal{D}_{m+1}\subset \mathcal{D}_m$ for each $m$. Intersection $\cap_m \mathcal{D}_m$?\\

4. Other
\end{comment}
\section{Representation of Ces\'aro and Hilbert matrices via semigroups}\label{semi}
Let $h:\D\to \C$ be a univalent and starlike function, with $h(0)=0$. One can also consider a spiral-like function $h$, but then some mild modifications in what follows are needed. Let $\varphi_t(z)=h^{-1}(th(z)),\;\;0\leq t\leq 1$, be a self-map of $\D$ for each $t$ and consider the path $\gamma(t)=\gamma_z(t)=\varphi_t(z)$, for each $z\in \D$, which is differentiable w.r.t. $t$, and evaluate the integral representation \eqref{Ces-g} of $C$  along the path $\gamma$. We get
\[
C(f)(z)=\dfrac{h(z)}{z}\int_0^1 w(\varphi_t(z)) f(\varphi_t(z))\,dt,
\]
where the weight function $w$ is given by
\[
w(z)=\dfrac{1}{(1-z)h'(z)}\,.
\]

For the Hilbert case, consider the function $\psi_t(z)=\dfrac{\varphi_t(z)}{z}$. Since $\varphi_t(0)=0$, by Schwarz's lemma, the function
$\psi_t$ maps $\D$ into itself and the path $\beta(t)=\beta_z(t)=\psi_t(z),\;0\leq t\leq 1$ is inside the disc, and joins $0$ to $1$. Evaluating the integral representation \eqref{Hil-g} of $H$, along the path $\beta$, gives
\[
\begin{split}
H(f)(z)&=\dfrac{h(z)}{z}\int_0^1 w(\varphi_t(z)) f(\psi_t(z))\,dt
\\
&=\dfrac{h(z)}{z}\int_0^1 w(z\psi_t(z)) f(\psi_t(z))\,dt,
\end{split}
\]
where the weight $w$ is as before.

Thus, each starlike (spiral-like) $h$ induces a representation for the operators $C$ and $H$. The choice $h(z)=\frac{z}{1-z}$ gives the classical representations. As another example, for $h(z)=\log\frac{1}{1-z}$, we obtain $\varphi_t(z)=1-(1-z)^t$, and consequently we get
\[
C(f)(z)=\dfrac{1}{z}\log\dfrac{1}{1-z} \int_0^1 f(\varphi_t(z))\,dt
\]
and
\[
H(f)(z)=\dfrac{1}{z}\log\dfrac{1}{1-z} \int_0^1 f\left(\dfrac{\varphi_t(z)}{z}\right)\,dt\,.
\]

Maybe an appropriate choice of the function $h$ can give estimates for the norms of the weighted composition operators
\[
T_t(f)(z)=w(\varphi_t(z)) f(\varphi_t(z))
\]
and
\[
S_t(f)(z)=w(\varphi_t(z)) f(\psi_t(z))\,,
\]
and then also information about the norms of $C$ and $H$.


\title{Ces\'aro and Hilbert operators and generalizations}


%\subjclass{Primary ......; Secondary ......}
\date{\today}

\maketitle

\section{Preliminaries}

Let $\D$ be the unit disc in the complex plane $\C$, and $\H(\D)$ the Frechet space of all analytic functions on $\D$.
The description below can be applied to several spaces  $X\hookrightarrow\H(\D)$, for which  the
monomials $e_n(z)=z^n$, $n=0,1, 2, ...$ form  a (Schauder) basis.

Let $0<p<\infty$. For $f\in \H(\mathbb{D})$ and $0\leq r<1$, define the integral means
\[
M_p(r,f)\,=\,\left(\dfrac{1}{2\pi}\int_{0}^{2\pi}\vert f(re^{i\theta})\vert^p\, d\theta\right)^{\frac{1}{p}}\,.
\]
For $p=\infty$, set
\[
M_\infty(r,f)\,=\,\mathop{\max}_{0\leq \theta <2\pi}\vert f(re^{i\theta})\vert
\]

For $0<p\leq\infty$, the Hardy space $H^p =H^p(\mathbb{D})$ consists of the functions $f\in \H(\D)$ for which
\[
\parallel f \parallel_{H^p}\,=\,\sup_{r<1}M_p(r,f) <+\infty\,.
\]
If $p=\infty$, $H^\infty$ is the space of bounded analytic functions on $\mathbb{D}$, equipped with the supremum norm
\[
\n{f}_{\infty}:=\sup_{z\in\D}|f(z)|\,.
\]

For $\,0\,<\,p\,<\,\infty$, the Bergman space $A^p_a$ consists of the functions $f \in \H(\mathbb{D})$ for which
\[
\parallel f \parallel_{A^p_a}\,=\, \left((a+1)\int_\mathbb{D} \vert f(z)\vert^p(1-|z|^2)^a\,dA(z) \right)^{\frac{1}{p}}<\infty,
\]
where $dA(z)=\,\dfrac{dx\, dy}{\pi}=\,\dfrac{r \,dr\, dt}{\pi}$, is the normalized Lebesgue area measure on $\mathbb{D} $. For $a=0$, denote $A^p_0=A^p$.





\subsection{The Ces\'aro operator}
Consider the Ces\'aro operator acting on a function $f(z)=\sum_{n=0}^{\infty}a_nz^n\in X$
\begin{equation}\label{Ces-m}
C(f)(z)=\frac{1}{z}\int_0^zf(\z)\frac{1}{1-\z}\,d\z
=\sum_{n=0}^{\infty}\left(\frac{1}{n+1}\sum_{k=0}^na_k\right)z^n,
\end{equation}
which can also be written as
\begin{equation}\label{Ces-g}
 C(f)(z)=\frac{1}{z}\int_0^zf(\zeta)g'(\zeta)\,d\zeta=\int_0^1f(tz)g'(tz)\,dt,
\end{equation}
where $g(z)=\log\frac{1}{1-z}$.


Its matrix w.r.t. to the basis $\{e_n(z)=z^n\}$ is
$$
C=\left(%
\begin{array}{ccccc}
            1 & 0  & 0  & .  \\
  \frac{1}{2} & \frac{1}{2}  & 0  & .  \\
  \frac{1}{3} & \frac{1}{3}  & \frac{1}{3}  & . \\
  .  & . & . & .  \\
\end{array}%
\right).
$$


\subsection{The Hilbert matrix}
Consider also the  the Hilbert matrix
$$
H=\left(%
\begin{array}{ccccc}
            1 & \frac{1}{2}  & \frac{1}{3}  & .  \\
  \frac{1}{2} & \frac{1}{3}  & \frac{1}{4}  & .  \\
  \frac{1}{3} & \frac{1}{4}  & \frac{1}{5}  & . \\
  .  & . & . & .  \\
\end{array}%
\right),
$$
 acting on $X$ as
\begin{equation}\label{Hil-m}
H(f)(z)
=\sum_{n=0}^{\infty}\left(\sum_{k=0}^{\infty}\frac{a_k}{n+k+1}\right)z^n.
\end{equation}
It has the integral representation
\begin{equation}\label{Hil-g}
H(f)(z)=\int_0^1f(t)\frac{1}{1-tz}\,dt=\int_0^1f(t)g'(tz)\,dt,
\end{equation}
where $g$ is as in \eqref{Ces-g}.

\subsection{A relation between $C$ and $H$}
 Notice that $C(f)$ is well defined, analytic on $\D$ for all $f\in \H(\D)$, while
$H(f)$ is not always defined (for example if $f(z)=1/(1-z)$ the series \eqref{Hil-m}
does not make sense).
However  Hardy's inequality for the Hardy space $H^1$ implies  that $H(f)(z)$
is analytic on $\D$ for each $f\in H^1$. Moreover, one can prove that $H(f)(z)$ is an analytic function of $\D$ if and only if
\[
\sum_{n\geq 0}\dfrac{a_n}{n+1}<\infty.
\]

It is well known that $C$ is bounded on the Hardy spaces $ H^p$,  $1\leq p<\infty$, and
$H$ is bounded on $H^p$ for $1< p<\infty$ (but not bounded on $H^1$).

Changing the path of integration in (\ref{Ces-m}) and in (\ref{Hil-g}) (see \cite[Section 2]{DS}), we obtain the following representations
$$
C(f)(z)= \int_0^1\frac{t}{1-(1-t)z}f(\frac{tz}{1-(1-t)z})\,dt,
$$
and
$$
H(f)(z)=\int_0^1\frac{t}{1-(1-t)z}f(\frac{t}{1-(1-t)z})\,dt
$$
in terms of weighted composition operators.

Notice that the matrix $C$ is obtained from $H$ by the operation:

{\bf ``Move down the $k^{th}$ column of $H$  $k-$places, and fill the empty entries with $0$'s, $k=0,1,2,...$''}

Thus  for $e_n(z)=z^n$
 $$
C(e_n)(z)=e_n(z)H(e_n)(z) = \frac{1}{z}\left(\log\frac{1}{1-z}- \sum_{k=1}^n\frac{z^k}{k}\right),
 $$
or in terms of the shift operator $S(f)(z)=zf(z)$
$$
C(e_n)=S^nH(e_n), \,\,  n=0,1,2,...
$$
\subsection{Plan of this note}
In section \ref{Volt} we comment on the connection between generalized Volterra operators and a generalized Hilbert operator $H_g$, studied in \cite{GGPS}. In section \ref{Haus}, we describe the generalized Hilbert operator (or generalized Hausdorff operator) $\G_\m$ that we propose to study and state some preliminary results. Finally, in section \ref{semi}, we describe a representation of the classical Hilbert and Ces\'aro matrices via semigroups of analytic functions.


\section{Volterra and generalized Hilbert.}\label{Volt}

The   Volterra-type operator with general symbol $g\in \H(\D)$,
$$
T_g(f)(z)=\frac{1}{z}\int_0^zf(\z)g'(\z)\,d\z=\int_0^1f(tz)g'(tz)\,dt,
$$
was studied in \cite{AS1}, \cite{AS2} and in several other articles. The  generalized Hilbert operator
$$
H_g(f)(z)=\int_0^1f(t)g'(tz)\,dt
$$
was studied in \cite{GGPS}. The last  integral is finite for  $f\in H^1$ due to the Fejer-Riesz inequality. It is
also finite on other spaces such as on  Bergman $A^p$ for $p>2$, on Dirichlet-type spaces, etc.

Notice that if $g(z)=\sum_{n=0}^{\infty}b_nz^n$ then
$$
H_g(f)(z)=\sum_{n=0}^{\infty}\left((n+1)b_{n+1}\int_0^1f(t)t^n\,dt\right)z^n,
$$
so  if $g$ is a polynomial of degree $m$ then $H_g(f)$ is also a polynomial  of degree $\leq m-1$ for every $f$
(so $H_g$ is then of finite rank $m$). More generally $H_g(f)$ inherits any  gaps in the power series of $g$. Also
$$
T_g(z^n)(z)=z^n H_g(z^n)
$$
which means that  properties of the two operators are related. However there are basic differences. For example
the space of symbols for which $T_g$ is bounded on Hardy spaces $H^p$ is the same for all $p$ and is the space BMOA, while
if  $H_g$  is  bounded on $H^p$ then  $g$ belongs to  the mean Lipschitz space $\Lambda(p, \frac{1}{p})\subsetneq BMOA$
i.e. the space of symbols  depends on $p$. This  is proved in \cite{GGPS} for $1<p<\infty $.  For $1<p\leq 2$
 the converse is also true, that is if
$g\in\Lambda(p, \frac{1}{p})$ then $H_g$ is bounded on $H^p$ (for $p>2$ the converse is open).

 Analogous results hold for weighted Bergman spaces
$$
A^p_{\a}=L_a^p(\D, (1-|z|^2)^{\a}dA(z))
$$
and for the weighted Dirichlet spaces $\mathcal{D}^p_{\a}=\{f: f'\in A^p_{\a}\} $. The class of all functions $g$ for which $H_g$ is
bounded is, in both cases, the same  $\Lambda(p, \frac{1}{p})$ (with $-1<\a<p-2 $ for $A^p_{\a}$ and $ p-2 <\a \leq p-1 $
for $\mathcal{D}^p_{\a})$. Further $H_g$ is  Hilbert-Schmidt on  $A^2_{\a}$ or on  $\mathcal{D}^2_{\a}$ or on  $H^2$ if
and only if $g$ is in the classical Dirichlet space $\mathcal{D}$. The Schatten classes were studied in \cite{PS}, and the
resulting spaces for the symbols  $g$ for which $H_g$ is in the Schatten class $S^p$ are some kind  of mixed norm spaces.

Notice that the operator transformation $H\to C$ defined by   $ C(e_n)=S^nH(e_n) $ can be applied more generally,
 i.e. starting with a   bounded operator  $T:X\to X$, one may consider the operator $V$ defined on the
  basis $\{e_n\}$ by $V(e_n)=S^nT(e_n)$. For
example if  $h_{\mu}$ is a Hankel operator with matrix induced by a sequence $(\mu_n)$,
$$
h_{\mu}=\left(%
\begin{array}{ccccc}
            \mu_0 & \mu_1  & \mu_2  & .  \\
  \mu_1 & \mu_2  & \mu_3  & .  \\
  \mu_2 & \mu_3  & \mu_4  & . \\
  .  & . & . & .  \\
\end{array}%
\right)
$$
the operator obtained    $V_{\mu}(e_n) = S^n h_{\mu}(e_n) $ has   matrix
$$
V_{\mu}=\left(%
\begin{array}{ccccc}
            \mu_0 & 0  & 0  & .  \\
  \mu_1 & \mu_1  & 0  & .  \\
  \mu_2 & \mu_2  & \mu_2  & . \\
  .  & . & . & .  \\
\end{array}%
\right).
$$
These are the so called "terraced matrices" \cite{LE}, \cite{RA}. If the sequence $(\mu_n)$ inducing $h_{\mu}$
 is the moment sequence of a measure $\mu$, i.e.
$
\mu_n=\int_0^1t^n\,d\mu,
$
 then $V_{\mu}$ acts as
 $$
V_{\mu}(f)(z)=\int_0^1f(tz)\frac{1}{1-tz}\,d\mu(t).
 $$
It follows from the work of Widom on Hankel matrices that, if $\mu$ is Carleson i.e. $\mu((t, 1))=O(1-t)$ for
$t$ near $1$, then $V_{\mu}$ is bounded  on $H^2$.
%I am not aware of other work studying $C_{\mu}$.
There are some articles on terraced matrices  in which  the above connection to Hankel operators is not used or mentioned.


\section{Hilbert matrices from Hausdorff matrices}\label{Haus}

Let  $(\mu_n)$  be a  sequence and $\Delta\mu_n=\mu_n-\mu_{n+1}$ the forward
difference operator, with iterates
$$
\Delta^{k}\mu_n=\Delta(\Delta^{k-1}\mu_n),
\quad  k=1,2,\cdots, \quad \Delta^0\mu_n=\mu_n.
$$
 The Hausdorff matrix induced by $(\m_n)$ is
\begin{equation}
Ha_{\m}= \left(
\begin{matrix}
c_{0,0}& 0 & 0&  \cdots     \\
c_{1,0}& c_{1,1} &0& \cdots     \\
c_{2,0} & c_{2,1} & c_{2,2} & \cdots    \\
\vdots &\vdots  &\vdots     &\vdots
\end{matrix}
\right)
\notag
\end{equation}
with entries
\begin{equation}
c_{n, k}=\binom{n}{k}\Delta^{n-k}\mu_k, \quad     k\leq n.
\notag
\end{equation}
In the special case when  $\mu_n$ is the moment sequence
\begin{equation}
\mu_n=\int_0^1t^n\,d\mu(t), \qquad n=0,1,\cdots
\notag
\end{equation}
of a finite positive Borel measure, the entries of $Ha_{\m}$ are
\begin{equation}
c_{n, k}=\binom{n}{k}\int_0^1t^k(1-t)^{n-k}\,d\mu(t),
\qquad k\leq n.
\notag
\end{equation}
Hausdorff matrices were studied in connection to the classical summability theory on $\ell^p$ spaces.

% (I am not
%aware of  analogous study  on  the matrix $\G_{\m}$ acting on $l^p$)

If  $d\mu(t)=dt$ then $Ha_\m$ reduces to the Ces\'aro matrix.
Erasing the zeros and shifting up each column of $C$ to the first non-zero entry
gives the Hilbert matrix $H$. If we do the same on $Ha_\m$ then we obtain a  Hilbert-like matrix
$$
\G_{\m}=\left(%
\begin{array}{ccccc}
            \g_{00} & \g_{01}  & \g_{02}  & .  \\
\g_{10} & \g_{11}  & \g_{12}  & .  \\
  \g_{20} & \g_{21}  & \g_{22}  & . \\
  .  & . & . & .  \\
\end{array}%
\right)
$$
with entries
$$
\g_{n, k}=c_{n+k, k}=\binom{n+k}{k}\int_0^1t^k(1-t)^n\,d\m(t).
$$
Notice here that if $d\m(t)=dt$, then $\g_{n,k}=\frac{1}{n+k+1}$.


This matrix is not necessarily symmetric, and can be studied on spaces of sequences (for example $\ell^p$).
Here we  consider $\G_{\m}$ acting on spaces of analytic functions
on the unit disc, thus obtaining a formal power series transformation
$$
f(z)=\sum_{k=0}^{\infty} a_kz^k \rightarrow
\G_{\m}(f)(z)=\sum_{n=0}^{\infty}\left(\sum_{k=0}^{\infty}\g_{nk}a_k\right)z^n.
$$
If $f$ is a polynomial then the sum defining the coefficient
$A_n=\sum\limits_{k=0}^{\infty}\g_{nk}a_k$ of the transformed series is finite for each
$n$, so the series is well defined. But in general the finiteness of the coefficients $A_n$ and the convergence on $\D$
will depend on $\m$ and on $f$. The question of when $\G_{\m}$ is well defined can be split in two parts:
\begin{enumerate}
\item Given a measure $\m$  we can ask for which spaces  $X=X_{\m}\subset \mathcal{H}(\D)$  the power series $\G_{\m}(f)(z)$ defines an analytic function on $\D$ for each $f\in X$. \\

\item Given a space $X\subset \mathcal{H}(\D)$ we can ask for which measures $\m$ the power series
$\G_{\m}(f)(z)$ defines an analytic function on $\D$  for each $f\in X$.
\end{enumerate}


In order to study $\G_\m$, we want to find an integral representation of the operator. We will assume that $\mu$ and $f$ are such that the operator is well defined as an analytic function of $\mathbb{D}$ (for example when $f$ is a polynomial or a function whose Taylor coefficients converge rapidly to $0$). For the sake of simplicity, we omit the proof that all sums and integrals are interchangeable.
We then have:
\[
\begin{split}
\G_\mu(f)(z) &= \sum_{n\geq 0}\left(\sum_{k\geq 0}a_k\, \binom{n+k}{k}\int_0^1
(1-t)^n t^k\,d\mu(t)\right)\,z^n
\\
&=\int_0^1\,\sum_{k\geq 0} a_k t^k \, \sum_{n\geq 0} \binom{n+k}{k} (1-t)^n z^n\,d\mu(t)\,.
\end{split}
\]
Next, using the fact that
\[
\sum_{n\geq 0}\binom{n+b-1}{n}z^n=\dfrac{1}{(1-z)^b}\,,
\]
we get:
\[
\begin{split}
\G_\mu(f)(z) &=\int_0^1\,\sum_{k\geq 0} a_k t^k \, \sum_{n\geq 0} \binom{n+k}{n} [(1-t) z]^n\,d\mu(t)
\\
&=\int_0^1\,\sum_{k\geq 0} a_k t^k \dfrac{1}{(1-(1-t)z)^{k+1}}\,d\mu(t)
\\
&=\int_0^1\,\sum_{k\geq 0} a_k  \left(\dfrac{t}{(1-(1-t)z)}\right)^k\,\dfrac{1}{(1-(1-t)z)}\,d\mu(t)
\\
&=\int_0^1 f\left(\dfrac{t}{(1-(1-t)z)}\right)\,\dfrac{1}{(1-(1-t)z)}\,d\mu(t)\,.
\end{split}
\]
Hence we have that
\[
\G_\mu(f)(z)=\int_0^1 f(\phi_t(z))w_t(z)\,d\mu(t)=\,\int_0^1 T_t(f) \,d\mu(t)
\]
where $\phi_t(z)=\dfrac{t}{(1+(t-1)z)}\,$ and $\,w_t(z)=\dfrac{1}{(1+(t-1)z)}$.
In other words, $\G_{\m}$ is an average of weighted composition operators, as is the case with the classical Hilbert matrix operator,
but with the Lebesgue measure $dt$ replaced by $d\m(t)$. This formula can be
used to study the operator on spaces where the weighted composition operators
$$
T_t(f)(z)=\frac{1}{1-(1-t)z}f\left(\frac{t}{1-(1-t)z}\right)
$$
are well understood.

The obvious questions that arise are to determine the measures $\m$ for
which $\G_{\m}$ is bounded on Hardy, Bergman and other spaces, questions of compactness, etc.

From \cite{DS} and \cite{D} we know that
\[
\n{T_t(f)}_{H^p}\leq\, \dfrac{1}{t^{1-\frac{1}{p}}\,(1-t)^{\frac{1}{p}}} \;\n{f}_{H^p},\;\;p\geq 2\,,
\]
and
\[
\n{T_t(f)}_{A^p}=\dfrac{1}{t^{1-\frac{1}{p}}\,(1-t)^{\frac{1}{p}}}\int_{\phi_t(\mathbb{D})}\abs{w}^{p-4}\abs{f(w)}^p\, | dw |,\;\;p> 2\,.
\]
By an application of the generalized Minkowski's inequality, we get
\[
\n{\G_\mu(f)}_X \leq \int_0^1 \n{T_t(f)}_X\,d\mu(t)\,,
\]
where $X=H^p$ or $X=A^p$, hence it follows that

\[
\n{\G_\mu(f)}_{H^p}\leq \int_0^1\dfrac{1}{t^{1-\frac{1}{p}}\,(1-t)^{\frac{1}{p}}}\,d\mu(t)\;\n{f}_{H^p},\;\;p\geq 2\,,
\]
and
\[
\n{\G_\mu(f)}_{A^p}\leq \int_0^1\dfrac{1}{t^{1-\frac{1}{p}}\,(1-t)^{2/p}}\,d\mu(t)\;\n{f}_{A^p},\;\;p\geq 4\,.
\]



The cases $H^p,\;1<p<2$ and $A^p,\;2<p<4$ require further study. Maybe we can get the remaining Bergman case from \cite{BK}. One also needs to check if we can use the lower bounds for the norm of ${H}$, from \cite{DJV}.

  Because  $\G_{\m}$ is not symmetric for all measures $\m$, its transpose $\G_{\m}^{\ast}$
is in general another operator that also deserves to be studied.

\begin{remark}
 If $\G_{\m}: H^1\to H^1$ is bounded then
\[
\int_0^1\frac{1}{1-t}\log(\frac{1}{t})\,d\m(t)<\infty.
\]
\end{remark}
\begin{proof}
Indeed, for $f=1$ we have
\begin{align*}
\G_{\m}(1)(z)&=\int_0^1\frac{1}{1-(1-t)z}\,d\m(t)\\
&=\int_0^1\sum_{n=0}^{\infty}(1-t)^nz^n\,d\m(t)\\
&=\sum_{n=0}^{\infty}\left(\int_0^1(1-t)^n\,d\m(t)\right)z^n
\end{align*}
Then by Hardy's inequality we have
$$
\sum_{n=0}^{\infty}\frac{1}{n+1}\int_0^1(1-t)^n\,d\m(t)\leq \pi\n{\G_{\m}(1)}_{H^1}
$$
and thus
\begin{align*}
\int_0^1\sum_{n=0}^{\infty}\frac{1}{n+1}(1-t)^n\,d\m(t)&=\int_0^1\frac{1}{1-t}\log\frac{1}{1-(1-t)}d\m(t)\\
&=\int_0^1\frac{1}{1-t}\log(\frac{1}{t})\,d\m(t)<\infty.
\end{align*}
\end{proof}



\begin{remark}
 If $\G_{\m}: H^1\to H^1$ is bounded then
\[
\int_0^1\frac{1}{1-t}\log(\frac{1}{t})\,d\m(t)<\infty.
\]
\end{remark}





\section{Spectral radius of $T_{t}$}

\textbf{Some comments on} $\varphi_t$:\\

\noindent i) We have that
\begin{equation}\label{imagephi}
   \varphi_{t}(\mathbb{D})=D(\frac{1}{2-t},\frac{1-t}{2-t}).
\end{equation}

\noindent ii) For $0<t<1/2$,  $\varphi_{t}$ has $z=\frac{t}{1-t}\in \mathbb{D}$ as a DW fixed point.\\

\noindent iii) For $1/2\leq t<1$,  $\varphi_{t}$ has $z=1$ as a DW fixed point, $0<\varphi_{t}'(1)=\frac{1}{t}-1<1$.\\

\noindent iv)  $\varphi_{t}'(z)=\frac{1-t}{t}\cdot(\varphi_{t}(z))^{2}$.\\

Let
$$
C_{\varphi_{t}}(f)(z) =f\left(\frac{t}{(t-1)z + 1}\right)
$$
then $T_{t}(f)(z)=\frac{1}{t}C_{\varphi_{t}}(Sf)(z)$ where $S$ is the shift operator.


The spectral radius of a bounded linear operator $A$ is the supremum of the absolute values of the elements of its spectrum.
The spectral radius is often denoted by $\rho(\cdot)$.  Gelfand's formula, also known as the spectral radius formula, also holds for bounded linear operators: letting $ \|\cdot \|$ denote the operator norm, we have

\[ \rho (A)=\lim _{k\to \infty }\|A^{k}\|^{\frac {1}{k}}=\inf _{k\in \mathbb {N} }\|A^{k}\|^{\frac {1}{k}}.\]

It is easy to see that $\rho(T_{t})=\frac{1}{t}\rho(\varphi_{t}C_{\varphi_{t}})$. We have that


$$
(\varphi_{t}C_{\varphi_{t}})^{n}(f)=\left(\prod_{i=1}^{n}\varphi_{t,i}\right)\cdot f\circ\varphi_{t,n}
$$
where $\varphi_{t,n}=\varphi_{t}\circ...\circ\varphi_{t}$, $n-$ times. Set $\prod_{i=1}^{n}\varphi_{t,i}=u_{t,n}$



Let $0<t<1/2$ then $\frac{t}{1-t}$ is the D.W point. We have for $w\in\mathbb{D}$


\begin{align*}
|u_{t,n}(w)|^{\frac{p}{2}}||K_{\varphi_{t,n}(w)}||_{2}&=||\overline{\left(u_{t,n}(w)^{\frac{p}{2}}\right)}\cdot K_{\varphi_{t,n}(w)}||_{2}\\
&=\left\|\left(u_{t,n}^{\frac{p}{2}}\cdot C_{\varphi_{t,n}}\right)^{*}(K_{w})\right\|\\
&\leq\left\|\left(u_{t,n}^{\frac{p}{2}}\cdot C_{\varphi_{t,n}}\right)^{*}\right\|_{2} \left\|K_{w}\right\|_{2}\\
&=\left\|u_{t,n}^{\frac{p}{2}}\cdot C_{\varphi_{t,n}}\right\|_{2} \left\|K_{w}\right\|_{2}\\
\end{align*}




%Thus for $w=t/(1-t)$ we have $\varphi_{t,n}(t/(1-t))=t/(1-t)$ and

Thus for $0<w=r<1$ we have



%\begin{align*}
%\left\|u_{t,n}^{\frac{p}{2}}\cdot C_{\varphi_{t,n}}\right\|_{2}
%&\geq |u_{t,n}(\frac{t}{1-t})|^{p/2}\\
%&=|u_{t,n}(\frac{t}{1-t})|\\
%&=(\frac{t}{1-t})^{pn/2}.
%\end{align*}


\begin{align*}
|u_{t,n}(r)|^{\frac{p}{2}}\frac{||K_{\varphi_{t,n}(r)}||_{2}}{\left\|K_{r}\right\|_{2}}
\leq \left\|u_{t,n}^{\frac{p}{2}}\cdot C_{\varphi_{t,n}}\right\|_{2}
\end{align*}

Therefore
\begin{align*}
\rho_{H^{p}}(\varphi_{t}C_{\varphi_{t}})&=\lim_{n\to\infty}\left\|u_{t,n}\cdot C_{\varphi_{t,n}}\right\|_{p}^{1/n}\\
 &=\lim_{n\to\infty}\left\|u_{t,n}^{p/2}\cdot C_{\varphi_{t,n}}\right\|_{2}^{\frac{1}{p}n}\\
&\geq1.
\end{align*}
This, since $||\varphi_{t}C_{\varphi_{t}}||_{p}\leq ||C_{\varphi_{t}}||_{p}$ implies that  $\rho_{H^{p}}(\varphi_{t}C_{\varphi_{t}})=1$ and thus

$$
\frac{1}{t}=\rho_{H^{p}}(T_{t}),\quad 0<t<1/2.
$$






\begin{thm}\cite[Theorem 6]{HyNi15}
Let $1/2<t<1$ and $\mathcal{A}$ is either $H^p$ or $A^p$, then
$$
\sigma_{e}(\varphi_{t}C_{\varphi_{t}})=\sigma(\varphi_{t}C_{\varphi_{t}})=\{\lambda\in\mathbb{C}: |\lambda|\leq \varphi_{t}'(1)^{-s}\}
$$
where $s=\frac{1}{p} $ for $H^p$ and $s=2/p$ for $A^p$.
\end{thm}

The above Theorem implies that

\begin{equation}\label{spectral rad Hp DW1}
\rho_{H^{p}}(T_{t})= \frac{t^{\frac{1}{p}-1}}{(1-t)^{\frac{1}{p}}},\quad 1/2\leq t<1
\end{equation}
and
\begin{equation}\label{spectral rad Ap DW1}
\rho_{A^{p}}(T_{t})= \frac{t^{\frac{2}{p}-1}}{(1-t)^{2/p}},\quad 1/2\leq t<1.
\end{equation}

\newpage
%%%%%%%%%%%%%%%%%%%%%%%%%%%%%%%%%%%%%%%%%%%%%%%%%%%%%%%%%%%%%%%%%%%%%%%%%%%%%%%%%%%%%%%%%%%%%%%%%%%%%%%%%%%%%%%%%%%%%%%%%%%%%%%%%%%%
\bibliographystyle{amsalpha}
\begin{thebibliography}{BPSSV}

\bibitem[AS1]{AS1} A.  Aleman, A. G.  Siskakis, \textit{An integral operator on $H^p$},
 Complex Variables Theory Appl. 28 (1995), 149--158.
\\
\bibitem[AS2] {AS2}  A.  Aleman, A. G. Siskakis, \textit{Integration operators on Bergman spaces}. Indiana Univ. Math. J. 46 (1997),
 337--356.
\\

\bibitem[DS]{DS} E. Diamantopoulos, A. G.  Siskakis, \textit{Composition operators and the
Hilbert matrix},  Studia Math. 140 (2000), 191--198.
\\
\bibitem[D]{D} E. Diamantopoulos, \textit{Hilbert matrix on Bergman spaces},  Illinois J. Math. 48 (2004),  1067--1078.
\\
\bibitem[DJV]{DJV}  M. Dostanic, M.  Jevtic, D.  Vukotic, \textit{Norm of the Hilbert matrix on Bergman and
    Hardy     spaces     and a theorem of Nehari type},  J. Funct. Anal. 254 (2008), 2800--2815
\\
\bibitem[GGPS]{GGPS} P. Galanopoulos, D.  Girela,  H. A. Pel\'aez,
A. G. Siskakis, \textit{Generalized Hilbert operators},  Ann. Acad. Sci. Fenn. Math. 39 (2014), 1, 231--258.
\\
\bibitem[BK]{BK}
V. Bo\v zin - B. Karapetrovi\'c, \emph{Norm of the Hilbert matrix on Bergman spaces}, J. Funct. Anal. 274 (2018), 525 - 543
\\
\bibitem[LE]{LE}  G. Leibowitz, \textit{Rhaly matrices}, J. Math. Anal. Appl. 128 (1987),  272--286.
\\
\bibitem[MA]{MA}
W. Magnus, On the spectrum of Hilbert's matrix, \textit{Amer. J. Math.\/} \textbf{72} (1950), 699--704.
\\
\bibitem[PS]{PS}  J. A. Pelaez, D.  Seco, \textit{Schatten classes of generalized Hilbert operators}, Collect. Math. 69 (2018),  83--105.
\\
\bibitem[RA]{RA}  H. C. Rhaly Jr. \textit{Terraced matrices},  Bull. London Math. Soc. 21 (1989),  399--406.
\\
\bibitem[S]{S} A. G.  Siskakis, \textit{Volterra operators on spaces of analytic functions, a survey}.
Proceedings of the First Advanced Course in Operator Theory and Complex Analysis, 51--68, Univ. Sevilla Secr. Publ., Seville, 2006.
\\
\bibitem[SZ]{SZ}   A. G. Siskakis, R.  Zhao, \textit{A Volterra type operator on spaces of analytic functions},
Function spaces (Edwardsville, IL, 1998), 299--311, Contemp. Math., 232, AMS, 1999.\\

\bibitem[AlCi01]{AlCi01} Aleman, Alexandru; Cima, Joseph A. An integral operator on Hp and Hardy's inequality, Journal d'Analyse Mathematique,  2001, 157-176.\\

\bibitem[Akh65]{Akh65} N. Akhiezer, The classical moment problem, oliver and Boyd LDT, 1965.\\

%\bibitem[5]{CwKal} M. Cwikel and N. J. Kalton, Interpolation of compact operators by the method of Calder\'on and Gustavsson-Peetre. Proceedings  of  the  Edinburgh  Mathematical  Society  (1995)  38,  261-276.

\bibitem[DiSi00]{DiSi00} E. Diamantopoulos, A. G. Siskakis, Composition operators and the Hilbert matrix, Studia Math. 140 (2000), 191-198.\\

\bibitem[Ga28]{Ga28}    Gabriel, R. M.; Some Results Concerning the Integrals of Moduli of Regular Functions Along Curves of Certain Types. Proc. London Math. Soc. (2) 28 (1928), no. 2, 121-127.\\

\bibitem[HyNi15]{HyNi15} Hyv$\ddot{a}$rinen, O., Nieminen, I.: Essential spectra of weighted composition operators with hyperbolic symbols.
Concr. Oper. 2, 110-119 (2015)\\


\bibitem[Pav12]{Pav12} M. Pavlovic, Analytic functions with decreasing coefficients and Hardy and Bloch spaces, Proceedings of the Edinburgh Mathematical Society, (2) 56,  (2013), 623-635.\\
\end{thebibliography}

{\color{red}By an application of the generalized Minkowski's inequality, we get
\[
\n{\G_\mu(f)}_{H^p} \leq \int_0^1 \n{T_t(f)}_{H^p}\,d\mu(t)\,,
\]
hence it follows that
\begin{equation}\label{norma}
\n{\G_\mu(f)}_{H^p}\leq \int_0^1\dfrac{1}{t^{1-\frac{1}{p}}\,(1-t)^{\frac{1}{p}}}\,d\mu(t)\;\n{f}_{H^p},\;\;p\geq 2\,.
\end{equation}The case $H^p,\;1<p<2$ requires further study. It is worth noting that the integral in \eqref{norma} is equal to the constant $\frac{\pi}{\sin(\pi/p)}$ when $d\mu=dt$.}
