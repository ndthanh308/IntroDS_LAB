\documentclass[12pt]{amsart}
\usepackage{amssymb, amscd, amsmath, amsthm, latexsym, enumerate}

\renewcommand{\geq}{\geqslant}
\renewcommand{\leq}{\leqslant}

\newcommand{\g}{\Gamma}
\newcommand{\F}{\mathbb F}
\newcommand{\R}{\mathbb R}
\newcommand{\Z}{\mathbb Z}

\newcommand{\ol}[1]{{\overline{#1}}}

\newtheorem{theorem}{Theorem}
\newtheorem{lemma}[theorem]{Lemma}
\newtheorem{cor}[theorem]{Corollary}

\newtheorem*{thm}{Theorem}
\newtheorem*{lem}{Lemma}
\newtheorem*{cor*}{Corollary}
\newtheorem*{add}{Addendum}

\begin{document}
\title[4-manifolds with 3-manifold fundamental groups]
{Homotopy types of 4-manifolds with 3-manifold fundamental groups}

\author{Jonathan A. Hillman }
\address{School of Mathematics and Statistics\\
     University of Sydney, NSW 2006\\
      Australia }

\email{jonathanhillman47@gmail.com}

\begin{abstract}
We show that the homotopy type of a
4-manifold $M$ whose fundamental group is a finitely
presentable $PD_3$-group $\pi$,
and with  $w_1(M)=w_1(\pi)$ is determined by $\pi$, 
$\pi_2(M)$,  $k_1(M)$ and the equivariant intersection pairing $\lambda_M$.
\end{abstract}

\keywords{4-manifold, homotopy type, $PD_3$-group, $PD_4$-complex}

%\subjclass{57K41, 57P10}


\maketitle
The basic algebraic invariants of a closed 4-manifold $M$ are the fundamental group 
$\pi=\pi_1(M)$,  the $\mathbb{Z}[\pi]$-module $\Pi=\pi_2(M)$,
the equivariant homotopy intersection pairing $\lambda_M$ on $\Pi$,
the first $k$-invariant $\kappa=k_1(M)\in{H^3(\pi;\Pi)}$,
the Euler characteristic $\chi(M)$,  
and the Stiefel-Whitney classes $w=w_1(M)$ and $w_2(M)$.
(Strictly speaking, $\lambda_X$ determines $\Pi$,
and it also determines $w$ if $\pi\not=0$.)
These invariants determine the stable homeomorphism type of $M$
(with respect to sums with $S^2\times{S^2}$), 
if $\pi$ is the group of an aspherical closed orientable 3-manifold \cite{KLPT}.
(The $k$-invariant is determined by the other data in this situation.)

We shall show that the homotopy type of a $PD_4$-complex $X$ 
whose fundamental group is a finitely presentable $PD_3$-group $\pi$,
and with  $w_1(X)=w_1(\pi)$ is determined by the  invariants
$\pi$, $\Pi$, $\kappa$ and $\lambda_X$.
(However we do not yet know the possible values of $\kappa$ or $\lambda_X$.)
This suggests an alternative approach to some of the results of \cite{KLPT},
and also leads to an unstabilized result when $\pi$ is solvable.

\section{Notation and terminology}

We shall assume throughout that $X$ is a $PD_4$-complex such that $\pi=\pi_1(X)$ 
is a finitely presentable $PD_3$-group and $w_1(X)=w_1(\pi)$.
We shall write $w=w_1(\pi)$, for simplicity.
Let $p:\widetilde{X}\to{X}$ be the universal covering.
The homology of $\widetilde{X}$ is given by 
$H_i(\widetilde{X};\mathbb{Z})=H_i(X;\mathbb{Z}[\pi])$, for all $i$.
We assume that $\pi$ acts on the left of $\widetilde{X}$,
and so these are left $\mathbb{Z}[\pi]$-modules.
Since $\pi$ has one end, $H_i(X;\mathbb{Z}[\pi])=0$ if $i\not=0,2$.
Therefore $\Pi={H_2(X;\mathbb{Z}[\pi])}$ is not 0, 
since $\pi$ is not a $PD_4$-group.

The homologies of $\widetilde{X}$ and $X$ 
are related by the Cartan-Leray spectral sequence for the covering, 
which has the form
\[
E_{p,q}^2=H_p(\pi;H_q(X;\mathbb{Z}[\pi]))\Rightarrow{H_{p+q}(X;\mathbb{Z})}.
\]
(Note that the groups $H_*(\pi;A)=Tor_*^{\mathbb{Z}[\pi]}(\mathbb{Z},A)$
 with coefficients in a left module $A$ are defined via a resolution of $\mathbb{Z}$ by 
 {\it right\/} $\mathbb{Z}[\pi]$-modules.)
 There is also a Universal Coefficient spectral sequence
 \[
 E_2^{p,q}=Ext^q_{\mathbb{Z}[\pi]}(H_p(X;\mathbb{Z}[\pi]), \mathbb{Z}[\pi])
 \Rightarrow{H^{p+q}(X;\mathbb{Z}[\pi])},
 \]
 which relates the homology and equivariant cohomology of the universal cover.
 
Let $\varepsilon:\mathbb{Z}[\pi]\to\mathbb{Z}$ be the augmentation homomorphism
and $I_\pi=\mathrm{Ker}(\varepsilon)$ be the augmentation ideal.
Clearly $H^0(\pi;I_\pi)=0$.
Applying the fixed point functor $Hom_{\mathbb{Z}[\pi]}(\mathbb{Z},-)$ 
to the augmentation exact sequence
\begin{equation*}
\begin{CD}
0\to{I_\pi}\to\mathbb{Z}[\pi]@>\varepsilon>>\mathbb{Z}\to0
\end{CD}
\end{equation*}
gives $H^i(\pi;\mathbb{Z})\cong{H^{i+1}(\pi;I_\pi)}~\mathrm{for}~i\leq2$,
while applying the functor $Hom_{\mathbb{Z}[\pi]}(-,\mathbb{Z}[\pi])$ gives
$Hom_{\mathbb{Z}[\pi]}(I_\pi,\mathbb{Z}[\pi])\cong\mathbb{Z}[\pi]$ and
$Ext^i_{\mathbb{Z}[\pi]}(I_\pi,\mathbb{Z}[\pi])\cong{H^{i+1}(\pi;\mathbb{Z}[\pi])},
~\mathrm{for}~ i>0$.
(In fact every homomorphism from $I_\pi$ to $\mathbb{Z}[\pi]$ is the restriction 
of right multiplication by an element of $\mathbb{Z}[\pi]$.)

If $M$ is a left $\mathbb{Z}[\pi]$-module let $M^w=\mathbb{Z}^w\otimes{M}$ 
be the left $\mathbb{Z}[\pi]$-module with the same underlying abelian group 
and diagonal left $\pi$-action,
given by $g(1\times{x})=w(g)(1\times{gx})$ for all $g\in\pi$ and $x\in\Pi$.
Then $\mathbb{Z}[\pi]^w\cong\mathbb{Z}[\pi]$, 
since we may define an isomorphism  
$f:\mathbb{Z}[\pi]\to\mathbb{Z}[\pi]^w$ by $f(g)=w(g)\otimes{g}$ for all $g\in\pi$.
The linear extension of $w$ defines the {\it$w$-twisted augmentation\/} 
$\varepsilon_w:\mathbb{Z}[\pi]\to\mathbb{Z}^w$,
with kernel $I_\pi^w$.
Arguments similar to those of the previous paragraph apply to $I_\pi^w$.

If $R$ is a right $\mathbb{Z}[\pi]$-module let $\overline{R}$ be the left module
with the same underlying group and $\mathbb{Z}[\pi]$-action determined 
by $g.r=w(g)rg$ for all $r\in{R}$  and $g\in\pi$.
We use a similar strategy and notation to obtain a right module $\overline{L}$ 
from a left $\mathbb{Z}[\pi]$-module $L$.
Free right modules give rise to free left modules of the same rank, and conversely.
We may define the dual of a left module $M$ as the left module
$M^\dagger=\overline{Hom_{\mathbb{Z}[\pi]}(M,\mathbb{Z}[\pi])}$.

Two (left) $\mathbb{Z}[\pi]$-modules $L$ and $L'$ are {\it stably projective equivalent\/} 
if $L\oplus{P}\cong{L'}\oplus{P'}$ for some finitely generated projective 
$\mathbb{Z}[\pi]$modules $P,P'$.
They are {\it stably equivalent\/} if we may assume that $P$ and $P'$ are
each free modules.
We shall let $[L]_{pr}$ and $[L]$ denote the equivalence classes 
corresponding to these two equivalence relations.
As our concern in this paper is mainly with finite $PD_4$-complexes, 
which correspond most closely to manifolds, 
stable equivalence is the more useful notion.
However our arguments apply with little change to the more general setting 
of finitely dominated $PD_4$-complexes
(and even $PD_4$-spaces in the sense of \cite{Hi20a}),
for which the broader notion of stably projective equivalence is needed.

If $A$ is an abelian group let $rk(A)=\dim_\mathbb{Q}\mathbb{Q}\otimes{A}$ be its rank.

\section{the basic examples}

Let $Y=K(\pi,1)$. 
We may assume that $Y=Y_o\cup{D^3}$, where $(Y_o,S^2)$  is a $PD_3$-pair
and $Y_o$ is cohomologically 2-dimensional.
Let $\tau$ be a self-homeomorphism of $S^2\times{S^1}$
which does not extend over $S^2\times{D^2}$.
(There is an unique isotopy class of such maps $\tau$.)
Then $X(\pi)=(Y_o\times{S^1})\cup{S^2}\times{D^2}$
and  $X(\pi)_\tau=(Y_o\times{S^1})\cup_\tau{S^2}\times{D^2}$
are $PD_4$-complexes with fundamental group $\pi$,
orientation character $w=w_1(Y)$ and  Euler characteristic 2.
Since $X(\pi)=\partial(Y_o\times{D^2})$, it retracts onto $Y_o$.

The arguments of \cite[\S2]{Pl86} for the case when $Y$ is a 3-manifold 
are essentially homological and apply equally well in our situation.
Let  $U=Y_o\times{S^1}$.
The long exact sequence of the pair $(X(\pi),U)$  
with coefficients $\mathbb{Z}[\pi]$ gives a five-term exact sequence
\[
H_3(X(\pi),U;\mathbb{Z}[\pi])\to{H_2(U;\mathbb{Z}[\pi])}\to{H_2(X(\pi);\mathbb{Z}[\pi])}\to
\]
\[{H_2(X(\pi),U;\mathbb{Z}[\pi])}\to{H_1(U;\mathbb{Z}[\pi])}\to0,
\]
since $H_i(X;\mathbb{Z}[\pi])=0$ for $i\not=0, 2$.
Now $H_3(X(\pi),U;\mathbb{Z}[\pi])=0$ and 
$H_2(X(\pi),U;\mathbb{Z}[\pi])\cong\mathbb{Z}[\pi]$, 
by excision, 
while\\ $H_2(U;\mathbb{Z}[\pi])\cong{H_2(Y_o;\mathbb{Z}[\pi])}(\cong
\pi_2(Y_o))\cong\mathbb{Z}[\pi]$ and $H_1(U;\mathbb{Z}[\pi])\cong\mathbb{Z}$.
Hence this sequence reduces to
\[
0\to\pi_2(Y_o)\cong\mathbb{Z}[\pi]\to\pi_2(X(\pi))\to\mathbb{Z}[\pi]\to\mathbb{Z}\to0.
\]
Hence $\pi_2(X(\pi))$ is an extension of $I_\pi$ by $\pi_2(Y_o)\cong\mathbb{Z}[\pi]$.
The extension splits,
since  $Ext^1_{\mathbb{Z}[\pi]}(I_\pi,\mathbb{Z}[\pi])=H^2(\pi;\mathbb{Z}[\pi])=0$,
and so  $\pi_2(X(\pi))\cong{\mathbb{Z}[\pi]\oplus{I_\pi}}$.
The retraction of $X(\pi)$ onto $Y_o$ determines a splitting.

A similar argument shows that 
$\pi_2(X(\pi)_\tau)\cong{\mathbb{Z}[\pi]\oplus{I_\pi}}$ also,
but we do not know whether $X(\pi)_\tau$ retracts onto $Y_o$.

If $\pi$ has a balanced presentation then 
there is a closed 4-manifold with $\pi_1(M)\cong\pi$ and $\chi(M)=2$.
Let $K$ be a finite 2-complex with $\pi_1(K)\cong\pi$ and $\chi(K)=1$
and let $N$ be a 4-dimensional handlebody thickening of $K$.
Then the double of $N$ is a closed 4-manifold $M$ with 
$\pi_1(M)\cong\pi$ and $\chi(M)=2$, and which retracts onto $K$.

If $Y$ is a closed 3-manifold then the corresponding closed 4-manifolds
are the manifolds obtained by elementary surgery on the second factor
of $Y\times{S^1}$. 
(There are two possible framings of the normal bundle.)
Plotnick uses manifold topology, 
first to define a splitting of the above exact sequence and then 
to show that the image of $\pi_2(Y_o)$ in $\pi_2(X)$ is self-annihilating,
with respect to the  equivariant homotopy intersection pairing $\lambda_X$,
in either case, 
and that $\pi_2(X(\pi))$ is the direct sum of two self-annihilating summands.
The equivariant homotopy intersection pairings of these 4-manifolds 
are not isometric \cite[Theorem 3.1]{Pl86}.


\section{$\Pi$, $\chi$ and $\lambda_X$}

In this section we shall summarize the key properties of $\Pi$ and $\chi$,
which were determined in \cite[Theorem 3.13]{FMGK}, 
and define the equivariant intersection pairing,
using the cohomological formulation.


We shall first state without proof a result from \cite{FMGK}.
\begin{thm}
\cite[Theorem 3.13]{FMGK}
Let $X$ be a $PD_4$-complex such that $\pi=\pi_1(X)$ is a finitely presentable
$PD_3$-group and $w_1(X)=w_1(\pi)$.
Then $\chi(X)\geq2$ and $[\Pi]_{pr}=[I_\pi]_{pr}$.
\qed
\end{thm}

The Euler characteristic is in fact determined by $\pi$ and $\Pi$.
This follows easily from the invariance of $\chi$ between the pages of a spectral sequence, 
and the special nature of $\Pi$.

\begin{cor}
 $\chi(X)=rk(\Pi)+1-\beta_1(\pi)$.
\end{cor}

\begin{proof}
The only nonzero entries in the Cartan-Leray homology spectral sequence for the universal cover of $X$ are when $0\leq{p}\leq3$ and $q=0$ or 2,
and then $E_{p,0}^2=H_p(\pi;\mathbb{Z})$,
while $E_{0,2}^2=\mathbb{Z}\otimes_{\mathbb{Z}[\pi]}\Pi$ and
$E_{p,2}^2=H_{p+1}(\pi;\mathbb{Z})$ for $p>0$, since $[\Pi]_{pr}=[I_\pi]_{pr}$.
Since $\chi(X)=\Sigma_{p,q}(-1)^{p+q}rk(E_{p,q}^2)$, the corollary follows.
\end{proof}

If $\chi(X)=2$ then $\Pi$ is stably projective equivalent to $\mathbb{Z}[\pi]\oplus{I_\pi}$.
Hence $H^3(\pi;\Pi)\cong\mathbb{Z}\oplus{H_1(\pi;\mathbb{Z}^w)}$,
since $H^3(\pi;\mathbb{Z}[\pi])\cong\mathbb{Z}$
and $H^3(\pi;I_\pi)\cong{H^2(\pi;\mathbb{Z})}\cong{H_1(\pi;\mathbb{Z}^w)}$,
by Poincar\'e duality.
If $w=1$ this is just $\pi^{ab}$.

If $X$ is a closed 4-manifold, $\pi=\pi_1(X)$ and $w=w_1(X)$ 
then geometric intersection numbers can be used to define a
$w$-hermitean equivariant intersection pairing on $\Pi$, 
with values in $\mathbb{Z}[\pi]$.
In the Poincar\'e duality complex case we cannot count geometric intersection numbers
and so weshall use the cohomological formulation of the intersection pairing instead.
In fact this cohomological formulation is well suited to application of
\cite[Theorem 2]{Hi20} in Theorem \ref{mainthm} below.

\begin{lemma}
\label{evalseq}
There is an exact sequence of left $\mathbb{Z}[\pi]$-modules
\begin{equation*}
\begin{CD}
0\to\overline{H^2(X;\mathbb{Z}[\pi])}@>{ev}>>\Pi^\dagger\to\mathbb{Z}\to0.
\end{CD}
\end{equation*}
 \end{lemma}
 
 \begin{proof}
 This follows from the Universal Coefficient spectral sequence, 
 since $H^2(\pi;\mathbb{Z}[\pi])=0$, 
 $H^2(\pi;\mathbb{Z}[\pi])\cong\mathbb{Z}$
 and 
 $\Pi\cong{H_2(X;\mathbb{Z}[\pi])}\cong\overline{H^2(X;\mathbb{Z}[\pi])}$,
 by Poincar\'e duality.
 (See  \cite[Lemma 3.3]{FMGK}.)
 \end{proof}
 
Let $D:\Pi\to{H^2(X;\mathbb{Z}[\pi])}$ be the isomorphism given by Poincar\'e duality.
Then the intersection pairing may be defined by
\[
\lambda_X(u,v)=ev(v)([X]\cap{u}),\quad\forall~u,v\in{H^2(X;\mathbb{Z}[\pi])}.
\]
(See \cite[Proposition 4.58]{Ra}.)
Then $\lambda_X(gu,gv)=w(g)\lambda_X(u,v)$ for all $g\in\pi$ 
and $u,v\in{H^2(X;\mathbb{Z}[\pi])}$.
Since $\Pi\not=0$ and $ev$ is a monomorphism, $\lambda_X$ is non-zero,
and so $w$ is determined by $\lambda_X$.
If $r:X\to{Y_o}$ is a retraction then $r_*[X]=0$, 
and so $r^*H^2(Y_o;\mathbb{Z}[\pi])\cong\mathbb{Z}[\pi]$
is a direct summand of $\Pi$ which is self-annihilating under $\lambda_X$.

It is clear from the argument in \cite[Theorem 3.13]{FMGK}
that if $X$ is finite and $\pi$ is of type $FF$ then  
$\Pi\oplus\mathbb{Z}[\pi]^r\cong
\mathbb{Z}[\pi]^{\chi(X)-1}\oplus{I_\pi}\oplus\mathbb{Z}[\pi]^r$ 
for $r$ large, and so $[\Pi]=[I_\pi]$.
The minimal value $\chi(X)=2$ is realized by the complexes $X(\pi)$ and $X(\pi)_\tau$ 
defined above.

In the 3-manifold group case $\widetilde{K}_0(\mathbb{Z}[\pi])=0$,
by work of Farrell and Jones, 
anticipating the Geometrization Theorem \cite{FJ87}.
In this case we may again  assume that $[\Pi]=[I_\pi]$,
and some of our statements can be simplified.
We shall assume that $\widetilde{K}_0(\mathbb{Z}[\pi])=0$ wherever convenient.
However even with this assumption there may be difficulties.
If $\pi$ is polycyclic but not abelian then there are ideals $J<\mathbb{Z}[\pi]$ 
which are not free but for which $\mathbb{Z}[\pi]\oplus{J}\cong\mathbb{Z}[\pi]^2$ \cite{Ar81}.

The case $\pi=\mathbb{Z}^3$ is exceptional.
All projectives are then free.
In this case $\Pi^\dagger$ is stably free, hence free, 
and it follows from Lemma \ref{evalseq}
that $\Pi\cong\mathbb{Z}[\pi]^{\chi(X)-1}\oplus{I_\pi}$.
Moreover,  if $K$ is any finite 2-complex with $\pi_1(K)\cong\mathbb{Z}^3$ and
$\chi(K)=1$ then $\pi_2(K)$ is free of rank 1, and so $K\simeq{T^3_o}$.

\section{$k_1$ and retractions onto $Y_o$}

The first $k$-invariant is an element of $H^3(\pi;\Pi)$,
and is well-defined up to the actions of $Aut(\pi)$ and $Aut_\pi(\Pi)$.
If $Z$ is a cell complex we may assume that the Postnikov 2-stage $f_2(Z):Z\to{P_2(Z)}$ 
is an inclusion, and that $P_2(Z)$ is obtained from $Z$ by adding cells of dimension $\geq4$.

\begin{lemma}
\label{pdnk}
Let $G$ be a $PD_n$-group and let $C_*$ be a projective resolution
of the augmentation module $\mathbb{Z}$ of length $n$ 
such that $C_n\cong\mathbb{Z}[G]$.
Then the class $[C_*]$ of $C_*$ in 
$H^n(G;C_n)=Ext^n_{\mathbb{Z}[G]}(\mathbb{Z},C_n)\cong\mathbb{Z}$ is a generator.
\end{lemma}

\begin{proof}
Let $D_*$ be the chain complex with $D_i=C_i$ for $i<n$
and $D_i=0$ for $i\geq{n}$,
and let $E_*$ be the chain complex with $E_i=D_i$ for $i\not=n-1$ and 
$E_{n-1}=C_{n-1}/Z_{n-1}\oplus{C_n}$.
Then $D_*$ and $E_*$ are of type $(\mathbb{Z},0, C_n, n-1)$ 
as defined in \cite[Definition 7.1]{Do60}.
The $k$-invariant of $D_*$ is represented by the class  $[C_*]$, 
while the $k$-invariant of $E_*$ is 0.

If $[C_*]=0$ then there is a c.h.e, $f:D_*\to{E_*}$ \cite[Satz 7.6]{Do60},
and since $H_{n-1}(f)$ is an isomorphism we see that $C_*$ is chain homotopy 
equivalent to a sequence in which $\partial_n$ is a split injection.
Hence $H^n(C_*;\mathbb{Z}[G])=0$, 
contrary to hypothesis.

Similarly, if $[C_*]$ is a $p$-fold multiple of some other class for some prime $p$
then $H^n(C_*;\mathbb{F}_p[G])=0$,  again contrary to hypothesis.
Therefore $[C_*]$ is indivisible, and so is a generator of 
$H^n(G;\mathbb{Z}[G])\cong\mathbb{Z}$.
\end{proof}

\begin{theorem}
Let $X$, $Y$ and $\pi$ be as in \S1 and \S2  above.
Then 
\begin{enumerate}
\item$P_2(X)$ retracts onto $P_2(Y_o)$ if and only if $\Pi\cong\mathbb{Z}[\pi]\oplus{L}$,  
for some $L$,
and the image of $k_1(X)$ in $H^3(\pi;\mathbb{Z}[\pi])$ is a generator;
\item{}if $X'$ is another $PD_4$-complex with $\pi_1(X')\cong\pi$, $\pi_2(X')\cong\Pi$
and which retracts onto $Y_o$ then $P_2(X')\simeq{P_2(X)}$.
\end{enumerate}
\end{theorem}

\begin{proof}
If $P_2(X)$ retracts onto $P_2(Y_o)$ then there is a pair of maps
 $j:P_2(Y_o)\to{P_2(X)}$ and $r:P_2(X)\to{P_2(Y_o)}$ such that 
$rj\sim{id_{Y_o}}$.
It follows immediately that $\pi_2(Y_o)\cong\mathbb{Z}[\pi]$ 
is a direct summand of $\Pi$, and that $j^*k_1(X)=k_1(Y_o)$,
up to the action of automorphisms.
The chain complex $C_*(Y_o;\mathbb{Z}[\pi])$ is chain homotopy equivalent to
a finite projective complex $D_*$ with $D_i=0$ for $i>2$.
The complex $C_*$ with $C_i=D_i$ for $i\not=3$ and $C_3=H_2(D_*)\cong\pi_2(Y_o)$
is a projective resolution of $\mathbb{Z}$, 
and $k_1(Y_o)$ is the class of $C_*$ in $H^3(\pi;\mathbb{Z}[\pi])$.
Hence $k_1(Y_o)$ 
is a generator of $H^3(\pi;\mathbb{Z}[\pi])$,
by Lemma \ref{pdnk}.

Conversely, if the conditions hold then there are morphisms between the algebraic 2-types
$[\pi,\Pi,k_1(X)]$ and $[\pi,\pi_2(Y_o),k_1(Y_o)]$ which can be realized by maps defining
a retraction.

Suppose now that $\Pi\cong\mathbb{Z}[\pi]\oplus{L}$,
for some $\mathbb{Z}[\pi]$-module $L$,
and let $e$ generate the first summand. 
If $k$ and $k'$ in $H^3(\pi;\Pi)$ each represent generators of
$H^3(\pi;\mathbb{Z}[\pi])\cong\mathbb{Z}$ then there is an automorphism 
$\phi$ of $\Pi$ such that $\phi(e)=\pm{e}+\ell$, 
for some $\ell\in{L}$, and $\phi|_L=id_L$,
and such that the induced automorphism of $H^3(\pi;\Pi)$ carries $k$ to $k'$.
This proves the second assertion.
\end{proof}

Assertion (2) extends partially an observation in \cite{KLPT}, 
namely that the $k$-invariant plays no role in their stable classification.

\begin{cor}
$P_2(X)\simeq{P_2(\partial(Y_o\times{D^2}))}\Leftrightarrow
\Pi\cong\mathbb{Z}[\pi]\oplus{I_\pi}$.
\qed
\end{cor}

If $P_2(X)$ retracts onto $P_2(Y_o)$ then the composite of $f_2(Y_o)$ with the
inclusion ${P_2(Y_o)}\to{P_2(X)}$ factors through $f_2(X)$, 
since $Y_o$ has dimension $\leq3$,
while $P_2(X)=X\cup{e^{\geq4}}$.

\section{the main theorem}

Let $\Gamma$ be the quadratic functor of Whitehead.
Let $L$ be a finitely generated left $\mathbb{Z}[\pi]$-module,
and let $Her_w(L^\dagger)$ be the abelian group of $w$-hermitean pairings on $L^\dagger$.
Then there is a natural homomorphism 
$B_L: \mathbb{Z}^w\otimes_{\mathbb{Z}[\pi]}\Gamma(L)\to{Her_w(L^\dagger)}$,
which is an isomorphism if $\pi$ is 2-torsion-free and $L$ is projective 
\cite[Theorem 1]{Hi20}.

If $\pi$ is a $PD_3$-group then it is torsion-free.
However the modules of interest to us are not projective, 
since $I_\pi$ has projective dimension 2.
We shall show that when $\Pi$ is stably equivalent to $I_\pi$ then $B_\Pi$ remains injective.
We first recall some details about  $\Gamma$ from \cite[Chapter 1.\S4]{Ba}.
Let $\gamma_A:A\to\Gamma(A)$ be the canonical quadratic map.
We may define a homomorphism $[-]:A\odot{A}\to\Gamma(A)$ by 
\[
[a\odot{b}]=\gamma(a+b)-\gamma(a)-\gamma(b).
\]
Then $[a\odot{a}]=2\gamma(a)$ for all $a\in{A}$.

As abelian groups, 
$\Gamma(A\oplus{B})\cong\Gamma(A)\oplus\Gamma(B)\oplus(A\otimes{B})$.
If $A$ and $B$ are $\mathbb{Z}[G]$-modules the summands are invariant 
under the action of $G$,
and so this direct sum splitting is a $\mathbb{Z}[G]$-module splitting.

The following lemma is close to the first part of \cite[Lemma 2.3]{HK88}
(which considered only finite groups $G$).

\begin{lemma}
\label{HKrerun}
Let $G$ be a group.
Then $\Gamma(\mathbb{Z}[G])\cong\mathbb{Z}[G]\oplus\Gamma(I_G)$
as a left $\mathbb{Z}[G]$-module.
\end{lemma}

\begin{proof}
Let $i_g=g-1$, for $g\in{G}$, 
and let $j:\mathbb{Z}\to\mathbb{Z}[G]$ be the canonical ring homomorphism.
Then $I_G$ is free with basis $\{i_g\mid{g\in{G}}\}$, and
$\mathbb{Z}[G]\cong\mathrm{Im}(j)\oplus{I_G}$ as abelian groups.
Hence $\Gamma(\mathbb{Z}[G])$ splits as a direct sum of abelian groups
$\Gamma(\mathbb{Z})\oplus\Gamma(I_G)\oplus(I_G\otimes\mathbb{Z})$.
The middle summand is a $\mathbb{Z}[G]$-submodule,
but the others are not.

The complement of $\Gamma(I_G)$ in $\Gamma(\mathbb{Z}[G])$
is freely generated (as an abelian group)
by the elements $e=\gamma_{\mathbb{Z}[G]}(1)$ and
$\{i_g\otimes1\mid{g\in{G}}\}$,
and so the quotient $\Gamma(\mathbb{Z}[G])/ \Gamma(I_G)$
is freely generated by the images of these elements.
The group $G$ acts on the basis elements by $h.1=1+i_h$
and $h.i_g=i_{hg}-i_h$.
Hence 
\[
2(h.e)=
h.2\gamma_{\mathbb{Z}[G]}(1)=h([1\odot1])\equiv[1\odot1]+2i_h\otimes1
~mod~\Gamma(I_G).
\]
Since $[1\odot1]=2e$ and $\Gamma(\mathbb{Z}[G])/\Gamma(I_G)$ is torsion-free (as an abelian group), 
\[
h.e\equiv{e+i_h\otimes1}~mod~\Gamma(I_G).
\]
We also have
\[
h.(i_g\otimes1)\equiv{i_{hg}\otimes1-i_g\otimes1}~mod~\Gamma(I_G).
\]
Thus the bijection sending $i_g$ to $i_g\otimes1$ and $1$ to $e$ defines 
an isomorphism $\mathbb{Z}[G]\cong\Gamma(\mathbb{Z}[G])/ \Gamma(I_G)$.
Hence $\Gamma(\mathbb{Z}[G])\cong\mathbb{Z}[G]\oplus\Gamma(I_G)$.
\end{proof}

We may strengthen this result as follows.

\begin{lemma}
\label{summand}
If $\Pi\oplus\mathbb{Z}[\pi]^r\cong\mathbb{Z}[\pi]^s\oplus{I_\pi}$ 
then $\Gamma(\Pi)$ is a direct summand of $\Gamma(\mathbb{Z}[\pi]^{s+1})$.
\end{lemma}

\begin{proof}
Since $\Gamma(\Pi)$ is a direct summand of $\Gamma(\Pi\oplus\mathbb{Z}[\pi]^r)$,
it shall suffice to assume that $\Pi\cong\mathbb{Z}[\pi]^s\oplus{I_\pi}$.
We may compare the splittings
\[
\Gamma(\mathbb{Z}[\pi]^s\oplus{I_\pi})=\Gamma(\mathbb{Z}[\pi]^s)\oplus
\Gamma(I_\pi)\oplus(\mathbb{Z}[\pi]^s\otimes{I_\pi})
\]
and
\[
\Gamma(\mathbb{Z}[\pi]^{s+1})=\Gamma(\mathbb{Z}[\pi]^s)\oplus
\Gamma(\mathbb{Z}[\pi])\oplus(\mathbb{Z}[\pi]^s\otimes\mathbb{Z}[\pi]).
\]
If the abelian group underlying a $\mathbb{Z}[\pi]$-module $M$ is free abelian 
with basis $\{m_i\}$ then the tensor products $M\otimes\mathbb{Z}[\pi]$ and 
$\mathbb{Z}[\pi]\otimes{M}$ with the diagonal left $\mathbb{Z}[\pi]$-structures 
are free $\mathbb{Z}[\pi]$-modules with bases $\{m_i\otimes1\}$ and $\{1\otimes{m_i}\}$,
respectively.
Hence
\[
(\mathbb{Z}[\pi]^s\otimes\mathbb{Z}[\pi])\oplus\Gamma(\mathbb{Z}[\pi])\cong
(\mathbb{Z}[\pi]^s\otimes{I_\pi})\oplus\mathbb{Z}[\pi]\oplus\Gamma(I_\pi)\oplus\mathbb{Z}[\pi],
\]
and so $\Gamma(\mathbb{Z}[\pi]^s\oplus{I_\pi})$ is a direct summand of 
$\Gamma(\mathbb{Z}[\pi]^{s+1})$.
\end{proof}
 
If $[\Pi]=[I]$  then $[\Pi^w]=[I^w]$.
Therefore $H_i(\pi;\Pi^w)\cong{H_{i+1}(\pi;\mathbb{Z}^w)}$ for $i>0$.
Hence $H_2(\pi;\Pi^w)\cong\mathbb{Z}$ and $H_3(\pi;\Pi^w)=0$.

In the formulation of the next theorem, 
we have included $\Pi$ among the relevant invariants, 
although (as observed earlier) it is determined by 
the intersection pairing.

\begin{theorem}
\label{mainthm}
Let $X$ be a $PD_4$-complex such that $\pi=\pi_1(X)$ is a $PD_3$-group
and $w_1(X)=w_1(\pi)$.
Then the homotopy type of $X$ is determined by $\pi$, 
$\Pi=\pi_2(X)$,  $\kappa=k_1(X)$ and $\lambda_X$.
\end{theorem}

\begin{proof}
The homotopy type of a $PD_4$-complex $X$ is determined by 
$P_2(X)$ and the image of a fundamental class $[X]$ in $H_4(P_2(X);\mathbb{Z}^w)$.
where $w=w_1(X)$  \cite[Theorem 3.1]{BB}.
(This was first proven in \cite[Theorem 1.1]{HK88}, 
assuming also that $\beta_2(X;\mathbb{Q})>0$.)
The invariants $\pi,\Pi$ and $\kappa$ determine $P_2(X)$.
We shall show that $\lambda_X$ determines the image of $[X]$ 
in  $H_4(P_2(X);\mathbb{Z}^w)$.

The Cartan-Leray spectral sequences for the universal covers give epimorphisms
$\delta_X:H_4(X;\mathbb{Z}^w)\to{H_2(\pi;\Pi^w)}$ and 
$\delta_P:H_4(P_2(X);\mathbb{Z}^w)\to{H_2(\pi;\Pi^w)}$,
since $c.d.\pi=3$.
Since $\pi$ has one end $\delta_X$ is an isomorphism,
while there is an exact sequence 
\begin{equation*}
\begin{CD}
0\to\mathbb{Z}^w\otimes_{\mathbb{Z}[\pi]}H_4(\Pi,2;\mathbb{Z})@>\phi>>
{H_4(P_2(X);\mathbb{Z}^w)}@>\delta_P>>{H_2(\pi;\Pi^w)}\to0.
\end{CD}
\end{equation*}
The universal cover of $P_2(X)$ is a $K(\Pi,2)$-space,
and the `boundary" homomorphism $b:H_4(\Pi,2;\mathbb{Z})\cong\Gamma(\Pi)$
of Whitehead is an isomorphism, since $\pi_i(K(\Pi,2))=0$ for $i\not=2$ 
(see \cite[1.3.7]{Ba}.
Hence $\psi=\mathbb{Z}^w\otimes_\Gamma{b}$ is also an isomorphism.

Since $\mathbb{Z}^w\otimes_{\mathbb{Z}[\pi]}\Gamma(\Pi)$ is a direct summand 
of $\mathbb{Z}^w\otimes_{\mathbb{Z}[\pi]}\Gamma(\mathbb{Z}[\pi]^{s+1})$,
by Lemma \ref{summand}, 
and since $B_M$ is an isomorphism if $M$ is a finitely generated projective module \cite[Theorem 2]{Hi20}, $B_\Pi$ is a monomorphism.

Let $\theta:H_4(P_2(X);\mathbb{Z}^w)\to{Her_w(\Pi^\dagger)}$
be the function defined by
\[
\theta(\xi)(u,v)=v(u\cap\xi)\quad\forall~u,v\in{H^2(X;\mathbb{Z}[\pi])}~\mathrm{and}~
\xi\in{H_4(P_2(X);\mathbb{Z}^w)}.
\]
Then $\theta\phi=B_\Pi\psi$, and so $\theta\phi$ is also a monomorphism.

Suppose that $X_1$ is a second such $PD_4$-complex 
and $h:P_2(X_1)\to{P_2(X)}$ is a homotopy equivalence 
which induces an isometry $f:\lambda_{X_1}\cong\lambda_X$.
Then the images of $h_*[X_1]$ and $[X]$ in $H_2(\pi;\Pi^w)$ agree,
and so $h_*[X_1]-[X]$ is in the image of $\phi$.
Since $h$ induces an isometry $\lambda_{X_1}\cong\lambda_X$
and since $\theta\phi$ is a monomorphism,
it follows that $h_*[X_1]=[X]$.
Hence $X_1\simeq{X}$ \cite[Theorem 3.1]{BB}.
\end{proof}

\begin{cor}
If $M$ is a closed $4$-manifold, 
$\pi=\pi_1(M)$ is a solvable $PD_3$-group and $w_1(M)=w_1(\pi)$ 
then the homeomorphism type of $M$ is determined by $\lambda_M$.
\end{cor}

\begin{proof}
Solvable $PD_n$-groups are torsion free and polycyclic, 
and satisfy the Farrell-Jones conjectures \cite{FJ87,FJ88}.
Since such groups $\pi$ are ``good",  standard surgery techniques
apply in this 4-dimensional setting.
\end{proof}

With present knowledge,  these
are the only known good $PD_3$-groups.

%\section{the stable homeomorphism classification}

Theorem \ref{mainthm} in conjunction with the Farrell-Jones conjectures 
for 3-manifold groups \cite[Corollary 1.3]{BFL14}
suggests an alternative route to some of the results of \cite{KLPT}
on the stable homeomorphism classification of closed orientable 4-manifolds 
with COAT fundamental groups,
using stable 4-dimensional surgery \cite{CS71} 
rather than the modified surgery of Kreck.



%\section{degree-1 maps}



%
\newpage
\begin{thebibliography}{99}

\bibitem{Ar81} Artamanov, V. A. 
Projective nonfree modules over group rings of solvable groups,
Mat. Sbornik 116 (1981), 232--244.

\bibitem{BFL14} Bartels, A, Farrell, F.-T. and L\"uck, W. 
The Farrell-Jones conjectures for cocompact lattices in virtually connected Lie Groups,

J. Amer. Math. Soc 27 (2014),  339--388.

\bibitem{Ba} Baues, H. J.  
\textit{Combinatorial Homotopy and 4-Dimensional Complexes},

Expositions in Math. 2,  W. De Gruyter, Berlin -- New York (1991).

\bibitem{BB} Baues, H. J.  and Bleile, B.  
Poincar\'e duality complexes in dimension four,

Alg. Geom. Top. 8 (2008),  2355--2389.

\bibitem{CS71} Cappell, S.  E. and Shaneson, J. L. 
On four-dimensional surgery and applications,
Comment. Math. Helvetici 46 (1971),  500--528.

\bibitem{Do60} Dold, A.  Zur Homotopietheorie der Kettenkomplexe,

Math. Ann. 140 (1960), 278--298.

\bibitem{FJ87} Farrell, F. T. and Jones, L. E.  
Implications of the Geometrization Conjecture for the algebraic $K$-theory of 3-manifolds,
in \textit{Geometry and Topology, Athens Ga (1985)},
Lecture Notes in Pure and Applied Math. 105 (1987), 109--113.

\bibitem{FJ88}  Farrell, F. T. and Jones, L. E.  
The surgery $L$-groups of poly-(finite or cyclic) groups,
Invent. Math. 91 (1988), 559--586.

\bibitem{HK88} Hambleton, I, and Kreck, M. 
On the classification of topological 4-manifolds with finite fundamental group, 

Math. Ann. 280 (1988),  85--104. 
Corrigendum {\it ibid} 372 (2018), 527--530.

\bibitem{HKT} Hambleton, I., Kreck, M. and Teichner, P. 
Topological 4-manifolds with 

geometrically two-dimensional fundamental groups,

J.  Topol.  Anal.  1 (2009),  123--151.

\bibitem{FMGK} Hillman, J.A. \textit{Four-Manifolds, Geometries and Knots},

GT Monographs, vol. 5, 

Geometry and Topology Publications, Warwick (2002 -- revised 2007, 2022).

\bibitem{Hi20} Hillman, J.A.  
$PD_4$-complexes and 2-dimensional duality groups,

2019-20 MATRIX Annals (2020), 57--109.

\bibitem{Hi20a} Hillman, J.A.  
\textit{Poincar\'e Duality in Dimension 3},

Open Book Series vol.  3, MSP, Berkeley (2020).

\bibitem{KLPT} Kasprowski, D., Land,  M., Powell, M. A.  and Teichner, P.
Stable classification of 4-manifolds with 3-manifold fundamental groups,

J.  Topol. 10 (2017), 821-887.

\bibitem{Pl86} Plotnick, S.  P. 
Equivariant intersection forms, knots in $S^4$, 
and rotations in 2-spheres,
Trans.  Amer.  Math.  Soc.  296 (1986),  543--575.

\bibitem{Ra} Ranicki, A. A. 
\textit{Algebraic and Geometric Surgery},

Oxford Mathematical Monographs (2002).

\end{thebibliography}

\end{document}



\section{some questions}

Can we recover an analogue  of \cite[Theorem 3.1]{Pl86}:
are $\lambda_{X(\pi)}$ and $\lambda_{X(\pi)_\tau}$ inequivalent?
Does $X(\pi)_\tau$ retract onto $Y_o$?

Are there other possible homotopy types with $\chi=2$?

There a degree-1 map to $X(\pi)$ or $X(\pi)_\tau$ if and only if
 $\lambda_X$ splits appropriately \cite[Theorem 5]{Hi20}.


Which modules stably equivalent to $I$ are realizable as $\Pi$?

Which $k$-invariants are realizable? (Special case: $\pi=\pi'$.)

Which intersection pairings on $\Pi$? (Check \cite{KLPT}?)

If $f:N\to{M}$ induces a $\pi_1$-isomorphism must it have degree $\pm1$?

What can we say about $\Pi$ if $w_1(X)\not=w_1(\pi)$?

Wall finiteness of $X$ and $\pi$, in relation to $\Pi$?





If there is a finite 2-dimensional $K(\pi,1)$-complex then there is a degree-1 map 
$f:M\to{X}$ to a ``strongly minimal" closed 4-manifold $X$ with $\pi_1(X)=\pi$,
$w_1(X)=w_1(M)$, $\chi(X)=2\chi(\pi)$ and $\lambda_X$ trivial.
If moreover $\pi$ is free, a $PD_2$-group or a solvable Baumslag-Solitar group 
then the homotopy type of $M$ is determined by the above data \cite{Hi20}.
 
These invariants determine the stable homeomorphism type of $M$
(with respect to sums with $S^2\times{S^2}$), firstly
if there is a finite 2-dimensional $K(\pi,1)$-complex,
$\pi$ satisfies the Farrell-Jones conjectures and $w=1$ \cite{HKT},
and secondly if $\pi$ is the group of an aspherical closed orientable 3-manifold \cite{KLPT}.
In these cases the $k$-invariant is 0 or is determined by the other data.
However in the latter case it is no longer clear that every such 4-manifold 
admits a degree-1 map to one with minimal Euler characteristic.


(When $L=I_\pi$ we can derive a more precise description of the action of $Aut(\pi)$ 
on $H^3(\pi;\Pi)$ from \cite[Lemmas 3.2 and 3.3]{Pl86}, but we do not need this here.)



$\pi_3(X)\cong\Gamma_W(\pi)$,  since $H_3(X;\Gamma)0$.

$H_3(c_X)=0$, by Poincar\'e duality and the fact that $H^4(\pi;\mathbb{Z})=0$.


\bibitem{Ha92} Hambleton, I. Intersection forms, fundamental groups and 4-manifolds,

G\"ok\"ova

\bibitem{SW} Sun, Wang ??




