\section{Conclusion}
\label{sec:conclusion}

We introduce an generalized OOD-aware semi-supervised object detection approach, based on a Teacher-Student paradigm, which includes both ID and OOD data. We also propose, an OOD Explorer to achieve two goals: 1) effective separation of OOD objects from ID objects and 2) localization of OOD objects. Our results show that this method significantly improves the robustness of semi-supervised object detection methods for ID objects through its capability to not only detect OOD objects but also integrate them into the training process and learn from them.   

\noindent\textbf{Limitations and future work.} While our method advances the state of the art, there is much room for improvement with respect to the localization and classification of OOD objects.
Most importantly, new techniques are needed to localize OOD objects more precisely. Our future work will include a repeated and continual testing of the proposed approach with a varying number of new classes. New approaches such as continuous, in-field learning and the incorporation of domain understanding into current data-driven models may hold promise in achieving these goals. 
