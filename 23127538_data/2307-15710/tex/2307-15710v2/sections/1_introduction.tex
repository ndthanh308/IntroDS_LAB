\section{Introduction}

Object detection based on deep learning techniques is highly dependent on two assumptions (1) availability of large-scale labeled datasets, and (2) a common, fixed set of object classes that appear in both training and unlabeled data, i.e., the ``closed-world'' assumption. While these assumptions may hold in ideal conditions, they often manifest as limitations in real-world situations. Semi-supervised object detection (SSOD) approaches  \citep{sohn2020detection, berthelot2019mixmatch, jeong2019consistency} address the first limitation by leveraging unlabeled data to boost model performance. However, these methods assume the classes in both the training and unlabeled data are sampled from the same distribution, i.e., in-distribution (ID) objects. In practice, the unlabeled data might contain objects that were \textit{unknown} and \textit{unseen} during training, which we call out-of-distribution (OOD) objects. 

Prior work \citep{dhamija2020overlooked, liuopen, miller2021uncertainty} shows that the existing SSOD techniques perform worse in the open-set setting, i.e in the presence of OOD objects. This issue is termed as the `semantic expansion' problem, where OOD objects end up being detected as ID objects and incorrectly used as pseudo-labels for unlabeled data. 
\cite{liuopen} addresses the open-set semi-supervised object detection problem, by proposing an offline OOD detector to filter out (reject) any OOD data, to improve the performance of ID classes. These methods, however, disregard any novel OOD data, thereby limiting the learning process to a predefined set of classes.

In this work, we propose an extension of the open-set to a more generalizable open-world setting that not only improves the performance for ID classes but also discovers and includes OOD classes in the learning process. Specifically, in the generalized open-world setting, we are given a small labeled dataset that includes ID categories of data and a large unlabeled dataset that contains both ID and OOD categories. The aim is to use both the labeled and unlabeled data to improve the performance of ID categories and simultaneously discover OOD data as `unknown' and adapt the semi-supervised object detection pipeline (Figure \ref{fig:owssd-problem}).

To achieve this, we propose an \textit{OOD Explorer} which is designed to address two primary tasks: the classification of objects as OOD or ID and the localization of novel OOD objects. For the OOD classification task, we propose a simple yet effective OOD detector that leverages an ensemble of lightweight auto-encoder networks trained \textit{only} on ID data. For OOD localization, we systematically analyze multiple unsupervised, class-agnostic object detection techniques, as they align with the inherently open-world nature of the problem. The second component is an \textit{OOD-aware semi-supervised learning framework}, structured into two learning stages. The first stage leverages labeled data, while the second stage incorporates both labeled and unlabeled data, with the help of the OOD Explorer that introduces OOD data.

In our experiments, we examine different open-world scenarios and evaluate the performance of (1) the proposed OOD Explorer on both OOD classification and localization metrics (2) OOD-aware semi supervised object detection. The results show that our method performs competitively against state-of-the-art OOD detection algorithms and that the proposed OOD-aware semi-supervised learning method significantly improves the robustness of ID objects classification and identification through its ability to detect OOD objects and integrate them into the model and learn from them.
Our contributions are summarized as follows:
\begin{itemize}
     \item We address the problem of generalized open-world semi-supervised object detection problem, by improving the  performance on \textit{both} ID and OOD classes.
     \item We propose a novel \textit{OOD Explorer} for handling novel OOD data, capable of both classification and localization) (Section~\ref{sec:methodology:ood-explorer}). In combination with the OOD Explorer, we develop an OOD-aware semi-supervised object detection algorithm capable of detecting and labeling objects of both ID and OOD classes (Section~\ref{sec:methodology:ssl-pipeline}).
    \item We demonstrate that our proposed method is able to detect OOD classes while substantially improving the robustness of ID class detection. (Section~\ref{sec:expts}).
\end{itemize}

% Figure environment removed
    