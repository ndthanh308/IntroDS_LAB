
\documentclass{article} % For LaTeX2e
\usepackage{iclr2024_conference,times}

% Optional math commands from https://github.com/goodfeli/dlbook_notation.
%%%%% NEW MATH DEFINITIONS %%%%%
\newtheorem{property}{Property}
\newtheorem{definition}{Definition}
\newtheorem{theorem}{Theorem}
\newtheorem{lemma}{Lemma}
\newtheorem{corollary}{Corollary}
\DeclarePairedDelimiter\abs{\lvert}{\rvert}
\DeclarePairedDelimiter\norm{\lVert}{\rVert}
\makeatletter
\let\oldabs\abs
\def\abs{\@ifstar{\oldabs}{\oldabs*}}
\let\oldnorm\norm
\def\norm{\@ifstar{\oldnorm}{\oldnorm*}}
\makeatother

% Mark sections of captions for referring to divisions of figures
\newcommand{\figleft}{{\em (Left) }}
\newcommand{\figcenter}{{\em (Center) }}
\newcommand{\figright}{{\em (Right)}}
\newcommand{\figtop}{{\em (Top) }}
\newcommand{\figbottom}{{\em (Bottom) }}
\newcommand{\captiona}{{\em (a) }}
\newcommand{\captionb}{{\em (b) }}
\newcommand{\captionc}{{\em (c) }}
\newcommand{\captiond}{{\em (d) }}

% Highlight a newly defined term
\newcommand{\newterm}[1]{{\bf #1}}


\def\figref#1{figure~\ref{#1}}
\def\Figref#1{Figure~\ref{#1}}
\def\twofigref#1#2{figures \ref{#1} and \ref{#2}}
\def\quadfigref#1#2#3#4{figures \ref{#1}, \ref{#2}, \ref{#3} and \ref{#4}}
\def\secref#1{section~\ref{#1}}
\def\Secref#1{Section~\ref{#1}}
\def\twosecrefs#1#2{sections \ref{#1} and \ref{#2}}
\def\secrefs#1#2#3{sections \ref{#1}, \ref{#2} and \ref{#3}}
\def\eqref#1{equation~\ref{#1}}
\def\Eqref#1{Equation~\ref{#1}}
% A raw reference to an equation---avoid using if possible
\def\plaineqref#1{\ref{#1}}
% Reference to a chapter, lower-case.
\def\chapref#1{chapter~\ref{#1}}
% Reference to an equation, upper case.
\def\Chapref#1{Chapter~\ref{#1}}
% Reference to a range of chapters
\def\rangechapref#1#2{chapters\ref{#1}--\ref{#2}}
% Reference to an algorithm, lower-case.
\def\algref#1{algorithm~\ref{#1}}
% Reference to an algorithm, upper case.
\def\Algref#1{Algorithm~\ref{#1}}
\def\twoalgref#1#2{algorithms \ref{#1} and \ref{#2}}
\def\Twoalgref#1#2{Algorithms \ref{#1} and \ref{#2}}
% Reference to a part, lower case
\def\partref#1{part~\ref{#1}}
% Reference to a part, upper case
\def\Partref#1{Part~\ref{#1}}
\def\twopartref#1#2{parts \ref{#1} and \ref{#2}}

% Random variables
\def\reta{{\textnormal{$\eta$}}}
\def\ra{{\textnormal{a}}}

% Random vectors
\def\rvepsilon{{\mathbf{\epsilon}}}
\def\rvtheta{{\mathbf{\theta}}}
\def\rva{{\mathbf{a}}}

% Elements of random vectors
\def\erva{{\textnormal{a}}}
\def\ervb{{\textnormal{b}}}

% Random matrices
\def\rmA{{\mathbf{A}}}
\def\rmB{{\mathbf{B}}}

% Elements of random matrices
\def\ermA{{\textnormal{A}}}
\def\ermB{{\textnormal{B}}}

\def\fvec{{\mathbf{f}}}
\def\bff{{\mathbf{f}}}
\def\bfg{{\mathbf{g}}}
% Vectors
\def\vzero{{\bm{0}}}
\def\vone{{\bm{1}}}
\def\vmu{{\bm{\mu}}}
\def\vtheta{{\bm{\theta}}}
\def\va{{\bm{a}}}
\def\vb{{\bm{b}}}
\def\vc{{\bm{c}}}
\def\vd{{\bm{d}}}
\def\ve{{\bm{e}}}
\def\vf{{\bm{f}}}
\def\vg{{\bm{g}}}
\def\vh{{\bm{h}}}
\def\vi{{\bm{i}}}
\def\vj{{\bm{j}}}
\def\vk{{\bm{k}}}
\def\vl{{\bm{l}}}
\def\vm{{\bm{m}}}
\def\vn{{\bm{n}}}
\def\vo{{\bm{o}}}
\def\vp{{\bm{p}}}
\def\vq{{\bm{q}}}
\def\vr{{\bm{r}}}
\def\vs{{\bm{s}}}
\def\vt{{\bm{t}}}
\def\vu{{\bm{u}}}
\def\vv{{\bm{v}}}
\def\vw{{\bm{w}}}
\def\vx{{\bm{x}}}
\def\vy{{\bm{y}}}
\def\vz{{\bm{z}}}

% Matrix
\def\mA{{\bm{A}}}

% Tensor
\DeclareMathAlphabet{\mathsfit}{\encodingdefault}{\sfdefault}{m}{sl}
\SetMathAlphabet{\mathsfit}{bold}{\encodingdefault}{\sfdefault}{bx}{n}
\newcommand{\tens}[1]{\bm{\mathsfit{#1}}}
\def\tA{{\tens{A}}}
\def\tB{{\tens{B}}}
\def\tC{{\tens{C}}}
\def\tD{{\tens{D}}}
\def\tE{{\tens{E}}}
\def\tF{{\tens{F}}}
\def\tG{{\tens{G}}}
\def\tH{{\tens{H}}}
\def\tI{{\tens{I}}}
\def\tJ{{\tens{J}}}
\def\tK{{\tens{K}}}
\def\tL{{\tens{L}}}
\def\tM{{\tens{M}}}
\def\tN{{\tens{N}}}
\def\tO{{\tens{O}}}
\def\tP{{\tens{P}}}
\def\tQ{{\tens{Q}}}
\def\tR{{\tens{R}}}
\def\tS{{\tens{S}}}
\def\tT{{\tens{T}}}
\def\tU{{\tens{U}}}
\def\tV{{\tens{V}}}
\def\tW{{\tens{W}}}
\def\tX{{\tens{X}}}
\def\tY{{\tens{Y}}}
\def\tZ{{\tens{Z}}}


% Graph
\def\gA{{\mathcal{A}}}
\def\gB{{\mathcal{B}}}
\def\gC{{\mathcal{C}}}
\def\dataset{{\mathcal{D}}}
\def\gE{{\mathcal{E}}}
\def\gF{{\mathcal{F}}}
\def\fourier{{\mathcal{F}}}
\def\gG{{\mathcal{G}}}
\def\gH{{\mathcal{H}}}
\def\gI{{\mathcal{I}}}
\def\gJ{{\mathcal{J}}}
\def\gK{{\mathcal{K}}}
\def\gL{{\mathcal{L}}}
\def\loss{{\mathcal{L}}}
\def\gM{{\mathcal{M}}}
\def\gN{{\mathcal{N}}}
\def\normal{{\mathcal{N}}}
\def\gaussian{{\mathcal{N}}}
\def\gO{{\mathcal{O}}}
\def\gP{{\mathcal{P}}}
\def\gQ{{\mathcal{Q}}}
\def\gR{{\mathcal{R}}}
\def\gS{{\mathcal{S}}}
\def\gT{{\mathcal{T}}}
\def\gU{{\mathcal{U}}}
\def\uniform{{\mathcal{U}}}
\def\gV{{\mathcal{V}}}
\def\gW{{\mathcal{W}}}
\def\gX{{\mathcal{X}}}
\def\gY{{\mathcal{Y}}}
\def\gZ{{\mathcal{Z}}}

\def\algebra{{\mathscr{A}}}
\def\borel{{\mathscr{B}}}
\def\manifold{{\mathscr{M}}}

% Sets
\def\sA{{\mathbb{A}}}
\def\sB{{\mathbb{B}}}
\def\complex{{\mathbb{C}}}
\def\sD{{\mathbb{D}}}
\def\expectation{{\mathbb{E}}}
\newcommand{\E}{\mathbb{E}}
\def\sF{{\mathbb{F}}}
\def\sG{{\mathbb{G}}}
\def\sH{{\mathbb{H}}}
\def\sI{{\mathbb{I}}}
\def\sJ{{\mathbb{J}}}
\def\sK{{\mathbb{K}}}
\def\sL{{\mathbb{L}}}
\def\sM{{\mathbb{M}}}
\def\natural{{\mathbb{N}}}
\def\sO{{\mathbb{O}}}
\def\sP{{\mathbb{P}}}
\def\rational{{\mathbb{Q}}}
\def\real{{\mathbb{R}}}
\newcommand{\R}{\mathbb{R}}
\def\sS{{\mathbb{S}}}
\def\sphere{{\mathbb{S}}}
\def\sT{{\mathbb{T}}}
\def\sU{{\mathbb{U}}}
\def\sV{{\mathbb{V}}}
\def\sW{{\mathbb{W}}}
\def\sX{{\mathbb{X}}}
\def\sY{{\mathbb{Y}}}
\def\integer{{\mathbb{Z}}}
\def\indicator{{\mathbbm{1}}}

% Entries of a matrix
\def\emLambda{{\Lambda}}
\def\emA{{A}}
\def\emB{{B}}
\def\emC{{C}}
\def\emD{{D}}
\def\emE{{E}}
\def\emF{{F}}
\def\emG{{G}}
\def\emH{{H}}
\def\emI{{I}}
\def\emJ{{J}}
\def\emK{{K}}
\def\emL{{L}}
\def\emM{{M}}
\def\emN{{N}}
\def\emO{{O}}
\def\emP{{P}}
\def\emQ{{Q}}
\def\emR{{R}}
\def\emS{{S}}
\def\emT{{T}}
\def\emU{{U}}
\def\emV{{V}}
\def\emW{{W}}
\def\emX{{X}}
\def\emY{{Y}}
\def\emZ{{Z}}
\def\emSigma{{\Sigma}}

% entries of a tensor
% Same font as tensor, without \bm wrapper
\newcommand{\etens}[1]{\mathsfit{#1}}
\def\etLambda{{\etens{\Lambda}}}
\def\etA{{\etens{A}}}
\def\etB{{\etens{B}}}
\def\etC{{\etens{C}}}
\def\etD{{\etens{D}}}
\def\etE{{\etens{E}}}
\def\etF{{\etens{F}}}
\def\etG{{\etens{G}}}
\def\etH{{\etens{H}}}
\def\etI{{\etens{I}}}
\def\etJ{{\etens{J}}}
\def\etK{{\etens{K}}}
\def\etL{{\etens{L}}}
\def\etM{{\etens{M}}}
\def\etN{{\etens{N}}}
\def\etO{{\etens{O}}}
\def\etP{{\etens{P}}}
\def\etQ{{\etens{Q}}}
\def\etR{{\etens{R}}}
\def\etS{{\etens{S}}}
\def\etT{{\etens{T}}}
\def\etU{{\etens{U}}}
\def\etV{{\etens{V}}}
\def\etW{{\etens{W}}}
\def\etX{{\etens{X}}}
\def\etY{{\etens{Y}}}
\def\etZ{{\etens{Z}}}

\def\ceil#1{\lceil #1 \rceil}
\def\floor#1{\lfloor #1 \rfloor}
\def\eps{{\epsilon}}

\newcommand{\pder}[1]{\frac{\partial}{\partial #1}}

\newcommand{\half}{\frac{1}{2}}
\newcommand{\limNinf}{\lim_{N \to \infty}}
\newcommand{\limTzero}{\lim_{\tau \to 0}}


\newcommand{\cmark}{\ding{51}}
\newcommand{\xmark}{\ding{55}}

\newcommand{\layer}{\mathcal{H}}
\newcommand{\defeq}{\triangleq}
%\newcommand{\defeq}{vcentcolon=}
\newcommand{\domain}{\Omega}
\newcommand{\grad}{\nabla}

\newcommand{\cin}{c_{\rm{in}}}
\newcommand{\cout}{c_{\rm{out}}}
\newcommand{\intdomain}{\int_{\domain}}
\newcommand{\network}{\gT}
\newcommand{\subnet}{\gK}
\newcommand{\map}{\gR} %\gR

\newcommand{\innerproduct}[2]{\langle #1, #2 \rangle}
\newcommand{\mcsum}[1][j]{\frac{1}{N}\sum_{#1=1}^N}

\newcommand{\inrspace}[1][c]{\gF_{#1}}

\DeclareMathOperator*{\argmax}{arg\,max}
\DeclareMathOperator*{\argmin}{arg\,min}

\let\ab\allowbreak


\usepackage{amsmath}
\usepackage[utf8]{inputenc} % allow utf-8 input
\usepackage[T1]{fontenc}    % use 8-bit T1 fonts
\usepackage{hyperref}       % hyperlinks
\usepackage{url}            % simple URL typesetting
\usepackage{booktabs}       % professional-quality tables
\usepackage{amsfonts}       % blackboard math symbols
\usepackage{nicefrac}       % compact symbols for 1/2, etc.
\usepackage{microtype}      % microtypography
\usepackage{xcolor}         % colors
\usepackage{graphicx}    
\usepackage{multirow}
\usepackage{algorithm}
\usepackage{algorithmic}
\usepackage{subfigure}
\usepackage{float}
\usepackage{tabularx}
\usepackage{placeins}
\usepackage{makecell}


\title{A Private Watermark for Large Language \\ Models}

% Authors must not appear in the submitted version. They should be hidden
% as long as the \iclrfinalcopy macro remains commented out below.
% Non-anonymous submissions will be rejected without review.

\author{
Aiwei Liu\textsuperscript{1},
~~~ Leyi Pan\textsuperscript{1},
~~~ Xuming Hu\textsuperscript{1},
~~~ \bf{Shu'ang Li}\textsuperscript{1},
~~~ \bf{Lijie Wen}\textsuperscript{1}, \\
~~~ \bf{Irwin King}\textsuperscript{2},
~~~ \bf{Philip S. Yu}\textsuperscript{3}\\
\textsuperscript{1}Tsinghua University~~~
\textsuperscript{2}The Chinese University of Hong Kong~~~\\
\textsuperscript{3}University of Illinois at Chicago~~~\\
{\tt\small liuaw20@mails.tsinghua.edu.cn, wenlj@tsinghua.edu.cn,}
{\tt\small  king@cse.cuhk.edu.hk}
}

% The \author macro works with any number of authors. There are two commands
% used to separate the names and addresses of multiple authors: \And and \AND.
%
% Using \And between authors leaves it to \LaTeX{} to determine where to break
% the lines. Using \AND forces a linebreak at that point. So, if \LaTeX{}
% puts 3 of 4 authors names on the first line, and the last on the second
% line, try using \AND instead of \And before the third author name.

\newcommand{\fix}{\marginpar{FIX}}
\newcommand{\new}{\marginpar{NEW}}

\iclrfinalcopy % Uncomment for camera-ready version, but NOT for submission.
\begin{document}


\maketitle

\begin{abstract}
Recently, text watermarking algorithms for large language models (LLMs) have been proposed to mitigate the potential harms of text generated by LLMs, including fake news and copyright issues. However, current watermark detection algorithms require the secret key used in the watermark generation process, making them susceptible to security breaches and counterfeiting.
To address this limitation, we propose the first private watermarking algorithm that uses two different neural networks for watermark generation and detection, instead of using the same key at both stages. Meanwhile, the token embedding parameters are shared between the generation and detection networks, which makes the detection network achieve a high accuracy very efficiently.
Experiments demonstrate that Our algorithm attains high detection accuracy and computational efficiency through neural networks with a minimized number of parameters. Subsequent analysis confirms the high complexity involved in reverse-engineering the watermark generation algorithms from the detection network.   Our code and data are available at \href{https://github.com/THU-BPM/private_watermark}{https://github.com/THU-BPM/private\_watermark}.
%  Recently, text watermarking algorithms for large language models (LLMs) have been mitigating the potential harms of text generated by the LLMs, including fake news and copyright issues. However, the watermark detection of current algorithms requires the key from the watermark generation process, making them susceptible to breaches and counterfeiting.
% In this work, we propose the first private watermarking algorithm, which extends the current text watermarking algorithms by using two different neural networks respectively for watermark generation and detection, rather than using the same key at both stages. Meanwhile, part of the parameters of the watermark generation and detection networks are shared, which makes the detection network achieve a high accuracy very efficiently.
% Experiments show that our algorithm ensures high detection accuracy with minimal impact on generation and detection speed, due to the small parameter size of both networks. Additionally, our subsequent analysis demonstrates the difficulty of reverting the watermark generation rules from the detection network.
\end{abstract}

\section{Introduction}

With the development of current large language models (LLMs), many LLMs, like GPT-4 \citep{openai2023gpt4} and Claude\footnote{https://claude.ai/chat}, could rapidly generate human-like texts. This has led to numerous risks, such as the generation of a vast amount of false information on the Internet \citep{pan2023risk}, and the infringement of copyrights of creative works \citep{chen2023pathway}. Therefore, texts generated by LLMs need to be detected and tagged.

At present, some watermarking algorithms for LLM have proved successful in making machine-generated texts detectable by adding implicit features during the text generation process that are difficult for humans to discover but easily detected by the specially designed method  \citep{christ2023undetectable, kirchenbauer2023watermark}. However, current watermarking algorithms are all public, which means the detection of watermarks requires the key from the watermark generation process. This allows attackers to easily remove and forge the text watermarks with these public keys. Although \citet{kirchenbauer2023watermark}  have suggested that the watermark detection process could be placed behind the web API to achieve the effect of private watermarking, this approach requires substantial server resources and robust designs against hacking (even social engineering). Additionally, the server-based detection process inherently poses data privacy risks. A watermarking algorithm that obfuscates the generation key during detection could substantially ameliorate these challenges.


In this work, we propose the first private watermarking algorithm for large language models (LLMs). Our approach adopts the commonly used watermark schema, which embeds small watermark signals into LLM's logits during generation \cite{kirchenbauer2023watermark,zhao2023provable}. The key difference is that we implement watermarking in a privacy-preserving manner. To conceal the details of the watermark in the detection process, we propose using two separate neural networks for watermark generation and detection, rather than relying on the same secret key for both stages. The privacy of our algorithm stems from a computational asymmetry: constructing a watermark generation network from a watermark detection network is notably more complex than the reverse process (Section \ref{sec:ana}). 
To enhance the accuracy of the detection network while alleviating the complexity of the training process, we have shared the token embedding parameters between the detector and the generator. This provides some prior information to the detector while maintaining privacy. Specifically, our watermark generator takes a window of $w$ tokens as input and predicts the watermark signals of the last token. The text detector directly takes all tokens from the text as input and outputs whether the complete text contains the watermark.

% In this work, we propose the first private watermark algorithm for LLMs. Our work is built on the common watermark paradigm, which adds small watermark logits to the LLM's logits \cite{kirchenbauer2023watermark,zhao2023provable}. The difference is we implement these concepts in a private way. In order to hide the detail of the watermark generation method during the detection process, we propose two separate neural networks for watermark generation and detection instead of using the same key for both stages. The privacy of our algorithm derives from the black-box nature of neural networks, that is, it's nearly impossible to infer the watermark generation detail from the parameters of the detection network. Also, we analyze the difficulty of reverting watermarking generation detail from the output of the detection network in section \ref{sec:ana}.
% However, in practice, training such a detection network from scratch requires a vast amount of data, and achieving a high accuracy is challenging due to the complexity of the problem. Therefore, we also propose a neural network for the watermark generation process.  To achieve a high-accuracy detection network with relatively small data,  we share the token embedding layers between the watermark generation network and the watermark detection network, which essentially provides some prior information to the detection network.
% Specifically, our watermark generation network takes the input of $w$ (local window size) tokens  and outputs whether the last token belongs to the green list, which differs from the origin method \cite{kirchenbauer2023watermark} of 
% splitting the vocabulary into the green and red list based on the local window text's hash value and the secret key. Meanwhile, the text detection network directly inputs all the token lists from the text, with the output being a classification indicating whether the entire text contains the watermark added by the generation network.

% While constructing the training data for the watermark detection network, the presence of the watermark is also determined by considering the labels (red or green) of the first `window size - 1' tokens. These labels are generated by treating the text as a cyclic document connected from head to tail. In this way, we prevent attackers from easily deducing the watermarking rule by continually altering the last token and observing the output changes.

% Figure environment removed

% In our experiments, we demonstrate that the watermark detection algorithm could achieve a nearly 99\% detection accuracy rate, which is only marginally inferior to the public watermark algorithm. Given that the detection accuracy of the public watermark algorithm represents our theoretical upper bound, this is already a remarkable result. Moreover, because the amount of parameters of our watermark generation and detection network is negligible compared to the large language model, it brings almost no additional computation burden to the text generation process.  Subsequent experiments also illustrate the critical importance of sharing the token embedding layer between the generation and detection networks.

In our experiments, the watermark detection algorithm exhibited a detection performance almost identical to our theoretical upper bound, represented by the public watermark algorithm, with a nearly 99\% F1 score.
% In our experiments, we found that the watermark detection algorithm achieves a detection F1 score of nearly 99\%, only slightly below that of the public watermark algorithm, which serves as our theoretical upper bound.
Notably, the computational burden of our watermark generation and detection network is minimal, especially when compared to large language models, due to its significantly lower parameter count. Furthermore, we have demonstrated the difficulty of cracking the watermarking rules through detailed experiments. We find that both attacking strategies, employing a detector to construct training data for the generator training and using word frequency analysis, yield exceedingly low rates of successful decryption in scenarios where the window size is not exceptionally small.

The main contributions of this work can be summarized as follows:

\begin{itemize}
\item Introduction of a novel private watermark algorithm that employs separate neural networks for watermark generation and detection, enhancing resistance to erasure and counterfeiting.

\item Utilization of shared token embedding between the watermark generation and detection networks, thereby improving the efficiency of training the detector.

\item Through empirical analysis, we have demonstrated that our watermarking algorithm are highly resistant to decryption.
\end{itemize}

\section{Related work}

As large language models (LLMs) generate increasingly high-quality text, detecting and tagging machine-generated content grows more vital. Currently, there are two primary methods for identifying LLM-produced text: text watermarking and classifier-based detection. With watermarking, implicit features are incorporated into the text during generation, allowing specially designed techniques to identify marked passages. Classifier-based detection leaves text generation unchanged and uses classifiers trained to differentiate machine-from human-authored content. This excerpt will focus on introducing these two approaches separately.

% Current classifier-based detection methods usually directly employ a binary classification model. \citet{zhan2023g3detector} utilized generation text from GPT2 \citep{radford2019language}, BART \citep{lewis2019bart}, and GPT3.5-turbo \footnote{https://chat.openai.com} to fine-tune the \textit{Roberta-large} \citep{liu1907roberta} model, resulting in a highly accurate GPT text detector. Similarly, \citet{mireshghallah2023smaller} discovered that smaller language models perform well for the detection of machine-generated text.
% In an effort to improve the robustness of detection algorithms, \citet{su2023detectllm} incorporated log-rank information from language models into the detector as a crucial feature. Meanwhile, \citet{hu2023radar} introduced a paraphraser and utilized adversarial learning to enhance robustness.
% To distinguish text from more LLMs, \cite{wu2023llmdet} utilized the prior information of the model's next-token probabilities to design a better detection model. However, whether machine-generated text can fundamentally be detected remains an open question. \citet{chakraborty2023possibilities} believe that with enough data collection, it is possible to train a good detector. On the contrary, \cite{sadasivan2023can} argue that as language models become more complex and the distance between human and AI-generated text decreases, the optimal detector's performance may be only slightly better than a random classifier.
% In conclusion, some classifier-based detection methods can achieve impressive results. However, due to their limited explainability, their performance in real-world scenarios may be still doubted.

Recent work has explored classifier-based approaches for detecting machine-generated text. These methods typically fine-tune large pre-trained language models like GPT-2 \citep{radford2019language}, BART\citep{lewis2019bart}, and RoBERTa \citep{liu1907roberta} for binary text classification \citep{zhan2023g3detector, mireshghallah2023smaller}. To improve model robustness, some methods incorporate language model features \citep{su2023detectllm} or adversarial training \citep{hu2023radar}. However, the fundamental detectability of machine-generated text remains an open question. Some argue that with sufficient data collection, high-performing detectors are achievable \citep{chakraborty2023possibilities}. Others contend that as language models become more advanced, the performance gap between detectors and random classifiers may diminish \citep{sadasivan2023can}. While current classifiers can be highly accurate, their real-world effectiveness is uncertain due to their limited explainability. In summary, classifier-based detection shows promise but requires further research to address robustness, explanability, and theoretical detectability limits.

Compared to the classifier-based methods, text watermarking is more explainable due to the injected implicit features in the text. There are typically two categories of text watermarking methods. The first is to add a watermark to the existing text. For example, \citet{abdelnabi2021adversarial} designed a data-hiding network to embed watermark information in the text, and utilized a data-revealing network to recover the embedded information. \citet{yoo2023robust} injected the watermark by substituting some words in the text. However, adding a watermark to the existing text struggles to keep the semantics of the text unchanged which limits its use in real-world scenarios. Another line of methods is injecting the watermark during the text decoding process. \citet{christ2023undetectable} used a pre-selected value sequence to sample the tokens and subsequently detected the watermark by observing the correlation between the preset pseudorandom numbers and the generated tokens. The work of \cite{kuditipudi2023robust} introduced Levenshtein distance to measure the distance between generated text and random number sequences, improving the robustness of watermarks. \citet{kirchenbauer2023watermark} divided the vocabulary into red and green lists and preferred to generate tokens from the green list.  \citet{zhao2023provable} enhanced the robustness of this approach by using a global fixed red-green vocabulary. \citet{lee2023wrote} designed a watermarking method for low-entropy code generation scenarios. However, the above methods are all public, which means the key used to generate the watermark is required during detection. This makes the watermark susceptible to removal and counterfeiting. In this work, we propose the first private watermarking method for LLM to alleviate these issues.





% % Figure environment removed

\section{Problem definition}


To facilitate subsequent discussions, this section introduces the key concepts used in this work: language models and the watermarking algorithm.
% \textbf{A language model} $\mathcal{M}$ is essentially a function for the next token prediction, which is typically implemented using neural networks. Given an input sequence $\boldsymbol{x} = [x_0,....,x_{n-1}]$, it outputs the probability of the next token $\textbf{P}(x_n)$ over the vocabulary $\mathcal{V}$: $\textbf{P}(x_n)=\mathcal{M}(x_n |\boldsymbol{x}_{1:n-1})$. The next token to be generated is then selected from this probability distribution, which can be achieved through sampling decode, choosing the token with the highest probability (greedy decode), or using other decode algorithms such as beam search to select a list of tokens with the highest probability.

% \textbf{A  watermarking algorithm} is the combination of two interconnected algorithms: the watermark generation algorithm and the watermark detection algorithm.
% \begin{itemize}
%     \item \textbf{Watermark Generation}: This can be seen as a subtle modification to a language model's probability distribution. If $\hat{\mathcal{M}}$ denotes a watermarked language model, the token prediction probability is expressed as 
% $ {\bf p}_n:=P_{\hat{\mathcal{M}}(\boldsymbol{x})}[ x_n = \cdot | \boldsymbol{x}_{1:n-1} ]. $
%     \item \textbf{Watermark detection:} This algorithm takes a text sequence $\boldsymbol{x} = [x_0, \dots, x_{n}]$ and determines the presence of a watermark. There's a one-to-one correspondence between the watermark detection model, denoted as ${\sf Detect}$, and the watermarked language model $\hat{\mathcal{M}}$.
% \end{itemize}
\textbf{Language Model:} A language model, denoted by $\mathcal{M}$, is fundamentally a function for predicting the next token in a sequence. Typically realized through neural networks, $\mathcal{M}$ takes an input sequence $\vx = [x_0, \ldots, x_{n-1}]$ and produces the probability distribution $P_n$ over a vocabulary set $\mathcal{V}$. Formally, we have $P_n = \mathcal{M}(\vx_{0:n-1})$. The subsequent token is selected from this distribution, either via sampling methods, greedy decoding, or alternative decoding algorithms such as beam search.

\textbf{Watermarking Algorithm:} This construct comprises two synergistic algorithms—the watermark generation algorithm and the watermark detection algorithm.
\begin{itemize}
    \item \textbf{Watermark Generation:} This can be seen as a subtle modification to a language model's probability distribution. If $\hat{\mathcal{M}}$ denotes a watermarked language model, the token prediction probability is expressed as: $\hat{P}_n := \hat{\mathcal{M}}(\boldsymbol{x}_{0:n-1})$.
    \item \textbf{Watermark Detection:} This algorithm takes a text sequence $\boldsymbol{x} = [x_0, \dots, x_{n}]$ and determines the presence of a watermark. There's a one-to-one correspondence between the watermark detection model, denoted as $\mathcal{D}$, and the watermarked language model $\hat{\mathcal{M}}$.
\end{itemize}

\section{Proposed Method}

This section provides a detailed description of our private watermark algorithm. The watermarked LLM decoding step is first explained (Section \ref{sec:step}). We then detail the watermark generation network (Section \ref{sec:generate}) and principles of watermark detection (Section \ref{sec:detect}). Next, we describe the detection network (Section \ref{sec:net}). Finally, we analyze the privacy of the algorithm (Section \ref{sec:ana}).

\subsection{Watermarked Large Language Model}
\label{sec:step}

% Algorithm \ref{alg:wm} outlines the process for generating watermarked text using a target language model $\mathcal{M}$. Given input $vx = [x_0....x_{n-1}]$, $\mathcal{M}$ generates logit scores $P(x_n)$ for the next token. The top $K$ tokens based on the probability are input to the watermark generation network $\mathcal{W}$, which determines the watermarked token set $G$. The probabilities of tokens in $G$ are increased by $\delta$, while others remain unchanged. The modified logits form the output of the watermarked model $\hat{\mathcal{M}}$.

Algorithm \ref{alg:wm} describes how to generate watermarked text with a target language model $\mathcal{M}$. Given an input sequence $\vx = [x_0, \ldots, x_{n-1}]$, $\mathcal{M}$ produces logit scores $P_n$ for the next token. Afterwards, the candidate tokens will be processed by the watermark generation network $\mathcal{W}$ to produce the watermarked token set G.  Logit scores for tokens in $G$ are incremented by $\delta$, while others remain unchanged, forming the logits of the modified model $\hat{\mathcal{M}}$.

Not all tokens in the logits are labeled during each generation step. In the case of top-$K$ sampling, only the highest-ranking $K$ tokens are tagged. During beam search with a beam size of $B$, tokens are labeled only if their scores surpass the $B$th highest score $S_B$ by a margin of $\delta$, formally defined as $\{x_i \in \mathcal{V} \mid P_n^{(i)} > S_B - \delta \}$. Due to that our watermark generation network could compute token labels in batches and with its minimal parameter size, even with a large K (e.g. all vocabulary), computations can be conducted swiftly. A more detailed analysis is provided in the appendix.

% Note that not all tokens require labeling by $\mathcal{W}$ during generation. With top-$K$ sampling, only the top $K$ tokens are tagged. For beam search with beam size $B$, tokens with scores exceeding the $B$th largest score $S_B$ by $\delta$ are tagged: $\mathbf{x}_n^{B} = { x_i \in \mathcal{V} \mid P(x_n)^{(i)} > S_B - \delta } $.


\begin{algorithm}[tbh]
   \caption{Watermark Generation Step}
   \label{alg:wm}
\begin{algorithmic}[1]
   \STATE {\bfseries Input:}  Watermark generation network: $\mathcal{W}$.
 Fixed integer: $K$.  Watermark strength: $\delta$. Language model: $\mathcal{M}$.  Previously generated text sequence: $\vx = [x_0, \ldots, x_{n-1}]$.  Local window size: $w$.
   \STATE Compute logits $P_n$ of token $x_n$ given $\vx_{0:n-1}$: $P_n =\mathcal{M}(\vx_{0:n-1})$.
   \STATE Extract the top-$K$ tokens from logits, denoting them as $\vx_n^K = \textsf{topK}(P_n)$.
   \STATE Initialize sequence matrix $\mathbf{S}$ where each row corresponds to the sequence $[x_{n-w+1},\ldots,x_n^{(i)}]$ for every $x_n^{(i)} \in \vx_n^K$.
   \STATE Calculate the watermark result $\vr = \mathcal{W}(\mathbf{S})$, where each entry $\vr_i$ is 0 ($x_n^{(i)}$ is not watermarked) or 1 ($x_n^{(i)}$ is watermarked).
   \STATE Generate the watermarked token set $G$ containing items with value 1 in $\mathbf{\vr}$: $G = \{x_n^{(i)} | \vr_i = 1\}$.
   \STATE Construct an augmented language model $\hat{\mathcal{M}}$ such that for an input sequence $\vx = [x_0, \ldots, x_{n-1}]$, the adjusted logits are:
   \[
   \hat{\mathcal{M}}(\vx_{0:n-1})^{(i)} = \mathcal{M}(\vx_{0:n-1})^{(i)} + \delta \mathbf{1}(i \in G),
   \]
where $\mathbf{1}(\cdot)$ is an indicator function: it returns 1 if $i \in G$ and 0 otherwise.
   \STATE {\bfseries Output:} Modified language model $\hat{\mathcal{M}}$.
\end{algorithmic}
\end{algorithm}


% \begin{algorithm}[tbh]
%    \caption{Watermark Generation Step (Top K sampling)}
%    \label{alg:wm}
% \begin{algorithmic}[1]
%    \STATE {\bfseries Input:} Watermark generation network $\mathcal{W}$, fixed integer $K$, watermark strength $\delta$, language model $\mathcal{M}$, previously generated text sequence $\vx = [x_0, \ldots, x_{n-1}]$, local window size $w$.
%    \STATE Compute the logits $P_n$ for token $x_n$ conditioned on $\vx_{0:n-1}$: $P_n =\mathcal{M}(x_n |\vx_{0:n-1})$.
%    \STATE Obtain the top-$K$ tokens in logits, denoted as $\vx_n^K = \textsf{topK}(P_n)$.
%    \FOR{$x_n^{(i)} \in \vx_n^K$}
%    \IF{$\mathcal{W}([x_{n-w+1},\ldots,x_n^{(i)}]) = 1$}
%     \STATE  Include token $x_n^{(i)}$ in the watermarked token set $G$.
%     \ENDIF
%    \ENDFOR
%    \STATE  Construct a new language model $\hat{\mathcal{M}}$ such that for an input sequence $\vx = [x_0, \ldots, x_{n-1}]$, the modified logits are given by:
%    $$\hat{\mathcal{M}}(x_n |\vx_{0:n-1})^{(i)} = \mathcal{M}(x_n |\vx_{0:n-1})^{(i)} + \delta \mathbf{1}(i \in G),$$
% where $\mathbf{1}(\cdot)$ is an indicator function that returns 1 if $i \in G$ and 0 otherwise.
%    \STATE {\bfseries Output:} Watermarked language model $\hat{\mathcal{M}}$.
% \end{algorithmic}
% \begin{algorithmic}[1]
%    \STATE {\bfseries Input:} a watermark generation network $\mathcal{W}$, a fixed number $K$, watermark strength $\delta$, a language model $\mathcal{M}$, previous generated text $vx = [x_0....x_{n-1}]$, local window size $w$.
%    \STATE Generate the token logits of $x_n$: $P_n =\mathcal{M}(x_n |\boldsymbol{x}_{1:n-1})$.\\
%    \STATE Get the top K tokens in logits $\vx_n^K = \sf{topK}$$(P(x_n))$.\\
%    \FOR{$x_{n}^{(i)}$ $\in$ $\vx_n^K$}
%    \IF{$\mathcal{W}([x_{n-w+1},...,x_{n}^{(i)}]) = 1$}
%     \STATE  Add the token $x_{n}^{(i)}$ to the watermarked token set $G$.
%     \ENDIF
%    \ENDFOR
% \STATE  Define a new language model $\hat{\mathcal{M}}$ where given input $\boldsymbol{x} = [x_0....x_{n-1}]$, the resulting logits satisfy $$\hat{\mathcal{M}}(x_n |\boldsymbol{x}_{1:n-1})^{(i)} := \mathcal{M}(x_n |\boldsymbol{x}_{1:n-1})^{(i)} + \delta \mathbf{1}(i\in G),$$
% where $\mathbf{1}(\cdot)$ is the indicator function that returns 1 if $i \in G$ and 0 otherwise.
%    \STATE {\bfseries Output:} Watermarked language model $\hat{\mathcal{M}}$.
% \end{algorithmic}
% \end{algorithm}

\subsection{Watermark Generation Network}

\label{sec:generate}

The watermark generation network's architecture is depicted in the middle part of Figure \ref{fig:intro}. The shared embedding network, denoted as $E$, first generates the embedding for each input token. Subsequently, embeddings from a local window $w$ are concatenated and processed by the fully connected classification network to ascertain if the last token in the watermarked token set:

% The structure of our watermark generation network is illustrated in the middle part of figure \ref{fig:intro}. The embedding of each input token is first generated by the shared embedding network $\mathbf{E}$.
% Then, the embeddings within a local window $w$ are concatenated and fed into the subsequent full connected classification network to determine if the last token belongs to the watermarked token set:

\begin{equation}
    \mathcal{W}(\boldsymbol{x}_{n-w+1:n}) = \text{FFN}(\mathrm{E}(x_{n-w+1})\oplus \mathrm{E}(x_{n-w+2})\oplus ....\oplus\mathrm{E}(x_n)).
\end{equation}

The embedding network accepts the binary representation of token IDs as input. The required number of encoding bits is contingent upon the vocabulary size. For instance, GPT2 \cite{radford2019language} has a vocabulary size $|\mathcal{V}|$ of 50,000, requiring 16 bits for its binary representation. 

% The embedding network is a fully connected network and its input is the binary representation of token IDs, where the number of encoding bits depends on the size of the vocabulary. For example, GPT2 \cite{radford2019language} has a vocabulary size $|\mathcal{V}|$ of 50,000, which requires 16 bits for its vocabulary representation. Common language models typically require bits between 15 and 17 for binary vocabulary representations.

For effective watermark detection, it's imperative that the proportion of watermarked tokens generated by the generation network remains invariant. Specifically, for any local window  $[x_{n-w+1}, \ldots, x_{n}]$, the probability that $x_n$ belongs to the watermarked token set should consistently be \( \gamma \):
% To facilitate the subsequent watermark detection, the proportion of the watermarked token set generated by the watermark generation network requires to remain constant.  Specifically, for any local window prefix $[x_{n-w+1}, \ldots, x_{n-1}]$, the probability of $x_n$ belongs to the watermarked token set is always a fixed value $\gamma$:
\begin{equation}
    \forall [x_{n-w+1}, \ldots, x_{n-1}], P(\mathcal{W}([x_{n-w+1}, \ldots, x_{n-1}, x_n]) = 1) = \gamma,
\end{equation}

where the $\gamma$ has the same meaning as the green list ratio in the previous public watermark algorithms \cite{kirchenbauer2023watermark, zhao2023provable}.
% However, due to the black-box nature of neural networks, it is challenging to get a fixed ratio by pre-defined parameters. We achieve this by constructing a training dataset strictly with the desired proportion $\gamma$. It's worth noting that this method does not guarantee that the ratio of green to red will be strictly the same under every local window. Still, the expected value of this ratio is $\gamma$, and there is also a standard deviation $\sigma$. We will show the standard deviation $\sigma$ only has a very slight impact on the final detection process in the following section \ref{sec:detect}.

Due to the inherent complexity of neural networks, maintaining a static ratio via predefined parameters is non-trivial. We address this challenge by curating a training dataset maintaining the precise proportion \( \gamma \). This approach guarantees an expected watermarked token ratio \(\gamma\) with standard deviation \(\sigma\). Section \ref{sec:detect} shows the minimal influence of \(\sigma\) on the detection.

\subsection{Watermark Detection}
\label{sec:detect}

In this section, we introduce how to detect a given watermark using the z-value test. 
Given a vocabulary partitioned into watermarked and non-watermarked tokens based on a fixed ratio, $\gamma$, the expected number of tokens from the watermarked set in a standard text of length $T$ is $\gamma T$, with a variance of $\gamma (1-\gamma)T$. Using the z-value test method as proposed by \citet{kirchenbauer2023watermark},  we reject the null hypothesis and detect the watermark in text if the z-score below surpasses a threshold:
\begin{align} 
z = (|s|_G - \gamma T)/\sqrt{T\gamma(1-\gamma)}.
\end{align}
However, as discussed previously, our watermark generation network does not ensure a constant ratio  $\gamma$. Instead, a ratio $\hat{\gamma}$ is achieved with an expected value $\gamma$ and a standard deviation $\sigma$. This necessitates a modification of the earlier formula. While the expected count of the green tokens remains $\gamma T $, the variance must be adjusted. Utilizing the law of total variance, the updated variance can be expressed as:
\begin{align} 
Var(\gamma T) = E[Var(\gamma T|\gamma)] + Var(E[\gamma T|\gamma]) = \gamma (1-\gamma)T + \sigma^2  T, 
\end{align}
and the new z-score could be calculated as follows:
\begin{align} 
\label{new-z}
z = (|s|_G - \gamma T)/\sqrt{\gamma (1-\gamma)T + \sigma^2  T}.
\end{align}
Since our standard deviation $\sigma$ is very small in practice, the increase in variance, $\sigma^2 T$, is also quite minimal. In the process of subsequent experiments, we will initially estimate the variance of the generation network and then include the variance during the z-score test calculation.

\subsection{Watermark Detection Network}
\label{sec:net}

The z-value test effectively detects watermarks in text but requires knowing the label (watermarked or not) of each token during detection. This allows the watermark to be more easily removed or forged. To address this, we propose a watermark detection neural network that accepts only the text sequence as input and outputs whether it contains a watermark.

The detailed structure of our watermark detection network is illustrated in the right part of Figure \ref{fig:intro}. The input to the entire network is the ID sequence of all tokens in the target sentence, where an output of 1 indicates the presence of a watermark in the entire sentence, and 0 signifies its absence. 

Specifically, all tokens first pass through a shared embedding network. Notably, the parameters of this token embedding are consistent with those of the watermark generation network and remain frozen during subsequent training. The motivation behind this novel approach is the shared embedding could give prior information to the detection networks and substantially reduce the difficulty of training.

The token embeddings are then combined and fed into an LSTM network \cite{hochreiter1997long} for binary classification of whether the text contains a watermark:
\begin{equation}
    \mathcal{D}(\boldsymbol{x}) = \text{LSTM}(\mathrm{E}(x_0)\oplus \mathrm{E}(x_1)\oplus ....\oplus\mathrm{E}(x_{T})).
\end{equation}
The watermark detection network functions as a discriminator, judging if the z-value of input text exceeds a threshold. We construct the training data using equation \ref{new-z} with a predefined threshold $z$. 

% Specifically, we sample texts with varying proportions of watermarked tokens, assigning a 0 or 1 label based on the resulting z-value relative to the threshold.

Notably, the input of the detection network need not be meaningful text; any token ID list suffices. Thus, the randomly generated ID sequence as training data theoretically avoids out-of-domain issues.

To impede attackers from inferring watermarking rules, we treat the text as a cyclic document, labeling initial tokens based on trailing ones. This results in random labels for the first few tokens, minimizing their effect on detection given the small window relative to full text length.

% The entire watermark detection network could be viewed as a discriminator to judge whether the z-value of a given input text is greater or less than a certain threshold. Therefore, we use equation \ref{new-z} with a certain threshold to construct the training dataset. Specifically, during the training dataset construction, we sample texts with different proportions of green tokens and then assign a label of 0 or 1 depending on whether the calculated z-value exceeds a certain threshold. 

% It should be noted that the input for the training of the entire watermark detection network does not need to be a meaningful text - any number ID list is acceptable. Therefore, the detection model trained in this way theoretically will not encounter out-of-domain issues. We will further illustrate this point in subsequent experiments.

% Moreover, under normal circumstances, the first $w-1$ tokens of a string of text sequences are usually not labeled as red or green. To make it more difficult for attackers to infer the watermark generation rules from the watermark detection network, we also labeled the first w-1 tokens by treating the text as a cyclic document connected head-to-tail. For instance, we can determine the label for $x_0$ through $x_{n-w+1}....x_{n}$. Normally, the labels of the first w-1 tokens are usually random, but since the window size is much smaller than the overall length of the text, this can be neglected in the overall watermark detection.

\subsection{Analysis of the Privacy}
\label{sec:ana}

% To demonstrate the efficacy of our private watermarking algorithm in obfuscating the watermark generation process, we analyzed the difficulty of deducing the watermark generation rules from the watermark detection network.

% The most direct way to crack watermarks is to use the detection network to construct training data and reversely train the generation network. However, directly leveraging the detection network to assemble training data for the generation network presents a significant obstacle: the inability to acquire precise labels. Specifically, the detection network yields only an aggregate of multiple outputs from the watermarking network, represented as $\sum_{i}^{T}{\mathcal{W}(\mathbf{x}_{i:i+w})}>\gamma T$ if text is watermarked, rather than precise labels for the generation network. This lack of accurate labeling complicates the training of the generation network, even when both networks share token embedding parameters. The intricacies of this process will be further elaborated in the experiment section. This characteristic adequately satisfies computational asymmetry, which is vital for privacy-preserving fields like asymmetric cryptography. The theoretical training difficulty and computational asymmetry inherent in training the generation network from the detector network robustly safeguard the privacy of our algorithm.

To further demonstrate the privacy of our watermarking algorithm, this section provides a detailed analysis of the difficulty in cracking the watermark generation method.

In the process of watermark generation, we use watermark generation networks to produce labels for training watermark detection networks. This process is quite straightforward. For a given text $\vx$, the gold label of the detection network $\mathcal{D}$ is a simple function of watermark generation network $\mathcal{W}$:
\begin{equation}
\mathcal{D}(\vx) = f\left(\mathcal{W}(\vx_{0:w}), \mathcal{W}(\vx_{1:w+1}), ..., , \mathcal{W}(\vx_{T-w:T})\right).
\end{equation}

Conversely, utilizing the watermark detection network to produce training labels for the watermark generation network presents substantial challenges. Given the input text $\vx$ and the detection network $\mathcal{D}$, it is not straightforward to obtain a specific label $\mathcal{W}(\mathbf{x}_{i:i+w})$. The inference is only possible with respect to the relationships among multiple labels. This necessitates accounting for output interdependencies within the generation network, as shown below:
\begin{equation}
\mathcal{W}(\vx_{i:i+w}) = g\left(\vx, \mathcal{D}, \{ \mathcal{W}(\vx_{j:j+w}) \}_{j \neq i} \right).
\end{equation}
The uncertainty and complex dependencies between labels make training watermark generation networks from watermark detection networks extremely difficult. The intricacies of this process will be further elaborated in the experiment section. This characteristic adequately satisfies computational asymmetry, which is vital for privacy-preserving fields like asymmetric cryptography. The theoretical training difficulty and computational asymmetry inherent in training the generation network from the detection network robustly safeguard the privacy of our algorithm.

In addition to using the watermark detection network to train the generation network, one can infer the watermark rules by analyzing word frequency differences across large amounts of text, i.e. the spoofing attack mentioned by \cite{sadasivan2023can}.  The principle is that when using a watermarking method with a window size $w$, the frequency of the $w$th token with fixed previous $w-1$ tokens will differ from that in unwatermarked text. However, this approach is largely ineffective when the window size is not particularly small, as will be detailed in the experiments section.




\section{Experiment}

In this section, we validate our private watermark algorithm through extensive experiments. 


\subsection{Experiment Setup}


\documentclass[a4paper,11pt]{article}
\pdfoutput=1 % if your are submitting a pdflatex (i.e. if you have
             % images in pdf, png or jpg format)

%\usepackage[utf8]{inputenc}
%\usepackage{mathrsfs, amssymb, amsmath}  
%\usepackage{comment}
%\usepackage{dcolumn}
%\usepackage{multirow}
%\usepackage{color}
%\usepackage{amsfonts,amssymb,amsmath, txfonts}
%\usepackage{float}

\usepackage{jcappub} % for details on the use of the package, please
                     % see the JCAP-author-manual

\usepackage[T1]{fontenc} % if needed

\hypersetup{ linktoc=all,
    colorlinks=true, linkcolor={blue},  
       citecolor={red}, urlcolor={darkred}
}
\definecolor{Redgreen}{RGB}{153,76,0}
\definecolor{vividviolet}{rgb}{0.62, 0.0, 1.0}
\definecolor{green}{RGB}{11,98,17}
\definecolor{darkgreen}{RGB}{40,150,65}
\definecolor{darkblue}{rgb}{0,0,0.3}
\definecolor{darkred}{rgb}{0.7,0,0}

\def\blue{\textcolor{blue}}
\def\red{\textcolor{red}}
\def\be{\begin{equation}}
\def\ee{\end{equation}}
\def\bea{\begin{eqnarray}}
\def\eea{\end{eqnarray}}


\title{MCMC Marginalisation Bias and $\Lambda$CDM tensions}
%\title{Overcoming bias in MCMC marginalisation to elucidate $\Lambda$CDM tensions}
%\title{Temp}

%%Markov Chain Monte Carlo

%% %simple case: 2 authors, same institution
%% \author{A. Uthor}
%% \author{and A. Nother Author}
%% \affiliation{Institution,\\Address, Country}

% more complex case: 4 authors, 3 institutions, 2 
\author[a]{Eoin \'O Colg\'ain}
\author[b]{Saeed Pourojaghi}
\author[b, c]{M. M. Sheikh-Jabbari}
\author[a]{Darragh Sherwin}

% The "\note" macro will give a warning: "Ignoring empty anchor..."
% you can safely ignore it.

\affiliation[a]{Atlantic Technological University, Ash Lane, Sligo, Ireland}
\affiliation[b]{School of Physics, Institute for Research in Fundamental Sciences (IPM), P.O.Box 19395-5531, Tehran, Iran}
\affiliation[c]{The Abdus Salam ICTP, Strada Costiera 11, I-34014 Trieste, Italy}

% e-mail addresses: one for each author, in the same order as the authors
\emailAdd{eoin.ocolgain@atu.ie}
\emailAdd{pourojaghi@ipm.ir}
\emailAdd{jabbari@theory.ipm.ac.ir}
\emailAdd{darragh.sherwin@research.atu.ie}




\abstract{Probability distributions become non-Gaussian when the flat $\Lambda$CDM model is fitted to redshift binned data in the late Universe. We explain mathematically why this non-Gaussianity arises and confirm that Markov Chain Monte Carlo (MCMC) marginalisation leads to biased inferences in observational Hubble data (OHD). In particular, in high redshift bins we find that $\chi^2$ minima, as identified from both least squares fitting and the MCMC chain, fall outside of the $1 \sigma$ confidence intervals. We resort to profile distributions to correct this bias. Doing so, we observe that $z \gtrsim 1$ cosmic chronometer (CC) data currently prefers a non-evolving (constant) Hubble parameter over a Planck-$\Lambda$CDM cosmology at $\sim 2 \sigma$. We confirm that both mock simulations and profile distributions agree on this significance. Moreover, on the assumption that the Planck-$\Lambda$CDM cosmological model is correct, using profile distributions we confirm  a $> 2 \sigma$ discrepancy with Planck-$\Lambda$CDM in a combination of  CC and baryon acoustic oscillations (BAO) data beyond $ z \sim 1.5$ that was noted earlier through comparison of least square fits of observed and mock data.}



\begin{document}
\maketitle
\flushbottom

\section{Introduction}
\label{sec:intro}
The flat $\Lambda$CDM model is the minimal model that fits Cosmic Microwave Background (CMB) data. Remarkably, CMB data from the Planck satellite \cite{Planck:2018vyg} constrains the $\Lambda$CDM model to sub-percent errors, thereby not only providing the strongest constraints, but also a concrete prediction for cosmological probes in the late Universe. The unmitigated success of the $\Lambda$CDM model is that CMB, Type Ia supernovae (SN) \cite{Riess:1998cb, Perlmutter:1998np} and baryon acoustic oscillations (BAO) \cite{Eisenstein:2005su} agree on a $\Lambda$CDM Universe that is approximately $30 \%$ matter. Thus, one key prediction of the Planck-$\Lambda$CDM model agrees across early and late Universe cosmological probes. Given this non-trivial agreement, any discrepancies that arise elsewhere constitute challenging puzzles. 

Nevertheless, one cannot define any \textit{model} for a dynamical system, especially a complicated system like the Universe, using data from a cosmic snapshot.\footnote{Here, we mean CMB data with an effective redshift $z \sim 1100$.} At best, one has a \textit{prediction} and not a model. In recent years, key predictions of Planck data have been challenged by late Universe determinations of the Hubble constant $H_0$ \cite{Riess:2021jrx, Freedman:2021ahq, Pesce:2020xfe, Blakeslee:2021rqi, Kourkchi:2020iyz} and the $S_8:= \sigma_8 \sqrt{\Omega_m/0.3}$ parameter \cite{HSC:2018mrq, KiDS:2020suj, DES:2021wwk, Boruah:2019icj, Said:2020epb}. Given the diversity of the late Universe probes (see reviews \cite{Perivolaropoulos:2021jda, Abdalla:2022yfr}), it is highly unlikely that any single systematic can be found to explain the discrepancies. That being said, in astrophysics one can never preclude systematics; 3 decades after Phillips' seminal paper \cite{Phillips:1993ng}, we are still debating an ad hoc correction for the mass of the host galaxy in Type Ia SN \cite{NearbySupernovaFactory:2018qkd, Kang:2019azh, Brout:2020msh, Lee:2021txi}. Bearing in mind that Type Ia SN are one of our best understood cosmological probes, one quickly understands that any systematics debate may be endless. 

Thus, it is far more expedient to assume that the $\Lambda$CDM model is breaking down and to look for tell-tale signatures of model breakdown. If signatures cannot be found, one arrives at a contradiction, and revisits the assumption that the model is breaking down. For physicists, \textit{model breakdown comes about when model fitting parameters return discrepant values at different time slices or epochs}. Translated into astronomy, this equates to discrepant cosmological parameters in different redshift ranges. The usual $H_0, S_8$ tensions  may also be viewed in the same light: a discrepancy between high and low redshift inferences/measurements of the parameters \cite{Perivolaropoulos:2021jda, Abdalla:2022yfr}. Nevertheless, early and late Universe observables are typically not the same, so one is confronted with a rich set of potential systematics. 

Within the context of $\Lambda$CDM tensions, it was recently observed that the integration constant from the Friedmann equations, aka the Hubble constant $H_0$, picks up redshift dependence whenever our model assumption - required to close the Friedmann equations - disagrees with the Hubble parameter $H(z)$ extracted from observations \cite{Krishnan:2020vaf, Krishnan:2022fzz}. %\footnote{One is free to speculate about the nature of the missing physics \cite{Liao:2020zko, Montani:2023xpd}.} 
Similarly, $\rho_{m0}=H_0^2\Omega_m$, an integration constant of the matter continuity equation, implies matter density $\Omega_m$ is a mathematically constant quantity. 
These are irrefutable predictions from mathematics, i. e. a prediction that is \textit{robust to systematics}. However, observationally $H_0$ and $\Omega_{m}$ are model fitting parameters and nothing precludes them picking up redshift dependence (except of course if one assumes they do not!), and providing a signature of model breakdown. If this happens in the late Universe within the $\Lambda$CDM model, $H_0$ is correlated with matter density $\Omega_m$, 
while $\Omega_m$ is correlated with $S_8 \propto \sigma_8 \sqrt{\Omega_m}$. Thus, there is at least one simple scenario, namely redshift evolution of cosmological parameters in the late Universe, where ``$H_0$ tension'' and ``$S_8$ tension'' are not independent and simply symptoms of $\Lambda$CDM model breakdown. 

The next relevant question is, where is the evidence for evolving cosmological parameters in the late Universe? Starting with strong lensing time delay \cite{Wong:2019kwg, Millon:2019slk},\footnote{Systematics are explored in \cite{Millon:2019slk} and the descending trend is not an obvious systematic. The lensed system RXJ1131-1231 \cite{Sluse:2003iy}, which partly drives the trend, has recently been re-analysed using spatially resolved stellar kinematics of the host galaxy \cite{Shajib:2023uig}, and the higher $H_0$ value remains robust, admittedly with inflated errors. As TDCOSMO project to analyse 40 lenses, the prospect of a discovery of a descending $H_0$ trend assuming the $\Lambda$CDM model remain strong.} descending trends of $H_0$ with redshift have been reported in Type Ia SN \cite{Dainotti:2021pqg, Colgain:2022nlb, Colgain:2022rxy,  Malekjani:2023dky, Hu:2022kes, Jia:2022ycc} and combinations of data sets \cite{Krishnan:2020obg, Dainotti:2022bzg}. On the other hand, larger values of $\Omega_m$ have been noted in high redshift observables, primarily quasars (QSOs) \cite{Risaliti:2015zla, Risaliti:2018reu, Lusso:2020pdb, Yang:2019vgk, Khadka:2020vlh, Khadka:2020tlm, Khadka:2021xcc, Pourojaghi:2022zrh},\footnote{Just as with Type Ia SN, the systematics of QSOs are being investigated \cite{Zajacek:2023qjm}.} but also Type Ia SN \cite{Colgain:2022nlb, Colgain:2022rxy, Malekjani:2023dky, Pasten:2023rpc} (see also \cite{Wagner:2022etu, Sakr:2023hrl}). Note, as emphasised earlier, if $H_0$ evolves at the background level, correlated fitting parameters are expected to also evolve. Moreover, mock analysis within the $\Lambda$CDM setting reveals that evolution of best fit $(H_0, \Omega_m)$ parameters cannot be precluded, and conversely possesses a finite likelihood, in either observational Hubble data (OHD) \textit{or} angular diameter distance data \textit{or} luminosity distance data \cite{Colgain:2022tql}. We stress that this result \textit{rests on mock analysis}; it represents a purely mathematical statement about the $\Lambda$CDM model that is independent of systematics. 

Separately, at the perturbative level, redshift evolution of $S_8$ or $\sigma_8$ has been reported in galaxy cluster number counts and Lyman-$\alpha$ spectra \cite{Esposito:2022plo}, $f \sigma_8$ constraints from peculiar velocities and redshift space distortions (RSD) 
 \cite{Adil:2023jtu}, comparison between weak \cite{HSC:2018mrq, KiDS:2020suj, DES:2021wwk} and CMB lensing \cite{ACT:2023dou, ACT:2023kun}. What is important here is that these observations appear to restrict the evolution in $S_8$ to the late Universe. In \cite{ACT:2023ipp} the possibility was raised that \textit{``tracers at higher redshift and probing larger scales prefer higher $S_8$''}.\footnote{There are also conflicting observations of high redshift $\sigma_8$ or $S_8$ values that are lower than Planck in the late Universe \cite{Miyatake:2021qjr, Alonso:2023guh}, so either this trend is not universal, or systematics are at play.} Nevertheless, one can argue against evolution with scale on the grounds that cosmic shear \cite{HSC:2018mrq, KiDS:2020suj, DES:2021wwk}, which is sensitive to smaller scales (larger $k$), and peculiar velocity constraints \cite{Boruah:2019icj, Said:2020epb}, which are sensitive to larger scales (smaller $k$), both prefer lower values of $S_8$. Moreover, both galaxy clusters and Lyman-$\alpha$ spectra are expected to probe similar scales.\footnote{We thank Matteo Viel for correspondence on this point.} Thus, if systematics are not impacting results, then redshift evolution is the only point of agreement in the observations \cite{Esposito:2022plo, Adil:2023jtu, HSC:2018mrq, KiDS:2020suj, DES:2021wwk, ACT:2023dou, ACT:2023kun, ACT:2023ipp}. Note also that redshift is more fundamental than scale in FLRW cosmology; one must solve the Friedmann equations in either time or redshift before one contemplates any discussion of scale.  

 The purpose of this letter is to revisit the analysis presented in \cite{Colgain:2022rxy,Colgain:2022tql}, where the evidence for evolution was quantified on the basis of mock simulations and not Markov Chain Monte Carlo (MCMC), the technique most familiar in cosmology. The fundamental problem is that once one bins low redshift data and studies evolution of cosmological parameters with bin redshift, one quickly encounters projection effects in MCMC analyses. These effects are not just the preserve of exotic models \cite{Herold:2021ksg, Gomez-Valent:2022hkb, Meiers:2023gft}, such as Early Dark Energy (EDE) \cite{Poulin:2018cxd, Niedermann:2019olb}, and happen in the simplest model when one bins data. The most striking demonstration of the resulting bias is that the peaks of MCMC posteriors no longer coincide with the minimum of the likelihood (see \cite{Gomez-Valent:2022hkb}). Ultimately, this bias is expected  because one is working in a regime of the $\Lambda$CDM model with non-Gaussian probability distributions   \cite{Colgain:2022tql}  (see also \cite{Colgain:2022rxy}).

 The structure of this paper is as follows. In section \ref{sec:MCMC_bias} we confirm the bias in MCMC marginalisation. In section \ref{sec:PD} we introduce profile distributions (PDs) \cite{Gomez-Valent:2022hkb} as a means of addressing the bias and confirm that the statistical significance of discrepancies from mock simulations agree well with PD analysis. In section \ref{sec:tension}, we revisit and confirm the high redshift OHD tensions reported in \cite{Colgain:2022rxy}. We end in section \ref{sec:discussion} with concluding remarks. 
 %A short appendix is also added on Fisher matrix for $\Lambda$CDM mdoel. 

\section{A bias in MCMC marginalisation}
\label{sec:MCMC_bias}
In this section we illustrate a bias in MCMC marginalisation that arises in the (flat) $\Lambda$CDM model when data is binned by redshift. This bias can be traced to a regime of the $\Lambda$CDM model with non-Gaussian distributions and is independent of systematics  \cite{Colgain:2022rxy, Colgain:2022tql}. 

\subsection{Mathematical Foundations}
\label{sec:math}
Consider an exercise where one bins OHD and confronts it to the $\Lambda$CDM Hubble Parameter $H(z)$ in the late Universe, a setting where the radiation sector can be safely decoupled. In high redshift bins ($z \gg 0$) in the matter-dominated regime, the Hubble parameter becomes insensitive to the dark energy (DE) sector: 
\be
\label{eq:lcdm}
H(z) = H_0 \sqrt{1-\Omega_m + \Omega_m (1+z)^3} \xrightarrow[z \gg 0]{} H_0 \sqrt{\Omega_m} (1+z)^{\frac{3}{2}}.  
\ee
More concretely, taking $z \rightarrow \infty$ we see that data can only constrain the combination $\rho_{m0}=H_0^2{\Omega_m}$. For \textit{hypothetical} data in a redshift bin with effective redshift $z = \infty$, this means that one can only constrain the combination $\Omega_m h^2$ ($h:= H_0/100)$, but $H_0$ and $\Omega_m$ remain unconstrained. Alternatively put, for any given $\Omega_m h^2$ constraint, there is an infinite number of corresponding $(H_0, \Omega_m)$ pairs. Translated into a probability density function (PDF), this is simply the statement that in a very high redshift bin at $z = \infty$, one expects uniform or flat distributions for $H_0$ and $\Omega_m$ with the model (\ref{eq:lcdm}).  

Of course, observed data resides at finite $z$ and not $z = \infty$. As a result, one does not encounter \textit{exactly} flat PDFs in $H_0$ and $\Omega_m$ at high redshift, but \textit{almost} flat PDFs. More important to us is the observation that these PDFs must flatten in a non-Gaussian manner. To appreciate this fact, we observe that high redshift OHD only constrains $\Omega_m h^2$ well.\footnote{Note that observables like SN or QSO that measure $D_L(z)=c (1+z)\int_0^z \textrm{d} z'/H(z')$ are mainly sensitive to the low redshift part of $H(z)$, i. e. the combination $H_0^2 (1-\Omega_m)$, and in this sense they are complementary to the OHD data which is more sensitive to high redshift part of $H(z)$, $H_0^2\Omega_m$. The complementarity can be demonstrated by combining $H(z)$ and $D_{L}(z)$ constraints and checking that one recovers mock data input parameters in all redshift bins \cite{Colgain:2022tql}. } For this reason, best fit parameters are constrained to a $\Omega_m h^2 = \textrm{constant}$ curve in the $(H_0, \Omega_m)$-plane. The almost flat $H_0$ and $\Omega_m$ PDFs can only arise if this curve stretches in the $(H_0, \Omega_m)$-plane. As a result of this stretching, one ends up with a relatively uniform distribution on a curve. At the extremes of the curve, one finds a distribution of large $H_0$ values, which do not differ greatly in $\Omega_m$, and they get projected to a peak at small values on the $\Omega_m$ axis. Conversely, at the other end of the curve, one finds a distribution of small $\Omega_m$ values, which do not differ greatly in $H_0$, and they get projected onto a peak at large values on the $H_0$ axis.  This is a ``projection effect'' in common cosmology parlance.  It is driven by the irrelevance of the DE sector at high redshift and the constraint $\Omega_m h^2 = \textrm{constant}$ from the $\Lambda$CDM model (\ref{eq:lcdm}). Together these features distort the distribution away from a Gaussian configuration. 

Thus, simply by binning and fitting OHD to the $\Lambda$CDM model one enters a non-Gaussian regime as the effective redshift of the bin increases. This effect, which is expected from the purely mathematical arguments above, has been confirmed in mock data \cite{Colgain:2022rxy, Colgain:2022tql}, and in line with expectations, we demonstrate that it impacts MCMC inferences with observed data in the next subsection.  

% Figure environment removed

\subsection{Cosmic Chronometer (CC) Data}
\label{sec:CCbias}
Here we work with OHD from the cosmic chronometer (CC) program \cite{Jimenez:2001gg}. Concretely, we work with 34 $H(z)$ constraints spanning the redshift range $0.07 \leq z \leq 1.965$ \cite{Stern:2009ep, Moresco:2012jh, Zhang:2012mp, Moresco:2016mzx, Ratsimbazafy:2017vga, Borghi:2021rft, Jiao:2022aep, Tomasetti:2023kek}. We illustrate the data in Fig.~\ref{fig:CC}, where it is consistent with Fig. 9 of \cite{Tomasetti:2023kek} {modulo the fact that we have an additional data point at $z = 0.8$, which is not independent. See Table 1.1 of \cite{Moresco:2023zys}. While CC data may eventually be good enough to arbitrate on Hubble tension \cite{Moresco:2023zys}, the data is not good enough on its own to do cosmology. To put this comment in context, we observe that the errors in Fig.~\ref{fig:CC} do not include systematic errors (see \cite{Moresco:2020fbm} for an account of the systematics). As a result the constraints we get on cosmological parameters will be underestimated. Thus, from our perspective the data in Fig.~\ref{fig:CC} is simply some representative cosmological data in the OHD class.}

\paragraph{Methodology:} We impose a low redshift cut-off on the OHD $z_{\textrm{min}}$, removing all data points with redshifts $z_i < z_{\textrm{min}}$, and then extremising the $\chi^2$ likelihood, 
\be
\label{eq:chi2}
\chi^2 = Q^{T} \cdot C^{-1} \cdot Q, 
\ee
where $C$ is the covariance matrix, which is simply the square of the $H_i$ errors on the diagonal, and $Q$ is the vector, 
\be
\label{eq:Q}
Q_i = H_i - H_{\textrm{model}}(z_i), 
\ee
where $H_i:=H(z_i)$ denotes OHD and $H_{\textrm{model}}(z)$ is the model (\ref{eq:lcdm}) without the high redshift limit. The best fit $(H_0, \Omega_m)$ parameters correspond to the minumum of the $\chi^2$, while on the assumption of Gaussian errors, we estimate the errors from a Fisher matrix (appendix \ref{sec:fisher}). In parallel, we perform MCMC marginalisation through \textit{emcee} \cite{Foreman-Mackey:2012any}. More concretely, subject to the priors $H_0 \in [0, 200 ]$ and $\Omega_m \in [ 0, 1]$, the latter restricting us to a physical regime, we record $16^{\textrm{th}}$, $50^{\textrm{th}}$ and $84^{\textrm{th}}$ percentiles for MCMC posteriors, as is common practice with Gaussian distributions. Thus, both techniques are tailored to Gaussian posteriors, yet non-Gaussianities will be evident in MCMC posteriors. By comparing the output from these two techniques in Table \ref{tab:LCDM_CC} for different values of $z_{\textrm{min}}$ we observe that error estimates from Fisher matrix and MCMC quickly disagree as $z_{\textrm{min}}$ increases. 

From Table \ref{tab:LCDM_CC}, we see that MCMC inferences lead to non-Gaussian $1 \sigma$ confidence intervals, where in line with the expectations from \cite{Colgain:2022tql}, $H_0$ errors are larger for smaller values, and $\Omega_m$ errors are larger for larger values, respectively. This is expected if the $H_0$ and $\Omega_m$ posteriors are peaked at larger and smaller values, respectively, in line with our earlier mathematical argument. Only for the full data set with $z_{\textrm{min}} = 0$  do we find reasonable agreement between the Fisher matrix and MCMC $1 \sigma$ confidence intervals. As can be seen from the lopsided MCMC confidence intervals, the non-Gaussianity becomes more pronounced with increasing $z_{\textrm{min}}$. Interestingly, beyond $z_{\textrm{min}} = 1$, the minimum of the $\chi^2$ falls outside of the MCMC $1 \sigma$ confidence intervals. Nevertheless, by evaluating the MCMC chains on the $\chi^2$ likelihood (\ref{eq:chi2}), we confirm that the parameters corresponding to the minimum $\chi^2$ value are tracking the best fit. Note, the peak of the MCMC posterior is no longer a measure of goodness of fit and inferences have become biased in a regime of model parameter space where distributions are expected to be inherently non-Gaussian. Our analysis here underscores potential problems with a blind MCMC analysis with the traditional $16^{\textrm{th}}$, $50^{\textrm{th}}$ and $84^{\textrm{th}}$ percentiles.       



\begin{table}[htb]
    \centering
    \begin{tabular}{c|c|c|c|c|c}
    \rule{0pt}{3ex} $z_{\textrm{min}}$ & \# CC & \multicolumn{2}{c}{Fisher Matrix}  & \multicolumn{2}{|c}{MCMC} \\
    \hline
    \rule{0pt}{3ex} & & $H_0$ (km/s/Mpc) & $\Omega_m$ & $H_0$ (km/s/Mpc) & $\Omega_m$ \\
    \hline
    \rule{0pt}{3ex} $0$ & $34$ & $68.14 \pm 3.07$ & $0.320 \pm 0.059$ & $67.76^{+3.03}_{-3.09}$  ($68.12$) & $0.328^{+0.065}_{-0.055}$ ($0.321$) \\
    \hline 
    \rule{0pt}{3ex} $0.2$ & $27$ & $65.03 \pm 6.65$ & $0.368 \pm 0.118$ & $63.05^{+6.64}_{-7.23}$ ($64.98$) & $0.405^{+0.170}_{-0.111}$ ($0.369$) \\
    \hline 
    \rule{0pt}{3ex} $0.4$ & $22$ & $62.42 \pm 8.38$ & $0.411 \pm 0.161$ & $59.54^{+8.30}_{-8.22}$ ($62.39$) & $0.470^{+0.229}_{-0.151}$ ($0.411$)\\
    \hline 
    \rule{0pt}{3ex} $0.6$ & $15$ & $59.83 \pm 17.21$ & $0.454 \pm 0.338$ & $56.45^{+13.16}_{-9.33}$ ($59.86$) & $0.526^{+0.288}_{-0.225}$ ($0.453$) \\
    \hline 
    \rule{0pt}{3ex} $0.7$ & $14$ & $79.11 \pm 19.40$ & $0.222 \pm 0.162$ & $67.59^{+19.19}_{-16.57}$ ($79.18$) & $0.344^{+0.344}_{-0.178}$ ($0.222$) \\
    \hline 
    \rule{0pt}{3ex} $0.8$ & $11$ & $103.97 \pm 24.94$ & $0.097 \pm 0.088$ & $82.43^{+28.33}_{-27.03}$ ($104.02$) & $0.206^{+0.357}_{-0.131}$ ($0.096$) \\
    \hline 
    \rule{0pt}{3ex} $1$ & $8$ & $150.37 \pm 31.21$ & $0.010 \pm 0.035$ & $108.92^{+33.94}_{-44.47}$ ($150.38$) & $0.087^{+0.304}_{-0.068}$ ($0.010$) \\
    \hline 
    \rule{0pt}{3ex} $1.2$ & $7$ & $154.35 \pm 42.95$ & $0.006 \pm 0.042$ & $83.07^{+48.52}_{-32.19}$ ($154.47$) & $0.194^{+0.439}_{-0.159}$ ($0.006$) \\
    \hline 
    \rule{0pt}{3ex} $1.4$ & $4$ & $125.41 \pm 79.55$ & $0.039 \pm 0.132$ & $65.32^{+44.88}_{-20.30}$ ($125.44$) & $0.320^{+0.423}_{-0.250}$ ($0.039$) \\
    \hline 
    \rule{0pt}{3ex} $1.5$ & $3$ & $36.12 \pm 72.69$ & $1.000 \pm 4.269$ & $55.19^{+34.64}_{-14.73}$ ($36.16$) & $0.393^{+0.387}_{-0.283}$ ($0.999$)
    \end{tabular}
    \caption{Comparison between Fisher matrix and MCMC analysis for CC data with a low redshift cut-off $z_{\textrm{min}}$. We record the number of data points, the extremum of the $\chi^2$ and $1 \sigma$ confidence interval estimated from the Fisher matrix,  $16^{\textrm{th}}$, $50^{\textrm{th}}$ and $84^{\textrm{th}}$ percentiles from MCMC posteriors corresponding to $1 \sigma$ confidence intervals, and the minimum $\chi^2$ from the MCMC chain in brackets. MCMC marginalisation exhibits non-Gaussian $1 \sigma$ confidence intervals, and for $z_{\textrm{min}} > 1$, the minimum value of the $\chi^2$ from the MCMC chain falls outside of this interval. The latter tracks the best fit up to small numbers in line with expectations. }
    \label{tab:LCDM_CC}
\end{table}

\subsection{Features in CC Data}
\label{sec:features}
Once one accounts for biases, it is clear from Table \ref{tab:LCDM_CC} that there are trends in CC data when it is binned. Starting from $z_{\textrm{min}} = 0$ through to $z_{\textrm{min}} = 0.6$ we see a decreasing trend in best fit values of $H_0$ (also central $H_0$ values from MCMC), which is compensated by a increasing trend in $\Omega_m$ best fit values. From Fig.~\ref{fig:CC} it is difficult to visibly discern any trend from the raw data. From $z_{\textrm{min}} = 0.7$ through to $z_{\textrm{min}} = 1.4$, there is in contrast a preference for larger $H_0$ and smaller $\Omega_m$ values. This trend is evident from the raw data, where at higher redshifts one sees large scatter and large fractional errors in the data. For $z_{\textrm{min}} = 1$, it is clear that the best fit line in magenta corresponding to $(H_0, \Omega_m) = (150.4, 0.01)$ (Table \ref{tab:LCDM_CC}) is closer to horizontal line than the Planck-$\Lambda$CDM cosmology in red. To be more explicit, for $z_{\textrm{min}} = 0$, $\rho_{m0}:=H_0^2\Omega_m\simeq 1500$ which is close to the Planck value, whereas for $z_{\textrm{min}} = 1$, $\rho_{m0}\simeq 225$. The sharp drop in $\rho_{m0}$ means the magenta line should be almost horizontal. For $z_{\textrm{min}} = 1.5$, we switch to an opposite regime of parameter space with unexpectedly low and high values of $H_0$ and $\Omega_m$, respectively, a trend which is evident in the data, but there are only three data points. Despite, the small number of data points, the tendency for smaller $H_0$ and larger $\Omega_m$ inferences within $\Lambda$CDM cosmology at high redshifts has been documented across three independent observables \cite{Colgain:2022rxy}. We will come back to this claim in section \ref{sec:tension}. Finally, it is worth noting that for large $z_{\textrm{min}}$ and samples with few data points, one expects broad MCMC posteriors. These posteriors are severely impacted by the prior on $\Omega_m$, as is evident from Table \ref{tab:LCDM_CC}. 

For the moment we leave physical speculations to the discussion and return to the trend in CC data above $z=1$ favouring less evolution in the Hubble parameter than the Planck-$\Lambda$CDM model. We would like to quantify the significance of this trend, but since we are working in a non-Gaussian regime of the model, we can expect both Fisher matrix and MCMC to give biased results. In Fig.~\ref{fig:CCsplit1} we show MCMC posteriors for $z>1$ CC data in blue alongside posteriors for low redshift ($z < 1$) CC data, which is simply added to aid comparison and also highlight the Gaussianity of the low redshift posteriors. One notes that the peaks of the $z > 1$ distributions are a little displaced from to the values minimising the $\chi^2$. However, the emergence of the lower peak in the $H_0$ posterior at $H_0 \sim 50$ km/s/Mpc has the hallmarks of a projection effect. To appreciate this, note that the configurations in the blue curve in the top left corner of the 2D posterior are projected onto the lower $H_0$ peak. Moreover, if one shifts the $H_0$ peak from $H_0 \sim 150$ to $H_0 \sim 50$ km/s/Mpc while maintaining $\Omega_m \sim 0$, this shifts the magenta curve in Fig. \ref{fig:CC} outside of all the data points, so the lower $H_0$ peak is a phantom artefact unrelated to the goodness of fit. We also observe a shift in the higher $H_0$ peak away from the minimum of the $\chi^2$.

Ignoring these features, one could attempt to interpret the overlap in the 2D posteriors in Fig. \ref{fig:CCsplit1}. Doing so, one may conclude that low and high redshift CC data are consistent within $1 \sigma$. However, since Hubble tension is a 1D problem (local $H_0$ determinations are insensitive to other parameters), to compare with locally observed values of $H_0$ one needs to project onto the $H_0$ axis. Alternatively put, Hubble tension is a problem in 1D posteriors. Projecting onto the $H_0$ axis by determining $16^{\textrm{th}}$, $50^{\textrm{th}}$ and $84^{\textrm{th}}$ percentiles, one sees from Table \ref{tab:LCDM_CC} that the $z_{\textrm{min}} = 1$ MCMC confidence interval encloses the $z_{\textrm{min}} = 0$ central values within $1 \sigma$,\footnote{Note, removing the eight high redshift data points from the $z_{\textrm{min}} = 0$ sample will not shift the central values much.} but not the point in parameter space that best fits the data!


% Figure environment removed



Evidently, given the non-Gaussian posteriors, care is required when interpreting the significance of the trend towards a non-evolving (horizontal) $H(z)$ at higher redshifts in Fig.~\ref{fig:CC}. We cannot use the errors from the Fisher matrix as we are clearly in a non-Gaussian regime, whereas MCMC inferences are impacted by projection effects to the extent that the minimum of the $\chi^2$ (confirmed from the MCMC chain) falls outside of the $1 \sigma$ confidence interval. For this reason, we resort to mock simulations. While this may seem a little redundant if we are going to employ profile distributions in section \ref{sec:PD}, there is motivation for this exercise. In \cite{Colgain:2022rxy} the significance of a descending $H_0$/increasing $\Omega_m$ trend with effective redshift in OHD, Type Ia SN and QSOs was estimated to be a $\sim 3 \sigma$ effect on the basis of combining $\sim 2 \sigma$ effects in each of the \textit{independent} data sets using Fisher's method. Here, working with the same data throughout, we can directly compare the significance of a discrepancy estimated through mock simulations from the significance of a discrepancy estimated through profile distributions. In particular, we will address the question: how significant is a constant $H(z)$ with $z_{\textrm{min}}=1$ (8 data points) against the Planck consistent cosmology favoured by the full data set ($z_{\textrm{min}}=0$ entry in Table \ref{tab:LCDM_CC})? Note, the significance will be overestimated due to missing systematic uncertainties (see \cite{Moresco:2020fbm}), but we can still make comparison between the two techniques.

\paragraph{{Mock simulations:}} To address this question using mock simulations, we begin with the MCMC chains for the full sample. For each entry in the MCMC chain (approximately 15,000 entries in total), we generate a new realisation of the 8 high redshift data points $(z > 1)$ that are by construction statistically consistent with both the best fits from the full sample and also the Planck-$\Lambda$CDM values \cite{Planck:2018vyg}. More concretely, for each $(H_0, \Omega_m)$ entry in our MCMC chain, we displace the data points to the corresponding $\Lambda$CDM Hubble parameter before generating new data points in a normal distribution where the errors serve as standard deviations. We then fit back the $\Lambda$CDM model to each realisation of the mock data and record the best fit $(H_0, \Omega_m)$ values, which give us a distribution of expected $(H_0, \Omega_m)$ best fits. The distributions are presented in Fig.~\ref{fig:CCsims} alongside the best fits from observed data. Throughout, we assume canonical values $(H_0, \Omega_m) = (70, 0.3)$ for the initial guess of the fitting algorithm. Best fits can saturate our bounds, i. e. $\Omega_m = 0$ and $\Omega_m = 1$, and this leads to an unsightly pile up of best fits at $\Omega_m = 0$ and $\Omega_m = 1$ in Fig.~\ref{fig:CCsims} \cite{Colgain:2022rxy}. It is important to retain all the configurations, otherwise one is not accounting for the probability that a best fit falls outside our priors. As a consistency check, we see that the median or 50$^{\textrm{th}}$ percentile, $(H_0, \Omega_m) = (68.32, 0.321)$ agrees well with the mock input parameters, thereby demonstrating that there are an equal number of best fits with values above and below the injected parameters in the mocks. We find that probability of a more extreme (larger) $H_0$ value to be $p = 0.022$, while the probability of a more extreme (smaller) $\Omega_m$ value to be $p = 0.035$, respectively. Converted into a Gaussian statistic, these correspond to $2 \sigma$ and $1.8 \sigma$, respectively, for a one-sided normal distribution. Thus, on the basis of mock simulations, we estimate the non-evolving constant $H(z)$ with $z_{\textrm{min}} = 1$ as a $\sim 2 \sigma$ effect. In the next section we will recover this number more or less from the profile distribution analysis. 

% Figure environment removed


\section{Profile Distributions}
\label{sec:PD}
Having explained the mathematics behind the bias, which gives rise to a projection effect, in subsection \ref{sec:math}, and having illustrated how it affects MCMC inferences in subsection \ref{sec:CCbias} - the minimum of the $\chi^2$ may fall outside of $1 \sigma$ confidence intervals - we turn to profile distributions (PDs) \cite{Gomez-Valent:2022hkb}, an extension of the profile likelihood, e. g. \cite{Trotta:2017wnx}, in order to address the bias. Consider two sets of parameters $\theta_1$ and $\theta_2$ and a normalised distribution $\mathcal{P}(\theta_1, \theta_2)$. The basic idea \cite{Gomez-Valent:2022hkb} is to study the ratio 
\be
\label{R}
R(\theta_1) = \frac{\tilde{\mathcal{P}}(\theta_1)}{\max_{\theta_1} \tilde{\mathcal{P}}(\theta_1) } = \frac{\tilde{\mathcal{P}}(\theta_1)}{\max_{\theta_1, \theta_2} \mathcal{P}(\theta_1, \theta_2) },  
\ee
where $\tilde{\mathcal{P}}(\theta_1)$ is the PD, defined to be the maximum of $\mathcal{P}$ for each $\theta_1$ along the $\theta_2$ direction: 
\be
\label{PD}
\tilde{\mathcal{P}} (\theta_1) = \max_{\theta_2} \mathcal{P}(\theta_1, \theta_2). 
\ee
The advantage of this approach is that $R(\theta_1)$ can serve as a probability distribution function (up to an overall normalization), however we do not need to perform any integration, so $R(\theta_1)$ is not prone to volume or projection effects. At this juncture, given the simplicity of our setup with only two parameters $(H_0, \Omega_m)$, we can be more explicit. Consider the probability distribution,   
\be
\mathcal{P}(\theta_1, \theta_2) = \exp \left( - \frac{1}{2} \chi^2(\theta_1, \theta_2) \right), 
\ee
where $\theta_i \in \{H_0, \Omega_m \}$  and $\chi^2(H_0, \Omega_m)$ is our earlier likelihood (\ref{eq:chi2}). The maximum value of $\mathcal{P}$ occurs for the minimum value of $\chi^2$ from the MCMC chain, $\mathcal{P}_{\textrm{max}} = e^{-\frac{1}{2} \chi^2_{\textrm{min}}}$. In this concrete setting, the PD becomes 
\be
\tilde{\mathcal{P}}(\theta_1) = e^{-\frac{1}{2} \chi^2_{\textrm{min}}(\theta_1)}, 
\ee
where $\chi^2_{\textrm{min}}(\theta_1)$ denotes the minimum value of the $\chi^2$ along the $\theta_2$ direction for a fixed $\theta_1$ value. It should not be confused with the overall minimum $\chi^2_{\textrm{min}}$, which can be extracted easily from the MCMC chain. In practice, one can also determine $\chi^2_{\textrm{min}}(\theta_1)$ from the MCMC chain by breaking the $\theta_1$ direction up into bins and finding the minimum of the $\chi^2$ for each bin. Having done so, we are in a position to define a PDF \cite{Gomez-Valent:2022hkb}: 
\be
\label{eq:w}
w(\theta_1) = \frac{e^{-\frac{1}{2} \chi^2_{\textrm{min}}(\theta_1)}}{\int e^{-\frac{1}{2} \chi^2_{\textrm{min}}(\theta_1)} \, \textrm{d} \theta_1} = \frac{R(\theta_1)}{\int R(\theta_1) \, \textrm{d} \theta_1}, 
\ee
where in the second equality we have divided top and bottom by $\mathcal{P}_{\textrm{max}} = e^{-\frac{1}{2} \chi^2_{\textrm{min}}}$. As a result, $R(\theta_1) = e^{-\frac{1}{2} \Delta \chi_{\textrm{min}}^2}$, where $\Delta \chi^2_{\textrm{min}} := \chi_{\textrm{min}}^2(\theta_1) - \chi^2_{\textrm{min}}$, so that $R(\theta_1)$ peaks at $R(\theta_1) = 1$. Note that $\int_{-\infty}^{+\infty} w(\theta_1) \, \textrm{d} \theta_1 = 1$ by construction, so $w(\theta_1)$ describes a properly normalised PDF. Thus we can identify the $1 \sigma, 2 \sigma$ and $3 \sigma$ confidence intervals corresponding to the 68\%, 95\% and 99.7\% confidence level, respectively, by simply identifying $\theta_1^{(1)}$ and $\theta_1^{(2)}$ such that \cite{Gomez-Valent:2022hkb}
\be
\label{eq:wsigma}
\int_{\theta_1^{(1)}}^{\theta_1^{(2)}} w(\theta_1) \, \textrm{d} \theta_1 = I, \quad w(\theta_1) = w(\theta_2), \quad I \in \{0.68, 0.95, 0.997\}. 
\ee
We will outline how these conditions can most easily be satisfied when we turn to explicit examples. 

Our first port of call is making sure that the PD methodology gives sensible results. This can be best judged by applying it to the CC data with $z_{\textrm{min}} = 0$, since this is where we expect a distribution closest to a Gaussian distribution, as is evident from the agreement between Fisher matrix and MCMC results in Table \ref{tab:LCDM_CC}. In particular, we will be interested in a comparison between $1 \sigma$ confidence intervals to make sure that (\ref{eq:wsigma}) is not underestimating or overestimating the $1 \sigma$ confidence interval. 

% Figure environment removed

We start by running a long MCMC chain (100,000 iterations) in order to ensure bins are well populated, and begin by analysing $\theta_1 = H_0$ with $\theta_2 = \Omega_m$. From the MCMC chain we identify the smallest and largest value of $H_0$ in the chain and break up this range into approximately 200 uniform bins, which we label using the $H_0$ value at the centre of the bin. We omit any empty bins. One can increase the number of bins by simply running a longer MCMC chain. In each $H_0$ bin we identify the minimum value of the $\chi^2$, $\chi^2_{\textrm{min}}(H_0)$, and calculate $R(H_0)$. One then repeats the steps for $\Omega_m$. In Fig.~\ref{fig:R_zmin0} we plot $R(H_0)$ against $H_0$ and $R(\Omega_m)$ against $\Omega_m$, noting that the distributions are Gaussian to first approximation. 

Since the distributions from the MCMC chain are sparse in the tails, empty bins are evident in Fig.~\ref{fig:R_zmin0}. Nevertheless, with 200 bins, modulo any empty bins, we have sufficient density of points to calculate the total area under the $R(H_0)$ and $R(\Omega_m)$ curve using Simpson's rule. Any concern about precision can simply be mitigated by running a longer MCMC chain and increasing the number of bins. 
One may directly use $R(H_0)\leq 1$ and $R(\Omega_m)\leq 1$   to find $68$, $95$ and $99.7$ percentiles,  respectively corresponding to $1 \sigma, 2 \sigma$ and $3 \sigma$ confidence intervals. Consider $F_\kappa:= \int_{R\geq \kappa} R (\theta_1) \, \textrm{d} \theta_1$, where $\kappa \leq 1$. Observe that $F_{\kappa=1}=0$ and $F_{\kappa=0}:=F_0=\int R(\theta_1) \textrm{d} \, \theta_1$. Then move $\kappa$ through and terminate the process when $F_\kappa/F_0$ is equal to $0.68$, $0.95$ and $0.997$. This gives the corresponding range for $\theta_1$ that defines the confidence interval.
Working with the precision afforded to us by approximately 200 bins, the $H_0$ and $\Omega_m$ $1 \sigma$ confidence intervals are presented in Fig.~\ref{fig:R_zmin0} and the first entry in Table \ref{tab:LCDM_CC_PD}. The outcome is in excellent agreement with both Fisher matrix and MCMC analysis. In particular, a mild non-Gaussianity in $\Omega_m$ is evident in both Fig.~\ref{fig:R_zmin0} and the errors. 
Thus, we have succeeded in recovering results in the (almost) Gaussian regime that are consistent with Fisher matrix and MCMC analysis and this provides an important check of the methodology.  

% Figure environment removed

We now apply the same PD methodology to the non-Gaussian regime where MCMC marginalisation leads to biased results. To be concrete, we focus on the eight data points in the range $1 < z < 2$ where a non-evolving $H(z)$ trend is evident in the raw data in Fig.~\ref{fig:CC}. Our goal here is to quantify the disagreement with the full data set, where one infers $H_0 \sim 68$ km/s/Mpc and $\Omega_m \sim 0.32$. A similar exercise was performed in subsection \ref{sec:features} with mock simulations and the disagreement was estimated to be approximately $2 \sigma$. Repeating the steps outlined above for the CC data with $z_{\textrm{min}} = 1$ we find the distributions in Fig.~\ref{fig:R_zmin1}. The first observation is that the distributions are non-Gaussian, but a comparison to the MCMC posteriors from the same data in blue in Fig.~\ref{fig:CCsplit1} reveals that there is no secondary $H_0$ peak at $H_0 \sim 50$ km/s/Mpc. Thus, we confirm the secondary peak to be a projection effect. That being said, the primary $H_0$ peak from Fig.~\ref{fig:CCsplit1} has shifted to the dashed line corresponding to the minimum of the $\chi^2$, since the peak of the distribution and $\chi^2$ minimum agree by construction. Comparing the blue $\Omega_m$ distribution from Fig.~\ref{fig:CCsplit1} to the $R(\Omega_m)$ distribution in Fig.~\ref{fig:R_zmin1}, we see that the peak is close to $\Omega_m = 0$ and that the tails continue to $\Omega_m = 1$. In both plots we see that there is a non-zero probability of inferring $\Omega_m = 1$. In some sense, this is not so surprising, the reason being that one is free to adopt generous priors for $H_0$, so that probability of large and small $H_0$ values is zero, but the priors on $\Omega_m$ in the flat $\Lambda$CDM model are restricted. For this reason, as a distribution spreads one invariably finds that distributions are impacted by the $\Omega_m$ priors.\footnote{Note, this is a problem for the flat $\Lambda$CDM model. In particular, one may easily find that the peak of the $\Omega_m$ distribution is larger than $\Omega_m=1$, as is the case with Hubble Space Telescope SN with redshifts $z > 1$ in the Pantheon+ sample \cite{Malekjani:2023dky}.}

It is evident from Fig.~\ref{fig:R_zmin1} that any tension that exists is confined to the $H_0$ parameter. Moreover, since there may be only one binned value of $\Omega_m$ below the $R(\Omega_m)$ peak, at the precision afforded to us by 200 bins, the $R(\Omega_m)$ distribution in Fig.~\ref{fig:R_zmin1} is essentially one-sided and the $1 \sigma$ confidence interval stretches beyond $\Omega_m \sim 0.32$, so there is no disagreement in the $\Omega_m$ parameter. Nevertheless, in the $H_0$ parameter we see that $H_0 \sim 68$ km/s/Mpc, the value favoured by the full data set is just under $2 \sigma$ removed from the peak. The main point here is that, as is obvious from the raw data, current CC data with $z > 1$ has a preference for a non-evolving Hubble parameter $H(z)$ with a large constant $H_0 \sim 150$ km/s/Mpc. The disagreement is just under $2 \sigma$, more accurately $1.9 \sigma$ from $R(H_0)$, and only $0.9 \sigma$ from $R(\Omega_m)$. Although this may not be a serious discrepancy, essentially because of the poor data quality (8 data points), this disagreement supports the $\sim 2 \sigma$ discrepancy seen in the mock simulations. It should be borne in mind that systematic uncertainties have been omitted and these will reduce this discrepancy once properly propagated. Given the agreement between the PD and mock simulation analysis, there is nothing to suggest that the three independent trends highlighted in \cite{Colgain:2022rxy} across OHD, Type Ia SN and QSOs are not \textit{bona fide} disagreements and that redshift evolution is present in the sample. The task remains to combine them at the level of a $\chi^2$ likelihood instead of combining them using Fisher's method on the basis that they are independent probabilities. We leave this exercise for a forthcoming paper, but revisit the tension in OHD data in the following section.  %\ref{sec:tension}. 
For completeness, in Table \ref{tab:LCDM_CC_PD} we perform a reanalysis of CC data subsets with the PD approach and record the $1 \sigma$ intervals.  

\begin{table}[htb]
    \centering
    \begin{tabular}{c|c|c|c}
    \rule{0pt}{3ex} $z_{\textrm{min}}$ & \# CC & \multicolumn{2}{c}{PD}  \\
    \hline
    \rule{0pt}{3ex} & & $H_0$ (km/s/Mpc) & $\Omega_m$ \\
    \hline
    \rule{0pt}{3ex} $0$ & $34$ & $68.15^{+3.04}_{-3.11}$ & $0.320^{+0.065}_{-0.055}$ \\
    \hline 
    \rule{0pt}{3ex} $0.2$ & $27$ & $65.03^{+6.52}_{-7.03}$ & $0.368^{+0.167}_{-0.110}$ \\
    \hline 
    \rule{0pt}{3ex} $0.4$ & $22$ & $62.42^{+7.78}_{-8.74}$ & $0.411^{+0.236}_{-0.113}$ \\
    \hline
    \rule{0pt}{3ex} $0.6$ & $15$ & $59.75^{+11.73}_{-13.97}$ & $0.455^{+0.355}_{-0.160}$ \\
    \hline
    \rule{0pt}{3ex} $0.7$ & $14$ & $79.10^{+16.42}_{-20.56}$ & $0.222^{+0.386}_{-0.117}$ \\
    \hline
    \rule{0pt}{3ex} $0.8$ & $11$ & $103.94^{+22.88}_{-28.54}$ & $0.097^{+0.378}_{-0.074}$ \\
    \hline
    \rule{0pt}{3ex} $1$ & $8$ & $150.35^{+17.12}_{-35.95}$ & $ < 0.339$ \\
    \hline
    \rule{0pt}{3ex} $1.2$ & $7$ & $154.26^{+14.88}_{-54.82}$ & $ < 0.570$ \\
    \hline
    \rule{0pt}{3ex} $1.4$ & $4$ & $124.81^{+35.38}_{-52.60}$ & $ < 0.661$ \\
    \hline
    \rule{0pt}{3ex} $1.5$ & $3$ & $36.11^{+72.87}_{-2.43}$ & $ > 0.354$
    \end{tabular}
    \caption{Same as Table \ref{tab:LCDM_CC} but with the PD methodology in lieu of Fisher matrix and MCMC analysis. The high redshift $R(\Omega_m)$ distributions are typically one-sided, so one encounters $1 \sigma$ upper and lower bounds.}
    \label{tab:LCDM_CC_PD}
\end{table}




\section{A tension with Planck}
\label{sec:tension}
A $2 \sigma$ ($p = 0.021$) tension with Planck has been reported in OHD through best fits and mock simulations in \cite{Colgain:2022rxy}. In particular, it was noted that a combination of 7 CC and BAO data points above $z = 1.45$ resulted in a $(H_0, \Omega_m) = (37.8, 1)$ best fit, where in line with analysis here, an $\Omega_m \in [0, 1]$ uniform prior was assumed. Based on mock simulations, the probability of such a best fit configuration arising by chance in mocks assuming input parameters consistent with Planck was estimated to be $p = 0.021$ \cite{Colgain:2022rxy}. A similar best fit appears in the last entry of Table \ref{tab:LCDM_CC} and Table \ref{tab:LCDM_CC_PD}, but there is no tension with Planck within the errors, even with our PD analysis, because CC data is inherently of poorer quality than BAO data. One further difference between the analysis is that \cite{Colgain:2022rxy} imposes a Gaussian Planck prior $\Omega_m h^2 = 0.1430 \pm 0.0011$ \cite{Planck:2018vyg} \footnote{This prior essentially prevents high redshift CC data from tracking a non-evolving $H(z)$.} to fix the high redshift behaviour of $H(z)$, whereas our analysis here so far has not introduced a prior. 

% Figure environment removed

Nevertheless, armed with a new PD methodology, we are in a position to revisit the earlier result and see if we can recover the $2 \sigma$ tension with Planck. Since \cite{Colgain:2022rxy} made use of older BAO data, here we replace QSO and Lyman-$\alpha$ BAO with the latest eBOSS results \cite{Hou:2020rse, Neveux:2020voa, duMasdesBourboux:2020pck}. Moreover, we work directly with the $D_{H}/r_d$ constraints and do not invert them. This entails assuming a value for the radius of the sound horizon, which we take to be the Planck value, $r_d = 147.09 \pm 0.26$ Mpc \cite{Planck:2018vyg}. In addition, we reinstate the prior $\Omega_m h^2 = 0.1430 \pm 0.0011$, so that the only difference with \cite{Colgain:2022rxy} is simply to update OHD BAO to the latest constraints. We stress that the priors we introduce are consistent with the Planck cosmology, so \textit{they cannot be driving any disagreement}. Moreover, the $\Omega_m h^2$ prior restricts one to a curve in the $(H_0, \Omega_m)$, but it cannot dictate where one is on the curve, this is done by the remaining 3 CC and 3 BAO data points.  

We again marginalise over the free parameters $(H_0, \Omega_m, r_d)$ with MCMC. In Fig.~\ref{fig:CC_BAO_MCMC} we present the posteriors. While $r_d$ is Gaussian and peaked on our Planck prior, as expected, the $\Omega_m$ posterior is peaked at $\Omega_m \sim 0.6$ and the fact that the fall off in the distribution is gradual beyond the peak leads to a pile up of configurations in the top left corner of the $(H_0, \Omega_m)$-plane. This fall off continues beyond $\Omega_m = 1$ and if the prior is relaxed, the $H_0$ peak shifts to smaller values. So,  once again all the hallmarks of projection effects are present. That being said, given the sharp fall off in the $\Omega_m$ distribution to smaller $\Omega_m$ values, some tension appears to be evident with the Planck values (dashed lines). 

% Figure environment removed

We now run the MCMC chain through our PD methodology. From Fig.~\ref{fig:CC_BAO}, we can see that the $R(H_0)$ and $R(\Omega_m)$ distributions prefer smaller values of $H_0$ and larger values of $\Omega_m$. The peak of the distributions occurs at $H_0 = 42.40$ km/s/Mpc and $\Omega_m = 0.795$.  The lone dot in the $R(H_0)$ distribution at low values of $H_0$ tells us that the distribution falls off sharply below $H_0 = 40$ km/s/Mpc. Note, since we employed generous uniform priors $H_0 \in [0, 200]$, the priors are not impacting the $R(H_0)$ distribution, so it is expected that the distribution falls off to zero on both sides. In contrast, the $R(\Omega_m)$ distribution is one-sided and fails to fall off in the direction of larger values within the uniform priors $\Omega_m \in [0, 1]$. The tension with Planck falls between $2 \sigma$ and $3 \sigma$. By integrating the PDF as far as the black lines corresponding to the Planck values in Fig.~\ref{fig:CC_BAO}, we estimate that the Planck $H_0$ is located at $2.1 \sigma$ from the peak, while the Planck $\Omega_m$ value is $2.5 \sigma$ from the peak.

The main take-away from this section is that OHD data comprising CC and BAO data points beyond $z=1.45$ is inconsistent with the Planck cosmology at in excess of $2 \sigma$. We have employed Planck priors to arrive at this result, but these priors cannot drive the disagreement. Moreover, independent analysis based on least squares fitting and mock simulations presented in \cite{Colgain:2022rxy} also points to a $2 \sigma$ tension, albeit with less up-to-date high redshift BAO data. In summary, different methodologies agree on a $2 \sigma$ discrepancy with Planck, which is robust to interchanging older and newer BAO data. 

\section{Concluding remarks}
\label{sec:discussion}
A $\chi^2$ likelihood is a metric or measure of how well a model fits data. The point in model parameter space that fits the data the best possesses the lowest $\chi^2$. Once one has identified this point, the problem remains to establish $1 \sigma$, $2 \sigma$, etc, confidence intervals in parameter space. In cosmology and astrophysics, MCMC is the prevailing technique for estimating confidence intervals. Its great advantage is that it allows one to i) globally sample the parameter space and ii) arrive at posteriors that serve as an estimate of the errors even with non-Gaussian distributions. In contrast, if one minimises the $\chi^2$ by gradient descent, there is always a risk that one ends up in a local minimum, i. e. the global minimum is missed, while error estimation through Fisher matrix assumes any distribution is Gaussian. The appeal of MCMC marginalisation is that it is widely applicable. However, the point of this paper is that limitations exist, even in the simplest model. 

Indeed, what happens when the MCMC posterior no longer tracks points in parameter space that fit the data better? Traditionally, volume effects are seen as the preserve of higher-dimensional models, e. g. \cite{Herold:2021ksg, Gomez-Valent:2022hkb, Meiers:2023gft}, but projection effects also occur in the minimal $\Lambda$CDM model when one fits the model to data binned by redshift in the late Universe \cite{Colgain:2022tql}. As explained in \cite{Colgain:2022tql}, this ``projection effect'' is driven by OHD, $H(z_i)$, and angular diameter or luminosity distance data, $D_{A}(z_i)$ or $D_{L}(z_i)$, {respectively} only constraining the combinations $\Omega_m h^2$ and $ (1-\Omega_m) h^2$ well, with high redshift data $z_i \gg 0$. In practice, this restricts MCMC configurations to constant $\Omega_m h^2$ and constant $(1-\Omega_m) h^2$ curves in the $(H_0, \Omega_m)$ plane, and as the curves stretch due to DE or matter being less well constrained in high redshift bins, projection effects lead to shifts in the peaks of MCMC posteriors and the emergence of non-Gaussian tails \cite{Colgain:2022tql}. We stress that one sees the same effect in PDFs of best fit $(H_0, \Omega_m)$ parameters in a large number of mock data realisations \cite{Colgain:2022tql}, so the problem is more general than MCMC; there is an inherent bias in the $\Lambda$CDM model when one fits it to redshift binned $H(z)$ \textit{or} $D_{A}(z)$ \textit{or} $D_{L}(z)$ data. Within MCMC, one sees this effect in the errors, but also in the drift of the parameters corresponding to the $\chi^2$ minimum outside of the $1 \sigma$ confidence intervals. Highlighting this (expected) bias in MCMC using OHD is the opening salvo (result) of this paper.     

Why should one care? This is evidently only a problem if one bins data and confronts the $\Lambda$CDM model. First, note that some data sets are inherently binned. For example, effective redshifts are assigned to CC and BAO analysed in a given redshift bin, while each strongly lensed system constitutes its own bin. Working with binned data is unavoidable. Secondly, $\Lambda$CDM tensions point to a problem with the $\Lambda$CDM model once the tensions become widespread and persistent. As explained in \cite{Krishnan:2020vaf}, if the minimal $\Lambda$CDM model is too simple, one expects redshift evolution of $\Lambda$CDM cosmological parameters as it is confronted to redshift binned data. Hints of these trends are now evident in $H_0$ \cite{Wong:2019kwg, Millon:2019slk, Dainotti:2021pqg, Colgain:2022nlb, Colgain:2022rxy, Malekjani:2023dky, Hu:2022kes, Jia:2022ycc, Krishnan:2020obg, Dainotti:2022bzg}, $\Omega_m$ \cite{Risaliti:2015zla, Risaliti:2018reu, Lusso:2020pdb, Yang:2019vgk, Khadka:2020vlh, Khadka:2020tlm, Khadka:2021xcc, Pourojaghi:2022zrh, Colgain:2022nlb, Colgain:2022rxy, Malekjani:2023dky, Pasten:2023rpc, Sakr:2023hrl} and $S_8$/$\sigma_8$ \cite{Esposito:2022plo, Adil:2023jtu, ACT:2023dou, ACT:2023kun} (also \cite{Miyatake:2021qjr, Alonso:2023guh}) across a host of different observables. This evolution is an expected hallmark of model breakdown, which must happen at some redshift if systematics are not universally at play. 

The main problem with redshift dependent $\Lambda$CDM cosmological parameters\footnote{There is a separate interpretation problem as the cosmology literature works with  parameters ``defined today''. In more mathematical language, this is simply the statement that one solves an ordinary differential equation (ODE), namely the Friedmann equation or equivalent, by specifying an integration constant, e.g. $H_0 = H(z=0)$ or $\rho_m(z=0)=\rho_{m0}=H_0^2\Omega_{m}$. However, this is a mathematical statement and it still needs to be confirmed observationally that $H_0$ or $\rho_{m0}$ are \textit{bona fide} constants. This cannot be \textit{a priori} assumed, because it is mathematical prediction of the model. If the model is correct, a constant $H_0$ and $\Omega_m$  will be supported by the data. See \cite{Krishnan:2020vaf} for further discussion.} is one needs to assign a statistical significance to any trend. At a purely practical level, this entails constructing bins centered on different redshifts and identifying discrepancies in $\Lambda$CDM parameters between bins, \textit{ideally in the same observable}, so that the potential systematics are under greatest control. As demonstrated both mathematically and observationally with the CC data in section \ref{sec:MCMC_bias}, MCMC marginalisation leads to biased inferences when one bins the data. In this paper we have resorted to profile distributions \cite{Gomez-Valent:2022hkb} to overcome this bias and have applied the technique to a setting where $\Lambda$CDM distributions are expected to be non-Gaussian for the reasons outlined above and in section \ref{sec:MCMC_bias}. This new technique, provides a complementary perspective that confirms the least square fits of observed and mock data presented in \cite{Colgain:2022nlb, Colgain:2022rxy, Malekjani:2023dky}, where evidence for redshift evolution in $H_0$ and $\Omega_m$ was presented. Regardless of the methodology, the objective is to drill down on the prevailing \textit{assumption} that cosmological parameters are constants. \textit{In the era of tensions in cosmology, nothing can be assumed, especially noting that the tensions are in essence showing an example of evolution of these parameters with redshift.}

More concretely, in this paper with both mock simulations and profile distributions we have shown that high redshift CC data has a preference for a non-evolving $H(z)$ over Planck-$\Lambda$CDM at approximately $\sim 2 \sigma$. This trend, which constitutes the second result of the paper, is unquestionable, as it is visible in the data. Note, we have not propagated systematic uncertainties, so the significance will be less when these are properly propagate. Nevertheless, low and high redshift CC data currently have a preference for different $\Lambda$CDM cosmological parameters. This is important because if the CC program is claiming an 8\% constraint on the Hubble constant, $H_0 = 66.7 \pm 5.5$ km/s/Mpc \cite{Moresco:2023zys}, it is imperative that \textit{all subsets of the data are consistent with this result}. If they are not, then we are staring at either systematics or model breakdown. Admittedly, demanding self-consistency of subsets of a data set confronted to a model is a high bar, but it is important that data sets result in overlapping constraints on $\Lambda$CDM parameters, otherwise this makes cosmological inferences moot. Note, the $\Lambda$CDM model is largely only well tested in the DE dominated regime $z \lesssim 1$ and at very high redshifts $z \sim 1100$, which leaves a wide expanse of redshifts to be explored in order to confirm or refute the model. Given the existing $\Lambda$CDM tensions \cite{Perivolaropoulos:2021jda, Abdalla:2022yfr}, and the hints of evolution in $H_0$, $\Omega_m$ and $S_8$ across assorted probes in the late Universe $z \lesssim 5$, it would be surprising if all discrepancies could be explained away by systematics.\footnote{We are open to the possibility, we just consider it a bad bet at the moment. The odds can of course change as observations improve.}

As an aside, it is intriguing that CC data has a preference for larger best fit values of $H_0$ and smaller best fit values of $\Omega_m$ beyond $z_{\textrm{min}} = 0.7$, as this is traditionally the transition redshift between decelerated and accelerated expansion. % where $\ddot{a} = 0$. 
Moreover, at higher redshifts $z \sim 2.3$, there is not only a longstanding anomaly in Lyman-$\alpha$ BAO \cite{duMasdesBourboux:2020pck}, but QSOs also show a preference for a lower luminosity distance, $D_{L}(z)$, relative to Planck-$\Lambda$CDM \cite{Risaliti:2015zla, Risaliti:2018reu}. Translated into $\Lambda$CDM parameters, this corresponds to conversely larger $\Omega_m$ values, e. g.  \cite{Yang:2019vgk, Khadka:2020vlh, Khadka:2020tlm, Khadka:2021xcc, Pourojaghi:2022zrh}, and consequently smaller $H_0$ values. Thus, the emerging probes CC and QSOs  \cite{Moresco:2022phi} do not appear to be in sync on high redshift $\Lambda$CDM inferences. Nevertheless, neither may be inconsistent with the anomaly in Lyman-$\alpha$ BAO. Relative to Planck-$\Lambda$CDM, Lyman-$\alpha$ BAO prefers \textit{smaller} values of $D_{M}(z) := c \int_{0}^z 1/H(z^{\prime}) \, \textrm{d} z$ and \textit{smaller} values of $H(z)$ (larger values of $D_{H}(z) := c/H(z)$).\footnote{In this statement we assumed the Planck value $r_d \sim 147$ Mpc \cite{Planck:2018vyg} If we reinstate the radius of the sound horizon in these expressions, one recognises that changing the sound horizon, as advocated by early Universe resolutions to Hubble tension, cannot consistently address the Lyman-$\alpha$ BAO anomaly. In general, even for the Planck-$\Lambda$CDM sound horizon, one cannot get both a smaller $D_{M}(z)$ and smaller $H(z)$ from a strictly increasing function, such as the $\Lambda$CDM $H(z)$. As a result, deviations from the Planck-$\Lambda$CDM model that address this anomaly are expected to lead to wiggles in $H(z)$ \cite{Akarsu:2022lhx}, which are unsurprisingly seen in data reconstructions \cite{Zhao:2017cud, Wang:2018fng, Escamilla:2021uoj}. Finally, evolution in $H_0, \Omega_m$ discussed here cannot be explained or accommodated by early resolutions to Hubble tension relying on a change in the $r_d$ at very high $z$.}. If CC data prefer less evolution in $H(z)$ in the matter-dominated regime, then this is consistent with the preference for a smaller $H(z)$ from Lyman-$\alpha$ BAO. Furthermore, QSO data prefers smaller luminosity distances $D_{L}(z)$ relative to Planck, which are consistent with the smaller $D_{M}(z) \propto D_{L}(z)$ values preferred by Lyman-$\alpha$ BAO. Thus, even if CC and QSOs appear to be showing diverging behaviour in the cosmological parameters $(H_0, \Omega_m)$, this may still turn out to be consistent with Lyman-$\alpha$ BAO. We await future DESI \cite{DESI:2023ytc} data releases to ascertain if the non-evolving $H(z)$ trend in high redshift CC data is physical or not. 

Finally, we come to our third and main result outlined in section \ref{sec:tension}. We have revisited a $\sim 2 \sigma$ tension between high redshift CC and BAO data reported in \cite{Colgain:2022rxy}, where the significance was estimated through mock simulations. Here, we have upgraded the BAO data to the latest constraints and again  recover a $>2 \sigma$ discrepancy in $(H_0, \Omega_m)$ with different methodology. This provides a consistency check that there is evolution in OHD between low and high redshifts in the late Universe. Note, this evolution runs contrary to the non-evolving $H(z)$ seen in high redshift CC data because it assumes Planck has accurately constrained the high redshift behaviour of the Hubble parameter in (\ref{eq:lcdm}). Nevertheless, both with and without a Planck prior on $\Omega_m h^2$, evolution at $ \gtrsim 2 \sigma$ is evident in OHD data. It should be stressed that evolution is evident in PDFs of best fit $\Lambda$CDM parameters fitted to a large number of Planck-$\Lambda$CDM mocks \cite{Colgain:2022tql}, so evolution in observed data can be expected. It is imperative to revisit the remaining observations in \cite{Colgain:2022rxy, Malekjani:2023dky} in order to confirm the significance of $\sim 2 \sigma$ hints of evolution found separately in Type Ia SN and QSO data sets. 




\acknowledgments
We would like to thank Adri\`a G\'omez-Valent for discussions and comments on the draft. We thank Gabriela Marques, Mike Hudson and Matteo Viel for related discussions on late Universe evolution in $S_8$. E\'OC thanks Yonsei University and Asia Pacific Center for Theoretical Physics for hospitality. 
This article/publication is based upon work from COST Action CA21136 – “Addressing observational tensions in cosmology with systematics and fundamental physics (CosmoVerse)”, supported by COST (European Cooperation in Science and Technology). SP and MMShJ acknowledge SarAmadan grant No. ISEF/M/401332. MMShJ thanks the support from ICTP associates office (under Senior Associate program) and ICTP HECAP section for hospitality.  


\appendix
\section{Fisher Matrix}
\label{sec:fisher}
Consider the $\chi^2$ (\ref{eq:chi2}). 
Defining $H_{\textrm{model}}(z) = H_0 \sqrt{1-\Omega_m + \Omega_m (1+z)^3}$ and $Q_i$ as in \eqref{eq:Q}, we can now work out the derivatives
\begin{equation}
    \begin{split}
\partial_{H_0} Q_i &= -\sqrt{1-\Omega_m + \Omega_m (1+z_i)^3}, \\  \partial_{\Omega_m} Q_i &= - \frac{1}{2} H_0 (z_i^3 + 3 z_i^2 + 3 z_i)/\sqrt{1-\Omega_m + \Omega_m (1+z_i)^3}, \\
\partial^2_{H_0} Q_i &= 0, \\
\partial_{H_0} \partial_{\Omega_m} Q_i &= - \frac{1}{2} (z_i^3 + 3 z_i^2 + 3 z_i)/\sqrt{1-\Omega_m + \Omega_m (1+z_i)^3}, \\
\partial^2_{\Omega_m} Q_i =& \frac{1}{4} H_0 (z_i^3 + 3 z_i^2 + 3 z_i)^2/(1-\Omega_m + \Omega_m (1+z_i)^3)^{\frac{3}{2}}.      
    \end{split}
\end{equation}
We can then define the Fisher matrix 
\be
F_{ij} = \frac{1}{2} \frac{\partial^2 \chi^2(H_0, \Omega_m)}{\partial p_i \partial p_j}
\ee
where $p_i \in \{ H_0, \Omega_m \}$. Note that the Fisher matrix is evaluated on the best fit parameters. The result is a $2 \times 2$ matrix, which one inverts and the estimated errors are the square root of the diagonal entries. 








\begin{thebibliography}{99}

\bibitem{Planck:2018vyg}
N.~Aghanim \textit{et al.} [Planck],
``Planck 2018 results. VI. Cosmological parameters,''
Astron. Astrophys. \textbf{641} (2020), A6
% doi:10.1051/0004-6361/201833910
%[arXiv:1807.06209 [astro-ph.CO]].

\bibitem{Riess:1998cb}
A.~G.~Riess \textit{et al.} [Supernova Search Team],
``Observational evidence from supernovae for an accelerating universe and a cosmological constant,''
Astron. J. \textbf{116} (1998), 1009-1038
% doi:10.1086/300499
%[arXiv:astro-ph/9805201 [astro-ph]].
%13031 citations counted in INSPIRE as of 02 Feb 2021

\bibitem{Perlmutter:1998np}
S.~Perlmutter \textit{et al.} [Supernova Cosmology Project],
``Measurements of $\Omega$ and $\Lambda$ from 42 high redshift supernovae,''
Astrophys. J. \textbf{517} (1999), 565-586
% doi:10.1086/307221
%[arXiv:astro-ph/9812133 [astro-ph]].
%13057 citations counted in INSPIRE as of 02 Feb 2021

\bibitem{Eisenstein:2005su}
D.~J.~Eisenstein \textit{et al.} [SDSS],
``Detection of the Baryon Acoustic Peak in the Large-Scale Correlation Function of SDSS Luminous Red Galaxies,''
Astrophys. J. \textbf{633} (2005), 560-574
%doi:10.1086/466512
%[arXiv:astro-ph/0501171 [astro-ph]].
%3380 citations counted in INSPIRE as of 08 Oct 2020

\bibitem{Riess:2021jrx}
A.~G.~Riess, W.~Yuan, L.~M.~Macri, D.~Scolnic, D.~Brout, S.~Casertano, D.~O.~Jones, Y.~Murakami, L.~Breuval and T.~G.~Brink, \textit{et al.}
``A Comprehensive Measurement of the Local Value of the Hubble Constant with 1 km s$^{?1}$ Mpc$^{?1}$ Uncertainty from the Hubble Space Telescope and the SH0ES Team,''
Astrophys. J. Lett. \textbf{934} (2022) no.1, L7
%doi:10.3847/2041-8213/ac5c5b
%[arXiv:2112.04510 [astro-ph.CO]].
%370 citations counted in INSPIRE as of 09 Jan 2023

\bibitem{Freedman:2021ahq}
W.~L.~Freedman,
``Measurements of the Hubble Constant: Tensions in Perspective,''
Astrophys. J. \textbf{919} (2021) no.1, 16
%doi:10.3847/1538-4357/ac0e95
%[arXiv:2106.15656 [astro-ph.CO]].
%179 citations counted in INSPIRE as of 09 Jan 2023

\bibitem{Pesce:2020xfe}
D.~W.~Pesce, J.~A.~Braatz, M.~J.~Reid, A.~G.~Riess, D.~Scolnic, J.~J.~Condon, F.~Gao, C.~Henkel, C.~M.~V.~Impellizzeri and C.~Y.~Kuo, \textit{et al.}
%``The Megamaser Cosmology Project. XIII. Combined Hubble constant constraints,''
Astrophys. J. Lett. \textbf{891} (2020) no.1, L1
%doi:10.3847/2041-8213/ab75f0
%[arXiv:2001.09213 [astro-ph.CO]].
%96 citations counted in INSPIRE as of 12 Jul 2021

\bibitem{Blakeslee:2021rqi}
J.~P.~Blakeslee, J.~B.~Jensen, C.~P.~Ma, P.~A.~Milne and J.~E.~Greene,
%``The Hubble Constant from Infrared Surface Brightness Fluctuation Distances,''
Astrophys. J. \textbf{911} (2021) no.1, 65
%doi:10.3847/1538-4357/abe86a
%[arXiv:2101.02221 [astro-ph.CO]].
%11 citations counted in INSPIRE as of 12 Jul 2021

\bibitem{Kourkchi:2020iyz}
E.~Kourkchi, R.~B.~Tully, G.~S.~Anand, H.~M.~Courtois, A.~Dupuy, J.~D.~Neill, L.~Rizzi and M.~Seibert,
%``Cosmicflows-4: The Calibration of Optical and Infrared Tully\textendash{}Fisher Relations,''
Astrophys. J. \textbf{896} (2020) no.1, 3
%doi:10.3847/1538-4357/ab901c
%[arXiv:2004.14499 [astro-ph.GA]].
%15 citations counted in INSPIRE as of 12 Jul 2021

\bibitem{HSC:2018mrq}
C.~Hikage \textit{et al.} [HSC],
``Cosmology from cosmic shear power spectra with Subaru Hyper Suprime-Cam first-year data,''
Publ. Astron. Soc. Jap. \textbf{71}, 43  (2019).
%doi:10.1093/pasj/psz010

\bibitem{KiDS:2020suj}
M.~Asgari \textit{et al.} [KiDS],
``KiDS-1000 Cosmology: Cosmic shear constraints and comparison between two point statistics,''
Astron. Astrophys. \textbf{645} (2021), A104
%doi:10.1051/0004-6361/202039070
%[arXiv:2007.15633 [astro-ph.CO]].
%113 citations counted in INSPIRE as of 18 Aug 2021

\bibitem{DES:2021wwk}
T.~M.~C.~Abbott \textit{et al.} [DES],
``Dark Energy Survey Year 3 results: Cosmological constraints from galaxy clustering and weak lensing,''
Phys. Rev. D \textbf{105} (2022) no.2, 023520
%doi:10.1103/PhysRevD.105.023520
%[arXiv:2105.13549 [astro-ph.CO]].
%519 citations counted in INSPIRE as of 14 Jul 2023

\bibitem{Boruah:2019icj}
S.~S.~Boruah, M.~J.~Hudson and G.~Lavaux,
``Cosmic flows in the nearby Universe: new peculiar velocities from SNe and cosmological constraints,''
Mon. Not. Roy. Astron. Soc. \textbf{498} (2020) no.2, 2703-2718
%doi:10.1093/mnras/staa2485
%[arXiv:1912.09383 [astro-ph.CO]].
%54 citations counted in INSPIRE as of 14 Jul 2023

\bibitem{Said:2020epb}
K.~Said, M.~Colless, C.~Magoulas, J.~R.~Lucey and M.~J.~Hudson,
``Joint analysis of 6dFGS and SDSS peculiar velocities for the growth rate of cosmic structure and tests of gravity,''
Mon. Not. Roy. Astron. Soc. \textbf{497} (2020) no.1, 1275-1293
%doi:10.1093/mnras/staa2032
%[arXiv:2007.04993 [astro-ph.CO]].
%49 citations counted in INSPIRE as of 14 Jul 2023

\bibitem{Perivolaropoulos:2021jda}
L.~Perivolaropoulos and F.~Skara,
``Challenges for \ensuremath{\Lambda}CDM: An update,''
New Astron. Rev. \textbf{95}, 101659  (2022).
%doi:10.1016/j.newar.2022.101659
%\href{https://arxiv.org/abs/2105.05208}{2105.05208}

\bibitem{Abdalla:2022yfr}
E.~Abdalla, G.~Franco Abell\'an, A.~Aboubrahim, A.~Agnello, O.~Akarsu, Y.~Akrami, G.~Alestas, D.~Aloni, L.~Amendola and L.~A.~Anchordoqui, \textit{et al.}
``Cosmology intertwined: A review of the particle physics, astrophysics, and cosmology associated with the cosmological tensions and anomalies,''
JHEAp \textbf{34}, 49  (2022).
%doi:10.1016/j.jheap.2022.04.002
%\href{https://arxiv.org/abs/2203.06142}{2203.06142}

\bibitem{Phillips:1993ng}
M.~M.~Phillips,
``The absolute magnitudes of Type IA supernovae,''
Astrophys. J. Lett. \textbf{413} (1993), L105-L108
%doi:10.1086/186970
%1245 citations counted in INSPIRE as of 24 Aug 2021

\bibitem{NearbySupernovaFactory:2018qkd}
M.~Rigault \textit{et al.} [Nearby Supernova Factory],
``Strong Dependence of Type Ia Supernova Standardization on the Local Specific Star Formation Rate,''
Astron. Astrophys. \textbf{644} (2020), A176
%doi:10.1051/0004-6361/201730404
%[arXiv:1806.03849 [astro-ph.CO]].
%143 citations counted in INSPIRE as of 20 Jul 2023

\bibitem{Kang:2019azh}
Y.~Kang, Y.~W.~Lee, Y.~L.~Kim, C.~Chung and C.~H.~Ree,
``Early-type Host Galaxies of Type Ia Supernovae. II. Evidence for Luminosity Evolution in Supernova Cosmology,''
Astrophys. J. \textbf{889} (2020) no.1, 8
%doi:10.3847/1538-4357/ab5afc
%[arXiv:1912.04903 [astro-ph.GA]].
%56 citations counted in INSPIRE as of 20 Jul 2023

\bibitem{Brout:2020msh}
D.~Brout and D.~Scolnic,
``It\textquoteright{}s Dust: Solving the Mysteries of the Intrinsic Scatter and Host-galaxy Dependence of Standardized Type Ia Supernova Brightnesses,''
Astrophys. J. \textbf{909} (2021) no.1, 26
%doi:10.3847/1538-4357/abd69b
%[arXiv:2004.10206 [astro-ph.CO]].
%82 citations counted in INSPIRE as of 20 Jul 2023

\bibitem{Lee:2021txi}
Y.~W.~Lee, C.~Chung, P.~Demarque, S.~Park, J.~Son and Y.~Kang,
``Evidence for strong progenitor age dependence of type Ia supernova luminosity standardization process,''
Mon. Not. Roy. Astron. Soc. \textbf{517} (2022) no.2, 2697-2708
%doi:10.1093/mnras/stac2840
%[arXiv:2107.06288 [astro-ph.GA]].
%5 citations counted in INSPIRE as of 20 Jul 2023


\bibitem{Krishnan:2020vaf}
C.~Krishnan, E.~\'O~Colg\'ain, M.~M.~Sheikh-Jabbari and T.~Yang,
``Running Hubble Tension and a H0 Diagnostic,''
Phys. Rev. D \textbf{103} (2021) no.10, 103509
%doi:10.1103/PhysRevD.103.103509
%[arXiv:2011.02858 [astro-ph.CO]].
%65 citations counted in INSPIRE as of 14 Jul 2023 

\bibitem{Krishnan:2022fzz}
C.~Krishnan and R.~Mondol,
``$H_0$ as a Universal FLRW Diagnostic,''
[arXiv:2201.13384 [astro-ph.CO]].
%12 citations counted in INSPIRE as of 14 Jul 2023

%\bibitem{Liao:2020zko}
%K.~Liao, A.~Shafieloo, R.~E.~Keeley and E.~V.~Linder,
%``Determining Model-independent H 0 and Consistency Tests,''
%Astrophys. J. Lett. \textbf{895} (2020) no.2, L29
%doi:10.3847/2041-8213/ab8dbb
%[arXiv:2002.10605 [astro-ph.CO]].
%51 citations counted in INSPIRE as of 14 Jul 2023

%\bibitem{Montani:2023xpd}
%G.~Montani, M.~De Angelis, F.~Bombacigno and N.~Carlevaro,
%``Metric $f(R)$ gravity with dynamical dark energy as a paradigm for the Hubble Tension,''
%[arXiv:2306.11101 [gr-qc]].
%1 citations counted in INSPIRE as of 14 Jul 2023

\bibitem{Wong:2019kwg}
K.~C.~Wong, S.~H.~Suyu, G.~C.~F.~Chen, C.~E.~Rusu, M.~Millon, D.~Sluse, V.~Bonvin, C.~D.~Fassnacht, S.~Taubenberger and M.~W.~Auger, \textit{et al.}
``H0LiCOW \textendash{} XIII. A 2.4 per cent measurement of H0 from lensed quasars: 5.3\ensuremath{\sigma} tension between early- and late-Universe probes,''
Mon. Not. Roy. Astron. Soc. \textbf{498} (2020) no.1, 1420-1439
%doi:10.1093/mnras/stz3094
%[arXiv:1907.04869 [astro-ph.CO]].
%804 citations counted in INSPIRE as of 18 May 2023

\bibitem{Millon:2019slk}
M.~Millon, A.~Galan, F.~Courbin, T.~Treu, S.~H.~Suyu, X.~Ding, S.~Birrer, G.~C.~F.~Chen, A.~J.~Shajib and D.~Sluse, \textit{et al.}
``TDCOSMO. I. An exploration of systematic uncertainties in the inference of $H_0$ from time-delay cosmography,''
Astron. Astrophys. \textbf{639} (2020), A101
%doi:10.1051/0004-6361/201937351
%[arXiv:1912.08027 [astro-ph.CO]].
%114 citations counted in INSPIRE as of 18 May 2023

\bibitem{Sluse:2003iy}
D.~Sluse, J.~Surdej, J.~F.~Claeskens, D.~Hutsemekers, C.~Jean, F.~Courbin, T.~Nakos, M.~Billeres and S.~V.~Khmil,
``A Quadruply imaged quasar with an optical Einstein ring candidate: 1RXS J113155.4-123155,''
Astron. Astrophys. \textbf{406} (2003), L43-L46
%doi:10.1051/0004-6361:20030904
%[arXiv:astro-ph/0307345 [astro-ph]].
%83 citations counted in INSPIRE as of 14 Jul 2023

\bibitem{Shajib:2023uig}
A.~J.~Shajib, P.~Mozumdar, G.~C.~F.~Chen, T.~Treu, M.~Cappellari, S.~Knabel, S.~H.~Suyu, V.~N.~Bennert, J.~A.~Frieman and D.~Sluse, \textit{et al.}
``TDCOSMO. XIII. Improved Hubble constant measurement from lensing time delays using spatially resolved stellar kinematics of the lens galaxy,''
Astron. Astrophys. \textbf{673} (2023), A9
%doi:10.1051/0004-6361/202345878
%[arXiv:2301.02656 [astro-ph.CO]].
%3 citations counted in INSPIRE as of 18 May 2023

\bibitem{Dainotti:2021pqg}
M.~G.~Dainotti, B.~De Simone, T.~Schiavone, G.~Montani, E.~Rinaldi and G.~Lambiase,
``On the Hubble constant tension in the SNe Ia Pantheon sample,''
Astrophys. J. \textbf{912}, 150  (2021).
%doi:10.3847/1538-4357/abeb73


\bibitem{Colgain:2022nlb}
E.~\'O~Colg\'ain, M.~M.~Sheikh-Jabbari, R.~Solomon, G.~Bargiacchi, S.~Capozziello, M.~G.~Dainotti and D.~Stojkovic,
``Revealing intrinsic flat \ensuremath{\Lambda}CDM biases with standardizable candles,''
Phys. Rev. D \textbf{106}, L041301  (2022).
%doi:10.1103/PhysRevD.106.L041301

\bibitem{Colgain:2022rxy}
E.~\'O~Colg\'ain, M.~M.~Sheikh-Jabbari, R.~Solomon, M.~G.~Dainotti and D.~Stojkovic,
``Putting Flat $\Lambda$CDM In The (Redshift) Bin,''
[arXiv:2206.11447 [astro-ph.CO]].
%42 citations counted in INSPIRE as of 14 Jul 2023

%\cite{Colgain:2022tql}
%\bibitem{Colgain:2022tql}
%E.~\'O.~Colg\'ain, M.~M.~Sheikh-Jabbari and R.~Solomon,
%``High redshift \ensuremath{\Lambda}CDM cosmology: To bin or not to bin?,''
%Phys. Dark Univ. \textbf{40} (2023), 101216
%doi:10.1016/j.dark.2023.101216
%[arXiv:2211.02129 [astro-ph.CO]].
%10 citations counted in INSPIRE as of 25 Jul 2023


\bibitem{Malekjani:2023dky}
M.~Malekjani, R.~M.~Conville, E.~\'O.~Colg\'ain, S.~Pourojaghi and M.~M.~Sheikh-Jabbari,
``Negative Dark Energy Density from High Redshift Pantheon+ Supernovae,''
[arXiv:2301.12725 [astro-ph.CO]].
%13 citations counted in INSPIRE as of 17 Jul 2023

\bibitem{Hu:2022kes}
J.~P.~Hu and F.~Y.~Wang,
``Revealing the late-time transition of H0: relieve the Hubble crisis,''
Mon. Not. Roy. Astron. Soc. \textbf{517}, 576  (2022).

\bibitem{Jia:2022ycc}
X.~D.~Jia, J.~P.~Hu and F.~Y.~Wang,
``Evidence of a decreasing trend for the Hubble constant,''
Astron. Astrophys. \textbf{674} (2023), A45
%doi:10.1051/0004-6361/202346356
%[arXiv:2212.00238 [astro-ph.CO]].
%10 citations counted in INSPIRE as of 17 Jul 2023

\bibitem{Krishnan:2020obg}
C.~Krishnan, E.~\'O~Colg\'ain, Ruchika, A.~A.~Sen, M.~M.~Sheikh-Jabbari and T.~Yang,
``Is there an early Universe solution to Hubble tension?,''
Phys. Rev. D \textbf{102} (2020) no.10, 103525
%doi:10.1103/PhysRevD.102.103525
%[arXiv:2002.06044 [astro-ph.CO]].
%69 citations counted in INSPIRE as of 17 Jul 2023

\bibitem{Dainotti:2022bzg}
M.~G.~Dainotti, B.~De Simone, T.~Schiavone, G.~Montani, E.~Rinaldi, G.~Lambiase, M.~Bogdan and S.~Ugale,
``On the Evolution of the Hubble Constant with the SNe Ia Pantheon Sample and Baryon Acoustic Oscillations: A Feasibility Study for GRB-Cosmology in 2030,''
Galaxies \textbf{10}, 24  (2022).
%doi:10.3390/galaxies10010024

\bibitem{Risaliti:2015zla}
G.~Risaliti and E.~Lusso,
``A Hubble Diagram for Quasars,''
Astrophys. J. \textbf{815} (2015), 33
%doi:10.1088/0004-637X/815/1/33
%[arXiv:1505.07118 [astro-ph.CO]].
%146 citations counted in INSPIRE as of 16 Jun 2023

\bibitem{Risaliti:2018reu}
G.~Risaliti and E.~Lusso,
``Cosmological constraints from the Hubble diagram of quasars at high redshifts,''
Nature Astron. \textbf{3}, 272  (2019).

\bibitem{Lusso:2020pdb}
E.~Lusso, G.~Risaliti, E.~Nardini, G.~Bargiacchi, M.~Benetti, S.~Bisogni, S.~Capozziello, F.~Civano, L.~Eggleston and M.~Elvis, \textit{et al.}
``Quasars as standard candles III. Validation of a new sample for cosmological studies,''
Astron. Astrophys. \textbf{642}, A150  (2020).


\bibitem{Yang:2019vgk}
T.~Yang, A.~Banerjee and E.~\'O~Colg\'ain,
``Cosmography and flat $\Lambda$CDM tensions at high redshift,''
Phys. Rev. D \textbf{102}, 123532  (2020).

\bibitem{Khadka:2020vlh}
N.~Khadka and B.~Ratra,
``Using quasar X-ray and UV flux measurements to constrain cosmological model parameters,''
Mon. Not. Roy. Astron. Soc. \textbf{497}, 263  (2020).


\bibitem{Khadka:2020tlm}
N.~Khadka and B.~Ratra,
``Determining the range of validity of quasar X-ray and UV flux measurements for constraining cosmological model parameters,''
Mon. Not. Roy. Astron. Soc. \textbf{502}, 6140  (2021).


\bibitem{Khadka:2021xcc}
N.~Khadka and B.~Ratra,
``Do quasar X-ray and UV flux measurements provide a useful test of cosmological models?,''
Mon. Not. Roy. Astron. Soc. \textbf{510}, 2753  (2022).
%doi:10.1093/mnras/stab3678

\bibitem{Pourojaghi:2022zrh}
S.~Pourojaghi, N.~F.~Zabihi and M.~Malekjani,
``Can high-redshift Hubble diagrams rule out the standard model of cosmology in the context of cosmography?,''
Phys. Rev. D \textbf{106}, 123523  (2022).


\bibitem{Zajacek:2023qjm}
M.~Zaja\v{c}ek, B.~Czerny, N.~Khadka, R.~Prince, S.~Panda, M.~L.~Mart\'\i{}nez-Aldama and B.~Ratra,
``Extinction biases quasar luminosity distances determined from quasar UV and X-ray flux measurements,''
[arXiv:2305.08179 [astro-ph.GA]].
%0 citations counted in INSPIRE as of 17 Jul 2023

\bibitem{Pasten:2023rpc}
E.~Past\'en and V.~H.~C\'ardenas,
``Testing \ensuremath{\Lambda}CDM cosmology in a binned universe: Anomalies in the deceleration parameter,''
Phys. Dark Univ. \textbf{40} (2023), 101224
%doi:10.1016/j.dark.2023.101224
%[arXiv:2301.10740 [astro-ph.CO]].

\bibitem{Wagner:2022etu}
J.~Wagner,
``Casting the $H_0$ tension as a fitting problem of cosmologies,''
[arXiv:2203.11219 [astro-ph.CO]].
%5 citations counted in INSPIRE as of 28 Jul 2023

\bibitem{Sakr:2023hrl}
Z.~Sakr,
``One matter density discrepancy to alleviate them all or further trouble for $\Lambda$CDM model,''
[arXiv:2305.02846 [astro-ph.CO]].
%0 citations counted in INSPIRE as of 24 Jul 2023


\bibitem{Colgain:2022tql}
E.~\'O~Colg\'ain, M.~M.~Sheikh-Jabbari and R.~Solomon,
``High redshift \ensuremath{\Lambda}CDM cosmology: To bin or not to bin?,''
Phys. Dark Univ. \textbf{40} (2023), 101216
%doi:10.1016/j.dark.2023.101216
[arXiv:2211.02129 [astro-ph.CO]].
%10 citations counted in INSPIRE as of 28 Jun 2023

\bibitem{Esposito:2022plo}
M.~Esposito, V.~Ir\v{s}i\v{c}, M.~Costanzi, S.~Borgani, A.~Saro and M.~Viel,
``Weighing cosmic structures with clusters of galaxies and the intergalactic medium,''
Mon. Not. Roy. Astron. Soc. \textbf{515}, 857  (2022).
%doi:10.1093/mnras/stac1825
[arXiv:2202.00974 [astro-ph.CO]].

\bibitem{Adil:2023jtu}
S.~A.~Adil, \"O.~Akarsu, M.~Malekjani, E.~\'O~Colg\'ain, S.~Pourojaghi, A.~A.~Sen and M.~M.~Sheikh-Jabbari,
``$S_8$ increases with effective redshift in $\Lambda$CDM cosmology,''
[arXiv:2303.06928 [astro-ph.CO]].
%1 citations counted in INSPIRE as of 14 Jul 2023

\bibitem{ACT:2023dou}
F.~J.~Qu \textit{et al.} [ACT],
``The Atacama Cosmology Telescope: A Measurement of the DR6 CMB Lensing Power Spectrum and its Implications for Structure Growth,''
[arXiv:2304.05202 [astro-ph.CO]].
%10 citations counted in INSPIRE as of 14 Jul 2023

\bibitem{ACT:2023kun}
M.~S.~Madhavacheril \textit{et al.} [ACT],
``The Atacama Cosmology Telescope: DR6 Gravitational Lensing Map and Cosmological Parameters,''
[arXiv:2304.05203 [astro-ph.CO]].
%10 citations counted in INSPIRE as of 14 Jul 2023

\bibitem{ACT:2023ipp}
G.~A.~Marques \textit{et al.} [ACT and DES],
``Cosmological constraints from the tomography of DES-Y3 galaxies with CMB lensing from ACT DR4,''
[arXiv:2306.17268 [astro-ph.CO]].
%0 citations counted in INSPIRE as of 14 Jul 2023

\bibitem{Miyatake:2021qjr}
H.~Miyatake, Y.~Harikane, M.~Ouchi, Y.~Ono, N.~Yamamoto, A.~J.~Nishizawa, N.~Bahcall, S.~Miyazaki and A.~A.~Plazas Malag\'on,
``First Identification of a CMB Lensing Signal Produced by 1.5~Million Galaxies at z\ensuremath{\sim}4: Constraints on Matter Density Fluctuations at High Redshift,''
Phys. Rev. Lett. \textbf{129} (2022) no.6, 061301
%doi:10.1103/PhysRevLett.129.061301
[arXiv:2103.15862 [astro-ph.CO]].
%7 citations counted in INSPIRE as of 25 Jul 2023

\bibitem{Alonso:2023guh}
D.~Alonso, G.~Fabbian, K.~Storey-Fisher, A.~C.~Eilers, C.~Garc\'\i{}a-Garc\'\i{}a, D.~W.~Hogg and H.~W.~Rix,
``Constraining cosmology with the Gaia-unWISE Quasar Catalog and CMB lensing: structure growth,''
[arXiv:2306.17748 [astro-ph.CO]].
%0 citations counted in INSPIRE as of 25 Jul 2023


\bibitem{Herold:2021ksg}
L.~Herold, E.~G.~M.~Ferreira and E.~Komatsu,
``New Constraint on Early Dark Energy from Planck and BOSS Data Using the Profile Likelihood,''
Astrophys. J. Lett. \textbf{929} (2022) no.1, L16
%doi:10.3847/2041-8213/ac63a3
%[arXiv:2112.12140 [astro-ph.CO]].
%43 citations counted in INSPIRE as of 17 Jul 2023

\bibitem{Gomez-Valent:2022hkb}
A.~G\'omez-Valent,
``Fast test to assess the impact of marginalization in Monte~Carlo analyses and its application to cosmology,''
Phys. Rev. D \textbf{106} (2022) no.6, 063506
%doi:10.1103/PhysRevD.106.063506
%[arXiv:2203.16285 [astro-ph.CO]].
%20 citations counted in INSPIRE as of 11 Jul 2023

\bibitem{Meiers:2023gft}
M.~Meiers, L.~Knox and N.~Sch\"oneberg,
``Exploration of the Pre-recombination Universe with a High-Dimensional Model of an Additional Dark Fluid,''
[arXiv:2307.09522 [astro-ph.CO]].
%0 citations counted in INSPIRE as of 22 Jul 2023

\bibitem{Poulin:2018cxd}
V.~Poulin, T.~L.~Smith, T.~Karwal and M.~Kamionkowski,
``Early Dark Energy Can Resolve The Hubble Tension,''
Phys. Rev. Lett. \textbf{122} (2019) no.22, 221301
%doi:10.1103/PhysRevLett.122.221301
%[arXiv:1811.04083 [astro-ph.CO]].
%608 citations counted in INSPIRE as of 17 Jul 2023

\bibitem{Niedermann:2019olb}
F.~Niedermann and M.~S.~Sloth,
``New early dark energy,''
Phys. Rev. D \textbf{103} (2021) no.4, L041303
%doi:10.1103/PhysRevD.103.L041303
[arXiv:1910.10739 [astro-ph.CO]].
%140 citations counted in INSPIRE as of 24 Jul 2023

\bibitem{Jimenez:2001gg}
R.~Jimenez and A.~Loeb,
``Constraining cosmological parameters based on relative galaxy ages,''
Astrophys. J. \textbf{573} (2002), 37-42
%doi:10.1086/340549
%[arXiv:astro-ph/0106145 [astro-ph]].
%598 citations counted in INSPIRE as of 28 Jun 2023

\bibitem{Stern:2009ep}
D.~Stern, R.~Jimenez, L.~Verde, M.~Kamionkowski and S.~A.~Stanford,
``Cosmic Chronometers: Constraining the Equation of State of Dark Energy. I: H(z) Measurements,''
JCAP \textbf{02} (2010), 008
%doi:10.1088/1475-7516/2010/02/008
%[arXiv:0907.3149 [astro-ph.CO]].
%740 citations counted in INSPIRE as of 20 May 2022

\bibitem{Moresco:2012jh}
M.~Moresco, A.~Cimatti, R.~Jimenez, L.~Pozzetti, G.~Zamorani, M.~Bolzonella, J.~Dunlop, F.~Lamareille, M.~Mignoli and H.~Pearce, \textit{et al.}
``Improved constraints on the expansion rate of the Universe up to z\textasciitilde{}1.1 from the spectroscopic evolution of cosmic chronometers,''
JCAP \textbf{08} (2012), 006
%doi:10.1088/1475-7516/2012/08/006
%[arXiv:1201.3609 [astro-ph.CO]].
%508 citations counted in INSPIRE as of 20 May 2022

\bibitem{Zhang:2012mp}
C.~Zhang, H.~Zhang, S.~Yuan, T.~J.~Zhang and Y.~C.~Sun,
``Four new observational $H(z)$ data from luminous red galaxies in the Sloan Digital Sky Survey data release seven,''
Res. Astron. Astrophys. \textbf{14} (2014) no.10, 1221-1233
%doi:10.1088/1674-4527/14/10/002
%[arXiv:1207.4541 [astro-ph.CO]].
%425 citations counted in INSPIRE as of 20 May 2022

\bibitem{Moresco:2016mzx}
M.~Moresco, L.~Pozzetti, A.~Cimatti, R.~Jimenez, C.~Maraston, L.~Verde, D.~Thomas, A.~Citro, R.~Tojeiro and D.~Wilkinson,
``A 6\% measurement of the Hubble parameter at $z\sim0.45$: direct evidence of the epoch of cosmic re-acceleration,''
JCAP \textbf{05} (2016), 014
%doi:10.1088/1475-7516/2016/05/014
%[arXiv:1601.01701 [astro-ph.CO]].
%505 citations counted in INSPIRE as of 17 May 2022

\bibitem{Ratsimbazafy:2017vga}
A.~L.~Ratsimbazafy, S.~I.~Loubser, S.~M.~Crawford, C.~M.~Cress, B.~A.~Bassett, R.~C.~Nichol and P.~V\"ais\"anen,
``Age-dating Luminous Red Galaxies observed with the Southern African Large Telescope,''
Mon. Not. Roy. Astron. Soc. \textbf{467} (2017) no.3, 3239-3254
%doi:10.1093/mnras/stx301
%[arXiv:1702.00418 [astro-ph.CO]].
%162 citations counted in INSPIRE as of 17 May 2022

\bibitem{Borghi:2021rft}
N.~Borghi, M.~Moresco and A.~Cimatti,
``Toward a Better Understanding of Cosmic Chronometers: A New Measurement of H(z) at z \ensuremath{\sim} 0.7,''
Astrophys. J. Lett. \textbf{928} (2022) no.1, L4
%doi:10.3847/2041-8213/ac3fb2
%[arXiv:2110.04304 [astro-ph.CO]].
%10 citations counted in INSPIRE as of 17 May 2022

\bibitem{Jiao:2022aep}
K.~Jiao, N.~Borghi, M.~Moresco and T.~J.~Zhang,
``New Observational H(z) Data from Full-spectrum Fitting of Cosmic Chronometers in the LEGA-C Survey,''
Astrophys. J. Suppl. \textbf{265} (2023) no.2, 48
%doi:10.3847/1538-4365/acbc77
%[arXiv:2205.05701 [astro-ph.CO]].
%14 citations counted in INSPIRE as of 17 Jul 2023

\bibitem{Tomasetti:2023kek}
E.~Tomasetti, M.~Moresco, N.~Borghi, K.~Jiao, A.~Cimatti, L.~Pozzetti, A.~C.~Carnall, R.~J.~McLure and L.~Pentericci,
``A new measurement of the expansion history of the Universe at z=1.26 with cosmic chronometers in VANDELS,''
[arXiv:2305.16387 [astro-ph.CO]].
%1 citations counted in INSPIRE as of 28 Jun 2023

\bibitem{Moresco:2023zys}
M.~Moresco,
``Addressing the Hubble tension with cosmic chronometers,''
[arXiv:2307.09501 [astro-ph.CO]].
%0 citations counted in INSPIRE as of 24 Jul 2023

\bibitem{Moresco:2020fbm}
M.~Moresco, R.~Jimenez, L.~Verde, A.~Cimatti and L.~Pozzetti,
``Setting the Stage for Cosmic Chronometers. II. Impact of Stellar Population Synthesis Models Systematics and Full Covariance Matrix,''
Astrophys. J. \textbf{898} (2020) no.1, 82
%doi:10.3847/1538-4357/ab9eb0
[arXiv:2003.07362 [astro-ph.GA]].
%57 citations counted in INSPIRE as of 28 Jul 2023

\bibitem{Foreman-Mackey:2012any}
D.~Foreman-Mackey, D.~W.~Hogg, D.~Lang and J.~Goodman,
``emcee: The MCMC Hammer,''
Publ. Astron. Soc. Pac. \textbf{125} (2013), 306-312
%doi:10.1086/670067
%[arXiv:1202.3665 [astro-ph.IM]].
%3393 citations counted in INSPIRE as of 17 Jul 2023


\bibitem{Hou:2020rse}
J.~Hou, A.~G.~S\'anchez, A.~J.~Ross, A.~Smith, R.~Neveux, J.~Bautista, E.~Burtin, C.~Zhao, R.~Scoccimarro and K.~S.~Dawson, \textit{et al.}
``The Completed SDSS-IV extended Baryon Oscillation Spectroscopic Survey: BAO and RSD measurements from anisotropic clustering analysis of the Quasar Sample in configuration space between redshift 0.8 and 2.2,''
Mon. Not. Roy. Astron. Soc. \textbf{500} (2020) no.1, 1201-1221
%:10.1093/mnras/staa3234
%[arXiv:2007.08998 [astro-ph.CO]].
%135 citations counted in INSPIRE as of 28 Jun 2023

\bibitem{Neveux:2020voa}
R.~Neveux, E.~Burtin, A.~de Mattia, A.~Smith, A.~J.~Ross, J.~Hou, J.~Bautista, J.~Brinkmann, C.~H.~Chuang and K.~S.~Dawson, \textit{et al.}
``The completed SDSS-IV extended Baryon Oscillation Spectroscopic Survey: BAO and RSD measurements from the anisotropic power spectrum of the quasar sample between redshift 0.8 and 2.2,''
Mon. Not. Roy. Astron. Soc. \textbf{499} (2020) no.1, 210-229
%doi:10.1093/mnras/staa2780
%[arXiv:2007.08999 [astro-ph.CO]].
%133 citations counted in INSPIRE as of 28 Jun 2023

\bibitem{duMasdesBourboux:2020pck}
H.~du Mas des Bourboux, J.~Rich, A.~Font-Ribera, V.~de Sainte Agathe, J.~Farr, T.~Etourneau, J.~M.~Le Goff, A.~Cuceu, C.~Balland and J.~E.~Bautista, \textit{et al.}
``The Completed SDSS-IV Extended Baryon Oscillation Spectroscopic Survey: Baryon Acoustic Oscillations with Ly\ensuremath{\alpha} Forests,''
Astrophys. J. \textbf{901} (2020) no.2, 153
%doi:10.3847/1538-4357/abb085
%[arXiv:2007.08995 [astro-ph.CO]].
%172 citations counted in INSPIRE as of 28 Jun 2023

\bibitem{Trotta:2017wnx}
R.~Trotta,
``Bayesian Methods in Cosmology,''
[arXiv:1701.01467 [astro-ph.CO]].
%96 citations counted in INSPIRE as of 18 Jul 2023

\bibitem{Moresco:2022phi}
M.~Moresco, L.~Amati, L.~Amendola, S.~Birrer, J.~P.~Blakeslee, M.~Cantiello, A.~Cimatti, J.~Darling, M.~Della Valle and M.~Fishbach, \textit{et al.}
``Unveiling the Universe with emerging cosmological probes,''
Living Rev. Rel. \textbf{25} (2022) no.1, 6
%doi:10.1007/s41114-022-00040-z
%[arXiv:2201.07241 [astro-ph.CO]].
%71 citations counted in INSPIRE as of 16 Jun 2023

\bibitem{DESI:2023ytc}
G.~Adame \textit{et al.} [DESI],
``The Early Data Release of the Dark Energy Spectroscopic Instrument,''
%doi:10.5281/zenodo.7964161
[arXiv:2306.06308 [astro-ph.CO]].
%13 citations counted in INSPIRE as of 26 Jul 2023

\bibitem{Akarsu:2022lhx}
O.~Akarsu, E.~\'O~Colg\'ain, E.~\"Ozulker, S.~Thakur and L.~Yin,
``Inevitable manifestation of wiggles in the expansion of the late Universe,''
Phys. Rev. D \textbf{107} (2023) no.12, 123526
%doi:10.1103/PhysRevD.107.123526
%[arXiv:2207.10609 [astro-ph.CO]].
%6 citations counted in INSPIRE as of 17 Jul 2023

\bibitem{Zhao:2017cud}
G.~B.~Zhao, M.~Raveri, L.~Pogosian, Y.~Wang, R.~G.~Crittenden, W.~J.~Handley, W.~J.~Percival, F.~Beutler, J.~Brinkmann and C.~H.~Chuang, \textit{et al.}
``Dynamical dark energy in light of the latest observations,''
Nature Astron. \textbf{1} (2017) no.9, 627-632
%doi:10.1038/s41550-017-0216-z
%[arXiv:1701.08165 [astro-ph.CO]].
%356 citations counted in INSPIRE as of 17 Jul 2023

\bibitem{Wang:2018fng}
Y.~Wang, L.~Pogosian, G.~B.~Zhao and A.~Zucca,
``Evolution of dark energy reconstructed from the latest observations,''
Astrophys. J. Lett. \textbf{869} (2018), L8
%doi:10.3847/2041-8213/aaf238
%[arXiv:1807.03772 [astro-ph.CO]].
%92 citations counted in INSPIRE as of 17 Jul 2023

\bibitem{Escamilla:2021uoj}
L.~A.~Escamilla and J.~A.~Vazquez,
``Model selection applied to reconstructions of the Dark Energy,''
Eur. Phys. J. C \textbf{83} (2023) no.3, 251
%doi:10.1140/epjc/s10052-023-11404-2
%[arXiv:2111.10457 [astro-ph.CO]].
%13 citations counted in INSPIRE as of 17 Jul 2023

\end{thebibliography}
\end{document}


\textbf{Language Model and Dataset:} 
We utlize three language models - GPT-2\citep{radford2019language}, OPT-1.3B\citep{zhang2022opt}, and LLaMA-7B\citep{touvron2023llama} - to generate watermarked text. Consistent with previous works, we employ two decoding methods: Top-K sampling and Beam search, to produce the watermarked text.  We select the C4\citep{raffel2020exploring} and Dbpedia Class \citep{raffel2020exploring} datasets and use the first 30 words of each text as the prompt. For each prompt, the language model would generate the next 200 ± 5 tokens.

\textbf{Evaluation:} The objective in evaluating watermarking algorithms is to distinguish human-written text from language model-generated text. Specifically, we select 500 texts from the dataset as human text and 500 language model-generated texts for evaluating binary classification metrics. We thoroughly document the false positive rate, false negative rate and F1 score.

\textbf{Hyper-parameters:} The default hyperparameters are configured as follows: watermark token ratio $\gamma$ of 0.5, window size $w$ of 5, five token embedding layers, and $\delta=2$ for the generator. The detector comprises two LSTM layers as well as the same token embedding layers. For decoding, Top-K employs \(K=20\) and Beam Search utilizes a beam width of 8. Both networks adopt a learning rate of 0.01, optimized via the Adam optimizer \citep{kingma2014adam}.  Due to the varying sensitivity of language models to the watermark, for ease of comparison, the detection z-score thresholds for GPT-2, OPT 1.3B, and LLaMA 7B are set to 1, 1 and 3, respectively.

% We evaluate our watermarking method using GPT-2 \citep{radford2019language}, OPT-1.3B \citep{zhang2022opt}, and OPT-2.7B \citep{zhang2022opt} models with both top-K sampling and beam search text generation. The models generate completions to C4 \citep{raffel2020exploring} and Dbpedia \citep{gangemi2012automatic} prompts of length 30. Each model generates 200$\pm$5 tokens per prompt. We compare model-generated text (watermarked) to original human-written text from the datasets (non-watermarked). The false positive rate (human text incorrectly flagged as watermarked) and false negative rate (undetected watermarked text) measure effectiveness.

% The default hyperparameters are: 0.5 watermarked token ratio, window size 5, 5 token embedding layers,  $\delta$=2 for the generator. For the detector, there are 2 LSTM layers. Top-K uses K=20. Beam search uses a beam width of 8. And the learning rate is 0.01 with Adam optimizer for both network.

% We utilize GPT-2 \citep{radford2019language}, OPT-1.3B \citep{zhang2022opt}, and OPT-2.7B \citep{zhang2022opt} as the models for generating watermarks. For each model, we adopt both top-K sampling and beam search methods for text generation. The specific details of the two sampling methods have already been mentioned in section \ref{sec:generate}.

% Meanwhile, we use the C4 \citep{raffel2020exploring} and Dbpedia Class datasets \citep{gangemi2012automatic} to evaluate our watermark algorithm. Specifically, following the approach of \citet{kirchenbauer2023watermark}, we selected texts with length 30 from these datasets as prompts, and let the language models perform completions given these prompts. For each prompt, the models would generate $T=200\pm5$ tokens. We used the completions from the original datasets as the non-watermarked text (human text), and the text generated by our models as the watermarked text. The effectiveness was evaluated based on the ratio of false positive errors (human text falsely flagged as watermarked)  and false negative errors (watermarked text not detected).

% Unless specified otherwise, the hyperparameters used in the experiment are as follows: for the generator network, the ratio of green labels generated is 0.5, the window size is 5, the layer number of the token embedding network is 5, and the value of $\delta$ is set to 2. For the detector network,  the value of z used in training is 4, and the number of LSTM network layers is 2. When using the top-K sampling method, the K is set to 20 and the beam size of the beam search method is set to 8.

\subsection{Main Results}

Table \ref{tab:main} shows a comparison of detection efficacy between our private watermarking algorithm and current public watermarking algorithms. The public baseline watermarking algorithm is the method of \cite{kirchenbauer2023watermark}, which directly computes the number of watermarked (green) tokens and then calculates the z-score.

% Table \ref{tab:main} demonstrates the detection accuracy of the private watermarking algorithm. We refer to the method which utilizes the label of each token to calculate the z-value (section \ref{sec:detect}) as the public watermarking algorithm and use this algorithm as the baseline for our comparison. The hyper-parameters used are $\delta=2.0$ and $\gamma=0.5$. The detection network is trained following a z-value threshold of 4.

As illustrated in Table \ref{tab:main}, similar to the public watermarking algorithm, our private watermarking algorithm also scarcely produces false positive results (0.1\% and 0.4\% on average), meaning that human text is almost never mistakenly identified as watermarked text. Moreover, in most scenarios, the false negative probability is only marginally higher than the public watermarking algorithm by an average of $1.2\%$. Considering that the performance of the public watermarking algorithm represents the strict upper bound of our method, this is indeed an outstanding result.  We also find that the performance trends of our algorithm across different settings (language models, datasets) are similar to the public watermarking algorithm.  This demonstrates the general applicability of our private watermarking method without using watermark generation rules. We will analyze the difficulty of inferring watermark generation details from the detector in subsequent analyses.

% performing slightly better on smaller language models. Using beam search enables deeper integration of the watermark into the text.

% For some special cases when the z-value is not properly selected (using the top-K sampling with OPT 1.3B and 2.7B models to test the DBpedia CLASS dataset), even the public detection algorithm generates more false negatives cases and our private detection methods would also decrease in performance.  With a properly selected z-value threshold, the private watermarking algorithm exhibits similar performance across different decoding methods, various language models, and disparate domain datasets, which demonstrates its strong generalizability and adaptability.

\section{Ablation study on YCBV}
\label{sec:ablation_ycbv}

In Tab.~\ref{tab:ablation_ycbv} we report the results of our ablation study on YCBV~\cite{ycbv}.
We choose the Large Marker object and train a single model on it for each modification we applied.
Each model is trained for 20 epochs on the standard training set.
For the computation of the Feature Matching Recall (FMR), we set the distance threshold $\tau_1=10$ voxels and the inlier ratio threshold $\tau_2=5$\%, to account for the different density of the scene point cloud in YCBV.
All the other settings and parameters are the same as those in our ablation study on LMO~\cite{lmo} in the main paper.

We can observe that some changes do not increment performance, but instead cause a slight drop, in particular when adapting the safety threshold to the object dimension (third row, $-0.4$) and when colour augmentation is applied (sixth row, $-$0.3).
These additions do not benefit this particular object, but are instead advantageous when averaging all the object in the dataset.

We can note that, as in the ablation study on the LMO dataset in the main paper, the most significant improvements in ADD-S AUC result from applying the safety threshold ($+$1.5), adding RGB information ($+$5.5), and using the Adam optimiser ($+$12.3).
\renewcommand{\arraystretch}{0.9}
\begin{table}%[t!]
\centering
\tabcolsep 3pt
\caption{
Ablation study on the Large Marker object of YCBV.
Performance are compared in terms of RRE [radiants] and RTE [cm] errors (the lower the better), and FMR and ADD-S AUC (shortened to ADD) scores (the higher the better).
$\Delta$ shows the improvement of each contribution in terms of ADD-S AUC with respect to the previous row.
}
\vspace{-3mm}
\resizebox{\columnwidth}{!}{%
\begin{tabular}{clrrrrr}
\toprule
& Improvements &
RRE{\color{black!50}{$\,\downarrow$}} &
RTE{\color{black!50}{$\,\downarrow$}} & 
FMR{\color{black!50}{$\,\uparrow$}} & 
ADD{\color{black!50}{$\,\uparrow$}} & 
$\Delta$ \\ 
\toprule
& Baseline & 2.0 & 4.6 & 0.00 & 77.2 & -- \\
\midrule
\multirow{2}{*}{\rotatebox{90}{Loss}} & $+$ $\tau_{NS} = 0.1 D_S$ & 2.0 & 4.2 & 0.00 & 78.7 & $+$1.5 \\
& $+$ $\tau_{NS} = 0.1 D_O$ & 2.0 & 4.3 & 0.00 & 78.3 & $-$0.4 \\
\midrule
\multirow{2}{*}{\rotatebox{90}{Arch.}} & $+$ Independent weights & 2.0 & 4.1 & 0.00 & 79.4 & $+$1.1 \\
& $+$ Add RGB information & 1.2 & 3.2 & 49.1 & 84.9 & $+$5.5 \\
\midrule
\multirow{2}{*}{\rotatebox{90}{Aug.}} & $+$ Color augmentation & 1.2 & 3.3 & 50.0 & 84.6 & $-$0.3 \\
& $+$ Random erasing & 1.2 & 3.1 & 53.4 & 85.2 & $+$0.6 \\
\midrule
\multirow{2}{*}{\rotatebox{90}{Optim.}} & $+$ SGD $\to$ Adam & 0.0 & 0.4 & 100 & 97.5 & $+$12.3 \\
& $+$ Adam $\to$ AdamW  & 0.0 & 0.4 & 100 & 97.5 & 0 \\
& $+$ Exp $\to$ Cosine & 0.0 & 0.4 & 100 & 97.4 & $-$0.1 \\\bottomrule
\end{tabular}}
\label{tab:ablation_ycbv}
\end{table}
\renewcommand{\arraystretch}{1}

\section{Additional ablation study on LMO}

We include an ablation study on the $t_\text{scale}$ hyperparameter, which is used to set the radius of the ball volume in which negative mining around a certain point is not allowed. We train on the Can object of LMO using the standard setting, and varying only $t_\text{scale}$. The results are shown in Tab.~\ref{tab:ablation_ycbv}.
We can observe that our choice of $t_\text{scale} = 0.1$ leads to the best result. When $t_\text{scale}$ is increased, many candidate points are forbidden to be used as negatives, therefore decreasing the final performance. On the other hand, a lower $t_\text{scale}$ implies negative pairs composed by points which are near in the 3D space. This reduces the performance, as similar points are forced to have different descriptors. Notably, the worst results is obtained when $t_\text{scale} = 0.1$, i.e. when no negative candidates are excluded.

\begin{table}
\tabcolsep 3pt
\caption{
Ablation study on the Can object of LMO. Performance is shown in terms of ADD-0.1 (the higher the better) in function of the hyperparameter $t_\text{scale}$.}
\centering
\resizebox{.9\columnwidth}{!}{
\begin{tabular}{c|ccccc}
    \toprule
    $t_\text{scale}$ & 0.0 & 0.01 & 0.05 & \textbf{0.1} & 0.5 \\
    ADD-0.1d & 66.55 & 91.80 & 93.79 & \textbf{93.95} & 81.28 \\
    \bottomrule
\end{tabular}
\label{tab:tscale}
}
\end{table}



\subsection{Analysis of Shared Embedding}

To demonstrate the necessity of shared embedding layers between the generation network and detection network, we compared the watermark detection performance under three settings in Table \ref{tab:ablation}: without using shared embedding layers, using shared embedding layers but not fine-tuning them during detection network training, and using shared embedding layers while also fine-tuning them during detection network training. All the experiments employ top-K sampling.

% An ablation study in Table \ref{tab:ablation} illustrates the effectiveness of shared token embedding for the detection network. Experiments on GPT2, OPT1.3B, and OPT2.7B models using the C4 and DBPEDIA CLASS datasets compare three settings: shared token embedding, no shared token embedding, and fine-tuned shared token embedding.

Without the shared layer, F1 score decreases dramatically, by 72.0\% on average. This renders the detection algorithm nearly useless.  Furthermore, fine-tuning the shared embedding layers decreases F1 score by 11.1\%. These results demonstrate using the generator's token embedding layers without further fine-tuning is optimal  for the detection network.

% Although fine-tuning the shared layer reduces false negatives, it also introduces some false positives. Since incorrectly identifying human text as watermarked has more severe consequences, we opt not to fine-tune the shared layer.

% To further analyze the private watermark algorithm, we conduct an ablation study in Table \ref{tab:ablation}  to illustrate the effectiveness of shared token embedding for the detection network. Specifically, the experiment is conducted on the GPT2, OPT1.3B, and OPT2.7B language models on the C4 and DBPEDIA CLASS datasets. We have presented results under three different settings: using shared token embedding, not using shared token embedding, and fine-tuning shared token embedding.

% As seen from Table \ref{tab:ablation}, without the shared layer, the proportion of false negatives (watermarked text not
% detected) and false positives (human text falsely flagged as watermarked) dramatically decreases on an average of $15.1\%$ and $32.0\%$ respectively. This renders the entire detection algorithm almost inapplicable. Concurrently, although fine-tuning the shared layer reduces the occurrence of false negatives, it also introduces some instances of wrongly tagged human text. Given that mistakenly recognizing human text as the watermarked text presents more severe consequences, we eventually choose to adopt the method without fine-tuning the shared layer.


% Figure environment removed

% It is crucial that the details of how our private watermark is generated remain difficult to infer. Thus, we further analyzed the accuracy of attack methods that attempt to obtain the rules used to generate the watermark. Specifically, we employed the two attack methods mentioned in Section 4.5: reverse training a generation network from a detection network, and directly cracking the watermark based on token frequency.




\subsection{Privacy Analysis}
\label{sec:pri}

It is crucial that the details of watermark generation remain difficult to infer. Thus, we conducted an analysis of the accuracy of potential attack methods aimed at discerning the watermark generation rules. Specifically, we examined two attack methods mentioned in Section \ref{sec:ana}: reverse training of the generation network from the detection network, and direct cracking of the watermark based on token frequency. For the reverse training generation network from the detection network attack method, the retrained generation network also shares parameters in the token embedding layers of the detection network without further fine-tuning. The setting \textit{w\_ft. params} indicates that the token embedding parameters have been fine-tuned during the training of the detection network or and the setting \textit{w fixed. params} indicates the parameters are the same as the origin generation network.

From Figure \ref{fig:z}(a), our watermark detection algorithm can maintain a relatively stable detection accuracy as the window size increases. However, the effectiveness of the attack method based on reverse training decreases gradually as the window size increases until it drops to the minimum value of 0.5 (random guess). This strongly validates our explanation in Section \ref{sec:ana} that training the generator with the detector is a computational asymmetrical inverse process. At the same time, the cracking method directly based on word frequency is completely useless when the window size is not particularly small. This experiment demonstrates that our method has good privacy properties. More details about the attack algorithms could be seen in the appendix.


\subsection{Hyper-parameter and Error Analysis}

% To further analyze how our private watermarking algorithm works, we examine the effects of key hyperparameters, specifically the impact of different $\delta$ values on detection accuracy and text quality. Text quality is evaluated using the LLaMMA 13B \citet{touvron2023llama}.  As shown in Figure \ref{fig:z} (b), with the increase in the value of $\delta$, the accuracy of the detection network also increases. However, the corresponding perplexity (PPL) value of the generated text will also rise. Weighing these factors, we finally chose a $\delta$ value of 2, which maintains detection accuracy without excessively impacting the text quality.

In investigating the efficacy of our private watermark algorithm, we focus on the influence of the $\delta$ values on the detection F1 score and the text quality. Text quality is assessed by perplexity using the LLaMA 13B  \citep{touvron2023llama}. Figure \ref{fig:z} (b) demonstrates that as $\delta$ increases, so does the F1 score of the detection network; however, this comes at the cost of an increased perplexity (PPL) value in the generated text. After considering these trade-offs, we selected a $\delta$ value of 2 to optimize detection accuracy without severely compromising text quality.

To analyze the error cases, we present the z-score distributions of human and watermarked texts, as well as the algorithm's detection F1 score at different z-score ranges in Figure \ref{fig:err}(a). These results are generated by GPT2 on the C4 dataset. Figure \ref{fig:err}(a) reveals that the human and watermarked texts exhibit nearly normal distributions around 0 and 9, respectively. The detection accuracy of the private watermark algorithm drops significantly around the z-score threshold of 4 but approaches 100\% in other ranges. This indicates that our algorithm is highly reliable for inputs with definite labels.

% To better understand how the private watermark algorithm works, we perform a series of analyses on several key hyper-parameters. Specifically, we tested the influence of different z value thresholds and $\delta$ values in Figure \ref{fig:z} (a) and Figure \ref{fig:z} (b) respectively. When analyzing different z value thresholds, the value of $\delta$ is set to 2.0, and when analyzing different $\delta$ values, the value of z is set to 4. We use the LLaMA 13B model \citet{touvron2023llama} to calculate the perplexity (PPL) value in Figure \ref{fig:z}(b).



% \subsection{Error Analysis}

% To better analyze the error cases of the private watermark algorithm, we present the z-score distributions of both the human text and the watermarked text, as well as the detection accuracy of the algorithm at different z-score ranges in Figure \ref{fig:err}(a). These results are generated by GPT2 on the C4 dataset. As can be observed from Figure \ref{fig:err}(a), the human text and watermarked text exhibited a normal-like distribution centered around 0 and 9 respectively. The detection accuracy of the private watermark algorithm is relatively low only around the z-score threshold 4, while it is almost 100\% in other ranges. This suggests that for inputs with highly certain labels, our algorithm is quite reliable.

 % To conducted a more detailed analysis of the error cases of our private watermark algorithm, as shown in figure \ref{fig:err} (a). In this experiment, the watermark detection network uses a z value of 4 during training.  In this figure, we first compute the actual z-scores for all the test data, divide these z-scores into different ranges, and then perform statistical analysis on the detection accuracy for each range. As can be seen from figure \ref{fig:err} (a), all errors are concentrated near the threshold $z=4$. Therefore, the model only makes judgment errors on the confusing examples (near the threshold), while for cases with high confidence, it rarely makes mistakes.

 % Figure environment removed


\subsection{Watermark Generation Network Analysis}

It is critical that the watermark generation network produce a stable watermarked token ratio, as the modified z-score calculation (equation \ref{new-z}) depends on the variance of this ratio. Therefore, in this section, we calculate the actual mean and variance of the watermark labels. 

Specifically, we train the watermark generation network using 5000 data items with strictly a 0.5 ratio of watermarked tokens. As seen in figure \ref{fig:err} (b), the ratio approaches the target value 0.5 as the training loss decreases, and its standard deviation also diminishes. The standard deviation can be controlled within 0.02, corresponding to a variance of less than $4e-4$. According to equation \ref{new-z}, $\sigma^2  T$ could be nearly neglected in the final z-value calculation. 

% We adopt the value 0.02 in the revised z-score calculation.

% Based on our analysis in the section \ref{sec:detect}, it is critical for the watermark generation network to generate a stable label ratio because the modified z-score calculation (equation \ref{new-z}) is dependent on the variance of the label ratio. Therefore, in this section, we calculate the actual mean and variance of the labels generated by the watermark generation network. 

% Specifically, we train the watermark generation network using 5000 data items with strictly a 0.5 ratio of green labels, using the Adam optimizer \cite{kingma2014adam} with a learning rate of 0.001. As can be seen from figure \ref{fig:err} (b), the ratio of green labels gradually approaches the target value 0.5 with the training loss decreases, and its standard deviation also gradually diminishes. Ultimately, the standard deviation can be controlled within 0.02, corresponding to a variance of less than $4e-4$. According to equation \ref{new-z}, $\sigma^2  T$ could be nearly neglected in the final z-value calculation. We adopt the value 0.02 in the revised z-score calculation.

\subsection{Time Complexity Analysis}

The private watermark generation process requires an additional network, potentially introducing computational overhead. However, our watermark generation network contains only 43k parameters, negligible compared to the 124M to 2.7B parameters in GPT2, OPT-1.3B, and OPT-2.7B models. Empirically, decoding a token takes 30ms in GPT2  on a single Tesla V100 GPU, and our watermarking adds only 1ms on average, even less for larger models. Thus, our watermark generation algorithm adds minimal computational burden. In prior experiments, we selected relatively small top-K values. However, given that networks can perform batch computations, even large K values do not affect the computation time. A detailed analysis will be provided in the appendix.

% Due to the private watermark generation process employing an additional watermark generation network, there is a risk of introducing an extra computational burden. Therefore, we analyze the time complexity of the watermark generation process in this section.

% First, we compare the number of parameters in the watermark generation network and the language model. Our  watermark generation network only consists of 43k parameters, whereas GPT2, OPT1.3B, and OPT2.7B have 124M, 1.3B, and 2.7B parameters respectively. It is evident that compared to the large language models with an enormous number of parameters, the number of parameters in our watermark generation network can be considered almost negligible.

% Then we analyze the actual running time. On a single Tesla V100 GPU, decoding a token in GPT2 requires 30ms, whereas incorporating our watermark generation network only adds an average of 1ms to the cost. For models with a larger number of parameters, such as OPT1.3B and OPT2.7B, the influence on the decoding time is even smaller. Hence, our watermark generation algorithm does not cause significant additional computational overhead.

\section{Conclusion}

In this paper, we propose the first private watermarking algorithm for large language models. Unlike previous detection methods that require the watermark key for detection, our method uses a separate detection neural network to detect the watermark. In experiments, we demonstrate the difficulty of inferring the watermarking method from the detection network, while our method achieves similar F1 scores to direct z-score computation and applies to various language models and datasets. In the future, details of the watermark generation and detection networks can be further optimized. Furthermore, as the focus of this study is on private watermarks, the robustness of watermarking algorithms in text modification scenarios is not our priority. In fact, our robustness in this setting is similar to corresponding public watermarking methods. Future work could investigate improving watermark robustness under private watermarking constraints.


\bibliography{iclr2024_conference}
\bibliographystyle{iclr2024_conference}
\newpage
\appendix



\begin{table*}[t]
\tiny
\centering
\resizebox{0.99\linewidth}{!}{
\begin{tabular}{p{2cm}|p{3cm}|p{3cm}|p{3cm}|p{0.35cm}|p{0.44cm}|p{0.3cm}|p{0.3cm}|p{0.3cm}}
\toprule
prompt & real completion &  no watermark (NW) &  watermarked (W) &(NW) $z$   &(W) $z$ &(Real) PPL & (NW) PPL & (W) PPL \\
\midrule
DPR members Jim Ragsdale, Diane Kane, Angeles Liera and pro tem chair Mike Costello discuss condo conversion projects.\textbackslash n During the Jan  & .10 meeting of the La Jolla Development Permit Review committee (DPR), board members voted unanimously to form a research subcommittee that will look into the consequences of condo conversion in the neighborhoods south of Pearl Street.[...continues] &  . 27 meeting, Ragsdale and Kane discussed the need for a condo market and how to get there. Costello spoke about the need to have a condo market in the area but also said there is a need to be able to rent a condo and that the area is growing. [...continues]& . 11-12 meeting, the city announced that the development of condos to be built in the historic downtown has been approved by the city.\textbackslash n \"We feel it's important to be able to provide affordable housing for the people of the city in a way that the community feels they [...continues] &  0.71 &  10.5 & 5.42& 8.26 &  7.15 \\\midrule
In their first game since dropping out of the top five, the Irish delivered a redemption performance against Boston College, picking up a 50-point win over &    
 the Eagles while simultaneously moving one step closer to cementing Arike Ogunbowale’s legacy, as the senior guard passed current associate coach Beth Cunningham on the list of all-time scorers in the program.\textbackslash n No. 6 Notre Dame (23-2, 10-2 ACC) wasted [...continues] 
&    the Eagles. The Irish also defeated the Bulldogs in the final and will face the Bulldogs in the final.\textbackslash n The Irish also defeated the Eagles in the final and will face the Bulldogs in the final. Boston College:\textbackslash n The Eagles had a very good game against the Irish, [...continues] 
& South Carolina in the College Football Playoff Tournament to secure a berth in the NCAA Tournament. The Eagles will meet Notre Dame in the Big 12 Tournament on Sept. 14 in Austin, Texas.\textbackslash n  n\"We've got to win in the first round [...continues] &  1.13 &      11.3 & 4.78 & 6.78 &  9.15 \\\midrule
Two officers of a company that operates three Manhattan hotels were indicted yesterday in a scheme to help homeless people fraudulently obtain welfare checks and split the money  &   with the hotels.\textbackslash n District Attorney Robert M. Morgenthau of Manhattan, who announced the indictments, said they resulted from a study of the three hotels announced last December by the city. He said his office was delayed in moving more quickly on the case because of difficulty[...continues] &  they received from the government. \textbackslash n The indictment, which was released on Tuesday, said that police officers, who arrived at the hotel on a routine shift, met with a homeless man who asked for a check from his landlord. man told the officers about the scheme, which involved[...continues] &  between themselves and the homeless.\textbackslash n The scheme, alleged to be connected to the New York City Department of Health and Welfare, was uncovered in the wake of the 2012 Sandy Hook Elementary School shooting and the 2011 bombing of the Boston Marathon. Authorities say that the scheme  [...continues] &    -1.9 &  9.94 & 4.83 & 7.02 &  7.05 \\\midrule
Buddhadev had written a strong letter of protest to Manmohan Singh objecting to Mulford's behaviour. \textbackslash n Taking serious exception to  & US Ambassador David Mulford writing directly to West Bengal Chief Minister Buddhadev Bhattacharjee for his remarks against the American President, the CPI-M on Friday said the party[...continues] &  the comments made by Mulford, the BJP MP also called on the CM to resign immediately and the Centre to make a statement in the coming weeks.\textbackslash n In his letter to Manmohan Singh, the MP said he was not opposed to[...continues] & Mr Mulford's behaviour in the media and in the Parliament, the Union Minister has directed the Union Secretaries of Parliament and the Secretaries of the Supreme Court to take action against him in the matter[...continues] &  1.85 & 12.36 & 4.25 & 7.35 &      8.15 \\
\bottomrule
\end{tabular}}
\caption{Selected output examples from non-watermarked (NW) and watermarked (W) top-K sampling using $\gamma=0.5$, $\delta=2.0$ and $k=20$.  
}
\label{tab:demo-examples}
\end{table*}



\section{Case study}

To better illustrate the text generated by the watermarked LLM, we have listed some text examples from both the watermarked LLM and the non-watermarked LLM in Table \ref{tab:demo-examples}. We compare the z-scores and PPL scores between these texts. Specifically, when calculating PPL scores, we utilize the LLaMA 13B model \cite{touvron2023llama}. The results from table \ref{tab:demo-examples} demonstrate that the z-scores for texts generated by the watermarked LLM are significantly higher than those from the non-watermarked LLM, while there isn't a significant increase in the PPL scores.


\section{Details of Reverse Training}

In Sections \ref{sec:ana} and \ref{sec:pri}, we analyzed the challenges of training a watermark generation network using training data from a watermark detection network, both theoretically and experimentally. Here we elaborate on the intricacies of this reverse training process.

A key aspect is the definition of loss function for reverse training. Given a sentence's token list $\mathbf{x} = [x_1, ..., x_n]$, the detection network $\mathcal{D}$ assigns a label indicating watermarked (1) or not (0). This could also calculated by the new generator network $\mathcal{W}$.  Specifically, when this proportion is 0.5, the generation network $\mathcal{W}$'s prediction is:

\begin{equation}
\hat{p} = \frac{\sum_{i}^{T-w}\mathcal{W}(x_{i:i+w})}{T-w}
\end{equation}

The final reverse training loss is:

\begin{equation}
L(\mathcal{W}) = -\mathcal{D}(\mathbf{x}) \log(\hat{p}) - (1 - \mathcal{D}(\mathbf{x})) \log(1 - \hat{p})
\end{equation}

For training, we used 10,000 random token lists of length 100-200, labeled by detector $\mathcal{D}$.


% In Sections \ref{sec:ana} and \ref{sec:pri}, we analyzed the challenge of training a watermark generation network using training data generated by a watermark detection network from both theoretical and experimental perspectives. In this section, we provide a more detailed description of the intricacies involved in this reverse training process.

% A pivotal aspect concerns the loss function for reverse training. Given a sentence's token list $\mathbf{x} = [x_1, …, x_n]$, the detection network $ \mathcal{D}$ assigns a label to the sentence, indicating whether it's watermarked (1) or not (0). This label depends on the watermark status of all tokens in the sentence. Specifically, when the proportion of watermarked tokens is 0.5, the predictive result of the generate network $\mathcal{W}$ is given by:
% \begin{equation}
% \hat{p} = \frac{\sum_{i}^{T-w}\mathcal{W}(x_{i:i+w})}{T-w} 
% \end{equation}

% The final loss can be represented by the following equation:
% \begin{equation}
% L(\mathcal{W}) = -\mathcal{D}(\mathbf{x}) \log(\hat{p}) - (1 - \mathcal{D}(\mathbf{x})) \log(1 - \hat{p}) .
% \end{equation}

% During training, we used 10,000 random token lists of length 100-200 as training data, labeled by the detection network $\mathcal{D}$.

\section{Details of Direct Cracking Attack}

\begin{table}[ht]
\centering
\caption{Occurrence frequencies of the most common (w-1)-grams under varying window sizes.}
\vspace{10pt}
\label{tab:my_table}
\begin{tabular}{c|c|c}
\hline
\textbf{Window Size} & \textbf{Most Common Prefix} & \textbf{Frequency} \\
\hline
2 & \textit{the} & 0.04 \\
\hline
3 & \textit{. a} & 0.018 \\
\hline
4 & \textit{âĢ Ļ s} & 0.004 \\
\hline
5 &,  âĢ  Ļ  Ġhe  Ġsaid & 0.00008  \\
\hline
\end{tabular}
\label{tab:fre}
\end{table}

In this section, we detail another attack method mentioned in section \ref{sec:pri}: the direct cracking attack. This method, originating from the work of \cite{sadasivan2023can}, does not require a watermark detector. Instead, it solely analyzes changes in token frequency distributions. Specifically, with a window size of w, it examines the frequency the wth token appears given a prefix of length w-1 tokens. If a token's frequency significantly increases in the presence of a watermark, it is deemed to be part of the watermark token set.

As in \cite{sadasivan2023can}'s method, we examine the 181 most common (w-1)-token n-grams. Since even the most common (w-1)-grams become increasingly rare as w grows, predicting watermarked text also becomes increasingly difficult with this approach.  As shown in Table \ref{tab:fre}, as the window size increases, the frequency of the most common prefixes decreases markedly. Analyzing this may require an extremely large amount of data, but through our analysis in Figure \ref{fig:data_size}, when the window size grows to a certain extent, even increasing the data amount cannot significantly improve the effect.

 % Figure environment removed


\section{Expanding the K In top-k sampling}

% \documentclass{article}
% \usepackage{tabularx}

% \begin{document}

\begin{table}[h]
\centering
\caption{Resource consumption metrics of watermark generation network during 200-Token generation task under various k values and window sizes.}
\vspace{10pt}
\resizebox{0.8\linewidth}{!}{
\begin{tabular}{cccc}
\hline
\textbf{K Value} & \textbf{Window Size} & \textbf{Computation Time (s)} & \textbf{Memory Consumption (MB)} \\
\hline
\multirow{2}{*}{20} & 5 & 0.62 & 23 \\
 & 10 & 0.62 & 23 \\
\hline
\multirow{2}{*}{200} & 5 & 0.65 & 25 \\
 & 10 & 0.65 & 27 \\
\hline
\multirow{2}{*}{2000} & 5 & 0.67 & 46 \\
 & 10 & 0.70 & 48 \\
\hline
\multirow{2}{*}{20000} & 5 & 1.20 & 228 \\
 & 10 & 1.60 & 354 \\
\hline
\end{tabular}}
\label{tab:time}
\end{table}

% 
% \end{document}


In Table \ref{tab:time}, we comprehensively examined the impact of varying k-values on computation time and GPU memory utilization. All timings were recorded on a single V100 32G GPU. It can be observed that even when processing 20,000 tokens simultaneously, the computation time is only twice that of processing 20 tokens, while the memory consumption increases tenfold (with a window size of 5). This indicates that our algorithm demonstrates exceptionally low computational complexity with respect to increasing k-values.


\section{Detection accuracy across different domains}




\begin{table}[bt!]
\centering
\caption{Comparison of detection accuracy of our network-based watermark detector across text domains in the DBPEDIA CLASS dataset versus key-based watermark detectors.}
\vspace{10pt}
% \resizebox{0.93\linewidth}{!}{
\begin{tabular}{llccc}
\toprule
\multicolumn{2}{c}{\textbf{Setting}} & \multicolumn{1}{c}{\textbf{FPR}} & \multicolumn{1}{c}{\textbf{FNR}} & \multicolumn{1}{c}{\textbf{F1}}   \\
% \midrule 
\midrule 
\multirow{2}{*}{\textbf{Domain: Agent}}  & Key-based Detector  &0.5 & 0.0 & 99.8  \\
& Network-based Detector &0.0 & 0.7 & 99.6\\
\midrule 
% \midrule 
% \midrule 
\multirow{2}{*}{\textbf{Domain: Place}}  & Key-based Detector  &0.0 &0.5 & 99.7  \\
& Network-based Detector &0.1 & 2.4 & 98.9\\
% \midrule 
% \midrule 
\midrule 
\multirow{2}{*}{\textbf{Domain: Species}}  & Key-based Detector  &0.0 &1.0 & 99.5  \\
& Network-based Detector &0.0 & 2.8 & 98.3\\

\bottomrule
\end{tabular}

\label{tab:domain}
% \vspace{-0.3in}
\end{table}

To validate our claim in Section \ref{sec:net} that our watermark detection algorithm is robust to differences in text domain, we compare our private watermark detector to current public watermark detectors across various text domains in the DBPEDIA CLASS dataset (Table \ref{tab:domain}). Our algorithm achieves consistently high detection F1 score across domains, primarily because the random token ID lists used during training explore the space of $|V|^T$ possible inputs, covering all plausible texts. Thus, our detector can theoretically achieve similar detection accuracy across text domains.

\end{document}
