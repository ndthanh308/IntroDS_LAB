% !TEX root=main.tex
\documentclass[10pt, a4paper, twocolumn, aps, pra, showpacs, longbibliography, nofootinbib,superscriptaddress]{revtex4-2}

%\usepackage[utf8x]{inputenc}
\usepackage{ucs}
\usepackage{amsmath}
\usepackage{amsfonts}
\usepackage{amssymb}
\usepackage{mathtools}
\usepackage{makeidx}
\usepackage{cellspace,booktabs}
\usepackage{natbib}
\usepackage{lipsum}
\usepackage{bm}
\usepackage{bbm}
\usepackage{relsize}
%\usepackage{bibunits}
%begin hyperref setup
\usepackage{hyperref}
\hypersetup{colorlinks = true, linkcolor=magenta,citecolor=blue, urlcolor=magenta, bookmarksnumbered =  true}
%End hyperref setup
%\usepackage{subcaption}
%\usepackage[format=hang,justification=justified,singlelinecheck=false]{caption}

\usepackage[usenames,dvipsnames]{color}
\definecolor{light-gray}{gray}{0.55}

\usepackage{microtype}

\usepackage{graphicx}


\renewcommand{\dag}{^{\dagger}}
\newcommand{\ssm}{\rm\scriptscriptstyle}
\newcommand{\exv}[1]{ \langle #1 \rangle }

\newcommand{\bra}[1]{ \langle #1 \rvert }
\newcommand{\ket}[1]{ \lvert #1 \rangle}
\newcommand{\braket}[2]{\langle #1 \vert #2 \rangle }
\newcommand{\innerbraket}[3]{\langle #1 \vert #2 \vert #3 \rangle }
\newcommand{\tr}[2][]{\text{Tr}_{ #1 } ( #2 )}
\newcommand{\up}{\uparrow}
\newcommand{\down}{\downarrow}

%\newcommand{\overlr}[1]{\overleftarrow{\overrightarrow{#1}}}
\newcommand{\overlr}[1]{\overset{\leftrightarrows}{#1}}
\renewcommand{\overleftarrow}[1]{\overset{\leftarrow}{#1}}
\renewcommand{\overrightarrow}[1]{\overset{\rightarrow}{#1}}


\newcommand{\pfrac}[2]{\frac{\partial #1}{\partial #2}}
\newcommand{\intinf}{\int_{-\infty}^{\infty}}

\newcommand{\parspace}{\vspace{0.4cm}}

\DeclareRobustCommand{\EG}[1]{{\color{blue}#1}}
\DeclareRobustCommand{\CKA}[1]{{\color{red}{#1}}}

\DeclareRobustCommand\VC[1]{{\color{ForestGreen}VC: #1 }}

\DeclareRobustCommand\AV[1]{{\color{Magenta}AV: #1 }}
\begin{document}
%TC:ignore
\begin{abstract}
The two-dimensional electron gas is 
of fundamental importance in quantum many-body physics.
We study a minimal extension of this model with $C_4$ (as opposed to full rotational) symmetry and an 
electronic dispersion 
with two valleys with anisotropic effective masses. 
Using variational Monte Carlo simulations, we find a broad intermediate range of densities 
with a   metallic valley-polarized, spin-unpolarized  
ground state.  Our results are of direct relevance to the recently discovered ``nematic'' state in AlAs quantum wells. For the effective mass anisotropy relevant to this system, $m_x/m_y\approx 5.2$, we obtain a transition from an anisotropic metal to a valley-polarized metal at $r_s \approx 12$ (where $r_s$ is the dimensionless Wigner-Seitz radius). At still lower densities, we find a (possibly metastable) 
valley and spin-polarized state with a reduced electronic anisotropy.
\end{abstract}


\date{\today}
\author{Agnes Valenti}
\affiliation{Institute for Theoretical Physics, ETH Zurich, CH-8093, Switzerland}
\author{Vladimir Calvera}
\affiliation{Department of Physics, Stanford University, Stanford, CA 94305, USA}
\author{Steven A. Kivelson}
\affiliation{Department of Physics, Stanford University, Stanford, CA 94305, USA}
\author{Erez Berg}
\affiliation{Department of Condensed Matter Physics, Weizmann Institute of Science, Rehovot 76001, Israel}
\author{Sebastian D. Huber}
\affiliation{Institute for Theoretical Physics, ETH Zurich, CH-8093, Switzerland}



\title{Nematic metal in a multi-valley electron gas: \\ Variational Monte Carlo analysis and application to AlAs}

\maketitle
%TC:endignore


The two-dimensional electron gas (2DEG) serves as a starting point for the description of a plethora of intriguing quantum phases \cite{ando1982electronic}. 
Among other things, it exhibits a transition from a fluid phase (or phases) at high density, $n$, to an insulating Wigner crystal phase at low $n$ where the Coulomb repulsion dominates over the kinetic energy.  In order to  describe this transition, methods which faithfully capture fluctuations beyond mean-field approaches are required. Quantum Monte Carlo methods such as variational and diffusion Monte Carlo have emerged as highly successful in the treatment of the 2DEG \cite{tanatar1989ground, kwon1993effects, rapisarda1996diffusion, attaccalite2002correlation, gori2003two, drummond2009phase, marchi2009correlation}. Unfortunately, the simplicity of the uniform isotropic electron gas is also reflected in its phase diagram: While an ongoing debate does not preclude the existence of subtle forms of order \cite{spivak2004phases, jamei2005universal, raghu2011superconductivity, kim2022interstitial}, there is no conclusive evidence for a {\em metallic, symmetry-broken state} stabilized by interactions according to quantum Monte Carlo calculations so far \cite{ drummond2009phase, loos2016uniform, marchi2009correlation}. 
In the present manuscript we show how a minimal extension of the two-dimensional electron gas can lead to an intermediate symmetry-broken {\em metallic} state.

With the advent of two-dimensional van der Waals materials~\cite{zhou2021superconductivity, seiler2022quantum, cao2018unconventional, chen2019signatures, park2021tunable, bistritzer2011moire, cao2018correlated, shen2020correlated, zhao2023gate, mak2022semiconductor}, the importance of multi-valley physics including several anisotropic Fermi-pockets has %become apparent.
come into renewed focus.  However, in most of these systems, such as graphene- or transition-metal-dichalcogenide multilayers, the complexity of the observed physics poses a challenge to accurate theoretical modelling. Recent QMC studies \cite{hofmann2022fermionic, Yang2023Metal} on 2D van der Waals materials showcase the potential of accurate treatments of the interactions in strongly correlated systems. Here, we aim at an in-depth study of a simpler model, which however shares a key feature with the aforementioned layered materials: the existence of a valley degree of freedom. Moreover, to account for more intricate effects arising from a non-trivial Fermi surface, we consider a model with two {\em anisotropic} pockets, related by $90^\circ$ rotation. 

Beyond its theoretical appeal, 
our model provides a good description of the 2DEG in ultra clean AlAs quantum wells~\cite{shayegan2006two}. 
 Recent experiments~\cite{hossain2020observation, hossain2021spontaneous, hossain2022anisotropic}  
have reported the existence of at least three distinct phases that appear in this system upon decreasing $n$: i) The symmetric Fermi liquid gives way to a {\em valley-polarized} phase appearing around the Wigner-Seitz radius $r_s\propto 1/\sqrt{n}\approx 20$ \cite{footnote1, hossain2021spontaneous}; ii) an insulating phase which is spin unpolarized is stabilized for $27 < r_s < 35$ and iii) An {\em insulating spin- and valley-polarized state} appears at $r_s \sim 35$ \cite{hossain2020observation, hossain2022anisotropic}. 


As we shall elaborate below, Hartree-Fock mean-field theory cannot explain these findings, even at a qualitative level. 
The experimental observations are further in apparent disagreement \cite{Ahn2022} with the quantum Monte Carlo results of the 2D {\em isotropic} electron gas that do not predict any polarized fluid phase, even in its multi-valley variant \cite{marchi2009correlation}. On the contrary, introducing multi-valley components alone has the rather opposite effect of further stabilizing the paramagnetic fluid.
These observations call for a quantum Monte Carlo study taking into account both the multi-valley nature of the system {\em and} the anisotropy of the Fermi surface.


We study the 2D multi-valley anisotropic electron gas beyond mean-field by performing variational Monte Carlo (VMC) calculations using a Slater-Jastrow-Backflow trial wave-function. We focus on densities larger than the expected transition to a Wigner crystal and therefore explore different fluid phases. We find that valley-polarization emerges as a result of both multi-valley physics and anisotropy and thereby resolve the apparent contrast to the isotropic 2D electron gas.
For the special case of AlAs, we are able to approximately match the relevant density regime at which valley-polarization is experimentally observed.
We  further predict 
a correlation-induced renormalization of the Fermi surface, which results in a reduced anisotropy relative to the band value. 

%\parspace

 %model

% Figure environment removed



{\it Multi-valley anisotropic electron gas.}
We consider a system with two conduction band minima that lie at the $X$-points of the Brillouin zone. For small electron densities, the conduction band electrons near the minima can be treated as two separate flavours, or isospins, of a 2D electron gas. We include anisotropy by allowing for an elliptic shape of the valleys, with distinct longitudinal and transverse effective masses. Concretely, we consider the Hamiltonian
\begin{align}
\label{eq:H}
H=-\sum \limits_{i} \frac{1}{2m^{*}}\big(\eta^{\tau_i/2} \partial_{i,x}^2+ \eta^{-\tau_i/2} \partial_{i,y}^2 \big) + \sum \limits_{i<j} V(|{\bf r}_i- {\bf r}_j|),
\end{align}
where $i,j$ run over all electrons, $\tau_i \in \{+1, -1 \}$  denotes the valley flavor (i.e. the isospin) of electron $i$ \cite{SI}, and the effective mass is given by $m^{*}$. The parameter $\eta$ controls the anisotropy of the system: $\eta\neq 1$ corresponds to an elliptical shape of the Fermi surface, as schematically depicted in Fig.~\ref{fig:fig1}. 
%Do we want to explain/justify the choice?
For computational simplicity, we consider a Coulomb interaction that is screened by two symmetric metal gates 
\begin{align}
V(|{\bf r}_i-{\bf r}_j|)&=\frac{1}{(2\pi)^2}\int {\rm d} {\bf q}\, {\rm e}^{i{\bf q} \cdot ({\bf r}_i-{\bf r}_j)}v({\bf q}), \\
v({\bf q})&= \frac{e^2}{2 \epsilon_0 \epsilon} \frac{\tanh (d |{\bf q}|)}{|{\bf q}|}.
\label{eq:V}
\end{align}
Here, the distance between the gates is $2d$ and $\epsilon$ %corresponds to the dielectric screening 
is the dielectric constant of the system. The above dual-gate screened interaction decreases exponentially in real-space for long distances as $V(r)\sim{\rm e}^{-r/2d}$ \cite{throckmorton2012fermions}.

{\it Variational Monte Carlo.} %method
We capture correlations beyond mean-field theory by performing VMC calcuations utilizing a trial wave function of Slater-Jastrow-Backflow \cite{jastrow1955many, lee1981green, kwon1993effects, kwon1998effects} type
\begin{align}
\Psi({\bf r}_1, ...,{\bf r}_N)= e^{-J({\bf r}_1, ...{\bf r}_N)}\Psi_D ({\bf r}_1, ...,{\bf r}_N). \label{eq:J_PsiD}
\end{align}
The trial wave-function is separated into the Jastrow pre-factor $e^{-J}$ and an antisymmetric determinantal part $\Psi_D$. Both parts contain parameters that are optimized using stochastic reconfiguration \cite{sorella1998green, sorella2007weak} to minimize 
\begin{align}
    E_{\ssm var}=\frac{\langle \Psi | H | \Psi \rangle}{\langle \Psi |\Psi \rangle}.
\end{align}
Expectation values of interest can then be computed using Monte Carlo sampling of a finite number of electron positions ${\bf r}_i$.


The determinantal part $\Psi_D$ is of the form
\begin{align}
    \Psi_D ({\bf r}_1, ...,{\bf r}_N)&=\prod \limits_{\alpha=\{\tau, \sigma\}} {\rm Det} \big( {\bf M}^{\alpha} \big), \\
    \big({\bf M}^{\alpha}\big)_{ij}&=\varphi_j({\bf r}_i)\propto e^{i {\bf k}_j \cdot {\bf r}_i}.
\end{align}
Given that we consider fluid phases, where translation symmetry fixes the form of the orbitals to be plane waves, the variational degrees of freedom reduce to the choice of which orbitals are filled. 
We choose the Slater determinant to be a filled elliptical Fermi sea \cite{ahn2021anisotropic} whose ratio between semi-axes $\tilde{\eta}_{\ssm{FS}}$ is treated as a variational parameter. We also allow for a symmetry-breaking imbalance of the filling of the isospin  and spin %flavors 
$\tau$ and $\sigma$. 


We include correlation effects into our ansatz in two ways. First, we choose a Jastrow factor \cite{jastrow1955many} $J({\bf r}_1,\dots,{\bf r}_N)$ which includes two-body correlations. The Jastrow factor is real and positive and its role is thus to refine the many-body wave-function by effectively keeping electrons apart through the creation of a {\em correlation hole}. We capitalize on the anisotropy of the system in the paramterization of $J({\bf r}_i-{\bf r}_j)$ \cite{whitehead2016jastrow, kim2018qmcpack}, as detailed in \cite{SI}.

A second improvement over a Hartree-Fock type wave-function is achieved through the introduction of a backflow transformation \cite{lee1981green, kwon1993effects, kwon1998effects}: Instead of evaluating the determinantal part of the wave-function $\Psi_D$ at the bare electron coordinates ${\bf r}_i$, the determinant is evaluated at
\begin{equation}
    %{\bf r}_i \to 
    \tilde{\bf r}_i={\bf r}_i+\sum_{j}{\boldsymbol \xi}_{\tau_i,\tau_j}({\bf r}_i-{\bf r}_j).
\end{equation}
The functional form of ${\boldsymbol \xi}_{\tau_i,\tau_j}({\bf r}_i-{\bf r}_j)$ is another optimizable degree of freedom which mostly affects the {\em nodal structure} of the wave-function. We make use of the short-ranged parametrization implemented in \cite{kim2018qmcpack}.



% Figure environment removed

Our VMC simulations include up to $N=176$ electrons, and are carried out using the package {\em qmcpack} \cite{kim2018qmcpack} with appropriate modifications. 
 There are two main sources of finite size errors: First, a discretization of the momenta ${\bf k}_j$ leads to discrete steps in the {\em kinetic energy} when the filling of the orbitals is determined. This can be alleviated via the introduction of twisted boundary conditions: We make use of the special twist method of Ref~\cite{dagrada2016exact}. Second, finite-size corrections to the potential pertain to how the {\em total energy} scales with the number of electrons. For the present case of screened interactions we resort to a phenomenological extrapolation to the thermodynamic limit, as detailed in \cite{SI}.

\parspace

{\it Phase diagram.} The VMC phase diagram of the metallic phases of Hamiltonian (\ref{eq:H}) as a function of $r_s$ and anisotropy $\eta$ is shown in Fig.~\ref{fig:fig1}. 
We simulate the regime $r_s \le 31$, but note that the AlAs experiment already observes a transition to an insulating phase at $r_s \approx 27$~\cite{hossain2022anisotropic}.
We find three stable phases of different isospin polarization for the considered range of densities and anisotropies. The unpolarized symmetric state is stable at large densities, whereas at very low denisities, the ground state is fully spin and valley polarized (SVP).  
The key feature, however, lies in the emergence of an intermediate, valley-polarized and spin-unpolarized  (VP) phase for $\eta  \gtrsim 3.5$. The density range where the VP phase is %stabilized 
stable further increases with the anisotropy of the system: For the largest simulated anisotropy ($\eta=12$), the VP state spans the whole range of simulated densities.


Simulations were performed for electron fluids with different isospin polarization on a discrete grid in the $r_s-\eta$ plane and the phase boundaries estimated by fitting the energies on horizontal and vertical lines \cite{SI}. The difference of the variational energies with respect to the symmetric state are displayed in Fig.~\ref{fig:fig2} for $\eta\approx 5.2$ and $r_{\ssm s}\approx 15.4$, respectively. We observe that of all the simulated polarization patterns, only the VP and the SVP states are ever the lowest energy states. Both a $3/4$-polarized state (where three of the four isospin-combinations are filled, with the same density for each of the isospin-combinations) and the spin-polarized state (SP) where both valleys are filled symmetrically are always higher in energy.  We have not calculated the energies of partially polarized phases, other than the $3/4$ state.
We shall comment further on the stability of the SVP phase below.



% Figure environment removed


Figure~\ref{fig:fig2} further demonstrates the growth of the energy difference between states with and without valley polarization as a function of increasing anisotropy.
Intuitively, this behaviour can  be understood as resulting from a correlation effect that penalizes states without full valley-polarization: Electrons from different valleys have different dispersions, which may hinder efficiently avoiding each other. The non-trivial nature of this effect becomes particularly apparent in comparison to mean-field theory. 

{\it Comparison to Hartree-Fock theory.} Within mean-field theory correlation effects cannot be captured, resulting in an accidental degeneracy of spin-and valley polarization (see Fig.~\ref{fig:fig3}) \cite{SI}.
Hartree-Fock theory faces a second shortcoming in the considered strongly interacting density regime. In particular, the phase diagram is determined by subtle correlation effects outside of the reach of mean-field theory. An exemplary case is illustrated in Fig.~\ref{fig:fig3} for $\eta=5.2$, $r_s=15.4$: While VMC results indicate the prevalence of a VP ground state, Hartree-Fock calculations \cite{SI} predict a SVP ground state, showing the well-known tendency of Hartree-Fock to over-favor symmetry-broken states. In addition, the energy differences between the states of different isospin polarization are two orders of magnitude smaller than within Hartree-Fock, demonstrating the relevance of subtle correlation effects.  %{\color{red} We might want to say that this shows the well known tendency of HF to over-favor broken symmetry states.}

Figure~\ref{fig:fig3} further shows that the energy gain resulting from introducing a Jastrow factor (correlation hole) is strongly isospin-polarization-dependent. In particular, the gain is smallest for the SVP state. The same observation can be made for the energy gain resulting from an optimizable nodal structure by introducing Backflow transformations; the gain is smallest for the SVP wave-function.
We can understand both observations by considering the nodal structure of the trial wave-functions, i.e. the positions in real space at which $\Psi_T({\bf r}_1, .. {\bf r}_N)=0$.
The nodal structure is solely determined by the determinantal part. Comparing the nodes of a bare Slater determinant (corresponding to a Hartree-Fock ansatz) and a Slater determinant with optimized Backflow transformations is a useful step to understand the validity of a mean-field approach.

The results are depicted in Fig.~\ref{fig:fig3} for the VP as well as the SVP state, showing the nodal structure of a two-dimensional ``slice'': It is obtained by fixing the position of all electrons but one and moving the latter through the simulation cell. As illustrated, Backflow transformations induce a much larger change in the nodal structure between different isospin flavours, as electrons with the same isospin are already kept apart by Pauli exclusion \cite{drummond2009phase, marchi2009correlation}. A similar reasoning might also be responsible for the polarization-dependent effect of the Jastrow factor.

These considerations are of particular relevance in connection to the stability of the SVP phase.
Concretely, one can e.g. obtain a more accurate results by including 3-body terms into the Jastrow factor \cite{kwon1993effects} or by performing diffusion Monte Carlo calculations \cite{foulkes2001quantum, varsano2001spin, drummond2009phase}. Following the above reasoning, when employing one of these options we expect a more enhanced effect on the energy of states with more than one isospin flavour. Consequently, the transition to the SVP state is likely to shift to larger $r_s$ upon using more accurate methods \cite{varsano2001spin}. The results on the two-dimensional {\it isotropic} electron gas \cite{rapisarda1996diffusion, drummond2009phase, marchi2009correlation} suggest that the transition may even be shifted to a density regime where Wigner crystallization becomes dominant. Whether a SVP phase exists for the considered model before entering the Wigner crystal is currently unclear.





% Figure environment removed

{\it Effective anisotropy}. 
The effective Fermi surface anisotropy $\tilde{\eta}_{\ssm FS}$ of the ground state has physical implications, e.g. on transport properties. We determine the effective anisotropy by scanning through a range $1\leq \tilde{\eta}_{\ssm FS} \leq \eta$ and picking the solution with lowest energy. Here, $\eta$ is the bare anisotropy of the system. Figure~\ref{fig:fig4} depicts the so-obtained variational energies as a function of $\tilde{\eta}_{\ssm FS}$ for different isospin polarizations. The position of the minimum can be understood as the effective anisotropy of the Fermi surface within VMC. We can compare this result with the renormalized anisotropy $\tilde{\eta}_{\ssm HF}$ obtained in Hartree-Fock calculations, also shown in Fig.~\ref{fig:fig4}. We observe that Hartree-Fock strongly underestimates the effective anisotropy for all polarizations except the SVP state: As detailed above, correlation effects beyond Hartree-Fock are expected to play a more dominant role for states with different isospin flavours, which explains the polarization-dependent discrepancy on the effective anisotropy obtained via Hartree-Fock and VMC.



In order to explore the interplay between anisotropy and correlation effects, we turn to a more concrete study of the correlation hole and the effective anisotropy $\tilde{\eta}_{\ssm C}$ thereof. 
Our objective here is to compare the correlation hole between two electrons of different (iso)spin flavour. The underlying motivation lies in acquiring intuition about the emergence of the energy gap between valley-polarized and valley-unpolarized states with increasing ansiotropy.
Direct access to the correlation hole is given in form of the two-body Jastrow factor. We obtain the effective ansiotropy $\tilde{\eta}_{\ssm C}$ by fitting an ellipse to the equipotential lines of the Jastrow factor and taking the mean over the three largest system sizes. 
The results are depicted in Fig.~\ref{fig:fig4} and illuminate the comparative energy gain that valley-polarization induces in the presence of anisotropy. While two electrons within the same valley are able to optimally avoid each other by forming an anisotropic correlation hole, the correlation hole in between valleys is isotropic: Intuitively one can deduce, that not being able to capitalize on the anisotropy of the system in the formation of a correlation hole will introduce an energy offset for valley-unpolarized states. 



{\em Comparison to experiment.} At the anisotropy relevant for AlAs ($\eta\approx 5.2$) \cite{hossain2021spontaneous}, our VMC calculations predict a transition from a symmetric state to a stable valley-polarized phase in the relevant density regime, which is in agreement with the experimental observations. 
There is, however, a quantitative mismatch between the critical $r_s$ obtained numerically ($r_s \approx 12$) and in the experiment ($r_s \approx 20$). In order to understand the root of this mismatch, we turn to a more detailed discussion of the Hamiltonian~(\ref{eq:H}) and neglected terms therein. 

Concretely, we used symmetric gates at a distance of $d=100$ nm, whereas in the experiment there is only a single gate at a much larger distance. However, simulations at selected values of $d$ in the range $70$ nm $\leq d \leq 300$ nm did not indicate a significant shift of the phase boundaries. 

In addition, we have neglected several effects, including electron-phonon coupling, inter-valley scattering terms, and the finite thickness of the quantum well. In \cite{SI}, we estimate the typical electron-phonon coupling and the inter-valley scattering terms, and find that they are both small compared to the energy differences between states of different spin and valley polarization (Fig.~\ref{fig:fig2}). 
Potentially more significant is the thickness of the quantum well, which is $w\approx 20\,\rm{nm}$ in Ref.~\cite{hossain2021spontaneous}; this modifies the Coulomb interaction from $v(|\mathbf{r}|)\sim 1/r$ at $r\gg w$ to $\ln(w/r)$ at $r\ll w$. Taking this effect into account is expected to shift the transition between the symmetric and the valley-polarized states to larger $r_s$, closer to the experimental value (since increasing $w$ weakens the Coulomb interaction at short distances). 


Another possible explanation of the apparent quantitative mismatch in the critical $r_s$ is an uncertainty in the anisotropy of AlAs: In particular, a smaller anisotropy than $\eta\approx 5.2$ results in a larger critical $r_s$.  

At even lower densities ($r_s \approx 27 $), we find a transition to a SVP metallic state. The existence and stability of this phase is, however, as detailed above, unclear.
Experimentally a preference for spin-valley polarization has been found at $r_s \approx 35 $, but the observed phase is insulating rather than metallic. For more conclusive statements about this phase, numerical consideration of insulating states would be required.

In addition, we note that according to Ref.~\cite{shchepetilnikov2020spin} the influence of spin-orbit coupling might be non-neglibible in AlAs. We leave its consideration to future studies.


{\it Perspectives} 
We explored the low-density phase diagram of an anisotropic multi-valley electron gas with VMC using a Slater-Jastrow-Backflow wave-function.
In contrast to the isotropic case, we find that there is a strong evidence for the existence of a symmetry-broken {\em metallic} phase. In particular, as the anisotropy increases, a stable valley-polarized, spin-unpolarized phase emerges. This result is in qualitative agreement with recent experiments on AlAs, where a spontaneous transition to a nematic valley-polarized state is observed at $r_s\gtrsim 20$ \cite{hossain2021spontaneous}.
While we find this transition at higher densities ($r_s\gtrsim 12$), we identify the finite thickness of the well (neglected in our calculation) as the most probable source of this difference. 

Here, we focused on spin or valley polarized, metallic phases: Study of other phases such as inter-valley coherent order \cite{chatterjee2022inter} as well as crystalline phases \cite{wigner1934interaction, hossain2022anisotropic, calvera2022pseudo} pose an interesting avenue for future research.
In addition, more accurate methods such as diffusion Monte Carlo \cite{foulkes2001quantum}, auxiliary-field quantum Monte Carlo \cite{zhang2003quantum, zhang201315} or other classes of wave-functions \cite{pescia2023message, cassella2023discovering, wilson2022wave, li2022ab} may reveal a quantitative change in behaviour and yield further physical insight.
Comparison to mean-field calculations reveal the relevance of correlation effect and call of strongly correlated methods as the one presented here.


\vspace{0.5cm}

\begin{acknowledgments}
We thank Sankar Das Sarma and Mansour Shayegan for their comments on this manuscript. 
This work has received funding from the
European Research Council under grant agreement no. 771503. S.D.H. acknowledges support by the Benoziyo Endowment Fund for the Advancement of Science. S.A.K and E.B. were supported by NSF-BSF award DMR-2000987. E.B. acknowledges support from the European Research Council (ERC) under grant HQMAT (grant agreement No. 817799). E.B. thanks the hospitality of the Aspen Center for Physics, supported by National Science Foundation grant PHY-2210452, where part of this work was done.
\end{acknowledgments}


\bibliographystyle{unsrt}
\bibliography{papers.bib}



\end{document}
