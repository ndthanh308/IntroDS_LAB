\documentclass[aps,nofootinbib,showpacs,twocolumn]{revtex4-1}
\usepackage{graphicx}
\usepackage{dcolumn}
\usepackage{amsmath}
\usepackage{latexsym}
\usepackage{mathrsfs}
\usepackage[usenames,dvipsnames]{color}
\def\clr{\color{red}}
\def\clb{\color{blue}}
\def\clm{\color{magenta}}
\usepackage{hyperref}
\urlstyle{same}

\newcommand{\be}{\begin{equation}} 
\newcommand{\ee}{\end{equation}} 
\newcommand{\bea}{\begin{eqnarray}} 
\newcommand{\eea}{\end{eqnarray}} 
\newcommand{\vdt}{\Delta T}
\begin{document}
\title{A unified picture of continuous variation of critical exponents}

\author{Indranil Mukherjee}
\email{im20rs148@iiserkol.ac.in}
\author{P. K. Mohanty}
\email{pkmohanty@iiserkol.ac.in}
\affiliation {Department of Physical Sciences, Indian Institute of Science Education and Research Kolkata, Mohanpur, 741246 India.}


%\title{ Anamolous fluctuations in nonequilibrium steady states}
\begin{abstract}
Renormalization group (RG) theory allows continuous variation of critical exponents along a marginal direction (if any) but does not predict the functional form. Recently it was proposed that continuous variation must occur in a way that leaves scaling relations invariant. For magnetic phase transition, the variation can occur in three different ways: I. $\gamma$ varies keeping $\delta$ fixed, II. $\delta$ varies keeping $\gamma$ fixed and III. both $\delta$ and $\gamma$ vary. In this article, we study the isotropic Ashkin Teller model on a two-dimensional square lattice and show that the magnetic and electric phase transition along the self-dual line exhibit continuous variation of critical exponents which are of Type-I (formally known as weak universality) and Type-III respectively. We show that the scaling functions of both electric and magnetic phase transition are only a scaled version of the universal scaling function  of  Ising  universality class in two dimensions.
\end{abstract}\maketitle
\section{Introduction}
Phase transition and critical phenomena \cite{Baxter, Stanley_1971, Hu_2014, Zhu_2020} have been emergent topics of research for several decades. Criticality is associated with two basic features, (a)  universality \cite{Griffiths_1970, Stanley_1999}, which states that the associated critical exponents and scaling functions are universal up to symmetries and space dimensionality, and (b) scaling theory \cite{Kadanoff_wsp}, that describes the general properties of the scaling functions and relates different critical exponents. The divergence of correlation length at the critical point of a second-order phase transition ensures that  microscopic details of the system have no roles to play - which explains both   the  scale invariance and  the universal behavior observed there. However, several experimental systems \cite{Guggenheim, Back, Suzuki} appears to violate this universality hypothesis and exhibit continuous variation of critical exponents w.r.t the system parameters. A clear example  is the eight-vertex (8V) model,  solved exactly by Baxter \cite{Baxter,Baxter2, Baxter3},  where   the critical exponents of the  Ferromagnetic  transition  $\beta$, $\gamma$, $\nu$, and  $\alpha$ change continuously but their ratios $\beta / \nu$, $\gamma / \nu$, $(2-\alpha)/\nu$ remain invariant. Later, Suzuki \cite{Suzuki} proposed an explanation:  the  critical exponents should rather be measured w.r.t. the correlation length, an emergent length scale of the system,   instead of the distance from the critical point which is an {\it external} tuning  parameter. This proposal, formally known as the weak universality scenario, explains several experimental features where the continuous variation of exponents is similar to those obtained in 8V model. Now we know that weak universality appears in interacting dimers \cite{Alet}, frustrated spin systems \cite{Queiroz, Jin_Sen}, magnetic hard squares \cite{Pearce}, Blume-Capel models \cite{Malakis}, quantum critical points \cite{Suzuki_Harada}, models of percolation \cite{Andrade_Herrmann, Sahara}, reaction-diffusion systems \cite{Newman}, absorbing phase transition \cite{Noh_Park}, fractal structures \cite{Monceau} etc. 

Most systems that exhibit continuous variation of critical exponents obey weak universality - there are a few exceptions though \cite{Bernardi-Campbell, Corti, Fisher, Bekhechi, Kondo}. Violation of universality and weak universality is observed in Ising spin glasses \cite{Bernardi-Campbell}, frustrated spin systems \cite{Bekhechi} micellar solutions \cite{Corti, Fisher}, strong coupling QED \cite{Kondo} and magnetic phase transitions in chemically doped materials \cite{Butch, Fuchs, pkm_pmandal}. A continuous variation of critical exponents is not ruled out by the renormalization group (RG) theory \cite{Wilson}; the critical points are fixed points of RG flow governed by a unique set of relevant operators. The critical exponents are related to the scaling dimensions of the relevant operators - which are universal in the sense that they are unaltered when irrelevant operators are added to the theory. On the other hand, if added operators are relevant, the system flows to a new fixed point. Continuous variation of critical exponents can occur in a system  if  there are  marginal operators in the theory; these   marginal operators produce a line of RG fixed points. The RG theory, however,  does not specify   how  the  critical exponents  must vary  along  the marginal direction.   

To explain the unusual variation of exponents recently Khan et. al. \cite{pkm_pmandal} have proposed a super  universality hypothesis (SUH): the continuous variation of critical exponents w.r.t. the marginal parameter of the system  must  occur in a way that the scaling relations remain invariant.  For magnetic phase transitions, this hypothesis naturally leads to three different kinds of continuous variations of the parent universality class, Type-I: when $\delta$ is fixed but $\gamma$ changes, Type-II,  when $\gamma$ is fixed but $\delta$ changes and Type-III when both $\delta$ and $\gamma$ vary. It turns out that  Type-I  scenario is the well-known weak universality, and Type-II variation  has been  observed in strong coupling QED \cite{Kondo} and  magnetic systems  which  are coupled to strain-fields \cite{Puri}.  Type-III variation  is observed in   a  Nd-doped  single crystal  (Sm$_{1-y}$Nd$_y$)$_{0.52}$Sr$_{0.48}$MnO$_3$  with $y$ in the range $(0.5, 1)$ \cite{pkm_pmandal}.  To the best of  our knowledge,  a theoretical model  of  phase transition  where  all critical  exponents vary continuously is  not  known in the literature. 
%Since magnetic phase transitions in composite materials are often masked by the existence of disorder, structural transitions, crystal defects, and effective long-range interactions, it is desirable that, like scenarios (a) and (b) we have a theoretical example of scenario (c). 
In this work, we propose that the electric phase transition in the Ashkin Teller (AT) model is indeed an example of  a Type-III variation of SUH.

First, we revisit the AT model and calculate both the magnetic and electric critical exponents from Monte Carlo simulations. We show that the magnetic exponents are consistent with the weak universality scenario, whereas, the electric exponents follow the most generic form of continuous variation, i.e.  Type-III variation of SUH. We also determine explicitly the functional form that describes this  variation.

The paper is organized as follows. For completeness, in Sec. II, we define AT model on the square lattice and report on some exact results, including the equation of the critical line on the phase plane and the critical exponents.   We then propose a generalized universality hypothesis, termed 'super universality hypothesis (SUH)', and show that magnetic phase transition in AT model is the Type-I scenario of  SUH. We provide examples from the literature that exhibit  the Type-II variation.   Electric phase transition in AT model is an example of the most generic variation, namely Type-III,  but we wait until section III  to discuss this in detail.  
In section III, we describe the results obtained from Monte Carlo simulations.  We first study the magnetic critical exponents and their dependencies on the model parameters and show that they are consistent with Baxter's exact results \cite{Baxter, Baxter2} and weak universality (Type-I variation SUH). Numerical results for the electric counterparts are presented in Sec. III B. It shows that the electric critical exponents  AT models vary in the most generic way, where the exponent ratios too vary with the model parameter. In Section III C. we briefly recapitulate the generalized universality hypothesis and show that variation of electric exponents follows it; we also obtain the functional form that characterizes the continuous variation.  We further show that 
 the scaling functions of magnetic and electric phase transition are only a scaled form of the universal scaling function of the Ising model in two dimensions (2D).  Finally, we summarise the results in  Sec. IV.

\section{Exact results and phase transition of Ashkin Teller model}

The  Ashkin Teller (AT) model \cite{AT_1943, Kadanoff_1977, Zisook} is a two-layer Ising system with a marginal four-body interaction between the layers \cite{Fan_Wu}. 
The model  on a square lattice  can be mapped  exactly \cite{Kadanoff_1971} to   the  well-known  eight-vertex (8V) model \cite{Baxter}, where each site of a square lattice is allowed to have non-zero stationary weights for only eight out of sixteen different possible vertices.  AT model naturally lead to two different kinds of order parameters - namely magnetic and electric ones. The usual ferromagnetic phase transition clearly belongs to a weak universality scenario whereas the nature of the critical behavior of the electric phase transition  has been  debated. Recently Kr\ifmmode \check{c}\else \v{c}\fi{}m\'ar et. al. \cite{Roman_Ladislav} have proposed that the electric-phase transition in a symmetric 8V model (and thus the  electric transition  in AT model) is {\it fully non-universal}.  
%In the following, we  show that this transition belongs to the class (c) scenario of the generalized universality hypothesis.


Ashkin-Teller model is defined on a $L\times L$ square lattice with periodic boundary conditions in both directions. Each site ${\bf i}$ of the lattice carry two different Ising spin variables $\sigma_{\bf i}=\pm$ and $\tau_{\bf i}=\pm,$ the neighbouring spins interact following a Hamiltonian, 
\begin{equation} \label{eq:AT_H}
 H = -J_{\sigma}\sum_{\langle {\bf ij}\rangle} \sigma_{\bf i} \sigma_{\bf j} - J_{\tau} \sum_{\langle {\bf ij}\rangle}\tau_{\bf i} \tau_{\bf j} - \lambda \sum_{\langle {\bf ij}\rangle}\sigma_{\bf i} \sigma_{\bf j} \tau_{\bf i} \tau_{\bf j}. 
\end{equation}
Here ${\bf j}$ is the nearest neighboring site of ${\bf i}$ and $\langle {\bf ij}\rangle$ denotes a pair of nearest-neighbor sites. Here $J_{\sigma}, J_{\tau} >0$ are the strengths of the intra-spin Ferro-magnetic interactions of $\sigma$ and $\tau$ neighboring spins and $\lambda$ represents interactions among them. We  consider only the isotropic case $J_{\sigma} = J_{\tau}=J$ where exact results are available from mapping of the model to 8V model \cite{Baxter}.

The Hamiltonian \eqref{eq:AT_H} is invariant under any of the following transformations: $\sigma \rightarrow -\sigma,$ $\tau \rightarrow -\tau,$ or $\sigma \rightarrow \tau.$ Thus one can treat  either of  $\langle\sigma\rangle= \langle\tau\rangle,$ or $\langle\sigma\tau\rangle$ as the order parameters  of the system; the  first one  characterizes   Ferro to paramagnetic transition where $\langle\sigma\rangle= \langle\tau\rangle \equiv M$ takes a nonzero value  whereas   $\langle\sigma\tau\rangle\equiv P$ (formally known as the polarization) becomes nonzero  during electric phase transition.  
 Unlike the magnetic phase transitions, the electric transitions are less studied in this model \cite{Roman_Ladislav}. A particular question is, whether this transition obeys the universality hypothesis.

 \subsection{The Phase Diagram}
The phase diagram of the AT model  on a square lattice  is known exactly from the  duality transformations, and  from  renormalization-group studies \cite{Wu_Lin,Domany_Riedel}. The phase diagram of the system    at temperature $T=1$ and $J\ge0$ is  shown in  Fig. \ref{fig:phasediag}   where   $\lambda$ is defined   by  the duality relation $\sinh (2J) = e^{-2\lambda}.$ 
% Figure environment removed

\noindent{\bf $\lambda = 0$:} For $\lambda = 0$, $\sigma$ and $\tau$ spins are decoupled and Eq. \eqref{eq:AT_H} reduces  to  two independent Ising systems on a square lattice. Thus, the critical point  is $J=J_{c}=\frac{1}{2}\ln(1+\sqrt{2})$ (marked as $Z_{2}$ in Fig. \ref{fig:phasediag}). \\

\noindent { \bf $J=0$:} For $J=0$  the model reduces to Ising model with a redefined  Ising-like  spin variable $s_{\bf i} \equiv \sigma_{\bf i} \tau_{\bf i}$ at every  site ${\bf i}$ which  interact with neighboring spins  with interaction strength $\lambda.$    Thus  corresponding magnetization $\sum_{\bf i} s_{\bf i}$ can undergo  a ferromagnetic  transition when $\lambda > \lambda_c = \frac{1}{2}\ln(1+\sqrt{2})$  or an anti-ferromagnetic transition $\lambda <-\lambda_c.$ Note that  $\lambda_c$ is same as $J_c$\\

\noindent{ \bf $\lambda=J$:} For $\lambda=J$ AT model has a $Z_4$ 
symmetry as Hamiltonian \eqref{eq:AT_H} with $J_\sigma =J_\tau =\lambda$ is invariant under the permutations of the four states ($\{\sigma=\pm\}$, $\{\tau=\pm\}$). Thus in this case we have $q=4$ Potts model with the critical point located at $J_p=\lambda=\frac{1}{4}\ln(3)\simeq 0.2746.$ This point is marked as $Z_{4}$ in Fig. \ref{fig:phasediag}.\\

\noindent{\bf $\lambda \rightarrow \infty$:} When $\lambda$ is very large, the terms of $\sigma_{\bf i} \sigma_{\bf j}$ and $\tau_{\bf i} \tau_{\bf j}$ in 
Eq. \eqref{eq:AT_H} must take the same value and their product becomes unity. In this limit, the Hamiltonian reduces to a single site Ising model with coupling $2J.$ Corresponding ferromagnetic Ising critical point is $(J,\lambda ) = (J_c /2, + \infty)$.
%which corresponds to four critical lines. These lines act as the boundaries of four different phases in the phase space. 

AT  model  has  four different phases. Phase I (paramagnetic and electrically disordered): a paramagnetic phase where the couplings are sufficiently weak and none of $M$ and $P$ are ordered., $\langle \sigma \rangle=\langle \tau \rangle=0 = \langle \sigma \tau \rangle.$ Phase II (Ferromagnetic and electrically ordered): the ferromagnetic phase where the couplings are sufficiently strong so that both $M$ and $P$ attain a nonzero value. Phase III and IV (Paramagnetic and  electrically ordered):   partial ferromagnetic ordering is  observed,  where $\langle \sigma \tau \rangle$ is ordered ferromagnetically but $\langle \sigma \rangle=\langle \tau \rangle=0$. Phase IV is similar to phase III except that $\langle \sigma \tau \rangle$ is ordered anti-ferromagnetically. 
%The boundaries between Phase III with I or II and Phase I with IV are not critical.


\subsection{Magnetic and electric critical exponents}
Let us define the critical exponents of the electric and magnetic transition. For the magnetic phase transition, the order parameter is
the statistical average of $M= \sum _{\bf i} \tau_{\bf i}$ $= \sum _{\bf i} \sigma_{\bf i}.$ For electric phase transition, the order parameter is $P= \sum _{\bf i} \phi_{\bf i},$ where $\phi_{\bf i} = \sigma_{\bf i} \tau_{\bf i}.$ When the temperature of the system is close to critical value $T_c,$
\begin{equation} \label{eq:beta_mag}
 \langle M \rangle \sim \Delta T^{\beta_m} ; ~~ \langle P \rangle\sim \Delta T^{\beta_e}
\end{equation}
where $\Delta T =T_{c}-T$ and $\beta_{m,e}$ are the order parameter 
the exponent of magnetic, electric transition. The critical exponents $\gamma_{m,e}$ are associated with the  susceptibilities 
\begin{equation} \label{eq:gamma_mag}
 \chi_m = \langle M^{2} \rangle - \langle M \rangle^{2} \sim \Delta T^{-\gamma_m}; \chi_e = \langle P^{2} \rangle - \langle P \rangle^{2} \sim\Delta T^{-\gamma_e}
\end{equation}

The correlation functions can be defined as $G_m({\bf r}) =\langle \sigma_{\bf i} \sigma_{\bf i+r} \rangle - \langle M \rangle^{2}$ $=\langle \tau_{\bf i} \tau_{\bf i+r} \rangle - \langle M \rangle^{2}$ and $G_e({\bf r}) =\langle \phi_{\bf i} \phi_{\bf i+r} \rangle- \langle P \rangle^{2}.$ Near the critical point $T=T_c$ 
\begin{equation}
 G_{m,e}(r) \sim \frac{1}{r^{\eta_{m,e}}} e^{(-r/\xi)}
\end{equation}
where $\xi,$ the correlation length, being an emergent length scale of the system does not depend on other details. As one approach the critical point $T_c,$ it diverges as, 
\begin{equation}\label{eq:nu}
 \xi \propto (\Delta T)^{-\nu}.
\end{equation}
Like $\nu$ another critical exponent $\alpha,$ associated with the specific heat $C_v$ of the system, 
does not carry subscripts $e,m.$ 
\begin{equation}\label{eq:alpha}
 C_v= \langle E^{2} \rangle - \langle E \rangle^{2} \sim (\Delta T)^{-\alpha}
\end{equation}


Now we look at some other exponents defined at 
the critical point $T=T_{c},$ but with a small applied fields $h$ and $\tilde h$ that couples to $M$ and $P$ respectively, i.e., 
the Hamiltonian is modified as, 
\bea \label{eq:AT_H_h}
 H &=& -J_{\sigma}\sum_{\langle {\bf ij}\rangle} \sigma_{\bf i} \sigma_{\bf j} - J_{\tau} \sum_{\langle {\bf ij}\rangle}\tau_{\bf i} \tau_{\bf j} - \lambda \sum_{\langle {\bf ij}\rangle}\sigma_{\bf i} \sigma_{\bf j} \tau_{\bf i} \tau_{\bf j}\cr
 &-& h \sum_{\bf i}\left(\sigma_{\bf i}+ \tau_{\bf i}\right) - \tilde h\sum_{\bf i}\sigma_{\bf i} \tau_{\bf i} 
\eea
Now the order parameters $M(h), P(\tilde h)$ are expected to increase with the increase of respective fields as, 
\begin{equation}\label{eq:delta_mag}
 M(h) \sim h^{1/\delta_m};~~ \phi(\tilde h) \sim \tilde h^{1/\delta_e}.
\end{equation}
These relations define the critical exponents $\delta_m$ and $\delta _e.$ 

All the critical exponents are not independent; they are related by scaling relations \cite{Baxter}; one of them is, 
\bea \label{eq:scale}
%\beta = \frac{\gamma}{\delta-1}; \eta= \frac4{1+ \delta}; ~~
%\eta=2\frac{\beta}{\nu}= \frac4{1+ \delta}; ~~
 2-\alpha= d \nu = \gamma \frac{\delta+1} {\delta- 1}.
\eea
We expect these scaling relations to be satisfied by the exponents of magnetic and electric transitions. 



The critical exponents of AT model are known from its mapping to 8V model introduced by Baxter \cite{Baxter, Baxter1_1972, Baxter2}, and from re-normalization group arguments \cite{Wu_Lin, Domany_Riedel}. 
\begin{eqnarray}
&&\nu=\frac{2 (\mu-\pi)}{4\mu-3\pi} ~{\rm with} ~ \cos\mu = e^{2\lambda} \sinh(2\lambda); ~ \alpha=2(1-\nu); \cr
&&\beta_e= \frac{2 \nu -1}{4};~
%~ \eta_e= \frac{2 \nu -1}{2 \nu};~
\delta_e=\frac{6 \nu +1}{2 \nu -1}; ~\gamma_e =\frac{1}{2}+\nu
\label{eq:exact_ele_AT}
\\
&&\beta_m=\frac{\nu}{8}; ~
%\eta_m=\frac{1}{4}; ~ 
\delta_m=15; ~\gamma_m =\frac{7\nu}{4} \label{eq:exact_mag_AT}
\end{eqnarray}
Let's discuss the kind of continuous variation of critical exponents that occurs here. For magnetic transition, the ratio of exponents are same as that of the parent universality class (which is the Ising model in $d=2$), i.e., $\nu=1, \alpha_0=0, \beta_0=\frac18,$ $\gamma_0= \frac74,$ $\delta_0 = 15$ and $\eta_0=\frac14,$
\be
\frac{\beta_m}{\nu} = \beta_0,\frac{\gamma_m}{\nu} = \gamma_0, \delta_m = \delta_0, \eta_m=\eta_0.
\ee
This scenario, that $\beta$ and $\gamma$ varies continuously keeping 
$\frac{\beta}{\nu}$ and $\frac{\gamma}{\nu}$ same as their parent value, is observed in several experiments \cite{Guggenheim, Back} and Suzuki \cite{Suzuki} referred to it as a weak universality scenario. 
For electric transition, however, all the critical exponents vary with
 the marginal parameter $\lambda$ and it breaks both the universality and weak universality hypothesis; it is not clear if the exponents have any relation to the parent universal exponents $ \{ \beta_0, \gamma_0, \delta_0, \eta_0\}$. %Recently Krcmar and Samaj \cite{Roman_Ladislav} have proposed that the electric-phase transition of AT model is {\it fully non-universal}. We must emphasize that magnetic phase transition where all the above exponents vary continuously has been observed experimentally \cite{pkm_pmandal}. 
 
 
 In the following, we propose a generic  universality hypothesis, namely SUH,  and show that both kinds of variations observed here are related to the parent universality. This hypothesis, not only allows variation of all the critical exponents it includes weak universality and some 
 other kind of variations \cite{Puri, Kondo} as special cases. 
 
 \subsection{Super universality hypothesis (SUH)} 
 The basic assumption of super universality is that scaling relations which are properties of homogeneous functions,  must be obeyed by the critical exponents even when they vary continuously. 
 Notice that Eq. \ref{eq:scale} is satisfied by both electric and magnetic transitions in AT model. In Eq. (\ref{eq:scale}), since exponent $\alpha$ is related to the energy fluctuations (not related to order parameter) and $\nu$ is associated with a global length scale of the system, and continuous variation must occur in three different ways. 
\bea
&{\rm Type-I.} & \gamma ~{\rm varies ~ keeping} ~ \delta ~{\rm fixed }\\
& {\rm Type-II.} & ~\delta ~ {\rm varies ~ keeping} \gamma ~ {\rm fixed }\\
& {\rm Type-III.} & ~{\rm both} ~ \delta ~ {\rm and} ~ \gamma ~ {\rm vary }
\eea 
Let us consider the most generic variation Type-III. We assume the existence of a marginal parameter $\mu$ that gives rise to a continuous variation of exponents. 
Starting from the parent universality with exponents $\{\beta_0, \gamma_0, \nu_0, \delta_0, \eta_0 \}$ the variation must occur generically as follows: 
%$ \gamma = \frac{\gamma_0}{f(\mu)}$ and $\nu = \frac{\nu_0}{h(\mu)}$ which can also be expressed as, 
\be
\label{eq:var1}
\gamma = \frac{\gamma_0}{f(\mu)}; ~~ \nu = \frac{\nu_0}{g(\mu)}. 
%\gamma = \frac{\gamma_0}{c}; ~~ \nu = \frac{\nu_0}{g(c)}
\ee
%where $c \equiv c(\mu)$ and $g(c) \equiv f( \mu(c)).$ 
Then, the other critical exponents are then determined by using the scaling relations ( \ref{eq:scale}) for $d=2,$ and using the other scaling relations $\beta = \frac{2-\alpha-\gamma}{2}$,  $\eta=2- \frac\gamma\nu,$
\bea\label{eq:Type-III}
{\rm Type-III}:~ &&2-\alpha=\frac{2-\alpha_0}{g(\mu)};
\frac{\delta +1}{\delta -1}=\frac{f(\mu)}{g(\mu)}\frac{\delta_0 +1}{\delta_0 -1}\cr 
&& \eta = \frac{g(\mu)} {f(\mu)} \eta_0 + 2\left( 1- \frac{g(\mu)} {f(\mu)}\right) \cr 
&& \beta = \frac{\beta_0}{g(\mu)} + \frac{\gamma_0}{2g(\mu)}\left( 1 - \frac{g(\mu)} {f(\mu)}\right)
%\beta = \frac12( \frac{\tilde \delta_0}{g(\mu)} - \frac{1}{f(\mu)}); \eta= 
\eea
Type-I and Type-II variations are now special cases, $f(\mu)=g(\mu)
$ and $f(\mu)=1$ respectively. For notational convenience we use $g (\mu) \equiv g,$ then 
\bea \label{eq:Type-I}
 &&{\rm Type-I:} \nu = \frac{\nu_0}{g}; \gamma = \frac{\gamma_0}{g}; \beta=\frac{\beta_0}{g}; \delta= \delta_0, \eta=\eta_0 \\ \label{eq:Type-II}
 &&{\rm Type-II:} \nu = \frac{\nu_0}{g}; \gamma =\gamma_0; \beta=\frac{\beta_0}{g} + \frac{\gamma_0}{2 g}(1-g);\cr
&&~~~~~~~~\delta= \frac{1+\delta_0 + g (\delta_0-1)}{ 1+\delta_0 - g (\delta_0-1)}; \eta = g\eta_0 + 2 (1-g)
\eea

Note that one can have other  types of continuous variations, besides  Type-I, II,  and  III.  For example,  when   $f(\mu)=1$  leads  to a continuous variation of  all  the  critical exponents except the  correlation-length exponent $\nu$. The reason for defining three  specific types of variation is that they have been already observed in some systems. Type-I scenario is well known as the weak universality scenario and it has been observed both theoretically and experimentally. The ferromagnetic phase transition of AT model is also an example of weak universality. Type-II variation has been observed in some other models.  Continuously varying critical exponents in quenched QED violates not only universality but also weak universality \cite{Kondo};  all the critical exponents except $\gamma$ vary continuously with the gauge coupling constant.  This is indeed the Type-II scenario proposed by SUH.   Another example is magnetic phase transition in systems where spins interact with a  long-ranged strain field \cite{Puri}. Here, in the mean-field limit, all the critical exponents except $\gamma$ vary continuously with the parameter that dictates the long-range nature of the strain field. 


We must emphasize that the universality hypothesis is not violated when all exponents vary continuously, there is a super-universality hypothesis given by Eqs. (\ref{eq:var1}) are (\ref{eq:Type-III}). The SUH is consistent with RG theory and scaling theories of critical phenomena. If the continuous variations are  related to the parent universality class then we must observe that the scaling functions of both magnetic and electric transitions of AT model must be the same (up to a multiplicative factor). 
In the following, we revisit AT model, find the magnetic and electric critical exponents, and show that they obey Eqs. (\ref{eq:var1}) are (\ref{eq:var1}) and Eq. \eqref{eq:fg}. We also calculate the scaling functions of electric and magnetic phase transitions for different $\lambda$ and show that up to a scale factor,  they match quite well with the scaling functions of  Ising Universality class (IUC)  in 2D.


\section{Monte Carlo simulation of AT model}

In this section, we calculate both the magnetic and electric critical exponents of AT model using Monte-Carlo simulations. We first set the critical points on the self-dual line as \cite{Wegner}
\be \label{eq:TcJclam}
T_{c} =1, \lambda= \lambda_c, J_c = \frac12 \sinh^{-1}(e^{-2\lambda_c}).
\ee

% Figure environment removed


Now the critical exponents of the system for different $\lambda$ values can be obtained by varying $T.$ We consider $\lambda =$ $-0.2$, $-0.1$, $0$, $0.1$ and $0.2$ and set $\Delta T = T_c-T.$ First we look for correlation exponent $\nu$ and specific-heat-exponent $\alpha$ which are common for both electric and magnetic phase transitions. $\nu$ can be obtained from the finite size scaling of the Binder cumulant \cite{K_binder, K_binder1}, 
\begin{equation}\label{eq:scaling_binder}
 B= 1- \frac {\langle M^4 \rangle} {3\langle M^2\rangle}=g_{B}(\Delta T L^{1/\nu})
\end{equation}
where $g_B(.)$ is the universal scaling function. For each $\lambda,$ variation of $B$ as a function of temperature is obtained for different $L;$ they collapse onto a unique scaling function 
when plotted against $\Delta T L^{1/\nu}.$ Figures \ref{fig:scaling_binder}
(a), (b),(c), (d), (e) shows the data collapse for $\lambda =$ $-0.2$, $-0.1$, $0$, $0.1$, and $0.2$ respectively. The value of $\nu$ that gives rise to the data collapse is listed in Table \ref{tab:mag_cri_1}. To calculate the specific-heat exponent $\alpha$ we look at the $C_v= \langle E^2\rangle - \langle E\rangle^2\sim \Delta T ^{-\alpha};$ at $T=T_c,$ it scales with the system size $L$ as $C_v \sim L^{\frac\alpha\nu}.$ The plot of $C_v$ vs. $1/L$ is shown in Fig. \ref{fig:scaling_binder}(f) and the values obtained for large $L,$ for different $\lambda$ are shown in Table \ref{tab:mag_cri_1}. 


\subsection{Magnetic critical exponents:} Using Monte Carlo simulations of a larger system, $L=2^{10}$, we calculate magnetization $M=\langle \sigma \rangle=\langle \tau \rangle$ and the susceptibility $\chi_m = \langle M^{2} \rangle - \langle M \rangle ^{2}$ for different temperature near the critical value $T_c=1.$ In absence of magnetic field $h,$ $M \sim (\Delta T) ^ {\beta_m}$ and $\chi_m \sim (\Delta T) ^ {\gamma_m}.$ Also at $T= T_c=1,$ $M \sim h ^ {\frac{1}{\delta_m}}.$ Figure  \ref{fig:exponents_mag} (a) and (b) shows variation of $M$ and $\chi_m$ with $\Delta T$ and (c) there shows how $M$ varies as $h$ at $T=T_c.$
In Fig. \ref{fig:exponents_mag} (d) we plot of $M$ as a function of $1/L$ at $T=T_c$ and $h=0,$ to obtain the exponent ration $\frac{\beta_m}{\nu}$ for different 
$\lambda.$ Our best estimate of all these magnetic critical exponents, $\beta_m, \gamma_m, \delta_m, \beta_m/\nu$ obtained from simulations are listed in Table. \ref{tab:mag_cri_1}. 


% Figure environment removed
\begin{table}[h]
\caption{\label{tab:mag_cri_1}
Results of magnetic critical exponents along the AT critical line varying the model parameter $\lambda$
}
\begin{ruledtabular}
\begin{tabular}{lcccccr}
\textrm{$\lambda$}&
\textrm{$\beta_{m}$}&
\textrm{$\gamma_{m}$}&
\textrm{$\delta_{m}$}&
\textrm{$\frac{\beta_m}{\nu}$}&
\textrm{$\nu$}&
\textrm{$\frac{\alpha}{\nu}$}\\
\colrule
-0.2 & 0.159 &	2.237 & 15 & 0.125 & 1.134 & -0.412\\
-0.1 & 0.142 & 1.995 & 15 & 0.125 & 1.283 & -0.227\\
0 &	0.125 & 1.750 & 15 & 0.125 & 1.0 & 0\\
0.1& 0.109 & 1.541 & 15 & 0.125 & 0.772 & 0.275\\
0.2 & 0.097 & 1.372 & 15 & 0.125 & 0.881 & 0.585\\ 
\end{tabular}
\end{ruledtabular}
\end{table}



\subsection{Electric critical exponents} 


Critical exponents $\beta_e$ and $\gamma_e$ are calculated in a similar way, respectively from the log scale plot of $P=\langle \sigma\tau \rangle $ and $\chi_e= \langle P^2\rangle - \langle P\rangle^2$ as a function of $\Delta T$, for $\lambda =$ $-0.2$, $-0.1$, $0$, $0.1$ and $0.2$ respectively (see Figs. \ref{fig:exponents_elec} (a) and (b)). 
At $T=T_c,$ we calculate variation of $P$ as a function of its conjugate filed $\tilde h$, in Fig. \ref{fig:exponents_elec}(c) and variation of $P$ with $L^{-1}$ in Fig. \ref{fig:exponents_elec}(c) to 
to obtain the critical exponents $\delta_e$ (from the relation $P \sim \tilde h ^{1/\delta_e})$ and $\beta_e/\nu$ (from the relation $P \sim L^{\beta_e/\nu}$) respectively. 
All these electric critical exponents are estimated for 
$\lambda =$ $-0.2$, $-0.1$, $0$, $0.1$ and $0.2$ 
 are listed in Table \ref{tab:electric_cri_1}.


% Figure environment removed

\begin{table}[h]
\caption{\label{tab:electric_cri_1}
Results of the electric critical exponents along the AT critical line varying the model parameter $\lambda$
}
\begin{ruledtabular}
\begin{tabular}{lcccr}
\textrm{$\lambda$}&
\textrm{$\beta_{e}$}&
\textrm{$\gamma_{e}$}&
\textrm{$\delta_{e}$}&
\textrm{$\frac{\beta_e}{\nu}$}\\
\colrule
-0.2 &	0.387 &	1.775 &	5.578 & 0.304\\
-0.1 &	0.316 &	1.634 &	6.156 & 0.278\\
0. & 0.250	& 1.500	& 7.000 & 0.250\\
0.1	 &	0.189 & 1.379 & 8.264 & 0.216\\
0.2	 &	0.135 & 1.272 & 10.359 & 0.176\\
\end{tabular}
\end{ruledtabular}
\end{table}

\subsection{Confirmation of SUH in AT model}
It is evident from Table \ref{tab:mag_cri_1} that the magnetic critical exponents follow Type-I scenario or Weak-universality scenario where $\delta$ remains same but $\gamma$ varies with $\lambda$; as a consequence, from Eq. \ref{eq:Type-I}, we have $\beta_m = \frac{\beta_m^0} {g(\mu)}, \gamma_m = \frac{\gamma_m^0} {g(\mu)}$ and $\nu = \frac{\nu^0}{g(\mu)},$ where the base exponents are the magnetic critical exponents of Ising model in 2D, $\nu^0=1, \beta_m^0= \frac18, \gamma_m^0= \frac74.$ Form Eq. (\ref{eq:exact_mag_AT}) it is clear that 
\begin{eqnarray}
g(\mu) =\frac{4\mu-3\pi}{2 (\mu-\pi)} ~{\rm with} ~ \cos\mu = e^{2\lambda} \sinh(2\lambda),\nonumber
\end{eqnarray}
which provides  the numerical value  of  $g(.)$ that matches well with the value of $\nu^{-1}$ listed in Table \ref{tab:mag_cri_1}.


Now we turn our attention to the electric critical exponents. 
All of them vary continuously, and here we expect the most generic super universality scenario, i.e., Type-III stated in Eq. \eqref{eq:Type-III}. To know the  exact form of the functions $f(\mu)= \gamma_e^0/\gamma_e$ and $g(\mu)= \nu_e^0/\nu_e$ we must know 
$\gamma_e^0$ and $\nu_e^0.$ The correlation exponent, derived from the spatial correlation functions, must be unique (i.e., $\nu_e^0=\nu=1$) and it does not depend on what kind of phase transition one looks at. Since, at $\lambda=0,$ the spin variables $\sigma$ and $\tau$ are independent of each other, 
 the polarization $P=\langle \sigma \tau \rangle$ must vary as $P= \langle M_\sigma \rangle \langle M_\tau\rangle\sim(\Delta T)^{1/4},$ as $\langle M_{\sigma,\tau} \rangle \sim (\Delta T)^{1/8}.$
 Corresponding variance is then $\chi_e = \langle M_\sigma^2 \rangle \langle M_\tau^2 \rangle - \langle M_\sigma\rangle^2\langle M_\tau\rangle^2.$ Since $\langle M_{\sigma,\tau}^2 \rangle = \chi_{\sigma,\tau} + 2 \langle M_{\sigma,\tau} \rangle^2$ and $\chi_{\sigma,\tau}\sim (\Delta T)^{-7/4},$ we get the dominant variation of $\chi_e$ is $\chi_e \sim (\Delta T)^{-\frac{3}{2}}$ and thus 
 $\gamma_e^0= \frac32.$ Other exponents at $\lambda=0$ can be determined following the scaling relation (\ref{eq:scale}). Finally, the base exponents of the electric phase transition are
 \begin{eqnarray}\label{eq:elec_Ising}
\nu_e^0=1;~\gamma_e^0=3/2;~\beta_e^0=1/4;~\delta_e^0=7.
\end{eqnarray}
Using this in Eq. (\ref{eq:exact_ele_AT}) we obtain 
%Continuous variation of exponents as described in Eqs. (\ref{eq:var1}) are (\ref{eq:var1}) are the most generic ones, with 
\be
\label{eq:fg}
 f(\mu) = \frac{\gamma_e^0}{\gamma_e} =\frac{3(4 \mu-3\pi)}{8 \mu - 7 \pi} ;~ g(\mu)= \frac{1} \nu= \frac{ 4 \mu - 3\pi} {2(\mu-\pi)}
\ee
where $\cos\mu = e^{2\lambda} \sinh(2\lambda).$ In Fig. 
\ref{fig:exponents_lambda} (a)-(d) we have shown respectively the variation of the exponents $\beta_e$, $\gamma_e$, $\delta_e$, and $\beta_e/\nu$ as a function of $\lambda$ (symbols) along with the theoretical values obtained 
from Eq. \eqref{eq:Type-III} with $g(.)$ and $f(.)$ taken from Eq. \eqref{eq:fg} (in solid lines) - they match quite well, indicating that critical exponents of electric phase transition belong to Ising super-universality class. 
% Figure environment removed
% Figure environment removed

To emphasize that the generalized scaling hypothesis is indeed in work, we further investigate the scaling functions. If the phase transitions have an intrinsic Ising nature, then the scaling functions must match with that of the IUC in 2D (up to a constant multiplicative factor) even when the critical exponents are different. This must be true for both electric and magnetic phase transitions which exhibit Type-III and Type-I continuous variation of exponents respectively. 
In the vicinity of a second order phase transition, the order parameter $\phi(\vdt, B)$ depends on $\Delta T = T_c-T,$  the field conjugate $B,$ and follows a scaling relation \cite{Stanley_1971}, 
\begin{equation} \label{eq:all_scaling}
\vdt \phi(\vdt, B)^{-1/\beta}=F(\vdt B^{-1/(\beta+\gamma)}).
\end{equation}
For magnetic and electric phase transitions we use $\phi \equiv M,P$ and the conjugate fields as $B=h, \tilde h$ and plot $\vdt \phi(\vdt, B)^{-1/\beta}$ as a function of 
$\vdt B^{-1/(\beta+\gamma)}$ respectively in Fig. \ref{fig:all_scaling} (a) and (b) respectively using corresponding critical exponents. In each figure, three different plots correspond to data-collapse observed for $\lambda = -2, 0,$ and $2.$ A good data collapse is observed in all cases. Now we re-scale $x$- and $y$-axis $\lambda = -2$ and $2$ curves in (a) and (b) so that they all collapse onto the $\lambda=0$ curve, which is 
shown in Figs. \ref{fig:all_scaling} (c) and (d) respectively for 
magnetic and electric transitions. With this universal scaling function, we also plot a scaling function (dashed line) separately obtained for Monte Carlo simulation of the Ising model on a $2^{10} \times 2^{10}$ square lattice, for comparison. An excellent match clearly indicates that the scaling properties of both magnetic and electric phase transitions in the AT model can be derived from that of the parent Ising universality class. 


\section{Summary} 
 We revisit Ashkin Teller model on a two-dimensional square lattice  and calculate   the critical exponents of magnetic and electric phase transitions from  the  Monte Carlo simulations  and  compared them with  results  known  from Renormalization group  arguments and from the mapping of the model to  eight vertex model \cite{Baxter2, Baxter3}.  It is 
 well known that the magnetic critical exponents follow a weak universality scenario where critical exponents $\beta, \gamma, \nu$ vary but their ratio $\beta/\nu, \gamma/ \nu$ keeps the same value as that of the IUC in 2D. However, all the exponents of the electric phase transition vary continuously following no specific patterns and it was proposed that they are non-universal. In this article, we show that there is an in-built pattern in the variation of critical exponents, which happened due to a parameter of the model becoming marginal in RG-sense. The most generic variation that can occur, obeying the scaling relations, is derived in Eq. \eqref{eq:Type-III}. It turns out that Weak Universality is a specific example of the generic variation proposed and we call it Type-I variation. Some other examples, like mass-generation in QED \cite{Kondo} and 
magnetic phase transition in systems where spins interact with a long-ranged strain field \cite{Puri} studied in different contexts are also specific examples of the super universality hypothesis proposed here - we call it Type-II variation. 


The super universality scenario suggests that the critical feature of both the electric and magnetic transitions in AT model must have been related to the parent universality class, namely Ising Universality in 2D. We show that that is indeed the case. Although the critical exponents of both transitions vary differently, the underlying scaling functions turn out to be the scaled form of the corresponding scaling function of IUC. 

In our opinion, the super-universality scenario is quite general and it provides a guideline on the functional form of the continuous variation of critical exponents with the marginal parameter of the model that generates it. It also suggests that a marginal operator can force the critical exponents to vary continuously but it does not alter the the scaling functions; all the scaling functions are only a scaled form of the respective scaling function of the parent universality class. 



{\it \bf  Acknowledgement:}
IM acknowledges the support of the Council of Scientific and Industrial Research, India  in the form of  a research fellowship (Grant No. 09/921(0335)/2019-EMR-I).

\begin{thebibliography}{99}
 \bibitem{Baxter} R. J. Baxter, {\it Exactly Solved Model in Statistical Mechanics}, Academic Press, London, 1982.

%\bibitem{Baxter3} R. J. Baxter, Ann. Phys. (NY) {\bf 70}, 193 (1972).

\bibitem{Stanley_1971} H. E. Stanley, {\it Introduction to Phase Transition and Critical Phenomena}, Oxford Univ. Press, New York, 1971.

\bibitem{Hu_2014} C. K. Hu, Chinese J. Phys. \textbf{52}, 1-76 (2014).

\bibitem{Zhu_2020} C. P. Zhu, L. T. Sun, B. J. Kim, B. H. Wang, C. K. Hu, H. E. Stanley, Chinese J. Phys. \textbf{64}, 25-34 (2020).

\bibitem{Griffiths_1970} R. B. Griffiths, Phys. Rev. Lett. {\bf 24}, 1479 (1970). 

\bibitem{Stanley_1999} H. E. Stanley, Rev. Mod. Phys. {\bf 71}, S358 (1999).

\bibitem{Kadanoff_wsp} L. P. Kadanoff, {\it Statistical physics: statics, dynamics, and renormalization}, World Scientific Publishing, 2000.

\bibitem{Guggenheim} E. A. Guggenheim, Chem. Phys. {\bf 13}, 253 (1945).

\bibitem{Back} C. H. Back et al., Nature {\bf 378}, 597 (2005).


\bibitem{Suzuki} M. Suzuki, Prog. Theor. Phys. \textbf{51}, 1992 (1974).

\bibitem{Baxter2} R. J. Baxter, Phys. Rev. Lett. {\bf 26}, 832 (1971).

\bibitem{Baxter3} R. J. Baxter, Ann. Phys. (NY) {\bf 70}, 193 (1972).

\bibitem{Alet} F. Alet et al., Phys. Rev. Lett. {\bf 94}, 235702 (2005).

\bibitem{Queiroz} S. L. A. de Queiroz, Phys. Rev. E {\bf 84}, 031132 (2011).
 
 
\bibitem{Jin_Sen} S. Jin, A. Sen, and A. W. Sandvik, Phys. Rev. Lett. {\bf 108}, 045702 (2012).

\bibitem{Pearce} P. A. Pearce, and D. Kim, J. Phys. A: Math. Gen. {\bf 20}, 6471 (1987).

\bibitem{Malakis} A. Malakis, A. N. Berker, I. A. Hadjiagapiou and N. G. Fytas, Phys. Rev. E {\bf 79}, 011125 (2009).

\bibitem{Suzuki_Harada} T. Suzuki, K. Harada, H. Matsuo, S. Todo, N. Kawashima, Phys. Rev. B {\bf 91}, 094414 (2015).

\bibitem{Andrade_Herrmann} R. F. S. Andrade and H. J. Herrmann, Phys. Rev. E {\bf 88}, 0421(2013).

\bibitem{Sahara} R. Sahara, H. Mizuseki, K. Ohno, Y. Kawazoe, 
Mater. Trans. JIM {\bf 40}, 1314 (1999).

\bibitem{Newman} T. J. Newman, J. Phys. A: Math. Gen. {\bf 28}, L183 (1995).

\bibitem{Noh_Park} J. D. Noh and H. Park, Phys. Rev. E {\bf 69}, 016122 (2004).

\bibitem{Monceau} P. Monceau, and P. Y. Hsiao, Physica A {\bf 331}, 1 (2004).

\bibitem{Bernardi-Campbell} L. Bernardi and I. A. Campbell, Phys. Rev. B {\bf 52}, 12501 (1995).

\bibitem{Corti} M. Corti, V. Degiorgio and M. Zulauf, Phys. Rev. Lett. {\bf 48}, 1617 (1982).

\bibitem{Fisher} M. E. Fisher, Phys. Rev. Lett. {\bf 57}, 1911 (1986).

\bibitem{Bekhechi} S. Bekhechi and B. W. Southern, Phys. Rev. B {\bf 67} 144403 (2003).

\bibitem{Kondo} K. I. Kondo, Int. J. Mod. Phys. A {\bf 6} 5447 (1991).

\bibitem{Butch} N. P. Butch and M. B. Maple, Phys. Rev. Lett. {\bf 103}, 076404 (2009).

\bibitem{Fuchs} D. Fuchs et al, Phys. Rev. B {\bf 89}, 174405 (2014).

\bibitem{pkm_pmandal} N. Khan, P. Sarkar, A. Midya, P. Mandal and P. K. Mohanty, Sci. Rep. {\bf 7}, 45004 (2017).

\bibitem{Wilson} K. G. Wilson and J. Kogut, Phys. Rep. {\bf 12}, 75 (1974).

\bibitem{Puri} R. Singh and S. Puri, J. Stat. Mech. 033205.(2023).  %Strain fields and critical phenomena in manganites I: spin-lattice Hamiltonians

\bibitem{AT_1943} J. Ashkin and E. Teller, Phys. Rev. {\bf 64}, 178 (1943).

\bibitem{Kadanoff_1977} L. P. Kadanoff, Phys. Rev. Lett. {\bf 39}, 903 (1977).

\bibitem{Zisook} A. B. Zisook, J. Phys. A: Math. Gen. {\bf 13}, 2451 (1980).

\bibitem{Fan_Wu} C. Fan and F. Y. Wu, Phys. Rev. B. {\bf 2}, 723 (1970).

\bibitem{Kadanoff_1971} L. P. Kadanoff and F. J. Wagner, Phys. Rev. B. {\bf 4}, 3989 (1971).

\bibitem{Roman_Ladislav} R. Kr\ifmmode \check{c}\else \v{c}\fi{}m\'ar and L. \ifmmode \check{S}\else \v{S}\fi{}amaj, Phys. Rev. E \textbf{97}, 012108 (2018).

\bibitem{Wu_Lin} F. Y. Wu and K.Y'. Lin, J. Phys. C {\bf 7}, L181 (1974).

\bibitem{Domany_Riedel} E. Domany and E. K. Riedel, Phys. Rev. B {\bf 19}, 5817 (1979).

\bibitem{Baxter1_1972} R. J. Baxter, Ann. Phys. (NY) {\bf 70}, 323 (1972).

\bibitem{Wegner} F. J. Wegner, J. Phys. C: Solid State Phys. {\bf 5}, L131 (1972).

\bibitem{K_binder}  K. Binder, Phys. Rev. Lett. {\bf 47}, 693 (1981)

\bibitem{K_binder1} K. Binder, Z. Physik B: Condensed Matter \textbf{43}, 119 (1981).



\end{thebibliography}


\end{document}

