\documentclass[10pt,twocolumn,letterpaper]{article}

\usepackage{iccv}
\usepackage{times}
\usepackage{epsfig}
\usepackage{graphicx}
\usepackage{amsmath}
\usepackage{amssymb}

% Include other packages here, before hyperref.

\usepackage{booktabs}
\usepackage{enumerate}
\usepackage{float}
\usepackage{multirow}
\usepackage{bbding}
\usepackage{makecell}


\usepackage{subcaption}
\captionsetup{compatibility=false}

% If you comment hyperref and then uncomment it, you should delete
% egpaper.aux before re-running latex.  (Or just hit 'q' on the first latex
% run, let it finish, and you should be clear).
\usepackage[pagebackref=true,breaklinks=true,letterpaper=true,colorlinks,bookmarks=false]{hyperref}

% Support for easy cross-referencing
\usepackage[capitalize]{cleveref}
\crefname{section}{Sec.}{Secs.}
\Crefname{section}{Section}{Sections}
\Crefname{table}{Table}{Tables}
\crefname{table}{Tab.}{Tabs.}

\iccvfinalcopy % *** Uncomment this line for the final submission

\def\iccvPaperID{3035} % *** Enter the ICCV Paper ID here
\def\httilde{\mbox{\tt\raisebox{-.5ex}{\symbol{126}}}}

% Pages are numbered in submission mode, and unnumbered in camera-ready
\ificcvfinal\pagestyle{empty}\fi

\def\mywork{U2MOT}

% \newcommand{\lk}[1]{\textcolor{blue}{#1}}
% \newcommand{\liuk}[1]{\textcolor{cyan}{#1}}
\newcommand{\lk}[1]{#1}
\newcommand{\liuk}[1]{#1}
\newcommand{\todo}[1]{\textcolor{cyan}{(TODO: #1)}}
\newcommand{\minus}{\text{-}}
% \newcommand{\minus}{\scalebox{0.75}[1.0]{$-$}}
\newcommand{\topa}[1]{\underline{\textbf{#1}}}
\newcommand{\topb}[1]{#1}
% \newcommand{\topb}[1]{\textbf{#1}}

\begin{document}

%%%%%%%%% TITLE
\title{Uncertainty-aware Unsupervised Multi-Object Tracking}

\author{
Kai Liu$^{1}$\footnotemark[1], $\;$ Sheng Jin$^{2}$, $\;$ Zhihang Fu$^{2}$, $\;$ Ze Chen$^{2}$, $\;$ Rongxin Jiang$^{1}$, $\;$ Jieping Ye$^{2}$
\vspace{0.5em}\\
$^{1}$Zhejiang University, $\;$ $^{2}$Alibaba DAMO Academy\\
% For a paper whose authors are all at the same institution,
% omit the following lines up until the closing ``}''.
% Additional authors and addresses can be added with ``\and'',
% just like the second author.
% To save space, use either the email address or home page, not both
% \and
% Second Author\\
% Institution2\\
% First line of institution2 address\\
% {\tt\small secondauthor@i2.org}
}


\maketitle
% Remove page # from the first page of camera-ready.
\ificcvfinal\thispagestyle{empty}\fi

\renewcommand{\thefootnote}{\fnsymbol{footnote}}
\footnotetext[1]{This work was done when Kai Liu worked as an research intern in Alibaba DAMO Academy. Email: kail@zju.edu.cn.}
% \footnotemark[2] 
% \footnotetext[2]{Corresponding author.}

%%%%%%%%% ABSTRACT
\begin{abstract}
    Without manually annotated identities, unsupervised multi-object trackers are inferior to learning reliable feature embeddings.
    It causes the similarity-based inter-frame association stage also be error-prone, where an \textbf{uncertainty} problem arises.
    The frame-by-frame accumulated uncertainty prevents trackers from learning the consistent feature embedding against time variation.
    To avoid this uncertainty problem, recent self-supervised techniques are adopted, whereas they failed to capture temporal relations. The inter-frame uncertainty still exists.
    In fact, this paper argues that though the uncertainty problem is inevitable, it is possible to leverage the uncertainty itself to improve the learned consistency in turn.
    Specifically, an uncertainty-based metric is developed to verify and rectify the risky associations. The resulting accurate pseudo-tracklets boost learning the feature consistency.
    And accurate tracklets can incorporate temporal information into spatial transformation. This paper proposes a tracklet-guided augmentation strategy to simulate the tracklet's motion,  which adopts a hierarchical uncertainty-based sampling mechanism for hard sample mining.
    The ultimate unsupervised MOT framework, namely~\mywork, is proven effective on MOT-Challenges and VisDrone-MOT benchmark. 
    \mywork~
    achieves a SOTA performance among the published supervised and unsupervised trackers.
    %This work presents a novel unsupervised MOT framework, namely~\mywork, to alleviate these problems.
    %现象 --- 问题 --- 设计原则 --- 具体方法拆解,目前除了方法的细节性描述外,没有一个总览
    %An uncertainty-based object association and propagation strategy is first developed to preserve the temporal consistency within the pseudo-tracklets. Then we design an anchor-based temporal-prior augmentation special for multi-object tracking, which simulates realistic motion. Together with a momentum memory dictionary that collects diverse negative samples, \mywork~effectively improves the discriminability.

\end{abstract}

%%%%%%%%% BODY TEXT


\section{Introduction}
Deep learning models have been widely used in many applications.
For example, BERT~\citep{devlin_bert_2019}, GPT-3~\citep{brown_language_2020}, and T5~\citep{raffel_exploring_2020} achieved state-of-the-art~(SOTA) results on different natural language processing~(NLP) tasks. 
For computer vision~(CV), Transformer-like models such as ViT~\citep{dosovitskiy_image_2021} and Swin Transformer~\citep{liu_swin_2021} deliver excellent accuracy performance upon multiple tasks. 


At the same time, training deep learning models has been a critical problem troubling the community due to the long training time, especially for those large models with billions of parameters~\citep{brown_language_2020}. 
In order to enhance the training efficiency, researchers propose some manually designed parallel training strategies~\citep{narayanan_efficient_2021,shazeer_mesh-tensorflow_2018,xu_gspmd_2021}. 
However, selecting, tuning, and combining these strategies require extensive domain knowledge in deep learning models and hardware environments. With the increasing diversity of modern hardware architectures~\cite{flynn_very_1966,flynn_computer_1972} and the rapid development of deep learning models, these manually designed approaches are bringing heavier burdens to developers. 
Hence, \emph{automatic parallelism} is introduced to automate the parallel strategy searching for training models.


There are two main categories of parallelism in deep learning models: inter-layer parallelism~\citep{huang_gpipe_2019,narayanan_pipedream_2019,narayanan_memory-efficient_2021,fan_dapple_2021,li_chimera_2021,lepikhin_gshard_2021,du_glam_2022,fedus_switch_2022} and intra-layer parallelism~\citep{li_pytorch_2020,narayanan_efficient_2021,rasley_deepspeed_2020,fairscale_authors_fairscale_2021}. 
Inter-layer parallelism partitions the model into disjoint sets on different devices without slicing tensors. 
Alternatively, intra-layer parallelism partitions tensors in a layer along one or more axes and distributes them across different devices.


Current automatic parallelism techniques focus on optimizing strategies within these two categories. However, they treat these two categories separately. 
Some methods~\citep{zhao_vpipe_2022,jia_exploring_2018,cai_tensoropt_2022,wang_supporting_2019,jia_beyond_2019,schaarschmidt_automap_2021,liu_colossal-auto_2023} overlook potential opportunities for inter- or intra-layer parallelism, the others optimize inter- and intra-layer parallelism hierarchically and sequentially~\citep{narayanan_pipedream_2019,fan_dapple_2021,he_pipetransformer_2021,tarnawski_efficient_2020,tarnawski_piper_2021,zheng_alpa_2022}. 
As a result, current automatic parallelism techniques often fail to achieve the global optima and instead become trapped in local optima. 
Therefore, a unified inter- and intra-layer approach is needed to enhance the effectiveness of automatic parallelism.


This paper aims to find the optimal parallelism strategy while simultaneously considering inter- and intra-layer parallelism. 
It enables us to search in a more extensive strategy space where the globally optimal solution lurk. 
However, unifying inter- and intra-layer parallelism in automatic parallelism brings us two challenges. 
Firstly, to adopt a unified perspective on the inter- and intra-layer automatic parallelism, we should not formalize them with separate formulations as prior works. Therefore, how can we express these parallelism strategies in a unified formulation? 
Secondly, previous methods take a long time to obtain the solution with a limited strategy space. Therefore, how can we ensure that the best solution can be obtained in a reasonable time while expanding the strategy space?


To solve the above challenges, we propose UniAP. For the first challenge, UniAP adopts the mixed integer quadratic programming~(MIQP)~\citep{lazimy_mixed_1982} to search for the globally optimal parallel strategy automatically. 
It unifies the inter- and intra-layer automatic parallelism in a single MIQP formulation. 
For the second challenge, our complexity analysis and experimental results show that UniAP can obtain the globally optimal solution in a significantly shorter time.


The contributions of this paper are summarized as follows: 
\begin{itemize}
    \item We propose UniAP, the first framework to unify inter- and intra-layer automatic parallelism in model training.
    \item The optimal parallel strategies discovered by UniAP exhibit scalability on training throughput and strategy searching time.
    \item The experimental results show that UniAP speeds up model training on four Transformer-like models by up to 1.70$\times$ and reduces the strategy searching time by up to 16$\times$, compared with the SOTA method.
\end{itemize}

\section{Related Work}
\label{sec:related}

\begin{table}[t]
\small
\centering
\caption{Comparison of our method with related settings}
\begin{tabular}{cccc}
\toprule
Setting & Detect Novel OOD Data & Semi-Supervised & Learns from Novel OOD Data \\
\midrule
SSOD & \xmark & \cmark & \xmark \\ 
Open-World OD & \cmark & \xmark & \cmark \\
Open-Set SSOD & \cmark & \cmark & \xmark \\ \midrule
\textbf{Our Method} & \cmark & \cmark & \cmark \\
\bottomrule
\end{tabular}
\label{tab:comparison}
\end{table}

\paragraph{Semi-Supervised Object Detection.} Semi-supervised object detection (SSOD) approaches have become popular to reduce the need for labeling \cite{sohn2020detection, berthelot2019mixmatch, jeong2019consistency}. Pseudo-labeling based methods such as FlexMatch \cite{zhang2021flexmatch}, TSSDL \cite{shi2018transductive}, and others \cite{iscen2019label, luo2018smooth, yan2019semi, liu2021unbiased, xu2021end}, first train a teacher model using only labeled data and then use that model to create pseudo-labels for unlabeled images. The pseudo-labels are then used along with the original labeled data to train a student model. On the other hand, consistency regularization approaches such as \cite{sajjadi2016regularization, laine2017temporal, tarvainen2017mean, liu2021certainty, luo2018smooth, jeong2019consistency, iscen2019label, liu2021unbiased, xu2021end}, aim to minimize a consistency loss between differently augmented versions of an image. All of these semi-supervised learning approaches assume a ``closed-world'' setting with a fixed set of classes in both training and testing, which is not a valid assumption in real-world applications.

\paragraph{Open-World Object Detection.} Open-world object detection enables the detection of novel objects by incrementally adding novel object classes to the set of known classes. Previous work \cite{kim2022learning, kuo2015deepbox, o2015learning, wang2020leads, Maaz2022Multimodal} has studied different methods of object proposals for novel objects by attempting to remove the notion of class (all objects are regarded the same). ORE \cite{joseph2021towards} is the first to propose an open-world object detector that identifies novel classes as ‘unknown’ and proceeds to learn the unknown classes once the labels become available. \cite{han2022expanding} aims to identify unknown objects by separating high/low-density regions in the latent space. Both these approaches work in a fully-supervised setting. Our setup goes a step further and situates the open-world problem in the context of semi-supervised learning, with limited amounts of labeled ID data \textit{only}, that more closely resembles the real-world settings. 

\paragraph{Unsupervised Object Localization.} Recently proposed methods such as CutLER \cite{wang2023cut}, FreeSolo \cite{wang2022freesolo}, LOST \cite{LOST}, and MOST \cite{rambhatla2023most} propose to localize objects in an unsupervised manner, either by segmentation masks or bounding boxes. Some of these \cite{wang2023cut, LOST, rambhatla2023most} use features from self-supervised trained transformers to localize objects in the scene. In our work, we evaluate the capabilities of such methods for localizing OOD objects, as they present open-world capabilities. Based on our evaluation (\ref{sec:expts:ablation}), we use CutLER as part of the OOD Explorer to localize OOD classes. Section \ref{sec:expts} provides the details of our evaluation. 

\paragraph{Open-Set/Open-world Semi-Supervised Object Detection.}
The open-set semi-supervised object detection problem \cite{liuopen} addressed some of the limitation of the above mentioned work. Furthermore, they address like the performance of ID classes in the presence of OOD data, but they do not learn from it or improve OOD performance. They propose an offline OOD detector to filter out OOD data, thus limiting the risk of ID performance in the presence of OOD data. In contrast, our approach \textit{both} improves performance for ID classes \textit{as well as} OOD classes, i.e., our proposed framework solves a strictly stronger problem. Specifically speaking, \cite{liuopen} solves for identifying novel classes and filters it out, but does not re-introduce the classes back into the training pipeline in order to be able to learn its features. \cite{mullappilly2024semi} addresses some of the limitations of the previous mentioned methods by extending the problem to a semi-supervised setting. However, their problem setting is similar to an incremental learning setting, access to unknown class labels is provided in subsequent tasks. Our generalized setting, on the other hand, does not require access to any unknown class labels. 

\section{Methodology}
\label{sec:method}

\subsection{Overview}
\label{sec:method_fmwk}

As shown in~\cref{fig:method_fmwk}, the proposed unsupervised MOT framework is trained with the widely-used contrastive learning technique~\cite{chen2020simple,he2020momentum}. 
\lk{Specifically, for multi-object tracking}, objects within the tracklet ($\boldsymbol{k}_{+}$) should be pulled together and different tracklets ($\boldsymbol{k}_{-}$) should be separated. It can be mathematically formulated as:

\begin{equation}
% \begin{split}
    \mathcal{L}_{cl}( \boldsymbol{q}; \boldsymbol{k}_{+}; \boldsymbol{k}_{-} )= 
    - \log \frac{\exp(\boldsymbol{q} \cdot \boldsymbol{k}_{+} / \epsilon)}{\sum_{i}\exp(\boldsymbol{q} \cdot \boldsymbol{k}_{i} / \epsilon)}
  \label{eq:method_nce}
% \end{split}  
\end{equation}

\noindent where $\mathcal{L}_{cl}$ denotes the InfoNCE~\cite{oord2018representation} loss function, and $\epsilon$ is the temperature hyper-parameter~\cite{wu2018unsupervised}. 
Within a video, following the unsupervised tracking fashion~\cite{liu2022online,shuai2022id}, the positive and negative keys mainly come from two sources, \ie pseudo-labeled historical frame and self-augmented frame. 

\lk{However, two issues occur: (1) the uncertainty reduces the accuracy of pseudo-tracklets and (2) the randomly augmented samples fail to learn the inter-frame consistency.} 
We argue the above issues are not independent. 
\lk{By leveraging the uncertainty in turn,} the accurate pseudo-tracklets can guide the qualified positive and negative augmentations.

To address these two issues, we propose an uncertainty-aware pseudo-tracklet labeling strategy in \cref{sec:method_uoap}, which integrates a verification-and-rectification mechanism into the tracklet generation. Our method significantly improves the accuracy of pseudo-identities, especially in long-term interval. 
Then we propose a tracklet-guided augmentation strategy in \cref{sec:method_ada_aug}, which brings the temporary information into spatial augmentation. The augmented samples simulates the objects' motion. A hierarchical uncertainty-based sampling strategy is proposed for hard sample mining. More details are described in the following section.


\subsection{Uncertainty-aware Tracklet-Labeling}
\label{sec:method_uoap}

Accurate pseudo tracklet is critical in \liuk{learning feature consistency}. 
However, without manual annotation, \lk{the aggravated uncertainty makes} the tracklet-labeling a huge challenge due to various interference factors, including similar appearance among objects, frequent object cross and occlusions, \etc. 
\lk{In fact, the uncertainty can also be leveraged to improve the pseudo-accuracy in turn.}
In this section, we propose an \textbf{U}ncertainty-aware \textbf{T}racklet-\textbf{L}abeling (\textbf{UTL}) strategy for better pseudo-tracklets.

Given an input video sequence $V = \{I^{1}, I^{2}, \cdots, I^{N}\}$, each frame $I^{t}$ is annotated with the bounding boxes $B^{t} = \{b_{1}^{t}, b_{2}^{t}, \cdots, b_{M^{t}}^{t}\}$ of $M^{t}$ objects in $t_{th}$ frame, where $b_{i}^{t} = (cx_{i}^{t}, cy_{i}^{t}, w_{i}^{t}, h_{i}^{t})$ is the center coordinate and shape of the $i_{th}$ object $o_{i}^{t}$. As shown in~\cref{fig:method_fmwk}, \mywork~generates accurate pseudo-tracklets in four main steps:

1) \textbf{Association}. For a certain object $o_{i}^{t}$ in frame $I^{t}$, the $\ell_2$-normalized representation $\boldsymbol{f}_{i}^{t}$ can be expressed as $\boldsymbol{f}_{i}^{t} = {\phi}(I^{t}, b_{i}^{t})$, 
% \begin{equation}
%   \boldsymbol{f}_{i}^{t} = {\phi}(I^{t}, b_{i}^{t})
%   % / {\Vert {\phi}(I^{t}, b_{i}^{t}) \Vert}_{2}
%   \label{eq:method_feat}
% \end{equation}
where the embedding encoder is denoted as $\phi$.

To associate the objects in frame $I^{t}$ with the objects or trajectories in previous $I^{t \minus 1}$, a similarity matrix is constructed with their appearance embeddings:

\begin{equation}
  \boldsymbol{C} \in \mathbb{R}^{M^{t} \times M^{t \minus 1}}, \;
  c_{i,j} = {\boldsymbol{f}_{i}^{t}} \cdot  \boldsymbol{f}_{j}^{t \minus 1}
  \label{eq:method_matrix}
\end{equation}

\noindent where $c_{i,j}$ represents the cosine similarity between the $i_{th}$ object in frame $I^{t}$ and the $j_{th}$ object (or trajectory) in frame $I^{t \minus 1}$. Then the Hungarian algorithm~\cite{kuhn1955hungarian} is adopted to generate the identity association results.

2) \textbf{Verification}. However, the appearance representations are sometimes unreliable, especially in the unsupervised scenario. To solve this issue, an uncertainty metric is proposed to evaluate the association after the first stage.

% For an object $o_{i}^{t}$ in frame $I^{t}$, the similarities against the $M^{t \minus 1}$ objects in the previous frame can be expressed as:

% \begin{equation}
%   \boldsymbol{s}_{i} = \boldsymbol{C}_{i} = [c_{i,1}, c_{i,2}, \cdots, c_{i,M^{t \minus 1}}]^T
%   \label{eq:method_svec}
% \end{equation}

% Inspired by margin-based OOD detection~\cite{hendrycks2016baseline}, we assume that the assignment ($o_{i}^{t} \!\sim\! o_{j}^{t \minus 1}$) in the association stage is not convincing under the following circumstances:

% \begin{itemize}
%     \setlength{\itemsep}{0pt}
%     \item The assigned similarity between $o_{i}^{t}$ and $o_{j}^{t \minus 1}$ is relatively low (\ie, $c_{i,j} < m_1$).
%     \item The second-highest similarity with others ($c_{i,j_{2}}$) is close to the assigned $o_{j}^{t \minus 1}$ (\ie, $c_{i,j} - c_{i,j_{2}} < m_2$).
% \end{itemize}

% Based on these assumptions, we define an association-level uncertainty metric, which is formulated as:



Object association can be viewed as multi-category classification.
And confidence-score has been proved efficient and effective on detecting mis-classified examples~\cite{hendrycks2016baseline}.
Inspired by this, we propose to detect the mis-associated objects through the similarity-scores.


Given an object $o_{i}^{t}$ associated with $o_{j}^{t \minus 1}$ in the previous frame based on \cref{eq:method_matrix}, the association ($o_{i}^{t} \!\sim\! o_{j}^{t \minus 1}$) is unconvincing in two cases: 
1) the assigned similarity $c_{i,j}$ is relatively low (\eg, partial occlusion or motion blur) and 
2) there are other objects whose similarities are close to the assigned $c_{i,j}$ (\eg, similar appearance or indistinguishable embedding).
It can be formulated as:

\begin{equation}
  c_{i,j} < m_1; \quad c_{i,j_{2}} > c_{i,j} - m_2
  \label{eq:method_margin}
\end{equation}


\noindent 
where $m_1,m_2$ are constant margins. Only the second-highest similarity with others ($c_{i,j_{2}}$) is considered for simplicity.
In an ideal association, $c_{i,j}$ should be close to 1 and $c_{i,j_{2}}$ close to 0.
We thus proposed to estimate the association \lk{risk} as:

% \sigma_{i,j} = - \left( 
% \log c_{i,j} + \log \left( 1 - c_{i,j_{2}} \right)
% + \overline{\log \left( 1 - c_{i,l} \right) }
% \right)  
\begin{equation}
  \sigma_{i,j} = - \log c_{i,j} - \log \left( 1 - c_{i,j_{2}} \right)
  \label{eq:method_energy}
\end{equation}

Detailed derivation process refers to the supplementary materials.
Combining with \cref{eq:method_margin} and \cref{eq:method_energy} , an adaptive threshold is proposed:

\begin{equation}
  % \gamma_{i,j} = -\log \left( 1 + m_2 - c_{i,j} \right) -\log m_1 \left( 1 - m_3 \right)
  \gamma_{i,j} =  -\log m_1 - \log \left( 1 + m_2 - c_{i,j} \right)
  \label{eq:method_border}
\end{equation}

As shown in~\cref{fig:method_verify}, when the \lk{risk} $\sigma_{i,j}$ is higher than the threshold $\gamma_{i,j}$, the assignment ($o_{i}^{t} \!\sim\! o_{j}^{t \minus 1}$) should be re-considered. 
\lk{The \textbf{association uncertainty} is quantified as:}

\begin{equation}
  \delta_{i,j} = \sigma_{i,j} - \gamma_{i,j}
  \label{eq:method_uncertain}
\end{equation}

The results are not sensitive to the exact margins. We set $m_1 = 0.5$ and $m_2 = 0.05$ for a slightly better performance.
% More experimental details are shown in the supplementary materials.

The uncertain pairs after the verification stage and unmatched objects after the association stage are gathered as uncertain candidates for the rectification stage.


3) \textbf{Rectification}. 
The rectification stage is performed among the uncertain candidate. The similarities between two adjacent frames are no longer convincing.
% due to irregular motion, severe occlusion, and so on. 
More information should be taken into account, including motion \lk{estimation} and appearance \lk{variation} within a tracklet. 
% Specifically, intersection-over-union (IoU)~\cite{bewley2016simple} is the widely-used motion metric. At the same time, the tracklet embeddings can provide complementary appearance information.

For the uncertain candidates, \mywork~constructs another similarity matrix for the secondary rectification. 
First, \lk{the motion constraints should be relaxed}, so the association shares overlap \lk{higher than} $\beta$ 
% in adjacent frames 
\lk{are preserved}. 
Second, \lk{the appearance should not vary extremely fast}, so we adopt the averaged similarity between object $o_{i}^{t}$ and tracklet $trk_{j} = \{o_{j}^{t \minus K}, \cdots, o_{j}^{t \minus 1}\}$ within previous $K$ frames. 
In this stage, we solve the sub-problem of global identity assignments, which can be formulated as:

\begin{equation}
\begin{split}
  \boldsymbol{C}^\prime \in \mathbb{R}^{{M^{t}}^\prime \times {M^{t \minus 1}}^\prime} & \\
  c^\prime_{i,j} = \left( \frac{1}{K} \sum_{\hat{t} = t \minus K}^{t \minus 1} {\boldsymbol{f}_{i}^{t}} \cdot  \boldsymbol{f}_{j}^{\hat{t}} \right) 
            \times \mathbb{I} & \left( \text{IoU} \left( b_{i}^{t}, b_{j}^{t \minus 1} \right) > \beta \right) 
  \label{eq:method_recti}
\end{split}
\end{equation}

\noindent where $\mathbb{I}(*)$ is the indicator function. Then the match set is updated based on the Hungarian algorithm.

\lk{
\textit{Remark.} Our core contribution is the uncertainty-based verification mechanism, rather than the specific rectification, which shall be adjusted in practice. Empirically we set $\beta=0.1$ and $K=5$.
}

% Figure environment removed

4) \textbf{Propagation}. The pseudo-tracklets are propagated frame-by-frame. As shown in~\cref{fig:method_reidacc}, our strategy brings \lk{consistently} accurate pseudo-identities, \lk{\eg, reaching 97\% accuracy across 100 frames}.
% The pseudo-tracklets are progressively updated during the training process.
The long-term intra-tracklet consistency is successfully maintained.
% by the accurate pseudo-identities.

It is worth mentioning that the \lk{verification and rectification} stages can be naturally applied to the inference process to boost the performance, \lk{which does not conflict with existing association methods}.

\subsection{Tracklet-Guided Augmentation}
\label{sec:method_ada_aug}

The accurate pseudo-tracklets can guide the sample augmentation in the unsupervised MOT framework.
To learn the \liuk{inter-frame consistency}~\cite{chen2020simple,zhang2021fairmot}, good training samples should be diverse and \liuk{temporal-aware}. 
However, as illustrated in~\cref{fig:method_ada_aug}, existing methods usually treat augmentation and multi-object tracking as two isolated tasks, leading to ineffective augmentations. 
Instead, this paper utilizes the tracklet's spatial displacements to guide the augmentation process. 
According to a properly selected anchor pair, the proposed strategy makes the augmented frames aligned to the historical frames, simulating realistic tracklet movements. The proposed method concurrently focuses on the hard negative samples.
Details \lk{of the \textbf{T}racklet-\textbf{G}uided \textbf{A}ugmentation (TGA)} are given below.

% Figure environment removed

We introduce the temporal information into spatial transformation. 
First, given a current frame $I^{t}$ with $M^{t}$ objects, we select a source-anchor object $o_{a}^{t}$, whose bounding box is denoted as $b_{a}^{t} = (cx_{a}^{t}, cy_{a}^{t}, w_{a}^{t}, h_{a}^{t})$. Then, we choose a target-anchor $o_{a}^{t \minus \tau}$ in $(t \minus \tau)_{th}$  historical frame from the pseudo-tracklet $trk_{a} = \{o_{a}^{t_0}, o_{a}^{t_1}, \cdots, o_{a}^{t}\}$. 
Finally, to augment the current $I^{t}$ to align with historical $I^{t \minus \tau}$,  a tracklet-guided affine transformation can be expressed as:

\begin{equation}
  \begin{bmatrix}
      x^{t \minus \tau} \\ y^{t \minus \tau} \\ 1
  \end{bmatrix}
  =
  \boldsymbol{M}_{t}^{t \minus \tau}
  \begin{bmatrix}
      x^{t} \\ y^{t} \\ 1
  \end{bmatrix}
  =
  \begin{bmatrix}
      m_{11} & m_{12} & m_{13} \\
      m_{21} & m_{22} & m_{23} \\
      0      & 0      & 1
  \end{bmatrix}
  \begin{bmatrix}
      x^{t} \\ y^{t} \\ 1
  \end{bmatrix}
  \label{eq:method_affine}
\end{equation}

\noindent where $x^*,y^*$ are spatial coordinates, and $\boldsymbol{M}_{t}^{t \minus \tau}$ can be solved by direct linear transform (DLT) algorithm ~\cite{detone2016deep}. 
% with the corner locations of the anchor pair $(o_{a}^{t} \!\sim\! o_{a}^{t \minus \tau})$. 
Then an augmented frame $\tilde{I}^{t}$ is generated based on the tracklet-guided affine transformation with perspective jitter, which can be expressed as $\tilde{I}^{t} = \mathcal{T}\left(I^{t}, M_{t}^{t \minus \tau} \right)$.
% \begin{equation}
%   \tilde{I}^{t} = \mathcal{T}\left(I^{t}, M_{t}^{t \minus \tau} \right)
%   \label{eq:method_aug}
% \end{equation}

Intuitively, a proper anchor-selection is vitally important for our augmentation strategy. 

First, the identity accuracy of anchor pair $(o_{a}^{t} \!\sim\! o_{a}^{t \minus \tau})$ is important. In other words, the consistency of anchor tracklet $trk_{a}$ should be guaranteed. We thus design a tracklet-level uncertain metric based on the propagated association-level uncertainty defined in \cref{eq:method_uncertain}, which is formulated as:

\begin{equation}
  \Omega_{i} = \frac{1}{n} \sum_{s=t_0}^{t} \exp (\delta_{i}^{s})
  % \Omega_{i} = \sqrt[n]{\sigma_{i}^{t_0} \cdot \sigma_{i}^{t_1} \cdots \sigma_{i}^{t}}
  \label{eq:method_tenergy}
\end{equation}

\noindent where $\Omega_{i}$ represents the uncertainty of tracklet $trk_{i}$, \lk{and $n$ is the tracklet length}.
An uncertainty-based sampling strategy is designed to select the source anchor $o_{a}^{t}$ (along with the anchor $trk_{a}$) from the $M^{t}$ objects in frame $I^{t}$, which can be formulated as:

\begin{equation}
  p\left(a=i \mid t \right) 
  % = softmax\left(-\Omega_{i}\right)
  = \frac{\exp{\left(-\Omega_{i}\right)}}{\sum_{\hat{i}=1}^{M^{t}}\exp{\left(-\Omega_{\hat{i}}\right)}}
  \label{eq:method_sel_an_src}
\end{equation}

\noindent where $p\left(a=i \mid t \right)$ represents the probability to choose the $i_{th}$ tracklet $trk_{i}$ as the anchor $trk_{a}$.
The uncertain tracklet with high $\Omega$ is less likely to be selected, avoiding dramatic augmentations from erroneous pseudo-tracklets.

Second, hard negative samples matters in discriminablity learning. We tend to choose an indistinguishable (or, high uncertain) target anchor $o_{a}^{t \minus \tau}$ along the tracklet $trk_{i}$. The selection probability can be formulated as:

\begin{equation}
  p\left(\pi=t \minus \tau \mid a \right) 
  = \frac{\exp{\left(\delta_{a}^{t \minus \tau}\right)}}{\sum_{\hat{\tau}=t_0}^{t-1}\exp{\left(\delta_{a}^{t-\hat{\tau}}\right)}}
  \label{eq:method_sel_an_tgt}
\end{equation}

\lk{A visualization example are displayed in the supplementary material to illustrate the hierarchical sampling process.}

Compared with conventional random transformation, the proposed tracklet-guided augmentation is well-directed and tracking-related. 
\lk{Together with accurate pseudo-tracklets, \mywork~successfully improves the inter-frame consistency, as shown in \cref{fig:method_disc_vis}. }


% Figure environment removed

% \subsection{Momentum Memory Dictionary}
% \label{sec:method_md}


%To reuse the encoded samples from the intermediate mini-batches, we maintain a queue for each video in the memory dictionary by enqueueing the $M^{t}$ objects in the current frame and removing the oldest samples.
%In representation learning, high-quality negative samples play an essential role~\cite{chen2020simple,he2020momentum}. However, existing unsupervised trackers only take negative samples from adjacent frames, augmented frames, and the current frame itself. The lack of negative sample diversity prevents trackers from learning discriminative representations. \mywork~adopts a momentum dictionary mechanism to alleviate this problem.

%As shown in~\cref{fig:method_fmwk}, we build a memory dictionary for each \textit{independent} video input during training. Given an input image $I^{t}$ from video $V$, we randomly sample a number of negative object samples from other videos in the memory dictionary, so as to enrich the negative sample diversity. To reuse the encoded samples from the intermediate mini-batches, we maintain a queue for each video in the memory dictionary by enqueueing the $M^{t}$ objects in the current frame and removing the oldest samples.
\section{Experiment}

\subsection{Experimental Setup}

% \begin{table}[]
% \caption{  Dataset Statistics }
% \begin{tabular}{cccc}
% \hline
% Dataset     & Users & Items  & Avg. Length of Sequence \\ \hline
% Electronics & 22685 & 20712  & 15.26                   \\
% Movies      & 26933 & 18855  & 28.97                   \\
% Beauty      & 82659 & 119365 & 28.97                   \\ \hline
% \end{tabular}
% \end{table}

\begin{table}[]
\caption{  Dataset Statistics }
\label{table:statistics}
\begin{tabular}{cccc}
\hline
Dataset     & Users & Items  & Avg. Length of Sequence \\ \hline
Electronics & 29710 & 20712  & 13.51                   \\
Movies      & 199435 & 155527  & 10.87                   \\
Beauty      & 82659 & 124859 & 6.96                   \\ \hline
\end{tabular}
\end{table}

\subsubsection{Datasets}
%\textbf{\textcolor{red}{@Bardia: please write about each dataset and its preprocessing that we used + statistics table}}

From Amazon, we have adopted three widely used real-world datasets, which are shown in Table \ref{table:statistics}. The Electronics dataset is derived from the public Amazon review dataset. This includes reviews of Amazon products belonging to the "Electronics" category from May 1996 to July 2014. Both The Movies and The Beauty are drawn from the same "Movie" and "Beauty" Amazon review categories. User reviews are treated as an interaction between them. These interactions are treated equally on all items. The $K$ parameter specifies the minimum number of transactions a user must keep. In addition, we delete users with fewer interactions in the system. We sort the data in order of the first transaction time, user ID, and transaction time. We then assign a new label to the items and users according to their appearance time, so that the first user is one, the second user is two, etc. Lastly, we will separate the test data from the test items. We will only keep the test data that contains interactions that use items from the test items in the test data. For each user node in the test and validation sets, we take each observed edge as a positive sample of the user. We then randomly select 100 items that did not interact with the current user as negative samples. Then based on the rank of the positive sample's score among negative samples, evaluation metrics will be calculated as in 


\cite{wang2021sequential,wei2020fast,kang2018self}.


\subsubsection{Baselines}


% Please add the following required packages to your document preamble:
% \usepackage{booktabs}
% \usepackage{multirow}
% Please add the following required packages to your document preamble:
% \usepackage{multirow}
% \begin{table}[]
% \caption{ Comparison of Different Models }
% \begin{tabular}{|c|cc|cc|cc|}
% \hline
% \multirow{}{}{Methods} & \multicolumn{2}{c|}{Electronics}   & \multicolumn{2}{c|}{Movie}         & \multicolumn{2}{c|}{Beauty}        \\ \cline{2-7} 
%                          & \multicolumn{1}{c|}{Hit@1} & MRR   & \multicolumn{1}{c|}{Hit@1} & MRR   & \multicolumn{1}{c|}{Hit@1} & MRR   \\ \hline
% BERT4Rec                 & \multicolumn{1}{c|}{0.200} & 0.323 & \multicolumn{1}{c|}{0.220} & 0.351 & \multicolumn{1}{c|}{0.214} & 0.341 \\ \hline
% MeLU                     & \multicolumn{1}{c|}{0.136} & 0.243 & \multicolumn{1}{c|}{0.168} & 0.289 & \multicolumn{1}{c|}{0.160} & 0.279 \\ \hline
% MAMO                     & \multicolumn{1}{c|}{0.127} & 0.296 & \multicolumn{1}{c|}{0.194} & 0.320 & \multicolumn{1}{c|}{0.195} & 0.310 \\ \hline
% MetaTL                   & \multicolumn{1}{c|}{0.241} & 0.320 & \multicolumn{1}{c|}{0.267} & 0.337 & \multicolumn{1}{c|}{0.231} & 0.328 \\ \hline
% MetaCF                   & \multicolumn{1}{c|}{0.210} & 0.330 & \multicolumn{1}{c|}{0.234} & 0.365 & \multicolumn{1}{c|}{0.220} & 0.340 \\ \hline
% \textbf{Our Model}                & \multicolumn{1}{c|}{}      &       & \multicolumn{1}{c|}{}      &       & \multicolumn{1}{c|}{}      &       \\ \hline
% \end{tabular}
% \end{table}


% \usepackage{tabularray}
% \begin{table}
% \centering
% \caption{Baselines}
% \label{table:evaltable}
% \begin{tblr}{
%   cells = {c},
%   cell{1}{1} = {r=2}{},
%   cell{1}{2} = {c=2}{},
%   cell{1}{4} = {c=2}{},
%   cell{1}{6} = {c=2}{},
%   % vline{2-3,5} = {1}{},
%   % vline{4,6} = {2}{},
%   % vline{2,4,6} = {3-8}{},
%   hline{1,3-9} = {-}{},
%   hline{2} = {2-7}{},
% }
% Methods            & Electronics &         & Movie   &         & Beauty  &         \\
%                    & $Hit@1$     & $MRR$   & $Hit@1$ & $MRR$   & $Hit@1$ & $MRR$   \\
% BERT4Rec           & $0.200$     & $0.323$ & $0.220$ & $0.351$ & $0.214$ & $0.341$ \\
% MeLU               & $0.136$     & $0.243$ & $0.168$ & $0.289$ & $0.160$ & $0.279$ \\
% MAMO               & $0.127$     & $0.296$ & $0.194$ & $0.320$ & $0.195$ & $0.310$ \\
% MetaTL             & $0.241$     & $0.320$ & $0.267$ & $0.337$ & $0.231$ & $0.328$ \\
% MetaCF             & $0.210$     & $0.330$ & $0.234$ & $0.365$ & $0.220$ & $0.340$ \\
% \textbf{Our Model} & \textbf{0.271} & \textbf{0.383} & \textbf{0.251} & \textbf{0.371} & \textbf{0.251} & \textbf{0.360} 
% \end{tblr}
% \end{table}





%\textbf{\textcolor{red}{@Bardia: please write about baselines and their corresponding adaptations}}

We compare the proposed model with the following methods: 

(i) Sequential recommendation baselines utilize different methods to capture the sequential patterns in the interaction sequences of users:

\begin{itemize}
    % \item SASRec:  Rely on Gated Recurrent Units, the simple convolutional generative network, and the self-attention layers to learn sequential user behaviors, respectively.
    \item SASRec: presents a self-attentive sequential recommendation model that utilizes Gated Recurrent Units, a simple convolutional generative network, and a self-attention mechanism to capture sequential patterns in user behavior and improve recommendation accuracy. The model is trained using a modified BPR loss function. 

    % \item BERT4Rec: adopts the bi-directional transformer to extract the sequential patterns, which is state-of-the-art for the sequential recommendation.
    \item BERT4Rec: proposes a novel recommendation model that uses the BERT architecture to capture sequential patterns in user behavior and improve recommendation accuracy. The model is trained using the bi-directional transformer to extract sequential patterns, outperforming other state-of-the-art models in accuracy and robustness to cold-start and long-tail item problems. The paper acknowledges some BERT4Rec limitations, such as its computational complexity and data requirements. However, it argues that the model's benefits justify the additional computational resources. 

\end{itemize}

(ii) Cold-start baselines include methods that provide accurate recommendations for customers with limited information. We modify these cold-start baselines to fit the case without auxiliary information. To deal with this issue, in the no side-information setting, for the datasets, we convert them into implicit recommendations by setting rated items to 1 and others to 0, and we utilize the Marginal Ranking loss function, which is the same as in our model, as we make implicit recommendations for binary signals. We just use the ID embedding of users and items as a feature (some methods like NGCF and LightGCN use this kind of embedding). In the training phase of recommendations, we sample data from the user and corresponding positive and negative items to calculate the loss at each step.

\begin{itemize}
    \item MeLU: Resolve the cold-start problem faced by existing recommender systems. The MeLU method uses meta-learning to estimate new users' preferences based on items they have consumed in the past. Moreover, the system provides a strategy to select evidence candidates to estimate customized preferences. It is shown that MeLU has a lower mean absolute error than two comparative models when tested on two benchmark datasets. In addition, the evidence selection strategy is tested in a user study. It aims to overcome the limitations of previous recommendation studies. These studies provided poor recommendations for users who consumed few items and inadequate evidence for candidates to identify user preferences.

    \item MetaTL: For cold-start users with minimal logged interactions, capturing sequential patterns of users for sequential recommenders is challenging. Models with limited interactions lose their predictive power due to difficulties in learning sequential patterns. Using meta-learning, the method proposes an innovative MetaTL framework that models users' transition patterns. A translation-based architecture extracts dynamic transition patterns from sequential recommendations in MetaTL, and meta-transitional learning facilitates fast learning for cold-start users with limited interaction. Meta-learning can improve sequential recommendations for cold-start users by inferring accurate sequential interactions.

    \item MAMO: Two memory matrices are used to store task-specific and feature-specific memories to support personalized parameter initialization and fast user preference prediction. 
    
    \item MetaCF: Discusses the cold-start problem in Collaborative Filtering (CF), where limited data is available for new users in the system. Previous approaches use user profiles, but these are not always available due to privacy concerns. MetaCF is a novel learning paradigm that leverages meta-learning to enable fast adaptation for new users. MetaCF learns a suitable initialization model for rapidly adapting to a new user. Adaptation rates are optimized in a fine-grained manner using Dynamic Subgraph Sampling to account for the dynamic arrival of new users. The proposed framework outperforms state-of-the-art baselines by a large margin in the cold-start scenario with limited user-item interactions.

\end{itemize}

\subsubsection{Evaluation Metrics}
%\textbf{\textcolor{red}{@Bardia: please write about them like metaTL with different sentences.}}

% In the experiment, each user only has one positive and true item for testing. With the predicted scores, we take each observed edge as a positive sample of the user and then randomly select 100 items that did not interact with the current user as the negative samples. This method has been widely used in many other works. Then we rank the list of positive and 100 negative items. We use Hit Ratio at rank 10 (HR@10)as the evaluation metric to measure the ranking performance. Mean Reciprocal Rank ($MRR$) indicates the rankings of the positive items. We also evaluate the Hit Rate ($Hit$) for the top-1 prediction. $Hit$@1 = 1 if the positive item is ranked top-1, otherwise $HR$@1 = 0. Also, note that $HR$@1 equals the recall or NDCG for top-1 prediction.

Each user was tested on only one positive and true item during the experiment. Based on the predicted scores, observed edges were taken as positive samples for the user. 100 items without interaction with the user were randomly selected as negative samples. This method is commonly used in other works. The list of 100 negative and positive items was ranked, and Hit Ratio at rank 10 (HR@10) was applied as the evaluation metric to measure ranking performance. Mean Reciprocal Rank (MRR) was used to indicate the ranking of positive items, and Hit Rate (Hit) was evaluated for the top-1 prediction. If the positive item was ranked top-1, Hit@1 was equal to 1; otherwise, it was 0. It should be noted that HR@1 is equivalent to recall or NDCG for top-1 prediction.

\subsection{Overall Performance}

\begin{table*}[!t]
\centering
\caption{Experimental results of different methods under K=3 on three data sets}
\label{table:evaltable}
\begin{tblr}{
  cells = {c},
  cell{1}{1} = {r=2}{},
  cell{1}{2} = {c=3}{},
  cell{1}{5} = {c=3}{},
  cell{1}{8} = {c=3}{},
  hline{1,3-13} = {-}{},
  hline{2} = {2-11}{},
}
Methods            & Electronics    &                &                & Movies         &                &                & Beauty         &                &                \\
                   & $MRR$          & $Hit@1$        & $NDCG@5$       & $MRR$          & $Hit@1$        & $NDCG@5$       & $MRR$          & $Hit@1$        & $NDCG@5$       \\
BERT4Rec           & $0.323$        & $0.200$        & $0.319$        & $0.421$        & $0.220$        & $0.357$        & $0.341$        & $0.214$        & $0.338$        \\
MeLU               & $0.243$        & $0.136$        & $0.265$        & $0.336$        & $0.168$        & $0.302$        & $0.279$        & $0.160$        & $0.292$        \\
MAMO               & $0.296$        & $0.127$        & $0.313$        & $0.384$        & $0.194$        & $0.345$        & $0.310$        & $0.195$        & $0.336$        \\
MetaTL             & $0.320$        & $0.241$        & $0.324$        & $0.438$        & $0.319$        & $0.412$        & $0.328$        & $0.231$        & $0.335$        \\
MetaCF             & $0.330$        & $0.210$        & $0.313$        & $0.474$        & $0.276$        & $0.397$        & $0.340$        & $0.220$        & $0.322$        \\
\textbf{ClusterSeq} & $\mathbf{0.383}$ & $\mathbf{0.262}$ & $\mathbf{0.391}$ & $\mathbf{0.660}$ & $\mathbf{0.542}$ & $\mathbf{0.685}$ & $\mathbf{0.443}$ & $\mathbf{0.254}$ & $\mathbf{0.341}$ \\


\end{tblr}
\end{table*}

In this study, we evaluated the performance of ClusterSeq and state-of-the-art models under K = 3 on several datasets. The results are presented in Table \ref{table:evaltable}. The best-performing method in each column is highlighted in bold. The findings indicate that ClusterSeq outperforms the competing models in all datasets, demonstrating its effectiveness in providing accurate recommendations for cold-start users with limited interactions.

We started with basic neural models for sequential recommendations. We discovered that BERT4Rec performed poorly due to its inability to capture patterns in user interaction sequences and learn effective embeddings for cold-start users. However, utilizing transformers to extract sequential patterns proved more effective as they aggregate items with attention scores. This leads to more informative representations for users with limited interactions.

MeLU, MAMO, MetaCF, and MetaTL are meta-learning-based methods that provide cold-start recommendations. As MeLU and MAMO require side information about users and items, we used their historical interactions as side information. However, MeLU and MAMO failed to produce satisfying results, as they are designed for scenarios with abundant auxiliary user/item information, which is not the case here. On the other hand, MetaCF and MetaTL performed well in the sequential recommendation, highlighting the importance of fast adaptation in cold-start scenarios. Nevertheless, they still fell short of ClusterSeq's proposed clustering patterns for a cold-start sequential recommendation.

\subsection{Ablation Study}


We compare the proposed model with its variants and some baselines under different K values (i.e., how many interactions are initially present) to evaluate its effectiveness. Our original experiment demonstrated that BERT4Rec is the state-of-the-art sequential recommendation method, and MetaTL is one of the strongest cold-start baselines (and illustrates meta-transitional learning). Despite its high prediction power, BERT4Rec performs poorly on cold-start sequential recommendation tasks with a limited number of items. In sequential and cold-start user recommendations with different numbers of initial interactions, the proposed model can outperform state-of-the-art methods due to the well-designed optimization steps and clustering of users within the graph.

% Figure environment removed

We compare the performance of our entire model (with the clustering module) to several model variants that do not include the clustering module. We evaluate these models on a standard benchmark dataset and report the results regarding our evaluation metrics.
Figure \ref{fig:ablation_results} shows the results of the ablation study. Clearly, the entire model achieves the highest evaluation metrics, indicating that the clustering module is necessary for the model to achieve its best performance. Performance is significantly reduced when the clustering module is removed.
% \textcolor{red}{k=3 with clustering module: mrr:35.2 with clustering module:38.3}


\subsection{Parameter Analysis}
In this section, we investigate the impact of model parameters on the recommendation performance of our proposed model under cold-start scenarios. We examined how cluster number affects performance, followed by the impact of dimensions of user representations and learning rates.

\subsubsection{Number of clusters}
To study the effect of the number of clusters, we vary the number of clusters and plot the performance of the proposed method in terms of MRR in Figure \ref{fig:num_clusters}. We observed that the performance of the proposed method is generally stable for different number of clusters. In particular, we find that the proposed model is robust against this hyperparameter which is hard to estimate.

% Figure environment removed

\subsubsection{Impact of Embedding Dimensions}
Next, we explore the influence of the dimension of user embeddings on the recommendation performance of the proposed model. We vary the embedding sizes from 32 to 512 and plot the resulting performance in terms of MRR in Figure \ref{fig:embedding_dim}. Our model achieves optimal performance when the embedding dimension is set to 256. Our model is not generally stable around the optimal setting, indicating that it is important to set the embedding dimensions carefully.

% Figure environment removed


\subsubsection{Impact of Batch size}
Lastly, we analyzed the impact of batch size on the performance of our model. We conducted experiments by varying the batch size from 64 to 4096 and evaluated the resulting performance in terms of MRR, as presented in Figure \ref{fig:batch_size}. Our observations show that the optimal performance of our model is achieved when the batch size is set to 1024. Additionally, we note that the training process can converge even with smaller batch sizes like 512 or 256. Overall, this analysis highlights the importance of selecting an appropriate batch size to achieve high performance in recommendation tasks.

% Figure environment removed

To sum up, our analysis of the model parameters indicates that the recommendation performance of our model is stable for a reasonable range of hyperparameter values. However, some parameters are found to be critical for achieving optimal performance. Therefore, our findings emphasize the significance of meticulously tuning the model parameters to attain high recommendation performance in cold-start scenarios.

% Tables are considered illustrations containing data. Therefore, they should also appear floated to the top (preferably) or bottom of the page, and with the captions below them.

% \begin{table}
%     \centering
%     \begin{tabular}{lll}
%         \hline
%         Scenario  & $\delta$ & Runtime \\
%         \hline
%         Paris     & 0.1s     & 13.65ms \\
%         Paris     & 0.2s     & 0.01ms  \\
%         New York  & 0.1s     & 92.50ms \\
%         Singapore & 0.1s     & 33.33ms \\
%         Singapore & 0.2s     & 23.01ms \\
%         \hline
%     \end{tabular}
%     \caption{Latex default table}
%     \label{tab:plain}
% \end{table}

% \begin{table}
%     \centering
%     \begin{tabular}{lrr}
%         \toprule
%         Scenario  & $\delta$ (s) & Runtime (ms) \\
%         \midrule
%         Paris     & 0.1          & 13.65        \\
%                   & 0.2          & 0.01         \\
%         New York  & 0.1          & 92.50        \\
%         Singapore & 0.1          & 33.33        \\
%                   & 0.2          & 23.01        \\
%         \bottomrule
%     \end{tabular}
%     \caption{Booktabs table}
%     \label{tab:booktabs}
% \end{table}

% If you are using \LaTeX, you should use the {\tt booktabs} package, because it produces better tables than the standard ones. Compare Tables \ref{tab:plain} and~\ref{tab:booktabs}. The latter is clearly more readable for three reasons:

% \begin{enumerate}
%     \item The styling is better thanks to using the {\tt booktabs} rulers instead of the default ones.
%     \item Numeric columns are right-aligned, making it easier to compare the numbers. Make sure to also right-align the corresponding headers, and to use the same precision for all numbers.
%     \item We avoid unnecessary repetition, both between lines (no need to repeat the scenario name in this case) as well as in the content (units can be shown in the column header).
% \end{enumerate}


We proposed a machine-learning based method to approximate diagonal as well as non-diagonal elements of the Hessian of a molecule. The representation used is specific for every internal coordinates, and takes explicitly into account the bond order, which is sensible because a single point DFT calculation is computationally considerably less expensive that the explicit calculation of the Hessian.
We trained our ML model on a relatively small dataset (subset of QM7) of less than 7000 molecules. The Hessian was computed at the B3LYP/cc-pVDZ level of theory. 
The agreement between ML and DFT was satisfactory. In particular, the calculated MAPE for bond stretching force constant was below 2\%, and was particularly small for bonds involving hydrogen atoms because they point outwards and are less affected by the chemical environment. The MAPE for bending and torsion was of 5\% and 10\%, respectively. 
From the ML model trained on QM7 we were also able to predict the Hessian of some molecules representative of the QM9 dataset. The Hessian predicted in internal coordinates was then transformed into the mass-weighted Cartesian Hessian, the diagonalization of which yields the harmonic vibrational frequencies and normal modes, that can be compared to the ones calculated  explicitly from DFT.

High frequency vibrations and normal modes were predicted accurately, while lower frequency ones were not. This behaviour is analogous to the IR spectroscopy theory, where stretchings and bendings can be identified accurately, while torsion and delocalized vibrations are more difficult to be interpreted.

The approximate Hessian obtained with ML is computational inexpensive and can be used as an initial Hessian guess for geometry optimization, or in the context of alchemical geometry relaxation \cite{Domenichini2020,domenichini2022alchemical, shiraogawa2022exploration,shiraogawa2023optimization}. 
A good starting Hessian may speed up the convergence of the geometrical optimization. An in detail assessment of the performance of the ML Hessian proposed is not yet provided, but should carefully take into account many parameters on which the optimization depends, \textit{e.g.} the type of molecule, the initial geometry, the optimization algorithm, and the Hessian update scheme.




{\small
\bibliographystyle{ieee_fullname}
\bibliography{egbib}
}

\end{document}