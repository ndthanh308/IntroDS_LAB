\section{Methodology Limitation}
\label{sec:lim}
While \mywork~can enhance the ability of unsupervised trackers by leveraging uncertainty during training, the current implementation has some limitations. One of these limitations is that the uncertainty assessment is conducted offline, which is isolated from the network training process. This means that the model cannot adjust and improve in real-time during training based on the uncertainty analysis, potentially limiting its ability to optimize its performance. Moreover, this offline uncertainty assessment has led to an increase in train time, with the current implementation taking twice as long to train the network. This could be problematic in scenarios where time is a critical factor or when there are large amounts of data to process.



\section{Conclusion}
\label{sec:conclu}
This paper introduces a novel unsupervised tracking framework, \mywork, which addresses the challenging and underexplored issue of uncertainty in tracking. The proposed method improves the quality of pseudo-tracklets through an uncertain-aware tracklet-labeling strategy and enhances tracklet consistency through a tracklets-guide augmentation method that employs a hierarchical uncertainty-based sampling approach for generating hard samples. Experimental results demonstrate the effectiveness of \mywork~,  showing the potential of uncertainty. Moving forwards, we will continue to investigate a general video-related uncertainty metric and its applications in various downstream tasks.

%In this section, we would like to first discuss about the limitations of this work and then draw the conclusion.

% \textbf{Future work.} 
% We first introduce a simple uncertainty metric to evaluate the associations, where more manifestations and theoretical analysis of uncertainty estimation can be further explored. 

% In addition, \mywork~is a generalized unsupervised MOT framework to bridge the gap against supervised trackers. The throughout tracking problems (\ie, the competition of detection and re-identification) have not been specifically alleviated. 

% Surveillance security.

% \textbf{Conclusion.}

%In this paper, we present a novel unsupervised tracking framework. The \mywork~method aims to solve a challenging but underexplored problem, i.e. the inevitable uncertainty pronlem. Specially, an uncertain-aware tracklet-labeling strategy is proposed to generate precise pseudo-tracklets for temporal consistency. Second, a tracklets-guide augmentation method is introduced to further improve the consistency of tracklets which adopt a hierarchical uncertainty-based sampling for hard samples genration. Extensive experiments on several benchmarks demonstrate the effectiveness of \mywork.
%(\ie, MOT16, MOT17, MOT20, and VisDrone-MOT) 
%The tracklet-uncertainty sampling strategy is further proposed 
%This work argues that existing unsupervised methods fail to gurantee intra-tracklet consistency and inter-tracklet discriminability. The propose OurTracking aims to solve these issues

%To solve these issues, we develop a novel unsupervised MOT framework, namely \mywork.  

% uncertainty-based object association and propagation strategy to maintain the temporal consistency of pseudo-identities. 
%, to maintain the temporal consistency of pseudo-identities and construct realistic and diverse object samples to learn discriminative representations
%Uncertainty estimation is introduced to verify and rectify the associations during the inference stage of unreliable unsupervised tracking. 