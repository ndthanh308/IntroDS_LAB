\section{Related Work}
\label{sec:related}
%We introduce the most related works from three main aspects: Unsupervised Multi-Object Tracking, Uncertainty Estimation, and Augmentation Strategy.
% \subsection{Unsupervised Multi-Object Tracking}
\textbf{Pseudo-label-based Unsupervised MOT}.
Existing unsupervised methods generate pseudo-identities in three main ways, including motion-based, cluster-based and similarity-based methods. In terms of motion-based methods, SimUMOT~\cite{karthik2020simple} adopts SORT~\cite{bewley2016simple} to generate the pseudo-tracklets, which is used to guide the training of re-identification networks. Very recently, UEANet~\cite{li2022unsupervised} uses ByteTrack~\cite{zhang2022bytetrack} to improve the quality of pseudo labels, where ByteTrack excavated the values of low-confident detection boxes. 
However, long-term dependency within pseudo-tracklets are hard to guarantee, and the spatial information is not reliable in irregular camera motions. 
Cluster-based methods ~\cite{fan2018unsupervised,lin2019bottom,shuai2022id} try to iteratively cluster the objects in the whole video to get pseudo-identities. These methods usually lead to a sub-optimal performance. A possible reason is that the the temporary association within the tracklet is totally ignored.
The similarity-based methods, including Cycas~\cite{wang2020cycas} and OUTrack~\cite{liu2022online} utilize the cycle-consistency~\cite{wang2019learning} of the objects similarities between adjacent frames. As time interval extends, the noise of pseudo-label becomes a inconvenient truth. 
Different from existing methods, our \mywork~ designs an uncertainty-based refinement mechanism to obtain accurate associations. The long-term consistency is preserved through identity propagation.

%the UOAP strategy to generate accurate pseudo-tracklets, which is able to preserve the long-term consistency. 

%Similarity-based mapping. Cycas~\cite{wang2020cycas} and OUTrack~\cite{liu2022online} utilize the cycle-consistency~\cite{wang2019learning} of the objects similarities between adjacent frames. 
%However, the performance of these methods is limited by the noise of pseudo labels. 
%However, the consistency of pseudo-identities corrupts when the time interval extends.
% without the guide of reliable object embeddings. 
%2) Cluster-based identities. Other methods~\cite{fan2018unsupervised,lin2019bottom,shuai2022id} try to iteratively cluster the objects in the whole video to get pseudo-identities, with which the re-identification model is then fine-tuned. 
% The procedure repeats until convergence. 
%However, cluster algorithms usually produce poor results, and the tracklet correlation among frames has not even been exploited. 

%3) Similarity-based mapping. Cycas~\cite{wang2020cycas} and OUTrack~\cite{liu2022online} utilize the cycle-consistency~\cite{wang2019learning} of the objects similarities between adjacent frames. However, long-term dependency has been totally ignored.

% Besides, we develop an anchor-based temporal-prior augmentations strategy and a momentum dictionary mechanism to guarantee the discriminability among tracklets.

\textbf{Uncertainty Estimation}.
In recent years, uncertainty estimation has been widely explored in classification calibration (\eg, detecting misclassified or out-of-distribution samples) from three main aspects.
Some researchers adopt deterministic networks~\cite{malinin2018predictive,sensoy2018evidential,devries2018learning} or ensembles~\cite{lakshminarayanan2017simple,wen2019batchensemble} to explicitly represent the uncertainty. 
Others adopt the widely-used softmax probability distribution~\cite{hendrycks2016baseline} to evaluate the credibility according to the classification confidence. 
Very recently, energy model~\cite{liu2020energy,wang2021can} emerges as the widely-exploited metric in the uncertainty estimation, which is theoretically aligned with the probability density of the inputs. 
However, for multi-object tracking, objects occlusion and similar appearance always lead to mismatching.
Thus, the uncertain estimation is worth exploring. In this paper, we design an uncertain metric specially for tracklets-based tasks, which is proved effective. 

%We hope uncertainty estimation can draw more attention in the MOT community.
%erroneous assignments also exist in the association part of multi-object tracking, the uncertain estimation has not yet been explored. 


% Bayesian inference~\cite{gal2016dropout} and gradient information~\cite{oberdiek2018classification} can be used to estimate the model uncertainty.
%1) Some works adopt deterministic networks~\cite{malinin2018predictive,sensoy2018evidential} or auxiliary branches~\cite{devries2018learning} to explicitly represent the model uncertainty.
%2) Softmax scores and margins~\cite{hendrycks2016baseline} can also be adopted to measure classification uncertainty.
%3) Energy model~\cite{liu2020energy,wang2021can} has recently been widely exploited in estimating the OOD uncertainty.

%The association part in multi-object tracking can be viewed as a classification task. However, uncertainty estimation has not been utilized. 
%To the best knowledge, we are the first to estimate the association uncertainty and further rectify the matches. 
%We hope uncertainty estimation can draw more attention in the MOT community.

\textbf{Augmentation Strategy}.
Adaptive augmentation strategies have been extensively studied in image classification~\cite{fawzi2016adaptive,liu2021divaug}, object detection~\cite{wang2019data,ghiasi2021simple}, and representation learning~\cite{bai2022directional,zhang2022rethinking}.
However, random perspective transformation still dominates in unsupervised multi-object tracking~\cite{zhang2021fairmot,shuai2022id}. 
Other researchers present GAN-based augmentation strategies~\cite{jiang2021exploring,zhan2021spatial} for person re-identification. However, these methods fail to generate realistic object tracklets in MOT situation. 
This work integrates the tracklets property into augmentation and focus on negative hard samples generations, which makes our augmentation strategy task-specific and effective.

%assists in learning better multi-object trackers.


%the mainly-adopted augmentation strategy for self-supervised re-identification learning~\cite{zhang2021fairmot,shuai2022id}. 
% However, random transformation usually produces too similar or dramatically corrupted samples, which is neither efficient nor effective. 
%However, the expensive training cost and unstable construction results prevent GANs from practical applications. 

%In this work, we propose to leverage temporal-prior information to efficiently generate realistic and diverse augmentations. The task-specific augmentation strategy assists in learning better multi-object trackers.



% Figure environment removed