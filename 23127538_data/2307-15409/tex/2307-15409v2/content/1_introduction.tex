\section{Introduction}
\label{sec:intro}

% Figure environment removed


Multi-object tracking (MOT)~\cite{milan2016mot16,bewley2016simple,wojke2017simple} has been widely deployed in real-world applications, including surveillance analysis~\cite{oh2011large,zhu2018vision}, autonomous driving~\cite{geiger2012we,sun2020scalability}, intelligent robots~\cite{said2012real,bescos2021dynaslam}, \etc. 
% However, the label-hungry property makes it intractable in practice~\cite{karthik2020simple,wang2020cycas}. Yet a large amount of free unlabeled data remains an underused resource.
% The predominant MOT methods follow the \textit{tracking-by-detection} paradigm~\cite{wojke2017simple,bergmann2019tracking,wang2020towards,zhang2021fairmot,zhang2022bytetrack}, which includes two steps, \ie, detection and association. Those fully-supervised trackers have achieved remarkable performance ~\cite{milan2016mot16,dendorfer2020mot20}. However, the label-hungry property makes it intractable in practice~\cite{karthik2020simple,wang2020cycas}. Yet a large amount of free unlabeled data remains an underused resource.
\liuk{The goal of MOT task is to detect all target objects and simultaneously keep their respective feature embeddings \textbf{consistent}, regardless of the change of their shapes and angles over a period of time~\cite{wojke2017simple,zhang2021fairmot}.
However, the core issue of unsupervised MOT task is lacking the annotated ID-supervision to confirm the consistency of a certain target, especially when its shape and angle are varied over time~\cite{karthik2020simple,wang2020cycas,liu2022online,li2022unsupervised,shuai2022id}. 
When training an unsupervised tracker, since the learned feature embedding is unreliable, the similarity-based association stage is error-prone. Propagating pseudo identities frame-by-frame leads to uncertainty in the resulting pseudo-tracklets, which accumulates in per-frame associations. This prevents trackers from learning a consistent feature embedding~\cite{wang2020cycas,liu2022online}.
%As the pseudo identities are propagated frame-by-frame, the constructed pseudo-tracklet becomes \textbf{uncertain}. The uncertainty accumulated from frame-by-frame associations prevents trackers from learning the consistent feature embedding~\cite{wang2020cycas,liu2022online}.
}


\lk{To avoid this problem, self-supervised techniques~\cite{liu2022online,shuai2022id,zhan2021spatial} are utilized to generate augmented samples with perfectly-accurate identities. 
However, \liuk{these commonly-used methods merely take a single frame for augmentation,} while the inter-frame temporal relation is totally ignored. \liuk{It usually leads to sub-optimal performance~\cite{zhang2021fairmot}. The uncertainty problem seems inevitable but remains under-explored.}
% We observe that uncertainty is accompanied by various phenomena, including similar appearance and object occlusion and etc. Inspired by these findings, the customized metric for uncertainty provides a promising alternative, which can be leveraged to maintain consistency in turn.
In fact, we argue that the uncertainty can be leveraged to maintain consistency in turn, as shown in~\cref{fig:intro_moti}.}

%\liuk{
%In this paper, we find that the uncertainty is accompanied by two phenomenons, including small similarity margin (similar appearance \etc) and low confidence (object occlusion \etc).
%Inspired by these findings, we argue that the uncertainty can be leveraged to maintain the consistency in turn, as shown in~\cref{fig:intro_moti}.
% reducing tracklet uncertainty and preserving embedding consistency can be mutually optimized.
% }


% To make better use of widely-available unlabeled data, several unsupervised trackers~\cite{karthik2020simple,wang2020cycas,liu2022online,li2022unsupervised,shuai2022id} are proposed, involving pseudo-label-based frameworks~\cite{karthik2020simple,liu2022online} and self-supervised techniques~\cite{zhang2021fairmot,shuai2022id}. However, the performance of unsupervised methods is still far from satisfactory for practical applications. 
% We attribute the performance degradation to 
% \lk{the aggravated \textbf{uncertainty} during the object association stage.}
% % TODO: why uncertainty, fix what problem
% % TODO: why TGA
% % TODO: UTL overcome more id-switch
% % \liuk{the unreliable appearance embedding without the manually-annotated identity supervision. As the pseudo identities are generated frame-by-frame, the noise in pseudo-tracklets increases as the time intervals extend~\cite{liu2022online,li2022unsupervised}.}

% \lk{Due to the lack of manually-annotated identity supervision, the learned appearance embedding is not always reliable. \liuk{It causes the similarity-based association stage error-prone.} As the pseudo identities are generated frame-by-frame, the noise in pseudo-tracklets increases as the time intervals extend~\cite{liu2022online,li2022unsupervised}. 
% % It is hard to preserve the long-term \textbf{intra-tracklet consistency}.
% }
% \lk{To alleviate the uncertainty problem, self-supervised techniques~\cite{liu2022online,shuai2022id,zhan2021spatial} are utilized to generate augmented samples with perfectly-accurate identities. 
% % \liuk{However, the random image transformations usually produce easy-learning samples~\cite{shuai2022id}, and  }
% However, \liuk{these methods merely take a single frame for augmentation,} while the consideration of inter-frame temporal relations is totally ignored. 
% % These recent random augmentation strategies~\cite{shuai2022id} or GAN-based approaches~\cite{zhan2021spatial} are either ineffective or inefficient. 
% \liuk{In the inter-frame association, the uncertainty problem still exists.} 
% % the \textbf{inter-tracklet discriminability} is not well-maintained. 
% }

% To address the aforementioned phenomena, we propose an association-uncertainty metric that takes into account both a small similarity margin and a low confidence score. \liuk{A propagation mechanism is designed based on this metric, termed \textbf{Uncertainty-aware Tracklet-Labeling (UTL)}, which generates highly-accurate pseudo-tracklets to learn the embedding consistency.} \lk{The proposed UTL has two features: (1) it can directly boost the tracking performance during inference as well. (2) it is complementary to existing methods and can be incorporated with consistent performance.}

First, uncertainty can guide the construction of pseudo-tracklets.
When the similarity-based inter-frame object association is inaccurate, we propose to introduce a quantified uncertainty measure to find out the possibly wrong associations and further re-associate them. 
Specifically, we find that the association mismatching is accompanied by two phenomenons, including a small similarity margin (similar appearance \etc) and low confidence (object occlusion \etc).
Inspired by these findings, an \textbf{association-uncertainty} metric is proposed to filter the uncertain candidate set, which is further rectified using the tracklet appearance and motion clues. 
The proposed mechanism, termed \textbf{Uncertainty-aware Tracklet-Labeling (UTL)}, generates highly-accurate pseudo-tracklets to learn the embedding consistency.
% Not surprisingly, we find that UTL can directly boost the tracking performance during inference as well.
The proposed UTL has two features: (1) it can directly boost the tracking performance during inference as well. (2) it is complementary to existing methods and can be incorporated with consistent performance.

Second, uncertainty can guide the hard sample augmentation.
Trustworthy pseudo-tracklets can incorporate temporal information into sample augmentation, thereby overcoming the key limitation of current augmentation-based methods. 
To this end, a \textbf{Tracklet-Guided Augmentation (TGA)} strategy is developed to simulate the real motion of pseudo-tracklets. 
TGA generates augmentation samples aligned to the highly-uncertain objects in the pseudo-tracklet for hard example mining, as a high association-uncertainty basically indicates the presence of challenging negative examples.
We specifically develop a hierarchical uncertainty-based sampling mechanism to ensure trustworthy pseudo-tracklets and hard sample augmentations.

% \liuk{Specifically, we develop a \textbf{Tracklet-Guided Augmentation (TGA)} strategy, which is a training-free way with negligible computational cost. TGA is designed to simulate the real motion of pseudo-tracklets in a training-free way with negligible computational cost, which only involves displacement mapping from the source to the target frame. The tracklet-level uncertainty is proposed in the selection of the target frame. The proposed TGA method prefers the frames with high tracklet-level uncertainty, which indicates the presence of challenging negative examples.  }

%to generate samples to simulate the tracklet's motion. 

%The TGA algorithm typically selects the target frame based on high association uncertainty, which indicates the presence of challenging negative examples.
% We further develop a \textbf{Tracklet-Guided Augmentation (TGA)} strategy to enrich the discriminability among tracklets. 
%\lk{TGA is designed to simulate the real motion of pseudo-tracklets in a training-free way with negligible computational cost, which only involves displacement mapping from the source to the target frame.} \lk{First, the real motion should be guaranteed, where a \textbf{tracklet-uncertainty} metric is developed to evaluate the pseudo-tracklet accuracy.
%Then, to further improve the discriminability, TGA tends to choose the target frame with high association uncertainty, which basically means the existence of hard negative examples.}
%\lk{With the hierarchical uncertainty-based sampling mechanism, the proposed TGA is concurrently effective and efficient to enhance the inter-tracklet discriminability.}

The ultimate \mywork~framework is evaluated on MOT17~\cite{milan2016mot16}, MOT20~\cite{dendorfer2020mot20}, and the challenging VisDrone-MOT~\cite{zhu2021visdrone} benchmarks.
The experiments show that \mywork~significantly outperforms previous unsupervised methods (\eg, 62.7\% $v.s.$ 58.6\% of HOTA on MOT20), and achieves SOTA (\eg, 64.2\% HOTA on MOT17) among existing unsupervised and supervised trackers.
\lk{Extensive ablation studies demonstrate the effectiveness of leveraging uncertainty in improving the consistency in turn.}

Contributions of this paper are summarized as follows:
\begin{enumerate}[1)]
    \setlength{\itemsep}{0pt}
    \item \lk{We are the first to leverage uncertainty} in unsupervised multi-object tracking, where an association-level uncertain metric is introduced to verify the pseudo-tracklets, and a hierarchical uncertainty-based sampling mechanism is developed for hard sample generation.  
    \item We propose a novel unsupervised \mywork~framework, where UTL is developed to guarantee the intra-tracklet consistency and TGA is adopted to learn the consistent feature embedding.
    \item \lk{We achieve a SOTA tracking performance among existing methods, and demonstrate the generalized application prospects for the uncertainty metric.} 
\end{enumerate}