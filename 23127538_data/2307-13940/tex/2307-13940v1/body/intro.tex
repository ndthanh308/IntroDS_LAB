Topological shape analysis seeks to summarize information about the shape of
data via topological invariants, which can then be used in subsequent analysis
tasks like classification or regression.  A classical topological invariant is the \emph{Euler Characteristic} (EC), which can easily be computed from data.
A popular extension of the EC for a filtered topological space is the
\emph{Euler characteristic function} (\fec),\footnote{
In some sources, this is referred to as the \emph{Euler characteristic curve},
but it is not a curve (as it is a step function).}
which tracks the EC as the filtration parameter changes.
When considering the shape of data embedded in $\R^d$, the shape is often filtered in different directions such as with the \emph{Euler Characteristic Transform} (ECT) and the \emph{Persistent Homology Transform} (PHT) \cite{turner2014persistent,ghrist2018persistent}.
These transforms have been used in a variety of settings such as
biology \cite{turner2014persistent,amezquita2022measuring}, oncology~\cite{crawford2020predicting}, and organoids \cite{marsh2022detecting}.
A recent generalization of the ECT that allows for a weighted simplicial complex
is the \emph{weighted Euler characterisitc transform}~(\wect)~\cite{jiang2020weighted}.
Jiang, Kurtek, and Needham~\cite{jiang2020weighted} extend work
of~\cite{turner2014persistent,ghrist2018persistent} and show that the \wect uniquely represents weighted simplicial complexes.
%We consider the case where we now have $k$ weight functions, one for each color channel.
The motivation of \cite{jiang2020weighted} is image data where the intensity of a pixel is used to assign the weights for the weighted simplicial complex.
We investigate the effectiveness of the \wect at discriminating shapes with
weights sampled from different distributions, and find the expected weighted EC
and the expected \wect of images under different weight distributions.

The effectiveness of the \wect at discriminating different types of images was
demonstrated in \cite{jiang2020weighted} where they considered the MNIST
handwritten digit dataset \cite{LeCun:1998aa} and magnetic resonance images of
malignant brain tumors (Glioblastoma Multiforme).  In the former example, the
\wect-based classification model outperformed classification models that used
either the image directly or the (unweighted) ECT in terms of ten-fold
cross-validation classification rate.
In this paper, we more generally explore the performance of the \wect when the
pixel intensities are sampled from different distributions.  In addition to
evaluating the performance of the \wect in a variety of settings, we improve the
interpretability of the \wect by estimating the \emph{expected} weighted EC and
the \emph{expected} \wect.
