While the ECT has been successful at summarizing shapes in a variety of contexts, image data poses a unique challenge because the pixel intensities potentially carry significant and relevant information.  The WECT was proposed to generalize the ECT and allow for the incorporation of pixel intensities as weights.  However, the intuition and interpretation of the WECT is not as clear.
%
In this paper, we explored using the \wect to represent shapes found in images, and developed an understanding of the importance of the weights.  Indeed, we found that the \wect captured more than the EC and
the ECF alone.  
This was especially important when images have the same shape, but different intensity distributions as assessed in our empirical study.  As the intensity distributions become more similar between two classes that have the same shape, the test classification accuracy decreases.
When the two classes had different image shapes, the \wect correctly classified new images perfectly (or almost perfectly with test classification accuracy greater than 0.99).  While we limited our classification models to SVM, and explored KNN models, \wects can be used in other classification models.

The effect on changing inputs to the \wect was also explored.  Two function
extensions, the maximum and average function extensions, were considered.
Though the maximum function extension outperformed the average function
extension in our settings, this was likely dependent on the pixel intensity
distributions considered.  In particular, the four distributions ($U(0,1)$,
$N(0.5,0.17)$,~$N(0.5,0.5)$, and~$N(0.5,5)$; all truncated to $(0,1]$) had
averages of $0.5$ so differences were near the extremes (near zero or one).
Therefore, the maximum function extension would amplify these differences better
than the~average.  

We also explored how the test classification accuracy changed with different
numbers of directions considered.  Shapes with rotational symmetry did not
benefit from increasing the number of directions, while those without rotational
symmetry did benefit.  How to appropriately select the number of directions in
this setting is an open question that will be explored in future~work.

Though classification was explored in this paper, the \wect could also be used for inference tasks such as hypothesis testing.  In light of this, we developed equations for the expected weighted EC and expected WECT for images, which may be useful for developing test statistics in future studies.
%
While we have found a clear benefit in using the \wect for image classification when the pixel intensity distributions differ, questions remain regarding its advantage for inference or for different data types.  
