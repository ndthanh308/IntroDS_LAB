Since academic language differs from everyday social language and is an essential acquisition target in current education, extracting salient features contributes to linguistic, register, genre and disciplinary feature identification that can benefit students, teachers and educational app developers alike. Compiling an Academic Word List (AWL) is an effective solution to support both language teaching and NLP tasks. From the didactic perspective, AWLs reflecting either the L1 (i.e. mother tongue) or the L2 (i.e. foreign language) academic vocabulary can be used to offer linguistic support to novice academic writers in the form of discipline-specific and general lexical prompts. Teachers of all disciplines can integrate AWLs into teaching materials to help students write better (see, for example, \citet{wangdi2022investigating}).

NLP studies can exploit AWL datasets on topics such as text classification \cite{zampieri2012evaluating} and topic modelling \cite{murakami2017corpus}. For example, field-specific academic lists can be used to automatically classify texts into disciplinary areas. The same can be applied for the automatic distribution of texts in academic versus non-academic batches. In machine learning methods for language modelling tasks, AWLs are essential in training models to generate accurate academic writing samples. By combining NLP tasks with linguistic approaches in relation to AWLs, important advances can be achieved in the frame of lexical and syntactic analyses that evaluate the use of collocations and phraseology specific to the academic varieties. For the Romanian language, there have been few attempts to extract a valid Romanian Word List \cite{szabo2015introducing} and only one study has extracted and analysed multiword units using academic writing corpora \cite{muresan2022phraseology}. 

In recent years, researchers have worked to create several academic writing corpora. EXPRES – Corpus of Expert Writing in Romanian and English \cite{chitez2022expres} is one of them. It is the only bilingual multidisciplinary corpus capturing the Romanian academic writing context. By combining datasets representing the Romanian Frequency List \cite{szabo2015introducing} based on the ROMBAC Corpus \cite{ion2012rombac}, and EXPRES disciplinary datasets \cite{chitez2022expres}, we were able to generate an empirically based Romanian Academic Word List. Ro-AWL is made publicly available\footnote{\url{https://github.com/bucuram/Ro-AWL}} and can be used for teaching, text classification and language modelling.