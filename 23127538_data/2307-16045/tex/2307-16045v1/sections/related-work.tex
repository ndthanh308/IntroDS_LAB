Most academic vocabulary lists have been developed in the context of English for Academic Purposes (EAP). On the whole, two categories of lists exist. One list type aims to identify academic words commonly used in EAP across disciplines, which students could be made aware of. The studies aiming to provide cross-disciplinary academic word lists usually use large corpora containing expert academic writing from various disciplines. The widely used lists of this type are the Academic Word List (AWL) \cite{coxhead2000new} and the Academic Vocabulary List (AVL) \cite{gardner2014new}. The second type of list seeks to identify discipline or field-specific words worth teaching. Various specialised lists have been developed for fields such as veterinary medicine \cite{ohashi2020esp} or nursing \cite{yang2015nursing}.

While there is a growing interest in building cross-disciplinary academic word lists for languages other than English, these academic word lists remain few. See, for example studies conducted for French \cite{cobb2004there}, Persian \cite{rezvanifirst}, Portuguese \cite{baptista2010p}, Swedish \cite{carlund2012academic}, and Norwegian \cite{johannessen2016constructing}. An explanation for this scarcity might be that academic language data sets are rare and often not freely available due to copyright. This can be especially true for low-resource languages, such as Romanian. Access to a representative corpus is crucial, as the validity and reliability of an academic word list highly depend on the quality of the data set. 

Apart from the limited availability of academic writing corpora, an additional challenge may be that there is no standard procedure for extracting academic word lists. Scholars are still exploring and testing various methodologies. For example, some studies build on the methods used for the AWL or the AVL \cite{johannessen2016constructing,rezvanifirst}. One study uses the translated version of the AVL in Portuguese as a starting point for its investigation \cite{baptista2010p}. Another study proposes a new word list extraction method different from previous ones \cite{carlund2012academic}.  

In the case of Romanian, no previous studies have compiled specialised or general academic word lists. Although in the last 10-15 years, several research institutions and projects have been involved in developing corpus resources in Romanian, relatively few have focused exclusively on general academic writing. Some of the most significant corpora recently compiled, such as ROMBAC (Romanian Balanced Annotated Corpus, see \citet{ion2012rombac}), with more than 30 million words, CoRoLa (Corpus of Contemporary Romanian Language, see \citet{mititelu2014corola}), or The Balanced Romanian Corpus (BRC, see \citet{midrigan2020resources}) cover only few disciplines or subsets: 5 sections for ROMBAC (journalism, literature, medical texts, legal texts, biographies), uneven and unfiltered distribution of resources in CoRoLa (the collection of academic writing texts is based on agreements with publishing houses and journals, without filtering of the content on quality criteria) and BRC (literary text samples, research articles, news, spoken data). The ROMBAC corpus (excluding the medical subcorpus) was already used to develop the Romanian Word List (RWL, see \citet{szabo2015introducing}), targeted at Romanian L2 learners (e.g. from the Hungarian minority in Romania). The list is a general list of words, not focused on academic language. As far as discipline-specific corpora are concerned, smaller corpora such as SiMoNERo (medical corpus, \citet{mitrofan2019monero}), BioRo \cite{mitrofan2018bioro}, PARSEME-Ro (news articles), LegalNERo (legal, \citet{paiș2021named}), MARCELL (legal, multilingual, see \citet{varadi2020marcell}), CURLICAT (multilingual, containing several domains: Economics, Education, Health, Sciences, etc., see \citet{varadi2022introducing}) have been compiled. However, apart from compiling the datasets and conducting a series of descriptive studies, no special attention is given to the lexical level. 

In this context, the EXPRES corpus (Corpus of Expert Writing in Romanian and English) is the first corpus of discipline-specific academic writing in the Romanian context (academic writing in Romanian L1 and academic writing in English L2 produced by Romanians) \cite{bucur2022expres,chitez2022write}. Covering four disciplines – Linguistics, Economics, Political Sciences, Information Technology –, the Romanian subset contains 200 open-access research articles from each domain, published in the past 5-10 years in peer-reviewed journals (see \citet{chitez2022expres}). The rigorous selection criteria \cite{rogobete2021challenges} contribute to the representativeness of the corpus, making it a suitable candidate for testing a possible Romanian Word List and narrowing it down to an Academic Word List. Furthermore, the EXPRES corpus is the first Romanian expert academic corpus available on an open-access query platform. Unlike other Romanian corpora, which offer limited access to third parties and poor resources for downloading search results or statistics, the EXPRES corpus support platform has been implemented as a cross-platform distributed web application  \cite{chitez2022expres}.
