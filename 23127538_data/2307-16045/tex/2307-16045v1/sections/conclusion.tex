This study reports the extraction of the first Romanian Academic Word List (Ro-AWL), which can be used to check the degree of academic vocabulary coverage in discipline-specific and general language samples. Ro-AWL consists of 673 lemmas, distributed among the main part-of-speech categories (nouns, verbs, adverbs, adjectives). Our methodology adopted measures used for the Academic Vocabulary List for the English language, such as ratio, range, dispersion and discipline measures. The percentages calculated by testing Ro-AWL on the disciplinary datasets in the EXPRES corpus \cite{chitez2022expres}, indicate a lower coverage for Linguistics (11.82\%) and Political Sciences (13.17\%) and a higher coverage for Information Technology (17.03\%) and Economics (17.75\%). Also, the academic vocabulary coverage in ROMBAC, a general language reference corpus, is 6.73\%, while the coverage is much higher (15.25\%) in EXPRES, an expert academic writing corpus. This aligns with previous research, since Ro-AWL coverage is similar to thresholds for academic vocabulary \cite{nation2001learning}.


Despite several computation constraints (e.g. Romanian POS tagger not being able to distinguish between adjectives and adverbs), our study provides important insights into the academic writing vocabulary in Romanian by proposing a validated Romanian Academic Word List. Our findings also have pedagogical implications, as the list can be used to support academic writing teaching activities and NLP tasks focusing on Romanian. For example, the Ro-AWL can be paired up with the freely available EXPRES corpus platform to develop corpus-assisted learning activities commonly known as Data-Driven Learning (DDL) (see e.g., \citet{bennett2010using}). However, even if the coverage test results in the EXPRES are encouraging, further research is needed to test the validity of the Ro-AWL on corpora containing academic writing from more disciplines. Future work can be conducted in at least two directions: refining the lists from a contrastive perspective, by developing a discipline-specific AWL, or, on the contrary, by searching for highly frequent academic words present in an extended corpus containing more disciplines.

