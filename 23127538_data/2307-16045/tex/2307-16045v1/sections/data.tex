This work uses two different corpora: the academic corpus EXPRES and the Romanian Academic Word List \cite{szabo2015introducing} compiled from the general corpus ROMBAC. The Romanian language sub-corpus of EXPRES\footnote{\url{https://expres-corpus.org/}} \cite{chitez2022expres} consists of 800 research articles, 200 articles for each of the four fields: Linguistics (LG), Economics (EC), Information Technology (IT) and Political Sciences (PS). The articles from the corpus were manually processed to preserve the anonymity of the authors (e.g., the name of the authors were replaced with {AUTHOR\_NAME}) and the beginning and end of the title, abstract and sections are annotated with corresponding XML tags (e.g., $<$TITLE$>$, $<$/TITLE$>$) \cite{chitez2022expres}. Table \ref{tab:expres} shows the distribution of words in EXPRES, without counting the manually added tags. The corpus contains more than 3 million words, with more than 200 thousand unique words.

\begin{table}[hbt!]
    \centering
    \caption{EXPRES Statistics}
    \resizebox{0.6\linewidth}{!}{
    \begin{tabular}{l|ll}
    \textbf{Domain}  & \textbf{Tokens} & \textbf{Types}\\
    \hline
    \textbf{EC} & 1,092,846 & 48,807 \\
    \textbf{LG} & 674,277 & 73,667 \\
    \textbf{IT} & 750,236 & 40,494 \\
    \textbf{PS} & 963,061 & 62,096 \\
    \hline
    \textbf{Total} & 3,480,420 & 225,064\\
    \end{tabular}
    }
    \label{tab:expres}
\end{table}

The Romanian Academic Word List \cite{szabo2015introducing} contains a frequency list for all the words in the Romanian Balanced Annotated Corpus (ROMBAC) \cite{ion2012rombac}. ROMBAC \cite{ion2012rombac} is a large general collection of texts from the Romanian language. It contains texts from five domains: news, medical, legal, biographies and fiction. The texts from ROMBAC are tokenized and lemmatized. The version we use in this paper contains more than 25 million lemmas, of which 1 million are unique (Table \ref{tab:rombac}). The dataset was previously used to derive other linguistic resources, such as the Romanian Word List and Romanian Vocabulary Levels Test \cite{szabo2015introducing}. We use the ROMBAC corpus in our work because it is the largest corpus available in Romanian that was not web-scraped, and it is a reference corpus for the contemporary Romanian language \cite{ion2012rombac}. Even if another larger corpus for the contemporary Romanian language exists, namely CoRoLa \cite{mititelu2014corola}, it is not publicly available and cannot be downloaded; it can only be queried online\footnote{\url{https://korap.racai.ro/}}. The other reference corpus recently compiled, BRC \cite{midrigan2020resources}, was not an option either, since its size is smaller than ROMBAC and lacks disciplinary variation.


\begin{table}[hbt!]
    \centering
    \caption{ROMBAC Statistics}
    \resizebox{0.7\linewidth}{!}{
    \begin{tabular}{l|ll}
    \textbf{Domain}  & \textbf{Tokens} & \textbf{Types}\\
    \hline
    \textbf{News} & 1,922,109 & 50,945 \\
    \textbf{Medical} & 6,783,005 & 362,782 \\
    \textbf{Legal} & 6,269,543 & 248,354 \\
    \textbf{Biographies} & 3,716,031 & 223,592 \\
    \textbf{Fiction} & 6,950,371 & 105,346 \\ 
    \hline
    \textbf{Total} & 25,641,059 & 991,019 \\ 
    \end{tabular}
    }
    \label{tab:rombac}
\end{table}