A first observation concerns the different coverages of Ro-AWL in the EXPRES corpus (see Table \ref{tab:coverage}). The lower percentages in Linguistics and Political Sciences (with a total coverage ranging between 11\% and 14\%) and the higher ones in Economics and IT confirm that “The SSH community is characterised by the embedment of research in the local context and by linguistic diversity in producing and disseminating knowledge” \cite{kancewicz2020does}. Researchers in the Romanian context in SSH (Social Sciences and Humanities) tend to favour a more “creative” dimension of the language used in academic writing, using figurative language in constructing rhetorical structures. Although in English language academic writing “the dichotomy of soft and hard sciences is rather fluid and as such insignificant” \cite{stotesbury2003evaluation}, discipline-specific peer-review practice in the Romanian setting seems to influence the academic writing style. This is particularly visible in the EXPRES subset of Political Sciences and Linguistics. Romanian academic writing in SSH seems rather unfocused, descriptive and rich in rhetorical structures.
In contrast, research articles in Economics and Information Technology contain many statistics, tables, and formulas, making the writing in the discipline less descriptive. 
 

Secondly, although our extraction measures were successful in filtering most of the technical vocabulary, small amount of technical language remains in the Ro-AWL (terms such as “dauna”, En: “damage” - in contexts related to insurances; “institutional”, “security”, “electronic” etc.). Nevertheless, the majority of the Ro-AWL components are discipline neutral, thus contributing to academic discourse cohesion and coherence.

Thirdly, a technical challenge regarding the functionality and accuracy of the Romanian POS tagger should be mentioned. An overview of the assigned tags revealed the difficulty of the tagger to distinguish between adjectives and adverbs (for instance: “important”, “social”, “european” were assigned as adverbs, but the contexts prove their prevalent use as adjectives). It also confused past participles ending with “-t” (e.g. “accentuat”, En: “emphasised”. This technical difficulty can be observed in Table \ref{tab:coverage}, with the coverage of adverbs being higher than the one of adjectives, because most of the adjectives had the part-of-speech mislabeled by the POS tagger. These errors of the POS tagger are due to the homonymy between the two POS, most adverbs being homonymous to their adjective counterparts \cite{vasile2017properties}.

A technical advantage of the Romanian POS tagger, however, is its capacity to recognise nouns with a definite article while being a part of prepositional phrases (“în pofida”, En: “despite”, “în jurul”, En: “around”). This also explains the increased percentage levels of nouns, adverbs and verbs and the lower values for adjectives (see Figure \ref{fig:pos}). 

Despite some of the technical challenges, the extraction of the Romanian AWL using the EXPRES corpus resulted in successfully identifying the recurrent discourse conventions used by Romanian researchers. During the process and alongside the extraction procedure per se, translating the Academic Vocabulary List (AVL) \cite{gardner2014new} was a helpful procedure, as it is well accepted that academic writing, irrespective of the language, contains a large number of words of Greek and Latin origin (see e.g., \citet{rasinski2008greek,green2020greek}). 
