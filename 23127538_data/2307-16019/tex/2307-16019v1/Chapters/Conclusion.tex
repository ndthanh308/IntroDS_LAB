Building upon principles from the recent ZSL literature \cite{APN} and incorporating them within a NeSy framework \cite{Martone2022PROTOtypicalLT}, we introduced a novel NeSy architecture, named FLVN, for ZSL and GZSL tasks. FLVN incorporates axioms that combine prior class-specific knowledge (e.g., class hierarchies), with high-level inductive biases to handle exceptions within the dataset (e.g., ``there exists a zebra that is not agile'') and establish relationships between images (e.g., ``if two images belong to the same class, they must be similar''). Such axioms act as semantic priors and compensate for the lack of annotations, thereby providing a solid NeSy foundation for GZSL tasks. FLVN does not require multiple backbones and maintains roughly the same parameter count of standard embedding-based methods. The proposed approach can be also incorporated into other architectures by, e.g., changing the grounding of the predicates. Experimental results prove that FLVN achieves performance on par or exceeding that of current literature on common GZSL benchmarks.  

We believe that our approach can be further extended in several ways. For example, different formulations of the $\texttt{isOfClass}$ predicate, or the introduction of a \texttt{hasAttribute} predicate to predict image-level attributes, could incorporate attention to discriminative regions within the image and increase explainability. Introducing attention mechanisms based on object detection~\cite{manigrasso2021faster} could support the definition useful axioms that identify the most discriminative parts of the image. Introducing axioms that solely consider the similarity between images (with no knowledge of their class membership) could improve results for unseen classes without requiring additional labels in the dataset. The knowledge base could be also extended to consider other types of relationships, beyond class hierarchies, and from emerging sources, such as language models.  Finally, the proposed method could be explored in a trasductive ZSL setting. 

