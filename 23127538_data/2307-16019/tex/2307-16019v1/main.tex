% This is samplepaper.tex, a sample chapter demonstrating the
% LLNCS macro package for Springer Computer Science proceedings;
% Version 2.21 of 2022/01/12
%
\pdfoutput=1 
\documentclass[runningheads]{llncs}
%
\usepackage[T1]{fontenc}
% T1 fonts will be used to generate the final print and online PDFs,
% so please use T1 fonts in your manuscript whenever possible.
% Other font encondings may result in incorrect characters.
%
\usepackage{graphicx}
%\usepackage[finalizecache,cachedir=minted-cache]{minted} 
\usepackage{amsmath}
\usepackage{eucal}
\usepackage{amsfonts}
\usepackage{booktabs}
\usepackage{multirow}
\usepackage{textcomp}
\usepackage{amssymb}
\usepackage{graphicx}
\usepackage{bibnames}
\usepackage{breqn}
\usepackage{csquotes}
\usepackage{siunitx}
\usepackage{url}
% Used for displaying a sample figure. If possible, figure files should
% be included in EPS format.
%
% If you use the hyperref package, please un the following two lines
% to display URLs in blue roman font according to Springer's eBook style:
\usepackage{color}
\renewcommand\UrlFont{\color{blue}\rmfamily}
%
\begin{document}
%\newcommand{\FM}[1]{\textcolor{blue}{\textsf{\textbf{FM:}~#1}}}
%
%\newcommand{\LM}[1]{\textcolor{red}{\textsf{\textbf{LM:}~#1}}}
%
\title{Fuzzy Logic Visual Network (FLVN): A neuro-symbolic approach for visual features matching }
%
\titlerunning{FLVN:  A neuro-symbolic approach for visual features matching}
% If the paper title is too long for the running head, you can set
% an abbreviated paper title here
%




\author{Francesco Manigrasso\inst{1}\orcidID{0000-0002-4151-8880} \and
Lia Morra\inst{1}\orcidID{0000-0003-2122-7178} \and
Fabrizio Lamberti\inst{1}\orcidID{0000-0001-7703-1372}}
%
\authorrunning{F. Manigrasso et al.}
% First names are abbreviated in the running head.
% If there are more than two authors, 'et al.' is used.
%
\institute{Politecnico di Torino, Dipartimento di Automatica e Informatica, Torino, Italy\\
\email\{francesco.manigrasso, lia.morra, fabrizio.lamberti\}@polito.it}

%
\maketitle              % typeset the header of the contribution
%
\begin{abstract}
Neuro-symbolic integration aims at harnessing the power of symbolic knowledge representation combined with the learning capabilities of deep neural networks. In particular, Logic Tensor Networks (LTNs) allow to incorporate background knowledge in the form of logical axioms by grounding a first order logic language as differentiable operations between real tensors. Yet, few studies have investigated the potential benefits of this approach to improve zero-shot learning (ZSL) classification. In this study, we present the Fuzzy Logic Visual Network (FLVN) that formulates the task of learning a visual-semantic embedding space within a neuro-symbolic LTN framework. FLVN incorporates prior knowledge in the form of class hierarchies (classes and macro-classes) along with robust high-level inductive biases. The latter allow, for instance, to handle exceptions in class-level attributes, and to enforce similarity between images of the same class, preventing premature overfitting to seen classes and improving overall performance.  FLVN reaches state of the art performance on the Generalized ZSL (GZSL) benchmarks AWA2 and CUB, improving by 1.3\% and 3\%, respectively. Overall, it achieves competitive performance to recent ZSL methods with less computational overhead.
FLVN is available at
\url{https://gitlab.com/grains2/flvn}.

\keywords{Zero shot learning  \and NeuroSymbolic AI  \and Logic Tensor Networks}
\end{abstract}
%
%

\makeatletter
\let\oldabs\abs
\def\abs{\@ifstar{\oldabs}{\oldabs*}}
%
\let\oldnorm\norm
\def\norm{\@ifstar{\oldnorm}{\oldnorm*}}
\makeatother

\newcommand{\XX}{\mathcal{X}}
\newcommand{\CC}{\mathcal{C}}
\newcommand{\FF}{\mathcal{F}}
\newcommand{\PP}{\mathcal{P}}
\newcommand{\GG}{\mathcal{G}}
\newcommand{\DD}{\mathcal{D}}
\newcommand{\LL}{\mathcal{L}}
\newcommand{\KK}{\mathcal{K}}
\newcommand{\prob}{\mathbb{P}}
\newcommand{\reals}{\mathbb{R}}
\newcommand{\Din}{D_{\mathrm{in}}}
\newcommand{\Dout}{D_{\mathrm{out}}}
\newcommand{\naturals}{\mathbb{N}}
\newcommand{\sigmoid}{\mathrm{sigmoid}}
\newcommand{\fuz}{\mathrm{FuzzyOp}}
\newcommand{\satagg}{\mathrm{SatAgg}}
\newcommand{\argmaxx}{\mathrm{argmax}}
\newcommand{\argminn}{\mathrm{argmin}}

\section{Introduction}
The problem of the presence or absence of phase transition is central in statistical mechanics. To prove the existence of phase transition, the standard idea is to define a notion of contour and use \textit{Peierls' argument} \cite{Peierls.1936}. In the usual Ising model \cite{Ising_25}, particles of the system interact only with their nearest-neighbors. On ferromagnetic long-range Ising models \cite{Anderson_Yuval_69}, there is interaction between each pair of spins in the lattice. The Hamiltonian of the model is given formally by
\begin{equation*}
    H(\sigma) = - \sum_{x,y\in \Z^d}J_{xy}\sigma_x\sigma_y,
\end{equation*}
where $J_{xy}=J|x-y|^{-\alpha}$, $J>0$, $\alpha > d$. It is well-known that the Peierls' argument in dimension 2 implies phase transition for Ising models with nearest-neighbors or long-range interactions when $d\geq 2$, using correlation inequalities. For the unidimensional lattice, it was known that short-range models do not present phase transition. In the long-range case, a different behavior was expected depending on the exponent $\alpha$ (see \cite{Kac_Thompson_69}), but the problem was challenging since contours were first created as multidimensional objects.

In dimension $d=1$, phase transition was proved first in 1969 by Dyson \cite{Dyson.69}, for $\alpha \in (1,2)$, by proving phase transition in an auxiliary model and then using correlation inequalities. In 1982, Fr{\"o}hlich and Spencer \cite{Frohlich.Spencer.82} introduced a notion of one-dimensional contours and then applied the Peierls' argument to show phase transition for the critical value $\alpha = 2$. These contours were inspired by the multiscale techniques previously introduced to study the Berezinskii-Kosterlitz-Thouless transition in two-dimensional continuous spin systems \cite{FS81}. Later, Cassandro, Ferrari, Merola and Presutti  \cite{Cassandro.05} extended the contour argument previously available for $\alpha=2$ to exponents $\alpha\in (3-\frac{\ln 3}{\ln 2}, 2)$, with the additional restriction that the nearest-neighbor interaction is strong, i.e.,  ${J(1)\gg 1}$; this restriction was removed for a subclass of interactions in \cite{Bissacot.Endo.18}. Further results were obtained using contour arguments, such as the decay of correlations, cluster expansions, phase transition with random interactions, etc; some references with these results are \cite{ Cassandro.Merola.Picco.17, Cassandro.Merola.Picco.Rozikov.14, Imbrie.82, Imbrie.Newman.88, Johansson.91}. 

In the multidimensional setting ($d\geq 2$), Ginibre, Grossmann, and Ruelle, in \cite{Ginibre.Grossmann.Ruelle.66}, proved the phase transition for $\alpha > d+1$, using an enhanced version of Peierls' argument and the usual contours. Park proposed a different notion of contour for long-range systems in \cite{Park.88.I, Park.88.II}, extending the Pirogov-Sinai theory available for short-range interactions assuming $\alpha > 3d+1$, although he can also consider Potts models with his methods. Some results in the literature suggest that truly long-range effects appear only when $d < \alpha \leq d+1$, see for instance, \cite{Biskup_Chayes_Kivelson_07}. Recently, Affonso, Bissacot, Endo and Handa \cite{Affonso.2021}, inspired by the ideas from Fr{\"o}hlich and Spencer in \cite{FS81, Frohlich.Spencer.82}, introduced a version of multiscale multidimensional contour and proved phase transition by a contour argument in the whole region $\alpha > d$. They can consider long-range Ising models with deterministic decaying fields, first introduced in the context of nearest-neighbor interactions in \cite{Bissacot_Cioletti_10}. For these models, the lack of analyticity of the free energy does not imply phase transition since these models have the same free energy as the models with zero field. It is expected that fields decaying slowly imply uniqueness. In this setting, a contour argument is useful for proofs of phase transitions as well for uniqueness, some papers with models with deterministic decaying fields are \cite{Aoun_Ott_Velenik_23, Bissacot_Cass_Cio_Pres_15, Bissacot.Endo.18, Cioletti_Vila_2016}.

The Random Field Ising model (RFIM) \cite{Imry.Ma.75} is the nearest-neighbor Ising model with an additional external field acting on each site $(h_x)_{x\in\Z^d}$ that is a family of i.i.d. Gaussian random variable with mean 0 and variance 1. Formally, the Hamiltonian of the model is given by
\begin{equation*}
    H(\sigma) = - \sum_{\substack{x,y\in \Z^d \\|x-y|=1}}J\sigma_x\sigma_y  - \varepsilon\sum_{x\in\Z^d}h_x\sigma_x,
\end{equation*}
where $J>0$, $\varepsilon>0$, $\alpha > d$ and $d \geq 1$. A detailed account of the history of the phase transition problem for this model, as well as detailed proofs, was given in \cite{Bovier.06}. Here we present a brief overview.

During the 1980s, the question of the specific dimension where phase transition for the RFIM should happen attracted much attention and was a topic of heated debate. Two convincing arguments were dividing the physics community. One of them, due to Imry and Ma \cite{Imry.Ma.75}, was a non-rigorous application of the Peierls' argument together with the use of the isoperimetric inequality. The key idea of Peierls' argument is to define a notion of contour and calculate the energy cost of "erasing" each contour, i.e., the energy cost of flipping all spins inside the contour. When there is no external field, that energy necessary to flip the spins in a region $A\subset \Z^d$ is of the order of the boundary $|\partial A|$. When we add an external field, we get an extra cost depending on this field. Imry and Ma argued that this cost should be approximately $\sqrt{|A|}$, which is smaller than $|\partial A|$ for all regions only when $d\geq 3$, so this should be the region where phase transition occurs. The other argument, due to Parisi and Sourlas \cite{Parisi.Sourlas.79}, based on dimensional reduction, predicted that the $d$-dimensional RFIM would behave like the $d-2$-dimensional nearest-neighbor Ising model, therefore presenting phase transition only when $d\geq 4$. 

The question was settled by two celebrated papers showing that Imry and Ma's prediction was correct. First, in 1988, Bricmont and Kupiainen \cite{Bricmont.Kupiainen.88} showed that there is phase transition almost surely in $d\geq3$, for low temperatures and variance $\varepsilon$ small enough. Their proof uses a rigorous renormalization group analysis for the short-range case and it is considered involved. Still, they claimed that the result works for any model with a suitable contour representation and centered sub-gaussian external field. Later on, Aizenman and Wehr \cite{Aizenman.Wehr.90} proved uniqueness for $d\leq 2$. For detailed proofs of these results, we refer the reader to \cite{Bovier.06} (see also \cite{Berretti.85, Camia.18, Frohlich.Imbre.84,  Klein.Masooman.97} for more uniqueness results). 

Recently, Ding and Zhuang, see \cite{Ding2021}, provided a simpler proof of the phase transition, not using RGM. And in  \cite{Ding.Liu.Xia.22}, Ding, Liu and Xia proved that if $\beta_c(d)$ is the critical inverse of the temperature of the Ising model with no field, for all $\beta>\beta_c(d)$ there exists a critical value $\varepsilon_0(d, \beta)$ such that the RFIM with $\varepsilon \leq \varepsilon_0$ presents phase transition. 

In the present paper, we are considering a long-range Ising model with a random field, whose Hamiltonian is given formally by
\begin{equation*}
    H(\sigma) = - \sum_{x,y\in \Z^d}J_{xy}\sigma_x\sigma_y - \varepsilon\sum_{x\in\Z^d}h_x\sigma_x,
\end{equation*}
where $J_{xy}=J|x-y|^{-\alpha}$, $J, \varepsilon>0$, $\alpha > d$ and $h_x\in\mathbb{R}$, $d\geq 3$.
Until now, the only known result in the long-range setting is for the one-dimensional long-range Ising model with a random field, by Cassandro, Orlandi, and Picco \cite{Cassandro.Picco.09}. They used the contours of \cite{Cassandro.05} to show the phase transition for the model when $\alpha\in (3-\frac{\ln 3}{\ln 2}, \frac{3}{2})$, under the assumption $J(1) \gg 1$. We stress that, as remarked by Aizenman, Greenblatt, and Lebowitz \cite{Aizenman_Greenblatt_Lebowitz_2012}, although their argument does not work for the whole region for the exponent $\alpha$, the phase transition holds for values close to the critical value $\alpha=3/2$, since by the Aizenman-Wehr theorem we know that there is uniqueness for $\alpha>3/2$.

The argument from Ding and Zhuang in \cite{Ding2021}, for $d\geq3$, involves controlling the probability of a bad event, which is closely related to controlling the quantity $$\sup_{\substack{0\in A\subset\Z^d \\ A \text{ connected }}}\frac{\sum_{x\in A}h_x}{|\partial A|},$$ known as the greedy animal lattice normalized by the boundary. The greedy animal lattice normalized by the size, instead of the boundary, was extensively studied for general distributions of $(h_x)_{x\in\Z^d}$, see \cite{Cox_Gandolfi_Griffin_Kesten_93, Gandolfi_Kesten_94, Hammond_06, Martin_02}. When we normalize by the boundary, an argument by Fisher, Fr\"{o}hlich and Spencer \cite{FFS84} shows that the expected value of the greedy animal lattice is constant. In dimension $d=2$, the expected value is not finite, see \cite{Ding.Wirth.20}. The supremum is taken over connected regions containing the origin since the interiors of the usual Peierls contours are of this form.


For the long-range model, the interior of contours is not necessarily connected. In fact, long-range contours may have considerably large diameters with respect to their size, so their interiors can be very sparse. To avoid this, we define contours, strongly inspired by the $(M,a,r)$-partition in \cite{Affonso.2021}, using a multiscaled procedure that assures that the contours have no cluster with small density.  With them, we generalize the arguments by Fisher-Fr\"{o}hlich-Spencer \cite{FFS84}, and prove that the expected value of the greedy animal lattice is constant, even considering regions not necessarily connected in the supremum. Then, we prove the phase transition for $d\geq 3$. The main result of this paper is the following.
\begin{theorem*}Given $d\geq 3$, $\alpha>d$, there exists $\beta_c\coloneqq\beta(d, \alpha)$ and $\varepsilon_c\coloneqq\varepsilon(d, \alpha)$ such that, for $\beta >\beta_c$ and $\varepsilon\leq \varepsilon_c$, the extremal Gibbs measures $\mu_{\beta, \varepsilon}^+$ and $\mu_{\beta, \varepsilon}^-$ are distinct, that is, $\mu_{\beta, \varepsilon}^+ \neq \mu_{\beta, \varepsilon}^-$ $\mathbb{P}$-almost surely. Therefore the long-range random field Ising model presents phase transition.
\end{theorem*}

This paper is divided as follows. In Section 2, we define the model and the contours, and suitable generalizations to the constructions in \cite{Affonso.2021} are introduced.  In Section 3, we define two bad events of the external field and prove that they occur with a small probability.  In Section 4, we present the proof of the phase transition.

\section{Related work}
\label{sec:related}

\subsection{Neural-symbolic AI in semantic image interpretation}
In recent years, there has been significant research focus on NeSy architectures for addressing semantic image interpretation tasks \cite{Yu2021RecentAI,LTN,manigrasso2021faster,donadello2019compensating,Martone2022PROTOtypicalLT,li2021calibrating}. 
The present studies falls within the class of NeSy
techniques that seeks to incorporate symbolic information as a prior \cite{van2021modular}. Specifically, we rely on LTNs, which are modular architectures capable of incorporating FOL constraints \cite{LTN,Martone2022PROTOtypicalLT} and can be jointly trained with  neural module in an end-to-end manner \cite{manigrasso2021faster}.


In the LTN framework, the concept of  \textit{grounding} plays a crucial role in interpreting FOL within a specific subset of the domain, denoted as $\mathbb{R}^{n}$. In this approach, logical predicates and axioms are represented as vectors, which are then grounded (interpreted) as real numbers in the range [0, 1] using a technique called  \textit{Real Logic}. By employing this grounding mechanism, the LTN framework maps each term, denoted as $x$, to a vector representation in $\mathcal{G}(x) = \mathbb{R}^{n}$, while each predicate symbol, represented by $p \in \mathcal{P}$, is mapped to $\mathcal{G}\left(D(p)\right) \rightarrow [0,1]$. The concept can be illustrated by considering the frequently encountered \texttt{isOfClass} predicate in neuro-symbolic architectures, which quantifies the probability of a given term belonging to a specific class $c$ \cite{manigrasso2021faster,Martone2022PROTOtypicalLT}. The training objective is formulated by constructing a knowledge base $\mathcal{K}$ of FOL axioms, and finding the \textit{best satisfiability} (sat). %Given the aggregate truth value of the knowledge base, the loss function corresponds to $Loss = 1 - \text{sat}$ \cite{LTN}.



\subsection{Zero-shot learning}
ZSL tasks entail recognizing objects from previously unseen classes by exploiting some form of auxiliary knowledge, usually attribute-based, learned from seen classes. GZSL extends ZSL by assuming both seen and unseen classes are present at test time. Different strategies have been proposed to tackle ZSL, including embedding-based, attention mechanism-based, and generative strategies.

\textbf{Embedding-based} techniques compare semantic attributes and visual information by mapping them onto a suitable embedding space. Some methods embed images into the attribute space using an embedding function and consider semantic attributes as the common space~\cite{APN,CC-ZSL}. Other techniques have proposed to adopt the image embedding space as common space~\cite{Martone2022PROTOtypicalLT,dem,vse} to mitigate the hubness problem~\cite{Zhang2016LearningAD}. %adopting  images as a common space~\cite{dem,Martone2022PROTOtypicalLT,vse}.  
A third class of approaches~\cite{SP-AEN,EDEZSL,TCN} utilizes a shared space distinct from both image and attribute domains. To prevent overfitting to seen classes, these approaches require the use of pseudo-labeling techniques or a transductive setting~\cite{EDEZSL}, assuming that unlabeled images from unseen classes are provided during training. \textbf{Attention mechanisms} have been proposed to find the image regions that contribute the most to the categorization of a certain class and improve the embedding space~\cite{APN,CC-ZSL,Li2021AttributeModulatedGM,AREN}.



A critical aspect of embedding-based method is preventing overfitting to seen classes using, e.g., regularization \cite{APN,EDEZSL} or contrastive techniques \cite{CC-ZSL}. 
FLVN accomplishes this objective by incorporating a symbolic prior to aggregate visually and semantically similar features, while also explicitly establishing relationships between classes within the same macro class (a characteristic often implicitly addressed in alternative methodologies).

Previous approaches have used semantic class descriptions for the classification, but attributes linked to a specific class may not be consistently expressed or detectable in individual images. FLVN addresses this limitation by incorporating axioms that encode the existence of ``exceptions to the rule'' within the dataset. This aspect, which has received limited consideration in previous studies, has been experimentally shown to improve classification accuracy.


\textbf{Generative techniques}, on othe other hand, exploit auxiliary models, such as Generative Adversarial Networks (GANs), to generate artificial examples representing unseen classes by learning a conditional probability for each class~\cite{E-PGN,TGMZ,LisGAN,cycle-CLSWGAN}. Recently, feature generation models were integrated with embedding-based models in a contrastive setting~\cite{CEGZSL}. Generative methods require prior knowledge regarding unseen classes when generating training data. In contrast, the training process of FLVN relies on a subset of distinct seen classes and solely assumes knowledge of the class hierarchy at training time, so that the training can be easily extended to new unseen classes. Nonetheless, our approach is complementary to generative techniques, and could be in principle combined. 






\section{Fuzzy Logic Visual Network}
\label{sec:flvn}
We introduce a comprehensive and trainable framework for the task of ZSL, depicted in Figure~\ref{Figure:architecture}. 
It comprises two main modules: a feature extractor and a LTN that formulates the training objective.
\subsubsection{Feature extractor.} 
 The feature (embedding) extractor is a CNN that maps the input $x$ to a feature space $f_{\theta}(x) \in \mathbb{R} ^{H \times W \times B}$, where $H$, $W$ and $B$ represent the height, width, and number of channels of the features, respectively. Through mean pooling over $H$ and $W$, we obtain global discriminative characteristics $g_{\theta}(x) \in \mathbb{R}^{B \times 1}$, and utilize a linear projection to transform them into a semantic space represented by $V \in \mathbb{R}^{B \times M}$, where $B$ is the dimension of the vector space of features, and $M$ denotes the length of the attribute vector \cite{APN}. 



% Figure environment removed


\subsubsection{Logic Tensor Network.}
The LTN  formulates the learning objective as the maximum satisfiability of a $\KK$ .  For each training batch, the $\KK$ is updated introducing axioms that represent labelled examples ($\phi_1$), as well as prior knowledge ($\phi_2, \phi_3, \phi_4, \phi_5,\phi_6$). The maximum satisfiability loss is then defined based on the aggregation of all axioms as follows:
\begin{equation}
        \mathcal{L}^{\text{ep}} = 1 -\left(\bigwedge_{\phi \in \mathcal{K}} \phi \right)  = 1 -\GG(\phi) %, \GG(\phi_{\text{neg}})
        \label{eq:loss}
\end{equation}

This section first define the variables, predicates and domain that form the FOL language, followed by the definition of the knowledge base $\KK$. 

\textbf{Groundings}. Variables and their domains are grounded as follows:

\begin{equation}
\GG(l)= \mathbb{N}^{C}, 
\GG(q)= \mathbb{N}^{Q}
\end{equation}
\begin{equation}
\GG(a)=  \GG(a^{\text{mask}})= \mathbb{R}^{M \times C} 
\end{equation}
\begin{equation}
\GG(a^{\text{macro}})= \mathbb{R}^{M \times Q} 
\end{equation}
\begin{equation}
\GG(x)=  g_\theta(f_\theta(\GG(\texttt{images}))) = \mathbb{R}^{M}  \\
\end{equation}
where the variable $l$ represents the class labels belonging to set of classes $C$, $q$ represent the macroclass label belonging to set of macroclasses $Q$, and each class/macroclass is described by a set of non-binary semantic attributes denoted by $a$ and $a^{\text{macro}}$, respectively. Functions $f_\theta$ and $g_\theta$ are employed to embed images into the attribute space, resulting in the final representation $\GG(x)$. The FOL language contains four main predicates: $\texttt{isOfClass}(x,l)$ and $\texttt{isOfClass}_{\text{masked}}(x,l)$ denote the fact that an image $x$ belongs to class $l$,
$\texttt{isOfMacro}(x,q)$ that an image $x$ belongs to the macroclass $q$, and $\texttt{hasSameAttribute}(x_1, x_2)$ that two images have the same attributes. 

The $\GG(\texttt{isOfClass})$ predicate is grounded by the similarity  between the input image and the corresponding class attribute vectors. First, we compute the similarity between the image $x$ and a class $l_c$ by calculating the scaled product of the global features mapped in the attribute space with the semantic vectors:
\begin{equation}
p\left(x,l\right)=\frac{\exp \left(x^T V a_{l} \right)}{\sum_{s=1}^S \exp \left(x^T V a_s \right)}
\label{eq:prob}
\end{equation}
where $a_{y}$ represents the semantic attribute vector associated with class $l$. To obtain a prediction score for an example $x$, we calculate the dot product between the output of $p$ (Eq. \ref{eq:prob}) and the one-hot encoding $l_{c}^T$ for class $c \in C$, as follows:

\begin{equation}
\GG(\texttt{isOfClass}):x,l_c \rightarrow {l_c}^T p(\GG(x),l_c)
\end{equation}


 Similarly, we define $\GG(\texttt{isOfMacro})$ and $\GG(\texttt{isOfClass}_{\text{masked}})$. Since attributes for macro-classes are in principle unknown, we define a trainable attribute vector $a_m^{\text{macro}}$ for macro-class $q_m$ to compute $\GG(\texttt{isOfMacro})$. On the other hand, $\GG(\texttt{isOfClass}_{\text{masked}})$ uses the masked attribute vector $a_c^{\text{masked}}$ for class $l_c$, in which missing attributes $k$ are set to 0, while preserving the rest of the attributes in $a_c$. Finally, the grounding for $\texttt{hasSameAttribute}$  is defined as:
\begin{equation}
\GG(\texttt{hasSameAttribute}):x_1,x_2 \rightarrow   \texttt{sigmoid}(\alpha  \texttt{d}(\GG(x_1),\GG(x_2))) 
\label{eq:gt_sameattributes}
\end{equation}
where $\texttt{d}$ is the cosine similarity, $\alpha$ a scale factor and $\GG(x_1)$, $\GG(x_2)$ correspond to the embeddings of the two images.


\textbf{Learning from labeled examples}. 
We incorporate labelled examples by introducing an axiom $\phi_1$ stating that all facts about labeled example should be true, that is, all labeled samples should be classified correctly:
\begin{equation}
\phi_{\text{1}} = \forall \text{Diag}(x,l_c) (\texttt{isOfClass}(x,l_c))   \label{eq:isOfClass} \footnote{Diagonal Quantification quantifies over pairs of instances, e.g., images and their labels. A more formal definition can be found in \cite{LTN}.}
\end{equation}

To account for the class hierarchy, we also introduce an axiomatic statement $\phi_2$ to indicate that ``if an image contains a zebra'', then ``the image belongs to the family of ungulates'':

\begin{equation}
\phi_{\text{2}} = \forall \text{Diag}(x,l_c,q_m) (\texttt{isOfClass}(x,l_c) \implies \texttt{isOfMacro}(x,q_m)) 
\label{eq:isOfMacro}
\end{equation}

\textbf{Learning better feature representations}. The following axiom encodes the assumption that features extracted from two images of the same class should possess the same attributes:

\begin{equation}
\phi_{\text{3}} = \forall \text{Diag}(x_1,l_{c_1}) \bigg( \forall\text{Diag}(x_2, l_{c_2} ):{c_1}={c_2} \texttt{ }
 \texttt{hasSameAttribute}(x_1,x_2) \bigg)
 \label{eq:sameattribute}
\end{equation}

Likewise, images from different classes should possess different attributes:
\begin{equation}
\phi_{\text{4}} = \forall \text{Diag}(x_1,l_{c_1}) \bigg(\forall\text{Diag}(x_2, l_{c_2} ):{c_1}!={c_2} \texttt{ }
 \neg\texttt{hasSameAttribute}(x_1,x_2)\bigg)
 \label{eq:notsameattribute}
\end{equation}

To further emphasize the similarity between visual attributes and semantic vectors, the following axiom enforces the similarity between image embeddings and attribute vectors of the same class:
\begin{equation}
\phi_{\text{5}} = \forall \text{Diag}(x,l_{c}) \bigg( \forall\text{Diag}(a, l_{a} ): c=a
 \texttt{ hasSameAttribute}(x,a) \bigg)
 \label{eq:sameattribute_class}
\end{equation} 

\textbf{Learning with refutation}. In classical ZSL benchmarks such as AWA2, attributes are associated with class labels using a crisp or fuzzy matrix. However, this association does not entail that all examples of a class will exhibit exactly the same attributes: attributes may be expressed by a subset of training samples, or may be occluded. The existential statement $\phi_6$ represents the fact that some class attributes may not be present for all samples (e.g, ``there exists a zebra that is not agile''). Given that image-level attributes are not available, we simply remove randomly selected attributes by defining the $\texttt{isOfClass}_{\text{masked}}$ predicate:
%\begin{equation}
%\phi_{\text{6}} = \forall \text{Diag} (a_{masked}^{seen}, l^{seen})
%( \exists x , \texttt{isOfClass}_{masked}(x,a_{masked}^{seen})  )
%\label{eq:isOfClassmasked}
%\end{equation}
\begin{equation}
\phi_{\text{6}} = \forall  l^{\text{seen}}
( \exists x , \texttt{isOfClass}_{\text{masked}}(x,l^{\text{seen}})  )
\label{eq:isOfClassmasked}
\end{equation}
where $l^{\text{seen}}$ denotes the list of seen classes.

\textbf{Grounding logical connectives and aggregators}.
The knowledge base $\KK$ is an aggregation of formulas  updated at each training step. To solve the maximum satisfiability problem using gradient descent, logical connectives and aggregators must be grounded into Real Logic. Given  two truth values $a$ and $b$ in $[0,1]$, we adopted the symmetric configuration from \cite{LTN}, using the standard negation $\neg: N_S(a) =1-a $ and the Reichenbach implication $\rightarrow: I_R(a, b)  =1-a+a b$. The existential quantifier $\exists$ was approximated by the generalized mean $A_{pM}$, and the universal quantifier $\forall$ by the generalized mean w.r.t. the error $A_{pME}$, respectively \cite{LTN,van2008visualizing}.  Given $n$ truth values $a_1, \ldots, a_n$ all in $[0,1]$:
\begin{equation}
 \exists: A_{p M}\left(a_1, \ldots, a_n\right)=\left(\frac{1}{n} \sum_{i=1}^n a_i^{p_{\exists}}\right)^{\frac{1}{p_{\exists}}} \quad p_{\exists} \geqslant 1 \\
 \label{eq:aggregmean}
\end{equation}
\begin{equation}
\forall: A_{p M E}\left(a_1, \ldots, a_n\right)=1-\left(\frac{1}{n} \sum_{i=1}^n\left(1-a_i\right)^{p_{\forall}}\right)^{\frac{1}{p_{\forall}}} \quad p_{\forall} \geqslant 1\\
\label{eq:aggregmeanerror}
\end{equation}

$A_{p M E}$ is a measure of how much, on average, truth values $a_i$ deviate from the true value of 1. The $A_{p M E}$ was also used to approximate $\bigwedge$ in Eq.[\ref{eq:loss}]. Further details on the role of $p_{\exists}$ and $p_{\forall}$ can be found in previous works \cite{LTN}.

%Introducing general logical prepositions allows the architecture to recognize the characteristics of the classes being examined; however, demonstrating the presence of a false or invalid logical preposition (refutation) is even more important in order to manage exceptions to these rules and strengthen the validity of the developed theory. For this reason, we introduce a series of axioms useful for identifying the classes represented in the dataset with a hierarchical relationship.





%- w_{\text{n}} where the weight $w_{\text{n}}$ reflects the expectation that negations play a less discriminative role than affirmation in classification, in our case $w_{\text{n}}=0$ since we do not introduce negative axioms, 

\textbf{Querying the knowledge base}. 
At inference time, the class with the highest score is selected as the predicted class:

\begin{equation}
\hat{y}=\underset{\tilde{y} \in \mathcal{Y}^U}{argmax} (g(x)^{\mathrm{T}} V a_{\tilde{y}})
\end{equation}

FVLN was evaluated in both ZSL and GZSL settings.  In the ZSL setting, only unseen images are assumed to be present at test time, whereas in the GZSL setting, the model is tested on both seen and unseen classes. This setup induces a bias towards seen classes. To mitigate it, we employed the Calibrated Stacking method, as proposed in \cite{Chen2022MSDNMS,Wang2023GeneralizedZA}, to diminish the classification score of seen classes. The class score is thus calculated as $\hat{y}$:
\begin{equation}
\hat{y}=\underset{\tilde{y} \in \mathcal{Y}^U \cup \mathcal{Y}^S}{argmax }( g(x)^{\mathrm{T}} V a_{\tilde{y}}-\gamma \mathbb{I}\left[\tilde{y} \in \mathcal{Y}^S\right])
\end{equation}
where $\mathbb{I}=1$ if $\tilde{y}$ is from a seen class and zero otherwise, $\gamma$ is a calibration coefficient tuned on a validation set and $\mathcal{Y}^S$ are the labels of seen classes.

\textbf{Construction of the training batch}. Following the approach in \cite{CC-ZSL}, for a positive input image $x_{i}$, we select a set of positive examples $x^{+}$ and $K$ negative examples ${ x_{1}^{-},...,x_{K}^{-} }$. Positive examples are selected from the same category as $x_{i}$, while negative examples are randomly selected from the remaining classes.



\section{Experimental settings}
\label{sec:Experiments}
\section{Experimental Evaluations}\label{sec:experiment}

\textbf{Implementation.}
We implement \puma\ on top of SecretFlow~\citep{spu} in \textrm{C++} and Python. SecretFlow compiles a high-level Flax code to secure computation protocols, which are then executed by our designed cryptographic backends, and we encode the floating-ponit values as $64$-bit integers in ring $\mathbb{Z}_{2^{64}}$ with $18$-bit fractional part. 
Our experiments are run on 3 Alibaba Cloud ecs.g7.8xlarge servers with 32 vCPU and 128GB RAM each. The CPU model is Intel Xeon(Ice Lake) Platinum 8369B CPU @ 2.70GHz. We evaluate \puma\ on Ubuntu 20.04.6 LTS with Linux kernel 5.4.0-144-generic. Our bandwidth is about 5Gbps and round trip time is about 1ms. %\cheng{Describe fixed point parameters: scale, share bits.}

\textbf{Models \& Datasets.}
We evaluate \puma\ on seven NLP models: Bert-Base, Roberta-Base, and Bert-Large~\citep{bert}; GPT2-Base, GPT2-Medium, and GPT2-Large~\citep{gpt}; and LLaMA-7B~\citep{touvron2023llama}. We measure the Bert performance for three NLP tasks over the datasets of Corpus of Linguistic Acceptability (CoLA), Recognizing Textual Entailment (RTE), Stanford Question Answering Dataset (QNLI) from GLUE benchmarks~\citep{wang2018glue}, and GPT2 performance on Wikitext-103 V1~\citep{merity2016pointer}.

\textbf{Baseline.}
We compare \puma\ to the most similar prior work \mpcformer~\citep{li2023mpcformer}. But for fair comparison, we have the following considerations:
\romannumeral1) As \mpcformer\ neither supports loading pretrained transformer models nor implements LayerNorm faithfully\footnote{ As \mpcformer~does not support loading pre-trained Transformer models, we did an experiment in plaintext Bert-Base that replaced LayerNorm with BatchNorm  as \mpcformer~did. This  resulted in a significant drop in the MCC score for CoLA task from $0.616$ to $-0.020$. On the contrary, \puma~achieves an MCC score of $0.613$. }, we cannot achieve meaningful secure inference results using their framework.
Therefore, we compare our secure Transformer models inference performance to that of plaintext (floating-point) to show our precision guarantee.
\romannumeral2) \mpcformer\ with \textit{Quad} approximations (for both $\gelu$ and $\softmax$) requires retraining the  modified models. As \puma\ does not require retraining, we compare our cost to that of \mpcformer\ without \textit{Quad} approximations. Also, we re-run \mpcformer~in our environment.



\subsection{Precision}\label{sec:accuracy}

% Figure environment removed

%\begin{table}
\centering
\caption{Performance on GLUE benchmark of Bert-Base, Roberta-Base, and Bert-Large on CoLA, RTE, and QNLI, Matthews correlation is reported for CoLA. Accuracy is reported for other datasets.}\label{table:bertacc}
\begin{tabular}{c|ccc|ccc|ccc}
\hline \hline
 Model & \multicolumn{3}{c|}{Bert-Base} & \multicolumn{3}{c|}{Roberta-Base} & \multicolumn{3}{c}{Bert-Large} \\ \hline
 TASK & CoLA & RTE & QNLI & CoLA & RTE & QNLI & CoLA & RTE & QNLI \\ \hline
CPU & $0.616$     & $0.700$      & $0.916$     & $0.629$ & $0.805$ & $0.920$  & $0.686$   & $0.755$ & $0.922$ \\
\puma   & $0.613$     & $0.700$     & $0.916$     & $0.618$ & $0.805$ & $0.918$ & $0.690$ & $0.747$ & $0.918$ \\ \hline \hline
\end{tabular}
\end{table}

\begin{table}[]
    \centering
    \caption{Perplexity of GPT2-Base, GPT2-Medium, and GPT2-Large on Wikitext-103 V1.}
    \label{tab:gpot2ppl}
    \begin{tabular}{c|c|c|c}
    \hline \hline
      Model & GPT2-Base & GPT2-Medium & GPT2-Large \\ \hline
      CPU & $16.284$ & $12.536$ & $10.142$ \\
      \puma & $16.284$ & $12.540$ & $10.161$ \\
      \hline \hline
    \end{tabular}
    
\end{table}

We compare our secure model 
inference performance to that of plaintext (floating-point) in Figure~\ref{fig:performance} to show our precision guarantee.

In Figure~\ref{fig:bert-base}-\ref{fig:bert-large}, we show the Matthews correlation/accuracy of plaintext and \puma\ on the Bert-Base, Roberta-base, and Bert-Large. We observe that the accuracy achieved by \puma~ matches the accuracy of the plaintext Flax code. Specifically, the accuracy difference does
not exceed $0.011$ over all datasets. 

Moreover, in Figure~\ref{fig:gpt2}, we also compare our perplexity on dataset Wikitext-103 V1 with the plaintext baseline on models GPT2-Base, GPT2-Medium, and GPT2-Large. The results are similar and the perplexity differences do not exceed $0.02$ over all models.

The above accuracy and perplexity advantages experimentally validate that our protocols are numerically precise. 

\subsection{Inference cost}\label{sec:efficiency}
\begin{table}[h]
    \centering
    \caption{Costs of Bert-Base, Roberta-Base, and Bert-Large for one sentence of length $128$. Time is in seconds and Communication (Comm. for short) is in GB, which is the same for the following tables.}\label{tab:costbert}
    \begin{tabular}{c|cc|cc|cc}
    \hline \hline
       Model & \multicolumn{2}{c|}{Bert-Base} & \multicolumn{2}{c|}{Roberta-Base} & \multicolumn{2}{c}{Bert-Large} \\ \hline
       Costs & Time & Comm. & Time & Comm. & Time & Comm. \\ \hline
       \mpcformer & $55.320$ & $12.089$ & $57.256$ & $12.373$ & $141.222$ & $32.577$ \\
       \puma & $33.913$ & $10.773$ & $41.641$ & $11.463$ & $73.720$ & $27.246$ \\
       \cellcolor{mygray} Improv. & \cellcolor{mygray} $1.631\times$ & \cellcolor{mygray} $1.122\times$ & \cellcolor{mygray} $1.375\times$ & \cellcolor{mygray} $1.079\times$ & \cellcolor{mygray} $1.916\times$ & \cellcolor{mygray} $1.195\times$ \\
       \hline \hline
    \end{tabular}
    \vspace{-0.2cm}
\end{table}

\begin{table}[]
    \centering
    \caption{Costs of GPT2-Base, GPT2-Medium, and GPT2-Large. The input sentence is of length $32$, all of the costs are for generating $1$ token.}\label{tab:costgpt2}
    \begin{tabular}{c|cc|cc|cc}
    \hline \hline
       Model & \multicolumn{2}{c|}{GPT2-Base} & \multicolumn{2}{c|}{GPT2-Medium} & \multicolumn{2}{c}{GPT2-Large} \\ \hline
       Costs & Time & Comm. & Time & Comm. & Time & Comm. \\ \hline
       \mpcformer & $34.889$ & $4.999$ & $73.078$ & $11.766$ & $129.095$ & $22.522$  \\
       \puma & $15.506$ & $3.774$ & $30.272$ & $7.059$ & $54.154$ & $11.952$ \\
       \cellcolor{mygray} Improv. & \cellcolor{mygray} $2.250\times$ & \cellcolor{mygray} $1.325\times$ & \cellcolor{mygray} $2.414\times$ & \cellcolor{mygray} $1.667\times$ & \cellcolor{mygray} $2.383\times$ & \cellcolor{mygray} $1.884\times$ \\
       \hline \hline
    \end{tabular}
    \vspace{-0.2cm}
\end{table}

In this subsection, we compare \puma's inference cost to that of \mpcformer. 
We evaluate  three Bert models (Bert-Base, Roberta-Base, and Bert-Large) and three GPT2 models (GPT2-Base, GPT2-Medium, and GPT2-Large).
The costs are for processing one input sentence: \romannumeral1) For Bert models the input sentence is of length $128$. \romannumeral2) GPT2 models input one length-32 sentence and generate $1$ new word. 

On the 3 Bert models in Table~\ref{tab:costbert}, \puma\ is  $1.375\sim 1.916\times$ faster than  \mpcformer, and is $1.079\sim 1.195\times$ more communication-efficient. For the GPT2 models in Table~\ref{tab:costgpt2}, \puma\ is $2.250\sim 2.414\times$ faster than \mpcformer, and is $1.325\sim 1.884\times$ more communication-efficient. 
    
We observe that \puma's improvements increase as the model size grows, particularly for the GPT2 models. This trend is because our specialized optimizations are more effective when processing large-scale evaluations.



\subsection{Scalability}\label{sec:scala}

In this subsection, we measure the costs of evaluating \puma\ on Bert-Base and GPT2-Base models for varying-length inputs, and varying-length outputs (only for GPT2-Base). We also compare our costs to those of \mpcformer~to demonstrate our improvements.





\begin{table}[]
    \centering
    \caption{Costs of Bert-Base and GPT2-Base for different input length (denoted as \#Input). The input lengths for Bert-Base and GPT2-Base are respective $\{64, 128, 256, 512\}$ and $\{16, 32, 64, 128\}$. GPT2-Base generates $1$ token.}\label{tab:costbertinput}
    \begin{tabular}{cc|cc|cc|cc|cc}
    \hline \hline
       \multicolumn{2}{c|}{\#Input} & \multicolumn{2}{c|}{$64 / 16$} & \multicolumn{2}{c|}{$128 / 32$} & \multicolumn{2}{c|}{$256 / 64$} & \multicolumn{2}{c}{$512 / 128$}  \\ \hline
       \multicolumn{2}{c|}{Costs} & Time & Comm. & Time & Comm. & Time & Comm. & Time & Comm. \\ \hline
       \multirow{3}{*}{Bert}& \mpcformer & $46.428$ & $4.750$ & $85.887$ & $9.673$ & $196.372$ & $23.443$ & $582.787$ & $68.069$ \\
       & \puma & $24.345$ & $1.627$ & $42.525$ & $3.591$ & $87.561$ & $8.668$ & $212.600$ & $23.439$\\
       & \cellcolor{mygray} Improv. & \cellcolor{mygray} $1.907\times$ & \cellcolor{mygray} $2.919\times$ & \cellcolor{mygray} $2.020\times$ & \cellcolor{mygray} $2.694\times$ & \cellcolor{mygray} $2.243\times$ & \cellcolor{mygray} $2.705\times$ & \cellcolor{mygray} $2.741\times$ & \cellcolor{mygray} $2.904$ \\
       \hline
       \multirow{3}{*}{GPT2}& \mpcformer & $34.522$ & $3.767$ & $42.615$ & $4.516$ & $60.451$ & $6.281$ & $105.028$ & $11.225$  \\
       & \puma & $20.692$ & $0.625$ & $29.248$ & $1.258$ & $40.968$ & $2.607$ & $74.529$ & $5.611$\\
       &\cellcolor{mygray} Improv. & \cellcolor{mygray} $1.668\times$ & \cellcolor{mygray} $6.027\times$ & \cellcolor{mygray} $1.457\times$ & \cellcolor{mygray} $3.590\times$ & \cellcolor{mygray} $1.476\times$ & \cellcolor{mygray} $2.409\times$ & \cellcolor{mygray} $1.409\times$ & \cellcolor{mygray} $2.001\times$\\
       \hline \hline
    \end{tabular}
\end{table}
\textbf{Input Length Evaluation.}
Table~\ref{tab:costbertinput} shows our costs on varying-length inputs, we evaluate Bert-Base on the inputs of length $\{64, 128, 256, 512\}$, and GPT2-Base on the inputs of length $\{16, 32, 64, 128\}$.
For Bert-Base, \puma\ is $1.720\sim 2.282\times$ faster, and for GPT2-Base, \puma\ is $1.550\sim 2.686\times$ faster. Unlike the observations in Section~\ref{sec:efficiency}, our efficiency gains decrease with increasing input sizes in GPT2, and \puma\ requires more communication when the input length is greater than 64. This phenomenon is attributed to the interesting fact: To directly support pre-trained plaintext models, \puma\ strictly follows the plaintext model format that only accept token ids as input, so \puma\ has to compute the one-hot vectors from token ids in an MPC way. On the other hand, \mpcformer\ uses modified models that accept one-hot vectors as input, so the one-hot function could be computed at the client side in plaintext. Nevertheless, \puma\ remains faster than \mpcformer.

%\begin{table}[]
    \centering
    \caption{Costs of GPT2-small for generating different output tokens (denoted as \#Output), the input length is set as $32$.}\label{tab:costgpt2tokens}
    \begin{tabular}{c|cc|cc|cc|cc}
    \hline \hline
       \#Output & \multicolumn{2}{c|}{2} & \multicolumn{2}{c|}{4} & \multicolumn{2}{c|}{8} & \multicolumn{2}{c}{16}  \\ \hline
       Costs & Time & Comm. & Time & Comm. & Time & Comm. & Time & Comm. \\ \hline
       \mpcformer & $72.833$ & $7.676$ & $132.644$ & $13.998$ & $252.796$ & $26.648$ & $494.509$ & $51.972$ \\
       \puma & $53.191$ & $2.549$ & $111.457$ & $5.167$ & $215.352$ & $11.115$ & $457.994$ & $24.917$ \\
       Improv. & $1.369\times$ & $3.011\times$ & $1.190\times$ & $2.709\times$ & $1.174\times$ & $2.397\times$ & $1.080\times$ & $2.086\times$ \\
       \hline \hline
    \end{tabular}
\end{table}

\begin{wrapfigure}{r}{0.4\textwidth}
    % Figure removed
    \caption{Runtime of GPT2-Base for generating different number of output tokens, the input length is of length $32$.} 
    \label{fig:gptwoutcosts}
\end{wrapfigure}

\textbf{Output Length Evaluation.}
Fig~\ref{fig:gptwoutcosts} presents our costs on varying-length outputs for GPT2-Base, and compares our costs to those of \mpcformer. Our improvements in runtime range from $1.279\sim 2.700\times$ respectively.
As more output tokens are generated, both costs increase in a linear way, this is because each output token must be input back into the model to generate the next token, increasing the required one-hot embedding costs. We should emphasize
again that although the time costs might be close for long outputs, \puma\ could achieve a similar accuracy as plaintext models while \mpcformer\  could not. 


\begin{table}[]
    \centering
    \caption{Costs of the secure inference of LLaMA-7B, \#Input denotes the length of input sentence and \#Output denotes the number of generated tokens.}\label{tab:llama7b}
    \begin{tabular}{c|cc|cc|cc}
    \hline \hline
       (\#Input, \#Output) & \multicolumn{2}{c|}{$(4,1)$} & \multicolumn{2}{c|}{$(8,1)$} & \multicolumn{2}{c}{$(8,2)$} \\ \hline
       Costs & Time & Comm. & Time & Comm. & Time & Comm. \\ \hline
       \puma & $122.004$ & $0.907$ & $200.473$ & $1.794$ & $364.527$ & $3.857$ \\
       \hline \hline
    \end{tabular}
    \vspace{-0.2cm}
\end{table}

\textbf{Scale to LLaMA-7B in Five Minutes.}
We evaluated the large language model LLaMA-7B using \puma\ under 3 Alibaba Cloud
ecs.r7.32xlarge servers, each has 128 threads and 1TB RAM, with 20GB bandwidth, 0.06ms round-trip-time. 
As shown in Table~\ref{tab:llama7b}, \puma\ can support the secure inference of large language model LLaMA-7B with reasonable costs. For example, given an input sentence of 8 tokens, \puma\ can output one token in around $346.126$ seconds with communication costs of $1.865$ GB. To our knowledge, this is the first time that LLaMA-7B has been evaluated using MPC.


%Llama-7B, LAN=(20GB, 0.06ms), 128 threads, input length=8, output=1 token, costs: 346.126s, 2002213760 bytes

\section{Results}
\label{sec:Results}
\section{Results of RL active flow control}\label{sec:Results}

In this section, we discuss the converge of the RL algorithms for the three FM and PM cases (\S\ref{subsec:Convergence}) and evaluate their drag reduction performance (\S\ref{Result_drag_reduction}). A parametric analysis of the effect of NARX memory length is presented (\S\ref{subsec:Nfs}) and the isolated effect of including past actions as observations during the RL training and control (\S\ref{subsec:past_actions}). Studies of reward function (\S\ref{subsec:Rewards_Study}), sensor placement (\S\ref{subsec:Sensor_study}) and generalisability to Reynolds number changes (\S\ref{subsec:Res}) are presented, followed by a comparison of SAC and TQC algorithms (\S\ref{subsec:SACvsTQC}). 

\subsection{Convergence of learning}\label{subsec:Convergence}

We perform RL with the maximum entropy TQC algorithm to discover control policies for the three cases shown in figure \ref{fig:Case_Demo}, which maximise the net-power-saving reward function given by \req{eq: PowerR}. During the learning stage, each episode (1 DNS simulation) corresponds to $200$ non-dimensional time units.  To accelerate learning, $65$ environments run in parallel.


Figure \ref{fig:Learning_Curve} shows the learning curves of the three cases.  Table \ref{tab:LearningConvergence} shows the number of episodes needed for convergence and relevant parameters for each case.
It can be observed from the curve of episode reward that the RL agent is updated after every 65 episodes, i.e. $1$ iteration, where the episode reward is defined as 
\begin{equation}
R_{ep} = \sum_{k=1}^{N_k} r_{k},
\label{eq:Epi_R}
\end{equation}
where $k$ denotes the $k^{th}$ RL step in one episode and $N_k$ is the total number of samples in one episode.
The root mean square (RMS) value of the drag coefficient, $C_D^{RMS}$, at the asymptotic regime of control, is also shown to demonstrate convergence, defined as 
$C_D^{RMS} = \sqrt { (\mathcal{D}(\langle C_D\rangle_{env}))^2 }$,
where the operator $\mathcal{D}$ detrends the signal with a $9^{th}$-order polynomial and removes the transient part, and $\langle ~ \rangle_{env}$ denotes the average value of parallel environments in a single iteration. 

% Figure environment removed

\begin{table}
  \begin{center}
\def~{\hphantom{0}}
  \begin{tabular}{lcccccc}
    
      Environment  & Algorithm  &  $N_{c}$ & $R_{ep,c}$ & (Layers, Neurons) & $N_{fs}$ & Number of Inputs \\ 
       FM-Static   & TQC & $325$ & $37.72$ & (3,512) & $0$ & $64p_t+2a_{t-1}$\\
       PM-Static   & TQC & $1235$ & $21.87$ & (3,512) & $0$ & $64p_t+2a_{t-1}$\\
       PM-Dynamic  & TQC & $715$ & $34.35$ & (3,512) & $27$ & $N_{fs} (64p_t+2a_{t-1})$\\
  \end{tabular}
  \caption{Number of episodes $N_{c}$ required for RL convergence in different environments. The episode reward $R_{ep,c}$ at the convergence point, the configuration of NN and the dimension of inputs are presented for each case. $N_{fs}$ is the finite-horizon length of past actions-measurements.}
  \label{tab:LearningConvergence}
  \end{center}
\end{table}

In figure \ref{fig:Learning_Curve}, it can be noticed that in the FM environment, RL converges after approximately $325$ episodes ($5$ iterations) to a   {nearly} optimal policy using a static   {feedback} controller. As will be shown in \S\ref{Result_drag_reduction}, this policy is globally optimal since the vortex shedding is fully attenuated and the jets converge to zero mass flow actuation, thus recovering the unstable base flow and the minimum drag state.  However, with the same static   {feedback} controller in a PM environment (POMDP), the RL agent fails to discover the   {nearly} optimal solution, requiring around $1235$ episodes for convergence but only obtaining a relatively low episode reward.
Introducing a dynamic   {feedback} controller in the PM environment, the RL agent convergences to a near-optimal solution in 735 episodes. The dynamic   {feedback} controller trained by RL achieves a higher episode reward (34.35) than the static   {feedback} controller in the PM case (21.87), which is close to the FM case (37.72). The learning curves illustrate that using a finite horizon of past actions-measurements ($N_{fs} = 27$) to train a dynamic   {feedback} controller in the PM case improves learning in terms of speed of convergence and accumulated reward achieving nearly optimal performance with only wall pressure measurements. 


\subsection{Drag reduction with dynamic RL controllers} \label{Result_drag_reduction}

% Figure environment removed

The trained controllers for the cases shown in figure \ref{fig:Case_Demo} are evaluated to obtain the results shown in figure \ref{fig:TQC_FMPM}.   {Evaluation tests are performed for 120 non-dimensional time units to show both transient and asymptotic dynamics of the closed-loop system.}
Control is applied at $t=0$ with the same initial condition for each case, i.e. steady vortex shedding with average drag coefficient $\langle C_{D0}\rangle \approx 1.45$ (baseline without control). Consistent with the learning curves, the difference in control performance in the three cases can be observed both from the drag coefficient $C_D$ and the actuation $Q_1$.
  {The drag reduction is quantified by a ratio $\eta$ using the asymptotic time-averaged drag coefficient with control $C_{Da} = \langle C_{D}\rangle_{t \in [80,120]}$, the drag coefficient $C_{Db}$ of the base flow (details presented in Appendix \ref{App:BaseFlow}), and the baseline time-averaged drag coefficient without control $\langle C_{D0}\rangle$, as
\begin{equation}
\eta = \frac{\langle C_{D0}\rangle - C_{Da}}{\langle C_{D0}\rangle - C_{Db}} \times 100\%.
\label{eq:drag_reduction}
\end{equation}}

\begin{itemize}

\item {\bf FM-Static:} With a static   {feedback} controller trained in a full-measurement environment, a drag reduction of $\eta = 101.96\%$ is obtained with respect to the base flow (steady unstable fixed point; maximum drag reduction). This indicates that an RL controller informed with full-state information can entirely stabilise the vortex shedding and cancel the unsteady part of the pressure drag.

\item {\bf PM-Static:} A static/memoryless controller in a partial-measurement environment leads to performance degradation and a drag reduction of   {$\eta = 56.00\%$} in the asymptotic control stage, i.e. after $t=80$, compared to the performance of ``FM-Static''. This performance loss can also be observed from the control actuation curve, as $Q_1$ oscillates with a relatively large fluctuation in ``PM-Static'' while it stays about zero in the ``FM-Static'' case. 
The discrepancy between FM and PM environments using a static   {feedback} controller reveals the challenge of designing a controller with a POMDP environment. The RL agent cannot fully identify the dominant dynamics with only partial measurements on the   {downstream} surface of the bluff body, resulting in sub-optimal control behaviour.

\item{\bf PM-Dynamic:} With a dynamic   {feedback} controller (NARX model presented in \S\ref{subsec:PM_Dynamic}) in a partial-measurement environment, the vortex shedding is stabilised and the dynamic   {feedback} controller achieves   {$\eta = 97.00\%$} of the maximum drag reduction after time $t=60$. Although there are minor fluctuations in the actuation $Q_1$, the energy spent in the synthetic jets is significantly lower compared to the ``PM-Static'' case. Thus, a dynamic   {feedback} controller in PM environments can achieve nearly optimal drag reduction, even if the RL agent only collects information from pressure sensors on the   {downstream} surface of the body. The improvement in control indicates that the POMDP due to the PM condition of the sensors can be reduced to an approximate MDP by training a dynamic   {feedback} controller with a finite horizon of past actions-measurements. Furthermore, high-frequency action oscillations, which can be amplified with static   {feedback} controllers, are attenuated in the case of dynamic   {feedback} control. These encouraging and unexpected results support the effectiveness and robustness of model-free RL control in practical flow control applications, in which sensors can only be placed on a solid surface/wall.

\end{itemize}


% Figure environment removed

In figure \ref{fig:Contour}, snapshots of the velocity magnitude   {$|\boldsymbol{u}| = \sqrt{u^2+v^2}$} are presented for ``Baseline'' without control, ``PM-Static'', ``PM-Dynamic'' and ``FM-Static'' control cases. Snapshots are captured at $t=100$ in the asymptotic regime of control. A vortex-shedding structure of different strengths can be observed in the wake of all three controlled cases. In ``PM-Static'', the recirculation area is lengthened compared to the baseline flow, corresponding to base pressure recovery and pressure drag reduction. A longer recirculation area can be noticed in ``PM-Dynamic'' due to the enhanced attenuation of vortex shedding and pressure drag reduction. The dynamic   {feedback} controller in the PM case renders a $326.22\%$ increase of recirculation area with respect to the baseline flow, while only a $116.78\%$ increase is achieved by a static   {feedback} controller. The ``FM-Static'' case has the longest recirculation area, and the vortex shedding is almost fully stabilised, which is consistent with the drag reduction shown in figure \ref{fig:TQC_FMPM}.

% Figure environment removed

Figure \ref{fig:Obs} presents first- and second-order base pressure statistics for the baseline case without control and PM cases with control. In figure \ref{fig:Obs}(a), the time-averaged value of base pressure, $\overline{p}$, demonstrates the base pressure recovery after control is applied. Due to flow separation and recirculation, the time-averaged base pressure is higher at the middle of the   {downstream surface}, which is retained with control. The base pressure increase is directly linked to pressure drag reduction, which quantifies the control performance of both static and dynamic   {feedback} controllers. Up to $49.56\%$ of pressure increase at the centre of the   {downstream surface}  is obtained in the ``PM-Dynamic'' case, while only $21.15\%$ can be achieved by a static   {feedback} controller. In figure \ref{fig:Obs}(b), the base pressure RMS is shown. For the baseline flow, strong vortex-induced fluctuations of the base pressure can be noticed around the top and bottom   {on the downstream surface} of the bluff body. In the ``PM-Static'' case, the RL controller   {partially suppresses} the vortex shedding, leading to a sub-optimal reduction of the pressure fluctuation. The sensors close to the top and bottom corners are also affected by the synthetic jets, which change the RMS trend for the two top and bottom measurements. In the ``PM-Dynamic'' case,  the pressure fluctuations are nearly zero for all the measurements on the   {downstream surface}, highlighting the success of vortex shedding suppression by a dynamic RL controller in a PM environment.

% Figure environment removed

The differences between static and dynamic controllers in PM environments are further elucidated in figure \ref{fig:Action_analysis} by examining  the time series of pressure differences $\Delta p_t$ from surface sensors (control input) and control actions $a_{t-1}$ (output). The pressure differences are calculated from sensor pairs at $y=\pm y_{sensor}$, where $y_{sensor}$ is defined in Eq. \req{eq:Probe_base}. For $N=64$, there are 32 time series of $\Delta p_t$ for each case. 
%
During  the initial stages of control ($t \in [0,11]$), the control actions are similar  for the two PM cases and they deviate for $t>11$, resulting in discernible control performance at the asymptotic regime. 
At the initial stages, the controllers operate in nearly anti-phase to $\Delta p_t$, in order to eliminate the antisymmetric pressure component due to vortex shedding. The inability of the static controller to have a frequency dependent amplitude (and phase), manifests as well through the amplification of high frequency noise. For $t>11$, the static feedback controller continues to operate in nearly anti-phase to the pressure difference, resulting in partial stabilisation of unsteadiness. However, the dynamic feedback controller adjusts its phase and amplitude significantly, which attenuates the antisymmetric fluctuation of base pressure and drives $\Delta p_t$ to near zero. 

% Figure environment removed

Figure \ref{fig:ContourComparision} shows instantaneous vorticity contours for PM-Dynamic and PM-Static cases, showing both the similarities and discrepancies between the two cases. At $t=2$, flow is expelled from the bottom jet for both cases, generating a clockwise vortex, termed V1. This V1 vortex, shown in black, works against the primary counter-clockwise vortex labelled as P1, depicted in red, emerging from the bottom surface. At $t=5.5$, a secondary vortex, V2, forms from the jets to oppose the primary vortex shedding from the top surface (labelled as P2). 
%
 At $t=13$, the suppression of the two primary vortices near the bluff body is evident in both cases, indicated by their less tilted shapes compared to the previous time instances. At $t=13$, the PM-Dynamic adjusted the phase of the control signal, which corresponds to a marginal action at this time instance at figure \ref{fig:Action_analysis}. Consequently, no additional counteracting vortex is formed in PM-Dynamic. However, in the PM-Static scenario, the jets generate a third vortex, labelled V3, which emerges from the top surface. This corresponds to a peak in the action of the PM-Static controller at this time. The inability of the PM-Static controller to adapt the amplitude/phase of the input/output behaviour results in suboptimal performance.

\subsection{Horizon of the finite-history sufficient statistic}\label{subsec:Nfs}

A parametric study on the horizon of the finite history in NARX (equation \req{eq:NARX}), i.e. the number of frames stacked $N_{fs}$, is presented in this section. Since the NARX model uses a finite horizon of past actions-measurements in  \req{eq:Sufficient_statistic}, the horizon of the finite history affects the convergence of the approximation \citep{yu_near_2008}. This approximation affects the optimisation during the learning of RL because it determines whether the RL agent can observe sufficient information to converge to an optimal policy. 

Since vortex shedding is the dominant instability to be controlled, the choice of $N_{fs}$ should intuitively link to the timescale of the vortex shedding period. The ``frames'' of observations are obtained every RL step ($0.5$ time units), while the vortex shedding period is $t_{vs}\approx6.85$ time units. Thus, $N_{fs}$ is rounded to integer values for different numbers of vortex shedding periods, as shown in table \ref{tab:Frame_Stack}.


% Figure environment removed

\begin{table}
  \setlength{\tabcolsep}{12pt}
  \begin{center}
\def~{\hphantom{0}}
  \begin{tabular}{ccc}
      Number of  & Non-dimensional &  History length \\
      VS periods &    time units          &  ($N_{fs}$)         \\ [3pt]
      \hline
       0.5   & 3.43 & 7 \\
       1   & 6.85 & 14 \\
       2  & 13.70 & 27 \\
       3 & 20.55 & 41\\
       4 & 27.40 & 55\\
       5 & 34.25 & 68\\
  \end{tabular}
  \caption{Correspondence between the number of vortex shedding (VS) periods and frame stack (history) length in samples $N_{fs}$. The RL control step size is $t_a =0.5$, and $N_{fs}$ is rounded to an integer.}
  \label{tab:Frame_Stack}
  \end{center}
\end{table}

The results of time-averaged drag coefficients $\langle C_{D}\rangle$ after control and the average episode rewards $\langle R_{ep}\rangle$ in the final stage of training are presented in figure \ref{fig:Frame_Stack}. As $N_{fs}$ increases from 0 to 27, the performance of RL control improves, resulting in a lower $\langle C_{D}\rangle$ and a higher $\langle R_{ep}\rangle$. $N_{fs}=2$ is specially examined because the latent dimension of the vortex shedding limit cycle is 2. However, the control performance with $N_{fs}=2$ is marginally improved to the one with $N_{fs}=0$, i.e. a static   {feedback} controller. This result indicates that the horizon consistent with the vortex shedding dimension is not long enough for the finite horizon of past action measurements. The optimal history length to achieve stabilisation of the vortex shedding   {in PM environments} is 27 samples, which are equivalent to 13.5 convective time units or $\sim 2$ vortex shedding periods. 

With $N_{fs}=41$ and $N_{fs}=55$, the drag reduction and episode rewards drop slightly compared to $N_{fs}=27$. The decline in performance is non-negligible as $N_{fs}$ increases further to 68. This decline shows that excessive inputs to the neural networks (see table \ref{tab:LearningConvergence}), may impede training because more parameters need to be tuned or larger neural networks need to be trained. 

\subsection{Observation sequence with past actions}\label{subsec:past_actions}

Past actions (exogenous terms in NARX) facilitate reducing a POMDP to an MDP problem, as discussed in \S\ref{subsec:PM_Dynamic}. In the near-optimal control of a PM environment using a dynamic   {feedback} controller with inputs $\left( o_t, o_{t-1}, ..., o_{t-N_{fs}} \right)$, a sequence of observations $o_t = \left \{ p_t, a_{t-1}\right \}$ at step $t$ is constructed to include pressure measurements and actions. In the FM environment, due to the introduction of one-step delayed action due to the first-order-hold interpolation given by \req{eq:FOH_action}, the inclusion of the past action along with the current pressure measurement, meaning $o_t = \left \{ p_t, a_{t-1} \right \}$, is required even when the sensors are placed in the wake and cover the wavemaker region. 

Figure \ref{fig:ActionInObs} presents the control performance for the same environment with and without past actions included.
In the FM case, there is no apparent difference between RL control with $o_t = \left \{ p_t, a_{t-1} \right \}$ or $o_t = \left \{ p_t \right \}$, which indicates that the inclusion of the past action is negligible to the performance. This is the case when the RL sampling frequency is sufficiently faster than the timescale of the vortex shedding dynamics. 
In PM cases, if exogenous action terms are not included in the observations but only the finite history of pressure measurements is used, the RL control fails to converge to a near-optimal policy, with only   {$\eta = 67.45\%$}  drag reduction. With past actions included, the drag reduction of the same environment increases up to   {$\eta = 97.00\%$}. 

The above results show that in PM environments, sufficient statistics cannot be constructed only from the finite history of measurements. Missing state information needs to be reconstructed by both state-related measurements and control actions. 

% Figure environment removed

\subsection{Reward study}
\label{subsec:Rewards_Study}

In \S\ref{Result_drag_reduction}, a power-based reward function given by \req{eq: PowerR} has been implemented, and stabilising controllers can be learned by RL, as shown. In this section, RL control results with other forms of reward functions (introduced in \S\ref{subsec:Reward}) are provided and discussed.

% Figure environment removed

The control performance of RL control with the different reward functions is evaluated based on the drag coefficient $C_D$ shown in figure \ref{fig:Reward_Study}. Static   {feedback} controllers are trained in FM environments, and dynamic   {feedback} controllers are trained in PM environments. In FM cases, control performance is not sensitive to the choice of reward function (power or force-based).  
In PM cases, the discrepancies between RL-step time-averaged and instantaneous rewards can be observed in the asymptotic regime of control. The controllers with both rewards (power or force-based) achieve nearly optimal control performance, but there is some unsteadiness in the cases using instantaneous rewards due to slow statistical convergence of the rewards and limited correlation to the partial observations.

All four types of reward functions studied in this work achieve nearly optimal drag reduction around $100\%$. However, the energy-based reward (``PowerR'') offers an intuitive reward design, attributable to its physical properties and the dimensionally consistent addition of the constituent terms of the reward function. Further enhancing its practicality, since the power of the actuator can be directly measured, it avoids the necessity for hyperparameter tuning, as in the force-based reward. Additionally, the results show similar performance with both time-averaged between RL steps and instantaneous rewards, avoiding the necessity for faster sampling for the calculation of the rewards. This choice of reward function can be extended to various RL flow control problems and can be beneficial to experimental studies.


\subsection{Sensor configuration study with partial measurements}\label{subsec:Sensor_study}

% Figure environment removed

In the PM environment, the configuration of sensors (number and location on the downstream surface) may also affect the information contained in the observations and thus control performance. 
Control results of drag coefficient $C_D$ for different sensor configurations in PM-dynamic cases are presented in figure \ref{fig:Sensor_config}. In the configuration with $N = 2$, two sensors are placed at $y=\pm 0.25$, and for $N = 1$, only one sensor is placed at $y = 0.25$. Other configurations are consistent with equation \req{eq:Probe_base}. 

The $C_D$ curves in figure \ref{fig:Sensor_config} show that, as the number of sensors is reduced from 64 to 2, RL control achieves the same level of performance with minor discrepancies due to randomness in different learning cases. However, if RL control uses observations from only one sensor at $y = 0.25$, performance degradation can be observed in the asymptotic stage with 19.79\% on average less drag reduction. The sub-figure presents the relationship between the number of sensors and asymptotic drag coefficient $\langle C_D \rangle$. These results indicate a limit on sensor configuration for the use of the NARX-modeled controller to stabilise the vortex shedding. 

% Figure environment removed

To understand the cause of performance degradation in the $N=1$ case, the pressure measurements from two sensors in both baseline and PM-Dynamic cases are presented in figure \ref{fig:Pressure2Sensors}. In the baseline case, two sensors are placed at the same location as the $N=2$ case ($y=\pm 0.25$) only for observations. It can be observed that the pressure measurements from two sensors are anti-symmetric since they are placed symmetrically on the downstream surface.
In the PM-Dynamic case, the NARX controller is used, and control is applied at $t=0$. In this closed-loop system, the anti-symmetric relationship between two sensors (from the symmetric position) is broken by the control actuation, and no correlation is evident. This can be seen during the transient dynamics, e.g. in $t \in [0,10]$. Therefore, when the number of sensors is reduced to $N=1$ by removing one sensor from the $N=2$ case, the dynamic feedback from the removed sensor cannot be fully reflected by the remaining sensor in the closed-loop system. This loss of information affects the fidelity of the control response to the dynamics of the sensor-removing side, causing suboptimal drag reduction in the $N=1$ scenario.

It should be noted that the configuration of 64 sensors is not necessary for control, as $N = 2$ or $N = 16$ also achieves nearly optimal performance. The number of sensors $N = 64$ in PM-Static environments is used for comparison with the FM-Static configuration (Eq. \ref{eq:Probe_wake}), which eliminates the effect from different input dimensions between two static cases. Also, 64 sensors sufficiently cover the downstream surface of the bluff body to avoid missing spatial information. 
The optimal configuration of sensors can be tuned with optimisation techniques such as \cite{paris_robust_2021}, but the results in figure \ref{fig:Sensor_config} indicate that RL adapts with nearly optimal performance to non-optimised sensor placement in the present environment.

\subsection{Performance of RL controllers to unseen $Re$} \label{subsec:Res}

% Figure environment removed

The RL controller is tested at different Reynolds numbers, in order to examine its generalisability to environment changes. The controllers have been trained at $Re=100$ with both FM and PM conditions, and tested at $Re= 80, 90, 100, 110, 120, 150$. The controllers were further trained at $Re=150$, denoted as continual learning (CL), and tested again at $Re=150$. 

As shown in figure \ref{fig:Res}, in both ``PM-Dynamic'' and ``FM-Static'' cases, the RL controllers are able to reduce drag by $\eta=64.68\%$ in the worst case, when $Re$ is close to the training point at $Re=100$, i.e. the test cases with $Re= 80, 90, 100, 110, 120$. 
However, when applying the controllers trained at $Re=100$ to an environment at $Re=150$, the drag reduction drops to $\eta=41.98\%$ and $\eta = 74.04\%$ in PM-Dynamic and FM-Static cases, respectively.

Performing CL at $Re=150$, the drag reduction is improved to $\eta = 78.07\%$ in PM-Dynamic after 1105 training episodes while $\eta = 88.13\%$ in FM-Static after 390 episodes, with the same RL parameters as the training at $Re=100$.
Overall, the results of these tests indicate that the RL-trained controllers can achieve significant drag reduction in the vicinity of the training point (i.e. $\pm\%20$ $Re$ change). If the test point is far from the training point, a CL procedure can be implemented to achieve nearly optimal control.

\subsection{TQC vs SAC}\label{subsec:SACvsTQC}

% Figure environment removed

Control results with TQC and SAC are presented in figure \ref{fig:TQCvsSAC} in terms of $C_D$. TQC shows a more robust control performance. In the case of FM, SAC might demonstrate a slightly more stable transient behaviour attributed to the fact that the quantile regression process in TQC introduced complexity to the optimisation process. Both controllers achieved an identical level of drag reduction in the FM case. 

However, in the context of the PM cases, it is observed that TQC outperforms SAC in drag reduction with both static and dynamic   {feedback} controllers. For static   {feedback} control, TQC achieved an average drag reduction of   {$\eta = 56.00\%$}, compared to the   {$\eta = 46.31\%$}  reduction achieved by SAC. The performance under dynamic   {feedback} control conditions is more compelling, where TQC fully reduced the drag, achieving   {$\eta = 97.00\%$}  of drag reduction, reverting it to a near-base-flow scenario. In contrast, SAC managed to achieve an average drag reduction of   {$\eta = 96.52\%$}.

The fundamental mechanism for updating Q-functions in RL involves selecting the maximum expected Q-functions among possible future actions. This process, however, can potentially lead to overestimation of certain Q-functions \citep{hasselt_double_2010}. In POMDP, this overestimation bias might be exacerbated due to the inherent uncertainty arising from the partial-state information. Therefore, the Q-learning-based algorithm, when applied to POMDPs, might be more prone to choosing these overestimated values, thereby affecting the overall learning and decision-making process.

As mentioned in \S\ref{subsec:SACTQC}, the core benefit of TQC under these conditions can be attributed to its advanced handling of the overestimation bias of rewards. By constructing a more accurate representation of possible returns, TQC provides a more accurate Q-function approximation than SAC. This process of modulating the probability distribution of the Q-function assists TQC in managing the uncertainties inherent in environments with only partial-state information. In this case, TQC can adapt more robustly to changes and uncertainties, leading to better performance in both static and dynamic feedback control tasks.

\section{Conclusions and future work}
\label{sec:Conclusion}
\section{Conclusion}\label{sec:conclusion}

This paper presents our empirical domain knowledge distillation framework using ChatGPT and discusses our observations from the framework application experiments in the autonomous driving domain. The key finding is that: 1) with proper design of prompt engineering and execution flow, fully automated domain knowledge (in the ontology format) distillation is possible. However, due to the randomness in the response and the butterfly effect, the quality of fully automated distillation results is not guaranteed. To address this, we develop a web-based assistant to enable manual supervision and early intervention at runtime. We hope our findings and tools inspire future research toward revolutionizing the engineering processes of knowledge-based systems across domains.

\section*{Acknowledgements}
This study was carried out within the FAIR - Future Artificial Intelligence Research and received funding from the European Union Next-GenerationEU (PIANO NAZIONALE DI RIPRESA E RESILIENZA (PNRR) – MISSIONE 4 COMPONENTE 2, INVESTIMENTO 1.3 – D.D. 1555 11/10/2022, PE00000013). This manuscript reflects only the authors’ views and opinions, neither the European Union nor the European Commission can be considered responsible for them.

\bibliographystyle{splncs05}
\bibliography{main.bib}

\end{document}
