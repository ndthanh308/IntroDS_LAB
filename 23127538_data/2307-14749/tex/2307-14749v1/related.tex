\section{Background and Related Work}
\label{sec:related}



The large efforts that game developers invest in the game development process do not always allow them to discover or fix all the bugs in a game before releasing it to the market. Several works have focused the attention on the quality assurance of video games analyzing the differences between traditional software development and video games development \citep{murphy2014cowboys, santos2018computer}. Many studios employ discussion forums or specific features in their games for gamers to report bugs (\eg Steam Community). Previous work shows that 80\% of the Steam games release urgent updates to fix issues such as feature malfunctions or game crashes \citep{lin2017studying}. The large amount of gameplay videos continuously produced and publicly released by many gamers on platforms such as Twitch and YouTube could be helpful to developers: Sometimes, gamers indirectly report issues while they play. Since \approach aims to support video game developers by extracting information from gameplay videos, the discussion focuses mainly on approaches aimed at extracting and manipulating gameplay videos for different purposes. In addition, since our approach aims to automatically categorize video segments, we also discuss existing taxonomies of video game topics that we use as a starting point for defining our categories.

\subsection{Mining of Gameplay Videos}
Some works targeted the automated generation of a comprehensive description of what happens in gameplay videos (\ie game commentary). Examples of these works are the framework by \citet{guzdial2018towards} and the approach presented by \citet{li2019end} modeling the generation of commentaries as a sequence-to-sequence problem, converting video clips to commentary. On the same line of research, an approach to generate automatic comments for videos by using deep convolutional neural networks was presented by \citet{shah2019automated}.
\citet{lewis2010went} described the gameplay videos as ``a rich resource.''
The main goal of \approach is to detect issues in gameplay videos. To the best of our knowledge, the only work aimed at achieving a similar goal is the one by \citet{lin2019identifying}. The authors conducted an in-depth study of gameplay videos posted by players on the Steam platform aiming at automatically identifying the ones that report bugs. They observe that na\"{\i}ve approaches based on keywords matching are inaccurate. Therefore, they propose an approach that uses a Random Forest classifier \citep{ho1995random} to categorize gameplay videos based on their probability of reporting a bug.
\citet{lin2019identifying} rely on Steam\footnote{\url{https://steamcommunity.com/}} to find videos related to specific games. While Steam is mainly a marketplace for video games, it also allows users to interact with each other and share videos. On a daily basis, for 21.4\% of the games on Steam, users share 50 game videos, and a median of 13 hours of video runtime \citet{lin2019identifying}. 
Still, their approach works at video-level, and manually watching long gameplay videos classified as buggy still requires a considerable manual effort since a whole video can even last several hours. Also, they only distinguish bug-reporting videos from non-bug-reporting ones, without a more specific classification regarding the type of issue reported (\eg glitch or logic bug).
We fill this gap and further aid developers by classifying the video segment according to the type of problem encountered, and by trying to classify video segments (\ie parts of videos) instead of whole videos. To achieve this goal, \approach augments the provided information, by including also (i) the type of issue found, (ii) the context (\ie area of the game) in which it occurred, and (iii) other segments in which the same issue was reported (possibly from different videos).

\subsection{Taxonomies of Video Game Issues}
\label{sec:related:taxonomy}
Video games can suffer from a vast variety of problems. Lin \etal \citep{lin2019identifying} do not distinguish among the types of issues reported in the videos identified as ``bug reporting'', while this is one of our goals. 

To determine meaningful categories in which it is worth categorizing video segments, we rely on a recent taxonomy of issues in video games introduced by \citep{truelove2021we} (which extends the one by  \citep{lewis2010went}). In their taxonomy, the authors reports 20 different kinds of issues. 

\noindent We use such a taxonomy as a base to define the labels we want to assign to the video segments. However, all such labels might be counterproductive since it is likely to observe a long-tail distribution (\ie a few types of issues appear in most of the video fragments, while several other issues are quite rare or do not even appear). Therefore, starting from such a taxonomy, we define macro-categories by clustering similar fine-grained categories. We identified four labels, as reported in \tabref{tab:taxonomy}: \textit{Logic}, \textit{Presentation}, \textit{Balance}, and \textit{Performance}. 

\begin{table}
  \centering
  \caption{Mapping between types of issues identified by \approach and categories from the taxonomy by Truelove \etal \citep{truelove2021we}.}
  \label{tab:taxonomy}%
  \resizebox{0.9\linewidth}{!}{
    \begin{tabular}{l p{3cm} l}
    \toprule
    \textbf{Issue Type} & \textbf{Description} & \textbf{Categories \citep{truelove2021we}}\\
    \midrule
    \multirow{11}{*}{\textbf{Logic}} & 
    \multirow[t]{11}{*}{
    \parbox[t]{3cm}{Issues related to the game logic, regardless of how information is presented to the player.
    }} & 
        Object Persistence \\
    & & Collision of Objects\\
    & & Inter. btw. Obj. Prop.\\
    & & Position of Object\\
    & & Context State\\
    & & Crash\\
    & & Event Occurrence\\
    & & Interrupted Event\\
    & & Triggered Event\\
    & & Action\\
    & & Value\\
    \midrule
    
    \multirow{6}{*}{\textbf{Presentation}} & 
    \multirow[t]{6}{*}{
    \parbox[t]{3cm}{Issues related to the game interface (graphical- or audio- related).
    }} & 
        Game Graphics\\
    & & Information\\
    & & Bounds\\
    & & Camera\\
    & & Audio\\
    & & User Interface\\ % SIMONE: merged with the one below
    \midrule
    
    \multirow{2}{*}{\textbf{Balance}} & 
    \multirow[t]{2}{*}{
    \parbox[t]{3cm}{
    Detrimental aspects in terms of ``fun''.
    }} & 
        Artificial Intelligence\\
    & & Exploit\\
    \midrule
    
    \multirow{2}{*}{\textbf{Performance}} & Performance-related issues (\eg FPS drops). & Implem. Response\\
    \bottomrule
    \end{tabular}%
    }
\end{table}%
