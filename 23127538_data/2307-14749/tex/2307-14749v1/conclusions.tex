\section{Conclusion}
\label{sec:conclusions}
In recent years, there has been a growing interest in video games. During game development, many bugs go undetected prior to release because of the difficulty of fully testing all aspects of a video game.
We introduce \approach, a novel approach for detecting anomalies in video games from gameplay videos to support developers by providing them with useful information on how to improve their games. We validated the single steps of \approach in an empirical study involving 604 segments extracted from 80 hours of gameplay videos related to 3 video games (Conan Exiles, DayZ, and New World). We obtained mixed results: The effectiveness of both segmentation (step 1) and issue-based clustering (step 4) are satisfactory, while we observed that categorization (step 2) and context-based clustering (step 3) of segments still do not work sufficiently well to be used in practice.
Future work should aim at addressing these two problems. To foster research in this field, we publicly release all the (manually annotated) datasets in our replication package \cite{replicationpackage}.


%In recent years, there has been a growing interest in video games. During game development, many bugs go undetected prior to release because of the difficulty of fully testing all aspects of a video game.  We propose a novel approach for detecting anomalies in video games from gameplay videos to support developers through the analysis of subtitles by (i) providing a labeled dataset that contains the issues found in gameplay videos and (ii) proposing an approach to automatically identify issues in the gameplay videos. However, the analysis of the subtitles is complex as, following the segmentation of the video, the sentences are incomplete and ambiguous also due to the use of slang. Based on the qualitative analysis we conducted, future work will be aimed at evaluating further techniques for textual analysis of subtitles (e.g., Bag of Words). Besides, we will try to get broader context for the subtitles to try to capture the use of interesting keywords that might be pronounced after the issue appears. We will also evaluate additional features that can be extracted through the analysis of gameplay videos, including audio and game frame analysis.
