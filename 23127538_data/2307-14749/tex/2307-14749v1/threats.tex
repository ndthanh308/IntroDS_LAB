\section{Threats to Validity}
\label{sec:threats}

\textbf{Threats to construct validity} mainly pertain the possible imprecisions made while defining the test set used to evaluate \approach and to answer all our research questions. As explained in \secref{sec:design}, to reduce this threat, two evaluators independently tagged each instance and discussed conflicts aiming at reaching consensus. This occurred in 1.2\% of the cases.

\textbf{Threats to internal validity} concern factors internal to our study that could have affected the results. 
A first threat regarding \RQ{2} is related to the specific set of ML techniques we decided to use and to the preprocessing pipelines we tested. As for the first, we took into account the main categories of classic ML approaches. It is possible that Deep Learning-based approach achieve better results, but we avoided using such approaches because even a small Neural Network (Multilayer Perceptron) achieves very poor results given the small size of our training set. \REV{Another limitation related to \RQ{2} is the choice not to tune the hyperparameters and to use the predefined hyperparameters provided by Weka. To understand the impact of this decision, we tried to replicate the results of binary classification of segments as \textit{informative} and \textit{non-informative} while varying the main hyperparameter for Random Forest (\ie the maximum number of features). We report in \tabref{tab:rq2_hyperparameters} the results of such an analysis on the Conan Exiles dataset.\footnote{Note that the results differ from the ones reported in \tabref{tab:rq2binary_videogames} because we did not run any preprocessing step here.} Although this analysis revealed some improvements in model performance while varying such a parameter, we found that the impact of not tuning it was rather small (+4 percentage points for F-Measure and +0.01 for AUC). Thus, we believe this is not the cause of the negative results we obtained.}

\begin{table*}[t]

\newcommand{\I}{\faIcon[regular]{info}}
\newcommand{\NI}{\faIcon[regular]{trash}}
\newcommand{\OV}{\faIcon[regular]{globe}}
\centering
\caption{\REV{\RQ{2}: Hyperparameter Tuning of the Random Forest categorization model on Conan Exiles. We use the icon \I{} to indicate the \textit{informative} class and the icon \NI{} to indicate the \textit{non-informative} class, while \OV{} indicates their weighted mean.}}
\label{tab:rq2_hyperparameters}
\resizebox{\linewidth}{!}{%
\begin{tabular}{l|rrr|rrr|rrr|rrr}
\toprule
\multirow{2}{*}{\textbf{NumFeatures}}     & \multicolumn{3}{c|}{\textbf{Precision}} & \multicolumn{3}{c|}{\textbf{Recall}} & \multicolumn{3}{c|}{\textbf{F-Measure}} & \multicolumn{3}{c}{\textbf{AUC}} \\
                                   & \I    & \NI    & \OV                     & \I    & \NI    & \OV                 & \I    & \NI    & \OV                    & \I    & \NI    & \OV             \\
\midrule                                                                                                                                                                                         
{Unlimited (default)}                    & 55\% & 72\%  & 63\%                      & 88\% & 29\%  & 59\%                  & 68\% & 42\% & 55\%                     & 0.68  & 0.68 & 0.68              \\
{1}                             & 55\% & 65\%  & 60\%                       & 80\% & 37\% & 58\%                 & 65\% & 47\% & 56\%                     & 0.61  & 0.61 & 0.61              \\
{2}                        & 56\% & 74\%  & 65\%                      & 87\% & 34\%  & 60\%                  & 68\% & 47\% & 58\%                     & 0.63  & 0.63 & 0.63              \\
{3}                     & 57\% & 80\%  & 69\%                      & 92\% & 32\%  & 62\%                  & 70\% & 45\% & 57\%                     & 0.66  & 0.66 & 0.66              \\
{4}                             & 54\% & 66\%  & 60\%                      & 85\% & 27\%  & 56\%                  & 66\% & 39\% & 52\%                     & 0.62  & 0.62 & 0.62              \\
{5}                        & 56\% & 79\%  & 68\%                      & 92\% & 31\%  & 61\%                  & 70\% & 45\% & 57\%                     & 0.65  & 0.65 & 0.65              \\
{6}                     & 55\% & 80\%  & 68\%                      & 93\% & 26\%  & 59\%                  & 70\% & 39\% & 54\%                     & 0.69  & 0.69 & 0.69              \\
{7}                             & 56\% & 77\%  & 69\%                      & 70\% & 45\%  & 57\%                  & 69\% & 45\% & 57\%                     & 0.67  & 0.67 & 0.67              \\
{8}                        & 65\% & 47\%  & 58\%                      & 56\% & 79\%  & 68\%                  & 92\% & 31\% & 61\%                     & 0.69  & 0.69 & 0.69              \\
{9}                     & 55\% & 72\%  & 63\%                      & 88\% & 30\%  & 59\%                  & 68\% & 42\% & 55\%                     & 0.69  & 0.69 & 0.69              \\
{10}                             & 57\% & 84\%  & 70\%                      & 93\% & 32\%  & 63\%                  & 71\% & 47\% & 59\%                     & 0.68  & 0.68 & 0.68              \\
\bottomrule
\end{tabular}
}
\end{table*}

The classes we consider for the multi-class categorization problem (\RQ{2} might be incomplete: It is possible that we do not consider some relevant categories of issues. To mitigate this threat, we avoided defining such categories based on our personal experience, but we relied on a state-of-the-art taxonomy \cite{truelove2021we}.
A key threat regards the features considered for step 2 (and, thus, to answer \RQ{2}). It is worth noting that we relied on features that proved to be useful in other contexts (\eg categorization and clustering of mobile app reviews \cite{chen2014ar, scalabrino2017listening}), and we also augmented them with video-based features. Still, it is possible that a different set of features leads to better results.
As for clustering (both \RQ{3} and \RQ{4}), it is possible that we chose sub-optimal parameters (\ie $\epsilon$ values). To reduce this threat, we used a rigorous procedure \cite{ozkok2017new} to set these values for each tested video game.

Finally, \textbf{threats to external validity} concern the generalizability of our findings. Our test set is composed of gameplay videos related to only 3 video games. We could not select videos from a more diverse set of video games because we needed multiple segments related to the same game areas to address \RQ{3} and \RQ{4}. However, it is worth noting that we also report in \tabref{tab:rq2binary} and \tabref{tab:rq2multiclass} the results of a 10-fold cross validation performed on the training set, which, differently from the test set, is composed of videos from many video games 110, specifically. Nevertheless, we acknowledge that most of our results are not necessarily generalizable to the vast quantity of video game genres and video games available in the market. 

\REV{We believe that the variety of video games is not as relevant as the variety and type of streamers involved. GELID heavily relies on (captioned) spoken content for segmentation and categorization. To this end, having verbose streamers could benefit GELID. On the other hand, the video game selection might mostly impact the two clustering-related steps: For example, games with many graphically similar levels or areas might deceive GELID while it cluster segments.}
