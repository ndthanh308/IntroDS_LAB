\documentclass[%
  reprint,
 superscriptaddress,
%groupedaddress,
%unsortedaddress,
%runinaddress,
%frontmatterverbose, 
%preprint,
%preprintnumbers,
%nofootinbib,
%nobibnotes,
%bibnotes,
 amsmath,amssymb,
 aps,
pra,
%prb,
%rmp,
%prstab,
%prstper,
%floatfix,
]{revtex4-2}
\usepackage{orcidlink}
\usepackage{graphicx}
\usepackage{physics}
\usepackage{xspace}
\newcommand{\ps}{phase space\xspace}
\usepackage[encapsulated]{CJK}
%\usepackage{dsfont}
\usepackage{hyperref}
\usepackage{color}
\newcommand{\VEC}[1]{{\mbox{\boldmath${#1}$}}}
\newcommand{\RED}[1]{{\color{red}#1}}
\newcommand{\BLUE}[1]{{\color{blue}#1}}
\newcommand{\GREEN}[1]{{\color{green}#1}}
\usepackage[normalem]{ulem}
\newcommand{\emphCaption}[1]{{\bf{#1}}}
%%%%%%%%%%%%%%%%%%%%%%%%%%%%%%%%%%%%%%%%%%%%%%%%%%%%%%%%%%%%%%%%%%%%%%%
%	Symbols << and >>
%%%%%%%%%%%%%%%%%%%%%%%%%%%%%%%%%%%%%%%%%%%%%%%%%%%%%%%%%%%%%%%%%%%%%%%
\makeatletter
\newsavebox{\@brx}
\newcommand{\llangle}[1][]{\savebox{\@brx}{\(\m@th{#1\langle}\)}%
  \mathopen{\copy\@brx\kern-0.5\wd\@brx\usebox{\@brx}}}
\newcommand{\rrangle}[1][]{\savebox{\@brx}{\(\m@th{#1\rangle}\)}%
  \mathclose{\copy\@brx\kern-0.5\wd\@brx\usebox{\@brx}}}
\makeatother

\newcommand{\Llangle}{\big\langle\!\!\:\!\big\langle}
\newcommand{\Rrangle}{\big\rangle\!\!\:\!\big\rangle}

%%%%%%%%%%%%% Appendix %%%%%%%%%%%%%%%%
\usepackage{forloop,ifthen} % for loops and if statements
\newcommand{\emphLabel}[1]{\textbf{({#1})}}
\usepackage{xifthen}% provides \isempty test
\newcommand{\refAppendix}[6]{#1
  \ifthenelse{\isempty{#2}}%
    {}% if #2 is empty
    {\protect\cite{#2}}% if #1 is not empty
    #3\protect\ref{#4}#5#6\xspace
}
%%% #1 contextual text, #2 previous citation, #3 contextual text, #4 section in Appendix, #5 contextual text, #6 TEXT (always shown)

\def\tr{\operatorname{tr}}
\def\Pr{\operatorname{Pr}}
\def\sq{\operatorname{sq}}
\def\SEP{\operatorname{SEP}}
\def\Ent{\operatorname{Ent}}
\def\PPT{\operatorname{PPT}}
\def\supp{\operatorname{supp}}
\def\LOCC{\operatorname{LOCC}}
\def\T{\operatorname{T}}

\begin{document}

\title{Photon Creation viewed from Wigner's Phase Space Current Perspective:\\
The Simplest Possible Derivation of a Lindblad Superoperator Form}

\author{Ole Steuernagel\orcidlink{0000-0001-6089-7022}}
\email{Ole.Steuernagel@gmail.com}
\affiliation{Institute of Photonics Technologies, National Tsing Hua University, Hsinchu 30013, Taiwan}
\affiliation{Tulane University, New Orleans, LA 70118, USA}

\author{Ray-Kuang Lee\orcidlink{0000-0002-7171-7274}}
\email{rklee@ee.nthu.edu.tw}
\affiliation{Institute of Photonics Technologies, National Tsing Hua University, Hsinchu 30013, Taiwan}
\affiliation{Department of Physics, National Tsing Hua University, Hsinchu 30013, Taiwan}
\affiliation{Physics Division, National Center for Theoretical Sciences, Taipei 10617, Taiwan}
\affiliation{Center for Quantum Technology, Hsinchu 30013, Taiwan}
 
\date{\today}
\begin{abstract}
  For the study of quantum dynamics the use of Wigner's \ps representation can be rewarding. It
  describes the state by Wigner's real-valued distribution~$W$ and its dynamics by a vector field in
  \ps, the Wigner current~$\VEC{J}$. Basically, only the Wigner representation can be used for this
  type of visual study of quantum dynamics so conveniently and directly. What does it teach us about
  the most fundamental ingredient of quantum dynamics, the quantum jump between energy levels? Quite
  a lot, as it turns out.
\end{abstract}

\maketitle

\section{Motivation}

In quantum theory, the dynamics of systems with a discrete spectrum is progressing through quantum
jumps. We are trying to formulate this in terms of the \ps current of the Wigner distribution and do
something absurd: we try to describe a quantum jump in a continuous fashion -- specifically -- the
addition of a photon to a light field.

This should not work. It would be astonishing if one hundred and twenty years after the inception of
the photon concept a continuous transition description could be formulated.

But how does the description break down?

We feel that sharing our analysis will illuminate the mathematical description of the creation of a
single photon, traditionally done through a bosonic field creation operator, from a new
perspective. Maybe this can throw extra light on the question of how to understand the differences
between classical and quantum dynamics.

We will first, in Sect.~\ref{sec:Wign-Distr-Continuity}, remind the reader of Wigner's
representation of quantum theory and contrast it with the Schr\"odinger--von Neumann representation.
Then, in Sect.~\ref{sec:vec-J-from_Moyal}, we will derive and prove, for hamiltonians that can be
Taylor-expanded, that in the Wigner representation a \ps current,~$\VEC{J}$, fulfilling a continuity
equation, must exist. In Sect.~\ref{sec:Ja}, we then attempt to apply this know\-ledge about the
current~$\VEC{J}$ to the question of its construction in order to describe quantum jumps
associated with adding photons to or removing them from a light field. Failing in this attempt, we
show, in Sect.~\ref{secsec:_Lindblad}, that the failure can only be remedied by constructing the
correct Lindblad superoperator, after this, we put our results into context and conclude.

\section{Wigner distribution and its Continuity Equation~\label{sec:Wign-Distr-Continuity}}

The regular time-evolution of Wigner's quantum \ps distribution $W(x,p,{t})$, for a one-dimensional
continuous system, is governed by the \ps current~$\VEC{J}$ and obeys a continuity
equation~\cite{Oliva_PhysA17}
\begin{eqnarray}\label{eq:continuity}
\frac{\partial W(x,p,{t})}{\partial {t}} + \VEC{\nabla} \cdot \VEC{J}(x,p,{t}) = 0 \; .
\end{eqnarray}
Here, $\VEC{\nabla} = (\partial/\partial_x, \partial/\partial_p)$ is the \ps divergence operator
with respect to position~$x$ and momentum~$p$, ${t}$ is time, and~$\VEC{J}=(J_x,J_p)$ has two
components and is a functional of $W$ and the system hamiltonian~$H(x,p)$.

\subsection{Why use the Wigner Representation?~\label{sec:why-wign-distr}}

The Wigner distribution's evolution seems always to be governed by a continuity equation, see
Sect.~\ref{sec:vec-J-from_Moyal}, below.

This gives us a differential operator,~$\VEC{\nabla} \cdot \VEC{J}$, as a function of~$x$,~$p$
and~$t$, to describe the infinitesimal changes in $W$, and which we can hope to integrate over time
to describe larger changes in states.

That is why we choose the Wigner distribution and its continuity equation as the starting point
for our investigation of photon addition.


\subsection{Contrast with Schr{\"o}dinger Equation}

The conventional approach of using Schr\"odinger's or von~Neumann's equation forces us to
investigate complex-valued wave functions to represent the state of the system, this already poses a
stumbling block for the direct visualization of state changes.

Wigner's representation of quantum theory, encapsulated by Eq.~(\ref{eq:continuity}), has the great
advantage of allowing us to visua\-lize quantum dynamics in \ps more directly since $W$ is
real-valued~\cite{Kakofengitis_EPJP17}.
Additionally, Wigner's representation permits the study of its real-valued \ps current  $\VEC{J}$
and formal line integrals along $\VEC{J}$ which yield momentary snapshots of `field lines',
reminiscent of classical phase portraits~\cite{Ole_PRL13}.

Wigner's continuity equation~(\ref{eq:continuity}) is the \ps equivalent of the conventional von
Neumann's equation~\cite{Cabrera_PRA15}.

Recent experimental work shows that the reconstruction of Wigner's \ps current~$\VEC{J}$ and its
effects on the evolution of the system can be studied with high resolution~\cite{Chen_ArXiv23}.

\section{${\VEC J}(x,p,{t})$ from Moyal's bracket~\label{sec:vec-J-from_Moyal}}

We will now remind the reader of how Wigner's and Schr{\"o}dinger's representation of quantum theory
are connected mathematically~\cite{Zachos_book_21}.

Consider a single-mode operator,~$\hat O$, given in coordinate
representation~$\langle x-y| \hat O | x+y \rangle = O(x-y,x+y)$. To map to Wigner's \ps
representation we employ the
Wigner-transform,~${\cal W}[\hat O]$~\cite{Hancock_EJP04,Cohen_LectureNotes18,Zachos_book_21},
\begin{align}\label{eq:WignerWeyl_Trafo}
  {\cal W}[\hat O](x,p) =
  \int_{-\infty}^\infty dy\; O(x-\frac{y}{2},x+\frac{y}{2})\; {\rm e}^{\frac{{\rm i}}{\hbar} p y}\, .
\end{align}

If $\hat O$ is a normalized single-mode density matrix~$\hat \rho$, then the associated
\emph{normalized} distribution in the Wigner repre\-sentation
is~$W(x,p) \equiv {\cal W}[\hat \rho]/(2 \pi \hbar)$.

Assuming that the hamiltonian ${\cal W}[ \hat H(\hat x, \hat p) ] = H(x,p)$ is smooth enough, namely,
has a global Taylor expansion, the Wigner transform of the von~Neumann time evolution equation
\begin{equation}\label{eq:W_of_vNeumann}
  {\cal W}\left[ \frac{\partial \hat \rho}{\partial {t}} = \frac{1}{{\rm i}\hbar} [\hat H, \hat
    \rho] \right]
  \end{equation}
  is
\begin{equation}\label{eq:moyal_motion}
  \frac{\partial W}{\partial {t}} = \{\!\!\{ {H} , W \}\!\!\}  = \frac{1}{\rm i \hbar} \left( H \star W - W \star H\right)\; ,
\end{equation}
where the $\star$-product is defined in Eq.~(\ref{Eq:GroenewoldStar}) and `Moyal's
bracket',~$\{\!\!\{ {H} , W \}\!\!\}$,~\cite{Groenewold_Phys46,Moyal_MPCPS49,Zachos_book_21} is of
`Moyal-Sine' form
\begin{align}\label{EqMoyalBraket}
  \{\!\!\{ f, g\}\!\!\} 
  %\equiv & \frac{f \star g - g \star f}{i\hbar}    \notag\\
     = \frac{2}{\hbar} f(x,p) \sin\!\!\left[\! \frac{\hbar}{2} \!\!\left( 
        \overleftarrow{\frac{\partial}{\partial x}} \overrightarrow{\frac{\partial}{\partial p}}
         - \overleftarrow{\frac{\partial}{\partial p}} \overrightarrow{\frac{\partial}{\partial x}}
  \right)\!\!\right] g(x,p) \; ,
\end{align}
with arrows indicating the `direction' of differentiation:
$f\overrightarrow{\frac{\partial}{\partial x}} g = g\overleftarrow{\frac{\partial}{\partial x}} f =
f \frac{\partial}{\partial x} g$.

Let us use the shorthand notation~$\frac{\partial}{\partial_z} = \partial_z$ and let us consider the
operator
$\overleftrightarrow{\partial} = \!\!\left( \overleftarrow{\partial_x} \overrightarrow{\partial_p} -
  \overleftarrow{\partial_p} \overrightarrow{\partial_x} \right)\!\!$ inside the Moyal
bracket~$\frac{2}{\hbar}\sin(\frac{\hbar}{2}\overleftrightarrow{\partial})$~(\ref{EqMoyalBraket}). It
is applied to $H$ and $W$ at least once and therefore enforces that all terms of
$\{\!\!\{ {H} , W \}\!\!\} = H \overleftrightarrow{\partial} W + {\cal
  O}(\overleftrightarrow{\partial}^3)$ contain derivatives of~$W$.

It can be useful to decompose Moyal's Sine-bracket~(\ref{EqMoyalBraket}) into
Groenewold~\cite{Groenewold_Phys46,Zachos_book_21} star-products,~$\star$, given by
\begin{align}\label{Eq:GroenewoldStar}
  \star &\equiv \exp\left[\frac{i\hbar}{2} 
          \overleftrightarrow {\partial} \right]
          = \sum_{n=0}^{\infty} \frac{(i\hbar \overleftrightarrow {\partial})^n}{2^n n!}  
          \; ;
\end{align}
see Eq.~(\ref{eq:moyal_motion}). A series expansion of Eq.~(\ref{eq:moyal_motion})
using Eq.~(\ref{EqMoyalBraket}) or~(\ref{Eq:GroenewoldStar}) reveals
that to first order the Moyal bracket equals the Poisson bracket of classical mechanics:
\begin{align}\label{Eq:classicalLimit}
  \lim_{\hbar \downarrow 0} \{\!\!\{ {H} , W \}\!\!\} = H \overleftrightarrow{\partial} W = \{H , W \} \; .
\end{align}


\subsection{Groenewold's $\star$-product is more fundamental than Moyal's
  bracket \label{subsec:GroenewoldFundamental}}

Moyal's bracket is a specific combination of terms using Groenewold's $\star$-product.  Whereas
Moyal's bracket translates the commutator of the von Neumann equation to \ps, Groenewold's
$\star$-product is a translation of the operator product of regular quantum theory into \ps
language.  To be explicit, the non-commutative nature of the composition of Hilbert space operators
manifests itself in the complicated and non-commutative structure of Groenewold's $\star$-product in
the Wigner representation. The $\star$-product is the key-ingredient to map quantum behaviour into
\ps.

Clearly, Groenewold's $\star$-product is more fundamental than Moyal's bracket.

\subsection{Proof that $\{\!\!\{ {H} , W \}\!\!\} = -{\VEC \nabla \cdot
    J}(x,p,{t})$ \label{subsec:proofMoyalDiv}}

If~$H(x,p)$ has a global Taylor expansion, evolution equation~(\ref{eq:moyal_motion}) can be
rewritten as the divergence of Wigner's \ps current,~$\VEC
J$~\cite{Oliva_Kerr_18,Oliva_PhysA17}, %yielding
see Eq.~(\ref{eq:continuity}).


The lowest order yields
$ H \overleftrightarrow{\partial} W = (\partial_x H) ( \partial_p W) - (\partial_p H) (\partial_x W)
= \{H,W\} = \partial_p(W \partial_x H) - \partial_x(W \partial_p H) \equiv -{\VEC \nabla \cdot
  J_1}(x,p,{t}) $, where~$\{H,W\}$ is the Poisson bracket of classical mechanics and the \ps current
${\VEC J_1 = (W \partial_p H,-W \partial_x H)^T}$ therefore is of classical form.

Because of the linearity of the operators involved we can now formally apply the same reasoning to
the higher order terms: start out with
$H \overleftrightarrow{\partial}^{N+1} W = \sum_{m,n} H_m \overleftrightarrow{\partial}^{N} W_n =
{\VEC \nabla} \cdot \sum_{m,n} \tilde H_m \overleftrightarrow{\partial}^{N} \tilde W_n = {\VEC
  \nabla} \cdot ( \sum_{m,n} \tilde H^x_{m} \overleftrightarrow{\partial}^{N} \tilde W^x_{n},
\sum_{m,n} \tilde H^p_{m} \overleftrightarrow{\partial}^{N} \tilde W^p_{n} )^T \equiv -{\VEC \nabla}
\cdot {\VEC J_{N+1}} $.

Similarly, for multimode systems, tracing out degrees of freedom commutes with pulling out the
gradient opera\-tor in front. For example, consider a two-mode system with modes $a$ and $b$ and let
us focus on mode $a$, tracing out~$b$. This gives us
${\rm Tr}_b \{H_{ab} \overleftrightarrow{\partial_{ab}} W_{ab}\} = {\rm Tr}_b\{ - \VEC \nabla_{ab}
\cdot \VEC j_{ab} \} = \VEC \nabla_{a} \cdot {\rm Tr}_b\{ - \VEC \nabla_{b} \cdot \VEC j_{ab} \}$,
where $\VEC j$ has two components in mode $a$, as desired. This result generalizes to terms of higher
order in~$\overleftrightarrow{\partial_{ab}}$ using the arguments presented in the previous
paragraph.

What about non-conservative dynamics, as encountered in dissipative systems? The multimode
considera\-tions just given, tracing out (the bath) modes, cover this case. Alternatively, for a
standard weakly coupled dissipative system the Wigner current has been derived~\cite{Braasch_PRA19}
in analy\-tical form and been investigated experimentally~\cite{Chen_ArXiv23}.

\subsection{Remark on the `--' sign in the Moyal bracket
  $\{\!\!\{ {H} , W \}\!\!\} = \frac{1}{\rm i \hbar} \left( H \star W \right.$ `--'
  $\left. W \star H\right)$~\label{subsec:MoyalMinus}}

It is the minus-sign in the Moyal bracket which is responsible for removing all terms from it that
do not carry derivatives of~$W$. This is, of course, inherited from the commutator in von~Neumann's
equation.  But there it might appear to be less fundamental since it might be understood to `just
arise' from the combination of phases of the unitary evolution operators for the bra- and ket-part
which form the commutator.

Viewing it from the Wigner \ps perspective, knowing that ${\partial_t} W $ has to have terms of the
form~$-{\VEC \nabla} \cdot {\VEC J}$, reinforces that this is a fundamental feature.

Let us try to understand better still what goes wrong if we do not enforce that time changes in $W$
have to be described by terms carrying derivatives of~$W$.  Let us look at a few very simple
cases.

Assume, for instance, that we were to find ${\partial_t} W = \epsilon(x,p) W$,
where~$ \epsilon(x,p)$ is a polynomial. Trying to shoehorn this into the form
${\partial_t} W = - {\VEC \nabla} \cdot {\VEC J}$ we might attempt something like this:

Assuming $\epsilon = p$, we might attempt to use $H = \frac{\epsilon}{2} W / (\partial_x W)$ to
generate a term of the form $J_x = - W H$. But, this yields
$- \partial_x J_x = p W + \frac{p}{2} W^2 \partial_x(1/\partial_xW)$, so we end up with non-linear
terms and derivatives of singular functions: our attempt fails.

Even something as simple as trying to shoehorn a constant term
${\partial_t} W = c W = - \partial_x J_x$ into the form of a current, such as $J_x= -c x W$,
fails. It generates an escalating sequence such as $c W= - \partial_x(J_x) - c x \partial_x W$, etc.

Essentially, we just spent futile effort on braking out of the straight-jacket of polynomials (and
their anti-derivatives) forming systems of linearly independent functions.

% Figure environment removed

\section{Photon addition: ${\cal W}[\hat a^\dag \hat \rho \; \hat a ]$ and
  $\VEC J_{\hat a^\dag \hat \rho \; \hat a}$ \label{sec:Ja}}

Let us now consider a bosonic single mode system, specifi\-cally a harmonic oscillator whose creation
operator is~$\hat a^\dag(\hat x, \hat p)$ $=$
$\sqrt{\frac{m \omega }{2 \hbar }} (\hat x-\frac{i \hat p}{m \omega })$. This can describe
an optical mode and its excitations. 

Setting $m$=1, $\omega$=1 and $\hbar$=1, the corresponding \ps functions are
${\cal W}[\hat a^\dag]=a^*(x,p)=(x-ip)/\sqrt{2}$ and ${\cal W}[\hat a]=a(x,p)=(x+ip)/\sqrt{2}$.  The
associated photon addition operator applied to the Wigner distribution (see
Fig.~\ref{fig:W_0_and_1}) therefore is
\begin{subequations}
    \label{eq:addPhotonWoverallEq}
\begin{align}
    \label{eq:addPhotonW_singularPart}
  \!\! 
  \frac{{\cal W}[\hat a^\dag \hat \rho_0 \; \hat a ]}{2 \pi \hbar } = \; & a^*  \star W_0 \star a =  \; \frac{1}{2} \left(p^2+x^2+1\right) W_0 + \\ \label{eq:addPhotonW}
& + \left[ -\frac{1}{2} {\VEC \nabla} \cdot \left(\begin{array}{c} x W_0
                                                       \\
                                                       p W_0
                                                     \end{array}\right)
   + \frac{\VEC \Delta W_0}{8} \right]    \; .
  \end{align}
  \end{subequations}

The differential operator in the square bracket~(\ref{eq:addPhotonW})
obviously has the desired form of $ \VEC{\nabla} \cdot {\VEC J} $ where
$ {\VEC J} = \frac{1}{8}([-4x + \partial_x]W_0 , [-4p + \partial_p]W_0 )^T$
and ${\VEC \Delta W_0} = \VEC{\nabla} \cdot (\VEC{\nabla} W_0)$ the Laplacian
in \ps.

We now seek a current ${\VEC J}$ such that
$ \VEC{\nabla} \cdot {\VEC J} \approx - \Delta {\cal W}[\hat a^\dag \hat \rho_0 \; \hat a ] / \Delta
{t}$ can be integrated over one unit of time giving us
expression~(\ref{eq:addPhotonWoverallEq}):~$W_0 - \int_0^1 dt \VEC{\nabla} \cdot {\VEC J} = a^*
\star W_0 \star a$.

We do not have in mind to use a `trivial' interpolation
$(1-t) W_0 + t \frac{{\cal W}[\hat a^\dag \hat \rho \; \hat a ]}{2 \pi \hbar} $, with $t=0,...,1$.

The term~(\ref{eq:addPhotonW_singularPart}), ${\frac{1}{2}(p^2+x^2+1)} W_0$, is troubling, it cannot be
rendered in the form $\VEC{\nabla} \cdot \VEC{J}(x,p)$, see remarks~\ref{subsec:MoyalMinus} on the
minus-sign in Moyal's bracket:

If we attempt moving the polynomial terms of~(\ref{eq:addPhotonW_singularPart}) from the outside to
the inside of a derivative operator we end up increasing the order of the polynomials in $x$ and $p$
that are involved.

For instance $ p^2 W = \partial_x \; (p^2 x \; W) - p^2 x \; \partial_x W$. But the second term shows
an unwelcome increase to a higher order polynomial (here $p^2 x$). Its polynomial degree, if we
similarly try to move it inside a derivative, will increase yet further; ad infinitum.


\subsection{Compensating unphysical terms~\label{sec:enforce_Lindbald}}

Similarly to $a^*  \star W \star a$, let us determine the \ps form for photon removal
from the light field~$W$
\begin{subequations}
    \label{eq:removePhotonWoverallEq}
  \begin{align}\label{eq:a_W_aD_singular}
   a \star W \star a^* = & \; \frac{1}{2} \left(p^2+x^2-1\right) W + \\
& + \left[ +\frac{1}{2} {\VEC \nabla} \cdot \left(\begin{array}{c} x W
                                                       \\
   \label{eq:a_W_aD_divergence}
                                                       p W
                                                     \end{array}\right)
   + \frac{\VEC \Delta W}{8} \right]    \; .
\end{align}
\end{subequations}

Combining Eqs.~(\ref{eq:addPhotonW_singularPart}) and (\ref{eq:a_W_aD_singular}), we find, neither
$a^* \star W \star a + a \star W \star a^*$ nor $a^* \star W \star a - a \star W \star a^*$ can be
given in the form of the divergence of a current, see remarks~\ref{subsec:MoyalMinus}.

Given that $a^*(x,p)$ and $a(x,p)$ are complex functions (representing the fact that $\hat a^\dag$
and $\hat a$ are not hermitian) we cannot use them by themselves, let us therefore consider the
simplest extension possible, for finding a current. We try quadratic combinations of $a^*$ and
$a$ to compensate for the unphysical terms identified above:

We know that the \ps dynamic of the quantum harmonic oscillator behaves
classically~\cite{Oliva_PhysA17}:
 \begin{align}\label{eq:J_HarmOSC}
   a^* \star a \star W - W \star a^* \star a  = {\VEC \nabla} \cdot \left(\begin{array}{c} p \; W
                                                       \\
                                                      - x \; W
                                                                          \end{array}\right)    \; .
\end{align}
Instead of subtracting such terms, adding them gives us
 \begin{align}\label{eq:_nW+Wn}
   \!\!\!\!   a^* \star a \star W + & W \star a^* \star a  = \\ \label{eq:_nW+Wn_b}
   & \; \left(p^2+x^2-1\right) W - \frac{\VEC \Delta W}{4}    \; ,
% \end{align}
% and
% \begin{align}
   \\  \label{eq:_nDW+WnD}
   \hspace{-0.7cm}\text{and}\qquad  
   a \star a^* \star W + & W \star a \star a^*  = \\ \label{eq:_nDW+WnD_b}
                                    & \; \left(p^2+x^2+1\right) W - \frac{\VEC \Delta W}{4}    \; .
\end{align}
We see that expressions~(\ref{eq:_nW+Wn_b}) and~(\ref{eq:_nDW+WnD_b}) can
compensate the terms~(\ref{eq:addPhotonW_singularPart}) and~(\ref{eq:a_W_aD_singular}).
\\
Namely, (only) the combinations
 \begin{align}\label{eq:aD_W_a__Lindblad1}
   2  & a^*  \star W \star a - (  a \star a^* \star W +  W \star a \star a^* )=
   \\   \label{eq:a_W_aD__Lindblad1b}
&  \left[ - {\VEC \nabla} \cdot \left(\begin{array}{c} x W
                                                       \\
                                                       p W
                                                     \end{array}\right)
   + \frac{\VEC \Delta W}{2} \right] 
   \; , \\
% \end{align}
% and
% \begin{align}
  \label{eq:a_W_aD__Lindblad2}
   \hspace{-0.4cm}\text{and}\qquad   2  & a \star W \star a^* - (  a^* \star a \star W +  W \star a^* \star a )=
   \\\label{eq:a_W_aD__Lindblad2b}   
&  \left[ {\VEC \nabla} \cdot \left(\begin{array}{c} x W
                                                       \\
                                                       p W
                                                     \end{array}\right)
   + \frac{\VEC \Delta W}{2} \right] 
   \; ,
\end{align}
can be rendered as~$-{\VEC \nabla} \cdot {\VEC J_\mp}$, respectively, where
 \begin{align}\label{eq:J_minusplus}
  {\VEC J_\mp} = - \left(\begin{array}{c} \mp x + \frac{1}{2} \partial_x \; W
                                        \vspace{0.1cm} \\
                \mp p + \frac{1}{2} \partial_p \; W
                                                                          \end{array}\right)    \; .
\end{align}

\section{Lindblad form \label{secsec:_Lindblad}}

As such, our result for~${\VEC J_\mp}$ of Eq.~(\ref{eq:J_minusplus}) is not new, expressions of open
systems dynamics in \ps have been derived by Wigner-transforming standard Lindblad master equation
terms~\cite{Cabrera_PRA15,Braasch_PRA19}
\begin{eqnarray}
\frac{d \rho}{dt}&=&-\frac{i}{\hbar}[H,\rho]+\frac{\gamma}{2}(\bar{n}+1)\left(2 a\rho a^{\dagger}-a^{\dagger}a\rho-\rho a^{\dagger}a\right)\cr
&&+\frac{\gamma}{2}\bar{n}\left(2a^{\dagger}\rho a-a a^{\dagger}\rho-\rho a a^{\dagger}\right).
\label{mastereq}
\end{eqnarray}
Including natural units, the currents~$ \vec{J}_{\text{env}}$, associated with the $\gamma$-terms
in~(\ref{mastereq}) coupling to the environment, have the concise form~\cite{Braasch_PRA19} (also
see experiment~\cite{Chen_ArXiv23})
\begin{eqnarray}
  \label{eq:Jenv}
\!\!\!\!
\!\!\!\!
  \vec{J}_{\text{env}} &=& -\frac{\gamma}{2} W \left(\begin{array}{c} x
                                                       \\
                                                       p
                                                     \end{array}\right) -\frac{\gamma}{2} \frac{\hbar}{\omega_0} (\overline{n}+\frac{1}{2}) \left( \begin{array}{c}  \partial_x W
                                                                                                                                                     \\  \partial_p W
                                                                                                                                                   \end{array}\right), \\
                        &\equiv& \vec{J}_{\text{damp}} + \vec{J}_{\text{diff}} \; .
\end{eqnarray}

What is new, however, is our proof that the continuity equation of \ps quantum
dynamics~(\ref{eq:continuity}) enforces this form.

\subsection{Lindblad form put into context\label{secsec:_LindbladLiterature}}

For the specific coupling for photon creation~(\ref{eq:addPhotonWoverallEq}) or
annihilation~(\ref{eq:removePhotonWoverallEq}), our analysis essentially gives an abstract
rederivation of the well known Lindblad form~(\ref{eq:aD_W_a__Lindblad1})
and~(\ref{eq:a_W_aD__Lindblad2}) of a quantum system coupled to a simple bath.

We think it is astonishing that the mere insistence to express dynamical terms as the divergence of
Wigner's \ps current,~$-{\VEC \nabla} \cdot {\VEC J}$, should have such far reaching and at the same
time specific consequences.

This is an interesting observation since the standard consistency requirements imposed on the
coupling terms~\cite{Lindblad_CMP76,Pearle_EJP12}, when using the Schr\"odinger--von Neumann
representation, do not bear resemblance to the requirement to be of the form of the divergence of a
current.

Of course, our analysis was restricted to one case (photon addition to, or subtraction from, a light
field) and, unsurprisingly, it is neither more general nor does it allow us to check on more subtle
consistency requirements, such as those elucidate in Ref.~\cite{Salmilehto_PRA12}.  It also does not
enforce the fluctuation-dissipation theorem, namely, that all terms have to appear simultaneously,
as in Eq.~(\ref{eq:Jenv}) (also see~\cite{Chen_ArXiv23}). 


\section{Conclusions and Outlook\label{sec:Conclusion}}

We have found that the explicit requirement that quantum dynamics (in \ps) has to fulfil continuity
equation~(\ref{eq:continuity}) enforces that photon addition to, or subtraction from, a system has
to obey a Lindblad form~\cite{Braasch_PRA19,Chen_ArXiv23}. For this special case, the derivation
consists of a few lines of transparent calculations of extreme simplicity.

Using the standard (von Neumann) approach yields  more subtle implicit requirements, namely, trace-,
hermiticity- and complete positivity-conservation~\cite{Pearle_EJP12}.

When contrasting the two approaches we feel, in the light of the simplicity of the \ps approach, it
is fair to argue that the Lindblad form could have been discovered earlier, if Wigner's \ps
representation would have been widely adopted sooner~\cite{Zachos_book_21}.

Wigner's \ps representation of quantum theory should probably be investigated further and in greater
depth.

\section*{Acknowledgements}

O.~S. thanks Hsien-Yi Hsieh for fruitful feedback.  This work is partially supported by the Ministry
of Science and Technology of Taiwan (Nos 112-2123-M-007-001, 112-2119-M-008-007,
112-2119-M-007-006), Office of Naval Research Global, US Army Research Office (ARO), and the
collaborative research program of the Institute for Cosmic Ray Research (ICRR), the University of
Tokyo.

%\clearpage

%\setcounter{page}{1}
\setcounter{section}{0}
%\renewcommand{\thesection}{A.~\arabic{section}}
\renewcommand{\thesection}{Appendix~\arabic{section}}
\renewcommand{\thefigure}{A.~\arabic{figure}}
%\renewcommand{\thefigure}{\arabic{figure}}
\setcounter{figure}{0}
\setcounter{equation}{0}
\renewcommand{\theequation}{A.\arabic{equation}}%.\arabic{figure} 

%\onecolumngrid
%\vspace{\columnsep}

  
% \begin{center}
%   {   \large \bf -- Appendix -- \\ \vspace{0.25cm}
% %    % Figure environment removed

% {Experimental Reconstruction of Wigner Phase Space Current}
% % Experimental Demonstration of Topological Charge Protection in Wigner Current}
% }
% %   --Supplement--}%al Material}
% \\ \vspace{0.25cm}

% {Yi-Ru Chen, Hsien-Yi Hsieh, Jingyu Ning, Hsun-Chung Wu, Hua Li Chen, You-Lin Chuang, Popo Yang, Ole
%   Steuernagel, Chien-Ming Wu, and Ray-Kuang Lee}

%  \end{center}


%\onecolumngrid  




% \setcounter{figure}{0}

% \section{Wigner Current Reconstruction\label{sec:Appendix_ExpDensityMatrix}}

% Note, the Wigner current stagnation point at the origin of an OPO system has an orientation winding
% number topolo\-gi\-cal charge of $\omega = -1$. This orientation winding number is
% defined as ~\cite{Ole_PRL13}  
% \begin{eqnarray}
%   \omega({\cal L}(x,p)) \equiv \frac{1}{2\pi}\oint_{{\cal L}(x,p)} \,d\phi,
%   \label{eq:WindingNumber}
% \end{eqnarray}
% and tracks the Wigner current's orientation along a simple closed-loop ${\cal L}$, i.e., around a single
% stagnation point $(x,p)$, see Fig.~\ref{fig:Stagnation_Point_Current_Schematic}.

\bibliography{AnnihilationCreation_continuous__W_Current_J.bbl}

\end{document}

%%% Local Variables:
%%% mode: latex
%%% TeX-master: t
%%% End:
