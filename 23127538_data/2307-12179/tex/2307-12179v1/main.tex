\documentclass[10pt,twocolumn,letterpaper]{article}

%%%%%%%%% PAPER TYPE  - PLEASE UPDATE FOR FINAL VERSION
%\usepackage[review,algorithms]{wacv}      % To produce the REVIEW version for the algorithms track
%\usepackage[review,applications]{wacv}      % To produce the REVIEW version for the applications track
%\usepackage{wacv}              % To produce the CAMERA-READY version
\usepackage[pagenumbers]{wacv} % To force page numbers, e.g. for an arXiv version

% Include other packages here, before hyperref.
\usepackage{graphicx}
\usepackage{amsmath}
\usepackage{amssymb}
\usepackage{booktabs}


\usepackage{times}
\usepackage{epsfig}

\usepackage{diagbox}
\usepackage{threeparttable}
\usepackage[table]{xcolor}
\usepackage{multirow}
% Include other packages here, before hyperref.


% If you comment hyperref and then uncomment it, you should delete
% egpaper.aux before re-running latex.  (Or just hit 'q' on the first latex
% run, let it finish, and you should be clear).

\def\httilde{\mbox{\tt\raisebox{-.5ex}{\symbol{126}}}}
\usepackage[pagebackref,breaklinks,colorlinks]{hyperref}


% Support for easy cross-referencing
\usepackage[capitalize]{cleveref}
\crefname{section}{Sec.}{Secs.}
\Crefname{section}{Section}{Sections}
\Crefname{table}{Table}{Tables}
\crefname{table}{Tab.}{Tabs.}
%%%%%%%%% PAPER ID  - PLEASE UPDATE
\def\wacvPaperID{839} % *** Enter the WACV Paper ID here
\def\confName{WACV}
\def\confYear{2024}



\begin{document}

%%%%%%%%% TITLE - PLEASE UPDATE
\title{Leveraging Knowledge Graphs for  Zero-Shot\\ Object-agnostic State Classification}

\author{Filipos Gouidis$^{1,2}$, Theodore Patkos$^1$,  Antonis Argyros$^{1,2}$,Dimitris Plexousakis$^{1,2}$ \\
$^1$Institute of Computer Science, FORTH, Greece\\                 
$^2$Computer Science Department, University of Crete, Greece\\
{\tt\small \{gouidis,patkos,argyros,dp\}@ics.forth.gr}
} 

\maketitle



%%%%%%%%% ABSTRACT
\begin{abstract}
We investigate  the problem of Object State Classification (OSC) as a zero-shot learning problem. Specifically, we propose the first Object-agnostic State Classification (OaSC) method that infers the state of a certain object without relying on the knowledge or the estimation of the object class. In that direction, we capitalize on Knowledge Graphs (KGs) for structuring and organizing knowledge, which, in combination with visual information, enable the inference of the states of objects in object/state pairs that have not been encountered in the method's training set. 
A series of experiments investigate the performance of the proposed method in various settings, against several hypotheses and in comparison with state of the art approaches for object attribute classification. The experimental results demonstrate that the knowledge of an object class is not decisive for the prediction of its state. Moreover, the proposed OaSC method outperforms existing methods in all datasets and benchmarks by a great margin. 
\end{abstract}



The problem of the presence or absence of phase transition is central in statistical mechanics. To prove the existence of phase transition, the standard idea is to define a notion of contour and use \textit{Peierls' argument} \cite{Peierls.1936}. In the usual Ising model \cite{Ising_25}, particles of the system interact only with their nearest-neighbors. On ferromagnetic long-range Ising models \cite{Anderson_Yuval_69}, there is interaction between each pair of spins in the lattice. The Hamiltonian of the model is given formally by
\begin{equation*}
    H(\sigma) = - \sum_{x,y\in \Z^d}J_{xy}\sigma_x\sigma_y,
\end{equation*}
where $J_{xy}=J|x-y|^{-\alpha}$, $J>0$, $\alpha > d$. It is well-known that the Peierls' argument in dimension 2 implies phase transition for Ising models with nearest-neighbors or long-range interactions when $d\geq 2$, using correlation inequalities. For the unidimensional lattice, it was known that short-range models do not present phase transition. In the long-range case, a different behavior was expected depending on the exponent $\alpha$ (see \cite{Kac_Thompson_69}), but the problem was challenging since contours were first created as multidimensional objects.

In dimension $d=1$, phase transition was proved first in 1969 by Dyson \cite{Dyson.69}, for $\alpha \in (1,2)$, by proving phase transition in an auxiliary model and then using correlation inequalities. In 1982, Fr{\"o}hlich and Spencer \cite{Frohlich.Spencer.82} introduced a notion of one-dimensional contours and then applied the Peierls' argument to show phase transition for the critical value $\alpha = 2$. These contours were inspired by the multiscale techniques previously introduced to study the Berezinskii-Kosterlitz-Thouless transition in two-dimensional continuous spin systems \cite{FS81}. Later, Cassandro, Ferrari, Merola and Presutti  \cite{Cassandro.05} extended the contour argument previously available for $\alpha=2$ to exponents $\alpha\in (3-\frac{\ln 3}{\ln 2}, 2)$, with the additional restriction that the nearest-neighbor interaction is strong, i.e.,  ${J(1)\gg 1}$; this restriction was removed for a subclass of interactions in \cite{Bissacot.Endo.18}. Further results were obtained using contour arguments, such as the decay of correlations, cluster expansions, phase transition with random interactions, etc; some references with these results are \cite{ Cassandro.Merola.Picco.17, Cassandro.Merola.Picco.Rozikov.14, Imbrie.82, Imbrie.Newman.88, Johansson.91}. 

In the multidimensional setting ($d\geq 2$), Ginibre, Grossmann, and Ruelle, in \cite{Ginibre.Grossmann.Ruelle.66}, proved the phase transition for $\alpha > d+1$, using an enhanced version of Peierls' argument and the usual contours. Park proposed a different notion of contour for long-range systems in \cite{Park.88.I, Park.88.II}, extending the Pirogov-Sinai theory available for short-range interactions assuming $\alpha > 3d+1$, although he can also consider Potts models with his methods. Some results in the literature suggest that truly long-range effects appear only when $d < \alpha \leq d+1$, see for instance, \cite{Biskup_Chayes_Kivelson_07}. Recently, Affonso, Bissacot, Endo and Handa \cite{Affonso.2021}, inspired by the ideas from Fr{\"o}hlich and Spencer in \cite{FS81, Frohlich.Spencer.82}, introduced a version of multiscale multidimensional contour and proved phase transition by a contour argument in the whole region $\alpha > d$. They can consider long-range Ising models with deterministic decaying fields, first introduced in the context of nearest-neighbor interactions in \cite{Bissacot_Cioletti_10}. For these models, the lack of analyticity of the free energy does not imply phase transition since these models have the same free energy as the models with zero field. It is expected that fields decaying slowly imply uniqueness. In this setting, a contour argument is useful for proofs of phase transitions as well for uniqueness, some papers with models with deterministic decaying fields are \cite{Aoun_Ott_Velenik_23, Bissacot_Cass_Cio_Pres_15, Bissacot.Endo.18, Cioletti_Vila_2016}.

The Random Field Ising model (RFIM) \cite{Imry.Ma.75} is the nearest-neighbor Ising model with an additional external field acting on each site $(h_x)_{x\in\Z^d}$ that is a family of i.i.d. Gaussian random variable with mean 0 and variance 1. Formally, the Hamiltonian of the model is given by
\begin{equation*}
    H(\sigma) = - \sum_{\substack{x,y\in \Z^d \\|x-y|=1}}J\sigma_x\sigma_y  - \varepsilon\sum_{x\in\Z^d}h_x\sigma_x,
\end{equation*}
where $J>0$, $\varepsilon>0$, $\alpha > d$ and $d \geq 1$. A detailed account of the history of the phase transition problem for this model, as well as detailed proofs, was given in \cite{Bovier.06}. Here we present a brief overview.

During the 1980s, the question of the specific dimension where phase transition for the RFIM should happen attracted much attention and was a topic of heated debate. Two convincing arguments were dividing the physics community. One of them, due to Imry and Ma \cite{Imry.Ma.75}, was a non-rigorous application of the Peierls' argument together with the use of the isoperimetric inequality. The key idea of Peierls' argument is to define a notion of contour and calculate the energy cost of "erasing" each contour, i.e., the energy cost of flipping all spins inside the contour. When there is no external field, that energy necessary to flip the spins in a region $A\subset \Z^d$ is of the order of the boundary $|\partial A|$. When we add an external field, we get an extra cost depending on this field. Imry and Ma argued that this cost should be approximately $\sqrt{|A|}$, which is smaller than $|\partial A|$ for all regions only when $d\geq 3$, so this should be the region where phase transition occurs. The other argument, due to Parisi and Sourlas \cite{Parisi.Sourlas.79}, based on dimensional reduction, predicted that the $d$-dimensional RFIM would behave like the $d-2$-dimensional nearest-neighbor Ising model, therefore presenting phase transition only when $d\geq 4$. 

The question was settled by two celebrated papers showing that Imry and Ma's prediction was correct. First, in 1988, Bricmont and Kupiainen \cite{Bricmont.Kupiainen.88} showed that there is phase transition almost surely in $d\geq3$, for low temperatures and variance $\varepsilon$ small enough. Their proof uses a rigorous renormalization group analysis for the short-range case and it is considered involved. Still, they claimed that the result works for any model with a suitable contour representation and centered sub-gaussian external field. Later on, Aizenman and Wehr \cite{Aizenman.Wehr.90} proved uniqueness for $d\leq 2$. For detailed proofs of these results, we refer the reader to \cite{Bovier.06} (see also \cite{Berretti.85, Camia.18, Frohlich.Imbre.84,  Klein.Masooman.97} for more uniqueness results). 

Recently, Ding and Zhuang, see \cite{Ding2021}, provided a simpler proof of the phase transition, not using RGM. And in  \cite{Ding.Liu.Xia.22}, Ding, Liu and Xia proved that if $\beta_c(d)$ is the critical inverse of the temperature of the Ising model with no field, for all $\beta>\beta_c(d)$ there exists a critical value $\varepsilon_0(d, \beta)$ such that the RFIM with $\varepsilon \leq \varepsilon_0$ presents phase transition. 

In the present paper, we are considering a long-range Ising model with a random field, whose Hamiltonian is given formally by
\begin{equation*}
    H(\sigma) = - \sum_{x,y\in \Z^d}J_{xy}\sigma_x\sigma_y - \varepsilon\sum_{x\in\Z^d}h_x\sigma_x,
\end{equation*}
where $J_{xy}=J|x-y|^{-\alpha}$, $J, \varepsilon>0$, $\alpha > d$ and $h_x\in\mathbb{R}$, $d\geq 3$.
Until now, the only known result in the long-range setting is for the one-dimensional long-range Ising model with a random field, by Cassandro, Orlandi, and Picco \cite{Cassandro.Picco.09}. They used the contours of \cite{Cassandro.05} to show the phase transition for the model when $\alpha\in (3-\frac{\ln 3}{\ln 2}, \frac{3}{2})$, under the assumption $J(1) \gg 1$. We stress that, as remarked by Aizenman, Greenblatt, and Lebowitz \cite{Aizenman_Greenblatt_Lebowitz_2012}, although their argument does not work for the whole region for the exponent $\alpha$, the phase transition holds for values close to the critical value $\alpha=3/2$, since by the Aizenman-Wehr theorem we know that there is uniqueness for $\alpha>3/2$.

The argument from Ding and Zhuang in \cite{Ding2021}, for $d\geq3$, involves controlling the probability of a bad event, which is closely related to controlling the quantity $$\sup_{\substack{0\in A\subset\Z^d \\ A \text{ connected }}}\frac{\sum_{x\in A}h_x}{|\partial A|},$$ known as the greedy animal lattice normalized by the boundary. The greedy animal lattice normalized by the size, instead of the boundary, was extensively studied for general distributions of $(h_x)_{x\in\Z^d}$, see \cite{Cox_Gandolfi_Griffin_Kesten_93, Gandolfi_Kesten_94, Hammond_06, Martin_02}. When we normalize by the boundary, an argument by Fisher, Fr\"{o}hlich and Spencer \cite{FFS84} shows that the expected value of the greedy animal lattice is constant. In dimension $d=2$, the expected value is not finite, see \cite{Ding.Wirth.20}. The supremum is taken over connected regions containing the origin since the interiors of the usual Peierls contours are of this form.


For the long-range model, the interior of contours is not necessarily connected. In fact, long-range contours may have considerably large diameters with respect to their size, so their interiors can be very sparse. To avoid this, we define contours, strongly inspired by the $(M,a,r)$-partition in \cite{Affonso.2021}, using a multiscaled procedure that assures that the contours have no cluster with small density.  With them, we generalize the arguments by Fisher-Fr\"{o}hlich-Spencer \cite{FFS84}, and prove that the expected value of the greedy animal lattice is constant, even considering regions not necessarily connected in the supremum. Then, we prove the phase transition for $d\geq 3$. The main result of this paper is the following.
\begin{theorem*}Given $d\geq 3$, $\alpha>d$, there exists $\beta_c\coloneqq\beta(d, \alpha)$ and $\varepsilon_c\coloneqq\varepsilon(d, \alpha)$ such that, for $\beta >\beta_c$ and $\varepsilon\leq \varepsilon_c$, the extremal Gibbs measures $\mu_{\beta, \varepsilon}^+$ and $\mu_{\beta, \varepsilon}^-$ are distinct, that is, $\mu_{\beta, \varepsilon}^+ \neq \mu_{\beta, \varepsilon}^-$ $\mathbb{P}$-almost surely. Therefore the long-range random field Ising model presents phase transition.
\end{theorem*}

This paper is divided as follows. In Section 2, we define the model and the contours, and suitable generalizations to the constructions in \cite{Affonso.2021} are introduced.  In Section 3, we define two bad events of the external field and prove that they occur with a small probability.  In Section 4, we present the proof of the phase transition.

\section{Related works}
In the climate research community, a celebrated work~\cite{north_sampling_1982}---often cited as \textit{North's rule-of-thumb}---warns scientists against close eigenvalues in the Karhunen-Loève expansion of a meteorological field. Indeed, the associated principal components---referred to as \textit{empirical orthogonal functions} (EOF)---suffer from large sampling errors, which is very problematic due to the key role EOF's play in this field for exploratory data analysis. The authors provide a perturbation-theoretical rule-of-thumb to decide which eigenvalues form \textit{degenerate multiplets}. The rule as stated in the paper is quite vague, however we are able (cf. Sec.~\ref{appsec:MS}) to reformulate its practical software implementation as a relative eigengap threshold and to compare it to our criterion~\eqref{eq:releigengap_threshold}. We show that this threshold is much lower than ours (e.g. $8.6\%$ instead of $21\%$ for $1000$ samples), therefore our result has a much larger impact on the practical methodology of PCA.

More broadly, several works have mentioned close-eigenvalues in PCA or in general symmetric matrices. 
A paper from Jolliffe~\cite{jolliffe_rotation_1989} shows the advantages of factor rotation inside subspaces spanned by principal components with close eigenvalues for structured data. 
Eigenvalue equality has also been studied formally in the context of oscillatory systems~\cite{arnold_modes_1972,lazutkin_kam_1993,gershkovich_problem_2004} diffusion tensor imaging~\cite{groisser_geometric_2017}, spectral geometry~\cite{besson_multiplicy_1988}, spectral shape analysis~\cite{lombaert_diffeomorphic_2013}, statistical tests~\cite{anderson_asymptotic_1963,tyler_asymptotic_1981,rabenoro_geometric_2024} etc.

Finally, the use of flags for statistical analysis has been particularly well illustrated with the example of \textit{independent subspace analysis}~\cite{hyvarinen_emergence_2000}, from which the name of our model is drawn. The authors notice the emergence of phase and shift-invariant features by maximizing the independence between the norms of projections of samples into so-called \textit{independent feature subspaces}. The learning algorithm is later recast as an optimization problem on flag manifolds~\cite{nishimori_riemannian_2006}. 
Flags also implicitly arise in general subspace methods under the name \textit{mutually orthogonal subspaces}, like in the mutually-orthogonal class-subspaces of Watanabe and Pakvasa~\cite{watanabe_subspace_1973} and the adaptive-subspace self-organizing maps of Kohonen~\cite{kohonen_emergence_1996}.
More recently, PCA was also reformulated as an optimization problem on flag manifolds~\cite{pennec_barycentric_2018}, raising perspectives for multilevel data analysis on manifolds.

\section{Methodology}
\label{sec:method}
% Figure environment removed


%\vspace*{0.2cm}\noindent\textbf{Problem formulation.} 
% Let $O$ be the set of Objects,  $S$ be the set of
%  States and $\textit{I}$ the set of
%   Images which consists of the disjoint sets ${I^{S}}$ and ${I^{U}}$ that are used during the training and testing phase respectively. 
%    Each image $i_{k} \in \textit{I}$  contains an object $o_{i} \in O$ which  is situated in a state $s_{j} \in S$. The OSC task deals with the yielding  of a  predicted state label $sp_{j} \in S$ for an image $i_{k} \in {I^U}$ that has been given as an input. In the zero-shot variation of OSC, ${S^{S}} 
%   \not \supseteq {S^{U}}$, i.e. some of the states contained in the testing images do not appear in the training images. 
% Let $O$ denote a set of objects, 
% \textcolor{red}{$S^S$ as the set of known object states found in the training images, $S^U$ as the test, yet unknown, object state labels} and $I$ the set of images, which is partitioned into the training set $I^T$ and the test set $I^U$. 
% % Each image $i \in I$ contains an object $o \in O$ in a state $s \in S$. 
% \textcolor{red}{Each image $i \in I^T$ contains an object $o \in O$ in a state $s \in S^S$, while an image $i \in I^U$ contains an object $o \in O$ in any state $s \in S = \{S^S \cup S^U\}$. }
% The goal of OSC is to predict the state $s \in S$, given the object $o$ in $i \in I^U$. In the zero-shot variation of OSC, the set of states observed in the test images $S^U$ is not a subset of the set of states observed in the training images $S^S$, i.e., there exists some states in the test image set that do not appear in the training set. Furthermore, the task should be addressed in an object-agnostic manner, i.e. no information concerning the object classes is to be utilized explicitly.  However,  although the set of object classes does not directly affect the task of OaSC, its size is proportional to the complexity of the problem. 
% The workflow of the proposed method is shown in \autoref{fig:pipeline}.

Let $O$ denote a set of objects, $S$ denote the set of states and $I$ denote the set of images, which is partitioned into the training set $I^T$ and the testing set $I^U$. Each image $i \in I$ contains an object $o \in O$ in a state $s \in S$. 
The goal of OSC is to predict the state $s \in S$, given the object $o$ in $i \in I^U$. In the zero-shot variation of OSC, the set of states observed in the test images $S^U$ is not a subset of the set of states observed in the training images $S^S$, i.e., there exists some states in the test image set that do not appear in the training set. Furthermore, the task should be addressed in an object-agnostic manner, i.e. no information concerning the object classes is to be utilized explicitly.  However,  although the set of object classes does not directly affect the task of OaSC, its size is proportional to the complexity of the problem. 
The workflow of the proposed method is shown in \autoref{fig:pipeline}.




% and will be analyzed in the following sections.

% \vspace*{0.2cm}\noindent\textbf{Approach.}

\subsection{Overview}
% Our method is inspired by works that address the problem of zero-shot object classification \cite{}. The main idea behind this line of work is that the necessary information for the classification of the unseen classes can be found in a Knowledge Graph (KG) if processed appropriately by a Graph Neural Network (GNN). Obviously, the most crucial component of this approach lies in the combination of the visual information stemming from the training images and referring to the seen classes with the semantic information stemming from the KG  and referring to the unseen classes.

We are inspired by prior research on zero-shot object classification and leverage the potential of KGs and GNNs to classify previously unseen objects~\cite{Kampffmeyer2019,nayak:tmlr22}. 
The core idea is that semantic information that is stored in the KG can be used by GNNs to learn graph embeddings that can be utilized jointly with visual information extracted from training images. 
This enables the model to generalize to new object classes by leveraging the semantic and contextual information encoded in the graph embeddings of the KG.

% More in detail, the GNN architecture is adopted to the architecture of the Classifier  that is used for the training on seen classes, the GNN last layer has the same size  with the Classifier last layer. This way the GNN can produce semantic embedding features that correspond to all the classes, both seen and unseen, that will be encountered during the inference. These embedding features  replace the last layer of the Classifier. Holding this layer fixed, the body of the Classifier is then fine-tuned with the training images.

GNNs are designed to operate on graph-structured data, such as KGs~\cite{kipf2016semi,Monka2022}. KGs are typically represented as labeled multi-graphs, where nodes correspond to entities, and edges represent entity relationships. GNNs process this graph by iteratively aggregating information from neighboring nodes, using neural network-based operations.

At each iteration, a GNN receives a feature vector for each graph node, which is initially set to the node's embedding vector. Then, the GNN performs a message-passing step that aggregates information from neighboring nodes, based on the edge weights and the features of the nodes. This message-passing operation can be formulated as a neural network layer, which applies a learnable function to the features of the neighboring nodes and returns an aggregated message for each node. After the message-passing step, the GNN updates the node features by applying a learnable transformation that takes into account the original features of the node and the received messages from its neighbors. This updated feature vector is then passed to the next iteration of the message-passing step. The process continues until a fixed number of epochs or convergence.
%%%AAA: Endexetai na mas rethrown gia tis times aytwn twn parametrwn?
% KP edw anaferetai genika mia diadikasia GNN training. Na anaferoume edw times parametrwn h sto 4 - see implementation details ?

The proposed method leverages GNN training using a visual classifier that is trained on seen state classes as supervision. In particular, the last layer of the GNN is designed to have the same size as the last layer of the classifier. This enables the GNN to generate semantic embedding features that correspond to all classes, including both seen and unseen classes that will be encountered during inference. Subsequently, the semantic embedding features replace the last layer of the classifier while this layer is kept fixed. The body of the classifier is then fine-tuned with the training images to optimize the overall model for state recognition.

% \vspace*{0.2cm}\noindent\textbf{GNN Details.} 
Overall, we experimented with four different model architectures and opted for the Transformer Graph Convolutional network (Tr-GCN)~\cite{nayak:tmlr22}. Further details are provided in Section~\ref{sec:abl} and the supplementary material of this work. 
The Tr-GCN mode is capable of combining input sets non-linearly by utilizing multilayer perceptrons and self-attention. Tr-GCN refers to an inductive model that can learn node representations by aggregating local neighborhood features allowing the trained model to make predictions on new graph structures without retraining. 
We leverage the aforementioned property of the Tr-GCN to train a permutation invariant non-linear aggregator that captures the intricate structure of a common sense knowledge graph. 
% , rendering it well-suited for zero-shot learning. 
% It is worth noting that a similar network architecture has been effectively employed for zero-shot object classification~\cite{nayak:tmlr22}.

% A critical aspect of the proposed method involves calibrating the weights of the GNN in a manner that its predictions in the semantic space are useful for the classifier deployed in the visual space. To accomplish this, we adopt an approach based on prior research \cite{Kampffmeyer2019, Wang2018b, nayak:tmlr22} that involves learning the semantic class representations by minimizing the L2 distance between the learned class representations and the weights of a fully connected layer in a ResNet classifier pre-trained on the ILSVRC 2012 dataset \cite{russakovsky2015imagenet}. Once the class representations are learned, we fix them and fine-tune the ResNet backbone using the training images from the dataset.




% \vspace*{0.2cm}\noindent\textbf{Building of the KG.}
% The KG is created by the querying  of a common sense repository. The repositories that we are ConceptNet \cite{} and WordNet\cite{}. The procedure takes place as follows. Initially we create a set of nodes that correspond to the target stace classes. Subsequently, the repository is queried for each of these nodes and its neighbours in the repository of  added to the KG if  certain criteria are met (see ablation section for more details). This procedure is repeated for the newly added nodes and henceforth until a number of hops has been reached.  

\subsection{The proposed OaSC approach}
\label{sec:pipeline}
Overall, the proposed method consists of four stages, as shown in \autoref{fig:pipeline}: (1) construction of the KG, (2) GNN training and learning of semantic graph embeddings, (3) fine-tuning of the visual classifier and (4) deployment of the fine-tuned state classifier.

\vspace*{0.0cm}\noindent\textbf{Construction of the KG (Stage 1)}:
To create the KG, we query a common sense repository to compile a generic solution and to avoid the construction of a task-specific KG, tailored to the entities at hand and their relationships. First, a set of nodes that correspond to the words of the target state classes $S^U$ and $S^S$ is generated. Then, we query the repository for each of these nodes and add their neighbors in the KG, if they meet specific criteria (see also Section~\ref{sec:abl}). This process is repeated for the newly added nodes until a specified number of node hops is reached.

This technique for building a generic KG offers several advantages in comparison to other problem-specific approaches. First, it allows the same KG to be used for different variations of the task. It also enables transfer learning since KGs can be reused to tackle other related problems. Moreover, the construction of such a KG does not rely on expert knowledge. Besides, the structured representation of relationships between entities and concepts that KGs provide can be leveraged to generate robust embeddings for zero-shot learning.
% which is expensive and time-consuming.  
The trade-off is that such KGs are prone to noisy information in the used repositories. 

% In comparison, language models, such as BERT~\cite{devlin2018bert}, often rely on large amounts of unstructured text data to generate embeddings. While language models are highly effective at capturing semantic relationships between words and phrases, they can also be prone to create associations between concepts that are not actually related. This can lead to noisy or unreliable embeddings, which can in turn degrade the performance of zero-shot learning models. By contrast, the structured nature of KGs allows for more accurate and precise capture of relationships between entities and concepts, leading to more robust embeddings that can improve the accuracy and reliability of zero-shot learning models~\cite{brown2020language}.


\vspace*{0.0cm}\noindent\textbf{Computation of  Graph Embeddings (Stage 2)}:
% Given the KG constructed in Stage 1, a word features embedding matrix corresponding to the KG nodes is created by utilizing the pre-computed word features of GloVe~\cite{pennington2014glove}. 
% % Subsequently,  random walks are performed in the KG and a sample of neighbors for each node is obtained.
% By taking the word features embedding matrix, the KG topology, and a target node as inputs, the GNN estimates the node's embeddings: the features of the node and its neighbors are  aggregated and submitted to a series of convolutions and pooling operations before the  output is produced in the form of a feature vector, the length of which is tailored to be the same as the size dimension of the last layer of a ResNet-101 classifier. 
% This procedure is repeated for all KG nodes and results in the computation of the semantic embeddings for all target state classes with each embedding being a feature vector of length equal to 2048. By combining these embeddings for the \mathcal{d} target classes  a  $ d \times 2048$ features matrix is created which serves as the last layer of a CNN classifier that is utilized during Stages 3 and 4.
% which serves as the last layer of a CNN classifier that is utilized during Stages 3 and 4 .
% We employ an established approach~\cite{Kampffmeyer2019, Wang2018b} that involves training of a transformer-based Graph Convolutional Model using graph embeddings of a set of semantic entities acquired by a common sense repository by minimizing the L2 distance between the learned class representations and the weights of a fully connected layer in a ResNet classifier, pre-trained on the ILSVRC 2012 dataset~\cite{russakovsky2015imagenet}, ensuring that the semantic class representations are meaningfully embedded.
We employ an established approach~\cite{Kampffmeyer2019, Wang2018b} that involves the training of a transformer-based Graph Convolutional Network (GCN)
 \textcolor{black}{ that utilizes a KG as input  %Training is performed %using features of a set of semantic entities acquired by a common sense repository, \textcolor{red}{(e.g. the ConceptNet, CSKG, or other)}  
 and generates an embedding vector for each node of the  KG. %. For the production of the embeddings vectors the GCM employs a sequence of transformations to the semantic features that correspond to the concepts linked to each node.
This process defines pre-computed GloVe word, i.e. semantic features~\cite{pennington2014glove}, for the KG nodes with each node representing a concept class.
% To compute node embeddings, the GNN is applied to encode the KG topology and the word feature embedding matrix. 
The GNN  aggregates each node's and its neighbors' features through a sequence of convolutions and pooling operations. %This results in the generation of a feature vector having a length equal to the dimension of the last layer in a visual CNN-based classifier that is instantiated using a ResNet-101 model. 
%By pre-training the visual classifier in a set of target classes 
The visual classifier is pre-trained on a set of target classes and using the weights of its fully connected layer, the GCN learns to produce visual feature representations, i.e. visual embeddings,  corresponding to the concept classes of the KG`s nodes.}
\textcolor{black}{
Formally,  the training involves the minimization of the L2 distance   $\mathcal{L_G}$ between the generated visual embeddings and the ground truth visual embeddings stemming from the visual classifier.} 
\textcolor{black}{In notation, 
\begin{equation}
\mathcal{L_G} = \frac{1}{2N} \sum_{n \in N} \sum_{p \in P} (W_{n,p} - \tildea{W}_{n,p})^2,
 \end{equation}
where $\tildea{W} \in \mathbb{R}^{|N|xP}$ denotes the weights of the GCN for the set of known concept classes $N$ and the dimensionality $P$ of the weight vector. Similar to~\cite{Kampffmeyer2019}, the ground truth weights, denoted as $W \in \mathbb{R}^{|N|xP}$, are obtained by extracting the last layer weights of a pre-trained CNN.}
% This process is repeated for all KG nodes corresponding to $S^U$ and $S^S$, generating semantic graph embeddings for all target state classes. 
%Each embedding comes in the form of a feature vector of length 2048. 


%By combining these embeddings for the $d$ target classes, a  $d \times 2048$ features matrix is defined that is integrated as the final layer of the visual CNN-based classifier that is employed in Stages~3-4.
%A critical aspect of this process is adjusting the GNN weights to align its predictions with the semantic space. This ensures that the semantic embeddings effectively aid the classifier used in Stages 3 and 4, operating in the visual space. 



\textcolor{black}{ 
The KG  given as an input to the GCN model is a hierarchical graph created for the requirements of the   ILSVRC 2012 dataset~\cite{russakovsky2015imagenet} and represents the WordNet hierarchical structure of the $1,000$ classes comprising the dataset. These 1,000 concept labels constitute the set of classes upon which the visual classifier used for the extraction of the ground truth visual embeddings is pre-trained.
}
% A critical aspect of this process is adjusting the GNN weights to align its predictions with the semantic space. This ensures that the semantic embeddings effectively aid the classifier used in Stages 3 and 4, operating in the visual space. 
% The concepts  that are used for the training refer to a set of 1K object classes of the ILSVRC 2012 dataset~\cite{russakovsky2015imagenet}, while the pre-trained ResNet101-based classifier is used for supervision to ensure that the GNN outputs, thus the semantic object class representations, are meaningfully embedded into the visual feature space. 
After the training is completed, the GCN model is employed to process the KG (constructed in Stage 1) and generate visual embeddings for the KG nodes that correspond to the object state classes,  by taking as input the  KG that was constructed during Stage 1. Each embedding comes in the form of a feature vector of length 2048, i.e. dimension of the last layer of the  pre-trained visual CNN-based classifier.
By combining these embeddings for the $d$ target classes, a  $d \times 2048$ features matrix is defined that is integrated as the final layer of the visual CNN-based classifier that is employed in Stages~3-4.
% First, using their pre-computed GloVe word features~\cite{pennington2014glove}, a matrix of word, i.e. semantic, features embeddings is defined for each of the KG nodes.
% % To compute node embeddings, the GNN is applied to encode the KG topology and the word feature embedding matrix. 
% Subsequently,  the GNN takes as input every target node that corresponds to any class in $S^U$ and $S^S$ and aggregates the features about the node and its neighbors through a sequence of convolutions and pooling operations. This results in the generation of a feature vector having a length equal to the dimension of the last layer in the visual CNN-based classifier that is instantiated using a ResNet-101 model.
% % This process is repeated for all KG nodes corresponding to $S^U$ and $S^S$, generating semantic graph embeddings for all target state classes. 
% Each embedding comes in the form of a feature vector of length 2048. By combining these embeddings for the $d$ target classes, a  $d \times 2048$ features matrix is defined that is integrated as the final layer of the visual CNN-based classifier that is employed in Stages~3-4.
% A critical aspect of this process is adjusting the GNN weights to align its predictions with the semantic space. This ensures that the semantic embeddings effectively aid the classifier used in Stages 3 and 4, operating in the visual space. 

% A critical aspect of this procedure involves calibrating the weights of the GNN to embed its predictions in the semantic space, i.e. semantic embeddings, are useful for the classifier deployed in the visual space during Stages 3 and 4. To accomplish this, we adopt an approach based on prior research~\cite{Kampffmeyer2019, Wang2018b} that involves learning the semantic class representations by minimizing the L2 distance between the learned class representations and the weights of a fully connected layer in a ResNet classifier pre-trained on the ILSVRC 2012 dataset~\cite{russakovsky2015imagenet}.  

\vspace*{0.0cm}\noindent\textbf{Fine-tuning of the Visual Classifier (Stage 3)}:
The estimated semantic embeddings are integrated into a visual CNN classifier that relies on the ResNet backbone and is initially pre-trained for object classification. The embeddings serve as the final layer of the network, encapsulating the representations essential for predicting the train state classes $S^S$. To enable this adaptation, the visual classifier undergoes re-training, specifically tailored to the classification of the train classes. 
During this fine-tuning process, input images $I^T$ contain states sourced exclusively from the training set $S^S$, i.e. ``seen states''. The primary objective is to harness the classifier capabilities to classify these familiar states, accurately. Notably, fine-tuning involves keeping the weights of the last layer fixed, safeguarding the integrity of the acquired semantic representations from Stage 2. Consequently, adjustments are only applied to the weights of preceding layers to ensure they effectively match the ``frozen'' last-layer weights.
% Apart from this detail, the procedure takes place in the same manner as the training of a CNN classifier.
% in every training epoch a loss is computed the value of which guides the update of all layers weights except the last one. 
Following the notation introduced 
\textcolor{black}{in the beginning of Section~\ref{sec:method}, the loss function is defined as:}
\begin{equation}
% \mathcal{L} = -\sum_{i \in S^{S}} y_i \cdot \log(P(y=i|X))
\mathcal{L_V} = -\sum_{s \in S^S, i \in I^{T}} y_s \cdot \log(P(s|i)),
 \end{equation}
\textcolor{black}{for the predicted \textit{$y_s$} state label in the \textit{$S^S$} set of state labels. $P(s|i)$ denotes the probability of state label \textit{s} based on the softmax vector given an image \textit{i} from the $I^T$ training set.}

\noindent\textbf{Zero-shot OaSC (Stage 4)}:
Upon the completion of fine-tuning, the visual state classifier can be utilized for  prediction by choosing the most likely class
\begin{equation}
% \^y = \arg\max_{i \in S} \left( P(y=i|X) \right)
\hat{y} = \arg\max_{s \in S^U i \in I^{U}} \left( P(s|i) \right),
\end{equation}
\textcolor{black}{where $I^U$ denotes the test image set and $S^U$ the test state classes respectively.} 
We highlight that the classifier is well-suited for predicting either only unseen classes, i.e. zero-shot classification, or both seen and unseen classes, i.e. generalized zero-shot classification.
\vspace{-.15cm}
% \subsection{Pipeline}

% Overall, the pipeline of our method consists of four stages (\autoref{fig:pipeline}}). During \textbf{Stage 1}, the KG is constructed.

% \vspace*{0.2cm}\noindent\textbf{Construction of the KG (Stage 1)}:
% The KG creation process involves querying a common sense repository to enable generalization instead of creating a custom KG tailored to specific entities and relationships. Initially, nodes corresponding to the target state classes are generated. The repository is then queried for each node, and neighbors meeting specific criteria are added to the knowledge graph. This process continues for the newly added nodes until a specified number of hops is reached. More details can be found in the ablation section.


% \vspace*{0.2cm}\noindent\textbf{Computation of semantic embeddings (Stage 2)}:


% \vspace*{0.2cm}\noindent\textbf{Finetuning of the Classfier (Stage 3)}:

% \vspace*{0.2cm}\noindent\textbf{Deployment  (Stage 4)}:


\section{Experimental Evaluation}

\newcommand{
  \pbt}{\color{blue}{}
}

\begin{table*}[t]
    \centering

    \begin{tabular}{|l|rrr|rrrrrr|}

\hline\hline 
\textbf{Dataset} & \textbf{Train}  &  \textbf{Val}   &  \textbf{Test} & \textbf{Seen} & \textbf{Unseen} & \textbf{Objects} & \textbf{VOSC} & \textbf{TOSC} & \textbf{S\textbackslash O} \\ \hline \hline

OSDD \cite{gouidis2022} &   6,977 & 1,124  & 5,275 & 5 & 4 & 14 & 35 &126 &2.36 \\ \hline 
CGQA-states \cite{Mancini2022} &   244 & 46 & 806 & 2 & 3 & 17  &41 & 75 & 1.71  \\ \hline
MIT-states \cite{Isola2015} &  170 & 34 & 274 & 2 & 3 & 14 & 20 &  70  & 1.57 \\ 
 
   \hline\hline 
  \end{tabular}
 

  
   \caption{Dataset details. Train/Val/Test: Number of Training/Validation/Testing Images. Seen/Unseen: Number of seen/unseen State classes. Objects: Number of Object classes. VOSC/TOSC: Valid/Total Object-State combinations. S\textbackslash O: Average number of states than an Object can be situated in.}
    \label{tab:datasets}
\end{table*}


Our study involves a sequence of experiments that entail comparing our approach to another SoA model, as well as conducting an extensive ablation study to investigate various aspects of the problem. Specifically, we aim at an in depth exploration of three Hypotheses. First, we examine the degree to which the KG contributes to the success of the OSC task. Second, we evaluate the impact of the GNN architecture on the method's overall performance. Additionally, we investigate whether knowledge of the object class has an effect on the performance of the OSC task. The previous hypotheses can be formulated as follows:

\vspace*{0.1cm}\noindent\textbf{Hypothesis 1}: The KG is beneficial to the task. Its impact depends primarily on the type of the knowledge it contains. 

\vspace*{0.1cm}\noindent\textbf{Hypothesis 2}: The GNN architecture is crucial  to the performance of the method. 

\vspace*{0.1cm}\noindent\textbf{Hypothesis 3}:\label{hyp:3} The knowledge of an object class is not decisive for the prediction of its state. Therefore, a method that is agnostic to the object class, can perform equally well to  a method that relies on it.
%%%AAA: To "or better" alougetai too much... To na ksereis to antikeimeno mporei na mhn voitha. Wstoso, giati na dyskoleyei thn katastasi???



\subsection{Implementation and evaluation issues}

\noindent\textbf{Implementation details}: 
The GNN was trained following the method outlined in Nayak et al. \cite{nayak:tmlr22}. The model was trained for 1000 epochs on 950 randomly selected classes from the ILSVRC 2012 dataset \cite{russakovsky2015imagenet}, while the remaining 50 classes were held out for validation. The model with the lowest validation loss was chosen to generate the seen and unseen class embeddings using the graph. For the seen classes, the embeddings were frozen, and a pre-trained ResNet101-backbone was fine-tuned on the individual datasets for 50 epochs using stochastic gradient descent  with a learning rate of 0.0001 and momentum of 0.9.

% For the training of the  GNN we follow the strategy propose in \cite{nayak:tmlr22} and train the model for 1000 epochs on 950 random classes
% from the ILSVRC 2012 \cite{russakovsky2015imagenet} while the remaining 50 classes are used for validation. The model with
% the least loss on the validation classes is used to generate the seen and unseen class embeddings with
% the graph.  We freeze the class embeddings for the seen classes and fine-tune a pretrained ResNet101-backbone on the individual datasets for 50 epochs using SGD with a learning rate 0.0001 and
% momentum of 0.9. 


% Currently with the exception of the OSDD dataset \cite{},  there are not exclusive object states dataset available, but rather attributes  dataset which include among their classes and object states. Therefore, we adopt two of the most popular attributes  datasets \cite{Isola2015} \cite{Mancini2022}  to our needs by taking the subsets that refer to object states in order to be used in the context of the experimental evaluation.

\begin{table*}[t]
	\small
    \centering
     % \resizebox{1\textwidth}{!}
     % {\begin{minipage}{\textwidth} 
     \begin{threeparttable}
    \begin{tabular}{|cc|c|c|c|c|c|c|c|c|c|c|c|c|c|c|}
    
   \hline \hline 

      
 % \diagbox[innerleftsep=.5cm,innerrightsep=0pt]{{\bf \multirow{2}{*}{Method}}}{{\bf \multirow{2}{*}{Dataset}}} 
 
 


 
 \multirow{2}{*}{\textbf{\centering  Method}}  & \multirow{2}{*}{\textbf{\centering  Version}}  &\multicolumn{4}{|c|}{\textbf{OSDD}}  & \multicolumn{4}{|c|}{\textbf{CGQA-States}}  & \multicolumn{4}{|c|}{\textbf{MIT-States}}  \\      \cline{3-14} 

& &\textbf{Seen} & \textbf{Unseen}   & \textbf{HM} & \textbf{AUC} & \textbf{Seen} & \textbf{Unseen}   & \textbf{HM} & \textbf{AUC}
& \textbf{Seen} &\textbf{Unseen}  & \textbf{HM} & \textbf{AUC}
 \\ \hline \hline
 

   



\hline  

&OO & 84.2 &  18.2  & 14.0 &  6.7
 & 93.2 & 45.7 & 33.9 & 24.7
& 97.8 &  55.5 &  37.5 &  30.6 \\ 

AoP\cite{nagarajan2018attributes}  &OW\tnote{\dag} & 69.7 &  33.3  & 21.6 &  9.2
 & 90.3 & 40.1 & 22.5 & 13.2
& 38.7 &  12.3 &  7.2 &  1.3 \\ 

 &CW\tnote{\dag} & 75.9 &  41.1  & 28.7 & 13.4
 & 95.5 & 50.0 & 35.6 & 27.7
& 48.5 &  20.8 &  15.1 &  4.1\\  \hline


 & OO & 31.2 &     39.8  &  23.8 & 9.2
     & 96.7 & 13.0 & 14.0 & 6.2
 & 99.8 & 20.7 &   20.7 & 10.3 \\  
LE+\cite{misra2017red}&   OW\tnote{\dag} & 71.6 &     14.3 &  20.8 & 6.5
     & 76.7 & 11.1 & 9.5 & 3.0
 & 63.6 & 14.6 &   20.3 & 7.1 \\  

 & CW\tnote{\dag} & 68.6 &     31.7  &  34.5 & 16.9
     & 93.5 & 16.1 & 16.1 & 8.1
 & 99.4 & 20.5 &   19.4 & 10.0 \\    \hline

 &OO & 85.5 &  67.2  & 38.5 &  27.5
 & 99.0 & 17.1 & 16.5 & 8.5
& 99.2 &  9.5 & 17.2 &  8.9 \\ 

 TMN\cite{purushwalkam2019task} &OW\tnote{\dag} & 73.4 &  43.6 & 33.7 &  19.0
 & 99.0 & 17.1 & 16.5 & 8.5
& 69.7&  18.4 & 22.4 & 6.3 \\ 


 &CW\tnote{\dag} & 71.5 &  49.8  & 34.9 &  20.7
 & 97.0 & 76.0 & 39.9 & 32.3
& 84.9 &  30.7 & 27.4 &  16.1 \\   \hline
 
 &OO & 83.2 &     36.7  &  28.3 & 16.3 
     & 98.5 & 66.4 & 34.3 & 27.2
 & 97.3 & 29.1 &   29.6 & 17.3 \\  

 SymNet\cite{Li2020}& OW\tnote{\dag} & 77.7 &     14.0  &  21.1 & 7.5 
     & 94.0 & 7.1 & 13.2 & 6.1
 & 79.3 & 17.2 &   10.2 & 5.1 \\  

  & CW\tnote{\dag} & 77.7 &     59.4 &  44.2 & 31.1
     & 95.5 & 27.4 & 39.4 & 24.4
 & 96.9 & 27.5 &   26.8 & 15.7 \\    \hline

\hline  

 &OO & 76.1 &  51.1  & 35.9 &  25.4
 & 95.7 & 72.8 & 21.9 & 10.1
& 97.6 &  49.4 &  21.8 &  10.2 \\ 


  Compcos\cite{Mancini2022}    &OW\tnote{\dag} & 79.9 &     3.7  &  30.1 & 22.1 
     & 86.8 & 42.8 & 7.7 & 3.4
 & 97.3 & 29.1 &   28.3 & 16.9 \\  


  & CW\tnote{\dag} &   81.7 &  33.2  & 38.7 & 23.8
   & 94.2 & 73.9 & 48.1 & 41.5  
   &   94.6 & 33.7 &   44.9 &  23.8 \\
%      Compcos OW& 79.9 &     3.7  &  6.9  & 2.5 
%      & 86.8 & 42.8 & 29.8 & 17.0
%  & 97.3 & 29.1 &   28.3 & 16.9 \\  

%  Compcos (Object-Oracle ) & 76.1 &  51.1  & 27.1 &  18.7 
%  & 95.7 & 72.8 & 41.1 & 35.4 
% & 97.6 &  49.4 &  42.4 &  33.2 \\ 

%   Compcos CW\tnote{\dag} &   81.7 &  33.2  & 22.9 & 13.4 
%    & 94.2 & 73.9 & 38.7 & 32.6  
%    &   94.6 & 33.7 &   27.7 & 18.2\\  
  
    \hline  \hline 
 % \multicolumn{2}{|c|}{\centering  OaSC CN+WN\_H2\_TH} 
 % & 86.5 & 64.2 & 45.6 & \textbf{35.4} & 97.1 & 68.6 & \bf 43.0 & \bf 35.0 & 85.7 & 69.6 & 52.7 & 40.9 \\ 

 %  \multicolumn{2}{|c|}{\centering  OaSC  CN+WN\_H3\_TH} 
 % &  85.8 & 65.8 & \textbf{47.3} & 32.3 & 81.2 & 59.7 & 38.8 & 28.6 & 83.9 & 66.9 & \bf 57.2 &  \bf 46.5  \\ \hline  
  \multicolumn{2}{|c|}{\centering  OaSC } 
  & 86.5 & 64.2 & \bf 45.6 & \textbf{35.4} & 97.1 & 68.6 & \bf 43.0 & \bf 35.0 & 85.7 & 69.6 & \bf 52.7 & \bf 40.9 \\  \hline 


% CN+WN\_H3\_TH & 87.1 & 56.3 & 44.6 & 31.9 & 97.1 & 60.5 & 41.0 & 32.5 & 83.3 & 68.6 & 55.9 & 41.0 \\ 
% CN\_H2\_TH & 85.7 & 63.7 & 45.6 & 34.5 & 97.1 & 70.0 & 43.5 & 35.6 & 85.7 & 70.2 & 51.6 & 40.5 \\ 
% CN\_H2 & 86.4 & 60.6 & 45.1 & 34.3 & 97.1 & 73.4 & \textbf{46.3} & \textbf{39.5} & 88.1 & 69.6 & \textbf{56.2} & \textbf{43.5} \\ 

 



  
  
  \end{tabular}
  

   \caption{   \label{tab:table_aggr}Aggregate results. Seen: Best Accuracy on seen classes. Unseen: Best accuracy on unseen classes. HM: Best harmonic mean. AUC: Area under curve for the  pairs of accuracy for seen and unseen classes.  OO: Object-Oracle version. OO: Open-World Vversion. CO: Closed-World version.
   % CN: ConceptNet-based model. WN: WordNet-based model. H2(3): Maximum number of hops equal to 2(3). TH: Thresholding policy for the nodes of the KG.
   }

\footnotesize
\begin{tablenotes}

\item[\dag] Serves as external reference but is not considered a baseline due to violation of experiment assumption.

\end{tablenotes}
        \end{threeparttable}

     % \end{minipage}}
   
\end{table*}

\vspace*{0.2cm}\noindent\textbf{Datasets}: 
At present, there is a lack of datasets exclusively dedicated to object states, with the exception of the OSDD~\cite{gouidis2022} which is a dataset tailored for state detection. Instead, existing attribute datasets include object states among their classes. To address this, we utilized two of the most widely used attribute datasets~\cite{Isola2015, Mancini2022} and extracted subsets that specifically relate to object states for use in our experimental evaluation. Regarding the OSDD, we extracted the bounding boxes of the original images in order to create images suitable for the OSC task.  The complexity of each dataset can be assessed primarily  by (a)~the number of unseen state classes and (b)~the average number of states per object class. 
The details for the three  datasets are presented in  \autoref{tab:datasets}. 

\vspace*{0.1cm}\noindent\textbf{Metrics}: 
Our evaluation protocol follows the standard generalized zero-shot evaluation described in~\cite{Purushwalkam}:  we calculate the Area Under the Curve (AUC)   measuring the accuracy on both seen and unseen compositions at different operating points based on the bias term that is added to the scores of the unseen classes.  The optimal zero-shot performance occurs when the bias term is positive, leading the classifier to prioritize the unseen labels. Conversely, the best seen performance is achieved with a negative bias term, which result in a focus on the seen labels. Additionally, we report the best harmonic mean (HM) which expresses balance between the seen and unseen accuracy, respectively.

\vspace*{0.1cm}\noindent\textbf{Comparison with SoA methods}: 
% Since there are no actual zero-shot state classifiers, we utilize SoA methods from the fields of  czsl and zero-shot attribute classification. In the case of czsl, they only method that can be used for ZS-OSC is  Compcos  which also exhibits the best performance in the czsl task. 
% As there are currently no pure zero-shot state classifiers available, we utilize  the state-of-the-art method from the field of CZSL\cite{Mancini2022} which involves both the prediction of object and state label for an image is  the most related field to to OSC.  We report the methods performance for three different settings: closed world, open world and object-oracle. For the first setting the method has to predict only among the valid object-state pairs, whereas for the second the method has to predict among all object-state pairs. For the third setting, all object labels are set to the generic term \'object\' enabling thus the method to predict only the state label. Although the closed world setting violates the zero-shot assumptions we include it as a baseline.
Given that there are currently no zero-shot state classifiers available, we resort to employing   5 state-of-the-art models \cite{misra2017red,nagarajan2018attributes,purushwalkam2019task,Li2020,Mancini2022}
  from the field of CZSL that deal with predicting both object and state labels and are, therefore, closely related to OSC. As these models are capable of producing state labels, they can be used in the context of OSC without any modifications.
We evaluate the performance of this approach on three different versions: closed world, open world, and object-oracle: 
\begin{itemize}
    \item {\bf Closed World (CW) version}: the method is tasked with predicting only among the valid object-state pairs.
    \item  {\bf Open World (OW) version}: the method is tasked with predicting  among all object-state pairs. 
    \item {\bf Object Oracle (OO) version}: all object labels are replaced with the generic term ``object'', allowing the method to solely predict the state label.
\end{itemize}
 While the closed world version setting violates the zero-shot conditions since it assumes that the valid states for each object are known in advance, we include it as a baseline for comparison. Moreover, the open world version is less generic than our approach since it presupposes that the set of object labels to which the states corresponds is closed, i.e. the same during training and inference, whereas our method is totally agnostic to this parameter. In addition, both the closed and the open world versions of the models make use of the information corresponding to the object categories, something that violates the object-agnostic assumption. Therefore, the most fair comparison to our method is the object oracle version of the models. However, we report  the results of the  closed  and open world version of each model as frames of reference. 
% Specifically, in the former case we can only use the Compcos method \cite{Mancini2022}, which has demonstrated the best performance in the context of czsl.


% The best zero-shot performance is achieved when the bias term is large, predicting only the unseen labels. The best seen performance is achieved when the bias term is negative, predicting only the seen labels.

% % Figure environment removed


\subsection{Results}


\autoref{tab:table_aggr} summarizes the results of the evaluation for the three employed datasets. We report the performance of the version of
our model that was selected by the ablation study described in the next version. It should be noted that this version of the model does not
exhibit the best performance in all categories.
% (further information about the specific characteristics of the two models are provided in the next section).
% Each model is based on a different KG and/or a different calibration. Further information about the specific characteristics of the four models are provided in the next section.   
The results indicate that under the Object Oracle version, our method clearly outperforms the competing methods in both metrics, namely AUC  and HM, across  all three datasets. Furthermore, our method achieves superior performance than the Closed-World setting of all the competing methods in most cases (it   scores best 14 out of 15 cases, e.g.  5 competing models X 3 datasets, in the AUC  metric and  13 out of 15 times in the HM metric,  respectively). This attests to the robustness of our method, since the Closed-World setting makes use of additional information regarding the object classes  and the valid object-state combinations. In the following, we refer only to the performance of the object oracle versions of the competing methods.

The largest performance margin in favor of our method is observed in the MIT-states dataset, with an increase of  10.3\%/ for AUC  and  15.2\% for HM   in comparison to the scores of the best performed competing method (AoP). In the case of the OSDD dataset, there is a difference of  7.9\% for AUC and 7.1\% for HM in favor of our method w.r.t.  TMN method which is the competing method that performs best in this dataset. Finally, for the CGQA-States dataset  a difference of   7.8\%   for AUC and for 8.7\%  HM is observed between our method and the SymNEt model which scores best among the competing methods in this setting.   

% The largest performance margin in favor of our method is observed in the MIT-states dataset, with an increase of 15.2\%/19.7\% for HM  and   10.3\%/15.9\% for AUC between the two versions method and the best of the competing methods (TMN). In the case of the OSDD dataset, there is an increase of  7.9\%/4.8\% for AUC and 8.8\%/7.1\%  for HM. Finally, for the CGQA-States dataset  a difference of 24.9\% for AUC and 9.7\%/5.5\%  for HM is observed. 
% This pattern indicates that our method performs even better than the competing method as the complexity of the dataset increases. 
  % Regarding the absolute performance of the method across the three datasets we see that  it is proportional to  the complexity of each dataset with the greater scores achieved for the MIT-states and the lowest for the CGQA-states respectively. This behavior is also observed in the performance of the majority of the competing methods.
% Overall, the best performance for the OSDD is achieved by the model CN+WN\_H2\_TH, while the model CN\_H2 scores best for the two other datasets.  


The substantial margin by which our proposed object-agnostic OaSC method outperforms the competing object-based method in every experiment supports strongly  the {\bf Hypothesis 3}, namely that that object information does not provide any advantages in the context of zero-shot OSC.
Additionally, the behavior of the three  versions of the competing  models provides further insights regarding the problem. Specifically, the Open-World version performs very poorly, while the performance of the Object-Oracle version can be deemed only average, given the significantly smaller search space (i.e., set of states) in comparison to the search spaces of the Closed (i.e., set of valid object-state pairs) and Open (i.e., set of all object-state pairs) Worlds, respectively.



% Finally, examining the  performance of the models across the three datasets we can see that for  most of the models the scores achieved are proportional to  the complexity of each dataset with the greater scores achieved for the MIT-states and the lowest for the OSDD respectively. 


% The results strongly support \textbf{Hypothesis 3}, which suggests that object information does not confer any advantages in the context of zero-shot Object State Classification, as evidenced by the clear margin of outperformance of our object-agnostic method over the competing object-based method in all experiments.



% Furthermore, the behavior of the three different versions of CompCos provides further insights regarding the problem. Specifically, we observe that the performance of the object-oracle version  

\subsection{Ablation Study}\label{sec:abl}


% We ablate our method across the following categories: the GNN architecture, the KG source, the number of max hops that were used for the creation of the KG and the policy that were followed w.r.t. the inclusion of nodes to the KG. For each ablated model we report the best accuracy achieved on seen and unseen classes, the best harmonic mean and the best AUC for each of the three datasets. The results are presented in \autoref{tab:ablation}.

We conducted a host of ablation experiments across several problem dimensions with  the purpose of selecting the optimal parameters for our model and of 
 investigating more thoroughly the three hypotheses stated previously.  Specifically, we explored the impact of varying the GNN architecture, the KG source, the maximum number of hops used for KG creation, and the policy for including nodes in the KG. Due to space consideration, it is not possible to present the performance exhibited by every ablated  model  that was tested. Instead, we present  aggregated means of all models across each of the ablated dimensions  reporting the  best harmonic mean and the AUC for each of the three datasets, respectively. 
% The results of these experiments are presented in \autoref{tab:abl1} - \autoref{tab:abl3}.

%%%AAA: Parakatw, prwta parousiazeis oles tis diastaseis tou ablation, kai meta ola ta relevant results. Gia na veltiwseis to locality of reference kanto alliws: ablation diastasi 1 - results 1, ablation diastasi 2 - results 2...
 


\vspace*{0.2cm}\noindent\textbf{GNN architecture}:
We experiment with 4 different GNN architectures:  GCN \cite{kipf2016semi}, R-GCN \cite{schlichtkrull2018modeling},   LSTM \cite{hamilton2017inductive} and Tr-GCN \cite{nayak:tmlr22}.The ablation results for the different  architectures are presented in \autoref{tab:abl1}. 
We can see that   the Tr-GCN framework  outperforms the other frameworks in all datasets w.r.t. AUC metric. whereas it scores best w.r.t. HM metric in the OSSD and comes second in the two other datasets. The R-GCN framework exhibits the second-best performance, while the GCN framework comes in third and the LSTM framework exhibits the worst performance respectively. These findings are consistent with prior research in the domain of zero-shot object classification and  substantiate   \textbf{Hypothesis 2}.



\vspace*{0.2cm}\noindent\textbf{KG source}:
% The KG sources that we use are: ConceptNet\cite{speer2017conceptnet} and WordNet\cite{fellbaum2010wordnet}. We also include in our experiments some KGs that were created using information from both sources. We mention also that we attempted to utilize other sources such as Dbpedia\cite{} and WikiData\cite{} but did not succeed at finding the necessary information for the creation of a KG.
We employed two KG sources, namely ConceptNet~\cite{speer2017conceptnet} and WordNet~\cite{fellbaum2010wordnet}, and also experimented with combining information from both sources. Other sources such as Dbpedia~\cite{auer2007dbpedia} and WikiData~\cite{vrandevcic2014wikidata} were also considered, but the necessary information for constructing a KG could not be obtained. Moreover, to assess more accurately the contribution of the KGs we include a ConceptNet-based model in which the target states classes were mapped to other unrelated state embeddings of the KG and a random model  where the embeddings corresponding to the target state classes were generated by a random process. 

Consulting the results in \autoref{tab:abl2}, we can observe that ConceptNet outperforms WordNet in all three datasets, while combining both sources results in  performance gains for the HM metric in all three datasets and for the AUC metric in two of the datasets. The difference in favor of ConceptNet can be attributed to the difference between the type of information that each KG holds. Specifically, ConceptNet contains mainly common-sense knowledge and also includes some lexicographic information, in contrast to WordNet which contains only lexicographic information.  Nonetheless, the fact that the best results are achieved by a model that uses both sources suggests that they may be complementary to each other.  Taken together, these findings offer substantial support for \textbf{Hypothesis 1}.  

Furthermore, we can see that the performance of the model using the random embeddings is very low, whereas the  ConceptNet-based model using unrelated state embeddings achieves a clearly better performance which yet remains significantly lower than that of the other CN-based models. 
The distinction between these approaches can be attributed to the
distribution of their embeddings: the former model employs a balanced and representative distribution enabled by GNN which permits the model to map the learned  representations to the visual information of seen classes during the fine-tuning procedure. In contrast, the latter
model has a completely random distribution which cannot be mapped to the semantic representations. The unrelated embeddings do not provide leverage for the recognition of unseen classes, thus resulting in the overall mediocre performance of the model.  
% The distinction between these approached lies in their
% distribution: the former model employs a balanced and rep-
% resentative distribution enabled by GNN, while the latter
% model has a completely random distribution. This suggests
% that the fine-tuning process can yield competitive seen accuracy even with unrelated embeddings to the target labels as long as the distribution is appropriate. In contrast, achieving accuracy for unseen classes requires an exact mapping
% between the embeddings and the target states







\vspace*{0.2cm}\noindent\textbf{Number of max hops}:
We experiment with a hop equal to 2 and to 3 for both KGs. The results are shown in the first two columns of \autoref{tab:abl3}. We can observe that no consistent pattern can be identified.  The best average performance is achieved for the OSDD dataset at a hop count of 2, while best average performance is exhibited for  the CGQA-State dataset   at a hop count of 3. In the case of MIT-States, there is no clear winner, as hop 2 shows superior AUC and hop 3 exhibits superior HM. This suggests that introducing additional nodes beyond a certain limit may introduce noise, potentially  impacting negatively overall performance in specific cases, as observed in the OSDD dataset. This outcome is consistent with the \textbf{Hypothesis 1}.

\vspace*{0.2cm}\noindent\textbf{Node policy}:
We investigate two strategies for adding nodes to our knowledge graph: indiscriminate inclusion of all neighboring nodes and selective inclusion of only relevant nodes. To determine relevance in ConceptNet, we use the edge weight between the queried node and its neighbors as the inclusion criterion. In WordNet, we use the Wu-Palmer Similarity metric~\cite{wu1994verb} 
between the two nodes. Additionally, in WordNet, we explore a hierarchical policy of accepting candidate nodes only if their ancestors belong to certain generic categories, such as attributes or objects. 

From the results (as shown in the last two columns of \autoref{tab:abl3})   it is evident that adopting this policy leads to significant performance improvements across all three datasets. This finding complements the previous observation regarding the number of hops and further strengthens the notion that the presence of noisy nodes can have a detrimental effect on model performance. These results align with \textbf{Hypothesis 1}.
 

% More details about the characteristics of the different KGs that were used during the ablation  can be found in \autoref{tab:KGS}.

% \begin{table*}[t]
% %	\small
%     % \centering
%       \resizebox{0.73\textwidth}{!}{\begin{minipage}{\textwidth} 
%     \begin{tabular}{|c|cccc|cccc|cccc|}

% \hline\hline
%  \bf \multirow{2}{*}{Method}& \multicolumn{4}{|c|}{\textbf{OSDD}}  & \multicolumn{4}{|c|}{\textbf{CGQA-States}}  & \multicolumn{4}{|c|}{\textbf{MIT-States}}  \\      \cline{2-13} 
%  & \textbf{LSTM}& \textbf{GCN}   & \textbf{R-GCN}& \textbf{Tr-GCN}& \textbf{LSTM}& \textbf{GCN}   & \textbf{R-GCN}& \textbf{Tr-GCN}& \textbf{LSTM}& \textbf{GCN}   & \textbf{R-GCN}& \textbf{Tr-GCN} \\
 
% \cline{2-13}
% \hline

% \textbf{H2\_WN}  &   33.0/18.1 &   30.8/15.4  &   35.1/21.5 &  28.6/13.0  & 
%  37.7/27.5 &   39.9/32.9 &  40.4/31.5 &  38.2/28.1 & 
% 31.1/11.4  &  35.2/22.3 & 43.0/20.0 & 39.7/27.3 \\ 

% \textbf{H2\_CN}  
%  & 38.1/24.6 &  44.0/31.0 &  42.2/29.2 &   40.2/27.7
%  & 43.0/34.0 &   36.2/26.7 & 34.5/24.5 &   39.2/28.8 
% & 44.0/20.1 & 59.2/29.3  & 51.2/30.3 &  53.6/43.7 \\

% \textbf{H2\_CNWN} & 38.4/27.8 & 47.7/35.0 &  37.9/25.2 &  47.1/33.4 & 
%  43.0/32.2&42.6/35.2 &   43.5/33.7 &  35.8/24.5 & 

% 54.0/47.1& 53.8/2.1 &  47.8/37.2 &  52.7/42.7 \\  

% \textbf{H2\_RN}  & 15.7 / 7.1 &  12.3 / 5.4 &  16.9 / 8.0 & 19.4 / 9.3 & 
%  33.7 / 19.0 &   13.1 / 6.8 &     12.8 / 5.8 &   20.1 / 9.0 & 
% 23.0 / 4.3  &   34.7 / 22.1 &  24.7 / 17.6  &  33.8 / 21.1 \\   \hline \hline  

% \textbf{Aggr} & 39.0/25.7 & 40.0/27.0 & 42.9/29.9 &  43.2/30.3 & 28.3/37.8 & 30.6/40.2 &  29.0/38.1 &28.2/38.5 & 47.7/30.7 & 50.7/34.3  & 53.7/36.6 & 51.2/39.8  \\   \hline


    
% \hline
% \end{tabular}
% \end{minipage}}
% \label{tab:abl1}
% \caption{Ablation results for the framework architecture. The first four rows present the performance of four different models for each architecture, whereas the last row presents the aggregated average of all models. The first (second) value in each cell corresponds to the HM(AUC).}

% \end{table*}


\begin{table}[t]
%	\small

       \resizebox{0.81\textwidth}{!}{\begin{minipage}{\textwidth} 
    \begin{tabular}{|c|cccc|}

\hline\hline
 \bf \diagbox[innerleftsep=.05cm,innerrightsep=4pt]{{\bf {Dataset}}}{{\bf {Arch}}} & 
  \textbf{LSTM}& \textbf{GCN}   & \textbf{R-GCN}& \textbf{Tr-GCN} \\
 
%\cline{2-13}
\hline

\textbf{OSDD}  & 39.0/25.7 & 40.0/27.0 & 42.9/29.9 & \bf 43.2/30.3  \\
\hline  
\textbf{CGQA-States}  & 28.3/37.8 & 30.6/40.2 &  \textbf{29.0}/38.1 &28.2/\textbf{38.5} 
  \\ \hline  



\textbf{MIT-States} & 47.7/30.7 & 50.7/34.3  & \textbf{53.7}/36.6 & 51.2/\textbf{39.8}  \\   \hline  



    
\hline
\end{tabular}
\end{minipage}}

\caption{Ablation results for the framework architecture.  The first (second) value in each cell corresponds to the best HM (AUC). } 
% All values are aggregate averages.}
\label{tab:abl1}
\end{table}



\begin{table}[t]
%	\small
    % \centering
       \resizebox{0.72\textwidth}{!}{\begin{minipage}{\textwidth} 
    \begin{tabular}{|c|ccccc|}

\hline\hline
 \bf \diagbox[innerleftsep=.05cm,innerrightsep=4pt]{{\bf {Dataset}}}{{\bf {KG }}} & 
  \textbf{CN}& \textbf{WN}   & \textbf{CN+WN}&  \textbf{IE}&   \textbf{RN} \\ 
 

\hline

\textbf{OSDD}  &43.5/30.5 & 32.6/18.5 &  \bf 45.4/34.7  & 19.4/9.3   & 8.2/3.1\\ \hline

\textbf{CGQA-States}  &  39.2/29.1  & 37.9/27.4 & \bf 44.5/34.7 &  20.1/9.0  & 11.1/5.7 
  \\ \hline



\textbf{MIT-States} & 53.3/\textbf{42.6} & 38.5/26.6  & \textbf{54.0}/42.1    & 33.8/22.1 & 18.6/13.0 \\   \hline  
% \textbf{AGR} & 43.5/30.5 & 32.6/18.5 &  45.4/34.7 & 39.2/29.1  & 37.9/27.4 & 44.5/34.7 & 53.3/42.6 & 38.5/26.6  & 54.0/42.1 \\     


\end{tabular}
\end{minipage}}
\caption{Ablation results for the KG source. The first (second) value in each cell corresponds to the best HM (AUC). CN: ConceptNet. WN: WordNet, WN+CN: Model based on both ConceptNet and WordNet. IE: ConceptNet-Based Model with irrelevant embeddings.  RN:  Model with random embeddings. }
% The values in the three fist columns are aggregate averages.}
\label{tab:abl2}
\end{table}


% \begin{table*}[t]
%	\small
    % \centering
%       \resizebox{0.74\textwidth}{!}{\begin{minipage}{\textwidth} 
%     \begin{tabular}{|c|cccc|cccc|cccc|}

% \hline\hline
%  \bf \multirow{2}{*}{Method}& \multicolumn{4}{|c|}{\textbf{OSDD}}  & \multicolumn{4}{|c|}{\textbf{CGQA-States}}  & \multicolumn{4}{|c|}{\textbf{MIT-States}}  \\      \cline{2-13} 
%  & \textbf{H2\_NP}& \textbf{H2\_TH}   & \textbf{H3\_NP}&  \textbf{H3\_TH}&  \textbf{H2\_NP}& \textbf{H2\_TH}   & \textbf{H3\_NP}
%  &   \textbf{H3\_TH}&\textbf{H2\_NP}& \textbf{H2\_TH}   & \textbf{H3\_NP} & \textbf{H3\_TH} \\
 
% \cline{2-13}
% \hline
% \textbf{CN} & 
% 39.2 /25.5 &   43.0 /28.5 &  40.7 /26.1 & 43.5/31.5
% & 43.0 /32.2 &  38.8 /31.9 & 39.6 /31.5 &  38.5/26.4  &
%  42.5 /17.4 &   48.2 /25.3 & 52.4 /26.1  & 49.2/38.7 \\ 
% \textbf{WN}  & 43.5/30.6 & 43.7/30.7 &   42.4/28.2 &   40.1/26.3 & 
%  42.6/35.2 & 40.5/32.0  &  37.0/27.6 & 36.4/27.1 &
%  59.0/29.7 & 56.6/29.4 & 56.9/27.2   & 54.8/25.6  \\
% \textbf{CN+WN}  & 41.2/28.7 &   41.2/28.5 &   40.0/27.3 & 47.3/32.3 & 
% 43.5/33.7 &   41.6/31.6 &     46.2/39.2 &    38.8/28.6 &
% 51.2/30.2 &  51.8/30.0  &  58.4/28.4  & 57.2/46.5 \\  \hline 
% \textbf{Aggr} &42.3/29.6  &  43.2/30.4 & 39.6/26.6  & 41.0/27.6 &
% 30.1/39.8  & 28.4/37.7 & 22.1/32.3  &  31.4/41.0 &
% 51.7/35.5 & 52.5/36.8 & 44.0/32.1 & 54.8/36.5 \\
% \hline
% \end{tabular}
% \label{tab:abl3}
% \end{minipage}}
% \caption{Ablation results for the number of hops and the threshold policy. The first three rows present the average performance of each KG source combination for each option, whereas the last row presents the aggregated average of all models. The first (second) value in each cell corresponds to the best HM (AUC).}
% \end{table*}


% \begin{table}[t]
% %	\small
%     % \centering
%        \resizebox{0.86\textwidth}{!}{\begin{minipage}{\textwidth} 
%     \begin{tabular}{|c|cc||cc|}

% \hline\hline
%  \bf \diagbox[innerleftsep=.05cm,innerrightsep=4pt]{{\bf {Dataset}}}{{\bf {Arch}}} & 
%   \textbf{H2\_NP}& \textbf{H2\_THR}   &  \textbf{H3\_NP}& \textbf{H3\_THR}
 
% \cline{2-13}
% \hline

% \textbf{OSDD}  & 42.3/29.6  &  \bf 43.2/30.4 & 39.6/26.6  & 41.0/27.6 \\
% \hline
% \textbf{CGQA-States}  &30.1/39.8  & 28.4/37.7 & 22.1/32.3  &  \bf 31.4/41.0 
%   \\ \hline



% \textbf{MIT-States} & 51.7/35.5 & 52.5/\textbf{36.8} & 44.0/32.1 & \textbf{54.8}/36.5 \\   \hline  

% % \textbf{Aggr} &42.3/29.6  &  43.2/30.4 & 39.6/26.6  & 41.0/27.6 &
% % 30.1/39.8  & 28.4/37.7 & 22.1/32.3  &  31.4/41.0 &
% % 51.7/35.5 & 52.5/36.8 & 44.0/32.1 & 54.8/36.5 \\
% \hline
% \end{tabular}
% \label{tab:abl3}
% \end{minipage}}
% \caption{Ablation results for the number of hops and the threshold policy. The first three rows present the average performance of each KG source combination for each option, whereas the last row presents the aggregated average of all models. The first (second) value in each cell corresponds to the best HM (AUC).}
% \end{table*}




\begin{table}[t]
%	\small
    % \centering
       \resizebox{0.75\textwidth}{!}{\begin{minipage}{\textwidth} 
    \begin{tabular}{|c|cc||cc|}

\hline\hline
 \bf \diagbox[innerleftsep=.05cm,innerrightsep=3pt]{{\bf {Dataset}}}{{\bf {Hops/Policy}}} & 
  \textbf{H2}& \textbf{H3}   &  \textbf{NP}& \textbf{THR}
 
\\
\hline

\textbf{OSDD}  & \bf 43.1/30.6  &   41.0/27.6 & 38.8/25.3   & \bf 42.5/28.5 \\
\hline
\textbf{CGQA-States}  &30.3/39.5 & \bf 31.4/41.0 &  25.9/36.0  &   \bf 29.8/39.5
  \\ \hline



\textbf{MIT-States} & 52.3/\textbf{36.9} &  \textbf{54.8}/36.5 & 45.9/31.7 & \bf 56.0/42.3 \\   \hline  






% \textbf{Aggr} &42.3/29.6  &  43.2/30.4 & 39.6/26.6  & 41.0/27.6 &
% 30.1/39.8  & 28.4/37.7 & 22.1/32.3  &  31.4/41.0 &
% 51.7/35.5 & 52.5/36.8 & 44.0/32.1 & 54.8/36.5 \\
\hline
\end{tabular}
\end{minipage}}
\caption{Ablation results for the number of hops (column 1 and 2) and the threshold policy (column 3 and 4). The first (second) column refers to the average performance of  models which are based   on a KG with hop equal to 2 (3). The third (fourth) column refers to the average performance of  models which are based  on a KG created without (with) threshold policy. The first (second) value in each cell corresponds to the best HM (AUC).}
% All values are aggregate averages.}
\label{tab:abl3}

\end{table}


% \vspace*{0.2cm}\noindent\textbf{Results Analysis}:
% Regarding the GNN architecture, we can see that the Tr-GCN framework outperforms the rest of the framework for every different model combination by a clear margin in every dataset.
% The ablation for the different GNN architectures is presented in \autoref{tab:abl1}. We can see that   the Tr-GCN framework  outperforms the other frameworks in all datasets w.r.t. AUC metric. whereas it scores best w.r.t. HM metric in the OSSD and comes second in the two other datasets. The R-GCN framework exhibits the second-best performance, while the GCN framework comes in third and the LSTM framework exhibits the worst performance respectively. These findings are consistent with prior research in the domain of zero-shot object classification and  substantiate   \textbf{Hypothesis 2}.



% Furthermore, we can see that the model with the unrelated embeddings (CN\_H3\_UN\_Tr-GCN) achieves an accuracy in seen classes a performance similar to the model of the same characteristics and standard embeddings (CN\_H3\_Tr-GCN). However, CN\_H3\_UN\_Tr-GCN accuracy in unseen classes and its HM and AUC score is about three to four times inferior to CN\_H3\_Tr-GCN. In contrast, the random model performs poor in all four metrics.  The difference between the embeddings of the CN\_H3\_UN\_Tr-GCN and the random model concerns their distribution: in the former case the GNN allows a balanced and representative distribution, whereas in the latter case it is totally random.
% This indicates that the fine-tuning process can result in competing seen accuracy even if the embeddings utilized are unrelated to the target labels  as long as they are  distributed adequately. In contrast, the accuracy for the unseen classes depends on an exact mapping between the embeddings and the target states. 
% Additionally, it can be observed that the model employing irrelevant embeddings (CN\_H3\_UN\_Tr-GCN) performs similarly to the model using standard embeddings (CN\_H3\_Tr-GCN) in terms of accuracy for seen classes. However, the accuracy for unseen classes, as well as HM and AUC scores, are three to four times lower in CN\_H3\_UN\_Tr-GCN compared to CN\_H3\_Tr-GCN. On the other hand, the random model exhibits poor performance in all four metrics. The distinction between the embeddings of CN\_H3\_UN\_Tr-GCN and the random model lies in their distribution: the former model employs a balanced and representative distribution enabled by GNN, while the latter model has a completely random distribution. This suggests that the fine-tuning process can yield competitive seen accuracy even with unrelated embeddings to the target labels as long as the distribution is appropriate. In contrast, achieving accuracy for unseen classes requires an exact mapping between the embeddings and the target states. 


% Regarding the maximum number of hops (as shown in the first two columns of \autoref{tab:abl3}), no consistent pattern can be identified.  The best average performance is achived for the OSDD dataset at a hop count of 2, while best average performance is exhibited for  the CGQA-State dataset   at a hop count of 3. In the case of MIT-States, there is no clear winner, as hop 2 shows superior AUC and hop 3 exhibits superior HM. This suggests that introducing additional nodes beyond a certain limit may introduce noise, potentially  impacting negatively overall performance in specific cases, as observed in the OSDD dataset. This outcome is consistent with the \textbf{Hypothesis 1}.

% In terms of the maximum number of hops (first two columns of \autoref{tab:abl3}),  no consistent pattern detect can be detected , since in the case of OSDD the best average performance is achieved for hop equal to 2, while in the case of CGQA-State  the best average performance is achieved for hop equal to 3 and in the case of MIT-States there is no obvious winner(superior AUC for hop 2 and superior HM for hop 3). This outcome suggests that the introduction of additional nodes beyond a certain limit introduces noise that could affect adversely overall performance in some instances(in our case, this happens for OSDD dataset). 


% Concerning the node inclusion policy (as shown in the last two columns of \autoref{tab:abl3}),   it is evident that adopting this policy leads to significant performance improvements across all three datasets. This finding complements the previous observation regarding the number of hops and further strengthens the notion that the presence of noisy nodes can have a detrimental effect on model performance. These results align with the hypothesis stated earlier (\textbf{Hypothesis 1}).

% From the previous, it follows that the optimal version of our method should employ  the Tr-GCN arhictecture for its GNN module. Moreover, for the construction of its KG both  Concept and WordNet should be queried and a node inclusion policy should be followed during this process. Finally, regarding the number of maximum hops the optimal number seems to depend on the size of the KG and the type of the dataset. Determining the ideal number of hops may require experimentation and careful consideration of the specific requirements of the task at hand.

% Based on the previous findings, the optimal version of our method would utilize the Tr-GCN architecture for its GNN module. Additionally, for constructing the knowledge graph (KG), querying both Concept and WordNet would be beneficial. During this process, it is recommended to follow a node inclusion policy to ensure comprehensive representation.
% Regarding the number of maximum hops in the KG, the optimal value appears to be dependent on the size of the KG and the characteristics of the dataset. Finding the ideal number of hops may require experimentation and considering the specific requirements of the task at hand.


% Based on our previous findings, we recommend using the Tr-GCN architecture as the GNN module in our method, as it has demonstrated superior performance. Furthermore, querying both the Concept and WordNet repositories is advantageous for constructing the knowledge graph (KG). It is advisable to follow a node inclusion policy during the construction process to ensure a comprehensive representation of the data.
% Regarding the number of maximum hops in the KG, the optimal value depends on factors such as the size of the KG and the characteristics of the dataset. Determining the ideal number of hops may require experimentation and careful consideration of the specific requirements of the task at hand.


% \begin{table}[t]
	
%     \centering
%       \resizebox{0.7\textwidth}{!}{\begin{minipage}{\textwidth} 
%     \begin{tabular}{|l|r|r|r|l|}

% \hline\hline
% \textbf{KG} &   \textbf{N}  &  \textbf{E}   &  \textbf{RT} & \textbf{RC}\\ \hline\hline



% WN\_H2 & 70/54/49 &  321/223/105 &       5 &LX\\ \hline
% WN\_H3 & 429/311/295 &  873/680/655 &       5 &LX\\ \hline\hline


% CN\_H2 &   715/552/504  & 2,132/1,981/1,864  &    13 &CS    \\ \hline
% CN\_H3 & 2,139/1,872/1,788  &  2,542/2,194/2,103 &       24 &CS    \\ \hline 
% CN\_H2\_TH & 611/505/485 & 1710/1521/1415  &       12 &CS  \\ \hline
% CN\_H3\_TH & 12,733/9,839/9,212 &  29,794/25,105/24,292&       29 & CS \\ \hline \hline
  
  
% CN+WN\_H2 & 667/581/506 & 1,906/1,682/1,602  &       13 &CS \\ \hline
% CN+WN\_H2\_TH & 590/492/431 & 1,442/1,167/1,089 &       12 &CS/LX    \\ \hline
% CN+WN\_H3\_TH &  10,165/8,842/7,948 &  26,735/23,176/22,602   &       29 & CS/LX   \\ \hline 
%   \end{tabular}
  

   
%      \end{minipage}}
  
%    \caption{KGs Details. N: Number of Nodes. E: Number of Edges.  RN: Number of Different Relation Types between nodes. RC: Category of Relation Types. CS: Common-Sense. LX: Lexicographic. First/Second/Third number in the N and E columns refers to the KG for OSDD/CGQA-States/Mit-States dataset respectively.}

%     \label{tab:KGS}
        
% \end{table}


 






\section{Summary and Conclusions}
\label{sec:conclusions}
We have outlined the design of a small electromagnetic calorimeter, the Few-Degree Calorimeter (FDC), which is designed to cover the range of $-4.6 < \eta < -3.6$. The primary objective of this detector is to tag electrons within the $Q^2$ range of 0.1 to 1.0 GeV$^2$, thus enabling future research on the transition to perturbative QCD and the gluon-saturation regime.

The FDC design we present here incorporates the latest advancements in SiPM-on-tile calorimetry to create a modern and improved version of the ZEUS Beam Pipe Calorimeter and H1 Very Low $Q^{2}$ calorimeter. The incorporation of high-granularity 5D shower measurements (position, time, and energy) offered by this technology holds great potential for background tagging.

In conclusion, this document presents the first design that has the potential to close the EIC $Q^2$ gap while maintaining a compact and cost-effective solution. Considering the larger crossing-angle envisioned for the second-interaction region at the EIC, which results in a larger hole in the crystal ECAL acceptance, this design may offer further opportunities for optimization for the EIC Detector 2.


 %\newpage
{\small
\bibliographystyle{ieee_fullname}
\bibliography{egbib}
}
% \newpage
% \appendix

% % This is samplepaper.tex, a sample chapter demonstrating the
% LLNCS macro package for Springer Computer Science proceedings;
% Version 2.20 of 2017/10/04
%
\documentclass[runningheads]{llncs}
%
\usepackage{graphicx}
% Used for displaying a sample figure. If possible, figure files should
% be included in EPS format.
%
% If you use the hyperref package, please uncomment the following line
% to display URLs in blue roman font according to Springer's eBook style:
% \renewcommand\UrlFont{\color{blue}\rmfamily}

\begin{document}
%
\title{Multi-View Vertebra Localization and Identification from CT Images \\ Supplementary Material}
%
%\titlerunning{Abbreviated paper title}
% If the paper title is too long for the running head, you can set
% an abbreviated paper title here
%
\author{Paper ID: 534}
%
\authorrunning{Paper ID: 534}
% First names are abbreviated in the running head.
% If there are more than two authors, 'et al.' is used.
%
\institute{}
%
\maketitle              % typeset the header of the contribution

\begin{table}
\centering
\caption{Evaluation results on a large-scale in-house dataset collected from the practical clinics with 500 CT scans divided into 300 for training, 100 for testing, and 100 for validation. We train the model on the training dataset, and further evaluate it on the test and validation dataset with $K$ set to 10.}
\begin{tabular}{l|ll|ll}
\hline
     & \multicolumn{2}{l|}{Test dataset}              & \multicolumn{2}{l}{Validation dataset}         \\ \cline{2-5} 
     & \multicolumn{1}{l|}{Id-Rate(\%)} & L-Error(mm) & \multicolumn{1}{l|}{Id-Rate(\%)} & L-Error(mm) \\ \hline
Cer. & \multicolumn{1}{l|}{99.67}       & 1.31       & \multicolumn{1}{l|}{99.55}       & 1.51       \\
Tho. & \multicolumn{1}{l|}{98.24}       & 1.34       & \multicolumn{1}{l|}{99.00}       & 1.48       \\
Lum. & \multicolumn{1}{l|}{99.31}       & 1.35       & \multicolumn{1}{l|}{99.54}       & 1.50       \\ 
All  & \multicolumn{1}{l|}{98.62}       & 1.34       & \multicolumn{1}{l|}{99.04}       & 1.49       \\ \hline
\end{tabular}
\end{table}

% Figure environment removed


\end{document}

\end{document}