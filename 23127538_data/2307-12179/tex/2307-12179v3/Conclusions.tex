%\vspace{-.25cm}
\section{Summary}

This work introduced OaSC, a novel method for zero-shot object-agnostic state classification.  OaSC leverages knowledge graphs and graph neural networks to infer object states without relying on object class information, enabling it to generalize to unseen objects.  Our extensive evaluation on four benchmark datasets demonstrated OaSC superior performance compared to SOTA CZSL methods. Furthermore,   the extensive comprehensive ablation study provided valuable insights into the impact of different design choices on the method's performance. %, highlighting the importance of KG construction and GNN architectures. 
%Future work will explore incorporating larger, more diverse knowledge graphs and investigate the applicability of OaSC in real-world scenarios, such as robotics and human-computer interaction, where recognizing states of unknown objects is crucial


%We proposed OaSC, a novel approach for the task of zero-shot object state classification.  OaSC is the first truly object-agnostic state classification approach. By being object-agnostic, OaSC has the potential to infer the states of objects that have never been observed during training. 
%of objects it has is more robust and generic compared to existing methods which depend on  object class classification.
%We evaluated our approach on four benchmark datasets in comparison with several recent SOTA CZSL approaches that perform object-aware state classification. OaSC outperforms all of them in all four  datasets. Moreover, an extensive ablation study evaluated several design options and shed light on important aspects of the object state estimation problem. 

%\cite{gouidis2022,Mancini2022,Isola2015,Pham2021CVPR} 
%In terms of future directions, firstly, we aim to experiment with the tuning of the GNN by using a classifier pre-trained on attribute classes or object-attributes pairs instead of object classes. Secondly, we would like to examine whether the exclusive inclusion of nodes related to objects into KGs leads to better results. 
% Overall,  we consider that the zero-shot state classification in the object-agnostic setting is worth further investigation. We hope that our work will encourage future efforts towards this direction.
 %However, implementing this approach is challenging as the available KG sources do not provide a reliable method for measuring the relevance of concepts.
 
\vspace*{0.0cm}\noindent\textbf{Acknowledgements} The Hellenic Foundation for Research and Innovation (H.F.R.I.) funded this research project under the 3rd Call for  H.F.R.I. Research Projects to support Post-Doctoral Researchers (Project Number 7678 InterLinK: Visual Recognition and Anticipation of Human-Object Interactions using Deep Learning, Knowledge Graphs and Reasoning) and under the “1st Call for H.F.R.I Research Projects to support Faculty members and Researchers and the procurement of high-cost research equipment”, project I.C.Humans, (Project Number  91).

%\vspace*{0.0cm}\noindent\textbf{Acknowledgements} The Hellenic Foundation for Research and Innovation (H.F.R.I.) funded this research under the Projects 1) InterLinK (n. 7678) and 2) I.C.Humans (n. 91).