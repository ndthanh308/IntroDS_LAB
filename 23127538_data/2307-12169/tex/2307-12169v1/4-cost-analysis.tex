\section{Network Cost Analysis} \label{sec:cost}

Our rail-only network architecture judiciously reduces the network resources for LLM training by eliminating unused network connections. This section compares the network cost of our proposed approach with the state-of-the-art rail-optimized GPU clusters. We calculate the number of switches (\#SW) and transceivers (\#TR) required for each network design and derive the total network equipment cost based on numbers reported in prior work~\cite{topoopt}\footnote{\$374 per transceiver, \$748 per switch port for 400~Gbps.}. This section focuses on using only electrical packet switches to construct the network; however, using optical direct connect technology can provide further cost reductions~\cite{jupiterevolving}. We defer the discussion about direct-connect networks to Section~\ref{sec:discussions}.

We enumerate the number of switches and transceivers required to build both the state-of-the-art network architecture and our proposed architecture in Table~\ref{tab:nw_swich}, accounting for variable cluster sizes and network switch radix. Note that for the state-of-the-art architecture, to use the least amount of network resources, each rail of GPUs is not physically separated in some cases. Thus, the datacenter operator must manually configure the switch to achieve the desired isolation across rails to achieve the rail-optimized design. 

The last column of Table~\ref{tab:nw_swich} illustrates our design's cost savings over that of the state-of-the-art for the total cost of switches and transceivers. Our rail-only design notably reduces the network cost by 37\% to 75\% compared to the state-of-the-art design while achieving equivalent performance. This reduction stems from eliminating core layer switches and decreasing the number of switch tiers within each rail. Surprisingly, switches with a radix of 64 provide the worst-case cost reduction in both cluster sizes. In this case, the state-of-the-art design requires a three-tier \fattree network, while the rail-only design requires two tiers for each rail. Still, our design only requires three-quarters of the total number of switches while simultaneously achieving the same performance as the state-of-the-art design. 

\begin{table}[t]
\scriptsize
\centering
\renewcommand{\arraystretch}{1} 
\linespread{1.05}\selectfont\centering
    \begin{tabular}{|P{0.8cm}|P{0.8cm}|P{0.8cm}|P{0.8cm}|P{0.8cm}|P{0.8cm}|P{0.8cm}|}
    \hline
    \textbf{\# of GPUs (N)} & \textbf{Switch Radix} & \textbf{SOTA \#SW} & \textbf{Rail-only \#SW} & \textbf{SOTA \#TR} & \textbf{Rail-only \#TR} & \textbf{Cost Reduction} \\ 
    \hline
        \multirow{4}{*}{32768}   & \multicolumn{1}{c|}{32} & \multicolumn{1}{c|}{7168} & \multicolumn{1}{c|}{3072} & \multicolumn{1}{c|}{262144} & \multicolumn{1}{c|}{131072} & \multicolumn{1}{c|}{54\%} \\\cline{2-7}
                                 & \multicolumn{1}{c|}{64} & \multicolumn{1}{c|}{2560} & \multicolumn{1}{c|}{1536} & \multicolumn{1}{c|}{196608} & \multicolumn{1}{c|}{131072} & \multicolumn{1}{c|}{37\%} \\\cline{2-7}
                                 & \multicolumn{1}{c|}{128} & \multicolumn{1}{c|}{1280} & \multicolumn{1}{c|}{256} & \multicolumn{1}{c|}{196608} & \multicolumn{1}{c|}{65536} & \multicolumn{1}{c|}{75\%} \\\cline{2-7}
                                 & \multicolumn{1}{c|}{256} & \multicolumn{1}{c|}{384} & \multicolumn{1}{c|}{128} & \multicolumn{1}{c|}{131072} & \multicolumn{1}{c|}{65536} & \multicolumn{1}{c|}{60\%} \\\hline
        \multirow{3}{*}{65536}   & \multicolumn{1}{c|}{64} & \multicolumn{1}{c|}{5120} & \multicolumn{1}{c|}{3072} & \multicolumn{1}{c|}{393216} & \multicolumn{1}{c|}{262144} & \multicolumn{1}{c|}{37\%} \\\cline{2-7}
                                 & \multicolumn{1}{c|}{128} & \multicolumn{1}{c|}{2560} & \multicolumn{1}{c|}{1536} & \multicolumn{1}{c|}{393216} & \multicolumn{1}{c|}{262144} & \multicolumn{1}{c|}{37\%} \\\cline{2-7}
                                 & \multicolumn{1}{c|}{256} & \multicolumn{1}{c|}{1280} & \multicolumn{1}{c|}{256} & \multicolumn{1}{c|}{393216} & \multicolumn{1}{c|}{131072} & \multicolumn{1}{c|}{75\%} \\\hline
    \end{tabular}
    \caption{Number of switches for different clusters.}
    \vspace*{-0.3cm}
    \label{tab:nw_swich}
\end{table}

