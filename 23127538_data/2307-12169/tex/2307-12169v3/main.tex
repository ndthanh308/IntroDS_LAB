%%%%%%%% mlsys 2024 EXAMPLE LATEX SUBMISSION FILE %%%%%%%%%%%%%%%%%

\documentclass{article}

% Recommended, but optional, packages for figures and better typesetting:
\usepackage{microtype}
\usepackage{graphicx}
\usepackage{subfigure}
\usepackage{booktabs} % for professional tables

% hyperref makes hyperlinks in the resulting PDF.
% If your build breaks (sometimes temporarily if a hyperlink spans a page)
% please comment out the following usepackage line and replace
% \usepackage{mlsys2024} with \usepackage[nohyperref]{mlsys2024} above.
\usepackage{hyperref}

% Attempt to make hyperref and algorithmic work together better:
\newcommand{\theHalgorithm}{\arabic{algorithm}}

% Use the following line for the initial blind version submitted for review:
\usepackage[accepted]{mlsys2024}
%\usepackage[subtle,title=tight]{savetrees} 
%\usepackage[small,compact]{titlesec}
% \usepackage{amsthm}
% \usepackage[english]{babel}
\usepackage{ifthen}
\usepackage{xcolor}
% \usepackage{blindtext}
% \usepackage{algorithm}
% %\usepackage{subfigure}
% \usepackage{graphicx}
\usepackage{amsmath}
% \usepackage[noend]{algpseudocode}
% \usepackage{subfig}
\usepackage{xspace}
\usepackage{amssymb} 
\usepackage{multirow}
\usepackage{url}
\usepackage{hyperref}
% \usepackage{balance}
% \newtheorem{remark}{Remark}
% \usepackage{graphicx}
% \usepackage{footnote}
% \usepackage{comment}
\usepackage{array}
\newcolumntype{P}[1]{>{\centering\arraybackslash}p{#1}}

\usepackage{url}
\def\UrlBreaks{\do\/\do-}

\newcommand{\exclude}[1]{}
\newcommand{\showComments}{yes}
\newcommand{\note}[2]{
    \ifthenelse{\equal{\showComments}{yes}}{\textcolor{#1}{#2}}{}
}
\newcommand{\TODO}[1]{%
  \addcontentsline{tdo}{todo}{\protect{#1}}%
  \note{red}{TODO: #1}
}

\newcommand\numberthis{\addtocounter{equation}{1}\tag{\theequation}}

\newcommand{\frank}[1]{\note{brown}{[WW: #1]}}
\newcommand{\manya}[1]{\note{red}{[MG: #1]}}
\newcommand{\naader}[1]{\note{brown}{[NH: #1]}}
\newcommand{\kayvon}[1]{\note{violet}{[KS: #1]}}
\newcommand{\ying}[1]{\note{green}{[YZ: #1]}}


\newcommand{\psass}{\ensuremath{\mathbin{{=}}\ }}
\newcommand{\addeq}{\ensuremath{\mathbin{{+}{=}}\ }}
\newcommand{\subeq}{\ensuremath{\mathbin{{-}{=}}\ }}
\newcommand{\muleq}{\ensuremath{\mathbin{{\times}{=}}\ }}
\newcommand{\diveq}{\ensuremath{\mathbin{{\divides}{=}}\ }}
\newcommand{\eqeq}{\ensuremath{\mathbin{{=}{=}}\ }}
\newcommand{\todo}[1]{{\color{red} #1}}
\newcommand{\name}{{\sc{OEAINet}}\xspace}
\newcommand{\cc}{{\sc{c$^2$}}\xspace}

\newcommand{\fattree}{{Clos}\xspace}
\newcommand{\fattrees}{{Clos}\xspace}
\newcommand{\LBE}{{\fattree}\xspace}
\newcommand{\SBE}{{Ideal Switch}\xspace}
\newcommand{\OBE}{{Oversub. \fattree}\xspace }
\newcommand{\para}[1]{{\textbf{{#1}}}}
\newcommand{\net}{{{Big-Net}}\xspace}
\newcommand{\MP}{{MP}\xspace}
\newcommand{\allreduce}{{AllReduce}\xspace}
\newcommand{\ata}{{All-to-All}\xspace}
\newcommand{\allgather}{{AllGather}\xspace}
\newcommand{\redsca}{{Reduce-Scatter}\xspace}
\newcommand{\fbd}{{Full-Bisection Domain}\xspace}
\newcommand{\dcd}{{Direct-Connected Domain}\xspace}



\newcommand{\captionvspace}{0em}
\pagestyle{plain}


\newenvironment{CompactItemize}
  {\def\usecounter{\compactify\latexusecounter}
   \begin{itemize}}
  {\end{itemize}\let\usecounter=\latexusecounter}
\date{}
\usepackage{lipsum} % for dummy text
\usepackage{enumitem}
\setlist{nosep} % or \setlist{noitemsep} to leave space around whole list


% If accepted, instead use the following line for the camera-ready submission:
% \usepackage[accepted]{mlsys2024}

% The \mlsystitle you define below is probably too long as a header.
% Therefore, a short form for the running title is supplied here:
\mlsystitlerunning{Submission and Formatting Instructions for MLSys 2024}
% \renewcommand{\vspace}[1]{\null}

\begin{document}
% \setlength{\abovedisplayskip}{3pt}
% \setlength{\belowdisplayskip}{3pt}
% \everypar{\looseness=-1}
\twocolumn[
\mlsystitle{How to Build Low-cost Networks for Large Language Models (without Sacrificing Performance)?}

% It is OKAY to include author information, even for blind
% submissions: the style file will automatically remove it for you
% unless you've provided the [accepted] option to the mlsys2024
% package.

% List of affiliations: The first argument should be a (short)
% identifier you will use later to specify author affiliations
% Academic affiliations should list Department, University, City, Region, Country
% Industry affiliations should list Company, City, Region, Country

% You can specify symbols, otherwise they are numbered in order.
% Ideally, you should not use this facility. Affiliations will be numbered
% in order of appearance and this is the preferred way.
\mlsyssetsymbol{equal}{*}

\begin{mlsysauthorlist}
\mlsysauthor{Weiyang Wang}{mit}
\mlsysauthor{Manya Ghobadi}{mit}
\mlsysauthor{Kayvon Shakeri}{meta}
\mlsysauthor{Ying Zhang}{meta}
\mlsysauthor{Naader Hasani}{meta}
\end{mlsysauthorlist}

\mlsysaffiliation{mit}{MIT CSAIL}
\mlsysaffiliation{meta}{Meta}

% \mlsyscorrespondingauthor{Weiyang Wang}{weiyangw@mit.edu}
% \mlsyscorrespondingauthor{Manya Ghobadi}{ghobadi@csail.mit.edu}
% You may provide any keywords that you
% find helpful for describing your paper; these are used to populate
% the "keywords" metadata in the PDF but will not be shown in the document
\mlsyskeywords{Machine Learning, MLSys}

\vskip 0.3in

\begin{abstract}

This paper presents a low-cost network architecture for training large language models (LLMs) at hyperscale. We study the optimal parallelization strategy of LLMs and propose a novel datacenter network design tailored to LLM's unique communication pattern. We show that LLM training generates sparse communication patterns in the network and, therefore, does not require any-to-any full-bisection network to complete efficiently. As a result, our design eliminates the spine layer in traditional GPU clusters. We name this design a \textit{Rail-only} network and demonstrate that it achieves the same training performance while reducing the network cost by 38\% to 77\% and network power consumption by 37\% to 75\% compared to a conventional GPU datacenter. Our architecture also supports Mixture-of-Expert (MoE) models with all-to-all communication through forwarding, with only 8.2\% to 11.2\% completion time overhead for all-to-all traffic. We study the failure robustness of Rail-only networks and provide insights into the performance impact of different network and training parameters. \looseness=-1


\end{abstract}


]

% this must go after the closing bracket ] following \twocolumn[ ...

% This command actually creates the footnote in the first column
% listing the affiliations and the copyright notice.
% The command takes one argument, which is text to display at the start of the footnote.
% The \mlsysEqualContribution command is standard text for equal contribution.
% Remove it (just {}) if you do not need this facility.

%\printAffiliationsAndNotice{}  % leave blank if no need to mention equal contribution
\printAffiliationsAndNotice{} % otherwise use the standard text.

\section{Introduction} \label{sec:introduction}

LLMs are among the largest and most computationally intensive Deep Neural Networks (DNNs). The latest GPT4 model is estimated to have trillions of parameters and take months to train~\cite{openai2023gpt4, gpt4params}. 
Conventionally, researchers seek to enhance the performance of distributed DNN training and inference through optimizing parallelization strategies~\cite{flex_flow, alpa, unity, tofu}, sophisticated scheduling~\cite{muri, gandiva, tiresias}, advanced compression~\cite{grad_comp}, and even the reconfiguration of the network topology itself~\cite{sipml, topoopt, zhao2022optimal}. Despite these efforts, LLMs still require significant raw computing power. The GPT3 model from 2020 already requires 355 GPU-years on Nvidia's V100 GPUs~\cite{gpt3, gpt3time}. As Moore's law slows down, the growth rate of LLM size and computation requirement exceeds the advancement of accelerators, making hyper-scale GPU clusters inevitable. 
Our conversations with lead machine learning architects in the industry indicate that the next-generation LLMs likely require over 30,000 GPUs of computing power to finish training within a reasonable time. At the same time, scaling the cluster to 32,000 GPUs also allows LLM designers to train smaller models like LLaMa-65B~\cite{touvron2023llama} within a day~\cite{meta-ethernet}, expediating future development.

A GPU-centric cluster typically employs two types of interconnection domains~\cite{dgxh100archdoc}. First, a high-bandwidth domain where a few GPUs (e.g., eight for a DGX H100 server) are interconnected with high bandwidth, but short-range communication links like NVLinks~\cite{nvlnvs}. We refer to this type of interconnection as the HB domain. 
The second interconnection domain forms a network capable of any-to-any GPU communication using RDMA-capable NICs, connected in a \fattree network architecture. The cluster uses the RDMA protocol on this network to benefit from bypassing CPU and OS entirely through GPU-Direct~\cite{mlnxrdmagpu, gpudirect}. 

However, scaling up RDMA networks to tens of thousands of GPUs is challenging. Previous work demonstrated that large-scale RDMA networks are prone to deadlocking and PFC storms~\cite{pfc_storm, rdma_azure, ib_deadlock, bfc,ddlindc}, degrading the performance. Furthermore, as the scale increases, \fattree architectures become prohibitively costly~\cite{topoopt}. Datacenter providers resort to over-subscription to tame costs, worsening the deadlocking problems. Prior work proposed several techniques to enable large-scale RDMA networks and reduce their cost~\cite{rdma_cc,rdma_azure,srnic,revisitrdmanet}. These approaches fundamentally depend on the assumption that the network is capable of any-to-any communication.
This assumption forms the bedrock upon which datacenters have been conceptualized and developed for several decades. 

In this paper, we challenge this assumption and show that LLM training traffic \textit{does not} require any-to-any connectivity across all GPUs in the network. This paper makes three primary contributions. First, we analyze the traffic pattern of training dense LLMs (\S\ref{sec:traffic_ana}). We demonstrate that with an optimal parallelization strategy, an LLM training workload requires high-bandwidth any-to-any connectivity \textit{only within discrete subsets of GPUs}, and each subset fits within an HB domain. Across the HB domains, communication only occurs between a few GPU pairs with the same rank in their respective HB domains, and the traffic volume is insignificant. As a result, the conventional any-to-any approach for building datacenter networks adds unnecessary complexity and cost for distributed LLM training.

Motivated by the above observations, we propose a low-cost, high-efficiency network architecture that accurately reflects LLM communication requirements that we name ``rail-only" (\S\ref{sec:system-arch}). In our architecture, a cluster is partitioned into multiple HB domains, similar to conventional \fattree architectures. Across the HB domains, however, instead of forming a \fattree to support any-to-any communication, the network only connects sets of GPUs with non-zero network traffic. Compared to the state-of-the-art \fattree design, our network architecture removes the network equipment that does not carry traffic and achieves the same performance as a \fattree network. We also examine our design's fault-tolerance properties and provide recovery methods from failure cases. \looseness=-1

Finally, we derive an analytical formula for accurately estimating the training iteration time of LLMs (\S\ref{sec:sys-model}). This formulation provides insights into the training performance of our network design with different LLM parallelization strategies. Unlike previous approaches~\cite{narayanan2021efficient}, our formulation explicitly considers both the computation and communication time, given the LLM hyperparameters, the parallelization strategy, and the training infrastructure. We compare our formulation to published results to validate its accuracy and show that for LLMs with over one trillion parameters, our formulation estimates the training iteration time within 0.15\% of the ground truth in hardware FLOP utilization (\S\ref{sec:accuracy_model}). 
\looseness=-1

We evaluate the performance of a rail-only network architecture using our analytical formulation and provide insights into the performance impact of different network and training parameters. Our evaluation indicates that an HB domain of size 256 provides near-optimal performance, within 8.9\% of the optimal training iteration time compared to the ideal case where all GPUs reside in a monolithic HB domain. We also show that surprisingly small LLMs exhibit more network overhead than larger ones and demand more network optimization. We discuss the reasons behind this counter-intuitive result (\S\ref{subsec:hbd-size}). 
In addition, we compare the cost of our proposal (\S\ref{sec:cost}) to a full-bisection bandwidth any-to-any \fattree cluster that achieves the same performance and show that our LLM-centric network architecture reduces the network cost by 37\% to 75\%. \looseness=-1

\section{Uncertainty}
\label{sec:background}
Uncertainty is a rich concept that has received various reasonable treatments before today's understanding of it.\footnote{From uncertainty's connection to (mostly abandoned) views on what is `knowable' \citep{knight1921risk}, to its central role in decision theories  \citep{ramsey1931foundations,von1947theory,Wald1951StatisticalDF,bernardo1994bayesian} and mathematical statistics \citep{savage1972foundations} to its modern understanding in terms of state of knowledge \citep{morgan_henrion_1990,lindley2013understanding}, to its  mathematical representation detached from philosophical interpretation \citep{halpern2017reasoning}.} We begin discussing it through common language. The online edition of the Oxford English dictionary listed five senses of uncertainty (retrieved in May 2023), two of which we partly quote here (those general enough to include the others as special case): \emph{(i) the state of not being definitely known or perfectly clear}; and \emph{(ii) the state or character of being uncertain in mind}. Both definitions regard uncertainty as \emph{a state of affairs}: in \emph{(i)}, the state of the world; in \emph{(ii)}, the state of an agent contemplating the world. They are subtly different: \emph{(i)} encompasses situations of inherent randomness (\eg, the result of a coin flip), \emph{(ii)} concerns one's inability to predict the state of the world regardless of any inherent randomness (\eg, a reader wondering about the content of the next paragraph). As we shall see, this difference leads to rather different interpretations %
of uncertainty as an aspect of reality. Yet, at the level of mathematical treatment, they share the same formal devices. Hence, with no loss of generality, we choose to talk about uncertainty from the point of view of an agent contemplating or interacting with the world, while possessing limited knowledge about it. Our presentation is inspired by various reference texts,  in particular, \citet{dubois2009formal} and \citet{halpern2017reasoning}. 

\paragraph{Agents.} We posit that any one agent shall represent the state of their knowledge in a way sufficient for reasoning about the truth value of claims (or propositions) that they make about aspects of the world. In particular, the agent is able to state their preference for claims they find themselves less uncertain about (\ie, possessing better information about those).\footnote{Agents and worlds are abstractions to be adapted and tailored to each application, commonly in NLG an agent is a model and a world is a response to a given prompt.}  An agent then uses this \emph{uncertainty representation} to interact with the environment (\eg, inform their actions) and, when they acquire new knowledge, they update the representation in a coherent manner. 
To illustrate formal concepts, we use three example agents. \textbf{A1}\hspace{0.5mm}\twemoji{game die} rolls a six-sided die; we seek to represent their state of knowledge about the outcome. \textbf{A2}\hspace{0.5mm}\twemoji{busts in silhouette} resolves mentions of entities to unique names in a knowledge base (KB); we seek to represent their state of knowledge about entity names given any one mention. Last, \textbf{A3}\hspace{0.5mm}\twemoji{speech balloon} provides written answers to questions; we seek to represent their state of knowledge about answers given any one question. 
For simplicity, we assume that our agents already acquired their knowledge, by means which are not relevant for now, and their state of knowledge is frozen.  
We begin by outlining the formal tools common to all frameworks for uncertainty representation we are aware of, we then zoom into the most commonly used framework (probability) and discuss the role of statistics in acquisition and revision of knowledge.  

\paragraph{Possible worlds.} Our agent does not know the state of the actual world, but they assume that it must be one of a collection of possible worlds (the universe). They represent a world as a unique symbol (or string, or collection of attributes; the level of detail being dictated by the agent's needs), and the universe of what is possible as a set $\Omega$ of mutually exclusive worlds.\footnote{This framework, \textit{possible worlds}, is familiar to linguists and philosophers alike \citep{hintikka1957modality,hintikka1961modality,sep-possible-worlds}.
} 
\textbf{A1}\hspace{0.5mm}\twemoji{game die}  might represent a world as a symbol $f_k$, with $k$ denoting the number of pips the die shows as a result of the roll; they might assume the die always lands showing one of six numbered faces and thus take $\{f_1, \ldots, f_6\}$ to represent all possible worlds. 
For \textbf{A2}\hspace{0.5mm}\twemoji{busts in silhouette}, a world is a symbol like $e_i$, with $i$ denoting an entity's identifier (\eg, a standardised unique name), and the universe is the finite set of entities in the English Wikipedia. 
For \textbf{A3}\hspace{0.5mm}\twemoji{speech balloon}, a world is a symbol like $u_s$, with $s$ an English sentence produced in response to a question. This agent happens to be unable to describe the set of all valid English sentences (they cannot enumerate its elements nor state a finite set of properties that all valid sentences must satisfy). Motivated by convenience, \textbf{A3}\hspace{0.5mm}\twemoji{speech balloon} uses a set large enough to encompass most of it while being specifiable in a compact manner: the set of all finite-length strings made by concatenation of known symbols (\eg, words, punctuation, \etc). These examples show that the agent's choice of universe can be a difficult one, often requiring simplifying assumptions: on soft or irregular terrain, a die could land on an edge; a KB may be incomplete (sometimes in known ways, \eg, under-representing the contributions of Black women to science); a regular language is a too loosely constrained representation of the English language (\eg, it includes infinitely many strings that will never correspond to any actual world). 

\paragraph{Propositions.} The possible worlds framework gives agents a mechanism to represent claims about specific aspects of the world. A \emph{proposition}  $E_t$ is the claim that the actual world $\omega$ is one where some property $t$ holds (which we denote $t(\omega)$). A property is something that can be assessed for any one world (\eg, $f_2$ is even and prime, $e_{\operatorname{Katherine\_Johnson}}$ is African-American, female and mathematician, $u_\text{`Biden is the 46th US president'}$ expresses the relation $\operatorname{presidentof}(\operatorname{Joe\_Biden}, \operatorname{USA})$). Not knowing the state of the actual world,  our agent represents $E_t$ by the set  $E_t = \{w \in \Omega: t(w)\} \subseteq \Omega$ of all possible worlds where the property holds. If the agent knew the state of the actual world $\omega$, then the truth value of the proposition would be determined simply by set membership (\ie, $\omega \in E_t$ or $\omega \not\in E_t$). 
For example, \textbf{A1}\hspace{0.5mm}\twemoji{game die}  represents the claim ``the roll is odd'' as $E_{\text{odd}} = \{f_1, f_3, f_5\}$.
\textbf{A2}\hspace{0.5mm}\twemoji{busts in silhouette} represents the claim ``mention to a female mathematician'' by the set $\{e_i \in \Omega: \operatorname{female}(e_i) \wedge \operatorname{mathematician}(e_i)\}$. %
\textbf{A3}\hspace{0.5mm}\twemoji{speech balloon} might use equivalence classes, for example, they use the set $E_a = \{u_s \in \Omega : \operatorname{equivalent}_{a}(u_s)\}$ to claim that the answer is a sentence semantically equivalent to some other sentence $u_a \in \Omega$ (\eg, $u_\text{`The 46th US president is Joe Biden'}$). Because propositions are semantic in nature, they can be difficult to represent explicitly. For example, \textbf{A3}\hspace{0.5mm}\twemoji{speech balloon}'s equivalence classes require sophisticated natural language understanding. 
A representation of all propositions an agent deems possible is a set $\mathcal E$ of subsets of $\Omega$.\footnote{If an agent has no knowledge of the impossibility of any proposition, or does not care to exclude those from the representation, the powerset of (countable) $\Omega$ is a reasonable choice for $\mathcal E$. In NLG, we often implicitly make this choice.}

\paragraph{Preferences.} The agent's imperfect knowledge of the actual world $\omega$ translates to limited knowledge about propositions. However, the agent's ignorance is qualitatively different depending on the claims they make. Intuitively, some claims are compatible with many of the possible worlds, while others hold in but a few (\eg, \textbf{A1}\hspace{0.5mm}\twemoji{game die}  knows that only one prime number is even), and though the various worlds are all possible, they may not be equally plausible (\eg, \textbf{A2}\hspace{0.5mm}\twemoji{busts in silhouette} knows that most mentions resolve to politicians, \textbf{A3}\hspace{0.5mm}\twemoji{speech balloon} knows that most answers are only a few words long), \etc.
Considerations of those kinds motivate an agent to express a \emph{preference} for claims they find themselves less uncertain about (\ie, possessing better information about those). The agent does so by prescribing a \emph{plausibility measure} \citep{friedman96}, a function that attaches a token of uncertainty---a qualifier that the agent knows how to sort---to each proposition in $\mathcal E$. Plausibility measures are very diverse, the most well known instance of it being axiomatic probability \citep{kolmogorov1960foundations}.\footnote{Other plausibility measures include belief functions \citep{shafer76}, possibility measures \citep{duboisprade90}, ordinal ranking functions \citep{GoldszmidtPearl92} and (non-numerical) preference orders \citep{friedman96}. 
Concrete instances of plausibility measures vary in descriptive power. Under certain documented assumptions \citep{friedman96}, they enable something like a `calculus of uncertainty' which formalises the procedures the agent must follow to incorporate additional information about the world and revise their uncertainty representation coherently (in axiomatic probability, this is known as \emph{conditioning}).} 

\paragraph{Probability.} Probability is a numerical qualifier that we can attach to events in $\mathcal E$.\footnote{In probability, propositions are  \emph{events}, worlds are \emph{outcomes} and universes are \emph{sample spaces}.} 
For any event, this qualifier is a positive real number bounded to be at most $1$. Probability values inherit various properties of real numbers: we can add, multiply and sort them. A function $\Pr$ over $\mathcal E$ is a \emph{probability measure} if a) it maps $\Omega$ to $1$, and b) it maps any two disjoint sets $U$ and $V$ in $\mathcal E$ to $\Pr(U)+\Pr(V)$, which is known as additivity. 
 Additivity, in particular, implies that we can identify a probability measure over all of $\mathcal E$ by assigning probability to elementary outcomes in $\Omega$, for example, using a probability mass function (pmf) or probability density function (pdf; for uncountable $\Omega$). This has massive consequences for uncertainty representation, since working with elementary outcomes is much simpler than working with sets of outcomes (for example, difficulty in prescribing equivalence classes such as `all sentences that talk about Joe Biden' need not stop \textbf{A3}\hspace{0.5mm}\twemoji{speech balloon} from identifying a probability measure for their reasoning needs).
 
\paragraph{Interpretations.} 
Probability has been motivated and justified from different angles, each building on a specific interpretation of what probability as a number must signify \citep{hacking1975emergence}. However different they are, they all lead to the same formal device. Under certain idealisations, \textit{objectivists} regard events as \emph{repeatable} (\eg, we may prompt a human speaker multiple times). Repetitions allow an agent to perceive what may be thought of as an inherent \emph{property} of an event: its \emph{frequency} in a large enough number of repetitions. The \emph{subjectivist} interpretation \citep{ramsey1931foundations,definetti2017theory} views probability as a degree of belief, personal to an agent, and deprived of any interpretation beyond its formal role as an expression of the agent's preferences.\footnote{Dictionary definition \textit{(i)} is objectivist; \textit{(ii)} subjectivist.} 
Different interpretations have an impact on the procedures that an agent acknowledges as logical or rational for knowledge acquisition and revision, as we discuss next.

\paragraph{Statistics.} We have described the general tools that agents can use to represent and convey their uncertainty. But where do their preferences (probabilities, in particular) come from? The \emph{Frequentist} agent is essentially an objectivist who assumes the existence of a precise statistical law that describes the phenomena in consideration. They assume to have access to this law up to an unknown parameter $\theta^\star \in  \mathbb R^D$. %
Given a parameter $\theta$, their preferences are specified via a pmf (or pdf) $p(x|\theta)$. Given data $\mathbf x = \langle x_1, \ldots, x_N \rangle$, this law identifies the so called likelihood function $\ell_{\mathbf x}(\theta) = \prod_n p(x_n|\theta)$, a measure of the compatibility between observed data $\mathbf x$ and the statistical model identified by $\theta$. The agent uses $\mathbf x$ to estimate the parameter $\theta^\star$: they pick the parameter $\hat\theta$ that maximises the likelihood function, %
this is known as maximum likelihood estimation (MLE). They do not entertain parameters as part of the possible worlds, hence have no uncertainty representation about them. Given the  parameter  estimate $\hat\theta$, the agent uses $p(x_{n+1}|\hat\theta)$ to make predictive inferences about future data $x_{n+1}$. When necessary (\eg, the agent suspects to have found a better statistical law), the agent studies properties of their parameter estimator(s) by repeated experimentation, for example to establish confidence intervals %
 and other tools for model selection (see for example \citealp{lehmann1993fisher}).
The \emph{Bayesian} agent, a subjectivist, %
also picks a statistical law, but makes no assumption about its correctness. Given some data $\mathbf x$, they too construct a likelihood function $\ell_{\mathbf x}(\theta)$, but use it differently. As a formal tool, probability comes with a mechanism for belief revision: conditioning. %
To make use of it, the agent augments their possible worlds to include possible values of $\theta$ and its interaction with possible values of the observable variable, they then state their preferences over parameters in the form of a pdf $p(\theta)$. %
This is called a \emph{prior} (conveys one's knowledge and experience before observing $\mathbf x$).
The agent then revises their preferences using Bayes rule %
to obtain a posterior pdf $p(\theta|\mathbf x) \propto p(\theta)\ell_{\mathbf x}(\theta)$. This object supports all inferences the agent will ever make (\eg, 
about parameters, or about future data $x_{n+1}$---for which the agent builds a posterior predictive function  $p(x_{n+1}|\mathbf x) = \int p(x_{n+1}|\theta)p(\theta|\mathbf x) \dd{\theta}$).
In essence, Frequentist procedures revolve around point estimation (\eg, MLE) and null hypothesis significance testing \citep{LehmCase98,lehmann2005testing}, %
Bayesian theory \citep{bernardo1994bayesian} and practice \citep{gelmanbda04}, instead, frame statistical inference as an iterative process of belief revision  (\eg, conditioning, marginalisation, expectation). 


\paragraph{Natural Language Generation.} 
Most NLG models (like \textbf{A3}\hspace{0.5mm}\twemoji{speech balloon}) acquire knowledge through MLE. Alternatives include Bayesian inference \cite[\eg,][]{malinin2020uncertainty,sankararaman2022bayesformer} and utility- and reward-based training (\eg, minimum risk \citep{shen-etal-2016-minimum},  reinforcement learning \citep{paulus2018a}). Recently, pre-training on enormous unlabelled corpora, and reinforcement learning from human feedback \cite[RLHF, \eg,][]{christiano2017deep,stiennon2020learning,ouyang2022training} or \textit{instruction tuning} \cite[\eg,][]{mishra-etal-2022-cross,wei2022finetuned} have become popular to refine the representation of uncertainty towards something that decodes more easily into strings preferred by human users. %

Generating a response is simulating an outcome. %
The event space is the powerset of all token sequences from a fixed vocabulary \cite[BPE tokens, \eg,][]{sennrich-etal-2016-neural}. Rather than prescribing a probability measure (mapping each event to a probability value) directly, we parameterise a pmf (typically parameterised via an autoregressive factorisation of the probability of any one sequence) with a neural network and exploit countable additivity to assign probability to any event (\eg, all token sequences that map to the same sentence \citep{cao-rimell-2021-evaluate,chirkova-etal-2023-marginalize} or all sentences that map to the same meaning representation \cite{kuhn2023semantic}). %


\paragraph{Key Takeaways.}
(1) Uncertainty is a state to be represented. %
(2) To represent uncertainty about something observable or not (\eg, responses, parameters, modelling assumptions) we need to acknowledge and order a whole space of alternatives: our choice of possible worlds must capture interaction amongst possible values of the variables we aim to express our uncertainty about. 
(3) Probability is not constrained to abide by any one interpretation. To regard probabilities in a specific human-interpretable way (\eg, relative frequencies), we need learning techniques yielding that result, and we need to verify that our setting actually meets all necessary conditions (\eg, the Frequentist interpretation of probability is sensitive to modelling choices, local optimality, and  data sparsity: most practical NLG agents are unable to meet the necessary formal requirements).
\vspace*{-3mm}
\section{LLM Traffic Pattern Analysis}
\label{sec:traffic_ana}

\subsection{Traffic Pattern of MegatronLM}
We now analyze the traffic pattern generated by LLMs with PTD-P parallelism by computing the network transfers from the model hyperparameters and the parallelization strategy. We first look at the 145.6 billion, the 310.1 billion, the 539.6 billion, and the 1 trillion parameter model described in Table~1 of MegatronLM~\cite{narayanan2021efficient}, distributed in a cluster composed of DGX A100 servers with an HB domain of size eight. Our analysis uses the same parallelization strategy from MegatronLM to ensure optimal GPU utilization. We use the ring-based collective communication since it is bandwidth-optimal and the default algorithm in NCCL, the communication library backend of MegatronLM. \looseness=-1

Figure~\ref{fig:traffic_dist}a illustrates the volume percentage for each type of traffic for one training iteration, and Figure~\ref{fig:traffic_dist}b shows the traffic type distribution across GPU pairs. There are three primary types of communication: AllGather and ReduceScatter traffic from TP, \allreduce traffic from DP, and point-to-point traffic from PP. The TP traffic happens within GPUs participating in a TP rank, which occupies an HB domain. The DP and PP traffic happen in the network domain, and their volume is significantly lesser than TP traffic, as illustrated by Figure~\ref{fig:traffic_dist}a. While these types of traffic do not overlap between different pairs of GPUs, Figure~\ref{fig:traffic_dist}b indicates that over 99\% of GPU pairs carry \textit{no traffic} and less than 0.04\% of GPU pairs carry TP traffic. Simultaneously, Figure~\ref{fig:traffic_dist}a suggests these traffic types account for over 75\% of the total transmitted data. Recall that TP traffic stays within HB domains, suggesting efficient usage of HB domain bandwidth and low demand in the network domain. This pattern is consistent across all LLM models, indicating that building a cluster with any-to-any connectivity on top of HB domains for LLM models is excessive. 

% Figure environment removed

% Figure environment removed


\subsection{Traffic in the Network Domain}

The parallelization strategy employed in MegatronLM induced an insignificant amount of network traffic across the HB domains compared to within them. Figure~\ref{fig:heatmap} shows the traffic heatmap for training the GPT-1T model. In this plot, every consecutive set of eight GPUs resides within the same HB domain (highlighted in orange), and GPUs with a distance of 8 between them belong to the same rail (highlighted in red). Figures~\ref{fig:heatmap}a demonstrates the traffic pattern within one pipeline stage, while Figures~\ref{fig:heatmap}b shows the traffic across the first four pipeline stages. The traffic volume is significant ($O$(100~GB) across GPU pairs) in an HB domain, while the communication drops to only about 6~GB across them. Furthermore, the communications across HB domains never traverse through the spine switches -- these network transfers only happen within a rail. 

We argue that \textit{all} LLM distributed with an optimal PTD-P parallelization strategy always induces sparse, low-volume traffic across HB domains \textit{within the rails}. By design, the only traffic exiting the HB domains is point-to-point traffic from pipeline parallelism or collective communication traffic (AllGather, ReduceScatter, and AllReduce) from TP and DP when DP and TP dimensions traverse beyond one HB domain. Due to the symmetry of LLM parallelization, each pipeline stage contains the same number of GPUs. As a result, the pipeline stages can always be placed such that traffic across stages always traverses on GPUs of the same rank across HB domains, hence staying within the same rail. 

On the other hand, for some parallelization strategies, TP and DP can induce collective communication traffic across HB domains. For example, training a model in pure DP causes all GPUs to participate in the same DP rank and, thus, the same collective communication operation. The cluster uses hierarchical collective communication algorithms that achieve near-optimal performance in these cases.

Hierarchical collective communication algorithms are designed for a multi-tiered network topology. We introduce the method for the AllGather collective and note that ReduceScatter achieves the same performance by inverting the schedule of AllGather, and AllReduce is equivalent to a ReduceScatter followed by an AllGather. We focus on the bandwidth analysis and ignore the latency in this analysis, as the data transmission is significant during LLM training; thus, the communication runtime is bandwidth-dominated. Logically, we arrange the GPUs conducting an AllGather operation into an $x\times y$ grid, where each $x$ GPU belongs to the same HB domain and across $y$ total HB domains. The basic hierarchical AllGather finishes the operation in two phases: first, the algorithm collects partial data for each rail of GPUs without transferring data in the HB domain. If the total data to run AllGather is of size $D$, then the amount of data exchanged in the network by all GPUs is ${D(y-1)}/{x}$. This operation effectively creates larger data shards for the HB domains to rerun AllGather within each HB domain. Therefore, each HB domain conducts an AllGather in the second phase, inducing a total transfer of $D(x-1)$. Assume the $x$ GPUs within an HB domain have bandwidth capacity $C_F$ and $y$ GPUs in the network domain have bandwidth $C_S$, then the total runtime is 
\begin{equation}\label{eq:agtime}
    \mathtt{AGtime}(D, x, y, C_F, C_S)=\frac{(y-1)D}{xyC_S}+\frac{(x-1)D}{xC_F}
\end{equation} 
Like PP communication, by appropriately mapping the logical $x\times y$ GPUs to the cluster, this algorithm only induces traffic for GPUs within the same rail. Furthermore, based on a recent result on bandwidth-optimal AllGather algorithms, we argue that as HB domain size increases, having full-bisection bandwidth in the network domain does not improve performance compared to only using connections within a rail. We defer the details of this argument to the Appendix~\ref{appendix:allgather}. \looseness=-1

As an example with hierarchical collective communication, we now compute and analyze the traffic pattern of training the GPT-1T model, with a batch size 4096, distributed in a cluster composed of 16 Nvidia GH200 supercomputers~\cite{gh200} (4096 GPUs). Each GH200 supercomputer comprises a two-tier NVSwitch architecture, facilitating  1.8~Pbps of full-bisection bandwidth (7.2~Tbps per GPU) across 256 H100 GPUs. Additionally, each GPU has a Connect-X7 HCA Infiniband network interface~\cite{gh200}, which provides 400~Gbps network bandwidth in/out of each GPU. In this setup, each GH200 supercomputer forms an HB domain. Figure~\ref{fig:traffic_dist_gh200} illustrates the traffic volume percentage and heatmap in this setting. The parallelization strategy has a total data parallel degree of 64, which spans 32 GPUs in each HB domain and two HB domains across the network. Figure~\ref{fig:traffic_dist}b and~\ref{fig:traffic_dist}c illustrate the traffic heatmap of the hierarchical AllReduce algorithm, which splits the AllReduce traffic among each DP group. Note that the network traffic stays within a rail (GPUs with a distance of 256 apart). The hierarchical algorithm efficiently utilized the bandwidth in the HB domain to carry $98\%$ of the AllReduce traffic, as suggested by Figure~\ref{fig:traffic_dist}a.

\section{Rail-only Network Design}\label{sec:system-arch}
Based on the observations above, this section proposes \textit{Rail-only}, a novel network architecture that diverts from the any-to-any GPU connectivity paradigm. We first introduce the architecture design of a Rail-only network. We then discuss routing policy and fault-tolerance properties of Rail-only interconnects.


\subsection{Architecture Design}
Figure~\ref{fig:Rail-only} illustrates our \textit{Rail-only} network architecture. Compared to a conventional Rail-optimized GPU cluster, shown in Figure~\ref{fig:Rail-optimized}, our Rail-only network keeps the HB domains but omits the full-bisection connectivity \textit{for all GPUs} in the NIC domain. Instead, we only ensure that \textit{GPUs within each rail} are connected with a full-bisection network.\looseness=-1 

% Figure environment removed 
A straightforward illustration of our proposed architecture is to remove the spine switches (Figure~\ref{fig:Rail-optimized}) and re-purpose the uplinks connecting rail switches to the spine as downlinks to GPUs. Hence, a dedicated \fattree network connects each rail. Compared to the Rail-optimized architecture, the Rail-only design saves the number of switches and transceivers by building multiple smaller \fattree networks for individual rails, requiring fewer layers of switches in the network.

\subsection{Routing in Rail-only Networks} \label{sec:routing}
Our Rail-only network architecture removes network connectivity across GPUs with different ranks in different rails. Such communication is still possible by forwarding the data through HB domains. For instance, a message from \texttt{GPU 1, Domain 1} to \texttt{GPU 2, Domain 2} in Figure~\ref{fig:Rail-only} can first route through the first HB domain to \texttt{GPU 2, Domain 1} and then be transmitted to the final destination through the network. Previous work has shown that such a routing scheme induces a \textit{bandwdith-tax}~\cite{rotornet, opera, topoopt}, where the physical traffic increases in the network due to forwarding. However, in this section, we show that due to the bandwidth asymmetry between the HB and the NIC domain, the performance degradation caused by bandwidth tax is negligible.  \looseness=-1

We use LLMs with MoE layers described in Section~\ref{sec:moe_traffic} as an example, as it generates a challenging all-to-all communication pattern. At first glance, this traffic pattern is challenging for the Rail-only network. Most of the all-to-all traffic requires forwarding. However, since the HB domain is much faster than the NIC domain, forwarding network traffic on the HB domain incurs a slight overhead. Below, we derive the slow-down factor for uniform all-to-all traffic using a two-step forwarding routing algorithm for the Rail-only network. Note that this strategy has already been implemented in NCCL as ``PCIe $\times$ NVLink" (PXN) to avoid congestion in cases where the spine layer of the Rail-optimized network is oversubscribed~\cite{nccl212}.\looseness=-1

We use the same variables defined in Table~\ref{tab:var_anal} in the following derivation. Consider a grid of $x\times y$ GPUs where $x$ GPUs are placed in an HB domain, and a Rail-only or Rail-optimized network connects $y$ HB domains. Recall that the HB domain has bandwidth $C_F$ while the NIC domain has bandwidth $C_S$ per GPU pair. For a Rail-optimized network, every GPU sends traffic directly to its destinations through the HB and full-bisection NIC domains. Assuming each pair of GPU communicates traffic of size $D$, the total all-to-all completion time is:
\begin{equation}
T_{a2a}^{Rail-opt}=\max(\frac{(x-1)D}{C_F}, \frac{x(y-1)D}{C_S})= \frac{x(y-1)D}{C_S}
\end{equation}
For the Rail-only network, the two-step algorithm first runs an all-to-all \textit{within each rail}, preparing each GPU to have all data to send on its rail. Then, within each rail, each GPU runs a second all-to-all \textit{in the HB domain} to finish the transfers with an effective shard size of $xD$. Note that the second step here contains the \textit{bandwidth-tax}. The total transmission time is
\begin{equation}
T_{a2a}^{Rail-only}=\frac{y(x-1)D}{C_F} + \frac{x(y-1)D}{C_S}
\end{equation}
The two terms differ by ${y(x-1)D}/{C_F}$, which is the cost of forwarding within HB domains. When $ y(x-1)\approx x(y-1)$, the slow-down factor is approximately $C_S/C_F$, which equals to $8.2\%$ and $11.2\%$ for the DGX A100 and DGX H100 generations of GPU platforms, respectively. This factor is low because of the bandwidth asymmetry between the two domains. Furthermore, this slow-down only applies to the all-to-all communication, which accounts for $27\%$ of the total traffic as shown in Figure~\ref{fig:moe_traffic}. Therefore, this small overhead is negligible by Amdah's law. We note that such properties also make our network design suitable for other classes of DNN models.   \looseness=-1


\subsection{Fault Tolerance Properties of Rail-only Networks} \label{sec:fail_tol}
Fault tolerance is crucial for large GPU clusters with long-lasting LLM training jobs. This section investigates the fault tolerance properties of  Rail-only networks compared to traditional Rail-optimized networks.

\para{Link and switch failures.} Suppose a rail switch or a link fails. GPUs connected to the failed switch or link become unavailable for both Rail-optimized and Rail-only network architectures, rendering the two designs identical regarding fault tolerance on switches and links. However, our design requires fewer switches, which naturally reduces the points of failure. Datacenter operators can add redundant capacity by including extra rail switches, and our design remains more cost-effective than the any-to-any network design. \looseness=-1 

\para{GPU platform failure.} For a GPU cluster composed of DGX-like servers, each server is an HB domain. When a server fails, the network operator migrates the task to another healthy server. The Rail-only connectivity remains the same for the new server. For a GB200-like cluster, a super-chip module is analogous to a server; thus, the failure mode is the same as a single GPU failure, which we will discuss next.  \looseness=-1

\para{Single GPU failures with idle GPU in the HB domain.} We discuss two distinct scenarios separately for single GPU failures. The first case is when another idle GPU presents the same HB domain as the failed one. In this case, a Rail-optimized network directly replaces the failed GPU with the idle one without breaking the HB domain integrity. One possible solution is to leverage optical reconfigurable switches for the Rail-only network. To improve robustness, we add a small number of optical switches between the GPU and rail switches to allow the dynamic reconfiguration of rails. When a GPU fails, the optical switch reconfigures to bring a healthy, idle GPU to replace the failed one. This approach is conceptually similar to the failure recovery mechanism of network designs that primarily uses optical-reconfigurable switches~\cite{jouppi2023tpu, topoopt, jupiterevolving}. We leave an in-depth analysis of rail-only with optical switch to future work.

\para{Single GPU failure in fully occupied HB domains.} Another failure mode occurs when a GPU fails in a fully occupied HB domain and requires a substitute GPU from different HB domains. In this case, the Rail-only design prevents migrating the task on the failed GPU to another idle one in the cluster, which is possible in a Rail-optimized network. However, such a solution is undesirable even with the Rail-optimized network. The new GPU no longer belongs to the same HB domain as the failed one, creating a bottleneck that slows the HB domain into a NIC domain. Instead, we propose two solutions. For platforms with small HB domains, we migrate the tasks on the entire HB domain with the failed GPU to a new one. For larger HB domains (e.g., DGX GH200 supercomputers with $K=256$), these HB domains comprise a multi-tiered topology with an optical core-layer~\cite{gh200}. One potential approach is to add optical switches, like in the previous failure case. When a GPU failure occurs, the optical switch reconfigures, replacing a small set of GPUs (including the failed one) with healthy ones, thus maintaining the integrity of an HB domain.  \looseness=-1


\section{Iteration Time Modeling}
\label{sec:sys-model}

An accurate and detailed model provides fundamental guidance for choosing the right parallelization strategy and network design. Previous research has mathematically modeled the computational time and memory requirements for various LLM training parallelization strategies~\cite{shoeybi2020megatronlm, narayanan2021efficient, korthikanti2022reducing}. Nevertheless, these works omit a detailed derivation for communication time during a training iteration considering the network infrastructure. This section presents an analytical model of the training iteration time, incorporating both the parallelization strategy and the training infrastructure.  This formulation is the foundation for evaluating the rail-only network design in Section~\ref{sec:evaluation}. Table~\ref{tab:parameters} outlines the parameters used in this analysis. The section assumes mixed-precision training with 16-bit model parameters, activations, and gradients.\looseness=-1

\subsection{Critical Path Analysis}

Figure~\ref{fig:critical_path} displays the 1F1B pipeline schedule without interleaving. Given the uniform structure of LLMs under the PTD-P parallelism, both forward and backward execution times for each micro-batch across GPUs remain the same. This consistency allows us to identify a critical path in a training iteration, highlighted by a red arrow in Figure~\ref{fig:critical_path}. This path further decomposes into three parts: the pipeline bubble time, the computation and communication time in the last stage (LS) of the pipeline, and finally, the parameter synchronization time after the pipeline flush. Note that the bubbling and the last pipeline stage are strictly disjointed. 
%The computation on the last pipeline stage can only begin after the first micro-batch arrives, and the backpropagation of the last micro-batch only starts after the last stage's computation finishes. 
However, parameter synchronization traffic can start immediately for each transformer layer after the last micro-batch finishes processing, overlapping itself with the backward propagation. Another potential overlapping happens within the last stage, between the PP and TP traffic across different micro-batches. For simplicity, we provide conservative modeling of the iteration time, in which we start parameter synchronization after all computation finishes and disallow cross micro-batch TP and PP traffic overlapping.  

With these observations, the iteration time is 
\begin{equation} \label{eq:itertime}
T_{iter}=T_{bubble}+T_{LS}+T_{sync}
\end{equation}
This model also holds for the interleaved schedule. Interleaved schedules reduce the pipeline bubble time while requiring additional communication within the last stage of the pipeline. We factor such cost into $T_{bubble}$ and $T_{LS}$ in Eq.~\ref{eq:itertime}. The rest of this section dissects each term with explicit computation and communication cost. \looseness=-1

\begin{table}[t]
\scriptsize
\centering
\vspace*{-1.5mm}
\caption{Parameters utilized in the calculation in this section.}
\newcommand{\centered}[1]{\begin{tabular}{l} #1 \end{tabular}}
\renewcommand{\arraystretch}{0.92} 
\linespread{1.05}\selectfont\centering
    \begin{tabular}{|p{1cm}|p{6.5cm}|}
    \hline
    \textbf{Name} & \textbf{Description} \\ \hline
    $p, t, d$        & Pipeline, Tensor and Data parallelism dimensions, respectively \\ \hline
    % $p$              & Pipeline parallelism dimension \\ \hline 
    % $d$              & Data parallelism dimension \\ \hline 
    $p_h, t_h, d_h$   & The portion $p, t, d$ mapped into an HB domain, respectively \\ \hline
    $p_l, t_l, d_l$   & The portion $p, t, d$ mapped into the network domain, respectively \\ \hline
    $h$              & LLM Embedding dimension (hidden size) \\ \hline 
    $s$              & Sequence length \\ \hline
    $v$              & Number of interleaved stages \\ \hline  
    $K$              & HB domain size\\ \hline
    $l$              & Number of transformer block layers \\ \hline
    $C_{F}$          & HB domain bandwidth \\ \hline
    $C_{S}$          & GPU Network bandwidth \\ \hline
    $S_T$            & Number of parameters in a transformer block \\ \hline 
    $b$              & Micro-batch size per pipeline  \\ \hline 
    $m$              & Number of micro-batches per iteration \\ \hline
    \end{tabular}
    \label{tab:parameters}
    \vspace{-5mm}
\end{table}

\subsection{Analytical Iteration Time Modeling}
This section provides a quantitative analysis, considering each term's computation and communication cost in Eq.~\ref{eq:itertime}.

\para{Pipeline bubble time}. For the pipeline bubble time, we break down the cost into the communication and the computation as
% \begin{equation}\label{eq:bubbletime}
$T_{bubble} = T_{bubble}^{comp} + T_{bubble}^{comm}$.  
% \end{equation}
Assume a micro-batch size's total compute time (forward and backward pass) is $t(b)$. With interleaved scheduling of $v$, the computation time spent in the pipeline is 
\begin{equation}\label{eq:bubblecomptime}
    T_{bubble}^{comp} = \frac{(p-1)t(b)}{v}
\end{equation}
Narayanan et al. observed that the computational efficiency of GPU varies with $b$~\cite{narayanan2021efficient}; therefore, it is best to profile $t(b)$ for each micro-batch size in practice. For simplicity, we provide an alternative measurement analytically by modeling the computational requirement (FLOPs) of an LLM and GPU's computation speed in Appendix~\ref{appendix:mb_est}.

For the communication, each micro-batch induces $D_{\mu b}^{p}={2bhs}/{t}$ bytes of traffic across two pipeline stages when sequence parallelism is enabled together with TP. Such transfer will happen for a total of $2(p-1)$ times throughout the pipeline filling and emptying, where $2(p_s-1)$ times will happen in the network domain and $2p_s(p_f-1)$ times in HB domains. Hence, the pipeline bubble communication time is \looseness=-1
\begin{equation}\label{eq:bubblecommtime}
T_{bubble}^{comm}=\frac{2(p_s-1) D_{\mu b}^{p}}{C_S} + \frac{2p_s(p_f-1) D_{\mu b}^{p}}{C_F}
\end{equation}
Unlike the computation time, the communication time for bubbling is unaffected by the interleaved scheduling. 

\para{Last stage time.} Similar to the bubble time, we analyze the computation and computation costs separately in the last stage of the pipeline. The last stage has $m$ micro-batches going through it, and therefore the computational cost is  
\begin{equation}\label{eq:lscomptime}
T_{LS}^{comp}=mt(b)
\end{equation}
For communication, TP and PP both generate network traffic across GPUs. We first focus on the TP traffic: for each micro-batch, four AllGather and four ReduceScatter happen in total for each layer of the transformer layer when sequence parallelism is applied. Since ReduceScatter is the inverse of AllGather and generates the same amount of traffic, this analysis will focus on AllGather only. Recall $\mathtt{AGtime}(D, x, y, C_F, C_S)$ (Eq.~\ref{eq:agtime}) is the time of running AllGather on data of size $D$ bytes on an $x\times y$ grid of GPUs with HB domain and network bandwidth $C_F$, $C_S$ respectively. The amount of data to run AllGather for each micro-batch is $D_{\mu b}^{t}=2bhs$ per transformer block, and since each pipeline stage holds ${l}/{p}$ transformer blocks, the total runtime for all $m$ micro-batches is ${8lm\mathtt{AGtime}(D_{\mu b}^{t}, t_h, t_l, C_F, C_S)}/{p}$. 

Next, we look at the pipeline traffic in the last stage. The pipeline traffic can overlap with the computation, even with interleaved scheduling; however, with conservative modeling, we assume that GPUs do not perform computation when sending and receiving pipeline parallel traffic. Each interleaved part of each micro-batch at least sends or receives ${D_{\mu b}^{p}}$ bytes of traffic both on the forward and backward pass and every micro-batch needs to traverse the network domain whenever $p_s>1$. Therefore, we model the pipeline communication overhead in the last stage as ${2mvD_{\mu b}^{p}}/{C_*}$ where $C_*=C_S$ if $p_s>1$, else $C_*=C_F$. Adding the tensor and pipeline communication time together,
\begin{equation}
\small
T_{LS}^{comm} = \frac{8lm\mathtt{AGtime}(D_{\mu b}^{t}, t_h, t_l, C_F, C_S)}{p} + \frac{2mvD_{\mu b}^{p}}{C_*} \label{eq:lscommtime} 
\end{equation}

% Figure environment removed 

\para{Parameter Synchronization.} Finally, we have the parameter synchronization time, consisting of an AllReduce operation of the model parameters in the first stage of the pipeline. We only model the communication time of the AllReduce operation since the computation of AllReduce incurs minimal overhead. We follow the same hierarchical collective algorithm described in Section~\ref{sec:system-arch}. For a $d=d_h\times d_l$ way DP, the amount of data to AllReduce is $D^{d}={2lS_T}/{pt}$. An AllReduce induces twice the runtime as an AllGather for the same amount of data; therefore,
\begin{equation} 
    T_{sync}=2\mathtt{AGtime}(D^d, d_h, d_l, C_F, C_S) \label{eq:dptime}
\end{equation} 
Together, Eq.~\ref{eq:bubblecomptime},~\ref{eq:bubblecommtime},~\ref{eq:lscomptime},~\ref{eq:lscommtime} and~\ref{eq:dptime} provide a closed-form expression for Eq.~\ref{eq:itertime} as the training iteration time for an LLM. While this section only presents the transmission delay, in our evaluation, we also consider the propagation delay (network latency) for the communication times to get the best iteration time accuracy. \looseness=-1

\vspace*{-1mm}
\subsection{Constraints in Choosing Parameters}
In addition to the cost model, we derive the set of constraints of the hyperparameters that describe a complete parallelization strategy.
We derive a program that exhaustively generates all the possible parallelization configurations in Appendix~\ref{appendix:constraints}. The iteration time model from Eq.~\ref{eq:itertime} then serves as the cost of an optimization problem to derive the optimal parallelization strategy. In the next section, we evaluate the accuracy of this modeling and use it as the basis for analyzing our rail-only network design for training LLMs. \looseness=-1

\section{Evaluation} \label{sec:evaluation}

\subsection{Iteration Time Modeling} \label{sec:accuracy_model}
We evaluate our Rail-only network design's performance through an analytical model of the training iteration time. Our analytical model considers the critical path for LLM training with TP, DP, and PP, similar to the approach in Calculon~\cite{calculon}. 

We demonstrate the accuracy of our analytical model by comparing its estimation of hardware FLOPs utilization (HFU) to that of published results in the literature. The HFU refers to the hardware's floating point operation performed in an iteration over the peak floating point operations. Prior work provided the full set of hyperparameters in their evaluation setup, enabling us to compare the estimated HFUs from our analytical model to the ground truth~\cite{korthikanti2022reducing}. In our evaluations, we assume a cluster of 1 to 280 DGX A100 servers. 
To compute the total required FLOPs for training per iteration of a DNN model, we use the same formula proposed by Korthikanti et al.~\cite{korthikanti2022reducing}.

% Figure environment removed

Figure~\ref{fig:hfu} compares the HFUs approximated by our analytical model with the ground truth for different GPT models and cluster scales. For GPT-1T, our computed HFU only differs from the ground truth by $0.15\%$. The discrepancy between our analytical model and the ground truth comes from our idealistic modeling of GPU computation and communication, assumptions on how computation and communication overlap, and underestimation of memory overhead. Such discrepancy goes up as the model size decreases. For the GPT-22B model, our estimation error is $15.7\%$ compared to ground truth MFU. 
The rest of this section utilizes our analytical model to estimate the training iteration time of Rail-only interconnects.

\subsection{What is the Optimal Size of an HB domain?} \label{subsec:hbd-size}

Intuitively, increasing HB domain size reduces the inter-platform network overhead during training. This section answers the following question: \textit{what is the optimal size of an HB domain in a Rail-only interconnect?} In Figure~\ref{fig:iter_time_dom_sz}, we vary the HB domain size ($K$) and plot the training iteration time for GPT-1T and GPT-146B MegatronLMs for clusters with 16384, 32768, and 65536 H100 GPUs. The global batch size in this evaluation is 4096 and 1024 for GPT-1T and GPT-146B, respectively. We enumerate all possible parallelization strategies for each cluster size and use the optimal parallelization strategy found in our analytical model, using the bandwidth and computational ability parameters of DGX GH200. In addition, to capture the ideal iteration time, we also compute training iteration time where a full-bisection monolithic NVLink fabric connects every GPU (i.e., the case where $K=N$, where $N$ is the total GPU count). \looseness=-1

As depicted in Figure~\ref{fig:iter_time_dom_sz},  
iteration times decrease as the HB domain size increases, indicating that larger HB domains reduce the network overhead during training. In both GPT models, the iteration time achieved with an HB domain size of 256 is nearly optimal. Compared to the ideal case (all GPUs are under a monolithic HB domain), GPT-146B with an HB domain of 256 is $8.9\%$ slower, while GPT-1T is $1.3\%$ slower. However, the \textit{marginal gain} decreases as the HB domain size increases.
For the larger GPT-1T model, the train iteration time plateaus above 32 GPUs due to Amdhahl's law, suggesting diminishing returns from the increased cost of bigger HB domains. 

% Figure environment removed

For a smaller GPT-146B model, shown in Figure~\ref{fig:iter_time_bw}b, the marginal gain of increasing HB domain size is higher than that of GPT-1T. Providing an HB domain of size eight reduces the iteration time by $50.6\%$ compared to the HB domain of size one, while increasing the HB domain size from 8 to 256 further achieves a $39.1\%$ reduction in iteration time. The more significant marginal gain for smaller LLMs incurs more communication overhead when distributed to the same cluster than larger models. This effect arises from how computation and communication costs scale as LLM grows. The communication requirement increases linearly with the model's hidden size and sequence length. On the other hand, the model FLOPs increase quadratically with these two parameters, as indicated by previous work~\cite{narayanan2021efficient}. Therefore, we conclude that the optimal HB domain size depends on the size of the GPT model.

\subsection{Impact of HB domain and Network Bandwidth}
% Figure environment removed 

The available bandwidth of HB and NIC domains determines the communication time during training. We analyze the impact of these bandwidths on the iteration time of a GPT model with one trillion parameters. Figure~\ref{fig:iter_time_bw}a and~\ref{fig:iter_time_bw}b show the iteration time variation for different HB domain bandwidths (different lines) and network bandwidths in the rails (on the $x$-axis) for HB domain size $K=8$ and 256, respectively. As expected, the iteration time decreases when either bandwidth increases. However, the $K=8$ case is less sensitive to the HB domain bandwidth. Increasing per-GPU bandwidth by a factor of four (from 2.4~Tbps to 9.6~Tbps) only improves the iteration time by $8.0\%$ on average for $K=8$, compared to the improvement of $13.3\%$ for $K=256$. On the other hand, larger HB domain sizes are more pronounced on network bandwidth improvement. Increasing the bandwidth from 100~Gbps to 400~Gbps, results in $35.9\%$ improvement for $K=8$ but only $8.0\%$ for $K=256$. Hence, improving the HB domain bandwidth is more beneficial than the network bandwidth for LLMs as future HB domain size increases. \looseness=-1

% Figure environment removed


\subsection{Impact of Batch Size on Network Design} \label{subsec:batch-size}

While the batch size is typically an ML-centric metric for optimization, our analysis indicates that the impact of batch size on the total training time goes beyond the total number of iterations required for convergence. To further understand such impact, we analyze the iteration time of a GPT-1T model on a 32768 GPU cluster while changing the HB domain size from $K=8$ to $32768$. In this study, we vary the global batch size from 256 to 4096. Figure~\ref{fig:iter_time_palm}a plots the change in iteration time as the batch size varies. The iteration time exhibits a similar trajectory for all HB domain sizes; however, the \textit{relative performance} (the ratio of the iteration time for an HB domain size to that of the ideal case) improves as the batch size increases. Figure~\ref{fig:iter_time_palm}b represents this trend. When $K=256$, the relative performance increases from $93\%$ to $99\%$ as the batch size increases from 256 to 4096 sequences. The iteration time advantage is prominent when the HB domain size is small. For $K=8$, increasing the batch size from 256 to 4096 improves the relative performance from $58\%$ to $82\%$, suggesting a larger batch size is preferable for a cluster with a smaller HB domain. Prior studies have shown that LLM training benefits from a larger batch size~\cite{kaplan2020scaling, gpt3}, especially for bigger models, making it a perfect fit for our Rail-only design. \looseness=-1

\subsection{Network Cost and Power Analysis} \label{sec:cost}

Our Rail-only network architecture judiciously reduces the network resources for LLM training by eliminating unused network hardware. This section compares our proposed approach's network cost and power with the state-of-the-art Rail-optimized GPU clusters. We calculate the number of switches ($N_{SW}$) and transceivers ($N_{TR}$) required for each network design and derive the network equipment cost and peak power consumption based on numbers reported in prior work and by vendors~\cite{topoopt, nvidia400gpower, qm7900spec}.\footnote{Cost: $Price_{TR}=\$199$ per transceiver, $Price_{SW}=\$694$ per switch port for 400~Gbps; Power: $Power_{TR}=9$W per transceiver, $Power_{SW}=18$W per switch port}
We enumerate the number of switches and transceivers required to build the state-of-the-art network architecture and our proposed architecture in Table~\ref{tab:nw_swich}, accounting for variable cluster sizes and network switch radix. 
For each architecture, we build a minimum layer \fattree network with the provided switch radix and calculate the minimum number of required switches and transceivers to achieve the desired connectivity. The total cost of each architecture is 
\begin{equation}
    \text{Total\ cost} = Price_{SW}\times N_{SW} + Price_{TR}\times N_{TR}
\end{equation}
while the power is
\begin{equation}
    \text{Total\ power} = Pwr_{SW}\times N_{SW} + Pwr_{TR}\times N_{TR}
\end{equation}

The last two columns of Table~\ref{tab:nw_swich} provide the cost and power savings of a Rail-only interconnect over the state-of-the-art. Our Rail-only design notably reduces the network cost by 38\% to 77\% (117 to 234 million dollars) and power consumption by 37\% to 75\% (1.7 to 6.9 megawatts) compared to the state-of-the-art design while achieving equivalent performance.
This reduction stems from eliminating spine layer switches and decreasing the number of switch tiers within each rail. Surprisingly, switches with a radix of 64 provide the worst-case cost and power reduction in both cluster sizes. In this case, the state-of-the-art design requires a three-tier \fattree network, while the Rail-only design requires two tiers for each rail. Still, our design only requires three-quarters of the total number of switches while achieving the same performance as the state-of-the-art design. 

\begin{table}[t]
\scriptsize
\centering
\caption{Number of switches and transceivers for different clusters.}

\renewcommand{\arraystretch}{1} 
\linespread{1.05}\selectfont\centering
    \begin{tabular}{|P{0.55cm}|P{0.55cm}|P{0.55cm}|P{0.65cm}|P{0.65cm}|P{0.65cm}|P{0.55cm}|P{0.55cm}|}
    \hline
    \multirow{2}{*}{\#GPUs} & \multirow{2}{*}{{\shortstack{Switch\\ Radix}}} & \multicolumn{2}{c|}{\#Switches ($N_{SW}$)} & \multicolumn{2}{c|}{\#Transceivers ($N_{TR}$)} & \multicolumn{2}{c|}{Savings} \\\cline{3-8}
       & & \multicolumn{1}{c|}{SOTA} & \multicolumn{1}{c|}{Rail-only} & \multicolumn{1}{c|}{SOTA} & \multicolumn{1}{c|}{Rail-only} & \multicolumn{1}{c|}{Cost} & \multicolumn{1}{c|}{Pwr} \\
    \hline
        \multirow{3}{*}{32768}  
                                 & \multicolumn{1}{c|}{64} & \multicolumn{1}{c|}{2560} & \multicolumn{1}{c|}{1536} & \multicolumn{1}{c|}{196608} & \multicolumn{1}{c|}{131072} & \multicolumn{1}{c|}{38\%}  & \multicolumn{1} {c|}{37\%} \\\cline{2-8}
                                 & \multicolumn{1}{c|}{128} & \multicolumn{1}{c|}{1280} & \multicolumn{1}{c|}{256} & \multicolumn{1}{c|}{196608} & \multicolumn{1}{c|}{65536} & \multicolumn{1}{c|}{77\%} & \multicolumn{1}{c|}{75\%} \\\cline{2-8}
                                 & \multicolumn{1}{c|}{256} & \multicolumn{1}{c|}{384} & \multicolumn{1}{c|}{128} & \multicolumn{1}{c|}{131072} & \multicolumn{1}{c|}{65536} & \multicolumn{1}{c|}{62\%} & \multicolumn{1}{c|}{60\%} \\\hline
        \multirow{3}{*}{65536}   
                                 & \multicolumn{1}{c|}{64} & \multicolumn{1}{c|}{5120} & \multicolumn{1}{c|}{3072} & \multicolumn{1}{c|}{393216} & \multicolumn{1}{c|}{262144} & \multicolumn{1}{c|}{38\%} & \multicolumn{1}{c|}{37\%} \\\cline{2-8}
                                 & \multicolumn{1}{c|}{128} & \multicolumn{1}{c|}{2560} & \multicolumn{1}{c|}{1536} & \multicolumn{1}{c|}{393216} & \multicolumn{1}{c|}{262144} & \multicolumn{1}{c|}{38\%}  & \multicolumn{1}{c|}{37\%} \\\cline{2-8}
                                 & \multicolumn{1}{c|}{256} & \multicolumn{1}{c|}{1280} & \multicolumn{1}{c|}{256} & \multicolumn{1}{c|}{393216} & \multicolumn{1}{c|}{131072} & \multicolumn{1}{c|}{77\%} & \multicolumn{1}{c|}{75\%} \\\hline
    \end{tabular}
    \label{tab:nw_swich}
\end{table}
\section{Related Work} \label{sec:discussions}

\para{LLM trend.}  
The current growth rate of LLM computational and speed requirement outpaces the advancements in AI accelerators and network speed as Moore's law slows down, necessitating hyper-scale clusters 
and more efficient interconnects~\cite{sirius, dnnmodelgrowth}. The MegatornLM line of work pioneers LLM parallelization~\cite{shoeybi2020megatronlm,narayanan2021efficient,korthikanti2022reducing}. Our position to remove any-to-any network connectivity complements MegatronLM.
We also acknowledge ongoing efforts to reduce language models' size and resource requirements without compromising performance~\cite{dolly}. These works complement our design as our design reduces network resources and maintains performance even for smaller language models and clusters. Similarly, research directions that aim to directly reduce the amount of communication through quantization and compression, like DeepSpeed Zero++, also complement our approach~\cite{deepspeedzeropp}.


\para{LLM Inference.} This paper explores the training workload of LLMs, yet inference represents another significant part of the LLM product cycle. Inference demands fewer resources as it involves moving less data through the LLM and only computes the forward pass and multiple passes to generate response tokens~\cite{alpaserve}. Pope et al. developed specific parallelism for inference on TPU-v4 architecture~\cite{pope2022efficiently}. For our design, each HB domain becomes an inference-serving domain with low latency, and the Rail-only connections help load-balance multiple inference domains. We leave a detailed investigation of LLM inference to future work. \looseness=-1 

\para{Multi-job training.} It is common for a GPU cluster to train multiple smaller jobs simultaneously. Existing works focus on \fattree-based GPU clusters and provide techniques for better fairness and shorter job-completion time~\cite{gandiva,tiresias,muri,rajasekaran2023cassini}. While this paper focuses on training a single LLM on a large cluster, the Rail-only network design is also suitable for a multi-job setting. The entire cluster can be arbitrarily partitioned by tiling into smaller rectangular partitions, similar to the case of TPU-v4~\cite{jouppi2023tpu}. Each partition then independently executes a smaller training job.  \looseness=-1 

\para{ML infrastructures and other ML workloads.} Prior works illustrated the benefits of co-designing hardware and software for ML models. For instance, Google's TPU cluster is optimized for training large models with 3D parallelism on TPUs~\cite{jouppi2023tpu}, while Meta's Neo focuses on training recommendation models with large embedding tables~\cite{mudigere2023softwarehardware}. Our work focuses on designing a cost-efficient network to train LLMs efficiently. Although our proposed Rail-only architecture focuses on network design specifically for LLMs, our design is efficient for many other DNN workloads when combined with other efforts, as the forwarding overhead is low~(\S\ref{sec:routing}). Recent works attempt to make the parallelization strategy and collective communication algorithms bandwidth-aware for any DNN model~\cite{alpa, unity, zhao2023bandwidth}, producing traffic patterns ideal for the Rail-only network. \looseness=-1  




\section{Conclusion}
In this paper, we have argued for the importance of a principled and fundamental understanding of representing, learning and reasoning about uncertainty in NLG. We identified and organised the main sources of uncertainty, and highlighted the many important applications this perspective can power.

To do so, we laid down central concepts, their formal mathematical frameworks and the necessary vocabulary. We specifically drew attention to the possible worlds framework; probability as a way to express preference over these possible worlds; its two main interpretations; statistical tools to acquire and revise knowledge; and how these are commonly used in NLG. 

Then, building on the triangle of reference, we identified and organised the main sources of uncertainty in language production: input \textit{ambiguity}, \textit{errors}, and \textit{complexity}; the \textit{open-endedness of the communicative task}, the \textit{agent's personal perspective}, and the final \textit{linguistic realisation}--- and modelling: \textit{model specification}, \textit{parameter estimation}, and \textit{distribution shift}. We proposed a two-dimensional taxonomy to organise sources as a richer alternative to the aleatoric/epistemic distinction.

Finally, we highlighted exciting applications of disentangled representations of uncertainty in NLG. These span from applications related data uncertainty (decoding, controllable generation, explicit modelling of sub-populations), to model uncertainty (self-assessment, selective answering, OOD detection, active learning).

We hope to spark a shared understanding of uncertainty and inspire more principled and focused research in NLG. Crucially, we believe this perspective allows for systems that are more flexible, representative of the diversity of human language and its speakers, and reliable and trustworthy.


\bibliography{ref}
\bibliographystyle{mlsys2024}

\appendix 
\newpage 
\setlength{\abovedisplayskip}{4pt}
\setlength{\belowdisplayskip}{4pt}
\section{Bandwidth Time of AllGather Algorithm for Rail-Optimized and Rail-Only Network} \label{appendix:allgather}
Zhao et al.~\cite{zhao2023bandwidth} proved that a bandwidth-optimal AllGather schedule exists for arbitrary network topology. This section examines the best-case AllGather time for a grid of $x\times y$ GPUs in the rail-optimized and rail-only network. Following Eq.~1 in Zhao et al.~\cite{zhao2023bandwidth}, we derive an analytical expression of the bandwidth AllGather time for these two types of networks. \looseness=-1

Let $\mathbb{M}_{x y}$ be the space of all boolean matrices of size $x\times y$ except all ones and all zeros. A matrix $A\in\mathbb{M}_{x y}$ represents a specific partition of the GPUs in an $x\times y$ configuration. Then, the optimal AllGather time for one unit of data in a rail-optimized network is

\begin{align*}
\scriptsize
\max\limits_{A\in\mathbb{M}_{x y}}\frac{\max\limits_{A'\in\{A, \Bar{A}\}}\sum\limits_{i,j} A'_{ij}}{\min\limits_{A''\in\{A, \Bar{A}\}}((C_F + C_S)\sum\limits_{i,j} A''_{ij}-xR(A'')C_F)} \numberthis\label{eq:ag_ropt}
\end{align*}

Here, $\bar{A}$ denotes the negation of the boolean matrix $A$. The numerator finds the partition that sends the largest data, which equals the sum of the binary entries of $A$ or $\bar{A}$. The denominator finds the partition with the lowest ingress and egress bandwidth. For each GPU included in the partition, the total bandwidth of the partition increases by $C_F+C_S$, hence the first term in the minimization. However, whenever the partition includes an entire row of GPUs (i.e., an HB domain), the bandwidth internal to this HB domain no longer contributes to the ingress or egress bandwidth. The function $R(A)$ counts the number of rows with all ones as $A$'s entries, implying one HB domain entirely inside the partition. The second term in the minimization removes this part of the bandwidth from the egress bandwidth. 

For the rail-only network, going through the same analysis, we get the AllGather time of

{\scriptsize
\begin{align*}
\max\limits_{A\in\mathbb{M}_{x y}}\frac{\max\limits_{A'\in\{A, \Bar{A}\}}\sum\limits_{i,j} A'_{ij}}{\min\limits_{A''\in\{A, \Bar{A}\}}((C_F + C_S)\sum\limits_{i,j} A''_{ij}-xR(A'')C_F-yC(A'')C_S)} \numberthis\label{eq:ag_ronly}
\end{align*}
}

The formula remains the same as Eq.~\ref{eq:ag_ropt}, except for the last term in the denominator. This term accounts for the fact that whenever an entire rail is included in the partition, this rail no longer contributes its bandwidth as the ingress or egress bandwidth of the partition. Here, the $C(A)$ function counts the number of columns with all ones as their entries, hence the number of entire rails inside the partition. 

Intuitively, to maximize either expression, the choice of $A$ should be skewed (i.e., having a large portion of GPU in one partition) so that $\sum\limits_{i,j} A'_{ij}$ on the numerator is large but $\sum\limits_{i,j} A''_{ij}$ on the denominator is small. In addition, the GPU choice should be arranged such that the denominator can exclude as much bandwidth from the partition as possible. For Eq.~\ref{eq:ag_ropt} and Eq.~\ref{eq:ag_ronly}, one such configuration is obtained when the partition has $y-1$ HB domains. In this case. $R(A'')=1$ and $C(A'')=0$ which yield an AllGather time of ${(y-1)}/{C_S}$ per unit of data for both networks. We postulate that for $C_F >> C_S$ and $x \geq y$, this choice of partition yields the lower bound (thus bandwidth optimal) AllGather time for both of this network, as perturbing the partition by adding or removing single GPUs or entire HB domains only relaxes the bound. We leave concrete proof of the optimality in future work. \looseness=-1

\section{Estimation of Microbatch Compute Time}\label{appendix:mb_est}

While LLM designers can use profiling to obtain micro-batch computation times accurately, it is also possible to analytically estimate the execution time, especially when the hardware is unavailable. Table~\ref{tab:extra_para} shows the extra parameters used in this calculation and other sections in the appendix, in addition to Table~\ref{tab:parameters}.

\begin{table}[t]
\scriptsize
\centering
\caption{Extra parameters utilized in the appendix.}
\newcommand{\centered}[1]{\begin{tabular}{l} #1 \end{tabular}}

\renewcommand{\arraystretch}{0.92} 
\linespread{1.05}\selectfont\centering
    \begin{tabular}{|p{1cm}|p{6.5cm}|}
    \hline
    \textbf{Name} & \textbf{Description} \\ \hline
    $N$              & Total number of GPUs in the cluster\\ \hline
    $M_{FF}$         & Amount of feed-forward FLOPs required for an iteration  \\ \hline
    $M_{Attn}$       & Amount of attention block FLOPs required for an iteration \\ \hline
    $F$              & GPU Compute Speed (FLOPs)\\ \hline
    $B$              & Global batch size \\ \hline 
    $R$              & GPU memory size   \\ \hline
    % $\alpha$         & Network latency  \\ \hline
    % \hline
    % $T_c$            & Compute time of a transformer block \\ \hline
    % $T_{d}$          & Exposed communication time for data parallelism \\ \hline
    % $T_{t}$          & Communication time for tensor parallelism \\ \hline    
    % $T_{p}$          & Communication time for tensor parallelism \\ \hline 
    \end{tabular}
    \label{tab:extra_para}
\end{table}

Most FLOP for LLMs comes from the attention mechanism and general matrix multiplication (GEMM) in the feed-forward layers. Prior work reports that while GPUs operate at peak speed for GEMM in feed-forward layers, they are at most $40\%$ efficient for attention blocks without architecture-specific optimization~\cite{dao2023flashattention2}. Therefore, we model the computation time of a micro-batch accordingly, as
\begin{equation}\label{eq:computetimemb}
T_{\mu b}^{comp} = \frac{(M_{FF} + \gamma M_{Attn})b}{FB p t} \approx t(b)
\end{equation}
where $\gamma$ is a slowdown factor for attention. We use $\gamma=1/0.4$ for Section~\ref{sec:evaluation}. $M_{FF}$ and $M_{Attn}$ depend on the model's hyperparameter and whether activation recomputation is utilized. Our evaluation in Section~\ref{sec:evaluation} assumes the training uses selective activation recomputation presented by Korthikanti et al.~\cite{korthikanti2022reducing}, and the FLOPs estimation follows from Eq.~8 in the same paper. 

\section{Constraints of Parallelization Configuration} \label{appendix:constraints}
\begin{table*}[t]
\scriptsize
\centering
\caption{Model hyperparameters and iteration time comparison.}
\renewcommand{\arraystretch}{1} 
\linespread{1.05}\selectfont\centering
    \begin{tabular}{|P{0.6cm}|P{0.8cm}|P{0.6cm}|P{0.6cm}|P{0.6cm}|P{0.6cm}|P{0.6cm}|P{0.6cm}|P{0.6cm}|P{1.2cm}|P{2cm}|P{1.5cm}|}
    \hline
    {Model Size} & {Attention Heads} & {Hidden Size} & {Layers} & {TP Size} & {PP Size} & {Number of GPUs} & {Global Batch Size} & {Micro-batch Size} & {Number of Interleaved Stages} & {Measured Iteration Time in~\cite{korthikanti2022reducing}} & {Computed Iteration Time from Section~\ref{sec:sys-model}}  \\ \hline
    22B & 64 & 6144 & 48 & 8 & 1 & 8 & 4 & 4 & 1 & 1.10 & 0.78  \\ \hline
    175B & 96 & 12288 & 96 & 8 & 8 & 64 & 64 & 1 & 3 & 13.75 & 11.89  \\ \hline
    530B & 128 & 20480 & 105 & 8 & 35 & 280 & 280 & 1 & 3 & 37.83 & 35.29  \\ \hline
    530B & 128 & 20480 & 105 & 8 & 35 & 2240 & 2240 & 1 & 3 & 39.15 & 35.56  \\ \hline
    1T & 160 & 25600 & 128 & 8 & 64 & 512 & 512 & 1 & 1 & 71.49 & 70.69  \\
    \hline
    \end{tabular}
    \label{tab:model_acc_eval}
\end{table*}
Given a cluster with $N$ GPUs and an HB domain size of $K$, the following set of constraints hold for a valid parallelization strategy: \looseness=-1
\begin{align}
    pdt &= N      \label{eq:totalgpu} \\
    d_ht_hp_h &= K   \label{eq:hbd}  \\
    mb &= \frac{B}{d}  \label{eq:mbs} \\
    d_ld_h &= d     \label{eq:dpgpu} \\
    t_lt_h &= t    \label{eq:tpgpu} \\
    p_lp_h &= p    \label{eq:ppgpu} \\
    d_l, d_h, t_l, t_h, p_l, p_h, v, m, b  &\in \mathbb{Z_{++}} \label{eq:param_int} \\
     \frac{l}{pv}, \frac{s}{t}, \frac{h}{t}, \frac{B}{d} &\in \mathbb{Z_{++}} \label{eq:ratio_int}  \\
     \mathtt{MemoryConsumption}(p, t, d) &\leq R \label{eq:gpuram}
\end{align}


In the constraints above, $\mathbb{Z_{++}}$ refers to all positive integers. The first two constraints are guarantees that the parallelization strategy uses all physical GPUs in the cluster. Constraint~\ref{eq:totalgpu} ensures that all GPUs in the cluster participate in the training. Constraint~\ref{eq:hbd} ensures that the total parallelization dimensions map to the HB domain covers the entire domain. 

Then, Constraint~\ref{eq:mbs} establishes the relation of micro-batch size, number of micro-batches, and local batch size.  

Constraint~\ref{eq:dpgpu} to~\ref{eq:ratio_int} ensures that the mapping from the parallelization strategy to the physical cluster is valid. Constraint~\ref{eq:dpgpu} to~\ref{eq:ppgpu} divides each of the parallelization dimension into grids of GPUs that spans HB and network domains. Constraints~\ref{eq:param_int} guarantees that each parallelization dimension, interleaving size, micro-batch size, and micro-batch counts are integers in the final solution. In the end, Constraints~\ref{eq:ratio_int} requires the ratio of a few parameters also to be integers. $\frac{l}{pv}$ is the number of transformer blocks per interleaved scheduling stage. $\frac{s}{t}$ and $\frac{h}{t}$ refers to the number of sequences and hidden dimensions each GPU gets when $t$-way tensor (and sequence) parallelism is applied. Finally, the local batch size is $\frac{B}{d}$.

Finally, Constraint~\ref{eq:gpuram} ensures that LLMs parallelized with the generated parallelization strategy fit in the GPU RAM. We follow Korthikanti et al.~\cite{korthikanti2022reducing} for calculating the memory consumption of the LLM with selective activation recomputation. 

We exhaustively generate all valid configurations for a given GPT model and the network cluster in the evaluation. While the nature of minimizing Eq.~\ref{eq:itertime} subject to all constraints in this section is non-convex, there are a limited number of valid configurations. For instance, the GPT-1T with 128 GH200 computers yielded $O(10000)$ possible configurations, making an exhaustive search possible. 

\section{Rail-only Network Slowdown for All-to-All Traffic}\label{appendix:a2aslowdown}

This section computes the slowdown of All-to-All traffic in a rail-only network compared to a rail-optimized network. We utilize the hierarchical All-to-All algorithm described in DeepSpeed Zero++~\cite{deepspeedzeropp} for the rail-only network, but the result remains the same for a straightforward two-hop forwarding scheme.

Consider a grid of $x\times y$ GPUs where each $x$ lives in an HB domain, and the $y$ HB domains are connected with a slower network domain. For All-to-All traffic in a rail-optimized network, every GPU sends to all other GPUs within an HB domain through the local fast interconnect and sends the rest of the traffic directly to their destinations through the full-bisection network. Assuming each shard to send is of size $D$, the total completion time is:
\begin{align*}
T_{a2a}^{Full\ Bisec}&=\max(\frac{(x-1)D}{C_F}, \frac{x(y-1)D}{C_S}) \\
 &= \frac{x(y-1)D}{C_S} \numberthis
\end{align*}
The hierarchical algorithm runs an All-to-All locally for the rail-only network that prepares each GPU to have all data to send on its rail. In this step, the effective data shard size is $xD$. Then, within each rail, the GPU runs a second All-to-All to finalize the algorithm with an effective shard size of $xD$. The total transmission time is
\begin{equation}
T_{a2a}^{Rail\ Only}=\frac{y(x-1)D}{C_F} + \frac{x(y-1)D}{C_S}
\end{equation}
Note the two terms differ by ${y(x-1)D}/{C_F}$, which is the cost of forwarding All-to-All traffic within HB domains. When $ y(x-1)\approx x(y-1)$, the slow down factor is approximately 
\begin{equation}
    \frac{T_{a2a}^{Rail\ Only}-T_{a2a}^{Full\ Bisec}}{T_{a2a}^{Full\ Bisec}}\approx\frac{C_F}{C_S}
\end{equation}

\newpage 

\section{Details of Model Hyperparameters in Section~\ref{sec:evaluation}}\label{appendix:modeldetails}
Table~\ref{tab:model_acc_eval} presents the detailed parameters for the LLM models evaluated in Section~\ref{sec:evaluation}. We utilize the same batch size, micro-batch size, parallelization strategy, schedule interleaving, and enabled selective activation computation, as in Korthikant et al.~\cite{korthikanti2022reducing}, to obtain the result in Figure~\ref{fig:hfu}. Although Narayanan~\cite{narayanan2021efficient} et al. also present a similar evaluation in MegatronLM, the paper did not report the exact micro-batch size and interleaving count. Therefore, we cannot get an accurate estimation to compare for our formulation. The analysis in Section~\ref{sec:system-arch} is unaffected by the micro-batch size and assumes no interleaving, and adding interleaving does not change the analysis result. 


%%%%%%%%%%%%%%%%%%%%%%%%%%%%%%%%%%%%%%%%%%%%%%%%%%%%%%%%%%%%%%%%%%%%%%%%%%%%%%%
%%%%%%%%%%%%%%%%%%%%%%%%%%%%%%%%%%%%%%%%%%%%%%%%%%%%%%%%%%%%%%%%%%%%%%%%%%%%%%%
% SUPPLEMENTAL CONTENT AS APPENDIX AFTER REFERENCES
%%%%%%%%%%%%%%%%%%%%%%%%%%%%%%%%%%%%%%%%%%%%%%%%%%%%%%%%%%%%%%%%%%%%%%%%%%%%%%%
%%%%%%%%%%%%%%%%%%%%%%%%%%%%%%%%%%%%%%%%%%%%%%%%%%%%%%%%%%%%%%%%%%%%%%%%%%%%%%%
% \appendix
% \section{Please add supplemental material as appendix here}
% %
% Put anything that you might normally include after the references as an appendix here, {\it not in a separate supplementary file}. Upload your final camera-ready as a single pdf, including all appendices.

%%%%%%%%%%%%%%%%%%%%%%%%%%%%%%%%%%%%%%%%%%%%%%%%%%%%%%%%%%%%%%%%%%%%%%%%%%%%%%%
%%%%%%%%%%%%%%%%%%%%%%%%%%%%%%%%%%%%%%%%%%%%%%%%%%%%%%%%%%%%%%%%%%%%%%%%%%%%%%%


\end{document}


% This document was modified from the file originally made available by
% Pat Langley and Andrea Danyluk for ICML-2K. This version was created
% by Iain Murray in 2018. It was modified from a version from Dan Roy in
% 2017, which was based on a version from Lise Getoor and Tobias
% Scheffer, which was slightly modified from the 2010 version by
% Thorsten Joachims & Johannes Fuernkranz, slightly modified from the
% 2009 version by Kiri Wagstaff and Sam Roweis's 2008 version, which is
% slightly modified from Prasad Tadepalli's 2007 version which is a
% lightly changed version of the previous year's version by Andrew
% Moore, which was in turn edited from those of Kristian Kersting and
% Codrina Lauth. Alex Smola contributed to the algorithmic style files.
