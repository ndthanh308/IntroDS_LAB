% ****** Start of file apssamp.tex ******
%
%   This file is part of the APS files in the REVTeX 4.2 distribution.
%   Version 4.2a of REVTeX, December 2014
%
%   Copyright (c) 2014 The American Physical Society.
%
%   See the REVTeX 4 README file for restrictions and more information.
%
% TeX'ing this file requires that you have AMS-LaTeX 2.0 installed
% as well as the rest of the prerequisites for REVTeX 4.2
%
% See the REVTeX 4 README file
% It also requires running BibTeX. The commands are as follows:
%
%  1)  latex apssamp.tex
%  2)  bibtex apssamp
%  3)  latex apssamp.tex
%  4)  latex apssamp.tex
%
\documentclass[
% reprint,
twocolumn,
superscriptaddress,
%groupedaddress,
%unsortedaddress,
%runinaddress,
%frontmatterverbose, 
%preprint,
preprintnumbers,
%nofootinbib,
%nobibnotes,
%bibnotes,
% aps,
%pra,
prl,
%rmp,
%prstab,
%prstper,
%floatfix,
]{revtex4-2}

\usepackage{amsmath,amssymb}
\usepackage{graphicx}% Include figure files
\usepackage{dcolumn}% Align table columns on decimal point
\usepackage{bm}% bold math
\usepackage{float}
%\usepackage{xcolor}
\usepackage[dvipsnames]{xcolor}
\usepackage{hyperref}
\hypersetup{colorlinks=true,allcolors=MidnightBlue}
%\usepackage{hyperref}% add hypertext capabilities
\usepackage{braket}  
\usepackage{xcolor}

%\usepackage[mathlines]{lineno}% Enable numbering of text and display math
%\linenumbers\relax % Commence numbering lines

%\usepackage[showframe,%Uncomment any one of the following lines to test 
%%scale=0.7, marginratio={1:1, 2:3}, ignoreall,% default settings
%%text={7in,10in},centering,
%%margin=1.5in,
%%total={6.5in,8.75in}, top=1.2in, left=0.9in, includefoot,
%%height=10in,a5paper,hmargin={3cm,0.8in},
%]{geometry}

\begin{document}

\title{The tilted-plane structure of the
energy of open quantum systems}% Force line breaks with \\


\author{Andrew C. Burgess}
\affiliation{%
School of Physics, Trinity College Dublin, The University of Dublin, Ireland} 

\author{Edward Linscott}%
\affiliation{%
Theory and Simulation of Materials (THEOS), École Polytechnique Fédérale de Lausanne, 1015 Lausanne, Switzerland
}

 \author{David D. O'Regan}%
 \email{david.o.regan@tcd.ie}
\affiliation{%
School of Physics, Trinity College Dublin, The University of Dublin, Ireland} 



\date{\today}% It is always \today, today,
             %  but any date may be explicitly specified

\begin{abstract}
The piecewise linearity condition
on the total energy with respect to total magnetization  of open quantum
systems is derived,  using the infinite-separation-limit technique. 
This generalizes the well known
constancy condition, related
to static correlation error, in 
approximate density functional theory
(DFT).
The magnetic analog of the DFT Koopmans' theorem is also derived.
Moving to fractional electron count, 
the tilted plane condition
is derived, lifting certain assumptions
in previous works. This generalization of the flat plane condition characterizes the total energy curve of a  finite system for all values of electron count $N$ and magnetization $M$. This result is used in combination
with tabulated  spectroscopic data
to show the flat plane-structure
of the oxygen atom, as an example.
A diverse set of tilted plane structures is shown to occur in  $d$-orbital subspaces, depending
on their chemical coordination.
General occupancy-based expressions 
for total energies are 
demonstrated thereby to be 
necessarily dependent 
on the symmetry-imposed degeneracies. 
\end{abstract}

%\keywords{Suggested keywords}%Use showkeys class option if keyword
                              %display desired
\maketitle


%\tableofcontents

Density Functional Theory (DFT), a workhorse of computational chemistry and condensed matter physics \cite{hohenbergInhomogeneousElectronGas1964a,kohnSelfConsistentEquationsIncluding1965,m.tealeDFTExchangeSharing2022}, owes its success to the development of relatively efficient, reliable and accurate approximations to the exchange-correlation (XC) functional \cite{voskoAccurateSpindependentElectron1980a,perdewGeneralizedGradientApproximation1996a,beckeDensityfunctionalExchangeenergyApproximation1988a,beckeDensityFunctionalThermochemistry1993a,heydHybridFunctionalsBased2003a,leeDevelopmentColleSalvettiCorrelationenergy1988a, perdewGeneralizedGradientApproximation1996b,perdewAccurateSimpleAnalytic1992}. These approximations can be developed by enforcing exact conditions and appropriate norms on a functional of given mathematical form \cite{sunStronglyConstrainedAppropriately2015}. 
Two well-known such exact conditions are the piecewise linearity condition with respect to electron count $N$ \cite{perdewDensityFunctionalTheoryFractional1982a} and the constancy condition with respect to magnetization $M$ \cite{yangDegenerateGroundStates2000a,cohenFractionalSpinsStatic2008a} that, 
as we will discuss, is a special case
or a more general linearity condition. 
Although these two exact conditions are 
most commonly discussed in the context
of DFT, they are quite general and
may  find applications in quantum 
science more widely. 
The former condition states that the total ground state energy of a system with external potential $v(\bf{r})$ and electron count $N=N_0+\omega$ is given by
\begin{equation}
\label{eqn:pwl}
E_v(N_0+\omega)=(1-\omega)E_v(N_0)+\omega E_v(N_0+1),
\end{equation}
where $N_0 \in \mathbb{N}^0$ and $0 \leq \omega \leq 1$. A DFT calculation for a finite system with a non-integer electron count necessitates fractional occupancy of at least one Kohn Sham (KS) orbital. Assuming this fractional occupancy is limited to one KS orbital, the slope of the linear segment of the $E_v(N)$ curve is given by
\begin{equation}
\label{eqn:janaks_theorem}
\bigg(\frac{\partial E_v}{\partial N}\bigg)_v=\epsilon_f,
\end{equation}
where $\epsilon_f$ is the fractionally occupied KS eigenvalue \cite{janakProofThatFrac1978,koopmansUeberZuordnungWellenfunktionen1934,kronikPiecewiseLinearityFreedom2020}. A derivative discontinuity must therefore occur at integer values of electron count \cite{perdewPhysicalContentExact1983, shamDensityFunctionalTheoryEnergy1983, yangDerivativeDiscontinuityBandgap2012,  mori-sanchezDerivativeDiscontinuityExchange2014, gouldKohnShamPotentialsExact2014}. The left-hand partial derivative is given by the highest occupied KS eigenvalue. The right-hand partial derivative is given by the lowest unoccupied KS eigenvalue, $\epsilon_{\rm LUKS}$, with an additional contribution from the derivative discontinuity of the exact XC functional, that is by
\begin{equation}
\label{eqn:janaks_theorem_luks}
\lim_{\delta \to 0} \bigg(\frac{\partial E_v}{\partial N}\bigg)_v\bigg|_{N_0+\delta}=\epsilon_{\rm LUKS}+\Delta_{xc}^N.
\end{equation}
The piecewise linearity condition with respect to electron count \cite{perdewDensityFunctionalTheoryFractional1982a}, as given by Eq.~\ref{eqn:pwl}, assumes that the convexity condition is satisfied, namely that
\begin{equation}
\label{eqn:convexity_condition}
2E_v[N_0] \leq E_v[N_0-1]+E_v[N_0+1].
\end{equation}
While this condition is well-supported empirically, no first-principles proof of it has been given \cite{parrDensityFunctionalTheoryAtoms}.

The second of the two aforementioned exact conditions is the constancy condition with respect to magnetization $M$ \cite{yangDegenerateGroundStates2000a,cohenFractionalSpinsStatic2008a}, which states that the total ground state energy of a system with electron count $N_0$ and magnetization $M$ satisfies
\begin{align}
\label{eqn:constancy}
&E_v(N_0,M)=E_v(N_0,M_0)=E_v(N_0,-M_0)
\end{align}
for any magnetization $|M| \leq M_0$.  
Here,  $M_0 \in \mathbb{N}^0$ is the maximum magnetization of the lowest-energy state for a given integer electron count $N_0 \in \mathbb{N}^0$. Critically, the magnetization constancy condition does not apply to high magnetization states where $|M| > M_0$. 
Here, and in what follows,  
it is supposed that no ambient 
field coupling to spin is present,
notwithstanding that  
spin-\emph{asymmetric} energy terms  
can sometimes be needed~\cite{burgessMathrmDFTTexttypeFunctional2023}.


The linearity and constancy conditions given by equations \ref{eqn:pwl} and \ref{eqn:constancy} can be combined and generalized \cite{chanFreshLookEnsembles1999,yangDegenerateGroundStates2000a,mori-sanchezDiscontinuousNatureExchangeCorrelation2009a,perdewExactExchangecorrelationPotentials2009,devriendtQuantifyingDelocalizationStatic2021,devriendtUncoveringPhaseTransitions2022,g.janeskoReplacingHybridDensity2021} to give the flat plane condition
\begin{align}
\label{eqn:flat_plane_condition}
E_v(N,M) ={}& (1-\omega)E_v(N_0,M_0)\nonumber \\ 
{}&+ \omega E_v(N_0+1,M_1)
\end{align}
for $|M| \leq M_0+\omega(M_1-M_0)$, where $N=N_0+\omega$ and $M_1 \in \mathbb{N}^0$ is the maximum magnetization of the lowest-energy state for the $N_0 +1$ electron system.

Many approximate XC functionals fail to obey these exact conditions, and this failure has been directly linked to poor performance in the prediction of band gaps \cite{perdewDensityFunctionalTheory1985a,borlidoLargeScaleBenchmarkExchange2019a,cohenFractionalChargePerspective2008a}, molecular dissociation \cite{ruzsinszkySpuriousFractionalCharge2006a,dutoiSelfinteractionErrorLocal2006a,nafzigerFragmentbasedTreatmentDelocalization2015a,bryentonDelocalizationErrorGreatest}, and electronic transport \cite{toherSelfInteractionErrorsDensityFunctional2005}. However, a small number of functionals have been developed that, fully or to some extent, enforce the flat plane condition \cite{bajajCommunicationRecoveringFlatplane2017,bajajNonempiricalLowcostRecovery2019,burgessMathrmDFTTexttypeFunctional2023,janeskoDerivingExtendedDFT2023,suDescribingStrongCorrelation2018,proynovCorrectingChargeDelocalization2021,kongDensityFunctionalModel2016,johnsonCommunicationDensityFunctional2011,prokopiouOptimalTuningPerspective2022}. 

The magnetization constancy condition is limited to the lowest energy magnetization states at a given value of electron count. However, higher-energy magnetization states are also of significant interest. For example, the lowest energy triplet state plays a crucial role in phosphorescence \cite{maSupramolecularPurelyOrganic2021,yeConfiningIsolatedChromophores2021}, thermally activated delayed fluorescence \cite{yangRecentAdvancesOrganic2017,amybrydenOrganicThermallyActivated2021} and singlet fission \cite{smithSingletFission2010,buddenSingletExcitonFission2021}. Within spin density functional theory (SDFT), such higher energy magnetization states are distinct from excited states as one may compute the lowest energy state of each symmetry within the KS scheme \cite{gunnarssonExchangeCorrelationAtoms1976}. Thus, we wish to investigate the structure of the exact $E_v(N=N_0,M)$ curve at all values of magnetization $M$, as opposed to the limited interval $|M| \leq M_0$ of Eq.~\ref{eqn:constancy}. 

{\bf Theorem 1.1.} {\it The $E_v[N_0,M]$ curve is piecewise linear with respect to magnetization M.}

The structure of the $E_v(N_0,M)$ curve can be elucidated by employing the technique, developed in W. Yang et al.\ \cite{yangDegenerateGroundStates2000a}, of constructing a system with external potential $v({\bf r})$ that is composed of $q$ copies of the same finite system with external potential $v_{{\bf R}_l}({\bf r})$, with all copies being infinitely separated in space. We then have that
\begin{equation}
\label{eqn:external_potential}
v({\bf r})=\sum_{l=1}^q v_{{\bf R}_l}({\bf r}).
\end{equation}
The total number of sites $q$ is chosen so that the ground state energy per site is minimized. The total magnetization of the system is denoted by $qM$, where $q$ and $qM \in \mathbb{Z}$ but typically $M \notin \mathbb{Z}$. The ground state of the system will be composed of $p$ sites with a magnetization $M_i$, and $q-p$ sites with a magnetization $M_j$, where
\begin{align}
\label{eqn:total_system_magnetization}
&{}qM = pM_i +(q-p)M_j, \quad
 p,q \in \mathbb{N}^0, \\   &{}p \leq q,  \quad M_i,M_j \in \mathbb{Z}, \quad \mbox{and} \quad M_i \leq M \leq M_j. \nonumber
\end{align}
Care must be taken in the choice of site magnetizations $M_i$ and $M_j$. Often $M_i=M_j \pm 2$ but this is not always the case. For example, in the case of the nitrogen atom $M_i=+3$ and $M_j=-3$ for all $-3<M<+3$. The correct choices of $M_i$ and $M_j$ are those that minimize the total energy of the system with magnetization $qM$ (which is discussed further in SI-I.

At the infinite-separation-limit, the ground state wave function of the system is the anti-symmetric product of ground state wave functions at each site. One possible ground state is for the first $p$ sites to have magnetization $M_i$ and wave function $\Phi_{M_i}$ and the remaining $q-p$ sites to have magnetization $M_j$ and wave function $\Phi_{M_j}$. The ground state wave function of the total system is 
\begin{align}
\Psi_1=\hat{A}\bigg(\Phi_{M_i}({\bf{R}}_1)&\cdots\Phi_{M_i}({\bf{R}}_p)\nonumber \\ &\Phi_{M_j}({\bf{R}}_{p+1})\cdots\Phi_{M_j}({\bf{R}}_q)\bigg).
\end{align}
Exchanging the magnetizations $M_i$ and $M_j$ at any two sites results in a degenerate ground state wave function. Therefore, the average of all such wave functions $\Psi_{{\rm avg}}$ is also a ground state wave function of the system. The ground state spin $\sigma$ density of $\Psi_{{\rm avg}}$ is given by
\begin{equation}
\label{eqn:spin-density-expression}
\rho^{\sigma}({\bf r})=\sum_{l=1}^q \frac{p}{q} {\rm \hspace{0.75mm}} \rho_{l}^{\sigma }({\bf r};M_i)+\frac{q-p}{q} {\rm \hspace{1mm}} \rho_{l}^{\sigma}({\bf r};M_j),
\end{equation}
where $\rho_{l}^{\sigma }({\bf r};M_i)$ is the spin $\sigma$ spin density of site $l$ with a magnetization of $M_i$. In this case, each of the $q$ non-interacting sites have identical spin-resolved densities.
To deduce the piecewise linearity condition for magnetization we make three reasonable assertions about the nature of the total energy functional, namely that it is (a) exact for all $v$-representable spin-densities, (b) size-consistent, and (c) translationally invariant.

% Figure environment removed

From (a), the total energy function should be exact for the spin resolved density of Eq.~\ref{eqn:spin-density-expression} so that
\begin{equation}
\label{eqn:exact-energy}
E_v[\rho^{\sigma }%({\bf r})
]=pE_{v_{{\bf R}_l}}(M_i)+(q-p)E_{v_{{\bf R}_l}}(M_j).
\end{equation}
From (b), the total energy functional should be size-consistent, whereupon
\begin{equation}
\label{eqn:size-consistent-energy}
E_v(\rho^{\sigma }%({\bf r})
)=\sum_{l=1}^qE_{v_{{\bf R}_l}}\bigg(\frac{p}{q} {\rm \hspace{0.75mm}} \rho_{l}^{\sigma }(%{\bf r};
M_i)+\frac{q-p}{q} {\rm \hspace{1mm}} \rho_{l}^{\sigma }(%{\bf r};
M_j)\bigg).
\end{equation}
Eq.~\ref{eqn:size-consistent-energy} can be simplified by application of (c), translational invariance, following which
\begin{equation}
\label{eqn:translational-invariance-energy}
E_v[\rho^{\sigma }%({\bf r})
]=qE_{v_{{\bf R}_l}}\bigg[\frac{p}{q} {\rm \hspace{0.75mm}} \rho_{l}^{\sigma }(%{\bf r};
M_i)+\frac{q-p}{q} {\rm \hspace{1mm}} \rho_{l}^{\sigma }(%{\bf r};
M_j)\bigg].
\end{equation}
From Eqs.~\ref{eqn:exact-energy} and \ref{eqn:translational-invariance-energy}, the total energy of the isolated site $l$ with magnetization $M_l$ is given by
\begin{align}
E_{v_{{\bf R}_l}}(N_0,M_l)=&\frac{p}{q}E_{v_{{\bf R}_l}}(N_0, M_i)\nonumber \\ &+\frac{q-p}{q}E_{v_{{\bf R}_l}}(N_0, M_j),
\end{align}
where the site magnetization is given by
\begin{equation}
\label{eqn:site_magnetization}
M_l=\frac{p}{q}M_i+\frac{q-p}{q}M_j=M.
\end{equation}
An equivalent argument will hold for any value of $M$ in the range $M_i \leq M \leq M_j$. Making the change of variable $\omega =p/q$ and relabelling the site potential ${v_{{\bf R}_l}}({\bf r})$ simply as $v({\bf r})$, we may succinctly state that
\begin{align}
\label{eqn:magnetization_set_up}
&E_v(N_0,M)=\omega E_v(N_0, M_i)+(1-\omega)E_v(N_0, M_j),  \\ 
&M=\omega M_i+(1-\omega)M_j, {\rm \hspace{0.3cm}} M_i, M_j \in \mathbb{Z} {\rm \hspace{0.3cm}} \& {\rm \hspace{0.3cm}} 0 \leq \omega \leq 1. \nonumber
\end{align}
Therefore, the exact spin density functional $E_v[\rho^{\upharpoonright }({\bf r}),\rho^{\downharpoonright }({\bf r})]$ obeys a piecewise linearity condition with respect to magnetization as opposed to simply a constancy condition. We note that G\'{a}l and Geerlings \cite{galEnergySurfaceChemical2010} arrived at a similar prediction of a piecewise linearity condition for magnetization, by invoking a zero-temperature grand canonical ensemble.

Approximate total energy functionals, $E^{\rm aprx}_v$ typically do not obey the piecewise linearity condition with respect to magnetization. We may refer to the deviation of $E^{\rm aprx}_v$ from the exact piecewise linear $E_v(N_0,M)$ curve as magnetic piecewise linearity error (MPLE), 
\begin{align}
E^{\rm MPLE}&{}(N_0,M) = E^{\rm aprx}_v(N_0,M) - \\ &{} \bigg[\omega E^{\rm aprx}_v(N_0,M_i) +(1-\omega)E^{\rm aprx}_v(N_0,M_j)\bigg], \nonumber
\end{align}
where the electron count is a constant integer value $N_0$ and $M$, $M_i$, $M_j$ and $\omega$ are given by equation \ref{eqn:magnetization_set_up}. A plot of $E^{\rm MPLE}(M)$ for the neutral helium atom using the PBE XC functional \cite{perdewGeneralizedGradientApproximation1996a}, is shown in Fig.~\ref{figure:PWL_figure}. 

MPLE differs from static correlation error (SCE) \cite{cohenFractionalSpinsStatic2008a,burtonVariationsHartreeFock2021}, which is defined as the spurious energy difference between degenerate states due to use of an approximate XC functional. Not all MPLE can be described as SCE. SCE defines errors in the total energy of states with non-integer magnetization that are {\emph{degenerate}} in energy to a state with integer magnetization, while MPLE defines errors in the total energy of all non-integer magnetization states. The converse is also true, not all SCE can be described as MPLE, for example the error in the total energy of the spherically symmetric boron atom is an SCE but not an MPLE \cite{cohenFractionalSpinsStatic2008a}.

{\bf Theorem 1.2.} {\it The partial derivative of the $E_v[N_0,M]$ curve with respect to magnetization M is equal to half the difference of the frontier KS eigenvalues, whenever the left and right partial derivatives both exist.}

% Figure environment removed
%
The magnetic analogue of the DFT Koopmans’ theorem can be derived by simple application of the chain rule in conjunction with the well studied (spin resolved) DFT Koopmans’ theorem. This dispenses with the need to invoke total single particle energies or grand canonical ensembles used in previous proofs \cite{galEnergySurfaceChemical2010,capelleSpinGapsSpinflip2010} of this theorem. The partial derivative of the total energy with respect to magnetization may be expressed in terms of the spin resolved electron counts
\begin{equation}
\label{eqn:partial_deriv_decomposition}
\bigg(\frac{\partial E_v}{\partial M}\bigg)_{v, N}=\sum_{\sigma}\bigg(\frac{\partial N^{\sigma}}{\partial M}\bigg)_N \bigg(\frac{\partial E_v}{\partial N^{\sigma}}\bigg)_{v, N^{\bar{\sigma}}},
\end{equation}
whenever the necessary partial 
derivatives exist. The right and left  partial derivatives of $E_v$ with respect to $N^{\sigma}$ are given, respectively,  by
\begin{align}
\label{eqn:deriv_with_respect_to_nsigma}
\lim_{\delta \to 0^+} \bigg(\frac{\partial E_v}{\partial N^{\sigma}}\bigg)\bigg|_{N^{\sigma}_0} =& \epsilon_{\rm LUKS}^{\sigma} +\Delta_{xc}^{N^{\sigma}}, {\mbox{ and}} \nonumber \\
\lim_{\delta \to 0^-} \bigg(\frac{\partial E_v}{\partial N^{\sigma}}\bigg)\bigg|_{N^{\sigma}_0} =& \epsilon_{\rm HOKS}^{\sigma},
\end{align} 
where $\epsilon_{\rm HOKS}^{\sigma}$ is the highest occupied spin-$\sigma$ KS eigenvalue and $\epsilon_{\rm LUKS}^{\sigma}$ is the lowest unoccupied spin-$\sigma$ KS eigenvalue. $\Delta_{xc}^{N^{\sigma}}$ is the explicit derivative discontinuity of the exact XC functional through its explicit dependence on $N^{\sigma}$. Eq.~\ref{eqn:deriv_with_respect_to_nsigma} is the spin resolved analogue of Eqs.~\ref{eqn:janaks_theorem} and \ref{eqn:janaks_theorem_luks}. For further details,  see Refs.~\cite{chanFreshLookEnsembles1999,yangDerivativeDiscontinuityBandgap2012,gritsenkoAnalogKoopmansTheorem2002,gritsenkoSpinunrestrictedMolecularKohn2004,galNonuniquenessMagneticFields2009}. The right and left partial derivatives of the total energy with respect to magnetization of a system with no fractional KS orbital occupations is thus given by
\begin{align}
\label{eqn:koopmans_for_magnetization}
\lim_{\delta \to 0^+} \bigg(\frac{\partial E_v}{\partial M}\bigg)\bigg|_{M_0} =& \frac{1}{2}\bigg[\epsilon_{\rm LUKS}^{\upharpoonright}-\epsilon_{\rm HOKS}^{\downharpoonright} + \Delta_{xc}^{N^{\upharpoonright}}\bigg] \quad {\mbox{and}} \nonumber \\
\lim_{\delta \to 0^-} \bigg(\frac{\partial E_v}{\partial M}\bigg)\bigg|_{M_0} =& \frac{1}{2}\bigg[\epsilon_{\rm HOKS}^{\upharpoonright}-\epsilon_{\rm LUKS}^{\downharpoonright} - \Delta_{xc}^{N^{\downharpoonright}}\bigg],
\end{align}
since the partial derivative of $N^{\sigma}$ with respect to $M$ is equal to one half. For systems with one fractionally occupied spin up and one fractionally occupied spin down frontier KS orbital, the expression given by Eq.~\ref{eqn:koopmans_for_magnetization} simplifies to
\begin{equation}
\label{eqn:koopmans_for_magnetization_fractional}
\frac{\partial E_v}{\partial M}= \frac{1}{2}\bigg[\epsilon_{f}^{\upharpoonright}-\epsilon_{f}^{\downharpoonright} \bigg].
\end{equation}
Typical XC functionals will break this exact condition. In Fig. \ref{figure:koopmans_figure} the slope of the $E_v[N_0,M]$ curve for the Helium atom as evaluated from the PBE KS eigenvalues is plotted as a function of magnetization. Use of the exact XC functional would result in a perfect step function. 

{\bf Theorem 1.3.} {\it The $E_v[N,M]$ curve obeys the tilted plane condition.}

% Figure environment removed

Analysis of the $E_v[N_0,M]$ curve may be extended to include states with not only fractional magnetization $M$ but also fractional electron count $N$. Again, one may construct an external potential given by Eq.~\ref{eqn:external_potential}, in this case typically both $N$ and $M \notin \mathbb{Z}$, but the total system electron count and magnetization, $qN$ and $qM \in \mathbb{Z}$. The ground state of this system will be composed of $qc_i$ sites with electron count $N_i$ and magnetization $M_i$, where $i$ ranges from 1 to $V_{\rm count}$ with:
\begin{equation}
 \sum_{i=1}^{V_{\rm count}} c_i=1,  {\rm \hspace{0.4cm}} 0  \leq c_i \leq 1,   {\rm \hspace{0.4cm}} N_i,{\rm \hspace{1.0mm}} M_i,{\rm \hspace{1.0mm}} qc_i \in \mathbb{Z}.
\end{equation}
The values of $c_i$, $N_i$ and $M_i$ are constrained so that:
\begin{equation}
N=\sum_{i=1}^{V_{\rm count}} c_i N_i \quad \mbox{and} \quad M=\sum_{i=1}^{V_{\rm count}} c_iM_i,
\end{equation}
Following an analogous derivation to that outlined in theorem 1.1, one finds that, 
for $N,M \notin \mathbb{Z}$, 
\begin{equation}
\label{eqn:tilted_plane_condition}
E_v[N,M]=\sum_{i=1}^{V_{\rm count}} c_iE_v[N_i,M_i].
\end{equation}




Vertices in the energy landscape will occur at the specified integer values of electron count $N_i$ and magnetization $M_i$. $V_{\rm count}$ is the number of vertices associated with a particular plane, typically  equal to 3 or 4, however higher numbers of vertices are possible in very rare circumstances, further details of which can be found in SI-II. For clarity in the below discussion, we restrict $V_{\rm count}=4$, %(with the possibility of $c_4=0)$, 
however, the 
method used to generate Fig.~\ref{figure:oxygen} and Fig.~\ref{figure:tilted_planes-d-orbitals} includes no restriction on the vertex count. 

% Figure environment removed

In cases where the electron counts of the four states satisfy $N_1=N_2=N_3-1=N_4-1$ and the magnetizations  $M_1=-M_2$,  $M_3=-M_4$, with $E_v[N_i,M_i]=\min_M\{E_v[N_i,M]\}$, Eq.~\ref{eqn:tilted_plane_condition} simplifies to the \emph{flat} plane condition as outlined in Eq.~\ref{eqn:flat_plane_condition}. X. Yang et al.~\cite{yangCommunicationTwoTypes2016a} and Cuevas-Saavedra et al.~\cite{cuevas-saavedraSymmetricNonlocalWeighted2012a} report the existence of two types of flat plane structures. These two different flat plane structures will occur as special cases of the more general condition outlined by Eq.~\ref{eqn:tilted_plane_condition}, specifically when $c_4=0$. However, restricting $c_4$ to be zero prohibits the correct flat plane structure of systems, e.g., the oxygen atom for $7 \leq N\leq 8$ and $|M| \leq 10-N$, where the  correct expression for $E_v[N,M]$ may
be written as 
\begin{widetext}
\begin{align}
\label{eqn:oxygen_flat_plane}
E_v[N,M]=&c_1E_v[8,2]+c_2E_v[8,-2]+c_3E_v[7,3]+c_4E_v[7,-3]=\frac{(N-7)(10-N+M)}{2(10-N)}E_v[8,2]+ \\ & \frac{(N-7)(10-N-M)}{2(10-N)}E_v[8,-2]+ \frac{(8-N)(10-N+M)}{2(10-N)}E_v[7,3]+\frac{(8-N)(10-N-M)}{2(10-N)}E_v[7,-3]. \nonumber
\end{align}
\end{widetext}
Despite $E_v[8,2]=E_v[8,-2]$ and $E_v[7,3]=E_v[7,-3]$ for the oxygen atom, reduction  of Eq.~\ref{eqn:oxygen_flat_plane} to the sum of two terms with coefficients $c_1^{\prime}$ and $c_2^{\prime}$ would  wrongly give 
\begin{equation}
M \neq \sum_{i=1}^2 c_i^{\prime} M_i.
\end{equation}
G\'{a}l and Geerlings \cite{galEnergySurfaceChemical2010} reported the existence of a tilted plane energy surface but their energy expressions also have the $c_4=0$ restriction, meaning that $(N_i,M_i)$ values in their energy expression will not always represent vertices in the energy landscape. The same is true for G. K.-L. Chan's $E_v[N,M]$ energy expression \cite{chanFreshLookEnsembles1999}. To the best of our knowledge, a lifting of the $c_4 = 0$ restriction of the generalized flat plane condition, has only been discussed to date in the unpublished Ref.~\cite{malekDiscontinuitiesEnergyDerivatives2013}. If we assume that $(N_i,M_i)$ values represent vertices in the energy landscape, the $c_4=0$ restriction allows for triangular shaped planes but neglects planes of other shapes, such as isosceles trapezoids, which for example occur for the oxygen atom/cation as shown in Fig.~\ref{figure:oxygen}, for $5\leq N \leq 8$ at low values of magnetization.

We refer to Eq.~\ref{eqn:tilted_plane_condition} as the \emph{tilted plane condition}. In analogy to the piecewise linearity condition with respect to magnetization, the $E_v[N,M]$ curve described by Eq.~\ref{eqn:tilted_plane_condition} will often exhibit a non-zero partial derivative with respect to magnetization, resulting in a `tilted plane' as opposed to a `flat plane'. Knowledge of the occurrence of a tilted plane energy surface has already been applied to correct the PBE \cite{perdewGeneralizedGradientApproximation1996a} total energy of dissociated triplet H$_5^+$ in Ref.~\cite{burgessMathrmDFTTexttypeFunctional2023}. An analysis of the `tilted plane' shape of the $E_v[N,M]$ curve for the He atom as opposed to simpler `flat plane' shape is shown in detail in SI-III.

The energies of localized electronic states within a solid material also obey the tilted plane condition in the limit where the subspace-bath interaction energy varies linearly with spin resolved occupancy. This of course includes the case where the subspace-bath interaction becomes negligible. Fig.~\ref{figure:tilted_planes-d-orbitals} displays sample tilted plane structures for a $d$-orbital subspace in the subspace-bath linear interaction limit under octahedral, tetrahedral, and square-planar crystal field splittings. The total energy of the electronic states with integer spin resolved orbital occupancies were approximated using the model 
\begin{equation}
\label{eqn:model_energy}
E^{\rm mod}=\sum_i \epsilon_i n_i+U \sum_i n_i^{\upharpoonright}n_i^{\downharpoonright}+\frac{V}{2}\sum_{i \neq j}n_i n_j - \frac{J}{2} M^2,
\end{equation}
where $\epsilon_i$ is the energy level of orbital $i$, $U$ and $V$ are the intra-orbital and inter-orbital interaction parameters, respectively, and ${n}_i^{\sigma}$ is the spin-$\sigma$ electron count. $J M^2$ is the magnetization term, used to favor either low or high spin states, where $M$ is the total magnetization. For each crystal field splitting, and 
hence degeneracy pattern, a wide variety of possible tilted pane structures exist.

In conclusion, the piecewise linearity condition with respect to magnetization and the tilted plane condition were derived from first principles using the infinite-separation-limit technique, and the magnetic analogue of the DFT Koopmans’ theorem was derived from the chain rule. 
%As shown in Fig.~\ref{figure:tilted_planes-d-orbitals}, subspace specific tilted plane energy surfaces can now be generated through use of a simple model Hamiltonian and a mathematica script. 
These three exact quantum mechanical conditions may aid in the development of DFT+U,  post-DFT methods, and 
techniques yet further beyond, 
as we find that in order to approach
the exact limit, energy functionals of occupancies
must \emph{necessarily} take different forms depending on 
symmetry-imposed degeneracies.  

The research conducted in this publication was funded by the Irish Research Council under grant number GOIPG/2020/1454. EL gratefully acknowledges financial support from the Swiss National Science Foundation (SNSF -- project number 213082).




\bibliography{main}% Produces the bibliography via BibTeX.



%\renewcommand{\thepage}{S\arabic{page}}
\renewcommand{\theequation}{S\arabic{equation}}
\renewcommand{\thefigure}{S\arabic{figure}}
\renewcommand{\thetable}{S\arabic{table}}
\setcounter{equation}{0}
\setcounter{figure}{0}

\newpage


\onecolumngrid
\vspace{\columnsep}
\section{SI-I. The correct choice of site magnetizations}
\label{sec:BLOR-derivation}
\vspace{\columnsep}
\twocolumngrid
In the derivation of the piecewise linearity condition for magnetization, the correct choice in values of $M_i$ and $M_j$ are that which minimize the total energy of the system with magnetization $qM$. This will be achieved if the following two conditions are satisfied:
\begin{enumerate}
    \item
For any values of site magnetization $M_h, M_k \in \mathbb{Z}$ where $M_i \notin \{M_h, M_k\}$, the ground-state energy of the system with magnetization $M_i$ must satisfy
\begin{align}
\label{eqn:condition2_on_mi}
&{\rm \hspace{1cm}}E_{v_{{\bf R}_l}}(M_i) < c_hE_{v_{{\bf R}_l}}(M_h)+c_kE_{v_{{\bf R}_l}}(M_k), \\
&{\rm \hspace{1cm}}M_i =c_hM_h+c_kM_k,{\rm \hspace{0.3cm}} c_h, c_k \geq 0 {\rm \hspace{0.3cm}}\&{\rm \hspace{0.3cm}} c_h+c_k=1. \nonumber 
\end{align}
The equivalent condition for $M_j$ must also be true.
    \item 
For any integer value of site magnetization $M^{\prime}$ in the range $M_i \leq M^{\prime} \leq M_j$, the ground-state energy must satisfy
\begin{align}
\label{eqn:condition1_on_mi}
&{\rm \hspace{1cm}}E_{v_{{\bf R}_l}}(M^{\prime}) \geq c_iE_{v_{{\bf R}_l}}(M_i)+c_jE_{v_{{\bf R}_l}}(M_j), \\
&{\rm \hspace{1cm}}M^{\prime}=c_iM_i+c_jM_j,{\rm \hspace{0.3cm}} c_i, c_j \geq 0 {\rm \hspace{0.3cm}}\&{\rm \hspace{0.3cm}} c_i+c_j=1. \nonumber 
\end{align}
\end{enumerate}


\onecolumngrid
\vspace{\columnsep}
\section{SI-II. Cases of high vertex count}
\label{sec:BLOR-derivation}
\vspace{\columnsep}
\twocolumngrid
In special cases, the vertex count of a given plane can be higher than four. For example, assume that the maximum magnetization of the lowest energy state of the $N-1$, $N$ and $N+1$ electron systems are given by $M$, $M+1$, and $M$ respectively, where $M \geq 1$. It is assumed that the convexity condition of Eq. \ref{eqn:convexity_condition} is satisfied for the $N-1$, $N$ and $N+1$ electron systems. In the special case where
\begin{widetext}
\begin{equation}
\label{eqn:fortuitous_slopes}
\frac{E_v[N,M+1]-E_v[N-1,M]}{N-(N-1)}=\frac{E_v[N+1,M]-E_v[N,M+1]}{(N+1)-N},
\end{equation}
\end{widetext}
the plane in question will have an hexagonal shape (stretched in the magnetization direction whenever $M>1$). Such a plane would require the summation in Eq. \ref{eqn:tilted_plane_condition} to be extended over six vertices
\begin{equation}
E_v[N,M]=\sum_{i=1}^6c_1E_v[N_i,M_i].
\end{equation}
Higher numbers of vertices will occur when there is a fortuitous equality of certain energy derivatives. Eq.~\ref{eqn:fortuitous_slopes} specifies the energy derivative equality for this particular example. It is worth noting that Eq.~\ref{eqn:fortuitous_slopes} simplifies to
\begin{equation}
2E_v[N,M+1]=E_v[N-1,M]+E_v[N+1,M],
\end{equation}
however this simplification will not occur in all cases of high vertex count. 

\clearpage

\onecolumngrid
\vspace{\columnsep}
\section{SI-III. Visualizing the tilted plane condition for Helium}
\label{sec:BLOR-derivation}
\vspace{\columnsep}
\twocolumngrid
The $E_v[N,M]$ curve for the He atom with electron count in the range $1 \leq N\leq 2$ and magnetization $M \geq 2-N$ is given by
\begin{align}
\label{eqn:he_tilted_plane}
E_v[N,M] ={\mbox{ }} & (2-N)E_v[1,1]+\frac{N-M}{2}E_v[2,0] \nonumber \\
& + \frac{M+N-2}{2}E_v[2,2].
\end{align}
This plane will have a non-zero partial derivative with respect to magnetization and thus has a `tilted plane' shape as opposed to a simple `flat plane' shape.

Approximate XC functionals typically do not obey the tilted plane condition. For example, Fig.~\ref{figure:tilted_plane} displays the energy energy of the helium atom when using the PBE XC functional, 
with respect to the exact tilted plane outlined in Eq.~\ref{eqn:he_tilted_plane}.

\vspace{5cm}


% Figure environment removed









\end{document}

%
% ****** End of file apssamp.tex ******

