
A simple form of the \emph{root extraction} is as follows:
Let $G  = \langle P,Q \rangle$ where 
\begin{equation}\label{struct1}
    G \approx \frac{\ZZ}{\ell^e\ZZ}\times\frac{\ZZ}{\ell^e\ZZ}
\end{equation}
 Consider an element $K\in G$ and $m,n\in \frac{\ZZ}{\ell^e\ZZ}$ such that,\begin{equation}\label{eqn1}
    K=mP+nQ
\end{equation}

If element $K$ and the multipliers $m,n$ are known, the root extraction problem is to find a basis $P,Q$ of $G$ such that \ref{eqn1} holds. The following table summarizes this problem and related problems.
\begin{table}[htbp]
	\centering
		\begin{tabular}{|l|c|c|c|}
			\hline
			\hline
			Problem & Element & Multipliers & Base Pts. \\
			        &  $K$    &   $m,n$ & $P,Q$   \\
			\hline
			\hline
			Exponentiation & ? & \checkmark & \checkmark \\
			\hline
			Extended DLP & \checkmark & ? & \checkmark \\
			\hline
			Root Extraction & \checkmark & \checkmark & ?\\
			\hline
            Basis computation & - & - & ?\\
			\hline
            \hline
		\end{tabular}
\end{table}

The root extraction problem has been solved for the groups of the form (\ref{struct1})  by Srinath in \cite{MSS}. Similarly for the algorithms to solve the discrete logarithm problem see \cite{Teske} and \cite{Sutherland2008StructureCA}, for the basis computation problem see \cite{Sutherland2008StructureCA}, and the square and multiply algorithm for exponentiation is given in \cite[Chapter 9, \S 9.2]{UdiManberAlgoBook}.
Lower bound for the root extraction problem in generic finite Abelian groups is given in \cite{cryptoeprint:2002/013}.

The method to solve the root extraction problem discussed in \cite{MSS} can be easily extended to any group $G$ of the form:\begin{equation}\label{Struct2}
  G \approx
\underbrace{\frac{\ZZ}{p^{e}\ZZ}\times...\times\frac{\ZZ}{p^e\ZZ}}_{N\text{-times}}
\end{equation}


Our objective is to extend the techniques given in \cite{MSS} for finite Abelian groups which, by the fundamental theorem of finite Abelian groups, can be written as \[
G \approx \prod_{i=1}^{N}\frac{\ZZ}{p_i^{e_i}\ZZ}
\]
For this, we will first solve the problem for finite Abelian $p$-groups and then extend the techniques to finite Abelian groups.

\subsection{Contributions of this work}
The following are the contributions of this work:
\begin{enumerate}
    \item We have arrived at the necessary and sufficient conditions for the existence of a solution to the root extraction
    problem in finite Abelian $p$-groups.
    \item We have provided an algorithm for root extraction in finite Abelian $p$-groups. The SageMath implementation of which can be found in \url{https://github.com/uacharjee14/Root-Extraction}.
    \item We have extended the algorithm for extracting roots in finite Abelian $p$-groups to finite Abelian groups.
    \item We conclude that root extraction in finite Abelian groups is no harder than solving discrete logarithms and computing basis. 
    %\item Our work together with \cite{cryptoeprint:2002/013} implies that either solving discrete logarithms or basis computation on groups of unknown orders has an exponential lower bound.
\end{enumerate}