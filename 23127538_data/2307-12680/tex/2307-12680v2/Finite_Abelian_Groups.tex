
We will extend our results from the previous section to solve the \emph{root extraction} problem in generic finite Abelian groups.\\
From the fundamental theorem of finite Abelian groups, the structure of a finite Abelian group $G$ is \[
G \approx \prod_{i=1}^{N}\frac{\ZZ}{p_i^{e_i}\ZZ}.
\]
We will assume that this structure is already known. This can be found using Sutherland's structure computation algorithm given in \cite[\S5]{Sutherland2008StructureCA}.
This structure can be restated as:
\[
G \approx \prod_{j=1}^{N_1}\frac{\ZZ}{p_1^{e_{1j}}\ZZ}\times \cdots\times\prod_{j=1}^{N_r}\frac{\ZZ}{p_r^{e_{rj}}\ZZ}\]
Here $e_{ij}\leq e_{ik}$ for $j \leq k$ and $p_i\neq p_j$ for $i\neq j$.
The \emph{Root Extraction Problem} can be stated as:
\begin{problem}[Root Extraction Problem in finite Abelian groups]\label{REP:finite Abelian}
    Given $m_{11},\ldots,m_{rN_r}$ such that $m_{ij} \in \frac{\ZZ}{p_i^{e_j}\ZZ}$ and $K \in G$ find a basis $\{P_{11},P_{12},\ldots,P_{rN_r}\}$ of $G$ such that $K= m_{11}P_{11}+\cdots+m_{rN_r}P_{rN_r}$.
\end{problem}
By using Sutherland's algorithm  \cite[\S5]{Sutherland2008StructureCA}) we may find a basis $\{Q_1,\ldots,Q_N\}$ for the group $G$ and then compute their orders. Then using the basis, (with known orders) we can partition the given group into Sylow $p$-subgroups and solve our problem in each of them using algorithm \ref{alg: check} discussed in the previous section.
\par
Let $\{Q_{1},Q_2,\ldots,Q_N\}$ be a basis for $G$ which can be renumbered as $\{Q_{11}, Q_{12},\ldots,Q_{rN_r} \}$ so that $|Q_{ij}|= p_i^{e_{ij}}$.\\
Then we use the Generalized Discrete Logarithm Algorithm \cite[\S3, Algorithm 2]{Sutherland2008StructureCA} to express $K$ as \[K = q_{11}Q_{11}+q_{12}Q_{12}+\cdots+q_{rN_r}Q_{rN_r}.\]
Let $G_i = \langle Q_{i1},Q_{i2},\ldots,Q_{i(N_{i})}\rangle$ and $K_i = q_{i1}Q_{i1}+\cdots+q_{i(N_{i})}Q_{i(N_{i})}$ where $1\leq i\leq r$. Now it can be seen that \[
G=G_1\oplus\cdots\oplus G_r \text{ and } K=K_1+\cdots+K_r.
\]
Note that subgroup $G_i$ is the Sylow $p_i$-subgroup of $G$ and $K_i\in G_i$ for all $i$ such that $1\leq i\leq r$.
We then pass the following inputs to Algorithm \ref{alg: check}:
\begin{enumerate} 
    \item $G_i = span\{Q_{i1},Q_{i2},\ldots,Q_{i(N_{i})}\}$ (the Group description for Sylow-$p_i$ group)
    \item $K_i = q_{i1}Q_{i1}+\cdots+q_{i(N_{i})}Q_{i(N_{i})}$ (the element of the group $G_i$)
    \item $m_{i1},\ldots,m_{iN_i}$ (the coefficients)
\end{enumerate}
This will define our $i$-th subproblem (for the prime $p_i$) \label{sub-problem}\\\\
Finally, we would make an important statement:
\begin{theorem}
    The REP for finite Abelian groups has a solution if and only if each of the sub-problems has a solution
\end{theorem}
\emph{Proof: }\begin{enumerate}
    \item \emph{If each of the sub-problems has a solution then the REP has a solution}: Let, $\beta_i=\{P_{i1},\ldots, P_{iN_i}\}$ is a solution for the $i$-th subproblem. Then, clearly $K=m_{11}P_{11}+\cdots+m_{rN_r}P_{rN_r}$. Also, $\beta=\cup_{i=1}^r\beta_i=\{P_{11},\ldots,P_{rN_r}\}$ is the \emph{basis} for the group $G$ as it is the union of the basis of all the Sylow $p_i$-subgroups of $G$.
    \item \emph{If a solution to REP exists then each of the subproblems also has a solution: } Let $\{P_{11},\ldots, P_{rN_r}\}$ be a solution to the REP. Now since $\langle P_{i1},\ldots, P_{iN_i}\rangle$ is a Sylow $p_i$-subgroup of $G$. Since, a Sylow $p_i$-subgroup of finite Abelian group is unique, so $\langle P_{i1},\ldots, P_{iN_i}\rangle=\langle Q_{i1},\ldots, Q_{iN_i}\rangle$. Therefore, $\{P_{i1},..., P_{iN_i}\}$ is a solution to the $i$-th subproblem.\hfill $\square$
\end{enumerate} 
\par
\begin{algorithm}
\caption{Root Extraction in Finite Abelian Groups}
    \label{alg: finite Abelian}
Input:$G, K, m_{11},...,m_{rN_r}$,a \emph{sorted} basis $\{Q_{11},Q_{12},...,Q_{rN_r}\}$ for $G$, $|Q_{ij}|=p_i^{e_{ij}}$, $q_{11},q_{12},...,q_{rN_r}$ such that $K= q_{11}Q_{11}+q_{12}Q_{12}+...+q_{rN_r}Q_{rN_r}$ \\
Precondition:$K\in G, m_{ij}\in \frac{\ZZ}{p_i^{e_{ij}}\ZZ}$, $e_{ij}\leq e_{ik}$ for $j \leq k$ , $p_i\neq p_j$ for $i\neq j$\\
Output:$P_{11},P_{12},...,P_{rN_r}$\\
Postcondition: $K=m_{11}P_{11}+m_{12}P_{12}+...+m_{rN_r}P_{rN_r}$ and $\{ P_{11},P_{12},...,P_{rN_r} \}$ is a basis of $G$\\
\textbf{begin}
\begin{algorithmic}[1]
    \STATE set $i=1, A=[\ ]$
    \WHILE{$i \leq r$}
        \STATE define $G' = span\{Q_{i1},Q_{i2},...,Q_{iN_i}\}$
        \STATE set $K' = q_{i1}Q_{i1}+...+q_{iN_i}Q_{iN_i}$
        \STATE pass $G',K', m_{i1},...,m_{iN_i} $ to Algorithm \ref{alg: check}.
        \STATE append output of Algorithm \ref{alg: check} to $A$
        \STATE set $i=i+1$
    \ENDWHILE
    \RETURN $A$
    
\end{algorithmic}
\end{algorithm}
\par
\textbf{The number of group operations required in Algorithm \ref{alg: finite Abelian}}\\
In step 4 we perform $N_i$ additions for each $i$. Now $N_i$ is the rank of the Sylow $p_i$-subgroup and suppose $N$ is the maximum among all such ranks i.e., $N=\max_{i=1}^rN_i$. Then, the number of group operations for additions is bounded by $O(N)$, and exponentiations take $O(e_{N_i}\log_2p_i)$ group operations. So in the loop, the total number of group operations required is $O(re_{iN_i}\log_2p_i+rN)$. Let $e=\max_{i=1}^re_{iN_i}$ and $p=\max_{i=1}^rp_i$. Algorithm \ref{alg: check} uses $O((e+N)Ne\log_2p)$. This we do for each of the $r$ primes in the structure of $G$ so the total number of group operations is bounded by $O((e+N)rNe\log_2p)$.\label{complexity}\\

\subsection{Discussion}
In all the algorithms presented till now, we have assumed that a basis $\{Q_1,\ldots, Q_N\}$ of the group and the group structure is known. We have also assumed that the discrete logarithm of the element $K$ from problem \ref{REP:finite Abelian} with respect to the basis $\{Q_1,\ldots,Q_N\}$ is known.  
The bound on the number of group operations required in the generalized discrete logarithm algorithm can be found in \cite[\S3, equation 15]{Sutherland2008StructureCA} and for the basis computation algorithm this can be found in \cite[\S5, corollary 3]{Sutherland2008StructureCA}.  
From the bound on the number of group operations required for algorithm \ref{alg: finite Abelian} we can see that \emph{root extraction in finite Abelian groups is no harder than solving discrete logarithms and computing basis}.

Further, Damgard and Koprowski \cite{cryptoeprint:2002/013} have proved that the root extraction problem in finite Abelian groups of unknown orders has an exponential lower bound. We have shown that the number of group operations for root extraction is dominated by that of basis computation and solving discrete logarithm which have exponential complexities. Thus, our results agree with that of Damgard and Koprowski. 