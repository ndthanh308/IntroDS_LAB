\vspace{-.7cm}

\section{Introduction}
\label{intro}

\vspace{-.2cm}
% Sequential data, including time series and video data, are ubiquitous today and are critical in many technical systems. These data types are characterized by their time-based ordering of observations, and the accurate modelling of time dependencies is often critical for predicting future behaviour or identifying meaningful patterns.

% By modelling how different variables change over time, we can gain insights into the underlying mechanisms driving the dynamic behaviour of these systems. Additionally, time series analysis and other techniques for modelling sequential data can help us predict future behaviour, detect anomalies or changes over time, and identify causal relationships between variables. 

% While basic strategies for modelling dynamics in technical systems may rely on techniques such as differential equations and stochastic processes, their effectiveness can be limited by computational resources, mainly when dealing with high-dimensional data.
% and can be used to simulate the behavior of the system over time or make predictions about future behavior.

Sequential data, such as time series and video data, are crucial for technical systems and require accurate modelling of time dependencies to predict future behaviour and identify meaningful patterns. Understanding how variables change over time can gain insights into underlying mechanisms driving the dynamic behaviour of a particular system. Analysis of sequential data can furthermore assist in detecting anomalies or identifying causal relationships. Although one may use basic modelling strategies such as differential equations and stochastic processes, computational limitations arise when dealing with high-dimensional data.
%The versatility of data-driven algorithms %,particularly deep-learning strategies, 
%enables the analysis of a diverse range of sequential data, including heterogeneous time series data, video data capturing sequences of images or frames depicting changes over time, audio signals, text data, and sensor data from Internet-of-Things devices.
While modelling and analyzing sequential data is crucial in multiple application areas~\cite{barra_deep_2020,martinez_time_2022, masini_machine_2023},
%, including finance [Ref], healthcare [Ref], manufacturing [Ref], and transportation? [Ref], 
there is currently no default approach concerning splitting sequential information for data-driven algorithms. Moreover, there is a lack of research on evaluating the generalization ability of sequential models trained for specific tasks and assessing the data quality used to train, e.g., machine learning (ML) models. The upcoming section will summarise the main research challenges regarding modelling time dependencies in technical systems, especially during the training phase of an ML cycle. 
We will then frame these challenges in two use cases, namely real-world applications for time series and video data analysis.

%We will then Then, we will present challenges in two specific application areas and conclude with an outlook and discussion of future work.

% Figure environment removed



% Submission to ICML 2023 will be entirely electronic, via a web site
% (not email). Information about the submission process and \LaTeX\ templates
% are available on the conference web site at:
% \begin{center}
% \textbf{\texttt{http://icml.cc/}}
% \end{center}

% The guidelines below will be enforced for initial submissions and
% camera-ready copies. Here is a brief summary:
% \begin{itemize}
% \item Submissions must be in PDF\@. 
% \item \textbf{New to this year}: If your paper has appendices, submit the appendix together with the main body and the references \textbf{as a single file}. Reviewers will not look for appendices as a separate PDF file. So if you submit such an extra file, reviewers will very likely miss it.
% \item Page limit: The main body of the paper has to be fitted to 8 pages, excluding references and appendices; the space for the latter two is not limited. For the final version of the paper, authors can add one extra page to the main body.
% \item \textbf{Do not include author information or acknowledgements} in your
%     initial submission.
% \item Your paper should be in \textbf{10 point Times font}.
% \item Make sure your PDF file only uses Type-1 fonts.
% \item Place figure captions \emph{under} the figure (and omit titles from inside
%     the graphic file itself). Place table captions \emph{over} the table.
% \item References must include page numbers whenever possible and be as complete
%     as possible. Place multiple citations in chronological order.
% \item Do not alter the style template; in particular, do not compress the paper
%     format by reducing the vertical spaces.
% \item Keep your abstract brief and self-contained, one paragraph and roughly
%     4--6 sentences. Gross violations will require correction at the
%     camera-ready phase. The title should have content words capitalized.
% \end{itemize}

% \subsection{Submitting Papers}

% \textbf{Paper Deadline:} The deadline for paper submission that is
% advertised on the conference website is strict. If your full,
% anonymized, submission does not reach us on time, it will not be
% considered for publication. 

% \textbf{Anonymous Submission:} ICML uses double-blind review: no identifying
% author information may appear on the title page or in the paper
% itself. \cref{author info} gives further details.

% \textbf{Simultaneous Submission:} ICML will not accept any paper which,
% at the time of submission, is under review for another conference or
% has already been published. This policy also applies to papers that
% overlap substantially in technical content with conference papers
% under review or previously published. ICML submissions must not be
% submitted to other conferences and journals during ICML's review
% period.
% %Authors may submit to ICML substantially different versions of journal papers
% %that are currently under review by the journal, but not yet accepted
% %at the time of submission.
% Informal publications, such as technical
% reports or papers in workshop proceedings which do not appear in
% print, do not fall under these restrictions.

% \medskip

% Authors must provide their manuscripts in \textbf{PDF} format.
% Furthermore, please make sure that files contain only embedded Type-1 fonts
% (e.g.,~using the program \texttt{pdffonts} in linux or using
% File/DocumentProperties/Fonts in Acrobat). Other fonts (like Type-3)
% might come from graphics files imported into the document.

% Authors using \textbf{Word} must convert their document to PDF\@. Most
% of the latest versions of Word have the facility to do this
% automatically. Submissions will not be accepted in Word format or any
% format other than PDF\@. Really. We're not joking. Don't send Word.

% Those who use \textbf{\LaTeX} should avoid including Type-3 fonts.
% Those using \texttt{latex} and \texttt{dvips} may need the following
% two commands:

% {\footnotesize
% \begin{verbatim}
% dvips -Ppdf -tletter -G0 -o paper.ps paper.dvi
% ps2pdf paper.ps
% \end{verbatim}}
% It is a zero following the ``-G'', which tells dvips to use
% the config.pdf file. Newer \TeX\ distributions don't always need this
% option.

% Using \texttt{pdflatex} rather than \texttt{latex}, often gives better
% results. This program avoids the Type-3 font problem, and supports more
% advanced features in the \texttt{microtype} package.

% \textbf{Graphics files} should be a reasonable size, and included from
% an appropriate format. Use vector formats (.eps/.pdf) for plots,
% lossless bitmap formats (.png) for raster graphics with sharp lines, and
% jpeg for photo-like images.

% The style file uses the \texttt{hyperref} package to make clickable
% links in documents. If this causes problems for you, add
% \texttt{nohyperref} as one of the options to the \texttt{icml2023}
% usepackage statement.


% \subsection{Submitting Final Camera-Ready Copy}

% The final versions of papers accepted for publication should follow the
% same format and naming convention as initial submissions, except that
% author information (names and affiliations) should be given. See
% \cref{final author} for formatting instructions.

% The footnote, ``Preliminary work. Under review by the International
% Conference on Machine Learning (ICML). Do not distribute.'' must be
% modified to ``\textit{Proceedings of the
% $\mathit{40}^{th}$ International Conference on Machine Learning},
% Honolulu, Hawaii, USA, PMLR 202, 2023.
% Copyright 2023 by the author(s).''

% For those using the \textbf{\LaTeX} style file, this change (and others) is
% handled automatically by simply changing
% $\mathtt{\backslash usepackage\{icml2023\}}$ to
% $$\mathtt{\backslash usepackage[accepted]\{icml2023\}}$$
% Authors using \textbf{Word} must edit the
% footnote on the first page of the document themselves.

% Camera-ready copies should have the title of the paper as running head
% on each page except the first one. The running title consists of a
% single line centered above a horizontal rule which is $1$~point thick.
% The running head should be centered, bold and in $9$~point type. The
% rule should be $10$~points above the main text. For those using the
% \textbf{\LaTeX} style file, the original title is automatically set as running
% head using the \texttt{fancyhdr} package which is included in the ICML
% 2023 style file package. In case that the original title exceeds the
% size restrictions, a shorter form can be supplied by using

% \verb|\icmltitlerunning{...}|

% just before $\mathtt{\backslash begin\{document\}}$.
% Authors using \textbf{Word} must edit the header of the document themselves.