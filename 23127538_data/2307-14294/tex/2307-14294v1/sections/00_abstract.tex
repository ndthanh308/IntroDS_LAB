\begin{abstract}
\vspace{-.05cm}
% \todo[inline]{Kürzer fassen und anpassen}

Splitting of sequential data, such as videos and time series, is an essential step in various data analysis tasks, including object tracking and anomaly detection. However, splitting sequential data presents a variety of challenges that can impact the accuracy and reliability of subsequent analyses. This concept article examines the challenges associated with splitting sequential data, including data acquisition, data representation, split ratio selection, setting up quality criteria, and choosing suitable selection strategies. We explore these challenges through two real-world examples: motor test benches and particle tracking in liquids.

%uneven segment lengths, varying frame rates, data noise and variability, subjectivity in defining segments, limited labeled data, complex interactions between segments, and high computational requirements. Understanding and addressing these challenges is critical for accurately analyzing sequential data and obtaining meaningful insights. By exploring these challenges, this article aims to provide insights into the complexities of splitting sequential data and highlight potential solutions for overcoming these challenges.

% We illustrate this by using two real-world examples from motor test benches and particle tracking in liquids.


\end{abstract}

