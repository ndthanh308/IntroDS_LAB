Image outlier detection (OD) is an essential tool to ensure the quality of images used in computer vision tasks. 
Existing algorithms often involve training a model to represent the inlier distribution, and outliers are determined by some deviation measure.
Although existing methods proved effective when trained on strictly inlier samples, their performance remains questionable when undesired outliers are included during training.
As a result of this limitation, it is necessary to carefully examine the data when developing OD models for new domains.
In this work, we present a novel image OD algorithm called RANSAC-NN that eliminates the need of data examination and model training altogether.
Unlike existing approaches, RANSAC-NN can be directly applied on datasets containing outliers by sampling and comparing subsets of the data.
Our algorithm maintains favorable performance compared to existing methods on a range of benchmarks. 
Furthermore, we show that RANSAC-NN can enhance the robustness of existing methods by incorporating our algorithm as part of the data preparation process.

% explore the use of improving existing methods 


% applicability of RANSAC-NN by improving the performance of existing methods under influenced by outliers during training. 


% using RANSAC-NN.

% that avoids model training altogether. RANSAC-NN is applied during   
% , and we observe a noticeable drop in performance across existing algorithms when outliers are present during model training. 
% In addition, we present an unsupervised image OD algorithm called RANSAC-NN that can perform OD without any training. 

% on existing OD algorithms, and we notice deteriorating performance 
% To avoid xxx, we propose an unsupervised OD algorithm applied difrectly 


% , and we observe deteriorating performance amongst existing OD algorithms when outliers are present during model training. 

% caused by outliers when training existing OD algorithms. 
% Furthermore, we propose a


% A challenging problem faced by these algorithms are when the training set consi


% The performance of these algorithms are largely influenced by the images within the training set. 

% using images sampled from the distribution before outlier prediction. 
% The resulting performance is therefore dependent on the images within the sampled set.
% Previous works often assumed a strictly inlier set for model training, but
% In this work, we explore the impact of outlier, and we demonstrate the 

% Previous methods often assume the sampled images to be strictly inliers, but in this work, we explore the 


% In previous works, the assumption is that the sampled set are strictly inliers, but 


% % a sampled set of in-distribution data prior to inference. 



% , causing the resulting performance to 

% and the resulting performance is highly dependent on the quality and quantity of the given in-distribution training set.
% However, to preparing such a .. cumbersome process, and existing methods do not address such a challenge. 




% Existing image OD algorithms perform optimally when trained on a set of in-distribution images, but their performance deteriorates when outliers are present in the training set. 



% are highly dependent on the dataset for which they have been trained with. 
% When trained on a 

% and they often yield promising performance when trained on a large set of in-distribution images. 



% However, the presence of outliers in the supposedly in-distribution training set 

% highly dependent

% the performance of the 


% obtaining an in-distribution set requires considerable effort. This limits the applicability of existing methods 

% considerable effort is required to obtain such a set of in-distribution images

% to obtain such a set of in-distribution image requires considerable effort, which inherintly limits the applicability 

% limiting the  




% real-world use cases 

% Existing approaches often depend on a  

% often require a readily available set of in-distribution images to train an OD model. 
% The OD performance of the model is 
% which is then 

% % The performance of the algorithm can depend on the:
% % - data quality
% % - data quantity
% % - 

% % Present algorithm that is :
% % - not dependent on 

% consider the train-and-deploy approach when dealing with outlier images. Specifically, a model is first trained on a set of in-distribution images before being deployed on 


% existing approach use two stage approach: train and deploy.


% Most existing approaches, however, require a set of in-distribution images to train an OD model for outlier prediction. The quality and quantity of the image set can determine the resulting performance. Thus, selecting a suitable in-distribution set often requires considerable effort.
% In this work, we propose RANSAC-NN, an unsupervised image OD algorithm designed to detect outliers within contaminated sets in a one-class classification fashion. Without any training, RANSAC-NN performs favorably in comparison to other well-established methods in a variety of OD benchmarks. Furthermore, we show that our method can enhance the robustness of existing OD methods by simply applying RANSAC-NN during pre-processing.