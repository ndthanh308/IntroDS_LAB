\section{Conclusions and Future Work}
\label{sec:conclusions}

This paper explores the problem of outlier contamination during the development of image OD models. We discuss the strengths and weaknesses of existing algorithms by comparing their performance when trained on different datasets. In our experiments, we observe solid performance in all existing methods when trained on clean inlier images. The performance of these algorithms, however, declines when outliers images contaminate the training data. This issue limits the use of existing OD methods on new image domains as datasets must be carefully examined before undergoing training.

This paper also introduces an alternate image OD algorithm called RANSAC-NN that avoids the need of data preparation or model training. Unlike existing methods, whose objective is to train a task-specific OD model for inference, the goal of RANSAC-NN is to detect outlier in contaminated datasets using the data distribution alone. 
% It does so by sampling the dataset in a two-stage approach, and the output is an outlier score for each image quantifying is likelihood of being an outlier image. 
Despite the lack of training, RANSAC-NN remains competitive against existing methods on a range of benchmarks. Furthermore, we demonstrate the use of RANSAC-NN to increase the robustness of existing OD methods when training on data containing potential outliers. Additional ablation studies on the components of RANSAC-NN have also been explored. 

% RANSAC-NN samples the dataset 


% , an image OD algorithm that is robust in detecting outliers within contaminated image sets. Unlike past methods that require training a model on a clean dataset, our algorithm does not require data preparation or training. Yet it maintains a competitive performance compared to existing OD methods across a range of benchmarks.

% We explored the strengths and weaknesses of existing image OD algorithms by comparing their performance when trained on different data combinations. When existing methods are trained on a clean inlier set, we observe their effectiveness as almost every method achieves near-perfect performance in our benchmark. However, when existing methods are trained on a contaminated set, their performance declines significantly. This challenge limits the applicability of existing methods on new domains as the training data must be carefully examined before model training takes place. 

% We highlighted the importance of training OD models on a clean inlier set by demonstrating the deteriorative influence of contaminated training. We then explored the use of RANSAC-NN an an outlier removal mechanism. We validated its effectiveness on existing OD algorithms by improving their performance in the contamination setting. To understand why our algorithm works well with contaminated sets, we conducted a detailed analysis of the different properties of the RANSAC-NN components. 

Future works may include extending RANSAC-NN into other domains such as text or audio. Possible applications in image mislabeled detection may also be explored (see Appendix). With the findings thus far, we hope RANSAC-NN may be a valuable tool for future image OD applications.

% We believe our algorithm is an ideal choice image data cleaning tasks, and it has the potential to be paired with existing OD algorithms to form a fully-automated image OD pipeline. 