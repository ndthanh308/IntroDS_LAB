\label{sec:related_work}

Most works in the field of computer vision addressing image outliers fall under the categories of anomaly detection, open-set recognition, and out-of-distribution detection. In contrast to the algorithms developed for OD, methods under such categories require a model to be trained on a set of inlier images before applying them to detect outliers \cite{zero_shot_ood, exploring_limits_ood, unified_framework, energy_based_ood, anomaly_detection_zero_shot_outlier, deep_one_class_classification, iForest_on_NN}. These methods consider the setting where a model has been deployed, and the objective is to flag inputs to the model that deviate from the training distribution. This is a different problem from OD, which does not assume the availability of an inlier set to begin with. 
As mentioned in a comprehensive survey by Yang \textit{et al.} \cite{ood_survey}, OD methods in the field of computer vision are rarely seen in recent deep learning venues. In the following, we list the few methods that have been found to address outliers in image data. 

KNN-Distance by Kuan \textit{et al.} \cite{knn-distance-algo} is an algorithm that computes the likelihood of each image being an outlier by calculating the cosine distance between each image and its $k$ nearest neighbors. Although it was initially proposed as an out-of-distribution detection method, it has been later shown to be effective in image OD as well \cite{knn-distance-blogpost}. 

Luan \textit{et al.} \cite{iForest_on_NN} demonstrated an approach that applies existing OD methods for tabular data onto the image domain. Specifically, the authors extract representations from a neural network for each input image and apply Isolation Forest \cite{isolation_forest} or LOF \cite{local_outlier_factor} on the set of extracted representations. As reported in their paper, this approach demonstrated good results for image OD as an attempt to bridge the gap between tabular and image data. In this work, we consider additional OD algorithms \cite{algo-inne, algo-lunar, loda} designed for tabular data, and apply them in a similar manner during our evaluation. 

The last method we considered is the Outlier Gallery function from FastDup \cite{algo-FastDup}. FastDup is a publicly available tool for obtaining insight analysis on computer vision datasets. Its Outlier Gallery function lists potential outlier images in a given dataset ranked by an internal algorithm. The number of presented outliers can be adjusted by an \verb|outlier_percentile| parameter, which controls the sensitivity threshold of the ranking mechanism. The underlying algorithm, however, has not been disclosed, limiting our discussion to only the results from running FastDup.
