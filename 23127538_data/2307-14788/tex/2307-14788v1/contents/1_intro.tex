\section{INTRODUCTION}
\label{sec:intro}
% 1) the hook
% 2) the gap
% 3) our study
% 4) contribution and conclusion
% 5) outline 

For a wide range of applications, an accurate estimation of the future position of moving agents is paramount. 
In the transportation sector, several branches benefit from effective forecasting systems such as (1) flexible, reliable and 
robust road traffic network relies on the prediction of traffic to prevent congestion and provide safety and efficiency~\cite{yu22,li23}; 
(2) Advanced Driver Assistance Systems (ADAS) capable of foreseeing the future can then adapt accordingly~\cite{singh22}; (3) the task of 
transporting goods may resort to a simulation of the future, which foster more informed planning and resource allocation~\cite{dalgkitsis21}. 
Further, trajectory prediction enhances situational awareness 
in complex dynamic environments such as industrial facilities, public sites, or home environments where multiple agents are moving. Finally, multimodal probabilistic forecasting of trajectories yields a richer 
representation of possible future events, which triggers subsequent decision-making systems~\cite{dahl23}.

The trajectory forecasting task mainly underlies three pillars such as agents interactions modeling~\cite{kosaraju19,sadeghian19,tang22,su22,minoura23,kothari23},
semantic context learning~\cite{xia22,kress23}, and multimodal predictions~\cite{deo22,bae22,chen22-itsc}. Also, multimodal approaches brought
a new view to the trajectory prediction domain as accounting for a set of diverse behaviors, instead of a unique point estimate, offers a much more
reasonable and safe solution~\cite{gupta18}. The problem with previous approaches is that they do not 
provide probabilistic or score measures for each trajectory estimate. In these cases, depending on the model's variability (induced usually by the
\emph{k-variety loss}~\cite{gupta18}), the predictions can be considered accurate or not~\cite{kothari23}. 
Most works following previous benchmarks~\cite{kothari21} opt by sampling $K$ trajectories (usually, $K=20$) and evaluating the closest to the ground truth. 
The premise that the closest trajectory to the ground truth can give a reasonable overview of the forecaster's performance is ambiguous as the model's 
variability plays a major role in the reported results~\cite{kothari23}. Therefore, the research community urges probabilistic and accurate predictions with 
adequate results reporting. In addition, context-agnostic 
approaches have had little attention from the research community~\cite{becker18,scholler20,giuliari20}, which we see as a gap to explore novel strategies 
for adding trajectory-based information to improve current context-free methods.

% Figure environment removed

Following recent works~\cite{sun21,chen22,miao22}, our method encompasses multiple stages where the final objective is to provide a distribution of trajectories and the respective probabilities. To that end, our 
system sequentially clusters the input data, trains a conditional Deep Generative Model (cDGM) to map the input data and clusters' ids to the 
respective future trajectory, and assigns likelihoods to the trajectories in a post-hoc fashion (see Fig.~\ref{fig:system_overview}). The clusters ids 
aim to represent akin similar behaviors that drive the final predictions whereas the ranking proposals methods allow us to assign probabilities to each 
generated trajectory. In this way, we can provide diverse and probabilistic accurate predictions.
Furthermore, inspired by~\cite{kothari23}, we consider Top-3 scores for the assessment and comparison of methods. Previous works have considered the Top-20 
trajectories, which we agree can mislead the interpretation of the results.

In summary, this work aims to improve standalone context-free Deep Generative Models by adding trajectory-related information encoded in the conditioning cluster class.
Inspired by~\cite{liu20,tiago22}, our first contribution is a novel deep generative clustering algorithm, Full Path Self-Conditioned GAN (FP SC-GAN). 
Besides clustering the input data, this framework generates complete sets of displacements that can be part of a downstream data augmentation process.
However, in this work, we solely focus on the clustering capability of FP SC-GAN. 
Alternatively to~\cite{sun21,chen22,miao22}, 
our second contribution is the distance-based ranking proposals methods, which do not rely on training an auxiliary neural network but are still rather effective
and efficient.
The proposed distance-based ranking proposals methods uniquely require access to the clustering space. In this work, these methods perform likewise to an auxiliary deep neural network, and 
run in \emph{linear time} whereas a Multilayer Perceptron (MLP) requires \emph{quadratic time}. 
Finally, to the best of our knowledge, we propose the first quantitative metrics report where the likelihoods provided by the system directly affect the Top-3 trajectory prediction
metrics. 

