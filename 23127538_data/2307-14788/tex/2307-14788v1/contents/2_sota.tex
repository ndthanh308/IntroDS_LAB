\section{RELATED WORK}
\label{sec:related_work}

% first approaches for trajectory forecasting
% probabilistic trajectory forecasting
% clusters based trajectory forecasting
% prob trajectory generation


Trajectory forecasting has been predominantly researched on two fronts: context-agnostic and context-aware approaches. 
Context-agnostic methods solely forecast based on observed trajectory patterns, whereas context-aware methods include social and scene layout cues. 
While context-agnostic approaches have received little attention~\cite{becker18,scholler20,giuliari20} from the research community, context-aware 
methods have been comprehensively investigated~\cite{gupta18,kosaraju19,dendorfer20,sun20,salzmann20,kothari23}. 
This paper focuses on context-agnostic approaches and breeding mechanisms for probabilistic trajectory forecasting based on encoded clustering 
information.


In the trajectory generation domain, some works tackled the problem from a probabilistic 
view in road scenarios~\cite{phan-minh20,ma21,ivanovic22,calem22} and human trajectory data~\cite{sun21,chen22,miao22}.
CoverNet~\cite{phan-minh20} comprises a Convolutional Neural Network (CNN) to extract contextual features from a road scene 
and a trajectory generator module to produce a set of possible predictions. Then, the system directly classifies the set of 
plausible trajectories yielding the score of each prediction. Conversely, we aim to produce context-agnostic samples to open 
the domain of applications of our system and reduce its requirements.
In~\cite{ma21}, the authors propose a post-hoc method named Likelihood-Based Diverse Sampling (LDS). In that paper, a novel objective function and a non-i.i.d sampling method encourage diverse predictions by suppressing similar predictions from the generated set of samples and leveraging the likelihoods from a pre-trained generative model. However, this work does not determine the scores (likelihoods) of the set of plausible trajectories, which we claim is paramount for risk-aware downstream
decision processes. 
\emph{HAICU}~\cite{ivanovic22} is a system that relies on perception and classification modules to give the 
class distribution of road agents. This system stands for a Conditional Variational Autoencoder (cVAE) conditioned on the class distribution and the observed trajectory of 
road agents to produce multimodal predictions. Therefore, this work heavily relies on upstream supervised methods to yield the class distribution, which is not easily generalizable to human trajectory data. To cope with this limitation, we propose unsupervised techniques to cluster akin trajectories, which we consider agnostic to the trajectory domain and more generalizable. 
In~\cite{calem22}, the authors enforce underlying physical admissibility constraints and diversity in a post-hoc trajectory sampling process 
based on a determinantal point process (DDP). Although it proposes a robust strategy for considering context using admissibility
constraints induced in the objective function, it does not provide a probabilistic view of the predicted trajectories.
\cite{sun21} devises a three-step method based on clustering, classification, and synthesis to predict. Contrarily to this work,
we first predict by using a cDGM and then propose a ranking proposals step. In this work, we investigate ranking proposals methods based on the distance 
to the already conceived clustering space and, therefore, not relying on a classification network. Further, \cite{chen22} 
proposes a method based on clustered goal points and a final classification step. While both \cite{sun21} and \cite{chen22}, 
at the classification step, learn the mapping between the past trajectory and the cluster class, we rank complete trajectories emphasizing the generated {\em tracklets}. Finally, in \cite{miao22}, 
the authors investigated a system with two branches: a motion pattern selector 
and a multimodal trajectory generator. The former produces a \emph{gallery} of diverse motion patterns, while the latter refines them and 
generates future trajectories. Then, a scoring method produces the most diverse predictions.


Our work encompasses mechanisms under the same umbrella as~\cite{sun21,chen22,miao22}, but our main objective is to propose a system that can 
improve current deep generative models by including information from clustered data. In our system
the clusters drive the multimodality, and the ranking proposals methods run on the generated trajectories. 
In addition, our distance-based ranking 
proposals methods does not rely on training any auxiliary neural network conversely to previous works~\cite{sun21,chen22,miao22}. In our case,
the ranking proposals step depends exclusively on the intrinsic nature of the clustering space and a similarity measure to the generated trajectories. Further, 
we propose a new deep clustering method inspired by a self-supervised deep generative model developed for the image generation task~\cite{liu20}. 
Finally, contrarily to previous works, the probabilities of each trajectory strongly affect the evaluation of our method, while in previous works, 
the probabilities only give a sense of the likelihood of each particular event.