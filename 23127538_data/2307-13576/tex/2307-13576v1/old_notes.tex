\documentclass{article}
\usepackage{fullpage} % Package to use full page
\usepackage{parskip} % Package to tweak paragraph skipping
\usepackage{tikz} % Package for drawing
\usepackage{graphicx}% Include figure files
\usepackage{amsmath, amssymb}
\usepackage{physics}
\usepackage{bbold}
\usepackage{braket}
\usepackage{xcolor}
\usepackage{wrapfig}
\usepackage{dcolumn}%
\usepackage[thinc]{esdiff}
\usepackage{subcaption}
\usepackage{bm}
\usepackage[export]{adjustbox}
\usepackage{hyperref}
 \usepackage{listings}
\usepackage{wrapfig}
\bibliographystyle{plain}
\usepackage[font=small,labelfont=bf]{caption}
\usepackage[ruled,vlined]{algorithm2e}
\usepackage[utf8]{inputenc}
\usepackage[margin=0.5in]{geometry}
\newcommand{\comment}[1]{} 
\newcommand{\dpc}[1]{\textcolor{blue}{(DP: #1)}}
\newcommand{\rebecca}[1]{\textcolor{red}{(rebecca: #1)}} 
\newcommand{\highlight}[1]{%
  \colorbox{yellow!50}{$\displaystyle#1$}}
\title{plots}
\author{rebeccaerbanni }
\date{June 2022}

\begin{document}

\maketitle
\section{Introduce briefly each section}
\section{note}
When the angle=pi/2, cos=0, which corresponds to max collisions. When angle$ \sim 0$,cos$\sim$1 and the partial swap is the identity, there are little to no collisions.
\section{Plots rectification, dt=0.001}

% Figure environment removed



% Figure environment removed



% Figure environment removed

% Figure environment removed



% Figure environment removed


% Figure environment removed


\newpage
d
\newpage
\section{Plots rectification, dt=0.01}

% Figure environment removed

% Figure environment removed


% Figure environment removed



% Figure environment removed
\newpage
\section{T=0.2, dt=0.1,0.2}
Here we consider a one and two-step trotterized evolutions and the setting with J=1, $h=\Delta=4$, dt=0.1,0.2, T=0.2. \\
We can see the minimum number of steps required to have an insulator in one direction is 2.\\

% Figure environment removed

% Figure environment removed
\newpage
\subsection{Note}
\comment{
\rebecca{T=0.2 is a good guess cause it gives a high R while not needing too many collisions (which is the case for instance for T=0.05).
We could also increase T and check the minimum number of steps required to have insulating behaviour in one direction. For T=0.2, 2 steps are enough, for T=0.4, 4 steps are needed, and the number increases with increasing T. So another plot could be, we pick a range of small T (to stay in regime of instantaneous collisions) and plot min number of steps required to obtain insulator in forward direction (although be careful when plotting R because for T=0.4, dt=0.15, n=3 steps, we get a high R but it's just because $I_f=10^{-4}$ and $I_r=10^{-2}$ )} }

\newpage
\section{Transport without rectification,baths at 0.5 and -0.5}
In the next two subsections we use baths at 0.5 and -0.5, hz = 0 and $\Delta$ = 1.5 and 0.5.
\subsection{Transport without rectification, $\Delta=1.5$}
% Figure environment removed


% Figure environment removed




% Figure environment removed




% Figure environment removed

\newpage
\subsection{Transport without rectification, $\Delta=0.5$}


% Figure environment removed


% Figure environment removed





% Figure environment removed




% Figure environment removed


\newpage
d\newpage
\section{Transport without rectification,baths at 1 and -1}
\subsection{Transport without rectification, $\Delta=1.5$}
In the next two subsections we use baths at 1 and -1, hz = 0 and $\Delta$ = 1.5 .
% Figure environment removed


% Figure environment removed




% Figure environment removed
\newpage
\section{Comparison with Lindblad ME}
We consider the following set-up: baths at $\pm 0.5$, $\Delta=1.5$, hz=0,J=1 and plot:
\begin{itemize}
    \item the magnetization profiles obtained with a collision model for different values of the collision time/evolution time, and from a NESS derived by the LME
    \item 
\end{itemize}
% Figure environment removed



% Figure environment removed


\end{document}