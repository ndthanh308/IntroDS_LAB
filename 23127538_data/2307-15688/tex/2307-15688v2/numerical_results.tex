

\section{Numerical results}
While the SoS proofs in the previous section only cover a very small fraction of possible uniformly weighted graphs, the SDP algorithm actually solves surprisingly many graphs exactly, in the sense that the obtained upper bound value matches the exact maximum eigenvalue. This is true for both the $\mathcal{Pauli}$ and $\mathcal{Proj}$ SDP relaxations, and in this section we will go through the numerical results showing this. 
We further observe that the SDP algorithm can be used for calculating expectation values of operators that are of physical interest. 
This is demonstrated in section \ref{subsec:HeisenbergChain}, where the $NPA_1(\mathcal{Proj})$ is applied to the Heisenberg Chain up to size $L=60$, and the critical correlation functions show the correct criticality up to error bars. 

In the following of this section, the term ``solve exactly" means that the upper bound value obtained by SDP theoretically matches exactly with the maximum eigenvalue.


\subsection{Exhaustive numerical results on small graphs}


Here, we show the results of $NPA_2(\mathcal{Pauli})$ applied to all possible uniform graphs up to $n=8$ vertices. 
The main observation is that $NPA_2(\mathcal{Pauli})$ is exact for many graphs with $n\leq 6$ vertices. While the percentage of such graphs seems to shrink as we go to larger system sizes, it suggests that there are many cases where an exact SoS exists that are not covered in the previous section. 

\subsubsection{Probing exact solvability numerically}

Before presenting the main numerical results, here we address the subtle issue arising from numerical precision of the SDP algorithm. 
It is fundamentally impossible to determine whether the SDP algorithm obtains the correct energy value for a particular Hamiltonian solely from numerical results. 
This is because the SDP algorithm always requires a precision parameter which is usually referred to as ``error-tolerance" $\epsilon$, and the algorithm only optimizes up to that $\epsilon$. 
Even if the algorithm seems to give very close values to the true energy we cannot {\it a priori} conclude if that is actually obtaining the exact solution, or if the error of the algorithm is merely small yet non-zero. 


% Figure environment removed

To address this issue systematically, we analyzed the optimal value (upper bound of the maximum eigenvalue) obtained by the SDP algorithm as a function of the error tolerance. 
More precisely, as shown in Fig. \ref{fig:Precision}, we plot the discrepancy of the SDP-obtained optimal value and the exact maximum eigenvalue $\Delta E :=|E_{\mathrm{SDP}} - E_{\mathrm{GS}}|$ as a function of $\epsilon^{-1}$. 
This plot, especially for $n=5$, shows a very clear dichotomy of $n=5$ connected graphs. While 7 graphs (red curves) have an almost constant $\Delta E >0$, the rest of the 14 graphs (blue curves) show a decay in $\Delta E$, roughly proportionally to $\epsilon$. 
This could be regarded as strong numerical evidence that the 14 graphs are exactly-solvable instances by the SDP algorithm while the 7 graphs are not. It is quite surprising that a simple five-vertex graph can naturally yield a very small error value around $0.00034$ (the graph shown in Fig. \ref{fig:Precision} with arrows in magenta). 

However, we must note that this method is not entirely decisive. 
As depicted in the center and right panels of Fig. \ref{fig:Precision}, the dichotomy becomes less clear as we go to larger sizes $n=6,7$, and is even worse for $n=8$ (not shown). 
This is because as we proceed to larger system size, an unweighted graph can potentially have extremely small error values $\Delta E$, such as $\sim 10^{-8}$ and even smaller. 
At some point, it practically becomes impossible, since smaller error tolerance $\epsilon$ requires longer iterations in the SDP optimization. 

We can also see that the theoretical error bound of $\Delta E < \epsilon$ (drawn in yellow lines in the figure) for any exactly-solvable graph, is not necessarily satisfied always. For example, although we rigorously prove that the star graph is exactly solvable by the SDP algorithm (see \S\ref{subsec:compbip}), the error of the star graph in Fig. \ref{fig:Precision}, $n=5$ (in cyan) is slightly above the error tolerance $\epsilon$. 
This arises from subtleties in how the error tolerances are handled inside the SDP package, and is difficult to control in general. 

Despite these subtleties, the behavior of the absolute energy error $\Delta E$ as a function of the error tolerance $\epsilon$ serves as a good rule of thumb for distinguishing exactly-solvable graphs from instances with merely small errors. 
For instance, we can be fairly confident that the graphs with magenta arrows indeed do have extremely small but non-zero errors such as $\sim 10^{-6}$.


\subsubsection{Exactly solvable small graphs and their statistics}\label{subsubsec:smallstat}

% Figure environment removed


Once we can confidently determine whether or not the SDP algorithm obtains the true ground state energy, we can start to ask questions such as ``When and how often does the SDP algorithm give us the exact solution?". 
To address this question, we present an exhaustive study for all connected graphs with $n=5,6,7$ and $8$ vertices. 


Figure \ref{fig:MetaGraphs} shows all of the 7 (out of 21) $n=5$ connected graphs and the 17 (out of 112) $n=6$ connected graphs that the $NPA_2(\mathcal{Pauli})$ SDP algorithm fails to obtain the exact ground state energy (colored in red/magenta). The numbers are labeling of the graphs according to a convention introduced in \cite{cve84ata}. 
It is rather surprising that the algorithm obtains the exact ground state energy for the vast majority of the graphs (colored in blue/cyan) up to this system size, noting that for most of the graphs the SoS is unknown and most likely very complicated (graphs in cyan). 

The figure also shows the topological relations of the graphs, by connecting them with a thick bond whenever two graphs only differ by one edge. 
In this way, we can see that for $n=5$ the red/magenta graphs (SDP fail) form one cluster. In other words, any two $n=5$ connected graphs that Lv. 2 Pauli SDP fails, can be transformed into one from the other by adding and subtracting one edge at a time, always maintaining the SDP algorithm to be failing. 
This is not the case for $n=6$, where the magenta graphs seem to form one big cluster and also three disconnected ``islands" (namely, graphs 20, 28, and 69). However, as we will see in the following, the ``single-clusteredness" of the hard graphs recovers once we focus on the errors from the $NPA_1(\mathcal{Proj})$. 

The ``failing cluster" includes the complete graph for $n=5$ but not for $n=6$. This is exactly as expected as we explained in \cref{subsec:complete}. This raises the question whether we can actually further constrain the SDP algorithm, not with a higher level, but simply by adding a constraint corresponding to the minimum total spin of the ground state. More specifically, the constraint would be 
\begin{equation}
    \sum_{1\leq i<j\leq n} M(\mathbb{I}, h_{ij})\leq \frac{(n+3)(n-1)}{8}, 
\end{equation}
from \cref{eq:xyz_sos_proof} for odd $n$. 
When we add this constraint, $NPA_1(\mathcal{Proj})$ not only was able to solve the complete graph $K_5$ exactly, but other graphs in the vicinity. 
This information is indicated in \cref{fig:MetaGraphs}, by showing graph 12 and 18 in red, being the only two graphs that $NPA_1(\mathcal{Proj})$ with this additional constraint still failed. 
Note that we cannot do the same thing when we have even number of qubits, because $NPA_1(\mathcal{Proj})$ already succeeds for the complete graphs, i.e., already {\it know} about this constraint on total spin. 



% Figure environment removed

We also compare the performance of the different SDP algorithms ($NPA_2(\mathcal{Pauli})$, $NPA_2^{\mathbb{R}}(\mathcal{Pauli})$, and $NPA_1(\mathcal{Proj})$) for all of these graphs up to $n=8$ in Fig. \ref{fig:PauliProjCompare}. 
The scatter plot shows the energy errors for $NPA_2(\mathcal{Pauli})$ and $NPA_1(\mathcal{Proj})$. The fact that the scattered points roughly forms four different clusters could be understood in the following way. 

Firstly, the cluster on the top right corresponds to graphs that the SDP algorithms with either bases fail to obtain the exact ground state. 
If we believe in typical hardness of the random {\scshape QMaxCut} instances, the ratio of graphs in this cluster in the scatter plot should reach 1 in the large problem size limit. 
The fact that all of the points in this cluster are on the left of the black line indicating $x=y$ reflects the fact that the $NPA_2(\mathcal{Pauli})$ SDP can never perform worse than the $NPA_1(\mathcal{Proj})$ SDP. This could be easily seen from the fact that you can always convert an SoS proof using degree-1 polynomials of projectors into SoS that uses degree-2 Pauli polynomials, but not necessarily the other way around. 


Whether the aforementioned inequality $NPA_2(\mathcal{Pauli}(H))\geq NPA_1(\mathcal{Proj}(H))$ is actually an equality or not for {\scshape QMaxCut} instances is  not obvious until we actually see examples. 
The second cluster on the top left of \cref{fig:PauliProjCompare} reflects exactly that there are indeed graphs where $NPA_2(\mathcal{Pauli})$ SDP is exact but $NPA_1(\mathcal{Proj})$ SDP fails, i.e., that the inequality is {\it strict} in general. 
We list up all the $n=5$ and 6 graphs that fall under this second cluster on the right side of \cref{fig:PauliProjCompare}. 
Furthermore, we also checked how $NPA_2^{\mathbb{R}}(\mathcal{Pauli})$ SDP performs on these graphs to find that the inequality $NPA_1(\mathcal{Proj}(H)) > NPA_2^\mathbb{R}(\mathcal{Pauli}(H)) > NPA_1 (\mathcal{Pauli}(H))$ is also strict in general\footnote{The nonstrict inequality could be quickly understood in the same manner as the argument in the previous paragraph}. 
Specifically, we find that $NPA_2^\mathbb{R}(\mathcal{Pauli})$ fails for all of  the graphs shaded in \cref{fig:PauliProjCompare}, while it succeeds for all of the other graphs with $n=5$ and 6.
This means that the exact Pauli SoS for unshaded graphs are ``breaking the SU(2) symmetry" in the individual squares possibly by having one-body Pauli terms in them. Those effects must cancel out as a whole when all the SoS terms are added since the final Hamiltonian has SU(2) symmetry and has no one-body terms. 
For the shaded graphs, this ``symmetry breaking" trick is not enough to obtain the exact SoS, and complex SoS are required to do so. 
As a concrete example, the graph labeled 8 in \cref{fig:PauliProjCompare} has errors $1.4\cdot{} 10^{-2}$, $5.7\cdot{}10^{-4}$ and $8.06\cdot{}10^{-12}$ for $NPA_1(\mathcal{Proj}(H))$, $NPA_1^\mathbb{R}(\mathcal{Pauli}(H))$ and $NPA_1 (\mathcal{Pauli}(H))$ respectively, which we interpret as the complex Pauli hierarchy being exact on this instance, but the real Pauli and complex projector hierarchy have nonzero errors.


The third cluster on the bottom left corresponds to graphs where the SDP algorithm succeeds with either of the bases. The ratio of the graphs in this third category seems to decrease as we get to larger sizes of graphs, which we will discuss further later. 
Noticing that the separation between $NPA_2(\mathcal{Pauli})$, $NPA_2^{\mathbb{R}}(\mathcal{Pauli})$, and $NPA_1(\mathcal{Proj})$ are strict in general from the previous paragraph, it seems more natural to regard this cluster as instances where $NPA_1(\mathcal{Proj}(H))=0$ forces the other two SDPs to have 0 error as well. 
From this perspective, it is more intriguing when $NPA_1(\mathcal{Proj}(H))=NPA_2(\mathcal{Pauli}(H))>0$, i.e., exactly on top of the $x=y$ line in \cref{fig:PauliProjCompare}, but in the top right cluster. 
Up to $n=8$ connected graphs we have computed, the only cases when that happens are all graphs related to complete graphs (simplest cases discussed in \cref{subsec:complete}).



There is a rather small fourth cluster on the right bottom, that extends beyond to the right side of the $x=y$ line. 
Since $NPA_2(\mathcal{Pauli})$ must always perform no worse than $NPA_1(\mathcal{Proj})$, this suggests a numerical error of some sort. We have observed that the SDP packages for these graphs do not converge as quickly as other graphs, and tends to give results that have larger duality gaps than specified. This practically does not become a problem since the errors are very small (around $10^{-6}$), and all graphs which we explicitly exemplify as ``NPA failing" in this work are not from this group\footnote{This may occur strange to the physicist readers that a convex optimization which theoretically does not have a local minimum, still seems to ``get stuck" in practice. This is actually not uncommon in the field of convex optimization, since e.g. a very narrow feasible region can cause practically slow convergences like this. \lunote{There's no local minimum in convex optimization problem, but it's uncommon that convex optimization problems have slow convergence towards the global minimum due to the shape of the convex cone. I think it's better to remove this sentence.} \jnote{How about in a footnote with a wording like this?}} .
Notably, instances falling on the right side of the $x=y$ line only occur at very small errors (bottom right), while none are observed in the top right cluster. This is 
encouraging, since we can be confident that these practically pathological cases only arise when we demand high numerical precision. This allows us to consider all of the graphs in the fourth cluster (bottom right) to be theoretically easy for both bases of SDP, i.e., actually belonging to the third cluster (bottom left).


%We obtained many interesting findings from the results, including a separation between real and complex hierarchies.  The graphs labeled $8$ and $15$  provide examples where the numerics suggest $NPA_1(\mathcal{Proj}(H)) > NPA_1^\mathbb{R}(\mathcal{Pauli}(H)) > NPA_1 (\mathcal{Pauli}(H))$ with {\it strict} inequalities.  Specifically the graph labeled $8$ obtains errors $1.4\cdot{} 10^{-2}$, $5.7\cdot{}10^{-4}$ and $8.06\cdot{}10^{-12}$ for $NPA_1(\mathcal{Proj}(H))$, $NPA_1^\mathbb{R}(\mathcal{Pauli}(H))$ and $NPA_1 (\mathcal{Pauli}(H))$ respectively.  So, while the complex Pauli hierarchy appears to be exact on this instance, the real Pauli and complex projector hierarchy have significant errors.  \jnote{I changed things a bit.}

% Figure environment removed


In order to see the statistics of the errors more closely, in \cref{fig:errorstats} (a), we show the values of the error for the $NPA_1(\mathcal{Proj})$ SDP in descending order for each size of graphs $n=5,6,7$ and $8$. 
The $x$-axis is rescaled so that the data of 21, 112, 853, and 11,117 graphs all fit into $[0,1]$. Thus, the figure is the inverse of the cumulative distribution function of errors. 

For example, all four curves display an acute decline at some point corresponding to the separation between graphs that have nonzero errors and (essentially) zero error. The graph shows that the ratio of such non-exactly solvable graphs are roughly $46\%, 37\%, 91\%$ and $94\%$ among all connected $n=5,6,7$ and $8$-vertex graphs respectively. This means that the ratio of exactly solvable graphs tend to decrease as the number of vertices increases, possibly converging to 0 in the $n\rightarrow \infty$ limit. 
%This is not too bad of a news for the SDP algorithm, since the number of connected graphs itself grows very fast. 
%\knote{This sentence is confusing me.  Do you mean that there are still a lot of graphs solved exactly via SDP? {\bf Jun}: Yes. I deleted it and changed things around.}
Yet still, the actual {\it number} of connected graphs that are exactly solvable seems to grow with $n$ at least for this size regime: 11, 67, 77, and 670, for $n=5,6,7$ and 8. 

Another piece of information in the graph, represented as the points in the figure, is how the {\it bipartite} graphs are distributed among this descending-error ordering. The {\scshape QMaxCut} problem on bipartite graphs is oftentimes described as having ``no geometric frustration" in condensed matter physics, since the singlet projector $h_{ij}$ could be seen as a constraint that favors the two qubits to be pointing in the opposite direction\footnote{Not to be confused with ``frustration-free" explained in section \ref{subsubsec:MG}.}. From this point of view, we would consider an odd-length loop as geometrically frustrated because the interaction would not be (even relatively) satisfied with a simple approach of having the qubits point the opposite directions alternately. 
This difference has practical applications, such as bipartite cases allowing the quantum Monte Carlo method to efficiently\footnote{Only known empirically, in terms of precise complexity theory statements. While the time complexity scaling is known to scale as $\mathcal{O}(\epsilon^{-2})$ with respect to the error tolerance $\epsilon$, the scaling with number of qubits $n$ is hard to bound rigorously for Markov-chain Monte Carlo methods in general, albeit cases of quantum Monte Carlo methods being applied to hundreds or thousands of qubits is common in computational physics \cite{san10com}.} obtain the ground state classically. 
Therefore, it is not so surprising that the bipartite graphs in Fig. \ref{fig:errorstats} (a) are distributed relatively on the right side of each curves, implying (exponentially) smaller errors. In some sense, the surprise is in the other direction, that SDP fails to obtain the exact ground states of such ``easily classically simulable" instances most of the time. 
It is unclear if the tendency of bipartite graphs having relatively small errors will remain for larger $n$, since it is already apparent that the position of the largest-error bipartite graph shifts to the left in Fig. \ref{fig:errorstats} (a) from $n=7$ to $n=8$.

%In the $n\rightarrow\infty$ limit, the uniform distribution among all connected graphs with $n$ vertices will be indistinguishable from a random Erd{\"o}s-R{\'e}nyi graph ensemble with ${n \choose 2}/2$ edges in total  \cite{rad67uni}.\knote{I don't understand this.  A vertex in a graph cant have degree ${n\choose 2}/2$ since this could be larger than the number of vertices in the graph.  Also you might need to say in what sense the graphs are indistinguishable.}\jnote{Thanks, that was a typo and I corrected the statement, hopefully making it clearer. But maybe we can just delete some sentences here?}
%This means that the value of the relative error will be tightly concentrated to a single value, making the curves in Fig. \ref{fig:errorstats} (a) to have a flat plateau covering the entire $(0,1)$. This explains the similarity of the two curves for $n=7$ and 8 that is already apparent due to the large number of graphs (853 and 11,117), despite $n$ itself being rather small.

In order to test the difference between bipartite graphs and non-bipartite graphs in a more systematic way, we also ran the SDP algorithm for random regular graphs with degree-3. When such graphs are generated uniformly randomly, for sufficiently large $n$, the graph is almost certainly non-bipartite. 
We generate 100 of such samples, and compare the performance of $NPA_1(\mathcal{Proj})$ 
against exact diagonalization for $n=12, 16$ and 24. 
It is also possible to generate uniformly random graphs that are bipartite and regular, and both results are displayed in Fig. \ref{fig:errorstats} (b). 
It is immediately apparent that the non-bipartite random regular graphs have a broader distribution in the two-dimensional scatter plot, compared to the bipartite cases. The cluster is also located farther away from the $x=y$ line in black, showing a larger relative error compared to bipartite random graphs. The bipartite random graph data also seem %agree well with a polynomial fit, 
 to form a ``line" in the scatter plot, 
indicating that the optimal SDP objective can give a fairly narrow estimate of the true energy value by a properly fitted linear function. In contrast, the non-bipartite random graph data extends in a two-dimensional manner forming a oval-like shape, resulting in broader estimates of the true energy given the SDP energy. 



\subsubsection{Transition points in the solvability of small graphs}

% Figure environment removed

The clusteredness of hard and easy graphs shown in Fig. \ref{fig:MetaGraphs} leads to the question of what happens at the boundary between them. 
If there is a pair of graphs which one is exactly solvable while the other is not, with only one edge difference as graphs, then we can add that one different edge with weight $x\in[0,1]$. 
This procedure continuously connects the graphs and demonstrates where exactly SDP starts to fail. 

In \cref{fig:PauliProjCompare}, we show three different cases of such a procedure. 
On each panel, we show the graph we use for demonstration, with the dotted edge being the weighted one. 
The left most panel shows the case for interpolating between the $n=4$ star graph and the Y-shaped $n=5$ graph (graph \# 20 in \cref{fig:PauliProjCompare} (a)), which is the easiest case of such. In this case, we can see that the moment we add $\epsilon>0$ amount of the new edge, SDP starts to fail. 
This could be argued that the solvability of the star graph in this situation is rather {\it fragile}, and immediately fails when perturbed away. 

The same thing could be argued for the case shown in the middle panel connecting graph \#69 and \#47 of \cref{fig:MetaGraphs} right. 
Again in this case, the moment the graph diverges away from the exactly solvable \#47, the SDP algorithm starts to fail. 
However, there exist cases where the ``transition" happens not at the edges but at a nontrivial value, as shown in the right panel. The error becomes as small as the duality gap set for the SDP solver for $x<0.22$. In this case, we can say that the solvability of graph \#47 is somewhat robust, and survives the perturbation in the direction considered here (towards graph \#28). 

Curiously, for all cases we have checked for interpolations between solvable and unsolvable graphs with $NPA_1(\mathcal{Proj})$, we always observe a quadratic initial increase of the error, as shown with the red dotted lines in \cref{fig:SmallTrans}. The quadratic fit is extremely good at the vicinity of the ``transition points" where the error starts to become nonzero, as shown in the insets of the figures. 
This resembles universal critical behavior seen in physics, where phase transition {\it points} vary largely depending on the details of the statistical physics model, but an indicator of the phase transition (called the order parameter) behaves as $\propto |T-T_{c}|^{\beta}$ with a universal exponent denoted by $\beta$. 
Although our observed exponent $\beta=2$ is clearly present numerically, we were unable to provide a general explanation, and leave it for future studies. 




\subsection{Numerical results for some condensed matter physics models}\label{subsec:cmp}

Here, we demonstrate the power of the SDP algorithm when applied to a number of condensed matter physics models. The message is two-fold: first, the SDP algorithm could be used to probe exact-solvability of models in some settings, giving rise to the possibility of numerical exploration for exactly-(analytically) solvable systems. Second, the method could be seen as the first-order approximation of the ground state, it actually gives very accurate numbers in practice, with errors only up to $\sim 4\%, 7\%, 2\%$ for the models we study. 

Both the Majumdar-Ghosh model and the Shastry-Sutherland model are known to be ``frustration-free" in the quantum spin system literature \cite{deb10sol,sat16whe,wou21int,ans22ana}. 
This means that the Hamiltonian could be rewritten as sum of terms that could all be satisfied simultaneously in the ground state. 
The standard way to show this is to rewrite the physical Hamiltonian (thus with the opposite sign from our {\scshape QMaxCut} convention as in Eq. (\ref{eq:QMCHamDef})) as a sum of projectors with additive and multiplicative constants. 
If there exists a state that the all the projectors evaluate to 0, that must be the ground state (note that the physics convention here is a minimization of the eigenvalue). 

Because all projectors are square of themselves, we can immediately obtain an SoS of some degree by flipping the entire sign, and redefining the Hamiltonian with the {\scshape QMaxCut} convention. 
In most cases in physics, the definition of ``frustration-free" requires the rewritten terms of the Hamiltonian to be spatially local. Thus, the SoS hierarchy could be seen as a generalization of the frustration-free notion, where we do not necessarily require spatial locality, but restrict the degree of the terms as polynomials.
The fact that the degree-restriction could be arbitrarily relaxed by the level of the hierarchy, and that SDP algorithms can solve the optimization problem efficiently for $\mathcal{O}(1)$ level, provides us a systematic approach to explore frustration-free Hamiltonians in a computational way. 

\subsubsection{The Majumdar-Ghosh model}\label{subsubsec:MG}

% Figure environment removed

The Majumdar-Ghosh (MG) model \cite{maj69nex} was one of the earliest proposed quantum spin models where the ground state could be obtained exactly. 
The Hamiltonian being considered is 
\begin{equation}\label{eq:MGHam}
    H = \sum_{i=1}^L \Bigl(J_1 h_{i,i+1} + J_2  h_{i,i+2}\Bigr), 
\end{equation}
where we use the {\scshape QMaxCut} convention (i.e. the ``ground state" we are searching for now is the maximum eigen state of this operator). The lattice structure is shown in Fig. \ref{fig:lattices}(a). 
The Hamiltonian above would typically be referred to as the $J_1$-$J_2$ Heisenberg chain in the context of condensed matter physics, and the MG model corresponds to the case where $J_2/J_1=1/2$, also known as the MG {\it point}. 
At the MG point, the ground state is two-fold degenerate with two different singlet-product states where closest neighbors are forming singlets periodically, as depicted in Fig. \ref{fig:lattices}(a): 
\begin{equation} 
    |\mathrm{GS}\rangle = 
    \prod_{i\in \mathrm{even}}^{\otimes}|s_{i,i+1}\rangle
    +\prod_{i\in \mathrm{odd}}^{\otimes}|s_{i,i+1}\rangle ,
\end{equation}
where by $|s_{i,j}\rangle$ we denote a singlet state between the spins on sites $i$ and $j$.

The exact ground state could be understood from the fact that the Hamiltonian at the MG point decomposes in the same sense of Eq. (\ref{eq:CBhamdecomp}). 
Specifically,
\begin{equation}
    H=\frac{J_1}{2}\sum_{i=1}^L \Bigl(h_{i,i+1} + h_{i+1,i+2} + h_{i,i+2}\Bigr), ~~~~
    \lVert H\rVert=\frac{J_1}{2}\sum_{i=1}^L \Big\lVert h_{i,i+1} + h_{i+1,i+2} + h_{i,i+2}\Big\rVert 
\end{equation}
holds. 
Since the individual terms after this decomposition reduces to a triangle with equal weights, we can reuse the exact $\SoS$ from Eq. (\ref{eq:WeakTriangleSoS}), obtaining 
\begin{equation}
    \frac{3L}{4}\mathbb{I} - H 
    = 
    \sum_{k=1}^{L}
    \frac{3J_1}{2}
    \biggl\{
        \mathbb{I}-\frac{2}{3}
        \bigl(
            h_{k-1,k} + h_{k,k+1} + h_{k-1,k+1}
        \bigr)
    \biggr\}^2
\end{equation}
for the periodic boundary condition ($L+i\equiv i$). 
This corresponds to the standard projector expression for frustration-free models as we mentioned earlier, and implies that $NPA_1(\mathcal{Proj})$ is able to obtain the exact ground state energy at the MG point. 
%Since the above SoS is constructed using only linear terms of the singlet projector operators, $NPA_1(\mathcal{Proj})$ should be able to obtain the exact ground state at the MG point. 

% Figure environment removed

This is demonstrated in Fig. \ref{fig:MG}, where we compare the exact energy $E_{\mathrm{GS}}$ and the SDP energy $E_{\mathrm{SDP}}$ for the $L=16$ case with periodic boundary condition. 
The fact that the SDP algorithm obtains the exact ground state energy is reflected as the two energy density values coinciding at the MG point in the left panel, and correspondingly in the center panel the relative error becomes 0. 
Interestingly, the SDP algorithm seems to be ``more sensitive" to the MG point in this $J_1$-$J_2$ model compared to the actual ground state energy, when we try to detect it by looking into the derivatives of the energy (right panel). 


The fact that in Fig. \ref{fig:MG} we see the SDP algorithm only obtaining the energy exactly at the MG point establishes that there are no other exactly-solvable points in the $J_1$-$J_2$ model, even if we allow non-local terms as long as they are limited to degree-2 in polynomials of singlet projectors. 




\subsubsection{The Shastry-Sutherland model}\label{subsubsec:SS}

% Figure environment removed

The Shastry-Sutherland (SS) model \cite{sha81exa} is a two-dimensional Heisenberg model that also admits an exact ground state representation for a certain parameter region. 
The Hamiltonian in the {\scshape QMaxCut} convention would read 
\begin{equation}
    H=J\sum_{\langle ij\rangle} h_{ij} + 2\alpha \sum_{\langle\hspace{-0.5mm}\langle ij\rangle\hspace{-0.5mm}\rangle} h_{ij}
\end{equation}
where $\langle ij\rangle$ represents bonds of the $L\times L$ square lattice (with periodic boundary condition) and $\langle\hspace{-1mm}\langle ij\rangle\hspace{-1mm}\rangle$ represents diagonal bonds of the Shastry-Sutherland lattice as illustrated in Fig. \ref{fig:lattices}(b). For simplicity, we fix $J=1$. 

This model has an obvious unique ground state when $\alpha$ is large enough, since the diagonal bonds with weight $2\alpha$ gives a perfect matching of the sites. In that parameter region, the unique ground state could be written as 
\begin{equation}\label{eq:SSGS}
    |\mathrm{GS}\rangle = 
    \prod_{\langle\hspace{-0.5mm}\langle ij\rangle\hspace{-0.5mm}\rangle}^{\otimes}|s_{i,j}\rangle , 
\end{equation}
again illustrated in Fig. \ref{fig:lattices}(b). 

For the SS model with $\alpha>1$, we are again able to decompose the Hamiltonian into triangles with weights 1, 1, and $\alpha$, as previously discussed in section \ref{subsubsec:crown}. 
It is easy to check that the Shastry Sutherland lattice (Fig. \ref{fig:lattices} (b)) can be decomposed into such triangles geometrically, with all triangles having two edges from the square lattice and one from the diagonal edges. 
Now, we can reuse Eq. (\ref{eq:StrongTriangleSoS}) to obtain 
\begin{eqnarray}\label{eq:SS-SOS}
    &&\left(\alpha+\frac{J}{2}\right)N\mathbb{I} - H \nonumber\\
    &=& 
    \sum_{\triangle}
    \left(\alpha+\frac{J}{2}\right)
    \biggl\{
    \mathbb{I}
    -\sum_{\mathrm{edges}\in\triangle}    \frac{4\alpha+2J\pm(-2)^{j}\sqrt{2\alpha(2\alpha-J)-2J^2}}{3J+6\alpha}h_{\mathrm{edge}}
    \biggr\}^2 , 
\end{eqnarray}
which gives the exact value only when $\alpha\geq J$. 
The summation $\sum_{\triangle}$ is taking the summation for all right triangles in the SS lattice as in the decomposition, and the summation inside of the square is for the three different edges for each such triangle. The $(-2)^j$ factor only appears for the edges with weight $J$ belonging to the square lattice where we set $j=1$, and not for the diagonal edges with weight $\alpha$ which we set $j=0$. Just as in Eq. (\ref{eq:StrongTriangleSoS}), the SoS has a degree of freedom in choosing $\pm$ for the square root term. 


In Fig. \ref{fig:SS}, we demonstrate the performance of the $NPA_1(\mathcal{Proj})$ SDP algorithm applied to the SS model with system size $n=L^2=16$ and $J=1$ fixed. We can see that the algorithm obtains the exact ground state energy for the entirety of the $\alpha\geq 1$ region, which exactly coincides where Eq.~(\ref{eq:SS-SOS}) gives a proper SoS (otherwise it has no real coefficients), and also the decomposition exists. 
The true ground state actually becomes the dimer singlet state \cref{eq:SSGS} from $\alpha\geq3/4$ for this system size, although the SDP algorithm fails to obtain that. 
This means that while $\alpha\geq 1$ was the condition used to show frustration-freeness in \cite{sha81exa}, relaxing the notion to allow non-local terms (but still only having degree-1 terms in the SoS) does not enlarge the region of exact-solvability. It would be interesting to see how the exactly solvable region changes as a function of the level of the NPA hierarchy. 

Furthermore, by looking at the first derivative of the SDP energy as a function of $\alpha$, we can clearly see that there are two points where $\partial E/\partial \alpha$ has a singularity (\cref{fig:SS} right), namely $\alpha \simeq 0.73$ and $\alpha =1$. 
The existence and the nature of different phases in the SS model is actively discussed in the condensed-matter physics context, where there is expected to be at least two phase transition points, i.e., singular points \cite{kog00qua,lee19sig,yan22qua}. 
The fact that the SDP energy derivative exhibit two singular points from relatively small system sizes suggest the possibility of this approach be used to detect phase transitions in similar models, without relying on comparison with exactly obtained ground states. The SDP algorithm also allows us to calculate observables other than energy such as the squared N{\'e}el order parameter from the moments, with a guarantee that they converge to the true value when the NPA hierarchy converges to the true ground state energy value. 
This lets us to interpret such physical observables obtained this way to be regarded as a first-order approximation. In the next section we see that even such approximated quantities can show essential characteristics in physical systems. 

Another thing to note is that both Hamiltonians for the SS model and the MG model allowed decomposition of the Hamiltonian as in \cref{eq:CBhamdecomp} and \cref{eq:stardecomp}. In the cases of SS and MG models, the sub-Hamiltonians were the triangles with $\alpha,J,J$ bonds and $J_1/2, J_1/2, J_2=J_1/2$ bonds respectively. 
However, it should be noted that the existence of such decomposition is not a necessary condition for obtaining exact SoSs. For example, for the crown graph which we present an exact SoS in section \ref{subsubsec:crown}, it appears that there are no such decomposition, while still having an exact SoS. 
The same could be said for complete graphs with even number of vertices. 
This fact gives us hope on discovering new exactly-solvable Hamiltonians, since oftentimes the search for frustration-free Hamiltonians relies on the existence of such decomposition \cite{gho23exa,kum02qua}. 

\subsubsection{The Heisenberg chain}\label{subsec:HeisenbergChain}
The nearest neighbor antiferromagnetic Heisenberg chain is one of the simplest yet nontrivial quantum spin system that also happens to be a {\scshape QMaxCut} instance. 
The Hamiltonian we consider here is simply the chain 
\begin{equation}
H=\sum_{i=1}^L h_{i,i+1}, 
\end{equation}
with a periodic boundary condition $L+k \equiv k$. 
This corresponds to setting $J_2=0$ for the $J_1$-$J_2$ model in section \ref{subsubsec:MG}. 
Although the Heisenberg chain has an exact solution thanks to the Bethe ansatz \cite{bet31zur}, the exact solvability of the model is quite different from the previous two models: it does not involve frustration-freeness, and our SDP algorithm is therefore not expected to solve it exactly. 

% Figure environment removed

In the left panel of \cref{fig:Cycle}, we show our numerical results on how the SDP algorithm performs on the Heisenberg chain, by comparing the $NPA_1(\mathcal{Proj})$, $NPA_2(\mathcal{Pauli})$, and the exact value for various system sizes. 
We plot the energy {\it density} $E/L$ here, so the fact that all three cases converge to different values indicate that the absolute error of the total energy increases linearly with the system size $L$ for large enough $L$. Still, the qualitative behavior of approaching the limiting value from above and below for even and odd $L$ is reproduced in both of the SDP methods. 

In the right panel, we show the correlation function 
\begin{equation}\label{eq:corr}
    C(r) := \langle \mathrm{GS} | (-1)^r Z_i Z_{i+r} |\mathrm{GS}\rangle = (-1)^r\frac{1-4 ~M(\mathbb{I}, h_{i,i+r})}{3}, 
\end{equation}
obtained by $NPA_1(\mathcal{Proj})$ and Monte Carlo (virtually exact value) for system sizes $L=18, 28,$ and $60$. 
Note that the translation symmetry of the cycle ensures the well-definedness of $C(r)$ regarding the choice of $i$ in the definition. 
The second equality in \cref{eq:corr} is valid only in the case where the SDP algorithm obtains the exact ground state. However, we can still measure the RHS quantity even in cases where the algorithm fails and consider the obtained result as an approximation.
Remarkably, when utilizing the SDP-obtained correlation function in this manner, it aligns very well with the true correlation function for small values of $r$ as shown in the figure. This can be attributed to the fact that the energy density exhibits a relative error of only $\sim2\%$.

One characteristic feature of the Heisenberg chain is that the ground state displays a power-law decaying correlation with critical exponent $\eta =1$, i.e., $C(r)\propto r^{-1}$, which is closely linked to the long-range entanglement it has. 
The fact that the SDP-obtained correlation function displays the same type of power-law decay with essentially the correct exponent (albeit the jagged feature) is quite interesting especially when it is compared to the exact correlation function of finite systems, since it appears to have even smaller finite-size corrections. 
This raises the intriguing possibility that SDP-derived quantities capture the underlying ``physics" of the ground state, even when there is no physical quantum state corresponding to the optimal moment matrix. 

Finally, an alert reader may notice from the figure that both $NPA_1(\mathcal{Proj})$ and $NPA_2(\mathcal{Pauli})$ are exact for the $L=6$ hexagon case. 
Although we were unable to obtain an analytic $\SoS$, we were able to study the structure of the ground state from the SDP perspective, which we provide in Appendix \ref{app:hexagon}. 
