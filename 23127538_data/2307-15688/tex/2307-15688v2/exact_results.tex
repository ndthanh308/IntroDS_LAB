
\section{Exact results on some families of graphs}\label{sec:exact}

Here we detail our results concerning the exactness/inexactness of $NPA_1(\mathcal{Proj})/SoS_1(\mathcal{Proj})$ on many interesting classes of {\scshape QMaxCut} Hamiltonians.  
%We focus on $NPA_1(\mathcal{Proj})/SoS_1(\mathcal{Proj})$ and demonstrate several proofs which show exactness, as well as some considerations from symmetry which prove inexactness for some family of graphs. %complete graphs with odd number of vertices. 
First of these classes is the positive weighted star graphs. The proof technique for this class involves reconstructing a quantum state with the exact same energy as the output of the $NPA_1(\mathcal{Proj})$ program.  A crucial component of this proof is a reinterpretation of ``monogamy of entanglement'' inequalities in terms of the possible angles for Gram vectors from $NPA_1(\mathcal{Proj})$.  
We show the constraints on these angles from the polynomial inequalities derived in \cite{par22opt} are actually saturated for the case of star graphs. This provides an interesting geometric perspective for monogamy of entanglement in the context of $NPA_1(\mathcal{Proj})$.%, and how the constraint from the monogamy essentially determines the ground state for the star graph case. 

The other proofs for exactness relies on SoS proofs, which we analytically construct. Since the SDP hierarchies defined in \Cref{sec:hierarchy} are relaxations of the Local Hamiltonian problem, it is sufficient to construct a feasible solution to $SoS_\ell$ which achieves the optimal eigenvalue as the objective. 
To state concretely, we will be utilizing the following theorem: 
\begin{theorem}
The upper bound obtained by the $NPA_\ell(\mathcal{Proj})$ matches exactly with the maximum eigenvalue if and only if there exists a $SoS_\ell(\mathcal{Proj})$ that upper bounds the maximum eigenvalue tightly.
\end{theorem}
%Techniques used 
Some results in the exactness proofs of other graphs will have some overlap with the first case of weighted star graph, but the explicitly constructive nature of SoS proof gives a complementary understanding of how the SDP algorithm obtains the exact solution. Finally, we discuss the sharp contrast in the SDP performance for complete graphs with even and odd number of vertices, which could be seen as a quantum version of the parity problem addressed in \cite{gri01lin}. 
Here, we prove cases where $NPA_1(\mathcal{Proj})$ is always {\it insufficient} to obtain the maximum-eigenvalue state. 

%For many interesting classes of {\scshape QMaxCut} Hamiltonians, it is possible to prove that $NPA_1(\mathcal{Proj})$ is sufficient to obtain the maximum-eigenvalue state exactly. In other cases, we can also prove that the $NPA_1(\mathcal{Proj}$ is always {\it insufficient} to obtain the maximum-eigenvalue state. 

%For the rest of the proofs, the main tool we will be using is the so-called {\it Sum of Squares} (SOS), which is dual to the Lasserre/NPA hierarchy \cite{lau09sum,pir10con}.



\subsection{Positive weighted star graph}
In this section we generalize the result of \cite{par21app} and prove that $NPA_1(\mathcal{Proj}(H))$ has optimal objective matching the extremal energy if the Hamiltonian is a positively weighted star.  Since $NPA_1^{\mathbb{R}}(\mathcal{Proj}(H))=NPA_1(\mathcal{Proj}(H))$ this implies also that $NPA_1(\mathcal{Proj})=\mu_{max}(H)$.  To our knowledge the first known proof of this statement is from unpublished personal correspondence \cite{per_comm}, however the proof we present here is simpler and has an intuitive geometric interpretation. 
The following theorem, proved in \cite{par22opt} about monogamy of entanglement on a triangle (three qubits $i, j$ and $k$), will be the starting point for our proof. %that $NPA_1(\mathcal{Proj})$ will calculate the maximum value of the Hamiltonian exactly on star graphs with positive weights.

\begin{theorem}[\protect{\cite[Lemma~7]{par22opt}}] \label{thm:MonogomyOfEntanglementOnTraingle}
    For any feasible moment matrix $M_1$ of $NPA_1(\mathcal{Proj})$, the following inequalities are true:
    \begin{align}
        0 \leq M_1(\mathbb{I}, h_{ij}) &+ M_1(\mathbb{I}, h_{jk}) + M_1(\mathbb{I}, h_{ki}) \leq 3/2 \\
        M_1(\mathbb{I},h_{ij})^2 + M_1(\mathbb{I},h_{jk})^2  +M_1(\mathbb{I},h_{ki})^2& \nonumber \\
        \leq 2\Big[M_1(\mathbb{I},h_{ij})M_1(\mathbb{I},h_{jk}) &+M_1(\mathbb{I},h_{jk})M_1(\mathbb{I},h_{ki})+M_1(\mathbb{I},h_{ki})M_1(\mathbb{I},h_{ij})\Big]. \label{eq:MonogomyOfEntanglementOnTraingle}
    \end{align}
\end{theorem}
Note that in \cite{par22opt}, the variables are defined by swap operators (\cref{eq:swap_def} in this work), and the above form could be derived by simply using the relation between swap operators and singlet projectors $h_{ij} = (1-p_{ij})/2$. 


\begin{lemma} \label{thm:60DegreeAngle}
    For any feasible moment matrix $M_1$ of $NPA_1(\mathcal{Proj})$, indexed by $\{I, h_{ij} \text{ where }i,j \in \{0,1,\ldots,n\} \text{ and } i < j\}$, the angle between any two normalized Gram vectors of indices sharing one vertex, i.e. $h_{ij}$ and $h_{jk}$ where $i,j,k$ are all distinct, is no less than $60^\circ$ and no greater than $120^\circ$ .
\end{lemma}

\begin{proof}
    Let $\ket{\mathbb{I}}$, $\ket{h_{ij}}$ be the Gram vectors of $M_1$ corresponding to indices $\mathbb{I}$ and $h_{ij}$ for any $i\neq j$ respectively. With the standard bra-ket notation, we can then simply write
    \begin{align}
        M_1(h_{ij},h_{kl}) = \braket{h_{ij}|h_{kl}}. 
    \end{align}
    Now recall that we have the following constraints on $M_1$: whenever $AB = CD$ where $A,B,C,D$ are all degree-1 polynomials in singlet projectors, $M_1(A,B) = M_1(C,D)$. From this, $h_{ij}^2 = h_{ij}$ implies that 
    \begin{align}
        M_1(h_{ij},h_{ij}) = M_1(\mathbb{I},h_{ij}).
    \end{align}
    Similarly, the anti-commutation relation for singlet projectors \cref{eq:anticommproj} implies that 
    \begin{align}
        4M_1(h_{ij}, h_{jk}) = M_1(\mathbb{I}, h_{ij}) + M_1(\mathbb{I}, h_{jk}) - M_1(\mathbb{I}, h_{ki}).
    \end{align}
    Starting from \cref{eq:MonogomyOfEntanglementOnTraingle}, we can derive the following.
    \begin{align}
        \big[M_1(\mathbb{I},h_{ij}) + M_1(\mathbb{I},h_{jk})  -M_1(\mathbb{I},h_{ki})\big]^2
        &\leq 4M_1(\mathbb{I},h_{ij})M_1(\mathbb{I},h_{jk}) \\
        \iff 16 M_1(h_{ij}, h_{jk})^2 &\leq 4M_1(\mathbb{I},h_{ij})M_1(\mathbb{I},h_{jk}) \\
        \iff \braket{h_{ij}|h_{jk}}^2 &\leq 4 \braket{\mathbb{I}|h_{ij}} \braket{\mathbb{I}|h_{jk}} = \braket{h_{ij}|h_{ij}} \braket{h_{jk}|h_{jk}} \\
        \iff \left| \braket{h_{ij}|h_{jk}}/\sqrt{\braket{h_{ij}|h_{ij}} \braket{h_{jk}|h_{jk}}} \right| &\leq 1/2 .\label{eq:derivation60}
    \end{align}
    \Cref{eq:derivation60} implies that the angle between the Gram vectors $\ket{h_{ij}}$ and $\ket{h_{jk}}$ must be between $60^\circ$ and $120^\circ$.
\end{proof}



\begin{theorem}
    The first level of the NPA hierarchy with $\mathcal{Proj}$ solves {\scshape QMaxCut} exactly for any positively weighted star graphs, i.e.,  
    $NPA_1(\mathcal{Proj}(H))=QMC(H)=\mu_{max}(H)$ for 

    
    \begin{align}\label{eq:starham}
    H = \sum_{i=1}^{n} w_i h_{0i},\quad w_i > 0\ ~~\forall i .
    \end{align}
\end{theorem}

   \begin{comment}
    Level-$1$ $\mathcal{Proj}$ program is exact in finding the ground state energy of the {\scshape QMaxCut} Hamiltonian over a star graph with positive weights
    \end{comment}

\begin{comment}
\knote{I have never been in this situation before so Im not sure what the proper thing to do is, but we might need to cite personal communication from John Wright as the original proof and then say we have an alternative proof.}
\cnote{I agree. We should mention that we were aware of SOS proof by john wright before developing our proof, and maybe we can emphasis that our proof is significantly simpler and has the potential to generalize to higher levels.}
\end{comment}

\begin{proof}
    Recall that the moment matrix $M_1$ of $NPA_1(\mathcal{Proj})$ program is indexed by $\{\mathbb{I}, h_{ij} \text{ where }i,j \in \{0,1,\ldots,n\} \text{ and } i < j\}$. Let $\ket{\widetilde{\mathbb{I}}}$, $\ket{\widetilde{h_{ij}}}$ be the normalized Gram vectors of $M_1$ corresponding to indices $\mathbb{I}$ and $h_{ij}$. Restating \Cref{thm:60DegreeAngle}, for any feasible moment matrix $M_1$, the angle between any two normalized Gram vectors of singlet projector indices sharing one vertex ($\ket{\widetilde{h_{ij}}}$ and $\ket{\widetilde{h_{jk}}}$ where $i,j,k$ are all distinct) is no less than $60^\circ$ and no greater than $120^\circ$, i.e.,
    \begin{align}
        \left| \braket{\widetilde{h_{ij}}|\widetilde{h_{jk}}}\right| \leq \frac{1}{2}.
    \end{align}
    The rest of the proof involves showing that when the objective function is a Hamiltonian of the form \cref{eq:starham},
    for the optimal solution the above inequality saturates for any two normalized Gram vectors with distinct indices from $\left\{h_{0i}\right\}_{i=1}^{n}$
    \begin{align}
        \braket{\widetilde{h_{0i}}|\widetilde{h_{0j}}} = \frac{1}{2} \quad \forall\ 0< i < j \leq n.
        \label{eq: sixty_angle_property}
    \end{align}
    When this equality holds, we can construct an actual quantum state that has the same energy as the objective value of $NPA_1(\mathcal{Proj})$ program. 
    We do this by mapping each of the normalized Gram vectors $\ket{\widetilde{h_{0i}}}$ to a state that has a singlet between $0^\text{th}$ and $i^\text{th}$ qubits , i.e., $(\ket{0}_0\otimes\ket{1}_i - \ket{1}_0\otimes \ket{0}_i)/\sqrt{2}$, and the rest of the qubits forming a maximal total spin state, e.g., all qubits in the spin up state. 
    This mapping preserves the property \cref{eq: sixty_angle_property} and also determines the normalized Gram vectors corresponding to other indices $h_{ij}$ to be $\ket{\widetilde{h_{ij}}} = \pm (\ket{\widetilde{h_{0i}}} - \ket{\widetilde{h_{0j}}})$ for  $0< i < j$ where the positive or negative sign in the front depending on whether $M_1(\mathbb{I}, h_{0i})$ is greater or less than $M_1(\mathbb{I}, h_{0j})$ respectively.
    Furthermore, when \cref{eq: sixty_angle_property} is satisfied, the $\ket{\widetilde{\mathbb{I}}}$ that maximizes the objective function will also be in the span of $\{\ket{\widetilde{h_{0i}}}\}_{i=1}^{n}$. 
    Then, since the output of the $NPA_1(\mathcal{Proj})$ program is an upper bound on the maximum eigenvalue of the Hamiltonian, this implies that the output of $NPA_1(\mathcal{Proj})$ is exact.

    Let $A$ be a matrix where the $i^\text{th}$ row of the matrix is the weighted Gram vector $\sqrt{w_i}\bra{\widetilde{h_{0i}}}$. %in some fixed chosen basis. 
    Without loss of generality, we can assume that $A$ is of size $n \times n$.
    Let $A = PU$ be its polar decomposition where $P$ is a positive semi-definite matrix and $U$ is an orthogonal matrix.
    We can rewrite the objective function as the following:
    \begin{align}
        M_1(\mathbb{I}, H) = \sum_{i=1}^{n} w_i M_1(\mathbb{I}, h_{0i}) = \sum_{i=1}^{n} w_i \braket{\widetilde{\mathbb{I}}|\widetilde{h_{0i}}}^2 = \braket{\widetilde{\mathbb{I}}|A^{T}A|\widetilde{\mathbb{I}}} = \braket{\widetilde{\mathbb{I}}|U^T P^2 U|\widetilde{\mathbb{I}}}.
    \end{align}
    Since we want to maximize the objective function, its value cannot be greater than the maximum eigenvalue of $P^2$, and it is equal to maximum eigenvalue when $U\ket{\widetilde{\mathbb{I}}}$ is the maximum eigenvector of $P^2$.
    
    The set of constraints that $\left|\braket{\widetilde{h_{0i}}|\widetilde{h_{0j}}} \right| \leq \frac{1}{2}$, where $i\neq j$, can be written as $\left|(AA^{T})_{ij} \right| = \left|(P^2)_{ij} \right| \leq \frac{1}{2}\sqrt{w_i w_j}$, where $i\neq j$, and the $ij$ subscript indicates that it's the $i^{\text{th}}$ row  $j^{\text{th}}$ column element of the particular matrix. 
    The $\ket{\widetilde{h_{ij}}}$ vectors being normalised implies $(AA^T)_{ii} = (P^2)_{ii} = w_i$ for $i \in \{1,2,...,n\}$. Given these constraints on $P^2$, the maximum eigenvalue of $P^2$ is maximized when $(P^2)_{ij} = \frac{1}{2}\sqrt{w_i w_j}$. To see this, consider $P^2$ with its maximum eigenvector $\ket{v}$ where some of the matrix elements of $P^2$ are negative. Let ${abs}(P^2)$ and $\ket{{abs}(v)}$ be the matrix where we take element wise absolute value of the matrix and the vectors. It is easy see that $\braket{{abs}(v)|{abs}(P^2)|{abs}(v)} \geq \braket{v|P^2|v}$, which implies that the maximum eigenvalue of $abs(P^2) \geq $ maximum eigenvalue of $P^2$. When all the elements of a matrix are non-negative, Perron-Frobenius theorem implies that the maximum eigenvalue is a non-decreasing function of each of the individual matrix elements and strictly increasing in the case of irreducible matrix thus implying that the optimal $P^2$ has $(P^2)_{ij} = \frac{1}{2}\sqrt{w_i w_j}$. The Gram vectors of this optimal $P^2$ are exactly $\sqrt{w_i} \ket{\widetilde{h_{0i}}}$ which satisfy the property $\braket{\widetilde{h_{0i}}|\widetilde{h_{0j}}} = \frac{1}{2}$. For the optimal $P^2$, the maximum eigenvector $U\ket{\widetilde{\mathbb{I}}}$ is also in the span of its gram vectors and so is $\ket{\widetilde{\mathbb{I}}}$ since the dimension of subspace formed by the span of $\{\ket{\widetilde{h_{0i}}}\}_{i=1}^{n}$ is $n$.
\end{proof}

\subsection{Complete bipartite graphs and some extensions}\label{subsec:compbip}

In this section, we show explicit $SoS_1(\mathcal{Proj})$ proofs for several family of graphs (shown in \cref{fig:SmallGraphs} (a)). An important tool for demonstrating exact SoS proofs is the decomposition of graphs into smaller graphs leading to a decomposition of the SoS proof into smaller SoS proofs (schematically shown in the figure).  The simplest example of such decomposition arises naturally when thinking of the SoS for the complete bipartite graph, which decomposes into several star graphs. 

%Let us first look into the easiest special case of them: the star graph. 
The weighted star graph can be solved exactly by $NPA_1(\mathcal{Proj})$ as shown in the previous section, however,  the explicit $\SoS$ cannot be analytically written down in general. The unweighted case however, gives us the simplest case of an exact $\SoS$ : 
\begin{equation}\label{eq:StarSOS}
   \left( \sqrt{\frac{n+1}{2}}\mathbb{I} - \sqrt{\frac{2}{n+1}}\sum_{i=1}^{n} h_{0i} \right)^{\bf 2}
   + \frac{1}{n+1}\sum_{1\leq j<k\leq n} h_{jk}^{\bf 2}
   = \frac{n+1}{2}\mathbb{I}-\sum_{i=1}^{n} h_{0i} .
\end{equation}
This $\SoS$ equation could be interpreted in the following way. Since the left hand side is a {\it sum of squares}, it implies that the right hand side is positive semidefinite, i.e., $0\preceq$ RHS. 
By reordering, we get $\mathbbm{I}(n+1)/2 \succeq \sum h_{0i}$, which upper bounds the eigenvalue of $\sum h_{0i}$, the Hamiltonian of interest here. 
For this particular case, %the equation gives the exact extremal eigenvalue $(n+1)/2$ for the Hamiltonian $\sum_{i=1}^{n}h_{0i}$, a star graph with $n+1$ qubits in total. 
the bound we obtain matches exactly to the actual maximum eigenvalue for the uniform star graph with $n$ edges ($n+1$ qubits in total). 
%By observing the exact $\SoS$, we can tell some properties of the ground state. For example, b
Note that \cref{eq:StarSOS} could be confirmed straightforwardly by using the anticommutation relation \cref{eq:anticommproj}. 
Also, by applying the ground state $|\mathrm{GS}\rangle$ from the left and right to \cref{eq:StarSOS}, we can see that all the terms inside the square on the left hand side must have the $|\mathrm{GS}\rangle$ as a 0-eigenvector. Indeed, expectation values of $\langle h_{jk}\rangle$ for any $0 < j<k\leq n$ should be 0 in the ground state. 
%\footnote{This could be seen as the SoS way of telling that all spins on the same sublattice in a perfect N{\'e}el state should be forming a maximal spin state that has 0 singlet density, a well known fact in condensed matter physics.}. 


Now let us consider the complete bipartite graph with $n+m$ vertices ($n\geq m$). 
The Hamiltonian could be written as
\begin{equation}
    H=\sum_{i\in A, j\in B}h_{ij}
\end{equation}
where we assume that the vertices are divided into two groups $A$ and $B$, with the edge set being $E=\{(i,j)|i\in A, j\in B\}$ and $|A|=n, |B|=m$. 
To our advantage, we can reuse the above SoS because of the decomposition property as follows: 
The maximum eigenvalue of $H$ on $K_{n,m}$ is exactly the same as that of
$K_{n,1}$ (i.e., a star graph with $n$ leaves) multiplied by $m$. Note that this relation only holds in one direction for $m<n$. Furthermore, the Hamiltonian itself could be viewed as comprising $m$ copies of the $n$-leaved star graph as well. 
In other words, 
\begin{equation}\label{eq:CBhamdecomp}
    H = \sum_{i\in B} ~ \biggl( \sum_{j\in A} h_{ij}\biggr)  ~~~\text{and}~~~ \| H \| = \sum_{i\in B} ~ \biggl\lVert \sum_{j\in A} h_{ij} \biggr\rVert
\end{equation}
holds simultaneously. 
This implies that if we can find an exact SoS for the decomposed Hamiltonian, we can combine $m$ copies of that SoS with appropriate relabeling to obtain the SoS for the entire Hamiltonian. 
Since we already have \cref{eq:StarSOS}, it is rather easy to confirm that
\begin{equation}\label{eq:CompBipSOS}
   \frac{2}{n+1} \sum_{i\in B}\Bigl(\frac{n+1}{2}\mathbb{I}-\sum_{j\in A}h_{ij}\Bigr)^{\bf 2} +
   \frac{m}{n+1}\sum_{j<k\in A}h_{jk}^{\bf 2}
   ~=~
   \frac{m(n+1)}{2}\mathbb{I}-H,
\end{equation}
which gives the exact energy for complete bipartite graphs $K_{n,m}$. %The complete bipartite case includes star graphs ($m=1$) and the square ($n=m=2$). 

The complete bipartite graph considered here are known as the Lieb-Mattis {\it model} in condensed matter physics \cite{lie62ord,lou19exa}, where the full energy spectrum is well-understood. The Lieb-Mattis {\it theorem} states that Heisenberg models with bipartite graphs (with sublattices $A$ and $B$) have ground states with total spin $\left( |A|-|B|\right) /2$, using the complete bipartite case as a starting point of the proof. 
The $\SoS$ we have here for complete bipartite graphs immediately tells you that the ``singlet density" $\langle h_{ij}\rangle$ among the same sublattice sites will always be 0, just like in the case we have mentioned for the star graph. 
This means that the two sublattices are forming the maximum total spin state, which is equivalent to the claim of the Lieb-Mattis theorem. 
We could say that our SoS is an alternative proof for the Lieb-Mattis theorem, restricted to the case of complete bipartite graphs with uniform weights. 

% Figure environment removed


\subsubsection{Crown Graphs}\label{subsubsec:crown}
Graphs with one additional edge to $K_{n,2}$ ($n\geq2$) connecting the two vertices of the B-sublattice 
(i.e. a complete tripartite graph $K_{n,1,1}$) also admits an exact $\SoS$ and thus $NPA_1(\mathcal{Proj})$ obtains the exact maximum eigenvalue as the upper bound.
These graphs, which we call the ``crown" graph (\cref{fig:SmallGraphs} (a)), have maximum eigenvalue $n+1$, the same value for the $K_{n,2}$ complete bipartite graphs. The additional edge does not change the maximum eigenvalue nor the maximum eigenvalue state itself. 

We can modify the SoS in \cref{eq:CompBipSOS} so that the Hamiltonian now includes the one additional edge on the right hand side. 
If we label the two vertices in the B-sublattice to be $a$ and $b$, then the $\SoS$ reads
\begin{equation}\label{eq:CrownSOS}
   \frac{2}{n+1} \sum_{k=a,b}\Bigl(\frac{n+1}{2}\mathbb{I}-\sum_{i\in A}h_{ik} -\frac{2+n\pm n}{4}h_{ab}
   \Bigr)^2 +
   \frac{2}{n+1}\sum_{i<j\in A}h_{ij}^2
   ~=~
   (n+1)\mathbb{I}-H, 
\end{equation}
where there is a degree of freedom for the coefficient of $h_{ab}$, coming from two solutions of a quadratic equation. 
%Note that the only difference with \Cref{eq:CompBipSOS} is the term $-h_{ab}/2$. 
%In fact, when the additional edge to $K_{n,2}$ has weight $x\leq (n+2)^2/4(n+1)=:x_c$ in general, we can always have an exact SOS of the above form just by modifying the coefficient of $h_{ab}$ to $(2+n)/4\pm\sqrt{(n+2)^2-4(n+1)x}/4$. 
%The actual ground state remains to be exactly the same for $x_c<x\leq 1+n/2$, but the simple SOS eq. (\ref{eq:CrownSOS}) ceases to exist. 

The observation that the only difference between this SoS and \cref{eq:CompBipSOS} is the $h_{ab}$ term encourages us to ask if this form of SoS is general in some sense. 
Indeed, as it turns out, we can consider a crown graph with the term $h_{ab}$ being weighted with weight $x$, and the above form of the SoS is exact for the entirety of $x\leq (n+2)^2/4(n+1)$. 
The precise $\SoS$ becomes 
\begin{eqnarray}
   &&(n+1)\mathbb{I}-\biggl(\sum_{k=a,b}\sum_{i\in A}h_{ik}+x h_{ab}\biggr)\nonumber\\
   &=&
   \frac{2}{n+1} \sum_{k=a,b}\Bigl(\frac{n+1}{2}\mathbb{I}-\sum_{i\in A}h_{ik} -\frac{2+n\pm \sqrt{(n+2)^2-4(n+1)x}}{4}h_{ab}
   \Bigr)^2 +
   \frac{2}{n+1}\sum_{i<j\in A}h_{ij}^2
   , \label{eq:WeightedCrownSOS}
\end{eqnarray}
which only has a real solution when $x\leq (n+2)^2/4(n+1)$. 

We can regard this SoS to be heuristically constructed in two steps. First, the case corresponding to $x=0$ was decomposable as in \cref{eq:CBhamdecomp}, yielding an SoS that retains the symmetry of the graph ($\mathbb{Z}_2$ between $a$ and $b$, and $\mathcal{S}_n$ for the A-sublattice sites). 
Next, when another edge is added also in a symmetry-preserving way, we can have an ansatz for the SoS that also still preserves the symmetry but now also includes the additional term. 
In this sense, the above SoS could be thought of as a ``perturbative" SoS from the complete-bipartite case, since if we gradually increase $x$ from 0, the SoS also can be changed continuously, always being exact. Since $1<(n+2)^2/4(n+1)$, the uniformly weighted crown graph is also exactly solvable, and we can say that the $\SoS$ for the complete bipartite graph and the crown graph are {\it adiabatically connected}. 
Intuitively, when $x$ is small enough, the ``physics" should not change a lot from the $x=0$ case, and in this case we can show that the ``radius of convergence" extends to $x=(n+2)^2/4(n+1)$, including $x=1$. 

The fact that the ansatz fails alone does not necessarily imply that no exact SoS exist, but it does suggest that even {\it if} such $\SoS$ exist, it will look very different from the SoS in the $x\leq(n+2)^2/4(n+1)$ region. 
As a matter of fact, we numerically observe that $NPA_1(\mathcal{Proj})$ starts to have nonzero error exactly from $x=(n+2)^2/4(n+1)$, implying that such a SoS proof indeed does not exist.  

Conversely, when we increase $x$ large enough, $NPA_1(\mathcal{Proj})$ starts to obtain the exact ground state energy again starting from $x \geq n$. Intuitively, in the $x\rightarrow\infty$ limit, the ground state should trivially become a state where there is simply one singlet placed for $h_{ab}$, and it seems natural for an SDP algorithm to be able to obtain such a simple state exactly. 
This intuition could be made rigorous by noticing that when $x\geq n$, the Hamiltonian regains the decomposition property, but now into triangles: 
\begin{equation}\label{eq:crowndecomp}
H=    \sum_{i\in A} \biggl( h_{ia}+h_{ib} + \frac{x}{n} h_{ab} \biggr), ~~~
\|H\|=    \sum_{i\in A} \Bigl\lVert h_{ia}+h_{ib} + \frac{x}{n} h_{ab} \Bigr\rVert \text{   when } x\geq n.
\end{equation}
Since a triangle with weight $(1,1,\alpha)$ has the exact $\SoS$ of
\begin{equation}\label{eq:StrongTriangleSoS}
    \begin{split}
    &\left(\alpha+\frac{1}{2}\right)\mathbb{I} - (h_{12}+h_{23}+\alpha h_{13})\\
    = &\left(\alpha+\frac{1}{2}\right)
    \biggl\{
    \mathbb{I}
    -\sum_{1\leq i<j\leq 3}    \frac{4\alpha+2\pm(-\frac{1}{2}+(-1)^{i+j})\sqrt{2\alpha (2\alpha -1)-2}}{3+6\alpha}h_{ij}
    \biggr\}^2,
    \end{split}
\end{equation}
% \begin{equation}\label{eq:StrongTriangleSoS}
%     \left(\alpha+\frac{1}{2}\right)\mathbb{I} - (h_{12}+h_{23}+\alpha h_{13})
%     = \\
%     \left(\alpha+\frac{1}{2}\right)
%     \biggl\{
%     \mathbb{I}
%     -\sum_{1\leq i<j\leq 3}    \frac{4\alpha+2\pm(-\frac{1}{2}+(-1)^{i+j})\sqrt{2\alpha (2\alpha -1)-2}}{3+6\alpha}h_{ij}
%     \biggr\}^2
% \end{equation}
for $\alpha\geq 1$, together with the decomposition, this can be turned into an $\SoS$ for the crown graph when $x\geq n$. 
Note that again, the SoS is not unique, and it has a degree of freedom in choosing $\pm$ to be fixed. 
%However, unlike the perturbation away from the complete bipartite graph case, 
When $\alpha<1$ the above form no longer gives a real coefficient.  However, the true ground state of the triangle also changes, and still allows an exact $\SoS$: 
\begin{equation}\label{eq:WeakTriangleSoS}
    \frac{3}{2}\mathbb{I} - (h_{12}+h_{23}+\alpha h_{13})
    =
    \frac{3}{2}\left(
        \mathbb{I} 
        -\frac{2}{3}h_{12}-\frac{2}{3}h_{23}
        -\frac{2\pm\sqrt{6(1-\alpha )}}{3}h_{13}\right)^2 ,
\end{equation}
which again only gives valid coefficients for $\alpha\leq 1$. 
For our current objective of constructing a $\SoS$ for the crown graph, the existence of SoS for $\alpha<1\Leftrightarrow x<n$ does not help since the Hamiltonian no longer has the decomposition \cref{eq:crowndecomp}. 

Again, like the case for the small $x$ region, although this decomposition is just {\it one} possible heuristic method for finding the exact SoS, it turns out that the $NPA_1(\mathcal{Proj})$ does start to fail exactly for $x<n$. 
Furthermore, it is possible to prove this failure for the region $(n+2)/3<x<n$ which rigorously establishes the right-side boundary at $x=n$ but leaves an unproved open space for the left-side boundary at $x=(n+2)^2/4(n+1)$. We provide this proof in Appendix \ref{app:crown}. 
The situation for the whole $x\in\mathbb{R}$ is illustrated in \cref{fig:SmallGraphs} (b). It is rather intriguing that the ``phase transition" points for the SDP ($x=(n+2)^2/4(n+1)$ and $n$), and the phase transition for the true ground state ($x=1+n/2$) are well-separated\knote{edited sentence}. This means that there are broad regions of the $x$ parameter where the SDP algorithm fails despite having exactly the same ground state as other points where SDP succeeds, which interestingly seems to be caused by the lack of real solutions in a quadratic equation \cref{eq:StrongTriangleSoS}.
In \cref{subsubsec:MG} and \cref{subsubsec:SS}, we will see more nontrivial phase transitions in condensed matter physics models. 





\subsubsection{Double Star Graphs}
While the crown graphs do not have the nice decomposition property that the complete bipartite graphs had, the double-star graphs have such a decomposition into two weighted star graphs. 
The double-star graphs are the ones with $n$ vertices connected to one vertex $a$, and the other set of $n$ vertices all connected to the other vertex $b$, and having an edge between $a$ and $b$ (thus $2n+2$ vertices in total). 

In this case, the decomposition works as 
\begin{equation}\label{eq:stardecomp}
     H 
    = \left( \frac{1}{2}h_{ab}+\sum_{i=1}^{n} h_{ai} \right)
    +\left( \frac{1}{2}h_{ab}+ \sum_{j=n+1}^{2n} h_{bj} \right), ~~~
    \| H \|
    = \biggl\lVert \frac{1}{2}h_{ab}+\sum_{i=1}^{n} h_{ai} \biggr\rVert
    +\biggl\lVert \frac{1}{2}h_{ab}+ \sum_{j=n+1}^{2n} h_{bj} \biggr\rVert,
\end{equation}
and the SoS reduces to the case of a weighted graph (with only one edge having weight $1/2$).
While the existence of exact $\SoS$ is provable for arbitrary weighted star graphs \cite{per_comm}, for the particular case corresponding to the double star we can have relatively simple analytical forms:%\knote{This might be tedious for the reader to verify, can you flesh it out in the appendix? {\bf Jun}: I'm good with maybe even moving this entire section to appendix. What do you think is best?} 
\knote{Double star graphs are pretty nice to include in the body, could definitely move the explicit SoS though}\jnote{I'm totally cool with that. I guess you are suggesting moving the Lv2 SoS to appendix and keeping Lv1 SoS = eq94 here?}\knote{yeah}
\begin{eqnarray}
\hspace{-5mm}
&~&\sum_{x=a,b}\biggl\{\biggl(\frac{E}{2} \mathbb{I} - \frac{2}{E} \sum_{i\in\partial x} h_{ix} - \frac{1}{E} h_{ab} 
    \biggr)^2
    +\frac{1}{E}\sum_{i\neq j\in\partial x}h_{ij}^2 ~~~~~~\nonumber\\
    &~& ~~+\frac{\sqrt{n/(n+2)}}{2n+1}\sum_{i\in\partial x}
    \left( h_{iy} - \left(n+1-\sqrt{n(n+2)}\right)h_{ix}+\frac{1}{2E-2}h_{ab}\right)^2 \biggr\} 
   = E\mathbb{I} - H,\label{eq:dssos}
\end{eqnarray}
where $E$ denotes the maximum eigenvalue, i.e., $E=(n+2+\sqrt{n(n+2)})/2$, and $\sum_{i\in\partial x}$ indicates summation over all vertices $i$ that are the $n$ leaves adjacent to $x=a$ or $b$. $y$ denotes the vertex $a$ or $b$ {\it other than} the one chosen for $x$ in the summation. 

While the above $\SoS$ shows that $NPA_1(\mathcal{Proj})$ obtains the ground state energy exactly for the double stars, the following $SoS_2(\mathcal{Proj})$ is simpler in form : 
\begin{eqnarray}\label{eq:lv2sos}
    &&\sum_{x=a,b}\left\{\biggl(\alpha \mathbb{I} - \beta \sum_{i\in\partial x} h_{ix} - \gamma h_{ab} 
    +\delta \sum_{i\in\partial x} h_{iy}
    \biggr)^2
    +\sum_{i\neq j\in\partial x}
    \left( \frac{\beta^2+\delta^2}{2}h_{ij}^2 + 2\beta\delta ( h_{ix}h_{jy} )^2\right) \right\}\nonumber\\
    %+2\beta\delta\sum_{x=a,b}\sum_{i\neq j\in \partial x } ( h_{ix}h_{jy} )^2 \\
    &=&
    \frac{n+2+\sqrt{n(n+2)}}{2}\mathbb{I} - H
    ,
    \end{eqnarray}
where the specific coefficients are 
\begin{comment}
\begin{eqnarray}
    \alpha &=& \frac{1}{2}\sqrt{n+2+s}\\
    \beta &=& \frac{1}{n+2}\left( \sqrt{n+2+s} +S\right)\\
    %\frac{\sqrt{2+n+s}}{2+n} + \sqrt{\frac{s+n(-2-n+s)}{n(2+n)^2}}\\
    \gamma &=& \frac{1}{n+2}\left( \sqrt{n+2+s} -\frac{n+s}{2}S\right)\\
    %\frac{2+n+s}{2+n}-\frac{1}{2}(n+s)\sqrt{\frac{s+n(-2-n+s)}{n(2+n)^2}}\\
    \delta &=& \frac{1}{n+2}    \left(    (n+1+s)S    -\sqrt{n+2+s}    \right) 
    ,
\end{eqnarray}    
\end{comment}
$\alpha=\sqrt{n+2+s}/2$, $\beta=(2\alpha+S)/(n+2)$, $\gamma=(2\alpha-(n+s)S/2)/(n+2)$, $\delta=((4\alpha^2-1)S-2\alpha)/(n+2)$,
with 
$s=\sqrt{n(n+2)}$, and $S=\sqrt{(1+2/n)^{1/2}+s-n-2}$.
Note that the $\SoS$ we provide here could again be viewed as an extension of the $\SoS$ for the complete bipartite case, just by adding another term to \cref{eq:CrownSOS}. Although this $SoS_2(\mathcal{Proj})$ Eq. (\ref{eq:lv2sos}) is weaker than $\SoS$ in terms of the SoS hierarchy, Eq. (\ref{eq:lv2sos}) straightforwardly shows that the ``two-singlet density" $\langle h_{ix}h_{jy}\rangle$ is always 0, a piece of information that was not obvious from the $\SoS$ Eq. (\ref{eq:dssos}). 

Interestingly, $NPA_1(\mathcal{Proj})$ starts to {\it fail} once the ``double star" becomes imbalanced, 
i.e. having different number of leaves on the two sides. 
This implies that the decomposition of the double graph \cref{eq:stardecomp} only holds for very precise cases with balanced double graphs and does not exist in general. 


\subsection{Complete graphs: Contrast between even and odd}\label{subsec:complete}
While the complete graphs $K_n$ do not admit similar decomposition as in \cref{eq:stardecomp}, 
we can still obtain the exact $\SoS$ by exploiting the high symmetry of the graph -- if the number of vertices $n$ is even: 
\begin{equation}\label{eq:EvenCompSOS}
    \sum_{i=1}^n 
    \left(
    \sqrt{\frac{n+2}{8}}\mathbb{I} - \sqrt{\frac{2}{n+2}}\sum_{j\neq i} h_{ij}
    \right)^2 
    = 
    \frac{n(n+2)}{8}\mathbb{I} - \sum_{1\leq i<j\leq n} h_{ij}. 
\end{equation}
Here again, the SoS is essentially a summation of the SoS for star graphs, but with slightly different coefficients, which makes them different from the simple decompositions we have been seeing. 
%This seems to be reflective of the fact that the maximum eigenvalue of the complete graph $K_n$ is exactly the same as the complete bipartite graph $K_{n/2, n/2}$, and therefore we can recycle the form of the SoS that worked for the complete bipartite graphs. 

The situation becomes quite different when the number of vertices is odd. The maximum eigenvalue is $(n+3)(n-1)/8$, but $NPA_1(\mathcal{Proj})$ gives $n(n+2)/8$ as the upper bound, which is $3/8$ bigger (observed numerically). 
% which has exactly the same formula for the even case but happens to be $3/8$ bigger (observed numerically). 
We can see that for the odd case the $NPA_1(\mathcal{Proj})$ must do {\it at least as good as} $n(n+2)/8$ from the fact that the SoS we have above works perfectly fine even when $n$ is odd. 
% The problem is that this value is simply not the maximum eigenvalue when $n$ is odd. 
%To further understand the problem, we can %A simple manipulation of the Hamiltonian for odd complete graphs reveals that 

Ideally for odd $n$, the exact SoS should give 
\begin{equation}\label{eq:xyz_sos_proof}
    \frac{(n+3)(n-1)}{8}\mathbb{I} - \sum_{1\leq i<j\leq n} h_{ij} \succeq 0
    ~~\Leftrightarrow ~~
    \sum_{i<j} \left( X_i X_j + Y_i Y_j + Z_i Z_j \right)+ \frac{3n-3}{2}\mathbb{I} \succeq 0. 
\end{equation}
%As we will see in the following, there are no $SoS_1(\mathcal{Pauli})$ proofs of this fact, so l
Let $\ell^*$ be the smallest integer such that $NPA_{\ell^*}(\mathcal{Pauli}(H))=\mu_{max}(H)$.  Since $NPA_\ell(\mathcal{Pauli})$ converges at $\ell=n$ we know $\ell^* \leq n$.
By exploiting the $SU(2)$ symmetry of the LHS, we can see that obtaining a degree-$\ell$ SoS proof for
\begin{equation}\label{eq:z_sos_proof}
    \sum_{i<j} Z_i Z_j + \frac{n-1}{2}\mathbb{I} \succeq 0 
    ~~\Leftrightarrow ~~
    \left(\sum_{i=1}^n Z_i \right)^2
    \succeq \mathbb{I} 
\end{equation}
%implies (i.e., a sufficient condition) 
would be a {\it sufficient} condition for showing
that $NPA_\ell(\mathcal{Pauli})=\mu_{max}\left( \sum_{i <j } h_{ij}\right)$.  
Since the Pauli operators $Z_i$ all commute,  
%Here, the $Z_i$'s are Pauli operators, so 
the problem essentially becomes classical and could be regarded as a {\scshape MaxCut} instance for the same odd complete graph. 
The problem then is equivalent to proving the following statement with SoS: 
\begin{center}
    {\it When you have odd numbers of $\pm 1$ values, their sum can never become 0.}
\end{center}
This trivial statement about parity becomes surprisingly hard to prove with SoS and is known to require $\lceil n/2 \rceil$-degree SoS \cite{gri01lin,lau03low,kun22spec}, so $\ell^* \leq \lceil n/2 \rceil$. 
While we believe that the same is most likely to be true for our case ($\ell^* =\Omega(n)$)\footnote{The tight SoS proof for {\scshape QMaxCut} on odd complete graphs can be reasonably named as the quantum version of the parity problem mentioned in the references.}, we were only able to prove the impossibility with $\SoS$. 

\begin{theorem}
    $NPA_1(\mathcal{Proj}(H))=n(n+2)/8$ for complete graphs with $n$ vertices, which gives the exact maximum eigenvalue when $n$ is even and is exactly $3/8$ larger than the exact maximum eigenvalue $(n+3)(n-1)/8$ when $n$ is odd.
\end{theorem}
% \begin{theorem}
%     The level-$1$ $\mathcal{Proj}$ obtains the energy upper bound $n(n+2)/8$ for complete graphs with $n$ vertices, even when $n$ is odd. In that case, the value is exactly $3/8$ larger than the true ground state energy $(n+3)(n-1)/8$. 
% \end{theorem}

\begin{proof}
    We show that the following constructed $M_1$ is a feasible solution for $NPA_1(\mathcal{Proj})$ that achieves the value $n(n+2)/8$. Together with the $\SoS$ in \cref{eq:EvenCompSOS}, this proves that the $NPA_1(\mathcal{Proj})$ gets the optimal value $n(n+2)/8$. 

    Now, consider the following moment matrix%\footnote{We drop the subscripts and superscripts to avoid cluttering.}
\begin{equation}
M=
\begin{pmatrix}
1      & a     & a     & \ldots & a      & \ldots \\
a      & a     & a/4   & \ldots & a/4    & \ldots \\
a      & a/4   & a     & \ldots & b      & \ldots \\
\vdots &\vdots &\vdots & \ddots &\vdots  &        \\
a      & a/4   & b     & \ldots & a      & \ldots \\
\vdots &\vdots &\vdots &        &\vdots  & \ddots 
\end{pmatrix},
\end{equation}
with 
\begin{equation}
    a=\frac{n+2}{4(n-1)}, ~~~ b=\frac{n(n+2)}{16(n-1)(n-3)},
\end{equation}
where the shown rows and columns are indexed by operators $\mathbb{I}, h_{12}, h_{13}, \ldots, h_{24}, \ldots$. In other words, 
\begin{equation}\label{eq:oddcompmmnew}
    M(h_{ij},h_{kl})=
    \begin{cases}
        a,   & (ij)=(kl), \\
        a/4, & (ij) ~\mathrm{and}~ (kl) \mathrm{~have~exactly~one~overlap},\\
        b,   & (ij) ~\mathrm{and}~ (kl) \mathrm{~have~no~overlaps}.
    \end{cases}
\end{equation}
It is easy to verify that this moment matrix has size $(1 + {n \choose 2})\times (1 + {n \choose 2})$, achieves energy $a\times {n \choose 2} = n(n+2)/8$, and satisfies the anti-commutation relation constraint: $((a+a-a)/2)/2=a/4$. 


All we need to do now is to show $M\succeq 0$, and we do this by constructing Gram vectors of $M$ \footnote{Alternatively, one can list all the eigenvalues of $M$ to show positive semidefiniteness, which has been the more traditional way to prove analogous results for the classical case \cite{gri01lin}. For completeness, we provide this in Appendix \ref{app:compeigen}.}. 
Specifically, we construct $1 + {n \choose 2}$ column vectors $\ket{\mathbb{I}}$ and $\{\ket h_{ij}\}$ for all $i,j\in [n]$ with $i<j$. 
Each column vector's elements are also indexed with the operators $\mathbb{I}$ and $h_{ij}$ as well, which we will denote as the subscript below.
We can then express the Gram vectors in the following way:
\begin{eqnarray}
    \ket{\mathbb{I}}_{\hat O}&=& 
    \begin{cases}
        1, & \text{ if } \hat{O}=\mathbb{I}\\
        0, & \text{otherwise},
    \end{cases}\label{eq:oddcompvec1}\\
    \ket{h_{ij}}_{\hat{O}}&=& 
    \begin{cases}
        %(n+2)/4(n-1), & \hat{O}=\mathbb{I}\\
        a, & \text{ if } \hat{O}=\mathbb{I}\\
        %\sqrt{3(n-3)(n^2-4)}/4\sqrt{{(n-1)^3}}, & \hat{O}=h_{ij}\\      
        \alpha, & \text{ if } \hat{O}=h_{ij}\\  
        %\sqrt{3(n+2)}/2\sqrt{(n-1)^3(n-2)(n-3)}, & \hat{O}=h_{jk} ~\text{with exactly one overlap}\\
        \beta, & \text{ if } \hat{O}=h_{kl} \text{ (no overlap with $ij$) }\\  
        %-\sqrt{3(n+2)(n-3)}/4\sqrt{{(n-1)^3(n-2)}}, & \hat{O}=h_{kl}~\text{with no overlap}.
        \gamma, & \text{ if } \hat{O}=h_{jk} \text{ (exactly one overlap with $ij$) },
    \end{cases}\label{eq:oddcompvec2}
\end{eqnarray}
with 
\begin{eqnarray}
    \alpha=\frac{\sqrt{3(n-3)(n^2-4)}}{4\sqrt{{(n-1)^3}}} , ~~
    \beta=\frac{\sqrt{3(n+2)}}{2\sqrt{(n-1)^3(n-2)(n-3)}} , ~~
    \gamma=-\frac{\alpha}{(n-2)}.
\end{eqnarray}
It is straightforward to confirm that these vectors Eq.~(\ref{eq:oddcompvec1}) and Eq.~(\ref{eq:oddcompvec2}) are indeed Gram vectors for the moment matrix  (\cref{eq:oddcompmmnew}) by a counting argument:
\begin{eqnarray}
    \braket{h_{ij}|h_{ij}}=& a^2 + \alpha^2 + {n-2 \choose 2}\beta^2 + (2n-4)\gamma^2 = a = \braket{\mathbb{I}|h_{ij}},\\
    \braket{h_{ij}|h_{kl}}=& a^2 + 2\alpha\beta + {n-4 \choose 2}\beta^2 + (4n-16)\beta\gamma +4\gamma^2 =b,\\
    \braket{h_{ij}|h_{jk}}=& a^2 + 2\alpha\gamma + {n-3 \choose 2}\beta^2 + (2n-6)\beta\gamma +(n-2)\gamma^2=a/4,
\end{eqnarray}
thus concluding that $M$ is the optimal $M_1$ of $NPA_1(\mathcal{Proj})$ achieving the value $n(n+2)/8$. 
\end{proof}

We can observe that the moment matrix that SDP creates is essentially ``blind to the fact that $n$ is an integer" \cite{gam22dis}
and is the reason for obtaining the wrong value $n(n+2)/8$. 
This is the energy you would get when you naively plug in an odd number to the formula for even complete graphs. 
Motivated by this fact and realizing that most of the higher order terms in the higher level moment matrix would reduce to lower degree moments (just like $a/4$ in the example above), 
we conjecture that the only independent moment matrix elements in higher levels would be 
\begin{equation}
    \left\langle 
    %\prod_{\mathrm{independent}}^{k} h_{ij}
    \overbrace{h_{ij}\cdot h_{st}\cdot \ldots}^{k \text{ independent op.s}}
    \right\rangle 
    = \prod_{l=0}^{k-1}\left( \frac{n+2-2l}{4(n-2l-1)}\right), 
\end{equation}
which is the formula for an even complete graph, but simply formally replacing $n$ with an odd number, resembling the classical case \cite{gri01lin,lau03low,kun22spec}. 
All other matrix elements would be calculable from the projector algebra constraints. 
