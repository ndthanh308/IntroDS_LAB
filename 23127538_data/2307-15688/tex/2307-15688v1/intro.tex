\section{Introduction}

The study of spin models plays a fundamental role both in physics and computer science. While 
%many physically relevant 
most models are %expected to be 
generally
too difficult to solve exactly \cite{bet31zur,lie62ord,pid17}, they provide insights into 
%uniquely quantum 
physical phenomena by %providing good representations of important systems in condensed matter physics 
serving as an effective description of condensed matter systems \cite{san10com,sac23qua}. 
%,sha81exa,maj69nex,san07evi,afl}. 
%One prominent model in the field of condensed matter physics is the antiferromagnetic Heisenberg model, which has been extensively studied and has provided insights into various physical phenomena. 
The antiferromagnetic Heisenberg model has been well-studied in physics and forms the focus of a recent flurry of work in optimization \cite{gha19alm, par22opt, ans20bey, par21app}, with the goal of extending the rich field of approximation algorithms to quantum problems. 
%applying classical techniques to approximate quantum problems REF.  %on approximation algorithms in theoretical computer science, and the search for quantum extensions have recently gathered %in computer science has shed light on approximation algorithms, recently gathering 
%attention to the antiferromagnetic Heisenberg model from an optimization point of view as well. 
Already as a classical spin model, the Ising model plays a central role in the intersection between statistical physics and combinatorial optimization. The problem of computing the ground state energy for an antiferromagnetic Ising model on an arbitrary graph is known to be equivalent to the classical ``Max Cut" ({\scshape MaxCut}) problem, one of the {\sf NP}-complete problems originally listed by Karp \cite{karp1972red}. While computing the ground state energy exactly is therefore hopeless in general, the celebrated Goemans-Williamson (GW) algorithm \cite{goe95imp} obtains an approximate solution %which has an approximation guarantee of $0.878\ldots$, 
which is optimal under the assumption of the {\it Unique Games Conjecture} \cite{khot07} and {\sf P}$\neq${\sf NP}. 

The {\it quantum} Max Cut ({\scshape QMaxCut}) problem is closely related to the antiferromagnetic quantum {\it Heisenberg} model 
and plays a crucial role in understanding the 
%serves as a stepping stone for extending our understanding of
hardness of approximation of Local Hamiltonian Problems.  
Finding the ground state of the antiferromagnetic  Heisenberg model corresponds to finding the {\it maximum} energy state of {\scshape QMaxCut}, yet the complexity of approximating these problems likely differs.  Polynomial-time approximation algorithms with constant-factor guarantees are known for {\scshape QMaxCut} on arbitrary interaction graphs, while these are not expected to exist for the antiferromagnetic Heisenberg model.  The {\scshape QMaxCut} Hamiltonian was designed to bear similarity to {\scshape MaxCut}, and this has enabled new types of approximation algorithms for quantum local Hamiltonians that draw inspiration from classical approximations for {\scshape MaxCut} and other constraint satisfaction problems.
%and the difference in convention allows us to discuss approximation algorithms from the computer science perspective. 
%these problems differ in how well they can be approximated, 
{\scshape QMaxCut} parallels {\scshape MaxCut} in that the decision version of the problem is known to be {\sf QMA}-complete \cite{pid17}, but unlike {\scshape MaxCut} the precise approximability of {\scshape QMaxCut} remains largely enigmatic. Recent works have been steadily improving the achievable approximation factor \cite{gha19alm, par22opt, ans20bey, par21app, kin22, lee22}, as well as conjecturing limitations on the achievable approximation factor \cite{hwa22uni}, but these upper and lower bounds have a sizable gap. In contrast, many combinatorial optimization problems are conjectured to be optimally approximated by techniques similar to the GW algorithm \cite{khot07}. 
%Hence, an important open problem is to find a simple approximation algorithm that is optimal under some well motivated complexity theoretic conjecture, just as the GW algorithm for the MC problem. In fact many combinatorial optimization problems are conjectured to be optimally approximated 
In general %results 
techniques of this form can be regarded as the first order of a family of approximation algorithms derived from the Lasserre hierarchy. %In general many combinatorial approximation algorithms can be optimally approximated, up to unique games, by similar techniques.

The Lasserrre hierarchy (or its dual, the Sum of Squares Hierarchy) is the tool of choice for many combinatorial optimization problems, with a well developed theory and practice (see e.g., \cite{lau09sum}). For a given problem this hierarchy corresponds to a set of semidefinite programs of increasing size and complexity (with increasing level).  At high level these hierarchies converge to the optimal solution of combinatorial optimization problems under fairly general assumptions \cite{las01glo, par03sem}, and at low level they relax the optimization problem.
%, producing an ``approximation''\cnote{this has the potential to create confusion for someone who doesn't know about these hierarchies} to the optimal solution.  
The hierarchy has the benefit of providing an explicit proof that the objective achieved by the SDP bounds the optimal objective value (a ``sum-of-squares'' proof).  On the other hand, there are known limitations on these hierarchies, with very simple objective functions provably non-convergent until the SDP reaches exponential size \cite{gri01lin}.

Navascu{\'e}s, Pironio, and Ac{\'i}n (NPA) \cite{pir10con, nav08con} generalized Lasserre's construction \cite{las01glo} to the quantum setting, producing a powerful tool for quantum information problems. Working directly with quantum states is infeasible on classical computers since they require exponential resources in space and time in general, so in many cases NPA and similar hierarchies provide new avenues for understanding quantum systems. Such hierarchies are also referred to as noncommutative or quantum Lasserre hierarchies.  Many authors have used hierarchies to characterize quantum correlations \cite{nav07bou, nav08con}, design entanglement witnesses \cite{bac17}, and probe questions in entangled games \cite{kem07ent,kem09uni,Bam15sum,joh16ext,wat18alg,cui20gen,ji22mip}. In quantum Chemistry \cite{maz07red} it is generally referred to as the {\it variational $2$-RDM method} and is used to provide computational bounds on the electronic structure problem when the dimension is too large for direct computation. More generally, SDP relaxations have been used for studying quantum many-body problems in various settings \cite{kho21sca,hai20var,bau12low,bar12sol}. Our primary application of interest is using NPA for the local Hamiltonian problem, along the lines of a recent thrust of work in quantum optimization \cite{bra16pro, gha19alm, par21app, par22opt, hwa22uni, has22opt}.


The main difference between NPA-like hierarchies and the Lasserre hierarchy is that NPA relaxes optimization over non-commuting rather than commuting variables. One might expect that quantum optimization landscape would parallel the classical one and that largely the same techniques would be useful for a breadth of problems, however, it appears that quantum optimization is richer in many ways. There are not known techniques which apply to many different problems, and, contrary to the classical case, it is known that the simplest ``first order'' algorithm is {\it not} optimal \cnote{By saying not optimal, we might be underselling the nuances in the quantum setting a bit more than we might want to {\bf Jun}: I think this is about the Pauli NPA, so it's OK. But maybe people will be confused exactly in that way is your point?}\knote{I beefed it up a little} for {\scshape QMaxCut} \cite{par22opt}, which sharply contrasts with the case for {\scshape MaxCut} \cite{goe95imp}. 
Interestingly, it is unclear at this point what form the optimal algorithm should take or even if there is an optimal classical algorithm. Since QMA-hard problems have witnesses which are highly entangled, it is likely difficult to describe them and to determine what kind of quantum state/algorithm is best for the problem.  %Essentially the difference between the hierarchies is that the Lasserre hierarchy optimizes over commuting variables while NPA treats the variables as non-commuting entities.  
%There has been a recent flurry of progress on using the NPA hierarchy in quantum approximation algorithms, 
%especially in understanding the anti-ferromagnetic Heisenberg model or ``quantum Max Cut'' ({\scshape QMaxCut}).  {\scshape QMaxCut} is a natural QMA-hard analog of {\scshape MaxCut}, with much less currently understood about its complexity of approximation.  
%especially in understanding the complexity of approximating {\scshape QMaxCut}. 
%The performance of approximation algorithms has been steadily improving using increasingly advanced ansatz, but it remains unclear what the best ansatz is for approximation algorithms since it is not understood what form the ground state should take for generic {\scshape QMaxCut} instances.
%For MC it is known that the lowest level of the Lasserre hierarchy is essentially optimal for approximation under the assumption of the \textit{Unique Games Conjecture} (UGC) and $\textsf{P} \neq \textsf{NP}$, but for {\scshape QMaxCut} it is unclear what level should be optimal as well as what ``kind'' of the NPA hierarchy should be used. 
%From this perspective, the GW algorithm fits nicely as the lowest level of the Lasserre hierarchy achieving optimal approximation ratio for MC with respect to UGC and $\textsf{P}\neq\textsf{NP}$.
Consequently, it is unclear what the best form of NPA is for {\scshape QMaxCut}, since NPA is defined using abstract non-commutative operators, and it could be that the optimal approximation algorithm takes advantage of a clever choice of the operators. 
%Symmetry is a clear theme in much of physics REF so it is reasonable to expect that the relaxation used for an approximation algorithm might take into account the important symmetry present in the quantum optimization problem. 
The generic $2$-Local Hamiltonian problem generalizes many classical problems \cite{woc03}, including those which are inapproximable (with constant approximation factor) under ${\sf P}\neq {\sf NP}$ \cite{zuc06}, so it is reasonable to expect that the optimal approximation algorithm takes advantage of the specific family or Hamiltonians it is designed for. There is precedent in this direction in that symmetry has already been used to drastically reduce the size of SDP relaxations on both the quantum \cite{ioa21} and classical \cite{gat04} side.
%since the optimal form of the ground state is not known, it is not known what Hence it reamins unclear what set of SDPs to use for Unlike MC, since the optimal algorithm is not know and since the optimal ansatz is not known, it remains unclear what set of SDPs deriving from NPA should be used.There is generalyl freedom However for {\scshape QMaxCut}, it remains unclear what level is optimal as well as what ``kind'' of the NPA hierarchy shall be used, since unlike the Lasserre hierarchy, the NPA hierarchy naturally allows a degree of freedom in the basis choice. 
One immediate inconvenience of the Pauli-based NPA hierarchy used in the past \cite{gha19alm, par21, bra19} is that the first level of the hierarchy fails to solve {\scshape QMaxCut} for the simplest types of nontrivial instances one can think of (star graphs \cite{par21app}). This again is in sharp contrast with the classical {\scshape MaxCut} case, since the GW algorithm solves {\it all} of the bipartite graph instances exactly, which includes the star graphs as the simplest subclass. 
So far, some works have focused on an NPA hierarchy based on using the Pauli operators as variables, as well as hinted at another kind of hierarchy using the anti-ferromagnetic local terms of the Hamiltonian as non-commuting variables in the optimization \cite{par22opt}. 


%Since arbitrary quantum states\knote{probably need to be more clear in preceding text that we're talking about classical alg for quantum problems} cannot be efficiently described on a classical computer the key design choice in approximation algorithms for {\scshape QMaxCut} is the ``ansatz'' or ``kind'' of state which the algorithm produces.  Many papers focus on product states \cite{bra16pro, gha19alm, par22opt}, with a recent thrust dedicated to finding approximation algorithms using entangled ansatz \cite{ans20bey, par21app, ans21imp}.  Extremal eigenvectors for QMA-complete problems are expected to be highly entangled in general, so the product state ansatz is vastly limiting. Therefore, it is an important research direction to find algorithms which natively provide states that are highly entangled.  In order to use NPA to find entangled states, it has been important in these works to understand what properties of entanglement NPA can ``mimic''.  More generally understanding which kind of instances NPA produces an exact solution at a low level has been a key component.   


\subsection{Our Contributions}

\begin{comment}
Things to mention/highlight:
\begin{enumerate}
    \item Implications for approximating QMaxCut
    \item We give explicit SoS proofs for many examples
    \item Separation between real and complex NPA.  Real version is important because it is easier to analyze and reason about, and this has enabled approximations for QMaxCut
    \item Evidence for perhaps unexpected behavior on cycles (in contrast to classical Max Cut)
    \item Easy-to-read table of analytical/numerical results (i.e., which graphs are solvable at which level)
    \item include a guide to the rest of the paper
\end{enumerate}
\end{comment}

%Our starting point is the definition of a set of SDPs approximating {\scshape QMaxCut} which respect the natural symmetry present in the problem.  
\onote{[Explain QMaxCut]}
\onote{[Summarize contributions]}
{\scshape QMaxCut} is the problem of solving for the largest eigenvalue of a class of instances of the $2-$Local Hamiltonian problem.  {\scshape QMaxCut} instances are paramterized by weighted graphs.  Given a vertex set $V$ and a functions $w$ from pairs of vertices to $\mathbb{R}_{\geq 0}$ such a Hamiltonian is written as 
$$
H=\sum_{\substack{i, j \in V\\i<j}} w_{ij} \frac{\mathbb{I}-X_i X_j -Y_i Y_j -Z_i Z_j}{4},
$$
where $\sigma_i$ stands for Pauli matrix $\sigma$ on qubit $i$.  {\scshape QMaxCut} Hamiltonians are naturally invariant under conjugation by any local unitary transformation on all qubits, so Schur-Weyl duality implies that the optimal eigenstate lies in an irreducible representation of the symmetric group.  \onote{[sentence relating SWAP and symmetric group]} Hence in defining NPA it is sensible to use permutation operators or, equivalently, polynomials in the $2$-local {\scshape QMaxCut} (Heisenberg) terms.  We demonstrate that an abstractly defined {\it operator program} has objective matching the extremal eigenvalue and that the objective of NPA defined using this operator program converges at some finite level to the optimal solution.  We show that the (weaker) real valued version of the NPA hierarchy agrees with the standard one at level-$1$, while giving an explicit example which we (numerically) demonstrate separates the real and complex versions in general.  The real version is studied in many works \cite{par21app, ans20bey, par22opt} so we motivate these works while also providing evidence that they could be improved.  The lowest level of our SDP family roughly corresponds to the second lowest level of the Pauli hierarchy used in those works (one could use the lowest level of NPA with {\scshape QMaxCut} terms in that work instead and achieve the same approximation factor) while being smaller by an order of magnitude.  Hence our results also contain an implicit run time speedup for approximating {\scshape QMaxCut}.

In the direction of improving existing algorithms, we give several new families of graphs where we demonstrate exactness/inexactness of our family of SDPs.  In existing approximation algorithms \cite{par21app, par22opt, kin22, lee22} a deep understanding of instances which the low level SDP gets correct is an integral part of the analysis (the so called ``star bound''), so it is possible that results established here could lead to approximation algorithms with better performance.  One particularly prominent example where we demonstrate SDP exactness is for weighted star graphs.  We are aware of an unpublished proof of this preceding our results \cite{per_comm}, but here we provide a different proof of this fact which gives a pleasing ``geometric'' interpretation of monogamy of entanglement inequalities in the context of NPA hierarchies.  The weighted star bound seems likely to have many applications, here we demonstrate that it implies exactness for another family of graphs, the ``double star'' graphs.  We complement this with many other classes of graphs where we can show exactness, some of which correspond to condensed matter physics models including the Majumdar Ghosh-model and the Shastry-Sutherland model.  Additionally, we provide two families (complete graphs with odd number of vertices, and ``crown graphs" with certain weights) of graphs where we can analytically prove looseness of NPA at the first level. In fact we are able to provide an analytic characterization of when low levels of the hierarchy are exact on crown graphs.

%and show that at level-$1$ the real version of NPA agrees with the complex version.  Since many approximation algorithms use only the real version \cite{par21app, ans20bey, par22opt} this provides motivation for those works.  Indeed, the second lowest level of NPA defined with Pauli terms is used in \cite{par21app} and roughly corresponds to the lowest level of NPA defined using {\scshape QMaxCut} terms (one could use the lowest level of NPA with {\scshape QMaxCut} terms in that work instead and achieve the same approximation factor).  

Our above results yield a characterization of the algebra of SWAP operators, using polynomial equations of degree at most two over the operators.  To the best of our knowledge, this is the first such characterization.  
%Previous works have conjectured such characterizations using polynomial equations of higher degree 
It has been rather ambiguous what degree a complete characterization requires \cite{ili18squ,egg01sep}, and in this context, it may be surprising that a degree-two characterization is possible.\onote{Can we give some applications for understanding the SWAP algebra?  What do the previous works say?} 
The algebra of the SWAP operators we introduce could be regarded as a variant of the well-studied Temperley-Lieb algebra \cite{tem71rel}, potentially giving new directions for studying the mathematical structure of Heisenberg models with general interaction graphs.\jnote{Probably overselling, but some kind of an application...} 

%We discuss several families of {\scshape QMaxCut} instances and provide analytic proofs that the NPA hierarchy we introduced obtains the exact solution at the lowest level.  These families include %many natural classes including 
%natural simple graphs like
%weighted star graphs and (nearly) complete bipartite graphs, as well as some condensed matter physics models including the Majumdar Ghosh-model and the Shastry-Sutherland model, as we will see later.  
%Additionally, we provide two families (complete graphs with odd number of vertices, and ``crown graphs" with certain weights) of graphs where we can analytically prove looseness of NPA at the first level. 
%Although it is far from a complete categorization on when the lowest level NPA obtains the exact ground state, it serves as a theoretical steppingstone to understand the strength and weakness of the most natural NPA hierarchy for {\scshape QMaxCut}. 


Equipped with the new SDP family, we then provide extensive numerical results studying the exactness/inexactness of NPA at low levels. 
We first provide results for an exhaustive search among all possible unweighted graphs up to $8$ vertices, and then proceed to physically interesting cases with up to $60$ vertices. 
With the exhaustive search, we find no unifying features among examples where NPA is exact, and examples which are seemingly ``simple'' where the optimal SDP objective at low level is far from the extremal eigenvalue as well. For cycles we find that neither the first level of our SDP family or second level the hierarchy previously considered in \cite{par21app, par22opt} is exact at low levels in sharp contrast to MaxCut where the lowest level is exact on all even cycles and the second level is exact on all cycles\knote{maybe cite Boaz Barak lecture notes here}\jnote{We don't actually have numerical evidence about Lv2 Proj, if I remember correctly. We couldn't go to large enough systems to test if it's actually failing for large cycles.}. 
%We find that the landscape of exact solvability with NPA is extremely complicated, with many complicated-appearing graphs being exactly solved with no apparent unifying feature. 
It is impossible to rigorously certify that NPA achieves the optimal eigenvalue using purely numerics, since we have many cases where the optimal SDP objective is only different from the extremal eigenvalue in the 4th or 5th decimal place. 
%and many examples have optimal SDP objective which agrees with the extremal eigenvalue to 4 or 5 decimal places, so a careful analysis is required. 
We classify graphs according to how the error of the SDP optimal solution behaves as a function of the tolerance parameter for the SDP. This lets us confidently conclude from numerics, whether the NPA is giving the exact extremal eigenvalue or not. 
 In doing so, we are able to explicitly show separation of different NPA hierarchies, which is otherwise subtle. 

Moreover, we run numerical simulations on some condensed matter physics models, demonstrating that the the lowest level of our NPA hierarchy obtains exact ground states of ``frustration-free" quantum spin systems such as the Majumdar-Ghosh and Shastry-Sutherland models. \cnote{This sentence doesn't make sense {\bf Jun}: How about now?} We point out that this is a natural consequence from the connection between frustration-freeness and sum of squares proof, showing that the NPA hierarchy as a whole is essentially a generalization of the frustration-free notion. 

The salient feature of our numerical results is that the SDP seems to predict many important physical properties even on instances where it is not achieving the optimal eigenvalue.  For instance, in models with a phase transition, the SDP also appears to reflect that, by having a discontinuous optimal SDP objective as a function of the parameters. 
Additionally, the SDP obtains the correct decaying exponent for the correlation function as on the Heisenberg spin chain, even though there is strong evidence that it does not correctly predict the optimal energy.  This suggests the capability of SDP solutions to exhibit nontrivial long-range entangled features of a critical ground state to some extent.  Using ``pseudo entanglement'' to model quantum systems and predict their physical properties seems to be a relatively open and exciting research direction with only a few results known \cite{has22per}. Since simulating large quantum systems is intractable on classical computers, the NPA hierarchy provides the possibility of probing features of quantum systems using (non-physical) pseudo states on a classical computer which would be unobtainable otherwise. 
This type of numerical analysis is only possible with our projector-based NPA hierarchy, since with the Pauli-based NPA hierarchy, the matrix size for SDP grows faster. Although the scaling difference is theoretically only a constant factor, the largest computable system size being $\sim 60$ qubits rather than $\sim 20$ makes a practical difference in terms of how deeply we can actually probe their performance. Moreover, in the projector SDP formulation, most of the variables in the moment matrix are free variables, an important feature that can significantly improve the numerical efficiency of solving SDPs when implemented in SDP solvers like MOSEK \cite{mosek}. This difference has enabled us to conduct both the exhaustive search and probing statistical physics model of sizes beyond what is reachable with exact diagonalization. 

%We define a family of SDPs which arise form applying the NPA hierarchy to the anti-ferromagnetic Heisenberg terms in the Hamiltonian.  We are able to demonstrate convergence of this family of SDPs to the maximal eigenvalue of the Hamiltonian in two ways, from relatively simple arguments using $C^*$ algebras (SECTION BLANK) as well as from the representation theory of the symmetric group (APPENDIX SECTION BLANK).\knote{now need to write about the numerical stuff observed}

%Our results shed light on the ``kind'' of {\scshape QMaxCut} instances for which NPA provides an exact solution.  Since similar facts have been critical components of approximation algorithms, we expect our results could have application in the design of approximation algorithms for these problems, especially for low degree graphs.  Additionally, by defining and examining what we feel is the most natural family of SDPs for {\scshape QMaxCut} instances we have opened the door to the use of this family in approximation algorithms.  Indeed we have motivated this family as natural since we have demonstrated its convergence to the optimal (extremal) quantum eigenvalue.  

%\subsection{Applications}

%State-of-the-art approximation algorithms for {\scshape QMaxCut} all have analyses which crucially depend on understanding that NPA at a low level is exact on certain families of graphs.  For instance \cite{kin22, par21app, lee22} all require use of the ``star bound'' which implies that NPA is exact on unweighted star graphs, while \cite{par22opt} crucially used the fact that NPA is exact on any weighted triangle graph ($K_3$). We provide several large new classes of graphs for which NPA provides exact solutions. This has the potential to lead to new approximation algorithms and more generally a deeper understanding of the strengths of low levels of the NPA hierarchy. If we know the strengths of a given level of the hierarchy this could help in the design of better approximations and might point to the optimal rounding strategy at that level of the hierarchy. Our extensive numerical results on small graphs also point toward graphs for which NPA has particularly bad performance. It is conceivable that a good approximation algorithm will need to find ways to ``get around'' these examples in order to achieve a good approximation factor.

%Another key feature of the numerics is the suggestion that NPA be used to model quantum systems. While this theme has already been investigated in many specific contexts (as highlighted in the previous section), we provide clear indications that the hierarchy seems to be ``predicting'' phase transitions for quantum systems and mimicking their properties, even for those on which NPA is not exact. Using ``pseudo entanglement'' to model quantum systems and predict their physical properties seems to be a relatively open and exciting research direction with only a few results known \cite{has22per}. Since simulating large quantum systems is intractable on classical computers, the NPA hierarchy provides the possibility of probing features of quantum systems using (non-physical) pseudo states on a classical computer which would be unobtainable otherwise. For quantum chemistry problems, existing semi-classical methods model chemical reactions by treating a very small portion of it quantum mechanically while using a classical (balls and springs) model on the rest. We believe that the use of a cleverly chosen NPA program has the potential to model important features while treating all the pieces on the same footing, potentially pointing to reaction mechanisms missed by standard methods.

\textbf{Note Added:}  \onote{[Our papers have some shared results but are larelgy complementary; our focus is more on understanding SDPs for appproximating or understanding ground states of local Hamiltonians.  More complicated or powerful formulations are possible [other paper], but it's important to understand ``simple'' formulations, because they're easier to analyze, SDPs are slow to solve, and it's good to understand their power.  Thus there is strong motivation to understand the power of the simplest SDPs that offer strong bounds.]}

After preparing this draft we became aware of an independent group of researchers with complementary results to ours \cite{other_guys}.  The explicit technical results established in our two papers are largely disparate with the central overlap being the finite presentation of the SWAP algebra.  The two papers have different themes in that our paper is largely focused on understanding the performance of low level SDP relaxations for Quantum Max Cut problems, while \cite{other_guys} establishes a more sophisticated hierarchy and uses representation theory to analyze the extremal energies for certain Quantum Max Cut instances.  There is clearly value in understanding powerful SDP relaxations, but we argue that it is also important to understand ``simple'' formulations because they are easier to analyze and SDPs are generally practically slow to solve.  Thus there is a strong motivation to understand the simplest SDPs which offer strong bounds.  We establish numerically and analytically that low levels of the hierarchy are exact on certain families of graphs with an eye toward solid state physics and approximation algorithms, while \cite{other_guys} is able to calculate the exact extremal eigenvalue for Hamiltonians which have a {\it signed clique decomposition}.  Here the two papers are very different in that we focus on the SDP solution rather than the exact solution for the Hamiltonian problem. \cite{other_guys} also investigates non-commutative Groebner bases for the $\SWAP$ algebra which we do not touch on and establishes finite convergence of their hierarchy at a lower level that we were able to show (\Cref{prop:fin_conv} in this work versus Theorem 4.8 in \cite{other_guys}).  We expect that both papers have much to offer one another, but we leave the full set of implications from the combined results for future work.  

