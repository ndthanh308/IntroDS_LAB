\section{Notation}\label{sec:notation}
The Pauli matrices are defined as:
\begin{equation*}
\label{eq:paulis}
 \mathbb{I}=\begin{bmatrix}
1 & 0 \\
0 & 1
\end{bmatrix},
\,\,\,\,\,\,
X=\begin{bmatrix}
0 & 1 \\
1 & 0
\end{bmatrix},
\,\,\,\,\,\,
Y=\begin{bmatrix}
0 & -i \\
i & 0
\end{bmatrix}, \,\text{and}
\,\,\,\,\,\,
Z=\begin{bmatrix}
1 & 0 \\
0 & -1
\end{bmatrix}.
\end{equation*}
\noindent Subscripts indicate quantum subsystems among $n$ qubits.  For instance, the notation $\sigma_i$ is used to denote a Pauli matrix $\sigma \in \{X,Y,Z\}$ acting on qubit $i$, i.e.\ $\sigma_i := \mathbb{I} \otimes \mathbb{I} \otimes \ldots \otimes \sigma \otimes \ldots \otimes \mathbb{I} \in \mathbb{C}^{2^n \times 2^n}$, where the $\sigma$ occurs at position $i$.  $\mathbb{I}$ will also be used to denote identity matrices of arbitrary context dependent size.  

We will be considering weighted graphs, $(V, \{w_{ij}\}_{ij\in V\times V})$, where each weight is non-negative.  Without loss of generality we can assume the graph is complete by possibly setting some weights to zero, so we need not include an edge set in the description of the graph.  The complex conjugate transpose of a given matrix will take the standard notation, $A^*$ and we will denote the $\max/\min$ eigenvalue of a given operator $A$ as $\mu_{max}(A)/\mu_{min}(A)$ respectively.  We will be considering Hermitian operators and operators which differ from a Hermitian matrix by a similarity transform, i.e. $T A T^{-1}$ is Hermitian, so this can be well-defined by:
$$
\mu_{min} (A) :=\min \lambda\in \mathbb{R}: \det(\lambda \mathbb{I} - A)=0;\,\,\,\,\,\,\text{and}\,\,\,\,\,\,\mu_{max} (A) :=\max \lambda\in \mathbb{R} : \det(\lambda \mathbb{I} - A)=0.
$$
%If $A$ is Hermitian, 
%$$
%\mu_{min} =\min_{\ket{\phi}: \braket{\phi|\phi}=1} \braket{\phi|A|\phi} \,\,\,\,\,\,\text{and}\,\,\,\,\,\, \mu_{max} =\max_{\ket{\phi}: \braket{\phi|\phi}=1} \braket{\phi|A|\phi}.
%$$
We will have need to discuss matrix/scalar variables and will generally denote these with lower case letters, while upper case letters will be generally used to denote assignments to those variables.  Polynomials in matrix variables will generally be denoted with Greek letters.  