\documentclass[11pt]{article}

\usepackage[utf8]{inputenc}
\usepackage{mathtools}
\usepackage{amsmath, amssymb, amsfonts, amsthm}
\usepackage{bbm}
\usepackage{physics}
\usepackage[toc,page]{appendix}
\usepackage{authblk}
\usepackage[linesnumbered,ruled,vlined]{algorithm2e}
\usepackage{boldline, multirow}
\usepackage[top=2cm,margin=0.9in]{geometry}
\usepackage{graphicx}
\usepackage{xcolor,soul}
\usepackage{url}
\usepackage{pdfpages}
\usepackage{comment}
% \usepackage{showkeys}
% \allowdisplaybreaks
% \usepackage[square,numbers]{natbib}
\usepackage{dutchcal}
\usepackage{ytableau}
\usepackage{fixmath}
%\usepackage{biblatex}
\usepackage[
backend=biber,
style=alphabetic,
sorting=ynt,
maxbibnames=9
]{biblatex}
\addbibresource{refs.bib}


% The following order is extremely important! 
\usepackage{hyperref}
\usepackage[capitalise]{cleveref}

\usepackage{tikz}
\usepackage{braket}
\usetikzlibrary{quantikz}


\newtheorem{theorem}{Theorem}[section]
\newtheorem{lemma}[theorem]{Lemma}
\newtheorem{proposition}[theorem]{Proposition}
\newtheorem{corollary}[theorem]{Corollary}
\newtheorem*{propositions}{Proposition}
\theoremstyle{definition}
\newtheorem{definition}[theorem]{Definition}
\newtheorem{example}[theorem]{Example}
\theoremstyle{remark}
\newtheorem{remark}[theorem]{Remark}
\newtheorem*{remarks}{Remark}


\DeclareMathOperator{\D}{D}
\DeclareMathOperator{\spec}{spec}
\DeclareMathOperator{\Det}{Det}
\DeclareMathOperator*{\argmin}{arg\,min}
\DeclareMathOperator{\diag}{diag}
%\DeclareUnicodeCharacter{0301}{*************************************}

\newcommand\inp[2]{\left\langle #1,\,#2 \right\rangle}
% \newcommand{\norm}[1]{\left\lVert#1\right\rVert}
% \newcommand{\abs}[1]{\lvert#1\rvert}
\newcommand{\id}{\mathrm{I}}
\newcommand*{\intzerotoinfty}[2]{\int_0^{+\infty}#1\mathop{}\!\mathrm{d}#2}
\newcommand{\SWAP}{SW\hspace{-1.2mm}AP}
\newcommand{\SoS}{SoS_1(\mathcal{Proj})}

\DeclarePairedDelimiterX{\infdivx}[2]{(}{)}{%
  #1\;\delimsize\|\;#2%
}
\newcommand{\infdiv}{\mathcal{D}\infdivx}
\newcommand{\brak}[1]{\left\{#1\right\}}
\newcommand{\pmone}{\brak{-1, 1}}
\newcommand{\R}{\mathbb{R}}

\newcommand{\knote}[1]{\footnote{{\color{red} {\bf Kevin}: {#1}}}}
% \newcommand{\lunote}[1]{{\color{red} {\bf Lu}: {#1}}}
\newcommand{\lunote}[1]{\footnote{{\color{red} {\bf Lu}: {#1}}}}
\newcommand{\cnote}[1]{\footnote{{\color{red} {\bf Chaithanya}: {#1}}}}
\newcommand{\jnote}[1]{\footnote{{\color{red} {\bf Jun}: {#1}}}}
\newcommand{\onote}[1]{\footnote{{\color{blue} {\bf Ojas}: {#1}}}}

\renewcommand{\knote}[1]{}
\renewcommand{\lunote}[1]{}
\renewcommand{\cnote}[1]{}
\renewcommand{\jnote}[1]{}
\renewcommand{\onote}[1]{}


\newcommand{\red}[1]{\textcolor{red}{#1}}
\newcommand{\req}{\red{\stackrel{?}{=}}}
\renewcommand\footnotemark{}


\title{An SU(2)-symmetric Semidefinite Programming Hierarchy for Quantum Max Cut}
%Convergence, Exactness and Computational Results of an $SU(2)$-symmetric NPA Hierarchy for Quantum Max Cut}
%Singlet projector semidefinite programming \\for the Quantum Max-Cut problem}
\author[1]{Jun Takahashi\thanks{\texttt{ \{juntakahashi, chaithanyarss, czhou\} @unm.edu, wking@caltech.edu, \{kevthom, odparek\}@sandia.gov}}}
\author[1]{Chaithanya Rayudu}%\thanks{chaithanyarss@unm.edu}}
\author[1]{Cunlu Zhou}%\thanks{czhou@unm.edu}}
\author[2]{Robbie King\thanks{\texttt{}}}
\author[3]{Kevin Thompson\thanks{\texttt{ }}}
\author[3]{Ojas Parekh}%\thanks{odparek@sandia.gov}}
\affil[1]{Department of Physics and Astronomy and Center for Quantum Information and Control, University of New Mexico, Albuquerque, New Mexico 87131, USA}
\affil[2]{Department of Computing and Mathematical Sciences, Caltech, Pasadena, CA, USA}
\affil[3]{Sandia National Laboratories, Albuquerque, NM, USA}
\date{}

\begin{document}

\maketitle

\begin{abstract}
Understanding and approximating extremal energy states of local Hamiltonians is a central problem in quantum physics and complexity theory.  Recent work has focused on developing approximation algorithms for local Hamiltonians, and in particular the ``Quantum Max Cut'' ({\scshape QMaxCut}) problem, which is closely related to the %well-established 
antiferromagnetic Heisenberg model. 
In this work, we introduce a family of semidefinite programming (SDP) relaxations based on the Navascu{\'e}s-Pironio-Ac{\'i}n (NPA) hierarchy which is tailored for {\scshape QMaxCut} by taking into account its SU(2) symmetry. 
We show that the hierarchy converges to the optimal {\scshape QMaxCut} value at a finite level, 
% In this work, we take the approach of constructing a hierarchy of semidefinite programming (SDP) relaxations of the problem, 
% %Much of this work relies on relaxations obtained via noncommutative hierarchies 
% %of semidefinite programs (SDPs), 
% which had been extremely successful in the classical analogous case. We specialize the noncommutative Navascu{\'e}s-Pironio-Ac{\'i}n (NPA) hierarchy to introduce a family of SDPs tailored for {\scshape QMaxCut} by taking its SU(2) symmetry into account, proving that for sufficiently large levels of the hierarchy, the  optimal SDP value converges to the correct {\scshape QMaxCut} value.  
%and our results may also be 
which is based on a new characterization of the algebra of SWAP operators.
%as well. 
We give several analytic proofs and computational results %proving or suggesting 
showing
exactness/inexactness of our hierarchy at the lowest level on several important families of graphs.

%as well as computational results 
%and computational results proving exactness of this hierarchy  
%concerning this family. We demonstrate that the family of SDPs converges to the solution of {\scshape QMaxCut} at finite level for any instance, as well as several other exactness/inexactness results for other interesting families of graphs at low levels.

%The NPA hierarchy with its increasing levels provide tighter bounds on the {\scshape QMaxCut} solution. We prove that the SDP family we introduce converges to the true {\scshape QMaxCut} value at a finite level, justifying the use of low-level NPA hierarchy as an approximation.
%opening up new directions for approximation algorithms more tailored to the problem
%We give a proof that the first-level NPA already obtains the exact values for arbitrary positively weighted star graphs, as well as several other exactness/inexactness results for other interesting families of graphs. Our proof for weighted star graph naturally offers a geometric interpretation of monogamy of entanglement equalities in the context of NPA. 

We also discuss relationships between SDP approaches for {\scshape QMaxCut} and frustration-freeness in condensed matter physics and numerically demonstrate that the SDP-solvability practically becomes an efficiently-computable generalization of frustration-freeness.
Furthermore, by numerical demonstration we show the potential of SDP algorithms to perform as an approximate method to compute physical quantities and capture physical features of some Heisenberg-type statistical mechanics models even away from the frustration-free regions. 


\end{abstract}

\newpage
\tableofcontents


\newpage
% Figure environment removed

\section{Introduction}
Automatic 3D reconstruction of clothed humans using image inputs has gained increasing significance due to its potential applications in a wide array of AR/VR scenarios. High-fidelity reconstructions typically depend on sophisticated capture systems, which are developed with dense camera arrays~\cite{collet2015high,joo2015panoptic,joo2018total}, programmable light-stages~\cite{Vlasic2009, guo2019relightables}, and depth sensors~\cite{newcombe2011kinectfusion,DoubleFusion,BodyFusion,dou2016fusion4d,newcombe2015dynamicfusion}. However, stringent capture environments equipped with complex hardware pose significant challenges for consumer-level applications.


In this context, considerable research effort has been dedicated to developing methods that allow for more flexible capture configurations, such as utilizing a few RGB inputs. Among these works, learning implicit functions \cite{iccv2020PIFu, saito2020pifuhd, hong2021stereopifu} has proven effective in achieving highly detailed reconstructions by integrating the advancements of deep neural networks. These methods employ large multi-layer perceptrons (MLPs) to predict the occupancy probability or truncated signed distance function (TSDF) value of every queried 3D point based on its associated local feature, which is extracted from images. They can recover a continuous surface at arbitrary resolutions without topology restrictions.


However, in typical MLP-based implicit networks, the occupancy or TSDF value at each location is solved independently with planar image features, rendering them less capable of addressing challenging cases such as occlusions. Consequently, these methods suffer from generalization and robustness issues, particularly when tackling strong occlusions caused by large motion or multiple interacting humans. 
Some follow-up studies  \cite{zheng2021deepmulticap,zheng2021pamir,huang2020arch} utilize an extra geometric model, SMPL~\cite{Loper2015}, to improve robustness by introducing strong shape priors. 
Their success typically relies on the assumption of geometrical similarity \cite{huang2020arch} between the shape prior and target reconstruction, making them intractable for handling complex cases with loose clothes and sensitive to errors in SMPL model fitting.



%\ping{this paragraph sounds like `TSDF is better than MLP/SMPL, and we use TSDF to solve the problem'. But in Sec 3, we are telling a different story, saying `MLP needs a 3D convolutional encoder'. We need to make these two sections consistent.}\sicong{I think in this paragraph we claim that the TSDF}


%We opt for Trucated Signed Distance Funtion (TSDF) volumetric representations as they are naturally suitable for convolution operations, which have shown remarkable performance for learning hierarchical features on 2D visual perception tasks \cite{SunXLW19}. 
%Meanwhile, TSDF also describes the gradual geometry change around shape surface, which is not reflected by occupancy volume. 

We instead revisit the 3D volumetric representation and resort to 3D convolutional neural networks (CNNs) for feature learning, due to their impressive performance in feature learning and the ability to incorporate spatial context. However, volumetric methods and 3D convolution involve discretization, which might raise concerns regarding whether a discretized volume can preserve subtle geometric details as continuous representations learned in implicit functions. We investigate the relationship between volume resolution and quantization error on synthetic data by converting target mesh objects to TSDF volumes, as shown in Figure~\ref{fig:quantization_error}. We observe that the quantization errors are significantly reduced by increasing volume resolution and become nearly negligible when reaching a relatively high resolution (e.g., 512 or higher). In other words, achieving fine-detailed reconstruction is not supposed to be restricted by the use of volume representations as long as a proper volume resolution is utilized. Therefore, we present a method with high-resolution feature volumes, e.g., 256 and 512, while traditional volumetric methods \cite{varol18_bodynet,gilbert2018volumetric} are often limited to much lower resolutions, such as 32 or 128.



On the other hand, an increase in volume resolution may lead to a cubic growth of memory overhead \cite{8100085}. Reducing memory costs while guaranteeing the granularity of volumetric representations is necessary for pursuing high-quality reconstruction. Thus, we adopt a coarse-to-fine approach and cull away irrelevant voxels to build a sparse high-resolution feature volume. At the coarse level, the network computes an initial TSDF by applying a U-Net with sparse 3D CNN \cite{3DSemanticSegmentationWithSubmanifoldSparseConvNet} on the sparse feature volume, which is carved by a visual hull. Through our experiments, it turns out that more than 95\% of the volume grids are discarded by the visual hull culling, making the sparse 3D CNN efficient. At the fine level, the network focuses on a narrow band near the zero-level set of the initial TSDF and discretizes the narrow band with smaller voxels. By employing this narrow-band culling, we further shrink the sampling space, resulting in a relatively small range of grid numbers (usually 300K--500K in our experiments) even with a high volume resolution of 512. The remaining voxels in the narrow band are associated with features that fuse high-frequency information from the computed normal maps upon the low-frequency shape from the coarse level to compute the TSDF at high resolution. The final mesh is then extracted from the TSDF using the Marching-Cube algorithm ~\cite{Lorensen87marchingcubes}.
% Different from the u-net sturcture to preserve global topology context, we then apply a shallow 3dcnn to compute the final TSDF $D_{final}$ which contain more local geometry detail.




% \ping{this paragraph can be expanded. It is an important contribution and often ignored by other works. stress on the novel idea of regressing blending weights instead of colors}

In addition to geometry, high-quality mesh texture is also a crucial factor contributing to visual appearance. Directly computing a color field in 3D space, as in \cite{iccv2020PIFu}, struggles to capture high-frequency texture details, while the neural radiance field (NeRF) \cite{yu2020pixelnerf} or the DoubleField~\cite{shao2022doublefield} require expensive per-instance optimization and are often unstable for sparse input images. In contrast, we adopt an image-based rendering approach to compute a texture atlas map, which is efficient and widely supported in existing computer graphics tools. 
Specifically, we compute a blending weight at each 3D point on the mesh surface to determine its color as a weighted average of the colors at its image projections. The blending weights can be computed at a relatively coarse resolution, e.g., 512 volume resolution in our case, and leave texture details to the high-resolution images, such as 1K or 2K. Unlike previous methods that generate blurry texturing results under sparse input, our method generalizes well on both synthetic and real data with just a few input views. 
Figure~\ref{fig:teaser} shows two examples reconstructed by our method. Despite the challenging garment, pose, and occlusion, our method recovers faithful shape, normal, and texture on the right.

%with a wide variety of poses and clothing styles, and it is also adaptive to handle input image with arbitrary resolutions.
%\sicong{For this concern we claim that when the resolution of dicretized volume meets certain threshold (which is 256 in our experiment), the quantization error can be neglected.} 



In summary, the main contributions of this paper are as follows:
\begin{itemize}
\vspace{-0.1in}
  \item 
  We revisit the 3D volumetric representation and demonstrate that it can support clothed human reconstruction with equal or even better performance compared to implicit representation. 
  \item 
  We develop a memory and computation-efficient method for high-resolution volumetric reconstruction using sophisticated sparse 3D CNN, coarse-to-fine estimation, and voxel culling by visual hull and narrow bands. 
  \item 
  We introduce a novel method to compute a texture atlas map, which captures rich appearance details from high-resolution input images.
  \item 
  We achieve impressive results on standard benchmark datasets Twindom and MultiHuman, significantly reducing the point-2-surface (P2S) precision to approximately 0.2cm from just six input views, with more than $50\%$ error reduction compared to the state-of-the-art methods, including DoubleField~\cite{shao2022doublefield} and PIFuHD~\cite{saito2020pifuhd}.
\end{itemize}
\section{NOTATION AND BACKGROUND}

We denote the set of real numbers by $\R$ and the set of natural numbers by $\N$. The set $\{1, \dots, a\} \subset \N$ is denoted by $[a]$ for $a \in \N$, and similarly ${a, \dots, b} \subset \N$ is denoted by $[a \sdots b]$. For a pair of boolean variables $x$ and $y$, the notation $\wedge$ denotes the ``and'' operator while $\vee$ denotes ``or.'' For a set of boolean variables $\{ x_1, x_2, \ldots, x_n \}$, the notations $\bigwedge_{i=1}^n x_i$ and $\bigvee_{i=1}^n x_i$ denote $x_1 \wedge x_2 \wedge \ldots \wedge x_n$ and $x_1 \vee x_2 \vee \ldots \vee x_n$, respectively. The logical negation of a boolean variable or vector $x$ is denoted by $\neg x$. We denote the identically zero function on a domain by $\zf$, and we write $f(\cdot) \not \equiv \zf$ to mean that $f(\cdot)$ is not equivalent to the zero function over its argument---i.e., there exists an input where $f$ is nonzero.


\subsection{Measure theory and probability}

For a random variable $X$, we introduce the notation ${\Pdi{x} \in M(X)}$ to represent a probability measure over the values $x$ in the domain of $X$, contained in the space of measures $M(X)$. The uniform measure over an interval $[a,b] \subset \R$ is denoted by $\cU(a,b)$. For two measures $\mu$ and $\nu$, we say that $\nu$ is absolutely continuous with respect to $\mu$ if for every $\mu$-measurable set $A$, $\mu(A) = 0$ implies $\nu(A) = 0$. If $\nu$ is absolutely continuous with respect to $\mu$, we let $d\nu / d\mu$ denote the Radon-Nikodym derivative of $\nu$ with respect to $\mu$. The standard Lebesgue measure on $\R$ is denoted $\lambda$. For a measure $\mu$ which is absolutely continuous with respect to $\lambda$, we define its $L_1$ norm in the typical manner
\begin{align*}
    \Pnormi{\mu} \defeq \int \biggl \lvert \frac{d \mu}{d \lambda} \biggr \rvert d \lambda,
\end{align*}
which we take to be the default norm in the Banach space of measures on $\R$. We denote independence between two random variables using $\PI$ and its negation by $\nPI$. 


\subsection{Causal graphs and structural causal models}

We denote a directed acyclic graph by $\G$, with the presence of a direct edge between nodes $X$ and $Y$ denoted $X \cau Y$. For a given node $X$ in $\G$, we let $\G_{\underline{X}}$ denote the graph obtained by deleting outgoing edges from $X$. We denote sets of nodes in a graph using bold font (e.g., $\bZ$). The set of parents of a node $X$ in a graph is denoted by $\pa{X}$. A path between two nodes $X$ and $Y$ can consist of arbitrarily directed edges and is said to be blocked by a set of nodes $\bZ$ if the path contains any of the following \cite{Pearl09}:
\begin{itemize}
    \item A chain $I \cau M \cau J$ with $M \in \bZ$.
    \item A fork $I \leftarrow M \cau J$ with $M \in \bZ$.
    \item A collider $I \cau M \leftarrow J$ such that $M \not \in \bZ$ and no descendent of $M$ is in $\bZ$.
\end{itemize}
     Two nodes $X$ and $Y$ are said to be d-separated by $\bZ$ if $\bZ$ blocks every path between $X$ and $Y$. We call a path with all edges oriented the same direction a directed path.

We leverage Pearl's structural causal model (SCM) formalism \cite{Pearl09}. An SCM $\M = \SCM$ consists of endogenous variables $\bV$, exogenous variables $\bU$, and structural equations $\cF$. Each $V \in \bV$ is represented by a node in the causal graph $\G$ and associated with an independently distributed exogenous variable $U_V \in \bU$. The structural equations $f_V \in \cF$ assign values of a particular node $V \in \bV$ as a function $V \defeq f_V(\pa{V}, U_V)$ of its parents and associated exogenous variable. The SCM $\M$ induces a joint distribution $\Pd{\bv}$ over the endogenous variables $\bV$. We say that an SCM $\M$ is faithful to its causal graph $\G$ if the distribution $\Pd{\bv}$ induced by $\M$ contains only the pairwise conditional independencies implied by $\G$; i.e. $X \PI Y \mid \bZ$ in the joint distribution from $\M$ iff $X$ and $Y$ are d-separated by $\bZ$ in $\G$ \cite{Spirtes2000}. As a notable special case, if $\bZ$ is empty and there exists a path from $X$ to $Y$ with no colliders then $X \nPI Y$. 

We define an intervention on a particular node $V$ to be a reassignment of the associated structural equation $f_V$. This intervention can take the form of a constant intervention $V \defeq v$, which we denote by $\doc(V = v)$ for a constant $v$ and may abbreviate to $\doc(v)$. We also define a distributional intervention, denoted by $\doc(V \sim \Pdint{v})$, where we assign $V$ to be drawn from a specified distribution $\Pdint{v}$. We denote the post-intervention SCM by $\Mi$, with an associated causal graph $\Gi$ identical to $\G$ but with incoming edges to $V$ removed. Note that reassigning the associated structural equation for any particular node $V$ induces a new distribution generated by $\Mi$ over the set of all endogenous variables $\bV$, which we denote by $\Pdi{\bv \mid \doc(V = v)}$ or $\Pdi{\bv \mid \doc(V \sim \Pdint{v})}$.


\subsection{Behavior cloning} \label{sec: imitation_learning}
Behavior cloning uses expert trajectories to train an imitating policy. For the system of interest, we use $\stdim$, $\imdim$, $\obdim$, and $\acdim$ to denote the dimensionality of the bounded state space $\stspace \subseteq \R^{\stdim}$, raw image observation space $\imspace \subseteq \R^{\imdim}$, disentangled observation space $\obspace \subseteq \R^{\obdim}$, and action space $\acspace \subseteq \R^{\acdim}$. Let $\St$, $\Im$, $\Ob$, and $\Ac$ be vector random variables taking on values in $\stspace$, $\imspace$, $\obspace$, and $\acspace$, respectively, for a discrete time step $t \in \N$. States variables $\St$ represent the intrinsic low-dimensional dynamics of the system (e.g. simulator variables) while observations $\Ob$ are distilled using a VAE-style framework from high-dimensional image measurements $\Im$, with $\imdim \gg \obdim$. The system dynamics assume that $\St[t+1]$ is strictly a function of $\St$ and $\Ac$. Lower-case script letters $\si \in [\stdim]$, $\oi \in [\obdim]$, and $\ai \in [\acdim]$ denote specific indices in the state, observation, and action vectors. For example, $\St[1][\si]$ refers to the real-valued random variable corresponding to the $\si \tth$ state variable at the first time step. We model $\Hi[1] \sim \cU(a,b)$ to be an unobserved variable capturing uncontrolled and unknown initialization stochasticity (i.e. a random ``seed'').

The collection of states, observations, and actions, along with $\Hi[1]$, comprise endogenous variables in an SCM defining our system. We denote the system SCM by $\Ms$ and denote the corresponding faithful causal graph by $\Gs$. Note that the SCM depends on the choice of policy. Since we aim to infer causalities regarding the expert policy, we generally let any causal relationships refer to the $\Ms$ and $\Gs$ induced by the expert policy unless otherwise stated. We pair the system SCM and causal graph with the tuple $\sys$. Although nodes in $\Gs$ are individual elements in our vector-valued random variables (i.e., $\St[t][\si]$ is a node, not $\St[t]$), with some abuse of notation, we let the edge symbol $\St[t] \cau X$ signify that $\St[t][\si] \cau X$ for some $\si \in [\stdim]$. Similarly, $X \cau \St[t]$ denotes that $X \cau \St[t][\si]$ for some $\si$.

% Figure environment removed

This work evaluates the importance of interventionally assigning the initial state to a particular distribution ${\St[1] \sim \Pdint{\st[1]}}$. This intervention yields a modified SCM $\Msi$ with a corresponding (not necessarily faithful) causal graph $\Gsi$, which removes the edge $\Hi[1] \to \St[1]$ in $\Gs$ (Figure~\ref{fig: structure}). We collect $N$ arbitrary-length expert trajectories from $\Msi$. The collection of all such trajectories is denoted $\trajs$. Among these $N$ trajectories, the $i^{\textrm{th}}$ trajectory consists of the tuple
\[
	\traj = \langle \st[1], \dots, \st[T];\ \im[1], \dots, \im[T];\ \ob[1], \dots, \ob[T];\ \ac[1], \dots, \ac[T] \rangle,
\]
where lowercase letters represent a concrete random variable value (to avoid confusion with indices, we use $\im$ to denote a value of $\Im$). Implicit in this definition is the existence of an \emph{encoder} $\enc: \imspace \to \obspace$ mapping each image $\im$ to a disentangled observation $\ob$. We characterize trajectories as containing observations for simplicity; our environment only provides the images $\im$, and the extraction of disentangled observations $\ob$ is method-dependent.

When training agents on $\trajs$, we parameterize policies as a neural network $\net: \imspace^L \to \acspace$. The neural policy maps some history of observations to an action $a_t$ via
\begin{align} \label{eqn: policy}
    \ac[t] = \net(\im[t], \im[t-1], \dots, \im[t-L+1]).
\end{align}
We then train $\net$ via standard behavior cloning by randomly sampling batches of images and expert actions from $\trajs$ and performing supervised regression.


\subsection{Statistical independence tests} \label{sec: hoeffding}
\newcommand{\Nhoeff}{N_{\textrm{Hoeff}}}

Our method relies on identifying whether two random variables are statistically dependent. While this is a challenging problem with a rich literature \cite{sheskin2020handbook}, in this paper, we only briefly introduce a well-known independence test for continuous distributions based on Hoeffding's D statistic \cite{hoeffding1994non, even2020independence}. Consider two real-valued random variables $X$ and $Y$ with a joint cumulative distribution function ${F(x, y) = \Pd{X \leq x, Y \leq y}}$. Hoeffding's D statistic operates on $\Nhoeff$ independent pairs of observations $\{(X_1, Y_1), \dots (X_{\Nhoeff}, Y_{\Nhoeff})\}$ and outputs a real number $D$ in the range $[-0.5, 1]$, with $D > 0$ indicating dependence. The computational complexity of calculating this statistic is $\mathcal{O} (\Nhoeff \log \Nhoeff)$. For absolutely continuous joint distributions, the D statistic is unbiased and consistent as $\Nhoeff \to \infty$, meaning that the dependence is correctly represented with probability arbitrarily close to $1$. Subsequent variations of the D statistic maintain consistency even for non-absolutely continuous joint distributions \cite{blum1961distribution}, although these complications are outside the scope of our work. We refer to the independence test based on the Hoeffding's D statistic as Hoeffding's independence test.

\section{NPA Hierarchy}
\subsection{Phrasing the Local Hamiltonian problem as an Operator Program}

Operator programs are a powerful and flexible way of stating difficult problems.  These are generally stated as the problem of optimization over non-commuting (nc) polynomials over sets of non-commuting variables ($\{a_i\}$).  In this context the variables $\{a_i\}$ are unspecified complex matrices of some finite, fixed, unspecified size ($a_i$ has the same size as $a_j$ so that multiplication is well defined).  It could be the case that the objective goes to infinity as the matrices get larger or that the objective converges to some fixed value in the limit of large matrices, but for the cases we will consider here an optimal feasible solution to the problem will consist of matrices of finite size, so the programs discussed here are all well-defined and explicitly obtain their maxima/minima.  %In fact, each of the programs described here has an explicit upper bound on the optimal matrix size which is exponential in the size of the problem.  
Depending on the convention \cite{pir10con}, one often also includes variables $\{a_i^*\}$ for denoting the complex conjugate of the matrix variables, however, in this paper these are redundant since we will always optimize over Hermitian matrices.  Polynomials in these variables will consist of linear combinations of monomials in the nc variables.  The set of monomials of degree $\leq \ell$ is denoted $\Gamma_\ell=\{a_{i_1} a_{i_2} ...a_{i_q}: q \leq \ell\}$ so an arbitrary degree-$\ell$ nc polynomial can be denoted $\theta(\{a_i\})=\sum_{\phi \in \Gamma_\ell} \theta_\phi \phi$ where $\theta_\phi \in \mathbb{C}$ for all $\phi$.  $\Gamma_\ell$ will always contain a term of degree $0$, $\mathbb{I}$.  $\mathbb{I}$ varies inside the program since it will have size matching the $\{a_i\}$ but will always denote an identity of the appropriate size.  
\begin{definition}\label{def:op_prog} Given nc polynomials $\theta$ and $\{\eta_i\}$ with $\theta^*=\theta$, an operator program $\mathcal{O}$ is an optimization problem of the following form:
% \begin{align}
% \min/\max \,\,\, \braket{\boldsymbol{\phi}| q(\{\mathbf{A}_i\})|\boldsymbol{\phi}}\\
% s.t. \,\,\,\, p_j(\{\mathbf{A}_i\})&=0 \,\, \text{  for all $j$}\\
% \braket{\boldsymbol{\phi}| \boldsymbol{\phi}}&=1\\
% \label{eq:herm_ness}\mathbf{A}_i&=\mathbf{A}_i^* \,\,\, \forall i
% \end{align}
\begin{align}
\min/\max \quad & \braket{g| \theta(\{a_i\})|g}\\
\mathrm{s.t.} \quad & \eta_j(\{a_i\}) =0 \,\, \text{  for all $j$},\\
& \label{eq:herm_ness}a_i =a_i^*,\, \forall i, \\
& \braket{g| g} =1.
\end{align}
\end{definition}

To build intuition we will first consider an ncp optimization problem where the constraints force the variables to be commuting, and hence the problem reduces to a combinatorial optimization problem.  Given a graph $(V, w)$, the {\scshape MaxCut} problem is equivalent to the following optimization problem:
\begin{align}
MC(V, w):=\max \,\, \sum_{ij} w_{ij} \frac{1-z_i z_j}{2}\,\,s.t. \,\, z_i \in \{\pm 1\} \,\, \forall \,\, i \in [n]. 
\end{align}
We could also have phrased {\scshape MaxCut} as a local Hamiltonian problem where the Local terms of the Hamiltonian are diagonal in the $Z$ basis \cite{woc03}.  In this case the largest eigenvalue would be $MC(V, w)$.  Since the matrix is diagonal, the extremal eigenvector can be assumed to be a computational basis state WLOG and this basis state provides the optimal assignment for Max Cut: 
\begin{align}\label{eq:ham_max_cut_def}
    MC(V, w)=\max \,\, \bra{g} \sum_{ij} w_{ij} \frac{\mathbb{I}-Z_i Z_j}{2} \ket{g} \,\, s.t. \,\, \braket{g|g}=1.
\end{align}
Note that the operators above are {\it not} variables, they are the explicitly defined Pauli matrices from \cref{sec:notation}.  Additionally the vector $\ket{g}$ is a vector varaible of fixed size, unlike \Cref{def:op_prog}.  A natural direction for stating \cref{eq:ham_max_cut_def} as a ncp optimization problem is ``promoting'' the actual Pauli matrices $Z_i$ to matrix variables $z_i$.  This would lead to a {\it relaxation} where the optimal solution to the operator program would be at least the solution to the relevant {\scshape MaxCut} instance. 
 To get the objectives to match we will need to explicitly enforce constraints on $z_i$ which are satisfied by $Z_i$.  We must demand that the $z_i$ commute, as well as that they square to the identity.  The resulting operator program is
\begin{align}\label{eq:op_max_cut_def}
    \max \quad & \bra{g} \sum_{ij} w_{ij} \frac{\mathbb{I}-z_i z_j}{2} \ket{g} \\ 
    \label{eq:mc1}\mathrm{s.t.}  \quad &z_i^2=\mathbb{I} \,\, \forall \,\, i \in[n],\\
     \label{eq:mc2}&z_i z_j-z_j z_i=0 \,\, \forall \,\,i, j \in [n],\\
     \label{eq:mc_end}&z_i^*=z_i\,\, \forall \,\, i \in [n],\\
     &\braket{g|g}=1.
\end{align}

 \begin{proposition}
     The program defined in \cref{eq:op_max_cut_def} has optimal objective $MC(V, w)$.
 \end{proposition}
\begin{proof}
    Let $z_i=Z_i'$ and $\ket{g}=\ket{\psi}$ be the optimal solution to \cref{eq:op_max_cut_def}.  $Z_i'$ all square to the identity and are Hermitian so they have at most two eigenvalues, $\{\pm 1\}$.  Since the $Z_i'$ all commute we can construct a basis which simultaneously diagonalizes all the $Z_i'$.  The objective is diagonal in this basis so we may assume WLOG that $\ket{\psi}$ is one of these basis elements and that $\bra{\psi} Z_i' \ket{\psi} \in \{\pm 1\}$.  Let us define $z_i'=\bra{\psi} Z_i' \ket{\psi} \in \{\pm 1\}$, so $(z_i')^2=1$.  By the eigenvector property,
    \begin{align*}
    \bra{\psi} \sum_{ij} w_{ij} \frac{\mathbb{I}-Z_i' Z_j'}{2} \ket{\psi}
    %=\sum_{ij} w_{ij} \frac{1-\bra{\psi}Z_i Z_j\ket{\psi}}{2} 
    =\sum_{ij} w_{ij} \frac{1-\bra{\psi}Z_i'\ket{\psi} \bra{\psi} Z_j'\ket{\psi}}{2} 
    =\sum_{ij} w_{ij} \frac{1-z_i' z_j'}{2},
    \end{align*}
    so the optimal objective of \cref{eq:op_max_cut_def} is less than or equal to $MC(V, w)$.  We already know that optimal objective of \cref{eq:op_max_cut_def} is greater than or equal to $MC(V, w)$ since it is a relaxation.
    
\end{proof}

Naturally we may consider a generic $2$-Local Hamiltonian problem and ask similar questions.  Arbitrary $2$-Local Hamiltonians may be written as $H=\sum_{ij} H_{ij}$ where $H_{ij}$ acts only on qubits $i$ and $j$.  We can express each $H_{ij}$ in the Pauli basis as 
\begin{equation}
    H_{ij}=\sum_{\sigma, \gamma \in \{\mathbb{I}, X, Y, Z\}} c_{\sigma, \gamma}^{ij} \,\, \sigma_i \gamma_j,
\end{equation}
for $c_{\sigma, \gamma}^{ij} \in \mathbb{R}$.  This lets us express the overall Hamiltonian as
\begin{equation}
    H=\sum_{ij} \sum_{\sigma, \gamma \in \{\mathbb{I}, X, Y, Z\}} c_{\sigma, \gamma}^{ij} \,\, \sigma_i \gamma_j.
\end{equation}
The maximum eigenvalue problem is then
\begin{equation}
   \mu_{max}(H)= \max \,\, \bra{g} \sum_{ij} \sum_{\sigma, \gamma \in \{\mathbb{I}, X, Y, Z\}} c_{\sigma, \gamma}^{ij} \,\, \sigma_i \gamma_j \ket{g} \,\, s.t. \,\, \braket{g|g}=1.
\end{equation}
We may promote the Pauli matrices above to operator variables, $X_i \rightarrow x_i, Y_i \rightarrow y_i, Z_i \rightarrow z_i$, to get a relaxation, but we will need to know what constraints to enforce to ensure that the operator problem has the same objective as the explicit local Hamiltonian problem we have in mind, just as for {\scshape MaxCut}.  Enforcing constraints of the form \cref{eq:mc1,eq:mc2,eq:mc_end} plus additional anti-commutation constraints is sufficient: 
\begin{definition}
Given a $2$-Local Hamiltonian $H$ on $n$ qubits,  

\begin{align}
\mathcal{Pauli}(H):= \max\quad & \bra{g}  \left(\sum_{ij} \sum_{\sigma, \gamma \in \{\mathbb{I}, x, y, z\}} c_{\sigma, \gamma}^{ij}\,\, \sigma_i \gamma_i\right)\ket{g}\\
\nonumber \mathrm{s.t.} \quad &\text{for all distinct $j, k\in [n]$}:\\
&\mathbb{I}=x_j^2=y_j^2=z_j^2, \\
\label{pauli_anti_commute_const} &\{x_j, y_j\}=0, \,\,\{x_j, z_j\}=0, \,\,\{y_j, z_j\}=0,\\
&a_j b_k-b_k a_j=0 \,\, \forall \,\,a, b \in \{x, y, z\},\\
&x_j^*=x_j, \,\,y_j^*=y_j, \,\,z_j^*=z_j,\\
&\braket{g|g}=1.
\end{align}
\end{definition}
\begin{proposition}[Theorem 2.3 in \cite{cha17}]
    $\mu_{max}(H)=\mathcal{Pauli}(H)$
\end{proposition}
The proof of this statement proceeds by showing that {\it any} operators which satisfy the relations above must be equal to the Pauli matrices up to overall unitary and tensoring with identity matrices.  In a sense the smallest feasible solution to $\mathcal{Pauli}(H)$ are the Pauli matrices themselves and larger solutions must have the same objective. 

\subsection{{\scshape QMaxCut} as an Operator Program}\label{subsec:qmcop}

While the $\mathcal{Pauli}$ program is very nice because of its generality, Hamiltonians are often best studied with the natural symmetry present taken into account.  Our interest is in a specific family of Local Hamiltonians known as ``Quantum Max Cut'' ({\scshape QMaxCut}) in many works \cite{gha19alm, par22opt, ans20bey, par21app}, so our aim is to produce the ``natural'' operator programs for these Hamiltonians.  Given a weighted graph $(V, w)$ with non-negative weights $w_{ij}\geq0$, the corresponding {\scshape QMaxCut} instance is defined on $n=|V|$ qubits \footnote{In later sections, $n$ is not always $|V|$ depending on the graph we focus on, which should be clear from the context.} by 
\begin{equation}\label{eq:QMCHamDef}
QMC(V, w):= \mu_{max} \left( \sum_{ij} w_{ij} H_{ij}\right),
\end{equation}
where $H_{ij} := \frac{1}{4} \left(\mathbb{I} - X_i X_j-Y_i Y_j -Z_i Z_j \right)$. 
The term $H_{ij}$ is a projector to the singlet state $|\psi^-_{ij}\rangle := (|0_i1_j\rangle - |1_i0_j\rangle)/\sqrt{2}$. Note that the singlet state is order sensitive ($|\psi^-_{ij}\rangle = - |\psi^-_{ji}\rangle$), but the Hamiltonian is not ($H_{ij}=H_{ji}$). 
This Hamiltonian has been well-studied in physics for decades, serving as central model for quantum magnetism. It has the nice property that it is rotation-invariant; that is, for any single-qubit unitary $U$, we have $(U^\dag)^{\otimes n} H_{ij} U^{\otimes n} = H_{ij}$. $H \succeq 0$ since we only consider non-negative weights $w$ ($\succeq 0$ denotes that a matrix is positive semidefinite.).  

It will be convenient for us to have a definition of another Hamiltonian which is simply an affine shift of the {\scshape QMaxCut} Hamiltonian.  If we define the usual quantum SWAP operators as 
\begin{equation}\label{eq:swap_def}
P_{ij}=\begin{bmatrix} 
1 & 0 & 0 & 0\\
0 & 0 & 1 & 0\\
0 & 1 & 0 & 0\\
0 & 0 & 0 & 1
\end{bmatrix}_{ij}=\frac{\mathbb{I} +X_i X_j +Y_i Y_j +Z_i Z_j}{2},
\end{equation}
we can then define
\begin{equation}
\SWAP(V, w)=\mu_{min} \left(\sum_{ij \in E} w_{ij} P_{ij} \right).
\end{equation}
The extremal eigenvalues are related as:
\begin{equation}
\SWAP (V, w)= \sum_{jk } w_{jk} -2 QMC (V, w).
\end{equation}

Our approach is to promote the operators $H_{ij}$ and $P_{ij}$ to variables, but we are left with the same question of deciding what constraints to include to accurately capture the local Hamiltonian problem.  Our work naturally extends that of ~\cite{par22opt}, who used such operators to obtain an optimal approximation for {\scshape QMaxCut} using product states.

One of the results of this work is showing the sufficiency of the following sets of constraints for {\scshape QMaxCut} and SWAP Hamiltoniants respectively:
\begin{comment}
By expressing the local terms in the Pauli basis we can go through a similar procedure 




optimal cut would correspond to the computational basis state which would also be the eigenvector corresponding to the largest eigenvalue, and 

The varaiables $\{\mathbf{z}_i\}$ are scalar (commuting) varaibles.  If we wanted to phrase this as an operator program the natural direction would be to promote 

We will assume that the matrices $\mathbf{A}_i$ are all Hermitian matrices of the same size (hence the product of matrices is well-defined).  The set of monomials of degree $\leq \ell$ is denoted $\Gamma_\ell=\{\mathbf{A}_{i_1} \mathbf{A}_{i_1} ...\mathbf{A}_{i_q}: q \leq \ell\}$ so an arbitrary degree-$\ell$ nc polynomial can be denoted $r(\{\mathbf{A}_i\})=\sum_{\mathbf{\Phi} \in \Gamma_\ell} r_\mathbf{\Phi} \mathbf{\Phi}$ where $r_\mathbf{\Phi} \in \mathbb{C}$ for all $\mathbf{\Phi}$.  

While their primary application is in optimizing over non-commuting operator variables, generally in the context of information, they can also be used to optimize over commuting operators.  Evaluating the solution to these commuting operator progmras is generally NP-hard.  As an example we may express the Max Cut problem as an operator program



All matrix varaibles will be assumed Hermitian, sp $(\mathbf{A}_{i_1} ... \mathbf{A}_{i_q})^*=\mathbf{A}_{i_q}^* ... \mathbf{A}_{i_1}^*=\mathbf{A}_{i_q} ... \mathbf{A}_{i_1}$, but note that degree-$\ell$ nc polynomials are not generally Hermitian (unless $\ell=1$): $r^*(\{\mathbf{A}_i\})=\sum_{\mathbf{\Phi} \in \Gamma_\ell} r_\mathbf{\Phi}^* \mathbf{\Phi}^*$.  If $\{A_i\}$ is some assignment to the variables $\{\mathbf{A}_i\}$ and $\mathbf{\Phi} =\mathbf{A}_{i_1} \mathbf{A}_{i_2} ...\mathbf{A}_{i_q} \in \Gamma_l$ then we will denote $\mathbf{\Phi}(\{A_i\})=A_{i_1} A_{i_2} ... A_{i_q}$.  A nc polynomial optimization problem is then an optimization problem with nc polynomials for the objective and constraints.  Specifically the objective will be the optimization of an extremal eigenvalue of a nc polynomial:

\noindent All of the operator programs in this paper satisfy $q^*=q$, so the objective is well-defined and equal to $\mu_{min}/\mu_{max}$ of $q(\{\mathbf{A}_i\})$.  

In our study of QMC we will be interested in three kinds of operator programs.  The first, $\mathcal{Pauli}$, is studied in all the existing works on approximation algorithms for QMC \cite{gha19alm,hwa22uni,par22opt,par21app}.  The other two operator programs, $\mathcal{Perm}$ and $\mathcal{Proj}$, are first presented in this work (although a ``low level'' of $\mathcal{Perm}$ is implicitly studied in \cite{par22opt}).

\begin{definition}
Given a weighted graph $(V, \{w\})$ on $n$ vertices,  

\begin{align}
\mathcal{Pauli}(G, w):= \,\,\max\bra{\boldsymbol{\phi}}  \left(\sum_{jk } w_{jk} \frac{\mathbb{I}-\mathbf{X}_j \mathbf{X}_k -\mathbf{Y}_j \mathbf{Y}_k- \mathbf{Z}_j \mathbf{Z}_k}{4}\right)\ket{\boldsymbol{\phi}}\\
s.t. \,\,\, \mathbb{I}=\mathbf{X}_j^2=&\mathbf{Y}_j^2=\mathbf{Z}_j^2 \,\, \forall j \in [n], \\
\label{pauli_anti_commute_const} \mathbf{X}_j \mathbf{Y}_j =&i \mathbf{Z}_j \,\, \forall j\in [n],\\
\mathbf{A}_j \mathbf{B}_k-\mathbf{B}_k \mathbf{A}_j=0 \,\, \forall \,\,\mathbf{A}, \mathbf{B} \in \{\mathbf{X}, &\mathbf{Y}, \mathbf{Z}\}, \text{ and distinct }j, k\in [n],\\
\braket{\boldsymbol{\phi}|\boldsymbol{\phi}}&=1,\\
\mathbf{X}_i^*=\mathbf{X}, \,\,\mathbf{Y}_i^*=\mathbf{Y},& \,\,\mathbf{Y}_i^*=\mathbf{Y} \,\,\, \forall i \in [n].
\end{align}
In contrast to the usual setting when $X$, $Y$ and $Z$ are fixed matrices, we emphasize that in the above program they should be thought of as unspecified matrix variables.  Note that \Cref{pauli_anti_commute_const} implies that $\mathbf{X}_j \mathbf{Y}_j +\mathbf{Y}_j \mathbf{X}_j=0$ (the usual anti-commutation constraint).  We will primarily be interested in the $\mathcal{Proj}$ and $\mathcal{Perm}$ operator programs since they are essentailly ``tailored'' to the QMC problem.  We define these programs as follows:
\end{definition}
\end{comment} 

\begin{definition}[$\mathcal{Proj}(V, w)$]  Given Hamiltonian $H=\sum_{ij} w_{ij} H_{ij}$ corresponding to graph $(V, w)$, define 
\begin{align}
\mathcal{Proj}(H)=\mathcal{Proj}(V, w):= \max \quad & \bra{g} \left(\sum_{jk} w_{jk} h_{jk}  \right) \ket{g}&\\
\nonumber \mathrm{s.t.}\quad &\forall \text{ distinct }\,\, i, j, k, l\in [n]:\\
  &h_{ij}^2 = h_{ij}, \label{eq:singproj1}\\
  &h_{ij}h_{kl} = h_{kl} h_{ij}, \label{eq:singprojcomm}\\
  &h_{ij} h_{jk} +h_{jk} h_{ij}= \frac{1}{2}(h_{ij}+h_{jk} -h_{ik}),\label{eq:anticommproj}\\
  &h_{ij}^* =h_{ij},\label{eq:singproj2}\\
  &\braket{g|g}=1.\label{eq:singprojnormalize}
\end{align}
\end{definition}

\begin{definition}[$\mathcal{Perm}(V, w)$]  Given Hamiltonian $H=\sum_{ij} w_{ij} P_{ij}$ corresponding to graph $(V, w)$, define 
\begin{align}
\mathcal{Perm}(H)=\mathcal{Perm}(V, w):=\min \quad & \bra{g} \left(  \sum_{jk} w_{jk} p_{jk}  \right)\ket{g}&\\
\nonumber \mathrm{s.t.} \quad & \forall \text{ distinct }\,\, i, j, k, l\in [n]:\\
\label{eq:sym_const_1}&p_{ij}^2=\mathbb{I},\\
\label{eq:sym_const_2}&p_{ij} p_{kl}=p_{kl}p_{ij},\\
\label{eq:anti_comm}&p_{ij} p_{jk} +p_{jk} p_{ij} =p_{ij} +p_{jk} +p_{ik}-\mathbb{I},\\
\label{eq:herm_const} &p_{ij}^*=p_{ij},\\
&\braket{g|g}=1.
\end{align}
\end{definition}

It is easy to verify that $\mathcal{Proj}$ and $\mathcal{Perm}$ are equivalent in the sense that optimal objectives of $\mathcal{Perm}$ and $\mathcal{Proj}$ are affine shifts of one another: 
\begin{equation}
\mathcal{Perm} (V, w)= \sum_{jk \in E} w_{jk} -2 \mathcal{Proj} (V, w).
\end{equation}  This can be verified by observing that if $\{P_{jk}'\}$ is a feasible solution for $\mathcal{Perm}$ then $\{(\mathbb{I}- P_{jk}')/2\}$ is a feasible solution for $\mathcal{Proj}$ and that if $\{H_{jk}'\}$ is feasible for $\mathcal{Proj}$ then $\{\mathbb{I}-2 H_{jk}'\}$ is feasible for $\mathcal{Perm}$.  

The intuition behind the $\mathcal{Perm}$ constraints can most easily be understood in the context of the representation theory of the symmetric group.  It is well known that the symmetric group has a ``finite presentation''.  Loosely speaking this means that there is a finite set of generators such that any element of the symmetric group can be written as the product of generators, and any product of elements from the symmetric group can be inferred from some finite set of multiplication rules on those generators.  Using standard notation, the symmetric group $S_n$ is generated by transpositions $(i, j)$ subject to the following rules:
\begin{align}
    \nonumber \forall \text{ distinct }\,\, i, j, k, l\in [n]:\\
\label{eq:fin_pres_1}(i, j)^2&=1,\\
\label{eq:fin_pres_2}(i, j) (k, l)&=(k, l)(i, j),\\
\label{eq:fin_pres_3}(i, j) (j, k) (i, j)&=(j, k)(i, j)(j, k).
\end{align}
Since all multiplicative identities can be derived from these rules, if we have operators $p_{ij}$ which satisfy analogous relations then multiplication of products of monomials in $\{p_{ij}\}$ must behave exactly like products of transpositions.  Hence feasible solutions $\{p_{ij}'\}$ must correspond to a representation of the symmetric group (see Appendix \ref{sec:rep_theory}).
%Note that \cref{eq:sym_const_1} corresponds to \cref{eq:fin_pres_1}, and \cref{eq:sym_const_2} corresponds to \cref{eq:fin_pres_2}, but \cref{eq:fin_pres_3} has no corresponding equation. 
Note that there is a correspondence between \cref{eq:sym_const_1} and \cref{eq:fin_pres_1} as well as between \cref{eq:sym_const_2} and \cref{eq:fin_pres_2}, but apparently none for \cref{eq:fin_pres_3}.
Instead, the operator program $\mathcal{Perm}$ contains an additional ``anti-commuting constraint'' \cref{eq:anti_comm}.  $\mathcal{Perm}$ actually has an implicit constraint corresponding to \cref{eq:fin_pres_3} (\Cref{prop:add_const}), so the constraints present enforce that the operators $p_{ij}$ ``look like'' a finite presentation of the symmetric group, plus an additional anti-commuting constraint. 
%An understanding of 
This fact is crucial for understanding why the operator programs listed are accurately capturing the relevant local Hamiltonian problems (\cref{thm:perm_is_opt}).  \Cref{eq:sym_const_1,eq:sym_const_2,eq:anti_comm,eq:herm_const}
%\jnote{Just to confirm, this one also needs Eq 35 (AC) right?}\knote{Yes, thank you for checking.  Prop 3.7 explicitly uses Hermitianness of the variables.  You have a simpler proof int eh appendix though which doesnt use it?  }\jnote{Yes, I have a shorter proof with less assumptions used for a stronger result. I replaced the proof with that because I think it's simply better and it makes things easier in \cref{app:relationderivation}. I left the previous proof in the comments.} 
force the operators to correspond to a representation of the Symmetric group, and \cref{eq:anti_comm} further forces the operators to correspond to the correct representation for the Hamiltonian.  

\begin{proposition}\label{prop:add_const}
    For all distinct $i, j, k\in [n]$, $\mathcal{Perm}$ satisfies the additional constraint 
    \begin{equation}\label{eq:totalswap}
    p_{ij} p_{jk}p_{ij}
    %=p_{jk}p_{ij}p_{jk}
    =p_{ik},
    \end{equation}
    and $\mathcal{Proj}$ satisfies the additional constraint
    \begin{equation}\label{eq:quarterformula}
    4 h_{ij}h_{jk}h_{ij}=h_{ij}.
    %=4h_{jk}h_{ij}h_{jk}-h_{jk}=0.
    \end{equation}
\end{proposition}
\begin{proof}
    From the anticommutation relation \cref{eq:anticommproj}, we can expand $h_{jk}=h_{ij}+h_{ik}-2(h_{ij}h_{ik}+h_{ik}h_{ij})$ to obtain
    \begin{eqnarray}
    h_{ij}h_{jk}h_{ij}&=&
    h_{ij}\bigl(h_{ij}+h_{ik}-2(h_{ij}h_{ik}+h_{ik}h_{ij})\bigr)h_{ij}\\
        &=&h_{ij}-3h_{ij}h_{ik}h_{ij},
    \end{eqnarray}
    where we used \cref{eq:singprojnormalize} as well. 
    Repeating the same substitution for $h_{ik}$ in the second term gives 
$h_{ij}h_{jk}h_{ij}=h_{ij}-3\bigl(h_{ij}-3h_{ij}h_{jk}h_{ij}\bigr)$,  
    which results in \cref{eq:quarterformula} after solving the linear equation. To obtain \cref{eq:totalswap}, apply the same proof with $h_{ij}=(\mathbb{I}-p_{ij})/2$. 
\end{proof}

This additional constraint \cref{eq:quarterformula} is actually one of the basic relations in the Temperley-Lieb algebra \cite{tem71rel} describing the 1-dimensional Heisenberg chain and variants. The factor of 4 could be understood in relation with other algebraic structures used for analyzing the Heisenberg model \cite{san05gro,bea06som}. 
Furthermore, an immediate corollary of \cref{prop:add_const} is that 
\begin{equation}\label{eq:sym_const_3}
p_{ij}p_{jk}p_{ij}=p_{ik}=p_{jk}p_{ij}p_{jk},
\end{equation}giving the constraint corresponding to \cref{eq:fin_pres_3}.
This lets us prove our first main result. 

\begin{comment}
\begin{proof}
By \cref{eq:anti_comm},
\begin{align}\label{eq:removing_const}
    &\quad p_{ij} p_{jk}p_{ij}-p_{jk}p_{ij}p_{jk}\\
    \nonumber &=(-p_{jk}p_{ij}+p_{ij}+p_{jk}+p_{ik}-\mathbb{I}) p_{ij}-(-p_{ij} p_{jk}+p_{ij}+p_{jk}+p_{ik}-\mathbb{I})p_{jk}\\
    \nonumber &=(p_{ij}+p_{jk}+p_{ik}) p_{ij}-(p_{ij}+p_{jk}+p_{ik})p_{jk}.
\end{align}
By applying \cref{eq:anti_comm} to every term on the expanded R.H.S we may derive:
\begin{align}
    &\quad (p_{ij}+p_{jk}+p_{ik}) p_{ij}-(p_{ij}+p_{jk}+p_{ik})p_{jk}\\
    &=-p_{ij}(p_{ij}+p_{jk}+p_{ik}) +p_{jk}(p_{ij}+p_{jk}+p_{ik})\\
    \nonumber 
    &=-((p_{ij}+p_{jk}+p_{ik}) p_{ij}-(p_{ij}+p_{jk}+p_{ik})p_{jk})^*,
\end{align}
where the last equality follows from the Hermitian assumption on the variables $p_{ij}$.  It is easily seen that the L.H.S of \cref{eq:removing_const} is Hermitian, and we have demonstrated that the R.H.S of \cref{eq:removing_const} is skew-Hermitian hence they are both zero.  For the $\mathcal{Proj}$ constraint apply the same proof with $h_{ij}=(\mathbb{I}-p_{ij})/2$.
\end{proof}
\end{comment}


\begin{theorem}\label{thm:perm_is_opt}
$\mathcal{Perm}(V, w)=\SWAP(V, w)$.
\end{theorem}
\begin{proof}
See \Cref{sec:perm_conv}.
\end{proof}

\begin{corollary}
$\mathcal{Perm}$ is a {\sf QMA}-complete operator program.  
\end{corollary}
\begin{proof}
    Determining $\SWAP(G, w)$ is {\sf QMA}-complete \cite{pid17} and by \Cref{thm:perm_is_opt} finding $\SWAP(G, w)$ is equivalent to finding $\mathcal{Perm}(G, w)$.
\end{proof}


\Cref{thm:perm_is_opt} establishes that the optimal operator variables in $\mathcal{Perm}$ and actual quantum operators in $\SWAP$ are the same, or that $\mathcal{Perm}$ can be optimized using the fixed assignments $p_{ij}=P_{ij}$ WLOG. 
Similarly, the abstract operator variables in $\mathcal{Proj}$ denoted by $h_{ij}$ become indistinguishable from the actual quantum operators of {\scshape QMaxCut} denoted by $H_{ij}$ when they are optimized according to the program. 
This is somewhat surprising, since in general, there are infinite amount of nontrivial relations between singlet projectors $H_{ij}$ that include higher order terms. The theorem implies that all of them must be derivable purely from relations in the program, namely \crefrange{eq:singproj1}{eq:singproj2} for $h_{ij}$, and \crefrange{eq:sym_const_1}{eq:herm_const} for $p_{ij}$. 
Proposition \ref{prop:add_const} could be seen as one such derivation, and we show other examples in Appendix \ref{app:relationderivation}. 
Later in section \ref{sec:exact}, we will take advantage of this equivalence, and identify $h_{ij}$ and $H_{ij}$ for practical purposes. If we explicitly add the constraints \cref{eq:sym_const_3} to the program then the above conclusions hold even if only a single anti-commuting constraint is included.  So if we only enforced $p_{12}p_{23}+p_{23}p_{12}=p_{12}+p_{23}+p_{13}-\mathbb{I}$ and did not include any other anti-commuting constraints then we would still have $\SWAP(V, w)=\mathcal{Perm}(V, w)$, essentially because a single constraint rules out all the incorrect representations.


\subsection{The Hierarchy}\label{sec:hierarchy}
The NPA hierarchy \cite{pir10con} gives a general procedure for constructing a family of semidefinite optimization programs parameterized by ``level'' which provide increasingly accurate relaxations on operator programs $\mathcal{O}$ (\Cref{def:op_prog}).  The definition given in \cite{pir10con} is more general than the one presented here we have simplified it for our particular application.  We motivate the definition of the hierarchy by considering a ``moment matrix''.  Let $\{A_i\}$, $\ket{G}$ be some feasible solution to $\mathcal{O}$.  $M_\ell$ is defined to be a complex Hermitian matrix with rows/columns indexed by elements of $\Gamma_\ell :=\{a_{i_1} a_{i_2} ...a_{i_m}: m \leq \ell\}$, the set of monomials of degree $\leq \ell$.
Entries of $M_\ell$ are defined so that 
\begin{align}
M_\ell(a_{i_1} ... a_{i_m} , a_{j_1}...a_{j_r})&:=\braket{G | (A_{i_1}...A_{i_m})^* A_{j_1} ... A_{j_r} |G }\\
\nonumber &= \braket{G| A_{i_m} ... A_{i_1} A_{j_1} ... A_{j_r}| G}.
\end{align}
By definition $M_\ell$ gives consistent values for all monomials in $\Gamma_{2\ell}$, and we can unambiguously define 
\begin{align}\label{eq:val_def}
 M_\ell( a_{i_1} ... a_{i_r} ):=\begin{cases}
     M_{\ell} (\mathbb{I}, a_{i_1} ... a_{i_r}) \text{ if $r\leq \ell$},\\
     M_\ell(a_{i_\ell} ... a_{i_1}, a_{i_{\ell+1}} ... a_{i_r}) \text{ otherwise}.
 \end{cases}   
\end{align}
Similarly for any degree-$\ell$ nc polynomials $\beta(\{A_i\})=\sum_{\phi \in \Gamma_\ell} \beta_\phi \phi(\{A_i\})$ and $\gamma(\{A_i\})=\sum_{\phi \in \Gamma_\ell}\gamma_\phi \phi(\{A_i\})$,  with $ \beta_\phi, \gamma_\phi\in \mathbb{C}$ for all $\phi$, we define 
\begin{align}
    M_\ell(\beta, \gamma):=\sum_{\phi, \phi' \in \Gamma_\ell} {\beta_\phi}^* \gamma_{\phi'} \braket{G | \phi^*(\{A_i\}) \phi'(\{A_i\})|G} = \sum_{\phi, \phi' \in \Gamma_\ell} {\beta_\phi}^* \gamma_{\phi'} M_\ell(\phi, \phi') = \sum_{\phi, \phi' \in \Gamma_\ell} (\beta_\phi)^* \gamma_{\phi'} M_\ell(\phi^*\phi') .
\end{align}
\begin{comment}
For degree-$\ell$ polynomials in $\{A_i\}$ we may define $M_\ell(\beta, \gamma)=\sum_{\phi, \phi'} \beta_\phi^* \gamma_{\phi'} \braket{G | \phi^*(\{A_i\}) \phi'(\{A_i\})|G}$.  Further, for any polynomial $\beta$ of degree-$2\ell$ we can unambiguously define $M_\ell(\beta) =\sum_{\phi \in \Gamma_{2 \ell}} \beta_{\phi}\braket{G | \phi (\{A_i\})|G}$ (for any monomial in $\theta$ of degree $>\ell$ we can split it up into two operators of size $\leq \ell$ and evaluate that term from an entry of $M_\ell$).  It can be easily verified that $M_\ell$ satisfies the following constraints:
\end{comment}
Furthermore, since the polynomials $\{\eta_k\}$ corresponding to the constraints must evaluate to zero for the variables $\{A_i\}$, we must also have that $0=\eta_k(\{A_i\})=\bra{G} \eta_k(\{A_i\}) \ket{G}=M_\ell(\eta_k)$.  More generally, feasibility for $\mathcal{O}$ implies that for any nc polynomials $\beta, \gamma$ with $\deg(\beta\eta_k\gamma)\leq 2 \ell$, 
\begin{equation}\label{eq:npa_const_0}
    M_\ell(\beta \eta_k \gamma)=0, 
\end{equation}
since $\bra{G}\beta \eta_k \gamma \ket{G}=\bra{G}\beta 0 \gamma \ket{G}$. The matrix $M_{\ell}$ also naturally satisfies 
\begin{align}\label{eq:npa_const_M}
    M_\ell=M_{\ell}^*  \text{ ~~~and~~~ }
    M_\ell \succeq 0,
\end{align}
where the latter constraint can be seen from the fact that for any complex column vector $v$, $v^* M_\ell v=M_{\ell} (\beta, \beta)=\bra{G} \beta^* \beta \ket{G}\geq 0$ for some $\beta$ that is a degree-$\ell$ nc polynomial.  

\begin{comment}
\begin{align}
M_\ell(\phi, \phi')&=M_\ell(\gamma, \gamma') \text{ if } \phi, \phi', \gamma, \gamma' \in \Gamma_\ell \\ \nonumber &\text{ and } \phi^* \phi' = \gamma^* \gamma',\\
M_\ell &=M_\ell^*, \\
M_\ell(\beta, \beta) \geq 0 \,\forall\,\,\, \text{degree-$\ell$ }&\beta \Leftrightarrow M_\ell \succeq 0,\\
\label{op_prog_const_p_i} M_\ell(\beta \eta_i \gamma)&=0 \text{ if } deg(\beta)+deg(\gamma)+deg(\eta_i) \leq 2\ell.
\end{align}

\begin{align}
\label{eq:mom_consistency}M_\ell(\phi, \phi')&=M_\ell(\phi^* \phi') \,\, \forall \,\, \phi, \phi'\in \Gamma_\ell,\\
M_\ell &=M_\ell^*, \\
M_\ell(\beta, \beta) \geq 0 \,\forall\,\,\, \text{degree-$\ell$ }&\beta \Leftrightarrow M_\ell \succeq 0,\\
\label{op_prog_const_p_i} M_\ell(\beta \eta_i \gamma)&=0 \text{ if } deg(\beta)+deg(\gamma)+deg(\eta_i) \leq 2\ell.
\end{align}
In the first set of constraints, \cref{eq:mom_consistency}, $M_\ell(\phi^* \phi')$ is meant to 

Note that $\eta_i$ are the nc polynomial constraints defined in \Cref{def:op_prog}.  \Cref{op_prog_const_p_i} applies to all polynomials $\beta$, $\gamma$ of sufficiently low degree, however linearity implies all constraints of the form \Cref{op_prog_const_p_i} can be simultaneously enforced simply by enforcing them on $\beta, \gamma\in \Gamma_{2\ell}$.  
\end{comment}

%Given an operator program $\mathcal{O}$, the $\ell$-th level of the NPA hierarchy, $NPA_\ell (\mathcal{O})$, is simply optimization of the variable $M_\ell$ subject to the above constraints without reference to a legitimate assignment to the variables.  
Given an operator program $\mathcal{O}$, the $\ell$-th level of the NPA hierarchy, denoted by $NPA_\ell (\mathcal{O})$, is the optimization of the matrix $M_\ell$ as a variable, subject to the above properties viewed as constraints, not requiring a corresponding legitimate vector $\ket{G}$ or operators $\{A_i\}$. $NPA_\ell (\mathcal{O})$ relaxes $\mathcal{O}$ since we could have used the optimal $\ket{G}, \{A_i\}$ to define $M_\ell$, so if $\mathcal{O}$ is a maximization problem, $NPA_\ell(\mathcal{O})\geq \mathcal{O}$. 
For a given monomial $a_{i_1}...a_{i_r}$, $M_\ell(a_{i_1}...a_{i_r})$ is defined as a specific entry of $M_\ell$ according to \cref{eq:val_def}. 
However, there are many other distinct entries of $M_\ell$ which will match. For example, $M_2(a_1a_2):= M_2(\mathbb{I}, a_1 a_2)$, but $ M_2(\mathbb{I}, a_1 a_2)=M_2(a_1, a_2)=M_2(a_2 a_1, \mathbb{I})$.  In $NPA_\ell$ we will force the value of $M_\ell(a_{i_1}...a_{i_r})$ to be consistent with the other distinct entires by enforcing that these other entries of $M_\ell$ are equal to the ``cannonical'' value $M_2(a_1a_2):= M_2(\mathbb{I}, a_1 a_2)$.  Note that $M_\ell$ will be {\it indexed} by variables $\phi \in \Gamma_\ell$ of the original operator program, but these $\phi$ do not vary inside $NPA_\ell$. 
\begin{definition}[$NPA_\ell (\mathcal{O})$]
%Given an operator program $\mathcal{O}$ with objective to $\max/\min $ the extremal $\max/\min$ eigenvalue of $\theta\in \text{span}(\Gamma_{2 \ell})$ subject to constraints $\{\eta_i\}$ 
Given an operator program $\mathcal{O}$ with objective to max / min the extremal eigenvalue of a nc polynomial $\theta\in \text{span}(\Gamma_{2 \ell})$ satsifying $\theta^*=\theta$ and subject to constraints $\{\eta_k\}$, define
\begin{comment}
\begin{align}
NPA_\ell(\mathcal{O}):= \max /\min \,\,&M_\ell(q)\\
\nonumber s.t. \\
\label{eq:id_const}&M_\ell(\mathbb{I})=1\\
\label{concat_const}&M_\ell(\phi, \phi')=M_\ell(\gamma, \gamma') \text{ if } \phi, \phi', \gamma, \gamma' \in \Gamma_\ell \\ \nonumber &\text{ and } \phi^* \phi' = \gamma^* \gamma'\\
\label{npa_const_p_i} &M_\ell (\beta \eta_i \gamma)=0 \,\,\,\,\forall \beta, \gamma \in \Gamma_{2 \ell}, \eta_i: deg(\eta_i )+deg(\beta)+deg(\gamma) \leq 2\ell\\
&M_\ell\succeq 0, Hermitian
\end{align}
\end{comment}
\begin{align}
NPA_\ell(\mathcal{O}):= \max /\min \quad & M_\ell(\theta)\\
\label{eq:id_const}\mathrm{s.t.}\quad & M_\ell(\mathbb{I})=1,\\
\label{concat_const}&M_\ell(\phi, \phi')=M_\ell(\phi^* \phi') ~~~~ \forall  \phi, \phi'\in \Gamma_\ell,\\
%\label{npa_const_p_i} &M_\ell (\beta \eta_k \gamma)=0 ~~~~ \forall \beta, \gamma \in \Gamma_{2 \ell}, \eta_k: \deg(\eta_k )+\deg(\beta)+\deg(\gamma) \leq 2\ell,\\
\label{npa_const_p_i} &M_\ell (\beta \eta_k \gamma)=0 ~~~~ \forall \beta, \gamma \in \Gamma_{2 \ell}, \eta_k: \deg(\beta\eta_k\gamma)\leq 2 \ell,\\
&M_\ell\succeq 0, ~\mathrm{Hermitian}.
\end{align}
\end{definition}
\begin{comment}
It is important to note that if a constraint polynomial has $deg(\eta_i) > 2 \ell$ then it is \textbf{not} included in $NPA_\ell$ (we ignore that constraint for that level).  
\end{comment}
In several works \cite{par21app, ans20bey, par22opt} a ``real version'' of $NPA_\ell$ is studied, $NPA_\ell^{\mathbb{R}}$.  The key difference is that $NPA_\ell^{\mathbb{R}}$ takes only the real portion of the moment matrix, effectively ``zeroing out'' skew-Hermitian polynomials.
\begin{definition}[$NPA_\ell^{\mathbb{R}} (\mathcal{O})$]
Given an operator program $\mathcal{O}$ with objective to max / min the extremal eigenvalue of a nc polynomial $\theta\in \text{span}(\Gamma_{2 \ell})$ satsifying $\theta^*=\theta$ and subject to constraints $\{\eta_k\}$, define\begin{align}
NPA_\ell^{\mathbb{R}} (\mathcal{O}):= \max /\min \quad &M_\ell^{\mathbb{R}}(\theta)\\
\mathrm{s.t.} \quad & M_\ell^{\mathbb{R}} (\mathbb{I})=1,\\
&M_\ell^{\mathbb{R}} (\phi, \phi')=M_\ell^{\mathbb{R}} (\phi^* \phi') ~~~~ \forall  \phi, \phi'\in \Gamma_\ell,\\
&\label{npa_const_p_i_real}M_\ell^{\mathbb{R}} ((\beta \eta_k \gamma + \gamma^* \eta_k^* \beta^*)/2)=0 ~~~~ \forall \beta, \gamma \in \Gamma_{2 \ell}, \eta_k: \deg(\beta\eta_k\gamma )\leq 2\ell,\\
%&\label{npa_const_p_i_real}M_\ell^{\mathbb{R}} ((\beta \eta_k \gamma + \gamma^* \eta_k^* \beta^*)/2)=0 ~~~~ \forall \beta, \gamma \in \Gamma_{2 \ell}, \eta_k: \deg(\eta_k )+\deg(\beta)+\deg(\gamma) \leq 2\ell,\\
%&M_\ell ~~\mathrm{Hermitian},\\ 
%&M_\ell^{\mathbb{R}}=\frac{1}{2}(M_\ell+(M_\ell^*)^T),\\
&M_\ell^\mathbb{R} \succeq 0, ~\mathrm{Symmetric}.
\end{align}
\end{definition}
As noted in other works \cite{bra19, gha19alm}, $ NPA_\ell^\mathbb{R}(\mathcal{O})$ is a relaxation on $NPA_\ell(\mathcal{O})$ ($NPA_\ell(\mathcal{O}) \leq NPA_\ell^\mathbb{R}(\mathcal{O})$ if $\mathcal{O}$ is a maximization problem) since given a feasible $M_\ell$ for $NPA_\ell(\mathcal{O})$, we can take $M_\ell^\mathbb{R}:=(M_\ell+(M_\ell^*)^T)/2$ to obtain a feasible solution to $NPA_\ell^\mathbb{R}$ with the same objective, but not necessarily the other way around. 
%$NPA_\ell^\mathbb{R}(\mathcal{O})$ is a relaxation on $NPA_\ell(\mathcal{O})$, so it generally upper/lower bounds $NPA_\ell(\mathcal{O})$ for max/min problems. 
We show explicit examples in \cref{subsubsec:smallstat} where the separation is strict for the $\mathcal{Pauli}$ case. 
%However, for the specific operator program we focus on most in this paper $\mathcal{Proj}$, this is a distinction without a difference:
However, for the specific SDP we focus on most in this paper, $NPA_1(\mathcal{Proj})$, this is a distinction without a difference:
\begin{proposition}
    For any weighted graph $(V, w)$, $NPA_1(\mathcal{Proj}(V, w))=NPA_1^\mathbb{R}(\mathcal{Proj}(V, w))$.
\end{proposition}
\begin{proof}
     Let $M_1^\mathbb{R}$ be a feasible solution to $NPA_1^\mathbb{R}(\mathcal{Proj})$.  We claim that $M_1^\mathbb{R}$ is feasible for $NPA_1(\mathcal{Proj})$ and hence $NPA_1^\mathbb{R}(\mathcal{Proj}) \leq NPA_1(\mathcal{Proj})$.  This is demonstrated by showing that all the constraints $\eta$ enforced on $M_1^\mathbb{R}$ (constraints of the form \Cref{npa_const_p_i_real}) are Hermitian and hence $M_1^\mathbb{R}(\eta+\eta^*)=0$ implies $M_1^\mathbb{R}(\eta)=0$.  It is easy to see that the nc polynomials $\eta$ of the form $h_{ij}^2-h_{ij}$ and $h_{ij}h_{jk}+h_{jk}h_{ij}-(h_{ij}+h_{jk}-h_{ik})/2$ are Hermitian, hence $M_1^\mathbb{R}(\eta)=0$ for these polynomials.  Since we are looking at moment matrices with a maximum degree $2$ if $\beta \eta \gamma$ has max degree $2$ for $\eta$ one of the polynomials above $\beta=\mathbb{I}=\gamma$ so there are no other constraints to check and $M_1^\mathbb{R}$ must be feasible for $NPA_1(\mathcal{Proj})$
     
     %This is shown by demonstrating that all the constraints enforced on $M_1^\mathbb{R}$ are actually Hermitian polynomials, The constraint $\eta=h_{ij}-h_{ij}^2=0$ is Hermitian ($\eta^*=\eta$) and we know $M_1^\mathbb{R}(\eta+\eta^*)=0$ by the constraints of $NPA_1^\mathbb{R}$ so $M_1^\mathbb{R}(\eta)=0$. Since $M_1$ is Hermitian and by the constraints $M_1(h_{ij})=M_1(h_{ij}^2)=M_1(h_{ij}, h_{ij})\in \mathbb{R}$, $M_1^\mathbb{R}(h_{ij}^2)=M_1^\mathbb{R}(h_{ij})$ for all distinct $i, j$.  Similarly, $M_1(h_{ij}, h_{kl})=M_1(h_{kl}, h_{ij})$ by the constraints, and since $M_1$ is Hermitian $M_1(h_{ij}, h_{kl})\in \mathbb{R}$, $M_1^\mathbb{R}(h_{ij}, h_{kl})=M_1^\mathbb{R}(h_{kl}, h_{ij})$ for all distinct $i, j, k, l$.  Lastly, 
    %\begin{equation}
    %        M_1(h_{ij}, h_{jk})+M_1(h_{jk}, h_{ij})=\frac{1}{2}(M_1(h_{ij})+M_1(h_{jk})-M_1(h_{ik})).
    %\end{equation}
    %$M_1$ being Hermitian implies that the right hand side is strictly real, so the L.H.S must also be strictly real.  Hence the above also holds with $M_1^\mathbb{R}$ in place of $M_1$.  We have demonsrataed that $M_1^\mathbb{R}$ satisfies all the constraints of $NPA_1(\mathcal{Proj})$.    
\end{proof}


Each $NPA_\ell$ can be solved via semidefinite programming, and the dual set of programs is called the {\it Sum of Squares} (SoS) hierarchy due to the interpretation of the hierarchy as an optimization over {\it sum of squares proofs}.  To motivate this imagine we are trying to maximize an nc polynomial $\theta$ and obtain an expression of the form
\begin{equation}\label{eq:sos_mot}
    \lambda \mathbb{I}-\theta=\sum_i \psi_i^* \psi_i+\sum_j \beta_j \eta_j \gamma_j,
\end{equation}
where $\lambda\in \mathbb{R}$, $\psi_i\in \mathrm{span}_\mathbb{C}(\Gamma_\ell)$ for all $i$ and $\deg(\beta_j\eta_j \gamma_j)\leq 2 \ell$ for all $j$.  Since $\eta_k$ are constraints of $\mathcal{O}$ we must have that 
\begin{equation}
    \lambda \mathbb{I}-\theta=\sum_i \psi_i^* \psi_i 
\end{equation}
for {\it any} feasible solution to $\mathcal{O}$.  $\sum_i \psi^* \psi_i$ is manifestly $\succeq 0$ so for all feasible solutions $\bra{G} \lambda \mathbb{I}-\theta\ket{G} \geq 0\Rightarrow \lambda \geq \bra{G} \theta \ket{G}$ which implies $\lambda \geq NPA_\ell(\mathcal{O})$.  An expression of the form \cref{eq:sos_mot} is generally referred to as a {\it sum of squares proof}.  The SoS optimization problem at level $\ell$ is to find the smallest $\lambda$ such that $\lambda \mathbb{I}-\theta$ can be deformed via the constraints to an expression of the form $\sum_i \psi^* \psi$: $\lambda \mathbb{I}-\theta-\sum_j \beta_j \eta_j \gamma_j =\sum_i \psi_i^* \psi_i$.  Crucially, the constraints that the SoS proof uses are of degree at most $\ell$ at the corresponding level ($\deg(\beta_j\eta_j\gamma_j)\leq 2\ell$).  For ease of reference, define the set of constraint deformation polynomials as
\begin{equation}
    U^\ell(\{\eta_k\}):= \mathrm{span}_\mathbb{C}\{\beta \eta_k \gamma: \beta, \gamma\in \Gamma_\ell, \deg(\beta\eta_k\gamma)\leq 2\ell\}
\end{equation}


\begin{comment}
In the interest of providing an explicit SDP for which the dual can be easily verified we will define a number of matrices.  These will be all be ``constraint matrices'' used to enforce  \Cref{eq:id_const}-\Cref{npa_const_p_i}, so they will all have rows and columns indexed by $\Gamma_\ell$.  For ease of reference let $U_1^\ell:=\{(\mathbf{\Phi}, \mathbf{\Phi}', \mathbf{\Theta}, \mathbf{\Theta}')\in \Gamma_\ell \times \Gamma_\ell \times\Gamma_\ell \times \Gamma_\ell: \mathbf{\Phi}^* \mathbf{\Phi}'=\mathbf{\Theta}^* \mathbf{\Theta}'\}$, and let $U_2:= \{ p\in span(\Gamma_{2\ell}): \,\,p=r p_i s \text{ with } r, s\in \Gamma_{2\ell}  \text{ and } deg(s)+deg(r)+deg(p_i) \leq 2\ell\}$.  Let
\begin{equation}\label{eq:b_I}
    B_\mathbb{I}(\mathbf{\Phi}, \mathbf{\Phi}')=\begin{cases}
    1 \text{  if } \mathbf{\Phi}=\mathbb{I}=\mathbf{\Phi}'\\
    0 \text{   otherwise}
    \end{cases}.
\end{equation}
For all $(\mathbf{\Phi}, \mathbf{\Phi}', \mathbf{\Theta}, \mathbf{\Theta}')\in U_1$ define $C_{(\mathbf{\Phi}, \mathbf{\Phi}', \mathbf{\Theta}, \mathbf{\Theta}')}, C_{(\mathbf{\Phi}, \mathbf{\Phi}', \mathbf{\Theta}, \mathbf{\Theta}')}'$\footnote{We will require matrices $C_{(\mathbf{\Phi}, \mathbf{\Phi}', \mathbf{\Theta}, \mathbf{\Theta}')}$ and  $C_{(\mathbf{\Phi}, \mathbf{\Phi}', \mathbf{\Theta}, \mathbf{\Theta}')}'$ to enforce \Cref{concat_const} using only \textbf{Hermitian} constraint matrices .} according to:
\begin{equation}
    C_{(\mathbf{\Phi}, \mathbf{\Phi}', \mathbf{\Theta}, \mathbf{\Theta}')}(\mathbf{\Psi}, \mathbf{\Psi}')=\begin{cases}
    1 \text{  if  $\mathbf{\Psi}=\mathbf{\Phi} $ and $\mathbf{\Psi}'=\mathbf{\Phi}'$}\\
    1 \text{  if  $\mathbf{\Psi}=\mathbf{\Phi}' $ and $\mathbf{\Psi}'=\mathbf{\Phi}$}\\
    -1 \text{  if  $\mathbf{\Psi}=\mathbf{\Theta} $ and $\mathbf{\Psi}'=\mathbf{\Theta}'$}\\
    -1 \text{  if  $\mathbf{\Psi}=\mathbf{\Theta}' $ and $\mathbf{\Psi}'=\mathbf{\Theta}$}\\
    0 \text{  otherwise}\\
    \end{cases},
\end{equation}
and 
\begin{equation}
    C_{(\mathbf{\Phi}, \mathbf{\Phi}', \mathbf{\Theta}, \mathbf{\Theta}')}'(\mathbf{\Psi}, \mathbf{\Psi}')=\begin{cases}
    i \text{  if  $\mathbf{\Psi}=\mathbf{\Phi} $ and $\mathbf{\Psi}'=\mathbf{\Phi}'$}\\
    -i \text{  if  $\mathbf{\Psi}=\mathbf{\Phi}' $ and $\mathbf{\Psi}'=\mathbf{\Phi}$}\\
    -i \text{  if  $\mathbf{\Psi}=\mathbf{\Theta} $ and $\mathbf{\Psi}'=\mathbf{\Theta}'$}\\
    i \text{  if  $\mathbf{\Psi}=\mathbf{\Theta}' $ and $\mathbf{\Psi}'=\mathbf{\Theta}$}\\
    0 \text{  otherwise}\\
    \end{cases}.
\end{equation}
For all $p\in U_2$ define $D_p, D_p'$ as follows:  Let $p=\sum_{\mathbf{\Phi}\in \Gamma_{2\ell}}c_\mathbf{\Phi} \mathbf{\Phi}$ for $c_\mathbf{\Phi} \in \mathbb{C}$ for all $\mathbf{\Phi}$.  For all $\mathbf{\Phi} \in \Gamma_{2\ell}$ choose a pair $(\mathbf{\Theta}_\mathbf{\Phi}, \mathbf{\Theta}_{\mathbf{\Phi}}')\in \Gamma_\ell \times \Gamma_\ell$ such that $\mathbf{\Theta}_\mathbf{\Phi}^* \mathbf{\Theta}_\mathbf{\Phi}'=$ Then,
\begin{equation}
    D_p(\mathbf{\Psi}, \mathbf{\Psi}')=\begin{cases}
        c_\mathbf{\Phi} \text{  if $\mathbf{\Psi}=\mathbf{\Theta}_\mathbf{\Phi}$ and $\mathbf{\Psi}'=\mathbf{\Theta}_\mathbf{\Phi}'$}\\
        c_\mathbf{\Phi}^* \text{  if $\mathbf{\Psi}=\mathbf{\Theta}_\mathbf{\Phi}'$ and $\mathbf{\Psi}'=\mathbf{\Theta}_\mathbf{\Phi}$}\\
        0 \text{  otherwise}
    \end{cases},
\end{equation}
and 
\begin{equation}\label{eq:dp_p}
    D_p(\mathbf{\Psi}, \mathbf{\Psi}')'=\begin{cases}
        i c_\mathbf{\Phi} \text{  if $\mathbf{\Psi}=\mathbf{\Theta}_\mathbf{\Phi}$ and $\mathbf{\Psi}'=\mathbf{\Theta}_\mathbf{\Phi}'$}\\
        -i c_\mathbf{\Phi}^* \text{  if $\mathbf{\Psi}=\mathbf{\Theta}_\mathbf{\Phi}'$ and $\mathbf{\Psi}'=\mathbf{\Theta}_\mathbf{\Phi}$}\\
        0 \text{  otherwise}
    \end{cases}.
\end{equation}
Using these matrices and denoting $A\cdot B:= Tr(A^* B)$, $NPA_\ell$ can be succinctly written as:
\begin{align}
\label{eq:npa_as_sdp} NPA_\ell(\mathcal{O})= \max /\min \,\,&\mathbf{M}_\ell \cdot \frac{1}{2} D_q\\
\nonumber s.t. \\
 \label{eq:sdp_const1}&B_\mathbb{I} \cdot \mathbf{M}_\ell =1\\
 \label{eq:sdp_const2}&C_{(\mathbf{\Phi}, \mathbf{\Phi}', \mathbf{\Theta}, \mathbf{\Theta}')} \cdot \mathbf{M}_\ell=0 \text{  for all $(\mathbf{\Phi}, \mathbf{\Phi}', \mathbf{\Theta}, \mathbf{\Theta}')\in U_1$}\\
 \label{eq:sdp_const3}&C_{(\mathbf{\Phi}, \mathbf{\Phi}', \mathbf{\Theta}, \mathbf{\Theta}')}' \cdot \mathbf{M}_\ell=0 \text{  for all $(\mathbf{\Phi}, \mathbf{\Phi}', \mathbf{\Theta}, \mathbf{\Theta}')\in U_1$}\\
 \label{eq:sdp_const4}&D_p \cdot \mathbf{M}_\ell =0 \text{ for all $p \in U_2$}\\
 \label{eq:sdp_const5}&D_p' \cdot \mathbf{M}_\ell =0 \text{ for all $p\in U_2$}\\ 
\nonumber & \mathbf{M}_\ell\succeq 0, Hermitian
\end{align}
Note that \Cref{eq:sdp_const1} enforces \Cref{eq:id_const}, \Cref{eq:sdp_const2}-\Cref{eq:sdp_const3} enforce \Cref{concat_const}, and \Cref{eq:sdp_const4}-\Cref{eq:sdp_const5} enforce \Cref{npa_const_p_i}.  Assuming $\mathcal{O}$ is a maximization problem, the dual of \Cref{eq:npa_as_sdp} is 
\begin{gather}
    \min \boldsymbol{\lambda}\\
    \nonumber s.t.\\
    \boldsymbol{\lambda} B_\mathbb{I}+\sum_{(\mathbf{\Phi}, \mathbf{\Phi}', \mathbf{\Theta}, \mathbf{\Theta}')\in U_1} \mathbf{y}_{(\mathbf{\Phi}, \mathbf{\Phi}', \mathbf{\Theta}, \mathbf{\Theta}')} C_{(\mathbf{\Phi}, \mathbf{\Phi}', \mathbf{\Theta}, \mathbf{\Theta}')}+\mathbf{y}_{(\mathbf{\Phi}, \mathbf{\Phi}', \mathbf{\Theta}, \mathbf{\Theta}')}' C_{(\mathbf{\Phi}, \mathbf{\Phi}', \mathbf{\Theta}, \mathbf{\Theta}')}'+\sum_{p \in U_2} \mathbf{z}_p D_p +\mathbf{z}_p' D_p' \succeq \frac{1}{2} D_q \\
    \mathbf{y}_{(\mathbf{\Phi}, \mathbf{\Phi}', \mathbf{\Theta}, \mathbf{\Theta}')}, \mathbf{y}_{(\mathbf{\Phi}, \mathbf{\Phi}', \mathbf{\Theta}, \mathbf{\Theta}')}', \mathbf{z}_p, \mathbf{z}_p' \in \mathbb{R} \,\, \forall \,\,(\mathbf{\Phi}, \mathbf{\Phi}', \mathbf{\Theta}, \mathbf{\Theta}') \in U_1, \,\, p \in U_2,
\end{gather}
which is equivalent to:
\begin{gather}
    \min \boldsymbol{\lambda}\\
    \nonumber s.t.\\
    \boldsymbol{\lambda} B_\mathbb{I}- \frac{1}{2}D_q = \mathbf{S}+\mathbf{T}\\
    \mathbf{T}=\sum_{(\mathbf{\Phi}, \mathbf{\Phi}', \mathbf{\Theta}, \mathbf{\Theta}')\in U_1} \mathbf{y}_{(\mathbf{\Phi}, \mathbf{\Phi}', \mathbf{\Theta}, \mathbf{\Theta}')} C_{(\mathbf{\Phi}, \mathbf{\Phi}', \mathbf{\Theta}, \mathbf{\Theta}')}+\mathbf{y}_{(\mathbf{\Phi}, \mathbf{\Phi}', \mathbf{\Theta}, \mathbf{\Theta}')}' C_{(\mathbf{\Phi}, \mathbf{\Phi}', \mathbf{\Theta}, \mathbf{\Theta}')}'+\sum_{p \in U_2} \mathbf{z}_p D_p +\mathbf{z}_p' D_p'\\
    \mathbf{y}_{(\mathbf{\Phi}, \mathbf{\Phi}', \mathbf{\Theta}, \mathbf{\Theta}')}, \mathbf{y}_{(\mathbf{\Phi}, \mathbf{\Phi}', \mathbf{\Theta}, \mathbf{\Theta}')}', \mathbf{z}_p, \mathbf{z}_p' \in \mathbb{R} \,\, \forall \,\,(\mathbf{\Phi}, \mathbf{\Phi}', \mathbf{\Theta}, \mathbf{\Theta}') \in U_1, \,\, p \in U_2\\
    \mathbf{S}\succeq 0, \,\,Hermitian.
\end{gather}

The above program can be interpreted as trying to find the smallest $\boldsymbol{\lambda}$ such that $\boldsymbol{\lambda} B_\mathbb{I}- \frac{1}{2}D_q$ may be deformed to a PSD matrix $\mathbf{S}$ using constraint matrices of $NPA_\ell(\mathcal{O})$.  $\mathbf{T}$ is such a ``constraint deformation'' matrix.  We will further simplify the above to make the interpretation more clear.  Let $(\lambda, S, T)$ be a feasible solution.  $S$ is PSD so it has a decomposition $S=\sum_i v_i^* v_i$ where $v_i$ are unnormalized complex row vectors.  Let $\mathbf{w}$ be a vector of variables indexed by elements of $\Gamma_\ell$ and define $s_i=v_i \cdot{} \mathbf{w}=\sum_{\mathbf{\Phi} \in \Gamma_\ell} (v_i)_{\mathbf{\Phi}} \,\, \mathbf{\Phi}$.  We can evaluate\knote{May want a notation which makes it clear polynomials are a function of the variables but are not variate themselves.  Also double check the below, could be off by a factor of $2$}:
\begin{align}
    \mathbf{w}^* \left( \lambda B_\mathbb{I}- \frac{1}{2}D_q \right) \mathbf{w}=\lambda \mathbb{I}-q(\{\mathbf{A}_i\})\\
    \nonumber \text{and}\\
    \mathbf{w}^* \left( S+T\right) \mathbf{w}=\sum_i s_i^* s_i+\sum_{(\mathbf{\Phi}, \mathbf{\Phi}', \mathbf{\Theta}, \mathbf{\Theta}')\in U_1}y_{(\mathbf{\Phi}, \mathbf{\Phi}', \mathbf{\Theta}, \mathbf{\Theta}')} (\mathbf{\Phi}^* \mathbf{\Phi}'-\mathbf{\Theta}^* \mathbf{\Theta}'+(\mathbf{\Phi}')^*\mathbf{\Phi} -(\mathbf{\Theta}')^* \mathbf{\Theta})\\
     \nonumber + y_{(\mathbf{\Phi}, \mathbf{\Phi}', \mathbf{\Theta}, \mathbf{\Theta}')}' i(\mathbf{\Phi}^* \mathbf{\Phi}'- \mathbf{\Theta}^* \mathbf{\Theta}'+(\mathbf{\Theta}')^* \mathbf{\Theta}- (\mathbf{\Phi}')^* \mathbf{\Phi})+\sum_{p\in U_2}z_p (p+p^*)+z_p i (p-p^*).
\end{align}
Note that for any feasible solution to $\mathcal{O}$ all of the terms in $\mathbf{w}^* T \mathbf{w}$ will evaluate to zero, so a feasible solution to $NPA_\ell$ provides a way to write $\lambda \mathbb{I}-q=\sum_i s_i^* s_i$ where equality is defined modulo the constraints.  Since $\sum_i s_i^* s_i$ will always be PSD this provides an algebraic proof that $\lambda \geq \mu_{max}(q)$ independent of the values that the variables $\{\mathbf{A}_i\}$ take.  The set of duals to programs in $NPA_\ell$ is generally referred to as a ``Sum of Squares'' hierarhcy for this reason.  It may be interpreted as looking for an algebraic deformation of $\lambda \mathbb{I}- q$ to a sum of squares via the constraints which is then guaranteed to be $\succeq 0$.  Let us define $U^\ell$ as the real span of all terms of the form 
\begin{align}
   (\mathbf{\Phi}^* \mathbf{\Phi}'-\mathbf{\Theta}^* \mathbf{\Theta}'+(\mathbf{\Phi}')^*\mathbf{\Phi} -(\mathbf{\Theta}')^* \mathbf{\Theta}) \text{ for }  (\mathbf{\Phi}, \mathbf{\Phi}', \mathbf{\Theta}, \mathbf{\Theta}')\in U_1^\ell,\\
   i(\mathbf{\Phi}^* \mathbf{\Phi}'- \mathbf{\Theta}^* \mathbf{\Theta}'+(\mathbf{\Theta}')^* \mathbf{\Theta}- (\mathbf{\Phi}')^* \mathbf{\Phi}) \text{ for }  (\mathbf{\Phi}, \mathbf{\Phi}', \mathbf{\Theta}, \mathbf{\Theta}')\in U_1^\ell,\\
   (p+p^*) \text{ for } p \in U_2^\ell, \\
   \text{ and } i(p-p^*) \text{ for } p \in U_2^\ell.
\end{align}


Each $NPA_\ell(\mathcal{O})$ may be expressed as a semidefinite program (SDP), so we may construct a dual hierarchy where each level corresponds to the SDP dual of $NPA_\ell(\mathcal{O})$.  This ``dual hierarchy'' is generally referred to as the ``Sum of Squares'' (SoS) hierarchy and can be interpreted as searching for a degree$-\ell$ nc polynaomil in $span(\Gamma_\ell)$ which provides an explicit (rigorous) bound on the objective of the polynomial $q$.  Each level in the dual hierarchy will use constraints present at that level to write $\boldsymbol{\lambda} \mathbb{I}-q$ as a nc polynomial of the form $\sum_i s_i^* s_i+ t$ for $s_i \in span(\Gamma_\ell)$ for all $i$ and $t$ a linear combination of constraints from $NPA_\ell(\mathcal{O})$.  Since $t$ will be zero for any feasible solution and $\sum_i s_i^* s_i \succeq 0$, this implies that $\boldsymbol{\lambda} \geq \mu_{max}(q)$.  Define the following lists of operators:
\begin{align}
    U_1^\ell=[\mathbf{\Phi}^* \mathbf{\Phi}'-\mathbf{\Theta}^* \mathbf{\Theta}'+(\mathbf{\Phi}')^* \mathbf{\Phi}-(\mathbf{\Theta}')^* \mathbf{\Theta}: \mathbf{\Phi}, \mathbf{\Phi'}, \mathbf{\Theta}, \mathbf{\Theta}' \in \Gamma_\ell \text{ with } \mathbf{\Phi}^* \mathbf{\Phi}'=\mathbf{\Theta}^* \mathbf{\Theta}']\\
    U_2^\ell=[i(\mathbf{\Phi}^* \mathbf{\Phi}'-\mathbf{\Theta}^* \mathbf{\Theta}'-(\mathbf{\Phi}')^* \mathbf{\Phi}+(\mathbf{\Theta}')^* \mathbf{\Theta}): \mathbf{\Phi}, \mathbf{\Phi'}, \mathbf{\Theta}, \mathbf{\Theta}' \in \Gamma_\ell \text{ with } \mathbf{\Phi}^* \mathbf{\Phi}'=\mathbf{\Theta}^* \mathbf{\Theta}']\\
    U_3^\ell=[r p_j +(r p_j)^*: r\in \Gamma_{2 \ell},\,\, p_j \text{ is a constraint of $\mathcal{O}$}, \,\,deg(r)+deg(p_j)\leq 2\ell]\\
    U_4^\ell=[i(r p_j -(r p_j)^*): r\in \Gamma_{2 \ell},\,\, p_j \text{ is a constraint of $\mathcal{O}$}, \,\,deg(r)+deg(p_j)\leq 2\ell]
\end{align}
\knote{I dont know howq to say waht I want, but it's technically important that these be lists rather than sets.}  Let us further define $U^\ell=span_\mathbb{R}(U_1^\ell \cup U_2^\ell \cup U_3^\ell \cup U_4^\ell)$ as the set of nc polynomials corresponding to all real linear combinations of the constraints of $\mathcal{O}$.
\end{comment}


\begin{definition}[$SoS_\ell(\mathcal{O})$]\label{def:SoS}
Given an operator program $\mathcal{O}$ with objective to max/min the extremal eigenvalue of $\theta$ subject to the constraints $\{\eta_k\}$,
\begin{align}
    SoS_\ell(\mathcal{O}):= \min/\max \quad & \lambda\\
    \mathrm{s.t.}\quad & \sum_i \psi_i^* \psi_i +\beta =\begin{cases} \lambda \mathbb{I}-\theta \text{ if $\mathcal{O}$ is a maximization problem}\\ \theta-\lambda \mathbb{I} \text{ if $\mathcal{O}$ is a minimization problem} \end{cases},\\
    &\psi_i\in \mathrm{span}_\mathbb{C}(\Gamma_\ell) \,\, \forall i,\\
    &\beta \in U^\ell(\{\eta_k\}),\\
    &\lambda \in \mathbb{R}.
\end{align}
\end{definition}

\textbf{Equivalence}.  SDPs for $\mathcal{Perm}$ and $\mathcal{Proj}$ are equivalent since a feasible solution for $NPA_\ell(\mathcal{Perm})$ can be used to construct a feasible solution to $NPA_\ell(\mathcal{Proj})$ and vice versa.  The reasoning for this is analogous to the reasoning behind the operator programs $\mathcal{Perm}$ and $\mathcal{Proj}$ being equivalent: Given $M_\ell$ feasible for $NPA_\ell$ one can construct $M_\ell'$ feasible for $NPA_\ell(\mathcal{Proj})$ according to
$$
%M_\ell'(h_{i_1, j_1}  ... h_{i_a, j_a}, h_{k_1, m_1} ... h_{k_b, m_b}):= M_\ell\left(\frac{\mathbb{I}+p_{i_1, j_1}}{2}...\frac{\mathbb{I}+p_{i_a, j_a}}{2}, \frac{\mathbb{I}+p_{k_1, m_1}}{2} ... \frac{\mathbb{I}+p_{k_b, m_b}}{2} \right).
M_\ell'(h_{i j}  ... h_{k l}, h_{m n} ... h_{o p}):= M_\ell\left(\frac{\mathbb{I}+p_{i j}}{2}...\frac{\mathbb{I}+p_{k l}}{2}, \frac{\mathbb{I}+p_{m n}}{2} ... \frac{\mathbb{I}+p_{o p}}{2} \right).
$$
Note that the R.H.S is evaluated using the expression for $M_\ell(\beta, \gamma)$ for nc polynomials $\beta$ and $\gamma$ as in \cref{sec:hierarchy}.  The primal/dual pair $NPA_\ell/SoS_\ell$ satisfies strong duality, so it holds that $NPA_\ell(\mathcal{Perm})=SoS_\ell(\mathcal{Perm})$, $NPA_\ell(\mathcal{Proj})=SoS_\ell(\mathcal{Proj})$, and that the objectives of the two sets of programs differ by a known affine shift.  

\textbf{Convergence.  }It is simple to check that any $\mathcal{Perm}$ satisfies a boundedness condition (it is {\it Archimedean} \cite{pir10con}).  For this we need to show the existence of a constant $C$ such that $C^2 \mathbb{I}-\sum_{i<j} P_{ij}\succeq 0$ for any feasible assignment to the variables $p_{ij}=P_{ij}$.  Since $P_{ij}^2=\mathbb{I}$, $P_{ij} \preceq \mathbb{I}$ so $C^2 \mathbb{I}-\sum_{i<j} P_{ij}\succeq \left(C^2-\binom{n}{2}\right) \mathbb{I}$.  Hence we may choose $C=\sqrt{\binom{n}{2}}$.  The results of \cite{pir10con} then imply that $NPA_{\ell}(\mathcal{Perm})$ converges to the optimal objective of $\mathcal{Perm}$ in the limit of large $\ell$.  In fact, the constraints present are strong enough to guarantee finite convergence of $NPA_\ell$.  Essentially the proof of this statement involves showing that moment matrices will satisfy the ``rank condition'' of \cite{pir10con} hence an optimal operator solution can be constructed from the optimal solution to $NPA_\ell(\mathcal{Perm})$ at some finite $\ell$.
\begin{proposition}\label{prop:fin_conv}
    Let $\ell^*=\binom{n}{2}$, then $NPA_{\ell^*}(\mathcal{Perm}(V, w)) = NPA_{\ell^*+1}(\mathcal{Perm}(V, w))$ 
    
    \noindent $ = NPA_{\ell^* + k}(\mathcal{Perm}(V, w))$ for any $k\in \mathbb{Z}_{\geq 0}$.
\end{proposition}
\begin{proof}
    We show this by demonstrating that the entries of a feasible moment matrix for $NPA_{\ell^*+k}$ are in fact determined by the submatrix corresponding to $NPA_{\ell^*}$.  Let $\phi\in \Gamma_{\ell^*+k}$ of degree $\ell^*+j$ for $0\leq j\leq k$.  Since there are only $\binom{n}{2}$ many varaibles $p_{ij}$, $\phi$ must contain $j$ varaibles which are repeated.  Using the anti-commuting constraint \cref{eq:anti_comm} we can commute the variables $p_{ij}$ through each other while potentially picking up lower degree monomials ($p_{ij} p_{jk}\rightarrow -p_{jk} p_{ij} +p_{ij} +p_{jk}+p_{ik}-\mathbb{I}$).  We can commute repeated $p_{ij}$ through to cancel them using \cref{eq:sym_const_1} and proceed inductively to cancel any repeating transpositions in each monomial in the linear combination of monomials generated by applying the anti-commuting constraint.  It follows that $\phi$ may be written as a linear combination of monomials in the variables $\{p_{ij}\}$ containing only distinct $p_{ij}$.  Hence, the constraints imply that for any $ \phi, \phi'\in \Gamma_{\ell^*+k}$, $M_{\ell^*+k}(\phi, \phi')=\sum_{\theta, \theta' \in \Gamma_{\ell^*}}c_{\theta, \theta'} M_{\ell^*+k}(\theta, \theta')$ 
    %\lunote{Should it be $M_{\ell^*+k}(\phi, \phi')=\sum_{\theta, \theta' \in \Gamma_{\ell^*}}c_{\theta, \theta'} M_{\ell^*}(\theta, \theta')$?}\knote{No, I think in the context of this proof $M_{\ell^*}$ isnt defined.  The point is to show that the matrix $M_{\ell^*+k}$ is really determined by a submatrix.} 
    for scalars $c_{\theta, \theta'}$.
\end{proof}
A feasible solution of $NPA_{\ell^*}(\mathcal{Perm})$ is ``effectively'' of size $ 2^{\binom{n}{2}}$.  This is because we may use the anti-commuting constraint to relate any row of $M_{\ell^*}$ to a row in which the $p_{ij}$ are given in some predetermined canonical order. Hence a linearly independent row is given by choosing whether or not each $p_{ij}$ is included in this fixed order. 
$NPA_{\ell}(\mathcal{Pauli})$ is known to have finite convergence at a lower level $\ell^*=n$ with effective SDP size $4^n \ll 2^{n \choose 2}$.  However, we suspect that \cref{prop:fin_conv} is loose and $NPA_{\ell}(\mathcal{Perm})$ actually has finite convergence at $\ell=n/2$\footnote{Indeed, this has been established by an independent group of researchers \cite{other_guys} for a slightly different hierarchy.}. As a matter of fact, for many Hamiltonians of interest $NPA_\ell(\mathcal{Perm})$ converges with smaller SDP sizes.  
For example, most of the graphs we address in the following section (star graphs, complete (bipartite) graphs, crown graphs, etc.) will be exactly solved by $NPA_1(\mathcal{Perm})$\footnote{Recall that solvability with $NPA_1(\mathcal{Perm})$ is equivalent to $NPA_1(\mathcal{Proj})$. While we will stick to $\mathcal{Proj}$ notation in the following sections, here we are using $\mathcal{Perm}$ for the simplicity of the finite convergence proof.} but not by $NPA_1(\mathcal{Pauli})$. 
The matrix size required for convergence then reads
$\binom{n}{2}3^2+3n+1$ and $\binom{n}{2}+1$ for $\mathcal{Pauli}$ and $\mathcal{Perm}$ respectively, where the latter is more efficient by an approximate factor of 9, and is far less than $2^{n \choose 2}$. 
%\lunote{Why are we using $NPA_1(\mathcal{Perm})$ instead of $NPA_1(\mathcal{Proj})$ in the proposition and this paragraph?}\knote{$\mathcal{Perm}$ is nice because you can cancel out same factors, i.e. $p_{ij}^2=\mathbb{I}$.  Without that you have to actually use the fact that you have every possible $(i, j)$. Perm is easier for an inductive proof, but its moslty a matter of taste.  }
%For instance the weighted/unweighted star graphs require a $\mathcal{Pauli}$ SDP of size $\binom{n}{2}3^2+3n+1$ for convergence while $\mathcal{Perm}$ requires only $\ell=1$, or size $1+\binom{n}{2}$. 
%The same could be said for any other family of graphs that are not exactly solvable by $NPA_1(\mathcal{Pauli})$ but is solvable by $NPA_1(\mathcal{Perm})$, which includes 


\begin{comment}
\subsection{Hierarchies as Relaxations of Quantum Systems}

In our context both the operator programs and the SDP hierarchies have the property taht they ``relax'' the local Hamiltonian problem.  This means taht the optimal eigenvector for the local Hamiltonian problem can be used to construct wfeasible solutions to the operator programs/SDPs in such a way that the extermal eigenvalue equals the objective of the corresponding feasible solution.  If $\ket{\psi}$ is the extremal eigenvecotr of a SWAP Hamiltonian then we may set $\ket{\boldsymbol{\phi}}=\ket{\psi}$ and $\mathbf{P}_{ij}= P_{ij}$ (defined in equation BLANK) to obtain a feasible solution for the program corresponding to $\mathcal{Perm}(G, \{w\})$ with objective matching the extremal eigenvalue.  Similarly, we can set
\begin{equation}
    \mathbf{M}_\ell(\mathbf{\Phi}, \mathbf{\Theta})=Tr[\mathbf{\Phi}^* (\{P_{ij}\} \mathbf{\Theta}(\{P_{ij}\}) \ket{\psi} \bra{\psi}]
\end{equation}
to obtain a feasible solution to $NPA_\ell(\mathcal{Perm})$ with objective matching the extremal eigenvalue.  It follows that $\SWAP(G, \{w\})\leq \mathcal{Perm}(G, \{w\})$ and $\SWAP(G, \{w\})\leq NPA_\ell(\mathcal{Perm}(G, \{w\})$\knote{Notation is failing here}.  Similar considerations hold for $\mathcal{Proj}$ and $\mathcal{Pauli}$ for QMC instances and general $2-$Local Hamiltonian instances respectively.

Both $NPA_\ell$ and $SoS_\ell$ provide computationally efficient upper bounds on the corresponding local Hamiltonian problem, but $SoS_\ell$ has the additional feature that the optimal solution to the SDP provides an explicit algebraic proof that $OPT(\mathcal{O}) \leq SoS_\ell$.


In order to explicitly give an SDP for $NPA_\ell$ we will define constraint matrices\knote{uinsg this expression for two different things} and express the above in SDP ``standard form''.  Let $B_\mathbb{I}$ be a matrix indexed by $\Gamma_\ell$ such that
\begin{equation}\label{eq:b_I}
    B_\mathbb{I}(\Phi, \Phi')=\begin{cases}
    1 \text{  if } \Phi=\mathbb{I}=\Phi'\\
    0 \text{   otherwise}
    \end{cases}.
\end{equation}
If $(\Phi, \Phi', \Theta, \Theta')\in \Gamma_\ell \times \Gamma_\ell \times \Gamma_\ell \times \Gamma_\ell $ is a tuple satisfying $\Phi^* \Phi'=\Theta^* \Theta'$ which can seen by concatenation, define $C_{(\Phi, \Phi', \Theta, \Theta')}, C_{(\Phi, \Phi', \Theta, \Theta')}'$\footnote{We will require matrices $C_{(\Phi, \Phi', \Theta, \Theta')}$ and  $C_{(\Phi, \Phi', \Theta, \Theta')}'$ to enforce \Cref{concat_const} using only \textbf{Hermtian} constraint matrices .}\knote{ToDo: Use D matrix notation to enforce these $C$ things should be shorter/less clunky.  One issue is that the (theta, theta') pair cannot be chosen arbitrarily anymore for enforcing the C conditions.  In general the notation is kind of bad here so think about how to make it better.  } as Hermitian matrices indexed by $\Gamma_\ell$ such that
\begin{equation}
    C_{(\Phi, \Phi', \Theta, \Theta')}(\Psi, \Psi')=\begin{cases}
    1 \text{  if  $\Psi=\Phi $ and $\Psi'=\Phi'$}\\
    1 \text{  if  $\Psi=\Phi' $ and $\Psi'=\Phi$}\\
    -1 \text{  if  $\Psi=\Theta $ and $\Psi'=\Theta'$}\\
    -1 \text{  if  $\Psi=\Theta' $ and $\Psi'=\Theta$}\\
    0 \text{  otherwise}\\
    \end{cases},
\end{equation}
and 
\begin{equation}
    C_{(\Phi, \Phi', \Theta, \Theta')}'(\Psi, \Psi')=\begin{cases}
    i \text{  if  $\Psi=\Phi $ and $\Psi'=\Phi'$}\\
    -i \text{  if  $\Psi=\Phi' $ and $\Psi'=\Phi$}\\
    -i \text{  if  $\Psi=\Theta $ and $\Psi'=\Theta'$}\\
    i \text{  if  $\Psi=\Theta' $ and $\Psi'=\Theta$}\\
    0 \text{  otherwise}\\
    \end{cases}.
\end{equation}
For any $p\in span(\Gamma_{2 \ell})$ define $D_p, D_p'$ as matrices indexed by $\Gamma_\ell$ as follows:  Let $p=\sum_{\Phi\in \Gamma_{2\ell}}c_\Phi \Phi$ for $c_\Phi \in \mathbb{C}$ for all $\Phi$.  For all $\Phi \in \Gamma_{2\ell}$ choose a pair $(\Theta_\Phi, \Theta_{\Phi}')\in \Gamma_\ell \times \Gamma_\ell$ such that $\Theta_\Phi^* \Theta_\Phi'=$ Then,
\begin{equation}
    D_p(\Psi, \Psi')=\begin{cases}
        c_\Phi \text{  if $\Psi=\Theta_\Phi$ and $\Psi'=\Theta_\Phi'$}\\
        c_\Phi^* \text{  if $\Psi=\Theta_\Phi'$ and $\Psi'=\Theta_\Phi$}\\
        0 \text{  otherwise},
    \end{cases}
\end{equation}
and 
\begin{equation}\label{eq:dp_p}
    D_p(\Psi, \Psi')'=\begin{cases}
        i c_\Phi \text{  if $\Psi=\Theta_\Phi$ and $\Psi'=\Theta_\Phi'$}\\
        -i c_\Phi^* \text{  if $\Psi=\Theta_\Phi'$ and $\Psi'=\Theta_\Phi$}\\
        0 \text{  otherwise}.
    \end{cases}
\end{equation}
For ease of reference let $U_1:=\{(\Phi, \Phi', \Theta, \Theta')\in \Gamma_\ell \times \Gamma_\ell \times\Gamma_\ell \times \Gamma_\ell: \Phi^* \Phi'=\Theta^* \Theta'\}$ and $U_2:= \{ p\in span(\Gamma_{2\ell}): \,\,p=r p_i \text{ with } r\in \Gamma_{2\ell}  \text{ and } deg(r)+deg(p_i) \leq 2\ell\}$.  Using these sets and denoting $A\cdot B:= Tr(A^* B)$, $NPA_\ell$ can be succinctly written as:
\begin{align}
NPA_\ell(\mathcal{O}):= \max /\min \,\,&M_\ell \cdot \frac{1}{2} D_q\\
\nonumber s.t. \\
\nonumber B_\mathbb{I} \cdot M_\ell &=1\\
\nonumber \,\, C_{(\Phi, \Phi', \Theta, \Theta')} \cdot M_\ell&=0 \text{  for all $(\Phi, \Phi', \Theta, \Theta')\in U_1$}\\
\nonumber C_{(\Phi, \Phi', \Theta, \Theta')}' \cdot M_\ell&=0 \text{  for all $(\Phi, \Phi', \Theta, \Theta')\in U_1$}\\
\nonumber D_p \cdot M_\ell &=0 \text{ for all $p \in U_2$}\\
\nonumber D_p' \cdot M_\ell &=0 \text{ for all $p\in U_2$}\\ 
\nonumber M_\ell&\succeq 0, Hermitian
\end{align}

The dual hierarhcy, where individual levels are SDP dual to the corresponding NPA level, is normally referred to as the ``sum of squares'' (SoS) hierarchy.  To motivate this hierarchy we will need to note a natural correspondence between PSD matrices indexed by $\Gamma_\ell$ and ``sums of squares'' of operators in $span(\Gamma_\ell)$. The SoS hierarchy will be slightly different depending on whether the objective of $\mathcal{O}$ is to maximize or minimize.  We will motivate it for the maximization version and give a general definition for $\max/\min$.  Let $\sum_i S_i^* S_i$ be a degree $\leq 2 \ell$ sum of squares.  This means each $S_i=\sum_{\Phi\in Gamma_\ell} c_\Phi^i \Phi$ for some constants $c_\Phi^i\in \mathbb{C}$.  While this is not obviously a sum of squares, if $c_\Phi^i\in \mathbb{R}$ for all $i$, $\Phi$ then $S_i^*=S_i$ and $\sum_i S_i^* S_i=\sum_i S_i^2$.  Having $\sum_i S_i^* S_i$ allows us to generalize this notion to complex numbers.  Let $Q$ be a matrix indexed by $\Gamma_\ell$  such that $Q(\Phi, \Phi')=\sum_i (c_\Phi^i)^* c_{\Phi'}^i$.  $Q$ is easily seen to be PSD: For each $i$ define a row vector $v_i$ indexed by $\Gamma_\ell$ such that $v_i(\Phi)=c_\Phi^i$.  Then, $Q=\sum_i v_i^* v_i$.  Observe also that any $Q\succeq 0$ has a corresponding sum of squares $\sum_i S_i^* S_i$ which can be calculated via diagonalization.  We will also need to define a ``constraint deformation matrix'' \knote{think of better thing to call this}.  Given the matrices in \Cref{eq:b_I}-\Cref{eq:dp_p} a constraint deformation matrix is simply a matrix of the form:
\begin{align}
T=\sum_{(\Phi, \Phi', \Theta, \Theta')\in U_1} y_{(\Phi, \Phi', \Theta, \Theta')} C_{(\Phi, \Phi', \Theta, \Theta')}+y_{(\Phi, \Phi', \Theta, \Theta')}' C_{(\Phi, \Phi', \Theta, \Theta')}'+\sum_{p \in U_2} z_p D_p +z_p' D_p',
\end{align}
where all $y_{(\Phi, \Phi', \Theta, \Theta')}, y_{(\Phi, \Phi', \Theta, \Theta')}', z_p, z_p' \in \mathbb{R}$.

Now suppose we are able to obtain an expression of the form $\lambda B_\mathbb{I}-\frac{1}{2}D_q= Q+T $ where $Q$ is an SoS matrix ($\succeq 0$) and $T$ is a constraint deformation matrix.  We claim that such an expression implies that the optimal value of $\mathcal{O}$ is $\leq \lambda$ (recall $\mathcal{O}$ is currently a maximization problem).  The left and right side are both matrices indexed by $\Gamma_\ell$ so by summing up matrix entries an multiplying by the corresponding operators we may derive:
\begin{align} 
\lambda\mathbb{I}-\frac{1}{2}\sum_{\Phi, \Phi' \in \Gamma_\ell}D_q(\Phi, \Phi') =\sum_{\Phi, \Phi'} Q(\Phi, \Phi') +T(\Phi, \Phi')\\
\Rightarrow \lambda\mathbb{I}- q =\sum_i S_i^* S_i+\sum_{(\Phi, \Phi', \Theta, \Theta')\in U_1}y_{(\Phi, \Phi', \Theta, \Theta')} (\Phi^* \Phi'-\Theta^* \Theta'+(\Phi')^*\Phi -(\Theta')^* \Theta)\\ 
\nonumber + y_{(\Phi, \Phi', \Theta, \Theta')}' i(\Phi^* \Phi'- \Theta^* \Theta'+(\Theta')^* \Theta- (\Phi')^* \Phi)+\sum_{p\in U_2}z_p (p+p^*)+z_p i (p-p^*).
\end{align}
The algebraic constraints of $\mathcal{O}$ imply the second and third sums on the R.H.S evaluate to $0$.  So for feasible operators for $\mathcal{O}$, $\lambda \mathbb{I}-q=\sum_i S_i^* S_i \succeq 0$, hence $\mu_{max}(q) \leq \lambda$.  Essentially a level-$\ell$ SoS proof involves taking a SoS of operator in $span(\Gamma_\ell)$ and deforming it using constraints which hold \textbf{at that level} to obtain an expression of the form $\lambda \mathbb{I}-q$.  Higher levels give access to more constraints (the set $U_2$ gets larger), and hence stronger proofs.  The optimal SoS proof at a level-$\ell$ can be computed by SDP and is in fact the dual SDP to $NPA_\ell$:

\begin{definition}[$SoS_\ell(\mathcal{O})$] Given an operator program of the form BLANK, define matrices as in \Cref{eq:b_I}-\Cref{eq:dp_p}.  If $\mathcal{O}$ is a maximization problem then 
\begin{align}
    SoS_\ell(\mathcal{O}):= \min \lambda\\
    s.t.\\
    \lambda B_\mathbb{I}-\frac{1}{2} D_q= Q+T\\
    T=\sum_{(\Phi, \Phi', \Theta, \Theta')\in U_1} y_{(\Phi, \Phi', \Theta, \Theta')} C_{(\Phi, \Phi', \Theta, \Theta')}+y_{(\Phi, \Phi', \Theta, \Theta')}' C_{(\Phi, \Phi', \Theta, \Theta')}'+\sum_{p \in U_2} z_p D_p +z_p' D_p'\\
    Q \succeq, Hermitian\\
    y_{(\Phi, \Phi', \Theta, \Theta')}, y_{(\Phi, \Phi', \Theta, \Theta')}', z_p, z_p' \in \mathbb{R} \,\, \forall \,\, (\Phi, \Phi', \Theta, \Theta')\in U_1, p\in U_2
\end{align}
If $\mathcal{O}$ is a minimization problem then:
\begin{align}
    SoS_\ell(\mathcal{O}):= \max \lambda\\
    s.t.\\
    \frac{1}{2} D_q-\lambda B_\mathbb{I}= Q+T\\
    T=\sum_{(\Phi, \Phi', \Theta, \Theta')\in U_1} y_{(\Phi, \Phi', \Theta, \Theta')} C_{(\Phi, \Phi', \Theta, \Theta')}+y_{(\Phi, \Phi', \Theta, \Theta')}' C_{(\Phi, \Phi', \Theta, \Theta')}'+\sum_{p \in U_2} z_p D_p +z_p' D_p'\\
    Q \succeq, Hermitian\\
    y_{(\Phi, \Phi', \Theta, \Theta')}, y_{(\Phi, \Phi', \Theta, \Theta')}', z_p, z_p' \in \mathbb{R} \,\, \forall \,\, (\Phi, \Phi', \Theta, \Theta')\in U_1, p\in U_2
    \end{align}
\end{definition}


\subsection{Convergence}
Under relatively mild assumptions \cite{pir10con} it is know that $NPA_\ell(\mathcal{O})$ converges to the optimal solution of $\mathcal{O}$ as $\ell \rightarrow \infty$.  So it is generally possible to solve a high enough level $NPA_\ell$ (corresponding to an exponentially large SDP in $\ell$) to obtain the optimal solution for $\mathcal{O}$.  However, it is unclear what the relationship is between the optimal solution of $\mathcal{O}$ and the extremal eigenvalue of the explicit Hamiltonian problem we have in mind (i.e. QMC).  One issue is that the optimal solution to the operator program could have operator assignments with size which are larger or smaller than the relevant Hilbert space, so it is not obvious how to interpret a the optimal solution of the operator program in the context of the Hamiltonian problem.  

Luckily for $\mathcal{Pauli}$ operator programs it is known by representation theory that the optimal solution to the operator program exactly equals the optimal solution to the local Hamiltonian problem (even for Hamiltonians outside QMC).  This is because any representation of the Pauli operators must be isomorphic to the Pauli operators themselves up to a global unitary and tensoring with the identity \cite{cha17}\knote{find better ref for this}.  In a sense the ``only'' representation of the Pauli group is the matrices themselves.  Additionally there is an operator program not discussed here for fermionic creation and annihilation operators \cite{pir10con} which is known to converge to the optimal eigenvalue under similar reasoning.  In both these cases we can understand the optimal solution to the operator program as the extremal eigenvalue because the representation theory is strict enough to guarantee that a set of optimal operators will be equal to the explicit operators we have in mind (again up to tensoring with identity and overall unitary).  

It is natural to ask if there is an operator program with optimal solution achieving the extremal quantum eigenvalue, \textbf{without} the representation theory of the underlying group being as strict.  Indeed one of the results of this paper is that $\mathcal{Perm}/\mathcal{Proj}$ achieves the optimal solution of the corresponding $\SWAP/QMC$ instance.  In contrast to $\mathcal{Pauli}$ the proof demonstrates that $\mathcal{Perm}$ obtains the optimal value with operators which are strictly smaller than the corresponding quantum SWAP operators.

\begin{theorem}\label{thm:perm_is_opt}
$\mathcal{Perm}(G, w)=\SWAP(G, w)$.
\end{theorem}
\begin{proof}
See \Cref{sec:perm_conv}.
\end{proof}

\begin{corollary}
$\mathcal{Perm}$ is a QMA-complete operator program.  
\end{corollary}
\begin{proof}
    Determining $\SWAP(G, w)$ is QMA-complete \cite{pid17} and by \Cref{thm:perm_is_opt} finding $\SWAP(G, w)$ is equivalent to finding $\mathcal{Perm}(G, w)$.
\end{proof}
\end{comment}

\section{Exact results on some families of graphs}\label{sec:exact}

Here we detail our results concerning the exactness/inexactness of $NPA_1(\mathcal{Proj})/SoS_1(\mathcal{Proj})$ on many interesting classes of {\scshape QMaxCut} Hamiltonians.  
%We focus on $NPA_1(\mathcal{Proj})/SoS_1(\mathcal{Proj})$ and demonstrate several proofs which show exactness, as well as some considerations from symmetry which prove inexactness for some family of graphs. %complete graphs with odd number of vertices. 
First of these classes is the positive weighted star graphs. The proof technique for this class involves reconstructing a quantum state with the exact same energy as the output of the $NPA_1(\mathcal{Proj})$ program.  A crucial component of this proof is a reinterpretation of ``monogamy of entanglement'' inequalities in terms of the possible angles for Gram vectors from $NPA_1(\mathcal{Proj})$.  
We show the constraints on these angles from the polynomial inequalities derived in \cite{par22opt} are actually saturated for the case of star graphs. This provides an interesting geometric perspective for monogamy of entanglement in the context of $NPA_1(\mathcal{Proj})$.%, and how the constraint from the monogamy essentially determines the ground state for the star graph case. 

The other proofs for exactness relies on SoS proofs, which we analytically construct. Since the SDP hierarchies defined in \Cref{sec:hierarchy} are relaxations of the Local Hamiltonian problem, it is sufficient to construct a feasible solution to $SoS_\ell$ which achieves the optimal eigenvalue as the objective. 
To state concretely, we will be utilizing the following theorem: 
\begin{theorem}
The upper bound obtained by the $NPA_\ell(\mathcal{Proj})$ matches exactly with the maximum eigenvalue if and only if there exists a $SoS_\ell(\mathcal{Proj})$ that upper bounds the maximum eigenvalue tightly.
\end{theorem}
%Techniques used 
Some results in the exactness proofs of other graphs will have some overlap with the first case of weighted star graph, but the explicitly constructive nature of SoS proof gives a complementary understanding of how the SDP algorithm obtains the exact solution. Finally, we discuss the sharp contrast in the SDP performance for complete graphs with even and odd number of vertices, which could be seen as a quantum version of the parity problem addressed in \cite{gri01lin}. 
Here, we prove cases where $NPA_1(\mathcal{Proj})$ is always {\it insufficient} to obtain the maximum-eigenvalue state. 

%For many interesting classes of {\scshape QMaxCut} Hamiltonians, it is possible to prove that $NPA_1(\mathcal{Proj})$ is sufficient to obtain the maximum-eigenvalue state exactly. In other cases, we can also prove that the $NPA_1(\mathcal{Proj}$ is always {\it insufficient} to obtain the maximum-eigenvalue state. 

%For the rest of the proofs, the main tool we will be using is the so-called {\it Sum of Squares} (SOS), which is dual to the Lasserre/NPA hierarchy \cite{lau09sum,pir10con}.



\subsection{Positive weighted star graph}
In this section we generalize the result of \cite{par21app} and prove that $NPA_1(\mathcal{Proj}(H))$ has optimal objective matching the extremal energy if the Hamiltonian is a positively weighted star.  Since $NPA_1^{\mathbb{R}}(\mathcal{Proj}(H))=NPA_1(\mathcal{Proj}(H))$ this implies also that $NPA_1(\mathcal{Proj})=\mu_{max}(H)$.  To our knowledge the first known proof of this statement is from unpublished personal correspondence \cite{per_comm}, however the proof we present here is simpler and has an intuitive geometric interpretation. 
The following theorem, proved in \cite{par22opt} about monogamy of entanglement on a triangle (three qubits $i, j$ and $k$), will be the starting point for our proof. %that $NPA_1(\mathcal{Proj})$ will calculate the maximum value of the Hamiltonian exactly on star graphs with positive weights.

\begin{theorem}[\protect{\cite[Lemma~7]{par22opt}}] \label{thm:MonogomyOfEntanglementOnTraingle}
    For any feasible moment matrix $M_1$ of $NPA_1(\mathcal{Proj})$, the following inequalities are true:
    \begin{align}
        0 \leq M_1(\mathbb{I}, h_{ij}) &+ M_1(\mathbb{I}, h_{jk}) + M_1(\mathbb{I}, h_{ki}) \leq 3/2 \\
        M_1(\mathbb{I},h_{ij})^2 + M_1(\mathbb{I},h_{jk})^2  +M_1(\mathbb{I},h_{ki})^2& \nonumber \\
        \leq 2\Big[M_1(\mathbb{I},h_{ij})M_1(\mathbb{I},h_{jk}) &+M_1(\mathbb{I},h_{jk})M_1(\mathbb{I},h_{ki})+M_1(\mathbb{I},h_{ki})M_1(\mathbb{I},h_{ij})\Big]. \label{eq:MonogomyOfEntanglementOnTraingle}
    \end{align}
\end{theorem}
Note that in \cite{par22opt}, the variables are defined by swap operators (\cref{eq:swap_def} in this work), and the above form could be derived by simply using the relation between swap operators and singlet projectors $h_{ij} = (1-p_{ij})/2$. 


\begin{lemma} \label{thm:60DegreeAngle}
    For any feasible moment matrix $M_1$ of $NPA_1(\mathcal{Proj})$, indexed by $\{I, h_{ij} \text{ where }i,j \in \{0,1,\ldots,n\} \text{ and } i < j\}$, the angle between any two normalized Gram vectors of indices sharing one vertex, i.e. $h_{ij}$ and $h_{jk}$ where $i,j,k$ are all distinct, is no less than $60^\circ$ and no greater than $120^\circ$ .
\end{lemma}

\begin{proof}
    Let $\ket{\mathbb{I}}$, $\ket{h_{ij}}$ be the Gram vectors of $M_1$ corresponding to indices $\mathbb{I}$ and $h_{ij}$ for any $i\neq j$ respectively. With the standard bra-ket notation, we can then simply write
    \begin{align}
        M_1(h_{ij},h_{kl}) = \braket{h_{ij}|h_{kl}}. 
    \end{align}
    Now recall that we have the following constraints on $M_1$: whenever $AB = CD$ where $A,B,C,D$ are all degree-1 polynomials in singlet projectors, $M_1(A,B) = M_1(C,D)$. From this, $h_{ij}^2 = h_{ij}$ implies that 
    \begin{align}
        M_1(h_{ij},h_{ij}) = M_1(\mathbb{I},h_{ij}).
    \end{align}
    Similarly, the anti-commutation relation for singlet projectors \cref{eq:anticommproj} implies that 
    \begin{align}
        4M_1(h_{ij}, h_{jk}) = M_1(\mathbb{I}, h_{ij}) + M_1(\mathbb{I}, h_{jk}) - M_1(\mathbb{I}, h_{ki}).
    \end{align}
    Starting from \cref{eq:MonogomyOfEntanglementOnTraingle}, we can derive the following.
    \begin{align}
        \big[M_1(\mathbb{I},h_{ij}) + M_1(\mathbb{I},h_{jk})  -M_1(\mathbb{I},h_{ki})\big]^2
        &\leq 4M_1(\mathbb{I},h_{ij})M_1(\mathbb{I},h_{jk}) \\
        \iff 16 M_1(h_{ij}, h_{jk})^2 &\leq 4M_1(\mathbb{I},h_{ij})M_1(\mathbb{I},h_{jk}) \\
        \iff \braket{h_{ij}|h_{jk}}^2 &\leq 4 \braket{\mathbb{I}|h_{ij}} \braket{\mathbb{I}|h_{jk}} = \braket{h_{ij}|h_{ij}} \braket{h_{jk}|h_{jk}} \\
        \iff \left| \braket{h_{ij}|h_{jk}}/\sqrt{\braket{h_{ij}|h_{ij}} \braket{h_{jk}|h_{jk}}} \right| &\leq 1/2 .\label{eq:derivation60}
    \end{align}
    \Cref{eq:derivation60} implies that the angle between the Gram vectors $\ket{h_{ij}}$ and $\ket{h_{jk}}$ must be between $60^\circ$ and $120^\circ$.
\end{proof}



\begin{theorem}
    The first level of the NPA hierarchy with $\mathcal{Proj}$ solves {\scshape QMaxCut} exactly for any positively weighted star graphs, i.e.,  
    $NPA_1(\mathcal{Proj}(H))=QMC(H)=\mu_{max}(H)$ for 

    
    \begin{align}\label{eq:starham}
    H = \sum_{i=1}^{n} w_i h_{0i},\quad w_i > 0\ ~~\forall i .
    \end{align}
\end{theorem}

   \begin{comment}
    Level-$1$ $\mathcal{Proj}$ program is exact in finding the ground state energy of the {\scshape QMaxCut} Hamiltonian over a star graph with positive weights
    \end{comment}

\begin{comment}
\knote{I have never been in this situation before so Im not sure what the proper thing to do is, but we might need to cite personal communication from John Wright as the original proof and then say we have an alternative proof.}
\cnote{I agree. We should mention that we were aware of SOS proof by john wright before developing our proof, and maybe we can emphasis that our proof is significantly simpler and has the potential to generalize to higher levels.}
\end{comment}

\begin{proof}
    Recall that the moment matrix $M_1$ of $NPA_1(\mathcal{Proj})$ program is indexed by $\{\mathbb{I}, h_{ij} \text{ where }i,j \in \{0,1,\ldots,n\} \text{ and } i < j\}$. Let $\ket{\widetilde{\mathbb{I}}}$, $\ket{\widetilde{h_{ij}}}$ be the normalized Gram vectors of $M_1$ corresponding to indices $\mathbb{I}$ and $h_{ij}$. Restating \Cref{thm:60DegreeAngle}, for any feasible moment matrix $M_1$, the angle between any two normalized Gram vectors of singlet projector indices sharing one vertex ($\ket{\widetilde{h_{ij}}}$ and $\ket{\widetilde{h_{jk}}}$ where $i,j,k$ are all distinct) is no less than $60^\circ$ and no greater than $120^\circ$, i.e.,
    \begin{align}
        \left| \braket{\widetilde{h_{ij}}|\widetilde{h_{jk}}}\right| \leq \frac{1}{2}.
    \end{align}
    The rest of the proof involves showing that when the objective function is a Hamiltonian of the form \cref{eq:starham},
    for the optimal solution the above inequality saturates for any two normalized Gram vectors with distinct indices from $\left\{h_{0i}\right\}_{i=1}^{n}$
    \begin{align}
        \braket{\widetilde{h_{0i}}|\widetilde{h_{0j}}} = \frac{1}{2} \quad \forall\ 0< i < j \leq n.
        \label{eq: sixty_angle_property}
    \end{align}
    When this equality holds, we can construct an actual quantum state that has the same energy as the objective value of $NPA_1(\mathcal{Proj})$ program. 
    We do this by mapping each of the normalized Gram vectors $\ket{\widetilde{h_{0i}}}$ to a state that has a singlet between $0^\text{th}$ and $i^\text{th}$ qubits , i.e., $(\ket{0}_0\otimes\ket{1}_i - \ket{1}_0\otimes \ket{0}_i)/\sqrt{2}$, and the rest of the qubits forming a maximal total spin state, e.g., all qubits in the spin up state. 
    This mapping preserves the property \cref{eq: sixty_angle_property} and also determines the normalized Gram vectors corresponding to other indices $h_{ij}$ to be $\ket{\widetilde{h_{ij}}} = \pm (\ket{\widetilde{h_{0i}}} - \ket{\widetilde{h_{0j}}})$ for  $0< i < j$ where the positive or negative sign in the front depending on whether $M_1(\mathbb{I}, h_{0i})$ is greater or less than $M_1(\mathbb{I}, h_{0j})$ respectively.
    Furthermore, when \cref{eq: sixty_angle_property} is satisfied, the $\ket{\widetilde{\mathbb{I}}}$ that maximizes the objective function will also be in the span of $\{\ket{\widetilde{h_{0i}}}\}_{i=1}^{n}$. 
    Then, since the output of the $NPA_1(\mathcal{Proj})$ program is an upper bound on the maximum eigenvalue of the Hamiltonian, this implies that the output of $NPA_1(\mathcal{Proj})$ is exact.

    Let $A$ be a matrix where the $i^\text{th}$ row of the matrix is the weighted Gram vector $\sqrt{w_i}\bra{\widetilde{h_{0i}}}$. %in some fixed chosen basis. 
    Without loss of generality, we can assume that $A$ is of size $n \times n$.
    Let $A = PU$ be its polar decomposition where $P$ is a positive semi-definite matrix and $U$ is an orthogonal matrix.
    We can rewrite the objective function as the following:
    \begin{align}
        M_1(\mathbb{I}, H) = \sum_{i=1}^{n} w_i M_1(\mathbb{I}, h_{0i}) = \sum_{i=1}^{n} w_i \braket{\widetilde{\mathbb{I}}|\widetilde{h_{0i}}}^2 = \braket{\widetilde{\mathbb{I}}|A^{T}A|\widetilde{\mathbb{I}}} = \braket{\widetilde{\mathbb{I}}|U^T P^2 U|\widetilde{\mathbb{I}}}.
    \end{align}
    Since we want to maximize the objective function, its value cannot be greater than the maximum eigenvalue of $P^2$, and it is equal to maximum eigenvalue when $U\ket{\widetilde{\mathbb{I}}}$ is the maximum eigenvector of $P^2$.
    
    The set of constraints that $\left|\braket{\widetilde{h_{0i}}|\widetilde{h_{0j}}} \right| \leq \frac{1}{2}$, where $i\neq j$, can be written as $\left|(AA^{T})_{ij} \right| = \left|(P^2)_{ij} \right| \leq \frac{1}{2}\sqrt{w_i w_j}$, where $i\neq j$, and the $ij$ subscript indicates that it's the $i^{\text{th}}$ row  $j^{\text{th}}$ column element of the particular matrix. 
    The $\ket{\widetilde{h_{ij}}}$ vectors being normalised implies $(AA^T)_{ii} = (P^2)_{ii} = w_i$ for $i \in \{1,2,...,n\}$. Given these constraints on $P^2$, the maximum eigenvalue of $P^2$ is maximized when $(P^2)_{ij} = \frac{1}{2}\sqrt{w_i w_j}$. To see this, consider $P^2$ with its maximum eigenvector $\ket{v}$ where some of the matrix elements of $P^2$ are negative. Let ${abs}(P^2)$ and $\ket{{abs}(v)}$ be the matrix where we take element wise absolute value of the matrix and the vectors. It is easy see that $\braket{{abs}(v)|{abs}(P^2)|{abs}(v)} \geq \braket{v|P^2|v}$, which implies that the maximum eigenvalue of $abs(P^2) \geq $ maximum eigenvalue of $P^2$. When all the elements of a matrix are non-negative, Perron-Frobenius theorem implies that the maximum eigenvalue is a non-decreasing function of each of the individual matrix elements and strictly increasing in the case of irreducible matrix thus implying that the optimal $P^2$ has $(P^2)_{ij} = \frac{1}{2}\sqrt{w_i w_j}$. The Gram vectors of this optimal $P^2$ are exactly $\sqrt{w_i} \ket{\widetilde{h_{0i}}}$ which satisfy the property $\braket{\widetilde{h_{0i}}|\widetilde{h_{0j}}} = \frac{1}{2}$. For the optimal $P^2$, the maximum eigenvector $U\ket{\widetilde{\mathbb{I}}}$ is also in the span of its gram vectors and so is $\ket{\widetilde{\mathbb{I}}}$ since the dimension of subspace formed by the span of $\{\ket{\widetilde{h_{0i}}}\}_{i=1}^{n}$ is $n$.
\end{proof}

\subsection{Complete bipartite graphs and some extensions}\label{subsec:compbip}

In this section, we show explicit $SoS_1(\mathcal{Proj})$ proofs for several family of graphs (shown in \cref{fig:SmallGraphs} (a)). An important tool for demonstrating exact SoS proofs is the decomposition of graphs into smaller graphs leading to a decomposition of the SoS proof into smaller SoS proofs (schematically shown in the figure).  The simplest example of such decomposition arises naturally when thinking of the SoS for the complete bipartite graph, which decomposes into several star graphs. 

%Let us first look into the easiest special case of them: the star graph. 
The weighted star graph can be solved exactly by $NPA_1(\mathcal{Proj})$ as shown in the previous section, however,  the explicit $\SoS$ cannot be analytically written down in general. The unweighted case however, gives us the simplest case of an exact $\SoS$ : 
\begin{equation}\label{eq:StarSOS}
   \left( \sqrt{\frac{n+1}{2}}\mathbb{I} - \sqrt{\frac{2}{n+1}}\sum_{i=1}^{n} h_{0i} \right)^{\bf 2}
   + \frac{1}{n+1}\sum_{1\leq j<k\leq n} h_{jk}^{\bf 2}
   = \frac{n+1}{2}\mathbb{I}-\sum_{i=1}^{n} h_{0i} .
\end{equation}
This $\SoS$ equation could be interpreted in the following way. Since the left hand side is a {\it sum of squares}, it implies that the right hand side is positive semidefinite, i.e., $0\preceq$ RHS. 
By reordering, we get $\mathbbm{I}(n+1)/2 \succeq \sum h_{0i}$, which upper bounds the eigenvalue of $\sum h_{0i}$, the Hamiltonian of interest here. 
For this particular case, %the equation gives the exact extremal eigenvalue $(n+1)/2$ for the Hamiltonian $\sum_{i=1}^{n}h_{0i}$, a star graph with $n+1$ qubits in total. 
the bound we obtain matches exactly to the actual maximum eigenvalue for the uniform star graph with $n$ edges ($n+1$ qubits in total). 
%By observing the exact $\SoS$, we can tell some properties of the ground state. For example, b
Note that \cref{eq:StarSOS} could be confirmed straightforwardly by using the anticommutation relation \cref{eq:anticommproj}. 
Also, by applying the ground state $|\mathrm{GS}\rangle$ from the left and right to \cref{eq:StarSOS}, we can see that all the terms inside the square on the left hand side must have the $|\mathrm{GS}\rangle$ as a 0-eigenvector. Indeed, expectation values of $\langle h_{jk}\rangle$ for any $0 < j<k\leq n$ should be 0 in the ground state. 
%\footnote{This could be seen as the SoS way of telling that all spins on the same sublattice in a perfect N{\'e}el state should be forming a maximal spin state that has 0 singlet density, a well known fact in condensed matter physics.}. 


Now let us consider the complete bipartite graph with $n+m$ vertices ($n\geq m$). 
The Hamiltonian could be written as
\begin{equation}
    H=\sum_{i\in A, j\in B}h_{ij}
\end{equation}
where we assume that the vertices are divided into two groups $A$ and $B$, with the edge set being $E=\{(i,j)|i\in A, j\in B\}$ and $|A|=n, |B|=m$. 
To our advantage, we can reuse the above SoS because of the decomposition property as follows: 
The maximum eigenvalue of $H$ on $K_{n,m}$ is exactly the same as that of
$K_{n,1}$ (i.e., a star graph with $n$ leaves) multiplied by $m$. Note that this relation only holds in one direction for $m<n$. Furthermore, the Hamiltonian itself could be viewed as comprising $m$ copies of the $n$-leaved star graph as well. 
In other words, 
\begin{equation}\label{eq:CBhamdecomp}
    H = \sum_{i\in B} ~ \biggl( \sum_{j\in A} h_{ij}\biggr)  ~~~\text{and}~~~ \| H \| = \sum_{i\in B} ~ \biggl\lVert \sum_{j\in A} h_{ij} \biggr\rVert
\end{equation}
holds simultaneously. 
This implies that if we can find an exact SoS for the decomposed Hamiltonian, we can combine $m$ copies of that SoS with appropriate relabeling to obtain the SoS for the entire Hamiltonian. 
Since we already have \cref{eq:StarSOS}, it is rather easy to confirm that
\begin{equation}\label{eq:CompBipSOS}
   \frac{2}{n+1} \sum_{i\in B}\Bigl(\frac{n+1}{2}\mathbb{I}-\sum_{j\in A}h_{ij}\Bigr)^{\bf 2} +
   \frac{m}{n+1}\sum_{j<k\in A}h_{jk}^{\bf 2}
   ~=~
   \frac{m(n+1)}{2}\mathbb{I}-H,
\end{equation}
which gives the exact energy for complete bipartite graphs $K_{n,m}$. %The complete bipartite case includes star graphs ($m=1$) and the square ($n=m=2$). 

The complete bipartite graph considered here are known as the Lieb-Mattis {\it model} in condensed matter physics \cite{lie62ord,lou19exa}, where the full energy spectrum is well-understood. The Lieb-Mattis {\it theorem} states that Heisenberg models with bipartite graphs (with sublattices $A$ and $B$) have ground states with total spin $\left( |A|-|B|\right) /2$, using the complete bipartite case as a starting point of the proof. 
The $\SoS$ we have here for complete bipartite graphs immediately tells you that the ``singlet density" $\langle h_{ij}\rangle$ among the same sublattice sites will always be 0, just like in the case we have mentioned for the star graph. 
This means that the two sublattices are forming the maximum total spin state, which is equivalent to the claim of the Lieb-Mattis theorem. 
We could say that our SoS is an alternative proof for the Lieb-Mattis theorem, restricted to the case of complete bipartite graphs with uniform weights. 

% Figure environment removed


\subsubsection{Crown Graphs}\label{subsubsec:crown}
Graphs with one additional edge to $K_{n,2}$ ($n\geq2$) connecting the two vertices of the B-sublattice 
(i.e. a complete tripartite graph $K_{n,1,1}$) also admits an exact $\SoS$ and thus $NPA_1(\mathcal{Proj})$ obtains the exact maximum eigenvalue as the upper bound.
These graphs, which we call the ``crown" graph (\cref{fig:SmallGraphs} (a)), have maximum eigenvalue $n+1$, the same value for the $K_{n,2}$ complete bipartite graphs. The additional edge does not change the maximum eigenvalue nor the maximum eigenvalue state itself. 

We can modify the SoS in \cref{eq:CompBipSOS} so that the Hamiltonian now includes the one additional edge on the right hand side. 
If we label the two vertices in the B-sublattice to be $a$ and $b$, then the $\SoS$ reads
\begin{equation}\label{eq:CrownSOS}
   \frac{2}{n+1} \sum_{k=a,b}\Bigl(\frac{n+1}{2}\mathbb{I}-\sum_{i\in A}h_{ik} -\frac{2+n\pm n}{4}h_{ab}
   \Bigr)^2 +
   \frac{2}{n+1}\sum_{i<j\in A}h_{ij}^2
   ~=~
   (n+1)\mathbb{I}-H, 
\end{equation}
where there is a degree of freedom for the coefficient of $h_{ab}$, coming from two solutions of a quadratic equation. 
%Note that the only difference with \Cref{eq:CompBipSOS} is the term $-h_{ab}/2$. 
%In fact, when the additional edge to $K_{n,2}$ has weight $x\leq (n+2)^2/4(n+1)=:x_c$ in general, we can always have an exact SOS of the above form just by modifying the coefficient of $h_{ab}$ to $(2+n)/4\pm\sqrt{(n+2)^2-4(n+1)x}/4$. 
%The actual ground state remains to be exactly the same for $x_c<x\leq 1+n/2$, but the simple SOS eq. (\ref{eq:CrownSOS}) ceases to exist. 

The observation that the only difference between this SoS and \cref{eq:CompBipSOS} is the $h_{ab}$ term encourages us to ask if this form of SoS is general in some sense. 
Indeed, as it turns out, we can consider a crown graph with the term $h_{ab}$ being weighted with weight $x$, and the above form of the SoS is exact for the entirety of $x\leq (n+2)^2/4(n+1)$. 
The precise $\SoS$ becomes 
\begin{eqnarray}
   &&(n+1)\mathbb{I}-\biggl(\sum_{k=a,b}\sum_{i\in A}h_{ik}+x h_{ab}\biggr)\nonumber\\
   &=&
   \frac{2}{n+1} \sum_{k=a,b}\Bigl(\frac{n+1}{2}\mathbb{I}-\sum_{i\in A}h_{ik} -\frac{2+n\pm \sqrt{(n+2)^2-4(n+1)x}}{4}h_{ab}
   \Bigr)^2 +
   \frac{2}{n+1}\sum_{i<j\in A}h_{ij}^2
   , \label{eq:WeightedCrownSOS}
\end{eqnarray}
which only has a real solution when $x\leq (n+2)^2/4(n+1)$. 

We can regard this SoS to be heuristically constructed in two steps. First, the case corresponding to $x=0$ was decomposable as in \cref{eq:CBhamdecomp}, yielding an SoS that retains the symmetry of the graph ($\mathbb{Z}_2$ between $a$ and $b$, and $\mathcal{S}_n$ for the A-sublattice sites). 
Next, when another edge is added also in a symmetry-preserving way, we can have an ansatz for the SoS that also still preserves the symmetry but now also includes the additional term. 
In this sense, the above SoS could be thought of as a ``perturbative" SoS from the complete-bipartite case, since if we gradually increase $x$ from 0, the SoS also can be changed continuously, always being exact. Since $1<(n+2)^2/4(n+1)$, the uniformly weighted crown graph is also exactly solvable, and we can say that the $\SoS$ for the complete bipartite graph and the crown graph are {\it adiabatically connected}. 
Intuitively, when $x$ is small enough, the ``physics" should not change a lot from the $x=0$ case, and in this case we can show that the ``radius of convergence" extends to $x=(n+2)^2/4(n+1)$, including $x=1$. 

The fact that the ansatz fails alone does not necessarily imply that no exact SoS exist, but it does suggest that even {\it if} such $\SoS$ exist, it will look very different from the SoS in the $x\leq(n+2)^2/4(n+1)$ region. 
As a matter of fact, we numerically observe that $NPA_1(\mathcal{Proj})$ starts to have nonzero error exactly from $x=(n+2)^2/4(n+1)$, implying that such a SoS proof indeed does not exist.  

Conversely, when we increase $x$ large enough, $NPA_1(\mathcal{Proj})$ starts to obtain the exact ground state energy again starting from $x \geq n$. Intuitively, in the $x\rightarrow\infty$ limit, the ground state should trivially become a state where there is simply one singlet placed for $h_{ab}$, and it seems natural for an SDP algorithm to be able to obtain such a simple state exactly. 
This intuition could be made rigorous by noticing that when $x\geq n$, the Hamiltonian regains the decomposition property, but now into triangles: 
\begin{equation}\label{eq:crowndecomp}
H=    \sum_{i\in A} \biggl( h_{ia}+h_{ib} + \frac{x}{n} h_{ab} \biggr), ~~~
\|H\|=    \sum_{i\in A} \Bigl\lVert h_{ia}+h_{ib} + \frac{x}{n} h_{ab} \Bigr\rVert \text{   when } x\geq n.
\end{equation}
Since a triangle with weight $(1,1,\alpha)$ has the exact $\SoS$ of
\begin{equation}\label{eq:StrongTriangleSoS}
    \begin{split}
    &\left(\alpha+\frac{1}{2}\right)\mathbb{I} - (h_{12}+h_{23}+\alpha h_{13})\\
    = &\left(\alpha+\frac{1}{2}\right)
    \biggl\{
    \mathbb{I}
    -\sum_{1\leq i<j\leq 3}    \frac{4\alpha+2\pm(-\frac{1}{2}+(-1)^{i+j})\sqrt{2\alpha (2\alpha -1)-2}}{3+6\alpha}h_{ij}
    \biggr\}^2,
    \end{split}
\end{equation}
% \begin{equation}\label{eq:StrongTriangleSoS}
%     \left(\alpha+\frac{1}{2}\right)\mathbb{I} - (h_{12}+h_{23}+\alpha h_{13})
%     = \\
%     \left(\alpha+\frac{1}{2}\right)
%     \biggl\{
%     \mathbb{I}
%     -\sum_{1\leq i<j\leq 3}    \frac{4\alpha+2\pm(-\frac{1}{2}+(-1)^{i+j})\sqrt{2\alpha (2\alpha -1)-2}}{3+6\alpha}h_{ij}
%     \biggr\}^2
% \end{equation}
for $\alpha\geq 1$, together with the decomposition, this can be turned into an $\SoS$ for the crown graph when $x\geq n$. 
Note that again, the SoS is not unique, and it has a degree of freedom in choosing $\pm$ to be fixed. 
%However, unlike the perturbation away from the complete bipartite graph case, 
When $\alpha<1$ the above form no longer gives a real coefficient.  However, the true ground state of the triangle also changes, and still allows an exact $\SoS$: 
\begin{equation}\label{eq:WeakTriangleSoS}
    \frac{3}{2}\mathbb{I} - (h_{12}+h_{23}+\alpha h_{13})
    =
    \frac{3}{2}\left(
        \mathbb{I} 
        -\frac{2}{3}h_{12}-\frac{2}{3}h_{23}
        -\frac{2\pm\sqrt{6(1-\alpha )}}{3}h_{13}\right)^2 ,
\end{equation}
which again only gives valid coefficients for $\alpha\leq 1$. 
For our current objective of constructing a $\SoS$ for the crown graph, the existence of SoS for $\alpha<1\Leftrightarrow x<n$ does not help since the Hamiltonian no longer has the decomposition \cref{eq:crowndecomp}. 

Again, like the case for the small $x$ region, although this decomposition is just {\it one} possible heuristic method for finding the exact SoS, it turns out that the $NPA_1(\mathcal{Proj})$ does start to fail exactly for $x<n$. 
Furthermore, it is possible to prove this failure for the region $(n+2)/3<x<n$ which rigorously establishes the right-side boundary at $x=n$ but leaves an unproved open space for the left-side boundary at $x=(n+2)^2/4(n+1)$. We provide this proof in Appendix \ref{app:crown}. 
The situation for the whole $x\in\mathbb{R}$ is illustrated in \cref{fig:SmallGraphs} (b). It is rather intriguing that the ``phase transition" points for the SDP ($x=(n+2)^2/4(n+1)$ and $n$), and the phase transition for the true ground state ($x=1+n/2$) are well-separated\knote{edited sentence}. This means that there are broad regions of the $x$ parameter where the SDP algorithm fails despite having exactly the same ground state as other points where SDP succeeds, which interestingly seems to be caused by the lack of real solutions in a quadratic equation \cref{eq:StrongTriangleSoS}.
In \cref{subsubsec:MG} and \cref{subsubsec:SS}, we will see more nontrivial phase transitions in condensed matter physics models. 





\subsubsection{Double Star Graphs}
While the crown graphs do not have the nice decomposition property that the complete bipartite graphs had, the double-star graphs have such a decomposition into two weighted star graphs. 
The double-star graphs are the ones with $n$ vertices connected to one vertex $a$, and the other set of $n$ vertices all connected to the other vertex $b$, and having an edge between $a$ and $b$ (thus $2n+2$ vertices in total). 

In this case, the decomposition works as 
\begin{equation}\label{eq:stardecomp}
     H 
    = \left( \frac{1}{2}h_{ab}+\sum_{i=1}^{n} h_{ai} \right)
    +\left( \frac{1}{2}h_{ab}+ \sum_{j=n+1}^{2n} h_{bj} \right), ~~~
    \| H \|
    = \biggl\lVert \frac{1}{2}h_{ab}+\sum_{i=1}^{n} h_{ai} \biggr\rVert
    +\biggl\lVert \frac{1}{2}h_{ab}+ \sum_{j=n+1}^{2n} h_{bj} \biggr\rVert,
\end{equation}
and the SoS reduces to the case of a weighted graph (with only one edge having weight $1/2$).
While the existence of exact $\SoS$ is provable for arbitrary weighted star graphs \cite{per_comm}, for the particular case corresponding to the double star we can have relatively simple analytical forms:%\knote{This might be tedious for the reader to verify, can you flesh it out in the appendix? {\bf Jun}: I'm good with maybe even moving this entire section to appendix. What do you think is best?} 
\knote{Double star graphs are pretty nice to include in the body, could definitely move the explicit SoS though}\jnote{I'm totally cool with that. I guess you are suggesting moving the Lv2 SoS to appendix and keeping Lv1 SoS = eq94 here?}\knote{yeah}
\begin{eqnarray}
\hspace{-5mm}
&~&\sum_{x=a,b}\biggl\{\biggl(\frac{E}{2} \mathbb{I} - \frac{2}{E} \sum_{i\in\partial x} h_{ix} - \frac{1}{E} h_{ab} 
    \biggr)^2
    +\frac{1}{E}\sum_{i\neq j\in\partial x}h_{ij}^2 ~~~~~~\nonumber\\
    &~& ~~+\frac{\sqrt{n/(n+2)}}{2n+1}\sum_{i\in\partial x}
    \left( h_{iy} - \left(n+1-\sqrt{n(n+2)}\right)h_{ix}+\frac{1}{2E-2}h_{ab}\right)^2 \biggr\} 
   = E\mathbb{I} - H,\label{eq:dssos}
\end{eqnarray}
where $E$ denotes the maximum eigenvalue, i.e., $E=(n+2+\sqrt{n(n+2)})/2$, and $\sum_{i\in\partial x}$ indicates summation over all vertices $i$ that are the $n$ leaves adjacent to $x=a$ or $b$. $y$ denotes the vertex $a$ or $b$ {\it other than} the one chosen for $x$ in the summation. 

While the above $\SoS$ shows that $NPA_1(\mathcal{Proj})$ obtains the ground state energy exactly for the double stars, the following $SoS_2(\mathcal{Proj})$ is simpler in form : 
\begin{eqnarray}\label{eq:lv2sos}
    &&\sum_{x=a,b}\left\{\biggl(\alpha \mathbb{I} - \beta \sum_{i\in\partial x} h_{ix} - \gamma h_{ab} 
    +\delta \sum_{i\in\partial x} h_{iy}
    \biggr)^2
    +\sum_{i\neq j\in\partial x}
    \left( \frac{\beta^2+\delta^2}{2}h_{ij}^2 + 2\beta\delta ( h_{ix}h_{jy} )^2\right) \right\}\nonumber\\
    %+2\beta\delta\sum_{x=a,b}\sum_{i\neq j\in \partial x } ( h_{ix}h_{jy} )^2 \\
    &=&
    \frac{n+2+\sqrt{n(n+2)}}{2}\mathbb{I} - H
    ,
    \end{eqnarray}
where the specific coefficients are 
\begin{comment}
\begin{eqnarray}
    \alpha &=& \frac{1}{2}\sqrt{n+2+s}\\
    \beta &=& \frac{1}{n+2}\left( \sqrt{n+2+s} +S\right)\\
    %\frac{\sqrt{2+n+s}}{2+n} + \sqrt{\frac{s+n(-2-n+s)}{n(2+n)^2}}\\
    \gamma &=& \frac{1}{n+2}\left( \sqrt{n+2+s} -\frac{n+s}{2}S\right)\\
    %\frac{2+n+s}{2+n}-\frac{1}{2}(n+s)\sqrt{\frac{s+n(-2-n+s)}{n(2+n)^2}}\\
    \delta &=& \frac{1}{n+2}    \left(    (n+1+s)S    -\sqrt{n+2+s}    \right) 
    ,
\end{eqnarray}    
\end{comment}
$\alpha=\sqrt{n+2+s}/2$, $\beta=(2\alpha+S)/(n+2)$, $\gamma=(2\alpha-(n+s)S/2)/(n+2)$, $\delta=((4\alpha^2-1)S-2\alpha)/(n+2)$,
with 
$s=\sqrt{n(n+2)}$, and $S=\sqrt{(1+2/n)^{1/2}+s-n-2}$.
Note that the $\SoS$ we provide here could again be viewed as an extension of the $\SoS$ for the complete bipartite case, just by adding another term to \cref{eq:CrownSOS}. Although this $SoS_2(\mathcal{Proj})$ Eq. (\ref{eq:lv2sos}) is weaker than $\SoS$ in terms of the SoS hierarchy, Eq. (\ref{eq:lv2sos}) straightforwardly shows that the ``two-singlet density" $\langle h_{ix}h_{jy}\rangle$ is always 0, a piece of information that was not obvious from the $\SoS$ Eq. (\ref{eq:dssos}). 

Interestingly, $NPA_1(\mathcal{Proj})$ starts to {\it fail} once the ``double star" becomes imbalanced, 
i.e. having different number of leaves on the two sides. 
This implies that the decomposition of the double graph \cref{eq:stardecomp} only holds for very precise cases with balanced double graphs and does not exist in general. 


\subsection{Complete graphs: Contrast between even and odd}\label{subsec:complete}
While the complete graphs $K_n$ do not admit similar decomposition as in \cref{eq:stardecomp}, 
we can still obtain the exact $\SoS$ by exploiting the high symmetry of the graph -- if the number of vertices $n$ is even: 
\begin{equation}\label{eq:EvenCompSOS}
    \sum_{i=1}^n 
    \left(
    \sqrt{\frac{n+2}{8}}\mathbb{I} - \sqrt{\frac{2}{n+2}}\sum_{j\neq i} h_{ij}
    \right)^2 
    = 
    \frac{n(n+2)}{8}\mathbb{I} - \sum_{1\leq i<j\leq n} h_{ij}. 
\end{equation}
Here again, the SoS is essentially a summation of the SoS for star graphs, but with slightly different coefficients, which makes them different from the simple decompositions we have been seeing. 
%This seems to be reflective of the fact that the maximum eigenvalue of the complete graph $K_n$ is exactly the same as the complete bipartite graph $K_{n/2, n/2}$, and therefore we can recycle the form of the SoS that worked for the complete bipartite graphs. 

The situation becomes quite different when the number of vertices is odd. The maximum eigenvalue is $(n+3)(n-1)/8$, but $NPA_1(\mathcal{Proj})$ gives $n(n+2)/8$ as the upper bound, which is $3/8$ bigger (observed numerically). 
% which has exactly the same formula for the even case but happens to be $3/8$ bigger (observed numerically). 
We can see that for the odd case the $NPA_1(\mathcal{Proj})$ must do {\it at least as good as} $n(n+2)/8$ from the fact that the SoS we have above works perfectly fine even when $n$ is odd. 
% The problem is that this value is simply not the maximum eigenvalue when $n$ is odd. 
%To further understand the problem, we can %A simple manipulation of the Hamiltonian for odd complete graphs reveals that 

Ideally for odd $n$, the exact SoS should give 
\begin{equation}\label{eq:xyz_sos_proof}
    \frac{(n+3)(n-1)}{8}\mathbb{I} - \sum_{1\leq i<j\leq n} h_{ij} \succeq 0
    ~~\Leftrightarrow ~~
    \sum_{i<j} \left( X_i X_j + Y_i Y_j + Z_i Z_j \right)+ \frac{3n-3}{2}\mathbb{I} \succeq 0. 
\end{equation}
%As we will see in the following, there are no $SoS_1(\mathcal{Pauli})$ proofs of this fact, so l
Let $\ell^*$ be the smallest integer such that $NPA_{\ell^*}(\mathcal{Pauli}(H))=\mu_{max}(H)$.  Since $NPA_\ell(\mathcal{Pauli})$ converges at $\ell=n$ we know $\ell^* \leq n$.
By exploiting the $SU(2)$ symmetry of the LHS, we can see that obtaining a degree-$\ell$ SoS proof for
\begin{equation}\label{eq:z_sos_proof}
    \sum_{i<j} Z_i Z_j + \frac{n-1}{2}\mathbb{I} \succeq 0 
    ~~\Leftrightarrow ~~
    \left(\sum_{i=1}^n Z_i \right)^2
    \succeq \mathbb{I} 
\end{equation}
%implies (i.e., a sufficient condition) 
would be a {\it sufficient} condition for showing
that $NPA_\ell(\mathcal{Pauli})=\mu_{max}\left( \sum_{i <j } h_{ij}\right)$.  
Since the Pauli operators $Z_i$ all commute,  
%Here, the $Z_i$'s are Pauli operators, so 
the problem essentially becomes classical and could be regarded as a {\scshape MaxCut} instance for the same odd complete graph. 
The problem then is equivalent to proving the following statement with SoS: 
\begin{center}
    {\it When you have odd numbers of $\pm 1$ values, their sum can never become 0.}
\end{center}
This trivial statement about parity becomes surprisingly hard to prove with SoS and is known to require $\lceil n/2 \rceil$-degree SoS \cite{gri01lin,lau03low,kun22spec}, so $\ell^* \leq \lceil n/2 \rceil$. 
While we believe that the same is most likely to be true for our case ($\ell^* =\Omega(n)$)\footnote{The tight SoS proof for {\scshape QMaxCut} on odd complete graphs can be reasonably named as the quantum version of the parity problem mentioned in the references.}, we were only able to prove the impossibility with $\SoS$. 

\begin{theorem}
    $NPA_1(\mathcal{Proj}(H))=n(n+2)/8$ for complete graphs with $n$ vertices, which gives the exact maximum eigenvalue when $n$ is even and is exactly $3/8$ larger than the exact maximum eigenvalue $(n+3)(n-1)/8$ when $n$ is odd.
\end{theorem}
% \begin{theorem}
%     The level-$1$ $\mathcal{Proj}$ obtains the energy upper bound $n(n+2)/8$ for complete graphs with $n$ vertices, even when $n$ is odd. In that case, the value is exactly $3/8$ larger than the true ground state energy $(n+3)(n-1)/8$. 
% \end{theorem}

\begin{proof}
    We show that the following constructed $M_1$ is a feasible solution for $NPA_1(\mathcal{Proj})$ that achieves the value $n(n+2)/8$. Together with the $\SoS$ in \cref{eq:EvenCompSOS}, this proves that the $NPA_1(\mathcal{Proj})$ gets the optimal value $n(n+2)/8$. 

    Now, consider the following moment matrix%\footnote{We drop the subscripts and superscripts to avoid cluttering.}
\begin{equation}
M=
\begin{pmatrix}
1      & a     & a     & \ldots & a      & \ldots \\
a      & a     & a/4   & \ldots & a/4    & \ldots \\
a      & a/4   & a     & \ldots & b      & \ldots \\
\vdots &\vdots &\vdots & \ddots &\vdots  &        \\
a      & a/4   & b     & \ldots & a      & \ldots \\
\vdots &\vdots &\vdots &        &\vdots  & \ddots 
\end{pmatrix},
\end{equation}
with 
\begin{equation}
    a=\frac{n+2}{4(n-1)}, ~~~ b=\frac{n(n+2)}{16(n-1)(n-3)},
\end{equation}
where the shown rows and columns are indexed by operators $\mathbb{I}, h_{12}, h_{13}, \ldots, h_{24}, \ldots$. In other words, 
\begin{equation}\label{eq:oddcompmmnew}
    M(h_{ij},h_{kl})=
    \begin{cases}
        a,   & (ij)=(kl), \\
        a/4, & (ij) ~\mathrm{and}~ (kl) \mathrm{~have~exactly~one~overlap},\\
        b,   & (ij) ~\mathrm{and}~ (kl) \mathrm{~have~no~overlaps}.
    \end{cases}
\end{equation}
It is easy to verify that this moment matrix has size $(1 + {n \choose 2})\times (1 + {n \choose 2})$, achieves energy $a\times {n \choose 2} = n(n+2)/8$, and satisfies the anti-commutation relation constraint: $((a+a-a)/2)/2=a/4$. 


All we need to do now is to show $M\succeq 0$, and we do this by constructing Gram vectors of $M$ \footnote{Alternatively, one can list all the eigenvalues of $M$ to show positive semidefiniteness, which has been the more traditional way to prove analogous results for the classical case \cite{gri01lin}. For completeness, we provide this in Appendix \ref{app:compeigen}.}. 
Specifically, we construct $1 + {n \choose 2}$ column vectors $\ket{\mathbb{I}}$ and $\{\ket h_{ij}\}$ for all $i,j\in [n]$ with $i<j$. 
Each column vector's elements are also indexed with the operators $\mathbb{I}$ and $h_{ij}$ as well, which we will denote as the subscript below.
We can then express the Gram vectors in the following way:
\begin{eqnarray}
    \ket{\mathbb{I}}_{\hat O}&=& 
    \begin{cases}
        1, & \text{ if } \hat{O}=\mathbb{I}\\
        0, & \text{otherwise},
    \end{cases}\label{eq:oddcompvec1}\\
    \ket{h_{ij}}_{\hat{O}}&=& 
    \begin{cases}
        %(n+2)/4(n-1), & \hat{O}=\mathbb{I}\\
        a, & \text{ if } \hat{O}=\mathbb{I}\\
        %\sqrt{3(n-3)(n^2-4)}/4\sqrt{{(n-1)^3}}, & \hat{O}=h_{ij}\\      
        \alpha, & \text{ if } \hat{O}=h_{ij}\\  
        %\sqrt{3(n+2)}/2\sqrt{(n-1)^3(n-2)(n-3)}, & \hat{O}=h_{jk} ~\text{with exactly one overlap}\\
        \beta, & \text{ if } \hat{O}=h_{kl} \text{ (no overlap with $ij$) }\\  
        %-\sqrt{3(n+2)(n-3)}/4\sqrt{{(n-1)^3(n-2)}}, & \hat{O}=h_{kl}~\text{with no overlap}.
        \gamma, & \text{ if } \hat{O}=h_{jk} \text{ (exactly one overlap with $ij$) },
    \end{cases}\label{eq:oddcompvec2}
\end{eqnarray}
with 
\begin{eqnarray}
    \alpha=\frac{\sqrt{3(n-3)(n^2-4)}}{4\sqrt{{(n-1)^3}}} , ~~
    \beta=\frac{\sqrt{3(n+2)}}{2\sqrt{(n-1)^3(n-2)(n-3)}} , ~~
    \gamma=-\frac{\alpha}{(n-2)}.
\end{eqnarray}
It is straightforward to confirm that these vectors Eq.~(\ref{eq:oddcompvec1}) and Eq.~(\ref{eq:oddcompvec2}) are indeed Gram vectors for the moment matrix  (\cref{eq:oddcompmmnew}) by a counting argument:
\begin{eqnarray}
    \braket{h_{ij}|h_{ij}}=& a^2 + \alpha^2 + {n-2 \choose 2}\beta^2 + (2n-4)\gamma^2 = a = \braket{\mathbb{I}|h_{ij}},\\
    \braket{h_{ij}|h_{kl}}=& a^2 + 2\alpha\beta + {n-4 \choose 2}\beta^2 + (4n-16)\beta\gamma +4\gamma^2 =b,\\
    \braket{h_{ij}|h_{jk}}=& a^2 + 2\alpha\gamma + {n-3 \choose 2}\beta^2 + (2n-6)\beta\gamma +(n-2)\gamma^2=a/4,
\end{eqnarray}
thus concluding that $M$ is the optimal $M_1$ of $NPA_1(\mathcal{Proj})$ achieving the value $n(n+2)/8$. 
\end{proof}

We can observe that the moment matrix that SDP creates is essentially ``blind to the fact that $n$ is an integer" \cite{gam22dis}
and is the reason for obtaining the wrong value $n(n+2)/8$. 
This is the energy you would get when you naively plug in an odd number to the formula for even complete graphs. 
Motivated by this fact and realizing that most of the higher order terms in the higher level moment matrix would reduce to lower degree moments (just like $a/4$ in the example above), 
we conjecture that the only independent moment matrix elements in higher levels would be 
\begin{equation}
    \left\langle 
    %\prod_{\mathrm{independent}}^{k} h_{ij}
    \overbrace{h_{ij}\cdot h_{st}\cdot \ldots}^{k \text{ independent op.s}}
    \right\rangle 
    = \prod_{l=0}^{k-1}\left( \frac{n+2-2l}{4(n-2l-1)}\right), 
\end{equation}
which is the formula for an even complete graph, but simply formally replacing $n$ with an odd number, resembling the classical case \cite{gri01lin,lau03low,kun22spec}. 
All other matrix elements would be calculable from the projector algebra constraints. 



\section{Numerical results}
While the SoS proofs in the previous section only cover a very small fraction of possible uniformly weighted graphs, the SDP algorithm actually solves surprisingly many graphs exactly, in the sense that the obtained upper bound value matches the exact maximum eigenvalue. This is true for both the $\mathcal{Pauli}$ and $\mathcal{Proj}$ SDP relaxations, and in this section we will go through the numerical results showing this. 
We further observe that the SDP algorithm can be used for calculating expectation values of operators that are of physical interest. 
This is demonstrated in section \ref{subsec:HeisenbergChain}, where the $NPA_1(\mathcal{Proj})$ is applied to the Heisenberg Chain up to size $L=60$, and the critical correlation functions show the correct criticality up to error bars. 

In the following of this section, the term ``solve exactly" means that the upper bound value obtained by SDP theoretically matches exactly with the maximum eigenvalue.


\subsection{Exhaustive numerical results on small graphs}


Here, we show the results of $NPA_2(\mathcal{Pauli})$ applied to all possible uniform graphs up to $n=8$ vertices. 
The main observation is that $NPA_2(\mathcal{Pauli})$ is exact for many graphs with $n\leq 6$ vertices. While the percentage of such graphs seems to shrink as we go to larger system sizes, it suggests that there are many cases where an exact SoS exists that are not covered in the previous section. 

\subsubsection{Probing exact solvability numerically}

Before presenting the main numerical results, here we address the subtle issue arising from numerical precision of the SDP algorithm. 
It is fundamentally impossible to determine whether the SDP algorithm obtains the correct energy value for a particular Hamiltonian solely from numerical results. 
This is because the SDP algorithm always requires a precision parameter which is usually referred to as ``error-tolerance" $\epsilon$, and the algorithm only optimizes up to that $\epsilon$. 
Even if the algorithm seems to give very close values to the true energy we cannot {\it a priori} conclude if that is actually obtaining the exact solution, or if the error of the algorithm is merely small yet non-zero. 


% Figure environment removed

To address this issue systematically, we analyzed the optimal value (upper bound of the maximum eigenvalue) obtained by the SDP algorithm as a function of the error tolerance. 
More precisely, as shown in Fig. \ref{fig:Precision}, we plot the discrepancy of the SDP-obtained optimal value and the exact maximum eigenvalue $\Delta E :=|E_{\mathrm{SDP}} - E_{\mathrm{GS}}|$ as a function of $\epsilon^{-1}$. 
This plot, especially for $n=5$, shows a very clear dichotomy of $n=5$ connected graphs. While 7 graphs (red curves) have an almost constant $\Delta E >0$, the rest of the 14 graphs (blue curves) show a decay in $\Delta E$, roughly proportionally to $\epsilon$. 
This could be regarded as strong numerical evidence that the 14 graphs are exactly-solvable instances by the SDP algorithm while the 7 graphs are not. It is quite surprising that a simple five-vertex graph can naturally yield a very small error value around $0.00034$ (the graph shown in Fig. \ref{fig:Precision} with arrows in magenta). 

However, we must note that this method is not entirely decisive. 
As depicted in the center and right panels of Fig. \ref{fig:Precision}, the dichotomy becomes less clear as we go to larger sizes $n=6,7$, and is even worse for $n=8$ (not shown). 
This is because as we proceed to larger system size, an unweighted graph can potentially have extremely small error values $\Delta E$, such as $\sim 10^{-8}$ and even smaller. 
At some point, it practically becomes impossible, since smaller error tolerance $\epsilon$ requires longer iterations in the SDP optimization. 

We can also see that the theoretical error bound of $\Delta E < \epsilon$ (drawn in yellow lines in the figure) for any exactly-solvable graph, is not necessarily satisfied always. For example, although we rigorously prove that the star graph is exactly solvable by the SDP algorithm (see \S\ref{subsec:compbip}), the error of the star graph in Fig. \ref{fig:Precision}, $n=5$ (in cyan) is slightly above the error tolerance $\epsilon$. 
This arises from subtleties in how the error tolerances are handled inside the SDP package, and is difficult to control in general. 

Despite these subtleties, the behavior of the absolute energy error $\Delta E$ as a function of the error tolerance $\epsilon$ serves as a good rule of thumb for distinguishing exactly-solvable graphs from instances with merely small errors. 
For instance, we can be fairly confident that the graphs with magenta arrows indeed do have extremely small but non-zero errors such as $\sim 10^{-6}$.


\subsubsection{Exactly solvable small graphs and their statistics}\label{subsubsec:smallstat}

% Figure environment removed


Once we can confidently determine whether or not the SDP algorithm obtains the true ground state energy, we can start to ask questions such as ``When and how often does the SDP algorithm give us the exact solution?". 
To address this question, we present an exhaustive study for all connected graphs with $n=5,6,7$ and $8$ vertices. 


Figure \ref{fig:MetaGraphs} shows all of the 7 (out of 21) $n=5$ connected graphs and the 17 (out of 112) $n=6$ connected graphs that the $NPA_2(\mathcal{Pauli})$ SDP algorithm fails to obtain the exact ground state energy (colored in red/magenta). The numbers are labeling of the graphs according to a convention introduced in \cite{cve84ata}. 
It is rather surprising that the algorithm obtains the exact ground state energy for the vast majority of the graphs (colored in blue/cyan) up to this system size, noting that for most of the graphs the SoS is unknown and most likely very complicated (graphs in cyan). 

The figure also shows the topological relations of the graphs, by connecting them with a thick bond whenever two graphs only differ by one edge. 
In this way, we can see that for $n=5$ the red/magenta graphs (SDP fail) form one cluster. In other words, any two $n=5$ connected graphs that Lv. 2 Pauli SDP fails, can be transformed into one from the other by adding and subtracting one edge at a time, always maintaining the SDP algorithm to be failing. 
This is not the case for $n=6$, where the magenta graphs seem to form one big cluster and also three disconnected ``islands" (namely, graphs 20, 28, and 69). However, as we will see in the following, the ``single-clusteredness" of the hard graphs recovers once we focus on the errors from the $NPA_1(\mathcal{Proj})$. 

The ``failing cluster" includes the complete graph for $n=5$ but not for $n=6$. This is exactly as expected as we explained in \cref{subsec:complete}. This raises the question whether we can actually further constrain the SDP algorithm, not with a higher level, but simply by adding a constraint corresponding to the minimum total spin of the ground state. More specifically, the constraint would be 
\begin{equation}
    \sum_{1\leq i<j\leq n} M(\mathbb{I}, h_{ij})\leq \frac{(n+3)(n-1)}{8}, 
\end{equation}
from \cref{eq:xyz_sos_proof} for odd $n$. 
When we add this constraint, $NPA_1(\mathcal{Proj})$ not only was able to solve the complete graph $K_5$ exactly, but other graphs in the vicinity. 
This information is indicated in \cref{fig:MetaGraphs}, by showing graph 12 and 18 in red, being the only two graphs that $NPA_1(\mathcal{Proj})$ with this additional constraint still failed. 
Note that we cannot do the same thing when we have even number of qubits, because $NPA_1(\mathcal{Proj})$ already succeeds for the complete graphs, i.e., already {\it know} about this constraint on total spin. 



% Figure environment removed

We also compare the performance of the different SDP algorithms ($NPA_2(\mathcal{Pauli})$, $NPA_2^{\mathbb{R}}(\mathcal{Pauli})$, and $NPA_1(\mathcal{Proj})$) for all of these graphs up to $n=8$ in Fig. \ref{fig:PauliProjCompare}. 
The scatter plot shows the energy errors for $NPA_2(\mathcal{Pauli})$ and $NPA_1(\mathcal{Proj})$. The fact that the scattered points roughly forms four different clusters could be understood in the following way. 

Firstly, the cluster on the top right corresponds to graphs that the SDP algorithms with either bases fail to obtain the exact ground state. 
If we believe in typical hardness of the random {\scshape QMaxCut} instances, the ratio of graphs in this cluster in the scatter plot should reach 1 in the large problem size limit. 
The fact that all of the points in this cluster are on the left of the black line indicating $x=y$ reflects the fact that the $NPA_2(\mathcal{Pauli})$ SDP can never perform worse than the $NPA_1(\mathcal{Proj})$ SDP. This could be easily seen from the fact that you can always convert an SoS proof using degree-1 polynomials of projectors into SoS that uses degree-2 Pauli polynomials, but not necessarily the other way around. 


Whether the aforementioned inequality $NPA_2(\mathcal{Pauli}(H))\geq NPA_1(\mathcal{Proj}(H))$ is actually an equality or not for {\scshape QMaxCut} instances is  not obvious until we actually see examples. 
The second cluster on the top left of \cref{fig:PauliProjCompare} reflects exactly that there are indeed graphs where $NPA_2(\mathcal{Pauli})$ SDP is exact but $NPA_1(\mathcal{Proj})$ SDP fails, i.e., that the inequality is {\it strict} in general. 
We list up all the $n=5$ and 6 graphs that fall under this second cluster on the right side of \cref{fig:PauliProjCompare}. 
Furthermore, we also checked how $NPA_2^{\mathbb{R}}(\mathcal{Pauli})$ SDP performs on these graphs to find that the inequality $NPA_1(\mathcal{Proj}(H)) > NPA_2^\mathbb{R}(\mathcal{Pauli}(H)) > NPA_1 (\mathcal{Pauli}(H))$ is also strict in general\footnote{The nonstrict inequality could be quickly understood in the same manner as the argument in the previous paragraph}. 
Specifically, we find that $NPA_2^\mathbb{R}(\mathcal{Pauli})$ fails for all of  the graphs shaded in \cref{fig:PauliProjCompare}, while it succeeds for all of the other graphs with $n=5$ and 6.
This means that the exact Pauli SoS for unshaded graphs are ``breaking the SU(2) symmetry" in the individual squares possibly by having one-body Pauli terms in them. Those effects must cancel out as a whole when all the SoS terms are added since the final Hamiltonian has SU(2) symmetry and has no one-body terms. 
For the shaded graphs, this ``symmetry breaking" trick is not enough to obtain the exact SoS, and complex SoS are required to do so. 
As a concrete example, the graph labeled 8 in \cref{fig:PauliProjCompare} has errors $1.4\cdot{} 10^{-2}$, $5.7\cdot{}10^{-4}$ and $8.06\cdot{}10^{-12}$ for $NPA_1(\mathcal{Proj}(H))$, $NPA_1^\mathbb{R}(\mathcal{Pauli}(H))$ and $NPA_1 (\mathcal{Pauli}(H))$ respectively, which we interpret as the complex Pauli hierarchy being exact on this instance, but the real Pauli and complex projector hierarchy have nonzero errors.


The third cluster on the bottom left corresponds to graphs where the SDP algorithm succeeds with either of the bases. The ratio of the graphs in this third category seems to decrease as we get to larger sizes of graphs, which we will discuss further later. 
Noticing that the separation between $NPA_2(\mathcal{Pauli})$, $NPA_2^{\mathbb{R}}(\mathcal{Pauli})$, and $NPA_1(\mathcal{Proj})$ are strict in general from the previous paragraph, it seems more natural to regard this cluster as instances where $NPA_1(\mathcal{Proj}(H))=0$ forces the other two SDPs to have 0 error as well. 
From this perspective, it is more intriguing when $NPA_1(\mathcal{Proj}(H))=NPA_2(\mathcal{Pauli}(H))>0$, i.e., exactly on top of the $x=y$ line in \cref{fig:PauliProjCompare}, but in the top right cluster. 
Up to $n=8$ connected graphs we have computed, the only cases when that happens are all graphs related to complete graphs (simplest cases discussed in \cref{subsec:complete}).



There is a rather small fourth cluster on the right bottom, that extends beyond to the right side of the $x=y$ line. 
Since $NPA_2(\mathcal{Pauli})$ must always perform no worse than $NPA_1(\mathcal{Proj})$, this suggests a numerical error of some sort. We have observed that the SDP packages for these graphs do not converge as quickly as other graphs, and tends to give results that have larger duality gaps than specified. This practically does not become a problem since the errors are very small (around $10^{-6}$), and all graphs which we explicitly exemplify as ``NPA failing" in this work are not from this group\footnote{This may occur strange to the physicist readers that a convex optimization which theoretically does not have a local minimum, still seems to ``get stuck" in practice. This is actually not uncommon in the field of convex optimization, since e.g. a very narrow feasible region can cause practically slow convergences like this. \lunote{There's no local minimum in convex optimization problem, but it's uncommon that convex optimization problems have slow convergence towards the global minimum due to the shape of the convex cone. I think it's better to remove this sentence.} \jnote{How about in a footnote with a wording like this?}} .
Notably, instances falling on the right side of the $x=y$ line only occur at very small errors (bottom right), while none are observed in the top right cluster. This is 
encouraging, since we can be confident that these practically pathological cases only arise when we demand high numerical precision. This allows us to consider all of the graphs in the fourth cluster (bottom right) to be theoretically easy for both bases of SDP, i.e., actually belonging to the third cluster (bottom left).


%We obtained many interesting findings from the results, including a separation between real and complex hierarchies.  The graphs labeled $8$ and $15$  provide examples where the numerics suggest $NPA_1(\mathcal{Proj}(H)) > NPA_1^\mathbb{R}(\mathcal{Pauli}(H)) > NPA_1 (\mathcal{Pauli}(H))$ with {\it strict} inequalities.  Specifically the graph labeled $8$ obtains errors $1.4\cdot{} 10^{-2}$, $5.7\cdot{}10^{-4}$ and $8.06\cdot{}10^{-12}$ for $NPA_1(\mathcal{Proj}(H))$, $NPA_1^\mathbb{R}(\mathcal{Pauli}(H))$ and $NPA_1 (\mathcal{Pauli}(H))$ respectively.  So, while the complex Pauli hierarchy appears to be exact on this instance, the real Pauli and complex projector hierarchy have significant errors.  \jnote{I changed things a bit.}

% Figure environment removed


In order to see the statistics of the errors more closely, in \cref{fig:errorstats} (a), we show the values of the error for the $NPA_1(\mathcal{Proj})$ SDP in descending order for each size of graphs $n=5,6,7$ and $8$. 
The $x$-axis is rescaled so that the data of 21, 112, 853, and 11,117 graphs all fit into $[0,1]$. Thus, the figure is the inverse of the cumulative distribution function of errors. 

For example, all four curves display an acute decline at some point corresponding to the separation between graphs that have nonzero errors and (essentially) zero error. The graph shows that the ratio of such non-exactly solvable graphs are roughly $46\%, 37\%, 91\%$ and $94\%$ among all connected $n=5,6,7$ and $8$-vertex graphs respectively. This means that the ratio of exactly solvable graphs tend to decrease as the number of vertices increases, possibly converging to 0 in the $n\rightarrow \infty$ limit. 
%This is not too bad of a news for the SDP algorithm, since the number of connected graphs itself grows very fast. 
%\knote{This sentence is confusing me.  Do you mean that there are still a lot of graphs solved exactly via SDP? {\bf Jun}: Yes. I deleted it and changed things around.}
Yet still, the actual {\it number} of connected graphs that are exactly solvable seems to grow with $n$ at least for this size regime: 11, 67, 77, and 670, for $n=5,6,7$ and 8. 

Another piece of information in the graph, represented as the points in the figure, is how the {\it bipartite} graphs are distributed among this descending-error ordering. The {\scshape QMaxCut} problem on bipartite graphs is oftentimes described as having ``no geometric frustration" in condensed matter physics, since the singlet projector $h_{ij}$ could be seen as a constraint that favors the two qubits to be pointing in the opposite direction\footnote{Not to be confused with ``frustration-free" explained in section \ref{subsubsec:MG}.}. From this point of view, we would consider an odd-length loop as geometrically frustrated because the interaction would not be (even relatively) satisfied with a simple approach of having the qubits point the opposite directions alternately. 
This difference has practical applications, such as bipartite cases allowing the quantum Monte Carlo method to efficiently\footnote{Only known empirically, in terms of precise complexity theory statements. While the time complexity scaling is known to scale as $\mathcal{O}(\epsilon^{-2})$ with respect to the error tolerance $\epsilon$, the scaling with number of qubits $n$ is hard to bound rigorously for Markov-chain Monte Carlo methods in general, albeit cases of quantum Monte Carlo methods being applied to hundreds or thousands of qubits is common in computational physics \cite{san10com}.} obtain the ground state classically. 
Therefore, it is not so surprising that the bipartite graphs in Fig. \ref{fig:errorstats} (a) are distributed relatively on the right side of each curves, implying (exponentially) smaller errors. In some sense, the surprise is in the other direction, that SDP fails to obtain the exact ground states of such ``easily classically simulable" instances most of the time. 
It is unclear if the tendency of bipartite graphs having relatively small errors will remain for larger $n$, since it is already apparent that the position of the largest-error bipartite graph shifts to the left in Fig. \ref{fig:errorstats} (a) from $n=7$ to $n=8$.

%In the $n\rightarrow\infty$ limit, the uniform distribution among all connected graphs with $n$ vertices will be indistinguishable from a random Erd{\"o}s-R{\'e}nyi graph ensemble with ${n \choose 2}/2$ edges in total  \cite{rad67uni}.\knote{I don't understand this.  A vertex in a graph cant have degree ${n\choose 2}/2$ since this could be larger than the number of vertices in the graph.  Also you might need to say in what sense the graphs are indistinguishable.}\jnote{Thanks, that was a typo and I corrected the statement, hopefully making it clearer. But maybe we can just delete some sentences here?}
%This means that the value of the relative error will be tightly concentrated to a single value, making the curves in Fig. \ref{fig:errorstats} (a) to have a flat plateau covering the entire $(0,1)$. This explains the similarity of the two curves for $n=7$ and 8 that is already apparent due to the large number of graphs (853 and 11,117), despite $n$ itself being rather small.

In order to test the difference between bipartite graphs and non-bipartite graphs in a more systematic way, we also ran the SDP algorithm for random regular graphs with degree-3. When such graphs are generated uniformly randomly, for sufficiently large $n$, the graph is almost certainly non-bipartite. 
We generate 100 of such samples, and compare the performance of $NPA_1(\mathcal{Proj})$ 
against exact diagonalization for $n=12, 16$ and 24. 
It is also possible to generate uniformly random graphs that are bipartite and regular, and both results are displayed in Fig. \ref{fig:errorstats} (b). 
It is immediately apparent that the non-bipartite random regular graphs have a broader distribution in the two-dimensional scatter plot, compared to the bipartite cases. The cluster is also located farther away from the $x=y$ line in black, showing a larger relative error compared to bipartite random graphs. The bipartite random graph data also seem %agree well with a polynomial fit, 
 to form a ``line" in the scatter plot, 
indicating that the optimal SDP objective can give a fairly narrow estimate of the true energy value by a properly fitted linear function. In contrast, the non-bipartite random graph data extends in a two-dimensional manner forming a oval-like shape, resulting in broader estimates of the true energy given the SDP energy. 



\subsubsection{Transition points in the solvability of small graphs}

% Figure environment removed

The clusteredness of hard and easy graphs shown in Fig. \ref{fig:MetaGraphs} leads to the question of what happens at the boundary between them. 
If there is a pair of graphs which one is exactly solvable while the other is not, with only one edge difference as graphs, then we can add that one different edge with weight $x\in[0,1]$. 
This procedure continuously connects the graphs and demonstrates where exactly SDP starts to fail. 

In \cref{fig:PauliProjCompare}, we show three different cases of such a procedure. 
On each panel, we show the graph we use for demonstration, with the dotted edge being the weighted one. 
The left most panel shows the case for interpolating between the $n=4$ star graph and the Y-shaped $n=5$ graph (graph \# 20 in \cref{fig:PauliProjCompare} (a)), which is the easiest case of such. In this case, we can see that the moment we add $\epsilon>0$ amount of the new edge, SDP starts to fail. 
This could be argued that the solvability of the star graph in this situation is rather {\it fragile}, and immediately fails when perturbed away. 

The same thing could be argued for the case shown in the middle panel connecting graph \#69 and \#47 of \cref{fig:MetaGraphs} right. 
Again in this case, the moment the graph diverges away from the exactly solvable \#47, the SDP algorithm starts to fail. 
However, there exist cases where the ``transition" happens not at the edges but at a nontrivial value, as shown in the right panel. The error becomes as small as the duality gap set for the SDP solver for $x<0.22$. In this case, we can say that the solvability of graph \#47 is somewhat robust, and survives the perturbation in the direction considered here (towards graph \#28). 

Curiously, for all cases we have checked for interpolations between solvable and unsolvable graphs with $NPA_1(\mathcal{Proj})$, we always observe a quadratic initial increase of the error, as shown with the red dotted lines in \cref{fig:SmallTrans}. The quadratic fit is extremely good at the vicinity of the ``transition points" where the error starts to become nonzero, as shown in the insets of the figures. 
This resembles universal critical behavior seen in physics, where phase transition {\it points} vary largely depending on the details of the statistical physics model, but an indicator of the phase transition (called the order parameter) behaves as $\propto |T-T_{c}|^{\beta}$ with a universal exponent denoted by $\beta$. 
Although our observed exponent $\beta=2$ is clearly present numerically, we were unable to provide a general explanation, and leave it for future studies. 




\subsection{Numerical results for some condensed matter physics models}\label{subsec:cmp}

Here, we demonstrate the power of the SDP algorithm when applied to a number of condensed matter physics models. The message is two-fold: first, the SDP algorithm could be used to probe exact-solvability of models in some settings, giving rise to the possibility of numerical exploration for exactly-(analytically) solvable systems. Second, the method could be seen as the first-order approximation of the ground state, it actually gives very accurate numbers in practice, with errors only up to $\sim 4\%, 7\%, 2\%$ for the models we study. 

Both the Majumdar-Ghosh model and the Shastry-Sutherland model are known to be ``frustration-free" in the quantum spin system literature \cite{deb10sol,sat16whe,wou21int,ans22ana}. 
This means that the Hamiltonian could be rewritten as sum of terms that could all be satisfied simultaneously in the ground state. 
The standard way to show this is to rewrite the physical Hamiltonian (thus with the opposite sign from our {\scshape QMaxCut} convention as in Eq. (\ref{eq:QMCHamDef})) as a sum of projectors with additive and multiplicative constants. 
If there exists a state that the all the projectors evaluate to 0, that must be the ground state (note that the physics convention here is a minimization of the eigenvalue). 

Because all projectors are square of themselves, we can immediately obtain an SoS of some degree by flipping the entire sign, and redefining the Hamiltonian with the {\scshape QMaxCut} convention. 
In most cases in physics, the definition of ``frustration-free" requires the rewritten terms of the Hamiltonian to be spatially local. Thus, the SoS hierarchy could be seen as a generalization of the frustration-free notion, where we do not necessarily require spatial locality, but restrict the degree of the terms as polynomials.
The fact that the degree-restriction could be arbitrarily relaxed by the level of the hierarchy, and that SDP algorithms can solve the optimization problem efficiently for $\mathcal{O}(1)$ level, provides us a systematic approach to explore frustration-free Hamiltonians in a computational way. 

\subsubsection{The Majumdar-Ghosh model}\label{subsubsec:MG}

% Figure environment removed

The Majumdar-Ghosh (MG) model \cite{maj69nex} was one of the earliest proposed quantum spin models where the ground state could be obtained exactly. 
The Hamiltonian being considered is 
\begin{equation}\label{eq:MGHam}
    H = \sum_{i=1}^L \Bigl(J_1 h_{i,i+1} + J_2  h_{i,i+2}\Bigr), 
\end{equation}
where we use the {\scshape QMaxCut} convention (i.e. the ``ground state" we are searching for now is the maximum eigen state of this operator). The lattice structure is shown in Fig. \ref{fig:lattices}(a). 
The Hamiltonian above would typically be referred to as the $J_1$-$J_2$ Heisenberg chain in the context of condensed matter physics, and the MG model corresponds to the case where $J_2/J_1=1/2$, also known as the MG {\it point}. 
At the MG point, the ground state is two-fold degenerate with two different singlet-product states where closest neighbors are forming singlets periodically, as depicted in Fig. \ref{fig:lattices}(a): 
\begin{equation} 
    |\mathrm{GS}\rangle = 
    \prod_{i\in \mathrm{even}}^{\otimes}|s_{i,i+1}\rangle
    +\prod_{i\in \mathrm{odd}}^{\otimes}|s_{i,i+1}\rangle ,
\end{equation}
where by $|s_{i,j}\rangle$ we denote a singlet state between the spins on sites $i$ and $j$.

The exact ground state could be understood from the fact that the Hamiltonian at the MG point decomposes in the same sense of Eq. (\ref{eq:CBhamdecomp}). 
Specifically,
\begin{equation}
    H=\frac{J_1}{2}\sum_{i=1}^L \Bigl(h_{i,i+1} + h_{i+1,i+2} + h_{i,i+2}\Bigr), ~~~~
    \lVert H\rVert=\frac{J_1}{2}\sum_{i=1}^L \Big\lVert h_{i,i+1} + h_{i+1,i+2} + h_{i,i+2}\Big\rVert 
\end{equation}
holds. 
Since the individual terms after this decomposition reduces to a triangle with equal weights, we can reuse the exact $\SoS$ from Eq. (\ref{eq:WeakTriangleSoS}), obtaining 
\begin{equation}
    \frac{3L}{4}\mathbb{I} - H 
    = 
    \sum_{k=1}^{L}
    \frac{3J_1}{2}
    \biggl\{
        \mathbb{I}-\frac{2}{3}
        \bigl(
            h_{k-1,k} + h_{k,k+1} + h_{k-1,k+1}
        \bigr)
    \biggr\}^2
\end{equation}
for the periodic boundary condition ($L+i\equiv i$). 
This corresponds to the standard projector expression for frustration-free models as we mentioned earlier, and implies that $NPA_1(\mathcal{Proj})$ is able to obtain the exact ground state energy at the MG point. 
%Since the above SoS is constructed using only linear terms of the singlet projector operators, $NPA_1(\mathcal{Proj})$ should be able to obtain the exact ground state at the MG point. 

% Figure environment removed

This is demonstrated in Fig. \ref{fig:MG}, where we compare the exact energy $E_{\mathrm{GS}}$ and the SDP energy $E_{\mathrm{SDP}}$ for the $L=16$ case with periodic boundary condition. 
The fact that the SDP algorithm obtains the exact ground state energy is reflected as the two energy density values coinciding at the MG point in the left panel, and correspondingly in the center panel the relative error becomes 0. 
Interestingly, the SDP algorithm seems to be ``more sensitive" to the MG point in this $J_1$-$J_2$ model compared to the actual ground state energy, when we try to detect it by looking into the derivatives of the energy (right panel). 


The fact that in Fig. \ref{fig:MG} we see the SDP algorithm only obtaining the energy exactly at the MG point establishes that there are no other exactly-solvable points in the $J_1$-$J_2$ model, even if we allow non-local terms as long as they are limited to degree-2 in polynomials of singlet projectors. 




\subsubsection{The Shastry-Sutherland model}\label{subsubsec:SS}

% Figure environment removed

The Shastry-Sutherland (SS) model \cite{sha81exa} is a two-dimensional Heisenberg model that also admits an exact ground state representation for a certain parameter region. 
The Hamiltonian in the {\scshape QMaxCut} convention would read 
\begin{equation}
    H=J\sum_{\langle ij\rangle} h_{ij} + 2\alpha \sum_{\langle\hspace{-0.5mm}\langle ij\rangle\hspace{-0.5mm}\rangle} h_{ij}
\end{equation}
where $\langle ij\rangle$ represents bonds of the $L\times L$ square lattice (with periodic boundary condition) and $\langle\hspace{-1mm}\langle ij\rangle\hspace{-1mm}\rangle$ represents diagonal bonds of the Shastry-Sutherland lattice as illustrated in Fig. \ref{fig:lattices}(b). For simplicity, we fix $J=1$. 

This model has an obvious unique ground state when $\alpha$ is large enough, since the diagonal bonds with weight $2\alpha$ gives a perfect matching of the sites. In that parameter region, the unique ground state could be written as 
\begin{equation}\label{eq:SSGS}
    |\mathrm{GS}\rangle = 
    \prod_{\langle\hspace{-0.5mm}\langle ij\rangle\hspace{-0.5mm}\rangle}^{\otimes}|s_{i,j}\rangle , 
\end{equation}
again illustrated in Fig. \ref{fig:lattices}(b). 

For the SS model with $\alpha>1$, we are again able to decompose the Hamiltonian into triangles with weights 1, 1, and $\alpha$, as previously discussed in section \ref{subsubsec:crown}. 
It is easy to check that the Shastry Sutherland lattice (Fig. \ref{fig:lattices} (b)) can be decomposed into such triangles geometrically, with all triangles having two edges from the square lattice and one from the diagonal edges. 
Now, we can reuse Eq. (\ref{eq:StrongTriangleSoS}) to obtain 
\begin{eqnarray}\label{eq:SS-SOS}
    &&\left(\alpha+\frac{J}{2}\right)N\mathbb{I} - H \nonumber\\
    &=& 
    \sum_{\triangle}
    \left(\alpha+\frac{J}{2}\right)
    \biggl\{
    \mathbb{I}
    -\sum_{\mathrm{edges}\in\triangle}    \frac{4\alpha+2J\pm(-2)^{j}\sqrt{2\alpha(2\alpha-J)-2J^2}}{3J+6\alpha}h_{\mathrm{edge}}
    \biggr\}^2 , 
\end{eqnarray}
which gives the exact value only when $\alpha\geq J$. 
The summation $\sum_{\triangle}$ is taking the summation for all right triangles in the SS lattice as in the decomposition, and the summation inside of the square is for the three different edges for each such triangle. The $(-2)^j$ factor only appears for the edges with weight $J$ belonging to the square lattice where we set $j=1$, and not for the diagonal edges with weight $\alpha$ which we set $j=0$. Just as in Eq. (\ref{eq:StrongTriangleSoS}), the SoS has a degree of freedom in choosing $\pm$ for the square root term. 


In Fig. \ref{fig:SS}, we demonstrate the performance of the $NPA_1(\mathcal{Proj})$ SDP algorithm applied to the SS model with system size $n=L^2=16$ and $J=1$ fixed. We can see that the algorithm obtains the exact ground state energy for the entirety of the $\alpha\geq 1$ region, which exactly coincides where Eq.~(\ref{eq:SS-SOS}) gives a proper SoS (otherwise it has no real coefficients), and also the decomposition exists. 
The true ground state actually becomes the dimer singlet state \cref{eq:SSGS} from $\alpha\geq3/4$ for this system size, although the SDP algorithm fails to obtain that. 
This means that while $\alpha\geq 1$ was the condition used to show frustration-freeness in \cite{sha81exa}, relaxing the notion to allow non-local terms (but still only having degree-1 terms in the SoS) does not enlarge the region of exact-solvability. It would be interesting to see how the exactly solvable region changes as a function of the level of the NPA hierarchy. 

Furthermore, by looking at the first derivative of the SDP energy as a function of $\alpha$, we can clearly see that there are two points where $\partial E/\partial \alpha$ has a singularity (\cref{fig:SS} right), namely $\alpha \simeq 0.73$ and $\alpha =1$. 
The existence and the nature of different phases in the SS model is actively discussed in the condensed-matter physics context, where there is expected to be at least two phase transition points, i.e., singular points \cite{kog00qua,lee19sig,yan22qua}. 
The fact that the SDP energy derivative exhibit two singular points from relatively small system sizes suggest the possibility of this approach be used to detect phase transitions in similar models, without relying on comparison with exactly obtained ground states. The SDP algorithm also allows us to calculate observables other than energy such as the squared N{\'e}el order parameter from the moments, with a guarantee that they converge to the true value when the NPA hierarchy converges to the true ground state energy value. 
This lets us to interpret such physical observables obtained this way to be regarded as a first-order approximation. In the next section we see that even such approximated quantities can show essential characteristics in physical systems. 

Another thing to note is that both Hamiltonians for the SS model and the MG model allowed decomposition of the Hamiltonian as in \cref{eq:CBhamdecomp} and \cref{eq:stardecomp}. In the cases of SS and MG models, the sub-Hamiltonians were the triangles with $\alpha,J,J$ bonds and $J_1/2, J_1/2, J_2=J_1/2$ bonds respectively. 
However, it should be noted that the existence of such decomposition is not a necessary condition for obtaining exact SoSs. For example, for the crown graph which we present an exact SoS in section \ref{subsubsec:crown}, it appears that there are no such decomposition, while still having an exact SoS. 
The same could be said for complete graphs with even number of vertices. 
This fact gives us hope on discovering new exactly-solvable Hamiltonians, since oftentimes the search for frustration-free Hamiltonians relies on the existence of such decomposition \cite{gho23exa,kum02qua}. 

\subsubsection{The Heisenberg chain}\label{subsec:HeisenbergChain}
The nearest neighbor antiferromagnetic Heisenberg chain is one of the simplest yet nontrivial quantum spin system that also happens to be a {\scshape QMaxCut} instance. 
The Hamiltonian we consider here is simply the chain 
\begin{equation}
H=\sum_{i=1}^L h_{i,i+1}, 
\end{equation}
with a periodic boundary condition $L+k \equiv k$. 
This corresponds to setting $J_2=0$ for the $J_1$-$J_2$ model in section \ref{subsubsec:MG}. 
Although the Heisenberg chain has an exact solution thanks to the Bethe ansatz \cite{bet31zur}, the exact solvability of the model is quite different from the previous two models: it does not involve frustration-freeness, and our SDP algorithm is therefore not expected to solve it exactly. 

% Figure environment removed

In the left panel of \cref{fig:Cycle}, we show our numerical results on how the SDP algorithm performs on the Heisenberg chain, by comparing the $NPA_1(\mathcal{Proj})$, $NPA_2(\mathcal{Pauli})$, and the exact value for various system sizes. 
We plot the energy {\it density} $E/L$ here, so the fact that all three cases converge to different values indicate that the absolute error of the total energy increases linearly with the system size $L$ for large enough $L$. Still, the qualitative behavior of approaching the limiting value from above and below for even and odd $L$ is reproduced in both of the SDP methods. 

In the right panel, we show the correlation function 
\begin{equation}\label{eq:corr}
    C(r) := \langle \mathrm{GS} | (-1)^r Z_i Z_{i+r} |\mathrm{GS}\rangle = (-1)^r\frac{1-4 ~M(\mathbb{I}, h_{i,i+r})}{3}, 
\end{equation}
obtained by $NPA_1(\mathcal{Proj})$ and Monte Carlo (virtually exact value) for system sizes $L=18, 28,$ and $60$. 
Note that the translation symmetry of the cycle ensures the well-definedness of $C(r)$ regarding the choice of $i$ in the definition. 
The second equality in \cref{eq:corr} is valid only in the case where the SDP algorithm obtains the exact ground state. However, we can still measure the RHS quantity even in cases where the algorithm fails and consider the obtained result as an approximation.
Remarkably, when utilizing the SDP-obtained correlation function in this manner, it aligns very well with the true correlation function for small values of $r$ as shown in the figure. This can be attributed to the fact that the energy density exhibits a relative error of only $\sim2\%$.

One characteristic feature of the Heisenberg chain is that the ground state displays a power-law decaying correlation with critical exponent $\eta =1$, i.e., $C(r)\propto r^{-1}$, which is closely linked to the long-range entanglement it has. 
The fact that the SDP-obtained correlation function displays the same type of power-law decay with essentially the correct exponent (albeit the jagged feature) is quite interesting especially when it is compared to the exact correlation function of finite systems, since it appears to have even smaller finite-size corrections. 
This raises the intriguing possibility that SDP-derived quantities capture the underlying ``physics" of the ground state, even when there is no physical quantum state corresponding to the optimal moment matrix. 

Finally, an alert reader may notice from the figure that both $NPA_1(\mathcal{Proj})$ and $NPA_2(\mathcal{Pauli})$ are exact for the $L=6$ hexagon case. 
Although we were unable to obtain an analytic $\SoS$, we were able to study the structure of the ground state from the SDP perspective, which we provide in Appendix \ref{app:hexagon}. 




\section{Acknowledgements}
J.T. thanks Hosho Katsura, Cristopher Moore, Sho Sugiura, and Seiji Takahashi for valuable discussions. C.Z. thanks Micha{\l} Adamaszek from MOSEK for helpful discussions regarding improving the efficiency of solving SDPs using MOSEK. 
K.T. and O.P. acknowledge discussions with Anirban Chowdhury.  J.T., C.R., and C.Z. thank Elizabeth Crosson for initially igniting this fruitful project. 
J.T. and C.Z. acknowledge support from the U.S. National Science Foundation under Grant No. 2116246, the U.S. Department of Energy, Office of Science, National Quantum Information Science Research Centers, and Quantum Systems Accelerator. 
O.P. and K.T. are supported by Sandia National Laboratories. Sandia National Laboratories is a multimission laboratory managed and operated by National Technology and Engineering Solutions of Sandia, LLC., a wholly owned subsidiary of Honeywell International, Inc., for the U.S. Department of Energy’s National Nuclear Security Administration under contract DE-NA-0003525. This work was supported by the U.S. Department of Energy, Office of Science, Office of Advanced Scientific Computing Research, Accelerated Research in Quantum Computing.




% only use with natbib
% \bibliography{bibvqe}
% \bibliographystyle{abbrvnat}

% with biblatex

\printbibliography

\begin{comment}
\section{System Architecture}
\label{appendix:architecture}
\system has a novel modularized system architecture with three key components: 
\emph{StreamManager}, 
\emph{TxnManager} and \emph{TxnScheduler}. 
These components are instantiated in each thread locally.
The execution outline of \system is presented in Algorithm~\ref{alg:algo}.
Transactional stream processing is continuous and potentially never ends (Line 1$\sim$8).
The dependency resolution and execution of state transactions are separated into two non-overlapping phases by punctuations~\cite{Tucker:2003:EPS:776752.776780} (Line 2 and 5), which guarantees that no subsequent input event will have a smaller timestamp. 
Effectively, a batch of state transactions is collected during the first phase, and processed during the second phase.

In the first phase (i.e., stream processing phase), 
the \emph{StreamManager} conducts preprocessing for every input event ($e$). Similar to some prior works~\cite{tstream}, state transactions may be issued but not immediately processed during preprocessing (Line 3).
The \emph{pre\_processing} and \emph{post\_processing} functions are exposed as APIs to users.
The \emph{TxnManager} handles dependency resolution (Line 4) among state transactions and insert decomposed operations to construct a \tpg. We discuss the detailed two-phase \tpg construction process in Section~\ref{subsec:construction}.

In the second phase  (i.e., transaction processing phase), 
the \emph{TxnManager} is first involved again to refine (Line 6) the constructed \tpg with further dependency resolution.
The \emph{TxnScheduler} 
schedules operations for concurrent execution based on the constructed \tpg according to the three dimensions of scheduling decisions (Line 7). 
In particular, a scheduling decision model $M$ is instantiated based on the constructed \tpg (Line 14).
\textbf{\circled{1}} Guided by $M$, execution threads adopt an exploration strategy (Section~\ref{subsec:explore}) to explore the constructed \tpg for operations available to be scheduled constrained by dependencies. 
\textbf{\circled{2}} 
During exploration, one or multiple operations may be treated as the 
% basic 
unit of scheduling (Section~\ref{subsec:granularity}). 
Subsequently, \textbf{\circled{3}} every thread executes operation(s) in the unit of scheduling with various abort handling mechanisms (Section~\ref{subsec:abort_handling}).
Only when state transactions are processed (i.e., committed or aborted) can the associated input events be postprocessed (Line 8) by the \emph{StreamManager} based on transaction processing results.
\end{comment}

\begin{comment}
\begin{algorithm}
\footnotesize
    \KwData{$e$ \tcp{Input event}}
    \KwData{$txn_{ts}$ \tcp{State transaction}}
    \KwData{$G$ \tcp{The currently constructed TPG}}
    \While{!finish processing of input streams}{
        \eIf(\tcp*[h]{Phase 1}){\text{$e$ is not a $punctuation$}}{
                $txn_{ts}$ $\gets$ PRE\_Processing($e$)\;
                \textbf{TPG\_Construction}($G$, $txn_{ts}$)\; 
          }(\tcp*[h]{Phase 2}){
                \textbf{TPG\_Refinement}($G$)\; 
                \textbf{TXN\_Scheduling}($G$)\; 
                POST\_Processing()\;
          }
    }
    
    \SetKwFunction{FMain}{TPG\_Construction}
    \SetKwProg{Fn}{Function}{:}{}
    \Fn{\FMain{$G$, $txn_{ts}$}}{
        $O_{1..k}$ $\gets$ \textbf{Partition} $txn_{ts}$\;
        \ForEach{\text{operation $O_{i}$ $\in$ $O_{1..k}$}}{
            \textbf{Identify} its \ld\;
            $G$ $\gets$ $G$ + $O_{i}$ \;
        }
    }
    \SetKwFunction{FMain}{TPG\_Refinement}
    \SetKwProg{Fn}{Function}{:}{}
    \Fn{\FMain{$G$}}{
        \ForEach{\text{vertex $e_{i}$ $\in$ $G$}}{
            \textbf{Identify} its \td, \pd\;
        }
    }
    
    \SetKwFunction{FMain}{TXN\_Scheduling}
    \SetKwProg{Fn}{Function}{:}{}
    \Fn{\FMain{$G$}}{
        $M$ $\gets$ Instantiated with $G$;\tcp{A decision model}
        \While{!finish scheduling of $G$
        }{
          \textbf{\circled{2}} $Scheduling Unit$ $\gets$ \textbf{\circled{1}} \emph{Explore}($G$, $M$)\; 
            \textbf{\circled{3}} \emph{Execute with Abort Handling} ($Scheduling Unit$)\; 
        }
    }
  \caption{Execution Outline of \system}
  \label{alg:algo}
\end{algorithm}
\end{comment}

\end{document}
