\documentstyle[12pt]{gandb}
\def\hal{{\textstyle \frac{1}{2}}}
\def\oqa{{\textstyle \frac{1}{4}}}
\def\eff{\rm eff}
\def\mbx{\rm max}
\def\ex{\rm ex}
\def\sc{\rm sc}
\def\similar{(\mbox{similar {\em y\/}-terms})}
\def\simi{(\mbox{sim.\ {\em y\/}-terms})}
\def\kv{\mbox{K-V} }
\def\<{\langle}
\def\>{\rangle}
\makeatletter
\input epsfig.sty
\makeatother
\begin{document}
%\count0=219
\setcounter{page}{219}
\evensidemargin 29pt
\fussy
\title{SELF-CONSISTENT DISTRIBUTIONS}
{THE PROBLEM OF SELF-CONSISTENT
\protect\\PARTICLE PHASE SPACE DISTRIBUTIONS
\protect\\FOR PERIODIC FOCUSING CHANNELS}
\author{J\"URGEN STRUCKMEIER and INGO HOFMANN}
\authorhead{J.~STRUCKMEIER AND I.~HOFMANN}
\address{\it Gesellschaft f\"ur Schwerionenforschung (GSI),
Postfach~11~05~52, 64220~Darmstadt,\protect\\Germany}
\received{(Received 15 July 1992; in final form 26 August 1992)}
\abstract{Charged particle beams that remain stationary
while passing through a transport channel are
represented by `self-consistent' phase space distributions.
As the starting point, we assume the external focusing forces
to act {\em continuously\/} on the beam.
If Liouville's theorem applies, an infinite variety of
self-consistent particle phase space distributions exists then.
The method is reviewed how to determine the Hamiltonian of the
focusing system for a given phase space density function.
Subsequently, this Hamiltonian is transformed canonically
to yield the appropriate Hamiltonian that pertains to a beam
passing through a {\em non-continuous\/} transport system.
It is shown that the total transverse beam energy is a
conserved quantity, if the beam stays rotationally symmetric along the channel.
It can be concluded that charged particle beams can be transmitted
through periodic solenoid channels without loss of quality.
Our computer simulations, presented in the second part
of the paper, confirm this result.
In contrast, the simulation for a periodic quadrupole channel yields a
small but constant growth rate of the rms-emittance.}
\keywords{Self-consistent phase space distributions, space-charge,
focusing channels\newline~\newline
\hspace*{-22mm}Published in: Part.~Acc.~\underline{39}, 219 (1992)}
\maketitle
\section{INTRODUCTION}
The transport of intense charged particle beams is of
considerable interest for many applications.
That is why intensive theoretical and numerical studies
as well as experiments have been carried out on this subject.
Nevertheless, a fundamental question has not yet been answered:
do `physically realistic' self-consistent phase space
distributions exist for {\em periodic\/} beam transport systems?
In other words, is it possible to transport intense
charged particle beams -- at least in principle -- without any loss
of beam quality through periodic focusing channels?
The answer is especially important for applications where
the conservation of the beam quality is crucial.
This is the case for example in heavy ion fusion, where
beam losses cannot be tolerated.
We emphasize that ideal beam transport conditions are assumed, i.e.\
no field errors, fringe fields,
misalignment effects etc.\ are taken into account.
Furthermore, we require that no particle collisions occur and that
the paraxial ray approximation applies, i.e.\ the longitudinal
particle velocity is much larger than the transverse ones.
No assumptions have been made with regard to the particle phase
space distribution and its time evolution.
It is, for instance, not required that the beam profile remains
self-similar during variations of its cross section, as it was
assumed in Ref.~\cite{leecoo}.

For charged particle beams in transport devices with continuous external
focusing, we can easily set up the condition for a beam to be stationary.
Seen from the co-moving beam frame, the external forces that act on
the particles are constant, hence the total system is conservative.
Under these circumstances a particle distribution $f$ is stationary
if the phase space surfaces of constant energy $w$
are also surfaces of constant density\cite{lapo,gluck}, i.e.\ if $f=f(w)$.

For non-continuous focusing channels, the only particle phase space
distribution known to be stationary for unbunched beams is the so-called
`Kapchinskij-Vladimirskij' \mbox{(K-V)} distribution\cite{kapvla}.
It is obtained if all particles are assumed to have exactly the same
transverse total energy, and if all states of that energy
have the same probability.
It has the advantage that the resulting equations
of motion turn out to be linear.
On the other hand this type of distribution is not `physically realistic',
since `real' beams always have some finite transverse energy spread.

Moreover, computer simulation of this distribution exhibits
instabilities\cite{hola,hofm} which are absent for more realistic
beam models in simulation as well as in the experiment.
Hence, it can be concluded that high current beams
cannot be described adequately by the \kv model
and other models need to be investigated.

After a short recall of the basic equations in section \ref{recall}, the
method of constructing stationary phase space distributions for
continuous beam transport systems is reviewed in section \ref{stationary}.
For unbunched beams a closed differential equation is
derived that allows us to directly determine the effective
potential, hence the system's Hamiltonian for any given phase space
density function $f = f(w)$.
We solve this differential equation for
the two simplest possible examples in section \ref{examp}.

In section \ref{catra}, a canonical transformation is suggested,
which performs in a formal way a transformation from a {\em continuous\/}
focusing system into a {\em periodic\/} system.
For the linear case, i.e.\ for a \kv particle phase space
distribution, this canonical transformation is just the inverse
of the transformation used earlier by Courant and Synder\cite{cousny}
to transform from a periodic channel into a continuous.
It is also equivalent to the transformation theory developed
by Lewis\cite{lewis} and Leach\cite{leach} for the time-dependent
harmonic oscillator.

Applied to non-K-V distributions we show that this transformation is
physically correct as infinitesimal canonical transformation.
We also show that it implies conservation of the total (transverse)
beam energy, whereas the individual particle energies as well as the
rms-emittances are no longer constants of motion.

Finally in section \ref{final}, we re-derive
the formula that allows to evaluate the upper limit
of emittance growth due to the thermalization of `excess'
field energy for a non-stationary initial distribution\cite{strekla}.
All results derived in this article are confirmed by
computer simulations shown in section \ref{simu}.
\section{RECAPITULATION OF THE BASIC EQUATIONS} \label{recall}
Before outlining the transformation theory, we recapitulate
some definitions and results needed in the subsequent sections.
We consider a charged particle beam that passes through
a focusing system.
In general, regarded from the beam system,
the external forces are time-dependent.
As a consequence, the whole system is not conservative.
The total Hamiltonian $H$
of the system of $N$ particles of the charge $q$ interacting
through their Coulomb forces is given by
$$
H = m_0 c^2 \beta^2 \gamma \sum_{i=1}^{N}
\left( x_i'^2 + y_i'^2 + z_i'^2 \right) ~-~ L \; ,
$$
$L$ denoting the Lagrangian function
$$
L = T - \left( q V_{\ex} + \frac{1}{\gamma^2} W \right) \; ,
$$
with $T$ the kinetic energy of all particles, $q V_{\ex}$ the
potential energy resulting from the external focusing devices,
and $W$ the Coulomb interaction energy.
If the external focusing forces are linear,
the complete Hamiltonian is given by:
\begin{eqnarray} \label{ham0}
H & = & \hal m_0 c^2 \beta^2 \gamma \sum_{i=1}^{N}
\left( x_i'^2 + y_i'^2 + z_i'^2 \right) \nonumber \\
& + & \hal m_0 c^2 \beta^2 \gamma \sum_{i=1}^{N}
\left( k_x^2(s) x_i^2 + k_y^2(s) y_i^2 + k_z^2(s) z_i^2 \right) +
\frac{q}{2\gamma^2} \sum_{i=1}^{N} V_{\sc}(x_i,y_i,z_i)\;~
\end{eqnarray}
with $V_{\sc}$ the space charge potential at the position of
the $i$-th particle
\begin{equation} \label{scpot}
V_{\sc}(x_i,y_i,z_i) = \sum_{j \neq i}
\frac{q}{\sqrt{(x_i-x_j)^2 + (y_i-y_j)^2 + (z_i-z_j)^2}} \; ,
\end{equation}
and $k_x^2(s)$, $k_y^2(s)$, $k_z^2(s)$ the focusing force functions in
$x$, $y$, $z$.

For convenience,
the time has been replaced by the distance $s = c \beta \cdot t$
along the focusing structure
$$
\dot{x}_i = c \beta \cdot x_i' \quad , \quad
\dot{p}_{x,i} = m_0 c^2 \beta^2 \gamma \cdot x_i'' \; ,
$$
and similarly in $y$ and $z$.

The equations of motion for particle $i$ result
from (\ref{ham0}) as:
\begin{eqnarray} \label{bewe}
x_i'' + k_x^2(s) \, x_i - \frac{q}{m_0 c^2 \beta^2 \gamma^3}
E_x(x_i,y_i,z_i) & = & 0 \nonumber \\
y_i'' + k_y^2(s) \, y_i - \frac{q}{m_0 c^2 \beta^2 \gamma^3}
E_y(x_i,y_i,z_i) & = & 0           \\
z_i'' + k_z^2(s) \, z_i - \frac{q}{m_0 c^2 \beta^2 \gamma^3}
E_z(x_i,y_i,z_i) & = & 0 \; , \nonumber
\end{eqnarray}
with the electric field from space charge, given by
\begin{equation} \label{e_x}
E_x(x_i,y_i,z_i) = q \, \sum_{j \neq i} \frac{x_i - x_j}
{\bigl[ (x_i - x_j)^2 + (y_i - y_j)^2 + (z_i - z_j)^2 \bigr]^{3/2}} \; .
\end{equation}
With $N$ the total number of particles, the beam moments
can be defined as
\begin{eqnarray*}
\<x^2\>   & = & N^{-1} \cdot \sum_{i=1}^{N} x_i^2 \\
\<xx'\>   & = & N^{-1} \cdot \sum_{i=1}^{N} x_i x_i' \\
\<x'^2\>  & = & N^{-1} \cdot \sum_{i=1}^{N} x_i'^2 \\
\<xE_x\>  & = & N^{-1} \cdot \sum_{i=1}^{N} x_i E_x(x_i,y_i,z_i) \\
\<x'E_x\> & = & N^{-1} \cdot \sum_{i=1}^{N} x_i' E_x(x_i,y_i,z_i) \; .
\end{eqnarray*}
Using the equations of motion (\ref{bewe}), we
obtain for their derivatives:
\begin{eqnarray} \label{deri}
\frac{d}{ds} \<x^2\> - \, 2\<xx'\> & = & 0 \nonumber \\
\frac{d}{ds} \<xx'\> - \<x'^2\> + \, k_x^2(s) \<x^2\> -
\frac{q}{m_0 c^2 \beta^2 \gamma^3} \<x E_x\> & = & 0 \\
\frac{d}{ds} \<x'^2\> + \, 2 k_x^2(s) \<xx'\> -
\frac{2q}{m_0 c^2 \beta^2 \gamma^3} \<x' E_x\> & = & 0 \nonumber \; .
\end{eqnarray}
With the following definition of the rms-emittance $\varepsilon_x(s)$
\begin{equation} \label{epsrms}
\varepsilon_x^2(s) ~=~ \<x^2\> \<x'^2\> - \<x x'\>^2 \; ,
\end{equation}
and (\ref{deri}) we can set up a differential equation
for the moment $\sqrt{\<x^2\>}$, which is proportional to the actual
beam width in $x$.
It is known as the `rms envelope equation' that has been
first derived by Sacherer\cite{sach}:
\begin{equation} \label{kv1}
\frac{d^2}{ds^2} \sqrt{\<x^2\>} + k_x^2(s) \, \sqrt{\<x^2\>} -
\frac{q}{m_0 c^2 \beta^2 \gamma^3} \frac{\<xE_x\>}{\sqrt{\<x^2\>}} -
\frac{\varepsilon_x^2(s)}{\sqrt{\<x^2\>^3}} = 0 \; .
\end{equation}
A beam is called `matched' if the initial conditions of (\ref{kv1})
have been adjusted so that the rms-beam widths
have the same periodicities as the focusing functions $k_{x,y,z}^2(s)$.
Due to the fact that $\<xE_x\>$ and $\varepsilon_x^2$ are in general
functions of $s$, this equation is not closed.

The change of the rms-emittance is also readily
derived from (\ref{deri})
\begin{equation} \label{epsderi}
\frac{d}{ds} \varepsilon_x^2(s) = \frac{2q}{m_0 c^2 \beta^2 \gamma^3}
\; \bigl( \<x^2\> \<x'E_x\> - \<xx'\> \<xE_x\>\bigr) \; .
\end{equation}
The total interaction energy $W$ is half the sum over the
space charge potential contributions (\ref{scpot}) of all particles
\begin{eqnarray} \label{wgen}
W & = & \frac{q}{2} \, \sum_i V_{\sc}(x_i,y_i,z_i) \nonumber \\
& = & \frac{q^2}{2} \sum_i \sum_{j \neq i}
\frac{1}{\sqrt{(x_i-x_j)^2 + (y_i-y_j)^2 + (z_i-z_j)^2}} \; .
\end{eqnarray}
The beam moments $\<x' E_x\>$, $\<y' E_y\>$ and $\<z' E_z\>$ can be
correlated to the change of the interaction energy $W$, since
we are allowed to interchange the order of summation:
\begin{eqnarray} \label{dw}
\frac{dW}{ds} & = & -\frac{q^2}{2} \, \sum_i \sum_{j \neq i}
\frac{(x_i - x_j)(x_i' - x_j') + (y_i - y_j)
(y_i' - y_j') + (z_i - z_j)(z_i' - z_j')}
{\bigl[ (x_i - x_j)^2 + (y_i - y_j)^2 +
(z_i - z_j)^2 \bigr]^{3/2}} \nonumber \\
& = & -q \, \sum_i \bigl[
x'_i E_x(x_i,y_i,z_i) +
y'_i E_y(x_i,y_i,z_i) +
z'_i E_z(x_i,y_i,z_i) \bigr] \nonumber \\
& = & -N q \, \bigl( \<x' E_x\> + \<y' E_y\> + \<z' E_z\> \bigr) \; .
\end{eqnarray}
If the charge distribution is {\em uniform\/}, the space charge field components
$E_x$, $E_y$ and $E_z$ are {\em linear\/} functions of $x$, $y$ and $z$,
respectively.
We thus conclude from (\ref{dw}), that the derivative of
the field energy $W_u$ of a uniform charge distribution
can be expressed as
\begin{eqnarray} \label{dwu}
\frac{d\/W_u}{ds} & = & -q \, \sum_i \left[
x_i x'_i \, \frac{\<x E_x\>}{\<x^2\>} +
y_i y'_i \, \frac{\<y E_y\>}{\<y^2\>} +
z_i z'_i \, \frac{\<z E_z\>}{\<z^2\>} \right] \nonumber \\ & = & -N q \left(
\<xx'\> \frac{\<x E_x\>}{\<x^2\>} \, +
\<yy'\> \frac{\<y E_y\>}{\<y^2\>} \, +
\<zz'\> \frac{\<z E_z\>}{\<z^2\>} \right) \, . \nonumber \\
& &
\end{eqnarray}
With (\ref{dw}) and (\ref{dwu}), we can rewrite (\ref{epsderi})
to obtain a differential equation that correlates
the change of the rms-emittances with the change of the
excess space charge field energy $W-W_u$ of a charged particle beam.
It has been derived first by Wangler\cite{wang} for
unbunched round beams, and generalized subsequently in Ref.~\cite{host}:
\begin{equation} \label{sr2}
\frac{1}{\<x^2\>} \frac{d \varepsilon_x^2(s)}{ds} +
\frac{1}{\<y^2\>} \frac{d \varepsilon_y^2(s)}{ds} +
\frac{1}{\<z^2\>} \frac{d \varepsilon_z^2(s)}{ds} =
\frac{-2}{m_0 c^2 \beta^2 \gamma^3 N}
\frac{d}{ds} \bigl[ W(s) - W_u(s) \bigr] \, .
\end{equation}
Corresponding to (\ref{wgen}), the total
interaction energy $W_u$ of a uniform charge distribution writes
$$
W_u = \frac{q}{2} \, \sum_i V_{\sc}^u (x_i,y_i,z_i) \; ,
$$
with $qV_{\sc}^u (x_i,y_i,z_i)$ denoting the contribution of the $i\/$-th
particle to the interaction energy of a uniform charge distribution.
If $V_{\sc}^u$ is normalized to be zero at the beam center,
it takes the following form
\begin{equation} \label{vu1}
V_{\sc}^u(x_i,y_i,z_i) ~=~ \mbox{} -
\frac{\<xE_x\>}{\<x^2\>} \, x_i^2 -
\frac{\<yE_y\>}{\<y^2\>} \, y_i^2 -
\frac{\<zE_z\>}{\<z^2\>} \, z_i^2 \; ,
\end{equation}
which is easily shown to be compatible with equation (\ref{dwu}).
\section{CONSTRUCTION OF STATIONARY PHASE SPACE DISTRIBUTIONS
IN CONTINUOUS FOCUSING CHANNELS} \label{stationary}
\subsection{Bunched Beams}
The 6-dimensional phase space distribution function $f$
$$
f(x,y,z,x',y',z',s) = \sum_{i=1}^{N} \delta(x - x_i) \cdot \:
\ldots \: \cdot \delta(z' - z'_i)
$$
satisfies Liouville's theorem
\begin{equation} \label{liou}
\frac{d\/f}{ds} = 0 \; ,
\end{equation}
if there are no particle-particle interactions.
Equation (\ref{liou}) holds too if the space charge forces
on a particle can be regarded as
derived from an equivalent external potential.
This is the case if the space charge potential is a smooth function
of the spatial coordinates, i.e.\ if there are no
particle collisions.

In order to set up the condition for a charged particle beam to be
stationary, we define $w$ as the energy of
a particle moving inside the effective potential bucket
$V_{\eff}$ formed by the sum of the focusing potential and the
collective space charge potential
\begin{equation} \label{energy}
w(x,x',y,y',z,z') = \hal m_0 c^2 \beta^2 \gamma
\left( x'^2 + y'^2 + z'^2 \right) + q V_{\eff}(x,y,z) \; ,
\end{equation}
wherein
$$
qV_{\eff}(x,y,z) = \hal m_0 c^2 \beta^2 \gamma
\left( k_x^2 x^2 + k_y^2 y^2 + k_z^2 z^2 \right) +
\frac{q}{\gamma^2} V_{\sc}(x,y,z) \; .
$$
With the phase space energy function $w$,
Liouville's theorem (\ref{liou}) can be rewritten as
$$
\frac{d\/f}{ds} \equiv \frac{\partial f}{\partial s} + \{w,f\} = 0 \; .
$$
If the external forces seen from the beam are constant, the particle
phase space distribution can be considered to be stationary if its
partial $s$-derivative vanishes
$$
\frac{\partial f}{\partial s} = 0 \; .
$$
Consequently, a vanishing Poisson bracket $\{w,f\}$ is a
sufficient condition for $f$ to be stationary.
This condition is always fulfilled if $f$ is a function of $w$
only, since a Poisson bracket $\{w,f(w)\}$ vanishes identically
for arbitrary differentiable functions $f$ and $w$
$$
f = f(w) \; \Rightarrow \; f \; \mbox{is stationary} \; .
$$
In other words, a particle phase space distribution is stationary if
its phase space surfaces of constant energy
(`Isohamiltonians') are also surfaces of constant density $f$.

In this context, the problem of `self-consistency' arises, since
due to the space charge potential $V_{\sc}$, the surfaces of constant
energy are, in turn, a function of the phase space distribution
$$
w = w(f) \; .
$$
With $q n(x,y,z)$ denoting
the macroscopic charge density function,
the space charge potential $V_{\sc}$ follows from Poisson's equation
\begin{equation} \label{poiss}
\Delta V_{\sc} = -4 \pi \, q \cdot n(x,y,z) \; .
\end{equation}
The particle density $n(x,y,z)$ in real space is obtained by
integrating the phase space distribution function over all velocities.
The self-consistent boundary of this integration is given by the
Isohamiltonian defined by the particle of the maximum allowable
energy $w_{\mbx}$
\begin{equation} \label{pro2}
n(x,y,z) = \int\limits_{-\zeta'}^{\zeta'}
\int\limits_{-\eta'}^{\eta'} \int\limits_{-\xi'}^{\xi'} \,
f(x,y,z,x',y',z') \; dx' \, dy' \, dz'
\end{equation}
with
\begin{eqnarray*}
\xi' ~\equiv  & x'_{\mbx}(x,y,z,y',z') & =~~\sqrt{\frac{2}{m_0 c^2\beta^2\gamma}
\bigl[ w_{\mbx} - q V_{\eff}(x,y,z) \bigr] - y'^2 - z'^2} \\
\eta' ~\equiv & y'_{\mbx}(x,y,z,z')~~~ & =~~\sqrt{\frac{2}{m_0 c^2\beta^2\gamma}
\bigl[ w_{\mbx} - q V_{\eff}(x,y,z) \bigr] - z'^2} \\
\zeta' ~\equiv &z'_{\mbx}(x,y,z)~~~~~~ & =~~\sqrt{\frac{2}{m_0 c^2\beta^2\gamma}
\bigl[ w_{\mbx} - q V_{\eff}(x,y,z) \bigr] } \; .
\end{eqnarray*}
Equation (\ref{pro2}) forms together with
Poisson's equation (\ref{poiss}) a closed set allowing the
determination of the self-consistent space charge potential
$V_{\sc}(x,y,z)$, since the type of distribution function
$f = f(w)$, the maximum energy $w_{\mbx}$
and the focusing constants $k_x$, $k_y$, $k_z$ are given.
\subsection{Unbunched Beams}
The procedure to construct a self-consistent phase space
distribution can be expressed in a simplified form if we consider a
4-dimensional transverse phase space distribution $f$ that describes
an unbunched beam in a rotationally symmetric focusing channel.
Using the abbreviations $r^2 = x^2 + y^2$ and $\rho^2 = x'^2 + y'^2$,
the energy function (\ref{energy}) takes on the form
\begin{eqnarray*}
w(r,\rho)
 & = & \hal m_0 c^2\beta^2\gamma\cdot\rho^2 + q V_{\eff}(r) \\
 & = & \hal m_0 c^2\beta^2\gamma\cdot \left( \rho^2 + k_0^2 \, r^2\right) +
\frac{q}{\gamma^2} V_{\sc}(r) \; .
\end{eqnarray*}
If $I$ denotes the longitudinal beam current,
the projection (\ref{pro2}) can then be written as
\begin{equation} \label{pro3}
n(r) = \pi \frac{I}{q c \beta}
\int\limits_{0}^{\rho_{\mbx}^2(r)} f(r,\rho) \; d\/\left(\rho^2\right) \; ,
\end{equation}
with the self-consistent upper limit of integration
\begin{eqnarray*}
\rho^2_{\mbx}(r) & = & \frac{2}{m_0 c^2 \beta^2 \gamma}
\bigl[ w_{\mbx} - q V_{\eff}(r) \bigr] \\
& = & \frac{2q}{m_0 c^2 \beta^2 \gamma}
\bigl[ V_{\eff}(a) - V_{\eff}(r) \bigr] \; .
\end{eqnarray*}
Since stationary phase space distributions $f$ are functions of the energy
$w$ only, we substitute $w$ as the new integration variable
\begin{eqnarray*}
n(r) & = & \frac{2 \pi I}{q m_0 c^3 \beta^3 \gamma}
\int\limits_{q V_{\eff}(r)}^{q V_{\eff}(a)} \, f(w) \; dw \\
& = & \frac{2 \pi I}{q m_0 c^3 \beta^3 \gamma}
\Bigl[ F\bigl( qV_{\eff}(a)\bigr) - F\bigl( qV_{\eff}(r)\bigr) \Bigr] \; ,
\end{eqnarray*}
if $F(w)$ denotes an integral function of
the given phase space distribution function $f(w)$.
With this expression for the charge density, we can rewrite Poisson's
equation (\ref{poiss}) in cylindrical coordinates
to obtain an inhomogeneous differential equation for the
self-consistent effective potential $V_{\eff}(r)$:
\begin{equation} \label{intdif}
\frac{1}{r} \frac{d}{dr} \left( r \frac{d}{dr} qV_{\eff}(r) \right) -
4 \pi^2 K \cdot F\Bigl( qV_{\eff}(r)\Bigr)
= 2 m_0 c^2 \beta^2 \gamma \cdot k_0^2 -
4 \pi^2 K \cdot F\Bigl( qV_{\eff}(a)\Bigr)\; .
\end{equation}
$K$ abbreviates the dimensionless `generalized perveance', defined as
\begin{equation} \label{perve}
K = \frac{2qI}{m_0 c^3 \beta^3 \gamma^3} \; .
\end{equation}
Starting with equation (\ref{intdif}), two examples how to construct
a self-consistent phase space distribution will be outlined
in the next section.
\section{EXAMPLES FOR SELF-CONSISTENT PARTICLE PHASE SPACE
DISTRIBUTIONS IN CONTINUOUS FOCUSING CHANNELS} \label{examp}
\subsection{`K-V' Distribution} \label{k-v}
For unbunched beams, the transverse phase space distribution that
defines all particles to have the same transverse total energy $w_{\mbx}$,
is commonly called \kv distribution.
Mathematically it is expressed using Dirac's $\delta$-distribution:
\begin{equation} \label{kvdef}
f(w)=\frac{m_0c^2\beta^2\gamma}{2\pi^2 a^2}\cdot\delta\bigl(w-w_{\mbx}\bigr)\; .
\end{equation}
The region of integration includes $w_{\mbx} = qV_{\eff}$,
thus (\ref{intdif}) becomes:
\begin{equation} \label{kvgl1}
\frac{1}{r} \frac{d}{dr} \left( r \frac{d}{dr} qV_{\eff}(r) \right) =
2 m_0 c^2 \beta^2 \gamma \cdot \left( k_0^2 - \frac{K}{a^2} \right) \; .
\end{equation}
The expression in parentheses
\begin{equation} \label{tune}
k^2 = k_0^2 - \frac{K}{a^2}
\end{equation}
shows that the zero current wave number $k_0$ is reduced
due to the space charge forces acting between the particles,
yielding the `depressed tune' of $k$.
Nevertheless the right hand side is still a constant, so (\ref{kvgl1})
can easily be integrated:
$$
qV_{\eff}(r) = \hal m_0 c^2 \beta^2 \gamma \cdot k^2r^2 \; ,
$$
wherein the space charge potential $V_{\sc}$ has been normalized to vanish
on the beam axis:
$$
V_{\sc}(r) = -\frac{I}{c\beta} \cdot \frac{r^2}{a^2} \;\; .
$$
We observe that the self-consistent space charge potential is strictly
quadratic.
This leads to linear space charge forces and a homogeneous
space charge density \mbox{$qn_0 = I / c \beta\pi a^2$}.
Together with the linear focusing forces, the
equations of motion are completely linear.
The phase space energy function $w(r,\rho)$ for this case is given by:
$$
w(r,\rho) = \hal m_0 c^2 \beta^2 \gamma \cdot \bigl( \rho^2 + k^2 r^2 \bigr)\;.
$$
The lines of constant energy (\ref{energy}) are thus
hyperellipsoids in phase space.
This reflects the fact that the net forces acting on all particles are linear.

The assumption that the particle beam coordinates are distributed
according to (\ref{kvdef}) is convenient because of the resulting
linear equations of motion.
On the other hand it has the disadvantage of being oversimplified from the
physical point of view.
The distribution of the transverse particle energies of a `real' beam
should be somewhat `smeared' and not as extremely peaked as
suggested by (\ref{kvdef}).
This requirement is met by the `water bag' distribution
that will be discussed in the next section.
\subsection{`Water Bag' Distribution} \label{water}
A particle distribution ansatz $f(w)$ that comes closer to reality
and also allows an analytical treatment of (\ref{intdif}) is the so-called
`water bag' distribution\cite{lapo}.
For this case we assume the transverse particle energies to be
distributed uniformly between zero and a maximum energy $w_{\mbx}$.
Thus each transverse particle energy in that range has the same probability,
whereas no particles occur beyond $w_{\mbx}$.
Mathematically this behavior can be expressed using the unit step function
$\Theta(x)$
\begin{equation} \label{wbdef}
f(w) = c_n \cdot \Theta \bigl( w_{\mbx} - w \bigr)
= \cases{c_n &if $w \leq w_{\mbx}$ \cr ~ \cr 0 &if $w > w_{\mbx}$ \cr} \;\; ,
\end{equation}
with $c_n$ as normalization constant that will be determined later.
The integral function of (\ref{wbdef}) evaluates to $F(w)= c_n w$ for
$w \leq w_{\mbx}$, thus equation (\ref{intdif}) becomes
\begin{equation} \label{wbgl1}
\frac{1}{r} \frac{d}{dr} \left( r \frac{d}{dr} qV_{\eff}(r) \right) =
2 m_0 c^2 \beta^2 \gamma \cdot k_0^2 +
4 \pi^2 K c_n q \cdot \Bigl[ V_{\eff}(r) - V_{\eff}(a) \Bigr] \;\; .
\end{equation}
In order to include the constants into a single expression,
we define temporarily the function $U(r)$
\begin{equation} \label{bessel}
U(r) = V_{\eff}(r) - V_{\eff}(a) + \frac{2 m_0 c^2 \beta^2 \gamma \cdot k_0^2}
{q \cdot \kappa^2} \quad {\rm with} \quad \kappa^2 = 4 \pi^2 K c_n \;\; .
\end{equation}
With the help of $U(r)$, (\ref{wbgl1}) is converted into
a differential equation of the Bessel type
$$
\frac{1}{r} \frac{d}{dr} \left( r \frac{d}{dr} U(r) \right) -
\kappa^2 \cdot U(r) = 0 \;\; ,
$$
whose only physical solution for this case is the zeroth modified
Bessel function $I_0$
$$
U(r) = {\rm const.} \cdot I_0(\kappa r) \;\; .
$$
With this solution, we resolve (\ref{bessel}) for $V_{\eff}(r)$:
\begin{equation} \label{wbskpot}
V_{\eff}(r) = V_{\eff}(a) \cdot \frac{I_0(\kappa r) - 1}{I_0(\kappa a) - 1} \;,
\end{equation}
wherein the space charge potential has been again adjusted to be zero
on the beam axis.
The shapes of the normalized effective potentials for different
$\kappa a$ values are plotted in Fig.~\ref{effpotential}.
The space charge parameter $\kappa$, hence
the normalization constant $c_n$ in (\ref{wbdef}) can now be determined
via normalization of the charge density function
$ \int_0^a \, n(r) \, r \, dr = I / (2\pi q c \beta) $.
We thus obtain
$$
n(r) = n_0 \cdot \left( 1 - \frac{I_0(\kappa r)}{I_0(\kappa a)} \right)
$$
with the charge density constant $qn_0$:
$$
q n_0 = \hal m_0 c^2 \beta^2 \gamma^3 \cdot \frac{k_0^2}{\pi q} \; .
$$
A uniform line charge density $qn_0$ produces a quadratic space charge
potential that is equal to the negative external focusing potential, thus
exactly cancels the latter.
The self-consistent maximum charge density, occurring on the beam axis,
is smaller by the factor $1 - 1/I_0(\kappa a)$ than the charge density $qn_0$
that fully screens out the external focusing potential at that point.
Due to the sharp rise of $I_0(\kappa a)$, this factor is close to unity
even for small beam currents.

The effect that charged particles arrange themselves to screen out the external
potential is commonly designated in plasma physics as `Debye shielding'.
We observe that -- in contrast to the \kv distribution --
the `water bag' distribution exhibits this property.
Therefore, it is much better suited to simulate the behavior of real beams.

The shapes of $n(r)/n_0$ for various values of $\kappa a$ are shown
in Fig.~\ref{chargedensity}.
% Figure environment removed
Unfortunately, the dimensionless space charge parameter $\kappa a$
cannot be determined directly, but only via numerical solution of
the implicit equation
$$
\frac{I_2(\kappa a)}{I_0(\kappa a)} = \frac{K}{(k_0a)^2} \;.
$$
The surfaces of constant energy for the `water bag' distribution are
$$
w(r,\rho) = \hal m_0 c^2\beta^2\gamma\cdot\rho^2 + q V_{\eff}(a)
\cdot \frac{I_0(\kappa r) - 1}{I_0(\kappa a) - 1} \;.
$$
As self consistent phase space boundary follows
\begin{equation} \label{border}
\frac{\rho_{\mbx}^2(r)}{\rho_{\mbx}^2(0)} +
\frac{I_0(\kappa r) - 1}{I_0(\kappa a) - 1} = 1 \; .
\end{equation}
In the zero current limit, the effective potential turns out to be quadratic
$$
\lim_{\kappa \rightarrow 0} \, \frac{I_0(\kappa r) - 1}{I_0(\kappa a) - 1} =
\frac{r^2}{a^2} \; ,
$$
as expected for linear external focusing forces.
Thus Eq.~(\ref{border}) leads again to an ellipsoidal phase space symmetry.
For the infinite charge density limit the potential term of (\ref{border})
exhibits a `reflecting wall' shape
$$
\lim_{\kappa \rightarrow \infty}\, \frac{I_0(\kappa r) - 1}{I_0(\kappa a) - 1} =
 \cases{0 &if $r < a$ \cr ~ \cr 1 &if $r = a$ \cr} \;\; .
$$
This means that the particles are moving force-free in the interior of the
beam due to the vanishing effective potential for $r<a$.
They are reflected at the beam boundaries, yielding a
rectangular phase space symmetry, as shown in Fig.~\ref{phasespace}.
% Figure environment removed
% Figure environment removed
\section{CANONICAL TRANSFORMATION OF A CONTINUOUS CHANNEL INTO
A PERIODIC CHANNEL FOR UNBUNCHED BEAMS} \label{catra}
\subsection{Preliminaries}
As we have seen in the previous section, any given type of
phase space distribution function $f = f(w)$ whose
Isohamiltonians are also surfaces of constant particle density
is stationary, provided that the beam experiences constant
external forces.

Realistic devices for charged particle optics consist of discrete lenses.
In order to investigate whether stationary particle phase space
distributions exist in realistic beam transport channels, we shall
correlate the particle motion in a continuous focusing
channel to the particle motion in a system,
where the focusing forces are $s$-dependent.
For that purpose, we map the
Hamiltonian that pertains to a continuous focusing channel canonically
to the corresponding Hamiltonian of a $s$-dependent focusing channel.
For the sake of transparency, this mapping will be carried out in two steps.
The quantities obtained after the first step will be marked by a tilde.
In the same way, over-lined quantities denote those
resulting from the second step.
It is understood that all quantities denoted with the index $x$
are defined likewise for the $y$-coordinate.

For unbunched beams, we have no $z$-dependency for the charge density.
If in addition no longitudinal focusing forces are acting, the
Hamiltonian (\ref{ham0}) reduces to its transverse part $H_{\perp}$.
We can thus integrate (\ref{scpot}) over $z$ to obtain the
corresponding expression for the space charge potential
$V_{\sc}(x_i,y_i)$ of a line charge distribution:
\begin{equation} \label{scpot_l}
V_{\sc}(x_i,y_i) = -\frac{I}{c\beta}\sum_{j \neq i}
\ln{\frac{(x_i-x_j)^2 + (y_i-y_j)^2}
{{\left(r_{max}-\sqrt{x_j^2+y_j^2}\,\right)}^{2}}}\;,
\end{equation}
normalized to zero on the beam pipe with radius $r_{max}$.
The electric space charge field $E_x(x_i,y_i)$ of a line
charge distribution is evaluated as:
\begin{equation} \label{e_x_l}
E_x(x_i,y_i) = \frac{2I}{c\beta} \cdot \sum_{j \neq i} \frac{x_i - x_j}
{(x_i - x_j)^2 + (y_i - y_j)^2 } \; .
\end{equation}
Before performing the first canonical transformation, we
write the Hamiltonian (\ref{ham0}) in an alternate form
using the moment equations (\ref{deri}).
For a stationary beam passing through a continuous focusing channel,
$k_x^2$ and $\<x^2\>$ are constant, thus $\<xx'\> \equiv 0$.
Hence the second equation of (\ref{deri}) reduces to
\begin{equation} \label{kv2}
k_x^2(s) \equiv k_x^2 ~=~ \kappa_x^2 +
\frac{q}{m_0 c^2 \beta^2 \gamma^3} \frac{\<xE_x\>}{\<x^2\>} \; ,
\end{equation}
with $\kappa_x^2 = \<x'^2\> / \<x^2\>$ .
It is thus possible to substitute the focusing functions
$k_x$ and $k_y$ contained in the Hamiltonian (\ref{ham0})
according to (\ref{kv2}).
The beam moments $\<xE_x\>$ and $\<yE_y\>$ can be replaced using expression
(\ref{vu1}) for the quadratic potential $V_{\sc}^u$ we would deal with
if the actual charge distribution would be uniform:
\begin{equation} \label{ham0p}
H_{\perp} = \hal m_0 c^2 \beta^2 \gamma \sum_i
\left( x_i'^2 + \kappa_x^2 x_i^2 \, + \,
       y_i'^2 + \kappa_y^2 y_i^2 \right) \, + \,
\frac{q}{2\gamma^2}\sum_i \bigl[ V_{\sc}(x_i,y_i)-V_{\sc}^u (x_i,y_i)\bigr] \, .
\end{equation}
In other words, the last term describes the difference of the actual
space charge potential contributions $V_{\sc}$ of all particles
to the corresponding quadratic space charge potential contributions
$V_{\sc}^u$ of a uniform charge distribution that possesses the same moments.
Therefore, this difference vanishes, if the actual space charge potential
$V_{\sc}$ itself is a quadratic function of the particle positions.
This is true for a \kv phase space distribution, as shown in section \ref{k-v}.
For all other distribution functions, $V_{\sc} - V_{\sc}^u$ is a function of
all particle coordinates and must be transformed canonically as well.
This case will be treated in section \ref{non-kv}.
\subsection{The first Canonical Transformation for a \kv distribution}
\label{first_kv}
For a \kv distributed beam, we define the first canonical transformation
as a $s$-dependent time displacement $\Delta t$
of the beam within the continuous focusing system.
This means that all particles are displaced
a distance $\ell_x(s) = c \beta \cdot \Delta t$
along the continuous focusing structure given by
\begin{equation} \label{dx}
\ell_x(s) = \int\limits_{s_0}^{s} \left(
\frac{\<x^2\>}{\<\bar{x}^2\>(z)} - 1 \right) \, dz \; .
\end{equation}
As can be seen directly from the definition of the
`phase advances' $\sigma_x$ and $\bar{\sigma}_x$
of both types of focusing systems
$$
\sigma_x(s) = \frac{\varepsilon_x}{\<x^2\>} \cdot \bigl( s - s_0 \bigr)
\quad , \quad \bar{\sigma}_x(s) = \bar{\varepsilon}_x \cdot
\int\limits_{s_0}^{s} \frac{dz}{\<\bar{x}^2\>(z)} \; ,
$$
the phase advance of the $s$-dependent system at
the position $s$ along the beam axis agrees
with the phase advance of the continuous system at $s+\ell_x$:
\begin{eqnarray*}
\sigma_x(s+\ell_x) & = & \frac{\varepsilon_x}{\<x^2\>} \cdot
\bigl[ s-s_0+\ell_x\bigr] \\
& = & \frac{\varepsilon_x}{\<x^2\>} \int\limits_{s_0}^{s}
\frac{\<x^2\>}{\<\bar{x}^2\>(z)} \, dz \; = \; \bar{\sigma}_x(s) \; ,
\end{eqnarray*}
since $\varepsilon_x = \bar{\varepsilon}_x$ for a \kv distribution.
We thus transform the particles of the continuous focusing channel
from the actual time to that instant, where the phase advance
in the continuous focusing system agrees with the $s$-dependent system.
The meaning of the transformation is illustrated graphically
in Fig.~\ref{sigma-x}.
% Figure environment removed
% Figure environment removed

The $s$-dependent generating function $F_2$
of the transformation is given by
\begin{eqnarray} \label{gen1}
F_2(x,\tilde{x}',y,\tilde{y}',s) & = &
\hal m_0 c \beta \gamma \, \sum_i \left[
\left( \kappa_x^{-1} \, \tilde{x}_i'^2 + \kappa_x \, x_i^2 \right)
\tan \Psi_x(s) \, + \, \frac{2x_i \tilde{x}_i'}{\cos \Psi_x(s)} \right]
\nonumber \\ & & \mbox{} + \similar \; .
\end{eqnarray}
$\Psi_x(s)$ abbreviates the difference of the `phase advances' of
both systems:
\begin{equation} \label{phadv}
\Psi_x(s) = \bar{\sigma}_x(s) - \sigma_x(s) = \kappa_x \cdot \ell_x(s)\; ;
\end{equation}
all quantities likewise in $y$.
The linear transformation following from (\ref{gen1})
can be written in matrix notation:
\begin{equation} \label{tra1}
\pmatrix{x_i \cr ~ \cr x_i'\cr} =
\pmatrix{\hfill \cos \Psi_x(s) & -\kappa_x^{-1} \cdot \sin \Psi_x(s) \cr ~ \cr
\kappa_x \cdot \sin \Psi_x(s) & \hfill \cos \Psi_x(s) \cr}
\pmatrix{\tilde{x}_i \cr ~ \cr \tilde{x}_i' \cr} \; ,
\end{equation}
again likewise in $y$.
The partial time derivative of (\ref{gen1}) is given by
$$
c \beta \frac{\partial F_2}{\partial s} = \hal m_0 c^2 \beta^2 \gamma
\left(\frac{\<x^2\>}{\<\bar{x}^2\>} - 1 \right) \cdot \sum_i
\left(\tilde{x}_i'^2 + \kappa_x^2 \, \tilde{x}_i^2 \right) + \similar \; .
$$
The transformed Hamiltonian $\tilde{H}_{\perp}$ is obtained by expressing
the original one in the form (\ref{ham0p})
with the new coordinates according to (\ref{tra1})
and adding the partial time derivative of the generating
function $F_2$:
\begin{eqnarray} \label{ham1}
\tilde{H}_{\perp} \!\!\! & = \!\!\! & H_{\perp} +
c \beta \, \frac{\partial F_2}{\partial s} \nonumber \\
\!\!\! & = \!\!\! & \hal m_0 c^2 \beta^2 \gamma \sum_i \left[
\frac{\<x^2\>}{\<\bar{x}^2\>}\left(\tilde{x}_i'^2+
\frac{\<x'^2\>}{\<x^2\>}\tilde{x}_i^2\right)+
\frac{\<y^2\>}{\<\bar{y}^2\>}\left(\tilde{y}_i'^2+
\frac{\<y'^2\>}{\<y^2\>}\tilde{y}_i^2\right) \right] \nonumber \\
& &
\end{eqnarray}
\subsection{The first Canonical Transformation for non-\kv distributions}
\label{non-kv}
If the space charge potential terms contained in (\ref{ham0p})
do not cancel, the resulting equations of motion are no longer linear.
Therefore, is is not possible to provide
an analytical solution of these equations in general.
The generating function (\ref{gen1}) generates a {\em finite\/}
time step $\Delta t$ that is valid for a {\em linear\/} beam transformation.
For a {\em nonlinear\/} beam transformation, we can only write a
function that generates an {\em infinitesimal\/} time step $\delta t$.
The function $F_2$ for this case corresponds to (\ref{gen1}),
but the finite distance $\ell_x$ is replaced by the infinitesimal
$\delta\ell_x = \ell_x'(s) \delta s$:
$$
\delta\ell_x = \left( \frac{\<x^2\>}{\<\bar{x}^2\>} - 1 \right) \delta s \; .
$$
Thus
\begin{eqnarray} \label{gen1p}
F_2(x,\tilde{x}',y,\tilde{y}',s) & = &
\hal m_0 c \beta \gamma \, \sum_i \left[
\left( \tilde{x}_i'^2 + \kappa_x^2\, x_i^2 \right) \delta\ell_x(s) \, + \,
2x_i \tilde{x}_i'\left( 1 + \hal \kappa_x^2\, \delta\ell_x^2(s)\right) \right]
\nonumber \\ & & \mbox{} + \similar \; .
\end{eqnarray}
Consequently, the linear transformation following from (\ref{gen1p}) is
the infinitesimal limit of (\ref{tra1}):
\begin{equation} \label{tra1p}
\pmatrix{x_i \cr ~ \cr x_i'\cr} =
\pmatrix{1 & - \delta\ell_x(s) \cr ~ \cr
\kappa_x^2 \cdot \delta\ell_x(s) & 1 \cr}
\pmatrix{\tilde{x}_i \cr ~ \cr \tilde{x}_i' \cr} \; .
\end{equation}
The partial time derivative of (\ref{gen1p}) agrees with the \kv case:
$$
c \beta \frac{\partial F_2}{\partial s} = \hal m_0 c^2 \beta^2 \gamma
\left(\frac{\<x^2\>}{\<\bar{x}^2\>} - 1 \right) \cdot \sum_i
\left(\tilde{x}_i'^2 + \kappa_x^2 \, \tilde{x}_i^2 \right) + \similar \; .
$$
If we insert (\ref{tra1p}) into the expression (\ref{scpot_l}) for
the space charge potential at $s_0$, we obtain the potential at
$s_0+\delta\ell_x$.
Since (\ref{tra1p}) is an infinitesimal canonical transformation,
we can restrict ourselves to terms up to the first order in $\delta s$:
$$
V_{\sc}(x_i,y_i) = V_{\sc}(\tilde{x}_i,\tilde{y}_i) -
\delta\ell_x\cdot\frac{d}{ds}\,V_{\sc}(\tilde{x}_i,\tilde{y}_i) \; .
$$
This is true if $\delta\ell_x = \delta\ell_y$, i.e.\ if
the beam fulfills the similarity condition
\begin{equation} \label{simi}
\frac{\<x^2\>}{\<\bar{x}^2\>} = \frac{\<y^2\>}{\<\bar{y}^2\>} \; .
\end{equation}
Under these circumstances, the transformed Hamiltonian
$\tilde{H}_{\perp}$ can be written as
\begin{eqnarray} \label{ham1p}
\tilde{H}_{\perp} \!\!\! & = \!\!\! & \hal m_0 c^2 \beta^2 \gamma \sum_i \left[
\frac{\<x^2\>}{\<\bar{x}^2\>}\left(\tilde{x}_i'^2+
\frac{\<x'^2\>}{\<x^2\>}\tilde{x}_i^2\right)+
\frac{\<y^2\>}{\<\bar{y}^2\>}\left(\tilde{y}_i'^2+
\frac{\<y'^2\>}{\<y^2\>}\tilde{y}_i^2\right)
\right] \nonumber \\ \!\!\! & \!\!\! & \mbox{} +
\frac{q}{2\gamma^2}\sum_i \bigl[
V_{\sc}(\tilde{x}_i,\tilde{y}_i)-V_{\sc}^u (\tilde{x}_i,\tilde{y}_i)\bigr]
\nonumber \\ \!\!\! & \!\!\! & \mbox{} -
\frac{q}{2\gamma^2} \delta\ell_x \, \frac{d}{ds} \sum_i \bigl[
V_{\sc}(\tilde{x}_i,\tilde{y}_i)-V_{\sc}^u (\tilde{x}_i,\tilde{y}_i)\bigr]\; .
\end{eqnarray}
Again all terms proportional to $(\delta s)^2$ have been neglected.
\subsection{The second Canonical Transformation}
After transforming the particles from corresponding time to corresponding
phase advance, as has been done by the first transformation,
the coordinates of both systems can be correlated linearly.
We can write the generating function $\tilde{F}_2$
of the second canonical transformation as
\begin{eqnarray} \label{gen2}
\tilde{F}_2(\tilde{x},\bar{x}',\tilde{y},\bar{y}',s) & = &
m_0 c \beta \gamma \sum_i \left( \sqrt{\frac{\<\bar{x}^2\>}{\<x^2\>}}
\cdot \tilde{x}_i \bar{x}_i' - \hal
\frac{\<\bar{x} \bar{x}'\>}{\<x^2\>} \cdot \tilde{x}_i^2 \right) \nonumber \\
& & \mbox{} + \similar \; .
\end{eqnarray}
The following linear transformation is calculated from (\ref{gen2}):
\begin{equation} \label{tra2}
\pmatrix{\tilde{x}_i \cr ~ \cr \tilde{x}_i' \cr} =
\pmatrix{\sqrt{\<x^2\>/\<\bar{x}^2\>} & 0 \cr ~ \cr
-\<\bar{x} \bar{x}'\>/\sqrt{\<x^2\>\<\bar{x}^2\>} \quad & \quad
\sqrt{\<\bar{x}^2\>/\<x^2\>} \cr}
\pmatrix{\bar{x}_i \cr ~ \cr \bar{x}_i' \cr} \; .
\end{equation}
The partial time-derivative of (\ref{gen2}) is given by
$$
c \beta \frac{\partial \tilde{F}_2}{\partial s} =
m_0 c^2 \beta^2 \gamma \sum_i \left(
\frac{\<\bar{x} \bar{x}'\>}{\<\bar{x}^2\>} \bar{x}_i \bar{x}_i' -
\frac{\bar{x}_i^2}{4\<\bar{x}^2\>}
\frac{d^2}{ds^2} \<\bar{x}^2\> \right) + \similar .
$$
Due to the fact that
$V_{\sc}(\tilde{x}_i,\tilde{y}_i) \equiv V_{\sc}(\bar{x}_i,\bar{y}_i)$
for the space charge potential of a line charge distribution (\ref{scpot_l}),
the new Hamiltonian $\bar{H}_{\perp}$ is
\begin{eqnarray} \label{ham2}
\bar{H}_{\perp} \!\!\!\! & = \!\!\!\! & \tilde{H}_{\perp} +
c \beta \frac{\partial \tilde{F}_2}{\partial s} \nonumber \\
\!\!\!\! & = \!\!\!\! & \hal m_0 c^2 \beta^2 \gamma \sum_i \left[ \bar{x}_i'^2 +
\bar{x}_i^2
\left(\frac{-1}{\sqrt{\<\bar{x}^2\>}} \frac{d^2}{ds^2} \sqrt{\<\bar{x}^2\>} +
\frac{\bar{\varepsilon}_x^2(s)}{\<\bar{x}^2\>^2} \right) \right]
\nonumber \\ \!\!\!\! & \!\!\!\! & \mbox{} + \simi
\nonumber \\ \!\!\!\! & \!\!\!\! & \mbox{} + \frac{q}{2\gamma^2}\sum_i \bigl[
V_{\sc}(\bar{x}_i,\bar{y}_i)-V_{\sc}^u (\bar{x}_i,\bar{y}_i)\bigr] -
\frac{q}{2\gamma^2} \delta\ell_x \frac{d}{ds} \sum_i
\bigl[ V_{\sc}(\bar{x}_i,\bar{y}_i)-V_{\sc}^u (\bar{x}_i,\bar{y}_i)\bigr]
\nonumber \\ \!\!\!\! & \!\!\!\! & \mbox{} -
\hal m_0 c^2 \beta^2 \gamma \cdot \delta\ell_x \sum_i \left[
\frac{\bar{x}_i^2}{\<\bar{x}^2\>^2}\, \frac{d}{ds}\,\bar{\varepsilon}_x^2(s) +
\frac{\bar{y}_i^2}{\<\bar{y}^2\>^2}\, \frac{d}{ds}\,\bar{\varepsilon}_y^2(s)
\right] \; .
\end{eqnarray}
Herein, the emittance term $\<x^2\>\<x'^2\>$ contained in (\ref{ham1p})
has been replaced by
$$
\<x^2\>\<x'^2\> = \bar{\varepsilon}_x^2(s_0) = \bar{\varepsilon}_x^2(s) -
\delta\ell_x \frac{d}{ds}\, \bar{\varepsilon}_x^2(s) \; .
$$
We remind that the quadratic space charge potential
$V_{\sc}^u(\bar{x}_i,\bar{y}_i)$
can be expressed in terms of the beam moments
$\<\bar{x} \bar{E}_x\>$ and $\<\bar{y} \bar{E}_y\>$ as already stated
in (\ref{vu1}).
To obtain the new Hamiltonian $\bar{H}_{\perp}$ in the form
corresponding to (\ref{ham0}), we define the focusing functions
$\bar{k}_x^2(s)$ and $\bar{k}_y^2(s)$
of the $s$-dependent focusing channel according to (\ref{kv1}):
$$
\bar{k}_x^2(s) =
\frac{-1}{\sqrt{\<\bar{x}^2\>}} \frac{d^2}{ds^2} \sqrt{\<\bar{x}^2\>} +
\frac{\bar{\varepsilon}_x^2(s)}{\<\bar{x}^2\>^2} +
\frac{q}{m_0 c^2 \beta^2 \gamma^3} \frac{\<\bar{x}\bar{E}_x\>}{\<\bar{x}^2\>} \; .
$$
We thus finally have at $s=s_0+\delta\ell_x$:
\begin{eqnarray} \label{hbar}
\bar{H}_{\perp} & = & \hal m_0 c^2 \beta^2 \gamma \sum_{i=1}^N
\left( \bar{x}_i'^2 + \bar{y}_i'^2 + \bar{k}_x^2(s) \, \bar{x}_i^2 +
\bar{k}_y^2(s) \, \bar{y}_i^2 \right) +
\frac{q}{2\gamma^2} \sum_{i=1}^N V_{\sc}(\bar{x}_i,\bar{y}_i)
\nonumber \\ & & \mbox{} -
\hal m_0 c^2 \beta^2 \gamma \cdot
\delta\ell_x \sum_{i=1}^N r(\bar{x}_i,\bar{y}_i,s) \; ,
\end{eqnarray}
with functions $r(\bar{x}_i,\bar{y}_i,s)$ defined as:
\begin{eqnarray*}
r(\bar{x}_i,\bar{y}_i,s) & = &
\frac{\bar{x}_i^2}{\<\bar{x}^2\>^2}\, \frac{d}{ds}\,\bar{\varepsilon}_x^2(s) +
\frac{\bar{y}_i^2}{\<\bar{y}^2\>^2}\, \frac{d}{ds}\,\bar{\varepsilon}_y^2(s) \\
& & \mbox{} + \frac{q}{m_0 c^2 \beta^2 \gamma^3} \frac{d}{ds}\,
\bigl[ V_{\sc}(\bar{x}_i,\bar{y}_i)-V_{\sc}^u (\bar{x}_i,\bar{y}_i)\bigr] \, .
\end{eqnarray*}
A sequence of canonical transformations can always be regarded
as a single canonical transformation.
Therefore, the Hamiltonian at a finite position $s = s_0 + \ell_x$
is obtained if the canonical transformations (\ref{tra1p}) and (\ref{tra2})
are applied repeatedly.

For a \kv distribution, we have $V_{\sc} \equiv V_{\sc}^u$ and
$\varepsilon_{x,y} = {\rm const.}$, thus
\mbox{$r(\bar{x}_i,\bar{y}_i,s) = 0$} for all indices $i$.
In this case, (\ref{hbar}) agrees exactly with Hamiltonian (\ref{ham0}),
from which the equations of motion (\ref{bewe}) can be derived
for all particles.

For non-\kv distributions, we would obtain additional unphysical terms
if we set up the Hamiltonian canonical equations using (\ref{hbar}).
Nevertheless, we know from Eq.~(\ref{sr2}) that
the sum of $r(\bar{x}_i,\bar{y}_i,s)$ over all particles $i$ vanishes:
\begin{eqnarray*}
\frac{1}{N}\sum_{i=1}^N r(\bar{x}_i,\bar{y}_i,s) & = &
\frac{1}{\<\bar{x}^2\>} \frac{d \bar{\varepsilon}_x^2(s)}{ds} +
\frac{1}{\<\bar{y}^2\>} \frac{d \bar{\varepsilon}_y^2(s)}{ds} \\
& & \mbox{} + \frac{2}{m_0 c^2 \beta^2 \gamma^3 N}
\frac{d}{ds} \bigl[ \bar{W}(s) - \bar{W}_u(s) \bigr] \\
 & = & 0 \; .
\end{eqnarray*}
We thus learn that the total sum (\ref{hbar}) is correlated to the
Hamiltonian (\ref{ham0p}) that defines the total beam energy,
which is a constant of motion.
We conclude that the total beam energy is a constant of motion
for periodic solenoid systems also.

Furthermore, the additional last term of the Hamiltonian (\ref{hbar})
vanishes for each index $i$, if \mbox{$\ell_x(s) = 0$}.
As we have already seen in section~\ref{first_kv},
this occurs at locations where the phase advances of both
the continuous and the periodic focusing system agree.
At these positions, (\ref{tra1}) resp.\ (\ref{tra1p}) reduce to unit matrices,
thus the particles in both systems are correlated simply by (\ref{tra2}).
The rms-emittances of both systems agree at these locations due to
the unit determinant of (\ref{tra2}).

In periodic quadrupole channels, the beam does not stay
rotationally symmetric.
As a consequence, the transformations (\ref{tra1p}) and (\ref{tra2})
no longer establish a strict correlation to a beam that passes through
a continuous solenoid channel.
The computer simulations presented in section~\ref{simu} show
that a slight but steady increase of the rms-emittance occurs
if a matched beam passes through a periodic quadrupole channel,
whereas no differences of the rms-emittance growth
factors are observed in periodic and continuous solenoid channels.
This suggests that the beam is reproduced completely only if
it keeps its rotational symmetry.
The emittance growth resulting from the lack of rotational symmetry
is however very small.
\section{INITIAL EMITTANCE GROWTH}
If we wish to match a beam to a periodic focusing system at the
starting point $s_0$, (\ref{tra1}) reduces to the unit matrix.
Consequently, the correlation of the particle positions is simply
determined by (\ref{tra2}):
$$
\bar{x}_i = \sqrt{\frac{\<\bar{x}^2\>}{\<x^2\>}} \cdot x_i \qquad , \qquad
\bar{y}_i = \sqrt{\frac{\<\bar{y}^2\>}{\<y^2\>}} \cdot y_i \;\;.
$$
If at this point the similarity condition
$$
\frac{\<\bar{x}^2\>}{\<x^2\>}=\frac{\<\bar{y}^2\>}{\<y^2\>}
$$
is fulfilled, the transformed electric space charge fields at
$s_0$ can be correlated to the original ones according to (\ref{e_x_l}):
$$
E_x(x_i,y_i) = \sqrt{\frac{\<\bar{x}^2\>}{\<x^2\>}} \cdot
\bar{E}_x\bigl(\bar{x}_i,\bar{y}_i\bigr) \; .
$$
We can now express $x_i'E_x(x_i,y_i)$ at $s_0$ in terms of the barred
coordinates and sum over all particles subsequently:
\begin{equation} \label{epsdericorr}
\<x^2\>\<x'E_x\> \, = \, \<\bar{x}^2\>\<\bar{x}'\bar{E}_x\> -
\<\bar{x}\bar{x}'\>\<\bar{x}\bar{E}_x\> \; .
\end{equation}
Due to the similarity, the same correspondence holds for the $y$-direction.
Since \mbox{$\<xx'\> = 0$} for a beam that is matched to a continuous focusing
channel, we can easily correlate the change of the rms-emittances
in that channel to the corresponding emittance change in the periodic channel
using Eqs.~(\ref{epsderi}) and (\ref{epsdericorr}):
$$
\frac{d}{ds}\, \varepsilon_x^2\, (s_0) = \frac{2q}{m_0 c^2 \beta^2 \gamma^3}
\<x^2\>\<x'E_x\> = \frac{d}{ds} \, \bar{\varepsilon}_x^2\, (s_0) \; .
$$
In other words, the matching transformation (\ref{tra2}) does not
change the gradient of the rms-emittance function at $s_0$.
If we transform a beam distribution that is stationary in a continuous
focusing channel, i.e.\ whose rms-emittance is a constant of motion,
we obtain for the beam transformed by (\ref{tra2}):
$$
\frac{d}{ds}\, \bar{\varepsilon}_x^2\bigl(s_0\bigr) = 0 \;\; .
$$
We thus avoid any {\em initial\/} emittance growth, if we match
a stationary phase space distribution via (\ref{tra2})
to a periodic focusing channel.
\section{MAXIMUM EMITTANCE GROWTH FOR NON-STATIONARY
\protect\\DISTRIBUTIONS} \label{final}
For an unbunched beam, Eq.~(\ref{sr2}) can be rewritten in the following
form
\begin{equation} \label{sr3}
\frac{1}{\<x^2\>(s)}\, \frac{d\, \varepsilon_x^2(s)}{ds} +
\frac{1}{\<y^2\>(s)}\, \frac{d\, \varepsilon_y^2(s)}{ds} +
\frac{K}{w_0}\cdot \frac{d}{ds}\bigl[ W(s)-W_u(s)\bigr] = 0 \; ,
\end{equation}
with $K$ denoting the `generalized perveance' (\ref{perve})
and $w_0$ the field energy normalization factor.
For unbunched beams, the latter can be expressed as
$$
w_0 = \frac{q I N}{c\beta} = \frac{q^2 N^2}{L} \; ,
$$
wherein $N$ denotes the number of particles per unit length $L$.
If $\<x^2\>$ and $\<y^2\>$ are assumed to be constant, we
can integrate Eq.~(\ref{sr3}) to obtain a correlation
between the change of the emittances $\varepsilon_x(s), \varepsilon_y(s)$
and the change of excess field energy $W(s) - W_u(s)$:
$$
\frac{\Delta \varepsilon_x^2(s)}{\<x^2\>} +
\frac{\Delta \varepsilon_y^2(s)}{\<y^2\>} +
K\cdot \frac{\Delta\bigl( W(s) - W_u(s) \bigr) }{w_0} = 0 \; .
$$
With $\Delta\varepsilon_x^2(s) = \varepsilon_x^2(s) - \varepsilon_x^2(0)$,
this equation reduces for a round beam to
$$
\frac{\varepsilon_x(s)}{\varepsilon_x(0)} = \sqrt{ 1 -
\frac{K \<x^2\>}{2\varepsilon_x^2(0)}\cdot
\frac{\Delta\bigl( W(s)-W_u(s)\bigr) }{w_0} } \; .
$$
The factor $K\<x^2\>/ \varepsilon_x^2(0)$ can be converted into a more
transparent form, if we use the transverse particle oscillation
frequencies with and without consideration of the space charge forces
(`tunes').
These oscillation frequencies are valid for linear equations of motion,
i.e.\ for a \kv beam.
As can be concluded from Eq.~(\ref{tune}),
the space charge forces cause the zero current
tune $k_0$ to be depressed down to $k$ according to
$$
\frac{k_0^2}{k^2} - 1 = \frac{K}{4 \<x^2\> k^2}\; .
$$
In order to obtain an expression that is valid for all types
of particle phase space distributions, the \kv beam radius $a$ contained in
(\ref{tune}) has been replaced by the corresponding beam moment, namely
$a^2 = 4 \<x^2\>$.
In this context, $k$ denotes the depressed betatron frequency
the beam particles would have, if the entire particle phase space
distribution was of the \kv type.
We can use this notation for non-\kv beams also, if we keep in mind
that $k$ no longer means the actual single particle oscillation frequency,
but only a global measure for the strength of the space charge effects.
The depressed tune $k\equiv \kappa_x$ is correlated to the
equilibrium beam moments for continuous focusing, as stated in Eq.~(\ref{kv2}):
$$
k^2 = \frac{\<x'^2\>}{\<x^2\>} = \frac{\varepsilon_x^2(0)}{\<x^2\>^2}\; ,
$$
thus
\begin{equation} \label{egdw}
\frac{\varepsilon_x(s)}{\varepsilon_x(0)} = \sqrt{
1 - 2\left( \frac{k_0^2}{k^2} - 1 \right) \cdot
\frac{\Delta\bigl( W(s)-W_u(s)\bigr) }{w_0} } \; .
\end{equation}
If we compare different charge distributions of the same rms-size,
we observe that the {\em minimum\/} field energy is associated
with the {\em uniform\/} charge density\cite{host}.
Consequently, only the excess field energy $W-W_u$ is eligible
for being converted into transverse kinetic energy during the
rearrangement of all charges in case that the initial
phase space distribution is not self-consistent.
The upper limit for the rms-emittance growth can be calculated
assuming that the entire excess field energy is released, i.e.\
if we neglect the equilibrium excess field energy:
$$
\Delta\bigr( W(s) - W_u(s)\bigr) = W(s) - W_u(s) - W(0) + W_u(0)
\approx -\bigr( W(0) - W_u(0) \bigr)\; .
$$
We thus obtain the following simple formula
for the upper limit of the emittance growth\cite{strekla,reiser}:
\begin{equation} \label{eg-formula}
\frac{\varepsilon_x^{\mbx}}{\varepsilon_x(0)} = \sqrt{ 1 +
2 f \left( \frac{k_0^2}{k^2} - 1 \right) }\; ,
\end{equation}
with the dimensionless geometry factor $f$ as a measure for
the initial inhomogeneity of the charge distribution:
$$
f = \frac{W(0) - W_u(0)}{w_0}\; .
$$
Eq.~(\ref{eg-formula}) leads to good estimates especially in the high
current regime, where equilibrium charge densities are nearly uniform,
as shown in section~\ref{water}.
For the non-stationary `water bag' distribution, we have
$f = \frac{5}{24} - \frac{1}{2}\ln\frac{3}{2}$ as calculated in
Ref.~\cite{strekla}.
In our simulation examples, we adjusted the zero current phase advance to
$\sigma_0 = k_0 / S = 60^\circ$ and the actual phase advance to
$\sigma = k / S = 15^\circ$.
If we insert these values into (\ref{eg-formula}), we readily obtain
for the upper limit of the emittance growth resulting from the
rearrangement of the charges towards a stationary, i.e.\
more uniform state:
$$
\frac{\varepsilon_x(s)}{\varepsilon_x(0)} \leq
\frac{\varepsilon_x^{\mbx}}{\varepsilon_x(0)} =
\sqrt{1 + \left(\frac{5}{12} - \ln{\frac{3}{2}} \right)
\left( \frac{60^2}{15^2} - 1\right) } \approx 1.081 \; .
$$
Fig.~\ref{zcwb_egf} shows that this upper limit value is excellently
confirmed by our computer simulations, although the implicit assumption
that $\<x^2\>$ and $\<y^2\>$ would be constant is not strictly fulfilled.
\section{RESULTS OF NUMERICAL SIMULATIONS} \label{simu}
\subsection{Overview}
We started our simulations of charged particle beams
with three different types of initial phase space distributions:
\begin{itemize}
\item \kv distribution as described in section~\ref{k-v},
\item stationary `water bag' distribution
as described in section~\ref{water},
\item `water bag' distribution generated for the zero current limit.
In consequence, it is {\em not\/} stationary for finite beam currents.
\end{itemize}
As transport devices we used the three basic types, i.e.\
\begin{itemize}
\item a continuous focusing solenoid channel (Fig.~\ref{systems}a),
\item a periodic focusing solenoid channel (Fig.~\ref{systems}b),
\item a periodic focusing quadrupole channel (Fig.~\ref{systems}c).
\end{itemize}
% Figure environment removed
In all three channels, the focusing strength has been adjusted to generate
a zero current phase advance of $\sigma_0 = 60^\circ$ over the
distance of one focusing period $S$.
The beam current and the initial value of the rms-emittance have been
chosen equally for all three types of distributions
to yield a depressed tune of $\sigma = 15^\circ$.

Figures~\ref{kv_egf}, \ref{scwb_egf}, and \ref{zcwb_egf} show
the emittance growth factors as they occur while the respective
beam passes through the various types of transport channels.
To illustrate the behavior of the different particle phase space
distributions, two-dimensional projections of the phase space points
are plotted at specific positions $s$ along the beam line.
Since these plots look nearly identical for all three types of
channels, only the plots pertaining to the
periodic quadrupole channel are being displayed.
% Figure environment removed
\subsection{`K-V' Distribution}
In Fig.~\ref{kv_egf}, the ratios
$\varepsilon_{x,y}(s)/\varepsilon_{x,y}(0)$
are plotted for an initial \kv distribution.
Since the \kv distribution is stationary for all cases, no
emittance change should be observed at first sight.
The slight increase of the emittance just after launching the beam can be
attributed to the inaccuracy introduced by the small number of simulation
particles ($10^5$) that represent the \kv beam.
In contrast, the decrease of the rms-emittance beyond
cell no.~$7$ has a physical background.
As can be observed in the top picture of Fig.~\ref{wbperi}, the \kv distribution is not
sustained.
The $x,x'$- and the $y,y'$-phase space projections become rectangular
rather than elliptical.
In the $x,y$-plane, the charges rearrange to form a uniform density
in the beam center and a decreasing density at the beam edge.
This reflects the fact that the \kv distribution represents an unstable
equilibrium distribution.
It is converted into a stable distribution that has `screening'
properties.
The resulting charge distribution is less uniform than the
strictly uniform \kv state.
This transition is associated with an increase of the -- initially
vanishing -- excess field energy.
Corresponding to equation (\ref{egdw}), this is in turn related to a
{\em decrease\/} of the rms-emittance.
We observe that the emittance growth factors agree
remarkably for all types of transport channels.
This indeed indicates that the beam transport through continuous
and periodic channels are correlated as shown in section~\ref{catra}.
% Figure environment removed
\subsection{Stationary `Water Bag' Distribution}
Fig.~\ref{scwb_egf} shows the emittance ratios for the
stationary `water bag' case.
In the first step, the phase space distribution is generated
for the continuous focusing channel.
It is characterized by a uniform density of phase space points
within the boundary given by Eq.~(\ref{border}).
The space charge parameter contained in (\ref{border})
evaluates to $\kappa a = 8.19$ for our simulation examples.
This distribution can be transported directly through the
continuous solenoid channel.
For both types of periodic channels, we have to apply the matching
transformation (\ref{tra2}) before starting
the beam transport simulations.
The second order beam moments are thus adjusted to the initial moments
evaluated earlier for the \kv distribution that proved to yield
$S$-periodic (`matched') solutions of the envelope equation (\ref{kv1}).

Regarding Fig.~\ref{scwb_egf}, we observe practically no changes
of the rms-emittance values for all three types of transport channels.
For the continuous solenoid channel, the emittance is
a constant of motion.
Therefore, the residual emittance fluctuations must be statistical
errors due to the limited number of simulation particles and the
necessarily limited accuracy of the computer simulations.
The range of these fluctuations agrees for all types of transport channels.
A slight increase of the rms-emittance is observed for the quadrupole channel.
Nevertheless the increase is fairly low -- only a factor of $1.002$
is obtained after $50$ periods in our example.
Maybe this effect can be attributed to the fact that
the beam no longer stays rotationally symmetric in a quadrupole channel.
Under these circumstances, the beam transformation through periodic
and continuous channels can be correlated only approximately, as already
stated at the end of section~\ref{catra}.

If we examine how the phase space projections evolve bottom picture of (Fig.~\ref{wbperi}),
practically no changes in the real space plane can be observed.
On the other hand, we see that the sharp boundary in the $x',y'$-plane
is not sustained.
The latter effect is an artifact of the `water bag' distribution
(\ref{wbdef}), which assumes that the uniformly populated phase space
ends abruptly at $w=w_{\mbx}$.
Since the computer simulations can never be completely accurate,
this boundary becomes smeared.
% Figure environment removed
\subsection{Non-stationary `Water Bag' Distribution}
% Figure environment removed
%% Figure environment removed
% Figure environment removed
In the zero current limit, the stationary `water bag' distribution
is characterized by an elliptical phase space boundary, and a charge
density in real space decreasing quadratically:
$n(r) \propto 1-r^2/r^2_{\mbx}$,
as has been shown in section~\ref{water}.
If this distribution is launched under space charge conditions,
it is in a non-equilibrium state, even if the second order beam
moments are perfectly matched.
This results in a complete rearrangement of the whole
populated phase space within the first plasma period.
In real space, this leads to a more uniform charge density profile,
which means that the excess field energy $W-W_u$ is reduced.
According to Eq.~(\ref{egdw}), this reduction is accompanied by
an {\em increase\/} of the transverse rms-emittances.
If we assume the total excess field energy to be converted
into transverse thermal energy, hence emittance, we obtain the
upper limit for the growth of the transverse rms-emittances.
As already calculated in section~\ref{final}, this increase
amounts to $\approx 8 \,\%$ for the non-stationary `water bag'
distribution transformed under space charge conditions that
cause a tune depression of $k_0/k = 4$.
This result is confirmed by our simulation examples,
as we can see in Fig.~\ref{zcwb_egf}.
Again, the emittance growth factors
agree for all three types of transport channels.
We thus learn that our channels can be regarded as
`equivalent' even for non-stationary beams.

The reordering of all phase space points that occurs if the initial state
of the distribution is not stationary can be examined in Fig.~\ref{zcwbperi}.
The real space $x,y$-projection takes on a uniform density, except at
the beam edge.
In the $x,x'$- and the $y,y'$-subspaces, we see that
the initial elliptical symmetry becomes rectangular,
although oscillations have not completely relaxed
at the end of our calculations (cell no.~$50$).
The $x',y'$-plots show that a considerable heating takes place, which
reflects the effect that excess field energy is converted into
transverse thermal energy.
\subsection{Long Distance Beam Transport}
% Figure environment removed
Fig.~\ref{long-term} shows the rms-emittance growth factors as they
occur if we transform a stationary `water bag' beam
over a long distance through our three types of transport channels.

We observe that the emittance change factors for periodic and
continuous focusing solenoid channels agree completely.
The factors converge to a value slightly below $1.0$ because the
initially sharp boundary of the populated phase space is smeared out.
In real space this means that the charge density of the beam
becomes less homogeneous,
hence with regard to Eq.~(\ref{egdw}) the rms-emittance is reduced.
Once this rearrangement has taken place, no further gradient
of the rms-emittance change factors is observed any more.

This is not true for the periodic quadrupole channel.
We observe a small but steady increase of the rms-emittance growth
factors which in our simulation example
($\sigma_0 = 60^\circ$, $\sigma = 15^\circ$) accumulates
to a factor of $1.015$ after $400$ focusing periods.

In contrast to periodic solenoid channels, we could not show
that the total transverse beam energy is a conserved quantity
in periodic quadrupole channels.
If this energy is indeed not conserved {\em strictly\/} in quadrupole channels,
a steady degradation of the beam quality must be anticipated.
We suggest that this is the reason why there is no
saturation of the emittance growth factors.
\section{CONCLUSIONS}
We have compared the transformation of intense ion beams passing
through both continuous and periodic focusing channels.
It is already known that the motion of the beam particles in both types
of transport systems can be correlated if the equations of motion are linear.
This occurs if the particle phase space distribution is of the \kv type.
Explicitly, the relationship is given by a canonical transformation,
that was derived earlier in order to map a time-dependent harmonic
oscillator onto a time-independent.
This transformation has been generalized
for `physically realistic' phase space distributions,
where the equations of motion are no longer linear.

We have shown that at least for periodic solenoid channels,
where the beam stays rotationally symmetric, stationary non-\kv
particle phase space distributions exist.
No additional assumption about the time evolution of the
phase space distribution function has been made.
We thus conclude that it is possible in principle to transport intense
ion beams without loss of quality over arbitrary long distances by
periodic solenoid channels.
This result is excellently confirmed by our computer simulations,
where after a distance of $400$ focusing periods identical rms-emittances
have been observed in continuous and periodic focusing solenoid channels.

For quadrupole channels, the beam no longer stays round.
The canonical transformation does not apply strictly for these cases
since the space charge potential terms do not keep their form.
Computer simulations show a fairly low but not saturating
increase of the rms-emittance in periodic quadrupole channels.
It can be concluded that periodic quadrupole channels are also capable
to reproduce these distributions to a high degree after each cell.
But in contrast to periodic solenoid channels,
the beam is not reproduced {\em exactly\/} by quadrupole channels.

\begin{thebibliography}{99}
\bibitem{lapo}{P.M.~Lapostolle, CERN, ISR/300 LIN/69-19~(1969)}
\bibitem{gluck}{R.L.~Gluckstern, R.~Chasman, and K.~Crandall,
    Linac Conference, FNAL, 823 (1970)}
\bibitem{kapvla}{I.M.~Kapchinskij and V.V.~Vladimirskij, Proc.\ Int.\ Conf.\
     on High Energy Accelerators, CERN, Geneva, 1959, p.~247}
\bibitem{hola}{I.~Hofmann, L.J.~Laslett, L.~Smith, and I.~Haber,
     {\em Particle Accelerators} {\bf 13}, 145 (1983)}
\bibitem{hofm}{I.~Hofmann,
     {\em Advances in Electronics and Electron Physics, Supplement 13C\/}
     (A.~Septier, ed.), Academic Press, New York, 49-140 (1983)}
\bibitem{cousny}{E.D.~Courant and H.S.~Snyder,
     {\em Annals of Physics} {\bf 3}, 1-48 (1958)}
\bibitem{lewis}{H.R.~Lewis,~Jr.,
     {\em J.~Mathematical~Phys.\/} {\bf 9}, 1976-1986 (1968)}
\bibitem{leach}{P.G.L.~Leach,
     {\em SIAM J.~Appl.~Math.\/} {\bf 34}, No.~3, 496-503 (1978)}
\bibitem{sach}{F.J.~Sacherer,
     {\em IEEE Trans.~Nucl.~Sci.\/} {\bf NS-18}, 1105 (1971)}
\bibitem{leecoo}{E.P.~Lee and R.K.~Cooper,
     {\em Particle Accelerators\/} {\bf 7}, 83-95 (1976)}
\bibitem{wang}{T.P.~Wangler, K.R.~Crandall, R.S.~Mills, and M.~Reiser,
     {\em IEEE Trans.~Nucl.~Sci.\/} {\bf NS-32}, 2196 (1985)}
\bibitem{strekla}{J.~Struckmeier, J.~Klabunde, and M.~Reiser,
     {\em Particle Accelerators\/} {\bf 15}, 47-65 (1984)}
\bibitem{host}{I.~Hofmann and J.~Struckmeier,
     {\em Particle Accelerators\/} {\bf 21}, 69-98 (1987)}
\bibitem{reiser}{M.~Reiser, {\em J.~Appl.~Phys.\/} {\bf 70}, 1919 (1991)}
\end{thebibliography}
\end{document}
\typeout{get arXiv to do 4 passes: Label(s) may have changed. Rerun}
