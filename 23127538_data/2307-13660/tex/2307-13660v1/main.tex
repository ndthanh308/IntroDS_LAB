\documentclass[12pt]{article}
\usepackage{standalone}
\usepackage{amsmath,amssymb, amsthm}
\usepackage{mathrsfs}
\usepackage{indentfirst}
\usepackage{fullpage}
\usepackage{tikz}
\usepackage{bm}
\usepackage{float}
\usepackage{hyperref}
\usepackage{nicefrac}
\usepackage[title]{appendix}
\usepackage{thmtools}
\usepackage{thm-restate}
\usepackage{cleveref}

\delimitershortfall-1sp
\usepackage{mleftright}
\mleftright % make \left & \right behave like \mleft & \mright 
% \usetikzlibrary{external}
% \tikzexternalize[prefix=tikz/]
\usetikzlibrary{shapes.geometric, shapes.misc, arrows, arrows.meta, decorations.markings, decorations.pathreplacing, calc, positioning}

\definecolor{Xcolor}{HTML}{FF0000}%FF312E
\definecolor{Ycolor}{HTML}{00FFFF}

\newtheorem{theorem}{Theorem}
\newtheorem{example}[theorem]{Example}
%\renewcommand*{\thetheorem}{\Alph{theorem}}
\newtheorem{lemma}{Lemma}
\newtheorem{claim}{Claim}
\newtheorem*{corollary}{Corollary}
\theoremstyle{definition}
\newtheorem{definition}{Definition}
\theoremstyle{remark}
\newtheorem*{remark}{Remark}

\makeatletter
\newcommand{\subalign}[1]{%
  \vcenter{%
    \Let@ \restore@math@cr \default@tag
    \baselineskip\fontdimen10 \scriptfont\tw@
    \advance\baselineskip\fontdimen12 \scriptfont\tw@
    \lineskip\thr@@\fontdimen8 \scriptfont\thr@@
    \lineskiplimit\lineskip
    \ialign{\hfil$\m@th\scriptstyle##$&$\m@th\scriptstyle{}##$\hfil\crcr
      #1\crcr
    }%
  }%
}
\makeatother

\DeclareMathOperator*{\argmax}{arg\,max}
\DeclareMathOperator*{\argmin}{arg\,min}
\DeclareMathOperator{\diam}{diam}
\DeclareMathOperator{\rad}{rad}
\DeclareMathOperator{\ecc}{ecc}
\DeclareMathOperator{\dis}{dis}
\DeclareMathOperator{\codis}{codis}
\DeclareMathOperator{\tr}{tr}
\DeclareMathOperator*{\defeq}{\buildrel \mathrm{def}\over =}

\newcommand{\XX}{\mathbf{X}}
\newcommand{\YY}{\mathbf{Y}}
\newcommand{\FF}{\mathbf{F}}
\newcommand{\GG}{\mathbf{G}}
\newcommand{\VV}{\mathbf{V}} % X__Y
\newcommand{\UU}{\mathbf{U}} % Y__X
\newcommand{\WW}{\mathbf{W}} % _Y_X
\newcommand{\R}{\mathbb{R}}
\newcommand{\RR}{\mathbf{R}}
\renewcommand{\SS}{\mathbf{S}}
\newcommand{\DD}{\mathbf{R}}
\newcommand{\HH}{\mathbf{H}}
\newcommand{\KK}{\mathbf{K}}
\newcommand{\RRR}{\mathcal{R}}
\newcommand{\SSS}{\mathcal{S}} % bi-mapping polytope
\newcommand{\dGH}{d_{\mathrm{GH}}}
\newcommand{\nconv}{\mathrm{nconv}}
\newcommand{\lmax}{\lambda_{\max}}
\newcommand{\dmax}{d_{\max}}
\newcommand{\pmax}{p_{\max}}
\newcommand{\obj}{\sigma}
\newcommand{\ch}{\mathrm{ch}}
\newcommand{\xx}{\mathbf{x}}
\newcommand{\yy}{\mathbf{y}}
\newcommand{\zz}{\mathbf{z}}

%opening
\title{Computing the Gromov--Hausdorff distance using first-order methods}
\author{Vladyslav Oles\thanks{\texttt{vlad.oles@proton.me}}}
\date{}
\begin{document}

\maketitle

\begin{abstract}
The Gromov--Hausdorff distance measures the difference in shape between compact metric spaces and poses a notoriously difficult problem in combinatorial optimization. We introduce its quadratic relaxation over a convex polytope whose solutions provably deliver the Gromov--Hausdorff distance. The optimality guarantee is enabled by the fact that the search space of our approach is not constrained to a generalization of bijections, unlike in other relaxations such as the Gromov--Wasserstein distance.

We suggest the Frank--Wolfe algorithm with $O(n^3)$-time iterations for solving the relaxation and numerically demonstrate its performance on metric spaces of hundreds of points. In particular, we obtain a new upper bound of the Gromov--Hausdorff distance between the unit circle and the unit hemisphere equipped with Euclidean metric. Our approach is implemented as a Python package \texttt{dgh}.
\end{abstract}

\section*{Notation}
\def\arraystretch{1.5}
\vspace{-3mm}
\begin{table}[H]
\begin{tabular}{rp{0.8\textwidth}}
$X \to Y$ & the set of mappings from set $X$ to set $Y$ \\
$d_X$ & metric on set $X$ \\
$\dGH(X, Y)$ & the Gromov--Hausdorff distance between metric spaces $X$ and $Y$\\
 $\mathbf{A}$ & finite matrix with entries $A_{ij}$\\
$c^\mathbf{A}$ & entrywise exponentiation with entries $c^{A_{ij}}$ \\
$\left\|\mathbf{A}\right\|_p$ & entrywise $p$-norm $\left(\sum_{i,j}\left|A_{ij}\right|^p\right)^{1/p}$\\
$\|\mathbf{A}\|_\infty$ & entrywise $\infty$-norm $\max_{i,j}\left|A_{ij}\right|$\\
$\overline{\mathbf{A}\mathbf{B}}$ & line segment $\left\{\alpha\mathbf{A} + (1-\alpha)\mathbf{B}: \alpha \in [0,1]\right\}$\\
\end{tabular}
\end{table}
\def\arraystretch{1}


\section{Introduction}
Deep learning models have been widely used in many applications.
For example, BERT~\citep{devlin_bert_2019}, GPT-3~\citep{brown_language_2020}, and T5~\citep{raffel_exploring_2020} achieved state-of-the-art~(SOTA) results on different natural language processing~(NLP) tasks. 
For computer vision~(CV), Transformer-like models such as ViT~\citep{dosovitskiy_image_2021} and Swin Transformer~\citep{liu_swin_2021} deliver excellent accuracy performance upon multiple tasks. 


At the same time, training deep learning models has been a critical problem troubling the community due to the long training time, especially for those large models with billions of parameters~\citep{brown_language_2020}. 
In order to enhance the training efficiency, researchers propose some manually designed parallel training strategies~\citep{narayanan_efficient_2021,shazeer_mesh-tensorflow_2018,xu_gspmd_2021}. 
However, selecting, tuning, and combining these strategies require extensive domain knowledge in deep learning models and hardware environments. With the increasing diversity of modern hardware architectures~\cite{flynn_very_1966,flynn_computer_1972} and the rapid development of deep learning models, these manually designed approaches are bringing heavier burdens to developers. 
Hence, \emph{automatic parallelism} is introduced to automate the parallel strategy searching for training models.


There are two main categories of parallelism in deep learning models: inter-layer parallelism~\citep{huang_gpipe_2019,narayanan_pipedream_2019,narayanan_memory-efficient_2021,fan_dapple_2021,li_chimera_2021,lepikhin_gshard_2021,du_glam_2022,fedus_switch_2022} and intra-layer parallelism~\citep{li_pytorch_2020,narayanan_efficient_2021,rasley_deepspeed_2020,fairscale_authors_fairscale_2021}. 
Inter-layer parallelism partitions the model into disjoint sets on different devices without slicing tensors. 
Alternatively, intra-layer parallelism partitions tensors in a layer along one or more axes and distributes them across different devices.


Current automatic parallelism techniques focus on optimizing strategies within these two categories. However, they treat these two categories separately. 
Some methods~\citep{zhao_vpipe_2022,jia_exploring_2018,cai_tensoropt_2022,wang_supporting_2019,jia_beyond_2019,schaarschmidt_automap_2021,liu_colossal-auto_2023} overlook potential opportunities for inter- or intra-layer parallelism, the others optimize inter- and intra-layer parallelism hierarchically and sequentially~\citep{narayanan_pipedream_2019,fan_dapple_2021,he_pipetransformer_2021,tarnawski_efficient_2020,tarnawski_piper_2021,zheng_alpa_2022}. 
As a result, current automatic parallelism techniques often fail to achieve the global optima and instead become trapped in local optima. 
Therefore, a unified inter- and intra-layer approach is needed to enhance the effectiveness of automatic parallelism.


This paper aims to find the optimal parallelism strategy while simultaneously considering inter- and intra-layer parallelism. 
It enables us to search in a more extensive strategy space where the globally optimal solution lurk. 
However, unifying inter- and intra-layer parallelism in automatic parallelism brings us two challenges. 
Firstly, to adopt a unified perspective on the inter- and intra-layer automatic parallelism, we should not formalize them with separate formulations as prior works. Therefore, how can we express these parallelism strategies in a unified formulation? 
Secondly, previous methods take a long time to obtain the solution with a limited strategy space. Therefore, how can we ensure that the best solution can be obtained in a reasonable time while expanding the strategy space?


To solve the above challenges, we propose UniAP. For the first challenge, UniAP adopts the mixed integer quadratic programming~(MIQP)~\citep{lazimy_mixed_1982} to search for the globally optimal parallel strategy automatically. 
It unifies the inter- and intra-layer automatic parallelism in a single MIQP formulation. 
For the second challenge, our complexity analysis and experimental results show that UniAP can obtain the globally optimal solution in a significantly shorter time.


The contributions of this paper are summarized as follows: 
\begin{itemize}
    \item We propose UniAP, the first framework to unify inter- and intra-layer automatic parallelism in model training.
    \item The optimal parallel strategies discovered by UniAP exhibit scalability on training throughput and strategy searching time.
    \item The experimental results show that UniAP speeds up model training on four Transformer-like models by up to 1.70$\times$ and reduces the strategy searching time by up to 16$\times$, compared with the SOTA method.
\end{itemize}

\section{Relaxing the Gromov--Hausdorff distance}
\label{relaxation}

\subsection{Matrix reformulation}
Let $X = \{x_1, \ldots, x_n\}$ and $Y = \{y_1, \ldots, y_m\}$ be metric spaces whose finite cardinalities satisfy $n \geq m > 1$, and let $\XX \in \R^{n\times n}$ and $\YY \in \R^{m\times m}$ denote their corresponding distance matrices. Recall that the Gromov--Hausdorff distance between $X$ and $Y$ can be formulated as a combinatorial minimization over the bi-directional mapping pairs $(f,g)$
$$\dGH(X, Y) = \frac{1}{2}\min_{\substack{f:X\to Y,\\ g:Y\to X}} \dis\left(\left\{\left(x, f(x)\right): x \in X\right\} \cup \left\{\left(g(y), y\right): y \in Y\right\}\right),\quad\cite{kalton1999distances}$$ where the distortion of some relation $R \subseteq X \times Y$ is defined as the largest absolute difference in distances incurred by it: $$\dis R \defeq \max_{\subalign{(x, y),(x', y') \in R}} \left|d_X(x, x') - d_Y(y, y')\right|.$$

For some $f:X\to Y$, consider its ``one-hot encoded'' matrix representation $\FF \in \{0, 1\}^{n \times m}$ s.t. $F_{ij} = \begin{cases}
    1, & f(x_i) = y_j\\
    0, & \text{otherwise}
\end{cases}$ and note that $\left(\FF\YY\FF^T\right)_{ij} = d_Y\left(f(x_i), f(x_j)\right)$. It follows that $$\max_{x,x' \in X} \Big|d_X(x, x') - d_Y\left(f(x), f(x')\right)\Big| = \big\|\XX - \FF\YY\FF^T\big\|_\infty.$$
From the analogous construction of $\GG \in \{0, 1\}^{m \times n}$ for an arbitrary $g: Y \to X$, $$\max_{y,y' \in Y} \left|d_X\left(g(y), g(y')\right) - d_Y(y, y')\right| = \big\|\YY - \GG\XX\GG^T\big\|_\infty.$$
Because $(\XX\GG^T)_{ij} = d_X\left(g(y_j), x_i\right)$ and $(\FF\YY)_{ij} = d_Y\left(f(x_i), y_j\right)$, $$\max_{x \in X, y \in Y} \left|d_X\left(x, g(y)\right) - d_Y\left(f(x), y\right)\right| = \big\|\XX\GG^T - \FF\YY\big\|_\infty.$$

Denote $R = R(f, g) \defeq \left\{\left(x, f(x)\right): x \in X\right\} \cup \left\{\left(g(y), y\right): y \in Y\right\}$. Any pair $(x, y), (x', y') \in R$ satisfies
\begin{align*}
\left|d_X(x, x') - d_Y(y, y')\right| \in \bigg\{&\left|d_X(x, x') - d_Y\left(f(x), f(x')\right)\right|,\\
&\left|d_X\left(g(y), g(y')\right) - d_Y(y, y')\right|,\\
&\left|d_X\left(x, g(y)\right) - d_Y\left(f(x), y\right)\right|,\\
&\left|d_X\left(x', g(y')\right) - d_Y\left(f(x'), y'\right)\right|\bigg\},
\end{align*}
% Arranging the three matrices inside $\|\cdot\|_\infty$ as blocks of an $(n+m)\times(n+m)$ matrix gives
and therefore
\begin{align*}
    \dis R &= \max\bigg\{\begin{aligned}[t]
        &\max_{x,x' \in X} \left|d_X(x, x') - d_Y\left(f(x), f(x')\right)\right|, \\ &\max_{y,y' \in Y} \left|d_X\left(g(y), g(y')\right) - d_Y(y, y')\right|,\\ &\max_{x \in X, y \in Y} \left|d_X\left(x, g(y)\right) - d_Y\left(f(x), y\right)\right|\bigg\}
    \end{aligned}  \\
    &= \left\|\begin{bmatrix}\XX - \FF\YY\FF^T & \XX\GG^T - \FF\YY \\ \GG\XX - \YY\FF^T & \YY - \GG\XX\GG^T\end{bmatrix}\right\|_\infty \\
    &= \big\|\VV - \RR\UU\RR^T + \RR\WW - \WW^T\RR^T\big\|_\infty,
\end{align*}
where $\RR \defeq \begin{bmatrix}\FF & \\ &\GG\end{bmatrix}$ is a matrix representation of $R$ and $\VV \defeq \begin{bmatrix}\XX & \\ &\YY\end{bmatrix}$, $\UU \defeq \begin{bmatrix}\YY & \\ &\XX\end{bmatrix}$, $\WW \defeq \begin{bmatrix}& \YY \\ \XX&\end{bmatrix}$. We note that the presence of redundant block $\GG\XX-\YY\FF^T$ in the distance difference matrix $\VV - \RR\UU\RR^T + \RR\WW - \WW^T\RR^T$ is motivated by symmetry of the latter.

By construction, $\RR \in \{0, 1\}^{(n+m)\times(n+m)}$ is row-stochastic and has $m\times m$ and $n\times n$ blocks of zeros in the upper right and lower left, respectively. Let $\RRR \subset \{0, 1\}^{(n+m)\times(n+m)}$ denote the set of all such matrices, which is in a 1-to-1 correspondence with the mapping pairs $(X\to Y)\times(Y\to X)$. We will write $\dis \RR \defeq \left\|\VV - \RR\UU\RR^T + \RR\WW - \WW^T\RR^T\right\|_\infty$ to denote the matrix-based formulation of distortion, and assume that the distinction between $\dis:\RRR\to \R$ and $\dis:\mathscr{P}(X\times Y)\to\R$ is clear from the context. Then an equivalent formulation of the Gromov--Hausdorff distance can be given as
\begin{equation*}
    \label{eqn:matrix_reformulation}
    \dGH(X, Y) = \frac{1}{2}\min_{\RR \in \RRR} \dis \RR.
    \tag{$\star$}
\end{equation*}

\subsection{Relaxing the objective}
The $\infty$-norm in $\dis \RR$ deprives the (otherwise quadratic) objective of (\ref{eqn:matrix_reformulation}) of its differentiability. A standard trope in smooth relaxations of the maximum function is to involve the sum of exponents of its arguments. It turns out that, using a sufficiently large base for exponentiation, we can construct a smooth relaxation of (\ref{eqn:matrix_reformulation}) that is minimized only by solutions to (\ref{eqn:matrix_reformulation}).

Let $\Delta \defeq \left\{\left|d_X(x, x') - d_Y(y, y')\right|: x, x' \in X, y, y' \in Y\right\}$, and note that $\dis \RR \in \Delta$ for any $\RR \in \RRR$.
% \begin{definition}
The \textit{distortion gap} between $X$ and $Y$ %for a pair of finite metric spaces $X, Y$
is then defined as $$\rho = \rho(X, Y) \defeq \min \left\{|\delta - \delta'|: \delta, \delta' \in \Delta, \delta \neq \delta'\right\}.$$ Trivially, $\dis \RR \neq \dis \RR'$ for some $\RR, \RR' \in \RRR$ implies that $|\dis \RR - \dis \RR'| \geq \rho$, which provides justification for the name.
% \end{definition}

\begin{restatable}{theorem}{cthreshold}
    \label{thm:c_threshold}
    Let $c \geq \left(\frac{(n+m)^2-n-m}{2}\right)^{1/\rho}$. Then
    \begin{align*}
    \argmin_{\RR\in\RRR}\left\|c^{\VV - \RR\UU\RR^T + \RR\WW - \WW^T\RR^T} + c^{\RR\UU\RR^T - \VV + \WW^T\RR^T - \RR\WW}\right\|_1 \subseteq \argmin_{\RR\in\RRR}\dis\RR,
    \end{align*}
    where the exponentials are taken entry-wise.

\end{restatable}


Theorem \ref{thm:c_threshold} is based on the idea that a decrease in $\left\|\VV - \RR\UU\RR^T + \RR\WW - \WW^T\RR^T\right\|_\infty$, the largest magnitude in the distance difference matrix, must decrease the above 1-norm relaxation even when it leads to increasing the magnitudes of all other distance differences from zero to the new maximum. In practice, however, suboptimal choices of $\RR \in \RRR$ do not tend to align the distances in $X$ and $Y$ better than solutions to (\ref{eqn:matrix_reformulation}), and therefore much smaller values of $c$ than $\left(\frac{(n+m)^2-n-m}{2}\right)^{1/\rho}$ can satisfy the statement of Theorem \ref{thm:c_threshold}.

Note that $c$ and $c^{-1}$ behave identically in the 1-norm relaxation, which means that both $c \in (0, 1]$ and $c \in [1, \infty)$ can be considered for analogous results. For simplicity, we focus on the latter option and assume $c \geq 1$ throughout this work.

Recall that the two parts of the distance difference matrix $\VV - \RR\UU\RR^T$ and $\RR\WW - \WW^T\RR^T$ have complementary block sparsity: the former contains $n\times m$ zeros in the upper right and $m\times n$ zeros in the lower left, while the latter has $n\times n$ zeros in the upper left and $m\times m$ zeros in the lower right. It follows that
\begin{align*}
    &\left\|c^{\VV - \RR\UU\RR^T + \RR\WW - \WW^T\RR^T} + c^{\RR\UU\RR^T - \VV + \WW^T\RR^T - \RR\WW}\right\|_1 \\ &\hspace{2cm}= \left\|c^{\VV - \RR\UU\RR^T} + c^{\RR\WW - \WW^T\RR^T} - c^\mathbf{0} + c^{\RR\UU\RR^T - \VV} + c^{\WW^T\RR^T - \RR\WW} - c^\mathbf{0}\right\|_1 \\ &\hspace{2cm}= \left\|c^{\VV - \RR\UU\RR^T}\right\|_1 + \left\|c^{\RR\UU\RR^T - \VV}\right\|_1 + \left\|c^{\RR\WW - \WW^T\RR^T}\right\|_1 + \left\|c^{\WW^T\RR^T - \RR\WW}\right\|_1 - 2(n+m)^2.
\end{align*}

Leveraging the structure of $\RR$ and subsequently applying the trace trick gives
\begin{align*}
&\left\|c^{\VV - \RR\UU\RR^T}\right\|_1 = \left\langle c^{\VV}, c^{-\RR\UU\RR^T}\right\rangle = \left\langle c^{\VV}, \RR c^{-\UU}\RR^T\right\rangle = \left\langle \RR , c^{\VV}\RR c^{-\UU}\right\rangle
\end{align*}
and
\begin{align*}
&\left\|c^{\RR\WW - \WW^T\RR^T}\right\|_1 = \left\langle c^{\RR\WW}, c^{-\WW^T\RR^T}\right\rangle = \left\langle \RR c^{\WW},
\big(\RR c^{-\WW}\big)^T\right\rangle = \left\langle \RR , \big(c^{\WW}\RR c^{-\WW}\big)^T\right\rangle,
\end{align*}
as well as $\left\|c^{\RR\UU\RR^T - \VV}\right\|_1 = \Big\langle \RR, c^{-\VV}\RR c^{\UU}\Big\rangle$ and $\left\|c^{\WW^T\RR^T - \RR\WW}\right\|_1 = \left\langle \RR , \left(c^{-\WW}\RR c^{\WW}\right)^T\right\rangle$. Combining the four equations casts the 1-norm relaxation of (\ref{eqn:matrix_reformulation}) as a quadratic minimization
%with optimality guarantees on its solutions for large enough $c$:
\begin{align*}
    \label{eqn:quadratic_objective}
    \min_{\RR \in \RRR} \obj(\RR) \defeq \left\langle \RR, c^{\VV}\RR c^{-\UU} + c^{-\VV}\RR c^{\UU} + \big(c^{\WW}\RR c^{-\WW} + c^{-\WW}\RR c^{\WW}\big)^T\right\rangle.
    \tag{$\star\star$}
\end{align*}


\subsection{Relaxing the domain}
In order to enable first-order methods for solving (\ref{eqn:quadratic_objective}), its objective $\obj$ needs to be considered over a continuous domain. A common approach in combinatorial optimization is to relax the discrete domain to its convex hull, employ a gradient-based algorithm on the convex region, and project the resulting solution back onto the original domain.
% The quadratic formulation enables first-order methods to find local minima if $\obj$ is extended to a continuous domain. A common practice is to relax a discrete domain to its convex hull, and to project the found solution back according to some choice of matrix distance.
The convex hull of $\RRR$, henceworth denoted as $\SSS$, is the set of all $(n+m)\times(n+m)$ row-stochastic matrices with $m\times m$ and $n\times n$ blocks of zeros in the upper right and lower left, respectively.
% the so-called $\SSS$ comprised of $(n+m)\times(n+m)$ row-stochastic matrices with, referred to as the \textit{bi-mapping polytope} .
This is a direct consequence of generalizing the Birkhoff-von Neumann theorem to the row-stochastic matrices, which was done e.g. in \cite{gubin2008subgraph} and \cite{cao2022centrosymmetric}. Because the points of $\SSS$ represent bi-directional pairs of generalized (or ``soft'') mappings, we refer to it as the \textit{bi-mapping polytope}.

Importantly, projecting a solution to the continuous relaxation 
% \begin{equation*}
%     \label{eqn:continuous_optimization}
%     \min_{\SS \in \SSS} \obj(\SS) \defeq \left\langle \SS, c^{\VV}\SS c^{-\UU} + c^{-\VV}\SS c^{\UU} + \left(c^{\WW}\SS c^{-\WW} + c^{-\WW}\SS c^{\WW}\right)^T\right\rangle
%     \tag{$\star\star$$\star$}
% \end{equation*}
$\min_{\SS \in \SSS}\obj(\SS)$
back onto the vertices $\RRR$ always yields a solution to (\ref{eqn:quadratic_objective}). Note that the faces of $\SSS$ can be characterized similarly to those of the $(n+m)$-th Birkhoff polytope, see e.g. \cite{paffenholz2013faces}. Specifically, every face of $\SSS$ corresponds to a (possibly empty) set of forbidden assignments between the points of $X$ and $Y$ and is comprised by the convex combinations of all the compliant mapping pairs. We write it formally as
\begin{restatable}{lemma}{polytopegeometry}
    \label{lem:polytope_geometry}
    Any face $\Phi$ of the bi-mapping polytope $\SSS$ is characterized by an index set $(\mathcal{I}, \mathcal{J}) \subset \{1, \ldots, n+m\}\times\{1,\ldots,n+m\}$ s.t. $\Phi = \left\{\SS \in \SSS: S_{ij} = 0 \quad\forall (i,j)\in (\mathcal{I}, \mathcal{J})\right\}$.
\end{restatable}
Using the above result, we can show that any face of $\SSS$ containing a solution on its interior must be a part of the solution set:
\begin{restatable}{theorem}{solutiononinterior}
\label{thm:solution_on_interior}
Let $\SS^* \in \argmin_{\SS \in \SSS}\obj(\SS)$ and $\Phi$ denote the face of $\SSS$ s.t. $\SS^* \in \Phi \setminus \partial\Phi$. Then $$\Phi \subseteq \argmin_{\SS \in \SSS}\obj(\SS).$$
\end{restatable}

In particular, Theorem \ref{thm:solution_on_interior} implies that $\obj$ attains its minimum on the vertices of $\SSS$.
%(and, as a consequence, that relaxing $\RRR$ to its convex hull $\SSS$ does not change the minimum of $\obj$).
In order to guarantee that the projection of $\argmin_{\SS \in \SSS} \obj(\SS)$ onto $\RRR$ is included in $\argmin_{\RR \in \RRR} \obj(\RR)$, it remains to show that the nearest vertices to any $\SS \in\SSS$ belong to the same faces as $\SS$.
%for any face $\Phi$ and point $\SS \in \Phi$, all the nearest vertices to $\SS$ are in $\Phi$.
\begin{restatable}{theorem}{voronoiface}
    \label{thm:voronoi_face}
    Let $\Phi$ be a face of $\SSS$. For any $\SS \in \Phi$, $$\argmin_{\RR \in \RRR}\|\SS - \RR\|_2 \subseteq \Phi.$$
\end{restatable}

Searching over the bi-mapping polytope $\SSS$ is a key difference between our approach and other relaxations of the Gromov--Hausdorff distance constrained to the Birkhoff polytope. The resulting continuous relaxation %(\ref{eqn:continuous_optimization})
\begin{equation*}
    \label{eqn:continuous_optimization}
    \min_{\SS \in \SSS} \obj(\SS) \defeq \left\langle \SS, c^{\VV}\SS c^{-\UU} + c^{-\VV}\SS c^{\UU} + \big(c^{\WW}\SS c^{-\WW} + c^{-\WW}\SS c^{\WW}\big)^T\right\rangle
    \tag{$\star\star$$\star$}
\end{equation*}
%along with the optimality guarantees for its solutions
whose solutions provably deliver the Gromov--Hausdorff distance
is the main theoretical result of this work.

%which can be proved as a modification to the Birkhoff-von Neumann theorem for row-stochastic matrices.
% \begin{theorem}
%     \label{thm:convex_hull}
%     The convex hull of $\RRR$ is $\SSS$.
% \end{theorem}
'
\section{Solving the relaxation}
\label{minimization}
\subsection{The Frank--Wolfe algorithm}
(\ref{eqn:continuous_optimization}) is an indefinite %(see Section \ref{convexity})
quadratic minimization with affine constraints and a Lipschitz gradient $$\nabla\obj(\SS) = 2\left(c^{\UU}\SS c^{-\VV} + c^{-\UU}\SS c^{\VV} + \big(c^{\WW}\SS c^{-\WW} + c^{-\WW}\SS c^{\WW}\big)^T\right).$$ While finding its global minimum remains an NP-hard problem, approximate solutions can be efficiently obtained by the Frank--Wolfe algorithm \cite{frank1956algorithm}, also known as conditional gradient descent. The iterative algorithm starts at some $\SS_0 \in \SSS$. At every iteration, it finds the descent direction as a point in $\SSS$ that minimizes the cosine similarity with the gradient at the current point $\SS_i$,
$$\DD_i \in \argmin_{\SS \in \SSS}\left\langle \SS, \nabla\obj(\SS_i)\right\rangle.$$
The descent direction $\DD_i$ chosen by the algorithm is always a vertex of $\SSS$ (note that the minimum must be attained at some vertex due to the linearity of the problem).
The algorithm then finds a point on the line segment $\overline{\SS_i\DD_i}$ that minimizes $\obj$, $$\SS_{i+1} \in \argmin_{\gamma \in [0, 1]} \obj\left(\gamma\SS_i + (1-\gamma)\DD_i\right),$$
which concludes the $i$-th iteration. The algorithm's convergence is measured as the \textit{Frank--Wolfe gap} $\left\langle \SS_i - \DD_i, \nabla \obj(\SS_i) \right\rangle \geq 0$, which is zero if and only if $\SS_i$ is a stationary point. 
The algorithm terminates when the Frank--Wolfe gap becomes sufficiently small (or after reaching the iteration limit). It takes the Frank--Wolfe algorithm $O(\epsilon^{-2})$ iterations to approach a stationary point with the gap of $\epsilon$ \cite{lacoste2016convergence}, each iteration requiring $O(n^3)$ time.
% \begin{theorem}
%     \label{thm:time_complexity}
%     One Frank--Wolfe iteration for solving (\ref{eqn:continuous_optimization}) takes $O(n^3)$ time.
% \end{theorem}

\subsection{Stationary points}
The structure of $\SSS$ helps characterize the stationary points of (\ref{eqn:continuous_optimization}). The following result suggests that the trend of saddle point prevalence in high-dimensional non-convex optimization \cite{dauphin2014identifying} is unlikely to manifest in (\ref{eqn:continuous_optimization}) for a broad class of metric spaces (e.g. random metric spaces from \cite{vershik2004random}).
\begin{restatable}{theorem}{oneminimizer}
    \label{thm:one_minimizer}
    Let the distances stored in $\XX$ be realized by continuous random variables $\mathrm{D}_1, \ldots, \mathrm{D}_{\frac{n(n-1)}{2}}$ such that any $\mathrm{D}_i$ restricted to any permissible realization of the rest of the variables $\left\{\mathrm{D}_j=d_j:j\neq i\right\}$ has support of non-zero measure. If $c > 1$, then the linear minimization from the first-order necessary optimality condition for any $\SS^*\in\SSS$ almost surely has a single solution:
    $$\mathbb{P}_\XX\left[\Big|\argmin_{\SS\in\SSS} \left\langle \SS, \nabla\obj(\SS^*) \right\rangle\Big| > 1\right] = 0.$$
    %and any permissible realization of all but one of them $\{X_j=x_j:j\neq i\}$ the support of $X_i|x_1,\ldots,x_{i-1},x_{i+1},\ldots,x_\frac{n(n-1)}{2}$ has positive measure.
\end{restatable}

% $\big|\argmin_{\SS\in\SSS} \langle \SS, \nabla\obj(\SS^*) \rangle$ combined with the first-order condition $\SS^* \in \argmin$ immediately fullfills the second-order condition as it pertains to the points in $\argmin _{\SS\in\SSS} \langle \SS, \nabla\obj(\SS^*) \rangle\setminus \{\SS^*\} = \emptyset.$

For any $\SS^* \in \SSS$, the probability distribution of the distances in $\XX$ induces a random matrix $\nabla\obj(\SS^*)$ and a Bernoulli variable representing whether $\SS^*$ is a stationary point of (\ref{eqn:continuous_optimization}). Under the assumption of independence of these two random objects, Theorem \ref{thm:one_minimizer} entails that any stationary point $\SS^*$ almost surely satisfies its second-order sufficient optimality condition as the latter ends up being imposed on the empty set $\argmin _{\SS\in\SSS} \left\langle \SS, \nabla\obj(\SS^*) \right\rangle\setminus \left\{\SS^*\right\}.$ In practice, it means that the approximate solutions to (\ref{eqn:continuous_optimization}) recovered by the Frank--Wolfe algorithm are likely to be points of local minima and not saddle points.

\subsection{Extent of non-convexity}
\label{convexity}
The Hessian of $\obj$ is a constant $(n+m)^2\times(n+m)^2$ matrix $$\HH \defeq \nabla^2\obj(\cdot) = 2\left(c^{\UU}\otimes c^{-\VV} + c^{-\UU}\otimes c^{\VV} + \left(c^{\WW}\otimes c^{-\WW^T} + c^{-\WW}\otimes c^{\WW^T}\right)\KK\right),$$ where $\otimes$ is the Kronecker product and $\KK \in \{0, 1\}^{(n+m)^2\times(n+m)^2}$ denotes the commutation matrix. %Let $\Lambda \subset \R$ denote the multi-set of its $(n+m)^2$ eigenvalues.
One approach to navigating the non-convexity of $\obj$ is to consider the space spanned by the eigenvectors corresponding to negative eigenvalues of $\HH$. However, doing so has $O(n^6)$-time and $O(n^4)$-memory complexity and is impractical for most cases. As a cheaper substitute, we assess the extent of non-convexity of $\obj$ using the normalized nuclear norm-induced distance from $\HH$ to the set of positive semidefinite matrices $\mathcal{M}^+$: $$\nconv(\obj) \defeq \inf_{\mathbf{M} \in \mathcal{M}^+} \frac{\|\HH - \mathbf{M}\|_*}{\|\HH\|_*} = \frac{\lambda^-}{\lambda^+ + \lambda^-} \in [0, 1],$$ where $\|\cdot\|_*$ denotes the nuclear norm and $\lambda^-$ and $\lambda^+$ are the total magnitudes of respectively negative and positive eigenvalues of $\HH$ \cite{davydov2019searching}. In particular, this distance is equal to 0 (or 1) if and only if the function is convex (or concave).

Notice that $\lambda^+ - \lambda^- = \tr\HH = 8(n+m)^2$ due to the zero main diagonals of $\VV$ and $\UU$ and
\begin{align*}
    \tr\left(\left(c^\WW \otimes c^{-\WW^T}\right)\KK\right) = \left\langle \KK,  c^\WW \otimes c^{-\WW^T} \right\rangle 
    = \sum_{i,j=1}^{n+m} c^{W_{ij}}c^{-W_{ij}}
    = (n+m)^2
    = \tr\left(\left(c^{-\WW} \otimes c^{\WW^T}\right)\KK\right),
\end{align*}
and therefore $$\nconv(\obj) = \frac{\lambda^+ - 8(n+m)^2}{2\lambda^+ - 8(n+m)^2}\in \left[0, \frac{1}{2}\right).$$
%  = 1 - \frac{1}{2 - \frac{8(n+m)^2}{\lambda^+}} 
Let $\lmax > 0$ denote the dominant eigenvalue of $\HH$ as per the Perron--Frobenius theorem. In the degenerate case of $c=1$ this is the only non-zero eigenvalue of the rank-1 Hessian $\HH=8\left(\mathbf{1}\mathbf{1}^T\right)$, which makes $\obj$ convex (though uninformative). %so $\lambda^+ = 8(n+m)^2=\lmax$.
%the problem becomes convex because its rank-$1$ Hessian with all entries equaling 8 rhas $\lambda^-=0$ and $\lambda^+=8(n+m)^2$.
On the other hand, the Courant--Fischer theorem yields $$\lmax \geq \frac{\mathbf{1}^T\HH\mathbf{1}}{\mathbf{1}^T\mathbf{1}} = \frac{\|\HH\|_1}{(n+m)^2} \xrightarrow[c \to \infty]{} \infty,$$
and
$\lambda^+ \geq \lmax$
then implies
$\nconv(\obj) \xrightarrow[c \to \infty]{} \frac{1}{2}.$
%$\lambda^+ \geq \lmax \geq \frac{\|\HH\|_1}{(n+m)^2} \to \infty$ as $c \to \infty$, and thus $\lim_{c \to \infty} \nconv = \frac{1}{2}$.
Therefore, the landscape of (\ref{eqn:continuous_optimization}) starts flat at $c=1$ and becomes increasingly non-convex as $c$ grows. The following result provides a tractable (specifically, $O(n^2)$-time and -space) bound on its non-convexity based on the value of $c$.
%to $(n^2-n)^{\frac{1}{\epsilon}}$, the largest value of interest in the context of finding the GH distance.


\begin{restatable}{theorem}{nonconvexity}
    \label{thm:nonconvexity}
    Let $\alpha \in \left[0, \frac{1}{2}\right)$ and $c \geq 1$ satisfy 
    $$\frac{2\alpha}{\frac{1}{2}-\alpha} = \frac{\sqrt{16(n+m)^4 + \left(c^{\dmax} + c^{-\dmax} - 2\right)p_{\max} - \frac{16}{(n+m)^4}\left\|c^{\WW}\right\|_1^2\left\|c^{-\WW}\right\|_1^2}}{n+m},$$
    %$$1-2\alpha = \frac{n+m}{\sqrt{16(n+m)^4 + !!p_{\max}(2\cosh(\dmax \ln c) - 2) - \frac{16}{(n+m)^2}\|c^\WW\|_1^2\|c^{-\WW}\|_1^2} + n+m},$$
    where $\dmax \defeq \max\{\diam X, \diam Y\}$ % = \|\WW\|_\infty$
    and $p_{\max} \defeq \left(\frac{\left(2\sqrt{2} + 4\right)(n+m)^2\sqrt{(n+m)^2\|\WW\|_2^2 - \|\WW\|_1^2}}{\dmax}+6\right)$. Then $\nconv(\obj) \leq \alpha$.% \quad \forall c \in [1, c_*].$$
\end{restatable}

% Negative eigenvalues correspond to dimensions in which $u$ is concave, so everywhere is a saddle point. So that 

% \subsection{Warm starts}
% The Frank--Wolfe algorithm converges to a stationary point that is fully determined by its starting point $\SS_0$. This induces a view of $\SSS$ as partitioned into the \textit{basins of attraction} %(see e.g. \cite{traonmilin2020basins, asenjo2013visualizing, tsang2018basin})
% of the stationary points of $\obj$. It follows that the chance of recovering a (global) solution to (\ref{eqn:continuous_optimization}) from a randomly sampled $\SS_0 \in \SSS$ is proportional to the total volume of the solution set's basins of attraction. Intuitively, this volume is expected to decrease as the value of $c$ and therefore $\nconv(\obj)$ increases. This presents a trade-off between the probability of a stationary point recovered by the Frank--Wolfe algorithm being a solution to (\ref{eqn:continuous_optimization}) and the likelihood of such a solution delivering the GH distance. To address this issue, we propose solving a sequence of (\ref{eqn:continuous_optimization}) for the increasing values $c_1 < c_2 < \ldots$, using each approximate solution produced by the Frank--Wolfe algorithm as a warm start in the subsequent minimization. The idea is illustrated !!!!
% This approach is based on the assumption that for a sufficiently small ratio $\frac{c_{k+1}}{c_k}$ a solution to (\ref{eqn:continuous_optimization}) for $c_k$ is more likely to be in the basin of attraction of some solution for $c_{k+1}$ than a randomly chosen point in $\SSS$. We compare its efficiency against solving the minimization for a single value of $c$ in the following section.


% Warm starts. Want upper bound of $c$ as a function of convexity.
% Notice that lambda- is bounded by ...sum lambda squared - lambda max squared $\|\HH\|_2^2 - \|\HH\|_1^2$. (*)
% It follows then that $conv = 1$ for $c = 1$ and $conv —> \frac{1}{2}$ for $c \to \inf$. 

% One way to do it is to use (*), but this will require $O(n^4)$ enumerations of the Hessian. Instead, we rely on the following result allowing for constructing an upper bound for $c$ in $O(n^2)$.

% We can express the relationship between $c$ and the degree of convexity as $$$$
% This allows for deriving $c$ for the desired degree of convexity of the landscape, and therefore for constructing a progression of minimization problems each using the previous output as a starting point. Additional bounds on $\lambda_{\max}$ and $\sum{\lambda_i^2}$ to avoid $O(n^4)$ enumerations of the Hessian are discussed in next section.


% $$\lambda_{\max} \geq \frac{8}{(n+m)^2}\sum c**\mathbf{C}\sum c**-\mathbf{C} \geq \frac{8}{(n+m)^2}$$ (the last transition is by Karamata/Jensen inequality: $\sum_1^n c**s_i \geq n\sum c**\frac{s_i}{n}$)

% Remark [IN THE PROOF!]: a tighter $O(n^2)$-time bound can be given for $\lambda_{\max}$ as $\sum\HH = 8\sumc^\mathbf{C}\sumc^{-\mathbf{C}}$. However, this would prevent from analytically solving for the upper bound of $c$ as a function of desired convexity, as the resulting inequality will be a polynomial of $O(n^4)$ degree. It would still be possible to solve using e.g. Newthon's method.

% SAME AS ABOVE REMARK. If using first LB for $\lambda_{\max}$, Newthon's method because of many power of $c$ in the equation. If using second LB, can solve directly.
\section{Numerical experiments}
\label{numerical}
We demonstrate the method's performance using our implementation from a Python package \texttt{dgh}.
Given a choice of $c$ and a budget of Frank--Wolfe iterations, it starts from a random $\SS_0 \in \SSS$ and iterates until the Frank--Wolfe gap is at most $10^{-8}$ (or there are no iterations left), after which the last $\SS_i$ is projected to the nearest $\RR$. The cycle repeats until the budget is depleted, after which the smallest found $\frac{1}{2}\dis \RR$ is returned as an estimate and upper bound of the Gromov--Hausdoff distance (we note that this approach is easily parallelizable as there is no interdependency between the random restarts).
In a bid to save on redundant computations, \texttt{dgh} also compares every new best $\frac{1}{2}\dis \RR$ against the trivial lower bound $$\dGH(X, Y) \geq \frac{1}{2}\max\left\{\left|\diam X - \diam Y\right|, \left|\rad X - \rad Y\right|\right\}\quad\cite{memoli2012some}$$ and terminates whenever they match, which means that the Gromov--Hausdorff distance was found exactly. Prior to the computations, \texttt{dgh} normalizes all distances in the input metric spaces so that $\max\{\diam X, \diam Y\} = 1$ to avoid floating-point arithmetic overflow, and scales the resulting $\dis\RR$ back afterwards. %Every time a smaller $\dis\RR$ is obtained, \texttt{dgh} compares it to the trivial lower bound of the Gromov--Hausdorff distance $$\dGH(X, Y) \geq \max\left\{\right\}$$ and terminates if they match
%--- note that this is equivalent to solving (\ref{eqn:continuous_optimization}) for $c^{\dmax}$ and does not affect the optimization landscape for the GH distance.

In the following, we describe our numerical experiments on synthetic and model geometric spaces. All computations were performed on a standard 2016 Intel i7-7500U processor.

% We implement our approach to computing the GH distance as a Python package \texttt{dgh}. Given $c > 1$ and the budget of Frank--Wolfe iterations, our implementation solves a sequence of (\ref{eqn:continuous_optimization}) for $c_1 = c$ and $c_{k+1} = 1 + 10(c_k - 1)$ using the Frank--Wolfe algorithm with warm starts. The starting point for the first minimization in a sequence is selected from $\SSS$ uniformly at random. For every $c_k$, the minimization stops when the Frank--Wolfe gap gets below $10^{-8}$ or when no iterations are left in the budget. Once $c_{k+1}$ exceeds $10^8$, the (approximate) solution for $c_k$ is stored and the sequence is restarted until the iteration budget is depleted. The final solution is selected for the smallest distortion of its projection onto $\RRR$ which delivers an upper bound of the GH distance. All distances in the input metric spaces are scaled by $\dmax$ (so that $\max\{\diam X, \diam Y\} = 1$) prior to the computations to avoid floating-point arithmetic overflow --- note that this is equivalent to solving (\ref{eqn:continuous_optimization}) for $c^{\dmax}$ and does not affect the optimization landscape for the GH distance.

% Next, we describe numerical experiments on synthetic, real-world, and theoretical metric spaces demonstrating the performance and applicability of our approach.

\subsection{Benchmarking on synthetic spaces}

We evaluate the speed and accuracy of \texttt{dgh} on synthetic point clouds and graphs by computing the Gromov--Hausdorff distance from each space to its isometric copy. A point cloud is generated by uniformly sampling $n=200$ points from the unit cube in $\R^3$ and taking the Euclidean distance between them. A graph is generated according to the Erd\H{o}s-R\'{e}nyi model with $n=200$ vertices and the edge probability of $p=0.05$ until it is connected, and then endowed with the shortest path metric. We generate 100 point clouds and 100 graphs, and run an experiment on each metric space for $c=1+10^{-4},1+10^{-3},\ldots,1+10^8$ and with the budgets of 100 and 1000 iterations. For each metric space type, we show the percent of experiments where the (zero) Gromov--Hausdorff distance was found exactly and the average time taken by \texttt{dgh} on Figure \ref{fig:performance}.
%, and another 100 graphs for $p=0.1$.
%We provide each experiment with the budgets of 100 and 1000 iterations and replicate it for every $c=1+10^{-4},1+10^{-3},\ldots,1+10^8$.

\begin{table*}[tbp]
\centering
\small
\begin{tabular}{cccccccccc}
\toprule
& \multicolumn{3}{c}{\msr} & \multicolumn{3}{c}{\negc} & \multicolumn{3}{c}{\wsj} \\
& Acc. & F1 & wF1 & Acc. & F1 & wF1 & Acc. & F1 & wF1 \\ \cmidrule(lr){2-4} \cmidrule(lr){5-7} \cmidrule(lr){8-10} 
\udel & 66.86 & 56.76 & 64.3 & \textbf{80.80} & 55.45 & 77.9 & 63.74 & 64.23 & 63.2 \\
\icsi & \underline{71.19} & 64.73 & 70.4 & 80.36 & 64.53 & \underline{78.6} & 64.62 & 64.15 & 63.4 \\
\cnts & 68.59 & 61.39 & 67.2 & 78.68 & 61.62 & 76.8 & 64.31 & 64.59 & 64.4 \\
\osu & 68.02 & 60.28 & 66.6 & 79.24 & 57.04 & 76.5 & 69.20 & 69.63 & 68.9 \\
\isg & 67.05 & 58.83 & 65.3 & 77.34 & 59.52 & 75.6 & 69.15 & 69.35 & 69.2 \\ \midrule
\bert & \textbf{71.68} & \underline{66.70} & \textbf{71.4} & 77.79 & \underline{72.87} & 77.7 & \underline{80.95} & \underline{80.93} & \underline{80.9} \\
\roberta & 70.91 & \textbf{67.53} & \underline{70.7} & \textbf{80.80} & \textbf{77.29} & \textbf{80.7} & \textbf{82.61} & \textbf{82.70} & \textbf{82.6} \\ \midrule
Average & 69.19 & 62.32 & 67.99 & 79.29 & 64.05 & 77.69 & 70.65 & 70.80 & 70.37 \\
\bottomrule
\end{tabular}
\caption{\label{tab:performance} Overall accuracy (Acc.), macro-averaged F1 (F1), and weighted-macro F1 (wF1) scores of the algorithms depicted in Section~\ref{sec:algorithm}. For instance, \msr-\udel refers to a C5.0 classifier trained on the \msr~corpus, using the feature set mentioned in \citet{greenbacker-mccoy-2009-udel}.}
%Its Acc., F1 and wF1 of this model are 66.86, 56.76, and 64.3, respectively.}
\end{table*}


Interestingly, the method behaves very differently on the two metric space types. While the accuracy on graphs drops drastically as the value of $c$ grows, this trend is somewhat reversed for point clouds. We hypothesize that the large distortion gap $\rho=1$ in graphs relative to their $O(\log{n})$ diameter %($\rho=1$ due to their integer distances)
allows the solutions to be preserved by (\ref{eqn:continuous_optimization}) even for small $c$,  %(see Theorem \ref{thm:c_threshold})
whose near-convex optimization landscape enables their efficient recovery.
While the non-convexity of the optimization landscape growing together with $c$ is expected to drive the accuracy down for any metric space, this phenomenon may be outweighed by the solution-preserving effects of larger values of $c$ on point clouds.

% At the same time, the non-convexity of the optimization landscape may be a factor behind the worsened accuracy on graphs for larger values of $c$. An analogous dynamic for point clouds may be obscured

% Every experiment is replicated for the values of $c$ in $\{1+10^p: p = -4, \ldots, 8\}$ and the budgets of 100 and 1000 iterations.
% %provided with a budget of 100 and, separately, 1000 iterations every time.
% Figure \ref{fig:performance} aggregates the resulting performance measurements across the 100 spaces for each experimental setup.

% for each metric space type and the value of $c$.

% space c exact dGH time restarts


% For each collection of metric spaces $X_1, \ldots, X_{100}$, we compute (an upper bound of) the GH distance between $X_i$ and $\pi(X_i)$ for some isometry $\pi$ for every $i=1,\ldots,100$. In addition, we compare the warm start approach implemented in \texttt{dgh} to the vanilla Frank--Wolfe algorithm for a fixed value of $c$ with the same iteration budget. Similarly to the warm start approach, the algorithm restarts from a randomly sampled point after converging and the best solution is selected after the iteration budget is depleted. Both warm start and vanilla approaches are tried for different values of $c \in \{1 + 10^p: p = -2, -3, \ldots, 7\}$ and given the same iteration budget of 100 iterations every time. !!!! Performance metrics

% Beside the warm start approach implemented in \texttt{dgh}, we additionally 

% \subsection{Classification on biological data}
\subsection{Bounding the distance between model spaces}
\label{spheres}
% Recall that any compact metric space $X$ admits a finite approximation of arbitrary precision $\varepsilon$ in the form of its epsilon-net
Recall that a compact metric space $X$ contains its finite $\varepsilon$-net $X_\varepsilon$ for any $\varepsilon > 0$. Because $$\left|\dGH(X,Y) - \dGH(X_\varepsilon,Y_\varepsilon)\right| \leq \epsilon,\quad\cite{oles2022lipschitz}$$ such a finite approximation enables numerical estimation of the Gromov--Hausdorff distance between infinite $X,Y$ to an arbitrary precision. This can be particularly useful for estimating the distance between model metric spaces. We demonstrate it by refining an upper bound of the Gromov--hausdorff distance between the unit circle $S^1 \in \R^2$ and the upper hemisphere $H^2 \subset \R^3$ of the unit sphere, established to be strictly below $\frac{\sqrt{3}}{2}$ in \cite{lim2021gromov}.

To generate an $\varepsilon$-net of $H^2$, we use a slight modification of the regular construction described in \cite{deserno2004generate}.
%on the upper hemisphere $\left\{(\theta, \phi): \theta \in [0, \frac{\pi}{2}], \phi \in [0, 2\pi]\right\}$.
Given some small $\delta$, we consider evenly spaced polar angles $\theta_i$ covering the hemisphere range $\left[0, \frac{\pi}{2}\right]$ so that the geodesic distance from any $\mathbf{x} = (\theta, \phi) \in H^2$ to the nearest $\yy = (\theta_i, \phi\}$ is $|\theta - \theta_i| \leq \frac{\delta}{2}$. The law of cosines then bounds the Eucledian distance between $\xx$ and $\yy$ by
\begin{align*}
    \|\xx - \yy\|^2 &= 2 - 2\cos(\theta - \theta_i)\\
    &\leq 2\left(1-\cos\frac{\delta}{2}\right).
\end{align*}
%$\forall \theta \in [0, \frac{\pi}{2}] \quad \min_i |\theta - \theta_i| \leq \frac{\delta}{2}$.
For each $\theta_i$, we choose evenly spaced azimuthal angles $\phi_j(\theta_i)$ so that %$\forall \phi \in [0, 2\pi] \quad \min_j |\phi - \theta_i| \leq \frac{\delta}{2}$
the geodesic distance between any $\mathbf{y} = (\theta_i, \phi)$ and the nearest $\mathbf{z} = \left(\theta_i, \phi_j(\theta_i)\right)$ is at most $\frac{\delta}{2}$. Because $\yy$ and $\zz$ are on a circle of radius $\sin\theta_i$, this implies $\left|\phi - \phi_j(\theta_i)\right| \leq \frac{\delta}{2\sin\theta_i}$ and therefore 
\begin{align*}
    \|\yy - \zz\|^2 &= 2\sin^2\theta_i - 2\sin^2\theta_i \cos\left(\phi - \phi_j(\theta_i)\right) \\
    &\leq 2\sin^2\theta_i\left(1-\cos\frac{\delta}{2\sin\theta_i}\right).
\end{align*}
%, and place a point at each $(\theta_i, \phi_j(\theta_i))$ to construct the $\varepsilon$-net $H^2_\varepsilon$.
The $\varepsilon$-net $H^2_\varepsilon$ is comprised of the points at $\left(\theta_i, \phi_j(\theta_i)\right)$ for every $i,j$ pair. Its covering radius $\varepsilon$ can be bounded using
\begin{align*}
    \left|\cos\angle\xx\yy\zz\right| &= \left| \langle \xx - \yy, \yy - \zz\rangle\right| \\
    &= \sin\theta_i\left|\sin\theta_i - \sin\theta\right|\left(\cos\left(\phi - \phi_j(\theta_i)\right)\right) \\
    &\leq \sin\theta_i\left(\sin\theta_i\left(1-\cos\frac{\delta}{2}\right)+\cos\theta_i\sin\frac{\delta}{2}\right)\left(1 - \cos\frac{\delta}{2\sin\theta_i}\right),
\end{align*}
which entails
\begin{align*}
    \varepsilon^2 &\leq \sup_{\xx}\|\xx-\zz\|^2 \\
    &\leq \sup_{\xx} \left(\|\xx - \yy\|^2 + \|\yy - \zz\|^2 + 2\|\xx - \yy\|\|\yy - \zz\|\left|\cos{\angle \xx\yy\zz}\right|\right) \\
    &\leq 2\left(1-\cos\frac{\delta}{2}\right) + 2 \max_i \sin^2\theta_i \left(1 - \cos\frac{\delta}{2\sin\theta_i}\right)\Bigg[1 + \\&\hspace{2cm}2\sqrt{1-\cos\frac{\delta}{2}}\sqrt{1 - \cos\frac{\delta}{2\sin\theta_i}}\left(\sin\theta_i\left(1-\cos\frac{\delta}{2}\right)+\cos\theta_i\sin\frac{\delta}{2}\right)\Bigg].
\end{align*}

By letting $\delta=\frac{\sqrt{2\pi}}{15}$ in the above, we construct $H^2_\varepsilon$ of 168 points and $\varepsilon \approx 0.1385$. We match this covering radius on $S^1$ by constructing its $\varepsilon$-net $S^1_\varepsilon$ as a regular lattice of 23 points. Setting $c=10^7$ and running \texttt{dgh} with a permissive iteration budget for about 16 minutes yields a mapping pair delivering $\dGH(S^1_\varepsilon, H^2_\varepsilon) < 0.6913$, shown in Figure \ref{fig:spheres}. It follows that $$\dGH\left(S^1, H^2\right) \leq \dGH\left(S^1_\varepsilon, H^2_\varepsilon\right) + \varepsilon < 0.8298,$$ a refinement over $\frac{\sqrt{3}}{2} \approx 0.8660.$ We note that this bound holds for both the open and closed hemispheres as well as for the ``helmet'' of $S^2$ --- its upper hemisphere that contains $(0, \phi) \in S^2$ if and only if $\phi \in \left[0, \pi\right)$, a construction from \cite{lim2021gromov} facilitating antipode-preserving mappings.

\input{figures/spheres}
\section*{Acknowledgement}
I am grateful to Oleksandr Dykhovychniy, Kostiantyn Lyman, and Kevin R. Vixie for the numerous insightful conversations that helped shape this paper. I would also like to acknowledge the convenience of an online tool for computing matrix derivatives  \cite{laue2018computing} used in this study.

% \appendix
\begin{appendices}
\section{Proofs}
\label{proofs}
\cthreshold*
\begin{proof}
    Recall that $\VV - \RR\UU\RR^T + \RR\WW - \WW^T\RR^T$ is by construction symmetric with zero main diagonal, and $\dis \RR$ is the largest magnitude among its entries. Let     $\RR, \RR' \in \RRR$ be such that $\dis \RR > \dis \RR'$ and therefore $\dis \RR \geq \dis\RR' + \rho$.
    % Let $R, R' \subseteq X \times Y$ be mapping pair-induced relations s.t. $\dis R > \dis R'$ with their matrix representations denoted as $\RR, \RR' \in \RRR$, respectively. By construction, $\VV - \RR\UU\RR^T + \RR\WW - \WW^T\RR^T$ is symmetric, has zero main diagonal, and only contains entries from $[0, \dis R]$ with at least one entry equal to $\dis R$.
    Because $a \mapsto c^a + c^{-a}$ is convex and attains its minimum at $a = 0$,
    \begin{align*}
        \left\|c^{\VV - \RR\UU\RR^T + \RR\WW - \WW^T\RR^T} + c^{\RR\UU\RR^T - \VV + \WW^T\RR^T - \RR\WW}\right\|_1 &\geq 2\left(c^{\dis \RR} + c^{-\dis \RR} + (n+m)^2-2\right) \\
        &> 2\left(c^{\dis \RR} + (n+m)^2 - 2\right).
    \end{align*}
    At the same time,
    \begin{align*}
        \hspace{-.5cm}\left\|c^{\VV - \RR'\UU\RR'^T + \RR'\WW - \WW^T\RR'^T} + c^{\RR'\UU\RR'^T - \VV + \WW^T\RR'^T - \RR'\WW}\right\|_1 &\leq \left((n+m)^2 - n - m\right)\left(c^{\dis \RR'} + c^{-\dis \RR'}\right) + 2(n+m) \\ &< \left((n+m)^2-n-m\right)c^{\dis \RR'} + (n+m)^2+n+m.
    \end{align*}
    % By the definition of distortion gap, $\dis R' \leq \dis R - \epsilon$ and therefore
    Then
    \begin{align*}
        &\left\|c^{\VV - \RR\UU\RR^T + \RR\WW - \WW^T\RR^T} + c^{\RR\UU\RR^T - \VV + \WW^T\RR^T - \RR\WW}\right\|_1 - \left\|c^{\VV - \RR'\UU\RR'^T + \RR'\WW - \WW^T\RR'^T} + c^{\RR'\UU\RR'^T - \VV + \WW^T\RR'^T - \RR'\WW}\right\|_1 \\
        &\quad> 2c^{\dis \RR} - \left((n+m)^2-n-m\right)c^{\dis \RR'} + (n+m)^2-n-m-4 \\
        &\quad\geq \left(2c^\rho - (n+m)^2-n-m\right)c^{\dis \RR'}  + (n+m)^2-n-m-4 \\
        &\quad\geq (n+m)^2-n-m-4 \\
        &\quad> 0.
    \end{align*}
    It follows that the 1-norm relaxation $\left\|c^{\VV - \RR\UU\RR^T + \RR\WW - \WW^T\RR^T} + c^{\RR\UU\RR^T - \VV + \WW^T\RR^T - \RR\WW}\right\|_1$ monotonically decreases with $\dis \RR$.
\end{proof}

% Proof of Theorem \ref{thm:convex_hull}:
% \begin{proof}
%     A generalization of the Birkhoff-von Neumann theorem for row-stochastic matrices states than the set of row-stochastic matrices is the convex hull of the set of binary row-stochastic matrices, see e.g. \cite{gubin2008subgraph} or \cite{cao2022centrosymmetric}. Trivially then, the Cartesian product of the $n\times m$ and $m \times n$ row-stochastic matrices is the convex hull of the Cartesian product of their binary counterparts. Replacing every matrix pair with their direct sum in both sets concludes the proof.
% \end{proof}

\polytopegeometry*
\begin{proof}
    The bi-mapping polytope $\SSS$ is defined by the $2nm$ inequalities of the form $S_{ij} \geq 0$, $n^2+m^2$ equality constraints of the form $S_{ij} = 0$, and the requirement of unit row sums $\SS\bf{1} = \bf{1}$. In particular, the facets of $\SSS$ lie in the hyperplanes $S_{ij} = 0$ for $(i,j) \in \left(\{1,\ldots,n\}\times \{1,\ldots,m\}\right) \cup \left(\{n+1,\ldots,n+m\}\times \{m+1,\ldots,m+n\}\right)$. Because every face is the intersection of a set of facets, $\Phi$ is given by the collective indices of zero entries describing the corresponding hyperplanes.
\end{proof}

\solutiononinterior*
\begin{proof}
The statement trivially holds for $\SS^* \in \RRR$ (recall that a vertex is its own interior). Otherwise, $\exists h,k,l$ s.t. $0 < S^*_{hk}, S^*_{hl} < 1$, which means that $\SS^*$ lies on the open segment whose endpoints $\SS', \SS''$ are given by $S'_{ij} = \begin{cases}
    S^*_{hk} + S^*_{hl} & \text{if $(i,j) = (h,k)$} \\
    0 & \text{if $(i,j) = (h,l)$}\\
    S^*_{ij} & \text{otherwise}
\end{cases}$, $S''_{ij} =
\begin{cases}
    0 & \text{if $(i,j) = (h,k)$} \\
    S^*_{hk} + S^*_{hl} & \text{if $(i,j) = (h,l)$}\\
    S^*_{ij} & \text{otherwise}
\end{cases}$. By applying the trace trick, we derive the following identity:
\begin{align*}
    \left\langle \SS', \grad \obj(\SS'') \right\rangle &= 2\left\langle \SS', c^\VV\SS'' c^{-\UU} + c^{-\VV}\SS''c^\UU + \big(c^\WW\SS'' c^{-\WW} + c^{-\WW}\SS'' c^\WW\big)^T \right\rangle \\
    &= 2\left\langle \SS'', c^\VV\SS' c^{-\UU} + c^{-\VV}\SS'c^\UU + \big(c^{-\WW}\SS' c^{\WW} + c^{\WW}\SS' c^{-\WW}\big)^T \right\rangle \\
    &= \left\langle \SS'', \grad \obj(\SS') \right\rangle.
\end{align*}
Denoting $\alpha \defeq \frac{S^*_{hk}}{S^*_{hk} + S^*_{hl}} \in (0, 1)$, we get
\begin{align*}
    \obj(\SS^*) &= \obj\left(\alpha \SS' + (1-\alpha)\SS''\right) \\
    &= \Big\langle \alpha \SS' + (1-\alpha)\SS'', \frac{1}{2}\grad\obj\left(\alpha \SS' + (1-\alpha)\SS''\right)\Big\rangle \\
    &= \alpha^2\obj(\SS') + \alpha(1-\alpha)\left\langle \SS'', \grad \obj(\SS') \right\rangle + (1-\alpha)^2\obj(\SS'') \\
    &\leq (1 - 2\alpha(1-\alpha)\obj(\SS^*) + \alpha(1-\alpha)\left\langle \SS'', \grad \obj(\SS') \right\rangle,
\end{align*} which implies $$\left\langle \SS'', \grad \obj(\SS') \right\rangle \leq 2 \obj(\SS^*) \leq \left\langle \SS', \grad \obj(\SS')\right\rangle$$ and therefore
\begin{equation*}
    \grad\obj(\SS')_{hl} \leq \grad\obj(\SS')_{hk}. \tag{1}
\end{equation*}
By the analogous reasoning using $\left\langle \SS', \grad \obj(\SS'') \right\rangle$ in place of $\left\langle \SS'', \grad \obj(\SS') \right\rangle$,
\begin{equation*}
    \grad\obj(\SS'')_{hk} \leq \grad\obj(\SS'')_{hl}.    \tag{2}
\end{equation*}
At the same time, %$\SS' - \SS'' = \begin{bmatrix}    \ddots & &  \end{bmatrix}$

\begin{align*}
    \grad\obj(\SS')_{ij} - \grad\obj(\SS'')_{ij} &= \grad\obj(\SS' - \SS'')_{ij} \\
    &= 2\big(c^\VV(\SS'-\SS'')c^{-\UU} + c^{-\VV}(\SS'-\SS'')c^\UU\big)_{ij} \\
    &\hspace{2.5cm}+ \big(c^\WW(\SS'-\SS'')c^{-\WW} + c^{-\WW}(\SS'-\SS'')c^\WW\big)_{ji} \\
    &= 2\begin{aligned}[t](S^*_{hk} + S^*_{hl})&\big(c^{V_{ih}}(c^{-U_{kj}} - c^{-U_{lj}}) + c^{-V_{ih}}(c^{U_{kj}} - c^{U_{lj}}) \\ &+ c^{W_{jh}}(c^{-W_{ki}} - c^{-W_{li}}) + c^{-W_{jh}}(c^{W_{ki}} - c^{W_{li}})\big),
    \end{aligned}
\end{align*}
and in particular
\begin{equation*}
\begin{aligned}
    \grad\obj(\SS')_{hk} - \grad\obj(\SS'')_{hk} &= 2(S^*_{hk} + S^*_{hl})\big(2 - c^{-U_{lk}} - c^{U_{lk}} + 2 - c^{W_{kh}}c^{-W_{lh}} - c^{-W_{kh}}c^{W_{lh}}\big) \\ &\leq 0,
\end{aligned} \tag{3}
\end{equation*}
\begin{equation*}
\begin{aligned}
        \grad\obj(\SS')_{hl} - \grad\obj(\SS'')_{hl} &= 2(S^*_{hk} + S^*_{hl})\big(c^{-U_{kl}} + c^{U_{kl}} - 2 + c^{W_{lh}}c^{-W_{kh}} + c^{-W_{lh}}c^{W_{kh}} - 2\big) \\ &\geq 0. 
\end{aligned}\tag{4}
\end{equation*}
Combining inequalities (1), (2), (3), and (4) gives $\grad\obj(\SS')_{hk} = \grad\obj(\SS')_{hl}$. Then $$\left\langle \SS'', \grad \obj(\SS')\right\rangle = \left\langle \SS', \grad \obj(\SS') \right\rangle = 2 \obj(\SS^*)$$ and therefore $\SS' \in \argmin_{\SS \in \SSS}\obj(\SS)$.

We showed that for any $(h, k)$ s.t. $0 < S^*_{hk} < 1$ there exists a solution $\SS' \in \SSS$ s.t. $S'_{ij} = 0$ whenever $S^*_{ij}=0$ or $(i,j)=(h,k)$. By Lemma \ref{lem:polytope_geometry}, it means that every facet of $\Phi$ contains a solution on its interior. Recursively repeating this argument establishes the existence of a solution in the interior of every non-empty face of $\Phi$, which also implies that every vertex of $\Phi$ is a solution.

Since $\obj$ is quadratic, attaining the same value at any three points on the same line renders the function constant on the entire line. In particular, any face $\Psi$ is included in $\argmin_{\SS\in\SSS}\obj(\SS)$ provided that $$\partial\Psi \cup \{\SS^\mathrm{o}\} \subseteq \argmin_{\SS\in\SSS}\obj(\SS)$$ for some interior point $\SS^\mathrm{o} \in \Psi\setminus\partial\Psi$. It follows that any positive-dimension face of $\Phi$ must be a part of the solution set if all the faces of $\Phi$ of dimension smaller by 1 are. Starting from the vertices of $\Phi$ and applying induction then yields $\Phi \subseteq \argmin_{\SS\in\SSS}\obj(\SS)$.
\end{proof}

\voronoiface*
\begin{proof}
    Note that $$\|\SS - \RR\|_2^2 = \|\SS\|_2^2 - 2\langle \SS, \RR \rangle + n+m\quad\forall\RR \in \RRR$$ and therefore $$\proj_\RRR \SS \in \argmin_{\RR \in \RRR}\,\|\SS - \RR\|_2 = \argmax_{\RR \in \RRR} \;\langle \SS, \RR \rangle.$$
    Let $s_i$ denote the maximum entry in the $i$-th row of $\SS$ for $i=1,\ldots,n+m$. Then $$\max_{\RR \in \RRR} \;\langle \SS, \RR \rangle = \sum_{i=1}^n s_i,$$ which means that any $\RR^* \in \argmax_{\RR \in \RRR} \;\langle \SS, \RR \rangle$ can have $R^*_{ij} \neq 0$ only if $S_{ij} = s_i$. Because $\SS$ is a convex combination of the vertices of $\Phi$, $S_{ij} = s_i > 0$ in turn implies the existence of $\RR \in \Phi \cap \RRR$ s.t. $R_{ij} \neq 0$. By Lemma \ref{lem:polytope_geometry}, the index of every non-zero entry of $\RR^*$ is not contained in the index set of zero entries characterizing $\Phi$, and therefore $\RR^* \in \Phi$.
\end{proof}

% \guarantees*
% \begin{proof}
%     By ..., the solution set of $\min_{\SS\in\SSS}\obj(\SS)$ is closed under inclusion 
%     By the corollary to Theorem \ref{thm:c_threshold}, $$\argmin_{\RR\in\RRR} \obj(\RRR) \subseteq \argmin_{\RR\in\RRR}\dis\RR.$$ It remains
% \end{proof}
%     \label{thm:guarantees}
%     Let $\proj_\RRR: \SSS \to \RRR$ denote the projection of the bi-mapping polytope onto the set of its vertices. If $c \geq \left(\frac{(n+m)^2-n-m}{2}\right)^{1/\rho}$, then $$\proj_\RRR \left(\argmin_{\SS\in\SSS} \obj(\SSS)\right) \subseteq \argmin_{\RR\in\RRR}\dis\RR.$$

% Proof of Theorem \ref{thm:time_complexity}:
% \begin{proof}
%     The time complexity of a single Frank--Wolfe iteration is defined by the calculations of gradient at the current point, the direction of smallest correlation with it, and the step size to take along said direction. Computing the gradient $\grad \obj(\SS_i)$ is comprised of multiplying and summing $(n+m)\times(n+m)$ matrices which has the time complexity of $O(n^3)$. Finding the descent direction $\DD_i$ amounts to solving $\min_{\RR\in\RRR}\langle \RR, \grad\obj(\SS_i) \rangle$, which can be done in $O(n^2)$ time by locating the smallest entry in each row of $\grad\obj(\SS_i)$. Finally, the quadratic minimization step $\min_{\gamma \in [0, 1]} \obj(\gamma\SS_i + (1-\gamma)\DD_i)$ admits a closed-form solution that requires $O(n^3)$-time computations including of $\ob2j(\DD_i)$.
% \end{proof}

\oneminimizer*
\begin{proof}
    $\left|\argmin_{\SS\in\SSS} \left\langle \SS, \grad\obj(\SS^*) \right\rangle\right| = 1$ if and only if the smallest entry in each row of $\grad\obj(\SS^*)$ is unique.  In order for $\grad\obj(\SS^*)_{hk} = \grad\obj(\SS^*)_{hl}$ to hold for some $h$ and $k\neq l$, the realizations $d_1, \ldots, d_\frac{n(n-1)}{2}$ of the distances in $X$ must satisfy
    $$\sum_{i=1}^{n+m}\sum_{j=1}^{n+m} \left(c^{U_{hi}}S_{ij}c^{-V_{jk}} + c^{-U_{hi}}S_{ij}c^{V_{jk}} +c^{-W_{kj}}S_{ji}c^{W_{ih}} + c^{W_{kj}}S_{ji}c^{-W_{ih}}\right) = 0,$$
    which can be rewritten as $$\sum_{i=0}^\frac{n(n-1)}{2}\sum_{j=0}^\frac{n(n-1)}{2}a_{ij}c^{d_i - d_j} = 0$$ for some $a_{ij} \in \R$ and $d_0 \defeq 0$. Because the left-hand side can be cast as a generalized polynomial through a change of variables, it must have a finite number of solutions. The probability of the distances in $X$ to form a particular solution $d_1^*, \ldots, d_\frac{n(n-1)}{2}^*$ is
    \begin{align*}
        &\mathbb{P}_\XX\left[\mathrm{D}_1=d^*_1,\ldots,\mathrm{D}_\frac{n(n-1)}{2}=d_\frac{n(n-1)}{2}^*\right] \\ &\hspace{2cm} = \mathbb{P}_\XX\left[\mathrm{D}_1=d^*_1|\mathrm{D}_2=d_2^*,\ldots,\mathrm{D}_\frac{n(n-1)}{2}=d_\frac{n(n-1)}{2}^*\right]\mathbb{P}_\XX\left[\mathrm{D}_2=d_2^*,\ldots,\mathrm{D}_\frac{n(n-1)}{2}=d_\frac{n(n-1)}{2}^*\right] \\
        &\hspace{2cm} = 0\cdot\mathbb{P}_\XX\left[\mathrm{D}_2=d_2^*,\ldots,\mathrm{D}_\frac{n(n-1)}{2}=d_\frac{n(n-1)}{2}^*\right]
        \\
        &\hspace{2cm} = 0,
    \end{align*} and therefore
    $\mathbb{P}_\XX\left[\grad\obj(\SS^*)_{hk} = \grad\obj(\SS^*)_{hl}\right] = 0.$
    Then
    \begin{align*}
        \mathbb{P}_\XX\left[\Big|\argmin_{\SS\in\SSS} \langle \SS, \grad\obj(\SS^*) \rangle\Big| > 1\right] &\leq \sum_{h=1}^{n+m}\sum_{k=1}^{n+m}\sum_{l=k+1}^{n+m} \mathbb{P}_\XX\left[\grad\obj(\SS^*)_{hk} = \grad\obj(\SS^*)_{hl}\right] \\ &= 0.
    \end{align*}
\end{proof}

\nonconvexity*
\begin{proof}
    Let $\lmax = \lambda_1 \geq \ldots \geq \lambda_{(n+m)^2}$ denote the eigenvalues of $\HH$. We already established using the Courant-Fischer theorem that
    \begin{align*}
        \lambda_1 &\geq \frac{\|\HH\|_1}{(n+m)^2} \\
        &= \frac{2}{(n+m)^2}(\big\|c^\UU\big\|_1\big\|c^{-\VV}\big\|_1 + \big\|c^{-\UU}\big\|_1\big\|c^{\VV}\big\|_1 + \big\|c^\WW\big\|_1\big\|c^{-\WW^T}\big\|_1 + \big\|c^{-\WW}\big\|_1\big\|c^{\WW^T}\big\|_1) \\
        &= \frac{8}{(n+m)^2}\big\|c^\WW\big\|_1\big\|c^{-\WW}\big\|_1.
    \end{align*}
    To bound $\lambda_{-}\defeq\sum_{\lambda_i < 0}|\lambda_i|$ from above%the total magnitude of eigenvalues
    , we will first introduce two new notations for convenience. For any two matrices $\mathbf{A}$ and $\mathbf{B}$, let $\mathbf{A} \ominus \mathbf{B}$ denote a matrix operation analogous to the Kronecker product but with subtraction in place of multiplication. In addition, define $\ch:\R\to\R$ as $\ch(a) \defeq c^a + c^{-a}$ and its entrywise counterpart for matrices $\ch:\R^{p\times q}\to\R^{p\times q}$, so that we are able to compactly write
    \begin{align*}
    \frac{1}{2}\HH &= \ch(\UU\ominus\VV) + \ch(\WW\ominus\WW^T)\KK \\
    &= \ch(\UU\ominus\VV) + \ch\big((\WW\ominus\WW^T)\KK\big).
    \end{align*}
    Note that $\big\langle \ch(\mathbf{A}), \ch(\mathbf{B}) \big\rangle = \big\|\ch(\mathbf{A}+\mathbf{B})\big\|_1 + \big\|\ch(\mathbf{A}-\mathbf{B})\big\|_1$. Furthermore, if $\|\mathbf{A}\|_1$ and $\|\mathbf{A}\|_\infty$ are fixed, $\big\|\ch(\mathbf{A})\big\|_1$ is maximized by the highest possible count of entries of $\mathbf{A}$ equal to $\big\|\mathbf{A}\big\|_\infty$ due to the superadditivity of $\ch$ (see also a proof based on Lagrange multipliers under ``\href{https://math.stackexchange.com/questions/1355638/upper-bound-on-sum-of-exponential-functions}{upper bound on sum of exponential functions}'' on Mathematics Stack Exchange). As a consequence, for $\mathbf{A} \in \R^{p\times q}$
    \begin{align*}
        \big\|\ch(\mathbf{A})\big\|_1 &\leq \left\lceil\frac{\|\mathbf{A}\|_1}{\|\mathbf{A}\|_\infty}\right\rceil\ch\big(\|\mathbf{A}\|_\infty\big) + \left\lfloor pq - \frac{\|\mathbf{A}\|_1}{\|\mathbf{A}\|_\infty}\right\rfloor \ch(0)\\
        &\leq \left(\frac{\|\mathbf{A}\|_1}{\|\mathbf{A}\|_\infty} + 1\right)\left(\ch\big(\|\mathbf{A}\|_\infty\big) - 2\right) + 2pq \\
        &\leq \left(\sqrt{pq}\frac{\|\mathbf{A}\|_2}{\|\mathbf{A}\|_\infty} + 1\right)\left(\ch\big(\|\mathbf{A}\|_\infty\big) - 2\right) + 2pq.
    \end{align*}
    The looser bound in terms of the Frobenius norm allows for its tractable computation when $\mathbf{A} = \mathbf{B}\ominus\mathbf{C}$ for some $\mathbf{B} \in \R^{p\times q}, \mathbf{C} \in \R^{r \times s}$ with non-negative entries, as then
    $$\|\mathbf{A}\|_2^2 = \|\mathbf{B}\ominus\mathbf{C}\|_2^2 = pq\|\mathbf{B}\|_2^2 + rs\|\mathbf{C}\|_2^2 - 2\|\mathbf{B}\|_1\|\mathbf{C}\|_1.$$
    
    Next, we bound the eigenvalues' total magnitude from above. Using the symmetry of $\HH$,
    \begin{align*}
        \hspace{-1.5cm}\frac{1}{4}\sum_{i=1}^{(n+m)^2}\lambda_i^2 &= \left\|\frac{1}{2}\HH\right\|_2^2 \\
        &= \big\|\ch(\UU\ominus\VV)\big\|_2^2 + \big\|\ch(\WW\ominus\WW^T)\big\|_2^2 + 2\Big\langle \ch(\UU\ominus\VV), \ch\big((\WW\ominus\WW^T)\KK\big)\Big\rangle \\
        &= \big\|\ch(2\UU\ominus2\VV)\big\|_1 + \big\|\ch(2\WW\ominus2\WW^T)\big\|_1 + 4(n+m)^4
        \\&\hspace{1cm}+ 
        % \begin{aligned}
        2\Big\|\ch\big(\UU\ominus\VV + (\WW\ominus\WW^T)\KK\big)\Big\|_1 +2\Big\|\ch\big(\UU\ominus\VV - (\WW\ominus\WW^T)\KK\big)\Big\|_  1 \\
        %&< ((n+m)^2\frac{\|\UU\ominus\VV\|_2 + \|\WW\ominus\WW^T\|_2}{\dmax}+2)(\ch(2\dmax) - 2) + 4(n+m)^4 + 4(n+m)^2 \\
        %&\hspace{1cm}+ ((n+m)^2\frac{\|\UU\ominus\VV + (\WW\ominus\WW^T)\KK)\|_2 + \|\ldots\|_2}{\dmax} + 4)(\ch(2\dmax) - 2) + 8(n+m)^4 \\
        &\leq \Bigg((n+m)^2\frac{\|\UU\ominus\VV\|_2 + \|\WW\ominus\WW^T\|_2 + \big\|\UU\ominus\VV + (\WW\ominus\WW^T)\KK\big\|_2 + \big\|\UU\ominus\VV - (\WW\ominus\WW^T)\KK\big\|_2}{\dmax}
        \\
        &\hspace{1cm}+6\Bigg)\big(\ch(2\dmax) - 2\big) + 16(n+m)^4.
        % \end{aligned}
        % &= 4(\|c^\WW \otimes c^{-\WW^T} + c^{-\WW} \otimes c^{\WW^T}\|_2^2 + \|c^\UU \otimes c^{-\VV} + c^{-\UU} \otimes c^{\VV}\|_2^2 \\ &\hspace{2cm} +  \langle(c^\WW \otimes c^{-\WW^T} + c^{-\WW} \otimes c^{\WW^T})\KK, c^\UU \otimes c^{-\VV} + c^{-\UU} \otimes c^{\VV}\rangle) \\
        % &= 4(\sum_{i,j,h,k=1}^{n+m}((c^{W_{ij}-W_{hk}} + c^{W_{hk}-W_{ij}})^2 + (c^{U_{ij}-V_{hk}} + c^{V_{hk}-U_{ij}})^2) + ).
    \end{align*}
    From
    \begin{align*}
        \big\|\UU\ominus\VV \pm (\WW\ominus\WW^T)\KK\big\|_2 &= \sqrt{\|\UU\ominus\VV\|_2^2 + \|\WW\ominus\WW^T\|_2^2 \pm 2\big\langle(\WW\ominus\WW^T)\KK, \UU\ominus\VV\big\rangle}
    \end{align*}
    and \begin{align*}
        \big\langle(\WW\ominus\WW^T)\KK, \UU\ominus\VV\big\rangle &= \sum_{i,j,h,k=1}^{n+m}(W_{ik} - W_{jh})(U_{ij} -V_{hk}) \\ &=\sum_{i,j,h,k=1}^{n+m}(W_{ik}U_{ij} - W_{jh}U_{ji} - W_{ik}V_{hk} + W_{jh}V_{kh}) \\
        &= (n+m)\sum_{i,j,k=1}^{n+m}(W_{ij}U_{ik} - W_{ij}U_{ik} + W_{ji}V_{ki} - W_{ji}V_{ki}) \\
        &= 0,
    \end{align*}
    it follows that
    \begin{align*}
        \frac{1}{4}\sum_{i=1}^{(n+m)^2}\lambda_i^2
        &\leq \Bigg((n+m)^2\frac{\|\UU\ominus\VV\|_2 + \|\WW\ominus\WW^T\|_2 + 2\sqrt{\|\UU\ominus\VV\|_2^2 + \|\WW\ominus\WW^T\|_2^2}}{\dmax}
        \\
        &\hspace{1cm}+6\Bigg)\big(\ch(2\dmax) - 2\big) + 16(n+m)^4 \\
        &= \left((n+m)^2\frac{(2\sqrt{2} + 4)\sqrt{(n+m)^2\|\WW\|_2^2 - \|\WW\|_1^2}}{\dmax}+6\right)\big(\ch(2\dmax) - 2\big) + 16(n+m)^4 \\
        &= \pmax\big(\ch(2\dmax) - 2\big) + 16(n+m)^4
    \end{align*}
    and therefore
    \begin{align*}
        \lambda_- &\leq \sum_{i=2}^{(n+m)^2}|\lambda_i| \\
        &\leq \sqrt{\big((n+m)^2 - 1\big)\left(-\lambda_1^2 + \sum_{i=1}^{(n+m)^2}\lambda_i^2\right)} \\
        &\leq 2(n+m)\sqrt{16(n+m)^4 + \pmax\big(\ch(2\dmax) -2\big) - \frac{16}{(n+m)^4}\big\|c^\WW\big\|_1^2\big\|c^{-\WW}\big\|_1^2} \\
        &= 4(n+m)^2\frac{\alpha}{\frac{1}{2} - \alpha}.
    \end{align*}
    Finally,
    \begin{align*}
        \nconv(\sigma) &= \frac{\lambda_-}{2\lambda_- + 8(n+m)^2} \\
        &= \frac{1}{2} - \frac{2(n+m)^2}{\lambda_- + 4(n+m)^2} \\
        &\leq \frac{1}{2} - \frac{1}{\frac{2\alpha}{\nicefrac{1}{2}-\alpha} + 2} \\
        &= \alpha.
    \end{align*}
\end{proof}

\end{appendices}

\bibliographystyle{alpha}
\bibliography{references/references.bib}


\end{document}
