\section{Introduction}
The Gromov--Hausdroff distance \cite{gromov1999metric} measures the difference in shape between geometric objects, and is a natural metric on the space of (isometry classes of) compact metric spaces. It is widely used in Riemannian geometry and appears in other areas of modern mathematics such as computational topology and graph theory (see e.g. \cite{gromov1999metric, latschev2001vietoris, chazal2009gromov, tuzhilin2020lectures}). Starting from the late 2000s, it has become increasingly popular in data science as a model for dissimilarity measures between shapes such as point clouds \cite{memoli2007use, chazal2009gromov, villar2016polynomial, bronstein2010gromov} and graphs \cite{lee2012persistent, chung2017topological, fehri2018characterizing}).

Formally, the Gromov--Hausdorff distance between a pair of compact metric spaces $X, Y$ minimizes the distortion of distances between the points of $X$ and $Y$ and their images under some bi-directional mapping pair $(f,g)\in(X\to Y)\times(Y\to X)$. Solving this combinatorial minimization is an NP-hard problem, and in fact even approximating it up to a multiplicative factor in the general case remains an intractable task \cite{schmiedl2017computational}. High computational cost of the distance has motivated several relaxations such as the modified Gromov--Hausdorff distance \cite{memoli2012some}, the Gromov--Wasserstein distance \cite{memoli2011gromov}, and GHMatch \cite{villar2016polynomial}, as well as more feasible algorithms for approximating the Gromov--Hausdorff distance for the special cases of subsets of $\R^1$ \cite{majhi2023approximating}, metric trees \cite{agarwal2018computing, touli2018fpt}, and ultrametric spaces \cite{memoli2021gromov}.

\begin{tikzpicture}
\node[circle,draw,fill=Xcolor] (X1) at (0,0){};
\node[circle,draw,fill=Xcolor,left of=X1] (X2){};
\node[circle,draw,fill=Xcolor,below of=X1,node distance=4cm] (X3){};
\node[circle,draw,fill=Xcolor,left of=X3] (X4){};
\node[below left=1.6cm and 0cm of X1]{$X$};

\node[circle,draw,fill=Ycolor,right of=X1,node distance=5cm] (Y1){};
\node[circle,draw,fill=Ycolor,below left=3.25cm and .25cm of Y1] (Y2){};
\node[circle,draw,fill=Ycolor,below right=3.25cm and .25cm of Y1] (Y3){};
\node[circle,draw,fill=Ycolor,below of=Y1,node distance=4.5cm] (Y4){};
\node[below right=1.6cm and 4.6cm of X1]{$Y$};

\newcommand{\arrow}{{Stealth[length=3mm, width=1.5mm]}}

\draw[dotted, \arrow-\arrow] (X1) to[out=-20,in=-160] (Y1);
\draw[dotted, -\arrow] (X2) to[out=-50,in=-120] (Y1);

\draw[dotted, \arrow-\arrow] (X4) to[out=30,in=160] (Y2);
\draw[dotted, \arrow-\arrow] (X3) to[out=-20,in=-150] (Y3);
\draw[dotted, \arrow-] (X3) to[out=-60,in=-130] (Y4);


\end{tikzpicture}


% Similarly to this work, the Gromov--Wasserstein distance and GHMatch algorithm are amenable to gradient methods that can obtain approximate solutions.
The Gromov--Wasserstein distance and GHMatch relaxation 
are similar to our work in being amenable to gradient methods for retrieval of approximate solutions.
While the Gromov--Wasserstein distance is a broadly used metric in its own right, both of these relaxations are constrained to some generalization of the bijections between $X$ and $Y$ and can therefore retrieve only those solutions $(f,g)$ to the original minimization that satisfy $f = g^{-1}$. However, the existence of such solutions is not guaranteed even for the case of $|X|=|Y|=n$, as is illustrated by Figure \ref{fig:nonbijective}. The figure also demonstrates that the smallest distortion of a bijection can be arbitrarily large in comparison to the minimum taken over all mapping pairs.
%all mapping pairs can be arbitrary small in comparison to its counterpart constrained to the set of bijections.
This is perhaps unsurprising considering that $(X\to Y)\times (Y\to X)$ grows super-exponentially faster than the set of bijections between $X$ and $Y$. % due to  $\frac{n^{2n}}{n!} \geq \left(\sqrt{2}n\right)^n$.
From a practical standpoint, such an expressivity of the Gromov--Hausdorff distance can be useful for aligning metric spaces of irregular density, e.g. those obtained by uneven sampling from some underlying shapes.% or simply of distinct cardinality.

In the following Section \ref{relaxation}, we propose a parametrized relaxation of the Gromov--Hausdorff distance over the same search space $(X\to Y)\times (Y\to X)$. We derive a threshold for the parameter ensuring that any solution to the relaxed problem minimizes the distortion and therefore delivers the Gromov--Hausdorff distance. Section \ref{minimization} describes a gradient-based approach to solving the relaxation, and discusses the associated computational complexity and optimization landscape. We detail on our implementation and demonstrate its performance in numerical experiments in Section \ref{numerical}. As a byproduct, we tighten an upper bound on the Gromov--Hausdorff distance between the unit circle $S^1 \subset \R^2$ and a hemisphere $H^2 \subset \R^3$ of the unit sphere (Section \ref{spheres}). For brevity, theorem proofs are relegated to the appendix.% \ref{proofs}.

% The scope of this work is restricted to computing the GH distance between finite metric spaces.% as those appearing in the computational context.
% Such a capacity however would be sufficient for approximating the GH distance between any compact spaces with an arbitrary precision. This is because the GH distance between any pair of spaces is within $\epsilon$ from the GH distance between their finite $\epsilon$-nets, existent for any $\epsilon > 0$ \cite{oles2022lipschitz}. This approach is exemplified in Section \ref{} where we numerically bound the GH distance between continuous objects.


% Intuitively speaking, dGH tries to align distances in $X$ and $Y$ by mapping the two spaces into each other

% GHMatch in \cite{villar2016polynomial}:
% - its objective is p-norm relaxation of the original $\infty$-norm formulation (no guarantee of coinciding minima);
% - its search space ($\mathbf{y}$) is Birkhoff polytope which generalizes bijections and does not fully capture the space of all mapping pairs;
% - its iteration is $O(n^4)$
