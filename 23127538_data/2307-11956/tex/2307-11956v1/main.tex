%\RequirePackage{lineno}
%\usepackage[left]{lineno}
%\linenumbers
\def\CTeXPreproc{Created by ctex v0.2.12, don't edit!}%\documentclass[aps,preprint,groupedaddress]{revtex4}
%\documentclass[aps,preprint,superscriptaddress,showpacs]{revtex4}
%\documentclass[aps,twocolumn,superscriptaddress,showpacs]{revtex4}
\documentclass[prl,aps,amsfonts,amsmath,amssymb,nofootinbib,twocolumn,superscriptaddress]{revtex4}
%\documentclass[aps,twocolumn,prl,preprintnumbers,amsmath,amssymb,superscriptaddress]{revtex4-1}
%\documentclass[aps,twocolumn,prb,preprintnumbers,amsmath,amssymb,superscriptaddress,floats]{revtex4}
%\documentclass[preprint,showpacs,preprintnumbers,amsmath,amssymb]{revtex4}
%\documentclass[preprint,showpacs,superscriptaddress,preprintnumbers,amsmath,amssymb]{revtex4}
\usepackage{mathtools}
\usepackage{upgreek}
\usepackage{graphicx}% Include figure files
%usepackage{dcolumn}% Align table columns on decimal point
\usepackage{bm}% bold math
\usepackage{hyperref}
%\usepackage{IEEEtrantools}
\bibliographystyle{naturemag}
\usepackage{braket}
\newcommand{\RNum}[1]{\uppercase\expandafter{\romannumeral #1\relax}}

\usepackage{natbib}
\usepackage{float}
\usepackage{xcolor}

\restylefloat{table}

%\bibliographystyle{apsrev}
%\nofiles


\begin{document}
\title{Manipulating the electronic polarization in a magnetoelectric antiferromagnet via the two-photon Stark effect}
%\title{Light induced electronic symmetry change in Cr$_2$O$_3$ }
\author{Xinshu Zhang}
\affiliation{Department of Physics and Astronomy, University of California Los Angeles, Los Angeles, CA 90095, USA}

\author{Tyler Carbin}
\affiliation{Department of Physics and Astronomy, University of California Los Angeles, Los Angeles, CA 90095, USA}

\author{Adrian B. Culver}
\affiliation{Department of Physics and Astronomy, University of California Los Angeles, Los Angeles, CA 90095, USA}
\affiliation{Mani L. Bhaumik Institute for Theoretical Physics, Department of Physics and Astronomy, University of California Los Angeles, Los Angeles, CA 90095, USA}

\author{Kai Du}
\affiliation{Rutgers Center for Emergent Materials, Rutgers University, Piscataway, NJ, USA}


\author{Kefeng Wang}
\affiliation{Rutgers Center for Emergent Materials, Rutgers University, Piscataway, NJ, USA}


\author{Sang-Wook Cheong }
\affiliation{Rutgers Center for Emergent Materials, Rutgers University, Piscataway, NJ, USA}

\author{Rahul Roy}
\affiliation{Department of Physics and Astronomy, University of California Los Angeles, Los Angeles, CA 90095, USA}
\affiliation{Mani L. Bhaumik Institute for Theoretical Physics, Department of Physics and Astronomy, University of California Los Angeles, Los Angeles, CA 90095, USA}

\author{Anshul Kogar}
\email{anshulkogar@physics.ucla.edu}
\affiliation{Department of Physics and Astronomy, University of California Los Angeles, Los Angeles, CA 90095, USA}



\date{\today}

\maketitle


\textbf{
When intense light is shone through a transparent medium, the strong, time-periodic potential from the radiation field reshapes the many-body Hamiltonian. Crucially, this interaction does not involve light absorption and, in principle, does not generate any heat. The incident radiation can nonetheless be used to optically tailor various degrees of freedom, leading to the possibility of photo-controlling macroscopic properties of matter.
Here, we show that when inversion symmetry is broken by the antiferromganetic spin arrangement in Cr$_2$O$_3$, transmitting linearly polarized light through the crystal gives rise to a purely electronic dipole moment by way of a two-photon Stark effect. Using interferometric time-resolved second harmonic generation, we show that the threefold rotational symmetry of the crystal is broken only while the pump pulse is present; the timescale indicates that an electronic response is generated without affecting the magnetic or crystal structures. The orientation of the induced moment depends on the incident light polarization, which allows for contact-free control of the dipole moment vector. 
Our results establish a dissipationless optical protocol by which to selectively polarize the electronic subsystem and provides a method to manipulate electronic symmetries in noncentrosymmetric insulators.} 

The macroscopic electric polarization, $\textbf{\textit{P}}$, is the most fundamental quantity that describes dielectric media. For $\textbf{\textit{P}}$ to be finite in crystalline solids, the medium must lack an inversion center, more than one rotation axis, and an improper rotation axis. These  symmetry requirements can be met by the application of an appropriate external field or through the spontaneous breaking of symmetries across a phase transition \cite{symmetry1,symmetry2}.
Two contributions determine the electric polarization -- the ionic and the electronic subsystems \cite{polarization1,polarization2}. Experimentally, it is extremely difficult to isolate either component. However, this separation is essential to understanding the driving mechanisms behind the ferroelectric and magnetoelectric effects in correlated materials, where the electronic subsystem is thought to play a major role~\cite{reviewME1,reviewME2,reviewME3,transitionmetal,contribution1, contribution2}.

One method to isolate the electronic subsystem is to use a ``Floquet engineering" protocol \cite{Floquet,nonlinearFloquet,shirley}. In this approach, an ultrashort light pulse is shone below the electronic gap to avoid absorption; the electrons can then be driven coherently by the light's oscillating electric field. So far, this concept has led to the manipulation of spin dynamics in magnets and to the engineering of energy level shifts in semiconductors. To selectively address the spins, light is used to magnetize the system through the inverse Faraday effect or the inverse Cotton-Moutton effect \cite{InverseFaraday, InverseCM}. On the other hand, energy levels shifts can be induced through the AC Stark and/or Bloch-Siegert effects which give rise to a light-assisted level repulsion \cite{MnPS3,WS2,WS22,Yihua,Fahad}. 
While these cases exemplify dynamical control of magnetic and spectral properties of matter, similar sub-gap optical manipulation of the electronic polarization has yet to be definitively demonstrated \cite{TaAs,TaAsprl}. It has long been known, however, that in systems lacking inversion symmetry, a strong periodic drive can induce a finite quasi-DC polarization \cite{Boyd}.

In this work, we show that, by shining light in the transparency window of a noncentrosymmetric but nonpolar antiferromagnet, optical rectification can be used to selectively and controllably polarize the electronic subsystem through a two-photon Stark effect. Optical rectification is a nonlinear process whereby light's oscillating electric field generates a quasi-DC electric polarization~\cite{OR}. To leading order, the induced polarization is:

\begin{equation}
    \boldsymbol{P}(\omega_0) = \boldsymbol{\chi}^{e}_{OR}(\omega_0; \omega_1, -\omega_2) \boldsymbol{E}(\omega_1)\boldsymbol{E}(-\omega_2),
    \label{eq:Rectification}
\end{equation}
where $\boldsymbol{E}(\omega_j)$ ($j$=1, 2) is the electric field vector of the incident light, $\boldsymbol{\chi}^{e}_{OR}(\omega_0; \omega_1, -\omega_2)$ is the second-order nonlinear susceptibility tensor that describes optical rectification, and the frequency of the rectified response is given by $\omega_0 = \omega_1 - \omega_2$ with $\omega_1 \approx\omega_2$. When light pulses shorter than the typical structural response timescales are used to generate a rectified field, one could, in principle, distinguish between a structural and an electronic polarization based on the system's relaxation dynamics. 

% Figure environment removed 

To observe the induced dipole moment, we probe the system with rotational anisotropy second harmonic generation (RA-SHG), a technique sensitive to electronic symmetries \cite{RASHG1,RASHG2,RASHG3}. In our experimental configuration, both the incident fundamental and detected second harmonic light beams are polarized along the same direction (Fig.~\ref{fig:1}(b)). As a model compound, we select the prototypical linear magnetoelectric Cr$_2$O$_3$ \cite{controlCr2O3,SHGastool,FiebigSHG,Cr2O3book,ME0,ME1,ME2,ME3,ME4,ME5,ME6,ME7,timeCr2O3,timeCr2O32,timeCr2O3wall}. 
Above $T_{N}~\approx$~307~K, Cr$_2$O$_3$ crystallizes in the point group $\bar{3}m$ (D$_{3d}$) in which electric dipole SHG is forbidden due to the presence of inversion symmetry; however, when the second harmonic energy is tuned to a Cr $d$-$d$ electronic transition ($^{4}\hspace{-0.1cm}A_{2g} \hspace{0.1cm} (t_{2g})^3 \hspace{-0.15cm}\rightarrow^4\hspace{-0.1cm}T_{2g} \hspace{0.1cm} (t_{2g})^2e_g$) at 2.1~eV (590~nm), resonant magnetic dipole SHG is observed (Fig.~\ref{fig:1}(c)-(d))~\cite{FiebigSHG}. As shown in Fig.~\ref{fig:1}(d), the RA-SHG pattern is consistent with the threefold symmetry of the crystal when the probe light propagates along the out-of-plane direction (and is polarized in-plane). The RA-SHG pattern can be fit with a magnetic dipole source term of the following form (black line in Fig.~\ref{fig:1}(d)):
\begin{equation}
    \begin{split}
    I(2\omega; T>T_N) \propto &|\chi^m\textrm{sin}(3\theta)|^2
    \label{eq:RA-SHG_AboveTN}
    \end{split}
\end{equation}
which contains a single fit parameter, $\chi^m$, the in-plane magnetic dipole second harmonic susceptibility. Here, $\omega$ is the frequency of the probe light and $\theta$ represents the angle of the incident and detected light polarization with respect to the sample's $y$-axis (Fig.~\ref{fig:1}~(b)).
% Figure environment removed 

Below $T_{N}$, Cr$_2$O$_3$ orders antiferromagnetically with the four Cr spins in the unit cell alternating in an up and down sequence along the rhombohedral optical axis (Fig.~\ref{fig:1}(a)). This spin structure breaks inversion symmetry, and the magnetic point group becomes $\underline{\bar{3}m}$~\cite{Birss}. Electric dipole SHG is then allowed (through the spin-orbit interaction) and interferes with the pre-existing magnetic dipole signal~\cite{SHGastool,FiebigSHG}. 
Due to the interference between the magnetic and electric dipole signals, the nodes present in the RA-SHG pattern above $T_N$ are lifted below $T_N$ (Fig.~\ref{fig:1}(e)). That the nodes are lifted implies a phase difference between the magnetic and electric dipole SHG amplitudes at the probed wavelength. The RA-SHG pattern below $T_N$ can be fit with a simple function that includes electric and magnetic dipole radiation, as shown in Fig.~\ref{fig:1}(e)~\cite{FiebigSHG, SHGastool}: 

\begin{equation}
    I(2\omega; T<T_N) \propto |e^{i\gamma} \chi^m\textrm{sin}(3\theta) \pm \chi^e\textrm{cos}(3\theta)|^2 \\
    \label{eq:RA-SHG}
\end{equation}
where the $\pm$ depends on the AFM domain and $\gamma$ denotes the relative phase between magnetic and electric dipole second harmonic radiation (see Supplementary Note \RNum{1}). Near but below $T_N$, the electric dipole SHG susceptibility, $\chi^e$, is proportional to the antiferromagnetic (AFM) order parameter, \textbf{L}. Thus, $\chi_e$ differs in sign between the two AFM domains, which we denote $\alpha$ and $\beta$ (Fig.~\ref{fig:1}(a))~\cite{FiebigSHG,topography}. In this equation, it is important to note that $\gamma$ depends sensitively on the second harmonic energy (Supplementary Note \RNum{6}). At the 2.1~eV second harmonic energy used here, $\gamma \approx$ 85$^\circ$, and the two domains exhibit almost identical RA-SHG patterns (left panels of Fig.~\ref{fig:2}(b) and (d)).

We now move on to pump the Cr$_2$O$_3$ crystal with 1.2~eV (1030~nm) light pulses at a fluence of 20~mJ/cm$^2$. The wavelength of the pump light lies in the transparency window of the crystal and away from electronic resonances to avoid significant absorption (orange arrow in Fig.~\ref{fig:1}(c)). As the linearly polarized pump light is transmitted through the crystal, we observe a drastic symmetry change in the RA-SHG pattern (right panels of Fig.~\ref{fig:2}(b) and (d)). The RA-SHG pattern, which formerly respected the underlying threefold crystal symmetry, transiently exhibits only twofold symmetry. Additionally, the two domains, which are almost indistinguishable at equilibrium with RA-SHG, exhibit differing responses to the pump (the two domains are distinguishable at equilibrium with circularly polarized SHG~\cite{FiebigSHG}). With the SHG probe polarization fixed to $\theta=60^\circ$, the second harmonic intensity decreases by about 60\% in the $\beta$ domain, while the intensity increases by 30\% in the $\alpha$ domain (Fig.~\ref{fig:2}(a) and (c)). In these figures, the pump polarization is aligned along the $x$-axis, as indicated with the double-headed orange arrow in the right panels of Fig.~\ref{fig:2}(b) and (d). Notably, the symmetry reduction in the  RA-SHG pattern is not observed above $T_N$ or when the pump light is circularly polarized; instead the antiferromagnetism in addition to a well-defined polarization axis are both necessary to observe this symmetry breaking effect in Cr$_2$O$_3$ (Supplementary Notes \RNum{3} and \RNum{5}).

% Figure environment removed 
Importantly, we observe this effect only when the pump and probe pulses are temporally overlapped; the timescale characterizing this transient symmetry breaking is on the order of the laser pulse width ($\sim$200~fs) (Fig.~\ref{fig:2}(a) and (c)). (The slight asymmetry in the background levels of the time traces is attributed to two-photon absorption and is unrelated to the symmetry-breaking.) In the insets of Fig.~\ref{fig:2}(a) and (c), we show RA-SHG patterns 500~fs before and after the pump pulse propagates through the probed region as well as the pattern when the pump and probe pulses are perfectly overlapped (the latter are also shown in the right panels of Fig.~\ref{fig:2}(b) and (d)). Noticeably, there is not a measurable relaxation timescale for the symmetry change, which suggests that neither the structural nor spin degrees of freedom bring about this reduction in symmetry. Instead, the electronic charge degree of freedom is solely involved.

To show that the origin of this transient symmetry breaking stems from an optically induced electronic dipole moment, we measure the dependence of the RA-SHG pattern on the pump polarization. Figure~\ref{fig:3}(a) reveals how the RA-SHG pattern in the $\beta$ domain varies as the pump polarization is rotated clockwise in 20$^\circ$ steps. As the pump polarization angle, $\varphi$, is tuned, the shape and orientation of the RA-SHG pattern is correspondingly modified. However, every $\Delta\varphi=60^\circ$, the shape of the RA-SHG reappears but is rotated clockwise by the same amount due to the crystal symmetry. These observations indicate that the modification of the RA-SHG pattern is due to a vectorial perturbation which can be used to control the angle of the second harmonic emission. Based on symmetry arguments, we show below that the vectorial nature of the perturbation, coupled with the timescales involved and the differing SHG response from opposite AFM domains (Fig.~\ref{fig:2}), stems from a light induced polarization of the electronic subsystem.

% Figure environment removed 

Under this interpretation, the fits to the RA-SHG patterns are highly constrained. Mathematically, the electric/magnetic dipole susceptibility tensor describing SHG can be expanded to first order in the induced moment:

\begin{equation}
    \begin{split}
    \boldsymbol{\chi}^{e/m}_{SHG}(\textbf{\textit{P}}(\omega_0)) = \left.\boldsymbol{\chi}^{e/m}_{SHG}\right|_{\textbf{\textit{P}}=0} + \underbrace{\left.\frac{\partial\boldsymbol{\chi}_{SHG}^{e/m}}{\partial\textbf{\textit{P}}(\omega_0)}\right|_{\textbf{\textit{P}}=0}\hspace{-0.5cm}\textbf{\textit{P}}(\omega_0)}_{\equiv\delta\boldsymbol{\chi}^{e/m}_{SHG}}.
    \label{eq:cascade}
    \end{split}
\end{equation}
$\textbf{\textit{P}}({\omega_0})$ can be calculated based on the magnetic symmetry of the crystal and yields the following expression (see Supplementary Note \RNum{2}A): 

\begin{equation}
 \textbf{\textit{P}}({\omega_0})=  \chi^{e}_{OR} I_{pump} (\textrm{sin}(-2\varphi) \hat{\textbf{x}}+ \textrm{cos}(-2\varphi)  \hat{\textbf{y}}),
\label{eq:DipoleMoment}
\end{equation} 
where $\chi^e_{OR}$ is the in-plane susceptibility associated with optical rectification and $I_{pump}$ is the intensity of the pump beam.
By combining the generalized susceptibility tensor (Eq.~\ref{eq:cascade}) and the expression for the pump induced polarization (Eq.~\ref{eq:DipoleMoment}), we can derive the full expression for the SHG intensity as function of the pump polarization angle for the two domains (Supplementary Note \RNum{2}B):

\begin{equation}
\begin{split}
I(2\omega; &\varphi) \propto |e^{i\gamma}\chi^m\textrm{sin}(3\theta) \pm \chi^e\textrm{cos}(3\theta) \\
&+ie^{i\gamma}\delta\chi^m\textrm{sin}(\theta-2\varphi) \pm i\delta\chi^e\textrm{cos}(\theta-2\varphi)|^2.
\label{eq:PerturbationPhase}
\end{split}
\end{equation}
Compared to the equilibrium case, only two additional independent terms are permitted to model the RA-SHG pattern -- perturbations to the in-plane electric and magnetic dipole susceptibilities, $\delta\chi^e$ and $\delta\chi^m$, respectively. 
In Fig.~\ref{fig:3}(b), we demonstrate that Eq.~\ref{eq:PerturbationPhase} yields an excellent fit to the RA-SHG pattern when the pump polarization is along $x$-axis (i.e.  $\varphi=-90^\circ$). 


Several non-trivial predictions of our fit model are consistent with our experimental observations and interpretation. First, below $T_N$, Cr$_2$O$_3$ loses an inversion center and an improper rotation axis. Only the presence of two separate rotation axes, one in-plane and one out-of-plane, forbids a finite polarization. The perturbation terms in Eq.~\ref{eq:PerturbationPhase} break the in-plane rotational symmetry of the crystal (Fig.~\ref{fig:3}(b)), which then permits a finite dipole moment. 
Second, as the pump polarization angle, $\varphi$, is varied, the induced dipole moment is correspondingly rotated by an angle $-2\varphi$ (Eq.~\ref{eq:DipoleMoment} and \ref{eq:PerturbationPhase}). In Fig.~\ref{fig:3}(c), we illustrate this concept visually with the orange and red arrows which represent the orientations of the pump polarization and induced dipole moment, respectively. Once the RA-SHG pattern at $\varphi=-90^\circ$ is fit (Fig.~\ref{fig:3}(b)), the dependence on the pump polarization angle, $\varphi$, is completely determined by Eq.~\ref{eq:PerturbationPhase}. The results generated using Eq.~\ref{eq:PerturbationPhase} are shown in the black curves of Fig.~\ref{fig:3}(a) and show good agreement with the data. We emphasize that these are not fits to the individual RA-SHG patterns; only the $\varphi =-90^\circ$ pattern is fit and the remaining black lines are generated by this function, where only a single parameter, $\varphi$, is varied. Third, the perturbation terms are 90$^\circ$ out of phase with respect to the corresponding unperturbed terms (i.e. there is a 90$^\circ$ phase difference between $\delta\chi^e$ and $\chi^e$ as well as between $\delta\chi^m$ and $\chi^m$) which is accounted for by the coefficient $i$ multiplying the perturbation terms in the fit model. This phase difference arises due to the $\mathcal{PT}$ symmetry (parity, $\mathcal{P}$; time-reversal, $\mathcal{T}$) in Cr$_2$O$_3$ (Supplemental Note \RNum{2}C). Lastly, by choosing the opposite sign of $\chi^e$ and $\delta\chi^e$ in Eq.~\ref{eq:PerturbationPhase} (terms proportional to the order parameter near $T_N$), we obtain excellent agreement to the RA-SHG pattern of the opposite antiferromagnetic domain. Again, the black line in the right panel of Fig~\ref{fig:2}(b) is not a fit, but is generated from this procedure. Overall, our fit model is able to reproduce various experimentally observed aspects of the data and provides strong evidence in favor of an electronic dipole moment. However, the model itself does not explain how the moment is generated microscopically.

To understand how the dipole is induced, we use a single-ion model of the Cr atoms in the crystal field environment. In Cr$_2$O$_3$, the Cr atoms are surrounded by a trigonally-distorted oxygen octahedron. The Cr 3$d^3$ electrons occupy the lower three $t_{2g}$ states, which are split off from the higher $e_g$ states by at least 2.1~eV (Fig.~\ref{fig:4}(a)). However, an electronic dipole moment cannot develop by considering only the Cr 3$d$ subspace. A necessary (but not sufficient) condition for observing a dipole is the superposition of states with opposite parity. Above $T_N$, the trigonal distortion of the oxygen octahedron breaks inversion symmetry at the Cr site locally and therefore permits a superposition of the Cr 4$p$ and 3$d$ states. However, because global inversion symmetry is present in the crystal, both an equilibrium dipole moment and the optical rectification process are forbidden. Below $T_N$, in contrast, global inversion symmetry is broken by the magnetic structure, but an equilibrium dipole moment is still disallowed because Cr$_2$O$_3$ retains two rotational symmetry axes (threefold along the $z$- and twofold along the $y$-axis; Fig.~\ref{fig:1}(a)). 
A rectified response to a periodic drive, though, becomes possible.

To comprehend the effect of the drive, we appeal to the useful picture provided by Floquet theory. In the leftmost panel of Fig.~\ref{fig:4}(a), we schematically illustrate the energy levels in Cr$_2$O$_3$ and show examples of the corresponding in-plane probability densities. In this sketch, the ground state represents the $^4A_{2g}$ state while the excited states, $\ket{e}$, include all six $^4T_{2g}$ and $^4T_{1g}$ states. (For the purposes of this illustration, we are neglecting the splitting between the $^4T_{2g}$ and $^4T_{1g}$ which is roughly 0.6~eV). When a periodic potential is applied to the system, a series of Floquet sidebands or ``dressed states" emerge, which we label $\ket{j, n}$, where $j$ labels the equilibrium state from which the $n^{th}$ Floquet sideband derives. In the middle panel of Fig.~\ref{fig:4}(a), we make the rotating wave approximation and only show the dressed states of the excited state manifold, $\ket{e, -1}$. Although this truncation works best when the drive is resonant between $\ket{g}$ and $\ket{e}$, this picture already provides a mechanism by which a static electronic dipole can be generated. 

In this effectively time-independent scheme, a ``hopping" is only possible between states separated by $\pm$1 in the Floquet index, where hopping matrix elements are given by terms like $\lambda\sim\bra{j',n\pm1}\boldsymbol{d}\cdot\boldsymbol{E}\ket{j,n}$\cite{shirley}. Here, $\boldsymbol{d}$ is the dipole operator and $\boldsymbol{E}$ is the time-independent electric field vector, and therefore the value of the hopping matrix elements sensitively depends on the polarization of the incoming light. Schematically, a superposition state, $\ket{g'} = \ket{g,0} + \lambda\ket{e,-1} + \lambda^2\ket{e,0}$, then forms due to this ``hopping", where the power of the coefficient $\lambda$ indicates a first or second order perturbative correction to the ground state. The superposition between $\ket{g,0}$ and $\ket{e,0}$ must occur via the $\ket{e,-1}$ sideband, which is why the coefficient scales with $\lambda^2$. $\ket{g'}$ is capable of exhibiting a dipole moment (either oscillating at the laser frequency or static), which can be seen by calculating the expectation value $\bra{g'}\boldsymbol{d}\ket{g'}$ (Supplementary Note \RNum{8}).

With the orange arrows in the middle panel of Fig.~\ref{fig:4}(a), we illustrate the two pathways through which a \textit{static} dipole can develop. The first path is shown with the dotted arrows and indicates that a static dipole can arise due to terms of the form $\lambda^2\bra{g,0}\boldsymbol{d}\ket{e,0}\sim|\boldsymbol{E}|^2$. Selection rules associated with such matrix elements have been previously been calculated for Cr$_2$O$_3$ in Refs.~\cite{ME0, ME1, ME2}. We use the matrix elements therein to sketch representative probability densities for this kind of process in the rightmost panel of Fig.~\ref{fig:4}(a). Clearly, the in-plane threefold symmetry of the crystal is broken by this superposition and a dipole moment can be observed. In Supplementary Note \RNum{7}, we discuss the symmetry of the $t_{2g}$ and $e_g$ states to clarify how these probability densities are sketched. 

A second pathway, indicated with the dashed orange arrows, also gives rise to a static dipole moment. In this case, the expectation value of the dipole operator yields terms of the form $\lambda^2\bra{e,-1}\boldsymbol{d}\ket{e,-1}\sim |\boldsymbol{E}|^2$. Since the Floquet indices on either side of the matrix element are identical, the dipole moment is static. %Naively, one would expect that such a hybridization to give rise to a dipole oscillating at the frequency of the radiation field, but a static dipole can also arise when calculating the expectation value of the dipole operator, $\bra{\phi}\boldsymbol{d}\ket{\phi}$.
%Because multiple states are present in the excited state manifold, terms like these are allowed. 
In both hopping pathways, non-vanishing dipole matrix elements among the excited state degenerate subspace is crucial for the observation of the static dipole.
It is important to note that both pathways also give rise to terms that scale with $\lambda^2 \sim |\boldsymbol{E}|^2$. Such a relation implies that the lowest order static electric dipole moment scales with $|\boldsymbol{E}|^2$ (i.e. fluence), which is confirmed experimentally in Fig.~\ref{fig:4}(b). We thus refer to this electronic process as a two-photon Stark effect. Further details of the calculation using time-dependent perturbation theory and Floquet theory in addition to using the single-particle and multiparticle bases are presented in Supplementary Note \RNum{8}. 


In conclusion, our experimental observations and simple theoretical account demonstrate how a purely electronic dipole moment develops in antiferromagnetic Cr$_2$O$_3$ via a two-photon Stark effect. By breaking in-plane rotational symmetry, light's polarization is used to establish optical control of the dipole moment vector. Our experiment builds on previous work showing that purely electronic macroscopic effects in crystals can be induced with laser pulses using a ``Floquet engineering" protocol. 
In principle, the generation of an electronic dipole moment in linear magnetoelectrics should produce a corresponding magnetic response; in compounds where the electronic component of the magnetoelectric effect is dominant, optical control of magnetism should be possible \cite{Floquet,magnetism1,type2MEa,type2MEb}. Our work paves the way towards isolating and quantifying the electronic contribution of the magnetoelectric effect as well as the optical manipulation of magnetism through the magnetoelectric effect.





\footnotesize
\vspace{-0.75em}
 \section{Acknowledgements:}
We thank Mengxing Ye, Honglie Ning, Carina Belvin and Wesley Campbell for helpful conversations related to this work.
Research at UCLA was supported by the U.S. Department of Energy (DOE), Office of Science, Office of Basic Energy Sciences under Award No. DE-SC0023017 (experiment and theory). The work at Rutgers was supported by W. M. Keck Foundation (materials synthesis).

\section{Author contributions:  }
X.Z. and T.C. built the SHG setup and performed the time-resolved SHG experiments under the supervision of A.K. X.Z. analysed the data under the supervision of A.K. K.D. and K.W. grew the single crystals under the supervision of S.-W.C. Theoretical calculations were carried out by A.B.C. under the supervision of R.R. The manuscript was written by X.Z., A.B.C. and A.K. with input from all authors.
\vspace{-0.75em}

\section{Competing interests:  }

The authors declare no competing interests.


\section{Methods }
\vspace{-1em}
\subsection{Sample synthesis}
Cr$_2$O$_3$ single crystals were grown using a laser diode heated floating zone (LFZ) technique. Cr$_2$O$_3$ powders (Alfa Aesar, $99.99\%$) were pressed into 3 mm diameter rods under 8000 PSI hydrostatic pressure. The compressed rod was sintered at 1600$^{\circ}$C in a box furnace for 10~hours. The crystals were grown with growth speed of 2 to 4 mm/h in oxygen flow of 0.1 l/min, and counter rotation of the feed and seed rods at 15 and 15 rpm, respectively.

\subsection{Experimental details}
The primary laser used in our experiment is based on a Yb:KGW gain medium that outputs a power of 10~W. The laser pulses have a Gaussian-like profile with an approximately 180~fs pulse duration and a 1030~nm central wavelength. In our experiment, we used a laser pulse repetition rate of 5~kHz. The fundamental output of the laser at 1030~nm was used as the pump pulse, which was focused obliquely on the sample at a 10 degree angle of incidence. The pump laser spot size was $\sim$ 500~$\mu$m and the maximum fluence was $\sim$ 20 $\mathrm{mJ}/\mathrm{cm}^{2}$. The probe pulse was generated from an optical parametric amplifier with tunable wavelength, which we use for the second harmonic spectroscopy between 900-1200~nm. The probe pulse was focused normally on the sample with a 100~$\mu$m spot size, and the probe fluence was $\sim$~2~$\mathrm{mJ}/\mathrm{cm}^{2}$. Detection of the second harmonic light was conducted with a commercial photo-multiplier tube. The sample was cooled to 150~K with a standard optical cryostat with fused silica windows to prevent distortions to the light polarization.


 
 \section{Data availability}
 \vspace{-1em}
 The data that supports the findings of this study are present in the paper and/or in the supplementary information, and are deposited in the Zenodo repository.  Additional data related to the paper is available from the corresponding authors upon reasonable request.
\begin{thebibliography}{1}
\bibitem{symmetry1}    Powell, R.  Symmetry, group theory, and the physical properties of crystals \textit{Springer 
}  (2010).
\bibitem{symmetry2}   Landau, L  Electrodynamics of continuous media  \textit{elsevier}  (2013).
\bibitem{polarization1}    Khomskii, D.  Transition metal compounds  \textit{Cambridge University Press 
}  (2014).
\bibitem{polarization2}    Bonfim, O. $\&$ Gehring, G. Magnetoelectric effect in antiferromagnetic crystals  \textit{Advances in Physics 
}  \textbf{29,} 731-769 (1980).
\bibitem{reviewME1}   Spaldin, N.  $\&$  Ramesh, R.  Advances in magnetoelectric multiferroics \textit{Nat. Mater.} \textbf{18,}   203–212 (2019).
\bibitem{reviewME2}  Spaldin, N.  $\&$  Fiebig, M.   The Renaissance of Magnetoelectric Multiferroics \textit{Science } \textbf{309,}   391-392 (2005).
\bibitem{reviewME3}   Cheong, S. $\&$  Mostovoy, M.  Multiferroics: a magnetic twist for ferroelectricity \textit{Nat. Mater. } \textbf{6,}  13–20 (2007).
\bibitem{transitionmetal}  Khomskii, D.  Transition metal compounds \textit{Cambridge University Press}  (2014).
\bibitem{contribution1}   Malashevich, A. et al.  Full magnetoelectric response of Cr$_2$O$_3$ from first principles \textit{Phys. Rev. B } \textbf{86,}   094430 (2012).
\bibitem{contribution2}   Bousquet, E.  Spaldin, N. $\&$  Delaney, K.  Unexpectedly Large Electronic Contribution to Linear Magnetoelectricity \textit{Phys. Rev. Lett. } \textbf{106,}   107202 (2011).

\bibitem{Floquet}  Oka, T. $\&$ Kitamura, S.  Floquet Engineering of Quantum Materials. \textit{Annu. Rev. Condens. Matter Phys.} \textbf{10,}  387-408 (2019).
\bibitem{nonlinearFloquet}  Morimoto, T.  $\&$ Nagaosa, N. Topological nature of nonlinear optical effects in solids. \textit{Sci. Adv.} \textbf{2,  } 5 (2016).
\bibitem{shirley}  Shirley, J. Solution of the Schrodinger Equation with a Hamiltonian Periodic in Time \textit{Phys. Rev} \textbf{138,  } 4B (1965).
\bibitem{InverseFaraday}    Kimel, A. et al. Ultrafast non-thermal control of magnetization by instantaneous photomagnetic pulses. \textit{Nature} \textbf{435,   }655–657 (2005).
\bibitem{InverseCM}     Baranga, A. et al. Observation of the inverse Cotton-Mouton effect. \textit{Europhysics Letters} \textbf{84,   } 44005 (2011).
\bibitem{MnPS3}  Shan, J. et al. Giant modulation of optical nonlinearity by Floquet engineering \textit{Nature} \textbf{ 600, } 235–239 (2021).
\bibitem{WS2}   Sie, E. et al. Valley-selective optical Stark effect in monolayer WS$_2$. \textit{Nat. Mater.} \textbf{14, } 290–294 (2015).
\bibitem{WS22}   Sie, E. et al. Large, valley-exclusive Bloch-Siegert shift in monolayer WS$_2$. \textit{Science} \textbf{355,  } 6329 (2015).

\bibitem{Yihua}  Wang, Y. et al. Observation of Floquet-Bloch States on the Surface of a Topological Insulator \textit{Science} \textbf{342,    } 453–457 (2013).
\bibitem{Fahad}  Mahmood, F. et al. Selective scattering between Floquet–Bloch and Volkov states in a topological insulator. \textit{Nat. Phys.} \textbf{12,    } 306–310 (2016).

\bibitem{TaAs}   Sirica, N. et al.  Photocurrent-driven transient symmetry breaking in the Weyl semimetal TaAs. \textit{Nat. Mater.} \textbf{ 21, } 62–66 (2022).
\bibitem{TaAsprl}   Sirica, N. et al. Tracking Ultrafast Photocurrents in the Weyl Semimetal TaAs Using THz Emission Spectroscopy. \textit{Phys. Rev. Lett. } \textbf{ 122, } 197401 (2019).
\bibitem{Boyd}   Boyd, R.  Nonlinear optics \textit{Academic press } (2020).

\bibitem{OR}  Bass, M. et al. Optical Rectification. \textit{Phys. Rev. Lett. } \textbf{ 9,  } 446  (1962).
\bibitem{RASHG1}     Harter, J. et al. A parity breaking electronic nematic phase transition in the spin orbit coupled metal Cd$_2$Re$_2$O$_7$ \textit{Science} \textbf{ 356, } 295-299 (2017).
\bibitem{RASHG2}   Zhao, L. et al. A global inversion-symmetry-broken phase inside the pseudogap region of YBa$_2$Cu$_3$O$_y$ \textit{Nat. Phys.} \textbf{ 13,   } 250–254 (2017)
\bibitem{RASHG3}    Wu, L. et al. Giant anisotropic nonlinear optical response in transition metal monopnictide Weyl semimetals \textit{Nat. Phys.} \textbf{ 13,  }350–355 (2017).

\bibitem{controlCr2O3}   He, X. et al. Robust isothermal electric control of exchange bias at room temperature \textit{Nat. Mater.} \textbf{9,  }579–585 (2010).
\bibitem{SHGastool}  Fiebig, M.,  Pavlov, V.  $\&$ Pisarev, R.   Second-harmonic generation as a tool for studying electronic and magnetic structures of crystals: review \textit{J. Opt. Soc. Am. B} \textbf{22,    } 1, 96-118 (2005).
\bibitem{FiebigSHG}  Fiebig, M. et al.   Second Harmonic Generation and Magnetic-Dipole —Electric-Dipole Interference in Antiferromagnetic Cr$_2$O$_3$ \textit{Phys. Rev. Lett.} \textbf{73,  } 2127 (1994).
\bibitem{Cr2O3book}  Sugano, S. $\&$ Kojima, N.    Magneto-optics \textit{Springer} \textbf{128,    }  (2013).
\bibitem{ME0}   Muthukumar, V.,   Valentí, R.  $\&$  Gros, C.  Microscopic Model of Nonreciprocal Optical Effects in  Cr$_2$O$_3$  \textit{Phys. Rev. Lett.} \textbf{ 75, 2766 } 2766  (1995).
\bibitem{ME1}   Muto, M. et al.   Magnetoelectric and second-harmonic spectra in antiferromagnetic  Cr$_2$O$_3$ \textit{Phys. Rev. B  } \textbf{57,  } 9586 (1998).
\bibitem{ME2}    Muthukumar, V.,   Valentí, R.  $\&$  Gros, C.   Theory of nonreciprocal optical effects in antiferromagnets: The case of  Cr$_2$O$_3$ \textit{Phys. Rev. B  } \textbf{54,   } 433 (1996).
\bibitem{ME3}    Krichevtsov, B. et al.   Magnetoelectric Spectroscopy of Electronic Transitions in Antiferromagnetic  Cr$_2$O$_3$ \textit{Phys. Rev. Lett.} \textbf{76, }  4628  (1996).
\bibitem{ME4}    Malashevich, A. et al.  Full magnetoelectric response of  Cr$_2$O$_3$  from first principles \textit{Phys. Rev. Lett.} \textbf{ 86, } 094430  (2012).
\bibitem{ME5}    Iyama, A. et al.  Magnetoelectric hysteresis loops in Cr$_2$O$_3$ at room temperature \textit{Phys. Rev. Lett.} \textbf{ 87, } 180408  (2013).
\bibitem{ME6}    Pisarev, R.  Crystal optics of magnetoelectrics \textit{Ferroelectrics} \textbf{ 162,  } 191-209  (1994).
\bibitem{ME7}    Hayashida, T. et al.  Observation of antiferromagnetic domains in 
 Cr$_2$O$_3$  using nonreciprocal optical effects \textit{Phys. Rev. Res.} \textbf{ 4,  } 043063  (2022).
 \bibitem{timeCr2O3}  Satoh, T. et al.  Ultrafast spin and lattice dynamics in antiferromagnetic Cr$_2$O$_3$ \textit{Phys. Rev. B  } \textbf{ 75, } 155406 (2007).
\bibitem{timeCr2O32}  Satoh, T. et al.  Time-resolved demagnetization in  Cr$_2$O$_3$ by phase sensitive second harmonic generation \textit{Phys. Rev. B  } \textbf{310, } 1604-1606 (2007).
\bibitem{timeCr2O3wall}  Sala, V. et al.  Resonant optical control of the structural distortions that drive ultrafast demagnetization in  Cr$_2$O$_3$  \textit{Phys. Rev. B  } \textbf{94,  }014430 (2015).

\bibitem{Birss}     Birss, R.  Symmetry and Magnetism  \textit{North Holland, 
}  (1966).
\bibitem{topography}    Fiebig, M.,Fröhlich, D. $\&$   Sluyterman, G. Domain topography of antiferromagnetic Cr$_2$O$_3$  by second‐harmonic generation\textit{Appl. Phys. Lett.} \textbf{  66, }  2906 (1995).

\bibitem{magnetism1}   Kirilyuk, A.,  Kimel, A. $\&$  Rasing, T. Ultrafast optical manipulation of magnetic order  \textit{Rev. Mod. Phys.} \textbf{82,    } 2731 (2010).
\bibitem{type2MEa}  Khomskii, D. Classifying multiferroics: mechanisms and effects. \textit{Physics } \textbf{2,    } 20 (2009).
\bibitem{type2MEb}   Tokura, Y., Seki, S. $\&$ Nagaosa, N. Multiferroics of spin origin. \textit{Rep. Prog. Phys.} \textbf{77,    } 076501 (2014)

%\bibitem{magnetism2}  Stanciu, C. et al.  All-Optical Magnetic Recording with Circularly Polarized Light \textit{Phys. Rev. Lett. } \textbf{99,   }047601 (2007).
%\bibitem{fatigue}    Lebeugle, D. et al.  Very large spontaneous electric polarization in BiFeO$_3$ single crystals at room temperature
%and its evolution under cycling fields \textit{Appl. Phys. Lett.} \textbf{91,}  022907 (2007)
%\bibitem{spectrum1}    McClure, D.  Comparison of the Crystal Fields and Optical Spectra of Cr$_2$O$_3$  and Ruby \textit{J. Chem. Phys.} \textbf{ 38, }2289 (1963).
%\bibitem{spectrum2}    Allen, J.,  MACFARLANE, R. $\&$  WHITE, R.  Magnetic Davydov Splittings in the Optical Absorption Spectrum of  Cr$_2$O$_3$  \textit{Phys. Rev. } \textbf{ 179, }523 (1969).

%\bibitem{exponent1}   Murtazaev, A.  Critical properties of the model of antiferromagnet Cr$_2$O$_3$ \textit{Low Temperature Physics  } \textbf{25,   } 344 (1999);
%\bibitem{exponent2}    Al-Mahdawi, M. et al. Apparent critical behaviour of sputter-deposited magnetoelectric antiferromagnetic Cr$_2$O$_3$ films near Néel temperature \textit{J. Phys. D: Appl. Phys.} \textbf{ 50,  }155004 (2017).


%\bibitem{THz1}  Nahata, A. et al. A wideband coherent terahertz spectroscopy system using optical rectification and electro‐optic sampling. \textit{Appl. Phys. Lett.  } \textbf{ 90,  }  171121 (2007).
%\bibitem{THz2}   Yeh, K. et al. Generation of 10$\mu$J ultrashort terahertz pulses by optical rectification. \textit{Appl. Phys. Lett.} \textbf{69,  } 2321 (1996).


\end{thebibliography}






\normalsize

 
\end{document}


 
 
