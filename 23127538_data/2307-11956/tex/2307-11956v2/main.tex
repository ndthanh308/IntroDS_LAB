%\RequirePackage{lineno}

%\linenumbers
\def\CTeXPreproc{Created by ctex v0.2.12, don't edit!}%\documentclass[aps,preprint,groupedaddress]{revtex4}
%\documentclass[aps,preprint,superscriptaddress,showpacs]{revtex4}
%\documentclass[aps,twocolumn,superscriptaddress,showpacs]{revtex4}
\documentclass[prl,aps,amsfonts,amsmath,amssymb,nofootinbib,twocolumn,superscriptaddress]{revtex4}
%\documentclass[aps,twocolumn,prl,preprintnumbers,amsmath,amssymb,superscriptaddress]{revtex4-1}
%\documentclass[aps,twocolumn,prb,preprintnumbers,amsmath,amssymb,superscriptaddress,floats]{revtex4}
%\documentclass[preprint,showpacs,preprintnumbers,amsmath,amssymb]{revtex4}
%\documentclass[preprint,showpacs,superscriptaddress,preprintnumbers,amsmath,amssymb]{revtex4}
\usepackage{mathtools}
\usepackage{upgreek}
%\usepackage{lineno}
\usepackage{graphicx}% Include figure files
%usepackage{dcolumn}% Align table columns on decimal point
\usepackage{bm}% bold math
\usepackage{hyperref}
%\usepackage{IEEEtrantools}
\bibliographystyle{naturemag}
\usepackage{braket}
\newcommand{\RNum}[1]{\uppercase\expandafter{\romannumeral #1\relax}}

\usepackage{natbib}
\usepackage{float}
\usepackage{xcolor}

\restylefloat{table}

%\bibliographystyle{apsrev}
%\nofiles


\begin{document}
\title{Light-induced polar nematicity in antiferromagnetic Cr$_2$O$_3$}
%\title{Light induced electronic symmetry change in Cr$_2$O$_3$ }
\author{Xinshu Zhang}
\affiliation{Department of Physics and Astronomy, University of California Los Angeles, Los Angeles, CA 90095, USA}

\author{Tyler Carbin}
\affiliation{Department of Physics and Astronomy, University of California Los Angeles, Los Angeles, CA 90095, USA}

\author{Adrian B. Culver}
\affiliation{Department of Physics and Astronomy, University of California Los Angeles, Los Angeles, CA 90095, USA}
\affiliation{Mani L. Bhaumik Institute for Theoretical Physics, Department of Physics and Astronomy, University of California Los Angeles, Los Angeles, CA 90095, USA}

\author{Kai Du}
\affiliation{Rutgers Center for Emergent Materials, Rutgers University, Piscataway, NJ, USA}


\author{Kefeng Wang}
\affiliation{Rutgers Center for Emergent Materials, Rutgers University, Piscataway, NJ, USA}


\author{Sang-Wook Cheong }
\affiliation{Rutgers Center for Emergent Materials, Rutgers University, Piscataway, NJ, USA}

\author{Rahul Roy}
\affiliation{Department of Physics and Astronomy, University of California Los Angeles, Los Angeles, CA 90095, USA}
\affiliation{Mani L. Bhaumik Institute for Theoretical Physics, Department of Physics and Astronomy, University of California Los Angeles, Los Angeles, CA 90095, USA}

\author{Anshul Kogar}
\email{anshulkogar@physics.ucla.edu}
\affiliation{Department of Physics and Astronomy, University of California Los Angeles, Los Angeles, CA 90095, USA}



\date{\today}

\maketitle

%\linenumbers
\textbf{In a solid, the electronic subsystem can exhibit incipient order with lower point group symmetry than the crystal lattice. External fields that couple to electronic order parameters have rarely been investigated, however, despite their potential importance to inducing exotic effects. Here, we show that when inversion symmetry is broken by the antiferromagnetic (AFM) order in Cr$_2$O$_3$, transmitting a linearly polarized light pulse through the crystal gives rise to a rotational symmetry breaking, characteristic of a nematic state, via optical rectification. Using interferometric time-resolved second harmonic generation, we show that the ultrafast timescale of the symmetry reduction is indicative of a purely electronic response; the underlying spin and crystal structures remain unaffected. The nematic state exhibits polar character, and its axis can be controlled with the incident light. %Remarkably, the light-indcued nematic state has polar character in contrast to common nematicity with two fold symmetry. We are able to manipulate the polar nematicity in a dissipationless and contact-free fashion, even if the orientation of dipole need not lie along the polarization axis of the incident light.  
Our results establish a dissipationless nonlinear optical protocol by which to break electronic symmetries and produce novel electronic phases of matter.
} 

%The macroscopic electric polarization, $\textbf{\textit{P}}$, is the most fundamental quantity that describes dielectric media. For $\textbf{\textit{P}}$ to be finite in crystalline solids, the medium must lack an inversion center, more than one rotation axis, and an improper rotation axis. These  symmetry requirements can be met by the application of an appropriate external field or through the spontaneous breaking of symmetries across a phase transition \cite{symmetry1,symmetry2}.
%Two contributions determine the electric polarization -- the ionic and the electronic subsystems \cite{polarization1,polarization2}. Experimentally, it is extremely difficult to isolate either component. However, this separation is essential to understanding the driving mechanisms behind the ferroelectric and magnetoelectric effects in correlated materials, where the electronic subsystem is thought to play a major role~\cite{reviewME1,reviewME2,reviewME3,transitionmetal,contribution1, contribution2}.Evidence of these states is present in correlated systems like cuprate, iron-pnictide, and heavy fermion superconductors. 

%When intense light is shone through a medium, the strong, time-periodic potential from the radiation field reshapes the many-body Hamiltonian.
Neumann's principle, formulated in 1885, stipulates that the physical properties of a perfect crystal must possess at least the point group symmetries of the underlying crystal~\cite{symmetry1}. It serves as a cornerstone for interpreting experiments on crystalline matter. In recent years, however, this paradigm has been increasingly questioned. Electronic nematics, %, where the electronic system breaks the rotational symmetry of the crystal lattice without affecting the translational symmetry, defy this framework. 
which are characterized by a lowering of rotational symmetry that is not simply a consequence of reduced lattice symmetry, challenge this framework~\cite{kivelson}. Evidence of electronic nematicity is present in quantum Hall states, in Sr$_3$Ru$_2$O$_7$ at high magnetic fields, as well as in the normal states of cuprate, iron-pnictide, ruthenate, and heavy fermion superconductors~\cite{eisenstein, feldman, mackenzie, harter, bozovic, fisher, bozovic2, moll, matsuda}. This body of work illustrates that the symmetry of the electronic system can potentially be decoupled from that of the lattice. %These studies raise an important question: is it possible to apply an external perturbation solely to the electronic subsystem to lower the symmetry? 

An outstanding related question is whether external fields can be found that break only the point group symmetries of the electronic subsystem~\cite{TaAs, TaAsprl}. In the current experimental landscape, it is difficult to disentangle the extent to which the electrons or ions couple to applied static fields like strain, pressure or electric fields.  %Hence, there is a tremendous need to develop external fields that couple solely to the point group symmetries of the electronic subsystem. 
However, ultrafast light pulses can potentially overcome this barrier and thereby address several long-standing issues in condensed matter physics. For instance, the relative importance of the structural versus the electronic contribution to the ferroelectric and magnetoelectric effects is still controversial despite decades of research~\cite{polarization1, polarization2, reviewME1, reviewME2, reviewME3, contribution1, contribution2}. In addition, the true nematic susceptibility of correlated materials could be measured without coupling the system to a lattice strain field, which instead yields a renormalized susceptibility~\cite{fisher}. At large strengths, such applied fields would also open up the possibility of tuning across purely electronic quantum phase transitions.

%An outstanding related question is whether externally applied fields can be found that couple solely to \textit{electronic} order parameters. Such fields would potentially enable the electronic contribution to various effects in solids to be isolated and address several long-standing issues in condensed matter physics. For instance, the structural versus the electronic contributions to the ferroelectric and magnetoelectric effects is still controversial despite decades of research. In addition, the true nematic susceptibility of correlated materials could be measured without the need to couple the system to a lattice strain field, which instead yields a renormalized susceptibility. At large strengths, such applied fields would also open up the possibility of tuning across purely electronic quantum phase transitions. 
%In the current experimental landscape, it is difficult to disentangle the extent to which fields like strain, pressure or electric fields break electronic versus lattice symmetries. 
%Hence, there is a tremendous need to develop external fields that couple solely to the point group symmetries of the electronic subsystem. %Recent work in this direction has hinted that this goal may be achievable by applying transient photocurrents.

%One method to isolate the electronic subsystem is to use a ``Floquet engineering" protocol \cite{Floquet,nonlinearFloquet,shirley}. In this approach, an ultrashort light pulse is shone below the electronic gap to avoid absorption; the electrons can then be driven coherently by the light's oscillating electric field. So far, this concept has led to the manipulation of spin dynamics in magnets and to the engineering of energy level shifts in semiconductors. To selectively address the spins, light is used to magnetize the system through the inverse Faraday effect or the inverse Cotton-Moutton effect \cite{InverseFaraday, InverseCM}. On the other hand, energy levels shifts can be induced through the AC Stark and/or Bloch-Siegert effects which give rise to a light-assisted level repulsion \cite{MnPS3,WS2,WS22,Yihua,Fahad}. 
%While these cases exemplify dynamical control of magnetic and spectral properties of matter, similar sub-gap optical manipulation of the electronic polarization has yet to be definitively demonstrated \cite{TaAs,TaAsprl}. It has long been known, however, that in systems lacking inversion symmetry, a strong periodic drive can induce a finite quasi-DC polarization \cite{Boyd}.

In this work, by shining an ultrashort light pulse through a noncentrosymmetric antiferromagnet, we show that optical rectification can induce a quasi-DC response that selectively breaks the rotational symmetry of the electronic subsystem. Optical rectification is a nonlinear process whereby light's oscillating electric field brings about a quasi-DC electric dipole moment~\cite{Boyd, OR}. To leading order, the induced dipole moment is given by:
\begin{equation}
    \boldsymbol{P}(\omega_0) = \boldsymbol{\chi}^{e}_{OR}(\omega_0; \omega, -\omega) \boldsymbol{E}(\omega)\boldsymbol{E}(-\omega),
    \label{eq:Rectification}
\end{equation}
where $\boldsymbol{E}(\omega)$ is the electric field vector of the incident light with frequency $\omega$, $\boldsymbol{\chi}^{e}_{OR}(\omega_0; \omega, -\omega)$ is the second-order nonlinear susceptibility tensor that describes optical rectification, and the frequency of the rectified response is given by $\omega_0 \approx 0$. Crucially, when light pulses shorter than the typical structural response timescales are used to generate a rectified polarization, one could, in principle, distinguish between a structural and an electronic response via the system's relaxation dynamics. 

% Figure environment removed 

To observe the induced symmetry breaking, we probe the system with rotational anisotropy second harmonic generation (RA-SHG), a technique sensitive to electronic symmetries \cite{SHGastool}. In our experimental configuration, both the incident fundamental and detected second harmonic light beams are polarized along the same direction (Fig.~\ref{fig:1}(b)). As a model compound, we select the prototypical linear magnetoelectric Cr$_2$O$_3$ because, as we describe below, interference of the electric and magnetic dipole second harmonic radiation allows for high-sensitivity detection of broken point group symmetries \cite{FiebigSHG, topography, timeCr2O3,timeCr2O32,timeCr2O3wall}. 

At equilibrium, Cr$_2$O$_3$ undergoes an antiferromagntic transition at $T_{N}\approx$~307~K. Above $T_N$, Cr$_2$O$_3$ possesses $\bar{3}m$ (D$_{3d}$) point group symmetry. Electric dipole SHG is forbidden in this state due to the presence of inversion symmetry. However, when the second harmonic energy is tuned to a Cr $d$-$d$ electronic transition ($^{4}\hspace{-0.1cm}A_{2g} \hspace{0.1cm} (t_{2g})^3 \hspace{-0.15cm}\rightarrow^4\hspace{-0.1cm}T_{2g} \hspace{0.1cm} (t_{2g})^2e_g$) at 2.1~eV (590~nm), we observe resonant magnetic dipole SHG (Fig.~\ref{fig:1}(c)-(d))~\cite{FiebigSHG}. As shown in Fig.~\ref{fig:1}(d), the RA pattern is consistent with the threefold symmetry of the crystal when the probe light propagates along the out-of-plane direction (and is polarized in-plane). %The RA-SHG pattern can be fit with a magnetic dipole source term of the following form (black line in Fig.~\ref{fig:1}(d)):
% \begin{equation}
%     \begin{split}
%     I(2\omega; T>T_N) \propto &|\chi^m\textrm{sin}(3\theta)|^2
%     \label{eq:RA-SHG_AboveTN}
%     \end{split}
% \end{equation}
%which contains a single fit parameter, $\chi^m$, the in-plane magnetic dipole second harmonic susceptibility. Here, $\omega$ is the frequency of the probe light and $\theta$ represents the angle of the incident and detected light polarization with respect to the sample's $y$-axis (Fig.~\ref{fig:1}~(b)).
% Figure environment removed 

Below $T_{N}$, Cr$_2$O$_3$ orders antiferromagnetically with the four Cr spins in the unit cell alternating in an up and down sequence along the rhombohedral optical axis (Fig.~\ref{fig:1}(a)). This spin structure breaks inversion symmetry, and the magnetic point group becomes $\underline{\bar{3}m}$~\cite{Birss}. Electric dipole SHG is then allowed (through the spin-orbit interaction) and interferes with the pre-existing magnetic dipole signal~\cite{SHGastool,FiebigSHG}. 
Due to the presence of both magnetic and electric dipole signals, the nodes present in the RA pattern above $T_N$ are lifted below $T_N$ (Fig.~\ref{fig:1}(e)). That the nodes are lifted implies a phase difference between the magnetic and electric dipole SHG amplitudes at the probed wavelength. The RA pattern below $T_N$ can be fit with a simple function that includes electric and magnetic dipole radiation (Fig.~\ref{fig:1}(e))~\cite{FiebigSHG, SHGastool}:
\begin{equation}
    I(2\omega_{pr}) \propto |e^{i\gamma} \chi^m\textrm{sin}(3\theta) \pm \chi^e\textrm{cos}(3\theta)|^2 \\
    \label{eq:RA-SHG}
\end{equation}
where $\chi^{e/m}$ is the in-plane electric/magnetic dipole second harmonic susceptibility, the $\pm$ depends on the AFM domain and $\gamma$ denotes the relative phase between magnetic and electric dipole second harmonic radiation (see Supplementary Note \RNum{1}). Here, $\omega_{pr}$ is the frequency of the probe light and $\theta$ represents the angle of the incident and detected light polarization with respect to the sample's $y$-axis (Fig.~\ref{fig:1}~(b)). Near but below $T_N$, the electric dipole SHG susceptibility, $\chi^e$, is proportional to the AFM order parameter, \textbf{L}~\cite{FiebigSHG}. Thus, $\chi^e$ differs in sign between the two AFM domains, which we denote $\alpha$ and $\beta$ (Fig.~\ref{fig:1}(a))~\cite{FiebigSHG,topography}. In Eq.~\ref{eq:RA-SHG}, it is also important to note that $\gamma$ depends sensitively on the second harmonic energy (Supplementary Note \RNum{6}). At the 2.1~eV second harmonic energy used here, $\gamma \approx$ 85$^\circ$, and the two domains exhibit almost identical RA patterns (left panels of Fig.~\ref{fig:2}(b) and (d)). This equivalence can be understood by noting that cross terms in Eq.~\ref{eq:RA-SHG} vanish when $\gamma = 90^\circ$ which eliminates the domain contrast. The interference between the electric and magnetic dipole SHG equips the crystal with an inherent phase sensitivity, which is key to observing the large symmetry-breaking response that we now demonstrate (Supplementary Note \RNum{6}).

We now move on to pump the Cr$_2$O$_3$ crystal with 1.2~eV (1030~nm) light pulses with fluences up to 20~mJ/cm$^2$. The pump polarization is aligned along the $x$-axis, as indicated with the double headed arrows in the right panels of Fig.~\ref{fig:2}(b) and (d). The wavelength of the pump light lies in the transparency window of the crystal and away from electronic resonances to avoid significant absorption (orange arrow in Fig.~\ref{fig:1}(c)). As the pump is transmitted through the crystal, we observe a drastic symmetry change in the RA pattern (right panels of Fig.~\ref{fig:2}(b) and (d)). Additionally, the two domains, which are almost indistinguishable with RA-SHG at equilibrium, exhibit differing responses to the pump. (The two domains are distinguishable at equilibrium with circularly polarized SHG~\cite{FiebigSHG, topography}). %With the SHG probe polarization fixed to $\theta=60^\circ$, the second harmonic intensity decreases by about 60\% in the $\beta$ domain, while the intensity increases by 30\% in the $\alpha$ domain (Fig.~\ref{fig:2}(a) and (c)). %In these figures, the pump polarization is aligned along the $x$-axis, as indicated with the double-headed orange arrow in the right panels of Fig.~\ref{fig:2}(b) and (d). 
This domain-dependent symmetry reduction is not observed above $T_N$ or when the pump light is circularly polarized; instead both the antiferromagnetism in addition to a well-defined pump polarization axis are necessary to observe this symmetry breaking effect in Cr$_2$O$_3$ (Supplementary Notes \RNum{3} and \RNum{5}).

% Figure environment removed 
Importantly, we observe this effect only when the pump and probe pulses are temporally overlapped; the timescale characterizing this transient symmetry breaking is on the order of the laser pulse width, $\sim$180~fs (Fig.~\ref{fig:2}(a) and (c)). (The slight asymmetry in the background levels of the time traces is attributed to two-photon absorption and is unrelated to the symmetry-breaking.) In the insets of Fig.~\ref{fig:2}(a) and (c), we show RA patterns 500~fs before and after the pump pulse propagates through the probed region as well as the pattern when the pump and probe pulses are perfectly overlapped (the latter are also shown in the right panels of Fig.~\ref{fig:2}(b) and (d)). Notably, there is not a measurable relaxation timescale for the symmetry change (apart from the pulse width), which suggests that neither the structural nor spin degrees of freedom bring about this reduction in rotational symmetry (Supplementary Note~\RNum{7}). Instead, the electronic charge degree of freedom is solely involved, and the system can be said to exhibit a nematic response.

%To determine how the symmetry breaking axis is chosen, we measure the dependence of the RA pattern on the pump polarization. Figure~\ref{fig:3}(a) reveals how the RA pattern in the $\beta$ domain varies as the pump polarization angle, $\varphi$, is rotated clockwise in 20$^\circ$ steps (double-headed orange arrows). As $\varphi$ is tuned, the shape and orientation of the RA pattern is correspondingly modified. However, every $\Delta\varphi=60^\circ$, the shape of the RA pattern reappears but is rotated clockwise by the same amount due to the crystal symmetry. These observations suggest that the modification of the RA pattern is due to a vector perturbation. Based on symmetry arguments, we show below that these observations are consistent with a light-induced polarization of the electronic subsystem; a polar nematic state emerges when the in-plane threefold symmetry is broken.
% Figure environment removed 

The transient nematicity has a polar character and originates from a light-induced electronic dipole moment via optical rectification. 
%Under this interpretation, the fits to the RA patterns are highly constrained. 
Under this interpretation, the electric/magnetic dipole susceptibility tensor describing SHG can be expanded to first order in the induced moment:
\begin{equation}
    \begin{split}
    \boldsymbol{\chi}^{e/m}_{SHG}(\textbf{\textit{P}}(\omega_0)) = \left.\boldsymbol{\chi}^{e/m}_{SHG}\right|_{\textbf{\textit{P}}=0} + \underbrace{\left.\frac{\partial\boldsymbol{\chi}_{SHG}^{e/m}}{\partial\textbf{\textit{P}}(\omega_0)}\right|_{\textbf{\textit{P}}=0}\hspace{-0.5cm}\textbf{\textit{P}}(\omega_0)}_{\equiv\delta\boldsymbol{\chi}^{e/m}_{SHG}}.
    \label{eq:cascade}
    \end{split}
\end{equation}
The rectified dipole moment, $\textbf{\textit{P}}({\omega_0})$, can be calculated based on the magnetic symmetry of the crystal, and we obtain the following expression (see Supplementary Note \RNum{2}A): 
\begin{equation}
 \textbf{\textit{P}}({\omega_0})=  \chi^{e}_{OR} I_{pump} (\textrm{sin}(-2\varphi) \hat{\textbf{x}}+ \textrm{cos}(-2\varphi)  \hat{\textbf{y}}),
\label{eq:DipoleMoment}
\end{equation} 
where $\chi^e_{OR}$ is the in-plane susceptibility associated with optical rectification, $I_{pump}$ is the intensity of the pump beam, and $\varphi$ represents the pump polarization angle with respect to the sample's $y$-axis.
By combining the effective susceptibility tensor (Eq.~\ref{eq:cascade}) with the expression for the pump induced polarization (Eq.~\ref{eq:DipoleMoment}), we can derive the full expression for the SHG intensity as function of the pump polarization angle for the two domains (Supplementary Note \RNum{2}B):
\begin{equation}
\begin{split}
I(2\omega_{pr},  \hspace{0.05cm}&\varphi) \propto |e^{i\gamma}\chi^m\textrm{sin}(3\theta) \pm \chi^e\textrm{cos}(3\theta) \\
&+ie^{i\gamma}\delta\chi^m\textrm{sin}(\theta-2\varphi) \pm i\delta\chi^e\textrm{cos}(\theta-2\varphi)|^2.
\label{eq:PerturbationPhase}
\end{split}
\end{equation}
Compared to the equilibrium case, only two additional independent terms are permitted to model the RA pattern: perturbations to the in-plane electric and magnetic dipole susceptibilities, $\delta\chi^e$ and $\delta\chi^m$, respectively. 
In Fig.~\ref{fig:3}(b), we demonstrate that Eq.~\ref{eq:PerturbationPhase} yields an excellent fit to the RA pattern when the pump polarization is along $x$-axis (i.e.  $\varphi=-90^\circ$). (In Supplementary Note \RNum{2}C, we explain how the combined parity and time-reversal symmetry accounts for the factor of $i$ multiplying the perturbative terms in Eq.~\ref{eq:PerturbationPhase}).


Our assignment of a polar nematic response is corroborated by two non-trivial predictions of our fit model. First, as the pump polarization angle, $\varphi$, is varied, the induced dipole moment is correspondingly rotated by an angle $-2\varphi$ (Eq.~\ref{eq:DipoleMoment} and \ref{eq:PerturbationPhase}). In Fig.~\ref{fig:3}(c), we illustrate this concept visually with the orange and red arrows which represent the orientation of the pump polarization and induced dipole moment, respectively. Once the RA pattern at $\varphi=-90^\circ$ is fit (Fig.~\ref{fig:3}(b)), the dependence on the pump polarization angle, $\varphi$, is completely determined by Eq.~\ref{eq:PerturbationPhase}. In Fig.~\ref{fig:3}(a), we show the dependence of the RA pattern on the pump polarization angle $\varphi$, while the corresponding results generated using Eq.~\ref{eq:PerturbationPhase} are shown in black. Even though the RA patterns change shape as the pump polarization is tuned, the black curves nonetheless show remarkable agreement with the data. We emphasize that these are not fits to the individual RA patterns; only the $\varphi =-90^\circ$ pattern is fit and the remaining black lines are generated with Eq.~\ref{eq:PerturbationPhase}, where only a single parameter, $\varphi$, is varied. Second, by choosing the opposite sign of $\chi^e$ and $\delta\chi^e$ in Eq.~\ref{eq:PerturbationPhase} (terms proportional to the order parameter near $T_N$), we obtain excellent agreement to the RA pattern of the opposite antiferromagnetic domain. Again, the black line in the right panel of Fig~\ref{fig:2}(b) is not a fit, but is generated from this procedure. This scheme implies that the two AFM domains exhibit opposite induced polarity. Our fit model thus corroborates the interpretation of an all-electronic polar nematic state induced through broken in-plane rotational symmetry. 

%To understand how the pump light induces an electronic symmetry breaking, we use a single-ion model of the Cr atoms in the crystal field environment. In Cr$_2$O$_3$, the Cr atoms are surrounded by a trigonally-distorted oxygen octahedron. The Cr 3$d^3$ electrons occupy the lower three $t_{2g}$ states, which are split off from the higher $e_g$ states by at least 2.1~eV (Fig.~\ref{fig:4}(a)). However, an electronic dipole moment cannot develop by considering only the Cr 3$d$ subspace. A necessary (but not sufficient) condition for observing a dipole is the superposition of states with opposite parity. Above $T_N$, the trigonal distortion of the oxygen octahedron breaks inversion symmetry at the Cr site locally and therefore permits a superposition of the Cr 4$p$ and 3$d$ states. However, because global inversion symmetry is present in the crystal, both an equilibrium dipole moment and the optical rectification process are forbidden. Below $T_N$, in contrast, global inversion symmetry is broken by the magnetic structure, but an equilibrium dipole moment is still disallowed because Cr$_2$O$_3$ retains two rotational symmetry axes (threefold along the $z$- and twofold along the $y$-axis; Fig.~\ref{fig:1}(a)). 
%A rectified response to a periodic drive, though, becomes possible.

The model itself does not explain how the symmetry is broken from a microscopic standpoint, however. To understand how this is accomplished, we appeal to the useful picture provided by Floquet theory~\cite{shirley, nonlinearFloquet}. In the leftmost panel of Fig.~\ref{fig:4}(a), we schematically illustrate the energy levels of the Cr atoms in the crystal field environment with the ground state  $\ket{g}$ and the excited states $\ket{e}$, and show examples of the corresponding in-plane probability densities.  When a periodic potential is applied to the system, a series of Floquet sidebands or ``dressed states" emerge, which we label $\ket{j, n}$, where $j$ labels the equilibrium state from which the $n^{th}$ Floquet sideband derives. In the middle panel of Fig.~\ref{fig:4}(a), we make the rotating wave approximation and only show a single dressed state of the excited state manifold, $\ket{e, -1}$. %Although this truncation works best when the drive is resonant between $\ket{g}$ and $\ket{e}$, this picture already provides a mechanism by which a static electronic dipole can be generated. 
In this effectively time-independent scheme, a superposition state, $\ket{g'} = \ket{g,0} + \lambda\ket{e,-1} + \lambda^2\ket{e,0}$ forms, where the numerator of $\lambda$ is given by matrix elements of the form $\bra{j',n\pm1}\boldsymbol{d}\cdot\boldsymbol{E}\ket{j,n}$, and the power of $\lambda$ indicates a first or second order perturbative correction to the ground state~\cite{shirley}. Here, $\boldsymbol{d}$ is the dipole operator and $\boldsymbol{E}$ is the time-independent electric field vector. Such matrix elements sensitively depend on the polarization of the incoming light. %Importantly, the superposition between $\ket{g,0}$ and $\ket{e,0}$ must occur via the $\ket{e,-1}$ sideband, due to the fact that the perturbation matrix elements can only change the Floquet index by $\pm$1. 


The new hybridized state, $\ket{g'}$, is capable of exhibiting a static dipole moment, which can be seen by calculating the expectation value $\bra{g'}\boldsymbol{d}\ket{g'}$ (Supplementary Note \RNum{9}). With the orange arrows in the middle panel of Fig.~\ref{fig:4}(a), we illustrate the two pathways through which the dipole can develop. The first path is shown with the dotted arrows and indicates that a dipole can arise due to terms of the form $\lambda^2\bra{g,0}\boldsymbol{d}\ket{e,0}\sim|\boldsymbol{E}|^2$. A second pathway, shown with the dashed arrows, yields terms of the form $\lambda^2\bra{e,-1}\boldsymbol{d}\ket{e,-1}\sim |\boldsymbol{E}|^2$. Selection rules associated with such matrix elements have been previously been calculated for Cr$_2$O$_3$ in Refs.~\cite{Cr2O3book, ME0, ME1, ME2}. We use the matrix elements therein to sketch representative probability densities for the first kind of process, which we show in the rightmost panel of Fig.~\ref{fig:4}(a). Clearly, the in-plane threefold symmetry of the crystal is broken and a finite dipole moment can be observed. (Supplementary Note \RNum{8} clarifies the symmetries of the sketched probability densities. Further details of the calculation using time-dependent perturbation theory and Floquet theory are presented in Supplementary Note \RNum{9}.)
%A second pathway, indicated with the dashed orange arrows, also gives rise to a static dipole moment. In this case, the expectation value of the dipole operator yields terms of the form $\lambda^2\bra{e,-1}\boldsymbol{d}\ket{e,-1}\sim |\boldsymbol{E}|^2$. Since the Floquet indices on either side of the matrix element are identical, the dipole moment is static. %Naively, one would expect that such a hybridization to give rise to a dipole oscillating at the frequency of the radiation field, but a static dipole can also arise when calculating the expectation value of the dipole operator, $\bra{\phi}\boldsymbol{d}\ket{\phi}$.
%Because multiple states are present in the excited state manifold, terms like these are allowed. 

%In both pathways, non-vanishing dipole matrix elements among the excited state degenerate subspace is crucial for the observation of the static dipole.
It is important to note that both pathways give rise to terms that scale with $\lambda^2 \sim |\boldsymbol{E}|^2$. Such a relation implies that the lowest order static electric dipole moment scales with $|\boldsymbol{E}|^2$ (i.e. fluence), which is confirmed experimentally in Fig.~\ref{fig:4}(b). This scaling is consistent with the second order nonlinear optical process producing a field conjugate to the polar nematic order parameter.


In conclusion, our experimental observations and simple theoretical account demonstrate how a symmetry breaking field can be applied solely to the electronic subsystem to induce polar nematicity. %The nonlinear light field conjugate to the electronic dipole moment induces a response fast enough to discriminate between the electronic and lattice subsystems. 
This experiment builds on previous work showing that purely electronic macroscopic effects in crystals can be brought about with laser pulses using a nonlinear optical protocol~\cite{Floquet, mciver, InverseFaraday, InverseCM, MnPS3, WS2, WS22, TaAs, TaAsprl}. 
%In principle, the generation of an electronic dipole moment in linear magnetoelectrics should produce a corresponding magnetic response; in compounds where the electronic component of the magnetoelectric effect is dominant, optical control of magnetism should be possible. 
Our work paves the way towards isolating and quantifying the electronic contribution to various effects in solids, engineering electronic phases with light, and observing optically rectified magnetoelectric and ferroelectric effects.


%\nolinenumbers
\footnotesize
\vspace{-0.75em}
 \section{Acknowledgements:}
We thank Mengxing Ye, Honglie Ning, Carina Belvin and Wesley Campbell for helpful conversations related to this work.
Research at UCLA was supported by the U.S. Department of Energy (DOE), Office of Science, Office of Basic Energy Sciences under Award No. DE-SC0023017 (experiment and theory). The work at Rutgers was supported by W. M. Keck Foundation (materials synthesis).  A.B.C. and R.R. acknowledge financial support from the University of California Laboratory Fees Research Program funded by the UC Office of the President (UCOP), grant number LFR-20-653926.  A.B.C acknowledges financial support from the Joseph P. Rudnick Prize Postdoctoral Fellowship (UCLA).

\section{Author contributions:  }
X.Z. and T.C. built the SHG setup and performed the time-resolved SHG experiments under the supervision of A.K. X.Z. analysed the data under the supervision of A.K. K.D. and K.W. grew the single crystals under the supervision of S.-W.C. Theoretical calculations were carried out by A.B.C. under the supervision of R.R. The manuscript was written by X.Z., A.B.C. and A.K. with input from all authors.
\vspace{-0.75em}

\section{Competing interests:  }

The authors declare no competing interests.


\section{Methods }
\vspace{-1em}
\subsection{Sample synthesis}
Cr$_2$O$_3$ single crystals were grown using a laser diode heated floating zone (LFZ) technique. Cr$_2$O$_3$ powders (Alfa Aesar, $99.99\%$) were pressed into 3 mm diameter rods under 8000 PSI hydrostatic pressure. The compressed rod was sintered at 1600$^{\circ}$C in a box furnace for 10~hours. The crystals were grown with growth speed of 2 to 4 mm/h in oxygen flow of 0.1 l/min, and counter rotation of the feed and seed rods at 15 and 15 rpm, respectively.

\subsection{Experimental details}
The regeneratively amplified laser used in our experiment is based on a Yb:KGW gain medium that outputs a power of 10~W. The laser pulses have a Gaussian-like profile with an approximately 180~fs pulse duration and a 1030~nm central wavelength. In our experiment, we used a laser pulse repetition rate of 5~kHz. The fundamental output of the laser at 1030~nm was used as the pump pulse, which was focused obliquely on the sample at a 10 degree angle of incidence. The pump laser spot size was $\sim$ 500~$\mu$m and the maximum fluence was $\sim$ 20 $\mathrm{mJ}/\mathrm{cm}^{2}$. The probe pulse was generated from an optical parametric amplifier with tunable wavelength, which we use for the second harmonic spectroscopy between 900-1200~nm. The probe pulse was focused normally on the sample with a 100~$\mu$m spot size, and the probe fluence was $\sim$~2~$\mathrm{mJ}/\mathrm{cm}^{2}$. Detection of the second harmonic light was conducted with a commercial photo-multiplier tube. The sample was cooled to 150~K with a standard optical cryostat with fused silica windows to prevent distortions to the light polarization.


 
 \section{Data availability}
 \vspace{-1em}
 The data that supports the findings of this study are present in the paper and/or in the supplementary information, and are deposited in the Zenodo repository.  Additional data related to the paper is available from the corresponding authors upon reasonable request.
\begin{thebibliography}{1}
\bibitem{symmetry1}    Powell, R.  Symmetry, group theory, and the physical properties of crystals \textit{Springer 
}  (2010).
\bibitem{kivelson} Kivelson, S.A., Fradkin, E. $\&$ Emery, V. Electronic liquid-crystal phases of a doped Mott insulator. \textit{Nature} \textbf{393}, 550-553 (1998).
\bibitem{eisenstein} Lilly, M.P. et al. Evidence for an anisotropic state of two-dimensional electrons in high Landau levels. \textit{Phys. Rev. Lett.} \textbf{82}, 394 (1999).
\bibitem{feldman} Feldman, B.E. et al. Observation of a nematic quantum Hall liquid on the surface of bismuth. \textit{Science} \textbf{354}, 316-321 (2016).
\bibitem{mackenzie} Borzi, R.A. et. al. Formation of a nematic fluid at high fields in Sr$_3$Ru$_2$O$_7$ at high magnetic fields. \textit{Science} \textbf{315}, 214-217 (2007).
\bibitem{harter} Harter, J. et al. A parity breaking electronic nematic phase transition in the spin orbit coupled metal Cd$_2$Re$_2$O$_7$ \textit{Science} \textbf{356}, 295-299 (2017).
\bibitem{bozovic} Wu, J. et al. Spontaneous breaking of rotational symmetry in copper oxide superconductors. \textit{Nature} \textbf{547}, 432-435 (2017).
\bibitem{fisher} Chu, J.-H. et al. Divergent nematic susceptibility in an iron arsenide superconductor. \textit{Science} \textbf{337}, 710-712 (2012).
\bibitem{bozovic2} Wu. J. et al. Electronic nematicity in Sr$_2$RuO$_4$. \textit{Proc. Natl. Acad. Sci.} \textbf{117}, 10654-10659 (2020).
\bibitem{moll} Ronning, F. et al. Electronic in-plane symmetry breaking at field-tuned quantum criticality in CeRhIn$_5$. \textit{Nature} \textbf{548}, 313-317 (2017).
\bibitem{matsuda} Okazaki, R. et al. Rotational symmetry breaking in the hidden-order phase of URu$_2$Si$_2$. \textit{Science} \textbf{331}, 439-442
(2011)
%\bibitem{symmetry2}   Landau, L  Electrodynamics of continuous media  \textit{elsevier}  (2013).
\bibitem{TaAs}   Sirica, N. et al.  Photocurrent-driven transient symmetry breaking in the Weyl semimetal TaAs. \textit{Nat. Mater.} \textbf{ 21, } 62–66 (2022).
\bibitem{TaAsprl}   Sirica, N. et al. Tracking Ultrafast Photocurrents in the Weyl Semimetal TaAs Using THz Emission Spectroscopy. \textit{Phys. Rev. Lett. } \textbf{ 122, } 197401 (2019).

\bibitem{polarization1}    Khomskii, D.  Transition metal compounds  \textit{Cambridge University Press 
}  (2014).
\bibitem{polarization2}    Bonfim, O. $\&$ Gehring, G. Magnetoelectric effect in antiferromagnetic crystals  \textit{Advances in Physics 
}  \textbf{29,} 731-769 (1980).
\bibitem{reviewME1}   Spaldin, N.  $\&$  Ramesh, R.  Advances in magnetoelectric multiferroics \textit{Nat. Mater.} \textbf{18,}   203–212 (2019).
\bibitem{reviewME2}  Spaldin, N.  $\&$  Fiebig, M.   The Renaissance of Magnetoelectric Multiferroics \textit{Science } \textbf{309,}   391-392 (2005).
\bibitem{reviewME3}   Cheong, S. $\&$  Mostovoy, M.  Multiferroics: a magnetic twist for ferroelectricity \textit{Nat. Mater. } \textbf{6,}  13–20 (2007).
\bibitem{contribution1}   Malashevich, A. et al.  Full magnetoelectric response of Cr$_2$O$_3$ from first principles \textit{Phys. Rev. B } \textbf{86,}   094430 (2012).
\bibitem{contribution2}   Bousquet, E.,  Spaldin, N. $\&$  Delaney, K.  Unexpectedly Large Electronic Contribution to Linear Magnetoelectricity \textit{Phys. Rev. Lett. } \textbf{106,}   107202 (2011).

%\bibitem{Yihua}  Wang, Y. et al. Observation of Floquet-Bloch States on the Surface of a Topological Insulator \textit{Science} \textbf{342,    } 453–457 (2013).
%\bibitem{Fahad}  Mahmood, F. et al. Selective scattering between Floquet–Bloch and Volkov states in a topological insulator. \textit{Nat. Phys.} \textbf{12,    } 306–310 (2016).

\bibitem{Boyd}   Boyd, R.  Nonlinear optics. \textit{Academic Press } (2020).

\bibitem{OR}  Bass, M. et al. Optical Rectification. \textit{Phys. Rev. Lett.} \textbf{9}, 446 (1962).

%\bibitem{RASHG2}   Zhao, L. et al. A global inversion-symmetry-broken phase inside the pseudogap region of YBa$_2$Cu$_3$O$_y$ \textit{Nat. Phys.} \textbf{ 13,   } 250–254 (2017)
%\bibitem{RASHG3}    Wu, L. et al. Giant anisotropic nonlinear optical response in transition metal monopnictide Weyl semimetals \textit{Nat. Phys.} \textbf{ 13,  }350–355 (2017).

%\bibitem{controlCr2O3}   He, X. et al. Robust isothermal electric control of exchange bias at room temperature \textit{Nat. Mater.} \textbf{9,  }579–585 (2010).
\bibitem{SHGastool}  Fiebig, M.,  Pavlov, V.  $\&$ Pisarev, R.   Second-harmonic generation as a tool for studying electronic and magnetic structures of crystals: review \textit{J. Opt. Soc. Am. B} \textbf{22,    } 1, 96-118 (2005).
\bibitem{FiebigSHG}  Fiebig, M. et al.   Second Harmonic Generation and Magnetic-Dipole —Electric-Dipole Interference in Antiferromagnetic Cr$_2$O$_3$ \textit{Phys. Rev. Lett.} \textbf{73}, 2127 (1994).
\bibitem{topography}    Fiebig, M.,Fröhlich, D. $\&$   Sluyterman, G. Domain topography of antiferromagnetic Cr$_2$O$_3$  by second‐harmonic generation. \textit{Appl. Phys. Lett.} \textbf{ 66},  2906 (1995).
%\bibitem{ME3}    Krichevtsov, B. et al.   Magnetoelectric Spectroscopy of Electronic Transitions in Antiferromagnetic  Cr$_2$O$_3$ \textit{Phys. Rev. Lett.} \textbf{76, }  4628  (1996).
%\bibitem{ME4}    Malashevich, A. et al.  Full magnetoelectric response of  Cr$_2$O$_3$  from first principles \textit{Phys. Rev. Lett.} \textbf{ 86, } 094430  (2012).
%\bibitem{ME5}    Iyama, A. et al.  Magnetoelectric hysteresis loops in Cr$_2$O$_3$ at room temperature \textit{Phys. Rev. Lett.} \textbf{ 87, } 180408  (2013).
%\bibitem{ME6}    Pisarev, R.  Crystal optics of magnetoelectrics \textit{Ferroelectrics} \textbf{ 162,  } 191-209  (1994).
%\bibitem{ME7}    Hayashida, T. et al.  Observation of antiferromagnetic domains in 
%Cr$_2$O$_3$  using nonreciprocal optical effects \textit{Phys. Rev. Res.} \textbf{ 4, } 043063  (2022).
\bibitem{timeCr2O3}  Satoh, T. et al.  Ultrafast spin and lattice dynamics in antiferromagnetic Cr$_2$O$_3$ \textit{Phys. Rev. B} \textbf{ 75}, 155406 (2007).
\bibitem{timeCr2O32}  Satoh, T. et al.  Time-resolved demagnetization in  Cr$_2$O$_3$ by phase sensitive second harmonic generation \textit{Phys. Rev. B} \textbf{310}, 1604-1606 (2007).
\bibitem{timeCr2O3wall}  Sala, V. et al.  Resonant optical control of the structural distortions that drive ultrafast demagnetization in  Cr$_2$O$_3$  \textit{Phys. Rev. B} \textbf{94}, 014430 (2015).

\bibitem{Birss} Birss, R. Symmetry and Magnetism. \textit{North Holland},  (1966).

\bibitem{shirley}  Shirley, J. Solution of the Schrodinger equation with a Hamiltonian periodic in time \textit{Phys. Rev} \textbf{138,  } 4B (1965).
\bibitem{nonlinearFloquet}  Morimoto, T.  $\&$ Nagaosa, N. Topological nature of nonlinear optical effects in solids. \textit{Sci. Adv.} \textbf{2,  } 5 (2016).

\bibitem{Cr2O3book}  Sugano, S. $\&$ Kojima, N.    Magneto-optics \textit{Springer} \textbf{128,    }  (2013).
\bibitem{ME0}   Muthukumar, V.,   Valentí, R.  $\&$  Gros, C.  Microscopic Model of Nonreciprocal Optical Effects in  Cr$_2$O$_3$  \textit{Phys. Rev. Lett.} \textbf{ 75, 2766 } 2766  (1995).
\bibitem{ME1}   Muto, M. et al.   Magnetoelectric and second-harmonic spectra in antiferromagnetic  Cr$_2$O$_3$ \textit{Phys. Rev. B  } \textbf{57,  } 9586 (1998).
\bibitem{ME2}    Muthukumar, V.,   Valentí, R.  $\&$  Gros, C.   Theory of nonreciprocal optical effects in antiferromagnets: The case of  Cr$_2$O$_3$ \textit{Phys. Rev. B  } \textbf{54,   } 433 (1996).

\bibitem{InverseCM} Pershan, P.S., van der Ziel, J.P. $\&$ Malmstrom, L.D. \textit{Phys. Rev.} \textbf{143}, 574-583 (1966).
\bibitem{InverseFaraday}    Kimel, A. et al. Ultrafast non-thermal control of magnetization by instantaneous photomagnetic pulses. \textit{Nature} \textbf{435,   }655–657 (2005).
\bibitem{MnPS3}  Shan, J. et al. Giant modulation of optical nonlinearity by Floquet engineering \textit{Nature} \textbf{ 600, } 235–239 (2021).
\bibitem{WS2}   Sie, E. et al. Valley-selective optical Stark effect in monolayer WS$_2$. \textit{Nat. Mater.} \textbf{14, } 290–294 (2015).
\bibitem{WS22}   Sie, E. et al. Large, valley-exclusive Bloch-Siegert shift in monolayer WS$_2$. \textit{Science} \textbf{355,  } 6329 (2015).
\bibitem{mciver} McIver, J.W. et. al. Light-induced anomalous Hall effect in graphene. \textit{Nat. Phys.}, \textbf{16} 38-41 (2020). 
\bibitem{Floquet}  Oka, T. $\&$ Kitamura, S.  Floquet Engineering of quantum materials. \textit{Annu. Rev. Condens. Matter Phys.} \textbf{10,}  387-408 (2019).


%\bibitem{magnetism1}   Kirilyuk, A.,  Kimel, A. $\&$  Rasing, T. Ultrafast optical manipulation of magnetic order  \textit{Rev. Mod. Phys.} \textbf{82,    } 2731 (2010).
%\bibitem{type2MEa}  Khomskii, D. Classifying multiferroics: mechanisms and effects. \textit{Physics } \textbf{2,    } 20 (2009).
%\bibitem{type2MEb}   Tokura, Y., Seki, S. $\&$ Nagaosa, N. Multiferroics of spin origin. \textit{Rep. Prog. Phys.} \textbf{77,    } 076501 (2014)

%\bibitem{magnetism2}  Stanciu, C. et al.  All-Optical Magnetic Recording with Circularly Polarized Light \textit{Phys. Rev. Lett. } \textbf{99,   }047601 (2007).
%\bibitem{fatigue}    Lebeugle, D. et al.  Very large spontaneous electric polarization in BiFeO$_3$ single crystals at room temperature
%and its evolution under cycling fields \textit{Appl. Phys. Lett.} \textbf{91,}  022907 (2007)
%\bibitem{spectrum1}    McClure, D.  Comparison of the Crystal Fields and Optical Spectra of Cr$_2$O$_3$  and Ruby \textit{J. Chem. Phys.} \textbf{ 38, }2289 (1963).
%\bibitem{spectrum2}    Allen, J.,  MACFARLANE, R. $\&$  WHITE, R.  Magnetic Davydov Splittings in the Optical Absorption Spectrum of  Cr$_2$O$_3$  \textit{Phys. Rev. } \textbf{ 179, }523 (1969).

%\bibitem{exponent1}   Murtazaev, A.  Critical properties of the model of antiferromagnet Cr$_2$O$_3$ \textit{Low Temperature Physics  } \textbf{25,   } 344 (1999);
%\bibitem{exponent2}    Al-Mahdawi, M. et al. Apparent critical behaviour of sputter-deposited magnetoelectric antiferromagnetic Cr$_2$O$_3$ films near Néel temperature \textit{J. Phys. D: Appl. Phys.} \textbf{ 50,  }155004 (2017).


%\bibitem{THz1}  Nahata, A. et al. A wideband coherent terahertz spectroscopy system using optical rectification and electro‐optic sampling. \textit{Appl. Phys. Lett.  } \textbf{ 90,  }  171121 (2007).
%\bibitem{THz2}   Yeh, K. et al. Generation of 10$\mu$J ultrashort terahertz pulses by optical rectification. \textit{Appl. Phys. Lett.} \textbf{69,  } 2321 (1996).


\end{thebibliography}






\normalsize

 
\end{document}
 
