%\usepackage{amssymb}
%\usepackage{amsmath}
%\usepackage{commath}
%\usepackage{graphicx}
%\usepackage{verbatim}
%\usepackage{epsfig}
%\usepackage{float}
%\usepackage{epstopdf}
%\usepackage{color}
%\usepackage{siunitx}
%\usepackage{array}
%\usepackage{physics}
%\newcolumntype{P}[1]{>{\centering\arraybackslash}p{#1}}
%\newcolumntype{M}[1]{>{\centering\arraybackslash}m{#1}}
%\renewcommand{\thefigure}{S\arabic{figure}}
%\newcommand{\bl}[1]{{\color{blue} #1}}
%\setlength{\extrarowheight}{5px}
%\setlength{\parskip}{0pt}
%\RequirePackage{lineno}
%\def\CTeXPreproc{Created by ctex v0.2.12, don't edit!}%\documentclass[aps,preprint,groupedaddress]{revtex4}
%\documentclass[aps,preprint,superscriptaddress,showpacs]{revtex4}
%\documentclass[aps,twocolumn,superscriptaddress,showpacs]{revtex4}
\documentclass[rmp,aps,amsfonts,amsmath,amssymb,nofootinbib,superscriptaddress]{revtex4}
%\documentclass[aps,twocolumn,prl,preprintnumbers,amsmath,amssymb,superscriptaddress]{revtex4-1}
%\documentclass[aps,twocolumn,prb,preprintnumbers,amsmath,amssymb,superscriptaddress,floats]{revtex4}
%\documentclass[preprint,showpacs,preprintnumbers,amsmath,amssymb]{revtex4}
%\documentclass[preprint,showpacs,superscriptaddress,preprintnumbers,amsmath,amssymb]{revtex4}

 

\usepackage{graphicx}% Include figure files
%usepackage{dcolumn}% Align table columns on decimal point
\usepackage{bm}% bold math
\usepackage{hyperref}
%\usepackage{IEEEtrantools}
\bibliographystyle{naturemag}
\renewcommand{\thefigure}{S\arabic{figure}}
\renewcommand{\theequation}{S\arabic{equation}}

\usepackage{mathtools}
\usepackage{upgreek}
\usepackage{natbib}
\usepackage{float}
\usepackage{braket}
\restylefloat{table}

%% ABC
\usepackage{physics}
%% ABC

%\bibliographystyle{apsrev}
%\nofiles


\begin{document}

%\linenumbers
%\title{An ideal Cobalt based Kitaev spin liquid }

%% ABC
\newcommand{\pd}{\partial}
\newcommand{\beq}{\begin{equation}}
\newcommand{\eeq}{\end{equation}}
\newcommand{\bseq}{\begin{subequations}}
\newcommand{\eseq}{\end{subequations}}
\newcommand{\bpmat}{\begin{pmatrix}}
\newcommand{\epmat}{\end{pmatrix}}
%% ABC\
%\nolinenumbers
\title{Supplementary Information for Light-induced polar nematicity in antiferromagnetic Cr$_2$O$_3$}
%\title{Light induced electronic symmetry change in Cr$_2$O$_3$ }
\author{Xinshu Zhang}

\affiliation{Department of Physics and Astronomy, University of California Los Angeles, Los Angeles, CA 90095, USA}

\author{Tyler Carbin}
\affiliation{Department of Physics and Astronomy, University of California Los Angeles, Los Angeles, CA 90095, USA}

\author{Adrian B. Culver}
\affiliation{Department of Physics and Astronomy, University of California Los Angeles, Los Angeles, CA 90095, USA}
\affiliation{Mani L. Bhaumik Institute for Theoretical Physics, Department of Physics and Astronomy, University of California Los Angeles, Los Angeles, CA 90095, USA}

\author{Kai Du}
\affiliation{Rutgers Center for Emergent Materials, Rutgers University, Piscataway, NJ, USA}


\author{Kefeng Wang}
\affiliation{Rutgers Center for Emergent Materials, Rutgers University, Piscataway, NJ, USA}


\author{Sang-Wook Cheong }
\affiliation{Rutgers Center for Emergent Materials, Rutgers University, Piscataway, NJ, USA}
\author{Rahul Roy}
\affiliation{Department of Physics and Astronomy, University of California Los Angeles, Los Angeles, CA 90095, USA}
\author{Anshul Kogar}
\email{anshulkogar@physics.ucla.edu}
\affiliation{Department of Physics and Astronomy, University of California Los Angeles, Los Angeles, CA 90095, USA}

\date{\today}


\maketitle
%\vfill


%\tableofcontents

%\linenumbers
\section{Analysis of equilibrium SHG }
\subsection{Rotational anisotropy SHG}
Radiation can come from various sources, including electric dipoles, magnetic dipoles, electric quadrupoles, and higher order radiation sources. In Cr$_2$O$_3$, magnetic and electric dipole radiation are known to dominate the response~\cite{Cr2O3book}. The total source term entering Maxwell's equations is given by 
\begin{equation}
    \boldsymbol{S}=\left(\nabla \times \frac{\partial  \boldsymbol{M}}{\partial t}+\frac{\partial^{2}  \boldsymbol{P}}{\partial t^{2}}\right).
\end{equation}
The second harmonic intensity measured by the detector in our experiment is $I \propto |\textbf{e}(2\omega)\cdot \boldsymbol{S} |^2$, where  $\textbf{e}(2\omega)$ is the unit vector of the output polarizer that selects second harmonic light of a particular polarization. The magnetization at the second harmonic frequency is allowed at all temperatures and has the form of $\boldsymbol{M}(2\omega)=\boldsymbol{\chi}^{m}\boldsymbol{E}(\omega)\boldsymbol{E}(\omega)$, where $\boldsymbol{\chi}^{m}$ is the second order nonlinear magnetic susceptibility tensor. The electric polarization at the second harmonic frequency is allowed only below $T_{N}$ and is given by $\boldsymbol{P}(2\omega)=\boldsymbol{\chi}^{e}\boldsymbol{E}(\omega)\boldsymbol{E}(\omega)$.  Below $T_N$, we can write the SHG intensity as 
$I \propto |M(\underline{\bar{3}m})+P(\underline{\bar{3}m})|^2$, where $M(\underline{\bar{3}m})$ and $P(\underline{\bar{3}m})$ represent the magnetic and electric dipole contributions to SHG signal at a particular polarization angle. Additionally, $\underline{\bar{3}m}$ is the magnetic point group of the crystal. For light along $z$-direction (optical axis of the crystal), the electric and magnetic susceptibility tensors have the same independent tensor elements with the relation $\chi_{yyy}$=$-\chi_{yxx}$=$-\chi_{xyx}$=$-\chi_{xxy}$=$\chi^{e/m}$. By taking the polarization of both the incident fundamental light and the outgoing SHG light into account, we obtain the magnetic dipole contribution, $M(\underline{\bar{3}m}) \propto \chi^{m}\textrm{sin}(3\theta)$, and the electric dipole contribution, $P(\underline{\bar{3}m}) \propto \chi^{e}\textrm{cos}(3\theta)$, where $\theta$ is taken to be with respect to the $y$-axis. The maximum of the magnetic contribution, $M(\underline{\bar{3}m})$, is rotated by 90 degrees with respect to the electric contribution, $P(\underline{\bar{3}m})$, because of the curl operation in the source term. Above $T_{N}$, the SHG data can be described by just magnetic contribution, while below $T_N$, both electric and magnetic contributions are required to fit the data.  $\chi^{m}$ and $\chi^{e}$ are complex quantities, but only the relative phase between them is important. The relative phase can be obtained by two methods. First, it can be obtained directly from the fit of the rotational anisotropy second harmonic generation (RA-SHG) patterns. The electric and magnetic contribution have roughly an 85 degree phase difference at 1180 nm (the wavelength used in the main manuscript) due to local field effects.  The fact that the nodes are lifted below $T_N$ is due to this relative phase between the two contributions. Second, the phase can be computed using intensity of the left and right circularly polarized second harmonic emission in combination with intensity of the linearly polarized second harmonic emission as shown in the following section. 


\subsection{Left and right circular probe}

In this section, we show the two AFM domains can be distinguished at equilibrium by using left and right circularly polarized incident light. Compared to rotational anisotropy SHG measurements, here we replace the polarizer for the incident light with a quarter wave plate that can generate circularly polarized light. We also remove the analyzer for the second harmonic light. The source term in linear basis is given by

\begin{equation*}
\boldsymbol{S} \propto \left(\nabla \times \frac{\partial \boldsymbol{M}}{\partial t}+\frac{\partial^{2} \boldsymbol{P}}{\partial t^{2}}\right)\propto 
\begin{pmatrix}
\chi^{m}(E_{x}^{2}-E_{y}^{2})+2\chi^{e}E_{x}E_{y}  \\
\chi^{e}(E_{x}^{2}-E_{y}^{2})-2\chi^{m}E_{x}E_{y}  \\
0 
\end{pmatrix}
\end{equation*}
We can convert the source term into circular basis with $\boldsymbol{E}=E_{R}\textbf{e}_{R}+E_{L}\textbf{e}_{L}+E_{z}\textbf{e}_{z}$, where $
 \textbf{e}_{R}=-1/\sqrt{2}(\textbf{e}_{x}+i \textbf{e}_{y})$ and $\textbf{e}_{L}=1/\sqrt{2}(\textbf{e}_{x}-i \textbf{e}_{y})$. Then one gets 
 
 \begin{equation*}
\boldsymbol{S}=\begin{pmatrix}
S_{R}\\
S_{L}\\
S_{z}
\end{pmatrix}
\propto  
\begin{pmatrix}
(-\chi^{m}+i\chi^{e}) E_{L}^{2} \\
(\chi^{m}+i\chi^{e}) E_{R}^{2}  \\
0 
\end{pmatrix}
\end{equation*}
 $ I_{R/L} \propto  |\boldsymbol{S}|^{2}$. Hence, the SHG intensity is 
 \begin{equation}
    I_{R/L} \propto  \left(|\chi^{m}|^2+|\chi^{e}|^2\right)\mp 2\left(|\chi^{m}| |\chi^{e}|\right)\textrm{sin}(\gamma ), 
    \label{eq:S1}
   \end{equation} where $\gamma$ is the phase difference between $\chi^{m}$ and $\chi^{e}$. The first term is quadratic in $\chi^{m}$ and $\chi^{e}$ and is
always positive, while the second term is the so-called interference term and may be positive or negative. The sign of the interference term can be switched either by changing the handedness of the circular light or by probing the opposite antiferromagnetic domain. Additionally, the interference is also affected by the wavelength of the probe. The largest contrast in the second harmonic response to the incident left and right circularly polarized light occurs when the magnitude of $\chi^{m}$ and $\chi^{e}$ are the same and possess a 90 degree phase difference.

\subsection{Computing the relative phase}
The relative phase between  $\chi^{m}$ and $\chi^{e}$ can be obtained either from directly fitting the SHG pattern or  derived from Eq.~\ref{eq:S1} as \begin{equation}
    \gamma = \textrm{sin}^{-1}\left(\frac{I_R - I_L}{I_R+I_L}\cdot\frac{I_{M}+I_{E}}{2\sqrt{I_{M}I_{E}}}\right).
    \label{eq:phase}
\end{equation}
Here, the $I_R$ and $I_L$ denote the SHG intensities due to right and left circularly polarized light, and $I_{M}$ and $I_{E}$ denote the SHG intensities due to the magnetic dipole and electric dipole radiation (i.e.$\propto |\chi^{m}|^2$ and $|\chi^{e}|^2$), respectively. The phase obtained using both this method and fitting the RA-SHG patterns are consistent.


\section{analysis of non-equilibrium SHG }
\subsection{Derivation of the $\mathbf{-2\phi}$ in optical rectification process}

The optical rectification caused by the pump is given by $\boldsymbol{P}(\omega_{0})=\boldsymbol{\chi}^{e}_{OR}\boldsymbol{E}^{p}(\omega)\boldsymbol{E}^{p}(-\omega)$.  Under the magnetic point group symmetry $\underline{\bar{3}m}$, the relevant susceptibility tensor elements are $\chi_{yyy}$=$-\chi_{yxx}$=$-\chi_{xyx}$=$-\chi_{xxy}$=$\chi^{e}_{OR}$.
The light induced polarization can be decomposed into $x$ and $y$ components $\boldsymbol{P}(\omega_{0})=P_{x} \textbf{e}_{x}+P_{y} \textbf{e}_{y}$ with $P_{x}=2\chi_{xyx}E_{y}E_{x}$ and $P_{y}=\chi_{yyy}E_{y}E_{y}+\chi_{yxx}E_{x}E_{x}$. For a linearly polarized pump we have $E_{x}=E^{p} \textrm{sin}\phi$ and $E_{y}=E^{p}\textrm{cos}\phi$, where the $E^{p}$ is the magnitude of the electric field from the pump and the $\phi$ is the angle between the pump polarization and the $y$-axis, as defined in the main text.  Therefore, we obtain
\begin{equation}
 \boldsymbol{P}(\omega_{0})=\chi^{e}_{OR}  (E^{p})^2 [\textrm{sin}(-2\phi)\textbf{e}_{x}+ \textrm{cos}(-2\phi)\textbf{e}_{y}]
\label{eq:2}
\end{equation}

\subsection{Functional form of pump induced electric and magnetic dipole contributions to the second harmonic intensity}
The pump induced nonlinear magnetization and polarization are given by 
\begin{equation} 
\begin{split}
 \boldsymbol{M}^{p}(2\omega)= &  \boldsymbol{\chi}^{m} \boldsymbol{E}(\omega) \boldsymbol{E}(\omega) \boldsymbol{P}(\omega_{0})\\
 \boldsymbol{P}^{p}(2\omega)= &  \boldsymbol{\chi}^{e} \boldsymbol{E}(\omega) \boldsymbol{E}(\omega) \boldsymbol{P}(\omega_{0}),
\end{split}
\end{equation} where the $\boldsymbol{\chi}^{e/m}$ are third order susceptibility tensors constrained by the $\underline{\bar{3}m}$ symmetry. The pump gives rise to additional contributions to the second order susceptibility tensors $\delta \chi_{ijk}^{m}=\chi_{ijkl}^{m}P_{l}(\omega_{0})$ and $\delta \chi_{ijk}^{e}=\chi_{ijkl}^{e}P_{l}(\omega_{0})$ that modify the equilibrium tensors, $\chi_{ijk}^{m}$ and $\chi_{ijk}^{e}$. When the pump and probe are normally incident on the sample, only eight matrix elements of the second order tensor, $\chi_{ijk}$, are experimentally accessible. The tensor elements related to the $z$ components can be ignored. We can then write the effective equilibrium second order susceptibility tensor as:

\begin{equation*}
      \boldsymbol{\chi}^{e/m} = \begin{pmatrix} 0 & -\chi^{e/m}  \\
            -\chi^{e/m} & 0    \\
            -\chi^{e/m} & 0 \\
            0 & \chi^{e/m}\\
            \end{pmatrix}  \\
\end{equation*}
When considering the effect of the pump, we therefore need to consider only eight tensor elements of $\chi^{e/m}_{ijkl}$:

\begin{equation*}
\left\lbrace \begin{array}{l}
\chi_{xxyy}=\chi_{yyxx}\\
\chi_{xyyx}=\chi_{yxxy}\\
\chi_{xyxy}=\chi_{yxyx}\\
\chi_{xxxx}=\chi_{yyyy}=\chi_{xxyy}+\chi_{xyyx}+\chi_{xyxy}
\end{array}\right.\end{equation*}
The perturbation due to induced polarization can then be included into an effective second order susceptibility tensor by writing:

\begin{align*}
       \delta\boldsymbol{\chi}^{e/m} = \begin{pmatrix} \chi_{xxxx}\cdot P_{x} & \chi_{xyxy}\cdot P_{y}  \\
            \chi_{xxyy}\cdot P_{y} & \chi_{xyyx}\cdot P_{x}    \\
             \chi_{yxxy}\cdot P_{y} & \chi_{yyxx}\cdot P_{x} \\
       \chi_{yxyx}\cdot P_{x} & \chi_{yyyy}\cdot P_{y}  \\
       \end{pmatrix}\\
\end{align*} 
%where  the indices in $\chi_{ijk}$ run over $x,y$  and tensor elements including $z$ component are ignored as the light is propagating along $z$ direction.  
From the previous section, we have  $\boldsymbol{P}(\omega_{0})=P_{x} \textbf{e}_{x}+P_{y} \textbf{e}_{y}=\chi^{e}_{OR}  (E^{p})^2 [\textrm{sin}(-2\phi)\textbf{e}_{x}+ \textrm{cos}(-2\phi)\textbf{e}_{y}]$. Then we obtain the  $M(C_{2}) \propto \delta\chi^{m}\textrm{sin}(\theta-2\phi)$ and $P(C_{2}) \propto \delta\chi^{e}\textrm{cos}(\theta-2\phi)$ as mentioned in the main text, where $\delta\chi^{m} \propto \chi^{m}_{yyyy}\chi^{e}_{OR}(E^{p})^{2}$ and $\delta\chi^{e} \propto \chi^{e}_{yyyy}\chi^{e}_{OR}(E^{p})^{2}$.
In summary, we have 
\begin{equation}
\begin{split}
I(2\omega; &\varphi) \propto |e^{i\gamma}\chi^m\textrm{sin}(3\theta) + \chi^e\textrm{cos}(3\theta) \\
&+ie^{i\gamma}\delta\chi^m\textrm{sin}(\theta-2\varphi) + i\delta\chi^e\textrm{cos}(\theta-2\varphi)|^2.
\label{eq:PerturbationPhase}
\end{split}
\end{equation}

\subsection{Property of susceptibility tensors and complexity of each contribution}

For physical fields, $A(t)$, we have $\Theta A(t)=\pm A(t)$, where $\Theta$ represents the time reversal operator.  The ``$+$'' sign corresponds to i-type fields and ``$-$'' sign to c-type fields. In the Fourier domain, we get the corresponding relationship: $\Theta A(\omega)=\pm A^{*}(\omega)$. In Cr$_{2}$O$_{3}$, below $T_N$, inversion and time reversal symmetry are both broken by the antiferromagnetic
arrangement of spins, but the combined symmetry of inversion and time reversal remains
a symmetry element \textit{in the absence of dissipation}. Hence we have 
\begin{equation}
 \Theta \mathbf{I} \boldsymbol{\chi}^{e/m}=\boldsymbol{\chi}^{e/m},
\label{eq:TI}
\end{equation} where $\Theta$ and $\mathbf{I}$ are the time reversal and inversion operators, respectively. Applying the combined time reversal and inversion operators  $\Theta \mathbf{I}$ to $\boldsymbol{P}(\omega_0)=\boldsymbol{\chi}^{e}_{OR}\boldsymbol{E}(\omega_1)\boldsymbol{E}(-\omega_2)$, we get  $-\boldsymbol{P}^{*}(\omega_0)=\boldsymbol{\chi}^{e}_{OR}(-\boldsymbol{E}^{*}(\omega))(-\boldsymbol{E}^{*}(\omega))$, where we have used Eq.~\ref{eq:TI}. After taking the complex conjugate of this equation, we get  $\boldsymbol{P}(\omega_0)=-(\boldsymbol{\chi}^{e}_{OR})^{*}\boldsymbol{E}(\omega_1)\boldsymbol{E}(-\omega_2)$. The susceptibility tensor describing optical rectification is then constrained to obey $\boldsymbol{\chi}^{e}_{OR}=-(\boldsymbol{\chi}^{e}_{OR})^{*}$, indicating $\boldsymbol{\chi}^{e}_{OR}$ is purely imaginary. Therefore, in the absence of dissipation (as satisfied by pumping below the gap), the light-induced polarization $\boldsymbol{P}(\omega_0)$ is purely imaginary, and $\delta\chi^{e}$ ($\delta\chi^{m}$) and $\chi^{e}$ ($\chi^{m}$) must possess a 90 degree phase difference. This is consistent with our results and can explain the domain contrast observed upon pumping. 



Above the Néel temperature $T_{N}$, Cr$_{2}$O$_{3}$ possesses symmetries of the centrosymmetric point group $\bar{3}m$. Axial i-type tensors of odd rank, such as $\chi_{ijk}^{m}$, and polar i-type tensors of even rank, such as $\chi_{ijkl}^{e}$, are allowed. Below $T_{N}$, inversion symmetry is broken due to the spin arrangement and the magnetic point group of Cr$_{2}$O$_{3}$ is $\underline{\bar{3}m}$. In addition to $\chi_{ijk}^{m}$ and $\chi_{ijkl}^{e}$, 
polar c-type tensors of odd rank, such as $\chi_{ijk}^{e}$,  and axial c-type tensors of even rank, such as $\chi_{ijkl}^{m}$, become allowed. Both of these tensors are proportional to the order parameter (i.e. the Néel vector) and flip sign when examining opposite AFM domains. 
Therefore, $\delta \chi^{e}$ is an axial c-type tensor and $\delta \chi^{m}$ is a polar i-type tensor. Hence, $\delta \chi^{e}$ flips sign when examining opposite antiferromagnetic domains, while $\delta \chi^{m}$ remains unchanged. 


\section{Temperature dependence of SHG}
In this section, we present pumped and unpumped RA-SHG patterns at various temperatures, and show that the pump induced susceptibility tensor $\delta\chi^{e}$ has an order parameter-like behavior as a function of temperature. We plot the unpumped and pumped SHG patterns in Fig.~\ref{FigS1} at five selected temperatures close to $T_{N} \sim 307$~K. Laser heating increases the sample temperature by roughly 3-5~K. Therefore, for the 305~K data, the actual temperature is above  $T_{N}$ as evidenced by the appearance of the nodes in RA-SHG pattern (and equal SHG intensities when examining the left and right circularly polarized SHG signals). From our fits, we can extract $\chi^{e}$ from the unpumped data and $\delta\chi^{e}$ from the pumped data at each temperature. In Fig.~\ref{FigS2}, we present $\chi^{e}$ and $\delta\chi^{e}$ versus temperature. We find that, consistent with previous studies, $\chi^{e}$ is proportional to the Néel vector. The critical exponent of the order parameter $\beta\approx$ 0.32. In the previous section we showed that $\delta\chi^{e} \propto \chi^{e}_{yyyy}\chi^{e}_{OR}$, which indicates that  $\delta\chi^{e}$ is also related to the order parameter and should also disappear above  $T_{N}$. Under the assumption that the i-type tensor element $\chi^{e}_{yyyy}$ is constant across the AFM transition, $\delta\chi^{e}$ is linear proportional to the order parameter. It should be kept in mind, however, that $\chi^{e}_{yyyy}$ may scale quadratically with the order parameter. From our data alone, it is difficult to ascertain whether $\delta\chi^{e}$ is linearly proportional to the order parameter due to the large measurement uncertainty. The best fit gives rise to an exponent $\beta \sim$ 0.5, but $\beta \sim$0.32 also falls within the error bars. Nevertheless, the pump induced  $\delta\chi^{e}$ is related to the order parameter and should flip signs as switching to the other domain.

% Figure environment removed 


% Figure environment removed 

\newpage
\vfill

\section{SHG in the perpendicular channel}
In the main text and previous section, we showed data with the two polarizers (for the incident and outgoing light) in a parallel geometry. Here, we show the data in the perpendicular geometry. In our setup, we rotate the half wave plate so that the input polarization is rotated by 90 degrees to achieve the perpendicular geometry. In contrast to the parallel geometry, we now get $M(\underline{\bar{3}m}) \propto \chi^{m}\textrm{cos}(3\theta)$ and $P(\underline{\bar{3}m}) \propto \chi^{e}\textrm{sin}(3\theta)$ for the equilibrium RA-SHG pattern, while the terms $M(C_{2}) \propto \delta\chi^{m}\textrm{cos}(\theta-2\phi)$ and $P(C_{2}) \propto \delta\chi^{e}\textrm{sin}(\theta-2\phi)$ need to be included upon pumping. In the perpendicular channel, $\delta\chi^{m} \propto \chi^{m}_{yxxy}\chi^{e}_{OR}(E^{p})^{2}$ and $\delta\chi^{e} \propto \chi^{e}_{yxxy}\chi^{e}_{OR}(E^{p})^{2}$. Fig.~\ref{FigS4} shows the unpumped RA-SHG pattern with a fundamental probe wavelength of 1180~nm in the perpendicular channel. This figure also shows the corresponding pumped RA-SHG pattern with pump polarization along the $x$ axis. Fig.~\ref{FigS5} shows the pumped RA-SHG pattern as the pump polarization is rotated clockwise in 20 degree steps. Again, these are not fits to the individual RA-SHG patterns -- instead, only the $\varphi = -90^\circ$ pattern is fit and the black curves pertaining to the remaining RA-SHG patterns are generated from our fit function. 
%The small discrepancy between data and fit come from the imperfection of the setup or the alignment of the sample is not perfect (e.g. z axis of sample is not perfectly along light propagation direction.)



% Figure environment removed 


% Figure environment removed 
\newpage
\section{Effect of circular pump}
In this section, we demonstrate that a circularly polarized pump pulse does not induce a quasi-static electric dipole in Cr$_2$O$_3$ when pumping in the transparency window away from any electronic transitions. %In analogy to how we derived the $-2\phi$ dependence of the light-induced polarization for the linearly polarized pump, we first consider the induced polarization via optical rectification 
This calculation is more easily demonstrated in the time-domain, where we make the assumption that the medium is lossless and dispersionless. We can write that $\boldsymbol{\tilde{P}}(t)=\boldsymbol{\chi}^{e}_{OR}\boldsymbol{\tilde{E}}^{p}(t)\boldsymbol{\tilde{E}}^{p}(t)$, where the tilde denotes a time-dependent quantity. Under the magnetic point group symmetry $\underline{\bar{3}m}$, the relevant susceptibility tensor elements are   $\chi_{yyy}$=$-\chi_{yxx}$=$-\chi_{xyx}$=$-\chi_{xxy}$=$\chi^{e/m}$. 
%For circularly polarized light, the electric field is given by $\boldsymbol{E}(\omega) = \boldsymbol{E}_0\textrm{cos}(\omega t) (\textbf{e}_x + i \textbf{e}_y$). In this case, we see that:
We can write $\boldsymbol{\tilde{P}}(t)=\tilde{P}_{x} \textbf{e}_{x}+\tilde{P}_{y} \textbf{e}_{y}$, and the $x$ and $y$ components are $\tilde{P}_{x}=2\chi_{xyx}\tilde{E}_{y}\tilde{E}_{x}$ and $\tilde{P}_{y}=\chi_{yyy}\tilde{E}_{y}\tilde{E}_{y}+\chi_{yxx}\tilde{E}_{x}\tilde{E}_{x}$ as in the previous section. Now with a circularly polarized pump, the $x$ and $y$ components of pump electric field are out of phase, and we have (for right circular pump) $\tilde{E}_{x}=\frac{1}{\sqrt{2} }(E^{p})\textrm{sin}(\omega t) $ and $\tilde{E}_{y}=\frac{1}{\sqrt{2} } (E^{p}) \textrm{cos}(\omega t)$ , where the $E^{p}$ is the electric field from the pump.   We obtain
\begin{equation}
 \boldsymbol{\tilde{P}}(t)=\frac{1}{2}\chi^{e}_{OR}  (E^{p})^2 [\textrm{sin}(-2\omega t)\textbf{e}_{x}+\textrm{cos}(-2\omega t)\textbf{e}_{y}].
\label{eq:S7}
\end{equation}
Because the calculation does not consist of a constant part (i.e. a time-independent contribution), but only a time-dependence at twice the incident frequency, there is no rectified signal. Circularly polarized light only generates second harmonic light, and not optical rectification, in Cr$_2$O$_3$.
%instead of a quasi-static dipole orientated along particular direction as for linear polarized pump, we get a circular dipole for circularly polarized pump, which rotates at twice the pump frequency, $2\omega \sim 600$ THz, and the time average is zero on scale of 200~fs.

In Fig.\ref{FigS6}, we show that when the $\alpha$ domain is pumped with circularly polarized light, there is no change in symmetry, consistent with the calculation presented above. The circularly polarized light was generated using quarter wave plate and the fluence used was approximately $20$~mJ/cm$^{2}$, the same as that used in the main manuscript with linearly polarized light.

% Figure environment removed 

\newpage
\section{wavelength dependence and phase}
% Figure environment removed 
In this section, we highlight the key role played by the equilibrium interference between electric and magnetic dipole radiation in observing the symmetry-breaking effects in Cr$_2$O$_3$. As we tune the second harmonic energy across several electronic $d$-$d$ transitions, the relative phase difference  $\gamma$ between electric and magnetic radiation changes correspondingly, and can be experimentally determined using Eq.~\ref{eq:phase} and is shown in Fig.~\ref{fig:S7}(d).

In the top panel of Fig.~\ref{fig:S7}(b), we show four equilibrium RA-SHG patterns at representative second harmonic energies corresponding to the dashed vertical lines in Fig.\ref{fig:S7}(a), (c) and (d). These values were chosen because they constitute extremal values of $\gamma$ ($\sim$0$^\circ$ at 2.56~eV and 2.43~eV, $\sim90^\circ$ at 2.64~eV and $\sim-90^\circ$ at 2.10~eV). In Fig.~\ref{fig:S7}(b), we show that when $\gamma \approx 0^\circ$, nodes are observed in the equilibrium RA-SHG pattern, and the induced dipole is barely visible when pumped. On the other hand, when $\gamma \approx \pm90^\circ$, the equilibrium nodes are lifted, and the pump has a dramatic effect on the symmetry of the RA-SHG. That the effect is most drastic at $\gamma\approx\pm90^\circ$ is also captured by Eq.~\ref{eq:PerturbationPhase}.  When $\gamma\neq0^\circ$, interference between the equilibrium and perturbing terms arise. These interference terms are linear in $\delta \chi^{e/m}$. In contrast, when $\gamma=0^\circ$ Eq.~\ref{eq:PerturbationPhase} factorizes and only terms $\mathcal{O}(\delta^2)$ remain. The lack of interference thus renders the induced dipole imperceptible.
\newpage

\section{time dependent RA-shg }

In Fig.~\ref{fig:S8}, we show how the rotational anisotropy pattern of the second harmonic light evolves as a function of time as the pump pulse propagates through Cr$_2$O$_3$. Specifically, we show the evolution of the RA-SHG in the $\alpha$ domain. The change due to the pump pulse (at a fluence of $\sim$20~mJ/cm$^2$) is roughly symmetric about $t=0$. This symmetry about $t=0$, which would not be anticipated if the spin or lattice degrees of freedom were involved, is evidence of a purely electronic process that breaks the underlying symmetry of the lattice.
% Figure environment removed 
\newpage
\section{wave-functions for microscopic mechanism of optical rectification }
At equilibrium,  the ground state $| g\rangle$ in  Cr$_2$O$_3$ is $^{4}\hspace{-0.1cm}A_{2g}$, which comprises three electrons in $t_{2g}$ state.   The relevant excited states, $| e\rangle$, comprise the $^{4}T_{2g}$ and $^{4}T_{1g}$ levels, both of which possess two $t_{2g}$ electrons and one $e_{g}$ electron. The $t_{2g}$ and  $e_{g}$ states can be constructed based on the $\underline{\bar{3}m}$ point group symmetry as shown by Eq.~\ref{eq:wavefunction}:%It can be checked easily that the three electron wave-functions respect the symmetry of Cr$_2$O$_3$ such as three fold rotation symmetry and $\mathcal{PT}$ symmetry \cite{ME0,ME1}.
\begin{equation} 
\begin{split}
&\ket{t_{2g}^{(1)}}=d_{x^2+y^2}\\
&\ket{t_{2g}^{(2)}}=((id_{xy}-d_{x^2-y^2})+ \eta(p_{x}+i p_{y}))/(1+\eta)\\
&\ket{t_{2g}^{(3)}}=((id_{xy}+d_{x^2-y^2})+ \eta(-p_{x}+i p_{y}))/(1+\eta)\\
&\ket{e_{g}^{(2)}}=((id_{xy}+d_{x^2-y^2})+ \eta(p_{x}-i p_{y}))/(1+\eta)\\
&\ket{e_{g}^{(1)}}=((id_{xy}-d_{x^2-y^2})+\eta (-p_{x}-i p_{y}))/(1+\eta)
\end{split}
\label{eq:wavefunction}
\end{equation}

where, for this choice of basis, $z$ is along the direction of the trigonal elongation, (i.e. the [111]-direction of the oxygen octahedron). Here, we already take into consideration the mixing of 4p orbitals due to the trigonal field, where $\eta$ denotes the strength of trigonal distortion and an arbitrary number is used for illustration purposes ($\eta =0.7$). As will be shown in next section, a single-electron approach is equivalent to multi-electron approach. Therefore the total electric dipole can be obtained by adding up the dipole moments of each  single electron.  We can draw  the corresponding single electron probability densities (in the $x-y$ plane) for $| g\rangle$ and $| e\rangle$ based on the above wave-functions. In the Fig.~4(a), we show the electron probability densities of ($t_{2g}^{(1)}$, $t_{2g}^{(2)}$, $t_{2g}^{(3)}$) for $| g\rangle$ and one combination $(t_{2g}^{(1)}, t_{2g}^{(2)}, e_{g}^{(1)})$ as an example for the $| e\rangle$ manifold.

In the presence of time periodic pump pulse, Floquet states are created and we show only the lowest order (within the rotating wave approximation) $| e,-1\rangle$ state in Fig.~4. There are two pathways that can give rise to a static dipole as explained in the main text. %The first pathway is hybridization between $| g,0\rangle$ and $| e,0\rangle$ mediated by $| e,-1\rangle$ with form of $\bra{g,0}\boldsymbol{d}\ket{e,0}$, and second pathway is hybridization between $| g,0\rangle$ and $| e,-1\rangle$ with form of $\bra{e,-1}\boldsymbol{d}\ket{e,-1}$. 
The consequences of these two pathways are the mixing among $(t_{2g}^{(1)}, t_{2g}^{(2)}, t_{2g}^{(3)}, e_{g}^{(1)}, e_{g}^{(2)})$ states in terms of the single electron approach. We show a representative electron probability density for the hybridized state $\ket{g,0} + \epsilon^2\ket{e,0}$ (corresponding to the first pathway): $|t_{2g}^{(1)}\rangle \rightarrow|t_{2g}^{(1)}+ \zeta(t_{2g}^{(2)}+t_{2g}^{(3)}) + \xi e_{g}^{(1)}\rangle $, $|t_{2g}^{(2)}\rangle \rightarrow|t_{2g}^{(2)}+ \zeta (t_{2g}^{(1)}+t_{2g}^{(3)}) + \xi e_{g}^{(1)}\rangle $ and $|t_{2g}^{(3)}\rangle \rightarrow|t_{2g}^{(3)}+ \zeta (t_{2g}^{(1)}+t_{2g}^{(2)}) + \xi e_{g}^{(1)}\rangle $. %Each of these three superposition states are occupied by a single 3$d$ electron. 
The coefficients characterizing the hybridization strength are determined by transition matrix elements (such as intensity and light polarization). Arbitrary values $\xi$ and $\zeta$  are chosen for illustration purposes, as shown in Fig.4 in the main text (e.g.,  $\zeta =0.1$ and $\xi =0.2$ ). As shown in Fig.4, the hybridized states break the threefold in-plane rotational symmetry and can give rise to the light-induced electronic dipole moment.



\section{Perturbative calculation of dipole moment}

We set up a toy model for the pumped Cr$_2$O$_3$ and calculate the static dipole moment using time-dependent perturbation theory.  Then, we repeat the same calculation using the Floquet approach.  Finally, we show that the same answer must be obtained if the dipole moment is written as the sum of dipole moments in time-evolving single-electron states.

\subsubsection{Time-dependent perturbation theory}
We consider an atomic model consisting of a non-degenerate ground state $\ket{0}$ at energy $w_0$ (we set $\hbar=1$ throughout) and excited states $\ket{1},\dots,\ket{N}$ at energy $w_1=\dots=w_N  > w_0$, with dipole coupling to light.  Thus, we consider
\beq
    H(t) = H^{(0)}+ H^{(1)}(t),
\eeq
where $H^{(0)}$ is the unperturbed Hamiltonian and $H^{(1)}(t)$ is the dipole coupling to the pump field $\mathbf{E}(t)$: 

\bseq
\begin{align}
    H^{(0)}&= \sum_{i=0}^N w_i \ket{i}\bra{i}\\
    H^{(1)}(t) &= - \mathbf{d}\cdot \mathbf{E}(t).
\end{align}
\eseq
We do not include dissipation because the experiment was done in a regime without significant absorption. 

Setting $N=6$ corresponds to a toy model of a given Cr atom in Cr$_2$O$_3$.  The ground state is the $^{4}\hspace{-0.1cm}A_{2g}$ state ($\ket{g}$ in the main text), and the only excited states kept are the three $^{4}T_{1g}$ states and the three $^{4}T_{2g}$ states ($\ket{e}$ in the main text).  We need not consider excited states that flip the spin because they are dipole forbidden.  In this toy model, we assume that the six excited states are exactly degenerate.  While the energy splitting (approximately $0.5$ eV) between the $^{4}T_{1g}$ states and $^{4}T_{2g}$ states may have a non-negligible quantitative effect on the static dipole moment, we set this splitting to zero as a simplifying assumption to convey the basic idea.  Non-zero splitting can be included straightforwardly in this approach.

A perturbative treatment of $H^{(1)}$ is appropriate because the linear dependence on fluence [Fig. 4(b) in the main text] indicates that the static dipole depends quadratically on the electric field amplitude.  Also, we can roughly estimate the electric field amplitude $E$ from the fluence $f=20$ mJ/cm$^2$ via $f\approx |\frac{1}{\mu_0}\mathbf{E}\times\mathbf{B}| t_\text{pulse} = c\epsilon_0 E^2 t_\text{pulse}$  (where $t_\text{pulse}=190$ fs), yielding $E\approx 6 * 10^8$ V/m. Then, estimating the dipole moment as the electron charge times the Bohr radius, we obtain the order of magnitude of $H^{(1)}(t)$ as $0.06$ eV, which is small compared to $w_2-w_1=2.1$ eV (the energy gap between $^{4}T_{1g}$ and $^{4}\hspace{-0.1cm}A_{2g}$).

We take the electric field $\mathbf{E}(t)$ of the pump pulse to be linearly polarized:
\beq
    \mathbf{E}(t) = \varphi(t)\mathbf{E} =  \int\frac{dw}{2\pi}\ e^{-i wt}\varphi(w)\mathbf{E},
\eeq
where $\mathbf{E}$ is a fixed vector defining the polarization axis, $\varphi(t)$ is the time dependence of pump pulse, and $\varphi(w) = \int dt\ e^{i wt} \varphi(t)$ is the pump pulse in the frequency domain.

Since $\varphi(t)$ must be real, $\varphi(w) = \varphi^*(-w)$; we therefore focus on $w>0$ in the following discussion of the pump pulse.  We take $\varphi(w)$ to be sharply peaked, with some pulse width $\Delta$, about the laser frequency $\Omega$; in particular, we assume
\beq
    \varphi(w) \overset{\Delta\to0^+}{\longrightarrow} \pi \delta(w-\Omega) \qquad (w>0).
\eeq
Thus, $\varphi(t)\overset{\Delta\to0^+}{\longrightarrow} \cos(\Omega t)$, while for small $\Delta>0$, $\varphi(t)$ is a broad wave train at frequency $\Omega$ that goes to zero at $t\to\pm\infty$.  We are interested in time $t$ in the broad middle of the wave train.  For simplicity, we assume that $\varphi(w)$ is strictly zero for $w$ outside of $[\Omega-\frac{1}{2}\Delta,\Omega+\frac{1}{2}\Delta]$.

We consider evolving the system starting from the ground state at time $t_0$ (later sending $t_0\to-\infty$).  We wish to calculate the dipole moment at a time $t$:
\beq
    \langle \mathbf{d}\rangle_t \equiv \bra{\Psi(t)}\mathbf{d} \ket{\Psi(t)},
\eeq
where
\beq
    \ket{\Psi(t)} = U(t,t_0)\ket{0}= \sum_{i=0}^N a_i(t) \ket{i}.
\eeq

We calculate the coefficients $a_i(t)$ perturbatively:
\beq
    a_i(t) = a_i^{(0)}(t)+a_i^{(1)}(t) + a_i^{(2)}(t)+ \dots,
\eeq
where the superscripts indicate the order in $|\mathbf{E}|$.  It is convenient to first calculate the coefficients in the interaction picture (relative to $t=t_0$), i.e., to define
\beq
    a_{i,I}(t) = e^{i w_i (t-t_0)}a_i(t).
\eeq
The interaction picture coefficients are determined order-by-order by $a_{i,I}^{(n)}(t_0) = \delta_{i0}\delta_{n 0}$ and

\bseq
\begin{align}
    i\frac{d}{d t}a_{i,I}^{(n+1)}(t)&= \sum_{j=0}^N [H_I^{(1)}(t)]_{ij} a_{j,I}^{(n)}(t)\\
    &= -\sum_{j=0}^N \mathbf{d}_{ij}\cdot \mathbf{E}\varphi(t) e^{i(w_i - w_j)(t-t_0)} a_{j,I}^{(n)}(t),
\end{align}
\eseq
where we use the notation $\mathcal{O}_{ij}= \bra{i}\mathcal{O}\ket{j}$ for the matrix elements of any operator $\mathcal{O}$.

Throughout this calculation, we always take the vector $\mathbf{d}$ to be in-plane ($x$-$y$).  This is justified because the pump field was polarized in-plane [hence only the in-plane components of $\mathbf{d}$ appear in $H^{(1)}(t)$] and because the measurement could only detect in-plane components of $\mathbf{d}$ [hence we only need to calculate $\langle \mathbf{d}\rangle_t$ in-plane].  Due to the $C_3$ site symmetry of the Cr atom, the in-plane components of the dipole vanish in the ground state:
\beq
    \mathbf{d}_{00}= \mathbf{0}.
\eeq

We work to second order in $|\mathbf{E}|$ because, as we will see, this yields the leading contribution to the static part of the dipole moment.  Once the $a_{i}(t)$ coefficients are determined up to second order, the dipole moment is obtained as
\beq
    \langle \mathbf{d}\rangle_t = \langle \mathbf{d}\rangle_t^{(0)} + \langle \mathbf{d}\rangle_t^{(1)}+\langle \mathbf{d}\rangle_t^{(2)} + \dots,   
\eeq
where one finds, after noting $a_i^{(0)}(t) = \delta_{i0}e^{-iE_0 (t-t_0)}$ and $\mathbf{d}_{00}=\mathbf{0}$ [which imply $a_0^{(1)}(t)=0$],

\bseq
\begin{align}
    \langle \mathbf{d}\rangle_t^{(0)}&= \mathbf{0},\\
    \langle \mathbf{d}\rangle_t^{(1)}&= 2\sum_{i=1}^N\text{Re}\left( e^{iw_0 (t-t_0)}\mathbf{d}_{0 i} a_i^{(1)}(t) \right),\\
    \langle \mathbf{d}\rangle_t^{(2)} &= \sum_{i,j=1}^N  a_i^{(1)*}(t)\mathbf{d}_{i j}a_j^{(1)}(t)+2\sum_{i=1}^N\text{Re}\left( e^{iw_0 (t-t_0)}\mathbf{d}_{0 i} a_i^{(2)}(t)   \right).\label{eq:d2} 
\end{align}
\eseq

We send the start time $t_0\to-\infty$ before sending the pulse width to zero ($\Delta\to 0^+$).  We take these limits with the observation time $t$ held fixed so that $t$ is in the broad middle of the wave train ($\Delta |t| \ll 1$).  We send the pulse width to zero to yield an approximate answer that is independent of the particular way the pulse turns on and off in time.  This results in some contributions to $\langle \mathbf{d}\rangle_t$ being strictly static.  If we instead were to keep a small non-vanishing pulse width, then additional quasistatic contributions would appear (with frequencies no larger than the pulse width) that depend on further details of how the pulse turns on and off in time.

We readily obtain the first-order correction to the wavefunction:

\bseq
\begin{align}
    a_i^{(1)}(t) &\overset{t_0\to-\infty}{\longrightarrow} e^{-iw_0(t-t_0)}  \mathbf{d}_{i0}\cdot\mathbf{E}\int\frac{dw}{2\pi}\ \frac{1}{w_i- w_0- w}e^{-i w t} \varphi(w)\\
    &\overset{\Delta\to 0^+}{\longrightarrow} e^{- iw_0(t-t_0)}\mathbf{d}_{i0}\cdot\mathbf{E}\frac{1}{2}\left( \frac{1}{w_i - w_0 - \Omega} e^{-i\Omega t} + \frac{1}{w_i- w_0 + \Omega} e^{i\Omega t}\right)\\
    &= e^{- iw_0(t-t_0)} \mathbf{d}_{i0}\cdot\mathbf{E}\frac{1}{2}\left( \frac{1}{w_1 - w_0 - \Omega} e^{-i\Omega t} + \frac{1}{w_1- w_0 + \Omega} e^{i\Omega t}\right)
\end{align}
\eseq
where $t_0$ still appears in the overall phase but always cancels in $\langle \mathbf{d}\rangle_t$.  In the last line, we have replaced $w_i\to w_1$ because $\mathbf{d}_{00}=\mathbf{0}$.

The first order part of the dipole moment therefore does not have any static part (the only oscillation frequency is $\Omega$).  So, we proceed to the second order, finding
\beq
    a_i^{(2)}(t)\overset{t_0\to-\infty}{\longrightarrow} e^{-iw_0(t-t_0)}\sum_{j=1}^N (\mathbf{d}_{ij}\cdot\mathbf{E})(\mathbf{d}_{j0}\cdot\mathbf{E})\int \frac{dw}{2\pi}\frac{dw'}{2\pi}\ \varphi(w)\varphi(w')  \frac{1}{w_j - w_0 - w'}\frac{e^{-i(w+w')t}}{w_i-w_0-w-w'}.
\eeq
Except for $i=0$ [which we see from Eq. \eqref{eq:d2} is not needed], we can send the pulse width to zero, yielding:

\bseq
\begin{align}
     a_i^{(2)}(t)&\overset{\Delta\to 0^+}{\longrightarrow}   e^{-iw_0(t-t_0)}\sum_{j=1}^N (\mathbf{d}_{ij}\cdot\mathbf{E})(\mathbf{d}_{j0}\cdot\mathbf{E})\biggr[ \frac{1}{4} \left( \frac{1}{w_j - w_0 - \Omega} +  \frac{1}{w_j - w_0 +  \Omega}\right) \frac{1}{w_i-w_0} + ( e^{\pm i 2  \Omega t} \text{ terms})\biggr]\\
     &= e^{-iw_0(t-t_0)}\sum_{j=1}^N (\mathbf{d}_{ij}\cdot\mathbf{E})(\mathbf{d}_{j0}\cdot\mathbf{E})\biggr[ \frac{1}{2} \frac{1}{w_1 - w_0 - \Omega}  \frac{1}{w_1 - w_0 +  \Omega} + ( e^{\pm i 2  \Omega t} \text{ terms})\biggr].
\end{align}
\eseq
We ignore the $e^{\pm i 2  \Omega t}$ terms because they do not contribute to the static part of the dipole moment at second order.

From Eq. \eqref{eq:d2}, we thus obtain the following expression for the static part of the dipole moment at leading order:

\begin{multline}
     \langle \mathbf{d}\rangle_t \overset{\text{static}}{=} \frac{1}{4}\left[ \left(\frac{1}{w_1-w_0 -\Omega}\right)^2 + \left(\frac{1}{w_1-w_0 +\Omega}\right)^2 \right] \sum_{i,j=1}^N (\mathbf{d}_{0i}\cdot\mathbf{E}) \mathbf{d}_{ij} (\mathbf{d}_{j0}\cdot\mathbf{E})\\ + \frac{1}{w_1-w_0-\Omega}\frac{1}{w_1-w_0+\Omega} \sum_{i,j=1}^N \text{Re}\left[\mathbf{d}_{0i} (\mathbf{d}_{ij}\cdot\mathbf{E})(\mathbf{d}_{j0} \cdot\mathbf{E})\right].\label{eq:static dipole toy model}
\end{multline}

There is a simple interpretation of this expression: the first term is the contribution from the first-order population of the excited states (resulting in a second-order contribution to the dipole moment through $\bra{\Psi^{(1)}(t)}\mathbf{d}\ket{\Psi^{(1)}(t)}$), and the second term is the contribution from the second-order population of the excited states, which must be mediated by a virtual state (which in this model can only be another excited state).  We have kept both rotating and counter-rotating contributions.

We may further simplify Eq. \eqref{eq:static dipole toy model} by using a basis for the $N$ degenerate excited states in which the dipole operator is diagonal:
\beq
    \mathbf{d}_{ij} = \mathbf{d}_{ii}\delta_{ij}.
\eeq
In such a basis, we obtain

\begin{multline}
     \langle \mathbf{d}\rangle_t \overset{\text{static}}{=} \frac{1}{4}\left[ \left(\frac{1}{w_1-w_0 -\Omega}\right)^2 + \left(\frac{1}{w_1-w_0 +\Omega}\right)^2 \right] \sum_{i=1}^N |\mathbf{d}_{i0}\cdot\mathbf{E}|^2 \mathbf{d}_{ii} \\ + \frac{1}{w_1-w_0-\Omega}\frac{1}{w_1-w_0+\Omega} \sum_{i=1}^N \text{Re}\left[\mathbf{d}_{0i} (\mathbf{d}_{ii}\cdot\mathbf{E})(\mathbf{d}_{i0}\cdot\mathbf{E}) \right].
\end{multline}


\subsubsection{Floquet perturbation theory}

We present an alternate calculation of the static dipole moment using the Floquet formalism.  We consider the same model as in the previous section, now with the pulse width set to zero from the beginning, i.e.,
\beq
    H^{(1)}(t) = -\mathbf{d}\cdot \mathbf{E}\cos(\Omega t).
\eeq
The Floquet formalism directly accesses the regime of the broad middle of the wave train in time; in particular, the turning on and off of the pump pulse are not included in this calculation.  We consider the start time to be $t=0$ for convenience.

We now set up Floquet perturbation theory.  The Floquet Hamiltonian $H_F$ is an infinite-dimensional, Hermitian matrix defined by
\beq
    (H_F)_{ij,mn}= \frac{1}{\tau}\int_0^\tau dt\ e^{i(m-n)\Omega t} H_{ij}(t),
\eeq
where $\tau=1/\Omega$ is the period and where the Floquet indices $m$ and $n$ go over all integers.  With $H(t) = H^{(0)}+H^{(1)}(t)$, we obtain $H_F= H_F^{(0)}+H_F^{(1)}$ with

\bseq
\begin{align}
   (H_F^{(0)})_{ij,mn}&= (w_i - m \Omega) \delta_{ij}\delta_{mn},\\
   (H_F^{(1)})_{ij,mn}&= -\frac{1}{2} \mathbf{d}_{ij}\cdot\mathbf{E}(\delta_{m-n,1} + \delta_{m-n,-1}).
\end{align}
\eseq

By Floquet's theorem, there is a complete basis of time-evolving states of the form
\beq
    \ket{\Psi(t)} = e^{-i\epsilon t}\ket{u(t)},\label{eq:Floquet theorem}
\eeq
where $\epsilon$ is the quasienergy and where the time-evolving states $\ket{u(t)}$ are $\tau$-periodic.  We may write
\beq
    \ket{u(t)} = \sum_{i=0}^N\sum_m e^{-i m\Omega t}u_{im}\ket{i} = \sum_\alpha e^{-i \alpha_2\Omega t} u_\alpha \ket{\alpha_1},\label{eq:u(t)}
\eeq
where $\alpha = (i,m)$ is a two-component index consisting of the level index $i$ and the Floquet index $m$.

It is convenient to consider an auxiliary Hilbert space spanned by orthogonal basis vectors $|\alpha)$, where we write $|\alpha)$ instead of $\ket{\alpha}$ to avoid confusion with the original Hilbert space.  Then we write
\beq
    |u) = \sum_\alpha u_\alpha|\alpha).
\eeq
It may be shown that the quasienergies $\epsilon$ and the coefficients $u_\alpha$ are determined by an eigenvalue equation with the same structure as the time-independent Schrodinger equation:
\beq
    H_F |u) = \epsilon |u).
\eeq
We can thus apply standard time-independent perturbation theory to calculate $\epsilon = \epsilon^{(0)}+\epsilon^{(1)}+\dots$ and $|u)= |u^{(0)})+|u^{(1)})+\dots$, yielding the time-dependent wavefunction.

To determine the static dipole moment in the case that the ground state $\ket{0}$ is perturbed by light, we consider the case of $|u^{(0)}) = |0,0)$ and $\epsilon^{(0)}= w_0$.  The static dipole moment is given by

\bseq
\begin{align}
    \bra{\Psi(t)}\mathbf{d}\ket{\Psi(t)} &= \sum_{i,i'=0}^N \sum_m u_{(i',m)}^{*}\mathbf{d}_{i' i }u_{(i,m)}\\
    &= 2\sum_{i=1}^N\text{Re}\left[\mathbf{d}_{0 i }u_{(i,0)}^{(1)} \right] + \sum_{i,i'=1}^N \sum_m u_{(i',m)}^{(1)*}\mathbf{d}_{i' i }u_{(i,m)}^{(1)} + 2\sum_{i=1}^N \text{Re}\left[ \mathbf{d}_{0 i }u_{(i,0)}^{(2)}\right],\label{eq:static dipole in terms of u toy model Floquet}
\end{align}
\eseq
where, in the second line, we have expanded to second order and recalled $\mathbf{d}_{00}=\mathbf{0}$.

At first order, we readily obtain
\beq
    \epsilon^{(1)}= (0,0|H_F^{(1)}|0,0) = 0,
\eeq
and, for $\alpha \ne (0,0)$,
\beq
    u_\alpha^{(1)} = - \frac{(\alpha|H_F^{(1)}|0,0)}{\epsilon_\alpha - w_0}= \frac{1}{2}\mathbf{d}_{\alpha_1 0}\cdot\mathbf{E}
    \begin{cases}
        \frac{1}{w_1 - w_0 -\Omega} & \alpha_2 = 1,\\
        \frac{1}{w_1 - w_0 +\Omega} & \alpha_2 = -1,\\
        0 & \text{otherwise},
    \end{cases}
\eeq
where $\epsilon_\alpha = w_{\alpha_1}-\alpha_2\Omega$ is the unperturbed quasienergy of state $|\alpha)$.  The remaining coefficient, $u_{(0,0)}^{(1)}$, can be shown to be pure imaginary (due to the normalization requirement $\bra{\Psi(t)}\ket{\Psi(t)}=1$) and represents a phase degree of freedom in the choice of initial state; for convenience, we set $u_{(0,0)}^{(1)}=0$.  Note from Eq. \eqref{eq:static dipole in terms of u toy model Floquet} that there is no static dipole at first order, since $u_{i0}^{(1)}=0$ for $i=1,\dots,N$.

At second order, we obtain
\beq
    \epsilon^{(2)} = - \sum_{\alpha\ne(0,0)} \frac{|(\alpha|H_F^{(1)}|0,0)|^2}{\epsilon_\alpha-w_0}= -\frac{1}{4}\left(\frac{1}{w_1-w_0 -\Omega} +\frac{1}{w_1-w_0 +\Omega}\right)\sum_{i=1}^N |\mathbf{d}_{i0}\cdot\mathbf{E}|^2, 
\eeq
and, for $i=1,\dots,N$,
\beq
    u_{(i,0)}^{(2)} = \frac{1}{w_1-w_0}\sum_{\alpha\ne(0,0)} \frac{(i,0|H_F^{(1)}|\alpha)(\alpha|H_F^{(1)}|0,0)}{\epsilon_\alpha-w_0}= \frac{1}{2}\frac{1}{w_1-w_0-\Omega}\frac{1}{w_1-w_0+\Omega} \sum_{j=1}^N (\mathbf{d}_{ij}\cdot\mathbf{E})(\mathbf{d}_{j0}\cdot\mathbf{E}).
\eeq

Eq. \eqref{eq:static dipole in terms of u toy model Floquet} then recovers Eq. \eqref{eq:static dipole toy model}.  The second term on the right-hand side of Eq. \eqref{eq:static dipole in terms of u toy model Floquet} is the contribution of the first-order hybridization between $|0,0)$ and either $|i,1)$ or $|i,-1)$, while the third term is the contribution from the second-order hybridization between $|0,0)$ and $|i,0)$ mediated by either $|j,1)$ or $|j,-1)$ (both possibilities being summed, in addition to summing $i,j=1,\dots,N$). 

From Eqs. \eqref{eq:Floquet theorem} and \eqref{eq:u(t)}, we can also confirm that Floquet perturbation theory recovers the same time-dependent wavefunction as calculated in the previous section (in the limit of $t_0\to-\infty$ followed by $\Delta\to0^+$) except for the following two features.  First, the second-order correction to the quasienergy, $\epsilon^{(2)}$, did not appear in the previous section; however, in the second-order expansion of $\ket{\Psi(t)}$, $\epsilon^{(2)}$ can only appear as a (time-diverging) term in the coefficient $a_{0,I}^{(2)}$ that we did not need to calculate.  Second, there is an unimportant overall phase of $e^{-iw_0 t_0}$ between the two calculations, which arises from changing the switch-on time from $t_0$ to $0$.

If we wish to make the approximation of dropping the counter-rotating terms, we can do this directly by setting $\Omega\approx E_1-E_0$ in Eq. \eqref{eq:static dipole toy model}.  Alternatively, we can truncate the Floquet Hamiltonian from the beginning so that it acts on the basis states $|0,0), |i,1)$, and $|i,0)$ ($i=1,\dots, N$).  Then, the static dipole moment consists of a contribution from the first-order hybridization between $|0,0)$ and $|i,1)$, and a contribution from the second-order hybridization between $|0,0)$ and $|i,0)$ mediated by $|j,1)$.

\subsubsection{Equivalence to single-electron approach}
As an alternative to the calculations we presented above, that considered the multielectron states directly, we can instead work with single-electron states.  Note that, due to strong on-site Coulomb interaction, we can work with spinless electrons.  However, it seems that we must neglect the splitting between the $^{4}T_{1g}$ states and $^{4}T_{2g}$ states, because this is a more complicated interaction effect.

In the single-electron approach, we consider
\beq
    H^{(0)} = \sum_a w_a c_a^\dagger c_a, 
\eeq
where the index $a$ goes over five orthonormal, single-electron states: the three $t_{2g}$ states (which we label as $g_1,g_2,g_3$) and the two $e_g$ states ($e_1,e_2$).  We are interested in the time-evolving dipole moment starting from the following three-electron state:
\beq
    \ket{\Psi} = c_{g_3}^\dagger c_{g_2}^\dagger c_{g_1}^\dagger\ket{\Omega},
\eeq
where $\ket{\Omega}$ is the empty state (annihlated by all $c_a$).  The state $\ket{\Psi}$ is the same as the ground state $\ket{0}=\ket{g}$ considered earlier.

The coupling to the pump field is as before.  Now we note that the dipole operator is quadratic in the single-electron fields:
\beq
    \mathbf{d} = \sum_{a,a'} c_{a'}^\dagger\mathbf{d}_{a' a}c_a^\dagger.
\eeq

It is then straightforward to show that the time-evolving dipole moment is the sum of time-evolving dipole moments in the three single-electron states in $\ket{\Psi}$.  That is,
\beq
    \bra{\Psi}U^\dagger(t,t_0) \mathbf{d} U(t,t_0)\ket{\Psi} = \sum_{a_0= g_1,g_2,g_3}  \bra{\Omega} c_{a_0} U^\dagger(t,t_0) \mathbf{d} U(t,t_0)c_{a_0}^\dagger \ket{\Omega}.\label{eq:dipole as single-electron} 
\eeq
We can show Eq. \eqref{eq:dipole as single-electron} by, e.g., using the time-dependent creation operators.  In particular, we may define
\beq
    c_a^\dagger(t) = U^\dagger(t,t_0)c_a^\dagger U(t,t_0),
\eeq
where we note that the time evolution is in the opposite sense compared to the Heisenberg picture.  We then have
\beq
    U(t,t_0)\ket{\Psi} = c_{g_3}^\dagger(t)c_{g_2}^\dagger(t)c_{g_1}^\dagger(t) \ket{\Omega}.
\eeq
Noting that the canonical anticommutation relation holds at all time ($\{ c_a(t),c_{a'}^\dagger(t)\} = \delta_{aa'}$), and noting that the dipole operator is quadratic, we then obtain
\beq
    \bra{\Psi}U^\dagger(t,t_0) \mathbf{d} U(t,t_0)\ket{\Psi} = \sum_{a_0=g_1,g_2,g_3} \sum_{a,a'}\{ c_{a_0}(t), c_{a'}^\dagger\}\mathbf{d}_{a' a} \{c_a, c_{a_0}^\dagger(t) \},
\eeq
which is equivalent to Eq. \eqref{eq:dipole as single-electron}.  

Thus, an alternate approach to calculating the dipole moment is to add up the dipole moments in the three single-electron problems corresponding to the three $t_{2g}$ states.



%%Recall that $E_e- E_g = 65\ meV$ and  $E_4-E_g =\ 1.7 eV$.  This suggests that the dominant term in the sum over $i$ (in the first two lines) is $i=4$, in particular, the term $\frac{1}{E_4 - E_g -\Omega}\approx \frac{1}{0.5 eV}$.%%  In the terms in the first line, we thus encounter $(\frac{1}{0.5 eV})^2$ as the strength of the dominant term.  In the terms in the second line, the dominant contribution seems to be from $i=j=4$, but this term vanishes because $\mathbf{d}_{ji}=0$.  However, suppose that the next several states are nearly degenerate with $E_4$.  Then there could be several contributions with $(\mathbf{d}\cdot\mathbf{E})_{ji}\ne 0$, with prefactor $\frac{1}{E_i - E_g - \Omega}\frac{1}{E_j- E_g} \sim \frac{1}{0.5 eV}\frac{1}{1.7 eV}$, which is not so much smaller than $\left(\frac{1}{0.5 eV}\right)^2$.  A degeneracy factor could make these contributions comparable.  On the other hand, maybe the matrix elements could cause these contributions to partially cancel against each other rather than adding.  In that case, the terms in the first line of Eq. \eqref{eq:static_dipole_calculated} would be the dominant contributions. %%


%\nolinenumbers




%% Here is the endmatter stuff: Supplementary Info, etc.
%% Use \item's to separate, default label is "Acknowledgements"



\end{document}