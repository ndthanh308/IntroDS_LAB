\section{Preliminaries}
\label{sec:preliminaries}
In this section we recap definitions used throughout the rest of the paper.

\subsection{Complexity classes}
The computational model introduced in this work uses complexity classes. 
Typically these classes are defined as classes of decision problems solvable by some type of circuits. 
Below we give definitions of some of these complexity classes in terms of the circuits contained in that class. 

\begin{definition}
The class $\NC^k$ consists of all decision problems solvable by circuits of $\mathO((\log n)^k)$ depth and polynomial size, and consisting of bounded-fan-in AND- and OR-gates.

The class $\AC^k$ consists of all decision problems solvable by circuits of $\mathO((\log n)^k)$ depth and polynomial size, and consisting of unbounded-fan-in AND- and OR-gates.

The class $\TC^k$ consists of all decision problems solvable by circuits of $\mathO((\log n)^k)$ depth and polynomial size, and consisting of unbounded-fan-in AND-, OR- and Threshold$_t$-gates. A Threshold$_t$-gate evaluates to one if and only if the sum of the inputs is at least $t$. 
\end{definition}
These classes also have a quantum equivalent class. 
\begin{definition}
The class $\QNC^k$ consists of all decision problems solvable by quantum circuits of $\mathO((\log n)^k)$ depth and polynomial size, and consisting of single- and two-qubit quantum gates.
\end{definition}
Definitions for the quantum versions of $\AC^k$ and $\TC^k$ also exist.
However, when equipping the class $\QNC^k$ with unbounded-fan-in parity gates, all three classes intersect~\cite{GreenHomerMoorePollett:2002,Moore:1999,TakahashiTani_CollapseOfHierarchyConstantDepthExactQuantumCircuits_2013}.

Two other often used classes are $\mathsf{P}$ and $\mathsf{L}$.
\begin{definition}
The class $\mathsf{P}$ consists of all decision problems solvable in polynomial time by a Turing machine. 

The class $\mathsf{L}$ consists of all decision problems solvable using only a logarithmic amount of memory. 
\end{definition}
The first class poses a limit on the depth of the operations performed by the Turing machine. 
The second class instead limits the available memory. 
Note however that with a logarithmic amount of memory, only a polynomial number of states is available, and hence the computation time is polynomial as well. 

Relations between different complexity classes exists: for instance, for all $k$, $\NC^k\subseteq\AC^k\subseteq\TC^k$. 
By \citeauthor{Johnson:1990}, we also have the inclusion $\mathsf{L} \subseteq \AC^1 \subseteq \TC^1$~\cite{Johnson:1990}.

In the remainder of this work we abuse notation and refer to $\mathsf{X}$-circuits as circuits that correspond to a decision problem in the class $\mathsf{X}$.
For example, an $\NC^k$ circuit is a circuit of $\mathO((\log n)^k)$ depth and polynomial size that corresponds to a decision problem in $\NC^k$.

Finally, we define the class of polynomial sized quantum circuits:~\footnote{Due to the bounded-error-aspect associated to decision problems in $\BQP$, we follow this definition instead of talking about $\BQP$-circuits.}
\begin{definition}
The class $\mathsf{QPoly}(n)$ consists of all polynomial-sized quantum circuits (in $n$) that use single and two-qubit quantum gates.
\end{definition}

\subsection{Quantum gate sets}
First, recall the definition of the Pauli group and the Clifford group. 
\begin{definition}
The one qubit Pauli-group $P_1$ consists of four matrices, the identity matrix $I$ and the three Pauli matrices: 
$$X = \begin{pmatrix}0&1\\1&0\end{pmatrix}, \qquad Y = \begin{pmatrix}0&-i\\i&0\end{pmatrix}, \qquad Z = \begin{pmatrix}1&0\\0&-1\end{pmatrix},$$
together with a global phase of $\pm 1$ or $\pm i$. 

The $n$-qubit Pauli-group $P_n$ is the set of all $4^{n+1}$ possible tensors of length $n$ of matrices from $P_1$, together with a global phase of $\pm 1$ or $\pm i$. 
\end{definition}
The Clifford group forms the other well-known group of quantum circuits, as it stabilizes the Pauli group. 
\begin{definition}
The Clifford group $C_n$ consists of all $n$-qubit unitaries that leave the Pauli group 
$P_n$ invariant under conjugation. 
That is, let $c\in C_n$ be any Clifford circuit, then for any $P\in P_n$, there exists a $P'\in P_n$, such that $cP = P'c$.

The Clifford group is generated by the $CNOT$-gate, the Hadamard gate $H$ and the phase gate $S$, that act on computational basis states $\ket{x}$ and $\ket{y}$ via
$$CNOT: \ket{x}\ket{y}\mapsto \ket{x}\ket{y\oplus x}, \qquad H: \ket{x} \mapsto \ket{0}+(-1)^x \ket{1}, \qquad S: \ket{x} \mapsto i^x \ket{x}.$$
Any circuit constructed using only these three gates is called a Clifford circuit. 
\end{definition}

Unsurprisingly, Clifford circuits only cover a small part of the possible quantum circuits. 
Moreover, on a linear nearest-neighbor architecture, $\mathO(n)$ deep Clifford circuits suffice to simulate any Clifford unitary of size $2^n\times 2^n$~\cite{maslov2018shorter}.
We can furthermore simulate Clifford circuits efficiently~\cite{GottesmanKnill:1998}.
Universal quantum computations require additional gates, though almost any quantum gate suffices. 
For example, adding the single qubit $T$-gate, $T:\ket{x}\mapsto e^{i\pi x/4}\ket{x}$, to the Clifford group gives a universal gate set. 

\subsection{Two quantum subroutines}
This section discusses two quantum subroutines used in later sections. 
The first concerns Grover's algorithm with zero failure probability~\cite{Long_GroverZeroFailureRate_2001}. 
The second concerns parallelization of commuting gates using quantum fan-out gates~\cite{HoyerSpalek:2005}.

Grover's search algorithm gives a quadratic speed-up for unstructured search~\cite{Grover:1996}. 
After sufficient iterations, a measurement returns a target state with high probability. 
Surprisingly, if the exact number of target states is known, a slight modification of the Grover iterates allows for returning a target state with certainty, assuming noiseless computations. 
Lemma~\ref{lem:grover_constant_fraction} uses the next lemma to prepare quantum states instead of to find a target state.
\begin{lemma}[\cite{Long_GroverZeroFailureRate_2001}]\label{lem:exact_grover}
Let $L$ be a set of items and $T\subseteq L$ a set of targets, with $N=|L|$ and $m=|T|$ both known. 
Let $g : L \rightarrow \{0,1\}$ label the items in $L$ and define the oracle $O_g: \ket{x}\ket{b} \mapsto \ket{x}\ket{b\oplus g(x)}$.

Then, there exists a quantum amplitude amplification algorithm that makes $\mathO(\sqrt{N/m})$ queries to $O_g$ and prepares the quantum state $\sum_{x\in T}\ket{x}$.
\end{lemma}

For the other result, we use the quantum fan-out gate, that implements the map 
$$\ket{x}\ket{y_1}\hdots\ket{y_n}\mapsto\ket{x}\ket{y_1\oplus x}\hdots\ket{y_n\oplus x}.$$
Section~\ref{sec:gates_created_in_LAQCC} gives more details on how to implement this gate. 
\citeauthor{HoyerSpalek:2005} introduced this gate and analyzed its properties. 
The state preparation protocols given in Section~\ref{sec:state_prep_in_LAQCC} use the property that the quantum fan-out gate allows parallelization of commuting quantum gates~\cite{HoyerSpalek:2005}. 
\begin{lemma}\emph{(\cite[Theorem~3.2]{HoyerSpalek:2005})}
\label{lem:unitar_parallelization}
Let $\{U_i\}_{i=1}^n$ be a pairwise commuting set of gates on $k$ qubits. 
Let $U_i^{x_i}$ be the gate $U_i$ controlled by qubit $\ket{x_i}$. 
Let $T$ be the unitary that mutually diagonalizes all $U_i$. 
Then there exists a quantum circuit, using quantum fan-out gates, computing $U = \prod_{i=1}^n U_i^{x_i}$ with depth $\max_{i = 1}^n \mathrm{depth}(U_i) + 4\cdot\mathrm{depth}(T) + 2$ and size $\sum_{i = 1}^n \mathrm{size}(U_i) + (2n + 2) \cdot \mathrm{size}(T)+ 2n$, using $(n - 1)k$ ancilla qubits.
\end{lemma}

