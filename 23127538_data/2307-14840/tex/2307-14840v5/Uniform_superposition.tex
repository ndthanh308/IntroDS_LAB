\subsection{Uniform superposition of size \texorpdfstring{$q$}{q}\label{sec:superposition_modulo_q}}
The uniform superposition is often used as initial state in other quantum algorithms. 
A simple Hadamard gate applied to $n$ qubits prepares the uniform superposition $\frac{1}{\sqrt{2^n}}\sum_{i=0}^{2^n-1}\ket{i}$.
Preparing the state $\frac{1}{\sqrt{q}}\sum_{i=0}^{q-1}\ket{i}$, the superposition up to size $q$, is already harder for arbitrary $q$. 

A simple probabilistic approach works as follows: 
1) create a superposition $\frac{1}{\sqrt{2^n}}\sum_{i=0}^{2^n-1}\ket{i}$  with $n = \ceil{\log_2(q)}$ qubits;
2) mark the states $i< q$ using an ancilla qubit;
3) measure this ancilla qubit.
Based on the measurement result, the desired superposition is found, which happens with probability at least one half. 

The next theorem modifies this probabilistic approach to a protocol that deterministically prepares the uniform superposition modulo $q$ in $\LAQCC$.
\begin{theorem}
\label{thm:uniform_superposition_mod_q}
There is a deterministic $\LAQCC$ circuit that prepares the uniform superposition of size $q$. This circuit requires $\mathO(\ceil{\log_2(q)}^2)$ qubits.
\end{theorem}
\begin{proof}
Let $n = \ceil{\log_2(q)}$ and define $\mathcal{G}=\{i\mid 0\le i<q\}$ and $\mathcal{B}=\{i\mid q\le i\le2^n-1\}$. 
Construct the unitary
$$U_q:\ket{y}\ket{b}\mapsto\begin{cases}
\ket{y}\ket{b\oplus 1} & \text{if } y<q \\
\ket{y}\ket{b} & \text{if } y\ge q
\end{cases}.$$
The Greaterthan-gate of Table~\ref{tab:Add_Equality_Greaterthan} implements the operator $U_q$, note that this gate requires $\mathO(n^2)$ qubits. 

As $|\mathcal{G}|/2^n \ge 1/2$ and known, applying Lemma~\ref{lem:grover_constant_fraction} with the sets $\mathcal{G}$ and $\mathcal{B}$ and the constant-depth implementation of $U_q$, gives a $\LAQCC$ algorithm that boosts the amplitude of $\ket{\mathcal{G}}$ to~$1$.
\end{proof}

\begin{remark}
Note that in Lemma~\ref{lem:grover_constant_fraction} it was implicitly assumed that $|\mathcal G| + |\mathcal B|$ is a power of two (allowing for a simple reflection over the uniform superposition state). This $\LAQCC$ implementation of creating a uniform superposition modulo any $q$ removes this requirement.
\end{remark}