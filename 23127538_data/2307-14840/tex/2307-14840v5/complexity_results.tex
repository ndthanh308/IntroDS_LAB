\section{Complexity results for $\LAQCC(\mathcal{Q}, \mathcal{C},d)$}
\label{sec:Complexity results}

Up until now we have considered the class $\LAQCC$. 
In this section, we investigate the state-preparation complexity of $\LAQCC(\mathcal Q, \mathcal C, d)$ when we increase the power of the quantum circuits to polynomial depth, and we allow for unbounded classical computational power.

\begin{notation}
The class $\LAQCC^*$ is the instantiation $\LAQCC(\mathsf{QPoly}(n), \mathsf{ALL}, \mathrm{poly}(n))$:
The class of polynomially many alternating polynomial-sized quantum circuits and arbitrary powerful classical computations, together with feed-forward of the classical information to future quantum operations. 
The quantum computations are restricted to all single-qubit gates and the two-qubit CNOT gate. 
\end{notation}

Recall the definition of $\mathsf{State}$-classes:
\StateClassX*
From which we directly have a definition for the class $\mathsf{State}$$\LAQCC^*_{\varepsilon}$ for any $\varepsilon>0$.

\begin{remark}
Note that for any non-zero $\varepsilon$, we can restrict ourselves to finite universal gate-sets. 
The Solovay-Kitaev theorem~\cite{Kitaev1997,Nielsen2012} says that any multi-qubit unitary can be approximated to within precision $\delta$ by a quantum circuit with size depending on $\delta$. 
Therefore, with a finite universal gate-set, any $\LAQCC^*$ circuit with a continuous gate-set can be approximated by an $\LAQCC^*$ circuit with gates from the finite set. 
\end{remark}

To upper bound the state preparation complexity of $\LAQCC^*$, we compare it to the class $\mathsf{PostQPoly}$ and its circuit variant.

\PostQPolyDef*

Next, we prove $\mathsf{State}$$\LAQCC^*_{\varepsilon}\subseteq\mathsf{StatePostQPoly}_{\varepsilon}$. 
Section~\ref{sec:LAQCC_in_QNC1} describes how any $\LAQCC$ circuit decomposes in alternating unitaries, measurements and classical calculation layers. 
A similar decomposition follows for $\LAQCC^*$ circuits: 
Any $\LAQCC^*$ can be written as 
\begin{equation}
    \Pi_{i=0}^{\text{poly}(n)} M_i U_i(y_i) C_i(x_i)\ket{0}^{\otimes \text{poly}(n)},
\end{equation}
where each $x_i\in\{0,1\}^*$ denotes the measurement outcome of the $i$-th measurement layer and $y_i\in \{0,1\}^*$ the output bitstring of the $i$-th classical calculation layer. 
Note, all $x_i$ and $y_i$ have length at most polynomial in $n$. 
The $U_i$'s denote unitary operations that represent a polynomial-deep quantum circuit consisting of single qubit gates and the two-qubit CNOT gate. 
The $M_i$'s denote a measurement of a subset of the qubits, and the $C_i$'s represent unbounded classical computations, with input $x_i$. 

\begin{theorem}
\label{thm:state_prep_post-BQP}
It holds that $\mathsf{State}$$\LAQCC^*_{\varepsilon}\subseteq\mathsf{StatePostQPoly}_{\varepsilon}$.
\end{theorem}
\begin{proof}
Fix $\varepsilon>0$ and a positive integer $n$ and 
let $\ket{\psi}\in\mathsf{State}$$\LAQCC^*_{\varepsilon}$. 
By definition, there exists an $\LAQCC^*$ circuit $A=\Pi_{i=0}^{\text{poly}(n)} M_i U_i(y_i) C_i(x_i)$, which prepares a state $\ket{\phi}$ with inner product at least $1-\varepsilon$ with $\ket{\psi}$.

Then consider the following $\mathsf{PostQPoly}$-circuit:
Let $B=\Pi_{i=0}^{\text{poly}(n)} \text{Equal}_{x_i}(x_i)U_i(y_i)\ket{0}^{\otimes \text{poly}(n)}$, where the $y_i$ are hardwired. 
The Equal$_{x_i}$ gate replaces the measurement layer $M_i$, by checking if the subset of qubits that would be measured are in $\ket{x_i}$ computational basis state. It stores the output in an ancilla qubit. 
As a last step, apply an $\text{AND}_{\text{poly}(n)}$-gate on the ancilla qubits, which hold the outputs of the Equal$_{x_i}$ gates, and store the result in an ancilla qubit. 
Conditional on this last ancilla qubit being one, the circuit prepares the state $\ket{\phi}$.
\end{proof}

Figure~\ref{fig:laqcc_to_post_bqp} gives a schematic overview of the proof and the translation of an $\LAQCC^*$ circuit in a $\mathsf{PostQPoly}$-circuit. 
% Figure environment removed

An interesting direction for future work is to extend the inclusion proved above to a true separation or an oracular separation. 
One approach is to use a similar oracle as used in~\cite{AaronsonKuperberg:2007} to separate $\mathsf{QMA}$ and $\mathsf{QCMA}$ with respect to an oracle and use a counting argument to argue that $\mathsf{State}$${\LAQCC^*_{\varepsilon}}^{\mathcal{O}}\neq \mathsf{StatePostQPoly}_{\varepsilon}^{\mathcal{O}}$, for some oracle $\mathcal{O}$ and $\varepsilon=1-\frac{1}{\text{poly}(n)}$. 
The $\LAQCC$ model allows for a constant number, more than one, of rounds of measurements and corrections. This was required for our three new state generation protocols. However other models considered only one round of measurements and corrections, for instance in the paper \cite{Cirac:2021}. One may wonder if there is a hierarchy of model power allowing one or multiple measurements, and if there is a way to reduce the number of measurements rounds. A starting effort towards classifying types of states based on such a hierarchy can be found in \cite{Tantivasadakarn_2023}. It would be interesting to see a more extensive complexity theoretic analysis comparing models with different number of allowed rounds.
