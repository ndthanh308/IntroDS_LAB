\documentclass[a4paper,onecolumn,accepted=2024-10-22]{quantumarticle}
\pdfoutput=1

\usepackage[utf8]{inputenc}
\usepackage{amsmath,amsthm,amsfonts,amssymb,mathrsfs,thmtools,thm-restate}
\usepackage{bbm}
% \usepackage[english]{babel}
\usepackage{csquotes}
\usepackage{graphicx}
\graphicspath{ {./images/} }
% \usepackage{cleveref}
\usepackage{mathtools}
\usepackage{xcolor}
\usepackage{cuted}
\usepackage{tikz}
\usepackage{authblk}
\usepackage{mdframed}
\usepackage{hyperref}
\usetikzlibrary{quantikz}
%\usepMackage[braket,qm]{qcircuit}
\usepackage[bibencoding=auto,style=alphabetic,defernumbers=true,backend=biber,giveninits=true,doi=true,isbn=false,url=false]{biblatex}
\addbibresource{main_final_quantum.bib}


\usepackage{geometry}
\geometry{
a4paper,
total={150mm,227mm},
}

\mdfsetup{%
middlelinewidth=2pt,
backgroundcolor=orange!10,
roundcorner=10pt
}


% \newcommand{\TODO}[1]{\textcolor{orange}{TODO: #1}}
 %\newcommand{\marten}[1]{\textcolor{cyan}{Marten: #1}}
 %\newcommand{\niels}[1]{\textcolor{blue}{Niels: #1}}
% \newcommand{\bruno}[1]{{\color{red} Bruno: #1}}
 %\newcommand{\harry}[1]{{\color{teal} Harry: #1}}
%\newcommand{\new}[1]{\textcolor{purple}{New: #1}}
% \newcommand{\remove}[1]{\textcolor{red}{Remove: #1}}

%\newcommand{\TODO}[1]{}
\newcommand{\marten}[1]{}
\newcommand{\niels}[1]{}
\newcommand{\bruno}[1]{}
\newcommand{\harry}[1]{}


\newtheorem{theorem}{Theorem}[section]
\newtheorem{lemma}[theorem]{Lemma}
\newtheorem{corollary}[theorem]{Corollary}
\newcounter{mycount}
% definition of the command
\newcommand\myprob[3]{%
  \stepcounter{mycount}
  \par
  \vspace{1em}
  \par\noindent\rule{\textwidth}{0.4pt}
  \vspace{0.5em}
  \indent Problem\ \themycount . #1\\
  \indent {\bfseries Input}: #2\\
  \indent {\bfseries Output}: #3
  \vspace{0.5em}
  \par\noindent\rule{\textwidth}{0.4pt}
  \vspace{1em}\par
}

\theoremstyle{definition}
\newtheorem{definition}[theorem]{Definition}
\newtheorem{notation}[theorem]{Notation}
\newtheorem{example}[theorem]{Example}

\theoremstyle{plain}
\newtheorem{remark}[theorem]{Remark}

\DeclareMathOperator\supp{supp}
\DeclareMathOperator\BQP{\mathsf{BQP}}
\DeclareMathOperator\NC{\mathsf{NC}}
\DeclareMathOperator\AC{\mathsf{AC}}
\DeclareMathOperator\TC{\mathsf{TC}}
\DeclareMathOperator\QNC{\mathsf{QNC}}
\DeclareMathOperator\QTC{\mathsf{QTC}}
\DeclareMathOperator\LAQCC{\mathsf{LAQCC}}
\newcommand{\floor}[1]{\left\lfloor #1 \right\rfloor}
\newcommand{\ceil}[1]{\left\lceil #1 \right\rceil}
\newcommand{\kb}[1]{\ket{#1}\bra{#1}}
\newcommand{\mathO}{\mathcal{O}}

\title{State preparation by shallow circuits using feed forward}

\author[1]{Harry Buhrman}
\author[1]{Marten Folkertsma}
\author[2]{Bruno Loff}
\author[1,3]{Niels M. P. Neumann}
\date{}%September 23, 2022}

\affil[1]{QuSoft, CWI \& University of Amsterdam, Amsterdam, the Netherlands}
\affil[2]{LASIGE \& Department of Mathematics, University of Lisbon}
\affil[3]{The Netherlands Organisation for Applied Scientific Research (TNO), Delft, the Netherlands}


\begin{document}
\maketitle
\begin{abstract}
Fault tolerant quantum computers repetitively apply a four-step procedure: 
First, perform a few one and two-qubit quantum gates. 
Second, perform a syndrome measurement on a subset of the qubits. 
Third, perform fast classical computations to establish if and where errors occurred. 
And, fourth, correct the errors with a correction step. 
The next iteration applies the same procedure with new one and two-qubit gates. 
Even though current error-rates prohibit this procedure to work and fault tolerant quantum computing remains a distant goal, the same procedure can already prove useful today. 
In this work we make use of this four-step scheme not to carry out fault-tolerant computations, but to enhance short, {\em constant}-depth, quantum circuits that perform 1 qubit gates and {\em nearest-neighbor} 2 qubit gates.

We introduce a new computational model called \emph{Local Alternating Quantum Classical Computations} ($\LAQCC$). 
In this model, qubits are placed in a grid and they can only interact with their direct neighbors; the quantum circuits are of constant depth with intermediate measurements; a classical controller can perform log-depth computations on these intermediate measurement outcomes and control future quantum operations based on the outcome.
This model fits naturally between quantum algorithms in the NISQ era and full-fledged fault-tolerant quantum computation. 
We first prove that any Clifford circuit has an equivalent $\LAQCC$ circuit, and that any $\LAQCC$ circuit can be simulated by a $\QNC^1$ circuit. 
Next, we conjecture the non-simulatability of $\LAQCC$ by showing that $\LAQCC$ contains the class of Instantaneous Quantum Polynomial-time circuits.
We also show that any $\LAQCC$ circuit with polynomial-sized quantum circuits and unbounded classical computations is contained in the class of quantum circuits equipped with post-selection gates with respect to the task of state preparation.
We continue by presenting $\LAQCC$ implementations of different subroutines, including OR-gates, quantum Fourier transforms and Threshold gates. 
These subroutines prove vital in constructing three state preparation routines in the main part of this work. 
Preparing a uniform superposition uses constant-depth arithmetic gates, combined with an exact Grover implementation by Long. 
For the $W$-state, we employ a compress-uncompress method to switch between a binary and one-hot encoding. 
This method extends to the more generalized Dicke-states, the superposition of $n$-bit strings of Hamming weight $k$, for $k=\mathO(\sqrt{n})$, but fails for higher $k$ due to the birthday paradox. 
We extend this protocol to a protocol that prepares many-body scar states, highly excited states with low entanglement and longer coherence times than states with the same energy density. 
We present a circuit for preparing Dicke-states for larger $k$ requiring log-depth circuits that maps between the factoradic number system and the combinatorial number system. 
%This paper concludes with a first exploration of the computational limits of $\LAQCC$ by showing its inclusion in the class of quantum circuits equipped with post-selection gates with respect to the task of state preparation. 
%This inclusion even holds if we allow for polynomial-deep quantum circuits, unbounded classical computations and a polynomial number of alternations between the two. 


%In order to achieve fault-tolerant quantum computation, we need to repeat the following sequence of four steps after we have initialized the quantum device. 
%First, we perform 1 or 2 qubit quantum gates (in parallel if possible). Second, we do a syndrome measurement on a subset of the qubits. Third, we perform a fast classical computation to establish which errors have occurred (if any). And, fourth, depending on the errors, we apply a correction step. 
%Then the procedure repeats with the next sequence of gates. 
%These four steps are essential to accomplish fault-tolerant quantum computing. 

%In order for these four steps to succeed, we need the error rate of the  gates to be below a certain threshold. Unfortunately, the error rates of current quantum hardware are still too high and do not meet this requirement. On the other hand, current quantum hardware platforms are designed with these four steps in mind.  In this work we make use of this four-step scheme not to carry out fault-tolerant computations, but to enhance short, {\em constant}-depth, quantum circuits that perform 1 qubit gates and {\em nearest-neighbor} 2 qubit gates. 
%To explore how this can be useful, we study a computational model which we call \emph{Local Alternating Quantum Classical Computations} ($\LAQCC$). 
%In this model, qubits are placed in a grid and they can only interact with their direct neighbors; the quantum circuits are of constant depth with intermediate measurements; a classical controller can perform log-depth computations on these intermediate measurement outcomes and control future quantum operations based on the outcome.
%This model fits naturally between quantum algorithms in the NISQ era and full fledged fault-tolerant quantum computation. 

%We show how an $\LAQCC$ circuit can create long-ranged interactions, which constant-depth quantum circuits cannot achieve, and use it to construct a range of useful multi-qubit operations. 
%With these gates, we create three new state preparation protocols for a uniform superposition over an arbitrary number of states, $W$-states and Dicke states, the generalization of $W$-states. Furthermore, we show that this type of model contains circuits which are unlikely to be classically simulatable, as well as bound the power of this model by showing an inclusion into $\QNC^1$

%We conclude by studying the power of intermediate classical calculations, by defining a more powerful version of $\LAQCC$. 
%This version has quantum circuits of with  polynomial size and uses {\em unbounded} classical computations. 
%We use ideas from Kolmogorov complexity to show that this new class is bounded in capabilities and that post-selecting on measurement outcomes is more powerful. 
\end{abstract}

\section{Introduction}
Current quantum hardware is unable to carry out universal quantum computations due to the buildup of errors that occur during the computation. 
The magnitude of the individual error is currently above the value that the Threshold Theorem requires in order to kick-start quantum error correction and fault-tolerant quantum computation~\cite[Section 10.6]{nielsen_chuang_2010}. 
Although the experimentally achieved fidelity rates are promising and the error bounds are inching closer to the required threshold, we will have to work for the foreseeable future with quantum hardware with errors that build-up during the computation.  This implies that we can only do a limited number of steps before the output of the computation has become completely uncorrelated with the intended one.

For fault-tolerant quantum computing, we repeat four steps: 
1) We apply a number of single and two-qubit quantum gates, in parallel whenever possible; 
2) We perform a syndrome measurement on a subset of the qubits; 
3) We perform fast classical computations to determine which errors have occurred and how to correct them; 
and, 4) We apply correction terms based on the classical computations.
We then repeat these four steps with a next sequence of gates. 
These four steps are essential to fault-tolerant quantum computing. 


The starting point of this work is to use the four steps outlined above, not to carry out error correction and fault-tolerant computation, but to enhance short, constant-depth, {\em uncorrected} quantum circuits that perform single qubit gates and {\em nearest-neighbor} two qubit gates. 
Since in the long run we will have to implement error-correction and fault-tolerant computation anyhow, and this is done by such a four-step process, why not make other use of this architecture? Moreover, on some of the quantum hardware platforms, these operations are already in place.
Embracing this idea we naturally arrive at the question: what is the computational power of \textit{low-depth} quantum-classical circuits organized as in the four steps outlined above? 
We thus investigate circuits that execute a small, ideally constant, number of stages, where at each stage we may apply, in parallel, single qubit gates and {\em nearest-neighbor} two qubit gates, followed by measurements, followed by low-depth classical computations of which the outcome can control quantum gates in later stages. 
It is not clear, at first, whether such circuits, especially with constant depth, can do anything remotely useful. 
But we will see that this is indeed the case: many quantum computations can be done by such circuits in constant depth. 
By parallelizing quantum computations in this way, we improve the overall computational capabilities of these circuits, as we do not incur errors on qubits that are idle, simply because qubits are not idle for a very long time. 
Furthermore, reducing the depth of quantum circuits, at the cost of increasing width, allows the circuit to be run faster even if errors occur.

The first usage of such a four-step layout, not to do error correction, but to perform computations, can be found in the paradigm of measurement-based quantum computing~\cite{gottesman1999demonstrating,raussendorf2001one,jozsa2006introduction,clark2007generalised}: 
A universal form of quantum computing where a quantum state is prepared and operations are performed by measuring qubits in different bases, depending on previous measurements and intermediate measurements.

\citeauthor{PhamSvore2013} were the first to formalize the four-step protocol for performing computations~\cite{PhamSvore2013}. They included specific hardware topologies by considering two-dimensional graphs for imposing constraints on qubit interactions. In their model, they develop circuits for particularly useful multi-qubit gates, including specifying costs in the width, number of qubits, depth, number of concurrent time steps, size, and total number of non-Identity operations.
As a result, they find an algorithm that factors integers in polylogarithmic depth.
\citeauthor{Browne:2011} showed that the main tool in the work by \citeauthor{PhamSvore2013}, the fan-out gate, can also be replaced by additional log-depth classical computations in the measurement-based quantum computing setting~\cite{Browne:2011}.

More recently, \citeauthor{Cirac:2021} introduced a scheme to implement unitary operations involving quantum circuits combined with Local Operations and Classical Communication ($\mathsf{LOCC}$) channels: $\mathsf{LOCC}$-assisted quantum circuits~\cite{Cirac:2021}. Similarly to the four-step scheme we just described, they allow for a short depth circuit to be run on the qubits, followed by one round of $\mathsf{LOCC}$, in which ancilla qubits are measured and local unitaries are applied based on the measurement outcomes. They show that in this model any 1D transitionally invariant matrix-product state (MPS) with fixed bond dimension is in the same phase of matter as the trivial state. Similar ideas can be found in~\cite{TVV_NonAbelianTopologicalOrder_2022, tantivasadakarn2021long}.

In this work, we introduce a new model, called \textit{Local Alternating Quantum-Classical Computations} ($\LAQCC$). In this model we alternate between running quantum circuits (constrained by locality), ending in the measurement of a subset of qubits, and fast classical computations based on the measurement results. The outcome of the classical computations are then used to control future quantum circuits. We allow for flexibility in this model, by giving different constraints to the power of both the quantum circuits and the classical circuits as well as the number of alternations between them. 
Most attention will be given to $\LAQCC$ containing quantum circuits of constant depth, classical circuits of logarithmic depth and at most a constant number of alternations between them. 
Any circuit constructed in this model is considered to be of constant depth. 
We restrict ourselves to logarithmic depth classical computations, as this is the first natural and non-trivial extension beyond constant-depth classical computations. 
Constant-depth classical computations do however also have an equivalent constant-depth quantum implementation.

The definition of $\LAQCC$ sharpens the original definition of \citeauthor{PhamSvore2013} by adding constraints to the intermediate classical computations. This allows us to bound the power of $\LAQCC$ from above. 

The main result of \citeauthor{Cirac:2021}, that 1D translational invariant MPS with fixed bond dimension can be prepared by $\mathsf{LOCC}$-assisted circuits, relies on local symmetries of the MPS. These symmetries allow them to prepare local states (on a constant number of qubits) and glue them together by doing one round of the appropriate entangling measurement and corrections, after which they run a round of local unitaries to get the desired result. This general scheme for preparing states that exhibit an MPS description with the appropriate local symmetries requires only geometrically local unitaries and one round of measurement and corrections an therefore is accessible in $\LAQCC$. Studying different local symmetries, known as Symmetry Protected Topological (SPT) phases of matter, to find measurement-based constant depth circuits for states is a broad ongoing field of research~\cite{TVV_NonAbelianTopologicalOrder_2022, tantivasadakarn2021long, smith2023deterministic}. 
All these schemes have a $\LAQCC$ implementation.

%$\LAQCC$-circuits also exist for general schemes of preparing local states, based on the local tensors, and gluing them together using one round of entangled measurement and corrections, based on the local symmetry. 
%The main result of \citeauthor{Cirac:2021}, that 1D translational invariant MPS with fixed bond dimension can be prepared by $\mathsf{LOCC}$-assisted circuits, relies heavily on local symmetries of the MPS and as a result also has an equivalent $\LAQCC$ implementation. 
%The corrections applied after the measurement round are local unitaries depending on the local symmetries of the MPS. 

 

%This general scheme of preparing local states, based on the local tensors, and gluing it together by doing one round of entangled measurement and corrections, based on the local symmetry, is accessible in $\LAQCC$.
Note however that \citeauthor{Cirac:2021} also suggest a circuit for the $W$-state.
This circuit uses sequentially and dependent measurement-based corrections of the ancilla qubits. 
These dependent measurements translate to sequential alternations between the quantum and classical circuits and therefore increase the total depth to linear depth, exceeding the constant-depth constraints imposed by $\LAQCC$-circuits. 

We study the power of the $\LAQCC$ model with respect to state preparation, showing that even with only constant quantum-depth and logarithmic classical depth it remains possible to prepare states with long-range entanglement.
Another surprising result is that it is unlikely that $\LAQCC$ circuits are classically simulatable. We show that any instantaneous quantum polynomial-time (IQP) circuit~\cite{Bremner2010,Shepherd2009} has an $\LAQCC$ implementation.
Classical simulation of IQP circuits implies the collapse of the polynomial hierarchy to the third level, which is not believed to be true~\cite{Bremner2017}. Therefore, we expect that $\LAQCC$ circuits are unlikely to be classically simulatable. We bound the power of $\LAQCC$ by showing that it is contained in $\QNC^1$, the class of polynomial-size, log-depth circuits.

Next, we also study the power that intermediate classical calculations can add to quantum computations, by considering a new model that alternates between polynomially many polynomial-depth quantum circuits and unbounded classical computations
We study this model by doing a complexity theoretical analysis, where we draw inspiration from the notions of complexity given by \citeauthor{RosenthalYuen:2022}, \citeauthor{MetgerYuen:2023}, and \citeauthor{Aaronson:2004}.
All three complexity notions are based on the notion of state preparation, instead of more traditional definition of complexity such as the decidability of a computational problem. 
The first two consider classes based on sequences of quantum states preparable by a polynomial-sized quantum circuit, where the circuits are uniformly generated by a computational class, for instance, the class $\mathsf{PSPACE}$, which results in the complexity class $\mathsf{StatePSPACE}$~\cite{RosenthalYuen:2022,MetgerYuen:2023}.
The third notion considers a relative complexity, where the complexity is measured between two given states, and is measured by the number of gates, from a given gate-set, required to transform one state in another state~\cite{Aaronson:2004}. 
For our definition of state preparation complexity, we drop the uniformity constraint from~\cite{RosenthalYuen:2022,MetgerYuen:2023} and define a class as $\mathsf{StateX}$, which refers to states preparable by circuits of type $\mathsf{X}$. 
As an example, if $\mathsf{X} = \QNC^0$, this results in the class $\mathsf{StateQNC^0}$, which is the set of states preparable from the $\ket{0}^n$ state by poly-size constant-depth circuits. 
This notion is similar to the relative complexity from~\cite{Aaronson:2004}, where one state is the  $\ket{0}^n$ state and instead of counting the number of gates we consider the set of states preparable by a fixed number of gates. Using this notion of complexity we show that any state preparable by an $\LAQCC^*$ circuit is also preparable by a $\mathsf{PostQPoly}$ circuit, the class of circuits of polynomial depth with an additional post-selection gate. 

All Clifford circuits have a constant-depth $\LAQCC$ implementation, implying that any stabilizer state can be implemented by a constant-depth $\LAQCC$ circuit, see Section~\ref{sec:clifford_circuits} for a proof of this statement. 
Efficient circuits for stabilizer states have been known already through measurement-based quantum computing. Therefore this paper focuses on the preparation of non-stabilizer states, and as a surprising result we find novel constant-depth protocols for four very natural classes of non-stabilizer states.
Despite the extensive research into these four classes of non-stabilizer states and the many applications of them, no efficient constant- or low-depth state preparation protocols are known yet. We specifically consider these four classes as they are all often used as initial states in other algorithms.

The first state is a uniform superposition over an arbitrary number of states. 
This state finds applications in many quantum algorithms, as they often start with a uniform superposition over multiple states. 
This superposition is often achieved by applying Hadamard gates to every qubit due to its simplicity to prepare. 
Yet, the analysis of many algorithms, such as Shor's algorithm~\cite{Shor:1997}, would benefit from a different initial superposition. 
The circuit to prepare the uniform superposition over an arbitrary number of states uses an exact version of Grover search as a subroutine, that turns a probabilistic circuit, with a known constant probability of success, into a deterministic circuit. 
We use the circuit for preparing a uniform superposition over an arbitrary number of states as a subroutine in the next two quantum state preparation protocols. 

The second state is the $W$-state, the uniform superposition over all computational basis states of Hamming-weight~$1$, a natural long-ranged entangled state that displays a fundamentally nonequivalent type of entanglement from the Greenberger–Horne–Zeilinger state~\cite{WState:2000}, for which $\LAQCC$-type constant-depth circuits were previously known~\cite{PhamSvore2013, Cirac:2021}. 
The $W$-state is often used as benchmark for new quantum hardware~\cite{Haffner2005,Neeley2010,GarciaPerez:2021}. 
A novel way to prepare the $W$-state therefore gives a new way to benchmark different quantum devices with each other. 
A circuit for preparing the $W$-state was given in~\cite{Cirac:2021}, but this implementation requires sequentially alternating measurements followed by local unitaries, which in the $\LAQCC$ model is not considered to be of constant depth. 
We improve this protocol by giving an $\LAQCC$ implementation of the $W$-state, based on a compress-uncompress method that links the one-hot and binary encoding of integers.

The third state considered is the Dicke state, a generalization of the $W$-state, a superposition over all computational basis states with Hamming-weight $k$~\cite{Dicke:1954}. 
Dicke states have relevance in various practical settings.
For instance, for quantum game theory~\cite{zdemir2007}, quantum storage~\cite{Bacon_Compress:2006,Plesch:2010}, quantum error correction~\cite{ouyang2014permutation}, quantum metrology~\cite{toth2012multipartite}, and quantum networking~\cite{prevedel2009experimental}. 
Dicke states have been used as a starting state for variational optimization algorithms, most notably Quantum Alternating Operator Ansatz (QAOA)~\cite{Hadfield2019}, to find solutions to problems such as Maximum k-vertex Cover~\cite{Brandhofer2022,cook2020quantum}.
The ground states of physical Hamiltonians describing one-dimensional chains tend to show a resemblance to Dicke states such as states resulting from the Bethe ansatz, making them an ideal starting state when investigating the ground state behavior of these Hamiltonians~\cite{TDL_BetheAnsatzDerivation:2010,B_ExcitedStateQuantumPhaseTransitions:2013,DickeTransitions:2021}. 
For instance, the algorithm by \citeauthor{van2021preparing}, who give an algorithm to prepare the Bethe ansatz eigenstates of the spin-1/2 XXZ spin chain, starts by first preparing a Dicke state~\cite{van2021preparing}. 
A Dicke-state preparation protocol based on the compress-uncompress methodology used in the $W$-state furthermore finds applications in entanglement distillation, where the entanglement of a large state is concentrated on only a few qubits. 
Efficient deterministic circuits for preparing Dicke states have been proposed by \citeauthor{bartschi2019deterministic}~\cite{bartschi2019deterministic, bartschi2022deterministic_short_depth}. 
They provide a quantum circuit of depth $\mathO(k \log(\frac{n}{k}))$, allowing arbitrary connectivity, to prepare a Dicke state, which they conjecture to be optimal when $k$ is constant. 
In this work, we provide a constant-depth $\LAQCC$ circuit below their conjectured bound already for constant $k$. 
However, this does not directly disprove their conjecture, as we allow for intermediate measurements and classical computations. 
More significantly, we even construct constant-depth $\LAQCC$ circuits for $k = \mathO(\sqrt{n})$ greatly improving their bound.
This construction extends the compress-uncompress method for the $W$-state combined with additional subroutines. 

We continue with a log-depth state preparation protocol for the Dicke-state for arbitrary $k$. 
This protocol implements an efficient transformation between the factoradic number representation and the combinatorial number representation of a positive integer. 
The combinatorial number representation relates directly to the Dicke state. 
The provided efficient transformation between number representation systems might be of independent interest. 

We conclude by modifying our protocol for preparing a Dicke-state to a protocol that prepares quantum many-body scar states in constant-depth. 
These states have low entanglement and longer coherence times than states with similar energy density.
These characteristics make many-body scar states interesting to analyze and relevant within physics.
Many-body scar states appear for instance in the AKLT model~\cite{AKLT:1987,MRBAR:2018,MRB:2018} and different spin models~\cite{SI:2019,MOBFR:2020}.
Known methods for preparing these states have polynomial-depth~\cite{Gustafson:2023}, whereas our circuit has constant depth. 

% We conclude by studying the power that intermediate classical calculations can add to quantum computations. 
% In this study, we define a new model that relaxes constant-depth quantum circuits to polynomial depth quantum circuits, log-depth classical calculations to unbounded classical computations and a constant number of alternations to a polynomial number of alternations. 
% We call this model $\LAQCC^*$. 
% We study this model by doing a complexity theoretical analysis, where we draw inspiration from the notions of complexity given by \citeauthor{RosenthalYuen:2022}, \citeauthor{MetgerYuen:2023}, and \citeauthor{Aaronson:2004}.
% All three complexity notions are based on the notion of state preparation, instead of more traditional definition of complexity such as the decidability of a computational problem. 
% The first two consider classes based on sequences of quantum states preparable by a polynomial-sized quantum circuit, where the circuits are uniformly generated by a computational class, for instance, the class $\mathsf{PSPACE}$, which results in the complexity class $\mathsf{StatePSPACE}$~\cite{RosenthalYuen:2022,MetgerYuen:2023}.
% The third notion considers a relative complexity, where the complexity is measured between two given states, and is measured by the number of gates, from a given gate-set, required to transform one state in another state~\cite{Aaronson:2004}. 
% For our definition of state preparation complexity, we drop the uniformity constraint from~\cite{RosenthalYuen:2022,MetgerYuen:2023} and define a class as $\mathsf{StateX}$, which refers to states preparable by circuits of type $\mathsf{X}$. 
% As an example, if $\mathsf{X} = \QNC^0$, this results in the class $\mathsf{StateQNC^0}$, which is the set of states preparable from the $\ket{0}^n$ state by poly-size constant-depth circuits. 
% This notion is similar to the relative complexity from~\cite{Aaronson:2004}, where one state is the  $\ket{0}^n$ state and instead of counting the number of gates we consider the set of states preparable by a fixed number of gates. Using this notion of complexity we show that any state preparable by an $\LAQCC^*$ circuit is also preparable by a $\mathsf{PostQPoly}$ circuit, the class of circuits of polynomial depth with an additional post-selection gate. 

\paragraph{Summary of results}
\begin{itemize}
    \item We give a new definition of a computational model that captures the power of the four step process: applying a constant number of layers of one- and two-qubit gates; performing a syndrome measurement; perform a fast classical computation determining corrections; apply corrections. We call this model \emph{Local Alternating Quantum Classical Computations}, or $\LAQCC$ for short. In this model we bound the allowed quantum operations, intermediate classical calculations, and number of rounds separately. In Section~\ref{sec:LAQCC_model} we define this model and give a list of operations based on results from literature contained in this computational model. In some of these operations we explicitly use that we allow for multiple, but at most constant, rounds  of corrections.
    \item  We show show that there exist $\LAQCC$ circuits that can not be weakly simulated in Section~\ref{sec:IQP_in_LAQCC}. We further show that for every $\LAQCC$ circuit there exists a $\QNC^1$ circuit simulating it perfectly, in Section~\ref{sec:LAQCC_in_QNC1}.
    \item We introduce a new type computational complexity for preparing states and show that the extension of $\LAQCC$ where we allow a polynomial number of rounds and unbounded classical computation, is contained in $\mathsf{PostQPoly}$, the class of polynomial circuits with post-selection, in Section~\ref{sec:Complexity results}.
    \item We show a protocol to prepare the uniform superposition state of size $q$ in $\LAQCC$ using $\mathO(\ceil{\log_2(q)}^2)$ qubits in Section~\ref{sec:superposition_modulo_q}. 
    \item We show a protocol to prepare the $W_n$ state in $\LAQCC$ using $\mathO(n\log(n))$ qubits in Section~\ref{sec:W_state_in_LAQCC}.
    \item We show two ways of preparing the Dicke-$(n,k)$ state. The first method is in $\LAQCC$, works up to $k = \mathO(\sqrt{n})$, uses $\mathO(n^2\log(n))$ qubits, and is found in Section~\ref{sec:dicke:small_k}. The second method is in $\LAQCC\text{-}\mathsf{LOG}$ (an extension of $\LAQCC$ allowing for logarithmic number of alterations instead of constant), works for any $k$, uses $\mathO(\text{poly}(n))$ qubits, and is found in Section~\ref{sec:Dicke_in_LAQCC_LOG}. 
    \item We extend on our $\LAQCC$ method of generating Dicke-$(n,k)$ states for $k = \mathO(\sqrt{n})$ and show a protocol to generate many-body scar states for a particular Hamiltonian in $\LAQCC$ (Section~\ref{sec:many_body_scar}). 
\end{itemize}
Summarized in a table, we provide the following state generation protocols:
\begin{table}[htb]
\centering
\begin{tabular}{l|l|l|l}
\textbf{State description} & \textbf{Width} & \textbf{Depth} & \textbf{Implementation}\\
\hline 
Uniform superposition mod $q$: $\frac{1}{\sqrt{q}} \sum_{i = 0}^{q-1}\ket{i}$ & $\mathO(\ceil{\log^2 q})$ & $\mathO(1)$ & Section~\ref{sec:superposition_modulo_q}\\

$W$-state: $\frac{1}{\sqrt{n}}\sum_{i = 0}^{n-1}\ket{e_i}$ & $\mathO(n \log n)$ & $\mathO(1)$ & Section~\ref{sec:W_state_in_LAQCC}\\

Dicke-$(n,k)$, $k = \mathO(\sqrt{n})$: $\binom{n}{k}^{-1/2}\sum_{x \in \{0,1\}^n: |x| = k} \ket{x}$ &  $\mathO(n^2\log n)$ & $\mathO(1)$ 
&Section~\ref{sec:dicke:small_k}\\

Dicke-$(n,k)$: $\binom{n}{k}^{-1/2}\sum_{x \in \{0,1\}^n: |x| = k} \ket{x}$ & $\mathO(\text{poly}(n))$ & $\mathO(\log n)$ &Section~\ref{sec:Dicke_in_LAQCC_LOG}\\

QMBS: $\ket{S_k} = \frac{1}{k! \sqrt{\mathcal N(n,k)}}(Q^\dagger)^k \ket{\Omega}$ &  $\mathO(n^2\log n)$ & $\mathO(1)$  &  Section~\ref{sec:many_body_scar}
\end{tabular}
\caption{Summary of state preparation protocols given in this paper.}
\label{tab:sate_prep}
\end{table}
In the entry for the quantum many-body scar state $Q$ denotes the raising operator and $\mathcal N(n,k)=\binom{n-k-1}{k}$. 
Section~\ref{sec:many_body_scar} will provide more details on the variables and the implementation. 

\paragraph{Organization of the paper}
\noindent We first introduce relevant preliminaries in Section~\ref{sec:preliminaries}. 
In Section~\ref{sec:LAQCC_model} we formally define the class of Local Alternating Quantum-Classical Computations ($\LAQCC$). We also show that any Clifford circuit can be implemented in constant depth $\LAQCC$ (a result based on a result from measurement-based quantum computing~\cite{jozsa2006introduction}). 
This result allows us to give many useful multi-qubit gates and routines in Section~\ref{sec:gates_created_in_LAQCC}. 
Beyond that we show that constant depth $\LAQCC$ circuits are contained in $\QNC^1$ and that any $\mathsf{IQP}$ circuit has an $\LAQCC$ implementation.
We conclude this section with an analysis of a more powerful instantiation of $\LAQCC$ and show an inclusion with respect to the class $\mathsf{PostQPoly}$, which is the class of circuits of polynomial depth with one additional post-selection gate. 
In Section~\ref{sec:state_prep_in_LAQCC} we give $\LAQCC$ circuit implementations for preparing the uniform superposition over an arbitrary number of states, the $W$-state and the Dicke state up to $k = \mathO(\sqrt{n})$. We furthermore give a log-depth circuit implementation for preparing the Dicke state for any $k$. We conclude by showing a $\LAQCC$ circuit for generating many body scar states of a particular type of Hamiltonian.


\section{Preliminaries}
In this section, we describe the necessary background for automated planning and the significance of the International Planning Competition. 

% \subsection{Ontology}
% A formal ontology is typically represented as a set of concepts, relations, and axioms. A concept represents a set of objects or entities that share common properties, while a relation represents a connection or association between two or more concepts. Axioms are statements that define the relationships between concepts and relations. It is a formal representation of knowledge that is designed to facilitate automated reasoning and information processing. It acts as a structured vocabulary that describes a domain and promotes interoperability, data integration, and communication between humans and machines. Formally, an ontology $O$ can be represented as a tuple $(C, R, A)$, where $C$ is the set of concepts, $R$ is the set of relations, and $A$ is the set of axioms. Each concept \textit{c} $\in$ $C$ can be represented as a set of attributes, denoted as $Att(c)$. Similarly, each relation \textit{r} $\in$ $R$ can be represented as a set of attributes, denoted as $Att(r)$.

% Ontology is a branch of philosophy that deals with the nature of existence and being. In the field of computer science, however, ontology refers to a formal representation of knowledge that is designed to facilitate automated reasoning and information processing. It is a structured vocabulary that describes a domain and promotes interoperability, data integration, and communication between humans and machines. Various tools and methodologies, including Protege and ontology editors, are available for ontology creation. Ontologies are increasingly important in artificial intelligence, knowledge engineering, and the semantic web, and researchers are exploring their potential in diverse domains and applications.

% Figure environment removed

\subsection{Automated Planning}

Automated planning, also known as AI planning, is the process of finding a sequence of actions that will transform an initial state of the world into a desired goal state \cite{ghallab2004automated}. It involves constructing a plan or a sequence of actions that will achieve a specified objective while respecting any constraints or limitations that may be present. Formally, automated planning can be defined as a tuple $(S, A, T, I, G)$, where:
\begin{itemize}
    \item $S$ is the set of possible states of the world
    \item $A$ is the set of possible actions that can be taken
    \item $T$ is the transition function that describes the effects of taking an action on the current state of the world
    \item $I$ is the initial state of the world
    \item $G$ is the desired goal state
\end{itemize}
Using this notation, the problem of automated planning can be framed as finding a sequence of actions $\prec a_1, a_2, ..., a_k\succ$ that will transform the initial state $I$ into the goal state $G$, while respecting any constraints or limitations on the actions. 
 % In automated planning, 
 A problem is defined in terms of a domain and a problem instance. The domain defines the possible actions that can be taken and the effects of each action, while the problem instance specifies the initial state of the world and the desired goal state. 
Various techniques can be used to solve the planning problem, such as search algorithms, constraint-based reasoning, and optimization methods. These techniques involve exploring the space of possible plans and selecting the one that satisfies the objective and any constraints. Figure \ref{fig:planning_bw} illustrates an automated planning scenario for the blocksworld domain, where an initial state can be transformed into a goal state by executing a sequence of actions.

% \noindent \textbf{Attributes modeled about a domain.}
%   %\noindent \textbf{Attributes modeled in a domain file}
%  \begin{enumerate}
%      \item \textbf{Requirements:} A list of requirements that the planner must satisfy in order to solve the domain. Requirements include durative actions, conditional effects, or negative preconditions. For example, in blocksworld domain with types involved, one of the requirements is \emph{typing}.
%     \item \textbf{Predicates:} Predicates are fundamental elements in the planning domain that define the properties of the world. They are used to describe the initial and goal states, as well as the preconditions and effects of actions. Predicates are usually defined as logical expressions over a set of variables, where each variable can take on a finite number of values. In the context of planning, predicates are typically used to represent facts about the world that can be true or false, such as the location of an object or the status of a machine. For example, in blocksworld domain, the predicate \verb|(on b1 b2)| could indicate that block 'b2' is on top of block 'b1'.
%      \item \textbf{Actions:} Actions are the basic units of change in the planning domain. They represent atomic operations that can be performed to transform the world from one state to another. Each action has a name, a set of parameters, preconditions that must be satisfied before the action can be executed, and effects that describe the changes that the action makes to the world. Actions can be used to model a wide variety of operations, ranging from simple movements or transformations to complex processes such as planning or decision-making. For example, in blocksworld domain, the action \verb|unstack b2 b1| can be used to unstack block 'b2' from block 'b1'. 
     
%      \item \textbf{Preconditions:} Preconditions are the conditions that must be true before an action can be executed. They are usually defined using predicates and can involve multiple variables. Preconditions can also be negative, which means that a certain condition must not be true for an action to be executed. In planning, preconditions ensure that actions are only executed when the necessary conditions have been met, such as ensuring that a machine is turned off before it is serviced. For example, in blocksworld domain, the action \verb|unstack b2 b1| has a precondition of \verb|(on b1 b2)|, meaning that for the action to be valid, the block 'b2' should be on top of block 'b1'.
     
%      \item \textbf{Effects:} Effects describe the changes that an action makes to the world. They are usually defined using predicates and can involve multiple variables. Effects can be positive, which means that a certain condition becomes true after the action is executed, or negative, which means that a certain condition becomes false after the action is executed. In the context of planning, effects are used to model the changes that result from executing an action, such as moving an object from one location to another or turning a machine on. For example, in blocksworld domain, when the action \verb|unstack b2 b1| is executed, one of its effect is \verb|(not (on b1 b2))|, indicating that block 'b2' is no longer on top of block 'b1'.
     
%      \item \textbf{Constants:} Constants are values that are fixed and do not change during the execution of the planning problem. They are used to represent objects or entities in the world that have a fixed value, such as the speed limit on a road. Constants can be used to simplify the planning problem by reducing the number of variables that need to be considered and by providing a fixed set of values that can be used in predicates and actions. For example, in blocksworld domain, the constant \emph{table} could represent the surface on which the blocks are initially placed.
     
%      \item \textbf{Types:} Types are used to classify objects or entities in the world based on their attributes or properties. They are used to define the domain of values that a variable can take on and can be used to constrain the values that are assigned to variables. In the context of planning, types are typically used to group related objects or entities together, such as cars or bicycles, and to specify the properties that are common to all members of a type, such as their color or size. For example, in blocksworld domain with types involved, one can represent the predicate as \verb|(on ?x - block ?y - block)| stating that the parameters in the predicate are of type \emph{block}.

%  \end{enumerate}


% ######### Shorter version for AI Planning preliminaries
% \subsection{Automated Planning}

% Automated planning, also known as AI planning, finds actions transforming an initial world state into a goal state \cite{ghallab2004automated}. It involves creating a plan, respecting constraints, defined as $(S, A, T, I, G)$ where $S$ is the world states set, $A$ is the actions set, $T$ is the state transition function, $I$ is the initial state, and $G$ is the goal state. The challenge is to find actions $\prec a_1, a_2, ..., a_k\succ$ converting $I$ to $G$ under constraints. 

% A problem has a domain (defining actions and effects) and an instance (specifying initial and goal states). Various techniques can be used to solve the planning problem, such as search algorithms, constraint-based reasoning, and optimization methods. These techniques involve exploring the space of possible plans and selecting the one that satisfies the objective and any constraints. Figure \ref{fig:planning_bw} illustrates an automated planning scenario for the blocksworld domain, where an initial state can be transformed into a goal state by executing a sequence of actions.

\noindent \textbf{Attributes modeled about a domain.}
 \begin{enumerate}
     \item \textbf{Requirements:} A list of requirements that the planner must satisfy to solve the given domain, e.g., \emph{typing} in blocksworld with types.
     \item \textbf{Predicates:} Define world properties, e.g., \verb|(on b1 b2)| in blocksworld.
     \item \textbf{Actions:} Units of change with preconditions and effects, e.g., \verb|unstack b2 b1| in blocksworld.
     \item \textbf{Preconditions:} Conditions for action execution, e.g., \verb|(on b1 b2)| for \\ \verb|unstack b2 b1|.
     \item \textbf{Effects:} Post-action world changes, e.g., \verb|(not (on b1 b2))| after \\ \verb|unstack b2 b1|.
     \item \textbf{Constants:} Fixed values, e.g., \emph{table} in blocksworld.
     \item \textbf{Types:} Classifications based on attributes, e.g., \\ \verb|(on ?x - block ?y - block)| in typed blocksworld.
 \end{enumerate}

\noindent \textbf{Attributes modeled about a problem instance from a domain.}
\begin{enumerate}
    \item \textbf{Name:} The name of the planning problem.
    \item \textbf{Domain:} The name of the planning domain that the problem belongs to.
    \item \textbf{Objects:} A list of objects that are present in the planning problem. Objects are typically defined in terms of their type and name. In the example shown in Figure \ref{fig:planning_bw}, objects are b1, b2, and b3.
    \item \textbf{Initial State:} A description of the initial state of the world, including the values of all relevant predicates. Figure \ref{fig:planning_bw} represents an example initial state.
    \item \textbf{Goal State:} A description of the desired goal state of the world, including the values of all relevant predicates. Figure \ref{fig:planning_bw} represents an example goal state.
\end{enumerate}

% \vspace{2cm}
\subsection{International Planning Competition (IPC)}

% IPC serves as a significant means of assessing and comparing various planning systems. By presenting new planners and benchmark problems each year, the competitions aim to stimulate the advancement of new planning methodologies and reflect current trends and challenges in the field. The competition comprises multiple tracks, each covering various planning problems such as classical, temporal, and probabilistic planning. These tracks include benchmark problems that evaluate the performance of planners concerning parameters such as plan quality, plan length, and run time. The results of these competitions provide insights into the current state-of-the-art in planning and help identify the strengths and weaknesses of different planning systems. IPC can serve as an excellent starting point for building a planning-related ontology as the benchmark problems used in these competitions can provide a comprehensive overview of the domain and the types of problems that planners need to solve. 

IPC is pivotal for evaluating and contrasting planning systems. Introducing new planners and benchmarks, it promotes innovative planning methodologies and reflects the field's evolving challenges. The competition has multiple tracks, such as classical and probabilistic planning, with benchmarks assessing plan quality, length, and run time. IPC results offer a glimpse into the latest in planning, highlighting system pros and cons. The benchmarks from IPC are ideal for crafting a planning-related ontology, encapsulating the domain's breadth and planners' challenges.

% !TEX program = pdflatex
% !TEX root = main.tex


\section{The Model}

We represent a series of interactions between $N$ individuals as a sequence of weighted directed networks with adjacency matrix $A^t$ for $t=0,1,2,\ldots,T$. For each $t$, its entry $A_{ij}^t$ is the outcome of interactions $i \rightarrow j$ suggesting that $i$ is ranked above $j$. This allows both cardinal and ordinal inputs. For instance, in team sports, $A_{ij}^t$ could be the number of points by which team $i$ beat team $j$, or we could simply set $A_{ij}^t=1$ to indicate that $i$ won and $j$ lost. We can include the case where individuals interact multiple times at time $t$ by summing the corresponding entries.

We assume that the values of $A_{ij}^t$ are influenced by a vector of real-valued ranks $\v{s}^t=(s_{1}^t,\dots, s_{N}^t)$, where $s_i^t$ is $i$'s skill, strength or prestige at time $t$.
To model these interactions, we follow SpringRank's approach of imagining the network as a physical system~\cite{de2018physical}. Specifically, each node $i$ is embedded in $\mathbb{R}$ at position $s_i^t$, and each directed edge $i \rightarrow j$ becomes an oriented spring with a non-zero resting length and displacement $s_i^t-s_j^t$. Since we are free to rescale latent space and the energy scale, we set the spring constant and resting length to $1$. The spring corresponding to an edge $i \rightarrow j$ at time $t$ then has energy
\be\label{eqn:staticH}
H_{ij}(s_i^t,s_j^t)=\f{1}{2} \bup{s_i^t-s_j^t-1}^{2} \, .
\ee
If there were no other effects, the total energy of the system at time $t$ would then be 
\be\label{eqn:totalstaticH}
H^t(\v{s}^t) = \sum_{i,j=1}^{N} A_{ij}^t \,H_{ij}(s_i^t,s_j^t) \, .
\ee
If we determined $\v{s}^t$ by minimizing $H^t$ for each $t$ separately, we would simply be applying the static SpringRank model separately to each ``snapshot'' of the network. This would ignore all previous (and future) interactions, and ignore the hypothesis that ranks change smoothly from one time-step to the next.

% Figure environment removed

To model this smoothness, we also assume a dependence between ranks at successive time-steps. Specifically, we extend the Hamiltonian~\eqref{eqn:totalstaticH} with an extra term that models the \emph{self-interaction} between past and current ranks,
\begin{equation}\label{eqn:selfH}
\Hself^t(\v{s}^t,\v{s}^{t-1}) 
= \frac{\kself}{2} \sum_{i=1}^N (s_i^t-s_i^{t-1})^2 \, .
\end{equation}
This can be seen as a set of additional ``self-springs'' that connect the rank of each individual with its own previous rank. The spring constant $\kself$ parametrizes how smoothly we want the ranks to change from one step to the next. In inference terms, $\kself$ is a hyperparameter which we tune using cross-validation.

Summing over all time-steps $0 < t \le T$ and adding this to the pairwise interactions at each time-step then gives a total energy

\begin{align}\label{eqn:fullH}
\Htotal(\{\v{s}^t\}) = \sum_{t=0}^T H^t(\v{s}^t) + \sum_{t=1}^T \Hself^t(\v{s}^t,\v{s}^{t-1}) \, .
\end{align}
We call this the dynamical SpringRank Hamiltonian. The optimal ranks $\v{s}^0,\v{s}^1,\ldots,\v{s}^T$ are those that minimize it.


There are two ways to minimize $\Htotal$. One is to proceed in an online way, moving forward in time. In this approach, we use the static SpringRank model Eq.~\eqref{eqn:totalstaticH} to find the initial ranks $\v{s}^0$ by minimizing $H^0(\v{s}^0)$. As in Ref.~\cite{de2018physical}, the energy is unchanged if we add a constant to all the ranks; we can break this translational symmetry by setting the mean initial rank $(1/N) \sum_{i=1}^N v_i^0$ to zero.
Then, at each subsequent time-step $t \ge 1$, we update the ranks by taking into account both the new pairwise interactions and the self-springs connecting the ranks with their previous values. Namely, given $\v{s}^{t-1}$ and $A^t$, we find the ranks $\v{s}^t$ that minimize $H^t(\v{s}^t) + \Hself^t(\v{s}^t,\v{s}^{t-1})$.

Since this is a convex function of $\v{s}^t$, we can find its minimum by setting its gradient to zero, or equivalently by balancing all the forces $v_i^t$. This yields a system of linear equations:
\begin{align}\label{eqn:fullsolution}
\rup{ D^{out,t}+D^{in,t}- \bup{A^t + (A^t)^\dagger}+\kself\id} \,\v{s}^t
&=\rup{D^{out,t}-D^{in,t}}\v{1} \nonumber \\& +\kself\, \v{s}^{t-1} \, . 
\end{align}

Here 
$D^{out,t}$ and $D^{in,t}$ are diagonal matrices whose entries are the weighted out- and in-degrees $D^{out,t}_{ii}=\sum_{j}A^t_{ij}$ and $D^{in,t}_{ii}=\sum_{j}A^t_{ji}$; 
$\dagger$ denotes the transpose; 
$\id$ is the identity matrix; 
and $\v{1}$ is the all-ones vector.

The matrix on the left side of~\Cref{eqn:fullsolution} is invertible if $\kself > 0$. In particular, its eigenvector $\v{1}$ has eigenvalue $N \kself$. Thus for each $A^t$ and each $\v{s}^{t-1}$, Eq.~\eqref{eqn:fullsolution} has a unique solution $\v{s}^t$. Overall, Eq.~\eqref{eqn:fullsolution} is similar to the regularized version of SpringRank~\cite{de2018physical} with regularization parameter $\alpha= \kself$. However, unlike the static model, there is a term on the right-hand side containing the previous ranks $\v{s}^{t-1}$, creating a Markovian dependence between successive time-steps. We refer to this model as \dsrfull\ (\dsr).

Importantly the online DSR approach does not actually minimize $\Htotal$, instead solving a sequence of minimization problems, one for each time step. To minimize $\Htotal$ instead, we set $\nabla \Htotal(\v{s}^t) = 0$, solving for the minimizers $\v{s}^t$ over all $N(T+1)$ ranks simultaneously, yielding the following system of equations (SI \Cref{sec:h_total_derive}):

\begin{align}\label{eqn:h_total}
\rup{ D^{out,t}+D^{in,t} - \bup{A^t+(A^t)^\dagger} + 2\kself\id}\,\v{s}^t 
&=\rup{D^{out,t}-D^{in,t}}\v{1} \nonumber\\ 
& +\kself \,\bup{\v{s}^{t-1} + \v{s}^{t+1}} \, . 
\end{align}
This differs from \Cref{eqn:fullH} in that the right-hand side now includes both past and future ranks (which doubles the contribution of $\kself$ on the left). We remove the terms $\v{s}^{t-1}$ and $\v{s}^{t+1}$ for $t=0$ and $t=T$ respectively. This entire system has translational symmetry, since the energy Eq.~\eqref{eqn:fullH} remains the same if we add the same constant to all ranks at all times, but we can again break this symmetry by setting the mean rank to zero.

Additionally, in contrast to \Cref{eqn:fullsolution}, the ranks at $t$ now depend on both $t-1$ and $t+1$, which themselves depend on ranks at adjacent time-steps, so that ranks are affected by interactions in both the past and the future. In computer science, methods like this where the entire history is provided to the algorithm are called \emph{offline}, to distinguish them from \emph{online} approaches that update their results in real time as data becomes available. Thus we refer to this model as \nmdsrfull\ (\nmdsr).  

The cost of solving \Cref{eqn:fullsolution} for a single time-step is the same as static SpringRank with only one additional parameter to be tuned using cross-validation, and there are $T$ such $N$-dimensional equations to be solved successively. On the other hand, \Cref{eqn:h_total} requires solving a single  system of dimension $NT$, whose operator consists of $T$ blocks, each of dimension $N\times N$. While these two approaches feature numbers of non-zero entries that are fundamentally determined by the number of total edges across all time steps, the cost of solving \dsr vs \nmdsr will depend on the particular choice of linear solver~\cite{peng2021solving}.

Philosophically, Eqns.~\eqref{eqn:fullsolution} and~\eqref{eqn:h_total} are trying to do two different things. If we are given all the data $A^0,A^1,\ldots,A^T$ and we want to infer retrospectively how each individual's rank changed over time, it makes sense to include both past and future interactions as in~\eqref{eqn:h_total} so that $s_i^t$ is affected by $i$'s entire history. 

In contrast, \eqref{eqn:fullsolution} can be viewed as modeling each individual's perceived rank at the time, based only on the interactions that have occurred so far.

In principle, one could envisage other ways to formally incorporate an explicit dependence on  $\v{s}^{t-1}$ into the model, and we provide one example in SI \Cref{sec:sidynl}. However, we found that the approaches presented in this Section provide a natural interpretation, result in good prediction performance on both real and synthetic datasets (see \Cref{sec:results}) and are computationally scalable. 

We close this section with two possible extensions to these models. First, in some settings we might have timestamps $t$ that are not successive integers $0,1,\ldots,T$. In this case, if the time interval between two successive times is $\Delta t$, one could scale the spring constant of the self-springs between time-steps as $\kself/\Delta t$. This corresponds to the fact that if we have $\Delta$ identical springs in series, each of which is stretched by $(s^t-s^{t-1})/\Delta$, their total energy is $(1/2)(\kself/\Delta)(s^t-s^{t-1})^2$. The same expression applies if the timestamps are real-valued so that $\Delta$ is not an integer.

Second, if we believe that not just the ranks themselves but their rates of change behave smoothly over time, one could add a momentum term to the Hamiltonian which is quadratic in the discrete second derivative of the ranks. Since
\begin{gather*}
\left( (s^{t+1}-s^t) - (s^t-s^{t-1}) \right)^2
= \left( s^{t+1} - 2 s^t + s^{t-1} \right)^2 \\
= 2 (s^t-s^{t-1})^2 + 2 (s^{t+1}-s^t)^2 - (s^{t+1} - s^{t-1})^2 \, ,
\end{gather*}
this is equivalent to adding a repulsive force, i.e., a spring with negative spring constant, between ranks two time-steps apart. Note that the system nevertheless remains convex: this momentum term is positive semidefinite, so adding it to~\eqref{eqn:fullH} keeps the coupling matrix positive definite except for translational symmetry. Of course, these terms are second-order in time. In the online approach, one would have to determine $\v{s}^0$ from the static model, $\v{s}^1$ from the first-order model~\eqref{eqn:fullsolution}, and then use the model including this momentum term for $\v{s}^t$ for $t \ge 2$. We have not pursued this here, but it may make sense for certain datasets.


\subsection{Moving-window SpringRank}\label{subsec:mwsr}

Before we test the various versions of \dsrfull\ defined above, we consider a simpler model as a baseline. 
The simplest way to extend SpringRank to a dynamical context is to apply the static model to the interactions in a series of ``windows,'' where in each window we sum the interactions over a series of consecutive time-steps. For instance, we can compute $\v{s}^t$ for each $t$ by applying the static model to a window of width $\tau$, i.e., replacing $A^t$ with $\sum_{t'=t}^{t+\tau-1} A^{t'}$. Since these windows overlap, the resulting estimates $\v{s}^t$ will be smooth to some extent, even without imposing an explicit dependence between $\v{s}^t$ and $\v{s}^{t-1}$. We use this method, which we call \mwsrfull\ (\mwsr), as a baseline to compare with the dynamical models presented above.

Roughly speaking, a larger $\tau$ is like a larger self-spring constant $\kself$, since it induces more overlap between windows and thus a stronger correlation between the inferred ranks. However, like a decaying-history approach, \mwsr\ assumes a particular kernel for the importance of past time-steps: namely, that all $t'$ in the window are equally important. In contrast, \dsrfull\ infers the importance of past time-steps by coupling $\v{s}^t$ with $\v{s}^{t-1}$.

However, both models have a free parameter that needs to be tuned, i.e., $\kself$ and $\tau$. A shorter window $\tau$ or smaller spring constant $\kself$ allows the ranks to respond quickly to new interactions, while a longer window or larger spring constant more tightly couples nearby estimates. This trade-off suggests the existence of an optimal window length $\tau_{\opt}$. We tune $\tau$ using a cross-validation procedure as explained in SI \Cref{sisec:tuning}.


\subsection{Generative Model and Synthetic Data}
\label{sec:genmod}

Analogous to a model presented in~\cite{de2018physical}, we propose a probabilistic generative model for dynamic data. It takes as input the ranks $\v{s}^t$ and generates a sequence of weighted directed networks with adjacency matrix $A^t$ at time $t$. One can also imagine models that generate the ranks, for instance with a random walk with Gaussian steps whose log-probability is the self-spring Hamiltonian~\eqref{eqn:selfH}, but we treat $\v{s}^t$ as an input since we want the user of this model to have control over how the ground-truth ranks vary with time.  For instance, in our experiments below we generate synthetic data where the ranks vary sinusoidally.

The generative model has two real-valued parameters: a signal-to-noise ratio or inverse temperature $\beta$, and an overall density of edges $c$. Given the ranks $\v{s}^t$, it generates weighted, directed edges between each pair of nodes $i,j$ independently, as follows. The probability $P_{ij}^t(\beta)$ of $i$ ``beating'' $j$ at time $t$, giving a directed edge $i \to j$, is a logistic function as in~\cite{de2018physical} or the Bradley-Terry-Luce model~\cite{bradley1952,luce1959}:
\bea
\nonumber P_{ij}^t(\beta)=\frac{1}{1+\e^{-2\beta(s_i^t-s_j^t)}} \, .
\eea
The number of such edges, which gives the integer weight $A_{ij}^t$, is then drawn from a Poisson distribution whose mean $\lambda_{ij}^t$ is $cP^t_{ij}\,(\beta)$: 
\be
\label{generative_poiss}
A^t_{ij} \sim \Poi\left(\lambda_{ij}^t=\frac{c}{1+\e^{-2\beta(s_i^t-s_j^t)}}\right).
\ee
Since $P_{ij}^t(\beta) + P_{ji}^t(\beta)=1$, for any pair $i,j$ the total number of interactions $A_{ij}^t + A_{ji}^t$ is Poisson-distributed with mean $c$. The rank differences $s_i^t-s_j^t$ are used only to choose the directions of these edges. This  is equivalent to a model where we define a random multigraph where the number of edges between $i$ and $j$ is $\Poi(c)$, and then we choose the direction of each edge independently according to $P_{ij}^t$.

This is different from the generative model proposed in the static case in~\cite{de2018physical}. In that model the probability that $i$ and $j$ interact depends on $s_i-s_j$ so that nodes are more likely to interact if their ranks are fairly close. This is consistent with SpringRank's assumption that if $i$ beats $j$ then $j$ is below $i$, but not too far below it (since the springs have resting length $1$). This assumption makes sense for some datasets but not for others. By generating synthetic data without this dependence, our intent is to pose a greater challenge to SpringRank by modeling (for example) round-robin tournaments where every team plays each other.

\subsection{Model Evaluation}
\label{sec:testing}

Assessing a ranking model on real datasets is not straightforward since we do not know the true values of the underlying ranks. Nevertheless, we may measure the extent to which inferred ranks are accurate in the sense that they can predict the outcome of new observations. 

There are several performance metrics that can be used for prediction evaluation. From coarse-grained measures capable of predicting the likely winner to more fine-grained measures that also estimate odds, we consider four main metrics in our experiments, detailed in \Cref{sisec:evaluation}. We measure prediction performance using a cross-validation protocol where datasets are divided into training and test sets. The training set is used for hyperparameter tuning and parameter estimation while performance is evaluated on the test set. In order to preserve the chronological ordering of the data, the test set contains future observations, i.e., observations that chronologically follow those used in training. Hyperparameters for each method are tuned using grid-search in order to maximize the performance metrics as described in SI \Cref{sisec:tuning}.





%%% Local Variables:
%%% mode: latex
%%% TeX-master: "main"
%%% End:


\section{State preparation in \texorpdfstring{$\LAQCC$}{LAQCC}}\label{sec:state_prep_in_LAQCC}
In this section we consider what quantum states we can prepare using an $\LAQCC$ circuit beyond the stabilizer states and Clifford circuits discussed in the previous section. 
Specifically, as mentioned in the introduction, we consider quantum states that are widely used, in other quantum algorithms, for bench marking purposes and within physics. 
First, we show how to create a uniform superposition of computational basis states up to size $q$, where $q$ is not a power of $2$, a state that is often used as initial state in other algorithms (including the following other state preparation protocols presented in this work). 
We then use this procedure to create $W$-states, the uniform superposition over all $n$-bitstrings of Hamming-weight $1$, using a compress-uncompress method. 
This compress-uncompress method generalizes to preparing Dicke-$(n,k)$ states for $k=\mathO(\sqrt{n})$, uniform superpositions over all $n$-bitstrings of Hamming-weight $k=\mathO(\sqrt{n})$. 
Dicke states find many applications, and especially the compress-uncompress approach might prove useful for entanglement distillation protocols.
Preparing general Dicke-$(n,k)$ states requires a novel method to map between two integer representation systems, the factoradic representation and the combinatorial number representation. 
Finally, we present a state preparation protocol for quantum many-body scar states, states often used in physics, based on the Dicke-$(n,k)$ state preparation protocol for $k=\mathO(\sqrt{n})$

\subsection{Uniform superposition of size \texorpdfstring{$q$}{q}\label{sec:superposition_modulo_q}}
The uniform superposition is often used as initial state in other quantum algorithms. 
A simple Hadamard gate applied to $n$ qubits prepares the uniform superposition $\frac{1}{\sqrt{2^n}}\sum_{i=0}^{2^n-1}\ket{i}$.
Preparing the state $\frac{1}{\sqrt{q}}\sum_{i=0}^{q-1}\ket{i}$, the superposition up to size $q$, is already harder for arbitrary $q$. 

A simple probabilistic approach works as follows: 
1) create a superposition $\frac{1}{\sqrt{2^n}}\sum_{i=0}^{2^n-1}\ket{i}$  with $n = \ceil{\log_2(q)}$ qubits;
2) mark the states $i< q$ using an ancilla qubit;
3) measure this ancilla qubit.
Based on the measurement result, the desired superposition is found, which happens with probability at least one half. 

The next theorem modifies this probabilistic approach to a protocol that deterministically prepares the uniform superposition modulo $q$ in $\LAQCC$.
\begin{theorem}
\label{thm:uniform_superposition_mod_q}
There is a deterministic $\LAQCC$ circuit that prepares the uniform superposition of size $q$. This circuit requires $\mathO(\ceil{\log_2(q)}^2)$ qubits.
\end{theorem}
\begin{proof}
Let $n = \ceil{\log_2(q)}$ and define $\mathcal{G}=\{i\mid 0\le i<q\}$ and $\mathcal{B}=\{i\mid q\le i\le2^n-1\}$. 
Construct the unitary
$$U_q:\ket{y}\ket{b}\mapsto\begin{cases}
\ket{y}\ket{b\oplus 1} & \text{if } y<q \\
\ket{y}\ket{b} & \text{if } y\ge q
\end{cases}.$$
The Greaterthan-gate of Table~\ref{tab:Add_Equality_Greaterthan} implements the operator $U_q$, note that this gate requires $\mathO(n^2)$ qubits. 

As $|\mathcal{G}|/2^n \ge 1/2$ and known, applying Lemma~\ref{lem:grover_constant_fraction} with the sets $\mathcal{G}$ and $\mathcal{B}$ and the constant-depth implementation of $U_q$, gives a $\LAQCC$ algorithm that boosts the amplitude of $\ket{\mathcal{G}}$ to~$1$.
\end{proof}

\begin{remark}
Note that in Lemma~\ref{lem:grover_constant_fraction} it was implicitly assumed that $|\mathcal G| + |\mathcal B|$ is a power of two (allowing for a simple reflection over the uniform superposition state). This $\LAQCC$ implementation of creating a uniform superposition modulo any $q$ removes this requirement.
\end{remark}
\subsection{$W$-state in $\LAQCC$}
\label{sec:W_state_in_LAQCC}
In this section we consider the $W_n$-state and how to prepare this state in $\LAQCC$. 
The $W_n$-state is a uniform superposition over all $n$-qubit states with a single qubit in the $\ket{1}$-state and all others in the $\ket{0}$-state:
$$\ket{W_n}  = \frac{1}{\sqrt{n}}\sum_i \ket{e_i},$$
where $\ket{e_i}$ is the state with a one on the $i$-th position and zeroes elsewhere. 

A first observation is that the $W$-state can be seen as a one-hot encoding of a uniform superposition over $n$ elements. 
We can label the $n$ states with non-zero amplitude of the $W$-state with an index. More precisly, we want to find circuits that implement the following map:
\begin{align}
\label{eqn:i_to_ei}
    \ket{i}\ket{0}\mapsto \ket{0}\ket{e_i},
\end{align}
with $i$ an index and $e_i$ the one-hot encoding of $i$.
This index -- which equals the position of the $1$ -- compresses the representation from $n$ to $\log(n)$ bits. 
This compression naturally defines two operations: 
\begin{align}
\text{\textbf{Uncompress}: }& \ket{i}_{\log(n)}\ket{0}_{n} \mapsto \ket{i}_{\log(n)}\ket{e_i}_{n}, \\
\text{\textbf{Compress}: }& \ket{i}_{\log(n)}\ket{e_i}_{n} \mapsto \ket{0}_{\log(n)}\ket{e_i}_{n}.
\end{align}
Implementing both and combining them implements Mapping~\ref{eqn:i_to_ei} giving an efficient $W$-state preparation protocol. 

The \textbf{Compress} and \textbf{Uncompress} operations map between a one-hot and binary represenation of an integer $i$. We call the registers containing the binary representation index registers, and the register containing the one-hot representation the system register. The index registers serve as ancilla qubits and the $W$-state is prepared in the system register. 

\begin{lemma}
\label{lem:uncompress}
There exists an $\LAQCC$ circuit for any $n$ implementing \textbf{Uncompress}, more specifically implementing the map: $\frac{1}{\sqrt{n}}\sum_{i = 0}^{n-1}\ket{i}_{\log(n)}\ket{0}_n \mapsto \frac{1}{\sqrt{n}}\sum_{i = 0}^{n-1}\ket{i}\ket{e_i}_n$. This circuit uses $\mathO(n \log(n))$ qubits placed in a grid pattern of size $n\times (\log(n))$. 
\end{lemma}
\begin{proof}
One column of the grid of length $n$ consists of system qubits placed in a line.
Adjacent to this line are $\log(n)$ index qubits. 
The left grid in Figure~\ref{fig:W_state_uncompress} shows the initial layout. 
The same figure also shows the steps to prepare the $W$-state in the system qubits. 
\begin{align*}
\frac{1}{\sqrt{n}}\sum_{i = 0}^{n-1}\ket{i}_{\log(n)}\ket{0}_{\log(n)}^{\otimes n-1}\ket{0}_n & \xrightarrow{(1)} \frac{1}{\sqrt{n}}\sum_{i = 0}^{n-1}\ket{i}_{\log(n)}^{\otimes n}\ket{0}_n \\
    								   & \xrightarrow{(2)} \frac{1}{\sqrt{n}}\sum_{i = 0}^{n-1}\ket{i}_{\log(n)}^{\otimes n}\ket{e_i}_n \\
     								   & \xrightarrow{(3)} \frac{1}{\sqrt{n}}\sum_{i = 0}^{n-1}\ket{i}\ket{0}^{\otimes n-1}\ket{e_i}_n 
\end{align*}
Step (1) uses fanout-gates to create a fully entangled state between the different index registers.
Step (2) applies $\text{Equal}_i$ gates in parallel from each index register to its corresponding system qubit to create the state $\ket{e_i}$ in the system register. 
Step (3) uses fanout-gates to disentangle and reset the index registers. 
Combined the \textbf{Uncompress} operations maps  $\frac{1}{\sqrt{n}}\sum_{i = 0}^{n-1}\ket{i}_{\log(n)}\ket{0}_{\log(n)}^{\otimes n-1}\ket{0}_n \mapsto \frac{1}{\sqrt{n}}\sum_{i = 0}^{n-1}\ket{i}\ket{0}^{\otimes n-1}\ket{e_i}_n$ as required.
\end{proof}
% Figure environment removed

\begin{lemma}
\label{lem:compress}
There exists an $\LAQCC$ circuit for any $n$ implementing \textbf{Compress}, more specifically implementing the map: $\frac{1}{\sqrt{n}}\sum_{i = 0}^{n-1}\ket{i}_{\log(n)}\ket{e_i}_n \mapsto \frac{1}{\sqrt{n}}\sum_{i = 0}^{n-1}\ket{0}\ket{e_i}_n$. This circuit uses $\mathO(n \log(n))$ qubits placed in a grid pattern of size $n\times (\log(n))$. 
\end{lemma}
\begin{proof}
To implement \textbf{Compress}, the index registers are uncomputed using parallel $CNOT$-operations, controlled by the system register. 
These controlled gates commute for different indices in the system register and hence by Lemma~\ref{lem:unitar_parallelization} a parallel circuit for the uncomputation exists. The \textbf{Compress} operation, also shown in Figure~\ref{fig:W_state_compress}, consists of the operations:
\begin{align*}
\frac{1}{\sqrt{n}}\sum_{i = 0}^{n}\ket{i}_{\log(n)}\ket{0}_{\log(n)}^{\otimes n - 1}\ket{e_i}_n & \xrightarrow{(1)} \frac{1}{n}\sum_{i, j = 0}^{n}(-1)^{i \cdot j}\ket{j}_{\log(n)}\ket{0}_{\log(n)}^{\otimes n - 1}\ket{e_i}_n \\
    & \xrightarrow{(2)} \frac{1}{n}\sum_{i, j = 0}^{n}(-1)^{i \cdot j}\ket{j}_{\log(n)}^{\otimes n}\ket{e_i}_n \\
    & \xrightarrow{(3)} \frac{1}{n}\sum_{i, j = 0}^{n}\ket{j}_{\log(n)}^{\otimes n}\ket{e_i}_n \\
    & \xrightarrow{(4)} \frac{1}{n}\sum_{i, j = 0}^{n}\ket{j}_{\log(n)}\ket{0}_{\log(n)}^{\otimes n-1}\ket{e_i}_n \\
    & \xrightarrow{(5)} \frac{1}{\sqrt{n}}\sum_{i =0 }^{n}\ket{0}_{\log(n)}^{\otimes n}\ket{e_i}_n 
\end{align*}
Step (1) applies Hadamard gates to the first index register, changing from the computational to the Hadamard basis, in which the $NOT$-operation is diagonal;
Step (2) uses fanout-gates to create a fully entangled state in the index registers;
Step (3) applies controlled-$Z$ gates, controlled by the system qubit $i$ and with targets the qubits in the $i$-th index register corresponding to the ones in the binary representation of $i$;
Step (4) disentangles the index registers using fanout-gates; 
and, Step (5) applies Hadamard gates to clean the index register.

The controlled-$Z$ gates in Step (3) apply phases that precisely cancel the phases already present, which disentangles the index registers from the system register. 
\end{proof}
% Figure environment removed

\begin{theorem}
\label{thm:W_state}
There exists a circuit in $\LAQCC$ that prepares the $\ket{W_n}$ state. This circuit requires $\mathO(n\log(n)$ qubits placed in a grid of size size $n\times (\log(n))$.
\end{theorem}

\begin{proof}
The circuit combines the circuits of Theorem~\ref{thm:uniform_superposition_mod_q}, Lemma~\ref{lem:uncompress} and Lemma~\ref{lem:compress}.
It consists of three steps:
\begin{align*}
\ket{0}^{\otimes n}_{\log(n)}\ket{0}_n &\xrightarrow[]{(1)} \frac{1}{\sqrt{n}}\sum_{i = 0}^{n-1}\ket{i}\ket{0}^{\otimes n-1}\ket{0}\\
        &\xrightarrow[]{(2)} \frac{1}{\sqrt{n}}\sum_{i = 0}^{n-1}\ket{i}\ket{0}^{\otimes n-1}\ket{e_i} \\
        &\xrightarrow[]{(3)}\frac{1}{\sqrt{n}}\sum_{i =0 }^{n}\ket{0}^{\otimes n}\ket{e_i} 
\end{align*}
Step one prepares the uniform superposition over indices, this can be done either by applying a layer of Hadamard gates, if $n$ is a power of $2$, requiring $\mathO(log(n))$ qubits,  or using Theorem~\ref{thm:uniform_superposition_mod_q} if $n$ is not a power of $2$ requiring $\mathO(log(n)^2)$ qubits.; Step (2) is by Lemma~\ref{lem:uncompress} and requires $\mathO(n\log(n))$ qubits; and, Step(3) is by Lemma~\ref{lem:compress} and requires $\mathO(n\log(n))$ qubits. 
\end{proof}




\subsection{Dicke states for small $k$}
\label{sec:dicke:small_k}
In this section we generalize our method of preparing the $\ket{W}$-state in Theorem~\ref{thm:W_state} to a more general set of states, Dicke states. 
A Dicke-$(n,k)$ state is the uniform superposition over bitstrings of Hamming weight $k$ and length $n$ (which we again assume to be a power of $2$ for simplicity): 
\begin{align}
    \ket{D_k^n} = \binom{n}{k}^{-1/2}\sum_{x \in \{0,1\}^n: |x| = k} \ket{x}.
\end{align}
For $k=1$, this state is precisely the $W$-state. 
There exists an efficient deterministic method to prepare a $\ket{D_k^n}$ state that requires a circuit of width $\mathO(n)$ and depth $\mathO(n)$, independent of $k$~\cite{bartschi2019deterministic}. 
This methods starts from the $\ket{1}^{\otimes k}\ket{0}^{\otimes k - n}$ state and relies on a recursive formula for the Dicke state
\begin{align*}
    \ket{D_k^n} = \alpha_{k,n} \ket{D_k^{n-1}}\otimes \ket{0} + \beta_{k,n} \ket{D_{k-1}^{n-1}}\otimes \ket{1}.
\end{align*}
This relation implies a protocol that is inherently sequential, which is unsuited for an $\LAQCC$ implementation. 

Instead, we present an $\LAQCC$ approach similar to the $W$-state preparation protocol. 
We apply the \textbf{Uncompress} operation of the $W$-state in parallel to put $k$ ones into the bitstring. 
This method allows for the preparation of Dicke states with $k=\mathO(\sqrt{n})$, using $\mathO(n^2 \log(n)^3)$ qubits. The bound on $k$ comes from the fact that using the \textbf{Uncompress} operation in parallel might cause overlaps to where the $1$'s are put into the system register. Having two ones in the same system qubit in effect negates the \textbf{Uncompress} operation. 
Following the lines of the birthday paradox, we find that overlaps between different indices happen not that often for $k = \mathO(\sqrt{n})$.
Lemma~\ref{lem:grover_constant_fraction} allows us to boost the amplitudes and make the protocol deterministic.

Again, consider two groups of qubits: Index registers with $\log(n)$ qubits each; 
and, system registers of $n$ qubits each. 
Contrary to the $W$-state, the Dicke state requires multiple system registers during the preparation. 
The state is prepared in only one system register. 
Denote the index registers with subscripts $i_1$ up to $i_k$ and the system registers with $s_1$ up to $s_n$. 
For clarity, these indices may be omitted if it is clear from the context. 

The algorithm consists of four steps:
\begin{enumerate}
    \item \textbf{Filling}: $\ket{0}_{i_1}\dots\ket{0}_{i_k}\ket{0}_{s_1} \rightarrow \frac{1}{n^{k/2}}\sum_{j_1, \dots, j_k = 0}^{n-1} \ket{j_1}_{i_1} \dots \ket{j_k}_{i_k}\ket{e_{j_1} \oplus \dots \oplus e_{j_k} }_{s_1}$
    \item \textbf{Filtering}: $\rightarrow \sqrt{\frac{(n-k)!}{n!}}\sum_{j_1 \neq \dots \neq j_k}^{n-1} \ket{j_1} \dots \ket{j_k}\ket{e_{j_1} \oplus \dots \oplus e_{j_k}}$
    \item \textbf{Ordering}: $\rightarrow \frac{1}{\sqrt{\binom{n}{k}}}\sum_{j_1 < \dots < j_k}^{n-1} \ket{j_1} \dots \ket{j_k}\ket{e_{j_1} \oplus \dots \oplus e_{j_k}}$
    \item \textbf{Cleaning}: $\rightarrow \frac{1}{\sqrt{\binom{n}{k}}}\sum_{j_1 < \dots < j_k}^{n-1} \ket{0} \dots \ket{0}\ket{e_{j_1} \oplus \dots \oplus e_{j_k}}$
\end{enumerate}
Note that after \textbf{Filling} there is a multiplicity in states. First, \textbf{Filtering} removes those states in which different indices $j_l$ are the same, resulting in an incorrect state in the $s_1$ register. 
Second, \textbf{Ordering} removes the multiplicity from having multiple permutations of the index registers creating the same state in the $s_1$ register, by forcing a choice of ordering on the indices.
These two steps give a unique pairing between index registers and system registers allowing the operation \textbf{Cleaning}.

We will now proof that these four steps are achievable in $\LAQCC$ and explicitly visualize the corresponding circuits for $n=4$ and $k=2$. 

\begin{lemma}
\label{lem:Dicke_filling}
An $\LAQCC$ circuits exists that implements \textbf{Filling}:
$$\ket{0}_{i_1}\dots\ket{0}_{i_k}\ket{0}_{s_1} \rightarrow \frac{1}{n^{k/2}}\sum_{j_1, \dots, j_k = 0}^{n-1} \ket{j_1}_{i_1} \dots \ket{j_k}_{i_k}\ket{e_{j_1} \oplus \dots \oplus e_{j_k} }_{s_1}.$$ 
This circuit uses $\mathO(k n\log(n))$ qubits.
\end{lemma}
\begin{proof}
To achieve a circuit implementing \textbf{Filling} we use \textbf{Uncompress} from Lemma~\ref{lem:uncompress} $k$ times in parallel.
Note that two \textbf{Uncompress} operations commute, hence by Lemma~\ref{lem:unitar_parallelization} $k$ \textbf{Uncompress} operations can be implemented in parallel.
Each of these parallel operations requires an index register, a system register and $\mathO(n\log(n))$ extra ancilla qubits.

The corresponding circuit consists of five steps: 
\begin{align*}
\ket{0}_{i_1} \dots \ket{0}_{i_k} \ket{0}_{s_1}\dots\ket{0}_{s_k} &\xrightarrow{(1)} \frac{1}{n^{k/2}}\sum_{j_1\dots j_k = 0}^{n-1}\ket{j_1}\dots \ket{j_k} \frac{1}{\sqrt{2^n}}\sum_{l=0}^{2^n - 1}\ket{l}_{s_1} \ket{0}_{s_2} \dots \ket{0}_{s_k}
\\
&\xrightarrow{(2)} \frac{1}{n^{k/2}}\sum_{j_1\dots j_k = 0}^{n-1}\ket{j_1}\dots \ket{j_k} \frac{1}{\sqrt{2^n}}\sum_{l=0}^{2^n - 1}\ket{l}_{s_1} \ket{l}_{s_2} \dots \ket{l}_{s_k}
\\
&\xrightarrow{(3)}\frac{1}{n^{k/2}}\sum_{j_1\dots j_k = 0}^{n-1}\ket{j_1}\dots \ket{j_k} \frac{1}{\sqrt{2^n}}\sum_{l=0}^{2^n - 1}(-1)^{(2^{j_1} + \dots + 2^{j_k}) \cdot l}\ket{l}_{s_1} \ket{l}_{s_2} \dots \ket{l}_{s_k}
\\
&\xrightarrow{(4)} \frac{1}{n^{k/2}}\sum_{j_1\dots j_k = 0}^{n-1}\ket{j_1}\dots \ket{j_k} \frac{1}{\sqrt{2^n}}\sum_{l=0}^{2^n - 1}(-1)^{(2^{j_1} + \dots + 2^{j_k}) \cdot l}\ket{l}_{s_1} \ket{0}_{s_2} \dots \ket{0}_{s_k}
\\ 
&\xrightarrow{(5)} \frac{1}{n^{k/2}}\sum_{j_1\dots j_k = 0}^{n-1}\ket{j_1}\dots \ket{j_k} \ket{e_{j_1} \oplus \dots \oplus e_{j_k}}_{s_1} \ket{0}_{s_2} \dots \ket{0}_{s_k}
\end{align*}
Step (1) brings all index registers in a uniform superposition of size $n$, use Theorem~\ref{thm:uniform_superposition_mod_q} if required, and one system register in a uniform superposition size $2^n$; 
Step (2) uses fan-out gates to create entangled copies of the system register; 
Step (3) applies a phase flip between every pair of index and system register using \textbf{Uncompress} of Lemma~\ref{lem:uncompress}, except instead of applying not gates to the system registers, apply phase gates;
Step (4) uses fan-out gates to disentangle and uncompute all but one of the system registers; 
Step (5) applies Hadamard gates on the system register to obtain the one-hot representation of the index registers. 
Step(3), the step that requires most qubits, requires $\mathO(n \log(n))$ qubits for every pair of index and system register, of which there are $k$, resulting in the requirement of $\mathO(kn \log(n))$ qubits. 
\end{proof}

Figure~\ref{fig:Dicke_2_state} shows these five steps graphically. 
Ancilla qubits are omitted for clarity. 
Note that some of the $j_i$ in the index registers may intersect. 
The next Filtering step takes care of that.
% Figure environment removed

\begin{lemma}
\label{lem:Dicke_filtering}
An $\LAQCC$ circuit exists that implements \textbf{Filtering}:
$$\frac{1}{n^{k/2}}\sum_{j_1, \dots, j_k = 0}^{n-1} \ket{j_1}_{i_1} \dots \ket{j_k}_{i_k}\ket{e_{j_1} \oplus \dots \oplus e_{j_k} }_{s_1} 
\rightarrow \sqrt{\frac{(n-k)!}{n!}}\sum_{j_1 \neq \dots \neq j_k}^{n-1} \ket{j_1} \dots \ket{j_k}\ket{e_{j_1} \oplus \dots \oplus e_{j_k}}.$$
\end{lemma}
\begin{proof}
First note that the state produced by the \textbf{Filling} step,
$$\frac{1}{n^{k/2}}\sum_{j_1\dots j_k = 0}^{n-1}\ket{j_1}\dots \ket{j_k} \ket{e_{j_1} \oplus \dots \oplus e_{j_k}}_{s_1} ,$$
contains states in which some of the indices $j_l$ overlap. Let $\ket{\psi} = \sum_{j_1 \neq \dots \neq j_k } \ket{j_1}\dots \ket{j_k} \ket{e_{j_1} \oplus \dots \oplus e_{j_k}}$, be the state in which none of the indices overlap, the desired output state. Then we can write
$$\frac{1}{n^{k/2}}\sum_{j_1\dots j_k = 0}^{n-1}\ket{j_1}\dots \ket{j_k} \ket{e_{j_1} \oplus \dots \oplus e_{j_k}}_{s_1} = \alpha \ket{\psi} + \beta \ket{\psi^{\perp}},$$
with $\ket{\psi^\perp}$ containing the states in which at least two of the indices $j_l$ ovelap. Note that $\braket{\psi}{\psi^\bot} = 0$, so $\alpha$ can be exactly calculated by counting the number of quantum states with distinct $j_i$'s, which gives $|\alpha|^2 = \frac{n!}{(n-k)! n^k}$.
Lemma~\ref{lem:birthday_paradox} gives a lower bound on $|\alpha|^2$: 
$$|\alpha|^2 = \frac{n!}{(n-k)! n^k} > e^{\frac{-2k^2}{n}},$$
which is at least constant for $k = \mathO(\sqrt{n})$. 

The state $\ket{\psi^{\bot}}$ is a superposition of states in which the system register state has Hamming weight less than $k$, because at least two of the $j_i$'s are the same causing a cancellation in the system register. We can use this to create a unitary $U_{flag}$ that flags $\ket{\psi^{\bot}}$. We implement this in two steps:
\begin{align*}
&\frac{1}{n^{k/2}}\sum_{j_1,\dots, j_k = 0}^{n-1}\ket{j_1}\dots \ket{j_k} \ket{e_{j_1} \oplus \dots \oplus e_{j_k}}_{s_1}\ket{0}_{\log(k)}\ket{0} \\
&\xrightarrow[]{(1)} \frac{1}{n^{k/2}}\sum_{j_1,\dots, j_k = 0}^{n-1}\ket{j_1}\dots \ket{j_k} \ket{e_{j_1} \oplus \dots \oplus e_{j_k}}_{s_1}\ket{|e_{j_1} \oplus \dots \oplus e_{j_k}|}\ket{0}\\
&\xrightarrow[]{(2)} \frac{1}{n^{k/2}}\sum_{j_1,\dots, j_k = 0}^{n-1}\ket{j_1}\dots \ket{j_k} \ket{e_{j_1} \oplus \dots \oplus e_{j_k}}_{s_1}\ket{0}\ket{\mathbbm{1}_{|e_{j_1} \oplus \dots \oplus e_{j_k}|=k}} \\
& = \alpha\ket{\psi}\ket{1} + \beta \ket{\psi^{\bot}}\ket{0}
\end{align*}
Where $|x|$ denotes the Hamming weight of bitstring $x$.
Step (1) follows from a Hamming-weight gate (see Table~\ref{tab:QFT_Hammingweight_Threshold}), which requires $\mathO(n \log(n))$ qubits; Step (2) follows from applying an Exact$_k$ gate, requiring $\mathO(\log(n)^2)$ qubits. This same step also uncomputes the Hamming-weight gate of the first step.

Lemma~\ref{lem:grover_constant_fraction} now allows us to amplify $\alpha$ to $1$ using the oracle $U_{flag}$. This produces the state
$$\sqrt{\frac{(n-k)!}{(n)!}}\sum_{j_1 \neq \dots \neq j_k} \ket{j_1} \dots \ket{j_k} \ket{e_{j_1} \oplus \dots \oplus e_{j_k}},$$
using $\mathO(k n\log(n))$ qubits.
\end{proof}

To uncompute the index registers, we have to know which one in the system register corresponds to which index register, as any permutation of the index registers results in the same state in the system register. 
The \textbf{Ordering} step imposes an ordering on the index registers, thereby removing the redundancy in the ordering. 

\begin{lemma}
\label{lem:dicke_ordering}
An $\LAQCC$ circuit exists that implements \textbf{Ordering}:
$$\sqrt{\frac{(n-k)!}{n!}}\sum_{j_1 \neq \dots \neq j_k}^{n-1} \ket{j_1} \dots \ket{j_k}\ket{e_{j_1} \oplus \dots \oplus e_{j_k}} \rightarrow  \frac{1}{\sqrt{\binom{n}{k}}}\sum_{j_1 < \dots < j_k}^{n-1} \ket{j_1} \dots \ket{j_k}\ket{e_{j_1} \oplus \dots \oplus e_{j_k}}.$$
This circuit uses $\mathO(k^2 \log(n)^2)$ qubits.
\end{lemma}
\begin{proof}
The first step of the $\LAQCC$ circuit that implements \textbf{Ordering} is to evaluate a Greaterthan-gate on all pairs of index registers, which requires $k$ copies of each index register. We require $k$ extra qubits per index register to store the outcome of the Greaterthan-gates. 
The copies of the index registers are created by doing a fan-out gate. Note that the distribution of the index registers should be set up in such a way that every possible pair can be compared by a Greaterthan-gate. 
\begin{align*}
    & \sqrt{\frac{(n-k)!}{n!}}\sum_{j_1 \neq \dots \neq j_k} \ket{j_1}^{\otimes k} \ket{0}^{\otimes k} \dots \ket{j_k}^{\otimes k} \ket{0}^{\otimes k} \ket{e_{j_1} \oplus \dots \oplus e_{j_k}} \xrightarrow[]{(1)} \\
    & \sqrt{\frac{(n-k)!}{n!}} \sum_{j_1 \neq \dots \neq j_k} 
    \big[\ket{j_1}^{\otimes k}  \ket{\mathbbm{1}_{j_1 > j_2}} \dots \ket{\mathbbm{1}_{j_1 > j_k}}\big]
    \dots \big[ \ket{j_k}^{\otimes k}  \ket{\mathbbm{1}_{j_k > j_1}} \dots \ket{\mathbbm{1}_{j_k > j_{k-1}}}\big] 
    \ket{e_{j_1} \oplus \dots \oplus e_{j_k}}.
\end{align*}
Each $\mathbbm{1}_{j_k > j_{k'}}$ is an indicator variable that evaluates to one if and only if $j_k > j_{k'}$. This step requires $\mathO(k^2\log(n)^2)$ qubits.

Next, we compute and measure the Hamming weight of the ancilla qubits $\ket{\mathbbm{1}_{j_1 > j_2}} \dots \ket{\mathbbm{1}_{j_1 > j_k}}$, using the Hamming-weight gate. 
We measure the calculated Hamming weights. 
As all index registers were distinct before measuring, these measurements directly impose an ordering on the index registers. 

\begin{align*}
&\xrightarrow[]{(\mathrm{Hamming weight})}\sqrt{\frac{(n-k)!}{n!}}\sum_{j_1 \neq \dots \neq j_k} \big[\ket{j_1} \ket{\mathbbm{1}_{j_1 > j_2} + \dots + \mathbbm{1}_{j_1 > j_k}}\big] \\
& \qquad\qquad\qquad\qquad\quad \dots \big[ \ket{j_k} \ket{\mathbbm{1}_{j_k > j_1} + \dots + \mathbbm{1}_{j_k > j_{k-1}}}\big] \ket{e_{j_1} \oplus \dots \oplus e_{j_k}}\\
&\xrightarrow[]{(measure)} \binom{n}{k}^{-1/2} \sum_{j_1 < \dots < j_k} \big[\ket{j_1} \ket{0}\big] \dots \big[ \ket{j_k} \ket{k}\big] \ket{e_{j_1} \oplus \dots \oplus e_{j_k}}
\end{align*}
This step costs $\mathO(k^2 \log(k))$ qubits.
Assume without loss of generality that the measurement outcomes impose the ordering ${j_1 < \dots < j_k}$. 
Otherwise, a permutation of the index registers achieves the same ordering, using the Permutation gate from Table~\ref{tab:Fanout_Perm}. 

Uncomputing the Hamming weights and the Greaterthan-gates gives the state 
$$\binom{n}{k}^{-1/2} \sum_{j_1 < \dots < j_k} \big[\ket{j_1} \dots \ket{j_k} \big] \ket{e_{j_1} \oplus \dots \oplus e_{j_k}}.$$
\end{proof}

The \textbf{Cleaning} step cleans the index registers for the Dicke state in a similar fashion as in the \textbf{Compress} method in the $W$-state protocol. 
In the cleaning process, we have to take the added ordering of the index registers into account. 
Suppose the $l$-th qubit of the system register is a $1$.
If this is the first $1$ in the system register, it belongs to index register $j_1$, and if it is the $m$-th $1$ it belongs to index register $j_m$. 
Computing the Hamming weight of the first $l-1$ qubits gives precisely this information.
Combined, this shows that if the $l$-th qubit is a $1$ and the Hamming weight of the first $l-1$ qubits equals $m$, then the $l$-th qubit should uncompute the $m+1$-th index register $j_{m+1}$. 

\begin{lemma}
\label{lem:dicke_cleaning}
An $\LAQCC$ circuit exists that implements \textbf{Cleaning}:
$$\frac{1}{\sqrt{\binom{n}{k}}}\sum_{j_1 < \dots < j_k}^{n-1} \ket{j_1} \dots \ket{j_k}\ket{e_{j_1} \oplus \dots \oplus e_{j_k}} \rightarrow \frac{1}{\sqrt{\binom{n}{k}}}\sum_{j_1 < \dots < j_k}^{n-1} \ket{0} \dots \ket{0}\ket{e_{j_1} \oplus \dots \oplus e_{j_k}}.$$
This circuit uses $\mathO(n^2 \log(n))$ qubits.
\end{lemma}
\begin{proof}
The first step, as described above, is to acquire the Hamming weight from all the substrings of the system register. This requires $n$ copies of the system register as well as a $\log(k)$-qubit register to store the Hamming weight value. The copies follow from the fanout-gate.
\begin{align*}
    &\binom{n}{k}^{-1/2} \sum_{j_1 < \dots < j_k} \ket{j_1} \dots \ket{j_k} \ket{e_{j_1} \oplus \dots \oplus e_{j_k}}\ket{0}_n^{\otimes n - 1}\ket{0}_{\log(n)}^{\otimes n}\xrightarrow{(1)} \\ 
    &\binom{n}{k}^{-1/2} \sum_{j_1 < \dots < j_k} \ket{j_1} \dots \ket{j_k} \ket{e_{j_1} \oplus \dots \oplus e_{j_k}}^{\otimes n}\ket{0}_{\log(n)}^{\otimes n} \xrightarrow{(2)}\\
    &\binom{n}{k}^{-1/2} \sum_{j_1 < \dots < j_k} \ket{j_1} \dots \ket{j_k} \ket{e_{j_1} \oplus \dots \oplus e_{j_k}}^{\otimes n}\bigotimes_{l=0}^{n-1}\ket{|(e_{j_1} \oplus \dots \oplus e_{j_k})_{[l,n]}|},
\end{align*}
where $|(e_{j_1} \oplus \dots \oplus e_{j_k})_{[l,n]}|$ denotes the Hamming weight of the substring consisting of qubits $l$ up until $n$ of the system register.
Step (1) copies the system qubits using fan-out gates; 
Step (2) computes the Hamming weight of all the qubits $1$ up until $j-1$ using the Hammingweight-gate shown in Table~\ref{tab:QFT_Hammingweight_Threshold};
Step (3) cleans the copies of the system register by applying fan-out. 
This step is omitted from the equations, but is included in the graphical explanation of the circuit, shown in Figure~\ref{fig:Hammingweight} for $n=4$.
Note that at the end of the calculation, it is convenient to teleport the Hamming weight registers next to the system register. There are now $n$ new registers containing the information of the Hamming weight, we will refer to them as the Hamming weight registers.
This step requires $\mathO(n^2 \log(n))$ qubits. 
% Figure environment removed

The last step that remains is to clean the $k$ index registers. Cleaning the $k$ index registers follows similar steps as the \textbf{Compress} method in the $W$-state protocol, with the added Hamming-weight information taken into account. This step requires $k$ copies of the system registers well as $k$ copies of the Hamming-weight registers. Every index register is paired with one copy of the system register and a copy of the $n$ Hamming-weight registers.
Cleaning the $j$-th register consists of five steps, similar to the \textbf{Compress} method of the $W$-state:
Step (1) applies Hadamard gates to bring the index register to phase space, in which $CNOT$-gates are diagonalized;
Step (2) copies the index register;
Step (3) uses the information in the Hamming-weight and system register to apply the phases to the correct index register qubits;
Step (4) cleans the index register copies;
and, Step (5) applies Hadamard gates to reset the index register qubits to the $\ket{0}$ state

Figure~\ref{fig:compress_dicke} shows the steps taken to clean a single index register $j$. 
The black dots represent the qubits in the system register and the upper row of blue dots represent the qubits in index register $j$.  The pink squares represent the ancilla Hamming weight register, where each square represents a group of $\log(k)$ qubits. This step requires $\mathO(n k \log(k)\log(n))$ qubits. At the end of the \textbf{Cleaning} operation the state is as desired:
$$
 \frac{1}{\sqrt{\binom{n}{k}}}\sum_{j_1 < \dots < j_k}^{n-1} \ket{0} \dots \ket{0}\ket{e_{j_1} \oplus \dots \oplus e_{j_k}}.
$$
The \textbf{Cleaning} step requires $\mathO(n^2 \log(n))$ qubits.
\end{proof}

% Figure environment removed

\begin{theorem}
\label{thm:dicke_cnst_depth}
For any $n$ and $k = \mathO(\sqrt{n})$ there exists an $\LAQCC$ circuit preparing the Dicke-$(n,k)$ state, $\ket{D^n_k}$, using $\mathO(n^2\log(n))$ qubits.
\end{theorem}

\begin{proof}
The circuit combines the circuits resulting from Lemmas~\ref{lem:Dicke_filling}, \ref{lem:Dicke_filtering}, \ref{lem:dicke_ordering} and \ref{lem:dicke_cleaning}.
It consists of four steps:
\begin{align*}
\ket{0}_{i_1}\dots\ket{0}_{i_k}\ket{0}_{s_1} &\xrightarrow{(1)} \frac{1}{n^{k/2}}\sum_{j_1, \dots, j_k = 0}^{n-1} \ket{j_1}_{i_1} \dots \ket{j_k}_{i_k}\ket{e_{j_1} \oplus \dots \oplus e_{j_k} }_{s_1}\\
&\xrightarrow{(2)} \sqrt{\frac{(n-k)!}{n!}}\sum_{j_1 \neq \dots \neq j_k}^{n-1} \ket{j_1} \dots \ket{j_k}\ket{e_{j_1} \oplus \dots \oplus e_{j_k}}\\
&\xrightarrow{(3)} \frac{1}{\sqrt{\binom{n}{k}}}\sum_{j_1 < \dots < j_k}^{n-1} \ket{j_1} \dots \ket{j_k}\ket{e_{j_1} \oplus \dots \oplus e_{j_k}}\\
&\xrightarrow{(4)} \frac{1}{\sqrt{\binom{n}{k}}}\sum_{j_1 < \dots < j_k}^{n-1} \ket{0} \dots \ket{0}\ket{e_{j_1} \oplus \dots \oplus e_{j_k}}
\end{align*}
Step (1) implements \textbf{Filling} using Lemma~\ref{lem:Dicke_filling} requiring $\mathO(k n \log(n))$ qubits; 
Step (2) implements \textbf{Filtering} using Lemma~\ref{lem:Dicke_filtering} requiring $\mathO(k n \log(n))$ qubits;
Step (3) implements \textbf{Ordering} using Lemma~\ref{lem:dicke_ordering} requiring $\mathO(k^2\log(n)^2)$ qubits; 
Step (4) implements \textbf{Cleaning} using Lemma~\ref{lem:dicke_cleaning} requiring $\mathO(n^2\log(n))$ qubits. After every step ancilla qubits are cleaned, so that they can be reused. As $k=\mathO(\sqrt{n})$ the largest amount of qubits required for a step is Step (4) requiring $\mathO(n^2\log(n))$ qubits.
\end{proof}
\citeauthor{bartschi2022deterministic_short_depth} posed a conjecture on the optimal depth of quantum circuits that prepare the Dicke-$(n,k)$ state. They give an algorithm for generating Dicke-$(n,k)$ states in depth $\mathO(k \log(\frac{n}{k}))$, given all-to-all connectivity, and conjecture that this scaling is optimal when $k$ is constant. Our result shows that there is a $\LAQCC$ implementation in this regime, when one has access to intermediate measurements and feed forward. This does not disprove their conjecture. However the circuits shown here are also accessible in $\QNC^1$ by Lemma~\ref{lem:LAQCC_QNC1}, giving ``pure" quantum circuits with depth $\mathO(\log(n))$ for $k = \mathO(\sqrt{n})$ and achieving better scaling when $k = \omega(1)$. 

\subsection{Dicke states for all $k$ using log-depth quantum circuits}
\label{sec:Dicke_in_LAQCC_LOG}
The previous section gave a constant-depth protocol to prepare the Dicke-$(n,k)$ state for $k=\mathO(\sqrt{n})$. 
We developed a different method for creating Dicke-$(n,k)$ states which requires logarithmic (in $n$) quantum depth to prepare Dicke-$(n,k)$, but works for arbitrary $k$. We first define what we mean with logarithmic quantum depth:
\begin{notation}
We let $\LAQCC\text{-}\mathsf{LOG}$ refer to the instance $\LAQCC(\QNC^0,\NC^1, \mathO(\log(n)))$, similar to $\LAQCC$ except that we allow for a logarithmic number of alterations between quantum and classical calculations. This results in a circuit of logarithmic quantum depth.
\end{notation}
In this section we show a $\LAQCC\text{-}\mathsf{LOG}$ circuit that creates the Dicke-$(n,k)$ state.

One way of studying the creation of Dicke states is by looking at efficient algorithms that convert numbers from one representation to another. 
An example of this is the \textbf{Uncompress}-\textbf{Compress} method in the $W$-state protocol, that converts numbers from a binary representation to a one-hot representation. 
Dicke states are a generalization of the $W$-state, hence the one-hot representation no longer suffices for preparing the state. 
Instead, we use a construction based on number conversion between the combinatorial representation and the factoradic representation.
Below we introduce both representations and present quantum circuits that map between the two. Theorem~\ref{thm:Dicke:Log_depth} proves that a $\LAQCC\text{-}\mathsf{LOG}$ circuit can prepare the Dicke-$(n,k)$ state for any $k$. 

\subsubsection{Combinatorial number system}
An interesting result showed that any integer $m\ge 0$ can be written as a sum of $k$ binomial coefficients~\cite{Beckenbach:1964}. 
For fixed $k$, this is even unique as the next lemma shows.
\begin{lemma}[\cite{Beckenbach:1964}]
\label{lem:comb_numbers}
For all integers $m\geq 0$ and $k \geq 1$, there exists a unique decreasing sequence of integers $c_k, c_{k-1},\dots, c_1$ with $c_j > c_{j-1}$ and $c_1 \geq 0$ such that 
$$m = \binom{c_k}{k}  + \binom{c_{k-1}}{k-1} \dots \binom{c_1}{1} = \sum_{i=1}^k \binom{c_i}{i}.$$
\end{lemma}

This lemma allows for the definition of the combinatorial number representation:
\begin{definition}
Let $k \in \mathbb{N}$ be a constant. Any integer $m \in \mathbb{N}$ can be represent by a unique string of numbers $(c_k, c_{k-1} \dots, c_1)$, such that $c_k > c_{k-1} \dots >  c_1 \geq 0$ and $c_k \leq m$. 
This string is given by the unique decreasing sequence of Lemma~\ref{lem:comb_numbers}. 
We call this string the \emph{index representation} denoted by $m^{indx(k)}$.

The bit string of $k$ ones at indices $(c_k,\dots, c_1)$ is the $m$-th bit string with $k$ ones in the lexicographical order. 
This bit string is called the \emph{combinatatorial representation}.
We denote the $m$-th bit string with $k$ ones as $m^{comb(k)}$.
\end{definition}

The $W$-state protocol used the conversion between the binary representation of a number $m$ and its combinatorial representation $m^{comb(1)}$.
A generalized number conversion is precisely the protocol needed to prepare Dicke states. 

A sketch of the protocol would be as follows: 
given positive integers $k$ and $n$:
Create a superposition state 
$${\binom{n}{k}}^{-\frac{1}{2}}\sum_{i=0}^{\binom{n}{k} - 1}\ket{i}\ket{0};$$
Use number conversion to go from label $i$ to $i^{comb(k)}$
$${\binom{n}{k}}^{-\frac{1}{2}}\sum_{i=0}^{\binom{n}{k} - 1}\ket{i}\ket{i^{comb(k)}};$$
Use number conversion from $i^{comb(k)}$ to $i$ to clean up the label register
$${\binom{n}{k}}^{-\frac{1}{2}}\sum_{i=0}^{\binom{n}{k} - 1}\ket{0}\ket{i^{comb(k)}} = \ket{D^n_k}.$$


The conversion map from the combinatorial representation to the binary representation is given by Lemma~\ref{lem:comb_numbers}. 
This calculation requires iterative multiplication and addition, both of which are in $\TC^0$, hence this calculation is in $\TC^0$. 

The converse mapping, from binary to combinatorial representation for given $k$, can be achieved by a greedy iterative algorithm:
On input $m$, find the biggest $c_k$ such that $m \geq \binom{c_k}{k}$ and subtract this from $m$: $\tilde{m} = m - \binom{c_k}{k}$. 
This gives $c_k$ and a residual $\tilde{m}$. 
Repeat this process for $\tilde{m}$: 
Find the largest $c_{j}$ such that $\tilde{m}\geq \binom{c_j}{j}$ and update residual $\tilde{m} = \tilde{m} - \binom{c_j}{j}$, until all $c_j$ are found. 

This greedy algorithm is inherently linear in $k$ as it requires all previously found $\{c_i\}_{i=j}^k$ to find $c_{j-1}$. 
Hence, it is not immediately obvious if and how to achieve this mapping in constant or even logarithmic depth. 

\subsubsection{Mapping between factoradic representation and combinatorial number system}
A number representation closely related to the combinatorial number representation is the \textit{factoradic representation}. 
This number system uses factorials instead of binomials to represent numbers. 

\begin{definition} 
\label{def:factoradics}
A sequence $y = (y_{n-1}, y_{n-2}, \dots, y_0)$ of integers, such that $j \geq y_j \geq 0$ is called a \textit{factoradic}, or more explicitly an \textit{$n$-factoradic}. 
The elements of an $n$-factoradic is called an \textit{$n$-digit}.
An $n$-factoradic $y$ can represent a number $m$ between $0$ and $n!-1$, in the following way
\begin{align}
\label{eqn:fact_to_int}
m = \sum_j^{n} y_j \cdot j!.
\end{align}
For a given $m \in \{0, \ldots, n!-1\}$, we call the $n$-factoradic $y$ obeying the equality above, the \emph{factoradic representation of $m$}.  
Denote $\text{Fact}(n)$ as the set of all $n$-factoradics.
\end{definition}

The following lemma shows that Equation~\ref{eqn:fact_to_int} is a bijection, showing that the factoradic representation is unique.

\begin{lemma}
\label{factoradic_summation}
For $k \geq 0$ it holds that:
$$\sum_{i=0}^k i \cdot i!= (k + 1)! - 1.$$ 
\end{lemma}
\begin{proof}
Proof by induction.\\
\textbf{BASE STEP}: Let $k$ be $0$:
$$0\cdot 0! = 1! - 1$$
\textbf{INDUCTION STEP}: Assume the lemma holds for some $j$, then
$$\sum_{i=0}^{j+1} i \cdot i! = (j+1) \cdot (j+1)! + \sum_{i = 0}^{j} i \cdot i = (j+1) \cdot (j+1)! + (j+1)! - 1 = (j+2)! - 1,$$
which completes the proof.
\end{proof}
This identity allows for using factorials as a base for a number system.
The next lemma gives a log-space algorithm to convert a factoradic representation to its combinatorial representation. 
\begin{lemma}
\label{lem:fac_to_comb}
There is a logspace algorithm $\mathcal{A}$ that, given $k \in \{0, \ldots, n\}$, and a uniformly random $n$-factoradic, outputs a uniformly random $n$-bit string of Hamming weight $k$.
\end{lemma}
\begin{proof}
The algorithm $\mathcal{A}$ is given $k$ and an $n$-factoradic $y = (y_{n-1}, \dots, y_0)$. It will then output an $n$-bit string $y^{comb(k)} = y^{comb(k)}_{n-1} \dots y^{comb(k)}_0 \in \{0,1\}^n$ of Hamming weight $k$, one bit at a time, from left to right, according to the following rule. Let $H_{>n-j} = \sum_{i=n-j+1}^{n-1} y^{comb(k)}_i$ be the Hamming weight of the bits produced before we reach bit $n-j$. Then $y^{comb(k)}_{n-j}$ is given by:
\begin{align}
\label{eqn:fac_to_comb}
    (\mathcal{A} (y))_{n-j} = y^{comb(k)}_{n-j} = \begin{cases} 1 & \text{if } y_{n-j} < k - H_{>n-j} \\
    0 & \text{otherwise}
    \end{cases}.
\end{align}
This conversion requires comparing an $n$-digit with a constant and the Hamming weight of a bitstring.
The only information that $\mathcal A$ needs to remember, as it goes from bit $n-j+1$ to bit $n-j$ , is the Hamming weight $H_{>n-j}$ of the bits it produced so far, and this can be stored in logarithmic space. 

Now note that the number of factoradic $n$-digit strings that map to the same combinatorial bit string is always $k!(n-k)!$:
Let $y^{comb(k)} \in \{0,1\}^n$ have Hamming weight $k$. 
For any bit position $y^{comb(k)}_{n-j}$, there are $n - j + 1 - (k - H_{>n-j})$ possible choices for the $n$-digit $y_{n-j} \in \{0, \ldots, n-j\}$ that set $y^{comb(k)}_{n-j} = 0$. 
For the leftmost index $n-j$ such that $y^{comb(k)}_{n-j} = 0$, it holds that $H_{>n-j} = j-1$, and then there are $n-k$ possible $n$-digits $y_{n-j}$ that set $y^{comb(k)}_{n-j} = 0$. 
Then, for the second index $n-j$ such that $y^{comb(k)}_{n-j} = 0$ it holds that $H_{>n-j} = j - 2$, hence there are $n - k - 1$ possible $n$-digits $y_{n-j}$ causing $y^{comb(k)}_{n-j} = 0$. And so forth. 
This results in $(n-k)!$ different possible choices for the $(n-k)$-many $n$-digits where $y^{comb(k)}=0$. 

Similarly, for the leftmost position $n-j$ where $y^{comb(k)}_{n-j} = 1$, there are $k$ possible choices for the $n$-digit $y_{n-j}$ that cause $y^{comb(k)}_{n-j} = 1$. 
The second leftmost position $n-j$ gives $k-1$ possible choices, and so forth, for a total of $k!$ possible settings of the $k$-many $n$-digits where $y^{comb(k)}=0$.

Combined, we conclude that, for every $n$-bit string $y^{comb(k)} \in \{0,1\}^n$ of Hamming weight $k$, there are exactly (the same number of) $k!(n-k)!$ $n$-factoradics $y$ such that $\mathcal A(y) = y^{comb(k)}$.
Hence, a uniformly random $n$-factoradic is mapped by $\mathcal A$ to a uniformly random $n$-bit string of Hamming weight $k$, as claimed.
\end{proof}

This lemma gives a logspace algorithm to convert a uniformly random $n$-factoradic to a uniformly random $n$-bit string of Hamming weight $k$, for any $k$.
It is well known that logspace is contained in $\TC^1$, allowing this calculation to be performed in parallel log-depth when one has access to threshold gates~\cite{Johnson:1990}. As we saw in Section~\ref{sec:gates_created_in_LAQCC}, we can compute a threshold gate in $\LAQCC$. Hence, an $\LAQCC\text{-}\mathsf{LOG}$ can perform any $\TC^1$ calculation. We conclude:

\begin{corollary}\label{cor:fac_to_comb}
The following map can be implemented in $\LAQCC\text{-}\mathsf{LOG}$.
\[
\frac{1}{\sqrt{n!}} \sum_{y \in \text{Fact}(n)} \ket{y}\ket{0} \xrightarrow{} \frac{1}{\sqrt{n!}} \sum_{y \in \text{Fact}(n)} \ket{y}\ket{\mathcal A (y)}.
\]
\end{corollary}
 

\noindent In the next lemma, we show that a $\TC^0$ circuit can implement the inverse of $\mathcal A$. 
\begin{lemma}
\label{lem:comb_to_fac}
There exists a $\TC^0$ algorithm which, when given an $n$-bit string $y^{comb(k)}$ of Hamming weight $k$, a uniformly-random $k$-factoradic, and a uniformly-random $(n-k)$-factoradic, outputs a uniformly random $n$-factoradic $y$ among those such that $\mathcal A(y) = y^{comb(k)}$.
\end{lemma}

\begin{proof}
The conversion can be done in parallel, generating an $n$-digit for every bit in $y^{comb(k)} = y_{n-1} \dots y_0\in\{0,1\}^n$. Recall that we are given as input a uniformly-random $k$-factoradic $O_{k-1}, \dots, O_0$ and a uniformly-random $(n-k)$-factoradic $Z_{n-k-1}, \dots, Z_0$.

For every bit position $n-j$, for $1 \le j \le n$, calculate the Hamming weight of the bits from $n-j+1$ to $n-1$: $H_{>n-j} = \sum_{i=j+1}^{n-1} y^{comb(k)}_i$. Recall that iterated addition is in $\TC^0$~\cite{vollmer1999introduction}.

If $y^{comb(k)}_{n-j} = 1$, set $y_{n-j} = O_{k - H_{> n-j}}$. This gives a uniform random $n$-digit between $0$ and $k - H_{> n-j} - 1$. If $y^{comb(k)}_{n-j} = 0$, set $y_{n-j} = k - H_{> n-j} + Z_{n-k-H_{>n-j}}$. 
Note that this gives a uniform random $n$-digit between $k - H_{>n-j}$ and $n - j$. 
By construction, it now follows that $\mathcal A(y) = y^{comb(k)}$. 
Computing each $n$-digit in this way requires summation and indexing, both of which are in $\AC^0 \subseteq \TC^0$~\cite{vollmer1999introduction}.
\end{proof}

\begin{remark}\label{rem:comb_to_fac}
The above algorithm establishes a bijection $(y^{comb(k)}, Z, O) \leftrightarrow y$  between triples $(y^{comb(k)}, Z, O)$ with $y^{comb(k)} \in \{0,1\}^n$ of Hamming weight $k$, $Z \in \text{Fact}(n-k)$ and $O\in\text{Fact}(k)$ and $n$-factoradics $y \in \text{Fact}(n)$. Let $(\mathcal A(y), \mathcal Z(y), \mathcal O(y))$ be the image of an $n$-factoradic $y$ under this bijection. The previous lemma shows that one can compute $y$ from $(y^{comb(k)}, Z, O)$ in $\TC^0$. 
It is not hard to see that the map $(\mathcal A(y), y) \mapsto (\mathcal A(y), y, \mathcal Z(y), \mathcal O(y))$ is also in $\TC^0$. Indeed, to find $\mathcal Z(y)$ and $\mathcal O(y)$, we need only invert the two defining equalities $y_{n-j} = O_{k - H_{> n-j}}$ and $y_{n-j} = k - H_{> n-j} + Z_{n-k-H_{>n-j}}$.
\end{remark}

\begin{corollary}\label{cor:comb_to_fac}
The following map can be implemented in $\LAQCC$.
\[
\begin{pmatrix}n \\ k\end{pmatrix}^{-\frac{1}{2}}\sum_{y^{comb(k)}} \ket{0} \ket{y^{comb(k)}} \xrightarrow{} \frac{1}{\sqrt{n!}} \sum_{y \in \text{Fact}(n)} \ket{y}\ket{\mathcal A (y)}
\]
where $y^{comb(k)}$ ranges over all $n$-bit strings of Hamming weight $k$.
\end{corollary}

\begin{proof}
The transformation consists of three steps: 
\begin{align*}
& \binom{n}{k}^{-\frac{1}{2}}\sum_{y^{comb(k)}} \ket{y^{comb(k)}} \ket{0}\ket{0}\ket{0}\\
\xrightarrow{(1)} \;\; &\binom{n}{k}^{-\frac{1}{2}}\sum_{y^{comb(k)}} \ket{y^{comb(k)}} \left(\bigotimes_{j = 0}^{n-k-1} \sum_{i = 0}^{j} \ket{i}\right)\left(\bigotimes_{j = 0}^{k-1} \sum_{i = 0}^{j} \ket{i}\right)\ket{0}\\
= \;\; & \frac{1}{\sqrt{n!}} \sum_{y^{comb(k)}} \ket{y^{comb(k)}} \left(\sum_{Z\in\text{Fact}(n-k)} \ket{Z}\right)\left(\sum_{O\in\text{Fact}(k)} \ket{O}\right)\ket{0}\\
\xrightarrow{(2)}\;\; & \frac{1}{\sqrt{n!}} \sum_{y\in\text{Fact}(n)} \ket{\mathcal A(y)} \ket{\hat Z(y)}\ket{\hat O(y)}\ket{y}\\
\xrightarrow{(3)}\;\; & \frac{1}{\sqrt{n!}} \sum_{y\in\text{Fact}(n)} \ket{\mathcal A(y)} \ket{0}\ket{0}\ket{y}
\end{align*}
Step (1) prepares a uniform superposition over all $n$-factoradics using Theorem~\ref{thm:uniform_superposition_mod_q}. 
Step (2)  is Lemma~\ref{lem:comb_to_fac}, and Step (3) follows from Remark~\ref{rem:comb_to_fac}.
In the above steps we implicitly used that the inverse of the used $\LAQCC$ operations are also $\LAQCC$ operations. 
Even though it is unclear if this inverse-property holds in general, it does hold for the considered $\LAQCC$ operations. 
The measurement steps, which might not be reversible, in this algorithm are used to implement fan-out gates.
The inverse of a fan-out gate is the fan-out gate itself and hence is contained in $\LAQCC$.
\end{proof}

\begin{theorem}
\label{thm:Dicke:Log_depth}
There exists a $\LAQCC\text{-}\mathsf{LOG}$-circuit for preparing Dicke-$(n,k)$ states, for any positive integers $n$ and $k \le n$, it uses $\mathO(\text{poly}(n))$ qubits. 
\end{theorem}
\begin{proof}
The circuit combines the circuits resulting from Lemma~\ref{lem:fac_to_comb} and Lemma~\ref{lem:comb_to_fac}. 

It consists of three steps: 
\begin{align*}
\ket{0}^{\otimes n \log(n)}\ket{0}^{\otimes n} &\xrightarrow{(1)} \frac{1}{\sqrt{n!}}\left(\bigotimes_{j = 0}^{n-1} \sum_{i = 0}^{j} \ket{i}\right)\ket{0}^{\otimes n} = \sum_{y \in \text{Fact}(n)} \ket{y}\ket{0} \\
&\xrightarrow{(2)} \frac{1}{\sqrt{n!}} \sum_{y \in \text{Fact}(n)} \ket{y}\ket{\mathcal A (y)}\\
&\xrightarrow{(3)} \begin{pmatrix}n \\ k\end{pmatrix}^{-\frac{1}{2}}\sum_{y \in \text{Fact}(n)} \ket{0}\ket{\mathcal A (y)} = \ket{D^n_k}.
\end{align*}
Step (1) prepares a uniform superposition over all $n$-factoradics using Theorem~\ref{thm:uniform_superposition_mod_q};
Step (2) is by Corollary \ref{cor:fac_to_comb};
and, Step (3) reverses the algorithm of Corollary \ref{cor:comb_to_fac}.
\end{proof}

\subsection{Quantum many-body scar states}
\label{sec:many_body_scar}
There is a particular set of states in many-body physics called, many-body scar states, which are highly excited states that exhibit atypically low entanglement~\cite{turner2018weak}. These states exhibit long coherence times relative to other states at the same energy density and seem to avoid thermalization and thereby they do not follow the eigenstate thermalization hypothesis.
This makes studying the lifetime of quantum many body scar states under perturbations particularly interesting.
Studying this lifetime is quite challenging, as even though scarred eigenstates often have modest entanglement and therefore have efficient matrix product state representations, perturbations typically couple them to states nearby in energy which typically have volume-law scaling entanglement, making classical simulations difficult.

A overview paper by \citeauthor{Gustafson:2023} studied methods of preparing quantum many-body scar states on quantum computers, with the goal to simulate time dynamics directly on the quantum system~\cite{Gustafson:2023}. 
They found several approaches for generating quantum many-body scars for a particular model, which require polynomial depth. They look at quantum many-body scar states of the $n$-qubit spin-1/2 Hamiltonian of~\cite{iadecola2020quantum}:
\[
    H = \lambda \sum_{i = 2}^{n - 1} (X_i - Z_{i - 1} X_i Z_{i +1}) + \Delta \sum_{i = 1}^n Z_i + J \sum_{i = 1}^{n-1} Z_i Z_{i + 1}
\]

The quantum many-body scar states of $n$-qubits are given by:
\[
\ket{S_k} = \frac{1}{k! \sqrt{\mathcal N(n,k)}}(Q^\dagger)^k \ket{\Omega},
\]
where $\mathcal N(n,k) = \binom{n - k - 1}{k}$, $\ket{\Omega} = \ket{0}^{\otimes n}$ and $k = 0, \dots, n/2 - 1$. The raising operator $Q^\dagger$ is given by:
\[
 Q^\dagger = \sum_{i = 2}^{n-1} (-1)^i P_{i-1} \sigma^+_{i} P_{i +1}, 
\]
with $P_j = \ket{0}\bra{0}$ and $\sigma^+_j = X_j + Y_j$.
They show that up to local $Z$ gates these states are very closely related to Dicke states:
\[
    \prod_{\text{i odd}} Z_i \ket{S_k} = \ket{0} \otimes P_{fib} \ket{D^n_k}\otimes \ket{0},
\]
where $P_{fib}$ is known as the Fibonacci constraint, which is a projector that removes all states where there are two ones next to each other:
\[
P_{fib} = I - \sum_{i = 1}^{n-1} \ket{11}\bra{11}_{i,i+1}.
\]
The goal of this section is to show that these states, for $k = \mathO(\sqrt{n})$ are accessible in $\LAQCC$. First note that by Theorem~\ref{thm:dicke_cnst_depth} there exists a $\LAQCC$ protocol to generate $\ket{D_k^n}$ up to $k = \mathO(\sqrt{n})$.
We will show that there is a $\LAQCC$ protocol that applies $P_{fib}$ to these $\ket{D^n_k}$ states. The first step will be to show that there exists a unitary accessible in $\LAQCC$ that flags the correct state.

\begin{lemma}
\label{lem:Ufib}
There exists a unitary $U_{fib}$, accessible in $\LAQCC$, that flags all the states that obey the Fibonacci constraint, more precisely:
\[
    U_{fib} \ket{D_k^n}\ket{0} = \alpha P_{fib}\ket{D_k^n}\ket{0} + \beta (I - P_{fib})\ket{D_k^n}\ket{1}
\]
\end{lemma}
\begin{proof}
Add $n$ extra qubits prepared in $\ket{0}$, one for every sequential pair of qubits. 
For all $i\in\{1,\hdots,n-1\}$, apply a Toffoli gate with control qubits $i$ and $i + 1$ and target qubits the $i$-th auxillary qubit. 
The second step is to apply the $OR_{n}$ gate on the $n$ auxillary qubits.
The $n$-th auxillary qubit is also used as the output qubit. 
Clean the extra qubits by again applying Toffoli gates. These steps are accessible in $\LAQCC$ therefore implements the flag unitary with an $\LAQCC$ protocol.
\end{proof}

The second step is to show that $\alpha$ is bounded by a constant in the case that $k = \mathO(\sqrt{n})$, this is shown using the following two lemma's.

\begin{lemma}

The total number of bitstrings of length $n$ with $k$ ones, such that no two ones are adjacent is given by:
\[
    \binom{n - k}{k} + \binom{n - k - 1}{k - 1}
\]
\end{lemma}

\begin{proof}
We can count the number of possible bitstrings, after first noticing that every $1$ must be followed by a $0$, unless the last bit is a $1$. 
As a result, we have two situations, in the first, we can consider all possible rearrangements $n-k$ elements, consisting of $k$ pairs `$10$' and $n-2k$ ones. 
This gives $\binom{n-k}{k}$ possible bitstrings. 
In the second situation, the last bit is $1$. 
This leave $k-1$ pairs `$10$' in a total of $n-k-1$ elements. 
With the same reasoning, this gives $\binom{n-k-1}{k-1}$ possible bitstrings. 
Summing the two situations proves the lemma. 
\end{proof}

We now consider the relative fraction of this type of bitstrings among all possible bitstrings with Hamming-weight $k$. 
\begin{lemma}
\label{lem:fraction_good_strings}
Let $k = c\sqrt{n}$ for some constant $c>0$. 
Then the following inequality holds
\[
\frac{\binom{n - k}{k} + \binom{n - k - 1}{k - 1}}{\binom{n}{k}} \geq \exp\big(-c^2\big)
\]
\end{lemma}

\begin{proof}
We have
\[
\frac{\binom{n - k}{k} + \binom{n - k - 1}{k - 1}}{\binom{n}{k}} > \frac{\binom{n - k}{k}}{\binom{n}{k}} = \frac{(n-k)! / k!(n-2k)!)}{n!/ k!(n-k!)} = \frac{(n-k)!^2}{n!(n-2k)!}.
\]
Expanding the factorials and only consider the terms that do not cancel, we obtain
\[
\frac{(n-k)!}{n!} \frac{(n-k)!}{(n-2k)!} = \frac{(n-k)(n- k - 1)\dots (n- (2k - 1))}{n (n-1)\dots (n - (k-1))}.
\]
Both the numerator and denominator have $K$ terms, which we can pair. 
next we note that $\tfrac{a}{b} > \tfrac{a-1}{b-1}$ whenever $b>a$ (and $b\not\in\{0,1\}$). 
Using this idea, we obtain the following expression:
\[
\frac{n-k}{n}\frac{n-k-1}{n-1}\dots \frac{n-(2k-1)}{n-(k-1)} > (\frac{n-k}{n})^k = (1 - \frac{k}{n})^k.
\]
Now using that $k = c \sqrt{n}$, we have
\[
\left(1 - \frac{c \sqrt{n}}{n}\right)^{c\sqrt{n}}  = \left(\left(1-\frac{c}{\sqrt{n}}\right)^{\frac{\sqrt{n}}{c}}\right)^{c^2} > \exp\big(-c^2\big),
\]
which is a constant.
\end{proof}

This allows us construct the state $\ket{S_k}$ using the steps for the Dicke-state preparation together with Lemma~\ref{lem:grover_constant_fraction}. 
Note that Lemma~\ref{lem:grover_constant_fraction} requires us to implement both $U$ and $U^{\dagger}$, where $U$ implements the initial superposition.
The \textbf{Ordering}-step (Lemma~\ref{lem:dicke_ordering}) does however use measurements, which stops us makes implementing the inverse of the circuit hard. 
Still, we can work around this, by applying Lemma~\ref{lem:grover_constant_fraction} between the \textbf{Filtering} and \textbf{Ordering} step:
\begin{theorem}
For any $n$ and $k = \mathO(\sqrt{n})$ there exists a $\LAQCC$ circuit preparing the many-body scar state $\ket{S_k}$, using $\mathO(n^2\log(n))$ qubits.
\end{theorem}
\begin{proof}
We follow the same steps as for the Dicke-state preparation (see Theorem~\ref{thm:dicke_cnst_depth}). 
After the second \textbf{Filtering} step however, we apply use the unitary $U_{fib}$ together with Lemma~\ref{lem:grover_constant_fraction} to filter out all states with subsequent ones in the state. 
Note that by Lemma~\ref{lem:fraction_good_strings}, the number of states with no consecutive ones is a constant fraction of the total number of strings of Hamming-weight~$k$. 

Furthermore, the state after the Filtering step,
\begin{equation*}
    \sqrt{\frac{(n-k)!}{(n)!}}\sum_{j_1 \neq \dots \neq j_k} \ket{j_1} \dots \ket{j_k} \ket{e_{j_1} \oplus \dots \oplus e_{j_k}},
\end{equation*}
is still entangled with the index registers. In effect there are many copies of the Dicke-$(n,k)$ state in on the system register, each with a diferent ordering of the index registers, however this does not affect the fraction of states with no consecutive ones compared to the states with consecutive ones. 
Next, the \textbf{Ordering} and \textbf{Cleaning} step of the protocol work similarly on the resulting state and will give the state $\ket{S_k}$.
\end{proof}


%\section{Complexity results for $\LAQCC(\mathcal{Q}, \mathcal{C},d)$}
\label{sec:Complexity results}

Up until now we have considered the class $\LAQCC$. 
In this section, we investigate the state-preparation complexity of $\LAQCC(\mathcal Q, \mathcal C, d)$ when we increase the power of the quantum circuits to polynomial depth, and we allow for unbounded classical computational power.

\begin{notation}
The class $\LAQCC^*$ is the instantiation $\LAQCC(\mathsf{QPoly}(n), \mathsf{ALL}, \mathrm{poly}(n))$:
The class of polynomially many alternating polynomial-sized quantum circuits and arbitrary powerful classical computations, together with feed-forward of the classical information to future quantum operations. 
The quantum computations are restricted to all single-qubit gates and the two-qubit CNOT gate. 
\end{notation}

Recall the definition of $\mathsf{State}$-classes:
\StateClassX*
From which we directly have a definition for the class $\mathsf{State}$$\LAQCC^*_{\varepsilon}$ for any $\varepsilon>0$.

\begin{remark}
Note that for any non-zero $\varepsilon$, we can restrict ourselves to finite universal gate-sets. 
The Solovay-Kitaev theorem~\cite{Kitaev1997,Nielsen2012} says that any multi-qubit unitary can be approximated to within precision $\delta$ by a quantum circuit with size depending on $\delta$. 
Therefore, with a finite universal gate-set, any $\LAQCC^*$ circuit with a continuous gate-set can be approximated by an $\LAQCC^*$ circuit with gates from the finite set. 
\end{remark}

To upper bound the state preparation complexity of $\LAQCC^*$, we compare it to the class $\mathsf{PostQPoly}$ and its circuit variant.

\PostQPolyDef*

Next, we prove $\mathsf{State}$$\LAQCC^*_{\varepsilon}\subseteq\mathsf{StatePostQPoly}_{\varepsilon}$. 
Section~\ref{sec:LAQCC_in_QNC1} describes how any $\LAQCC$ circuit decomposes in alternating unitaries, measurements and classical calculation layers. 
A similar decomposition follows for $\LAQCC^*$ circuits: 
Any $\LAQCC^*$ can be written as 
\begin{equation}
    \Pi_{i=0}^{\text{poly}(n)} M_i U_i(y_i) C_i(x_i)\ket{0}^{\otimes \text{poly}(n)},
\end{equation}
where each $x_i\in\{0,1\}^*$ denotes the measurement outcome of the $i$-th measurement layer and $y_i\in \{0,1\}^*$ the output bitstring of the $i$-th classical calculation layer. 
Note, all $x_i$ and $y_i$ have length at most polynomial in $n$. 
The $U_i$'s denote unitary operations that represent a polynomial-deep quantum circuit consisting of single qubit gates and the two-qubit CNOT gate. 
The $M_i$'s denote a measurement of a subset of the qubits, and the $C_i$'s represent unbounded classical computations, with input $x_i$. 

\begin{theorem}
\label{thm:state_prep_post-BQP}
It holds that $\mathsf{State}$$\LAQCC^*_{\varepsilon}\subseteq\mathsf{StatePostQPoly}_{\varepsilon}$.
\end{theorem}
\begin{proof}
Fix $\varepsilon>0$ and a positive integer $n$ and 
let $\ket{\psi}\in\mathsf{State}$$\LAQCC^*_{\varepsilon}$. 
By definition, there exists an $\LAQCC^*$ circuit $A=\Pi_{i=0}^{\text{poly}(n)} M_i U_i(y_i) C_i(x_i)$, which prepares a state $\ket{\phi}$ with inner product at least $1-\varepsilon$ with $\ket{\psi}$.

Then consider the following $\mathsf{PostQPoly}$-circuit:
Let $B=\Pi_{i=0}^{\text{poly}(n)} \text{Equal}_{x_i}(x_i)U_i(y_i)\ket{0}^{\otimes \text{poly}(n)}$, where the $y_i$ are hardwired. 
The Equal$_{x_i}$ gate replaces the measurement layer $M_i$, by checking if the subset of qubits that would be measured are in $\ket{x_i}$ computational basis state. It stores the output in an ancilla qubit. 
As a last step, apply an $\text{AND}_{\text{poly}(n)}$-gate on the ancilla qubits, which hold the outputs of the Equal$_{x_i}$ gates, and store the result in an ancilla qubit. 
Conditional on this last ancilla qubit being one, the circuit prepares the state $\ket{\phi}$.
\end{proof}

Figure~\ref{fig:laqcc_to_post_bqp} gives a schematic overview of the proof and the translation of an $\LAQCC^*$ circuit in a $\mathsf{PostQPoly}$-circuit. 
% Figure environment removed

An interesting direction for future work is to extend the inclusion proved above to a true separation or an oracular separation. 
One approach is to use a similar oracle as used in~\cite{AaronsonKuperberg:2007} to separate $\mathsf{QMA}$ and $\mathsf{QCMA}$ with respect to an oracle and use a counting argument to argue that $\mathsf{State}$${\LAQCC^*_{\varepsilon}}^{\mathcal{O}}\neq \mathsf{StatePostQPoly}_{\varepsilon}^{\mathcal{O}}$, for some oracle $\mathcal{O}$ and $\varepsilon=1-\frac{1}{\text{poly}(n)}$. 
The $\LAQCC$ model allows for a constant number, more than one, of rounds of measurements and corrections. This was required for our three new state generation protocols. However other models considered only one round of measurements and corrections, for instance in the paper \cite{Cirac:2021}. One may wonder if there is a hierarchy of model power allowing one or multiple measurements, and if there is a way to reduce the number of measurements rounds. A starting effort towards classifying types of states based on such a hierarchy can be found in \cite{Tantivasadakarn_2023}. It would be interesting to see a more extensive complexity theoretic analysis comparing models with different number of allowed rounds.


\section*{Acknowledgements}
We want to thank Jonas Helsen, Joris Kattem{\"o}lle, Ido Niesen, Kareljan Schoutens, Florian Speelman, Dyon van Vreumingen and Jordi Weggemans for insight full discussions. 
%Furthermore, we would like to thank Chris Cade for suggesting to look at the oracle from~\cite{AaronsonKuperberg:2007} for separating $\LAQCC^*$ and $\mathsf{Post}$-$\BQP$ and Georgios Styliaris for first mentioning how to parallelize Clifford ladder circuits. 
Furthermore, we would like to thank Georgios Styliaris for first mentioning how to parallelize Clifford ladder circuits. 
HB and MF were supported by the Dutch Ministry of Economic Affairs and Climate Policy (EZK), as part of the Quantum Delta NL programme. 
\iffalse
NN was supported by the quantum application project of TNO. 
This work was supported by the Dutch Research Council (NWO/OCW), as part of the Quantum Software Consortium programme (project number 024.003.037).
\fi


\printbibliography


\appendix

\begin{comment}
\section{System Architecture}
\label{appendix:architecture}
\system has a novel modularized system architecture with three key components: 
\emph{StreamManager}, 
\emph{TxnManager} and \emph{TxnScheduler}. 
These components are instantiated in each thread locally.
The execution outline of \system is presented in Algorithm~\ref{alg:algo}.
Transactional stream processing is continuous and potentially never ends (Line 1$\sim$8).
The dependency resolution and execution of state transactions are separated into two non-overlapping phases by punctuations~\cite{Tucker:2003:EPS:776752.776780} (Line 2 and 5), which guarantees that no subsequent input event will have a smaller timestamp. 
Effectively, a batch of state transactions is collected during the first phase, and processed during the second phase.

In the first phase (i.e., stream processing phase), 
the \emph{StreamManager} conducts preprocessing for every input event ($e$). Similar to some prior works~\cite{tstream}, state transactions may be issued but not immediately processed during preprocessing (Line 3).
The \emph{pre\_processing} and \emph{post\_processing} functions are exposed as APIs to users.
The \emph{TxnManager} handles dependency resolution (Line 4) among state transactions and insert decomposed operations to construct a \tpg. We discuss the detailed two-phase \tpg construction process in Section~\ref{subsec:construction}.

In the second phase  (i.e., transaction processing phase), 
the \emph{TxnManager} is first involved again to refine (Line 6) the constructed \tpg with further dependency resolution.
The \emph{TxnScheduler} 
schedules operations for concurrent execution based on the constructed \tpg according to the three dimensions of scheduling decisions (Line 7). 
In particular, a scheduling decision model $M$ is instantiated based on the constructed \tpg (Line 14).
\textbf{\circled{1}} Guided by $M$, execution threads adopt an exploration strategy (Section~\ref{subsec:explore}) to explore the constructed \tpg for operations available to be scheduled constrained by dependencies. 
\textbf{\circled{2}} 
During exploration, one or multiple operations may be treated as the 
% basic 
unit of scheduling (Section~\ref{subsec:granularity}). 
Subsequently, \textbf{\circled{3}} every thread executes operation(s) in the unit of scheduling with various abort handling mechanisms (Section~\ref{subsec:abort_handling}).
Only when state transactions are processed (i.e., committed or aborted) can the associated input events be postprocessed (Line 8) by the \emph{StreamManager} based on transaction processing results.
\end{comment}

\begin{comment}
\begin{algorithm}
\footnotesize
    \KwData{$e$ \tcp{Input event}}
    \KwData{$txn_{ts}$ \tcp{State transaction}}
    \KwData{$G$ \tcp{The currently constructed TPG}}
    \While{!finish processing of input streams}{
        \eIf(\tcp*[h]{Phase 1}){\text{$e$ is not a $punctuation$}}{
                $txn_{ts}$ $\gets$ PRE\_Processing($e$)\;
                \textbf{TPG\_Construction}($G$, $txn_{ts}$)\; 
          }(\tcp*[h]{Phase 2}){
                \textbf{TPG\_Refinement}($G$)\; 
                \textbf{TXN\_Scheduling}($G$)\; 
                POST\_Processing()\;
          }
    }
    
    \SetKwFunction{FMain}{TPG\_Construction}
    \SetKwProg{Fn}{Function}{:}{}
    \Fn{\FMain{$G$, $txn_{ts}$}}{
        $O_{1..k}$ $\gets$ \textbf{Partition} $txn_{ts}$\;
        \ForEach{\text{operation $O_{i}$ $\in$ $O_{1..k}$}}{
            \textbf{Identify} its \ld\;
            $G$ $\gets$ $G$ + $O_{i}$ \;
        }
    }
    \SetKwFunction{FMain}{TPG\_Refinement}
    \SetKwProg{Fn}{Function}{:}{}
    \Fn{\FMain{$G$}}{
        \ForEach{\text{vertex $e_{i}$ $\in$ $G$}}{
            \textbf{Identify} its \td, \pd\;
        }
    }
    
    \SetKwFunction{FMain}{TXN\_Scheduling}
    \SetKwProg{Fn}{Function}{:}{}
    \Fn{\FMain{$G$}}{
        $M$ $\gets$ Instantiated with $G$;\tcp{A decision model}
        \While{!finish scheduling of $G$
        }{
          \textbf{\circled{2}} $Scheduling Unit$ $\gets$ \textbf{\circled{1}} \emph{Explore}($G$, $M$)\; 
            \textbf{\circled{3}} \emph{Execute with Abort Handling} ($Scheduling Unit$)\; 
        }
    }
  \caption{Execution Outline of \system}
  \label{alg:algo}
\end{algorithm}
\end{comment}

\end{document}
