\section{Useful lemmas}
This section gives two lemmas.
The first lemma upper bounds the computational power of $\LAQCC$.
The second lemma helps in preparing Dicke states for $k\in\mathO(\sqrt{n})$. 
\begin{lemma}
\label{lem:NC1toQauntum}
Let $\Pi = (\Pi_{yes}, \Pi_{no})$ be a decision problem in $\NC^1$. Then there is a uniform log-depth quantum circuit that decides on $\Pi$.
\end{lemma}
\begin{proof}
Let $B$ be the uniform Boolean circuit of logarithmic depth deciding on $\Pi$. 
As $\Pi\in\NC^1$, such a circuit exists. 

For fixed input size $n$, write $B$ as a Boolean tree of depth $\mathO(\log(n))$, with at its leaves the $n$ input bits $x_i$ and as root an output bit. 
This Boolean tree directly translates in a classical circuit using layers of AND, OR and NOT gates.

Each of these gates has a direct quantum equivalent gate, provided that we use ancilla qubits: 
First replace all OR gates by AND and NOT gates. 
Then replace all AND gates by Tofolli gates, which has three inputs.
The third input is a clean ancilla qubit and will store the AND of the other two inputs. 
Finally, replace all NOT gates by $X$-gates. 
\end{proof}
\begin{lemma}
\label{lem:birthday_paradox}
Let $n, k \in \mathbb{N}$ and $k < \frac{n}{2}$ then:
\begin{align*}
    \frac{n!}{n^k (n-k)!} > e^{-\frac{2 k^2}{n}}
\end{align*}
\end{lemma}
\begin{proof}
The result follows by a simple computation
\begin{align*}
    \frac{n!}{n^k (n-k)!} &= e^{\sum_{i = 1}^k \log(1 - \frac{i}{n})}\\
    & > e^{\sum_{i = 1}^k \frac{-i}{n-i}} \\
    & > e^{\sum_{i = 1}^k \frac{-i}{n-k}} \\
    & > e^{-\frac{k^2}{n-k}} \\
    & > e^{-\frac{2 k^2}{n}},
\end{align*}
where we use that $\log(1 + x)\geq \frac{x}{1+x}$ for $x > - 1$.

\end{proof}
\section{Gate implementations}

\subsection{OR-gate}\label{gate:OR_implementation}
In this section we discuss the implementation of the OR$_n$-gate and why we can implement it using local gates in a nearest-neighbor architecture.
We show how the gate works for any basis state $\ket{x}=\ket{x_1}\otimes \hdots\otimes\ket{x_n}$. 
By linearity, the gate then works for arbitrary superpositions. 
The OR-gate by \citeauthor{TakahashiTani_CollapseOfHierarchyConstantDepthExactQuantumCircuits_2013} consists of two steps: First they apply the OR-reduction introduced in Ref.~\cite{HoyerSpalek:2005}, which prepares a state on $\log n$ qubits such that the OR evaluated on these $\log n$ qubits yields the same result as the OR evaluated on the original $n$ qubits.
Second, they  evaluate an exponential circuit on these $\log n$ qubits to calculate the OR-gate. This results in a polynomial-sized circuit for the OR-gate. 

Let $m=\ceil{\log_2(n+1)}$, then the OR-reduction implements the map
\begin{equation*}
    \ket{x}\ket{0}^{\otimes m}\to \ket{x}\bigotimes_{j=1}^m\ket{\mu^{|x|}_{\phi_j}},
\end{equation*}
where $\varphi_j=\frac{2\pi}{2^j}$ and $\ket{\mu_{\varphi}^{|x|}} = H R_Z(\varphi\cdot |x|) H\ket{0}$. 
The OR-reduced state thus depends on the weighted Hamming weight of $x$,
more precisely, every $\ket{\mu^{|x|}_{\phi_j}}$ depends on the entire string $x$. For every $\ket{\mu^{|x|}_{\phi_j}}$ this requires a copy of $\ket{x}$ which can be created by using the fanout gate. The circuit applying the rotation $R_Z(\varphi\cdot |x|)$ consists of sequential $R_Z$ gates, which are diagonal and hence can be implemented in parallel  by Lemma~\ref{lem:unitar_parallelization}. This results in a circuit of width $\mathO(n \log(n))$ for creating the OR-reduced state.

%For every input qubit $\ket{x_i}$, we can prepare the states $\ket{\mu_{\varphi_j}^{|x|}}$ for every $j$ using two rows of qubits. 
%First apply a fanout-gate in the second row to prepare a GHZ state there. Next, in parallel, apply the $R_Z$ gates controlled by the input qubits $\ket{x_i}$.
%And finally, apply a fanout-gate to disentangle the second row of qubits. 
%We can prepare the state $\ket{\mu_{\varphi_j}^{|x|}}$ for every $j$ in parallel by first copying the two rows of qubits $m$ times using fanout-gates.
%This first step has a local nearest-neighbor implementation for the OR-reduced state. 

We have for every bitstring $x\in\{0,1\}^n$ that 
\begin{equation*}
    \text{OR}_n(x) = \frac{1}{2^{n-1}}\sum_{a\in\{0,1\}^n\setminus \{0^n\}} \text{PA}_n^a(x),
\end{equation*}
where PA$_n^a(x) = \oplus_{j=0}^{n-1} a_i x_i$ is the parity of $x$, weighted by the non-zero binary vector~$a$.
Hence, computing the OR of the input is now reduced to computing all parities of the subsets of the inputs. 
The parity gate is equivalent to a fanout-gate conjugated by Hadamard gates on every qubit, see also Table~\ref{tab:Fanout_Perm}.

We now copy the $m$ qubits of the OR-reduced state $2^m = n$ times, using fanout gates. 
For each copy we require two additional auxiliary qubits. 
The first will hold the result of the parity computation, for which we already have a nearest-neighbor implementation. 
We will entangle the second auxiliary qubit using a fanout-gate to obtain a GHZ-state in these qubits. 

We now compute in parallel the parity of the subsets of the inputs. 
For every subset we use a single copy of the inputs and we store the result in the corresponding auxiliary qubit. 
We then apply a controlled-$R_Z$ gate from the first auxiliary qubit to the second auxiliary qubit in the GHZ state. 
This prepared the state (omitting other registers)
\begin{equation*}
    \frac{1}{\sqrt{2}}(\ket{0}^{n-1} + (-1)^{\frac{1}{2^{n-1}}\sum_{a\in\{0,1\}^m\setminus \{0^m\}} \text{PA}_m^a(x)}\ket{1}^{\otimes n-1}) = \frac{1}{\sqrt{2}}(\ket{0}^{n-1} + (-1)^{\text{OR}_{n}(x)}\ket{1}^{\otimes n-1}).
\end{equation*}
Uncomputing this final state using a fanout-gate gives a single qubit that holds the desired answer in its phase. 
A single Hadamard gate applied to this qubit will then give the answer in a single qubit. 

Combining all steps thus gives an implementation for the OR gate using a geometrically local nearest-neighbor circuit. 
For more details on the implementation as well as a proof of correctness, we refer to the original proof~\cite{TakahashiTani_CollapseOfHierarchyConstantDepthExactQuantumCircuits_2013}.

\subsection{Equality-gate}
\label{gate:equality}
Define the Equality gate on two $n$-qubit computational basis states as
$$\mathrm{Equality}: \ket{x}\ket{y}\ket{0}\mapsto\ket{x}\ket{y}\ket{\mathbbm{1}_{x=y}}.$$
This gate is implemented in three steps: 
(1) subtract the first register from the second using a subtraction circuit; 
(2) apply $\mathrm{Equal}_0$ on the second register and store the result in the third register; 
(3) add the first register to the second, undoing the subtraction computation:
\begin{align*}
    \ket{x}\ket{y}\ket{0}&\xrightarrow[(1)]{} \ket{x}\ket{y - x}\ket{0}\\
    &\xrightarrow[(2)]{} \ket{x}\ket{y - x}\ket{\mathbbm{1}_{x=y}}\\
    &\xrightarrow[(3)]{}\ket{x}\ket{y}\ket{\mathbbm{1}_{x=y}}
\end{align*}
Addition and subtraction both have width $\mathO(n^2)$, which, as a result, the $\mathrm{Equality}$-gate also has.

\subsection{Greaterthan-gate}
\label{gate:greaterthan}
Define the Greaterthan gate on two $n$-qubit computational basis states as
$$\mathrm{Greaterthan}: \ket{x}\ket{y}\ket{0}\mapsto\ket{x}\ket{y}\ket{\mathbbm{1}_{x>y}}.$$
This gate is implemented in four step: 
(1) Add an extra clean qubit to the second register and interpret this as an $n+1$-qubit register with most significant bit zero; 
(2) subtract the first register from the second. The subtraction is modulo $2^{n+1}$; 
(3) apply a CNOT-gate from most significant bit of the second register to the third register; 
(4) add the first register to the second, undoing the subtraction computation:
\begin{align*}
    \ket{x}\ket{y}\ket{0}&\xrightarrow[(1)]{} \ket{x}\ket{0y}\ket{0}\\
    &\xrightarrow[(2)]{} \ket{x}\ket{y- x \bmod 2^{n+1}}\ket{0}\\
    &\xrightarrow[(2)]{} \ket{x}\ket{y - x \bmod 2^{n+1}}\ket{\mathbbm{1}_{x>y}}\\
    &\xrightarrow[(3)]{}\ket{x}\ket{0}\ket{y}\ket{\mathbbm{1}_{x>y}}
\end{align*}
This construction works, as after step (2), the most significant bit of the second register is one, precisely if $x$ is larger than $y$. 

Addition and subtraction both have width $\mathO(n^2)$, which, as a result, the $\mathrm{Greaterthan}$-gate also has.

\subsection{Exact\texorpdfstring{$_t$}{}-gate}
\label{gate:exact_t}
Define the Exact$_t$ gate on an $n$-qubit computational basis state as
$$\mathrm{Exact}_t: \ket{x}\ket{0}\mapsto\ket{x}\ket{\mathbbm{1}_{|x|=t}}.$$
Here, $|x|$ denotes the Hamming weight of the $n$-bit string $x$. 

This gate follows by combining the Hammingweight-gate and the Equality-gate: First, compute the Hamming weight of $x$ in an ancilla register and then apply the Equality gate to check that this ancilla register equals $t$. 

Another approach is to modify the circuit for $OR$ slightly. 
In the $OR$-reduction step, add a gate $R_Z(-\varphi t)$, which adjusts the angle to be zero precisely if $|x|=t$ (see Theorem 4.6 of~\cite{HoyerSpalek:2005}).
Then apply the circuit for $OR$ and negate the output. 
The circuit for $OR$ evaluates to zero, precisely if the input had Hamming weight $t$. 


\subsection{Threshold\texorpdfstring{$_t$}{}-gate}
\label{gate:threshold_t}
Define the Threshold$_t$ gate on an $n$-qubit computational basis state as
$$\mathrm{Threshold}_t: \ket{x}\ket{0}\mapsto\ket{x}\ket{\mathbbm{1}_{|x|\ge t}}.$$

Taking the $OR$ over the outputs of $\mathrm{Exact}_j$-gates for all $j\ge t$, gives the $\mathrm{Threshold}_t$-gate. 
An improved implementation with better scaling in $t$ is given in Theorem~2 of~\cite{TakahashiTani_CollapseOfHierarchyConstantDepthExactQuantumCircuits_2013}.

A weighted threshold gate uses weights $w_i$ and evaluates to one precisely if $\sum_i w_i x_i \ge t$. 
Assume without loss of generality that $\sum_i w_i x_i$ evaluates to an integer. 
Otherwise, we can use the same ideas, but up to some precision. 

Use the same $OR$-reduction as for the normal threshold gate. 
Instead of rotations $R_Z(\varphi)$ controlled by $x_i$, we use rotations $R_Z(w_i \varphi)$ controlled by $x_i$. 
This implements the weighted threshold gate. 

\iffalse
\section{Kolmogorov states}
In this section, we would like to construct a state that is hard to generate -- in fact, even hard to approximate -- by any $\mathsf{Post}\BQP$-circuit. 
In other words, we are looking for an $n$-qubit state $\ket{\psi} \not\in \mathsf{CircuitPost}\BQP_{n, 1- \frac{1}{poly(n)}}$. 
To achieve this we take inspiration from Kolmogorov complexity. For a formal definition of Kolmogorov complexity we refer to~\cite{Li:2008}, we give an informal definition instead:
\begin{definition}
Kolmogorov complexity $K(s)$ of a string $s$ is the shortest description of a computer program that outputs the string $s$ and then halts.
\end{definition}
\begin{example}
The $n$-bitstring, $s = 0^n$ of only zeros has Kolmogorov complexity $K(s)\leq \log(n) + \mathO(1)$ as the description: ``write $n$ times $0$'' requires logarithmic number of bits in the length $n$.
\end{example}

Note that any $n$-bit string $s$ has Kolomogorov complexity at most $K(s) \leq n + \mathO(1)$, as the program ``write $s$'' produces $s$ and then halts. 

It actually turns out that most bit-strings have large Kolmogorov complexity. 
Consider for instance the set of all bit strings with Kolomogorov complexity less than $n - 2$. 
We can calculate the fraction of these bit strings over all bit strings of length $n$:
$$\frac{|\{s| K(s) \leq n - 2\}|}{2^n} \leq \frac{|\{P| \text{$P$ a program of description length at most $n-2$}\}|}{2^n} = \frac{2^{n-1}-1}{2^n} \le \frac{1}{2}.$$

Instead of writing down bit strings, we are interested in quantum states that are hard to generate. 
Our strategy is to take a $2^n$-bit string with high Kolmogorov complexity and show that any quantum state that encodes this bit string, is hard to generate by a $\mathsf{PostBQP}$ circuit.
Let $z$ be a $2^n$-bit string with Kolomogorov complexity $K(z) \geq 2^n$. The following state encodes~$z$:
$$\ket{z} = \frac{1}{\sqrt{2^n}}\sum_{i=0}^{2^n-1} (-1)^{z_i} \ket{i},$$
where $z_i$ denotes the $i$-th bit of $z$. Now our goal is to argue that there exists no circuit that approximates this up to polynomial overlap. 

\begin{lemma}
\label{lem:exact_circuit_z}
There is no quantum circuit of depth poly$(n)$ that generates $\ket{z}$ exactly.
\end{lemma}
\begin{proof}
We follow a proof by contradiction strategy.
Assume there exist a circuit $C$ of polynomial depth that generates $\ket{z}$ exactly.
Now let $P$ be a classical program that simulates $C$ exactly and generates a $2^n$ long vector $v$ containing the exact amplitudes on all the basis states of $\ket{z}$. Now by outputting the sign of each basis state, this program $P$ generates the string $z$. 
As the description of $P$ is to simulate $C$, its Kolmogorov complexity is constant. 
Hence, the length of the description of this program is: 
$$desc(P) = \mathO(1) + desc(C) = \mathO(1) + \mathO(\mathrm{poly}(n))$$
which is much less than $2^n$ which is the Kolmogorov complexity of $z$ by assumption. Hence, such a circuit can not exist. 
\end{proof}

We can extend this proof by looking at equal weight superposition states with different phases:
$$\ket{\tilde{z}} = \frac{1}{\sqrt{2^n}}\sum_{i=0}^{2^n - 1} (-1)^{\tilde{z}_i}\ket{i},$$
and show that any such state prepared by a poly-depth quantum circuit must have a small overlap with $\ket{z}$.

\begin{lemma}
\label{lem:overlap_z_tildez}
Any state of the form $\ket{\tilde{z}}$ prepared by a poly-depth quantum circuit has small overlap with $\ket{z}$.
\end{lemma}
\begin{proof}
For any state $\ket{\tilde{z}}$, we can consider its overlap with $\ket{z}$, 
$$\braket{z}{\tilde{z}} = \frac{1}{2^n}(G  - B),$$
where $G = \#_i(z_i = \tilde{z_i})$ is the number of bits where $z$ and $\tilde{z}$ are the same, and $B = \#_i(z_i \neq \tilde{z_i})$ is the number of bits where $z$ and $\tilde{z}$ differ.
Now for every string $\tilde{z}$ there exists a string $R$ such that $\tilde{z}\oplus R = z$.
Note that the number of $\#1$'s in $R$ should exactly be the number of bits in which $z$ and $\tilde{z}$ differ. 
Given $\tilde{z}$ and $R$ there is a program that outputs $z$, hence it follows that:
\[
    K(z) \leq K(\tilde{z}) + K(R),
\]
where we know that $K(\tilde{z})$ is at most some polynomial in $n$ and describes a program that gives the quantum circuit that prepares the state.
For $R$ we know that there are $\binom{2^n}{B}$ total possible strings, hence $K(R)\leq \log(\binom{2^n}{B}) + \log(B)$ by giving the weight of the string and its index. 
This gives the following equation for the overlap
\[
2^n\leq \mathrm{poly}(n) + \log\left(\binom{2^n}{2^{n-1}(1 - \braket{z}{\tilde{z}})}\right) + \log\left(2^{n-1}(1 - \braket{z}{\tilde{z}}\right).
\]
As the last term is upper bounded by $n^2$, it can be absorbed in the polynomial term. 

Using the following inequality for binomials: $\binom{n}{\alpha n}\leq \frac{2^{H(\alpha) n}}{\sqrt{n \alpha (1 - \alpha)}}$, with $H(\alpha)$ being the Shannon entropy and $0 \leq \alpha \leq 1$, and setting $\alpha = \frac{ 1- \braket{z}{\tilde{z}}}{2}$, we obtain
\begin{equation}
    2^n\leq \mathrm{poly}(n) + 2^n H(\alpha) - \frac{n}{2} - \log(\alpha (1 - \alpha))/2 \leq \mathrm{poly}(n) + 2^n H(\alpha) - \log(\alpha (1 - \alpha))/2.
    \label{eq:intermediate_kolmogorove_inequality}
\end{equation}
By Lemma~\ref{lem:exact_circuit_z} we know that any poly-depth circuit fails to prepare $\ket{z}$ exactly.
Any state $\ket{\tilde{z}}$ that is not exactly $\ket{z}$ has overlap at most $1 - \frac{1}{2^{n}}$, by having at least one bit wrong. 
Therefore, the last term is upper bounded by a polynomial in $n$. 

Similarly, we can bound $H(\alpha)$ by taking the Taylor series expansion of $\log(x)$ to obtain $H(\frac{ 1- \braket{z}{\tilde{z}}}{2}) \leq 1 - \frac{1}{4}\braket{z}{\tilde{z}}^2$. 
Filling this inequality in in Eq.~\eqref{eq:intermediate_kolmogorove_inequality} gives a bound on $\braket{z}{\tilde{z}}$:
\[
\braket{z}{\tilde{z}} \leq \sqrt{\frac{\text{poly}(n)}{2^n}}
\]
which is exponentially small.
\end{proof}

Let $S \subset [2^n]$ denote a subset of all bit strings of length $n$, then define the state
\[
    \ket{\tilde{z}_S} = \frac{1}{\sqrt{|S|}}\sum_{i \in S} (-1)^{\tilde{z}_i}\ket{i}, 
\]
as a subset state with $\tilde{z}_i$ being the $i$-th bit of some string $\tilde{z}$. 

\begin{lemma}
Any state of the form $\ket{\tilde{z}_S}$ prepared by a poly-depth quantum circuit has small overlap with $\ket{z}$.
\end{lemma}
\begin{proof}
We can proof this by splitting into two cases based on the number of states in the subset: (i) small subset $|S|\leq 2^{3n/4}$ and (ii) large subset $|S| > 2^{3n/4}$.\\

\noindent\textbf{Case (i):} In this case we can calculate the overlap explicitly,
\[
 \braket{z}{\tilde{z}_S} = \frac{1}{\sqrt{2^n |S|}}(G - B),
\]
because the weight of $\ket{\tilde{z}_S}$ is only distributed on the basis states $i \in S$ we have that $G + B = |S|$, where $G$ again denotes the number of `good' or agreeing bits, and $B$ the number of `bad' or disagreeing bits.
Setting $G = |S|$ and $B = 0$, we upper bound this expression by:
\[
    \braket{z}{\tilde{z}_S} \leq \frac{|S|}{\sqrt{2^n |S|}} = \sqrt{\frac{|S|}{2^n}} \leq \sqrt{\frac{2^{3n/4}}{2^n}} = \frac{1}{2^{n/8}},
\]
where in the second inequality we used that $|S| \leq 2^{3n/4}$ as by assumption.\\

\noindent\textbf{Case (ii):} Let $z_S$ be the string such that $z_{S,i} = z_i$ if $i \in S$ and $0$ otherwise. 
Then $z_{S^\perp} \oplus z_S = z$, where $S^\perp$ is the set of indices not in $S$ and has size $2^n - |S|$. 
As a result we have $2^n \leq K(z_S) + K(z_{S^\perp})$.
We can rewrite this expression to $2^n -  K(z_{S^\perp})\leq K(z_S)$, and lower bound $K(z_S)$ by using $K(z_{S^\perp}) \leq |S^\perp| + O(1)$:
\[
    K(z_S)\geq 2^n - 2^n + |S| - O(1) = |S| - O(1) \geq 2^{n/2}.
\]
From $z_S$ we can construct the quantum state $\ket{z_S} = \frac{1}{\sqrt{|S|}}\sum_{i \in S}(-1)^{z_i}\ket{i}$, which allows us to bound the overlap between $\ket{z}$ and $\ket{\tilde{z}_S}$:
\[
    \braket{z}{\tilde{z}_S}\leq \braket{z_S}{\tilde{z}_S},
\]
due to the fact that all support of $\ket{z_S}$ is on $S$ and it has no support on the rest. But now this is the same situation as Lemma~\ref{lem:overlap_z_tildez} except that $K(z_S)\geq 2^{n/2}$ instead of $2^{n}$. By following the same steps as Lemma~\ref{lem:overlap_z_tildez} we get:
\[
 \braket{z}{\tilde{z}_S}\leq \braket{z_S}{\tilde{z}_S}\leq \sqrt{\frac{\text{poly}(n) + n^2}{2^{n/2}}}
\]
\end{proof}

The next step in the proof is to allow for non-uniform non-negative real amplitudes:
\[
\ket{\tilde{z}_A} = \sum_{i=0}^{2^n - 1} (-1)^{\tilde{z_i}}\alpha_i \ket{i}.
\]

\begin{lemma}
Any state of the form $\ket{\tilde{z}_A}$ created by a poly-depth circuit has small overlap with $\ket{z}$.
\end{lemma}

\begin{proof}
First we show that any basis state $\ket{i}$ with amplitude $\alpha_i\geq \frac{1}{2^{n/4}}$ gives a negligible contribution to the overlap.
Let $S_>$ be the set of indices such that $\alpha_i\geq \frac{1}{2^{n/4}}$, note that the size of $|S_>|$ is at most $2^{n/2}$ due to normalization. \marten{continue this...}
Let $\ket{\tilde{z}_{A|S_>}}$ be the state $\ket{\tilde{z}_{A}}$ restricted to the subset $S_>$, then the contribution of this part of the state to the overlap is upper bounded by
\[
\braket{z}{\tilde{z}_{A|S_>}} \leq \sum_{i\in S_>} \frac{\alpha_i}{2^{n/2}} \leq \frac{2^{n/2}}{2^{n/2}\cdot2^{n/4}} = \frac{1}{2^{n/4}}.
\]
The first inequality comes from the assumption that for all bits in $\tilde{z}_i$ and $z$ agree, and hence, every amplitude contributes positively to the overlap.
For the second inequality, we assumed that all $\alpha_i$ are of minimal value and have the same size. 
This assumption is valid, as we can only increase an $\alpha_i$ by decreasing other $\alpha_j$'s or by decreasing the size of $|S_>|$. In both cases, the value of the entire sum also decreases.

\noindent Second, we show that any basis state $\ket{i}$ with amplitude $\alpha_i \leq \frac{1}{2^{n}}$ gives a negligible contribution to the overlap. 
Let $S_<$ be the set of indices such that $\alpha_i \leq \frac{1}{2^{n}}$, note that the size of $|S_<| \leq 2^{n} - 1$ as there are $2^n$ possible basis states and only all but one can be lower than $ \frac{1}{2^{n}}$ due to normalization. \niels{Should it be in inequality instead of $\leq$?} 
Let $\ket{\tilde{z}_{A|S_<}}$ be the state restricted to the subset $S_{<}$, then the contribution of these amplitudes is given by the overlap between this state and $\ket{z}$:
\[
    \braket{z}{\tilde{z}_{A|S_<}} \leq \sum_{i \in S_<} \frac{\alpha_i}{2^{n/2}} \leq  \frac{|S_{<}|}{2^{n} 2^{n/2}} \leq \frac{2^n}{2^n 2^{n/2}} = \frac{1}{2^{n/2}}.
\]
\marten{this trick below does not work yet! Try to figure out a different method?}
The remaining amplitudes can be discretized into small buckets of increasing size:
\[
    [2^{-n + \frac{i}{n}}, 2^{-n + \frac{i + 1}{n}}]_{i = 0}^{i = \frac{3n^2}{4} } 
\]
\end{proof}


\subsection{Attempt 2) different type of state}

Let $z$ be a string of length $|z| = N (3+\epsilon) n$, where $N = 2^n$, and let it have maximal Kolmogorov complexity $K(z)\geq |z|$. We consider $z$ as a string consisting of $N$ sub-strings of length $(3+\epsilon)n$ labeld by an inxed $i$: $z = [z_0, z_1 \dots z_i \dots, z_{N-1}]$. Consider the following state
\[
    \ket{z} = \frac{1}{\sqrt{N}}\sum_{i=0}^{N-1} \ket{i}\ket{z_i},
\]
of $(4+\epsilon)n$ qubits. We will show that any state $\ket{\psi}$ created by a circuit of $\mathO(poly(n))$ depth has negligible overlap with $\ket{z}$. Note that we can write any state $\ket{\psi}$ as
\[
    \ket{\psi} = \sum_{i = 0}^{N-1} \alpha_i \ket{i}\ket{\phi_i}
\]
where $\sum_i |\alpha_i|^2 = 1$ and $\ket{\phi_i}$ is some state, which in turn can be written as
\[
    \ket{\phi_i} = \sum_j \beta_{i,j} \ket{j}.
\]
This gives the following formula for the overlap between $\ket{z}$ and $\ket{\psi}$
\begin{align*}
    \braket{\psi}{z} = \sum_i \frac{\alpha_i}{\sqrt{N}} \braket{\phi}{z_i} = \sum_i \frac{\alpha_i}{\sqrt{N}} \sum_j \beta_{i,j} \braket{j}{z_i} = \sum_i \frac{\alpha_i  \beta_{i,z_i}}{\sqrt{N}},
\end{align*}
where $\beta_{i,z_i}$ is the amplitude corresponding to the part of $\phi_i$ where the bit string $\ket{j}$ is equal to $z_i$. Using the Cauchy-Schwartz inequality we can bound this by:
\begin{align}
\label{eqn:overlap_z_psi}
 |\braket{\psi}{z}| \leq  \frac{1}{\sqrt{N}} \sqrt{(\sum_{i} |\alpha_i|^2) (\sum_i |\beta_{i,z_i}|^2)} = \frac{1}{\sqrt{N}} \sqrt{\sum_i |\beta_{i,z_i}|^2}.
\end{align}
\begin{lemma}
\label{lem:bound_num_beta_big}
Let $\ket{\psi}$ be a state created by a circuit $C$ of depth $p(n)$. At most $\frac{4 p(n)}{\epsilon}$ of the $\beta_{i,z_i}$ are bigger than $\frac{1}{N}$.
\end{lemma}
\begin{proof}
First assume that for a particular $\ket{\phi_i}$ the exact size of $\beta_{i,z_i}$ is known. Than for this $\ket{\phi_i}$ there are at most $\frac{1}{|\beta_{i,z_i}|^2}$ possible basis states that have exactly that amplitude. This also holds when a lower bound on $\beta_{i,z_i}$ is known. This means that given $i$ and an index over the basis states with size $\beta_{i,z_i}$ one can recover $z_i$ from the state $\ket{\psi}$. Note that this can be done by doing exact classical simulation, which requires exponential space and time, but the description of the program is very short, given by the poly-depth circuit for creating $\ket{\psi}$. For every $i$, where $\beta_{i,z_i}$ is bigger than $\frac{1}{N}$, we can use some additional bits to recover $z_i$ efficiently. The number of bits required are:
\[
    K(i) + K(\log(\frac{1}{|\beta_{i,z_i}|^2})) \leq n + \log(N^2) = 3n.
\]
Note that any $z_i$ consists of $(3 + \epsilon)n$ bits, hence for any $\beta_{i,z_i}\geq \frac{1}{N}$ we win $\epsilon n$ bits for the description of $z_i$. Note that the description of circuit $C$ is at most $K(C)\leq 4n p(n)$ as there are at most that many gates.  If more than $\frac{4n p(n)}{n \epsilon}$ of the $\beta_{i,z_i}\geq \frac{1}{N}$, then $C$ together with the description of which $\beta_{i,z_i}$'s would be a true compression of $z$ which by the assumption $K(z)\geq |z|$ is not possible. 
\end{proof}

\begin{theorem}
Any state $\ket{\psi}$ created by a circuit of depth $p(n)$, with $p$ some polynomial, has overlap with the state $\ket{z}$ of at most:
\[
    |\braket{\psi}{z}|\leq \frac{4 p(n)}{\epsilon \sqrt{N}} + \frac{1}{N}
\]
\end{theorem}
\begin{proof}
By Lemma~\ref{lem:bound_num_beta_big} we know that at most $\frac{4 p(n)}{\epsilon}$ of the $\beta_{i,z_i}$ are bigger than $\frac{1}{N}$. This lemma combined with Equation~\ref{eqn:overlap_z_psi} gives:
\begin{align*}
     |\braket{\psi}{z}| \leq \frac{1}{\sqrt{N}} \sum_{i} \beta_{i,z_i} \leq  \frac{1}{\sqrt{N}} (\frac{4 p(n)}{\epsilon}) +  \frac{1}{\sqrt{N}} \sqrt{\frac{N}{N^2}} = \frac{4 p(n)}{\epsilon \sqrt{N}} + \frac{1}{N}.
\end{align*}
In the first inequality we used that all the $\beta_{i,z_i}$ that are  bigger than $\frac{1}{N}$ are upper bounded by $1$ and the rest of the $\beta_{i,z_i}$  are upperbounded by $\frac{1}{N}$. 
\end{proof}
\fi