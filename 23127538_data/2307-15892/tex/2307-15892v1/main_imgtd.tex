\documentclass[twoside,11pt]{article}

% Any additional packages needed should be included after jmlr2e.
% Note that jmlr2e.sty includes epsfig, amssymb, natbib and graphicx,
% and defines many common macros, such as 'proof' and 'example'.
%
% It also sets the bibliographystyle to plainnat; for more information on
% natbib citation styles, see the natbib documentation, a copy of which
% is archived at http://www.jmlr.org/format/natbib.pdf

\usepackage{jmlr2e}
%\usepackage[abbrvbib, preprint]{jmlr2e}

% Definitions of handy macros can go here

%% Choose your variant of English; be consistent
\usepackage[american]{babel}
% \usepackage[british]{babel}

%% Some suggested packages, as needed:
%\usepackage{natbib} % has a nice set of citation styles and commands
    % \bibliographystyle{plainnat}
%\bibliographystyle{ieee_fullname}
\renewcommand{\bibsection}{\subsubsection*{References}}
\usepackage{mathtools} % amsmath with fixes and additions
% \usepackage{siunitx} % for proper typesetting of numbers and units
\usepackage{booktabs} % commands to create good-looking tables
\usepackage{tikz} % nice language for creating drawings and diagrams

%% Provided macros
% \smaller: Because the class footnote size is essentially LaTeX's \small,
%           redefining \footnotesize, we provide the original \footnotesize
%           using this macro.
%           (Use only sparingly, e.g., in drawings, as it is quite small.)
\usepackage[utf8]{inputenc} % allow utf-8 input
\usepackage[T1]{fontenc}    % use 8-bit T1 fonts
% \usepackage[colorlinks=true]{hyperref}      % hyperlinks
% \usepackage{url}            % simple URL typesetting
%\usepackage{booktabs}       % professional-quality tables  
\usepackage{amsmath}
\usepackage{bbm}
% blackboard math symbols
\usepackage{nicefrac}       % compact symbols for 1/2, etc.
\usepackage{microtype}      % microtypography
\usepackage{xcolor}         % colors
\usepackage[export]{adjustbox}
\usepackage{wrapfig}
\usepackage{caption}
\usepackage{subcaption}
\usepackage{enumitem}
% \usepackage{apptools}
% \AtAppendix{\counterwithin{lemma}{section}}
\usepackage{algorithm}
\usepackage{algpseudocode}

\newtheorem{thm}{Theorem}
\newtheorem{lem}[theorem]{Lemma}
\newtheorem{assumption}{Assumption}

\renewcommand{\SS}{\mathcal{S}}
\renewcommand{\AA}{\mathcal{A}}
\newcommand{\PP}{\mathbbm{P}}
\newcommand{\RR}{\mathcal{R}}
\newcommand{\EE}{\mathbbm{E}}
\newcommand{\EED}{\mathbbm{E}_{\mathcal{D}}}
\newcommand{\Efgrad}{\mathbbm{E}\norm{f'(x_t)}^2}
\newcommand{\normg}{\norm{g_t(x_t)}^2}
\newcommand{\norm}[1]{\left\|#1\right\|}
\newcommand{\tr}{\top}
\newcommand{\avg}{\bold{avg}_m(x_t)}
\newcommand{\avgx}{\bold{avg}_m(x)}
\newcommand{\avgt}{\bold{avg}_m(\theta_t)}
% Heading arguments are {volume}{year}{pages}{submitted}{published}{author-full-names}
\newcommand{\oneexptd}{R1-GTD\,}

%\jmlrheading{1}{2000}{1-48}{4/00}{10/00}{Marina Meil\u{a} and Michael I. Jordan}

% Short headings should be running head and authors last names

\ShortHeadings{A new Gradient TD algorithm with only one step-size}{**}
\firstpageno{1}

\begin{document}

\title{A new Gradient TD Algorithm with only One Step-size:\\  Convergence Rate Analysis using $L$-$\lambda$ Smoothness
}
% \title{A New Gradient TD Algorithm with Only One Step-size:\\
% Baird Counterexample Mystery is Solved
% }

\author{\name Hengshuai Yao\email hengshuai.yao@sony.com \\
       \addr SonyAI
       }

\editor{***}

\maketitle

\begin{abstract}

The Fast Reciprocal Square Root Algorithm is a well-established approximation technique consisting of two stages: first, a coarse approximation is obtained by manipulating the bit pattern of the floating point argument using integer instructions, and second, the coarse result is refined through one or more steps, traditionally using Newtonian iteration but alternatively using improved expressions with carefully chosen numerical constants found by other authors. The algorithm was widely used before microprocessors carried built-in hardware support for computing reciprocal square roots. At the time of writing, however, there is in general no hardware acceleration for computing other fixed fractional powers. This paper generalises the algorithm to cater to all rational powers, and to support any polynomial degree(s) in the refinement step(s), and under the assumption of unlimited floating point precision provides a procedure which automatically constructs provably optimal constants in all of these cases. It is also shown that, under certain assumptions, the use of monic refinement polynomials yields results which are much better placed with respect to the cost/accuracy tradeoff than those obtained using general polynomials. Further extensions are also analysed, and several new best approximations are given.

\end{abstract}


\begin{keywords}
  Off-policy learning, Gradient-based Temporal Difference learning, The NEU objective, MSPBE, SGD, Convergence rate analysis, Batch size effect, expected smoothness, linear convergence rate %Baird Counterexample 
\end{keywords}


\section{Introduction}
Current quantum hardware is unable to carry out universal quantum computations due to the buildup of errors that occur during the computation. 
The magnitude of the individual error is currently above the value that the Threshold Theorem requires in order to kick-start quantum error correction and fault-tolerant quantum computation~\cite[Section 10.6]{nielsen_chuang_2010}. 
Although the experimentally achieved fidelity rates are promising and the error bounds are inching closer to the required threshold, we will have to work for the foreseeable future with quantum hardware with errors that build-up during the computation.  This implies that we can only do a limited number of steps before the output of the computation has become completely uncorrelated with the intended one.

For fault-tolerant quantum computing, we repeat four steps: 
1) We apply a number of single and two-qubit quantum gates, in parallel whenever possible; 
2) We perform a syndrome measurement on a subset of the qubits; 
3) We perform fast classical computations to determine which errors have occurred and how to correct them; 
and, 4) We apply correction terms based on the classical computations.
We then repeat these four steps with a next sequence of gates. 
These four steps are essential to fault-tolerant quantum computing. 


The starting point of this work is to use the four steps outlined above, not to carry out error correction and fault-tolerant computation, but to enhance short, constant-depth, {\em uncorrected} quantum circuits that perform single qubit gates and {\em nearest-neighbor} two qubit gates. 
Since in the long run we will have to implement error-correction and fault-tolerant computation anyhow, and this is done by such a four-step process, why not make other use of this architecture? Moreover, on some of the quantum hardware platforms, these operations are already in place.
Embracing this idea we naturally arrive at the question: what is the computational power of \textit{low-depth} quantum-classical circuits organized as in the four steps outlined above? 
We thus investigate circuits that execute a small, ideally constant, number of stages, where at each stage we may apply, in parallel, single qubit gates and {\em nearest-neighbor} two qubit gates, followed by measurements, followed by low-depth classical computations of which the outcome can control quantum gates in later stages. 
It is not clear, at first, whether such circuits, especially with constant depth, can do anything remotely useful. 
But we will see that this is indeed the case: many quantum computations can be done by such circuits in constant depth. 
By parallelizing quantum computations in this way, we improve the overall computational capabilities of these circuits, as we do not incur errors on qubits that are idle, simply because qubits are not idle for a very long time. 
Furthermore, reducing the depth of quantum circuits, at the cost of increasing width, allows the circuit to be run faster even if errors occur.

The first usage of such a four-step layout, not to do error correction, but to perform computations, can be found in the paradigm of measurement-based quantum computing~\cite{gottesman1999demonstrating,raussendorf2001one,jozsa2006introduction,clark2007generalised}: 
A universal form of quantum computing where a quantum state is prepared and operations are performed by measuring qubits in different bases, depending on previous measurements and intermediate measurements.

\citeauthor{PhamSvore2013} were the first to formalize the four-step protocol for performing computations~\cite{PhamSvore2013}. They included specific hardware topologies by considering two-dimensional graphs for imposing constraints on qubit interactions. In their model, they develop circuits for particularly useful multi-qubit gates, including specifying costs in the width, number of qubits, depth, number of concurrent time steps, size, and total number of non-Identity operations.
As a result, they find an algorithm that factors integers in polylogarithmic depth.
\citeauthor{Browne:2011} showed that the main tool in the work by \citeauthor{PhamSvore2013}, the fan-out gate, can also be replaced by additional log-depth classical computations in the measurement-based quantum computing setting~\cite{Browne:2011}.

More recently, \citeauthor{Cirac:2021} introduced a scheme to implement unitary operations involving quantum circuits combined with Local Operations and Classical Communication ($\mathsf{LOCC}$) channels: $\mathsf{LOCC}$-assisted quantum circuits~\cite{Cirac:2021}. Similarly to the four-step scheme we just described, they allow for a short depth circuit to be run on the qubits, followed by one round of $\mathsf{LOCC}$, in which ancilla qubits are measured and local unitaries are applied based on the measurement outcomes. They show that in this model any 1D transitionally invariant matrix-product state (MPS) with fixed bond dimension is in the same phase of matter as the trivial state. Similar ideas can be found in~\cite{TVV_NonAbelianTopologicalOrder_2022, tantivasadakarn2021long}.

In this work, we introduce a new model, called \textit{Local Alternating Quantum-Classical Computations} ($\LAQCC$). In this model we alternate between running quantum circuits (constrained by locality), ending in the measurement of a subset of qubits, and fast classical computations based on the measurement results. The outcome of the classical computations are then used to control future quantum circuits. We allow for flexibility in this model, by giving different constraints to the power of both the quantum circuits and the classical circuits as well as the number of alternations between them. 
Most attention will be given to $\LAQCC$ containing quantum circuits of constant depth, classical circuits of logarithmic depth and at most a constant number of alternations between them. 
Any circuit constructed in this model is considered to be of constant depth. 
We restrict ourselves to logarithmic depth classical computations, as this is the first natural and non-trivial extension beyond constant-depth classical computations. 
Constant-depth classical computations do however also have an equivalent constant-depth quantum implementation.

The definition of $\LAQCC$ sharpens the original definition of \citeauthor{PhamSvore2013} by adding constraints to the intermediate classical computations. This allows us to bound the power of $\LAQCC$ from above. 

The main result of \citeauthor{Cirac:2021}, that 1D translational invariant MPS with fixed bond dimension can be prepared by $\mathsf{LOCC}$-assisted circuits, relies on local symmetries of the MPS. These symmetries allow them to prepare local states (on a constant number of qubits) and glue them together by doing one round of the appropriate entangling measurement and corrections, after which they run a round of local unitaries to get the desired result. This general scheme for preparing states that exhibit an MPS description with the appropriate local symmetries requires only geometrically local unitaries and one round of measurement and corrections an therefore is accessible in $\LAQCC$. Studying different local symmetries, known as Symmetry Protected Topological (SPT) phases of matter, to find measurement-based constant depth circuits for states is a broad ongoing field of research~\cite{TVV_NonAbelianTopologicalOrder_2022, tantivasadakarn2021long, smith2023deterministic}. 
All these schemes have a $\LAQCC$ implementation.

%$\LAQCC$-circuits also exist for general schemes of preparing local states, based on the local tensors, and gluing them together using one round of entangled measurement and corrections, based on the local symmetry. 
%The main result of \citeauthor{Cirac:2021}, that 1D translational invariant MPS with fixed bond dimension can be prepared by $\mathsf{LOCC}$-assisted circuits, relies heavily on local symmetries of the MPS and as a result also has an equivalent $\LAQCC$ implementation. 
%The corrections applied after the measurement round are local unitaries depending on the local symmetries of the MPS. 

 

%This general scheme of preparing local states, based on the local tensors, and gluing it together by doing one round of entangled measurement and corrections, based on the local symmetry, is accessible in $\LAQCC$.
Note however that \citeauthor{Cirac:2021} also suggest a circuit for the $W$-state.
This circuit uses sequentially and dependent measurement-based corrections of the ancilla qubits. 
These dependent measurements translate to sequential alternations between the quantum and classical circuits and therefore increase the total depth to linear depth, exceeding the constant-depth constraints imposed by $\LAQCC$-circuits. 

We study the power of the $\LAQCC$ model with respect to state preparation, showing that even with only constant quantum-depth and logarithmic classical depth it remains possible to prepare states with long-range entanglement.
Another surprising result is that it is unlikely that $\LAQCC$ circuits are classically simulatable. We show that any instantaneous quantum polynomial-time (IQP) circuit~\cite{Bremner2010,Shepherd2009} has an $\LAQCC$ implementation.
Classical simulation of IQP circuits implies the collapse of the polynomial hierarchy to the third level, which is not believed to be true~\cite{Bremner2017}. Therefore, we expect that $\LAQCC$ circuits are unlikely to be classically simulatable. We bound the power of $\LAQCC$ by showing that it is contained in $\QNC^1$, the class of polynomial-size, log-depth circuits.

Next, we also study the power that intermediate classical calculations can add to quantum computations, by considering a new model that alternates between polynomially many polynomial-depth quantum circuits and unbounded classical computations
We study this model by doing a complexity theoretical analysis, where we draw inspiration from the notions of complexity given by \citeauthor{RosenthalYuen:2022}, \citeauthor{MetgerYuen:2023}, and \citeauthor{Aaronson:2004}.
All three complexity notions are based on the notion of state preparation, instead of more traditional definition of complexity such as the decidability of a computational problem. 
The first two consider classes based on sequences of quantum states preparable by a polynomial-sized quantum circuit, where the circuits are uniformly generated by a computational class, for instance, the class $\mathsf{PSPACE}$, which results in the complexity class $\mathsf{StatePSPACE}$~\cite{RosenthalYuen:2022,MetgerYuen:2023}.
The third notion considers a relative complexity, where the complexity is measured between two given states, and is measured by the number of gates, from a given gate-set, required to transform one state in another state~\cite{Aaronson:2004}. 
For our definition of state preparation complexity, we drop the uniformity constraint from~\cite{RosenthalYuen:2022,MetgerYuen:2023} and define a class as $\mathsf{StateX}$, which refers to states preparable by circuits of type $\mathsf{X}$. 
As an example, if $\mathsf{X} = \QNC^0$, this results in the class $\mathsf{StateQNC^0}$, which is the set of states preparable from the $\ket{0}^n$ state by poly-size constant-depth circuits. 
This notion is similar to the relative complexity from~\cite{Aaronson:2004}, where one state is the  $\ket{0}^n$ state and instead of counting the number of gates we consider the set of states preparable by a fixed number of gates. Using this notion of complexity we show that any state preparable by an $\LAQCC^*$ circuit is also preparable by a $\mathsf{PostQPoly}$ circuit, the class of circuits of polynomial depth with an additional post-selection gate. 

All Clifford circuits have a constant-depth $\LAQCC$ implementation, implying that any stabilizer state can be implemented by a constant-depth $\LAQCC$ circuit, see Section~\ref{sec:clifford_circuits} for a proof of this statement. 
Efficient circuits for stabilizer states have been known already through measurement-based quantum computing. Therefore this paper focuses on the preparation of non-stabilizer states, and as a surprising result we find novel constant-depth protocols for four very natural classes of non-stabilizer states.
Despite the extensive research into these four classes of non-stabilizer states and the many applications of them, no efficient constant- or low-depth state preparation protocols are known yet. We specifically consider these four classes as they are all often used as initial states in other algorithms.

The first state is a uniform superposition over an arbitrary number of states. 
This state finds applications in many quantum algorithms, as they often start with a uniform superposition over multiple states. 
This superposition is often achieved by applying Hadamard gates to every qubit due to its simplicity to prepare. 
Yet, the analysis of many algorithms, such as Shor's algorithm~\cite{Shor:1997}, would benefit from a different initial superposition. 
The circuit to prepare the uniform superposition over an arbitrary number of states uses an exact version of Grover search as a subroutine, that turns a probabilistic circuit, with a known constant probability of success, into a deterministic circuit. 
We use the circuit for preparing a uniform superposition over an arbitrary number of states as a subroutine in the next two quantum state preparation protocols. 

The second state is the $W$-state, the uniform superposition over all computational basis states of Hamming-weight~$1$, a natural long-ranged entangled state that displays a fundamentally nonequivalent type of entanglement from the Greenberger–Horne–Zeilinger state~\cite{WState:2000}, for which $\LAQCC$-type constant-depth circuits were previously known~\cite{PhamSvore2013, Cirac:2021}. 
The $W$-state is often used as benchmark for new quantum hardware~\cite{Haffner2005,Neeley2010,GarciaPerez:2021}. 
A novel way to prepare the $W$-state therefore gives a new way to benchmark different quantum devices with each other. 
A circuit for preparing the $W$-state was given in~\cite{Cirac:2021}, but this implementation requires sequentially alternating measurements followed by local unitaries, which in the $\LAQCC$ model is not considered to be of constant depth. 
We improve this protocol by giving an $\LAQCC$ implementation of the $W$-state, based on a compress-uncompress method that links the one-hot and binary encoding of integers.

The third state considered is the Dicke state, a generalization of the $W$-state, a superposition over all computational basis states with Hamming-weight $k$~\cite{Dicke:1954}. 
Dicke states have relevance in various practical settings.
For instance, for quantum game theory~\cite{zdemir2007}, quantum storage~\cite{Bacon_Compress:2006,Plesch:2010}, quantum error correction~\cite{ouyang2014permutation}, quantum metrology~\cite{toth2012multipartite}, and quantum networking~\cite{prevedel2009experimental}. 
Dicke states have been used as a starting state for variational optimization algorithms, most notably Quantum Alternating Operator Ansatz (QAOA)~\cite{Hadfield2019}, to find solutions to problems such as Maximum k-vertex Cover~\cite{Brandhofer2022,cook2020quantum}.
The ground states of physical Hamiltonians describing one-dimensional chains tend to show a resemblance to Dicke states such as states resulting from the Bethe ansatz, making them an ideal starting state when investigating the ground state behavior of these Hamiltonians~\cite{TDL_BetheAnsatzDerivation:2010,B_ExcitedStateQuantumPhaseTransitions:2013,DickeTransitions:2021}. 
For instance, the algorithm by \citeauthor{van2021preparing}, who give an algorithm to prepare the Bethe ansatz eigenstates of the spin-1/2 XXZ spin chain, starts by first preparing a Dicke state~\cite{van2021preparing}. 
A Dicke-state preparation protocol based on the compress-uncompress methodology used in the $W$-state furthermore finds applications in entanglement distillation, where the entanglement of a large state is concentrated on only a few qubits. 
Efficient deterministic circuits for preparing Dicke states have been proposed by \citeauthor{bartschi2019deterministic}~\cite{bartschi2019deterministic, bartschi2022deterministic_short_depth}. 
They provide a quantum circuit of depth $\mathO(k \log(\frac{n}{k}))$, allowing arbitrary connectivity, to prepare a Dicke state, which they conjecture to be optimal when $k$ is constant. 
In this work, we provide a constant-depth $\LAQCC$ circuit below their conjectured bound already for constant $k$. 
However, this does not directly disprove their conjecture, as we allow for intermediate measurements and classical computations. 
More significantly, we even construct constant-depth $\LAQCC$ circuits for $k = \mathO(\sqrt{n})$ greatly improving their bound.
This construction extends the compress-uncompress method for the $W$-state combined with additional subroutines. 

We continue with a log-depth state preparation protocol for the Dicke-state for arbitrary $k$. 
This protocol implements an efficient transformation between the factoradic number representation and the combinatorial number representation of a positive integer. 
The combinatorial number representation relates directly to the Dicke state. 
The provided efficient transformation between number representation systems might be of independent interest. 

We conclude by modifying our protocol for preparing a Dicke-state to a protocol that prepares quantum many-body scar states in constant-depth. 
These states have low entanglement and longer coherence times than states with similar energy density.
These characteristics make many-body scar states interesting to analyze and relevant within physics.
Many-body scar states appear for instance in the AKLT model~\cite{AKLT:1987,MRBAR:2018,MRB:2018} and different spin models~\cite{SI:2019,MOBFR:2020}.
Known methods for preparing these states have polynomial-depth~\cite{Gustafson:2023}, whereas our circuit has constant depth. 

% We conclude by studying the power that intermediate classical calculations can add to quantum computations. 
% In this study, we define a new model that relaxes constant-depth quantum circuits to polynomial depth quantum circuits, log-depth classical calculations to unbounded classical computations and a constant number of alternations to a polynomial number of alternations. 
% We call this model $\LAQCC^*$. 
% We study this model by doing a complexity theoretical analysis, where we draw inspiration from the notions of complexity given by \citeauthor{RosenthalYuen:2022}, \citeauthor{MetgerYuen:2023}, and \citeauthor{Aaronson:2004}.
% All three complexity notions are based on the notion of state preparation, instead of more traditional definition of complexity such as the decidability of a computational problem. 
% The first two consider classes based on sequences of quantum states preparable by a polynomial-sized quantum circuit, where the circuits are uniformly generated by a computational class, for instance, the class $\mathsf{PSPACE}$, which results in the complexity class $\mathsf{StatePSPACE}$~\cite{RosenthalYuen:2022,MetgerYuen:2023}.
% The third notion considers a relative complexity, where the complexity is measured between two given states, and is measured by the number of gates, from a given gate-set, required to transform one state in another state~\cite{Aaronson:2004}. 
% For our definition of state preparation complexity, we drop the uniformity constraint from~\cite{RosenthalYuen:2022,MetgerYuen:2023} and define a class as $\mathsf{StateX}$, which refers to states preparable by circuits of type $\mathsf{X}$. 
% As an example, if $\mathsf{X} = \QNC^0$, this results in the class $\mathsf{StateQNC^0}$, which is the set of states preparable from the $\ket{0}^n$ state by poly-size constant-depth circuits. 
% This notion is similar to the relative complexity from~\cite{Aaronson:2004}, where one state is the  $\ket{0}^n$ state and instead of counting the number of gates we consider the set of states preparable by a fixed number of gates. Using this notion of complexity we show that any state preparable by an $\LAQCC^*$ circuit is also preparable by a $\mathsf{PostQPoly}$ circuit, the class of circuits of polynomial depth with an additional post-selection gate. 

\paragraph{Summary of results}
\begin{itemize}
    \item We give a new definition of a computational model that captures the power of the four step process: applying a constant number of layers of one- and two-qubit gates; performing a syndrome measurement; perform a fast classical computation determining corrections; apply corrections. We call this model \emph{Local Alternating Quantum Classical Computations}, or $\LAQCC$ for short. In this model we bound the allowed quantum operations, intermediate classical calculations, and number of rounds separately. In Section~\ref{sec:LAQCC_model} we define this model and give a list of operations based on results from literature contained in this computational model. In some of these operations we explicitly use that we allow for multiple, but at most constant, rounds  of corrections.
    \item  We show show that there exist $\LAQCC$ circuits that can not be weakly simulated in Section~\ref{sec:IQP_in_LAQCC}. We further show that for every $\LAQCC$ circuit there exists a $\QNC^1$ circuit simulating it perfectly, in Section~\ref{sec:LAQCC_in_QNC1}.
    \item We introduce a new type computational complexity for preparing states and show that the extension of $\LAQCC$ where we allow a polynomial number of rounds and unbounded classical computation, is contained in $\mathsf{PostQPoly}$, the class of polynomial circuits with post-selection, in Section~\ref{sec:Complexity results}.
    \item We show a protocol to prepare the uniform superposition state of size $q$ in $\LAQCC$ using $\mathO(\ceil{\log_2(q)}^2)$ qubits in Section~\ref{sec:superposition_modulo_q}. 
    \item We show a protocol to prepare the $W_n$ state in $\LAQCC$ using $\mathO(n\log(n))$ qubits in Section~\ref{sec:W_state_in_LAQCC}.
    \item We show two ways of preparing the Dicke-$(n,k)$ state. The first method is in $\LAQCC$, works up to $k = \mathO(\sqrt{n})$, uses $\mathO(n^2\log(n))$ qubits, and is found in Section~\ref{sec:dicke:small_k}. The second method is in $\LAQCC\text{-}\mathsf{LOG}$ (an extension of $\LAQCC$ allowing for logarithmic number of alterations instead of constant), works for any $k$, uses $\mathO(\text{poly}(n))$ qubits, and is found in Section~\ref{sec:Dicke_in_LAQCC_LOG}. 
    \item We extend on our $\LAQCC$ method of generating Dicke-$(n,k)$ states for $k = \mathO(\sqrt{n})$ and show a protocol to generate many-body scar states for a particular Hamiltonian in $\LAQCC$ (Section~\ref{sec:many_body_scar}). 
\end{itemize}
Summarized in a table, we provide the following state generation protocols:
\begin{table}[htb]
\centering
\begin{tabular}{l|l|l|l}
\textbf{State description} & \textbf{Width} & \textbf{Depth} & \textbf{Implementation}\\
\hline 
Uniform superposition mod $q$: $\frac{1}{\sqrt{q}} \sum_{i = 0}^{q-1}\ket{i}$ & $\mathO(\ceil{\log^2 q})$ & $\mathO(1)$ & Section~\ref{sec:superposition_modulo_q}\\

$W$-state: $\frac{1}{\sqrt{n}}\sum_{i = 0}^{n-1}\ket{e_i}$ & $\mathO(n \log n)$ & $\mathO(1)$ & Section~\ref{sec:W_state_in_LAQCC}\\

Dicke-$(n,k)$, $k = \mathO(\sqrt{n})$: $\binom{n}{k}^{-1/2}\sum_{x \in \{0,1\}^n: |x| = k} \ket{x}$ &  $\mathO(n^2\log n)$ & $\mathO(1)$ 
&Section~\ref{sec:dicke:small_k}\\

Dicke-$(n,k)$: $\binom{n}{k}^{-1/2}\sum_{x \in \{0,1\}^n: |x| = k} \ket{x}$ & $\mathO(\text{poly}(n))$ & $\mathO(\log n)$ &Section~\ref{sec:Dicke_in_LAQCC_LOG}\\

QMBS: $\ket{S_k} = \frac{1}{k! \sqrt{\mathcal N(n,k)}}(Q^\dagger)^k \ket{\Omega}$ &  $\mathO(n^2\log n)$ & $\mathO(1)$  &  Section~\ref{sec:many_body_scar}
\end{tabular}
\caption{Summary of state preparation protocols given in this paper.}
\label{tab:sate_prep}
\end{table}
In the entry for the quantum many-body scar state $Q$ denotes the raising operator and $\mathcal N(n,k)=\binom{n-k-1}{k}$. 
Section~\ref{sec:many_body_scar} will provide more details on the variables and the implementation. 

\paragraph{Organization of the paper}
\noindent We first introduce relevant preliminaries in Section~\ref{sec:preliminaries}. 
In Section~\ref{sec:LAQCC_model} we formally define the class of Local Alternating Quantum-Classical Computations ($\LAQCC$). We also show that any Clifford circuit can be implemented in constant depth $\LAQCC$ (a result based on a result from measurement-based quantum computing~\cite{jozsa2006introduction}). 
This result allows us to give many useful multi-qubit gates and routines in Section~\ref{sec:gates_created_in_LAQCC}. 
Beyond that we show that constant depth $\LAQCC$ circuits are contained in $\QNC^1$ and that any $\mathsf{IQP}$ circuit has an $\LAQCC$ implementation.
We conclude this section with an analysis of a more powerful instantiation of $\LAQCC$ and show an inclusion with respect to the class $\mathsf{PostQPoly}$, which is the class of circuits of polynomial depth with one additional post-selection gate. 
In Section~\ref{sec:state_prep_in_LAQCC} we give $\LAQCC$ circuit implementations for preparing the uniform superposition over an arbitrary number of states, the $W$-state and the Dicke state up to $k = \mathO(\sqrt{n})$. We furthermore give a log-depth circuit implementation for preparing the Dicke state for any $k$. We conclude by showing a $\LAQCC$ circuit for generating many body scar states of a particular type of Hamiltonian.



\vspacebeforesection
\section{Background}
\label{sec:background}

In this section, we provide the necessary background information to ensure a comprehensive understanding of the attack described in this paper. We start with a description of the Distributed Hash Table (DHT) used by IPFS, followed by its content resolution mechanisms. We also detail techniques for network size estimation, necessary for our attack detection and mitigation mechanisms.

\vspacebeforesection
\subsection{IPFS DHT}
\label{sec:kad_dht}

We review the features of the Kademlia DHT~\cite{maymounkov2002kademlia} and its \texttt{libp2p} implementation~\cite{libp2p_github} that are the most relevant to our attack.
To participate in the DHT, each peer generates a public/private key pair and derives an identity $\peerid \in \{0,1\}^{256}$ as the hash of its public key.
Ideally, each peer generates a random key pair and, therefore, peer IDs are distributed uniformly and independently over the space $\{0,1\}^{256}$.
While honest nodes follow this rule, malicious nodes may generate and choose from an arbitrary number of key pairs.
Each peer maintains a routing table consisting of $m=256$ buckets.
The $i$-th bucket contains the addresses of up to $k=20$ peers whose peer IDs share a common prefix of exactly $i$ bits with the peer's own peer ID. 

%
A new participant node joins the IPFS network by contacting one of the hardcoded bootstrap nodes. This bootstrap node provides the new node with some initial peers allowing it to join the DHT. The new node uses this information to perform a walk through the DHT towards its own peer ID.
The walk allows to: \textit{(i)}~make sure that there is no other node in the network with the same ID; \textit{(ii)}~discover new peers and fill the newcomer's DHT routing table. At the same time, the newcomer establishes \bitswap~\cite{de2021accelerating} connections to a subset of encountered peers (usually around 300 of them). The core role of the \bitswap protocol is to enable bilateral content transfer and to play the role of a cache for recently-accessed content.

The main DHT operation $\Call{GetClosestPeers}{\key}$ returns the $k=20$ closest peers to $\key$. 
%
In Kademlia, the distance between two keys $x$ and $y$ in the key space is given by $x \oplus y \in \{0,...,2^{256}-1\}$, where $\oplus$ denotes the bitwise XOR operation on the keys; the resulting binary string is interpreted as an integer.
%
When a client wants to find the peers with IDs closest to $\key$, it sends a request to the $\alpha=3$ peers in its routing table whose peer IDs are closest to $\key$. Each of these peers returns the $k$ closest peers to $\key$ in its own routing table and the addresses of these peers. 
%
The client again sends a request to the $\alpha$ peers closest to $\key$, among peers in its routing table and those whose addresses it just received. This process repeats until the client does not find any more peers closer to $\key$.
Due to network churn and imperfect routing tables, we observed in our experiments that successive calls to $\Call{GetClosestPeers}{\key}$ do not always return the same set of $k=20$ peers (we provide more details in \Cref{sec:evaluation}, \Cref{fig:20closest}). This is an important limitation affecting our attack.

\vspacebeforesection
\subsection{Content Resolution in IPFS}
\label{sec:ipfs}

IPFS is a content-centric network.
It allows its participant to request files without specifying their location. 
%
Content is indexed by content IDs $\cid \in \{0,1\}^{256}$ that are derived from a hash of that content.
Both peer IDs and CIDs are used as keys in the DHT.
Each node can play the role of a \provider, \downloader, or \resolver. 
The process of content advertisement and resolution is illustrated in \Cref{fig:add_get_provider}.

%
When a \provider wishes to publish content with a given $\cid$ on IPFS, it creates a \emph{provider record} that contains $cid$ and the \provider's address.
During a $\Call{Provide}{\cid}$ operation, the \provider first uses $\Call{GetClosestPeers}{\cid}$ to locate the $k=20$ peers with their peer IDs closest to $\cid$, 
%
and then sends them a $\mathsf{PutProvider}$ message including the provider record (\Cref{fig:add_get_provider}(a)).
We call the peers that hold provider records for $\cid$ the \emph{resolvers} for $\cid$.

Each CID can have several \providers. In fact, by default, each IPFS client becomes a provider for each piece of content it downloads for a fixed amount of time (12h, 24h, or 48h depending on the client version or custom configuration). As a result, the system provides an auto-scaling feature with supply automatically rising with demand.

%
When a \downloader wishes to fetch a piece of content, it first sends a request to all its \bitswap peers. If none of them has the content, the \downloader uses the DHT-based resolution system. We stress that the \bitswap protocol plays the supporting role of a cache in the dissemination of popular files. However, the mechanism does not provide reliable content resolution, in particular for new or less popular content. %

When \bitswap unstructured search fails, the \downloader resolves $\cid$ using $\Call{FindProviders}{\cid}$. This operation uses a DHT walk identical to that of $\Call{GetClosestPeers}{\cid}$ to find $k$ \resolvers but also queries encountered nodes for a provider record for $\cid$ (\Cref{fig:add_get_provider}(b)). The process terminates when either 20 \providers have been found, or all \resolvers have been asked. Querying all encountered nodes (\ie, not only the designated \resolvers) is useful because some of the encountered nodes may have a provider record in their cache.
%

Upon receiving a provider record, the client connects to the address specified in the provider record to retrieve the actual content (\Cref{fig:add_get_provider}(c)).
Provider records are not authenticated, and therefore malicious \providers may respond with incorrect provider records (or may not respond at all). However, the integrity of the content is preserved because the hash of the retrieved content can be verified against its $\cid$.
%


%

\input{img/add_get_provider.tex}

\vspacebeforesection
\subsection{Network Size Estimator}
\label{sec:netsize}

The number of nodes in a decentralized system is generally unknown due to the avoidance of centralized membership management.
This number is nonetheless useful for optimizations, deciding on individual node configurations, or security mechanisms.
Various methods were proposed for the decentralized estimation of unstructured and structured networks~\cite{eli-sohl-dht-size-estimation,kostoulas2005decentralized, manku2003symphony}.
We use in this work a mechanism developed initially by Protocol Labs as part of a mechanism for decreasing the latency of publishing content in IPFS~\cite{network-size-estimation-notion,network-size-estimation-github-pr}.

%
%
%
%
%
%
%
%
%
%

Each node in the DHT refreshes its routing table periodically (every $10$ minutes in \texttt{libp2p}). 
For this, the node samples $m$ random keys (one for each bucket of its routing table)
%
and queries the DHT to obtain the $k=20$ closest peer IDs to each key.
Using these, the node then computes the average distance between each one of these keys $\key_j$ for $j=1,\dots,m$ and their $i$-th closest peer ID for $i=1,...,k$ (with $m=256$ and $k=20$).
\begin{equation}
    \label{equ:avg-dist}
    \overline{D}_i = \frac{1}{m} \sum_{j=1}^m \operatorname{dist}(\key_j, \peerid_{j}^{(i)})
\end{equation}
where $\peerid_{j}^{(i)}$ is the $i$-th closest peer ID to $\key_j$.
With $N$ peers in the DHT and peer IDs uniformly distributed in the hash space, the expected distance between a $\key$ and its $i$-th closest peer ID is $\frac{2^{256}i}{N+1}$. The node then runs a least square regression to compute the value of $N$ for which the expected distances best fit the empirical average distances, \ie,
\begin{equation}
    \label{equ:netsize-least-squares}
    \hat{N} = \arg\min_{N} \sum_{i=1}^k \left(\overline{D}_i - \frac{2^{256}i}{N+1}\right)^2.
\end{equation}
The resulting estimate $\hat{N}$ can be computed in closed form.
%

When a node starts running, it must perform DHT queries for a few random keys to initialize its network size estimate. 
Since a larger number of queries will result in higher accuracy, making more queries than what is needed to initialize one's routing table is recommended.
Thereafter, keeping the estimate up-to-date does not require any excess DHT queries beyond what is already used for refreshing the routing table as this is done frequently (every 10 minutes).

While the network size estimate has a stochastic variance resulting from the probability distribution of the honest peer IDs, it is hard for an attacker to bias the estimate significantly. Since the estimator uses the density of peer IDs around keys chosen uniformly at random, the adversary would require numerous Sybil nodes (on the order of the whole network size) to significantly affect the peer ID density around those keys.


 \section{Impression GTD}\label{sec:imgtd}
GTD has two step-sizes. In this section, we introduce a new Gradient TD algorithm that has only one step-size, e.g., see the six design desiderata as discussed in Section \ref{sec:introduction}. 
 
 Our idea is to decouple the two estimations in GTD by a special sampling method that is going to be detailed later. To do this, we use a buffer that stores transitions. At a time step $t$, we sample two i.i.d. transitions from the buffer,   
$(\phi_1, r_1, \phi_1')$ and $(\phi_2, r_2, \phi_2')$. Note the shorthand $\phi_1=\phi(s_1)$ is for some state $s_1 \in \SS$, and $\phi_2=\phi(s_2)$ for some $s_2\in \SS $. 

Our algorithm updates the parameter vector by
\begin{equation}\label{eq:one_update}
\Delta\theta_{t} = - \alpha_t (\gamma\phi'_1 - \phi_1) \bold{sim}(\phi_1,\phi_2)\left[(\gamma\phi_2' - \phi_2)^\tr\theta_t + r_2\right],
\end{equation}
where $\alpha_t$ is a positive step-size and $\bold{sim}$ is some similarity measure for the two input feature vectors. The update is interesting that the similarity seemingly ``pairs'' the gradient of a TD error on a transition with the TD error on another transition.  

Let's understand this update. If $r_2$ is a big reward, it likely creates a large TD error (the last term in the bracket). This TD error is bridged to adjust $V(s_1)$ and $V(s_1')$. That is, a TD error {\em impresses} another (independent) sample, based on which the parameters are adjusted. 
The bigger is the similarity between the two feature vectors, the larger impression of the TD error from one sample is going to make on the other. We call this new algorithm the {\em Impression} GTD.  

In this paper, we focus on the similarity measure being the correlation between the two feature vectors. 
Let us define $\phi = (\gamma\phi'_1 - \phi_1) \phi_1^\tr\phi_2$. The update can be rewritten into
\begin{align*}
\theta_{t+1} &=  \theta_t - \alpha_t \left[\gamma{\phi_2'}^\tr\theta_t +  r_2 - \phi_2^\tr\theta_t \right]\phi\\
 &=  \theta_t - \alpha_t \left[\gamma V(s_2')+  r_2 - V(s_2) \right]\nabla_{\theta} J.
\end{align*}
where $\alpha$ is the step-size and $\nabla_{\theta} J=\phi$. The overloading notation $\nabla_{\theta} J$ will be explained shortly. 

Interestingly, most incremental $O(d)$ TD algorithms known to the authors update the parameter based on one sample. This algorithm use two independent transitions for the update. It looks like the TD update, but not exactly so (because the transposed term has $\phi_2$ in the first line instead of $\phi$). In fact, it is a modification of the TD(0) update (or the so-called bootstrapping), whose key idea is to treat $V(s_{t+1})$ as a constant target in taking the gradient of the TD error, by combing the two sample transitions to form {\em truly an SGD} algorithm that minimizes the NEU objective.\footnote{\citet{gtd} had a comment that GTD is a SGD method. This is not very precise. GTD is two-time scale, and it is not the standard, single-time-scale SGD. We noted in literature this interpretation of GTD (and GTD2 and TDC) is not rare, e.g., see \citep{td_survey}. See also the discussions on page 35 by \citet{csaba_book}. A better terminology for GTD, GTD2 and TDC may be that they are pseudo-gradient methods as suggested. The exception is when GTD uses exactly the same step-size for the two iterators in the saddle-point formulation \citep{bo_gtd_finite}. Empirical results show that in order for good convergence across domains, one has to use different ratios for the two step-sizes, e.g., see \citep{tdc,martha2020gradient}. For example, Figure 2 of \citep{martha2020gradient} shows that GTD2 generally prefers a larger step-size for the helper iterator in four out of five domains.  However, for TDC, in three domains, it prefers actually slower update for the helper iterator. This is not covered by the theory of two-time scale stochastic approximation. }

There has been a mystery about the function $J$ for decades. In particular, what form should $J$ take for the convergence guarantee of TD methods? The TD methods were developed by treating $V(s_2')$ as the target and taking just $-\nabla_\theta V(s_2)$ as $\nabla_\theta J$. For example, it is common in literature to call $V(s_2')$ (or $\gamma V(s_2')+r_2$ ) the ``TD target'', the essential quantity for bootstrapping \citep{sutton2018reinforcement}.
Treating $V(s_2')$ as the target is also the essential idea for using neural networks for TD methods. For example, \citet{tdgammon}'s TD-gammon is the first such successful example. In DQN, \citet{mnih2015human} used the target network that is a historical snapshot of the network to generate relatively stable targets.  
Counterexamples show that TD can diverge if (1) nonlinear function approximation is used (even for on-policy learning); (2) learning is off-policy (even in the linear case); and (3) bootstrapping (TD methods with the eligibility trace factor smaller than one). This is referred to as the deadly triad \citep{sutton2018reinforcement,zhang2021breaking}.
Historical efforts that research into what form of $J$ guarantees convergence include re-weighted least-squares \citep{bertsekas1995counterexample}, residual gradient \citep{baird1995residual}, and Grow-Support \citep{boyan1994generalization}, etc. These algorithms attempted to derive an algorithm that is either a contraction mapping or a stochastic gradient with the current transition. See also \citep{bo_gtd_finite} for a good discussion and the long history of seeking gradient descent methods for temporal difference learning.  

Our algorithm may imply that this cannot be done with a single sample, if one wants to achieve the TD solution. In order to achieve that, we have to use two samples, in particular, 
\[
\nabla_\theta J = \left[\gamma \nabla_\theta V(s_1') - \nabla_\theta V(s_1)\right] \nabla_\theta V(s_1)^\tr \nabla_\theta V(s_2).
\]
In contrast, residual gradient takes $\nabla_\theta J=\gamma \nabla_\theta V(s_2') - \nabla_\theta V(s_2)$,  calculated on the same transition as where the TD error is computed. 
This shows why the residual gradient algorithm does not converge to the TD solution as discussed by \citet{tdc}. In order to converge to the TD solution, one needs to {\em compute the TD error and the gradient on two different (and independent) samples, also with a similarity measure to bridge them, instead of computing the TD error and the gradient on a single sample}. While the resulting algorithm is indeed an SGD algorithm, the independence sampling mechanism of two samples is different from supervised learning. That is, in supervised learning, one i.i.d. sample suffices for a well-defined SGD update. It has guaranteed convergence (with probability one) to the correct optimum. However, in the reinforcement learning setting, only one sample is not enough for ensuring this unless for deterministic environments.  
Although this still requires (at least) two i.i.d. samples at a time, note that the two samples do not need to be the i.i.d. transitions from the same state, because it is not practical to reset our state to the previous state to start over from there, ``passed is passed''.

Note the above update does not use the reward signal $r_1$. To take advantage of the two transitions, we also perform
\[
\theta_{t+1} =  \theta_t - \alpha_t \left[\gamma{\phi_1'}^\tr\theta_t + r_1 - \phi_1 ^\tr\theta_t\right]\phi,
\]
in which $\phi = (\gamma\phi'_2 - \phi_2) \phi_2^\tr\phi_1$ this time. This is due to that in using the two samples, the operation is symmetric.  
To ensure the two transitions are independent, in sampling we also require that they are from two different episodes. This can be done by an adding 
the episode index for each transition. Ours uses this special and novel sampling method to the best of our knowledge.\footnote{\citet*{shangtong_averagereward_two_iid} considered two i.i.d. samples from a given distribution in an average-reward off-policy learning algorithm, but not in a buffer setting like our method.} To differentiate from the uniform random sampling and prioritized sampling methods widely practised in literature, we call it the {\em independence sampling} method.   


The merit of independence sampling and Impression GTD is that together they remove the two steps-sizes and the resulting  tuning efforts and slow convergence. They achieve the decoupling of the two terms in GTD in a novel way. 
From a practical view, carrying a buffer is acceptable. Similar ideas appear in experience replay \citep{lin1992experiencereplay}, %model-based RL such as the linear Dyna \citep{lin_dyna} implementation \citep{multi-step-dyna} that saves the online feature vectors for later planing, 
and deep reinforcement learning \citep{mnih2015human,schaul2015prioritized}. 

Recently, \citet{shangtong_imgtd} developed a GTD algorithm that is very similar to ours as in equation \ref{eq:one_update}. They started with the same observation as ours, in that the gradient $A^\tr$ and the expected TD error in GTD's O.D.E. can be estimated separately. Their algorithm has a buffer as well, but the buffer length does not need to grow linearly as learning proceeds, while our analysis does have such a limitation.  
They also focused on the infinite-horizon setting, and the analysis is very much involved in the discretization of the underlying O.D.E. Our analysis is focused on the episodic problems though occasionally there are also discussions about infinite horizon problems as well. Our independence sampling is also an important ingredient, which facilitates an SGD analysis framework.  
In general, their direct GTD and our Impression GTD can be viewed as algorithms in the same family, with the same motivation and similar algorithmic flavour.   

In a summary, Impression GTD is guaranteed to converge under the same conditions on the MDP and linear features as GTD. Together with direct GTD \citep{shangtong_imgtd}, ours is the first theoretically sound, truly single-time-scale SGD off-policy learning algorithm, with $O(d)$ complexity and one step-size. In Section \ref{sec:theory} and Section \ref{sec:experiments}, we conduct theoretical analysis and empirical studies to show that the new algorithm converges much faster than GTD, GTD2 and TDC. 

We will detail the sampling process in the next section, which also introduces a more general form of this algorithm. 

\section{Mini-batch Policy Evaluation}\label{sec:minibatchPE}
This section further extends the Impression GTD. It is common to use mini-batch training in deep learning and deep reinforcement learning. There the mini-batch training paradigm is necessary mostly because the size of the data sets and the high dimensional inputs. Here we show that it also makes sense to use mini-batch training for off-policy learning, even in the linear case and even the problem size is not big. The motivation of using a buffer here has a different motivation from in deep learning and deep reinforcement learning though it also has the merit of improving sample efficiency and scaling to large problems. In short, the buffer is a tool for decoupling the error and gradient estimations in GTD. 

Let's start with on-policy learning. Suppose we maintain a buffer that is large enough. At each time step, we take an action according to the policy that is evaluated, observing a transition, $(\phi_t, \phi_t', r_t)$. We put the sample into the buffer. Next we sample a mini-batch of samples, $\{(\phi_i, \phi_i', r_i)\}, i=1, 2, \ldots, m$, where $m$ is the batch size.  We then update the parameter vector by the averaged TD update:
\[
\theta_{t+1} = \theta_t + \alpha_t \frac{1}{m}\sum_{i=1}^m \left(\gamma {\phi_i'}^\tr\theta_t +r_t - \phi_i^\tr\theta_t\right)\phi_i.
\]
We call this algorithm the {\em mini-batch TD}.

We follow by extending the Impression GTD for off-policy learning to work with mini-batch sampling. The buffer saves for each sample also the episode index within which a sample is encountered. At a time step, we sample two batches of samples,
\begin{equation}\label{eq:buffer1}
b_1=\{(\phi_i, \phi_i', r_i, e_i)|i=1, 2, \ldots, m_1\}, \quad b_2=\{(\phi_j, \phi_j', r_j, e_j)|j=1, 2, \ldots, m_2\}
\end{equation}
where $e_k$ is the episode index for the $k$th sample. In order for the samples in $b_1$ and $b_2$ to be independent, for any sample index pair, $i$ of $b_1$ and $j$ of $b_2$, we require that they are from different episodes:
\begin{equation}\label{eq:buffer2}
e_i\neq e_j, \quad for \quad \forall i=1, 2, \ldots, m_1; j=1, 2, \ldots, m_2.
\end{equation}
We first generate the averaged TD update from the samples in $b_2$, just like in the mini-batch TD:
\[
\bar{u}_{t} = \frac{1}{m_2} \sum_{j=1}^{m_2} \left(\gamma \phi_j'^\tr\theta_t +r_j - \phi_j^\tr\theta_t\right)\phi_j.
\]
Then for each sample in $b_1$, we compute $\bar{\delta}_t(i)=\phi_i^\tr \bar{u}_t$. 
Finally, the {\em mini-batch Impression GTD} update is
\begin{equation}\label{eq:imgtd}
\theta_{t+1} = \theta_t - \alpha_t \frac{1}{m_1}\sum_{i=1}^{m_1}(\gamma \phi_i' - \phi_i) \bar{\delta}_t(i).
\end{equation}

In the lookup table case,\footnote{In this case, with batch sizes $m_1=m_2=1$, the algorithm is a variant of Baird's RG, equipped with double-sampling. The algorithm converges to the true value function, while RG does not because RG only converges to the correct value function for deterministic MDPs. 
In fact, this is the place where double-sampling and independence sampling meet. Update to the weights happens only when $\phi_i$ and $\phi_j$ are the same, or, the two i.i.d. transitions are from the same state of the MDP. This is rare though, which also shows why mini-batch sampling leads to faster convergence than using batch sizes equal one. This observation was due to James MacGlashan.} this means the bigger is this $\bar{\delta}_t(i)$, the more eligible is this sample for a big update. Thus the update for $\theta(s_i)$ (or $V(s_i)$), and $\theta(s_i')$ (or $V(s_i')$) is big if $\bar{\delta}_t(i)$ is large. Note because $\bar{\delta}_t(i)= u_t(s_i)$ in this case, this largely agrees with prioritized sweeping \citep{moore1993prioritized}. Consider the table lookup case. When $|u_t(s_i)|$ is large, it means the TD update for the the component, $\theta(s_i)$, is big. Thus we can view Impression GTD as a way of adjusting the magnitude of the TD update in the original TD(0) algorithm and update based on the adjusted.



Consider for batch $b_1$, we have only one sample, e.g., the latest online sample, and $b_2$ has $m$ samples. This in fact is the standard online learning paradigm, hereby aided with some historical samples:\footnote{This is actually a ``shrinked'' version of R1-GTD. }
\begin{align*}
\Delta \theta_t &=  - \alpha_t (\gamma \phi_t' - \phi_t) \phi_t^\tr\frac{1}{m} \sum_{j=1}^m \left(\gamma \phi_j'^\tr\theta_t +r_j - \phi_j^\tr\theta_t\right)\phi_j\\
&=  - \alpha_t (\gamma \phi_t' - \phi_t) \frac{1}{m} \sum_{j=1}^m \left(\gamma \phi_j'^\tr\theta_t +r_j - \phi_j^\tr\theta_t\right)\phi_t^\tr\phi_j\\
&=  - \alpha_t (\gamma \phi_t' - \phi_t) \frac{1}{m} \sum_{j=1}^m \delta_j(\theta_t) \phi_t^\tr\phi_j=  - \alpha_t (\gamma \phi_t' - \phi_t) \frac{1}{m} \sum_{j=1}^m \delta_j(\theta_t) \mbox{sim}(s_t, s_j).
\end{align*}
The first line means, if the current feature vector is greatly correlated the averaged TD update from historical samples, the update for $\theta$ is likely to be big for the current transition, to reduce the difference between  $V(s_t)$ and $\gamma V(s_t')$. We could also say that it reduces the difference between $V(s_t)$ and $\gamma V(s_t')+r_t$ because the reward is a constant bias whose gradient is zero. The reward does not appear in $(\gamma \phi_t' - \phi_t)$ because it is already taken care of in the averaged TD update, which will be driven to zero as the update proceeds. The effect is that we use the averaged TD update (estimated independently) {\em projected} on the current feature vector for the parameter update.  

The second and third lines give a different interpretation of the algorithm. The algorithm replaces the TD error in the standard TD with an average TD error, {\em similarity weighted}. In particular, instead of using the current TD error, $\delta_t$, calculated on the latest transition, to trigger learning, as in the standard TD(0), it uses an average of the TD errors that are computed on independent samples, weighted by the similarity of the sampled historical feature vectors to the current feature vector, for learning. Thus our algorithm takes an approach that comes with an improved estimation for the error signal to prevent the divergence of TD(0) for off-policy learning, for which using the latest TD error is problematic.

Notably, this interpretation gives a connection to \citet{baird1995residual}'s Residual Gradient (RG) algorithm. If we replace the weighted averaged TD error in the  third line with the latest TD error, it becomes exactly RG. RG is guaranteed to converge, however, not to the TD solution, e.g., see \citep{tdc}. TD(0) uses the latest TD error in another way, however, it suffers from divergence for off-policy learning. This update is guaranteed to converge to the TD solution under general and the same conditions as GTD. Furthermore, the convergence is orders faster than GTD, as we will show in Section \ref{sec:theory}.

The complexity of mini-batch Impression GTD is $O((m_1+m_2)d)$ per step, where $m_1$ and $m_2$ are the batch sizes. It is more complex than the Impression GTD in Section \ref{sec:imgtd} and GTD. However, it is still a linear complexity that is scalable to large problems.

An easier implementation for independence sampling is to have two buffers. Before the start of an episode, we can choose a random number that is either zero or one with equal probability. If it’s zero, then all the samples in this episode will be saved to the first buffer; otherwise, they will be saved to the second buffer. At sampling time, we just sample a batch from the first buffer and another batch from the second. In this way, we can also save extra memory for the episode index in each sample. Using the odd-even episode number for switching the buffers also works. This two-buffer implementation is shown in Algorithm \ref{alg:imGTD}. The similarity computation is also consumed so as to vectorize. 

\begin{algorithm}[t]
\caption{Impression GTD for off-policy learning, with independence sampling.}\label{alg:imGTD}
\begin{algorithmic}
\Require $\gamma \in (0, 1)$, the discount factor; $\alpha>0$, the step-size; $\phi(\cdot): \SS \to \RR^d$, the features
\State $\theta \gets \theta_0$
\Comment{Initialize the parameter vector}
\State buffer $B_1 \gets []$
\State buffer $B_2 \gets []$
\Comment{Initialize the buffers}
\For{episode $e=0, 1, \ldots$}

\State Environment resets to an initial state, $s_0$, drawn i.i.d. from some distribution 

\State $s\gets s_0$

\For{time step $t=0, 1, \ldots$}

    \State Observe $\phi(s)=\phi$, and take an action according to the behavior policy $\pi_b$

    \State Observe the next feature vector $\phi(s')=\phi'$ and reward $r$
    
\If{$e$ is odd} 
\Comment{Append the data in the same episode to same buffer}
    \State $B_1.\mbox{append}((\phi, \phi', r))$
\Else
    \State $B_2.\mbox{append}((\phi, \phi', r))$
\EndIf

\If{len$(B_2)>M$}
\State  
Sample a batch of $m_2$ samples from $B_2$, $\{(\phi_j, \phi_j', r_j)\}$, and compute
\[
\bar{u} \gets \frac{1}{m_2} \sum_{j=1}^{m_2} \left(\gamma \phi_j'^\tr\theta +r_j - \phi_j^\tr\theta\right)\phi_j.
\]

\State Sample $\{(\phi_i, \phi_i', r_i), i=1, \ldots, m_1\}$ from $B_1$ 

\State Form a feature matrix $\Phi$ with $\Phi[i,:] = \phi_i^\tr$
\Comment{The $i$th row of the matrix is $\phi_i^\tr$}

\State Compute $\bar{\bold{\delta}}=\Phi \bar{u}$

\State Update the parameters by
\[
\theta \gets \theta - \alpha \frac{1}{m_1}\sum_{i=1}^{m_1}(\gamma \phi_i' - \phi_i) \bar{\bold{\delta}}(i).
\]

\EndIf

\State $s\gets s'$
\EndFor
\EndFor
\end{algorithmic}
\end{algorithm}



In terms of the similarity measure used by our algorithm, the most relevant work is a recent new loss,  called the ``K-loss'' function \citep*{lihong_kernel_sim}, %\footnote{This paper was noted by Declan Oller internally at Sony AI after the writing and experiments of the paper were almost finished with exactly the same algorithm layout presented herein (with the $\bold{sim}$ notation and independence sampling). It is a great research coincidence that \citet{lihong_kernel_sim} used kernel embedding for the similarity measure. }, 
defined by the product of the Bellman errors calculated on two i.i.d. transition samples, weighted by a kernel encoding of the similarity between the two samples. They are probably the first to find that considering the similarity interplay between i.i.d. transitions can circumvent the double-sampling problem for reinforcement learning. Using their method, we can actually derive our algorithm in a second way. In particular, the NEU objective is
\begin{align}
\bold{NEU} &= \norm{\EE\delta\phi}^2\label{eq:neu_L2}\\
&=\EE[\delta\phi]^\top \EE[\delta\phi] \nonumber\\ 
&= \EE[\delta_1\phi_1]^\top \EE[\delta_2\phi_2] \Longleftrightarrow \EE[\delta_1\phi_1^\top \delta_2\phi_2] \nonumber \\
&= \EE[\phi_1^\top \phi_2 \delta_1 \delta_2] \nonumber\\
&= \EE[\bold{sim}(s_1, s_2) \delta_1 \delta_2].\label{eq:neu_2deltas}
\end{align}
This is exactly the place where \citeauthor*{lihong_kernel_sim} and we converge to. The  double-sampling problem arises when one aims to optimize using the single, online sample (the first line), which has held back the off-policy learning field for decades. 
The third line means we are realizing the $\ell_2$ norm on two independent transitions instead of on a single transition. The arrow annotated equality is due to the independence sampling.\footnote{Strictly speaking, the K-loss was defined using the Bellman errors, e.g., equation (3) of \citep{lihong_kernel_sim} uses the Bellman operator. The proof of their Corollary 3.5 mentioned "TD error". However the proof was done for deterministic MDPs for which TD error is the same as Bellman error.  Writing the loss in terms of  the weighted independent TD errors is more direct, also easier to interpret (without a model) and it entails direct optimization for practitioners. This is also interesting because minimizing the usual, online TD error via gradient descent has pitfalls \citep{tdc}, in particular the way represented by residual gradient \citep{baird1995residual}.} 
Therefore besides minimizing NEU, another way of interpreting our method is that it is a SGD method for minimizing the {\em expected product of two i.i.d. TD errors}, weighted by the similarity between the two feature vectors where the TD errors happened.

\begin{proposition}
Using independence sampling, we sample two independent transitions, $(s_1, r_1, s_1')$ and $(s_2, r_2, s_2')$ from the two buffers that have an infinity length. 
Consider a generic loss $\bold{N}(\theta) = \EE[\bold{sim}(s_1, s_2) \delta_1 \delta_2]$, where $\bold{sim}(s_1, s_2)$ is some similarity measure. Define $C= \EE[\phi(s)\phi(s)^\tr]$, where the expectation is taken with respect to the behavior policy (i.e., the distribution of $s$). Assume $C$ is non-singular. If $\bold{sim}(s_1, s_2)= \phi(s_1)^\tr C^{-1} \phi(s_2)$, then we have, $\bold{N}(\theta) = \bold{MSPBE}(\theta)$.  
\end{proposition}
\begin{proof}
We have 
\begin{align*}
\bold{N}(\theta) &= \EE[\bold{sim}(s_1, s_2) \delta_1 \delta_2] \\
&= \EE[\phi_1^\tr C^{-1} \phi_2 \delta_1\delta_2]\\
&= \EE[\delta_1 \phi_1^\tr C^{-1} \delta_2\phi_2 
]\\
&= \EE[\delta_1 \phi_1^\tr]  C^{-1} \EE[\delta_2\phi_2 
]\\
&= \bold{MSPBE}(\theta), 
\end{align*}
where the last second line is because of independence sampling, and $C^{-1}$ is a constant. The last line is because the buffers are sufficiently long so that the empirical distribution is the true data distribution. 
\end{proof}
Thus this shows that NEU and MSPBE belong to the same family of objective functions that are only different in a similarity measure, under independence sampling. Note this observation actually holds for any S.P.D matrix $U$ besides $C$. In particular, the generic loss $E(\theta)$ discussed in Section \ref{tdc} is also a special case of $\bold{N}(\theta)$.  
While these observations are interesting, we focus on minimizing NEU in this paper. 

Our method of deriving the Impression GTD algorithm by decoupling the estimations of $A^\top$ and $A\theta+b$ in GTD also entails an empirical form of the NEU loss, given multiple samples:
\[
\widehat{\bold{NEU}}(\theta|B_1, B_2) =\sum_{s_1 \in B_1}\sum_{s_2 \in B_2}\bold{sim}(s_1, s_2) \delta_1(\theta) \delta_2(\theta).
\]
For episodic problems, $B_1$ and $B_2$ are from our two-buffer implementation, for which samples in $B_1$ are always independent from the samples in $B_2$. 

For infinite horizon problems, $B_1$ and $B_2$ can be collected such that samples in them have a sufficiently large time window. For example, every $10000$ steps, we switch the collection buffer. In the first $T_0$ time steps, all the samples are inserted into $B_1$ and for the next $T_0$ time steps, the samples go into $B_2$; etc. A large $T_0$ ensures that no samples, for which the similarities are computed, happened close to each other in time, thus controlling their dependence strength at sampling time.    

Writing the NEU loss in terms of multiple samples is more reminiscent of the general machine learning problem where one minimizes an empirical loss on data sets. This is especially interesting because it transforms off-policy learning, an important problem of reinforcement learning, into a supervised learning problem, except that the data still needs to be collected for which there is the issue of exploration, etc. Nonetheless, we think it is an important connection to establish between reinforcement learning and supervised learning. This view is also interesting because the $\bold{sim}$ is a matrix form now, which measures inter-similarity between independent samples across the two buffers.\footnote{Note that \citeauthor{lihong_kernel_sim} did not have this form of the loss. Instead, they estimated the loss and the gradient using V-statistics. It has a problem that is discussed later in this section.}
Suppose the buffers keep adding the data and never drop any sample. Taking the gradient descent for the empirical NEU gives
\begin{align*}
\theta_{t+1} &= \theta_t - \frac{\alpha}{2} \nabla \widehat{\bold{NEU}} \\
&= \theta_t - \frac{\alpha}{2} \sum_{s_1 \in B_1}\sum_{s_2 \in B_2}\bold{sim}(s_1, s_2) \nabla(\delta_1 \delta_2) \\
&= \theta_t - \frac{\alpha}{2} \sum_{s_1 \in B_1}\sum_{s_2 \in B_2}\phi_1^\tr\phi_2 (\delta_1\nabla\delta_2 + \nabla\delta_1 \delta_2 )\\
&= \theta_t - \frac{\alpha}{2} \sum_{s_1 \in B_1}\sum_{s_2 \in B_2}\phi_1^\tr\phi_2 \left[ (r_1+\gamma\phi_1'^\tr \theta_t - \phi_1^\tr \theta_t) \nabla\delta_2 + (r_2+\gamma\phi_2'^\tr \theta_t - \phi_2^\tr \theta_t) \nabla\delta_1 \right]
\end{align*}
The two terms in the bracket is similar. For example, the first one is  (dropping the subscript of $\theta$ for simplicity)
\begin{align*}
 \phi_1^\tr \phi_2(r_1+\gamma\phi_1'^\tr \theta - \phi_1'^\tr \theta) \nabla\delta_2
% &= \phi_1^\tr\phi_2(r_1+\gamma\phi_1'^\tr \theta - \phi_1^\tr \theta) (\gamma \phi'_2 - \phi_2)\\
% &=(r_1+\gamma\phi_1'^\tr \theta - \phi_1^\tr \theta)\phi_1^\tr\phi_2 (\gamma \phi'_2 - \phi_2)\\
% &=(r_1+\gamma\phi_1'^\tr \theta - \phi_1^\tr \theta)\phi_2^\tr \phi_1 (\gamma \phi'_2 - \phi_2)\\
&=\phi_2^\tr \phi_1(r_1+\gamma\phi_1'^\tr \theta - \phi_1^\tr \theta) (\gamma \phi'_2 - \phi_2). 
\end{align*}
Suppose $T_1$ samples are stored in buffer $B_1$ and $T_2$ samples are in buffer $B_2$. We have 
\begin{align*}
&\sum_{s_1 \in B_1}\sum_{s_2 \in B_2} 
    \phi_2^\tr \phi_1(r_1+\gamma\phi_1'^\tr \theta - \phi_1^\tr \theta) (\gamma \phi'_2 - \phi_2) \\
    % =& \sum_{t_1=1}^{T_1}\sum_{t_2=1}^{T_2} 
    % \phi_2^\tr \phi_1(r_1+\gamma\phi_1'^\tr \theta - \phi_1^\tr \theta) (\gamma \phi'_2 - \phi_2) \\
    =& \sum_{t_2=1}^{T_2} \sum_{t_1=1}^{T_1}\phi_2^\tr\phi_1(r_1+\gamma\phi_1'^\tr \theta - \phi_1^\tr \theta)
     (\gamma \phi'_2 - \phi_2) \\
    =& \sum_{t_2=1}^{T_2} \phi_2^\tr(\tilde{A}_1 \theta+\tilde{b}_1)
  (\gamma \phi'_2 - \phi_2) \\
      =& \sum_{t_2=1}^{T_2}   (\gamma \phi'_2 - \phi_2)\phi_2^\tr(\tilde{A}_1 \theta+\tilde{b}_1)
 \\
  %    =& \sum_{t_2=1}^{2T+1} \phi_2^\tr(A_{2T} \theta+b_{2T})
  % (\gamma \phi'_2 - \phi_2) \\
      =& \tilde{A}_2^\tr (\tilde{A}_1 \theta+\tilde{b}_1),
\end{align*}
in which we define $\tilde{A}_1 = \sum_{t_1=1}^{T_1} \phi_1(\gamma \phi_1' - \phi_1)^\tr$, and $\tilde{A}_{2} = \sum_{t_2=1}^{T_2} \phi_2(\gamma \phi_2' - \phi_2)^\tr$. 
The normalized matrices, i.e., $\tilde{A}_1/T_1$ and $\tilde{A}_2/T_2$, are both consistent estimations of the matrix, $A=E[\phi(\gamma \phi' - \phi)^\tr]$. Note this algorithm can be implemented in a complexity that is linear in the number of samples ($n$), i.e., $O(d^2)$ per sample, where $d$ is the number of features, by forming the matrices explicitly. 


Therefore, if we go for a direct approach of minimizing the empirical NEU, it ends up with a variant of the expected GTD \citep{ptd_yao}, 
\[
  \theta_{t+1}   = \theta_t -\frac{\alpha}{2} \left[\tilde{A}_1^\tr (\tilde{A}_2 \theta_t+\tilde{b}_2) + \tilde{A}_2^\tr (\tilde{A}_1 \theta_t+\tilde{b}_1) \right],
\]
which is $O(d^2)$ per step (the two matrices can be aggregated incrementally).
Though an interesting variant of the expected GTD, this algorithm is presented purely for the understanding of Impression GTD. Our convergence and convergence rate analysis apply to this variant in a straightforward way. 

 The Impression GTD applies the successful mini-batch training in deep learning to off-policy learning and reduces to a linear complexity in the number of features, without resorting to two-time scale stochastic approximation. This observation was also made by \citet{lihong_kernel_sim}. They noted that their loss function ``coincides'' with NEU in the linear case (see their Section 3.3). However, the reason was not well understood or explained. Hopefully it is clear that our derivation above showed that this is not an coincidence. In matrix notation, the minibatch Impression TD uses the batch samples to build two matrices, $\tilde{A}_1^\tr = \sum_{b=1}^m (\gamma\phi_1' - \phi_1)\phi_1^\tr $, from the batch samples in $B_1$, and  $\tilde{A}_2 = \sum_{b=1}^m \phi_2  (\gamma\phi_2' - \phi_2)^\tr$ (and $\tilde{b}_2 = \sum_{b=1}^m \phi_2 r_2$), from the minibatch samples of $B_2$. The terms were transformed equivalently using the $\bold{sim}$ measure such that these matrices do not form explicitly and thus avoid the $O(d^2)$ complexity, e.g., see 
equation \ref{eq:imgtd}.\footnote{We found it's interesting that the two implementations, one that forms the matrices explicitly, and the other that doesn't (instead using $\bold{sim}$), gives the flexibility of switching for the higher computation efficiency given different numbers of samples (e.g., the batch sizes). }


\citeauthor{lihong_kernel_sim} also had a batch version of their algorithm, e.g., see their equation 4 and Section B.1 therein. However, the implementation is not technically sound because the independence of samples would break.  
The nature is a bit tricky.\footnote{We also refer the readers to \citep{shangtong_imgtd} for more detailed discussions about this problem.} Random sampling from the buffer does not necessarily means the samples in the buffer are i.i.d. 
Let's say we have two samples, $(s_1, s_1', r_1)$ and $(s_2, s_2', r_2)$. They are sampled i.i.d. from the buffer. However, what if they occurred in the same episode when we inserted them? Let's say $(s_2, s_2', r_2)$ was inserted into the buffer right after $(s_1, s_1', r_1)$. That is, $s_2=s_1'$. The second sample is dependent on the first one. In general, as long as the two samples are from the same episode, the one that happens at a later time depends on the former one and they are not independent.
It may be easier to understand in the infinite horizon setting. Suppose the Markov chain is irreducible and aperiodic and there exists a unique stationary distribution under the behavior policy. Then the samples are only independent of each other if the empirical distribution of the states in the buffer gets sufficiently close to the stationary distribution. Before this happens, the samples in the buffer are all dependent on each other. This depends on how fast the chain is mixing. It can take a very long time to reach the stationary distribution for slowly mixing chains. After the stationary distribution is reached, the Markovian argument for the above two samples still holds. However, because the distribution of states becomes stationary, the samples from the chain exhibit independence: the distribution of a state is just a property on its own. 
Consider a simple example. Assume that $s_2$ can only be reached from $s_1$. Note, however, from $s_1$ one can reach other state(s) than $s_2$. 
Let $\mu$ be the empirical distribution of the states in the buffer (a single buffer that stores all the samples up to the current time step). 
Then we have 
\[
\mbox{Prob}(s_2|s_1) = \mu(s_1) \mbox{Prob}(s1\to s_2)
\]
As long as $\mu(s_1) \mbox{Prob}(s1\to s_2)\neq \mu(s_2)$, $\mbox{Prob}(s_2|s_1)$ is not equal to $\mbox{Prob}(s_2)$, and thus the dependence between the two states holds. Before the chain reaches the stationary distribution, $\pi_0$, we don't have the equality. Only after $\mu$ gets sufficiently close to $\pi_0$, we have $\mu(s_1)\mbox{Prob}(s_1\to s_2)\approx \mu(s_2)$, and the independence between the states starts to exhibit. 

For episodic problems, one can define a similar chain from the distribution of the initial states (where the episodes are started), the behavior policy and the transition dynamics of the MDPs. If a unique stationary distribution exists, similar argument holds for the episodic problems. Most reinforcement learning problems in practice are episodic. Luckily, our independence sampling ensures the samples are independent even when the underlying chain has not reached the stationary distribution yet. This is shown by Lemma \ref{lem:independence} in Section \ref{sec:theory}. 

 



\section{Analysis}\label{sec:theory}
This section contains convergence rate analysis of Impression GTD %The convergence of Impression GTD under the diminishing step-size is first established using standard stochastic approximation, thanks to independence sampling. 
 with constant step-sizes. The first result is an $O(1/t)$ rate. For the second result, we first give a new condition of smoothness, called $L$-$\lambda$ smoothness. Under this weaker smoothness condition, we establish a tighter convergence rate for SGD than Theorem 3.1 of \citet{gower2019sgd_general}. 
Then by showing that the NEU objective and the independence sampling satisfies $L$-$\lambda$ smoothness, we prove that Impression GTD converges at a linear rate.

Our algorithm analysis is conducted in a generic GTD algorithmic framework. The $O(1/t)$ rate and the linear rate are both applicable to Expected GTD, A$^\tr$TD, and R1-GTD.  

Both the $O(1/t)$ rate and the linear rate depend on the i.i.d. sampling ensured by our independence sampling method. Thus we first introduce a lemma for that. 

\begin{lem}[Independence Sampling]\label{lem:independence}
For episodic problems, our sampling method according to equations \ref{eq:buffer1} and \ref{eq:buffer2} ensures that the transition samples from the two mini-batches are independent: 
\[
Pr(i_{t_1}  =s_1 \cap j_{t_2} = s_2) =Pr(i_{t_1}  =s_1 ) Pr(j_{t_2} = s_2),
\]
where $i_{t_1}$ and $j_{t_2}$ are the time steps that we insert the two samples into buffer $B_1$ and buffer $B_2$, respectively. 
\end{lem}
\begin{proof}
Without loss of generality, let us consider the batch size equal to 1. 
Let $(s_1, r_1, s_1')$ and $(s_2, r_2, s_2')$ be two samples drawn at time step $t$ by the sampling method. Then it suffices to prove that $s_1$ and $s_2$ are independent. For notation convenience, let $i_{t_1}$ be $s_1$, and the state sequence up to $s_1$ is $\{i_0, i_1, \ldots, i_{t_1}\}$, in the episode where we put $s_1$ into the buffer. Similarly, $j_{t_2}$ is for aliasing $s_2$. 
We just need to prove that $Pr(i_{t_1}=s_1 \cap j_{t_2} = s_2) = Pr(i_{t_1}=s_1) Pr(j_{t_2} = s_2)$. To see this, we first have
\begin{align*}
&Pr(i_{t_1}  =s_1 \cap j_{t_2} = s_2) \\
=& \sum_{i_0, i_1, \ldots, i_{t_1}-1}  \quad \sum_{j_0, j_1, \ldots, j_{t_2}-1}
Pr(\underline{i_0, i_1, \ldots, i_{t_1}  =s_1 } \cap \underline{j_0, j_1, \ldots, j_{t_2} = s_2})  \\
=&\sum_{i_0, i_1, \ldots, i_{t_1}-1}  Pr(i_0, i_1, \ldots, i_{t_1}  =s_1 )  \sum_{j_0, j_1, \ldots, j_{t_2}-1} Pr( j_0, j_1, \ldots, j_{t_2} = s_2)  
\end{align*}
The first equality is according to the law of total probability, which sums over all possible trajectories that lead to these two observations.  
The second equality is because the two episodes are independent due to that $i_0$ and $j_0$ are i.i.d. samples (which is ensured by the environment). 

It suffices to focus on the first term in the second equality. The second term  can be calculated similarly. We have 
\begin{align*}
&\sum_{i_0, i_1, \ldots, i_{t_1}-1}  Pr(i_0, i_1, \ldots, i_{t_1}  =s_1 )  \\
=& \sum_{i_0, i_1, \ldots, i_{t_1}-1}  Pr(i_{t_1}  =s_1 ) Pr(i_0, \ldots, i_{t_1}-1| i_{t_1}  =s_1 )   \\
=&   Pr(i_{t_1}  =s_1 ) \sum_{i_0, i_1, \ldots, i_{t_1}-1} Pr(i_0, \ldots, i_{t_1}-1)    \\
=&   Pr(i_{t_1}  =s_1 ) \sum_{i_0, i_1, \ldots, i_{t_1}-1} Pr(i_{t_1}-1)Pr(i_0, \ldots, i_{t_1}-2| i_{t_1}-1)    \\
=&   Pr(i_{t_1}  =s_1 ) \sum_{i_0, i_1, \ldots, i_{t_1}-2}\sum_{i_{t_1}-1}  Pr(i_{t_1}-1)Pr(i_0, \ldots, i_{t_1}-2| i_{t_1}-1)   \\
=&   Pr(i_{t_1}  =s_1 ) \sum_{i_0, i_1, \ldots, i_{t_1}-2}Pr(i_0, \ldots, i_{t_1}-2)   \\
=&   Pr(i_{t_1}  =s_1 ) \sum_{i_0} d_0(i_0)    \\
=& Pr(i_{t_1}  =s_1 ).
\end{align*}
The first equality is according to the conditional probability formula. The next equality is because historical observations are independent of later ones.

The remaining of the derivation breaks down according to the conditional probability formula. The third equality applies this one step,   
Then the next equality splits the sum over $i_{t_1}-1$, and the law of total probability follows. 
We recursively apply to the beginning to get the last second equality. Note $d_0$ is the sampling distribution of the initial state, and $\sum_{i_0}d_0(i_0)=1$.

\end{proof}

We analyze the convergence rates of Impression GTD under constant step-sizes. 
Many SGD analysis is conducted in the setting of the finite-sum loss function, e.g., see \citep{gower2019sgd_general,sps_stepsize}. In that setting, the function is of the form, $f(x)=\frac{1}{n}\sum_{i=1}^nf_i(x)$. This setting covers important applications in machine learning, especially supervised learning problems, where there are $n$ training samples and each $f_i$ is the  loss on sample $i$. However, it does not cover the application we consider in this paper, because the NEU objective is not a finite-sum form in a straightforward sense. Towards this end, we consider the ``expected form'' of the loss. That is, the loss function can be sampled via simulation, in particular, 
\[
f(x)=\EE[f_t(x)],
\]
where $f_t$ is the loss on the sample drawn at simulation step $t$,  according to a distribution $\mathcal{D}$. This covers the finite-sum loss and it is general enough to cover our GTD setting. Let $x^*$ be the optimum and $f(x^*)=\min_x f(x)$.   

We assume the gradient of the loss can be queried for each stochastic sample. Thus equivalently, we can also say that our simulation process keeps drawing the stochastic gradient. In particular, we draw $f_t'(x)$ to get a random sample for the true gradient $f'(x)$. 
At drawing step $t$, denote the gradient sample as $g_t(x)=f_t'(x)$ for a given $x\in \RR^d$.
\begin{lem}\label{lem:ED_avg}
Let us draw a batch of $m$ i.i.d. samples according to $\mathcal{D}$. 
Let $\avgx$ be the average of the sampled gradients in this batch, i.e.,  
\[
\avgx=\frac{1}{m}\sum_{t=1}^m g_t(x).
\]
We have, for any distribution $\mathcal{D}$ that satisfies $\EED [g_t(x)]=f'(x)$, the following holds:
    \[
    \EED \norm{\avgx}^2 = \frac{1}{m}\EED\norm{g_t(x)}^2+ \left(1-\frac{1}{m}\right) \norm{f'(x)}^2.
    \]
\end{lem}
The proof is in Appendix \ref{appendix:ED_avg}. 

Instead of analyzing Impression GTD and each of the three GTD algorithms that are discussed in Section \ref{sec:background} individually, 
we use a generic algorithmic framework that enables us to study their convergence rates at one time. 
Define $\tilde{A}_m=\frac{1}{m}\sum_{i=1}^m \phi_i (\gamma \phi_{i+1}- \phi_i)^\tr$ as a normalized matrix from $m$ samples.
Consider this algorithm:
\begin{equation}\label{alg:generic_gtd}
\theta_{t+1} = \theta_t - \alpha_t\tilde{A}_{m_1}^\tr (\tilde{A}_{m_2} \theta_t + \tilde{b}_{m_2}). 
\end{equation}
The above the algorithm is for mathematical definition only.  
Note the matrix and the transpose may not be explicitly formed or computed using matrix-vector product for certain algorithms. 
We compute $\tilde{A}_{m_1}$ and $\tilde{A}_{m_2}$ for different algorithms as follows:
\begin{itemize}
\item Impression GTD. 
$\tilde{A}_{m_1}$ and $\tilde{A}_{m_2}$ are computed
from buffer $B_1$ and buffer $B_2$, respectively. 

\item Expected GTD. For the algorithm that is discussed in Section \ref{sec:background} (equation \ref{eq:expectedGTD}), a single matrix is built from all the samples in the two buffers. To fit into the independence sampling and the generic TD framework, we consider here the version described in Algorithm \ref{alg:imGTD}. Thus $m_1=|B_1|$ and $m_2=|B_2|$. For simplicity of argument and without loss of generality, we assume $|B_1|=|B_2|=t/2$.

\item A$^\tr$TD. $\tilde{A}_{m_1}$ is computed from both buffers and $\tilde{A}_{m_2}$ is the rank-1 matrix from the latest transition, $\phi_t(\gamma \phi_{t+1} - \phi_t)^\tr$. Thus  
$m_1=t$ and $m_2=1$ for A$^\tr$TD.  

\item \oneexptd. $\tilde{A}_{m_1}$ is the rank-1 matrix from the latest transition, and $\tilde{A}_{m_2}$ is computed from both buffers instead. Thus $m_1=1$ and $m_2=t$ for \oneexptd. 

 
\end{itemize}

We first introduce an assumption that is fairly general in the analysis of TD methods, e.g., see \citep*{tsi_td,bertsekas1996neuro,gtd,zhang2020gradientdice,zhang2020provably}. 


\begin{assumption}\label{assumption:phir}
The feature functions in $\phi(\cdot): \SS\to \RR^d$, are independent. All the feature vectors and rewards are finite. 
\end{assumption}


We show that all the four discussed GTD algorithms are faster than GTD2 and TDC, even though the latter two were developed to improve the convergence rate of the GTD algorithm. Note that the above four GTD algorithms all have the same O.D.E. as the GTD algorithm. Thus the convergence is accelerated by them not by improving the conditioning of the problem (like GTD2 and TDC do). Instead, improvement is achieved by a single-time scale formulation of minimizing NEU. 
%The following theorem shows that all the four GTD algorithms converge at a rate of $O(1/t)$.  

First consider this term, $\delta_i \phi_i^\tr\phi_j \delta_j$, which is a sample of NEU using two independent sample transitions from $\phi_i$ and $\phi_j$. Its gradient is $(\gamma \phi_{i+1}-\phi_i) \phi_i^\tr \phi_j \delta_j + (\gamma \phi_{j+1}-\phi_j) \phi_j^\tr \phi_i \delta_i $. For simplicity, we only consider the first term in the following analysis. The second term is symmetric and our proof can be extended to including it in a straightforward way. Define $f'_{i,j}=(\gamma \phi_{i+1}-\phi_i) \phi_i^\tr \phi_j \delta_j$. One can show that $\EE[f'_{i,j}]=\EE\nabla[\delta_i \phi_i^\tr\phi_j \delta_j]$. 
We have
\begin{align*}
\norm{f'_{i,j}(x) - f'_{i,j}(y)} &= \norm{(\gamma \phi_{i+1}-\phi_i) \phi_i^\tr \phi_j (\gamma \phi_{j+1}-\phi_j)^\tr (x-y)}  \\
&\le \underbrace{\norm{(\gamma \phi_{i+1}-\phi_i) \phi_i^\tr \phi_j (\gamma \phi_{j+1}-\phi_j)^\tr}}_\text{$L_{i,j}$}\norm{x-y}   = L_{i,j} \norm{x-y}.
\end{align*}
That is, each $f_{i,j}(x)$ is $L_{i,j}$ smooth. Given that all the feature vectors are finite according to Assumption \ref{assumption:phir}, $L_{i,j}$ must be finite.  

We are now ready to give the $O(1/t)$ rate. 
\begin{thm}\label{thm:rates_1_over_t}
Let Assumption \ref{assumption:phir} hold. Also assume matrix $A$ is non-singular. 
Impression GTD, Expected GTD, A$^\tr$TD and \oneexptd converge at a rate of $O(1/t)$ with $\alpha \le \frac{2}{L_{\max}}$, 
where $L_{\max}=\max_{i,j}L_{i,j}$. 
In particular, 
\[
\min_{k=0, \ldots, t-1} 
 f(\theta_k) \le \max\left\{\frac{2}{t \alpha \left(2- \alpha L_{\max} \right)\sigma^2_{\min}(A)}f (\theta_{0}) - \frac{1}{m_1m_2\sigma^2_{\min}(A)} \sigma_v^2, \, 0\right\}.
\]
\end{thm}
\begin{proof}
Consider the generic GTD update in \ref{alg:generic_gtd}. 
Because each $f_{i,j}$ is $L_{i,j}$-smooth, we have
\begin{align*}
f_{i,j}(\theta_{t+1}) &\le f_{i,j}(\theta_t) + f_{i,j}'(\theta_t)^\tr (\theta_{t+1}-\theta_t) +\frac{L_{i,j}}{2} \norm{\theta_{t+1} - \theta_t}^2 \\
&= f_{i,j}(\theta_t) - \alpha_t f_{i,j}'(\theta_t)^\tr  \tilde{A}_{m_1}^\tr (\tilde{A}_{m_2} \theta_t + \tilde{b}_{m_2}) +\frac{\alpha_t^2 L_{i,j}}{2} \norm{\tilde{A}_{m_1}^\tr (\tilde{A}_{m_2} \theta_t + \tilde{b}_{m_2})}^2. 
\end{align*}
Summing above for all the samples $i$ in batch $b_1$ and all the samples $j$ in batch $b_2$ gives
\begin{align*}
\sum_{i,j} f_{i,j}(\theta_{t+1}) &\le \sum_{i,j}f_{i,j}(\theta_t)- \alpha_t \sum_{i,j}f_{i,j}'(\theta_t)^\tr  \tilde{A}_{m_1}^\tr (\tilde{A}_{m_2} \theta_t + \tilde{b}_{m_2}) \\
&\quad +\frac{\alpha_t^2 \sum_{i,j}L_{i,j}}{2} \norm{\tilde{A}_{m_1}^\tr (\tilde{A}_{m_2} \theta_t + \tilde{b}_{m_2})}^2.
\end{align*}
Note that 
\begin{align*}
\frac{1}{m_1m_2}\sum_{i,j}f_{i,j}'(\theta_t)
&=\frac{1}{m_1} \sum_{i=1}^{m_1} (\gamma \phi_{i+1}-\phi_i) \phi_i^\tr\frac{1}{m_2}\sum_{j=1}^{m_2}  \phi_j \delta_j\\
%two terms
% &=\frac{1}{m_1} \sum_{i=1}^{m_1} (\gamma \phi_{i+1}-\phi_i) \phi_i^\tr\frac{1}{m_2}\sum_{j=1}^{m_2}  \phi_j \delta_j + \frac{1}{m_2}\sum_{j=1}^{m_2} (\gamma \phi_{j+1}-\phi_j) \phi_j^\tr \frac{1}{m_1} \sum_{i=1}^{m_1}\phi_i \delta_i\\
&= \tilde{A}_{m_1}^\tr (\tilde{A}_{m_2} \theta_t + \tilde{b}_{m_2}).
\end{align*}
Thus 
\[
\frac{1}{m_1m_2}\sum_{i,j} f_{i,j}(\theta_{t+1}) \le \frac{1}{m_1m_2} \sum_{i,j}f_{i,j}(\theta_t)- \alpha_t\left(1 -\frac{\alpha_t\sum_{i,j}L_{i,j}}{2m_1m_2}\right) \norm{\tilde{A}_{m_1}^\tr (\tilde{A}_{m_2} \theta_t + \tilde{b}_{m_2})}^2
\]
Let $L_{\max}=\max_{i,j}L_{i,j}$, then for $\alpha_t \le \frac{2}{L_{\max}}$, the averaged pair-wise loss across the two batches is guaranteed to reduce because the following also holds:
\begin{equation}\label{eq:fb1b2}
\frac{1}{m_1m_2}\sum_{i,j} f_{i,j}(\theta_{t+1}) \le \frac{1}{m_1m_2} \sum_{i,j}f_{i,j}(\theta_t)- \alpha_t\left(1 -\frac{\alpha_tL_{\max}}{2}\right) \norm{\tilde{A}_{m_1}^\tr (\tilde{A}_{m_2} \theta_t + \tilde{b}_{m_2})}^2.
\end{equation}
For notation convenience, let $\bar{f}_{b_1, b_2}(x)= \frac{1}{m_1m_2}\sum_{i\in b_1,j\in b_2} f_{i,j}(x)$. We have, for a constant step-size $\alpha \le \frac{2}{L_{\max}}$,
\begin{align}
\bar{f}_{b_1,b_2}(\theta_t) &\le  \bar{f}_{b_1,b_2}(\theta_{t-1}) - \alpha\left(1 -\frac{\alpha L_{\max}}{2}\right) \norm{\tilde{A}_{m_1}^\tr (\tilde{A}_{m_2} \theta_{t-1} + \tilde{b}_{m_2})}^2\label{eq:fbatch}\\
&= \bar{f}_{b_1,b_2} (\theta_{0}) - \alpha\left(1 -\frac{\alpha L_{\max}}{2}\right)  \sum_{k=0}^{t-1}\norm{\tilde{A}_{m_1}^\tr (\tilde{A}_{m_2} \theta_{k} + \tilde{b}_{m_2})}^2 \nonumber\\
&\le \bar{f}_{b_1,b_2} (\theta_{0}) - \alpha\left(1 -\frac{\alpha L_{\max}}{2}\right)  t \min_{k=0, \ldots, t-1} \norm{\tilde{A}_{m_1}^\tr (\tilde{A}_{m_2} \theta_{k} + \tilde{b}_{m_2})}^2. \nonumber
\end{align}
Thus 
\begin{align*}
\min_{k=0, \ldots, t-1} \norm{\tilde{A}_{m_1}^\tr (\tilde{A}_{m_2} \theta_{k} + \tilde{b}_{m_2})}^2 &\le \frac{2}{t \alpha \left(2- \alpha L_{\max} \right)}\left(\bar{f}_{b_1,b_2} (\theta_{0}) - \bar{f}_{b_1,b_2} (\theta_{t})\right)\\
&\le \frac{2}{t \alpha \left(2- \alpha L_{\max} \right)}\left(\bar{f}_{b_1,b_2} (\theta_{0}) - \bar{f}_{b_1,b_2} (\theta^*)\right), 
\end{align*}
where the second line is because the averaged loss keeps decreasing according to equation \ref{eq:fbatch}, and furthermore, as $t$ goes to infinity, the loss is bounded below and thus $\theta^* = \lim_{t\to\infty}\theta_t$. 

This proves the $\ell_2$ norm of the update of the generic GTD converges at a rate of $O(1/t)$. The above steps can also be conducted after taking expectation of equation \ref{eq:fb1b2} with respect to the sampling. This gives 
\begin{align*}
\min_{k=0, \ldots, t-1} \EED \norm{\tilde{A}_{m_1}^\tr (\tilde{A}_{m_2} \theta_{k} + \tilde{b}_{m_2})}^2 &\le \frac{2}{t \alpha \left(2- \alpha L_{\max} \right)}f (\theta_{0}). 
\end{align*}
Note that in the context of generic GTD, the stochastic gradient is $f'_{i,j}(\theta)$, and the batch size is actually $m_1m_2$ because the generic GTD  essentially uses this number of pairs of the correlated TD errors from the two buffers (one has $m_1$ samples and the other $m_2$). Thus we can use Lemma \ref{lem:ED_avg} (which depends on the i.i.d. sampling that is ensured by Lemma \ref{lem:independence}) to get
\begin{align*}
\EED \norm{\tilde{A}_{m_1}^\tr (\tilde{A}_{m_2} \theta_{k} + \tilde{b}_{m_2})}^2 &=
\frac{1}{m_1m_2}\EED\norm{f'_{i,j}(x)}^2+ \left(1-\frac{1}{m_1m_2}\right) \norm{f'(x_t)}^2\\
&\ge \norm{f'(x_t)}^2 +\frac{1}{m_1m_2} \sigma_v^2 \\
&= \norm{A^\tr (A\theta_k+b)}^2+\frac{1}{m_1m_2} \sigma_v^2\\
&\ge \sigma_{\min}^2(A) \norm{A\theta_k+b}^2 +\frac{1}{m_1m_2} \sigma_v^2,
\end{align*}
where the first inequality is because of Lemma \ref{lem:g2andf2}.

Therefore, 
\begin{align*}
\min_{k=0, \ldots, t-1} 
 f(\theta_k) &= \min_{k=0, \ldots, t-1} 
 \norm{A\theta_k+b}^2 \\
 &\le \max\left\{\frac{2}{t \alpha \left(2- \alpha L_{\max} \right)\sigma^2_{\min}(A)}f (\theta_{0}) - \frac{1}{m_1m_2\sigma^2_{\min}(A)} \sigma_v^2, \, 0\right\}.
\end{align*}

\end{proof}
Theorem \ref{thm:rates_1_over_t} shows that out of the historical learning steps, we are guaranteed to find a moment with an $O(1/t)$ reduction of the initial loss.\footnote{It also shows that the minibatch update is helpful, because it enables more reduction than $O(1/t)$. For larger batch-sizes $m_1$ and $m_2$, this benefit grows smaller, indicating that smaller batch-sizes should converge faster. 
Although this is interesting, we found this is contradictory to our empirical results which we cannot explain why.} 

\citet{bo_gtd_finite} proved that certain variants of GTD and GTD2 converge at a rate of $O(t^{-1/4})$ with a high probability. The algorithm variants apply projections to the iterators to keep them bounded. The rate was proved by applying the rate analysis of the saddle point problem \citep{saddle_point_Nemirovski}. A key condition that guarantees this rate is the use of a {\em fixed} step-size, $1/\sqrt{t}$, by knowing the total number of iteration steps before hand. For example, if we want to learn $10000$ steps, at all the learning steps, the step-size is $0.01$. 

\citet*{dalal2018finite_twotimescale} established the convergence rates of a variant of GTD that projects the update back to a ball sparsely. With diminishing step-sizes $\alpha_t=t^{-(1-\tau)}$ and $\beta_t=t^{-2/3(1-\tau)}$, where $\kappa$ is some constant in $(0,1)$, they showed that the convergence rate is $O(t^{-1/3 + \kappa/3})$, which is roughly $O(t^{-1/3})$ at best. This is slightly faster than \citet{bo_gtd_finite}'s rate if $\kappa$ is small. One can understand that this is due to the use of a bigger step-size than the fixed $1/\sqrt{t}$ step-size.  
The rate also applies to GTD2 and TDC. This result was obtained by drawing inspiration from single-time-scale stochastic approximation \citep{borkar2008book}, in bounding the distance of the two-time-scale iterations to the trajectories that are generated by the O.D.E. 
An important condition for this distance to remain bounded is that the two step-sizes are scheduled to satisfy the two-time-scale condition.     

Later with the step-sizes $t^{-\alpha}$ and $t^{-\beta}$ respectively for the two iterators, which satisfy $0<\beta<\alpha<1$, they showed that the convergence rates are $O(t^{-\alpha/2})$ and $O(t^{-\beta/2})$ for the two iterators, and the bounds are tight \citep*{dalal2020tale_twotimescale}.\footnote{Note that in this paper, analysis of the GTD algorithm was applied with a projection operator very $2^i$ ($i=0, 1, \ldots$) steps to keep the update bounded. Our analysis does not use any projection. 
}  Given that GTD learns slower than GTD2 and TDC as found by empirical studies \citep{tdc,sutton2018reinforcement}, it is probably safe to say that GTD (without projection) converges no faster than this rate. 
In short, all the three GTD algorithms converge slower than $O(1/\sqrt{t})$. In fact, $O(1/\sqrt{t})$ is the theoretical rate limit of stochastic saddle-point problem \citep{bo_gtd_finite}. This means even if one uses advanced optimizers such as Stochastic Mirror-Prox \citep*{Mirror_prox}, GTD, GTD2 and TDC will not converge any faster than $O(1/\sqrt{t})$. 

In contrast, 
our Theorem \ref{thm:rates_1_over_t}  shows that the Impression GTD algorithm together with the three other GTD algorithms converge at least as fast as $O(1/t)$, much faster than GTD, GTD2 and TDC. \cite*{shangtong_imgtd} proved an $O(\xi(t)ln(t)/t)$ rate for their Direct GTD under the diminishing step-size that scales with $1/t$, where $\xi(t)$ is some slowly growing function such as $ln^2(t)$. This rate is almost $O(1/t)$, and it does not need to take the minimum over the historical learning steps.    

% \begin{assumption}\label{assumption:stepsize} 
% Let $\alpha_k>0$ be a sequence that satisfies $\sum_t \alpha_t^2 <\infty $ and $\sum_t \alpha_t =\infty$. 
% \end{assumption}



% The GTD paper \citep{gtd} contains the theorems and proofs for the cases of i.i.d. setting and the off-policy sub-sampling process individually. 
% There is actually no need to deal with them separately. In the following theorem, the distribution $\bold{d}$ is the state distribution of a data set that can be from any source, e.g., samples collected by a behavior policy or multiple policies that may be either behavior or target policies. 

% The proof for the following theorem is for the algorithm using the batch sizes equal to 1, which is easily generalized to any batch size. 


% \begin{thm} \label{thm:convergence_imgtd}
% Consider independence sampling with batch sizes $m_1=m_2=1$. 
% In particular, the algorithm maintains a buffer that inserts the samples up to the current time step according to the method in equations \ref{eq:buffer1} and \ref{eq:buffer2}.
% This means in the limit the buffer size grows to infinity as well. 
% Denote the distribution of the states in the buffer by $d_t$ at time step $t$. Assume $\lim_{t\to\infty}d_t$ exists, and let $\lim_{t\to\infty}d_t=d$. The feature matrix is $\Phi$, whose $i$th row is $\phi(i)^\tr$. Let the target policy be $\pi$. 


% The Impression GTD algorithm converges with probability one to the solution of a linear system, $A\theta + b=0$, where $A = \Phi^\tr D(\gamma P^\pi -I)\Phi$, and $b = \Phi^\tr D \bar{r}^\pi$. Here $D$ is the diagonal matrix with $D(i,i)=d(i)$, where $i\in \SS$; $P^\pi(i,j)=\sum_{a\in\AA}\pi(a|i)\PP(j|i,a)$; and $\bar{r}^\pi(i) = \sum_jP^\pi_{i,j}r(i,j) $. Assume that $A$ is non-singular. However, it is not necessarily negative definite.   
% \end{thm}
% \begin{proof}
% Let us start with the update, which can be written as
% \begin{align*}
% -\Delta \theta 
% &\propto \phi(\gamma\phi_2' - \phi_2)^\tr\theta + \phi r_2.\\
% &= (\gamma\phi'_1 - \phi_1) \phi_1^\tr \phi_2(\gamma\phi_2' - \phi_2)^\tr\theta + (\gamma\phi'_1 - \phi_1) \phi_1^\tr\phi_2 r_2.\\
% &=\left[(\gamma\phi'_1 - \phi_1)\phi_1^\tr \right]\phi_2\left[ (\gamma\phi_2' - \phi_2)^\tr\theta + r_2 \right],
% % &= (\gamma\phi'_1 - \phi_1) \phi_1^\tr \left[\phi_2(\gamma\phi_2' - \phi_2)^\tr\theta + \phi_2 r_2\right].\\
% \end{align*}
% where the second line is according to the definition of $\phi$. 

% Let us first understand the expected update of this algorithm, or the behavior of the algorithm in the ``average'' sense until time step $t$. Assume the buffer has samples $\{(s, s', r)\}$, whose number keeps growing as learning proceeds. Let the distribution of $s$ be $d_t$ at time step $t$, and the expectation operator conditioned on $d_t$ be $\EE_t$. At time step $t$, Impression GTD is a stochastic version of the following update:
% \begin{align*}
% \bar{\theta}_{\tau+1}
% &= \bar{\theta}_{\tau} -  \EE_t\left\{\Delta \theta|\theta_{\tau}\right\}\\
% &= \bar{\theta}_{\tau} - \alpha_\tau \EE_t\left\{ \left[(\gamma\phi'_1 - \phi_1)\phi_1^\tr \right]\phi_2\left[ (\gamma\phi_2' - \phi_2)^\tr\theta_{\tau} + r_2 \right]\right\}\\
% &= \bar{\theta}_{\tau} - \alpha_\tau \EE_t\left[(\gamma\phi'_1 - \phi_1)\phi_1^\tr \right] 
% \EE_t
%  \left[ \phi_2(\gamma\phi_2' - \phi_2)^\tr\theta_{\tau} + r_2 \right]\\
% &= \bar{\theta}_{\tau} - \alpha_\tau \tilde{A}_{1}^\tr(\tilde{A}_{2}\theta + \tilde{b}_{2}),
% \end{align*}
% where $\tilde{A}_1$ and $\tilde{A}_2$ are estimates of $A$ from the two buffers (by the empirical means), and $\tilde{b}_2$ is the estimate of $b$ from Buffer $B_2$. The second line is because of independence sampling. 
% The third line follows due to Lemma \ref{lem:independence}, and the last line is because $\EE
%  \left[ \phi(\gamma\phi' - \phi)^\tr\right]=A$ and $\EE
%   [\phi r]= b$, given a state distribution $d(s)$ where $\phi(s)$, the  feature vectors are computed, e.g., see \citep{tsi_td}. 

% Assume the two buffers have the same number of samples. According to the law of large numbers, we can re-write the last line by 
% \begin{align*}
% \bar{\theta}_{\tau+1}&= \bar{\theta}_{\tau} - \alpha_\tau (A+ O(2/t)I)^\tr\left[(A+O(2/t)I)\theta + (b+O(2/t) \bold{1})\right]\\
% &= \bar{\theta}_{\tau} - \alpha_\tau A^\tr(A\theta_\tau + b)\\
% & \quad\quad\,\,  - \alpha_\tau A^\tr\left[O(2/t)\theta + O(2/t) \bold{1}\right]\\
% & \quad\quad\,\,  - \alpha_\tau O(2/t)  \left[(A+O(2/t)I)\theta + (b+O(2/t) \bold{1})\right]. 
% \end{align*}
% Thus the second and third lines in the last equality show that the perturbation noises both diminish at a rate of $O(1/t)$. 

% The idea of the proof is to formulate the Impression GTD update using this expected update plus some noise $\epsilon_t$. 
% \begin{align*}
% \theta_{t+1} &= \theta_t - \alpha_t \left[ A^\tr(A\theta_t +b) + \epsilon_t \right].
% \end{align*}
% It is straightforward to show that $\EE[\epsilon_t]=0$ because $\EE[\epsilon_t(A)]=0$ and $\EE[\epsilon_t(b)]=0$, where
% \[
% \epsilon_t(A)= A^\tr A-  \left[(\gamma\phi'_1 - \phi_1)\phi_1^\tr \right]\phi_2  (\gamma\phi_2' - \phi_2)^\tr,
% \]
% and 
% \[
% \epsilon_t(b)= A^\tr b -  \left[(\gamma\phi'_1 - \phi_1)\phi_1^\tr \right]\phi_2 r_2.
% \]
% Furthermore, 
% $\EE\norm{\epsilon_t(A)}$ and $\EE\norm{\epsilon_t(b})$ are both bounded. 
% Standard results from stochastic approximation applies \citep{borkar2000ode,yin1997stochastic}, e.g., Theorem 2 of \citep{tsi_td} applies because $-A^\tr A$ is guaranteed to be negative definite due to that $A$ is non-singular.\footnote{The process does not need to be Markovian though, as required by \citet{tsi_td}'s Theorem 2.} With the bounded moments from Assumption \ref{assumption:phir} and the step-size sequence in Assumption \ref{assumption:stepsize},  
% it follows that $\theta_t$ converges to the solution with probability one to the linear system $A^\tr(A\theta^*+b)=0$, which also satisfies $A\theta^* +b=0$ because $A$ is a non-singular, square matrix in this setting.  
% \end{proof}

We further show that the four algorithms actually converge in a linear rate to a biased solution. For that purpose, we first establish a convergence rate result for SGD, under the $L$-$\lambda$ smoothness. 

\begin{definition}[$L$-$\lambda$ smoothness] \label{def:L-lambda-smooth}  
If for all $x\in \RR^d$, function $f$ and $\mathcal{D}$ satisfy
\[
\EE_{\mathcal{D}}\norm{g_t(x)}^2 \le 2L (f(x) - f(x^*)) + \lambda \norm{x-x^*}^2 +  \sigma^2, 
\] 
we say that $f$ is $L$-$\lambda$ smooth smooth under distribution $\mathcal{D}$, or simply, $(f, \mathcal{D})\sim L$-$\lambda(  \sigma^2)$,
\end{definition}
This new definition of expected smoothness has a background in our Impression GTD setting. 
Note that in this definition, $\sigma^2$ can be any positive real number. \citeauthor{gower2019sgd_general} used $\sigma^2=\EED\norm{g_t(x^*)}^2$. %and they showed that in the over-parameterization case, $\sigma^2=0$.   
We will show that in our analysis, $\sigma^2$ is some different number.  
The new definition adds a term of $\lambda\norm{x-x^*}$ to allow for convergence analysis of GTD algorithms. This term improves the expected smoothness to be more {\em noise tolerant}, and thus the induced smoothness is more general. %The following three lemmas are independent of the SGD setting.     


\begin{lem}\label{lem:utrongly_norm_grad}
If $f$ is $\mu$-strongly quasi-convex, then we have for any $x\in \RR^d$, 
\[
\norm{f'(x)} \ge \mu \norm{x-x^*}.  
\]
\end{lem}
Appendix \ref{appendix:u_norm_grad} has the proof. 

\begin{lem} \label{lem:ES_qLsmooth}
    If $(f, \mathcal{D})$ $\sim$ $L$-$\lambda$ $( \sigma^2)$ for some $\sigma^2\ge 0$, then for any $x\in \RR^d$, we have 
    \[
   f(x) - f(x^*) \ge \frac{\norm{f'(x)}^2- \lambda\norm{x-x^*}^2 - (\sigma^2-\sigma_v^2)}{2L}, 
    \]
    where $\sigma_v^2 = \min_{x}\EED \norm{g_t(x)-f'(x)}^2$.
\end{lem}
The proof is in Appendix \ref{appendix:es_qls}.


The following theorem improves \citet*{gower2019sgd_general}'s Theorem 3.1 by removing the factor of two in the bias term because of the use of a refined definition of expected smoothness. The rate is also tightened for a faster rate with a $\mu^2$ term. Analysis on SGD usually drops $\EE \norm{\nabla f(x_t)}^2$ by relating it to $f(x_t)-f(x^*)$ first, and then drops $f(x_t)-f(x^*)$ due to $L$-smoothness and $f(x)\ge f(x^*)$, e.g., see \citep{gower2019sgd_general} and \citep*{sps_stepsize}.
This means their bounds on the convergence rate can be further tightened.  
Our proof keeps $f(x_t)-f(x^*)$, relates it to $\EE \norm{\nabla f(x_t)}^2$, and bounds the latter.  
This can be done by noting that $f(x) - f(x^*)$ can be lower bounded by the norm of the gradient together with the perturbation and the constant. 
By using Lemma \ref{lem:utrongly_norm_grad} for the strongly quasi-convexity of $f$, we have further
$\EE \norm{\nabla f(x_t) }^2\ge \mu^2 \EE \norm{x_t-x^*}^2$. 
\begin{thm}\label{thm:sgd_rate}
Assume $(f, \mathcal{D})$ $\sim$ $L\mbox{-}\lambda(\sigma^2)$ and $f$ is $\mu$-strongly quasi-convex. For SGD with batch update:
\[
x_{t+1} = x_t - \alpha_t\avg,
\]
we have 
\begin{align*}
\EE \norm{x_{t+1}-x^*}^2  &\le   \left[1-\left(\mu -\frac{\lambda}{L} \right) \alpha_t-\mu^2\alpha_t\left(\frac{1}{L}-\alpha_t  \right)\right] \EE\norm{\Delta_t}^2  + \frac{\alpha_t }{L}(\sigma^2-\sigma^2_v) + \frac{\alpha_t^2\sigma^2_v}{m},
\end{align*}
A linear convergence rate 
can be guaranteed for $\lambda\le L\mu$. 
Specifically, for a constant step-size $\alpha\le \frac{1}{L}$, we have 
\[
\EE \norm{x_{t}-x^*}^2 \le q^t\EE \norm{x_0-x^*}^2 +  \alpha\frac{   m(\sigma^2-\sigma^2_v) + L\alpha\sigma^2_v }{L m\left[\left(\mu -\frac{\lambda}{L} \right)+\mu^2\left(\frac{1}{L}-\alpha\right)\right]},
\]
where 
\[
q= 1-\left(\mu -\frac{\lambda}{L} \right) \alpha-\mu^2\alpha\left(\frac{1}{L}-\alpha  \right).
\]
\end{thm}
\begin{proof}
Let $\Delta_t = x_t - x^*$.
We have $\Delta_{t+1}= \Delta_t -  \alpha_t \avg$. 
Taking the squared $\ell_2$ norm and the conditional expectation gives 
\begin{align*}
\EED  \norm{\Delta_{t+1}}^2  & = \EED(\Delta_t -  \alpha_t \avg)^\top (\Delta_t -  \alpha_t \avg\nonumber \\
&= \norm{\Delta_t}^2  - 2 \alpha_t \EED\left[ \avg^\top \Delta_t|x_t\right] + \alpha_t^2 \EED \norm{\avg}^2\nonumber\\
&= \norm{\Delta_t}^2  - 2 \alpha_t \EED\left[ \avg\right]^\top \EED\left[\Delta_t\right] + \alpha_t^2 \EED \norm{\avg}^2\nonumber\\
&=
\norm{\Delta_t}^2  - 2 \alpha_t  \nabla f(x_t)^\top \Delta_t + \alpha_t^2 \EED \norm{\avg}^2
\end{align*}
where the third line is because $\Delta_t$ is independent of $\avg$ given $x_t$. The last line is due to the expected form of $f$, which gives $\EE[ g_t(x)]=\nabla f(x)$ for any $x$. 

Taking expectation over $x_t$ gives
\begin{align*}
\EE \norm{\Delta_{t+1}}^2 
&=\EE\norm{\Delta_t}^2  - 2 \alpha_t  \EE\left[\nabla f(x_t)^\top \Delta_t\right] + \alpha_t^2\EE \EED \norm{\avg}^2\\
&\le 
\EE \norm{\Delta_t}^2  - 2 \alpha_t  \EE\left(f(x_t)-f(x^*)+\frac{\mu}{2}\norm{\Delta_t}^2\right)  + \alpha_t^2 \EE\EED \norm{\avg}^2
\end{align*}
where the inequality is by the $\mu$-strongly quasi-convexity of $f$.


We have
\begin{align*}
\EE \norm{\Delta_{t+1}}^2 &\le (1-\mu\alpha_t) \EE \norm{\Delta_t}^2  - 2 \alpha_t   \EE\left(f(x_t)-f(x^*)\right)  + \alpha_t^2 \EE\EED \norm{\avg}^2\\
&= (1-\mu\alpha_t) \EE \norm{\Delta_t}^2  - 2\alpha_t  \EE\left(f(x_t)-f(x^*)\right) \\
& \quad + \alpha_t^2 \left( \frac{1}{m}\EE\EED\norm{g_t(x)}^2+ \left(1-\frac{1}{m}\right) \EE\norm{f'(x_t)}^2 \right)\\
&\le (1-\mu\alpha_t) \EE \norm{\Delta_t}^2  - 2\alpha_t\EE  \left(f(x_t)-f(x^*)\right) \\
& \quad + \alpha_t^2 \left( \frac{1}{m}\left( 2L\EE(f(x_t)-f(x^*)) + \EE \lambda\norm{\Delta_t}^2 + \sigma^2 \right)+ \left(1-\frac{1}{m}\right) 
\EE\norm{f'(x_t)}^2 \right)\\
&= \left(1-\mu\alpha_t +\frac{\lambda \alpha_t^2}{m} \right) \EE \norm{\Delta_t}^2 - 2\alpha_t\left(1 - \frac{\alpha_tL}{m} \right)\EE(f(x_t)-f(x^*)) \\
& \quad + \alpha_t^2\left(1-\frac{1}{m}\right) \EE\norm{f'(x_t)}^2 + \frac{\alpha_t^2\sigma^2}{m}
\end{align*}
in which line 2 is by Lemma \ref{lem:ED_avg}, and line 3 is according to $L$-$\lambda$ smoothness.
The above holds for any $\alpha_t$. 
Then with $\alpha_t\le \frac{m}{L}$,
\begin{align*}
\EE \norm{\Delta_{t+1}}^2&\le \left(1-\mu\alpha_t+ \frac{\lambda \alpha_t^2}{m}  \right) \EE \norm{\Delta_t}^2 +
\alpha_t^2\left(1-\frac{1}{m}\right) \EE\norm{f'(x_t)}^2 + \frac{\alpha_t^2\sigma^2}{m} \\
& \quad  - 2\alpha_t\left(1 - \frac{\alpha_tL}{m} \right)\frac{1}{2L}\left( \EE \norm{f'(x_t)}^2 -\EE\lambda \norm{\Delta_t}^2 -(\sigma^2 - \sigma^2_v) \right)\\
% &= \left(1-\mu\alpha_t + \frac{\lambda\alpha_t}{L}  \right) \EE \norm{\Delta_t}^2  - \alpha_t\left( \frac{1}{L} - \alpha_t\right)   \EE \norm{f'(x_t)}^2 \\
% &\quad +  \alpha_t\left(1 - \frac{\alpha_tL}{m} \right)\frac{1}{L}(\sigma^2-\sigma^2_v)
% +  \frac{\alpha_t^2\sigma^2 }{m}  \\
&= \left(1-\mu\alpha_t+ \frac{\lambda \alpha_t}{L}\right) \EE \norm{\Delta_t}^2  - \alpha_t\left( \frac{1}{L} -\alpha_t\right)   \EE \norm{f'(x_t)}^2 +   \frac{\alpha_t }{L}(\sigma^2-\sigma^2_v) + \frac{\alpha_t^2\sigma^2_v}{m},  
\end{align*}
where line 1 is by Lemma \ref{lem:ES_qLsmooth}.
Furthermore, if $\alpha_t\le \frac{1}{L}$, we can use Lemma \ref{lem:utrongly_norm_grad} to get 
\begin{align*}
\EE \norm{\Delta_{t+1}}^2&\le 
\left(1-\mu\alpha_t+ \frac{\lambda \alpha_t}{L}\right) \EE \norm{\Delta_t}^2  - \alpha_t\left( \frac{1}{L} -\alpha_t\right)   \mu^2 \EE\norm{\Delta_t}^2 +   \frac{\alpha_t }{L}(\sigma^2-\sigma^2_v) + \frac{\alpha_t^2\sigma^2_v}{m} \\
&= 
\left(1-\left(\mu -\frac{\lambda}{L} \right)\alpha_t  - \mu^2 \alpha_t\left( \frac{1}{L} -\alpha_t\right)  \right) \EE \norm{\Delta_t}^2    +   \frac{\alpha_t }{L}(\sigma^2-\sigma^2_v) + \frac{\alpha_t^2\sigma^2_v}{m},
\end{align*}

In the constant step-size case, with the choice of $\alpha\le \frac{1}{L}$, a linear rate is guaranteed because
\begin{align*}
0\le q&\stackrel{def}{=}1-\left(\mu -\frac{\lambda}{L} \right)\alpha  - \mu^2 \alpha\left( \frac{1}{L} -\alpha\right) \le 1- \mu^2 \alpha\left( \frac{1}{L} -\alpha\right)\le 1,
\end{align*}
due to that $\lambda\le L\mu$.

We run the recursion repeatedly until the beginning and get
\begin{align*}
\EE\norm{\Delta_{t}}^2& \le q^t \EE\norm{\Delta_0}^2 +  \left( \frac{\alpha }{L}(\sigma^2-\sigma^2_v) + \frac{\alpha^2\sigma^2_v}{m} \right)  \sum_{s=0}^\infty q^s \\
&= q^t \EE\norm{\Delta_0}^2 + \alpha\frac{   m(\sigma^2-\sigma^2_v) + L\alpha\sigma^2_v }{L m\left[\mu - \frac{\lambda}{L}+\mu^2\left(\frac{1}{L}-\alpha\right)\right]}.
\end{align*} 
\end{proof}

This theorem extends Theorem 3.1 of \citep{gower2019sgd_general} in three ways. First, the SGD rate is established under $L$-$\lambda$ smoothness, which is weaker than expected smoothness. Second, the linear rate is tightened with a $\mu^2$ term even for $\lambda=0$. Third, the bias term is more refined in the numerator too, with the difference between $\sigma^2$ and the minimum variance. 


In the extreme case of $\lambda=L\mu$, although it guarantees a linear rate, for problems where $\mu$ is small, the rate can be very slow.
The factor $\lambda$ can be understood as the amount of perturbation to the expected smoothness condition. In particular, the more perturbation, the slower the rate. If the perturbation reduces, e.g.,  
if $\lambda \le (1-\rho) L\mu$ where $\rho \in [0, 1]$, then a much faster rate can be achieved:
\[
q\le 1- \rho\mu\alpha -\mu^2 \alpha\left( \frac{1}{L} -\alpha\right). 
\]
Luckily, as we will show later, in our Impression GTD setting, $\lambda$ can be reduced by increasing the batch sizes.  

%review later in the SGD paper
% Many SGD analysis works assume that $\EE \norm{g_t(x_t)}^2$ is bounded by a constant. For example, $\sigma$-bounded gradient \citep{reddi2016stochastic}.  \citet{ghadimi2013stochastic} assumes the global (unconditional) variance for non-convex but $L$-smooth functions.  Their final bounds have no batch size and it thus does not reveal its effect for variance reduction.



% The following theorems give the convergence rates of all the four GTD algorithms. 


%using two terms
% \begin{align*}
% &\quad \norm{f'_{i,j}(x) - f'_{i,j}(y)}\\ &= \norm{(\gamma \phi_{i+1}-\phi_i) \phi_i^\tr \phi_j (\gamma \phi_{j+1}-\phi_j)^\tr (x-y)  + 
% (\gamma \phi_{j+1}-\phi_j) \phi_j^\tr \phi_i (\gamma \phi_{i+1}-\phi_i)^\tr (x-y)} \\
% &\le \norm{(\gamma \phi_{i+1}-\phi_i) \phi_i^\tr \phi_j (\gamma \phi_{j+1}-\phi_j)^\tr (x-y)}  + \norm{
% (\gamma \phi_{j+1}-\phi_j) \phi_j^\tr \phi_i (\gamma \phi_{i+1}-\phi_i)^\tr (x-y)} \\
% &\le \norm{(\gamma \phi_{i+1}-\phi_i) \phi_i^\tr \phi_j (\gamma \phi_{j+1}-\phi_j)^\tr}\norm{x-y}  + \norm{
% (\gamma \phi_{j+1}-\phi_j) \phi_j^\tr \phi_i (\gamma \phi_{i+1}-\phi_i)^\tr}\norm{ x-y} \\
% &= L_{i,j} \norm{x-y}.
% \end{align*}

 
% We first give a general SGD rate result. For that purpose, we start with a variant of expected smoothness that was proposed by \citet{gower2019sgd_general}. After that, a few lemmas are proved, followed by the main SGD rate result under the new smoothness condition.


Now we are ready to prove the linear rate result of Impression GTD. This is achieved by applying the SGD rate in Theorem \ref{thm:sgd_rate}.
First we introduce a lemma to show that in the Impression GTD problem, the loss function (NEU) and the independence sampling is $L$-$\lambda$ smooth, which is required by Theorem \ref{thm:sgd_rate}. 

\begin{lem}\label{thm_item:Llambdasmoooth}
Let $\mu=\sigma^2_{\min}(A)$, i.e., the minimum singular values of $A$. Assume $\mu>0$. Let Assumption \ref{assumption:phir} hold. 
Let $\Sigma_A$ be the matrix of the standard deviations of the rank-1 sample matrix $\phi(\gamma \phi'-\phi)^\tr$. That is, $\Sigma_A(i,j)= \sqrt{Var(\phi(i)(\gamma \phi(j)-\phi(j)))}$. Let $\Sigma_b$ be the vector of the standard deviations of $\phi r$, i.e., $\Sigma_b(i)=\sqrt{Var(\phi(i)r)}$.\footnote{These are all properties of the considered MDP, feature functions, the behavior policy and the target policy.} 

Define the following constants due to NEU and the independence sampling, respectively:
\[
L_1 =4\left(\frac{\norm{\Sigma_A}^2}{m_1} +  \norm{A}^2\right), \quad \sigma^2 = 16\left(\frac{\norm{\Sigma_A}^2}{m_1} +  \norm{A}^2\right) \left(\frac{\norm{\Sigma_A}^2}{m_2} \norm{\theta^*}^2 + \frac{\norm{\Sigma_b}^2}{m_2}\right).
\]
and
\[
L_2 = \frac{\norm{\Sigma_A}^2}{m_2}, \quad \lambda = \frac{2\norm{\Sigma_A}^4}{m_1m_2}; 
\]
The NEU objective function and the independence sampling satisfy the $L$-$\lambda(\sigma^2)$ smoothness with $L=L_1+L_2$, $\lambda=\lambda$, and $\sigma^2=\sigma^2$.  
\end{lem}

\begin{proof}

First we have $x^\tr H x = x^\tr H^\tr x$ holds even for a non-symmetric matrix $H$. This is because 
\begin{align*}
x^\tr H x  = \sum_{i}\sum_j H_{i,j}x_ix_j= \sum_{j}\sum_i H_{i,j}x_ix_j= \sum_{i}\sum_j H_{j,i}x_ix_j=x^\tr H^\tr x,
\end{align*}
where equality 2 is by switching the order of the two sums, and equality 3 is by swapping $i$ and $j$. Thus $\norm{Hx} = \norm{H^\tr x}$ holds for any real matrix $H$ and real vector $x$. This equality is crucial in the analysis below. 

We have
\begin{align*}
&\EED\norm{\tilde{A}_{m_1}^\tr (\tilde{A}_{m_2} \theta_t + \tilde{b}_{m_2})}^2 \\
&= \EED\norm{(\tilde{A}_{m_1}-A+A) ^\tr \left((\tilde{A}_{m_2}-A) \theta_t + {A} \theta_t + b + (\tilde{b}_{m_2}-b)\right)}^2\\
&= \EED\norm{(\Delta^A_{m_1} + A)^\tr \left(\Delta^A_{m_2}(\theta_t-\theta^*) + \Delta^A_{m_2}\theta^* + (A\theta_t +b) + \Delta^b_{m_2} \right)}^2\\
&\le 2\EED\norm{(\Delta^A_{m_1} + A)^\tr \Delta^A_{m_2}(\theta_t-\theta^*)}^2 + 2\EED \norm{(\Delta^A_{m_1} + A)^\tr}^2 \norm{ \Delta^A_{m_2}\theta^*  +  (A\theta_t +b) + \Delta^b_{m_2} )}^2
\end{align*}
where we define $\Delta^A_{m}= \tilde{A}_m -A$, and $\Delta^b_{m}= \tilde{b}_m -b$.
Let's first examine the second term in the above equation: 
\begin{align*}
&\EED\norm{(\Delta^A_{m_1} + A)}^2 \norm{ \Delta^A_{m_2}\theta^*  +  (A\theta_t +b) + \Delta^b_{m_2} )}^2\\
&=\EED\norm{(\Delta^A_{m_1} + A)}^2 \EED \norm{ \Delta^A_{m_2}\theta^*  +  (A\theta_t +b) + \Delta^b_{m_2} )}^2\\
&\le 2
\EED \norm{\Delta^A_{m_1} + A}^2 \left(\EED \norm{ \Delta^A_{m_2}\theta^* + \Delta^b_{m_2}}^2   + \norm{A\theta_t +b}^2 \right)\\
&\le 8 \left(\frac{\norm{\Sigma_A}^2}{m_1} +  \norm{A}^2\right) \left(2\EED\norm{\Delta^A_{m_2}}^2 \norm{\theta^*}^2 + 2\EED \norm{\Delta^b_{m_2}}^2     + f(\theta_t) \right)\\
&\le \underbrace{16\left(\frac{\norm{\Sigma_A}^2}{m_1} +  \norm{A}^2\right) \left(\frac{\norm{\Sigma_A}^2}{m_2} \norm{\theta^*}^2 + \frac{\norm{\Sigma_b}^2}{m_2}\right)}_\text{$\sigma^2$}     + \underbrace{8 \left(\frac{\norm{\Sigma_A}^2}{m_1} +  \norm{A}^2\right)}_\text{2$L_1$}\left(f(\theta_t) -f(\theta^*) \right)\\
& = \sigma^2 + 2L_1 \left(f(\theta_t)-f(\theta^*) \right), 
\end{align*}
where the equality is due to the independence sampling. The first inequality uses Jensen's inequality. The second inequality uses Jensen's inequality, $Var(\frac{1}{m_1}\sum_{i=1}^m X_i) = \frac{1}{m_1}Var(X)$, where $\{X_i\}$ are i.i.d. samples of the random variable $X$; and the triangle inequality. The third inequality uses the above variance relationship again. 

The first term is 
\begin{align*}
&\quad\EED\norm{(\Delta^A_{m_1} + A)^\tr \Delta^A_{m_2}(\theta_t-\theta^*)}^2\\
&\le 2\EED\norm{{\Delta^A_{m_1}}^\tr \Delta^A_{m_2}(\theta_t-\theta^*)}^2 + 2\EED \norm{A^\tr \Delta^A_{m_2}(\theta_t-\theta^*)}^2\\
&\le 2\EED\norm{{\Delta^A_{m_1}}}^2 \norm{ \Delta^A_{m_2}}^2 \norm{\theta_t-\theta^*}^2 + 2\EED \norm{ {\Delta^A_{m_2}}^\tr A(\theta_t-\theta^*)}^2\\
&\le 
2\EED\norm{{\Delta^A_{m_1}}}^2 \EED\norm{ \Delta^A_{m_2}}^2 \norm{\theta_t-\theta^*}^2 + 2\EED \norm{\Delta^A_{m_2}}^2\norm{A(\theta_t-\theta^*)}^2\\
&= 2\frac{\norm{\Sigma_A}^2}{m_1}\frac{\norm{\Sigma_A}^2}{m_2}\norm{\theta_t-\theta^*}^2  + 2\frac{\norm{\Sigma_A}^2}{m_2}\norm{A\theta_t+b}^2\\
&= \underbrace{2\frac{\norm{\Sigma_A}^4}{m_1m_2}}_\text{$\lambda$}\norm{\theta_t-\theta^*}^2  + 2\underbrace{\frac{\norm{\Sigma_A}^2}{m_2}}_\text{$L_2$}\left(f(\theta_t) -f(\theta^*)  \right)\\
&= \lambda \norm{\theta_t-\theta^*}^2  + 2L_2 \left(f(\theta_t) -f(\theta^*)\right). 
\end{align*}
The first inequality uses Jensen's inequality. The second inequality uses the triangle inequality, and $\norm{Hx}= \norm{H^\tr x}$ due to that $x^\tr H x = x^\tr H^\tr x$. The third inequality uses the independence sampling and the triangle inequality. The first equality uses the variance equality that was used in proving the second term, and $b= -A\theta^*$. 

Therefore, $\EED\norm{\tilde{A}_{m_1}^\tr (\tilde{A}_{m_2} \theta_t + \tilde{b}_{m_2})}^2 \le 2(L_1+L_2) \left(f(\theta_t) -f(\theta^*)\right) + \lambda \norm{\theta_t-\theta^*}^2  +\sigma^2$. This proves that the NEU objective function and the independence sampling satisfy the $L$-$\lambda(\sigma^2)$ smoothness with the specified constants. 

\end{proof}

The following theorem is shows that, for Impression GTD, the linear rate can be obtained by large batch sizes, and we show a sufficient choice is $m_1=m_2\ge \lceil \frac{1}{\sqrt{2\mu}}\frac{\norm{\Sigma_A}^2}{\norm{A}^2} \rceil$. 
For Expected GTD, linear rate can be achieved after a key metric, $\frac{\norm{\Sigma_A}^2}{t^2}$ gets small for Expected GTD. For A$^\tr$TD and \oneexptd, the rate becomes linear once $\frac{\norm{\Sigma_A}^2}{t}$ gets small. This can be understood as that after we have a big enough number of samples that is proportional to the variance of the problem (or simply put, our buffers are {\em representative} of the true data distribution in the sense of the variances), the algorithms converge fast. The results also show that Expected GTD is faster than \oneexptd and A$^\tr$TD, \oneexptd is faster than A$^\tr$TD. 

\begin{thm}[Conv. Rates of Impression GTD, Expected GTD, A$^\tr$TD, \oneexptd]\label{thm:rates_all}  

We have the following convergence rate results. 

\begin{enumerate}

\item \label{thm_item:imGTD}
Impression GTD (\ref{eq:imgtd}). 
With batch sizes $m_1=m_2 \ge \lceil \frac{1}{\sqrt{2\mu}}\frac{\norm{\Sigma_A}^2}{\norm{A}} \rceil =m$,\footnote{The $1/\norm{A}$ can be roughly interpretted as the condition number of NEU. Thus this shows that the batch sizes should increase with the condition number of NEU and the variances of the feature transitions. The constant $\frac{1}{\sqrt{\mu}}$ is a good sign because it is much smaller than $1/\mu$, if $\mu$ is very small. }
the algorithm converges linearly and
the rate is given by Theorem \ref{thm:sgd_rate} by using a step-size
\[
\alpha \le \frac{1}{5\frac{\norm{\Sigma_A}^2}{m} +  4\norm{A}^2}.
\]  

\item \label{thm_item:expGTD}
Expected GTD (\ref{eq:expectedGTD}). 
There exists $t_0$, such that $t>t_0$, we have
$\frac{2\norm{\Sigma_A}^2}{t}\le \epsilon$, and with $\alpha \le \frac{1}{4\norm{A}^2}$, 
\begin{align*}
\EE \norm{x_{t+1}-x^*}^2  &\le   q \EE\norm{x_{t}-x^*}^2  + \frac{\alpha }{4\norm{A}^2}(\sigma^2-\sigma^2_v) + \frac{4\alpha^2\sigma^2_v}{t^2},
\end{align*}
where 
\[
q= 1- \mu \alpha - \mu^2  \alpha\left(\frac{1}{4\norm{A}^2+5\epsilon} -\alpha \right)  + \frac{2\alpha}{4\norm{A}^2}\epsilon^2.   
\]

 

\item \label{thm_item:attd} 
A$^\tr$TD (\ref{eq:attd}). For $t>t_0$ such that
$\frac{\norm{\Sigma_A}^2}{t}\le \epsilon$, with $\alpha\le \frac{1}{\max\{ 4\norm{A}^2, \norm{\Sigma_A}^2\}}$, we have
\begin{align*}
\EE \norm{x_{t+1}-x^*}^2  &\le   q \EE\norm{x_{t}-x^*}^2  + \frac{\alpha }{\max\{ 4\norm{A}^2, \norm{\Sigma_A}^2\}}(\sigma^2-\sigma^2_v) + \frac{\alpha^2\sigma^2_v}{t},
\end{align*}
where 
\[
q=  1- \mu \alpha - \mu^2  \alpha\left(\frac{1}{4\norm{A}^2 + {\norm{\Sigma_A}^2} + 4\epsilon} -\alpha \right)  + \frac{\norm{\Sigma_A}^2}{\max\{ 4\norm{A}^2, \norm{\Sigma_A}^2\}}\alpha\epsilon.  
\]

\item \label{thm_item:r1etd} 
\oneexptd (\ref{eq:R1-GTD}).\footnote{It is straightforward to extend this result to the shrinked R1-GTD algorithm that is discussed in Section \ref{sec:minibatchPE}. The result remains the same by just replacing $t$ with $m_2$ and requiring $m_2$ to be sufficiently large. Similarly, this can be done for a shrinked version of A$^\tr$TD. }
After $t>t_0$ such that
$\frac{\norm{\Sigma_A}^2}{t}\le \epsilon$, the algorithm converges linearly with $\alpha \le \frac{1}{4\left({\norm{\Sigma_A}^2} +  \norm{A}^2\right)}$: 
\begin{align*}
\EE \norm{x_{t+1}-x^*}^2  &\le   q \EE\norm{x_{t}-x^*}^2  + \frac{\alpha }{4\left({\norm{\Sigma_A}^2} +  \norm{A}^2\right)}(\sigma^2-\sigma^2_v) + \frac{\alpha^2\sigma^2_v}{t},
\end{align*}
where 
\[
q= 1- \mu \alpha - \mu^2  \alpha\left(\frac{1}{4\left({\norm{\Sigma_A}^2} +  \norm{A}^2\right) + \epsilon} -\alpha \right)  + \frac{\norm{\Sigma_A}^2}{4\left({\norm{\Sigma_A}^2} +  \norm{A}^2\right)}\alpha\epsilon.   
\]


\end{enumerate}

\end{thm}
\begin{proof}
\ref{thm_item:Llambdasmoooth}. 


\ref{thm_item:imGTD}. Impression GTD.  
Consider $m_1=m_2=m$.  
With 
\[
m \ge \left\lceil \frac{1}{\sqrt{2\mu}}\frac{\norm{\Sigma_A}^2}{\norm{A}} \right\rceil, 
\]
we have 
\[
\frac{4\norm{A}^2}{\norm{\Sigma_A}^2}m^2  + 5m -\frac{2\norm{\Sigma_A}^2}{\mu}> 0 
\]
This gives
\begin{align*}
 \frac{2\norm{\Sigma_A}^4}{\mu} &< 4\norm{A}^2m^2  + 5{\norm{\Sigma_A}^2}m \\
 &= 4\left(\norm{A}^2 + \frac{\norm{\Sigma_A}^2}{m} \right)m^2+\frac{\norm{\Sigma_A}^2}{m}m^2 \\
&=
{(L_1 + L_2)m^2}, 
\end{align*}
or equivalently, $\lambda < (L_1+L_2)\mu=L\mu$. Thus Theorem \ref{thm:sgd_rate} is applicable. The step-size condition can be derived by requiring that $\alpha \le \frac{1}{L}$. 

Next let's get the $\mu$ constant in the context of Impression GTD. 
Because $f'(\theta) = A^\tr (A\theta + b)$, 
we have  
\begin{align*}
 (x-y)^\tr (f'(x) -f'(y)) &= (x-y)^\tr  A^\tr A (x-y)\ge \sigma_{\min}^2(A)\norm{(x-y)}^2. 
\end{align*}
Thus $\mu = \sigma_{\min}^2(A)$. Thus we can apply Theorem \ref{thm:sgd_rate} and completes the proof for Impression GTD. 

\ref{thm_item:expGTD}. Expected GTD.  
For a sufficiently large $t>t_0$, we have
$\frac{2\norm{\Sigma_A}^2}{t}\le \epsilon$.
Note that 
\[
4\norm{A}^2\le L_1<L=L_1+L_2\le L_1 + \epsilon\le  4\norm{A}^2 + 5\epsilon,
\] 
which gives
\begin{align*}
\frac{\lambda}{L} &\le \frac{8\norm{\Sigma_A}^4}{4\norm{A}^2t^2} \le \frac{\epsilon^2}{2\norm{A}^2}; \quad -\frac{1}{L} \le - \frac{1}{4\norm{A}^2+5\epsilon}.
\end{align*}
According to Theorem \ref{thm:sgd_rate}, with $\alpha \le \frac{1}{4\norm{A}^2}$, 
\begin{align*}
\EE \norm{x_{t+1}-x^*}^2  &\le   q \EE\norm{\Delta_t}^2  + \frac{\alpha_t }{L}(\sigma^2-\sigma^2_v) + \frac{4\alpha_t^2\sigma^2_v}{t^2},
\end{align*}
where for the linear rate we have
\begin{align*}
q&=1-\left(\mu -\frac{\lambda}{L} \right)\alpha  - \mu^2 \alpha\left( \frac{1}{L} -\alpha\right) \\
&\le 1- \mu \alpha - \mu^2  \alpha\left(\frac{1}{4\norm{A}^2+5\epsilon} -\alpha \right)  + \frac{\alpha}{2\norm{A}^2}\epsilon^2.   
\end{align*}

\ref{thm_item:attd}. A$^\tr$TD. $m_1=t$ and $m_2=1$.
Note that $L_1$ still has a diminishing term but $L_2$ itself does not any more: 
\[
\quad L_1 = 4\left(\frac{\norm{\Sigma_A}^2}{t} +  \norm{A}^2\right), \quad 
L_2 = {\norm{\Sigma_A}^2}, \quad \lambda = \frac{2\norm{\Sigma_A}^4}{t},
\]
For $t>t_0$ such that
$\frac{\norm{\Sigma_A}^2}{t}\le \epsilon$, we have
\[
\max\{ 4\norm{A}^2, \norm{\Sigma_A}^2\}<L=L_1+L_2 \le  4\norm{A}^2 + {\norm{\Sigma_A}^2} + 4\epsilon.
\] 
Thus 
\begin{align*}
\frac{\lambda}{L} &< \frac{2\norm{\Sigma_A}^4}{\max\{ 4\norm{A}^2, \norm{\Sigma_A}^2\}t} \le \frac{2\norm{\Sigma_A}^2}{\max\{ 4\norm{A}^2, \norm{\Sigma_A}^2\}}\epsilon; \quad -\frac{1}{L} \le - \frac{1}{4\norm{A}^2 + {\norm{\Sigma_A}^2} + 4\epsilon}.
\end{align*}
Therefore, with $\alpha\le \frac{1}{\max\{ 4\norm{A}^2, \norm{\Sigma_A}^2\}}$,  the linear rate for A$^\tr$TD satisfies
\[
q= 1- \mu \alpha - \mu^2  \alpha\left(\frac{1}{4\norm{A}^2 + {\norm{\Sigma_A}^2} + 4\epsilon} -\alpha \right)  + \frac{2\norm{\Sigma_A}^2}{\max\{ 4\norm{A}^2, \norm{\Sigma_A}^2\}}\alpha\epsilon.   
\]
The bias in the rate can be bounded according to the lower bound of $L$. 

\ref{thm_item:r1etd}. \oneexptd. $m_1=1$ and $m_2=t$. The constants are now 
\[
L_1= 4\left({\norm{\Sigma_A}^2} +  \norm{A}^2\right), \quad L_2 =\frac{\norm{\Sigma_A}^2}{t}; \quad \lambda =  \frac{2\norm{\Sigma_A}^4}{t}.
\]
and the lower and upper bounds of $L$ are thus
\[
4\left({\norm{\Sigma_A}^2} +  \norm{A}^2\right)\le L_1<L=L_1+L_2\le L_1 + \epsilon\le  4\left({\norm{\Sigma_A}^2} +  \norm{A}^2\right) + \epsilon,
\] 
which gives
\begin{align*}
\frac{\lambda}{L} &< \frac{2\norm{\Sigma_A}^4}{4\left({\norm{\Sigma_A}^2} +  \norm{A}^2\right)t} \le \frac{\norm{\Sigma_A}^2}{2\left({\norm{\Sigma_A}^2} +  \norm{A}^2\right)}\epsilon; \quad -\frac{1}{L} \le - \frac{1}{4\left({\norm{\Sigma_A}^2} +  \norm{A}^2\right) + \epsilon}.
\end{align*}
With $\alpha \le \frac{1}{4\left({\norm{\Sigma_A}^2} +  \norm{A}^2\right)}$, the linear rate of \oneexptd  is thus
\[
q= 1- \mu \alpha - \mu^2  \alpha\left(\frac{1}{4\left({\norm{\Sigma_A}^2} +  \norm{A}^2\right) + \epsilon} -\alpha \right)  + \frac{\norm{\Sigma_A}^2}{2\left({\norm{\Sigma_A}^2} +  \norm{A}^2\right)}\alpha\epsilon.   
\]
\end{proof}
Comparing Theorem \ref{thm:rates_all} and Theorem \ref{thm:rates_1_over_t}, we can see that in Theorem \ref{thm:rates_1_over_t}, the step-size is much smaller, because in practice $L_{\max}$ can be very large. In that case, we are guaranteed to converge to the optimal solution, but with a slower rate. By using a much larger step-size in Theorem \ref{thm:rates_all}, we are only guaranteed to converge to a neighborhood of the optimal solution, however, with a much faster, linear rate.     


Theorem \ref{thm:rates_all}.\ref{thm_item:imGTD} shows that the step-size of Impression GTD for the fastest convergence depends on three factors: the variance in the transition, the $\ell_2$ norm of $A$ and the batch size. In particular, the higher is the variance of the transition or the bigger is $\norm{A}$, the smaller the step-size we need to use for Impression GTD. A bigger batch size enables a larger step-size and faster convergence. The side effect of a larger step-size, though, is that the bias term in the convergence rate increases, which means the final convergence point may be located in a larger neighborhood of the optimal solution.    

\begin{table}
\centering
% \begin{tabular}{l|l| l| l|l| l|l} 
%  \hline
%    & Batch size & $L_1-  4\norm{A}^2$ & $L_2$ & $\lambda$ & Linear rate if& Bias\\
%  \hline
%  Im.GTD &$m^2$ & ${4\norm{\Sigma_A}^2}/{m} $ & ${\norm{\Sigma_A}^2}/{m}$ & ${2\norm{\Sigma_A}^4}/{m^2}$ & $m \ge \frac{\norm{\Sigma_A}^2}{\norm{A}\sqrt{2\mu}}$ & ${\alpha^2\sigma^2_v}/{m}$ \\ 
%  Ex.GTD & $t^2/4$ & ${8\norm{\Sigma_A}^2}/{t} $ & ${2\norm{\Sigma_A}^2}/{t}$ & ${8\norm{\Sigma_A}^4}/{t^2}$ & ${2\norm{\Sigma_A}^2}/{t}\le \epsilon$ & ${4\alpha^2\sigma^2_v}/{t^2}$\\
%  A$^\tr$TD & $m_1=t, m_2=1$ & ${4\norm{\Sigma_A}^2}/{t}$ & $\norm{\Sigma_A}^2$ & ${2\norm{\Sigma_A}^4}/{t}$ & ${\norm{\Sigma_A}^2}/{t}\le \epsilon$ & ${\alpha^2\sigma^2_v}/{t}$ \\
%  R1-GTD& $m_1=1, m_2=t$ & $4{\norm{\Sigma_A}^2} $ & ${\norm{\Sigma_A}^2}/{t}$ & ${2\norm{\Sigma_A}^4}/{t}$& ${\norm{\Sigma_A}^2}/{t}\le \epsilon$& ${\alpha^2\sigma^2_v}/{t}$\\ 
%  \hline
% \end{tabular}
%delete the linear rate if column
\begin{tabular}{l|l| l| l| l|l} 
 \hline
   & Batch size & $L_1-  4\norm{A}^2$ & $L_2$ & $\lambda$ & Bias\\
 \hline
 Im.GTD &$m^2$ & ${4\norm{\Sigma_A}^2}/{m} $ & ${\norm{\Sigma_A}^2}/{m}$ & ${2\norm{\Sigma_A}^4}/{m^2}$ & ${\alpha^2\sigma^2_v}/{m}$ \\ 
 Expected GTD & $t^2/4$ & ${8\norm{\Sigma_A}^2}/{t} $ & ${2\norm{\Sigma_A}^2}/{t}$ & ${8\norm{\Sigma_A}^4}/{t^2}$ & ${4\alpha^2\sigma^2_v}/{t^2}$\\
 A$^\tr$TD & $m_1=t, m_2=1$ & ${4\norm{\Sigma_A}^2}/{t}$ & $\norm{\Sigma_A}^2$ & ${2\norm{\Sigma_A}^4}/{t}$  & ${\alpha^2\sigma^2_v}/{t}$ \\
 R1-GTD& $m_1=1, m_2=t$ & $4{\norm{\Sigma_A}^2} $ & ${\norm{\Sigma_A}^2}/{t}$ & ${2\norm{\Sigma_A}^4}/{t}$& ${\alpha^2\sigma^2_v}/{t}$\\ 
 \hline
\end{tabular}
\caption{GTD Algorithm factors. For Impression GTD, we consider $m_1=m_2=m$. The first column is the effective batch size. }
\label{table:alg_L_mu_lambda}
\end{table}

The constants are summarized in Table \ref{table:alg_L_mu_lambda} for comparison.
Let's take a look at the smoothness constants of A$^\tr$TD and \oneexptd. 
For A$^\tr$TD, 
\[
\quad L_1 = 4\left(\frac{\norm{\Sigma_A}^2}{t} +  \norm{A}^2\right), \quad 
L_2 = {\norm{\Sigma_A}^2}, \quad \lambda = \frac{2\norm{\Sigma_A}^4}{t},
\]
For \oneexptd,
\[
L_1= 4\left({\norm{\Sigma_A}^2} +  \norm{A}^2\right), \quad L_2 =\frac{\norm{\Sigma_A}^2}{t}; \quad \lambda =  \frac{2\norm{\Sigma_A}^4}{t}.
\]
Clearly, the two algorithms have the same $\lambda$. In one extreme (A$^\tr$TD), $L_1$ is small but $L_2$ is large. In the other extreme (\oneexptd), $L_1$ is big but $L_2$ is small and in fact diminishing. Thus Impression GTD can be viewed as a balance between the two algorithms in $L_1$ and $L_2$. Its complexity is much lighter than the two algorithms, but it is still linear in the number of features. Though still higher than GTD, the order is the same, both in $O(d)$. The storage of Impression GTD is much higher due to the buffers. However, memory is not usually not a concern in modern computers, with a wide application in deep learning and deep reinforcement learning. 
After a sufficiently large number of learning steps, Impression GTD converges slower than A$^\tr$TD and \oneexptd, but the rate is still a linear rate, which is much faster than GTD, GTD2 and TDC. Our result also shows that A$^\tr$TD is slower than R1-GTD and with a larger bias term.     

Comparing Expected GTD, A$^\tr$TD and \oneexptd, we can see that there is wait time for the algorithms to be converge linearly. In particular, the wait time is proportional to $\lambda/L$, i.e., the ratio of the perturbation to the expected smoothness. For Expected GTD, this perturbation in the rate $q$ is $\frac{\norm{\Sigma_A}^4}{4\norm{A}^2t^2}$.\footnote{
The constant of the perturbation is also interesting. In particular, this ratio shows that the condition number of NEU and the variances in the feature transitions all contribute to the perturbation.} For the latter two algorithms, the perturbations are  
\[
\mbox{A$^\tr$TD: }
\frac{2\norm{\Sigma_A}^4}{\max\{ 4\norm{A}^2, \norm{\Sigma_A}^2\}t}; \quad \mbox{\oneexptd: } \frac{2\norm{\Sigma_A}^4}{4\left({\norm{\Sigma_A}^2} +  \norm{A}^2\right)t}. 
\]
We can see that the perturbation is diminishing in time. For the case of Expected GTD, the diminishing rate is very fast, which is $O(1/t^2)$. 
Thus the wait time for Expected GTD to converge linearly is much shorter than A$^\tr$TD and \oneexptd.\footnote{One can show that the wait time of Expected GTD is $1/\sqrt{\epsilon}$ for achieving a bias proportional to $\epsilon$. In the theorem, we let the algorithm wait the same amount of time as A$^\tr$TD and \oneexptd, for which case, the bias of Expected GTD is $O(\epsilon^2)$. The two presentation forms are equivalent.} A$^\tr$TD and \oneexptd have similar perturbation, both in the order of $O(1/t)$. At a constant scale, the perturbation in R1-GTD is smaller, and thus it waits shorter than A$^\tr$TD for the linear rate to arrive. 
The results also show that there is no guarantee that the three GTD algorithms (and Impression GTD as well) would converge fast before a sufficiently large number of samples in the buffers. This can be understood as that we need a sufficient amount of statistics built in our buffers and it takes time to grow it.  

Note that if we use mini-batch versions for GTD, GTD2 and TDC, their convergence rate may be expected to converge faster as well. Algorithm 1 by \citet{xu2021sample_twotimescale} shows how such update can be done for TDC. They showed that this mini-batch TDC also converges at a linear rate. The rate was established by requiring that the two step-sizes are smaller than some upper bound number. The number for $\alpha$ is fairly complex, containing quite a few terms from the minimum eigenvalues of $A^\tr C^{-1}A$ and $C$, the maximum importance sampling ratio, the ergodicity factor of the underlying Markov chain and $\beta$, the other step-size as well. On one hand, their result and ours show that mini-batch training is indeed a very useful tool for accelerating stochastic approximation, and effective for both single-time scale and two-time scale algorithms. However, on the other hand, regardless of that the mini-batch TDC algorithm also has two step-sizes, which is hard to use in practice just like TDC, the rate they proved is a fairly slow one even though it is linear. To be concrete, in their Theorem 1, let $\lambda_1 = \lambda_{\min} (A^\tr C^{-1}A)$ and $\lambda_2 = \lambda_{\min}(C)$. The $\lambda_1$ and $\lambda_2$ factors correspond to $\mu$, the strong convexity factor for solving the underlying O.D.Es of the main iterator and helper iterator, respectively. Also let $\rho_{\max} = \max_{s, a} \frac{\pi(a|s)}{\pi_b(a|s)}$, the maximum importance sampling ratio across all state-action pairs. The condition for the theorem requires that $\alpha$ should be at least as small as the minimum of $\frac{\lambda_1\lambda_2}{12}$ and $\frac{\lambda_1\lambda_2^2}{256 \rho_{\max}^2} $. Both numbers are extremely small because in practice the minimum eigenvalues are usually small. The factor $\rho_{\max}$ is very large in off-policy learning, and scales like $1000$  or even much higher aren't uncommon. %Though earlier experiments on off-policy learning uses problems where the importance sampling ratios are not crazily big to demonstrate the efficacy of off-policy learning \citep{precup2001off}, in practice, the ratios can be very large. For example, see \citep{chen2022sufficiency} in the context of PPO \citep{ppo}, and \citep{munos2016safe,xie2019towards_IS} for methods to tame the ratios in the context of off-policy learning.   


In contrast, our result does not depend on $\rho_{\max}$ (at least not explicitly, it may still play a role in the conditioning number of $A$). In addition to the condition number of $A$, the (single) step-size in our result depends on the ratio between the variances in the feature transitions and the batch size(s), which means we can increase the step-size for larger batch sizes and also for problems in which feature transition variances are small.         

\footnote{This paragraph is due to a discussion with Csaba Szepesvari.} In literature, there is a result of $O(1/t)$ rate established for linear stochastic approximation algorithms, which also holds for a variant of GTD \citep*{csaba_lin_stochastic18}. The technique they used is iterate averaging \citep{polyak1992acceleration}, which iteratively averages the weight vector over all historical steps. Later, by adapting the step-size or using constant step-sizes that require prior knowledge of certain problem-dependent data structures, the same rate is also established for TDC with iterative averaging (over both the main and the helper iterators) \citep*{csaba_gtd_22}. This $O(1/t)$ is known to be information-theoretically near-optimal, e.g., see \citep{1_over_t_information_optiaml}. Thus it appears that our linear rate is contradictory to this well-known result. The catch is that our result has a bias because of the use of constant step-sizes. Although the results by \citeauthor{csaba_gtd_22} also contain the case of a special constant step-size, the averaging on the top of iterations provides a similar effect to the diminishing step-size, which enables their solution to converge to the true solution without a bias. To have a closer look of why our result is not contradictory, take the main result of \citeauthor{1_over_t_information_optiaml} (their Theorem 1) for example. The result states that, for any algorithm that comes up with a solution $\hat{x}$, there exists a data distribution (underlying the expectation operator in $f$) such that 
\begin{equation}\label{eq:worst_rate_1_over_t}
f(\hat{x}) - f(x^*) \ge c\, \min \left\{Y^2, \frac{B^2+dY^2}{t}, \frac{BY}{\sqrt{t}}
\right\}
\end{equation}
holds, 
where $B$ and $Y$ are some constants, $B\ge 2Y$, and $c$ is some positive constant. Now if $t$ is sufficiently large, the $O(1/t)$ term is the minimum of the three. Thus the result quantifies {\em the worst convergence rate to the optimal solution}. Precisely, the distance from any algorithmic solution to the optimal solution (in terms of the loss) cannot be anywhere closer than $O(1/t)$ for certain data distributions. For our result in Theorem \ref{thm:sgd_rate}, when $t$ is sufficiently large, for a constant step-size $\alpha\le \frac{1}{L}$, the first term becomes negligible, and we are left with the bias term, which is usually bigger than zero. Thus our theorem states that SGD converges to a {\em neighborhood} of the optimal solution $x^*$ {\em linearly fast}, but caution that it does not necessarily converge to $x^*$ linearly fast. The $O(1/t)$ rate to $x^*$ still applies to SGD and Impression GTD with diminishing step-size or iterate averaging. Note that the $O(1/t)$ information-theoretically near-optimal rate is the worst case, and it is realized on certain data distributions. For example, the proof of Theorem 1 in \citep{1_over_t_information_optiaml} is constructed by using an example in which the data distributions depend on the sample size. In practice, we are usually not that unlucky that our data distributions are screwed like such, and we may often get a faster rate than $O(1/t)$ when using SGD with mini-batch update. 

There is a special case that SGD will converge to the optimal solution $x^*$ linearly fast, no longer to just a neighbourhood of $x^*$. This corresponds to $B=Y=0$ in equation \ref{eq:worst_rate_1_over_t}. In this case, this bound is only an obvious fact instead of a rate. %In fact, even though the bound above is still correct for $B=Y=0$, it wasn't proved correctly. 
%This is because the proof of Theorem 1 in \citep{1_over_t_information_optiaml} fails to discuss this case. For example, the proof of Theorem 1 compares the sample size with $B^2/Y^2$, and it depends on Theorem 3, which uses $Y/B$ in the two designed distributions. 
Our bound such as Theorem \ref{thm:sgd_rate} correctly covers a subclass of this case, with $L$-$\lambda$ smoothness for the loss and the sampling, the convergence of SGD is linear, with a zero bias. 
Interestingly, in Baird counterexample (\ref{exp:baird}), we actually see this linear rate to the optimal solution in experiments, because the bias term there is zero due to that $B=Y=0$. 
%Because the four GTD algorithms we analyzed are SGD algorithms, the above arguments also hold for them.  


\section{Experiments}
% \haizhou{Follow the same way of introduction as we did in Section2.}
% \noindent In this section, we will introduce datasets and experimental setups that we used. Then we evaluate our method, other self-supervised methods, and supervised methods under different distribution shifts (\ie, concept shifts and covariate shifts) under common settings (\ie, transductive, inductive settings). It has to note that we focus on node-level tasks (\eg, node classification) in this work. As for graph-level tasks, we leave it as our future work and some simple experiments can be found in Appendix~\ref{app:graph_classification}. 
In this section, we first introduce the experimental setup including datasets, training, and evaluation protocol in Section~\ref{sec:dataset}~and~\ref{sec:unsupervised}. 
% Next, we present our experimental setup and conduct extensive experiments to evaluate our method in Section~\ref{sec:unsupervised}. 
We then perform an ablation study to demonstrate the effectiveness of each proposed component in Section~\ref{sec:ablation}. 
Additionally, we analyze the impact of important hyper-parameters in Section~\ref{sec:sensitivity}. 
Subsequently, we integrate our method with various encoding models, showcasing the model-agnostic nature of our recipe in Section~\ref{sec:other_models}. 
Finally, we provide some qualitative results such as feature visualization in Section~\ref{sec:vis}.
It is important to note that we focus on node-level tasks (\eg, node classification) in this work. As for graph-level tasks, we leave it as our future work, while some simple experiments are also provided in Appendix~\ref{app:graph_classification}.

\subsection{Datasets}\label{sec:dataset}
There exist some benchmarks for evaluating graph out-of-distribution generalization~\cite{good,ji2022drugood,gds}. 
Among them, GOOD~\cite{good} is the most representative and comprehensive benchmark that curates more diverse graph datasets with diverse tasks, including single/multi-task graph classification, graph regression, and node classification involving more distribution shifts (\ie, concept shifts and covariate shifts). Hence in this work, we follow the evaluation protocol proposed in \cite{good}. Furthermore, we validate the effectiveness of our method in the datasets (\ie, Amazon-Photo, Elliptic) that are used in EERM~\cite{eerm}. The statistics and detailed introduction to these datasets can be found in Table~\ref{tab:dataset} and Appendix~\ref{app:datasets}.

\begin{table*}[htp]
\caption{The descriptions of datasets. ``Domain-Level'' means splitting by graphs, ``Time-Aware'' denotes splitting according to chronological order.``Word'' and ``Degree'' represent splitting according to word diversity and node degree respectively. ``Language'' means splitting by user language, suggesting the prediction should not be impacted by the language the user use. ``University'' denotes splitting according to the domain university, implying that the prediction of webpages should be based on word contents and link connections rather than university features. ``Color'' means that nodes are split according to node differences in covariate shift and color-label correlations in concept shift.}
\label{tab:dataset}
\centering
\begin{tabular}{cccccccc}
\toprule
Datasets     & Network Type        & \#Nodes & \#Edges & \#Attributes &\#Classes& Train/Val/Test Split     & Metric   \\
% Cora         & Artificial Transformation & 2,703   &         &              &         &                      & Accuracy \\
Amazon-Photo\footnotemark
             & Co-purchasing network      & 7,650   & 119,081   & 755          & 10      & Domain-Level         & Accuracy \\
Elliptic\footnotemark  
             & Bitcoin transactions       & 203,769 & 234,355   & 165          & 2       & Time-Aware           & F1-Score \\
GOOD-Cora    & Scientific publications    & 19,793  & 126,842   & 8,710         & 70      & Word/Degree          & Accuracy \\
% GOOD-Arxiv   & arXiv papers               & 169,343 & 2,315,598 & 128          & 40      & Time/Degree          & Accuracy \\
GOOD-Twitch  & Gamer network              & 34,120  & 892,346   & 128          & 2       & Language             & ROC-AUC  \\
GOOD-CBAS    & A BA-house graph           & 700     & 3,962     & 4             & 4       & Color                & Accuracy \\
GOOD-WebKB   & Webpage network            & 617     & 1,138     & 1,703         & 5       & University           & Accuracy \\
\bottomrule
\end{tabular}
\end{table*}
\footnotetext[5]{This dataset is adopted from~\cite{yang2016revisiting}. \cite{eerm} constructs ten graphs with different environment id’s for each graph.} 
\footnotetext[6]{The original is available on \hyperlink{https://www.kaggle.com/ellipticco/elliptic-data-set}{https://www.kaggle.com/ellipticco/elliptic-data-set}}

\subsection{Unsupervised Representation Learning}\label{sec:unsupervised}
\subsubsection{Transductive Setting}~\label{sec:trans}
% \noindent\textbf{Baselines.}\quad We conduct experiments with 12 baselines which consist of three categories: supervised methods and self-supervised generative methods, self-supervised contrastive methods. Specifically, we compare with three supervised baselines: empirical risk minimization~(ERM)~\cite{erm}, invariant risk minimization (IRM)~\cite{irm}, and a recent proposed graph OOD method dubbed EERM~\cite{eerm}. We also compare various unsupervised node-level representation learning methods: three self-supervised generative methods including GAE~\cite{gae}, VGAE~\cite{gae}, GraphMAE~\cite{gmae} and seven self-supervised contrastive methods: DGI~\cite{dgi}, MVGRL~\cite{mvgrl}, GRACE~\cite{grace}, RoSA~\cite{rosa}, BGRL~\cite{bgrl}, COSTA~\cite{costa}, SwAV~\cite{swav}. The descriptions of these methods can be found in Appendix~\ref{app:baselines}.
In this subsection, we focus on validating our proposed algorithm under the transductive setting, where the test nodes will participate in message passing~\cite{gilmer2017neural} during training following~\cite{good}. 

\noindent\textbf{Baselines.} We conduct experiments with 12 baselines from three categories: (i)~supervised methods, including empirical risk minimization~(\textbf{ERM})~\cite{erm}, invariant risk minimization (\textbf{IRM})~\cite{irm}, and a recent proposed graph OOD method \textbf{EERM}~\cite{eerm}; (ii)~self-supervised generative methods including Graph Autoencoder (\textbf{GAE})~\cite{gae}, Variational Graph Autoencoder (\textbf{VGAE})~\cite{gae}, Self-Supervised Masked Graph Autoencoders (\textbf{GraphMAE})~\cite{gmae}; (iii)~self-supervised contrastive methods including Deep Graph Infomax (\textbf{DGI})~\cite{dgi}, Contrastive Multi-View Representation Learning on Graphs (\textbf{MVGRL})~\cite{mvgrl}, Deep Graph Contrastive Representation Learning (\textbf{GRACE})~\cite{grace}, A Robust Self-Aligned Framework for Node-Node Graph Contrastive Learning (\textbf{RoSA})~\cite{rosa}, Bootstrapped Representation Learning on Graphs (\textbf{BGRL})~\cite{bgrl}, Covariance-Preserving Feature Augmentation for Graph Contrastive Learning (\textbf{COSTA})~\cite{costa}, Unsupervised Learning of Visual Features by Contrasting Cluster Assignments (\textbf{SwAV})~\cite{swav}. The detailed descriptions of these baselines can be found in Appendix~\ref{app:baselines}.

\noindent\textbf{Experimental setup.} We use the same graph encoder across different datasets for a fair comparison following~\cite{good}. We use grid search to find other hyper-parameters (\eg, learning rate, epochs) for different methods. For all experiments, we select the best checkpoints for ID and OOD tests according to results on ID and OOD validation sets following~\cite{good}, respectively. Experimental details and hyper-parameter selections are provided in Appendix~\ref{app:hyper}. For evaluating unsupervised methods, a linear classifier will be built on the frozen trained encoder after finishing pre-training. The reported results are the mean performance with standard deviation after 10 runs following~\cite{good}.

\noindent\textbf{Analysis.}\quad Based on the experimental results listed in Table~\ref{tab:trans_concept} and \ref{tab:trans_covariate}, we can draw the following conclusions: firstly, we find strong self-supervised methods (\eg, GRACE, BGRL, COSTA) are more robust to distribution shifts (concept shift in Table~\ref{tab:trans_concept} and covariate shift in Table~\ref{tab:trans_covariate}) compared to supervised methods. For instance, on GOOD-CBAS and GOOD-WebKB datasets, GRACE surpasses the best supervised method by large margins (over 6\% absolute improvement). Interestingly, we find the methods designed for OOD generalization (\ie, IRM) and graph OOD generalization (\ie, EERM) do not attain superior performance than the standard ERM on most of the datasets. For example, EERM shows superior OOD performance compared to ERM in only one experiment, and IRM outperforms ERM in four out of ten experiments across the conducted evaluations. This phenomenon is also observed in \cite{good,ahuja2020empirical,rosenfeld2021risks}, showcasing the challenge of achieving invariant prediction in non-Euclidean graph settings. 

Furthermore, our method surpasses other SOTA self-supervised methods on the OOD test set of all datasets by a considerable margin while achieving comparable performance in the in-distribution test set. For instance, on small datasets such as GOOD-CBAS and GOOD-WebKB, our method outperforms GRACE\footnote{MARIO is built up on GRACE according to our recipe. So, we make a comparison with GRACE here.} by over 2\% absolute accuracy on the OOD test set. On larger datasets such as GOOD-Cora and GOOD-Twitch, our method still outperforms other methods which shows its superiority. For instance, under covariate shift, MARIO surpasses other methods by over 7\% absolute accuracy on the GOOD-Twitch OOD test set. These statistics prove the effectiveness of our design.


\begin{table*}[htp]
\caption{Experimental results of all methods under concept shift. The bold font means the top-1 performance and the underline represents the second performance across the unsupervised methods. 'ID' represents in-distribution test performance and 'OOD' means out-of-distribution test performance. (OOM: out-of-memory on a GPU with 24GB memory)}
\label{tab:trans_concept}
\centering
\scalebox{0.95}{
\begin{tabular}{l|cc|cc|cc|cc|cc}
\toprule
\toprule
\multirow{3}{*}{concept shift} & \multicolumn{4}{c|}{GOOD-Cora}                   & \multicolumn{2}{c|}{GOOD-CBAS} & \multicolumn{2}{c|}{GOOD-Twitch} & \multicolumn{2}{c}{GOOD-WebKB} \\
                           & \multicolumn{2}{c}{word} & \multicolumn{2}{c|}{degree}& \multicolumn{2}{c|}{color}    & \multicolumn{2}{c|}{language}   & \multicolumn{2}{c}{university} \\
                           & ID         & OOD         & ID          & OOD          & ID            & OOD           & ID             & OOD            & ID            & OOD            \\
\midrule
ERM                        & 66.38±0.45 & 64.44±0.18  & 68.60±0.40  & 60.76±0.34   & 89.79±1.39    & 83.43±1.19    & 80.80±1.00     & 56.92±0.92     & 62.67±1.53    & 26.33±1.09     \\
IRM                        & 66.42±0.41 & 64.29±0.31  & 68.57±0.35  & 61.45±0.24   & 89.64±1.21    & 82.29±1.14    & 78.87±1.04     & 59.30±1.79     & 62.67±1.10    & 26.88±1.42     \\
EERM                       & 65.10±0.44 & 62.45±0.19  & 66.95±0.44  & 56.58±0.25   & 79.07±2.12    & 64.50±1.01    & OOM            & OOM            & 62.50±2.01    & 28.07±3.23      \\
\midrule
% Random-Init                & 37.53±1.74 & 32.12±1.24  & 37.82±1.71  & 27.74±1.14   &               &               &                &                & 60.33±2.21    & 27.07±1.70     \\
GAE                        & 60.65±0.89 & 58.00±0.55  & 62.59±1.11  & 53.44±0.80   & 75.28±1.36    & 68.07±2.05    & 81.25±0.81     & 51.51±1.05     & 62.17±3.34    & 25.78±1.85     \\
VGAE                       & 63.19±0.53 & 60.35±0.47  & 61.65±0.66  & 54.28±0.28   & 76.50±0.50    & 59.07±0.56    & 80.46±0.53     & 55.56±4.53     & 62.50±2.38    & 24.40±2.57     \\
GraphMAE                   & \underline{66.44±0.46} & \underline{64.87±0.30}  & 67.95±0.46  & 59.41±0.39   & 89.14±0.89    & 82.93±0.93    & 80.05±0.64     & 59.38±1.49     & 61.83±3.37    & 29.27±2.15     \\
DGI                        & 63.33±0.56 & 60.71±0.49  & 65.93±1.02  & 55.83±0.53   & 91.22±1.47    & 85.00±1.66    & 80.05±0.87     & 59.16±1.88     & 61.83±2.83    & 28.63±1.92      \\
MVGRL                      & OOM        & OOM         & OOM         & OOM          & 88.57±1.15    & 76.50±1.17    & OOM            & OOM            & 62.00±3.79    & 28.26±4.20     \\
GRACE                      & 65.61±0.61 & 63.92±0.44  & \textbf{68.59±0.35}  & 60.15±0.45   & 92.00±1.39    & 88.64±0.67    & \textbf{83.43±0.63}     & \underline{60.45±1.46}     & 64.00±3.43    & \underline{34.86±3.43}  \\
RoSA                       & 64.06±0.67 & 62.44±0.39  & 67.07±0.65  & 57.68±0.44   & 90.78±2.27    & 85.93±2.14    & 82.39±0.42     & 57.45±2.16     & 64.17±4.10    & 32.20±2.15     \\
BGRL                       & 65.18±0.43 & 63.43±0.45  & 66.83±0.80  & 59.63±0.38   & 92.36±1.16    & 87.14±1.60    & 82.52±0.60     & 55.48±1.48     & 63.67±2.33    & 31.47±3.43     \\
COSTA                      & 65.05±0.80 & 62.37±0.45  & 66.76±0.87  & 55.73±0.36   & \underline{93.50±2.62}    & \underline{89.29±3.11}    & 83.15±0.30 & 55.03±3.22     & 61.66±2.58    & 32.39±2.13 \\
% ArCL                       &            &             & 67.64±0.57  & 59.71±0.44   &               &               &                &                & 65.00±3.94    & 35.41±1.97 \\      
SwAV                       & 62.22±0.53 & 59.79±0.53  & 64.65±0.94  & 55.06±0.39   & 89.00±0.79    & 81.72±0.66    & \underline{83.32±0.15}     & 59.69±1.97     & \underline{65.17±3.76}    & 29.36±2.01    \\
\midrule
MARIO                       & \textbf{67.11±0.46} & \textbf{65.28±0.34}  & \underline{68.46±0.40}  & \textbf{61.30±0.28}   & \textbf{94.36±1.21}    & \textbf{91.28±1.10}    & 82.31±0.54     & \textbf{63.33±1.72}     & \textbf{65.67±2.81}    & \textbf{37.15±2.37}     \\
\bottomrule
\end{tabular}}
\end{table*}

\begin{table*}[htp]
\caption{Experimental results of all methods under covariate shift. The bold font means the top-1 performance and the underline represents the second performance across the unsupervised methods. 'ID' represents in-distribution test performance and 'OOD' means out-of-distribution test performance. (OOM: out-of-memory on a GPU with 24GB memory)}
\label{tab:trans_covariate}
\centering
\scalebox{0.95}{
\begin{tabular}{l|cc|cc|cc|cc|cc}
\toprule
\toprule
\multirow{3}{*}{covariate shift} & \multicolumn{4}{c|}{GOOD-Cora}                                   & \multicolumn{2}{c|}{GOOD-CBAS} & \multicolumn{2}{c|}{GOOD-Twitch} & \multicolumn{2}{c}{GOOD-WebKB} \\
                           & \multicolumn{2}{c}{word} & \multicolumn{2}{c|}{degree}& \multicolumn{2}{c|}{color}    & \multicolumn{2}{c|}{language}   & \multicolumn{2}{c}{university} \\
                           & ID         & OOD         & ID          & OOD          & ID            & OOD           & ID             & OOD            & ID            & OOD            \\
\midrule
ERM                        & 70.50±0.41 & 64.69±0.33  & 72.46±0.49  & 55.53±0.50   & 92.00±3.08    & 77.57±1.29    & 70.98±0.41     & 49.35±5.09     & 39.34±1.79    & 14.52±3.14   \\
IRM                        & 70.48±0.26 & 64.53±0.57  & 71.98±0.34  & 53.72±0.46   & 90.86±2.41    & 78.86±1.67    & 69.81±0.95     & 49.11±2.82     & 38.52±3.30    & 13.97±2.80     \\
EERM                       & OOM        & OOM         & OOM         & OOM          & 65.00±2.57    & 57.43±3.60    & OOM            & OOM            & 46.07±4.55    & 27.40±7.65     \\
\midrule
GAE                        & 56.63±0.79 & 48.93±0.93  & 66.30±0.88  & 34.01±0.87   & 73.00±2.16    & 60.86±3.01    & 67.24±1.23     & 47.65±2.49     & 45.08±6.32    & 28.02±6.29    \\
VGAE                       & 62.02±0.66 & 54.12±0.86  & 69.41±0.57  & 44.20±1.29   & 62.29±2.04    & 63.29±1.11    & 66.99±1.43     & \underline{50.48±4.58}     & 48.85±4.68    & 20.87±6.69     \\
GraphMAE                   & 68.14±0.43 & 64.00±0.33  & \textbf{73.36±0.56}  & 53.75±0.55   & 67.28±3.03    & 67.28±1.49    & 68.84±1.20     & 48.02±2.79     & 48.03±4.34    & 30.00±8.09     \\
DGI                        & 60.85±0.75 & 57.03±0.67  & 68.97±0.41  & 41.75±0.88   & 69.57±4.09    & 59.71±3.43    & 68.43±1.05     & 44.83±1.61     & 48.52±5.04    & 21.11±7.50     \\
MVGRL                      & OOM        & OOM         & OOM         & OOM          & 65.00±1.94    & 64.15±0.77    & OOM            & OOM           & \textbf{54.10±5.39}    & 16.59±6.51     \\
GRACE                      & \underline{68.77±0.33} & \underline{64.21±0.41}  & 72.69±0.34  & \underline{56.10±0.63}   & \underline{93.57±1.83}    & \underline{89.29±3.40}    & \underline{71.12±0.87} & 46.21±1.54 & 49.67±5.82    & 28.10±4.68    \\
RoSA                       & 68.19±0.56 & 62.48±0.61  & 71.04±0.62  & 52.72±0.79   & 84.71±4.14    &79.14±3.51     & 70.58±0.36     & 45.83±1.72     & 52.30±4.24    & \underline{34.24±7.92}     \\
BGRL                       & 67.23±0.43 & 61.33±0.36  & 72.11±0.39  & 49.15±0.73   & 89.00±2.56    & 79.86±3.29    & \textbf{71.43±0.53}     & 43.86±0.94     & 51.80±5.55    & 30.32±7.61    \\
COSTA                      & 65.28±0.60 & 60.33±0.53  & 70.65±0.62  & 54.03±0.28   & 92.29±1.59    & 82.71±2.74    & 69.29±1.37     & 49.07±2.13     & 50.49±3.01    & 29.84±4.75   \\
SwAV                       & 63.29±1.01 & 56.98±0.94  & 70.27±0.73  & 43.00±0.52   & 89.57±1.12    & 81.43±1.69    & 69.19±0.93     & 49.37±2.96     & 49.84±4.82    & 30.55±6.72   \\
\midrule
MARIO                       & \textbf{69.99±0.54} & \textbf{65.06±0.34}  & \underline{72.73±0.43}  & \textbf{57.73±0.45}  & \textbf{94.57±2.46}    & \textbf{91.00±2.48}     & 68.31±0.78 & \textbf{57.37±1.37}     & \underline{53.94±3.23}    & \textbf{35.24±4.98}   \\
\bottomrule
\end{tabular}}

\end{table*}

\subsubsection{Inductive Setting}
In this subsection, we conduct experiments under the inductive settings, where the test nodes are kept unseen during training. This setting is more suitable for domain generalization.
% But we think it is more convincing that conduct experiments under inductive settings which means test nodes are unseen during training. This setting is more appropriate for domain generalization.

\noindent\textbf{Baselines:} For GOOD-WebKB and GOOD-CBAS datasets, we adopt ERM, IRM, GraphMAE, and GRACE as our baselines. And for Amazon-Photo and Elliptic datasets, we select ERM, EERM, and GRACE as our baselines.

\noindent\textbf{Experimental setup:} For GOOD-WebKB and GOOD-CBAS datasets, we use the same model configuration in Section~\ref{sec:trans}.
% Besides, we add experiments on Amazon-Photo dataset~\cite{yang2016revisiting} and Elliptic~\cite{elliptic} dataset in this subsection. 
For Amazon-Photo dataset~\cite{yang2016revisiting} and Elliptic~\cite{elliptic} dataset, they consist of many snapshots (training data and testing data use different snapshots) which are naturally inductive. For Amazon-Photo dataset, we use 2-layer GCN~\cite{gcn} as the encoder and for elliptic dataset, we use 5-layer GraphSAGE~\cite{sage} as encoder following~\cite{eerm}.

% Figure environment removed

\noindent\textbf{Analysis:}
According to Figure~\ref{fig:amazon},\ref{fig:elliptic},\ref{fig:ind_con},\ref{fig:ind_cov}, we can draw following conclusions:
firstly, based on Figure~\ref{fig:amazon}, it is evident that our method outperforms other representative supervised and self-supervised methods on all test graphs (T1$\sim$T8). This superiority is reflected in the larger median value of our method compared to others. For instance, MARIO achieves over a 3\% absolute improvement compared to ERM in terms of the mean value of eight median values. Additionally, our method demonstrates higher stability across different random initializations, as indicated by the closer proximity of the first and third quartile values to the median value~(\eg, the difference of first and third quartile values of ERM, EERM, GRACE and MARIO are 4.2, 3.3, 6.7 and 1.0 on T8 respectively which indicates MARIO is much more stable than other methods). Furthermore, our method exhibits consistent performance across different graphs (\eg, The standard deviation of median values on T1$\sim$T8 for ERM, EERM, GRACE, and MARIO are 0.4, 1.1, 1.2, and 0.3, respectively.), indicating its robustness to environmental variations and its ability to extract invariant features: $g(G^e) \approx g(G^{e'})$ for all $e, e' \in \mathcal{E}^\text{train}$. In summary, our method showcases enhanced OOD generalization capabilities.
% $g(G^e)g(G^e^\prime)$ where $any e, e^\prime in \mathcal{E}^{train}$

Secondly, from the results presented in Figure~\ref{fig:elliptic}, we can observe that our method averagely harvests 10.9\% absolute improvement over GRACE and 12.5\% absolute improvement over EERM in terms of F1 scores on Elliptic dataset. This demonstrates the effectiveness of our method in handling distribution shifts and improving performance compared to existing approaches. It is worth noting that GRACE's performance worsens over time, indicating its inability to handle distribution shifts effectively. In contrast, our method consistently achieves better F1 scores, except for T9, which is caused by the dark market shutdown occurred after T7~\cite{elliptic}. The emergence of such an event introduces significant variations in data distributions, which subsequently results in performance degradation for all methods. Indeed, this event serves as an unpredictable external factor that introduces significant challenges for models trained on limited training data. The results indicate that the performance heavily depends on available training data. Nonetheless, our approach outperforms other methods even in such an extreme case. This highlights the effectiveness of our method in addressing distribution shifts and improving generalization performance.

Finally, based on the observations from Figure~\ref{fig:ind_con} and Figure~\ref{fig:ind_cov} MARIO demonstrates the best performances on both ID and OOD test sets for GOOD-WebKB and GOOD-CBAS datasets, under both concept shift and covariate shift. Notably, MARIO outperforms other methods by more than 3\% and 10\% absolute improvement on GOOD-WebKB and GOOD-CBAS, respectively, under covariate shift. We can draw similar conclusions as discussed in Section~\ref{sec:trans}. Even under the inductive setting, our method continues to demonstrate excellent OOD generalization capabilities and achieves comparable or even improved in-distribution test performance. These statistical results further validate the effectiveness of our method in handling distribution shifts and enhancing generalization performance.

Overall, the observations we have made provide strong evidence of the great capacity of our method for handling distribution shifts, validating its effectiveness and potential for real-world applications.



% Figure environment removed

% Figure environment removed


% Figure environment removed


\subsection{Ablation Studies}\label{sec:ablation}
\noindent Table~\ref{tab:aba} provides a detailed analysis of the effect of each component according to our proposed recipe for improving OOD generalization in graph contrastive learning. Let's examine the different variants of our method and their impact on performance.
Specifically, MARIO~(w/o ad) represents MARIO without  adversarial augmentation. MARIO~(w/o cmi) denotes we only maximize the mutual information between positive pairs without considering conditional mutual information. MARIO~(w/o cmi, ad) means a vanilla graph contrastive method that is similar to GRACE. 

From Table~\ref{tab:aba}, we can find MARIO~(w/o cmi) lags far behind MARIO on OOD test set which demonstrates appropriately minimizing the redundant information (\ie, conditional mutual information) is essential to improve OOD generalization of GCL methods. And adversarial augmentation can also boost OOD generalization because it can approximately serve as a supermum operator to learn more invariant features  discussed in Section~\ref{sec:aug}. Based on the analysis of these variants, it is evident that the proposed improvements on data augmentation and contrastive loss in the recipe are both effective in enhancing graph OOD generalization. Each component contributes to the overall performance improvement, and their combination leads to a stronger self-supervised graph learner in terms of graph OOD generalization. 

In short, the findings from Table~\ref{tab:aba} support the rationale behind your proposed recipe and provide empirical evidence of the effectiveness of each proposed component. By incorporating these enhancements, our method achieves superior performance in handling distribution shifts and improving graph OOD generalization in graph contrastive learning.
\begin{table*}[htp]
\caption{Ablation studies for MARIO by masking each component.}
\label{tab:aba}
\centering
\scalebox{0.9}{
\begin{tabular}{l|cc|cc|cc|cc|cc}
\toprule
\toprule
\multirow{3}{*}{concept shift} & \multicolumn{4}{c|}{GOOD-Cora}                       & \multicolumn{2}{c|}{GOOD-CBAS} & \multicolumn{2}{c|}{GOOD-Twitch} & \multicolumn{2}{c}{GOOD-WebKB} \\
                           & \multicolumn{2}{c}{word} & \multicolumn{2}{c|}{degree}& \multicolumn{2}{c|}{color}    & \multicolumn{2}{c|}{language}   & \multicolumn{2}{c}{university} \\
                           & ID         & OOD         & ID          & OOD          & ID            & OOD           & ID             & OOD            & ID            & OOD            \\
\midrule
MARIO                      & \textbf{67.11±0.46} & \textbf{65.28±0.34}  & \textbf{68.46±0.40}  & \textbf{61.30±0.28}      & \textbf{94.36±1.21}  & \textbf{91.28±1.10}    & 82.31±0.54     & \textbf{63.33±1.72}     & \textbf{65.67±2.81}    & \textbf{37.15±2.37}     \\
MARIO(w/o ad)              & 66.23±0.53 & 64.02±0.18  & 67.88±0.38  & 60.46±0.29   & 93.21±1.25    & 90.29±0.91    & 82.42±0.73     & 60.50±1.02     & 64.83±2.83    & 36.51±3.25    \\
MARIO(w/o cmi)             & 65.32±0.60 & 63.51±0.32  & 68.14±0.32  & 61.19±0.34   & 94.15±1.23    & 90.57±1.96    & \textbf{82.51±0.56}     & 61.41±2.63     & 64.50±4.35    & 35.78±2.53     \\
MARIO(w/o cmi, ad)         & 64.67±0.55 & 63.11±0.32  & 67.95±0.65  & 60.01±0.57   & 93.36±1.66    & 89.64±1.73    & 81.90±0.75     & 60.12±1.60     & 64.17±3.67    & 34.13±2.38     \\
\bottomrule
\end{tabular}}
\end{table*}
% & 65.32±0.60 & 63.51±0.32 exchange 64.67±0.55 & 63.11±0.32
% 68.14±0.32       id ood test: 60.95±0.43       ood ood test: 61.19±0.34


\subsection{Sensitivity Analysis}\label{sec:sensitivity}
\noindent In this subsection, we will analyze some important hyper-parameters of our method. We conduct sensitivity analysis on GOOD-WebKB dataset with concept shift, we chose two sensitive hyper-parameters (\ie, the coefficient $\gamma$ of condition mutual information in Equation~\ref{equ:cmi} and the number of prototypes $|C|$ in Equation~\ref{equ:pq}). The coefficient of CMI range in $[0.001, 0.01, 0.1, 0.5, 1]$ and the number of prototypes $|C|$ ranges in $[10, 50, 100, 200, 300]$. From Figure~\ref{fig:sensitivity}, we can observe that $\gamma$ reaches 0.1 and $|C|$ reaches 100 or 200 can achieve the best OOD test accuracy. Both higher and lower values of $\gamma$ result in suboptimal performance. This finding aligns with previous research such as DIB~\cite{dib}, indicating that an appropriate compression level is crucial for achieving optimal performance. Extremely high or low compression values are not ideal. 

Regarding the number of prototypes $|C|$, based on the results shown in Figure~\ref{fig:sensitivity}, it is found that setting $|C|=100$ leads to the best performance in terms of OOD test accuracy. This choice provides a moderate number of pseudo labels, which is beneficial for the learning process. 

Based on the sensitivity analysis, we determined that setting $\gamma=0.1$ and $|C|=100$ on most datasets. These hyperparameter values strike a balance between compression level and the number of prototypes, resulting in improved graph OOD generalization.
% Figure environment removed


\subsection{Integrated with Other Models}\label{sec:other_models}
% Figure environment removed

\begin{table}[htp]
\caption{Results of different learning approaches with different encoding models (\ie, GCN, GraphSAGE, GAT).}
\label{tab:others}
\centering
\scalebox{0.9}{
\begin{tabular}{cc|cc|cc}
\toprule
\toprule
\multirow{3}{*}{Model}& \multirow{3}{*}{Method} & \multicolumn{2}{c|}{GOOD-CBAS} & \multicolumn{2}{c}{GOOD-WebKB} \\
                & & \multicolumn{2}{c|}{color}    & \multicolumn{2}{c}{university} \\
                &   & ID          & OOD         & ID          & OOD            \\
\midrule
\multirow{3}{*}{GCN} 
&ERM               & 89.79±1.39 & 83.43±1.19  &  62.67±1.53 & 26.33±1.09         \\
&GRACE             & 92.00±1.39 & 88.64±0.67  &  64.00±3.43 & 34.86±3.43        \\
&MARIO             & 94.36±1.21 & 91.28±1.10  &  65.67±2.81 & 37.15±2.37        \\ \bottomrule
\multirow{3}{*}{SAGE} 
&ERM               & 95.07±1.51 & 75.14±1.19  & 73.67±2.08  & 46.33±3.42       \\
&GRACE             & 95.29±1.11 & 74.43±2.36  & 70.50±5.06  & 49.54±3.83        \\
&MARIO             & 96.00±1.07 & 76.29±3.01  & 71.00±3.82  & 51.74±4.63        \\ \bottomrule
\multirow{3}{*}{GAT} 
&ERM               & 78.64±3.63 & 72.93±2.64  & 61.33±3.71  & 28.99±2.63        \\
&GRACE             & 84.57±1.79 & 78.36±1.60  & 59.50±2.36  & 35.78±3.26        \\
&MARIO             & 84.93±1.95 & 80.43±1.89  & 62.17±4.78  & 38.17±3.10        \\
\bottomrule
\end{tabular}}
\end{table}



\noindent In the subsection, we demonstrate the model-agnostic nature of the recipe by integrating it with various graph neural network (GNN) models, including GCN, GraphSAGE, and GAT.

From Table~\ref{tab:others}, it can be observed that regardless of the specific GNN model used as the encoder, our method consistently achieves the best performance on the OOD test set. This indicates the effectiveness and robustness of our method across different GNN models.
By achieving superior performance across different GNN models, MARIO demonstrates its versatility and ability to improve the OOD generalization of various graph neural models. This highlights the broad applicability and effectiveness of our recipe in enhancing the performance of different GNN encoders.

Furthermore, we integrate our recipe with other GCL methods in Appendix~\ref{app:other_methods}. The results demonstrate our recipe can boost the OOD generalization ability of various GCL methods which means our recipe can serve as a plug-in for many current classical GCL methods.

% Figure environment removed

\subsection{Visualization}\label{sec:vis}
\subsubsection{Metric Score Curves}
We present metric score curves for ERM and MARIO, including training, ID validation, ID testing, OOD validation, and OOD testing accuracy, in Figure~\ref{fig:curve2}. Notably, MARIO demonstrates superior convergence with approximately 10\% absolute improvement on the OOD test set compared to ERM. Furthermore, MARIO effectively narrows the performance gap between in-distribution and out-of-distribution performance, showcasing its efficacy in enhancing OOD generalization for graph data. More metric score curves can be found in Appendix~\ref{app:curves}.


\subsubsection{Feature Visualization}
In order to assess the quality of learned embeddings, we adopt t-SNE~\cite{tsne} to visualize the node embedding on GOOD-Cora dataset (concept shift in word domain) using random-init of GCN, EERM, GRACE, and MARIO, where different classes have different colors in Figure~\ref{fig:vis}. For clarity, we select eight classes with the largest number of nodes to enhance the informativeness and interpretability of the visualization. We can observe that the 2D projection of node embeddings learned by MARIO has a better separation of clusters, which indicates the model can help learn representative features for downstream tasks. It has to note that we depict both ID nodes and OOD nodes in the same figure. 

Besides, we also separately visualize ID nodes and OOD nodes in the different figures in the Appendix~\ref{app:feature}. And we can find MARIO performs a clearer separation of clusters whether on ID nodes or OOD nodes compared to other methods.




\section{Conclusion and Future Work}
In this work, I design corruption-robust algorithms for the Lipschitz contextual search problem. I present the \emph{agnostic checking} technique and demonstrate its effectiveness in designing corruption-robust algorithms. There are several open problems for future research. First, in the algorithm I propose for pricing loss, the schedule for agnostic checks is fixed upfront. Can the learner design an adaptive checking schedule for the pricing loss? Second, this work assumes the learner has knowledge of the Lipschitz constant $L$. Can the learner design efficient no-regret algorithms without knowledge of $L$? 

% Acknowledgements should go at the end, before appendices and references

\acks{
We appreciate Thomas Walsh, James MacGlashan, Peter Stone, Varun Kompella, Dustin Morrill and Ishan Durugkar for insightful discussions on the topics of off-policy learning and deep reinforcement learning, who also helped improve the paper in many ways. Tom spotted a problem in an early draft of Theorem 2. Tom and Peter also gave lots of advice that greatly helped improve the presentation of the paper.   
We would like to thank Declan Oller for the pointer to the paper by \citet*{lihong_kernel_sim}, which helped improve our understanding of Impression GTD. %At the time, an internal document at Sony AI of this paper was shared and reviewed, with writing,  experiments, and exactly the same algorithm layout presented herein (with the $\bold{sim}$ notation and independence sampling). 
We appreciate Sina Ghiassian, Andrew Patterson, Shivam Garg, Dhawal Gupta, Adam White and Martha White for making their TDRC code available, and Shangtong Zhang for the Baird counterexample, both of which greatly facilitate the experiment studies in this paper. We appreciate Shangtong Zhang also for helpful discussions on importance sampling and off-policy learning.  
}

% Manual newpage inserted to improve layout of sample file - not
% needed in general before appendices/bibliography.

%\newpage

% \appendix
% \section*{Appendix A.}
% \label{app:theorem}

% Note: in this sample, the section number is hard-coded in. Following
% proper LaTeX conventions, it should properly be coded as a reference:

%In this appendix we prove the following theorem from
%Section~\ref{sec:textree-generalization}:

% In this appendix we prove the following theorem from
% Section~6.2:

% \noindent
% {\bf Theorem} {\it Let $u,v,w$ be discrete variables such that $v, w$ do
% not co-occur with $u$ (i.e., $u\neq0\;\Rightarrow \;v=w=0$ in a given
% dataset $\dataset$). Let $N_{v0},N_{w0}$ be the number of data points for
% which $v=0, w=0$ respectively, and let $I_{uv},I_{uw}$ be the
% respective empirical mutual information values based on the sample
% $\dataset$. Then
% \[
% 	N_{v0} \;>\; N_{w0}\;\;\Rightarrow\;\;I_{uv} \;\leq\;I_{uw}
% \]
% with equality only if $u$ is identically 0.} \hfill\BlackBox

% \noindent
% {\bf Proof}. We use the notation:
% \[
% P_v(i) \;=\;\frac{N_v^i}{N},\;\;\;i \neq 0;\;\;\;
% P_{v0}\;\equiv\;P_v(0)\; = \;1 - \sum_{i\neq 0}P_v(i).
% \]
% These values represent the (empirical) probabilities of $v$
% taking value $i\neq 0$ and 0 respectively.  Entropies will be denoted
% by $H$. We aim to show that $\fracpartial{I_{uv}}{P_{v0}} < 0$....\\

% {\noindent \em Remainder omitted in this sample. See http://www.jmlr.org/papers/ for full paper.}


\vskip 0.2in
\bibliography{ref}

\appendix
\begin{comment}
\section{System Architecture}
\label{appendix:architecture}
\system has a novel modularized system architecture with three key components: 
\emph{StreamManager}, 
\emph{TxnManager} and \emph{TxnScheduler}. 
These components are instantiated in each thread locally.
The execution outline of \system is presented in Algorithm~\ref{alg:algo}.
Transactional stream processing is continuous and potentially never ends (Line 1$\sim$8).
The dependency resolution and execution of state transactions are separated into two non-overlapping phases by punctuations~\cite{Tucker:2003:EPS:776752.776780} (Line 2 and 5), which guarantees that no subsequent input event will have a smaller timestamp. 
Effectively, a batch of state transactions is collected during the first phase, and processed during the second phase.

In the first phase (i.e., stream processing phase), 
the \emph{StreamManager} conducts preprocessing for every input event ($e$). Similar to some prior works~\cite{tstream}, state transactions may be issued but not immediately processed during preprocessing (Line 3).
The \emph{pre\_processing} and \emph{post\_processing} functions are exposed as APIs to users.
The \emph{TxnManager} handles dependency resolution (Line 4) among state transactions and insert decomposed operations to construct a \tpg. We discuss the detailed two-phase \tpg construction process in Section~\ref{subsec:construction}.

In the second phase  (i.e., transaction processing phase), 
the \emph{TxnManager} is first involved again to refine (Line 6) the constructed \tpg with further dependency resolution.
The \emph{TxnScheduler} 
schedules operations for concurrent execution based on the constructed \tpg according to the three dimensions of scheduling decisions (Line 7). 
In particular, a scheduling decision model $M$ is instantiated based on the constructed \tpg (Line 14).
\textbf{\circled{1}} Guided by $M$, execution threads adopt an exploration strategy (Section~\ref{subsec:explore}) to explore the constructed \tpg for operations available to be scheduled constrained by dependencies. 
\textbf{\circled{2}} 
During exploration, one or multiple operations may be treated as the 
% basic 
unit of scheduling (Section~\ref{subsec:granularity}). 
Subsequently, \textbf{\circled{3}} every thread executes operation(s) in the unit of scheduling with various abort handling mechanisms (Section~\ref{subsec:abort_handling}).
Only when state transactions are processed (i.e., committed or aborted) can the associated input events be postprocessed (Line 8) by the \emph{StreamManager} based on transaction processing results.
\end{comment}

\begin{comment}
\begin{algorithm}
\footnotesize
    \KwData{$e$ \tcp{Input event}}
    \KwData{$txn_{ts}$ \tcp{State transaction}}
    \KwData{$G$ \tcp{The currently constructed TPG}}
    \While{!finish processing of input streams}{
        \eIf(\tcp*[h]{Phase 1}){\text{$e$ is not a $punctuation$}}{
                $txn_{ts}$ $\gets$ PRE\_Processing($e$)\;
                \textbf{TPG\_Construction}($G$, $txn_{ts}$)\; 
          }(\tcp*[h]{Phase 2}){
                \textbf{TPG\_Refinement}($G$)\; 
                \textbf{TXN\_Scheduling}($G$)\; 
                POST\_Processing()\;
          }
    }
    
    \SetKwFunction{FMain}{TPG\_Construction}
    \SetKwProg{Fn}{Function}{:}{}
    \Fn{\FMain{$G$, $txn_{ts}$}}{
        $O_{1..k}$ $\gets$ \textbf{Partition} $txn_{ts}$\;
        \ForEach{\text{operation $O_{i}$ $\in$ $O_{1..k}$}}{
            \textbf{Identify} its \ld\;
            $G$ $\gets$ $G$ + $O_{i}$ \;
        }
    }
    \SetKwFunction{FMain}{TPG\_Refinement}
    \SetKwProg{Fn}{Function}{:}{}
    \Fn{\FMain{$G$}}{
        \ForEach{\text{vertex $e_{i}$ $\in$ $G$}}{
            \textbf{Identify} its \td, \pd\;
        }
    }
    
    \SetKwFunction{FMain}{TXN\_Scheduling}
    \SetKwProg{Fn}{Function}{:}{}
    \Fn{\FMain{$G$}}{
        $M$ $\gets$ Instantiated with $G$;\tcp{A decision model}
        \While{!finish scheduling of $G$
        }{
          \textbf{\circled{2}} $Scheduling Unit$ $\gets$ \textbf{\circled{1}} \emph{Explore}($G$, $M$)\; 
            \textbf{\circled{3}} \emph{Execute with Abort Handling} ($Scheduling Unit$)\; 
        }
    }
  \caption{Execution Outline of \system}
  \label{alg:algo}
\end{algorithm}
\end{comment}

\end{document}