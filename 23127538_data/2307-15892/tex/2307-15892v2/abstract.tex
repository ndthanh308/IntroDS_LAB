\begin{abstract}
Gradient Temporal Difference (GTD) algorithms \citep{gtd,tdc} are the first $O(d)$ ($d$ is the number of features) algorithms that have convergence guarantees for off-policy learning with linear function approximation. 
\citet{bo_gtd_finite} and \citet{dalal2018finite_twotimescale} proved the convergence rates of GTD, GTD2 and TDC are $O(t^{-\alpha/2})$ for some $\alpha \in (0,1)$. This bound is tight \citep{dalal2020tale_twotimescale}, and slower than $O(1/\sqrt{t})$. 
GTD algorithms also have two step-size parameters, which are difficult to tune. In literature, there is a ``single-time-scale'' formulation of GTD. However, this formulation still has two step-size parameters.

This paper presents a truly single-time-scale GTD algorithm for minimizing the Norm of Expected (TD) Update (NEU) objective, and it has only one step-size parameter.  
We prove that the new algorithm, called Impression GTD, converges to the optimal solution in $O(1/t)$ rate with a constant step-size. Furthermore, based on a generalization of the expected smoothness \citep{gower2019sgd_general}, called $L$-$\lambda$ smoothness, we are able to prove that the new GTD with large constant step-sizes converges, in fact, with a {\em linear rate}, to a biased solution. 
Our rate actually also improves \citeauthor{gower2019sgd_general}'s result with a tighter bound under a weaker assumption.  
Besides Impression GTD, we also prove the rates of three other GTD algorithms, one by \citet{ptd_yao},  another called A$^\tr$TD \citep{gtd}, and a counterpart of A$^\tr$TD. 
It appears that these four algorithms are only different in how they build certain data structures from the buffers. 
The convergence rates of all the four GTD algorithms are proved in a {\em single} generic GTD framework to which $L$-$\lambda$ smoothness applies. Empirical results on Random walks, Boyan chain, and Baird counterexample show that Impression GTD converges much faster than existing GTD algorithms for both on-policy and off-policy learning problems, with well-performing step-sizes in a big range. 
\end{abstract}

