
\documentclass[aip,reprint,floatfix,nofootinbib,amsmath,amssymb,amsfont]{revtex4-1}

\usepackage{graphicx}\graphicspath{{figures}}
\usepackage{mathtools}
\usepackage{bm}
\usepackage{dcolumn}
\usepackage{esint}
\usepackage{subcaption}
\usepackage{braket}
\usepackage{lettrine}
\usepackage[utf8]{inputenc}
\usepackage[T1]{fontenc}
\usepackage{mathptmx}
\usepackage{appendix}
\usepackage{etoolbox}

\begin{document}

\title{Theory of electron and ion holes as vortices in the phase-space of collision-less plasmas}
	\author{Allen Lobo}
	\email{allen.e.lobo@outlook.com}
	\affiliation{Sikkim Manipal Institute of Technology, Sikkim Manipal University, Sikkim - 737136, India.}
	\author{Vinod Kumar Sayal}
 \email{vksayal@hotmail.com}
	\affiliation{Sikkim Manipal Institute of Technology, Sikkim Manipal University, Sikkim - 737136, India.}
	\date{\today}
 
	\begin{abstract}
    This article studies the vortical nature and structure of phase-space holes -- nonlinear B.G.K. trapping modes found in the phase-space collision-free plasmas. A fluid-like outlook of the particles' phase-space is introduced, which makes it convenient to analytically identify electron and ion holes as vortices -- similar to that of ordinary two-dimensional fluids. A fluid velocity and vorticity field is defined for the phase-space of the electrons and ions, Euler equations describing the flow of the phase-space fluid representing the particle system are then developed. Using these equations, electron holes and ion holes are analytically identified as vortices in the phase-space of the plasma. A relation between Schamel's trapping parameter ($\beta$), hole speed ($M$), hole phase-space depth ($-\Gamma$) and hole potential amplitude ($\chi_0$) is derived. The approach introduces a new technique to study the phase-space holes of collision-less plasmas, allowing fluid-vortex-like treatment to these kinetic structures. Phase-space distribution functions for electron hole regions can then be analytically derived from this model, reproducing the schamel-df equations and thus acting as a precursor to the pseudo-potential approach, avoiding the need to assume a solution to the phase-space density.
	\end{abstract}
	\maketitle
    \section{Introduction}
	\lettrine{T}{he} problem of electron and ion holes of collision-less plasmas is, perhaps, one of the most complex topics that the theory of collision-free plasma has to deal with, comprising many stochastic elements. It refers to phenomena for which there is no linear approach. Electron holes (and their positive counter-parts) are well-identifiable species in their respective phase-space with the core particle density much lower than the boundary, resembling a Gaussian-like potential in their respective configuration space, as seen in laboratory and simulation experiments \cite{Morse1969, turikov1978computer, Saeki1979, Lynov1979}. They are the result of nonlinear trapping mechanisms and are identified as B.G.K. modes \cite{Bernstein1957} with velocities ($M$) comparable to thermal speeds \cite{Iizuka1987, hutchinson2016plasma, Saeki1979, Lynov1979}. Such trapping may be the result of one of the numerous mechanisms \cite{Schamel2023PatternEquilibria}.\\
    \indent The study of electron (and ion) holes is conducted using differential or integral approaches, the former employing an assumed distribution function \cite{Schamel1979} for the trapped and free particles in the hole region and the latter employing an assumed potential function resembling a Gaussian or an inverted bell-shape \cite{Turikov1984, Chen2005, Aravindakshan2021StructuralPlasma}. Schamel's pseudopotential approach utilizes the schamel-df equations \cite{Schamel1979, schamel1986electron} (assumed electron phase-space density functions for free and trapped particles) and produces a nonlinear dispersion relation expressing hole amplitude $\chi_0$ as a function of a particle-trapping parameter $\beta$, wave number $k$ and hole speed $M$. However, the value of the $\beta$ parameter can not be exactly defined -- except that it must be a negative value -- allowing arbitrariness in the dispersion relation. \\
    \indent Interestingly, the electron and ion holes have been ubiquitously referred to as phase-space vortices \cite{Morse1969,Saeki1979, Schamel1979,schamel1982kinetic,Turikov1984, Schamel1998,Guio2003,Trivedi2017,hutch2017}. However, no analytical proof of their vortical nature has been presented till now. On the other hand, vortices and vortical behaviour of ordinary two-dimensional fluids present themselves with an abundant history of mathematical analyses.  The correspondence between an ordinary two-dimensional fluid and a one-dimensional (1D-1V) phase-space is well-known, with the Vlasov equation in the collision-less plasma case being identical to the Liouville equation for ideal in-compressible two-dimensional fluids \cite{schamel1986electron,schamel2012cnoidal}. It should be noted that for such an analysis, the phase-space of such systems should be studied in analogy to ordinary fluid surfaces -- electron and ion holes are vortices in the phase-space of plasma particles, and not in the configuration space (where they present themselves as nonlinear B.G.K. mono-polar potential waves). Hence, a fluid-like modelling of the particle phase-space must be introduced to analytically identify and treat electron holes and ion holes as phase-space vortices.\\
    \indent In this article, the electron phase-space of collision-less plasmas is portrayed in a fluid-like outlook. Analogous to conventional two-dimensional fluid dynamics, the 1D-1V phase-space is transformed into a two-dimensional plane, wherein a fluid-like dynamical model of the particle phase-space is described. Euler equations and diffusion equations are framed to demonstrate the fluid-like behaviour of the particle phase-space using the velocity and vorticity fields of the phase-space fluid. On the bases of these equations and pre-established techniques of vortex identification in fluids \cite{Tian2018, Lugt1979, Helmholtz1867}, electron holes (and by analogy, ion holes) are analytically identified as vortices in the phase-space of the system. The same equations are then used to derive a steady-state structure of the phase-space density in the hole region, reproducing the schamel-df equations \cite{Schamel1979}. The fluid description of the phase-space of the plasma particles (and not the plasma itself) provides a new outlook to understand electron and ion holes observed in the phase-space of collision-less plasma. The equations developed using this approach analytically identifies such holes as vortices in phase-space and provides techniques analogous to the study of a fluid vortex to explore their structure and behaviour. Using this approach, the phase-space density function of both free and trapped particles can be derived, without the need of an initial assumption. 
    
    \section{Theory of phase-space vortices}
    \indent The kinetic study of many-particle systems thoroughly describes the (stochastic) behaviour of the system -- which comprises particles with a range of energy, under influence of some interacting potential. This is represented by the Hamiltonian of the system itself --
  \begin{equation}\label{timeindependentH}
      T(v_x) = H(f_{x,v_x}) - V(x).
  \end{equation}
		Here $T(v_x)$, $H$, $f(x,v_x)$ and $V(x)$ represent the Kinetic Energy function, Hamiltonian, phase-space probability density and Potential Energy function. This analysis, however, when conducted for a dynamically evolving system, describes a flow-like evolution of the particle phase-space --
  \begin{equation}\label{timedependentH}
      p_x\dot{v}_x = \int\frac{df}{dt}dE - \dot{x}\frac{\partial V(x)}{\partial x}.
  \end{equation}
  Here, $p_x=\frac{\partial T}{\partial v_x}$ is the momentum of a particle, $E$ is its energy and $\dot q$ represents an ordinary time derivative of $q$. The terms $\dot v_x$ and $\dot x$ can be observed as the flow-rate of a particle in velocity and position axes as it slides along a phase-space trajectory, in which case the above equation (\ref{timedependentH}) describes flow-like evolution of a system in its phase-space, providing the dynamics of the many-particle phase-space with a fluid-like outlook.
  The position-velocity phase-space of a system can be described as a two-dimensional surface by multiplying a system-specific time-constant to the velocity axis. Such a time-factor can be a characteristic response time of the particles, characteristic to the system itself, and we represent it as $\tau$ for our analysis. The phase-space then describes a co-ordinate set $x-\tau v_x$ to locate an infinitesimal volume $\tau dx dv_x$. The particle density of this phase-space is the number of particles present in this volume -- 
  \begin{equation}\label{psdensity}
     \frac{\partial^2 N}{\tau\partial x \partial v_x}=\frac{1}{\tau}f(x,v_x) =\eta.
  \end{equation}
  Here, $N$ is the total number of particles in the (closed) system. The phase-space gradient operator can similarly be defined as the derivative along each axis --
  \begin{equation}\label{nablameaning}
      \nabla = \frac{\partial}{\partial x}\hat x + \frac{\partial}{\tau\partial v_x}\hat v_x.
  \end{equation}
  One can now describe the associated position $\bm{r}$ and velocity field $\mathcal{V}$ of the phase-space fluid\footnote{The stream function for describing the fluid-like self-consistency of the phase-space-fluid model can be described using a modified particle Hamiltonian ($H$). Please refer to Appendix for the mathematics.} --
  \begin{subequations}\label{vnjfields}
      \begin{align}
      \bm{r} = x\hat x + \tau v_x \hat v_x,\\
          \bm{\mathcal{V}} = \frac{d\bm{r}}{dt} = \dot x \hat x + \tau \dot v_x \hat v_x.\\
          \intertext{The flux density of the flow in the phase-space $\bm{J}$ will then be the multiple of the particle density and velocity field of the phase-space --}
          \bm{J} = \eta \bm{\mathcal{V}} = \frac{1}{\tau}f\dot x \hat x + f\dot v_x \hat v_x.
      \end{align}
  \end{subequations}
  
  \subsection{The kinetic Euler equations}
  The velocity and flux density fields of the particle flow in phase-space, described in equations (\ref{vnjfields}a-c) can now be used to understand the nature of the flow of this phase-space fluid. We begin by studying the time-rate of change of the particle density in the phase-space --
  \begin{subequations}\label{continuityeqn}\allowdisplaybreaks
      \begin{align}
          &\frac{d\eta}{dt} = \frac{1}{\tau}\left(\frac{\partial f}{\partial t}+\frac{\partial f}{\partial x}\frac{dx}{dt}+\frac{\partial f}{\partial v_x}\frac{dv_x}{dt}\right).\\
          \intertext{Also,}
          &\nabla\cdot\bm{\mathcal{V}} = \frac{\partial \dot x}{\partial x} + \frac{\partial \dot v_x}{\partial v_x} = \frac{\partial}{\partial x}\frac{\partial H}{\partial v_x}-\frac{\partial}{\partial v_x}\frac{\partial H}{\partial x} = 0.\\
          \begin{split}
               \nabla\cdot\bm{J} = \eta \nabla.\bm{\mathcal{V}} + \bm{\mathcal{V}}.\nabla \eta = \dot x \frac{\partial \eta}{\partial x} + \tau \dot v_x \frac{\partial \eta}{\tau\partial v_x}\\
               =\frac{1}{\tau}\dot x\frac{\partial f}{\partial x} + \frac{1}{\tau}\dot v_x\frac{\partial f}{\partial v_x}.
          \end{split}\\
          &\Rightarrow \frac{d\eta}{dt} = \frac{\partial \eta}{\partial t} + \nabla\cdot\bm{J} =\begin{cases}
              0&\text{collision-less system}\\
              \left(\frac{\delta f}{\delta t}\right)_{\text{coll.}}&\text{with collisions}
          \end{cases} 
      \end{align}
  \end{subequations}
  % Figure environment removed
 \\
 Equation (\ref{continuityeqn}d) describes the continuity equation of the phase-space flow in the cases of collision-less and collisional considerations of the system \cite{Gibbs1902, Vlasov1938}. The flow of the phase-space fluid representation of the one-dimensional system is in-compressible in a collision-free condition.\\
 \indent The dynamics of the phase-space fluid analogue of the system can be further explored by framing Euler phase-space momentum equations governing its flow. We do this by determining the time-derivatives of the fluid's velocity field. For the same, we introduce a general internal force field $\bm{F}(x)$ and kinetic pressure\footnote{The pressure term in the above equation is the kinetic equivalent of fluid pressure present in the phase-space of the system and arises (in fluid-analogue) due to particle density gradient in the phase-space of the particle, along the position axis. This pressure term must not be confused with the plasma fluid pressure which does not play a role in the above equations.} $\bm{P}$ into the system.
 \begin{subequations}\allowdisplaybreaks\label{Eulereqns}
     \begin{align}
         &\frac{d\bm{\mathcal{V}}}{dt}=\Ddot{x}\hat x + \tau \Ddot v_x \hat v_x.\quad \text{Introducing advection terms,}\\
		&\ddot x(x,t) = \frac{\partial \dot x}{\partial t} + v_x\frac{\partial \dot x}{\partial x}, \quad \ddot v_x(x,t) = \frac{\partial \dot v_x}{\partial t} + v_x\frac{\partial \dot v_x}{\partial x}\\
         \intertext{For a system experiencing an internal force field $\bm{F}(x)$ and pressure $\bm{P}$,}
        &\Ddot{x} = \frac{\bm{F}(x)}{m}-\frac{\nabla_x \bm{P}}{m\eta},\qquad \Ddot{v}_x = \frac{\dot{\bm{F}}(x)}{m}.\\
        \intertext{Splitting into components along each axis in phase-space, we get --}
       &\frac{\partial \dot x}{\partial t} + v_x\frac{\partial \dot x}{\partial x} = \frac{\bm{F}(x)}{m} -\frac{1}{m\eta}\frac{\partial \bm{P}}{\partial x} \\
       &\frac{\partial \dot v_x}{\partial t} +  v_x\frac{\partial \dot v_x}{\partial x}=\frac{\dot{\bm{F}}(x)}{m}. 
     \end{align}
 \end{subequations}
 The above equations (\ref{Eulereqns}d, e) describe the momentum equations for the flow of the phase-space fluid model of the system. 
 % Figure environment removed
 For different particles, the equations can be generalised as --
 \begin{equation}\label{kineticeuler1}
     \begin{split}
         m_i\eta_i\left(\frac{\partial \dot x}{\partial t} + v_x\frac{\partial \dot x}{\partial x}\right) = \eta_i\bm{F_i}(x) -\nabla_x \bm{P}, \\
        m_i\eta_i\left(\frac{\partial \dot v_x}{\partial t} + v_x\frac{\partial \dot v_x}{\partial x}\right)=\eta_i\dot{\bm{F_i}}(x).
     \end{split}
 \end{equation}
 One can introduce more terms on the right side of the equations (\ref{kineticeuler1}) for other interactions between particles in phase-space. Equations (\ref{continuityeqn}d) and (\ref{kineticeuler1}) together describe the kinetic Euler equations for the fluid model of the particle phase-space. These equations are in capacity to collectively describe the complete dynamics of the system in its phase-space, in the same nature as the classical Euler equations describe the behaviour of an ordinary fluid. For the collision-less condition, equation (\ref{continuityeqn}d) can be set to zero. 
 
 \subsection{Collision-less diffusion in phase-space}
  A form of particle diffusion in this phase-space fluid model can be observed as an outcome of the developed kinetic Euler equations. For this, we study the kinetic momentum equation along position-space, with a time-invariant, quasi-static flow of the fluid in phase-space, therefore equating $\frac{\partial \dot x}{\partial t}$ to 0. We assume a thermal equilibrium state, such that the pressure gradient $\nabla\bm{P}_i$ can be expressed in the isothermal, barotropic form $K_BT_i\nabla\eta_i$, $T_i$ being the temperature of the particle and $K_B$ the Boltzmann constant. Using $\dot x = v_x$, the particle velocity, the equation becomes --
 \begin{subequations}\label{diff1}\allowdisplaybreaks
 \begin{align}
     m_i\eta_i v_x\frac{\partial v_x}{\partial x}= \eta_i\bm{F_i}(x) -K_BT_i\nabla_x \eta_i\\
     \bm{J}_{x_i}=\eta_i v_x= \frac{\eta_i \bm{F}_i(x)}{m_i\frac{\partial v_x}{\partial x}} -\frac{K_B T_i}{m_i\frac{\partial v_x}{\partial x}}\nabla_x \eta_i.
 \end{align}
 \end{subequations}
 The absence of a force field $\bm{F}(x)$ in the system produces a non-diverging, non-rotating flow of the phase-space fluid, restricted along the position space, since the fluid velocity along $\hat v_x$ direction becomes zero (refer to FIG. \ref{Figure1}a). In such a case, $\frac{\partial v_x}{\partial x}$ becomes equal to zero. However, in regions with non-zero force field, the phase-space fluid velocity $\bm{\mathcal{V}}$ may retain a rotational velocity, thus exhibiting a solenoidal flow (refer to FIG. \ref{Figure1}b).
 \begin{equation}\label{modvorticity}
     \Bigl|\nabla \times \bm{\mathcal{V}}\Bigr| = \frac{\partial \tau \dot v_x}{\partial x} - \frac{1}{\tau}\frac{\partial \dot x}{\partial v_x}=\frac{\tau}{m_i}\frac{\partial \bm{F}(x)}{\partial x} - \frac{1}{\tau}\frac{\partial v_x}{\partial v_x}.
 \end{equation}
 Here, $\dot v_x$ has been substituted by $\bm{F}/m_i$. The presence of a force field also suggests that --
 \begin{equation}\label{dvdxsubstitution}\allowdisplaybreaks
     \begin{split}
         \nabla.\bm{\mathcal{V}} = \frac{\partial \dot x}{\partial x}+\frac{1}{m_i}\frac{\partial \bm{F}}{\tau\partial v_x}=0 \Rightarrow
        \frac{\partial v_x}{\partial x} = -\frac{1}{m_i}\frac{\partial \bm{F}(x)}{\partial x}\frac{\partial x}{\partial v_x}\\
         \Rightarrow \frac{\partial v_x}{\partial x} = \sqrt{\frac{1}{m_i}(-\nabla_x \bm{F})}.
     \end{split}
 \end{equation}
 Including equation (\ref{dvdxsubstitution}) in equation (\ref{diff1}b), we get --
 \begin{subequations}\label{diff2}\allowdisplaybreaks
     \begin{align}
     \begin{split}
        &\bm{J}_{x_i}=\eta_i v_x= \frac{\eta_i \bm{F}_i(x)}{\sqrt{-m_i\nabla_x \bm{F}}} -\frac{K_B T_i}{\sqrt{-m_i\nabla_x \bm{F}}}\nabla_x \eta_i,\\
         &\text{or,}\quad\bm{J}_x = \mu\eta_i\bm{F}(x) - \mathcal{D}\nabla_x \eta_i
         \end{split}\\
         &\text{where,} \quad \mu =\sqrt{\frac{-1}{m_i\nabla_x\bm{F}(x)}}, \quad \mathcal{D} = K_BT_i\sqrt{\frac{-1}{m_i\nabla_x\bm{F}(x)}}\\
         &\text{and}\qquad K_BT_i\mu = \mathcal{D}.
     \end{align}
 \end{subequations}
 In the above equations, $\mu$ represents particle mobility in phase-space along position ($x$) axis. $\mathcal{D}$ represents the fluid diffusion coefficient in phase-space, which is related to the mobility $\mu$ using Einstein's classical diffusion relation \cite{Einstein1905UberTeilchen, Sutherland1905LXXV.Albumin} shown in equation (\ref{diff2}c). Both particle mobility and diffusion coefficient are facilitated by the presence of a negative force-field gradient in these regions. Further understanding of this phenomena is therefore possible by introducing field-specific equations which describe this gradient. As an example, upon introducing an electrostatic interaction, $\bm{F} = q\bm{E} $ where $\bm{E}(x)$ is the electric field of this interaction, the field gradient will then become the spatial charge density, since, $\epsilon_0\nabla_x\bm{E} = \rho(x)$, suggesting that in an electrostatic domain, such a diffusion occurs in the presence of a charge accumulation in space.
 \subsection{Vorticity field in phase-space}
 \indent We now proceed to describe the vorticity field $\bm{\xi}(x,\tau v_x)$ associated with the phase-space fluid analogue of a system, which is simply the curl of the velocity field, as described in equation (\ref{modvorticity}).
 \begin{equation}\label{vorticity}
 \begin{split}
     \bm{\xi}=\nabla \times \bm{\mathcal{V}} = \left(\frac{\partial\tau \dot v_x}{\partial x} - \frac{1}{\tau}\frac{\partial \dot x}{\partial v_x}\right)\hat n\\
     =\left(\frac{\tau}{m_i}\frac{\partial \bm{F}(x)}{\partial x} - \frac{1}{\tau}\right)\hat n,\\
     \hat n = \hat x \times \hat v_x.
     \end{split}
 \end{equation}
 Here, $\hat n$ is a direction normal to the phase-space plane, and is a constant in a non-relativistic inertial frame\footnote{The relativistic transformations of position and velocity coordinates of the particle states and its affect on the preceding mathematics can be explored in future works on this field of study.}. It is observable that vorticity of the phase-space fluid, which varies only along the position-space, is influenced by the present force-field gradient, without which it remains a constant. It is hence in the nature of this phase-space fluid model of a system to exhibit some type of local deformation, sheering or rotational, in time. Presence of vorticities in either direction, regulated by the direction of the force-field gradient, can subsequently give rise to local turbulence which can then result in formation of various characteristic structures, such as vortices (refer to FIG. \ref{Figure1}c).\\
 \indent The vorticity field of the phase-space fluid model of the system can be further analysed using the vorticity transport equation, which can be simplified by introducing the barotropic pressure term and restricting within two-dimensions --
 \begin{equation}\label{vorticityeqn1}
    \frac{\partial \bm{\xi}(x,t)}{\partial t}+(\bm{\mathcal{V}}\cdot\nabla) \bm{\xi} = \mathcal{D}\nabla^2 \bm{\xi} +\nabla \times \left(\frac{\mathcal{F}}{m\eta}\right).
\end{equation}
Here, $\mathcal{F}$ represents the sum-total of all body forces acting on the system. Further simplification of this equation can be performed upon defining the vorticity term specific to a given form of interaction between particles in the system.\\
\indent The preceding mathematics in this section describes the phase-space of a many-particle system as a two-dimensional fluid. The Euler-like equations developed in this section describe flow-like behaviour of the particle phase-space and can now be used to analyse for vortical structures. We use this technique for collision-less plasmas and identify electron holes and ion holes as vortices in the phase-space of the plasma in the next section.

\section{Analytical identification of electron and ion holes as phase-space vortices}
 \indent Having described the phase-space as a two-dimensional dynamic fluid, we can now analytically identify electron and ion holes as vortices in the phase-space. For the same, we use some established techniques of vortex identification in two-dimensional fluids \cite{Helmholtz1867, Lugt1979, Tian2018}. We analyse the phase-space vorticity fields and observe particle trajectories in the phase-space of the plasma, and observe local concentrations of vorticities as well as locally rotating fluids.\\
 \indent Upon setting particle charge $q= \delta e$ with $\delta = \pm1$ for (singly ionised, positive or negative) ions or electrons, $\bm{F}=qE$  with $E(x,t)$ as electric field, $\dot x = v_x$ and $\dot v_x = \frac{\delta e}{m}E(x,t)$ in equation (\ref{vorticity}), the vorticity associated with a plasma particle (ion or electron) phase-space fluid flow becomes --
\begin{equation}\label{vorticityrho}\allowdisplaybreaks
	\begin{split}
	\bm{\xi}=&\Bigl(\tau\frac{\partial \dot v_x}{\partial x} - \frac{1}{\tau}\Bigr)\hat n = \Bigl(\frac{\delta e\tau}{m}\frac{\partial E(x,t)}{\partial x} - \frac{1}{\tau}\Bigr)\hat n\\
	=& \Bigl(\frac{\delta e\tau}{\epsilon_0m}\rho(x,t) - \frac{1}{\tau}\Bigr)\hat n = \omega_{p}\Bigl(\delta\bar{\rho}(x,t) -1\Bigr),\\
	& \text{where,}\quad  en_0\bar{\rho}(x,t) =\rho(x,t).
	\end{split}
\end{equation}
 Here, $\tau= \sqrt{\frac{\epsilon_0m}{n_0 q^2}}$ representing (inverse) plasma particle frequency $\omega_p$, which can be set to $1$ for even further simplification. It is a simple task to introduce multiple-ionised particles into these equations by replacing $\delta e$ with $\delta Z e$, $Z$ being the ionisation number. As shown in equation (\ref{vorticityrho}), vorticity associated with the plasma particle phase-space fluid flow is related to the local concentration of charge in a region of phase-space, defined by the spatial charge density $\rho(x)$. \\
 \indent The vorticity $\bm{\xi}$ associated with the phase-space, like any other ordinary fluid, describes the deformations in a region of the particle phase-space fluid. These deformations can be local sheering or rotational deformations, the latter of which describes a local spin of fluid elements about a common axis and is identified as a fluid vortex. These can also be defined as a local region of concentrated vorticity distinguished by closed or spiral streamlines \cite{Lugt1979}, or as described by \textcite{Tian2018}, using the vortex parameter $g_{z}$ (or $g_{z\theta\rvert_{max}}$ in an invariant form) which can be defined for our system as --
\begin{subequations}\label{tianvalues}\allowdisplaybreaks
 		\begin{align}
 			\begin{split}
 		&g_z=\frac{\tau\partial \dot v_x}{\partial x}\frac{\partial \dot x}{\tau\partial v_x} = \frac{\delta e}{m}\frac{\partial E(x)}{\partial x}\frac{\partial v_x}{\partial v_x}\\&=\frac{\delta e}{m \epsilon_0}\rho(x)= \delta \omega_{p}^2 \bar{\rho}(x)
 		\end{split}\\
 	&g_{z\theta}\lvert_{\max}=\alpha^2 - \beta^2\\
 	\begin{split}=\frac{1}{4}\Biggl[\left(\frac{\tau\partial \dot v_x}{\tau\partial v_x}-\frac{\partial \dot x}{\partial x}\right)^2 + \left(\frac{\tau\partial \dot v_x}{\partial x} + \frac{\partial \dot x}{\tau\partial v_x}\right)^2 \\- \left(\frac{\tau\partial \dot v_x}{\partial x}-\frac{\partial \dot x}{\tau\partial v_x}\right)^2 \Biggr]\end{split}\\
 	&=\frac{1}{4}\Biggl[ \Bigl(\frac{\partial \dot x}{\partial x}+\frac{\partial \dot v_x}{\partial v_x}\Bigr)^2 - 4\frac{\partial \dot x}{\partial x}\frac{\partial \dot v_x}{\partial v_x} + 4\frac{\partial \dot v_x}{\partial x}\frac{\partial \dot x}{\partial v_x} \Biggr] \\
 	\begin{split}=\Biggl(\frac{\partial \dot v_x}{\partial x}\frac{\partial \dot x}{\partial v_x}-\frac{\partial \dot x}{\partial x}\frac{\partial \dot v_x}{\partial v_x} \Biggr)=\\\Biggl(\frac{\delta e}{m}\frac{\partial E(x)}{\partial x}\frac{\partial v_x}{\partial v_x}-\frac{\partial v_x}{\partial x}\frac{\delta e}{m}\frac{\partial E(x)}{\partial v_x} \Biggr)\\
 		=\frac{\delta e}{m}\frac{\partial E(x)}{\partial x} = \frac{\delta e}{m\epsilon_0}\rho(x) = \delta \omega_{p}^2\bar \rho(x) = g_z.\end{split}\\
 	\intertext{From ref. \cite{Tian2018}, for a vortex region, }
 	 &g_{z\theta}\lvert_{\max}<0 \implies \delta \omega_{p}^2\bar \rho(x) <0 \implies \delta\bar \rho(x)<0.
 		\end{align}
 \end{subequations}
 % Figure environment removed
 \\\indent The results described in equation (\ref{tianvalues}a,e,f) state the relation between the local  accumulation of charge in position-space and the rotational deformation caused by this accumulation in the phase-space fluid. It is well-known that phase-space electron and ion holes produce a well-defined local depression of particle charge in their regions. This would be in complete agreement with equation (\ref{tianvalues}f) since a vortical presence will be detected in any region of reduced particle charge, i.e.,
 \begin{equation}\label{vortexcondition}
 	g_{z\theta\rvert_{max}} =\begin{cases}
 	 \bar\rho(x)>0 &\text{for a -ve particle phase-space vortex}\\
 	 \bar\rho(x)<0 &\text{for a +ve particle phase-space vortex }
 	\end{cases}
 \end{equation}
 \indent A key phenomena of this vortex region in the phase-space fluid model and its correspondence to the trapping condition must be included in the preceeding analysis -- the presence of fluid elements with both low and high velocities in the phase-space plane. Like any ordinary fluid, in the region of a vortex, fluid particles with low velocity produce complete rotational deformation, indicating the `trapping' of particles in the vortex. However, fluid elements with higher fluid velocity enter and exit the vortex region showing an arc-like trajectory.  Clearly, this trapping will be determined by the kinetic energy of the fluid element in comparison to the potential $\Phi$ developed due to the local accumulation of charge in the vortex, as shown in equation (\ref{tianvalues}a,e), i.e.,
 \begin{subequations}\label{trappingcondition}
 	\begin{align}
 	&\frac{1}{2}mv_x^2 + \delta e\Phi\geq 0 \implies \text{particles escape the vortex.}\\
 	&\frac{1}{2}mv_x^2 + \delta e\Phi<0 \implies \text{particle trapping.} 
 	\end{align}
 \end{subequations}
 The conditions defined by equations (\ref{tianvalues}f), (\ref{vortexcondition}) and (\ref{trappingcondition}) can be used to identify vortices in the phase-space fluid (see Fig. \ref{vortexidenti}).\\
 \indent We therefore conclude in this section that electron holes and ion holes are in-fact vortices in the phase-space of the collision-free plasma. Having found this analytical proof of their vortical nature, we can now approach the problem of electron phase-space holes using the fluid-like kinetic approach described in sec. II of this article. We do this in the succeeding section wherein we apply the phase-space-fluid model to study the structure of an electron phase-space hole.\\
% Figure environment removed

\section{Application of phase-space fluid model to derive the phase-space density equations for an electron hole and the validity of schamel-df equations}

As an application of the phase-space-fluid model towards the study of electron phase-space vortices, we determine a steady-state equation for the phase-space density structure of this vortical region in the fluid phase-space. For the same, we use the fluid-momentum equation derived in equation (\ref{kineticeuler1}) along-with the isothermal pressure gradient introduced in equation (\ref{diff1}a). In the region of a vortex, the temperature $T=T_e^{\text{tr.}}$, which is the temperature of the trapped electrons.

\begin{equation}\label{keuler}
		\eta m\left( \frac{\partial v_x}{\partial t} + v_x\cdot \nabla_x v_x\right)= \eta q E(x) - K_B T_e^{\text{tr.}}\nabla_x \eta
		\end{equation}
  
	We aim to find the steady state solution\footnote{Here, we have used the momentum equation along position-space (\ref{keuler}) which seems to be a more descriptive relation between the phase-space density function $f(x,v_x)$ and its coordinates, and not the Vlasov equation (\ref{continuityeqn}), which is the continuity equation of the phase-space fluid.} of this equation in the region of the electron vortex, by setting the material derivative of the phase-space fluid velocity (left side of equation (\ref{keuler})) equal to zero, as follows --
	\begin{subequations}\label{schamelderive1}\allowdisplaybreaks
		\begin{align}
		\eta m_e\frac{D( v_x)}{Dt}=0\Rightarrow\frac{q}{m_e\tau}fE(x) = \frac{1}{\tau}\nabla_x f\frac{K_B T_e^{\text{tr.}}}{m_e}\\
		\Rightarrow -fq\frac{\partial \phi(x)}{\partial x}=K_BT_e^{\text{tr.}}\frac{\partial f}{\partial x}\\
        - fq\frac{\partial \phi(x)}{\partial x}dx =K_BT_e^{\text{tr.}}\frac{\partial f}{\partial x}dx\Rightarrow  - fq d\phi =K_BT_e^{\text{tr.}}df\\
        \Rightarrow \int \frac{df}{f} =-\frac{q}{K_BT_e^{\text{tr.}}}\int \frac{\partial \phi}{\partial x}dx =-\frac{q}{K_BT_e^{\text{tr.}}}\int d\phi. 
	\end{align}
\end{subequations}
For the above integration, we set the limits from within the vortex region ( $0\leq v_x < \sqrt{2e\phi/m_e}$) till its boundary ($v_x = \sqrt{2e\phi/m_e}$). For a constant Hamiltonian $H\leq 0$, change in potential can be expressed in terms of particle kinetic energy ($\mathcal{K}$) as follows --
\begin{subequations}\label{schamelderiv2}\allowdisplaybreaks
	\begin{align}
		&-qd\phi =  d\mathcal{K}\Rightarrow -q\Delta \phi = \Delta \mathcal{K} =q\phi - \frac{1}{2}mv_x^2 \\\intertext{Inserting in equation (\ref{schamelderive1}b) and normalising $\phi=\chi(K_B T_e q^{-1})$, $\mathcal{K}=\bar{v}_x^2 K_B T_e$, we get --}
	\begin{split}\ln f_{\text{bd.}}- \ln f_{\text{tr.}} = -\frac{1}{K_BT_e^{\text{tr}}}\left(\frac{1}{2}mv_x^2 - q\phi\right) \\= -\frac{K_BT_e}{K_B T_e^{\text{tr}}}\left(\bar v_x^2 - \chi\right)\end{split}\\
		&\Rightarrow \ln f_{\text{tr}} = -\left[(-\ln f_{\text{bd.}}) + \Bigl(-\frac{T_e}{T_e^{\text{tr}}} \Bigr)(\bar v_x^2 -\chi)\right],\text{ or,}\\
		\begin{split}\ln f_{\text{tr}} = -\left(M^2+ \beta(\bar v_x^2 - \chi)\right), \quad \text{where,}\quad  \\M^2 = -\ln f_{\text{bd.}},\qquad \beta = -\frac{T_e}{T_e^{\text{tr}}}.\end{split} \\
		& \Rightarrow \bar f_{\text{tr}} = \frac{1}{\sqrt\pi}\exp\left[-\bigl\{M^2+ \beta(\bar v_x^2 - \chi)\bigr\}\right].
	\end{align}
\end{subequations}

Using the same approach around the hole, it can be shown that the phase-space density remains a shifted Maxwellian in the un-trapped region --
\begin{equation}\label{freedf}
    \bar f_{\text{free}} = \frac{1}{\sqrt{\pi}}\exp\Bigl[ -\left(\bar M^2+\bar v_x^2 - \chi \right)\Bigr] , \qquad \bar v_x^2 \geq \chi.
\end{equation}

\indent The mathematics presented in equations (\ref{schamelderiv2}a-e, \ref{freedf}) analytically derive the schamel-df\cite{Schamel1979} equations for an electron hole. These equations, which describe the phase-space density $f$ for both trapped and free electrons and were initially assumed B.G.K. solutions to the Vlasov-Poisson system, can be very-well derived upon treating the electron phase-space of collision-less plasmas as a fluid. \\
\indent The equations (\ref{schamelderiv2}a-e) also provide a relation between the normalised electron hole speed $M$ in the reference frame of the perturbation and its structure in phase-space (refer to FIG. \ref{fig:holespeed} ). The normalised electron hole speed $M$ is, as described in equation (\ref{schamelderiv2}d), equal to the square root of the phase-space density of the electron hole at its boundary, i.e.,
\begin{equation}\label{M}
	M=\sqrt{-\ln( f^{\text{tr.}}_{\text{bd.}})}=\sqrt{-\ln \Bigl[\sqrt{\pi}\bar{f}^{\text{tr.}}\bigl({x,v_x=\sqrt{{2e\phi}/{m_e}}}\bigr)\Bigr]}.
\end{equation}
\indent Third, the definition of the particle trapping parameter $\beta$, as presented by Schamel in his works on particle trapping solitary B.G.K. waves \cite{Schamel1971, Schamel1972, Schamel1975, Schamel1979} can be extracted as a ratio of the plasma electron temperature to trapped electron temperature, as shown in equation (\ref{schamelderiv2}d). This is also in well-agreement to the initially assumed relation \cite{hutch2017}. Upon further analysis of the parabolic structure of this phase-space vortex, as clearly depicted from equation (\ref{schamelderiv2}c), a relation between this $\beta$ parameter and the depth of this vortex in the phase-space logarithmic surface can be deduced --
\begin{subequations}\label{betavalue}\allowdisplaybreaks
	\begin{align}
		\log_e\sqrt\pi\bar f_{tr} = -M^2 + \beta\chi -\beta \bar v_x^2\\
		\intertext{The minima of this parabolic surface is equal to $\ln f_{\text{min.}}^{\text{tr.}}$ at the centre of the hole, for which $\chi=\chi_0$ and $\bar v_x^2 =0$, with $\chi_0$ being the amplitude of the hole potential. Therefore,  }
		\beta = \frac{\ln f_{\text{min.}}^{\text{tr.}} + M^2}{\chi_0} = \frac{1}{\chi_0}\ln\left(\frac{f_{\text{min.}}^{\text{tr.}}}{f_{\text{bd.}}^{\text{tr.}}}\right) = \frac{\Gamma}{\chi_0},\quad \Gamma <0.
	\end{align}
\end{subequations}
The term $\Gamma = \ln{f_{\text{min.}}^{\text{tr.}}} - ln{f_{\text{bd.}}^{\text{tr.}}}$ in the above expression has been used to represent the (negative) depth of this phase-space vortex in the $\log_e f(x,v_x)$ surface. Thus,
\begin{equation}\label{depth-amplituderel}
    |\beta \chi_0| = |\Gamma|
\end{equation}
It is clear from this relation as to why the electron trapping parameter must be negative -- the value of $f_{\text{min.}}^{\text{tr.}}$ lies between $0$ and $f_{\text{bd.}}^{\text{tr.}}$, implying that $\Gamma \in (-\infty,0]$. Hence, $\beta$ must always be a non-positive value, its minima representing a hole with an empty core and maxima describing a flat surface electron hole. It is also clear from equation (\ref{depth-amplituderel}) that $\beta$ does not represent the hole depth in the phase-space plane ($-\Gamma$)\cite{Schamel2023PatternEquilibriab}, but is actually the ratio of hole phase-space depth to the hole potential amplitude.\\
\indent Upon moving across an electron phase-space hole along position-space, one can observe that the hole phase-space depth ($-\Gamma$) must be directly proportional to the hole potential, keeping a constant $\beta$ for the hole. Thus, for a single hole, 
\begin{equation}
    \chi_0 \propto -\Gamma.
\end{equation}
The same is shown in the numerical study of a solitary hole formed using the Q-machine plasma kinetic simulation\cite{Saeki1979} with a pulse amplitude of $4.3K_BT_ee^{-1}$ (refer to FIG. \ref{phigamma}).\\
% Figure environment removed
% Figure environment removed

\section{Discussions and Conclusion}
In this article, a fluid outlook of the particle phase-space has been presented. This approach, as has been shown, analytically confirms the vortical nature of phase-space holes. This implies the validity not just of their vortical behaviour, but also of the applicability of a fluid-like treatment in the study of these phase-space vortices using a different mathematical framework -- the kinetic-Euler equations -- which use the fluid-like nature of the phase-space to describe the structure of the holes. It confirms that, when the particle phase-space is treated as a two-dimensional fluid, a velocity field and a vorticity field associated with the flow-like behaviour of the particles (electrons and ions) in their respective phase-space can be defined. Various contortions can then be analysed using these fields and analogous to the analytical techniques of two-dimensional fluid-dynamics, the phase-space characteristics of such kinetic structures can then be studied.\\
\indent The study of electron holes using this approach provides an exact derivation of the schamel-df equations \cite{Schamel1979} which depict the phase-space structure of electron holes. It reproduces the solutions both for free and trapped particle densities and derives the definition of the particle trapping $\beta$ parameter as it was originally defined (as the inverse, normalised trapped electron temperature \cite{hutch2017}). It also provides an analytical relation between electron hole speed $M$, potential amplitude $\chi_0$, the trapping parameter $\beta$, and phase-space hole depth ($-\Gamma$), expressing $\beta$ as a ratio of $\Gamma$ and $\chi_0$. This is a measurable definition of the electron trapping parameter and simplifies\footnote{The measurement of hole trapping parameter can be directly calculated from simulation data of phase-space particle density minima and reduces both the analytical as well as computational burden of measuring the ratio of trapped and free electron temperatures.} the dispersion relation of the hole to
\begin{equation}
      \beta=\beta(\chi_0,-\Gamma),\quad
      \Gamma = \Gamma(M,\ln f_{\text{min.}}^{\text{tr.}}).
\end{equation}
Hence, this approach provides a precursor to the B.G.K. differential approach -- allowing one to not assume but derive a well-defined particle distribution function for the study of the phase-space vortices using the pseudo-potential method \cite{Schamel1979, schamel1986electron, schamel2012cnoidal, Schamel2023PatternEquilibriab}. \\
\indent It is also noteworthy to observe that the electron hole B.G.K. wave is observed as a vorticity wave travelling in the phase-space fluid, and therefore can be studied using the vorticity transport relation presented in equation (\ref{vorticityeqn1}), with the diffusion coefficient $\mathcal{D}$ defined in equation (\ref{diff2}b), which can be further calculated using the exact solution to the phase-space structure in the trapped region of electron holes which has been already derived. Hence, an alternate approach to the pseudo-potential method is presented as an outcome of the fluid-phase-space model.\\
\indent One must note that the stochastic nature of the studied kinetic phenomena and the abundance in the trapping mechanisms can also be accounted for in this model upon further exploration and modifications of the kinetic-Euler equations. One such modification is the rejection of the thermal equilibria of the trapped particles and adoption of a non-equilibrium state, allowing a time-evolution and stabilisation of the electron-hole structure. Such phenomena can also be explored as a part of further development of this model, which we leave for future study in this domain.
\bibliography{authors.bib}
\bibliographystyle{apalike}
\section*{Appendix}
The self-consistency of a fluid requires solving a Laplace equation of the stream function $\mathcal{H}(x,y)$, such that\cite{schamel2012cnoidal} --
    \begin{equation}\label{streamer}
        \mathcal{V}(x,y) = \nabla \mathcal{H} \times \hat{n},
    \end{equation}
    where, $\hat{n}$ is a direction normal to the two-dimensional fluid. The same, in the case of the fluid-analogue of the particle phase-space, can be described as a modified Hamiltonian $H$ of the particles --
    \begin{equation}\label{streamfunc}
        \mathcal{H} = \frac{\tau}{m}H = \frac{\tau}{m}\left(\frac{1}{2}mv_x^2 + q\phi(x)\right).
    \end{equation}
    Here, $\phi(x)$ is the potential field experienced by the particle of charge $q$. Inserting equation (\ref{streamfunc}) in equation (\ref{streamer}), we see that\footnote{Since $\frac{d}{dx}\neq \partial_x$, the same can be implied to show that $\partial_x (\frac{1}{2}mv_x^2)=0$.} --
    \begin{subequations}
        \begin{align}
            \nabla \mathcal{H} &= \frac{\tau}{m}\left( q\frac{\partial \phi}{\partial x}\hat{x} + \frac{m}{2\tau}\frac{\partial v_x^2}{\partial v_x}\hat{v}_x\right)\\
            &=\frac{q\tau}{m}\frac{\partial \phi}{\partial x}\hat{x} + v_x\hat{v}_x.\\
            \nabla \mathcal{H}\times\hat{n} &=\frac{q\tau}{m}\frac{\partial \phi}{\partial x}(\hat{x}\times\hat{n}) + v_x(\hat{v}_x\times\hat{n})\\
            &=\frac{q\tau}{m}\left(-\frac{\partial \phi}{\partial x}\right)\hat{v}_x + v_x\hat{x} = \mathcal{V}.
        \end{align}
    \end{subequations}
    Therefore, upon using the relation between $\hat{n}$, $\hat{x}$ and $\hat{v}_x$ defined in equation (\ref{vorticity}), equation (\ref{streamer}) is satisfied. The stream function (\ref{streamfunc}) can then be acted upon by the Laplacian operator in phase-space to reproduce the vorticity expression defined in equation (\ref{vorticity}) --
    \begin{subequations}
        \begin{align}
            \nabla \cdot \nabla\mathcal{H} &=\frac{q\tau}{m}\frac{\partial^2 \phi}{\partial x^2} + \frac{1}{\tau}\frac{\partial v_x}{\partial v_x}\\
            &=-\left(\frac{\tau}{m}\frac{\partial \bm{F}(x)}{\partial x} - \frac{1}{\tau}\right) = - \bm{\xi}(x),\\
            \text{where,}&\quad \bm{F}(x)=-q\frac{\partial\phi}{\partial x}\hat{x}.
        \end{align}
    \end{subequations}
    This shows the self-consistency of the fluid-analogue of the particle phase-space.
\end{document}