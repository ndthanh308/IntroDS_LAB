%%%%%%%%%%%%%%%%%%%%%%%%%%%%%%%%%%%%%%%%%%%%%%%%%%%%%%%%%%%%%%%%%%%%%%%%%%%%
%% Trim Size : 11in x 8.5in
%% Text Area : 9.6in (include Runningheads) x 7in
%% ws-ijbc.tex, 24 Jan 2010
%% Tex file to use with ws-ijbc.cls written in Latex2E.
%% The content, structure, format and layout of this style file is the
%% property of World Scientific Publishing Co. Pte. Ltd.
%%%%%%%%%%%%%%%%%%%%%%%%%%%%%%%%%%%%%%%%%%%%%%%%%%%%%%%%%%%%%%%%%%%%%%%%%%%%
%%
%\documentclass[draft]{ws-ijbc}
\documentclass{ws-ijbc}
\usepackage{ws-rotating}     % used only when sideways tables/figures are used
\usepackage{graphicx}
\usepackage{epstopdf}
\usepackage{subfigure}
\usepackage{color}


\begin{document}

\catchline{}{}{}{}{} % Publisher's Area please ignore

\markboth{Yong Yao}{Leslie-Gower type predator-prey system with herd behavior and harvesting}

\title{Dynamics of a Leslie-Gower type predator-prey system with herd behavior and constant harvesting in prey}

\author{Yong Yao\footnote{ Author for correspondence}}

\address{School of Mathematics and Physics, Wuhan Institute of Technology,\\
Wuhan, Hubei 430205, P.R. China\\
mathyaoyong@163.com}


\maketitle

\begin{history}
\received{(to be inserted by publisher)}
\end{history}

\begin{abstract}
In this paper, the dynamics of a Leslie-Gower type predator-prey system with herd behavior and constant harvesting in prey are investigated.
Earlier work has shown that the herd behavior in prey merely induces a supercritical Hopf bifurcation in the classic Leslie-Gower predator-prey system in the absence of harvesting.
However, the work in this paper shows that the presence of herd behavior and constant harvesting in prey can give rise to numerous kinds of  bifurcation at the non-hyperbolic equilibria in the classic Leslie-Gower predator-prey system such as two saddle-node bifurcations and one Bogdanov-Takens bifurcation of codimension two at the degenerate equilibria and one degenerate Hopf bifurcation of codimension three at the weak focus.
Hence, the research results reveal that the herd behavior and constant harvesting in prey have a strong influence on the dynamics and also contribute to promoting the ecological diversity and maintaining the long-term economic benefits.


\end{abstract}

\keywords{Leslie-Gower system; herd behavior; constant harvesting;  Bogdanov-Takens bifurcation; degenerate Hopf bifurcation.}

%\begin{multicols}{2}
\section{Introduction}
\noindent
Mathematical modeling is a proven approach to explore the underlying
dynamics of interactions among populations in ecosystems.
A huge number of mathematical models taking various forms consequently appeared for the subject of
modeling the specific manners of those interactions.
Particularly, predator-prey models are frequently established as bi-dimensional autonomous systems of ordinary differential equations(ODEs) to model the predation interaction. One of the fundamental and classic model describing the predator-prey interaction is the Leslie-Gower model [Leslie, 1948; Leslie \& Gower, 1960]
\begin{eqnarray}
\left\{
\begin{array}{l}
\frac{dx}{dt}=r_1x(1-\frac{x}{k})-axy,\\
\frac{dy}{dt}=r_2y(1-\frac{y}{px}),
\end{array}
\right.
\label{(0.1)}
\end{eqnarray}
where $x$ and $y$ are the population densities of prey and predator, respectively, $r_1$ and $r_2$ represent the intrinsic growth rates of
prey and predator, respectively, $k$ and $px$ take on the role of carrying capacities of prey and predator, respectively,
$ax$ is the per-unit predator extraction rate of prey (the functional response).
The typical feature of the Leslie-Gower model is that
the growth for the predators is assumed as a logistic-like equation
with a carrying capacity proportional to the density of prey.
The dynamics of system (\ref{(0.1)}) is simple due to the linear functional response, that is the unique interior equilibrium is globally asymptotically stable [Hsu \& Hwang, 1995; Korobeinikov, 2001].
As a consequence of the advancement of ecological knowledge, more elements in predation interactions (antipredator behavior(APB), for instance) are recognized as being essential to gain realities in the process of modeling, which are formulated by nonlinear functional responses generally.
Furthermore, it is shown that more complicated dynamics such as more non-hyperbolic equilibria and bifurcations with higher codimension
in the Leslie-Gower type models may be attributed to the nonlinear functional responses, for instance, Holling types II [Braza, 2003; Hsu
\& Hwang, 1995], III [Dai {\it et al}., 2019; Hsu \& Hwang, 1995; Huang {\it et al}., 2008] and IV [Dai \& Zhao,
2018; Huang {\it et al}., 2016; Li \& Xiao, 2007; Gonz\'alez-Olivares {\it et al}., 2022; Zhang \& Su, 2021].


A type of APB, prey herd behavior, has received much attention recently, which means that each prey individual demonstrates collective social
conduct and realizes the reaction same as that effectuated by the other members to protect themselves against predation.
Generally, the prey species implements the behavior by forming groups with concentrating the weaker individuals in the interior and leaving the stronger ones on the boundary.
This  defensive strategy is also a common phenomenon, for example, large herbivores usually gather together in huge herds to avoid predation.
It is thereby more reasonable to take into account the element in predator-prey models, which is usually depicted by non-differentiable nonlinear functional responses (see e.g. [Ajraldi {\it et al}., 2011; Braza, 2012; Pal {\it et al}., 2016; Venturino \&
Petrovskii, 2013; Vilches {\it et al}., 2018; Xu {\it et al}., 2016]).
For instance, Rosenzweig [1971] proposed functional response $ax^\alpha$ with $\alpha\in(0,1)$ called Rosenzweig functional response to model the prey herd behavior, where exponent $\alpha$ stands for a kind of aggregation efficiency.
Gonz\'alez-Olivares {\it et al} [2022] established the Leslie-Gower type model with the Rosenzweig functional response and investigated the dynamics including qualitative properties of equilibria, existence of homoclinic and heteroclinic orbits and limit cycle as well as bifurcations.
They concluded that the dynamics have a clear difference compared with the Leslie-Gower model (\ref{(0.1)}), or rather the former has richer dynamics.
The special case of the Rosenzweig functional response is $a\sqrt{x}$ called  square root functional response and proposed by Gause [1934].
He and Li [2023] considered the Leslie-Gower type model with the square root functional response taking the form
\begin{eqnarray}
\left\{
\begin{array}{l}
\frac{dx}{dt}=r_1x(1-\frac{x}{k})-a\sqrt{x}y,\\
\frac{dy}{dt}=r_2y(1-\frac{y}{px}),
\end{array}
\right.
\label{(0.2)}
\end{eqnarray}
and discussed the global dynamics. They claimed that the unique interior equilibrium of system (\ref{(0.2)}) is either globally asymptotically stable or unstable and system (\ref{(0.2)}) admits a unique stable limit cycle induced by the supercritical Hopf bifurcation. Therefore,
the Leslie-Gower type model with prey herd behavior still has more complicated dynamics compared to the Leslie-Gower model (\ref{(0.1)}) even in the special case.




A type of interaction between humans and species, harvesting, actually is
an excellent strategy to keep species in balance and provide economic benefits for humans.
However, it is well known that several species become extinct and many others are the verge of
extinction nowadays resulting from over-exploitation of species.
Therefore, it is necessary to develop some sustainable development strategies depending on mathematical modeling to protect species.
Until now, several forms of harvesting rate have been proposed in predator-prey models mainly consisting of constant harvesting,
proportional harvesting and nonlinear harvesting.
For the specific situation that prey species is harvested at the constant harvesting rate and predator species is not of commercial importance,
some corresponding predator-prey models were built and investigated (see e.g.  [Etoua \& Rousseau, 2010; Gong \& Huang, 2014; Huang {\it et al}., 2013; Lan \& Zhu, 2011; Peng {\it et al}., 2009; Xiao \& Jennings, 2005; Zhu \& Lan, 2010]).
For example, researchers in [Gong \& Huang, 2014; Zhu \& Lan, 2010] shown that the Leslie-Gower model (\ref{(0.1)}) with constant prey harvesting can exhibit  rich bifurcations such as saddle-node bifurcation, supercritical and subcritical Hopf bifurcations and Bogdanov-Takens bifurcation of  codimension two,
which indicates that the basic prey harvesting has a strong impact on the dynamics of predator-prey system compared to the Leslie-Gower model (\ref{(0.1)}).
For the more specific situation that the prey of some species does exhibit the herd behavior when they simultaneously encounter human intervention and predation, some predator-prey models were established to simulate this situation (see e.g.  [Hacini {\it et al}., 2021; Kumar \& Kharbanda, 2019; Luo \& Zhao, 2017; Mortuja {\it et al}., 2021]).
For example, Luo and Zhao in  [Luo \& Zhao, 2017] considered a predator-prey model with prey herd behavior and harvesting under two cases, i.e.,
prey harvesting only and predator harvesting only.
They analyzed the saddle-node bifurcation and the Hopf bifurcation for both cases and the Bogdanov-Takens bifurcation of codimension two for the case of predator harvesting only.
Conclusively, the harvesting in predator-prey systems can make various dynamical behaviors regardless of functional responses.




Motivated by the previous works,  we assume that the prey species exhibits herd behavior against predation and is harvested continuously at the constant rate $H$ by humans but the predator species is not of commercial importance in this present paper, which is formulated as Leslie-Gower type predator-prey model (\ref{(0.2)}) with constant prey harvesting as follows
\begin{eqnarray}
\left\{
\begin{array}{l}
\frac{dx}{dt}=r_1x(1-\frac{x}{k})-a\sqrt{x}y-H,\\
\frac{dy}{dt}=r_2y(1-\frac{y}{px}).
\end{array}
\right.
\label{(0.3)}
\end{eqnarray}
For mathematical simplicity, we nondimensionalize model (\ref{(0.3)}) by
\begin{eqnarray*}
\left.
\begin{array}{l}
\bar{x}:=\frac{x}{k},~~ \bar{y}:=\frac{ay}{r_1\sqrt{k}}, ~~\bar{t}:=r_1t,~~ s:=\frac{r_2}{r_1},~~ n:=\frac{ap\sqrt{k}}{r_1},~~ h:=\frac{H}{kr_1}
\end{array}
\right.
\end{eqnarray*}
and drop the bars. Then model (\ref{(0.3)}) takes the form
\begin{eqnarray}
\left\{
\begin{array}{l}
\frac{dx}{dt}=x(1-x)-\sqrt{x}y-h,\\
\frac{dy}{dt}=sy(1-\frac{y}{nx}).
\end{array}
\right.
\label{(1.1)}
\end{eqnarray}
Because of both biological meaning and not well-defined on the vertical axis, we restrict our attention to system (\ref{(1.1)}) in $\Omega:=\{(x, y): x>0, y\geq0\}$. We are concerned with the dynamics of model (\ref{(1.1)}) in $\Omega$ so as to explore the dual influences of predation and human harvesting on prey species. This paper is organized as follows. In section 2, we discuss the existence of
equilibria and their various topological types of model (\ref{(1.1)}).  In section 3,  we discuss all possible bifurcations of  model (\ref{(1.1)}) including the saddle-node bifurcation, the Bogdanov-Takens bifurcation of codimension two and the degenerate Hopf bifurcation of codimension three. In section 3, we give a brief discussion about our results.






\section{Equilibria}
The objective of this section is to perform the existence and qualitative analysis of equilibria for system (\ref{(1.1)}), which can be given by the following theorem.
\begin{theorem}
System (\ref{(1.1)}) has at most two boundary equilibria. More precisely, system (\ref{(1.1)}) has no equilibrium if $h>\frac{1}{4}$,
one boundary equilibrium $E_{1*}: (\frac{1}{2}, 0)$ if $h=\frac{1}{4}$, which is a degenerate equilibrium,
two boundary equilibria $E_{11}: (x_{11}, 0)$ and $E_{12}: (x_{12}, 0)$ if $0<h<\frac{1}{4}$, where $E_{11}$ is an unstable node, $E_{12}$ is a saddle, and
\begin{eqnarray*}
\left.
\begin{array}{l}
x_{11}:=\frac{1-\sqrt{1-4h}}{2} ~~~\mbox{and}~~~
x_{12}:=\frac{1+\sqrt{1-4h}}{2}.
\end{array}
\right.
\end{eqnarray*}
System (\ref{(1.1)}) has at most two interior equilibria.
The exact number of interior equilibria and their qualitative properties  are described in Table 1.
\label{thm1}
\end{theorem}

\begin{table}[h]
\tbl{Interior equilibria.}
{\begin{tabular}{l l l}\\[-2pt]
\toprule
Parameters& Number &Equilibria \\[6pt]
\hline\\[-2pt]
$s>0$,~$n>0$,~$h>h_1$  & $0$ &\\[1pt]
$s>0$,~$n>0$,~$h=h_1$  & $1$& $E_{2*}$(degenerate)\\[1pt]
$s\geq\frac{2}{3}, ~n>0,~0<h<h_1$ & $2$ & $E_{21}$(saddle) ~ $E_{22}$(stable focus or node)\\[1pt]
$0<s<\frac{2}{3}, ~0<n\leq n_1, ~0<h<h_1$ & $2$ &$E_{21}$(saddle) ~ $E_{22}$(stable focus or node)\\[1pt]
$\frac{1}{2}\leq s<\frac{2}{3},~ n>n_1, ~0<h<h_2$ & $2$ &$E_{21}$(saddle) ~ $E_{22}$(stable focus or node)\\[1pt]
$0<s<\frac{1}{2}, ~n_1<n<n_2, ~0<h<h_2$ & $2$ & $E_{21}$(saddle) ~ $E_{22}$(stable focus or node)\\[1pt]
$\frac{1}{2}\leq s<\frac{2}{3}, ~n>n_1, ~h=h_2$ &$ 2 $&$E_{21}$(saddle) ~ $E_{22}$(center type)\\[1pt]
$0<s<\frac{1}{2}, ~n_1<n<n_2, ~h=h_2$ & $2$ &$E_{21}$(saddle) ~ $E_{22}$(center type)\\[1pt]
$0<s<\frac{1}{2}, ~n\geq n_2, ~0<h<h_1$ & $2$ &$E_{21}$(saddle)  ~~$E_{22}$(unstable focus or node)\\[1pt]
$\frac{1}{2}\leq s<\frac{2}{3}, ~n>n_1,~ h_2<h<h_1$ &$2$& $E_{21}$(saddle) ~ $E_{22}$(unstable focus or node)\\[1pt]
$0<s<\frac{1}{2},~n_1<n<n_2,~ h_2<h<h_1$&$2$&$E_{21}$(saddle) ~ $E_{22}$(unstable focus or node)\\[1pt]
\hline\\[-2pt]
\multicolumn{3}{c}{$n_1:=\frac{2s}{\sqrt{2-3s}}$, ~$n_2:=\frac{2(s+1)}{\sqrt{3-6s}}$, ~$h_1:=z_*^2(-z_*^2-n z_*+1)$, }
\\[8pt]
\multicolumn{3}{c}{~$h_2:=z_0^2(-z_0^2-n z_0+1)$,~$z_*:=\frac{-3 n+\sqrt{9 n^2+32}}{8}$,~$z_0:=\frac{-n+\sqrt{n^2-32 s+32}}{8}$
}
\\[2pt]
\botrule
\end{tabular}}
\end{table}

\begin{proof}
Equilibria of system (\ref{(1.1)}) are decided by the prey nullcline and predator nullcline, which are given by algebraic equations
\begin{eqnarray}
\left\{
\begin{array}{l}
x(1-x)-\sqrt{x}y-h=0,\\
sy(1-\frac{y}{nx})=0.
\end{array}
\right.
\label{(2.1)}
\end{eqnarray}
In the case $y=0$,  boundary equilibria are decided by the equation
$x^2-x+h=0$.
It is easy to see system (\ref{(1.1)}) has no boundary equilibrium, exactly one boundary equilibrium $E_{1*}: (\frac{1}{2}, 0)$, and exactly two boundary equilibria $E_{11}: (x_{11}, 0)$ and $E_{12}: (x_{12}, 0)$ with
\begin{eqnarray*}
\left.
\begin{array}{l}
x_{11}=\frac{1-\sqrt{1-4h}}{2} ~~~\mbox{and}~~~
x_{12}=\frac{1+\sqrt{1-4h}}{2}
\end{array}
\right.
\end{eqnarray*}
if and only if $h>\frac{1}{4}$, $h=\frac{1}{4}$, and $0<h<\frac{1}{4}$, respectively.
In the other case $y\neq0$, substituting $y=nx$ in the first equation of (\ref{(2.1)}), we have the quartic equation
\begin{eqnarray}
\left.
\begin{array}{l}
F(z):=z^4+nz^3-z^2+h
\end{array}
\right.
\label{(2.2)}
\end{eqnarray}
with $z:=\sqrt{x}$.
In order to find all the interior equilibria of system (\ref{(1.1)}), we need to discuss all the positive zeros of quartic equation (\ref{(2.2)}).
However, there are difficulties to solve the zeros of quartic equation by the formulae of quartic roots directly,  but we can use the idea of monotonic interval partition to discuss the distribution of positive zeros. The derivative of quartic equation (\ref{(2.2)}) is
\begin{eqnarray}
\left.
\begin{array}{l}
F'(z)=z(4z^2+3nz-2),
\end{array}
\right.
\label{(2.3)}
\end{eqnarray}
which has one positive zero $z_*$ given in Table 1.
The positive axis is partitioned into two monotonic intervals by $z_*$, i.e., $F(z)$ is strictly decreasing on $(0, z_*)$ and strictly increasing
on $(z_*, +\infty)$. Clearly, $F(z)$ also satisfies the fact $F(0)>0$ and $F(+\infty)=+\infty$.
The above discussion shows that $F(z)$ has no positive zero, exactly one positive zero $z_*$, and exactly two positive zeros $z_{21}$ and $z_{22}$  with $z_{21}<z_{2*}<z_{22}$ if and only if $F(z_*)>0$, $F(z_*)=0$,
and $F(z_*)<0$, respectively.
Therefore, system (\ref{(1.1)}) has no interior equilibrium, exactly one interior equilibrium $E_{2*}:(x_{2*}, nx_{2*})$, and exactly two interior equilibria $E_{21}:(x_{21}, nx_{21})$ and $E_{22}:(x_{22}, nx_{22})$ with $x_{2*}:=z_*^2$ and $x_{2i}:=z_{2i}^2$ $(i=1, 2)$ if and only if
$h>h_1$, $h=h_1$, and $0<h<h_1$, respectively. The expression of $h_1$ is given in Table 1.


In what follows, we further discuss qualitative properties of the equilibria of system (\ref{(1.1)}). The Jacobian matrix of system (\ref{(1.1)}) at an equilibrium $(x, y)$ is
$$
J(x, y):=
\left(\begin{array}{lr}\frac{-4 x\sqrt{x}+2 \sqrt{x}-y}{2\sqrt{x}}& -\sqrt{x}\\
\frac{s y^2}{n x^2}  & \frac{s (n x-2 y)}{n x}
\end{array}\right).
$$
Let $D(x, y)$ and $T(x, y)$ denote the determinant and trace of Jacobian matrix $J(x, y)$ respectively,
where
\begin{eqnarray}
\left.
\begin{array}{l}
D(x, y):=\frac{s \{4 y^2+(8 x-n \sqrt{x}-4) \sqrt{x} y-2 n x (2 x-1) \sqrt{x}\}}{2 n x \sqrt{x}},\\
T(x, y):=\frac{(-n \sqrt{x}-4 s) y+2 n x (s-2 x+1)}{2 n x}.
\end{array}
\right.
\label{(2.5)}
\end{eqnarray}
It is clear that $E_{11}$ is an unstable node since $D(x_{11}, 0)=s\sqrt{1-4h}>0$, $T(x_{11}, 0)=\sqrt{1-4h}+s>0$ and $(T(x_{11}, 0))^2-4D(x_{11}, 0)=(\sqrt{1-4h}-s)^2\geq0$, $E_{12}$ is a saddle since $D(x_{12}, 0)=-s\sqrt{1-4h}<0$, and $E_{1*}$ is a degenerate equilibrium since $D(\frac{1}{2}, 0)=0$ and $T(\frac{1}{2}, 0)=s>0$.
$D(x, y)$ and $T(x, y)$ at interior equilibria can be reduced to the following forms by eliminating with $x=z^2$ and $y=nz^2$
\begin{eqnarray*}
\left.
\begin{array}{l}
D(z)=\frac{sF'(z)}{2z}~~\mbox{and}~~
T(z)=-2z^2-\frac{n}{2}z-s+1,
\end{array}
\right.
\end{eqnarray*}
which shows that $D(z)$ has the same sign as $F'(z)$.
Therefore, $E_{21}$ is a saddle because of $F'(z_{21})<0$,
$E_{22}$ is a focus, node or of center type because of $F'(z_{22})>0$, and
$E_{2*}$ is a degenerate equilibrium because of  $F'(z_*)=0$.
To investigate the topological types of equilibrium $E_{22}$, the signs of $T(z_{22})$ need to be further discussed.
It follows that $T(z)$ is strictly decreasing on $(0, +\infty)$ and has one positive zero $z_0$ given in Table 1.
We consider two subcases: {(\bf C1)}~$s\geq 1$ and {(\bf C2)}~$0<s<1$.
In subcase {(\bf C1)}, it is clear that $T(z_{22})<0$, which implies that equilibrium $E_{22}$ is a stable focus or node.
In subcase {(\bf C2)}, we can obtain the signs of $T(z)$ by the monotonicity of $T(z)$ on $(0, +\infty)$ and the relative positions of $z_0$ and $z_{22}$. By analysis and calculation, it shows that $z_0<z_{22}$ under parameter conditions displayed from the third line to the sixth line in Table 1, i.e., $T(z_{22})<0$, which implies that $E_{22}$ is a stable focus or node, $z_0=z_{22}$ under parameter conditions displayed in the seventh and  eighth lines of Table 1, i.e., $T(z_{22})=0$, which implies that $E_{22}$ is of center type, $z_0>z_{22}$ under parameter conditions displayed from the ninth line to the eleventh line in Table 1, i.e., $T(z_{22})>0$, which implies that $E_{22}$ is an unstable focus or node.
\end{proof}

\begin{remark}
Incidentally, we can check that the two thresholds for existence of equilibria satisfy the inequality relation $0<h_1<\frac{1}{4}$, which means that system (\ref{(1.1)}) has no any equilibrium in $\Omega$ if $h>\frac{1}{4}$.
\label{rem1}
\end{remark}

On the basis of Theorem \ref{thm1}, the topological types of degenerate equilibria $E_{1*}$ and $E_{2*}$ need to be further identified.
Direct computation shows that $T(z_*)\neq0$ if $h=h_1$ and either $n>0$, $s\geq\frac{2}{3}$ or $0<n<n_1$, $0<s<\frac{2}{3}$ or $n>n_1$, $0<s<\frac{2}{3}$, which implies equilibrium $E_{2*}$ is degenerate with  one simple zero eigenvalue, $T(z_*)=0$ if $h=h_1$, $n=n_1$ and $0<s<\frac{2}{3}$, which implies equilibrium $E_{2*}$ is degenerate with one double zero eigenvalue. The expression of $n_1$ is displayed in Table 1.
The following theorem indicates that it is either a saddle-node or a cusp.

\begin{theorem}
Equilibrium $E_{1*}$ is always a saddle-node.
Equilibrium $E_{2*}$ is a saddle-node if $h=h_1$ and either $n>0$, $s\geq\frac{2}{3}$ or $n\neq n_1$, $0<s<\frac{2}{3}$, and a cusp if $h=h_1$, $n=n_1$ and $0<s<\frac{2}{3}$.
\label{thm2}
\end{theorem}

\begin{proof}
For equilibrium $E_{1*}$, applying an invertible linear transformation $x=u+v+\frac{1}{2}$, $y =-\sqrt{2}sv$  and a time-rescaling  $d\tau=sdt$ to normalize the linear part of system (\ref{(1.1)}), we obtain the Taylor expansion of system (\ref{(1.1)}) at the origin as follows
\begin{eqnarray*}
\left\{
\begin{array}{l}
\frac{du}{d\tau}=-\frac{1}{s}u^2+(1-\frac{2}{s})uv-\frac{2\sqrt{2}s^2-sn+n}{sn}v^2+O(|u,v|^3),\\
\frac{dv}{d\tau}=v+\frac{2\sqrt{2}s}{n}v^2+O(|u,v|^3).
\end{array}
\right.
\end{eqnarray*}
Then from the right-hand side of the second equation in the above system, we obtain the unique solution $v=0$ near the origin.
The right-hand side of the first equation in the above system becomes $-\frac{1}{s}u^2$ by substituting $v=0$,
which indicates that the origin is a saddle-node. Furthermore, its parabolic sector lies in the
left-half plane of $(u, v)$-coordinate  [Zhang {\it et al}., 1992]. Therefore, equilibrium $E_{1*}$ is a saddle-node.

For equilibrium $E_{2*}$ with $h=h_1$ and either $n>0$, $s\geq\frac{2}{3}$ or $n\neq n_1$, $0<s<\frac{2}{3}$, the proof is similar to the last.
Using the transformation $x=u+\frac{z_*}{s}v+z_*^2$, $y=nu+v+nz_*^2$ together with the time-rescaling $d\tau=(nz_*-s)dt$ to transform $E_{2*}$ to the origin and normalize the linear part, we can change system (\ref{(1.1)}) into the following
\begin{eqnarray*}
\left\{
\begin{array}{l}
\frac{du}{d\tau}=\frac{s (8 z_*+3 n)}{8 z_* (n z_*-s)^2} u^2-\frac{9 n^2 z_*^2-8 n z_*^3-20 n s z_*+8 s^2}{8 s z_* (n z_*-s)^2 n} v^2+\frac{8 z_*^2+n z_*+2 s }{4 z_* (n z_*-s)^2}u v+O(|u,v|^3):=\Phi(u,v),\\
\frac{dv}{d\tau}=v-\frac{s n (8 z_*+3 n)}{8 z_* (n z_*-s)^2} u^2+\frac{n^3 z_*^3-8 n^2 z_*^4+4 n^2 s z_*^2-16 n s^2 z_*+8 s^3}{8 z_*^2 s (n z_*-s)^2 n} v^2-\frac{n (8 z_*^2+n z_*+2 s)}{4 z_* (n z_*-s)^2} u v+O(|u,v|^3):=\Psi(u,v).
\end{array}
\right.
\end{eqnarray*}
Solving the equation $\Psi(u,v)=0$, we obtain the unique implicit function
\begin{eqnarray*}
\left.
\begin{array}{l}
v=\phi(u):=\frac{n s (8 z_*+3 n)}{8 z_* (n^2 z_*^2-2 n s z_*+s^2)} u^2+O(u^3).
\end{array}
\right.
\end{eqnarray*}
Substituting $v=\phi(u)$ into the other equation $\Phi(u,v)$, we obtain
\begin{eqnarray*}
\left.
\begin{array}{l}
\Phi(u,\phi(u))=\frac{s (8 z_*+3 n)}{8 z_* (n z_*-s)^2} u^2+O(u^3)
\end{array}
\right.
\end{eqnarray*}
with
\begin{eqnarray*}
\left.
\begin{array}{l}
\frac{s (8 z_*+3 n)}{8 z_* (n z_*-s)^2}=\frac{64s \sqrt{9 n^2+32}}{(-3 n+\sqrt{9 n^2+32}) ( n\sqrt{9 n^2+32}-3 n^2-8 s)^2}>0
\end{array}
\right.
\end{eqnarray*}
for either $n>0$, $s\geq\frac{2}{3}$ or $n\neq n_1$, $0<s<\frac{2}{3}$,
which shows that the origin is a saddle-node and its parabolic sector lies in the
right-half plane of $(u, v)$-coordinate  [Zhang {\it et al}., 1992]. Therefore, equilibrium $E_{2*}$ is a saddle-node if $h=h_1$ and either $n>0$, $s\geq\frac{2}{3}$ or $n\neq n_1$, $0<s<\frac{2}{3}$.

For the other case $h=h_1$, $n=n_1$ and $0<s<\frac{2}{3}$ with $h_1=\frac{(3s-2)(s-2)}{16}$, using the linear transformation
\begin{eqnarray*}
\left.
\begin{array}{l}
x =\frac{\sqrt{2-3s}}{2s}u+v-\frac{3s}{4}+\frac{1}{2}, ~~~y=u+\frac{s\sqrt{2-3s}}{2}
\end{array}
\right.
\end{eqnarray*}
and the time-rescaling $d\tau=\frac{2s^2}{\sqrt{2-3s}}dt$ to transform $E_{2*}: (\frac{2-3 s}{4}, \frac{s\sqrt{2-3s}}{2})$ to the origin and normalize the linear part, we can rewrite system (\ref{(1.1)}) as the following
\begin{eqnarray}
\left\{
\begin{array}{l}
\frac{du}{d\tau}=v+\frac{4}{3s-2}v^2+O(|u,v|^3),\\
\frac{dv}{d\tau}=-\frac{(3 s-2) (3 s-4)}{16 s^4 \sqrt{2-3 s}} u^2-\frac{(15 s-4) (3 s-2)}{4 s^2 (2-3 s)\sqrt{2-3 s}} v^2
+\frac{5 s-4}{4 s^3} u v+O(|u,v|^3).
\end{array}
\right.
\label{(2.7)}
\end{eqnarray}
Next, with the transformation $(u,v)\rightarrow(u_1,v_1)$, where $v_1$ denotes the right-hand side of the first equation in system (\ref{(2.7)}),
we obtain
\begin{eqnarray}
\left\{
\begin{array}{l}
\frac{du_1}{d\tau}=v_1,\\
\frac{dv_1}{d\tau}=-\frac{(3 s-2) (3 s-4)}{16 s^4 \sqrt{2-3 s}} u_1^2-\frac{(15 s-4) (3 s-2)}{4 s^2 (2-3 s)\sqrt{2-3 s}} v_1^2
+\frac{5 s-4}{4 s^3}u_1v_1+O(|u_1,v_1|^3).
\end{array}
\right.
\label{(2.8)}
\end{eqnarray}
Then, using the change of variables $u_1 = u_2$ and
\begin{eqnarray*}
\left.
\begin{array}{l}
v_1 = v_2+\frac{15 s-4}{4 s^2 \sqrt{2-3 s}} u_2 v_2,
\end{array}
\right.
\end{eqnarray*}
and the time-rescaling
\begin{eqnarray*}
\left.
\begin{array}{l}
dt=(1+\frac{15s-4}{4 s^2 \sqrt{2-3s}} u_2)d\tau
\end{array}
\right.
\end{eqnarray*}
to eliminate the term of $v_1^2$ in the second equation of system (\ref{(2.8)}), we obtain
\begin{eqnarray}
\left\{
\begin{array}{l}
\frac{du_2}{dt}=v_2,\\
\frac{dv_2}{dt}=-\frac{(3 s-2) (3 s-4)}{16 s^4 \sqrt{2-3 s}} u_2^2+\frac{5 s-4}{4 s^3} u_2v_2+O(|u_2,v_2|^3).
\end{array}
\right.
\label{(2.9)}
\end{eqnarray}
Further, using the rescaling
\begin{eqnarray*}
\left.
\begin{array}{l}
u_2=\frac{(3 s-4) s^2 \sqrt{2-3 s}}{(5 s-4)^2}u_3,~~~
v_2=\frac{(3 s-4)^2 (3 s-2)s}{4 (5 s-4)^3} v_3,~~~
d\tau=-\frac{(3 s-4) \sqrt{2-3 s}}{4 s (5 s-4)}dt
\end{array}
\right.
\end{eqnarray*}
to reduce the coefficients of the terms $u_2^2$ and $u_2v_2$ to $1$ and $-1$, system (\ref{(2.9)}) can be rewritten as
\begin{eqnarray}
\left\{
\begin{array}{l}
\frac{du_3}{d\tau}=v_3,\\
\frac{dv_3}{d\tau}=u_3^2-u_3v_3+O(|u_3,v_3|^3),
\end{array}
\right.
\label{(2.10)}
\end{eqnarray}
which implies that the origin is a cusp of codimension two [Zhang {\it et al}., 1992]. Therefore, equilibrium $E_{2*}$ is a cusp of codimension two if $h=h_1$, $n=n_1$ and $0<s<\frac{2}{3}$.
\end{proof}



\section{Bifurcations}
In this section, we further investigate various bifurcations of system (\ref{(1.1)}) at those nonhyperbolic equilibria $E_{1*}$, $E_{2*}$ and $E_{22}$ and derive corresponding bifurcation conditions. According to the theorems in Section 2, we obtain that system (\ref{(1.1)}) may undergo the saddle-node bifurcations at both degenerate equilibria $E_{1*}$ and $E_{2*}$, the Bogdanov-Takens bifurcation at equilibrium $E_{2*}$ and the Hopf bifurcation at nonhyperbolic equilibrium $E_{22}$, respectively.


\subsection{Saddle-node bifurcations}
Now, we discuss saddle-node bifurcations at degenerate equilibria $E_{1*}$ and $E_{2*}$ by perturbing parameter $h$ and obtain the following result.
\begin{theorem}
System (\ref{(1.1)}) undergoes saddle-node bifurcations at equilibria $E_{1*}$ and $E_{2*}$ with saddle-node bifurcation surfaces
\begin{eqnarray*}
\left.
\begin{array}{l}
\mathcal{SN}_1:=\{(s, n, h)\in\mathbb{R}_+^3: h=\frac{1}{4}\} ~~\mbox{and}~~
\mathcal{SN}_2:=\mathcal{SN}_2^0\cup\mathcal{SN}_2^+\cup\mathcal{SN}_2^-,
\end{array}
\right.
\end{eqnarray*}
where
\begin{eqnarray*}
\left.
\begin{array}{l}
\mathcal{SN}_2^0:=\{(s, n, h)\in\mathbb{R}_+^3: h=h_1, s\geq\frac{2}{3}\},\\
\mathcal{SN}_2^+:=\{(s, n, h)\in\mathbb{R}_+^3: h=h_1, n>n_1, 0<s<\frac{2}{3}\},\\
\mathcal{SN}_2^-:=\{(s, n, h)\in\mathbb{R}_+^3: h=h_1, n<n_1, 0<s<\frac{2}{3}\}.
\end{array}
\right.
\end{eqnarray*}
Moreover, (i) an unstable node $E_{11}$ and a saddle $E_{12}$ arise from a saddle-node bifurcation at $E_{1*}$ as $h$ crosses bifurcation surface $\mathcal{SN}_1$ from $h>\frac{1}{4}$ to $h<\frac{1}{4}$;
(ii) a saddle $E_{21}$ and a stable (resp., unstable) node $E_{22}$ arise from a saddle-node bifurcation at $E_{2*}$ as $h$ crosses bifurcation surface $\mathcal{SN}_2^0\cup\mathcal{SN}_2^-$ (resp., $\mathcal{SN}_2^+$) from $h>h_1$ to $h<h_1$.
\label{thm3}
\end{theorem}

\begin{proof}
Firstly, we consider boundary equilibrium $E_{1*}$.
Clearly, $y=0$ is an invariant curve of system (\ref{(1.1)}). Computing the gradient of the invariant curve, we can check that the inner product of the normal vector $(0,1)$ of the invariant curve and the center subspace spanned by the eigenvector at $E_{1*}$ for $h=\frac{1}{4}$ is $0$, which implies that the normal vector is perpendicular the center subspace. Therefore, invariant curve $y=0$ is the global center manifold of system (\ref{(1.1)}) if $h=\frac{1}{4}$.
Suspending system (\ref{(1.1)}) with variable $\epsilon:=h-\frac{1}{4}$ with $|\epsilon|$ sufficiently small  and  making the transformation $x=u+v+\frac{1}{2}$, $y=-\sqrt{2}sv$ to translate equilibrium $E_{1*}$ to the origin and normalize the linear part, we obtain
\begin{eqnarray}
\left\{
\begin{array}{l}
\frac{du}{dt}=-\epsilon-u^2+(s-2) uv-\frac{2 \sqrt{2} s^2-s n+n}{n} v^2+O(|u, v|^3),\\
\frac{dv}{dt}=s v+\frac{2 \sqrt{2} s^2}{n} v^2+O(|u, v|^3),\\
\frac{d\epsilon}{dt}=0.
\end{array}
\right.
\label{(2.11)}
\end{eqnarray}
Hence, system (\ref{(2.11)}) restricted to center manifold $v=0$ can be reduced to
\begin{eqnarray*}
\left.
\begin{array}{l}
\frac{du}{dt}=-\epsilon-u^2+O(|u|^3),
\end{array}
\right.
\end{eqnarray*}
which implies that two equilibria arise from  the saddle-node bifurcation at the origin as $\epsilon$ varies from $\epsilon>0$ to $\epsilon<0$ [Kuznetsov, 1995].
Thus,  an unstable node $E_{11}$ and a saddle $E_{12}$ arise from the saddle-node bifurcation at $E_{1*}$ as $h$ crosses bifurcation surface $\mathcal{SN}_1$ from $h>\frac{1}{4}$ to $h<\frac{1}{4}$.


Secondly, we similarly consider interior equilibrium $E_{2*}$. Let $\epsilon=h-h_1$ and consider $|\epsilon|$ sufficiently small.
Suspending system (\ref{(1.1)}) with variable $\epsilon$  and  making the transformation $x=u+\frac{z_*}{s} v+\frac{\epsilon}{n z_*-s}+z_*^2$, $y=n u+v+nz_*^2$ to translate equilibrium $E_{2*}$ to the origin and normalize the linear part, we obtain
\begin{eqnarray}
\left\{
\begin{array}{l}
\frac{du}{dt}=\frac{s }{n z_*-s}\epsilon+\frac{(3 n+8 z_*)s}{8 z_* (n z_*-s)} u^2-\frac{(9 n^2 z_*^2-8 n z_*^3-20 n s z_*+8 s^2)}{8 z_* s n (n z_*-s)} v^2-\frac{s (-8 z_*+9 n)}{8 z_* (n z_*-s)^3} \epsilon^2+\frac{(8 z_*^2+n z_*+2 s)}{4 z_* (n z_*-s)} u v\\
\phantom{\frac{du}{dt}=}
+\frac{s(n+8 z_*)}{4 z_* (n z_*-s)^2} u \epsilon-\frac{(9 n z_*-8 z_*^2-10 s)}{4 z_* (n z_*-s)^2} v\epsilon+O(|u, v|^3),\\
\frac{dv}{dt}=(n z_*-s) v-\frac{(3 n+8 z_*) n s}{8 z_* (n z_*-s)} u^2+\frac{(n^3 z_*^3-8 n^2 z_*^4+4 n^2 s z_*^2-16 n s^2 z_*+8 s^3)}{8 z_*^2 (n z_*-s) n s} v^2+\frac{s n (n z_*-8 z_*^2+8 s)}{8 (n z_*-s)^3 z_*^2} \epsilon^2\\
\phantom{\frac{du}{dt}=}
-\frac{(n z_*+8 z_*^2+2 s) n}{4 z_* (n z_*-s)} u v-\frac{(n+8 z_*) n s}{4 z_* (n z_*-s)^2} u \epsilon+\frac{(n^2 z_*^2-8 n z_*^3+6 n s z_*-8 s^2)}{4 z_*^2 (n z_*-s)^2} v\epsilon+O(|u, v|^3),\\
\frac{d\epsilon}{dt}=0.
\end{array}
\right.
\label{(2.12)}
\end{eqnarray}
System (\ref{(2.12)}) exists a $C^{\infty}$ center manifold $v=C(u,\epsilon)$ near the origin  [Carr, 1981], which is tangent to plane $v=0$ at the origin and has the form
\begin{eqnarray*}
\left.
\begin{array}{l}
C(u,\epsilon):=\frac{n s (3 n+8 z_*)}{8 (n z_*-s)^2 z_*}u^2+\frac{n^2 s (n z_*+8 z_*^2+2 s)}{4 z_* (n z_*-s)^4}u\epsilon-\frac{n s \{-8 n^2 z_*^4+n^3 z_*^3+4s(n^2-2 s)z_*^2-19 n s^2 z_*+8 s^3\}}{8 z_*^2 (n z_*-s)^6}\epsilon^2+O(|u, \epsilon|^3).
\end{array}
\right.
\end{eqnarray*}
Hence, system (\ref{(2.12)}) restricted to center manifold $v=C(u,\epsilon)$ can be reduced to
\begin{eqnarray}
\left.
\begin{array}{l}
\frac{du}{dt}=d_0(\epsilon)+d_1(\epsilon)u+d_2(\epsilon)u^2+O(|u|^3)
\end{array}
\right.
\label{(2.13)}
\end{eqnarray}
with
\begin{eqnarray*}
\left.
\begin{array}{l}
d_0(\epsilon):=\frac{s}{n z_*-s}\epsilon+O(|\epsilon|^2),~~~
d_1(\epsilon):=\frac{(n+8 z_*) s}{4 z_* (n z_*-s)^2}\epsilon+O(|\epsilon|^2),~~~
d_2(\epsilon):=\frac{(3 n+8 z_*) s}{8 z_* (n z_*-s)}+O(|\epsilon|).
\end{array}
\right.
\end{eqnarray*}
We can check either $d_2(0)>0$ if $n>n_1$ and $0<s<\frac{2}{3}$ or $d_2(0)<0$ if $0<n<n_1$ and $0<s<\frac{2}{3}$.
Applying translation $u=w-\frac{d_1(\epsilon)}{2d_2(\epsilon)}$ and time-rescaling $d\tau=d_2(\epsilon)dt$
to eliminate the term of $u$ and reduce the coefficient of the term $u_2$ to $1$, system (\ref{(2.13)}) can be rewritten as
\begin{eqnarray*}
\left.
\begin{array}{l}
\frac{dw}{d\tau}=\frac{8 z_*}{3 n+8 z_*} \epsilon+O(|\epsilon|^2)+w^2+O(|w|^3),
\end{array}
\right.
\end{eqnarray*}
which shows that two equilibria arise from  the saddle-node bifurcation at the origin as $\epsilon$ varies from $\epsilon>0$ to $\epsilon<0$ [Kuznetsov, 1995].
Therefore, a saddle $E_{21}$ and a stable (resp., unstable) node $E_{22}$ arise from the saddle-node bifurcation at $E_{2*}$ as $h$ crosses bifurcation surface $\mathcal{SN}_2^0\cup\mathcal{SN}_2^-$ (resp., $\mathcal{SN}_2^+$) from $h>h_1$ to $h<h_1$.

\end{proof}

\subsection{Bogdanov-Takens bifurcation}

Next, we discuss Bogdanov-Takens bifurcation of codimension two at cusp $E_{2*}: (\frac{2-3 s}{4}, \frac{s\sqrt{2-3s}}{2})$ by perturbing a pair of parameter $(h, n)$ near $(h_1, n_1)$ with $h_1=\frac{(3s-2)(s-2)}{16}$ for $0<s<\frac{2}{3}$.
\begin{theorem}
For $0<s<\frac{2}{3}$, system (\ref{(1.1)}) undergoes a Bogdanov-Takens bifurcation of codimension two at cusp $E_{2*}$ with four bifurcation curves in the neighborhood $U$ of $(h_1, n_1)$ for the $(h, n)$-parameter plane
\begin{eqnarray*}
\left.
\begin{array}{l}
\mathcal{SN}^+=\{(h, n)\in U: h=h_1+\frac{(2-3 s)^2}{8 \sqrt{2-3 s}}(n-n_1)+\frac{9 (2-3 s)^2}{32(3 s-4)}(n-n_1)^2+O(|n-n_1|^3), n<n_1\},\\
\mathcal{SN}^-=\{(h, n)\in U: h=h_1+\frac{(2-3 s)^2}{8 \sqrt{2-3 s}}(n-n_1)+\frac{9 (2-3 s)^2}{32(3 s-4)}(n-n_1)^2+O(|n-n_1|^3), n>n_1\},\\
\mathcal{H}=\{(h, n)\in U: h=h_1-\frac{(3 s-2)^2}{8 \sqrt{2-3 s}}(n-n_1)+\frac{(-243 s^3+504 s^2-348 s+80)}{32 (5 s-4)^2}(n-n_1)^2+O(|n-n_1|^3), n>n_1\},\\
\mathcal{HL}=\{(h, n)\in U: h=h_1-\frac{(3 s-2)^2}{8 \sqrt{2-3 s}}(n-n_1)+\frac{(9 s+4) (159 s-116) (3 s-2)^2}{800 (3 s-4) (5 s-4)^2}(n-n_1)^2+O(|n-n_1|^3), n>n_1\}.
\end{array}
\right.
\end{eqnarray*}
Moreover, system (\ref{(1.1)}) undergoes a saddle-node bifurcation near $E_{2*}$ as parameters $(h, n)$ cross bifurcation curves $\mathcal{SN}^+\cup\mathcal{SN}^-$, a Hopf bifurcation near $E_{2*}$ as parameters $(h, n)$ cross bifurcation curve $\mathcal{H}$, and a homoclinic bifurcation near $E_{2*}$ as parameters $(h, n)$ cross bifurcation curve $\mathcal{HL}$.
\label{thm4}
\end{theorem}

\begin{proof}
Let $\epsilon:=(\epsilon_1, \epsilon_2)$ with $\epsilon_1:=h-h_1$ and $\epsilon_2:=n-n_1$. First, translating equilibrium $E_{2*}$ to the origin and normalizing the linear part as $\epsilon=0$, the unfolding system of system (\ref{(1.1)}) takes the power series form
\begin{eqnarray}
\left\{
\begin{array}{l}
\frac{dx}{dt}=a_{00}+a_{10}x+a_{01}y+a_{02}y^2+O(|x, y|^3),\\
\frac{dy}{dt}=b_{00}+b_{10}x+b_{01}y+b_{20}x^2+b_{02}y^2+b_{11}xy+O(|x, y|^3),
\end{array}
\right.
\label{(2.14)}
\end{eqnarray}
where the coefficients are displayed in the Appendix.
Next, consider the near-identity transformation
\begin{eqnarray*}
\left.
\begin{array}{l}
u:=x,~~
v:=a_{00}+a_{10}x+a_{01}y+a_{02}y^2+O(|x, y|^3),
\end{array}
\right.
\end{eqnarray*}
which reduces system (\ref{(2.14)}) to the Kukles form
\begin{eqnarray}
\left\{
\begin{array}{l}
\frac{du}{dt}=v,\\
\frac{dv}{dt}=c_{00}+c_{10}u+c_{01}v+c_{20}u^2+c_{02}v^2+c_{11}uv+O(|u, v|^3),
\end{array}
\right.
\label{(2.15)}
\end{eqnarray}
where the coefficients are given in the Appendix.
Then, in order to eliminate the term of $v$ in the second equation of system (\ref{(2.15)}), we introduce the transformation
$u_1:=u+\frac{c_{01}}{c_{11}}$ and system (\ref{(2.15)}) becomes
\begin{eqnarray}
\left\{
\begin{array}{l}
\frac{du_1}{dt}=v,\\
\frac{dv}{dt}=d_{00}+d_{10}u_1+c_{20}u_1^2+c_{02}v^2+c_{11}u_1v+O(|u_1, v|^3),
\end{array}
\right.
\label{(2.16)}
\end{eqnarray}
where
\begin{eqnarray*}
\left.
\begin{array}{l}
d_{00}:=c_{00}-\frac{c_{10} c_{01}}{c_{11}}+\frac{c_{20} c_{01}^2}{c_{11}^2},~~
d_{10}:=c_{10}-\frac{2 c_{20} c_{01}}{c_{11}}.
\end{array}
\right.
\end{eqnarray*}
Further, making the transformation $v_1:=v-c_{02}u_1v$ and the time-rescaling $d\tau:=(1+c_{02}u_1)dt$ to eliminate the term of  $v^2$ in the second equation of system (\ref{(2.16)}),  we obtain
\begin{eqnarray}
\left\{
\begin{array}{l}
\frac{du_1}{d\tau}=v_1,\\
\frac{dv_1}{d\tau}=\mu_1+\mu_2u_1+e_{20}u_1^2+e_{11}u_1v_1+O(|u_1, v_1|^3),
\end{array}
\right.
\label{(2.17)}
\end{eqnarray}
where
\begin{eqnarray*}
\left.
\begin{array}{l}
\mu_1:=d_{00},~~\mu_2:=d_{10}-2 d_{00}c_{02},~~
e_{20}:=c_{20}-2d_{10}c_{02}+2d_{00}c_{02}^2,~~e_{11}:=c_{11}.
\end{array}
\right.
\end{eqnarray*}
Furthermore, applying the rescaling
$u_2:=\frac{e_{11}^2}{e_{20}}u_1$, $v_2:=-\frac{e_{11}^3}{e_{20}^2}v_1$
and $dt:=-\frac{e_{20}}{e_{11}}d\tau$ to reduce the coefficients of the terms $u_1^2$ and $u_1v_1$ to $1$ and $-1$,
the second order truncation of system (\ref{(2.17)}) can be changed into
\begin{eqnarray}
\left\{
\begin{array}{l}
\frac{du_2}{dt}=v_2,\\
\frac{dv_2}{dt}=\eta_1+\eta_2u_2+u_2^2-u_2v_2,
\end{array}
\right.
\label{(2.18)}
\end{eqnarray}
where
\begin{eqnarray}
\begin{array}{l}
\eta_1:=\frac{8 (5 s-4)^4}{(3 s-2) (3 s-4)^3 s^2} \epsilon_1+\frac{(5 s-4)^4 (3 s-2)^5 }{(2-3 s^4\sqrt{2-3 s} (3 s-4)^3 s^2}\epsilon_2+O(|\epsilon_1, \epsilon_2|^2),\\
\eta_2:=-\frac{4 (5 s-4) (51 s^2-48 s+16) }{(3 s-4)^2 (3 s-2) s^2}\epsilon_1+\frac{(5 s-4) (3 s-2)^2 (27 s^2-32 s+16)}{2 (2-3 s)\sqrt{2-3 s} (3 s-4)^2 s^2} \epsilon_2+O(|\epsilon_1, \epsilon_2|^2).
\end{array}
\label{(2.19)}
\end{eqnarray}
We can check that correspondence (\ref{(2.19)}) is invertible near $(\epsilon_1, \epsilon_2)=(0, 0)$ because of
$$
\begin{vmatrix}
\frac{\partial\eta_1}{\partial\epsilon_1}&\frac{\partial\eta_1}{\partial\epsilon_2}\\
\frac{\partial\eta_2}{\partial\epsilon_1}&\frac{\partial\eta_2}{\partial\epsilon_2}
\end{vmatrix}_{(\epsilon_1, \epsilon_2)=(0, 0)}
=\frac{32 (3 s-2)^5 (5 s-4)^5}{s^3 (2-3 s)^4\sqrt{2-3 s} (3 s-4)^5}<0
$$
for $0<s<\frac{2}{3}$, which implies that system (\ref{(2.18)}) is the universal unfolding for
the second order truncation of system (\ref{(2.10)}) in the proof of Theorem \ref{thm2} [Kuznetsov, 1995].
Therefore, system (\ref{(2.18)}) undergoes a saddle-node bifurcation as $(\eta_1,\eta_2)$ crossing $\mathcal{SN}^+\cup\mathcal{SN}^-$,
a Hopf bifurcation as $(\eta_1,\eta_2)$ crossing $\mathcal{H}$ and a homoclinic bifurcation as $(\eta_1,\eta_2)$ crossing $\mathcal{HL}$, where
\begin{eqnarray*}
\left.
\begin{array}{l}
\mathcal{SN}^+:=\{(\eta_1,\eta_2):\eta_1=\frac{1}{4}\eta_2^2, \eta_2>0\},~~
\mathcal{SN}^-:=\{(\eta_1,\eta_2):\eta_1=\frac{1}{4}\eta_2^2, \eta_2<0\},\\
\mathcal{H}:=\{(\eta_1,\eta_2):\eta_1=0, \eta_2<0\},~~
\mathcal{HL}:=\{(\eta_1,\eta_2):\eta_1=-\frac{6}{25}\eta_2^2+O(|\eta_2|^3), \eta_2<0\}.
\end{array}
\right.
\end{eqnarray*}
In the following, we present the above bifurcation curves in terms of the original parameters $h$ and $n$.
Restricted on the four bifurcation curves, we first obtain
\begin{eqnarray*}
\left.
\begin{array}{l}
\epsilon_2=\frac{(3 s-4)^2 s}{4 (2-3 s)\sqrt{2-3 s} (5 s-4)}\eta_2+O(|\eta_2|^2)
\end{array}
\right.
\end{eqnarray*}
with $\frac{(3 s-4)^2 s}{4 (2-3 s)\sqrt{2-3 s} (5 s-4)}<0$ for $0<s<\frac{2}{3}$ from (\ref{(2.19)}), which implies $\epsilon_2>0$ (resp. $<0$) as $\eta_2<0$ (resp. $>0$).
Then, for the saddle-node bifurcation curves $\mathcal{SN}^{\pm}$, we consider $\mathcal{SN}(\epsilon_1,\epsilon_2):=\eta_1-\frac{1}{4}\eta_2^2$.
Substituting (\ref{(2.19)}) into $\mathcal{SN}(\epsilon_1,\epsilon_2)$, we can check that $\mathcal{SN}(0,0)=0$ and $\frac{\partial\mathcal{SN}(0,0)}{\partial\epsilon_1}\neq0$. The implicit function theorem shows that there is a unique function
\begin{eqnarray*}
\left.
\begin{array}{l}
\epsilon_1(\epsilon_2)=\frac{(2-3 s)^2}{8 \sqrt{2-3 s}}\eta_2+\frac{9 (2-3 s)^2}{32(3 s-4)}\eta_2^2+O(|\eta_2|^3)
\end{array}
\right.
\end{eqnarray*}
such that $\epsilon_1(0)=0$ and $\mathcal{SN}(\epsilon_1(\epsilon_2),\epsilon_2)=0$ near $(\epsilon_1, \epsilon_2)=(0, 0)$.
Therefore, by translation transformations $\epsilon_1=h-h_1$ and $\epsilon_2=n-n_1$, we obtain the saddle-node bifurcation curves
\begin{eqnarray*}
\left.
\begin{array}{l}
\mathcal{SN}^+=\{(h,n)\in U: h=h_1+\frac{(2-3 s)^2}{8 \sqrt{2-3 s}}(n-n_1)+\frac{9 (2-3 s)^2}{32(3 s-4)}(n-n_1)^2+O(|n-n_1|^3), n<n_1\},\\
\mathcal{SN}^-=\{(h,n)\in U: h=h_1+\frac{(2-3 s)^2}{8 \sqrt{2-3 s}}(n-n_1)+\frac{9 (2-3 s)^2}{32(3 s-4)}(n-n_1)^2+O(|n-n_1|^3), n>n_1\}.
\end{array}
\right.
\end{eqnarray*}
Similarly, from functions $\mathcal{H}(\epsilon_1,\epsilon_2):=\eta_1$ and $\mathcal{HL}(\epsilon_1,\epsilon_2):=\eta_1+\frac{6}{25}\eta_2^2+O(|\eta_2|^3)$,
we obtain the Hopf bifurcation curve $\mathcal{H}$ and the homoclinic bifurcation curve $\mathcal{HL}$
\begin{eqnarray*}
\left.
\begin{array}{l}
\mathcal{H}=\{(h,n)\in U: h=h_1-\frac{(3 s-2)^2}{8 \sqrt{2-3 s}}(n-n_1)+\frac{(-243 s^3+504 s^2-348 s+80)}{32 (5 s-4)^2}(n-n_1)^2+O(|n-n_1|^3), n>n_1\},\\
\mathcal{HL}=\{(h,n)\in U: h=h_1-\frac{(3 s-2)^2}{8 \sqrt{2-3 s}}(n-n_1)+\frac{(9 s+4) (159 s-116) (3 s-2)^2}{800 (3 s-4) (5 s-4)^2}(n-n_1)^2+O(|n-n_1|^3), n>n_1\}.
\end{array}
\right.
\end{eqnarray*}

\end{proof}


The neighborhood $U$ of point $(n_1, h_1)$ is divided into the following four regions  by the four bifurcation curves in Theorem \ref{thm4}
\begin{eqnarray*}
\begin{array}{l}
\mathcal{R}_1:=\{(n, h)\in U: h>h_1+\frac{(2-3 s)^2}{8 \sqrt{2-3 s}}(n-n_1)+\frac{9 (2-3 s)^2}{32(3 s-4)}(n-n_1)^2+O(|n-n_1|^3) \},\\
\mathcal{R}_2:=\{(n, h)\in U:
h_1-\frac{(3 s-2)^2}{8 \sqrt{2-3 s}}(n-n_1)+\frac{(-243 s^3+504 s^2-348 s+80)}{32 (5 s-4)^2}(n-n_1)^2+O(|n-n_1|^3)<h\\
\phantom{\mathcal{R}_2:=\{(n, h)\in U:}
<h_1+\frac{(2-3 s)^2}{8 \sqrt{2-3 s}}(n-n_1)+\frac{9 (2-3 s)^2}{32(3 s-4)}(n-n_1)^2+O(|n-n_1|^3), n>n_1
\},\\
\mathcal{R}_3:=\{(n, h)\in U:
h_1-\frac{(3 s-2)^2}{8 \sqrt{2-3 s}}(n-n_1)+\frac{(9 s+4) (159 s-116) (3 s-2)^2}{800 (3 s-4) (5 s-4)^2}(n-n_1)^2+O(|n-n_1|^3)<h\\
\phantom{\mathcal{R}_2:=\{(n, h)\in U:}
<h_1-\frac{(3 s-2)^2}{8 \sqrt{2-3 s}}(n-n_1)+\frac{(-243 s^3+504 s^2-348 s+80)}{32 (5 s-4)^2}(n-n_1)^2+O(|n-n_1|^3), n>n_1
\},\\
\mathcal{R}_4:=\{(n, h)\in U:
h<h_1+\frac{(2-3 s)^2}{8 \sqrt{2-3 s}}(n-n_1)+\frac{9 (2-3 s)^2}{32(3 s-4)}(n-n_1)^2+O(|n-n_1|^3), n\leq n_1
\}\\
\phantom{\mathcal{R}_2:=}
~\cup\{(n, h)\in U:
h<h_1-\frac{(3 s-2)^2}{8 \sqrt{2-3 s}}(n-n_1)+\frac{(9 s+4) (159 s-116) (3 s-2)^2}{800 (3 s-4) (5 s-4)^2}(n-n_1)^2+O(|n-n_1|^3),  n>n_1
\}.
\end{array}
\end{eqnarray*}
The according dynamics of system (\ref{(1.1)}) near cusp $E_{2*}$ for parameters in neighborhood $U$ are listed in Table 2, which
is also demonstrated by some numerical simulations.
\begin{table}[h]
\tbl{Dynamics near equilibrium $E_{2*}$.}
{\begin{tabular}{l l l}\\[-2pt]
\toprule
 $(n,h)\in U$ & Equilibria and properties  & Closed orbits and homoclinic orbits \\[6pt]
\hline\\[-2pt]
 $\mathcal{R}_{1}$ & No equilibria  & No \\[1pt]
 $\mathcal{SN}^{\pm}$ & $E_{2*}$(saddle-node) & No \\[1pt]
 $\mathcal{R}_{2}$
& $E_{21}$(saddle) $E_{22}$(unstable focus or node)  & No \\[1pt]
 $\mathcal{H}$
& $E_{21}$(saddle) $E_{22}$(unstable weak focus) & No\\[1pt]
$\mathcal{R}_3$
&$E_{21}$(saddle)  $E_{22}$(stable focus) & An unstable limit cycle\\[1pt]
$\mathcal{HL}$
&$E_{21}$(saddle)  $E_{22}$(stable focus)  & A homoclinic orbit\\[1pt]
$\mathcal{R}_{4}$
&$E_{21}$(saddle) $E_{22}$(stable focus or node)  & No \\[1pt]
 $(n_1, h_1)$ & $E_{2*}$(cusp) & No\\[1pt]
\botrule
\end{tabular}}
\end{table}

Set parameter $s=0.1$, the bifurcation diagram of Bogdanov-Takens bifurcation in neighborhood $U$ of point $(n_1, h_1)=(0.1534, 0.2019)$ and cusp $E_{2*}=(0.425,0.0652)$ are displayed in Figure \ref{Figure1} (a) and (b), respectively.
% Figure environment removed

If $(n,h)=(0.3034, 0.17)\in\mathcal{R}_1$, system (\ref{(1.1)}) has no interior equilibrium near the cusp (Figure \ref{Figure2} (a)).

If $(n,h)=(0.3034, 0.164)\in\mathcal{R}_2$, system (\ref{(1.1)}) has the saddle $E_{21}=(0.425, 0.1019)$ and the unstable focus $E_{22}=(0.425,0.1184)$, the rise of which is due to the saddle-node bifurcation at $E_{2*}$ (Figure \ref{Figure2} (b)).

If $(n,h)=(0.3034, 0.1625)\in\mathcal{R}_3$, system (\ref{(1.1)}) has the saddle $E_{21}=(0.425,0.0966)$ and the stable focus $E_{22}=(0.425,0.1237)$ surrounded by an unstable limit cycle (Figure \ref{Figure2} (c)). The rise of the limit cycle is induced by the Hopf bifurcation.

If $(n,h)=(0.3034, 0.16)\in\mathcal{R}_4$, system (\ref{(1.1)}) has the saddle $E_{21}=(0.425,0.0907)$ and the stable focus $E_{22}=(0.425,0.1296)$ (Figure \ref{Figure2} (d)). The disappearance of the limit cycle is induced by the homoclinic bifurcation.
% Figure environment removed










\subsection{Hopf bifurcation}

Last section indicates that interior equilibrium $E_{22}$ is of center type if $(s, n, h)\in\mathcal{H}_1$, where
\begin{eqnarray*}
\left.
\begin{array}{l}
\mathcal{H}_1:=\{(s, n, h)\in\mathbb{R}_+^3: h=h_2, n>n_1, \frac{1}{2}\leq s<\frac{2}{3}\}\cup\{(s, n, h)\in\mathbb{R}_+^3: h=h_2, n_1<n<n_2, s<\frac{1}{2}\}
\end{array}
\right.
\end{eqnarray*}
with $n_1$, $n_2$ and $h_2$ given in Table 1.
This section further addresses that equilibrium $E_{22}$ is a weak focus and determines its final order, which implies that system (\ref{(1.1)}) may undergo Hopf bifurcation near equilibrium $E_{22}$.
Let
\begin{eqnarray}
\left.
\begin{array}{l}
l_1:=64 z_0^4-64 n z_0^3+(n^2+96 s) z_0^2-6 n s z_0+8 s^2,\\
l_2:=413696 n^2 z_0^{10}-1024 n (505 n^2+536 s) z_0^9+(116352 n^4+462848 n^2 s+32768 s^2) z_0^8\\
\phantom{l_2:=}
-16 n (505 n^4-36680 n^2 s-34304 s^2) z_0^7+(101 n^6-146656 n^4 s-2257920 n^2 s^2-622592 s^3) z_0^6\\
\phantom{l_2:=}
+16 n s (151 n^4+22212 n^2 s+127616 s^2) z_0^5-32 s^2 (591 n^4-7984 n^2 s+17024 s^2) z_0^4\\
\phantom{l_2:=}
+24 n s^3 (979 n^2-43264 s) z_0^3+32 s^4 (1685 n^2+17856 s) z_0^2-123136 n s^5 z_0+61440 s^6.
\end{array}
\right.
\label{(2.22)}
\end{eqnarray}
Regarding the Hopf bifurcation of system (\ref{(1.1)}) at equilibrium $E_{22}$ we have the following result.
\begin{theorem}
Equilibrium $E_{22}$ is a weak focus of order at most three if $(s, n, h)\in\mathcal{H}_1$. More exactly,
equilibrium $E_{22}$ is a weak focus of order one if $(s, n, h)\in\mathcal{H}_{11}$,
a weak focus of order two if $(s, n, h)\in\mathcal{H}_{12}$,
a weak focus of order three if $(s, n, h)\in\mathcal{H}_{13}$, where
\begin{eqnarray*}
\left.
\begin{array}{l}
\mathcal{H}_{11}:=\{(s, n, h)\in\mathcal{H}_1: l_1\neq0\},\\
\mathcal{H}_{12}:=\{(s, n, h)\in\mathcal{H}_1: l_1=0, l_2\neq0\},\\
\mathcal{H}_{13}:=\{(s, n, h)\in\mathcal{H}_1: l_1=l_2=0\}.
\end{array}
\right.
\end{eqnarray*}
Furthermore, system (\ref{(1.1)}) may undergo Hopf bifurcation of codimension up to three.
\label{thm5}
\end{theorem}

\begin{proof}
Making the transformation
\begin{eqnarray*}
\left.
\begin{array}{l}
x= \frac{u}{s n}+\frac{v}{n \sqrt{s (n z_0-s)}}+z_0^2,~~~
y= \frac{v}{\sqrt{s (n z_0-s)}}+n z_0^2
\end{array}
\right.
\end{eqnarray*}
and the time-rescaling $d\tau=\sqrt{s(nz_0-s)}dt$ to translate equilibrium $E_{22}$ to the origin and normalize the linear part of system (\ref{(1.1)}), we obtain
\begin{eqnarray}
\left\{
\begin{array}{l}
\frac{d\tilde{u}}{d\tau}=-v+\tilde{a}_{20} u^2+\tilde{a}_{02} v^2+\tilde{a}_{11} u v+\tilde{a}_{30} u^3+\tilde{a}_{03} v^3+\tilde{a}_{21} u^2 v+\tilde{a}_{12} u v^2+\tilde{a}_{04} v^4+\tilde{a}_{40} u^4\\
\phantom{\frac{d\tilde{u}}{d\tau}=}
+\tilde{a}_{13} u v^3+\tilde{a}_{31} u^3 v+\tilde{a}_{22} u^2 v^2+\tilde{a}_{50} u^5+\tilde{a}_{05} v^5+\tilde{a}_{14} u v^4+\tilde{a}_{23} u^2 v^3+\tilde{a}_{41} u^4 v+\tilde{a}_{32} u^3 v^2\\
\phantom{\frac{d\tilde{u}}{d\tau}=}
+\tilde{a}_{60} u^6+\tilde{a}_{06} v^6+\tilde{a}_{33} u^3 v^3+\tilde{a}_{15} u v^5+\tilde{a}_{42} u^4 v^2+\tilde{a}_{24} u^2 v^4+\tilde{a}_{51} u^5 v+\tilde{a}_{70} u^7+\tilde{a}_{07} v^7\\
\phantom{\frac{d\tilde{u}}{d\tau}=}
+\tilde{a}_{16} u v^6+\tilde{a}_{61} u^6 v+\tilde{a}_{43} u^4 v^3+\tilde{a}_{52} u^5 v^2+\tilde{a}_{25} u^2 v^5+\tilde{a}_{34} u^3 v^4+O(|u, v|^8),\\
\frac{d\tilde{v}}{d\tau}=u+\tilde{b}_{20} u^2+\tilde{b}_{30} u^3+\tilde{b}_{21} u^2 v+\tilde{b}_{40} u^4+\tilde{b}_{31} u^3 v+\tilde{b}_{22} u^2 v^2
+\tilde{b}_{50} u^5+\tilde{b}_{23} u^2 v^3+\tilde{b}_{32} u^3 v^2\\
\phantom{\frac{d\tilde{u}}{d\tau}=}
+\tilde{b}_{41} u^4 v+\tilde{b}_{60} u^6+\tilde{b}_{51} u^5 v+\tilde{b}_{42} u^4 v^2+\tilde{b}_{33} u^3 v^3+\tilde{b}_{24} u^2 v^4
+\tilde{b}_{70} u^7+\tilde{b}_{52} u^5 v^2+\tilde{b}_{43} u^4 v^3\\
\phantom{\frac{d\tilde{u}}{d\tau}=}
+\tilde{b}_{61} u^6 v+\tilde{b}_{34} u^3 v^4+\tilde{b}_{25} u^2 v^5+O(|u, v|^8),
\end{array}
\right.
\label{(2.20)}
\end{eqnarray}
where the coefficients are listed in the Appendix.
By the method of successive function  [Zhang {\it et al}., 1992], the first three focal values are given by
\begin{eqnarray}
\left.
\begin{array}{l}
L_1:=\frac{l_1}{256 s^2 n  z_0^3 (n z_0-s)^2 \sqrt{s (n z_0-s)}},\\
L_2:=\frac{-l_2}{1572864 s^5 n^3 z_0^7 (n z_0-s)^5 \sqrt{s (n z_0-s)}},\\
L_3:=\frac{l_3}{1159641169920 s^8 n^5z_0^{11} (n z_0-s)^8 \sqrt{s (n z_0-s)}},
\end{array}
\right.
\label{(2.21)}
\end{eqnarray}
where $l_1$ and $l_2$ are given in (\ref{(2.22)}), and $l_3$ is listed in the Appendix.

Note that $z_0$ is the unique positive zero of $l_0=0$, where
\begin{eqnarray*}
\left.
\begin{array}{l}
l_0:=-2z_0^2-\frac{n}{2}z_0-s+1.
\end{array}
\right.
\end{eqnarray*}
Thus, we discuss the common zeros of $l_1$, $l_2$ and $l_3$ together with $l_0$ rather than substituting the expression of  $z_0$ into
$l_1$, $l_2$ and $l_3$.
In the following, we first show that $l_1$, $l_2$ and $l_3$ have no common zeros for $(s, n, h)\in\mathcal{H}_1$ by proving that the algebraic variety $V(l_0, l_1, l_2, l_3)$ is empty in $\mathcal{H}_2:=\{(s, n, h, z_0)\in\mathcal{H}_1\times\mathbb{R}_+\}$, i.e.,
\begin{eqnarray*}
\left.
\begin{array}{l}
V(l_0, l_1, l_2, l_3)\cap\mathcal{H}_2=\{(s, n, h, z_0)\in\mathcal{H}_1\times\mathbb{R}_+: l_0=l_1=l_2=l_3=0\}=\emptyset,
\end{array}
\right.
\end{eqnarray*}
which implies that equilibrium $E_{22}$ is a weak focus of order at most three.
Compute the following Sylvester resultants by eliminating variable $z_0$
\begin{eqnarray*}
\begin{array}{l}
r_{11}:=\mbox{res}(l_0, l_1, z_0)\\
\phantom{r_{11}~}
=84 (3 s-1) (2 s-1) n^4+(12384 s^3-35616 s^2+34176 s-9984) n^2+1024 (3 s^2-2 s-2)^2,
\end{array}
\end{eqnarray*}
$r_{12}:=\mbox{res}(l_0, l_2, z_0)$  is given in the Appendix and the long expression of $r_{13}:=\mbox{res}(l_0, l_3, z_0)$ is omitted.
Obviously, $\mbox{lcoeff}(l_0, z_0)=-2\neq0$.
By Lemma 2 in  [Chen \& Zhang, 2009], we decompose the algebraic variety
\begin{eqnarray*}
\begin{array}{l}
V(l_0, l_1, l_2, l_3)=V(l_0, l_1, l_2, l_3, \mbox{lcoeff}(l_0, z_0))\cup V(\frac{l_0, l_1, l_2, l_3, r_{11}, r_{12}, r_{13}}{\mbox{lcoeff}(l_0, z_0)}).
\end{array}
\end{eqnarray*}
It therefore follows that
\begin{eqnarray*}
\begin{array}{l}
V(l_0, l_1, l_2, l_3)\cap\mathcal{H}_2=V(l_0, l_1, l_2, l_3, r_{11}, r_{12}, r_{13})\cap\mathcal{H}_2.
\end{array}
\end{eqnarray*}
Furthermore, compute the resultants by eliminating variable $n$
\begin{eqnarray*}
\begin{array}{l}
r_{21}:=\mbox{res}(r_{11}, r_{12}, n)\\
\phantom{r_{21}~}
=12446951863073631081856209256519876896173850624 s^4 (3 s-4)^4 (5 s-4)^8 (2 s-1)^2 r_1^2 r_2^2,\\
r_{22}:=\mbox{res}(r_{11}, r_{13}, n)\\
\phantom{r_{11}~}
=49923799453875410132818775116670519302868727877966921572734506879162767114240000 s^4 \\
\phantom{r_{11}~=}
\times (3 s-4)^6 (5 s-4)^8(2 s-1)^2 r_3^2 r_4^2,
\end{array}
\end{eqnarray*}
where
\begin{eqnarray*}
\begin{array}{l}
r_1:=915 s^5-7652 s^4+5980 s^3+53328 s^2-44996 s+8752,\\
r_2:=115491402588240 s^{22}-1953488427186537 s^{21}+15359878066659345 s^{20}-74470860636578055 s^{19}\\
\phantom{r_2:=}
+248840112199221720 s^{18}-605983660243125516 s^{17}+1106868061627144938 s^{16}\\
\phantom{r_2:=}
-1531784149311281514 s^{15}+1590601340020498464 s^{14}-1184755455901388517 s^{13}\\
\phantom{r_2:=}
+538618227994707893 s^{12}-13017113263954487 s^{11}-200469957259703824 s^{10}\\
\phantom{r_2:=}
+171635374856929418 s^9-73692083765786712 s^8+11751894622236952 s^7+5256302292625968 s^6\\
\phantom{r_2:=}
-3821292131749824 s^5+974288712572096 s^4-43731581922784 s^3-37910639252864 s^2\\
\phantom{r_2:=}
+9486073843200 s-754710784000,\\

r_3:=1205472812175 s^{12}-13715477533560 s^{11}+30486977408588 s^{10}+88571926756408 s^9\\
\phantom{r_2:=}
-314952100862532 s^8+141285865453424 s^7+477508614784096 s^6-759257933165280 s^5\\
\phantom{r_2:=}
+473179845238048 s^4-140286212879104 s^3+16911656927040 s^2+58099116544 s-98336631808,
\end{array}
\end{eqnarray*}
and $r_4$ is listed in the Appendix.
Clearly, $\mbox{lcoeff}(r_{11}, n)=84 (3 s-1) (2 s-1)=0$ if either $s=\frac{1}{3}$ or $s=\frac{1}{2}$, otherwise $\mbox{lcoeff}(r_{11}, n)\neq0$.
By Lemma 2 in [Chen \& Zhang, 2009] again, we have the decomposition
\begin{eqnarray*}
\begin{array}{l}
V(r_{11}, r_{12}, r_{13})=V(r_{11}, r_{12}, r_{13}, \mbox{lcoeff}(r_{11}, n))\cup V(\frac{r_{11}, r_{12}, r_{13}, r_{21}, r_{22},}{\mbox{lcoeff}(r_{11}, n)}).
\end{array}
\end{eqnarray*}
It follows from the above that
\begin{eqnarray*}
\begin{array}{l}
V(r_{11}, r_{12}, r_{13})\cap\mathcal{H}_2=V_1\cup V_2 \cup V_3,
\end{array}
\end{eqnarray*}
where
\begin{eqnarray*}
\begin{array}{l}
V_1:=V(r_{11}, r_{12}, r_{13}, 3 s-1)\cap\mathcal{H}_2,\\
V_2:=V(r_{11}, r_{12}, r_{13}, 2 s-1)\cap\mathcal{H}_2,\\
V_3:=V(\frac{r_{11}, r_{12}, r_{13}, r_{21}, r_{22}}{\mbox{lcoeff}(r_{11}, n)})\cap\mathcal{H}_2.
\end{array}
\end{eqnarray*}
We claim that $V_1=\emptyset$ because $n=\frac{2\sqrt{6}}{3}$ from $r_{11}=0$ if $s=\frac{1}{3}$, but both $r_{12}$ and $r_{13}$ are non-vanished. Similarly, we also obtain that $V_2=\emptyset$ since $n=\frac{12\sqrt{7}}{7}$ from $r_{11}=0$ if $s=\frac{1}{2}$, but $r_{12}$ and $r_{13}$ are still nonzero.
Further, we can  see that
\begin{eqnarray*}
\begin{array}{l}
V_3\subseteq V(\frac{r_{21}, r_{22}}{\mbox{lcoeff}(r_{11}, n)})\cap\mathcal{H}_2=V(r_1r_2, r_3r_4, r_{31})\cap\mathcal{H}_2=\emptyset,
\end{array}
\end{eqnarray*}
where $r_{31}:=\mbox{res}(r_1r_2, r_3r_4, s)$
is a nonzero constant.
Therefore, we obtain $V(r_{11}, r_{12}, r_{13})\cap\mathcal{H}_2=\emptyset$. Then, we conclude that
\begin{eqnarray*}
\begin{array}{l}
V(l_0, l_1, l_2, l_3)\cap\mathcal{H}_2\subseteq V(r_{11}, r_{12}, r_{13})\cap\mathcal{H}_2=\emptyset,
\end{array}
\end{eqnarray*}
which implies that equilibrium $E_{22}$ is a weak focus of order at most three.


Next, we show that $l_1$ and $l_2$ have common zeros for $(s, n, h)\in\mathcal{H}_1$ by proving that algebraic variety $V(l_0, l_1, l_2)$ is nonempty
in $\mathcal{H}_2:=\{(s, n, h, z_0)\in\mathcal{H}_1\times\mathbb{R}_+\}$, i.e.,
\begin{eqnarray*}
\left.
\begin{array}{l}
V(l_0, l_1, l_2)\cap\mathcal{H}_2=\{(s, n, h, z_0)\in\mathcal{H}_1\times\mathbb{R}_+: l_0=l_1=l_2=0\}\neq\emptyset,
\end{array}
\right.
\end{eqnarray*}
which implies that the order of weak focus $E_{22}$ is up to three.
From the above discussion, we have
\begin{eqnarray*}
\begin{array}{l}
V(l_0, l_1, l_2)\cap\mathcal{H}_2=V_4\cup V_5
\end{array}
\end{eqnarray*}
with
\begin{eqnarray*}
\begin{array}{l}
V_4:=V(l_0, l_1, l_2, r_{11}, r_{12}, r_1)\cap\mathcal{H}_2~~\mbox{and}~~
V_5:=V(l_0, l_1, l_2, r_{11}, r_{12}, r_2)\cap\mathcal{H}_2.
\end{array}
\end{eqnarray*}
The following discussion shows that $V_4\neq\emptyset$ and $V_5=\emptyset$.
Compute the resultants by eliminating variable $s$
\begin{eqnarray*}
\begin{array}{l}
\mbox{res}(r_1, r_{11}, s)=87824507904r_5r_6 ~~\mbox{and}~~
\mbox{res}(r_1, r_{12}, s)=-84934656r_5r_7,
\end{array}
\end{eqnarray*}
where
\begin{eqnarray*}
\begin{array}{l}
r_5:=19215 n^{10}-23702304 n^8-1196507136 n^6+90449117184 n^4+379567734784 n^2-420906795008,\\
r_6:=357110775 n^{10}-59055128160 n^8-273611724288 n^6+260968168488960 n^4+899787019190272 n^2\\
\phantom{r_2:=}
-3520670946295808,
\end{array}
\end{eqnarray*}
and the long expression of  polynomial $r_7$ is omitted.
Calculation shows that $\mbox{res}(r_6, r_7, n)$ is a nonzero constant, which implies that whether $r_1$, $r_{11}$ and $r_{12}$ have common zeros only depends on whether $r_5$ has zeros in $\mathcal{H}_2$.
We can check that $r_1$ and $r_5$ have isolated real zeros $s\doteq0.5117$ and $n\doteq6.6770$ in $\mathcal{H}_2$ such that all $r_1$, $r_{11}$ and $r_{12}$ vanish, which together with $z_0\doteq0.1353$ such that $l_0$, $l_1$ and $l_2$ are all vanished. Therefore, we obtain that
$V_4\neq\emptyset$.
Direct calculation shows that $r_2$ has no zero in $\mathcal{H}_2$ implying that $V_5=\emptyset$.
Hence, these analyses reveal that
\begin{eqnarray*}
\left.
\begin{array}{l}
V(l_0, l_1, l_2)\cap\mathcal{H}_2\neq\emptyset,
\end{array}
\right.
\end{eqnarray*}
which implies that the order of weak focus $E_{22}$ is up to three.

\end{proof}


We offer an example to exhibit the dynamics of system (\ref{(1.1)}) induced by the Hopf bifurcation.
Let $s=0.5$,  $n=4.5856$ and $h=0.0037$, system (\ref{(1.1)}) has two limit cycles around the unstable focus $E_{22}=(0.0351, 0.1611)$ (Figure \ref{Figure3}).

In order to determine their positions in the phase plane,  we plot three orbits starting from initial values $P_1(0.0351, 0.175)$,
$P_2(0.0351, 0.2)$ and $P_3(0.0351, 0.25)$ separately in Figure \ref{Figure3} (a) and (b).
The three orbits form two annular regions since they spiral inward and outward alternately. Then the Poincar\'e-Bendixson Theorem  [Zhang {\it et al}., 1992] indicates that there exist closed orbits in the annular regions.
Concretely, the orbit from initial value $P_3$ spirals outward as $t\rightarrow+\infty$ (Figure \ref{Figure3} (b)),
but it is blurred to distinguish the positive directions of  orbits starting from $P_1$ and $P_2$ in Figure \ref{Figure3} (a) and (b).
Zooming in the orbits near $P_1$ and $P_2$ in Figure \ref{Figure3} (c) and (d),
we obtain that the orbit from $P_1$ spirals outward while the orbit from $P_2$ spirals inward as $t\rightarrow+\infty$.
Therefore, there are a stable limit cycle lying in the annular
region bounded by the two orbits starting from $P_1$ and $P_2$ as well as an unstable limit cycle
lying in the annular region bounded by the two orbits starting from $P_2$ and $P_3$.
% Figure environment removed





\begin{remark}
In spite that both the saddle-node bifurcation at $E_{2*}$ and the Hopf bifurcation at $E_{22}$ are discussed separately, they  are actually exhibited in the Bogdanov-Takens bifurcation as well.
Accordingly, the saddle-node bifurcation curves $\mathcal{SN}^{\pm}$ and the Hopf bifurcation curve $\mathcal{H}$ are approximative expressions of $\mathcal{SN}_2^{\mp}$ and $\mathcal{H}_1$ in the above theorems.
\label{rem2}
\end{remark}













\section{Discussion}
In this paper, we dealt with the Leslie-Gower type predator-prey system (\ref{(1.1)}) with the presence of both herd behavior and constant harvesting in prey for the goal of revealing the influence of prey herd behavior and prey harvesting on the evolution of both prey and predator species.
System (\ref{(1.1)}) shows rich and varied dynamic behaviors as it can be seen that herd behavior and harvesting have a composite effect on the dynamic behaviors of the system including  the existence and qualitative properties of equilibria as well as different kinds of bifurcation such as the saddle-node bifurcations, the Bogdanov-Takens bifurcation of codimension two and the degenerate Hopf bifurcation of codimension three.


Biologically, we determine some critical thresholds for the harvesting of species and give conditions assuring the preservation of species.
The maximum sustainable yield (MSY) of harvesting is the maximum harvesting compatible with survival and also the threshold in the dynamics. If the harvesting of species exceeds the MSY (i.e., the species is over-exploited), then the species will go extinct, which means that harvesting with no limitation will cause extinction of species.
We obtain the MSY of system (\ref{(1.1)}) $h=\frac{1}{4}$. Despite the predator species is not affected directly by the harvesting activity,
both the prey and predator species become extinct if $h>\frac{1}{4}$, the reason for which is that
the prey species is reduced due to the harvesting and the predator species is reduced indirectly due to the availability of prey to the predators.
There are another two thresholds of coexistence $h_1$ and $h_2$ to control the appearance of interior equilibria and limit cycles, respectively.
Concretely, the interior equilibria appear due to the saddle-node bifurcation and the species will coexist in the form of a stable interior equilibrium if $h<h_1$.
The interior equilibrium loses the stability and a stable limit cycle appears due to the Hopf bifurcation resulting in the species coexist at the periodic oscillation if $h>h_2$.
Therefore, over-exploitation of prey species will lead to the extinction of both species, and only moderate harvesting can ensure the persistence of species and long-term economic benefits for humans.

Mathematically, the early researches have shown that even Leslie-Gower type predator-prey system (\ref{(0.2)}) with only prey herd behavior still has more complex dynamics than Leslie-Gower predator-prey system (\ref{(0.1)}) because the former undergoes a  supercritical Hopf bifurcation around the unique interior equilibrium while the unique interior equilibrium of the latter is always globally asymptotically stable [Korobeinikov,
2001; He \& Li, 2023],  which implies that the prey herd behavior can contribute to the coexistence of prey and predator species.
Our systematic work on system (\ref{(1.1)}) reveals that system (\ref{(1.1)}) is more interesting and richer in dynamics compared to system (\ref{(0.2)}) with the absence of prey constant harvesting because system (\ref{(1.1)}) exhibits more equilibria such as three kinds of non-hyperbolic equilibrium as well as more complex bifurcations such as the saddle-node bifurcations, the Bogdanov-Takens bifurcation of codimension two and the degenerate Hopf bifurcation of codimension three, which implies that
the study of predator-prey system with harvesting is more involved than that of the mere predator-prey system and
the harvesting alters the dynamical behaviors of the model significantly.
It has been shown that the elements of both herd behavior and constant harvesting in prey can not only have a strong influence on the dynamic evolution of both prey and predator species but also promote ecological diversity.
Actually, there is still many work to do in this area.
For instance, it would be interesting to explore how the constant harvesting in predators or the nonlinear harvesting impact the dynamics of
system (\ref{(0.2)}) and what is the difference between these systems with difference harvesting, which will be our future work in this area.



\nonumsection{Acknowledgments}
This work is supported by the National Natural Science Foundation of China (Grant Nos. 12101470) and
the Science Foundation of Wuhan Institute of Technology (Grant No. K2021077).


\begin{thebibliography}{9}

\bibitem[Ajraldi {\it et al.}, 2011]{Ajraldi} Ajraldi, V., Pittavino,  M. \& Venturino, E. [2011] ``Modeling herd behavior in population systems,'' {\it Nonlinear Anal.-Real} {\bf 12}, 2319-2338.

\bibitem[Braza, 2003]{Braza03} Braza, P. A. [2003] ``The bifurcation structure of the Holling-Tanner model for predator-prey interactions using two-timing,'' {\it SIAM J. Appl. Math.} {\bf 63}, 889-904.

\bibitem[Braza, 2012]{Braza12} Braza, P. A. [2012] ``Predator-prey dynamics with square root functional responses,'' {\it Nonlinear Anal.-Real} {\bf 13}, 1837-1843.

\bibitem[Carr(1981)]{Carr} Carr, J. [1981] {\it Applications of Center Manifold Theory} (Springer, NY).

\bibitem[Chen \& Zhang, 2009]{Chen} Chen, X. W. \& Zhang, W. N. [2009] ``Decomposition of algebraic sets and applications to weak centers of cubic systems,'' {\it J. Comput. Appl. Math.} {\bf 232}, 565-581.

\bibitem[Dai \& Zhao, 2018]{Dai21} Dai, Y. F. \&  Zhao, Y. L. [2018] ``Hopf cyclicity and global dynamics for a predator-prey system of Leslie type with simplified Holling type IV functional response,'' {\it Int. J. Bifurcat. Chaos} {\bf 28}, 1850166.

\bibitem[Dai {\it et al.}, 2019]{Dai19} Dai, Y. F., Zhao, Y. L.  \& Sang, B. [2019] ``Four limit cycles in a predator-prey system of Leslie type with generalized Holling type III functional response,'' {\it Nonlinear Anal.-Real} {\bf 50}, 218-239.

\bibitem[Etoua \& Rousseau, 2010]{Etoua10} Etoua, R. M. \& Rousseau, C. [2010] ``Bifurcation analysis of a generalized Gause model with prey harvesting and a generalized Holling response function of type III,''  {\it J. Differ. Equations} {\bf 249}, 2316-2356.

\bibitem[Gause(1934)]{Gause}Gause, G. F. [1934] {\it The Struggle for Existence} (Dover).

\bibitem[Gong \& Huang, 2014]{GongHuang14} Gong, Y. J. \&  Huang, J. C. [2014] ``Bogdanov-Takens bifurcation in a Leslie-Gower predator-prey model with prey harvesting,'' {\it Acta Math. Appl. Sin.-E.} {\bf 30}, 239-244.

\bibitem[Gonz\'alez-Olivares {\it et al.}, 2022]{Olivares22-1} Gonz\'alez-Olivares, E., Mosquera-Aguilar, A. \& Tintinago-Ruiz, P. [2022] ``Bifurcations in a Leslie-Gower type predator-prey model with a rational non-monotonic functional response,'' {\it Math. Model. Anal.} {\bf 27}, 510-532.

\bibitem[Gonz\'alez-Olivares {\it et al.}, 2022]{Olivares22-2} Gonz\'alez-Olivares, E., Rivera-Estay, V., Rojas-Palma, A. \& Vilches-Ponce, K. [2022] ``A Leslie-Gower type predator-prey model considering herd behavior,'' {\it Ric. Mat.}, 1-24.


\bibitem[Hacini {\it et al.}, 2021]{Hacini21}Hacini, M. E. M., Hammoudi, D., Djilali, S. \& Bentout, S. [2021] ``Optimal harvesting and stability of a predator-prey model for fish populations with schooling behavior,'' {\it Theor. Biosci.} {\bf 140}, 225-239.

\bibitem[He \& Li, 2023]{He} He, M. X. \& Li, Z. [2023] ``Global dynamics of a Leslie-Gower predator-prey model with square root response function,'' {\it Appl. Math. Lett.} {\bf 140}, 108561.

\bibitem[Hsu \& Hwang, 1995]{HsuHwang} Hsu, S. B. \&  Hwang, T. W. [1995] ``Global stability for a class of predator-prey systems,''  {\it SIAM J. Appl. Math.} {\bf 55}, 763-783.

\bibitem[Huang {\it et al.}, 2013]{Huang13} Huang, J. C., Gong, Y. J. \& Chen, J. [2013] ``Multiple bifurcations in a predator-prey system of Holling and Leslie type with constant-yield prey harvesting,'' {\it Int. J. Bifurcat. Chaos} {\bf 23}, 1350164.

\bibitem[Huang {\it et al.}, 2008]{Huang14} Huang, J. C., Ruan, S. G. \&  Song. J. [2014] ``Bifurcations in a predator-prey system of Leslie type with generalized Holling type III functional response,'' {\it J. Differ. Equations} {\bf 257}, 1721-1752.

\bibitem[Huang {\it et al.}, 2016]{Huang16} Huang, J. C, Xia, X. J., Zhang, X. A. \&  Ruan, S. G. [2016] ``Bifurcation of codimension 3 in a predator-prey system of Leslie type with simplified Holling type IV functional response,'' {\it Int. J. Bifurcat. Chaos} {\bf 26}, 1650034.

\bibitem[Korobeinikov, 2001]{Korobeinikov} Korobeinikov, A. [2001] ``A Lyapunov function for Leslie-Gower predator-prey models,'' {\it Appl. Math. Lett.} {\bf 14}, 697-699.

\bibitem[Kumar \& Kharbanda, 2019]{Kumar19} Kumar, S. \&  Kharbanda, H. [2019] ``Chaotic behavior of predator-prey model with group defense and non-linear harvesting in prey,'' {\it Chaos Soliton. Fract.} {\bf 119 }, 19-28.

\bibitem[Kuznetsov(1995)]{Kuznetsov}Kuznetsov, Y. A. [1995] {\it Elements of Applied Bifurcation Theory} (Springer, NY).

\bibitem[Lan \& Zhu, 2011]{LanZhu11} Lan, K. Q. \&  Zhu, C. R. [2011] ``Phase portraits of predator-prey systems with harvesting rates,'' {\it Discrete Cont. Dyn. B} {\bf 32 }, 901-933.

\bibitem[Leslie, 1948]{Leslie48} Leslie, P. H. [1948] ``Some further notes on the use of matrices in population mathematics,''  {\it Biometrika} {\bf 35}, 213-245.

\bibitem[Leslie \& Gower, 1960]{Leslie60} Leslie, P. H.  \&  Gower, J. C. [1960] ``The properties of a stochastic model for the predator-prey type of interaction between two species,'' {\it Biometrika} {\bf 47}, 219-234.

\bibitem[Li \& Xiao, 2007]{LiXiao} Li, Y. L. \&  Xiao, D. M. [2007] ``Bifurcations of a predator-prey system of Holling and Leslie types,'' {\it Chaos Soliton. Fract.} {\bf 34}, 606-620.

\bibitem[Luo \& Zhao, 2017]{Luo17} Luo, J. F. \& Zhao, Y. [2017] ``Stability and bifurcation analysis in a predator-prey system with constant harvesting and prey group defense,'' {\it Int. J. Bifurcat. Chaos} {\bf 27}, 1750179.

\bibitem[Mortuja {\it et al.}, 2021]{Mortuja21} Mortuja, M. G., Chaube, M. K. \& Kumar, S. [2021] ``Dynamic analysis of a predator-prey system with nonlinear prey harvesting and square root functional response,'' {\it Chaos Soliton. Fract.} {\bf 148}, 111071.

\bibitem[Pal {\it et al.}, 2016]{Pal} Pal, D., Santra, P.  \& Mahapatra, G. S. [2016] ``Predator-prey dynamical behavior and stability analysis with square root functional response,''  {\it Int. J. Appl. Comput. Math.}  {\bf 3}, 1833-1845.

\bibitem[Peng {\it et al.}, 2009]{Peng09} Peng, G. J., Jiang, Y. L. \&  Li, C. P. [2009] ``Bifurcations of a Holling-type II predator-prey system with constant rate harvesting,'' {\it Int. J. Bifurcat. Chaos} {\bf 19}, 2499-2514.

\bibitem[Rosenzweig, 1971]{Rosenzweig}  Rosenzweig, M. L. [1971] ``Paradox of enrichment: destabilization of exploitation ecosystem in ecological time,''  {\it Science} {\bf 171}, 385-387.

\bibitem[Venturino \& Petrovskii, 2013]{Venturino} Venturino, E. \& Petrovskii, S. [2013] ``Spatiotemporal behavior of a prey-predator system with a group defense for prey,''  {\it Ecol. Complex.} {\bf 14}, 37-47.

\bibitem[Vilches {\it et al.}, 2018]{Vilches} Vilches, K., Gonz\'alez-Olivares, E. \&  Rojas-Palma, A. [2018] ``Prey herd behavior modeled by a generic non-differentiable functional response,''  {\it Math. Model. Nat. Pheno.} {\bf 13}, 26.

\bibitem[Xiao \& Jennings, 2005]{Xiao05} Xiao, D. M. \& Jennings, L. S. [2005] ``Bifurcations of a ratio-dependent predator-prey system with constant rate harvesting,'' {\it SIAM J. Appl. Math.}  {\bf 65}, 737-753.

\bibitem[Xu {\it et al.}, 2016]{Xu} Xu, C. Q., Yuan, S. L. \& Zhang, T. H. [2016] ``Global dynamics of a predator-prey model with defense mechanism for prey,'' {\it Appl. Math. Lett.} {\bf 62}, 42-48.

\bibitem[Zhang {\it et al.}(1992)]{Zhang92}Zhang, Z. F., Ding, T. R., Huang, W. Z. \& Dong, Z. X. [1992] {\it Qualitative Theory of Differential Equations} (Amer. Math. Soc., Providence).

\bibitem[Zhang \& Su, 2021]{ZhangSu} Zhang, J. \& Su, J. [2021] ``Bifurcations in a predator-prey model of Leslie-type with simplified Holling type IV functional response'' {\it Int. J. Bifurcat. Chaos} {\bf 31},  2150054.

\bibitem[Zhu \& Lan, 2010]{ZhuLan10} Zhu, C. R. \& Lan, K. Q. [2010] ``Phase portraits, Hopf bifurcation and limit cycles of Leslie-Gower predator-prey systems with harvesting rates,'' {\it Discrete Contin. Dynam. Syst. Ser. B} {\bf 14}, 289-306.


\end{thebibliography}




\nonumsection{Appendix}
\noindent
Coefficients $a_{ij}$ and $b_{ij}$ in system (\ref{(2.14)}) are listed below
\begin{eqnarray*}
\left.
\begin{array}{l}
a_{00}:=\frac{(2-3 s) \sqrt{2-3 s} \epsilon_2}{4 (\epsilon_2 \sqrt{2-3 s}+2 s)},~~
a_{10}:=\frac{\epsilon_2 (2-3 s)}{2 s (\epsilon_2 \sqrt{2-3 s}+2 s)},~~
a_{01}:=\frac{ 2 s}{\epsilon_2 \sqrt{2-3 s}+2 s},~~
a_{02}:=\frac{8 s}{(3 s-2) (\epsilon_2 \sqrt{2-3 s}+2 s)},\\
b_{00}:=\frac{\{(3 s^2-2 s-4 \epsilon_1)\sqrt{2-3 s} \epsilon_2 -8 s \epsilon_1\} \sqrt{2-3 s}}{8 s^2 (\epsilon_2 \sqrt{2-3 s}+2 s)},~~
b_{10}:=\frac{(3 s-2) \sqrt{2-3 s}\epsilon_2}{4 s^2 (\epsilon_2 \sqrt{2-3 s}+2 s)},~~
b_{01}:=\frac{(2-3 s)\epsilon_2}{2 s (\epsilon_2 \sqrt{2-3 s}+2 s)},\\
b_{20}:=\frac{(3 s-4) (2-3 s)}{16 \sqrt{2-3 s} s^4},~~
b_{02}:=\frac{2 s \sqrt{2-3 s} (15 s-4)-(7 s-4) (3 s-2)\epsilon_2}{4 (\epsilon_2 \sqrt{2-3 s}+2 s) (2-3 s)s^2},~~
b_{11}:=\frac{5 s-4}{4 s^3}.
\end{array}
\right.
\end{eqnarray*}
Coefficients $c_{ij}$ in system (\ref{(2.15)}) are listed below
\begin{eqnarray*}
\left.
\begin{array}{l}
c_{00}:=-\frac{1}{a_{01}^9}(2 a_{00}^6 a_{02}^4 b_{02}+6 a_{00}^5 a_{01}^2 a_{02}^3 b_{02}+5 a_{00}^4 a_{01}^4 a_{02}^2 b_{02}-a_{00}^4 a_{01}^3 a_{02}^3 a_{10}-2 a_{00}^4 a_{01}^3 a_{02}^3 b_{01}-2 a_{00}^3 a_{01}^5 a_{02}^2 a_{10}\\
\phantom{c_{00}:=}
-4 a_{00}^3 a_{01}^5 a_{02}^2 b_{01}-a_{00}^2 a_{01}^8 b_{02}-a_{00}^2 a_{01}^7 a_{02} b_{01}+2 a_{00}^2 a_{01}^6 a_{02}^2 b_{00}+a_{00} a_{01}^9 b_{01}+2 a_{00} a_{01}^8 a_{02} b_{00}-a_{01}^{10} b_{00}),\\
c_{10}:=-\frac{1}{a_{01}^9}(12 a_{00}^5 a_{02}^4 a_{10} b_{02}-2 a_{00}^4 a_{01}^3 a_{02}^3 b_{11}+30 a_{00}^4 a_{01}^2 a_{02}^3 a_{10} b_{02}-4 a_{00}^3 a_{01}^5 a_{02}^2 b_{11}+20 a_{00}^3 a_{01}^4 a_{02}^2 a_{10} b_{02}\\
\phantom{c_{00}:=}
-4 a_{00}^3 a_{01}^3 a_{02}^3 a_{10}^2-8 a_{00}^3 a_{01}^3 a_{02}^3 a_{10} b_{01}-a_{00}^2 a_{01}^7 a_{02} b_{11}+2 a_{00}^2 a_{01}^6 a_{02}^2 b_{10}-6 a_{00}^2 a_{01}^5 a_{02}^2 a_{10}^2+a_{01}^9 a_{10} b_{01}\\
\phantom{c_{00}:=}
-12 a_{00}^2 a_{01}^5 a_{02}^2 a_{10} b_{01}+a_{00} a_{01}^9 b_{11}+2 a_{00} a_{01}^8 a_{02} b_{10}-2 a_{00} a_{01}^8 a_{10} b_{02}-2 a_{00} a_{01}^7 a_{02} a_{10} b_{01}-a_{01}^{10} b_{10}\\
\phantom{c_{00}:=}
+4 a_{00} a_{01}^6 a_{02}^2 a_{10} b_{00}+2 a_{01}^8 a_{02} a_{10} b_{00}),\\
c_{01}:=\frac{1}{a_{01}^9}(2 a_{00} a_{02}+a_{01}^2) (6 a_{00}^4 a_{02}^3 b_{02}+12 a_{00}^3 a_{01}^2 a_{02}^2 b_{02}+4 a_{00}^2 a_{01}^4 a_{02} b_{02}-2 a_{00}^2 a_{01}^3 a_{02}^2 a_{10}-4 a_{00}^2 a_{01}^3 a_{02}^2 b_{01}\\
\phantom{c_{00}:=}
-2 a_{00} a_{01}^6 b_{02}-2 a_{00} a_{01}^5 a_{02} a_{10}-4 a_{00} a_{01}^5 a_{02} b_{01}+a_{01}^7 a_{10}+a_{01}^7 b_{01}+2 a_{01}^6 a_{02} b_{00}),\\
c_{20}:=-\frac{1}{a_{01}^9}(30 a_{00}^4 a_{02}^4 a_{10}^2 b_{02}-8 a_{00}^3 a_{01}^3 a_{02}^3 a_{10} b_{11}+60 a_{00}^3 a_{01}^2 a_{02}^3 a_{10}^2 b_{02}+2 a_{00}^2 a_{01}^6 a_{02}^2 b_{20}-12 a_{00}^2 a_{01}^5 a_{02}^2 a_{10} b_{11}\\
\phantom{c_{00}:=}
+30 a_{00}^2 a_{01}^4 a_{02}^2 a_{10}^2 b_{02}-6 a_{00}^2 a_{01}^3 a_{02}^3 a_{10}^3-12 a_{00}^2 a_{01}^3 a_{02}^3 a_{10}^2 b_{01}+2 a_{00} a_{01}^8 a_{02} b_{20}-2 a_{00} a_{01}^7 a_{02} a_{10} b_{11}\\
\phantom{c_{00}:=}
+4 a_{00} a_{01}^6 a_{02}^2 a_{10} b_{10}-6 a_{00} a_{01}^5 a_{02}^2 a_{10}^3-12 a_{00} a_{01}^5 a_{02}^2 a_{10}^2 b_{01}-a_{01}^{10} b_{20}+a_{01}^9 a_{10} b_{11}+2 a_{01}^8 a_{02} a_{10} b_{10}\\
\phantom{c_{00}:=}
-a_{01}^8 a_{10}^2 b_{02}-a_{01}^7 a_{02} a_{10}^2 b_{01}+2 a_{01}^6 a_{02}^2 a_{10}^2 b_{00}),\\
c_{02}:=-\frac{1}{a_{01}^9}(30 a_{00}^4 a_{02}^4 b_{02}+60 a_{00}^3 a_{01}^2 a_{02}^3 b_{02}+30 a_{00}^2 a_{01}^4 a_{02}^2 b_{02}
-6 a_{00}^2 a_{01}^3 a_{02}^3 a_{10}-12 a_{00}^2 a_{01}^3 a_{02}^3 b_{01}-a_{01}^7 a_{02} b_{01}\\
\phantom{c_{00}:=}
-6 a_{00} a_{01}^5 a_{02}^2 a_{10}-12 a_{00} a_{01}^5 a_{02}^2 b_{01}-a_{01}^8 b_{02}+2 a_{01}^6 a_{02}^2 b_{00}),\\
c_{11}:=\frac{1}{a_{01}^9}(60 a_{00}^4 a_{02}^4 a_{10} b_{02}-8 a_{00}^3 a_{01}^3 a_{02}^3 b_{11}+120 a_{00}^3 a_{01}^2 a_{02}^3 a_{10} b_{02}-12 a_{00}^2 a_{01}^5 a_{02}^2 b_{11}+60 a_{00}^2 a_{01}^4 a_{02}^2 a_{10} b_{02}\\
\phantom{c_{00}:=}
-12 a_{00}^2 a_{01}^3 a_{02}^3 a_{10}^2-24 a_{00}^2 a_{01}^3 a_{02}^3 a_{10} b_{01}-2 a_{00} a_{01}^7 a_{02} b_{11}+4 a_{00} a_{01}^6 a_{02}^2 b_{10}-12 a_{00} a_{01}^5 a_{02}^2 a_{10}^2\\
\phantom{c_{00}:=}
-24 a_{00} a_{01}^5 a_{02}^2 a_{10} b_{01}+a_{01}^9 b_{11}+2 a_{01}^8 a_{02} b_{10}-2 a_{01}^8 a_{10} b_{02}-2 a_{01}^7 a_{02} a_{10} b_{01}+4 a_{01}^6 a_{02}^2 a_{10} b_{00}).
\end{array}
\right.
\end{eqnarray*}
Function $l_3$ used in the proof of theorem~\ref{thm5} is defined below
\begin{eqnarray*}
\left.
\begin{array}{l}
l_3:=225752383488 n^4 z_0^{16}-131072 n^3 (2583531 n^2+7447456 s) z_0^{15}+4096 n^2 (33585903 n^4+251047136 n^2 s\\
\phantom{l_2:=}
+284620800 s^2) z_0^{14}-4096 n (6028239 n^6-30947402 n^4 s+115164928 n^2 s^2+30285824 s^3) z_0^{13}\\
\phantom{l_2:=}
+(2149497792 n^8-110495629312 n^6 s-916320976896 n^4 s^2-1538969829376 n^2 s^3\\
\phantom{l_2:=}
-53888417792 s^4) z_0^{12}-32 n (2583531 n^8-1404511288 n^6 s+37943176704 n^4 s^2-17607974912 n^2 s^3\\
\phantom{l_2:=}
-63919882240 s^4) z_0^{11}+(861177 n^{10}-4544677984 n^8 s+243696475136 n^6 s^2+6736688414720 n^4 s^3\\
\phantom{l_2:=}
+742425362432 n^2 s^4-249477201920 s^5) z_0^{10}+2 n s (33226157 n^8-19405317120 n^6 s\\
\phantom{l_2:=}
-367048667136 n^4 s^2-5776862806016 n^2 s^3-642417426432 s^4) z_0^9+8 s^2 (64159695 n^8\\
\phantom{l_2:=}
+12692850816 n^6 s-263719462400 n^4 s^2+1106692308992 n^2 s^3+127567659008 s^4) z_0^8\\
\phantom{l_2:=}
-64 n s^3 (86338711 n^6-5998296800 n^4 s-137168962560 n^2 s^2+52037468160 s^3) z_0^7\\
\phantom{l_2:=}
+128 s^4 (15647447 n^6-10653378992 n^4 s-84959182848 n^2 s^2+6329958400 s^3) z_0^6\\
\phantom{l_2:=}
+128 n s^5 (425095541 n^4+7885825664 n^2 s+44151396352 s^2) z_0^5-512 s^6 (212224979 n^4\\
\phantom{l_2:=}
-1481374208 n^2 s+1881290752 s^2) z_0^4-24576 n s^7 (386861 n^2+53753760 s) z_0^3+16384 s^8 (12616373 n^2\\
\phantom{l_2:=}
+29268544 s) z_0^2-200822751232 n s^9 z_0+61032890368 s^{10}.
\end{array}
\right.
\end{eqnarray*}
Coefficients $\tilde{a}_{ij}$ and $\tilde{b}_{ij}$ in system (\ref{(2.20)}) are listed below
\begin{eqnarray*}
\left.
\begin{array}{l}
\tilde{a}_{20}:= \frac{n z_0-8 z_0^2+8 s}{8 z_0^2 s n \sqrt{s (n z_0-s)}},~~
\tilde{a}_{11}:= -\frac{n+8 z_0}{4 n s z_0 (n z_0-s)},~~
\tilde{a}_{02}:= -\frac{3 n+8 z_0}{8 n z_0 (n z_0-s)  \sqrt{s (n z_0-s)}},~~
\tilde{a}_{30}:= -\frac{n z_0+16 s}{16 z_0^4 s^2 n^2 \sqrt{s (n z_0-s)}},\\
\tilde{a}_{21}:= -\frac{n z_0+16 s}{16 z_0^4 n^2 s^2 (n z_0-s)},~~
\tilde{a}_{12}:= \frac{1}{16 n s z_0^3 (n z_0-s) \sqrt{s (n z_0-s)} },~~
\tilde{a}_{03}:= \frac{1}{16 s n  z_0^3 (n z_0-s)^2},~~
\tilde{a}_{40}:= \frac{5 n z_0+128 s}{128 z_0^6 s^3 n^3 \sqrt{s (n z_0-s)}},\\
\tilde{a}_{31}:= \frac{3 n z_0+64 s}{32 z_0^6 s^3 n^3 (n z_0-s)},~~
\tilde{a}_{22}:= \frac{3 n z_0+64 s}{64  n^3 z_0^6 s^2 (n z_0-s) \sqrt{s (n z_0-s)}},~~
\tilde{a}_{13}:= -\frac{1}{32 n^2z_0^5 s^2 (n z_0-s)^2},\\
\tilde{a}_{04}:= -\frac{3 s}{128 n^2 s^2 z_0^5 (n z_0-s)^2 \sqrt{s (n z_0-s)} },~~
\tilde{a}_{50}:= -\frac{7 n z_0+256 s}{256 z_0^8 s^4 n^4 \sqrt{s (n z_0-s)}},~~
\tilde{a}_{41}:= -\frac{25 n z_0+768 s}{256 z_0^8 s^4 n^4 (n z_0-s)},\\
\tilde{a}_{32}:= -\frac{3 (5 n z_0+128 s)}{128 z_0^8  n^4 s^3 (n z_0-s) \sqrt{s (n z_0-s)}},~~
\tilde{a}_{23}:= -\frac{5 n z_0+128 s}{128 z_0^8 n^4 s^3 (n z_0-s)^2},~~
\tilde{a}_{14}:= \frac{5}{256 n^3 s^2 z_0^7 (n z_0-s)^2 \sqrt{s (n z_0-s)} },\\
\tilde{a}_{05}:= \frac{3}{256 s^2 n^3z_0^7 (n z_0-s)^3},~~
\tilde{a}_{60}:= \frac{21 n z_0+1024 s}{1024 z_0^{10} s^5 n^5 \sqrt{s (n z_0-s)}},~~
\tilde{a}_{51}:= \frac{49 n z_0+2048 s}{512 z_0^{10} s^5 n^5 (n z_0-s)},\\
\tilde{a}_{42}:= \frac{175 n z_0+6144 s}{1024 z_0^{10} n^5 s^4 (n z_0-s)  \sqrt{s (n z_0-s)}},~~
\tilde{a}_{33}:=\frac{35 n z_0+1024 s}{256 z_0^{10} s^4 n^5 (n z_0-s)^2},~~
\tilde{a}_{24}:= \frac{35 n z_0+1024 s}{1024 z_0^{10} n^5 s^3 (n z_0-s)^2 \sqrt{s (n z_0-s)}},\\
\tilde{a}_{15}:= -\frac{7}{512 n^4 s^3 z_0^9 (n z_0-s)^3 },~~
\tilde{a}_{06}:= -\frac{7 s}{1024 n^4 s^3 z_0^9 (n z_0-s)^3 \sqrt{s (n z_0-s)}},~~
\tilde{a}_{70}:= -\frac{33 n z_0+2048 s}{2048 z_0^{12} s^6 n^6 \sqrt{s (n z_0-s)}},\\
\tilde{a}_{61}:= -\frac{189 n z_0+10240 s}{2048 z_0^{12} s^6 n^6 (n z_0-s)},~~
\tilde{a}_{52}:= -\frac{441 n z_0+20480 s}{2048 z_0^{12}n^6 s^5 (n z_0-s)  \sqrt{s (n z_0-s)}},~~
\tilde{a}_{43}:= -\frac{5 (105 n z_0+4096 s)}{2048 z_0^{12} s^5 n^6 (n z_0-s)^2},\\
\tilde{a}_{34}:= -\frac{5 (63 n z_0+2048 s)}{2048 z_0^{12} s^4 n^6 (n z_0-s)^2 \sqrt{s (n z_0-s)}},~~
\tilde{a}_{25}:= -\frac{63 n z_0+2048 s}{2048 z_0^{12} n^6 s^4 (n z_0-s)^3},~~
\tilde{a}_{16}:= \frac{21}{2048 n^5 s^3 z_0^{11} (n z_0-s)^3 \sqrt{s (n z_0-s)} },\\
\tilde{a}_{07}:= \frac{9}{2048 s^3 n^5z_0^{11} (n z_0-s)^4 },~~
\tilde{b}_{21}:= \frac{1}{n^2 s z_0^4 \sqrt{s (n z_0-s)} },~~
\tilde{b}_{20}:= -\frac{1}{s n z_0^2},~~
\tilde{b}_{31}:= -\frac{2}{n^3 s^2 z_0^6 \sqrt{s (n z_0-s)} },\\
\tilde{b}_{22}:= -\frac{1}{s^2 n^3 z_0^6 (n z_0-s) },~~
\tilde{b}_{30}:= \frac{1}{s^2 n^2 z_0^4},~~
\tilde{b}_{50}:= \frac{1}{s^4 n^4 z_0^8},~~
\tilde{b}_{41}:= \frac{3}{n^4  s^3 z_0^8 \sqrt{s (n z_0-s)}},~~
\tilde{b}_{32}:= \frac{3}{s^3 n^4 z_0^8 (n z_0-s) },\\
\tilde{b}_{23}:= \frac{1}{n^4 s^2 z_0^8 (n z_0-s) \sqrt{s (n z_0-s)} },~~
\tilde{b}_{40}:= -\frac{1}{s^3 n^3 z_0^6},~~
\tilde{b}_{60}:= -\frac{1}{s^5 n^5 z_0^{10}},~~
\tilde{b}_{51}:= -\frac{4}{n^5 s^4 z_0^{10} \sqrt{s (n z_0-s)} },\\
\tilde{b}_{42}:= -\frac{6}{s^4 n^5 z_0^{10} (n z_0-s) },~~
\tilde{b}_{33}:= -\frac{4}{n^5 s^3 z_0^{10} (n z_0-s) \sqrt{s (n z_0-s)} },~~
\tilde{b}_{24}:= -\frac{1}{n^5 s^3 z_0^{10} (n z_0-s)^2 },~~
\tilde{b}_{70}:= \frac{1}{s^6 n^6 z_0^{12}},\\
\tilde{b}_{61}:= \frac{5}{n^6 s^5 z_0^{12} \sqrt{s (n z_0-s)} },~~
\tilde{b}_{52}:= \frac{10}{s^5  n^6 z_0^{12} (n z_0-s)},~~
\tilde{b}_{43}:= \frac{10}{n^6 s^4  z_0^{12} (n z_0-s) \sqrt{s (n z_0-s)}},~~
\tilde{b}_{34}:= \frac{5}{n^6 s^4z_0^{12} (n z_0-s)^2 },\\
\tilde{b}_{25}:= \frac{1}{n^6 s^3 z_0^{12} (n z_0-s)^2 \sqrt{s (n z_0-s)} }.
\end{array}
\right.
\end{eqnarray*}
Expressions of resultant $r_{12}$ and function $r_4$ are listed below
\begin{eqnarray*}
\begin{array}{l}
r_{12}:=\mbox{res}(l_0, l_2, z_0)\\
\phantom{r_{11}~}
=-305424 (2 s-1) (297 s^5-11136 s^4+16619 s^3-7645 s^2+804 s+101) n^{12}+(-64391428224 s^7\\
\phantom{r_{11}~=}
+38324524032 s^6-11857197312 s^5+207789267456 s^4-281309940864 s^3+116379698688 s^2\\
\phantom{r_{11}~=}
-4688985600 s-4324571136) n^{10}+(810785452032 s^8-4108555210752 s^7+13324275068928 s^6\\
\phantom{r_{11}~=}
-27135190560768 s^5+30777281961984 s^4-19131506589696 s^3+6415787409408 s^2\\
\phantom{r_{11}~=}
-1099350786048 s+83232325632) n^8+(79609725444096 s^9-444285353385984 s^8\\
\phantom{r_{11}~=}
+1085278669119488 s^7-1488869127430144 s^6+1254116567482368 s^5-679083628494848 s^4\\
\phantom{r_{11}~=}
+241660389490688 s^3-55533925171200 s^2+7460613521408 s-449253998592) n^6\\
\phantom{r_{11}~=}
+(245454803828736 s^{10}-1132936831696896 s^9+2133846486417408 s^8-2076833578221568 s^7\\
\phantom{r_{11}~=}
+1129021261742080 s^6-373127968194560 s^5+100240638083072 s^4-33808405495808 s^3\\
\phantom{r_{11}~=}
+11265690828800 s^2-2194375966720 s+171144380416) n^4+33554432 s^2 (7356911 s^9-25673518 s^8\\
\phantom{r_{11}~=}
+30288576 s^7-10786805 s^6-3177302 s^5+2023014 s^4+530916 s^3-582400 s^2+161628 s-16340) n^2\\
\phantom{r_{11}~=}
+1073741824 s^4 (274 s^4-309 s^3-107 s^2+84 s-2)^2,\\
r_4:=44820126806841777107618930250 s^{38}-1257070491235707788295129142125 s^{37}\\
\phantom{r_2:=}
+16937951865522398143783584692130 s^{36}-145986364587773106370065248351217 s^{35}\\
\phantom{r_2:=}
+904008577756147373533147439545782 s^{34}-4282185166647555691630082193782400 s^{33}\\
\phantom{r_2:=}
+16127855049213976023181005391255716 s^{32}-49547607535630337740354999334092608 s^{31}\\
\phantom{r_2:=}
+126361503856638229588641232797718008 s^{30}-270737157415035017121746056781534910 s^{29}\\
\phantom{r_2:=}
+491085486274841237013209424014182344 s^{28}-757087732740541686283300863437673678 s^{27}\\
\phantom{r_2:=}
+992052499168168712625214889479144800 s^{26}-1099591126107439172354201419019968836 s^{25}\\
\phantom{r_2:=}
+1018288873028928777082440423919141308 s^{24}-766812131575295039595895988019403452 s^{23}\\
\phantom{r_2:=}
+439782632083540311634408011638674158 s^{22}-152678433272337069068276404793678781 s^{21}\\
\phantom{r_2:=}
-22210728149360221080226141696766842 s^{20}+81773260933294855052625517774031015 s^{19}\\
\phantom{r_2:=}
-70069640352949497366438605771044998 s^{18}+37061402054486524598400990575918444 s^{17}\\
\end{array}
\end{eqnarray*}
\begin{eqnarray*}
\begin{array}{l}
\phantom{r_2:=}
-11798579161906765831630359712768800 s^{16}+500360804256501118421951662434820 s^{15}\\
\phantom{r_2:=}
+1880804305883642573888577895655648 s^{14}-1212429294404991587710475126908448 s^{13}\\
\phantom{r_2:=}
+401358107337339560786048160265552 s^{12}-53835696130866710903930585078960 s^{11}\\
\phantom{r_2:=}
-17429346683118726135531396659488 s^{10}+12119864767479874103527073980128 s^9\\
\phantom{r_2:=}
-3276250730344330624672320883072 s^8+352367256451951211170524923328 s^7\\
\phantom{r_2:=}
+62652951889026256147092533632 s^6-31093749837416758205639631232 s^5\\
\phantom{r_2:=}
+4828176923979670029162405376 s^4-43041373277758211392729088 s^3\\
\phantom{r_2:=}-109348369901108819492454400 s^2+17833791328587200738918400 s\\
\phantom{r_2:=}
-1014846759850859053056000.
\end{array}
\end{eqnarray*}








\end{document}
