%%%%%%%%%%%%%%%%%%%%%%%%%%%%%%%%%%%%%
%%%%%%%%%%%%%%%%%%%%%%%%%%%%%%%%%%%%%%%%
%%%%%%%%%%%%%%%%%%%%%%%%%%%%%%%%%%%%%
%%%%%% Elliptic P1  %%%%%%%%%%%%%%%%%%%%%%%
%%%%%%%%%%%%%%%%%%%%%%%%%%%%%%%%%%
\documentclass[a4paper,12pt,twoside,reqno]{amsart}
%\pagestyle{myheadings}
\usepackage{amsmath}
\usepackage{amsthm}
\usepackage{amssymb}
\setlength{\textheight}{23cm}
\setlength{\textwidth}{16cm}
\setlength{\oddsidemargin}{0cm}
\setlength{\evensidemargin}{0cm}
\setlength{\topmargin}{0cm}
\renewcommand{\baselinestretch}{1.2}
\textwidth=15.7cm \textheight=24.0cm
\voffset-0.6cm
%%%%%%%%%%%%%%%%%%%%%%%%%%%%%%%%%%%%%%%%%%%%%%%%
%%%
%%%%%%%%%%%%%%%%%%%%%%%%%%%%%%%%%%%%%%%%%%%%%%%%
\theoremstyle{plain}
\newtheorem{thm}{\indent\bf Theorem}[section]
\newtheorem{lem}[thm]{\indent\bf Lemma}
\newtheorem{prop}[thm]{\indent\bf Proposition}
\newtheorem{cor}[thm]{\indent\bf Corollary}
\newtheorem*{thma}{\indent\bf Theorem A}
%\newtheorem*{lema}{\indent\sc Lemma A.2}
%%%
\theoremstyle{definition}
\newtheorem{rem}{\indent\it Remark}[section]
\newtheorem{exa}{\indent\it Example}[section]
%%%%%%%%%%%%%%%%%%%%%%%%%%%%%%%%%%%%%%%
%%%%%%%%%%%%%%%%%%%%%%%%%%
%%%%%%%%%%%%%%%%%%%%%%%%%%%%%%%%%%%%%%%
\numberwithin{equation}{section}
\numberwithin{figure}{section}
%%%%%%%%%%%%%%%%%%%%%%%%%%
\renewcommand{\proofname}{\indent\it Proof.}
%%%%%%%%%%%%%%%%%%%%%%%%%%%%%%%%%%%%%%%%%%%%%%%%%%
\def \i {\mathrm{i}}
\def \e {\mathrm{e}}
\def \d {\mathrm{d}}
\def \A {\mathbb{A}}
\def \C {\mathbb{C}}
\def \R {\mathbb{R}}
\def \Q {\mathbb{Q}}
\def \Z {\mathbb{Z}}
\def \N {\mathbb{N}}
\def \re {\mathrm{Re\,}}
\def \im {\mathrm{Im\,}}
\def \diag {\mathrm{diag}}
\def \dist {\mathrm{dist}}
\def \z {\mathfrak{z}}
\def \sn {\mathrm{sn}}
\def \cn {\mathrm{cn}}
%%%%%%%%%%%%%%%%%%%%%%%%%%%%%%%%%%%%%%%%%%%%%%%%%%
\begin{document}
\title[First Painlev\'e transcendents]
{Explicit error bound of the elliptic asymptotics 
for the first Painlev\'e transcendents}
\author[Shun Shimomura]{Shun Shimomura} 
%%%%%%%%%%%%%%%%%%%%%%%%%%%%%%%%%%%%%%%%%%%%%
%%%%%%%%%%%%%%%%%%%%%%%%%%%%%%%%%%%%%%%%%%%%%%
\address{Department of Mathematics, 
Keio University, 
3-14-1, Hiyoshi, Kohoku-ku,
Yokohama 223-8522 
Japan\quad
{\tt shimomur@math.keio.ac.jp}}
\date{}
%%%%%%%%%%%%%%%%%%%%%%%%%%%%%%%%%%%%%%%
%%%%%%%%%%%%%%%%%%%%%%%%%%%%%%%%%%%%%%%
\begin{abstract}
For the first Painlev\'e transcendents Kitaev established an asymptotic
representation in terms of the Weierstrass pe-function
in cheese-like strips near the point at infinity. 
We present an explicit error bound of this asymptotic 
expression, which leads to the order estimate of exponent $-1$. 
\par
2020 {\it Mathematics Subject Classification.} 
{34M55, 34M56, 34M40, 34M60, 33E05.}
\par
{\it Key words and phrases.} 
{Boutroux ansatz; first Painlev\'e transcendents; 
isomonodromy deformation; monodromy data; 
Weierstrass pe-function.}
\end{abstract}
\maketitle
\allowdisplaybreaks
%%%%%%%%%%%%%%%%%%%%%%%%%%%%%%%%%%%%%
%%%%%%%%%%%%%%%%%%%%%%%%%
%%%%% Section 1 %%%%%%%%
\section{Introduction}\label{sc1}
The first Painlev\'e equation 
%%%%%%% (1.1) %%%%%%%%%%%%%%%%%%%%%%%%
\begin{equation}\label{1.1}
  y''= 6y^2+x 
\end{equation}
governs the isomonodromy deformation of the two-dimensional linear system
%%%%%%%%%%%%%%%%%%%%%%%%
%%%%%% (1.2) %%%%%%%%
\begin{align}\label{1.2}
\frac{d\Psi}{d\xi}= & \Bigl( (4\xi^4 + x+ 2y^2)\sigma_3 
- i(4y \xi^2 + x+ 2y^2)\sigma_2 -(2y'\xi +\tfrac 12\xi^{-1})\sigma_1 
\Bigr)\Psi
\end{align}
with the Pauli matrices 
% $\sigma_j=(\sigma_{j,pq})$ $(j=1,2,3; p,q=1,2)$
% such that $\sigma_{1,12}=\sigma_{1,21}=1,$ $-\sigma_{2,12}=\sigma_{2,21}=i,$
% $\sigma_{3,11}=-\sigma{3,22}=1,$ and $\sigma_{j,pq}=0$ otherwise  
$$
 \sigma_1= \begin{pmatrix}  0 & 1 \\  1 & 0 \end{pmatrix},
 \quad
 \sigma_2= \begin{pmatrix}  0 & -i  \\  i & 0 \end{pmatrix},
 \quad
 \sigma_3= \begin{pmatrix}  1 & 0  \\  0 & -1 \end{pmatrix}
$$
(\cite{JM}, \cite{Ka}, \cite{FIKN}). Each solution of \eqref{1.1} 
is parametrised by $(s_j)_{1\le j \le 5}$ on the manifold of monodromy data
for \eqref{1.2}, which is a
two-dimensional complex manifold in $\mathbb{C}^5$ defined by $S_1S_2S_3
S_4S_5=i\sigma_2$ with the Stokes matrices
$$
S_{2l+1}=\begin{pmatrix} 1 & s_{2l+1} \\ 0 & 1 \end{pmatrix}, \quad  
S_{2l}=\begin{pmatrix} 1 & 0 \\ s_{2l}  & 1 \end{pmatrix}, \quad l=0,1,2,
\ldots
$$
(see \cite{Ka}, \cite{Ka-Ki}, \cite{K1}, \cite{K-3}).  
For a general solution of \eqref{1.1} thus parametrised, Kapaev \cite{Ka}
obtained asymptotic representations as $x \to \infty$ along the Stokes rays
$\arg x=\pi + 2\pi k/5$ $(k=0,\pm 1, \pm 2)$.
Along generic directions the Boutroux ansatz \cite{Boutroux} suggests the
asymptotic behaviour of a general solution of \eqref{1.1} expressed by the
Weierstrass $\wp$-function. An approach to such an expression was made by
Joshi and Kruskal \cite{J-K} by the method of multiple-scale expansions, in which
the labelling of the solution by $(s_j)$ is not considered. Based on the
isomonodromy deformation of \eqref{1.2}, using WKB analysis, Kitaev \cite{K1},
\cite{K-3}, \cite{Ka-Ki} presented the elliptic asymptotic representation
of a general solution of \eqref{1.1} for $3\pi/5<\phi -2\pi k/5 <\pi,$ $k\in
\mathbb{Z}.$ The result in a sector with $k=-2$ is described as follows
\cite[p. 593, Section 8]{K1}, \cite[Theorem 2]{K-3}, \cite[Theorem 2]{Ka-Ki}.  
%%%%%%%%%%%%%%%%%%% Theorem A %%%%%%%%%%%%%%%%%%%%%
\begin{thma}\label{thmA}
Let $y(x)$ be a solution of \eqref{1.1} corresponding to the monodromy data
$(s_j)_{1\le j\le 5}$ such that $s_1s_4\not=0.$ Set $t=\tfrac 45 (e^{-i\phi}
x)^{5/4},$ and suppose that $|\phi|<\pi/5.$ Then $y(x)$ admits the asymptotic
representation
%%%%% (1.3) %%%%%%%%
\begin{equation}\label{1.3}
y(x)=(e^{-i\phi}x)^{1/2}\left(\wp(e^{i\phi}t-t_0; g_2(\phi),g_3(\phi))
 +O(t^{-\delta})\right)
\end{equation}
as $x\to \infty$ through the strip $\mathcal{D}(\phi,c,\varepsilon),$ where
$\wp(u;g_2,g_3)$ is the $\wp$-function such that
$(\wp_{u})^2=4\wp^3-g_2 \wp -g_3$ with $g_2(\phi)= -2e^{i\phi},$ $g_3(\phi)
=-A_{\phi},$ $\delta$ is a small positive number, and $A_{\phi},$ $t_0$ and
$\mathcal{D}(\phi, c,\varepsilon)$ with given $c>0$ and given small 
$\varepsilon>0$ are as in {\bf (2)} and {\bf (3)} below.
\end{thma}
The leading term of \eqref{1.3} depends on the integration constant $x_0
=x_0((s_j)_{1\le j \le 5})$ only and the other one is contained in the
error term $O(t^{-\delta})$. In treating $y(x)$ as a general solution, it
is preferable to know the explicit error bound. Furthermore, this high-order
part is related, say, to
the $\tau$-function \cite[p.~121]{K-3} or to degeneration to trigonometric
asymptotics \cite[Section 4]{K-3}. For the $\tau$-function
associated with \eqref{1.1}, Iwaki \cite{Iwaki}, by the method of topological 
recursion, obtained a conjectural full-order formal expansion yielding the
elliptic expression. 
\par
In this paper we present the explicit representation of the error bound,
which leads to the error estimate $O(t^{-1})=O(x^{-5/4})$ in \eqref{1.3}.
These main results are
stated in Theorems \ref{thm2.1} and \ref{thm2.2}. Corollary
\ref{cor2.3} describes the dependence on the other integration
constant $\beta_0$ in this explicit error bound. 
In Section \ref{sc4} these results are derived 
from the $t^{-\delta}$-asymptotics of $y(x)$ in Theorem A and that
of the correction function $B_{\mathrm{as}}(\phi,t)$ 
\cite{K-3} combined with a system of integral equations 
equivalent to \eqref{1.1},
which is constructed in Section 3. Necessary lemmas in our argument are
shown in the final section. Our method is quite different from that in
\cite{I-K}, \cite{J-K} and \cite{Novo}.
\par 
Throughout this paper we use the following symbols (cf. \cite{K1}, \cite{K-3}).
\par
{\bf (1)} For each $\phi$ such that $|\phi|<\pi/5$, $A_{\phi}\in \mathbb{C}$ 
denotes a unique solution of the Boutroux equations
$$
\re \int_{\mathbf{a}} w d\lambda =\re \int_{\mathbf{b}} w d\lambda =0,
$$
with $w=w(A_{\phi},\lambda)=\sqrt{\lambda^3+\tfrac 12 e^{i\phi}\lambda 
+\tfrac 14
A_{\phi}}=\sqrt{\lambda-\lambda_1}\,\sqrt{\lambda-\lambda_2}\,
\sqrt{\lambda-\lambda_3}.$ 
The unique existence of $A_{\phi}$ is guaranteed by
\cite[Section 7]{K1} and $0.336 <A_0<0.380.$ 
Here $\mathbf{a}$ and $\mathbf{b}$ are basic cycles as in Figure \ref{cycles}
on the elliptic curve $\Gamma=\Gamma_+ \cup \Gamma_-$ defined by $w(A_{\phi},
\lambda)$ such that two sheets $\Gamma_+$ and $\Gamma_-$ are glued along 
the cuts
$[\lambda_1,\lambda_2]\cup[-\infty,\lambda_3]$; the branch points $\lambda_j
=\lambda_j(\phi)$ for $\phi$ around $\phi=0$ are specified in such a way that
$\re \lambda_1(0)>0,$ $\im \lambda_1(0)>0,$ 
$\lambda_2(0)=\overline{\lambda_1(0)},$ $\lambda_3(0)<0,$ and the branches of
$\sqrt{\lambda-\lambda_j}$ are determined by $\sqrt{\lambda-\lambda_j}
\to +\infty$ as $\lambda\to +\infty$ along the positive axis on the upper
sheet $\Gamma_+.$
%%%%%%%%%%%%%%%%%%%%%%%%%%%%%%%%%%%%%%%%%%%%%%%%%%%%%%%%%
%%%%%%%%%%%% Figure 1.1 %%%%%%%%%%%%%%%%%%%%% 
%%%%%%%%%%%%%%%%%%%%%%%%%%%%%%%%%%%%%%%%%%%%%
{\small
% Figure environment removed
}
%%%%%%%%%%%%%%%%%%%%%%%%%%%%%%%%%%%%%%%%%
\par
{\bf (2)} For the cycles $\mathbf{a}$ and $\mathbf{b}$ on $\Gamma$ write
\begin{align*}
& \omega_{\mathbf{a}}=\omega_{\mathbf{a}}(\phi)=\frac 12\int_{\mathbf{a}}
\frac{d\lambda}{w}, \quad
\omega_{\mathbf{b}}=\omega_{\mathbf{b}}(\phi)=\frac 12\int_{\mathbf{b}}
\frac{d\lambda}{w}, \quad
\\
& J_{\mathbf{a}}=J_{\mathbf{a}}(\phi)=2 \int_{\mathbf{a}} w d\lambda,
 \quad
 J_{\mathbf{b}}=J_{\mathbf{b}}(\phi)=2 \int_{\mathbf{b}} w d\lambda,
\end{align*}
and set
$$
t_0=\frac 1{2\pi i} \Bigl(\omega_{\mathbf{a}} \log(is_1) +\omega_{\mathbf{b}}
\log \frac{s_4}{s_1} \Bigr).
$$
\par
{\bf (3)} For given $c>0$ and given small $\varepsilon >0$,
\begin{align*}
 \mathcal{D}(\phi,c,\varepsilon )= & \{ x= e^{i\phi} (\tfrac 54 t)^{4/5}
\in \mathbb{C}\, :
\\
& \, \re t>0,\,\, |\im t|<c,\,\, |e^{i\phi}t-t_0-m\omega
_{\mathbf{a}}-n \omega_{\mathbf{b}}| >\varepsilon, \, m,n \in \mathbb{Z}\}, 
\\
 \check{\mathcal{D}}(\phi,c,\varepsilon )= & \mathcal{D}(\phi,c,\varepsilon )
\setminus \bigcup_{1\le j\le 3;\, m,n\in \mathbb{Z}} 
\mathcal{V}_{j,m,n}(\varepsilon),
\\
\mathcal{V}_{j,m,n}(\varepsilon)
=& \{ x= e^{i\phi} (\tfrac 54 t)^{4/5}\in \mathbb{C}\,:\, |e^{i\phi}(t-t_j)
-t_0-m\omega_{\mathbf{a}} -n\omega_{\mathbf{b}} | \le \varepsilon   \},
\end{align*}
where $t_j$ $(1\le j \le 3)$ are such that $\wp(e^{i\phi}t_j -t_0; g_2(\phi),
g_3(\phi))=\lambda_j$. Write $t_j^{m,n}=t_j+e^{-i\phi}(t_0+m\omega_{\mathbf{a}}
+n\omega_{\mathbf{b}})$ and $t_{\infty}^{m,n}=e^{-i\phi}(t_0+m\omega_{\mathbf
{a}}+n\omega_{\mathbf{b}})$ for $j=1,2,3$ and $m,n \in \mathbb{Z}$.
Let $l(t^{m,n}_j)$ $(j=1,2,3,\infty)$ be the lines defined by 
$x=e^{i\phi}(\tfrac 54 t)^{4/5}$: $t=\re t^{m,n}_j +i\eta$, $\eta \ge \im 
t^{m,n}_j$ if $\im t^{m,n}_j \ge 0$ 
(respectively, $\eta \le \im t^{m,n}_j$ if $\im t^{m,n}_j <0$) and let
$\check{\mathcal{D}}_{\mathrm{cut}}(\phi,c,\varepsilon)$ be
$\check{\mathcal{D}}(\phi,c,\varepsilon)$ with cuts along  
$l(t^{m,n}_j)$ for all $j=1,2,3, \infty$ and $m,n \in \mathbb{Z},$ where
some local segments contained in
$l(t^{m,n}_j)$ may be replaced with suitable arcs, if necessary, 
not to touch another small circle $|t-t^{m',n'}_{j'}|=\varepsilon$
with $(j',m',n')\not=(j,m,n).$ 
% $$
% \check{\mathcal{D}}_{\mathrm{cut}}(\phi,c,\varepsilon)
% =\check{\mathcal{D}}(\phi,c,\varepsilon)\setminus \bigcup_{j=1,2,3,\infty;
% \, m,n\in \mathbb{Z}} l(t^{m,n}_j).
% $$
Then $\check{\mathcal{D}}_{\mathrm{cut}}(\phi,c,\varepsilon)$ is simply
connected.
These domains are strips tending to the direction $\arg x=\phi$ in the
$x$-plane. In addition to $f(x)$ we deal with a function $f(t)$ with 
$t=\tfrac 45 (e^{-i\phi}x)^{5/4}$ or $f(z)$ with $z=e^{i\phi}t
=\tfrac 45 e^{i\phi}(e^{-i\phi}x)^{5/4}$ as $x\to \infty$, say, through
$\check{\mathcal{D}}_{\mathrm{cut}}(\phi, c,\varepsilon)$, and then
we write, in short, $f(t)$ or $f(z)$ in 
$\check{\mathcal{D}}_{\mathrm{cut}}(\phi, c,\varepsilon)$. 
\par
{\bf (4)} For $\im \tau>0$, 
$$
\vartheta(z,\tau)=\sum_{n\in \mathbb{Z}}e^{\pi i\tau n^2+2\pi izn},
$$ 
is the theta-function \cite{H}, \cite{WW} with $\tau=\omega_{\mathbf{b}}
/\omega_{\mathbf{a}}$, $\nu=(1+\tau)/2,$ which satisfies $\vartheta(z\pm 1,
\tau)=\vartheta(z,\tau),$ $\vartheta(z\pm \tau,\tau)=e^{-\pi i(\tau\pm 2z)}
\vartheta(z,\tau).$ 
\par
{\bf (5)} We write $f \ll g$ or $g \gg f$ if $f=O(g)$, and $f \asymp g$
if $f\ll g$ and $g\ll f.$
%%%%%%%%%%%%%%%%%%%%%%%%%%%%%%%%%%%%%%%
%%%%%%% Section 2 %%%%%%%%%%%%%%%%%%%
%%%%%%%%%%%%%%%%%%%%%%%%%%%%%%%%
\section{Main results}\label{sc2}
Suppose that $|\phi|<\pi/5.$ Let $y(x)$ be a solution of \eqref{1.1}
corresponding to $(s_j)_{1\le j\le 5}$ such that $s_1s_4 \not=0.$
Then we have the following.
%%%%%%%%%%%%%%%%%%%%%%%%%%%%%%%%%%%%%%%%%
%%%%%%%%% Theorem 2.1 %%%%%%%%%%%
\begin{thm}\label{thm2.1}
Let $c>0$ be a given number and $\varepsilon>0$ a given
small number. Then
$$
y(x)=(e^{-i\phi}x)^{1/2}
 \left(\wp(e^{i\phi}t-t_0; g_2(\phi),g_3(\phi))+O(t^{-1})\right)
$$
as $x=e^{i\phi}(\tfrac 54 t)^{4/5} \to \infty$ through $\mathcal{D}(\phi, c,
\varepsilon)$. 
\end{thm}
To deal with the error bound explicitly, let us write \eqref{1.3} in the form
$$
y(x)=(e^{-i\phi}x)^{1/2} \wp(e^{i\phi}t-t_0 +h(e^{i\phi}t);g_2(\phi),g_3(\phi)).
$$
Set $z=e^{i\phi}t$ and write
\begin{align*}
& \mathfrak{p}=\mathfrak{p}(z)=\wp(z-t_0; g_2(\phi),g_3(\phi)),
\\
&\beta=\beta(z)=\frac{8e^{i\phi}}{5\omega_{\mathbf{a}}}\Bigl(\beta_0 -\frac 54
e^{-i\phi}J_{\mathbf{a}} (z-t_0) +\frac{\vartheta'}{\vartheta}\Bigl(\frac{z-t_0}
{\omega_{\mathbf{a}}} +\nu,\tau \Bigr) \Bigr),
\\
& \beta_0=\log \frac{s_1}{s_4}-\frac 54 e^{-i\phi}J_{\mathbf{a}} t_0 +\pi i,
\end{align*}
in which $\vartheta'(z,\tau)=(d/dz)\vartheta(z,\tau)$ and 
$\beta(z)=e^{i\phi}B_{\mathrm{as}}
(\phi,t)$ and $\mathfrak{p}(z)$ are bounded in $\mathcal{D}(\phi,c,\varepsilon)$
\cite[Theorem 4]{K-3}.
Then we have the following expression of $h(z).$
%%%%%%%%%%%%%%%%%%%%%%%%%%%%%
%%%%%%%% Theorem 2.2 %%%%%%%%%%%%
\begin{thm}\label{thm2.2}
Let $P(\lambda)=4w(A_{\phi},\lambda)^2=4\lambda^3 +2e^{i\phi}\lambda+A_{\phi},$
and let $\delta>0$ be as in Theorem A.
Then as $x=e^{i\phi}(\tfrac 54 e^{-i\phi}z)^{4/5}
\to \infty$ through $\check{\mathcal{D}}_{\mathrm{cut}}(\phi,c,\varepsilon)$,
\begin{align*}
h(z)= &  \int^z_{\infty} F(\mathfrak{p},\beta)\frac{d\zeta}{\zeta}
-\int^z_{\infty} (F(\mathfrak{p},\beta)^2+G(\mathfrak{p},\beta)) \frac{d\zeta}{\zeta^2}
\\
& -\frac 1{10} \int^z_{\infty} K(\mathfrak{p},\beta) \mathcal{I}(\zeta) 
\frac{d\zeta}{\zeta} +O(z^{-1-\delta}),
\end{align*}
in which
\begin{align*}
F(\mathfrak{p},\beta) &=\frac{\beta}{2P(\mathfrak{p})}-\frac{2\mathfrak{p}}{5\sqrt{P(\mathfrak{p})}},
\quad
G(\mathfrak{p},\beta)=\frac{\beta^2}{8P(\mathfrak{p})^2},
\\
K(\mathfrak{p},\beta) &= 2(4e^{i\phi}\mathfrak{p} +3A_{\phi})F(\mathfrak{p},\beta)-\beta, \quad
\mathcal{I}(z)=\int^z_{\infty} \frac 1{P(\mathfrak{p})}\frac{d\zeta}{\zeta}
\end{align*}
and $\mathfrak{p}=\mathfrak{p}(\zeta),$ $\beta=\beta(\zeta)$ in the integrands. Here 
\begin{align*}
&\mathcal{I}(z) \ll z^{-1}, \quad \int^z_{\infty} F(\mathfrak{p},\beta)\frac{d\zeta}
{\zeta} \ll z^{-1},
\\
&\int^z_{\infty}(F(\mathfrak{p},\beta)^2+G(\mathfrak{p},\beta))\frac{d\zeta}{\zeta^2}\ll z^{-1},
\quad  \int^z_{\infty} K(\mathfrak{p},\beta)\mathcal{I}(\zeta) \frac{d\zeta}{\zeta}
\ll z^{-1}.
\end{align*} 
\end{thm}
%%%%%%%%%%%%%%%%%%%%%%%%%%%%%
%%%%%%% Remark 2.1 %%%%%%%%%%%%
\begin{rem}\label{rem2.1}
For the convenience' sake the integrals are considered in 
$\check{\mathcal{D}}_{\mathrm{cut}}(\phi,c,\varepsilon)$ to avoid the 
possible multi-valuedness around poles of the integrands.
\end{rem}
The integration constant $\beta_0$ appears in $h(z)$ as described in the 
following.
%%%%%%%%%%%%%%%%%%%%%%%%%%%%%
%%%%%%%% Corollary 2.3 %%%%%%%%
\begin{cor}\label{cor2.3}
The function $h(z)$ satisfies $zh(z)=h_0 \beta_0^2 + h_1(z) \beta_0 +h_2(z)+
O(z^{-\delta})$
with $h_1(z)=O(1),$ $h_2(z)=O(1)$ and 
$$
h_0= -\frac {24}{5\omega_{\mathbf{a}}^2} e^{2i\phi}(8e^{3i\phi}+27A_{\phi}
^2)^{-1} 
$$
in $\check{\mathcal{D}}_{\mathrm{cut}}(\phi,c,\varepsilon)$.  
\end{cor} 
%%%%%%%%%%%%%%%%%%%%%%
%%%%%%%%%%%%%%%%%%%%%%%%%%%%%
%%%%% Section 3 %%%%%%%%%%%%%
\section{System of equations and integral representations}\label{sc3}
By the change of variables $t=\tfrac 45 (e^{-i\phi}x)^{5/4},$
$y=(e^{-i\phi}x)^{1/2}v$, equation \eqref{1.1} is taken to
$$
t\frac d{dt}a_{\phi} +\frac 65 a_{\phi}= -\frac 85 e^{i\phi}v
$$
with
$$
a_{\phi}=e^{-2i\phi}\bigl(v_t+(2/5)t^{-1}v\bigr)^2 -4v^3-2e^{i\phi}v,
$$
which may be regarded as the Lagrangian or the Hamiltonian \cite[Theorem 4]
{K-3}. Setting $a_{\phi}=A_{\phi}+t^{-1}B(\phi,t)$ with $e^{i\phi}t=z,$
$e^{i\phi}B(\phi,t)=b,$ we have the system of equations 
%%%%%%%%%%%%%%%%%%%%%%%%%%%%%
%%%%%% (3.1) %%%%%%%%%%%%
\begin{equation}\label{3.1}
\begin{split}
&\bigl(v_z +(2/5) z^{-1}v\bigr)^2=4v^3+2 e^{i\phi}v+ A_{\phi}+z^{-1}b,
\\
&b_z= -\frac 85e^{i\phi}v -\frac 65 A_{\phi} -\frac 15 z^{-1}b.
\end{split}
\end{equation}
%%%%%%%%%%%%%%%%%%%%%%
The system
%%%%%%%%%%%%%%%%%%%%%%%%
%%%%%%%%% (3.2) %%%%%%%%%%%
\begin{equation}\label{3.2}
\begin{split}
&\mathfrak{p}_z^2 =4\mathfrak{p}^3+2e^{i\phi}\mathfrak{p} +A_{\phi},
\\
&\beta_z= -\frac 85e^{i\phi}\mathfrak{p} -\frac 65 A_{\phi}
\end{split}
\end{equation}
admits a solution $(\mathfrak{p},\beta)=(\mathfrak{p}(z),\beta(z))=(\wp(z-t_0;g_2(\phi),
g_3(\phi)), e^{i\phi}B_{\mathrm{as}}(\phi,t))$ \cite[Proposition 6]{K-3},
and is, at least formally, an approximation to \eqref{3.1}. 
%%%%%%%%%%%%%%%%%%%%%%%%%%%%%%%%%
%%%% Proposition 3.1 %%%%%%%%%%%
\begin{prop}\label{prop3.1}
For $y(x)$ given by \eqref{1.3} set $\mathfrak{p}(z+h(z))=(e^{-i\phi}x)
^{-1/2}y(x)$ with $e^{i\phi}t=z$ in $\check{\mathcal{D}}_{\mathrm{cut}}
(\phi,c,\varepsilon)$. Then $(v,b)=(\mathfrak{p}(z+h(z)), e^{i\phi}B(\phi,t))$ solves
\eqref{3.1}.
\end{prop}
By \eqref{1.3} and \cite[Theorem 4]{K-3} with
$b(z)-\beta(z)=e^{i\phi}(B(\phi,t)-B_{\mathrm{as}}(\phi,t))$, we have the
following.
%%%%%%%%%%%%%%%%%%%%%%%%%%%%%%%%%
%%%%%%% Proposition 3.2 %%%%%%%%%%%
\begin{prop}\label{prop3.2}
In $\check{\mathcal{D}}_{\mathrm{cut}}(\phi,c,\varepsilon)$,
$h(z) \ll z^{-\delta},$ and $b(z)$ is bounded and satisfies
$b(z)-\beta(z) \ll z^{-\delta}.$
\end{prop}
Recall that $P(\lambda)=4\lambda^3+2e^{i\phi}\lambda +A_{\phi}.$
Let us insert $v={\mathfrak{p}_{(h)}}(z):=\mathfrak{p}(z+h(z))$ into the first equation of 
\eqref{3.1}. Observing that $v_z=(1+h')\mathfrak{p}'(z+h)=(1+h')
\sqrt{P({\mathfrak{p}_{(h)}})},$ and that $(1+h')\sqrt{P({\mathfrak{p}_{(h)}})}+(2/5)z^{-1}
{\mathfrak{p}_{(h)}}=\sqrt{P({\mathfrak{p}_{(h)}})} (1+z^{-1}b P({\mathfrak{p}_{(h)}})^{-1})^{1/2},$
we have, in $\check{\mathcal{D}}(\phi,c,\varepsilon)$,
$$
h'=\Bigl( \frac{b}{2P({\mathfrak{p}_{(h)}})}
-\frac{2{\mathfrak{p}_{(h)}}}{5\sqrt{P({\mathfrak{p}_{(h)}})}}
\Bigr)z^{-1} - \frac{b^2}{8P({\mathfrak{p}_{(h)}})^2} z^{-2} +O(z^{-3}).
$$
Furthermore, by $\mathfrak{p}_{(h)}(z)=\mathfrak{p}(z)+h \mathfrak{p}'(z)+O(h^2),$
%%%%%%%%%%%%%%%% (3.3) %%%%%%%%%%%%%%
\begin{equation}\label{3.3}
h'=F(\mathfrak{p},b)z^{-1}-G(\mathfrak{p},b)z^{-2} +F_v(\mathfrak{p},b)\mathfrak{p}'hz^{-1} +O(z^{-1}(|z^{-1}|
+|h|)^2)
\end{equation}
with $F(v,b)=\tfrac 12 bP(v)^{-1}-\tfrac 25 vP(v)^{-1/2},$ $G(v,b)=\tfrac 18
b^2P(v)^{-2}$ as in Theorem \ref{thm2.2}.
Write $\chi:=b-\beta.$ Then the second equations of \eqref{3.1} and \eqref{3.2}
yield
\begin{align*}
\chi'=& -\frac 85 e^{i\phi}(\mathfrak{p}_{(h)}(z)-\mathfrak{p}(z)) -\frac{b(z)}5z^{-1}
\\
=& -\frac 85 e^{i\phi} \Bigl(\mathfrak{p}' h+\frac{\mathfrak{p}''}{2!}h^2 +\cdots + \frac
{\mathfrak{p}^{(m)}}{m!} h^m +E_m h^{m+1} \Bigr) -\frac{b}5z^{-1}
\end{align*}
for any positive integer $m$, where $E_m \ll 1$ in $\check{\mathcal{D}}(\phi,c,
\varepsilon).$
Let $\{z_n\} \subset \check{\mathcal{D}}(\phi,c,\varepsilon)$ be any sequence
such that $z_n \to\infty$. Then
\begin{align*}
\chi(z)-\chi(z_n) =& -\frac 85 e^{i\phi}\Bigl(\mathfrak{p} h +\frac{\mathfrak{p}'}2 h^2+
\cdots + \frac{\mathfrak{p}^{(m-1)}}{m!}h^m \Bigr) \Bigr]^z_{z_n}
\\
 +\frac 85 e^{i\phi}  \int^z_{z_n} &
\Bigl(\Bigl(\mathfrak{p} +{\mathfrak{p}'} h+
\cdots + \frac{\mathfrak{p}^{(m-1)}}{(m-1)!}h^{m-1} \Bigr)h'-E_mh^{m+1} \Bigr)d\zeta
 -\frac 15 \int^z_{z_n}(\beta+\chi)\frac{d\zeta}{\zeta}.
\end{align*}
By \eqref{3.3}, $h'=(F(\mathfrak{p},\beta) +\tfrac 12 \chi P(\mathfrak{p})^{-1}+\tilde{R}
 )z^{-1}$ with $\tilde{R} \ll |z^{-1}|+|h|,$ and hence the sum of the
integrals on the right-hand side is
$$
 \frac 15 \int^z_{z_n}  (8e^{i\phi} F(\mathfrak{p},\beta)\mathfrak{p}-\beta)\frac{d\zeta}{\zeta}
+\frac 15 \int^z_{z_n}R(h,\chi) \frac{d\zeta}{\zeta} -\frac 85 e^{i\phi}
\int^z_{z_n}E_m h^{m+1} d\zeta,
$$
with
\begin{align}
\notag
R(h,\chi):=& 8e^{i\phi}\Bigl(\mathfrak{p}'+\cdots+\frac{\mathfrak{p}^{(m-1)}}{(m-1)!}h^{m-2}
\Bigr) \Bigl(F(\mathfrak{p},\beta)+\frac{\chi}{2P(\mathfrak{p})}+\tilde{R} \Bigr)h
\\
%%%%%%%%%%%%%%%% (3.4) %%%%%%%%%%%%%%%
\label{3.4}
& +8e^{i\phi}\mathfrak{p}\Bigl(\frac {\chi}{2P(\mathfrak{p})}+\tilde{R} \Bigr)-\chi
\ll |h|+|\chi|+|\zeta^{-1}|.
\end{align}
Now suppose that, for a positive number $\mu$ satisfying $\delta\le \mu<1,$
%%%%%%%%%%%%%%% (3.5) %%%%%%%%%%%%%%%%
\begin{equation}\label{3.5}
h(z) \ll z^{-\mu}
\end{equation}
in $\check{\mathcal{D}}_{\mathrm{cut}}(\phi,c,\varepsilon).$ By Proposition 
\ref{prop3.2} this supposition is true for $\mu=\delta,$ and 
$\chi(z)\ll z^{-\delta}.$ Choose $m$ such that $\delta (m+1) \ge 3$. 
Under the passage to the limit $z_n\to \infty,$ we have
%%%%%%%%%%%%% (3.6) %%%%%%%%%%%%%
\begin{equation}\label{3.6}
\chi+\frac 85 e^{i\phi}\mathfrak{p} h =\frac 15 \int^z_{\infty} (8e^{i\phi}F(\mathfrak{p},\beta)
\mathfrak{p}-\beta)\frac{d\zeta}{\zeta}+\frac 15 \int^z_{\infty}R(h,\chi)\frac{d\zeta}
{\zeta} +O(z^{-2\mu}),
\end{equation}
in which the convergence of
$$
\int^z_{\infty} (8e^{i\phi}F(\mathfrak{p},\beta)\mathfrak{p}-\beta)\frac{d\zeta}{\zeta} \ll
z^{-\delta}
$$
is guaranteed by the absolute convergence of $\int^z_{\infty} R(h,\chi)
 \zeta^{-1} d\zeta \ll z^{-\delta}$ (cf. \eqref{3.4} and Proposition 
\ref{prop3.2}).
Under \eqref{3.5}, observing that, in \eqref{3.3},
$$
-G(\mathfrak{p},b)z^{-2} +F_v(\mathfrak{p},b)\mathfrak{p}'hz^{-1}=-G(\mathfrak{p},\beta)z^{-2}+F_v(\mathfrak{p},\beta)
\mathfrak{p}'hz^{-1}+O(z^{-1-\mu-\delta}),
$$
and that
\begin{align*}
\int^z_{z_n} F_v(\mathfrak{p},\beta)\mathfrak{p}' h\frac{d\zeta}{\zeta} =&
\int^z_{z_n} \Bigl(F(\mathfrak{p},\beta)'-\frac{\beta'}{2P(\mathfrak{p})} \Bigr)h\frac{d\zeta}
{\zeta} 
\\
=&F(\mathfrak{p},\beta)h \zeta^{-1} \Bigr]^z_{z_n} -\int^z_{z_n} 
 \Bigl(F(\mathfrak{p},\beta)h'+\frac{\beta'h}{2P(\mathfrak{p})} \Bigr)\frac{d\zeta}{\zeta}
+O(z^{-1-\mu})\Bigr]^z_{z_n}
\\
=& -\int^z_{z_n} 
 \Bigl(F(\mathfrak{p},\beta)^2\zeta^{-1}+\frac{\beta'h}{2P(\mathfrak{p})} \Bigr)\frac{d\zeta}
{\zeta}+O(z^{-1-\delta})\Bigr]^z_{z_n},
\end{align*}
we may apply a similar argument to \eqref{3.3}, and the convergence of 
$\int^z_{\infty} F(\mathfrak{p},\beta)\zeta^{-1}d\zeta$ follows. Thus we have
the following relations, in which the second equation follows from \eqref{3.6}
and \eqref{3.2}.
%%%%%%%%%%%%%%%%%%%%%%%%%%%%%%%%%%%%%%
%%%%%%%%%% Proposition 3.3 %%%%%%%%%%%%%
\begin{prop}\label{prop3.3}
Under the supposition \eqref{3.5}, in $\check{\mathcal{D}}_{\mathrm{cut}}
(\phi,c,\varepsilon)$,
\begin{align*}
h=& \int^z_{\infty} F(\mathfrak{p},\beta)\frac{d\zeta}{\zeta} -\int^z_{\infty}
(F(\mathfrak{p},\beta)^2+G(\mathfrak{p},\beta) )\frac{d\zeta}{\zeta^2} +\int^z_{\infty}
\frac{\chi-\beta'h}{2P(\mathfrak{p})} \frac{d\zeta}{\zeta} +O(z^{-\mu-\delta}), 
\\
\chi- &\beta'h= \frac 65A_{\phi}h +\frac 15 \int^z_{\infty} H(\mathfrak{p},\beta)\frac
{d\zeta}{\zeta} +\frac 15 \int^z_{\infty} R(h,\chi)\frac {d\zeta}{\zeta}
+O(z^{-2\mu}),
\end{align*}
in which $H(\mathfrak{p},\beta)=8e^{i\phi} F(\mathfrak{p},\beta)\mathfrak{p}-\beta,$ every integral
converges, and
\begin{align*}
&\int^z_{\infty} F(\mathfrak{p},\beta)\frac{d\zeta}{\zeta} \ll z^{-\delta}, \quad
\int^z_{\infty}( F(\mathfrak{p},\beta)^2 +G(\mathfrak{p},\beta)) \frac{d\zeta}{\zeta^2} 
\ll z^{-1}, \quad \int^z_{\infty}\frac{\chi-\beta'h}{2P(\mathfrak{p})}\frac{d\zeta}
{\zeta}\ll z^{-\delta},
\\
&\int^z_{\infty} H(\mathfrak{p},\beta)\frac{d\zeta}{\zeta} \ll z^{-\delta}, \quad
\int^z_{\infty} R(h,\chi) \frac{d\zeta}{\zeta} \ll z^{-\delta}. 
\end{align*}
\end{prop}
%%%%%%%%%%%%%%%%%%%%%%%%%
%%%%% Section 4 %%%%%%%%%%%%%%
\section{Proofs of the main results}\label{sc4}
By Proposition \ref{prop3.3},
$$
h-\frac 35A_{\phi}\int^z_{\infty} \frac h{P(\mathfrak{p})}\frac{d\zeta}{\zeta}
=\int^z_{\infty}F(\mathfrak{p},\beta)\frac{d\zeta}{\zeta} -\int^z_{\infty}(F(\mathfrak{p},\beta)
^2+G(\mathfrak{p},\beta))\frac{d\zeta}{\zeta^2} +I_1+O(z^{-\mu-\delta})
$$
with
$$
I_1=\frac 1{10}\int^z_{\infty}\frac 1{P(\mathfrak{p})} \int^{\zeta}_{\infty}(H(\mathfrak{p},
\beta)+R(h,\chi) )\frac{d\zeta_1}{\zeta_1} \frac{d\zeta}{\zeta}
$$
in $\check{\mathcal{D}}_{\mathrm{cut}}(\phi,c,\varepsilon)$ since $\mu\ge
\delta.$ Note that $\int^z_{\infty} P(\mathfrak{p})^{-1}\zeta^{-1}d\zeta \ll z^{-1}$ by
Lemma \ref{lem5.5}, and $\int^z_{\infty}H(\mathfrak{p},\beta)\zeta^{-1}d\zeta \ll
z^{-\delta}$, $\int^z_{\infty} R(h,\chi) \zeta^{-1}d\zeta \ll z^{-\delta}$ by
Proposition \ref{prop3.3} and \eqref{3.4}. Then integration by parts leads to
\begin{align*}
10 I_1 &= \int^z_{\infty}\frac 1{P(\mathfrak{p})} \frac{d\zeta}{\zeta} \int^z_{\infty}
(H(\mathfrak{p},\beta)+R(h,\chi))\frac{d\zeta}{\zeta} 
\\
&\phantom{---} -\int^z_{\infty}\int^{\zeta}
_{\infty}\frac 1{P(\mathfrak{p})} \frac{d\zeta_1}{\zeta_1}(H(\mathfrak{p},\beta)+R(h,\chi))
\frac{d\zeta}{\zeta}
\\
&= -\int^z_{\infty} H(\mathfrak{p},\beta)\int^{\zeta}_{\infty}\frac 1{P(\mathfrak{p})}\frac
{d\zeta_1}{\zeta_1} \frac{d\zeta}{\zeta}+O(z^{-1-\delta}).
\end{align*}
Furthermore, by \eqref{3.3}, \eqref{3.5}, Lemma \ref{lem5.5} and Proposition
\ref{prop3.2},
\begin{align*}
\int^z_{\infty}\frac{h}{P(\mathfrak{p})} \frac{d\zeta}{\zeta} &= h\int^z_{\infty}
\frac 1{P(\mathfrak{p})}\frac{\d\zeta}{\zeta}-\int^z_{\infty} h'\int^{\zeta}_{\infty}
\frac 1{P(\mathfrak{p})} \frac{d\zeta_1}{\zeta_1} d\zeta
\\
&= -\int^z_{\infty}\Bigl(F(\mathfrak{p},\beta)+\frac{\chi}{2P(\mathfrak{p})}\Bigr) \int^{\zeta}
_{\infty} \frac 1{P(\mathfrak{p})}\frac{d\zeta_1}{\zeta_1}\frac{d\zeta}{\zeta}
+O(z^{-1-\mu})
\\
&= -\int^z_{\infty}F(\mathfrak{p},\beta) \int^{\zeta}_{\infty}\frac 1{P(\mathfrak{p})} 
\frac{d\zeta_1}{\zeta_1}\frac{d\zeta}{\zeta} +O(z^{-1-\delta}). 
\end{align*}
Thus we obtain, under supposition \eqref{3.5},
%%%%%%%%%%%%% (4.1) %%%%%%%%%%%%
\begin{align}\notag
h=\int^z_{\infty}& F(\mathfrak{p},\beta)\frac{d\zeta}{\zeta} -\int^z_{\infty}(F(\mathfrak{p},\beta)
^2+G(\mathfrak{p},\beta)) \frac{d\zeta}{\zeta^2}
\\
\label{4.1}
& -\frac 1{10}\int^z_{\infty}K(\mathfrak{p},\beta)
\mathcal{I}(\zeta)\frac{d\zeta}{\zeta} +O(z^{-\mu-\delta})
\end{align}
in $\check{\mathcal{D}}_{\mathrm{cut}}(\phi,c,\varepsilon),$ in which the
implied constant possibly depends on $\mu$, and $K(\mathfrak{p},\beta)=H(\mathfrak{p},\beta)
+6A_{\phi}F(\mathfrak{p},\beta)=2(4e^{i\phi}\mathfrak{p}+3A_{\phi})F(\mathfrak{p},\beta)-\beta,$
$\mathcal{I}(z)=\int^z_{\infty} P(\mathfrak{p})^{-1}\zeta^{-1}d\zeta.$  
The integrals on the right-hand side of \eqref{4.1} satisfy
%%%%%%%%%%%%% (4.2) %%%%%%%%%%%%%%%%%%%%
\begin{equation}\label{4.2}
\begin{split}
&\int^z_{\infty}F(\mathfrak{p},\beta)\frac{d\zeta}{\zeta} \ll z^{-1}, \quad
\int^z_{\infty}(F(\mathfrak{p},\beta)^2 +G(\mathfrak{p},\beta))\frac{d\zeta}{\zeta^2}
 \ll z^{-1}, 
\\
&\int^z_{\infty}K(\mathfrak{p},\beta)\mathcal{I}(\zeta)\frac{d\zeta}{\zeta}
 \ll z^{-1}, \quad \mathcal{I}(z)\ll z^{-1}.
\end{split}
\end{equation}
The first estimate follows from Lemmas \ref{lem5.5} and \ref{lem5.4} with 
$\beta(z)=\tfrac 85 e^{i\phi}\omega_{\mathbf{a}}^{-1} (\beta_0 +
\mathfrak{b}(z-t_0))$ (cf. \eqref{5.6}).
%%%%%%%%%%%%%%%%%%% 4.1 %%%%%%%%%%%%%%%%%%%%%%%%%%%%%
\subsection{Derivation of Theorem \ref{thm2.2} and Corollary \ref{cor2.3}}
\label{ssc4.1}
%%%%%%%%%%%%%%%%%%%%%%%%%%%%%%%%%%%%%%%%%%%%%%%%%%%%%%%%%%%%%%%
By \eqref{4.2} the sum of the integrals on the right-hand side of \eqref{4.1}
is $O(z^{-1})$ in $\check{\mathcal{D}}_{\mathrm{cut}}(\phi,c,\varepsilon)$.
Note that \eqref{3.5} is valid for $\mu=\delta,$ i.e. $h(z)\ll z^{-\delta}.$
Then, by \eqref{4.1} with $\mu=\delta,$ we have $h(z)\ll |z^{-1}|
+|z^{-2\delta}|.$ If $2\delta <1,$ then \eqref{3.5} is valid for $\mu=2\delta,$
which implies asymptotic formula \eqref{4.1} with the error term 
$O(z^{-3\delta})$ and $h(z) \ll |z^{-1}|+|z^{-3\delta}|.$ 
For $m_0$ such that $m_0\delta <1 \le (m_0+1)\delta,$ $m_0$-times
repetition of this procedure results in
\eqref{4.1} with $\mu=m_0\delta$, which yields \eqref{3.5} with $\mu=1.$  
Then \eqref{4.1} with $\mu=1$ is valid in 
$\check{\mathcal{D}}_{\mathrm{cut}}(\phi,c,\varepsilon)$, implying 
Theorem \ref{thm2.2}. Calculation of
the coefficient of $\beta_0^2$ by the use of Lemma \ref{lem5.5} leads us to
Corollary \ref{cor2.3}.
%%%%%%%%%%%%%%%%%%% 4.2 %%%%%%%%%%%%%%%
\subsection{Derivation of Theorem \ref{thm2.1}}\label{ssc4.2}
%%%%%%%%%%%%%%%%%%%%%%%%%%%%%%%%%%%%%%%%%%%%%%%%%%%%%%%%%%%%%%%%
Theorem \ref{thm2.2} implies $h(z)\ll z^{-1}$ in $\check{\mathcal{D}}_{\mathrm
{cut}}(\phi,c,\varepsilon)$. Note that
\begin{align*}
& (e^{-i\phi}x)^{-1/2}y(x)-\wp(e^{i\phi}t-t_0; g_2(\phi),g_3(\phi))
\\
=& \wp(e^{i\phi}t-t_0 +h(e^{i\phi}t); g_2(\phi),g_3(\phi))
-\wp(e^{i\phi}t-t_0; g_2(\phi),g_3(\phi))
\\
\ll & \wp'(e^{i\phi}t-t_0; g_2(\phi),g_3(\phi)) h(e^{i\phi}t) \ll t^{-1}
\end{align*}
in $\check{\mathcal{D}}_{\mathrm{cut}}(\phi,c,\varepsilon)$, and that the
function on the first line is holomorphic in  
${\mathcal{D}}(\phi,c,\varepsilon)$. By using the maximal modulus principle
in excluded discs around $t_j^{m,n}$ $(1\le j \le 3;\, m,n\in\mathbb{Z}),$ 
we obtain Theorem \ref{thm2.1}.
%%%%%%%%%%%%%%%%%%%%%%%%%%%%%%%%%%%%%%%%%%%%%%%%%%%%%%%
%%%%%%%%% Section 5 %%%%%%%%%%%%%%%%%%
\section{Lemmas on primitive functions}\label{sc5}
%%%%%%%%%%%%%%%%%%%%%%%%%
It remains to show some lemmas on primitive functions used in the proofs of
the main results. Recall that $P(\lambda)=4w(A_{\phi},\lambda)^2$ as in
Theorem \ref{thm2.2}, and let us write
\begin{align*}
P(\lambda)&=4\lambda^3-g_2\lambda -g_3 =4(\lambda-\lambda_1)(\lambda-\lambda_2)
(\lambda-\lambda_3), \quad g_2=-2e^{i\phi}, \quad g_3=-A_{\phi},
\\
&\lambda_1=\wp(\omega_1),\quad  \lambda_2=\wp(\omega_2),\quad 
\lambda_3=\wp(\omega_3), 
\end{align*}
where 
$$
\omega_1=\frac 12\omega_{\mathbf{a}},\quad \omega_3=\frac 12
\omega_{\mathbf{b}}, \quad \omega_1+\omega_2+\omega_3=0.
$$
Then
\begin{align*}
&\lambda_1+\lambda_2+\lambda_3=0, \quad \lambda_2\lambda_3+\lambda_3\lambda_1
+\lambda_1\lambda_2=-g_2/4, \quad \lambda_1\lambda_2\lambda_3=g_3/4,
\\
& (\lambda_2-\lambda_3)^2(\lambda_3-\lambda_1)^2(\lambda_1-\lambda_2)^2
=\frac 1{16}(g_2^3-27g_3^2),
\end{align*}
and $\omega_{\mathbf{a}}$ and $J_{\mathbf{a}}$ are also written in the form
$$
\omega_{\mathbf{a}}=\int_{\mathbf{a}}
 \frac{d\lambda}{\sqrt{P(\lambda)}}, \quad
J_{\mathbf{a}}=\int_{\mathbf{a}}
 \sqrt{P(\lambda)} d\lambda.
$$
%%%%%%%%%%%%%%%%%%%%%%%%%%%%%%%%%%%%%%%%%
%%%%%%%%% Lemma 5.1 %%%%%%%%%%%%%%%%%%%
\begin{lem}\label{lem5.1}
Let 
\begin{align*}
&\gamma_1=(\lambda_1-\lambda_2)(\lambda_1-\lambda_3), \quad
\gamma_2=(\lambda_2-\lambda_3)(\lambda_2-\lambda_1), \quad
\gamma_3=(\lambda_3-\lambda_1)(\lambda_3-\lambda_2), 
\\
&\gamma_0=5(g_2^3-27g_3^2)^{-1}
\end{align*}
and $\wp(z)=\wp(z;g_2, g_3).$ Then
\begin{align*}
\int^z_0  \frac{dz}{P(\wp(z))} =& -\frac 1{4\omega_{\mathbf{a}}} \Bigl(
\gamma_1^{-2}\Bigl(\frac{5J_{\mathbf{a}}}{2g_2}z+\frac{\vartheta'}{\vartheta}
\Bigl(\frac z{\omega_{\mathbf{a}}} +\frac{\tau}2,\tau\Bigr)\Bigr)
+\gamma_2^{-2}\Bigl(\frac{5J_{\mathbf{a}}}{2g_2}z+\frac{\vartheta'}{\vartheta}
\Bigl(\frac z{\omega_{\mathbf{a}}} ,\tau\Bigr)\Bigr)
\\
&+\gamma_3^{-2}\Bigl(\frac{5J_{\mathbf{a}}}{2g_2}z+\frac{\vartheta'}{\vartheta}
\Bigl(\frac z{\omega_{\mathbf{a}}} +\frac{1}2,\tau\Bigr)\Bigr) \Bigr) +C_0,
\\
\int^z_0  \frac{dz}{P(\wp(z))^2} =& \gamma_0z 
-\frac 1{96\omega_{\mathbf{a}}} \Bigl(
\gamma_1^{-4} (\partial_z^2 -48\lambda_1)
\Bigl(\frac{5J_{\mathbf{a}}}{2g_2}z+\frac{\vartheta'}{\vartheta}
\Bigl(\frac z{\omega_{\mathbf{a}}} +\frac{\tau}2,\tau\Bigr)\Bigr)
\\
&+\gamma_2^{-4} (\partial_z^2 -48\lambda_2)
\Bigl(\frac{5J_{\mathbf{a}}}{2g_2}z+\frac{\vartheta'}{\vartheta}
\Bigl(\frac z{\omega_{\mathbf{a}}} ,\tau\Bigr)\Bigr)
\\
&+\gamma_3^{-4} (\partial_z^2 -48\lambda_3)
\Bigl(\frac{5J_{\mathbf{a}}}{2g_2}z+\frac{\vartheta'}{\vartheta}
\Bigl(\frac z{\omega_{\mathbf{a}}} +\frac{1}2,\tau\Bigr)\Bigr) \Bigr) +C_1,
\end{align*}
where $C_0$ and $C_1$ are some constants and $\partial_z=d/dz$.
\end{lem}
%%%%%%%%%%%%%%%%%%%%
\begin{proof}
Around $\omega_j$ $(j=1,2,3)$,
$$
\frac 1{P(\wp(z))}=\frac 14\gamma_j^{-2}(z-\omega_j)^{-2}(1+O(z-\omega_j)^2)
$$
since $\wp(z)=\lambda_j+\gamma_j(z-\omega_j)^2+O(z-\omega_j)^4.$ Hence
%%%%%%%%% (5.1) %%%%%%%%%%%%%%%%%
\begin{equation}\label{5.1}
\frac 1{P(\wp(z))}-\frac 14(\gamma_1^{-2}\wp(z-\omega_1)+
\gamma_2^{-2}\wp(z-\omega_2)+ \gamma_3^{-2}\wp(z-\omega_3) )\equiv \Gamma_0,
\end{equation}
in which, by putting $z=0$, we find $\Gamma_0= 9g_3 (g_2^3-27g_3^2)^{-1}.$
We may set, for some $c_0$,
%%%%%%%%%%% (5.2) %%%%%%%%%%%%%%%%%
\begin{equation}\label{5.2}
\wp(z)+\frac 1{\omega_{\mathbf{a}}}\partial_z \frac{\vartheta'}{\vartheta}
\Bigl(\frac z{\omega_{\mathbf{a}}}+\nu, \tau\Bigr)\equiv c_0,
\end{equation}
integration of which yields
$$
\omega_{\mathbf{a}} c_0 =c_0 \int^{\omega_1}_{-\omega_1}dz
= \int^{\omega_1}_{-\omega_1}\wp(z) dz =\int_{\mathbf{a}}\frac{\lambda}
{\sqrt{P(\lambda)}}d\lambda. 
$$
Observing that
\begin{align*}
J_{\mathbf{a}}= &\int_{\mathbf{a}}\sqrt{P(\lambda)}d\lambda
=\int_{\mathbf{a}}\Bigl(\frac 23\lambda\, \bigl(\!\sqrt{P(\lambda)}\bigr)' 
-\frac 23 
\frac{g_2\lambda}{\sqrt{P(\lambda)}} -\frac{g_3}{\sqrt{P(\lambda)}}\Bigr)d\lambda
\\
=& -\frac 23 J_{\mathbf{a}}-\frac 23 g_2 \int_{\mathbf{a}} \frac{\lambda}
{\sqrt{P(\lambda)}} d\lambda -g_3 \omega_{\mathbf{a}},
\end{align*}
we have $c_0=-5J_{\mathbf{a}}/(2g_2\omega_{\mathbf{a}})-3g_3/(2g_2).$
Inserting \eqref{5.2} with $z\mapsto z-\omega_j$ $(j=1,2,3)$ into \eqref{5.1}, 
and using
$$
\Gamma_0-\frac{3g_3}{8g_2}(\gamma_1^{-2}+\gamma_2^{-2}+\gamma_3^{-2})=0,
$$
we obtain the first primitive function.
To derive the second formula, we note that, around $z=\omega_j$ $(j=1,2,3)$,
\begin{align*}
\frac{16}{P(\wp(z))^2}=&
\Bigl(\gamma_j^{-2}\wp(z-\omega_j)-4\gamma_j^{-2}\lambda_j+O(z-\omega_j)^2
\Bigr)^2 
\\
=&\gamma_j^{-4}(\wp(z-\omega_j)^2-8\lambda_j\wp(z-\omega_j)+O(1))
\end{align*}
by \eqref{5.1}, since, say around $z=\omega_1$, 
$\wp(z-\omega_2)=\wp(\omega_3)+O(z-\omega_1)^2,$ 
$\wp(z-\omega_3)=\wp(\omega_2)+O(z-\omega_1)^2,$ with $\gamma_2^{-2}\lambda_3
+\gamma_3^{-2}\lambda_2+4\Gamma_0= -4\gamma_1^{-2}\lambda_1.$  
Then we set
%%%%%%%%%%%%%% (5.3) %%%%%%%%%%%%%%%%%
\begin{align}\notag
\frac{16}{P(\wp(z))^2} &-\gamma_1^{-4} (\wp(z-\omega_1)^2-8\lambda_1
\wp(z-\omega_1) )  -\gamma_2^{-4} (\wp(z-\omega_2)^2-8\lambda_2\wp(z-\omega_2) )
\\
\label{5.3}
 &-\gamma_3^{-4} (\wp(z-\omega_3)^2-8\lambda_3\wp(z-\omega_3) )\equiv \Gamma_1
\end{align}
with
$$
\Gamma_1=\frac{7((\lambda_2-\lambda_3)^4\lambda_1^2 +
(\lambda_3-\lambda_1)^4\lambda_2^2+(\lambda_1-\lambda_2)^4\lambda_3^2)}
{(\lambda_2-\lambda_3)^{4}(\lambda_3-\lambda_1)^{4}
(\lambda_1-\lambda_2)^{4}}.
$$
Insertion of \eqref{5.2} and
$$
\wp(z)^2+ \frac {\partial_z^3}{6\omega_{\mathbf{a}}}  \frac{\vartheta'}
{\vartheta}\Bigl(\frac{z}{\omega_{\mathbf{a}}}+\nu, \tau \Bigr)\equiv
\frac{g_2 }{12}
$$
with $z \mapsto z-\omega_j$ $(j=1,2,3)$ into \eqref{5.3} leads to the second
formula.
\end{proof}
%%%%%%%%%%%%%%%%%%%%%%%%%%%%%%%%%%%%%%%
%%%%%%%%%%%% Lemma 5.2 %%%%%%%%%%%%%%%%
\begin{lem}\label{lem5.2}
Under the same supposition as in Lemma \ref{lem5.1}, for $k=0,1,2$,
\begin{align*}
\int^z_{z_0} \frac{\wp(z)^k\wp'(z)}{P(\wp(z))} dz =&\frac 14 \Bigl(
\gamma_1^{-1}\lambda_1^k \log(\wp(z)-\lambda_1)+
\\
& \gamma_2^{-1}\lambda_2^k \log(\wp(z)-\lambda_2)+
\gamma_3^{-1}\lambda_3^k \log(\wp(z)-\lambda_3)\Bigr) +C(z_0),
\end{align*}
where $z_0\not\in \{0,\omega_1,\omega_2,\omega_3\}+\omega_{\mathbf{a}}\mathbb{Z}
+\omega_{\mathbf{b}}\mathbb{Z},$ and $C(z_0)$ is some constant.
\end{lem}
%%%%%%%%%%%%%%%%%%%%%%%%%%%%
\begin{proof}
Note that, for $k=0,1,2,$
$$
\frac{4\wp^k}{P(\wp)}=\gamma_1^{-1}\lambda_1^k(\wp-\lambda_1)^{-1}+
\gamma_2^{-1}\lambda_2^k(\wp-\lambda_2)^{-1}+
\gamma_3^{-1}\lambda_3^k(\wp-\lambda_3)^{-1},
$$
from which the lemma follows.
\end{proof}
%%%%%%%%%%%%%%%%%%%%%%%
Recall that $\mathfrak{p}(z)=\wp(z-t_0)$ and
%%%%%%% (5.6) %%%%%%%%%%%%%
\begin{equation}\label{5.6}
\beta(z)=\frac{8e^{i\phi}}{5\omega_{\mathbf{a}}}(\beta_0+\mathfrak{b}(z-t_0)),
\quad \mathfrak{b}(\sigma)=-\frac 54 e^{-i\phi}J_{\mathbf{a}}\sigma 
+\frac{\vartheta'}{\vartheta}\Bigl(\frac{\sigma}{\omega_{\mathbf{a}}} +\nu,
\tau\Bigr).
\end{equation}
Then we have the following.
%%%%%%%%%%%%%%%%%%%%%%%%%%%%%%%%
%%%%% Lemma 5.3 %%%%%%%%%%%%%%%%%%%
\begin{lem}\label{lem5.3}
Let $\alpha_0 \in \mathbb{C}$ be a given number. Then
$$
\int^s_{\infty} (\mathfrak{b}(\sigma-\alpha_0)+\mathfrak{b}(\sigma+\alpha_0)
)_{\sigma}\mathfrak{b}(\sigma) \frac{d\sigma}{\tilde{\sigma}} \ll s^{-1},
\quad \tilde{\sigma}=\sigma+t_0
$$
as $x=e^{i\phi} (\tfrac 54e^{-i\phi}s)^{4/5} \to \infty$ through $\check
{\mathcal{D}}_{\mathrm{cut}}(\phi,c,\varepsilon)$, in which the integral on
the left-hand side is convergent.
\end{lem}
%%%%%%%%%%%%%%%%%%%%%%%%%%%
\begin{proof}
Let $x_{\nu}=e^{i\phi}(\tfrac 54 e^{-i\phi}s_{\nu})^{4/5}$ be any sequence
tending to $\infty$ through $\check{\mathcal{D}}_{\mathrm{cut}}(\phi,c,
\varepsilon)$. Note that $\mathfrak{b}(s)$ and $\mathfrak{b}(s\pm \alpha_0)$
are bounded in $\check{\mathcal{D}}_{\mathrm{cut}}(\phi,c,\varepsilon)$.
Integration by parts leads to
\begin{align*}
\int^s_{s_{\nu}}\mathfrak{b}(\sigma-\alpha_0)_{\sigma} 
\mathfrak{b}(\sigma)\frac{d\sigma}{\tilde{\sigma}}
=& \mathfrak{b}(\sigma-\alpha_0)\mathfrak{b}(\sigma)\tilde{\sigma}^{-1}
\Bigr]^s_{s_{\nu}}
\\
&-\int^s_{s_{\nu}} \mathfrak{b}(\sigma-\alpha_0)\mathfrak{b}_{\sigma}(\sigma)
\frac{d\sigma}{\tilde{\sigma}}
+\int^s_{s_{\nu}} \mathfrak{b}(\sigma-\alpha_0)\mathfrak{b}(\sigma)
\frac{d\sigma}{\tilde{\sigma}^2}
\\
=&-\int^{s-\alpha_0}_{s_{\nu}-\alpha_0} \mathfrak{b}(\sigma)
\mathfrak{b}_{\sigma}(\sigma+\alpha_0)
\frac{d\sigma}{\tilde{\sigma}+\alpha_0} +O(s^{-1})+O(s_{\nu}^{-1})
\\
=&-\int^{s}_{s_{\nu}}
\mathfrak{b}(\sigma+\alpha_0)_{\sigma} \mathfrak{b}(\sigma)
\frac{d\sigma}{\tilde{\sigma}} +O(s^{-1})+O(s_{\nu}^{-1}).
\end{align*}
Under the passage to the limit $s_{\nu} \to \infty$, we obtain the lemma.
\end{proof}
By Lemma \ref{lem5.1} with $g_2=-2e^{i\phi},$ we may write
\begin{align*}
\frac 1{P(\wp(z))} =& -\frac 1{4\omega_{\mathbf{a}}} \frac d{dz}
\Bigl( \frac{\gamma_1^{-2}}2 (\mathfrak{b}(z-\omega_{\mathbf{a}}/2)
 +\mathfrak{b}(z+\omega_{\mathbf{a}}/2))
\\
&+ \frac{\gamma_2^{-2}}2 (\mathfrak{b}(z-\omega_{\mathbf{a}}\nu)
 +\mathfrak{b}(z+\omega_{\mathbf{a}}\nu))
+ \frac{\gamma_3^{-2}}2 (\mathfrak{b}(z-\omega_{\mathbf{a}}\tau/2)
 +\mathfrak{b}(z+\omega_{\mathbf{a}}\tau/2)) \Bigr).
\end{align*}
Substituting $z$ by $z-t_0$ and using Lemma 5.3 we have the following.
%%%%%%%%%%% Lemma 5.4 %%%%%%%%%
\begin{lem}\label{lem5.4}
In $\check{\mathcal{D}}_{\mathrm{cut}}(\phi,c,\varepsilon)$,
$$
\int^z_{\infty} \frac{\mathfrak{b}(\zeta-t_0)}{P(\mathfrak{p}(\zeta))} \frac{d\zeta}
{\zeta} \ll z^{-1},
$$
in which the integral on the left-hand is convergent.
\end{lem}
%%%%%%%%%%% Lemma 5.5 %%%%%%%%%
\begin{lem}\label{lem5.5}
In $\check{\mathcal{D}}_{\mathrm{cut}}(\phi,c,\varepsilon)$,
$$
\int^z_{\infty} \frac 1{P(\mathfrak{p}(\zeta))} \frac{d\zeta}{\zeta} \ll z^{-1},
\,\,\,
\int^z_{\infty} \frac 1{P(\mathfrak{p}(\zeta))^2} \frac{d\zeta}{\zeta^2} =-\gamma_0 z^{-1}
+O(z^{-2}),
\,\,\,
\int^z_{\infty} \frac{\mathfrak{p}(\zeta)\mathfrak{p}'(\zeta)}{P(\mathfrak{p}(\zeta))} \frac{d\zeta}
{{\zeta}} \ll z^{-1},
$$
where $\gamma_0=-5(8e^{3i\phi}+27A_{\phi}^2)^{-1}.$
\end{lem}
\begin{proof}
Let $\gamma_0z+F(z)$ be the primitive function of $1/P(\wp(z))^2$ given in
Lemma \ref{lem5.1}, where $F(z)$ is bounded in $\check{\mathcal{D}}_{\mathrm
{cut}}(\phi,c,\varepsilon)$. Then
\begin{align*}
\int^z_{\infty} \frac 1{P(\mathfrak{p}(\zeta))^2} \frac{d\zeta}{{\zeta}^2}
=& (F(\zeta-t_0)+\gamma_0\zeta)\frac 1{{\zeta}^{2}} \Bigl]^z_{\infty}
+2\int^z_{\infty}(F(\zeta-t_0)+\gamma_0\zeta) \frac{d\zeta}{{\zeta}^3}
\\
=& -\gamma_0 z^{-1}+ O(z^{-2}),
\end{align*}
which is the second integral. The remaining estimates are similarly obtained
by the use of Lemmas \ref{lem5.1} and \ref{5.2}.
\end{proof}


%%%%%%%%%%%%%%%%%%%%%%%%%%%%%%%%%%%%%%%%%%%%%%%%%%%%%%%%%%%%%%%%%%%%%%%% 
%%%%% References %%%%%%%%%%%%%%%%%%%%%%%%%%%%%%%%%%%%%%%%%%%%%%%%%%%
\begin{thebibliography}{99}


\bibitem{Boutroux} 
P.~{\sc Boutroux}, 
Recherches sur les transcendantes de M.~Painlev\'e et l'etude asymptotique des 
e\'quations diff\'erentielles du second ordre, Ann.\ Sci.\ \'Ecole Norm.\ Sup. 
(3) {\bf 30} (1913), 255--375.


\bibitem{FIKN}
A.~S.~{\sc Fokas}, A.~R.~{\sc Its}, A.~A.~{\sc Kapaev} and
V.~Yu.~{\sc Novokshenov},
{\it Painlev\'e Transcendents, The Riemann-Hilbert Approach},
Math.\ Surveys and Monographs 128, AMS Providence, 2006.

\bibitem{H}
A.~{\sc Hurwitz} and R.~{\sc Courant},
{\it Vorlesungen \"{u}ber allgemeine Funktionentheorie und elliptische 
Funktionen}, Berlin, Springer, 1922. 

\bibitem{I-K}
A.~R.~{\sc Its} and A.~A.~{\sc Kapaev},  
The nonlinear steepest descent approach to the asymptotics of the second 
Painlev'e transcendent in the complex domain,
{\it MathPhys odyssey}, 2001, 273--311, Prog. Math. Phys., 23, Birkhauser 
Boston, Boston, MA, 2002. 

\bibitem{Iwaki}
K.~{\sc Iwaki}, 
2-parameter $\tau$-function for the first Painlev\'e equation: topological 
recursion and direct monodromy problem via exact WKB analysis, Comm.\ Math.\ 
Phys. {\bf 377} (2020), 1047--1098.

\bibitem{JM}
M.~{\sc Jimbo} and T.~{\sc Miwa}, Monodromy preserving deformation 
of linear ordinary 
differential equations with rational coefficients. II, Phys.\ D
{\bf 2} (1981), 407--448.  

\bibitem{J-K}
N.~{\sc Joshi} and M.~D.~{\sc Kruskal}, 
An asymptotic approach to the connection problem for the first and the second 
Painlev\'e equations, 
Phys.~Lett. A {\bf 130} (1988), 129--137. 

\bibitem{Ka}
A.~A.~{\sc Kapaev}, 
Asymptotic behaviour of the solutions of the Painlev\'e equation of the first
kind. (Russian) Differentsial'nye Uravneniya {\bf 24} (1988), 1684--1695;
translation in Differential Equations {\bf 24} (1988), 1107--1115. 
 

%\bibitem{Ka}
%A.~A.~{\sc Kapaev}, 
%Essential singularity of the Painlev\'e function of the 
% 5second kind and the nonlinear Stokes phenomenon. (Russian) 
%Zap.~Nauchn.~Sem.~Leningrad.~Otdel.~Mat.~Inst.~Steklov. (LOMI) {\bf 187} 
%(1991), Differentsial'naya Geom.~Gruppy Li i Mekh. 12, 139--170, 173, 176; 
%translation in J.~Math.~Sci. {\bf 73} (1995), 500--517. 

\bibitem{Ka-Ki}
A.~A.~{\sc Kapaev} and A.~V.~{\sc Kitaev}, 
Connection formulae for the first Painlev\'e transcendent in the complex domain,
Lett.~Math.~Phys. {\bf 27} (1993), 243--252. 


%% \bibitem{K}
%% A.~V.~{\sc Kitaev}, 
%% The justification of asymptotic formulas that can be obtained by the method 
%% of isomonodromic deformations, (Russian) 
%% Zap.~Nauchn.~Sem.~Leningrad.~Otdel.~ Mat.~Inst.~Steklov. (LOMI) {\bf 179} 
%% (1989), Mat.~Vopr.~Teor.~Rasprostr.~Voln. 19, 101--109, 189--190; 
%% translation in J.~Soviet Math. {\bf 57} (1991), no. 3, 3131--3135. 


\bibitem{K1}
A.~V.~{\sc Kitaev}, 
The isomonodromy technique and the elliptic asymptotics of the first 
Painlev\'e transcendent, (Russian) Algebra i Analiz {\bf 5} (1993), 179--211; 
translation in St.~Petersburg Math.~J. {\bf 5} (1994), 577--605. 

\bibitem{K-3}
A.~V.~{\sc Kitaev}, 
Elliptic asymptotics of the first and second Painlev\'e transcendents, 
(Russian) Uspekhi Mat.~Nauk {\bf 49} (1994), 77--140; 
translation in Russian Math.~Surveys {\bf 49} (1994), 81--150. 

\bibitem{Novo}
V.~Yu.~{\sc Novokshenov}, The Boutroux ansatz for the second Painlev\'e 
equation in the complex domain, (Russian) Izv.~Akad.~Nauk SSSR Ser.~Mat. 
{\bf 54} (1990), 1229--1251; translation in Math.~USSR-Izv. {\bf 37} (1991), 
587--609. 

%% \bibitem{S}
%% S.~{\sc Shimomura}, Elliptic asymptotic representation of the fifth
%% Painlev\'e transcendents, Kyushu J. Math. 76 (2022), 43--99.

%% \bibitem{S1}
%% S.~{\sc Shimomura}, Corrigendum: Elliptic asymptotic representation of the 
%% fifth Painlev\'e transcendents, Kyushu J. Math. 77 (2023), 191--202.

\bibitem{WW}
E.~T.~{\sc Whittaker} and G.~N.~{\sc Watson},  
{\it A Course of Modern Analysisi}, Reprint of the fourth (1927) edition. 
Cambridge Mathematical Library. Cambridge University Press, Cambridge, 1996. 
\end{thebibliography}
%%%%%%%%%%%%%%%%%%%%%%%%%%%%%%%%%%%%
%%%%%%%%%%%%%%%%%%%%%%%%%%%%%%%
%%%%%%%%%%%%%%%%%%%%%%%%%%

\end{document}



