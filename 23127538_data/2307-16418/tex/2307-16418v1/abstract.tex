Digital images are vulnerable to nefarious manipulations. Passive image manipulation localization methods often lack robustness against post-processing operations, e.g., blurring, and generalizability to the continuously evolving adversaries. Proactive methods, on the other hand, show promising results in forged image classification via watermarking-based image protection. Nevertheless, the large embedding capacity required by robust forgery localization can easily compromise the visual quality of the image. This paper explores a novel image protection scheme that embeds the imperceptible protective signal in the RAW domain and blindly transfers it into the rendered RGB images regardless of the image signal processing pipeline. We find that this paradigm shift brings nontrivial benefits compared to direct RGB-domain protection in view of adaptive embedding and artifact reduction. Once the protected RGB images are manipulated, the localization networks can accurately and robustly identify the forged areas. Furthermore, for cellphones or cameras with limited computing resources, we propose a lightweight Multi-frequency Partial Fusion Network (MPF-Net) with frequency learning and partial feature fusion, which enables protection to be incorporated into images immediately after it is captured. Extensive experiments on famous RAW datasets, e.g., RAISE, FiveK and SIDD, prove the effectiveness of our method.

%%%%%% merged version by ChatGPT
% Digital images are vulnerable to malicious manipulations, which can compromise their authenticity and integrity. Passive image manipulation localization methods often lack robustness against post-processing operations and are limited in their generalizability to evolving adversaries. In this paper, we propose a proactive approach for protecting images from manipulation by introducing an invisible watermark into the RAW data at the moment of capture. We design a lightweight Multi-frequency Partial Fusion Network (MPF-Net) with frequency learning and partial feature fusion to embed the protective signal, which can be transferred into the rendered RGB images, even when different ISP algorithms are used. The proposed method accurately identifies manipulated areas using a dedicated localization network and is robust against post-processing operations such as blurring or recompression. Extensive experiments on popular RAW datasets (e.g., RAISE, FiveK, and SIDD) demonstrate the effectiveness of our approach. Our technique can potentially be used as an option for image protection in future cameras.

%%%%%% original version
% Digital images are vulnerable to nefarious manipulation attacks, which may not be easily detected since many image lossy operations like compression and rescaling would destroy the manipulation traces.
% Passive image manipulation localization algorithms rely on capturing traces vulnerable to various distortions, e.g. blurring.
% Images are vulnerable to nefarious manipulations where forensic traces can be easily wiped out.
% The explosive progress of Deepfake techniques poses unprecedented privacy and security risks to our society by creating real-looking but fake visual content
% The proliferation of image editing software 
% % has led to the rampant manipulation of images, 
% causes the authenticity of digital images doubtful.
% makes it critical to develop methods to localize manipulated areas within images.
% Detecting manipulated images is critical owing to the abuse of image editing softwares.
% and the fragility of forensics traces against lossy image operations.
% causes the authenticity of digital images questionable.
% Various lossy operations, e.g., compression and resizing can wipe out forensic traces. 
% for manipulation localization within forged images.
Digital images are vulnerable to nefarious manipulations.
Passive image manipulation localization methods often lack robustness against post-processing operations, e.g., blurring, and generalizability to the continuously evolving adversaries.  
Proactive methods, on the other hand, show promising results in forged image classification via watermarking-based image protection. 
Nevertheless, the large embedding capacity required by robust forgery localization can easily compromise the visual quality of the image.
% are often trained to discriminate between images manipulated with particular Generative Models (GMs) and genuine/real images, yet generalize poorly to images manipulated with GMs unseen in the training. 
% Conventional detection algorithms receive an input image passively.
This paper explores a novel image protection scheme that 
% defends RAW data against image manipulation,
% We propose to introduce imperceptible protective signal into the RAW data. 
embeds the imperceptible protective signal in the RAW domain and blindly transfers it into the rendered RGB images regardless of the image signal processing pipeline.
We find that this paradigm shift brings nontrivial benefits compared to direct RGB-domain protection in view of adaptive embedding and artifact reduction.
Once the protected RGB images are manipulated, the localization networks can accurately and robustly identify the forged areas.
% despite the presence of post-processing operations such as compression, blurring or color jittering. 
% Compared to RGB-domain protection, transferring protective signal from RAW brings nontrivial benefits of adaptive embedding and artifact reduction with the help of image signal processing pipelines.
Furthermore, for cellphones or cameras with limited computing resources, we 
% strengthen the practicality of image protection via a novelly-
propose a lightweight \textbf{M}ulti-frequency \textbf{P}artial \textbf{F}usion \textbf{Net}work (MPF-Net) with frequency learning and partial feature fusion 
% The novel lightweight architecture requires few computing resources, 
% , allowing it to be integrated into every image immediately after the camera shooting stage.
, which enables protection to be incorporated into images immediately after it is captured.
% making it possible to include protection countering manipulation in every image exactly after the camera shooting stage. 
Extensive experiments on famous RAW datasets, e.g., RAISE, FiveK and SIDD, prove the effectiveness of our method.
% robustness and transferability of our method.