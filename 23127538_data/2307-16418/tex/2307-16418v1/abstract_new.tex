The RAW format is widely used in most cameras, which contains the Bayer-Patterned data of the original image. Generally, the RAW data are further converted into JPEG images using the modules of Image Signal Processing (ISP). In this paper, we propose an idea of protecting the image at the beginning of camera-shooting. Specifically, we design a lightweight Multi-frequency Partial Fusion Network (MPF-Net) with the capabilities of frequency learning and partial feature fusion, which introduces an invisible watermark immediately into the RAW data. The protection capability can well be transferred into the rendered RGB images, even different ISP algorithms are used. Once the image is manipulated, we can accurately identify the forged areas with a dedicated localization network. The protection method is also robust against post-processing operations like blurring or recompression. Extensive experiments on famous RAW datasets, e.g., RAISE, FiveK and SIDD, indicate the effectiveness of our method. We hope that this technique can be used in future cameras as an option of image protection.