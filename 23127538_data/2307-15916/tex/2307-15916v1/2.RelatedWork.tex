\section{Related Work}
\subsection{Air Quality Forecasting}
% Air quality monitoring and forecasting: data-driven and model-driven approaches
Data-driven models for air quality forecasting has gained a huge popularity recently. Recent work \cite{zuo2023graph,liang2022airformer} studies graph-based representations of the air quality data by considering the sensor network as a graph structure, which extracts decent structural features between sensor data from a topological view. The air quality forecasting can be then formulated as a spatio-temporal forecasting problem.

% work on general spatio-temporal forecasting tasks, and what make Air Quality forecasting different? 
Works like DCRNN~\cite{li2018diffusion}, STGCN~\cite{yu2018spatio} and Graph WaveNet~\cite{wu2019graph}, have shown promising results in traffic forecasting tasks. These models can be adapted to air quality forecasting tasks owing to the shared spatio-temporal features present in the data. 
%The key differences lie in two aspects: i) Air quality exhibits continuous state changes between areas, while traffics change abruptly between nodes; ii) External factors impacting air quality are more complex, which can be from human activities, or cased by natural environment. 
% the typical research problems considered in air quality forecasting: multiple data sources -> data fusion -> forecasting target 
%DeepAir~\cite{yi2018deep}: spatial transformation for various sensor readings, a fusion network to model the relationships between different factors and AQIs.
%AirFormer~\cite{liang2022airformer}: self-attention for learning spatio-temporal representations; capture the intrinsic uncertainty of air quality data.
% Challenges/constrains of previous air quality forecasting models
However, in practice, the above-mentioned models often overlook the evolving nature of sensor networks as more data collection infrastructures are incrementally built. Consequently, these models require re-training from scratch on the most recent data that reflects the evolved sensor network. It may result in the loss of valuable information contained in outdated data collected from different network configurations.
%or using the previous model to do approximate inference of unlearned areas relying on neighbor predictions. 

% Learning models
%Some work models the air pollution in the whole city with an image-based representation. However, this representation may not be ideal, as air pollution and other impact factors have natural graph structures. 
% Challenges: huge amount of sensors are required

\subsection{Expandable Graph Neural Networks}
%Dynamic graph structures, meaning at different time stamps, the graph structure can be different. Although various works have studied the dynamic graphs with a focus on the dynamic edge weight but over a fixed set of graph nodes, they barely consider a dynamic graph with evolving number of nodes. 

In the field of graph learning, several works, such as ContinualGNN \cite{wang2020streaming} and ER-GNN \cite{zhou2021overcoming}, have incorporated the concept of Continual Learning to capture the evolving patterns within graph nodes.
While these approaches are valuable, it is important to consider spatio-temporal features in air quality forecasting tasks.
Designed for traffic forecasting, TrafficStream~\cite{chen2021trafficstream} considers evolving patterns on both temporal and spatial axes; ST-GFSL~\cite{lu2022spatio} introduces a meta-learning model for cross-city spatio-temporal knowledge transfer. However, these works primarily focus on shared (meta-)knowledge between nodes, and give less attention to expandable graph structures. 
Basically, spectral-based graph neural networks (GNNs) face challenges when scaling to graphs with different structures due to the complexity of reconstructing the Laplacian matrix. To address this issue, our paper explores the use of spatial-based GNNs, such as Graph Attention Networks (GAT) \cite{velivckovic2018graph}, for expandable graph learning in air quality forecasting tasks.
%ContinualGNN~\cite{wang2020streaming}: streaming GNN considering continual learning of new patterns and preservation of existing patterns.
%ER-GNN~\cite{zhou2021overcoming}: consider catastrophic forgetting problems using Memory Replay.

%ST-GFSL~\cite{lu2022spatio}: meta-learning for cross-city knowledge transfer, the graphs of different cities can be different. 

%GAT~\cite{velivckovic2018graph}; 

%TrafficStream~\cite{chen2021trafficstream}: evloving patterns on both temporal axis and spatial axis (expanded graphs) 


