\section{Experiments}
In this section, we demonstrate the effectiveness of EGAT with real-life air quality datasets. The experiments were designed to answer the following questions:

\begin{itemize}%[leftmargin=1in]
    
    \item[\textbf{Q1}] \textit{Continual learning with self-adaptation:} How well can our model make use of the ancient data with different graph structures, to improve the model's performance? % (Training process is different) compare between learning from all data and only from the data with the most recent graph structure 

    \item[\textbf{Q2}] \textit{Flexible Inference on unknown areas:} How well is our model at predicting air quality in areas without any sensors installed? i.e., no available data over these areas. % (Inference process is different) compare models' performance with or without integrating the training data for the 'unknown' areas.

        
\end{itemize}

\subsection{Experimental Settings}
\subsubsection{Dataset description}
We base our experiments on real air quality data~\cite{zuo2023unleashing} collected via PurpleAir API~\cite{purpleair}, which contains the AQIs and meteorological data in San Francisco (within $10 \, km^2$) between 2021-10-01 and 2023-05-15. The datasets are split to training, validation, test sets with \textit{7:1:2}. 
%The missing values are discarded during the training process, i.e., removing the training loss for missing values. 
Table \ref{AQIDatasets} shows more details of the collected datasets. For PurpleAirSF-1H, we adopt the last 12-hour data to predict the AQI (i.e., PM2.5) for the next 12 hours. For PurpleAir-6H, we consider the last 72 hours to predict the next 72 hours.

\begin{table}[!htbp]
\centering
\caption{Summary statistics of PurpleAirSF-1H/6H}
\label{AQIDatasets}
\scalebox{0.85}{
\begin{tabular}{ccccccc}
\toprule
\textbf{Data} & \textbf{\#Nodes} & \textbf{\#Features} & \textbf{Sampling} & \textbf{Observations} & \textbf{Missing} \\
\midrule
PurpleAirSF-1H  & 112              & 19                           & 1 hour               & 29 011 024            & 1.566\%                \\
PurpleAirSF-6H  & 232              & 19                           & 6 hours              & 10 054 648            & 1.231\%  \\
\bottomrule
\end{tabular}
}
\end{table}


\subsubsection{Execution and Parameter Settings}
%The proposed model is implemented by PyTorch 1.12.0 and is trained using the Adam optimizer with a learning rate of 0.001. 
We take Graph WaveNet as the backbone model. However, our proposal can be integrated to any air quality forecasting models. All the tests are done on a single Tesla A100 GPU of 40 Go memory. The forecasting accuracy of all tested models is evaluated by three metrics~\cite{zuo2023graph}: mean absolute error (MAE), root-mean-square error (RMSE) and mean absolute percentage error (MAPE).

\subsubsection{Baselines}
We compare EGAT with various model variants and with Graph WaveNet~\cite{wu2019graph}:
%a popular Spatio-temporal forecasting model based on Graph Convolutional Networks (GCNs). 
%We should note that our proposal can be integrated into any Spatio-temporal forecasting models by considering the expandable graph structures.

\begin{itemize}
    \item \textbf{GraphWaveNet} (GWN)~\cite{wu2019graph}: Trained on expanded graph data, as it is non-adaptable to different graph structures.
    \item \textbf{EGAT-Rec}: EGAT trained on data with expanded graph; 
    \item \textbf{EGAT-FI-SS}: EGAT trained on data over ancient graph, Flexible Inference (FI) with \textit{Spatial Smoothing} is applied; 
    \item \textbf{EGAT-FI-SRS}: EGAT trained on ancient data, FI with \textit{Spatial Representation Smoothing} is employed; 
    \item \textbf{EGAT}: EGAT trained on both ancient and recent data.
    
\end{itemize}

%We compare with ST-GFSL~\cite{lu2022spatio} and TrafficStream~\cite{chen2021trafficstream} for general expandable GNN models; besides, we compare with popular expandable GNN strategies:
%\begin{itemize}
 %   \item compare models' performance with or without integrating the training data for the 'unknown' areas;
 %   \item follow the settings in previous step, compare with the model with data for training; 
%\end{itemize}

\subsection{Experimental Results}
Table \ref{table:results_expand_node_ratio} and Table \ref{table:results_expand_time_ratio} reports the average errors (12/72H) regarding the expanding node ratio and expanding time ratio determined by the deployment.
\textbf{Bold} values indicate the best results, while \uline{underlined} values represent the second-best.

EGAT consistently outperforms other models in continual learning with different node ratios and time radios, owning to its ability to leverage rich data from various graph structures. While GWN performs better than EGAT-Rec, this can be attributed to the k-order diffusion process in GCN. Even so, EGAT surpasses GWN by incorporating ancient graph data, further validating our proposal in graph adaptations (\textbf{Q1}).

When forecasting in unknown areas, EGAT-FI-SS provides approximate AQIs through \textit{Spatial Smoothing}. However, its performance deteriorates with a high number of expanded nodes due to spatial sparsity. EGAT-FI-SRS performs better than EGAT-FI-SS and sometimes even better than GWN and comparable to EGAT, validating the viability of \textit{Spatial Representation Smoothing} for unknown areas' prediction (\textbf{Q2}).

\begin{table}[t]
\centering
\caption{Performance comparison regarding different ratios of \textbf{expanded nodes}, we fix the time ratio with expanded nodes as 10\%.}
\label{table:results_expand_node_ratio}
\scalebox{0.68}{
\begin{tabular}{clp{0.7cm}p{0.7cm}p{0.8cm}p{0.7cm}p{0.7cm}p{0.8cm}p{0.7cm}p{0.7cm}p{0.8cm}}
\toprule
                &       & \multicolumn{3}{c}{Expand node = 10\%}                                   & \multicolumn{3}{c}{Expand node = 20\%}                                    & \multicolumn{3}{c}{Expand node = 40\%}                                  \\
                
          & Models                & MAE                 & RMSE          & MAPE(\%)        & MAE                 & RMSE          & MAPE(\%)       & MAE                 & RMSE          & MAPE(\%)         \\
\midrule
\multirow{3}{*}{\rotatebox{90}{P.AirSF-1H}}    & Graph WaveNet               & \uline{3.62}       & \uline{10.77}  & \uline{10.76}  & \uline{3.60}       & \uline{10.83}  & \uline{10.00}  & \uline{3.62}       & \uline{10.85}  & \uline{10.61}  \\
 & EGAT-Rec                    & 3.83               & 11.02          & 12.51          & 3.76               & 10.96          & 11.02          & 3.86               & 11.09          & 12.18          \\
 & EGAT-FI-SS                  & 4.60               & 14.69          & 19.30          & 5.76               & 16.50          & 19.69          & 8.18               & 20.82          & 48.49          \\
 & EGAT-FI-SRS                 & 3.88               & 11.23          & 12.41          & 4.01               & 12.32          & 13.44          & 4.65               & 13.21          & 16.12          \\
 & \textbf{EGAT}          & \textbf{3.47}      & \textbf{10.73} & \textbf{7.73}  & \textbf{3.56}      & \textbf{10.82} & \textbf{8.47}  & \textbf{3.45}      & \textbf{10.76} & \textbf{8.78}          \\


\midrule
\multirow{3}{*}{\rotatebox{90}{P.AirSF-6H}}    & Graph WaveNet               & 6.65               & \uline{16.73}  & 31.47          & \uline{6.65}       & \uline{16.51}  & \uline{23.93}  & \uline{7.04}       & \uline{19.62}  & \uline{25.65}  \\
 & EGAT-Rec                    & 7.90               & 23.32          & 24.74          & 9.27               & 25.66          & 32.51          & 8.41               & 24.56          & 28.92          \\
 & EGAT-FI-SS                  & 6.34               & 22.40          & 26.24          & 8.42               & 29.20          & 36.74          & 11.10              & 35.26          & 80.79          \\
 & EGAT-FI-SRS                 & \uline{5.85}       & 17.21          & \uline{22.41}  & 7.21               & 22.45          & 26.85          & 9.45               & 26.12          & 31.21          \\
 & \textbf{\textbf{EGAT}} & \textbf{5.46}      & \textbf{13.96} & \textbf{18.69} & \textbf{5.18}      & \textbf{13.83} & \textbf{16.21} & \textbf{5.24}      & \textbf{13.85} & \textbf{19.23}       \\

\bottomrule
\end{tabular}
}
\end{table}

\begin{table}[t]
\centering
\caption{Performance comparison regarding different ratios of \textbf{time with expanded nodes}, we fix the expanded node ratio as 10\%.}
\label{table:results_expand_time_ratio}
\scalebox{0.68}{
\begin{tabular}{clp{0.7cm}p{0.7cm}p{0.8cm}p{0.7cm}p{0.7cm}p{0.8cm}p{0.7cm}p{0.7cm}p{0.8cm}}
\toprule
                &       & \multicolumn{3}{c}{Expand time = 10\%}                                   & \multicolumn{3}{c}{Expand time = 20\%}                                    & \multicolumn{3}{c}{Expand time = 40\%}                                  \\
                
          & Models                & MAE                 & RMSE          & MAPE(\%)        & MAE                 & RMSE          & MAPE(\%)       & MAE                 & RMSE          & MAPE(\%)         \\
\midrule
\multirow{3}{*}{\rotatebox{90}{P.AirSF-1H}}    & Graph WaveNet               & \uline{3.62}       & \uline{10.77}  & \uline{10.76}  & \uline{3.57}       & \uline{10.74}  & \uline{8.91}   & \uline{3.45}       & \textbf{10.75} & \uline{7.03}   \\
 & EGAT-Rec                    & 3.83               & 11.02          & 12.51          & 3.77               & 10.96          & 11.76          & 3.54               & 10.78                  & 8.87           \\
 & EGAT-FI-SS                  & 4.60               & 14.69          & 19.30          & 4.84               & 14.85          & 26.88          & 4.23               & 12.09                  & 17.50          \\
 & EGAT-FI-SRS                 & 3.88               & 11.23          & 12.41          & 4.01               & 12.32          & 14.33          & 3.60               & 11.12                  & 9.56           \\
 & \textbf{EGAT}          & \textbf{3.47}      & \textbf{10.73} & \textbf{7.73}  & \textbf{3.42}      & \textbf{10.60} & \textbf{7.43}  & \textbf{3.41}      & \uline{10.77}          & \textbf{6.61}            \\


\midrule
\multirow{3}{*}{\rotatebox{90}{P.AirSF-6H}}    & Graph WaveNet               & 6.65               & 16.73          & 31.47          & \uline{5.71}       & \uline{14.35}  & \uline{18.20}  & 5.40               & 14.21                  & 17.45          \\
 & EGAT-Rec                    & 7.90               & 23.32          & 24.74          & 6.38               & 16.36          & 19.33          & 6.10               & 14.99                  & 18.11          \\
 & EGAT-FI-SS                  & 6.34               & 22.40          & 26.24          & 7.00               & 25.85          & 23.99          & 6.72               & 21.90                  & 26.70          \\
 & EGAT-FI-SRS                 & \uline{5.85}       & 17.21          & \uline{22.41}  & 6.22               & 16.43          & 19.22          & \uline{5.23}       & \uline{14.11}          & \uline{14.26}  \\
 & \textbf{\textbf{EGAT}} & \textbf{5.46}      & \textbf{13.96} & \textbf{18.69} & \textbf{4.85}      & \textbf{13.82} & \textbf{13.87} & \textbf{4.78}      & \textbf{13.98}         & \textbf{12.18}           \\

\bottomrule
\end{tabular}
}
\vspace{-2em}
\end{table}

%\subsubsection{Q2: Flexible Inference on unknown areas}
%\subsubsection{Impact of recent data volume the number of new nodes}
%For air quality data at different time periods, the number of sensors are different. One strategy is to consider the installed sensor at previous period as missing values. Another one is to consider the transfer learning paradigm to time periods with different sensor infrastructures. We consider as well the model trained only on data with very recent network, which allows further validating the utility of the knowledge learned from ancient network data. 

%We compare as well the proposal with spectral graph models, which generally can not adapt to new networks structures, and we only train the models on data with the most recent sensor network.
% Meta-learning 
% Few-shot learning, why needed? a new learning task (not enough data)
% Simple strategy: only adapt the Adjacency Matrix
%% - can be distance-based, or attention-based 

\iffalse

\section{Supplementary Materials}
\subsection{Data collection}
We filter the air quality index with value between (0, 300), as the one which passed 300 is considered as heavily polluted. 
The data is collected from 2021-10-01 to 2023-05-15, with one-hour and six-hour granularity. 
\fi
