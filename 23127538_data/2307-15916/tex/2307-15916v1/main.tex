\documentclass[conference]{IEEEtran}
\IEEEoverridecommandlockouts
% The preceding line is only needed to identify funding in the first footnote. If that is unneeded, please comment it out.
\usepackage{cite}
\usepackage{amsmath,amsthm,amssymb,amsfonts}
\usepackage{algorithmic}
\usepackage{graphicx}
\usepackage{textcomp}
\usepackage{xcolor}
\usepackage{booktabs} % For formal tables
\usepackage[linesnumbered,ruled,vlined]{algorithm2e}
\usepackage{svg}
\usepackage{amsmath}
\usepackage{float}
\usepackage[font=small,labelfont=bf]{caption}
\usepackage[utf8]{inputenc}

\usepackage{multirow}
\usepackage{multicol}
\usepackage{makecell}
%\usepackage{subfigure}
\usepackage[font=footnotesize]{subfig}
\usepackage{rotating}
\usepackage{enumitem}
\usepackage{tabularx}
\usepackage[normalem]{ulem}

\theoremstyle{plain}
\newtheorem{thm}{Theorem}% reset theorem numbering for each chapter
\theoremstyle{definition}
\newtheorem{defn}[thm]{Definition} % definition numbers are dependent on theorem numbers


\SetAlFnt{\small}

\def\BibTeX{{\rm B\kern-.05em{\sc i\kern-.025em b}\kern-.08em
    T\kern-.1667em\lower.7ex\hbox{E}\kern-.125emX}}
\begin{document}



\title{\huge Opportunistic Air Quality Monitoring and Forecasting with Expandable Graph Neural Networks}

\author{\IEEEauthorblockN{Jingwei Zuo, Wenbin Li, Michele Baldo and Hakim Hacid}
\IEEEauthorblockA{Technology Innovation Institute, Abu Dhabi, UAE \\
Email: $\{$firstname.lastname$\}$@tii.ae}
}

\maketitle

\begin{abstract}
%% Text of abstract
Air Quality Monitoring and Forecasting has been a popular research topic in recent years. 
Recently, data-driven approaches for air quality forecasting have garnered significant attention, owing to the availability of well-established data collection facilities in urban areas. 
%Existing problems & challenges
Fixed infrastructures, typically deployed by national institutes or tech giants, often fall short in meeting the requirements of diverse personalized scenarios, e.g., forecasting in areas without any existing infrastructure. Consequently, smaller institutes or companies with limited budgets are compelled to seek tailored solutions by introducing more flexible infrastructures for data collection. 
%Our Contributions & Proposals 
%In this paper, we analyze the trade-off between model's performance and data collection cost, which provides a global view for industrial players to pick the most appropriate strategies between infrastructure building and model's performance. 
In this paper, we propose an expandable graph attention network (EGAT) model, which digests data collected from existing and newly-added infrastructures, with different spatial structures.
%Results
Additionally, our proposal can be embedded into any air quality forecasting models, to apply to the scenarios with evolving spatial structures. The proposal is validated over real air quality data from PurpleAir.
%we propose a pipeline for bridging the data collection, pre-processing and model building considering practical challenges on air pollution data analysis, e.g., missing value, sensor amount, collection duration, etc.  
%Experimental Results 
\end{abstract}

\begin{IEEEkeywords}
Air Quality Forecasting, Opportunistic Forecasting, Graph Neural Networks, Urban Computing
\end{IEEEkeywords}

\section{Introduction}
%the Introduction section need to be improved (writting & arguments, and reduce non-sense words)
Air quality forecasting using data-driven models has gained significant attention in recent years, thanks to the proliferation of data collection infrastructures such as sensor stations and advancements of telecommunication technologies. These infrastructures are typically managed by national institutes (e.g., AirParif\footnote{https://www.airparif.asso.fr/}, EPA\footnote{https://www.epa.gov/air-quality}) or large companies (e.g., PurpleAir\footnote{https://www2.purpleair.com/}) that specialize in air quality monitoring or forecasting services and products. Leveraging existing data collection infrastructures proves beneficial for initial research exploration or validating product prototypes.
However, reliance on fixed infrastructures presents practical constraints when customization is required for specific tasks. For instance, certain monitoring areas may be inadequately covered or completely absent from the existing infrastructures, or the density of coverage may not be sufficient. This issue particularly affects small or mid-sized industrial and academic players who face budget limitations that prevent them from investing in their own infrastructure from scratch, but have specific customization needs.

% give the motivation from another aspect: incrementally built infrastructure, no need to re-train the model
In addition to data collection, air quality forecasting models trained solely with data from public fixed infrastructures may not perform well for users' specific scenarios, such as forecasting at a higher spatial resolution. Deploying additional sensors as a cost-effective solution can enrich the data and improve forecasting performance without the need to build infrastructures from scratch. 
Subsequently, this targeted solution leads us to consider the practical question: \textit{how we can make use of the data collected from existing infrastructures, when integrating new sensor infrastructures?} 
%which can be equipped on fixed sensor stations or moving objects (e.g., drones) with a higher flexibility.

% Figure environment removed

As depicted in Figure \ref{fig:research_background}, the topological sensor network may change as the urban infrastructure evolves, resulting in varying network structures of air quality sensors. The data collected from the network $G_{\tau}$ needs to be augmented with enriched data from newly installed sensors $\Delta G_{\tau'}$ and $\Delta G_{\tau''}$. Training a model solely on recent data with $G_{\tau''}$ would overlook valuable information contained in the historical data with $G_{\tau}$ and $G_{\tau'}$.

In this paper, we propose an expandable graph attention network (EGAT) that effectively integrates data with various graph structures. This approach is versatile and can be seamlessly embedded into any existing air quality forecasting model. Furthermore, it applies to scenarios where sensors are not installed, enabling accurate forecasting in such areas.
We summarize our approach's main advantages as follows:
\begin{itemize}
    %\item \textbf{Air quality forecasting in real scenarios:} we consider the complex data quality issues, e.g., missing values

    \item \textbf{Less is more:} With fewer installed sensors, we can directly predict the air quality of other unknown area where sensors are not installed and achieve comparable performance to models relying on extensive data collection infrastructures with more sensors.
    \item \textbf{Continual learning with self-adaptation:} The proposed model enables continuous learning from newly collected data with expanded sensor networks, demonstrating self-adaptability to different topological sensor networks.
    \item \textbf{Embeddable module with scalability:} The proposed module can be seamlessly integrated into any air quality forecasting model, enhancing its ability to forecast in real-world scenarios.

\end{itemize}

The rest of this paper starts with a review of the most related work. Then, we formulate the problems of the paper. Later, we present in detail our proposal, which is followed by the experiments on real-life datasets and the conclusion.


\section{Related Work}
\subsection{Air Quality Forecasting}
% Air quality monitoring and forecasting: data-driven and model-driven approaches
Data-driven models for air quality forecasting has gained a huge popularity recently. Recent work \cite{zuo2023graph,liang2022airformer} studies graph-based representations of the air quality data by considering the sensor network as a graph structure, which extracts decent structural features between sensor data from a topological view. The air quality forecasting can be then formulated as a spatio-temporal forecasting problem.

% work on general spatio-temporal forecasting tasks, and what make Air Quality forecasting different? 
Works like DCRNN~\cite{li2018diffusion}, STGCN~\cite{yu2018spatio} and Graph WaveNet~\cite{wu2019graph}, have shown promising results in traffic forecasting tasks. These models can be adapted to air quality forecasting tasks owing to the shared spatio-temporal features present in the data. 
%The key differences lie in two aspects: i) Air quality exhibits continuous state changes between areas, while traffics change abruptly between nodes; ii) External factors impacting air quality are more complex, which can be from human activities, or cased by natural environment. 
% the typical research problems considered in air quality forecasting: multiple data sources -> data fusion -> forecasting target 
%DeepAir~\cite{yi2018deep}: spatial transformation for various sensor readings, a fusion network to model the relationships between different factors and AQIs.
%AirFormer~\cite{liang2022airformer}: self-attention for learning spatio-temporal representations; capture the intrinsic uncertainty of air quality data.
% Challenges/constrains of previous air quality forecasting models
However, in practice, the above-mentioned models often overlook the evolving nature of sensor networks as more data collection infrastructures are incrementally built. Consequently, these models require re-training from scratch on the most recent data that reflects the evolved sensor network. It may result in the loss of valuable information contained in outdated data collected from different network configurations.
%or using the previous model to do approximate inference of unlearned areas relying on neighbor predictions. 

% Learning models
%Some work models the air pollution in the whole city with an image-based representation. However, this representation may not be ideal, as air pollution and other impact factors have natural graph structures. 
% Challenges: huge amount of sensors are required

\subsection{Expandable Graph Neural Networks}
%Dynamic graph structures, meaning at different time stamps, the graph structure can be different. Although various works have studied the dynamic graphs with a focus on the dynamic edge weight but over a fixed set of graph nodes, they barely consider a dynamic graph with evolving number of nodes. 

In the field of graph learning, several works, such as ContinualGNN \cite{wang2020streaming} and ER-GNN \cite{zhou2021overcoming}, have incorporated the concept of Continual Learning to capture the evolving patterns within graph nodes.
While these approaches are valuable, it is important to consider spatio-temporal features in air quality forecasting tasks.
Designed for traffic forecasting, TrafficStream~\cite{chen2021trafficstream} considers evolving patterns on both temporal and spatial axes; ST-GFSL~\cite{lu2022spatio} introduces a meta-learning model for cross-city spatio-temporal knowledge transfer. However, these works primarily focus on shared (meta-)knowledge between nodes, and give less attention to expandable graph structures. 
Basically, spectral-based graph neural networks (GNNs) face challenges when scaling to graphs with different structures due to the complexity of reconstructing the Laplacian matrix. To address this issue, our paper explores the use of spatial-based GNNs, such as Graph Attention Networks (GAT) \cite{velivckovic2018graph}, for expandable graph learning in air quality forecasting tasks.
%ContinualGNN~\cite{wang2020streaming}: streaming GNN considering continual learning of new patterns and preservation of existing patterns.
%ER-GNN~\cite{zhou2021overcoming}: consider catastrophic forgetting problems using Memory Replay.

%ST-GFSL~\cite{lu2022spatio}: meta-learning for cross-city knowledge transfer, the graphs of different cities can be different. 

%GAT~\cite{velivckovic2018graph}; 

%TrafficStream~\cite{chen2021trafficstream}: evloving patterns on both temporal axis and spatial axis (expanded graphs) 



\section{Problem Formulation}
\begin{defn}(Air Quality Forecasting). Given an air quality sensor network $G=\mathcal{\{V, E\}}$, where $\mathcal{V}=\{v_1, ..., v_N\}$ is a set of $N$ sensor nodes/stations and $\mathcal{E}=\{e_1, ..., e_E\}$ is a set of $E$ edges connecting the nodes, the air quality data $\{AQI_{t}\}_{t=1}^{T}$ and meteorological data $\{M_{t}\}_{t=1}^{T}$ are collected over the $N$ stations, where $T$ is current timestamp. We aim to build a model $f$ to predict the $AQI$ over the next $T_p$ timestamps.
\end{defn}

To simplify, we denote input data as $\mathcal{X}$= $\{AQI_{t}, M_{t}\}_{t=1}^{T}$
= $\{x_t\}_{t=1}^{T}\in \mathbb{R}^{N \times F \times T }$. %which can be represented as a multivariate time series on a sensor network. 
Each node contains $F$ features representing $PM_{2.5}$, $PM_{10}$, humidity, temperature, etc. As $PM_{2.5}$ is \textit{most reported and most difficult-to-predict}~\cite{yi2018deep}, we take $PM_{2.5}$ as the AQI prediction target $\mathcal{Y}$=$\{y_t\}_{t = T+1}^{T+T_{p}} \in \mathbb{R}^{N\times T_{p}}$. 
%We use $\mathbf{X}_{t}=(\textbf{x}_{t}^{1},...,\textbf{x}_{t}^{N}) \in \mathbb{R}^{N \times F}$ denotes the observations at time $t$, where $\textbf{x}_t^{i}\in \mathbb{R}^{F}$ is the $i$-th variable of $X_t$.


\begin{defn}(Expanded Sensor Network). Given a sensor network at $\tau$: $G_{\tau}$ = $\{\mathcal{V}_{\tau}, \mathcal{E}_{\tau}\}$ with $N_{\tau}$ sensors, the network at $\tau'$: $G_{\tau'}$=$G_{\tau}$+$\Delta G_{\tau}$ = $\{V_{\tau'}, E_{\tau'}\}$ expands $G_{\tau}$ to $N_{\tau'}$ sensors. 
\end{defn}

% Can be simplified to TWO datasets with different sensor networks: it's OPTIONAL to have the training data on new sensor work, but the inference should be done over data collected from the new sensor network. 
% TODO: to improve the definition of training dataset, try to include the instance number and the definition of \tau.  
We aim to build a model $f$, which is firstly trained over a dataset $\{\mathcal{X}_{\tau}\}$ on a sensor network $G_{\tau}$ = $\{\mathcal{V}_{\tau}, \mathcal{E}_{\tau}\}$, and can be incrementally trained over $\{\mathcal{X}_{\tau'}\}$ on an expanded network $G_{\tau'}$. 
% several cases for inference
For inference, given a sequence $\mathcal{X} \in \mathbb{R}^{N_{\tau'} \times F \times T}$ and a sensor network $G_{\tau'}$, the model $f$ can predict the $AQI$ for the next $T_{p}$ time steps $\mathcal{Y}$=$\{y_t\}_{t = T+1}^{T+T_{p}} \in \mathbb{R}^{N\times T_{p}}$, where $N_{\tau'} \geq N_{\tau}$. 
%When $N_{\tau'} > N_{\tau}$, the model $f$ can predict the $AQI$ of any specific areas, without feeding the area's data for training.
\section{Our proposals}

%Spatial-based graph attention network (GAT)~\cite{velivckovic2018graph} can be adopted to encode the spatial relationships within the sensor network. As a weighted message-passing process, GAT is different from normal graph convolutional networks (GCNs)~\cite{zuo2023graph}. In GAT, the relations between nodes are not only decided by nodes' distance, but also by their inherent feature similarities. 
%Compared with spectral-based graph neural network [25, 26], GAT can adapt to various network structures.

In this paper, we adopt Graph WaveNet~\cite{wu2019graph} as the backbone model, which consists of $l$ Spatio-Temporal (ST) Blocks. However, our proposed EGAT can be integrated to any spatio-temporal models with adaptations on graph network layers. We employ Temporal Convolution Network (TCN) to encode the temporal dynamics of the AQIs. Specifically, as shown in Figure~\ref{fig:system_structure}, we designed an Expandable Graph Attention Network (EGAT) to learn from the data with evolving graph structures. 
%We adopt residual connections between the input and output of each ST block to avoid the gradient vanishing problem. 
The output forecasting layer takes skip connections on the output of the final ST Block and the hidden states after each TCN module for final predictions.
% Figure environment removed

%We consider two practical scenarios: i) new sensors are deployed, generating data with expanded structure for training; ii) for inference, we predict air qualities in areas W/o sensors.

\subsection{Temporal Dynamics with Temporal Convolution Network}
Compared to RNN-based approaches, Temporal Convolution Network (TCN)~\cite{wu2019graph} allows handling long-range sequences in a parallel manner, which is critical in industrial scenarios considering the model efficiency. %The output of the TCN's last layer is a representation that captures temporal dynamics in history. 

Given an input air quality sequence embedding $H$= $f_{linear}(\mathcal{X}) \in \mathbb{R}^{N \times d \times T}$, a filter $\mathcal{F}\in \mathbb{R}^{1 \times \mathrm{K}}$, $\mathrm{K}$ is the temporal filter size, $\mathrm{K}=2$ by default. The dilated causal convolution operation of $H$ with $\mathcal{F}$ at time $t$ is represented as:
\begin{equation}
\small
    H \star \mathcal{F}(t)= \textstyle\sum_{s=0}^{\mathrm{K}}\mathcal{F}(s) H(t-\textbf{d} \times s) \in \mathbb{R}^{N \times d\times T'}
\end{equation}
where $\star$ is the convolution operator, $\textbf{d}$ is the dilation factor, $d$ is the embedding size, $T'$ is the generated sequence length.
%, which equals to one on the last layer. 
We define the output of a gated TCN layer as:
\begin{equation}
\small
    \textbf{h} = tanh(W_{\mathcal{F}^{1}} \star H) \odot \sigma(W_{\mathcal{F}^{2}} \star H) \in \mathbb{R}^{N \times d \times T'}
\end{equation}
where $W_{\mathcal{F}^{1}}$, $W_{\mathcal{F}^{2}}$ are learnable parameters, $\odot$ is the element-wise multiplication operator, $\sigma(\cdot)$ denotes Sigmoid function.

\subsection{Expandable Graph Attention Networks (EGATs)}
%The spatial sensor nodes are deployed in an incremental way. How to inherit the knowledge from data collected in the old sensor network and combine it with the data collected from the new network shows a critical, and challenging technical gap in real-life scenarios. Bridging this gap allows industrial players to make use of existing data collection infrastructures as a basic staring point, to enrich and personalize the air quality monitoring and forecasting services with newly deployed sensors. 
Graph attention network (GAT)~\cite{velivckovic2018graph}, as a weighted message-passing process, models neighboring nodes' relationships via their inherent feature similarities.
%Compared with spectral-based graph neural network [25, 26], GAT can adapt to various network structures.
%The input of a graph attention layer is a set of node features at time $t$, 
Given a set of air pollution features at time $t$: $\textbf{h}(t)$ = $\{h_1, h_2, ..., h_N\}, h_i\in \mathbb{R}^{N \times d}$ as input of a graph attention layer, following~\cite{velivckovic2018graph}, we define the attention score between node $i$, $j$ as: 
%apply one learnable linear transformation to each interconnected node to obtain sufficient expressive power of the high-level features, the \textit{attention coefficients} $e_{i j}$ :
\begin{equation}\label{eq:attention_score}
\small
\alpha_{i j}=\frac{\exp 
\left(
\operatorname{a}\left(W h_i, W h_j\right)
\right)
}{\sum_{k \in \mathcal{N}_i} \exp \left(
\operatorname{a}\left(W h_i, W h_k\right)
\right)} 
\end{equation}

where $W \in \mathbb{R}^{d \times d'}$ is a weight matrix, $\operatorname{a}$ is the attentional mechanism as mentioned in~\cite{velivckovic2018graph}: $\mathbb{R}^{d'} \times \mathbb{R}^{d'} \to \mathbb{R} $, and $\mathcal{N}_i$ is a set of neighbor nodes of $v_i$.
A \textit{multi-head attention} with a nonlinearity $\sigma$ is employed to obtain abundant spatial representation of $v_i$ with features from its neighbor nodes $\mathcal{N}_i$:
\begin{equation}
\label{eq:multi_head_GAT}
\small
h_{i}'=\sigma\left(\frac{1}{K} \sum_{k=1}^K \sum_{j \in \mathcal{N}_i} \alpha_{i j} W^k h_j\right)
\end{equation}
Therefore, the GAT layer in $i$-th ST Block can be defined as:
\begin{equation}\label{eq:gat}
\small
H_{i+1} = \sigma \left(\frac{1}{K} \sum_{k=1}^K \mathcal{A} \textbf{h}_{i} W^k \right)
\end{equation}
where $\mathcal{A}$=$\{\alpha_{ij}\} \in \mathbb{R}^{N \times N}$, $H_{i+1} \in \mathbb{R}^{N \times d' \times T}$, $W^k \in \mathbb{R}^{d \times d'}$.


\iffalse 
%% Graph alignment (graph restruction)? L(G_dist, f(G_atten)), to check the graph construction methods using different node representations
\subsubsection{Graph Alignment}
In order to express the structural information of graphs and reduces structure deviation caused by different source data distribution, the graph is reconstructed by output embeddings for structure-aware learning. We predict the likelihood of an edge existing between nodes $v_i$ and $v_j$, by multiplying learned embeddings $h_i$ and $h_j$ as follows:
$$
p\left(a_{i j} \mid h_i, h_j \right)=\operatorname{sigmoid}\left(\left(h_i\right)^T, h_j\right) .
$$
As such, the graph $\mathbf{A}$ can be constructed as
$$
\mathbf{A}=\operatorname{sigmoid}\left[\left(H\right)^T \cdot H\right]
$$
where $(\cdot)^T$ is the transpose of the embedding matrix.
% 

Then, the re-construction loss can be defined as:
\begin{equation}
    L = | A - \mathbf{A}|
\end{equation}
 where $A$ is the original adjacency matrix, which can be basically represented by the distances between sensors.
\fi

When expanding the graph with new sensor nodes, we scale up the GAT layers on new nodes while conserving the information learned over the old ones. Basically, new nodes can be considered during both model's training and inference.

%We aim to build a model $f$, which is firstly trained over a dataset $\{\mathcal{X}_{\tau}\}$ on a sensor network $G_{\tau}$ = $\{\mathcal{V}_{\tau}, \mathcal{E}_{\tau}\}$, and can be incrementally trained over $\{\mathcal{X}_{\tau'}\}$ on an expanded network $G_{\tau'}$. 
% several cases for inference
%For inference, given a sequence $\mathcal{X} \in \mathbb{R}^{N_{\tau'} \times F \times T}$ and a sensor network $G_{\tau'}$, the model $f$ can predict the $AQI$ for the next $T_{p}$ time steps $\mathcal{Y}$=$\{y_t\}_{t = T+1}^{T+T_{p}} \in \mathbb{R}^{N\times T_{p}}$, where $N_{\tau'} \geq N_{\tau}$. 
%When $N_{\tau'} > N_{\tau}$, the model $f$ can predict the $AQI$ of any specific areas, without feeding the area's data for training.

\subsubsection{Expandable Graph Network Training}
We consider that the sensor network expands with the newly built infrastructures. The model learned from $G_{\tau}$ can be updated with recent data over $G_{\tau'}$ without re-training the model from scratch. 

From Equation \ref{eq:gat}, with new embeddings $\mathbf{h}_{\tau'}$$\in$$\mathbb{R}^{N_{\tau'} \times d \times T}$, the weight matrix $W^{k}$ stays unchanged; only the adjacency matrix requires updates: $\mathcal{A}_{\tau} \in \mathbb{R}^{N_{\tau} \times N_{\tau}}$ $\to$ $\mathcal{A}_{\tau'}\in \mathbb{R}^{N_{\tau'} \times N_{\tau'}}$.
%%
We re-define $\mathcal{N}_i$=$\{\mathcal{N}_{i,\tau}, \mathcal{N}_{i,\tau'}\}$ as the $k$ nearest neighbors of $v_{i}$, where $\mathcal{N}_{i,\tau}$ denotes neighbors from existing nodes, $\mathcal{N}_{i,\tau'}$ indicates those from newly added nodes. 
%%
Given a set of new sensors $\Delta \mathcal{V}_{\tau}$, we obtain new edge connections $\Delta \mathcal{E}_{\tau}$=$\{\mathcal{N}_{i}\}_{i=1}^{\Delta N}$, where $\Delta N$=$N_{\tau'}-N_{\tau}$, with $\mathcal{O}(N_{\tau'} \Delta N)$ time for distance computations.
According to Equation~\ref{eq:attention_score}, the attentional mechanism will apply to $\Delta \mathcal{E}_{\tau}$ with $\mathcal{O}(\Delta N k)$ time. Therefore, the attention score between node $i, j$ can be re-defined as:
\begin{equation}\label{eq:attention_score_new}
\footnotesize
\alpha_{i j}=\frac{\exp 
\left(
\operatorname{a}\left(W h_i, W h_j\right)
\right)
}{\sum\limits_{k \in \mathcal{N}_{i,\tau}} \! \exp \left(
\operatorname{a}\left(W h_i, W h_k\right)
\right)
\!+\!
\sum\limits_{k \in \mathcal{N}_{i,\tau'}} \! \exp \left(
\operatorname{a}\left(W h_i, W h_k\right)
\right)
} 
\end{equation}
In this manner, we can update the graph layer, i.e., $\mathcal{A}_{\tau'}$ incrementally by considering cached attention scores over $\mathcal{E}_{\tau}$, reducing the time complexity to $\mathcal{O}(N_{\tau'} \Delta N + \Delta N k)$. 
This is much faster than rebuilding the entire graph layer ($\mathcal{O}(N_{\tau'}^{2})$).
%In other words, when applying GAT, we need to re-calibrate the attention scores between the graph nodes. 
%Instead of rebuilding the model from scratch, the incremental learning process requires $\mathcal{O}(N'^{2}\text{-} N^{2})$ extra computation time, which calculates the relation scores between newly added and existing sensor nodes. In fact, it is not necessary because only the neighbors within the order $k$ are affected. %(cited from Wang et al., CIKM'20, Streaming GNNs via Continual Learning) An improved idea is to use breadth-first-search (BFS) to find all the L-order neighbors and calculate their scoring function, but the complexity is still related to the size of the neighborhood of ΔG. 
%Therefore, we generate a sub-graph by selecting neighbor nodes of newly added nodes on $\mathcal{G}$ within k-hops. This will greatly reduce the processing time to $\mathcal{O}((\Delta N k^{2})^2)$ by updating a sub-graph $\mathcal{G}_{sub}$ with $\Delta N$ nodes. % only use the new-sensor data to calculate the loss?
% Q1: how to compute the adjacency matrix? -> 
% Q2: attention and distanced-based adjacency matrix value
% Q3: why using GAT, but not a normal GCN layer?  
%% Answer to Q2: for training, we use attention-based (GAT), for inference, we use distance-based (GCN)
%% Answer to Q3: GAT can better capture the information in each node for enriching the node's embedding
% TODO: to clarify the time complexity for computing attention scores, formulate the complexity in math representation




\iffalse 
\subsubsection{Expandable Graph Network Inference}
During the inference step, the sensor network can be different from the existing one, with an expanded structure. The objective is to forecast the values in unseen sensors, which are not learned in the model. In this case, only the spatial location of the desired area is available. We aim to forecast the air quality of a specific area without any sensors. As the graph alignment allows constraining the graph structure to the distance-based graph topology, the air quality on unknown areas can be obtained by simply weighting the data from connected sensors. 
\begin{equation}
    Y_i = \sum_{j\in \mathcal{N}_i} a_{ij} Y_j, N_i = \{v_j | dist(v_i, v_j) < \varepsilon \}
\end{equation}
where $a_{ij} \in A$ is the reversed distance between $v_i$ and $v_j$, $\varepsilon$ is a threshold which decides the neighboring sensor nodes. 
% need to compare between i) model trained on new sensor data ii) model not trained on new sensor data  
% Q: do we have the data from new sensors? 1) without data -> more practical, can infer the air quality in any areas (worse performance); 2) with data -> re-train the model with updated sensor network (better performance)
\fi


% consider multiple scenarios for model inference
\subsubsection{Expandable Graph Network Inference}
When no sensors are installed in (unseen) areas, \textit{Spatial Smoothing} can be performed on the unseen node $v_i$. Based on its spatial location, we incorporate predictions from its neighbor nodes:
\begin{equation}
    Y_i = \sum_{j\in \mathcal{N}_i} a_{ij} Y_j, N_i = \{v_j | dist(v_i, v_j) < \varepsilon \}
\end{equation}
where $\mathcal{N}_i$ is the first-order neighbors of $v_i$ (excluding $v_i$, as the data on $v_i$ is unavailable), $a_{ij}=1-\frac{dist(v_{i}, v_{j})}{\sum_{k\in \mathcal{N}_{i}} dist(v_{i}, v_{k})}$ is the inverse Euclidean Distance (ED) between $v_i$ and $v_j$, $\varepsilon$ is a threshold which decides the neighboring sensor nodes. 

We propose a robust \textit{Spatial Representation Smoothing} technique that considers richer spatial relationships, in the embedding space, between unseen and existing nodes. Given an unseen node $v_i$, its embedding $h_i$ can be defined as follows:
\begin{equation}
\small
h_{i}=\sigma\left(\frac{1}{K} \sum_{k=1}^K \sum_{j \in \mathcal{N}_i} a_{i j} W^k h_j\right )
\end{equation}
where $a_{ij}$ is the inverse ED between $v_i$ and $v_j$, $W^{k}$ is the learned weights in each attention head as shown in Equation~\ref{eq:multi_head_GAT}.

    
% need to compare between i) model trained on new sensor data ii) model not trained on new sensor data  (inference W/o new sensor data) 

% Q: do we have the data from new sensors? 1) without data -> more practical, can infer the air quality in any areas; 2) with data -> sensors should be installed, may be more convincing for the paper/model performance; arguments for this point: no training effort, just forecast the air quality on newly installed sensors; 3) with data -> re-train the model with updated sensor network



\iffalse
\subsection{Dynamic Evolving Data Distributions}
% can be put in future work, as there is a huge effort to explore/collect the data with corresponding dynamic features 
The air quality data distribution evolves with time. Thus, a stable model can not adapt to the recent distribution, leading to deteriorated performance when forecasting real-time air quality. Re-training the model can be a simple solution, however, it requires a careful selection of the re-training period, a considerable re-training effort, and a non-ignorable re-training latency. 
%(To consider transfer learning or meta-learning)
\fi


\subsection{Output Forecasting Layer}
For final predictions, we take skip connections as shown in~\cite{wu2019graph} on the final ST Block's output and hidden states after each TCN.
The concatenated output features are defined as:
\begin{equation}
\small
    O = (\textbf{h}_{0} W^{0} + b^{0})\| ... \| (\textbf{h}_{l-1} W^{l-1} + b{l-1}) \ \| (\mathcal{H}_{l} W^{l} + b^{l})
\end{equation}
where $O \in \mathbb{R}^{N\times (l+1)d}$, $W_{s}^{i}$, $b_{s}^{i}$ are learnable parameters for the convolution layers. Two fully-connected layers are added to project the concatenated features into the desired dimension:
\begin{equation}
\small
    \hat{\mathcal{Y}} = (ReLU(OW_{fc}^{1} +  b_{fc}^{1})) W_{fc}^{2}  +  b_{fc}^{2} \in \mathbb{R}^{N\times T_{p}}
\end{equation}
where $W_{fc}^{1}$, $W_{fc}^{2}$, $b_{fc}^{1}$, $b_{fc}^{2}$ are learnable parameters. We use mean absolute error (MAE) \cite{wu2019graph} as loss function for training.


\section{Experiments}
In this section, we demonstrate the effectiveness of EGAT with real-life air quality datasets. The experiments were designed to answer the following questions:

\begin{itemize}%[leftmargin=1in]
    
    \item[\textbf{Q1}] \textit{Continual learning with self-adaptation:} How well can our model make use of the ancient data with different graph structures, to improve the model's performance? % (Training process is different) compare between learning from all data and only from the data with the most recent graph structure 

    \item[\textbf{Q2}] \textit{Flexible Inference on unknown areas:} How well is our model at predicting air quality in areas without any sensors installed? i.e., no available data over these areas. % (Inference process is different) compare models' performance with or without integrating the training data for the 'unknown' areas.

        
\end{itemize}

\subsection{Experimental Settings}
\subsubsection{Dataset description}
We base our experiments on real air quality data~\cite{zuo2023unleashing} collected via PurpleAir API~\cite{purpleair}, which contains the AQIs and meteorological data in San Francisco (within $10 \, km^2$) between 2021-10-01 and 2023-05-15. The datasets are split to training, validation, test sets with \textit{7:1:2}. 
%The missing values are discarded during the training process, i.e., removing the training loss for missing values. 
Table \ref{AQIDatasets} shows more details of the collected datasets. For PurpleAirSF-1H, we adopt the last 12-hour data to predict the AQI (i.e., PM2.5) for the next 12 hours. For PurpleAir-6H, we consider the last 72 hours to predict the next 72 hours.

\begin{table}[!htbp]
\centering
\caption{Summary statistics of PurpleAirSF-1H/6H}
\label{AQIDatasets}
\scalebox{0.85}{
\begin{tabular}{ccccccc}
\toprule
\textbf{Data} & \textbf{\#Nodes} & \textbf{\#Features} & \textbf{Sampling} & \textbf{Observations} & \textbf{Missing} \\
\midrule
PurpleAirSF-1H  & 112              & 19                           & 1 hour               & 29 011 024            & 1.566\%                \\
PurpleAirSF-6H  & 232              & 19                           & 6 hours              & 10 054 648            & 1.231\%  \\
\bottomrule
\end{tabular}
}
\end{table}


\subsubsection{Execution and Parameter Settings}
%The proposed model is implemented by PyTorch 1.12.0 and is trained using the Adam optimizer with a learning rate of 0.001. 
We take Graph WaveNet as the backbone model. However, our proposal can be integrated to any air quality forecasting models. All the tests are done on a single Tesla A100 GPU of 40 Go memory. The forecasting accuracy of all tested models is evaluated by three metrics~\cite{zuo2023graph}: mean absolute error (MAE), root-mean-square error (RMSE) and mean absolute percentage error (MAPE).

\subsubsection{Baselines}
We compare EGAT with various model variants and with Graph WaveNet~\cite{wu2019graph}:
%a popular Spatio-temporal forecasting model based on Graph Convolutional Networks (GCNs). 
%We should note that our proposal can be integrated into any Spatio-temporal forecasting models by considering the expandable graph structures.

\begin{itemize}
    \item \textbf{GraphWaveNet} (GWN)~\cite{wu2019graph}: Trained on expanded graph data, as it is non-adaptable to different graph structures.
    \item \textbf{EGAT-Rec}: EGAT trained on data with expanded graph; 
    \item \textbf{EGAT-FI-SS}: EGAT trained on data over ancient graph, Flexible Inference (FI) with \textit{Spatial Smoothing} is applied; 
    \item \textbf{EGAT-FI-SRS}: EGAT trained on ancient data, FI with \textit{Spatial Representation Smoothing} is employed; 
    \item \textbf{EGAT}: EGAT trained on both ancient and recent data.
    
\end{itemize}

%We compare with ST-GFSL~\cite{lu2022spatio} and TrafficStream~\cite{chen2021trafficstream} for general expandable GNN models; besides, we compare with popular expandable GNN strategies:
%\begin{itemize}
 %   \item compare models' performance with or without integrating the training data for the 'unknown' areas;
 %   \item follow the settings in previous step, compare with the model with data for training; 
%\end{itemize}

\subsection{Experimental Results}
Table \ref{table:results_expand_node_ratio} and Table \ref{table:results_expand_time_ratio} reports the average errors (12/72H) regarding the expanding node ratio and expanding time ratio determined by the deployment.
\textbf{Bold} values indicate the best results, while \uline{underlined} values represent the second-best.

EGAT consistently outperforms other models in continual learning with different node ratios and time radios, owning to its ability to leverage rich data from various graph structures. While GWN performs better than EGAT-Rec, this can be attributed to the k-order diffusion process in GCN. Even so, EGAT surpasses GWN by incorporating ancient graph data, further validating our proposal in graph adaptations (\textbf{Q1}).

When forecasting in unknown areas, EGAT-FI-SS provides approximate AQIs through \textit{Spatial Smoothing}. However, its performance deteriorates with a high number of expanded nodes due to spatial sparsity. EGAT-FI-SRS performs better than EGAT-FI-SS and sometimes even better than GWN and comparable to EGAT, validating the viability of \textit{Spatial Representation Smoothing} for unknown areas' prediction (\textbf{Q2}).

\begin{table}[t]
\centering
\caption{Performance comparison regarding different ratios of \textbf{expanded nodes}, we fix the time ratio with expanded nodes as 10\%.}
\label{table:results_expand_node_ratio}
\scalebox{0.68}{
\begin{tabular}{clp{0.7cm}p{0.7cm}p{0.8cm}p{0.7cm}p{0.7cm}p{0.8cm}p{0.7cm}p{0.7cm}p{0.8cm}}
\toprule
                &       & \multicolumn{3}{c}{Expand node = 10\%}                                   & \multicolumn{3}{c}{Expand node = 20\%}                                    & \multicolumn{3}{c}{Expand node = 40\%}                                  \\
                
          & Models                & MAE                 & RMSE          & MAPE(\%)        & MAE                 & RMSE          & MAPE(\%)       & MAE                 & RMSE          & MAPE(\%)         \\
\midrule
\multirow{3}{*}{\rotatebox{90}{P.AirSF-1H}}    & Graph WaveNet               & \uline{3.62}       & \uline{10.77}  & \uline{10.76}  & \uline{3.60}       & \uline{10.83}  & \uline{10.00}  & \uline{3.62}       & \uline{10.85}  & \uline{10.61}  \\
 & EGAT-Rec                    & 3.83               & 11.02          & 12.51          & 3.76               & 10.96          & 11.02          & 3.86               & 11.09          & 12.18          \\
 & EGAT-FI-SS                  & 4.60               & 14.69          & 19.30          & 5.76               & 16.50          & 19.69          & 8.18               & 20.82          & 48.49          \\
 & EGAT-FI-SRS                 & 3.88               & 11.23          & 12.41          & 4.01               & 12.32          & 13.44          & 4.65               & 13.21          & 16.12          \\
 & \textbf{EGAT}          & \textbf{3.47}      & \textbf{10.73} & \textbf{7.73}  & \textbf{3.56}      & \textbf{10.82} & \textbf{8.47}  & \textbf{3.45}      & \textbf{10.76} & \textbf{8.78}          \\


\midrule
\multirow{3}{*}{\rotatebox{90}{P.AirSF-6H}}    & Graph WaveNet               & 6.65               & \uline{16.73}  & 31.47          & \uline{6.65}       & \uline{16.51}  & \uline{23.93}  & \uline{7.04}       & \uline{19.62}  & \uline{25.65}  \\
 & EGAT-Rec                    & 7.90               & 23.32          & 24.74          & 9.27               & 25.66          & 32.51          & 8.41               & 24.56          & 28.92          \\
 & EGAT-FI-SS                  & 6.34               & 22.40          & 26.24          & 8.42               & 29.20          & 36.74          & 11.10              & 35.26          & 80.79          \\
 & EGAT-FI-SRS                 & \uline{5.85}       & 17.21          & \uline{22.41}  & 7.21               & 22.45          & 26.85          & 9.45               & 26.12          & 31.21          \\
 & \textbf{\textbf{EGAT}} & \textbf{5.46}      & \textbf{13.96} & \textbf{18.69} & \textbf{5.18}      & \textbf{13.83} & \textbf{16.21} & \textbf{5.24}      & \textbf{13.85} & \textbf{19.23}       \\

\bottomrule
\end{tabular}
}
\end{table}

\begin{table}[t]
\centering
\caption{Performance comparison regarding different ratios of \textbf{time with expanded nodes}, we fix the expanded node ratio as 10\%.}
\label{table:results_expand_time_ratio}
\scalebox{0.68}{
\begin{tabular}{clp{0.7cm}p{0.7cm}p{0.8cm}p{0.7cm}p{0.7cm}p{0.8cm}p{0.7cm}p{0.7cm}p{0.8cm}}
\toprule
                &       & \multicolumn{3}{c}{Expand time = 10\%}                                   & \multicolumn{3}{c}{Expand time = 20\%}                                    & \multicolumn{3}{c}{Expand time = 40\%}                                  \\
                
          & Models                & MAE                 & RMSE          & MAPE(\%)        & MAE                 & RMSE          & MAPE(\%)       & MAE                 & RMSE          & MAPE(\%)         \\
\midrule
\multirow{3}{*}{\rotatebox{90}{P.AirSF-1H}}    & Graph WaveNet               & \uline{3.62}       & \uline{10.77}  & \uline{10.76}  & \uline{3.57}       & \uline{10.74}  & \uline{8.91}   & \uline{3.45}       & \textbf{10.75} & \uline{7.03}   \\
 & EGAT-Rec                    & 3.83               & 11.02          & 12.51          & 3.77               & 10.96          & 11.76          & 3.54               & 10.78                  & 8.87           \\
 & EGAT-FI-SS                  & 4.60               & 14.69          & 19.30          & 4.84               & 14.85          & 26.88          & 4.23               & 12.09                  & 17.50          \\
 & EGAT-FI-SRS                 & 3.88               & 11.23          & 12.41          & 4.01               & 12.32          & 14.33          & 3.60               & 11.12                  & 9.56           \\
 & \textbf{EGAT}          & \textbf{3.47}      & \textbf{10.73} & \textbf{7.73}  & \textbf{3.42}      & \textbf{10.60} & \textbf{7.43}  & \textbf{3.41}      & \uline{10.77}          & \textbf{6.61}            \\


\midrule
\multirow{3}{*}{\rotatebox{90}{P.AirSF-6H}}    & Graph WaveNet               & 6.65               & 16.73          & 31.47          & \uline{5.71}       & \uline{14.35}  & \uline{18.20}  & 5.40               & 14.21                  & 17.45          \\
 & EGAT-Rec                    & 7.90               & 23.32          & 24.74          & 6.38               & 16.36          & 19.33          & 6.10               & 14.99                  & 18.11          \\
 & EGAT-FI-SS                  & 6.34               & 22.40          & 26.24          & 7.00               & 25.85          & 23.99          & 6.72               & 21.90                  & 26.70          \\
 & EGAT-FI-SRS                 & \uline{5.85}       & 17.21          & \uline{22.41}  & 6.22               & 16.43          & 19.22          & \uline{5.23}       & \uline{14.11}          & \uline{14.26}  \\
 & \textbf{\textbf{EGAT}} & \textbf{5.46}      & \textbf{13.96} & \textbf{18.69} & \textbf{4.85}      & \textbf{13.82} & \textbf{13.87} & \textbf{4.78}      & \textbf{13.98}         & \textbf{12.18}           \\

\bottomrule
\end{tabular}
}
\vspace{-2em}
\end{table}

%\subsubsection{Q2: Flexible Inference on unknown areas}
%\subsubsection{Impact of recent data volume the number of new nodes}
%For air quality data at different time periods, the number of sensors are different. One strategy is to consider the installed sensor at previous period as missing values. Another one is to consider the transfer learning paradigm to time periods with different sensor infrastructures. We consider as well the model trained only on data with very recent network, which allows further validating the utility of the knowledge learned from ancient network data. 

%We compare as well the proposal with spectral graph models, which generally can not adapt to new networks structures, and we only train the models on data with the most recent sensor network.
% Meta-learning 
% Few-shot learning, why needed? a new learning task (not enough data)
% Simple strategy: only adapt the Adjacency Matrix
%% - can be distance-based, or attention-based 

\iffalse

\section{Supplementary Materials}
\subsection{Data collection}
We filter the air quality index with value between (0, 300), as the one which passed 300 is considered as heavily polluted. 
The data is collected from 2021-10-01 to 2023-05-15, with one-hour and six-hour granularity. 
\fi


\section{Conclusion}\label{sec:conclusion}

This paper presents our empirical domain knowledge distillation framework using ChatGPT and discusses our observations from the framework application experiments in the autonomous driving domain. The key finding is that: 1) with proper design of prompt engineering and execution flow, fully automated domain knowledge (in the ontology format) distillation is possible. However, due to the randomness in the response and the butterfly effect, the quality of fully automated distillation results is not guaranteed. To address this, we develop a web-based assistant to enable manual supervision and early intervention at runtime. We hope our findings and tools inspire future research toward revolutionizing the engineering processes of knowledge-based systems across domains.

\bibliographystyle{IEEEtran}
\bibliography{references.bib}
\end{document}
