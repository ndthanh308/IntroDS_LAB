\section{Introduction}
%the Introduction section need to be improved (writting & arguments, and reduce non-sense words)
Air quality forecasting using data-driven models has gained significant attention in recent years, thanks to the proliferation of data collection infrastructures such as sensor stations and advancements of telecommunication technologies. These infrastructures are typically managed by national institutes (e.g., AirParif\footnote{https://www.airparif.asso.fr/}, EPA\footnote{https://www.epa.gov/air-quality}) or large companies (e.g., PurpleAir\footnote{https://www2.purpleair.com/}) that specialize in air quality monitoring or forecasting services and products. Leveraging existing data collection infrastructures proves beneficial for initial research exploration or validating product prototypes.
However, reliance on fixed infrastructures presents practical constraints when customization is required for specific tasks. For instance, certain monitoring areas may be inadequately covered or completely absent from the existing infrastructures, or the density of coverage may not be sufficient. This issue particularly affects small or mid-sized industrial and academic players who face budget limitations that prevent them from investing in their own infrastructure from scratch, but have specific customization needs.

% give the motivation from another aspect: incrementally built infrastructure, no need to re-train the model
In addition to data collection, air quality forecasting models trained solely with data from public fixed infrastructures may not perform well for users' specific scenarios, such as forecasting at a higher spatial resolution. Deploying additional sensors as a cost-effective solution can enrich the data and improve forecasting performance without the need to build infrastructures from scratch. 
Subsequently, this targeted solution leads us to consider the practical question: \textit{how we can make use of the data collected from existing infrastructures, when integrating new sensor infrastructures?} 
%which can be equipped on fixed sensor stations or moving objects (e.g., drones) with a higher flexibility.

% Figure environment removed

As depicted in Figure \ref{fig:research_background}, the topological sensor network may change as the urban infrastructure evolves, resulting in varying network structures of air quality sensors. The data collected from the network $G_{\tau}$ needs to be augmented with enriched data from newly installed sensors $\Delta G_{\tau'}$ and $\Delta G_{\tau''}$. Training a model solely on recent data with $G_{\tau''}$ would overlook valuable information contained in the historical data with $G_{\tau}$ and $G_{\tau'}$.

In this paper, we propose an expandable graph attention network (EGAT) that effectively integrates data with various graph structures. This approach is versatile and can be seamlessly embedded into any existing air quality forecasting model. Furthermore, it applies to scenarios where sensors are not installed, enabling accurate forecasting in such areas.
We summarize our approach's main advantages as follows:
\begin{itemize}
    %\item \textbf{Air quality forecasting in real scenarios:} we consider the complex data quality issues, e.g., missing values

    \item \textbf{Less is more:} With fewer installed sensors, we can directly predict the air quality of other unknown area where sensors are not installed and achieve comparable performance to models relying on extensive data collection infrastructures with more sensors.
    \item \textbf{Continual learning with self-adaptation:} The proposed model enables continuous learning from newly collected data with expanded sensor networks, demonstrating self-adaptability to different topological sensor networks.
    \item \textbf{Embeddable module with scalability:} The proposed module can be seamlessly integrated into any air quality forecasting model, enhancing its ability to forecast in real-world scenarios.

\end{itemize}

The rest of this paper starts with a review of the most related work. Then, we formulate the problems of the paper. Later, we present in detail our proposal, which is followed by the experiments on real-life datasets and the conclusion.

