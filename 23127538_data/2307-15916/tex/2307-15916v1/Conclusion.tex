\section{Perspectives and Conclusion}
In this paper, we propose an Expandable Graph Attention Network (EGAT) for Air Quality monitoring and forecasting. It incorporates historical and recent graph data, which prevents industrial players with budget limitations from investing in their own infrastructures from scratch. EGAT also allows predicting air quality in areas without installed sensors. Future work includes comparing additional expandable graph learning models and exploring transfer learning and node alignment techniques to reduce re-training effort in industrial scenarios.

%Recent work~\cite{lu2022spatio} starts studying the knowledge transfer between different cities. However, more practical scenarios can be considered. Such as the node alignment between cities, and topological expansion with cities. Besides, when modeling long term air quality data, the evolving data distributions can be considered, e.g., catastrophic forgetting problem of the temporal patterns during the expanded graph learning. 
%which can be handled by recent models, e.g., ER-GNN~\cite{zhou2021overcoming}.