% We know that we are awesome! :D
% Depression is a severe mental illness that not only affects an individual, but also has global social and economical implications. 
In this paper, we demonstrate the efficacy of elementary head motion units, termed \emph{kinemes}, for depression detection by utilising two approaches: (a) discovering kinemes from data of both patient and control cohorts, and (b) learning kineme patterns solely from the control cohort to compute statistical functional features derived from reconstruction errors for the two classes. Apart from effective depression detection, we also identify explainable kineme patterns for the two classes, consistent with prior research. 

Our study demonstrates the utility of head motion features for detecting depression, but our experiments are restricted to classification tasks involving a discretisation of the depression scores. In the future, we will investigate (a) the utility of kinemes for continuous prediction (regression) of depression severity, (b) the cross-dataset generalisability of models trained via kinemes, and (c) the development of multimodal methodologies combining kinemes with other behavioural markers, and evaluating their efficacy.
% \vspace{-4mm}
%\section*{Acknowledgement}
%Thanks to A. Samanta, IIT Kanpur for sharing the kineme code.
%This research was supported partially by the Australian Government through the Australian Research Council’s Discovery Projects funding scheme (project DP190101294).