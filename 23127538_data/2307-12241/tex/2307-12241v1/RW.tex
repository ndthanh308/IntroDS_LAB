%
In this section, we briefly review the literature focusing on (a) depression detection as a classification problem, and (b) depression detection using head motion patterns.
%----------------------------------------
%
\subsection{Depression Analysis as a Classification Task}
%
Traditionally, depression detection has been approached as a supervised binary classification task, with many studies relying on discriminative classifiers to distinguish between \emph{healthy controls} and \emph{patients} \cite{alghowinem2015cross, cohn2018multimodal, alghowinem2013head}. A typical recognition accuracy of up to 80\% demonstrates the promise of behavioural cues such as eye-blink and closed-eye duration rate, statistical features computed over the yaw, pitch and roll head-pose features, \etc. to differentiate the two classes. However, challenges involved in depression detection such as limited clinically validated, curated data and a skewed data distributions have been acknowledged in the literature~\cite{alghowinem2013head, nasir2016multimodal}. 

Recent efforts have sought to learn patterns indicative of only the target class and reformulate depression detection as a one-class classification problem to mitigate the issues with imbalanced datasets~\cite{opoku2019towards, aguilera2021depression}. Studies have attempted to learn features associated with control participants and treat inputs that deviate from these patterns as \textit{anomalous}~\cite{gerych2019classifying, mourao2011patient}. Gerych \emph{et. al.}~\cite{gerych2019classifying} formulate the task as anomaly detection by leveraging autoencoders to learn features of the non-depressed class and treating depressed user data as outliers. Similarly, Mourão-Miranda \emph{et. al.}~\cite{mourao2011patient} employ a one-class SVM to classify patients as outliers compared to healthy participants based on the fMRI responses to sad facial expressions. Conversely, a few studies explore one-class classification by learning features characterising the depressed class, and treating non-depressed subjects as outliers~\cite{aguilera2021depression, opoku2019towards}.

%
%----------------------------------------
%
\subsection{Depression Detection via Head Motion Cues}
Many studies have focused on non-verbal behavioural cues, such as body gestures~\cite{joshi2013can, joshi2013relative}, facial expressions~\cite{bourke2010processing, he2022intelligent, de2019combining}, their combination~\cite{Parekh2018} and speech features~\cite{cummins2011investigation, rejaibi2022mfcc, huang2019investigation} as biomarkers for depression diagnosis and rehabilitation utilising computational tools~\cite{ringeval2019avec}. Head motion patterns have nevertheless received little attention. Psychological research on depression assessment has identified head motion as a significant non-verbal cue for depression with more pronounced behavioural changes in hand and head regions as compared to other body parts for depressed patients \cite{pedersen1988ethological}. Waxer \emph{et. al.}~\cite{waxer1974nonverbal} found that depressed subjects are more likely to keep their heads in a downward position and exhibit significantly reduced head nodding compared to healthy subjects \cite{fossi1984ethological}. Another study focusing on social interactions identified the reduced involvement of depressed patients in conversations, where their behaviour was characterised by lesser encouragement (head nodding and backchanneling while listening) and fewer head movements~\cite{hale1997non}.

From a computational standpoint, only a few studies have employed head pose and movement patterns for automatic depression detection. Alghowinem~\emph{et al.}~\cite{alghowinem2013head} analysed head movements by modelling statistical features extracted from the 2D Active Appearance Model (AAM) projection of a 3D face and demonstrated the efficacy of head pose as behavioural cue. Another study~\cite{joshi2013can} generated a histogram of head movements normalised over time to highlight the diminished movements of depressed patients due to psychomotor retardation, characterised by a more frequent occurrence of static head positions than in healthy controls. Several studies \cite{song2020spectral, dibekliouglu2017dynamic,cohn2018multimodal, morales2017cross} explored the utilisation of head motion as a complementary cue to other modalities to enhance detection performance. For instance, several studies~\cite{alghowinem2016multimodal, alghowinem2020interpretation} combined head pose with speech behaviour and eye gaze to develop statistical features for depression analysis. Generalisation across different cross-cultural datasets was attempted in~\cite{alghowinem2015cross} by using head pose and eye gaze based temporal features. Kacem~\emph{et. al.}~\cite{kacem2018detecting} encoded head motion dynamics with facial expressions to classify depression based on severity, while Dibeklioglu \emph{et. al.}~\cite{dibekliouglu2015multimodal} included vocal prosody in combination with head and facial movements for depression detection. 
%
% Contemporary research has focused on multimodal approaches to depression detection~\cite{cohn2018multimodal, morales2017cross}. 

% It can be noted that multimodal approaches usually combine multiple unimodal cues. Therefore, investigating depression performance based on a novel modality can be considered as a valuable research contribution. 
%
%----------------------------------------
%
\subsection{Novelty of the Proposed Approach} 
From the literature review, it can be seen that while a number of studies have employed head movements as a complementary cue in multimodal approaches, only few studies have deeply explored head motion as a rich source of information. Further, the explainability of behavioural features, especially head motion features, for depression detection has not yet been explored in the literature. This study (a) is the first to propose the use of kinemes as depression biomarkers, (b) explores multimodal cues derived from head motion behaviour as potential biomarkers for depression; specifically, we show that kinemes learned for the depressed and control classes, or only the control class enable accurate depression detection, and (c) the learned kinemes also \textit{explain} depressed behaviours consistent with prior observations.


% Additionally, we implement depression detection by discovering fundamental head pose units solely from the healthy controls data. The head pose data from the depressed class are represented on basis of these discovered kineme values.

