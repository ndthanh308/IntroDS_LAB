


%
%----------------------------------------
%
% Figure environment removed
% Figure environment removed

% Initially introduced as a challenge in 2013  \cite{valstar2013avec}, AVEC2013 contains naturalistic videos of participants performing a series of PowerPoint guided tasks in front of the computer recorded by webcam, along with depression annotations consisting of self-reported 21-item multiple-choice inventory (Beck Depression Index) \cite{beck1996comparison}. BlackDog dataset, on the other hand, is a clinically validated dataset where the participants were recorded in an interview setting with a clinician, answering open-ended questions centered around their life experiences \cite{alghowinem2016multimodal}. 

We now examine kineme patterns obtained from the depression datasets, namely \textit{BlackDog}~\cite{alghowinem2016multimodal} and \textit{AVEC2013}~\cite{valstar2013avec} (described in Sec. ~\ref{sec:datasets}). Using the \textit{Openface}~\cite{Baltrusaitis16} toolkit, we extracted \textit{yaw}, \textit{pitch} and \textit{roll} angles per frame, and segmented each video into 2s and 5s-long chunks with 50\% overlap for the AVEC2013 and BlackDog datasets, respectively. Considering $K = 16$ \cite{samanta2017role}, we extracted kinemes from both patient and healthy control segments, following the procedure outlined in Sec.~\ref{Sec:KF}. We further examined the kinemes learned for each dataset to identify the set of distinctive kinemes for the two classes. To obtain the most discriminative kinemes, we computed the relative frequency of occurrence for each kineme for the control and patient data, and selected the top five kinemes per class based on their relative frequency difference (see Sec.~\ref{Sec:approach1}).  

Selected kinemes corresponding to the maximal difference in their relative frequency of occurrence for the control and patient classes are visualised in Figures~\ref{fig:kinemes_bdi} (\textit{BlackDog}) and~\ref{fig:kinemes_avec} (\textit{AVEC2013}). Examining the control-specific kinemes in Figs.~\ref{fig:kinemes_bdi} and~\ref{fig:kinemes_avec}, we observe a greater degree of movement for healthy subjects as compared to a predominantly static head pose conveyed by the depressed patient-specific kinemes. Head nodding, characterised by pitch oscillations, and considerable roll angle variations can be noted for at least one control-class kineme; conversely, patient-specific kinemes exhibit relatively small changes over all head pose angular dimensions. These findings are reflective of reduced head movements in the depressed cohort compared to healthy individuals, which is consistent with observations made in past studies \cite{hale1997non, alghowinem2013head}. 