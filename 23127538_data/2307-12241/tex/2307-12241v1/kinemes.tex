\documentclass[sigconf, nonacm]{acmart}
%\settopmatter{authorsperrow=2}
\settopmatter{printacmref=false}

% \copyrightyear{2023}
% \acmYear{2023}
% \setcopyright{acmcopyright}\acmConference[ICMI '23]{Proceedings of the 2023
% International Conference on Multimodal Interaction}
% \acmBooktitle{Proceedings of the 2023 International Conference on Multimodal
% Interaction (ICMI '23)}
% \acmPrice{15.00}
% \acmDOI{10.1145/3462244.3479901}
% \acmISBN{978-1-4503-8481-0/21/10}

\usepackage{multirow}
\usepackage{amsmath}
\usepackage[gen]{eurosym}
\usepackage{hyperref}
\usepackage{enumitem}% http://ctan.org/pkg/enumitem
\setlist[itemize]{noitemsep, topsep=0pt}
%\usepackage[cp1251]{inputenc}
%\usepackage{selinput}
%\SelectInputMappings{%
	%agrave={à},
	%ccedilla={ç},
	%Euro={€}
%}
\newcommand*{\Perm}[2]{{}^{#1}\!P_{#2}}%
\newcommand*{\Comb}[2]{{}^{#1}C_{#2}}%
\newcommand{\TG}[1]{{\color{blue}[TG: #1]}}
\newcommand{\RS}[1]{{\color{red}[RS: #1]}}
%\usepackage{amssymb}
%\renewcommand{\baselinestretch}{1.3}
\usepackage{graphicx}
%\usepackage{subfigure}
\usepackage{caption}
\usepackage{comment}
\usepackage{subcaption}
\usepackage[utf8]{inputenc}
\usepackage[linesnumbered,ruled,vlined,algo2e]{algorithm2e}
\usepackage{multirow}
%\usepackage[algo2e]{algorithm2e} 
% \usepackage{algpseudocode}
\usepackage{subcaption}
\AtBeginDocument{%
  \providecommand\BibTeX{{%
    \normalfont B\kern-0.5em{\scshape i\kern-0.25em b}\kern-0.8em\TeX}}}
\usepackage{balance }
%% Rights management information.  This information is sent to you
%% when you complete the rights form.  These commands have SAMPLE
%% values in them; it is your responsibility as an author to replace
%% the commands and values with those provided to you when you
%% complete the rights form.
% \setcopyright{acmcopyright}
% \copyrightyear{2020}
% \acmYear{2021}
% \acmDOI{10.1145/1122445.1122456}

%% These commands are for a PROCEEDINGS abstract or paper.

\begin{document}
\def\x{{\mathbf x}}
\def\L{{\cal L}}
\def\eg{\textit{e.g.}}
\def\ie{\textit{i.e.}}
\def\Eg{\textit{E.g.}}
\def\etal{\textit{et al.}}
\def\etc{\textit{etc}}
\newcommand{\blue}[1]{\textcolor{blue}{#1}}

\title{Explainable  Depression Detection via Head Motion Patterns}

% \author{Paper ID: 1222}
%

\author{Monika Gahalawat}
\affiliation{University of Canberra}
\email{monika.gahalawat@canberra.edu.au}
%  \institution{Indian Institute of Technology, Ropar}
  %\streetaddress{1 Th{\o}rv{\"a}ld Circle}
%  \city{Ropar}
%  \country{India}}
% \email{monika.20csz0003@iitrpr.ac.in}

\author{Raul Fernandez Rojas}
\affiliation{ University of Canberra}
\email{raul.fernandezrojas@canberra.edu.au}

\author{Tanaya Guha}
\affiliation{University of Glasgow}
%  \institution{University of Warwick}
%  \streetaddress{1 Th{\o}rv{\"a}ld Circle}
%  \city{Coventry}
%  \country{United Kingdom}}
\email{tanaya.guha@glasgow.ac.uk}
% \affiliation{}

\author{Ramanathan Subramanian}
\affiliation{University of Canberra}
\email{ram.subramanian@canberra.edu.au}

\author{Roland Goecke}
\affiliation{University of Canberra}
\email{roland.goecke@canberra.edu.au}

%  \institution{Indian Institute of Technology, Ropar}
  %\streetaddress{1 Th{\o}rv{\"a}ld Circle}
 % \city{India}
 % \country{India}}
% \email{ram.subramanian@canberra.edu.au}
% \authornote{Both authors contributed equally to this research.}
% \email{trovato@corporation.com}
% \orcid{1234-5678-9012}
% \author{G.K.M. Tobin}
% \authornotemark[1]
% \email{webmaster@marysville-ohio.com}
% \affiliation{%
%  \institution{Paper id: 2344}
%   \streetaddress{P.O. Box 1212}
%   \city{Dublin}
%   \state{Ohio}
%   \postcode{43017-6221}
%  }
%


%%
%% By default, the full list of authors will be used in the page
%% headers. Often, this list is too long, and will overlap
%% other information printed in the page headers. This command allows
%% the author to define a more concise list
%% of authors' names for this purpose.
% \renewcommand{\shortauthors}{ICMI '23}

%%
%% The abstract is a short summary of the work to be presented in the
%% article.
\begin{abstract}
While depression has been studied via multimodal non-verbal behavioural cues, head motion behaviour has not received much attention as a biomarker. This study demonstrates the utility of fundamental head-motion units, termed \emph{kinemes}, for depression detection by adopting two distinct approaches, and employing distinctive features: (a) discovering kinemes from head motion data corresponding to both depressed patients and healthy controls, and (b) learning kineme patterns only from healthy controls, and computing statistics derived from reconstruction errors for both the patient and control classes. Employing machine learning methods, we evaluate depression classification performance on the \emph{BlackDog} and \emph{AVEC2013} datasets. Our findings indicate that: (1) head motion patterns are effective biomarkers for detecting depressive symptoms, and (2) explanatory kineme patterns consistent with prior findings can be observed for the two classes. Overall, we achieve peak F1 scores of 0.79 and 0.82, respectively, over BlackDog and AVEC2013 for binary classification over episodic \emph{thin-slices}, and a peak F1 of 0.72 over videos for AVEC2013.   
% Still working on the observations
\end{abstract}
\vspace{-3mm}
% \begin{CCSXML}
% <ccs2012>
%    <concept>
%        <concept_id>10003120.10003121</concept_id>
%        <concept_desc>Human-centered computing~Human computer interaction (HCI)</concept_desc>
%        <concept_significance>500</concept_significance>
%        </concept>
%    <concept>
%        <concept_id>10010147.10010178.10010224.10010240</concept_id>
%        <concept_desc>Computing methodologies~Computer vision representations</concept_desc>
%        <concept_significance>500</concept_significance>
%        </concept>
%  </ccs2012>
% \end{CCSXML}

% \ccsdesc[500]{Human-centered computing~Human computer interaction (HCI)}
% \ccsdesc[500]{Computing methodologies~Computer vision representations}
\vspace{-2mm}
\keywords{Kinemes, Head-motion, Depression detection, Explainability}

%% A "teaser" image appears between the author and affiliation
%% information and the body of the document, and typically spans the
%% page.
%\begin{teaserfigure}
%  % Figure removed
%  \caption{Seattle Mariners at Spring Training, 2010.}
%  \Description{Enjoying the baseball game from the third-base
%  seats. Ichiro Suzuki preparing to bat.}
%  \label{fig:teaser}
%\end{teaserfigure}

%%
%% This command processes the author and affiliation and title
%% information and builds the first part of the formatted document.
%\maketitle
\maketitle
\vspace{-3.5mm}
\section{Introduction}\label{Sec:Intro}
%
%
%
%
%
Clinical depression, a prevalent mental health condition, is considered as one of the leading contributors to the global health-related burden \cite{greenberg2015economic, lepine2011increasing}, affecting millions of people worldwide \cite{vos_et_al_GBD2016,institute2021global}. As a mood disorder, it is characterised by a prolonged (> two weeks) feeling of sadness, worthlessness and hopelessness, a reduced interest and a loss of pleasure in normal daily life activities, sleep disturbances, tiredness and lack of energy. Depression can lead to suicide in extreme cases \cite{goldney2000suicidal} and is often linked to comorbidities such as anxiety disorders, substance abuse disorders, hypertensive diseases, metabolic diseases, and diabetes \cite{steffen_et_al_2020_BMCPsychiatry,campayo2011diabetes}. Although effective treatment options are available, diagnosing depression through self-report and clinical observations presents significant challenges due to the inherent subjectivity and biases involved.

Over the last decade, researchers from affective computing and psychology have focused on investigating objective measures that can aid clinicians in the initial diagnosis and monitoring of treatment progress of clinical depression \cite{cohn2018multimodal, pampouchidou_et_al_TAC_DepressionReview}. A key catalyst to this progress is the availability of relevant datasets, such as AVEC2013 and subsequent challenges~\cite{valstar2013avec}. In recent years, research on depression detection employing affective computing approaches has increasingly focused on leveraging non-verbal behavioural cues such as facial expressions \cite{bourke2010processing, de2019combining}, body gestures \cite{joshi2013relative}, eye gaze \cite{alghowinem2016multimodal}, head movements \cite{alghowinem2013head} and verbal features \cite{cummins2011investigation, huang2019investigation} extracted from multimedia data to develop distinctive features to classify individuals as depressed or healthy controls, or to estimate the severity of depression on a continuous scale. 

In this study, we examine the utility of inherently interpretable head motion units, referred to as \emph{kinemes} \cite{madan_gahalawat_guha_subramanian_ICMI2021_Kinemes}, for assessing depression. Initially, we utilise data from both healthy controls and depressed patients to discover a basis set of kinemes via the (\emph{pitch}, \emph{yaw}, and \emph{roll}) head pose angular data obtained from short overlapping time-segments (termed two-class kineme discovery or 2CKD). Further, we employ these kinemes to generate features based on the frequency of occurrence of distinctive, class-characteristic kinemes. Subsequently, we discover kineme patterns solely from head pose data corresponding to healthy controls (Healthy control kineme discovery or HCKD), and use them to represent both healthy and depressed class segments. A set of statistical features are then computed from the reconstruction errors between the raw and learned head-motion segments corresponding to both the depressed and control classes (see Figure ~\ref{fig:Depression_proposed_framework}). Using machine learning methodologies, we evaluate the performance of the features derived from the two approaches. Our results show that head motion patterns are effective behavioural cues for detecting depression. Additionally, explanatory class-specific kinemes patterns can be observed, in alignment with prior research.  

% Figure environment removed

This paper makes the following research contributions:
%
\begin{itemize}
    \item A study of head movements as a biomarker for clinical depression, which so far has been understudied.
    \item Proposing the \textit{kineme} representation of motion patterns as an effective and explanatory means for depression analysis.
    \item \begin{sloppypar} A detailed investigation of various classifiers for 2-class and 4-class categorisation on the AVEC2013 and BlackDog datasets. We obtain peak F1-scores of 0.79 and 0.82, respectively, on \textit{thin-slice} chunks for binary classification on the BlackDog and AVEC2013 datasets, which compare favorably to prior approaches. Also, a video-level F1-score of 0.72 is achieved for 4-class categorisation on AVEC2013.  \end{sloppypar}
\end{itemize}
%
The remainder of this paper is organised as follows. Section \ref{Sec:RW} provides an overview of related work. Section \ref{Sec:KF} describes the kineme formulation, followed by Section \ref{Sec:EKF} that details the explainable kineme features used as a representation of motion patterns. The methodology is presented in Section \ref{Sec:Meth}, while Section \ref{Sec:ER} provides details of the datasets, experimental settings, and classifiers used in this study. The experimental results are shown and discussed in Section \ref{sec:ResultsDiscussion}. Finally, the conclusions are drawn in Section \ref{Sec:DC}.


% Add the basics of kinemes, the approach used
% Add overview of the framework implemented
% Contribution






\section{Related Work}\label{Sec:RW}

%
In this section, we briefly review the literature focusing on (a) depression detection as a classification problem, and (b) depression detection using head motion patterns.
%----------------------------------------
%
\subsection{Depression Analysis as a Classification Task}
%
Traditionally, depression detection has been approached as a supervised binary classification task, with many studies relying on discriminative classifiers to distinguish between \emph{healthy controls} and \emph{patients} \cite{alghowinem2015cross, cohn2018multimodal, alghowinem2013head}. A typical recognition accuracy of up to 80\% demonstrates the promise of behavioural cues such as eye-blink and closed-eye duration rate, statistical features computed over the yaw, pitch and roll head-pose features, \etc. to differentiate the two classes. However, challenges involved in depression detection such as limited clinically validated, curated data and a skewed data distributions have been acknowledged in the literature~\cite{alghowinem2013head, nasir2016multimodal}. 

Recent efforts have sought to learn patterns indicative of only the target class and reformulate depression detection as a one-class classification problem to mitigate the issues with imbalanced datasets~\cite{opoku2019towards, aguilera2021depression}. Studies have attempted to learn features associated with control participants and treat inputs that deviate from these patterns as \textit{anomalous}~\cite{gerych2019classifying, mourao2011patient}. Gerych \emph{et. al.}~\cite{gerych2019classifying} formulate the task as anomaly detection by leveraging autoencoders to learn features of the non-depressed class and treating depressed user data as outliers. Similarly, Mourão-Miranda \emph{et. al.}~\cite{mourao2011patient} employ a one-class SVM to classify patients as outliers compared to healthy participants based on the fMRI responses to sad facial expressions. Conversely, a few studies explore one-class classification by learning features characterising the depressed class, and treating non-depressed subjects as outliers~\cite{aguilera2021depression, opoku2019towards}.

%
%----------------------------------------
%
\subsection{Depression Detection via Head Motion Cues}
Many studies have focused on non-verbal behavioural cues, such as body gestures~\cite{joshi2013can, joshi2013relative}, facial expressions~\cite{bourke2010processing, he2022intelligent, de2019combining}, their combination~\cite{Parekh2018} and speech features~\cite{cummins2011investigation, rejaibi2022mfcc, huang2019investigation} as biomarkers for depression diagnosis and rehabilitation utilising computational tools~\cite{ringeval2019avec}. Head motion patterns have nevertheless received little attention. Psychological research on depression assessment has identified head motion as a significant non-verbal cue for depression with more pronounced behavioural changes in hand and head regions as compared to other body parts for depressed patients \cite{pedersen1988ethological}. Waxer \emph{et. al.}~\cite{waxer1974nonverbal} found that depressed subjects are more likely to keep their heads in a downward position and exhibit significantly reduced head nodding compared to healthy subjects \cite{fossi1984ethological}. Another study focusing on social interactions identified the reduced involvement of depressed patients in conversations, where their behaviour was characterised by lesser encouragement (head nodding and backchanneling while listening) and fewer head movements~\cite{hale1997non}.

From a computational standpoint, only a few studies have employed head pose and movement patterns for automatic depression detection. Alghowinem~\emph{et al.}~\cite{alghowinem2013head} analysed head movements by modelling statistical features extracted from the 2D Active Appearance Model (AAM) projection of a 3D face and demonstrated the efficacy of head pose as behavioural cue. Another study~\cite{joshi2013can} generated a histogram of head movements normalised over time to highlight the diminished movements of depressed patients due to psychomotor retardation, characterised by a more frequent occurrence of static head positions than in healthy controls. Several studies \cite{song2020spectral, dibekliouglu2017dynamic,cohn2018multimodal, morales2017cross} explored the utilisation of head motion as a complementary cue to other modalities to enhance detection performance. For instance, several studies~\cite{alghowinem2016multimodal, alghowinem2020interpretation} combined head pose with speech behaviour and eye gaze to develop statistical features for depression analysis. Generalisation across different cross-cultural datasets was attempted in~\cite{alghowinem2015cross} by using head pose and eye gaze based temporal features. Kacem~\emph{et. al.}~\cite{kacem2018detecting} encoded head motion dynamics with facial expressions to classify depression based on severity, while Dibeklioglu \emph{et. al.}~\cite{dibekliouglu2015multimodal} included vocal prosody in combination with head and facial movements for depression detection. 
%
% Contemporary research has focused on multimodal approaches to depression detection~\cite{cohn2018multimodal, morales2017cross}. 

% It can be noted that multimodal approaches usually combine multiple unimodal cues. Therefore, investigating depression performance based on a novel modality can be considered as a valuable research contribution. 
%
%----------------------------------------
%
\subsection{Novelty of the Proposed Approach} 
From the literature review, it can be seen that while a number of studies have employed head movements as a complementary cue in multimodal approaches, only few studies have deeply explored head motion as a rich source of information. Further, the explainability of behavioural features, especially head motion features, for depression detection has not yet been explored in the literature. This study (a) is the first to propose the use of kinemes as depression biomarkers, (b) explores multimodal cues derived from head motion behaviour as potential biomarkers for depression; specifically, we show that kinemes learned for the depressed and control classes, or only the control class enable accurate depression detection, and (c) the learned kinemes also \textit{explain} depressed behaviours consistent with prior observations.


% Additionally, we implement depression detection by discovering fundamental head pose units solely from the healthy controls data. The head pose data from the depressed class are represented on basis of these discovered kineme values.



\section{Kineme Formulation}\label{Sec:KF}

%
This section describes our approach to discovering a set of elementary head motion units termed \emph{kinemes} from 3D head pose angles. These head pose angles are expressed as a time-series of short overlapping segments, which enables shift invariance. The segments are then projected onto a lower-dimensional space and clustered using a Gaussian Mixture Model \cite{samanta2017role}. 

We extracted 3D head pose angles using the OpenFace tool~\cite{Baltrusaitis16} in terms of 3D Euler rotation angles, \emph{pitch} ($\theta_p$), \emph{yaw} ($\theta_y$) and \emph{roll} ($\theta_r$). The head movement over a duration $T$ is denoted as a time-series: $\boldsymbol{\theta} = \{\theta_p^{1:T}, \theta_y^{1:T}, \theta_r^{1:T}\}$. We ensure that the rotation angles remain non-negative by defining the range in [0$^{\circ}$, 360$^{\circ}$]. 

For each video, the multivariate time-series $\boldsymbol{\theta}$ is divided into short overlapping segments of length $\ell$ with overlap $\ell/2$, where the $i^{th}$ segment is represented as a vector $\mathbf{h}^{(i)} = [\theta_p^{i:i+\ell}\, \theta_y^{i:i+\ell}\, \theta_r^{i:i+\ell}]$. Considering the total number of segments in any given video as $s$, the characterisation matrix  $\mathbf{H}_{\boldsymbol\theta}$ for this video is defined as
$ \mathbf{H}_{\boldsymbol\theta} = [\mathbf{h}^{(1)}, \mathbf{h}^{(2)},\cdots, \mathbf{h}^{(s)}]$. Thus, for a training set of $N$ samples, the head motion matrix is created as $\mathbf{H} = [\mathbf{H}_{\boldsymbol\theta_1}|\mathbf{H}_{\boldsymbol\theta_2}|\cdots|\mathbf{H}_{\boldsymbol\theta_N}]$ with each column of $\mathbf{H}$ representing a single head motion time-series for a given video sample. We decompose $\mathbf{H}\in\mathbb{R}_+^{m\times n}$ into a basis matrix $\mathbf{B}\in\mathbb{R}_+^{m\times q}$ and a coefficient matrix $\mathbf{C}\in\mathbb{R}_+^{q\times n}$ using Non-negative Matrix Factorization (NMF) such that $m = 3\ell$, $n = Ns$
%The head motion matrix $\mathbf{H}$ is decomposed into two non-negative matrices  $\mathbf{B}$ and $\mathbf{C}$ as follows: 
\vspace{-1mm}
%
\begin{equation}
    \underset{\mathbf{B} \geq 0, \mathbf{C} \geq 0}{\text{ min}} \lVert{\mathbf{H} - \mathbf{B}\mathbf{C}}\rVert_F^2
\end{equation}
%
where $q \leq min(m, n)$ and $\lVert \textbf{ . }  \rVert_F$ denotes the Frobenius norm. Rather than clustering the raw head motion segments, we employ a more interpretable and stable approach by clustering the coefficient vectors in the transformed space. To this end, we learn a Gaussian Mixture Model (GMM) using the columns of the coefficient matrix $\mathbf{C}$ to produce a ${\mathbf{C}^*}\in\mathbb{R}_+^{q\times k}$ where $k << Ns$. These vectors in the learned subspace are transformed back to the original head motion subspace defined by the Euler angles using $\mathbf{H}^*=\mathbf{B}\mathbf{C}^*$. The columns of matrix $\mathbf{H}^*$ represent the set of $K$ kinemes as $\{\mathcal{K}_i\}_{i=1}^K$. 

Now, we can represent any head motion time-series $\theta$ as a sequence of kinemes discovered from the input video set by associating each segment of length $\ell$ from $\theta$ with one of the kinemes. For each $i^{th}$ segment in the time-series, we compute the characterisation vector $\mathbf{h}^{(i)}$ and project it onto the transformed subspace defined by $\mathbf{B}$ to yield $\mathbf{c}^{(i)}$ such that:\vspace{-1mm}
%
\begin{equation}
    \hat{\mathbf{c}} = \underset{\mathbf{c}^{(i)} \geq 0}{\text{arg min}} \lVert{\mathbf{h}^{(i)} - \mathbf{B}\mathbf{c}^{(i)}}\rVert_F^2
\end{equation}
%
We then maximise the posterior probability $P({K}|\hat{\mathbf{c}})$ over all kinemes to map the $i^{th}$ segment with its corresponding kineme $K^{(i)}$. In the same way, we compute the corresponding kineme label for each segment of length $\ell$ to obtain a sequence of kinemes: $\{K^{(1)} \cdots K^{(s)}\}$, where $K^{(j)}\in \mathcal{K}$ for all segments of time-series $\theta$. 

\section{Explainable Kineme Features }\label{Sec:EKF}




%
%----------------------------------------
%
% Figure environment removed
% Figure environment removed

% Initially introduced as a challenge in 2013  \cite{valstar2013avec}, AVEC2013 contains naturalistic videos of participants performing a series of PowerPoint guided tasks in front of the computer recorded by webcam, along with depression annotations consisting of self-reported 21-item multiple-choice inventory (Beck Depression Index) \cite{beck1996comparison}. BlackDog dataset, on the other hand, is a clinically validated dataset where the participants were recorded in an interview setting with a clinician, answering open-ended questions centered around their life experiences \cite{alghowinem2016multimodal}. 

We now examine kineme patterns obtained from the depression datasets, namely \textit{BlackDog}~\cite{alghowinem2016multimodal} and \textit{AVEC2013}~\cite{valstar2013avec} (described in Sec. ~\ref{sec:datasets}). Using the \textit{Openface}~\cite{Baltrusaitis16} toolkit, we extracted \textit{yaw}, \textit{pitch} and \textit{roll} angles per frame, and segmented each video into 2s and 5s-long chunks with 50\% overlap for the AVEC2013 and BlackDog datasets, respectively. Considering $K = 16$ \cite{samanta2017role}, we extracted kinemes from both patient and healthy control segments, following the procedure outlined in Sec.~\ref{Sec:KF}. We further examined the kinemes learned for each dataset to identify the set of distinctive kinemes for the two classes. To obtain the most discriminative kinemes, we computed the relative frequency of occurrence for each kineme for the control and patient data, and selected the top five kinemes per class based on their relative frequency difference (see Sec.~\ref{Sec:approach1}).  

Selected kinemes corresponding to the maximal difference in their relative frequency of occurrence for the control and patient classes are visualised in Figures~\ref{fig:kinemes_bdi} (\textit{BlackDog}) and~\ref{fig:kinemes_avec} (\textit{AVEC2013}). Examining the control-specific kinemes in Figs.~\ref{fig:kinemes_bdi} and~\ref{fig:kinemes_avec}, we observe a greater degree of movement for healthy subjects as compared to a predominantly static head pose conveyed by the depressed patient-specific kinemes. Head nodding, characterised by pitch oscillations, and considerable roll angle variations can be noted for at least one control-class kineme; conversely, patient-specific kinemes exhibit relatively small changes over all head pose angular dimensions. These findings are reflective of reduced head movements in the depressed cohort compared to healthy individuals, which is consistent with observations made in past studies \cite{hale1997non, alghowinem2013head}. 

\section{Classification Methodology}\label{Sec:Meth}

\section{Method} \label{method_hybridaugment}
In this section, we formally define the problem, motivate our work and then present our proposed techniques.


\subsection{Preliminaries}
Let $\mathcal{F}(x;W)$ be an image classification CNN trained on the training set $\mathcal{T}_\text{train} = (x_{i}, y_{i})^{N}_{i=1}$  with $N$ samples, where $x$ and $y$ correspond to images and labels. The clean accuracy (CA) of $\mathcal{F}(x;W)$ is formally defined as its accuracy over a clean test set $\mathcal{T}_\text{test} = (x_{j}, y_{j})^{M}_{j=1}$. Assume two operators ${A}(\cdot)$ and ${C}(c, s)$ that adversarially attacks or corrupts a given set of images with the corruption category $c$ and severity $s$, respectively.  Let $A\mathcal{T}_\text{test}$ and $C\mathcal{T}_\text{test}$ be the adversarially attacked and corrupted versions of $\mathcal{T}_\text{test}$, and let $\mathcal{F}(x;W)$ have a robust accuracy (RA) on $A\mathcal{T}_\text{test}$ and a corruption accuracy (CRA) on $C\mathcal{T}_\text{test}$. 
The aim is to fit $\mathcal{F}(x;W)$ such that the model gains robustness (\ie. increased RA and CRA compared its the baseline version), while retaining (or improving) the clean accuracy of its baseline version trained without robustness concerns.


\noindent \textbf{What we know.} Our work builds on the following crucial observations: i) CNNs favour high-frequency content \cite{wang2020high}, ii) adversaries and corruptions often reside in high-frequency \cite{wang2020towards}, iii) images are dominated by low-frequency \cite{Saikia_2021_ICCV} and iv) models relying on low-frequency components are more robust \cite{li2022robust,wang2020towards}. The robustness-accuracy trade-off is visible; low-frequency reliant models are more robust, but tend to miss out on clean accuracy brought by the high-frequency components. 

\subsection{HybridAugment}
We hypothesize that a \textit{sweet spot} in the robustness-accuracy trade-off can be found. Unlike the \textit{hard} approaches that completely rule out the reliance on high-frequency components (i.e. low-pass filters), we propose to \textit{reduce} the reliance on them. To this end, we adopt a data augmentation approach that aims to diversify $\mathcal{T}_\text{train}$ by an operation $\mathcal{HA(\cdot)}$. Keeping the strong relation intact between labels and low-frequency content (i.e. labels come from low-frequency-component image), we propose to swap high and low-frequency components of images in a batch on-the-fly. Unlike \cite{mukai2022improving}, we \textit{do not} restrict the images to belong to the same class; this diversifies the training distribution even further while preserving the image semantics. We call this basic version of our approach \textit{HybridAugment}, which corresponds to: 
%
\begin{equation} \label{hybrid_augment_paired}
    \mathcal{HA_{P}}(x_{i}, x_{j}) = \mathcal{LF}(x_{i}) + \mathcal{HF}(x_{j})
\end{equation}
%
where $x_{i}$ is the input image and $x_{j}$ is a randomly sampled image from the whole training set, which we simply sample from the mini batch at each training iteration in practice. $\mathcal{HF}$ and $\mathcal{LF}$ operators select the high and low-frequency components of an input image, for which we use:
%
\begin{equation} \label{eq:cutoff}
\begin{split}
    \mathcal{LF}(x) = GaussBlur(x) \\
    \mathcal{HF}(x) = x - \mathcal{LF}(x)
    \end{split}
\end{equation}
%
where $GaussBlur$ is used as a low-pass filter. Note that a similar outcome is possible by using Discrete Fourier Transforms (DFT), swapping the frequency bands and then applying Inverse DFT (IDFT). We find the gaussian blur operation to be faster and better in practice. 


Inspired from \cite{chen2021amplitude}, in addition to the image-pair scheme in Eq.~\ref{hybrid_augment_paired}, we propose a single image variant of \textit{HybridAugment}. In the single image variant, instead of combining two images, $x_i$ and $x_{j}$ are obtained by applying randomly sampled augmentations to a single image. The single image variant $\mathcal{HA_{S}}$ can therefore be defined as 
%
\begin{equation} \label{hybrid_augment_single}
    \mathcal{HA_{S}}(x_{i}) = \mathcal{LF}(Aug(x_{i})) + \mathcal{HF}(\hat{Aug}(x_{i}))
\end{equation}
%
where $Aug$ and $\hat{Aug}$ correspond to two sets of randomly sampled augmentation operations. Note that paired and single versions can work in tandem ($\mathcal{HA_{PS}}$), and actually outperform single or paired image versions. 


\subsection{HybridAugment++}


The frequency analysis is a vast literature, however, two core aspects often stand out; frequency-band analysis (i.e. low, high) and the decomposition of signals into amplitude and phase. \textit{HybridAugment} covers the former and shows competitive results in various benchmarks (see Section \ref{sec:exp_hybridaugment}). The latter is investigated in $\mathcal{APR}$ \cite{chen2021amplitude}, where phase is shown to be the more relevant component for correct classification, and training models based on their phase labels and swapping amplitude components of images randomly lead to more robust models. Note that frequency-band and phase/amplitude discussions are arguably orthogonal, since frequency, phase and amplitude provide distinct characterizations of a signal: intuitively speaking, frequency, phase and amplitude can be seen as the separation of visual patterns in terms of scale, location and significance. 


We hypothesize these two approaches can be complementary; a model reliant on low-frequency and spatial information (i.e. phase) can further improve robustness. Inspired by the successes of cascaded augmentation methods \cite{hendrycks2019augmix,wang2021augmax,calian2022defending}, we unify these two core aspects into a single, hierarchical augmentation method. We refer to this method as \textit{HybridAugment++} and define its paired version as:
%
\begin{equation}
  \mathcal{HA_{P}}^{++}(x_{i}, x_{j}, x_{z}) = \mathcal{APR_{P}}(\mathcal{LF}(x_{i}), x_{z}) + \mathcal{HF}(x_{j})
\end{equation}
%
where $x_{i}$, $x_{j}$ and $x_{z}$ are images sampled from the same batch. Here, $\mathcal{APR_{P}}$~\cite{chen2021amplitude} is defined as
\begin{equation}
    \mathcal{APR_{P}}(x_{i}, x_{z}) = \mathcal{IDFT}(A_{x_{z}} \otimes e^{i. P_{x_{i}}}) \\
\end{equation}
%
where $\otimes$ is element-wise multiplication, $A$ is the amplitude and $P$ is the phase component. Similar to $\mathcal{HA}$ and $\mathcal{APR}$, we also define a single-image version of \textit{HybridAugment++} as
%
\begin{equation}
 \mathcal{HA_{S}}^{++}(x_{i}) = \mathcal{APR_{S}}(\mathcal{LF}(Aug(x_{i}))) + \mathcal{HF}(\hat{Aug}(x_{i}))
\end{equation}
%
where $\mathcal{APR_{S}}$~\cite{chen2021amplitude} is defined as
%
\begin{equation}
\mathcal{APR_{S}}(x_{i}) = \mathcal{IDFT}\left(A_{\bar{Aug}(x_{i})} \otimes e^{i. P_{\overline{Aug}\left(x_{i}\right)}}\right)    
\end{equation}
%
where $Aug$, $\hat{Aug}$, $\bar{Aug}$ and $\overline{Aug}$ are different sets of randomly sampled augmentation operations. Note that we essentially propose a framework; one can use different single and paired image augmentations, either individually or together, and can still achieve competitive results (see ablations in Section \ref{sec:exp_hybridaugment}). There are also other alternatives, such as swapping phase/amplitude first and then performing $\mathcal{HA}$, but we observe poor performance in practice; dividing the phase component into frequency-bands is not interpretable as frequencies of the phase component are not well defined. The pseudo-code of our methods can be found in the supplementary material.





\section{Experiments }\label{Sec:ER}

%
%----------------------------------------
%
\begin{sloppypar}
We perform binary classification on the BlackDog and AVEC2013 datasets, plus 4-class classification on AVEC2013. This section details our datasets, experimental settings and learning algorithms.
\end{sloppypar}

%
%----------------------------------------
%
\subsection{Datasets}
\label{sec:datasets}
We examine two datasets in this study: clinically validated data collected at the Black Dog Institute -- a clinical research facility focusing on the diagnosis and treatment of mood disorders such as anxiety and depression (referred to as \emph{BlackDog} dataset) -- and the \emph{Audio/Visual Emotion Challenge} (AVEC2013) depression dataset. 

\textbf{BlackDog Dataset~\cite{alghowinem2016multimodal}:} This dataset comprises responses from healthy controls and depression patients selected as per the criteria outlined in the Diagnostic and Statistic Manual of Mental Disorders (DSM-IV). Healthy controls with no history of mental illness and patients diagnosed with severe depression were carefully selected~\cite{alghowinem2016multimodal}. For our analysis, we focus on the structured interview responses in~\cite{alghowinem2016multimodal}, where participants answered open-ended questions about life events, designed to elicit spontaneous self-directed responses, asked by a clinician. 
%These videos capture subject responses to questions about life events, designed to elicit spontaneous self-directed responses. 
In this study, we analyse video data from 60 subjects (30 depressed patients and 30 healthy controls), with interview durations ranging from $183-1200s$. 

\textbf{AVEC2013 Dataset~\cite{valstar2013avec}:} Introduced for a challenge in 2013, this dataset is a subset of the audio-visual depressive language corpus (AViD-corpus) comprising 340 video recordings of participants performing different PowerPoint guided tasks detailed in~\cite{valstar2013avec}. The videos are divided into three nearly equal partitions (training, development, and test) with videos ranging from $20-50 min$. Each video frame depicts only one subject, although some participants feature in multiple video clips. The participants completed a multiple-choice inventory based on the Beck Depression Index (BDI)~\cite{beck1996comparison} with scores ranging from 0 to 63 denoting the severity of depression. For binary classification, we dichotomise the recordings into the non-depressed and depressed cohorts as per the BDI scores. Subjects with a BDI score $\leq 13$ are categorised as \emph{non-depressed}, while the others are considered as \emph{depressed}. 

\textbf{AVEC2013 Multi-Class Classification:} For fine-grained depression detection over the AVEC2013 dataset, we categorise the dataset based on the BDI score into four classes as detailed below: 
\begin{itemize}
    \item Nil or minimal depression: BDI score 0 - 13
    \item Mild depression: BDI score 14 - 19
    \item Moderate depression: BDI score 20 - 28
    \item Severe depression: BDI score 29 - 63
\end{itemize}


%
%----------------------------------------
%
\subsection{Experimental Settings}
\textbf{Implementation Details:} For binary classification, we evaluate performance for the smaller \emph{BlackDog} dataset via 5-repetitions of 10-fold cross-validation (10FCV). For the AVEC2013, the pre-partitioned train, validation and test sets are employed. We utilise the validation sets for fine-tuning classifier hyperparameters.

\textbf{Chunk vs Video-level Classification:} The videos from both datasets are segmented into smaller chunks of $15s - 135$s length, to examine the influence of \emph{thin-slice} chunk duration on the classifier performance. We repeated the video label for all chunks and metrics are computed over all chunks for chunk-level analysis. Additionally, video-level classification results are obtained by computing the majority label over all video chunks in the test set. 

\textbf{Performance Measures:} For the BlackDog dataset, results are shown as $\mu \pm \sigma$ values over 50 runs ($5\times$ 10FCV repetitions). For AVEC2013, performance on the test set is reported. For both, we evaluate performance via the accuracy (Acc), weighted F1 (F1), precision (Pr), and recall (Re) metrics. The weighted F1-score denotes the mean F1-score over the two classes, weighted by class size.

%
% We discover kinemes as mentioned in Sec.~\ref{Sec:approach2} using the minimally depressed participants of the AVEC2013 dataset and use these learned kinemes to represent the head pose values from all classes. After computing the reconstruction error between the raw and learned vectors, we employ the 8 statistical features over the 3 angular dimensions ($8 \times 3$) for classification using the pre-partitioned sets of train, test and development data. The results for AVEC2013 multi-class classification are presented in Sec.~\ref{sec: 4class_res}.


%
%----------------------------------------
%
\subsection{Classification Methods}
Given that our proposed features do not model spatial or temporal correlations, we employ different machine learning models for detecting depression as described below:
%
\begin{itemize}
    \item \textbf{Logistic Regression (LR)}, a probabilistic classifier that employs a sigmoid function to map input observations to binary labels. We utilise extensive grid-search to fine-tune parameters such as penalty $\in \{ l1, l2, None \}$ and regulariser $\lambda \in \{ 1e^{-6}, \cdots, 1e^{3}\}$.
    \item\textbf{Random Forest (RF)}, where multiple decision trees are generated from training data whose predictions are aggregated for labelling. Fine-tuned parameters include the number of estimators $N \in [2, \cdots,  8]$, maximum depth $\in [3, \cdots,  7]$, and maximum features in split $\in [3, \cdots,  7]$.
    \item \textbf{Support Vector Classifier (SVC)}, a discriminative classifier that works by transforming training data to a high-dimensional space where the two classes can be linearly separated via a hyperplane. For SVC, we examine different kernels $\in \{ rbf, poly, sigmoid\}$ and fine-tune regularisation parameter $C \in \{ 0.1, 1, 10, 100 \}$ and kernel coefficient $\gamma \in \{ 0.0001, \cdots, 1, scale, auto \}$. 
    \item \textbf{Extreme Gradient Boosting (XGB)}, a model built upon a gradient boosting framework, and focused on improving a series of weak learners by employing the gradient descent algorithm in a sequential manner. The fine-tuned hyperparameters include the number of estimators $\{ 50, 100, 150 \}$, maximum depth $\in [3, \cdots,  7]$ of the tree and learning rate $\in [0.0005, \cdots,  0.1]$.
    \item \textbf{Multi Layer Perceptron (MLP)}, where we employed a feed-forward neural network with two hidden dense layers comprising 12 and 6 neurons, resp., with a rectified linear unit (ReLU) activation. For training, we employ categorical cross-entropy as the loss function and fine-tune the following hyperparameters: learning rate $\in \{ 1e^{-4}, 1e^{-3}, 1e^{-2}\}$, and batch size $\in \{ 16, 24, 32, 64 \}$. We utilise the Adam optimiser for updating the network weights during training.
\end{itemize}


%
%--------------------------------------------------
%
\section{Results and Discussion}\label{sec:ResultsDiscussion}
% \vspace{-8mm}

% \vspace*{1mm}
\begin{table*}[ht]
 \caption{Chunk and Video-level classification results on the BlackDog dataset with the 2CKD and HCKD approaches. Accuracy (Acc), F1, Precision (Pr) and Recall (Re) are tabulated as ($\mu \pm \sigma$) values. } \vspace{-2mm}
% \begin{tabular}{|l l||cc||cc|} 
% \hline
%     \bf Classifier & \bf Features & \multicolumn{2}{c||}{\textbf{Chunk-level}} & \multicolumn{2}{c|}{\textbf{Video-level}} \\ 
%     & & \textbf{Acc} & \textbf{F1} & \textbf{Acc} & \textbf{F1} \\ 
%     \hline \hline

%     \textbf{Logistic Regression} & 2CKD & 0.60 ± 0.15 & 0.61 ± 0.14 & 0.60 ± 0.20 & 0.59 ± 0.21 \\ 
%     \textbf{Random Forest}& 2CKD & 0.58 ± 0.13 & 0.60 ± 0.12 &0.61 ± 0.19 & 0.62 ± 0.19\\ 
%     \textbf{SV Classifier}& 2CKD & 0.60 ± 0.15 & 0.62 ± 0.15 & 0.62 ± 0.19 & 0.63 ± 0.19 \\
%     \textbf{Logistic Regression}& HCKD  & 0.77 ± 0.13 & 0.78 ± 0.12 & 0.79 ± 0.16 & 0.78 ± 0.17 \\ 
%     %\hline
%     \textbf{Random Forest}& HCKD &0.71 ± 0.13 & 0.73 ± 0.12 &0.76 ± 0.15 & 0.76 ± 0.16\\ 
%     %\hline
%     \textbf{SV Classifier}& HCKD & \textbf{0.78 ± 0.14} & \textbf{0.79 ± 0.13} & \textbf{0.80 ± 0.18} & \textbf{0.80 ± 0.19} \\ 
%     %\hline
%     \textbf{XG Boost}& HCKD & 0.72 ± 0.13 & 0.72 ± 0.12 & 0.78 ± 0.17 & 0.78 ± 0.17 \\ 
%     %\hline
%     \textbf{MLP Classifier}& HCKD & 0.75 ± 0.13 & 0.76 ± 0.12 & 0.76 ± 0.16 & 0.76 ± 0.16 \\
%     \hline
% \end{tabular} 
%\vspace{-2mm}

%reformatted table
\begin{tabular}{|l l||cc cc||cc cc|} 
\hline
    \bf Condition & \bf Classifier    & \multicolumn{4}{c||}{\textbf{Chunk-level}} & \multicolumn{4}{c|}{\textbf{Video-level}}    \\ 
     & & \textbf{Acc} & \textbf{F1} & \textbf{Pr} & \textbf{Re} & \textbf{Acc} & \textbf{F1} & \textbf{Pr} & \textbf{Re}\\ 
    \hline\hline
    \multirow{5}{*}{\bf 2CKD}& \textbf{LR} &  0.60±0.15 & 0.61±0.14 &  0.67±0.22 & 0.65±0.22 & 0.60±0.20 & 0.59±0.21 &  0.55±0.30 & 0.65±0.33  \\ 
    & \textbf{RF} & 0.58±0.13 & 0.60±0.12 &  0.67±0.21 & 0.61±0.21  &0.61±0.19 & 0.62±0.19 &  0.59±0.32 & 0.59±0.32 \\ 
    & \textbf{SVC} & 0.60±0.15 & 0.62±0.15&  0.68±0.25 & 0.62±0.25  & 0.62±0.19 & 0.63±0.19 &  0.61±0.32 & 0.59±0.33 \\
    %\hline
    & \textbf{XGB}& 0.55±0.17 & 0.54±0.16 &  0.63±0.21 & 0.71±0.22  & 0.53±0.17 & 0.50±0.20 &  0.54±0.23 & 0.79±0.21 \\ 
    %\hline
    & \textbf{MLP} & 0.53±0.15 & 0.52±0.17 &  0.60±0.22 & 0.71±0.21 & 0.51±0.20 & 0.47±0.21 &  0.53±0.27 & 0.74±0.32 \\ \hline
    \multirow{5}{*}{\bf HCKD} & \textbf{LR}  & 0.77±0.13 & 0.78±0.12 &  0.85±0.19 & 0.74±0.21 & 0.79±0.16 & 0.78±0.17 &  0.81±0.30 & 0.66±0.31 \\ 
    %\hline
    & \textbf{RF} &0.71±0.13 & 0.73±0.12&  0.75±0.25 & 0.71±0.20  &0.76±0.15 & 0.76±0.16 &  0.75±0.26 & 0.82±0.25 \\ 
    %\hline
    & \textbf{SVC} & {0.78±0.14} & \textbf{0.79±0.13}&  0.87±0.18 & 0.74±0.20  & {0.80±0.18} & \textbf{0.80±0.19} &  0.83±0.30 & 0.70±0.31 \\ 
    %\hline
    & \textbf{XGB}& 0.72±0.13 & 0.72±0.12&  0.75±0.18 & 0.81±0.15  & 0.78±0.17 & 0.78±0.17 &  0.74±0.27 & 0.82±0.27  \\ 
    %\hline
    & \textbf{MLP} & 0.75±0.13 & 0.76±0.12 &  0.78±0.21 & 0.81±0.21 & 0.76±0.16 & 0.76±0.16 &  0.74±0.29 & 0.77±0.28 \\
    \hline
\end{tabular} 
\vspace{-2mm}
\label{tab:BDI_res1}
\end{table*}
%
% \vspace*{5mm}
%
\begin{table*}[ht]
\caption{\label{tab:AVEC_res1} Chunk and Video-level classification results on the AVEC2013 dataset with the 2CKD and HCKD approaches. Accuracy (Acc), F1, Precision (Pr) and Recall (Re) are tabulated as ($\mu \pm \sigma$) values.}
\begin{tabular}{|l l||cc cc||cc cc|} 
    \hline
    \bf Condition & \bf Classifier    & \multicolumn{4}{c||}{\textbf{Chunk-level}} & \multicolumn{4}{c|}{\textbf{Video-level}}    \\ 
     & & \textbf{Acc} & \textbf{F1} & \textbf{Pr} & \textbf{Re} & \textbf{Acc} & \textbf{F1} & \textbf{Pr} & \textbf{Re}\\ 
    \hline\hline

   \multirow{5}{*}{\bf 2CKD}&  \textbf{LR} & 0.58 & 0.58  & 0.54 & 0.65 & 0.61 & 0.61 & 0.57 & 0.71   \\ 
    %\hline
    & \textbf{RF} & 0.61 & 0.61 & 0.57 & 0.59 & 0.72 & 0.72 & 0.67 & 0.82 \\ 
    %\hline
    & \textbf{SVC} & 0.61 & 0.61 & 0.57 & 0.63  & 0.64 & 0.64 & 0.61 & 0.65 \\ 
    & \textbf{XGB} & 0.59 & 0.58 & 0.57 & 0.44 & 0.67 & 0.67 & 0.65 & 0.65 \\ 
    %\hline
    & \textbf{MLP}    & 0.56 & 0.56 & 0.52 & 0.60 & 0.58 & 0.58 & 0.65 & 0.65 \\ \hline
    \multirow{5}{*}{\bf HCKD} & \textbf{LR} & 0.80 & 0.80 & 0.77 & 0.81  & 0.94  & 0.94 & 0.94  & 0.94\\ 
    %\hline
    & \textbf{RF} & 0.78 & 0.78 & 0.78 & 0.75   & {1.00} & \textbf{1.00} & 1.00 & 1.00 \\ 
    %\hline
    & \textbf{SVC} & {0.82} & \textbf{0.82} & 0.83 & 0.77  &  {1.00} & \textbf{1.00} & 1.00 & 1.00 \\ 
    %\hline
    & \textbf{XGB} & 0.80 & 0.80  & 0.79 & 0.77 & {1.00} & \textbf{1.00} & 1.00 & 1.00\\ 
    %\hline
    & \textbf{MLP}    & 0.81 & 0.80 & 0.80 & 0.77 & {1.00} & \textbf{1.00} & 1.00 & 1.00\\
    \hline
\end{tabular}
% \vspace{-2mm}
\end{table*}
%
% \vspace*{5mm}
\begin{table*}[ht]
 \caption{Comparison with prior works for the two datasets.} \vspace{-2mm}
\begin{tabular}{|l l l||cc cc|} 
\hline
    \bf Dataset & \bf Methods  & \bf Features    & \multicolumn{4}{c|}{\textbf{Evaluation metrics}}     \\ 
     & & &\textbf{Acc} & \textbf{F1} & \textbf{Pr} & \textbf{Re}\\ 
    \hline\hline
    \multirow{4}{*} {\textbf{BlackDog}} & Alghowinem \etal \ \cite{alghowinem2013head} & Head movement &  - & - &  - & 0.71  \\ 
     & Joshi \etal \ \cite{joshi2013can} & Head movement &  0.72 & - &  - & -    \\ 
     & Ours (Chunk-level) & Kinemes &  \textbf{0.75} & {0.76} & { 0.78 }& \textbf{0.81}    \\
    & Ours (Video-level) & Kinemes &  \textbf{0.80} & {0.80} & { 0.83} &{0.70}  \\  \hline
    \multirow{4}{*}{\textbf{AVEC2013}}  & Senoussaoui \etal \ (AVEC2014)~\cite{senoussaoui2014model} & Video features &  0.82 & - &  - & - \\ 
    & Al-gawwam~\etal \ (AVEC2014 - Northwind) \cite{al2018depression} & Eye Blink &  0.85 & - &  - & -    \\ 
    & Al-gawwam~\etal \ (AVEC2014 - Freeform) \cite{al2018depression} & Eye Blink &  0.92 & - &  - & -    \\ 
     & Ours (AVEC2013 at chunk-level) & Kinemes &  0.82 & 0.82 &  0.83 & 0.87    \\
    & Ours (AVEC2013 at Video-level) & Kinemes &  \textbf{1.00} & {1.00} & {1.00} & {1.00}\\  \hline
\end{tabular} 
\vspace{-2mm}
\label{tab:Comp}
\end{table*}

\begin{table*}[ht]
% \hspace*{5mm}
\caption{\label{tab:AVEC_res_4class} Video-level 4-class categorization results on the AVEC dataset obtained with the HCKD approach. Accuracy (Acc), F1, Precision (Pre) and Recall (Re) are tabulated.} 
\vspace{-2mm}
\begin{tabular}{|l l||cc cc|} 
    \hline
    \bf Condition & \bf Classifier     & \multicolumn{4}{c|}{\textbf{Video-level}} \\ 
               & & \textbf{Acc} & \textbf{F1} & \textbf{Pr} & \textbf{Re} \\ 
    \hline\hline

    \multirow{5}{*}{\bf HCKD} & \textbf{LR}  & 0.71 & \textbf{0.72} & 0.73 & 0.71 \\ 
    %\hline
    & \textbf{RF}  & {0.74} & \textbf{0.72}  & 0.80 & 0.74 \\ 
    %\hline
    &  \textbf{SVC} & {0.74} & \textbf{0.72} & 0.75 & 0.74 \\ 
    %\hline
    &  \textbf{XGB}  & 0.71 & 0.69  & 0.68 & 0.71 \\ 
    %\hline
    &  \textbf{MLP}  & 0.69 & 0.66  & 0.64 & 0.69 \\
    \hline
\end{tabular}\vspace{-2mm}
\end{table*}

% \subsection{Results and Discussion}

Table ~\ref{tab:BDI_res1} shows the classification results obtained for the \emph{BlackDog} dataset with the 2CKD and HCKD approaches (Section~\ref{Sec:Meth}). Table ~\ref{tab:AVEC_res1} presents the corresponding results for the \emph{AVEC2013} dataset. These tables present classification measures obtained at the \emph{chunk-level} (best results achieved over $15 - 135s$-long chunks for the two datasets are presented), and the \emph{video-level} (label derived upon computing the mode over the chunk-level labels). Based on these results, we make the following observations:
%
\begin{itemize}
    \item It can be noted from Tables~\ref{tab:BDI_res1} and~\ref{tab:AVEC_res1} that relatively lower accuracies and F1 scores are achieved for both datasets using the 2CKD approach, implying that while class-characteristic kinemes are explanative as seen from Figs.~\ref{fig:kinemes_bdi} and~\ref{fig:kinemes_avec}, they are nevertheless not discriminative enough to effectively distinguish between the two classes. 
    \item In comparison, we note far superior performance with the HCKD method over all classifiers. As a case in point, we obtaine peak chunk-level F1-scores of 0.79 and 0.62, resp., for HCKD and 2CKD on BlackDog, while the corresponding F1-scores are 0.82 and 0.61, resp., on AVEC. This observation reveals considerable and distinguishable differences in the reconstruction errors for the patient and control classes, and convey that patient data are characterised as \emph{anomalies} when kinemes are only learned from the control cohort.
    \item Examining the HCKD precision and recall measures for both datasets, we note higher precision than recall at the chunk-level for the BlackDog dataset. Nevertheless, higher recall is achieved at the video-level with multiple classifiers. Likewise, higher chunk-level precision is noted for AVEC, even if ceiling video-level precision and recall are achieved.
    \item Comparing HCKD chunk and video-level F1-scores for both datasets, similar or higher video-level F1 values can be seen in Table~\ref{tab:BDI_res1}. F1-score differences are starker in Table \ref{tab:AVEC_res1}, where video-level scores are considerably higher than chunk-level scores. These results suggest that aggregating observations over multiple thin-slice chunks is beneficial and enables more precise predictions as shown in~\cite{madan_gahalawat_guha_subramanian_ICMI2021_Kinemes}.
    % This highlights that these selected feature sets are able to detect the two class differences over longer time-duration for the AVEC dataset.
    \item Examining measures achieved with the different classifiers, the support vector classifier achieves the best chunk-level F1-score on both datasets, with the LR classifier performing very comparably. All classifiers achieve very similar performance when video-level labels are compared.
     % \item Comparing against the depression detection model proposed in \cite{joshi2013can}, where head movement is analysed by estimating the displacement of rigid facial points over the BlackDog dataset with an accuracy of 71.7\%, we note that kineme discovery using healthy controls data approach achieves a better performance (80\%) highlighting the efficacy of considering depressed head pose values as anomalous data for the representation discovered solely using healthy subject data.
\end{itemize}
%


%
%----------------------------------------
%
\subsection{Comparison with the state-of-the-art} Our best results are compared against prior classification-based depression detection studies in Table~\ref{tab:Comp}. For the BlackDog dataset, Alghowinem \emph{et al.}~\cite{alghowinem2013head} analysed statistical functional features extracted from a 2D Active Appearance Model, whereas Joshi \emph{et al.}~\cite{joshi2013can} computed a histogram of head movements by estimating the displacement of fiducial facial points. Compared to \textit{N}-average recall of 0.71 reported in~\cite{alghowinem2013head}, and an accuracy of 0.72 noted in~\cite{joshi2013can}, our kineme-based approach achieves better chunk and video-level accuracies (0.75 and 0.80, resp.), and  superior chunk-level recall (0.81). As most previous studies on the AVEC2013 dataset focus on continuous prediction, we compare our model's performance with the AVEC2014~\cite{valstar2014avec} results examining visual features. 

AVEC2014 used the same subjects as AVEC2013, but with additional, specific task data (\emph{Northwind}, \emph{Freeform}) extracted from the AViD videos. For video analysis, Senoussaoui \etal~\cite{senoussaoui2014model} extracted LGBP-TOP features from frame blocks to obtain an accuracy of 0.82 using an SVM classifier. On the other hand, Al-gawwam \etal~\cite{al2018depression} extracted eye-blink features from video data using a facial landmark tracker to achieve an accuracy of 0.92 for the \emph{Northwind} task and 0.88 for the \emph{Freeform} task. Comparatively, our work achieves an accuracy of 0.82 at the chunk-level and 1.00 at the video-level. The next section will detail the performance of a more fine-grained 4-class categorisation on the AVEC2013 dataset.

%
%----------------------------------------
%
\subsection{\textbf{AVEC2013 Multi-class Classification}} 
\label{sec: avec_4class_res}
Table~\ref{tab:AVEC_res_4class} depicts video-level 4-class classification results achieved on the AVEC2013 dataset via the HCKD approach. The 4-class categorisation was performed to further validate the correctness of the HCKD approach, which produces ceiling video-level F1, Precision and Recall measures on AVEC2013 in binary classification. Results are reported on the test set, upon fine-tuning the classifier models on the development set. Reasonably good F1-scores are achieved even with 4-class classification, with a peak F1 of 0.72 obtained with the LR, RF and support vector classifiers. Cumulatively, our empirical results confirm that kinemes encoding atomic head movements are able to effectively differentiate between (a) the patient and control classes, and (b) different depression severity bands. 

% Figure environment removed

%
%----------------------------------------
%

% Figure environment removed

%
%----------------------------------------
%
\subsection{\textbf{Ablative Analysis over Thin Slices}}
\label{sec:Ablative_ts}
Tables~\ref{tab:BDI_res1} and~\ref{tab:AVEC_res1} evaluate detection performance over (\emph{thin-slice}) chunks or short behavioural episodes, and over the entire video, on the BlackDog and AVEC2013 datasets. We further compared labelling performance at the chunk and video-levels using chunks spanning $15-135s$. The corresponding results are presented in Figure~\ref{fig:Chunk_vs_Vid_kineme}. For both plots presented in the figure, the dotted curves denote video-level F1-scores, while solid curves denote chunk-level scores obtained for different classifiers.

For the BlackDog dataset (Fig.~\ref{fig:Chunk_vs_Vid_kineme} (left)), longer time-slices (of length $75-105s$) achieve better performance than shorter ($15-60s$ long) ones at both the chunk and video-levels across all classifiers; these findings are consistent with the finding that more reliable predictions can be achieved with longer observations in general~\cite{madan_gahalawat_guha_subramanian_ICMI2021_Kinemes}. However, a performance drop is noted for very long chunk-lengths of $120-135s$ duration. Decoding results on the AVEC2013 dataset, consistent with Table~\ref{fig:kinemes_avec} results, a clear gap is noted between the chunk and video-level results, with the latter demonstrating superior performance. Very similar F1-scores are observed across classifiers for various chunk lengths. No clear trends are discernible from video-level F1-scores obtained with different chunk-lengths, except that the performance in general decreases for all classifiers with very long chunks.

% analysing the chunk level performance for the AVEC2013 dataset, it can be seen that F1-scores are comparable over different chunk duration, with slight variation between different chunk-sizes.  Comparatively, better performance is achieved for video-level results of the dataset over shorter time slices, which declines with an increase in time length.
% In addition, comparing the chunk and video level performance over varying time lengths, it can be observed that video level classification achieves better F1-scores over both datasets demonstrating that combining the outcomes of smaller episodes can result in a better performance, despite potential inconsistencies in shorter time slices.  


%
%----------------------------------------
%
\subsection{\textbf{Ablative Analysis over Angular Dimensions}}\label{sec:Ablative_ad}
To investigate the impact of the head pose angular dimensions on chunk-level binary depression detection performance, we perform detection utilising ($8 \times 1$) statistical features over each of the (pitch, yaw, and roll) angular dimensions, and concatenate features ($8 \times 2$) for the dimensional pairs to evaluate which angular dimension(s) are more informative.

Figure~\ref{fig:descriptive_comp} presents F1-scores obtained with the different classifiers for uni-dimensional and pairwise-dimensional features. On the BlackDog dataset, a combination of the pitch and yaw-based descriptors produce the best performance across all models, while roll-specific descriptors perform worst. For the AVEC2013 dataset, pitch-based descriptors achieve excellent performance across models. The F1-scores achieved with these features are very comparable to the pitch + yaw and pitch + roll combinations. Here again, roll-specific features achieve the worst performance. Cumulatively, these results convey that pitch is the most informative head pose dimension, with roll being the least informative. With respect to combinations, the pitch + yaw combination in general produces the best results. These results again confirm that responsiveness in social interactions, as captured by pitch (capturing actions such as head nodding) and yaw (capturing head shaking), provides a critical cue for detecting depression, consistent with prior studies ~\cite{hale1997non, alghowinem2013head}.






\section{Conclusion}\label{Sec:DC}

% We know that we are awesome! :D
% Depression is a severe mental illness that not only affects an individual, but also has global social and economical implications. 
In this paper, we demonstrate the efficacy of elementary head motion units, termed \emph{kinemes}, for depression detection by utilising two approaches: (a) discovering kinemes from data of both patient and control cohorts, and (b) learning kineme patterns solely from the control cohort to compute statistical functional features derived from reconstruction errors for the two classes. Apart from effective depression detection, we also identify explainable kineme patterns for the two classes, consistent with prior research. 

Our study demonstrates the utility of head motion features for detecting depression, but our experiments are restricted to classification tasks involving a discretisation of the depression scores. In the future, we will investigate (a) the utility of kinemes for continuous prediction (regression) of depression severity, (b) the cross-dataset generalisability of models trained via kinemes, and (c) the development of multimodal methodologies combining kinemes with other behavioural markers, and evaluating their efficacy.
% \vspace{-4mm}
%\section*{Acknowledgement}
%Thanks to A. Samanta, IIT Kanpur for sharing the kineme code.
%This research was supported partially by the Australian Government through the Australian Research Council’s Discovery Projects funding scheme (project DP190101294).

% \section{Acknowledgement}
% This study was partially supported by research grant XXX-XXXX.


\bibliographystyle{ACM-Reference-Format}
\bibliography{references}


\end{document}