\section{introduction}

\textit{``People currently think of text as information to be consumed. I want text to be used as an environment to think in.'' --- Bret Victor~\cite{victor2011explorable}}

\ \\
Today's textbooks, whether in digital or physical format, primarily consist of \textit{static explanations} that only allow users to passively consume information, without facilitating a deep understanding that comes from interactive dialogue and exploration.
The concept of \textit{explorable explanations}~\cite{victor2011explorable} has been developed to shift this paradigm, transforming text into an interactive medium to think in, rather than just a source of information to be consumed. Such dynamic and interactive explanations have enormous potential as learning aids, as they foster deeper understanding of abstract concepts through engaging and playful explorations, which is otherwise difficult to achieve through static textbooks alone. Therefore, explorable explanations have emerged as a promising approach to changing the way we learn complex concepts across various educational domains, such as math~\cite{GeoGebra95:online}, physics~\cite{perkins2006phet}, programming~\cite{guo2013online, victor2012learnable, suzuki2017tracediff}, data science~\cite{conlen2022fidyll}, and machine learning~\cite{distill, wang2020cnn, smilkov2017direct}.

However, creating these explorable explanations often requires a significant amount of time and effort, as well as substantial programming expertise. This prevents non-technical users, such as teachers or students, from creating their own interactive content tailored to their specific needs and context. As a result, they are often limited to passively utilizing existing interactive content available on the internet, while the majority of their own textbooks remain static.

In this paper, we propose \system{}, a new machine learning (ML) enabled approach to creating explorable explanations by \textit{augmenting} static math textbooks, rather than programming them from scratch. 
Our system first scans a static document to extract values, symbols, and graphs using optical character recognition (OCR) and computer vision techniques. The system then localizes the position of detected math formulas and graphs, converting them into the computer understandable mathematical expression. 
The user can then author interactive content by binding and manipulating these extracted elements.
These interactive visual outputs are embedded and overlaid on the scanned document either through mobile AR for printed paper or through a desktop interface for scanned PDFs.
This empowers non-technical end users to transform their own textbooks into on-demand, personalized, explorable explanations without requiring programming knowledge. 

To design our system, we first analyze the common design strategies widely used in the existing interactive math explanations available on various websites. We identify three high-level categories of these strategies: 1) exemplify through concrete values, 2) visualize through interactive and animated graphs, and 3) guide through contextual hints or exercise.
Based on these design strategies, we develop a set of augmentation techniques for our proof-of-concept system, including 1) dynamic values, 2) interactive figures, 3) relationship highlights, 4) concrete examples, and 5) step-by-step hints. 
These features are designed to be automatically generated based on the extracted content, while covering a wide variety of strategies identified through our analysis. 

To evaluate our system, we conduct two user studies. First, we perform an exploratory study with eleven participants, comparing our system across three conditions: 1) static paper, 2) web-based interface, and 3) mobile AR. The study results confirm that the AR-based interface is the most engaging, while the web interface provides a better system usability score (SUS). Qualitative feedback informs us that the system helps them understand mathematical concepts through interactive and playful exploration, compared to static textbooks.
In the second study, we conducted expert interviews with five math instructors to quantitatively compare our approach to existing learning resources and practices, such as handouts, videos, and existing interactive websites. In addition, we investigate how our tool could fit their potential needs and practices, seeking the opportunity for real-world use scenarios. Their feedback suggests that our approach has great potential as an on-demand and personalized learning assistant tailored for their specific context, which is not well-supported with the currently available learning practices. Based on their feedback and insights, we discuss how to expand our proposed approach beyond the current proof-of-concept prototype for future deployment.

Finally, this paper contributes to the followings: 
\begin{enumerate}
\item \system{}, a novel machine learning-based approach to creating explorable explanations by augmenting static math textbooks.
\item A set of features that consist of our augmented math textbooks, informed by a taxonomy analysis of existing practices.
\item Results and insights from preliminary user testing and expert interviews.
\end{enumerate}
