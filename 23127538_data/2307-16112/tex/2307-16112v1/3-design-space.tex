\section{Design Strategies of Existing Explorable Math Explanations}

\subsection{Design Strategy Analysis}
To design our authoring interface, we first try to understand the common techniques and design strategies employed in existing explorable explanations. 

 

\subsubsection{Motivation}
Although previous work, such as \textit{Augmented Math}~\cite{Head2022}, have presented taxonomy analyses in similar domains, their focus is broader, encompassing the readability of math formulas in books, papers, slides, and videos. 
While these analyses provide insights into low-level augmentation and annotation techniques, such as visual notations or styles, we need to understand higher-level design strategies to inform our system features. 
Therefore, our analysis specifically focuses on \textit{interactive websites} to better understand the high-level \textit{design strategies} employed in these websites. 
Our analysis complements existing taxonomy analyses presented in~\cite{Head2022} to gain a deeper understanding of explorable math explanations.

\subsubsection{Dataset}
We collected 43 existing interactive math explanations available on the internet from various sources.
These datasets were gathered from notable collections of interactive websites, such as \textit{Explorabl.es}~\cite{Explorab37:online}, \textit{Gallery of Concept Visualization}~\cite{Galleryo94:online}, \textit{GeoGebra}~\cite{GeoGebra95:online}, \textit{PhET}~\cite{PhETFree6:online}, \textit{Explained Visually}~\cite{Setosada85:online}, \textit{Visualize It}~\cite{Visualiz92:online}, \textit{Interactive Maths}~\cite{InteractiveMaths:online}, \textit{Seeing Theory}~\cite{SeeingTh51:online}, \textit{Distill}~\cite{distill}, and \textit{Awesome Interactive Math Tools}~\cite{awesome:online}, as well as through the individual searches. 
A complete list of the collected dataset with each website link is available in \autoref{appendix}.  

\subsubsection{Method}
We identified and categorized high-level design strategies through an open-coding analysis. 
After collecting the examples, two of the authors first conducted an open-coding analysis to identify initial categories and dimensions. Next, all authors reflected on the initial categorization to discuss consistency and comprehensiveness. After refining the categories, three authors performed systematic coding with individual tagging to categorize the complete dataset. Finally, the authors reflected on individual tagging to resolve discrepancies and obtain the final coding results.

\subsubsection{Results}
Our analysis led to the identification of three high-level design strategies: 1) Exemplify through concrete values, 2) Visualize through interactive and animated graphs, and 3) Guide through contextual hints and exercises. For each strategy, we also identify two common design components, as illustrated in Figure~\ref{fig:design-strategies}. 
These collected design strategies span a wide range of math topics, including \textit{Algebra} (6/43), \textit{Geometry} (9/43), \textit{Calculus} (2/43), \textit{Probability} (11/43), \textit{Arithmetic} (1/43), \textit{Applied math} (19/43), and \textit{Graph theory} (3/43). 
In addition to the high-level design strategies, we also identify other low-level design and interaction elements, which include \textit{Slider and Scrubbable} (27/43), \textit{Direct Graph Manipulation} (18/43), \textit{Options} (3/43), \textit{Text Input}  (5/43), \textit{Scroll} (2/43), \textit{Button} (24/43), and \textit{Hover} (3/43). 
The results are summarized in \autoref{appendix} (\autoref{fig:taxonomy-analysis} as well as \autoref{tab:urls1} and \autoref{tab:urls2}). 
In the following sections, we discuss each design strategy and its examples in more detail. \autoref{fig:design-strategies} also illustrates the visual summary of each strategy and design component. 

\subsection*{Strategy 1. Exemplify through Concrete Values}
\subsubsection*{Explorable Values and Examples}
The first strategy is to exemplify abstract and symbolic math representations with concrete values. 
Such examples include providing concrete examples for abstract concepts, embedding numerical values in each symbol of complex math formulas, and showing concrete probabilities with simulation experiments. 
Most of the websites allow users to dynamically update these values with interactive components such as sliders or text inputs.

\subsubsection*{Dynamic Calculation}
Another important aspect of this strategy is that when users change these values, then corresponding equations also change based on dynamic calculations and simulations.
This allows students to immediately see how each value affects the outcome, helping them gain an intuition of how symbols and equations are related to each other. 
By leveraging explorable examples and dynamic calculations, students can develop a better understanding of abstract concepts by visualizing them. 

\subsection*{Strategy 2. Visualize through Interactive and Animated Graphs}
\subsubsection*{Reactive Graphs}
While the first strategy mainly focuses on textual representation, another common strategy leverages dynamic visual representation. 
In particular, existing websites often use reactive graphs that can be dynamically changed based on corresponding interactions or value changes.
Such examples include a reactive graph of a trigonometric function (algebra), an interactive diagram for the Pythagorean theorem (geometry), and an explorable explanation of Newton's method (calculus). 
This technique is often used in conjunction with the exemplifying strategy described above. 
For example, when a user changes a value in an equation, the graphs and diagrams are also updated. 
This helps users understand the relationship between the symbolic variable and its visual representation. 

\subsubsection*{Animated Figures}
Alternatively, animated figures focus primarily on animation with less emphasis on interactivity. 
Animated figures are particularly useful for visualizing simulation results.
In this way, animation can visualize how simulation evolves with temporal representation. 
Examples include showing simulation results of probability, demonstrating various geometric transformations, and temporal visualization of sine and cosine curves.
While this representation focuses more on animation than interaction, users can still manipulate parameters such as timelines. 

\subsection*{Strategy 3. Guide through Contextual Hints or Exercises}
\subsubsection*{Contextual Hints}
Contextual hints provide additional information to students when they need it. 
Examples include referencing definitions to remind the context or breaking down complex solutions through step-by-step instructions.
This support can be customized based on a student's progress and can adapt to their individual learning needs, ensuring that they receive the most appropriate help at the right time.

\subsubsection*{Interactive Exercises}
Incorporating exercises directly into the explanations allows users to actively engage with the content. 
Exercises can come in various forms, such as multiple-choice questions for a simple quiz, fill-in-the-blank problems to solve equations, or direct manipulation of figures to answer questions.  
Providing immediate feedback on these exercises can help students identify their mistakes and reinforce their understanding of the concepts.