\begin{abstract}
We introduce Augmented Math, a machine learning-based approach to authoring AR explorable explanations by augmenting static math textbooks without programming. To augment a static document, our system first extracts mathematical formulas and figures from a given document using optical character recognition (OCR) and computer vision. By binding and manipulating these extracted contents, the user can see the interactive animation overlaid onto the document through mobile AR interfaces. This empowers non-technical users, such as teachers or students, to transform existing math textbooks and handouts into on-demand and personalized explorable explanations. To design our system, we first analyzed existing explorable math explanations to identify common design strategies. Based on the findings, we developed a set of augmentation techniques that can be automatically generated based on the extracted content, which are 1) dynamic values, 2) interactive figures, 3) relationship highlights, 4) concrete examples, and 5) step-by-step hints. To evaluate our system, we conduct two user studies: preliminary user testing and expert interviews. The study results confirm that our system allows more engaging experiences for learning math concepts.
\end{abstract}

\begin{CCSXML}
<ccs2012>
   <concept>
       <concept_id>10003120.10003121.10003124.10010392</concept_id>
       <concept_desc>Human-centered computing~Mixed / augmented reality</concept_desc>
       <concept_significance>500</concept_significance>
   </concept>
 </ccs2012>
\end{CCSXML}

\ccsdesc[500]{Human-centered computing~Mixed / augmented reality}
% http://dl.acm.org/ccs.cfm

\keywords{Augmented Reality; Explorable Explanations; Interactive Paper; Augmented Textbook; Authoring Interfaces}





% create these explorable math textbooks by overlaying embedding interactive visuals. Our system provides 

% Explorable explanations promote more active and engaging learning experiences, but creating them often requires time and efforts, limiting the opportunity for non-technical users to create their own interactive content. 

%Repurposing the content in static Math textbooks

% encourage users to discover things about the concept for themselves, and test their expectations of its behaviour against its actual behaviour, promoting a more active form of learning than reading or listening
