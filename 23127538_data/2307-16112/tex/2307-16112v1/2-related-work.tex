\section{Related Work}

\subsection{Explorable Explanations}
Explorable explanations~\cite{victor2011explorable} have emerged as an increasingly popular practice to teach and learn abstract concepts across various educational contexts, such as math, science, and engineering (e.g. \textit{PhET}~\cite{perkins2006phet}, \textit{Distill}~\cite{distill}). Interactive explanations that leverage animations, simulations, and gaming provide more engaging experiences, naturally encouraging students to develop their understanding of complex concepts in a playful manner~\cite{adams2010student, hensberry2015effective, keller2007assessing, wang2020cnn}. Many of these math augmentation techniques were demonstrated in Bret Victor's KillMath projects~\cite{victor2011killmath}.
Various tools, such as \textit{Idyll}~\cite{conlen2018idyll}, \textit{Flapjax}~\cite{meyerovich2009flapjax}, \textit{GeoGebra}~\cite{GeoGebra95:online}, \textit{Mavo}~\cite{verou2016mavo}, and \textit{Data Theater}~\cite{lau2020data}, have been similarly developed to lower the barrier to create interactive explanations. For example, \textit{Idyll Studio}~\cite{conlen2021idyll} allows users to create data-driven explanations and \textit{GeoGebra}~\cite{GeoGebra95:online} lets users make interactive diagrams to teach various math topics such as geometry, algebra, and calculus.

However, existing tools still require the programming expertise that makes it difficult for non-technical users to adapt and create their own interactive content.
\citeauthor{Head2022}~\cite{Head2022} explicitly mention that the development of an appropriate authoring tool still requires future work, and there is a strong need for intelligent design assistants that automatically augment static math textbooks.
Towards this goal, \textit{CrossData}~\cite{chen2022crossdata} supports non-technical users in creating data-driven documents by binding text and data in the authoring process. This allows for easy updating of text, data, and graphs, but the output document is static once the authoring process is finished. 
Beyond textual explanations, some tools support non-technical users in creating interactive diagrams for math and engineering education. For example, \textit{MathPad2}~\cite{LaViola2007mathPad2} allows users to create math sketches that turn into interactive diagrams. In \textit{PhysicsBook}~\cite{cheema2012physicsbook} and \textit{PhysInk}~\cite{scott2013physink}, students can create interactive sketches that animate through given math formulas or physics simulation. Alternatively, \textit{Kitty}~\cite{kazi2014kitty} explores an authoring interface to let users make interactive animation by binding between different components, which is useful for engineering education. In these tools, however, the interactive explanations are based on sketches that are disconnected from existing material such as textbooks.

In contrast, this paper proposes to \textit{repurpose} existing static documents to turn them into explorable explanations.
This approach allows users to easily and quickly create interactive content with minimal time and effort as it does not require programming or generating content from scratch.
Additionally, to the best of our knowledge, our work is the first to investigate AR-based explorable explanations, which also have great potential for enhancing the learning experience by providing immersive and interactive content in a real-world context. This approach opens up new possibilities for engaging and personalized learning, bridging the gap between physical textbooks and digital interactivity.


\subsection{Augmented Paper}
Since Wellner's first demonstration of \textit{DigitalDesk}~\cite{wellner1991digitaldesk}, HCI researchers have actively explored the idea of \textit{augmented paper} by overlaying digital content onto physical documents~\cite{Han2021hybrid}. Researchers have investigated various ways to augment physical paper, including projecting digital information (e.g., \textit{AffinityLens}~\cite{subramonyam2019affinity}, \textit{MouseLight}~\cite{Song2010mouseLight}, \textit{Qook}~\cite{zhao2014qook}), illuminate physical paper (e.g., \textit{IllumiPaper}~\cite{klamka2017illumipaper}), and embedding content in mobile AR (e.g., \textit{Opportunistic Interfaces}~\cite{du2022opportunistic}, \textit{Teachable Reality}~\cite{monteiro2023teachable}) or see-through displays (e.g. \textit{MagicBook}~\cite{billinghurst2001magicbook}, \textit{Mixed Reality Book}~\cite{Grasset2007mixed}, \textit{HoloDoc}~\cite{Li2019holodoc}, \textit{Replicate and Reuse}~\cite{gupta2020replicate}).
However, most of these works assume prepared content, and only a few works have looked into the \textit{authoring aspect} of these augmented documents.
For example, \textit{PapARVis Designer}~\cite{chen2020augmenting} allows users to author embedded visualizations for static paper, and
\textit{Dually Noted}~\cite{qian2022dually} enables users to annotate physical documents by leveraging computer vision to recognize the layout of documents. 
Most closely related to our work, \textit{Paper Trail}~\cite{Rajaram2022papertrail} explores the authoring of augmented paper based on simple user-defined animation. 
Our work builds on top of these works to enable AR-based documents with a greater focus on interactive exploration compared to prior works.
By emphasizing explorability, we aim to create more engaging and personalized learning experiences through augmented paper.

\subsection{AR for Math Education}
Augmented reality has shown advantages in teaching math by enhancing visualizations~\cite{Ahmad2020Augmented} and creating playful learning experiences~\cite{Chen2019effect, Khan2018mathland}.
Such AR-based tools explore various math topics such as geometry~\cite{Kaufmann2002mathematics, Suselo2021using, Sarkar2019collaborative} and algebra~\cite{Chen2019effect}. 
For example, \textit{ARMath}\cite{Kang2020armath} demonstrates teaching early math skills such as counting and basic geometry by turning everyday objects into mathematical learning materials, using AR and computer vision.
Similarly, \textit{RealitySketch}\cite{suzuki2020realitysketch} also aims to visualize math and physics concepts in the real world by embedding responsive graphics through an AR sketching interface.
However, in most of these existing works, the AR contents are disconnected from existing learning materials like textbooks.
A few studies have focused on augmenting existing math textbooks. For example, Li et al.~\cite{Li2019turning} turn a math book into an interactive game to attract children's interest, however the connection between the physical book and AR content is limited to multiple-choice questions. \textit{GeoAR}~\cite{Kirner2012development} creates an augmented book to teach geometry, with AR geometric shapes appearing on the pages, but the AR content is predefined and only works with the specific book. In contrast, our work introduces a general-purpose authoring tool that enables teachers and students to create their own augmentations using existing formulas and figures in math textbooks.

\subsection{AR Authoring Tools}
Existing AR authoring tools often rely on user-sketched content (e.g. \textit{Rapido}~\cite{Leiva2021rapido}, \textit{Pronto}~\cite{Leiva2020pronto}, \textit{RealitySketch}~\cite{suzuki2020realitysketch}, \textit{Sketched Reality}~\cite{kaimoto2022sketched}, \textit{ProjectAR}~\cite{lunding2022projectar}), user-created physical models (e.g. \textit{ProtoAR}~\cite{Nebeling2018protoAR}), or pre-defined virtual models (e.g. \textit{Reality Composer}~\cite{RealityComposer}, \textit{Adobe Aero}~\cite{aero}, \textit{RealityTalk}~\cite{liao2022realitytalk}). However, there is a lack of authoring tools that enables the users to augment existing visual content such as physical paper.
Moreover, creating dynamic AR content typically requires user-defined demonstrations (e.g. \textit{Rapido}~\cite{Leiva2021rapido}, \textit{Pronto}~\cite{Leiva2020pronto}) or pre-defined animations (e.g. \textit{Reality Composer}~\cite{RealityComposer}, \textit{Adobe Aero}~\cite{aero}). While these approaches work for general-purpose AR prototyping, they are not well-suited for AR-based explorable explanations, where animation should be bound to mathematical formulas or simulations. 
To address these limitations, our proposed authoring tool leverages machine learning and data-binding techniques. This allows users to extract existing content from documents to create interactive AR explanations tailored to their needs, facilitating the creation of engaging educational materials.

% Figure environment removed

