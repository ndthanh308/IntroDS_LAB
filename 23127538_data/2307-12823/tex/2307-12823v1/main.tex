% !TEX TS-program = pdflatexmk
\documentclass{config}

\newcommand{\BLUE}[1]{{\color{blue} #1}}

\begin{document}

\preprint{APS/123-QED}

\title{Reliable confidence regions for quantum tomography using distribution moments}

\author{D.O. Norkin}
\affiliation{Russian Quantum Center, Skolkovo, Moscow 143025, Russia}
\affiliation{National University of Science and Technology ``MISIS”, Moscow 119049, Russia}

\author{E.O. Kiktenko}
\affiliation{Russian Quantum Center, Skolkovo, Moscow 143025, Russia}
\affiliation{National University of Science and Technology ``MISIS”, Moscow 119049, Russia}

\author{A.K. Fedorov}
\affiliation{Russian Quantum Center, Skolkovo, Moscow 143025, Russia}
\affiliation{National University of Science and Technology ``MISIS”, Moscow 119049, Russia}

\date{\today}

\begin{abstract}

    Quantum tomography is a widely applicable method for reconstructing unknown quantum states and processes. However, its applications in quantum technologies usually also require estimating the difference between prepared and target quantum states with relivable confidence intervals.
	In this work, we suggest a computationally efficient and reliable scheme for determining well-justified error bars for quantum tomography.
	We approximate the probability distribution of the Hilbert-Schmidt distance between the target state and the estimation, which is given by the linear inversion, by calculating its moments. 
	We also present a generalization of this approach for quantum process tomography.
	We benchmark our approach for a number of quantum tomography protocols using both simulated and experimental data.
	The obtained results pave a way to the use of the suggested scheme for complete characterization of quantum systems of various nature.
\end{abstract}

\maketitle

\section{Introduction}

Quantum technologies strongly require the ability to prepare quantum states with desired properties~\cite{Fedorov2022}.
In this regard, one of crucial tasks is to estimate the difference between the states that are actually prepared and the desired one, which requires solving the problem of complete characterization of quantum states and processes.
The art of solving this problem is known as quantum tomography that uses the measurement statistics in multiple bases (for a review, see Ref.~\cite{Lvovsky2009}).
Being a gold standard in quantum physics experiments, quantum tomography is routinely performed for systems of various nature, ranging from quantum-optical states~\cite{Lvovsky2001} to trapped-ion quantum simulators~\cite{Blatt2017}. 
However, along with the various benefits, this method encounters a number of challenges. 
The first issue related to the efficiency of quantum tomography, which becomes unwieldy for large and complex quantum systems, which are required for quantum technologies~\cite{Blume-Kohout2022}. 
A number of methods for improving the performance of quantum tomography protocols have been suggested~\cite{Weinfurter2010,Gross2010,Cramer2010,Blatt2017,Carleo2018,Carrasquilla2019}, 
and the latest results are related to the use of neural-network quantum state representation for tomographic reconstruction~\cite{Carleo2018,Carrasquilla2019,Tiunov2020,Hradil2022,Lvovsky2022}. 

Another important issue is supplying quantum tomography protocols with efficient methods 
for the estimation of their reconstruction accuracy~\cite{Bogdanov2009,Flammia2011,Silva2011,Blume-Kohout2012,Renner2012,Filippov2012,Flammia2012,Sugiyama2013,Renner2016,Renner2019}.  
Most of the existing practical techniques for estimating the accuracy of quantum tomography protocols are based on heuristic approaches,
while only a few methods are established rigorously~\cite{Sugiyama2013,Renner2019,Kiktenko20202,Kiktenko2021}.
Existing approaches for solving this problem encounter an obstacle of combining computational efficiency, which takes place for heuristic methods, and provable accuracy, which is a feature of rigorous methods 
(however, their estimations are typically quite pessimistic so that yielding error bars, which are unnecessarily large). 
Consequently, finding a trade-off between the computational efficiency of the method and provable accuracy is highly important for ongoing experiments. 
This problem becomes even more significant for quantum process tomography (QPT)~\cite{Poyatos1997,Chuang1997,DAriano2001,Kiktenko20202,Kiktenko2021}, 
which is an important tool for characterizing the quality of operations performed by quantum devices~\cite{White2004,Wineland2008}.  

In this work, we present a computationally-efficient approach for solving this problem, which demonstrates the best performance among existing rigorous approaches.
The idea is that the probability distribution of the Hilbert-Schmidt distance between the experimentally prepared state and the estimation, which is given by the linear inversion, can be approximated by calculating its moments. 
We benchmark our approach for a number of quantum tomography protocols for states and processes using both simulated and experimental data,
where the latter are generated using a publicly accessible superconducting quantum processor.
As we expect, our results provide a computationally efficient and still rigorously justified approach for quantum tomography of states and processes. 

Our work is organized as follows.
In Sec.~\ref{sec:QST}, we provide a short description of the general quantum state tomography protocol (QST) for a finite-dimensional system.
In Sec.~\ref{sec:RQST}, we discuss rigorous confidence regions for QST.
In Sec.~\ref{sec:RQPT}, we present a generalization of the method of obtaining confidence regions for QPT.
In Sec.~\ref{sec:affine}, we describe a method of computing confidence intervals in terms of more practical metrics, such as fidelity or mean value of observable.
In Sec.~\ref{sec:analysis}, we present a performance analysis of our approach, where we use several quantum states, in particular, 3-qubit GHZ state, for the QST, and also single-qubit depolarizing channel and noisy quantum teleportation channel for the QPT setting.
We conclude in Sec.~\ref{sec:conclusion}.

\section{Quantum state tomography}\label{sec:QST}

Let us first briefly consider a standard QST protocol.
We are interested in a quantum system assigned with a finite-dimensional Hilbert space $\hil$ with a corresponding space $\mathcal{L}(\hil)$ of Hermitian operators over $\hil$.
The quantum system under consideration is repeatedly prepared in the same unknown state, described by a density matrix $\rho$, 
which is an element of the set of density operators $\denset = \left\{\rho\in\mathcal{L}(\hil) \middle\vert\ \rho \ge 0,\ \Tr \rho = 1 \right\} \subset \mathcal{L}(\hil)$. 
After each preparation the system is measured with positive operator-valued measure (POVM), which is defined by a set of hermitian positive semidefinite operators $\boldsymbol{E}=\povm$ that sum into unity: $\sum_{i=1}^P E_i = \id$. 
After a series of $N$ such measurements one obtains data in a form of a vector $\boldsymbol{n}=(n_1,\ldots,n_P)$, where each element $n_i$ indicates on the number of occurrences of an outcome corresponding to $E_i$. 
Finally, a density matrix $\tilde\rho$, which is an approximation of the real state $\rho$, can be reconstructed. 
Mathematically, each quantum measurement can be viewed as sampling from a generalized Bernoulli distribution with probability of $i$th outcome determined as follows: 
\begin{equation}
	p_i = \Tr[E_i\rho].
	\label{eq:prob}
\end{equation} 
Quantum tomography, which is based on a series of such measurements, corresponds to sampling from the multinomial distribution: 
$\boldsymbol{n} \sim \Mult\left(N, \boldsymbol{p}\right)$ with $\boldsymbol{p}=(p_1,\ldots,p_P)$. 
The problem of reconstructing a density matrix thus can be reduced to the task of finding unknown parameters of this distribution. 

\section{Rigorous confidence regions for the QST protocol}\label{sec:RQST}

To obtain rigorous confidence regions for quantum tomography, we first fix the quantum tomography protocol.
Similar to Ref.~\cite{Sugiyama2013}, we use the least squares (LS) inversion that exploits the linearity of Eq.~(\ref{eq:prob}). 
Let us introduce an orthogonal set $\{\sigma_i\}_{i=1}^{d^2-1}$ of traceless Hermitian operators $\sigma_i\in{\cal L}({\cal H})$ satisfying the condition $\Tr(\sigma_i \sigma_j)=d\delta_{ij}$, where $\delta_{ij}$ is a Kronecker delta and $d$ is the dimension of ${\cal H}$.
Each matrix then can be represented by its Pauli vector $\boldsymbol{r}(\rho)=(1/d, r_1(\rho),\ldots, r_{d^2-1}(\rho))$, which satisfies $\rho = \boldsymbol{r}(\rho)\cdot\vec{\sigma}$ with $\vec{\sigma}=(\mathbb{1},\sigma_1,\ldots,\sigma_{d^2-1})$.
One can easily show that there exists a unique matrix $A$ such that 
\begin{equation}\label{eq:state_lin}
	\boldsymbol{p} = A\boldsymbol{r}(\rho).
\end{equation}
It is well known that LS solution to this equation is $\boldsymbol{r}(\rho) = A^+\boldsymbol{p}$, 
where $A^+ = (A^\top A)^{-1} A^\top$ is a left pseudo-inverse matrix~\cite{Watson1967}. 
Thus, by combining all these formulas one can derive the expression for $\tilde\rho$ as follows:
\begin{equation}\label{eq:lininv}
	\tilde\rho = \frac{1}{N}\vec{\sigma}\cdot \left(A^+\boldsymbol{n}\right)=
    \vec{\sigma}\cdot \left(A^+\boldsymbol{f}\right),
\end{equation}
where $\boldsymbol{f}:=\boldsymbol{n}/N$ is a vector of observed frequencies, and ${\Tr}\tilde\rho=1$, yet the condition $\tilde\rho\geq 0$ may or may not be fulfilled due to statistical fluctuations.

We are interested in constructing a QST procedure that computes not only a point estimate, but also a rigorous confidence region for this estimate with a desired confidence level ${\rm CL}$, 
which we denote as $\Gamma(\tilde\rho, {\rm CL})$~\cite{Sugiyama2013}. 
This means that $\Pr{\rho\in\Gamma(\tilde\rho, {\rm CL})} \ge {\rm CL}$, where $\Pr{\cdot}$ should be interpreted as the probability taken over all possible measurement outcomes $\boldsymbol{n}$. 
There are infinitely many regions in various shapes that satisfy the condition above. 
For example, as it shown in certain cases the constructed regions have the form of a polytope~\cite{Renner2019, Kiktenko2021}.

In this work, however, we only concentrate on spherically symmetric regions (i.e. having a shape of a ball) with respect to Hilbert-Schmidt norm, centered around the point estimate $\tilde\rho$.
Our task is then can be simplified to finding a mapping $\delta({\rm CL})$ such that the following condition holds:
\begin{equation}
	\Pr{\Delta_{\rm HS}\left(\rho, \tilde\rho\right) < \delta({\rm CL}) } \ge {\rm CL},
	\label{eq:pg}
\end{equation}
the Hilbert-Schmidt distance is defined as
\begin{equation}
	\Delta_{\rm HS}\left(a, b\right) := \sqrt{\frac{\Tr\left[(a - b)^2\right]}{2}}
\end{equation}
for arbitrary Hermitian matrices $a$ and $b$.
Employing the orthogonality of the Pauli basis, one can easily rewrite $\Delta_{\rm HS}\left(\rho, \tilde\rho\right)$ in terms of Pauli vectors:
\begin{equation} \label{eq:hs_norm_est}
	\Delta_{\rm HS}\left(\rho, \tilde\rho\right) = \sqrt{\frac{d}{2}} \|\boldsymbol{r}(\rho) - \boldsymbol{r}(\tilde\rho)\|_2 = \sqrt{\frac{d}{2}} \left\|A^+\Delta \boldsymbol{p}\right\|_2,
\end{equation}
where $\Delta\boldsymbol{p} := \boldsymbol{f} - \boldsymbol{p}$ and $\|\cdot\|_2$ is a standard $\ell^2$ norm. 
This implies that
\begin{equation}\label{eq:pg_bound}
	\Pr{\Delta_{\rm HS}\left(\rho, \tilde\rho\right) < \delta} = \Pr{\sqrt{\frac{d}{2}}\left\|A^+\Delta\boldsymbol{p}\right\|_2 < \delta}.
\end{equation}
Thus, the RHS could be used as the lower bound ${\rm CL}$ in Eq.~\eqref{eq:pg}. 
Moreover it is the tightest possible estimate, since Eq.~\eqref{eq:pg} turns into an equality with 
\begin{equation}
	{\rm CL} = \Pr{\sqrt{\frac{d}{2}}\left\|A^+\Delta\boldsymbol{p}\right\|_2 < \delta}.
\end{equation}
Below we demonstrate that it is possible to approximate this distribution with arbitrary precision.

We note that since we consider a finite-dimensional space both $\|\Delta\boldsymbol{p}\|_2$ and the corresponding operator norm $\left\|A^+\right\|_2$ are bounded. 
This combined with the definition of operator norm implies the boundedness of  $\left\|A^+\Delta\boldsymbol{p}\right\|_2^2 \leq \left\|A^+\right\|_2^2 \|\Delta\boldsymbol{p}\|_2^2$. 
Thus, its distribution is uniquely determined by its moments (see Ref.~\cite{Kjeldsen1993,Lin2017} on the \emph{moment problem}).

Let us now show that it is feasible to explicitly calculate all moments of $\|A^+\Delta\boldsymbol{p}\|_2^2$.
Indeed,
\begin{multline*}
	\|A^+\boldsymbol{\Delta p}\|_2^2 = \sum_i(\sum_j A^+_{ij}\Delta p_j)^2 \\ = \sum_i \left(\sum_j\sum_k A^+_{ij}A^+_{ik}\Delta p_j\Delta p_k\right) \\ 
	= \sum_{jk} \left((A^+)^\top A^+\right)_{jk}\Delta p_j\Delta p_k := \sum_{ij} T_{ij}\Delta p_i\Delta p_j \\ 
	= \sum_{ij} T_{ij}(f_i - p_i)(f_j - p_j) = \sum_{ij} T_{ij}\left(\frac{n_i}{N} - p_i\right)\left(\frac{n_j}{N} - p_j\right) \\ := a^{(0)} + \sum_i a^{(1)}_i n_i + \sum_{ij} a^{(2)}_{ij} n_i n_j,
\end{multline*}
where $a^{(0)}$, $a^{(1)}_i$, and $a^{(2)}_{ij}$ are some constants.

As can be seen from the equation above, $\|A^+\Delta\boldsymbol{p}\|_2^2$ (and hence $\|A^+\Delta\boldsymbol{p}\|_2^{2n}$ for any $n\in\mathbb{N}$) can be expressed as a multivariate polynomial of $n_i$, 
which are samples from multinomial distribution. 
Thus, $\mathbb{E}\|A^+\Delta\boldsymbol{p}\|_2^{2n}$ could be decomposed into a sum of expected values of monomials containing $n_i$ in arbitrary degrees. 
This implies that it is feasible to explicitly calculate all moments of $\|A^+\Delta\boldsymbol{p}\|_2^2$, since there are known formulae for said expected values~\cite{newcomer2008computation,stats4010002}.

We should note that we do not know true probabilities $\boldsymbol{p}$, but only their approximation $\boldsymbol{f}$, and simple substitution of $\boldsymbol{p}$ with $\boldsymbol{f}$ results in a biased estimate. 
However, asymptotically it is still unbiased and the precision of this approximation is of order $O\left({1}/{N}\right)$ and hence, as in the case of the QST, $N$ is usually a quite large number, and so we can neglect the second-order term.

Although knowing all moments ultimately allows us to reconstruct the distribution precisely, it is impractical due to the infinite number of required moments. 
To overcome this one could for example utilize a moment bound on $\Pr{\|A^+\Delta\boldsymbol{p}\|_2^2 < \delta}$~\cite{Nelson1995}. 
However, these bounds are not very tight, so we propose a more practical approach, which is to use only first $k$ moments and approximate with a parametric distribution.

% Figure environment removed

In this work we mostly restrict ourselves to using only two first central moments because utilizing higher moments provides a negligible improvement, in the same time significantly increasing the computational cost. 
Let us denote the cumulative distribution function (CDF) of this approximated distribution by $F$:
% To choose one we note that if the components of $\Delta\boldsymbol{p}$ had equal variance, then $\|\Delta\boldsymbol{p}\|_2^2$ would be distributed as a chi-squared distribution. 
% This is why we choose a generalization of a chi-squared, the gamma distribution. 
% As it has two parameters, it is sufficient to calculate only two first moments of $\|\Delta\boldsymbol{p}\|_2^2$ to fully describe the distribution and estimate the RHS in \eqref{eq:pg_bound}:
\begin{multline}\label{eq:gamma_approx}
    \Pr{\sqrt{\frac{d}{2}}\left\|A^+\Delta\boldsymbol{p}\right\|_2 < \delta} \\ = \Pr{\|A^+\Delta\boldsymbol{p}\|_2^2 < \frac{2\delta^2}{d}} \approx F\left(\frac{2\delta^2}{d}\right).
\end{multline}
Using this equation, we are finally able to calculate the confidence interval:
\begin{equation}\label{eq:state_pg}
	\delta({\rm CL}) = \sqrt{\frac{d}{2}} F^{-1}({\rm CL}).
\end{equation}

In practice, a set of POVMs is often used in QST protocols instead of a single POVM. 
Our method can be easily generalized to this multiple POVM setting with the only difference in operator $A$ in Eq.~\eqref{eq:state_lin} 
being represented by a 3-dimensional tensor instead of 2-dimensional matrix.

Thus, we obtain the method for calculating rigorous confidence regions for QST protocols. 
It should be noted that this method is efficient from the computation point of view, because it does not rely on Monte-Carlo procedures and it has well-justified error bars.
Below we provide results on the comparison of our approach with a Monte-Carlo approach, which is a gold standard in this task~\cite{EfroTibs93,Blume-Kohout2012}.

% We also apply our method to obtaining the confidence interval for the distance between real and target states in experimental QST of 4-qubit GHZ state.
% The state generated using a small-scale quantum information processor based on a string of ions~\cite{Blatt2013}.
% In this case, unlike in the simulations, the real state $\rho$, on which the tomography is performed, may not coincide with the target one $\rho_{\text{target}} = \text{GHZ}(4)$ because of noises and imperfections of an experimental setup. 
% Following the approach of Ref.~\cite{Sugiyama2013} we use the triangle inequality to estimate a confidence interval for $\Delta\left(\rho, \rho_{\text{target}}\right)$ instead of $\Delta\left(\rho, \tilde\rho\right)$:
% \begin{multline}
% 	\delta_{\text{target}}({\rm CL}) = \Delta\left(\rho, \rho_{\text{target}}\right) \leq \Delta\left(\rho, \tilde\rho\right) + \Delta\left(\tilde\rho, \rho_{\text{target}}\right) = \\ = \delta({\rm CL}) + \Delta\left(\tilde\rho, \rho_{\text{target}}\right).
% \end{multline}
% This confidence interval is shown in Fig.~\ref{fig:exp}.

% Overall, our method is computationally efficient (in particular, for 4-qubit system the real time of calculation on a standard quad-core processor is several seconds).
% In addition, it generates good quality confidence intervals in compare with existing alternative to other rigorous approaches.

\section{Rigorous confidence regions for QPT}\label{sec:RQPT}

Here we generalize our approach to the task of QPT.
For this purpose consider a completely-positive trace-preserving map (CPTP) also known as a channel $\Phi\!: {\cal L}({\cal H_{\rm in}})\rightarrow {\cal L}({\cal H_{\rm out}})$, 
where ${\cal H_{\rm in}}$ and ${\cal H_{\rm out}}$ are $d_{\rm in}$- and $d_{\rm out}$-dimensional Hilbert spaces ($d_{\rm in}, d_{\rm out}<\infty$), respectively.
In what follows, $\mathbb{1}_{{\rm in}({\rm out})}$ is the identity operator in ${\cal H}_{{\rm in}({\rm out})}$, and $\Tr_{{\rm in}({\rm out})}$ denotes partial trace over ${\cal H}_{{\rm in}({\rm out})}$.

To define the map $\Phi$ it is convenient to consider a corresponding Choi state~\cite{Jamiokowski1972,Choi1975,Jiang2013} 
$C_{\Phi}\in {\cal C}({\cal H}_{\rm in},{\cal H}_{\rm out}):=\{C\in {\cal L}({\cal H}_{\rm in}\otimes{\cal H}_{\rm out}): C\geq0, \Tr_{\rm out} C = \mathbb{1}_{\rm in} \}$, given by
\begin{equation}
	C_{\Phi} = \Sum_{i,j=0}^{d_{\rm in}-1} \ket{i}\bra{j}\otimes\Phi\left(\ket{i}\bra{j}\right),
\end{equation}
where $\{\ket{n}\}_{n=0}^{d_{\rm in-1}}$ is a computational basis in ${\cal H}_{\rm in}$.
Recall that given $C_{\Phi}$ one can compute an output of $\Phi$ for any input $\rho\in{\cal L}(\cal H_{\rm in})$ as follows:
\begin{equation}\label{eq:channel_choi}
	\Phi[\rho] = \Tr_{\rm in}( \rho^\top \otimes \mathbb{1}_{\rm out} C_{\Phi}),
\end{equation}
where $\top$ stands for standard transposition.

Consider a QPT protocol, where a set of input states $\boldsymbol\rho_{\rm in}=(\rho_{\rm in}^{(1)},\ldots,\rho_{\rm in}^{(M)})$ with $\rho_{\rm in}^{(i)}\in{\cal S}({\cal H}_{\rm in})$ is repeatedly prepared,
put through $\Phi$, and then is measured by a POVM $\boldsymbol{E}=\povm$.
According to Eq.~\eqref{eq:prob}, the probability of observing $j$-th outcome when $i$-th input state is sent to the channel is $p_{ij} = \Tr\left[\Phi(\rho_{\rm in}^{(i)})E_j\right]$. 
Together with Eq.~\eqref{eq:channel_choi}, it gives us the following formula:
\begin{equation}\label{eq:prob_channel}
	p_{ij} = \Tr\left[E_j\Phi(\rho_{\rm in}^{(i)})\right] = \Tr\left[\left((\rho_{\rm in}^{(i)})^\top\otimes E_j\right)C_{\Phi}\right].
\end{equation}
Let us then denote a Pauli vector of $C_{\Phi}$ by $\boldsymbol{c}(C_{\Phi})$: $C_{\Phi} = \boldsymbol{c}(C_{\Phi}) \cdot \vec{\sigma}_{\rm in,out}$, where
$\vec{\sigma}_{\rm in,out}=({\mathbb{1}\otimes\mathbb{1},\ldots,\sigma^{\rm in}_{d_{\rm in}^2-1}\otimes\sigma^{\rm out}_{d_{\rm out}^2-1}})$ with traceless Hermitian operators satisfying ${\rm Tr}(\sigma_i^{\rm in(out)}\sigma_j^{\rm in(out)})=d_{\rm in(out)}\delta_{ij}$.
Note that $\boldsymbol{c}(C_{\Phi})$ has $d_{\rm in}^2(d_{\rm out}^2-1)$ independent components since ${\rm Tr}_{\rm out}C_{\Phi}=\mathbb{1}_{\rm in}$.

In analogy with Eq.~\eqref{eq:prob}, due to linearity there exists a tensor $B_{ijk}$ such that:
\begin{equation}\label{eq:prob_channel_inv}
	c_k = B_{ijk}p_{ij}.
\end{equation}
Doing analogous calculations one can arrive at very similar expressions to Eq.~\eqref{eq:pg_bound} and Eq.~\eqref{eq:gamma_approx}:
\begin{multline}\label{eq:gamma_approx_channel}
	\Pr{\Delta_{\rm HS}\left(C_{\Phi}, \tilde C_{\Phi}\right) < \delta} = \Pr{\sqrt{\frac{d}{2}}\left\|B_{ijk}p_{ij}\right\|_2 < \delta} \\ = \Pr{\|B_{ijk}p_{ij}\|_2^2 < \frac{2\delta^2}{d}} \approx F\left(\frac{2\delta^2}{d}\right),
\end{multline}
and $\delta({\rm CL})$ in QPT case can also be described by Eq.~\eqref{eq:state_pg} (here $\tilde C_{\Phi}$ is an estimate of $C_{\Phi}$ obtained via linear inversion).

\section{Deriving confidence intervals for affine functions}\label{sec:affine}

Our method is specifically based on Hilbert-Schmidt norm as a measure of distance between quantum states.
However, often it is desired to compute confidence intervals in terms of more practical metrics, such as fidelity or mean value of an observable.
The method of obtaining confidence intervals for affine functions of quantum states (Choi states) with standard methods of linear programming based on QST (QPT) confidence polytopes is presented in Ref.~\cite{Kiktenko2021}.

Here we demonstrate how to generalize our method to other metrics employing convex optimization.
We consider only the case of QST, as the QPT setting is analogous.
Suppose one has already performed the procedure described in the section devoted to constructing confidence intervals, 
i.e. for a given confidence level ${\rm CL}$ calculated $\delta$ such that $\Pr{\Delta_{\rm HS}\left(\rho, \tilde\rho\right) < \delta} \ge {\rm CL}$, or in Pauli vector formalism: 
\begin{equation}
	\Pr{\|\boldsymbol{r}(\rho) - \boldsymbol{r}(\tilde\rho)\|_2 < \sqrt{\frac{2}{d}}\delta} \ge {\rm CL}.
\end{equation}
Suppose then one wants to compute a confidence interval in terms of an affine function of $\boldsymbol{r}(\rho)$, i.e. a function $\phi: {\cal S}({\cal H})\rightarrow \mathbb{R}$ that can be represented in the following form:
\begin{equation}
	\phi(\rho) = {\boldsymbol r}(\rho)\cdot\boldsymbol\phi + \phi_0,
\end{equation}
where $\boldsymbol\phi\in\mathbb{R}^{d^2}$ and $\phi_0\in\mathbb{R}$.

Thus, the task is to find optima of $\phi$ in the ball 
\begin{equation}
	\|\boldsymbol{r}(\rho) - \boldsymbol{\tilde r}(\rho)\|_2 < \sqrt{\frac{2}{d}}\delta.
\end{equation} 
This optimization problem can be solved efficiently with second-order cone programming~\footnote{Python Software for Convex Optimization, https://cvxopt.org}. 
The classical second-order cone programming formulation is the following:
\begin{equation}\label{eq:SOCP}
    \begin{aligned}
        &\text{minimize~} {\boldsymbol k}\cdot {\boldsymbol x}\\
        &\text{subject to} \\
        &\qquad \|\boldsymbol{A}_i \boldsymbol{x} + \boldsymbol{b}_i\|_2 \leq \boldsymbol{c}_i\cdot x + d_i, \quad i \in 1,\ldots,m \\
        &\qquad \boldsymbol{F}\boldsymbol{x} = \boldsymbol{g}.
    \end{aligned}
\end{equation}
It is easy to see that our task can be formulated in this classical form with straightforward substitutions.

% Figure environment removed

\section{Performance analysis}\label{sec:analysis}

In this section, we analyze the performance of the developed algorithm in cases of both QST and QPT. 
We consider different quantum states and processes, but utilize the same setup for all of them. 

For probing quantum processes, we use the set of $4^N$ (here and further $N$ denotes the number of qubits in a quantum state or process) input states consisting of pure product states in the following form:
\begin{equation}
	\rho_{\rm in}^{(i_1,\ldots,i_N)} = \rho_{\rm in}^{(i_1)}\otimes\ldots\otimes\rho_{\rm in}^{(i_N)},
\end{equation}
where $i_j\in\{1,\ldots,4\}$ and single-qubit states $\{\rho_{\rm in}^{(k)}\}_{k=1}^4$ form a tetrahedron inscribed in a cube with facet orthogonal to $x$, $y$, and $z$ axes.
For read-out in both QST and QPT we use $3^N$ ($N$ is a number of qubits) POVMs corresponding to single-qubit projective measurements with respect to $x$, $y$ and $z$ axes. 
In QST we consider the total number $n$ copies of a quantum state for each of $3^N$ POVMs and in QPT --- $n$ copies of a quantum process for each of $3^N \times 4^N$ combinations of POVMs and input states.

First, we illustrate how our confidence intervals compare with Monte-Carlo simulations on Fig.~\ref{fig:interval} in both QST and QPT settings. 
In case of QST, we consider a 3-qubit GHZ state:
\begin{equation}
    \ket{\psi^{(3)}} = \frac{\ket{000} + \ket{111}}{\sqrt{2}}.
\end{equation}
For QPT we use a single-qubit depolarizing channel:
\begin{equation}
    \Phi^{(N)}_p[\rho] = (1-p)\rho + \frac{\mathbb{1}}{2^N}\Tr \rho,
\end{equation}
where $N=1$ and $p=0.1$. 
We note that our intervals almost exactly coincide with those obtained by Monte-Carlo simulation, despite the latter being significantly more computationally intensive.

We also provide an example of the fidelity confidence intervals. 
We conducted a real quantum tomography of the teleportation channel on the superconducting quantum processor provided by IBM. 
The circuit used to build the channel is shown on Fig.~\ref{fig:fidelity}a and corresponds to the quantum teleportation of the 1st qubit onto the 3rd qubit.
To prepare input states, an operation $U$ was consequently taken in the form
\begin{equation}
    U = \mathbb{1},~RX(\theta),~ RZ(2\phi)RX(\theta),~RZ(\phi)RY(\theta),
\end{equation}
where $RX$, $RY$, and $RZ$ stand for standard Pauli rotations, $\phi:=2\pi/3$, $\theta:=\arccos(-1/3)$. 
Informationally complete set of output measurements was realized by taking
\begin{equation}
    V=\mathbb{1},~RX(\pi/2),~RY(\pi/2).
\end{equation}

In case of ideal input states, gates and output measurements, this channel should coincide with the identity channel. 
However, due to noise and imperfections real results differ from theory.
On Fig.~\ref{fig:fidelity}b a confidence interval in terms of fidelity with respect to the 1-qubit identity channel for the case of $n=2^{13}\approx 8\times10^3$ experiments for each configuration of an input state and a POVM.
We see that the results of the methods can be used as lower and upper bounds on the performance of NISQ devices.

We also provide a verification of our confidence intervals. 
We pick a set of quantum states and quantum processes, a set of predefined confidence levels, and for each combination of them do the following procedure.
We perform a simulated quantum tomography, reconstruct the quantum state (or process) from the results of the tomography, and calculate the confidence interval with respect to the corresponding confidence level. 
After that we check whether the reconstructed quantum state (or process) lies inside of the confidence interval. 
Then we repeat this experiment $D=10^4$ times. 
In the ideal case, if the confidence interval with a confidence level ${\rm CL}$ is valid and the most tight possible, the fraction of experiments $f_{\rm inside}$ in which the reconstructed state (or process) lies inside the interval should be exactly equal to ${\rm CL}$. 
The numerical results of our verification for QST and QPT can be found in Tables~\ref{tab:verification_qst} and \ref{tab:verification_qpt}, respectively. 
As it is seen, regardless of the quantum state or process and the confidence level our confidence intervals are very close to ideal.

To highlight the computational efficiency of our method we report the results of profiling it against the Monte-Carlo approach with 1000 bootstrap samples.
In Table~\ref{tab:profiling} we show time it took to construct a confidence interval depending on the chosen method both for QST with $\ket{\psi^{(3)}}$ and QPT with $\Phi^{(2)}_{0.1}[\rho]$. 
All experiments were conducted on the same machine with a Hexa-Core AMD Ryzen 5 and 16 GB of RAM.

\begin{table}[ht]
    \centering
    \begin{tabular}{c|ccccc}
        State & 0.5 & 0.75 & 0.9 & 0.95 & 0.99 \\ \hline
        $\ket{0}$ & 0.4985 & 0.7532 & 0.9006 & 0.9509 & 0.9899 \\
        $\cos{\frac{\pi}{8}}\ket{0} + \sin{\frac{\pi}{8}} e^{\frac{\imath\pi}{4}} \ket{1}$ & 0.4916 & 0.7503 & 0.9010 & 0.9521 & 0.9911 \\
        Fully mixed 1-qubit & 0.5002 & 0.7532 & 0.9014 & 0.9526 & 0.9896 \\
        $\ket{00}$ & 0.5043 & 0.7582 & 0.9034 & 0.9506 & 0.9885 \\
        $\ket{\Phi^+}$ & 0.5065 & 0.7514 & 0.8968 & 0.9436 & 0.9873 \\
        Fully mixed 2-qubit & 0.4977 & 0.7481 & 0.8987 & 0.9483 & 0.9878 \\
        3-qubit GHZ & 0.4967 & 0.7510 & 0.8990 & 0.9501 & 0.9895 \\
    \end{tabular}
    \caption{Verification of confidence intervals for QST.}
    \label{tab:verification_qst}
\end{table}

\begin{table}[ht]
    \centering
    \begin{tabular}{c|ccccc}
        Process & 0.5 & 0.75 & 0.9 & 0.95 & 0.99 \\ \hline
        Hadamard gate & 0.5028 & 0.7656 & 0.9018 & 0.9474 & 0.9867 \\
        $RX(\frac{\pi}{2})$ & 0.5096 & 0.7575 & 0.9021 & 0.9494 & 0.9872 \\
        $RY(\frac{\pi}{2})$ & 0.5133 & 0.7691 & 0.9088 & 0.9547 & 0.9893 \\
        $\Phi^{(1)}_{0.1}[\rho]$ & 0.5070 & 0.7566 & 0.9000 & 0.9485 & 0.9869 \\
        $\Phi^{(2)}_{0.1}[\rho]$ & 0.5162 & 0.7590 & 0.9017 & 0.9493 & 0.9864 \\
    \end{tabular}
    \caption{Verification of confidence intervals for QPT.}
    \label{tab:verification_qpt}
\end{table}

\begin{table}[ht]
    \centering
    \begin{tabular}{c|cc}
         & QST ($\ket{\psi^{(3)}}$) & QPT ($\Phi^{(2)}_{0.1}[\rho]$) \\ \hline
        Our method & 220~ms & 640~ms \\
        Monte-Carlo & 8.5~s & 870~s \\
    \end{tabular}
    \caption{Profiling of confidence regions.}
    \label{tab:profiling}
\end{table}

\section{Conclusion}\label{sec:conclusion}

We have proposed and tested the computationally efficient and reliable scheme for determining well-justified error bars for quantum tomography.
We have restricted ourselves to the use of two first moments of the distribution of $\|\Delta\boldsymbol{p}\|_2^2$ only. 
However, as shown above, it is possible to calculate an arbitrarily high moment. 
This opens a broad field of investigation of more precise approximations. 
One may use, for example, the generalized gamma distribution, estimating its parameters with three first moments. 
Overall, the more moments are used, the more accurate the estimation of intervals is expected to be.
A way to develop this technique further is to expand the range of quantum tomography protocols, 
including coherent-state quantum tomography~\cite{Lvovsky2008,Lvovsky2011,Anis2012,Fedorov2015}.

{\bf Acknowledgements.}
We acknowledge use of the IBM Q Experience for this work. 
The views expressed are those of the authors and do not reflect the official policy or position of IBM or the IBM Q Experience team. 
All the experiments that are used in this paper have been completed in 2021.
The research is supported by the Priority 2030 program at the National University of Science and Technology ``MISIS'' under the project K1-2022-027.
The work of E.O.K. (deriving confidence regions) was also supported by the RSF Grant No. 19-71-10091.

\bibliography{bibliography.bib}

\iffalse

\appendix

\section{Generalization to multiple POVMs setting}

In practice in the task of quantum tomography instead of a single POVM, a set of POVMs is often used. In this appendix we describe how our method can be generalized to the case of multiple POVMs. 
Let us consider a set of POVMs $\{\boldsymbol{E}^{(j)}\}_{j=1}^M$. 
According to \eqref{eq:prob} the probability of observing $i$-th outcome in $j$-th POVM is
\begin{equation}
	p_i^{(j)} = \Tr\left[E_i^{(j)}\rho\right].
\end{equation}
For the sake of universality of notation we normalize the vector $p_i^{(j)}$ to sum into unity multiplying it by a measurement ratio constant $r^{(j)} = N^{(j)}/N$, 
where $N^{(j)}$ is the number of measurements for $j$-th POVM and $N$ is the total number of measurements. 
Thus we get a resulting vector of probabilities $\tilde p_i^{(j)} = r^{(j)} p_i^{(j)}$. 
Moreover, we construct a single POVM, multiplying each of $E^{(j)}_i$ by the same constant $r^{(j)}$. 
Then, as in Eq.~\eqref{eq:lininv} it is possible to construct a matrix $A$ such that $\boldsymbol{\tilde p} = A\boldsymbol{r}(\rho)$. 
Then the moments of $\|\Delta\boldsymbol{\tilde p}\|_2^2$ can easily be calculated. Mean:
\begin{equation}
	\mathbb{E}\|\Delta\boldsymbol{\tilde p}\|_2^2 = \sum_j \left(r^{(j)}\right)^2 \mathbb{E}\|\Delta\boldsymbol{p}^{(j)}\|_2^2;
\end{equation}
and variance:
\begin{equation}
	\mathbb{V}\|\Delta\boldsymbol{\tilde p}\|_2^2 = \sum_j \left(r^{(j)}\right)^4 \mathbb{V}\|\Delta\boldsymbol{p}^{(j)}\|_2^2.
\end{equation}

\fi



\end{document}