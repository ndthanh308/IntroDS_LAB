\section{Experiments and Results}

\subsubsection{Dataset and Preprocessing}

Following the previous work~\cite{kim2022diffusemorph}, we used the publicly available 3D cardiac MR dataset ACDC~\cite{bernard2018deep} for experiments. The dataset includes 100 4D temporal cardiac MRI data with corresponding segmentation maps. We selected the 3D image at the end of the diastolic stage as the fixed image and the image at the end of the systolic stage as the moving image. 
We resampled all scans to the voxel spacing of $1.5\times1.5\times3.15 mm$, then cropped them to the voxel size of $128\times128\times32$. We normalized the intensity of all images to $\lbrack-1,1\rbrack$. The training set contains 90 image pairs, while the remaining 10 pairs form the test set.
The abovementioned preprocessing steps were performed in accordance with the approach described in prior work~\cite{kim2022diffusemorph} to ensure a fair comparison.

\subsubsection{Implementation Details}

The proposed framework was implemented using the PyTorch library, version 1.12.0.
Following~\cite{kim2022diffusemorph}, we used DDPM UNet's 3D encoder as our shared encoder and DDPM UNet's 3D decoder as our diffusion decoder. For the registration part, instead of a complete 3D UNet in~\cite{kim2022diffusemorph}, we only used DDPM UNet's 3D decoder as our registration decoder to generate the deformation field.
During the diffusion task, we gradually increased the noise schedule from $10^{-6}$ to $10^{-2}$ over 2000 timesteps.
We utilized an Nvidia RTX3090 GPU and the Adam optimization algorithm~\cite{kingma2014adam} to train the model with $\lambda=20$, $\lambda_\phi=20$, $\gamma=1$, batch size $\mathbb{B}=1$, a learning rate of $2\times10^{-4}$, and a maximum of 700 epochs.

\subsubsection{Evaluation Metrics}
We employed three evaluation metrics, i.e., DICE, $|J|\leq0(\%)$, and $SD(|J|)$ to measure the image registration performance, following existing registration methods~\cite{balakrishnan2019voxelmorph,kim2022diffusemorph,kim2021cyclemorph}. 
DICE measures the spatial overlap of anatomical segmentation maps between the warped moving image and the fixed reference image.
A higher Dice score indicates better alignment between the warped moving image and the fixed reference image, thus reflecting an improved registration quality.
$|J|\leq0(\%)$ indicates the percentage of non-positive values in the Jacobian determinant of the registration field. 
This metric indicates the percentage of voxels that lacks a one-to-one registration mapping relation,  causing unrealistic deformations and roughness. 
$SD(|J|)$ refers to the standard deviation of the Jacobian determinant of the registration field. A lower standard deviation indicates that the registration field is relatively smooth and consistent across the image.


\subsubsection{Compare with the State-of-the-art Methods}
Table~\ref{tab1} shows the comparison of our method with existing state-of-the-art methods including VoxelMorph~\cite{balakrishnan2019voxelmorph}, VoxelMorph-Diff~\cite{dalca2019unsupervised}, and DiffuseMorph~\cite{kim2022diffusemorph} on the same training and testing dataset. We produced baseline results using the recommended hyperparameters in their paper.
%we present the average Dice score for the segmentation masks of the myocardium(Myo), left ventricle(LV), right ventricle(RV), and the overall result. 
The result shows that our proposed method outperforms existing baseline methods by a substantial margin (Wilcoxon signed-rank test, $p<0.005$) (Also see Fig.~\ref{fig:compare_fig}). Furthermore, our method aligned better in areas where larger deformation happened, such as myocardium (myo).
%The average runtime of our method is 0.581s, which is on par with other deep-learning-based algorithms~\cite{balakrishnan2019voxelmorph,kim2022diffusemorph}.
%It demonstrates that guided by the diffusion model's feature-level semantics, the registration decoder can obtain more informative features for generating deformation fields, thus getting better registration accuracy (Also see Fig.~\ref{fig:attn_fig}). \xmli{this conclusion can't be obtained by seeing Table 1.} The results also show that guided by diffusion scores, the network's optimization process can better concentrate on hard-to-register areas, thus better preserving the deformation topology. \xmli{The same}

\begin{table}[!h]
\centering
\caption{Image registration results with standard deviation in parenthesis on the 3D cardiac dataset. ``LV'', ``Myo'', ``RV'' refers to Left Ventricle, Myocardium, and Right Ventricle, respectively. ``Overall'' refers to the averaged registration result of the left blood pool, myocardium, left ventricle, right ventricle, and these total region, following~\cite{kim2022diffusemorph}. $\uparrow$: the higher, the better results. $\downarrow$: the lower, the better results.}\label{tab1}
\resizebox{\textwidth}{!}{%
\begin{tabular}{c|cccc|cc}
\hline
\multirow{2}{*}{Method}                                 & \multicolumn{4}{c|}{DICE $\uparrow$}                                               & \multirow{2}{*}{$|J|\leq0(\%) \downarrow$} & \multirow{2}{*}{$SD(|J|) \downarrow$}\\ \cline{2-5}
                                       & LV           & Myo                & RV           & Overall                                           \\ \hline
Initial                                & 0.585(0.074) & 0.357(0.120)  & 0.741(0.069) & 0.655(0.188)          & - & -                       \\
VM~\cite{balakrishnan2019voxelmorph}    & 0.770(0.086) & 0.679(0.129)  & 0.816(0.065) & 0.799(0.110)          & 0.079(0.058)&0.183             \\
VM-Diff~\cite{dalca2019unsupervised}    & 0.755(0.092) & 0.659(0.137)  & 0.815(0.066) & 0.789(0.117)          & 0.083(0.063)&0.182             \\
DiffuseMorph~\cite{kim2022diffusemorph} & 0.783(0.086) & 0.678(0.148)  & 0.821(0.067) & 0.805(0.114)          & 0.061(0.038)&0.178             \\
\textbf{Ours}                          & \textbf{0.809(0.077)} & \textbf{0.724(0.119)}  & \textbf{0.827(0.061)} & \textbf{0.823(0.096)} & \textbf{0.054(0.026)}&\textbf{0.176}    \\ \hline
\end{tabular}%
}
\end{table}

% Figure environment removed

% Figure environment removed

\subsubsection{Ablation Study}
To validate the effectiveness of our proposed learning strategies, including the Feature-wise Diffusion-Guided module(FDG) and Score-wise Diffusion-Guided module(SDG), we conducted ablative experiments, as shown in Table~\ref{tab2}. The network without FDG also uses the denoising diffusion decoder but generates the deformation field from the encoded feature directly, and without SDG means that we optimize the network using the vanilla NCC loss. By integrating multi-scale intermediate latent diffusion features into generating deformation fields, we can see that the network's performance increased by 1\%. By deploying the reweighing loss, the Jacobian metric decreased by 60.5\%. The result achieved a balance when all components were deployed. These results demonstrated that our proposed components could effectively guide the deformation field generation by using multi-scale diffusion features (Also see Fig.~\ref{fig:attn_fig}). Optimization guided by diffusion score led to better preservation of deformation topology. It is worth noticing that the results without FDG or SDG showed only marginal improvement over baseline results, indicating the importance of feature-level deformation field generation and the reweighing scheme. The ablative study of hyperparameter $\lambda$ is illustrated in Supp. Fig. 1.

\begin{minipage}{0.95\textwidth}
\begin{minipage}[t]{0.48\textwidth}
\makeatletter\def\@captype{table}
%\begin{table}[htb]
\centering
\caption{Ablation study on FDG and SDG.}\label{tab2}
\resizebox{0.9\textwidth}{!}{%
\begin{tabular}{cc|cc}
\hline
FDG & SDG & DICE$\uparrow$ & $|J|\leq0(\%)\downarrow$ \\ \hline
                                   &                 &0.811(0.098)      &0.114(0.076)   \\
              $\surd$                    &                 & 0.818(0.102)     & 0.062(0.038)  \\
                                   & $\surd$               & 0.810(0.188)     & 0.045(0.025)  \\
 $\surd$                    & $\surd$               & 0.823(0.096) & 0.054(0.026)   \\\hline
\end{tabular}%
}
%\end{table}
\end{minipage}
\begin{minipage}[t]{0.48\textwidth}
\makeatletter\def\@captype{table}
\centering
\caption{Ablation Study on hyperparameter $\gamma$ in SDG.}\label{tab3}
\resizebox{\textwidth}{!}{%
\begin{tabular}{c|ccc}
\hline
$\gamma$ & 0.5 & 1 & 2 \\ \hline
DICE$\uparrow$                  &0.817(0.100)     &0.823(0.096)   &0.816(0.104)  \\
$|J|\leq0(\%)\downarrow$                     &0.069(0.029)     &0.054(0.026)   &0.042(0.023)  \\ \hline
\end{tabular}%
}
\end{minipage}
\end{minipage}


%lambda, exponential.

\subsubsection{Analysis of $\gamma$}
%\xmli{This para shows the new table with analysis of $\lambda$.} 
Furthermore, to validate SDG's effectiveness on topology preservation, we conducted another ablative study on SDG's hyperparameter $\gamma$, as Table~\ref{tab3} shows. Increased $\gamma$ indicates a more substantial reweighing effect. The results showed that by adding stronger reweighing influence, we could obtain deformation fields with better topology preservation almost without compromising accuracy.
\iffalse
\begin{table}[htb]
\centering
\caption{Ablation Study on hyperparameter $\gamma$}\label{tab3}
\resizebox{0.48\textwidth}{!}{%
\begin{tabular}{c|ccc}
\hline
$\gamma$ & 0.5 & 1 & 2 \\ \hline
DICE$\uparrow$                  &0.817(0.100)     &0.823(0.096)   &0.816(0.104)  \\
$|J|\leq0(\%)\downarrow$                     &0.069(0.029)     &0.054(0.026)   &0.042(0.023)  \\ \hline
\end{tabular}%
}
\end{table}
\fi

\iffalse
\begin{table}[]
\label{tab:tab1}
\begin{tabular}{cccccc}
\hline
Method        & DICE$\uparrow$                  & $|J|\leq0(\%)\downarrow$                   & PSNR(dB)$\uparrow$               & NMSE$\downarrow$                  & Time$\downarrow$                  \\ \hline
Initial       & 0.655(0.188)          & -                     & 23.813(2.852)          & 0.223                 & -                     \\
VM            & 0.781(0.110)          & 0.167(0.109)          & 28.092(4.850)          & 0.105(0.084)          & 0.561(1.259)          \\
VM-Diff       & 0.793(0.103)          & 0.284(0.188)          & 27.977(4.694)          & 0.163(0.072)          & 0.636(1.264)          \\
DiffuseMorph  & 0.809(0.117)          & 0.107(0.091)          & 28.105(3.545)          & 0.091(0.061)          & \textbf{0.438(1.126)} \\
\textbf{Ours} & \textbf{0.815(0.100)} & \textbf{0.066(0.037)} & \textbf{29.493(4.026)} & \textbf{0.072(0.055)} & 0.535(1.309) \\ \hline         
\end{tabular}
\end{table}
\fi