\section{Introduction}
Deformable image registration is the process of accurately estimating non-rigid voxel correspondences, such as the deformation field, between the same anatomical structure of a moving and fixed image pair. Fast, accurate, and realistic image registration algorithms are essential to improving the efficiency and accuracy of clinical practices. 
By observing dynamic changes, such as lesions, physicians can more comprehensively design treatment plans for patients~\cite{giger2013breast,jain2021amalgamation}.
When images during surgery align with preoperative ones, surgeons can locate instruments better and improve surgical prognosis~\cite{alam2018medical}. 
As reported in~\cite{khalil2018overview}, cardiac image registration is especially vital in improving heart chamber analysis accuracy, correcting cardiac imaging errors, and guiding cardiac surgeries. 
Thus, several studies have explored classical~\cite{klein2009elastix,avants2008symmetric} and deep-learning-based~\cite{balakrishnan2019voxelmorph,mok2020large,kim2021cyclemorph,huang2021difficulty} registration methods over the years.

% Accordingly, various works have studied the classical~\cite{klein2009elastix,avants2008symmetric} and deep-learning-based~\cite{balakrishnan2019voxelmorph,mok2020large,kim2021cyclemorph,huang2021difficulty} registration methods over the years.

Classical registration methods~\cite{klein2009elastix} used hand-crafted features to align images by solving computational-expensive optimization problems. Recently, researchers explored the deep-learning-based unsupervised deformable image registration~\cite{balakrishnan2019voxelmorph,kim2021cyclemorph,mok2020fast,mok2020large} to address the computational burden while reducing the need for accurate ground truth in the registration task. 
VoxelMorph~\cite{balakrishnan2019voxelmorph}, as the baseline, took moving and fixed image pairs as the input and maximized image pair similarity to train a registration network. To achieve higher accuracy, most unsupervised methods adopted a cascaded network with several sub-networks or an iterative refinement strategy~\cite{mok2020large,che2023amnet,huang2021difficulty,kim2021cyclemorph}. 
These strategies made the training procedure complicated and computational resources demanding. 
Meanwhile, to obtain smoother and more realistic deformation fields, i.e., topology preservation, many existing works introduced explicit diffeomorphic constraints~\cite{dalca2019unsupervised,mok2020fast,krebs2018unsupervised} or additional calculations on cycle consistency~\cite{kim2021cyclemorph}. For example, CycleMorph~\cite{kim2021cyclemorph} utilized the bidirectional registration consistency to preserve the topology during training. VoxelMorph-Diff~\cite{dalca2019unsupervised} adopted velocity field-based deformation field and new diffeomorphic estimation. SYMNet~\cite{mok2020fast} used symmetric deformation field estimation to achieve the goal. However, these schemes did not fully exploit the inherent network features, thereby overlooking these features' ability for better topology preservation.

%These schemes also need redundant calculation in the training process, therefore increasing the complexity and the need for computation resources.
%For instance, LapIRN[] utilized Laplacian feature pyramids to estimate the deformation field in multi resolution. AMNet[] estimated the local importance map and energy map in multiple scales to guide the generation of the deformation field. [2021MIA,diffculty aware] used multiple sub-networks to iterative refine the difficult patches of the registration input. 
% Figure environment removed

Recently, Kim~\textit{et al.}~\cite{kim2022diffusemorph} first proposed a diffusion model~\cite{ho2020denoising}, which is simpler to train than other generative models yet rich in semantics, for the registration task. They used the latent feature from the diffusion model's score function, i.e., the gradient field of a distribution's log-likelihood function~\cite{song2020score}, as one of the registration network's inputs for a better registration result. However, this method only used the final diffusion score as an image level guidance, which \textit{ignored diffusion model's rich task-specific semantics in the feature levels}, as proven in~\cite{kwon2023diffusion,baranchuk2022labelefficient,tumanyan2022plug}. 
%Therefore, the features learned at the hidden layers of the registration network could not be directly guided by the diffusion model's latent semantics, which reduced the informativeness of features for image registration. 
This resulted in the latent semantics of the diffusion model not being able to directly guide the features learned at the hidden layers of the registration network. As a result, the informativeness of these features for image registration was reduced. 
Moreover, this method only preserved deformation topology by simply using the diffusion score as the input, thereby \textit{ignoring the informative details about areas where significant deformations occur}; see Fig.~\ref{fig:intro_fig}.d for unexploited informative semantics. 
Therefore, the registration network was unable to explicitly prioritize hard-to-register areas, thereby limiting its effectiveness in preserving the deformation topology.
% Therefore, the registration network  could not explicitly focus on hard-to-register areas, limiting its ability to preserve the deformation topology effectively.

%We regard DiffuseMorph \cite{kim2022diffusemorph} as the baseline as it is the first and only work to integrate the score-based generative model with the registration task.

To address these issues, we present two novel modules, namely \textbf{F}eature-wise \textbf{D}iffusion-\textbf{G}uided Module (\textbf{FDG}) and \textbf{S}core-wise \textbf{D}iffusion-\textbf{G}uided Module (\textbf{SDG}) in the registration network. 
FDG introduces a direct feature-wise diffusion guidance technique for generating deformation fields by utilizing cross-attention to integrate the intermediate features of the diffusion model into the hidden layer of the registration network's decoder.
Furthermore, we embed the feature-wise guidance into multiple layers of the registration network and produce the feature-level deformation fields in multiple scales. 
Finally, after obtaining deformation fields at multiple scales, we upsample and average them to generate the full-resolution deformation field for registration.
%we combine the multiple feature-level deformation field to generate the final deformation field. \xmli{This sentence is confusing. }
Our SDG introduces explicit score-wise diffusion guidance for deformation topology preservation by reweighing the similarity-based unsupervised registration loss based on the diffusion score. Through this reweighing scheme, direct attention is given during the optimization process to ensure the preservation of the deformation topology. Our main contribution can be summarized as follows:

\begin{itemize}
\item We propose a novel feature-wise diffusion-guided module (FDG), which utilizes multi-scale intermediate features from the diffusion model to effectively guide the registration network in generating deformation fields. 
   
\item We also propose a score-wise diffusion-guided module (SDG), which leverages the diffusion model's score function to guide deformation topology preservation during the optimization process without incurring any additional computational burden. 

\item Experimental results on the cardiac dataset validated the effectiveness of our proposed method.
\end{itemize}

% more efficient -> converge faster, more stable, more compact, more accurate, multi-task way of doing this task.