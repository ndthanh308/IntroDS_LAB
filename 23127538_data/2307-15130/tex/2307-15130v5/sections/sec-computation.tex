\section{Computation}
\label{sec:computation}

In this section, we show that given an $n$-assignment $\phi$  and $\psi$, we can compute the loss function $L_B(\phi,\psi)$ in polynomial time. 
For simplicity, we describe the algorithm explicitly in the case where $d=1$ for clarity of exposition, before addressing the run time in higher dimensions. 

%
%

%
\subsection{Data Structures for $d=1$}
\label{ssec:DataStructures}
%
In this section, we build a pair of graphs representing a pair of input functors $F,G$, and use pointers between the vertex and edge sets to represent a given $n$-assignment $\phi$ and $\psi$.
For ease of exposition, we will carefully focus on the case $d=1$, following the example of Fig.~\ref{fig:DataStructureExample} to illustrate our construction. 
While we acknowledge an overuse of the term ``graph'' throughout this paper as it is used in many different contexts, for this section we use the term graph to mean a network; i.e.~a combinatorial pair consisting of vertices and edges. 
Throughout, we denote edges interchangeably by either $(x,y)$ or $xy$ depending on the context and notational complexity. 

At a high level, we will construct graphs for $F$ and $G$, which we denote by $(V_F,E_F)$ and $(V_G, E_G)$.
Then we will build data structures to encode the natural transformations $\phi$ and $\psi$.
For clarity, we use $\phitt$ and $\psitt$ to denote a collection of pointers that will store the information of $\phi$ and $\psi$, respectively.
These will be viewed as set maps\footnote{This notation is meant to imply that $\phitt$ is composed of two maps,  $\phitt:V_F \to V_G$ and $\phitt:E_F \to E_G$.} $\phitt:(V_F,E_F) \to (V_G,E_G)$ and $\psitt:(V_G,E_G) \to (V_F,E_F)$, which map each vertex to a vertex in the other graph and each edge to an edge in the other graph. 
We give explicit constructions of $\phitt$ and $\psitt$ as well as further restrictions on allowable maps in what follows.


Focusing on $d=1$,  the complex $K$, i.e.,~the discretization of $\R$, consists of vertices which we write as $\sigma_{-L},\cdots,\sigma_L$ with heights in our bounding box $[-L\delta,L\delta]$, and with edges $\tau_j = (\sigma_j, \sigma_{j+1})$. 
Then we construct the graph for $F:\Open(\cU) \to \Set$ by generating a vertex for every object in every $F(S_{\sigma_i})$ and connect them using the morphisms of the functor.
This results in a vertex set 
$$V_F = \coprod_{i =1}^B F(S_{\sigma_i}),$$
and an edge for every object in every $F(S_{\tau_i})$, giving edge set 
$$E_F = \coprod_{i =1}^{B-1} F(S_{\tau_i}).$$ 
The endpoints of any edge $e \in F(S_{\tau_i}) \subseteq E_F$ can be found via the attaching maps: 
\begin{align*}
F[S_{\tau_i} &\subseteq S_{\sigma_{i}}](e) \in F(S_{\sigma_i}) 
    \text{ and} \\
F[S_{\tau_i} &\subseteq S_{\sigma_{i+1}}](e) \in F(S_{\sigma_{i+1}}). 
\end{align*}
For example,  $e = (v_4,v_6) \in F(S_{\tau_4})$ in Fig.~\ref{fig:DataStructureExample} has endpoints $v_6 \in F(S_{\sigma_4})$ and $v_4 \in F(S_{\sigma_5})$. 
We store this data in a standard adjacency list.
In addition, each vertex also keeps track of its height, so a vertex $v \in F(S_{\sigma_i})$ also stores the value $i$ as a representation of its height.


% Figure environment removed

Next, we encode the information for an assignment $(\phi,\psi)$ between $F,G: \Open(\cU) \to \Set$ by constructing the set maps $\phitt$ and $\psitt$ using the graphs $(V_F,E_F)$ and $(V_G,E_G)$.
Assume we are given $n$-assignments $\phi$ and $\psi$. 
we start by focusing on the vertex set. 
For this, we need to represent the map
$\phi_{S_{\sigma_i}}:F(S_{\sigma_i}) \to G(S_{\sigma_i}^n)$. 
The elements of $F(S_{\sigma_i})$ are already given as vertices, however the elements of $G(S_{\sigma_i}^n)$ are not.
But, because of the cosheaf structure of $G$, the elements of $G(S_{\sigma_i}^n)$ can be seen as the connected components of particular subsets of the $(V_G,E_G)$. 

We emphasize that we are using the term ``subset" because the resulting objects will not be subgraphs in the usual sense. 
These subsets will consist of a subset of the vertex set and a subset of the edge set $(V', E')$, but without the promise that both endpoints of an edge are included in the set. 
However, we can still define connected components in this setting to consist of vertices and edges which can be connected by paths. 

For some vertex $\sigma_i$ of $K$, let 
\[
V_{G,\sigma_i,n} = \{ v \mid v \in G(S_{\sigma_j}), j \in [i-n,i+n]\}
\]
and 
\[
E_{G,\sigma_i,n} = \{e \mid e \in G(S_{\tau_j}), j \in [i-n-1,i+n] \}
\]
and write $(V_G,E_G)_{\sigma_i,n}:=(V_{G,\sigma_i,n}, E_{G,\sigma_i,n})$.
Similarly for the edges of $K$, we can define 
\[
V_{G,\tau_i,n} = \{ v \mid v \in G(S_{\sigma_j}), j \in [i-n+1,i+n]
\]
and 
\[
E_{G,\tau_i,n} = \{e \mid e \in G(S_{\tau_j}), j \in [i-n,i+n] \}
\]
with notation $(V_G,E_G)_{\tau_i,n}:=(V_{G,\tau_i,n}, E_{G,\tau_i,n})$. 
Note that because of the endpoints, these are not induced subgraphs; see Fig.~\ref{fig:Assignment} for examples.
We call either $(V_G,E_G)_{\tau_i,n}$ or $(V_G,E_G)_{\sigma_i,n}$ a \emph{slice} of the graph since they are each a portion of the graph which gives the connected components over an open interval as seen in the following lemma. 

\begin{lemma}
The elements of $G(S_{\sigma_i}^n)$ (resp.~$G(S_{\tau_i}^n)$) are in one-to-one correspondence with the connected components of the subset of the graph $(V_G,E_G)_{\sigma_i,n}$ (resp.~$(V_G,E_G)_{\tau_i,n}$). 
\end{lemma}
\begin{proof} 
This lemma is immediate from noting that $G(S_{\sigma_i}^n)$ is the colimit of the diagram given by $G(S_\sigma)$ for $S_\sigma \subset S_{\sigma_i}^n$ where $\sigma$ is taken over all cells  in $K$ (of both dimension 0 and 1), and then carefully tracking indices of these cells from the above notation.
The edge version is similar. 
\end{proof}




With this, we can return to storing a given $n$-assignment $\phi$ and $\psi$. 
To store the unnatural transformation $\phi$, for each $v \in F(S_{\sigma_i})$, we choose a vertex $\phitt(v) \in V_{G,\sigma_i,n}$, where $\phitt(v)$ is in the connected component of $(V_{G}, E_G)_{\sigma_i,n}$ represented by $\phi_{S_{\sigma_i}}(v) \in G(S_{\sigma_i}^n)$.
We note that given a collection of choices of vertices 
$$\{ \phitt(v) \in G(S_{\sigma_j})\subset V_G \mid v \in F(S_{\sigma_i}) \subset V_F,\, |i-j| \leq n\}$$
and edges 
$$\{ \phitt(e) \in G(S_{\tau_j}) \mid e \in F(S_{\tau_i}) \subset V_F,\, |i-j| \leq n\}$$
we can immediately reconstruct an unnatural transformation $\phi$  by setting $\phi_{S_{\sigma_i}}(v)$ to be the element of $G(S_{\sigma_i}^n)$ representing the connected component of $(V_G,E_G)_{\sigma_i,n}$ containing  $\phitt(v)$. 
As these processes are inverse of each other and the parallel setup can be done for $\psi$ and $\psitt$, we are justified in using this data structure to represent the assignment.  



%
%
%

For an example, consider Fig.~\ref{fig:Assignment} where we assume $n=1$. 
If  $\phitt(b) = w$, then $\phi_{S_{\sigma_i}}(b)$ is the connected component that includes $w$ of $(V_G,E_G)_{\sigma_i,1}$ as shown on the right.  
We can similarly find the edge map $\phitt(e)$ for $e \in F(S_{\tau_i})$ by setting it to be an edge in $E_{G,\tau_i,n}$ representing the connected component of $\phi_{S_{\tau_i}}(e) \in G(S_{\tau_i}^n)$ in $(V_G,E_G)_{\tau_i,n}$.
So, for example, in Fig.~\ref{fig:Assignment} where $n=1$, the input data might have $\phitt(ab) = (xy) \in E_G$ and $\phitt(bc) = (uv) \in E_G$. 


% Figure environment removed



%
\subsection{Algorithm and Complexity for $d=1$}
\label{ssec:Complexity}
In this section, we discuss the complexity of determining $L_B(\phi,\psi)$ given $\phitt$ and $\psitt$. 
First, we will proceed using a binary search on $k \in [0,\cdots, 2L]$ where the maximum is determined by the diameter of the bounding box. 
For a fixed $k$, we will determine if $L_B(\phi,\psi) \leq k$ by checking if $\Lpl^{S_\tau, S_\sigma}$,  $\Ltd^{S_\sigma}$, 
$\Lpr^{S_\tau, S_\sigma}$ and $\Ltu^{S_\sigma}$ are all less than $k$ for all $\sigma$ and $\tau$ in $K$. 
We will describe the cases for 
$\Lpl^{S_\tau, S_\sigma}$ and $\Ltd^{S_\sigma}$, as 
$\Lpr^{S_\tau, S_\sigma}$ and $\Ltu^{S_\sigma}$ are symmetric.
%

Start with $\Lpl^{S_\tau, S_\sigma}$ and note that in the case where $d=1$, there are two pairs necessary to check for each edge: $\tau_j, \sigma_j$ and $\tau_j,\sigma_{j+1}$.
Fix $\sigma_\ell$ to be either $\sigma_j$ or $\sigma_{j+1}$.
For each edge $e \in F(S_{\tau_j})$, we need to check if the two possible images in $G(S_{\sigma_\ell}^{n+k})$ under the diagram
\begin{equation}
\label{eqn:dgm:parallel_extend_basis}
\begin{tikzcd}
        F(S_{\tau_i})  
            \ar[r, "{F[\subseteq ]}"] 
            \ar[dr, "\phi_{S_{\tau_i}}"',  violet]
        & F(S_{\sigma_\ell})
            \ar[dr, "\phi_{S_{\tau_i}^n}",  violet]
        & & e \ar[r,mapsto] \ar[dr, mapsto]
        & v \ar[dr, mapsto] \\
        & G(S_{\tau_i}^n) 
            \ar[r, "{G[\subseteq ]}"'] 
        & G (S_{\sigma_\ell}^n) \ar[r] 
        & G(S_{\sigma_\ell}^{n+k})
        & \substack{\\{[e']}} \ar[r, mapsto, shift right] 
        & \substack{[w]\\{[e']}} \ar[r, mapsto, shift left] \ar[r, mapsto, shift right] 
        & \substack{[w]\\{[e']}}
\end{tikzcd}
\end{equation}
are the same. 
Note that we use $[-]$ to note that the elements represent the connected component in the relevant slice of the graph containing that edge or vertex. 
Following the top of diagram  Eq.~\eqref{eqn:dgm:parallel_extend_basis}, we know that $e$ has a unique endpoint vertex  $v \in F(S_{\sigma_\ell})$, and that vertex has an image under $\phi_{S_{\tau_i}^n}$,  which is a connected component represented by  $\phitt(v) = w \in V_G$. 
Following down, the edge $e$ has an edge image $\phitt(e) = e' \in E_G$. 
So the question becomes: are $e'$ and $w$ in the same connected component of $(V_G,E_G)_{\sigma_\ell,n+k}$? 
This can be done by filtering through the adjacency lists, keeping only vertices and edges in the correct strip, and then checking for connectivity using a standard graph traversal like breadth or depth first search. 
In particular, we run a variant of breadth first search one time which labels the connected components of this slice graph, and then the two images of each edge starting from $F(S_{\tau_i})$ can be checked in constant time~\cite[Sec.~5.6]{Jeffe-book}. 
This results in a total time (when $d=1$) of $O(|V_G|+|E_G|)$ time taken for checking the parallelogram.
There are $2L$ dimension 1 cells in $K$, and after computing connected components for the two parallelogram diagrams, all edges in a cell can have each parallelogram diagrams checked in $O(1)$ time, thus the time to determine if 
$\max_{\sigma \leq \tau} \Lpl^{S_\tau, S_\sigma} \leq k$ 
 is $O(L \cdot (|V_G|+|E_G|))$.
%

In the example of Fig.~\ref{fig:Assignment}, assume $n=k=1$ and assume the given input $\phitt$ is as noted. 
Then  for the diagram of Eq.~\eqref{eqn:dgm:parallel_extend_basis} with $\ell = j$ and chasing $bc \in F(S_{\tau_j})$,
this comes down to checking if the connected component of $\phitt(b) = w$ and $\phitt(bc) = xy$ are the same in the portion of $(V_G,E_G)_{\sigma_j,2}$.
In this particular example, there are two connected components in this slice and the images are not in the same component. 
Then we know that $\Lpl^{S_{\tau_j}, S_{\sigma_j}}>k$ so we would skip all other commutative diagram checks and immediately move on in our binary search. 
If it were the case that the two images were in the same connected component, then $\Lpl^{S_{\tau_j}, S_{\sigma_j}}\leq k$ and thus we would move on to the next commutative diagram check.


Checking if $\Ltd^{S_\tau} \leq k$ is similar so we briefly highlight the differences.  
First, there are two types of basis elements in our case where $d=1$, so we need to check  $\Ltd^{S_{\sigma_i}} \leq k$ (meaning checking vertices) and $\Ltd^{S_{\tau_i}} \leq k$ (meaning checking edges). 
We focus on the case of vertices since the edge version is similar. 
For any vertex $v \in F(S_{\sigma_i})$, we need to chase it around the diagram
\begin{equation}
\label{eqn:dgm:tri_extend_basis}
    \begin{tikzcd}
        F(S_{\sigma_i}) 
            \ar[rr, "{F[S_{\sigma_i} \subseteq S_{\sigma_i}^{2n}]}"]   
            \ar[dr, "\phi_{S_{\sigma_i}}"',violet]
            & & F(S_{\sigma_i}^{2n})  \ar[r] 
            & F(S_{\sigma_i}^{2(n+k)}).
        \\
        & G(S_{\sigma_i}^n)
            \ar[ur, "\psi_{S_{\sigma_i}^n}"', orange]  
        & \substack{v\\ \phantom{x}} 
            \ar[rr, mapsto, shift left] \ar[dr, mapsto]
        & & 
        \substack{{[v]}\\{[v']}}
            \ar[r, mapsto, shift left]
            \ar[r, mapsto, shift right]
        & \substack{{[v]}\\{[v']}}\\
        & & & w \ar[ur, mapsto]
    \end{tikzcd}
\end{equation}
If $\texttt{phi}: v \mapsto w$, and $\texttt{psi}: w \mapsto v'$, the question again becomes:  are $v$ and $v'$ in the same connected component of $(V_G,E_G)_{\sigma_j,2(n+k)}$?
Similar to the parallelogram case, we take a strip of the graph and check this connectivity question by finding connected components once in the slice in  $O(|V_G| +|E_G|)$ time, and then the check for each vertex in $F(S_{\sigma_i})$ is done in $O(1)$ time. 
As before, either the elements checked are in the same connected component of the relevant slice of the graph, in which case we move to the next diagram; or it does not, and we move to a different $k$ in our binary search.
There are $O(L)$ cells (counting both 0- and 1-dimensional cells) in $K$, meaning there are $O(L)$ triangle diagrams to check.
Thus, checking if $\max_{\sigma  \in K} \Ltd^{S_\sigma} \leq k $ can also be done in $O(L \cdot (|V_G| + |E_G| )$ time.

In our example case of Fig.~\ref{fig:Assignment} with $n=k=1$,  we have $2(n+k) = 4$.
Then chasing $b$, we need to check that $b$ and $\psitt \circ \phitt(b) = c$ are in the same connected component of $(V_G,E_G)_{\sigma_j,4}$. 
As this slice has one connected component, this triangle commutes. 
We can check another triangle $\Ltu^{S_{\sigma_j}} \leq k$ by chasing $w$. 
In this case, we must check if  $w$ and $\phitt \circ \psitt(w) = z$ are in the same component of $(V_G,E_G)_{\sigma_i,4}$, which again, they both are. 
In either case, if they were not, we would know the loss function is at least $k$ and continue in the binary search.

%
%
%
%
%
%
%
%
%
%
%
%
%
Putting this together, this means that if the graph representations of $F$ and $G$ are $(V_F,E_F)$ and $(V_G,E_G)$ respectively, the time for computing the loss function is 
\begin{equation*}
    O\Bigg( 
    %
    %
    L\log L  \cdot \max\Big\{|V_F|+|E_F|,|V_G|+|E_G| \Big\} \Bigg)
\end{equation*}
where the $\log L$ term comes from the binary search.

\subsection{Generalization for $d> 1$}

The generalization to higher dimensions makes relatively minor modifications for the algorithm, with the expected curse of dimensionality result in the running time. 
In this case, we build a graph with vertex set $V_F = \coprod_{\sigma \in K} F(S_\sigma)$ so that vertices represent all dimensional cubes rather than only 0 as earlier. 
An edge is given between every pair of vertices $v$ and $w$ for which $F[S_\tau \subseteq S_\sigma](v) = w$. 
Denote the sizes of these sets by $|V_F|$ and $|E_F|$. 
Note that these sizes are in some sense already hiding an exponential term in $d$ since the number of cells in the grid $K$ is $O(L^d)$. 



%
%
%

As in $d=1$, we proceed using a binary search on $[0,\cdots,2L]$ where $[-L\delta,L\delta]^d $ is the bounding box of the images of $f: \X \to \R^d$ and $g:\Y \to \R^d$.
Again, for a fixed $k$, we will determine if $L_B(\phi,\psi) \leq k$ by checking if $\Lpl^{S_\tau, S_\sigma}$,  $\Ltd^{S_\sigma}$, 
$\Lpr^{S_\tau, S_\sigma}$ and $\Ltu^{S_\sigma}$ are all less than $k$ for all $\sigma$ and $\tau$ in $K$. 

If we count in terms of the open sets $S_\sigma$, every set  $F(S_\sigma)$ needs to be checked as the starting point for one triangle diagram $\triangled_{\Phi, \Psi}(S_\sigma)$, and as the starting point for one parallelogram diagram $\Parallelograml_{\Phi}(S_\sigma,S_\tau)$ for every $\tau \geq \sigma$. 
The grid structure means there are worst case $O(2^d)$ adjacent cells, so this results in $1+O(2^d)$ diagram checks to be done per vertex. 
Each of these checks involves determining if two vertices are in the same connected component of the higher dimensional analogue of a ``strip'' for $\sigma_{\overrightarrow{\ell}}$ with indices $\overrightarrow\ell \in \Z^d$ involves checking  
a portion of the graph with indices in a $d$-dimensional box 
$[\ell_1-(n+k), \ell_1+(n+k) ] \times \cdots \times [\ell_d-(n+k), \ell_d+(n+k) ]$
 and hence takes worst case $O(|V_F|+|E_F|)$  time.
 However as before, this connected component needs to only be found once per diagram.
 %
 The result is a running time of 
 \begin{equation*}
O(\log L \cdot  2^d \cdot 
\max\{|V_F| + |E_F|, |V_G| + |E_G| \})
 \end{equation*}
