\section{Technical Background}
\label{sec:background}

We will assume several example types of inputs in this paper. 
All are tied together by having data of the form $f:\X \to \R^d$, where $\X$ is a topological space.
In particular, we will require that $\pi_0(f\inv(U))$ is a finite set for some reasonable collection of open sets $U \subset \R^d$. 
We thus assume that $\X \subset \mathbb{R}^k$ is a semi-algebraic set and that $f$ is a semi-algebraic map since this results in our desired restrictions.



%
%
%


\subsection{Functors and Cosheaves}
\label{ssec:cosheaf}

We give basic definitions for the category theoretic notions required in this paper, and direct the interested reader to \cite{Riehl2017,Curry2014} for further details. 
A \emph{category} $\cC$ consists of a collection of objects $X,Y,Z,\dots$ and morphisms $f,g,h,\dots$ with the following requirement: morphisms $f:X \to Y$ have designated domain $X$ and codomain $Y$; every object has a designated identity morphism $\1_X: X \to X$, and any pair of morphisms $f:X \to Y$ and $g:Y \to Z$ has a composite morphism $g \circ  f:X \to Z$. 
These objects and morphisms satisfy an identity axiom, where $f:X \to Y$ is the same as the $\1_Y \circ  f$ and $f \circ \1_X$; and composition (denoted by $\circ$ but often dropped when unnecessary) is associative, so $h\circ (g\circ f) = (h\circ g)\circ f$. 
Some example categories are $\Set$ where objects are finite sets and morphisms are set maps; 
$\Top$ where objects are topological spaces and morphisms are continuous functions;
and $\Open(\X)$ for a given topological space $\X$, where the objects are open sets and morphisms are given by inclusion. 

A \emph{functor} $F:\cC \to \cD$ is a map between categories preserving the relevant structures.
Specifically, for every object $X \in \cC$ there is a an object $F(X) \in \cD$, and for every morphism $f:X \to Y$, there is a morphism $F[f]:F(X) \to F(Y)$. 
To be a functor, $F$ must further satisfy that for  any $X \in \cC$, $F[\1_X] = \1_{F(X)}$ and for any composable pair $f,g \in \cC$, we have $F[gf] = F[g] F[f]$. 
Given functors $F,G: \cC \to \cD$, a \emph{natural transformation} $\eta: F \Rightarrow G$  consists of a map $\eta_X:F(X) \to G(X)$ for every $X \in \cC$ (called the components) so that for any morphism $f:X \to Y$ in $\cC$, the following diagram 
\begin{equation*}
\begin{tikzcd}
X\ar[d,  "f"] 
& F(X) 
    \ar[r, "\eta_X"] 
    \ar[d, "{F[f]}"']
& G(X) 
    \ar[d, "{G[f]}"]
\\
Y 
& F(Y) \ar[r, "\eta_Y"] 
& G(Y)
\end{tikzcd}
\end{equation*}
commutes. 
One example is $\pi_0:\Top \to \Set$, where $\pi_0(\X)$ is the set of path-connected components of $\X$, and morphisms are set maps $\pi_0[f]: \pi_0(\X) \to \pi_0(\Y)$ sending a connected component $A$ in $\X$ to the connected component of $f(A)$ in $\Y$.
Note that throughout the paper, we use the term \textit{component} to mean \textit{path-connected component}. 

A diagram is a functor $F:J \to \cC$ where $J$ is a small category. 
In essence, this construction picks out a collection of objects $F(j)$ and a collection of morphisms $F(j) \to F(k)$. 
A cocone on a given diagram is a natural transformation $\lambda:F \to c$ where we abuse notation to write that $c: J \to \cC$ is the constant functor returning the object $c(j) = c \in \cC$ for all $j \in J$. 
We often call the components $\lambda_j:F(j) \to c$ the \emph{legs}, and note that this requirement says that for any $f:j \to k$ in $J$, $\lambda_k \circ F[f] = \lambda_j$. 
A cocone $\lambda:F \to c$ is called a colimit if for any other cocone $\lambda':F \to c'$, there is a unique  morphism $u:c \to c'$ such that 
\begin{equation*}
\begin{tikzcd}
F(j) 
    \ar[rr, "{F[f]}"]
    \ar[dr, "\lambda_j"]
    \ar[ddr, "\lambda_j'"']
&& F(k)
    \ar[dl, "\lambda_k"']
    \ar[ddl, "\lambda_k'"]
\\
& c \ar[d, dashed, "u"] \\
& c'
\end{tikzcd}
\end{equation*}
commutes for all $f:j \to k$ in $J$. 

We will be particularly interested in functors of the form $F:\Open(X) \to \Set$, which can also be called \emph{pre-cosheaves}. 
A pre-cosheaf is a \emph{cosheaf} if it satisfies a gluing axiom,  meaning that $F(U)$ is entirely determined by $F(U_\alpha)$ for any cover $\{U_\alpha\}_\alpha$. 
Specifically, given an open set $U$ and a cover $\{ U_\alpha \mid \alpha \in A\}$ of $U$, define a category $\cU = \{U_\alpha \cap U_{\alpha'} \mid \alpha,\alpha' \in A \}$ with morphisms given by inclusion. 
Then we have a diagram $F:\cU \to \Set$, and as such can consider its colimit $\lambda:F \to L$.
If the unique map $L \to F(U)$ given by the colimit definition is an isomorphism, then $F$ is called a \emph{cosheaf}.

\subsection{Functorial Representation of Embedded Data}
\label{ssec:functorGraphs}

Assume we are given as input a pair of compact topological spaces with valued functions $f: \X \to \R^d$ and $g: \X \to \R^d$.
We will construct a cover  controlled  by diameter $\delta\geq 0$ of the images $f(\X) \cup g(\Y)$ in order to define the discretized mapper complex. 
Assume that a bounding box containing $f(\X) \cup g(\Y)$ can be written as $[-B,B]^d = [-L\delta, L\delta]^d$. 

Following \cite{Kaczynski2004}, $\delta$ induces a discretization of $[-L\delta, L\delta]^d$ into a cubical complex in the following way. 
An \emph{elementary interval} is an open interval in $\R$ of the form\footnote{Note our use of open intervals here in order to have open sets in our cover later, which differs from the definition given in \cite{Kaczynski2004}.} 
$(\ell\delta, (\ell+1)\delta)$ or a single point viewed as a degenerate interval
$[\ell]:= [\ell\delta, \ell\delta]$ for $\ell \in [-L,\cdots, L] \subset \Z$.
These are called non-degenerate and degenerate intervals, respectively. 
An elementary (open) cube $Q$ is a finite product of $d$ elementary intervals, i.e.
   $ \sigma = I_1 \times I_2 \times \cdots \times I_d \subset [-B,B]^d$. 
The dimension of a cube $\sigma$ is given by the number of intervals used which are non-degenerate. 
This means that 
$0$-cubes are vertices at grid locations 
$(i\delta, j \delta, \ldots, k \delta) \in \delta \cdot \Z^d$, 
$1$-cubes are edges (not including their endpoints), 
$2$-cubes are squares (not including their boundaries),
$3$-cubes are voxels, etc. 
The collection of elementary cubes discretizing $[-B,B]^d$ is a finite cubical complex $K$.
This construction comes with a face relation which gives a poset structure, where we write $\sigma \leq \tau$ iff $\sigma \subseteq \overline{\tau}$, where $\overline{\tau}$ denotes the closure of the set. 
In order to differentiate between the combintorial and continuous settings, we write $|\sigma|$ for the geometric realization  in $\R^d$ of a combinatorial object $\sigma \in K$. 

This complex $K$ induces a cover $\cU$ of $[-B,B]^d$ as follows. 
For any cube $\sigma \in K$, we can find the upper closure of $\sigma$ using the face relation, i.e.~$\sigma^\uparrow = \{ \tau \in K \mid \tau \geq \sigma\}$.
The \emph{cover element associated to $\sigma$} is  $U_\sigma = \bigcup_{\tau \in \sigma^{\uparrow}} |\tau|$.  
Then we write the cover as $\cU = \{U_\sigma \mid \sigma \in K\}$. 
Note that there is a poset relation on $\cU$ given by inclusion, and in particular, $U_\sigma \subseteq U_\tau$ iff $\tau \leq \sigma$. 




We next endow the poset $(\cU, \subseteq)$ with the Alexandroff topology, following \cite{Barmak2011}. 
For any set $S \subseteq \cU$, the upper closure, or \emph{upset}, is 
$S^{\uparrow} = \{U \in \cU \mid V \subseteq U, \, V \in S \}$ and the downward closure, or \emph{downset}, is
$S^{\downarrow} = \{ U \in \cU \mid U \subseteq V, \, V \in S\}$. 
We give $(\cU, \subseteq)$ the Alexandroff topology
$\Open(\cU)$, where a set $S\subseteq K$ is open iff the following holds: 
for any $U \in S$ and any $V \subseteq U$, $U \in S$. 
Equivalently, this means that $S$ is its own downset, i.e.~$S = S^\downarrow$.
See Fig.~\ref{fig:NewNotation} for an example of this notation in the case of $d=1$, where the open set $U_{\sigma_i}$ is associated to the point $\sigma_i  = i\delta$, and open set $U_{\tau_i}$ is associated to the edge $\tau_i = (i\delta,(i+1)\delta)$. 

% Figure environment removed


It can be checked that this topology has a basis given by the collection  
$\{S_\sigma\}_{\sigma \in K}$ 
where we write 
\begin{equation}
\label{eq:S_sigma}    
    S_\sigma := \{ U_\sigma\}^{\downarrow} 
        = \{U_\tau \in \cU  \mid U_\tau \subseteq U_\sigma \} 
        = \{U_\tau \in \cU \mid \tau \in \sigma^\uparrow\}
\end{equation}
and call $S_\sigma$ a \emph{basic open} set.
This complex is constructed so that for any subset $S \subset \cU$, the geometric realization $|S|:= \bigcup_{U \in S}U \subset \R^d$ is an open set; and further the notation is reasonable since the geometric realization of the basis set associated to $\sigma$ is the same as the open set associated to $\sigma$, i.e.~$|S_\sigma| = U_\sigma$ for all $\sigma \in K$. 
Again, see Fig.~\ref{fig:NewNotation} for an example of this notation.

We assume that the inputs $f:\X \to \R^d$ and $g:\Y \to \R^d$ are semi-algebraic maps defined on semi-algebraic sets in $\mathbb{R}^k$, 
%
%
as well as being compact as assumed earlier. Recall that the class of semi-algebraic sets is the smallest class of sets defined by a finite number of polynomial (in)equalities $\{x \in \mathbb{R}^k \mid p(x) \geq 0\}$ that is closed under complement, finite union, and finite intersection. A map $f\colon \X \to \mathbb{R}^d$ is semi-algebraic if the graph of $f$ is a semi-algebraic set in $\mathbb{R}^k \times \mathbb{R}^d$. A semi-algebraic set is \emph{semi-algebraically connected} if it cannot be written as the disjoint union of two non-empty open semi-algebraic sets. Analogously, a semi-algebraic set $X$ is \emph{path connected} if for any two points $x, x' \in X$ there is a continuous semi-algebraic map $\gamma \colon [0,1] \to X$ such that $\gamma(0) = x'$ and $\gamma(1) = y'$. Note that, by definition, a semi-algebraically path connected semi-algebraic set is also path connected. From this observation, combined with Theorem 2.4.5 and Proposition 2.5.13 in \cite{Bochnak1998}, we see that any connected semi-algebraic set is also path-connected. We make this restriction so that we have the following property. 

\begin{lemma}
Given a semi-algebraic map $f: \X \to \R^d$ with $\X$ semi-algebraic, and any semi-algebraic open set $U \subset \R^d$, $\pi_0(f\inv(U))$ is finite. 
\end{lemma}

We refer the reader to \cite{bpr} for an overview of the basic properties of semi-algebraic sets and maps, from which this lemma easily follows.

Because each grid cell itself is a semi-algebraic set, we have that for any open set $S \in \Open(\cU)$, the set of (path) connected components $\pi_0 f\inv(|S|)$ is finite. 

Then we have a functor $F:\Open(\cU) \to \Set$ given by 
\begin{equation*}
    \begin{matrix}
    F: &  \Open(\cU) & \to & \Set\\
    & S & \mapsto & \pi_0 f\inv(|S|).
    \end{matrix}
\end{equation*}
Functoriality of $\pi_0$ means that for $S \subseteq T$, there is an induced map
\[F[S \subseteq T] \colon \pi_0f\inv(|S|) \to \pi_0f\inv(|T|),\]
so that $F$ satisfies the requirements of a functor.
Indeed, this functor is actually a cosheaf so moving forward, we assume that $F$ and $G$ are cosheaves, even if they was not obtained from some input topological space. 
When the sets involves are obvious in the notation, we will  write the induced map as $F[\subseteq]:F(S) \to F(T)$. 

\subsection{Thickenings}
\label{ssec:thickenings}
% Figure environment removed
Given any set $S \in \Open(\cU)$, the 1-thickening\footnote{We note that this definition is distinct from the morphological operation of thickening that is a tool in image processing, although quite similar to the concept of dilation from that field.}  is defined by taking the downward closure of the upper closure  of $S$,  written as 
$   S^1 = (S^{\uparrow})^\downarrow$.
This operation can be thought of as taking the star of the closure of the set; see Fig.~\ref{fig:thickenings} for examples. 
The $n$-thickening is defined to be $n$ repetitions of the process given recursively as
\begin{equation*}
    S^n = 
    \begin{cases}
    S & n = 0\\
    (S^{n-1})^{\uparrow\downarrow} & n \geq 1.
    \end{cases}
\end{equation*}
Each $S^n$ is itself an open set in $\Open(\cU)$, and if $S \subseteq T$, then $S^n \subseteq T^n$. 
Thus we can view this operation as a functor on the category $\Open(\cU)$ with morphisms given by inclusion:
\begin{equation*}
    \begin{matrix}
    (-)^n: & \Open(\cU)  & \to &  \Open(\cU)\\
    & S & \mapsto & S^n.
    \end{matrix}
\end{equation*}
In Sec.~\ref{sec:technicalProofs}, 
we show that $(-)^n$ is a functor.
Because of the cosheaf assumption, we also have that $F(S^n) = \varinjlim_{S_\sigma \subset S^n}F(S_\sigma)$. 
Another useful property of this thickening, proved in Sec.~\ref{sec:technicalProofs}, is described in Lem.~\ref{lem:composedthickenings}.
\begin{restatable}{lemma}{composedThickenings}
\label{lem:composedthickenings}
For any $n, n' \geq 0$ and $S \in \Open(\cU)$, 
\[(S^{n})^{n'} = S^{n+n'}.\] 
\end{restatable}


We can use this thickening to build an interleaving distance on cosheaves of the form \linebreak
${F:\Open(\cU) \to \Set}$.
The first necessary ingredient is the composition of functors $F \circ ( - )^n: \Open(\cU) \to \Set$, which we  denote by $F^n$. 
This means $F^n(S) = F(S^n)$, followed by a similar setup for $G^n$. 
With this notation, an interleaving is a pair of natural transformations $\phi:F \Rightarrow G^n$ and $\psi:G \Rightarrow F^n$, so a component of $\phi$ is a set-map $\phi_S:F(S) \to G(S^n)$. 
There is another component at $S^n$, $\phi_{S^n}:F(S^n) \to G(S^{2n})$, which can also be viewed as a component of a different natural transformation $\phi^n:F^n \Rightarrow G^{2n}$. 
For this reason, we use the notation $\phi_{S^n}$ and $\phi_S^n$ interchangeably when $\phi$ is indeed a natural transformation.\footnote{We are implicitly using Lem.~\ref{lem:composedthickenings} to write the maps this way.}

We are now ready to introduce our notion of interleaving distance. 
\begin{definition}
\label{def:interleavingDistance}
Given cosheaves $F,G:\Open(\cU) \to \Set$ and $n \geq 0$, an \emph{$n$-interleaving} is a pair of natural transformations 
$\phi:F \Rightarrow G^n$
and 
$\psi:G \Rightarrow F^n$
such that the diagrams 
\begin{equation*}
    \begin{tikzcd}
        F(S) 
            \ar[rr, "{F[S \subseteq S^{2n}]}"]   
            \ar[dr, "\phi_S"',violet] 
            & & F(S^{2n}) & 
        & F(S^n) \ar[dr]
            \ar[dr, "\phi_{S^{n}}",violet]
        & \\
        & G(S^n)\ar[ur, "\psi_{S^n}"', orange]  & & 
        G(S) 
            \ar[rr, "{G[S \subseteq S^{2n}]}"']
            \ar[ur, "\psi_{S}", orange]  
        && G(S^{2n})
    \end{tikzcd}
\end{equation*} 
commute for all $S \in \Open(\cU)$.
The interleaving distance is given by 
\begin{equation*}
    d_I(F,G) = \inf\{ n \geq 0 \mid \text{there exists an $n$-interleaving} \}, 
\end{equation*}
and is set to be $d(F,G) = \infty$ if there is no interleaving for any $n$.
\end{definition}
As shown in Sec.~\ref{sec:technicalProofs}, this definition fits in the framework built by Bubenik et al.~\cite{Bubenik2014a} and thus it is an extended pseudometric. 
