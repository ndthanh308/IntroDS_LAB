\section{Extension to Reeb Graphs}
\label{sec:ReebLoss}

We now take a brief diversion into understanding how the loss function framework can be used to approximate the Reeb graph interleaving distance. 
In this case, we consider a 1-dimensional mapper graph to be an approximation of the Reeb graph 
\cite{Carriere2017,Carriere2018,Munch2016,Brown2020,botnan2020}. 
We show that in order to bound the Reeb graph interleaving distance, we can compute the mapper graph for a resolution $\delta$, and then use the loss function to provide a similar bound.

\subsection{Definitions}

Given input data $f:\X \to \R$, the Reeb graph of $(\X,f)$ is computed as follows. 
Define an equivalence relation by setting $x \sim y$ iff $x$ and $y$ are in the same path-connected component of the levelset $f\inv(a)$.
With enough restrictions on the space and function (for example, a Morse function on a manifold), the resulting Reeb graph is a topological graph; i.e.~a 1-dimensional stratified space. 
Similar to the vantage taken for the mapper graphs in this paper, the data of a Reeb graph can be stored in a cosheaf. 
\begin{definition}
    For a given $(\X,f)$, the associated Reeb cosheaf is given by 
    \begin{equation*}
        \begin{matrix}
            \tF: & \Int & \to & \Set \\
            & I & \mapsto & \pi_0 f\inv(I)\\
            & \rotatebox[origin=c]{-90}{$\subseteq$} &  & \downarrow \pi_0[\subseteq]\\ 
            & J & \mapsto & \pi_0 f\inv(J)\\
        \end{matrix}
    \end{equation*}
    where morphisms are induced by the $\pi_0$ functor. 
\end{definition}

For clarity, we write the Reeb cosheaf with a tilde to distinguish it from the mapper cosheaf without a tilde. 
Given this input, we have the Reeb graph interleaving distance \cite{deSilva2016}, given as follows.
\begin{definition}
\label{def:ReebInterleavingDistance}
    Define the functor $(-)^\e: \Int \to \Int$ by $(a,b) \mapsto (a-\e,b+\e)$ with morphisms induced by inclusion. 
    Then $\tF_\e:\Int \to \Set$ is given by $\tF_\e(J) = \tF(J^\e)$.

    For given $\tF,\tG: \Int \to \Set$, an $\e$-interleaving is a pair of natural transformations $\tphi:\tF \Rightarrow \tG_\e$ and $\tpsi:\tG \Rightarrow \tF_\e$ such that 
\begin{equation*}
    \begin{tikzcd}
        \tF(I) 
            \ar[rr, "{\tF[I \subseteq I^{2n}]}"]   
            \ar[dr, "\tphi_I"',violet]
            & & \tF(I^{2n}) & 
        & \tF(I^n) \ar[dr]
            \ar[dr, "\tphi_{I^{n}}",violet]
        & \\
        & \tG(I^n)\ar[ur, "\tpsi_{I^n}"', orange]  & & 
        \tG(I) 
            \ar[rr, "{\tG[I \subseteq I^{2n}]}"']
            \ar[ur, "\tpsi_{I}", orange] 
        && \tG(I^{2n})
    \end{tikzcd}
\end{equation*}
    commute for all $I \in \Int$. 
    The (categorical) Reeb graph interleaving distance is given by 
    \begin{equation*}
        d_R(\tF,\tG) = \inf\{ \e \geq 0 \mid \text{ there exists an $\e$-interleaving}\}.
    \end{equation*}
    
\end{definition}

Fix a $\delta$.
Following Sec.~\ref{sec:background}, denote the vertices of $K$ by $\{\sigma_{-L},\cdots,\sigma_L\}$ where $\sigma_i$ is at the point $i\delta \in \R$. 
Denote the edges by $\tau_i = (i\delta,(i+1)\delta)$ which has faces $\sigma_i$ and $\sigma_j$. 
Given some input data $f:\X \to \R$, we can either construct its Reeb cosheaf $\tF:\Int \to \Set$, or by fixing some choice of $\delta$, we can construct its mapper cosheaf $F:\Open(K) \to \Set,\, F(S) = f\inv(|S|)$. 

We next show that the loss function we have computed here on the mapper version $d_I$ can be used to similarly bound the Reeb interleaving distance $d_R$.
We do this by showing that $d_I$ is an approximation of $d_R$, which can be viewed as a special case of \cite[Thm.~5.15]{botnan2020};  however, for clarity, we include a direct proof in Sec.~\ref{sec:technicalProofs} as our setting allows a proof with considerably less use of category theoretic machinery.


\begin{restatable}{proposition}{reebvsmapperbound}
\label{prop:reebvsmapperbound}
%
    For inputs $f:\X \to \R$ and $g:\Y \to \R$, denote the respective Reeb cosheaves as $\tF,\tG:\Int \to \Set$, and the respective mapper cosheaves as $F,G: \Open(K) \to \Set$. Then 
    \begin{equation*}
        d_R(\tF,\tG) \leq \left(d_I(F,G) + 1\right)\delta. 
    \end{equation*}
\end{restatable}



Given this bound, we combine Prop.~\ref{prop:reebvsmapperbound} with Thm.~\ref{thm:secondBound} to show that the loss function for the mapper graph discretization bounds the Reeb graph interleaving as well and that, in particular, this bound is controlled by the diameter $\delta$ chosen for $K$. 

\begin{cor}
Given a basis $n$-assignment  
$\phi = \{\phi_{U_\sigma} \mid \sigma \in K\}$ 
and 
$\psi = \{\psi_{U_\sigma} \mid \sigma \in K\}$ for $F,G: \Open(K) \to \Set$, we have that 
\begin{equation*}
    d_R(\tF,\tG) \leq \delta(d_I(F,G) + 1)  \leq \delta(n + L_B(\phi,\psi)+1).
\end{equation*}
\end{cor}





