\section{Discussion}
\label{sec:discussion}

%

In this paper, we defined a loss function that quantifies how far a diagram is from being commutative, and used such a loss function to bound the interleaving distance, both for mapper and Reeb graph settings.
This work provides a way to evaluate a particular set of maps, which immediately suggests the question of utilizing this quantification to iteratively improve our comparison. 
Here, the quality of the bound is dependent on the quality of the input $n$-assignment, but we assume no control over that input in this paper and so we cannot evaluate the tightness of the bound. 
In the followup work \cite{Chambers2024}, we will use this bound in the context of an ILP framework, where an input $n$-assignment can be improved incrementally thus finding a better bound on the distance. 
The potential for not only getting better approximations but also returning the actual interleaving maps used in the bound is an exciting step toward computing interleaving distances for graph-based signatures available in practice. 
Of course, we know that deciding if two Reeb graphs are $\epsilon$-interleaved (for $\epsilon \ge 1$) is NP-hard
\cite{Bjerkevik2018}, so our ILP has no guarantee of reaching the global optimal solution. 
This is related to other available work providing bounds similarly lacking tightness guarantees for the interleaving distance restricted to merge trees \cite{Curry2022, Pegoraro2021}. 
The first of these (\cite{Curry2022}) uses the labeled merge tree distance \cite{Munch2018,Gasparovic2019} which has more limited properties when generalizing to Reeb graphs \cite{Lan2024} so it is unclear if this would generalize.
The latter uses the map formulations on the topological spaces that arise in the special case of merge trees, so major modifications would need to be made for this to apply to the Reeb graph setting as seen here. 
%

We believe that our loss function based framework is applicable in a broader context where data are modeled as sheaves or cosheaves in the category of sets, as sheaf theory is emerging as a tool in data science to study, e.g.,   distributed systems~\cite{Malcolm2009,Mansourbeigi2017}, sensor networks~\cite{Robinson2017}, model fit~\cite{KvingeJeffersonJoslyn2021}, and uncertainty quantification~\cite{JoslynCharlesDePerno2020}. 
In particular, one interesting next step is to study how to extend our framework to work with persistence modules as cosheaves in the category of vector spaces (e.g.,~\cite{BubenikMilicevic2021}). 
As the interleaving distance for multiparameter persistence modules is similarly NP-hard \cite{Bjerkevik2019}, this would be an exciting step toward computational efforts in this broad class of topological signatures.




