\documentclass[11pt]{article} 

%
\usepackage{geometry}
\geometry{verbose,letterpaper,vmargin=1.0in,hmargin=1.0in}

\usepackage[utf8]{inputenc}
\usepackage{amsthm}
\usepackage[dvipsnames]{xcolor}
\usepackage{tikz-cd}
\usepackage{quiver}
\usepackage{tipa}
\usepackage{float}
\usepackage[pdf]{pstricks}
\usepackage{pgf}
\usepackage{scalerel}
\usepackage{algorithm}
\usepackage{algpseudocode}

\usepackage{thmtools,thm-restate}

\usepackage[style=numeric,sorting=nyt,url = false]{biblatex}
\addbibresource{interleaving.bib}

\usepackage{authblk}

\algnewcommand\algorithmicinput{\textbf{Input:}}
\algnewcommand\INPUT{\item[\algorithmicinput]}
\algnewcommand\algorithmicoutput{\textbf{Output:}}
\algnewcommand\OUTPUT{\item[\algorithmicoutput]}

\algnewcommand\algorithmicproc{\textbf{Procedure:}}
\algnewcommand\PROCEDURE{\item[\algorithmicproc]}
\algnewcommand\algorithmiccomplexity{\textbf{Complexity:}}
\algnewcommand\COMPLEXITY{\item[\algorithmiccomplexity]}


\graphicspath{{./figures/}}

\usepackage{amsmath,amssymb,amsfonts}

\usepackage{hyperref}
\hypersetup{
    colorlinks=true,
    linkcolor=blue,
    filecolor=magenta,      
    urlcolor=cyan,
    citecolor = magenta,
    pdftitle={Approximate Interleavings},
}

\DeclareFontFamily{U}{mathb}{}
\DeclareFontShape{U}{mathb}{m}{n}{
  <-5.5> mathb5
  <5.5-6.5> mathb6
  <6.5-7.5> mathb7
  <7.5-8.5> mathb8
  <8.5-9.5> mathb9
  <9.5-11.5> mathb10
  <11.5-> mathb12
}{}
\DeclareSymbolFont{mathb}{U}{mathb}{m}{n}
\DeclareMathSymbol{\dl}{3}{mathb}{"EA}
\DeclareMathSymbol{\dr}{3}{mathb}{"EB}


\newtheorem{theorem}{Theorem}[section]
\newtheorem{lemma}[theorem]{Lemma}
\newtheorem{prop}[theorem]{Proposition}
\newtheorem{cor}[theorem]{Corollary}
\newtheorem{claim}[theorem]{Claim}
\newtheorem{definition}[theorem]{Definition}
\newtheorem{remark}[theorem]{Remark}
\newtheorem{conj}[theorem]{Conjecture}
\newtheorem{obs}[theorem]{Observation}
\newtheorem{notation}[theorem]{Notation}
\newtheorem*{theorem*}{Theorem}


%
%
%
%
%
%

\newcommand{\Asgn}{\mathrm{Asgn}}
\newcommand{\Func}{\mathrm{Func}}
%
%
\newcommand{\denselist}{\vspace{-5pt} \itemsep -2pt\parsep=-1pt\partopsep -2pt}

%
\newcommand{\rd}{\mathbin{\rotatebox[origin=c]{90}{$\dl$}}}

\newcommand*{\Parallelogramr}[1][]{%
  \pgfpicture\pgfsetroundjoin
    \pgftransformxslant{.6}%
    \pgfpathrectangle{\pgfpointorigin}{\pgfpoint{.60em}{.65em}}%
    \pgfusepath{stroke,#1}%
  \endpgfpicture}
  
\newcommand*{\Parallelograml}[1][]{%
  \pgfpicture\pgfsetroundjoin
    \pgftransformxslant{-.6}%
    \pgfpathrectangle{\pgfpointorigin}{\pgfpoint{.60em}{.65em}}%
    \pgfusepath{stroke,#1}%
  \endpgfpicture}
  
\newcommand{\triangled}{\raisebox{\depth}{$\bigtriangledown$}}
\newcommand{\triangleu}{\bigtriangleup}


%
\newcommand{\Lpl}{L_{\scaleobj{.7}{\Parallelograml}}}
\newcommand{\Lpr}{L_{\scaleobj{.7}{\Parallelogramr}}}
\newcommand{\Ltd}{L_{\bigtriangledown}}
\newcommand{\Ltu}{L_{\bigtriangleup}}



%
\newcommand{\R}{\mathbb{R}}
\newcommand{\X}{\mathbb{X}}
\newcommand{\Y}{\mathbb{Y}}
\newcommand{\Z}{\mathbb{Z}}

\newcommand{\cA}{\mathcal{A}}
\newcommand{\cB}{\mathcal{B}}
\newcommand{\cC}{\mathcal{C}}
\newcommand{\cD}{\mathcal{D}}
\newcommand{\cF}{\mathcal{F}}
\newcommand{\cG}{\mathcal{G}}
\newcommand{\cU}{\mathcal{U}}

\newcommand{\inv}{^{-1}}
\newcommand{\e}{\varepsilon}
\usepackage{dsfont}
\newcommand{\1}{\mathds{1}}
\renewcommand{\epsilon}{\varepsilon}
\renewcommand{\phi}{\varphi}



\newcommand {\mm}[1] {\ifmmode{#1}\else{\mbox{\(#1\)}}\fi}
\newcommand{\Rspace}        {\mm{{R}}}
\newcommand{\Xspace}        {\mm{{X}}}
\newcommand{\Sspace}        {\mm{{S}}}
\newcommand{\Yspace}        {\mm{{Y}}}
\newcommand{\Zspace}        {\mm{{Z}}}

\newcommand{\Ccal}        {\mm{{\mathcal C}}}
\newcommand{\Fcal}        {\mm{{\mathcal F}}}
\newcommand{\Pcal}        {\mm{{\mathcal P}}}
\newcommand{\Lcal}        {\mm{{\mathcal L}}}
\newcommand{\Xstrata}  {\mm{{\mathfrak X}}} 
\newcommand{\bdr}  {\mm{{\partial}}} 
\newcommand{\str}  {\mbox{{St}}} 
\newcommand{\lk}  {\mbox{{Lk}}}
\newcommand{\closure}  {\mbox{Cl}}
\newcommand{\dime}  {\mbox{dim}}

\newcommand{\Open}{\mathbf{Open}}
\newcommand{\Set}{\mathbf{Set}}
\newcommand{\Top}{\mathbf{Top}}
\newcommand{\Vect}{\mathbf{Vect}}

\newcommand{\id}{\mm{\text{id}}}


\newcommand{\Liz}[1]{{\color{violet}\textbf{Liz:} #1}}
\newcommand{\liz}[1]{\Liz{#1}} %
\newcommand{\Erin}[1]{{\color{WildStrawberry}\textbf{Erin:} #1}}
\newcommand{\Bei}[1]{{\color{magenta}\textbf{Bei:} #1}}
\newcommand{\Sarah}[1]{{\color{cyan}\textbf{Sarah:} #1}}
\newcommand{\todo}[1]{{\color{red}\textbf{Todo:} #1}}
%


%

%
\title{Bounding the Interleaving Distance \\ for Geometric Graphs with a Loss Function}
\author[1]{Erin W.~Chambers}
\author[2]{Elizabeth Munch}
\author[2]{Sarah Percival}
\author[3]{Bei Wang}
\affil[1]{St.~Louis University}
\affil[2]{Michigan State University}
\affil[3]{University of Utah}
%
\date{}

\begin{document}

\maketitle

\begin{abstract}
A geometric graph is an abstract graph along with an embedding of the graph into the Euclidean plane which can be used to model a wide range of data sets.  
The ability to compare and cluster such objects is required in a data analysis pipeline, leading to a need for distances or metrics on these objects.  
In this work, we study the interleaving distance on geometric graphs, where functor representations of data can be compared by finding pairs of natural transformations between them. 
However, in many cases, particularly those of the set-valued functor variety, computation of the interleaving distance is NP-hard.

For this reason, we take inspiration from the work of Robinson to find quality measures for families of maps that do not rise to the level of a natural transformation. 
Specifically, we call collections 
$\phi = \{\phi_U\mid U\}$ and $\psi = \{\psi_U\mid U\}$ which do not necessarily form a true interleaving an \textit{assignment}.  
In the case of embedded graphs, we impose a grid structure on the plane, treat this as a poset endowed with the Alexandroff topology $K$,  and encode the embedded graph data as functors $F: \Open(K) \to \Set$ where $F(U)$ is the set of connected components of the graph inside of the geometric realization of the set $U$. 
We then endow the image with the extra structure of a metric space and define a loss function $L(\phi,\psi)$ which measures how far the required diagrams of an interleaving are from commuting. 
Then for a pair of assignments, we use this loss function to bound the interleaving distance, with an eye toward computation and approximation of the distance. 
We expect these ideas are not only useful in our particular use case of embedded graphs, but can be extended to a larger class of interleaving distance problems where computational complexity creates a barrier to use in practice. 

%
%
%
%
%
%
%
%
%
%
%
%

%
\end{abstract}

\newpage
%
%
%
%
%
%
Clinical depression, a prevalent mental health condition, is considered as one of the leading contributors to the global health-related burden \cite{greenberg2015economic, lepine2011increasing}, affecting millions of people worldwide \cite{vos_et_al_GBD2016,institute2021global}. As a mood disorder, it is characterised by a prolonged (> two weeks) feeling of sadness, worthlessness and hopelessness, a reduced interest and a loss of pleasure in normal daily life activities, sleep disturbances, tiredness and lack of energy. Depression can lead to suicide in extreme cases \cite{goldney2000suicidal} and is often linked to comorbidities such as anxiety disorders, substance abuse disorders, hypertensive diseases, metabolic diseases, and diabetes \cite{steffen_et_al_2020_BMCPsychiatry,campayo2011diabetes}. Although effective treatment options are available, diagnosing depression through self-report and clinical observations presents significant challenges due to the inherent subjectivity and biases involved.

Over the last decade, researchers from affective computing and psychology have focused on investigating objective measures that can aid clinicians in the initial diagnosis and monitoring of treatment progress of clinical depression \cite{cohn2018multimodal, pampouchidou_et_al_TAC_DepressionReview}. A key catalyst to this progress is the availability of relevant datasets, such as AVEC2013 and subsequent challenges~\cite{valstar2013avec}. In recent years, research on depression detection employing affective computing approaches has increasingly focused on leveraging non-verbal behavioural cues such as facial expressions \cite{bourke2010processing, de2019combining}, body gestures \cite{joshi2013relative}, eye gaze \cite{alghowinem2016multimodal}, head movements \cite{alghowinem2013head} and verbal features \cite{cummins2011investigation, huang2019investigation} extracted from multimedia data to develop distinctive features to classify individuals as depressed or healthy controls, or to estimate the severity of depression on a continuous scale. 

In this study, we examine the utility of inherently interpretable head motion units, referred to as \emph{kinemes} \cite{madan_gahalawat_guha_subramanian_ICMI2021_Kinemes}, for assessing depression. Initially, we utilise data from both healthy controls and depressed patients to discover a basis set of kinemes via the (\emph{pitch}, \emph{yaw}, and \emph{roll}) head pose angular data obtained from short overlapping time-segments (termed two-class kineme discovery or 2CKD). Further, we employ these kinemes to generate features based on the frequency of occurrence of distinctive, class-characteristic kinemes. Subsequently, we discover kineme patterns solely from head pose data corresponding to healthy controls (Healthy control kineme discovery or HCKD), and use them to represent both healthy and depressed class segments. A set of statistical features are then computed from the reconstruction errors between the raw and learned head-motion segments corresponding to both the depressed and control classes (see Figure ~\ref{fig:Depression_proposed_framework}). Using machine learning methodologies, we evaluate the performance of the features derived from the two approaches. Our results show that head motion patterns are effective behavioural cues for detecting depression. Additionally, explanatory class-specific kinemes patterns can be observed, in alignment with prior research.  

% Figure environment removed

This paper makes the following research contributions:
%
\begin{itemize}
    \item A study of head movements as a biomarker for clinical depression, which so far has been understudied.
    \item Proposing the \textit{kineme} representation of motion patterns as an effective and explanatory means for depression analysis.
    \item \begin{sloppypar} A detailed investigation of various classifiers for 2-class and 4-class categorisation on the AVEC2013 and BlackDog datasets. We obtain peak F1-scores of 0.79 and 0.82, respectively, on \textit{thin-slice} chunks for binary classification on the BlackDog and AVEC2013 datasets, which compare favorably to prior approaches. Also, a video-level F1-score of 0.72 is achieved for 4-class categorisation on AVEC2013.  \end{sloppypar}
\end{itemize}
%
The remainder of this paper is organised as follows. Section \ref{Sec:RW} provides an overview of related work. Section \ref{Sec:KF} describes the kineme formulation, followed by Section \ref{Sec:EKF} that details the explainable kineme features used as a representation of motion patterns. The methodology is presented in Section \ref{Sec:Meth}, while Section \ref{Sec:ER} provides details of the datasets, experimental settings, and classifiers used in this study. The experimental results are shown and discussed in Section \ref{sec:ResultsDiscussion}. Finally, the conclusions are drawn in Section \ref{Sec:DC}.


% Add the basics of kinemes, the approach used
% Add overview of the framework implemented
% Contribution




 

%
%
%
%

%



%
%
%

%

%
%
%
%
%
%
%
%
%
%
%
%
%
%
%
%
%
%
%

%

%
%
%

%
%
%
%
%
%


%
%
%
%

%
%
%

%
%
%

%
\section{Technical Background}
\label{sec:Bkgd}

Our given data is a finite graph with an embedding function $f: G \to \R^2$. 
While more general mathematical assumptions are possible (such as immersions with an $o$-minimal image \cite{Dries1998}), for the computational goals of this paper, we will assume a straight line, plane embedding. 
%
We define a bounding box on the image $f(G)$, which for simplicity we will assume is of the form $[-B, B]^2$. 
%
We next outline the necessary categorical framework in order to formulate a precise notion of the interleaving distance.

%
\subsection{Functors and cosheaves}
\label{ssec:cosheaf}

We give basic definitions for the category theoretic notions required in this paper, but note that the introduction will be quite sparse. 
We direct the interested reader to \cite{Riehl2017,Curry2014} for further details. 

A category $\cC$ consists of a collection of objects $X,Y,Z,\cdots$ and morphisms $f,g,h,\cdots$ with the following data: 
morphisms $f:X \to Y$ have designated domain $X$ and codomain $Y$. 
Every object has a designated identity morphism $\1_X: X \to X$, and any pair of morphisms $f:X \to Y$ and $g:Y \to Z$ has a composite morphism $gf:X \to Y$. 
These objects and morphisms are required to satisfy an identity axiom, where $f:X \to Y$ is the same as the composites $\1_Yf$ and $f\1_X$, and composition of morphisms is associative, so $h(gf) = (hg)f$. 
Some examples used in this paper are $\Set$, where objects are sets and morphisms are set maps; 
$\Top$ where objects are topological spaces and morphisms are continuous functions;
and $\Open(K)$ for a given topological space $K$, where the objects are open sets and morphisms are given by inclusion. 
The latter example is a special case called a \emph{poset category}, where every pair of objects has at most one morphism between them. 

A functor $F:\cC \to \cD$ is a map between categories preserving the relevant structures.
Specifically, for every object $X \in \cC$ there is a an object $F(X) \in \cD$, and for every morphism $f:X \to Y$, there is a morphism $F[f]:F(X) \to F(Y)$. 
To be a functor, $F$ must further satisfy that for  any $X \in \cC$, $F[\1_X] = \1_{F(X)}$ and for any composable pair $f,g \in \cC$, we have $F[gf] = F[g] F[f]$. 
Given functors $F,G: \cC \to \cD$, a natural transformation $\eta: F \Rightarrow G$  consists of a map $\eta_X:F(X) \to G(X)$ for every $X \in \cC$ (called the components) so that for any morphism $f:X \to Y$ in $\cC$, the diagram 
\begin{equation*}
\begin{tikzcd}
X\ar[d,  "f"] 
& F(X) 
    \ar[r, "\eta_X"] 
    \ar[d, "{F[f]}"']
& G(X) 
    \ar[d, "{G[f]}"]
\\
Y 
& F(Y) \ar[r, "\eta_Y"] 
& G(Y)
\end{tikzcd}
\end{equation*}
commutes. 
One example we will use later is $\pi_0:\Top \to \Set$, where $\pi_0(\X)$ is the set of path connected components of $\X$, and morphisms are set maps $\pi_0[f]: \pi_0(\X) \to \pi_0(\Y)$ sending a connected component $A$ in $\X$ to the connected component of $f(A)$ in $\Y$.

A diagram is a functor $F:J \to \cC$ where $J$ is a small category. 
In essence, this construction picks out a collection of objects $F(j)$ and a collection of morphisms $F(j) \to F(k)$. 
A cocone on a given diagram is a natural transformation $\lambda:F \to c$ where we abuse notation to write that $c: J \to \cC$ is the constant functor returning the object $c(j) = c \in \cC$ for all $j \in J$. 
We often call the components $\lambda:F(j) \to c$ the \emph{legs}, and note that this requirement says that for any $f:j \to k$ in $J$, the triangle 
\begin{equation*}
\begin{tikzcd}
F(j) 
    \ar[rr, "{F[f]}"]
    \ar[dr, "\lambda_j"']
&& F(k)
    \ar[dl, "\lambda_k"]
\\
& c
\end{tikzcd}
\end{equation*}
commutes. 
A cocone $\lambda:F \to c$ is called a colimit if for any other cocone $\lambda':F \to c'$, there is a unique  morphism $u:c \to c'$ such that 
\begin{equation*}
\begin{tikzcd}
F(j) 
    \ar[rr, "{F[f]}"]
    \ar[dr, "\lambda_j"]
    \ar[ddr, "\lambda_j'"']
&& F(k)
    \ar[dl, "\lambda_k"']
    \ar[ddl, "\lambda_k'"]
\\
& c \ar[d, dashed, "u"] \\
& c'
\end{tikzcd}
\end{equation*}
commutes for all $f:j \to k$ in $J$. 

We will be particularly interested in functors of the form $F:\Open(X) \to \Set$, which can also be called \emph{pre-cosheaves}. 
A pre-cosheaf is a cosheaf if it satisfies a gluing axiom meaning $F(U)$ is entirely determined by $F(U_\alpha)$ for any cover $\{U_\alpha\}_\alpha$. 
Specifically, given an open set $U$ and a cover $\{ U_\alpha \mid \alpha \in A\}$ of $U$, define a category $\cU = \{U_\alpha \cap U_{\alpha'} \mid \alpha,\alpha' \in A \}$ with morphisms given by inclusion. 
Then we have a diagram $F:\cU \to \Set$, and as such can consider its colimit $\lambda:F \to L$.
If the unique map $L \to F(U)$ given by the colimit definition is an isomorphism, then $F$ is called a \emph{cosheaf}.
%
%
%
%
%
%
%
%
%
%
%
%
%
%
%
%



%
\subsection{Functorial Representation of Geometric Graphs}
\label{ssec:functorGraphs}
%
%



We start by defining a cubical complex on $[-B,B]^2$, where $[-B,B]^2$ is the bounding box for the graphs $X$ and $Y$.
Following \cite{Kaczynski2004}, we define a cubical complex given by diameter $\delta$.
For the sake of simplicity, assume that $B$ is a multiple of $\delta$, so that the bounding box can be written as $[-L\delta, L\delta]^2$. 
An \emph{elementary interval} is defined to be a closed interval in $\R$ of the form $[\ell\delta, (\ell+1)\delta]$ or $[\ell]:= [\ell\delta, \ell\delta]$ for $\ell \in [-L,\cdots, L] \subset \Z$.
These are called non-degenerate and degenerate intervals, respectively. 
An elementary cube $Q$ is a finite product of two elementary intervals, i.e.
%
   $ \sigma = I_1 \times I_2 \subset [-B,B]^2$. 
%
The dimension of a cube $\sigma$ is given by the number of intervals used which are non-degenerate. 
Note that this means 
$0$-cubes are vertices at grid locations $[i\delta, j \delta]$, 
$1$-cubes are products $[i\delta] \times [j\delta, (j+1)\delta]$ or $[i\delta, (i+1)\delta] \times [j\delta ]$, 
and $2$-cubes are products $[i\delta, (i+1)\delta] \times [j\delta, (j+1)\delta]$. 
The collection of elementary cubes $K$ is a finite cubical complex.
This construction comes with a face relation which gives a poset structure, where we write $\sigma \leq \tau$ iff $\sigma \subseteq \tau$. 


We now show how we endow the complex $K$ with the Alexandroff topology, following \cite{Barmak2011}. 
Given the poset $(K,\leq)$, for any set $S \subseteq X$, the up-set is the collection 
$S^{\uparrow} = \{x \mid x \geq y, \, y \in S \}$. 
Similarly, the down-set $S^{\downarrow}$ is the collection $\{ x \mid x \leq y, \, y \in S\}$. 
For any elementary cube $\sigma$, we have that 
\begin{equation*}
    U_\sigma := \{ \sigma\}^{\uparrow} = \{\tau  \mid \tau \geq \sigma \}
\end{equation*}
is the same as the star of the cube, to borrow terminology from the simplicial complex literature. 
We give $K$ the Alexandroff topology%
\footnote{Note that in the case of finite posets,  either down- or  up-sets can be used to define the topology; while in general the Alexandroff topology is defined using the down-sets as opens \cite{Barmak2011}.  
However, we are trying to avoid using the opposite poset as much as possible to alleviate notation woes, and the correspondence with stars in this setting is useful for our purposes.} 
$\Open(K)$, where a set $U\subseteq K$ is said to be open iff the following holds: 
for any $x \in U$, and any $y \geq x$, we have that $y \in U$. 
Equivalently, this means that $U$ is its own up-set, i.e.~$U = U^\uparrow$.
It can be checked that this topology has the collection  
$\{U_\sigma\}_{\sigma \in K}$ 
as a basis. 

%
%
%

The main objects of study here are embedded graphs. 
That is, we start with input data $f:\X \to \R^2$ where $\X$ is a finite topological graph and $f$ is a straight line plane embedding. 
%
%
Then we can encode this information in a functor $F:\Open(K) \to \Set$ given by 
\begin{equation*}
    \begin{matrix}
    F: &  \Open(K) & \to & \Set\\
    & U & \mapsto & \pi_0 f\inv(|U|).
    \end{matrix}
\end{equation*}
Note that functoriality of $\pi_0$ means that for $U \subseteq V$ there is an induced map
\[F[U \subseteq V] \colon \pi_0f\inv(|U|) \to \pi_0f\inv(|V|)\]
satisfying all requirements of a functor for $F$.
Indeed, this functor is actually a cosheaf. 
Throughout the paper, when the notation makes the sets involved obvious, we will simply write the induced map as $F[\subseteq]:F(U) \to F(V)$. 



%
\subsection{Thickenings}
\label{ssec:thickenings}
% Figure environment removed
Given any set $U \in \Open(K)$, the 1-thickening  is defined by taking the upset of the downset of $U$. This can be written as 
$   U^1 = (U^{\downarrow})^\uparrow$.
Thinking in parallel to simplicial complex settings, this operation can be thought of as taking the star of the closure of the set. 
See Fig.~\ref{fig:thickenings} for examples. 
We then define the $n$-thickening to be $n$ repetitions of the process given recursively as
\begin{equation*}
    U^n = 
    \begin{cases}
    U & n = 0\\
    %
    (U^{n-1})^{\downarrow\uparrow} & n \geq 1.
    \end{cases}
\end{equation*}
Note that each $U^n$ is itself an open set in $\Open(K)$, and that if $U \leq V$, then $U^n \subseteq V^n$. 
Thus we can view this operation as a functor on the category $\Open(K)$ with morphisms given by inclusion:
\begin{equation*}
    \begin{matrix}
    (-)^n: & \Open(K)  & \to &  \Open(K)\\
    & U & \mapsto & U^n.
    \end{matrix}
\end{equation*}

One property of this construction that will be useful is the following. 
For any $\sigma \in U^n$, there is a $\tau \in U$ and a sequence of cells
\begin{equation}
\label{eq:length_n_path}
    \tau 
    \geq \gamma_1 \leq \tau_1 
    \geq \gamma_2 \leq \tau_2 
    \geq  \cdots 
    \geq \gamma_n \leq \sigma.
\end{equation}
Further, given such a sequence with $\tau \in U$, we know that $\sigma \in U^n$. 
Two examples of this can be seen in Fig.~\ref{fig:Length_n_path}, where $\sigma$ and $\sigma'$ from $U^3$ are given, along with a path satisfying Eq.~\eqref{eq:length_n_path}.
Of course, the choice of sequence for Eq.~\eqref{eq:length_n_path} is not unique, so other options are possible. 
% Figure environment removed

\begin{lemma}
$(-)^n$ is a functor. 
\end{lemma}

\begin{proof}
%
First, we check that the images of morphisms are well defined, which is to say that if $U \subseteq V$, then $U^{n} \subseteq V^{n}$. 
The statement is clear if $n = 0$, so by induction, we assume that $U^{n-1} \subseteq V^{n-1}$. 
Given an arbitrary $\sigma \in U^{n}$, the statement is immediate if $\sigma \in U^{n-1} \subseteq U^{n}$, so we assume $\sigma \in U^n \setminus U^{n-1}$. 
For this to happen, there must be a $\gamma \in U^{n-1}$ and $\tau \in K$ with $\gamma \geq \tau \leq \sigma$. 
But as $\gamma \in U^{n-1} \subseteq V^{n-1}$, this sequence also implies that $\sigma \in V^n$ finishing the well-defined check. 

To ensure this is a functor, we need to check that the identity morphism is sent to the identity, and that composition holds. 
For the former, we see that $U \subseteq U$ gets sent to $U^n \subseteq U^n$, and each is an identity. 
The latter is immediate from the property that $\Open(K)$ is a poset category, meaning there is at most one morphism between any pair of objects.  
\end{proof}


We can use this construction to build an interleaving distance on functors of the form \linebreak
${F:\Open(K) \to \Set}$ using the superlinear family of translations framework of \cite{Bubenik2014a}.
Note that this construction can be generalized to the concept of a category with a flow \cite{deSilva2018}, but the added generality is not needed here. 

\begin{definition}[\cite{Bubenik2014a}]
Let $P = (P,\leq)$ be a preordered set. 
A \emph{translation} on $P$ is a functor $\Gamma: P \to P$ along with a natural transformation $\eta:\1_P \Rightarrow \Gamma$. 
A \emph{super-linear family of translations} is a collection $\{\Gamma_\e \}_{\e\geq 0}$  such that 
$\Gamma_\e \Gamma_{\e'}(p) \leq \Gamma_{\e + \e'}(p)$ for all $p \in P$, and $\e, \e' \geq 0$. 
%
\end{definition}

\begin{lemma}
\label{lem:composedthickenings}
For any $n, n' \geq 0$ and $U \in \Open(K)$, 
$(U^{n})^{n'} = U^{n+n'}$.
%
This is a stronger requirement than needed above, so the collection $\{ ( - )^n\}_{n \geq 0}$ forms a super-linear family of translations. 
%
\end{lemma}

\begin{proof}
First, we check that $(-)^n$ is indeed a translation using the above terminology. 
In particular, we define $\gamma^n:\1_{\Open(K)} \Rightarrow (-)^n$ to have components $\gamma^n_U: U \to U^n$ as simply the inclusion, and can easily check that this satisfies naturality requirements. 

Fix $U \in \Open(K)$. 
We need to show that $(U^n)^{n'} = U^{n+n'}$. 
Let $\sigma \in (U^n)^{n'}$. 
By previous remarks, this is true if and only if there is a sequence 
\begin{equation*}
    \tau 
    \geq \gamma_1 \leq \tau_1 
    \geq \gamma_2 \leq \tau_2 
    \geq  \cdots 
    \geq \gamma_{n'} \leq \sigma
\end{equation*}
with $\tau \in U^n$. 
But this property of $\tau$ happens iff there is also a sequence 
\begin{equation*}
    \tau' 
    \geq \gamma_1' \leq \tau_1 '
    \geq \gamma_2' \leq \tau_2' 
    \geq  \cdots 
    \geq \gamma_{n}' \leq \tau
\end{equation*}
with $\tau' \in U$. 
Concatenating the two sequences gives a sequence of length $(n+n')$ from $\tau \in U$ to $\sigma$. 
Thus $\sigma \in U^{n+n'}$ iff $\sigma \in (U^n)^{n'}$, and hence $(U^n)^{n'} = U^{n+n'}$. 
\end{proof}

Our next task is to use this structure to define an interleaving distance. 
%
%
%
%
The first necessary ingredient is the composition of functors $F \circ ( - )^n: \Open(K) \to \Set$, which we will denote it by $F^n$. 
This means $F^n(U) = F(U^n)$, and we have the similar setup for $G$. 
With this notation, the interleavings use natural transforms of the form $\phi:F \Rightarrow G^n$ and $\psi:G \Rightarrow F^n$. 
Note that a component of $\phi$ is a set-map $\phi_U:F(U) \to G(U^n)$. 
There is of course, another component at $U^n$, $\phi_{U^n}:F(U^n) \to G(U^{2n})$, which can also be viewed as a component of another natural transformation $\phi^n:F^n \Rightarrow G^{2n}$. 
For this reason, we use the notation $\phi_{U^n}$ and $\phi_U^n$ interchangeably when $\phi$ is indeed a natural transformation. 
Note that we are implicitly using Lem.~\ref{lem:composedthickenings} to write the maps this way. 

\begin{definition}
\label{def:interleavingDistance}
Given two cosheaves $F,G:\Open(K) \to \Set$ and $n \geq 0$, an \emph{$n$-interleaving} is given by a pair of natural transformations 
$\phi:F \Rightarrow G^n$
and 
$\psi:G \Rightarrow F^n$
such that the diagrams
%
%
%
\begin{equation*}
    \begin{tikzcd}
        F(U) 
            \ar[rr, "{F[U \subseteq U^{2n}]}"]   
            \ar[dr, "\phi_U"',violet]
            %
            & & F(U^{2n}) & 
        & F(U^n) \ar[dr]
            \ar[dr, "\phi_{U^{n}}",violet]
        & \\
        & G(U^n)\ar[ur, "\psi_{U^n}"', orange]  & & 
        G(U) 
            \ar[rr, "{G[U \subseteq U^{2n}]}"']
            \ar[ur, "\psi_{U}", orange] 
        %
        && G(U^{2n})
    \end{tikzcd}
\end{equation*}
%
%
%
%
%
%
%
%
%
%
%
%
%
%
%
%
%
%
%
commute for all $U \in \Open(K)$.
The interleaving distance is given by 
\begin{equation*}
    d(F,G) = \inf\{ n \geq 0 \mid \text{there exists an $n$-interleaving} \}
\end{equation*}
and is set to be $d(F,G) = \infty$ if there is no interleaving for any $n$.
\end{definition}
%

%
%
%
%
%
%
%
%
%
%
%
%
%
%
%
%
%
%
%
%

\begin{theorem}
    The interleaving distance of Defn.~\ref{def:interleavingDistance} is an extended pseudometric. 
\end{theorem}
\begin{proof}
This result is immediate from \cite[Theorem 3.21]{Bubenik2014a}.
    %
\end{proof}


%

%
%
%
%
%

%
%
%
%

%
%

%
%
%
%
%
%
%
%
%
%
%
%
%
%
%
%
%
%

%
%
%
%
%
%
%
%
%
%
%
%
%
%
%
%
%
%
%
%
%



%

%
    
%


%
%
%
%

%
%

%
%

%
\input{Sections/LossFunction}

%
%
%

%
\appendices
\section{The Proof of Proposition \ref{prop2}}
\label{appa}
For the jointly Gaussian random vectors $\bm{x}$ and $\bm{y}$, we have
\begin{equation}
\begin{aligned}
&    \left[\begin{matrix}\bm{x}\\\bm{y}\\\end{matrix}\right] \sim \mathcal{N}\left(\left[\begin{matrix}\bm{\mu}_x\\\bm{\mu}_y\\\end{matrix}\right],\left[\begin{matrix}A&C\\C^T&B\\\end{matrix}\right]\right) \\
& = \mathcal{N}\left(\left[\begin{matrix}\bm{\mu}_x\\\bm{\mu}_y\\\end{matrix}\right],\left[\begin{matrix}\widetilde{A}&\widetilde{C}\\{\widetilde{C}}^T&B\\\end{matrix}\right]^{-1}\right)
\end{aligned}
\end{equation}
then the marginal and conditional distribution of $\bm{x}$ are shown as follows according to \cite{williams2006gaussian}.
\begin{equation}
    \bm{x} \sim \mathcal{N}\left(\bm{\mu}_x,A\right)
\end{equation}
% and
\begin{equation}
\label{app2-1}
    \bm{x}|\bm{y} \sim \mathcal{N}\left(\bm{\mu}_x+CB^{-1}\left(\bm{y}-\bm{\mu}_y\right),A-CB^{-1}C^T\right)
\end{equation}
% or
\begin{equation}
\label{app2-2}
    \bm{x}|\bm{y} \sim \mathcal{N}\left(\bm{\mu}_x-{\widetilde{A}}^{-1}\widetilde{C}\left(\bm{y}-\bm{\mu}_y\right),{\widetilde{A}}^{-1}\right)
\end{equation}

Thus, \textbf{Proposition \ref{prop2}} is proved.










\section{The Proof of Proposition \ref{prop3}}
\label{appb}
The product of two Gaussian distributions is represented as
\begin{equation}
\mathcal{N}\left(\bm{x}\middle|\bm{a},A\right)\mathcal{N}\left(\bm{x}\middle|\bm{b},B\right)=Z^{-1}\mathcal{N}\left(\bm{x}\middle|\bm{c},C\right)
\end{equation}
where
\begin{equation}
\label{app4}
    \bm{c}=C\left(A^{-1}\bm{a}+B^{-1}\bm{b}\right)
\end{equation}
\begin{equation}
\label{app5}
    C=\left(A^{-1}+B^{-1}\right)^{-1}
\end{equation}
\begin{equation}
\label{app6}
    Z^{-1}=\left(2\pi\right)^{-\frac{D}{2}}\left|A+B\right|^{-\frac{1}{2}}\exp{\left(-\frac{\left(\bm{a}-\bm{b}\right)^T\left(\bm{a}-\bm{b}\right)}{2\left(A+B\right)}\right)}
\end{equation}

Thus, through multiplying the cavity distribution by $t_i$ from (\ref{11}), \textbf{Proposition \ref{prop3}} is proved.


\section{The Proof of Proposition \ref{prop4}}
\label{appc}
Consider
\begin{equation}
\label{app7}
Z=\int_{-\infty}^{\infty}{\Phi\left(\frac{x-m}{v}\right)\mathcal{N}(x|\mu,\sigma^2)dx}
\end{equation}
% where
% \begin{equation}
%     \Phi\left(x\right)=\int_{-\infty}^{x}{\mathcal{N}\left(y\right)dy}
% \end{equation}
When $v>0$, by combining$ z=y-x+\mu-m$ and $w=x-\mu$ we can get
\begin{equation}
\begin{aligned}
& Z_{v>0}=\frac{\int_{-\infty}^{\infty}\int_{-\infty}^{x}\exp{\left(-\frac{\left(y-m\right)^2}{2v^2}-\frac{\left(x-\mu\right)^2}{2\sigma^2}\right)}}{2\pi\sigma v}dydx \\
& =\frac{\int_{-\infty}^{\mu-m}\int_{-\infty}^{\infty}\exp{\left(-\frac{\left(z+w\right)^2}{2v^2}-\frac{w^2}{2\sigma^2}\right)}}{2\pi\sigma v}dwdz
\end{aligned}
\end{equation}
% and
\begin{equation}
\begin{aligned}
& Z_{v>0} \\
& =\frac{\int_{-\infty}^{\mu-m}\int_{-\infty}^{\infty}\exp{\left(-\frac{1}{2}\left[\begin{matrix}w\\z\\\end{matrix}\right]^T\left[\begin{matrix}\frac{1}{v^2}+\frac{1}{\sigma^2}&\frac{1}{v^2}\\\frac{1}{v^2}&\frac{1}{v^2}\\\end{matrix}\right]\left[\begin{matrix}w\\z\\\end{matrix}\right]\right)}}{2\pi\sigma v}dwdz \\
& =\int_{-\infty}^{\mu-m}\int_{-\infty}^{\infty}\mathcal{N}\left(\left[\begin{matrix}w\\z\\\end{matrix}\right]|\mathbf{0},\left[\begin{matrix}\sigma^2&-\sigma^2\\-\sigma^2&v^2+\sigma^2\\\end{matrix}\right]\right)dwdz
\end{aligned}
\end{equation}
According to (\ref{app2-1}) and (\ref{app2-2}), we can get
\begin{equation}
\label{app11}
    Z_{v>0}=\frac{\int_{-\infty}^{\mu-m}\exp{\left(-\frac{z^2}{2\left(v^2+\sigma^2\right)}\right)}dz}{\sqrt{2\pi(v^2+\sigma^2)}}=\Phi\left(\frac{\mu-m}{\sqrt{v^2+\sigma^2}}\right)
\end{equation}
When $v<0$, by combining $\Phi\left(-z\right)=1-\Phi\left(z\right)$ and (\ref{app7}),
% we can obtain
\begin{equation}
\label{app12}
Z_{v<0}=1-\Phi\left(\frac{\mu-m}{\sqrt{v^2+\sigma^2}}\right)=\Phi\left(-\frac{\mu-m}{\sqrt{v^2+\sigma^2}}\right)
\end{equation}

By collecting (\ref{app11}) and (\ref{app12}), we can get
\begin{equation}
\label{app13}
Z=\int\Phi\left(\frac{x-m}{v}\right)\mathcal{N}\left(x\middle|\mu,\sigma^2\right)dx=\Phi\left(z\right)
\end{equation}
where $z=\frac{\mu-m}{v\sqrt{1+\sigma^2/v^2}} (v\neq0)$. 
% We aim to get the moments of
% \begin{equation}
% q\left(x\right)=Z^{-1}\Phi\left(\frac{x-m}{v}\right)\mathcal{N}\left(x\middle|\mu,\sigma^2\right)
% \end{equation}
By differentiating with respect to $\mu$ on (\ref{app13}), we can obtain
\begin{equation}
\begin{aligned}
& \frac{\partial Z}{\partial\mu}=\int{\frac{x-\mu}{\sigma^2}\Phi\left(\frac{x-m}{v}\right)}\mathcal{N}\left(x\middle|\mu,\sigma^2\right)dx =\frac{\partial}{\partial\mu}\Phi\left(z\right) \\
& \Longleftrightarrow \frac{1}{\sigma^2}\int x\Phi\left(\frac{x-m}{v}\right)\mathcal{N}\left(x\middle|\mu,\sigma^2\right)dx-\frac{\mu Z}{\sigma^2} \\
& =\frac{\mathcal{N}(z)}{v\sqrt{1+\sigma^2/v^2}}
\end{aligned}
\end{equation}
where $\partial\Phi\left(z\right)/\partial\mu=\mathcal{N}(z)\partial z/\partial\mu$ is utilized. Multiplying through by $\sigma^2/Z$, (\ref{app16}) is obtained.
\begin{equation}
\label{app16}
\mathbb{E}_q\left[x\right]=\mu+\frac{\sigma^2\mathcal{N}\left(z\right)}{\Phi\left(z\right)v\sqrt{1+\frac{\sigma^2}{v^2}}}
\end{equation}
Similarly, we can obtain the second moment as
\begin{equation}
\label{app17}
\begin{aligned}
 & \frac{\partial^2Z}{\partial\mu^2} \\
 & =\int{[\frac{x^2}{\sigma^4}-\frac{2\mu x}{\sigma^4}+\frac{\mu^2}{\sigma^4}-\frac{1}{\sigma^2}] \Phi\left(\frac{x-m}{v}\right)\mathcal{N}\left(x\middle|\mu,\sigma^2\right)} dx  \\
 & =-\frac{z\mathcal{N}(z)}{v^2+\sigma^2} \Longleftrightarrow \\
 & \mathbb{E}_q\left[x^2\right]=2\mu\mathbb{E}_q\left[x\right]-\mu^2+\sigma^2-\frac{\sigma^4z\mathcal{N}\left(z\right)}{\Phi\left(z\right)\left(v^2+\sigma^2\right)}
\end{aligned}
\end{equation}
By combining (\ref{app16}) and (\ref{app17}), we can get
\begin{equation}
\begin{aligned}
& \mathbb{E}_q\left[{(x-\mathbb{E}_q\left[x\right])}^2\right]=\mathbb{E}_q\left[x^2\right]-\mathbb{E}_q[x]^2 \\
& =\sigma^2-\frac{\sigma^4\mathcal{N}\left(z\right)}{\left(v^2+\sigma^2\right)\Phi\left(z\right)}\left(z+\frac{\mathcal{N}\left(z\right)}{\Phi\left(z\right)}\right)
\end{aligned}
\end{equation}

Thus, \textbf{Proposition \ref{prop4}} is proved.

\section{The Proof of Proposition \ref{prop5}}
\label{appd}
We can obtain (\ref{19-1}), (\ref{19-2}), and (\ref{19-3}) according to (\ref{app4}), (\ref{app5}), and (\ref{app6}). Hence, \textbf{Proposition \ref{prop5}} is proved.



\section{The Proof of Proposition \ref{prop6}}
\label{appe}
The approximated mean for $f_\ast$ can be denoted as
\begin{equation}
\begin{aligned}
& \mathbb{E}_q\left[f_\ast|X,\bm{y},\bm{x}_\ast\right]=\bm{k}_\ast^TK^{-1}\bm{\mu} \\
& =\bm{k}_\ast^TK^{-1}\left(K^{-1}+{\widetilde{\Sigma}}^{-1}\right)^{-1}{\widetilde{\Sigma}}^{-1}\widetilde{\bm{\mu}} \\
& =\bm{k}_\ast^T\left(K+\widetilde{\Sigma}\right)^{-1}\widetilde{\bm{\mu}}
\end{aligned}
\end{equation}

The variance of $f_\ast|(X,\bm{y})$ under the Gaussian approximation can be denoted as
\begin{equation}
\begin{aligned}
& \mathbb{V}_q\left[f_\ast\middle| X,\bm{y},\bm{x}_\ast\right] = \mathbb{E}_{p(f_\ast|X,\bm{x}_\ast,\bm{f})} {f_\ast-\mathbb{E}[f_\ast|X,\bm{x}_\ast,\bm{f}]}^2 \\
& =k\left(\bm{x}_\ast,\bm{x}_\ast\right)-\bm{k}_\ast^TK^{-1}\bm{k}_\ast+\bm{k}_\ast^TK^{-1}\left(K^{-1}+\widetilde{\Sigma}\right)^{-1}K^{-1}\bm{k}_\ast \\
& =k\left(\bm{x}_\ast,\bm{x}_\ast\right)-\bm{k}_\ast^T\left(K^{-1}+\widetilde{\Sigma}\right)^{-1}\bm{k}_\ast
\end{aligned}
\end{equation}

Then, we can obtain
\begin{equation}
\begin{aligned}
& q\left(y_\ast\middle| X,\bm{y},\bm{x}_\ast\right)=\mathbb{E}_q\left[\pi_\ast|X,\bm{y},\bm{x}_\ast\right] \\
& =\int\Phi\left(f_\ast\right)q\left(f_\ast\middle| X,\bm{y},\bm{x}_\ast\right)df_\ast
\end{aligned}
\end{equation}

According to (\ref{app11}), we can obtain
\begin{equation}
\label{app22}
\begin{aligned}
& q\left(y_\ast\middle| X,\bm{y},\bm{x}_\ast\right) \\
& =\Phi\left(\frac{\bm{k}_\ast^T\left(K+\widetilde{\Sigma}\right)^{-1}\widetilde{\bm{\mu}}}{\sqrt{1+k\left(\bm{x}_\ast,\bm{x}_\ast\right)-\bm{k}_\ast^T\left(K+\widetilde{\Sigma}\right)^{-1}\bm{k}_\ast}}\right)
\end{aligned}
\end{equation}

By combining (\ref{13}) and (\ref{app22}), \textbf{Proposition \ref{prop6}} is proved.




\section{The Proof of Proposition \ref{prop7}}
\label{appf}
Given $f_s$ and $f_\ast$, $y_s$ and $y_\ast$ are conditionally independent. Hence, $p\left(y_s,y_\ast\middle|\bm{x}_s,\bm{x}_\ast\right)$ can be represented as
\begin{equation}
\begin{aligned}
& p\left(y_s=1,y_\ast=1\middle|\bm{x}_s,\bm{x}_\ast\right) \\
& =\iint{\Phi\left(f_s\right)\Phi\left(f_\ast\right)\phi\left(f_s,f_\ast\middle|\mu_{s\ast},\Sigma_{s\ast}\right)}df_sdf_\ast \\
& =\iint{\Phi\left(f_\ast\right)\phi\left(f_\ast\middle|{\widetilde{\mu}}_\ast\left(f_s\right),{\widetilde{\sigma}}_{\ast\ast}\right)df_\ast\Phi\left(f_s\right)}\phi\left(f_s\middle|\mu_s,\sigma_{ss}\right)df_s \\
& =\int\Phi\left(\frac{{\widetilde{\mu}}_\ast\left(f_s\right)}{\sqrt{{\widetilde{\sigma}}_{\ast\ast}+1}}\right)\Phi\left(f_s\right)\phi\left(f_s\middle|\mu_s,\sigma_{ss}\right)df_s
\end{aligned}
\end{equation}

Hence, \textbf{Proposition \ref{prop7}} is proved.

% \section{The Proof of Lemma \ref{lem}}
% \label{appg}
% \begin{equation}
% \begin{aligned}
% & R_e=\frac{1}{N_a}\sum_{n=1}^{N_a}\mathbb{I}\left(\bm{L}_n \neq \bm{Y}_n\right) \\
% & =\displaystyle\frac{FA+FL}{TL+TA+FL+FA} \\
% & =\displaystyle\frac{1}{\displaystyle\frac{TL+TA+FL+FA}{FA+FL}} \\
% & =\displaystyle\frac{1}{1+\displaystyle\frac{TL+TA}{FA+FL}} \\
% & =\displaystyle\frac{1}{1+\displaystyle\frac{\displaystyle\frac{TL}{TA}+1}{\displaystyle\frac{FA}{TA}+\displaystyle\frac{FL}{TA}}} \\
% & =\frac{1}{1+\displaystyle\frac{\displaystyle\frac{TL}{TA}+1}{\displaystyle\frac{1}{P_{md}-1}+\displaystyle\frac{1}{P_{fa}-1}}}
% \end{aligned}
% \end{equation}

% Hence, \textbf{Lemma \ref{lem}} is proved.

%
\printbibliography
%
%
%

\end{document}