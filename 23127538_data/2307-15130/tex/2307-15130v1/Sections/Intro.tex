


\section{Introduction}
%
%
%
%
%

A geometric graph is generally defined as an abstract graph along with a well behaved embedding of the graph into the Euclidean plane.
Such graphs are a fundamental object used to model a wide range of data sets, ranging from maps and trajectories to commodity networks (i.e. electrical grids) to skeletons for shape recognition.  This geometric data is necessarily quite noisy, so the ability to compare, cluster, and simplify such objects is essential in a data analysis pipeline, leading to a need for theoretically motivated and computable distances. 
Of course, comparing graphs in such a setting is closely connected to the graph isomorphism problem, albeit on graphs which are enriched with geometric labels on each vertex and edge.  While graph isomorpism is polynomial time on planar graphs~\cite{Hopcroft1974}, the goal of this work is a similarity score determining whether the underlying graphs are isomorphic or not, and hence such a binary classification of similarity based on isomorphism is useless for our purposes.  


There has been extensive work on metrics for general graphs; see for example~ \cite{Deza2009,Conci2017,Donnat2018,Wills2020} for surveys on this topic.  
When restricting to geometric graphs, there has been both practical and theoretical work; see~\cite{Buchin2023} for a recent survey.  
For example, from the theoretical end, the well-studied Fr\'echet distance can be extended to graphs~\cite{bos-cfdrvs-17,buchin2017distance,Buchin2020,Fang2021}, although the practicality of this distance is unclear on real world data sets. 
On the practical side, other recent work has attempted to apply such measures to either geometric graphs or trajectories, motivated by the need for such distances on GIS data~\cite{Ahmed2014,Toohey2015,Werner2018,Tang2021}.


In this paper, we will draw inspiration from tools in topological data analysis, which primarily uses techniques from topology like homology and category theory. 
More specifically, we develop a natural extension of the interleaving distance on Reeb graphs and other topological spaces~\cite{deSilva2018,botnan2020} to geometric graphs.   
A Reeb graph \cite{Reeb1946}, for our purposes, is a pair consisting of a finite 1-dimensional stratified space with a well-controlled map to $\R$, $f:X \to \R$. 
This can be approximated by the related mapper graph \cite{Singh2007,Carriere2017,Carriere2018,Munch2016,Brown2020}, where structure is encoded using the relationship between connected components in the inverse image of a fixed cover of $\R$. 
In parallel, this paper focuses on geometric graphs, which we view as a pair consisting of a finite graph (i.e.~a 1-dimensional stratified space) with a straight-line planar map to $\R^2$. 
Like the analogy with mapper graphs, we will work with a computational approximation, where we are interested in the connected components over a particular choice of cover of the plane, in our case, a grid. 


Interleaving distances use functor representations of data, which can then be compared by finding pairs of natural transformations between them;
this powerful framework can be applied to a wide range of input data.
The study of interleavings arose in the context of generalizing the bottleneck distance for persistence modules \cite{Chazal2009}, represented as functors $(\R,\leq) \to \Vect$, where its equivalence to the bottleneck distance means the interleaving distance is polynomial time to compute~\cite{Lesnick2015}.  
Bringing the concept of interleavings to category theory \cite{Bubenik2014a, deSilva2018} opened up a wide range of available categories of objects that could be compared using the framework, but due to computational issues, the existing work has been largely restricted to theoretical results. 
Staying with more algebraic objects,  the interleaving distance between two multi-graded persistence modules is NP-hard to compute \cite{Bjerkevik2018,Bjerkevik2019}. 
On more graph based representations \cite{Bollen2021}, for Reeb graphs the problem is graph isomorphism-hard~\cite{deSilva2018};  and for merge trees, it is fixed parameter tractable \cite{FarahbakhshTouli2019}.
%
%

% Figure environment removed

In this work, we initiate the study of interleavings on geometric graphs, such as the example in Fig.~\ref{fig:geomgraphs}, with an eye towards determining practical algorithms to bound and approximate theoretically motivated metrics in this space.  
The idea of the interleaving distance, in this context, is to represent the input geometric graph data $f:X \to \R^2$ as a cosheaf of the form $F:\Open(X) \to \Set$ where we store the connected components of inverse images of open sets $\pi_0(f\inv(U))$. 
Then, we compare two such cosheaves using a pair of natural transformations $\phi:F \Rightarrow G^n$ and $\psi:G \Rightarrow F^n$ mapping into an $n$-relaxed version of the original inputs. 
This idea was briefly mentioned as an example in prior work~\cite{botnan2020}, however, to the best of our knowledge, interleavings on geometric graphs embedded in $\R^2$ have never been fully studied. 
In particular,  the computational complexity of the construction has meant a lack of the use of the interleaving distance in practice.

To circumvent these issues, we take particular inspiration from recent work of Robinson \cite{Robinson2020} to find quality measures for families of maps that do not rise to the level of a natural transformation, and apply these quality measures on geometric graphs.
In \cite{Robinson2020}, the object of study is a single input assignment of data of the form $P: \Open(X) \to \Set$ and, with the added structure of a pseudometric for each set $P(U)$, provides a measurement for how far the input data is from having a global section. 
In our work, we modify this idea to work with collections of maps 
${\phi = \{\phi_U: F(U) \to G(U^n)\mid U\}}$ and $\psi = \{\psi_U:G(U) \to F(U^n)\mid U\}$ which we call an \textit{assignment} when they do not necessarily form a true interleaving.  
In the case of embedded graphs, we restrict our class of open sets by imposing a grid structure on the plane.
We then use this grid to construct a poset endowed with the Alexandroff topology $K$,  and encode the original embedded graph data as functors $F: \Open(K) \to \Set$ where $F(U)$ is the set of connected components of the graph inside of the geometric realization of the set $U$. 
Using the structure of the embedded graphs, we can endow the image with the extra structure of a metric space, so that we have pairs $(F(U),d_U)$ for every open set $U$. 
Using this metric structure, we define a loss function $L(\phi,\psi)$ which measures how far the required diagrams of an interleaving are from commuting.  We can then further improve this bound by only focusing on the loss function computed for a basis of the topology, $L_B(\phi,\psi)$. 


%
Our main result is the following theorem, where for a fixed input assignment, we can use the basis function to bound the interleaving distance. 
\begin{restatable*}{theorem}{SecondLossBound}
Given basis $n$-assignments 
$\phi = \{\phi_{U_\sigma} \mid \sigma \in K\}$ 
and 
$\psi = \{\psi_{U_\sigma} \mid \sigma \in K\}$, 
\begin{equation*}
    d_I(F,G) \leq n + L_B(\phi,\psi).
\end{equation*}
\end{restatable*}
\noindent This opens up the potential for algorithmic approximation of the interleaving distance. 
First, the proof follows by explicitly computing a true interleaving to determine the upper bound.
Second, in future work, we will use this framework to iteratively optimize the choice of assignments to minimize the upper bound on interleaving distance.

\paragraph{Outline}
In Section \ref{sec:Bkgd} we provide the necessary technical background to set up the interleaving distance for  geometric graph inputs. 
In Section \ref{sec:loss-function}, we define the loss function (Sec.~\ref{ssec:LossFunction}) and use it to prove a bound which is simpler to define in the mathematical setting. 
Then we restrict our focus to only input information for basis elements, and show in Sec.~\ref{ssec:BasisBound} that this not only improves the computational complexity but also improves the bound. 
Finally, we include technical proofs in Section \ref{sec:technicalProofs}.

%
%
%
%
%
%
%
%
%
%
%
%
%
%
%