\section{Computation}
\label{sec:computation}

In this section, we show that given an $n$-assignment $\phi$, $\psi$, we can compute the loss function $L_B(\phi,\psi)$ in polynomial time. For simplicity, we describe the algorithm explicitly in the case where $d=1$ for clarity of examples, before addressing the run time in higher dimensions. 

\subsection{Data structures}
\label{ssec:DataStructures}
In this section, we describe the encoding of the data structures for a pair of input functors $F,G$ and a given $n$-assignment $\phi$ and $\psi$.
We will start with the case $d=1$, and follow the example of Fig.~\ref{fig:DataStructureExample} to illustrate our construction. 
At a high level, we construct graphs for $F$ and $G$, which we denote by $(V_F,E_F)$ and $(V_G, E_G)$.
Then we build data structures to encode the natural transformations $\phi$ and $\psi$.
For clarity, we use $\phitt$ and $\psitt$ to denote the data structures that store information for $\phi$ and $\psi$.
These encode set maps $\phitt:(V_G,E_G) \to (V_H,E_H)$ and $\psitt:(V_H,E_H) \to (V_G,E_G)$, which will map each vertex to a vertex in the other graph and each edge to an edge in the other graph. 



When $d=1$, recall that the discretization of $\R$, $K$, consists of vertices  $\sigma_{-L},\cdots,\sigma_L$ with heights in our bounding box $[-L\delta,L\delta]$, and with edges $\tau_j = (\sigma_j, \sigma_j+1)$. 
Then we construct the graph for $F:\Open(K) \to \Set$ by generating a vertex for every object in every $F(\sigma_i)$ and connect them using the morphisms of the functor.
This results in a vertex set 
$V_F = \coprod_{i =1}^B F(U_{\sigma_i})$,
and an edge for every object in every $F(U_{\tau_i})$, giving edge set 
$E_F = \coprod_{i =1}^{B-1} F(U_{\tau_i})$. 
Note that the endpoints of any edge $e \in E_F$ can be found via the attaching maps: $F[U_{\tau_i} \subseteq U_{\tau_{i-1}}](e)$ and $F[U_{\tau_i} \subseteq U_{\tau_{i+1}}](e)$. 
For example,  $e = (v_4,v_6) \in F(U_{\tau_4})$ in Fig.~\ref{fig:DataStructureExample} has endpoints $v_6 \in F(U_{\sigma_4})$ and $v_4 \in F(U_{\sigma_5})$. 
We store this data in a standard adjacency list.
In addition, each vertex also keeps track of its height, so a vertex $v \in F(U_{\sigma_i})$ will also store the value $i$ as a representation of its height.


% Figure environment removed

Next, we encode the information for an assignment $(\phi,\psi)$ between $F,G: \Open(K) \to \Set$ by constructing the maps $\phitt$ and $\psitt$ using the graphs $(V_F,E_F)$ and $(V_G,E_G)$.
Specifically, for every $v \in V_F$, we  store a vertex $\phitt(v) \in V_G$ with the requirement that if $v \in F(U_{\sigma_i}$ and $\phitt(v) \in G(U_{\sigma_j})$, then $|i-j| \leq n$.
In addition, for every $e \in E_F$, we store an edge $\phitt(e) \in E_G$ again with the requirement that if $e \in F(U_{\tau_i})$  and  $\phitt(e) \in G(U_{\tau_j})$ then $|i-j|\leq n$.
The symmetric situation is setup for $\psitt$. 

To see how these maps arise from input $\phi$ and $\psi$, we start by focusing on the vertex set. 
For this, we need to encode the map
$\phi_{U_{\sigma_i}}:F(U_{\sigma_i}) \to G(U_{\sigma_i}^n)$. 
The elements of $F(U_{\sigma_i})$ are already encoded as vertices, however the elements of $G(U_{\sigma_i}^n)$ are not.
But, because of the cosheaf structure of $G$, the elements of $G(U_{\sigma_i}^n)$ can be seen as the connected components of particular subgraphs. 
Let 
$V_{G,\sigma_i,n} = \{ v \mid v \in G(\sigma_j), j \in [i-n,i+n]\}$ 
and 
$E_{G,\sigma_i,n} = \{e \mid e \in G(\tau_j), j \in [i-n-1,i+n] \}$.  
Then by the properties of colimits, the elements of $G(U_{\sigma_i}^n)$ are the connected components of the subset of the graph $(V_G,E_G)_{\sigma_i,n}:=(V_{G,\sigma_i,n}, E_{G,\sigma_i,n})$. 
Note that because of the endpoints, this is not an induced subgraph; see Fig.~\ref{fig:Assignment} for examples.
Similarly for the edges of $K$, we can define 
$V_{G,\tau_i,n} = \{ v \mid v \in G(\sigma_j), j \in [i-n+1,i+n]$ 
and 
$E_{G,\tau_i,n} = \{e \mid e \in G(\tau_j), j \in [i-n,i+n] \}$ 
so that the connected components of $(V_G,E_G)_{\tau_i,n}:=(V_{G,\tau_i,n}, E_{G,\tau_i,n})$ are the elements of $G(U_{\tau_i}^n)$.

%
%
%
%
%
%
%
%


So, for each $v \in F(U_{\sigma_i})$, we store a vertex $\phitt(v) \in V_{G,\sigma_i,n}$, where $\phitt(v)$ is in the connected component of $(V_{G}, E_G)_{\sigma_i,n}$ represented by $\phi_{U_{\sigma_i}}(v) \in F(U_{\sigma_i}^n)$.
For instance, consider the example of Fig.~\ref{fig:Assignment} where we assume $n=1$. 
If  $\phitt(b) = w$, then $\phi_{U_{\sigma_i}}(b)$ is the connected component that includes $w$ of $(V_G,E_G)_{\sigma_i,1}$ as shown at the right. 
%
We can similarly find the edge map $\phitt(e)$ for $e \in F(U_{\tau_i})$ by setting it to be an edge in $E_{G,\tau_i,n}$ representing the connected component of $\phi_{U_{\tau_i}}(e) \in G(U_{\tau_i}^n)$ in $(V_G,E_G)_{\tau_i,n}$.
So, for example, in Fig.~\ref{fig:Assignment} where $n=1$, the input data might have $\phitt(ab) = (xy) \in E_G$ and $\phitt(bc) = (uv) \in E_G$. 


% Figure environment removed




\subsection{Algorithm and Complexity}
\label{ssec:Complexity}
In this section, we determine the complexity of determining $L_B(\phi,\psi)$ given $\phitt$ and $\psitt$. 
First, we will proceed using a binary search on $k \in [0,\cdots, 2L]$ where the maximum is determined by the diameter of the bounding box. 
So, for a fixed $k$, we will determine if $L_B(\phi,\psi) \leq k$. 
We will focus on 
$\Lpl^{U_\tau, U_\sigma}$ and $\Ltd^{U_\sigma}$ as 
$\Lpr^{U_\tau, U_\sigma}$ and $\Ltu^{U_\sigma}$ are symmetric.

Start with $\Lpl^{U_\tau, U_\sigma}$ and note that in the case where $d=1$, there are two pairs necessary to check for each edge: $\tau_j, \sigma_j$ and $\tau_j,\sigma_{j+1}$.
Fixing $\sigma_\ell$ to be either $\sigma_j$ or $\sigma_{j+1}$, for each edge $e \in F(U_{\tau_j})$, we need to check if the two possible images in $G(U_{\sigma_\ell}^{n+k})$ under the diagram
\begin{equation}
\label{eqn:dgm:parallel_extend_basis}
\begin{tikzcd}
        F(U_{\tau_i})  
            \ar[r, "{F[\subseteq ]}"] 
            \ar[dr, "\phi_{U_{\tau_i}}"',  violet]
        & F(U_{\sigma_\ell})
            \ar[dr, "\phi_{U_{\tau_i}^n}",  violet]
        & & e \ar[r,mapsto] \ar[dr, mapsto]
        & v \ar[dr, mapsto] \\
        & G(U_{\tau_i}^n) 
            \ar[r, "{G[\subseteq ]}"'] 
        & G (U_{\sigma_\ell}^n) \ar[r] 
        & G(U_{\sigma_\ell}^{n+k})
        & \substack{\\{[e']}} \ar[r, mapsto, shift right] 
        & \substack{[w]\\{[e']}} \ar[r, mapsto, shift left] \ar[r, mapsto, shift right] 
        & \substack{[w]\\{[e']}}
\end{tikzcd}
\end{equation}
are the same. 
Note that we use $[-]$ to note that the elements represent the connected component in the relevant sliced graph containing that edge or vertex. 
Following the top, we know that $e$ has a unique endpoint vertex  $v \in F(U_\sigma)$, and that vertex has an image under $\phi_{U_{\tau_i}^n}$ which is a connected component represented by  $\phitt(v) = w \in V_G$. 
Following down, the edge $e$ has an edge image $\phitt(e) = e' \in E_G$. 
So the question becomes: are $e'$ and $w$ in the same connected component of $(V_G,E_G)_{\sigma_\ell,n+k}$? 
This can be done by filtering through the adjacency lists, keeping only vertices and edges in the correct strip, and then checking for connectivity using a standard graph traversal like breadth or depth first search; as we do this once per grid element, we get a total time (when $d=1$) of $O(V_G+E_G)$ time per parallelogram. 
If $d>1$, then the correct ``strip'' for $\sigma_{\overrightarrow{\ell}}$ with indices $\overrightarrow\ell \in \Z^d$ involves checking  
a portion of the graph with indices in a $d$-dimensional box 
$[\ell_1-(n+k), \ell_1+(n+k) ] \times \cdots \times [\ell_d-(n+k), \ell_d+(n+k) ]$
 and hence takes $O(d(V_G + E_G))$ time.

In the example of Fig.~\ref{fig:Assignment}, assume $n=k=1$ and assume our given input $\phitt$ is as noted. 
%
%
Then  for the diagram of \cref{eqn:dgm:parallel_extend_basis} with $\ell = j$ and chasing $bc \in F(U_{\tau_j})$,
this comes down to checking if the connected component of $\phitt(b) = w$ and $\phitt(bc) = xy$ are the same in the portion of $(V_G,E_G)_{\sigma_j,2}$.
In this particular example, there are two connected components in this slice and the images are not in the same component. 
Then we know that $\Lpl^{U_{\tau_j}, U_{\sigma_j}}>k$ so we would immediately move on in our binary search. 
If it were the case that the two images were in the same connected component, then $\Lpl^{U_{\tau_j}, U_{\sigma_j}}\leq k$ and thus we would move on to the next commutative diagram check.


Checking if $\Ltd^{U_\tau} \leq k$ is similar so we briefly highlight the differences.  
First there are two types of basis elements in our case where $d=1$, so we need to check  $\Ltd^{U_{\sigma_i}} \leq k$ (meaning checking vertices) and $\Ltd^{U_{\tau_i}} \leq k$ (meaning checking edges). 
We focus on the case of vertices since the edge version is similar. 
For any vertex element $v \in U_{\sigma_i}$, we need to chase it around the diagram
\begin{equation}
\label{eqn:dgm:tri_extend_basis}
    \begin{tikzcd}
        F(U_{\sigma_i}) 
            \ar[rr, "{F[U_{\sigma_i} \subseteq U_{\sigma_i}^{2n}]}"]   
            \ar[dr, "\phi_{U_{\sigma_i}}"',violet]
            & & F(U_{\sigma_i}^{2n})  \ar[r] 
            & F(U_{\sigma_i}^{2(n+k)}).
        \\
        & G(U_{\sigma_i}^n)
            \ar[ur, "\psi_{U_{\sigma_i}^n}"', orange]  
        & \substack{v\\ \phantom{x}} 
            \ar[rr, mapsto, shift left] \ar[dr, mapsto]
        & & 
        \substack{{[v]}\\{[v']}}
            \ar[r, mapsto, shift left]
            \ar[r, mapsto, shift right]
        & \substack{{[v]}\\{[v']}}\\
        & & & w \ar[ur, mapsto]
    \end{tikzcd}
\end{equation}
If $\texttt{phi}: v \mapsto w$, and $\texttt{psi}: w \mapsto v'$, the question again becomes:  are $v$ and $v'$ in the same connected component of $(V_G,E_G)_{\sigma_j,2(n+k)}$?
So similar to the parallelogram case, we take a strip of the graph and check this connectivity question in $O(d(V_G +E_G))$ time. 
As before, either the elements checked are in the same connected component of the relevant slice of the graph, in which case we move to the next diagram; or it does not, and we move to a different $k$ in our binary search. 
In our example case of Fig.~\ref{fig:Assignment} with $n=k=1$,  we have $2(n+k) = 4$.
Then chasing $b$, we need to check that $b$ and $\psitt \circ \phitt(b) = c$ are in the same connected component of $(V_G,E_G)_{\sigma_j,4}$. 
As this slice has one connected component, this triangle commutes. 
We can check another triangle $\Ltu^{U_{\sigma_j}} \leq k$ chasing $w$. 
In this case, we must check if  $w$ and $\phitt \circ \psitt(w) = z$ are in the same component of $(V_G,E_G)_{\sigma_i,4}$, which again, they both are. 
In either case, if they were not, we would know the loss function is at least $k$ and continue in the binary search.

%

To count the number of diagram checks done, a vertex $v \in F(U_{\sigma_i})$  is checked for one triangle loss $\Ltd^{U_{\sigma_i}}$; and a vertex $w \in G(U_{\sigma_i})$ is checked for one triangle loss $\Ltu^{U_{\sigma_i}}$. 
An edge 
$e \in F(U_{\tau_i})$ 
is checked for one triangle loss 
$\Ltd^{U_{\tau_i}}$ 
and for two parallelograms: 
$\Lpl^{(U_{\tau_i}, U_{\sigma_{i}})}$
and
$\Lpl^{(U_{\tau_i}, U_{\sigma_{i+1}})}$. 
Likewise, an edge $e' \in G(U_{\tau_i})$ is checked in diagrams 
$\Ltu^{U_{\tau_i}}$,
$\Lpr^{(U_{\tau_i}, U_{\sigma_{i}})}$ and
$\Lpr^{(U_{\tau_i}, U_{\sigma_{i+1}})}$. 
This means that if the graph representations of $F$ and $G$ are $(V_F,E_F)$ and $(V_G,E_G)$ respectively, the time for computing the loss function is 
\begin{equation*}
    O\Bigg( [(V_F+V_F) + 3 (E_F + E_G) ] \cdot \max\{(V_F+E_F),(V_G+E_G)\} \Bigg).
\end{equation*}
In $d$-dimensions, a similar construction holds, except that our $\sigma$ cells are now indexed by $B^d$, giving an extra multiplicative factor of $B^d$ in the run time. 
%



%

%
%
%


%
%


%
%
%
%
%
%
%

%
%
%
%
%
%
%
%
    
%


%
%

%

%
%
%
%
%
%
%
%
%
%
%
%
%
%
%
%
%
%
%
%
%
%
%
%

%
%


%

%

%
%
%
%
%
%
%

%

%