\section{Discussion}
\label{sec:discussion}

In this paper, we define a loss function that quantifies how far a diagram is from being commutative, and use such a loss function to bound the interleaving distance.  
We believe that our framework is applicable in a broader context where data are modeled as sheaves or cosheaves in the category of sets, as sheaf theory is emerging as a tool in data science to study, e.g.,   distributed systems~\cite{Malcolm2009,Mansourbeigi2017}, sensor networks~\cite{Robinson2017}, model fit~\cite{KvingeJeffersonJoslyn2021}, and uncertainty quantification~\cite{JoslynCharlesDePerno2020}. 
In terms of computation, we are interested in utilizing optimization techniques in combination with our  loss function to find better bounds on the interleaving distance, perhaps even with guaranteed approximation factors. 
Our work is also a first step toward bounding the interleaving distance of persistence modules. 
It is left for future work to extend our framework to work with persistence modules as cosheaves in the category of vector spaces (e.g.,~\cite{BubenikMilicevic2021}). 




