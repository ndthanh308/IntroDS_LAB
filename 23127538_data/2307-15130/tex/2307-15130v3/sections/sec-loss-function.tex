
%
\section{Loss Function and Bounds}
\label{sec:loss-function}

%
In this section, we  introduce a loss function for interleavings on $\R^d$-mapper graphs.  
%
%
In Sec.~\ref{ssec:LossFunction} we give the definition of the loss function (Defn.~\ref{def:Loss_v1}) and give our first version of the bound as Thm.~\ref{thm:bound} in Sec.~\ref{ssec:bound_v1}.
However, this version of the bound requires checking diagrams for all possible open sets $U \in \Open (K)$ which creates a combinatorial explosion that is not helpful for use in practice.
Thus, in Sec.~\ref{ssec:BasisBound} we prove this loss function can be replaced with an improved loss function which only needs to check the open sets for a basis of $\Open(K)$. 

%
%
%

%
\subsection{Loss function definition}
\label{ssec:LossFunction}
We start by turning each $F(U)$ (similarly $G(U)$) into a metric space, as follows:
\begin{definition}
Define the distance $d_U^{F}(A,B)$ for $A,B \in F(U)$ to be the smallest $n$ such that $A$ and $B$ represent the same connected component when included into $U^n$. 
Specifically,
\begin{equation*}
    d_U^{F}(A,B) = 
    \min\{ n\geq0 \mid F[U \subset U^n](A) = F[U \subset U^n](B)\}. 
\end{equation*}
If no such $n$ exists, then we we set $d_U^{F}(A,B) = \infty$.
\end{definition}

Consider the example of Fig.~\ref{fig:examplegraph-distance} with a single input graph encoded by a cosheaf  $F:\Open(K) \to \Set$. 
The set $F(U)$ has two elements, which we denote by $A$ and $B$ as they represent the connected components containing the points $a$ and $b$ respectively. 
Then $d_U^F(A,B) = 1$ since thickening the set $U$ by one puts $a$ and $b$ in the same connected component.
Likewise, denoting the elements of $F(V)$ by $W$ and $Z$, we see that $d_V^F(W,Z) = 2$ since we must expand the set $V$ twice before $w$ and $z$ are in the same connected component. 
% Figure environment removed


As a first useful property of this distance, thickening a set means the distance between components will only decrease.
For an example of this consider $W,Z \in F(V)$ representing points $w$ and $z$ in Fig.~\ref{fig:examplegraph-distance}.
As noted previously $d_V^F(W,Z) = 2$. 
However, if the elements ${W',Z' \in F(V^1)}$ represent the connected components in the 1-thickening of $V$, then $d_{V^1}^F(W',Z') = 1$, and in particular, $d_V^F(W,Z) \geq d_{V^1}^F(W',Z')$. 
This idea is formalized in the following lemma:

\begin{lemma}
\label{lem:distanceContraction}
Fix $k \geq 0$ and any $A,B \in F(U)$ with images $A' = F[U\subseteq U^k](A)$ and \linebreak $B' = F[U\subseteq U^k](B)$ in $F(U^k)$. 
Then $d_U^F(A,B) \geq d_{U^k}^F(A',B')$. 
\end{lemma}
\begin{proof}
Let $n = d_U^F(A,B)$, so that we know the image of $A$ and $B$ in $F(U^n)$ is the same. 
If $k \geq n$, then we use the functor maps $F(U) \to F(U^n) \to F(U^k)$ to see that the images of $A$ and $B$ are the same in $F(U^n)$ so they are the same in $F(U^k)$. 
Then $d_{U^k}^F(A',B') = 0 \leq d_U^F(A,B)$. 
If $k < n$, then we have the maps $F(U) \to F(U^k) \to F(U^n)$.
This means that $d_{U^k}^F(A',B') \leq n-k \leq n = d_U^F(A,B)$, completing the proof.
\end{proof}




%
%


%
%
%

We use this framework as follows: first, assume we are given $F$ and $G$ but our attempts at finding an interleaving  do not necessarily satisfy the requirements of a natural transformation. 
Normally, a natural transformation $\eta:F \Rightarrow G$ is a collection of component morphisms ${\eta:F(U) \to G(U)}$ which commute with the inclusions: 
\[
\begin{tikzcd}[sep=scriptsize]
	{F(U)} && {F(V)} \\
	\\
	{G(U)} && {G(V)}.
	\arrow["{F[\subseteq]}", from=3-1, to=3-3]
	\arrow["{\eta_u}"', from=1-1, to=3-1]
	\arrow["{F[\subseteq]}", from=1-1, to=1-3]
	\arrow["{\eta_V}", from=1-3, to=3-3]
\end{tikzcd}
\]
The following definitions, inspired by \cite{nlab:unnatural_transformation} and~\cite{Robinson2020}, give names to collections of component morphisms used to define an interleaving where the square might not commute. 

\begin{definition}
\label{def:assignment}
Given functors $H,H':\Open(K) \to Set$, an \emph{unnatural transformation}   $\eta:H \rightarrow H'$ is a collection of maps $\eta_U:H(U) \to H'(U)$ with no additional promise of commutativity. 
For a fixed $n \geq 0$ and cosheaves $F$ and $G$, an \emph{assignment}, or more specifically an \emph{$n$-assignment}, is a pair of unnatural transformations $\phi:F \Rightarrow G^n$ and $\psi:G \Rightarrow F^n$.

\end{definition}



%
%
%


In order to clarify notation, for the remainder of the paper, we will be using $n$-assignments to  build $(n+k)$-interleavings, which by definition will be required to be natural transformations. 
When the $n$-assignment might not commute, we  denote its maps by lower case $\phi$ and $\psi$;  for $(n+k)$-assignments which are constructed to be natural transformations, we  denote them by $\Phi$ and $\Psi$. 

In the spirit of  \cite{Robinson2020}, we will measure the quality of a choice of an  $n$-assignment $\phi, \psi$ using the collections of distances $\{d_U^F \mid U \in \Open(K)\}$ and $\{d_U^G \mid U \in \Open(K)\}$.  
First, note that checking that $\phi$ and $\psi$ are natural transformations means ensuring the diagrams
\begin{equation*}
    \begin{tikzcd}%
        F(U)  
            \ar[r, "{F[\subseteq ]}"] 
            \ar[dr, "\phi_U"', very near start, violet]
        & F(V)
            \ar[dr, "\phi_{U^n}"', very near start, violet]
        & \\
        & G(U^n) 
            \ar[r, "{G[\subseteq ]}"'] 
        & G (V^n)
    \end{tikzcd}
    \begin{tikzcd}%
        & F(U^n)
            \ar[r, "{F[\subseteq ]}"] 
        & F (V^n)\\
        G(U)
            \ar[r, "{G[\subseteq ]}"'] 
            \ar[ur, "\psi_U", very near start, orange, crossing over]
        & G(V) 
            \ar[ur, "\psi_{U^n}", very near start, orange, crossing over]
        & 
    \end{tikzcd}
\end{equation*}
commute. 
As we use these repeatedly, we will denote these diagrams by $\Parallelograml_\phi(U,V)$ and $\Parallelogramr_\psi(U,V)$, dropping the subscript when it is clear from context.
Then checking whether the pair constitutes an interleaving involves checking commutativity of the diagrams
\begin{equation*}
\label{eq:fourDiagrams}
%
%
%
%
%
%
%
%
%
%
%
%
%
%
%
%
%
%
%
%
%
%
%
%
    \begin{tikzcd}
        F(U) 
            \ar[rr, "{F[U \subseteq U^{2n}]}"]   
            \ar[dr, "\phi_U"',violet]
            %
            & & F(U^{2n}) & 
        & F(U^n) \ar[dr]
            \ar[dr, "\phi_{U^{n}}",violet]
        & \\
        & G(U^n)\ar[ur, "\psi_{U^n}"', orange]  & & 
        G(U) 
            \ar[rr, "{G[U \subseteq U^{2n}]}"']
            \ar[ur, "\psi_{U}", orange] 
        %
        && G(U^{2n})
    \end{tikzcd}
\end{equation*}
which we denote by $\triangled_{\phi,\psi}(U)$ and $\triangleu_{\phi,\psi}(U)$ respectively, again dropping the subscripts when unnecessary. 
%
%
We measure quality of the given assignments by checking how far these four diagrams are from commuting in the sense of the distances defined at the terminal vertex of the shape. 





\begin{definition}
\label{def:Loss_v1}
Fix an $n$-assignment
$(\phi,\psi)$.
%
%
We define four \emph{diagram loss functions}:
%
%
%
%
%
%
%
%
%
%
%
%
%
%
%
%
%
%
%
%
%
%
%
%
%
\begin{align*}
\Lpl^{U,V}(\phi)
    &= \max\limits_{\alpha \in F(U)} d_{V^n}^{G}(\varphi_U^n \circ F[U \subseteq V](\alpha),
    G[U^n \subseteq V^n] \circ \varphi_U(\alpha))\\
\Lpr^{U,V} (\psi)
    &= \max\limits_{\alpha \in G(U)} d_{V^n}^{F}(
    \psi_U^n \circ G[U \subseteq V](\alpha), 
    F[U^n \subseteq V^n] \circ \psi_U(\alpha)
    )\\
\Ltd^U (\phi,\psi)
    &= \max\limits_{\alpha \in F(U)}  \Big \lceil \tfrac{1}{2} \cdot d_{U^{2n}}^{F}(
    F[U \subseteq U^{2n}] (\alpha),
    \psi_{U^n} \circ \varphi_U(\alpha)
    ) \Big \rceil\\
\Ltu^U (\phi,\psi)
    &= \max\limits_{\alpha \in G(U)}\Big \lceil \tfrac{1}{2} \cdot d_{U^{2n}}^{G}(
    G[U \subseteq U^{2n}](\alpha),
    \varphi_{U^n} \circ \psi_U(\alpha)
    )\Big \rceil.
\end{align*}%
%
Then the loss for a given assignment is defined to be
\begin{equation*}
L(\phi,\psi) = \max_{U\subseteq V}\left\{\Lpl^{U,V}, \Lpr^{U,V}, \Ltu^U, \Ltd^U\right\}.
%
\end{equation*}
\end{definition}

These loss functions are defined in a way so that while the diagram in question might not commute, pushing $n$ forward by the loss value will send the elements to the same place. 
For example, if   $\Lpl^{U,V}(\phi)  =k$, then in the diagram 
\begin{equation}
\label{eqn:dgm:parallelExtendK}
\begin{tikzcd}
        F(U)  
            \ar[r, "{F[\subseteq ]}"] 
            \ar[dr, "\phi_U"', very near start, violet]
        & F(V)
            \ar[dr, "\phi_{U^n}"', very near start, violet]
        & \\
        & G(U^n) 
            \ar[r, "{G[\subseteq ]}"'] 
        & G (V^n) \ar[r] 
        & G(V^{n+k})
\end{tikzcd}
\end{equation}
the image of a point from $F(U)$ is the same  in $G(V^{n+k})$  no matter which path is followed. 
Similarly, if $\Ltd^U (\phi,\psi)=k$, then in the diagram 
\begin{equation}
\label{eqn:dgm:triExtendK}
    \begin{tikzcd}
        F(U) 
            \ar[rr, "{F[U \subseteq U^{2n}]}"]   
            \ar[dr, "\phi_U"',violet]
            %
            & & F(U^{2n})  \ar[r] 
            & F(U^{2(n+k)})
        \\
        & G(U^n)\ar[ur, "\psi_{U^n}"', orange]  & 
    \end{tikzcd}
\end{equation}
the image of a point in $F(U)$ is the same in $F(U^{2(n+k)})$ even if not in $F(U^{2n})$.

%
%
%


%
%
%
%
%
%




%
%
%
%
%
%
%


%





%
%
%
%
%
%
%
%
%
%
%
%
%

%

%

%

%

%
\textbf{An Example:} Consider Fig.~\ref{fig:nonzero_loss_finite} and fix $n=1$. 
Denote the connected component of the point $a$ in $F(U)$, $F(U^1)$, and $F(U^2)$ by $A$, $A'$, and $A''$, respectively.
Similarly, the connected component of the point $b$ is denoted by  $B'' \in G(U^{2})$. 
Follow the same form for the connected components of points $w$ and $z$ in $G$.
The interleaving diagrams can be collected together as 
\begin{equation}
\label{eq:interleavingLadder_example}
    \begin{tikzcd}[row sep=large, column sep=huge]
        {\color{blue}\{A\}}  
            \ar[r, "{F[U \subseteq U^1]}"] 
            \ar[dr, "\phi_U"', very near start, violet]
        & {\color{blue}\{A'\}} 
            \ar[r, "{F[U^1 \subseteq U^{2}]}"] 
            \ar[dr, "\phi_{U^n}"', very near start, violet]
        & {\color{blue}\{A'',B''\}} \\
        {\color{red}\{W,Z\}} 
            \ar[r, "{G[U \subseteq U^1]}"'] 
            \ar[ur, "\psi_U", very near start, orange, crossing over]
        & {\color{red}\{W',Z'\}}
            \ar[r, "{G[U^1 \subseteq U^{2}]}"'] 
            \ar[ur, "\psi_{U^1}", very near start, orange, crossing over]
        & {\color{red}\{W'',Z''\}}
    \end{tikzcd}
\end{equation}
noting that the horizontal maps are determined by sending a letter to the same letter with an additional prime. 
The distances between the points in their respective sets are
\begin{equation*}
    \begin{matrix}
    & & d_{U^2}^F(A'',B'')  = 1\\
    \\
     d_U^G(W,Z)  = 3& 
    d_{U^1}^G(W',Z')  = 2& 
    d_{U^2}^G(W'',Z'') = 1.       
    \end{matrix}
\end{equation*}
%
%
%
%
%
%
%
%
%
%
%
%
%
%
%
%
%
%
%
%
% Figure environment removed
Consider the following example assignment:
\begin{equation*}
\begin{matrix}
    \phi_U: A \mapsto W'& & 
    \psi_U: W,Z \mapsto A'\\
    \phi_{U^1}:A' \mapsto W''& &
    \psi_{U^1}: \substack{W' \mapsto A'' \\ Z' \mapsto B'' }.
\end{matrix}
\end{equation*}
In this case, we then have that 
$\Lpl^{U,U^n} = 0$,
$\Lpr^{U,U^n} = 1$, 
$\Ltd^U = 0 $, 
and $\Ltu^U= 1$
so again $L(\phi,\psi) \geq 1$.
Note for this particular example, no $n=1$ interleaving is possible so any choice of assignment will have a non-zero loss (the easiest check is to see that any choice of assignment will force $\Ltu^U =1$). 
%


%
\subsection{Bounding the interleaving distance}
\label{ssec:bound_v1}

We now use the loss function to give an upper bound for the interleaving distance. 
%
\begin{restatable}{theorem}{FirstLossBound}
\label{thm:bound} 
    For an $n$-assignment,  $\phi\colon F \Rightarrow G^n$ and $\psi\colon G \Rightarrow F^n$, 
    \begin{equation*}
        d_I(F, G) \leq  n+L(\phi, \psi). %
    \end{equation*}
\end{restatable}

To prove this, we require the following technical lemma, proved in Appx.~\ref{sec:technicalProofs}.

%
%
%
%

\begin{restatable}{lemma}{lossimpliescommutes}
\label{lem:lossimpliescommutes}
Assume we are given an $n$-assignment
$\phi:F \Rightarrow G^n$ and 
$\psi:G \Rightarrow F^n$. 
For a fixed $k$, define $(n+k)$-assignments
$\Phi_U = G[U^n\subseteq U^{n+k}]\circ \phi_U$
and 
$\Psi_U = F[U^n\subseteq U^{n+k}]\circ \psi_U$ for all $U \in \Open(K)$. 
Then the following hold:
\begin{enumerate}
    \item $\Lpl^{U,V}(\phi) \leq k$ implies $\Parallelograml_{\Phi}(U,V)$ commutes, and thus $\Lpl^{U,V}(\Phi) = 0$.
    \item $\Lpr^{U,V}(\psi) \leq k$ implies $\Parallelogramr_{\Psi}(U,V)$ commutes, and thus $\Lpr^{U,V}( \Psi) = 0$.
    \item 
    $\Ltd^{U}(\phi,\psi) \leq k$ 
    and
    $\Lpr^{U^n,U^{n+k}}(\psi) \leq k$     imply $\triangled_{\Phi, \Psi}(U)$ commutes, and thus $\Ltd^{U}(\Phi, \Psi) = 0$.
    \item $\Ltu^{U}(\phi,\psi) \leq k$  and
    $\Lpl^{U^n,U^{n+k}}(\phi) \leq k$ 
    imply $\triangleu_{\Phi, \Psi}(U)$ commutes, and thus $\Ltu^{U}(\Phi, \Psi) = 0$.
\end{enumerate}
In particular, if  $\phi$ and $\psi$ have $L(\phi,\psi) = 0$, then $\phi$ and $\psi$ constitute an interleaving, and so $d_I(F,G) \leq n$.
\end{restatable}







\begin{proof}[Proof of Thm.~\ref{thm:bound} ]
Set $k = L(\phi,\psi)$,
so by definition,  $\Lpl^{U,V}(\phi) \leq k$, $\Lpr^{U,V}(\psi) \leq k$, $\Ltd^{U}(\phi,\psi) \leq k$, and $\Ltu^{U}(\phi,\psi) \leq k$. 
As in Lem.~\ref{lem:lossimpliescommutes}, construct two $(n+k)$-assignments: 
$\Phi$ given by 
$\Phi_U = G[U^n \subseteq U^{n+k}] \circ \phi$,  and
$\Psi$ given by 
$\Psi_U = F[U^n \subseteq U^{n+k}] \circ \psi$.
By Lem.~\ref{lem:lossimpliescommutes}, this means the diagrams 
$\Parallelograml_{\Phi}(U,V)$,
$\Parallelogramr_{\Psi}(U,V)$,
$\triangled_{\Phi, \Psi}(U)$, and 
$\triangleu_{\Phi, \Psi}(U)$ 
commute for all pairs $U\subseteq V$. 
This implies that $\Phi$ and $\Psi$ are an $(n+k)$-interleaving, giving the theorem. 
\end{proof}

First, notice that this proof works by explicitly constructing an interleaving from a given $n$-assignment.
Second, we have no reason to believe that this bound is tight.
In particular, in Sec.~\ref{ssec:BasisBound} we improve the  bound by way of restricting the computation to the basis for the topology of $K$.
%

%

%
%
%
%
%
%

%

%
%
%
We include one additional note about when this loss function can be promised to be finite. 
Define the diameter of a metric space to be the largest distance between points, which we denote by 
%
$
    \mathrm{diam}(X,d) = \sup \{ d(a,b) \mid a,b \in X\}
$,
%
and note that here, the $\sup$ can be replaced with a $\max$ since we are working in finite metric spaces.
\begin{lemma}
The loss function for an $n$-assignment $(\phi,\psi)$ is bounded above by the distances in $F(U^{2n})$ and $G(U^{2n})$; specifically,
\begin{equation*}
    L(\phi,\psi) \leq 
    \max \left\{\mathrm{diam}(F(U^n),d_F^{U^n}) \mid U \in \Open(K)\right\} \cup \left\{\mathrm{diam}(G(U^n),d_G^{U^n})\mid U \in \Open(K)\right\}.
\end{equation*}
In particular, if the original graphs are each connected, then $L(\phi,\psi)$ is finite.
\end{lemma}

\begin{proof}
Note that the parallelogram portions of the loss function $\Lpl$ and $\Lpr$ take value from distances in $F(U^n)$ and $G(U^n)$. 
The triangle portions $\Ltd$ and $\Ltu$ take value from distances in $F(U^{2n})$ and $G(U^{2n})$. 
However, by Lem.~\ref{lem:distanceContraction}, we know that distances can only decrease from $F(U^n)$ to $F(U^{2n})$, meaning the maximum diameter is obtained on some open set $U^n$ giving the inequality. 
For the second statement, if the input graphs each have a single connected component, then any pair of elements $a, b \in F(U)$ map to the same element under the inclusion $F(U) \to F(U^K)$ for a large enough $K$. 
This in turn implies that the diameter of $d_U^F$ is finite for every $U$. 
\end{proof}

% Figure environment removed
Consider the example in Fig.~\ref{fig:infiniteLoss}.
Let $\{A,B\}$, $\{A',B'\}$, and $\{A'',B''\}$ be the representatives of the connected components of the points $a$ and $b$ in $F(U)$, $F(U^1)$ and $F(U^2)$ respectively. 
Note that because there is no $n$ for which the two points are the same connected component of $X$, the distance between $A$ and $B$ is $\infty$ in all three sets. 
Then no matter the choice of $1$-assignment, $\Ltd = \infty$ making the loss function infinite. 
%

%





%
\subsection{Restriction to basis elements}
\label{ssec:BasisBound}
To this point, we have measured the loss function by studying all possible open sets $U$. 
While this is helpful for definitions, it does not make for a reasonable computational setting. 
To that end, we now focus on a basis of the topology, and prove that this basis suffices.

\begin{definition}
    An open set defined by the upset of a cell $\sigma \in K$ (that is, a vertex, edge,  square, etc), $U_\sigma = \{ \tau \mid \tau \geq \sigma\}$, is called a \emph{basic open set}.
    %
\end{definition}

Note that  $\{U_\sigma \mid \sigma \in K \}$ is a basis for the Alexandroff topology. 
Also, this is an order reversing process, as for $\sigma \leq \tau$, $U_\tau \subseteq U_\sigma$.
We next give a name to the case where we are only given $n$-assignment information for basis elements, or equivalently if we are given a full assignment but ignore the maps for non-basis open sets.
\begin{definition}
A \emph{basis unnatural transformation} for functors $H$ and $H'$ is a collection of maps $\eta_{U_\sigma}:H(U_\sigma) \to H'(U_\sigma)$ for all basis elements $U_\sigma$ from $\sigma \in K$. 
A \emph{basis $n$-assignment} (or simply a basis assignment) is a pair of basis unnatural transformations
$$
\{\phi_{U_\sigma} :F(U_\sigma) \to G(U^n_\sigma) \mid \sigma \in K\} 
\qquad \text{and}\qquad 
\{\psi_{U_\sigma} :G(U_\sigma) \to F(U^n_\sigma) \mid \sigma \in K\} 
$$
\end{definition}

In this section, we  prove that we can focus our loss function efforts on only those diagrams associated to basic opens, and the solution can be extended to any open set.
\begin{definition}
\label{def:basisLoss}
    The \emph{basis loss function} is defined to be 
\begin{equation*}
L_B(\phi,\psi) = \max_{\sigma \leq \tau}
\left\{
\Lpl^{U_\tau, U_\sigma}, \Lpr^{U_\tau, U_\sigma}, \Ltu^{U_\sigma}, \Ltd^{U_\sigma}
\right\}.
\end{equation*}
\end{definition}
It is immediate from the definitions that $L_B \leq L$ as the $L_B$ maximum is taken over a subset of those used to determine $L$. 
This means, in particular, that if $L=0$ then $L_B = 0$. 
These values are not always equal; for instance we might have chosen a basis assignment for which every diagram commutes (making $L_B = 0$), but $\phi_V$ defined on non-basis elements causes a non-zero loss function so $L >0$. 
%
However in the special case where $L_B = 0$, and thus the basis open diagrams are commutative, we do have the ability to extend the information checked to a full interleaving. 
This can be seen in the following lemma, proved in Appx.~\ref{sec:technicalProofs}. 

\begin{restatable}{lemma}{extendToNatTrans}
\label{lem:extendToNatTrans}
Given a basis unnatural transformation
$\{\Phi_{U_\sigma}: F(U_\sigma) \to G(U_\sigma^N) \mid \sigma \in K\}$ with $\Lpl^{U_\tau, U_\sigma} = 0$ for all $\sigma \leq \tau$, we can extend this to a full natural transformation $\Phi$; i.e.~we can define $\Phi_U$ for all $U$ such that $\Lpl^{U,V} = 0 $ for all $U \subseteq V$. 
\end{restatable}

Note that the symmetric version extending a basis unnatural transformation $\Psi$ to a natural transformation $\Psi: F \Rightarrow G^N$ is obtained in exactly the same way. 
Next we can take these natural transformations and ensure the triangles commute (thus giving an interleaving) by only checking the basis set triangles, again proved in Appx.~\ref{sec:technicalProofs}.

\begin{restatable}{lemma}{extendTriangles}
\label{lem:extendTriangles}
    Given natural transformations $\Phi:F \Rightarrow G_N$ and $\Psi:G_N \Rightarrow F$ such that $\Ltd^{U_\sigma} = 0$ for all $\sigma \in K$, then $\Ltd^{U} = 0$ for all open sets $U$. 

%
%
%
%
%
\end{restatable}



Taken together, we immediately have the following proposition. 
\begin{proposition}
\label{prop:zeros}
Fix a basis $N$-assignment $(\Phi,\Psi)$. 
If $L_B(\Phi,\Psi) = 0$, then $\Phi$ and $\Psi$ can be extended to full $N$-assignments with $L(\Phi,\Psi) = 0$ and thus constitute an interleaving. 
\end{proposition}

Finally, we arrive at our main theorem, where we can use the provided basis $n$-assignment and the calculated loss function to give a bound for the interleaving distance. 

 
%
\begin{theorem}
\label{thm:secondBound}
Given a basis $n$-assignment  
$\phi = \{\phi_{U_\sigma} \mid \sigma \in K\}$ 
and 
$\psi = \{\psi_{U_\sigma} \mid \sigma \in K\}$, 
\begin{equation*}
    d_I(F,G) \leq n + L_B(\phi,\psi).
\end{equation*}
%
\end{theorem}
%
%
\begin{proof}
This proof proceeds in the same way as that of Thm.~\ref{thm:bound} with some minor modifications of input assumptions. 
First, let $k = L_B(\phi,\psi)$; and 
define basis $(n+k)$-assignments 
\begin{equation*}
\{\Phi_{U_\sigma} = G[\subseteq] \circ \phi_{U_\sigma} \mid \sigma \in K\}
\qquad \text{ and } \qquad 
\{\Psi_{U_\sigma} = F[\subseteq] \circ \psi_{U_\sigma} \mid \sigma \in K\}. 
\end{equation*}
By Lem.~\ref{lem:lossimpliescommutes}, we know that  $\Lpl^{U_\tau,U_\sigma}(\Phi)  =0$
and
$\Lpr^{U_\tau,U_\sigma}(\Psi)  =0$
for all $\tau \leq \sigma$.
Then by Lem.~\ref{lem:extendToNatTrans}, we can extend $\Phi$ and $\Psi$ to full natural transformations defined for all $U \in \Open(K)$. 

To show that $\Phi$ and $\Psi$ constitute an $(n+k)$-interleaving, we must check triangles; i.e.~ensure that $\Ltd^{U}(\Phi, \Psi) = \Ltu^{U}(\Phi, \Psi)= 0$. 
With the goal of using part 3 of Lem.~\ref{lem:lossimpliescommutes}, first note that $\Ltd^{U_\sigma} (\phi,\psi) \leq k$ for basis elements. 
We can see that $\Lpr^{U_\sigma^n, U_\sigma^{n+k}} \leq k$ by using the (non-commutative) diagram 
\begin{equation*}
\begin{tikzcd}
    & F(U_\sigma^{2n}) 
        \ar[rr , "{F[\subseteq]}"] 
    && F(U_\sigma^{2n+k}) 
        \ar[r, "{F[\subseteq]}"]  
    & F\left(U_\sigma^{2(n+k)}\right)\\
    G(U_\sigma^n) 
        \ar[rr, "{G[\subseteq]}"'] 
        \ar[ur, "\psi_{\bullet}"] 
        \ar[urrr, "\Psi_{\bullet}", orange]
    && G(U_\sigma^{n+k}). 
        \ar[ur, "\psi_{\bullet}", very near start] 
        \ar[urr, "\Psi_{\bullet}"', orange]
\end{tikzcd}
\end{equation*}
The leftmost and rightmost triangles commute by definition of $\Psi$, and the orange parallelogram commutes because $\Psi$ is a natural transformation. 
Then chasing any $x \in G(U_\sigma^n)$ up to the top right $F\left(U_\sigma^{2(n+k)}\right)$ results in the same element, giving the required bound on $\Lpr^{U_\sigma^n, U_\sigma^{n+k}} $. 
%
Using Lem.~\ref{lem:extendTriangles} for $\Phi$ and $\Psi$, $\Ltd^{U} (\Phi,\Psi) =0$ for all open sets $U$. 
The proof that $\Ltu^{U} (\Phi,\Psi) =0$ is similar.
Therefore $\Phi$ and $\Psi$ are an $(n+k)$-interleaving, giving the bound.
\end{proof}




What is surprising about this bound is that despite checking fewer open sets, the loss function for $L_B$ is actually lower than that found using $L$. 
One reason for this is that when we work with the smaller set of input maps, we extend the collection to a ``better'' full assignment, potentially getting rid of some of the causes of a nonzero loss function in the first place. 
For example, a full assignment would be required to provide a map $\phi_U$ for a $U$ with multiple connected components, say $U = V_1 \cup V_2$.
Since no requirements were made of this map based on the $\phi_{V_1}$ and $\phi_{V_2}$ maps, there is a reasonable chance that the loss function contribution from the $\Lpl^{V_1,U}$ is higher than necessary. 
However, in the basis version, we can build the best possible $\phi_V$ given the information over $\phi_{U_1}$ and $\phi_{U_2}$, providing a potentially better, but certainly no worse, bound. 








