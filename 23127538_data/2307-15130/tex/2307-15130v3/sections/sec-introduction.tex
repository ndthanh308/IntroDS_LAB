


\section{Introduction}
%
%
%
%
%
Graphs with additional geometric information arise in many contexts in data analysis. 
For instance, a \emph{geometric graph} is generally defined as an abstract graph along with a well-behaved embedding of the graph into $\R^2$, while a graph with a well-behaved map into $\R$ is called a \emph{Reeb graph}.
In particular from the viewpoint of the Reeb graph, these sorts of input data can arise by studying connected component structures from more general input $\R^d$-spaces, meaning a topological space $\X$ with a function $f:\X \to \R^d$. 
Such graphs are a fundamental object used to model a wide range of data sets, ranging from maps and trajectories to commodity networks (e.g. electrical grids) to skeletons for shape recognition. 
The input  data can be quite noisy, so the ability to compare, cluster, and simplify such representative objects is essential in a data analysis pipeline, leading to a need for theoretically motivated and computable distances. 
In this paper, we study a distance for a discretization of the input data: \emph{$\R^d$-mapper graphs} \cite{Singh2007}. 
That is, starting from data of the form $f:\X \to \R^d$ (or perhaps point cloud data $f: P \hookrightarrow \R^d$), the mapper graph is an encoding of the connected components (resp.~clusters) of $f\inv(U_\alpha)$ for some cover $\cU = \{U_\alpha\}$ of $\R^d$. 


%
%
%


There has been extensive work on metrics for general graphs, geometric graphs, and Reeb graphs 
(see surveys \cite{Deza2009,Conci2017,Donnat2018,Wills2020},
\cite{Buchin2023}, and \cite{Bollen2021} respectively). 
%
%
%
In this paper, we will draw inspiration from the interleaving distance;
specifically, we develop a natural extension of the interleaving distance on Reeb graphs~\cite{deSilva2018} to the setting of $\R^d$-mapper graphs.   
Interleaving distances arose in the context of generalizing the bottleneck distance for persistence modules \cite{Chazal2009} and were subsequently translated to more general categorical frameworks in \cite{Bubenik2014a, deSilva2018}. 
With the exception of 1-parameter persistence \cite{Lesnick2015}, the interleaving distance is NP-hard in many contexts including multi-parameter persistence \cite{Bjerkevik2018,Bjerkevik2019}, and Reeb graphs~\cite{deSilva2018}.
However, some additional structural information can give better algorithms such as FPT computation for merge trees \cite{FarahbakhshTouli2019}, and  polynomial time for formigrams \cite{Kim2019b} and labeled merge trees \cite{Munch2018,Gasparovic2019}.
%
%
See~\cite{Bjerkevik2018} for a recent summary of interleaving distance complexity results.



When $d=1$, there is already work using the interleaving distance to relate the Reeb graph and its mapper graph \cite{Carriere2017,Carriere2018,Munch2016,Brown2020,botnan2020}. 
We will encode the structure of our more general $\R^d$-mapper graphs in a discretized setting by imposing a grid structure $K$ on $\R^d$. 
Then we can represent the input data $f:X \to \R^d$ as a cosheaf of the form $F:\Open(K) \to \Set$ where we store the connected components of inverse images of open sets $\pi_0(f\inv(U))$. 
%
%
The idea of the interleaving distance, in this context, is to compare two cosheaves $F,G:\Open(K) \to \Set$ using a pair of natural transformations $\phi:F \Rightarrow G^n$ and $\psi:G \Rightarrow F^n$ mapping into an $n$-relaxed version of the original inputs. 
This idea was briefly mentioned as an example in prior work~\cite{botnan2020}, however, to the best of our knowledge, interleavings of this form have not been fully studied in the broader context of $\R^d$-graphs. 


% Figure environment removed

While powerful in theory, the computational complexity of the construction  has meant a lack of the use of the interleaving distance in practice.
To circumvent these issues, we take inspiration from recent work of Robinson \cite{Robinson2020} to define quality measures for families of maps that do not rise to the level of a natural transformation, and then apply these quality measures to $\R^d$-mapper graphs.
In \cite{Robinson2020}, the object of study is a single input assignment of data of the form $P: \Open(X) \to \Set$ and, with the added structure of a pseudometric for each set $P(U)$, provides a measurement for how far the input data is from having a global section. 
In our work, we instead work with collections of maps 
${\phi = \{\phi_U: F(U) \to G(U^n)\mid U\}}$ and $\psi = \{\psi_U:G(U) \to F(U^n)\mid U\}$ which we call an \textit{assignment} when they do not necessarily form a true interleaving.  
We then endow the image with the extra structure of a metric space, so that we have pairs $(F(U),d_U)$ for every open set $U$. 
Using this metric structure, we define a loss function $L(\phi,\psi)$ which measures how far the required diagrams of an interleaving are from commuting given any input assignment (Thm.~\ref{thm:bound}).  
We modify this bound by only focusing on the loss function computed for a basis of the topology, $L_B(\phi,\psi)$ (Thm.~\ref{thm:secondBound}), which not only improves the computational complexity but also improves the bound. 
Then, we show that the computation of the bound is polynomial, opening up the potential for algorithmic approximation of the interleaving distance. 
Throughout, we show examples encoding the data of a \textit{geometric graph} (i.e.~a graph $G$ with a straight line embedding $f:G \to \R^2$) or a Reeb graph (a graph $G$ with a straight line map to $\R$) but note that this kind of input is not a requirement for this work. 
%
%

\textbf{Outline}
In Sec.~\ref{sec:Bkgd} we provide the necessary technical background to set up the interleaving distance for  $\R^d$-mapper graph inputs. 
In Sec.~\ref{sec:loss-function}, we define the loss functions and bounds.
%
%
%
We discuss algorithmic requirements of the bound in Sec.~\ref{sec:computation}; and  include technical proofs in Appx.~\ref{sec:technicalProofs}.

%
%
%
%
%
%
%
%
%
%
%
%
%
%
%