%
%
%

%



%
%
%

%

%
%
%
%
%
%
%
%
%
%
%
%
%
%
%
%
%
%
%

%

%
%
%

%
%
%
%
%
%


%
%
%
%

%
%
%

%
%
%

%
\section{Technical Background}
\label{sec:Bkgd}
We will assume several example types of inputs in this paper. 
All are tied together by having some sort of data of the form $f:G \to \R^d$ where $G$ is a finite graph. 
We view these as topological graphs in the sense that we treat the graph as a 1-dimensional CW complex.
We call this input an $\R^d$-graph.
%
%
%
Prior to discussing the specifics of $\R^d$-graphs in our use cases, we  give the necessary categorical framework in order to formulate a precise notion of the interleaving distance.

%
%

%
\subsection{Functors and cosheaves}
\label{ssec:cosheaf}

%
We give basic definitions for the category theoretic notions required in this paper, and direct the interested reader to \cite{Riehl2017,Curry2014} for further details. 
A \emph{category} $\cC$ consists of a collection of objects $X,Y,Z,\cdots$ and morphisms $f,g,h,\cdots$ with the following data: 
morphisms $f:X \to Y$ have designated domain $X$ and codomain $Y$. 
Every object has a designated identity morphism $\1_X: X \to X$, and any pair of morphisms $f:X \to Y$ and $g:Y \to Z$ has a composite morphism $gf:X \to Y$. 
These objects and morphisms satisfy an identity axiom, where $f:X \to Y$ is the same as the $\1_Yf$ and $f\1_X$; and composition is associative, so $h(gf) = (hg)f$. 
Some example categories are $\Set$ where objects are finite sets and morphisms are set maps; 
$\Top$ where objects are topological spaces and morphisms are continuous functions;
and $\Open(K)$ for a given topological space $K$, where the objects are open sets and morphisms are given by inclusion. 
%

A \emph{functor} $F:\cC \to \cD$ is a map between categories preserving the relevant structures.
Specifically, for every object $X \in \cC$ there is a an object $F(X) \in \cD$, and for every morphism $f:X \to Y$, there is a morphism $F[f]:F(X) \to F(Y)$. 
To be a functor, $F$ must further satisfy that for  any $X \in \cC$, $F[\1_X] = \1_{F(X)}$ and for any composable pair $f,g \in \cC$, we have $F[gf] = F[g] F[f]$. 
Given functors $F,G: \cC \to \cD$, a \emph{natural transformation} $\eta: F \Rightarrow G$  consists of a map $\eta_X:F(X) \to G(X)$ for every $X \in \cC$ (called the components) so that for any morphism $f:X \to Y$ in $\cC$, $G[f] \circ \eta_X = \eta_Y F[f]$.
%
%
%
%
%
%
%
%
%
%
%
%
%
%
%
One example is $\pi_0:\Top \to \Set$, where $\pi_0(\X)$ is the set of path connected components of $\X$, and morphisms are set maps $\pi_0[f]: \pi_0(\X) \to \pi_0(\Y)$ sending a connected component $A$ in $\X$ to the connected component of $f(A)$ in $\Y$.

%
%
%
%
%
%
%
%
%
%
%
%
%
%
%
%
%
%
%
%
%
%
%
%
%
%
%
%
%
%
%
%
%

We will be particularly interested in functors of the form $F:\Open(X) \to \Set$, which can also be called \emph{pre-cosheaves}. 
A pre-cosheaf is a \emph{cosheaf} if it satisfies a gluing axiom meaning $F(U)$ is entirely determined by $F(U_\alpha)$ for any cover $\{U_\alpha\}_\alpha$. 
Specifically, given an open set $U$ and a cover $\{ U_\alpha \mid \alpha \in A\}$ of $U$, define a category $\cU = \{U_\alpha \cap U_{\alpha'} \mid \alpha,\alpha' \in A \}$ with morphisms given by inclusion. 
Then we have a diagram $F:\cU \to \Set$, and as such can consider its colimit $\lambda:F \to L$.
If the unique map $L \to F(U)$ given by the colimit definition is an isomorphism, then $F$ is called a \emph{cosheaf}.
%
%
%
%
%
%
%
%
%
%
%
%
%
%
%
%



%
\subsection{Functorial Representation of Discretized Graph Map data}
\label{ssec:functorGraphs}
%
%

%

We start by defining a cubical complex on $[-B,B]^d$, where $[-B,B]^d$ will be the bounding box for the image of the graphs $f:X\to\R^d$ and $g:Y \to \R^d$ to be compared.
Following \cite{Kaczynski2004}, we define a cubical complex given by diameter $\delta$.
For the sake of simplicity, assume that $B$ is a multiple of $\delta$, so that the bounding box can be written as $[-L\delta, L\delta]^d$. 
An \emph{elementary interval} is a closed interval in $\R$ of the form 
$[\ell\delta, (\ell+1)\delta]$ or 
$[\ell]:= [\ell\delta, \ell\delta]$ for $\ell \in [-L,\cdots, L] \subset \Z$.
These are called non-degenerate and degenerate intervals, respectively. 
An elementary cube $Q$ is a finite product of two elementary intervals, i.e.
%
   $ \sigma = I_1 \times I_2 \times \cdots \times I_d \subset [-B,B]^d$. 
%
The dimension of a cube $\sigma$ is given by the number of intervals used which are non-degenerate. 
Note that this means 
$0$-cubes are vertices at grid locations 
$(i\delta, j \delta, \ldots, k \delta) \in \delta \cdot \Z^d$, 
$1$-cubes are edges,
%
$2$-cubes are squares,
$3$-cubes are voxels, etc.
%
The collection of elementary cubes $K$ discretizing $[-B,B]^d$ is a finite cubical complex.
This construction comes with a face relation which gives a poset structure, where we write $\sigma \leq \tau$ iff $\sigma \subseteq \tau$. 


We next endow the complex $K$ with the Alexandroff topology, following \cite{Barmak2011}. 
Given the poset $(K,\leq)$, for any set $S \subseteq X$, the up-set is 
$S^{\uparrow} = \{x \mid x \geq y, \, y \in S \}$ and the down-set is
$S^{\downarrow} = \{ x \mid x \leq y, \, y \in S\}$. 
We give $K$ the Alexandroff topology%
\footnote{Note that in the case of finite posets,  either down- or  up-sets can be used to define the topology; while in general the Alexandroff topology is defined using the down-sets as opens \cite{Barmak2011}.  
However, we are trying to avoid using the opposite poset as much as possible to alleviate notation woes, and the correspondence with stars in this setting is useful for our purposes.} 
$\Open(K)$, where a set $U\subseteq K$ is open iff the following holds: 
for any $x \in U$ and any $y \geq x$, $y \in U$. 
Equivalently, this means that $U$ is its own up-set, i.e.~$U = U^\uparrow$.
It can be checked that this topology has a basis given by the collection  
$\{U_\sigma\}_{\sigma \in K}$ 
where 
%
$
    U_\sigma := \{ \sigma\}^{\uparrow} = \{\tau  \mid \tau \geq \sigma \}
$.
%
%
%

%
%
%

%
The example inputs here  are given by $f:G \to \R^d$ where $G$ is a finite topological graph. 
%
For our purposes, we only require that for any open set $U$, the set of connected components $\pi_0 f\inv(|U|)$ is finite.
%
%
%
Then we can encode $f$ in a functor $F:\Open(K) \to \Set$ given by 
\begin{equation*}
    \begin{matrix}
    F: &  \Open(K) & \to & \Set\\
    & U & \mapsto & \pi_0 f\inv(|U|).
    \end{matrix}
\end{equation*}
Note that functoriality of $\pi_0$ means that for $U \subseteq V$ there is an induced map
\[F[U \subseteq V] \colon \pi_0f\inv(|U|) \to \pi_0f\inv(|V|)\]
so that $F$ satisfies the requirements of a functor.
Indeed, this functor is actually a cosheaf. 
When the notation makes the sets involved obvious, we will  write the induced map as $F[\subseteq]:F(U) \to F(V)$. 



%
\subsection{Thickenings}
\label{ssec:thickenings}
% Figure environment removed
Given any set $U \in \Open(K)$, the 1-thickening  is defined by taking the upset of the downset of $U$,  written as 
$   U^1 = (U^{\downarrow})^\uparrow$.
This operation can be thought of as taking the star of the closure of the set; see Fig.~\ref{fig:thickenings} for examples. 
The $n$-thickening is defined to be $n$ repetitions of the process given recursively as
\begin{equation*}
    U^n = 
    \begin{cases}
    U & n = 0\\
    %
    (U^{n-1})^{\downarrow\uparrow} & n \geq 1.
    \end{cases}
\end{equation*}
Note that each $U^n$ is itself an open set in $\Open(K)$, and that if $U \subseteq V$, then $U^n \subseteq V^n$. 
Thus we can view this operation as a functor on the category $\Open(K)$ with morphisms given by inclusion:
\begin{equation*}
    \begin{matrix}
    (-)^n: & \Open(K)  & \to &  \Open(K)\\
    & U & \mapsto & U^n.
    \end{matrix}
\end{equation*}
In Appx.~\ref{sec:technicalProofs},
%
we show that $(-)^n$ is a functor.
Another useful property of this thickening, proved in the appendix, is as follows.




\begin{restatable}{lemma}{composedThickenings}
\label{lem:composedthickenings}
For any $n, n' \geq 0$ and $U \in \Open(K)$, 
$(U^{n})^{n'} = U^{n+n'}$.
%
%
\end{restatable}



We can use this thickening to build an interleaving distance on functors of the form \linebreak
${F:\Open(K) \to \Set}$.
%
%
%
%
%
The first necessary ingredient is the composition of functors $F \circ ( - )^n: \Open(K) \to \Set$, which we  denote by $F^n$. 
This means $F^n(U) = F(U^n)$ with  the similar setup for $G^n$. 
With this notation, an interleaving is a pair of natural transformations $\phi:F \Rightarrow G^n$ and $\psi:G \Rightarrow F^n$, so a component of $\phi$ is a set-map $\phi_U:F(U) \to G(U^n)$. 
There is another component at $U^n$, $\phi_{U^n}:F(U^n) \to G(U^{2n})$, which can also be viewed as a component of a different natural transformation $\phi^n:F^n \Rightarrow G^{2n}$. 
For this reason, we use the notation $\phi_{U^n}$ and $\phi_U^n$ interchangeably when $\phi$ is indeed a natural transformation.\footnote{We are implicitly using Lem.~\ref{lem:composedthickenings} to write the maps this way.}

\begin{definition}
\label{def:interleavingDistance}
Given cosheaves $F,G:\Open(K) \to \Set$ and $n \geq 0$, an \emph{$n$-interleaving} is a pair of natural transformations 
$\phi:F \Rightarrow G^n$
and 
$\psi:G \Rightarrow F^n$
such that the diagrams
%
%
%
\begin{equation*}
    \begin{tikzcd}
        F(U) 
            \ar[rr, "{F[U \subseteq U^{2n}]}"]   
            \ar[dr, "\phi_U"',violet]
            %
            & & F(U^{2n}) & 
        & F(U^n) \ar[dr]
            \ar[dr, "\phi_{U^{n}}",violet]
        & \\
        & G(U^n)\ar[ur, "\psi_{U^n}"', orange]  & & 
        G(U) 
            \ar[rr, "{G[U \subseteq U^{2n}]}"']
            \ar[ur, "\psi_{U}", orange] 
        %
        && G(U^{2n})
    \end{tikzcd}
\end{equation*}
%
%
%
%
%
%
%
%
%
%
%
%
%
%
%
%
%
%
%
commute for all $U \in \Open(K)$.
The interleaving distance is given by 
\begin{equation*}
    d(F,G) = \inf\{ n \geq 0 \mid \text{there exists an $n$-interleaving} \}
\end{equation*}
and is set to be $d(F,G) = \infty$ if there is no interleaving for any $n$.
\end{definition}
As shown in the appendix, this definition fits in the framework built by \cite{Bubenik2014a} and thus it is an extended pseudometric. 
%

%
%
%
%
%
%
%
%
%
%
%
%
%
%
%
%
%
%
%
%


%

%
%
%
%
%

%
%
%
%

%
%

%
%
%
%
%
%
%
%
%
%
%
%
%
%
%
%
%
%

%
%
%
%
%
%
%
%
%
%
%
%
%
%
%
%
%
%
%
%
%



%

%
    
%


%
%
%
%

%
%

%
%