\documentclass{article}
%
\usepackage{fullpage}
\usepackage[dvipsnames]{xcolor}
\usepackage{hyperref}
\hypersetup{
    colorlinks=true,
    linkcolor=blue,
    citecolor=violet,
    filecolor=magenta,      
    urlcolor=cyan,
}

\usepackage{authblk}
\usepackage{graphicx,graphbox}
\graphicspath{{./figures/}}
\usepackage{wrapfig}
%
%
\renewcommand\UrlFont{\color{blue}\rmfamily}

\usepackage{tikz-cd}

\usepackage{amssymb,amsmath,amsthm,bbm}
\usepackage{mathtools}%
\usepackage[capitalize]{cleveref}


\usepackage{amsthm,amssymb,amsmath}

\usepackage{thm-restate}
\usepackage{enumitem}

\numberwithin{equation}{section}

\newtheorem{theorem}[equation]{Theorem}
\newtheorem{corollary}[equation]{Corollary}
\newtheorem{lemma}[equation]{Lemma}
\newtheorem{remark}[equation]{Remark}

\newtheorem{proposition}[equation]{Proposition}
\newtheorem{not1}[equation]{Notation}
\newtheorem{con}[equation]{Conjecture}
\newtheorem{obs}[equation]{Observation}
\newtheorem{ex}[equation]{Example}
\newtheorem{defn}[equation]{Definition}
\newtheorem{definition}[equation]{Definition}
\newtheorem{mapdefn}[equation]{Map Definition}

%
\newtheorem{alg}[equation]{Algorithm}
\Crefname{defn}{Defn.}{Defns.}
\Crefname{lemma}{Lem.}{Lems.}
\Crefname{alg}{Alg.}{Algs.}


%
%
\usepackage[backend = biber, url=false,sorting=nyt,style=numeric-comp,maxbibnames = 99]{biblatex}
%
\addbibresource{interleaving.bib}     %

\usepackage[colorinlistoftodos,prependcaption,textsize=small]{todonotes}

%
\definecolor{ltgray}{gray}{0.9}
\definecolor{gray}{rgb}{0.95,0.95,0.96}
\definecolor{dkgray}{rgb}{0.7,0.7, 0.735}
\definecolor{ltblue}{rgb}{0.55,0.55, 0.95}
\definecolor{ltGreen}{rgb}{0.25,0.65, 0.25}
\definecolor{dkgreen}{RGB}{0, 100, 0}
\definecolor{dkred}{rgb}{0.75,0.0, 0.0}
\definecolor{dkorange}{rgb}{.82,.45, 0.0}
\definecolor{ltred}{rgb}{0.95,0.95, 0.85}
\definecolor{utahRed}{rgb}{.8, 0, 0}
\definecolor{oregonGreen}{rgb}{0, .41, .163}
\definecolor{albanyPurple}{rgb}{0.4, 0, .55}




%
%
%

%
\newcommand{\E}{\mathbb{E}}
\newcommand{\N}{\mathbb{N}}
\newcommand{\R}{\mathbb{R}}
\newcommand{\W}{\mathbb{W}}
\newcommand{\V}{\mathbb{V}}
\newcommand{\X}{{\mathbb X}}
\newcommand{\Y}{{\mathbb Y}}
\newcommand{\Z}{{\mathbb Z}}
%
\newcommand{\CC}{\mathcal{C}}
\newcommand{\DD}{\mathcal{D}}
\newcommand{\FF}{\mathcal{F}}
\newcommand{\GG}{\mathcal{G}}
\newcommand{\HH}{\mathcal{H}}
\newcommand{\II}{\mathcal{I}}
\newcommand{\LL}{\mathcal{L}}
\newcommand{\MM}{\mathcal{M}}
\newcommand{\PP}{\mathcal{P}}
\newcommand{\RR}{\mathcal{R}}
\renewcommand{\SS}{\mathcal{S}}
\newcommand{\TT}{\mathcal{T}}
\newcommand{\UU}{\mathcal{U}}

\newcommand{\cA}{\mathcal{A}}
\newcommand{\cB}{\mathcal{B}}
\newcommand{\cC}{\mathcal{C}}
\newcommand{\cD}{\mathcal{D}}
\newcommand{\cF}{\mathcal{F}}
\newcommand{\cG}{\mathcal{G}}
\newcommand{\cU}{\mathcal{U}}

%
\newcommand{\End}{\mathbf{End}}


\newcommand{\Asgn}{\mathrm{Asgn}}
\newcommand{\Func}{\mathrm{Func}}
\newcommand{\Open}{\mathbf{Open}}
\newcommand{\Set}{\mathbf{Set}}
\newcommand{\denselist}{\vspace{-5pt} \itemsep -2pt\parsep=-1pt\partopsep -2pt}

\newcommand{\rd}{\mathbin{\rotatebox[origin=c]{90}{$\dl$}}}

\newcommand*{\Parallelogramr}[1][]{
  \pgfpicture\pgfsetroundjoin
    \pgftransformxslant{.6}
    \pgfpathrectangle{\pgfpointorigin}{\pgfpoint{.60em}{.65em}}
    \pgfusepath{stroke,#1}
  \endpgfpicture}
  
\newcommand*{\Parallelograml}[1][]{
  \pgfpicture\pgfsetroundjoin
    \pgftransformxslant{-.6}
    \pgfpathrectangle{\pgfpointorigin}{\pgfpoint{.60em}{.65em}}
    \pgfusepath{stroke,#1}
  \endpgfpicture}
  
\newcommand{\triangled}{\raisebox{\depth}{$\bigtriangledown$}}
\newcommand{\triangleu}{\bigtriangleup}

\usepackage{scalerel}
\newcommand{\Lpl}{L_{\scaleobj{.7}{\Parallelograml}}}
\newcommand{\Lpr}{L_{\scaleobj{.7}{\Parallelogramr}}}
\newcommand{\Ltd}{L_{\bigtriangledown}}
\newcommand{\Ltu}{L_{\bigtriangleup}}
%


\newcommand{\e}{\varepsilon}
\renewcommand{\phi}{\varphi}
\newcommand{\inv}{^{-1}}
\newcommand{\radius}{\ensuremath{\mathrm{radius}}}
\newcommand{\1}{\mathbbm{1}}
\renewcommand{\Im}{\mathrm{Im}}
\newcommand{\id}{\mathrm{Id}}
\newcommand{\Id}{\id}

\newcommand{\from}{\colon}
\newcommand{\after}{\circ}
\DeclareMathOperator{\ord}{ord}
\newcommand{\dL}{\ensuremath{d_f}}
\newcommand{\Reeb}{\ensuremath{\mathcal{R}}}
\newcommand{\Sdiv}{\ensuremath{S}}
\newcommand{\Scon}{\ensuremath{S_\text{tr}}}
\newcommand{\Gcon}[1]{\ensuremath{G^{#1}_\text{tr}}}
\newcommand{\fcon}[1]{\ensuremath{f^{#1}_\text{tr}}}
%

%
%
%
%
%
\newcommand{\incirc}[1]{{\ensuremath{\mathrlap{\bigcirc}\!#1\,}}}
\newcommand{\olt}{\incirc{<}}
\newcommand{\oeq}{\incirc{=}}
\newcommand{\ogt}{\incirc{>}}

\newcommand{\econ}{\ensuremath{{\e_\text{trunc}^*}}}

\newcommand{\mypar}[1]{\medskip\noindent{\bfseries #1.}}


\newcommand{\phitt}{\texttt{phi}}
\newcommand{\psitt}{\texttt{psi}}
\newcommand{\vtt}{\texttt{v}}
\newcommand{\wtt}{\texttt{w}}
\newcommand{\Vtt}{\texttt{V}}
\newcommand{\Ett}{\texttt{E}}


\newcommand {\mm}[1] {\ifmmode{#1}\else{\mbox{\(#1\)}}\fi}
\newcommand{\Rspace}        {\mm{{R}}}
\newcommand{\Xspace}        {\mm{{X}}}
\newcommand{\Sspace}        {\mm{{S}}}
\newcommand{\Yspace}        {\mm{{Y}}}
\newcommand{\Zspace}        {\mm{{Z}}}

\newcommand{\Ccal}        {\mm{{\mathcal C}}}
\newcommand{\Fcal}        {\mm{{\mathcal F}}}
\newcommand{\Pcal}        {\mm{{\mathcal P}}}
\newcommand{\Lcal}        {\mm{{\mathcal L}}}
\newcommand{\Xstrata}  {\mm{{\mathfrak X}}} 
\newcommand{\bdr}  {\mm{{\partial}}} 
\newcommand{\str}  {\mbox{{St}}} 
\newcommand{\lk}  {\mbox{{Lk}}}
\newcommand{\closure}  {\mbox{Cl}}
\newcommand{\dime}  {\mbox{dim}}

\newcommand{\Top}{\mathbf{Top}}
\newcommand{\Vect}{\mathbf{Vect}}

%

\newcommand{\Liz}[1]{{\color{violet}\textbf{Liz:} #1}}
\newcommand{\liz}[1]{\Liz{#1}} 
\newcommand{\Erin}[1]{{\color{WildStrawberry}\textbf{Erin:} #1}}
\newcommand{\erin}[1]{{\color{WildStrawberry}\textbf{Erin:} #1}}
\newcommand{\Bei}[1]{{\color{magenta}\textbf{Bei:} #1}}
\newcommand{\Sarah}[1]{{\color{cyan}\textbf{Sarah:} #1}}
%

\begin{document}
%
\title{Bounding the Interleaving Distance for Mapper Graphs with a Loss Function
 \thanks{The authors would like to thank Vidit Nanda for helpful discussions related to non-commutative diagrams. This work was funded in part by the National Science Foundation through grants CCF-1907591, CCF-1907612, CCF-2106578, CCF-2142713, CCF-2106672, DMS-2301361, and IIS-2145499.}
}


\author[1]{Erin Chambers}
\author[2]{Elizabeth Munch}
\author[2]{Sarah Percival}
\author[3]{Bei Wang}
\affil[1]{St.~Louis University}
\affil[2]{Michigan State University}
\affil[3]{University of Utah}

\date{}

\maketitle

\begin{abstract}
Tropical cyclones (TCs) are among the most destructive weather systems. 
Realistically and efficiently detecting and tracking TCs are critical for assessing their impacts and risks. 
In particular, the eye is a signature feature of a mature TC. 
Therefore, knowing the eyes' locations and movements  is crucial for both operational weather forecasts and climate risk assessments. 
Recently, a multilevel robustness framework has been introduced to study the critical points of time-varying vector fields. 
The framework quantifies the robustness (i.e., structural stability) of critical points across varying neighborhoods. 
By relating the multilevel robustness with critical point tracking, the framework has demonstrated its potential in cyclone tracking. 
An advantage is that it identifies cyclonic features using only 2D wind vector fields, which is encouraging as most tracking algorithms require multiple dynamic and thermodynamic variables at different altitudes. 
A disadvantage is that the framework does not scale well computationally for datasets containing a large number of cyclones.  
This paper introduces a topologically robust physics-informed tracking framework (TROPHY) for TC tracking. 
The main idea is to integrate physical knowledge of TC to drastically improve the computational efficiency of  multilevel robustness framework for large-scale  climate datasets. 
First, during preprocessing, we propose a physics-informed feature selection strategy to filter $90\%$ of critical points that are short-lived and have low stability, thus preserving good candidates for TC tracking. 
Second, during in-processing, we impose constraints during the multilevel robustness computation to focus only on physics-informed neighborhoods of TCs. 
We apply TROPHY to 30 years of 2D wind fields from reanalysis data in ERA5 and generate a number of TC tracks. 
In comparison with the observed tracks, we demonstrate that TROPHY can capture TC characteristics (e.g., frequency, intensity, duration, latitudes with maximum intensity, and genesis) that are comparable to and sometimes even better than a well-validated TC tracking algorithm that requires multiple dynamic and thermodynamic scalar fields.
\end{abstract}

%

%
Nondeterministic bottom-avoiding choice is an important and useful idea. 
With the wide-spread use of hardware supporting parallel computation,
it 
can speed up practical computation and, at the same time,
%it is related
relates 
to computation over mathematical structures like real 
numbers~\cite{Escardo96,Tsuiki02}.
On the other hand, it is not easy to apply 
theoretical tools like denotational semantics to nondeterministic bottom-avoiding choice~\cite{HughesO89,Levy07}, and guaranteeing 
correctness and totality of such programs 
%through logical systems 
is a difficult task.
% and to our knowledge, not much work has been done in this direction.
% 

To explain the subtleties of the problem, let us start with an example.
%
Suppose that $M$ and $N$ are partial programs that,  
under the conditions $A$ and $\neg A$, respectively, 
are guaranteed to terminate and produce values satisfying specification $B$. 
Then, by executing $M$ and $N$ in parallel 
and taking the result 
%obtained
returned
first, we should always obtain a result satisfying $B$. 
This kind of bottom-avoiding nondeterministic program
is known as \emph{McCarthy's amb (ambiguous) operator} \cite{McCarthy1963}, and 
we denote such a program by  $\Amb(M, N)$.
$\Amb$ is called the angelic choice operator and
is usually studied as one of the three 
nondeterministic choice operators (the other two are erratic choice and demonic choice).



If one tries to formalize this idea naively, one will face some 
obstacles.  
Let $\ire{M}{B}$ (``$M$ realizes $B$'') denote the fact 
that a program $M$ satisfies a specification $B$ and 
let $\Set(B)$ be the specification that can be satisfied by 
a concurrent program of the form $\Amb(M, N)$ that  always terminates and 
produces a value satisfying $B$.  
Then, the above inference could be written as
\[
  \infer[\hbox{}]{
  \ire{\Amb(M, N)} \Set(B)
}{
A \to (\ire{M}{B})  \ \ \ \     \neg A \to (\ire{N}{B}) 
}
\]
%
However, this inference is not sound for the following reason.
Suppose that $A$ does not hold, that is, $\neg A$ holds. 
Then, the execution of $N$ will produce a
value %following refereee A's suggestion
%data 
satisfying $B$. But the execution of $M$ may terminate as well, and
with a data that does not satisfy $B$ since there is no condition on $M$
if $A$ does not hold.
Therefore,  if $M$ terminates first in the execution of $\Amb(M, N)$, 
%then 
we obtain a result that may not satisfy $B$.

To amend this problem, we add a new operator $\rt{A}{B}$ 
(pronounced ``$B$ restricted to $A$'')
%(pronounced ``$A$ restricts $B$'')
and consider the rule
\begin{equation}\label{eq0}
  \infer[\hbox{}]{
  \ire{\Amb(M, N)} \Set(B)
}{
\ire{M}{(\rt{A}{B})}\ \ \ \  \ire{N}{(\rt{\neg A}{B})}}
\end{equation}

Intuitively, $\ire{M}{(\rt{A}{B})}$ means  two things:
%(1) $M$ terminates if $A$ holds, and
(1) If $A$ holds, then $M$ terminates, and
(2) if 
%% the execution of 
$M$ terminates, then the result satisfies $B$,
even for the case $A$ does not hold.
As we will see in Section~\ref{sub-conc}, 
the above rule is derivable in classical logic
and can therefore be used to prove total correctness of Amb programs.

In this paper, we go 
a step further and
introduce a logical system $\CFP$ whose formulas
can be interpreted as specifications of 
%concurrent and 
nondeterministic programs
although they do not talk about programs explicitly.
%instead of proving correctness of a program given a program and a specification, 
$\CFP$ is defined by adding the two logical operators $\rt{A}{B}$ and 
$\Set(B)$ to the system $\IFP$, 
a logic for program extraction~\cite{IFP} 
(see also \cite{Berger11,SeisenBerger12,BergerPetrovska18}). 
$\IFP$ supports the extraction of lazy functional programs 
from inductive/coinductive proofs in intuitionistic first-order logic.
It has a prototype implementation in Haskell,
called Prawf \cite{DBLP:conf/cie/0001PT20}.
A related approach
has been developed in the
 proof system Minlog~\cite{SchwichtenbergMinlog06,BergerMiyamotoSchwichtenbergSeisenberger11,SchwichtenbergWainer12}.
 
We show that from a $\CFP$-proof of a formula, both
a program and a proof that the program
satisfies the specification can be extracted
(Soundness theorem, Theorem \ref{thm-soundnessI}). 
For example, in $\CFP$ we have the rule 
 \begin{equation}\label{eq00}
   \infer[\hbox{(Conc-lem)}]{
   \Set(B)
 }{
 \rt{A}{B}\ \ \ \    \rt{\neg A}{B}}
  \end{equation}
which is realized by the program %$\lambda a_0. \lambda b. \Amb(b, c)$ is extracted
  $\lambda a. \lambda b. \Amb(a, b)$,
  and whose correctness is expressed by the rule (\ref{eq0}).
Programs extracted from $\CFP$ proofs can be  
executed in Haskell,
implementing $\Amb$
%% with the Amb operator implemented
with  %primitives of 
the concurrent Haskell package.

 
Compared with program verification, the
%% Compared with proving properties of programs, this 
extraction approach has the benefit that 
%
(a) the proofs 
programs are extracted from 
take place in a formal system
that is of a very high level of abstraction and therefore is simpler and
easier to use than a logic that formalizes concurrent programs
(in particular, programs do not have to be written manually at all);
%
(b) not only the complete extracted program is proven correct but also
all its sub-programs come with their specifications and correctness proofs
since these correspond to sub-proofs. This makes it easier to locally
modify programs without the danger of compromising overall correctness.

%We extract, as a case study,  a concurrent 
%As a case study, 
As an application,
we extract a nondeterministic 
program that converts 
infinite Gray code to signed digit representation, where 
infinite Gray code is a 
coding of real numbers by partial digit streams
that are allowed to contain a $\bot$, that is, a digit
whose computation does not terminate~\cite{Gianantonio99,Tsuiki02}. 
Partiality and multi-valuedness are common phenomena in computable analysis 
and exact real number computation~\cite{Weihrauch00,LUCKHARDT1977321}.
This case study connects these two aspects through a nondeterministic and
concurrent program whose correctness is guaranteed by a CFP-proof. % with axioms for real numbers. 
The extracted Haskell programs are listed in the Appendix, and are also 
available in the repository~\cite{githubUB}.   

% 
Organization of the paper: 
In Sections~\ref{sec-ang} and~\ref{sec-ops}
we present the denotational and operational semantics of a functional 
%% programming 
language with $\Amb$ and prove that they match 
(Thorems.~\ref{thm:data} and~\ref{thm:dataconv}).
Sections~\ref{sec-cfp} and~\ref{sec-pe} describe the formal system $\CFP$ 
and its realizability interpretation 
%on which our program extraction method is based 
which our program extraction method is based on 
(Theorems~\ref{thm-soundnessI},~\ref{thm-soundnessII}, and~\ref{thm-pe}).
In Sections~\ref{sec-gray} we extract
%%, as a case study, 
a concurrent program that converts 
representations
 of real numbers 
and study its behaviour in Section~\ref{sec-experiments}.






 
\section{Technical Background}
\label{sec:background}

We first review the classic notion of robustness and multilevel robustness, which are customized to build {\tool} (see \cref{sec:TROPHY}).

\subsection{Robustness}
\label{sec:classicRobustness}

\para{Critical points of a 2D vector field.} 
Unless otherwise specified, we work with a 2D \emph{vector field} $f: \Xspace \subseteq \Rspace^2 \to \Rspace^2$, which assigns a 2D vector to each point in $\Xspace$. 
We use $u_{10}$ and $v_{10}$ to represent the 10-meter zonal (west-east) and meridional (south-north) wind vector components, respectively.  
Then, $f$ is expressed as $f(x) = (u_{10}(x), v_{10}(x))^T$. 

A \emph{critical point} $x \in \Xspace$ in $f$ is where the vector vanishes, that is, $|f(x)| = 0$. 
A critical point $x$ can be classified {\wrt} its \emph{degree} $\mydeg(x)$, defined as the number of field rotations while traveling along a closed curve counterclockwise surrounding $x$ (enclosing no other critical point). 
A source/sink/center has degree $+1$, whereas a saddle point has degree $-1$.
Critical points are important features in studying flow behavior in many applications; see \cref{fig:CriticalPoints} as an example.
%\cref{fig:CriticalPoints} shows four types of critical points from the $\EW$ dataset (see~\cref{sec:data} for details). 
% Figure environment removed


% Figure environment removed

In most cases, the center of a TC can be detected as a center in a vector field when it is intensified into a strong hurricane, with very low wind speed in the eye and extremely high wind speed along the eyewall. 
During the dissipating phase of a TC, such as at its landfall, the center of the TC can be detected as either a source or a sink. 
If a center transforms to a source, then it indicates a divergence in meteorology, which means the weather can be clear and calm. 
If a center transforms to a sink, then it indicates a convergence, which is associated with clouds and precipitation. 
An example of this phenomenon is Hurricane Florence in 2018: a clear sink forms in the 2D wind field  during its landfall, bringing 1-in-500-year expected flooding due to heavy precipitation.

%and can be detected as source points when it is weaken on its dissipating phase, such as at its landfall, \jiali{sentences after this need to be changed if there is no sink. what i wrote there is not true for source, it's true for sink..} which can still bring severe weather such as heavy precipitation, as the source is still a low pressure system, favorable for air rising, cooling and condensation.
 
% Arrow glyphs are placed on sampled points in the domain to indicate the directions of vectors. 






\para{Merge trees.} 
The computation of robustness relies on an \emph{augmented merge tree} modified from the classic merge tree. 
Given a scalar function $f_0$ defined in a 2D domain $\Xspace$, $f_0:\Xspace \to \Rspace$, let $\Xspace_r=f_0^{-1} (-\infty, r]$ denote the \emph{sublevel set} of $f_0$ for some $r \geq 0$. 
A classic \emph{merge tree} is constructed by tracking the evolution of (connected) components in $\Xspace_r$ as $r$ increases. Leaves in a merge tree represent the creation of a component at a local minimum of $f_0$, internal nodes represent the merging of components, and the root represents the entire space as a single component.

% Figure environment removed

% Figure environment removed

\myedit{To construct an augmented merge tree from a 2D vector field $f$, first we define a scalar field $f_0:\Xspace \to \Rspace$ by assigning the vector magnitude to each point $x \in \Xspace$, that is, $f_0(x) = ||f(x)||_2$. In this paper, $f_0$ can be expressed as wind speed.
Second, instead of using local minima of $f_0$ as leaves of the merge tree, the leaves of our augmented merge tree consist of $\Xspace_0$, which is precisely the set of critical points of $f$. 
The tracking of the merging behavior of components is the same as classic merge tree construction. 
Third, once the merge tree is constructed, it can be further augmented with the degrees of critical points (on leaves) and the degrees of components (on internal nodes). The degree of a component is defined as the sum of degrees of critical points the component contains. See~\cref{sec:exampleMT} for an example.}

\para{Robustness calculation.}
The topological notion of \emph{robustness} quantifies the stability of a critical point \wrt~perturbations of the vector field. Let us define the concept of vector field \emph{perturbation} first. A continuous mapping $h: \Xspace \to \Rspace^2$ is an \emph{$r$-perturbation} of $f$, if $d(f, h) \leq r$, where $d(f, h)=\sup_{x\in \Xspace}||f(x)-h(x)||_2$, and $\sup$ means supremum. 
See~\cite{WangRosenSkraba2013} for some mathematical properties of robustness and lemmas to support critical points cancellation under vector field perturbation. 

The robustness of a critical point can be calculated as the function value of its lowest zero-degree ancestor in the augmented merge tree~\cite{WangRosenSkraba2013}. \myedit{See~\cref{sec:exampleRob} for an example.}



%which ignores the possibility of the occurrences of perturbation within a local neighborhood.
%; see~\cite[Sect. 4]{YanUllrichVan-Roekel2022} for an example. 



\subsection{Multilevel Robustness}
\label{sec:ml-Robustness}
In practice, vector fields generated from large-scale ocean and atmospheric datasets contain features at different scales. The drawback of classic robustness comes from building a single merge tree with critical points in the entire domain, which suffer from undesirable boundary effects~\cite{YanUllrichVan-Roekel2022}. 
To mitigate such drawbacks, Yan~\etal\cite{YanUllrichVan-Roekel2022} introduced a notion of multilevel robustness (reviewed in~\cref{sec:MRdef}).
This notion captures the multiscale nature of the data and mitigates the boundary effects suffered by classic robustness computation. 
It also shows initial promise in critical point tracking in practice. 
%They also proposed a multilevel robustness framework to realize the robustness-based critical point tracking in practice. 
We review how the notion of multilevel robustness can improve the feature-tracking results in~\cref{sec:intrgrateWithFTK}, and we give the pipeline to implement the multilevel robustness framework in~\cref{sec:pipelineMR}. For simplicity and comparative purposes, we refer to this original multilevel robustness-based tracking~\cite{YanUllrichVan-Roekel2022} as the {\MR} framework in the remainder of this paper.

\subsubsection{The Multilevel Robustness}
\label{sec:MRdef}
Roughly speaking, the multilevel robustness of a critical point $x \in \Xspace$ can be defined as a sequence of robustness values computed from its neighborhoods of increasing radii. 
Formally, let $B_x(a)$ denote a ball of radius $a$ with a critical point $x \in \Xspace$ as its center.
The multilevel robustness of $x$ can be expressed as 
$R_x: [0,\infty) \to \Rspace$, where $R_x(a)$ is the (classic) robustness of $x$ computed {\wrt} the domain $B_x(a)$ for $a \in [0,\infty)$.
%see~\cref{fig:AdaptiveRegions} for multiple neighborhoods of $x_1$ and $x_3$ with different radii.
Assuming the domain $\Xspace$ contains $n$ critical points, then for a fixed critical point $x \in \Xspace$, its multilevel robustness will change at most $n-1$ times as $a$ increases, since $x$ gets one more candidate as its the cancellation partner as $B_x(a)$ passes through each critical point. 

In~\cref{fig:Partners} (D-F), we give the exact multilevel robustness of $x_1$, $x_2$, and $x_3$, respectively, where the $x$-axis corresponds to the increasing radii and the $y$-axis represents their classic robustness values. We highlight the radii when the neighborhood includes new critical points with blue points in~\cref{fig:Partners} (D-F).  
In~\cref{fig:Partners} (A-C), we visualize all cancellation partners for selected critical points when we use different sizes of neighborhoods in classic robustness computation. 
The cancellation partners are wrapped in bubbles and colored by the number of times that are referred to as cancellation partners of selected critical points. 
For example, $x_1$ and $y_1$ are paired as cancellation partners 12 times, whereas $x_2$ and $y_2$ are paired 122 times.
The classic robustness of $x_1$ calculated with the entire  domain is infinity, even if it can be canceled with $y_1$ within a $7.85$-degree region under a $17.6$-perturbation. 
This phenomenon happens because $x_1$ represents the center of a large-scale cyclone and is surrounded by flows of a large magnitude. If we build an augmented merge tree from the entire input domain during classic robustness calculation, the lowest ancestor of $x_1$ will be the ancestor of the most critical points in the domain. 
This limitation explains why $x_1$ has potential cancellation partners across the entire domain and may not be able to find its cancellation partner if the degree of the entire domain is not equal to zero. See~\cite[Fig. 2]{YanUllrichVan-Roekel2022} for another example. 
 
Therefore, the drawback of classic robustness comes from building a single merge tree with critical points in the entire domain, which ignores the possibility of the occurrences of cancellation within a local neighborhood. The definition of multilevel robustness successfully captures the multiscale nature of the data and mitigates the drawbacks of the classic robustness computation. However, computing the multilevel robustness exactly is time-consuming. For the vector field containing $n$ critical points, we need to conduct $n \times (n-1)$ classic robustness computations. 
In~\cite{YanUllrichVan-Roekel2022}, the {\MR} framework approximates the exact multilevel robustness by using $N$-level robustness. That is, for a critical point $x \in \Xspace$, the authors considered $N$ number of its neighborhoods at radius $\{a_0, \dots, a_{N-1}\}$, where each $a_i := L \times (i+1)/N$ and $L$ is the diameter of the domain $\Xspace$. 
In this case, the approximations of multilevel robustness for all critical points require $n \times N$ classic robustness computations and work well in their applications.




\subsubsection{Enhancing Feature Tracking with Multilevel Robustness}
\label{sec:intrgrateWithFTK} 
The multilevel robustness can be integrated with any existing feature-tracking algorithms to improve the understanding of vector field dynamics.
Yan et al.~\cite{YanUllrichVan-Roekel2022} utilized the minimum multilevel robustness $\minR_{x} := \min_{a \in [0,~L)} R_x(a) $ for their visualization tasks, since $\minR_{x}$ approximates the smallest possible amount of perturbation to the vector field necessary to cancel each critical point. 
The authors 
%\cite{YanUllrichVan-Roekel2022} 
integrated the $\minR_{x}$ with FTK~\cite{GuoLenzXu2021}, a state-of-the-art feature-tracking technique. 
We also utilize FTK in {\tool}.

The initial critical point tracks from FTK suffer from visual clutter when we deal with large-scale datasets. 
\cref{fig:filterBeforeMR} (A) shows the FTK tracking result for the $\EFour$ dataset whose time steps range from 06/01/2004 to 10/31/2004 with a six-hour time gap (see~\cref{sec:evaluation,sec:data} for details). 
Because of visual clutter among thousands of tracks, it is hard for us to identify the dominant features. 
Since the FTK algorithm considers only the correspondences of critical points based on 0-levelset extraction, some important features (\eg,~centers of cyclones) will be included in the same track with other noisy features. 
\cref{fig:filterBeforeMR} (C) shows one of the FTK tracks from \cref{fig:filterBeforeMR} (A). 
This long track contains a Category 3 hurricane, named Jeanne, as highlighted with the blue curve in~\cref{fig:filterBeforeMR} (C). However, it also contains unstable features on the Gulf of Mexico; indicated within the orange box of ~\cref{fig:filterBeforeMR} (C). 

The main idea of enhancing feature tracking with multilevel robustness is to segment and reconnect the initial tracks obtained by FTK considering the minimum multilevel robustness. 
The {\MR} framework can break initial FTK tracks into more meaningful segments with similar robustness values.  
In the example of~\cref{fig:filterBeforeMR} (C), the {\MR} framework can extract the part highlighted with the blue curve from the other part of the track.
This framework can also remove unstable features in the middle of a meaningful track and reconnect remaining parts as a new track after examining spatial faces and spacetime edges~\cite{GuoLenzXu2021} of breakpoints; see~\cite[Sect. 5.1]{YanUllrichVan-Roekel2022} for a concrete example.  


\subsubsection{Pipeline of the Multilevel Robustness Framework}
\label{sec:pipelineMR}
As shown in~\cref{fig:pipeline} (orange arrows and indices), the implementation of {\MR} framework involves the following three steps: 

\para{Step 1:~multilevel robustness calculation.} 
%Multilevel robustness is defined as a sequence of robustness values computed from its neighborhoods with increasing radii. 
The {\MR} framework calculates the multilevel robustness for all detected critical points with evenly increased radii until the neighborhood includes the entire input domain. 
Then, the minimum multilevel robustness is calculation for postprocessing.

\para{Step 2:~integration with feature tracking.}
The {\MR} framework integrates the minimum multilevel robustness with FTK~\cite{GuoLenzXu2021} to enhance the original FTK tracking results.
%The main idea is to add a step to segment/reconnect the original FTK tracks into more meaningful segments, based on the multilevel robustness.

\para{Step 3:~feature selection.}
The {\MR} framework utilizes two filters based on multilevel robustness and degree information of tracks for feature selection. 
These feature selection strategies can help users reduce visual clutter and highlight dominant features in the domain.

{\tool} 
%inherits two key capabilities of the multilevel robustness framework: capture the multiscale nature of the data and enhance critical points tracking results to understand the vector field dynamics. It 
reuses the notion of multilevel robustness, described in~\cref{sec:MRdef}, and the method to integrate the minimum multilevel robustness with FTK; see~\cref{sec:intrgrateWithFTK}. 
In the following section, we customize the {\MR} framework to {\tool} by integrating the physical knowledge of TCs in feature extraction and tracking.




\section{Loss Function and Bounds}
\label{sec:loss-function}

In this section, we introduce a loss function for interleavings on $\R^d$-mapper complexes.  
We give the definition of the loss function (Defn.~\ref{def:Loss_v1}) in Sec.~\ref{ssec:LossFunction}, and present our first version of the bound as Thm.~\ref{thm:bound} in Sec.~\ref{ssec:bound_v1}.
However, this version of the bound requires checking diagrams for all possible open sets $S \in \Open (K)$ which creates a combinatorial explosion that is counterproductive in practice.
Thus, in Sec.~\ref{ssec:BasisBound}, we prove this loss function can be replaced with an improved loss function which only needs to check the open sets for a basis of $\Open(\cU)$. 

%
\subsection{Loss Function Definition}
\label{ssec:LossFunction}
We start by turning each non-empty $F(S)$ (similarly $G(S)$) into a metric space, as follows.
\begin{definition}
Define the distance $d_S^{F}(A,B)$ for $A,B \in F(S)$ to be the smallest $n$ such that $A$ and $B$ represent the same connected component when included into $S^n$. 
Specifically,
\begin{equation*}
    d_S^{F}(A,B) = 
    \min\{ n\geq0 \mid F[S \subset S^n](A) = F[S \subset S^n](B)\}. 
\end{equation*}
If no such $n$ exists, then we  set $d_S^{F}(A,B) = \infty$.
\end{definition}

It is easy to see that this definition satisfies the definition of an extended metric. 
Indeed, it is actually an extended ultrametric since $d_S^F(A,C) \leq \max \{ d_S^F(A,B) , d_S^F(B,C) \}$, although we will not need that additional structure here.

Consider the example of Fig.~\ref{fig:examplegraph-distance} with a single input graph encoded by a cosheaf  $F:\Open(\cU) \to \Set$. 
The set $F(S)$ has two elements, which we denote by $A$ and $B$ as they represent the connected components containing the points $a$ and $b$ respectively. 
Then $d_S^F(A,B) = 1$, since thickening the set $S$ by $1$ puts $a$ and $b$ in the same connected component.
Likewise, denoting the elements of $F(T)$ by $W$ and $Z$, we see that $d_T^F(W,Z) = 2$ since we must expand the set $T$ twice before $w$ and $z$ are in the same connected component. 
% Figure environment removed


As a first useful property of this distance, thickening a set implies that the distance between components will only decrease.
For an example, consider $W,Z \in F(T)$ representing points $w$ and $z$ in Fig.~\ref{fig:examplegraph-distance}.
As noted previously, $d_T^F(W,Z) = 2$. 
However, if the elements ${W',Z' \in F(T^1)}$ represent the connected components in the 1-thickening of $T$, then $d_{T^1}^F(W',Z') = 1$, and in particular, $d_T^F(W,Z) \geq d_{T^1}^F(W',Z')$. 
This idea is formalized in the following lemma:

\begin{lemma}
\label{lem:distanceContraction}
%
Fix $k \geq 0$ and any $A,B \in F(S)$ with images $A' = F[S\subseteq S^k](A)$ and \linebreak $B' = F[S\subseteq S^k](B)$ in $F(S^k)$. 
Then 
\begin{equation*}
    d_{S^k}^F(A',B') = \max\{0,  d_S^F(A,B) - k \} =
    \begin{cases}
        0 & \text{if } k \geq n\\
        d_S^F(A,B) - k & \text{if }0 \leq k <n
    \end{cases}
\end{equation*}
and in particular, $d_S^F(A,B) \geq d_{S^k}^F(A',B')$. 
\end{lemma}
\begin{proof}
Let $n = d_S^F(A,B)$, so that we know the image of $A$ and $B$ in $F(S^n)$ is the same. 
If $k \geq n$, then we use the functor maps $F(S) \to F(S^n) \to F(S^k)$ to see that the images of $A$ and $B$ are the same in $F(S^n)$ so they are the same in $F(S^k)$. 
Then $d_{S^k}^F(A',B') = 0$. 
If $k < n$, then we have the maps $F(S) \to F(S^k) \to F(S^n)$.
Because we know that $A$ and $B$ do not map to the same thing prior to $n$, we have $d_{S^k}^F(A',B') = n-k$, completing the proof.
\end{proof}

We use this framework as follows: first, assume we are given $F$ and $G$ but our attempts at finding an interleaving  do not necessarily satisfy the requirements of a natural transformation. 
Normally, a natural transformation $\eta:H \Rightarrow H'$ is a collection of component morphisms ${\eta:H(S) \to H'(S)}$ which commute with the inclusions: 
\[
\begin{tikzcd}[sep=scriptsize]
	{H(S)} && {H(T)} \\
	\\
	{H'(S)} && {H'(T)}.
	\arrow["{H'[\subseteq]}", from=3-1, to=3-3]
	\arrow["{\eta_u}"', from=1-1, to=3-1]
	\arrow["{H[\subseteq]}", from=1-1, to=1-3]
	\arrow["{\eta_T}", from=1-3, to=3-3]
\end{tikzcd}
\]
The following definitions, inspired by~\cite{Robinson2020} and~\cite{nlab:unnatural_transformation}, give names to collections of component morphisms used to define an interleaving where the square might not commute. 

\begin{definition}
\label{def:assignment}
Given functors $H,H':\Open(\cU) \to Set$, an \emph{unnatural transformation}\footnote{A  natural transformation is an unnatural transformation which just happens to follow commutativity properties. In other words, natural and unnatural transformations are not mutually exclusive. This vocabulary follows from~\cite{nlab:unnatural_transformation} so we accept no responsibility for the linguistic implications.}   $\eta:H \rightarrow H'$ is a collection of maps $\eta_S:H(S) \to H'(S)$ with no additional promise of commutativity. 

For a fixed $n \geq 0$ and cosheaves $F$ and $G$, an \emph{assignment}, or more specifically an \emph{$n$-assignment}, is a pair of unnatural transformations $\phi:F \Rightarrow G^n$ and $\psi:G \Rightarrow F^n$.

\end{definition}

In order to clarify notation, for the remainder of the paper, we will be using $n$-assignments to  build $(n+k)$-interleavings, which by definition will be required to be natural transformations. 
When the $n$-assignment might not commute, we  denote its maps by lower case $\phi$ and $\psi$;  for $(n+k)$-assignments which are constructed to be natural transformations, we  denote them by $\Phi$ and $\Psi$. 

In addition, we assume for the remainder of the paper that $n$ is large enough for an assignment to exist. 
That is, it is possible that for some given $F(S)$, $G(S^n)$ might be empty for $n$ small enough and thus there is no available map from one to the other. 
However, because we have assumed a compact input, $f(\X)$ and $g(\Y)$ is contained in a compact interval, and thus, we have that the $\sigma$ for which  $F(S_\sigma)$  is not empty is contained in some interval (in the poset sense). 
So long as $n$ is large enough that the Hausdorff distance between the images $f(\X)$ and $g(\Y)$ is at most $\delta n$, $G(S^n)$ will be non-empty for any non-empty $F(S)$ (and vice versa).  

In the spirit of  \cite{Robinson2020}, we measure the quality of a choice of an  $n$-assignment $\phi, \psi$ using the collections of distances $\{d_S^F \mid S \in \Open(\cU)\}$ and $\{d_S^G \mid S \in \Open(\cU)\}$.  
First, note that checking that $\phi$ and $\psi$ are natural transformations means ensuring the diagrams
\begin{equation*}
    \begin{tikzcd}
        F(S)  
            \ar[r, "{F[\subseteq ]}"] 
            \ar[dr, "\phi_S"', very near start, violet]
        & F(T)
            \ar[dr, "\phi_T"', very near start, violet]
        & \\
        & G(S^n) 
            \ar[r, "{G[\subseteq ]}"'] 
        & G (T^n)
    \end{tikzcd}
    \begin{tikzcd}
        & F(S^n)
            \ar[r, "{F[\subseteq ]}"] 
        & F (T^n)\\
        G(S)
            \ar[r, "{G[\subseteq ]}"'] 
            \ar[ur, "\psi_S", very near start, orange, crossing over]
        & G(T) 
            \ar[ur, "\psi_T", very near start, orange, crossing over]
        & 
    \end{tikzcd}
\end{equation*}
commute. 
As we use them repeatedly, we will denote these diagrams by $\Parallelograml_\phi(S,T)$ and $\Parallelogramr_\psi(S,T)$, dropping the subscript when it is clear from context.
Then checking whether the pair constitutes an interleaving involves checking commutativity of the diagrams
\begin{equation*}
\label{eq:fourDiagrams}
\begin{tikzcd}
        F(S) 
            \ar[rr, "{F[S \subseteq S^{2n}]}"]   
            \ar[dr, "\phi_S"',violet] 
            & & F(S^{2n}) & 
        & F(S^n) \ar[dr]
            \ar[dr, "\phi_{S^{n}}",violet]
        & \\
        & G(S^n)\ar[ur, "\psi_{S^n}"', orange]  & & 
        G(S) 
            \ar[rr, "{G[S \subseteq S^{2n}]}"']
            \ar[ur, "\psi_{S}", orange]  
        && G(S^{2n})
    \end{tikzcd}
\end{equation*}
which we denote by $\triangled_{\phi,\psi}(S)$ and $\triangleu_{\phi,\psi}(S)$ respectively, again dropping the subscripts when unnecessary. 
We measure quality of the given assignments by checking how far these four diagrams are from commuting in the sense of the distances defined at the terminal vertex of the shape. 

\begin{definition}
\label{def:Loss_v1}
Fix an $n$-assignment
$(\phi,\psi)$. 
We define four \emph{diagram loss functions}: 
\begin{align*}
\Lpl^{S,T}(\phi)
    &= \max\limits_{\alpha \in F(S)} d_{T^n}^{G}(\varphi_S^n \circ F[S \subseteq T](\alpha),
    G[S^n \subseteq T^n] \circ \varphi_S(\alpha))\\
\Lpr^{S,T} (\psi)
    &= \max\limits_{\alpha \in G(S)} d_{T^n}^{F}(
    \psi_S^n \circ G[S \subseteq T](\alpha), 
    F[S^n \subseteq T^n] \circ \psi_S(\alpha)
    )\\
\Ltd^S (\phi,\psi)
    &= \max\limits_{\alpha \in F(S)}  \Big \lceil \tfrac{1}{2} \cdot d_{S^{2n}}^{F}(
    F[S \subseteq S^{2n}] (\alpha),
    \psi_{S^n} \circ \varphi_S(\alpha)
    ) \Big \rceil\\
\Ltu^S (\phi,\psi)
    &= \max\limits_{\alpha \in G(S)}\Big \lceil \tfrac{1}{2} \cdot d_{S^{2n}}^{G}(
    G[S \subseteq S^{2n}](\alpha),
    \varphi_{S^n} \circ \psi_S(\alpha)
    )\Big \rceil.
\end{align*}
Then the loss for the given assignment is defined to be
\begin{equation*}
L(\phi,\psi) = \max_{S\subseteq T}\left\{\Lpl^{S,T}, \Lpr^{S,T}, \Ltu^S, \Ltd^S\right\}.
\end{equation*}
\end{definition}

These loss functions are defined in a way so that while the diagram in question might not commute, pushing $n$ forward by the loss value will send the elements to the same place. 
For example, if   $\Lpl^{S,T}(\phi)  =k$, then in the diagram 
\begin{equation}
\label{eqn:dgm:parallelExtendK}
\begin{tikzcd}
        F(S)  
            \ar[r, "{F[\subseteq ]}"] 
            \ar[dr, "\phi_S"', very near start, violet]
        & F(T)
            \ar[dr, "\phi_T"', very near start, violet]
        & \\
        & G(S^n) 
            \ar[r, "{G[\subseteq ]}"'] 
        & G (T^n) \ar[r] 
        & G(T^{n+k})
\end{tikzcd}
\end{equation}
the image of a point from $F(S)$ is the same in $G(T^{n+k})$ following both paths. 
Similarly, if $\Ltd^S (\phi,\psi)=k$, then in the diagram 
\begin{equation}
\label{eqn:dgm:triExtendK}
    \begin{tikzcd}
        F(S) 
            \ar[rr, "{F[ \subseteq ]}"]   
            \ar[dr, "\phi_S"',violet] 
            & & F(S^{2n})  \ar[r] 
            & F(S^{2(n+k)})
        \\
        & G(S^n)\ar[ur, "\psi_{S^n}"', orange]  & 
    \end{tikzcd}
\end{equation}
the image of a point in $F(S)$ is the same (following both paths) in $F(S^{2(n+k)})$ even if not in $F(S^{2n})$.

\para{An Example:} 
Consider Fig.~\ref{fig:nonzero_loss_finite} and fix $n=1$. 
Denote the connected component of the point $a$ in $F(S)$, $F(S^1)$, and $F(S^2)$ by $A$, $A'$, and $A''$, respectively.
Similarly, the connected component of the point $b$ is denoted by  $B'' \in G(S^{2})$. 
Follow the same form for the connected components of points $w$ and $z$ in $G$.
The interleaving diagrams can be collected together as 
\begin{equation}
\label{eq:interleavingLadder_example}
    \begin{tikzcd}[row sep=large, column sep=huge]
        {\color{blue}\{A\}}  
            \ar[r, "{F[S \subseteq S^1]}"] 
            \ar[dr, "\phi_S"', very near start, violet]
        & {\color{blue}\{A'\}} 
            \ar[r, "{F[S^1 \subseteq S^{2}]}"] 
            \ar[dr, "\phi_{S^n}"', very near start, violet]
        & {\color{blue}\{A'',B''\}} \\
        {\color{red}\{W,Z\}} 
            \ar[r, "{G[S \subseteq S^1]}"'] 
            \ar[ur, "\psi_S", very near start, orange, crossing over]
        & {\color{red}\{W',Z'\}}
            \ar[r, "{G[S^1 \subseteq S^{2}]}"'] 
            \ar[ur, "\psi_{S^1}", very near start, orange, crossing over]
        & {\color{red}\{W'',Z''\}}
    \end{tikzcd}
\end{equation}
noting that the horizontal maps are determined by sending a letter to the same letter with an additional prime. 
The distances between the points in their respective sets are
\begin{equation*}
    \begin{matrix}
    %
    &&& d_{S^2}^F(A'',B'') = 1; \\
    \\
    & d_S^G(W,Z) = 3, & 
    d_{S^1}^G(W',Z')  = 2, & 
    d_{S^2}^G(W'',Z'') = 1. 
    \end{matrix}
\end{equation*}

% Figure environment removed
Consider the following example assignment:
\begin{equation*}
\begin{matrix}
    \phi_S: A \mapsto W',& & 
    \psi_S: W,Z \mapsto A',\\
    \phi_{S^1}:A' \mapsto W'',& &
    \psi_{S^1}: \substack{W' \mapsto A'' \\ Z' \mapsto B'' }.
\end{matrix}
\end{equation*}
In this case, we then have that 
$\Lpl^{S,S^n} = 0$,
$\Lpr^{S,S^n} = 1$, 
$\Ltd^S = 0 $, 
and $\Ltu^S= 1$, 
so again $L(\phi,\psi) \geq 1$.
For this particular example, no $n=1$ interleaving is possible so any choice of assignment will have a non-zero loss (the easiest check is to see that any choice of assignment will force $\Ltu^S =1$). 

%
\subsection{Bounding the Interleaving Distance}
\label{ssec:bound_v1}

We now use the loss function to give an upper bound for the interleaving distance.  
\begin{restatable}{theorem}{FirstLossBound}
\label{thm:bound} 
    For an $n$-assignment,  $\phi\colon F \Rightarrow G^n$ and $\psi\colon G \Rightarrow F^n$, 
    \begin{equation*}
        d_I(F, G) \leq  n+L(\phi, \psi).  
    \end{equation*}
\end{restatable}

To prove this, we require the following technical lemma, proved in Sec.~\ref{sec:technicalProofs}.

\begin{restatable}{lemma}{lossimpliescommutes}
\label{lem:lossimpliescommutes}
Assume we are given an $n$-assignment
$\phi:F \Rightarrow G^n$ and 
$\psi:G \Rightarrow F^n$. 
For a fixed $k$, define $(n+k)$-assignments
$\Phi_S = G[S^n\subseteq S^{n+k}]\circ \phi_S$
and 
$\Psi_S = F[S^n\subseteq S^{n+k}]\circ \psi_S$ for all $S \in \Open(\cU)$. 
Then the following hold:
\begin{enumerate}
    \item $\Lpl^{S,T}(\phi) \leq k$ implies $\Parallelograml_{\Phi}(S,T)$ commutes, and thus $\Lpl^{S,T}(\Phi) = 0$.
    \item $\Lpr^{S,T}(\psi) \leq k$ implies $\Parallelogramr_{\Psi}(S,T)$ commutes, and thus $\Lpr^{S,T}( \Psi) = 0$.
    \item 
    $\Ltd^{S}(\phi,\psi) \leq k$ 
    and
    $\Lpr^{S^n,S^{n+k}}(\psi) \leq k$     imply $\triangled_{\Phi, \Psi}(S)$ commutes, and thus $\Ltd^{S}(\Phi, \Psi) = 0$.
    \item $\Ltu^{S}(\phi,\psi) \leq k$  and
    $\Lpl^{S^n,S^{n+k}}(\phi) \leq k$ 
    imply $\triangleu_{\Phi, \Psi}(S)$ commutes, and thus $\Ltu^{S}(\Phi, \Psi) = 0$.
\end{enumerate}
In particular, if  $\phi$ and $\psi$ have $L(\phi,\psi) = 0$, then $\phi$ and $\psi$ constitute an interleaving, and so $d_I(F,G) \leq n$.
\end{restatable}


\begin{proof}[Proof of Thm.~\ref{thm:bound} ]
Set $k = L(\phi,\psi)$,
so by definition,  $\Lpl^{S,T}(\phi) \leq k$, $\Lpr^{S,T}(\psi) \leq k$, $\Ltd^{S}(\phi,\psi) \leq k$, and $\Ltu^{S}(\phi,\psi) \leq k$. 
As in Lem.~\ref{lem:lossimpliescommutes}, construct two $(n+k)$-assignments: 
$\Phi$ given by 
$\Phi_S = G[S^n \subseteq S^{n+k}] \circ \phi$,  and
$\Psi$ given by 
$\Psi_S = F[S^n \subseteq S^{n+k}] \circ \psi$.
By Lem.~\ref{lem:lossimpliescommutes}, this means the diagrams 
$\Parallelograml_{\Phi}(S,T)$,
$\Parallelogramr_{\Psi}(S,T)$,
$\triangled_{\Phi, \Psi}(S)$, and 
$\triangleu_{\Phi, \Psi}(S)$ 
commute for all pairs $S\subseteq T$. 
This implies that $\Phi$ and $\Psi$ are an $(n+k)$-interleaving, giving the theorem. 
\end{proof}

First, notice that this proof works by explicitly constructing an interleaving from a given $n$-assignment.
Second, we have no reason to believe that this bound is tight.
In particular, in Sec.~\ref{ssec:BasisBound} we improve the  bound by way of restricting the computation to the basis for the topology of $K$ but even that is depending on input quality and gives no guarantee.

We include one additional note about when this loss function can be promised to be finite. 
Define the diameter of a metric space to be the largest distance between points, which we denote by 
$
    \mathrm{diam}(X,d) = \sup \{ d(a,b) \mid a,b \in X\}
$,
and note that here, the $\sup$ can be replaced with a $\max$ since we are working in finite metric spaces.
To simplify statements, we define the diameter of the empty set to be zero.
\begin{lemma}
The loss function for an $n$-assignment $(\phi,\psi)$ is bounded above; specifically,
\begin{align*}
    L(\phi,\psi) \leq 
    \max \Bigg( &
    \left\{\mathrm{diam}(F(S^{k}),d_F^{S^{k}}) \mid S \in \Open(\cU), k \in \{ n, 2n\}\right\} \\
    &\cup \left\{\mathrm{diam}(G(S^{k}),d_G^{S^{k}})\mid S \in \Open(\cU), k \in \{ n, 2n\}\right\}
    \Bigg) .
\end{align*}
In particular, if the inputs come from $f:\X\to\R$ and $g:\Y\to\R$ with both $\X$ and $\Y$ connected, then $L(\phi,\psi)$ is finite.
\end{lemma}

\begin{proof}
The parallelogram portions of the loss function $\Lpl$ and $\Lpr$ take values from distances in $F(S^n)$ and $G(S^n)$. 
The triangle portions $\Ltd$ and $\Ltu$ take values  from distances in $F(S^{2n})$ and $G(S^{2n})$. 
So, the maximum for the loss function must be attained on one of these sets, giving the inequality.
%
For the second statement, if the input graphs each have a single connected component, then any pair of elements $a, b \in F(S)$ map to the same element under the inclusion $F(S) \to F(S^K)$ for a large enough $K$. 
This in turn implies that the diameter of $d_S^F$ is finite for every $S$. 
\end{proof}

% Figure environment removed
Consider the example in Fig.~\ref{fig:infiniteLoss}.
Let $\{A,B\}$, $\{A',B'\}$, and $\{A'',B''\}$ be the representatives of the connected components of the points $a$ and $b$ in $F(S)$, $F(S^1)$ and $F(S^2)$ respectively. 
Because there is no $n$ for which the two points are the same connected component of $X$, the distance between $A$ and $B$ is $\infty$ in all three sets. 
Then no matter the choice of $1$-assignment, $\Ltd = \infty$, making the loss function infinite. 

%
\subsection{Restriction to Basis Elements}
\label{ssec:BasisBound}
We have so far measured the loss function by studying all possible open sets $S$. 
While this is helpful for definitions, it does not make for a reasonable computational setting. 
To that end, we now focus on a basis of the topology, and prove that this basis suffices.

Recall that an open set $S_\sigma\in \Open(\cU)$ (Eqn.~\eqref{eq:S_sigma}) given by the downset of $U_\sigma$ for some $\sigma \in K$ is called a \emph{basic open set}. 
Note that  $\{S_\sigma \mid \sigma \in K \}$ is a basis for the Alexandroff topology. 
We next give a name to the case where we are only given $n$-assignment information for basis elements, or equivalently, if we are given a full assignment but ignore the maps for non-basis open sets.
\begin{definition}
A \emph{basis unnatural transformation} for functors $H$ and $H'$ is a collection of maps $\eta_{S_\sigma}:H(S_\sigma) \to H'(S_\sigma)$ for all basis elements $S_\sigma$ from $\sigma \in K$. 
A \emph{basis $n$-assignment} (or simply a basis assignment) is a pair of basis unnatural transformations
$$
\{\phi_{S_\sigma} :F(S_\sigma) \to G(S^n_\sigma) \mid \sigma \in K\} 
\qquad \text{and}\qquad 
\{\psi_{S_\sigma} :G(S_\sigma) \to F(S^n_\sigma) \mid \sigma \in K\} 
$$
\end{definition}

In this section, we  prove that we can focus our loss function efforts on only those diagrams associated to basic opens, and the solution can be extended to any open set.
\begin{definition}
\label{def:basisLoss}
    The \emph{basis loss function} is defined to be 
\begin{equation*}
L_B(\phi,\psi) = \max_{\sigma \leq \tau}
\left\{
\Lpl^{S_\tau, S_\sigma}, \Lpr^{S_\tau, S_\sigma}, \Ltu^{S_\sigma}, \Ltd^{S_\sigma}
\right\}.
\end{equation*}
\end{definition}
It is immediate from the definitions that $L_B \leq L$ as the $L_B$ maximum is taken over a subset of those used to determine $L$. 
This means, in particular, that if $L=0$ then $L_B = 0$. 
These values are not always equal; for instance, we might have chosen a basis assignment for which every diagram commutes (making $L_B = 0$), but $\phi_T$ defined on non-basis elements causes a non-zero loss function so $L >0$. 
However in the special case where $L_B = 0$, and thus the basis open diagrams are commutative, we do have the ability to extend the information checked to a full interleaving. 
This can be seen in the following lemma, proved in Sec.~\ref{sec:technicalProofs}. 

\begin{restatable}{lemma}{extendToNatTrans}
\label{lem:extendToNatTrans}
Given a basis unnatural transformation
\begin{equation*}
\{\Phi_{S_\sigma}: F(S_\sigma) \to G(S_\sigma^N) \mid \sigma \in K\} 
\end{equation*}
with $\Lpl^{S_\tau, S_\sigma} = 0$ for all $\sigma \leq \tau$, we can extend this to a full natural transformation $\Phi$; i.e.~we can define $\Phi_S$ for all $S$ such that $\Lpl^{S,T} = 0 $ for all $S \subseteq T$. 
\end{restatable}

Note that the symmetric version extending a basis unnatural transformation $\Psi$ to a natural transformation $\Psi: G \Rightarrow F^N$ is obtained in exactly the same way. 
Next, we can take these natural transformations and ensure the triangles commute (thus giving an interleaving) by only checking the basis set triangles, again proved in Sec.~\ref{sec:technicalProofs}.

\begin{restatable}{lemma}{extendTriangles}
\label{lem:extendTriangles}
    Given natural transformations $\Phi:F \Rightarrow G^N$ and $\Psi:G^N \Rightarrow F$ such that $\Ltd^{S_\sigma} = 0$ for all $\sigma \in K$, then $\Ltd^{S} = 0$ for all open sets $S$. 
\end{restatable}



Taken together, we immediately have the following proposition. 
\begin{proposition}
\label{prop:zeros}
Fix a basis $N$-assignment $(\Phi,\Psi)$. 
If $L_B(\Phi,\Psi) = 0$, then $\Phi$ and $\Psi$ can be extended to natural transformations with $L(\Phi,\Psi) = 0$, and thus constitute an interleaving. 
\end{proposition}

Finally, we arrive at our main theorem, where we can use the provided basis $n$-assignment and the calculated loss function to give a bound for the interleaving distance. 

 
\begin{theorem}
\label{thm:secondBound}
Given a basis $n$-assignment  
\begin{equation*}
\phi = \{\phi_{S_\sigma} \mid \sigma \in K\} 
\text{ and } 
\psi = \{\psi_{S_\sigma} \mid \sigma \in K\}, 
\end{equation*}
we have 
\begin{equation*}
    d_I(F,G) \leq n + L_B(\phi,\psi).
\end{equation*}
\end{theorem}

\begin{proof}
This proof proceeds in the same way as that of Thm.~\ref{thm:bound} with some minor modifications of input assumptions. 
First, let $k = L_B(\phi,\psi)$; and 
define a basis $(n+k)$-assignment by
\begin{equation*}
\{\Phi_{S_\sigma} = G[\subseteq] \circ \phi_{S_\sigma} \mid \sigma \in K\}
\qquad \text{ and } \qquad 
\{\Psi_{S_\sigma} = F[\subseteq] \circ \psi_{S_\sigma} \mid \sigma \in K\}. 
\end{equation*}
By Lem.~\ref{lem:lossimpliescommutes}, we know that  $\Lpl^{S_\tau,S_\sigma}(\Phi)  =0$
and
$\Lpr^{S_\tau,S_\sigma}(\Psi)  =0$
for all $\tau \leq \sigma$.
Then by Lem.~\ref{lem:extendToNatTrans}, we can extend $\Phi$ and $\Psi$ to full natural transformations defined for all $S \in \Open(\cU)$. 

To show that $\Phi$ and $\Psi$ constitute an $(n+k)$-interleaving, we must check triangles; i.e.~ensure that $\Ltd^{S}(\Phi, \Psi) = \Ltu^{S}(\Phi, \Psi)= 0$. 
With the goal of using part 3 of Lem.~\ref{lem:lossimpliescommutes}, first note that $\Ltd^{S_\sigma} (\phi,\psi) \leq k$ for basis elements. 
We can see that $\Lpr^{S_\sigma^n, S_\sigma^{n+k}} \leq k$ by using the (non-commutative) diagram 
\begin{equation*}
\begin{tikzcd}
    & F(S_\sigma^{2n}) 
        \ar[rr , "{F[\subseteq]}"] 
    && F(S_\sigma^{2n+k}) 
        \ar[r, "{F[\subseteq]}"]  
    & F\left(S_\sigma^{2(n+k)}\right)\\
    G(S_\sigma^n) 
        \ar[rr, "{G[\subseteq]}"'] 
        \ar[ur, "\psi_{\bullet}"] 
        \ar[urrr, "\Psi_{\bullet}", orange]
    && G(S_\sigma^{n+k}). 
        \ar[ur, "\psi_{\bullet}", very near start] 
        \ar[urr, "\Psi_{\bullet}"', orange]
\end{tikzcd}
\end{equation*}
The leftmost and rightmost triangles commute by definition of $\Psi$, and the orange parallelogram commutes because $\Psi$ is a natural transformation. 
Then chasing any $x \in G(S_\sigma^n)$ up to the top right $F\left(S_\sigma^{2(n+k)}\right)$ results in the same element, giving the required bound on $\Lpr^{S_\sigma^n, S_\sigma^{n+k}} $. 
Using Lem.~\ref{lem:extendTriangles} for $\Phi$ and $\Psi$, $\Ltd^{S} (\Phi,\Psi) =0$ for all open sets $S$. 
The proof that $\Ltu^{S} (\Phi,\Psi) =0$ is similar.
Therefore $\Phi$ and $\Psi$ are an $(n+k)$-interleaving, giving the bound.
\end{proof}

What is surprising about this bound is that despite checking fewer open sets, the loss function for $L_B$ is actually lower than that found using $L$. 
One reason for this is that when we work with the smaller set of input maps, we extend the collection to a ``better'' full assignment, potentially getting rid of some of the causes of a nonzero loss function in the first place. 
For example, a full assignment would be required to provide a map $\phi_S$ for a $S$ with multiple connected components, say $S = T_1 \cup T_2$.
Since no requirements were made of this map based on the $\phi_{T_1}$ and $\phi_{T_2}$ maps, there is a reasonable chance that the loss function contribution from the $\Lpl^{T_1,S}$ is higher than necessary. 
However, in the basis version, we can build the best possible $\phi_T$ given the information over $\phi_{S_1}$ and $\phi_{S_2}$, providing a potentially better, but certainly no worse, bound. 









\section{Computation}
\label{sec:computation}

In this section, we show that given an $n$-assignment $\phi$, $\psi$, we can compute the loss function $L_B(\phi,\psi)$ in polynomial time. For simplicity, we describe the algorithm explicitly in the case where $d=1$ for clarity of examples, before addressing the run time in higher dimensions. 

\subsection{Data structures}
\label{ssec:DataStructures}
In this section, we describe the encoding of the data structures for a pair of input functors $F,G$ and a given $n$-assignment $\phi$ and $\psi$.
We will start with the case $d=1$, and follow the example of Fig.~\ref{fig:DataStructureExample} to illustrate our construction. 
At a high level, we construct graphs for $F$ and $G$, which we denote by $(V_F,E_F)$ and $(V_G, E_G)$.
Then we build data structures to encode the natural transformations $\phi$ and $\psi$.
For clarity, we use $\phitt$ and $\psitt$ to denote the data structures that store information for $\phi$ and $\psi$.
These encode set maps $\phitt:(V_G,E_G) \to (V_H,E_H)$ and $\psitt:(V_H,E_H) \to (V_G,E_G)$, which will map each vertex to a vertex in the other graph and each edge to an edge in the other graph. 



When $d=1$, recall that the discretization of $\R$, $K$, consists of vertices  $\sigma_{-L},\cdots,\sigma_L$ with heights in our bounding box $[-L\delta,L\delta]$, and with edges $\tau_j = (\sigma_j, \sigma_j+1)$. 
Then we construct the graph for $F:\Open(K) \to \Set$ by generating a vertex for every object in every $F(\sigma_i)$ and connect them using the morphisms of the functor.
This results in a vertex set 
$V_F = \coprod_{i =1}^B F(U_{\sigma_i})$,
and an edge for every object in every $F(U_{\tau_i})$, giving edge set 
$E_F = \coprod_{i =1}^{B-1} F(U_{\tau_i})$. 
Note that the endpoints of any edge $e \in E_F$ can be found via the attaching maps: $F[U_{\tau_i} \subseteq U_{\tau_{i-1}}](e)$ and $F[U_{\tau_i} \subseteq U_{\tau_{i+1}}](e)$. 
For example,  $e = (v_4,v_6) \in F(U_{\tau_4})$ in Fig.~\ref{fig:DataStructureExample} has endpoints $v_6 \in F(U_{\sigma_4})$ and $v_4 \in F(U_{\sigma_5})$. 
We store this data in a standard adjacency list.
In addition, each vertex also keeps track of its height, so a vertex $v \in F(U_{\sigma_i})$ will also store the value $i$ as a representation of its height.


% Figure environment removed

Next, we encode the information for an assignment $(\phi,\psi)$ between $F,G: \Open(K) \to \Set$ by constructing the maps $\phitt$ and $\psitt$ using the graphs $(V_F,E_F)$ and $(V_G,E_G)$.
Specifically, for every $v \in V_F$, we  store a vertex $\phitt(v) \in V_G$ with the requirement that if $v \in F(U_{\sigma_i}$ and $\phitt(v) \in G(U_{\sigma_j})$, then $|i-j| \leq n$.
In addition, for every $e \in E_F$, we store an edge $\phitt(e) \in E_G$ again with the requirement that if $e \in F(U_{\tau_i})$  and  $\phitt(e) \in G(U_{\tau_j})$ then $|i-j|\leq n$.
The symmetric situation is setup for $\psitt$. 

To see how these maps arise from input $\phi$ and $\psi$, we start by focusing on the vertex set. 
For this, we need to encode the map
$\phi_{U_{\sigma_i}}:F(U_{\sigma_i}) \to G(U_{\sigma_i}^n)$. 
The elements of $F(U_{\sigma_i})$ are already encoded as vertices, however the elements of $G(U_{\sigma_i}^n)$ are not.
But, because of the cosheaf structure of $G$, the elements of $G(U_{\sigma_i}^n)$ can be seen as the connected components of particular subgraphs. 
Let 
$V_{G,\sigma_i,n} = \{ v \mid v \in G(\sigma_j), j \in [i-n,i+n]\}$ 
and 
$E_{G,\sigma_i,n} = \{e \mid e \in G(\tau_j), j \in [i-n-1,i+n] \}$.  
Then by the properties of colimits, the elements of $G(U_{\sigma_i}^n)$ are the connected components of the subset of the graph $(V_G,E_G)_{\sigma_i,n}:=(V_{G,\sigma_i,n}, E_{G,\sigma_i,n})$. 
Note that because of the endpoints, this is not an induced subgraph; see Fig.~\ref{fig:Assignment} for examples.
Similarly for the edges of $K$, we can define 
$V_{G,\tau_i,n} = \{ v \mid v \in G(\sigma_j), j \in [i-n+1,i+n]$ 
and 
$E_{G,\tau_i,n} = \{e \mid e \in G(\tau_j), j \in [i-n,i+n] \}$ 
so that the connected components of $(V_G,E_G)_{\tau_i,n}:=(V_{G,\tau_i,n}, E_{G,\tau_i,n})$ are the elements of $G(U_{\tau_i}^n)$.

%
%
%
%
%
%
%
%


So, for each $v \in F(U_{\sigma_i})$, we store a vertex $\phitt(v) \in V_{G,\sigma_i,n}$, where $\phitt(v)$ is in the connected component of $(V_{G}, E_G)_{\sigma_i,n}$ represented by $\phi_{U_{\sigma_i}}(v) \in F(U_{\sigma_i}^n)$.
For instance, consider the example of Fig.~\ref{fig:Assignment} where we assume $n=1$. 
If  $\phitt(b) = w$, then $\phi_{U_{\sigma_i}}(b)$ is the connected component that includes $w$ of $(V_G,E_G)_{\sigma_i,1}$ as shown at the right. 
%
We can similarly find the edge map $\phitt(e)$ for $e \in F(U_{\tau_i})$ by setting it to be an edge in $E_{G,\tau_i,n}$ representing the connected component of $\phi_{U_{\tau_i}}(e) \in G(U_{\tau_i}^n)$ in $(V_G,E_G)_{\tau_i,n}$.
So, for example, in Fig.~\ref{fig:Assignment} where $n=1$, the input data might have $\phitt(ab) = (xy) \in E_G$ and $\phitt(bc) = (uv) \in E_G$. 


% Figure environment removed




\subsection{Algorithm and Complexity}
\label{ssec:Complexity}
In this section, we determine the complexity of determining $L_B(\phi,\psi)$ given $\phitt$ and $\psitt$. 
First, we will proceed using a binary search on $k \in [0,\cdots, 2L]$ where the maximum is determined by the diameter of the bounding box. 
So, for a fixed $k$, we will determine if $L_B(\phi,\psi) \leq k$. 
We will focus on 
$\Lpl^{U_\tau, U_\sigma}$ and $\Ltd^{U_\sigma}$ as 
$\Lpr^{U_\tau, U_\sigma}$ and $\Ltu^{U_\sigma}$ are symmetric.

Start with $\Lpl^{U_\tau, U_\sigma}$ and note that in the case where $d=1$, there are two pairs necessary to check for each edge: $\tau_j, \sigma_j$ and $\tau_j,\sigma_{j+1}$.
Fixing $\sigma_\ell$ to be either $\sigma_j$ or $\sigma_{j+1}$, for each edge $e \in F(U_{\tau_j})$, we need to check if the two possible images in $G(U_{\sigma_\ell}^{n+k})$ under the diagram
\begin{equation}
\label{eqn:dgm:parallel_extend_basis}
\begin{tikzcd}
        F(U_{\tau_i})  
            \ar[r, "{F[\subseteq ]}"] 
            \ar[dr, "\phi_{U_{\tau_i}}"',  violet]
        & F(U_{\sigma_\ell})
            \ar[dr, "\phi_{U_{\tau_i}^n}",  violet]
        & & e \ar[r,mapsto] \ar[dr, mapsto]
        & v \ar[dr, mapsto] \\
        & G(U_{\tau_i}^n) 
            \ar[r, "{G[\subseteq ]}"'] 
        & G (U_{\sigma_\ell}^n) \ar[r] 
        & G(U_{\sigma_\ell}^{n+k})
        & \substack{\\{[e']}} \ar[r, mapsto, shift right] 
        & \substack{[w]\\{[e']}} \ar[r, mapsto, shift left] \ar[r, mapsto, shift right] 
        & \substack{[w]\\{[e']}}
\end{tikzcd}
\end{equation}
are the same. 
Note that we use $[-]$ to note that the elements represent the connected component in the relevant sliced graph containing that edge or vertex. 
Following the top, we know that $e$ has a unique endpoint vertex  $v \in F(U_\sigma)$, and that vertex has an image under $\phi_{U_{\tau_i}^n}$ which is a connected component represented by  $\phitt(v) = w \in V_G$. 
Following down, the edge $e$ has an edge image $\phitt(e) = e' \in E_G$. 
So the question becomes: are $e'$ and $w$ in the same connected component of $(V_G,E_G)_{\sigma_\ell,n+k}$? 
This can be done by filtering through the adjacency lists, keeping only vertices and edges in the correct strip, and then checking for connectivity using a standard graph traversal like breadth or depth first search; as we do this once per grid element, we get a total time (when $d=1$) of $O(V_G+E_G)$ time per parallelogram. 
If $d>1$, then the correct ``strip'' for $\sigma_{\overrightarrow{\ell}}$ with indices $\overrightarrow\ell \in \Z^d$ involves checking  
a portion of the graph with indices in a $d$-dimensional box 
$[\ell_1-(n+k), \ell_1+(n+k) ] \times \cdots \times [\ell_d-(n+k), \ell_d+(n+k) ]$
 and hence takes $O(d(V_G + E_G))$ time.

In the example of Fig.~\ref{fig:Assignment}, assume $n=k=1$ and assume our given input $\phitt$ is as noted. 
%
%
Then  for the diagram of \cref{eqn:dgm:parallel_extend_basis} with $\ell = j$ and chasing $bc \in F(U_{\tau_j})$,
this comes down to checking if the connected component of $\phitt(b) = w$ and $\phitt(bc) = xy$ are the same in the portion of $(V_G,E_G)_{\sigma_j,2}$.
In this particular example, there are two connected components in this slice and the images are not in the same component. 
Then we know that $\Lpl^{U_{\tau_j}, U_{\sigma_j}}>k$ so we would immediately move on in our binary search. 
If it were the case that the two images were in the same connected component, then $\Lpl^{U_{\tau_j}, U_{\sigma_j}}\leq k$ and thus we would move on to the next commutative diagram check.


Checking if $\Ltd^{U_\tau} \leq k$ is similar so we briefly highlight the differences.  
First there are two types of basis elements in our case where $d=1$, so we need to check  $\Ltd^{U_{\sigma_i}} \leq k$ (meaning checking vertices) and $\Ltd^{U_{\tau_i}} \leq k$ (meaning checking edges). 
We focus on the case of vertices since the edge version is similar. 
For any vertex element $v \in U_{\sigma_i}$, we need to chase it around the diagram
\begin{equation}
\label{eqn:dgm:tri_extend_basis}
    \begin{tikzcd}
        F(U_{\sigma_i}) 
            \ar[rr, "{F[U_{\sigma_i} \subseteq U_{\sigma_i}^{2n}]}"]   
            \ar[dr, "\phi_{U_{\sigma_i}}"',violet]
            & & F(U_{\sigma_i}^{2n})  \ar[r] 
            & F(U_{\sigma_i}^{2(n+k)}).
        \\
        & G(U_{\sigma_i}^n)
            \ar[ur, "\psi_{U_{\sigma_i}^n}"', orange]  
        & \substack{v\\ \phantom{x}} 
            \ar[rr, mapsto, shift left] \ar[dr, mapsto]
        & & 
        \substack{{[v]}\\{[v']}}
            \ar[r, mapsto, shift left]
            \ar[r, mapsto, shift right]
        & \substack{{[v]}\\{[v']}}\\
        & & & w \ar[ur, mapsto]
    \end{tikzcd}
\end{equation}
If $\texttt{phi}: v \mapsto w$, and $\texttt{psi}: w \mapsto v'$, the question again becomes:  are $v$ and $v'$ in the same connected component of $(V_G,E_G)_{\sigma_j,2(n+k)}$?
So similar to the parallelogram case, we take a strip of the graph and check this connectivity question in $O(d(V_G +E_G))$ time. 
As before, either the elements checked are in the same connected component of the relevant slice of the graph, in which case we move to the next diagram; or it does not, and we move to a different $k$ in our binary search. 
In our example case of Fig.~\ref{fig:Assignment} with $n=k=1$,  we have $2(n+k) = 4$.
Then chasing $b$, we need to check that $b$ and $\psitt \circ \phitt(b) = c$ are in the same connected component of $(V_G,E_G)_{\sigma_j,4}$. 
As this slice has one connected component, this triangle commutes. 
We can check another triangle $\Ltu^{U_{\sigma_j}} \leq k$ chasing $w$. 
In this case, we must check if  $w$ and $\phitt \circ \psitt(w) = z$ are in the same component of $(V_G,E_G)_{\sigma_i,4}$, which again, they both are. 
In either case, if they were not, we would know the loss function is at least $k$ and continue in the binary search.

%

To count the number of diagram checks done, a vertex $v \in F(U_{\sigma_i})$  is checked for one triangle loss $\Ltd^{U_{\sigma_i}}$; and a vertex $w \in G(U_{\sigma_i})$ is checked for one triangle loss $\Ltu^{U_{\sigma_i}}$. 
An edge 
$e \in F(U_{\tau_i})$ 
is checked for one triangle loss 
$\Ltd^{U_{\tau_i}}$ 
and for two parallelograms: 
$\Lpl^{(U_{\tau_i}, U_{\sigma_{i}})}$
and
$\Lpl^{(U_{\tau_i}, U_{\sigma_{i+1}})}$. 
Likewise, an edge $e' \in G(U_{\tau_i})$ is checked in diagrams 
$\Ltu^{U_{\tau_i}}$,
$\Lpr^{(U_{\tau_i}, U_{\sigma_{i}})}$ and
$\Lpr^{(U_{\tau_i}, U_{\sigma_{i+1}})}$. 
This means that if the graph representations of $F$ and $G$ are $(V_F,E_F)$ and $(V_G,E_G)$ respectively, the time for computing the loss function is 
\begin{equation*}
    O\Bigg( [(V_F+V_F) + 3 (E_F + E_G) ] \cdot \max\{(V_F+E_F),(V_G+E_G)\} \Bigg).
\end{equation*}
In $d$-dimensions, a similar construction holds, except that our $\sigma$ cells are now indexed by $B^d$, giving an extra multiplicative factor of $B^d$ in the run time. 
%



%

%
%
%


%
%


%
%
%
%
%
%
%

%
%
%
%
%
%
%
%
    
%


%
%

%

%
%
%
%
%
%
%
%
%
%
%
%
%
%
%
%
%
%
%
%
%
%
%
%

%
%


%

%

%
%
%
%
%
%
%

%

%
\section{Discussion}
\label{sec:discussion}

In this paper, we define a loss function that quantifies how far a diagram is from being commutative, and use such a loss function to bound the interleaving distance, both for mapper and Reeb graph settings.
This work provides a way to evaluate a particular set of maps, which immediately suggests the question of utilizing this quantification to iteratively improve our comparison. 
Here, the quality of the bound is dependent on the quality of the input $n$-assignment, but we assume no control over that input in this paper and so we cannot evaluate the tightness of the bound. 
In the long term, we envision this bound to be used in the context of a gradient descent style framework, where an input $n$-assignment can be improved incrementally thus finding a better bound on the distance. 
Of course, we know that deciding if two Reeb graphs are $\epsilon$-interleaved (for $\epsilon \ge 1$) is NP-hard
\cite{Bjerkevik2018}, so our gradient decent has no guarantee of reaching the global optimal solution. 
However, the potential for not only getting better approximations but also returning the actual  interleaving maps used in the bound is an exciting step toward computing interleaving distances for graph-based signatures available in practice. 
Furthermore, the current approach focuses on 0-dimensional interleavings involving connected components, it is possible to extend our framework in the future to study 1-dimensional interleavings by studying homologous cycles. 

We also believe that our loss function based framework is applicable in a broader context where data are modeled as sheaves or cosheaves in the category of sets, as sheaf theory is emerging as a tool in data science to study, e.g.,   distributed systems~\cite{Malcolm2009,Mansourbeigi2017}, sensor networks~\cite{Robinson2017}, model fit~\cite{KvingeJeffersonJoslyn2021}, and uncertainty quantification~\cite{JoslynCharlesDePerno2020}. 
In particular, one interesting next step is to study how to extend  our framework to work with persistence modules as cosheaves in the category of vector spaces (e.g.,~\cite{BubenikMilicevic2021}). 
As the interleaving distance for multiparameter persistence modules is similarly NP-hard \cite{Bjerkevik2019}, this would be an exciting step toward computational efforts in this broad class of topological signatures.





%
%
\printbibliography
%

\appendix
\input{sections/sec-appendix}
%



\end{document}