%
Data consisting of a graph with a function mapping into $\R^d$ arise in many data applications, encompassing structures such as Reeb graphs, geometric graphs, and knot embeddings. 
As such, the ability to compare and cluster such objects is required in a data analysis pipeline, leading to a need for distances between them.  
In this work, we study the interleaving distance on discretization of these objects, called mapper graphs when $d=1$, where functor representations of the data can be compared by finding pairs of natural transformations between them. 
However, in many cases, computation of the interleaving distance is NP-hard.
For this reason, we take inspiration from recent work by Robinson to find quality measures for families of maps that do not rise to the level of a natural transformation, called assignments. 
We then endow the functor images with the extra structure of a metric space and define a loss function which measures how far an assignment is from making the required diagrams of an interleaving commute. 
Finally we show that the computation of the loss function is polynomial with a given assignment.
We believe this idea is both powerful and translatable, with the potential to provide approximations and bounds on interleavings in a broad array of contexts.
