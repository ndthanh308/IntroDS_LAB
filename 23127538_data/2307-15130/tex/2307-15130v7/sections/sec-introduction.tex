\section{Introduction}
\label{sec:introduction}

Graphs with additional geometric information arise in many contexts in data analysis. 
For instance, a \emph{geometric graph} is generally defined as an abstract graph along with a well-behaved embedding of the graph into $\R^2$, while a graph with a well-behaved map into $\R$ is called a \emph{Reeb graph}.
In particular from the viewpoint of the Reeb graph, these types of input data can arise by studying connected component structures from more general input $\R^d$-spaces, meaning a topological space $\X$ with a function $f:\X \to \R^d$. 
Such graphs are a fundamental object used to model a wide range of data sets, ranging from maps and trajectories \cite{Ahmed2015, Buchin2023}, to commodity networks (e.g.~transportation networks \cite{Jiang2022,Myers2023}) and shape skeletons for object recognition \cite{Ayzenberg2022, Zeng2021}. 
The ability to compare, cluster, and simplify such representative objects is essential in a data analysis pipeline, leading to a need for theoretically motivated and computable distances. 
In this paper, we study a distance for a discretization of the input data, known as mapper \cite{Singh2007}. 
That is, starting from a topological space $\X$ with a function $f:\X \to \R^d$ (resp.~a point cloud $P$ with a function $f: P \to \R^d$), mapper is an encoding of the connected components (resp.~clusters) of $f\inv(U_\alpha)$ for some cover $\cU = \{U_\alpha\}$ of $\R^d$ defined as the nerve of the pullback cover. 
When $d=1$, this results in a graph structure called a \emph{mapper graph} (see Fig.~\ref{fig:geomgraphs}).
Since there are higher dimensional cells, we call the resulting construction for $d>1$ an $\R^d$-mapper complex; however we abuse terminology and generally call this construction a mapper graph whether the dimension is 1 or not.


There has been extensive work on metrics for general graphs, geometric graphs, and Reeb graphs 
(see surveys \cite{Deza2013,Conci2017,Donnat2018,Wills2020}, \cite{Buchin2023}, and \cite{YanMasoodSridharamurthy2021, Bollen2021} respectively). 
In this paper, we will draw inspiration from the interleaving distance;
specifically, we develop a natural extension of the interleaving distance on Reeb graphs~\cite{deSilva2018} to the setting of mapper graphs.   
Interleaving distances arose in the context of generalizing the bottleneck distance for persistence modules \cite{Chazal2009} and were subsequently translated to more general categorical frameworks in \cite{Bubenik2014a, deSilva2018}. 
With the exception of 1-parameter persistence \cite{Lesnick2015}, the interleaving distance is NP-hard in many contexts including multi-parameter persistence \cite{Bjerkevik2018,Bjerkevik2019}, and Reeb graphs~\cite{deSilva2018}.
However, some additional structural information can give better algorithms such as FPT computation for merge trees \cite{FarahbakhshTouli2019}, and  polynomial time for formigrams \cite{Kim2019b} and labeled merge trees \cite{Munch2018,Gasparovic2019}. 
Indeed, the closest work to our approach is work providing bounds for the interleaving distance restricted to merge trees: 
\cite{Curry2022} use the Gromov-Wasserstein distance to find a leaf labeling that gives an upper bound using the easy to compute labeled interleaving distance \cite{Gasparovic2019, Munch2018}; 
while \cite{Pegoraro2021} uses the map formulation of \cite{Morozov2013,FarahbakhshTouli2019} with an integer linear program to provide a bound.
See~\cite{Bjerkevik2018} for a recent summary of interleaving distance complexity results.

When $d=1$, there is already work using the interleaving distance to relate the Reeb graph and its mapper graph \cite{Carriere2017,Carriere2018,Munch2016,Brown2020,botnan2020}. 
We will encode the structure of our more general $\R^d$-mapper complexes in a discretized setting by imposing a grid structure $K$ on $\R^d$. 
Then we can represent the input data $f:X \to \R^d$ as a cosheaf of the form $F:\Open(K) \to \Set$ where we store the path-connected components of inverse images of open sets $\pi_0(f\inv(U))$. 
The idea of the interleaving distance, in this context, is to compare two cosheaves $F,G:\Open(K) \to \Set$ using a pair of natural transformations $\phi:F \Rightarrow G^n$ and $\psi:G \Rightarrow F^n$ mapping into relaxations of the original inputs.   
The complexity of computing this distance then relies on finding the smallest $n$ with available $\phi$ and $\psi$ maps, which in our setting immediately connects to hard underlying problems such as graph isomorphism.

The ideas building this distance are rooted in previous work that study interleavings in related contexts. 
In some sense, we can view this distance as a  discretized cosheaf version of the continuous sheaf version of the interleaving distance that was previously studied~\cite{Curry2014}. 
It can fit into the more general framework of an interleaving distance arising from a category with a flow \cite{deSilva2016}, or as an interleaving distance on generalized persistence modules with a family of translations \cite{Bubenik2014a}, but prior work in these areas focused on theoretical properties and did not address computational aspects, as the more general framework makes such study incredibly difficult. 
Perhaps the closest version of this distance is mentioned as a special case of a general categorical framework~\cite{botnan2020}; however, that setting keeps the thickening of the open sets structure tightly bound to thickening in $\R^d$, whereas we choose to define the distance entirely over the combinatorial structures. 


% Figure environment removed

While all this prior work is powerful in theory, the computational complexity of the construction in more general settings has meant a lack of the use of the interleaving distance in practice.
To circumvent these issues, we take inspiration from recent work of Robinson \cite{Robinson2020} to define quality measures for families of maps that do not rise to the level of a natural transformation, in order to allow for non-optimal maps $\phi$ and $\psi$ in this framework.  
We then apply these quality measures to $\R^d$-mapper complexes, in the hopes of utilizing algorithms from geometry and graph theory to make computation more feasible.

In particular, in \cite{Robinson2020} the object of study is a single input assignment of data of the form $P: \Open(X) \to \Set$ and, with the added structure of a pseudometric for each set $P(U)$, provides a measurement for how far the input data is from having a global section. 
In our work, we instead work with a pair of functors $F,G:\Open(K) \to \Set$ as input, and study collections of maps 
${\phi = \{\phi_U: F(U) \to G(U^n)\mid U\}}$ and $\psi = \{\psi_U:G(U) \to F(U^n)\mid U\}$ which we call an \textit{assignment} when they do not necessarily form a true interleaving.  
We then endow the image with the extra structure of a metric space, so that we have pairs $(F(U),d_U)$ for every open set $U$. 
Using this metric structure, we define a loss function $L(\phi,\psi)$ which measures how far the required diagrams of an interleaving are from commuting, given any input assignment (Thm.~\ref{thm:bound}).  
We modify this bound by only focusing on the loss function computed for a basis of the topology, $L_B(\phi,\psi)$ (Thm.~\ref{thm:secondBound}), which not only improves the computational complexity but also improves the bound. 
Then, we show that the computation of the bound is polynomial, opening up the potential for algorithmic approximation of the interleaving distance. 
Throughout, we show examples encoding the data of a \textit{geometric graph} (i.e.~a graph $G$ with a straight line embedding $f:G \to \R^2$) or a Reeb graph (a graph $G$ with a straight line map to $\R$); see Fig.~\ref{fig:geomgraphs} for an illustration.  However, this kind of input is not a requirement for our framework.
 
We note that this paper is the first step in a larger project. 
That is, the paper here presents a loss function that can be computed given an input $n$-assignment $\phi, \psi$ and results in a bound on the distance by explicitly constructing an $\left(n+L_B(\phi,\psi)\right)$-interleaving. 
As with many garbage-in-garbage-out settings, this bound is only as good as the input, but this study seeks to determine the most general possible bound with no guarantees on the input at all. 
In the followup work \cite{Chambers2025},  we include this loss function as part of an optimization strategy to update a given assignment in order to find better bounds as well as provide further study on how close to optimal is possible. 

\para{Outline.}
In Sec.~\ref{sec:background} we provide the necessary technical background to set up the interleaving distance for  mapper graph inputs. 
In Sec.~\ref{sec:loss-function}, we define the loss functions and bounds. 
We discuss algorithmic requirements of the bound in Sec.~\ref{sec:computation}.
Next, we show how this loss function can be used to similarly bound the Reeb graph interleaving distance by approximating the Reeb graph with a mapper graph in Sec.~\ref{sec:ReebLoss}.
We include all technical proofs in Sec.~\ref{sec:technicalProofs}.
Finally, we discuss broader implications of this work in Sec.~\ref{sec:discussion}. 

%
%
%
%
%
%
%
%
%
%
%
%
%
%
%