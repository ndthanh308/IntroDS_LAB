\section{Loss Function and Bounds}
\label{sec:loss-function}

In this section, we introduce a loss function for interleavings on $\R^d$-mapper complexes.  
We give the definition of the loss function (Defn.~\ref{def:Loss_v1}) in Sec.~\ref{ssec:LossFunction}, and present our first version of the bound as Thm.~\ref{thm:bound} in Sec.~\ref{ssec:bound_v1}.
However, this version of the bound requires checking diagrams for all possible open sets $S \in \Open (K)$ which creates a combinatorial explosion that is counterproductive in practice.
Thus, in Sec.~\ref{ssec:BasisBound}, we prove this loss function can be replaced with an improved loss function which only needs to check the open sets for a basis of $\Open(\cU)$. 

%
\subsection{Loss Function Definition}
\label{ssec:LossFunction}
We start by turning each non-empty $F(S)$ (similarly $G(S)$) into a metric space, as follows.
\begin{definition}
Define the distance $d_S^{F}(A,B)$ for $A,B \in F(S)$ to be the smallest $n$ such that $A$ and $B$ represent the same connected component when included into $S^n$. 
Specifically,
\begin{equation*}
    d_S^{F}(A,B) = 
    \min\{ n\geq0 \mid F[S \subset S^n](A) = F[S \subset S^n](B)\}. 
\end{equation*}
If no such $n$ exists, then we  set $d_S^{F}(A,B) = \infty$.
\end{definition}

It is easy to see that this definition satisfies the definition of an extended metric. 
Indeed, it is actually an extended ultrametric since $d_S^F(A,C) \leq \max \{ d_S^F(A,B) , d_S^F(B,C) \}$, although we will not need that additional structure here.

Consider the example of Fig.~\ref{fig:examplegraph-distance} with a single input graph encoded by a cosheaf  $F:\Open(\cU) \to \Set$. 
The set $F(S)$ has two elements, which we denote by $A$ and $B$ as they represent the connected components containing the points $a$ and $b$ respectively. 
Then $d_S^F(A,B) = 1$, since thickening the set $S$ by $1$ puts $a$ and $b$ in the same connected component.
Likewise, denoting the elements of $F(T)$ by $W$ and $Z$, we see that $d_T^F(W,Z) = 2$ since we must expand the set $T$ twice before $w$ and $z$ are in the same connected component. 
% Figure environment removed


As a first useful property of this distance, thickening a set implies that the distance between components will only decrease.
For an example, consider $W,Z \in F(T)$ representing points $w$ and $z$ in Fig.~\ref{fig:examplegraph-distance}.
As noted previously, $d_T^F(W,Z) = 2$. 
However, if the elements ${W',Z' \in F(T^1)}$ represent the connected components in the 1-thickening of $T$, then $d_{T^1}^F(W',Z') = 1$, and in particular, $d_T^F(W,Z) \geq d_{T^1}^F(W',Z')$. 
This idea is formalized in the following lemma:

\begin{lemma}
\label{lem:distanceContraction}
%
Fix $k \geq 0$ and any $A,B \in F(S)$ with images $A' = F[S\subseteq S^k](A)$ and \linebreak $B' = F[S\subseteq S^k](B)$ in $F(S^k)$. 
Then 
\begin{equation*}
    d_{S^k}^F(A',B') = \max\{0,  d_S^F(A,B) - k \} =
    \begin{cases}
        0 & \text{if } k \geq n\\
        d_S^F(A,B) - k & \text{if }0 \leq k <n
    \end{cases}
\end{equation*}
and in particular, $d_S^F(A,B) \geq d_{S^k}^F(A',B')$. 
\end{lemma}
\begin{proof}
Let $n = d_S^F(A,B)$, so that we know the image of $A$ and $B$ in $F(S^n)$ is the same. 
If $k \geq n$, then we use the functor maps $F(S) \to F(S^n) \to F(S^k)$ to see that the images of $A$ and $B$ are the same in $F(S^n)$ so they are the same in $F(S^k)$. 
Then $d_{S^k}^F(A',B') = 0$. 
If $k < n$, then we have the maps $F(S) \to F(S^k) \to F(S^n)$.
Because we know that $A$ and $B$ do not map to the same thing prior to $n$, we have $d_{S^k}^F(A',B') = n-k$, completing the proof.
\end{proof}

We use this framework as follows: first, assume we are given $F$ and $G$ but our attempts at finding an interleaving  do not necessarily satisfy the requirements of a natural transformation. 
Normally, a natural transformation $\eta:H \Rightarrow H'$ is a collection of component morphisms ${\eta:H(S) \to H'(S)}$ which commute with the inclusions: 
\[
\begin{tikzcd}[sep=scriptsize]
	{H(S)} && {H(T)} \\
	\\
	{H'(S)} && {H'(T)}.
	\arrow["{H'[\subseteq]}", from=3-1, to=3-3]
	\arrow["{\eta_u}"', from=1-1, to=3-1]
	\arrow["{H[\subseteq]}", from=1-1, to=1-3]
	\arrow["{\eta_T}", from=1-3, to=3-3]
\end{tikzcd}
\]
The following definitions, inspired by~\cite{Robinson2020} and~\cite{nlab:unnatural_transformation}, give names to collections of component morphisms used to define an interleaving where the square might not commute. 

\begin{definition}
\label{def:assignment}
Given functors $H,H':\Open(\cU) \to Set$, an \emph{unnatural transformation}\footnote{A  natural transformation is an unnatural transformation which just happens to follow commutativity properties. In other words, natural and unnatural transformations are not mutually exclusive. This vocabulary follows from~\cite{nlab:unnatural_transformation} so we accept no responsibility for the linguistic implications.}   $\eta:H \rightarrow H'$ is a collection of maps $\eta_S:H(S) \to H'(S)$ with no additional promise of commutativity. 

For a fixed $n \geq 0$ and cosheaves $F$ and $G$, an \emph{assignment}, or more specifically an \emph{$n$-assignment}, is a pair of unnatural transformations $\phi:F \Rightarrow G^n$ and $\psi:G \Rightarrow F^n$.

\end{definition}

In order to clarify notation, for the remainder of the paper, we will be using $n$-assignments to  build $(n+k)$-interleavings, which by definition will be required to be natural transformations. 
When the $n$-assignment might not commute, we  denote its maps by lower case $\phi$ and $\psi$;  for $(n+k)$-assignments which are constructed to be natural transformations, we  denote them by $\Phi$ and $\Psi$. 

In addition, we assume for the remainder of the paper that $n$ is large enough for an assignment to exist. 
That is, it is possible that for some given $F(S)$, $G(S^n)$ might be empty for $n$ small enough and thus there is no available map from one to the other. 
However, because we have assumed a compact input, $f(\X)$ and $g(\Y)$ is contained in a compact interval, and thus, we have that the $\sigma$ for which  $F(S_\sigma)$  is not empty is contained in some interval (in the poset sense). 
So long as $n$ is large enough that the Hausdorff distance between the images $f(\X)$ and $g(\Y)$ is at most $\delta n$, $G(S^n)$ will be non-empty for any non-empty $F(S)$ (and vice versa).  

In the spirit of  \cite{Robinson2020}, we measure the quality of a choice of an  $n$-assignment $\phi, \psi$ using the collections of distances $\{d_S^F \mid S \in \Open(\cU)\}$ and $\{d_S^G \mid S \in \Open(\cU)\}$.  
First, note that checking that $\phi$ and $\psi$ are natural transformations means ensuring the diagrams
\begin{equation*}
    \begin{tikzcd}
        F(S)  
            \ar[r, "{F[\subseteq ]}"] 
            \ar[dr, "\phi_S"', very near start, violet]
        & F(T)
            \ar[dr, "\phi_T"', very near start, violet]
        & \\
        & G(S^n) 
            \ar[r, "{G[\subseteq ]}"'] 
        & G (T^n)
    \end{tikzcd}
    \begin{tikzcd}
        & F(S^n)
            \ar[r, "{F[\subseteq ]}"] 
        & F (T^n)\\
        G(S)
            \ar[r, "{G[\subseteq ]}"'] 
            \ar[ur, "\psi_S", very near start, orange, crossing over]
        & G(T) 
            \ar[ur, "\psi_T", very near start, orange, crossing over]
        & 
    \end{tikzcd}
\end{equation*}
commute. 
As we use them repeatedly, we will denote these diagrams by $\Parallelograml_\phi(S,T)$ and $\Parallelogramr_\psi(S,T)$, dropping the subscript when it is clear from context.
Then checking whether the pair constitutes an interleaving involves checking commutativity of the diagrams
\begin{equation*}
\label{eq:fourDiagrams}
\begin{tikzcd}
        F(S) 
            \ar[rr, "{F[S \subseteq S^{2n}]}"]   
            \ar[dr, "\phi_S"',violet] 
            & & F(S^{2n}) & 
        & F(S^n) \ar[dr]
            \ar[dr, "\phi_{S^{n}}",violet]
        & \\
        & G(S^n)\ar[ur, "\psi_{S^n}"', orange]  & & 
        G(S) 
            \ar[rr, "{G[S \subseteq S^{2n}]}"']
            \ar[ur, "\psi_{S}", orange]  
        && G(S^{2n})
    \end{tikzcd}
\end{equation*}
which we denote by $\triangled_{\phi,\psi}(S)$ and $\triangleu_{\phi,\psi}(S)$ respectively, again dropping the subscripts when unnecessary. 
We measure quality of the given assignments by checking how far these four diagrams are from commuting in the sense of the distances defined at the terminal vertex of the shape. 

\begin{definition}
\label{def:Loss_v1}
Fix an $n$-assignment
$(\phi,\psi)$. 
We define four \emph{diagram loss functions}: 
\begin{align*}
\Lpl^{S,T}(\phi)
    &= \max\limits_{\alpha \in F(S)} d_{T^n}^{G}(\varphi_S^n \circ F[S \subseteq T](\alpha),
    G[S^n \subseteq T^n] \circ \varphi_S(\alpha))\\
\Lpr^{S,T} (\psi)
    &= \max\limits_{\alpha \in G(S)} d_{T^n}^{F}(
    \psi_S^n \circ G[S \subseteq T](\alpha), 
    F[S^n \subseteq T^n] \circ \psi_S(\alpha)
    )\\
\Ltd^S (\phi,\psi)
    &= \max\limits_{\alpha \in F(S)}  \Big \lceil \tfrac{1}{2} \cdot d_{S^{2n}}^{F}(
    F[S \subseteq S^{2n}] (\alpha),
    \psi_{S^n} \circ \varphi_S(\alpha)
    ) \Big \rceil\\
\Ltu^S (\phi,\psi)
    &= \max\limits_{\alpha \in G(S)}\Big \lceil \tfrac{1}{2} \cdot d_{S^{2n}}^{G}(
    G[S \subseteq S^{2n}](\alpha),
    \varphi_{S^n} \circ \psi_S(\alpha)
    )\Big \rceil.
\end{align*}
Then the loss for the given assignment is defined to be
\begin{equation*}
L(\phi,\psi) = \max_{S\subseteq T}\left\{\Lpl^{S,T}, \Lpr^{S,T}, \Ltu^S, \Ltd^S\right\}.
\end{equation*}
\end{definition}

These loss functions are defined in a way so that while the diagram in question might not commute, pushing $n$ forward by the loss value will send the elements to the same place. 
For example, if   $\Lpl^{S,T}(\phi)  =k$, then in the diagram 
\begin{equation}
\label{eqn:dgm:parallelExtendK}
\begin{tikzcd}
        F(S)  
            \ar[r, "{F[\subseteq ]}"] 
            \ar[dr, "\phi_S"', very near start, violet]
        & F(T)
            \ar[dr, "\phi_T"', very near start, violet]
        & \\
        & G(S^n) 
            \ar[r, "{G[\subseteq ]}"'] 
        & G (T^n) \ar[r] 
        & G(T^{n+k})
\end{tikzcd}
\end{equation}
the image of a point from $F(S)$ is the same in $G(T^{n+k})$ following both paths. 
Similarly, if $\Ltd^S (\phi,\psi)=k$, then in the diagram 
\begin{equation}
\label{eqn:dgm:triExtendK}
    \begin{tikzcd}
        F(S) 
            \ar[rr, "{F[ \subseteq ]}"]   
            \ar[dr, "\phi_S"',violet] 
            & & F(S^{2n})  \ar[r] 
            & F(S^{2(n+k)})
        \\
        & G(S^n)\ar[ur, "\psi_{S^n}"', orange]  & 
    \end{tikzcd}
\end{equation}
the image of a point in $F(S)$ is the same (following both paths) in $F(S^{2(n+k)})$ even if not in $F(S^{2n})$.

\para{An Example:} 
Consider Fig.~\ref{fig:nonzero_loss_finite} and fix $n=1$. 
Denote the connected component of the point $a$ in $F(S)$, $F(S^1)$, and $F(S^2)$ by $A$, $A'$, and $A''$, respectively.
Similarly, the connected component of the point $b$ is denoted by  $B'' \in G(S^{2})$. 
Follow the same form for the connected components of points $w$ and $z$ in $G$.
The interleaving diagrams can be collected together as 
\begin{equation}
\label{eq:interleavingLadder_example}
    \begin{tikzcd}[row sep=large, column sep=huge]
        {\color{blue}\{A\}}  
            \ar[r, "{F[S \subseteq S^1]}"] 
            \ar[dr, "\phi_S"', very near start, violet]
        & {\color{blue}\{A'\}} 
            \ar[r, "{F[S^1 \subseteq S^{2}]}"] 
            \ar[dr, "\phi_{S^n}"', very near start, violet]
        & {\color{blue}\{A'',B''\}} \\
        {\color{red}\{W,Z\}} 
            \ar[r, "{G[S \subseteq S^1]}"'] 
            \ar[ur, "\psi_S", very near start, orange, crossing over]
        & {\color{red}\{W',Z'\}}
            \ar[r, "{G[S^1 \subseteq S^{2}]}"'] 
            \ar[ur, "\psi_{S^1}", very near start, orange, crossing over]
        & {\color{red}\{W'',Z''\}}
    \end{tikzcd}
\end{equation}
noting that the horizontal maps are determined by sending a letter to the same letter with an additional prime. 
The distances between the points in their respective sets are
\begin{equation*}
    \begin{matrix}
    %
    &&& d_{S^2}^F(A'',B'') = 1; \\
    \\
    & d_S^G(W,Z) = 3, & 
    d_{S^1}^G(W',Z')  = 2, & 
    d_{S^2}^G(W'',Z'') = 1. 
    \end{matrix}
\end{equation*}

% Figure environment removed
Consider the following example assignment:
\begin{equation*}
\begin{matrix}
    \phi_S: A \mapsto W',& & 
    \psi_S: W,Z \mapsto A',\\
    \phi_{S^1}:A' \mapsto W'',& &
    \psi_{S^1}: \substack{W' \mapsto A'' \\ Z' \mapsto B'' }.
\end{matrix}
\end{equation*}
In this case, we then have that 
$\Lpl^{S,S^n} = 0$,
$\Lpr^{S,S^n} = 1$, 
$\Ltd^S = 0 $, 
and $\Ltu^S= 1$, 
so again $L(\phi,\psi) \geq 1$.
For this particular example, no $n=1$ interleaving is possible so any choice of assignment will have a non-zero loss (the easiest check is to see that any choice of assignment will force $\Ltu^S =1$). 

%
\subsection{Bounding the Interleaving Distance}
\label{ssec:bound_v1}

We now use the loss function to give an upper bound for the interleaving distance.  
\begin{restatable}{theorem}{FirstLossBound}
\label{thm:bound} 
    For an $n$-assignment,  $\phi\colon F \Rightarrow G^n$ and $\psi\colon G \Rightarrow F^n$, 
    \begin{equation*}
        d_I(F, G) \leq  n+L(\phi, \psi).  
    \end{equation*}
\end{restatable}

To prove this, we require the following technical lemma, proved in Sec.~\ref{sec:technicalProofs}.

\begin{restatable}{lemma}{lossimpliescommutes}
\label{lem:lossimpliescommutes}
Assume we are given an $n$-assignment
$\phi:F \Rightarrow G^n$ and 
$\psi:G \Rightarrow F^n$. 
For a fixed $k$, define $(n+k)$-assignments
$\Phi_S = G[S^n\subseteq S^{n+k}]\circ \phi_S$
and 
$\Psi_S = F[S^n\subseteq S^{n+k}]\circ \psi_S$ for all $S \in \Open(\cU)$. 
Then the following hold:
\begin{enumerate}
    \item $\Lpl^{S,T}(\phi) \leq k$ implies $\Parallelograml_{\Phi}(S,T)$ commutes, and thus $\Lpl^{S,T}(\Phi) = 0$.
    \item $\Lpr^{S,T}(\psi) \leq k$ implies $\Parallelogramr_{\Psi}(S,T)$ commutes, and thus $\Lpr^{S,T}( \Psi) = 0$.
    \item 
    $\Ltd^{S}(\phi,\psi) \leq k$ 
    and
    $\Lpr^{S^n,S^{n+k}}(\psi) \leq k$     imply $\triangled_{\Phi, \Psi}(S)$ commutes, and thus $\Ltd^{S}(\Phi, \Psi) = 0$.
    \item $\Ltu^{S}(\phi,\psi) \leq k$  and
    $\Lpl^{S^n,S^{n+k}}(\phi) \leq k$ 
    imply $\triangleu_{\Phi, \Psi}(S)$ commutes, and thus $\Ltu^{S}(\Phi, \Psi) = 0$.
\end{enumerate}
In particular, if  $\phi$ and $\psi$ have $L(\phi,\psi) = 0$, then $\phi$ and $\psi$ constitute an interleaving, and so $d_I(F,G) \leq n$.
\end{restatable}


\begin{proof}[Proof of Thm.~\ref{thm:bound} ]
Set $k = L(\phi,\psi)$,
so by definition,  $\Lpl^{S,T}(\phi) \leq k$, $\Lpr^{S,T}(\psi) \leq k$, $\Ltd^{S}(\phi,\psi) \leq k$, and $\Ltu^{S}(\phi,\psi) \leq k$. 
As in Lem.~\ref{lem:lossimpliescommutes}, construct two $(n+k)$-assignments: 
$\Phi$ given by 
$\Phi_S = G[S^n \subseteq S^{n+k}] \circ \phi$,  and
$\Psi$ given by 
$\Psi_S = F[S^n \subseteq S^{n+k}] \circ \psi$.
By Lem.~\ref{lem:lossimpliescommutes}, this means the diagrams 
$\Parallelograml_{\Phi}(S,T)$,
$\Parallelogramr_{\Psi}(S,T)$,
$\triangled_{\Phi, \Psi}(S)$, and 
$\triangleu_{\Phi, \Psi}(S)$ 
commute for all pairs $S\subseteq T$. 
This implies that $\Phi$ and $\Psi$ are an $(n+k)$-interleaving, giving the theorem. 
\end{proof}

First, notice that this proof works by explicitly constructing an interleaving from a given $n$-assignment.
Second, we have no reason to believe that this bound is tight.
In particular, in Sec.~\ref{ssec:BasisBound} we improve the  bound by way of restricting the computation to the basis for the topology of $K$ but even that is depending on input quality and gives no guarantee.

We include one additional note about when this loss function can be promised to be finite. 
Define the diameter of a metric space to be the largest distance between points, which we denote by 
$
    \mathrm{diam}(X,d) = \sup \{ d(a,b) \mid a,b \in X\}
$,
and note that here, the $\sup$ can be replaced with a $\max$ since we are working in finite metric spaces.
To simplify statements, we define the diameter of the empty set to be zero.
\begin{lemma}
The loss function for an $n$-assignment $(\phi,\psi)$ is bounded above; specifically,
\begin{align*}
    L(\phi,\psi) \leq 
    \max \Bigg( &
    \left\{\mathrm{diam}(F(S^{k}),d_F^{S^{k}}) \mid S \in \Open(\cU), k \in \{ n, 2n\}\right\} \\
    &\cup \left\{\mathrm{diam}(G(S^{k}),d_G^{S^{k}})\mid S \in \Open(\cU), k \in \{ n, 2n\}\right\}
    \Bigg) .
\end{align*}
In particular, if the inputs come from $f:\X\to\R$ and $g:\Y\to\R$ with both $\X$ and $\Y$ connected, then $L(\phi,\psi)$ is finite.
\end{lemma}

\begin{proof}
The parallelogram portions of the loss function $\Lpl$ and $\Lpr$ take values from distances in $F(S^n)$ and $G(S^n)$. 
The triangle portions $\Ltd$ and $\Ltu$ take values  from distances in $F(S^{2n})$ and $G(S^{2n})$. 
So, the maximum for the loss function must be attained on one of these sets, giving the inequality.
%
For the second statement, if the input graphs each have a single connected component, then any pair of elements $a, b \in F(S)$ map to the same element under the inclusion $F(S) \to F(S^K)$ for a large enough $K$. 
This in turn implies that the diameter of $d_S^F$ is finite for every $S$. 
\end{proof}

% Figure environment removed
Consider the example in Fig.~\ref{fig:infiniteLoss}.
Let $\{A,B\}$, $\{A',B'\}$, and $\{A'',B''\}$ be the representatives of the connected components of the points $a$ and $b$ in $F(S)$, $F(S^1)$ and $F(S^2)$ respectively. 
Because there is no $n$ for which the two points are the same connected component of $X$, the distance between $A$ and $B$ is $\infty$ in all three sets. 
Then no matter the choice of $1$-assignment, $\Ltd = \infty$, making the loss function infinite. 

%
\subsection{Restriction to Basis Elements}
\label{ssec:BasisBound}
We have so far measured the loss function by studying all possible open sets $S$. 
While this is helpful for definitions, it does not make for a reasonable computational setting. 
To that end, we now focus on a basis of the topology, and prove that this basis suffices.

Recall that an open set $S_\sigma\in \Open(\cU)$ (Eqn.~\eqref{eq:S_sigma}) given by the downset of $U_\sigma$ for some $\sigma \in K$ is called a \emph{basic open set}. 
Note that  $\{S_\sigma \mid \sigma \in K \}$ is a basis for the Alexandroff topology. 
We next give a name to the case where we are only given $n$-assignment information for basis elements, or equivalently, if we are given a full assignment but ignore the maps for non-basis open sets.
\begin{definition}
A \emph{basis unnatural transformation} for functors $H$ and $H'$ is a collection of maps $\eta_{S_\sigma}:H(S_\sigma) \to H'(S_\sigma)$ for all basis elements $S_\sigma$ from $\sigma \in K$. 
A \emph{basis $n$-assignment} (or simply a basis assignment) is a pair of basis unnatural transformations
$$
\{\phi_{S_\sigma} :F(S_\sigma) \to G(S^n_\sigma) \mid \sigma \in K\} 
\qquad \text{and}\qquad 
\{\psi_{S_\sigma} :G(S_\sigma) \to F(S^n_\sigma) \mid \sigma \in K\} 
$$
\end{definition}

In this section, we  prove that we can focus our loss function efforts on only those diagrams associated to basic opens, and the solution can be extended to any open set.
\begin{definition}
\label{def:basisLoss}
    The \emph{basis loss function} is defined to be 
\begin{equation*}
L_B(\phi,\psi) = \max_{\sigma \leq \tau}
\left\{
\Lpl^{S_\tau, S_\sigma}, \Lpr^{S_\tau, S_\sigma}, \Ltu^{S_\sigma}, \Ltd^{S_\sigma}
\right\}.
\end{equation*}
\end{definition}
It is immediate from the definitions that $L_B \leq L$ as the $L_B$ maximum is taken over a subset of those used to determine $L$. 
This means, in particular, that if $L=0$ then $L_B = 0$. 
These values are not always equal; for instance, we might have chosen a basis assignment for which every diagram commutes (making $L_B = 0$), but $\phi_T$ defined on non-basis elements causes a non-zero loss function so $L >0$. 
However in the special case where $L_B = 0$, and thus the basis open diagrams are commutative, we do have the ability to extend the information checked to a full interleaving. 
This can be seen in the following lemma, proved in Sec.~\ref{sec:technicalProofs}. 

\begin{restatable}{lemma}{extendToNatTrans}
\label{lem:extendToNatTrans}
Given a basis unnatural transformation
\begin{equation*}
\{\Phi_{S_\sigma}: F(S_\sigma) \to G(S_\sigma^N) \mid \sigma \in K\} 
\end{equation*}
with $\Lpl^{S_\tau, S_\sigma} = 0$ for all $\sigma \leq \tau$, we can extend this to a full natural transformation $\Phi$; i.e.~we can define $\Phi_S$ for all $S$ such that $\Lpl^{S,T} = 0 $ for all $S \subseteq T$. 
\end{restatable}

Note that the symmetric version extending a basis unnatural transformation $\Psi$ to a natural transformation $\Psi: G \Rightarrow F^N$ is obtained in exactly the same way. 
Next, we can take these natural transformations and ensure the triangles commute (thus giving an interleaving) by only checking the basis set triangles, again proved in Sec.~\ref{sec:technicalProofs}.

\begin{restatable}{lemma}{extendTriangles}
\label{lem:extendTriangles}
    Given natural transformations $\Phi:F \Rightarrow G^N$ and $\Psi:G^N \Rightarrow F$ such that $\Ltd^{S_\sigma} = 0$ for all $\sigma \in K$, then $\Ltd^{S} = 0$ for all open sets $S$. 
\end{restatable}



Taken together, we immediately have the following proposition. 
\begin{proposition}
\label{prop:zeros}
Fix a basis $N$-assignment $(\Phi,\Psi)$. 
If $L_B(\Phi,\Psi) = 0$, then $\Phi$ and $\Psi$ can be extended to natural transformations with $L(\Phi,\Psi) = 0$, and thus constitute an interleaving. 
\end{proposition}

Finally, we arrive at our main theorem, where we can use the provided basis $n$-assignment and the calculated loss function to give a bound for the interleaving distance. 

 
\begin{theorem}
\label{thm:secondBound}
Given a basis $n$-assignment  
\begin{equation*}
\phi = \{\phi_{S_\sigma} \mid \sigma \in K\} 
\text{ and } 
\psi = \{\psi_{S_\sigma} \mid \sigma \in K\}, 
\end{equation*}
we have 
\begin{equation*}
    d_I(F,G) \leq n + L_B(\phi,\psi).
\end{equation*}
\end{theorem}

\begin{proof}
This proof proceeds in the same way as that of Thm.~\ref{thm:bound} with some minor modifications of input assumptions. 
First, let $k = L_B(\phi,\psi)$; and 
define a basis $(n+k)$-assignment by
\begin{equation*}
\{\Phi_{S_\sigma} = G[\subseteq] \circ \phi_{S_\sigma} \mid \sigma \in K\}
\qquad \text{ and } \qquad 
\{\Psi_{S_\sigma} = F[\subseteq] \circ \psi_{S_\sigma} \mid \sigma \in K\}. 
\end{equation*}
By Lem.~\ref{lem:lossimpliescommutes}, we know that  $\Lpl^{S_\tau,S_\sigma}(\Phi)  =0$
and
$\Lpr^{S_\tau,S_\sigma}(\Psi)  =0$
for all $\tau \leq \sigma$.
Then by Lem.~\ref{lem:extendToNatTrans}, we can extend $\Phi$ and $\Psi$ to full natural transformations defined for all $S \in \Open(\cU)$. 

To show that $\Phi$ and $\Psi$ constitute an $(n+k)$-interleaving, we must check triangles; i.e.~ensure that $\Ltd^{S}(\Phi, \Psi) = \Ltu^{S}(\Phi, \Psi)= 0$. 
With the goal of using part 3 of Lem.~\ref{lem:lossimpliescommutes}, first note that $\Ltd^{S_\sigma} (\phi,\psi) \leq k$ for basis elements. 
We can see that $\Lpr^{S_\sigma^n, S_\sigma^{n+k}} \leq k$ by using the (non-commutative) diagram 
\begin{equation*}
\begin{tikzcd}
    & F(S_\sigma^{2n}) 
        \ar[rr , "{F[\subseteq]}"] 
    && F(S_\sigma^{2n+k}) 
        \ar[r, "{F[\subseteq]}"]  
    & F\left(S_\sigma^{2(n+k)}\right)\\
    G(S_\sigma^n) 
        \ar[rr, "{G[\subseteq]}"'] 
        \ar[ur, "\psi_{\bullet}"] 
        \ar[urrr, "\Psi_{\bullet}", orange]
    && G(S_\sigma^{n+k}). 
        \ar[ur, "\psi_{\bullet}", very near start] 
        \ar[urr, "\Psi_{\bullet}"', orange]
\end{tikzcd}
\end{equation*}
The leftmost and rightmost triangles commute by definition of $\Psi$, and the orange parallelogram commutes because $\Psi$ is a natural transformation. 
Then chasing any $x \in G(S_\sigma^n)$ up to the top right $F\left(S_\sigma^{2(n+k)}\right)$ results in the same element, giving the required bound on $\Lpr^{S_\sigma^n, S_\sigma^{n+k}} $. 
Using Lem.~\ref{lem:extendTriangles} for $\Phi$ and $\Psi$, $\Ltd^{S} (\Phi,\Psi) =0$ for all open sets $S$. 
The proof that $\Ltu^{S} (\Phi,\Psi) =0$ is similar.
Therefore $\Phi$ and $\Psi$ are an $(n+k)$-interleaving, giving the bound.
\end{proof}

What is surprising about this bound is that despite checking fewer open sets, the loss function for $L_B$ is actually lower than that found using $L$. 
One reason for this is that when we work with the smaller set of input maps, we extend the collection to a ``better'' full assignment, potentially getting rid of some of the causes of a nonzero loss function in the first place. 
For example, a full assignment would be required to provide a map $\phi_S$ for a $S$ with multiple connected components, say $S = T_1 \cup T_2$.
Since no requirements were made of this map based on the $\phi_{T_1}$ and $\phi_{T_2}$ maps, there is a reasonable chance that the loss function contribution from the $\Lpl^{T_1,S}$ is higher than necessary. 
However, in the basis version, we can build the best possible $\phi_T$ given the information over $\phi_{S_1}$ and $\phi_{S_2}$, providing a potentially better, but certainly no worse, bound. 








