\section{Technical Proofs}
\label{sec:technicalProofs}

In this section, we include the technical proofs from the previous sections. 

%
\subsection{Proofs from Sec.~\ref{sec:background}}

\begin{lemma}
\label{lem:thickeningIsFunctor}
$(-)^n$ is a functor. 
\end{lemma}

\begin{proof}
First, we check that the images of morphisms are well defined, which is to say that if $S \subseteq T$, then $S^{n} \subseteq T^{n}$. 
The statement is clear if $n = 0$, so by induction, we assume that $S^{n-1} \subseteq T^{n-1}$. 
Given an arbitrary $U_\sigma \in S^{n}$, the statement is immediate if $U_\sigma \in S^{n-1} \subseteq S^{n}$, so we assume $U_\sigma \in S^n \setminus S^{n-1}$. 
For this to happen, there must be a $U_\gamma \in S^{n-1}$ and $\tau \in K$ with $\gamma \geq \tau \leq \sigma$ and thus $U_\gamma \subseteq U_\tau \supseteq U_\sigma$. 
But as $U_\gamma \in S^{n-1} \subseteq T^{n-1}$, this sequence also implies that $U_\sigma \in T^n$,  finishing the well-defined check. 

To ensure this is a functor, we need to check that the identity morphism is sent to the identity, and that composition holds. 
For the former, we see that $S \subseteq S$ gets sent to $S^n \subseteq S^n$, and each is an identity. 
The latter is immediate from the property that $\Open(\cU)$ is a poset category, meaning that there is at most one morphism between any pair of objects.  
\end{proof}

One property of this construction that will be useful is as follows. 
For any $U_\sigma \in S^n$, there is a $U_\tau \in S$ and a sequence of cells of $K$
\begin{equation}
\label{eq:length_n_path_cells}
    \tau 
    \geq \gamma_1 \leq \tau_1 
    \geq \gamma_2 \leq \tau_2 
    \geq  \cdots 
    \geq \gamma_n \leq \sigma
\end{equation}
and thus also a sequence of sets in $\cU$
\begin{equation}
\label{eq:length_n_path_sets}
    U_\tau 
    \geq U_{\gamma_1} \leq U_{\tau_1} 
    \geq U_{\gamma_2} \leq U_{\tau_2} 
    \geq  \cdots 
    \geq U_{\gamma_n} \leq U_\sigma.
\end{equation}
Further, given such a sequence with $\tau \in U$, we know that $\sigma \in U^n$. 
Two examples of this can be seen in Fig.~\ref{fig:Length_n_path}, where $\sigma$ and $\sigma'$ from $U^3$ are given, along with a path satisfying Eq.~\eqref{eq:length_n_path_cells}.
Of course, the choice of sequence for Eq.~\eqref{eq:length_n_path_cells} is not unique, so other options are possible. 
% Figure environment removed

Next we show that the distance of Eq.~\eqref{def:interleavingDistance} is indeed a distance 
using the super-linear family of translations framework of \cite{Bubenik2014a}.
This construction can be generalized to the concept of a category with a flow \cite{deSilva2018}, but the added generality is not needed here. 

\begin{definition}[\cite{Bubenik2014a}]
\label{defn:superlinear}
Let $P = (P,\leq)$ be a preordered set. 
A \emph{translation} on $P$ is a functor $\Gamma: P \to P$ along with a natural transformation $\eta:\1_P \Rightarrow \Gamma$. 
A \emph{super-linear family of translations} is a collection $\{\Gamma_\e \}_{\e\geq 0}$  such that 
$\Gamma_\e \Gamma_{\e'}(p) \leq \Gamma_{\e + \e'}(p)$ for all $p \in P$, and $\e, \e' \geq 0$. 
\end{definition}

\composedThickenings*

\begin{proof}
First, we check that $(-)^n$ is indeed a translation using the above terminology. 
In particular, we define $\gamma^n:\1_{\Open(\cU)} \Rightarrow (-)^n$ to have components $\gamma^n_S: S \to S^n$ as simply the inclusion, and we can easily check that this satisfies the naturality requirements. 

Fix $S \in \Open(\cU)$. 
We need to show that $(S^n)^{n'} = S^{n+n'}$. 
Let $U_\sigma \in (S^n)^{n'}$. 
By previous remarks, this is true if and only if there is a sequence in $K$
\begin{equation*}
    \tau 
    \geq \gamma_1 \leq \tau_1 
    \geq \gamma_2 \leq \tau_2 
    \geq  \cdots 
    \geq \gamma_{n'} \leq \sigma
\end{equation*}
with $U_\tau \in S^n$. 
But this property of $\tau$ happens iff there is also a sequence in $K$
\begin{equation*}
    \tau' 
    \geq \gamma_1' \leq \tau_1 '
    \geq \gamma_2' \leq \tau_2' 
    \geq  \cdots 
    \geq \gamma_{n}' \leq \tau
\end{equation*}
with $U_{\tau'} \in S$. 
Concatenating the two sequences gives a sequence of length $(n+n')$ from $U_\tau \in S$ to $U_\sigma$. 
Thus $U_\sigma \in S^{n+n'}$ iff $U_\sigma \in (S^n)^{n'}$, and hence $(S^n)^{n'} = S^{n+n'}$. 
\end{proof}

\begin{theorem}
    The interleaving distance of Defn.~\ref{def:interleavingDistance} is an extended pseudometric. 
\end{theorem}
\begin{proof}
Because Lem.~\eqref{lem:composedthickenings} is a stronger requirement than needed for Defn.~\ref{defn:superlinear}, the collection $\{ ( - )^n\}_{n \geq 0}$ forms a super-linear family of translations. 
Then the result is immediate from \cite[Theorem 3.21]{Bubenik2014a}.
\end{proof}



%
\subsection{Proofs from Sec.~\ref{sec:loss-function}}

To simplify notation, throughout the proofs of this section we often use a $\bullet$ symbol to represent the set indexing a particular map when the subscript would be obvious from the given map. 
For example, we write $\phi_\bullet: F(S^n) \to G(S^{2n})$ rather than writing $\phi_{S^n}$. 


\lossimpliescommutes*

\begin{proof}[Proof of Lem.~\ref{lem:lossimpliescommutes}]
We prove the lemma for the first and third entries only as the other arguments are symmetric. 
Assume $\Lpl^{S,T}(\phi) \leq k$ and 
consider the diagram
\begin{equation}
\label{eq:dgm_cheese_wedge}
\begin{tikzcd}[column sep = 4em]
F(S) 
\ar[r, "{F[\subseteq]}"] 
\ar[dr, "\phi_S",] \ar[ddr, "\Phi_S"', very near start]
& F(T) \ar[dr, "\phi_T", very near start] \ar[ddr, "\Phi_T"',very near start]\\
&    G(S^n) \ar[r, crossing over,  very near start, "{G[\subseteq]}"'] \ar[d, "{G[\subseteq]}"] & G(T^n) \ar[d, "{G[\subseteq]}"] \\
&    G(S^{n+k}) \ar[r,  "{G[\subseteq]}"] & G(T^{n+k}).
\end{tikzcd}
\end{equation}
Note that the top of the diagram (Eq.~\ref{eq:dgm_cheese_wedge}) given by
\begin{equation}
\label{eq:dgm_cheese_top}
\begin{tikzcd}
    F(S) \ar[r, "{F[\subseteq]}"] \ar[d,"\phi_S"]
        & F(T) \ar[d, "\phi_T"]\\
    G(S^n) \ar[r, "{G[\subseteq]}"]  
        & G(T^n)
\end{tikzcd}
\end{equation}
does not necessarily commute in the case that $k\geq 1$, and the bottom of the diagram (Eq.~\ref{eq:dgm_cheese_wedge}) given by 
\begin{equation}
\label{eq:dgm_cheese_bottom}
\begin{tikzcd}
    F(S) \ar[r, "{F[\subseteq]}"] \ar[d,"\Phi_S"]
        & F(T) \ar[d, "\Phi_T"]\\
    G(S^{n+k}) \ar[r, "{G[\subseteq]}"]  
        & G(T^{n+k})
\end{tikzcd}
\end{equation}
is $\Parallelograml_{\Phi}(S,T)$, for which we wish to check for commutativity. 
For any $x \in F(S)$, following around the top square, Eq.~\ref{eq:dgm_cheese_top}, gives 
\begin{equation*}
\begin{tikzcd}
x \ar[r,mapsto] \ar[d,mapsto]
& x' \ar[d, mapsto] \\
\substack{ \\a} 
    \ar[r, mapsto, shift right,end anchor = {[yshift = -0.4ex]}] 
& \substack{ b'\\a'}
\end{tikzcd}
\end{equation*}
with $d_{T^n}^G(a',b') \leq k$.
By definition, the image of $a'$ and $b'$ is the same under the map \linebreak ${G(T^n) \to G(T^{n+k})}$.
Then since the front square of Eq.~\ref{eq:dgm_cheese_wedge} given by 
\begin{equation*}
\begin{tikzcd}
    G(S^n) \ar[r, "{G[\subseteq]}"] \ar[d,"{G[\subseteq]}"]
        & G(T) \ar[d, "{G[\subseteq]}"]\\
    G(S^n) \ar[r, "{G[\subseteq]}"]  
        & G(T^n)
\end{tikzcd}
\end{equation*}
commutes by functoriality of $G$, and the side triangles of Eq.~\ref{eq:dgm_cheese_wedge} given by 
\begin{equation*}
\begin{tikzcd}
    F(S) \ar[r, "\phi_S"] \ar[dr, "\Phi_S"']  & G(S^n) \ar[d, "{G[\subseteq]}"] 
    & F(T) \ar[r, "\phi_T"] \ar[dr, "\Phi_T"'] & G(T^n) \ar[d, "{G[\subseteq]}"] \\
    & G(S^{n+k}) && G(T^{n+k})
\end{tikzcd}
\end{equation*}
commute by definition of $\Phi$, we have that the image of $x$ under either direction of the back square, Eq.~\ref{eq:dgm_cheese_bottom},
commutes, proving claim (1). 

Turning to claim (3), consider the noncommutative diagram 
\begin{equation}
\label{eq:dgm_colors}
\begin{tikzcd}[execute at end picture={
\foreach \Nombre in  {A,B,...,F}
  {\coordinate (\Nombre) at (\Nombre.center);}
\fill[yellow,opacity=0.3] 
  (A) -- (E) -- (F) -- cycle;
\fill[yellow,opacity=0.3] 
  (F) -- (C) -- (D) -- cycle;
\fill[blue,opacity=0.1] 
  (B) -- (C) -- (F) -- (E) -- cycle;
}]
|[alias=A]| F(S) 
    \ar[rrr, "{F[\subseteq]}"] 
    \ar[dr, "\phi_\bullet"']
    \ar[drr, "\Phi_\bullet"]
&&& |[alias=B]| F(S^{2n}) 
    \ar[r, "{F[\subseteq]}"] 
& |[alias=C]| F(S^{2n+k}) 
    \ar[r, "{F[\subseteq]}"] 
& |[alias=D]| F(S^{2(n+k)}).
\\
& |[alias=E]| G(S^n) 
    \ar[r, "{G[\subseteq]}"'] 
    \ar[urr, "\psi_\bullet"]
& |[alias=F]| G(S^{n+k}) 
    \ar[urr, "\psi_\bullet"]
    \ar[urrr, "\Psi_\bullet"']
\end{tikzcd}
\end{equation}
The  two yellow triangles 
\begin{equation*}
\begin{tikzcd}
    F(S) \ar[dr, "\phi_{\bullet}"'] \ar[drr, "\Phi_{\bullet}"]
        &&& F(S^{2n+k}) \ar[r, "{F[\subseteq]}"]  & F(S^2(n+k)\\
    & G(S^n)\ar[r, "{G[\subseteq]}"']  & G(S^{n+k}) \ar[ur, "\psi_{\bullet}"] \ar[urr, "\Psi_{\bullet}"] 
\end{tikzcd}
\end{equation*}
commute by definition of $\Phi$ and $\Psi$. 
The blue parallelogram 
\begin{equation*}
\begin{tikzcd}
    G(S^{n}) \ar[r, "{G[\subseteq]}"] \ar[d, "\psi_\bullet"]  & G(S^{n+k}) \ar[d, "\psi_\bullet"]\\
    F(S^{2n}) \ar[r, "{G[\subseteq]}"] & F(S^{2n+k})
\end{tikzcd}
\end{equation*}
is the diagram $\Parallelogramr_\psi(S^n,S^{n+k})$ which also has loss function bounded by $k$, thus elements of $G(S^n)$ are not necessarily the same in the image of $F(S^{2n+k})$ following the parallelogram, but are the same in $F(S^{2(n+k)})$. 

Checking that $\triangled_{\Phi,\Psi}(S)$ commutes amounts to a diagram chase.
For an arbitrary $\alpha \in F(S)$, consider the following elements
\begin{equation*}
\begin{tikzcd}
\alpha
    \ar[rrr, mapsto, end anchor = {[yshift = 1ex]}] 
    \ar[dr, mapsto]
    \ar[drr, mapsto]
&&& \substack{a\\a'\\}
    \ar[r, mapsto, 
        end anchor = {[yshift = 1.5ex]},
        start anchor = {[yshift = 1ex]}] 
    \ar[r, mapsto, 
        end anchor = {[yshift = -0.2ex]},
        start anchor = {[yshift = -1ex]}] 
& \substack{b\\b'\\b''} \Big\}
    \ar[r, mapsto] 
& c
\\
& x
    \ar[r, mapsto] 
    \ar[urr, mapsto,end anchor = {[yshift = -.5ex]}]
& x'
    \ar[urr,mapsto, 
        end anchor = {[yshift = -1ex]}]
    \ar[urrr, mapsto, bend right = 10, start anchor = {[yshift = -.5ex]}]
\end{tikzcd}
\end{equation*}
 aligning with the diagram of Eq.~\ref{eq:dgm_colors}. 
Both $\alpha$ and $x$ map to $x'$ because of the yellow triangle commuting, and both $b''$ and $x'$ map to the same $c$ for the same reason.
Even if $\alpha$ and $x$ map to different elements in $F(S^{2n})$, they must map to the same element in $F(S^{2(n+k)})$, and this element must be $c$, since both $b'$ and $b''$ map to the same element by the bound on the blue parallelogram. 
As this was done for an arbitrary $\alpha$, we have that $\triangled_{\Phi, \Psi}(S)$ commutes. 

Claims (2) and (4) are similar with appropriate choices of diagrams. 
The final statement is immediate since $L(\phi,\psi) = 0$ implies all diagrams needed for an interleaving commute.
\end{proof}


\extendToNatTrans*
\begin{proof}[Proof of Lem.~\ref{lem:extendToNatTrans}] 
We start by defining $\Phi_S$ for arbitrary open sets. 
Note that since ${\Lpl^{S_\tau, S_\sigma} = 0}$, for any $\sigma \leq \tau$, the diagram of the form 
\begin{equation*}
\begin{tikzcd}
F(S_\tau) \ar[r,"{F[\subseteq]}"] \ar[d] \ar[d, "{\Phi_{S_\tau}}"']
    & F(S_\sigma) \ar[d, "{\Phi_{S_\sigma}}"] \\
G(S_\tau^n) \ar[r,"{G[\subseteq]}"] 
    & G(S_\sigma^n)
\end{tikzcd}
\end{equation*}
commutes.


For an arbitrary open $S$, define 
$\cU_S = \{ S_\sigma \mid U_\sigma \in S\}$. 
%
%
Any nonempty intersection $U_\sigma \cap U_\tau$ is also an element of $\cU$, and so any nonempty intersection of $S_\sigma \cap S_\tau$ is an element of $\cU_S$, so it is a cover of $S$. 
Then we use the fact that $F$ is a cosheaf, and in particular this means that $F(S)$ is the coequalizer of the diagram 
\begin{equation*}
\begin{tikzcd}[column sep = 1in]
    \displaystyle
    \coprod_{\sigma, \sigma'} F(S_\sigma \cap S_{\sigma'}) 
        \ar[r, shift left, "{F[S_\sigma \cap S_{\sigma'}\subseteq S_\sigma]}"] 
        \ar[r, shift right, "{F[S_\sigma \cap S_{\sigma'}\subseteq S_{\sigma'}]}"'] 
    &
    \displaystyle\coprod_{\tau} F(S_\tau).
\end{tikzcd}
\end{equation*} 
Rephrased, this means that for any set $Q$ with maps $F(S_\sigma) \to Q$ such that the  solid arrow diagrams of the form 
\begin{equation*}
\begin{tikzcd}
F(S_\sigma \cap S_{\sigma'}) \ar[d, "{F[\subseteq]}"'] \ar[r,"{F[\subseteq]}"] 
& F(S_\sigma) \ar[d, "{F[\subseteq]}"]  \ar[ddr, bend left] \\
F(S_{\sigma'}) \ar[r, "{F[\subseteq]}"] \ar[drr, bend right] 
& F(S)  \ar[dr, dashed, "\exists!"]\\
 & & Q
\end{tikzcd}
\end{equation*}
commute for any $\sigma, \sigma'$, then there is a unique map $F(S) \to Q$ whose addition still has all diagrams commute. 
In our case, set $Q = G(S^n)$, and define the legs of the cocone to be $G[\subseteq] \circ \Phi_{S_\sigma}$ as seen in the bold purple arrows of the diagram 
\begin{equation}
\label{eqn:cubedgm}
\begin{tikzcd}[row sep=1.5em, column sep = 1.5em]
F(S_{\sigma} \cap S_{\sigma'})
    \arrow[rr, "\Phi_{\bullet}"] \arrow[dr, "{F[\subseteq]}"] 
    \arrow[dd,swap, "{F[\subseteq]}"] 
    &&
G( (S_{\sigma} \cap S_{\sigma'})^n)\arrow[dd, "{G[\subseteq]}"', very near start] \arrow[dr, "{G[\subseteq]}"] \\
& 
F(S_{\sigma'})
    \arrow[rr, crossing over, thick, violet,"\Phi_{\bullet}", near start ] 
&&
G(S_{\sigma'}^n)
    \arrow[dd, thick, violet, "{G[\subseteq]}"] \\
F(S_\sigma) 
    \arrow[rr, thick, violet,"\Phi_{\bullet}", near start] 
    \arrow[dr, "{F[\subseteq]}"] 
&& 
G(S_{\sigma'}^n)\arrow[dr, thick,  violet, "{G[\subseteq]}"] \\
& 
F(S)
    \arrow[rr, dashed, "{\exists! \, \, \Phi_S}"] 
    \arrow[uu, leftarrow, crossing over, "{F[\subseteq]}", very near end]
&& G(S^n).
\end{tikzcd}
\end{equation}
Note that the diagram prior to the inclusion of the dotted line commutes, since we can check the relevant faces as follows. 
The left and right squares commute because $F$ and $G$ are functors. 
The back and top panels commute because they involve only basis opens; equivalently, because we assumed $\Lpl^{S_\sigma \cap S_{\sigma'}, S_\sigma}= \Lpl^{S_\sigma \cap S_{\sigma'}, S_{\sigma'}}  = 0$.
Then, because $F(S)$ is a colimit of the diagram, there exists a unique map $\Phi_S:F(S) \to G(S^n)$ as noted, making any diagram of this form commute. 

To ensure that the resulting $\Phi_S$ maps  make diagrams of the form 
\begin{equation*}
\begin{tikzcd}
    F(S) \ar[r] \ar[d] & G(S^n) \ar[d] \\
    F(T) \ar[r] & G(T^n)
\end{tikzcd}
\end{equation*}
commute for arbitrary $S \subseteq T$, fix such a pair and an $x \in F(S)$. 
Because $F(S)$ is the colimit, there is a $\sigma$ and an $x_\sigma \in F(S_\sigma)$ such that $x_\sigma \mapsto x$. 
In this case we have the diagram 
\begin{equation*}
\begin{tikzcd}
F(S_\sigma) 
\ar[r, "\Phi_\bullet"] 
\ar[dr] \ar[ddr]
& G(S_\sigma)^n \ar[dr] \ar[ddr]\\
&    F(S) \ar[r, crossing over, "\Phi_\bullet", very near start] \ar[d] & G(S^n) \ar[d] \\
&    F(T) \ar[r, "\Phi_\bullet"] & G(T^n)
\end{tikzcd}
\end{equation*}
The top and bottom squares of the wedge commute because they are the front of the cube of the diagram in Eq.~\eqref{eqn:cubedgm}. 
The left and right triangles commute since $F$ and $G$ are functors. 
Thus the front square commutes. 
This means the resulting $\Phi$ is a natural transformation, and thus $\Lpl^{S,T} = 0$. 
\end{proof}


\extendTriangles*
\begin{proof}[Proof of Lem.~\ref{lem:extendTriangles}] 
Because $\Ltd^{S_\sigma} = 0$ for all basis elements, diagrams of the form 
\begin{equation*}
\begin{tikzcd}
    F(S_\sigma) \ar[rr, "{F[\subseteq]}"] \ar[dr, "\Phi_{S_\sigma}"']&& F(S_\sigma^{2n})\\
    & G(S_\sigma^n) \ar[ur, "\Psi_{S_{\sigma}^n}"']
\end{tikzcd}
\end{equation*}
commute for any $\sigma \in K$.
Given an arbitrary open set $S$, let $x \in F(S)$ be given. 
As in the proof of Lem.~\ref{lem:extendToNatTrans}, there is a $\sigma$ and an $x_\sigma \in F(S_\sigma)$ with $x_\sigma \mapsto x$. 
Then consider the diagram 
\begin{equation*}
\begin{tikzcd}
	F(S_\sigma) &&&& F(S_\sigma^{2n}) \\
	&& F(S) &&& {} & F(S^{2n}) \\
	&& G(S_\sigma^n) \\
	&&&& G(S^n).
	\arrow[from=4-5, to=2-7, "\Psi_\bullet"']
	\arrow[from=3-3, to=1-5, "\Psi_\bullet", near end]
	\arrow[from=1-1, to=1-5, "{F[\subseteq]}"]
	\arrow[from=1-1, to=3-3, "\Phi_\bullet"']
	\arrow[from=1-1, to=2-3, "{F[\subseteq]}"]
	\arrow[from=1-5, to=2-7, "{F[\subseteq]}"]
	\arrow[from=3-3, to=4-5, "{G[\subseteq]}"']
	\arrow[from=2-3, to=2-7, crossing over, "{F[\subseteq]}"]
	\arrow[from=2-3, to=4-5, crossing over, "\Phi_\bullet", near end]
\end{tikzcd}
\end{equation*}
The top square commutes because $F$ is a functor. 
The back triangle commutes by this lemma's assumption.
The left and right squares commute because $\Phi$ and $\Psi$ are natural transformations.
Taken together, this means that the front triangle commutes as required.
\end{proof}


\subsection{Proof from Sec.~\ref{sec:ReebLoss}}
\reebvsmapperbound*
\begin{proof}
    Let $\phi,\psi$ be an $n$-interleaving for $F,G:\Open(\cU) \to \Set$. 
    We will construct an $\e=\delta(n+1)$-interleaving $\tphi$, $\tpsi$ for $\tF,\tG:\Int \to \Set$. 

    We start by defining $\tphi : \tF \Rightarrow \tG^\e$ as $\tpsi$ is analogous. 
    Given an arbitrary interval $I= (a,b)$, let $J = (j\delta, k\delta)$ be the smallest grid-aligned interval containing $I$; i.e. ${j\delta \leq a < (j+1)\delta}$ and $(k-1)\delta<b \leq k\delta$.
 
    Note that 
    $I \subseteq J \subseteq J^{\delta n} \subseteq I^{(n+1)\delta} = I^\e$.
    Let $S = \{S_{\tau_i} \mid j \leq i \leq k-1 \} \cup \{ S_{\sigma_i} \mid j < i < k \}$. 
    A quick check shows that ${S \in \Open(\cU)}$, that $J = |S|$, and that $J^{\delta n} = |S^n|$.
    Chasing definitions, this means that ${\tF(J) = \pi_0(f\inv(J))}$ and $F(S) = \pi_0(f\inv(|S|))$ are equal; similarly $\tF(J^{\delta n}) = F(S^n)$. 
    Then define $\tphi_I$ to be the map defined by the composition 
    \begin{equation*}
    \begin{tikzcd}
    \tF(I) \ar[rr,dashed, "\tphi_I"] \ar[d, "{\tF[\subseteq]}"'] 
            && \tG(I^{(n+1)\delta}) \\
        \tF(J)  \ar[d, "="'] \ar[r,dashed, "\tphi_J"]
        & \tG(J^{\delta n}) \ar[ur, "{\tG[\subseteq]}"'] \\ 
        F(S) \ar[r, "\phi_S"'] 
        & G(S^n). \ar[u, "="']
    \end{tikzcd}
    \end{equation*}
    Notice that setting $I$ to be an axis aligned interval $J$ gives the map $\tphi_J$ marked. 

    Now that we have built $\tphi$ and $\tpsi$, we need to check (i) that each is a natural transformation and (ii) that they satisfy the triangle diagrams of Defn.~\ref{def:ReebInterleavingDistance}. 
    For (i) we check only $\tphi$ as, again, $\tpsi$ is symmetric. 
    To this end, assume we have $I \subseteq I'$ with minimal grid-aligned intervals $J$ and $J'$, and let $S,S' \in \Open(\cU)$ be such that $|S|=J$ and $|S'|=J'$.
    Then consider the diagram 
    \begin{equation*}
    \begin{tikzcd}
    \tF(I) \ar[rr, "\tphi_I"] \ar[d, "{\tF[\subseteq]}"']
            && \tG(I^{(n+1)\delta}) 
            \ar[dddr, "{\tG[\subseteq]}"] \\
        \tF(J)  \ar[d, "="'] 
        && \tG(J^{\delta n}) \ar[u, "{\tG[\subseteq]}"] \\ 
        F(S) \ar[rr, "\phi_S"'] 
            \ar[dddr, "{F[\subseteq]}"']
        && G(S^n) \ar[u, "="']
            \ar[dddr, "{G[\subseteq]}"']\\
    &\tF(I') \ar[rr,"\tphi_{I'}", crossing over , near start] \ar[d, "{\tF[\subseteq]}"'] 
        \ar[uuul, leftarrow, crossing over, "{\tF[\subseteq]}"']
            && \tG((I')^{(n+1)\delta}) \\
    &    \tF(J')  \ar[d, "="] 
        && \tG((J')^{\delta n}) \ar[u, "{\tG[\subseteq]}"'] \\ 
    &    F(S') \ar[rr, "\phi_{S'}"'] 
        && G((S')^n). \ar[u, "="']    
    \end{tikzcd}
    \end{equation*}
    Note that the front and back panels of the cube are the diagrams that were used to define $\tphi_I$ and $\tphi_{I'}$, so they commute. 
    The bottom panel commutes because $\phi$ is a natural transformation. 
    The left and right panels commute because $F$ and $\tF$ arise from computing connected components on the same underling input data. 
    Thus, the top square commutes, and this is exactly what is needed to say that $\tphi$ is a natural transformation. 

    To check (ii), fix an interval $I$ with grid aligned $J \subseteq I$ and $S \in \Open(\cU)$ with $|S|=J$. 
    Then consider the diagram 
    \begin{equation*}
    \begin{tikzcd}
    \tF(I) 
        \ar[rr, "\tphi_I"] \ar[d, "{\tF[\subseteq]}"'] 
    && \tG(I^{(n+1)\delta})
        \ar[rr, "\tpsi_{I^\e}"]
        \ar[d]
    && \tF(I^{2(n+1)}\delta)
    \\
    \tF(J)  
        \ar[d, "="'] 
        %
    & \tG(J^{\delta n}) 
        \ar[ur, "{\tG[\subseteq]}"] 
        \ar[r, "{\tG[\subseteq]}"']
    & \tG(J^{(n+1)\delta}))  
        \ar[d, "="'] 
        %
    & \tF(J^{\delta (2n+1)}) 
        \ar[ur, "{\tF[\subseteq]}"'] 
        \\ 
    F(S) \ar[r, "\phi_S"] 
         \ar[drr, "{F[\subseteq]}"']
    & G(S^n)
        \ar[u, "="']
        \ar[r, "{G[\subseteq]}"]
        \ar[dr, "\psi_{S^n}"]
    &G(S^{n+1}) \ar[r, "\psi_{S^{n+1}}"] 
    & F(S^{2n+1}) \ar[u, "="']
    \\
    && F(S^{2n}).
    \ar[ur, "{F[\subseteq]}"']
    \end{tikzcd}
    \end{equation*}
    The left and right hexa-laterals commute by definition of $\tphi$ and $\tpsi$ respectively. 
    The middle top triangle commutes because $\tG$ is a functor, and the middle square commutes because $\tG$ and $G$ are defined as connected components of the same input data. 
    The bottom left triangle commutes because $\phi$ and $\psi$ are an $n$-interleaving. 
    The right quadrilateral commutes because $\psi$ is a natural transformation.
    All this shows that the outside boundary of the diagram commutes. 
    Swapping out the interior, we have 
    \begin{equation*}
    \begin{tikzcd}
    & \tG(I^{(n+1)\delta})
        \ar[dr, "\tpsi_{I^\e}"]\\
    \tF(I) 
        \ar[ur, "\tphi_I"] \ar[d, "{\tF[\subseteq]}"'] 
        \ar[rr, "{\tF[\subseteq]}"]
    & 
    & \tF(I^{2(n+1)}\delta)
    \\
    \tF(J)  
        \ar[d, "="'] 
        \ar[rr, "{\tF[\subseteq]}"]
    & 
    & \tF(J^{\delta (2n+1)}) 
        \ar[u, "{\tF[\subseteq]}"'] 
        \\ 
    F(S) 
         \ar[dr, "{F[\subseteq]}"']
         \ar[rr, "{F[\subseteq]}"]
    &
    & F(S^{2n+1}) \ar[u, "="']
    \\
    & F(S^{2n}).
    \ar[ur, "{F[\subseteq]}"']
    \end{tikzcd}
    \end{equation*}
    The bottom triangle commutes because $F$ is a functor, the next square up commutes by definition of $F$ and $\tF$, and the top square commutes because $\tF$ is a functor. 
    Combining this with the outside ring commuting means that the top triangle commutes, which is the final ingredient needed for the definition of an interleaving. 
\end{proof}