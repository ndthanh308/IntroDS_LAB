\section{Technical Proofs}
\label{sec:technicalProofs}

In this section, we include the technical proofs from the previous sections. 

%
\subsection{Proofs from Sec.~\ref{sec:background}}

\begin{lemma}
\label{lem:thickeningIsFunctor}
$(-)^n$ is a functor. 
\end{lemma}

\begin{proof}
First, we check that the images of morphisms are well defined, which is to say that if $U \subseteq V$, then $U^{n} \subseteq V^{n}$. 
The statement is clear if $n = 0$, so by induction, we assume that $U^{n-1} \subseteq V^{n-1}$. 
Given an arbitrary $\sigma \in U^{n}$, the statement is immediate if $\sigma \in U^{n-1} \subseteq U^{n}$, so we assume $\sigma \in U^n \setminus U^{n-1}$. 
For this to happen, there must be a $\gamma \in U^{n-1}$ and $\tau \in K$ with $\gamma \geq \tau \leq \sigma$. 
But as $\gamma \in U^{n-1} \subseteq V^{n-1}$, this sequence also implies that $\sigma \in V^n$,  finishing the well-defined check. 

To ensure this is a functor, we need to check that the identity morphism is sent to the identity, and that composition holds. 
For the former, we see that $U \subseteq U$ gets sent to $U^n \subseteq U^n$, and each is an identity. 
The latter is immediate from the property that $\Open(K)$ is a poset category, meaning that there is at most one morphism between any pair of objects.  
\end{proof}

One property of this construction that will be useful is as follows. 
For any $\sigma \in U^n$, there is a $\tau \in U$ and a sequence of cells
\begin{equation}
\label{eq:length_n_path}
    \tau 
    \geq \gamma_1 \leq \tau_1 
    \geq \gamma_2 \leq \tau_2 
    \geq  \cdots 
    \geq \gamma_n \leq \sigma.
\end{equation}
Further, given such a sequence with $\tau \in U$, we know that $\sigma \in U^n$. 
Two examples of this can be seen in Fig.~\ref{fig:Length_n_path}, where $\sigma$ and $\sigma'$ from $U^3$ are given, along with a path satisfying Eq.~\eqref{eq:length_n_path}.
Of course, the choice of sequence for Eq.~\eqref{eq:length_n_path} is not unique, so other options are possible. 
% Figure environment removed

Next we show that the distance of Eq.~\eqref{def:interleavingDistance} is indeed a distance 
using the super-linear family of translations framework of \cite{Bubenik2014a}.
This construction can be generalized to the concept of a category with a flow \cite{deSilva2018}, but the added generality is not needed here. 

\begin{definition}[\cite{Bubenik2014a}]
\label{defn:superlinear}
Let $P = (P,\leq)$ be a preordered set. 
A \emph{translation} on $P$ is a functor $\Gamma: P \to P$ along with a natural transformation $\eta:\1_P \Rightarrow \Gamma$. 
A \emph{super-linear family of translations} is a collection $\{\Gamma_\e \}_{\e\geq 0}$  such that 
$\Gamma_\e \Gamma_{\e'}(p) \leq \Gamma_{\e + \e'}(p)$ for all $p \in P$, and $\e, \e' \geq 0$. 
\end{definition}

\composedThickenings*

\begin{proof}
First, we check that $(-)^n$ is indeed a translation using the above terminology. 
In particular, we define $\gamma^n:\1_{\Open(K)} \Rightarrow (-)^n$ to have components $\gamma^n_U: U \to U^n$ as simply the inclusion, and we can easily check that this satisfies the naturality requirements. 

Fix $U \in \Open(K)$. 
We need to show that $(U^n)^{n'} = U^{n+n'}$. 
Let $\sigma \in (U^n)^{n'}$. 
By previous remarks, this is true if and only if there is a sequence 
\begin{equation*}
    \tau 
    \geq \gamma_1 \leq \tau_1 
    \geq \gamma_2 \leq \tau_2 
    \geq  \cdots 
    \geq \gamma_{n'} \leq \sigma
\end{equation*}
with $\tau \in U^n$. 
But this property of $\tau$ happens iff there is also a sequence 
\begin{equation*}
    \tau' 
    \geq \gamma_1' \leq \tau_1 '
    \geq \gamma_2' \leq \tau_2' 
    \geq  \cdots 
    \geq \gamma_{n}' \leq \tau
\end{equation*}
with $\tau' \in U$. 
Concatenating the two sequences gives a sequence of length $(n+n')$ from $\tau \in U$ to $\sigma$. 
Thus $\sigma \in U^{n+n'}$ iff $\sigma \in (U^n)^{n'}$, and hence $(U^n)^{n'} = U^{n+n'}$. 
\end{proof}

\begin{theorem}
    The interleaving distance of Defn.~\ref{def:interleavingDistance} is an extended pseudometric. 
\end{theorem}
\begin{proof}
Because Lem.~\eqref{lem:composedthickenings} is a stronger requirement than needed for Defn.~\ref{defn:superlinear}, the collection $\{ ( - )^n\}_{n \geq 0}$ forms a super-linear family of translations. 
Then the result is immediate from \cite[Theorem 3.21]{Bubenik2014a}.
\end{proof}



%
\subsection{Proofs from Sec.~\ref{sec:loss-function}}

\lossimpliescommutes*

\begin{proof}[Proof of Lem.~\ref{lem:lossimpliescommutes}]
We prove the lemma for the first and third entries only as the other arguments are symmetric. 
Assume $\Lpl^{U,V}(\phi) \leq k$ and 
consider the diagram
\begin{equation*}
\begin{tikzcd}[column sep = 4em]
F(U) 
\ar[r, "{F[\subseteq]}"] 
\ar[dr, "\phi_U",] \ar[ddr, "\Phi_U"', very near start]
& F(V) \ar[dr, "\phi_V", very near start] \ar[ddr, "\Phi_V"',very near start]\\
&    G(U^n) \ar[r, crossing over,  very near start, "{G[\subseteq]}"'] \ar[d, "{G[\subseteq]}"] & G(V^n) \ar[d, "{G[\subseteq]}"] \\
&    G(U^{n+k}) \ar[r,  "{G[\subseteq]}"] & G(V^{n+k}).
\end{tikzcd}
\end{equation*}
Note that the top of the diagram does not necessarily commute in the case that $k\geq 1$, and the bottom of the diagram is $\Parallelograml_{\Phi}(U,V)$, for which we wish to check for commutativity. 
For any $x \in F(U)$, following around the top square gives 
\begin{equation*}
\begin{tikzcd}
x \ar[r,mapsto] \ar[d,mapsto]
& x' \ar[d, mapsto] \\
\substack{ \\a} 
    \ar[r, mapsto, shift right,end anchor = {[yshift = -0.4ex]}] 
& \substack{ b'\\a'}
\end{tikzcd}
\end{equation*}
with $d_{V^n}^G(a',b') \leq k$.
By definition, the image of $a'$ and $b'$ is the same under the map \linebreak ${G(V^n) \to G(V^{n+k})}$.
Then since the front square commutes by functoriality of $G$, and the side triangles commute by definition of $\Phi$, we have that the image of $x$ under either direction of the back square commutes, proving claim (1). 

Turning to claim (3), consider the noncommutative diagram 
\begin{equation*}
\begin{tikzcd}[execute at end picture={
\foreach \Nombre in  {A,B,...,F}
  {\coordinate (\Nombre) at (\Nombre.center);}
\fill[yellow,opacity=0.3] 
  (A) -- (E) -- (F) -- cycle;
\fill[yellow,opacity=0.3] 
  (F) -- (C) -- (D) -- cycle;
\fill[blue,opacity=0.1] 
  (B) -- (C) -- (F) -- (E) -- cycle;
}]
|[alias=A]| F(U) 
    \ar[rrr, "{F[\subseteq]}"] 
    \ar[dr, "\phi_\bullet"']
    \ar[drr, "\Phi_\bullet"]
&&& |[alias=B]| F(U^{2n}) 
    \ar[r, "{F[\subseteq]}"] 
& |[alias=C]| F(U^{2n+k}) 
    \ar[r, "{F[\subseteq]}"] 
& |[alias=D]| F(U^{2(n+k)}).
\\
& |[alias=E]| G(U^n) 
    \ar[r, "{G[\subseteq]}"'] 
    \ar[urr, "\psi_\bullet"]
& |[alias=F]| G(U^{n+k}) 
    \ar[urr, "\psi_\bullet"]
    \ar[urrr, "\Psi_\bullet"']
\end{tikzcd}
\end{equation*}
The  two yellow triangles commute by definition of $\Phi$ and $\Psi$. 
The blue parallelogram is the diagram $\Parallelogramr_\psi(U^n,U^{n+k})$ which also has loss function bounded by $k$, thus elements of $G(U^n)$ are not necessarily the same in the image of $F(U^{2n+k})$ following the parallelogram, but are the same in $F(U^{2(n+k)})$. 

Checking that $\triangled_{\Phi,\Psi}(U)$ commutes amounts to a diagram chase.
For an arbitrary $\alpha \in F(U)$, consider the following elements aligning with the diagram above, 
\begin{equation*}
\begin{tikzcd}
\alpha
    \ar[rrr, mapsto, end anchor = {[yshift = 1ex]}] 
    \ar[dr, mapsto]
    \ar[drr, mapsto]
&&& \substack{a\\a'\\}
    \ar[r, mapsto, 
        end anchor = {[yshift = 1.5ex]},
        start anchor = {[yshift = 1ex]}] 
    \ar[r, mapsto, 
        end anchor = {[yshift = -0.2ex]},
        start anchor = {[yshift = -1ex]}] 
& \substack{b\\b'\\b''} \Big\}
    \ar[r, mapsto] 
& c
\\
& x
    \ar[r, mapsto] 
    \ar[urr, mapsto,end anchor = {[yshift = -.5ex]}]
& x'
    \ar[urr,mapsto, 
        end anchor = {[yshift = -1ex]}]
    \ar[urrr, mapsto, bend right = 10, start anchor = {[yshift = -.5ex]}]
\end{tikzcd}
\end{equation*}
Both $\alpha$ and $x$ map to $x'$ because of the yellow triangle commuting, and both $b''$ and $x'$ map to the same $c$ for the same reason.
Even if $\alpha$ and $x$ map to different elements in $F(U^{2n})$, they must map to the same element in $F(U^{2(n+k)})$, and this element must be $c$, since both $b'$ and $b''$ map to the same element by the bound on the blue parallelogram. 
As this was done for an arbitrary $\alpha$, we have that $\triangled_{\Phi, \Psi}(U)$ commutes. 

Claims (2) and (4) are similar with appropriate choices of diagrams. 
The final statement is immediate since $L(\phi,\psi) = 0$ implies all diagrams needed for an interleaving commute.
\end{proof}


\extendToNatTrans*
\begin{proof}[Proof of Lem.~\ref{lem:extendToNatTrans}] 
We start by defining $\Phi_U$ for arbitrary open sets. 
Note that since ${\Lpl^{U_\tau, U_\sigma} = 0}$, for any $\sigma \leq \tau$, the diagram of the form 
\begin{equation*}
\begin{tikzcd}
F(U_\tau) \ar[r,"{F[\subseteq]}"] \ar[d] \ar[d, "{\Phi_{U_\tau}}"']
    & F(U_\sigma) \ar[d, "{\Phi_{U_\sigma}}"] \\
G(U_\tau^n) \ar[r,"{G[\subseteq]}"] 
    & G(U_\sigma^n)
\end{tikzcd}
\end{equation*}
commutes.

For an arbitrary open $U$, define the cover 
$\cU = \{ U_\sigma \mid \sigma \in U\}$. 
It is straightforward to check that $U = \bigcup_{U_\sigma \in \cU} U_\sigma$ and that any nonempty intersection $U_\sigma \cap U_\tau$ is also an element of $\cU$. 
Then we use the fact that $F$ is a cosheaf, and in particular this means that $F(U)$ is the coequalizer of the diagram 
\begin{equation*}
\begin{tikzcd}[column sep = 1in]
    \displaystyle
    \coprod_{\sigma, \sigma'} F(U_\sigma \cap U_{\sigma'}) 
        \ar[r, shift left, "{F[U_\sigma \cap U_{\sigma'}\subseteq U_\sigma]}"] 
        \ar[r, shift right, "{F[U_\sigma \cap U_{\sigma'}\subseteq U_{\sigma'}]}"'] 
    &
    \displaystyle\coprod_{\tau} F(U_\tau).
\end{tikzcd}
\end{equation*} 
Rephrased, this means that for any set $S$ with maps $F(U_\sigma) \to S$ such that the  solid arrow diagrams of the form 
\begin{equation*}
\begin{tikzcd}
F(U_\sigma \cap U_{\sigma'}) \ar[d, "{F[\subseteq]}"'] \ar[r,"{F[\subseteq]}"] 
& F(U_\sigma) \ar[d, "{F[\subseteq]}"]  \ar[ddr, bend left] \\
F(U_{\sigma'}) \ar[r, "{F[\subseteq]}"] \ar[drr, bend right] 
& F(U)  \ar[dr, dashed, "\exists!"]\\
 & & S
\end{tikzcd}
\end{equation*}
commute for any $\sigma, \sigma'$, then there is a unique map $F(U) \to S$ whose addition still has all diagrams commute. 
In our case, set $S = G(U^n)$, and define the legs of the cocone to be $G[\subseteq] \circ \Phi_{U_\sigma}$ as seen in the bold purple arrows of the diagram 
\begin{equation}
\label{eqn:cubedgm}
\begin{tikzcd}[row sep=1.5em, column sep = 1.5em]
F(U_{\sigma} \cap U_{\sigma'})
    \arrow[rr, "\Phi_{\bullet}"] \arrow[dr, "{F[\subseteq]}"] 
    \arrow[dd,swap, "{F[\subseteq]}"] 
    &&
G( (U_{\sigma} \cap U_{\sigma'})^n)\arrow[dd, "{G[\subseteq]}"', very near start] \arrow[dr, "{G[\subseteq]}"] \\
& 
F(U_{\sigma'})
    \arrow[rr, crossing over, thick, violet,"\Phi_{\bullet}", near start ] 
&&
G(U_{\sigma'}^n)
    \arrow[dd, thick, violet, "{G[\subseteq]}"] \\
F(U_\sigma) 
    \arrow[rr, thick, violet,"\Phi_{\bullet}", near start] 
    \arrow[dr, "{F[\subseteq]}"] 
&& 
G(U_{\sigma'}^n)\arrow[dr, thick,  violet, "{G[\subseteq]}"] \\
& 
F(U)
    \arrow[rr, dashed, "{\exists! \, \, \Phi_U}"] 
    \arrow[uu, leftarrow, crossing over, "{F[\subseteq]}", very near end]
&& G(U^n)
\end{tikzcd}
\end{equation}
where $\Phi_{\bullet}$ means $\Phi_V$ for the appropriate set $V$, but is dropped to simplify the notation.
Note that the diagram prior to the inclusion of the dotted line commutes, since we can check the relevant faces as follows. 
The left and right squares commute because $F$ and $G$ are functors. 
The back and top panels commute because they involve only basis opens; equivalently, because we assumed $\Lpl^{U_\sigma \cap U_{\sigma'}, U_\sigma}= \Lpl^{U_\sigma \cap U_{\sigma'}, U_{\sigma'}}  = 0$.
Then, because $F(U)$ is a colimit of the diagram, there exists a unique map $\Phi_U:F(U) \to G(U)$ as noted, making any diagram of this form commute. 

To ensure that the resulting $\Phi_U$ maps  make diagrams of the form 
\begin{equation*}
\begin{tikzcd}
    F(U) \ar[r] \ar[d] & G(U^n) \ar[d] \\
    F(V) \ar[r] & G(V^n)
\end{tikzcd}
\end{equation*}
commute for arbitrary $U \subseteq V$, fix such a pair and an $x \in F(U)$. 
Because $F(U)$ is the colimit, there is a $\sigma$ and an $x_\sigma \in F(U_\sigma)$ such that $x_\sigma \mapsto x$. 
In this case we have the diagram 
\begin{equation*}
\begin{tikzcd}
F(U_\sigma) 
\ar[r, "\Phi_\bullet"] 
\ar[dr] \ar[ddr]
& G(U_\sigma)^n \ar[dr] \ar[ddr]\\
&    F(U) \ar[r, crossing over, "\Phi_\bullet", very near start] \ar[d] & G(U^n) \ar[d] \\
&    F(V) \ar[r, "\Phi_\bullet"] & G(V^n)
\end{tikzcd}
\end{equation*}
The top and back squares commute because they are the front of the cube of the diagram in Eq.~\eqref{eqn:cubedgm}. 
The left and right triangles commute since $F$ and $G$ are functors. 
Thus the front square, and in particular the element $x \in F(U)$, commute. 
This means the resulting $\Phi$ is a natural transformation, and thus $\Lpl^{U,V} = 0$. 
\end{proof}


\extendTriangles*
\begin{proof}[Proof of Lem.~\ref{lem:extendTriangles}] 
Because $\Ltd^{U_\sigma} = 0$ for all basis elements, diagrams of the form 
\begin{equation*}
\begin{tikzcd}
    F(U_\sigma) \ar[rr, "{F[\subseteq]}"] \ar[dr, "\Phi_{U_\sigma}"']&& F(U_\sigma^{2n})\\
    & G(U_\sigma^n) \ar[ur, "\Psi_{U_{\sigma}^n}"']
\end{tikzcd}
\end{equation*}
commute for any $\sigma \in K$.
Given an arbitrary open set $U$, let $x \in F(U)$ be given. 
As in the proof of Lem.~\ref{lem:extendToNatTrans}, there is a $\sigma$ and an $x_\sigma \in F(U_\sigma)$ with $x_\sigma \mapsto x$. 
Then consider the diagram 
\begin{equation*}
\begin{tikzcd}
	F(U_\sigma) &&&& F(U_\sigma^{2n}) \\
	&& F(U) &&& {} & F(U^{2n}) \\
	&& G(U_\sigma^n) \\
	&&&& G(U^n)
	\arrow[from=4-5, to=2-7, "\Psi_\bullet"']
	\arrow[from=3-3, to=1-5, "\Psi_\bullet", near end]
	\arrow[from=1-1, to=1-5, "{F[\subseteq]}"]
	\arrow[from=1-1, to=3-3, "\Phi_\bullet"']
	\arrow[from=1-1, to=2-3, "{F[\subseteq]}"]
	\arrow[from=1-5, to=2-7, "{F[\subseteq]}"]
	\arrow[from=3-3, to=4-5, "{G[\subseteq]}"']
	\arrow[from=2-3, to=2-7, crossing over, "{F[\subseteq]}"]
	\arrow[from=2-3, to=4-5, crossing over, "\Phi_\bullet", near end]
\end{tikzcd}
\end{equation*}
where we again replace map subscripts with $\bullet$ to simplify notation.
The top square commutes because $F$ is a functor. 
The back triangle commutes by this lemma's assumption.
The left and right squares commute because $\Phi$ and $\Psi$ are natural transformations.
Taken together, this means that the diagram commutes and in particular, the image of $x \in F(U)$ chased around the front triangle commutes.
\end{proof}


\subsection{Proof from Sec.~\ref{sec:ReebLoss}}
\reebvsmapperbound*
\begin{proof}
    Let $\phi,\psi$ be an $n$-interleaving for $F,G:\Open(K) \to \Set$. 
    We will construct an $\e=\delta(n+1)$-interleaving $\tphi$, $\tpsi$ for $\tF,\tG:\Int \to \Set$. 

    We start by defining $\tphi : \tF \Rightarrow \tG^\e$ as $\tpsi$ is analogous. 
    Given an arbitrary interval $I= (a,b)$, let $J = (j\delta, k\delta)$ be the smallest grid-aligned interval containing $I$; i.e. ${j\delta \leq a < (j+1)\delta}$ and $(k-1\delta)<b \leq k\delta$.
 
    Note that 
    $I \subseteq J \subseteq J^{\delta n} \subseteq I^{(n+1)\delta} = I^\e$.
    Let $S = \{\tau_i \mid j \leq i \leq k-1 \} \cup \{ \sigma_i \mid j < i < k \}$. 
    A quick check shows that ${S \in \Open(K)}$, that $J = |S|$, and that $J^{\delta n} = |S^n|$.
    Chasing definitions, this means that ${\tF(J) = \pi_0(f\inv(J))}$ and $F(S) = \pi_0(f\inv(|S|))$ are equal; similarly $\tF(J^{\delta n}) = F(S^n)$. 
    Then define $\tphi_I$ to be the map defined by the composition 
    \begin{equation*}
    \begin{tikzcd}
    \tF(I) \ar[rr,dashed, "\tphi_I"] \ar[d, "{\tF[\subseteq]}"'] 
            && \tG(I^{(n+1)\delta}) \\
        \tF(J)  \ar[d, "="'] \ar[r,dashed, "\tphi_J"]
        & \tG(J^{\delta n}) \ar[ur, "{\tG[\subseteq]}"'] \\ 
        F(S) \ar[r, "\phi_S"'] 
        & G(S^n). \ar[u, "="']
    \end{tikzcd}
    \end{equation*}
    Notice that setting $I$ to be an axis aligned interval $J$ gives the map $\tphi_J$ marked. 

    Now that we have built $\tphi$ and $\tpsi$, we need to check (i) that each is a natural transformation and (ii) that they satisfy the triangle diagrams of Defn.~\ref{def:ReebInterleavingDistance}. 
    For (i) we check only $\tphi$ as, again, $\tpsi$ is symmetric. 
    To this end, assume we have $I \subseteq I'$ with minimal grid-aligned intervals $J$ and $J'$, and let $S,S' \in \Open(K)$ be such that $|S|=J$ and $|S'|=J'$.
    Then consider the diagram 
    \begin{equation*}
    \begin{tikzcd}
    \tF(I) \ar[rr, "\tphi_I"] \ar[d, "{\tF[\subseteq]}"']
            && \tG(I^{(n+1)\delta}) 
            \ar[dddr, "{\tG[\subseteq]}"] \\
        \tF(J)  \ar[d, "="'] 
        && \tG(J^{\delta n}) \ar[u, "{\tG[\subseteq]}"] \\ 
        F(S) \ar[rr, "\phi_S"'] 
            \ar[dddr, "{F[\subseteq]}"']
        && G(S^n) \ar[u, "="']
            \ar[dddr, "{G[\subseteq]}"']\\
    &\tF(I') \ar[rr,"\tphi_{I'}", crossing over , near start] \ar[d, "{\tF[\subseteq]}"'] 
        \ar[uuul, leftarrow, crossing over, "{\tF[\subseteq]}"']
            && \tG((I')^{(n+1)\delta}) \\
    &    \tF(J')  \ar[d, "="] 
        && \tG((J')^{\delta n}) \ar[u, "{\tG[\subseteq]}"'] \\ 
    &    F(S') \ar[rr, "\phi_{S'}"'] 
        && G((S')^n). \ar[u, "="']    
    \end{tikzcd}
    \end{equation*}
    Note that the front and back panels of the cube are the diagrams that were used to define $\tphi_I$ and $\tphi_{I'}$, so they commute. 
    The bottom panel commutes because $\phi$ is a natural transformation. 
    The left and right panels commute because $F$ and $\tF$ arise from computing connected components on the same underling input data. 
    Thus, the top square commutes, and this is exactly what is needed to say that $\tphi$ is a natural transformation. 

    To check (ii), fix an interval $I$ with grid aligned $J \subseteq I$ and $S \in \Open(K)$ with $|S|=J$. 
    Then consider the diagram 
    \begin{equation*}
    \begin{tikzcd}
    \tF(I) 
        \ar[rr, "\tphi_I"] \ar[d, "{\tF[\subseteq]}"'] 
    && \tG(I^{(n+1)\delta})
        \ar[rr, "\tpsi_{I^\e}"]
        \ar[d]
    && \tF(I^{2(n+1)}\delta)
    \\
    \tF(J)  
        \ar[d, "="'] 
        %
    & \tG(J^{\delta n}) 
        \ar[ur, "{\tG[\subseteq]}"] 
        \ar[r, "{\tG[\subseteq]}"']
    & \tG(J^{(n+1)\delta}))  
        \ar[d, "="'] 
        %
    & \tF(J^{\delta (2n+1)}) 
        \ar[ur, "{\tF[\subseteq]}"'] 
        \\ 
    F(S) \ar[r, "\phi_S"] 
         \ar[drr, "{F[\subseteq]}"']
    & G(S^n)
        \ar[u, "="']
        \ar[r, "{G[\subseteq]}"]
        \ar[dr, "\psi_{S^n}"]
    &G(S^{n+1}) \ar[r, "\psi_{S^{n+1}}"] 
    & F(S^{2n+1}) \ar[u, "="']
    \\
    && F(S^{2n}).
    \ar[ur, "{F[\subseteq]}"']
    \end{tikzcd}
    \end{equation*}
    The left and right hexa-laterals commute by definition of $\tphi$ and $\tpsi$ respectively. 
    The middle top triangle commutes because $\tG$ is a functor, and the middle square commutes because $\tG$ and $G$ are defined as connected components of the same input data. 
    The bottom left triangle commutes because $\phi$ and $\psi$ are an $n$-interleaving. 
    The right quadrilateral commutes because $\psi$ is a natural transformation.
    All this shows that the outside boundary of the diagram commutes. 
    Swapping out the interior, we have 
    \begin{equation*}
    \begin{tikzcd}
    & \tG(I^{(n+1)\delta})
        \ar[dr, "\tpsi_{I^\e}"]\\
    \tF(I) 
        \ar[ur, "\tphi_I"] \ar[d, "{\tF[\subseteq]}"'] 
        \ar[rr, "{\tF[\subseteq]}"]
    & 
    & \tF(I^{2(n+1)}\delta)
    \\
    \tF(J)  
        \ar[d, "="'] 
        \ar[rr, "{\tF[\subseteq]}"]
    & 
    & \tF(J^{\delta (2n+1)}) 
        \ar[u, "{\tF[\subseteq]}"'] 
        \\ 
    F(S) 
         \ar[dr, "{F[\subseteq]}"']
         \ar[rr, "{F[\subseteq]}"]
    &
    & F(S^{2n+1}) \ar[u, "="']
    \\
    & F(S^{2n}).
    \ar[ur, "{F[\subseteq]}"']
    \end{tikzcd}
    \end{equation*}
    The bottom triangle commutes because $F$ is a functor, the next square up commutes by definition of $F$ and $\tF$, and the top square commutes because $\tF$ is a functor. 
    Combining this with the outside ring commuting means that the top triangle commutes, which is the final ingredient needed for the definition of an interleaving. 
\end{proof}