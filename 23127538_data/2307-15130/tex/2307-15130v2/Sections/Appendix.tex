\section{Technical proofs}
\label{sec:technicalProofs}

In this section, we include the technical proofs from the rest of the paper. 


\lossimpliescommutes*

\begin{proof}[Proof of Lemma \ref{lem:lossimpliescommutes}]
We prove the lemma for the first and third entries only as the other arguments are symmetric. 
Assume $\Lpl^{U,V}(\phi) \leq k$ and 
consider the diagram
\begin{equation*}
\begin{tikzcd}[column sep = 4em]
F(U) 
\ar[r, "{F[\subseteq]}"] 
\ar[dr, "\phi_U",] \ar[ddr, "\Phi_U"', very near start]
& F(V) \ar[dr, "\phi_V", very near start] \ar[ddr, "\Phi_V"',very near start]\\
&    G(U^n) \ar[r, crossing over,  very near start, "{G[\subseteq]}"'] \ar[d, "{G[\subseteq]}"] & G(V^n) \ar[d, "{G[\subseteq]}"] \\
&    G(U^{n+k}) \ar[r,  "{G[\subseteq]}"] & G(V^{n+k})
\end{tikzcd}
\end{equation*}
noting that the top of the diagram does not necessarily commute in the case that $k\geq 1$, and the bottom of the diagram is $\Parallelograml_{\Phi}(U,V)$ for which we wish to check commutativity. 
For any $x \in F(U)$, following around the top square gives 
\begin{equation*}
\begin{tikzcd}
x \ar[r,mapsto] \ar[d,mapsto]
& x' \ar[d, mapsto] \\
\substack{ \\a} 
    \ar[r, mapsto, shift right,end anchor = {[yshift = -0.4ex]}] 
& \substack{ b'\\a'}
\end{tikzcd}
\end{equation*}
with $d_{V^n}^G(a',b') \leq k$.
By definition, the image of $a'$ and $b'$ is the same under the map \linebreak ${G(V^n) \to G(V^{n+k})}$.
Then since the front square commutes by functoriality of $G$, and the side triangles commute by definition of $\Phi$, we have that the image of $x$ under either direction of the back square commutes, proving claim (1). 

Turning to claim (3), consider the noncommutative diagram 
%
%
%
\begin{equation*}
\begin{tikzcd}[execute at end picture={
\foreach \Nombre in  {A,B,...,F}
  {\coordinate (\Nombre) at (\Nombre.center);}
\fill[yellow,opacity=0.3] 
  (A) -- (E) -- (F) -- cycle;
\fill[yellow,opacity=0.3] 
  (F) -- (C) -- (D) -- cycle;
\fill[blue,opacity=0.1] 
  (B) -- (C) -- (F) -- (E) -- cycle;
%
  %
}]
|[alias=A]| F(U) 
    \ar[rrr, "{F[\subseteq]}"] 
    \ar[dr, "\phi_\bullet"']
    \ar[drr, "\Phi_\bullet"]
&&& |[alias=B]| F(U^{2n}) 
    \ar[r, "{F[\subseteq]}"] 
& |[alias=C]| F(U^{2n+k}) 
    \ar[r, "{F[\subseteq]}"] 
& |[alias=D]| F(U^{2(n+k)}).
\\
& |[alias=E]| G(U^n) 
    \ar[r, "{G[\subseteq]}"'] 
    \ar[urr, "\psi_\bullet"]
& |[alias=F]| G(U^{n+k}) 
    \ar[urr, "\psi_\bullet"]
    \ar[urrr, "\Psi_\bullet"']
\end{tikzcd}
\end{equation*}
The  two yellow triangles commute by definition of $\Phi$ and $\Psi$. 
The blue parallelogram is the diagram $\Parallelogramr_\psi(U^n,U^{n+k})$ which also has loss function bounded by $k$, thus elements of $G(U^n)$ are not necessarily the same in the image of $F(U^{2n+k})$ following the parallelogram, but are the same in $F(U^{2(n+k)})$. 

Checking that $\triangled_{\Phi,\Psi}(U)$ commutes amounts to a diagram chase.
For an arbitrary $\alpha \in F(U)$, consider the following elements aligning with the diagram above 
\begin{equation*}
\begin{tikzcd}
\alpha
    \ar[rrr, mapsto, end anchor = {[yshift = 1ex]}] 
    \ar[dr, mapsto]
    \ar[drr, mapsto]
&&& \substack{a\\a'\\}
    \ar[r, mapsto, 
        end anchor = {[yshift = 1.5ex]},
        start anchor = {[yshift = 1ex]}] 
    \ar[r, mapsto, 
        end anchor = {[yshift = -0.2ex]},
        start anchor = {[yshift = -1ex]}] 
& \substack{b\\b'\\b''} \Big\}
    \ar[r, mapsto] 
& c
\\
& x
    \ar[r, mapsto] 
    \ar[urr, mapsto,end anchor = {[yshift = -.5ex]}]
& x'
    \ar[urr,mapsto, 
        end anchor = {[yshift = -1ex]}]
    \ar[urrr, mapsto, bend right = 10, start anchor = {[yshift = -.5ex]}]
\end{tikzcd}
\end{equation*}
Both $\alpha$ and $x$ map to $x'$ because of the yellow triangle commuting, and both $b''$ and $x'$ map to the same $c$ for the same reason.
Even if $\alpha$ and $x$ map to different elements in $F(U^{2n})$, they must map to the same element in $F(U^{2(n+k)})$, and this must be $c$ since both $b'$ and $b''$ map to the same element by the bound on the blue parallelogram. 
%
%
%
%
%
%
%
%
%
As this was done for an arbitrary $\alpha$, we have that $\triangled_{\Phi, \Psi}(U)$ commutes. 

%


Claims (2) and (4) are similar with appropriate choices of diagrams. 
The final statement is immediate since $L(\phi,\psi) = 0$ implies all diagrams needed for an interleaving commute.
\end{proof}


\extendToNatTrans*
\begin{proof}[Proof of Lemma \ref{lem:extendToNatTrans}]
%
We start by defining $\Phi_U$ for arbitrary open sets. 
Note that since ${\Lpl^{U_\tau, U_\sigma} = 0}$, for any $\sigma \leq \tau$, the diagram of the form 
\begin{equation*}
\begin{tikzcd}
F(U_\tau) \ar[r,"{F[\subseteq]}"] \ar[d] \ar[d, "{\Phi_{U_\tau}}"']
    & F(U_\sigma) \ar[d, "{\Phi_{U_\sigma}}"] \\
G(U_\tau^n) \ar[r,"{G[\subseteq]}"] 
    & G(U_\sigma^n)
\end{tikzcd}
\end{equation*}
commutes.

For an arbitrary open $U$, define the cover 
$\cU = \{ U_\sigma \mid \sigma \in U\}$. 
It is straightforward to check that $U = \bigcup_{U_\sigma \in \cU} U_\sigma$ and that any nonempty intersection $U_\sigma \cap U_\tau$ is also an element of $\cU$. 
Then we use the fact that $F$ is a cosheaf, and in particular this means that $F(U)$ is the coequalizer of the diagram 
\begin{equation*}
\begin{tikzcd}[column sep = 1in]
    \displaystyle
    \coprod_{\sigma, \sigma'} F(U_\sigma \cap U_{\sigma'}) 
        \ar[r, shift left, "{F[U_\sigma \cap U_{\sigma'}\subseteq U_\sigma]}"] 
        \ar[r, shift right, "{F[U_\sigma \cap U_{\sigma'}\subseteq U_{\sigma'}]}"'] 
    &
    \displaystyle\coprod_{\tau} F(U_\tau).
\end{tikzcd}
\end{equation*}
%
Rephrased, this means that for any set $S$ with maps $F(U_\sigma) \to S$ such that the  solid arrow diagrams of the form 
\begin{equation*}
\begin{tikzcd}
F(U_\sigma \cap U_{\sigma'}) \ar[d, "{F[\subseteq]}"'] \ar[r,"{F[\subseteq]}"] 
& F(U_\sigma) \ar[d, "{F[\subseteq]}"]  \ar[ddr, bend left] \\
F(U_{\sigma'}) \ar[r, "{F[\subseteq]}"] \ar[drr, bend right] 
& F(U)  \ar[dr, dashed, "\exists!"]\\
 & & S
\end{tikzcd}
\end{equation*}
commute for any $\sigma, \sigma'$, then there is a unique map $F(U) \to S$ whose addition still has all diagrams commute. 
In our case, set $S = G(U^n)$, and define the legs of the cocone to be $G[\subseteq] \circ \Phi_{U_\sigma}$ as seen in the bold purple arrows of the diagram 
%
%
%
%
%
%
%
%
%
%
%
%
%
%
%
%
%
%
%
%
%
%
\begin{equation}
\label{eqn:cubedgm}
\begin{tikzcd}[row sep=1.5em, column sep = 1.5em]
F(U_{\sigma} \cap U_{\sigma'})
    \arrow[rr, "\Phi_{\bullet}"] \arrow[dr, "{F[\subseteq]}"] 
    \arrow[dd,swap, "{F[\subseteq]}"] 
    &&
G( (U_{\sigma} \cap U_{\sigma'})^n)\arrow[dd, "{G[\subseteq]}"', very near start] \arrow[dr, "{G[\subseteq]}"] \\
& 
F(U_{\sigma'})
    \arrow[rr, crossing over, thick, violet,"\Phi_{\bullet}", near start ] 
&&
G(U_{\sigma'}^n)
    \arrow[dd, thick, violet, "{G[\subseteq]}"] \\
F(U_\sigma) 
    \arrow[rr, thick, violet,"\Phi_{\bullet}", near start] 
    \arrow[dr, "{F[\subseteq]}"] 
&& 
G(U_{\sigma'}^n)\arrow[dr, thick,  violet, "{G[\subseteq]}"] \\
& 
F(U)
    \arrow[rr, dashed, "{\exists! \, \, \Phi_U}"] 
    \arrow[uu, leftarrow, crossing over, "{F[\subseteq]}", very near end]
&& G(U^n)
\end{tikzcd}
\end{equation}
where $\Phi_{\bullet}$ means $\Phi_V$ for the appropriate set $V$, but is dropped to simplify the notation.
Note that the diagram prior to the inclusion of the dotted line commutes, since we can check the relevant faces as follows. 
The left and right squares commute because $F$ and $G$ are functors. 
The back and top panels commute because they involve only basis opens; equivalently because we assumed $\Lpl^{U_\sigma \cap U_{\sigma'}, U_\sigma}= \Lpl^{U_\sigma \cap U_{\sigma'}, U_{\sigma'}}  = 0$.
Then, because $F(U)$ is a colimit of the diagram, there exists a unique map $\Phi_U:F(U) \to G(U)$ as noted, making any diagram of this form commute. 

To ensure that the resulting $\Phi_U$ maps  make diagrams of the form 
\begin{equation*}
\begin{tikzcd}
    F(U) \ar[r] \ar[d] & G(U^n) \ar[d] \\
    F(V) \ar[r] & G(V^n)
\end{tikzcd}
\end{equation*}
commute for arbitrary $U \subseteq V$, fix such a pair and an $x \in F(U)$. 
Because $F(U)$ is the colimit, there is a $\sigma$ and an $x_\sigma \in F(U_\sigma)$ such that $x_\sigma \mapsto x$. 
In this case we have the diagram 
\begin{equation*}
\begin{tikzcd}
F(U_\sigma) 
\ar[r, "\Phi_\bullet"] 
\ar[dr] \ar[ddr]
& G(U_\sigma)^n \ar[dr] \ar[ddr]\\
&    F(U) \ar[r, crossing over, "\Phi_\bullet", very near start] \ar[d] & G(U^n) \ar[d] \\
&    F(V) \ar[r, "\Phi_\bullet"] & G(V^n)
\end{tikzcd}
\end{equation*}
The top and back squares commute because they are the front of the cube of the diagram in Eqn.~\eqref{eqn:cubedgm}. 
The left and right triangles commute since $F$ and $G$ are functors. 
Thus the front square, and in particular the element $x \in F(U)$, commute. 
This means the resulting $\Phi$ is a natural transformation, and thus $\Lpl^{U,V} = 0$. 
\end{proof}


\extendTriangles*
\begin{proof}[Proof of Lemma \ref{lem:extendTriangles}]
%
Because $\Ltd^{U_\sigma} = 0$ for all basis elements, diagrams of the form 
\begin{equation*}
\begin{tikzcd}
    F(U_\sigma) \ar[rr, "{F[\subseteq]}"] \ar[dr, "\Phi_{U_\sigma}"']&& F(U_\sigma^{2n})\\
    & G(U_\sigma^n) \ar[ur, "\Psi_{U_{\sigma}^n}"']
\end{tikzcd}
\end{equation*}
commute for any $\sigma \in K$.
Given an arbitrary open set $U$, let $x \in F(U)$ be given. 
As in the proof of Lem.~\ref{lem:extendToNatTrans}, there is a $\sigma$ and an $x_\sigma \in F(U_\sigma)$ with $x_\sigma \mapsto x$. 
Then consider the diagram 
\begin{equation*}
\begin{tikzcd}
	F(U_\sigma) &&&& F(U_\sigma^{2n}) \\
	&& F(U) &&& {} & F(U^{2n}) \\
	&& G(U_\sigma^n) \\
	&&&& G(U^n)
	\arrow[from=4-5, to=2-7, "\Psi_\bullet"']
	\arrow[from=3-3, to=1-5, "\Psi_\bullet", near end]
	\arrow[from=1-1, to=1-5, "{F[\subseteq]}"]
	\arrow[from=1-1, to=3-3, "\Phi_\bullet"']
	\arrow[from=1-1, to=2-3, "{F[\subseteq]}"]
	\arrow[from=1-5, to=2-7, "{F[\subseteq]}"]
	\arrow[from=3-3, to=4-5, "{G[\subseteq]}"']
	\arrow[from=2-3, to=2-7, crossing over, "{F[\subseteq]}"]
	\arrow[from=2-3, to=4-5, crossing over, "\Phi_\bullet", near end]
\end{tikzcd}
\end{equation*}
where we again replace map subscripts with $\bullet$ to simplify notation.
The top square commutes because $F$ is a functor. 
The back triangle commutes by this lemma's assumption.
The left and right squares commute because $\Phi$ and $\Psi$ are natural transformations.
Taken together, this means that the diagram commutes and in particular, the image of $x \in F(U)$ chased around the front triangle commutes.
\end{proof}