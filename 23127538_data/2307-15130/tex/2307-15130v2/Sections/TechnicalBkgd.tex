%
%
%

%



%
%
%

%

%
%
%
%
%
%
%
%
%
%
%
%
%
%
%
%
%
%
%

%

%
%
%

%
%
%
%
%
%


%
%
%
%

%
%
%

%
%
%

%
\section{Technical Background}
\label{sec:Bkgd}

Our given data is a finite graph with an embedding function $f: G \to \R^2$. 
While more general mathematical assumptions are possible (such as immersions with an $o$-minimal image \cite{Dries1998}), for the computational goals of this paper, we will assume a straight line, plane embedding. 
%
We define a bounding box on the image $f(G)$, which for simplicity we will assume is of the form $[-B, B]^2$. 
%
We next outline the necessary categorical framework in order to formulate a precise notion of the interleaving distance.

%
\subsection{Functors and cosheaves}
\label{ssec:cosheaf}

We give basic definitions for the category theoretic notions required in this paper, but note that the introduction will be quite sparse. 
We direct the interested reader to \cite{Riehl2017,Curry2014} for further details. 

A category $\cC$ consists of a collection of objects $X,Y,Z,\cdots$ and morphisms $f,g,h,\cdots$ with the following data: 
morphisms $f:X \to Y$ have designated domain $X$ and codomain $Y$. 
Every object has a designated identity morphism $\1_X: X \to X$, and any pair of morphisms $f:X \to Y$ and $g:Y \to Z$ has a composite morphism $gf:X \to Y$. 
These objects and morphisms are required to satisfy an identity axiom, where $f:X \to Y$ is the same as the composites $\1_Yf$ and $f\1_X$, and composition of morphisms is associative, so $h(gf) = (hg)f$. 
Some examples used in this paper are $\Set$, where objects are sets and morphisms are set maps; 
$\Top$ where objects are topological spaces and morphisms are continuous functions;
and $\Open(K)$ for a given topological space $K$, where the objects are open sets and morphisms are given by inclusion. 
The latter example is a special case called a \emph{poset category}, where every pair of objects has at most one morphism between them. 

A functor $F:\cC \to \cD$ is a map between categories preserving the relevant structures.
Specifically, for every object $X \in \cC$ there is a an object $F(X) \in \cD$, and for every morphism $f:X \to Y$, there is a morphism $F[f]:F(X) \to F(Y)$. 
To be a functor, $F$ must further satisfy that for  any $X \in \cC$, $F[\1_X] = \1_{F(X)}$ and for any composable pair $f,g \in \cC$, we have $F[gf] = F[g] F[f]$. 
Given functors $F,G: \cC \to \cD$, a natural transformation $\eta: F \Rightarrow G$  consists of a map $\eta_X:F(X) \to G(X)$ for every $X \in \cC$ (called the components) so that for any morphism $f:X \to Y$ in $\cC$, the diagram 
\begin{equation*}
\begin{tikzcd}
X\ar[d,  "f"] 
& F(X) 
    \ar[r, "\eta_X"] 
    \ar[d, "{F[f]}"']
& G(X) 
    \ar[d, "{G[f]}"]
\\
Y 
& F(Y) \ar[r, "\eta_Y"] 
& G(Y)
\end{tikzcd}
\end{equation*}
commutes. 
One example we will use later is $\pi_0:\Top \to \Set$, where $\pi_0(\X)$ is the set of path connected components of $\X$, and morphisms are set maps $\pi_0[f]: \pi_0(\X) \to \pi_0(\Y)$ sending a connected component $A$ in $\X$ to the connected component of $f(A)$ in $\Y$.

A diagram is a functor $F:J \to \cC$ where $J$ is a small category. 
In essence, this construction picks out a collection of objects $F(j)$ and a collection of morphisms $F(j) \to F(k)$. 
A cocone on a given diagram is a natural transformation $\lambda:F \to c$ where we abuse notation to write that $c: J \to \cC$ is the constant functor returning the object $c(j) = c \in \cC$ for all $j \in J$. 
We often call the components $\lambda:F(j) \to c$ the \emph{legs}, and note that this requirement says that for any $f:j \to k$ in $J$, the triangle 
\begin{equation*}
\begin{tikzcd}
F(j) 
    \ar[rr, "{F[f]}"]
    \ar[dr, "\lambda_j"']
&& F(k)
    \ar[dl, "\lambda_k"]
\\
& c
\end{tikzcd}
\end{equation*}
commutes. 
A cocone $\lambda:F \to c$ is called a colimit if for any other cocone $\lambda':F \to c'$, there is a unique  morphism $u:c \to c'$ such that 
\begin{equation*}
\begin{tikzcd}
F(j) 
    \ar[rr, "{F[f]}"]
    \ar[dr, "\lambda_j"]
    \ar[ddr, "\lambda_j'"']
&& F(k)
    \ar[dl, "\lambda_k"']
    \ar[ddl, "\lambda_k'"]
\\
& c \ar[d, dashed, "u"] \\
& c'
\end{tikzcd}
\end{equation*}
commutes for all $f:j \to k$ in $J$. 

We will be particularly interested in functors of the form $F:\Open(X) \to \Set$, which can also be called \emph{pre-cosheaves}. 
A pre-cosheaf is a cosheaf if it satisfies a gluing axiom meaning $F(U)$ is entirely determined by $F(U_\alpha)$ for any cover $\{U_\alpha\}_\alpha$. 
Specifically, given an open set $U$ and a cover $\{ U_\alpha \mid \alpha \in A\}$ of $U$, define a category $\cU = \{U_\alpha \cap U_{\alpha'} \mid \alpha,\alpha' \in A \}$ with morphisms given by inclusion. 
Then we have a diagram $F:\cU \to \Set$, and as such can consider its colimit $\lambda:F \to L$.
If the unique map $L \to F(U)$ given by the colimit definition is an isomorphism, then $F$ is called a \emph{cosheaf}.
%
%
%
%
%
%
%
%
%
%
%
%
%
%
%
%



%
\subsection{Functorial Representation of Geometric Graphs}
\label{ssec:functorGraphs}
%
%



We start by defining a cubical complex on $[-B,B]^2$, where $[-B,B]^2$ is the bounding box for the graphs $X$ and $Y$.
Following \cite{Kaczynski2004}, we define a cubical complex given by diameter $\delta$.
For the sake of simplicity, assume that $B$ is a multiple of $\delta$, so that the bounding box can be written as $[-L\delta, L\delta]^2$. 
An \emph{elementary interval} is defined to be a closed interval in $\R$ of the form $[\ell\delta, (\ell+1)\delta]$ or $[\ell]:= [\ell\delta, \ell\delta]$ for $\ell \in [-L,\cdots, L] \subset \Z$.
These are called non-degenerate and degenerate intervals, respectively. 
An elementary cube $Q$ is a finite product of two elementary intervals, i.e.
%
   $ \sigma = I_1 \times I_2 \subset [-B,B]^2$. 
%
The dimension of a cube $\sigma$ is given by the number of intervals used which are non-degenerate. 
Note that this means 
$0$-cubes are vertices at grid locations $[i\delta, j \delta]$, 
$1$-cubes are products $[i\delta] \times [j\delta, (j+1)\delta]$ or $[i\delta, (i+1)\delta] \times [j\delta ]$, 
and $2$-cubes are products $[i\delta, (i+1)\delta] \times [j\delta, (j+1)\delta]$. 
The collection of elementary cubes $K$ is a finite cubical complex.
This construction comes with a face relation which gives a poset structure, where we write $\sigma \leq \tau$ iff $\sigma \subseteq \tau$. 


We now show how we endow the complex $K$ with the Alexandroff topology, following \cite{Barmak2011}. 
Given the poset $(K,\leq)$, for any set $S \subseteq X$, the up-set is the collection 
$S^{\uparrow} = \{x \mid x \geq y, \, y \in S \}$. 
Similarly, the down-set $S^{\downarrow}$ is the collection $\{ x \mid x \leq y, \, y \in S\}$. 
For any elementary cube $\sigma$, we have that 
\begin{equation*}
    U_\sigma := \{ \sigma\}^{\uparrow} = \{\tau  \mid \tau \geq \sigma \}
\end{equation*}
is the same as the star of the cube, to borrow terminology from the simplicial complex literature. 
We give $K$ the Alexandroff topology%
\footnote{Note that in the case of finite posets,  either down- or  up-sets can be used to define the topology; while in general the Alexandroff topology is defined using the down-sets as opens \cite{Barmak2011}.  
However, we are trying to avoid using the opposite poset as much as possible to alleviate notation woes, and the correspondence with stars in this setting is useful for our purposes.} 
$\Open(K)$, where a set $U\subseteq K$ is said to be open iff the following holds: 
for any $x \in U$, and any $y \geq x$, we have that $y \in U$. 
Equivalently, this means that $U$ is its own up-set, i.e.~$U = U^\uparrow$.
It can be checked that this topology has the collection  
$\{U_\sigma\}_{\sigma \in K}$ 
as a basis. 

%
%
%

The main objects of study here are embedded graphs. 
That is, we start with input data $f:\X \to \R^2$ where $\X$ is a finite topological graph and $f$ is a straight line plane embedding. 
%
%
Then we can encode this information in a functor $F:\Open(K) \to \Set$ given by 
\begin{equation*}
    \begin{matrix}
    F: &  \Open(K) & \to & \Set\\
    & U & \mapsto & \pi_0 f\inv(|U|).
    \end{matrix}
\end{equation*}
Note that functoriality of $\pi_0$ means that for $U \subseteq V$ there is an induced map
\[F[U \subseteq V] \colon \pi_0f\inv(|U|) \to \pi_0f\inv(|V|)\]
satisfying all requirements of a functor for $F$.
Indeed, this functor is actually a cosheaf. 
Throughout the paper, when the notation makes the sets involved obvious, we will simply write the induced map as $F[\subseteq]:F(U) \to F(V)$. 



%
\subsection{Thickenings}
\label{ssec:thickenings}
% Figure environment removed
Given any set $U \in \Open(K)$, the 1-thickening  is defined by taking the upset of the downset of $U$. This can be written as 
$   U^1 = (U^{\downarrow})^\uparrow$.
Thinking in parallel to simplicial complex settings, this operation can be thought of as taking the star of the closure of the set. 
See Fig.~\ref{fig:thickenings} for examples. 
We then define the $n$-thickening to be $n$ repetitions of the process given recursively as
\begin{equation*}
    U^n = 
    \begin{cases}
    U & n = 0\\
    %
    (U^{n-1})^{\downarrow\uparrow} & n \geq 1.
    \end{cases}
\end{equation*}
Note that each $U^n$ is itself an open set in $\Open(K)$, and that if $U \leq V$, then $U^n \subseteq V^n$. 
Thus we can view this operation as a functor on the category $\Open(K)$ with morphisms given by inclusion:
\begin{equation*}
    \begin{matrix}
    (-)^n: & \Open(K)  & \to &  \Open(K)\\
    & U & \mapsto & U^n.
    \end{matrix}
\end{equation*}

One property of this construction that will be useful is the following. 
For any $\sigma \in U^n$, there is a $\tau \in U$ and a sequence of cells
\begin{equation}
\label{eq:length_n_path}
    \tau 
    \geq \gamma_1 \leq \tau_1 
    \geq \gamma_2 \leq \tau_2 
    \geq  \cdots 
    \geq \gamma_n \leq \sigma.
\end{equation}
Further, given such a sequence with $\tau \in U$, we know that $\sigma \in U^n$. 
Two examples of this can be seen in Fig.~\ref{fig:Length_n_path}, where $\sigma$ and $\sigma'$ from $U^3$ are given, along with a path satisfying Eq.~\eqref{eq:length_n_path}.
Of course, the choice of sequence for Eq.~\eqref{eq:length_n_path} is not unique, so other options are possible. 
% Figure environment removed

\begin{lemma}
$(-)^n$ is a functor. 
\end{lemma}

\begin{proof}
%
First, we check that the images of morphisms are well defined, which is to say that if $U \subseteq V$, then $U^{n} \subseteq V^{n}$. 
The statement is clear if $n = 0$, so by induction, we assume that $U^{n-1} \subseteq V^{n-1}$. 
Given an arbitrary $\sigma \in U^{n}$, the statement is immediate if $\sigma \in U^{n-1} \subseteq U^{n}$, so we assume $\sigma \in U^n \setminus U^{n-1}$. 
For this to happen, there must be a $\gamma \in U^{n-1}$ and $\tau \in K$ with $\gamma \geq \tau \leq \sigma$. 
But as $\gamma \in U^{n-1} \subseteq V^{n-1}$, this sequence also implies that $\sigma \in V^n$ finishing the well-defined check. 

To ensure this is a functor, we need to check that the identity morphism is sent to the identity, and that composition holds. 
For the former, we see that $U \subseteq U$ gets sent to $U^n \subseteq U^n$, and each is an identity. 
The latter is immediate from the property that $\Open(K)$ is a poset category, meaning there is at most one morphism between any pair of objects.  
\end{proof}


We can use this construction to build an interleaving distance on functors of the form \linebreak
${F:\Open(K) \to \Set}$ using the superlinear family of translations framework of \cite{Bubenik2014a}.
Note that this construction can be generalized to the concept of a category with a flow \cite{deSilva2018}, but the added generality is not needed here. 

\begin{definition}[\cite{Bubenik2014a}]
Let $P = (P,\leq)$ be a preordered set. 
A \emph{translation} on $P$ is a functor $\Gamma: P \to P$ along with a natural transformation $\eta:\1_P \Rightarrow \Gamma$. 
A \emph{super-linear family of translations} is a collection $\{\Gamma_\e \}_{\e\geq 0}$  such that 
$\Gamma_\e \Gamma_{\e'}(p) \leq \Gamma_{\e + \e'}(p)$ for all $p \in P$, and $\e, \e' \geq 0$. 
%
\end{definition}

\begin{lemma}
\label{lem:composedthickenings}
For any $n, n' \geq 0$ and $U \in \Open(K)$, 
$(U^{n})^{n'} = U^{n+n'}$.
%
This is a stronger requirement than needed above, so the collection $\{ ( - )^n\}_{n \geq 0}$ forms a super-linear family of translations. 
%
\end{lemma}

\begin{proof}
First, we check that $(-)^n$ is indeed a translation using the above terminology. 
In particular, we define $\gamma^n:\1_{\Open(K)} \Rightarrow (-)^n$ to have components $\gamma^n_U: U \to U^n$ as simply the inclusion, and can easily check that this satisfies naturality requirements. 

Fix $U \in \Open(K)$. 
We need to show that $(U^n)^{n'} = U^{n+n'}$. 
Let $\sigma \in (U^n)^{n'}$. 
By previous remarks, this is true if and only if there is a sequence 
\begin{equation*}
    \tau 
    \geq \gamma_1 \leq \tau_1 
    \geq \gamma_2 \leq \tau_2 
    \geq  \cdots 
    \geq \gamma_{n'} \leq \sigma
\end{equation*}
with $\tau \in U^n$. 
But this property of $\tau$ happens iff there is also a sequence 
\begin{equation*}
    \tau' 
    \geq \gamma_1' \leq \tau_1 '
    \geq \gamma_2' \leq \tau_2' 
    \geq  \cdots 
    \geq \gamma_{n}' \leq \tau
\end{equation*}
with $\tau' \in U$. 
Concatenating the two sequences gives a sequence of length $(n+n')$ from $\tau \in U$ to $\sigma$. 
Thus $\sigma \in U^{n+n'}$ iff $\sigma \in (U^n)^{n'}$, and hence $(U^n)^{n'} = U^{n+n'}$. 
\end{proof}

Our next task is to use this structure to define an interleaving distance. 
%
%
%
%
The first necessary ingredient is the composition of functors $F \circ ( - )^n: \Open(K) \to \Set$, which we will denote it by $F^n$. 
This means $F^n(U) = F(U^n)$, and we have the similar setup for $G$. 
With this notation, the interleavings use natural transforms of the form $\phi:F \Rightarrow G^n$ and $\psi:G \Rightarrow F^n$. 
Note that a component of $\phi$ is a set-map $\phi_U:F(U) \to G(U^n)$. 
There is of course, another component at $U^n$, $\phi_{U^n}:F(U^n) \to G(U^{2n})$, which can also be viewed as a component of another natural transformation $\phi^n:F^n \Rightarrow G^{2n}$. 
For this reason, we use the notation $\phi_{U^n}$ and $\phi_U^n$ interchangeably when $\phi$ is indeed a natural transformation. 
Note that we are implicitly using Lem.~\ref{lem:composedthickenings} to write the maps this way. 

\begin{definition}
\label{def:interleavingDistance}
Given two cosheaves $F,G:\Open(K) \to \Set$ and $n \geq 0$, an \emph{$n$-interleaving} is given by a pair of natural transformations 
$\phi:F \Rightarrow G^n$
and 
$\psi:G \Rightarrow F^n$
such that the diagrams
%
%
%
\begin{equation*}
    \begin{tikzcd}
        F(U) 
            \ar[rr, "{F[U \subseteq U^{2n}]}"]   
            \ar[dr, "\phi_U"',violet]
            %
            & & F(U^{2n}) & 
        & F(U^n) \ar[dr]
            \ar[dr, "\phi_{U^{n}}",violet]
        & \\
        & G(U^n)\ar[ur, "\psi_{U^n}"', orange]  & & 
        G(U) 
            \ar[rr, "{G[U \subseteq U^{2n}]}"']
            \ar[ur, "\psi_{U}", orange] 
        %
        && G(U^{2n})
    \end{tikzcd}
\end{equation*}
%
%
%
%
%
%
%
%
%
%
%
%
%
%
%
%
%
%
%
commute for all $U \in \Open(K)$.
The interleaving distance is given by 
\begin{equation*}
    d(F,G) = \inf\{ n \geq 0 \mid \text{there exists an $n$-interleaving} \}
\end{equation*}
and is set to be $d(F,G) = \infty$ if there is no interleaving for any $n$.
\end{definition}
%

%
%
%
%
%
%
%
%
%
%
%
%
%
%
%
%
%
%
%
%

\begin{theorem}
    The interleaving distance of Defn.~\ref{def:interleavingDistance} is an extended pseudometric. 
\end{theorem}
\begin{proof}
This result is immediate from \cite[Theorem 3.21]{Bubenik2014a}.
    %
\end{proof}


%

%
%
%
%
%

%
%
%
%

%
%

%
%
%
%
%
%
%
%
%
%
%
%
%
%
%
%
%
%

%
%
%
%
%
%
%
%
%
%
%
%
%
%
%
%
%
%
%
%
%



%

%
    
%


%
%
%
%

%
%

%
%