\documentclass[prl,aps,twocolumn,showpacs,floatfix]{revtex4-2}
% \documentclass[preprint]{revtex4-2}
\usepackage{amsfonts}
\usepackage{amssymb}
\usepackage{hyperref}
\usepackage{graphicx}
\usepackage{dcolumn}
\usepackage{bm,amsmath,verbatim}
\usepackage{mathrsfs}
\usepackage{color}

\hypersetup{colorlinks,
	linkcolor=blue,          citecolor=blue,        filecolor=blue,      urlcolor=blue           }

\newcommand{\ket}[1]{|#1\rangle}
\newcommand{\bra}[1]{\langle#1|}

\begin{document}


%\preprint{}

\title{Higher-order Topological Insulators and Semimetals in Three Dimensions without Crystalline Counterparts}
\author{Yu-Feng Mao$^{1}$}
\author{Yu-Liang Tao$^{1}$}
\author{Jiong-Hao Wang$^{1}$}
\author{Qi-Bo Zeng$^{2}$}
\author{Yong Xu$^{1,3}$}
\email{yongxuphy@tsinghua.edu.cn}
\affiliation{$^{1}$Center for Quantum Information, IIIS, Tsinghua University, Beijing 100084, People's Republic of China}
\affiliation{$^{2}$Department of Physics, Capital Normal University, Beijing 100048, People's Republic of China}
\affiliation{$^{3}$Hefei National Laboratory, Hefei 230088, People's Republic of China}

\begin{abstract}
Quasicrystals allow for symmetries that are impossible in crystalline materials, such as eight-fold rotational symmetry,
enabling the existence of novel higher-order topological insulators in two dimensions without crystalline counterparts. 
However, it remains an open question whether three-dimensional higher-order topological insulators and Weyl-like semimetals
without crystalline counterparts can exist. Here, we demonstrate the existence of a second-order topological insulator
by constructing and exploring a three-dimensional model Hamiltonian in a stack of 
Ammann-Beenker tiling quasicrystalline lattices.
The topological phase has eight chiral hinge modes that lead to quantized longitudinal conductances of $4 e^2/h$. 
We show that the topological phase is characterized by the winding number of the quadrupole moment.
We further establish the existence of a second-order topological insulator with time-reversal symmetry,
characterized by a $\mathbb{Z}_2$ topological invariant. Finally, we propose a model that exhibits 
a higher-order Weyl-like semimetal phase, demonstrating both hinge and surface Fermi arcs. 
Our findings highlight that quasicrystals in three dimensions can give rise to higher-order topological insulators and 
semimetal phases that are unattainable in crystals.
\end{abstract}

\maketitle

Higher-order topological phases represent a significant expansion of conventional first-order topological phases 
and have experienced considerable advancements in recent years~\cite{Taylor2017Science,Taylor2017PRB,Fritz2012PRL,ZhangFan2013PRL,Slager2015PRB,Brouwer2017PRL,FangChen2017PRL,
	Schindler2018SA, Brouwer2019PRX,Seradjeh2019PRB,Roy2019PRB,Yang2019PRL,Hughes2020PRB, Xu2020PPR,
	Parameswaran2020PRL,Zhao2020PRL,AYang2020PRL,Xu2020NJP,Cho2020arXiv,Yang2023Review}. These phases possess edge states of 
dimension $(n-m)$ ($1< m \le n$) in an $n$-dimensional system, which is in stark contrast to 
first-order cases that support $(n-1)$-dimensional edge states. 
For instance, in two dimensions (2D), second-order topological insulators like the quadrupole insulator 
exhibit four corner states~\cite{Taylor2017Science,Taylor2017PRB}. In three dimensions (3D), second-order topological insulators give rise to 
four chiral (or helical pairs of) hinge modes~\cite{Schindler2018SA}. Furthermore, higher-order Weyl semimetals in 
3Ds display bulk Weyl nodes that feature both surface and hinge Fermi 
arcs~\cite{Jiang2020PRL,Hughes2020PRL,Wei2021NM,Luo2021NM}. 

Apart from crystalline systems, higher-order topological states have also been found
in non-crystalline systems such as quasicrystals~\cite{Fulga2019PRL,Xu2020PRL,Xu2020PRB,Huang2021NL,Huang2022FP,Huang2022PRL}, 
amorphous lattices~\cite{Roy2020PPR,Wang2021PRL,Zhou2022PRB,Tao2023arxiv} and 
hyperbolic lattices~\cite{Xu2023PRB,Zhou2023PRB},
despite the absence of translational symmetries. 
Remarkably, these systems in 2Ds can support higher-order topological phases that
cannot exist in crystals. 
For example, a quasicrystal in 2Ds with eight-fold rotational symmetry can possess 
eight corner modes~\cite{Fulga2019PRL,Xu2020PRL}, 
in stark contrast to crystalline counterparts with two, four or six corner modes~\cite{Taylor2017Science, Yang2019PRL,Zhao2020PRL,AYang2020PRL}. 
Similar cases occur in amorphous lattices protected by an average symmetry~\cite{Tao2023arxiv}.
Although significant progress has been made, it remains unclear whether topological insulators 
without crystalline counterparts can exist in 3Ds. The challenge lies in establishing the bulk-boundary 
correspondence in the case of 3D quasicrystals. 
Previously, a higher-order topological phase in a 2D quasicrystal
was characterized by a $\mathbb{Z}_2$ topological invariant determined by the sign of 
the Pfaffian of a transformed Hamiltonian at high-symmetry momenta that is an antisymmetric matrix~\cite{Fulga2019PRL,Xu2020PRL}. 
However, this invariant cannot be generalized to characterize the chiral hinge modes in 3Ds.
Fortunately, the winding number of the quadrupole moment can be employed to 
characterize the chiral hinge modes when their number is equal to four~\cite{Cho2020arXiv,Wang2021PRL}. 
Very recently, we have proposed a method to calculate the quadrupole moment in 2D
amorphous lattices with eight corner modes~\cite{Tao2023arxiv}. This method thus provides an opportunity 
to establish the existence of topological phases in 3Ds without crystalline counterparts.

% Figure environment removed

In this work, we theoretically predict the existence of a second-order topological insulator by 
constructing and exploring 3D model Hamiltonians in a stack of 
Ammann-Beenker tiling quasicrystals with eight-fold rotational symmetry [see Fig.~\ref{Fig1}(a)].
We find that there are eight gapless chiral hinge modes [see Fig.~\ref{Fig1}(b)] leading to 
longitudinal conductances of $4 e^2/h$.
These topological states are not allowed in crystalline materials due to the absence of 
eight-fold rotational symmetries. 
To establish the bulk-edge correspondence, we use the method proposed in Ref.~\cite{Tao2023arxiv} to calculate
the quadrupole moment and then evaluate its winding number, which confirms the agreement with
the observed conductances. Moreover, we show the existence of second-order topological insulators in 3D quasicrystals with 
time-reversal symmetry (TRS) that support eight helical pairs of hinge states, giving rise to 
the longitudinal conductance of $8 e^2/h$.    
We find that such a phase is protected by a $\mathbb{Z}_2$ topological invariant defined based on 
transformed site positions in quasicrystals. 
Finally, we present a model that showcases the existence of higher-order Weyl-like semimetal phase in 3D quasicrystals.
This phase exhibits both hinge Fermi arcs and surface arcs, which are characterized by the quadrupole moment and Bott index,
respectively.
Notably, unlike higher-order Weyl semimetals in crystals, the quasicrystal exhibits the presence of eight hinge arc states.

 % Figure environment removed

\emph{{\color{blue}Model Hamiltonian}}.--- 
To demonstrate the presence of 3D topological insulators that do not have crystalline counterparts, 
we stack 2D quasicrystalline lattices to create a 3D lattice as shown in Fig.~\ref{Fig1}(a)
and introduce a tight-binding model on the lattice described by the Hamiltonian
\begin{equation} \label{h_without_trs}
	\hat{H}_c=\sum_{\bm r} [M \hat{c}_{{\bm r}}^{\dagger} \tau_z \sigma_0 \hat{c}_{{\bm r}}
	+ \sum_{{\bm d} } \hat{c}_{{\bm r} +{\bm d}}^{\dagger} T_c(\hat{\bm d}) \hat{c}_{ {\bm r} } ],
\end{equation}
where $\hat{c}_{\bm r}^{\dagger} =  (\hat{c}_{{\bm r}, 1}^{\dagger}, \hat{c}_{{\bm r}, 2}^{\dagger}, 
\hat{c}_{{\bm r}, 3}^{\dagger}, \hat{c}_{{\bm r}, 4}^{\dagger})$
with $\hat{c}_{{\bm r},\alpha}^{\dagger}$ ($\hat{c}_{{\bm r},\alpha}$) creating (annihilating) a 
particle of the $\alpha$th component at the lattice 
site of position ${\bm r}$,
and  $\tau_\nu$ and $\sigma_\nu$ with $\nu=x,y,z$ are Pauli matrices acting on internal degrees of freedom.
The first term describes the on-site mass term, and the second one describes the hopping
between two connected sites ${\bm r}$ and ${\bm r} +{\bm d}$ [see Fig.~\ref{Fig1}(a)]
depicted by the hopping matrix $T_c(\hat{\bm d})$ with $\hat{\bm d}$ being the unit vector
of $\bm d$.
For the intra-layer hopping,  
$T_c(\hat{\bm{d}})=[t_0 \tau_z \sigma_0+i t_1 \tau_x(\hat{d}_x\sigma_x + \hat{d}_y\sigma_y) + 
g \cos \left(p \theta / 2\right) \tau_y \sigma_0]/2$ with $p=8$ for the Ammann-Beenker tiling quasicrystals
and $\theta$ being the polar angle of the vector $\bm{d}$, and
for the inter-layer hopping,
$T_c(\hat{\bm{d}})=(t_0 \tau_z \sigma_0+i t_1 \tau_x\sigma_z)/2$.
Here, $t_0$ and $t_1$ are system parameters, which will henceforth be set to one as the units of energy 
for simplicity without loss of generality.
While the term $g \cos \left(p \theta / 2\right) \tau_y \sigma_0$  breaks the TRS and eight-fold rotational symmetry,
their combination symmetry is preserved, which protects the eight chiral hinge modes. 
Without the hopping along $z$, the system reduces to a 2D quasicrystal model with eight zero-energy
corner modes~\cite{Fulga2019PRL,Xu2020PRL}. The hopping along $z$ clearly breaks chiral symmetry so that the 3D model 
is not a simple stacking of 2D models.

To map out the phase diagram with respect to the mass $M$, we numerically calculate the zero-temperature 
two-terminal longitudinal conductance $G$ along the $z$-direction based on the Landauer formula
\begin{equation}
	G = \frac{e^2}{h}T(E_F).
\end{equation}
Here $T(E_F)$ represents the transmission probability from one lead to the other at the energy $E_F$ 
for the 3D quasicrystal system connected to two infinite leads along $z$. 
We calculate the transmission probability $T(E_F)$ at zero energy using the 
nonequilibrium Green's function method~\cite{Dattabook,Xing2007PRB,Yanbin2019PRL}. 

Figure~\ref{Fig2}(a) displays the numerically computed conductance $G$ as a function of $M$, 
remarkably illustrating the existence of a region with a quantized value of $4 e^2/h$. 
Specifically, as we increase $M$ from $-4$, the conductance suddenly rises at $M\approx -3.1$, indicating
the occurrence of a topological phase transition. In fact, the transition point is associated with the
bulk energy gap closing as shown in the inset of Fig.~\ref{Fig2}(b). In the topological region, 
we find that the conductance becomes more perfectly quantized at $ 4e^2/h $ as we 
enlarge the system size [see Fig.~\ref{Fig2}(a)], confirming the existence of the topological phase 
in the thermodynamic limit. To further confirm the origin of the quantized conductance 
from chiral hinge modes, we plot the energy spectrum of the system at $M=-2$
with respect to the momentum $k_z$ under open boundaries in the $x$ and $y$ directions.
The figure clearly shows the presence of gapless chiral hinge modes, which 
result in the quantized conductance. For each chiral mode, there is four-fold degeneracy due
to the $C_8T$ symmetry, the combination of the eight-fold rotational and 
the time-reversal operations. Such hinge states are not allowed in a crystal since the eight-fold
rotational symmetry is not permitted in a crystalline lattice.

We now propose using the winding number of the quadrupole moment with respect to 
$k_z$ to establish the bulk-edge correspondence for the higher-order topological insulator, 
\begin{equation}
	W_Q=\int_0^{2 \pi} d k_z \frac{\partial Q_{x y}(k_z)}{\partial k_z},
\end{equation}
where $Q_{x y}(k_z)$ is the quadrupole moment at $k_z$. The winding number
has been used to characterize the higher-order topological insulator with {\it four} chiral
hinge modes~\cite{Wang2021PRL,Cho2020arXiv}. The quadrupole moment is defined as~\cite{Cho2019PRB,Hughes2019PRB,Xu2021PRB,Shen2020PRL}
\begin{equation}
	Q_{xy} (k_z) = [\frac{1}{2\pi} \operatorname{Im}{\log \det({U_o}^\dagger\hat{D }U_o)-Q_0}]\mod 1,
\end{equation}
where $U_o = (\ket{\psi_1},\ket{\psi_2},\dots,\ket{\psi_{N_{\text{occ} }}} )$ is a matrix consisting of 
occupied eigenstates of the first-quantization Hamiltonian at $k_z$ under periodic boundary conditions~\cite{supplement},
 $\hat{D}=\mathrm{diag} \{e^{2\pi \text{i} x_l y_l/L^2} \}_{l=1}^{2N_{\text{occ} } }$
with $(x_l,y_l)$ being the real-space coordinate of the $l$th degree of freedom, and
$Q_0$ is contributed by the background positive charge distribution.
If we use the original real-space coordinates of the quasicrystal lattice to calculate the quadrupole moment,
we always obtain zero results as clarified in Ref.~\cite{Tao2023arxiv} for the amorphous case. To obtain the reliable
quadrupole moment, we perform the transformations of site positions from $(x_l,y_l)$ to 
$(x_l^\prime,y_l^\prime)$ so that a half or one and a half of a quadrant of sites are transformed into a 
quadrant~\cite{Tao2023arxiv,supplement}. 
While $\hat{D}$ and $Q_0$ are changed accordingly, we use the same bulk states $U_o$ to 
evaluate the quadrupole moment as well as
its winding number. 

Figure~\ref{Fig2}(a) shows the winding number $W_Q$ with respect to $M$, which exhibits the 
quantized value of one in the topological regime and zero in the trivial regime, 
thereby establishing the bulk-edge correspondence of the 3D quasicrystal state.
For clarity, we also display the quadruple moment as a function of $k_z$ at $M_z=-2$
and $M_z=-3.6$, illustrating the presence and absence of the winding, respectively.  

% Figure environment removed

\emph{{\color{blue}Model with TRS}}.--- 
We now construct a model with TRS incorporating eight degrees of freedom per site described by the Hamiltonian
\begin{equation} \label{withtrs}
	\hat{H}_h=\sum_{\bm r} M \hat{c}_{\bm r}^{\dagger} \tau_z s_0 \sigma_0 \hat{c}_{\bm r}+
	\sum_{\bm d} \hat{c}_{{\bm r}+{\bm d}}^{\dagger} T_h(\hat{\bm d}) \hat{c}_{\bm r},
\end{equation}
where $\hat{c}_{\bm r}^{\dagger} =  (\hat{c}_{{\bm r}, 1}^{\dagger},\dots,\hat{c}_{{\bm r}, 8}^{\dagger} )$ 
with $\hat{c}_{{\bm r},\alpha}^{\dagger}$ creating a fermion of the $\alpha$th component at site ${\bm r}$,
and $\{s_\nu\}$ with $\nu=x,y,z$ is another set of Pauli matrices besides $\{\sigma_\nu\}$
and $\{\tau_\nu\}$.
For the intra-layer hopping, 
$T_h(\hat{\bm{d}})=[t_0 \tau_z s_0 \sigma_0+i t_1 \tau_x s_0 ( \hat{d}_x\sigma_x + \hat{d}_y\sigma_y ) 
+ g \cos (4 \theta ) \tau_y s_y \sigma_0 ]/2$
and for the inter-layer hopping along $z$,
$T_h(\hat{\bm{d}})=(t_0 \tau_z s_0 \sigma_0+i t_1 \tau_x s_0 \hat{d}_z\sigma_z  
+ i t_3 \hat{d}_z \tau_y s_x \sigma_0 )/2$.
The Hamiltonian now respects the TRS.
Similar to the case without TRS, we set $t_0=t_1=t_3=1$ as the units of energy.

Previously, we develop a $\mathbb{Z}_2$ invariant to characterize the higher-order topology in an 
amorphous system with TRS supporting four helical pairs of hinge modes in Ref.~\cite{Wang2021PRL}.
The topological invariant $\nu_Q$ is defined based on
\begin{equation}
{(-1)}^{\nu_Q}=\frac{\operatorname{Pf}[A(\pi)]}{\operatorname{Pf}[A(0)]} \sqrt{\frac{\det[A(0)]}{\det[A(\pi)]}},
\end{equation}
where $\operatorname{Pf}[\cdot]$ denotes the Pfaffian of an antisymmetric matrix
and $A\left(k_z\right) \equiv {U_o(-k_z)}^\dagger \hat{D} {T} U_o(k_z)$ with $T=\text{i} \sigma_y \kappa $,
$U_o (k_z) = (\ket{\psi_1 (k_z)},\ket{\psi_2 (k_z)},\dots,\ket{\psi_{N_{\text{occ} }}(k_z) } ) $ 
being a matrix consisting of occupied eigenstates of Hamiltonian (\ref{withtrs}) at the momentum $k_z$. 
Similar to the case without TRS, we need to perform the transformation of site positions for the
$\hat{D}$ matrix to evaluate the topological invariant. 

In Fig.~\ref{Fig3}(a), we plot the zero-temperature longitudinal conductance $G$ and the $\mathbb{Z}_2$
topological invariant. The figure clearly illustrates the existence of a topological regime identified by
the quantized conductance of $8 e^2/h$ and the quantized nontrivial value of the topological invariant.
The conductance is attributed to the eight helical pairs of hinge modes as shown in Fig.~\ref{Fig3}(b),
which are degenerate due to the $C_8 s_x$ symmetry. 

When $t_3=0$, the Hamiltonian (\ref{withtrs}) commutes with $s_y$ so that it can be written as a direct sum
of two copies of the Hamiltonian (\ref{h_without_trs}) with opposite signs of $g$.
We therefore can calculate the winding number of the quadrupole moment in each subspace
using the transformed site positions so as to evaluate the spin quadrupole moment winding number~\cite{Wang2021PRL}
to characterize the system's topology. In fact, despite the absence of the symmetry
when $t_3$ is nonzero, we can still compute the spin winding number 
and find that the results coincide with the $\mathbb{Z}_2$ invariant.

\emph{{\color{blue}Higher-order Weyl-like semimetal}}.---
We now proceed to introduce a model in the stack of 2D quasicrystals that exhibits both the first-order
surface modes and second-order hinge modes. The Hamiltonian reads
\begin{equation} \label{HWS}
	\hat{H}_W(M)=\sum_{\bm r}\hat{c}_{\bm r}^{\dagger} T_0 \hat{c}_{\bm r} +
	\sum_{{\bm d}} \hat{c}_{{\bm r}+{\bm d}}^{\dagger} T_W({\hat{\bm d}}) \hat{c}_{\bm r},
\end{equation}
where $T_0=M  \tau_z \sigma_0  + t_c  \tau_0 \sigma_z$ 
and $T_W({\hat{\bm d}})$ is the hopping matrix that 
reads $T_W(\hat{\bm {d}})= [ t_0 \tau_z \sigma_0 +i t_1 
(\cos \theta \tau_x \sigma_x +\sin \theta \tau_x \sigma_y) +g\cos(4 \theta) \tau_y \sigma_0 ]/2$
for the intra-layer hopping and 
$T_W(\hat{\bm {d}})=  t_0 \tau_z \sigma_0/2$
for the inter-layer hopping. 
Similar to the previous cases, we set $ t_0=t_1 = 1$.
The inter-layer hopping changes the mass $M$ to $M+t_0\cos k_z$.
Therefore, we can view the system as a stack of 2D systems on quasicrystalline lattices with
the mass controlled by $k_z$. Without $t_c$, the chiral symmetry $\Gamma= \tau_x\sigma_z$ 
(for the first-quantization Hamiltonian) is preserved so that each slice of a system at a fixed 
$k_z$ cannot develop quantum anomalous Hall insulating phase. In fact, in this case, 
the system can develop a four-fold degenerate point at the transition point between a normal insulator 
and a quadrupole insulator as $k_z$ varies, similar to the Dirac point in the regular case~\cite{Hughes2018PRB,Zeng2020PRB,Wieder2020NC}.
To generate the Weyl-like semimetal phase, we add the term $t_c  \tau_0 \sigma_z$ to break the chiral
symmetry, but leave the particle-hole symmetry 
$\Xi= \tau_x\sigma_x\kappa$ preserved. As a result, both the quadrupole insulator
and quantum anomalous Hall insulator can exist, similar to the higher-order Weyl semimetal~\cite{Jiang2020PRL,Hughes2020PRL}. 

Indeed, adding the $t_c$ term splits each four-fold degenerate point into two 
twofold degenerate ones. The split regions develop the quantum anomalous Hall insulating
phases characterized by the Bott index as shown in Fig.~\ref{Fig4}(a). Here, the Bott index
is defined as~\cite{Hastings2010EL}
\begin{equation}
	\operatorname{Bott} = \frac{1}{2\pi}\operatorname{Im}\operatorname{Tr}\log(U_y U_x U_y^{\dagger} U_x^{\dagger}),
\end{equation}
where $U_x$ and $U_y$ are given by $U_o^{\dagger} e^{2\pi i \hat{x}/L_x} U_o$ 
and $U_o^{\dagger} e^{2\pi i \hat{y}/L_y} U_o$, respectively, where $\hat{x}$ and $\hat{y}$ are
position operators. 
Apart from the first-order topological phases, the middle region with  $|k_z| \lesssim 1.2$
corresponds to the quadrupole insulator with the quadrupole moment of $0.5$, which
is calculated using the transformed site positions.
In this region, there exist eight hinge states at zero energy as shown in Fig.~\ref{Fig4}(b).

% Figure environment removed

To illustrate the real-space distribution of the hinge and surface states, we plot the $k_z$-resolved local DOS
at zero energy in Fig.~\ref{Fig4}(c). We see that the midgap states in the middle region in $k_z$ are mainly 
spatially localized on the hinges, while the states in the anomalous Hall region are localized on the surfaces. 
In contrast to the higher-order Weyl semimetal in a crystal~\cite{Jiang2020PRL,Hughes2020PRL}, this phase exhibits the local DOS with eight-fold
rotational symmetry as enforced by the ${C_8}{T}$ symmetry of the Hamiltonian.
Our phase is also different from the first-order quasicrystal Weyl-like semimetal
with only anomalous surface states~\cite{Fonseca2023arxiv}.
We therefore establish the bulk-edge correspondence for a higher-order Weyl-like semimetal in a stack of 2D
quasicrystals without crystalline counterparts. 

To further reveal Fermi arcs arsing from surface and hinge states, we calculate the spectral DOS
at the energy $E$~\cite{Fonseca2023arxiv},
$
	\rho (x, q_y, E, k_z) = \sum_{{\bm r}\in S_x } \bra{q_y, {\bm r},\alpha}\delta(E-{H}_W(k_z))\ket{q_y,{\bm r},\alpha},
$
where $S_x$ denotes the set of surface sites on a $x$-normal surface, 
and  $\ket{q_y,{\bm r},\alpha} $ is the plane wave with the momentum $q_y$. 
The quantity measures the DOS of the system that an incident plane wave of energy $E$ 
can couple to on the surface corresponding to 
an angle-resolved diffraction measurement~\cite{Fonseca2023arxiv}. 
Figure~\ref{Fig4}(d) shows the spectral DOS at zero energy,
illustrating the appearance of Fermi arcs of varying intensities with respect to $q_y$ 
that connect the projections of two degenerate points at $k_z \approx 2.2$. 
The spectral DOS distribution corresponds to the Bragg peaks of the structure factor~\cite{Fonseca2023arxiv}
$
	I(q_y)  \varpropto \sum_{q_x} | {\sum_{\bm R} e^{i {\bm q} \cdot {\bm R}}} |^2
	$ 
with lattice sites $ \bm R$ and the Fourier momentum $ \bm q = (q_x,q_y) $. 
Figure~\ref{Fig4}(e) shows that the positions of the Bragg peaks in $q_y$ agree perfectly
with those of the Fermi arcs in Fig.~\ref{Fig4}(d). See also the Supplemental Material for the 
bulk spectral DOS revealing the existence of the Weyl-like degenerate points~\cite{supplement}.

In summary, we have demonstrated the existence of 3D second-order topological insulators
and higher-order Weyl-like semimetals in a stack of 2D 
quasicrystals with eight-fold rotational symmetry. We have also established the
bulk-edge correspondence in these systems. Importantly,  
it is worth noting that such phases cannot appear in a 3D crystal due to the absence of eight-fold rotational symmetry.
Moreover, our findings can be generalized to other quasicrystals, 
such as Stampfli-tiling quasicrystals with twelve-fold rotational symmetry.
While our analysis focuses on the case with translational symmetry along the $z$-direction, 
the implications of our work may be significant for fully 3D quasicrystalline systems.
Additionally, these intriguing phases could potentially be experimentally observed not only in realistic 
quasicrystalline materials but also in metamaterials, such as phononic~\cite{Huber2018Nat}, photonic~\cite{Hafezi2019NP}, 
electric circuit~\cite{Thomale2018NP} and microwave systems~\cite{Bahl2018Nat}.

\emph{Note added}: Recently, we became aware of a related work~\cite{Chen2023arXiv} where higher-order Dirac semimetals in 
3D quasicrystals have been studied.

\begin{acknowledgments}
	The work is supported by the National Natural Science Foundation
	of China (Grant No. 11974201) and Tsinghua University Dushi Program.
\end{acknowledgments}

\begin{thebibliography}{99}
	%% higher-order
	\bibitem{Taylor2017Science}{W. A. Benalcazar, B. A. Bernevig, and T. L. Hughes, 
		%Quantized electric multipole insulators, 
		%\href{https://science.sciencemag.org/content/357/6346/61.full}
		{Science \textbf{357,} 61 (2017)}.}  %%2D
	
	\bibitem{Taylor2017PRB}{W. A. Benalcazar, B. A. Bernevig, and T. L. Hughes, 
		%Electric multipole moments, topological multipole moment pumping, and chiral hinge states in crystalline insulators, 
		%\href{https://doi.org/10.1103/PhysRevB.96.245115}
		{Phys. Rev. B \textbf{96,} 245115 (2017)}.}  %%2D
	
	\bibitem{Fritz2012PRL}{M. Sitte, A. Rosch, E. Altman, and L. Fritz, 
		%Topological Insulators in Magnetic Fields: Quantum Hall Effect and Edge Channels with a Nonquantized $\theta$ Term, 
		%\href{https://journals.aps.org/prl/abstract/10.1103/PhysRevLett.108.126807}
		{Phys. Rev. Lett. \textbf{108,} 126807 (2012)}.}%%3D
	
	\bibitem{ZhangFan2013PRL}{F. Zhang, C. L. Kane, and E. J. Mele, 
		%Surface State Magnetization and Chiral Edge States on Topological Insulators, 
		%\href{https://journals.aps.org/prl/abstract/10.1103/PhysRevLett.110.046404}
		{Phys. Rev. Lett. \textbf{110,} 046404 (2013)}.}%% 3D
	
	\bibitem{Slager2015PRB}{R.-J. Slager, L. Rademaker, J. Zaanen, and L. Balents, 
		%Impurity-bound states and Green's function zeros as local signatures of topology, 
		%\href{https://journals.aps.org/prb/abstract/10.1103/PhysRevB.92.085126}
		{Phys. Rev. B \textbf{92,} 085126 (2015).}} %%2D 3D
	
	\bibitem{Brouwer2017PRL}{J. Langbehn, Y. Peng, L. Trifunovic, F. von Oppen, and P. W. Brouwer, 
		%Reflection-Symmetric Second-Order Topological Insulators and Superconductors, 
		%\href{https://journals.aps.org/prl/abstract/10.1103/PhysRevLett.119.246401}
		{Phys. Rev. Lett. \textbf{119,} 246401 (2017)}.} %%2D 3D
	
	\bibitem{FangChen2017PRL}{Z. Song, Z. Fang, and C. Fang, 
		%$(d-2)$-Dimensional Edge States of Rotation Symmetry Protected Topological States, 
		%\href{https://journals.aps.org/prl/abstract/10.1103/PhysRevLett.119.246402}
		{Phys. Rev. Lett. \textbf{119,} 246402 (2017)}.}%%2D 3D
	
	\bibitem{Schindler2018SA}{F. Schindler, A. M. Cook, M. G. Vergniory, Z. Wang, S. S. P. Parkin, B. A. Bernevig, and T. Neupert, 
		%Higher-Order Topological Insulators, 
		%\href{https://advances.sciencemag.org/content/4/6/eaat0346}
		{Sci. Adv. \textbf{4,} eaat0346 (2018)}.}%%%%%%%%%%%% the nested entanglement spectrum         
	
%	\bibitem{Wang2018EL}{Q. Wang, D. Wang, and Q.-H. Wang, 
		%Entanglement in a second-order topological insulator on a square lattice, 
		%\href{https://doi.org/10.1209/0295-5075/124/50005}
%		{Europhys. Lett. \textbf{124,} 50005 (2018)}.}
	
	\bibitem{Brouwer2019PRX}{L. Trifunovic and P. W. Brouwer, 
		%Higher-Order Bulk-Boundary Correspondence for Topological Crystalline Phases, 
		%\href{https://journals.aps.org/prx/abstract/10.1103/PhysRevX.9.011012}
		{Phys. Rev. X \textbf{9,} 011012 (2019)}.}%% 2D 3D
	
	\bibitem{Seradjeh2019PRB}{M. Rodriguez-Vega, A. Kumar, and B. Seradjeh, 
		%Higher-order Floquet topological phases with corner and bulk bound states, 
		%\href{https://journals.aps.org/prb/abstract/10.1103/PhysRevB.100.085138}
		{Phys. Rev. B \textbf{100,} 085138 (2019)}.}
	
	\bibitem{Roy2019PRB}{D. C{\v{a}}lug{\v{a}}ru, V. Juri{\v{c}}i{\'c}, and B. Roy, 
		%Higher-order topological phases: A general principle of construction, 
		%\href{https://journals.aps.org/prb/abstract/10.1103/PhysRevB.99.041301}
		{Phys. Rev. B \textbf{99,} 041301(R) (2019)}.} %% 3D
	
	
	\bibitem{Yang2019PRL}{X.-L. Sheng, C. Chen, H. Liu, Z. Chen, Z.-M. Yu, Y. X. Zhao, and S. A. Yang, 
		%Two-Dimensional Second-Order Topological Insulator in Graphdiyne, 
		%\href{https://journals.aps.org/prl/abstract/10.1103/PhysRevLett.123.256402}
		{Phys. Rev. Lett. \textbf{123,} 256402 (2019)}.} %2D 6 corner
	
	\bibitem{Hughes2020PRB}{P. Zhu, K. Loehr, and Taylor L. Hughes, 
		%Identifying $C_n$-symmetric higher-order topology and fractional corner charge using entanglement spectra, 
		%\href{https://doi.org/10.1103/PhysRevB.101.115140}
		{Phys. Rev. B \textbf{101,} 115140 (2020)}.} 
	
	
	\bibitem{Xu2020PPR}{Y.-B. Yang, K. Li, L.-M. Duan, and Y. Xu, 
		%Type-II quadrupole topological insulators, 
		%\href{https://journals.aps.org/prresearch/abstract/10.1103/PhysRevResearch.2.033029}
		{Phys. Rev. Research \textbf{2,} 033029 (2020)}.}%% 2D
	
	\bibitem{Parameswaran2020PRL}{A. Tiwari, M.-H. Li, B. A. Bernevig, T. Neupert, and S. A. Parameswaran, 
		%Unhinging the Surfaces of Higher-Order Topological Insulators and Superconductors, 
		%\href{https://journals.aps.org/prl/abstract/10.1103/PhysRevLett.124.046801}
		{Phys. Rev. Lett. \textbf{124,} 046801 (2020)}.}%% 3D 1.27
	
	\bibitem{Zhao2020PRL}{K. Wang, J.-X. Dai, L. B. Shao, S. A. Yang, and Y. X. Zhao, 
		%Boundary Criticality of PT -Invariant Topology and Second-Order Nodal-Line Semimetals, 
		%\href{https://doi.org/10.1103/PhysRevLett.125.126403}
		{Phys. Rev. Lett. \textbf{125,} 126403 (2020)}.} 
	
	\bibitem{AYang2020PRL}{C. Chen, Z. Song, J.-Z. Zhao, Z. Chen, Z.-M. Yu, X.-L. Sheng, and S. A. Yang, 
		%Universal Approach to Magnetic Second-Order Topological Insulator, 
		%\href{https://journals.aps.org/prl/abstract/10.1103/PhysRevLett.125.056402}
		{Phys. Rev. Lett. \textbf{125,} 056402 (2020)}.}%% 2D 6 corner
	
	\bibitem{Xu2020NJP}{Y.-L. Tao, N. Dai, Y.-B. Yang, Q.-B. Zeng, and Y. Xu, 
		%Hinge solitons in three-dimensional second-order topological insulators, 
		%\href{https://iopscience.iop.org/article/10.1088/1367-2630/abc1f9}
		{New J. Phys. \textbf{22,} 103058 (2020)}.} %%3D
	%%%%%%%%% disorder	
	\bibitem{Cho2020arXiv}{B. Kang, W. Lee, and G. Y. Cho,
	%Many-Body Invariants for Chern and Chiral Hinge Insulators,
	Phys. Rev. Lett. \textbf{126}, 016402 (2021).}

    \bibitem{Yang2023Review}
    Y.-B. Yang, J.-H. Wang, K. Li, and Y. Xu,
    arXiv:2309.03688 (2023).
    
	%%%%%%%%%% Weyl semimetal
\bibitem{Jiang2020PRL}{H.-X. Wang, Z.-K. Lin, B. Jiang, G.-Y. Guo, and J.-H. Jiang, 
	%Higher-Order Weyl Semimetals, 
	%\href{https://doi.org/10.1103/PhysRevLett.125.146401}
	{Phys. Rev. Lett. \textbf{125,} 146401 (2020)}.}%%	

\bibitem{Hughes2020PRL}{S. A. A. Ghorashi, T. Li, and T. L. Hughes, 
	%Higher-Order Weyl Semimetals, 
	%\href{https://doi.org/10.1103/PhysRevLett.125.266804}
	{Phys. Rev. Lett. \textbf{125,} 266804 (2020)}.}%%	

\bibitem{Wei2021NM}{Q. Wei, X. Zhang, W. Dong, J. Lu, X. Huang, M. Yan, G. Chen, Z. Liu, and S. Jia, 
	%Higher-order topological semimetal in acoustic crystals, 
	%\href{https://doi.org/10.1038/s41563-021-00933-4}
	{Nat. Mater. \textbf{20,} 812 (2021)}.} %%

\bibitem{Luo2021NM}{L. Luo, H.-X. Wang, Z.-K. Lin, B. Jiang, Y. Wu, F. Li, and J.-H. Jiang , 
	%Observation of a phononic higher-order Weyl semimetal, 
	%\href{https://doi.org/10.1038/s41563-021-00985-6}
	{Nat. Mater. \textbf{20,} 794 (2021)}.} %%

	%%%%%%%%%%%	Quasicrystal
	\bibitem{Fulga2019PRL}{D. Varjas, A. Lau, K. P{\"o}yh{\"o}nen, A. R. Akhmerov, D. I. Pikulin, and I. C. Fulga, 
		%Topological Phases without Crystalline Counterparts, 
		%\href{https://journals.aps.org/prl/abstract/10.1103/PhysRevLett.123.196401}
		{Phys. Rev. Lett. \textbf{123,} 196401 (2019)}.}%%
	
	\bibitem{Xu2020PRL}{R. Chen, C.-Z. Chen, J.-H. Gao, B. Zhou, and D.-H. Xu, 
		%Higher-Order Topological Insulators in Quasicrystals, 
		%\href{https://journals.aps.org/prl/abstract/10.1103/PhysRevLett.124.036803}
		{Phys. Rev. Lett. \textbf{124,} 036803 (2020)}.}%% 
	
	\bibitem{Xu2020PRB}{C.-B. Hua, R. Chen, B. Zhou, and D.-H. Xu, 
		%Higher-order topological insulator in a dodecagonal quasicrystal, 
		%\href{https://doi.org/10.1103/PhysRevB.102.241102}
		{Phys. Rev. B \textbf{102,} 241102(R) (2020)}.}%% 
	
	\bibitem{Huang2021NL}{H. Huang, J. Fan, D. Li, and F. Liu, 
		%Generic Orbital Design of Higher-Order Topological QuasicrystallineInsulators with Odd Five-Fold Rotation Symmetry, 
		%\href{https://doi.org/10.1021/acs.nanolett.1c02661}
		{Nano Lett. \textbf{21,} 7056 (2021)}.}%% 
	
	\bibitem{Huang2022FP}{J. Fan and H. Huang, 
		%Topological states in quasicrystals, 
		%\href{https://doi.org/10.1007/s11467-021-1100-y}
		{Front. Phys. \textbf{17,} 13203 (2022)}.}%% 
	
	\bibitem{Huang2022PRL}{C. Wang, F. Liu, and H. Huang, 
		%Effective Model for Fractional Topological Corner Modes in Quasicrystals, 
		%\href{https://doi.org/10.1103/PhysRevLett.129.056403}
		{Phys. Rev. Lett. \textbf{129,} 056403 (2022)}.}%% 
	
	%%%%%%%%%%%	Amorphous
	\bibitem{Roy2020PPR}{A. Agarwala, V. Juri{\v{c}}i{\'c}, and B. Roy, 
		%Higher-order topological insulators in amorphous solids, 
		%\href{https://journals.aps.org/prresearch/abstract/10.1103/PhysRevResearch.2.012067}
		{Phys. Rev. Research \textbf{2,} 012067(R) (2020)}.} %%
	
	\bibitem{Wang2021PRL}{J.-H. Wang, Y.-B. Yang, N. Dai, and Y. Xu, 
		%Structural-Disorder-Induced Second-Order Topological Insulators in Three Dimensions, 
		%\href{https://journals.aps.org/prl/abstract/10.1103/PhysRevLett.126.206404}
		{Phys. Rev. Lett. \textbf{126,} 206404 (2021)}.}%%
	
	\bibitem{Zhou2022PRB}{T. Peng, C.-B. Hua, R. Chen, Z.-R. Liu, H.-M. Huang, and B. Zhou, 
		%Density-driven higher-order topological phase transitions in amorphous solids, 
		%\href{https://doi.org/10.1103/PhysRevB.106.125310}
		{Phys. Rev. B \textbf{106,} 125310 (2022)}.} %%
	
	\bibitem{Tao2023arxiv}{Y.-L. Tao, J.-H. Wang, and Y. Xu, 
		%Average Symmetry Protected Higher-order Topological Amorphous Insulators, 
		%\href{https://doi.org/10.48550/arXiv.2306.02246}
		{arXiv: 2306.02246 (2023).}} %%
	%%%%%%%%%%% hyperbolic
	\bibitem{Xu2023PRB}{Y.-L. Tao and Y. Xu, 
		%Higher-order topological hyperbolic lattices, 
		%\href{https://doi.org/10.1103/PhysRevB.107.184201}
		{Phys. Rev. B \textbf{107,} 184201 (2023)}.} %%
	
	\bibitem{Zhou2023PRB}{Z.-R. Liu, C.-B. Hua, T. Peng, R. Chen, and B. Zhou, 
		%Higher-order topological insulators in hyperbolic lattices, 
		%\href{https://doi.org/10.1103/PhysRevB.107.125302}
		{Phys. Rev. B \textbf{107,} 125302 (2023)}.} %%
	
	%%%%%%%%%%%%%  transmission probability
	\bibitem{Dattabook}{S. Datta, 
		\emph{Electronic Transport in Mesoscopic Systems} (Cambridge University Press, Cambridge, UK, 1997)} %%
	
	\bibitem{Xing2007PRB}{Y. Xing, Q.-F. Sun, and J. Wang, 
		%Symmetry and transport property of spin current induced spin-Hall effect, 
		%\href{https://doi.org/10.1103/PhysRevB.75.075324}
		{Phys. Rev. B \textbf{75,} 075324 (2007)}.} %%
	
	\bibitem{Yanbin2019PRL}
	{Y.-B. Yang, T. Qin, D.-L. Deng, L.-M. Duan, and Y. Xu,
	Phys. Rev. Lett. 123, 076401 (2019).}
	
	%%%%%%%%%%%% Quadrupole moment
	\bibitem{Cho2019PRB}{B. Kang, K. Shiozaki, and G. Y. Cho, 
		%Many-body order parameters for multipoles in solids, 
		%\href{https://journals.aps.org/prb/abstract/10.1103/PhysRevB.100.245134}
		{Phys. Rev. B \textbf{100,} 245134 (2019)}.}%% 
	
	\bibitem{Hughes2019PRB}{W. A. Wheeler, L. K. Wagner, and T. L. Hughes, 
		%Many-body electric multipole operators in extended systems, 
		%\href{https://journals.aps.org/prb/abstract/10.1103/PhysRevB.100.245135}
		{Phys. Rev. B \textbf{100,} 245135 (2019)}.}%% 
	
		\bibitem{Xu2021PRB}{Y.-B. Yang, K. Li, L.-M. Duan, and Y. Xu, 
		%Higher-order topological Anderson insulators, 
		%\href{https://journals.aps.org/prb/abstract/10.1103/PhysRevB.103.085408}
		{Phys. Rev. B \textbf{103,} 085408 (2021)}.} %%2D
	
	\bibitem{Shen2020PRL}{C.-A. Li, B. Fu, Z.-A. Hu, J. Li, and S.-Q. Shen, 
		%Topological Phase Transitions in Disordered Electric Quadrupole Insulators, 
		%\href{https://journals.aps.org/prl/abstract/10.1103/PhysRevLett.125.166801}
		{Phys. Rev. Lett. \textbf{125,} 166801 (2020).}}
	
	\bibitem{supplement}See the Supplemental Material.
	
	%%%%%%%%%% Dirac semimetal
	\bibitem{Hughes2018PRB}{M. Lin and T. L. Hughes, 
		%Topological quadrupolar semimetals, 
		%\href{https://doi.org/10.1103/PhysRevB.98.241103}
		{Phys. Rev. B \textbf{98,} 241103(R) (2018)}.}%% 
	
	\bibitem{Zeng2020PRB} {Q.-B. Zeng, Y.-B. Yang, and Y. Xu,
	Phys. Rev. B 101, 241104(R) (2020). }
	
	\bibitem{Wieder2020NC} B. J. Wieder, Z. Wang, J. Cano, et al. 
	%Strong and fragile topological Dirac semimetals with higher-order Fermi arcs. 
	Nat Commun \textbf{11,} 627 (2020).
	%%%%%%%% Bott
	\bibitem{Hastings2010EL}{T. A. Loring and M. B. Hastings, 
		%Disordered topological insulators via $C^*$-algebras, 
		%\href{https://doi.org/10.1209/0295-5075/92/67004}
		{Europhys. Lett. \textbf{92,} 67004 (2010)}.}%%
	
	%%%%%%% Weyl semimetal first-order
	\bibitem{Fonseca2023arxiv}{A. G. e Fonseca, T. Christensen, J. D. Joannopoulos, and M. Solja{\v{c}}i{\'c}, 
		%Quasicrystalline Weyl points and dense Fermi–Bragg arcs, 
		%\href{https://doi.org/10.48550/arXiv.2211.14299}
		{arXiv: 2211.14299 (2023).}} %%
	
	%%%%%%%%%%% higher-order experiment
	
	\bibitem{Huber2018Nat}{M. Serra-Garcia, V. Peri, R. S\"usstrunk, O. R. Bilal, T. Larsen, L. G. Villanueva, and S. D. Huber, 
		%Observation of a phononic quadrupole topological insulator, \href{https://www.nature.com/articles/nature25156}
		{Nature (London) \textbf{555,} 342 (2018)}.} %% phononic
	
	\bibitem{Hafezi2019NP}{S. Mittal, V. V. Orre, G. Zhu, M. A. Gorlach, A. Poddubny, and M. Hafezi, 
		%Photonic quadrupole topological phases, \href{https://www.nature.com/articles/s41566-019-0452-0}
		{Nat. Photonics \textbf{13,} 692 (2019)}.} %% CROW
	
	\bibitem{Thomale2018NP}{S. Imhof, C. Berger, F. Bayer, J. Brehm, L. W. Molenkamp, T. Kiessling, F. Schindler, C. H. Lee, M. Greiter, T. Neupert, and R. Thomale, 
		%Topolectrical-circuit realization of topological corner modes, \href{https://www.nature.com/articles/s41567-018-0246-1}
		{Nat. Phys. \textbf{14,} 925 (2018)}.} %% circuit
	
	\bibitem{Bahl2018Nat}{C. W. Peterson, W. A. Benalcazar, T. L. Hughe, and G. Bahl, 
		%A quantized microwave quadrupole insulator with topologically protected corner states, \href{https://www.nature.com/articles/nature25777#citeas}
		{Nature (London) \textbf{555,} 346 (2018)}.} %% microwave
	
	\bibitem{Chen2023arXiv}{R. Chen, B. Zhou, and D.-H. Xu, arXiv: 2307.04334 (2023).}
	
\end{thebibliography}

\begin{widetext}
	%%%%%%%%%% Prefix a "S" to all equations, figures, tables and reset the counter %%%%%%%%%%
	\setcounter{equation}{0} \setcounter{figure}{0} \setcounter{table}{0} %
	%\setcounter{page}{1} \makeatletter
	\renewcommand{\theequation}{S\arabic{equation}} \renewcommand{\thefigure}{S%
		\arabic{figure}}
	%\renewcommand{\bibnumfmt}[1]{[S#1]} \renewcommand{%
	%\citenumfont}[1]{S#1}
	%%%%%%%%%% Prefix a "S" to all equations, figures, tables and reset the counter %%%%%%%%%%
	
	In the Supplemental Material, we will show in detail how the periodic boundary conditions (PBCs)
	and site position transformations are realized in Section S-1
	and provide the bulk spectral DOS in Section S-2.
	
	\section{S-1. Details on constructions of periodic boundary conditions and site position transformations}
	In the main text, we have applied PBCs in the $(x,y)$ plane to 
	calculate the quadrupole moment and the Bott index. Figure~\ref{pbc_fig} shows how the PBCs 
	are realized in the Ammann-Beenker tiling quasicrystalline lattices. Specifically, the sites at a boundary 
	are connected to the sites at the other boundary. For example, the sites $B$ and $C$ at a boundary are 
	connected to the sites $B^\prime$ and $C^\prime$ at the other boundary, respectively. 
	The corner sites are connected to the other two corner sites, e.g., the corner site $A$ is connected to 
	both the sites $A^\prime$ and $A^{\prime \prime}$.
	
	In the main text, we use the winding number of the quadrupole moment and the $\mathbb{Z}_2$ topological invariant
	to characterize the topological phases. Their calculations involve the evaluation of the $\hat{D}$ matrix dependent 
	on the site positions. If we use original lattice 
	site positions in a quasicrystal, we obtain zero quadrupole moment [see Fig.~\ref{pbc_fig}(b)].
	To obtain the reliable quadrupole moment, we perform the transformation of site positions~\cite{Tao2023arxivS}.
	Specifically, we move the sites in the light green (blue) region in Fig.~\ref{pbc_fig}(c) to 
	the first or third (second or fourth) quadrant in Fig.~\ref{pbc_fig}(d) while keeping its
	distance from the center unchanged. After that, we modify the $\hat{D}$ matrix accordingly. 
	
	% Figure environment removed
	
	\section{S-2. Weyl-like points revealed by the bulk spectral DOS}
	In the main text, we have shown that the Fermi arcs appear in the spectral DOS resolved in the Fourier momentum $q_y$
	when we consider surface sites as the coupling sites. In this section, we provide the spectral DOS at zero
	energy by considering the bulk sites as the coupling sites in Fig.~\ref{bragg} and find that the Weyl-like degenerate points
	manifest in the spectral DOS as bright spots. The figure also illustrates that the positions of Bragg peaks of the
	structure factor in $q_y$ agree perfectly with those of Weyl-like points.
	
	% Figure environment removed
	
	
	\begin{thebibliography}{99}
		
		\bibitem{Tao2023arxivS}{Y.-L. Tao, J.-H. Wang, and Y. Xu, 
			%Average Symmetry Protected Higher-order Topological Amorphous Insulators, 
			%\href{https://doi.org/10.48550/arXiv.2306.02246}
			{arXiv: 2306.02246 (2023).}} %%
		
	\end{thebibliography}
	
\end{widetext}

\end{document}




