\section{Related Work}
The denoising U-Net~\cite{DBLP:conf/nips/HoJA20,DBLP:conf/miccai/RonnebergerFB15} is an essential design to the success of the DDPMs in generation tasks~\cite{DBLP:conf/cvpr/RombachBLEO22,DBLP:conf/icml/NicholD21,DBLP:conf/nips/DhariwalN21,DBLP:journals/corr/abs-2207-12598,DBLP:conf/icml/NicholDRSMMSC22}.
As important building blocks, 2D convolutions~\cite{DBLP:conf/cvpr/HeZRS16} and spatial self-attention~\cite{DBLP:conf/nips/VaswaniSPUJGKP17} effectively extract intermediate features from images, proven successful for the denoising task.
% with attentional capability which is beneficial to the denoising task.
Video diffusion models~\cite{DBLP:conf/nips/HoSGC0F22,DBLP:journals/corr/abs-2209-14792,DBLP:journals/corr/abs-2210-02303}, inspired by video understanding models 
% such activity recognition
~\cite{DBLP:conf/iccv/TranBFTP15,DBLP:conf/cvpr/CarreiraZ17,DBLP:conf/cvpr/TranWTRLP18,li2019global,li2019multi,chen2020temporal},
% and personal re-identification~\cite{},
propose a new type of U-Net architecture following the principle of jointly extracting spatial-temporal information from video frames.
Such developments aim to bridge the gap between tasks (\textit{e.g.}, from image to video) with a minimum design effort in optimization recipe, in which the domain/modality shift of the data may pose a non-trivial task to the model training if the network architecture remain unchanged.

Wavelet-based deep learning approaches~\cite{DBLP:conf/icmla/WilliamsL16,DBLP:conf/eccv/YaoPLNM22,DBLP:journals/corr/abs-2102-06108} have shown great potential in providing inherent advantages that are not available in the pixel space, inspiring researchers to 
% Recently, ~\cite{} 
incorporate wavelets in diffusion models for generation tasks.
WaveDiff~\cite{DBLP:journals/corr/abs-2211-16152} builds upon a GAN-based method, DDGAN~\cite{DBLP:conf/iclr/XiaoKV22}, and incorporates wavelet transformation from the perspective of image compression, hence achieving a better trade-off between efficiency and sample quality.
DiWa~\cite{DBLP:journals/corr/abs-2304-01994} combines wavelets and diffusion model to improve image super-resolution by leveraging the power of high-frequency information for detail enhancement.
~\cite{DBLP:journals/corr/abs-2302-00190} proposes a diffusion model on a continuous implicit representation in wavelet space for 3D shape generation.
% These works share a similar concept in the spirit of exploring the  wavelets space instead of the pixel space in diffusion models.
% However, wavelets in these works are used for different reasons and play different roles.
Among them, WaveDiff is perhaps the most relevant study to ours.
However, one important difference in design principle between our \ourmodel and WaveDiff is that: \ourmodel does not perform discrete/inverse wavelet transform (D/IWT) within each computational block. We note that, it may not be an appropriate practice to embed DWT and IWT to intermediate features, in which case, DWT and IWT just serves as differentiable linear operators.  
Moreover, with GAN~\cite{goodfellow2014generative,DBLP:conf/iclr/XiaoKV22} components embedded in WaveDiff, including the auxiliary discriminator and the adversarial training objectives, one may not have the flexibility to further explore whether the noisy wavelet signals can be effectively recovered only through the reversed diffusion process, which is the question we aim to answer in this paper.






