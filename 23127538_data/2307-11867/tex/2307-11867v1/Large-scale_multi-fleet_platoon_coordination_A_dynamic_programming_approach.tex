
%% bare_jrnl.tex
%% V1.4b
%% 2015/08/26
%% by Michael Shell
%% see http://www.michaelshell.org/
%% for current contact information.
%%
%% This is a skeleton file demonstrating the use of IEEEtran.cls
%% (requires IEEEtran.cls version 1.8b or later) with an IEEE
%% journal paper.
%%
%% Support sites:
%% http://www.michaelshell.org/tex/ieeetran/
%% http://www.ctan.org/pkg/ieeetran
%% and
%% http://www.ieee.org/

%%*************************************************************************
%% Legal Notice:
%% This code is offered as-is without any warranty either expressed or
%% implied; without even the implied warranty of MERCHANTABILITY or
%% FITNESS FOR A PARTICULAR PURPOSE! 
%% User assumes all risk.
%% In no event shall the IEEE or any contributor to this code be liable for
%% any damages or losses, including, but not limited to, incidental,
%% consequential, or any other damages, resulting from the use or misuse
%% of any information contained here.
%%
%% All comments are the opinions of their respective authors and are not
%% necessarily endorsed by the IEEE.
%%
%% This work is distributed under the LaTeX Project Public License (LPPL)
%% ( http://www.latex-project.org/ ) version 1.3, and may be freely used,
%% distributed and modified. A copy of the LPPL, version 1.3, is included
%% in the base LaTeX documentation of all distributions of LaTeX released
%% 2003/12/01 or later.
%% Retain all contribution notices and credits.
%% ** Modified files should be clearly indicated as such, including  **
%% ** renaming them and changing author support contact information. **
%%*************************************************************************


% *** Authors should verify (and, if needed, correct) their LaTeX system  ***
% *** with the testflow diagnostic prior to trusting their LaTeX platform ***
% *** with production work. The IEEE's font choices and paper sizes can   ***
% *** trigger bugs that do not appear when using other class files.       ***                          ***
% The testflow support page is at:
% http://www.michaelshell.org/tex/testflow/



\documentclass[journal]{IEEEtran}
%
% If IEEEtran.cls has not been installed into the LaTeX system files,
% manually specify the path to it like:
% \documentclass[journal]{../sty/IEEEtran}


% Some very useful LaTeX packages include:
% (uncomment the ones you want to load)

% *** MISC UTILITY PACKAGES ***
%
%\usepackage{ifpdf}
% Heiko Oberdiek's ifpdf.sty is very useful if you need conditional
% compilation based on whether the output is pdf or dvi.
% usage:
% \ifpdf
%   % pdf code
% \else
%   % dvi code
% \fi
% The latest version of ifpdf.sty can be obtained from:
% http://www.ctan.org/pkg/ifpdf
% Also, note that IEEEtran.cls V1.7 and later provides a builtin
% \ifCLASSINFOpdf conditional that works the same way.
% When switching from latex to pdflatex and vice-versa, the compiler may
% have to be run twice to clear warning/error messages.


% *** CITATION PACKAGES ***
%
%\usepackage{cite}
% cite.sty was written by Donald Arseneau
% V1.6 and later of IEEEtran pre-defines the format of the cite.sty package
% \cite{} output to follow that of the IEEE. Loading the cite package will
% result in citation numbers being automatically sorted and properly
% "compressed/ranged". e.g., [1], [9], [2], [7], [5], [6] without using
% cite.sty will become [1], [2], [5]--[7], [9] using cite.sty. cite.sty's
% \cite will automatically add leading space, if needed. Use cite.sty's
% noadjust option (cite.sty V3.8 and later) if you want to turn this off
% such as if a citation ever needs to be enclosed in parenthesis.
% cite.sty is already installed on most LaTeX systems. Be sure and use
% version 5.0 (2009-03-20) and later if using hyperref.sty.
% The latest version can be obtained at:
% http://www.ctan.org/pkg/cite
% The documentation is contained in the cite.sty file itself.




% *** GRAPHICS RELATED PACKAGES ***
%
\ifCLASSINFOpdf
  % \usepackage[pdftex]{graphicx}
  % declare the path(s) where your graphic files are
  % \graphicspath{{../pdf/}{../jpeg/}}
  % and their extensions so you won't have to specify these with
  % every instance of \includegraphics
  % \DeclareGraphicsExtensions{.pdf,.jpeg,.png}
\else
  % or other class option (dvipsone, dvipdf, if not using dvips). graphicx
  % will default to the driver specified in the system graphics.cfg if no
  % driver is specified.
  % \usepackage[dvips]{graphicx}
  % declare the path(s) where your graphic files are
  % \graphicspath{{../eps/}}
  % and their extensions so you won't have to specify these with
  % every instance of \includegraphics
  % \DeclareGraphicsExtensions{.eps}
\fi
% graphicx was written by David Carlisle and Sebastian Rahtz. It is
% required if you want graphics, photos, etc. graphicx.sty is already
% installed on most LaTeX systems. The latest version and documentation
% can be obtained at: 
% http://www.ctan.org/pkg/graphicx
% Another good source of documentation is "Using Imported Graphics in
% LaTeX2e" by Keith Reckdahl which can be found at:
% http://www.ctan.org/pkg/epslatex
%
% latex, and pdflatex in dvi mode, support graphics in encapsulated
% postscript (.eps) format. pdflatex in pdf mode supports graphics
% in .pdf, .jpeg, .png and .mps (metapost) formats. Users should ensure
% that all non-photo figures use a vector format (.eps, .pdf, .mps) and
% not a bitmapped formats (.jpeg, .png). The IEEE frowns on bitmapped formats
% which can result in "jaggedy"/blurry rendering of lines and letters as
% well as large increases in file sizes.
%
% You can find documentation about the pdfTeX application at:
% http://www.tug.org/applications/pdftex


% *** MATH PACKAGES ***
%
% \usepackage{amsmath}
% A popular package from the American Mathematical Society that provides
% many useful and powerful commands for dealing with mathematics.
%
% Note that the amsmath package sets \interdisplaylinepenalty to 10000
% thus preventing page breaks from occurring within multiline equations. Use:
%\interdisplaylinepenalty=2500
% after loading amsmath to restore such page breaks as IEEEtran.cls normally
% does. amsmath.sty is already installed on most LaTeX systems. The latest
% version and documentation can be obtained at:
% http://www.ctan.org/pkg/amsmath


% *** SPECIALIZED LIST PACKAGES ***
%
%\usepackage{algorithmic}
% algorithmic.sty was written by Peter Williams and Rogerio Brito.
% This package provides an algorithmic environment fo describing algorithms.
% You can use the algorithmic environment in-text or within a figure
% environment to provide for a floating algorithm. Do NOT use the algorithm
% floating environment provided by algorithm.sty (by the same authors) or
% algorithm2e.sty (by Christophe Fiorio) as the IEEE does not use dedicated
% algorithm float types and packages that provide these will not provide
% correct IEEE style captions. The latest version and documentation of
% algorithmic.sty can be obtained at:
% http://www.ctan.org/pkg/algorithms
% Also of interest may be the (relatively newer and more customizable)
% algorithmicx.sty package by Szasz Janos:
% http://www.ctan.org/pkg/algorithmicx


% *** ALIGNMENT PACKAGES ***
%
%\usepackage{array}
% Frank Mittelbach's and David Carlisle's array.sty patches and improves
% the standard LaTeX2e array and tabular environments to provide better
% appearance and additional user controls. As the default LaTeX2e table
% generation code is lacking to the point of almost being broken with
% respect to the quality of the end results, all users are strongly
% advised to use an enhanced (at the very least that provided by array.sty)
% set of table tools. array.sty is already installed on most systems. The
% latest version and documentation can be obtained at:
% http://www.ctan.org/pkg/array


% IEEEtran contains the IEEEeqnarray family of commands that can be used to
% generate multiline equations as well as matrices, tables, etc., of high
% quality.


% *** SUBFIGURE PACKAGES ***
%\ifCLASSOPTIONcompsoc
%  \usepackage[caption=false,font=normalsize,labelfont=sf,textfont=sf]{subfig}
%\else
%  \usepackage[caption=false,font=footnotesize]{subfig}
%\fi
% subfig.sty, written by Steven Douglas Cochran, is the modern replacement
% for subfigure.sty, the latter of which is no longer maintained and is
% incompatible with some LaTeX packages including fixltx2e. However,
% subfig.sty requires and automatically loads Axel Sommerfeldt's caption.sty
% which will override IEEEtran.cls' handling of captions and this will result
% in non-IEEE style figure/table captions. To prevent this problem, be sure
% and invoke subfig.sty's "caption=false" package option (available since
% subfig.sty version 1.3, 2005/06/28) as this is will preserve IEEEtran.cls
% handling of captions.
% Note that the Computer Society format requires a larger sans serif font
% than the serif footnote size font used in traditional IEEE formatting
% and thus the need to invoke different subfig.sty package options depending
% on whether compsoc mode has been enabled.
%
% The latest version and documentation of subfig.sty can be obtained at:
% http://www.ctan.org/pkg/subfig




% *** FLOAT PACKAGES ***
%
%\usepackage{fixltx2e}
% fixltx2e, the successor to the earlier fix2col.sty, was written by
% Frank Mittelbach and David Carlisle. This package corrects a few problems
% in the LaTeX2e kernel, the most notable of which is that in current
% LaTeX2e releases, the ordering of single and double column floats is not
% guaranteed to be preserved. Thus, an unpatched LaTeX2e can allow a
% single column figure to be placed prior to an earlier double column
% figure.
% Be aware that LaTeX2e kernels dated 2015 and later have fixltx2e.sty's
% corrections already built into the system in which case a warning will
% be issued if an attempt is made to load fixltx2e.sty as it is no longer
% needed.
% The latest version and documentation can be found at:
% http://www.ctan.org/pkg/fixltx2e


%\usepackage{stfloats}
% stfloats.sty was written by Sigitas Tolusis. This package gives LaTeX2e
% the ability to do double column floats at the bottom of the page as well
% as the top. (e.g., "% Figure environment removed

Based on the discrete decision and state spaces obtained above, the optimal solution to the problem (\ref{Eq.2}) is presented by the following Theorem~\ref{Theorem1}.  
\begin{theorem}\label{Theorem1}
The optimal solution to the problem in (\ref{Eq.2}) can be obtained by solving the following BOE. The optimal value function at the terminal stage is
\begin{align}
    J_{i,N_i}^{*}\big(t_{i,N_i}^a\big)=g_{i,N_i}(t_{i,N_i}^a), \quad ~\text{for all}~t_{i,N_i}^a\!\!\in\!\Gamma_{i,N_i}^S.\label{Eq.14}
\end{align}
The optimal value functions at stages $m\!=k,\dots,N_i\!-\!1$ are
\begin{align}
    &\!\!\!J_{i,m}^{*}\big(t_{i,m}^a\big)\nonumber\\
    &\!\!=\max\limits_{t_{i,m}^w\in\Gamma_{i,m}^D(t_{i,m}^a)}\!\!Q_{i,m}\big(t_{i,m}^a,t_{i,m}^w\big),\quad \text{for all $t_{i,m}^a\!\in\!\Gamma_{i,m}^S$},\label{Eq.15}
\end{align}
where $\Gamma_{i,m}^D(t_{i,m}^a)$ and $\Gamma_{i,m}^S$ are the discrete decision and state spaces computed by (\ref{Eq.13}) and Algorithm~\ref{Alg.1}, respectively.
\end{theorem}
\vspace{2pt}
\begin{proof} 
See the Appendix.
\end{proof}
\vspace{2pt}

The BOE in Theorem~\ref{Theorem1} can be solved by DP as in Algorithm~\ref{Alg.2}. The output of the algorithm is the optimal solution to the problem~(\ref{Eq.2}). We refer the readers to see, \emph{e.g.}, \cite{bellman1966dynamic,bertsekas2019reinforcement}, for detailed introductions on DP. 

In contrast to the conventional discretization methods using fixed time intervals to handle continuous state spaces, our method generates a discrete decision space $\Gamma_{i,m}^D(t_{i,m}^a)$ based on the discrete predicted departure times of other trucks. The constraint (\ref{Eq.3}) imposed by the delivery deadline and the conditions required to form platoons further contribute to the sparsity of the discretized decision space. As a result, the proposed approach is not subject to the curse of dimensionality associated with DP methods.  

\begin{remark}\label{Remark4}
Denote as $\tilde{n}\!=\!\max_{m\in\{1,\dots, N_i-1\}}|\Gamma_{i,m}^D|$ the maximum decision options of truck $i$ at each hub, where $\Gamma_{i,m}^D\!=\!\bigcup_{t_{i,m}^a\in\Gamma_{i,m}^S}\!\!\!\Gamma_{i,m}^D(t_{i,m}^a)$. The computational complexity of solving the problem (\ref{Eq.2}) by DP at the first hub of truck $i$ can be denoted as $O(\tilde{n}N_i)$, see, Example 1.3.1. in \cite{bertsekas2017dynamic}, where $\tilde{n}$ is no worse than $\max_{m\in\{1,\dots, N_i\}}\!\!\big|\Gamma_{i,m}^S\big|^2$.
\end{remark}

For a better understanding of the above results, we make use of Figure~\ref{Fig.3} to illustrate how one can generate the discrete decision space for truck $i$ that arrives at its $k$-th hub and obtain the DP graph associated with the problem (\ref{Eq.2}). By applying DP in such a DP graph, the optimal waiting time decisions of (\ref{Eq.2}) can be attained, as shown by the red arrows connecting the arrival and departure time nodes in orange in Figure~\ref{Fig.3}.

% Figure environment removed

The proposed platoon coordination method is applicable to large-scale transportation systems where hubs with cloud storage provide trucks with data storage and communication services while trucks conduct calculations independently for forming platoons. For any truck $i\!\in\!\mathcal{M}$, its workflow can be depicted in Figure~\ref{Fig.4} and described as follows:
\vspace{1pt}
\begin{itemize}
\item \textbf{Initialization:} Before starting the trip, truck $i$ uploads its route $e_{i,k}\!\in\!\mathcal{E}_i$, the fleet $i\!\in\!\mathcal{F}_s$ it belongs to, and its predicted departure times $t_{i,k}^d\!=\!t_{i,k}^a$ to the cloud storage for each hub $k$. At this stage, the predicted waiting time of truck $i$ at each hub is set to $0$. 
\vspace{2pt}

\item \textbf{Dynamic optimization:} When starting the trip, truck~$i$ is monitored at a regular time interval to check if it arrives at a hub $k\!\in\!\{1,\dots, N_i\!-\!1\}$. If arriving at a hub, truck $i$ downloads $\{j\!\in\!\mathcal{F}_t | \textbf{\textit{e}}_{i,m}\!\in\!\mathcal{E}_j, j\!\in\!\mathcal{M}\}$ from each of its hubs $m$ to compute its potential platoon partner-set $\mathcal{P}_{i,m}$ and downloads then the predicted departure times of other trucks $\{t_{i,m}^{d,j}|j\!\in\!\mathcal{P}_{i,m}\}$ with $m\!=\!k,\dots,N_i\!-\!1$. By applying Algorithms~\ref{Alg.1} and~\ref{Alg.2}, truck $i$ computes its optimal waiting times $\{t_{i,k}^{w*},\dots,t_{i,N_i-1}^{w*}\}$ and applies $t_{i,k}^{w*}$ at its hub $k$. Its predicted schedules at the following hubs are updated in accordance with $\{t_{i,k}^{w*},\dots,t_{i, N_i-1}^{w*}\}$.
\end{itemize}

\begin{table*}[t]
\caption{Fleet and Truck Assignment} % title name of the table
\vspace{-5pt}
\centering % centering table
\begin{tabular}{c|c|c|c|c|c|c|c}
  \hline\hline
  & & & & & &\\[-1.3ex]
  \raisebox{1.5ex}{\textbf{Fleet size}} & \raisebox{1.5ex}{\textbf{Nr. of trucks per fleet}} & \raisebox{1.5ex}{\textbf{Nr. of fleets}} &  \raisebox{1.5ex}{\textbf{Fleet index}} & \raisebox{1.5ex}{\textbf{Fleet percentage}} & \raisebox{1.5ex}{\textbf{Nr. of trucks}} & \raisebox{1.5ex}{\textbf{Truck index}} & \raisebox{1.5ex}{\textbf{Truck percentage}}\\
  [-1.1ex]
  \hline
  \multirow{3}*{Small fleet} & $1$ & $325$ & \multirow{3}*{$1$ -- $767$} & \multirow{3}*{$89.7\%$} & 325 &~ \multirow{3}*{$1-1971$} & \multirow{3}*{$39.4\%$}\\
   & 3 & 362 & ~ & ~ & 1086 & ~& ~\\
   & 7 & 80 & ~ & ~ & 560 &~ & ~\\
  \hline
  \multirow{3}*{Medium fleet} & 15 & 49 & \multirow{3}*{$768$ -- $851$}& \multirow{3}*{$9.8\%$} & 735 & \multirow{3}*{$1972$ -- $4216$}&\multirow{3}*{$44.9\%$}\\
  ~ & 34 & 27 & ~& ~ & 918 & ~&~\\
  ~ & 74 & 8 & ~& ~ & 592 & ~&~\\
  \hline
  \multirow{2}*{Large fleet} & 148 & 3 & \multirow{2}*{$852$ -- $855$} & \multirow{2}*{$0.5\%$} & 444 &\multirow{2}*{$4217$ -- $5000$} &\multirow{2}*{$15.7\%$}\\
  ~ & 340 & 1 & ~ & ~ & 340 &~ &~\\
  \hline
  \hline
\end{tabular}
\label{Table1}
\end{table*}

Note that our platoon coordination approach relies on two assumptions: (i) deterministic travel times in the dynamic model (\ref{Eq.1}), and (ii) free communication across different fleets to share routes and predicted schedules among trucks. As a result, the proposed method is limited to an ideal transportation network without travel time uncertainties and a reliable communication network without trust or privacy concerns between different fleets. Notwithstanding, travel time uncertainties can be incorporated with our method by allowing trucks to update their predicted schedules \textit{periodically} or in an event-triggered manner. In addition, the second assumption can be addressed by reaching data-sharing agreements among trucks or employing encrypted data transmission approaches. 

%==========Section V================
%------Simulation Study-------------
\section{Simulation Study}\label{Section V}
% Figure environment removed
In this section, we evaluate the improved platooning revenue from the developed multi-fleet platoon coordination approach in a large-scale transportation system. The simulation is conducted over the Swedish road network and based on the freight data in Sweden. In the following, the simulation setup and the simulation results will be introduced in detail.

%%%%%%%%%%%%%%%%%%%%%%%%%%%%%%%%%%%%%%%%%%
% Section V, Subsection A
\subsection{Simulation Setup}

% A-Sub-subsection 1)
\subsubsection{Road Network and Mission Generation} As shown in Figure~\ref{Fig.5}, we perform the simulation study in the Swedish road network with $105$ hubs, where each hub is a real road terminal in Sweden and represents the freight transport demands within one district. The coordinate of each hub is obtained from the SAMGODS model, which is the national model for freight transportation in Sweden and provides us with the truck flow between different districts based on the producer and consumer data. The mission of each truck is randomized such that its origin and destination belong to the set of hubs, and the truck flow from the SAMGODS model is used to get a realistic distribution of the transport missions. More specifically, the probability for two hubs $i$ and $j$ to be drawn as the origin and destination pair is computed by
\begin{align}
    P_{i,j}=\frac{F_{i,j}}{\sum_{ij}F_{i,j}},\nonumber
\end{align}
where $F_{i,j}$ is the truck flow from hub $i$ to hub $j$ in the SAMGODS model. The route between each pair of hubs is obtained from the open-source routing service \textit{OpenStreetMap}~\cite{OpenStreetMap}.

% A-Sub-subsection 2)
% Figure environment removed

\subsubsection{Fleet Distribution} We generate realistic fleet sizes from the data on the number of employees in transportation companies in Sweden \cite{TRAFA}. The fleet size distribution that we use in the simulation is given in Figure~\ref{Fig.6}, where the x-axis denotes the number of trucks in a fleet and the y-axis shows the fleet percentage. As is shown in the figure, $89.7\%$ of the fleets have no more than $10$ trucks in each fleet, and we refer to these fleets as small fleets. Fleets that include a number of trucks between $11$ and $100$, and over $100$ are referred to as medium and large fleets, respectively, which account for $9.8\%$ and $0.5\%$ in the total number of fleets.

According to the data reported in \cite{Lastbilstrafik}, approximately $5,000$ trucks start their transport trips each hour in Sweden. Following the fleet size distribution in Figure~\ref{Fig.6}, we can assign the $5,000$ trucks into $855$ fleets. More details about the assignment are provided in Table~\ref{Table1}.

% A-Sub-subsection 3)
\subsubsection{Parameter Settings}
We assume trucks start their trips at a random time between 08:00 to 09:00. The total travel time of a truck per day is less than $10$ hours, and the waiting budget (\emph{i.e.}, the maximum allowed waiting time at all hubs in the trip) of each truck is $10\%$ of its total travel time in the road network. In addition, we assume that trucks travel with a maximum and constant velocity of $80$ km per hour. The fuel consumption of each following truck in a platoon is assumed to be reduced by $10\%$, which leads to a monetary saving of $0.07$\texteuro ~per following truck per kilometer. Accordingly, the resulting platooning benefit $\xi$ is $5.6$\texteuro ~per following truck per hour. In line with the salary level of truck drivers in Sweden, the parameter $\epsilon_i$ representing the waiting loss of trucks is considered as $25$\texteuro ~per hour. 

%%%%%%%%%%%%%%%%%%%%%%%%%%%%%%%%%%%%%%
% Section V, Subsection B
\subsection{Solution Evaluation}
The simulation results and solution evaluation are introduced in this subsection. We conduct the simulation study over $5,000$ trucks belonging to $855$ different fleets and compare the platooning performance of three platoon coordination methods: the proposed predictive multi-fleet platooning method, the spontaneous multi-fleet platooning method, and the single-fleet platooning method. In spontaneous multi-fleet platooning, trucks communicate only with the trucks arriving at the same hub to optimize their schedules while having no access to the predicted schedules of other trucks at future hubs. The simulation results are evaluated from several perspectives, including the achieved reward, fuel savings, parameter sensitivity analysis, travel and waiting times, platooning rate and platoon formation rate, size of the formed platoons, and computation efficiency of the developed approach. For a better understanding of the dynamic optimization problem and the proposed solution scheme, the optimal scheduling solution of an illustrative example is first provided. 

\begin{table}[t]
\caption{The optimal schedule of one truck} % title name of the table
\vspace{-5pt}
\centering % centering table
\begin{tabular}{|c|c|c|c|c|} 
\hline
& & & &\\[-1.4ex]
  \raisebox{1.3ex}{\!\!\textbf{Hub}\!\!}& \raisebox{1.3ex}{\!\!\textbf{Arrival time}\!\!}&\raisebox{1.3ex}{\!\!\textbf{Departure time}\!\!} &\raisebox{1.3ex}{\!\!\textbf{Waiting time [s]}\!\!}&\raisebox{1.3ex}{\!\!\text{$|\mathcal{R}_{i,k}^{*}|$$\big/$$|\mathcal{P}_{i,k}|$\!\!}}
\\ [-0.5ex]
\hline % inserts single-line
& & & &\\[-1.1ex]
\raisebox{1.0ex}{\text{1}} & \raisebox{1.0ex}{\text{08:35:00}} & \raisebox{1.0ex}{\text{08:45:00}} & \raisebox{1.0ex}{\text{600}}& \raisebox{1.0ex}{\text{6}$\big/$\text{43}}\\[-0.4ex]
\hline 
& & & &\\[-1.1ex]
\raisebox{1.0ex}{\text{2}} & \raisebox{1.0ex}{\text{09:52:00}} & \raisebox{1.0ex}{\text{09:52:00}} & \raisebox{1.0ex}{\text{0}}& \raisebox{1.0ex}{\text{8}$\big/$\text{189}}
\\[-0.5ex]
\hline
& & & &\\[-1.1ex]
\raisebox{1.0ex}{\text{3}} & \raisebox{1.0ex}{\text{10:44:00}} & \raisebox{1.0ex}{\text{10:46:00}} & \raisebox{1.0ex}{\text{120}}& \raisebox{1.0ex}{\text{6}$\big/$\text{183}}
\\[-0.5ex]
\hline
& & & &\\[-1.1ex]
\raisebox{1.0ex}{\text{4}} & \raisebox{1.0ex}{\text{11:44:00}} & \raisebox{1.0ex}{\text{11:47:00}} & \raisebox{1.0ex}{\text{180}}& \raisebox{1.0ex}{\text{2}$\big/$\text{45}}
\\[-0.5ex]
\hline
& & & &\\[-1.1ex]
\raisebox{1.0ex}{\text{5}} & \raisebox{1.0ex}{\text{14:16:00}} & \raisebox{1.0ex}{\text{14:16:00}} & \raisebox{1.0ex}{\text{0}}& \raisebox{1.0ex}{\text{2}$\big/$\text{92}}
\\[-0.5ex]
\hline
& & & &\\[-0.9ex]
\raisebox{1.0ex}{\text{6}} & \raisebox{1.0ex}{\text{15:14:00}} & \raisebox{1.0ex}{/} & \raisebox{1.1ex}{/}& \raisebox{1.0ex}{/}\\[-0.1ex]
\hline
\end{tabular}
\label{Table2}
\end{table}


% Figure environment removed
% Figure environment removed

% B-Sub-subsection 1)
\subsubsection{Solution of One Truck} We provide in Table~\ref{Table2} the optimal schedule of one truck obtained by applying the predictive multi-fleet platoon coordination scheme. The transport route of the truck is given in Figure~\ref{Fig.5}. As shown in Table~\ref{Table2}, the truck starts its trip at $08\!:\!35\!:\!00$ and arrives at its destination hub at $15\!:\!14\!:\!00$. Its total travel time on roads is $384$ minutes, and the waiting budget at all hubs in its route is $38.4$ minutes. By the optimal schedule, $15$ minutes are spent at hubs for waiting and forming platoons. The last column of Table~\ref{Table2} gives the number of potential platoon partners of the truck, as well as that of its optimal platooning partners, at each hub.

% B-Sub-subsection 2)
\subsubsection{Reward}
Figure~\ref{Fig.7} shows the achieved reward (including the platooning reward and waiting loss) of each small, medium, and large fleet, compared among the three platooning schemes. In each sub-figure in Figure~\ref{Fig.7}, the fleet indices are sorted according to the fleet's reward achieved in the proposed predictive multi-fleet platooning. The simulation study shows that few small fleets form platoons and enjoy platooning benefits in single-fleet platooning, due to the small number of trucks included in each fleet. It also shows that the adoption of multi-fleet platoon coordination approaches leads to a substantial increase in fleet rewards. Additionally, it indicates that predictive multi-fleet platooning further improves the platooning profits based on the achievements of spontaneous multi-fleet platooning for each type of fleet.  
% Figure environment removed
% Figure environment removed
% Figure environment removed
% Figure environment removed

Figure~\ref{Fig.8} and Figure~\ref{Fig.9} show the total reward of trucks in small, medium, and large fleets, and the average reward per truck in each type of fleet, respectively. As we can see from Figure~\ref{Fig.8}, spontaneous multi-fleet platooning achieves higher monetary profits for small, medium, and large fleets compared to single-fleet platooning (approximately, $240$, $10$, and $1.7$ times higher, respectively). The predictive multi-fleet platooning further enhances these profits, with approximately $359$, $17$, and $3$ times higher profits for each fleet type, compared to single-fleet platooning. In total, the predictive multi-fleet platooning approach achieves around $15$ times higher monetary profit for all trucks in the system compared to single-fleet platooning, and results in about $0.5$ times higher monetary profit compared to spontaneous multi-fleet platooning. Although large fleets gain less increased reward compared to small fleets, they also have the incentive to cooperate with other fleets due to the substantial increase in their benefit. Furthermore, Figure~\ref{Fig.9} shows the average reward per truck in the three platoon coordination methods. The results indicate that the platooning profit gained by each truck in both the predictive and spontaneous multi-fleet platooning methods is independent of its fleet affiliation. 

% B-Sub-subsection 3)
\subsubsection{Fuel Saving}
% Figure environment removed
Figure~\ref{Fig.10} gives the total fuel savings of each type of fleet from platooning. The simulation results show that the proposed predictive multi-fleet platooning method achieves approximately $5.5\%$ fuel savings for small, medium, and large fleets, compared to $3.2\%$ in the spontaneous multi-fleet platooning scheme. In comparison with single-fleet platooning, where the fuel savings for small, medium, and large fleets are $0.03\%$, $0.38\%$, and $1.4\%$, respectively, the fuel economy is significantly improved by the predictive multi-fleet platooning. As a rough estimation, we utilize a linear model to convert fuel savings into reductions in CO$_2$ emissions, as adopted in~\cite{morrow2014assessment,xu2014energy}. Thereby, from a system perspective, the total fuel savings for all trucks achieved from predictive multi-fleet, spontaneous multi-fleet, and single-fleet platooning are $0.4\%$, $3.2\%$, and $5.5\%$, respectively. This corresponds to approximately $13$ times higher CO$_2$ emission reductions by employing the predictive multi-fleet platooning (\emph{i.e.}, $(5.5\!-\!0.4)/0.4\!=\!12.75$).

% B-Sub-subsection 4)
\subsubsection{Sensitivity Analysis}
Figure~\ref{Fig.11} shows how the fuel savings from truck platooning for the following trucks affect the trucks' total reward in the predictive multi-fleet platooning method. The y-axis shown on the right-hand side represents the platooning rate of the entire system, defined as the ratio of trucks' total travel time in platoons to that in the road network. The figure illustrates that as the percentage of fuel savings from platooning increases, the total reward tends to level off, indicating that the platooning rate of the system gradually approaches saturation status. Additionally, it shows that the total reward of the system drops sharply when the fuel-saving benefit of platooning becomes small. 

Figure~\ref{Fig.12} shows the total fuel savings achieved by all trucks in the platooning system applying the three platoon coordination methods. The results indicate that higher fuel savings from platooning lead to higher total fuel savings. Moreover, predictive multi-fleet platooning shows greater sensitivity to changes in platooning fuel savings compared to single-fleet and spontaneous multi-fleet platooning methods.

% B-Sub-subsection 5)
\subsubsection{Travel and Waiting Time}
The details of the travel and waiting times of individual trucks are provided in Figure~\ref{Fig.13}. Specifically, Figure~\ref{Fig.13}(a) gives the total travel time of each truck in the road network and in platoons from small, medium, and large fleets, compared between single-fleet and predictive multi-fleet platooning. In each sub-figure in Figure~\ref{Fig.13}(a), the x-axis represents trucks indices which are sorted according to trucks’ total travel times in the road network. By comparison, it can be seen that trucks' travel times in platoons increase significantly in the predictive multi-fleet platooning provided by our method for small, medium, and large fleets.  

Figure~\ref{Fig.13}(b) illustrates the waiting times of trucks at all hubs in their routes, where the waiting budget of each truck is assumed as $10\%$ of its total travel time in the road network. The truck index in the x-axis is consistent with the sorted truck indices in Figure~\ref{Fig.13}(a). As is shown, the average waiting time per truck in small, medium, and large fleets in the predictive multi-fleet platooning is $2.8$, $3.4$, and $3.7$ minutes, respectively. Compared to single-fleet platooning, there is an increase in the average waiting time for trucks in predictive multi-fleet platooning, while it leads to significant increases in trucks' travel times in platoons and the resulting platooning profits.

% B-Sub-subsection 6)
\subsubsection{Platooning Rate and Platoon Formation Rate} 
% Figure environment removed
We further evaluate trucks' platooning performance on roads and at hubs in the proposed predictive multi-fleet platoon coordination approach. As is shown in Figure~\ref{Fig.14}, the transport flow in the Swedish road network is given in sub-figure (a), which reflects the number of trucks traveling between different pairs of hubs per day in line with the SAMGODS data. Figure~\ref{Fig.14}(b) shows the platooning rate of all trucks in the system on each road segment in the road network, where the platooning rate on a road segment $\textbf{\textit{e}}\!\in\!\mathcal{E}$ is defined by
\begin{align}
    P_r(\textbf{\textit{e}})=\frac{\text{Nr. of following trucks on the road segment $\textbf{\textit{e}}$}}{\text{Nr. of trucks on the road segment $\textbf{\textit{e}}$}},\nonumber
\end{align}
where $P_r(\textbf{\textit{e}})$ takes values between $0$ and $1$. By comparing Figures~\ref{Fig.14}(a) and (b) one can see that trucks' platooning rate on a road segment is positively correlated with the transport flow on it. Moreover, Figure~\ref{Fig.14}(c) illustrates the platoon formation rate of each hub $k\!\in\!\mathcal{H}$ in the road network, which is defined as
\begin{align}
    P_{f}(k)\!=\!\frac{\text{Nr. of trucks finding new platoon partners at hub $k$}}{\text{Nr. of trucks in the road network}},\nonumber
\end{align}
where $P_f(k)$ measures each hub's contribution to the formation of platoons. For instance, a hub with a platoon formation rate of $0.04$ indicates that $4\%$ of trucks find new platoon partners at this hub. The higher the platoon formation rate of a hub, the more it contributes to forming platoons. We also show in Figure~\ref{Fig.14}(c) the trucks' average waiting time at each hub. The results show that hubs with a higher platoon formation rate have a relatively lower average waiting time. This is reasonable because hubs with high platoon formation rates generally also have a high throughput of trucks, as shown in Figure~\ref{Fig.14}(a), which leads to more platooning opportunities and a smaller time difference between truck arrivals. The evaluation results of hubs in Figure~\ref{Fig.14}(c) can be used to plan transport routes for trucks or to decide at which hubs to improve the capacity to facilitate more trucks in forming platoons. 
% Figure environment removed

% B-Sub-subsection 7)
\subsubsection{Platoon Size}
In our simulation study, $2453$ platoons are finally formed, and the size distribution of the formed platoons is shown in Figure~\ref{Fig.15}. As is shown, $55\%$ of the platoons consist of two trucks, $19.4\%$ and $11.2\%$ of the platoons are formed by $3$ and $4$ trucks, and around $97\%$ of the platoons have no more than $8$ trucks in the platoon. In practice, the platoon size might be limited due to safety concerns. Our method can be extended to handle platoon size constraints by limiting the size of the predicted platoon partner set or splitting the platoons exceeding the size limit.

% B-Sub-subsection 8)
\subsubsection{Computation Efficiency}
% Figure environment removed
\begin{table}[t]
\caption{Comparison of the computation efficiency} % title name of the table
\vspace{-5pt}
\centering % centering table
\begin{tabular}{|c|c|c|c|c|} 
\hline
& & & &\\[-1.4ex]
   \raisebox{1.3ex}{\!\!\textbf{Case}\!\!}&\raisebox{1.3ex}{\!\!\textbf{$\tilde{n}\!=\!\max_{m\in\{1,\dots,N_i\!-\!1\}}\!\!\big|\Gamma_{i,m}^D\big|$}\!\!}&\raisebox{1.3ex}{\textbf{$N_i$}} &\raisebox{1.3ex}{\!\textbf{Benchmark [s]}\!}&\raisebox{1.3ex}{\!\textbf{DP [s]}\!}
\\ [-0.5ex]
\hline % inserts single-line
& & & &\\[-1.0ex]
\raisebox{0.8ex}{\text{1}} & \raisebox{0.8ex}{\text{135}} & \raisebox{0.8ex}{\text{7}}& \raisebox{0.8ex} {\text{5.7}} &\raisebox{0.8ex}{\text{1.9}} \\[-0.5ex]
\hline 
& & & &\\[-1.0ex]
 \raisebox{0.8ex}{\text{2}} & \raisebox{0.8ex}{\text{283}} & \raisebox{0.8ex}{\text{6}}& \raisebox{0.8ex}{\text{45.9}} &\raisebox{0.8ex}{\text{4.8}}
\\[-0.5ex]
\hline
& & & &\\[-1.0ex]
 \raisebox{0.8ex}{\text{3}} & \raisebox{0.8ex}{\text{354}} & \raisebox{0.8ex}{\text{8}}& \raisebox{0.8ex}{\text{654.1}}&\raisebox{0.8ex}{\text{7.4}} 
\\[-0.5ex]
\hline
& & & &\\[-1.0ex]
\raisebox{0.8ex}{\text{4}} & \raisebox{0.8ex}{\text{387}} & \raisebox{0.8ex}{\text{7}}& \raisebox{0.8ex}{\text{1639.2}}&\raisebox{0.8ex}{\text{10.8}} 
\\[-0.5ex]
\hline
& & & &\\[-1.0ex]
 \raisebox{0.8ex}{\text{5}} & \raisebox{0.8ex}{\text{1319}} & \raisebox{0.8ex}{\text{7}}& \raisebox{0.8ex}{\text{7330.4}}&\raisebox{0.8ex}{\text{26.4}}
\\[-0.2ex]
\hline
\end{tabular}
\label{Table3}
\end{table}
Eventually, the computation efficiency of the proposed platoon coordination scheme is evaluated. We analyze the computation time that each truck takes to make its waiting time decision at every hub along its route and show the analysis results in Figure~\ref{Fig.16}. As we can see, over $98\%$ of the decision-making instances take less than $10$ seconds to compute the optimal waiting times, and more than $96\%$ of the decision-making instances take less than $5$ seconds. This demonstrates the high computation efficiency of the proposed method.

The computation efficiency of our solution scheme is further compared with a benchmark method, where all the possible combinations of the waiting time options at each hub are enumerated to obtain the optimal solution. We conduct simulation studies on $5$ trucks selected from the $5,000$ trucks, where each truck has a distinct route. For each truck (\textit{i.e.}, case), the computation time used to solve the predictive multi-fleet platoon coordination problem in DP and the benchmark solution method are given in Table~\ref{Table3}, where, as defined in Remark~\ref{Remark4}, $\tilde{n}$ denotes the maximum decision options of a truck at each hub. Based on our theoretical result, the optimal waiting time solution computed in the two methods is the same in each case, while the computation efficiency is significantly improved by DP. The above simulation studies demonstrate the high computation efficiency of the developed method, making our DP-based coordination approach suitable for real-time truck platooning in large transportation networks. 

%==========Section VI=================
\section{Conclusion}\label{Section VI}
This paper develops a DP-based platoon coordination method that schedules the waiting times of trucks in real-time and takes into account that trucks belong to different fleets interested in optimizing their own fleets' profits. A distributed dynamic optimization model is established to formulate the multi-fleet platoon coordination of trucks, where the reward function is modeled to capture the platooning reward and waiting loss of the fleet. Moreover, a DP-based optimal solution is presented for addressing the multi-fleet platoon coordination problem, where we show that the continuous decision space of the problem can be discretized without loss of optimality, which results in an efficient and real-time solution approach. Finally, the profit and efficiency of the platoon coordination method are demonstrated in a realistic simulation study performed over the Swedish road network, where we compare the case where trucks cooperate across fleets in forming platoons to the case where only single-fleet platoons are formed. 

The simulation study shows that multi-fleet platooning is essential to get significant economic and environmental benefits for the transportation system. Compared to single-fleet platooning, the developed multi-fleet platoon coordination approach achieves $359$, $17$, and $3$ times higher monetary profit for small, medium, and large fleets, respectively, and results in around $15$ times higher monetary profit for all trucks in the system. It also shows that compared to a system without any platooning, multi- and single-fleet platooning reduces the CO$_2$ emissions from $5,000$ trucks by $5.5\%$ and $0.4\%$, respectively. Moreover, the simulation study shows that over $98\%$ of the decision-making instances take a computation time of less than $10$ seconds, indicating that the proposed method is suitable for real-time platoon coordination in large transportation networks.

A limitation of the proposed method exists in the simplified truck dynamic model, where trucks' travel times are assumed to be deterministic, which in practice could be uncertain if considering traffic jams or various road conditions. In future work, we plan to extend the method in this paper to handle the travel time uncertainties by, for example, allowing trucks to update their predictions of arrival times at hubs. 

%==========Appendix=================
\appendix
\begin{prooflemma1}\label{proof_Lemma1}
Consider any waiting time $t_{i,m}^w\!\in\!{\Gamma_{i,m}(t_{i,m}^a)}\!\setminus\!{\Gamma_{i,m}^D(t_{i,m}^a)}$, meaning that $\Delta{f}\big(p_{i,m}^s,p_{i,m}^{-s}\big)\!=\!0$ and $t_{i,m}^w\!>\!0$. According to the stage reward function modeled in Eq.~(\ref{Eq.8}), we have that
\begin{align}
    \max\limits_{t_{i,m}^w\in{\Gamma_{i,m}(t_{i,m}^a)}\setminus{\Gamma_{i,m}^D(t_{i,m}^a)}}g_{i,m}(t_{i,m}^a,t_{i,m}^w)=0.\label{Eq.16}
\end{align}

Now consider another waiting time $t_{i,m}^{w'}\!\!=\!0\!\in\!\Gamma_{i,m}^D(t_{i,m}^a)$. By Eqs.~(\ref{Eq.7}) and (\ref{Eq.8}), the maximized stage reward function with respect to $t_{i,m}^a$ and $t_{i,m}^{w'}$ is denoted by
\begin{align}
    &\!\!\max\limits_{t_{i,m}^{w'}\in{\Gamma_{i,m}^D(t_{i,m}^a)}}g_{i,m}(t_{i,m}^a,t_{i,m}^{w'})\nonumber\\
    &\quad\quad\quad\quad\quad\quad~=\begin{cases}
        \!\Delta F_{i,m}^p(p_{i,m}^s,p_{i,m}^{-s})\!>\!0,& \text{if~$\exists\,t_{i,m}^{d,j}\!=\!t_{i,m}^a$}\\
        \!0, &\text{otherwise},
    \end{cases}\nonumber
\end{align}
where $t_{i,m}^{d,j}$ is the predicted departure time of other truck $j$ that belongs to the potential platoon partner set $\mathcal{P}_{i,m}$. Compared to the stage reward function in~(\ref{Eq.16}), it derives that
\begin{align}
    &\!\!\!\!\!\!\max\limits_{t_{i,m}^w\in{\Gamma_{i,m}(t_{i,m}^a)}\setminus{\Gamma_{i,m}^D(t_{i,m}^a)}}g_{i,m}(t_{i,m}^a,t_{i,m}^w)\nonumber\\
    &\quad \quad \quad \quad \quad \quad \quad \quad  \leq{\max\limits_{t_{i,m}^{w'}\in{\Gamma_{i,m}^D(t_{i,m}^a)}}g_{i,m}(t_{i,m}^a,t_{i,m}^{w'})}.\label{Eq.17}
\end{align}
Due to $t_{i,m}^{w'}\!<\!t_{i,m}^w$, and in line with the truck dynamics in (\ref{Eq.1}), the arrival time of truck $i$ at its next hub $(m\!+\!1)$ follows 
\begin{align}
    t_{i,m+1}^a>t_{i,m+1}^{a'},\label{Eq.18}
\end{align}
where $t_{i,m+1}^a\!=\!f_{i,m}(t_{i,m}^a,t_{i,m}^w)$ and $t_{i,m+1}^{a'}\!=\!f_{i,m}(t_{i,m}^a,t_{i,m}^{w'})$.
\vspace{-8pt}

Next, we will prove that $J_{i,m}^{*}\big(t_{i,m}^a\big)\!\leq\!{J_{i,m}^{*}\big(t_{i,m}^{a'}\big)}$ holds if there is $t_{i,m}^a\!>\!t_{i,m}^{a'}$, for $m\!=\!k,\dots,N_i$. We start with $m\!=\!N_i$.

\noindent (i) For $m\!=\!N_i$, by definition (\ref{Eq.10}), we have
\begin{align}
    J_{i,N_i}^{*}\big(t_{i,N_i}^a\big)\!=\!g_{i,N_i}(t_{i,N_i}^{a}),\quad \quad \text{for all}~t_{i,N_i}^a.\nonumber
\end{align}
If $t_{i,N_i}^a\!>\!t_{i,N_i}^{a'}$ holds, by Eq.~(\ref{Eq.9}), the following inequality holds
\begin{align}
    J_{i,N_i}^{*}\big(t_{i,N_i}^a\big)\!\leq\!J_{i,N_i}^{*}\big(t_{i,N_i}^{a'}\big).\label{Eq.19}
\end{align}

\noindent (ii) For $m\!=\!k,\dots,N_i\!-\!1$, the BOE in (\ref{Eq.11}) can be rewritten as
\begin{align}
    J_{i,m}^{*}\big(t_{i,m}^a\big)=\max\limits_{t_{i,m}^d\in\Gamma_{i,m}^d(t_{i,m}^a)}Q_{i,m}\big(t_{i,m}^a,(t_{i,m}^d\!\!-\!t_{i,m}^a)\big),\label{Eq.20}
\end{align}
where $t_{i,m}^d\!=t_{i,m}^a\!+t_{i,m}^w$ represents the departure time of truck $i$ at its $m$-th hub. The BOE in Eq.~(\ref{Eq.20}) shows that optimizing truck $i$'s waiting time $t_{i,m}^w$ at its $m$-th hub is equivalent to optimizing its departure time $t_{i,m}^d$ at the same hub. By (\ref{Eq.3}), $t_{i,m}^d$ is constrained by the set
\begin{align}
    \!\!\!\Gamma_{i,m}^d(t_{i,m}^a)\!=\!\Bigg\{t_{i,m}^d\,\bigg|\,t_{i,m}^a\!\leq\!{t_{i,m}^d}\!\leq\!{t_{i}^{dd}-\!\sum_{m=k}^{N_i-1}\tau(\textbf{\textit{e}}_{i,m})}\Bigg\}.\label{Eq.21}
\end{align}
So given two arrival times $t_{i,m}^a$ and $t_{i,m}^{a'}$, if $t_{i,m}^a\!>\!t_{i,m}^{a'}$ holds, by (\ref{Eq.21}), we have that
\begin{align}
    \Gamma_{i,m}^d(t_{i,m}^a)\!\subset\!{\Gamma_{i,m}^d(t_{i,m}^{a'})},\nonumber
\end{align}
which means that the later arrival time will cause a shrank decision space to the BOE in (\ref{Eq.20}), and it then will not result in a larger optimal reward function, \emph{i.e.}, 
\begin{align}
    J_{i,m}^{*}\big(t_{i,m}^a\big)\!\leq\!{J_{i,m}^{*}\big(t_{i,m}^{a'}\big)}.\label{Eq.22}
\end{align}

Given the above, if (\ref{Eq.18}) holds, there is
\begin{align}
    J_{i,m+1}^{*}\big(t_{i,m+1}^a\big)\!\leq\!{J_{i,m+1}^{*}\big(t_{i,m+1}^{a'}\big)}.\label{Eq.23}
\end{align}
Thus, by Eqs.~(\ref{Eq.11}), (\ref{Eq.12}), (\ref{Eq.17}) and (\ref{Eq.23}), it derives that 
\begin{align}
    &\!\max\limits_{t_{i,m}^w\in{\Gamma_{i,m}(t_{i,m}^a)}\setminus{\Gamma_{i,m}^D(t_{i,m}^a)}}\!\Big[g_{i,m}(t_{i,m}^a,t_{i,m}^w)\!+\!J_{i,m+1}^{*}\big(t_{i,m+1}^a\big)\Big]\nonumber\\
    &\quad \quad \quad~  \leq{\!\max\limits_{t_{i,m}^{w'}\in{\Gamma_{i,m}^D(t_{i,m}^a)}}\!\Big[g_{i,m}(t_{i,m}^a,t_{i,m}^{w'})\!+\!J_{i,m+1}^{*}\big(t_{i,m+1}^{a'}\big)\Big]}.\nonumber
\end{align}
Consequently, it proves that
\begin{align}
    J_{i,m}^{*}\big(t_{i,m}^a\big)&=\!\max\limits_{t_{i,m}^w\in{\Gamma_{i,m}(t_{i,m}^a)}}Q_{i,m}\big(t_{i,m}^a,t_{i,m}^w\big)\nonumber\\
    &={\!\max\limits_{t_{i,m}^w\in{\Gamma_{i,m}^D(t_{i,m}^a)}}Q_{i,m}\big(t_{i,m}^a,t_{i,m}^w\big)}.\nonumber
\end{align}
\end{prooflemma1}

\begin{proof}
The proof of Theorem~\ref{Theorem1} follows directly the proof of Lemma~\ref{Lemma1}, where we prove that the continuous decision space $\Gamma_{i,m}(t_{i,m}^a)$ of the BOE in (\ref{Eq.11}) can be discretized as $\Gamma_{i,m}^D(t_{i,m}^a)$, without loss of optimality. The discrete decision space $\Gamma_{i,m}^D(t_{i,m}^a)$ and the state transition function $f_{i,m}(t_{i,m}^a,t_{i,m}^w)$ lead to a discrete state space $\Gamma_{i,m}^S$. Thereby, it proves that the BOE in (\ref{Eq.11}) can be solved by DP using the discrete decision and state spaces generated by Lemma~\ref{Lemma1} and Algorithm~\ref{Alg.1}.
\end{proof}
\vspace{2pt}

\section*{Acknowledgment}
The authors would like to thank Albin Engholm for providing the simulation data from the SAMGODS model. The author Ting Bai would also like to thank the support of the Outstanding Ph.D. Graduate Development Scholarship from Shanghai Jiao Tong University, Shanghai, China.

\bibliographystyle{IEEEtran}
\bibliography{Ref}

\begin{IEEEbiography}[{% Figure removed}]{\textcolor{black}{Ting Bai}}\textcolor{black}{received the B.Sc. degree in Automation from Northwestern Polytechnical University, Xi'an, China, in 2013, and the Ph.D. degree in Electrical Engineering from Shanghai Jiao Tong University, Shanghai, China, in 2019. Since 2020, she has been with the Division of Decision and Control Systems, Department of Electrical Engineering and Computer Science, KTH Royal Institute of Technology, Stockholm, Sweden, where she is a postdoctoral researcher. Her research interests include distributed model predictive control, dynamic programming methods, and their applications to transportation systems.} 
\end{IEEEbiography}
\vspace{-120pt}

\begin{IEEEbiography}[{% Figure removed}]{Alexander Johansson} received a B.Sc. degree in vehicle engineering in 2015 and an M.Sc. degree in applied mathematics in 2017 from KTH Royal Institute of Technology, Stockholm, Sweden. In 2022, he received a Ph.D. from the Division and Control Systems, Department of Intelligent Systems, School of Electrical Engineering, KTH Royal Institute of Technology, Stockholm, Sweden. His research interests are optimization, game theory, and control for platoon coordination and swarm technology. 
\end{IEEEbiography}
\vspace{-120pt}

\begin{IEEEbiography}[{% Figure removed}]{Karl Henrik Johansson} is a Professor with the School of Electrical Engineering and Computer Science at KTH Royal Institute of Technology in Sweden and Director of Digital Futures. He received M.Sc. and Ph.D. degrees from Lund University. He has held visiting positions at UC Berkeley, Caltech, NTU, HKUST Institute of Advanced Studies, and NTNU. His research interests are in networked control systems and cyber-physical systems with applications in transportation, energy, and automation networks. He is a member of the Swedish Research Council's Scientific Council for Natural Sciences and Engineering Sciences. He has served on the IEEE Control Systems Society Board of Governors, the IFAC Executive Board, and is currently President of the European Control Association. He has received several best paper awards and other distinctions from IEEE, IFAC, and ACM. He has been awarded Distinguished Professor with the Swedish Research Council and Wallenberg Scholar with the Knut and Alice Wallenberg Foundation. He has received the Future Research Leader Award from the Swedish Foundation for Strategic Research and the triennial Young Author Prize from IFAC. He is Fellow of the IEEE and the Royal Swedish Academy of Engineering Sciences, and he is IEEE Control Systems Society Distinguished Lecturer.
\end{IEEEbiography}
\vspace{-120pt}

\begin{IEEEbiography}[{% Figure removed}]{Jonas M{\aa}rtensson} received the M.Sc. degree in vehicle engineering and the Ph.D. degree in automatic control from KTH Royal Institute of Technology, Stockholm, Sweden, in 2002 and 2007, respectively. In 2016, he was appointed as a Docent. He is currently a Professor with the Division of Decision and Control Systems, KTH Royal Institute of Technology. He is also engaged as the Director of the Integrated Transport Research Laboratory and the Thematic Leader for the area transport in the information age with the KTH Transport Platform. His research interests are cooperative and autonomous transport systems, in particular related to heavy-duty vehicle platooning. He is involved in several collaboration projects with Scania CV AB, S{\"o}dert{\"a}lje, Sweden, dealing with collaborative adaptive cruise control, look-ahead platooning, route optimization and coordination for platooning, path planning and predictive control of autonomous heavy vehicles, and related topics.
\end{IEEEbiography}
\end{document}


