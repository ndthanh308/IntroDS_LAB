\section{Borinot's hardware}
\label{sec:hardware}

\subsection{Design overview}

Borinot (bumblebee in Catalan, \figRef{fig:borinot}) is conceived as a compact and powerful thrust-controlled hexacopter to which different light and compliant torque-driven limbs \cite{grimminger_OpenTorqueControlledModular_2020} can be outfitted. 
The \gls{com} is located high to allow rapid and easy destabilization from the hovering stance. 
It has a powerful control unit constituted by an Intel NUC i7, able to perform high-demanding control tasks such as whole-body force-torque-based \gls{mpc} involving contacts. 


Borinot comprises two bodies (\figRef{fig:ro_platform_overview}): The upper body is a hexacopter flying platform. The lower body comprises one or more robotic limbs and the landing gear.
The following gives a detailed overview of both subsystems and how they are assembled.

\subsection{Upper body: hexacopter platform}

The Borinot base (\figref{fig:ro_platform_overview}, left and center) has been designed as a hexacopter.
This maximizes compactness, allowing its use in small-sized robotics laboratories.
The structural frame is composed of six light $\varnothing 8 \si{\milli\meter}\!\times\!2 \si{\milli\meter}$ carbon-fiber tubes forming two equilateral triangles that are disposed in opposition to form a  hexagram star. 
Motors are placed slightly outward of each vertex, resulting in a diagonal distance between propeller axes of $370 \si{\milli\meter}$.
This unusual hexagram design provides high rigidity and low weight (see \tabref{tab:ro_platform_weight_list}). 
The inner part of the star forms a generously sized hexagon, not spanned by the propellers and hence not interrupting the main airflow, where the main body with all other hardware resides.
3D-printed parts and regular M3 screws hold together the tubes and the rest of the components.
The control, power, and communication hardware is above the star plane, with the battery at the top. 
This provides two advantages for an agile motion: 
First, the \gls{com} is located high, allowing for rapid and easy destabilization from the hovering stance;
Second, the limb, which is attached below the base, can have the first joint's axis close to the body's propeller plane and \gls{com}, thereby improving the dynamic behavior of the assembly in the face of rapid tilting of the base.
Four small, mechanically fusible, and easily replaceable 3D-printed pieces are fixed to the carbon tubes to act as ports to fasten the lower body (\figRef{fig:ro_platform_overview}-right and section \ref{subsec:limb}).
The platform can be manufactured using only a 3D printer and common tools such as a soldering iron, metric Allen keys, and wrenches.

% Figure environment removed


\begin{table}[t]
  \caption{Weights and approximate prices of the platform's components.}
  \label{tab:ro_platform_weight_list}
  \begin{center}
    \begin{tabular}{@{}l c c@{}}
      \toprule
      & \textsc{Weight $[\si{\gram}]$} & \textsc{Cost (approx.) [\texteuro]} \\
      \midrule
      \textsc{Platform} \\
      \midrule
      \emph{Structural parts} \\
      3D printed parts & 544 & 20.00 \\ 
      Carbon fiber tubes & 82 & 41.00 \\
      \midrule
      \emph{Drive system} \\
      Motors & 348 & 165.00 \\
      ESCs   &   54 & 160.00 \\
      Propellers  &   36 & 23.00 \\
      \midrule
      \emph{Others} \\
      Battery 6S 3.0\,Ah 60C   &  498 & 83.00 \\ 
      Power module PM03      & 80 & 55.00 \\
      Pixhawk V5X           & 87 & 450.00 \\
      NUC7-i7-DNKE & 232 & 1050.00  \\
      Misc. &   151 & - \\
      \midrule
      \textbf{\textsc{Total platform}} &  2112 & 2047.00 \\     
      \bottomrule
    \end{tabular}
  \end{center}
\end{table}

\begin{table}[t]
  \caption{Weights and approximate prices of the 2 \gls{dof} limb's  components.}
  \label{tab:ro_arm_weight_list}
  \begin{center}
    \begin{tabular}{@{}l c c@{}}
      \toprule
      & \textsc{Weight $[\si{\gram}]$} & \textsc{Cost (approx.) [\texteuro]} \\     
      \midrule
      \textsc{2 \gls{dof} limb} \\
      \midrule
      \emph{Legs \& ODRI boards} \\
      3D printed parts & 130 & 3.0 \\ 
      Carbon fiber tubes & 37 & 14.00 \\
      ODRI micro driver & 30 & 50.00 \\
      ODRI master board & 18 & 40.00 \\
      \midrule
      \emph{Links} \\
      1st Link & 261 & 300.00 \\
      2nd Link & 139 & 300.00 \\
      End effector (tail) & 127 & 1.00 \\
      End effector (leg) & 25 & 1.00 \\
      \midrule
      \textbf{\textsc{Total limb (tail) }} & 742 & 709.00 \\     
      \bottomrule
    \end{tabular}
  \end{center}
\end{table}


The Borinot flying platform holds different subsystems:
\begin{enumerate*}
  \item the power and drive systems, comprised by the motors, the \glspl{esc} and the battery;
  \item the flight controller, which manages functionalities related to flying such as the state estimation, the communication with a radio controller, handling safety flight modes, and interfacing with the \glspl{esc}; and
  \item the onboard computer that is used to run the algorithms that operate the robot.
\end{enumerate*}

\subsubsection{Power \& Drive system}


% Figure environment removed

To achieve a high \gls{twr} we have to minimize weight (see \tabref{tab:ro_platform_weight_list}) and maximize thrust.
Regarding thrust, the range of available \gls{bldc} motors for multi-copters has increased significantly in recent years, leading to motors with special characteristics tailored to different applications (\gls{fpv}, aerial photography and cinema, surveillance, etc.). 
Our platform's dimensions limit the propellers' size to $7$ inches.
Thus, we target motors mainly used within the \gls{fpv} community, typically using small propellers.
The selected \textsc{TMotors F90-1300KV} \cite{tmotor_datasheet}, powered by a $6\text{S}$ Li-Po battery (nominal $22.2\si{\volt}$) and equipped with a 7-inch propeller with 4.2 inches pitch, can produce up to $16.1\si{\newton}$ of thrust. 
This results in a total thrust of $96.6\si{\newton}$ for the hexacopter.
More details on the performance of this motor-propeller set are provided in \secref{subsec:exp_motor_tests}. 

The motor selection highly restricts the choice of the \glspl{esc} and the battery.
The selected \gls{esc} is the \textsc{TMotor F35A 6s}, a drone racing component that allows operating the motor with a $6S$ battery and a fast throttle response.
As for the battery, we have selected a $6\text{S}$ unit with $3\si{\ampere\!\,\hour}$ and a discharge rate of $60\text{C}$.
This provides room to continuously operate the robot at $90\%$ of its throttle capacity, and having burst loads up to $100\%$. 
For reference, Borinot with a 2DoF limb has a \gls{twr} of 3,5: this means that its motors are at $100/\textrm{\gls{twr}}=28,6\%$ of throttle capacity during hovering.


\subsubsection{Flight controller}

The flight controller selection is usually tied to the type of firmware that we want to use.
We have opted for a unit compliant with the PX4 flight stack (see the justification of this choice in \secref{subsec:so_px4}).
We have selected the \textsc{Holybro Pixhawk 5X} because of its fast Ethernet communication with the main onboard computer.
The unit includes three IMUs, two barometers, and two GPS ports, has built-in communication with the radio controller and other inputs such as motion capture, and can drive up to 16 servos and \glspl{esc}.
%

\subsubsection{Onboard computer}
Borinot is equipped with an Intel NUC7-i7-DNKE, which mounts an Intel i7-8650U 4-cores CPU, a clock frequency of $1.9\si{\giga\hertz}$ and $32\si{\giga\byte}$ of DDR4 RAM.
This computer comfortably runs computation-intensive controllers, such as \gls{mpc}.
The CPU benchmarks 6300 at cpubenchmark\footnote{\url{https://www.cpubenchmark.net/cpu_lookup.php?cpu=Intel+Core+i7-8650U+\%40+1.90GHz\&id=3070}}, a website used in \cite{foehn_AgiliciousOpensourceOpenhardware_2022} to quantify the computing power of research multicopters. 
In this sense, Borinot compares favorably to Agilicious \cite{foehn_AgiliciousOpensourceOpenhardware_2022}, with a similar \gls{twr} but a CPU mark of 6300 vs. 1343 for Agilicious.


\subsection{Lower body: limbs and landing gear}
\label{subsec:limb}

The lower body (\figref{fig:ro_platform_overview}, right) comprises the landing gear and the limb or limbs. 
The assembly is attached to the carbon tubes of the upper body through 4 attachment ports so that different lower-body designs can be accommodated. 
The landing gear is a light structure made of carbon tubes whose mission is to protect the limbs upon landing. 
This section is, therefore, devoted to limb design.

To comply with the requirements in \secref{subsec:int_requirements} that affect the limb, we take advantage of the \gls{odri} ecosystem \cite{grimminger_OpenTorqueControlledModular_2020}.
This initiative proposes an open-source robotic actuator (electronic and mechanical parts) to build affordable agile legged robots such as quadrupeds or bipeds.
Adopting this technology allows us to work with different kinds of limbs attached to the same upper body.
A particular limb design can be selected according to the kind of task or research to conduct.
We judge this to be a better solution than having one universal  limb with many \glspl{dof}, which due to its weight, would undermine the final \gls{twr} and thereby the agility potential of Borinot.

The \gls{odri} actuator consists of a drone motor able to produce 0.3Nm of torque, a rotary encoder at the motor axle, and two pinion-belt 1:3 reduction stages. This results in a 1:9 gear ratio with an output torque of 2.7Nm. 

The electronics for the limb are composed of one \gls{odri} master board, which handles the communication with the onboard computer, connected to up to six uDrivers, which can drive two motors each. The uDrivers can perform torque control and variable-impedance control.


% Figure environment removed

In this work, we implement a planar $2$-\gls{dof} limb (see \figref{fig:ro_arm_design}), which allows us to investigate many of the main concepts of agility while keeping the weight low.
We have designed a new limb's base link that contains the same hardware parts as the \gls{odri} actuator at a smaller footprint.
We used a Solo12 leg link directly from \gls{odri} for the second link.
The third or terminal link can be selected depending on the application (\figref{fig:ro_arm_design}). 
When using the limb as a leg, we mount a regular Solo-12 terminal link with a rubber boot to maximize contact friction. 
When using it as a tail or arm, we mount a custom 3D-printed link on which a heavy object can be attached to increase the inertia of the end effector.

