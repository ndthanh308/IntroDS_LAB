\section{Introduction}
\label{sec:introduction}

The field of robotics is currently witnessing a notable surge in interest regarding the understanding and development of intricate and dynamic motion~\cite{Rubenson_SR2022, Ajanic_SR2020, foehn_AgiliciousOpensourceOpenhardware_2022}. 
This exploration is not only fascinating from a biomechanical perspective but also vital for effectively constructing and controlling inherently unstable robots, such as humanoids or aerial robots. 
This is particularly evident in the realm of  legged robotics (humanoids and quadrupeds), where all the multi-articulated body parts interact dynamically to generate overall motion, and consequently, locomotion and manipulation tasks are interdependent, necessitating the consideration of the concept of loco-manipulation and the existence of hybrid motion modes. 
In aerial robotics, the emergence of \glspl{uam} has increased the complexity of aerial robots with multiple articulations~\cite{Ollero_2022}. 
This presents new opportunities for hybrid motion, such as aerial loco-manipulation utilizing contacts, the action of tail inertia to modify flight dynamics (as observed in studies like \cite{Nabeshima-2019-arque-tail,schwaner-2021-tail-reorient,tang-2022-quadruped-tail}), or mixed locomotion modes like the jump-and-fly technique exhibited by certain animals such as chickens and locusts \cite{wei-19-fly-jump,badri-2022-birdbot,birn2014don,tobalske2007aerodynamics}. 
However, the topic of agile aerial loco-manipulation using contacts remains largely unexplored, and  research on \glspl{uam} usually concentrates only on the manipulation aspects, which are tackled at relatively low dynamics.

% Figure environment removed


In view of this, the main objective of this paper is to lay the groundwork for the study of whole-body agility at the convergence of aerial and legged robotics. 
We begin by carefully examining the essential characteristics that an agile aerial loco-manipulator robot should possess. 
To support our definition, we construct a practical prototype named Borinot (\figRef{fig:borinot}), which we make openly available through a comprehensive open-source initiative. 
Furthermore, we demonstrate the robot's capabilities by performing various typical movements that illustrate the use of its limb as a tail, as an arm, and as a leg, by means of a relatively straightforward \gls{mpc} controller. 
This investigation opens up a broad research area where we will subsequently share our progress in more advanced controllers, such as better MPC designs or reinforcement learning (RL).

This paper is organized as follows.
In the rest of this section, we explore the related work.
In  \secref{sec:conditions-agile-robots}, we elaborate on the conditions for building agile aerial loco-manipulators. 
In Secs. \ref{sec:hardware} and \ref{sec:software} we describe the  hardware and software of Borinot. 
In \secref{sec:control_architecture}, we present a preliminary design of a \gls{mpc} controller, which allows us to validate Borinot in terms of a capable agile platform.
In \secref{sec:experiments}, we present hardware tests relevant to an agile motion and real executions of different agile loco-manipulation tasks. 
Sec. \ref{sec:discussion} closes with a discussion. 



