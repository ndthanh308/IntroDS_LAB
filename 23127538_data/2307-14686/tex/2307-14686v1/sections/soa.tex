\subsection{Related work}
\label{sec:related-work}


Morphologically speaking, the kind of robot that most resembles our proposed robot is a \gls{uam}, a flying platform endowed with one or more robotic arms. 
So far, agility has not been seriously explored in the field of aerial manipulation and there is no \gls{uam} allowing for agile movements \cite{ruggiero_2018, Ollero_2022}.
Current \glspl{uam} equip arms with high gear ratios, hence not compliant, which are actuated using low-torque motors.
This prevents these robots to be controlled dynamically. 
Initially, \gls{uam}s were controlled in a decentralized manner, meaning that platform and arm controls are independent \cite{ruggiero_multilayer_2015,santamaria_nmpc_2017,rossi_trajgeneration_2017,lipiello_visualservoing_2016}.
This requires decoupled dynamics of platform and arm, resulting in quasi-static operations. 
Centralized whole-body control has been typically implemented using feedback linearization \cite{rajappa_tilthex_2015,tognon_control-aware_2018,kim_stabilizing_2018}, sometimes exploiting the differential flatness properties of the system  \cite{yuksel_differential_2016,tognon_dynamic_2017}. 
The drawback of feedback linearization is that it works at the kinematic level, and it is not obvious how to include the natural dynamics of multi-articulated bodies.

In many cases, \gls{uam}s have been built by adding arms to  existing commercial \glspl{uav}.
This helps solving a significant amount of practical problems related to flight, both in hardware and control, especially those regarding safety via recovery flight modes.
However, the capabilities of \gls{uav}s meant for commercial purposes are not always aligned with those needed for research and it is difficult to find manufacturers. 
For example, the dynamic control of such platforms demands direct control of individual rotor velocities, which is not always available.
In addition, in most cases these flying platforms have been withdrawn due to a reduced market (\eg~Asctec, and DracoR from Uvify).
This lack and/or inconvenience of platforms has led the robotics laboratories to create their own, although not many of them are open-source.
A recently published open-source \gls{uav} is the Agilicious drone~\cite{foehn_AgiliciousOpensourceOpenhardware_2022}, which is designed for agile motion but has no manipulation capabilities.
Non-commercial UAMs have been described in~\cite{Saeed_2018, Paul_2018, Perez-Jimenez-2020, Suarez_2020}, although they mount traditional arms with high gear ratios and still require quasi-static maneuvering.
Other existing works focus on presenting unconventional flying platform designs such as~\cite{Nguyen_2018, Park_2018} but they are not open-source.
%
In summary, new hardware proposals are required to grant \glspl{uam} with agile capabilities, for both platforms and arms, using also new control paradigms enabling the exploitation of the whole-body dynamics.

In many relevant aspects, it is more clarifying to regard Borinot not as an evolution of the \gls{uam}s, but rather of the legged robots, which are now allowed to fly.
Indeed, research on legged robots (quadrupeds, bipeds and humanoids) with strong dynamic capabilities is very mature, and already contemplates most of the requirements for agility \cite{hutter-16-anymal,feng-14-atlas,mastalli-20-crocoddyl,Carpentier_2018,grimminger_OpenTorqueControlledModular_2020} (see \secref{sec:conditions-agile-robots} for these requirements). 
In particular, and in contrast with what we said about \gls{uam}s, many legged robots are already conceived with compliant torque-controlled limbs, and their control schemes have been contemplating the whole-body dynamics already for a while.
It is therefore very reasonable, and this is precisely what we do, to draw from the state-of-the-art in contact locomotion research \cite{Carpentier_2018,mastalli-20-crocoddyl} and adapt its techniques to agile aerial loco-manipulation. 
Hardware-wise, the high dynamics nature of the actuators used in legged robots make them a natural choice for agile aerial maneuvering.
For example, Borinot takes advantage of the open dynamic robot initiative (ODRI\footnote{https://open-dynamic-robot-initiative.github.io/}) and its actuator design~\cite{grimminger_OpenTorqueControlledModular_2020} to conceive different kinds of powerful, light, compliant and torque-controlled limbs. 
%
As for the agile base platform, we have to design and build one from the ground up.

\subsection{Beyond the state-of-the-art}
Endowing legged robots with flying capabilities opens up a new area of research in the confluence of legged robotics and aerial manipulation. 
Unlike \glspl{uam}, such robots are now not meant exclusively for manipulation.
Interestingly, they can be used for research on agile aerial manipulation \textit{and} locomotion, and all the possible hybrid motion modes that may emerge from this new viewpoint.
Put otherwise, we do not concentrate on the semantics of the intended tasks (\eg~walk, jump, grasp, pick-and-place), but on the nature of the motion itself (the exploitation and control of the dynamic interactions between objects with mass).
One well-known robot intended for hybrid motion is Leonardo \cite{Kim_2021}, a bipedal hybrid robot that can walk and fly. 
The limitation of Leonardo is that it is built without agility or dynamism in mind, again resorting to limbs with high gear ratios, thereby suffering from the same limitations of the \glspl{uam} described above. 
With the introduction of Borinot we aim to overcome all these limitations.



