\section{Discussion}
\label{sec:discussion}

% New possibilities for legged robots: these experiments validate the hybridization between contact and flying locomotion. Opening new possibilities for legged robots, since a propeller-enabled legged robot could have the ability to jump higher, reorientate itself easier in the air, or fall from higher altitudes without breaking.

% \comment{What is missing to the current design, its drawbacks, and possible future work.}

% \comment{We have seen that the arm can help a lot in the general locomotion of the robot. Currently, the arm can only help in the X direction. Thus, we should think of other arm designs that allow the arm to also help in the other directions.}

In this paper, we have introduced Borinot, an open-source robot designed for research on agile aerial-contact loco-manipulation.
Throughout our study, we successfully validated the electromechanical capabilities of this robot by conducting a series of experiments to test the different operation modes.
Based on this work, we should direct our future efforts at two levels. 
The first level comprises engineering improvements at both hardware and software levels. 
The second level involves research in the field of motion control to fill the gaps that remain uncovered by this work.
Both are developed in the following paragraphs.

In terms of the design of the robot, we believe that the improvements should lead to a lighter and cheaper robot with fewer custom-made software repositories to maintain.
To achieve a lighter robot, we should target the platform's 3D-printed parts, which currently contribute to $25\%$ of the total weight.
In line with the weight reduction goal, we aim to evaluate alternative onboard computers that are lighter and more cost-effective compared to the current solution, which is half the cost of the total platform.
Furthermore, to eliminate the need for maintaining our custom version of \textsc{px4}, we intend to utilize the existing firmware version that already contains the capabilities we have developed in our version. 
This will streamline maintenance and align our system with established off-the-shelf solutions.
Still related to the flight controller, we believe exploring the use of more affordable hardware is worthwhile, as suggested in \cite{foehn_AgiliciousOpensourceOpenhardware_2022}.

Regarding motion control, we have presented a basic control architecture based on \gls{mpc}.
This allowed us to validate the robot, especially for agile flying locomotion, but left several topics unaddressed, which individually require intensive research.
First, the control architecture does not consider contacts.
The theoretical tools for including them in the \glspl{ocp} are known and used in legged locomotion. 
Fundamental difficulties arise because it is difficult or impossible to predict the time of contact. 
This can be a very acute problem under high dynamics, requiring accurate contact surface reconstruction and/or fast and reliable contact detection and fast control reaction.
Second, the control architecture lacks precision for manipulation tasks.
This should improve by endowing the controller loop with the notion of task, recovering at the same time the navigation phases.
This would allow the \gls{mpc} to recompute maneuvers to accommodate the tasks with precision.
Here, we face difficulties related to the discontinuities arising when the tasks first appear in the \gls{mpc} horizon.
In such cases, the new \gls{ocp} happens to be very different from the last, and the solver requires a sudden increase in the number of iterations, sometimes even failing to converge.
These issues can sometimes be tackled with heuristics or fine-tuning. 
Other times they require redesigning some components, a deeper theoretical insight, and/or adopting radically different techniques such as reinforcement learning in certain parts of the control scheme.

To conclude, our ultimate interest is to demonstrate hybrid motion modes that are particular to aerial-contact loco-manipulation.
With the help of the research platform Borinot, we believe we have established a consistent foundation to pursue this research.
By making it open source, we invite any interested actor to join the effort.


% Topics to discuss:

% 1. General concepts. Repeat some discourse bout Agility, open source, and research platform.

% 2. Platform improvements

% - Weight reduction

% - Simpler and cheaper controllers, with standard available code.

% 3. Roadmap for better control designs. (maybe this is not to be discussed here?)
  
%   - What are the gaps in this paper? (there're a ton)
%   - Why is it difficult? Contacts, precision, time-optimality. 

%   - What is different from purely legged robots?
%
%        Not much is different. Just the dynamical model:
%       Legged: non-actuated floating base
%       Chickens: partially actuated floating base.   
  
%   - What approaches are promising? Model-based (MPC), data-based (RL), hybrid (RL+MPC).

