\section{Experiments}
\label{sec:experiments}

There are different things that we aim at validating regarding Borinot's capabilities.
First, in \secref{subsec:exp_motor_tests} we want to gain insight into the performance of its hexacopter platform.
We are interested in quantitative details during a normal near-hovering operation and during a demanding scenario.
Quantities such as its thrust capacity, its power delivery, or its current consumption allow us to validate the component selection and to know where its limits are in terms of performance.

Then, we test the robot for the different motion modes involved in loco-manipulation, as explained in \secref{subsec:types_agile_locoman}, using the limb as a tail in \secref{subsec:exp_flying_locomotion}, as an arm in \secref{subsec:flying_manipulation}, and as a leg in \secref{subsec:exp_aerial_contact_locomotion}.
For the first two experiments, we use the control architecture presented in \secref{sec:control_architecture} running onboard the NUC i7 computer.
We use $35$ nodes and a node separation of $20\si{\milli\second}$, giving an \gls{mpc} horizon of $0.7$ seconds.
The last experiment involves contacts and jumps, and for the sake of simplicity, we resort to open-loop commands and the assistance of a guide rail.



\subsection{Motor-propeller tests} 
\label{subsec:exp_motor_tests}

% Figure environment removed

To get an estimate of the \gls{twr} and its implications in terms of energy consumption and autonomy, we have tested a single motor in a test bench for motor-propeller systems \cite{thrust_stand}.

In \figref{fig:exp_motors}-left, we show the thrust-speed curve of the platform motor with the $7 \times 4.2$ propeller.
The motor can deliver $16.1\si{\newton}$ of thrust, which translates into $96.6\si{\newton}$ of theoretical thrust for the platform.
This means a \gls{twr} of $4.7$ when using only the platform and $3.5$ with the platform and the limb.

In terms of energy consumption, \figref{fig:exp_motors}-left also shows the current that the motor draws to achieve different thrusts.
The hovering current consumption is at $30\si{\ampere}$ for the case with the limb and $20\si{\ampere}$ for the platform alone ($5\si{\ampere}$ and $3\si{\ampere}$ per motor to achieve a total thrust of $28\si{\newton}$ and $20\si{\newton}$, respectively).
Considering the $3\si{\ampere\hour}$ battery, this consumption leads to a theoretical hovering autonomy of $4\si{\minute}$ and $8\si{\minute}$, respectively.
This autonomy is enough for doing several attempts of an agile trajectory, which normally take less than a minute each.

The mapping from the desired thrust to motor command depending on battery voltage is shown in \figref{fig:exp_motors}-center.
Its accuracy and dynamic response are shown in \figref{fig:exp_motors}-right, where we observe a tight match between desired and achieved thrusts, with a sharp response to transients.


\subsection{Flying locomotion: limb as a tail}
\label{subsec:exp_flying_locomotion}

% Figure environment removed

The first set of experiments illustrates the use of the limb as a tail to participate in flight dynamics.
The setup is as follows.
We define the body axes of Borinot as X-forward, Y-left, and Z-up, and identify the XZ and YZ planes respectively as the sagittal and coronal planes.
We mount a 2\gls{dof} planar tail so that its motion occurs only in the sagittal plane. 
We thus expect it to contribute differently to trajectories evolving in the sagittal and coronal planes.
A weight of 100g is attached to the endpoint of the tail to increase its inertia and therefore its impact on the overall dynamics (see \figref{fig:ro_arm_design}).

We specify two trajectories, namely \emph{sagittal displacement} and \emph{coronal displacement}, which are rapid movements along the X and Y directions, respectively (see \figref{fig:sequence-displacements} for the initial acceleration phase of these trajectories).
Both share the same fundamental task of moving from an initial hovering position to a waypoint placed at $5\si{\meter}$ distance using a time interval of 2\,s, remaining there for $600\si{\milli\second}$, then returning to the starting point also in 2\,s, and finally hover.
The tail configuration at all waypoints (initial, intermediate, and final) is set to be fully stretched down.

The tail's influence on the platform dynamics can be of different nature.
On the one hand, the tail can be moved to adjust the overall \gls{cog} and reduce the moment of inertia, enabling the platform to tilt more easily. 
We refer to this as \emph{static tail assistance}, since this adjustment is usually done at low velocities.
On the other hand, a high torque can be applied at the tail's joints, producing a reaction torque at the platform and resulting in a rapid change in inclination.
We refer to this as \emph{dynamic tail assistance}, as the torque considerably alters the angular momentum.
In the present setup, dynamic assistance is not practicable in the coronal movements since the tail can only apply torques in the sagittal plane.

% Figure environment removed

We show in \figref{fig:exp_flying_displacement} a comparison of the sagittal and coronal trajectories with the same time interval of $2\si{\second}$.
In the sagittal case, the platform can tilt much quicker (top plot) while applying a smaller torque on the platform due to the propellers (center plot). 
We attribute this difference to the dynamic tail assistance, as seen in the bottom plot where the tail contributes over $2\si{\newton\meter}$ of torque.
It might be worth remarking that this assistance maneuver has been discovered and computed by the \gls{ocp}; in other words, it has not been enforced or suggested by any other means than optimality.
The dynamic and static assistance can also be appreciated in \figref{fig:sequence-displacements}. 
We observe rapid and ample tail movements in the sagittal case, typical of dynamic assistance.
We also observe static tail assistance in the coronal case, again discovered by the \gls{ocp}, where the tail configuration changes slowly to achieve a  better centered and slightly higher  \gls{com}, thereby easing the task of tilting the robot.
This tilting needs to be performed exclusively via the propellers' thrust differential.


% Figure environment removed

We conducted a second set of experiments to investigate the extent to which the limb's contribution can enhance the platform's performance during increasingly aggressive trajectories in the sagittal plane.
We now ask Borinot to go to and come back from the waypoint at increasing speeds, that is, with decreasing time intervals of $2.0\,\si{\second}$, $1.8\,\si{\second}$, and $1.6\,\si{\second}$.
We report the results in \figref{fig:exp_flying_x_displacement}.

We find that for the $2.0\,\si{\second}$ trajectory, the limb's action is already saturated at $2.5\,\mathrm{Nm}$ for the initial 5\% of the time.
During this initial kick, and since the tail torque is already saturated, the additional angular speed of the platform observed for the $1.8\si{\second}$ and $1.6\si{\second}$ trajectories is  due to the differential action of the propellers (second plot).
After this and up to 30\% of the time, we observe an increasing contribution of the tail as the time of the experiment shortens.
This tail assistance controls the tilting angle, which reaches values close to 60$^\circ$ for the 1.6\,s trajectory.
We also observe that the limb's saturation persists for extended periods during these quicker trajectories.
These results suggest that the trajectory of $2.0\si{\second}$ marks the threshold between a graceful motion (see \secref{sec:degrees-agility}), where controls remain smooth, and an aggressive motion, where the controls become saturated for extended intervals, and the transitions are sharper.
This increasing aggressiveness can be better appreciated in the accompanying video.

A final commentary concerns power and aggressiveness. 
More aggressive motions require more power, as observed in \figref{fig:exp_flying_x_displacement}-bottom. 
Interestingly, such an increase in power is well compensated by a shorter execution time, leading to very similar energy consumption for completing the three tasks.


\subsection{Flying manipulation: limb as an arm}
\label{subsec:flying_manipulation}

% Figure environment removed

\glsreset{ee}
In this experiment, we demonstrate an agile approach for a manipulation task.
The task is to maintain the \gls{ee} at a fixed position for a brief but non-trivial period  ($500\,\si{\milli\second}$) while the platform is prevented from hovering, which we achieve by demanding a pitch angle of 45\,$^\circ$ during the task.
This way, we emulate the conditions for a manipulation in which the robot base cannot stop and hover, such as when having to pick an object from a wall with an arm that is shorter than the propellers.
This forces a dynamic maneuver.

The results are shown in \figref{fig:exp_flying_agile_ee}. 
We observe that it is possible to accomplish this manipulation task by executing an agile maneuver where the tilted platform follows a parabolic trajectory governed by gravity. 
The manipulation occurs around the instant when the platform reaches the apex of this trajectory. 
During this short time, the arm continuously adjusts to ensure that the \gls{ee} remains at the desired position.

\figref{fig:exp_flying_agile_ee} indicates that the maneuver dynamics are properly captured. 
That is, the platform does a parabolic trajectory while the \gls{ee}'s position remains within a short range from a fixed point, and its velocity remains very low.
However, to maintain the \gls{ee} in a truly fixed position, its velocity should be zero.
This is not the case, and we attribute it to the lack of closed-loop control at the task level
(see \figref{fig:control_architecture}), which prevents the positioning task from reaching the desired accuracy.
Nonetheless, the results indicate that the approach is feasible and effective for an agile manipulation task. 
We discuss this control issue further in the conclusions.


\subsection{Hybrid aerial-contact locomotion: limb as a leg}
\label{subsec:exp_aerial_contact_locomotion}

% Figure environment removed

% Figure environment removed

The hybrid aerial-contact locomotion experiment consists of Borinot jumping using its limb as a leg while having its propellers holding a fraction of its weight.
We have attached the robot to a carriage that can move through a vertical rail (\figref{fig:exp_jump_trajectory}).
This allows us to eliminate the necessity of robot control and concentrate on showing the electro-mechanical capacity of Borinot for the jump-and-fly locomotion mode.

The jump-and-fly sequence of moves can be observed in \figref{fig:exp_jump_trajectory}.
To perform the jump, we set the platform's motors at a constant thrust below the total weight (Borinot + carriage).
Then, starting with the folded leg, we apply full torque to its joints ($2.7\si{\newton\meter}$) until the leg is fully stretched and the robot goes airborne.
During the flight, we bring the leg to a semi-folded configuration with medium impedance to allow it to touch the ground in a compliant soft landing.

We show in  \figref{fig:exp_jump_results} the results for three different jumps with the thrust set respectively at $50\%$, $70\%$, and $90\%$ of the total moving weight.
As expected, the robot is able to jump higher for higher thrust values and for a longer time (top plot).
During the airborne phase, the three trajectories should have a perfect parabolic shape, with a linear decay of velocity corresponding respectively to 50\%, 30\%, and 10\% of gravity (center plot).
This behavior can be distorted by the apparition of the ground effect\footnote{\url{https://en.wikipedia.org/wiki/Ground_effect_(aerodynamics)\#Rotorcraft}}, which is especially evident in the case of the $90\%$ thrust jump.

The bottom plot of \figref{fig:exp_jump_results} shows the total power required for each jump, together with a reference of the power required for hovering, which is 927W if we account for the carrier's weight (around 700\,g).
We see that the locomotion power required increases as we increase the contribution of the flight with respect to the jump. 
In other words, jump-and-fly locomotion is much more efficient than pure flight, and the more we can rely on the legs, the better. 
This reveals that exploring such hybrid locomotion modes is beneficial, for flying is an expensive endeavor for creatures with poor aerodynamics, but walking, running, or jumping alone might be insufficient to achieve certain tasks.
The fact that some animals perform this kind of hybrid locomotion suggests that for specific body architectures and in certain situations (\eg~escaping from predators or climbing steep slopes), this is the preferred mode of locomotion \cite{wei-19-fly-jump,Vidyasagar_2015,birn2014don,tobalske2007aerodynamics,badri-2022-birdbot}.
This should also apply to robots.



