

\section{The agile aerial loco-manipulator}
\label{sec:conditions-agile-robots}


Borinot is designed with agility in mind. 
For this reason, we believe it is necessary first to introduce our notion of agility, providing a detailed definition from which we can work on.
To do so, we start with the dictionary definition and refine it so that we can engineer it. 
We then discuss how such a concept of agility can be implemented in robots, both in terms of electro-mechanics and control, to finally conclude with a relation of the characteristics that we believe are required for conceiving agile aerial loco-manipulators.


\subsection{The concept of agility}

The concept of agility is defined in the dictionary\footnote{Oxford learner’s dictionaries: agility, accessed on April 3,
2023 at \url{https://www.oxfordlearnersdictionaries.com/definition/english/agility}} as  \textit{``the ability to move quickly and easily''}.
This definition is considerably vague as it overlooks some crucial factors that are key for robotics. 
The existence of a complex multi-articulated body with torso, head, and limbs is taken for granted in natural language and biomechanical studies when referring to agility, and so it is implicit in the definition, but requires careful attention when engineering machines. 
Two characteristics of such bodies that must be considered are under-actuation and redundancy. 
Under-actuation means the system cannot be commanded to follow arbitrary trajectories in configuration space, requiring the generation of non-trivial maneuvers~\cite{underactuated}. 
Redundancy means there are multiple ways to satisfy a task, and thus, certain \glspl{dof} remain available for secondary tasks.

A third characteristic of these bodies that demands consideration is that they often rely on contacts with the environment to generate all or a significant part of the forces that will produce the motion. 


With all these considerations made explicit, we shall enrich the definition of agility as \textit{``the ability of a complex body to combine quick maneuvers easily, using contacts and other forces against the environment''}. 
%
This explicit definition outlines the major necessary conditions for designing and controlling an agile robot, with each keyword representing a specific characteristic. 
For instance, the term \textit{`maneuver'} demands strong prediction capabilities, while \textit{`quick'} emphasizes dynamism. 
\textit{`Easy'} necessitates the minimization of effort, hence the importance of optimality when controlling the robot. 
Executing quick maneuvers with minimal effort requires the utilization of the \textit{`complex body'}'s natural dynamics: we must carefully predict those forces that will modify the robot's whole-body dynamics as little as necessary while accomplishing the desired tasks. 
These forces have to be produced by the robot, hence the need for force-torque actuation and control.
Whole-body control is also demanded by the keyword \textit{`combine'}, in the sense that the robot can do several things at once, such as flipping over while looking at a specific target. 
\textit{`Contacts}`, and especially unexpected or poorly predicted contacts in the context of high dynamics, demand light and compliant limbs to minimize the undesired effects of energetic impacts, but powerful enough to be used for locomotion.
Finally, to ensure easy dynamics, robots with high \gls{com} and small moment of inertia are preferred, with actuators placed where their effect is maximized.

\subsection{Degrees of agile motion}
\label{sec:degrees-agility}

To provide effective design guidelines, we must also consider different degrees of agility. 
In this regard, we can subdivide agile motion into categories and examine how they impact the requirements of the desired robot.
First, `soft' or `gentle' motion is characterized by small actuation effort and results in the execution of tasks with low dynamic content. 
It is the consequence of minimizing energy.
Second, `graceful' motion would employ a significant part of the control range but avoids frequent saturation and harsh control changes. 
It is a trade-off between energy and time.
Finally, `acrobatic' or even `aggressive' motion would be where more powerful actuators are often saturated, allowing sharp transitions between saturation extremes. 
It comes from the minimization of time. 
The upper limit for aggressiveness results in switched, permanently saturated actuation, which can be dealt with time-optimal control strategies, yielding actuation of the type bang-bang \cite{romero-2021-time-optimal-aerial}.
Clearly, a more powerful robot can be driven more aggressively, and therefore the overall ratio between motor power and mass (or inertia) will determine the final character of the motion a robot will be able to achieve.

\subsection{Types of agile aerial-contact loco-manipulation}
\label{subsec:types_agile_locoman}

We now discuss the concepts of locomotion and manipulation and their impact on the design and control of aerial robots. 

When there are no contacts involved, \textit{locomotion} refers to plain flight, with the limb functioning as a tail, while \textit{manipulation} involves modifying non-dynamic characteristics such as aesthetics or information (for tasks like spraying, filming, or spying), using the limb as an arm.

In the presence of contacts, various situations arise depending on the ratio of inertias between the robot and the object being contacted:

    - Locomotion occurs when the contact is utilized to alter the robot's motion, such as walking or jumping. 
    It is assumed that the inertia of the contacted object is infinitely larger than that of the robot. 
    In this case, the limb functions as a leg.

    - A first manipulation class involves using contact to influence the object's motion. 
    This corresponds to traditional object manipulation tasks like grasping, pick-and-place, pushing, or pulling. 
    Ideally, it is assumed that the object's inertia is negligibly small compared to the robot's. 
    Here, the limb acts as an arm.

    - A second manipulation class employs contact to modify non-dynamic characteristics, such as drawing or measuring. 
    In this scenario, the object's inertia is considered infinitely larger than the robot's, resulting in the interaction forces affecting the robot's motion.

We observe that it is the ratio of inertia between the robot and the object that determines whether the contact will affect the motion of the robot, the object, or both --- this stems directly from the principle of conservation of momentum. 
When the inertias are comparable, any interaction between the robot and the object will influence both dynamics, resulting in a hybrid motion combining locomotion and manipulation elements. 
This hybridization is particularly interesting in highly dynamic scenarios. 
We can refer to this as hybrid loco-manipulation, such as when a robot bounces off or intercepts an airborne object.
Similarly, we can consider hybrid motion modes such as jump-and-flight locomotion, which, as we show at the end of this paper, achieves much higher efficiency than plain flight.

Conceiving a robot for such situations and controlling it requires a unified approach encompassing loco-manipulation and considering dynamism as its core aspect. 
This approach emphasizes the nature of the motion and interactions themselves, focusing on how to apply forces to achieve the desired dynamic effects on the robot \textit{and/or} the object, rather than the specific categorization (locomotion versus manipulation) of the intended action.






\subsection{Required and desired characteristics}
\label{subsec:int_requirements}

Given all the ideas presented so far, we summarize a set of required and desired characteristics for conceiving agile aerial loco-manipulator robots, particularly when they are devoted to research.



\subsubsection{\bf{\Gls{mc} platform}}
\begin{enumerate}[(a)]
  \item \emph{Small sized, compact}: Provides smaller inertia. Allows operation in smaller indoor research arenas.
  \item \emph{High \gls{com}}: allows easier tilting for more agile maneuvers.
  \item \emph{High \gls{twr}}: %Allows higher levels of agility.
  The \gls{twr} is used in \cite{foehn_AgiliciousOpensourceOpenhardware_2022} to quantify the agility of a platform. 
  \item \emph{Direct motor control}: Control has to act directly on forces, and so we need to command propeller thrust. A direct command over motor velocity or similar is also practicable as long as a good mapping onto thrust is available.
\end{enumerate}


\subsubsection{\bf{Robotic limbs}}
%
\begin{enumerate}[a)]
  \item \emph{Lightweight}: To undermine as less as possible the overall \gls{twr} of the assembled loco-manipulator.
  \item \emph{Compliant, low friction, low gear ratios}: To avoid the introduction of non-conservative forces, such as friction. Also, to minimize the disrupting effects of impacts.
  \item \emph{High-torque motors}: To allow aggressive maneuvers. To support (a significant amount of) the robot's weight for locomotion. Important since we have low gear ratios.
  \item \emph{Torque-controllable}: 
  Torque can be either measured with a torque sensor at the joint, which is expensive in terms of weight and price, or estimated from the motor current. 
  The last option is only possible with the low friction provided by low gear ratios.
\end{enumerate}

\subsubsection{\bf{Control}} 
\begin{enumerate}[a)]
    \item \emph{Predictive}: To be able to generate maneuvers. 
    \item \emph{Optimal}: To be able to move easily. Optimality can be model-based (such as \gls{mpc}) or data-based (evolutionary algorithms, learning).
    \item \emph{Whole-body}: To account for all dynamic interactions between the parts.
    \item \emph{Torque-based}: To act on the dynamics of the robot.
    \item \emph{Phase switching}: To account for the appearance and disappearance of contacts.
\end{enumerate}

\subsubsection{\bf{Other desired characteristics}}

\begin{enumerate}[a)]
  \item \emph{Open-source}: Researchers should have access to the hardware and firmware designs. 
  Open-source platforms may have large communities of developers who can provide support and contribute to the platform's growth.
  \item \emph{Easy to build, cost-effective, flexible}: The  Hardware should be 3D-printable and require common tools for assembly. 
  This reduces costs related to production, maintenance and repairs, and allows easy customization.
  \item \emph{Conceptually simple}: Avoid design complications such as non-planar \glspl{mc} or coaxial propellers. They move the focus of research away from agility.
\end{enumerate}



