\section{Background.}
\label{sec:back}
Given a set of points and a pairwise distance metric, partitioned local depth (PaLD) algorithms determine the pairwise cohesion between all pairs of points in a dataset \cite{pald_pnas22}.
Assuming that the dataset comprises sufficiently separated subsets, cohesion values are invariant to contraction and dilation of distances within subset distances.
Community structure revealed by cohesion values capture the concept of near neighbors based on relative positioning, adapting to varying density.
This approach is more flexible than standard cluster labeling or nearest neighbor approaches.
Density-based approaches (e.g. DBSCAN) \cite{cm+13-pakdd,cm+15-tkdd,ek+96-kdd} that attempt to combine points into high- and low-density groups based on pairwise distances include thresholding (tuning) parameters to reflect locality and cluster size.
Likewise, $k$-nearest neighbor (KNN) approaches \cite{gh04-neurips} attempt to group points via comparisons against their $k$ nearest neighbors (using absolute distances).
The tuning parameter, $k$, controls the neighborhood size for a given point and is often fixed for all points.
Cohesion values depend on triplet distance comparisons (as opposed to absolute distances) which require only measures of relative similarity and can be more reliable than exact numerical distances for analyzing high dimensional, non-Euclidean data.
PaLD requires $O(n^3)$ operations to compute cohesion values without assumptions on underlying probability distribution or tuning parameters.

Given a set of points $\SSS$, the \emph{local focus} of a pair of points $x,y\in \SSS$ is the set of all points within distance $d_{xy}$ of either $x$ or $y$, where $d_{xy}$ is the distance between $x$ and $y$: $ {\cal U}_{xy} = \{z \in \SSS \; | \; d_{xz} \leq d_{xy} \; \Or \;  d_{yz} \leq d_{xy}\}.$
We let $u_{xy} = |{\cal U}_{xy}|$ denote the size of the local focus.

The \emph{local depth} of a point $x \in \SSS$ is the probability that, given a uniformly chosen random second point $Y\in \cal S$ and a random third point $Z$ chosen uniformly from the local focus ${\cal U}_{xY}$, $Z$ is closer to $x$ than $Y$:
\begin{equation}
\label{eq:LD}
    \ell_x = \Pr\left[d_{Zx} < d_{ZY} \; | \; Y \sim \unif(\SSS {\setminus} \{x\}), Z \sim \unif({\cal U}_{xY})\right].
\end{equation}

The cohesion of a point $z$ to another point $x$ is a part of the local depth $\ell_x$ and is defined as
\begin{equation}
\label{eq:cohesion}
    c_{xz} = \Pr\left[Z = z \; \tAnd \; d_{Zx} < d_{ZY} \right].
\end{equation}
The random variables $Y$ and $Z$ in \cref{eq:cohesion} are chosen from the same distributions as in \cref{eq:LD}; we drop the notation here and later.
This implies that $\ell_x = \sum_{z\in \SSS} c_{xz}$, or that cohesion is partitioned local depth.
The cohesion matrix, $C$, can be used to analyze community structure.
For example, two points have particularly strong cohesion if the impact of one of the points to the other is more than that expected from a random focus point of another random point.