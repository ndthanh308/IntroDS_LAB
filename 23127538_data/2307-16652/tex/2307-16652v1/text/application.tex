\section{Text Analysis Application}
\label{sec:application}

% \paragraph{Application.}
% Figure environment removed

We demonstrate the utility of PaLD on larger datasets than previously considered \cite{pald_pnas22} for semantic analysis of words extracted from Shakespeare sonnets \cite{matlab-text-toolbox}.
Words are converted to vectors using the pre-trained fastText word embedding \cite{BGJM16-arxiv,JGBM16-arxiv}, yielding a dataset of 2712 words.
We compute Euclidean distance between embedding vectors and generate the cohesion matrix $C$ using the OpenMP pairwise algorithm.
\Cref{fig:wordcloud} shows words associated with \emph{guilt} and \emph{halt} obtained from PaLD and from analyzing only the distance matrix $D$.
PaLD is parameter-free, with strong ties determined by a universal threshold (see \cite{pald_pnas22}), whereas analysis using $D$ requires a user-tuned distance or neighbor-count cutoff.
Note the differing sizes of strong-tie neighborhoods between the two words.
PaLD finds $20$ words with strong ties to \emph{guilt} and 5 words for \emph{halt}.
The $20$ closest words to \emph{guilt} based on distance correspond to a cutoff of $2.26$.
We observe significant overlap between the two sets, though PaLD reports stronger ties to  \emph{expiate} and \emph{conscience}.
PaLD finds $5$ words with strong ties to \emph{halt}.
To illustrate the pitfalls of tuning an absolute distance threshold, we apply the distance cutoff $2.26$ for \emph{halt}, which yields $23$ words including several unrelated ones (e.g. \emph{just} and \emph{say}).
This suggests that absolute distance thresholds are not robust to varying density and distance scales within word neighborhoods.
A distance cutoff of $2.14$ is required for \emph{halt} to match results obtained from PaLD.
Applying the cutoff to \emph{guilt} identifies only $8$ related words, missing several words like \emph{expiate}.
We attain a speedup of $16.7\times$ using the NUMA optimized OpenMP pairwise algorithm at $p = 32$ and an overall run time of $0.178$ seconds.
