%%%%%%%%%%%%%%%%%%%%%%%%%%  ltexpprt_twocolumn.tex  %%%%%%%%%%%%%%%%%%%%%%%%%%%%%%%%
%
% This is ltexpprt_twocolumn.tex, an example file for use with the SIAM LaTeX2E
% Preprint Series macros. It is designed to provide two-column output.
% Please take the time to read the following comments, as they document
% how to use these macros. This file can be composed and printed out for
% use as sample output.

% Any comments or questions regarding these macros should be directed to:
%
%                 Donna Witzleben
%                 SIAM
%                 3600 University City Science Center
%                 Philadelphia, PA 19104-2688
%                 USA
%                 Telephone: (215) 382-9800
%                 Fax: (215) 386-7999
%                 e-mail: witzleben@siam.org


% This file is to be used as an example for style only. It should not be read
% for content.

%%%%%%%%%%%%%%% PLEASE NOTE THE FOLLOWING STYLE RESTRICTIONS %%%%%%%%%%%%%%%

%%  1. There are no new tags.  Existing LaTeX tags have been formatted to match
%%     the Preprint series style.
%%
%%  2. Do not change the margins or page size!  Do not change from the default
%%     text font!
%%
%%  3. You must use \cite in the text to mark your reference citations and
%%     \bibitem in the listing of references at the end of your chapter. See
%%     the examples in the following file. If you are using BibTeX, please
%%     supply the bst file with the manuscript file.
%%
%%  4. This macro is set up for two levels of headings (\section and
%%     \subsection). The macro will automatically number the headings for you.
%%
%%  5. No running heads are to be used for this volume.
%%
%%  6. Theorems, Lemmas, Definitions, Equations, etc. are to be double numbered,
%%     indicating the section and the occurrence of that element
%%     within that section. (For example, the first theorem in the second
%%     section would be numbered 2.1. The macro will
%%     automatically do the numbering for you.
%%
%%  7. Figures and Tables must be single-numbered.
%%     Use existing LaTeX tags for these elements.
%%     Numbering will be done automatically.
%%
%%  8. Page numbering is no longer included in this macro.
%%     Pagination will be set by the program committee.
%%
%%
%%%%%%%%%%%%%%%%%%%%%%%%%%%%%%%%%%%%%%%%%%%%%%%%%%%%%%%%%%%%%%%%%%%%%%%%%%%%%%%



\documentclass[twoside,leqno,twocolumn]{article}

% Comment out the line below if using A4 paper size
\usepackage[letterpaper]{geometry}
\usepackage{ltexpprt}
\usepackage{algorithm}
\usepackage[noend]{algpseudocode}
\algnewcommand\tTo{\textbf{to}~}
\algnewcommand\tAnd{\textbf{and}~}
\algnewcommand\Or{\textbf{or}~}
\usepackage{amsmath}
\usepackage{standalone}
\usepackage{amssymb}
\usepackage{hyperref}
\usepackage{xcolor,graphicx}
\usepackage{caption}
\usepackage{subcaption}
\usepackage[inline]{enumitem}
\usepackage{wrapfig}
\usepackage[capitalise]{cleveref}
\usepackage{bm}
\usepackage{soul}
\usepackage{multirow}
\usepackage{tikz}
\usetikzlibrary{shapes}
\usetikzlibrary{matrix,positioning,shapes.geometric}

\Crefname{@theorem}{Theorem}{Theorems}
\newcommand{\todo}[1]{\textcolor{red}{TODO:~#1}}
\newcommand{\todoAD}[1]{\textcolor{blue}{TODO:AD: ~#1}}
\newcommand{\SSS}{{\cal S}}
\newcommand{\unif}{\mathbb{U}}
\newcommand{\indic}[1]{\mathbb{I}_{#1}}
\newcommand{\mindxz}{d_{xz} \leq \{d_{xy},d_{yz}\}}
\newtheorem{definition}{Definition}

\usepackage{listings}
\lstset{
    language=c,
    basicstyle=\footnotesize\ttfamily,
    breaklines=true,
    keywordstyle=\color{magenta},
    commentstyle=\itshape\color{green!75!black},
    escapechar=!,
    tabsize=1
    % numbers=left,
    % numberstyle=\tiny\color{gray},
}
\setlength{\belowcaptionskip}{-10pt}
\graphicspath{{./fig/}}
\begin{document}
%
\newcommand\relatedversion{}
% \renewcommand\relatedversion{\thanks{The full version of the paper can be accessed at \protect\url{https://arxiv.org/abs/1902.09310}}} % Replace URL with link to full paper or comment out this line


%\setcounter{chapter}{2} % If you are doing your chapter as chapter one,
%\setcounter{section}{3} % comment these two lines out.

\title{\Large Sequential and Shared-Memory Parallel Algorithms for Partitioned Local Depths}
\author{Aditya Devarakonda\footnote{Department of Computer Science, Wake Forest University.}
% \thanks{Department of Computer Science, Wake Forest University}
\and Grey Ballard\footnotemark[1]
% \thanks{Society for Industrial and Applied Mathematics.}
}

\date{}

\maketitle

% Copyright Statement
% When submitting your final paper to a SIAM proceedings, it is requested that you include
% the appropriate copyright in the footer of the paper.  The copyright added should be
% consistent with the copyright selected on the copyright form submitted with the paper.
% Please note that "20XX" should be changed to the year of the meeting.

% Default Copyright Statement
% \fancyfoot[R]{\scriptsize{Copyright \textcopyright\ 2023 by SIAM\\
% Unauthorized reproduction of this article is prohibited}}

% Depending on which copyright you agree to when you sign the copyright form, the copyright
% can be changed to one of the following after commenting out the default copyright statement
% above.

%\fancyfoot[R]{\scriptsize{Copyright \textcopyright\ 20XX\\
%Copyright for this paper is retained by authors}}

%\fancyfoot[R]{\scriptsize{Copyright \textcopyright\ 20XX\\
%Copyright retained by principal author's organization}}

%\pagenumbering{arabic}
%\setcounter{page}{1}%Leave this line commented out.
% \todo{10pgs excluding refs.}
% \todoAD{I think we decided to add Yixin as a co-author. Need to confirm...}


\begin{abstract} \small\baselineskip=9pt
In this work, we design, analyze, and optimize sequential and shared-memory parallel algorithms for partitioned local depths (PaLD).
Given a set of data points and pairwise distances, PaLD is a method for identifying strength of pairwise relationships based on relative distances, enabling the identification of strong ties within dense and sparse communities even if their sizes and within-community absolute distances vary greatly.
We design two algorithmic variants that perform community structure analysis through triplet comparisons of pairwise distances.
We present theoretical analyses of computation and communication costs and prove that the sequential algorithms are communication optimal, up to constant factors.
We introduce performance optimization strategies that yield sequential speedups of up to $29\times$ over a baseline sequential implementation and parallel speedups of up to $19.4\times$ over optimized sequential implementations using up to $32$ threads on an Intel multicore CPU.
\end{abstract}

% Figure environment removed

\section{Introduction}
Automatic 3D reconstruction of clothed humans using image inputs has gained increasing significance due to its potential applications in a wide array of AR/VR scenarios. High-fidelity reconstructions typically depend on sophisticated capture systems, which are developed with dense camera arrays~\cite{collet2015high,joo2015panoptic,joo2018total}, programmable light-stages~\cite{Vlasic2009, guo2019relightables}, and depth sensors~\cite{newcombe2011kinectfusion,DoubleFusion,BodyFusion,dou2016fusion4d,newcombe2015dynamicfusion}. However, stringent capture environments equipped with complex hardware pose significant challenges for consumer-level applications.


In this context, considerable research effort has been dedicated to developing methods that allow for more flexible capture configurations, such as utilizing a few RGB inputs. Among these works, learning implicit functions \cite{iccv2020PIFu, saito2020pifuhd, hong2021stereopifu} has proven effective in achieving highly detailed reconstructions by integrating the advancements of deep neural networks. These methods employ large multi-layer perceptrons (MLPs) to predict the occupancy probability or truncated signed distance function (TSDF) value of every queried 3D point based on its associated local feature, which is extracted from images. They can recover a continuous surface at arbitrary resolutions without topology restrictions.


However, in typical MLP-based implicit networks, the occupancy or TSDF value at each location is solved independently with planar image features, rendering them less capable of addressing challenging cases such as occlusions. Consequently, these methods suffer from generalization and robustness issues, particularly when tackling strong occlusions caused by large motion or multiple interacting humans. 
Some follow-up studies  \cite{zheng2021deepmulticap,zheng2021pamir,huang2020arch} utilize an extra geometric model, SMPL~\cite{Loper2015}, to improve robustness by introducing strong shape priors. 
Their success typically relies on the assumption of geometrical similarity \cite{huang2020arch} between the shape prior and target reconstruction, making them intractable for handling complex cases with loose clothes and sensitive to errors in SMPL model fitting.



%\ping{this paragraph sounds like `TSDF is better than MLP/SMPL, and we use TSDF to solve the problem'. But in Sec 3, we are telling a different story, saying `MLP needs a 3D convolutional encoder'. We need to make these two sections consistent.}\sicong{I think in this paragraph we claim that the TSDF}


%We opt for Trucated Signed Distance Funtion (TSDF) volumetric representations as they are naturally suitable for convolution operations, which have shown remarkable performance for learning hierarchical features on 2D visual perception tasks \cite{SunXLW19}. 
%Meanwhile, TSDF also describes the gradual geometry change around shape surface, which is not reflected by occupancy volume. 

We instead revisit the 3D volumetric representation and resort to 3D convolutional neural networks (CNNs) for feature learning, due to their impressive performance in feature learning and the ability to incorporate spatial context. However, volumetric methods and 3D convolution involve discretization, which might raise concerns regarding whether a discretized volume can preserve subtle geometric details as continuous representations learned in implicit functions. We investigate the relationship between volume resolution and quantization error on synthetic data by converting target mesh objects to TSDF volumes, as shown in Figure~\ref{fig:quantization_error}. We observe that the quantization errors are significantly reduced by increasing volume resolution and become nearly negligible when reaching a relatively high resolution (e.g., 512 or higher). In other words, achieving fine-detailed reconstruction is not supposed to be restricted by the use of volume representations as long as a proper volume resolution is utilized. Therefore, we present a method with high-resolution feature volumes, e.g., 256 and 512, while traditional volumetric methods \cite{varol18_bodynet,gilbert2018volumetric} are often limited to much lower resolutions, such as 32 or 128.



On the other hand, an increase in volume resolution may lead to a cubic growth of memory overhead \cite{8100085}. Reducing memory costs while guaranteeing the granularity of volumetric representations is necessary for pursuing high-quality reconstruction. Thus, we adopt a coarse-to-fine approach and cull away irrelevant voxels to build a sparse high-resolution feature volume. At the coarse level, the network computes an initial TSDF by applying a U-Net with sparse 3D CNN \cite{3DSemanticSegmentationWithSubmanifoldSparseConvNet} on the sparse feature volume, which is carved by a visual hull. Through our experiments, it turns out that more than 95\% of the volume grids are discarded by the visual hull culling, making the sparse 3D CNN efficient. At the fine level, the network focuses on a narrow band near the zero-level set of the initial TSDF and discretizes the narrow band with smaller voxels. By employing this narrow-band culling, we further shrink the sampling space, resulting in a relatively small range of grid numbers (usually 300K--500K in our experiments) even with a high volume resolution of 512. The remaining voxels in the narrow band are associated with features that fuse high-frequency information from the computed normal maps upon the low-frequency shape from the coarse level to compute the TSDF at high resolution. The final mesh is then extracted from the TSDF using the Marching-Cube algorithm ~\cite{Lorensen87marchingcubes}.
% Different from the u-net sturcture to preserve global topology context, we then apply a shallow 3dcnn to compute the final TSDF $D_{final}$ which contain more local geometry detail.




% \ping{this paragraph can be expanded. It is an important contribution and often ignored by other works. stress on the novel idea of regressing blending weights instead of colors}

In addition to geometry, high-quality mesh texture is also a crucial factor contributing to visual appearance. Directly computing a color field in 3D space, as in \cite{iccv2020PIFu}, struggles to capture high-frequency texture details, while the neural radiance field (NeRF) \cite{yu2020pixelnerf} or the DoubleField~\cite{shao2022doublefield} require expensive per-instance optimization and are often unstable for sparse input images. In contrast, we adopt an image-based rendering approach to compute a texture atlas map, which is efficient and widely supported in existing computer graphics tools. 
Specifically, we compute a blending weight at each 3D point on the mesh surface to determine its color as a weighted average of the colors at its image projections. The blending weights can be computed at a relatively coarse resolution, e.g., 512 volume resolution in our case, and leave texture details to the high-resolution images, such as 1K or 2K. Unlike previous methods that generate blurry texturing results under sparse input, our method generalizes well on both synthetic and real data with just a few input views. 
Figure~\ref{fig:teaser} shows two examples reconstructed by our method. Despite the challenging garment, pose, and occlusion, our method recovers faithful shape, normal, and texture on the right.

%with a wide variety of poses and clothing styles, and it is also adaptive to handle input image with arbitrary resolutions.
%\sicong{For this concern we claim that when the resolution of dicretized volume meets certain threshold (which is 256 in our experiment), the quantization error can be neglected.} 



In summary, the main contributions of this paper are as follows:
\begin{itemize}
\vspace{-0.1in}
  \item 
  We revisit the 3D volumetric representation and demonstrate that it can support clothed human reconstruction with equal or even better performance compared to implicit representation. 
  \item 
  We develop a memory and computation-efficient method for high-resolution volumetric reconstruction using sophisticated sparse 3D CNN, coarse-to-fine estimation, and voxel culling by visual hull and narrow bands. 
  \item 
  We introduce a novel method to compute a texture atlas map, which captures rich appearance details from high-resolution input images.
  \item 
  We achieve impressive results on standard benchmark datasets Twindom and MultiHuman, significantly reducing the point-2-surface (P2S) precision to approximately 0.2cm from just six input views, with more than $50\%$ error reduction compared to the state-of-the-art methods, including DoubleField~\cite{shao2022doublefield} and PIFuHD~\cite{saito2020pifuhd}.
\end{itemize}
\vspacebeforesection
\section{Background}
\label{sec:background}

In this section, we provide the necessary background information to ensure a comprehensive understanding of the attack described in this paper. We start with a description of the Distributed Hash Table (DHT) used by IPFS, followed by its content resolution mechanisms. We also detail techniques for network size estimation, necessary for our attack detection and mitigation mechanisms.

\vspacebeforesection
\subsection{IPFS DHT}
\label{sec:kad_dht}

We review the features of the Kademlia DHT~\cite{maymounkov2002kademlia} and its \texttt{libp2p} implementation~\cite{libp2p_github} that are the most relevant to our attack.
To participate in the DHT, each peer generates a public/private key pair and derives an identity $\peerid \in \{0,1\}^{256}$ as the hash of its public key.
Ideally, each peer generates a random key pair and, therefore, peer IDs are distributed uniformly and independently over the space $\{0,1\}^{256}$.
While honest nodes follow this rule, malicious nodes may generate and choose from an arbitrary number of key pairs.
Each peer maintains a routing table consisting of $m=256$ buckets.
The $i$-th bucket contains the addresses of up to $k=20$ peers whose peer IDs share a common prefix of exactly $i$ bits with the peer's own peer ID. 

%
A new participant node joins the IPFS network by contacting one of the hardcoded bootstrap nodes. This bootstrap node provides the new node with some initial peers allowing it to join the DHT. The new node uses this information to perform a walk through the DHT towards its own peer ID.
The walk allows to: \textit{(i)}~make sure that there is no other node in the network with the same ID; \textit{(ii)}~discover new peers and fill the newcomer's DHT routing table. At the same time, the newcomer establishes \bitswap~\cite{de2021accelerating} connections to a subset of encountered peers (usually around 300 of them). The core role of the \bitswap protocol is to enable bilateral content transfer and to play the role of a cache for recently-accessed content.

The main DHT operation $\Call{GetClosestPeers}{\key}$ returns the $k=20$ closest peers to $\key$. 
%
In Kademlia, the distance between two keys $x$ and $y$ in the key space is given by $x \oplus y \in \{0,...,2^{256}-1\}$, where $\oplus$ denotes the bitwise XOR operation on the keys; the resulting binary string is interpreted as an integer.
%
When a client wants to find the peers with IDs closest to $\key$, it sends a request to the $\alpha=3$ peers in its routing table whose peer IDs are closest to $\key$. Each of these peers returns the $k$ closest peers to $\key$ in its own routing table and the addresses of these peers. 
%
The client again sends a request to the $\alpha$ peers closest to $\key$, among peers in its routing table and those whose addresses it just received. This process repeats until the client does not find any more peers closer to $\key$.
Due to network churn and imperfect routing tables, we observed in our experiments that successive calls to $\Call{GetClosestPeers}{\key}$ do not always return the same set of $k=20$ peers (we provide more details in \Cref{sec:evaluation}, \Cref{fig:20closest}). This is an important limitation affecting our attack.

\vspacebeforesection
\subsection{Content Resolution in IPFS}
\label{sec:ipfs}

IPFS is a content-centric network.
It allows its participant to request files without specifying their location. 
%
Content is indexed by content IDs $\cid \in \{0,1\}^{256}$ that are derived from a hash of that content.
Both peer IDs and CIDs are used as keys in the DHT.
Each node can play the role of a \provider, \downloader, or \resolver. 
The process of content advertisement and resolution is illustrated in \Cref{fig:add_get_provider}.

%
When a \provider wishes to publish content with a given $\cid$ on IPFS, it creates a \emph{provider record} that contains $cid$ and the \provider's address.
During a $\Call{Provide}{\cid}$ operation, the \provider first uses $\Call{GetClosestPeers}{\cid}$ to locate the $k=20$ peers with their peer IDs closest to $\cid$, 
%
and then sends them a $\mathsf{PutProvider}$ message including the provider record (\Cref{fig:add_get_provider}(a)).
We call the peers that hold provider records for $\cid$ the \emph{resolvers} for $\cid$.

Each CID can have several \providers. In fact, by default, each IPFS client becomes a provider for each piece of content it downloads for a fixed amount of time (12h, 24h, or 48h depending on the client version or custom configuration). As a result, the system provides an auto-scaling feature with supply automatically rising with demand.

%
When a \downloader wishes to fetch a piece of content, it first sends a request to all its \bitswap peers. If none of them has the content, the \downloader uses the DHT-based resolution system. We stress that the \bitswap protocol plays the supporting role of a cache in the dissemination of popular files. However, the mechanism does not provide reliable content resolution, in particular for new or less popular content. %

When \bitswap unstructured search fails, the \downloader resolves $\cid$ using $\Call{FindProviders}{\cid}$. This operation uses a DHT walk identical to that of $\Call{GetClosestPeers}{\cid}$ to find $k$ \resolvers but also queries encountered nodes for a provider record for $\cid$ (\Cref{fig:add_get_provider}(b)). The process terminates when either 20 \providers have been found, or all \resolvers have been asked. Querying all encountered nodes (\ie, not only the designated \resolvers) is useful because some of the encountered nodes may have a provider record in their cache.
%

Upon receiving a provider record, the client connects to the address specified in the provider record to retrieve the actual content (\Cref{fig:add_get_provider}(c)).
Provider records are not authenticated, and therefore malicious \providers may respond with incorrect provider records (or may not respond at all). However, the integrity of the content is preserved because the hash of the retrieved content can be verified against its $\cid$.
%


%

\input{img/add_get_provider.tex}

\vspacebeforesection
\subsection{Network Size Estimator}
\label{sec:netsize}

The number of nodes in a decentralized system is generally unknown due to the avoidance of centralized membership management.
This number is nonetheless useful for optimizations, deciding on individual node configurations, or security mechanisms.
Various methods were proposed for the decentralized estimation of unstructured and structured networks~\cite{eli-sohl-dht-size-estimation,kostoulas2005decentralized, manku2003symphony}.
We use in this work a mechanism developed initially by Protocol Labs as part of a mechanism for decreasing the latency of publishing content in IPFS~\cite{network-size-estimation-notion,network-size-estimation-github-pr}.

%
%
%
%
%
%
%
%
%
%

Each node in the DHT refreshes its routing table periodically (every $10$ minutes in \texttt{libp2p}). 
For this, the node samples $m$ random keys (one for each bucket of its routing table)
%
and queries the DHT to obtain the $k=20$ closest peer IDs to each key.
Using these, the node then computes the average distance between each one of these keys $\key_j$ for $j=1,\dots,m$ and their $i$-th closest peer ID for $i=1,...,k$ (with $m=256$ and $k=20$).
\begin{equation}
    \label{equ:avg-dist}
    \overline{D}_i = \frac{1}{m} \sum_{j=1}^m \operatorname{dist}(\key_j, \peerid_{j}^{(i)})
\end{equation}
where $\peerid_{j}^{(i)}$ is the $i$-th closest peer ID to $\key_j$.
With $N$ peers in the DHT and peer IDs uniformly distributed in the hash space, the expected distance between a $\key$ and its $i$-th closest peer ID is $\frac{2^{256}i}{N+1}$. The node then runs a least square regression to compute the value of $N$ for which the expected distances best fit the empirical average distances, \ie,
\begin{equation}
    \label{equ:netsize-least-squares}
    \hat{N} = \arg\min_{N} \sum_{i=1}^k \left(\overline{D}_i - \frac{2^{256}i}{N+1}\right)^2.
\end{equation}
The resulting estimate $\hat{N}$ can be computed in closed form.
%

When a node starts running, it must perform DHT queries for a few random keys to initialize its network size estimate. 
Since a larger number of queries will result in higher accuracy, making more queries than what is needed to initialize one's routing table is recommended.
Thereafter, keeping the estimate up-to-date does not require any excess DHT queries beyond what is already used for refreshing the routing table as this is done frequently (every 10 minutes).

While the network size estimate has a stochastic variance resulting from the probability distribution of the honest peer IDs, it is hard for an attacker to bias the estimate significantly. Since the estimator uses the density of peer IDs around keys chosen uniformly at random, the adversary would require numerous Sybil nodes (on the order of the whole network size) to significantly affect the peer ID density around those keys.

\section{Algorithms}


\begin{algorithm}[H]
\caption{Federated Learning}
\label{alg:FedAvg}

% \DontPrintSemicolon
% \SetCommentSty{mycommfont}
% \SetAlgoLined
% \SetKwInOut{Req}{Require}
% \SetKwFunction{FClient}{ClientUpdate}
% \SetKwProg{Fn}{}{:}{}

\begin{algorithmic}[1]
\REQUIRE number of clients ($N$), sampling rate ($C\in(0,1]$), number of communication rounds ($T$), local dataset of client $k$ ($D_k$), number of local epoch ($E$), number of local batch size ($B$), learning rate ($\eta$).
%\SetKwProg{Fn}{Server Executes}{:}{}
\STATE \textbf{Server Executes:}
\STATE Initialize the server model with $\btheta_g^0$ \;
\FOR {each round $t = 0, 1, \ldots, T-1$}
\STATE $m \leftarrow {\rm{max}}(C \cdot N,1)$ \;
\STATE $S_t \leftarrow $(random set of m clients) \; %\tcp*{set of $m$ available clients}
\FOR {each client $k \in S_t$ \rm{\textbf{in parallel}}}
\STATE $\btheta^{t+1}_{k}\leftarrow {\mathtt{ClientUpdate}}(k; \btheta^t_{g} )$ \; 
%$\theta^{t+1}_{g}=\sum_{k \in S_t }{|D_{k}|\theta^{t+1}_{k}}  /\sum_{k \in S_t }{|D_{k}|}$
\ENDFOR
\STATE $\btheta^{t+1}_{g}=\mathtt{ModelFusion}(\{\btheta_k \}_{k \in S_t})$
\ENDFOR
\vspace{1mm}
\FUNCTION {$\mathtt{ClientUpdate}(k, \btheta^t_{g})$}
\STATE $\btheta^{t}_k \leftarrow \btheta^t_{g}$ \;
\STATE $\mathcal{B} \leftarrow$ ({randomly splitting $D_k^{train}$ into batches of Size $B$})\;
\FOR {each local epoch $ \in \{1, \ldots, E\}$}
\FOR {each batch $\mathbf{b} \in \mathcal{B}$}
\STATE $\btheta^{t}_k \leftarrow \btheta^{t}_k - \eta \nabla_{\btheta^{t}_k} \ell(f_k(\bx; \btheta^{t}_k), y)$\;
\ENDFOR
\ENDFOR
\STATE $\btheta^{t+1}_k \leftarrow \btheta^{t}_k$\;
\ENDFUNCTION
% \Fn{\FClient{$k, \btheta^t_{g}$}}{
%     \STATE $\btheta^{t}_k \leftarrow \btheta^t_{g}$ \;
%     \STATE $\mathcal{B} \leftarrow$ ({randomly splitting $D_k$ into batches of Size $B$})\;
%     \For {each local epoch $i = 1, \dots, E$} {
%         \For {each batch $\mathbf{b} \in \mathcal{B}$} {
%             \STATE $\btheta^{t}_k \leftarrow \btheta^{t}_k - \eta \nabla_{\btheta^{t}_k} f_k(\btheta^{t}_k; \mathbf{b})$\;
%         }
%     }
%     \STATE $\btheta^{t+1}_k \leftarrow \btheta^{t}_k$\;
%     \KwRet $\btheta^{t+1}_{k}$\;
% }
\end{algorithmic}
\end{algorithm}


% \begin{algorithm}[H]
% \caption{FedAvg}
% \label{alg:FedAvg}

% % \DontPrintSemicolon
% % \SetCommentSty{mycommfont}
% % \SetAlgoLined
% % \SetKwInOut{Req}{Require}
% % \SetKwFunction{FClient}{ClientUpdate}
% % \SetKwProg{Fn}{}{:}{}

% \begin{algorithmic}[1]
% \STATE \textbf{Require:} number of clients ($N$), sampling rate ($C\in(0,1]$), number of communication rounds ($T$), local dataset of client $k$ ($D_k$), number of local epoch ($E$), number of local batch size ($B$), learning rate ($\eta$).
% %\SetKwProg{Fn}{Server Executes}{:}{}
% \STATE \textbf{Server Executes:}
% \STATE $\quad$ Initialize the server model with $\btheta_g^0$ \;
% \FOR {each round $t = 0, 1, \dots, T-1$}
% \STATE $\quad$ $m \leftarrow {\rm{max}}(C \cdot N,1)$ \;
% \STATE $\quad$ $S_t \leftarrow $(random set of m clients) \; %\tcp*{set of $m$ available clients}
% $\quad$ \FOR {each client $k \in S_t$ \rm{\textbf{in parallel}}}
% \STATE $\quad$ $\btheta^{t+1}_{k}\leftarrow {\mathtt{ClientUpdate}}(k; \btheta^t_{g} )$ \; 
% %$\theta^{t+1}_{g}=\sum_{k \in S_t }{|D_{k}|\theta^{t+1}_{k}}  /\sum_{k \in S_t }{|D_{k}|}$
% \ENDFOR
% \STATE $\quad$ $\btheta^{t+1}_{g}=\mathtt{ModelFusion}(\{\btheta_k \}_{k \in S_t})$
% \ENDFOR
% \vspace{1mm}
% \FUNCTION {$\mathtt{ClientUpdate}(k, \btheta^t_{g})$}
% \STATE H
% \ENDFUNCTION
% % \Fn{\FClient{$k, \btheta^t_{g}$}}{
% %     \STATE $\btheta^{t}_k \leftarrow \btheta^t_{g}$ \;
% %     \STATE $\mathcal{B} \leftarrow$ ({randomly splitting $D_k$ into batches of Size $B$})\;
% %     \For {each local epoch $i = 1, \dots, E$} {
% %         \For {each batch $\mathbf{b} \in \mathcal{B}$} {
% %             \STATE $\btheta^{t}_k \leftarrow \btheta^{t}_k - \eta \nabla_{\btheta^{t}_k} f_k(\btheta^{t}_k; \mathbf{b})$\;
% %         }
% %     }
% %     \STATE $\btheta^{t+1}_k \leftarrow \btheta^{t}_k$\;
% %     \KwRet $\btheta^{t+1}_{k}$\;
% % }
% \end{algorithmic}
% \end{algorithm}

% \begin{algorithm}[H]
% \caption{FedProx}
% \label{alg:FedProx}

% \DontPrintSemicolon
% \SetCommentSty{mycommfont}
% \SetAlgoLined
% \SetKwInOut{Req}{Require}
% \SetKwFunction{FClient}{ClientUpdate}

% \Req{number of clients ($N$), sampling rate ($C\in(0,1]$), number of communication rounds ($T$), 
% local dataset of client $k$ ($D_k$), number of local epoch ($E$), number of local batch size ($B$), learning rate ($\eta$), FedProx parameter ($\mu$)\;}
% \SetKwProg{Fn}{Server Executes}{:}{}
% \Fn{}{
% Initialize the server model with $\theta_g^0$ \;
% \For {each round $t = 0, 1, \dots, T-1$} {
%  $m \leftarrow {\rm{max}}(C \cdot N,1)$\;
%  $S_t \leftarrow $(random set of m clients)\; %\tcp*{set of $m$ available clients}
% \For  {each client $k \in S_t$ \rm{\textbf{in parallel}}}
% { 
% $\theta^{t+1}_{k}\leftarrow {\mathtt{ClientUpdate}}(k; \theta^t_{g} )$ \; 
% }

% $\theta^{t+1}_{g}=\sum_{k \in S_t }{|D_{k}|\theta^{t+1}_{k}}  /\sum_{k \in S_t }{|D_{k}|}$
% }
% }
% %\Indm
% \SetKwProg{Fn}{}{:}{}
% \Fn{\FClient{$k, \theta^t_{g}$}}{
%      $\theta^{t}_k \leftarrow \theta^t_{g}$\;
%      $\mathcal{B} \leftarrow$ ({randomly splitting $D_k$ into batches of Size $B$})\;
%     \For {each local epoch $i = 1, \dots, E$} {
%         \For {each batch $\mathbf{b} \in \mathcal{B}$} 
%         {
%              $\theta^{t+1}_k \leftarrow \argmin_{\theta_k} \mathbb{E} \left[ f(\theta_k; \mathbf{b}) \right] + \frac{\mu}{2} ||\theta_k - \theta^t_k||^2$\;
%         }
%     }
%     $\theta^{t+1}_k \leftarrow \theta^{t}_k$\;
%     \KwRet $\theta^{t+1}_{k}$\;
% }
% \end{algorithm}


\subsection{Performance Indicators}\label{subsection:performance_indicators_definition}
We use three categories of indicators to analyze the performance of a scenario definition:

\textbf{Inefficiency Rate}: it is the difference in fuel spent between a traffic scenario in which the aircraft comply to DAA resolution advisories (RA), and the analogous scenario in which each vehicle follows its optimal path, without observation of traffic separation rules. In case of vertical deviations, a higher fuel rate is required for climb maneuvers, a lower fuel rate in descent maneuver, which increase the net total for the mission. The resulting value of this indicator is the average value over all traffic configurations in a scenario set. 

\textbf{Loss of Separation (LoS) Rate}: despite there being several ways of defining traffic separation, we examine just the simplest one, which is checking whether or not the vehicles are separated by at least a fixed \emph{minimum distance}. In order to include both DAIDALUS and ACAS sXu in the same tables, we use one separation distance from each one, respectively: 4,000 ft, which is the Horizontal Miss Distance, or HMD, used in DO-365B \cite{DO_365} to define the so-called \emph{Hazard Alert Zone} (HAZ), associated to the DAA Well-Clear (DWC) concept of separation; and 2,000 ft, used in DO-396 \cite{DO_396}, that defines the Loss of Well-Clear (LoWC) event in relation to large UAVs or manned aircraft. These indicators will denote the rate of scenarios, in a scenario set, where the distance between any aircraft pair fell below the afore mentioned threshold values. 

\textbf{Timeout Rate}: this indicates, in a scenario set, the rate of scenarios where any aircraft exceeded a maximum time without reaching its destination point. As pointed out in section \ref{section:scenario_definitions}, this phenomenon occurs because DAA Resolution Advisories (RAs) cause long chains of maneuvers that extend beyond the energy/fuel allowance of the vehicle, due to shortcomings in coordination. We use a time threshold of 1,000 seconds but, in practice, the timeout would be determined by the energy/fuel capacity of the vehicle.

\textbf{Scenario Computing Time}: the time needed to simulate a single scenario instance, in a single core of an Intel Xeon CPU, discounted the fact that multiple scenario instances can be run in parallel in a multi-core CPU. In our simulated scenarios, the DAA algorithm is called at least each 2 seconds, for each aircraft, but when the aircraft is in avoidance mode, that can happen more often. In the case of ACAS sXu, the requirement of receiving various messages to update a single track contribute to result in multiple calls per simulated second.


\subsection{Scenario Specifications and Labels}
In this study, a scenario specification is defined by features such as: the DAA algorithm used, the dimensionality (2-D or 3-D), if it uses extrinsic priorities or not, the target separation parameter, and possibly other features. The scenario labels used in this section encode these attributes:
\begin{itemize}
\item \texttt{dai\_ip\_2d\_4k}: DAIDALUS without extrinsic priorities, 2-D maneuvering and regular 4 kft Horizontal Miss Distance (HMD);
\item \texttt{dai\_ep\_2d\_4k}: similar to the above, with extrinsic priorities;
\item \texttt{sxu\_ip\_2d\_2k}: ACAS sXu with intrinsic priorities only, 2-D maneuvering and regular 2 kft LoWC threshold;
\item \texttt{sxu\_ep\_2d\_2k}: similar to the above, with extrinsic priorities;
\item \texttt{dai\_ep\_3d\_4k}: similar to \texttt{dai\_ep\_2d\_4k}, with 3-D maneuvering;
\item \texttt{dai\_ep\_2d\_2k}: similar to \texttt{dai\_ep\_2d\_4k}, with HMD reduced to 2 kft.
\end{itemize}

And there are other features and labels that will be mentioned below as needed.

\subsection{Performance Analysis}
The analysis of selected scenario specifications is summarized in table~\ref{table:performance_analysis}. The first notorious observation in this table is the effect of extrinsic priorities to decrease inefficiency. With regards to safety indicators, their effect is mixed, and we have to observe each case separately. In the case of DAIDALUS, priorities decreased the 2 kft LoS rate and, most drastically, the timeout rate, while increased the 4 kft LoS rate. In the case of ACAS sXu, priorities increased both LoS rate indicators, but decreased the timeout rate drastically. Based on our rule of thumb assessment, it can be said that DAIDALUS works better with extrinsic priorities, while ACAS sXu works better without them. We conjecture that the following reasons explain this fact: i) that DAIDALUS has more built-in symmetries than ACAS sXu; ii) that ACAS sXu has already built-in priority rules for multi-aircraft encounters, and extrinsic priorities may contradict with them.

\begin{table}[h]
    \caption{Summary of closed-loop performance indicators per scenario.}
    \label{table:performance_analysis}
    \begin{center}
    \begin{tabularx}{\columnwidth}{|p{0.20\columnwidth}|p{0.11\columnwidth}|p{0.09\columnwidth}|p{0.09\columnwidth}|p{0.1\columnwidth}|p{0.105\columnwidth}|}
        \hline
    Scenario spec. & Ineffici-ency rate & LoS rate 4~kft & LoS rate 2~kft & Timeout rate & Scenario comp. time (s)\\
    \hline
    \texttt{dai\_ip\_2d\_4k} & 9.71\% & 1.4E-2 & 6.5E-5 & 1.6E-2 & 6.5E-2\\
    \texttt{dai\_ep\_2d\_4k} & 4.83\% & 2.4E-2 & 4.1E-5 & 0 & 5.1E-2\\
    \texttt{sxu\_ip\_2d\_2k} & 20.7\% & 8.7E-1 & 4.1E-2 & 1.6E-3 & 7.8E+1\\
    \texttt{sxu\_ep\_2d\_2k} & 10.9\% & 9.0E-1 & 2.0E-1 & 1.8E-5 & 7.8E+1\\
    \texttt{dai\_ep\_3d\_4k} & 4.38\% & 8.9E-6 & 0 & 0 & 7.4E-2\\
    \texttt{dai\_ep\_2d\_2k} & 1.3\% & 9.0E-1 & 1.1E-1 & 0 & 3.9E-2\\
    \hline
    \end{tabularx}
    \end{center}
\end{table}

According to a line of reasoning, it would be expected, that, in the more efficient scenarios, the aircraft fly closer to each other and, therefore, there should be a higher probability of losing separation. But this is not the only principle at play, because, if the aircraft perform deviations with the least extra distance, while keeping separation, they stay less in the air and decrease the total number of conflicts. This becomes more understandable when we compare \texttt{dai\_ep\_2d\_4k} with \texttt{dai\_ep\_3d\_4k}, where the latter achieved a small advantage in efficiency, but a huge one in safety. \texttt{dai\_ep\_3d\_4k} is capable of shortening the total distances, but has a residual cost associated to vertical maneuvers, where the climb maneuvers spend fuel at higher rates. 

It cannot escape from observation that DAIDALUS performed much better than ACAS sXu in almost all indicators. In our opinion, it would be reasonable to expect that ACAS sXu would not excel in the LoS rates, especially that of 4 kft, because its first protection criterion is 2 kft, as a built-in feature. However, with smaller protection volumes, the deviations should be smaller and, by this reasoning, its expected inefficiency would be lower than that of DAIDALUS. But our results show otherwise when we compare the cases of DAIDALUS with those of ACAS sXu. The only case in which ACAS sXu obtained an advantage was for the LoS rates comparison between \texttt{sxu\_ip\_2d\_2k} and \texttt{dai\_ep\_2d\_2k}, which have the same separation target. In any case, the Los rate obtained for ACAS sXu meets the performance requirement of ASTM F3442 \cite{ASTM}, which uses the definition of LoWC Ratio (LR), which is the ratio between the LoS rate of 2 kft shown in table~\ref{table:performance_analysis}, with DAA active, and corresponding LoS rate with DAA inactive, the latter being 0.785 according  to our simulations. Thus, the resulting LR scores for ACAS sXu here are 0.052 and 0.253, respectively for the two ACAS sXu specs, which are well below the value of 0.4 from \cite{ASTM}, and consistent with the performance analysis of \cite{DO_396}. We conjecture that these scores would be lower in a future 3-D scenario spec of a ACAS sXu, by following the same improvement obtained with DAIDALUS. 

\subsection{Possible approximations to the closed-loop behavior}
Trying to alleviate the heavy computational load to simulate large numbers of different traffic configurations, in this multi-aircraft, closed-loop setup, we considered some approximated solutions, such as the use of Deep Neural Networks to emulate the ACAS Xu/sXu behavior, in the lines followed by \cite{Julian2018,Bak2022}. However, the existing solutions that we found available were developed for just one intruder aircraft, so they were not suitable for our study. Another possibility would be the exploitation of symmetry transformations \cite{Sibai2020}, however the history-dependent nature of the ACAS Xu/sXu algorithms, associated to the present closed-loop setup, make this possibility unpractical. Thus, we started exploring simpler ways of deducing closed-loop behavior without having to perform the full simulation of a scenario. So far, we tried to analyze correlations between measures of open-loop maneuvers and the closed loop performance. The features that we explored are:
\begin{itemize}
    \item \textbf{Distance flown until the end of the first deviation maneuver} ($\overline{M/D}$): we consider the total distance flown until a ``Clear-of-Conflict'' (CoC) event happens, that is, after one or more divergent maneuvers start in a scenario instance, we stop the scenario when the first divergent maneuver of any aircraft finishes and that individual aircraft is clear of conflict, the moment from which some decision must be made on how to continue the mission. We count the total number of maneuvers started, and divide it by the sum of the flown distances, across all scenario instances in an execution set associated to a scenario spec.
    \item \textbf{Average angle deviation maneuver} ($\overline{\alpha}$): using the same stopping rule of above, we account the angle difference between the heading angles of the aircraft at the beginning of the divergent maneuver and at the stopping moment. 
\end{itemize}
For each of these measures, we ran the 122,416 traffic configuration instances with the open-loop stopping rule. Here, all the scenario specifications are 2-dimensional, and we use abbreviated lables to achieve a better display in the graph legends. Namely, the scenario specifications in this subsection are defined as:
\begin{itemize}
    \item \texttt{D1}: DAIDALUS without extrinsic priorities and with deterministic sensor data;
    \item \texttt{D2}: DAIDALUS with extrinsic priorities and deterministic sensor data;
    \item \texttt{D3}: DAIDALUS with extrinsic priorities and Sensor Uncertainty Mitigation (SUM). This is a design feature \cite{Narkawicz2018} to mitigate uncertainty in sensor data, as for example, to determine the position of an intruder aircraft;
    \item \texttt{D4}: DAIDALUS with its Horizontal Miss Distance (HMD) set to 200 ft (the standard is 4,000 ft) and uncertain sensor data;
    \item \texttt{X1}: ACAS sXu without extrinsic priorities;
    \item \texttt{X2}: ACAS sXu with extrinsic priorities;
    \item \texttt{X3}: ACAS sXu with scenario downscaled to speed of 43 knots and cell radius of 1 km.
\end{itemize}
We used the values of $\overline{M/D}$ and $\overline{\alpha}$ obtained for each of the specs above as inputs to a linear regressor of inefficiency, as defined in section~\ref{subsection:performance_indicators_definition}, and generated a plot with the pairs of (true, predicted) values from this regression, as shown in fig.~\ref{fig:inefficiency_regression}. The trend line in the figure, which depicts the regressor, seems to represent a strong correlation, which is confirmed by the value of $R^2$. When considering each of the regression inputs separately, we obtain $R^2=0.73$ for $\overline{M/D}$ and $R^2=0.77$ for $\overline{\alpha}$, which show that they contribute with approximately equal predicting power. 
% Figure environment removed

We performed a similar analysis for the LoS indicators, but we found very little correlation, as $R^2=0.11$ for the 2~kft LoS indicator. Nevertheless, it can be concluded that these open-loop measurements are a good proxy for the closed-loop inefficiency, with the advantage that the predictor discounts the bias that may have been introduced by the closed-loop mission management system, which is not part of the DAA specification. A rough estimate for the computing time saved is of 68\% in the inefficiency case.      

\section{Sequential Performance Optimization.}\label{sec:seqexp}
We study the performance improvements achieved by each optimization, the tuning parameters introduced, and performance tradeoffs between the pairwise and triplet variants.
All algorithms were written in C and compiled with the Intel C compiler (\verb|icc|) release 2021.06.
The code was compiled with the following compiler flags: \verb|-Ofast -mavx512 -opt-zmm-usage=high|.
Experiments are performed on a single-node, dual-socket platform with two Intel Xeon Gold 6226R CPUs (16 cores per socket).
We run 5 trials for each experiment and use the mean to compute speedups.
We observe low runtime variance across trials, so we omit error bars for simplicity.
We perform experiments on randomly generated distance matrices for powers of two $n \in \{128, \ldots,4096\}$.
Our code can handle arbitrary square matrix sizes, but we limit performance evaluation to powers of two.
% Figure environment removed
We begin performance tuning by applying one level of blocking to \cref{alg:pairwise} (naive pairwise) and \cref{alg:triplet} (naive triplet).
We show speedups relative to the previous optimization tried in \cref{fig:seqopt} with a fixed $n = 2048$ matrix.
Overall speedup over naive pairwise (resp. naive triplet) may be obtained by multiplying speedups across all optimizations.
Naive triplet resulted in a speedup of $1.11\times$ over naive pairwise due to less computation.
Introducing one level of blocking to naive pairwise led to a speedup of $1.07\times$.
Applying blocking to the triplet variant led to speedups of $1.20\times$ over naive triplet ($1.33\times$ over naive pairwise).
\Cref{alg:pairwise,alg:triplet} require branches to correctly update $U$ and $C$ based on distance comparisons.
Distance comparisons can be vectorized, but updates to $U$ and $C$ cannot due to branching.
We avoid branches in both algorithms by computing auxiliary mask variables and performing FMAs with these explicit masks.
For \cref{alg:pairwise}, we compute the masks: $r = d_{xz} < d_{xy}~\Or d_{yz} < d_{xy}$ and $s = d_{xz} < d_{yz}$. %, where $\Or$represents the boolean or operator.
The variable $r$ indicates that $z$ is in the $(x,y)$ local focus and $s$ determines the entry of $C$ to update.
$C$ can be updated via two FMAs: $c_{xz} = c_{xz} + r \cdot s \cdot (1/u_{xy})$ and $c_{yz} = c_{yz} + (r)(1 - s)(1/u_{xy})$.
Branch avoidance introduces a performance tradeoff by increasing computation (e.g. performing FMAs with explicit zeros) but eliminates branch misprediction overhead.
For \cref{alg:pairwise}, branch avoidance enables a fixed stride length for updates of $C$ and facilitates other compiler optimizations (e.g. auto-vectorization and loop unrolling).
Branch avoidance alone yielded a speedup of $1.7\times$ over naive pairwise.
While branch avoidance allows for vectorization, updates to $c_{xz}$ and $c_{yz}$ require a stride length of $n$.
After blocking, we reduce the stride length to $1$ by updating columns of $C$ instead (see \cref{fig:pairwise_dependency}).
The combination achieved speedups of $20.2\times$ over naive pairwise.

\Cref{alg:triplet} must determine the closest pair of points from a triplet $(x,y,z)$.
We avoid branches in \cref{alg:triplet} by computing three masks from three floating point comparisons: $r = d_{xy} < d_{xz}~\tAnd d_{xy} < d_{yz}$, $s = (1 - r)(d_{xz} < d_{yz})$, and $t = (1 - r)(1 - s)$. %, where $\tAnd$represents the boolean and operator.
$C$ can then be updated using six FMAs:
\begin{align*}
    c_{xy} = c_{xy} + r\left(1/u_{xz}\right), \quad &
    c_{yx} = c_{yx} + r\left(1/u_{yz}\right),\\
    c_{xz} = c_{xz} + s\left(1/u_{xy}\right), \quad &
    c_{zx} = c_{zx} + s\left(1/u_{yz}\right),\\
    c_{yz} = c_{yz} + t\left(1/u_{xy}\right), \quad &
    c_{zy} = c_{zy} + t\left(1/u_{xz}\right).
\end{align*}
Applying branch avoidance to the triplet algorithm yields a speedup of $0.98\times$ due to the stride-$n$ updates to $C$.
When combined with blocking, however, we attain speedups of $20\times$ over naive triplet.
Triplet with branch avoidance and blocking yields a speedup of $1.1\times$ over pairwise with the same optimizations. % branch avoidance and blocking.
We were able to extract additional speedup by replacing floating point operations with integer operations during local focus updates, and ignoring equality in pairwise/triplet distance comparisons.
Each entry of $U$ counts the number of points in the local focus based on distance comparisons, with results stored in a mask register.
If $U$ is stored as a floating point array, then each increment to update $U$ requires an expensive integer mask to 32-bit floating point cast operation.
We avoid this by storing $U$ as an integer array during the local focus computation.
This allowed us to combine casting with computing reciprocals prior to cohesion updates.

The theoretical formulation of PaLD \cite{pald_pnas22} allows for ties in pairwise distances (e.g., $d_{xz} == d_{yz}$).
When ties occur, support is split between cohesion entries $c_{xz}$ and $c_{yz}$ (i.e. $c_{xz} = c_{xz} + r\cdot s \cdot \left(0.5/u_{xy}\right)$).
In finite arithmetic, floating point equality is unlikely due to round-off and truncation.
Avoiding ties is critical for \cref{alg:triplet} which contains more distance tie permutations than pairwise.
Introducing these additional optimizations yields self-relative speedups (over naive) of $25.5\times$ and $26.2\times$ for pairwise and triplet, respectively.
Overall, optimized triplet achieves a speedup of $1.14\times$ over optimized pairwise for $n = 2048$.
% Figure environment removed
We also perform block size tuning for each algorithm.
We experiment with (powers of two) block sizes in the range $[2^5, 2^{10}]$. Optimized pairwise attains a maximum speedup of $25.5\times$ for $n = 2048$ after tuning.

For optimized triplet, updates to $U$ require storing $3$ distinct blocks of $D$ and $3$ distinct blocks of $U$ in cache.
Updates to $C$ require $3$ distinct blocks of $D$, $3$ distinct blocks of $U$, and $6$ distinct blocks of $C$ in cache.
This suggests that different block sizes may be better than a fixed block size.
\Cref{fig:triplet_heatmap} (bottom) illustrates the speedups observed (over \cref{alg:triplet}) for various block size combinations for the optimized triplet algorithm.
We observe a maximum speedup of $26.2\times$ over naive triplet with $\hat{b} = 256$ and $\tilde{b} = 128$.
\begin{table}[t]
\footnotesize
    \centering
    \begin{tabular}{c|c|c}
        $n$ & Pairwise Optimized & Triplet Optimized\\ \hline\hline
        128 & {\bf 0.00117 (1.58$\times$)} & {0.00185} \\ \hline
        256  & {\bf 0.00497 (1.34$\times$)} & {0.00665} \\ \hline
        512  & {\bf 0.0188 (1.18$\times$)} & {0.0221} \\ \hline
        1024 & 0.1274 & {\bf 0.1208 (1.05$\times$)} \\ \hline
        2048 & 0.9942 & {\bf 0.8734  (1.14$\times$)} \\ \hline
        4096 & 8.3623 & {\bf 6.6111  (1.26$\times$)} \\% \hline
    \end{tabular}
    \caption{Running time in seconds (and speedup) comparison of  pairwise and triplet algorithms.}
    \label{tab:seqtimes}
\end{table}
In \cref{tab:seqtimes} we compare running times (and speedups) of optimized pairwise and optimized triplet over a range of input matrix sizes.
For small matrix sizes, where $D,U$ and $C$ all fit in cache, optimized pairwise is fastest (e.g. speedup of $1.58\times$ over triplet at $n = 128$).
This is because $n/b$ is a small integer where lower order terms dominate (see \cref{thm:triplet}).
For larger matrices, optimized triplet performs better (speedup of $1.26\times$ over pairwise at $n = 4096$) due to lower computation cost.
In practice, we expect triplet to be the better sequential variant for most applications of PaLD.
If distances ties must be handled correctly, then pairwise is the better variant due to fewer branches.

Finally, we note that optimized pairwise attains $27.7\%$ of hardware peak at $n = 2048$ and optimized triplet attains $28\%$ at $n = 8192$.
Our Intel CPU has a single-core, single precision peak of $249.6$ Gflops/sec.
Single precision comparisons on our CPU have a cycles-per-instruction (CPI) of $1$ while all other single precision ops have a CPI of $0.5$.
Thus, floating point comparisons are twice as expensive.
See \Cref{sec:pct-peak} for details on percentage of peak calculations for each algorithm.

The combination of all optimizations achieves speedups of $25.5\times$ and $29\times$ for pairwise and triplet, respectively, over naive pairwise (for $n = 2048$).
We observe speedups of $23\times$ and $26.2\times$ over naive triplet.
\section{Shared-Memory Parallel Algorithms}
\label{sec:par-algs}
This section presents the OpenMP parallelization of the optimized sequential pairwise and triplet algorithms.
\Cref{fig:omp_pairwise} shows the OpenMP version of the blocked pairwise algorithm.
The blocked pairwise algorithm first computes $U_{\mathcal{X},\mathcal{Y}}$ with a pass over all $n$ points $z$.
The local focus $z$-loop can be parallelized across $p$ threads using the OpenMP \verb|parallel for| construct.
All threads must write to $U_{\mathcal{X},\mathcal{Y}}$ so a sum-reduction is required to resolve write conflicts.
The cohesion update pass requires the quantities $1/u_{xy}~\forall~(x,y)~\in~\mathcal{X}{\times}\mathcal{Y}$, which can be parallelized without write conflicts.
Cohesion updates are within each column of $C$ to entries of $C_{\mathcal{X},z}$ and $C_{\mathcal{Y},z}$.
The cohesion pass can be parallelized without write conflicts by splitting the $z$-loop across $p$ threads.
\Cref{fig:pairwise_writes} illustrates the write patterns for optimized OpenMP pairwise for $n = 16$, $b = 4$, and $p = 8$.
Updates to entries of $C$ requires corresponding entries from $D$, so $D$ can also be partitioned column-wise.
The pairwise algorithm is amenable to NUMA optimizations due to the regular data dependencies.

\Cref{fig:omp_triplet} shows the OpenMP version of the blocked triplet algorithm. The triplet approach requires reading all of $D$ for local focus and cohesion update passes.
Blocking is performed over triplets of points, $\mathcal{X},\mathcal{Y},\mathcal{Z}$, and updates to $U$ and $C$ become irregular.
We use the OpenMP tasking model \cite{openmp-spec} for parallelism.
Each triplet block, $\mathcal{X}\times\mathcal{Y}\times\mathcal{Z}$, is a new task that can be executed by any available thread.
Tasks in the local focus pass write to $3$ blocks of $U$. $C$ is not symmetric, so the cohesion update pass writes to $6$ blocks.
Write conflicts arise when multiple tasks need to update the same blocks of $U$ or $C$.
We resolve conflicts by annotating dependencies using the \verb|depend| clause with the \verb|inout| modifier.
\Cref{fig:triplet-tasks} shows the write conflicts for the local focus pass.
Each vertex represents one of the $\binom{n/b + 2}{3}$ tasks and is labeled by $\mathcal{X},\mathcal{Y},\mathcal{Z}$ block values, and edges represent conflicts.
The degree for each vertex varies based on the symmetry in the block.
This leads to irregular dependencies which we will show in \cref{sec:omp-perf} are not as amenable to NUMA optimizations.

% Figure environment removed

% Figure environment removed
% Figure environment removed
% Figure environment removed

\subsection{OpenMP Performance.}
\label{sec:omp-perf}
We use OpenMP version 4.5 and test the OpenMP algorithms on randomly generated dense distance matrices with $n \in \{2048, 4096, 8192\}$. We incorporate NUMA optimizations into the pairwise algorithm by controlling thread affinity via the \verb|OMP_PROC_BIND| and \verb|OMP_PLACES| environment variables. We map OpenMP threads to physical cores, by assigning OpenMP thread ids $0$ to $16$ to CPU $0$ and threads $17$ to $31$ to CPU $1$.
% Figure environment removed
A static loop schedule yields best performance due to the pairwise algorithm's regular dependencies.
Each thread reads columns of $D$ and $C$ from thread-local fast memory so updates to $C$ are spatially local.
Thread binding ensures that accesses are temporally local by assigning fixed column blocks of $D$/$C$ to threads.
OpenMP allocates memory pages using a first-touch policy by default.
If a single thread allocates $D$, then $D$ resides in the memory hierarchy of the thread's CPU.
$D$ is typically computed outside the scope of the OpenMP algorithms, so we also study the effects of partitioning $D$ across sockets (i.e. memory binding).

\Cref{fig:numa} shows the speedup achieved by introducing thread binding only and thread + memory binding into the OpenMP pairwise algorithm across three matrix sizes, $n \in \{2048, 4096, 8192\}$.
We use the OpenMP pairwise algorithm without NUMA-aware optimizations as our baseline and report speedups for $32$ OpenMP threads.
When we use thread binding only, we observe average speedups of $1.4\times, 1.5\times,$ and $1.13\times$ for $n = 2048, 4098,$ and $8192$, respectively.
Thread binding with memory binding yields average speedups of speedup of $1.7\times, 1.69\times,$ and $1.2\times$ over the baseline.
We did not perform TLB optimizations, therefore, we observe decreasing speedups for large matrix sizes.
We also found that NUMA optimizations are useful at smaller thread counts, $2 \leq p \leq 16$, by mapping half the threads to CPU $0$ and the other half to CPU $1$.
This mapping provides access to the fast memory hierarchies on both CPUs.
We observe speedups ranging from $1.05\times$ ($n = 4096, p = 2$) to $1.33\times$ ($n = 2048, p = 16$) when splitting threads (where $p \leq 16$) across sockets. %to a baseline of mapping all threads to one CPU.
We experimented with thread binding for the OpenMP triplet algorithm but not memory binding due to the irregular data dependencies.
However, we did not observe significant performance improvements over the baseline, so we omit these results from \cref{fig:numa}.
We obtain best OpenMP scaling when using the \verb|untied| clause, which allows suspended tasks to be resumed on any available thread.
Suspended tasks may cause additional reads from slow memory after restart.
Hence, we do not expect NUMA optimizations to be helpful. %for the OpenMP triplet algorithm without a deterministic task schedule.
% Figure environment removed %
We perform strong scaling experiments in \cref{fig:strong} of the OpenMP variants under the same settings as for \cref{fig:numa} and report self-relative efficiency achieved.
We report efficiencies with and without NUMA optimizations. %Since our experimental platform is a dual-socket system we report the effects of NUMA optimizations across the entire range of $p$ in order to utilize the cache hierarchies on both CPUs.
The pairwise algorithm without NUMA optimizations achieves efficiencies of $24.2\%, 33.5\%,$ and $50.6\%$ at $p = 32$ for $n = 2048, 4096$ and $8192$, respectively.
Including NUMA optimizations yields efficiencies of $42.9\%, 56.6\%,$ and $60.5\%$ for $p = 32$.
The triplet algorithm achieves efficiencies of $28.0\%, 29.2\%,$ and  $40.9\%$ without NUMA optimizations and $36.9\%, 34.9\%$, and $41.2\%$ with NUMA optimizations for $p = 32$.
The triplet algorithm is the faster sequential baseline, hence the OpenMP triplet efficiencies are lower than those reported for OpenMP pairwise. % We focus on self-relative scaling behavior in this section.
We also study weak scaling of the two algorithms with and without NUMA optimizations.
We fix $n^3/p$ over the range of $p$ tested.
We use the matrix sizes $n_1 \in \{2048, 4096, 8192\}$, where $n_1$ is the matrix size at $p = 1$.
\Cref{fig:weak} shows the results of the weak scaling experiments.
The pairwise algorithm without NUMA optimizations attains weak scaling efficiencies of $30.6\%, 48.2\%$, and $61.4\%$ for $n_1 = 2048, 4096,$ and $8192$, respectively at $32$ threads.
With NUMA optimizations, the efficiencies increase to $59.1\%,~ 63.6\%$, and $65.6\%$ for each of the matrix size settings at $p = 32$.
Triplet without NUMA optimizations achieves weak scaling efficiencies of $44.2\%, 49.1\%,$ and $50.1\%$ and $47.6\%, 49.1\%,$ and $50.1\%$ with NUMA optimizations at $p = 32$. %for $n^3/p = 2048, 4096,$ and $8192$, respectively at $32$ threads. With NUMA optimizations included, we observed efficiencies of $47.6\%, 49.2\%,$ and $50.1\%$ for the same settings.
% Figure environment removed


\subsection{Atlas Construction}
In computational anatomy, atlases have been an essential tool for investigating the variability of human organs across populations and facilitating the segmentation of organs in individual patients. Typically, atlases are constructed through an iterative averaging process (\emph{i.e.}, \textit{procrustean averaging}~\citep{ma2008bayesian}) using a population of patient images~\citep{allassonniere2007towards, davis2004large, guimond2000average, joshi2004unbiased, ma2008bayesian, avants2010optimal}. This procedure commences with the registration of images to a common frame of reference, followed by the computation of an average based on the registered images, which serves as the atlas for the current iteration. The iteration cycle continues until convergence has been achieved, resulting in the final atlas. However, these traditional methods tend to blur regions exhibiting high-frequency deformations~\citep{dey2021generative}. This shortcoming arises from the averaging of intensities when constructing the atlas, which invariably results in the loss of high-frequency information essential for capturing anatomical details.

Recent advancements in learning-based registration have demonstrated significant improvements in the quality of constructed atlases while concurrently expediting the atlas construction process. 
\citet{dalca2019learning} pioneered the development of a brain atlas within a deep learning framework, in which an initial approximation of the atlas is derived from the mean of the brain images under study.
This atlas is then jointly optimized with a registration network, employing the VoxelMorph architecture to align the atlas with individual patient images.
Throughout the training process, both the atlas and the registration network weights are updated.
To promote an unbiased atlas and enhance spatial smoothness in the resulting deformation fields, the authors introduce a Gaussian-inspired prior.
This prior serves to penalize sharp deformation changes while simultaneously encouraging minimal average deformation across the entire dataset. 
Moreover, patient demographic information is conditioned into the network architecture, facilitating the generation of conditional atlases that vary according to the specific attributes of different individuals.
This work has inspired a variety of applications. For instance,~\citet{cheng2020unbiased} establish continuous spatio-temporal cortical surface atlases for neonatal brains.
Similarly, both~\citet{zhao2021learning} and~\citet{bastiaansen2022towards} construct continuous spatio-temporal atlases for fetal and infant brains.
Zhao~\emph{et al.} developed a multi-scale spherical registration network featuring group-wise registration, while Bastiaansen~\emph{et al.} applied group-wise registration to volumetric ultrasound images.
Alternatively,~\citet{yu2020learning} constructed an unconditional and universal atlas while incorporating demographic information into the displacement field generation.
This approach explicitly models morphological changes related to attributes as a diffeomorphic deformation, which captures variations in shape and size.
Recognizing that the necessity for images to be affinely aligned in a preprocessing step as suggested in~\citep{dalca2019learning} could not adequately capture the dynamic size and shape development of fetal brain structures,~\citet{chen2021construction} proposed incorporating an affine network, conditioned on patient demographic data, to register the constructed atlas to individual patient images.
This approach preserves the dynamic size and shape variations of patients at different ages.
\citet{li2021cas} proposed integrating the segmentation produced by a segmentation network into the atlas construction method proposed in~\citep{dalca2019learning}.
This method enables the joint training of segmentation and registration networks while simultaneously constructing both image and segmentation atlases. 
Similarly,~\citet{sinclair2022atlas} embraced the concept of jointly training segmentation and registration networks.
They were motivated by the observation that segmentation networks often yield spurious voxel-wise predictions.
By warping the label map of the constructed atlas to match the segmentation prediction through learning-based diffeomorphic registration, the topology of the original anatomical structure can be preserved, thus avoiding the potential segmentation errors produced by the segmentation network.
Drawing inspiration from \citet{dalca2019learning} and \citet{shu2018deforming},~\citet{siebert2021learning} proposed using a shared encoder to extract features from input images, followed by two decoders.
One decoder generates an unconditional atlas, while the other produces deformation fields that warp the atlas to individual images.
To improve registration performance and enforce unbiased atlas construction, they introduced an inverse consistency and a bias reduction loss, in addition to the commonly seen similarity measure and deformation regularizer.
In a related study,~\citet{wu2022hybrid} proposed a closed-form update for constructing the atlas by leveraging pre-trained registration networks as a priori knowledge of the deformation field.
Their approach involves an alternating update process for both the deformation field, which warps the atlas, and the atlas itself.
This method results in an atlas construction framework that is independent of the registration model choice, offering flexibility in its application.

Researchers have explored various strategies to enhance the quality of the constructed atlases.
\citet{dey2021generative} improved the constructed atlas by incorporating adversarial learning, which improved both the sharpness and centrality of the resulting atlas.
In a similar vein,~\citet{he2021learning} aimed to improve the atlas' sharpness through adversarial learning and by integrating edge information derived from anatomical label maps.
Additionally,~\citet{pei2021learning} leveraged anatomical label maps to improve the quality of the constructed atlas by applying anatomical consistency supervision.
However,~\citet{ding2022aladdin} contended that the importance of atlas sharpness is secondary to the registration model's ability to align corresponding points between images in the atlas space.
Therefore, they focused on the registration model upon which the atlas construction is based and proposed using the constructed atlas as a bridge.
In their method, an image is first warped to align with an atlas and then further warped to match the target image using the registration network. This process facilitates a direct comparison between the warped image and the target image while enabling the construction and evaluation of the atlas without requiring segmentation of the atlas itself.
Inspired by implicit neural shape representations~\citep{mescheder2019occupancy}, ~\citet{yang2022implicitatlas} proposed constructing atlases of anatomical shapes using a continuous occupancy grid instead of representing them in a voxel-based manner.
Given the latent representation of the shape, this alternative approach constructs an atlas based on the linear combination of a learned template matrix.
Their method offers a novel perspective on atlas representation, diverging from traditional voxel-based representations.

Advancements in learning-based atlas construction methods have facilitated the fast construction of high-quality atlases. The following subsection explores the application of the atlases and learning-based image registration in achieving the goal of image segmentation.
%Junyu
\subsection{Multi-atlas Segmentation}
Multi-atlas segmentation is a well-established registration-based segmentation technique in existence for several decades~\citep{rohlfing2003ipmi, iglesias2015multi}.
The typical approach involves registering atlas images or their patches to a target image and fusing the propagated atlas labels.
For deformable registration-based multi-atlas methods, the process of pairwise registration between atlas images and the target image can be computationally expensive and time-consuming. 
However, recent advancements in deep learning-based deformable registration algorithms provide a promising solution to address the speed issue and potentially improve the accuracy of registration, which can subsequently improve the accuracy of multi-atlas segmentation.
While many works have explored the use of deep networks to improve the fusion of multiple registered atlas images~\citep{zhu2020fcn, fang2019automatic, xie2019improving, xie2023deep, yang2018neural}, there are relatively few studies that incorporate deep learning-based deformable registration algorithms into their pipeline.

Ding~\emph{et al.}~\citep{ding2019votenet, ding2020votenet+} proposed VoteNet, which predicts a voxel probability of the agreement between registered atlas images and the segmentation target image.
They adopted Quicksilver~\citep{yang2017quicksilver} as their registration algorithm to speed up the pairwise registration process.
Their follow-up work~\citep{ding2021votenet++} experimented with improving the initial registration results from Quicksilver by incorporating a registration refinement step.
The results showed that registration accuracy is a critical factor in achieving accurate multi-atlas segmentation.
In~\citet{ding2022cross}, the authors addressed the challenging problem of cross-modality multi-atlas segmentation. They proposed a deep network that learns the bi-directional registration between atlas images and the target image, as well as a second network that estimates the weights for label fusion.
To account for the modality differences between the atlas and target images, they used Dice loss as a similarity measure to train their registration network and conditional entropy to train the fusion network.
For registering 3D first-trimester ultrasound images,~\citet{bastiaansen2022multi} proposed a two-stage network for learning an affine transformation.
They then applied the VoxelMorph architecture to perform deformable registration on the affinely aligned images. 
The segmentation of the target images was achieved by propagating the labels of the atlas images and combining them using majority voting.

The good performance of supervised training in image segmentation could be a reason for the relative lack of research on deep learning-based registration in multi-atlas segmentation.
Deep neural networks have demonstrated impressive results in supervised image segmentation tasks, making them a popular choice for many researchers. However, the performance of single atlas segmentation is often used to evaluate the accuracy of a registration algorithm, as discussed in Section~\ref{ss:anatomical_info}.
Due to the close relationship between registration and segmentation, there is an increasing interest in exploring the possibility of integrating the learning of segmentation and registration~\citep{sinclair2022atlas, khor2023anatomically, xu2019deepatlas}.
Overall, the use of deep learning-based registration in multi-atlas segmentation is still in its early stages, and there is a significant opportunity for further research.

\subsection{Uncertainty}
% Shuwen
Accurate registration is critical for many medical image analysis applications, such as image-guided surgery, radiation therapy, and longitudinal studies. 
However, registration uncertainty can arise due to factors such as training data artifacts or predictive model variances.
To address this issue, incorporating registration uncertainty into medical image analysis can help guide the interpretation of the registration results and improve the reliability of various analysis tasks. 

In clinical decision-making, understanding registration uncertainty is critical for image-guided surgery and radiation therapy. The absence of proper registration uncertainty awareness may lead surgeons to presume a substantial registration error throughout the entire region based on a large error in a single location, resulting in the total disregard of registration. Furthermore, the lack of registration uncertainty may also cause surgeons to place unwarranted confidence in regions with inaccurate registration, resulting in potentially severe consequences. 

For image-guided surgery,~\citet{risholm2013bayesian} showed that the registration uncertainty increased at the site of resection using clinical data from neurosurgery for resection of brain tumors, which demonstrated the potential utility of registration uncertainty in recognizing the surgical regions and guiding surgery.
For radiation therapy,~\citet{risholm2011estimation} had previously presented a probabilistic framework to estimate the accumulated radiation dose and corresponding dose uncertainty delivered to significant anatomical structures during radiation therapy, such as the primary tumor and healthy surrounding organs.
The uncertainty in the estimated dose directly results from registration uncertainty in the deformation used to align daily cone-beam CT images with planning CT. 
The accumulated radiation dose is an important metric to monitor during treatment, potentially requiring treatment plan adaptation to conform to the current patient anatomy. 

A study by~\citet{nenoff2020deformable} employed six different deformable registration algorithms to analyze dose uncertainty in proton therapy and investigate their impact on dose accumulation for non-small cell lung cancer patients with inter-fractional anatomy variations.
The results show that dose degradation caused by anatomical changes was more pronounced than the uncertainty arising from using different deformable image registration algorithms for dose accumulation. 
However, accumulated dose variations between these algorithms can still be substantial, leading to additional dose uncertainty.

In longitudinal medical image analysis, registration is an essential step because it enables the comparison of measurements taken at different time points, which is necessary for correcting anatomical variability and tracking changes over time.
Registration uncertainty estimation can be beneficial for longitudinal image processing tasks, such as image smoothing, segmentation prior propagation, joint label fusion, and others. \citet{simpson2011longitudinal}~proposed an approach to calculate the deformable registration uncertainty using a probabilistic registration framework, integrating the uncertainty into spatially normalized statistics for adaptive image smoothing.
This method showed improved classification results in longitudinal MR brain images acquired from Alzheimer's Disease Neuroimaging Initiative compared to not smoothing or using a straightforward Gaussian filter kernel.

In summary, incorporating registration uncertainty into medical image registration can facilitate interpreting registration results and improve the reliability of various medical image analysis tasks. 
It is crucial for clinicians to understand registration uncertainty and its potential applications in clinical decision-making.
Further research is needed to explore other potential applications of registration uncertainty.

\subsection{Motion Estimation}

In the context of medical images, deep learning-based motion estimation has been closely associated with the unsupervised optical flow~\citep{jonschkowski2020matters, stone2021smurf,bian2022learning} and point tracking~\citep{lai2019self, harley2022particle, ranjan2019competitive, bian2022learning} techniques within the computer vision domain. However, the application of motion estimation in medical imaging presents unique challenges, including limited training data, heterogeneous patient data for testing, and special desired properties on the motion field, such as diffeomorphism~(to preserve anatomical relationships) and incompressibility~(to preserve anatomical integrity).
Deep learning-based registration has demonstrated successful outcomes in estimating motion for various organs, such as the human heart, brain, lungs, and tongue.
Registration-based motion estimation plays a significant role in enabling the assessment of changes in the position, shape, and size of organs over time. 
Multiple dynamic imaging modalities are used for motion estimation in medical imaging, including but not limited to cine images,  tagged-MRI~\citep{axel1989heart, axel1989mr}, and echocardiography.

Cine images are a temporal sequence of MR images captured in quick succession, allowing for the monitoring of organ movement and deformation over time.
Recent research~\citep{qin2018joint, morales2019implementation, meng2022mulvimotion, yu2020foal, qin2023generative, lopez2022warppinn, yu2020motion} has successfully applied deep learning-based registration techniques to cine images.
For example, FOAL~\citep{yu2020foal} proposed online optimization to mitigate distribution mismatch between the training and testing datasets for motion estimation, using meta-learning techniques to enable more efficient online optimization with fewer gradient descent steps and smaller data samples, which differs from instance-specific optimization~\citep{balakrishnan2019voxelmorph}. 
\citet{yu2020motion}~applied similarity and smoothness loss to multiple scales of motion fields (pyramid) using a deep supervision strategy. %The study also developed a student model, which runs faster during inference, by training from a teacher model that estimates motion fields more accurately using two-step optimization.
 
The relatively uniform signal within tissues from cine images and the lack of reliable, identifiable landmarks have motivated the exploration of additional regularization methods for estimating motion that is biologically plausible and clinically reliable.
 For example, Qin~\emph{et al.}~\citep{qin2023generative} trained a variational autoencoder-based generative model to capture the prior of biomechanically plausible deformations by reconstructing simulated deformations using finite element models. This prior is then used as regularization during the training of the registration network. 
 Lopez~\emph{et al.}~\citep{lopez2022warppinn} incorporated hyperelastic regularization terms into the framework of physics-informed neural networks~\citep{raissi2019physics} to estimate incompressible motion fields.

Tagged-MRI, on the other hand, employs a spatially modulated periodic pattern to magnetize tissue temporarily, producing transient tags in the image sequence that move with the tissue and capture motion information. It allows for tracking the motion of inner tissue where the region does not have contrast on cine images.
DeepTag~\citep{ye2021deeptag} takes raw 2D tagged images as input and estimates the incremental motion between two consecutive frames using a bi-directional registration network. Then it composes the incremental motion field to estimate motion between any two time frames. Harmonic phase images~\citep{osman1999cardiac} have been found to be more robust to tag fading and imaging artifacts during motion tracking than raw tagged images. DRIMET~\citep{bian2023deep} proposed a simple sinusoidal transformation on the harmonic phase images,  enabling end-to-end training for estimating a 3D dense motion field from tagged-MRI. It also incorporates a Jacobian determinant-based loss that penalizes \textit{symmetrically} for contraction and expansion to estimate a biologically-plaussible incompressible motion field. DRIMET shows promising results in terms of superior registration accuracy, a comparable degree of incompressibility, and faster speed over its traditional iterative-based counterparts~\citep{PVIRA,ilogdemons2011}.

% Echocardiography is a non-invasive and cost-efficient medical imaging procedure that uses ultrasound to create images of the heart. \citep{ta2020semi,ahn2020unsupervised} xxx. 

% 2D vs 3D
Numerous deep learning-based techniques have been devised to estimate 2D motion, and although this may be adequate for certain applications, tracking dense 3D motion is typically necessary or highly desirable when estimating the motion of biological structures.  %The 2D methods require the acquisition of a large number of closely spaced image slices, making them impractical for routine clinical use due to the large amount of time required. 
To address this issue, Meng~\emph{et al.}~\citep{meng2022mulvimotion} integrate features extracted from multi-view 2D cine CMR images captured in both short-axis and long-axis planes to learn a 3D motion field of the heart. The edge map of myocardial wall is used as a shape regularization of the estimated motion field. Alternatively, DRIMET~\citep{bian2023deep} uses sparsely acquired tagged images and interpolates them onto an isotropic grid with a resolution based on the in-plane resolution. This approach is based on the observation that the tag pattern changes slowly in the through-plane direction and therefore will not cause aliasing issues during sampling. By doing so, DRIMET is capable of tracking \textit{dense} 3D motion.

Recent studies have shown that joint learning of segmentation and motion estimation can be mutually beneficial~\citep{qin2018joint,ta2020semi,ahn2020unsupervised}. For instance, Qin~\emph{et al.}~\citep{qin2018joint} employ a dual-branch framework consisting of a segmentation branch and a motion estimation branch to simultaneously estimate motion and segmentation from a sequence of cardiac cine images. During training, a \textit{shared} feature encoder is learned under the premise that joint features can complement both tasks. In contrast, Ta~\emph{et al.}~\citep{ta2020semi} and Ahn~\citep{ahn2020unsupervised} adopt a task-level approach to jointly tackle motion estimation and segmentation in the context of estimating cardiac motion from echocardiography. Specifically, they warp the segmentation (of one time frame) using the estimated motion field and regularize the motion field by incorporating shape information obtained from the segmentation. This approach differs from previous studies which couple motion estimation and segmentation at the feature-level, and may offer a novel perspective on joint learning of these tasks.

In addition to MRIs and echocardiography, numerous deep learning-based algorithms have been developed for motion estimation with 4D-CT~\citep{fu2020lungregnet,ho2023unsupervised,wolterink2022implicit,fechter2020one,hering2021cnn,ji2022one}. 4D-CT imaging captures images at different phases of respiratory or cardiac cycles, providing valuable insights for lung imaging applications, including radiation therapy planning and lung function assessment. DIR-LAB~\citep{castillo2009framework} is a widely-used dataset, containing 4D CT images of ten patients, to evaluate 4D-CT registration techniques, with the aim of registering inspiration images to expiration images. This task is challenging due to the superimposed motion of the heart and lungs, which is larger in scale than the small lung structures being studied.

LungRegNet~\citep{fu2020lungregnet} trains two separate networks to handle large lung motion. One network predicts large motion on a coarse scale, and the other network takes the coarsely warped image and fixed image as input to predict fine motion. In addition to similarity and smoothness losses, an adversarial loss is applied as extra regularization to prevent unrealistic deformed images. Hering \emph{et al.}~\citep{hering2021cnn} employs a coarse-to-fine multi-level optimization strategy. The deformations of coarse levels provide an initial guess for subsequent finer levels. Networks are trained progressively, with each handling one level and initialized with parameters from the previous level. It incorporates a penalty for volume change and utilizes an $l2$ loss function to match corresponding keypoints that are automatically detected. Ho \emph{et al.}~\citep{ho2023unsupervised} applied cycle-consistent training~\citep{kuang2019cycle} to reduce foldings using two networks. After the first network's forward pass, the warped and moving images are sent to the second network to predict inverse deformation, with a similarity loss applied to maximize the similarity between the moving images and inversely-deformed moving images. IDIR~\citep{wolterink2022implicit} use a multi-layer perceptron to represent the transformation function of coordinates and demonstrate the ability to incorporate the Jacobian regularizer, hyperelastic regularizer~\citep{burger2013hyperelastic}, and bending energy~\citep{rueckert1999nonrigid} into the framework. The resulting deformation is void of foldings and achieves a mean target registration error (TRE) of 1.07 mm on DIR-LAB datasets. However, this method requires more time compared to CNN-based approaches, prompting researchers to consider acceleration as a potential future direction. 

In order to accurately register previously unseen images outside of training datasets, the application of one-shot learning has been employed for the estimation of lung motion~\citep{fechter2020one,ji2022one}. Fechter \emph{et al.}~\citep{fechter2020one} concatenated images captured at different phases in the channel dimension in order to leverage temporal information. To minimize memory requirements, they partitioned images into non-overlapping patches and applied a boundary smoothness constraint on the transitions between patches. Additionally, they utilized a coarse-to-fine approach by constructing an image pyramid, where the estimated vector fields of finer scales were added to the upsampled vector fields of coarser scales. The proposed method showed a competitive performance without the need for training in advance. 

\subsection{2D-3D Registration}
Recent progress in the field of interventional procedures for invasive treatment protocols has been associated with high precision in surgeries performed at a reasonable cost~\citep{pfandler2019technical, dlouhy2014surgical}. In these procedures, 2D-3D registration plays a significant role in determining the spatial relationship between the 3D anatomical structures and 2D images, such as X-Ray fluoroscopic images, ultrasound image frames, or endoscopic images. 2D-3D medical image registration primarily involves registering 2D interventional images to 3D pre-operative CT/MR images, \emph{i.e.}, to obtain the 3D geometric transformation that aligns with the 2D view available. 
Conventional 2D-3D registration methods involve iterative optimization methods with similarity metrics \citep{maes1997multimodality}~based on image intensity as the objective function. 
Due to the sparsity of spatial information derived from 2D images, the problem is non-convex, which may lead to convergence at a local minimum if the initial estimate is not sufficiently close to the correct one. 2D-3D registration is a problem with a minimum of six degrees of freedom which may also lead to registration ambiguity as the spatial information along each projection line is compressed to a single point in the 2D plane. This high-dimensional optimization problem increases the difficulty of determining the parameters associated with the depth of anatomical features in the 3D volume. 
%Alternatively, since deep-learning methods do not require explicit functional mappings, it has gained much popularity for this application\citep{unberath2021impact}. 
Alternatively, deep-learning-based methods have gained popularity for this application as they do not require explicit functional mappings~\citep{unberath2021impact}.
In this discussion, we briefly highlight recent advancements in 2D-3D registration, while directing interested readers to \citep{unberath2021impact} for a comprehensive review of the influence of various learning-based methods in this area.

Common 2D-3D registration applications and examples include registration of 2D fluoroscopic/angiography images to 3D CT/MR images of pelvic, lung, or brain regions~\citep{Gu2020ExtendedCR, liao2019multiview, gao2020generalizing, gao2020fiducial, jaganathan2023self, huang2022novel}, registering endoscopy images to CT/MR images~\citep{liu2020extremely, bobrow2022colonoscopy}, and registering 2D Ultrasound (US) frames to 3D MR images to facilitate interventional procedures, such as liver tumor ablation \citep{Wei2021a} or prostate cancer biopsy\citep{guo2022ultrasound}. 

In~\citep{Gu2020ExtendedCR, Wei2021a, huang2022novel}, the 2D-3D registration problem was modeled as a regression learning problem where the network is trained to directly predict the desired geometric parameters. These models are trained by completely relying on the data, \emph{i.e.}, it has little to no tie to the actual image formation physics involved. Specifically, in \citep{Gu2020ExtendedCR} a 2D X-Ray image is registered to a 3D CT volume using a ConvNet, which takes the X-Ray image and a digitally reconstructed radiograph (DRR) from the CT volume at some known pose as input. 
The ConvNet regresses a geodesic loss function over the geometric parameter space to estimate the relative pose between the fixed X-ray image and the DRR from the CT volume without the need for accurate pose initialization.

In~\citep{Wei2021a}, Wei~\emph{et al.} propose a two-step registration process to determine the position and orientation of the ultrasound plane in the 3D MR volume data. In the first step, a ResNet-18 network is employed to determine the US probe orientation. Following this, a U-Net is used to regress a weighted dice loss function, which facilitates the determination of the orientation and position of the corresponding XY plane in the resampled 3D MR volume associated with the US frame.
In \citep{huang2022novel}, Huang~\emph{et al.} also implemented a two-step registration process for aligning 3D MR vessel wall images (VWI) with 2D Digital Subtraction Angiography (DSA) images. This approach encompasses a ConvNet regressor~\citep{miao2016cnn} that estimates the initial pose, followed by an instance-based centroid alignment, which serves to further minimize parameter estimation errors between the images.

As an alternative to formulating registration as a regression problem that necessitates ground truth transformation parameters, several recent studies~\citep{liao2019multiview, gao2020generalizing, gao2020fiducial, jaganathan2023self, guo2022ultrasound} have explored framing it as an unsupervised optimization problem. In such a formulation, the cost function is determined by a similarity metric measured between the transformed and fixed images. Liao~\emph{et al.}~\citep{liao2019multiview} trained a network to track a set of points of interest (POIs) derived from the 3D CT volume in the 2D DRR and in the multi-view fluoroscopic 2D images (used as fixed images), enabling the network to learn the spatial correspondences between the POIs. In this method, a Siamese U-Net architecture is employed to extract features from the DRRs and fixed images, subsequently tracking the POIs within the extracted features. A triangulation layer is incorporated to pinpoint the locations of the tracked POIs within the fixed image in 3D space. Finally, the geometric transformation between the estimated locations of POIs derived from the fixed image and their true positions is determined analytically. In \citep{gao2020generalizing}, Gao~\emph{et al.} proposed a novel differential volume rendering transformer network combined with a feature extraction encoder to approximate the image similarity metric in a manner that renders the geometric parameter estimation as a convex problem with respect to the pose parameters. %In ~\citep{jaganathan2023self} Jaganath~\emph{et al.} propose a novel unsupervised network to perform annotation-free 2D-3D registration between 2D X-Ray projection image and CT volumes. %The network is a combination of Barlow Twins network, adversarial feature discriminator and DL-based registration network. 

The examples and applications discussed thus far have primarily focused on rigid 2D-3D registration. However, non-rigid 2D-3D registration is essential in certain applications, such as cephalometry~\citep{Li2020Nonrigid} and lung tumor tracking in radiation therapy~\citep{foote2019real,dong20232d}. Cephalometry, for instance, involves formulating the problem as deformed 2D-3D registration with the objective of generating a 3D volumetric image from a 2D X-ray image using a 3D skull atlas. Li~\emph{et al.}~\citep{Li2020Nonrigid} developed a convolutional encoder that uniquely codes the cephalogram image into a volumetric image. The network is trained by minimizing the NCC between the synthesized DRR originating from the volumetric image and the 2D cephalogram. 

Numerous deep learning-based models and metrics have been developed to improve the performance of 2D-3D registration in specific applications, although these methods are specialized and not as versatile as traditional optimization methods. Nonetheless, Machine Learning/Deep Learning has been instrumental in tackling the persistent challenges associated with algorithmic approaches. These techniques have tackled a narrow optimal range of parameters, while also decreasing registration ambiguity. CNN-based approaches are also comparably fast. These factors encourage users to further improve learning-based 2D-3D registration pipeline. 

\section{Conclusion and Future Work}
In this work, I design corruption-robust algorithms for the Lipschitz contextual search problem. I present the \emph{agnostic checking} technique and demonstrate its effectiveness in designing corruption-robust algorithms. There are several open problems for future research. First, in the algorithm I propose for pricing loss, the schedule for agnostic checks is fixed upfront. Can the learner design an adaptive checking schedule for the pricing loss? Second, this work assumes the learner has knowledge of the Lipschitz constant $L$. Can the learner design efficient no-regret algorithms without knowledge of $L$? 

\bibliographystyle{siam}
\bibliography{refs}

\clearpage
\appendix
\begin{comment}
\section{System Architecture}
\label{appendix:architecture}
\system has a novel modularized system architecture with three key components: 
\emph{StreamManager}, 
\emph{TxnManager} and \emph{TxnScheduler}. 
These components are instantiated in each thread locally.
The execution outline of \system is presented in Algorithm~\ref{alg:algo}.
Transactional stream processing is continuous and potentially never ends (Line 1$\sim$8).
The dependency resolution and execution of state transactions are separated into two non-overlapping phases by punctuations~\cite{Tucker:2003:EPS:776752.776780} (Line 2 and 5), which guarantees that no subsequent input event will have a smaller timestamp. 
Effectively, a batch of state transactions is collected during the first phase, and processed during the second phase.

In the first phase (i.e., stream processing phase), 
the \emph{StreamManager} conducts preprocessing for every input event ($e$). Similar to some prior works~\cite{tstream}, state transactions may be issued but not immediately processed during preprocessing (Line 3).
The \emph{pre\_processing} and \emph{post\_processing} functions are exposed as APIs to users.
The \emph{TxnManager} handles dependency resolution (Line 4) among state transactions and insert decomposed operations to construct a \tpg. We discuss the detailed two-phase \tpg construction process in Section~\ref{subsec:construction}.

In the second phase  (i.e., transaction processing phase), 
the \emph{TxnManager} is first involved again to refine (Line 6) the constructed \tpg with further dependency resolution.
The \emph{TxnScheduler} 
schedules operations for concurrent execution based on the constructed \tpg according to the three dimensions of scheduling decisions (Line 7). 
In particular, a scheduling decision model $M$ is instantiated based on the constructed \tpg (Line 14).
\textbf{\circled{1}} Guided by $M$, execution threads adopt an exploration strategy (Section~\ref{subsec:explore}) to explore the constructed \tpg for operations available to be scheduled constrained by dependencies. 
\textbf{\circled{2}} 
During exploration, one or multiple operations may be treated as the 
% basic 
unit of scheduling (Section~\ref{subsec:granularity}). 
Subsequently, \textbf{\circled{3}} every thread executes operation(s) in the unit of scheduling with various abort handling mechanisms (Section~\ref{subsec:abort_handling}).
Only when state transactions are processed (i.e., committed or aborted) can the associated input events be postprocessed (Line 8) by the \emph{StreamManager} based on transaction processing results.
\end{comment}

\begin{comment}
\begin{algorithm}
\footnotesize
    \KwData{$e$ \tcp{Input event}}
    \KwData{$txn_{ts}$ \tcp{State transaction}}
    \KwData{$G$ \tcp{The currently constructed TPG}}
    \While{!finish processing of input streams}{
        \eIf(\tcp*[h]{Phase 1}){\text{$e$ is not a $punctuation$}}{
                $txn_{ts}$ $\gets$ PRE\_Processing($e$)\;
                \textbf{TPG\_Construction}($G$, $txn_{ts}$)\; 
          }(\tcp*[h]{Phase 2}){
                \textbf{TPG\_Refinement}($G$)\; 
                \textbf{TXN\_Scheduling}($G$)\; 
                POST\_Processing()\;
          }
    }
    
    \SetKwFunction{FMain}{TPG\_Construction}
    \SetKwProg{Fn}{Function}{:}{}
    \Fn{\FMain{$G$, $txn_{ts}$}}{
        $O_{1..k}$ $\gets$ \textbf{Partition} $txn_{ts}$\;
        \ForEach{\text{operation $O_{i}$ $\in$ $O_{1..k}$}}{
            \textbf{Identify} its \ld\;
            $G$ $\gets$ $G$ + $O_{i}$ \;
        }
    }
    \SetKwFunction{FMain}{TPG\_Refinement}
    \SetKwProg{Fn}{Function}{:}{}
    \Fn{\FMain{$G$}}{
        \ForEach{\text{vertex $e_{i}$ $\in$ $G$}}{
            \textbf{Identify} its \td, \pd\;
        }
    }
    
    \SetKwFunction{FMain}{TXN\_Scheduling}
    \SetKwProg{Fn}{Function}{:}{}
    \Fn{\FMain{$G$}}{
        $M$ $\gets$ Instantiated with $G$;\tcp{A decision model}
        \While{!finish scheduling of $G$
        }{
          \textbf{\circled{2}} $Scheduling Unit$ $\gets$ \textbf{\circled{1}} \emph{Explore}($G$, $M$)\; 
            \textbf{\circled{3}} \emph{Execute with Abort Handling} ($Scheduling Unit$)\; 
        }
    }
  \caption{Execution Outline of \system}
  \label{alg:algo}
\end{algorithm}
\end{comment}
\end{document}