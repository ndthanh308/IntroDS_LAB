\section{Architecture of Decentralized AI Service}

\subsection{Requirements}

We now describe a specific architecture based on the general six layer architecture outlined in the last section, allowing \sakshi to be concrete. Our decentralized AI service is designed to enable an open marketplace for AI models where any user can access inference service offered by multiple, untrusted AI service suppliers. Our goal is to ensure that the user is guaranteed a good quality of service and the suppliers get a fair payment for their service. 

There are several challenges that can hinder bootstrapping and growth of such a decentralized service:
\begin{enumerate}
    \item Individual suppliers may not be able to attract enough clients;
    \item The supplier may not apply a good model and return low quality results;
    \item The client may not pay after getting the service.
\end{enumerate}

Each of these challenges is addressed by our decentralized AI service model:
\begin{enumerate}
    \item We allow an aggregator to collectively offer service on behalf of multiple suppliers. The aggregator and suppliers engage in an SLA implemented as a smart contract to ensure that each gets a fair share of the revenue.
    \item We have a proof system for quality of AI services to ensure that suppliers provide the promised quality of service. The proof is implemented through a challenge-response setup executed using a decentralized pool of challenger nodes.
    \item We have smart contracts and payment channels to implement scalable and reliable payment service for the suppliers. This will be supported by an objective dispute resolution mechanism to ensure that suppliers can get paid if they deliver service.
\end{enumerate}

\subsection{The six layer architecture with Witness Chain }

These functionalities of \sakshi are enabled using the  architecture in Figure \ref{fig:LayerArchitecture}.  

% Figure environment removed
 
At the top is the marketplace, a decentralized two-sided platform for buying and selling AI services. A client (user) comes to our marketplace and places an order to access inference service from an aggregator. Both agree on an SLA which contains terms for quality of service and payments. 

Next comes the service layer that provides the APIs for clients to make inference requests to the aggregators. This request is appropriately passed to a matching supplier server using a router deployed as a part of the control layer. Both service and control layer are reminiscent of standard web 2.0 services with multiple servers, with the caveat that the supplier servers can now be hosted by different entities with their own business incentives and without any pre-existing reputation. These servers are bound to an SLA between them and the aggregator. 

All the SLAs that govern the service-payment rules between different parties are deployed as smart contracts as a part of the transaction layer, a decentralization middleware provided by Witness Chain \cite{witnesschain}. The Witness Chain transaction layer not only hosts and provides interfaces for the SLA smart contracts, but also provides state channels to maintain the payment and service state for interacting client, aggregator and supplier. Furthermore, it provides a dispute resolution framework to ensure that the client completes the payment after availing the service. 

 Finally, a proof layer deploys an appropriate Proof of Inference to ensure that the suppliers are using models agreed upon in the SLA. This challenge and verification for this proof is executed by a pool of challengers, Witnesses, provided by Witness Chain. These proofs interact with the transaction layer to ensure service quality promised in the SLA. The Witness Chain challenger nodes executing these proofs are incentivised by Witness Chain using a part of service payment. Witness Chain, in turn, provides a programmable layer for choosing the challenger nodes which can be used to specify how decentralized the challenger pool should be and how well-provisioned each challenger node needs to be. 

A detailed description of each layer is provided in Section \ref{sec:detailed}; the interactions discussed above are depicted at a high level in Figure \ref{fig:ServiceSteps} below. 

% Figure environment removed


\subsection{The economic layer with Eigen Layer}

All entities in the above ecosystem are incentivized to do their job fairly because of the economics underlying the SLA and the incentive system for the challengers. Often, each new blockchain ecosystem launches its own token to provide this cryptoeconomic security. However, this new token may not gain the necessary volume and spread to enforce reasonable security in the early stages, resulting in failure to bootstrap for the ecosystem. 

This problem was solved recently by Eigen Layer \cite{eigenlayer} which provides a framework for using Ethereum cryptoeconomic security by engaging Ethereum validators. Witness Chain integrates with Eigen Layer and uses Eigen Layer operators as challengers to extend Ethereum security to the decentralized AI marketplace. The challengers running the Proof of Inference, the ultimate root of trust in service quality, would have staked/restaked Eth using Eigen Layer.  Witness Chain deploys an additional proof of custody \cite{witnesschain} to ensure that these challengers are being diligent in their job, lest their stake  be slashed. Putting the restaking framework of Eigen Layer together with the proof of diligence/custody by Witness Chain provides a comprehensive economic security layer for \sakshi. 

