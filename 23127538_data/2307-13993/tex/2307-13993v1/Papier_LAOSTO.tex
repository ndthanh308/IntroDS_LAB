\documentclass[aps,prb,amssymb,showpacs,twocolumn]{revtex4-2}
%\documentclass[aps,prl,amssymb,showpacs,twocolumn]{revtex4}


%%%%%%% particular usepackages

\usepackage{graphics,graphicx,amsmath}
\usepackage{tabularx,float}
\usepackage{epsfig}
\usepackage{subfigure}
\usepackage{bm}
\usepackage{wasysym}
\usepackage{bbm}
\usepackage[T1]{fontenc}
\usepackage[english]{babel}
\usepackage[normalem]{ulem}

\usepackage{mathrsfs} %pour le math script
\usepackage{physics}

%%%%%%%%%%% color package allows for color highlighting text (for internal use) %%%
\usepackage{color}
\usepackage{verbatim}
\graphicspath{{fig/}}
\newcommand{\red}{\color{red}}
\newcommand{\blue}{\color{blue}}
\newcommand{\cyan}{\color{cyan}}
\newcommand{\green}{\color{green}}





\unitlength=1mm

\begin{document}

%\bibliographystyle{apsrev4-1}
%\bibliographystyle{apsrev}


\date{\today}



\title{Normal state quantum geometry and superconducting domes in (111) oxide interfaces}

\author{Florian Simon\email[Corresponding author: ]{florian.simon1@universite-paris-saclay.fr}}
\affiliation{Laboratoire de Physique des Solides, Universit\'e Paris Saclay,
CNRS UMR 8502, F-91405 Orsay Cedex, France} 
\author{Mark O. Goerbig}
\affiliation{Laboratoire de Physique des Solides, Universit\'e Paris Saclay,
CNRS UMR 8502, F-91405 Orsay Cedex, France}
\author{Marc Gabay}
\affiliation{Laboratoire de Physique des Solides, Universit\'e Paris Saclay,
CNRS UMR 8502, F-91405 Orsay Cedex, France}
\begin{abstract}
We theoretically investigate the influence of the normal state quantum geometry on the superconducting phase in (111) oriented oxide interfaces and discuss some of its implications in the case of the $\text{LaAlO}_3/\text{SrTiO}_3$ (LAO/STO) heterostructure. Based on a tight-binding representation of this interface, we introduce a low-energy model for which we compute the quantum geometry of the lowest band. The quantum metric exhibits a high peak around the $\Gamma$ point, owing to the closeness of the band to a degeneracy point, while the Berry curvature is negligible. We then compute the conventional and geometric contributions to the superfluid weight. The conventional part increases linearly with the chemical potential $\mu$, a generic behaviour for Schrödinger-like bands. The geometric part shows a dome upon varying $\mu$, and we argue that this is a generic behaviour when the quantum metric is peaked at the zero-filling point (where the filling starts). Both contributions can be of the same order when we include disorder effects, yielding a dome-shaped superfluid weight as a function of the chemical potential. Experimentally, a dome-shaped superconducting temperature is observed when the gate voltage $V_g$ is changed. We suggest that this effect stems from the variation of the chemical potential with $V_g$ and that it mirrors the evolution of the conventional part of the superfluid weight up to optimal doping. Furthermore, we propose that a \textit{second superconducting dome} could be found at larger values of $V_g$, as a result of the dominant contribution of the geometric superfluid weight, which would also matter in saturating the overdoped regime of the observed dome. Such features would underscore the impact of the normal state quantum geometry on the superconducting state.
\end{abstract}
\maketitle

\section{Introduction}
Superconductivity has, since 1911, become a flagship of condensed-matter physics. The main paradigm is
given by the Bardeen-Cooper-Schrieffer (BCS) theory \cite{bardeen_theory_1957} which, in its standard form, consists of quasiparticles in a single, partially filled band, pairing and thus
condensing in a single collective dissipationless state. This single-band approximation has its limits. Indeed, since the 1950s \cite{Adams1959,Blount1962},
it was realized that in a multiband situation, even in the adiabatic limit, each band carrries the influence of the other bands
in the form of two geometric contributions, namely the Berry curvature and the quantum metric \cite{berry_quantum_1989}. These quantities form what we call band/quantum geometry. In the context of superconductivity, this means that even if the Cooper pairing takes place within a single band, it is \textit{a priori} affected by the other electronic bands of the normal state, particularly through the \textit{normal state quantum geometry}. While BCS theory does not take these geometric effects into account, recent studies have theoretically pointed out the relevance of the quantum metric for the superfluid weight \cite{Peotta2015,liang_band_2017,Iskin2018,rossi_quantum_2021,torma_superconductivity_2022}, of flat-band models, as well as of the Berry curvature of Dirac-like systems \cite{simon_role_2022}, such as 2D transition-metal dichalcogenides.

Our study emphasizes the impact of the normal state quantum geometry on superconductivity for (111)-oriented oxide interfaces, and more specifically for the LAO/STO heterostructure \cite{gariglio_research_2016}. Let us point out that the results which we present here may be relevant for other materials, including other (111) oxide interfaces. Along the (111) orientation, the LAO/STO interface has a honeycomb structure with three orbitals per site and can, from that point of view, be seen as a three-orbital version of graphene \cite{doennig_massive_2013}. Starting from a tight-binding modeling of this interface, we derive a three-band low-energy model to quadratic order in the wave vector $k$,  close to the $\Gamma$ point. In this limit, the three bands are isotropic. The lowest one in energy is substantially flatter than the other two and is close to a degeneracy point, suggesting an enhanced quantum geometry. We then compute the quantum geometry of the lowest energy branch, again within the aforementioned low-energy model. Its quantum metric exhibits a large peak at the $\Gamma$ point, while its Berry curvature is much smaller and may thus be neglected. Using these results, we then compute the conventional and geometric superfluid weight \cite{liang_band_2017,torma_superconductivity_2022}, where the geometric contribution is a direct measure of influence of the normal-state quantum metric on the superconducting phase. For example, it has been used to explain the appearance of a superconducting dome in twisted bilayer graphene as a function of carrier density \cite{liang_band_2017,tian_evidence_2023}. For our low-energy model, we find that the conventional contribution is linear in the chemical potential $\mu$, which is a generic feature of Schrödinger-like bands. Similarly, the geometric weight shows a dome upon varying the chemical potential, and we argue that this is a generic behavior in the low-filling limit, when the metric is peaked at the zero-filling point. Taking disorder effects into account allows for the possibility of having a regime when both the conventional and the geometric contributions are of the same order, yielding a superconducting dome as a function of the chemical potential. In the last section, we describe the relevance of our findings for transport experiments performed on LAO/STO (111). Most of the results that were obtained in the framework of the quadratic band approximation carry over to the tight-binding form even when we take spin-orbit effects into account. In order to connect the theoretical and experimental data, one needs to establish the dependence of $\mu$ on $V_g$ or the conductivity. We propose a scenario such that the underdoped and optimally doped regimes of the 2D electronic fluid are dominated by the conventional contribution. The geometric contribution would play a sizeable role in the overdoped regime, resulting in a somewhat saturating plateau. A consequence of this scenario is the appearance of a \textit{second superconducting dome} at a higher range of gate voltages, this time originating from the dome produced by the geometric contribution upon changing the chemical potential. We also discuss the discrepancy between the obtained Berezinskii-Kosterlitz-Thouless (BKT) temperature and the experimentally measured value of the critical temperature.

The paper is organized as follows. In Sec. \ref{sec:TB}, we present our tight-binding model. Its continuum version in the low-energy limit is discussed in Sec. \ref{section_low_energy_model} and allows us to investigate analytically the basic quantum-geometric properties. The different contributions to the superfluid weight in the low-energy model are presented in Sec. \ref{sec:SFW}, and a connection with experimental findings and prospectives can be found in Sec. \ref{section exp}.

\section{Tight-binding model}\label{sec:TB}

We first introduce the relevant tight-binding modelling of the $(111)$ interface. and discuss its various terms. The values of the relevant energy scales, presented in detail in Sec. \ref{section_low_energy_model}, are mainly taken from \cite{rodel_orientational_2014,xiao_interface_2011,khanna_symmetry_2019}. The system has the geometry presented in Fig. \ref{Schema_LAOSTO}.
% Figure environment removed
\newline
The three-dimensional (3D) atomic structure forms a cubic lattice such that in the (111) direction it consists of two layers of two-dimensional (2D) triangular lattices displaced by the vector $\vec{a}_0$ (see Fig. \ref{Schema_LAOSTO}). 
  In the (111) direction, the system of two layers (red and blue) is thus equivalent to a honeycomb lattice.
On each site, we have the three conducting $t_{2g}$ Ti orbitals. We first set the spin-orbit coupling (SOC) to zero and discuss its impact later in Sec. \ref{section exp}. On a honeycomb lattice with two inequivalent sublattices, we thus have a six-band system. The \textit{orbital basis} which we use to write down the tight-binding model is $\big(d^1_{yz},d^1_{xz},d^1_{xy},d^2_{yz},d^2_{xz},d^2_{xy}\big)$, where the superscript $\{1,2\}$ is the sublattice/layer index.

    \subsection{Kinetic term}    
The kinetic part of the model takes into account hoppings between the different lattice sites and orbitals. This term  only considers hoppings between the same orbitals that are located in different layers with amplitudes $t$ and $t_d$ for nearest and next nearest neigbors respectively. The general form of the kinetic term is thus diagonal in the orbitals but off-diagonal in layers. Therefore, in the $\big(d^1_{yz},d^1_{xz},d^1_{xy},d^2_{yz},d^2_{xz},d^2_{xy}\big)$ basis the kinetic term reads as Eq. (\ref{kinetic_6}).
    \begin{equation}
        \begin{pmatrix}
            0&H_{\text{cin}}\\
            H_{\text{cin}}^*&0
        \end{pmatrix}=\tau_x\otimes\text{Re}(H_{\text{cin}})-\tau_y\otimes\text{Im}(H_{\text{cin}}),
        \label{kinetic_6}
    \end{equation}
with $H_{\text{cin}}=t\operatorname{diag}(e,f,g)$ in the basis $(d_{yz},d_{xz},d_{xy})$. The Pauli matrices $\tau_x$ and $\tau_y$ in Eq. (\ref{kinetic_6}) act on the layer index. Explicit expressions for $e,f$ and $g$ may be found in appendix \ref{expressions cinetiques}.


\subsection{Orbital mixing terms}
While the kinetic term does not couple the different orbitals, such couplings are generated at the interface by \textit{orbital mixing}. In appendix \ref{annexe OM}, we show by symmetry considerations that a natural choice is that of Eq. (\ref{orbital mixing matrix}).
    \begin{equation}
        \tau_x\otimes H_{\text{om}}=\tau_x\otimes c_0\begin{pmatrix}
           0&i\delta&-i\alpha\\
           -i\delta&0&i\beta\\
           i\alpha&-i\beta&0
        \end{pmatrix},
        \label{orbital mixing matrix}
    \end{equation} 
    where $\alpha=\sin(\sqrt{3}/2k_x+3/2k_y)$, $\beta=\sin(\sqrt{3}/2k_x-3/2k_y)$, $\delta=-\sin(\sqrt{3}k_x)$ and $c_0$ the strength of the orbital mixing.  Here, we measure the wave vectors in units of the inverse $a_0^{-1}$ of the distance between nearest-neighbor sites in the (111) plane (see Fig. \ref{Schema_LAOSTO}), and $\tau_x$ is again a Pauli matrix acting on the layer degree of freedom. Note that with inversion symmetry, these terms are prohibited. But in reality, interfaces between $\text{LaAlO}_3$ and $\text{SrTiO}_3$ always have corrugation \cite{khalsa_theory_2013,zhong_theory_2013}, such that inversion symmetry is broken and orbitals that would have been orthogonal are not, resulting in non-zero overlap and allowed interorbital hoppings. It will give rise to an orbital Rashba effect. 

    \subsection{Trigonal crystal field}
Note that the (111) interface has a different point symmetry than the orbitals whose symmetry is governed by the (cubic) bulk symmetry of LAO and STO. Therefore the $t_{2g}$ orbitals are not orthogonal to each other in the hexagonal lattice, resulting in a \textit{trigonal crystal field}, where the couplings have the same value because of the hexagonal symmetry. It lifts the degeneracy between the $e_{\pm g}$ orbitals and the $a_{1g}$ orbital within the conducting $t_{2g}$ orbitals of Ti. This trigonal crystal field, of strength $d$, thus couples the different orbitals in the same layers so that it may be written as 
\begin{equation}
    H_d=-d\tau_0\otimes \begin{pmatrix}
        0&1&1\\
        1&0&1\\
        1&1&0
    \end{pmatrix},
    \label{crystal field matrix}
\end{equation}
where $\tau_0$ is the identity matrix indicating that the trigonal crystal field is diagonal in the layer index. 

    \subsection{Confinement energy}
Finally, we need to take into account a confinement term that reflects the different onsite potential for the two sublattices, which reside in different layers. 
It is equivalent to the Semenoff mass in graphene, breaking the $\mathcal{C}_6$ symmetry down to $\mathcal{C}_3$. We have $-VI_3$ for layer 1 and $VI_3$ for layer 2, so that this term may be written as $\tau_z\otimes(-VI_3)$, in terms of the $3\times 3$ identity matrix $I_3$. While this term may be important for other properties of the LAO/STO interface, we will see that it does not affect those studied in this paper, and we will later omit it when reducing the six-band model to two effective three-band models that are related by particle-hole symmetry. 



    \subsection{Six-band model}
With these four terms, the six-band tight-binding model is written in the orbital basis as
\iffalse
\begin{equation}
    H=\begin{pmatrix}
        -VI_3+dH_d&tH_{\text{cin}}+c_0H_{\text{om}}\\
        tH_{\text{cin}}^*+c_0H_{\text{om}}&VI_3+dH_d
    \end{pmatrix}.
    \label{six band matrix}
\end{equation}
\fi
\begin{equation}
    H=\begin{pmatrix}
        -VI_3+H_d&H_{\text{cin}}+H_{\text{om}}\\
        H_{\text{cin}}^*+H_{\text{om}}&VI_3+H_d
    \end{pmatrix}.
    \label{six band matrix}
\end{equation}
A more convenient basis is the \textit{trigonal basis} in which the trigonal crystal field term is diagonal. The latter is detailed in appendix \ref{trigonal basis}. Hereafter, we discuss the band structure described by $H$ in the trigonal basis.

\section{Low-energy model}
\label{section_low_energy_model}
Numerical diagonalization shows that the low-filling regime occurs near the $\Gamma$ point. Moreover, in the vicinity of the latter, there are two band groups of three bands separated by several eV. This is because the gap between the two groups at the $\Gamma$ point can be found to be $2(2t+t_d)\sim6.5$eV, and the kinetic energy %$t$% 
is clearly the largest energy scale. Therefore, for low fillings, it appears possible to simplify the above six-band to two effective three-band models, one for each group. To make a similar structure appear explicitly in $H$, we apply the following unitary transformation
\begin{equation}
    U=\frac{1}{\sqrt{2}}\begin{pmatrix}
        -1&1\\
        1&1
    \end{pmatrix}\otimes I_3
\end{equation}
so that the Hamiltonian is transformed to
%%%%%%
\iffalse
\begin{widetext}
    \begin{equation}
        U^\dagger HU=\begin{pmatrix}
        dH_d-c_0H_{\text{om}}-t\text{Re}(H_{\text{cin}})&-VI_3+it\text{Im}(H_{\text{cin}})\\
        -VI_3-it\text{Im}(H_{\text{cin}})&dH_d+c_0H_{\text{om}}+t\text{Re}(H_{\text{cin}})
    \end{pmatrix}.
    \label{H transforme}
    \end{equation}
\end{widetext}
\fi 
%%%%%%%
\begin{widetext}
    \begin{equation}
        U^\dagger HU=\begin{pmatrix}
        H_d-H_{\text{om}}-\text{Re}(H_{\text{cin}})&-VI_3+i\text{Im}(H_{\text{cin}})\\
        -VI_3-i\text{Im}(H_{\text{cin}})&H_d+H_{\text{om}}+\text{Re}(H_{\text{cin}})
    \end{pmatrix}.
    \label{H transforme}
    \end{equation}
\end{widetext}
Numerical inspection confirms that the diagonal blocks pertain to the two  groups. Thus, we may focus on the lower diagonal block and take it as a low-energy three-band model that reads 
\begin{equation}
    H_3=H_d+H_{\text{om}}+\text{Re}(H_{\text{cin}}).
\end{equation}
A discussion of the validity of this approximation, done in appendix \ref{validity annexe}, shows that with a precision of a few meV, this \textit{three-band approximation} is valid over an area centered at $\Gamma$ and covering approximately ten percent of the Brillouin zone (BZ). To be consistent with this approximation, we need to expand $H_3$ to quadratic order in $k$.

    \subsection{Quadratic three-band model}
    In appendix \ref{annexe DL quad H3}, we show that to quadratic order, we have 
    \begin{widetext}
    \begin{equation}
    H_3=-(2t+t_d)\Big(1-\frac{1}{4}k^2\Big)I_3+\begin{pmatrix}
        d-t_{\text{eff}}(k_x^2-k_y^2)&-2t_{\text{eff}}k_xk_y&ick_x\\
        -2t_{\text{eff}}k_xk_y&d+t_{\text{eff}}(k_x^2-k_y^2)&ick_y\\
        -ick_x&-ick_y&-2d
    \end{pmatrix},
    \label{H3quad}
\end{equation}
\end{widetext}
with $t_{\text{eff}}=(t-t_d)/8$ and $c=3c_0/\sqrt{2}$. Note that $H_3$ is expressed in the trigonal basis (see appendix \ref{trigonal basis}). The trigonal crystal field lifts the threefold degeneracy at the $\Gamma$ point (between $a_{1g}$ and $e_{\pm g}$ states). The linear and quadratic terms arise from the orbital mixing and kinetic terms, respectively. $H_3$ can then be exactly diagonalized, and we find the following eigenvalues for the last term:
\begin{equation}
    \epsilon_1=d+t_{\text{eff}}k^2,\quad\epsilon_2=d+\bigg(\frac{c^2}{3d}-t_{\text{eff}}\bigg)k^2,
\end{equation}
and
\begin{equation}
    \epsilon_3=-2d-\frac{c^2}{3d}k^2,
\end{equation}
to quadratic order in the wave-vector components.
The values taken hereafter are those corresponding to Ref. \cite{rodel_orientational_2014}, i.e. $t=1.6$ eV, $t_d=70$ meV, $V=100$ meV, $d=3$ meV. Additionally, we estimate $c_0=40$ meV.
We thus find an isotropic electron-like band structure. In the remainder of  this section and in the following one, we highlight the most salient features of the quantum geometry in the low-energy limit, where analytical calculations can be readily performed and assess their impact on  superconductivity. We point out that these results may apply to other (111) oxide interfaces. In Section \ref{section exp}, we discuss the relevance of our results in an experimental context, illustrated with the LAO/STO (111) interface. The lowest energy band ($\epsilon_3$) is substantially flatter than the other two. Indeed, its band mass can be computed to be 
\begin{equation}
    m_B=\frac{\hbar^2}{2a_0^2}\bigg(\frac{2t+t_d}{4}-\frac{c^2}{3d}\bigg)^{-1}\simeq21m_0,
    \label{band mass}
\end{equation}
with $m_0\simeq9.1\times10^{-31}$ kg the rest mass of an electron. Note that beyond the low-energy model, and already at the cubic level, the interorbital effects give rise to an orbital Rashba effect which moves the minimum away from the $\Gamma$ point and therefore the actual band mass differs from Eq. (\ref{band mass}).
We then plot this band structure in Fig. \ref{figure structure de bandes}a, and contrast it with the one we get from the tight-binding form of the kinetic and orbital mixing terms (Fig. \ref{figure structure de bandes}b).
% Figure environment removed
\newline
We indeed achieve the aforementioned precision of a few meVs. Note that the general offset of 2 meV seen in Fig. \ref{figure structure de bandes} is due to the confinement potential which globally shifts the bands. Such a global shift does not have a physical relevance on the quantum geometry and superfluid weight as it can be absorbed if we define the chemical potential with respect to the lowest value of the lowest band. We then get a lower band that is substantially flatter than the other bands and that is close in energy to a level crossing at the $\Gamma$ point. This points to an enhanced quantum geometry, which is computed in the following section.

    \subsection{Quantum geometry of the lowest band}
In order to compute the quantum geometry of the state with dispersion  $\epsilon_3$, we write down the $\text{SU}(3)$ decomposition of our multiband Hamiltonian and use the formalism presented in Ref. \cite{graf_berry_2021}. The Hamiltonian vector form of Eq. (\ref{H3quad}) is given in appendix \ref{annexe hamiltonian vector}.
\subsubsection{Quantum metric}
We begin by the quantum metric, which is defined as the real part of the quantum geometric tensor \cite{berry_quantum_1989}
\begin{equation}
    Q^n_{\mu\nu}=\bra{\partial_\mu n}(\mathbbm{1}-\ket{n}\bra{n})\ket{\partial_\nu n}=g^n_{\mu\nu}-\frac{i}{2}\mathcal{B}^n_{\mu\nu},
\end{equation}
and has the physical dimension of a surface. Here, $|\partial_\mu n\rangle$ is the quantum state obtained by deriving the Bloch state $|n\rangle$ associated with the $n$-th band with respect to the component $k_\mu$ of the wave vector.
Using Ref. \cite{graf_berry_2021}, we compute the quantum metric associated with the quadratic three-band model shown in Fig. \ref{gxxyy}.

% Figure environment removed

The diagonal components $g_{xx}$ and $g_{yy} $ exhibit prononced peaks at $\Gamma$,  $g_{jj}(\Gamma)a_0^{-2}=c^2/9d^2\sim90$. This feature stems from the fact that $d$ is small; in the limit $d \to 0$ the quantum metric diverges at $\Gamma$ due to the degeneracy of the energy of the three bands.

% Figure environment removed

Then, we have the transverse component $g_{xy}$. As seen in Fig. \ref{gxy}, the transverse component $g_{xy}$ is odd in $k_x$ and $k_y$. In the next section, we will see that this results in a zero transverse geometric superfluid weight. Note that, within the low-energy model, the orbital mixing terms are necessary in order to get a non-vanishing quantum metric. 


\subsubsection{Berry curvature}
To quadratic order in $k$, the Berry curvature is identically zero. In order to obtain a non-zero Berry curvature one needs to include cubic order terms in the expansion of the orbital mixing contribution, which breaks the isotropy of the problem, reducing it to a $\mathcal{C}_3$ symmetry  and leading to an orbital Rashba effect. In that case, the resulting Berry curvature is finite  as we show in Fig. \ref{Berry}. The Hamiltonian vector with this cubic term  added is given in appendix \ref{annexe hamiltonian vector}. 
% Figure environment removed

The maximum value of the Berry curvature is approximately $0.1a_0^2$, i.e. almost three orders of magnitude smaller than the quantum metric. The impact of a normal state Berry curvature on the superconducting instability for a two-band model has been reported in a previous study \cite{simon_role_2022}. It was found that, in a paradigmatic example, the normal state Berry curvature lowers the attractive electron-electron interaction, thereby weakening superconductivity. However, in the example studied in Ref. \cite{simon_role_2022}, both the dispersion and the geometric quantities were isotropic in reciprocal space. In the present case, the spectrum is isotropic, while the quantum geometry is not. One way to mimic the rotational symmetry that is present in Ref. \cite{simon_role_2022} would be to perform an angular average of the Berry curvature. Inspection of Fig. \ref{Berry} shows that this would result in a vanishing Berry curvature. Furthermore, numerical calculation for the six-band tight-binding model Eq. (\ref{six band matrix}) finds that the Berry curvature is again much smaller than the quantum metric.

In summary, both the low-energy model at quadratic order and the tight-binding model of Eq. (\ref{six band matrix}) at "infinite" order give a negligible Berry curvature. One way to have a non-zero Berry curvature is to go at cubic order in $k$ in the orbital mixing. The term obtained is akin to that of Ref. \cite{Lesne2023}. We obtain a Berry curvature that is significantly weaker than the quantum metric. We will therefore hereafter focus only on the quantum metric. In the following, we then examine the role of the quantum metric on superconductivity.

\section{Superfluid weight}
\label{sec:SFW}

During the last decade, a significant number of papers have discussed the role of the normal state quantum metric on the superconducting state \cite{liang_band_2017,torma_superconductivity_2022,rossi_quantum_2021}. It was found that the expression for the London penetration depth has two terms. One is a well-known contribution coming from the intraband dispersion (see Ref. \cite{chandrasekhar_superconducting_1993}, for example), and the other is a geometric contribution which stems from interband coupling when the normal state is described by more than just one band. In the isolated-band limit (which we consider here), this geometric contribution depends on the normal state quantum metric \cite{liang_band_2017}. Initially, this theory was developed for flat bands where the conventional contribution vanishes and the geometric contribution then dominates. While we do not have flat bands in our model, we have found one band that is significantly flatter than the other two and has a strong quantum metric. It thus seems relevant to investigate whether the normal-state quantum metric has a sizeable effect on the superconducting state through this geometric superfluid weight. In this section, we discuss the two contributions in the context of our low-energy model. 

\subsection{BKT Temperature}
In addition to the superfluid weight (which has the dimension of an energy in 2D), we consider the associated Berezinskii-Kosterlitz-Thouless (BKT) temperature, using the (isotropic) Nelson-Kosterlitz criterion \cite{liang_band_2017},
\begin{equation}
    T_{\text{BKT}}=\frac{\pi}{8k_B}D(T_{\text{BKT}}),
    \label{Neslon-Kosterlitz}
\end{equation}
where $D(T)$ is the superfluid weight at temperature $T$.
The BKT temperature $T_\text{BKT}$ is the temperature above which  vortex-antivortex pairs start to unbind and thus destroy superconductivity. It is generically smaller than the critical temperature % $T_\text{MF}$ %
calculated within a mean-field approach. For $T_{\text{BKT}}$ not too close to $T_c$, we may approximate $D(T_{\text{BKT}})$ by $D(T=0)$. This defines a ``mean-field'' BKT temperature which is larger than the actual one.
\begin{equation}
    T_{\text{BKT}}=\frac{\pi}{8k_B}D(T=0),
    \label{BKT T zéro}
\end{equation}
%omitting the superscript 0.


\subsection{Conventional contribution}
The conventional contribution to the superfluid weight at $T=0$ is given by \cite{liang_band_2017,chandrasekhar_superconducting_1993},
\begin{equation}
    D_{\mu\nu,\text{conv}}=\int_{\mathscr{S}_{\text{occ}}(\mu)}\frac{d^2\vec{k}}{(2\pi)^2}\frac{\Delta^2}{E(\vec{k})^3}(\partial_\mu\epsilon)(\partial_\nu\epsilon),  
\end{equation}
where $\mathscr{S}_{\text{occ}}$ denotes the set of occupied states in the BZ. As discussed above, our low-energy model results in three Schrödinger-like bands. We analytically compute the conventional contribution (appendix \ref{calcul Dconv}) and find that it is isotropic $D_{xy,\text{conv}}=0$, $D_{xx,\text{conv}}=D_{yy,\text{conv}}=D_{\text{conv}}$ with
\begin{equation}
    D_{\text{conv}}=\frac{1}{2\pi}\Big(\sqrt{\Delta^2+\mu^2}-\Delta\Big).
\end{equation}
Fig. \ref{Figure poids conv mu} shows a plot of  $D_{\text{conv}}$ versus the chemical potential $\mu$.
% Figure environment removed
% \newpage

 
  \subsection{Geometric contribution}
  The geometric contribution at zero temperature can be written as \cite{liang_band_2017},
 \begin{equation}
     D_{\mu\nu,\text{geom}}=\int_{\mathscr{S}_{\text{occ}}(\mu)}\frac{d^2\vec{k}}{(2\pi)^2}\frac{4\Delta^2}{E(\vec{k})}g_{\mu\nu}.
     \label{def poids geom}
 \end{equation}
Note the factor of two difference with the expression given in Ref. \cite{liang_band_2017}. This is because the definition of the metric there is twice the usual one \cite{berry_quantum_1989,graf_berry_2021}. Again, we see that because of the parity of $g_{xy}$ we have $D_{xy,\text{geom}}=0$. Also, $D_{xx,\text{geom}}=D_{yy,\text{geom}}=D_{\text{geom}}$. We then plot the latter as a function of the chemical potential $\mu$ in Fig.\ref{fig poids geom}.
% Figure environment removed

The variation of the geometric contribution with the chemical potential features a dome. This can be explained by inspection of Eq. (\ref{def poids geom}). Indeed, the $1/E(\vec{k})$ factor in the integral enhances the contribution at the Fermi contour, making it dominant. Focusing on this contribution, we can propose a scenario explaining the emergence of a dome in the geometric superfluid weight when the metric has a peak where the filling starts, as it is the case here. We sketch this scenario in Fig. \ref{scenario metrique dome}.

% Figure environment removed

 At low $\mu$, the band starts to be filled around $\Gamma$. The Fermi contour is thus at the top of the peak, but it is also narrow, such that $D_{\text{geom}}$ is low. However, as the filling increases, the Fermi contour gets wider while still being high and thus $D_{\text{geom}}$ becomes larger. This is the \textit{underdoped regime}, shown in Fig. \ref{scenario metrique dome}a.  The chemical potential $\mu$ then reaches a value where the trade-off between the height and the extent of the Fermi contour is optimal, and $D_{\text{geom}}$ reaches its maximal value. This is the \textit{optimal doping} in Fig. \ref{scenario metrique dome}b. Beyond the optimal doping, the Fermi contour still gets wider but not enough to compensate the smaller values of $g_{\mu\nu}$, resulting in a decrease of $D_{\text{geom}}$. This is the \textit{overdoped regime} in Fig. \ref{scenario metrique dome}c.
 
 \newpage

\section{Connection to experiments}
\label{section exp}
When confronting our theoretical scenario with experiment, we need to address three main issues. The first is the extent of the difference in value of quantities obtained using the low-energy model results as opposed to using the actual tight-binding model. The second is the link between the dome that we theoretically find when we change the chemical potential and the dome that has been experimentally observed upon variation of a gate voltage.  The third concerns the relation between the value of  $T_{\text{BKT}}$ obtained theoretically and the experimental value of the critical temperature (Sec. \ref{section exp}.A).
 \subsection{ Effects beyond the low-energy model}
 \label{Correction desordre, Tbkt total}
\subsubsection{Rashba SOC, comparing the low-energy model and the tight-binding model.}
Our low-energy model produces isotropic constant energy contours and it features neither an atomic spin-orbit term nor a confinement potential contribution. We show below that it nevertheless captures the main thermodynamic characteristics of the superconducting phase in the experimentally relevant regime of small $\mu$. 

We now include a spin index $\sigma =\uparrow,\downarrow$ and write down the $12 \times 12$ tight-binding Hamiltonian in the $\big(d^1_{yz\uparrow},d^1_{xz\uparrow},d^1_{xy\uparrow},d^2_{yz\uparrow},d^2_{xz\uparrow},d^2_{xy\uparrow},d^1_{yz\downarrow},d^1_{xz\downarrow},d^1_{xy\downarrow},d^2_{yz\downarrow},d^2_{xz\downarrow},\\ d^2_{xy\downarrow}\big)$ basis. The kinetic, trigonal crystal field, orbital mixing and confinement potential parts are diagonal in spin. The atomic spin-orbit Hamiltonian is diagonal in the layer index and in layer $1\; \text{or} \; 2$ it reads

\begin{equation}\label{SOC_ham}
\left(
\begin{smallmatrix} 
0 & i\lambda^{'} &0 & 0 & 0 & -\lambda^{'}\\
-i\lambda^{'} & 0 & 0 & 0 & 0 &i\lambda^{'}\\
0 & 0 & 0 & \lambda^{'} & -i\lambda^{'} & 0\\
0 & 0 &\lambda^{'} & 0 & -i\lambda^{'} & 0\\
0 & 0 & i\lambda^{'} & i\lambda^{'}& 0 & 0\\
-\lambda^{'} & -i\lambda^{'} & 0 & 0 & 0 & 0
\end{smallmatrix}
\right)
\end{equation}
where $\lambda^{'}=\Delta_{SO}/3$. The spin-orbit energy $\Delta_{SO}$ is on the order of 8 meV. A numerical solution of the Hamiltonian yields two groups of eigen-energies, one corresponding to bonding and the other to anti-bonding states. Their energy separation at $\Gamma$ is on the order of $3$ eV such that we only consider the lower energy bonding solutions, $E_{i,\pm}(\mathbf{k}),$ $i=1...3$. The sign $\pm$ labels the time-reversed Kramers pairs. For $\mathbf{k}=0$, $E_{+}=E_{-}$ but $E_{1}\neq E_{2}\neq E_{3}$.
%Their energy is degenerate for all $\mathbf{k}$ when $\Delta_{SO}$ is set to zero.
Close to  $\Gamma$, the restriction of the Hilbert space to subspaces $1,\; 2\; \text{or}\; 3$ leads to spin-Rashba Hamiltonians. We may conclude that $E_{i,+}(\mathbf{k}) -E_{i,-}(\mathbf{k}) $ is the Rashba spin-splitting energy and that $\frac{1}{2}[E_{i,+}(\mathbf{k}) +E_{i,-}(\mathbf{k})]$ is the "orbital" energy at zero spin-splitting. For the experimentally relevant regime, the chemical potential intersects the lowest energy bands $E_{1,\pm}(\mathbf{k})$. A numerical computation of the quantum geometry yields $g_{xx}, \; g_{yy}, \text{and} \; g_{xy}$
profiles similar to those displayed in Figs. \ref{gxxyy} and \ref{gxy} albeit with a larger value of the peak at $\Gamma$. Similarly \cite{Iskin2018}, the computed variations of the conventional and geometric contributions to the superfluid weight with respect to $\mu$ are similar to those shown in Figs. \ref{Figure poids conv mu} and \ref{fig poids geom}. Moreover, from Ref. \cite{liang_band_2017}, we find that the ratio between the spin-Rashba and the orbital contributions to the superfluid weight is less that $10 \%$ such that a $6\times 6$ orbital Hamiltonian adequately models the low $\mu$ experimental regime.

The $\mu$-dependence of physical quantities, such as the band filling and the conductivity, derived in the  tight-binding model agree fairly well with those obtained in the  low-energy model. However, the low-magnetic-field Hall resistance displays a non-monotonic behavior in the tight-binding model, caused by changes in convexity of the Fermi contour.
 
 We now discuss the tight-binding model of Eq. (\ref{six band matrix}) numerically. First, we study the closeness of the results derived from the low-energy model with respect to that of Eq. (\ref{six band matrix}). For the band dispersions, we have already seen in Fig. \ref{figure structure de bandes} that the two satisfactorily agree. As for the quantum metric, there is a remarkable agreement in the direct vicinity of the $\Gamma$ point. Away from the latter we observe four additional branches, but these do not have a significant effect on the corresponding geometric superfluid weight. The latter also shows a good agreement with Fig. \ref{fig poids geom}, with a dome-shaped isotropic geometric contribution, an optimal doping located at $\mu\sim0.1$ meV and at a value close to the one in Fig. \ref{fig poids geom}. The low-energy model thus yields results that compare well to those obtained numerically in the model in Eq. (\ref{six band matrix}). 

 
 \subsubsection{Correction to $D_{\text{conv}}$ from disorder}
    The conventional superfluid weight $D_{\text{conv}}$ is proportional to the superfluid carrier density $n_s$. From Fig. \ref{Figure poids conv mu}, we see that $D_{\text{conv}}\sim0.1$meV in the plotted range of chemical potential. Using the Nelson-Kosterlitz criterion, the associated BKT temperature is $T_{\text{BKT}}^{\text{conv}}\sim500$mK, clearly larger than reported critical temperature $T_c\sim100-200$mK. This discrepancy can be explained by disorder. Indeed, we did not take disorder into account therefore our conventional superfluid weight corresponds to a superfluid density $n_s$ without disorder, and therefore of the order of the carrier density. It was estimated in Ref. \cite{bert_gate-tuned_2012} that because of disorder, the superfluid density only amounts to one to ten percent of the total carrier density which is of the order $10^{13-14}\,\text{cm}^{-2}$. Therefore, one way to take disorder into account would be applying such a renormalisation to the superfluid density and thus the conventional superfluid weight. As said earlier, this factor is between ten and a hundred. We choose to apply a factor of a hundred, in order to make the conventional and geometric contributions of the same order. 
 
    \subsubsection{Total superfluid weight}
Having introduced the two distinct contributions to the superfluid weight, we now discuss the total superfluid weight. We plot the corresponding BKT temperatures of Eq. (\ref{BKT T zéro}) as a function of the chemical potential in Fig. \ref{Tbkt}.

% Figure environment removed


We indeed see a dome as a function of the chemical potential. At low $\mu$, the geometric contribution dominates and beyond a value $\sim 0.5$ meV the conventional contribution is largest.
These are the \textit{theoretical} results coming from the low-energy model given in Eq. (\ref{H3quad}). The connection to the experimental results is however more subtle.

 \subsection{Superconducting domes}
 
As described above, we find a superconducting dome when we vary the chemical potential. In contrast, the experimentally observed superconducting dome \cite{rout_link_2017,rout_six-fold_2017,bert_gate-tuned_2012,gariglio_interface_2015,gariglio_research_2016} is measured upon tuning a gate voltage $V_g$ or a conductivity. It has been argued \cite{maniv_correlation-induced_2017,khanna_symmetry_2019} that the correspondence between these transport quantities and the (Hall) carrier density (or the chemical potential) is non-monotonic, possibly due to correlation effects or to the curvature of the Fermi surface. It may also be the case because of leakage of the surface's electrons in the substrate beyond a certain gate voltage. More precisely, the Hall carrier density itself displays a dome upon changing the gate voltage indicating a non-monotonic relationship between density and gate voltage. Therefore, there is no direct correspondence between the SC domes that result from changing $\mu$ as opposed to changing $V_g$. Based on the dependence of the Hall number on $V_g$, and that of $\mu$ on the carrier density, we propose that the variation of $\mu$ with $V_g$ is as depicted in the inset of Fig. \ref{Figure dôme Vg}, resulting in a gate voltage dependence of the critical temperature shown in Fig. \ref{Proposition dômes Vg}.
 % Figure environment removed
 The scenario depicted in Fig. \ref{Figure dôme Vg} goes as follows. The initial value of the chemical potential is at a point beyond 0.5 meV where the conventional contribution dominates, indicated by the point (1). At first, increasing the gate voltage also increases the chemical potential so that the BKT temperature also increases. This is the \textit{underdoped regime} from point (1) to point (2). It is followed by the \textit{optimal doping} region at the point (2), starting around the top of the dome. Further increase of the gate voltage leads to a decrease of the chemical potential and therefore to lower values of the BKT temperature, in the \textit{overdoped regime}, from point (2) to point (3). We can draw further consequences from this scenario. The experimentally observed dome happens in a regime where the conventional contribution dominates and the geometric contribution should be sizeable in the overdoped regime. But if we go one step beyond and assume that further increasing the gate voltage results in an even lower value of the chemical potential, we could reach the low-$\mu$ regime and reveal the dome due to the geometric contribution. In other words, while the measured dome would be a consequence of the conventional contribution and the non-monotonicity of the chemical potential with respect to the gate voltage, there should be a \textit{secondary superconducting dome},  coming from the geometric contribution, for higher values of the gate voltage. The evolution of the critical temperature (or superfluid density, BKT temperature) would be similar to that sketched in Fig. \ref{Proposition dômes Vg}, so long as only the lowest energy band contributes to the superfluid condensate. 
 
 In our picture, the two expected superconducting domes when measured as a function of the gate voltage have thus two different origins. The first one is essentially (up to the optimal point) due to the non-monotonic behavior of the chemical potential with respect to the gate voltage while $T_{BKT}$ varies monotonically with $\mu$ in this interval. In contrast, the second one would be due to the ``geometric'' superconducting dome that is revealed when $T_{BKT}$ is plotted as a function of the chemical potential.

 
 % Figure environment removed
 \newpage
 
 \subsection{BKT and critical temperatures}
The last question pertaining to the connection between the theoretical results presented here and the experimental results is the relation between the \textit{magnitude} of the BKT temperature as found in Fig. \ref{Tbkt} and that of the experimentally measured critical temperature $T_c\sim100-200$ mK. Indeed, Fig. \ref{Tbkt} indicates that the total BKT temperature barely reaches 10 mK, which is one order of magnitude lower than the reported value of the critical temperature. We may suggest possible explanations. First, we included the effect of disorder on the conventional contribution by renormalizing it, as suggested in Ref. \cite{bert_gate-tuned_2012} for example. The renormalization factor is said to be between ten and a hundred. While we chose a hundred, the actual factor may be closer to 10, thereby yielding a BKT temperature of the same order as the measured critical temperature. Further experimental studies, particularly on the variation of $V_g$ on $\mu$ and that of $T_c$ at high-$V_g$ where one could see the relative height of the two domes, might help clarifying the matter. Beyond that, the effect of disorder on the geometric contribution may be relevant. Further studies focusing on this relation may help enlighten the origins of this discrepancy.

\section{Conclusion}

Our study underscores the impact of the normal state quantum geometry on the superconducting state of the (111) $\text{LaAlO}_3/\text{SrTiO}_3$ interface. Starting from a tight-binding model, we first developed a low-energy three-band model to describe the electronic structure around the $\Gamma$ point. There, we found three Schrödinger-like bands, with one lower band being significantly flatter than the other two, which are degenerate at the $\Gamma$ point. Using a method developed in Ref. \cite{graf_berry_2021}, we computed the quantum geometry associated to this lower band. We found that its Berry curvature is negligible. By contrast, its quantum metric presents a strong peak at the $\Gamma$ point, owing to the closeness to a degeneracy point (coming from a low value of the trigonal crystal field). Then, using a theory developed in Ref. \cite{liang_band_2017}, we computed the superfluid weight of this band as a function of the chemical potential $\mu$, expecting a strong geometric contribution because of the strong quantum metric.
We found that this geometric contribution has a dome-shaped behavior as a function of $\mu$, and put forward a scenario explaining that the geometric contribution generically presents this dome behavior when the metric has a peak at the zero-filling point. For the conventional contribution, we analytically showed that, for a Schrödinger-like band, it has a linear behavior with respect to the chemical potential. In the last section, we discussed subtleties regarding the relation with experimental results.  We first took into account the effect of disorder by renormalizing the conventional contribution. The resulting total BKT temperature then has the form explicited in Fig. \ref{Tbkt}. The geometric contribution should dominate in a low-chemical-potential regime ($\leq0.5$ meV). Beyond that, the conventional contribution dominates. We then discussed the relation between the dome seen as a function of the chemical potential and the ones observed experimentally as function of a gate voltage or a conductivity. Using the reported dependence of the Hall carrier density as a function of the gate voltage and our theoretical results, we put forward a scenario explaining the emergence of the observed dome. The observed dome would be a consequence of the non-monotonic dependence of the chemical potential on the gate voltage and would rely mostly on the conventional contribution, the geometric one being sizeable in the overdoped regime. Extrapolating this scenario, we suggest the prediction of a second superconducting dome at a higher range of gate voltage, this time ruled by the geometric contribution.  Finally, we discussed the discrepancy between the predicted value of the BKT temperature and the measured superconducting critical temperature. Given the ubiquituousness of quantum geometry, this \textit{hidden influence} on the superconducting state might be apparent in other classes of materials. Finally, this positive effect of the normal-state quantum metric on superconductivity needs to be contrasted to a previous theoretical discussion \cite{simon_role_2022} suggesting a negative impact of the normal-state Berry curvature on superconductivity. This would suggest a \textit{normal state curvature-metric competition} towards superconductivity.
\newline
\section*{Acknowledgements}
  We wish to acknowledge Frédéric Piéchon for his insightful input on our work and careful reading of our manuscript. We thank Andrea Caviglia, and Roberta Citro for valuable discussions.










%\section*{Documents}
%The Python codes and Mathematica notebooks used to obtain the results %in this paper are available upon reasonable requests.

 \bibliographystyle{apsrev4-2}
\bibliography{Bibliolaosto}

\begin{appendix}
\section{Expression of the kinetic terms}
\label{expressions cinetiques}
Hoppings are between two neighboring layers, with amplitude $t$ for nearest neighbors and $t_d$ between second neighbors blue and red sites (Fig. 1). The origin of the basis lattice vectors is chosen at the center of an hexagon. $e$, $f$ and $g$ have the following expressions
\begin{widetext}
\begin{equation}
    \begin{cases}
        e=-\bigg\{\exp(ik_y)+\exp\bigg[i\Big(\frac{\sqrt{3}}{2}k_x-\frac{1}{2}k_y\Big)\bigg]\bigg\}-r\exp\bigg[-i\Big(\frac{\sqrt{3}}{2}k_x+\frac{1}{2}k_y\Big)\bigg]\\
        f=-\bigg\{\exp(ik_y)+\exp\bigg[-i\Big(\frac{\sqrt{3}}{2}k_x+\frac{1}{2}k_y\Big)\bigg]\bigg\}-r\exp\bigg[i\Big(\frac{\sqrt{3}}{2}k_x-\frac{1}{2}k_y\Big)\bigg]\\
        g=-\exp\Big(-\frac{i}{2}k_y\Big)\times2\cos\Big(\frac{\sqrt{3}}{2}k_x\Big)-r\exp(ik_y)
    \end{cases}
    \hspace{2mm}\text{with}\hspace{2mm}
    r=\frac{t_d}{t}.
\end{equation}
\end{widetext}

\section{Derivation of the orbital mixing term}
\label{annexe OM}
In the orbital basis, $H_{\text{om}}$ can be written as $A\otimes B$ with $A$ and $B$ acting in the layer and orbital subspaces respectively. The orbital mixing term consists of the interlayer and interorbital couplings so the diagonal elements of $A$ and $B$ must vanish. The hexagonal lattice structure seen in Fig. \ref{Schema_LAOSTO} has the $\mathcal{C}_{3v}$ symmetry. In order to respect the latter, we assume that all the couplings have the same magnitude in energy and that those between layer 1 to layer 2 are the same as those from layer 2 to layer 1. Therefore we can write 
\begin{equation}
    A\otimes B=c_0\begin{pmatrix}
        0&a\\
        a&0
    \end{pmatrix}
    \otimes
    \begin{pmatrix}
        0&b_1&b_2\\
        b_3&0&b_4\\
        b_5&b_6&0
    \end{pmatrix},
\end{equation}
where the coefficients are complex, and of modulus 1 in order to have the same magnitude $c_0$ in energy. They are further constrained by the fact that the term must be Hermitian. Using $(A\otimes B)^\dagger=A^\dagger\otimes B^\dagger$, this means that we must have $a^*=a$, $b_3=b_1^*$, $b_5=b_2^*$ and $b_6=b_3^*$. We choose $a=1$, such that
\begin{equation}
    A\otimes B=c_0\tau_x\otimes\begin{pmatrix}
        0&b_1&b_2\\
        b_1^*&0&b_3\\
        b_2^*&b_3^*&0
    \end{pmatrix}.
\end{equation}
We then introduce $(\phi_i,\psi_i)$ such that $b_i(\vec{k})=\operatorname{cos}(\phi_i(\vec{k}))+i\operatorname{sin}(\psi_i(\vec{k}))$. We look for $\phi_i$ and $\psi_i$ that are linear combinations of $k_x$ and $k_y$, which is natural for tight-binding models. These hoppings are also antisymmetric under an inversion operation $\vec{r}\longmapsto-\vec{r}$ \cite{khalsa_theory_2013}, which adds the constraint $b_i(-\vec{k})=-b_i(\vec{k})$ since in $\operatorname{exp}(i\vec{k}\cdot\vec{r})$ doing $\vec{r}\longmapsto-\vec{r}$ is equivalent to $\vec{k}\longmapsto-\vec{k}$. Hence, $\operatorname{cos}(\phi_i(\vec{k}))=0$. Writing the allowed hoppings between the red and blue sites explicitly (Fig. \ref{Schema_LAOSTO}) with the above requirements then gives $b_i(\vec{k})=i\operatorname{sin}(\psi_i(\vec{k}))=i\operatorname{sin}(\alpha_ik_x+\beta_ik_y)$, with $\psi_i(\vec{k})=\alpha_ik_x+\beta_ik_y$. Next, the orbital mixing term needs to obey the $\mathcal{C}_{3v}$ symmetry, i.e. a $2\pi/3$ rotation with an axis perpendicular to the $(111)$ plane and a mirror symmetry parallel to the $(\overline{11}2)$ orientation. The $2\pi/3$ rotation transforms $\vec{r}=(x,y,z)$ into $\vec{r}'=(z,x,y)$ in the original cubic unit cell. Therefore the orbitals are transformed as $(d_{yz},d_{xz},d_{xy})\longmapsto(d_{xy},d_{yz},d_{xz})$. In order to obey this $\mathcal{C}_3$ symmetry, we must therefore have $b_1(\vec{k}')=b_3(\vec{k})$, $b_2(\vec{k}')=b_1^*(\vec{k})$ and $b_3(\vec{k}')=b_2^*(\vec{k})$ with $\vec{k}'=(-1/2k_x-\sqrt{3}/2k_y,\sqrt{3}/2k_x-1/2k_y)$. The mirror operation maps $\vec{r}=(x,y,z)$ to $\vec{r}'=(y,x,z)$ and $\vec{k}=(k_x,k_y)$ to $\vec{k}'=(-k_x,k_y)$, so that the orbitals transform as $(d_{yz},d_{xz},d_{xy})\longmapsto(d_{xz},d_{yz},d_{xy})$. In order to obey this symmetry, we must have $b_1(\vec{k}')=b_1^*(\vec{k})$, $b_2(\vec{k'})=b_3(\vec{k})$ and $b_3(\vec{k}')=b_2(\vec{k})$. These constraints on the the $b_i$s put constraints on the coefficients $(\alpha_i,\beta_i)$ by taking the low-$k$ limit and identifying the $k_x$ and $k_y$ components (which is allowed since the constraints must be valid for all $\vec{k}$). The resulting system of equations puts five independent constraints such that $\beta_1=0$, $(\alpha_2,\beta_2)=(1/2\alpha_1,\sqrt{3}/2\alpha_1)$ and $(\alpha_3,\beta_3)=(-1/2\alpha_1,\sqrt{3}/2\alpha_1)$. We then recover Eq. (\ref{orbital mixing matrix}) with $\alpha_1=-\sqrt{3}$. 

\section{Trigonal basis}
\label{trigonal basis}
Let $U$ be the following unitary transformation in the orbital basis
\begin{equation}
        U=\tau_0\otimes P,\quad P=\begin{pmatrix}
        -\frac{1}{\sqrt{2}}&-\frac{1}{\sqrt{6}}&\frac{1}{\sqrt{3
        }}\\
        \frac{1}{\sqrt{2}}&-\frac{1}{\sqrt{6}}&\frac{1}{\sqrt{3}}\\
        0&\frac{2}{\sqrt{6}}&\frac{1}{\sqrt{3}}
    \end{pmatrix}.
\end{equation}
In this basis, the trigonal crystal field becomes diagonal,
\begin{equation}
    P^\dagger dH_d P=\begin{pmatrix}
        d&0&0\\
        0&d&0\\
        0&0&-2d
    \end{pmatrix}.
\end{equation}
The first two-fold degenerate eigenvalue represents the $e_{+g}$ and $e_{-g}$ orbitals while the third one represents the $a_{1g}$ orbital. They are separated by an energy gap of $3d\simeq 10$ meV. The kinetic term is transformed as
\begin{equation}
    H_{\text{cin}}=t\begin{pmatrix}
    \frac{e+f}{2}&\frac{e-f}{2\sqrt{3}}&-\frac{e-f}{\sqrt{6}}\\
        \frac{e-f}{2\sqrt{3}}&\frac{e+f+4g}{6}&-\frac{e+f-2g}{3\sqrt{2}}\\
        -\frac{e-f}{\sqrt{6}}&-\frac{e+f-2g}{3\sqrt{2}}&\frac{e+f+g}{3}
    \end{pmatrix},
\end{equation}
while the orbital mixing term becomes
\begin{equation}
    H_{\text{om}}=c_0\begin{pmatrix}
        0&iD&-iA\\
        -iD&0&iB\\
        iA&-iB&0
    \end{pmatrix},
\end{equation}
with
\begin{equation}
    \begin{cases}
        A=-\frac{1}{\sqrt{6}}\big(\alpha+\beta-2\delta\big)\\
        B=\frac{1}{\sqrt{2}}\big(\alpha-\beta\big)\\
        D=\frac{1}{\sqrt{3}}\big(\alpha+\beta+\delta\big).
    \end{cases}
\end{equation}

\section{Validity of the quadratic three-band approximation}
\label{validity annexe}
We now discuss the validity of the three-band approximation. Doing so amounts to neglecting the off-diagonal blocks in Eq. (\ref{H transforme}) which contain the confinement energy and the imaginary part of the kinetic term. From Ref. \cite{nakatsukasa_off-diagonal_2017}, the effect of such off-diagonal terms is in $\mathcal{O}\Big(\frac{||E||^2}{\text{gap}}\Big)$, where $E$ is the off-diagonal perturbation. The numerically observed gap (with our choice of parameters) is around $6-7$ eV. The biggest contribution of the two terms is at zeroth and first order in $k$, i.e. in terms of scalar quantities, we have $E\sim V\pm itk$ to linear order. Therefore the intrinsic error of the three-band approximation is roughly given by 
\begin{equation}
    \frac{V^2+t^2k^2}{6.5\text{eV}}\simeq (1.5+400k^2)\text{meV}.
\end{equation}
This means that if we want a precision on the order of 1 meV, we find that the approximation holds until $k\sim0.1$, so about a tenth of the Brillouin zone (BZ). This is indeed what we find when we compare the band structure with and without the off-diagonal blocks. More precisely, the confinement energy globally shifts every band by 1 to 2 meV while the imaginary part of the kinetic term breaks the isotropy of the band structure obtained within the low-energy model and gives rise to the $\mathcal{C}_3$ symmetric structure of ellipses seen in experimental studies (see Ref. \cite{rodel_orientational_2014} for example). So the validity of the low-energy model is restrained to the first tenth of the BZ around the $\Gamma$ point. Knowing this, what is the natural order of expansion we can do to the low-energy model ? The relevant terms will be the ones above or around our precision of a few meVs. For the orbital mixing term, the first two corrections are linear and cubic in $k$. We then have $c_0k\sim4$ meV and $c_0k^3\sim0.04$ meV for $k\sim0.1$, so we only take the linear term. As for the kinetic term, the first two corrections are of the order $tk^2$ and $tk^4$. This gives $tk^2\sim10$ meV and $tk^4\sim0.1$ meV, we therefore only keep the quadratic term. So in conclusion, we can expand our three-band model to quadratic order while being coherent with the three-band approximation.

\section{Quadratic expansion of $H_3$}
\label{annexe DL quad H3}
Here, we derive the quadratic expansion of $H_3$ in Eq. (\ref{H3quad}). We remind the reader that the matrices are written in the trigonal basis. For the orbital mixing term, we have
\begin{equation}
    H_{\text{om}}=\begin{pmatrix}
        0&0&ick_x\\
        0&0&ick_y\\
        -ick_x&-ick_y&0
    \end{pmatrix}+\mathcal{O}(k^3),
\end{equation}
with $c=(3/\sqrt{2})c_0$. For the kinetic term, we have
\begin{widetext}
    \begin{equation}
    \text{Re}(H_{\text{cin}})=-t(2+r)\bigg(1-\frac{1}{4}k^2\bigg)I_3+t\begin{pmatrix}
        -\frac{1}{8}(1-r)(k_x^2-k_y^2)&-\frac{1}{4}(1-r)k_xk_y&\frac{1}{2\sqrt{2}}(1-r)k_xk_y\\
        -\frac{1}{4}(1-r)k_xk_y&\frac{1}{8}(1-r)(k_x^2-k_y^2)&\frac{1}{4\sqrt{2}}(1-r)(k_x^2-k_y^2)\\
        \frac{1}{2\sqrt{2}}(1-r)k_xk_y&\frac{1}{4\sqrt{2}}(1-r)(k_x^2-k_y^2)&0
        \end{pmatrix}+\mathcal{O}(k^4),
\end{equation}
\end{widetext}
where we separated the traceful and traceless parts using 
\begin{equation}
    \begin{pmatrix}
        a&0&0\\
        0&b&0\\
        0&0&c
    \end{pmatrix}=\frac{a+b+c}{3}I_3+\frac{a-b}{2}\lambda_3+\frac{a+b-2c}{2\sqrt{3}}\lambda_8,
\end{equation}
in terms of the Gell-Mann matrices. Now, if we define $t_{\text{eff}}=(t-t_d)/8$, we indeed find Eq. (\ref{H3quad}), neglecting the quadratic terms where there already is a linear term.
\section{Hamiltonian vector}
\label{annexe hamiltonian vector}
The Hamiltonian vector of $H_3$ at quadratic order is given by 
    \begin{equation}
        \vec{h}=\left(-2t_{\text{eff}}k_xk_y,-t_{\text{eff}}\big(k_x^2-k_y^2\big),0,-ck_x,0,-ck_y,\sqrt{3}d\right).
    \end{equation}
We then have $H_3=\vec{h}\cdot\vec{\lambda}$ in terms of the vector  $\vec{\lambda}$ regrouping the eight Gell-Mann matrices, where we have omitted the traceful term which is irrelevant in the calculation of the quantum geometry \cite{graf_berry_2021}. As said in the corresponding subsection, we find an identically zero Berry curvature. To have a non-zero Berry curvature one way is to add a cubic term coming from the orbital mixing term, which breaks the isotropy of the problem and rather has a $\mathcal{C}_3$ symmetry. The Hamiltonian vector $\vec{h}$ then becomes
\begin{widetext}
\begin{equation}
    \vec{h}=\left(-2t_{\text{eff}}k_xk_y,-\frac{1}{4\sqrt{2}}ck_x\big(k_x^2-3k_y^2\big),-t_{\text{eff}}\big(k_x^2-k_y^2\big),0,-ck_x,0,-ck_y,\sqrt{3}d\right).
\end{equation}
\end{widetext}
\section{Calculation of $D_{\text{conv}}$}
\label{calcul Dconv}
 Let us consider a general isotropic and quadratic band $\epsilon=\epsilon_0+\alpha k^2$. Then, we can readily show that $D^{\text{conv}}_{xx}=D^{\text{conv}}_{yy}=D_{\text{conv}}$ and $D^{\text{conv}}_{xy}=0$. We also see that $\mathscr{S}_{\text{occ}}(\mu)=B\big(0;\sqrt{\mu/\alpha}\big)$. We then have
 \begin{align}
    D_{\text{conv}}&=\frac{1}{2}\int_{\mathscr{S}_{\text{occ}}(\mu)}\frac{\Delta^2}{\big[\Delta^2+(\alpha k^2-\mu)^2\big]^{3/2}}4\alpha^2k^2\frac{d^2\vec{k}}{(2\pi)^2}\nonumber\\
    &=\frac{2\alpha^2}{(2\pi)^2}\int_0^{2\pi}\int_0^{\sqrt{\mu/\alpha}}\frac{\Delta^2}{\big[\Delta^2+(\alpha k^2-\mu)^2\big]^{3/2}}k^3dkd\theta\nonumber\\
    &=\frac{2\alpha^2}{2\pi}\int_0^{\sqrt{\mu/\alpha}}\frac{\Delta^2}{\big[\Delta^2+(\alpha k^2-\mu)^2\big]^{3/2}}k^3dk\nonumber\\
    &=\frac{\alpha^2}{\pi}\frac{1}{2\alpha^2}\int_0^\mu\frac{\Delta^2}{\big[\Delta^2+(\epsilon-\mu)^2\big]^{3/2}}\epsilon d\epsilon\nonumber\\
    &=\frac{1}{2\pi}\Big(\sqrt{\Delta^2+\mu^2}-\Delta\Big).
\end{align}
\end{appendix}
\end{document}