\documentclass{article}

\usepackage{arxiv}
\usepackage{amsmath,amssymb,amsfonts}
\usepackage{algorithmic}
\usepackage{array}
\usepackage[caption=false,font=normalsize,labelfont=sf,textfont=sf]{subfig}
\usepackage{textcomp}
\usepackage{stfloats}
\usepackage{url}
\usepackage{cite}
\usepackage{booktabs}
\usepackage{etoolbox}
% 定义一个新的命令 \smalltabular,用于将表格内的字体大小设置为 \small
\newcommand{\smalltabular}{\small\tabular}

\usepackage{colortbl}
\usepackage{verbatim}
\usepackage{graphicx}
\usepackage{adjustbox}
% \hyphenation{op-tical net-works semi-conduc-tor IEEE-Xplore}
% \def\BibTeX{{\rm B\kern-.05em{\sc i\kern-.025em b}\kern-.08em
%     T\kern-.1667em\lower.7ex\hbox{E}\kern-.125emX}}
\usepackage{balance}

\usepackage[numbers]{natbib}
\usepackage[framemethod=tikz]{mdframed}
\usepackage[ruled, linesnumbered, lined, boxed, commentsnumbered]{algorithm2e} % 
\usepackage{color}

\usepackage{multirow}

% \usepackage[hyphens]{url}
\usepackage[switch]{lineno}
\usepackage{threeparttable}
\usepackage[normalem]{ulem} % cross out text
\usepackage{tabularx} % better tables
\usepackage{booktabs} % table support
% \usepackage[toc, page]{appendix} % appendix support
% \usepackage{subfigure}
% \usepackage[dvipsnames]{xcolor}
\usepackage{makecell}
\usepackage{lipsum} 
% Hao xiang
\usepackage{float} % float figure temporarily
\usepackage{enumitem} % itemize no indent
\usepackage{tcolorbox} % beautiful box
\newcommand{\rqbox}[1]{\begin{tcolorbox}[left=4pt,right=4pt,top=4pt,bottom=4pt,colback=gray!5,colframe=gray!40!black,before skip=6pt,after skip=6pt]#1\end{tcolorbox}}
\usepackage{textcomp} % arrow
\renewcommand{\textleftarrow}{$\leftarrow$}
\renewcommand{\textrightarrow}{$\rightarrow$}
\newtheorem{definition}{Definition}

% Wenhan (Cosmos)
\usepackage{hyperref}
\usepackage{graphics}
\usepackage[caption=false]{subfig}
% 限制图表在正文中
\usepackage[section]{placeins}
% \usepackage[utf8]{inputenc} % allow utf-8 input
% \usepackage[T1]{fontenc}    % use 8-bit T1 fonts
% \usepackage{hyperref}       % hyperlinks
% \usepackage{url}            % simple URL typesetting
% \usepackage{booktabs}       % professional-quality tables
% \usepackage{amsfonts}       % blackboard math symbols
% \usepackage{nicefrac}       % compact symbols for 1/2, etc.
% \usepackage{microtype}      % microtypography
% % \usepackage{cleveref}       % smart cross-referencing
% \usepackage{lipsum}         % Can be removed after putting your text content
% \usepackage{graphicx}
% \usepackage{natbib}
% \usepackage{doi}
% \usepackage{amsmath}
% \usepackage{cleveref} 
% \usepackage[ruled, linesnumbered, lined, boxed, commentsnumbered]{algorithm2e} % 

\title{Few-shot Image Classification based on Gradual Machine Learning}

% Here you can change the date presented in the paper title
%\date{September 9, 1985}
% Or remove it
%\date{}

\author{{Na Chen\textsuperscript{1}}, {Xianming Kuang\textsuperscript{2}}, {Feiyu Liu\textsuperscript{1}}, {Kehao Wang\textsuperscript{2}}, {Qun Chen\textsuperscript{1,2}}\thanks{Joint corresponding author:chenbenben@nwpu.edu.cn}\\ \\
\textsuperscript{1}School of software,Northwestern Polytechnical University,Xi'an, China, 710072 \\
\textsuperscript{2}School of Computer Science,Northwestern Polytechnical University,Xi'an, China, 710072 \\
}
		% School of software\\
		% Northwestern Polytechnical University\\
		% Xi'an, China, 710072 \\
	% \texttt{chenna05@mail.nwpu.edu.cn} \\
	%% examples of more authors
	% \And
	% School of Computer Science\\
	% Northwestern Polytechnical University\\
	% Xi'an, China, 710072\\
	% \texttt{kkjdc@mail.nwpu.edu.cn} \\
	% \And
	% \\
	% School of software\\
	% Northwestern Polytechnical University\\
	% Xi'an, China, 710072 \\
	% \texttt{lfy01@mail.nwpu.edu.cn} \\
	% \And
	%  \\
	% School of Computer Science\\
	% Northwestern Polytechnical University\\
	% Xi'an, China, 710072 \\
	% \texttt{iwangkehao@mail.nwpu.edu.cn} \\
	% \And
	%  \\
	% School of Computer Science\\
	% Northwestern Polytechnical University\\
	% Xi'an, China, 710072 \\
	% \texttt{chenbenben@nwpu.edu.cn} \\
	%% \AND
	%% Coauthor \\
	%% Affiliation \\
	%% Address \\
	%% \texttt{email} \\
	%% \And
	%% Coauthor \\
	%% Affiliation \\
	%% Address \\
	%% \texttt{email} \\
	%% \And
	%% Coauthor \\
	%% Affiliation \\
	%% Address \\
	%% \texttt{email} \\
	


% Uncomment to override  the `A preprint' in the header
\renewcommand{\headeright}{Technical Report}
\renewcommand{\undertitle}{Technical Report}
\renewcommand{\shorttitle}{Few-shot Image Classification based on Gradual Machine Learning}

%%% Add PDF metadata to help others organize their library
%%% Once the PDF is generated, you can check the metadata with
%%% $ pdfinfo template.pdf
\hypersetup{
pdftitle={Few-shot Image Classification based on Gradual Machine Learning},
pdfsubject={q-bio.NC, q-bio.QM},
pdfauthor={Na Chen, Xianming Kuang, Feiyu Liu, Kehao Wang and Qun Chen},
pdfkeywords={First keyword, Second keyword, More},
}

\begin{document}
\maketitle

\begin{abstract}
	% \lipsum[1]
	Few-shot image classification aims to accurately classify unlabeled images using only a few labeled samples. The state-of-the-art solutions are built by deep learning, which focuses on designing increasingly complex deep backbones. Unfortunately, the task remains very challenging due to the difficulty of transferring the knowledge learned in training classes to new ones. In this paper, we propose a novel approach based on the non-i.i.d paradigm of gradual machine learning (GML). It begins with only a few labeled observations, and then gradually labels target images in the increasing order of hardness by iterative factor inference in a factor graph. Specifically, our proposed solution extracts indicative feature representations by deep backbones, and then constructs both unary and binary factors based on the extracted features to facilitate gradual learning. The unary factors are constructed based on class center distance in an embedding space, while the binary factors are constructed based on k-nearest neighborhood. We have empirically validated the performance of the proposed approach on benchmark datasets by a comparative study. Our extensive experiments demonstrate that the proposed approach can improve the SOTA performance by 1-5\% in terms of accuracy. More notably, it is more robust than the existing deep models in that its performance can consistently improve as the size of query set increases while the performance of deep models remains essentially flat or even becomes worse.
\end{abstract}


% keywords can be removed
\keywords{Few-shot Image Classification \and  gradual machine learning \and Factor Graph}

\maketitle
% \AtBeginEnvironment{tabular}{\smalltabular}
% ######################################################
% Cycling
% ######################################################
To promote sustainability, cities worldwide are promoting a transition to public transportation and active transportation. From these, cycling has proven to provide numerous advantages, including benefits to health \cite{gotschi2016cycling}, economy \cite{clifton2013examining}, and reduction of carbon emissions \cite{NEVES2019130}. Despite these benefits, cycling numbers remain predominantly low in some cities. In contrast, barriers to cycling include hilliness, lack of cycling infrastructure, or appropriate bike storage or parking. Yet, the main deterrent to cycling relates to safety concerns \cite{aldred2015investigating, lawson2013perception, felix2019maturing}. If cyclists feel unsafe or are afraid to cycle, they will prefer other means of transportation. 


% ######################################################
% Perception of Safety
% ######################################################
Thus, for cities aiming to boost cycling numbers and the effectiveness of such strategies, it is increasingly important to understand what affects individuals' perceptions. Perception of cycling safety research explores how individuals subjectively experience cycling accident risk and what fears and events negatively impact one's perception of being involved in a cycling accident. Current research shows that infrastructure layout, fear of traffic, and distracted cycling are some aspects that influence this perception \cite{heinen2010commuting}. Most research focuses on surveys and in-loco and post-riding interviews to compare factors influencing perceptions \cite{sanders2015perceived}. Even though these approaches are vital to understanding cycling perception of safety, they need to be more scalable over space or time due to their high cost (human resources, time, and money). This prevents any analysis of perceptions over time, and qualitative non-scalable data analysis hampers any comparative study across cities or countries.


% !BIB TS-program =
\documentclass[12pt]{article}
\usepackage{color}
\usepackage{amsfonts,amssymb,amsmath}
\usepackage[export]{adjustbox}
\makeatletter
\setlength{\@fptop}{0pt}
 \makeatother
\usepackage{graphicx}
\usepackage[T1]{fontenc}
\usepackage[numbers,sort&compress]{natbib}
\graphicspath{ {./images/} } \textheight 9in \textwidth  6.5in
\topmargin -1cm \oddsidemargin -0.1in \evensidemargin -0.1in
\marginparwidth 17.57mm
%\renewcommand{\baselinestretch}{1.55}
\newcounter{tempeq}
\begin{document}
\title{\textsf{Enhancing the performance of an open quantum battery by adjusting its velocity}}
\author{B. Mojaveri\thanks{Email: bmojaveri@azaruniv.ac.ir; bmojaveri@gmail.com},
\hspace{2mm}R. Jafarzadeh Bahrbeig\thanks{Email:
r.jafarzadeh86@gmail.com},\hspace{2mm}M. A. Fasihi
\thanks{Email: ma-fasihi@azarunic.ac.ir}, and S. Babanzadeh\thanks{Email: s.babanzadeh@azaruniv.ac.ir}\\
{\small {Department of Physics, Azarbaijan Shahid Madani University,
PO Box 51745-406, Tabriz, Iran \,}}} \maketitle
\begin{abstract}
The performance of open quantum batteries (QBs) is severely limited
by decoherence due to the interaction with the surrounding
environment. So, protecting the charging processes against
decoherence is of great importance for realizing QBs. In this work
we address this issue by developing a charging process of a
qubit-based open QB composed of a qubit-battery and a qubit-charger,
where each qubit moves inside an independent cavity reservoir. Our
results show that, in both the Markovian and non-Markovian dynamics,
the charging characteristics, including the charging energy,
efficiency and ergotropy, regularly increase with increasing the
speed of charger and battery qubits. Interestingly, when the charger
and battery move with higher velocities, the initial energy of the
charger is completely transferred to the battery in the Markovian
dynamics. In this situation, it is possible to extract the total
stored energy as work for a long time. Our findings show that open
moving-qubit systems are robust and reliable QBs, thus making them a
promising candidate for experimental implementations.\\\\
{\bf Keywords:} Open quantum batter, Markovian and non-Markovian
charging process, Ergotropy, Atomic motion.
\end{abstract}
\section{introduction}
In recent years, with advancements in quantum thermodynamics, there
has been a radical change of perspective in the framework of energy
manipulation based on the electrochemical principles. The
possibility to create an alternative and efficient energy storage
device at small scale introduces the concept of the quantum battery
(QB), which was proposed by Alicki and Fennes in the 2013's
\cite{Alicki}, and  subsequently became into a significant field of
research. As their name indicates, QBs are finite dimensional
quantum systems that are able to temporarily store energy in their
quantum degrees of freedom for later use. The fundamental strategy
for developing the idea of QBs is based on their non-classical
features such as quantum coherence, entanglement and many-body
collective behaviors that can be cleverly exploited to achieve more
efficient and faster charging processes than the macroscopic
counterpart \cite{Strasberg, Vinjanampathy, Goold, Campisi,
Gelbwaser, Horodecki}. A QB is charged based on an interaction
protocol between QB itself with either an external field or a
quantum system which serves as a charger. It is then discharged into
a consumption hub based on the same protocol. When the battery
enters into an interaction with the charger, it transitions from a
lower energy level into the higher ones and will be charged. So far,
a variety of powerful charging protocols have been proposed in
different platforms, including two-level systems \cite{Farin, Zhang,
Fus}, harmonic oscillators \cite{Cata}, and hybrid light-matter
systems \cite{Maze, Manzo, Cond}. Some proposals have been also
devoted to implement QBs based on the two-level systems such as
trapped ions \cite{Forn, Lv}, cold atoms \cite{Bau} and
superconducting qubits \cite{Devoret}.

 Due to the fact that a real quantum system inevitably interacts with
its environment, studying QBs from the open quantum systems
perspective is attracting considerable interest. The interaction of
a QB with its surrounding environments causes the leakage of the
coherence of battery to the environment, leading to decoherence
effect in the battery. Such an adverse effect often plays a negative
role in the charging and discharging performance of QBs \cite{Camp,
Farin1, Carega}. Decoherence brought during the charging process
tends to lead QBs to a non-active (passive) equilibrium state in
which work extracting from the QBs is often impossible \cite{Barra}
in a cyclic unitary process. The environmental-induced noises also
affect QBs that are disconnected from both charger and consumption
hub and cause self-discharging of that QBs \cite{San0, Pedro,
Salimi}. Therefore, designing a more robust battery against the
environmental dissipations is valuable step for implementation of
QBs in the real-life. Recently, researchers have devoted efforts not
only to studying the effect of the environment on QBs, but also to
exploit non-classical effect as well as to developing open system
protocols to stabilize the charging cycle performance through
quantum control techniques. For example, Kamin et al \cite{Kamin1}
studied the charging performance of a qubit-based QB charged by the
mediation of a non-Markovian environment. They revealed the
non-Markovian property is beneficial for improving charging cycle
performance. In Ref. \cite{Squeezing}, the authors studied dynamics
of a continuous variable QB coupled weakly to the squeezed thermal
reservoir and managed to control the performance of the charging
process by boosting the quantum squeezing of reservoir. A feasible
route for harnessing loss-free dark states for stabilizing the
stored energy of a qubit-based open QB has been introduced in
\cite{Dark}. In addition to the above considerations, several other
protocols have been developed to protect the charging cycle of QBs
such as feedback control method \cite{Mitch, Shao, Ios}, convergent
iterative algorithm \cite{Borhan}, Bang-Bang modulation of the
intensity of an external Hamiltonian \cite{Franc}, inhiring an
auxiliary quantum system \cite{Behzadi}, modulating the detuning
between system and reservoir \cite{Yu0}, stimulated Raman adiabatic
passage technique \cite{Baris}, engineering quantum environments
\cite{Segal}, etc.

 On the other hand, according to the previous studies on
the Markovian and non-Markovian dynamics of open two-qubit systems,
translational motion of qubits provides novel insights for
stabilizing qubit-qubit entanglement against the environmental
induced dissipations by suitably adjusting the velocities of the
qubits \cite{Epjp0, morteza0, Chao0, sare0, Golkar1, Epjp1, MPLA,
Wang00}. We want here to use this safeguard capability of the
motional properties to improve the charging cycle performance of the
open qubit-based QBs. For this end, we consider a moving-biparticle
system composed of a qubit-battery and a qubit-charger that
independently interacts with their local environments. The battery
qubit here is charged with the help of the dipole-dipole interaction
with the charger qubit. We will investigate how the translational
motion of qubits affects the charging process of QB. Our results
show that translational motion of qubits always plays a constructive
role in protecting QB from decay induced by the environment. This
work is organized as follows: in Sec. 2, we introduce and describe
several figures of merit for characterizing the performance of QBs.
In Sec. 3, we illustrate our model and obtain explicit expressions
for the reduced density matrix of the QB and the charger. In Sec. 4
we present the results of our numerical simulations in the context
of their physical significance. Finally, Sec. 5 concludes this
paper.
\section{Figures of Merit}
Let us consider a QB modeled as a quantum system with d-dimensional
Hilbert space $\mathcal{H}$ and Hamiltonian $H_B$ such that
\renewcommand\theequation{\arabic{tempeq}\alph{equation}}
\setcounter{equation}{-1}
\addtocounter{tempeq}{1}\begin{eqnarray}\label{Bat}
H_B=\sum_{i=1}^{d} \varepsilon_i
|\varepsilon_i\rangle\langle\varepsilon_i|,
\end{eqnarray}
with non-degenerate energy levels $\varepsilon_i \leq
\varepsilon_{i+1}$. Internal energy of QB is given by $Tr(\rho_B
H_B)$, where $\rho_B$ is the state of the battery. Charging a QB
means brings the quantum system from a lower energy state $\rho_B$
to a higher energy state $\rho_B^\prime$, while discharging refers
to the inverse process, i.e., brings the quantum system from a
higher energy state $\rho_B^\prime$ to a lower one
$\rho_B^{\prime\prime}$:
\renewcommand\theequation{\arabic{tempeq}\alph{equation}}
\setcounter{equation}{-1}
\addtocounter{tempeq}{1}\begin{eqnarray}\label{den}
\texttt{Tr}\left\{\left(\rho_B^\prime-\rho_B\right) H_B\right\}\geq0,\qquad\qquad\qquad\qquad charging \nonumber \\
\texttt{Tr}\left\{\left(\rho_B^{\prime\prime}-\rho_B^\prime\right)
H_B\right\}\geq0.\qquad\qquad\qquad\quad \;\;discharging
\end{eqnarray}
Therefore, in a charging process, the actual stored energy of QB at
time $t$, regarding the initial energy, can be expressed as follows
\cite{Alicki}
\renewcommand\theequation{\arabic{tempeq}\alph{equation}}
\setcounter{equation}{-1} \addtocounter{tempeq}{1}\begin{equation}
\Delta E_B=\texttt{Tr}\{\rho_B(t) H_B\}-\texttt{Tr}\{\rho_B(0)
H_B\}.
\end{equation}
A complete converting the stored energy into valuable work is
impossible without dissipation of heat according to the second law
of thermodynamics. The maximum amount of energy extracted from a
given quantum state $\rho_B=\sum_{i} r_i |r_i\rangle\langle r_i|$,
($ r_i \geq r_{i+1}$) through a cyclic unitary operation is called
ergotropy \cite{Allahverdyan}. This quantity can be defined as
\cite{Allahverdyan, Franc0, Cakmak0}
\renewcommand\theequation{\arabic{tempeq}\alph{equation}}
\setcounter{equation}{-1}
\addtocounter{tempeq}{1}\begin{equation}\label{ergo}
\mathcal{W}=\texttt{Tr}\{\rho_B
H_B\}-\texttt{min}_U\,\texttt{Tr}\{U\rho_B U^{\dagger} H_B\},
\end{equation}
where the minimization is taken over all possible unitary
transformations acting locally on such system. It has been shown in
\cite{Allahverdyan} that no work can be extracted from the passive
counterpart of $\rho_B$ with the form $\sigma_{\rho_B}=\sum_{i} r_i
|\varepsilon_i\rangle\langle\varepsilon_i|$. The unique unitary
transformation $U=\sum_i |\varepsilon_i\rangle\langle r_i|$ on the
$\rho$ minimizes $\texttt{Tr}(U\rho_B U^{\dagger} H_B)$, and when
inserted in Eq. (\ref{ergo}) yields the following expression for the
ergotropy
\renewcommand\theequation{\arabic{tempeq}\alph{equation}}
\setcounter{equation}{-1} \addtocounter{tempeq}{1}\begin{equation}
\mathcal{W}=\sum_{i,j} r_j \varepsilon_i\left(|\langle
r_j|\varepsilon_i\rangle|^2-\delta_{ij}\right).
\end{equation}
In order to quantify the amount of extractable energy, the
efficiency $\eta$ is defined as the ratio between the ergotropy
$\mathcal{W}$ and the total charging energy $\Delta E_B$
\renewcommand\theequation{\arabic{tempeq}\alph{equation}}
\setcounter{equation}{-1} \addtocounter{tempeq}{1}\begin{equation}
\eta=\frac{\mathcal{W}}{\Delta E_B}.
\end{equation}% Figure environment removed
\section{Open Moving-Quantum Battery}
The open QB under consideration is composed of an atomic two-qubit
system, the qubit $A$ as a charger and the qubit $B$ as a quantum
battery, coupled to each other trough the dipole-dipole interaction.
The battery and charger qubits coupled locally to two independent
zero-temperature cavity reservoirs (see Fig. 1). We assume that each
qubit moves along the $z$-axis of its cavity at a constant
non-relativistic speed $v$. For simplicity we neglect here any
scattering \cite{Engl} or trapping \cite{Haro} effects and consider
the translational motion of the atom qubits being classically. Under
the dipole and rotating wave approximation, the entire system is
ruled by Hamiltonian (setting $\hbar=1$)
\renewcommand\theequation{\arabic{tempeq}\alph{equation}}
\setcounter{equation}{-1} \addtocounter{tempeq}{1}\begin{equation}
H=H_0+H_{int},
\end{equation}
with
\renewcommand\theequation{\arabic{tempeq}\alph{equation}}
\setcounter{equation}{-1}
\addtocounter{tempeq}{1}\begin{eqnarray}\label{Ham}
&&\hspace{-1.15cm}
H_0=H_A+H_B+H_{R_A}+H_{R_B}=\sum_{j=A,B}\left(\frac{\omega_0}{2}
\sigma_{z}^{j} + \sum_{k}\omega_{k}^j a_{k}^{j\dag} a_{k}^j\right),\nonumber\\
&&\hspace{-1.2cm}H_{int}=H_{A-B}+H_{A-R_A}+H_{B-R_B}=D\left(\sigma_{+}^{A}\sigma_{-}^{B}+\sigma_{-}^{A}
\sigma_{+}^{B}\right) +\sum_{j=A,B}\sum_{k} f_k^j(z)
\left(\mathfrak{g}_{k}^j \sigma_{+}^{j} a^j_k +H.c.\right).
\end{eqnarray}
Here, H.c. stands for Hermitian conjugate, $\sigma_z^j$,
$\sigma_+^j$, and $\sigma_-^j$ $(j=A,B)$ are, respectively, the
population inversion, raising and lowering operators of the $j$th
qubit with transition frequency $\omega_0$. $a_k^{j\dagger}$ and
$a^j_k$ are, respectively, the creation and annihilation operators
of the $k$th mode of the cavity reservoir $j$ with the frequency
$\omega_k^j$. Also, $D$ is coupling constant of the dipole-dipole
interaction between the battery and charger qubits, and
$\mathfrak{g}_{k}^j$ is the coupling constant between the $j$th
qubit and $k$th mode of in the cavity reservoir $j$. The effect of
translation motion of the battery and charger qubits has been
included in the model by introducing the $z$-dependent shape
function $f_k^j(z)$ in the Hamiltonian $H_{int}$. When the battery
and charger qubits are moving with same constant velocity $v$, the
shape function $f_k^j(z=vt)$ can be taken into account as
\renewcommand\theequation{\arabic{tempeq}\alph{equation}}
\setcounter{equation}{-1} \addtocounter{tempeq}{1}\begin{equation}
f_k^j(z)=\sin[\omega_k^j(\beta t-\Gamma)],\qquad\qquad j=A,B
\end{equation}
where, $\Gamma=L/c$ with $L$ being the size of the cavity. Also,
$\beta=v/c$ where $c$ refers to the speed of light in the vacuum
space. This particular form of the shape function can be obtained by
imposing an appropriate boundary condition on the cavity reservoirs
\cite{Lenard, morteza0}. Here we describe the translational motion
of both battery and charger qubits by classical mechanics ($z=vt$).
To this end, we will choose the values of the parameters in such a
way that the de Broglie wavelength of qubit $\lambda_B$ is
significantly smaller than the wavelength $\lambda_0$ associated
with the resonant transition $\omega_0=\omega_n$ ($\omega_n$ is the
central frequency of the cavity field mode) \cite{mortezapour,
Cook}. Furthermore, we consider a situation in which the photon
momentum is relatively small than the atomic momentum and thus we
neglect the atomic recoil caused by the interaction with the
electric field \cite{Wilkens}. In the optical regime, to ignore the
atomic recoil and consider the translational motion of atoms as
classical, the velocity of qubits should be $v\gg 10^{-3}$
\cite{morteza0}.

In the interaction picture (IP) generated by the unitary
transformation $U=e^{-iH_0t}$, the Hamiltonian (\ref{Ham}) can be
written as follows
\renewcommand\theequation{\arabic{tempeq}\alph{equation}}
\setcounter{equation}{-1}
\addtocounter{tempeq}{1}\begin{eqnarray}\label{HIP}
&&\hspace{-1.5cm}H_{IP}=D\left(\sigma_{+}^{A}
\sigma_{-}^{B}+\sigma_{-}^{A} \sigma_{+}^{B}\right)+
\sum_{j=A,B}\sum_{k} f_k^j(z)\left(\mathfrak{g}_{k}^j \sigma_{+}^{j}
a_k^{j} e^{i(\omega_0-\omega_k^j)t}+\mathfrak{g}_k^{j \ast}
\sigma_{-}^{j}a_{k}^{j\dag} e^{-i(\omega_0-\omega_k^j)t}\right).
\end{eqnarray}
It is straightforward to show that the total excitation operator
$\hat{\mathcal{N}}=\sum_{j=A,B}\left(\sum_k\hat{a_k}^{j\dagger}\hat{a_k}^j+
\frac{1}{2}\hat{\sigma}_{z}^{j}\right)+1$, commutes with the total
Hamiltonian, i.e. $[H,\hat{\mathcal{N}}]=0$ and therefor it is the
constant of the motion. This allows us to decompose Hilbert space of
the entire qubit-cavity system,
$\mathcal{H}=\mathcal{H}_q\otimes\mathcal{H}_R$ spanned by the basis
$\{\left|i_A,j_B\right\rangle\otimes\left|n_1,n_2, ...,n_k,
...\right\rangle_{R_A}|_{n_1,n_2,...=0}^{\infty}
\otimes\left|m_1,m_2, ...,m_k,
...\right\rangle_{R_B}|_{m_1,m_2,...=0}^{\infty}\}$
$\left(i,j=e,g\right)$ into the excitation subspaces, as follows
\renewcommand\theequation{\arabic{tempeq}\alph{equation}}
\setcounter{equation}{-1} \addtocounter{tempeq}{1}
\begin{eqnarray}
&&\hspace{-14mm} \mathcal{H}=\oplus_{n=0}^{\infty} \mathcal{H}_{n}.
\end{eqnarray}
As a result of this decomposition, the dynamics of the entire
qubit-reservoir system can be restricted to the excitation subspaces
labeled by the total excitation number $n$. Here we are interested
to explore dynamics of the entire system in the single-excitation
subspace $\mathcal{H}_1$ spanned by vectors
$\{\left|g_A,g_B\right\rangle\otimes\left|1_k\right\rangle_{R_A}\left|0_k\right\rangle_{R_B}|_{k=0}^\infty,
\left|g_A,g_B\right\rangle\otimes\left|0_k\right\rangle_{R_A}\left|1_k\right\rangle_{R_B}|_{k=0}^\infty,
\left|e_A,g_B\right\rangle\otimes\left|0_k\right\rangle_{R_A}\left|0_k\right\rangle_{R_B},
\left|g_A,e_B\right\rangle\otimes\left|0_k\right\rangle_{R_A}\left|0_k\right\rangle_{R_B}\}$
in which the single excitation is either in one of the qubits or in
the k-th mode of one of cavity reservoirs. We consider a normalized
initial state of entire qubit-reservoir as a superposition of
$\left|e_A,g_B\right\rangle\left|0_k\right\rangle_{R_A}\left|0_k\right\rangle_{R_B}$
and
$\left|g_A,e_B\right\rangle\left|0_k\right\rangle_{R_A}\left|0_k\right\rangle_{R_B}$
states with the following form
\renewcommand\theequation{\arabic{tempeq}\alph{equation}}
\setcounter{equation}{-1}
\addtocounter{tempeq}{1}\begin{eqnarray}\label{sai0}
|\Psi(0)\rangle=\big[c_1(0) |e_{A},g_{B}\rangle +c_2(0)
|g_{A},e_{B}\rangle\big]\otimes |0\rangle_{R_A}|0\rangle_{R_B}.
\end{eqnarray}
For times $t>0$, we expand the state vector $|\Psi(t)\rangle$ in
terms of the vector basis of the single-excitation subspace
$\mathcal{H}_1$ as
\renewcommand\theequation{\arabic{tempeq}\alph{equation}}
\setcounter{equation}{-1}
\addtocounter{tempeq}{1}{\footnotesize\begin{eqnarray}\label{sai}
&&\hspace{-3.5cm}\left|\Psi(t)\right\rangle=\big[c_1(t)\left |e_{A},
g_B\right\rangle +c_2(t) \left|g_A, e_B\right\rangle\big] \otimes
\left|0_k\right\rangle_{R_A}\left|0_k\right\rangle_{R_B}
\nonumber\\
&&\hspace{-2.35cm}+\left|g_A, g_B\right\rangle\otimes\sum_{k}
\big[d_{k}(t)\left|1_k\right\rangle_{R_A}\left|0_k\right\rangle_{R_B}+d_{k}^{\prime}(t)
\left|0_k\right\rangle_{R_A}\left|1_k\right\rangle_{R_B}\big],
\end{eqnarray}}
where the time-dependent amplitudes satisfy the normalization
requirement
\renewcommand\theequation{\arabic{tempeq}\alph{equation}}
\setcounter{equation}{-1} \addtocounter{tempeq}{1}\begin{eqnarray}
\sum_{i=1}^2|c_i(t)|^2+\sum_k(|d_{k}(t)|^2+|d_{k}^{\prime}(t)|^2)=1.
\end{eqnarray}
By taking the partial traces over the field modes and subsystem A
(B), the reduced time-dependent density operator for the battery
(charger) in the $\{\left|e\right\rangle, \left|g\right\rangle\}$
basis is obtained as
\renewcommand\theequation{\arabic{tempeq}\alph{equation}}
\setcounter{equation}{0} \addtocounter{tempeq}{1}\begin{eqnarray}
&&\hspace{-2cm}\rho_A(t)=|c_1(t)|^2\left|e_A\right\rangle\left\langle
e_A\right|-\left(1-|c_1(t)|^2\right)\left|g_A\right\rangle\left\langle
g_A\right|\label{rob2},\\
&&\hspace{-2cm}\rho_B(t)=|c_2(t)|^2\left|e_B\right\rangle\left\langle
e_B\right|-\left(1-|c_2(t)|^2\right)\left|g_B\right\rangle\left\langle
g_B\right|\label{rob1}.
\end{eqnarray}

 Inserting Eq. (\ref{sai}) into the time dependent Schr\"{o}dinger
equation $H_{IP}|\Psi(t)\rangle=i\frac{d}{d t}|\Psi(t)\rangle$, with
$H_{IP}$ given in (\ref{HIP}), leads to the following set of
differential equations for time-dependent amplitudes
\renewcommand\theequation{\arabic{tempeq}\alph{equation}}
\setcounter{equation}{0} \addtocounter{tempeq}{1}\begin{eqnarray}
&&\hspace{-4cm}i\dot{c_1}(t)=D c_2(t)+\sum_{k} \mathfrak{g}_{k}^A
f_k^A(z)d_{k}(t)e^{i(\omega_0-\omega_{k}^A)}\label{c1t},\\
&&\hspace{-4cm}i\dot{c_2}(t)= D c_1(t)+ \sum_{k} \mathfrak{g}_{k}^B
f_k^B(z)d_{k}^{\prime}(t)e^{i(\omega_0-\omega_{k}^B)}\label{c2t},\\
&&\hspace{-4cm}i\dot{d}_{k}(t)=\mathfrak{g}_k^{A\ast}f_k^A(z)
c_1(t)e^{-i(\omega_0-\omega_{k}^A)t},\label{d1t}\\
&&\hspace{-4cm}i\dot{d}_{k}^{\prime}(t)=
\mathfrak{g}_k^{B\ast}f_k^B(z)
c_2(t)e^{-i(\omega_0-\omega_{k}^B)t}\label{d2t}.
\end{eqnarray}
By integrating Eqs. (\ref{d1t}) and (\ref{d2t}) with the initial
condition $d_{k}(0)=0$ and $d_{k}^{\prime}(0)=0$ and putting their
solutions, respectively, in Eqs. (\ref{c1t}) and (\ref{c2t}), we get
the following integro-differential equations for the amplitudes
$c_1(t)$ and $c_2(t)$
\renewcommand\theequation{\arabic{tempeq}\alph{equation}}
\setcounter{equation}{0} \addtocounter{tempeq}{1}\begin{eqnarray}
&&\hspace{-2cm}\dot{c_1}(t)=-iDc_2(t)+\int_{0}^{t}F_A(t-t^\prime)c_1(t^\prime)dt^\prime,\label{mt}\\
&&\hspace{-2cm}\dot{c_2}(t)=-iDc_1(t)+\int_{0}^{t}F_B(t-t^\prime)c_2(t^\prime)dt^\prime,\label{nt}
\end{eqnarray}
where
\renewcommand\theequation{\arabic{tempeq}\alph{equation}}
\setcounter{equation}{0} \addtocounter{tempeq}{1}\begin{eqnarray}
&&\hspace{-2cm}F_{A}(t-t^\prime)=\sum_{k} |\mathfrak{g}_{k}^A|^2
e^{i(\omega_0-\omega_{k}^A)(t-t^\prime)}\sin[\omega_k^A(\beta^A
t-\Gamma)]\sin[\omega_k^A(\beta^A t^\prime-\Gamma)],\\
&&\hspace{-2cm}F_{B}(t-t^\prime)=\sum_{k} |\mathfrak{g}_{k}^B| ^2
e^{i(\omega_0-\omega_{k}^B)(t-t^\prime)}\sin[\omega_k^B(\beta^B
t-\Gamma)]\sin[\omega_k^B(\beta^B t^\prime-\Gamma)],
\end{eqnarray}
are the memory correlation function of the reservoirs $A$ and $B$,
respectively. For simplicity, we suppose
$F_{A}(t-t^\prime)=F_{B}(t-t^\prime)=F(t-t^\prime)$. In the limit of
a large number of modes ( in the continuum limit ), the correlation
function $F(t-t^\prime)$ takes the following form
\renewcommand\theequation{\arabic{tempeq}\alph{equation}}
\setcounter{equation}{-1}
\addtocounter{tempeq}{1}\begin{equation}\label{kernel}
F(t-t^\prime)=\int d\omega J(\omega)
e^{i(\omega_0-\omega)(t-t^\prime)}\sin[\omega(\beta
t-\Gamma)]\sin[\omega(\beta t^\prime-\Gamma)],
\end{equation}
in which $J(\omega)$ is the spectral density of the cavity
reservoirs and has the Lorentzian form \cite{Lenard, Breuer0}
\renewcommand\theequation{\arabic{tempeq}\alph{equation}}
\setcounter{equation}{-1}
\addtocounter{tempeq}{1}\begin{equation}\label{lorentz}
J(\omega)=\frac{1}{2\pi}\frac{\gamma\lambda^2}{(\omega_0-\omega-\Delta)^2+\lambda^2},
\end{equation}
where $\lambda$ defines the spectral width of the coupling which is
connected to the memory time $\tau_E$ by the relation
$\tau_E=\lambda^{-1}$ and $\gamma$ refers to the qubit-environment
coupling strength which is related to the relaxation time scale
$\tau_R$ by $\tau_R \approx \gamma^{-1}$. Also $\Delta$ is the
detuning of $\omega_0$ and the central frequency of the cavity. The
weak and strong coupling regimes can be distinguished by comparing
$\tau_E$ and  $\tau_R$, in other words with an increasing
$\frac{\tau_E}{\tau_R}=\frac{\gamma}{\lambda}$ ratio, the
interaction will transition into a strong coupling or a non-Markovian regime \cite{Breuer0}.\\
By inserting the Eq. (\ref{lorentz}) into the Eq. (\ref{kernel}) and
after some calculations, in the continuum limit ($\Gamma \rightarrow
\infty$), the correlation function is simplified as
\renewcommand\theequation{\arabic{tempeq}\alph{equation}}
\setcounter{equation}{-1}
\addtocounter{tempeq}{1}\begin{equation}\label{ft}
F(t-t^\prime)=\frac{\gamma \lambda}{4} \cosh[\beta
\overline{\lambda}(t-t^\prime)] e^{-(\lambda-i\Delta) |t-t^\prime|}
\end{equation}
with $\overline{\lambda}=\lambda+i(\omega_0-\Delta)$.\\
In view of (\ref{ft}), taking the Laplace transformations of both
sides of the differential Eqs. (\ref{mt}) and (\ref{nt}) and using
the convolution property
$\mathcal{L}[\int_{0}^{t}\mathbf{A}(t-t^\prime) \mathbf{B}(t^\prime)
dt^\prime]=\mathbf{A}(s)\mathbf{B}(s)$ yields
\renewcommand\theequation{\arabic{tempeq}\alph{equation}}
\setcounter{equation}{0} \addtocounter{tempeq}{1}\begin{eqnarray}
&&\hspace{-2cm}sc_1(s)-c_1(0)=-iDc_2(s)-F(s)c_1(s),\label{ms}\\
&&\hspace{-2cm}sc_2(s)-c_2(0)=-iDc_1(s)-F(s)c_2(s),\label{ns}
\end{eqnarray}
where the functions $c_1(s)$ and $c_2(s)$ are the Laplace
transformations of the $c_1(t)$ and $c_2(t)$, respectively, and
$F(s)$ is the Laplace transforms of $F(t-t^\prime)$ which has the
following explicit form
\renewcommand\theequation{\arabic{tempeq}\alph{equation}}
\setcounter{equation}{-1} \addtocounter{tempeq}{1}\begin{eqnarray}
F(s)=\frac{\gamma\lambda}{4}\frac{s+\overline{\lambda}}{(s+\overline{\lambda})^2-\beta^2\overline{\lambda}\,^2}.
\end{eqnarray}
By reformulating the Eqs. (\ref{ms}) and (\ref{ns}), we get a
general solution for $c_1(s)$ and $c_2(s)$ as follows
\renewcommand\theequation{\arabic{tempeq}\alph{equation}}
\setcounter{equation}{0} \addtocounter{tempeq}{1}\begin{eqnarray}
&&\hspace{-2cm}c_1(s)=\frac{s+F(s)}{\big(s+F(s)\big)^2+D^2}c_1(0)-i\frac{D}{(s+F(s))^2+D^2}c_2(0),\\
&&\hspace{-2cm}c_2(s)=\frac{s+F(s)}{\big(s+F(s)\big)^2+D^2}c_2(0)-i\frac{D}{(s+F(s))^2+D^2}c_1(0).
\end{eqnarray}
In continuation, by applying the inverse Laplace transformation on
the both side of the above equations, we obtain finally $c_1(t)$ and
$c_2(t)$, as
\renewcommand\theequation{\arabic{tempeq}\alph{equation}}
\setcounter{equation}{0} \addtocounter{tempeq}{1}\begin{eqnarray}
&&\hspace{-2cm}c_1(t)=\frac{1}{2}\bigg(c_1(0)\Re(\mathcal{M}(t))-ic_2(0)\Im(\mathcal{M}(t))\bigg)\label{ct12},\\
&&\hspace{-2cm}c_2(t)=\frac{1}{2}\bigg(c_2(0)\Re(\mathcal{M}(t))-ic_1(0)\Im(\mathcal{M}(t))\bigg)\label{ct122},
\end{eqnarray}
where, $\Re(x)$ ($\Im(x)$) is real (imaginary) part of $x$, and
\renewcommand\theequation{\arabic{tempeq}\alph{equation}}
\setcounter{equation}{-1} \addtocounter{tempeq}{1}\begin{equation}
\mathcal{M}(t)=\sum_{i,j,k=1}^3\varepsilon_{ijk}\frac{ e^{q_it}
(q_j-q_k)\bigg((q_i+\overline{\lambda})^2-\beta
^2\overline{\lambda}^2\bigg)}{\prod_{i=1}^{3}\prod_{j=i+1}^{3}(q_i-q_j)},
\end{equation}
with $\varepsilon_{ijk}$ is the Levi-Civita symbol and $q_i (i=  1,
2, 3)$ are the roots of
\renewcommand\theequation{\arabic{tempeq}\alph{equation}}
\setcounter{equation}{-1} \addtocounter{tempeq}{1}\begin{equation}
q^3+q^2(2 \overline{\lambda}-i \text{D} )+q \left(\frac{\gamma
\lambda }{4}+\overline{\lambda} (\overline{\lambda}-2 i
\text{D})-\beta ^2\overline{\lambda}^2\right)+\frac{\gamma  \lambda
\overline{\lambda}}{4}+i \text{D} \overline{\lambda}^2\left(\beta
^2-1\right)=0.
\end{equation}

 With substitution (\ref{ct12}) and (\ref{ct122}), respectively, into the reduced density matrices
(\ref{rob1}) and (\ref{rob2}), and then using the $\Delta
E_{A(B)}=\texttt{Tr}\{\rho_{A(B)}(t)
H_{A(B)}\}-\texttt{Tr}\{\rho_{A(B)}(0) H_{A(B)}\}$ , the internal
energy of the charger and battery are deduced as
\renewcommand\theequation{\arabic{tempeq}\alph{equation}}
\setcounter{equation}{-1} \addtocounter{tempeq}{1}\begin{equation}
\Delta
E_A=\omega_0\left(|c_1(t)|^2-|c_1(0)|^2\right),\quad\quad\Delta
E_B=\omega_0\left(|c_2(t)|^2-|c_2(0)|^2\right).
\end{equation}
On the other hand, one can obtain ergotropy of the battery by
substitution Eq. (\ref{rob1}) with Eq. (\ref{ergo}). So, we have
\renewcommand\theequation{\arabic{tempeq}\alph{equation}}
\setcounter{equation}{-1} \addtocounter{tempeq}{1}\begin{equation}
 W_B=\omega_0\left(2|c_2(t)|^2-1\right)\Theta
\left(|c_2(t)|^2-\frac{1}{2}\right),
\end{equation}
where $\Theta(x-x_0)$ is the Heaviside function, which satisfies
$\Theta(x-x_0)=0$ for $x<x_0$, $\Theta(x-x_0)=\frac{1}{2}$ for
$x=x_0$ and $\Theta(x-x_0)=1$ for $x>x_0$.
% Figure environment removed
% Figure environment removed
\section{Numerical Results and Discussion}
In this section, we will analyze the charging dynamics of the
introduced open moving-battery in the weak and strong coupling
regimes. In particular, we explore the role of the movement of QB on
the dynamical behavior of performance indicators including stored
energy, ergotropy and efficiency. In our following analysis, we
choose the optical regime parameters \cite{Hood, Pinkse} and
consider that qubit transition frequency as
$\omega_0=1.5\times10^{9}\lambda$. In what follows, we consider an
initial condition in which the battery is initially empty and the
charger has the maximum energy, i.e. $c_1(0)=0$, $c_2(0)=1$.
% Figure environment removed

 In Fig. 2, we plot the Markovian and non-Markovian dynamics of the stored energy $\Delta E_B$
for the initial state
$\left|\Psi(0)\right\rangle=\left|g\right\rangle_{A}\left|e\right\rangle_{B}\otimes
\left|0\right\rangle_{R_B}\left|0\right\rangle_{R_B}$, by
considering different values of the QB speed $\beta$. In panel (a),
the battery is charged in the Markovian dynamics with
$(\gamma=0.1\lambda)$, while in panel (b), it is charged in a
non-Markovian dynamics with $(\gamma=20\lambda)$. Here we consider a
situation at which the charger and battery's qubits are both in
resonance with the reservoir modes by setting $\Delta=0$. According
to this figure, the positive impact of the translational motion of
the charger and batter's qubits in controlling the stored energy of
battery is clearly visible in both Markovian and non-Markovian
charging processes. As can be seen in both Figs. 2(a) and (b), when
the charger and battery's qubits are at rest inside their cavity
reservoirs, the stored energy in the battery $\Delta E_B$ decays
into zero at sufficiently long times. However the rate of these
decays decreases regularly by gradual growth of the qubit velocity,
and therefore the energy stored in the battery and consequently the
charging process is strongly protected from the environmental
noises. Comparing Fig. 2(a) with Fig. 2(b) clearly reveals a
fundamental difference between Markovian and non-Markovian charging
processes. The maximal amount of stored energy in the Markovian
charging process is more than those of the non-Markovian charging
process. The reason stems from the nature of the qubit-cavity
coupling. In the non-Markovian charging process, the coupling
strength of charger's qubit to the cavity modes is greater than its
coupling to the battery's qubit, therefore, the initial internal
energy of charger has more tendency to evolve toward the reservoir
than to the battery. Moreover, since the motional effect of QB has
been included in battery-cavity and charger-cavity coupling
strength, it seems that increasing speed of QB decreases the
charger-cavity coupling strength in favor of to charger-battery
coupling strength, which increases the energy stored in the battery.

In order to get more insight to this area and a deeper understanding
of the relationship between the charger and battery energy, in Fig.
2 we have illustrated the energy stored in the battery at the end of
charging process as well as the energy that the charger loses at the
same time. Here $\Delta E_B$ and $|\Delta E_A|$ have been plotted as
a function of the dimensionless time $\lambda t$ for the qubit
velocities $\beta=0$ and $\beta=0.7\times 10^{-9}$ in the Markovian
and non-Markovian regimes. In the non-Markovian charging process,
$|\Delta E_A|$ is much more than $\Delta E_B$ for a given $\beta$ as
shown in Fig. 3(b). This implies that the internal energy of the
charger is not completely transferred to the battery. Fig. 3(b) also
shows that, when the charger and battery's qubits are at rest inside
their cavity reservoirs, the charger's qubit immediately loses a
large amount of its initial energy without being transferred to the
battery. However, increasing the qubit velocity (decreasing the
ratio of charger-cavity coupling strength to charger-battery
coupling strength) during the non-Markovian process, decreases the
initial loss-rate of the charger, and therefore improves the energy
transfer in the charging processes.

The relationship between the charger and battery energy in the
Markovian charging process is drastically different from that in the
non-Markovian charging process. One can infer from Fig. 3(a) that,
for the static battery-charger system ($\beta=0$), the total energy
of the charger can be transferred to the battery in the Markovian
short-charging process, where we have $|\Delta E_A|=\Delta E_B$.
Interestingly, when the qubits move with the velocity
$\beta=0.7\times10^{-9}$, $|\Delta E_A|=\Delta E_B$ holds at any
charging time. So, we conclude again that a robust Markovian
charging against the arisen dissipation can be achieved, when the
qubits move with higher velocities.
% Figure environment removed

 In the following, we examine the influence of translational motion
of the battery-charger system on the dynamics of ergotropy. In Fig.
4, we plot $W/W_{max}$ as a function of $\lambda t$ for the
different values of $\beta$ in the Markovian (Fig. 4(a)) and
non-Markovian (Fig. 4(b)) regimes. Our numerical results in Fig.
4(a) and (b) illustrate that, the effect of translational motion of
QB on the ergotropy is also constructive in both Markovian and
non-Markovian regimes. Fig. 4(b) shows that, in the non-Markovian
regime, in the cases of stationary ($\beta=0$) and slowly moving
($\beta=3\times10^{-9}$) qubits, we are not able to extract useful
work from the QB, but in this regime a considerable work can be
extracted, as the qubits move with a higher velocity
($\beta=0.8\times10^{-9}$). Our numerical results in Fig. 4(a)
illustrate that, the effect of translational motion of QB on the
ergotropy is more considerable in the Markovian case. We observe
that, in the Markovian regime, increasing the speed of QB $\beta$
(decreasing the qubit-reservoir coupling) not only boosts the
ergotropy, but also increases the number of time zones in which work
can be extracted. Accordingly, a strong robust charging process can
be established in the higher speed limit, in which the extractable
work approaches to its maximum value.

 Finally, we examine the effect of translational motion
of QB on the Markovian and non-Markovian charging efficiency. The
results for Markovian and non-Markovian charging processes are
presented in Fig. 5(a) and 5(b), respectively. Here we consider the
same parameter values as Fig. 4. Comparing Figs. 4 and 3 reveals
that both ergotropy and efficiency are positively affected by the
translational motion of QB. However the efficiency is influenced
more than the ergotropy; the amount of increment in efficiency is
more than the ergotropy in both Markovian and non-Markovian charging
processes.
\section{Outlook and summary}
To summarize, we proposed a mechanism for robust charging process of
an open qubit-based quantum battery (QB) whose robustness can be
well controlled by the translational motion of the charger and
battery in both Markovian and non-Markovian dynamical regimes. Both
the battery and charger's qubits move with a same speed inside two
separated identical environments, and are directly coupled by the
dipole-dipole interaction. We showed that the stored energy,
ergotropy and efficiency of the moving QB regularly increased with
the gradual growth of the charger and battery speed, thereby
improving its charging performance. The constructive role of the
translational movement of QB in controlling the charging process
arises from the attachment of qubits velocities to the
qubit-reservoir coupling strength (see Eq. (\ref{Ham})). According
to the adopted charging protocol, a weak qubit-reservoir coupling is
required for a strongly robust charging process which can be
fulfilled by adjusting $\beta$ to the higher velocities.

 Our results represent a further control strategy to have a robust QB with
a natural implementation in cavity-QED context. The strategy can be
easily implemented also in the circuit-QED setups where the qubit
position slowly varies linearly with time and also the qubit-cavity
interaction is tuned through a sinusoidal position-dependent
coupling \cite{Jones}.

  In perspective, we believe that this strategy can be used
to control the performance of the discharging of a qubit-based QB to
an available consumption hub. Further efforts in this field can be
devoted to use the proposed strategy for improving the performance
of the two-photon based charging process where the moving-QB is
coupled with a cavity reservoir by means of a two-photon
relaxation.\\\\
\textbf{\large{Data availability}}\\ The datasets used and analysed
during the current study available from the corresponding author on
reasonable request.
\begin{thebibliography}{99}
\bibitem{Alicki} R. Alicki and M. Fannes, Entanglement boost for extractable work from ensembles of quantum batteries, Phys. Rev. E 87, 042123 (2013).
\bibitem{Strasberg} P. Strasberg, G. Schaller, T. Brandes, and M. Esposito, Quantum and information thermodynamics: A unifying framework based on repeated interactions, Phys. Rev. X 7, 021003 (2016).
\bibitem{Vinjanampathy} S. Vinjanampathy and J. Anders, Quantum thermodynamics, Cont. Phy. 57, 545 (2016).
\bibitem{Goold} J. Goold, M. Huber, A. Riera, L. del Rio, and P. Skrzypczyk, The role of quantum information in thermodynamics: a topical review, J. Phys. A: Math. Theor. 49, 143001 (2016).
\bibitem{Campisi} M. Campisi, P. H\"{a}nggi, and P. Talkner, Colloquium: Quantum fluctuation relations: Foundations and applications, Rev. Mod. Phys. 83, 1653 (2011).
\bibitem{Gelbwaser} D. Gelbwaser-Klimovsky, W. Niedenzu and G. Kurizki, Thermodynamics of quantum systems under dynamical control, Adv. At. Mol. Opt. Phys., 64, 329 (2015).
\bibitem{Horodecki} M. Horodecki and J. Oppenheim,Fundamental limitations for quantum and nanoscale thermodynamics, Nature Comm. 4, 2059 (2013).
\bibitem{Farin} D. Farina, G. M. Andolina, A. Mari, M. Polini and V. Giovannetti, powerful charging of quantum batteries, Phys. Rev. B 99, 035421 (2019).
\bibitem{Zhang} Y-Y. Zhang, T-R. Yang, L. Fu and X. Wang, Powerful harmonic charging in a quantum battery, Phys. Rev. E 99, 052106 (2019).
\bibitem{Fus} L. Fusco, M. Paternostro, and G. D. Chiara, Work extraction and energy storage in the Dicke model, Phys. Rev. E 94, 052122 (2016).
\bibitem{Cata} R. R. Rodriguez, B. Ahmadi, P. Mazurek, S. Barzanjeh, R. Alicki and P. Horodecki, catalysis in charging quantum batteries, Phys. Rev. A 107, 042419 (2023).
\bibitem{Maze} J. Carrasco, J. R. Maze, C. Hermann-Avigliano and F. Barra, collective enhancement in dissipative quantum batteries, Phys. Rev. E. 105, 064119 (2022).
\bibitem{Manzo} M. Gumberidze, M. Kol\'{a}r and R. filip, Measurement induced Synthesis of coherent Quantum Batteries, Sci. Rep 9, 19628 (2019).
\bibitem{Cond} D. Ferraro, M. Campisi, G. M. Andolina, V. Pellegrini and M. Polini, High-power collective charging of a solid-state quantum battery, Phys. Rev. Lett. 120, 117702 (2018).
\bibitem{Forn} P. Forn-D\'{\i}laz, J. J. Garc\'{\i}la-Ripoll, B. Peropadre, J.-L. Orgiazzi, M. A. Yurtalan, R. Belyansky, C. M. Wilson, and A. Lupascu, Ultrastrong coupling of a single artificial atom to an electromagnetic continuum in the nonperturbative regime, Nat. Phys. 13, 39 (2016).
\bibitem{Lv} Bruzewicz, C.D.; Chiaverini, J.; McConnell, R.; Sage, J.M. Trapped-Ion Quantum Computing: Progress and Challenges. Appl. Phys. Rev. 2019, 6, 021314..
\bibitem{Bau} K. Baumann, C. Guerlin, F. Brennecke, and T. Esslinger, The dicke quantum phase transition with a superfluid gas in an optical cavity, Nature (London) 464, 1301 (2010)
\bibitem{Devoret} Devoret, M.H.; Schoelkopf, R. J. Superconducting Circuits for Quantum Information: An Outlook. Science 2013, 339, 1169
\bibitem{Farin1} D. Farina, G. M. Andolina, A. Mari, M. Polini, and V. Giovannetti, Charger-mediated energy transfer for quantum batteries: Anopen-system approach. Phys. Rev. B 99, 035421 (2019).
\bibitem{Camp} C. Ou, R. V. Chamberlin and S. Abe, Lindbladian operators, von Neumann entropy and energy conservation in time-dependent quantum open systems, Physica A 466, 450 (2017).
\bibitem{Carega} M. Carrega, A. Crescente, D. Ferraro, and M. Sassetti, Dissipative dynamics of an open quantum battery. New J. Phys. 22, 083085 (2020).
\bibitem{Barra} F. Barra, Dissipative charging of a quantum battery, Phys. Rev. Lett. 122, 210601 (2019).
\bibitem{San0} A. C. Santos, Quantum advantage of two-level batteries in
self-discharging process, Phys. Rev. E 103, 042118 (2021).
\bibitem{Pedro} L. P. Garcia-Pintos, A. Hamma, A. del Campo, Fluctuations in extractable work bound the charging power of quantum batteries. Phys. Rev. Lett. 125, 040601 (2020).
\bibitem{Salimi} F. H. Kamian, F. T. Tabesh, S. Salimi, F. Kheirandish, and A. C. Santos, Non-markovian effects on charging and selfdischarging processes of quantum batteries, New J. Phys. 22, 083007 (2020).
\bibitem{Kamin1} F. T. Tabesh, F. H. Kamin, and S. Salimi, Environmentmediated charging process of quantum batteries, Phys. Rev. A 102, 052223 (2020).
\bibitem{Squeezing} F. Centrone, L. Mancino, M. Paternostro, Charging batteries with quantum squeezing, https://doi.org/10.48550/arXiv.2106.07899.
\bibitem{Dark} J. Q. Quach and W. J. Munro, Using dark states to charge and stabilise open quantum batteries, Phys. Rev. Applied 14, 024092 (2020).
\bibitem{Mitch} M. T. Mitchison, J. Goold and J. Prior, Charging a quantum battery with linear feedback control, Quantum 5, 500 (2021).
\bibitem{Shao} Y. Yao and X. Q. Shao, Phys. Rev. E Optimal charging of open spin-chain quantum batteries via homodyne-based feedback control, 106, 014138 (2022).
\bibitem{Ios} S. Borisenok, Ergotropy of quantum battery controlled via target attractor feedback, J. Appl. Phys. 12, 43 (2020).
\bibitem {Borhan} R. R. Rodriguez, B. Ahmadi, G. Suarez, P. Mazurek, S. Barzanjeh, P. Horodecki, Optimal quantum control of charging quantum batteries, arXiv:2207.00094 [quant-ph].
\bibitem{Franc} F. Mazzoncini, V. Cavina, G. M. Andolina, P. A. Erdman and V. Giovannetti, Optimal control methods for quantum batteries, Phys. Rev. A 107 (2023) 032218.
\bibitem{Behzadi} N. Behzadi and H. Kassani, Mechanism of controlling robust and stable charging of open quantum batteries, J. Phys. A: Math. Theor. 55, 425303 (2022).
\bibitem{Yu0} J. L. Li, H. Z. Shen and X. X. Yi, Quantum batteries in non-Markovian reservoirs, Opt. Lett 21, 5614 (2022).
\bibitem{Baris} A. C. Santos, B. \c{C}akmak, S. Campbell and N.T. Zinner, Stable adiabatic quantum batteries, Phys. Rev. E 100, 032107 (2019).
\bibitem{Segal} J. Liu, D. Segal, Boosting quantum battery performance by structure engineering, arXiv:2104.06522 [quant-ph].
\bibitem{Epjp0} J. Taghipour, B. Mojaveri and A. Dehghani, Witnessing entanglement between two two-level atoms coupled to a leaky cavity via two-photon relaxation, Eur. Phys. J. Plus 137, 772 (2022).
\bibitem{morteza0} A. Mortezapour, M. A. Borji, and R. L. Franco, Protecting entanglement by adjusting the velocities of moving qubits inside non-Markovian environments, Laser Phys. Lett 14, 055201 (2017).
\bibitem{Chao0} W. Chao and F. Mao-Fa, The entanglement of two moving atoms interacting with a single-mode field via a three-photon process, Chin. Phys. B 19, 020309 (2010).
\bibitem{sare0} S. Golkar and M. K. Tavassoly and A. Nourmandipour, Entanglement dynamics of moving qubits in a common environment, J. Opt. Soc. Am. B 37, 400 (2020).
\bibitem{Golkar1} S. Golkar and M. K. Tavassoly And A. Nourmandipour, Qubit movement-assisted entanglement swapping, Chin. Phys. B. 29, 050304 (2020).
\bibitem{Epjp1} B. Mojaveri, A. Dehghani and J. Taghipour, Control of entanglement, single excited-state population and memory-assisted entropic uncertainty of two qubits moving in a cavity by using a classical driving field, Eur. Phys. J. Plus 137, 1065 (2022).
\bibitem{MPLA} J. Taghipour, B. Mojaveri and A. Dehghani, Witnessing entanglement between two two-level atoms moving inside a leaky cavity under classical control, Mod. Phys. Lett. A 37, 2250141 (2022).
\bibitem{Wang00} Q. Wang, R. Liu, H. M. Zou, D. Long and J. Wang, Entanglement dynamics of an open moving-biparticle system driven by classical-field, Phys. Scr. 97, 055101, (2022).
\bibitem{Allahverdyan} A. E. Allahverdyan, R. Balian and T. M. Nieuwenhuizen, Maximal work extraction from finite quantum systems. Eur. phys. Lett 67, 565 (2004).
\bibitem{Franc0} G. Francica, J. Goold, F. Plastina, and M. Paternostro, Daemonic ergotropy: enhanced work extraction from quantum correlations, npj Quantum Inf. 3, 12 (2017).
\bibitem{Cakmak0} B. \c{C}akmak, Ergotropy from coherences in an open quantum system, Phys. Rev. E 102, 042111 (2020).
\bibitem{Engl} B.G. Englert, J. Schwinger, A.O. Barut and M.O. Scully, Reflecting slow atoms from a micromaser field, Eur. Phys. Lett 14, 25 (1991).
\bibitem{Haro} S. Haroche, M. Brune and J.M. Raimond, Trapping atoms by the vacuum field in a cavity, Eur. Phys. Lett 14, 19 (1991).
\bibitem{Lenard} C. Leonardi and A. Vagliea, Non-markovian dynamics and spectrum of a moving atom strongly coupled to the field in a damped cavity, Opt. Commun 97, 130 (1993).
\bibitem{mortezapour} F. Nosrati, A. Mortezapour and R. Lo Franco, Validating and controlling quantum enhancement against noise by the motion of a qubit, Phys. Rev. A. 101, 012331 (2020).
\bibitem{Cook} R. J. Cook, Atomic motion in resonant radiation: An application of Ehrenfest's theorem, Phys. Rev. A. 20, 224 (1979).
\bibitem{Wilkens} M. Wilkens, Z. Bialynicka-Birula and P. Meystre, Spontaneous emission in a Fabry-P\'{e}rot cavity: The effects of atomic motion, Phys. Rev. A. 45, 477 (1992).
\bibitem{Breuer0} H. P. Breuer and F. Petruccione, \textit{The Theory of Open Quantum Systems} (Oxford University Press, Oxford, New York, 2002).
\bibitem{Hood} C. J. Hood et al., The Atom-Cavity Microscope: Single Atoms Bound in Orbit by Single Photons, Science 287, 1447 (2000).
\bibitem{Pinkse} P. W. H. Pinkse et al., Trapping an atom with single photons, Nature 404, 365 (2000).
\bibitem{Jones} P. J. Jones, J. A. M. Huhtam\"{a}ki, K. Y. Tan and M. M\"{o}tt\"{o}nen, Tunable electromagnetic environment for superconducting quantum bits, Sci. Rep. 3, 1987 (2013).
\end{thebibliography}
\end{document}

% ######################################################
% Pairwise Comparisons
% ######################################################
Studying such perceptions has traditionally been carried out using direct rating methods (users assign a score to each event or situation). This procedure requires a well-defined scale and user training and is particularly difficult to conduct when events or conditions substantially differ from one another \cite{perez2017practical}, which is the case when analyzing real-world environments. In contrast, using pairwise comparisons (users compare two situations and choose one of the two) is often simpler and faster to set up, well-suited for non-expert participants \cite{perez2017practical}, and presents lower measurement error compared to direct ratings \cite{shah2015estimation}. With this in mind, we employ pairwise comparisons to analyze cycling safety perceptions. Moreover, we draw current practice and knowledge from other research areas (e.g., sports outcome prediction and preference learning) about pairwise comparisons and how algorithms can be used to study cycling safety perceptions, something unexplored in cycling safety research. This paves the way to scale safety perception studies and ubiquitously understand how individuals perceive cycling risk.


% ######################################################
% Gap, Objectives & Contributions
% ######################################################
The main contributions of this paper are as follows. First, we draw knowledge from other research areas about pairwise comparisons and apply them to studying cycling safety perceptions. This novel approach
uses a survey showcasing images of two road environments and asking users which one they find safer, if any. % We use respondents' answers to compare different methodologies previously applied to sports prediction and preference learning, showcasing how these can be directly applied to our main goal: understanding cycling perception of safety.
With the respondents' answers, we compare different methodologies, previously applied to sports prediction and preference learning, and show how these can be directly applied to our main goal: understanding cycling perception of safety. Lastly, we draw from these results to objectively classify cycling environments based on urban characteristics and cycling environments. 


% ######################################################
% Outline of article
% ######################################################
We divide the article as follows. In the next section, we explore the current literature on pairwise comparisons and how traditional rating methods unravel such data. In Section \ref{sec:survey}, we detail our pairwise comparison survey and present different algorithms to rate cycling environments. Next, in Section \ref{sec:ranking}, we present the methodology, overviewing all pairwise ranking algorithms and environment classification. Section \ref{sec:results} presents the results and highlights what environments are perceived as safer or riskier. Finally, Section \ref{sec:conclusions} concludes the paper and draws possible paths forward.
\section{Related Work}

\paragraph{Model-based Reinforcement Learning.}
Model-based reinforcement learning methods show a promising prospect for real-world decision-making problems due to their data efficiency. However, learning an accurate model is challenging, especially in complex environments. Many papers \cite{pets,me-trpo,mbpo,pdml} commonly use ensemble probabilistic networks to construct uncertainty-aware environment models.


The previously proposed model-based methods \cite{mve,steve,emc-ac,vagram} allow the model rollout to a fixed depth, and value estimations are split into a model-based reward and a model-free value. To guarantee the monotonic improvement, the recent work %stochastic lower bounds optimization
\cite{slbo} builds a lower bound of the expected reward and then maximizes the lower bound jointly over the policy and the model. Furthermore, model-based policy optimization \cite{mbpo} utilizes short model-generated rollouts to do policy improvement and evaluation, and also provides a guarantee of monotonic improvement.


Current model-based RL mainly focuses on better model usage. For example, M2AC \cite{m2ac} implements a masking mechanism based on the model’s uncertainty to decide whether its prediction should be used or not. Another line of works \cite{gps,svg} aims to exploit the differentiability of the learned model in model-based RL. Model-augmented actor-critic \cite{maac} uses the path-wise derivative of the learned model and policy across future time steps.
Our work estimates value function by utilizing the model error as regularization.


\paragraph{Model-based Planning.}
Many recent papers on deep model-based RL \cite{pets,VisualForesight,mpc} optimize the future action trajectories over a given horizon starting from the current state, which is usually referred as model-based planning.
Model predictive control \cite{mpc} is a common control approach for model-based planning. It frequently solves the action planning over a limited horizon and conducts the first action on the environment.
Random Shooting optimizes the action sequence among the randomly generated candidates to maximize the expected reward under the learned dynamic model, and PETS \cite{pets} uses the cross entropy method \cite{cem} to improve the efficiency of the random search. However, shooting methods usually rely on the local search in the action space and are not effective on high-dimension environments. 
To solve this problem,  the latest work \cite{latco} utilizes the collocation-based planning in a learned latent space.
% MSAC \cite{msac} uses long-term planning with a simple overshooting. 
In contrast, we extend the policy improvement step of SAC to distill from model-based planning to the policy, which reduces the cost in the deployment phase.
% A further step is to build powerful planning methods, such as Monte Carlo Tree Search  \cite{mcts1,mcts2}, which is successfully implemented in AlphaZero \cite{alphazero} and Muzero \cite{muzero}.

In addition, some recent works distill the result from model-based policy planning into an RL policy. POPLIN \cite{poplin} formulates action planning at each time step as an optimization problem w.r.t. the parameters of the policy network, and uses behavior cloning to distill the resulted action into the policy network. GPS \cite{gps2,gps} uses KL divergence to minimize the distance between the policy and the planning result. However, the essential theoretical properties of such distillation are not well-understood.
Instead, we propose an algorithm to improve the policy with the solution of model-based planning over multiple time steps, and give the theoretical guarantee of its improvement and convergence.

\paragraph{Actor-Critic Methods.}
Actor-critic algorithms are typically derived from policy iteration, which alternates between policy evaluation and policy improvement.
Deep deterministic policy gradient \cite{ddpg} is a common model-free actor-critic method, however, the critic is usually overestimated to predict Q value, which leads to the worse policy.
Moreover, twin delayed deep deterministic policy \cite{td3} mainly utilizes the clipped double Q learning to alleviate the above overestimation.
SAC \cite{sac,taec} is the SOTA algorithm of policy learning under the model-based paradigm. In the framework of SAC, the actor aims to maximize expected reward with entropy and the critic evaluates the expected cumulative reward with entropy. Due to the splendid performance of SAC, we choose it as the RL instance to prove the theoretical properties, by distilling the planning into an RL policy.
\section{Task Statement}\label{sec:Task Statement}

%{\color{blue}

   The task of few-shot image classification is usually defined in the $N$-way $K$-shot manner, in which $N$ and $K$ denote the number of classes and the number of labeled samples per class respectively. A workload consists of three disjoint sets: a training set, $\mathcal{D}_{R}$, a validation set $\mathcal{D}_{V}$ and a test set $\mathcal{D}_{T}$, in which the classes of $\mathcal{D}_R$, $\mathcal{D}_V$ and $\mathcal{D}_T$ are distinct from each other.
The training set of $\mathcal{D}_R$ contains a large number of labeled samples, and is usually used to train a generic feature extractor.
The validation set of $\mathcal{D}_V$, used for model selection, also contains labeled samples. The test set of $\mathcal{D}_T$, used for final evaluation, consists of support sets and query sets. Following the standard few-shot setting, final evaluation is supposed to be performed on a set of $N$-way $K$-shot tasks. Concretely, each task of $T_i$ is composed of a support and a query set. The support set, $\mathcal{S}$ = $\{(x_{k,n}^S, y_{k,n}^S)|k\in [1,K], n\in [1,N]\}$, contains $K$ samples per class, and the query set, $\mathcal{Q}$ = $\{(x_{k,n}^Q, y_{k,n}^Q)|k\in [1,K], n\in [1,N]\}$, contains $Q$ samples per class. Given a test task, a classifier needs to select one class from the provided $N$ candidates for each sample in the query set.     
	
	
  In this paper, we consider the setting of transductive learning, in which a classifier has access to all the samples in $D_T$ when reasoning about the labels of samples in its query sets. 

\iffalse	
	Formally, the task of few-shot image classification can be defined as follows:

\begin{definition}

  Suppose that a workload of few-shot image classification consists of a training set, $D_{R}$, a validation set $D_{V}$ and a test set $D_{T}$, in which the classes of $D_R$, $D_V$ and $D_T$ are distinct from each other. The goal of $N$-way $K$-shot learning is to learn a classifier, , based on $D_R$, $D_V$ and $D_T$, such that 
	
	
	The input of the training is $D_{train} = \{(x_i,y_i)| y_i \in C_{train}\}$, the output is the feature extractor $f_{train}$. The validation set is composed of the labled support set and the unlabeled query set, $D_{val}=\{S_{val},Q_{val}\}$,where $S_{val}=\{(x_i,y_i) | y_i \in C_{val}\}$ and $Q_{val} = \{(x_j) | x_j \in C_{val}\}$, aimming to get $f_{val}$ which is the optimal $f_{train}$. The testing set is $D_{test}=\{S_{test},Q_{test}\}$ where $S_{test}=\{(x_i,y_i) | y_i \in C_{test}, C_{test} \cap C_{val} = \varnothing\}$ and $Q_{test} = \{(x_j) | x_j \in C_{test} \}$, $S_{test}$ has N classes and K images each of class and the classes of $Q_{test}$ are the same as $S_{test}$, obtaining the labed of $Q_{test}$ by using the $f_{val}$.
	 
	
\end{definition}
}
\fi

%$D_{train}$ $\cap$ $D_{val}$ $\cap$ $D_{test}$ $=$ $\varnothing$. The datasets $D_{train}$ contain a large number of labeled examples and can be used to train a generic feature extractor. $D_{val}$ is divided into N-way K-shot data distribution to select the optimal model in the training phase. the classes of $D_{test}$ are distinct from those in $D_{train}$ and $D_{val}$. we define $x_i$ is an image, $y_i$ is the category label of the $x_i$, $C_{train}$ are the label of the $D_{train}$, $C_{val}$ are the label of the $D_{val}$, $C_{test}$ are the label of the $D_{test}$. and $C_{train} \cap C_{val} \cap C_{test} = \varnothing$.

% , each of which contains $s$ labeled samples(support set) and $q$ unlabeled samples(query set)

% Our task is to construct a classification model based on the feature vectors of the support set and the query set in $D_{test}$, by which we can infer the label of query set.

% In training phase,Nerual models are first trained on $D_{train} = {(x_i,y_i|y_i \in C_{train} )}$, where $x_i$ is an image, $y_i$ is the category label of the $x_i$ and $C_{train}$ are the label of the $D_{train}$. In varing phase, the $D_{val}$ are selected or adjusted the Nerual models. Then, in testing phase, the $D_{test}$ is divided into labeled support set($S_{test}$) and unlabeled query set($Q_{test}$), the task predicts the classes of $Q_{test}={x_i}$ given $S_{test} = {(x_i,y_i)| y_i \in C_{test}, C_{train} \cap C_{test} = \varnothing }$, where $C_{test}$ are the label of the $D_{test}$, $S_{test}$ has N classes and K images each of class and the classes of $Q_{test}$ are the same as $S_{test}$.



\section{The GML Framework}\label{sec:The GML Framework}

As shown in Figure~\ref{fig:model structure}, consistent with the general paradigm, the GML framework for few-shot image classification consists of the following three components:

\iffalse
% Figure environment removed
\fi


\subsection{Evidential Instance Labeling}

%given a classification workload, accurately labeling all the instances in the workload by the machine is usually very challenging. However, the task usually becomes much easier if the machine only needs to label some easy instances in the workload. In practical applications, 

       GML begins with some initial evidential observations. In the unsupervised setting, initial instance labeling can be achieved by human-crafted rules or unsupervised clustering. For instance, an instance very close to a class centroid usually has a high chance of belonging to the class. Few-shot image classification supposes that the support sets contain some labeled sample images (e.g., 5 images per class for 5-shot learning). Therefore, these labeled images can naturally serve as initial evidential samples, even though their number is very limited. In the case of 5-shot learning, the labeled samples in support sets are considered as initial evidential observations. In the case of 1-shot learning, we expand the set of initial evidential observations by automatically labeling one additional sample per class. Specifically, for each class, GML automatically labels the unlabeled sample image closest to the class centroid. As a result, for 1-shot learning, GML actually begins with 2 evidential observations per class.     
				
			% given a set of initial seeds, GML iteratively labels each unlabeled image. After the algorithm converges, 
\subsection{Feature Extraction and Influence Modeling}
    
		In GML, features serve as the medium for gradual learning. This step extracts the common features shared by the labeled and unlabeled samples. To facilitate extensive knowledge conveyance, it is desirable that a wide variety of features are extracted to capture diverse information. For each extracted feature, this step also needs to model its influence over the class labels of relevant samples.
		
		For image classification, CNN has been shown to be much more effective at extracting discriminative features than previous alternatives (e.g., manually crafted mechanisms). Therefore, our solution leverages CNN backbones to extract implicit image features that are indicative of class status. Specifically, it transforms images into high-dimensional vector representations in an embedding space by a CNN, and then uses the monotonic metrics, e.g., class centroid distance and k-nearest neighborhood, to extract discriminative features. It can be observed that the closer to a class centroid an image is, the more likely it belongs to the class; similarly, the more same-class nearest neighbors an image has, the more likely it belongs to the same class. 
		
% Figure environment removed

		As in previous work, we model a feature's influence over images' class status by a monotonous Sigmoid function. As shown in Figure~\ref{fig:sigmoid}, a sigmoid function has two parameters, $\alpha$ and $\tau$, which denote the midpoint and steepness of the curve respectively. Formally, given a class label, $c$, and its feature, $f_c$, the influence of $f_c$ over the class status of an image, $d$, is represented by
\begin{equation}
\label{eq:sigmoid}
  P_{f_c}(d) = \frac{1}{{1 + {e^{ - {\tau _{f_c}}(x_{f_c}(d) - {\alpha _{f_c}})}}}},
\end{equation}

in which $P_{f_c}(d)$ denotes the probability of $d$ having the label of $c$ as indicated by $f_c$, $x_{f_c}(d)$ represents $d$'s feature value w.r.t $f_c$. According to Eq.~\ref{eq:sigmoid}, provided with the values of $\alpha_{f_c}$ and $\tau_{f_c}$, the influence model statistically dictates that any feature value of $x_{f_c}(d)$ corresponds to a label probability. As shown in Figure~\ref{fig:sigmoid}, different combinations of $\alpha_{f_c}$ and $\tau_{f_c}$ can result in different influence model shapes. Typically, the value of $P_{f_c}(d)$ increases with the feature value of $x_{f_c}(d)$.



		
		 %Specifically, for each feature, GML models its influence over pair labels by a monotonous

		
		%This step also needs to model the extracted features as factors in a factor graph, and quantify their influence over relevant instances. 
		  
		%Extracting classification features and building factor graphs to transfer knowledge from easy to difficult cases. By obtaining shared features between labeled and unlabeled samples, the acquired features possess knowledge sharing and transferability. Affect modeling is conducted based on the obtained features and their corresponding instance labels.
		
		

\subsection{Gradual Inference}
    
		
		GML fulfills gradual learning by iterative factor inference on a factor graph, $G$, which consists of a set of evidence variables, a set of inference variables, and a set of factors. For few-shot image classification, an evidence variable represents a labeled image in the support set, an inference variable represents an unlabeled image in the query set, and a factor represents the correlation between images. Typically, GML labels only one image at each iteration. Once an inference variable is labeled, its value remains unchanged and would serve as an evidence variable in the following iterations. 

\iffalse		
{\color{red} The process is shown in Figure~\ref{fig:gradual structure}.}

% Figure environment removed
\fi
        
    Formally, we denote the set of evidence variables by ${\bf{\Lambda}}$, the set of inference variables by $\bf{V_I}$, and the group of factor functions of variables indicating their correlations by ${\phi_{w_i}}(V_i)$. In the case of few-shot image classification, each variable in the factor graph is supposed to take one of several distinct values, each of which corresponds to a class label. 
Then, the joint probability distribution over $V=\{\Lambda, V_I\}$ of $G$ can be formulated as 
	\begin{equation}
		\label{eq:joint_prob}
		\begin{split}
			P_{\bf{w}}(\Lambda, V_I)=&\frac{1}{Z_{\bf{w}}}\prod_{i=1}^m\phi_{w_i}(V_i)
		\end{split}
	\end{equation}
	where $V_i$ denotes a set of variables, $w_i$ denotes a factor weight, $m$ denotes the total number of factors and $Z_{\bf{w}}$ denotes the normalization constant. Factor inference on $G$ learns factor weights by minimizing the negative log marginal likelihood of evidence variables as follows:
	\begin{equation} \label{eq:weight-learning}
		\hat {\bf{w}}  = arg \min \limits_{\bf{w}} -log \sum_{V_I} P_{\bf{w}}(\Lambda, V_I).
	\end{equation}	

	
 	In each iteration, GML typically labels the inference variable with the highest degree of evidential certainty, which is measured by the inverse of entropy as follows
%{\color{red}	
	 \begin{equation}
 E(v)=\frac{1}{H(v)}, % = \frac{1}{H_+(v) + H_-(v)}, 
 \end{equation}
where 
\begin{equation}
	H(v) = - (P_{max}({v})  \cdot {\log _2}P_{max}({v}) + 
	    (1-P_{max}({v}))  \cdot (1-{\log _2}P_{max}({v}))), 
	\end{equation}
where $E(v)$ and $H(v)$ denote the evidential certainty and entropy of $v$ respectively, and $P_{max}(v)$ denotes the max estimated class probability of $v$. 

%GML repeats the iteration until all the inference variables are labeled.



\begin{algorithm}[t]
	\caption{Scalable Gradual Inference.}
	\label{alg:gradualinference}
	%\begin{algorithmic}
	\While{there exists any unlabeled variable in $G$}
	{
		$V' \leftarrow$ all the unlabeled variables in $G$\;
		\For{$v\in V'$}
		{
			Measure the evidential support of $v$ in $G$\;
		}
		Select top-$m$ unlabeled variables with the most evidential support (denoted by $V_m$) \;
		\For{$v\in V_m$}
		{
			Approximately rank the entropy of $v$ in $V_m$\;
		}
		Select top-$n$ most promising variables in terms of entropy in $V_m$ (denoted by $V_n$) \;
		\For{$v\in V_n$}
		{
			Compute the probability of $v$ in $G$ by factor graph inference over a subgraph of $G$\;
		}
		Label the variable with the minimal entropy in $V_n$\;
	}
	%\end{algorithmic}
\end{algorithm}

To improve efficiency, as usual, the GML solution implements gradual inference by a scalable approach as sketched in Algorithm~\ref{alg:gradualinference}, which is essentially the same as what was previously proposed for entity resolution and sentiment analysis~\cite{hou2018r,wang2021aspect}. Scalable gradual inference consists of three steps: 1) measurement of evidential support; 2) approximate ranking of entropy; 3) subgraph factor inference. In the first step, it selects the top-$m$ unlabeled variables with the most evidential support in $G$ as the inference candidates. For each unlabeled variable, GML measures its evidential support from each feature by the degree of labeling confidence indicated by labeled observations and then aggregates them based on the Dempster-Shafer theory\footnote{https://en.wikipedia.org/wiki/Dempster-Shafer\_theory}. In the second step, it approximates entropy estimation by an efficient algorithm on the $m$ candidates and selects only the top-$n$ most promising variables among them for factor graph inference. Finally, the third step estimates the class probabilities of these $n$ selected variables by factor graph inference. 
	
It is noteworthy that the open-sourced GML engine has standardized the process of scalable gradual inference\footnote{https://chenbenben.org/gml.html}. Therefore, our GML implementation of few-shot image classification only needs to construct a factor graph, while scalable gradual inference on the factor graph can be automatically executed by the engine. Therefore, in the following section, we focus on how to construct the factor graph for few-shot image classification.
\section{Factor Graph Construction}\label{sec:Factor Graph Construction}


In this section, we first describe how to extract discriminative features by deep neural networks, and then elaborate how to model them as factors in a factor graph to enable gradual learning. 


\subsection{Discriminative Feature Extraction}\label{sec:Ifve}

%Since CNN has been widely used for few-shot image classification~\cite{2017Model,afrasiyabi2022matching,wertheimer2021few,huang2022few}, we leverage the existing CNN backbones for discriminative feature extraction. 

Our solution essentially trains a CNN model to learn high-dimensional vector representations of images in a class-sensitive embedding space, and then extracts monotonic features based on the learned representations to capture their class correlation. Since CNNs are prone to overfitting due to the limited amount of labeled images, we use two different backbones, ResNet-12 and WRN-28-10, for discriminative feature extraction to ensure feature diversity.  
	
	%Compared with increasing the depth, increasing the width has been empirically shown to be more effective in improving performance of residual networks. 
	
	 %Our proposed solution employs these two distinct, but to some extent complementary networks to extract diverse visual features to improve gradual learning. 
	
%{\color{blue} 
We have sketched the network structures of ResNet-12 and WRN-28-10 in Figure~\ref{fig:network structure} (b) and (c) respectively. ResNet-12 consists of four residual blocks, each of which contains three convolutional layers. Based on ResNet, the WRN-28-10 (Wide Residual Network) architecture increases the width of residual blocks from 12 to 28. By increasing the width of residual blocks, WRN-28-10 attains more advanced representational capabilities, and thus can capture finer-granular image features. In ResNet-12, we extract vector representations by the output of Average pooling2D and the input of the FC layer. In WRN-28-10, we extract vector representations by the output of the BatchNormalization layer and the input of the FC layer by the backbones. Both extracted vectors have the dimension of 640. Our ablation study has shown that using both ResNet-12 and WRN-28-10 for feature extraction is considerably better that using either of them. 
%}

We train our backbones using the methodology called EASY presented in~\cite{bendou2022easy}. Its basic principle is to take a standard classification architecture, and then set up two classifiers after the second-to-last layer of the network: one classifier for identifying the class of input samples and a new logistic regression classifier which can determine which one of the four possible rotations (one-quarter rotations of 360°) was applied to the input samples. EASY uses a two-step process to train model. In the first step, samples are directly input to the model and its first classifier. In the second step, samples are arbitrarily rotated and separately input into the two classifiers. Once training is complete, we freeze the backbone and use it, $f(\theta)$ as shown in Figure~\ref{fig:network structure} (a), to extract representation vectors for the images in $\mathcal{D}_V$ and $\mathcal{D}_T$. 
 

	%We apply ResNet-12 and WRN-28-10 for the aforementioned operations and obtain $f(\theta_1)$ and $f(\theta_2)$, respectively. 
	


Then, based on the learned vector representations, we extract two types of monotonic features as follows:
	
\begin{itemize}
  \item {\bf Class Centroid Distance (\emph{CCD}).} We estimate the prototype class centroid of each class by its support set, and then measure an image's similarity with a class by calculating its distance to the class centroid, which is defined as 1.0 minus vector cosine similarity. It is obvious that the smaller the distance is, the more likely the image belongs to the class. We denote the unary feature of class centroid distance by \emph{CCD}. 
	
	%For each image in the query set, we record its k-nearest neighbors in a class-sensitive deep embedding space. 
	
	\item {\bf K-nearest Neighborhood (\emph{KNN}).} Since a CNN classifier tends to separate the images with different class labels as far as possible while clustering the images with the same label, two images appearing very close in its corresponding embedding space usually have the same label. Therefore, we extract k-nearest neighborhood relations, ($v_i$, $v_j$, $sim_{i,i}$), in which $sim_{i,j}$ denotes the cosine similarity between the vector representations of $v_i$ and $v_j$. We denote the binary feature of k-nearest neighborhood by \emph{KNN}. In practical implementation, we suggest to set the value of $k$ within the reasonable range of [5,7].
\end{itemize}	

% Figure environment removed

% Figure environment removed

We have visualized the unary \emph{CCD} and the binary \emph{KNN} features in Figure~\ref{fig:distance structure} (b) and (c) respectively. 
It is noteworthy that \emph{CCD} and \emph{KNN} are complementary to each other, in that \emph{CCD} captures an image's correlation with global representatives, the prototype class centroids, while \emph{KNN} captures an image's correlation with local representatives, its nearest neighbors. 

%{\color{purple} Therefore, the hybrid mechanism of combining \emph{CCD} and \emph{KNN} can facilitate more effective knowledge conveyance compared with the homogeneous mechanism of using only \emph{CCD} or \emph{KNN}. Our ablation study in Section has also clearly demonstrated the efficacy of the hybrid mechanism.} Furthermore, our ablation study has also shown that using both ResNet-12 and WRN-28-10 for feature extraction is considerably better that using either of them. This observation clearly demonstrates that even though both ResNet-12 and WRN-28-10 are built upon the CNN backbone, they are to some extent complementary to each other in feature representation. 


\subsection{Feature Influence Modeling}\label{sec:GML}




 Since both \emph{CCD} and \emph{KNN} features are monotonic w.r.t class probability, we model their influence over class status by the sigmoid function as shown in Figure~\ref{fig:sigmoid}. Formally, denoting a \emph{CCD} feature by $f_c$, in which $c$ denotes a class, we model the influence of $f_c$ over a variable, $v$, by a unary factor defined as follows: 
\begin{equation}
	\label{eq:unary factor}
	\varphi_{f_c}(v) = 
	\left \{
	\begin{array}{ll}
	e^{w_{f_c}(v)}      &    
     if $ $ v = c;  \\
  1 &  if $ $ v\neq  c.
	\end{array} 
\right.
\end{equation}
where $w_{f_c}(v)$ denotes the factor weight of $v$, and 

\begin{equation}
\label{eq:unaryfactorweight}
  w_{f_c}(v) = \theta_{f_c}(v)\cdot \tau_{f_c}\cdot (x_{f_c}(v) - {\alpha_{f_c})},
\end{equation}
in which $\theta_{f_c}(v)$ denotes the confidence on influence modeling of $f_c$, $x_{f_c}(v)$ denotes the feature value of $v$, or the distance to the class centroid of $c$, and $\tau_{f_c}$ and $\alpha_{f_c}$ denote the steepness and mid-point of a sigmoid function respectively. In our implementation, as in previous GML work~\cite{hou2019gradual,HouTKDE}, we estimate $\theta_{f_c}(v)$ by the theory of regression error bound. The parameter values of $\tau_{f_c}$ and $\alpha_{f_c}$ are however supposed to be continuously optimized based on evidential observations in the process of gradual learning. 


	
	Similarly, we model the influence of the \emph{KNN} feature by the following binary factor: 
\begin{equation}
	\label{eq:binary factor}
	\varphi_{f_b}(v_i, v_j) = 
	\left \{
	\begin{array}{ll}
		e^{w_{f_b}(v_i,v_j)}     &      if \ v_i = v_j; \\
		1    &   otherwise.
	\end{array} 
	\right.
\end{equation}	
in which $f_b$ denotes the binary \emph{KNN} feature, $v_i$ and $v_j$ denote the two variables sharing the feature of $f_b$, $w_{f_b}(v_i,v_j)$ denotes the factor weight, and
\begin{equation}
\label{eq:unaryfactorweight}
  w_{f_b}(v_i,v_j) = \theta_{f_u}(v_i,v_j)\cdot \tau_{f_b}\cdot (x_{f_b}(v_i,v_j) - {\alpha_{f_b}}), 
\end{equation}
in which $\theta_{f_u}(v_i,v_j)$ denotes the confidence on binary feature influence modeling, $x_{f_b}(v_i,v_j)$ denotes the vector similarity of $v_i$ and $v_j$, and $\tau_{f_b}$ and $\alpha_{f_b}$ denote the steepness and mid-point of a sigmoid function. Similar to the unary factor, we estimate $\theta_{f_u}(v_i,v_j)$ by the theory of regression error bound, while $\tau_{f_b}$ and $\alpha_{f_b}$ need to be 
continuously optimized in the process of gradual learning. 

\section{Empirical Evaluation}\label{sec:experiments}


In this section, we empirically evaluate the performance of our proposed solution by a comparative study on real benchmark datasets. It is organized as follows: Subsection~\ref{sec:setup} describes the experimental setup. Subsection~\ref{sec:comparison} presents the comparative evaluation results. Subsection~\ref{sec:ablation} presents the evaluation results of ablation study. Finally, Subsection~\ref{sec:sensitivity} evaluates the performance sensitivity of GML w.r.t key parameters. 


\subsection{Experimental Setup} \label{sec:setup}

  We use four widely used benchmark datasets in our empirical study: 
\begin{itemize}
\item \textbf{MiniImageNet\cite{ravi2017optimization}}: it contains totally 100 classes, each of which contains 600 images with a size of 84$\times$84. The classes are split among training, validation, and test sets by the ratio of (64:16:20);

\item \textbf{TieredImageNet\cite{ren2018meta}}:it was created by selecting 34 categories from the ILSVRC-2012 imagenet, with each superclass containing 10-30 subclasses. There are totally 20 superclasses (351 subclasses) on the training set, 6 superclasses (97 subclasses) in the validation set and 8 superclasses (160 subclasses) in the test set. All the images are of the size of 84$\times$84; 

\item \textbf{Cifar-FS\cite{bertinetto2018meta}}: it contains totally 100 classes, each of which contains 600 images with the size of 32$\times$32. The classes are split among training, validation and test sets by the ratio of (64:16:20);

\item \textbf{CUB-200-2011\cite{wah2011caltech}}: also known as Caltech-UCSD Birds-200-2011, it contains totally 200 bird species, which are split among training, validation, and test sets by the ratio of (100:50:50).
\end{itemize}	

%methods involve utilizing common features or patterns in the samples for classification. Examples of inductive methods in the context of few-shot classification include 

 We compare the proposed GML solution with the SOTA methods of both inductive learning and transductive learning: 1) the methods of inductive learning include recently proposed ProtoNet~\cite{2017Prototypical}, MetaQDA~\cite{zhang2021shallow}, SetFeat-12~\cite{afrasiyabi2022matching} and PFENet~\cite{zhao2022self}. Among them, PFENet reported the overall best performance; 2) the methods of transductive learning include recently proposed TPN~\cite{sung2018learning}, LaplacianShot~\cite{ziko2020laplacian}, COM-FSC~\cite{liu2023cycle}, DFMN-MCT~\cite{kye2020meta}, PT-MAP~\cite{hu2021leveraging}, EASE+SIAMESE~\cite{zhu2022ease}, EASY~\cite{bendou2022easy} and PEM$_n$E-BMS*~\cite{hu2022squeezing}. Among them, the most recently proposed models, e.g., EASY, EASE+SIAMESE and PEM$_n$E-BMS*, have reported highly competitive performance. 

  Due to the large number of the compared methods, we directly compare the results of GML in term of accuracy with the results that have been reported in these methods' original papers. For fair comparison, on each workload, as usual, we report the average and standard variance over 10000 rounds of testing. Since most of the existing methods only report results on 2-3 test datasets of the 4 datasets we have used, we mark a compared method's result on a dataset as null (-) if it was not reported in the original paper.       

  For comparative evaluation, we have compared performance in both scenarios of intro-domain classification, where training and test sets come from the same data source, and cross-domain classification, where training and test sets come from different sources, e.g., a model is trained on MiniImageNet but tested on CUB-200-2011. Obviously, cross-domain classification is more challenging than intro-domain classification. In practical scenarios, it is usually expensive to manually label samples, but much easier to retrieve unlabeled samples. Therefore, we have also evaluated the robustness of the proposed GML solution by increasing the size of query set, or the samples to be labeled in the query set, and compared its performance with the existing SOTA alternatives.  

\iffalse
 models for few-shot image classification, which include: {\color{red}
\begin{itemize}
  \item Inductive Few-shot Learning: ProtoNet\cite{2017Prototypical}, MetaQDA\cite{zhang2021shallow}, SetFeat-12\cite{afrasiyabi2022matching}, and PFENet\cite{zhao2022self}. Among them, PFENet uses an improved convolutional structure, known as Self-Guided Information Convolution to obtains effective feature embeddings. PFENet is widely considered as the optimal inductive model.
  
  \item Transductive Few-shot Learning: {\color{blue}TPN\cite{sung2018learning}, LaplacianShot\cite{ziko2020laplacian}, COM-FSC\cite{liu2023cycle}, DFMN-MCT\cite{kye2020meta}, PT-MAP\cite{hu2021leveraging}, EASE+SIAMESE\cite{zhu2022ease}, EASY\cite{bendou2022easy}, and PEM$_n$E-BMS*\cite{hu2022squeezing}. The reported performance of these models, especially recently proposed ones, EASY, EASE+SIAMESE and PEM$_n$E-BMS* are highly competitive. }
\end{itemize}
}
\fi
%Among them, PEM$_n$E-BMS*\cite{hu2022squeezing} aims to make the distribution of feature vectors closer to a Gaussian distribution. It utilizes the idea of optimal transport algorithm to iteratively adjust the feature relationships between the support set and query set, aiming to obtain optimal classification performance.

% 归纳式方法:利用样本中的共同特征或规律来进行分类,例如,ProtoNet,MetaQDA,SetFeat12,FT+SGI-Conv+GCBNet等。其中,FT+SGI-Conv+GCBNet改进卷积结构the Self-Guided Information Convolution,通过获取有效的特征嵌入,将度量网络划分成多个块,以共享相邻矩阵构建多层图卷积网络,该方法是目前归纳式最优方法。
% 直推式学习:利用先验知识或已有的分类模型进行推理。例如,PT-MAP,EASE+SIAMESE,EASY,PEMnE_BMS等。其中,PEMnE_BMS将特征向量的分布更接近于高斯分布,利用最优传输算法的思想,通过迭代调整支持集和查询集间的特征关系,获取最优分类性能。
% 


We have implemented the proposed solution based on the open-sourced GML engine\footnote{https://github.com/gml-explore/gml}. In the GML implementation, we leverage both ResNet-12 and WRN-28-10 to extract deep vector representations. In the generation of \emph{KNN} features, by default, we set $k=6$. In the implementation of gradual inference, by default, we set the number of candidates with the highest evidential support at 50, and the number of candidates with the smallest approximate entropy at 10.  At each iteration of gradual inference, GML labels the 10 samples with the smallest approximate entropy by factor inference. After each iteration, given the newly labeled images, the algorithm correspondingly updates evidential support and approximate entropy estimation. In our sensitivity evaluation, we will show that the performance of GML is very stable w.r.t these parameters provided that their values are set within reasonable ranges. We have open-sourced the GML implementation\footnote{https://github.com/chn05/FSIC\_GML}.

%As usual, we measure performance by the metric of accuracy. {\color{blue} On each workload, as usual, we report the average and standard variance over 10000 rounds of testing. 

% We also set a distance threshold of 0.01 to filter out not-close-enough nearest neighbors.

\subsection{Comparative Evaluation} \label{sec:comparison}

%We have compared performance in two scenarios: 1) intro-domain classification, where training and test sets come from the same data source; 2) cross-domain classification, where training and test datasets come from different sources, e.g., a model is trained on MiniImageNet but tested on CUB-200-2011. It is obvious that compared with intro-domain classification, cross-domain classification is more challenging. 


% \hspace{-0.2in}
\textbf{Intro-domain classification:}
the comparative evaluation results have been presented in Table~\ref{table:minitiered} and~\ref{table:cifar-fs_cub}. It can be observed that the transductive approaches consistently perform considerably better than their inductive alternatives on all the workloads. It is worth pointing out that GML consistently outperforms the best transductive alternative by considerable margins on all the workloads. For instance, on MiniImageNet and TierImageNet, in the case of 5-way 1-shot learning, GML beats EASY+SIMESE, which is the best transductive approach based on the reported results, by the margins of 2.79\% and 2.41\% respectively. In the case of 5-way 5-shot learning, the improvement margins are 4.65\% and 1.5\% respectively. On Cifar-FS and CUB-100-2011, PEM$_n$E-BMS* is instead the best transductive approach. In the case of 5-way 1-shot learning, GML outperforms PEM$_n$E-BMS* by the margins of 4.58\% and 3.72\% respectively. In the case of 5-way 5-shot learning, the improvement margins are 3.89\% and 2.93\% respectively. 

\begin{table*}[htbp]
    %\setlength\tabcolsep{1.5pt}%设置表格列间距
    \centering
    \caption{Comparative results on the MiniImageNet and TieredImageNet datasets: GML clearly achieves the SOTA performance on both datasets, and the margins are considerable.}\label{table:minitiered}
    % \begin{tabular}{c,p{4cm},c,p{7cm}|}
    \small
    \begin{tabular}{c|c|c|c|c|c}
    \toprule
    \multicolumn{2}{c|}{\textbf{DataSets}} & \multicolumn{2}{c|}{\cellcolor[HTML]{9698ED}{\textbf{Mini-ImageNet}}} & \multicolumn{2}{c}{\cellcolor[HTML]{FFCE93}{\textbf{Tiered-ImageNet}}} \\ 
    \midrule
    Setting & Methods & 5-way 1-shot(\%) &5-way 5-shot(\%) & 5-way 1-shot(\%) & 5-way 5-shot(\%)\\ 
    \midrule
    \multirow{10}{*}{\textbf{Inductive}} &  Relation\cite{sung2018learning}  &$52.48\pm0.86$ & $69.83\pm0.68$  & $-$ & $-$  \\
    & Baseline++\cite{chen2019closer}  &  $53.97\pm0.79$ & $75.90\pm0.61$  & $-$ & $-$\\
    &MatchingNet\cite{2016Matching} &   $52.91\pm0.88$ & $68.88\pm0.69$ & $-$ & $-$ \\
    &ProtoNet\cite{2017Prototypical} &  $54.16\pm0.82$ & $73.68\pm0.65$ &  $65.65\pm0.92$ & $83.40\pm0.65$ \\
    &$S2M2_R$~\cite{mangla2020charting} &  $64.93\pm0.18$ & $83.18\pm0.11$ & $73.71\pm0.22$ & $88.59\pm0.14$\\
    &DeepEMD\cite{zhang2020deepemd} &  $65.91\pm0.82$ & $82.41\pm0.56$ & $71.16\pm0.87$ & $86.03\pm0.58$ \\
    &FRN\cite{wertheimer2021few} &   $66.45\pm0.19$ & $82.83\pm0.13$ & $71.16\pm0.22$ & $86.01\pm0.15$\\
    &MetaQDA\cite{zhang2021shallow} &  $67.83\pm0.64$ & $84.28\pm0.69$ & $74.33\pm0.65$ & $89.56\pm0.79$\\
    &SetFeat12\cite{afrasiyabi2022matching} &  $68.32\pm0.62$ & $82.71\pm0.46$ & $73.63\pm0.88$ & $87.59\pm0.57$ \\
    &PFENet\cite{zhao2022self} &  $68.76\pm0.75$ & $84.67\pm0.52$ & $74.93\pm0.84$ & $89.62\pm0.50$ \\
    \midrule
    \multirow{10}{*}{\textbf{Transductive}} &TPN\cite{sung2018learning} & $55.51\pm0.86$ & $69.86\pm0.65$ &  $59.91\pm0.94$ & $73.30\pm0.75$\\
    &COM-FSC\cite{liu2023cycle} &   $68.92\pm0.72$ & $85.37\pm0.49$ &  $79.69\pm0.74$ & $90.57\pm0.45$\\
    &LaplacianShot\cite{ziko2020laplacian} & $75.57\pm0.19$ & $84.72\pm0.13$ &  $80.30\pm0.22$ & $87.93\pm0.15$\\
    &DFMN-MCT\cite{kye2020meta} &  $78.55\pm0.86$ & $86.03\pm0.42$ &  $80.89\pm0.84$ & $87.30\pm0.49$\\
    &Transd-CNAPS+FETI\cite{bateni2022enhancing} &   $79.90\pm0.80$ & $91.50\pm0.40$ & $73.80\pm0.10$ & $87.70\pm0.60$ \\
    &PT-MAP\cite{hu2021leveraging} &   $82.92\pm0.26$ & $88.82\pm0.13$ & $85.67\pm0.26$ & $90.45\pm0.14$\\
    &EASE+SIAMESE\cite{zhu2022ease} &$83.00\pm0.21$ & $88.92\pm0.13$ & $88.96\pm0.23$ & $92.63\pm0.13$\\
    &EASY\cite{bendou2022easy} & $84.04\pm0.23$ & $89.14\pm0.11$ &  $84.29\pm0.24$ & $89.76\pm0.14$\\
    &PEM$_n$E-BMS*\cite{hu2022squeezing} &  $83.35\pm0.25$ & $89.53\pm0.13$ & $86.07\pm0.75$ & $91.09\pm0.14$\\
    % GiFeic & Transductive & WRN & \textcolor{red}{\textbf{0$\pm$0.32}} & \textbf{93.14$\pm$0.25} & \makecell{ResNet-12\\+WRN} & \textcolor{red}{\textbf{0$\pm$0.42}} & \textcolor{red}{\textbf{0$\pm$0.43}}\\
    \rowcolor[HTML]{FFFFFF}&\cellcolor[HTML]{96FFFB} \textbf{GML}  & \cellcolor[HTML]{96FFFB} \textbf{$85.79\pm0.32$} & \cellcolor[HTML]{96FFFB} \textbf{$93.57\pm0.25$} & \cellcolor[HTML]{96FFFB}  \textbf{$91.37\pm0.42$} & \cellcolor[HTML]{96FFFB} \textbf{$94.13\pm0.13$}\\
    \bottomrule
    \end{tabular}
    \end{table*}



\begin{table*}[htbp]
    %\setlength\tabcolsep{1.5pt}%设置表格列间距
    \centering
    \caption{Comparative results on the Cifar-FS and CUB-100-2011 datasets: GML clearly achieves the SOTA performance on both datasets, and the margins are considerable.}\label{table:cifar-fs_cub}
    % \begin{tabular}{c,p{4cm},c,p{7cm}|}
    \small
    \begin{tabular}{c|c|c|c|c|c}
    \toprule
    \multicolumn{2}{c|}{\textbf{DataSets}} & \multicolumn{2}{c|}{\cellcolor[HTML]{C0C0C0}{\textbf{Cifar-FS}}}  &  \multicolumn{2}{c|}{\cellcolor[HTML]{FE996B}{\textbf{CUB-100-2011}}} \\ 
    \midrule
    \textbf{Setting} &\textbf{Method} & \textbf{5-way 1-shot(\%)} & \textbf{5-way 5-shot(\%)} & \textbf{5-way 1-shot(\%)} & \textbf{5-way 5-shot(\%)} \\ 
    \midrule
    \multirow{9}{*}{\textbf{Inductive}} & MatchingNet\cite{2016Matching} & $43.88\pm0.75$ & $57.05\pm0.76$ & $-$ & $-$\\
    &ProtoNet\cite{2017Prototypical} & $41.54\pm0.76$ & $57.08\pm0.76$ & $-$ & $-$\\
    &DeepEMD\cite{zhang2020deepemd} &$46.47\pm0.78$ & $63.22\pm0.71$ &$-$ & $-$\\
    &SetFeat12\cite{afrasiyabi2022matching} &  $-$ & $-$ &  $79.60\pm0.80$ & $90.48\pm0.44$ \\
    &$S2M2_R$\cite{mangla2020charting} & $74.81\pm0.19$ & $87.47\pm0.13$ & $80.68\pm0.81$ & $90.85\pm0.44$\\
    &RENet\cite{2021Relational} &   $74.51\pm0.46$ & $86.60\pm0.32$ &  $79.49\pm0.44$ & $91.11\pm0.24$\\
    &FRN\cite{wertheimer2021few} & $-$ & $-$ &  $83.55\pm0.19$ & $92.92\pm0.10$ \\
    &PFENet\cite{zhao2022self} &  $-$ & $-$ & $86.09\pm0.19$ & $93.15\pm0.10$ \\
    &MetaQDA\cite{zhang2021shallow} &  $75.83\pm0.88$ & $88.79\pm0.75$ & $-$ & $-$\\
    \midrule
    \multirow{7}{*}{\textbf{Transductive}} & COM-FSC\cite{liu2023cycle} & $-$ & $-$ &  $83.93\pm0.66$ & $93.95\pm0.30$\\
    &EASY\cite{bendou2022easy} & $87.16\pm0.21$ & $90.47\pm0.15$ &  $90.56\pm0.19$ & $93.79\pm0.10$\\
    &iLPC\cite{Lazarou_2021_ICCV}& $86.51\pm0.23$ & $90.60\pm0.48$ &  $91.03\pm0.63$ & $94.11\pm0.30$\\
    &EASE+SIAMESE\cite{zhu2022ease} &  $87.60\pm0.23$ & $90.60\pm0.16$ & $91.68\pm0.19$ & $94.12\pm0.09$\\
    &PT-MAP\cite{hu2021leveraging} &  $87.69\pm0.23$ & $90.68\pm0.15$ & $91.55\pm0.19$ & $93.99\pm0.10$\\
    &PEM$_n$E-BMS*\cite{hu2022squeezing} & $87.83\pm0.22$ & $91.20\pm0.15$ & $91.91\pm0.18$ & $94.62\pm0.09$\\
    \rowcolor[HTML]{FFFFFF}  & \cellcolor[HTML]{96FFFB} \textbf{GML} & \cellcolor[HTML]{96FFFB} \textbf{$92.41\pm0.32$} & \cellcolor[HTML]{96FFFB} \textbf{$95.09\pm0.18$} &  \cellcolor[HTML]{96FFFB} \textbf{$95.63\pm0.06$} & \cellcolor[HTML]{96FFFB} \textbf{$97.55\pm0.43$}\\
    \bottomrule
    \end{tabular}
    \end{table*}
    
	Even with the existing inductive and transductive solutions being considered as a whole, GML consistently improves the reported SOTA performance on the four workloads by considerable margins. In the case of 5-way 1-shot learning, the improvement margins over the SOTA results are 1.75\%, 2.41\%, 4.58\%, and 3.95\% respectively. In the case of 5-way 5-shot learning, the improvement margins are 2.07\%, 1.5\%, 3.89\%, and 2.93\% respectively. Due to the widely recognized challenge of few-shot learning, these margins are truly considerable. Since our GML solution extracts discriminative features by the same deep neural models leveraged by the existing solutions, these evaluation results clearly demonstrate that compared with the existing transductive alternatives, gradual inference is a more effective mechanism for few-shot learning.       

%Our evaluation results clearly demonstrate the performance advantage of GML over the existing approaches in the scenario of few-shot learning. 


\begin{table}[htbp]
    %\setlength\tabcolsep{1.5pt}%设置表格列间距
    \centering
    \caption{Comparative results on cross-domain classification: the models are trained on MiniImageNet but tested on CUB-100-2011.}\label{table:cross}
    % \begin{tabular}{c,p{4cm},c,p{7cm}|}
    \small
    \begin{tabular}{c|c|c|c}
    \toprule
    \textbf{DataSets} & \multicolumn{3}{c}{\textbf{MiniImageNet$\rightarrow$CUB-100-2011 (5-way)}} \\
    \toprule
    \textbf{Setting} & \textbf{Methods}  &\textbf{1-shot(\%)} &\textbf{5-shot(\%)} \\ 
    \midrule
    \multirow{4}{*}{\textbf{Inductive}} &PFENet\cite{zhao2022self}   &$48.27$ & $69.51$ \\
    &$S2M2_R$\cite{mangla2020charting} & $48.24$ & $70.44$ \\
    &MetaQDA\cite{zhang2021shallow}  &  $53.75$ & $71.84$\\
    &FRN\cite{wertheimer2021few} & $54.11$ & $77.09$ \\
    \midrule
    \multirow{4}{*}{\textbf{Transductive}} &LaplacianShot\cite{ziko2020laplacian} & $55.46$ & $66.33$ \\
    &COM-FSC\cite{liu2023cycle} &  $53.14$ & $73.02$ \\
    &PEM$_n$E-BMS*\cite{hu2022squeezing}& $63.00$ & $79.15$ \\
    % \midrule
    \rowcolor[HTML]{FFFFFF}  & \cellcolor[HTML]{96FFFB} \textbf{GML} &\cellcolor[HTML]{96FFFB} $\textbf{67.29}$ & \cellcolor[HTML]{96FFFB} $\textbf{82.81}$ \\
    \bottomrule
    \end{tabular}
\end{table}

% \vspace{0.05in}
% \hspace{-0.2in}
\textbf{Cross-domain classification:} 
 cross-domain classification is usually performed on two datasets containing the same type of objects. Since both MiniImageNet and CUB-100-2011 contain images of bird species, as in previous work~\cite{mangla2020charting,zhang2021shallow,zhao2022self,ziko2020laplacian}, we train models on the MiniImageNet dataset and test its performance on another dataset of CUB-100-2011. 
	
  The comparative evaluation results have been presented in Table~\ref{table:cross}. It can be observed that on both cases of 1-shot and 5-shot learning, GML outperforms the existing alternatives by considerable margins. Specifically, on 1-shot learning, GML beats PEM$_n$E-BMS*, which is the best approach among the existing alternatives, by 4.29\% in terms of accuracy. On 5-shot learning, GML's improvement margin over PEM$_n$E-BMS* is similarly large at 3.66\%. Our experimental results clearly demonstrate that by gradual learning, the features learned in training classes can be better generalized to unseen classes. 

% It is interesting to point out that these observed margins are even more considerable than what have been observed on intro-domain classification. 


% Figure environment removed


\begin{table*}[htbp]
    %\setlength\tabcolsep{1.5pt}%设置表格列间距
    \centering
    \caption{The evaluation result of GML ablation study on MiniImagenet and Cifar-FS: using both ResNet-12 and WRN-28-10 vs using either of them.}\label{table:backbone}
    % \begin{tabular}{c,p{4cm},c,p{7cm}|}
    \small
    \begin{tabular}{c|c|c|c|c|c}
    \toprule
    \multicolumn{2}{c|}{\textbf{DataSets}}  &  \multicolumn{2}{c|}{\textbf{MiniImageNet}} &\multicolumn{2}{c}{\textbf{Cifar-FS}} \\
    \midrule
    \textbf{Methods} & \textbf{Network} & \textbf{5-way 1-shot(\%)} & \textbf{5-way 5-shot(\%)}  & \textbf{5-way 1-shot(\%)} & \textbf{5-way 5-shot(\%)} \\ 
    \midrule
    \multirow{3}{*}{\textbf{GML}} &  ResNet-12 & $84.74$ & $89.50$ & $85.69$ & $89.43$\\
    &  WRN-28-10 & $83.28$ & $88.80$ & $84.06$ & $88.35$\\
    \rowcolor[HTML]{FFFFFF}  &  \cellcolor[HTML]{96FFFB} ResNet-12+WRN-28-10 &\cellcolor[HTML]{96FFFB}$\textbf{85.79}$ &\cellcolor[HTML]{96FFFB}$\textbf{93.57}$ & \cellcolor[HTML]{96FFFB}$\textbf{92.41}$ &\cellcolor[HTML]{96FFFB}$\textbf{95.09}$\\
  
    \bottomrule
    \end{tabular}
    \end{table*}

\begin{table*}[htbp]
        %\setlength\tabcolsep{1.5pt}%设置表格列间距
        \centering
        \caption{ Parameter sensitivity evaluation results.}\label{table:sensitivity}
        % \begin{tabular}{c,p{4cm},c,p{7cm}|}
        \small
        \begin{tabular}{c|c|c|c|c|c}
        \toprule
        \multicolumn{2}{c|}{\textbf{DataSets}} & \multicolumn{2}{c|}{\textbf{MiniImageNet}} & \multicolumn{2}{c}{\textbf{Cifar-FS}} \\
        \midrule
         & \textbf{top-m/n/k}  &\textbf{5-way 1-shot(\%)} & \textbf{5-way 5-shot(\%)}  &\textbf{5-way 1-shot(\%)} & \textbf{5-way 5-shot(\%)} \\ 
        \midrule
        \rowcolor[HTML]{FFFFFF} \multirow{2}{*}{\textbf{$w.r.t$ m (n=10,k=6)}}  & m=40 &  $85.61$ & $ 93.28$  &  $92.16$ & $95.12$ \\
         & m=60 &  $ 85.56 $ & $ 93.23 $ &  $92.31$ & $ 94.59$  \\
         \midrule
         \multirow{2}{*}{\textbf{$w.r.t$ n (m=50,k=6)}} & n=8 &  $ 85.39$ & $ 93.48$  &  $92.09$ & $94.95$ \\
         & n=12 & $85.69$ & $ 93.56$  &  $92.24$ & $95.10$ \\
         \midrule
        \multirow{2}{*}{\textbf{$w.r.t$ k (m=50,n=10)}}  & k=5 & $ 85.76 $ & $ 93.33 $  &  $ 92.40 $ & $ 94.98 $   \\
        % & k=3 &  $85.59$ & $ 92.88$  &  $92.18$ & $94.57$ \\
        & k=7 & $ 85.15 $ & $ 93.04 $  &  $ 91.83 $ & $ 94.66 $   \\
        \midrule
        \rowcolor[HTML]{96FFFB}\textbf{GML}  & m=50, n=10, k=6 &$ 85.79 $ &$ 93.57 $  & $ 92.41 $ & $ 95.09 $   \\
        \bottomrule
        \end{tabular}
\end{table*}
%In practical scenarios, it is usually expensive to manually label many samples, but much easier to retrieve many unlabeled samples. Therefore, we evaluate the robustness of the proposed GML solution by increasing the size of query set, or the samples to be labeled in the query set. I


\textbf{Comparative evaluation with increasing size of query set:}
in the classical setting of few-shot learning, the number of queries per class is set at 15. Therefore, we increase the number of queries from 15 to 30, 50, 100, and finally up to 200, and compare the performance of GML with two recently proposed approaches, EASY and EASE, which can be considered as the SOTA representatives of the existing transductive approaches. 

%{\color{red} We report the evaluation results on MiniImageNet and Cifar-FS; the evaluation results on the other two datasets are similar, but omitted here.} 

The comparative evaluation results on MiniImageNet and Cifar-FS have been presented in Figure~\ref{fig:query structure}. The evaluation results on the other two datasets are similar, thus omitted due to space limit. It can be observed that on both 1-shot and 5-shot learning, the performance of GML consistently improves as the number of queries increases, even though by different margins on different workloads. The common pattern is that the performance of GML initially improves considerably as the number of queries increases from 15 to 30, but then gradually flattens out as it continues to increase. For instance, in the case of 1-shot learning on MiniImageNet, the accuracy improves by the margin of around 6\%. from 85.79\% to 91.88\%, when the number of queries increases from 15 to 30, and then continues to improve, even though by smaller margins, up to 94.3\% when the number of queries reaches 200. In comparison, the performance of EASE and EASY fluctuates only marginally when the number of queries increases. For instance, in the case of 1-shot learning on MiniImageNet, the performance of EASY even deteriorates marginally from 83.84\% to 82.98\%; but on 5-shot learning, its performance instead improves slightly, from 88.37\% to 89.06\%. These observations clearly demonstrate that GML is more robust than the existing transductive alternatives. They bode well for its application in real scenarios.  




\subsection{Ablation Study} \label{sec:ablation}



To verify the efficacy of leveraging two distinct backbones, i.e., ResNet-12 and WRN-28-10, for gradual learning, we have conducted an ablation study on the GML approach, which compares the solution using both models with the alternatives using either of them. The evaluation results on MiniImagenet and Cifar-FS have been presented in Table~\ref{table:backbone}. The evaluation results on the two other datasets are similar, thus omitted here. It can be observed that the GML using both of them performs considerably better that the alternatives using either of them. These results clearly demonstrate that even though both ResNet-12 and WRN-28-10 have been constructed based on the ResNet network, they are some extent complementary in feature extraction, and integrating them for knowledge conveyance can effectively improve the performance of gradual learning. 



%{\color{blue}
\textbf{An Illustrative Example:} in a run on MiniImageNet, inference accuracy based on ResNet-12 or WRN-28-10 is 85.33\% and 68\% respectively, but the accuracy is better at 94.67\% if gradual inference uses both of them. As shown in Figure~\ref{fig:example}, we take the sample with the id of 64 as an example. In the factor graph constructed based on ResNet, both unary and binary factors point to the class of $c_3$ for the sample. However, based on WRN-28-10, the factors point to its ground-truth class of $c_2$. It can be observed that the fused factor graph constructed based on both ResNet-12 and WRN-28-10 correctly point to the class of $c_2$ while labeling the sample of 64. It is noteworthy that the inference order in different factors may vary. This example clearly demonstrates that the framework of GML can effectively fuse diverse and noisy features to improve gradual knowledge conveyance.
%}     

% Figure environment removed

%a task is randomly extracted, and the feature vectors extracted by ResNet-12, WRN-28-10, ResNet-12+WRN-28-10 are used respectively. The prediction accuracy in GML are: 85.33\%, 68\% and 94.67\%. Take sample 64 as an example to illustrate the effectiveness of this method. (a) is the unary and binary factor constructed by the eigenvector of ResNet-12. When marking sample 64, the weight of the unary and binary factor of the evidence variable points to the $c_3$. (b) is the unary and binary factor constructed by obtaining the feature vector through WRN-28-10, and the weight of the unary and binary factor points the 64 sample to the $c_2$. (c) Constructing two types of unary and binary factors, due to the increase of features, the fusion of DS theory changes the order of reasoning, so sample 64 is indicated as the $c_2$ by more evidence variables. It is obvious that the method proposed in this paper adopts gradual inference to fuse different factors for feature complementation, so as to obtain higher accuracy.



%The accuracy on ResNet-12 and WRN-28-10 models is lower, while the accuracy on integrated feature models ResNet-12+WRN-28-10 is higher. Due to the issue of feature redundancy when adding ResNet-12 and WRN-28-10 features, its accuracy is lower than ResNet-12+WRN-28-10. gradual inference enhances detection accuracy by obtaining image feature vectors with different expressive capabilities and applying different features to factor graph modeling to improve evidence support for factors.

\subsection{Parameter Sensitivity Study} \label{sec:sensitivity}

In this subsection, we evaluate the performance sensitivity of the proposed GML solution w.r.t one key parameter of feature extraction, the $k$ value of k-nearest neighbors (\emph{KNN}) for binary feature extraction, and two key parameters of scalable gradual inference, the number of candidates with the most evidential support and the number of candidates with the smallest approximate entropy, or $m$ and $n$ as shown in Algorithm~\ref{alg:gradualinference}. By default, we set $m$ = 50, $n$ = 10 and $k$ = 6. In the sensitivity study, we vary the value of a parameter, but fix the values of the other two parameters. We set the values of parameters within reasonable ranges. Specifically, we vary the value of $k$ from 5 to 7, the value of $m$ from 40 to 60, and the value of $n$ from 8 to 12. We report the evaluation results on the MiniImageNet and Cifar-FS workloads; the results on other workloads are similar, thus omitted here. 
	
The detailed evaluation results have been presented in Table~\ref{table:sensitivity}. It can be observed that the performance of GML only fluctuates marginally ($\leq 0.5\%$ in most cases) as the values of $m$, $n$ and $k$ change. These observations clearly indicate that the performance of GML is very robust w.r.t these parameters. They bode well for their application in real scenarios. 

\section{Conclusion}\label{sec:Conclusion}
In this paper, we propose a novel solution for few-shot image classification based on the non-i.i.d paradigm of GML. Beginning with only a few labeled samples, it gradually labels unlabeled samples in the increasing order of hardness by iterative factor inference in a factor graph. To facilitate gradual knowledge conveyance, we leverage the existing CNN backbones to extract discriminative image features and model them as monotonous factors in a factor graph. Our extensive experiments on benchmark datasets have validated its efficacy. 

Our research on gradual machine learning is an ongoing endeavor, and there are several avenues for future exploration. First, while this paper targets few-shot image classification, the proposed approach is potentially applicable to other few-shot learning tasks (e.g., object detection, image segmentation). Detailed technical solutions however remain to be investigated. Second, we currently leverage the existing backbones and training procedure to extract deep image features. However, for few-shot gradual learning, deep feature generalization with only a few labeled samples remains its major performance bottleneck. It is very interesting to investigate how to design new backbones for feature extraction that can more effectively support gradual learning. 

% \section{Introduction}
% \lipsum[2]
% \lipsum[3]



% \section{Headings: first level}
% \label{sec:headings}

% \lipsum[4] See Section \ref{sec:headings}.

% \subsection{Headings: second level}
% \lipsum[5]
% \begin{equation}
% 	\xi _{ij}(t)=P(x_{t}=i,x_{t+1}=j|y,v,w;\theta)= {\frac {\alpha _{i}(t)a^{w_t}_{ij}\beta _{j}(t+1)b^{v_{t+1}}_{j}(y_{t+1})}{\sum _{i=1}^{N} \sum _{j=1}^{N} \alpha _{i}(t)a^{w_t}_{ij}\beta _{j}(t+1)b^{v_{t+1}}_{j}(y_{t+1})}}
% \end{equation}

% \subsubsection{Headings: third level}
% \lipsum[6]

% \paragraph{Paragraph}
% \lipsum[7]



% \section{Examples of citations, figures, tables, references}
% \label{sec:others}

% \subsection{Citations}
% Citations use \verb+natbib+. The documentation may be found at
% \begin{center}
% 	\url{http://mirrors.ctan.org/macros/latex/contrib/natbib/natnotes.pdf}
% \end{center}

% Here is an example usage of the two main commands (\verb+citet+ and \verb+citep+): Some people thought a thing \citep{kour2014real, keshet2016prediction} but other people thought something else \citep{kour2014fast}. Many people have speculated that if we knew exactly why \citet{kour2014fast} thought this\dots

% \subsection{Figures}
% \lipsum[10]
% See Figure \ref{fig:fig1}. Here is how you add footnotes. \footnote{Sample of the first footnote.}
% \lipsum[11]

% % Figure environment removed

% \subsection{Tables}
% See awesome Table~\ref{tab:table}.

% The documentation for \verb+booktabs+ (`Publication quality tables in LaTeX') is available from:
% \begin{center}
% 	\url{https://www.ctan.org/pkg/booktabs}
% \end{center}


% \begin{table}
% 	\caption{Sample table title}
% 	\centering
% 	\begin{tabular}{lll}
% 		\toprule
% 		\multicolumn{2}{c}{Part}                   \\
% 		\cmidrule(r){1-2}
% 		Name     & Description     & Size ($\mu$m) \\
% 		\midrule
% 		Dendrite & Input terminal  & $\sim$100     \\
% 		Axon     & Output terminal & $\sim$10      \\
% 		Soma     & Cell body       & up to $10^6$  \\
% 		\bottomrule
% 	\end{tabular}
% 	\label{tab:table}
% \end{table}

% \subsection{Lists}
% \begin{itemize}
% 	\item Lorem ipsum dolor sit amet
% 	\item consectetur adipiscing elit.
% 	\item Aliquam dignissim blandit est, in dictum tortor gravida eget. In ac rutrum magna.
% \end{itemize}


\bibliographystyle{unsrtnat}
\bibliography{references}  %%% Uncomment this line and comment out the ``thebibliography'' section below to use the external .bib file (using bibtex) .


%%% Uncomment this section and comment out the \bibliography{references} line above to use inline references.
% \begin{thebibliography}{1}

% 	\bibitem{kour2014real}
% 	George Kour and Raid Saabne.
% 	\newblock Real-time segmentation of on-line handwritten arabic script.
% 	\newblock In {\em Frontiers in Handwriting Recognition (ICFHR), 2014 14th
% 			International Conference on}, pages 417--422. IEEE, 2014.

% 	\bibitem{kour2014fast}
% 	George Kour and Raid Saabne.
% 	\newblock Fast classification of handwritten on-line arabic characters.
% 	\newblock In {\em Soft Computing and Pattern Recognition (SoCPaR), 2014 6th
% 			International Conference of}, pages 312--318. IEEE, 2014.

% 	\bibitem{keshet2016prediction}
% 	Keshet, Renato, Alina Maor, and George Kour.
% 	\newblock Prediction-Based, Prioritized Market-Share Insight Extraction.
% 	\newblock In {\em Advanced Data Mining and Applications (ADMA), 2016 12th International 
%                       Conference of}, pages 81--94,2016.

% \end{thebibliography}


\end{document}
