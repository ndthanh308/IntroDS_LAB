\section{Conclusion}\label{sec:Conclusion}
In this paper, we propose a novel solution for few-shot image classification based on the non-i.i.d paradigm of GML. Beginning with only a few labeled samples, it gradually labels unlabeled samples in the increasing order of hardness by iterative factor inference in a factor graph. To facilitate gradual knowledge conveyance, we leverage the existing CNN backbones to extract discriminative image features and model them as monotonous factors in a factor graph. Our extensive experiments on benchmark datasets have validated its efficacy. 

Our research on gradual machine learning is an ongoing endeavor, and there are several avenues for future exploration. First, while this paper targets few-shot image classification, the proposed approach is potentially applicable to other few-shot learning tasks (e.g., object detection, image segmentation). Detailed technical solutions however remain to be investigated. Second, we currently leverage the existing backbones and training procedure to extract deep image features. However, for few-shot gradual learning, deep feature generalization with only a few labeled samples remains its major performance bottleneck. It is very interesting to investigate how to design new backbones for feature extraction that can more effectively support gradual learning. 