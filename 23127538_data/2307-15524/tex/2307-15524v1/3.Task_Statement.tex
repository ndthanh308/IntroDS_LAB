\section{Task Statement}\label{sec:Task Statement}

%{\color{blue}

   The task of few-shot image classification is usually defined in the $N$-way $K$-shot manner, in which $N$ and $K$ denote the number of classes and the number of labeled samples per class respectively. A workload consists of three disjoint sets: a training set, $\mathcal{D}_{R}$, a validation set $\mathcal{D}_{V}$ and a test set $\mathcal{D}_{T}$, in which the classes of $\mathcal{D}_R$, $\mathcal{D}_V$ and $\mathcal{D}_T$ are distinct from each other.
The training set of $\mathcal{D}_R$ contains a large number of labeled samples, and is usually used to train a generic feature extractor.
The validation set of $\mathcal{D}_V$, used for model selection, also contains labeled samples. The test set of $\mathcal{D}_T$, used for final evaluation, consists of support sets and query sets. Following the standard few-shot setting, final evaluation is supposed to be performed on a set of $N$-way $K$-shot tasks. Concretely, each task of $T_i$ is composed of a support and a query set. The support set, $\mathcal{S}$ = $\{(x_{k,n}^S, y_{k,n}^S)|k\in [1,K], n\in [1,N]\}$, contains $K$ samples per class, and the query set, $\mathcal{Q}$ = $\{(x_{k,n}^Q, y_{k,n}^Q)|k\in [1,K], n\in [1,N]\}$, contains $Q$ samples per class. Given a test task, a classifier needs to select one class from the provided $N$ candidates for each sample in the query set.     
	
	
  In this paper, we consider the setting of transductive learning, in which a classifier has access to all the samples in $D_T$ when reasoning about the labels of samples in its query sets. 

\iffalse	
	Formally, the task of few-shot image classification can be defined as follows:

\begin{definition}

  Suppose that a workload of few-shot image classification consists of a training set, $D_{R}$, a validation set $D_{V}$ and a test set $D_{T}$, in which the classes of $D_R$, $D_V$ and $D_T$ are distinct from each other. The goal of $N$-way $K$-shot learning is to learn a classifier, , based on $D_R$, $D_V$ and $D_T$, such that 
	
	
	The input of the training is $D_{train} = \{(x_i,y_i)| y_i \in C_{train}\}$, the output is the feature extractor $f_{train}$. The validation set is composed of the labled support set and the unlabeled query set, $D_{val}=\{S_{val},Q_{val}\}$,where $S_{val}=\{(x_i,y_i) | y_i \in C_{val}\}$ and $Q_{val} = \{(x_j) | x_j \in C_{val}\}$, aimming to get $f_{val}$ which is the optimal $f_{train}$. The testing set is $D_{test}=\{S_{test},Q_{test}\}$ where $S_{test}=\{(x_i,y_i) | y_i \in C_{test}, C_{test} \cap C_{val} = \varnothing\}$ and $Q_{test} = \{(x_j) | x_j \in C_{test} \}$, $S_{test}$ has N classes and K images each of class and the classes of $Q_{test}$ are the same as $S_{test}$, obtaining the labed of $Q_{test}$ by using the $f_{val}$.
	 
	
\end{definition}
}
\fi

%$D_{train}$ $\cap$ $D_{val}$ $\cap$ $D_{test}$ $=$ $\varnothing$. The datasets $D_{train}$ contain a large number of labeled examples and can be used to train a generic feature extractor. $D_{val}$ is divided into N-way K-shot data distribution to select the optimal model in the training phase. the classes of $D_{test}$ are distinct from those in $D_{train}$ and $D_{val}$. we define $x_i$ is an image, $y_i$ is the category label of the $x_i$, $C_{train}$ are the label of the $D_{train}$, $C_{val}$ are the label of the $D_{val}$, $C_{test}$ are the label of the $D_{test}$. and $C_{train} \cap C_{val} \cap C_{test} = \varnothing$.

% , each of which contains $s$ labeled samples(support set) and $q$ unlabeled samples(query set)

% Our task is to construct a classification model based on the feature vectors of the support set and the query set in $D_{test}$, by which we can infer the label of query set.

% In training phase,Nerual models are first trained on $D_{train} = {(x_i,y_i|y_i \in C_{train} )}$, where $x_i$ is an image, $y_i$ is the category label of the $x_i$ and $C_{train}$ are the label of the $D_{train}$. In varing phase, the $D_{val}$ are selected or adjusted the Nerual models. Then, in testing phase, the $D_{test}$ is divided into labeled support set($S_{test}$) and unlabeled query set($Q_{test}$), the task predicts the classes of $Q_{test}={x_i}$ given $S_{test} = {(x_i,y_i)| y_i \in C_{test}, C_{train} \cap C_{test} = \varnothing }$, where $C_{test}$ are the label of the $D_{test}$, $S_{test}$ has N classes and K images each of class and the classes of $Q_{test}$ are the same as $S_{test}$.


