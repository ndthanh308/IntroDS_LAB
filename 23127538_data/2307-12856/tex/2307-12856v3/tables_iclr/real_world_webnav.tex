\begin{table*}[t]
\begin{center}
\begin{small}
\scalebox{0.76}{
\begin{tabular}{lccrrrrrrrrr}
\toprule
& \multicolumn{2}{c}{\textbf{Modules}} & \multicolumn{2}{c}{\housing{}} & \multicolumn{2}{c}{\socialmedia{}} & \multicolumn{2}{c}{\texttt{map}} & \multicolumn{3}{c}{\textbf{Error Ratio} (\%)} \\
 \cmidrule(r){2-3} \cmidrule(r){4-5} \cmidrule(r){6-7} \cmidrule(r){8-9} \cmidrule(r){10-12}
 & \textbf{Plan} & \textbf{Sum} & \textbf{Success} & \textbf{Score} & \textbf{Success} & \textbf{Score} & \textbf{Success} & \textbf{Score} & \textbf{Program} & \textbf{Plan} & \textbf{Sum} \\
\midrule
\textbf{Flan-U-PaLM} & \lno{} & \lno{} & 10.0 & 55.3 & 20.0 & 25.0 & 10.0 & 51.3 & 36 / \underline{88} / 11 & \underline{38} / 0 / \underline{78} & 26 / 12 / 11 \\
\textbf{Flan-U-PaLM+P} & \lyes{} & \lno{} & 50.0 & 79.5 & 20.0 & 38.3 & 30.0 & 73.8 & 39 / \underline{65} / 14 & \underline{56} / 30 / 29 & 5 / 5 / \underline{57} \\
\textbf{Flan-U-PaLM+S} & \lno{} & \lyes{} &  0.0 & 45.7 &  25.0 & 62.1 & 15.0 & 46.3 & 30 / \underline{67} / 0 & \underline{40} / 13 / \underline{100} & 30 / 20 / 0 \\
\textbf{WebAgent} & \lyes{} &\lyes{} &  \textbf{65.0} & \textbf{87.6} &  \textbf{70.0} & \textbf{85.8} & \textbf{80.0} & \textbf{93.8} & 20 / 33 / 25 & \underline{70} / \underline{50}  / \underline{50} & 10 / 17 / 25 \\
\bottomrule
\end{tabular}
}
%\end{sc}
\end{small}
\end{center}
\vskip -0.125in
\caption{
Success rate of real-world web automation on real estate, social media and map websites.
The score stands for the percentage of covered attributes specified in given instructions.
WebAgent, with language model modules for planning and summarization, achieves the best success (65\%, 70\%, 80\%, respectively), surpassing
other baselines, such as a single Flan-U-PaLM, that with a planning language model (Flan-U-PaLM+P), and that with a summarization language model (Flan-U-PaLM+S).
Without language model modules, prompted Flan-U-PaLM plans in an open-loop manner (\textbf{Plan}: \no{}) and  regular-expression-based retrieval summarizes HTML inputs (\textbf{Sum}: \no{}).
The results imply that self-experience supervision notably improve the performance, and task planning should be learned by finetuning domain language models for closed-loop planning, rather than by prompting single LLM for open-loop planning.
The error analysis describes the ratio across three types of errors in (\housing{}) / (\socialmedia{}) / (\texttt{map}) domains, which also points out that better adaptive planner to decompose the given instructions would contribute to further improvements of WebAgent.
}
\vskip -0.1in
\label{tab:realworld_results}
\end{table*}
