\begin{table*}[t]
\begin{center}
\begin{small}
%\begin{sc}
\scalebox{0.875}{
\begin{tabular}{lccrrrrrrr}
\toprule
& \multicolumn{2}{c}{\textbf{HTML-T5}} & \multicolumn{2}{c}{\housing{}} & \multicolumn{2}{c}{\socialmedia{}} & \multicolumn{3}{c}{\textbf{Error Analysis}} \\
 \cmidrule(r){2-3} \cmidrule(r){4-5} \cmidrule(r){6-7} \cmidrule(r){8-10}
 & \textbf{Plan} & \textbf{Sum} & \textbf{Success} & \textbf{Score} & \textbf{Success} & \textbf{Score} & \textbf{Program} & \textbf{Plan} & \textbf{Sum} \\
\midrule
\multirow{4}{*}{\textbf{WebAgent}} & \textcolor{cb_red}{\XSolidBrush} &  \textcolor{cb_red}{\XSolidBrush} & 10\% & 55.3 & 20\% & 25.0 & 35.5\% / \underline{87.5\%} & \underline{38.7\%} / 0.0\% & 25.8\% / 12.5\% \\
 & \textcolor{cb_green}{\CheckmarkBold} & \textcolor{cb_red}{\XSolidBrush} & 50\% & 79.5 & 20\% & 38.3 & 38.9\% / \underline{65.0\%} & \underline{55.5\%} / 30.0\% & 5.6\% / 5.0\% \\
 & \textcolor{cb_red}{\XSolidBrush} & \textcolor{cb_green}{\CheckmarkBold} &  0\% & 45.7 &  25\% & 62.1 & 30.4\% / \underline{66.7\%} & \underline{39.2\%} / 13.3\% & 30.4\% / 20.0\% \\
 & \textcolor{cb_green}{\CheckmarkBold} & \textcolor{cb_green}{\CheckmarkBold} &  \textbf{65}\% & \textbf{87.6} &  \textbf{70}\% & \textbf{85.8} & 20.0\% / 33.3\% & \underline{70.0\%} / \underline{50.0\%} & 10.0\% / 16.7\% \\
\bottomrule
\end{tabular}
}
%\end{sc}
\end{small}
\end{center}
% \vskip -0.1in
\caption{
Success rate of real-world web navigation on real estate and \socialmediaweb{}.
The score stands for the percentage of covered attributes specified in the given instructions.
WebAgent with HTML-T5 for both planning and summarization achieves the best success on both domains (65\% on \housing{}, 70\% on \socialmedia{}), surpassing
other baselines with open-loop planning with Flan-U-PaLM (\textbf{Plan}: \textcolor{cb_red}{{\scriptsize \XSolidBrush}}) or regular-expression-based retrieval (\textbf{Sum}: \textcolor{cb_red}{{\scriptsize \XSolidBrush}}).
The results imply that the planning with sub-instructions should be learned by finetuning domain language models for closed-loop planning grounded on HTML observations, rather than taken from LLMs with few-shot open-loop planning.
The error analysis describes the ratio across three types of errors in (\housing{}) / (\socialmedia{}) domains, which also points out that enhancing the adaptive planning to decompose the given instructions would contribute to further improvements of WebAgent.
}
\label{tab:realworld_results}
\end{table*}
