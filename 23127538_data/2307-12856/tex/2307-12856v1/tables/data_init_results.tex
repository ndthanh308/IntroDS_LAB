\begin{table}[t]
    \begin{center}
    \begin{small}
    %\begin{sc}
    \scalebox{1.0}{
        \begin{tabular}{lcrr}
            \toprule
            \textbf{CC-HTML} & \textbf{PEGASUS} & \housing{} & \textbf{MiniWoB++} \\
            \midrule
            Raw & \textcolor{cb_green}{\CheckmarkBold} & 80.56 & 56.7\% \\
            Extracted & \textcolor{cb_red}{\XSolidBrush} & 67.11 & 49.1\% \\
            % \midrule
            Extracted & \textcolor{cb_green}{\CheckmarkBold} & \textbf{82.46} & \textbf{57.0}\% \\
            \bottomrule
        \end{tabular}
    }
    %\end{sc}
    \end{small}
    \end{center}
    % \vskip -0.1in
    \caption{
        Ablations of HTML-T5-Base on dataset quality and initialization. For HTML-denoising, we prepare HTML corpus from CommonCrawl with (Extracted) or without (Raw) subtree extraction around label elements. We also compare the pre-training of base architectures with PEGASUS objective~\citep{zhang2020pegasus} before HTML-denoising.
        The results imply that PEGASUS pre-training is critical for the architectures and pre-processing with subtree extraction improves the downstream performance on HTML-based tasks.
        }
    \label{tab:data_init_results}
\end{table}
