% ****** Start of file aipsamp.tex ******
%
%   This file is part of the AIP files in the AIP distribution for REVTeX 4.
%   Version 4.1 of REVTeX, October 2009
%
%   Copyright (c) 2009 American Institute of Physics.
%
%   See the AIP README file for restrictions and more information.
%
% TeX'ing this file requires that you have AMS-LaTeX 2.0 installed
% as well as the rest of the prerequisites for REVTeX 4.1
% 
% It also requires running BibTeX. The commands are as follows:
%
%  1)  latex  aipsamp
%  2)  bibtex aipsamp
%  3)  latex  aipsamp
%  4)  latex  aipsamp
%
% Use this file as a source of example code for your aip document.
% Use the file aiptemplate.tex as a template for your document.
\documentclass[%
 aip,
% jmp,
% bmf,
% sd,
% rsi,
 amsmath,amssymb,
%preprint,%
 reprint,%
%author-year,%
%author-numerical,%
% Conference Proceedings
]{revtex4-1}

\usepackage{graphicx}% Include figure files
\usepackage{dcolumn}% Align table columns on decimal point
\usepackage{bm}% bold math
%\usepackage[mathlines]{lineno}% Enable numbering of text and display math
%\linenumbers\relax % Commence numbering lines
\usepackage[utf8]{inputenc}
\usepackage[T1]{fontenc}
\usepackage{mathptmx}
\usepackage{etoolbox}
\usepackage{float}
\usepackage{xcolor}
\newcommand{\red}[1]{{\color{red}#1}}
\usepackage{url}
%% Apr 2021: AIP requests that the corresponding 
%% email to be moved after the affiliations
\makeatletter
\def\@email#1#2{%
 \endgroup
 \patchcmd{\titleblock@produce}
  {\frontmatter@RRAPformat}
  {\frontmatter@RRAPformat{\produce@RRAP{*#1\href{mailto:#2}{#2}}}\frontmatter@RRAPformat}
  {}{}
}%
\makeatother
\begin{document}

%\preprint{AIP/123-QED}

%\maketitle
\title{Sub-Doppler laser cooling and magnetic trapping of natural-abundance fermionic potassium.}
% Force line breaks with \\
\author{Mateusz Boche\'nski}
 % \altaffiliation[Also at ]{Institute of Experimental Physics, University of Warsaw, Pasteura 5, 02-093 Warsaw, Poland}%Lines break automatically or can be forced with \\
\author{M. Semczuk}%
 \email{msemczuk@fuw.edu.pl}
\affiliation{ 
Institute of Experimental Physics, University of Warsaw, Pasteura 5, 02-093 Warsaw, Poland%\\This line break forced with \textbackslash\textbackslash
}%

%\date{\today}% It is always \today, today,
             %  but any date may be explicitly specified

\begin{abstract}
We report on reaching sub-Doppler temperatures of $^{40}$K in a single-chamber setup using a dispenser-based potassium source with natural (0.012$\%$ of $^{40}$K) isotopic composition. With gray molasses cooling on the $D_1$-line following a standard $D_2$-line magneto-optical trap, we obtain $3\times10^5$ atoms at $\sim$10~\textmu K. We reach densities high enough to measure the temperature via absorption imaging using the time-of-flight method. Directly after sub-Doppler cooling we pump atoms into the $F=7/2$ hyperfine ground state and transfer a mixture of $m_F=-3/2,-5/2$ and $-7/2$ Zeeman states into the magnetic trap. We trap $5\times10^4$ atoms with a lifetime of 0.6~s when the dispensers are heated up to maximize the atom number at a cost of deteriorated background gas pressure. When the dispensers have been off for a day and the magneto-optical trap loading rate has been increased by light induced atomic desorption we can magnetically trap $\sim$$10^3$ atoms with a lifetime of 2.8~s. The background pressure-limited lifetime of 0.6~s is a reasonable starting point for proof-of-principle experiments with atoms and/or molecules in optical tweezers as well as for sympathetic cooling with another species if transport to a secondary chamber is implemented.

Our results show that unenriched potassium can be used to optimize experimental setups containing $^{40}$K in the initial stages of their construction, which can effectively extend the lifetime of enriched sources needed for proper experiments. Moreover, demonstration of sub-Doppler cooling and magnetic trapping of a relatively small number of potassium atoms might influence experiments with laser cooled radioactive isotopes of potassium. 
\end{abstract}

\maketitle

\section{Introduction} 
%First 40K in Florence, then their following work, 40K on chip from Thywissen, un-unriched 40K from MIT. Our demonstration shows that un-unriched source could be used for optimisation of the experimental sequence without depleting the expensive enriched source. This is the first demonstration of subdoppler cooling of un-enriched 40K. The result are obtained with a laser system, that is designed to provide as much laser power as possible with a limited number of components. This limits our ability to simultaneously use the $$D_1$$- and $$D_2$$-line light as is common for other potassium experiments, but this limitation does not seem to have any negative influence on performance.

Observation of the first degenerate Fermi gas in 1999~\cite{demarco_DFG}, only four years after successful creation of a Bose-Einstein condensate~\cite{first_rb_bec}, widened the scope of experimental platforms where phenomena are predominantly governed by quantum statistics~\cite{Jeltes2007}. The choice of fermions for laser cooling is rather limited if compared to the number of available bosonic species. Among alkali atoms only lithium and potassium have long lived isotopes obeying Fermi statistics, $^6$Li and $^{40}$K. Laser cooling of the latter benefits from availability of high power CW lasers, nowadays primarily based on telecom technology~\cite{telecom_potassium_laser} or tapered amplifiers~\cite{potassium_TA_example}. Working with $^{40}$K has, however, a major drawback: natural potassium contains only 0.012\% of this isotope. Pioneering works that demonstrated laser cooling and trapping of several thousands $^{40}$K atoms loaded from a natural source~\cite{WilliamsonIII:98, cataliotti1998magneto} made it abundantly clear that an isotopically enriched source would be needed to provide sufficiently large atom number to implement evaporative cooling. Since then, nearly all experiments using fermionic potassium rely on sources enriched to 3\%-6\%~\cite{40K_enriched_demarco,40K_subdopp_cool_modugno}. The main issue with enriched potassium is its high price and limited availability - since the late nineties the price has gone up by more than an order of magnitude. As a result, experiments with enriched fermionic potassium do not use Zeeman slowers and almost exclusively rely on 2D MOTs~\cite{gokhroo2011sub} or double-stage MOTs~\cite{demarco_DFG} as a source of pre-cooled atoms. As opposed to other alkali species, single chamber apparatuses are rarely used with $^{40}$K, even though a single chamber design with a source located near the trapping region has enabled studies of the BEC-BCS crossover regime with $^6$Li~\cite{gunton_6Li2_bec}. In recent years some efforts have been made to bypass the need for potassium enrichment. Both a Zeeman slower~\cite{wu2011strongly} and a 2D MOT~\cite{Yang_2018} have been demonstrated but they have not been widely adopted by the community. 

In this work we demonstrate that state-of-the-art sub-Doppler cooling in a single chamber apparatus is possible with natural abundance potassium source. As a proof-of-principle measurement we trap 5$\times10^4$ atoms in a magnetic trap, sufficiently many to enable sympathetic cooling by another species~\cite{wu2011strongly} if the magnetically trapped clouds are transferred to a second chamber with a much better vacuum. Our results also show that during the construction of a new experimental setup one can use cheap, natural potassium sources for optimization, thus effectively prolonging the useful lifetime of the enriched source.

\section{Experimental setup} 

The experimental chamber is based on a fused silica glass cell without any antireflection coatings. Ultra-high vacuum is maintained by a 55~l/s ion pump (VacIon Plus 55 Noble Diode) supported by a titanium sublimation pump (fired no more than twice per year). The vacuum chamber is designed for laser cooling of cesium and potassium atoms; therefore, we use two dispensers (SAES Getters) for each species as atomic sources. We use potassium with a natural abundance of isotopes, that is, the content of $^{40}$K is only $\sim$0.012$\%$. All dispensers are located about 7~cm from the center of the magneto-optical trap (MOT). We use a LED emitting at the central wavelength of 370~nm (FWHM $\sim$10~nm) to increase the MOT loading rate due to the light-induced atomic desorption (LIAD) process~\cite{LIAD_K_Rb}. This method of increasing additional free-atom load in the trapping region allows us to release less atoms from the dispensers into the setup, thus minimizing the deterioration of the vacuum quality. The base pressure (i.e., when the dispensers have not been turned on for several days) reaches $10^{-11}$~mbar.

The quadrupole field for the magneto-optical and the magnetic trap is created by a pair of coils in a nearly anti-Helmholtz configuration. The coils are arranged such that the axis of the strongest gradient $B'_{\mathrm{axial}}$ is in the horizontal plane. Throughout the paper we report the radial gradient $B'_{\mathrm{r}}$ because it sets the effective trapping potential in the vertical direction.

The laser system is designed to maximize the available power by limiting double passes through acousto-optic modulators after the final amplification stage. In general, the design allows us to switch between potassium isotopes ($^{39}$K, $^{40}$K or $^{41}$K) on a sub-millisecond time scale and to implement sub-Doppler cooling on the $D_1$ line for all isotopes. Here, we only discuss the core elements of the laser system relevant to the current work on $^{40}$K. The details of the entire laser system will be the subject of a further publication.
% Figure environment removed

Figure~\ref{fig:laser_system_simplified_K} shows a simplified layout of the laser system. We use two master oscillator tapered amplifiers (Toptica TA pro) stabilized to crossover transitions in $^{39}$K, one on the $D_1$ line and the other on the $D_2$ line. The beams from both master lasers are combined on a polarizing beam splitting cube (PBS1) and are coupled into the same single mode, polarization maintaining fiber with perpendicular polarizations. The PBS2 splits each of the $D_1$ and $D_2$ beams into two paths, called cooling and repumping paths, where two double pass acousto-optic modulators DPAOM$_{\textrm{cool}}$ and DPAOM$_{\textrm{rep}}$ provide frequency tuning around optimized values of $+350$~MHz and $-370$~MHz, respectively, before seeding dedicated tapered amplifiers TA$_{\textrm{cool}}$ and TA$_{\textrm{rep}}$ (both are Toptica TA pro models, but we only use tapered amplifiers from these systems). Mechanical shutters S$_1$ and S$_2$ are used to choose whether the $D_1$ or $D_2$ line light is amplified. Single-pass acousto-optic modulators SPAOM$_{\textrm{cool}}$ and SPAOM$_{\textrm{rep}}$ provide final frequency shifts of $+80$~MHz and $-80$~MHz, respectively, and work as fast optical shutters. To eliminate light leakage caused by finite extinction of AOMs, mechanical shutters S$_3$-S$_5$ are used. Our design allows us to reach all frequencies required for efficient cooling on both $^{40}$K lines. %We are not able to implement a standard $D_1$\&$D_2$-line compressed magneto-optical trap as used in experiments with $^{39}$K and $^{41}$K \cite{salomon2014gray, chen2016production} due to the opening jitter of  .
In fermionic potassium, the splitting of the P$_{3/2}$ state is sufficiently large to allow efficient (compared to bosonic potassium isotopes) cooling and compression using the $D_2$ line, therefore we do not need to implement a $D_1$\&$D_2$-line compressed magneto-optical trap as used in experiments with $^{39}$K and $^{41}$K~\cite{salomon2014gray, chen2016production}. The design of our laser system provides an additional feature, namely the phase coherence of the cooling and the repumping beams which might be useful for driving Raman transitions between hyperfine states. The phase coherence has been shown to increase the cooling efficiency on the $D_2$-line for some species~\cite{Rosi2018}, but it is not clear if it provides any benefit in our setting since the role of phase coherence for $D_1$-line gray molasses has not been studied. 

The single chamber design of our vacuum system and the use of unenriched dispensers as a source of $^{40}$K heavily restrict the possibility to optimize the magneto-optical trap (MOT). In particular, we are forced to choose between maximizing the atom number and maximizing the lifetime. These two are strongly coupled unlike in setups where a 2D MOT or a Zeeman slower are used - even if these sources rely on natural abundance potassium~\cite{Yang_2018,wu2011strongly}.
We use three retro-reflected MOT beams, each collimated with a standard 30~mm diameter lens to a $1/e$ diameter of about 25~mm. This size guarantees that there is no noticeable distortion of the beams caused by the glass cell walls, which are separated by 30~mm. After passing the glass cell, the diameter of the beams is decreased to $\approx$9~mm with a telescope to fit through 0.5" waveplates. This choice is made to avoid purchasing custom made, 30~mm diameter waveplates which are rather costly if they have to be achromatic (our setup is also used for laser cooling of cesium). The retro-reflected beams are slightly focused to partially compensate for reflection losses on the un-coated glass cell walls.
% Figure environment removed
\section{Magneto-optical trap}
%known from experiments with alkali metals in which the energy structure is significantly separated \cite{metcalf2007laser}.
For magneto-optical trapping we use well established laser cooling methods where we close the $^2 S_{1 / 2}, F= 9 / 2 \to {^2P_{3 / 2}}, F'=11 / 2$ cooling transition by providing repumping light on the $^2 S_{1 / 2}, F= 7 / 2 \to {^2P_{3 / 2}}, F'=9 / 2$ transition to depopulate the ground hyperfine state with lower total angular momentum $F$. It is worth noting that this level has higher energy than the $F=9/2$ level, unlike in other alkali atoms. In figure~\ref{fig:HFS} we present the relevant energy levels and schematically show various detunings of the laser beams used at different stages of laser cooling on both the $D_1$ and the $D_2$ line. 


% Figure environment removed
During MOT loading we use maximum available laser power with total intensities of cooling (repumping) beams equal to 29$I_\mathrm{s}$ (18$I_\mathrm{s}$), where $I_\mathrm{s}$ is the saturation intensity of the $D_2$ line. We have found that we can obtain the highest atom number in the MOT for the magnetic field gradient of 11~G/cm, the cooling light red-detuned by \mbox{-3.5$\Gamma$~to~-4.5$\Gamma$} and the repumper red-detuned by \mbox{-0.5$\Gamma$~to~-4$\Gamma$}~(see figure~\ref{fig:mot_optimal_freq}). Here, $\Gamma$ is the natural linewidth of the $D_2$ line transitions. All measurements reported in the following paragraphs are obtained for the cooler (repumper) detunigs \mbox{equal to -4$\Gamma~(-1.5\Gamma$)}.
% Furthermore, to optimize the number of atoms, we used the photodiode to compare the loading curves for the situation when the blue LED diode ("LIAD") is on and off. The results are shown in Figure \ref{fig:HFS}, where we present photodiode signal in function of time. As can be observed, LIAD allowed us to achieve "n" times more atoms in "m" times shorter time.
% % Figure environment removed

With optimized parameters we have compared magneto-optical trap loading curves for two cases relevant to this investigation (see figure~\ref{fig:loading_curves}). In the first case, the number of atoms is maximized by heating up dispensers beyond typical temperatures we use while working with nearly {8,000} times more abundant $^{39}$K and by using light-induced atomic desorption (LIAD) during MOT loading. This reduces the quality of vacuum (see section~\ref{sec:magn_trap}) as dispensers release significant amounts of other, more abundant, potassium isotopes that collide with trapped $^{40}$K. Additionally, due to the nearby location of cesium dispensers we also observe an increase in partial pressure of cesium, enhanced even more by the use of LIAD. For regular operation, when working with $^{39}$K or Cs or their mixture, this cross-talk is minimized by choosing the temperature of dispensers which is a satisfactory compromise between the atom number and the lifetime of atoms trapped in an optical dipole trap ($\sim$5~s). For the study of sub-Doppler cooling, which takes on the order of 15~ms, this reduced lifetime is irrelevant thus the efficiency of the cooling process is investigated under these conditions.

The second case considers loading the magneto-optical trap using only LIAD, with atomic sources having been turned off for more than a day resulting in a much improved background pressure. As expected, the trapped atom number is significantly smaller and the lifetime of trapped atoms is longer (see section~\ref{sec:magn_trap}). The overall performance of sub-Doppler cooling is, however, essentially the same as in the first considered case.

Under conditions optimized for the maximum atom number we achieve a loading rate of $5\times10^5$~atoms/s, trapping about $5.5\times10^5$ atoms, nearly two orders of magnitude more than in the first reported magneto-optical traps of fermionic potassium~\cite{WilliamsonIII:98,cataliotti1998magneto}. For the background pressure optimized case the loading rate drops to $1\times10^4$~atoms/s and we can trap at best $8.5\times10^4$ atoms. Surprisingly, sub-Doppler cooling mechanisms that are typically present in magneto-optical traps of $^{40}$K~\cite{fernandes2012sub,sievers2015simultaneous,demarco_DFG, cataliotti1998magneto} do not seem to work in our setup and the steady state temperature of the MOT is 250~\textmu K, well above the Doppler limit of $T_{\mathrm{D}}=145$~\textmu K~\cite{tiecke2010properties}. We have not systematically investigated what could cause this difference because the figure of merit for us has been the combined efficiency of all cooling stages, not of each individual stage. As we show next, 250~\textmu K is a sufficiently good starting point to proceed with further cooling. 
% Figure environment removed

% Figure environment removed
\section{Sub-Doppler cooling}

We compress the cloud to cool it down and increase its density using a $D_2$-line compressed MOT stage. This is obtained by changing the cooling (repumping) beam detuning to -1$\Gamma$ (-5$\Gamma$) and increasing the magnetic field gradient to 18~G/cm in <~0.6~ms (set by the response time of the coil current driver). Once the magnetic gradient has reached the new value we ramp down the power of the cooling and the repumping beam in 10~ms to 6$I_\mathrm{s}$ and 1.4$I_\mathrm{s}$, respectively. We are not able to simultaneously ramp the frequency and the power due to the limitation of the RF source driving acousto-optic modulators (AOMs). This might be the reason that we do not observe a significant temperature drop, ending up with a cloud at 160~\textmu K. However, during this process there is no significant loss of atoms and we reach a phase space density (PSD) of $\mathrm{\rho = 1.6\times10^{-7}}$. 

We first investigate to what extent our setup allows for sub-Doppler cooling on the $D_2$ line. Previous works using the $D_2$ line reached temperatures below 50~\textmu K in optical molasses with the cooling beam red-detuned by more than -4$\Gamma$~\cite{40K_subdopp_cool_modugno,gokhroo2011sub} from the $F'=11/2$ level or by using coherent superposition of ground hyperfine states with red-detuned gray molasses~\cite{bruce2017sub}. % This stems from the fact that our laser setup has been optimized to provide sufficient frequency agility for $D_1$ gray molasses cooling, with the consequence of restricting the possibility of obtaining large detunings with respect to $P_{3/2}$ levels.
We essentially follow the work of Bruce~et.~al~\cite{bruce2017sub} where $D_2$-line gray molasses for $^{40}$K have been implemented for the first time. After the magnetic field of the compressed MOT has been switched off, we set the compensation coils currents to nullify the background magnetic field~\cite{dobosz2021bidirectional} and detune the repumping and the cooling light to produce a coherent superposition of ground state hyperfine levels via the excited state $F'=9/2$. We investigate the dependence of the final atom number and the final temperature on both the single photon detuning from the $F'=9/2$ level, $\Delta_\mathrm{D2}$, and on the two-photon detunig $\delta_\mathrm{D2}$ between the cooling and the repumping light. Here, we have used the cloud size after a fixed expansion time as a proxy for temperature to generate the plots shown in figures~\ref{fig:GMCD2}a)~and~b). We find that the maximum atom number and the minimum temperature can be obtained for two-photon detuning of about $\delta_\mathrm{D2} = -1\Gamma$ and single photon detuning of about $\Delta_\mathrm{D2}=-12\Gamma$, very similar to values reported by Bruce~et.~al~\cite{bruce2017sub}. Unfortunately, we observe a nearly 80\% loss of atoms with a negligible temperature drop to~140~\textmu K, as determined with a time of flight method.
Most of the reported sub-Doppler cooling techniques rely on simultaneous ramping of the frequency and the laser power which cannot be implemented with our current RF sources driving AOMs. This limitation might be the main reason for inefficient cooling using the $D_2$ line. Another possible source of the reduced cooling efficiency may be optimization of the cooling paths for the 770~nm light, i.e. $D_1$-line light. MOT beams and gray molasses beams are coupled into the same optical fiber, and they share the same paths upon leaving the fiber to form a MOT. We control the polarization of the overlapped beams with half-~and quarter-waveplates with the design wavelength of 767~nm ($D_2$ line) but we optimize the polarization of the 770~nm light, which slightly degrades the target polarization of the MOT beams and $D_2$ line molasses. This is a conscious choice as we have expected to use sub-Doppler cooling on the $D_1$ line all along while treating the MOT as a source of pre-cooled atoms. As such, the experimental setup has been optimized to maximize the atom number in the MOT and minimize the final temperature of the cloud after $D_1$-line cooling. From this point of view it may not be surprising that sub-Doppler cooling mechanisms on the $D_2$ line do not perform as well as reported by other groups~\cite{40K_subdopp_cool_modugno,gokhroo2011sub,bruce2017sub}.

We now focus our attention on the gray molasses cooling on the $D_1$ line. It has been shown by several authors~\cite{salomon2014gray,chen2016production,grier2013lambda,fernandes2012sub} that this cooling method assures highly efficient and fast cooling with negligible atom loss. In fact, it has been implemented already for all alkali atoms where sub-Doppler cooling mechanisms on the $D_2$ line do not work efficiently due to the small hyperfine structure energy splitting of the $P_{3/2}$ state. Gray molasses require good control over the background magnetic field and it is necessary to cancel all external stray magnetic fields for the best performance. We have nullified stray fields using a method we have developed and illustrated with $^{39}$K~\cite{dobosz2021bidirectional} which enabled us to cool that isotope to $\sim$8~\textmu K. Gray molasses cooling starts immediately after the $D_2$-line compressed MOT. The magnetic field and the $D_2$ light are turned off and the compensation of the stray magnetic fields is engaged. We block the light from the master laser (D2 ML) with a shutter, simultaneously opening a shutter that sends the $D_1$ line to tapered amplifiers. Due to the opening times of the shutters and their jitter we turn on the $D_1$ line light when the shutter is fully open. These technical restrictions introduce a delay of 0.8~ms between the beginning of the $D_1$ gray molasses stage and the turn off of the magnetic field. During that time the cloud drops freely and gets diluted due to its rather high temperature, but we are able to achieve nearly 100\% transfer of atoms to gray molasses with light intensities of 8.7$I_\mathrm{s}$ and 2.2$I_\mathrm{s}$ for the cooling and the repumping beam, respectively. We proceed with optimization of single- and two-photon detuning [see figures~\ref{fig:GMCD2}c) and~\ref{fig:GMCD2}d)] and the cooling time. We have found that during a 10 ms cooling stage during which we simultaneously ramp down the cooling (repumping) beam intensity from 8.7$I_\mathrm{s}$ to 2.9$I_\mathrm{s}$ (2.2$I_\mathrm{s}$ to 0.8$I_\mathrm{s}$) we are able to decrease the temperature to 10~\textmu K, on par with temperatures reported by other groups~\cite{fernandes2012sub,sievers2015simultaneous} while losing less than 20\% of atoms. With nearly $3\times 10^5$ atoms in a mixture of states we obtain free--space phase space density $\rho = 3.5\times 10^{-5}$. For the determination of the final temperature we have used a time-of-flight method using both fluorescence and absorption images (examples of both types of images taken at different expansion times are shown in figure~\ref{fig:abs_fluo}). Both methods have shown a very good agreement, giving essentially the same temperature of 10~\textmu K. Here, the shortest expansion time after release from gray molasses is 4.3~ms, limited by the 14~ms repetition time of mechanical shutters S1 and S2 (see figure~\ref{fig:laser_system_simplified_K}), whereas the longest expansion is 9~ms, limited by our ability to reliably image diluted samples.

% Figure environment removed

\section{Magnetic trapping\label{sec:magn_trap}}

High laser cooling efficiency of the sample enabled us to transfer atoms to a conservative magnetic potential. A magnetic trap has a large capture volume and enables sympathetic cooling of $^{40}$K with $^{23}$Na,$^{41}$K or $^{87}$Rb~\cite{Park_NaK_fermi_mix,aubin2006rapid,wu2011strongly}. In such a cooling scheme, the final temperature of fermions is set by the ability to evaporate the coolant (e.g. $^{23}$Na, $^{41}$K or $^{87}$Rb) and one can enter the degenerate regime without notable atom loss of $^{40}$K.

To transfer atoms from gray molasses to the magnetic trap we perform hyperfine pumping of the atomic population to the \mbox{$^2S_{1/2}, F=7/2$} hyperfine state by switching off the repumping light, then after 500~\textmu s switching off the cooling light and turning on the quadrupole field. We use the  highest gradient we can safely sustain in our setup, $B'_{\mathrm{r}}=57.6$~G/cm (note that the horizontal gradient is nearly $B'_{\mathrm{axial}}=115$~G/cm) which is sufficiently high to capture atoms distributed between $m_\mathrm{F}=-3/2,-5/2,-7/2$ states. If after hyperfine pumping the populations was eqaully distributed between Zeeman sublevels we would expect a 37.5\% transfer efficiency - we reach just under 50\% of that value. 

We have investigated the same two cases we have considered for the MOT loading: dispensers turned on to maximize the trapped atom number and dispensers turned off for a day and loading enhancement with LIAD. For the atom number maximized case we are able to magnetically trap about $5\times10^4$ atoms, with a lifetime of 0.6~s. This lifetime can be extended to over 2.8~s after keeping the dispensers off for a day and enhancing the loading rate with LIAD. This, however, results in a much lower number of trapped atoms, just above 3,000. The lifetime measurements are shown in figure~\ref{fig:lifetime}.
% Figure environment removed
We have not implemented spin polarization as we have focused on a proof-of-principle demonstration of magnetic trapping but there is nothing fundamental nor technical that would prevent us from trapping a little over $10^5$ spin polarized atoms in our current setup under the condition of reduced lifetime. It is worth emphasizing that that the reported lifetime is a lower bound on what can be achieved in our setup as both the sample's spin polarization and/or trapping atoms in the $F=9/2$ hyperfine state would lead to the increase of the storage time.  %of spin polarization of the sample have been implemented, and after the gray molasses process the cloud is slightly displaced from the center of magnetic field. 
 
We were not able to reliably measure the temperature of the magnetically trapped cloud as it got diluted already after a short expansion time. This might indicate that the transfer from gray molasses introduced some heating, which would not be surprising given the nature of optical pumping and the fact that gray molasses are slightly offset from the minimum of the magnetic potential. We believe, however, that it is primarily due to the small atom number that is approaching the detection sensitivity of our imaging system. For the lifetime measurement in the magnetic trap the small atom number is less of an issue. We release atoms after a given hold time and re-trap them in the MOT, where the fluorescence signal is collected for 20 ms. The imaging time has been chosen such that loading of the MOT from the background gas during imaging is negligible.




\section{Summary}

We have presented sub-Doppler cooling of $^{40}$K using a source with natural composition of potassium isotopes and a single chamber apparatus. With $3 \times 10^5$ atoms at a temperature of 10~\textmu K after $D_1$ gray molasses cooling stage we demonstrate that even without isotopically enriched sources it is possible to achieve state-of-the art cloud parameters in a single chamber setup. Our results open doors to using unenriched fermionic potassium in modern experiments including quantum computing with atoms in optical tweezers and sympathetic cooling of magnetically trapped $^{40}$K with another species. The observed deterioration of the vacuum quality when atomic sources are operational can be a serious problem in many experiments involving evaporation. For proof-of-principle works in optical tweezers, including formation of $^{40}$KCs ground state molecules that we are pursuing, the vacuum quality might be a secondary issue. Optical or magnetic transport can be later used to move the sample to another chamber with much better vacuum as has been demonstrated both for atoms~\cite{greiner_magnetic_transport,Unnikrishnan_transport} and for molecules~\cite{Bao_2022}, where the cloud was moved by 46~cm in just 50~ms.

Our work might encourage precision spectroscopic measurements of stable potassium isotopes~\cite{trans_freq_K_isotopes} in simplified experimental setups (no enriched sources, no Zeeman slowers, no 2D MOTs etc.) while still providing samples at $\sim$10~\textmu K thus significantly reducing the Doppler effect. Our results on sub-Doppler cooling of small numbers of atoms might be useful to $\beta$-decay experiments, where small numbers of laser cooled $^{37}$K and $^{38m}$K isotopes have been used~\cite{behr_rad_Kisotopes}. Here, it is not only a matter of the reduction of the Doppler effect but, we believe, the demonstrated gain in phase space density that might improve the quality of measurements. 
%Another somehow speculative, yet feasible direction that might be enabled by our work is the production of degenerate Fermi gases using magneto-optical cooling.   

%The last result shown in this paper was the magnetic trapping of potassium atoms, which indicates two important aspects. First, the possibility of simultaneous trapping of $^{40}$K and $^{41}$K atoms (also present in the natural potassium mixture) in a magnetic trap would allow for efficient further cooling of the fermionic isotope with the sympathetic cooling method. Furthermore, despite the fact that the entire cooling process took place in a single-chamber system, it was still possible to obtain lifetimes on the order of 3s, which should be sufficient to perform photoassociation or experiments with optical tweezers.


%\acknowledgments 
We would like to acknowledge P. Arciszewski and J. Dobosz for their contribution to the development of the experimental setup used in this work. This research was funded by the Foundation for Polish Science within the HOMING programme and the National Science Centre of Poland (grant No. 2016/21/D/ST2/02003 and a postdoctoral fellowship for M.S., grant No. DEC-2015/16/S/ST2/00425).


%\appendix
\section*{References}
%\nocite{*}
%\bibliography{sub-dopp_cool}% Produces the bibliography via BibTeX.
%merlin.mbs aipnum4-1.bst 2010-07-25 4.21a (PWD, AO, DPC) hacked
%Control: key (0)
%Control: author (8) initials jnrlst
%Control: editor formatted (1) identically to author
%Control: production of article title (0) allowed
%Control: page (1) range
%Control: year (1) truncated
%Control: production of eprint (0) enabled
%merlin.mbs aipnum4-1.bst 2010-07-25 4.21a (PWD, AO, DPC) hacked
%Control: key (0)
%Control: author (8) initials jnrlst
%Control: editor formatted (1) identically to author
%Control: production of article title (0) allowed
%Control: page (1) range
%Control: year (1) truncated
%Control: production of eprint (0) enabled
\begin{thebibliography}{30}%
\makeatletter
\providecommand \@ifxundefined [1]{%
 \@ifx{#1\undefined}
}%
\providecommand \@ifnum [1]{%
 \ifnum #1\expandafter \@firstoftwo
 \else \expandafter \@secondoftwo
 \fi
}%
\providecommand \@ifx [1]{%
 \ifx #1\expandafter \@firstoftwo
 \else \expandafter \@secondoftwo
 \fi
}%
\providecommand \natexlab [1]{#1}%
\providecommand \enquote  [1]{``#1''}%
\providecommand \bibnamefont  [1]{#1}%
\providecommand \bibfnamefont [1]{#1}%
\providecommand \citenamefont [1]{#1}%
\providecommand \href@noop [0]{\@secondoftwo}%
\providecommand \href [0]{\begingroup \@sanitize@url \@href}%
\providecommand \@href[1]{\@@startlink{#1}\@@href}%
\providecommand \@@href[1]{\endgroup#1\@@endlink}%
\providecommand \@sanitize@url [0]{\catcode `\\12\catcode `\$12\catcode
  `\&12\catcode `\#12\catcode `\^12\catcode `\_12\catcode `\%12\relax}%
\providecommand \@@startlink[1]{}%
\providecommand \@@endlink[0]{}%
\providecommand \url  [0]{\begingroup\@sanitize@url \@url }%
\providecommand \@url [1]{\endgroup\@href {#1}{\urlprefix }}%
\providecommand \urlprefix  [0]{URL }%
\providecommand \Eprint [0]{\href }%
\providecommand \doibase [0]{http://dx.doi.org/}%
\providecommand \selectlanguage [0]{\@gobble}%
\providecommand \bibinfo  [0]{\@secondoftwo}%
\providecommand \bibfield  [0]{\@secondoftwo}%
\providecommand \translation [1]{[#1]}%
\providecommand \BibitemOpen [0]{}%
\providecommand \bibitemStop [0]{}%
\providecommand \bibitemNoStop [0]{.\EOS\space}%
\providecommand \EOS [0]{\spacefactor3000\relax}%
\providecommand \BibitemShut  [1]{\csname bibitem#1\endcsname}%
\let\auto@bib@innerbib\@empty
%</preamble>
\bibitem [{\citenamefont {DeMarco}\ and\ \citenamefont
  {Jin}(1999)}]{demarco_DFG}%
  \BibitemOpen
  \bibfield  {author} {\bibinfo {author} {\bibfnamefont {B.}~\bibnamefont
  {DeMarco}}\ and\ \bibinfo {author} {\bibfnamefont {D.~S.}\ \bibnamefont
  {Jin}},\ }\bibfield  {title} {\enquote {\bibinfo {title} {Onset of fermi
  degeneracy in a trapped atomic gas},}\ }\href {\doibase
  10.1126/science.285.5434.1703} {\bibfield  {journal} {\bibinfo  {journal}
  {Science}\ }\textbf {\bibinfo {volume} {285}},\ \bibinfo {pages} {1703--1706}
  (\bibinfo {year} {1999})},\ \Eprint
  {http://arxiv.org/abs/https://www.science.org/doi/pdf/10.1126/science.285.5434.1703}
  {https://www.science.org/doi/pdf/10.1126/science.285.5434.1703} \BibitemShut
  {NoStop}%
\bibitem [{\citenamefont {Anderson}\ \emph {et~al.}(1995)\citenamefont
  {Anderson}, \citenamefont {Ensher}, \citenamefont {Matthews}, \citenamefont
  {Wieman},\ and\ \citenamefont {Cornell}}]{first_rb_bec}%
  \BibitemOpen
  \bibfield  {author} {\bibinfo {author} {\bibfnamefont {M.~H.}\ \bibnamefont
  {Anderson}}, \bibinfo {author} {\bibfnamefont {J.~R.}\ \bibnamefont
  {Ensher}}, \bibinfo {author} {\bibfnamefont {M.~R.}\ \bibnamefont
  {Matthews}}, \bibinfo {author} {\bibfnamefont {C.~E.}\ \bibnamefont
  {Wieman}}, \ and\ \bibinfo {author} {\bibfnamefont {E.~A.}\ \bibnamefont
  {Cornell}},\ }\bibfield  {title} {\enquote {\bibinfo {title} {Observation of
  bose-einstein condensation in a dilute atomic vapor},}\ }\href {\doibase
  10.1126/science.269.5221.198} {\bibfield  {journal} {\bibinfo  {journal}
  {Science}\ }\textbf {\bibinfo {volume} {269}},\ \bibinfo {pages} {198--201}
  (\bibinfo {year} {1995})},\ \Eprint
  {http://arxiv.org/abs/https://www.science.org/doi/pdf/10.1126/science.269.5221.198}
  {https://www.science.org/doi/pdf/10.1126/science.269.5221.198} \BibitemShut
  {NoStop}%
\bibitem [{\citenamefont {Jeltes}\ \emph {et~al.}(2007)\citenamefont {Jeltes},
  \citenamefont {McNamara}, \citenamefont {Hogervorst}, \citenamefont {Vassen},
  \citenamefont {Krachmalnicoff}, \citenamefont {Schellekens}, \citenamefont
  {Perrin}, \citenamefont {Chang}, \citenamefont {Boiron}, \citenamefont
  {Aspect},\ and\ \citenamefont {Westbrook}}]{Jeltes2007}%
  \BibitemOpen
  \bibfield  {author} {\bibinfo {author} {\bibfnamefont {T.}~\bibnamefont
  {Jeltes}}, \bibinfo {author} {\bibfnamefont {J.~M.}\ \bibnamefont
  {McNamara}}, \bibinfo {author} {\bibfnamefont {W.}~\bibnamefont
  {Hogervorst}}, \bibinfo {author} {\bibfnamefont {W.}~\bibnamefont {Vassen}},
  \bibinfo {author} {\bibfnamefont {V.}~\bibnamefont {Krachmalnicoff}},
  \bibinfo {author} {\bibfnamefont {M.}~\bibnamefont {Schellekens}}, \bibinfo
  {author} {\bibfnamefont {A.}~\bibnamefont {Perrin}}, \bibinfo {author}
  {\bibfnamefont {H.}~\bibnamefont {Chang}}, \bibinfo {author} {\bibfnamefont
  {D.}~\bibnamefont {Boiron}}, \bibinfo {author} {\bibfnamefont
  {A.}~\bibnamefont {Aspect}}, \ and\ \bibinfo {author} {\bibfnamefont {C.~I.}\
  \bibnamefont {Westbrook}},\ }\bibfield  {title} {\enquote {\bibinfo {title}
  {Comparison of the hanbury brown--twiss effect for bosons and fermions},}\
  }\href {\doibase 10.1038/nature05513} {\bibfield  {journal} {\bibinfo
  {journal} {Nature}\ }\textbf {\bibinfo {volume} {445}},\ \bibinfo {pages}
  {402--405} (\bibinfo {year} {2007})}\BibitemShut {NoStop}%
\bibitem [{\citenamefont {Cherfan}\ \emph {et~al.}(2021)\citenamefont
  {Cherfan}, \citenamefont {Denis}, \citenamefont {Bacquet}, \citenamefont
  {Gamot}, \citenamefont {Zemmouri}, \citenamefont {Manai}, \citenamefont
  {Clément}, \citenamefont {Garreau}, \citenamefont {Szriftgiser},\ and\
  \citenamefont {Chicireanu}}]{telecom_potassium_laser}%
  \BibitemOpen
  \bibfield  {author} {\bibinfo {author} {\bibfnamefont {C.}~\bibnamefont
  {Cherfan}}, \bibinfo {author} {\bibfnamefont {M.}~\bibnamefont {Denis}},
  \bibinfo {author} {\bibfnamefont {D.}~\bibnamefont {Bacquet}}, \bibinfo
  {author} {\bibfnamefont {M.}~\bibnamefont {Gamot}}, \bibinfo {author}
  {\bibfnamefont {S.}~\bibnamefont {Zemmouri}}, \bibinfo {author}
  {\bibfnamefont {I.}~\bibnamefont {Manai}}, \bibinfo {author} {\bibfnamefont
  {J.-F.}\ \bibnamefont {Clément}}, \bibinfo {author} {\bibfnamefont {J.-C.}\
  \bibnamefont {Garreau}}, \bibinfo {author} {\bibfnamefont {P.}~\bibnamefont
  {Szriftgiser}}, \ and\ \bibinfo {author} {\bibfnamefont {R.}~\bibnamefont
  {Chicireanu}},\ }\bibfield  {title} {\enquote {\bibinfo {title}
  {Multi-frequency telecom fibered laser system for potassium laser cooling},}\
  }\href {\doibase 10.1063/5.0070646} {\bibfield  {journal} {\bibinfo
  {journal} {Appl. Phys. Lett.}\ }\textbf {\bibinfo {volume} {119}},\ \bibinfo
  {pages} {204001} (\bibinfo {year} {2021})},\ \Eprint
  {http://arxiv.org/abs/https://pubs.aip.org/aip/apl/article-pdf/doi/10.1063/5.0070646/14556051/204001\_1\_online.pdf}
  {https://pubs.aip.org/aip/apl/article-pdf/doi/10.1063/5.0070646/14556051/204001\_1\_online.pdf}
  \BibitemShut {NoStop}%
\bibitem [{\citenamefont {Nyman}\ \emph {et~al.}(2006)\citenamefont {Nyman},
  \citenamefont {Varoquaux}, \citenamefont {Villier}, \citenamefont {Sacchet},
  \citenamefont {Moron}, \citenamefont {Le~Coq}, \citenamefont {Aspect},\ and\
  \citenamefont {Bouyer}}]{potassium_TA_example}%
  \BibitemOpen
  \bibfield  {author} {\bibinfo {author} {\bibfnamefont {R.~A.}\ \bibnamefont
  {Nyman}}, \bibinfo {author} {\bibfnamefont {G.}~\bibnamefont {Varoquaux}},
  \bibinfo {author} {\bibfnamefont {B.}~\bibnamefont {Villier}}, \bibinfo
  {author} {\bibfnamefont {D.}~\bibnamefont {Sacchet}}, \bibinfo {author}
  {\bibfnamefont {F.}~\bibnamefont {Moron}}, \bibinfo {author} {\bibfnamefont
  {Y.}~\bibnamefont {Le~Coq}}, \bibinfo {author} {\bibfnamefont
  {A.}~\bibnamefont {Aspect}}, \ and\ \bibinfo {author} {\bibfnamefont
  {P.}~\bibnamefont {Bouyer}},\ }\bibfield  {title} {\enquote {\bibinfo {title}
  {Tapered-amplified antireflection-coated laser diodes for potassium and
  rubidium atomic-physics experiments},}\ }\href {\doibase 10.1063/1.2186809}
  {\bibfield  {journal} {\bibinfo  {journal} {Rev. Sci. Instrum.}\ }\textbf
  {\bibinfo {volume} {77}},\ \bibinfo {pages} {033105} (\bibinfo {year}
  {2006})},\ \Eprint
  {http://arxiv.org/abs/https://pubs.aip.org/aip/rsi/article-pdf/doi/10.1063/1.2186809/14098285/033105\_1\_online.pdf}
  {https://pubs.aip.org/aip/rsi/article-pdf/doi/10.1063/1.2186809/14098285/033105\_1\_online.pdf}
  \BibitemShut {NoStop}%
\bibitem [{\citenamefont {{Williamson III}}\ \emph {et~al.}(1998)\citenamefont
  {{Williamson III}}, \citenamefont {Voytas}, \citenamefont {Newell},\ and\
  \citenamefont {Walker}}]{WilliamsonIII:98}%
  \BibitemOpen
  \bibfield  {author} {\bibinfo {author} {\bibfnamefont {R.~S.}\ \bibnamefont
  {{Williamson III}}}, \bibinfo {author} {\bibfnamefont {P.~A.}\ \bibnamefont
  {Voytas}}, \bibinfo {author} {\bibfnamefont {R.~T.}\ \bibnamefont {Newell}},
  \ and\ \bibinfo {author} {\bibfnamefont {T.}~\bibnamefont {Walker}},\
  }\bibfield  {title} {\enquote {\bibinfo {title} {A magneto-optical trap
  loaded from a pyramidal funnel},}\ }\href {\doibase 10.1364/OE.3.000111}
  {\bibfield  {journal} {\bibinfo  {journal} {Opt. Express}\ }\textbf {\bibinfo
  {volume} {3}},\ \bibinfo {pages} {111--117} (\bibinfo {year}
  {1998})}\BibitemShut {NoStop}%
\bibitem [{\citenamefont {Cataliotti}\ \emph {et~al.}(1998)\citenamefont
  {Cataliotti}, \citenamefont {Cornell}, \citenamefont {Fort}, \citenamefont
  {Inguscio}, \citenamefont {Marin}, \citenamefont {Prevedelli}, \citenamefont
  {Ricci},\ and\ \citenamefont {Tino}}]{cataliotti1998magneto}%
  \BibitemOpen
  \bibfield  {author} {\bibinfo {author} {\bibfnamefont {F.}~\bibnamefont
  {Cataliotti}}, \bibinfo {author} {\bibfnamefont {E.~A.}\ \bibnamefont
  {Cornell}}, \bibinfo {author} {\bibfnamefont {C.}~\bibnamefont {Fort}},
  \bibinfo {author} {\bibfnamefont {M.}~\bibnamefont {Inguscio}}, \bibinfo
  {author} {\bibfnamefont {F.}~\bibnamefont {Marin}}, \bibinfo {author}
  {\bibfnamefont {M.}~\bibnamefont {Prevedelli}}, \bibinfo {author}
  {\bibfnamefont {L.}~\bibnamefont {Ricci}}, \ and\ \bibinfo {author}
  {\bibfnamefont {G.}~\bibnamefont {Tino}},\ }\bibfield  {title} {\enquote
  {\bibinfo {title} {Magneto-optical trapping of fermionic potassium atoms},}\
  }\href@noop {} {\bibfield  {journal} {\bibinfo  {journal} {Phys. Rev. A}\
  }\textbf {\bibinfo {volume} {57}},\ \bibinfo {pages} {1136} (\bibinfo {year}
  {1998})}\BibitemShut {NoStop}%
\bibitem [{\citenamefont {DeMarco}, \citenamefont {Rohner},\ and\ \citenamefont
  {Jin}(1999)}]{40K_enriched_demarco}%
  \BibitemOpen
  \bibfield  {author} {\bibinfo {author} {\bibfnamefont {B.}~\bibnamefont
  {DeMarco}}, \bibinfo {author} {\bibfnamefont {H.}~\bibnamefont {Rohner}}, \
  and\ \bibinfo {author} {\bibfnamefont {D.~S.}\ \bibnamefont {Jin}},\
  }\bibfield  {title} {\enquote {\bibinfo {title} {An enriched $^{40}$k source
  for fermionic atom studies},}\ }\href {\doibase 10.1063/1.1149695} {\bibfield
   {journal} {\bibinfo  {journal} {Rev. Sci. Instrum.}\ }\textbf {\bibinfo
  {volume} {70}},\ \bibinfo {pages} {1967--1969} (\bibinfo {year} {1999})},\
  \Eprint
  {http://arxiv.org/abs/https://pubs.aip.org/aip/rsi/article-pdf/70/4/1967/11359111/1967\_1\_online.pdf}
  {https://pubs.aip.org/aip/rsi/article-pdf/70/4/1967/11359111/1967\_1\_online.pdf}
  \BibitemShut {NoStop}%
\bibitem [{\citenamefont {Modugno}\ \emph {et~al.}(1999)\citenamefont
  {Modugno}, \citenamefont {Benk\H{o}}, \citenamefont {Hannaford},
  \citenamefont {Roati},\ and\ \citenamefont
  {Inguscio}}]{40K_subdopp_cool_modugno}%
  \BibitemOpen
  \bibfield  {author} {\bibinfo {author} {\bibfnamefont {G.}~\bibnamefont
  {Modugno}}, \bibinfo {author} {\bibfnamefont {C.}~\bibnamefont {Benk\H{o}}},
  \bibinfo {author} {\bibfnamefont {P.}~\bibnamefont {Hannaford}}, \bibinfo
  {author} {\bibfnamefont {G.}~\bibnamefont {Roati}}, \ and\ \bibinfo {author}
  {\bibfnamefont {M.}~\bibnamefont {Inguscio}},\ }\bibfield  {title} {\enquote
  {\bibinfo {title} {Sub-doppler laser cooling of fermionic ${}^{40}\mathrm{K}$
  atoms},}\ }\href {\doibase 10.1103/PhysRevA.60.R3373} {\bibfield  {journal}
  {\bibinfo  {journal} {Phys. Rev. A}\ }\textbf {\bibinfo {volume} {60}},\
  \bibinfo {pages} {R3373--R3376} (\bibinfo {year} {1999})}\BibitemShut
  {NoStop}%
\bibitem [{\citenamefont {Gokhroo}\ \emph {et~al.}(2011)\citenamefont
  {Gokhroo}, \citenamefont {Rajalakshmi}, \citenamefont {Easwaran},\ and\
  \citenamefont {Unnikrishnan}}]{gokhroo2011sub}%
  \BibitemOpen
  \bibfield  {author} {\bibinfo {author} {\bibfnamefont {V.}~\bibnamefont
  {Gokhroo}}, \bibinfo {author} {\bibfnamefont {G.}~\bibnamefont
  {Rajalakshmi}}, \bibinfo {author} {\bibfnamefont {R.~K.}\ \bibnamefont
  {Easwaran}}, \ and\ \bibinfo {author} {\bibfnamefont {C.}~\bibnamefont
  {Unnikrishnan}},\ }\bibfield  {title} {\enquote {\bibinfo {title}
  {Sub-doppler deep-cooled bosonic and fermionic isotopes of potassium in a
  compact 2d$^+$--3d mot set-up},}\ }\href@noop {} {\bibfield  {journal}
  {\bibinfo  {journal} {J. Phys. B: At. Mol. Opt. Phys.}\ }\textbf {\bibinfo
  {volume} {44}},\ \bibinfo {pages} {115307} (\bibinfo {year}
  {2011})}\BibitemShut {NoStop}%
\bibitem [{\citenamefont {Gunton}, \citenamefont {Semczuk},\ and\ \citenamefont
  {Madison}(2013)}]{gunton_6Li2_bec}%
  \BibitemOpen
  \bibfield  {author} {\bibinfo {author} {\bibfnamefont {W.}~\bibnamefont
  {Gunton}}, \bibinfo {author} {\bibfnamefont {M.}~\bibnamefont {Semczuk}}, \
  and\ \bibinfo {author} {\bibfnamefont {K.~W.}\ \bibnamefont {Madison}},\
  }\bibfield  {title} {\enquote {\bibinfo {title} {Realization of
  bec-bcs-crossover physics in a compact oven-loaded magneto-optic-trap
  apparatus},}\ }\href {\doibase 10.1103/PhysRevA.88.023624} {\bibfield
  {journal} {\bibinfo  {journal} {Phys. Rev. A}\ }\textbf {\bibinfo {volume}
  {88}},\ \bibinfo {pages} {023624} (\bibinfo {year} {2013})}\BibitemShut
  {NoStop}%
\bibitem [{\citenamefont {Wu}\ \emph {et~al.}(2011)\citenamefont {Wu},
  \citenamefont {Santiago}, \citenamefont {Park}, \citenamefont {Ahmadi},\ and\
  \citenamefont {Zwierlein}}]{wu2011strongly}%
  \BibitemOpen
  \bibfield  {author} {\bibinfo {author} {\bibfnamefont {C.-H.}\ \bibnamefont
  {Wu}}, \bibinfo {author} {\bibfnamefont {I.}~\bibnamefont {Santiago}},
  \bibinfo {author} {\bibfnamefont {J.~W.}\ \bibnamefont {Park}}, \bibinfo
  {author} {\bibfnamefont {P.}~\bibnamefont {Ahmadi}}, \ and\ \bibinfo {author}
  {\bibfnamefont {M.~W.}\ \bibnamefont {Zwierlein}},\ }\bibfield  {title}
  {\enquote {\bibinfo {title} {Strongly interacting isotopic bose-fermi mixture
  immersed in a fermi sea},}\ }\href@noop {} {\bibfield  {journal} {\bibinfo
  {journal} {Phys. Rev. A}\ }\textbf {\bibinfo {volume} {84}},\ \bibinfo
  {pages} {011601} (\bibinfo {year} {2011})}\BibitemShut {NoStop}%
\bibitem [{\citenamefont {Yang}\ \emph {et~al.}(2018)\citenamefont {Yang},
  \citenamefont {Long}, \citenamefont {Gao}, \citenamefont {Jin}, \citenamefont
  {Zuo},\ and\ \citenamefont {Wang}}]{Yang_2018}%
  \BibitemOpen
  \bibfield  {author} {\bibinfo {author} {\bibfnamefont {J.-L.}\ \bibnamefont
  {Yang}}, \bibinfo {author} {\bibfnamefont {Y.}~\bibnamefont {Long}}, \bibinfo
  {author} {\bibfnamefont {W.-W.}\ \bibnamefont {Gao}}, \bibinfo {author}
  {\bibfnamefont {L.}~\bibnamefont {Jin}}, \bibinfo {author} {\bibfnamefont
  {Z.-C.}\ \bibnamefont {Zuo}}, \ and\ \bibinfo {author} {\bibfnamefont
  {R.-Q.}\ \bibnamefont {Wang}},\ }\bibfield  {title} {\enquote {\bibinfo
  {title} {Enhanced loading of $^{40}$k from natural abundance potassium source
  with a high performance 2d$^+$ mot},}\ }\href {\doibase
  10.1088/0256-307X/35/3/033701} {\bibfield  {journal} {\bibinfo  {journal}
  {Chin. Phys. Lett.}\ }\textbf {\bibinfo {volume} {35}},\ \bibinfo {pages}
  {033701} (\bibinfo {year} {2018})}\BibitemShut {NoStop}%
\bibitem [{\citenamefont {Klempt}\ \emph {et~al.}(2006)\citenamefont {Klempt},
  \citenamefont {van Zoest}, \citenamefont {Henninger}, \citenamefont {Topic},
  \citenamefont {Rasel}, \citenamefont {Ertmer},\ and\ \citenamefont
  {Arlt}}]{LIAD_K_Rb}%
  \BibitemOpen
  \bibfield  {author} {\bibinfo {author} {\bibfnamefont {C.}~\bibnamefont
  {Klempt}}, \bibinfo {author} {\bibfnamefont {T.}~\bibnamefont {van Zoest}},
  \bibinfo {author} {\bibfnamefont {T.}~\bibnamefont {Henninger}}, \bibinfo
  {author} {\bibfnamefont {O.}~\bibnamefont {Topic}}, \bibinfo {author}
  {\bibfnamefont {E.}~\bibnamefont {Rasel}}, \bibinfo {author} {\bibfnamefont
  {W.}~\bibnamefont {Ertmer}}, \ and\ \bibinfo {author} {\bibfnamefont
  {J.}~\bibnamefont {Arlt}},\ }\bibfield  {title} {\enquote {\bibinfo {title}
  {Ultraviolet light-induced atom desorption for large rubidium and potassium
  magneto-optical traps},}\ }\href {\doibase 10.1103/PhysRevA.73.013410}
  {\bibfield  {journal} {\bibinfo  {journal} {Phys. Rev. A}\ }\textbf {\bibinfo
  {volume} {73}},\ \bibinfo {pages} {013410} (\bibinfo {year}
  {2006})}\BibitemShut {NoStop}%
\bibitem [{\citenamefont {Salomon}\ \emph {et~al.}(2014)\citenamefont
  {Salomon}, \citenamefont {Fouch{\'e}}, \citenamefont {Wang}, \citenamefont
  {Aspect}, \citenamefont {Bouyer},\ and\ \citenamefont
  {Bourdel}}]{salomon2014gray}%
  \BibitemOpen
  \bibfield  {author} {\bibinfo {author} {\bibfnamefont {G.}~\bibnamefont
  {Salomon}}, \bibinfo {author} {\bibfnamefont {L.}~\bibnamefont {Fouch{\'e}}},
  \bibinfo {author} {\bibfnamefont {P.}~\bibnamefont {Wang}}, \bibinfo {author}
  {\bibfnamefont {A.}~\bibnamefont {Aspect}}, \bibinfo {author} {\bibfnamefont
  {P.}~\bibnamefont {Bouyer}}, \ and\ \bibinfo {author} {\bibfnamefont
  {T.}~\bibnamefont {Bourdel}},\ }\bibfield  {title} {\enquote {\bibinfo
  {title} {Gray-molasses cooling of $^{39}$k to a high phase-space density},}\
  }\href@noop {} {\bibfield  {journal} {\bibinfo  {journal} {EPL}\ }\textbf
  {\bibinfo {volume} {104}},\ \bibinfo {pages} {63002} (\bibinfo {year}
  {2014})}\BibitemShut {NoStop}%
\bibitem [{\citenamefont {Chen}\ \emph {et~al.}(2016)\citenamefont {Chen},
  \citenamefont {Yao}, \citenamefont {Wu}, \citenamefont {Liu}, \citenamefont
  {Wang}, \citenamefont {Wang}, \citenamefont {Chen},\ and\ \citenamefont
  {Pan}}]{chen2016production}%
  \BibitemOpen
  \bibfield  {author} {\bibinfo {author} {\bibfnamefont {H.-Z.}\ \bibnamefont
  {Chen}}, \bibinfo {author} {\bibfnamefont {X.-C.}\ \bibnamefont {Yao}},
  \bibinfo {author} {\bibfnamefont {Y.-P.}\ \bibnamefont {Wu}}, \bibinfo
  {author} {\bibfnamefont {X.-P.}\ \bibnamefont {Liu}}, \bibinfo {author}
  {\bibfnamefont {X.-Q.}\ \bibnamefont {Wang}}, \bibinfo {author}
  {\bibfnamefont {Y.-X.}\ \bibnamefont {Wang}}, \bibinfo {author}
  {\bibfnamefont {Y.-A.}\ \bibnamefont {Chen}}, \ and\ \bibinfo {author}
  {\bibfnamefont {J.-W.}\ \bibnamefont {Pan}},\ }\bibfield  {title} {\enquote
  {\bibinfo {title} {Production of large $^{41}$k bose-einstein condensates
  using $d_1$ gray molasses},}\ }\href@noop {} {\bibfield  {journal} {\bibinfo
  {journal} {Phys. Rev. A}\ }\textbf {\bibinfo {volume} {94}},\ \bibinfo
  {pages} {033408} (\bibinfo {year} {2016})}\BibitemShut {NoStop}%
\bibitem [{\citenamefont {Rosi}\ \emph {et~al.}(2018)\citenamefont {Rosi},
  \citenamefont {Burchianti}, \citenamefont {Conclave}, \citenamefont {Naik},
  \citenamefont {Roati}, \citenamefont {Fort},\ and\ \citenamefont
  {Minardi}}]{Rosi2018}%
  \BibitemOpen
  \bibfield  {author} {\bibinfo {author} {\bibfnamefont {S.}~\bibnamefont
  {Rosi}}, \bibinfo {author} {\bibfnamefont {A.}~\bibnamefont {Burchianti}},
  \bibinfo {author} {\bibfnamefont {S.}~\bibnamefont {Conclave}}, \bibinfo
  {author} {\bibfnamefont {D.~S.}\ \bibnamefont {Naik}}, \bibinfo {author}
  {\bibfnamefont {G.}~\bibnamefont {Roati}}, \bibinfo {author} {\bibfnamefont
  {C.}~\bibnamefont {Fort}}, \ and\ \bibinfo {author} {\bibfnamefont
  {F.}~\bibnamefont {Minardi}},\ }\bibfield  {title} {\enquote {\bibinfo
  {title} {$\lambda$-enhanced grey molasses on the $d_2$ transition of
  rubidium-87 atoms},}\ }\href {https://doi.org/10.1038/s41598-018-19814-z}
  {\bibfield  {journal} {\bibinfo  {journal} {Sci. Rep.}\ }\textbf {\bibinfo
  {volume} {8}},\ \bibinfo {pages} {1301} (\bibinfo {year} {2018})}\BibitemShut
  {NoStop}%
\bibitem [{\citenamefont {Tiecke}(2010)}]{tiecke2010properties}%
  \BibitemOpen
  \bibfield  {author} {\bibinfo {author} {\bibfnamefont {T.}~\bibnamefont
  {Tiecke}},\ }\bibfield  {title} {\enquote {\bibinfo {title} {Properties of
  potassium},}\ }\href@noop {} {\bibfield  {journal} {\bibinfo  {journal}
  {University of Amsterdam, The Netherlands, Thesis}\ ,\ \bibinfo {pages}
  {12--14}} (\bibinfo {year} {2010})}\BibitemShut {NoStop}%
\bibitem [{\citenamefont {Fernandes}\ \emph {et~al.}(2012)\citenamefont
  {Fernandes}, \citenamefont {Sievers}, \citenamefont {Kretzschmar},
  \citenamefont {Wu}, \citenamefont {Salomon},\ and\ \citenamefont
  {Chevy}}]{fernandes2012sub}%
  \BibitemOpen
  \bibfield  {author} {\bibinfo {author} {\bibfnamefont {D.~R.}\ \bibnamefont
  {Fernandes}}, \bibinfo {author} {\bibfnamefont {F.}~\bibnamefont {Sievers}},
  \bibinfo {author} {\bibfnamefont {N.}~\bibnamefont {Kretzschmar}}, \bibinfo
  {author} {\bibfnamefont {S.}~\bibnamefont {Wu}}, \bibinfo {author}
  {\bibfnamefont {C.}~\bibnamefont {Salomon}}, \ and\ \bibinfo {author}
  {\bibfnamefont {F.}~\bibnamefont {Chevy}},\ }\bibfield  {title} {\enquote
  {\bibinfo {title} {Sub-doppler laser cooling of fermionic $^{40}$k atoms in
  three-dimensional gray optical molasses},}\ }\href@noop {} {\bibfield
  {journal} {\bibinfo  {journal} {EPL}\ }\textbf {\bibinfo {volume} {100}},\
  \bibinfo {pages} {63001} (\bibinfo {year} {2012})}\BibitemShut {NoStop}%
\bibitem [{\citenamefont {Sievers}\ \emph {et~al.}(2015)\citenamefont
  {Sievers}, \citenamefont {Kretzschmar}, \citenamefont {Fernandes},
  \citenamefont {Suchet}, \citenamefont {Rabinovic}, \citenamefont {Wu},
  \citenamefont {Parker}, \citenamefont {Khaykovich}, \citenamefont {Salomon},\
  and\ \citenamefont {Chevy}}]{sievers2015simultaneous}%
  \BibitemOpen
  \bibfield  {author} {\bibinfo {author} {\bibfnamefont {F.}~\bibnamefont
  {Sievers}}, \bibinfo {author} {\bibfnamefont {N.}~\bibnamefont
  {Kretzschmar}}, \bibinfo {author} {\bibfnamefont {D.~R.}\ \bibnamefont
  {Fernandes}}, \bibinfo {author} {\bibfnamefont {D.}~\bibnamefont {Suchet}},
  \bibinfo {author} {\bibfnamefont {M.}~\bibnamefont {Rabinovic}}, \bibinfo
  {author} {\bibfnamefont {S.}~\bibnamefont {Wu}}, \bibinfo {author}
  {\bibfnamefont {C.~V.}\ \bibnamefont {Parker}}, \bibinfo {author}
  {\bibfnamefont {L.}~\bibnamefont {Khaykovich}}, \bibinfo {author}
  {\bibfnamefont {C.}~\bibnamefont {Salomon}}, \ and\ \bibinfo {author}
  {\bibfnamefont {F.}~\bibnamefont {Chevy}},\ }\bibfield  {title} {\enquote
  {\bibinfo {title} {Simultaneous sub-doppler laser cooling of fermionic $^6$li
  and $^{40}$k on the $d_1$ line: Theory and experiment},}\ }\href@noop {}
  {\bibfield  {journal} {\bibinfo  {journal} {Phys. Rev. A}\ }\textbf {\bibinfo
  {volume} {91}},\ \bibinfo {pages} {023426} (\bibinfo {year}
  {2015})}\BibitemShut {NoStop}%
\bibitem [{\citenamefont {Bruce}\ \emph {et~al.}(2017)\citenamefont {Bruce},
  \citenamefont {Haller}, \citenamefont {Peaudecerf}, \citenamefont {Cotta},
  \citenamefont {Andia}, \citenamefont {Wu}, \citenamefont {Johnson},
  \citenamefont {Lovett},\ and\ \citenamefont {Kuhr}}]{bruce2017sub}%
  \BibitemOpen
  \bibfield  {author} {\bibinfo {author} {\bibfnamefont {G.~D.}\ \bibnamefont
  {Bruce}}, \bibinfo {author} {\bibfnamefont {E.}~\bibnamefont {Haller}},
  \bibinfo {author} {\bibfnamefont {B.}~\bibnamefont {Peaudecerf}}, \bibinfo
  {author} {\bibfnamefont {D.~A.}\ \bibnamefont {Cotta}}, \bibinfo {author}
  {\bibfnamefont {M.}~\bibnamefont {Andia}}, \bibinfo {author} {\bibfnamefont
  {S.}~\bibnamefont {Wu}}, \bibinfo {author} {\bibfnamefont {M.~Y.}\
  \bibnamefont {Johnson}}, \bibinfo {author} {\bibfnamefont {B.~W.}\
  \bibnamefont {Lovett}}, \ and\ \bibinfo {author} {\bibfnamefont
  {S.}~\bibnamefont {Kuhr}},\ }\bibfield  {title} {\enquote {\bibinfo {title}
  {Sub-doppler laser cooling of $^{40}$k with raman gray molasses on the $d_2$
  line},}\ }\href@noop {} {\bibfield  {journal} {\bibinfo  {journal} {J. Phys.
  B: At. Mol. Opt. Phys.}\ }\textbf {\bibinfo {volume} {50}},\ \bibinfo {pages}
  {095002} (\bibinfo {year} {2017})}\BibitemShut {NoStop}%
\bibitem [{\citenamefont {Dobosz}, \citenamefont {Boche{\'n}ski},\ and\
  \citenamefont {Semczuk}(2021)}]{dobosz2021bidirectional}%
  \BibitemOpen
  \bibfield  {author} {\bibinfo {author} {\bibfnamefont {J.}~\bibnamefont
  {Dobosz}}, \bibinfo {author} {\bibfnamefont {M.}~\bibnamefont
  {Boche{\'n}ski}}, \ and\ \bibinfo {author} {\bibfnamefont {M.}~\bibnamefont
  {Semczuk}},\ }\bibfield  {title} {\enquote {\bibinfo {title} {Bidirectional,
  analog current source benchmarked with gray molasses-assisted stray magnetic
  field compensation},}\ }\href@noop {} {\bibfield  {journal} {\bibinfo
  {journal} {Appl. Sci.}\ }\textbf {\bibinfo {volume} {11}},\ \bibinfo {pages}
  {10474} (\bibinfo {year} {2021})}\BibitemShut {NoStop}%
\bibitem [{\citenamefont {Grier}\ \emph {et~al.}(2013)\citenamefont {Grier},
  \citenamefont {Ferrier-Barbut}, \citenamefont {Rem}, \citenamefont
  {Delehaye}, \citenamefont {Khaykovich}, \citenamefont {Chevy},\ and\
  \citenamefont {Salomon}}]{grier2013lambda}%
  \BibitemOpen
  \bibfield  {author} {\bibinfo {author} {\bibfnamefont {A.~T.}\ \bibnamefont
  {Grier}}, \bibinfo {author} {\bibfnamefont {I.}~\bibnamefont
  {Ferrier-Barbut}}, \bibinfo {author} {\bibfnamefont {B.~S.}\ \bibnamefont
  {Rem}}, \bibinfo {author} {\bibfnamefont {M.}~\bibnamefont {Delehaye}},
  \bibinfo {author} {\bibfnamefont {L.}~\bibnamefont {Khaykovich}}, \bibinfo
  {author} {\bibfnamefont {F.}~\bibnamefont {Chevy}}, \ and\ \bibinfo {author}
  {\bibfnamefont {C.}~\bibnamefont {Salomon}},\ }\bibfield  {title} {\enquote
  {\bibinfo {title} {$\lambda$-enhanced sub-doppler cooling of lithium atoms in
  $d_1$ gray molasses},}\ }\href@noop {} {\bibfield  {journal} {\bibinfo
  {journal} {Phys. Rev. A}\ }\textbf {\bibinfo {volume} {87}},\ \bibinfo
  {pages} {063411} (\bibinfo {year} {2013})}\BibitemShut {NoStop}%
\bibitem [{\citenamefont {Park}\ \emph {et~al.}(2012)\citenamefont {Park},
  \citenamefont {Wu}, \citenamefont {Santiago}, \citenamefont {Tiecke},
  \citenamefont {Will}, \citenamefont {Ahmadi},\ and\ \citenamefont
  {Zwierlein}}]{Park_NaK_fermi_mix}%
  \BibitemOpen
  \bibfield  {author} {\bibinfo {author} {\bibfnamefont {J.~W.}\ \bibnamefont
  {Park}}, \bibinfo {author} {\bibfnamefont {C.-H.}\ \bibnamefont {Wu}},
  \bibinfo {author} {\bibfnamefont {I.}~\bibnamefont {Santiago}}, \bibinfo
  {author} {\bibfnamefont {T.~G.}\ \bibnamefont {Tiecke}}, \bibinfo {author}
  {\bibfnamefont {S.}~\bibnamefont {Will}}, \bibinfo {author} {\bibfnamefont
  {P.}~\bibnamefont {Ahmadi}}, \ and\ \bibinfo {author} {\bibfnamefont {M.~W.}\
  \bibnamefont {Zwierlein}},\ }\bibfield  {title} {\enquote {\bibinfo {title}
  {Quantum degenerate bose-fermi mixture of chemically different atomic species
  with widely tunable interactions},}\ }\href {\doibase
  10.1103/PhysRevA.85.051602} {\bibfield  {journal} {\bibinfo  {journal} {Phys.
  Rev. A}\ }\textbf {\bibinfo {volume} {85}},\ \bibinfo {pages} {051602}
  (\bibinfo {year} {2012})}\BibitemShut {NoStop}%
\bibitem [{\citenamefont {Aubin}\ \emph {et~al.}(2006)\citenamefont {Aubin},
  \citenamefont {Myrskog}, \citenamefont {Extavour}, \citenamefont {LeBlanc},
  \citenamefont {McKay}, \citenamefont {Stummer},\ and\ \citenamefont
  {Thywissen}}]{aubin2006rapid}%
  \BibitemOpen
  \bibfield  {author} {\bibinfo {author} {\bibfnamefont {S.}~\bibnamefont
  {Aubin}}, \bibinfo {author} {\bibfnamefont {S.}~\bibnamefont {Myrskog}},
  \bibinfo {author} {\bibfnamefont {M.}~\bibnamefont {Extavour}}, \bibinfo
  {author} {\bibfnamefont {L.}~\bibnamefont {LeBlanc}}, \bibinfo {author}
  {\bibfnamefont {D.}~\bibnamefont {McKay}}, \bibinfo {author} {\bibfnamefont
  {A.}~\bibnamefont {Stummer}}, \ and\ \bibinfo {author} {\bibfnamefont
  {J.}~\bibnamefont {Thywissen}},\ }\bibfield  {title} {\enquote {\bibinfo
  {title} {Rapid sympathetic cooling to fermi degeneracy on a chip},}\
  }\href@noop {} {\bibfield  {journal} {\bibinfo  {journal} {Nat. Phys.}\
  }\textbf {\bibinfo {volume} {2}},\ \bibinfo {pages} {384--387} (\bibinfo
  {year} {2006})}\BibitemShut {NoStop}%
\bibitem [{\citenamefont {Greiner}\ \emph {et~al.}(2001)\citenamefont
  {Greiner}, \citenamefont {Bloch}, \citenamefont {H\"ansch},\ and\
  \citenamefont {Esslinger}}]{greiner_magnetic_transport}%
  \BibitemOpen
  \bibfield  {author} {\bibinfo {author} {\bibfnamefont {M.}~\bibnamefont
  {Greiner}}, \bibinfo {author} {\bibfnamefont {I.}~\bibnamefont {Bloch}},
  \bibinfo {author} {\bibfnamefont {T.~W.}\ \bibnamefont {H\"ansch}}, \ and\
  \bibinfo {author} {\bibfnamefont {T.}~\bibnamefont {Esslinger}},\ }\bibfield
  {title} {\enquote {\bibinfo {title} {Magnetic transport of trapped cold atoms
  over a large distance},}\ }\href {\doibase 10.1103/PhysRevA.63.031401}
  {\bibfield  {journal} {\bibinfo  {journal} {Phys. Rev. A}\ }\textbf {\bibinfo
  {volume} {63}},\ \bibinfo {pages} {031401} (\bibinfo {year}
  {2001})}\BibitemShut {NoStop}%
\bibitem [{\citenamefont {Unnikrishnan}\ \emph {et~al.}(2021)\citenamefont
  {Unnikrishnan}, \citenamefont {Beulenkamp}, \citenamefont {Zhang},
  \citenamefont {Zamarski}, \citenamefont {Landini},\ and\ \citenamefont
  {Nägerl}}]{Unnikrishnan_transport}%
  \BibitemOpen
  \bibfield  {author} {\bibinfo {author} {\bibfnamefont {G.}~\bibnamefont
  {Unnikrishnan}}, \bibinfo {author} {\bibfnamefont {C.}~\bibnamefont
  {Beulenkamp}}, \bibinfo {author} {\bibfnamefont {D.}~\bibnamefont {Zhang}},
  \bibinfo {author} {\bibfnamefont {K.~P.}\ \bibnamefont {Zamarski}}, \bibinfo
  {author} {\bibfnamefont {M.}~\bibnamefont {Landini}}, \ and\ \bibinfo
  {author} {\bibfnamefont {H.-C.}\ \bibnamefont {Nägerl}},\ }\bibfield
  {title} {\enquote {\bibinfo {title} {{Long distance optical transport of
  ultracold atoms: A compact setup using a Moiré lens}},}\ }\href {\doibase
  10.1063/5.0049320} {\bibfield  {journal} {\bibinfo  {journal} {Rev. Sci.
  Instrum.}\ }\textbf {\bibinfo {volume} {92}},\ \bibinfo {pages} {063205}
  (\bibinfo {year} {2021})},\ \Eprint
  {http://arxiv.org/abs/https://pubs.aip.org/aip/rsi/article-pdf/doi/10.1063/5.0049320/15922323/063205\_1\_online.pdf}
  {https://pubs.aip.org/aip/rsi/article-pdf/doi/10.1063/5.0049320/15922323/063205\_1\_online.pdf}
  \BibitemShut {NoStop}%
\bibitem [{\citenamefont {Bao}\ \emph {et~al.}(2022)\citenamefont {Bao},
  \citenamefont {Yu}, \citenamefont {Anderegg}, \citenamefont {Burchesky},
  \citenamefont {Gonzalez-Acevedo}, \citenamefont {Chae}, \citenamefont
  {Ketterle}, \citenamefont {Ni},\ and\ \citenamefont {Doyle}}]{Bao_2022}%
  \BibitemOpen
  \bibfield  {author} {\bibinfo {author} {\bibfnamefont {Y.}~\bibnamefont
  {Bao}}, \bibinfo {author} {\bibfnamefont {S.~S.}\ \bibnamefont {Yu}},
  \bibinfo {author} {\bibfnamefont {L.}~\bibnamefont {Anderegg}}, \bibinfo
  {author} {\bibfnamefont {S.}~\bibnamefont {Burchesky}}, \bibinfo {author}
  {\bibfnamefont {D.}~\bibnamefont {Gonzalez-Acevedo}}, \bibinfo {author}
  {\bibfnamefont {E.}~\bibnamefont {Chae}}, \bibinfo {author} {\bibfnamefont
  {W.}~\bibnamefont {Ketterle}}, \bibinfo {author} {\bibfnamefont {K.-K.}\
  \bibnamefont {Ni}}, \ and\ \bibinfo {author} {\bibfnamefont {J.~M.}\
  \bibnamefont {Doyle}},\ }\bibfield  {title} {\enquote {\bibinfo {title} {Fast
  optical transport of ultracold molecules over long distances},}\ }\href
  {\doibase 10.1088/1367-2630/ac900f} {\bibfield  {journal} {\bibinfo
  {journal} {New J. Phys.}\ }\textbf {\bibinfo {volume} {24}},\ \bibinfo
  {pages} {093028} (\bibinfo {year} {2022})}\BibitemShut {NoStop}%
\bibitem [{\citenamefont {Falke}\ \emph {et~al.}(2006)\citenamefont {Falke},
  \citenamefont {Tiemann}, \citenamefont {Lisdat}, \citenamefont {Schnatz},\
  and\ \citenamefont {Grosche}}]{trans_freq_K_isotopes}%
  \BibitemOpen
  \bibfield  {author} {\bibinfo {author} {\bibfnamefont {S.}~\bibnamefont
  {Falke}}, \bibinfo {author} {\bibfnamefont {E.}~\bibnamefont {Tiemann}},
  \bibinfo {author} {\bibfnamefont {C.}~\bibnamefont {Lisdat}}, \bibinfo
  {author} {\bibfnamefont {H.}~\bibnamefont {Schnatz}}, \ and\ \bibinfo
  {author} {\bibfnamefont {G.}~\bibnamefont {Grosche}},\ }\bibfield  {title}
  {\enquote {\bibinfo {title} {Transition frequencies of the $d$ lines of
  $^{39}\mathrm{K}$, $^{40}\mathrm{K}$, and $^{41}\mathrm{K}$ measured with a
  femtosecond laser frequency comb},}\ }\href {\doibase
  10.1103/PhysRevA.74.032503} {\bibfield  {journal} {\bibinfo  {journal} {Phys.
  Rev. A}\ }\textbf {\bibinfo {volume} {74}},\ \bibinfo {pages} {032503}
  (\bibinfo {year} {2006})}\BibitemShut {NoStop}%
\bibitem [{\citenamefont {Behr}\ \emph {et~al.}(1997)\citenamefont {Behr},
  \citenamefont {Gorelov}, \citenamefont {Swanson}, \citenamefont {H\"ausser},
  \citenamefont {Jackson}, \citenamefont {Trinczek}, \citenamefont {Giesen},
  \citenamefont {D'Auria}, \citenamefont {Hardy}, \citenamefont {Wilson},
  \citenamefont {Choboter}, \citenamefont {Leblond}, \citenamefont {Buchmann},
  \citenamefont {Dombsky}, \citenamefont {Levy}, \citenamefont {Roy},
  \citenamefont {Brown},\ and\ \citenamefont {Dilling}}]{behr_rad_Kisotopes}%
  \BibitemOpen
  \bibfield  {author} {\bibinfo {author} {\bibfnamefont {J.~A.}\ \bibnamefont
  {Behr}}, \bibinfo {author} {\bibfnamefont {A.}~\bibnamefont {Gorelov}},
  \bibinfo {author} {\bibfnamefont {T.}~\bibnamefont {Swanson}}, \bibinfo
  {author} {\bibfnamefont {O.}~\bibnamefont {H\"ausser}}, \bibinfo {author}
  {\bibfnamefont {K.~P.}\ \bibnamefont {Jackson}}, \bibinfo {author}
  {\bibfnamefont {M.}~\bibnamefont {Trinczek}}, \bibinfo {author}
  {\bibfnamefont {U.}~\bibnamefont {Giesen}}, \bibinfo {author} {\bibfnamefont
  {J.~M.}\ \bibnamefont {D'Auria}}, \bibinfo {author} {\bibfnamefont
  {R.}~\bibnamefont {Hardy}}, \bibinfo {author} {\bibfnamefont
  {T.}~\bibnamefont {Wilson}}, \bibinfo {author} {\bibfnamefont
  {P.}~\bibnamefont {Choboter}}, \bibinfo {author} {\bibfnamefont
  {F.}~\bibnamefont {Leblond}}, \bibinfo {author} {\bibfnamefont
  {L.}~\bibnamefont {Buchmann}}, \bibinfo {author} {\bibfnamefont
  {M.}~\bibnamefont {Dombsky}}, \bibinfo {author} {\bibfnamefont {C.~D.~P.}\
  \bibnamefont {Levy}}, \bibinfo {author} {\bibfnamefont {G.}~\bibnamefont
  {Roy}}, \bibinfo {author} {\bibfnamefont {B.~A.}\ \bibnamefont {Brown}}, \
  and\ \bibinfo {author} {\bibfnamefont {J.}~\bibnamefont {Dilling}},\
  }\bibfield  {title} {\enquote {\bibinfo {title} {Magneto-optic trapping of
  $\ensuremath{\beta}$-decaying ${}^{38}{K}^{\mathit{m}}$, ${}^{37}k$ from an
  on-line isotope separator},}\ }\href {\doibase 10.1103/PhysRevLett.79.375}
  {\bibfield  {journal} {\bibinfo  {journal} {Phys. Rev. Lett.}\ }\textbf
  {\bibinfo {volume} {79}},\ \bibinfo {pages} {375--378} (\bibinfo {year}
  {1997})}\BibitemShut {NoStop}%
\end{thebibliography}%


\end{document}
%

% ****** End of file aipsamp.tex ******