%
% The first command in your LaTeX source must be the \documentclass command.
% \documentclass[acmlarge, anonymous]{acmart}
\documentclass[acmlarge]{acmart}
% Load basic packages
\usepackage{balance}       % to better equalize the last page
\usepackage{graphics}      % for EPS, load graphicx instead 
\usepackage{hyperref}
\usepackage{color}
\usepackage{booktabs}
\usepackage{textcomp}
\usepackage{subcaption}
\usepackage{enumerate}
\usepackage{xcolor}
\usepackage{lipsum}% http://ctan.org/pkg/lipsum
\usepackage{makecell}
\usepackage{multicol}
\usepackage{multirow}
\usepackage{array}
\usepackage{verbatimbox}
\usepackage{enumitem}
\usepackage{amsmath}
\usepackage{stfloats}
\usepackage{graphicx}
\usepackage{amsthm}
\usepackage{listings}
\usepackage{caption} 
\usepackage[export]{adjustbox}
\usepackage{xspace}
\usepackage{epsfig}
\usepackage[linesnumbered]{algorithm2e}
\usepackage{algpseudocode}
\usepackage{tabularx}
\usepackage{arydshln}
\usepackage[bottom]{footmisc}
\usepackage{tcolorbox}
\usepackage{stackengine}
\usepackage{placeins}

\newcommand*{\eg}{\textit{e.g.},\xspace}
\newcommand*{\ie}{\textit{i.e.},\xspace}
\newcommand*{\vs}{\textit{vs.}\xspace}
\newcommand*{\etc}{\textit{etc.}}
\newcommand*{\st}{\textit{s.t.},\xspace}
\newcommand*{\etal}{\textit{et~al.}\xspace}
\newcommand*{\hlinespace}{\addlinespace[1ex]\hline\addlinespace[1ex]}
\newcommand*{\hdashlinespace}{\addlinespace[1ex]\hdashline\addlinespace[1ex]}
\newcommand*{\cdashlinespace}[1]{\addlinespace[1ex]\cdashline{#1}\addlinespace[1ex]}
\newcommand{\clinespace}[1]{\addlinespace[1ex]\cline{#1}\addlinespace[1ex]}

\newcolumntype{L}[1]{>{\raggedright\let\newline\\\arraybackslash\hspace{0pt}}m{#1}}
\newcolumntype{C}[1]{>{\centering\let\newline\\\arraybackslash\hspace{0pt}}m{#1}}
\newcolumntype{R}[1]{>{\raggedleft\let\newline\\\arraybackslash\hspace{0pt}}m{#1}}



\makeatletter
\def\thickhline{%
  \noalign{\ifnum0=`}\fi\hrule \@height \thickarrayrulewidth \futurelet
   \reserved@a\@xthickhline}
\def\@xthickhline{\ifx\reserved@a\thickhline
               \vskip\doublerulesep
               \vskip-\thickarrayrulewidth
             \fi
      \ifnum0=`{\fi}}
\makeatother

\makeatletter
\def\thickhlinespace{%
  \addlinespace[1ex]
  \noalign{\ifnum0=`}\fi\hrule \@height \thickarrayrulewidth \futurelet
   \reserved@a\@xthickhline
   \addlinespace[1ex]
   }
\def\@xthickhlinespace{\ifx\reserved@a\thickhline
               \vskip\doublerulesep
               \vskip-\thickarrayrulewidth
             \fi
      \ifnum0=`{\fi}}
\makeatother

\newlength{\thickarrayrulewidth}
\setlength{\thickarrayrulewidth}{3\arrayrulewidth}


\newlength\Origarrayrulewidth

% horizontal rule equivalent to \cline but with 2pt width
\newcommand{\Cline}[1]{%
 \noalign{\global\setlength\Origarrayrulewidth{\arrayrulewidth}}%
 \noalign{\global\setlength\arrayrulewidth{2pt}}\cline{#1}%
 \noalign{\global\setlength\arrayrulewidth{\Origarrayrulewidth}}%
}

% draw a vertical rule of width 2pt on both sides of a cell
\newcommand\Thickvrule[1]{%
  \multicolumn{1}{!{\vrule width 2pt}c!{\vrule width 2pt}}{#1}%
}

% draw a vertical rule of width 2pt on the left side of a cell
\newcommand\Thickvrulel[1]{%
  \multicolumn{1}{!{\vrule width 2pt}c|}{#1}%
}

% draw a vertical rule of width 2pt on the right side of a cell
\newcommand\Thickvruler[1]{%
  \multicolumn{1}{|c!{\vrule width 2pt}}{#1}%
}

\DeclareMathOperator*{\argmin}{argmin}   % Jan Hlavacek
\DeclareMathOperator*{\argmax}{argmax}   % Jan Hlavacek

\newcommand{\algrule}[1][.2pt]{\par\vskip.5\baselineskip\hrule height #1\par\vskip.5\baselineskip}

\algnewcommand{\IfThenElse}[3]{% \IfThenElse{<if>}{<then>}{<else>}
  \State \algorithmicif\ #1\ \algorithmicthen\ #2\ \algorithmicelse\ #3}
  
\newenvironment{s_itemize}{
\begin{itemize}
  \setlength{\itemsep}{3pt}
  \setlength{\parskip}{0pt}
  \setlength{\parsep}{0pt}
}{\end{itemize}}

\newenvironment{s_enumerate}{
\begin{enumerate}
  \setlength{\itemsep}{3pt}
  \setlength{\parskip}{0pt}
  \setlength{\parsep}{0pt}
}{\end{enumerate}}

\newcommand\acomment[1]{\textcolor{orange}{\textit{Anind: #1}}}
\newcommand\anind[1]{\textcolor{orange}{\textit{Anind: #1}}}
\newcommand\jcomment[1]{\textcolor{red}{\textit{Jen: #1}}}
\newcommand\jm[1]{\textcolor{red}{\textit{Jen: #1}}}
\newcommand\jen[1]{\textcolor{red}{\textit{Jen: #1}}}
\newcommand\tim[1]{\textcolor{magenta}{\textit{Tim: #1}}}
\newcommand\ocomment[1]{\textcolor{blue}{\textit{Orson: #1}}}
\newcommand\orson[1]{\textcolor{blue}{\textit{Orson: #1}}}
\newcommand\needinput[1]{\textcolor{red}{\textit{#1}}}


\newcommand\downred[1]{\textcolor{downredcolor}{#1}}
\newcommand\upgreen[1]{\textcolor{upgreencolor}{#1}}
\definecolor{downredcolor}{HTML}{e31a1c}
\definecolor{upgreencolor}{HTML}{33a02c}

\definecolor{DarkGreen}{HTML}{5DAC81}
% \newcommand\review[1]{\textcolor{DarkGreen}{#1}}
% \newcommand\minorreview[1]{\textcolor{DarkGreen}{#1}}
\newcommand\review[1]{\textcolor{black}{#1}}
\newcommand\minorreview[1]{\textcolor{black}{#1}}
% \newcommand\review[1]{\textcolor{black}{#1}}


\newcommand\projectname{}

% Rights management information. 
% This information is sent to you when you complete the rights form.
% These commands have SAMPLE values in them; it is your responsibility as an author to replace
% the commands and values with those provided to you when you complete the rights form.
%
% These commands are for a PROCEEDINGS abstract or paper.
% \copyrightyear{2018}
% \acmYear{2018}
% \setcopyright{acmlicensed}
% \acmConference[Woodstock '18]{Woodstock '18: ACM Symposium on Neural Gaze Detection}{June 03--05, 2018}{Woodstock, NY}
% \acmBooktitle{Woodstock '18: ACM Symposium on Neural Gaze Detection, June 03--05, 2018, Woodstock, NY}
% \acmPrice{15.00}
% \acmDOI{10.1145/1122445.1122456}
% \acmISBN{978-1-4503-9999-9/18/06}

%
% These commands are for a JOURNAL article.

\setcopyright{rightsretained}
\acmJournal{IMWUT}
\acmYear{2023}
% \acmVolume{6} \acmNumber{4} \acmArticle{190} \acmMonth{12} \acmPrice{}\acmDOI{10.1145/XXXXXXX}

%
% Submission ID. 
% Use this when submitting an article to a sponsored event. You'll receive a unique submission ID from the organizers
% of the event, and this ID should be used as the parameter to this command.
%\acmSubmissionID{123-A56-BU3}

%
% The majority of ACM publications use numbered citations and references. If you are preparing content for an event
% sponsored by ACM SIGGRAPH, you must use the "author year" style of citations and references. Uncommenting
% the next command will enable that style.
%\citestyle{acmauthoryear}

%
% end of the preamble, start of the body of the document source.
\begin{document}

%
% The "title" command has an optional parameter, allowing the author to define a "short title" to be used in page headers.
% \title{\projectname: Connecting Cough Detection and Lung Health Assessment with An End-to-End Deep Learning Model on Passively Sensed Audio}

\title{Mental-LLM: Leveraging Large Language Models for Mental Health Prediction via Online Text Data}

%
% The "author" command and its associated commands are used to define the authors and their affiliations.
% Of note is the shared affiliation of the first two authors, and the "authornote" and "authornotemark" commands
% used to denote shared contribution to the research.
% Xuhai Xu, Bingshen Yao, Yuanzhe Dong, Hong Yu, James Hendler, Anind K. Dey, Dakuo Wang

\author{Xuhai Xu}
\email{xuhaixu@uw.edu}
\orcid{0000-0001-5930-3899}
\affiliation{%
  \institution{Massachusetts Institute of Technology \& University of Washington}
  \country{USA}
}

\author{Bingsheng Yao}
% \email{yaob@rpi.edu}
\orcid{0009-0004-8329-4610}
\affiliation{%
  \institution{Rensselaer Polytechnic Institute}
  \country{USA}
}

\author{Yuanzhe Dong}
% \email{yzd@stanford.edu}
\orcid{0009-0006-2013-1157}
\affiliation{%
  \institution{Stanford University}
  \country{USA}
}

\author{Saadia Gabriel}
\orcid{}
\affiliation{%
    \institution{Massachusetts Institute of Technology}
    \country{USA}
}

\author{Hong Yu}
% \email{hong_yu@uml.edu}
\orcid{0000-0001-9263-5035}
\affiliation{%
    \institution{University of Massachusetts Lowell}
    \country{USA}
}

\author{James Hendler}
% \email{hendler@cs.rpi.edu}
\orcid{0000-0003-3056-1960}
\affiliation{%
  \institution{Rensselaer Polytechnic Institute}
  \country{USA}
}

\author{Marzyeh Ghassemi}
% \email{mghassem@csail.mit.edu}
\orcid{0000-0001-6349-7251}
\affiliation{%
    \institution{Massachusetts Institute of Technology}
    \country{USA}
}

\author{Anind K. Dey}
% \email{anind@uw.edu}
\orcid{0000-0002-3004-0770}
\affiliation{%
  \institution{University of Washington}
  \country{USA}
}
 
\author{Dakuo Wang}
% \email{d.wang@northeastern.edu}
\orcid{0000-0001-9371-9441}
\affiliation{%
  \institution{Northeastern University}
  \country{USA}
}


%
% By default, the full list of authors will be used in the page headers. Often, this list is too long, and will overlap
% other information printed in the page headers. This command allows the author to define a more concise list
% of authors' names for this purpose.
\renewcommand{\shortauthors}{Xu et al.}
\renewcommand{\shorttitle}{Mental-LLM}

%
% The abstract is a short summary of the work to be presented in the article.
\begin{abstract}
\begin{abstract}

This paper presents a low-cost network architecture for training large language models (LLMs) at hyperscale. We study the optimal parallelization strategy of LLMs and propose a novel datacenter network design tailored to LLM's unique communication pattern. We show that LLM training generates sparse communication patterns in the network and, therefore, does not require any-to-any full-bisection network to complete efficiently. As a result, our design eliminates the spine layer in traditional GPU clusters. We name this design a \textit{Rail-only} network and demonstrate that it achieves the same training performance while reducing the network cost by 38\% to 77\% and network power consumption by 37\% to 75\% compared to a conventional GPU datacenter. Our architecture also supports Mixture-of-Expert (MoE) models with all-to-all communication through forwarding, with only 8.2\% to 11.2\% completion time overhead for all-to-all traffic. We study the failure robustness of Rail-only networks and provide insights into the performance impact of different network and training parameters. \looseness=-1


\end{abstract}


\end{abstract}

%
% The code below is generated by the tool at http://dl.acm.org/ccs.cfm.
% Please copy and paste the code instead of the example below.
%
\begin{CCSXML}
<ccs2012>
<concept>
<concept_id>10003120.10003138</concept_id>
<concept_desc>Human-centered computing~Ubiquitous and mobile computing</concept_desc>
<concept_significance>500</concept_significance>
</concept>
<concept>
<concept_id>10010405.10010444</concept_id>
<concept_desc>Applied computing~Life and medical sciences</concept_desc>
<concept_significance>500</concept_significance>
</concept>
</ccs2012>
\end{CCSXML}
\ccsdesc[500]{Human-centered computing~Ubiquitous and mobile computing}
\ccsdesc[500]{Applied computing~Life and medical sciences}
%
% Keywords. The author(s) should pick words that accurately describe the work being
% presented. Separate the keywords with commas.
\keywords{Mental Health, Large Language Model, Instruction Finetuning}

%
% This command processes the author and affiliation and title information and builds
% the first part of the formatted document.
\maketitle

\section{Introduction}
\label{sec:introduction}

The recent surge of Large Language Models (LLMs), such as GPT-3.5/4~\cite{bubeck_sparks_2023}, PaLM~\cite{chowdhery_palm_2022}, FLAN-T5~\cite{chung_scaling_2022}, and Alpaca~\cite{taori_stanford_2023}, has shown a promising trend of large pre-trained models to do a variety of tasks in a zero-shot setting (\ie without any new training data). Example tasks include question answering~\cite{omar2023chatgpt,robinson2023leveraging}, logic reasoning~\cite{wei_chain--thought_2023,zhou_least--most_2023}, machine translation~\cite{brants2007large,gulcehre2017integrating} \etc\ 
A number of experiments have revealed that, built on hundreds of billions of parameters, these LLMs have started to show the capability to understand the human common sense beneath the natural language and do proper reasoning and inference accordingly~\cite{bubeck_sparks_2023,nori_capabilities_2023}.

Among different applications, one particular question yet to be answered is how well LLMs can understand human mental health states through natural language.
Mental health problems represent a significant burden for individuals and societies worldwide.
A recent report suggested that more than 20\% of adults in the U.S. would experience at least one mental disorder in their lifetime~\cite{mental2022state} and 5.6\% of adults experienced a serious psychotic disorder that significantly impairs functioning~\cite{mental2023stats}. The global economy loses around \$1 trillion annually in productivity due to depression and anxiety alone~\cite{mentalcost2023}.

In the past decade, there has been a plethora of research in natural language processing (NLP) and computational social science on detecting mental health issues via online text data such as social media~(\eg \cite{guntuku_detecting_2017,eichstaedt2018facebook,coppersmith_clpsych_2015,de_choudhury_social_2013,de_choudhury_mental_2014}). However, most of these studies have focused on building domain-specific machine learning (ML) models (\ie one model for one particular task, such as stress detection~\cite{nijhawan2022stress,guntuku2019understanding}, depression prediction~\cite{eichstaedt2018facebook,tadesse2019detection,xu_leveraging_2019}, or suicide risk assessment~\cite{de_choudhury_discovering_2016,coppersmith2018natural}). Even for traditional pre-trained language models such as BERT, it needs to be finetuned for specific downstream tasks~\cite{devlin_bert_2019,liu_roberta_2019}.
Since natural language is a major component of mental health assessment and treatment~\cite{sharma2018mental,gkotsis2016language}, LLMs might be a potentially powerful tool to understand end-users' mental states based on the language users' wrote. These instruction-finetuned and general-purpose models can understand a variety of inputs and obviate the need to train multiple models for different tasks. Thus, we can envision using one LLM for a variety of mental-health-related tasks, such as multiple question-answering, reasoning, and inference.
Such a vision opens up a wide range of opportunities for UbiComp, Human-Computer Interaction (HCI), and mental health communities, such as online public health monitoring systems~\cite{patel2018psyheal,graham2019artificial}, intelligent assistants for mental counselors and supporters~\cite{sharma_towards_2021,sharma_humanai_2023}, mental-health-aware personal chatbots~\cite{abd2021perceptions,denecke2020mental}, to just name a few.
However, there is a lack of investigation into understanding, evaluating, and improving the capability of LLMs for mental health prediction tasks.

There are few very recent studies on the evaluation of LLMs (\eg ChatGPT) on mental-health-related tasks, most of which are in zero-shot settings with simple prompt engineering~\cite{yang_evaluations_2023,amin_will_2023,lamichhane_evaluation_2023}. Researchers have shown preliminary results that LLMs have some initial capability of predicting mental health disorders with natural language with some promising but still limited performance compared to state-of-the-art domain-specific NLP models~\cite{yang_evaluations_2023,lamichhane_evaluation_2023}.
This remaining gap is expected since existing general-purpose LLMs are not specifically trained on mental health tasks.
However, to achieve our vision of leveraging LLMs for mental health support and assistance, we need to answer the research question: \textbf{How to empower LLMs with more mental health domain knowledge and become an expert}?

We conducted a series of experiments with multiple LLMs, including Alpaca~\cite{noauthor_stanford_2023}, Alpaca-LoRA~\cite{hu_lora_2021}, and GPT-3.5~\cite{noauthor_introducing_2022}.
Considering the data availability, we focused on online social media data with high-quality human-generated mental health labels.
Our experiments contained three stages: (1) zero-shot prompting, where we experimented with various prompts related to mental health, (2) few-shot prompting, where we inserted examples into prompt inputs, and (3) instruction-finetuning, where we finetuned LLMs on multiple mental-health datasets with various tasks.

Our results indicate that zero-shot obtained promising but limited performance on multiple mental health prediction tasks across all models. GPT-3.5 had relatively better results since it has a larger scale. But their performance is still far from task-specific models. 
Meanwhile, providing a few shots in the prompt can improve the model performance to some extent ($\overline{\Delta}$ = 4.7\%), but the advantage is limited.
Finally and most importantly, we found that instruction-finetuning can significantly improve the model performance across multiple mental-health-related tasks at the same time. Our finetuned Alpaca, namely \textbf{Mental-Alpaca}, significantly outperforms the original GPT-3.5 ($\times$25 times of model size) by an average of 16.7\% on balance accuracy. 
Meanwhile, Mental-Alpaca can further perform on par with the task-specific state-of-the-art Mental-RoBERTa~\cite{ji_mentalbert_2021}. It is noteworthy that Mental-RoBERTa needs to be trained on each task individually, 
while our Mental-Alpaca can solve different tasks off the shelf. 
% We open-source our training code and model at [github link].
Our experiments present the first comprehensive evaluation of various techniques to enhance LLMs' capability in the mental health domain.

The contribution of our paper can be summarized as follows:
\begin{s_enumerate}
\item We present the first comprehensive evaluation of prompt engineering, few-shot, and finetuning techniques on multiple LLMs in the mental health domain.
\item With online social media data, our results reveal that finetuning on a variety of datasets can significantly improve LLM's capability on multiple mental-health-specific tasks simultaneously.
% We release our model \textbf{Mental-Alpaca} as the first open-source LLM targeted at mental health prediction tasks.
\item We provide a few technical guidelines for future researchers and developers on turning LLMs into experts in specific domains.
\end{s_enumerate}


\section{Uncertainty}
\label{sec:background}
Uncertainty is a rich concept that has received various reasonable treatments before today's understanding of it.\footnote{From uncertainty's connection to (mostly abandoned) views on what is `knowable' \citep{knight1921risk}, to its central role in decision theories  \citep{ramsey1931foundations,von1947theory,Wald1951StatisticalDF,bernardo1994bayesian} and mathematical statistics \citep{savage1972foundations} to its modern understanding in terms of state of knowledge \citep{morgan_henrion_1990,lindley2013understanding}, to its  mathematical representation detached from philosophical interpretation \citep{halpern2017reasoning}.} We begin discussing it through common language. The online edition of the Oxford English dictionary listed five senses of uncertainty (retrieved in May 2023), two of which we partly quote here (those general enough to include the others as special case): \emph{(i) the state of not being definitely known or perfectly clear}; and \emph{(ii) the state or character of being uncertain in mind}. Both definitions regard uncertainty as \emph{a state of affairs}: in \emph{(i)}, the state of the world; in \emph{(ii)}, the state of an agent contemplating the world. They are subtly different: \emph{(i)} encompasses situations of inherent randomness (\eg, the result of a coin flip), \emph{(ii)} concerns one's inability to predict the state of the world regardless of any inherent randomness (\eg, a reader wondering about the content of the next paragraph). As we shall see, this difference leads to rather different interpretations %
of uncertainty as an aspect of reality. Yet, at the level of mathematical treatment, they share the same formal devices. Hence, with no loss of generality, we choose to talk about uncertainty from the point of view of an agent contemplating or interacting with the world, while possessing limited knowledge about it. Our presentation is inspired by various reference texts,  in particular, \citet{dubois2009formal} and \citet{halpern2017reasoning}. 

\paragraph{Agents.} We posit that any one agent shall represent the state of their knowledge in a way sufficient for reasoning about the truth value of claims (or propositions) that they make about aspects of the world. In particular, the agent is able to state their preference for claims they find themselves less uncertain about (\ie, possessing better information about those).\footnote{Agents and worlds are abstractions to be adapted and tailored to each application, commonly in NLG an agent is a model and a world is a response to a given prompt.}  An agent then uses this \emph{uncertainty representation} to interact with the environment (\eg, inform their actions) and, when they acquire new knowledge, they update the representation in a coherent manner. 
To illustrate formal concepts, we use three example agents. \textbf{A1}\hspace{0.5mm}\twemoji{game die} rolls a six-sided die; we seek to represent their state of knowledge about the outcome. \textbf{A2}\hspace{0.5mm}\twemoji{busts in silhouette} resolves mentions of entities to unique names in a knowledge base (KB); we seek to represent their state of knowledge about entity names given any one mention. Last, \textbf{A3}\hspace{0.5mm}\twemoji{speech balloon} provides written answers to questions; we seek to represent their state of knowledge about answers given any one question. 
For simplicity, we assume that our agents already acquired their knowledge, by means which are not relevant for now, and their state of knowledge is frozen.  
We begin by outlining the formal tools common to all frameworks for uncertainty representation we are aware of, we then zoom into the most commonly used framework (probability) and discuss the role of statistics in acquisition and revision of knowledge.  

\paragraph{Possible worlds.} Our agent does not know the state of the actual world, but they assume that it must be one of a collection of possible worlds (the universe). They represent a world as a unique symbol (or string, or collection of attributes; the level of detail being dictated by the agent's needs), and the universe of what is possible as a set $\Omega$ of mutually exclusive worlds.\footnote{This framework, \textit{possible worlds}, is familiar to linguists and philosophers alike \citep{hintikka1957modality,hintikka1961modality,sep-possible-worlds}.
} 
\textbf{A1}\hspace{0.5mm}\twemoji{game die}  might represent a world as a symbol $f_k$, with $k$ denoting the number of pips the die shows as a result of the roll; they might assume the die always lands showing one of six numbered faces and thus take $\{f_1, \ldots, f_6\}$ to represent all possible worlds. 
For \textbf{A2}\hspace{0.5mm}\twemoji{busts in silhouette}, a world is a symbol like $e_i$, with $i$ denoting an entity's identifier (\eg, a standardised unique name), and the universe is the finite set of entities in the English Wikipedia. 
For \textbf{A3}\hspace{0.5mm}\twemoji{speech balloon}, a world is a symbol like $u_s$, with $s$ an English sentence produced in response to a question. This agent happens to be unable to describe the set of all valid English sentences (they cannot enumerate its elements nor state a finite set of properties that all valid sentences must satisfy). Motivated by convenience, \textbf{A3}\hspace{0.5mm}\twemoji{speech balloon} uses a set large enough to encompass most of it while being specifiable in a compact manner: the set of all finite-length strings made by concatenation of known symbols (\eg, words, punctuation, \etc). These examples show that the agent's choice of universe can be a difficult one, often requiring simplifying assumptions: on soft or irregular terrain, a die could land on an edge; a KB may be incomplete (sometimes in known ways, \eg, under-representing the contributions of Black women to science); a regular language is a too loosely constrained representation of the English language (\eg, it includes infinitely many strings that will never correspond to any actual world). 

\paragraph{Propositions.} The possible worlds framework gives agents a mechanism to represent claims about specific aspects of the world. A \emph{proposition}  $E_t$ is the claim that the actual world $\omega$ is one where some property $t$ holds (which we denote $t(\omega)$). A property is something that can be assessed for any one world (\eg, $f_2$ is even and prime, $e_{\operatorname{Katherine\_Johnson}}$ is African-American, female and mathematician, $u_\text{`Biden is the 46th US president'}$ expresses the relation $\operatorname{presidentof}(\operatorname{Joe\_Biden}, \operatorname{USA})$). Not knowing the state of the actual world,  our agent represents $E_t$ by the set  $E_t = \{w \in \Omega: t(w)\} \subseteq \Omega$ of all possible worlds where the property holds. If the agent knew the state of the actual world $\omega$, then the truth value of the proposition would be determined simply by set membership (\ie, $\omega \in E_t$ or $\omega \not\in E_t$). 
For example, \textbf{A1}\hspace{0.5mm}\twemoji{game die}  represents the claim ``the roll is odd'' as $E_{\text{odd}} = \{f_1, f_3, f_5\}$.
\textbf{A2}\hspace{0.5mm}\twemoji{busts in silhouette} represents the claim ``mention to a female mathematician'' by the set $\{e_i \in \Omega: \operatorname{female}(e_i) \wedge \operatorname{mathematician}(e_i)\}$. %
\textbf{A3}\hspace{0.5mm}\twemoji{speech balloon} might use equivalence classes, for example, they use the set $E_a = \{u_s \in \Omega : \operatorname{equivalent}_{a}(u_s)\}$ to claim that the answer is a sentence semantically equivalent to some other sentence $u_a \in \Omega$ (\eg, $u_\text{`The 46th US president is Joe Biden'}$). Because propositions are semantic in nature, they can be difficult to represent explicitly. For example, \textbf{A3}\hspace{0.5mm}\twemoji{speech balloon}'s equivalence classes require sophisticated natural language understanding. 
A representation of all propositions an agent deems possible is a set $\mathcal E$ of subsets of $\Omega$.\footnote{If an agent has no knowledge of the impossibility of any proposition, or does not care to exclude those from the representation, the powerset of (countable) $\Omega$ is a reasonable choice for $\mathcal E$. In NLG, we often implicitly make this choice.}

\paragraph{Preferences.} The agent's imperfect knowledge of the actual world $\omega$ translates to limited knowledge about propositions. However, the agent's ignorance is qualitatively different depending on the claims they make. Intuitively, some claims are compatible with many of the possible worlds, while others hold in but a few (\eg, \textbf{A1}\hspace{0.5mm}\twemoji{game die}  knows that only one prime number is even), and though the various worlds are all possible, they may not be equally plausible (\eg, \textbf{A2}\hspace{0.5mm}\twemoji{busts in silhouette} knows that most mentions resolve to politicians, \textbf{A3}\hspace{0.5mm}\twemoji{speech balloon} knows that most answers are only a few words long), \etc.
Considerations of those kinds motivate an agent to express a \emph{preference} for claims they find themselves less uncertain about (\ie, possessing better information about those). The agent does so by prescribing a \emph{plausibility measure} \citep{friedman96}, a function that attaches a token of uncertainty---a qualifier that the agent knows how to sort---to each proposition in $\mathcal E$. Plausibility measures are very diverse, the most well known instance of it being axiomatic probability \citep{kolmogorov1960foundations}.\footnote{Other plausibility measures include belief functions \citep{shafer76}, possibility measures \citep{duboisprade90}, ordinal ranking functions \citep{GoldszmidtPearl92} and (non-numerical) preference orders \citep{friedman96}. 
Concrete instances of plausibility measures vary in descriptive power. Under certain documented assumptions \citep{friedman96}, they enable something like a `calculus of uncertainty' which formalises the procedures the agent must follow to incorporate additional information about the world and revise their uncertainty representation coherently (in axiomatic probability, this is known as \emph{conditioning}).} 

\paragraph{Probability.} Probability is a numerical qualifier that we can attach to events in $\mathcal E$.\footnote{In probability, propositions are  \emph{events}, worlds are \emph{outcomes} and universes are \emph{sample spaces}.} 
For any event, this qualifier is a positive real number bounded to be at most $1$. Probability values inherit various properties of real numbers: we can add, multiply and sort them. A function $\Pr$ over $\mathcal E$ is a \emph{probability measure} if a) it maps $\Omega$ to $1$, and b) it maps any two disjoint sets $U$ and $V$ in $\mathcal E$ to $\Pr(U)+\Pr(V)$, which is known as additivity. 
 Additivity, in particular, implies that we can identify a probability measure over all of $\mathcal E$ by assigning probability to elementary outcomes in $\Omega$, for example, using a probability mass function (pmf) or probability density function (pdf; for uncountable $\Omega$). This has massive consequences for uncertainty representation, since working with elementary outcomes is much simpler than working with sets of outcomes (for example, difficulty in prescribing equivalence classes such as `all sentences that talk about Joe Biden' need not stop \textbf{A3}\hspace{0.5mm}\twemoji{speech balloon} from identifying a probability measure for their reasoning needs).
 
\paragraph{Interpretations.} 
Probability has been motivated and justified from different angles, each building on a specific interpretation of what probability as a number must signify \citep{hacking1975emergence}. However different they are, they all lead to the same formal device. Under certain idealisations, \textit{objectivists} regard events as \emph{repeatable} (\eg, we may prompt a human speaker multiple times). Repetitions allow an agent to perceive what may be thought of as an inherent \emph{property} of an event: its \emph{frequency} in a large enough number of repetitions. The \emph{subjectivist} interpretation \citep{ramsey1931foundations,definetti2017theory} views probability as a degree of belief, personal to an agent, and deprived of any interpretation beyond its formal role as an expression of the agent's preferences.\footnote{Dictionary definition \textit{(i)} is objectivist; \textit{(ii)} subjectivist.} 
Different interpretations have an impact on the procedures that an agent acknowledges as logical or rational for knowledge acquisition and revision, as we discuss next.

\paragraph{Statistics.} We have described the general tools that agents can use to represent and convey their uncertainty. But where do their preferences (probabilities, in particular) come from? The \emph{Frequentist} agent is essentially an objectivist who assumes the existence of a precise statistical law that describes the phenomena in consideration. They assume to have access to this law up to an unknown parameter $\theta^\star \in  \mathbb R^D$. %
Given a parameter $\theta$, their preferences are specified via a pmf (or pdf) $p(x|\theta)$. Given data $\mathbf x = \langle x_1, \ldots, x_N \rangle$, this law identifies the so called likelihood function $\ell_{\mathbf x}(\theta) = \prod_n p(x_n|\theta)$, a measure of the compatibility between observed data $\mathbf x$ and the statistical model identified by $\theta$. The agent uses $\mathbf x$ to estimate the parameter $\theta^\star$: they pick the parameter $\hat\theta$ that maximises the likelihood function, %
this is known as maximum likelihood estimation (MLE). They do not entertain parameters as part of the possible worlds, hence have no uncertainty representation about them. Given the  parameter  estimate $\hat\theta$, the agent uses $p(x_{n+1}|\hat\theta)$ to make predictive inferences about future data $x_{n+1}$. When necessary (\eg, the agent suspects to have found a better statistical law), the agent studies properties of their parameter estimator(s) by repeated experimentation, for example to establish confidence intervals %
 and other tools for model selection (see for example \citealp{lehmann1993fisher}).
The \emph{Bayesian} agent, a subjectivist, %
also picks a statistical law, but makes no assumption about its correctness. Given some data $\mathbf x$, they too construct a likelihood function $\ell_{\mathbf x}(\theta)$, but use it differently. As a formal tool, probability comes with a mechanism for belief revision: conditioning. %
To make use of it, the agent augments their possible worlds to include possible values of $\theta$ and its interaction with possible values of the observable variable, they then state their preferences over parameters in the form of a pdf $p(\theta)$. %
This is called a \emph{prior} (conveys one's knowledge and experience before observing $\mathbf x$).
The agent then revises their preferences using Bayes rule %
to obtain a posterior pdf $p(\theta|\mathbf x) \propto p(\theta)\ell_{\mathbf x}(\theta)$. This object supports all inferences the agent will ever make (\eg, 
about parameters, or about future data $x_{n+1}$---for which the agent builds a posterior predictive function  $p(x_{n+1}|\mathbf x) = \int p(x_{n+1}|\theta)p(\theta|\mathbf x) \dd{\theta}$).
In essence, Frequentist procedures revolve around point estimation (\eg, MLE) and null hypothesis significance testing \citep{LehmCase98,lehmann2005testing}, %
Bayesian theory \citep{bernardo1994bayesian} and practice \citep{gelmanbda04}, instead, frame statistical inference as an iterative process of belief revision  (\eg, conditioning, marginalisation, expectation). 


\paragraph{Natural Language Generation.} 
Most NLG models (like \textbf{A3}\hspace{0.5mm}\twemoji{speech balloon}) acquire knowledge through MLE. Alternatives include Bayesian inference \cite[\eg,][]{malinin2020uncertainty,sankararaman2022bayesformer} and utility- and reward-based training (\eg, minimum risk \citep{shen-etal-2016-minimum},  reinforcement learning \citep{paulus2018a}). Recently, pre-training on enormous unlabelled corpora, and reinforcement learning from human feedback \cite[RLHF, \eg,][]{christiano2017deep,stiennon2020learning,ouyang2022training} or \textit{instruction tuning} \cite[\eg,][]{mishra-etal-2022-cross,wei2022finetuned} have become popular to refine the representation of uncertainty towards something that decodes more easily into strings preferred by human users. %

Generating a response is simulating an outcome. %
The event space is the powerset of all token sequences from a fixed vocabulary \cite[BPE tokens, \eg,][]{sennrich-etal-2016-neural}. Rather than prescribing a probability measure (mapping each event to a probability value) directly, we parameterise a pmf (typically parameterised via an autoregressive factorisation of the probability of any one sequence) with a neural network and exploit countable additivity to assign probability to any event (\eg, all token sequences that map to the same sentence \citep{cao-rimell-2021-evaluate,chirkova-etal-2023-marginalize} or all sentences that map to the same meaning representation \cite{kuhn2023semantic}). %


\paragraph{Key Takeaways.}
(1) Uncertainty is a state to be represented. %
(2) To represent uncertainty about something observable or not (\eg, responses, parameters, modelling assumptions) we need to acknowledge and order a whole space of alternatives: our choice of possible worlds must capture interaction amongst possible values of the variables we aim to express our uncertainty about. 
(3) Probability is not constrained to abide by any one interpretation. To regard probabilities in a specific human-interpretable way (\eg, relative frequencies), we need learning techniques yielding that result, and we need to verify that our setting actually meets all necessary conditions (\eg, the Frequentist interpretation of probability is sensitive to modelling choices, local optimality, and  data sparsity: most practical NLG agents are unable to meet the necessary formal requirements).

\section{Methods}
\label{sec:methods}
We introduce our experiment design with LMMs on multiple mental health prediction task setups, including zero-shot prompting (Sec.~\ref{sub:methods:zero-shot}), few-shot prompting (Sec.~\ref{sub:methods:few-shot}), and instruction finetuning (Sec.~\ref{sub:methods:finetuning}). These setups are model-agnostic, and we will present the details of language models and datasets employed for our experiment in the next section.

\subsection{Zero-shot Prompting}
\label{sub:methods:zero-shot}
The language understanding and reasoning capability of LLMs have enabled a wide range of applications without the need for any domain-specific data, but only providing appropriate prompts~\cite{kojima_large_2022,wei2021finetuned}.
Therefore, we start with prompt design for mental health tasks in a zero-shot setting.

The goal of prompt design is to empower a pre-trained general-purpose LLM to achieve good performance on tasks in the mental health domain. We propose a general zero-shot prompt template ($\textit{Prompt}_{ZS}$) that consists of four parts:
\begin{equation}
    \textit{Prompt}_{ZS} = \textit{TextData} + \textit{Prompt}_{\textit{Part1-S}} + \textit{Prompt}_{\textit{Part2-Q}} + \textit{OutputConstraint}
\label{eq:prompt-zs}
\end{equation}
where \textit{TextData} is the online text data generated by end-users. \textit{Prompt}$_{\textit{Part1-S}}$ provides specifications for a mental health prediction target. \textit{Prompt}$_{\textit{Part2-Q}}$ poses the question for LLMs to answer. And \textit{OutputConstraint} controls the output of models (\eg ``Only return yes or no'' for a binary classification task).

We propose several design strategies for \textit{Prompt}$_{\textit{Part1-S}}$, as shown in the top part of Table~\ref{tab:prompt_design}: (1) \textbf{Basic}, which leaves it as blank; (2) \textbf{Context Enhancement}, which provides more social media context about the \textit{TextData}; (3) \textbf{Mental Health Enhancement}, which inserts mental health concept by asking the model to act as an expert. (4) \textbf{Context \& Mental Health Enhancement}, which combines both enhancement strategies by asking the model to act as a mental health expert under the social media context.

As for \textit{Prompt}$_{\textit{Part2-Q}}$, we mainly focus on two categories of mental health prediction targets: (1) predicting critical mental states, such as stress or depression, and (2) predicting high-stake risk actions, such as suicide. We tailor the question description for each category. Moreover, for both categories, we explore binary and multi-class classification tasks
\footnote{We also conduct an exploratory case study on mental health reasoning tasks. Please see more details in Sec.~\ref{sub:results:reasoning}.}.
Thus, we also make small modifications based on specific tasks to ensure appropriate questions (see Sec.~\ref{sec:implementation} for our mental health tasks). The bottom part of Table~\ref{tab:prompt_design} summarizes the mapping.

\renewcommand{\arraystretch}{1}
\begin{table}[]
\centering
\caption{Prompt Design for Mental Health Recognition Tasks. Prompt Part 1 aims to provide a better specification for LLMs and Part 2 poses the questions for LLMs to answer. For Part 1, we propose three strategies: context enhancement, mental health enhancement, and the combination of both. As for Part 2, we design different content for multiple mental health problem categories and prediction tasks. For each part, we propose two to three versions to improve its variation.}
\label{tab:prompt_design}
\label{tab:my-table}
\resizebox{1\textwidth}{!}{
\begin{tabular}{ccl}
\thickhlinespace
 \multicolumn{2}{c}{\textbf{Strategy}} &
  \multicolumn{1}{c}{\textbf{Prompt Part 1 - S}} \\
\thickhlinespace 
\multicolumn{2}{c}{Basic} &
  \makecell[l]{$\bullet$ \{ blank \} } \\ \hlinespace
\multicolumn{2}{c}{Context Enhancement} &
  \makecell[l]{$\bullet$ This person wrote this paragraph on social media.\\$\bullet$ Consider this post on social media to answer the question.} \\ \hlinespace
\multicolumn{2}{c}{Mental Health Enhancement} &
  \makecell[l]{$\bullet$ As a psychologist, read the post on social media and answer the question.\\ $\bullet$ If you are a psychologist, read the post on social media and answer the question.} \\ \hlinespace
\multicolumn{2}{c}{Context \& Mental Health Enhancement} &
  \makecell[l]{ $\bullet$ This person wrote this paragraph on social media. As a psychologist, read the post on social media \\ \xspace\xspace\xspace\xspace and answer the question.\\ $\bullet$ This person wrote this paragraph on social media. As a psychologist, consider the mental well-being\\\xspace\xspace\xspace\xspace condition expressed in this post, read the post on social media, and answer the question.} \\
\thickhlinespace
\multicolumn{1}{c}{\textbf{Category}} &
  \multicolumn{1}{c}{\textbf{Task}} &
  \multicolumn{1}{c}{\textbf{Prompt Part 2 - Q}} \\
\thickhlinespace
\multirow{2}{*}{\makecell[c]{\\Mental state\\ (\eg stressed, \\ depressed)}} &
  \makecell[c]{Binary classification\\ (\eg yes or no)} &
  \makecell[l]{$\bullet$ Is the poster {[}stressed{]}?\\$\bullet$ Is the poster of this post {[}stressed{]}?\\$\bullet$ Determine if the poster of this post is {[}stressed{]}.} \\  \clinespace{2-3}
 &
  \makecell[c]{Multi-class classification\\ (\eg multiple levels)} &
  \makecell[l]{$\bullet$ Which level is the person {[}stressed{]}?\\$\bullet$ How {[}stressed{]} is the person?\\ $\bullet$ Determine how {[}stressed{]} the person is.} \\ \hlinespace
\multirow{2}{*}{\makecell[c]{\\Critical risk action\\ (\eg suicide)}} &
  \makecell[c]{Binary classification\\ (\eg yes or no)} &
  \makecell[l]{$\bullet$ Does the poster want to {[}suicide{]}?\\$\bullet$ Is the poster likely to {[}suicide{]}?\\$\bullet$ Determine if the poster of this post want to {[}suicide{]}.} \\  \clinespace{2-3}
 &
  \makecell[c]{Multi-class classification\\ (\eg multiple levels)} &
  \makecell[l]{$\bullet$ Which level of {[}suicide{]} risk does the person have?\\$\bullet$ How {[}suicidal{]} is the person?\\$\bullet$ Determine which level of {[}suicide{]} risk does the person have.}\\
\thickhlinespace
\end{tabular}
}
\end{table}
\renewcommand{\arraystretch}{1.0}

For both \textit{Prompt}$_{\textit{Part1-S}}$ and \textit{Prompt}$_{\textit{Part2-Q}}$, we propose several versions to improve its variability. We then evaluate these prompts on multiple LLMs on different datasets and compare their performance.

\subsection{Few-shot Prompting}
\label{sub:methods:few-shot}

In order to provide more domain-specific information, researchers have also explored few-shot prompting with LLMs by providing few-shot demonstrations to support in-context learning (\eg \cite{agrawal2022large,dang2022prompt}). Note that these few examples are used solely in prompts, and the model parameters remain unchanged. The intuition is to present a few ``examples'' for the model to learn domain-specific knowledge \textit{in situ}.
In our setting, we also test this strategy by adding additional randomly sampled [$\textit{Prompt}_{ZS} - \textit{label}$] pairs. The design of the few-shot prompt ($\textit{Prompt}_{FS}$) is straightforward:
\begin{equation}
    \textit{Prompt}_{FS} = [\textit{Sample Prompt}_{ZS} - \textit{label}]_{M} + \textit{Prompt}_{ZS}
\label{eq:prompt-fs}
\end{equation}
where $M$ is the number of prompt-label pairs and is capped by the input length limit of a model. Note that both the $\textit{Sample Prompt}_{ZS}$ and $\textit{Prompt}_{ZS}$ follow Eq.~\ref{eq:prompt-zs} and employ the same design of \textit{Prompt}$_{\textit{Part1-S}}$ and \textit{Prompt}$_{\textit{Part2-Q}}$ to ensure consistency.

\subsection{Instruction Finetuning}
\label{sub:methods:finetuning}

In contrast to the few-shot prompting strategy in Sec.~\ref{sub:methods:few-shot}, the goal of this strategy is closer to the traditional few-shot transfer learning, where we further train the model with a small amount of domain-specific data (\eg \cite{huang2022large,xu2021raise,liu_large_2023}).
We experiment with multiple finetuning strategies.

\subsubsection{Single-dataset Finetuning}
\label{subsub:methods:finetuning:single-dataset}
Following most of the previous work in the mental health field~\cite{yang_evaluations_2023,coppersmith_clpsych_2015,de_choudhury_discovering_2016}, we first conduct basic finetuning on a single dataset (the training set). This finetuned model can be tested on the same dataset (the test set) to evaluate its performance and different datasets to evaluate its generalizability.

\subsubsection{Multi-dataset Finetuning}
\label{subsub:methods:finetuning:multi-dataset}
From Sec.~\ref{sub:methods:zero-shot} to Sec.~\ref{subsub:methods:finetuning:single-dataset}, we have been focusing on one single mental health dataset $D$.
More interestingly, we further experiment with finetuning across multiple datasets simultaneously. Specifically, we leverage instruction finetuning to enable LLMs to handle multiple tasks in different datasets~\cite{brown_language_2020}.

It is noteworthy that such an instruction finetuning setup differs from the state-of-the-art mental-health-specific models (\eg Mental-RoBERTa~\cite{ji_mentalbert_2021}). The previous models are finetuned for a specific task, such as depression prediction or suicidal ideation prediction. Once trained on task A, the model becomes specific to task A and is only suitable for solving that particular task. In contrast, we finetune LLMs on several mental health datasets, employing diverse instructions for different tasks across these datasets in a single iteration. This enables them to handle multiple tasks without additional task-specific finetuning.

% It is noteworthy that such an instruction finetuning setup differs from the state-of-the-art mental-health-specific models (\eg Mental-RoBERTa~\cite{ji_mentalbert_2021}). The previous models are finetuned for a specific task, such as depression prediction or suicidal ideation prediction, or multiple predefined tasks in the multi-task setting. Once the model is trained on task(s) A, it becomes task-A-specific and only aims to solve task A. In contrast, we finetune LLMs on multiple mental health datasets with various instructions for different tasks across multiple datasets in a single round, empowering them to work on multiple tasks without the need for further task-specific finetuning.

For both single- and multi-dataset finetuning, we follow the same two steps:
\begin{align}
\begin{split}
\text{Step 1:} & \text{ Finetune with } [\textit{Prompt}_{ZS} - \textit{label}]_{\sum\limits^{I} N_{D_{i-train}}} \\
\text{Step 2:} & \text{ Test with } [\textit{Prompt}_{ZS}]_{\sum\limits^{I} N_{D_{i-test}}}
\end{split}
\label{eq:prompt-ft}
\end{align}
where $N_{D_{i-train}}$/$N_{D_{i-test}}$ is the total size of the training/test dataset $D_i$, $I$ represents the set of datasets used for finetuning, and $i$ indicates the specific dataset index ($i \in I, |I| \ge 1$). Both $\textit{Prompt}_{ZS\textit{-train}}$ and $\textit{Prompt}_{ZS\textit{-test}}$ follow Eq.~\ref{eq:prompt-zs}. Similar to the few-shot setup in Eq.~\ref{eq:prompt-fs}, they employ the same design of \textit{Prompt}$_{\textit{Part1-S}}$ and \textit{Prompt}$_{\textit{Part2-Q}}$.

% In addition, we also experiment with multiple variations:
% (1) \textbf{Data size variation}, where we sample a subset of each training dataset $D_{i-train}$ with a smaller $N'_{D_{i-train}}$ to explore the effect of data size on finetuning performance.
% (2) \textbf{Prompt variation}, where we randomly sample \textit{Prompt}$_{\textit{Part1-S}}$ from Table~\ref{tab:prompt_design} during the training to explore the effect of prompt diversity on finetuning performance.

\section{Implementation}
\label{sec:implementation}

Our method design is agnostic to specific datasets or models. In this section, we introduce the specific datasets (Sec.~\ref{sub:implementation:datasets}) and models (Sec.~\ref{sub:implementation:models}) involved in our experiments. In particular, we highlight our instructional-finetuned open-source model \textbf{Mental-Alpaca} (Sec.~\ref{subsub:implementation:models:mental}).

\subsection{Datasets and Tasks}
\label{sub:implementation:datasets}

Our experiment is based on four well-established datasets that are commonly employed for mental health analysis collected from social media platforms.
It is noteworthy that we intentionally avoid using datasets that generate labels based on certain linguistic patterns (\eg some datasets label a post based on whether a user ever stated ``I was diagnosed with X''), as it would be hard to remove artifacts or biases in these datasets.
Instead, we used ones with human expert annotations or supervision.
We define six diverse mental health prediction tasks based on these datasets.

\begin{s_itemize}
\item \textbf{Dreaddit}~\cite{turcan_dreaddit_2019}: This dataset collected posts from Reddit, which contains ten subreddits in the five domains (abuse, social, anxiety, PTSD, and financial). Multiple human annotators rated whether sentence segments showed the stress of the poster, and the annotations were aggregated to generate final labels. We used this dataset for a post-level binary stress prediction (\textbf{Task 1}).
\item \textbf{DepSeverity}~\cite{naseem_early_2022}: This dataset leveraged the same posts collected in \cite{turcan_dreaddit_2019}, but with a different focus on depression. Two human annotators followed DSM-5~\cite{regier2013dsm} and categorized posts into four levels of depression: minimal, mild, moderate, and severe. We employed this dataset for two post-level tasks: binary depression prediction (\ie whether a post showed at least mild depression, \textbf{Task 2}), and four-level depression prediction (\textbf{Task 3}).
\item \textbf{SDCNL}~\cite{haque_deep_2021}: This dataset also collected posts from Reddit, including r/SuicideWatch and r/Depression. Through manual annotation, they labeled whether each post showed suicidal thoughts. We employed this dataset for the post-level binary suicide ideation prediction (\textbf{Task 4}).
\item \textbf{CSSRS-Suicide}~\cite{gaur_knowledge-aware_2019}: This dataset contains posts from 15 mental health-related subreddits. Four practicing psychiatrists followed Columbia Suicide Severity Rating Scale (C-SSRS) guidelines~\cite{posner2008columbia} to manually annotate 500 users on suicide risks in five levels: supportive, indicator, ideation, behavior, and attempt. We leveraged this dataset for two user-level tasks: binary suicide risk prediction (\ie whether a user showed at least suicide indicator, \textbf{Task 5}), and five-level suicide risk prediction (\textbf{Task 6}).
\end{s_itemize}

Table~\ref{tab:datasets} summarizes the information of the four datasets and six mental health prediction tasks. For each dataset, we conducted an 80\%/20\% train-test split. It is noteworthy that one user's data were either in the training or test set to avoid leakage.
 
\begin{table*}[htbp]
  	\vspace{-0.8cm}
\setlength{\abovecaptionskip}{0pt} 
\setlength{\belowcaptionskip}{10pt}
  \centering
  \caption{Details of the selected 10 models}
  \begin{adjustbox}{width=\textwidth,center}
    \begin{tabular}{|c|c|c|c|c|}
    \noalign{\hrule height 1pt}
    \textbf{ID} & \textbf{App Name} &\textbf{Model in iOS} & \textbf{Model in Android} & \textbf{Model Function} \\
    \noalign{\hrule height 1pt}
    1     & hp smart & doc\_classification.mlmodelc & doc\_classification.tflite & document classification \\
    2     & paypal - send, shop, manage & ObamModel.mlmodelc & obamModel.tflite & image classification \\
    3     & merlin bird id by cornell lab & geo\_v18.mlmodelc & geo\_v17.tflite & bird classification   \\
    4     & seek by inaturalist  & optimized\_model.mlmodelc & optimized\_model.tflite & identify plants and animals in pictures\\
    5     & smart bird id & NABirdsImageClassifier.mlmodelc & BirdImageClassifier.tflite & identify birds in pictures\\
    6     & sticker maker studio & deeplabv3\_mnv2\_pascal\_trainval.mlmodelc & deeplabv3\_mnv2\_pascal\_trainval.tflite & image segmentation \\
    7     & scentbird & FindFour.mlmodelc & findfour.tflite & detect box in pictures \\
    8     & scentbird & FourRecognize.mlmodelc & fourrecognize.tflite & detect box in pictures\\
    9     & gradient: celebrity look alike & gender\_nn.mlmodelc & gender\_nn.tflite & identify gender in pictures\\
    10    & Hoop - Make new friends & Nudity.mlmodelc & optimized\_nudity\_graph.tflite & identify nudity in pictures\\
    \noalign{\hrule height 1pt}
    \end{tabular}%
    \end{adjustbox}
  \label{tab:dataset}%
  	\vspace{-0.2cm}
\end{table*}%


\subsection{Models}
\label{sub:implementation:models}
We experimented with multiple LLMs with different sizes, pre-training targets, and availability.

\begin{s_itemize}
\item \textbf{Alpaca} (7B)~\cite{taori_stanford_2023}: An open-source large model finetuned from another open-sourced LLaMA 7B model~\cite{touvron_llama_2023} on instruction following demonstrations. Experiments have shown that Alpaca behaves qualitatively similarly to OpenAI’s \texttt{text-davinci-003} on certain task metrics. We choose the relatively small 7B version so that we can run and finetune it on consumer hardware. 
\item \textbf{Alpaca-LoRA} (7B)~\cite{hu_lora_2021}: Another open-source large model finetuned from LLaMA 7B model using the same dataset as Alpaca~\cite{taori_stanford_2023}. This model leverages a different finetuning technique called low-rank adaptation (LoRA)~\cite{hu_lora_2021}, with the goal of reducing finetuning cost by freezing the model weights and injecting trainable rank decomposition matrices into each layer of the Transformer architecture. It is noteworthy that despite the name similarity, Alpaca-LoRA is completely different from Alpaca. They are trained on the same dataset but with different methods.
% \item \textbf{FLAN-T5} (13B)~\cite{chung_scaling_2022}: An open-source large model finetuned with a variety of task-based datasets on instruction. Compared to other LLMs, FLAN-T5 focuses more on task solving and is less optimized for natural language or dialogue generation. 
\item \textbf{GPT-3.5} (175B)~\cite{noauthor_introducing_2022}: This large model is closed-source and available through API provided by OpenAI. We picked the \texttt{gpt-3.5-turbo}, one of the most capable and cost-effective models in the GPT-3.5 family. Due to the limited availability of API, the cost of finetuning GPT-3.5 or experimenting with GPT-4 is prohibitive.
\end{s_itemize}

% It is worth noting that Alpaca, Alpaca-LoRA, and GPT-3.5 are all finetuned with natural dialogue as one of the optimization goals. In contrast, FLAN-T5 is more task-focused.
% In our case, the input posts written by users are closer to natural dialogue, while the mental health prediction tasks are defined as specific classification tasks. It is unclear and thus interesting to explore which LLM fits better with our target.

\subsubsection{Mental-Alpaca}
\label{subsub:implementation:models:mental}
Our methods of zero-shot prompting (Sec.~\ref{sub:methods:zero-shot}) and few-shot prompting (Sec.~\ref{sub:methods:few-shot}) do not update model parameters during the experiment.
In contrast, instruction finetuning (Sec.~\ref{sub:methods:finetuning}) will update model parameters and generate new models.
To enhance its capability in the mental health domains, we update Alpaca on six tasks across the four datasets in Sec.~\ref{sub:implementation:datasets} using the multi-dataset instruction finetuning method (Sec.~\ref{subsub:methods:finetuning:multi-dataset}), which leads to our new model \textbf{Mental-Alpaca}.
Specifically, we merge the four datasets together and provide instructions for all six tasks (in the training set). We use eight Nvidia A100 GPUs for instruction finetuning. With cross entropy as the loss function, we backpropagate and update model parameters with the optimizer as Adam, the learning rate as 2$e^{-5}$ (cosine scheduler, warmup ratio 0.03), and the epoch number as 3.
% We open-source the finetuning pipeline and release the model parameter difference from LLaMA\footnote{Due to the open-source legislation of LLaMA, we cannot release model parameters directly. We adopt a similar practice of Stanford Alpaca~\cite{taori_stanford_2023} and only the parameter difference.} at [github link].

\subsection{Numerical Results}

% Figure environment removed


% Figure environment removed

% Figure environment removed
% Figure environment removed


% Figure environment removed

% Figure environment removed

\if11
% Figure environment removed
\fi

We present results using LM and NL1 for evaluating the approximations in \eqref{eq:approximations} and discuss the effect of 
the initial input $u_0$ and the decay rate of the form $\eta_k = \eta_0 k^{-c}$ on ILC performance.

\subsubsection{Effect of the model}
The results presented in Fig. \ref{fig:error-vs-iteration-init=xi} and Fig. \ref{fig:error-vs-iteration} show that for the iterations taken using gradient information from both the LM and the NL1, the error converges to a value of the same order of magnitude.
However, we see LM converge to a solution with lower deviation and at a faster rate compared to NL1.
This result is at first surprising given that the prediction error of the LM is one order of magnitude higher than NL1. We note however that NL1 is built with LeakyReLu activation functions, and thus its gradient is piecewise constant and discontinuous. Despite its higher prediction error, the structure of the LM is found to provide more accurate gradient information.
Different shapes and tunings of the cost function show mostly similar trends of the ILC loop using the linear and nonlinear models (data not shown).
We have observed that the system is only mildly nonlinear, and since the ILC step relies on measurements that do not depend on the models used, the fidelity of the approximations~\eqref{eq:approximations} is apparently not of critical importance. 
We hypothesize that a system with more pronounced nonlinearities would experience faster convergence with ILC steps relying on the nonlinear model for the approximations.
In Fig. \ref{fig:deviation-linearmodel} and Fig. \ref{fig:deviation-nonlinearmodel}, we show the output deviation as a function of time before and after $20$ steps of the proposed method. The deviations are computed between the output and the target trajectory depicted in Fig. \ref{fig:xy-detail-linearmodel}, where the output trajectories are plotted in $x-y$ coordinates.
Empirically, in previous studies, we found that the best rms error that can be obtained in this setup is close to the steady state values to which the ILC converges.
\subsubsection{Effect of initialization}
In Fig. \ref{fig:error-vs-iteration}, we show the results for two different initial conditions.
The results illustrate that since the underlying problem is nonlinear it is possible to be in a local minimum and not achieve a better solution, depending on the initial conditions.



\subsubsection{Effect of step size}


Next, we study the effect of step size on convergence behavior. We take $\eta_k = \eta_0 k^{-c}$ and compare the convergence.
We show the error trajectories for $c = \{0.2, 0.5, 0.9\}$ (Fig. \ref{fig:normalised-vs-iteration}).
In Fig. \ref{fig:rms-vs-c} we plot the error after $10$ and $50$ iterations for different values of $c$.
We observe that for $c$ between $0.3$ and $0.6$, we obtain fast decreases in the error without compromising the value at a steady state.


\if 01
\eb{
\begin{itemize}
    \item Comparison between different models starting from the same initial condition. 
    \item Compare to linear ILC and use the linear model for the Hessian etc. to show what we gain in simulation results
    \item Use the best-performing model and maybe do a plot on the effect of the initial condition to show the sensitivity to the initial condition, which should be part of the limitations of the method. Could make sense to compare against the linear ILC maybe? 
    \item (A side note that we could add only if we have time/space) If we have the 500ms NN as both model and system, we can show that the method (hopefully) almost converges to the optimization of the 500ms NN directly. This would be an additional motivation to use this framework for the optimization of such NN models instead of doing expensive optimization directly. 
\end{itemize}
}
\fi



The novel approach for mitigating numerical instability presented in this work has demonstrated promising results with respect to the backpropagation phase of deep learning models when dealing with a simple Linear Layer. Our modification to the gradient computation equation within the Adam optimizer ensures that the denominator stays within a safe range, thereby providing a stable update for the gradient. This has the potential to significantly enhance the stability of the training process, and by extension, the accuracy of the resultant model. However, as in all scientific studies, this work is not without its limitations. The empirical validation of our proposed method has been conducted with image classification model such as the vision transformer models. While this forms the basis for more complex architectures, it remains only one part of training of the wide array of layers and structures currently utilized in modern deep learning architectures. Specifically, future research should aim to validate the robustness and utility of our approach across a broader spectrum of deep learning architectures. In particular, Transformer models for Natural Language Processing is a significant portion of the current deep learning landscape, particularly in language-related tasks. The complexity of these models, with their intricate hierarchies and highly non-linear transformations, may pose additional challenges that are not fully encapsulated by a Linear Layer. As such, it is imperative that further validation is conducted on these architectures to establish the generalizability of our method. Furthermore, while our focus has primarily been on mitigating numerical instability, it would also be beneficial to investigate any potential side effects this approach might have on other aspects of the model training process. For instance, it would be interesting to explore the implications for training time, memory requirements, and robustness to variations in hyperparameters. In conclusion, this work presents an exciting step forward in the pursuit of more robust and stable training methodologies for deep learning models. The road ahead, while challenging, is filled with opportunities for further innovation and refinement.

\section{Conclusion}
\label{sec:conclusion}
In this paper, we present the first comprehensive evaluation of multiple LLMs (Alpaca, Alpaca-LoRA, GPT-3.5) on mental health prediction tasks (binary and multi-class classification) via online text data. 
Our experiments cover zero-shot prompting, few-shot prompting, and instruction finetuning. The results reveal a number of interesting findings.
Our context enhancement strategy can robustly improve performance for all LLMs, and our mental health enhancement strategy can enhance models with large number of trainable parameters.
Meanwhile, few-shot prompting can also robustly improve model performance even by providing just one example per class.
Most importantly, our experiments show that instruction finetuning across multiple datasets can significantly boost model performance on various mental health prediction tasks at the same time. Our best finetuned model Mental-Alpaca performs on par with the state-of-the-art task-specific model Mental-RoBERTa.
We summarize our findings as a set of guidelines for future researchers, developers, and practitioners who want to empower LLMs with better knowledge of mental health for downstream tasks.



% \section*{ACKNOWLEDGMENTS}


\bibliographystyle{ACM-Ref-Format}
\bibliography{
bib/BehaviorIntervention,
bib/HumanComputerInteraction,
bib/Modeling_Behavior-General,
bib/MachineLearning,
bib/OrsonPublication,
bib/others
}

% \newpage
\section*{Appendix: Detailed Results Tables}
\label{appendix:detail_results}

\renewcommand{\arraystretch}{1}
% \vspace{-0.5cm}
\begin{table}[!b]
\centering
% \vspace{-0.5cm}
\caption{
\review{Balanced Accuracy Performance Summary of Zero-shot, Few-shot and Instruction Finetuning on LLMs. 
$context$, $mh$, and $both$ indicate the prompt design strategies of context enhancement, mental health enhancement, and their combination  (see Table.~\ref{tab:prompt_design}).
Small numbers represent standard deviation across different designs of \textit{Prompt}$_{\textit{Part1-S}}$ and \textit{Prompt}$_{\textit{Part2-Q}}$. The baselines at the top rows do not have standard deviation as the task-specific output is static, and prompt designs do not apply.
Due to token limit, computation cost, and resource constraints, some infeasible experiments are marked as ``--''.
For each column, the best result is \textbf{bolded}, and the second best is \underline{underlined}.
}
}
% \vspace{-0.3cm}
\label{tab:results_overall}
\resizebox{1\textwidth}{!}{
\begin{tabular}{llccccccccc}
\thickhlinespace
& \makecell[r]{\textbf{Dataset}} & \textbf{Dreaddit} & \multicolumn{2}{c}{\textbf{DepSeverity}} & \textbf{SDCNL} & \multicolumn{2}{c}{\textbf{CSSRS-Suicide}} & \textbf{Red-Sam} & \textbf{Twt-60Users} & \textbf{SAD} \\ \addlinespace[1ex]
\textbf{Category} & \textbf{Model}   & \textbf{Task \#1}       & \textbf{Task \#2}             & \textbf{Task \#3}             & \textbf{Task \#4}    & \textbf{Task \#5}     & \textbf{Task \#6} & \textbf{Task \#2} & \textbf{Task \#2} & \textbf{Task \#1} \\ \thickhlinespace
\multirow{23}{*}{\makecell{Zero-shot\\Prompting}}  & Alpaca$_{ZS}$ & 0.593$_{\pm0.039}$ & 0.522$_{\pm0.022}$ & 0.431$_{\pm0.050}$ & 0.493$_{\pm0.007}$ & 0.518$_{\pm0.037}$ & 0.232$_{\pm0.076}$ & 0.524$_{\pm0.014}$ & 0.521$_{\pm0.022}$ & 0.503$_{\pm0.004}$\\
 & Alpaca$_{ZS-context}$ & 0.612$_{\pm0.065}$ & 0.567$_{\pm0.077}$ & 0.454$_{\pm0.143}$ & 0.497$_{\pm0.006}$ & 0.532$_{\pm0.033}$ & 0.250$_{\pm0.060}$ & 0.525$_{\pm0.019}$ & 0.559$_{\pm0.064}$ & 0.501$_{\pm0.004}$\\
 & Alpaca$_{ZS\_mh}$ & 0.593$_{\pm0.031}$ & 0.577$_{\pm0.028}$ & 0.444$_{\pm0.090}$ & 0.482$_{\pm0.015}$ & 0.523$_{\pm0.013}$ & 0.235$_{\pm0.033}$ & 0.527$_{\pm0.006}$ & 0.569$_{\pm0.017}$ & 0.522$_{\pm0.027}$\\
 & Alpaca$_{ZS\_both}$ & 0.540$_{\pm0.029}$ & 0.559$_{\pm0.040}$ & 0.421$_{\pm0.095}$ & 0.532$_{\pm0.005}$ & 0.511$_{\pm0.011}$ & 0.221$_{\pm0.030}$ & 0.495$_{\pm0.016}$ & 0.499$_{\pm0.004}$ & 0.557$_{\pm0.041}$ \\ \cdashlinespace{2-11}
& Alpaca-LoRA$_{ZS}$ & 0.571$_{\pm0.043}$ & 0.548$_{\pm0.027}$ & 0.437$_{\pm0.044}$ & 0.502$_{\pm0.011}$ & 0.540$_{\pm0.012}$ & 0.187$_{\pm0.053}$ & 0.577$_{\pm0.004}$ & 0.607$_{\pm0.046}$ & 0.477$_{\pm0.016}$\\
 & Alpaca-LoRA$_{ZS_context}$ & 0.537$_{\pm0.047}$ & 0.501$_{\pm0.001}$ & 0.343$_{\pm0.152}$ & 0.472$_{\pm0.020}$ & 0.567$_{\pm0.038}$ & 0.214$_{\pm0.059}$ & 0.535$_{\pm0.017}$ & 0.649$_{\pm0.021}$ & 0.443$_{\pm0.047}$\\
 & Alpaca-LoRA$_{ZS\_mh}$ & 0.500$_{\pm0.000}$ & 0.500$_{\pm0.000}$ & 0.331$_{\pm0.145}$ & 0.497$_{\pm0.025}$ & 0.557$_{\pm0.023}$ & 0.216$_{\pm0.022}$ & 0.541$_{\pm0.016}$ & 0.569$_{\pm0.019}$ & 0.471$_{\pm0.033}$\\
 & Alpaca-LoRA$_{ZS\_both}$ & 0.500$_{\pm0.000}$ & 0.500$_{\pm0.000}$ & 0.386$_{\pm0.059}$ & 0.499$_{\pm0.023}$ & 0.517$_{\pm0.031}$ & 0.224$_{\pm0.049}$ & 0.507$_{\pm0.009}$ & 0.535$_{\pm0.025}$ & 0.420$_{\pm0.019}$\\ \cdashlinespace{2-11}
 & FLAN-T5$_{ZS}$  & 0.659$_{\pm0.086}$ & 0.664$_{\pm0.011}$ & 0.396$_{\pm0.006}$ & 0.643$_{\pm0.021}$ & 0.667$_{\pm0.023}$ & 0.418$_{\pm0.012}$ & 0.554$_{\pm0.034}$ & 0.613$_{\pm0.040}$ & 0.692$_{\pm0.093}$\\
 & FLAN-T5$_{ZS_context}$ & 0.663$_{\pm0.079}$ & 0.674$_{\pm0.014}$ & 0.378$_{\pm0.013}$ & 0.653$_{\pm0.011}$ & 0.649$_{\pm0.026}$ & 0.378$_{\pm0.029}$ & 0.563$_{\pm0.029}$ & 0.613$_{\pm0.046}$ & 0.738$_{\pm0.056}$\\
 & FLAN-T5$_{ZS\_mh}$ & 0.616$_{\pm0.070}$ & 0.666$_{\pm0.009}$ & 0.366$_{\pm0.012}$ & 0.648$_{\pm0.010}$ & 0.653$_{\pm0.018}$ & 0.372$_{\pm0.033}$ & 0.547$_{\pm0.035}$ & 0.613$_{\pm0.033}$ & 0.739$_{\pm0.039}$\\
 & FLAN-T5$_{ZS\_both}$ & 0.604$_{\pm0.074}$ & 0.661$_{\pm0.004}$ & 0.389$_{\pm0.051}$ & 0.645$_{\pm0.005}$ & 0.657$_{\pm0.019}$ & 0.382$_{\pm0.048}$ & 0.536$_{\pm0.027}$ & 0.606$_{\pm0.040}$ & 0.767$_{\pm0.050}$\\ \cdashlinespace{2-11}
 & LLaMA2$_{ZS}$ & 0.720$_{\pm0.012}$ & 0.693$_{\pm0.034}$ & 0.429$_{\pm0.013}$ & 0.589$_{\pm0.010}$ & 0.691$_{\pm0.014}$ & 0.261$_{\pm0.018}$ & 0.574$_{\pm0.008}$ & 0.735$_{\pm0.017}$ & 0.704$_{\pm0.026}$\\
 & LLaMA2$_{ZS_context}$ & 0.658$_{\pm0.025}$ & 0.707$_{\pm0.056}$ & 0.410$_{\pm0.019}$ & 0.588$_{\pm0.026}$ & 0.722$_{\pm0.039}$ & 0.367$_{\pm0.043}$ & 0.562$_{\pm0.011}$ & \underline{0.736}$_{\pm0.019}$ & 0.650$_{\pm0.027}$\\
 & LLaMA2$_{ZS\_mh}$ & 0.617$_{\pm0.012}$ & 0.711$_{\pm0.033}$ & 0.395$_{\pm0.017}$ & 0.642$_{\pm0.008}$ & 0.696$_{\pm0.021}$ & 0.291$_{\pm0.038}$ & 0.572$_{\pm0.012}$ & 0.689$_{\pm0.056}$ & 0.567$_{\pm0.021}$\\
 & LLaMA2$_{ZS\_both}$ & 0.584$_{\pm0.017}$ & 0.704$_{\pm0.036}$ & 0.444$_{\pm0.021}$ & 0.643$_{\pm0.014}$ & 0.689$_{\pm0.043}$ & 0.328$_{\pm0.058}$ & 0.559$_{\pm0.012}$ & 0.692$_{\pm0.069}$ & 0.560$_{\pm0.009}$\\ \cdashlinespace{2-11}
 & GPT-3.5$_{ZS}$ & 0.685$_{\pm0.024}$ & 0.642$_{\pm0.017}$ & 0.603$_{\pm0.017}$ & 0.460$_{\pm0.163}$ & 0.570$_{\pm0.118}$ & 0.233$_{\pm0.009}$ & 0.454$_{\pm0.007}$ & 0.536$_{\pm0.024}$ & 0.717$_{\pm0.017}$\\
 & GPT-3.5$_{ZS_context}$ & 0.688$_{\pm0.045}$ & 0.653$_{\pm0.020}$ & 0.543$_{\pm0.047}$ & 0.618$_{\pm0.008}$ & 0.577$_{\pm0.090}$ & 0.265$_{\pm0.048}$ & 0.473$_{\pm0.001}$ & 0.560$_{\pm0.002}$ & 0.723$_{\pm0.003}$\\
 & GPT-3.5$_{ZS\_mh}$ & 0.679$_{\pm0.017}$ & 0.636$_{\pm0.021}$ & 0.642$_{\pm0.034}$ & 0.576$_{\pm0.001}$ & 0.477$_{\pm0.014}$ & 0.310$_{\pm0.015}$ & 0.467$_{\pm0.004}$ & 0.571$_{\pm0.000}$ & 0.664$_{\pm0.061}$\\
 & GPT-3.5$_{ZS\_both}$ & 0.681$_{\pm0.010}$ & 0.627$_{\pm0.022}$ & 0.617$_{\pm0.014}$ & 0.632$_{\pm0.020}$ & 0.617$_{\pm0.033}$ & 0.254$_{\pm0.009}$ & 0.506$_{\pm0.004}$ & 0.570$_{\pm0.007}$ & 0.750$_{\pm0.027}$\\ \cdashlinespace{2-11}
 & GPT-4$_{ZS}$ & 0.700$_{\pm0.001}$ & 0.719$_{\pm0.013}$ & 0.588$_{\pm0.010}$ & 0.644$_{\pm0.007}$ & 0.760$_{\pm0.009}$ & 0.418$_{\pm0.009}$ & 0.434$_{\pm0.005}$ & 0.566$_{\pm0.017}$ & \textbf{0.854}$_{\pm0.006}$\\
 & GPT-4$_{ZS_context}$ & 0.706$_{\pm0.009}$ & 0.719$_{\pm0.009}$ & 0.590$_{\pm0.011}$ & 0.644$_{\pm0.011}$ & 0.753$_{\pm0.028}$ & \underline{0.441}$_{\pm0.057}$ & 0.465$_{\pm0.010}$ & 0.565$_{\pm0.006}$ & \underline{0.848}$_{\pm0.001}$\\
 & GPT-4$_{ZS\_mh}$ & 0.725$_{\pm0.009}$ & 0.684$_{\pm0.004}$ & 0.656$_{\pm0.001}$ & 0.645$_{\pm0.012}$ & 0.737$_{\pm0.005}$ & 0.396$_{\pm0.020}$ & 0.496$_{\pm0.005}$ & 0.527$_{\pm0.007}$ & 0.840$_{\pm0.003}$\\
 & GPT-4$_{ZS\_both}$ & 0.719$_{\pm0.021}$ & 0.689$_{\pm0.000}$ & 0.650$_{\pm0.011}$ & 0.647$_{\pm0.014}$ & 0.697$_{\pm0.005}$ & 0.411$_{\pm0.009}$ & 0.511$_{\pm0.000}$ & 0.546$_{\pm0.014}$ & 0.837$_{\pm0.002}$\\
\thickhlinespace
\multirow{4}{*}{\makecell{Few-shot\\Prompting}} &  Alpaca$_{FS}$         & 0.632$_{\pm0.030}$ & 0.529$_{\pm0.017}$ & 0.628$_{\pm0.005}$ & ---   & ---                & ---                & ---             & ---                & ---                \\
& FLAN-T5$_{FS}$         & 0.786$_{\pm0.006}$ & 0.678$_{\pm0.009}$ & 0.432$_{\pm0.009}$ & ---                & ---                & ---   & ---                & ---                & ---             \\
& GPT-3.5$_{FS}$         & 0.721$_{\pm0.010}$ & 0.665$_{\pm0.015}$ & 0.580$_{\pm0.002}$ & ---                & ---                & --- & ---                & ---                & ---               \\
& GPT-4$_{FS}$         & 0.698$_{\pm0.009}$ & 0.724$_{\pm0.005}$ & 0.613$_{\pm0.001}$ & ---                & ---                & --- & ---                & ---                & ---                \\\thickhlinespace
\multirow{2}{*}{\makecell{Instructional\\Finetuning}} & Mental-Alpaca         & \underline{0.816}$_{\pm0.006}$ & \underline{0.775}$_{\pm0.006}$ & \underline{0.746}$_{\pm0.005}$ & \textbf{0.724}$_{\pm0.004}$ & 0.730$_{\pm0.048}$ & 0.403$_{\pm0.029}$ & \textbf{0.604}$_{\pm0.012}$ & 0.718$_{\pm0.011}$ & 0.819$_{\pm0.006}$\\
& Mental-FLAN-T5 &  0.802$_{\pm0.002}$ & 0.759$_{\pm0.003}$ & \textbf{0.756}$_{\pm0.001}$ & 0.677$_{\pm0.005}$ & \textbf{0.868}$_{\pm0.006}$ & \textbf{0.481}$_{\pm0.006}$ & \underline{0.582}$_{\pm0.002}$ & \textbf{0.736}$_{\pm0.003}$ & 0.779$_{\pm0.002}$ \\ \thickhlinespace
\multirow{3}{*}{\xspace\xspace\xspace Baseline} & Majority          & 0.500$_{\pm ---}$  & 0.500$_{\pm ---}$  & 0.250$_{\pm ---}$  & 0.500$_{\pm ---}$  & 0.500$_{\pm ---}$  & 0.200$_{\pm ---}$ & ---                & ---                & --- \\
& BERT              & 0.783$_{\pm ---}$  & 0.763$_{\pm ---}$  & 0.690$_{\pm ---}$  & 0.678$_{\pm ---}$  & 0.500$_{\pm ---}$  & 0.332$_{\pm ---}$ & ---                & ---                & --- \\
& Mental-RoBERTa     & \textbf{0.831}$_{\pm ---}$  & \textbf{0.790}$_{\pm ---}$   & 0.736$_{\pm ---}$  & \underline{0.723}$_{\pm ---}$  & \underline{0.853}$_{\pm ---}$  & 0.373$_{\pm ---}$ & ---                & ---                & --- \\ \thickhlinespace
\end{tabular}
\vspace{-0.5cm}
}
\end{table}
\renewcommand{\arraystretch}{1}

\end{document}