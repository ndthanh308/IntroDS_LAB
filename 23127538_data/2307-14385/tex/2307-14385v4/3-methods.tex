\section{Methods}
\label{sec:methods}
We introduce our experiment design with LMMs on multiple mental health prediction task setups, including zero-shot prompting (Sec.~\ref{sub:methods:zero-shot}), few-shot prompting (Sec.~\ref{sub:methods:few-shot}), and instruction finetuning (Sec.~\ref{sub:methods:finetuning}). These setups are model-agnostic, and we will present the details of language models and datasets employed for our experiment in the next section.

\subsection{Zero-shot Prompting}
\label{sub:methods:zero-shot}
The language understanding and reasoning capability of LLMs have enabled a wide range of applications without the need for any domain-specific data, but only providing appropriate prompts~\cite{kojima_large_2022,wei2021finetuned}.
Therefore, we start with prompt design for mental health tasks in a zero-shot setting.

The goal of prompt design is to empower a pre-trained general-purpose LLM to achieve good performance on tasks in the mental health domain. We propose a general zero-shot prompt template ($\textit{Prompt}_{ZS}$) that consists of four parts:
\begin{equation}
    \textit{Prompt}_{ZS} = \textit{TextData} + \textit{Prompt}_{\textit{Part1-S}} + \textit{Prompt}_{\textit{Part2-Q}} + \textit{OutputConstraint}
\label{eq:prompt-zs}
\end{equation}
where \textit{TextData} is the online text data generated by end-users. \textit{Prompt}$_{\textit{Part1-S}}$ provides specifications for a mental health prediction target. \textit{Prompt}$_{\textit{Part2-Q}}$ poses the question for LLMs to answer. And \textit{OutputConstraint} controls the output of models (\eg ``Only return yes or no'' for a binary classification task).

We propose several design strategies for \textit{Prompt}$_{\textit{Part1-S}}$, as shown in the top part of Table~\ref{tab:prompt_design}: (1) \textbf{Basic}, which leaves it as blank; (2) \textbf{Context Enhancement}, which provides more social media context about the \textit{TextData}; (3) \textbf{Mental Health Enhancement}, which inserts mental health concept by asking the model to act as an expert. (4) \textbf{Context \& Mental Health Enhancement}, which combines both enhancement strategies by asking the model to act as a mental health expert under the social media context.

As for \textit{Prompt}$_{\textit{Part2-Q}}$, we mainly focus on two categories of mental health prediction targets: (1) predicting critical mental states, such as stress or depression, and (2) predicting high-stake risk actions, such as suicide. We tailor the question description for each category. Moreover, for both categories, we explore binary and multi-class classification tasks
\footnote{We also conduct an exploratory case study on mental health reasoning tasks. Please see more details in Sec.~\ref{sub:results:reasoning}.}.
Thus, we also make small modifications based on specific tasks to ensure appropriate questions (see Sec.~\ref{sec:implementation} for our mental health tasks). The bottom part of Table~\ref{tab:prompt_design} summarizes the mapping.

\renewcommand{\arraystretch}{1}
\begin{table}[]
\centering
\caption{Prompt Design for Mental Health Recognition Tasks. Prompt Part 1 aims to provide a better specification for LLMs and Part 2 poses the questions for LLMs to answer. For Part 1, we propose three strategies: context enhancement, mental health enhancement, and the combination of both. As for Part 2, we design different content for multiple mental health problem categories and prediction tasks. For each part, we propose two to three versions to improve its variation.}
\label{tab:prompt_design}
\label{tab:my-table}
\resizebox{1\textwidth}{!}{
\begin{tabular}{ccl}
\thickhlinespace
 \multicolumn{2}{c}{\textbf{Strategy}} &
  \multicolumn{1}{c}{\textbf{Prompt Part 1 - S}} \\
\thickhlinespace 
\multicolumn{2}{c}{Basic} &
  \makecell[l]{$\bullet$ \{ blank \} } \\ \hlinespace
\multicolumn{2}{c}{Context Enhancement} &
  \makecell[l]{$\bullet$ This person wrote this paragraph on social media.\\$\bullet$ Consider this post on social media to answer the question.} \\ \hlinespace
\multicolumn{2}{c}{Mental Health Enhancement} &
  \makecell[l]{$\bullet$ As a psychologist, read the post on social media and answer the question.\\ $\bullet$ If you are a psychologist, read the post on social media and answer the question.} \\ \hlinespace
\multicolumn{2}{c}{Context \& Mental Health Enhancement} &
  \makecell[l]{ $\bullet$ This person wrote this paragraph on social media. As a psychologist, read the post on social media \\ \xspace\xspace\xspace\xspace and answer the question.\\ $\bullet$ This person wrote this paragraph on social media. As a psychologist, consider the mental well-being\\\xspace\xspace\xspace\xspace condition expressed in this post, read the post on social media, and answer the question.} \\
\thickhlinespace
\multicolumn{1}{c}{\textbf{Category}} &
  \multicolumn{1}{c}{\textbf{Task}} &
  \multicolumn{1}{c}{\textbf{Prompt Part 2 - Q}} \\
\thickhlinespace
\multirow{2}{*}{\makecell[c]{\\Mental state\\ (\eg stressed, \\ depressed)}} &
  \makecell[c]{Binary classification\\ (\eg yes or no)} &
  \makecell[l]{$\bullet$ Is the poster {[}stressed{]}?\\$\bullet$ Is the poster of this post {[}stressed{]}?\\$\bullet$ Determine if the poster of this post is {[}stressed{]}.} \\  \clinespace{2-3}
 &
  \makecell[c]{Multi-class classification\\ (\eg multiple levels)} &
  \makecell[l]{$\bullet$ Which level is the person {[}stressed{]}?\\$\bullet$ How {[}stressed{]} is the person?\\ $\bullet$ Determine how {[}stressed{]} the person is.} \\ \hlinespace
\multirow{2}{*}{\makecell[c]{\\Critical risk action\\ (\eg suicide)}} &
  \makecell[c]{Binary classification\\ (\eg yes or no)} &
  \makecell[l]{$\bullet$ Does the poster want to {[}suicide{]}?\\$\bullet$ Is the poster likely to {[}suicide{]}?\\$\bullet$ Determine if the poster of this post want to {[}suicide{]}.} \\  \clinespace{2-3}
 &
  \makecell[c]{Multi-class classification\\ (\eg multiple levels)} &
  \makecell[l]{$\bullet$ Which level of {[}suicide{]} risk does the person have?\\$\bullet$ How {[}suicidal{]} is the person?\\$\bullet$ Determine which level of {[}suicide{]} risk does the person have.}\\
\thickhlinespace
\end{tabular}
}
\end{table}
\renewcommand{\arraystretch}{1.0}

For both \textit{Prompt}$_{\textit{Part1-S}}$ and \textit{Prompt}$_{\textit{Part2-Q}}$, we propose several versions to improve its variability. We then evaluate these prompts on multiple LLMs on different datasets and compare their performance.

\subsection{Few-shot Prompting}
\label{sub:methods:few-shot}

In order to provide more domain-specific information, researchers have also explored few-shot prompting with LLMs by providing few-shot demonstrations to support in-context learning (\eg \cite{agrawal2022large,dang2022prompt}). Note that these few examples are used solely in prompts, and the model parameters remain unchanged. The intuition is to present a few ``examples'' for the model to learn domain-specific knowledge \textit{in situ}.
In our setting, we also test this strategy by adding additional randomly sampled [$\textit{Prompt}_{ZS} - \textit{label}$] pairs. The design of the few-shot prompt ($\textit{Prompt}_{FS}$) is straightforward:
\begin{equation}
    \textit{Prompt}_{FS} = [\textit{Sample Prompt}_{ZS} - \textit{label}]_{M} + \textit{Prompt}_{ZS}
\label{eq:prompt-fs}
\end{equation}
where $M$ is the number of prompt-label pairs and is capped by the input length limit of a model. Note that both the $\textit{Sample Prompt}_{ZS}$ and $\textit{Prompt}_{ZS}$ follow Eq.~\ref{eq:prompt-zs} and employ the same design of \textit{Prompt}$_{\textit{Part1-S}}$ and \textit{Prompt}$_{\textit{Part2-Q}}$ to ensure consistency.

\subsection{Instruction Finetuning}
\label{sub:methods:finetuning}

In contrast to the few-shot prompting strategy in Sec.~\ref{sub:methods:few-shot}, the goal of this strategy is closer to the traditional few-shot transfer learning, where we further train the model with a small amount of domain-specific data (\eg \cite{huang2022large,xu2021raise,liu_large_2023}).
We experiment with multiple finetuning strategies.

\subsubsection{Single-dataset Finetuning}
\label{subsub:methods:finetuning:single-dataset}
Following most of the previous work in the mental health field~\cite{yang_evaluations_2023,coppersmith_clpsych_2015,de_choudhury_discovering_2016}, we first conduct basic finetuning on a single dataset (the training set). This finetuned model can be tested on the same dataset (the test set) to evaluate its performance and different datasets to evaluate its generalizability.

\subsubsection{Multi-dataset Finetuning}
\label{subsub:methods:finetuning:multi-dataset}
From Sec.~\ref{sub:methods:zero-shot} to Sec.~\ref{subsub:methods:finetuning:single-dataset}, we have been focusing on one single mental health dataset $D$.
More interestingly, we further experiment with finetuning across multiple datasets simultaneously. Specifically, we leverage instruction finetuning to enable LLMs to handle multiple tasks in different datasets~\cite{brown_language_2020}.

It is noteworthy that such an instruction finetuning setup differs from the state-of-the-art mental-health-specific models (\eg Mental-RoBERTa~\cite{ji_mentalbert_2021}). The previous models are finetuned for a specific task, such as depression prediction or suicidal ideation prediction. Once trained on task A, the model becomes specific to task A and is only suitable for solving that particular task. In contrast, we finetune LLMs on several mental health datasets, employing diverse instructions for different tasks across these datasets in a single iteration. This enables them to handle multiple tasks without additional task-specific finetuning.

% It is noteworthy that such an instruction finetuning setup differs from the state-of-the-art mental-health-specific models (\eg Mental-RoBERTa~\cite{ji_mentalbert_2021}). The previous models are finetuned for a specific task, such as depression prediction or suicidal ideation prediction, or multiple predefined tasks in the multi-task setting. Once the model is trained on task(s) A, it becomes task-A-specific and only aims to solve task A. In contrast, we finetune LLMs on multiple mental health datasets with various instructions for different tasks across multiple datasets in a single round, empowering them to work on multiple tasks without the need for further task-specific finetuning.

For both single- and multi-dataset finetuning, we follow the same two steps:
\begin{align}
\begin{split}
\text{Step 1:} & \text{ Finetune with } [\textit{Prompt}_{ZS} - \textit{label}]_{\sum\limits^{I} N_{D_{i-train}}} \\
\text{Step 2:} & \text{ Test with } [\textit{Prompt}_{ZS}]_{\sum\limits^{I} N_{D_{i-test}}}
\end{split}
\label{eq:prompt-ft}
\end{align}
where $N_{D_{i-train}}$/$N_{D_{i-test}}$ is the total size of the training/test dataset $D_i$, $I$ represents the set of datasets used for finetuning, and $i$ indicates the specific dataset index ($i \in I, |I| \ge 1$). Both $\textit{Prompt}_{ZS\textit{-train}}$ and $\textit{Prompt}_{ZS\textit{-test}}$ follow Eq.~\ref{eq:prompt-zs}. Similar to the few-shot setup in Eq.~\ref{eq:prompt-fs}, they employ the same design of \textit{Prompt}$_{\textit{Part1-S}}$ and \textit{Prompt}$_{\textit{Part2-Q}}$.

% In addition, we also experiment with multiple variations:
% (1) \textbf{Data size variation}, where we sample a subset of each training dataset $D_{i-train}$ with a smaller $N'_{D_{i-train}}$ to explore the effect of data size on finetuning performance.
% (2) \textbf{Prompt variation}, where we randomly sample \textit{Prompt}$_{\textit{Part1-S}}$ from Table~\ref{tab:prompt_design} during the training to explore the effect of prompt diversity on finetuning performance.