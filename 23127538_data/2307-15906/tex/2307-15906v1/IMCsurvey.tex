% interactapasample.tex
% v1.05 - August 2017

\documentclass[]{interact}

\usepackage{epstopdf}% To incorporate .eps illustrations using PDFLaTeX, etc.
\usepackage[caption=false]{subfig}% Support for small, `sub' figures and tables
%\usepackage[nolists,tablesfirst]{endfloat}% To `separate' figures and tables from text if required
%\usepackage[doublespacing]{setspace}% To produce a `double spaced' document if required
%\setlength\parindent{24pt}% To increase paragraph indentation when line spacing is doubled
%\usepackage[slovak]{babel}

\usepackage[T2A,T1]{fontenc}

\usepackage[longnamesfirst,sort]{natbib}% Citation support using natbib.sty
\bibpunct[, ]{(}{)}{;}{a}{,}{,}% Citation support using natbib.sty
\renewcommand\bibfont{\fontsize{10}{12}\selectfont}% To set the list of references in 10 point font using natbib.sty

%\usepackage[natbibapa,nodoi]{apacite}% Citation support using apacite.sty. Commands using natbib.sty MUST be deactivated first!
%\setlength\bibhang{12pt}% To set the indentation in the list of references using apacite.sty. Commands using natbib.sty MUST be deactivated first!
%\renewcommand\bibliographytypesize{\fontsize{10}{12}\selectfont}% To set the list of references in 10 point font using apacite.sty. Commands using natbib.sty MUST be deactivated first!

\theoremstyle{plain}% Theorem-like structures provided by amsthm.sty
\newtheorem{theorem}{Theorem}[section]
\newtheorem{lemma}[theorem]{Lemma}
\newtheorem{corollary}[theorem]{Corollary}
\newtheorem{proposition}[theorem]{Proposition}

\theoremstyle{definition}
\newtheorem{definition}[theorem]{Definition}
\newtheorem{example}[theorem]{Example}

\theoremstyle{remark}
\newtheorem{remark}{Remark}
\newtheorem{notation}{Notation}

\newcommand{\rd}{R\&D}

\begin{document}

%\articletype{ARTICLE TEMPLATE}% Specify the article type or omit as appropriate

\title{On the long-term impact of non-formal learning in particle physics}

\author{
\name{J\'ulia Kekel\'akov\'a\textsuperscript{a} and Boris Tom\'a\v{s}ik\textsuperscript{a,b}\thanks{Email: boris.tomasik@cvut.cz}}
\affil{\textsuperscript{a}Univerzita Mateja Bela, Fakulta pr\'irodn\'ych vied, Tajovsk\'eho 40, 97401 Bansk\'a Bystrica, Slovakia; \\
\textsuperscript{b}\v{C}esk\'e vysok\'e u\v{c}en\'i technick\'e v Praze, Fakulta jadern\'a a fyzik\'aln\v{e} in\v{z}en\'yrsk\'a, B\v{r}ehov\'a 7, 
11519 Praha 1, Czech Republic}
}

\maketitle

\begin{abstract}
Since 2005, the global flagship of outreach activities in high-energy 
physics have been 
the International Particle Physics Masterclasses. We report on a 
survey performed among the participants from Slovakia and the 
Czech Republic, where we have studied the impact of Masterclasses
on their further careers and their attitude towards science and 
especially particle physics. More than a half of our respondents 
does not work in science or research and development. However, 
most of them report positive shift in their attitude towards science. 
A positive nudge to physics career is indicated among  
those being open to such a possibility. 
\end{abstract}

\begin{keywords}
International Particle Physics Masterclasses, Motivation
\end{keywords}


\section{Introduction}

In various areas of active (physics) research, often a large gap exists between the cutting edge results and the knowledge taught at secondary 
schools or distributed among the general public. This gap is partially filled by non-formal learning and outreach activities. 

High-energy physics, a.k.a.\ particle physics,  has been traditionally communicated rather intensely also outside of the field, 
as there has often been demand for the knowledge by the public.
The reason is perhaps that HEP pursuits the most fundamental principles of our Universe, which is perceived as very exciting. 
Nevertheless, the appearance  of this research 
field and its relation to the public are also very important for its 
own sustainability. Firstly, the complexity of the problems and
the scale of HEP experiments require rather large numbers of top-qualified workforce which in longer-term 
perspective are recruited among the 
young people graduating from secondary schools. 
Secondly, particle physics belongs to those branches of science that eventually require building large and costly infrastructures. Hence, the support from the governments---representing the tax-paying citizens---is 
vital. 
Outreach is the part of the strategy that addresses these issues. Thus the motivation for outreach is on all sides. 


Since 2005---the 100th anniversary of Einstein's Annus Mirabilis---one of the internationally most important outreach activities in HEP
has been developed: the International Particle Physics Masterclasses (IMC). Institutes from Slovakia and the Czech Republic have participated 
in this initiative since its beginning. 

Due to its general attraction and longevity one could expect that IMC have already influenced a generation of young people and 
motivated some of the careers towards HEP, or generally to STEM. Furthermore, it could have contributed to the perception of science
among those who chose non-STEM careers. 
These are interesting hypotheses and until now we have had no data against which they could have been confronted. 
In this study, we report on a survey that addresses these hypotheses and demonstrates the long-term impact of IMC. 

Our survey shows that while more participants chose careers outside science or research\&development (\rd), the majority maintains positive
attitude towards science. The recollections about IMC are mostly very positive.

We describe the International Particle Physics Masterclasses in the next Section. In Section \ref{s:data} we explain how data were collected
on which our survey is based. Results are presented in Section \ref{s:results} and the conclusions are summarised in Section \ref{s:conc}.

%%%%%%%%%%%%%%%%%%%%%%%%%%%%%%%

\section{International Particle Physics Masterclasses}
\label{s:IMC}

The principal idea that governs the Masterclasses is to provide the upper secondary level pupils the genuine experience 
of particle physics. The advertising mottos `Hands on CERN' and `Become a particle physicist for one day' give the first idea about its 
motivation and agenda. 
The event (almost) always takes place at a university or research institute that runs active research programme in particle physics. 
One of the aims is to also bring the pupils into the environment where research is performed. 

The agenda of the event is planned for one day and the highlight is an analysis of data collected by one of the major particle 
physics experiments. Participants thus get hands-on experience mimicking the work of particle physicists. 

No previous knowledge is necessary in order to participate in IMC. Therefore, the morning programme usually includes lectures
that provide the elementary knowledge about theoretical concepts of particle physics and explain the basic experimental techniques
used. 

After the lunch, the actual hands-on activity is introduced and explained.
Participants work on it individually or in pairs, and it takes about two hours. 
Currently, all four large experiments at the LHC (CERN) offer activities that are being used in Masterclasses. 
Some of the collaborations even prepared two different activities. 
In addition to CERN experiments there are 
exercises prepared by BELLE~II (KEK Laboratory, Japan), MINER$\nu$A (Fermi National Laboratory, USA), 
Pierre Auger (cosmic ray detection array, Argentina) and one activity prepared by GSI Darmstadt on particle therapy 
in oncology.

The international character of the research in particle physics is also illustrated in the subsequent videoconference. There, 
up to five institutes connect and discuss---together with moderators from a major particle physics laboratory---their results. The videoconference
is arranged by the global management of the event in such a way, that at all participating institutes the same hands-on activity 
has been performed. The participants in the videoconference may also pose questions about particle physics and related issues 
in general, which are answered by the moderators---who are professional physicists themselves. 

The hands-on activity is rather involved. Such an endeavor, together with the international coordination of the videoconferences
is only possible thanks to the joint effort managed by the International Particle Physics Outreach Group (IPPOG)\footnote{{https://ippog.org}}.
Formally, IPPOG is an international collaboration, hosted by CERN, and facilitating exchange of ideas as well as coordinating outreach 
activities in the participating countries and with the major particle physics laboratories and experiments. 

For the first time on international level, IMC were organised by European Particle Physics Outreach Group (EPPOG) in 2005. 
It was in 2010 that EPPOG evolved internationally and became IPPOG.
Today, each year about 13,000 pupils participate in IMC that take place at some 220 institutes from 55 countries. 

More details about IMC can be found in the literature e.g.\ by \cite{Kobel05,Foka13,Bilow14,Bilow22}, 
and the information about IPPOG can be obtained from their web-site. 


%%%%%%%%%%%%%%%%%%%%%%%%%%%%%%%%

\section{Data collection}
\label{s:data}

First survey which focussed on the performance of IMC was undertaken 
already in 2005 and reported by \cite{Kobel05}.

Both Czech Republic and Slovakia have been participating in IMC since its first edition in 2005. Between 2011 and 2015
surveys were performed at most of the participating institutions in Slovakia and once also at the Czech Technical University in Prague,
Czech Republic. Those surveys were mainly focused on the 
educational background of the participants and the assessment of the IMC 
with the aim to further develop and improve the event. Results were partly summarised by \cite{Beniacikova11,Tomasik13,Cecire17}
and/or used internally. 

The surveys were performed by means of answer sheets that were (usually) distributed and collected just after the end of 
IMC while the participants were still at the venue. In one case 
two questionnaires were collected---one before and one after the IMC---in order 
to directly measure the impact of IMC. The total number of collected sets of answers over all years is somewhat below 1000. The 
surveys were anonymous. Nevertheless, in the end we suggested that if the participants agreed, they could leave us with their email addresses 
for the purpose of a later survey. 

In April and May 2023 we have performed two new surveys, from which we present results in this paper. 

\paragraph*{Survey 1} has been performed among former participants of IMC, who  left their email addresses 
with us. In this way we have retrieved 484 email addresses from the archived answer sheets. 

We administered an anonymous 
online questionnaire that has been implemented with the help of google forms. In included 10 questions. 
We deliberately kept the form short in order not to discourage the participants from filling it in. The questions were in Slovak for participants 
from both Slovakia and Czech Republic. The questions, translated to English, are listed in Table~\ref{t:q1}.
%
%%%%%%%%%%
%
\begin{table}[t]
\caption{Questions asked in the questionnaire of Survey 1.}
%
\centerline{
\begin{tabular}{| c | p{0.595\textwidth} | p{0.24\textwidth} |} 
\hline 
& Question & Mode of answer \\ 
\hline\hline
Q1.1 & In which year have you participated in Masterclasses? & multiple choice \\
\hline
Q1.2 & At which place have you participated in Masterclasses? & multiple choice \\ 
\hline
Q1.3 & I came to Masterclasses with the ambition to study particle or nuclear physics. 
& 5-point Likert scale \\
\hline
Q1.4 & Masterclasses have influenced my decision to study physics or related subject. & 5-point Likert scale \\
\hline
Q1.5 & I contribute to the development of science---I am scientific associate, or I work for a company focussed on R\&D, or 
I want to work in this field after I finish my study. & 5-point Likert scale \\
\hline
Q1.6 & The content of my current employment or study is: &  short free text  \\
\hline
Q1.7 & Thanks to Masterclasses I positively changed my opinion about science, research, and physics. &  free text  \\
\hline
Q1.8 & I am interested in the news from CERN and/or news from science and particle physics. & 5-point Likert scale \\
\hline
Q1.9 & If you want to leave us a message, you can type it here. & long free text \\
\hline
Q1.10 & This is the end of the questionnaire. If you agree with the comparison of your answers with the answers that you gave 
after the event, leave us your email address here. Thank you! & short free text \\
\hline
\end{tabular}
}
\label{t:q1}
\end{table}
%
%%%%%%%%%%
%
The first two questions serve statistic purposes, mainly. Q1.3 aims at reconstruction of the motivation prior to experiencing IMC. 
Questions Q1.4--Q1.8 explore the actual impact of IMC as perceived today. We will mention the comments that we received in 
Q1.9 below, and look at the correlation between the previous survey and this one thanks to emails collected in Q1.10.

Out of the 484 invitations to our survey, 133 messages bounced back as undeliverable, hence we assume that 351 messages 
were delivered. From those, 71 participants filled our online questionnaire. This is the sample we work with. Furthermore, 36 people 
revealed their email addresses and we were able to connect their responses with the ones that were collected in the past. 

\paragraph*{Survey 2} was performed in parallel to Survey 1 with the aim to identify if and how  IMC influenced young researchers 
in HEP. In this case, we asked the senior faculty members and team leaders at relevant Czech and Slovak institutions to 
forward our invitation to the study to younger colleagues. The participation could not be enforced, thus the sample may be biased 
and possibly not all eligible people may have responded. We collected 19 responses in total. This number is to be compared 
with the relevant size of the community, which we estimate as 100 young HEP practitioners in the Czech Republic and 25 in Slovakia, 
totalling to 125. 
The numbers include young colleagues that---due to their age---could have participated in IMC and can be clearly assigned to 
research in HEP. This results in a group from master study level up to about 35 years of age. The numbers were estimated  after 
requesting the (approximate) headcount from team leaders at all relevant institutions. 

Since our surveys were anonymous, unless the respondents volunteered to 
reveal their identity, we do not have information about possible overlap between participants to Survey 1 and Survey 2. From anecdotal discussions we know that there is some overlap, but our method does not allow for a more detailed information. 

%
%%%%%%%%%%
%
\begin{table}[t]
\caption{Questions asked in the questionnaire of Survey 2.}
%
\begin{tabular}{| c | p{0.595\textwidth} | p{0.24\textwidth} |} 
\hline 
& Question & Mode of answer \\ 
\hline\hline
Q2.1 & In which year have you participated in Masterclasses? & multiple choice \\
\hline
Q2.2 & At which place have you participated in Masterclasses? & multiple choice \\ 
\hline
Q2.3 & I came to Masterclasses with the ambition to study particle or nuclear physics. 
& 5-point Likert scale \\
\hline
Q2.4 & Masterclasses have influenced my decision to study particle physics. & 5-point Likert scale \\
\hline
Q2.5 & Later I have helped with the organisation of the Masterclasses at the institution where I studied or worked. 
& 5-point Likert scale \\
\hline
Q2.6 & I would like to help with the organisation of Masterclasses in the future. & 5-point Likert scale \\
\hline
Q2.7 & In particle or nuclear physics I am rather focussed on... & 
multiple choice from:\newline experiment; theory; \newline phenomenology; nothing yet, because I study
\\
\hline
Q2.8 & My current professional status could be described as... & 
radio buttons with the possibilities:\newline student (bachelor or master level);\newline doctoral student;\newline postdoc;\newline 
scientific staff;\newline university teacher; \newline other\\
\hline
Q2.9 & If you want to leave us with a message concerning Masterclasses, type it here. We will be grateful, e.g., for proposals 
what can be improved. In any case, thank you for filling the questionnaire.
& long free text \\
\hline
\end{tabular}

\label{t:q2}
\end{table}
%
%%%%%%%%%%
%
The questions of Survey 2 are summarised in Table~\ref{t:q2}. Part of them is similar to Survey 1, but in case of Survey 2 we know 
that the answers are given by practitioners in HEP. The other part of the questions thus rather  aims  on the research focus of the participants
and the attitude to currently organised IMC. 

For brevity, below we will refer to the respondents of Survey 1 as \emph{fans}, while respondents of Survey 2 will be called \emph{practitioners}.

%%%%%%%%%%%%%%%%%%%%%%%%%%%%%%%%

\section{Results}
\label{s:results}

%
%%%%%
%
% Figure environment removed
%
%%%
%
Histograms in Fig.~\ref{f:years} show the distribution of years when our respondents participated in IMC. Based on our data 
collection procedure it is not surprising that the fans mostly participated between 2011 and 2015, when the previous survey 
was performed. Individual cases outside of this time interval either indicate multiple participation to IMC by one respondent, 
or erroneous assignment of the year. There were 12 responses which could not recall the actual year and they are not included 
here. The distribution of years is wider for the practitioners. Participation after 2019 indicates the survey leaked also to 
current bachelor-level students. We decided to keep their responses in the sample, thus making it more informative. 

%
%%%%%
%
% Figure environment removed
%
%%%
%
Next, we study to what extent IMC have directed our respondents towards a career in particle physics. In Figure~\ref{f:bars-motiv}
we combine the results from both surveys. In the general group of fans, people with the ambition to go to particle physics 
make up slight minority. Not surprisingly, more than a half of the practitioners were motivated to proceed to particle physics 
after they experienced IMC. 
Interestingly enough, according to Q2.3, 
6 out of 19 practitioners had no such ambition before coming to IMC. 
Since they did end up as particle physicists, this means that they have changed their minds later. The impact of IMC
in directing towards particle physics is also quite different in the two groups. While practitioners were clearly influenced, 
slight majority of the fans feel rather not influenced. 


%
%%%%%
%
% Figure environment removed
%
%%%
%
In order to understand the motivating effect better, we look in Figure~\ref{f:amb-mot} into the correlation of answers to 
Q1.3 and Q1.4, i.e., questions about the pre-existing ambition to study particle physics and the influence of IMC on personal 
motivation, for the sample of fans. The histogram shows a clear peak for people neither planning nor being influenced towards 
particle physics. Nevertheless, on the 'yes'-side of the histogram we see a hint of anti-correlation between the ambition 
(Q1.3) and decision influence (Q1.4): people with ambivalent attitude towards particle physics seem to be nudged towards it
while those with positive attitude report no additional influence. 

There are not enough data to produce a similar plot for the practitioners, hence we refrain from it. 

%
%%%%%
%
% Figure environment removed
%
%%%
%
We were interested in further evolution of the careers of former IMC participants. As can be seen in Figure~\ref{f:careers},
slightly more than a half of the fans actually work outside science and R\&D. A more detailed information is summarised in 
Figure~\ref{f:professio}.
%
%%%%%
%
% Figure environment removed
%
%%%
%
By far the most populated profession group is related to IT. Another prominent group can be identified if we put 
together all healthcare-related professions. The education group mostly includes science teachers. In general, STEM-related 
careers prevail, with only 10 out of 71 respondents falling clearly out of this field, i.e., to social sciences, humanities, economy, 
or business. 

Linking the current responses with those collected about ten years ago allows us to gain some insight into the evolution of 
career plans. The correlation of previous plans with current reality is analysed in Figure~\ref{f:car-corr}.
%
%%%%%
%
% Figure environment removed
%
%%%
%
It is based on the 36 responses that could have been linked. We divided all professions into 10 groups, see the caption. 
If there was no change of plans, data would be aligned along the diagonal. Such a trend is roughly visible, with two pronounced
peaks; the main for IT and the next for  physics. In addition to the diagonal there are hints of two more effects. Firstly, 
there is a group of respondents who originally planned various careers but ended up with  physics. Secondly, 
a similar, though smaller group ended up finally in IT. 
 

%
%%%%%
%
% Figure environment removed
%
%%%
%
Another focus of our surveys was  in the perception of particle physics and science, in general. 
We summarise in Figure~\ref{f:attid} the answers to questions Q1.7, and Q1.8, which aim at the positive change 
in the attitude to science and research, and the interest in CERN and particle physics in general, respectively. 
In case of the attitude we see a positive effect of the IMC. In case of the interest in particle physics this is less pronounced. 

%
%%%%%
%
% Figure environment removed
%
%%%
%
It is interesting to see how the answers to Q1.7 and Q1.8 depend on the professional orientation of the respondent. To this end, 
we show in Figure~\ref{f:att-corr} two-dimensional histogram that combines answers to these questions with those to Q1.5. 
The histograms demonstrate that both the positive shift in attitude to science and the interest in CERN and particle physics 
are more pronounced with respondents working in science or R\&D. 

Survey 2 also clearly showed that the practitioners who participated in IMC are very happy to help with IMC at their institutes. 
Most of them work on experiments, in comparison  to theory or phenomenology, which is usual distribution in HEP. 

We would like to close this section with the messages that the respondents left us as free text. They unanimously evaluated IMC very positively
even if they decided to choose a different career path. Some of these decisions were motivated by the financial attractiveness of the IT sector. 

These answers illustrate that there seems to be a more profound idea that is communicated indirectly in the IMC events. 
It is related to the meaningfulness of the devoted (scientific) work, which is surely transferable beyond particle physics and is much 
more general asset in life. 

Since it may be interesting, where possible we identified the gender of the respondent, which in Slovak language can be inferred from 
the form of the verbs used in the answers. There were 5 responses from males and 7 from females, while 8 responses could not be uniquely 
assigned. We did not collect the data on gender because we were not interested in this aspect. Nevertheless, the observed distribution 
is in agreement with our experience that the participants to IMC are rather balanced. 

Selected messages follow: 

\begin{itemize}
\item 
'Very good event. (IT pays better.)`\\
('Ve\v{l}mi dobr\'e podujatie. (IT sektor platí lep\v{s}ie.)`)
\item
'As a teacher I would recommend it to pupils interested in physics.`\\ 
('Ako učiteľka by som to odporučila absolvovať žiakom so záujmom o fyziku.`)
\item 
'I participated in Masterclasses 12 years ago and I still recollect it very well. It was tremendous experience for me as a student who was interested in physics and planned to study this specialisation. It showed me new possibilities and I gained new encouragement to proceed this way also to the university. I am thankful for this opportunity.`\\
('Podujatie Masterclasses som absolvovala pred 12timi rokmi a doteraz si na to veľmi dobre spomínam. Bola to úžasná skúsenosť pre mňa ako študentku, ktorá sa zaujímala o fyziku a mala v pláne ist študovať tento odbor. Ukázalo mi to nové možnosti a nabrala som ešte väčšie odhodlanie ísť touto cestou aj na vysokej škole. Som vďačná za túto príležitosť.`)
\item 
'In my opinion, Masterclasses are a superb event. I would not say that it changed my opinion about science; I held it positive already before. It gave me kind of a first practical contact with science, how it works, what is the state of the art and what are open questions in the given area. It also helped me to grasp some things, that we learned in secondary school, because perhaps even our teachers did not understand them so deeply, so that they could present it so simply and logically. It was a contact with the fact that if you pursue something more deeply, it could make sense. I think that Masterclasses make sense, even if you finally decide to study something else than particle physics. Today, many projects are interdisciplinary and then it is an advantage if you had the possibility to get a flavor of the  terminology and get some basic knowledge also from different specialisations.`\\
('Masterclasses je pod\v{l}a m\v{n}a super podujatie. Nepovedala by som, \v{z}e by mi to zmenilo n\'azor na vedu, ten som mala u\v{z} aj pred t\'ym pozit\'ivny. Dalo mi to tak\'y prv\'y praktick\'y kontakt s vedou, ako prebieha a \v{c}o state of the art a \v{c}o s\'u open questions v danej oblasti. Tie\v{z} mi to pomohlo pochopi\v{t} niektor\'e veci, pochopi\v{t}, ktor\'e sme sa u\v{c}ili na strednej, lebo asi ani na\v{s}i u\v{c}itelia tomu tak do h\'lbky nerozumeli, aby to vedeli jednoducho a logicky poda\v{t}. Bol to kontakt s t\'ym, \v{z}e ke\v{d} sa \v{c}lovek nie\v{c}omu venuje trochu viac do h\'lbky, m\^o\v{z}e to d\'ava\v{t} zmysel. Mysl\'im si, \v{z}e projekt Masterclasses m\'a zmysel, aj ke\v{d} sa \v{c}lovek nakoniec rozhodne robi\v{t} a \v{s}tudova\v{t} nie\v{c}o in\'e ako \v{c}asticov\'u fyziku. Ve\v{l}a projektov je teraz medziodborov\'ych a tam je v\'yhodou, ak m\'a \v{c}lovek mo\v{z}nos\v{t} na\v{c}uchn\'u\v{t} aj do terminol\'ogie a z\'iska\v{t} nejak\'e z\'akladn\'e vedomosti aj z in\'ych odborov.`)
\item 
'I would like to praise the organisation of this event. During my study I participated in such events joyfully and often, and this one particularly stayed in my memory, because it seems to me that there I learned in a short time a lot, and it certainly increased my interest for this field, even though later---perhaps rather due to practical reasons---I decided to study informatics. Thank you!\hspace{0pt}`\\
('Chcem pochv\'ali\v{t} organiz\'aciu tohoto podujatia. Po\v{c}as \v{s}t\'udia som sa podobn\'ych akci\'i z\'u\v{c}ast\v{n}ovala rada a \v{c}asto, a tento mi obzvl\'a\v{s}\v{t} utkvel v pam\"ati, lebo m\'am pocit , \v{z}e som sa tam za kr\'atky \v{c}as nau\v{c}ila ve\v{l}a a ur\v{c}ite to v tom \v{c}ase zv\'y\v{s}ilo m\^oj z\'aujem o tento obor, hoci som sa nesk\^or, mo\v{z}no viac z praktick\'ych d\^ovodov, rozhodla pre \v{s}t\'udium informatiky. V\v{d}aka!\hspace{0pt}`)
\item
'Even though it was a long time ago, I have very good recollections of Masterclasses and I think that such activities are of great significance. Even if I did not study physics at last, but informatics, CERN still bugged me and finally I got there as a fellow and spent three years in the IT division. Just by chance, right now I am sitting in a train to Geneva on my way to visit my former colleagues. :)` \\
('Aj keď to už bolo dávno, na Masterclasses mám veľmi dobré spomienky a myslím si, že takéto aktivity majú veľký význam. Aj keď som nakoniec nešla študovať fyziku, ale informatiku, CERN mi zostal ako chrobák v hlave a nakoniec som sa tam dostala na fellowship a strávila som tri roky v IT oddelení. Zhodou okolností práve sedím vo vlaku do Ženevy a idem pozrieť bývalých kolegov :)`)
\end{itemize}
%%%%%%%%%%%%%%%%%%%%%%%%%%%%%%%%%

\section{Conclusions}
\label{s:conc}

We found an indication in our data that International Particle Physics Masterclasses do have a nudging effect on participants 
who do not exclude the possibility of becoming a scientist or even particle physicist. 
They appear as very effective learning environment where in short time great experience is provided to those who participate. 
 
 Nevertheless, a larger portion of former participants does not pursue scientific career, or a career in R\&D. IMC then provide a kind of 
 cultural transfer in which the (good) practices followed in particle physics are exported to other fields and professions. In line
 with this, IMC contributes to the generally positive acceptance of science. 
 
 The recruiting function is often stressed and even mentioned frequently in an IMC event. ('You can become particle physicist and 
 solve these open problems!\hspace{0pt}`)
 Nevertheless, it should be acknowledged and appreciated that perhaps more important impact of IMC is in the export of knowledge 
 and culture beyond particle physics and enriching the future generation of citizens. 


%%%%%%%%%%%%%%%%%%%%%%%%%%%%%%%%

\section*{Acknowledgements}

We thank Ivan Melo and Vojt\v{e}ch Pleskot for insightful discussions that helped us in performing the reported survey, as well as critical reading and comments to the manuscript. 
We are grateful to the colleagues who provided inputs and forwarded our requests to participate in Survey 2: Pavol Barto\v{s}, Jaroslav Biel\v{c}\'ik,
Marek Bombara, Peter Chochula, Michal Mere\v{s}, Vojt\v{e}ch Pleskot, Marek Ta\v{s}evsk\'y, Martin Venhart.
We also thank the students who helped us with processing data from the answer sheets: Nat\'alia Dzia{\l}ak and Dominika Kru\v{z}\'ikov\'a. 

%%%%%%%%%%%%%%%%%%%%%%%%%%%%%%%%

\section*{Funding}

IMC and this study have been supported by the Ministry of Education, Science, Research and Sports of the Slovak Republic. 


%%%%%%%%%%%%%%%%%%%%%%%%%%%%%%%%


\begin{thebibliography}{}

\bibitem[Benia\v{c}ikov\'a and Kri\v{s}kov\'a(2011)]{Beniacikova11}
Benia\v{c}ikov\'a, M., Kri\v{s}kov\'a, K., (2011). 
Zhodnotenie semin\'ara Masterclasses. (in Slovak)
\emph{Proceedings from the Student Conference, \v{S}VK 2011, 19.-20.5.2011, Ko\v{s}ice, Slovakia}.

\bibitem[Bilow and Kobel(2014)]{Bilow14}
Bilow, U., Kobel, M. (2014). International Masterclasses - bringing LHC data to school children. 
\emph{EPJ Web of Conferences}, \emph{71}, 00017.

\bibitem[Bilow and Cecire(2022)]{Bilow22}
Bilow, U., Cecire, K. (2022). International Masterclasses: Forward from Pandemic.
\emph{Proceedings of Science}. PoS(ICHEP2022)381.


\bibitem[Cecire et al.(2017)]{Cecire17}
Cecire, K., Melo, I., Tom\'a\v{s}ik, B.\ (2017). 
Bringing particle physics into classrooms.
\emph{Proceedings of the conference Physics Teaching in Engineering Education, PTEE 2017},
University of \v{Z}ilina, May 18-19, 2017. 

\bibitem[Foka(2013)]{Foka13}
Foka, P. (2013). IPPOG report on Masterclasses - Bringing LHC data into the classroom. 
\emph{Proceedings of Science}. PoS(Confinement X)029.

\bibitem[Kobel(2005)]{Kobel05}
Kobel, M.\ (2005). 
Masterclasses spreads the world for physics. 
\emph{CERN Courier}, \emph{vol. 45}, available from 
https://cerncourier.com/a/masterclass-spreads-the-word-for-physics/

\bibitem[Tom\'a\v{s}ik and Goceliakov\'a(2013)]{Tomasik13}
Tom\'a\v{s}ik, B., Goceliakov\'a, L.\ (2013). 
Majstrovské triedy z časticovej fyziky 2012: ohlas účastníkov. (in Slovak) 
\emph{Obzory matematiky, fyziky a informatiky}, \emph{42}, 53-62.




\end{thebibliography}


\end{document}


%%%%%%%%%%%%%%%%%%%%%%%%%%%%%%%%%%%%%%%%%%%%%%%%%%%%%%%%%%%%%%%%%%%%%%%%%%%%%%%%%%%%%%%%%%%%%%%%%%%%%%%%%%%%%%%%%%%%%%%%%%%%%%%%%%%%%%%%%%%%%%%%%%%%%%%%%%%%%%%%%%%%%%%%%%%%%%%%%%%%%%%%%%%%%%%%%%%%%%%%%%%%%%%%%%%%%%%%%%%%%%%%%%%%%%%%%%%%%%%%%%%%%%%%%%%%%%%%%%%%%%%%%%%%%%%%%%%%%%%%%%%%%%%%%%%%%%%%%%%%%%%%%%%%%%%%%%%%%%%%%%%%%%%%%%%%%%%%%%%%%%%%%%%%%%%%%%%%%%%%%%%%%%%%%%%%%%%%%%%%%%%%%%%%%%%%%%%




In order to assist authors in the process of preparing a manuscript for a journal, the Taylor \& Francis `\textsf{Interact}' layout style has been implemented as a \LaTeXe\ class file based on the \texttt{article} document class. A sample bibliography is also provided in order to assist with the formatting of your references.

Commands that differ from or are provided in addition to standard \LaTeXe\ are described in this document, which is \emph{not} a substitute for a \LaTeXe\ tutorial.

The \texttt{interactapasample.tex} file can be used as a template for a manuscript by cutting, pasting, inserting and deleting text as appropriate, using the preamble and the \LaTeX\ environments provided (e.g.\ \verb"\begin{abstract}", \verb"\begin{keywords}").


\subsection{The \textsf{Interact} class file}\label{class}

The \texttt{interact} class file preserves the standard \LaTeXe\ interface such that any document that can be produced using \texttt{article.cls} can also be produced with minimal alteration using the \texttt{interact} class file as described in this document.

If your article is accepted for publication it will be typeset as the journal requires in Minion Pro and/or Myriad Pro. Since most authors will not have these fonts installed, the page make-up is liable to alter slightly with the change of font. Also, the \texttt{interact} class file produces only single-column format, which is preferred for peer review and will be converted to two-column format by the typesetter if necessary during preparation of the proofs. Please therefore do not try to match the typeset format exactly, but use the standard \LaTeX\ fonts instead and ignore details such as slightly long lines of text or figures/tables not appearing in exact synchronization with their citations in the text: these details will be dealt with by the typesetter. Similarly, it is unnecessary to spend time addressing warnings in the log file -- if your .tex file compiles to produce a PDF document that correctly shows how you wish your paper to appear, such warnings will not prevent your source files being imported into the typesetter's program.


\subsection{Submission of manuscripts prepared using \emph{\LaTeX}}

Manuscripts for possible publication should be submitted to the Editors for review as directed in the journal's Instructions for Authors, and in accordance with any technical instructions provided in the journal's ScholarOne Manuscripts or Editorial Manager site. Your \LaTeX\ source file(s), the class file and any graphics files will be required in addition to the final PDF version when final, revised versions of accepted manuscripts are submitted.

Please ensure that any author-defined macros used in your article are gathered together in the preamble of your .tex file, i.e.\ before the \verb"\begin{document}" command. Note that if serious problems are encountered in the coding of a document (missing author-defined macros, for example), the typesetter may resort to rekeying it.


\section{Using the \texttt{interact} class file}

For convenience, simply copy the \texttt{interact.cls} file into the same directory as your manuscript files (you do not need to install it in your \TeX\ distribution). In order to use the \texttt{interact} document class, replace the command \verb"\documentclass{article}" at the beginning of your document with the command \verb"\documentclass{interact}".

The following document-class options should \emph{not} be used with the \texttt{interact} class file:
\begin{itemize}
  \item \texttt{10pt}, \texttt{11pt}, \texttt{12pt} -- unavailable;
  \item \texttt{oneside}, \texttt{twoside} -- not necessary, \texttt{oneside} is the default;
  \item \texttt{leqno}, \texttt{titlepage} -- should not be used;
  \item \texttt{twocolumn} -- should not be used (see Subsection~\ref{class});
  \item \texttt{onecolumn} -- not necessary as it is the default style.
\end{itemize}
To prepare a manuscript for a journal that is printed in A4 (two column) format, use the \verb"largeformat" document-class option provided by \texttt{interact.cls}; otherwise the class file produces pages sized for B5 (single column) format by default. The \texttt{geometry} package should not be used to make any further adjustments to the page dimensions.

%If your manuscript has supplementary content you can also use the \verb"interact" class file to prepare all or part of it using the \verb"suppldata" document-class option, which will suppress the `article history' date. This option \emph{must not} be used on any primary content. Note that authors are solely responsible for the preparation of all supplemental material.


\section{Additional features of the \texttt{interact} class file}

\subsection{Title, authors' names and affiliations, abstracts and article types}

The title should be generated at the beginning of your article using the \verb"\maketitle" command.
In the final version the author name(s) and affiliation(s) must be followed immediately by \verb"\maketitle" as shown below in order for them to be displayed in your PDF document.
To prepare an anonymous version for double-blind peer review, you can put the \verb"\maketitle" between the \verb"\title" and the \verb"\author" in order to hide the author name(s) and affiliation(s) temporarily.
Next you should include the abstract if your article has one, enclosed within an \texttt{abstract} environment.
The \verb"\articletype" command is also provided as an \emph{optional} element which should \emph{only} be included if your article actually needs it.
For example, the titles for this document begin as follows:
\begin{verbatim}
\articletype{ARTICLE TEMPLATE}

\title{Taylor \& Francis \LaTeX\ template for authors (\textsf{Interact}
layout + American Psychological Association reference style)}

\author{
\name{A.~N. Author\textsuperscript{a}\thanks{CONTACT A.~N. Author.
Email: latex.helpdesk@tandf.co.uk} and John Smith\textsuperscript{b}}
\affil{\textsuperscript{a}Taylor \& Francis, 4 Park Square, Milton
Park, Abingdon, UK; \textsuperscript{b}Institut f\"{u}r Informatik,
Albert-Ludwigs-Universit\"{a}t, Freiburg, Germany} }

\maketitle

\begin{abstract}
This template is for authors who are preparing a manuscript for a
Taylor \& Francis journal using the \LaTeX\ document preparation system
and the \texttt{interact} class file, which is available via selected
journals' home pages on the Taylor \& Francis website.
\end{abstract}
\end{verbatim}

An additional abstract in another language (preceded by a translation of the article title) may be included within the \verb"abstract" environment if required.

A graphical abstract may also be included if required. Within the \verb"abstract" environment you can include the code
\begin{verbatim}
\\\resizebox{25pc}{!}{% Figure removed}
\end{verbatim}
where the graphical abstract is to appear, where \verb"abstract.eps" is the name of the file containing the graphic (note that \verb"25pc" is the recommended maximum width, expressed in pica, for the graphical abstract in your manuscript).


\subsection{Abbreviations}

A list of abbreviations may be included if required, enclosed within an \texttt{abbreviations} environment, i.e.\ \verb"\begin{abbreviations}"\ldots\verb"\end{abbreviations}", immediately following the \verb"abstract" environment.


\subsection{Keywords}

A list of keywords may be included if required, enclosed within a \texttt{keywords} environment, i.e.\ \verb"\begin{keywords}"\ldots\verb"\end{keywords}". Additional keywords in other languages (preceded by a translation of the word `keywords') may also be included within the \verb"keywords" environment if required.


\subsection{Subject classification codes}

AMS, JEL or PACS classification codes may be included if required. The \texttt{interact} class file provides an \texttt{amscode} environment, i.e.\ \verb"\begin{amscode}"\ldots\verb"\end{amscode}", a \texttt{jelcode} environment, i.e.\ \verb"\begin{jelcode}"\ldots\verb"\end{jelcode}", and a \texttt{pacscode} environment, i.e.\ \verb"\begin{pacscode}"\ldots\verb"\end{pacscode}" to assist with this.


\subsection{Additional footnotes to the title or authors' names}

The \verb"\thanks" command may be used to create additional footnotes to the title or authors' names if required. Footnote symbols for this purpose should be used in the order
$^\ast$~(coded as \verb"$^\ast$"), $\dagger$~(\verb"$\dagger$"), $\ddagger$~(\verb"$\ddagger$"), $\S$~(\verb"$\S$"), $\P$~(\verb"$\P$"), $\|$~(\verb"$\|$"),
$\dagger\dagger$~(\verb"$\dagger\dagger$"), $\ddagger\ddagger$~(\verb"$\ddagger\ddagger$"), $\S\S$~(\verb"$\S\S$"), $\P\P$~(\verb"$\P\P$").

Note that any \verb"footnote"s to the main text will automatically be assigned the superscript symbols 1, 2, 3, etc. by the class file.\footnote{If preferred, the \texttt{endnotes} package may be used to set the notes at the end of your text, before the bibliography. The symbols will be changed to match the style of the journal if necessary by the typesetter.}


\section{Some guidelines for using the standard features of \LaTeX}

\subsection{Sections}

The \textsf{Interact} layout style allows for five levels of section heading, all of which are provided in the \texttt{interact} class file using the standard \LaTeX\ commands \verb"\section", \verb"\subsection", \verb"\subsubsection", \verb"\paragraph" and \verb"\subparagraph". Numbering will be automatically generated for all these headings by default.


\subsection{Lists}

Numbered lists are produced using the \texttt{enumerate} environment, which will number each list item with arabic numerals by default. For example,
\begin{enumerate}
  \item first item
  \item second item
  \item third item
\end{enumerate}
was produced by
\begin{verbatim}
\begin{enumerate}
  \item first item
  \item second item
  \item third item
\end{enumerate}
\end{verbatim}
Alternative numbering styles can be achieved by inserting an optional argument in square brackets to each \verb"item", e.g.\ \verb"\item[(i)] first item"\, to create a list numbered with roman numerals at level one.

Bulleted lists are produced using the \texttt{itemize} environment. For example,
\begin{itemize}
  \item First bulleted item
  \item Second bulleted item
  \item Third bulleted item
\end{itemize}
was produced by
\begin{verbatim}
\begin{itemize}
  \item First bulleted item
  \item Second bulleted item
  \item Third bulleted item
\end{itemize}
\end{verbatim}


\subsection{Figures}

The \texttt{interact} class file will deal with positioning your figures in the same way as standard \LaTeX. It should not normally be necessary to use the optional \texttt{[htb]} location specifiers of the \texttt{figure} environment in your manuscript; you may, however, find the \verb"[p]" placement option or the \verb"endfloat" package useful if a journal insists on the need to separate figures from the text.

Figure captions appear below the figures themselves, therefore the \verb"\caption" command should appear after the body of the figure. For example, Figure~\ref{sample-figure} with caption and sub-captions is produced using the following commands:
\begin{verbatim}
% Figure environment removed
\end{verbatim}
% Figure environment removed

To ensure that figures are correctly numbered automatically, the \verb"\label" command should be included just after the \verb"\caption" command, or in its argument.

The \verb"\subfloat" command requires \verb"subfig.sty", which is called in the preamble of the \texttt{interactapasample.tex} file (to allow your choice of an alternative package if preferred) and included in the \textsf{Interact} \LaTeX\ bundle for convenience. Please supply any additional figure macros used with your article in the preamble of your .tex file.

The source files of any figures will be required when the final, revised version of a manuscript is submitted. Authors should ensure that these are suitable (in terms of lettering size, etc.) for the reductions they envisage.

The \texttt{epstopdf} package can be used to incorporate encapsulated PostScript (.eps) illustrations when using PDF\LaTeX, etc. Please provide the original .eps source files rather than the generated PDF images of those illustrations for production purposes.


\subsection{Tables}

The \texttt{interact} class file will deal with positioning your tables in the same way as standard \LaTeX. It should not normally be necessary to use the optional \texttt{[htb]} location specifiers of the \texttt{table} environment in your manuscript; you may, however, find the \verb"[p]" placement option or the \verb"endfloat" package useful if a journal insists on the need to separate tables from the text.

The \texttt{tabular} environment can be used as shown to create tables with single horizontal rules at the head, foot and elsewhere as appropriate. The captions appear above the tables in the \textsf{Interact} style, therefore the \verb"\tbl" command should be used before the body of the table. For example, Table~\ref{sample-table} is produced using the following commands:
\begin{table}
\tbl{Example of a table showing that its caption is as wide as
 the table itself and justified.}
{\begin{tabular}{lcccccc} \toprule
 & \multicolumn{2}{l}{Type} \\ \cmidrule{2-7}
 Class & One & Two & Three & Four & Five & Six \\ \midrule
 Alpha\textsuperscript{a} & A1 & A2 & A3 & A4 & A5 & A6 \\
 Beta & B2 & B2 & B3 & B4 & B5 & B6 \\
 Gamma & C2 & C2 & C3 & C4 & C5 & C6 \\ \bottomrule
\end{tabular}}
\tabnote{\textsuperscript{a}This footnote shows how to include
 footnotes to a table if required.}
\label{sample-table}
\end{table}
\begin{verbatim}
\begin{table}
\tbl{Example of a table showing that its caption is as wide as
 the table itself and justified.}
{\begin{tabular}{lcccccc} \toprule
 & \multicolumn{2}{l}{Type} \\ \cmidrule{2-7}
 Class & One & Two & Three & Four & Five & Six \\ \midrule
 Alpha\textsuperscript{a} & A1 & A2 & A3 & A4 & A5 & A6 \\
 Beta & B2 & B2 & B3 & B4 & B5 & B6 \\
 Gamma & C2 & C2 & C3 & C4 & C5 & C6 \\ \bottomrule
\end{tabular}}
\tabnote{\textsuperscript{a}This footnote shows how to include
 footnotes to a table if required.}
\label{sample-table}
\end{table}
\end{verbatim}

To ensure that tables are correctly numbered automatically, the \verb"\label" command should be included just before \verb"\end{table}".

The \verb"\toprule", \verb"\midrule", \verb"\bottomrule" and \verb"\cmidrule" commands are those used by \verb"booktabs.sty", which is called by the \texttt{interact} class file and included in the \textsf{Interact} \LaTeX\ bundle for convenience. Tables produced using the standard commands of the \texttt{tabular} environment are also compatible with the \texttt{interact} class file.


\subsection{Landscape pages}

If a figure or table is too wide to fit the page it will need to be rotated, along with its caption, through 90$^{\circ}$ anticlockwise. Landscape figures and tables can be produced using the \verb"rotating" package, which is called by the \texttt{interact} class file. The following commands (for example) can be used to produce such pages.
\begin{verbatim}
\setcounter{figure}{1}
\begin{sidewaysfigure}
\centerline{\epsfbox{figname.eps}}
\caption{Example landscape figure caption.}
\label{landfig}
\end{sidewaysfigure}
\end{verbatim}
\begin{verbatim}
\setcounter{table}{1}
\begin{sidewaystable}
 \tbl{Example landscape table caption.}
  {\begin{tabular}{@{}llllcll}
    .
    .
    .
  \end{tabular}}\label{landtab}
\end{sidewaystable}
\end{verbatim}
Before any such float environment, use the \verb"\setcounter" command as above to fix the numbering of the caption (the value of the counter being the number given to the preceding figure or table). Subsequent captions will then be automatically renumbered accordingly. The \verb"\epsfbox" command requires \verb"epsfig.sty", which is called by the \texttt{interact} class file and is also included in the \textsf{Interact} \LaTeX\ bundle for convenience.

Note that if the \verb"endfloat" package is used, one or both of the commands
\begin{verbatim}
\DeclareDelayedFloatFlavor{sidewaysfigure}{figure}
\DeclareDelayedFloatFlavor{sidewaystable}{table}
\end{verbatim}
will need to be included in the preamble of your .tex file, after the \verb"endfloat" package is loaded, in order to process any landscape figures and/or tables correctly.


\subsection{Theorem-like structures}

A predefined \verb"proof" environment is provided by the \texttt{amsthm} package (which is called by the \texttt{interact} class file), as follows:
\begin{proof}
More recent algorithms for solving the semidefinite programming relaxation are particularly efficient, because they explore the structure of the MAX-CUT problem.
\end{proof}
\noindent This was produced by simply typing:
\begin{verbatim}
\begin{proof}
More recent algorithms for solving the semidefinite programming
relaxation are particularly efficient, because they explore the
structure of the MAX-CUT problem.
\end{proof}
\end{verbatim}
Other theorem-like environments (theorem, definition, remark, etc.) need to be defined as required, e.g.\ using \verb"\newtheorem{theorem}{Theorem}" in the preamble of your .tex file (see the preamble of \verb"interactapasample.tex" for more examples). You can define the numbering scheme for these structures however suits your article best. Please note that the format of the text in these environments may be changed if necessary to match the style of individual journals by the typesetter during preparation of the proofs.


\subsection{Mathematics}

\subsubsection{Displayed mathematics}

The \texttt{interact} class file will set displayed mathematical formulas centred on the page without equation numbers if you use the \texttt{displaymath} environment or the equivalent \verb"\[...\]" construction. For example, the equation
\[
 \hat{\theta}_{w_i} = \hat{\theta}(s(t,\mathcal{U}_{w_i}))
\]
was typeset using the commands
\begin{verbatim}
\[
 \hat{\theta}_{w_i} = \hat{\theta}(s(t,\mathcal{U}_{w_i}))
\]
\end{verbatim}

For those of your equations that you wish to be automatically numbered sequentially throughout the text for future reference, use the \texttt{equation} environment, e.g.
\begin{equation}
 \hat{\theta}_{w_i} = \hat{\theta}(s(t,\mathcal{U}_{w_i}))
\end{equation}
was typeset using the commands
\begin{verbatim}
\begin{equation}
 \hat{\theta}_{w_i} = \hat{\theta}(s(t,\mathcal{U}_{w_i}))
\end{equation}
\end{verbatim}

Part numbers for sets of equations may be generated using the \texttt{subequations} environment, e.g.
\begin{subequations} \label{subeqnexample}
\begin{equation}
     \varepsilon \rho w_{tt}(s,t) = N[w_{s}(s,t),w_{st}(s,t)]_{s},
     \label{subeqnparta}
\end{equation}
\begin{equation}
     w_{tt}(1,t)+N[w_{s}(1,t),w_{st}(1,t)] = 0,   \label{subeqnpartb}
\end{equation}
\end{subequations}
which was typeset using the commands
\begin{verbatim}
\begin{subequations} \label{subeqnexample}
\begin{equation}
     \varepsilon \rho w_{tt}(s,t) = N[w_{s}(s,t),w_{st}(s,t)]_{s},
     \label{subeqnparta}
\end{equation}
\begin{equation}
     w_{tt}(1,t)+N[w_{s}(1,t),w_{st}(1,t)] = 0,   \label{subeqnpartb}
\end{equation}
\end{subequations}
\end{verbatim}
This is made possible by the \texttt{amsmath} package, which is called by the class file. If you put a \verb"\label" just after the \verb"\begin{subequations}" command, references can be made to the collection of equations, i.e.\ `(\ref{subeqnexample})' in the example above. Or, as the example also shows, you can label and refer to each equation individually -- i.e.\ `(\ref{subeqnparta})' and `(\ref{subeqnpartb})'.

Displayed mathematics should be given end-of-line punctuation appropriate to the running text sentence of which it forms a part, if required.

\subsubsection{Math fonts}

\paragraph{Superscripts and subscripts}
Superscripts and subscripts will automatically come out in the correct size in a math environment (i.e.\ enclosed within \verb"\(...\)" or \verb"$...$" commands in running text, or within \verb"\[...\]" or the \texttt{equation} environment for displayed equations). Sub/superscripts that are physical variables should be italic, whereas those that are labels should be roman (e.g.\ $C_p$, $T_\mathrm{eff}$). If the subscripts or superscripts need to be other than italic, they must be coded individually.

\paragraph{Upright Greek characters and the upright partial derivative sign}
Upright lowercase Greek characters can be obtained by inserting the letter `u' in the control code for the character, e.g.\ \verb"\umu" and \verb"\upi" produce $\umu$ (used, for example, in the symbol for the unit microns -- $\umu\mathrm{m}$) and $\upi$ (the ratio of the circumference of a circle to its diameter). Similarly, the control code for the upright partial derivative $\upartial$ is \verb"\upartial". Bold lowercase as well as uppercase Greek characters can be obtained by \verb"{\bm \gamma}", for example, which gives ${\bm \gamma}$, and \verb"{\bm \Gamma}", which gives ${\bm \Gamma}$.


\section*{Acknowledgement(s)}

An unnumbered section, e.g.\ \verb"\section*{Acknowledgements}", may be used for thanks, etc.\ if required and included \emph{in the non-anonymous version} before any Notes or References.


\section*{Disclosure statement}

An unnumbered section, e.g.\ \verb"\section*{Disclosure statement}", may be used to declare any potential conflict of interest and included \emph{in the non-anonymous version} before any Notes or References, after any Acknowledgements and before any Funding information.


\section*{Funding}

An unnumbered section, e.g.\ \verb"\section*{Funding}", may be used for grant details, etc.\ if required and included \emph{in the non-anonymous version} before any Notes or References.


\section*{Notes on contributor(s)}

An unnumbered section, e.g.\ \verb"\section*{Notes on contributors}", may be included \emph{in the non-anonymous version} if required. A photograph may be added if requested.


\section*{Nomenclature/Notation}

An unnumbered section, e.g.\ \verb"\section*{Nomenclature}" (or \verb"\section*{Notation}"), may be included if required, before any Notes or References.


\section*{Notes}

An unnumbered `Notes' section may be included before the References (if using the \verb"endnotes" package, use the command \verb"\theendnotes" where the notes are to appear, instead of creating a \verb"\section*").


\section{References}

\subsection{References cited in the text}

References should be cited in accordance with \citeauthor{APA10} (APA) style, i.e.\ in alphabetical order separated by semicolons, e.g.\ `\citep{Ban77,Pia88,VL07}' or `\ldots see Smith (1985, p.~75)'. If there are two or more authors with the same surname, use the first author's initials with the surnames, e.g.\ `\citep{Lig08,Lig06}'. If there are three to five authors, list all the authors in the first citation, e.g.\ `\citep{GSSM91}'. In subsequent citations, use only the first author's surname followed by et al., e.g.\ `\citep{GSSM91}'. For six or more authors, cite the first author's name followed by et al. For two or more sources by the same author(s) in the same year, use lower-case letters (a,~b,~c, \ldots) with the year to order the entries in the reference list and use these lower-case letters with the year in the in-text citations, e.g.\ `(Green, 1981a,b)'. For further details on this reference style, see the Instructions for Authors on the Taylor \& Francis website.

Each bibliographic entry has a key, which is assigned by the author and is used to refer to that entry in the text. In this document, the key \verb"Nas93" in the citation form \verb"\citep{Nas93}" produces `\citep{Nas93}', and the keys \verb"Koc59", \verb"Han04" and \verb"Cla08" in the citation form \verb"\citep{Koc59,Han04,Cla08}" produce `\citep{Koc59,Han04,Cla08}'. The citation \verb"\citep{Cha08}" produces `\citep{Cha08}' where the citation first appears in the text, and `\citep{Cha08}' in any subsequent citation. The appropriate citation style for different situations can be obtained, for example, by \verb"\citet{Ovi95}" for `\citet{Ovi95}', \verb"\citealp{MPW08}" for `\citealp{MPW08}', and \verb"\citealt{Sch93}" for `\citealt{Sch93}'. Citation of the year alone may be produced by \verb"\citeyear{Sch00}", i.e.\ `\citeyear{Sch00}', or \verb"\citeyearpar{Gra05}", i.e.\ `\citeyearpar{Gra05}', or of the author(s) alone by \verb"\citeauthor{Rit74}", i.e.\ `\citeauthor{Rit74}'. Optional notes may be included at the beginning and/or end of a citation by the use of square brackets, e.g.\ \verb"\citep[p.~31]{Hay08}" produces `\citep[p.~31]{Hay08}'; \verb"\citep[see][pp.~73-–77]{PI51}" produces `\citep[see][pp.~73--77]{PI51}'; \verb"\citep[e.g.][]{Fel81}" produces `\citep[e.g.][]{Fel81}'. A `plain' \verb"\cite" command will produce the same results as a \verb"\citet", i.e.\ \verb"\cite{BriIP}" will produce `\cite{BriIP}'.


\subsection{The list of references}

References should be listed at the end of the main text in alphabetical order, then chronologically (earliest first), with full page ranges (where appropriate) and issue numbers (essential for journals paginated by issue). If a reference has more than seven named authors, list the first six names, followed by an ellipsis (\ldots), then the last author's name \cite[see for example][]{Gil04}.
The following list shows some sample references prepared in the Taylor \& Francis APA style.

\begin{thebibliography}{}

\bibitem[American Psychological Association(2010)]{APA10}
American Psychological Association. (2010). \emph {Publication manual of the
 American Psychological Association} (6th ed.). Washington, DC: Author.

\bibitem[Bandura(1977)]{Ban77}
Bandura, A.~J. (1977). \emph{Social learning theory}. Englewood Cliffs, NJ:
 Prentice Hall.

\bibitem[Briscoe(in press)]{BriIP}
Briscoe, R. (in press). {Egocentric spatial representation in action and
 perception}. \emph{Philosophy and Phenomenological Research}. Retrieved from
 http://cogprints.org/5780/1/ECSRAP.F07.pdf

\bibitem[Chamberlin et al.(2008)Chamberlin, Novotney, Packard, \& Price]{Cha08}
Chamberlin, J., Novotney, A., Packard, E., \& Price, M. (2008, May).
 Enhancing worker well-being: Occupational health psychologists convene to
 share their research on work, stress, and health. \emph{Monitor on
 Psychology}, \emph{39}(5), 26--29.

\bibitem[Clay(2008)]{Cla08}
Clay, R. (2008, June). Science vs. ideology: Psychologists fight back about the
 misuse of research. \emph{Monitor on Psychology}, \emph{39}(6). Retrieved from
 http://www.apa.org/monitor/

\bibitem[Feller(1981)]{Fel81}
Feller, B.~A. (1981). \emph{Health characteristics of persons with chronic
 activity limitation, United States, 1979} (Report No. VHS-SER10/137).
 Hyattsville, MD: National Center for Health Statistics (US).

\bibitem[Ganster et~al.(1991)Ganster, Schaubroeck, Sime, \& Mayes]{GSSM91}
Ganster, D.~C., Schaubroeck, J., Sime, W.~E., \& Mayes, B.~T. (1991).
 The nomological validity of the Type A personality among employed adults
 [Monograph]. \emph{Journal of Applied Psychology}, \emph{76}, 143--168.

\bibitem[Gilbert et~al.(2004)]{Gil04}
Gilbert, D.~G., McClernon, F.~J., Rabinovich, N.~E., Sugai, C., Plath, L.~C.,
 Asgaard, G., \ldots Botros, N. (2004). Effects of quitting smoking on EEG
 activation and attention last for more than 31 days and are more severe with
 stress, dependence, DRD2 A1 allele, and depressive traits. \emph{Nicotine
 and Tobacco Research}, \emph{6}, 249--267.

\bibitem[Graham(2005)]{Gra05}
Graham, G. (2005). Behaviorism. In E.~N. Zalta (ed.), \emph{The Stanford
 encyclopedia of philosophy} (Fall 2007 ed.). Retrieved from
 http://plato.stanford.edu/entries/behaviorism

\bibitem[Haney \& Wiener(2004)]{Han04}
Haney, C., \& Wiener, R.~L. (Eds.). (2004). Capital punishment in the United
 States [Special issue]. \emph{Psychology, Public Policy, and Law},
 \emph{10}(4).

\bibitem[Haybron(2008)]{Hay08}
Haybron, D.~M. (2008). Philosophy and the science of subjective well-being. In
 M. Eid \& R.~J. Larsen (Eds.), \emph{The science of subjective well-being}
 (pp. 17--43). New York, NY: Guilford Press.

\bibitem[Koch(1959-1963)]{Koc59}
Koch, S. (Ed.). (1959--1963). \emph{Psychology: A study of science}
 (Vols.~1--6). New York, NY: McGraw-Hill.

\bibitem[I. Light(2006)]{Lig06}
Light, I. (2006). \emph{Deflecting immigration: Networks, markets, and
 regulation in Los Angeles}. New York, NY: Russell Sage Foundation.

\bibitem[M.~A. Light \& Light(2008)]{Lig08}
Light, M.~A., \& Light, I.~H. (2008). The geographic expansion of Mexican
 immigration in the United States and its applications for local law
 enforcement. \emph{Law Enforcement Executive Forum Journal}, \emph{8}(1),
 73--82.

\bibitem[Marshall-Pescini \& Whiten(2008)]{MPW08}
Marshall-Pescini, S., \& Whiten, A. (2008). Social learning of nut-cracking
 behavior in East African sanctuary-living chimpanzees (\emph{Pan troglodytes
 schweinfurthii}) [Supplemental material]. \emph{Journal of Comparative
 Psychology}, \emph{122}, 186--194.

\bibitem[Nash(1993)]{Nas93}
Nash, M. (1993). Malay. In P.~Hockings (Ed.), \emph{Encyclopedia of world
 cultures} (Vol.~5, pp.~174--176). New York, NY: G.~K. Hall.

\bibitem[Oviedo(1995)]{Ovi95}
Oviedo, S. (1995). \emph{Adolescent pregnancy: Voices heard in the everyday
 lives of pregnant teenagers} (Unpublished master's thesis). University of
 North Texas, Denton, TX.

\bibitem[Piaget(1988)]{Pia88}
Piaget, J. (1988). Extracts from Piaget's theory (G.~Gellerier \& J.~Langer,
 Trans.). In K.~Richardson \& S.~Sheldon (Eds.), \emph{Cognitive development
 to adolescence: A reader} (pp. 3--18). Hillsdale, NJ: Erlbaum. (Reprinted
 from \emph{Manual of child psychology}, pp. 703--732, by P.~H. Mussen, Ed.,
 1970, New York, NY: Wiley)

\bibitem[Piaget \& Inhelder(1951)]{PI51}
Piaget, J., \& Inhelder, B. (1951). \emph{La gen{\`e}se de l'id{\'e}e de
 hasard chez l'enfant} [The origin of the idea of chance in the child]. Paris:
 Presses Universitaires de France.

\bibitem[Ritzmann(1974)]{Rit74}
Ritzmann, R.~E. (1974). \emph{The snapping mechanism of \emph{Alpheid} shrimp}
 (Unpublished doctoral dissertation). University of Virginia, Charlottesville,
 VA.

\bibitem[Schatz(2000)]{Sch00}
Schatz, B.~R. (2000, November 17). Learning by text or context? [Review of the
 book \emph{The social life of information}, by J.~S. Brown \& P. Duguid].
\emph{Science}, \emph{290}, 1304.

\bibitem[Schwartz(1993)]{Sch93}
Schwartz, J. (1993, September 30). Obesity affects economic, social status.
 \emph{The Washington Post}, pp.~A1, A4.

\bibitem[Von~Ledebur(2007)]{VL07}
Von~Ledebur, S.~C. (2007). Optimizing knowledge transfer by new employees in
 companies. \emph{Knowledge Management Research \& Practice}. Advance online
 publication. doi:10.1057/palgrave/kmrp.8500141

\end{thebibliography}
\bigskip
\noindent This was produced by typing:
\begin{verbatim}
\begin{thebibliography}{}

\bibitem[American Psychological Association(2010)]{APA10}
American Psychological Association. (2010). \emph {Publication manual
 of the American Psychological Association} (6th ed.). Washington, DC:
 Author.

\bibitem[Bandura(1977)]{Ban77}
Bandura, A.~J. (1977). \emph{Social learning theory}. Englewood Cliffs,
 NJ: Prentice Hall.

\bibitem[Briscoe(in press)]{BriIP}
Briscoe, R. (in press). {Egocentric spatial representation in action
 and perception}. \emph{Philosophy and Phenomenological Research}.
 Retrieved from http://cogprints.org/5780/1/ECSRAP.F07.pdf

\bibitem[Chamberlin et al.(2008)Chamberlin, Novotney, Packard, \& Price]{Cha08}
Chamberlin, J., Novotney, A., Packard, E., \& Price, M. (2008, May).
 Enhancing worker well-being: Occupational health psychologists convene
 to share their research on work, stress, and health. \emph{Monitor on
 Psychology}, \emph{39}(5), 26--29.

\bibitem[Clay(2008)]{Cla08}
Clay, R. (2008, June). Science vs. ideology: Psychologists fight back
 about the misuse of research. \emph{Monitor on Psychology},
 \emph{39}(6). Retrieved from http://www.apa.org/monitor/

\bibitem[Feller(1981)]{Fel81}
Feller, B.~A. (1981). \emph{Health characteristics of persons with chronic
 activity limitation, United States, 1979} (Report No. VHS-SER10/137).
 Hyattsville, MD: National Center for Health Statistics (US).

\bibitem[Ganster et al.(1991)Ganster, Schaubroeck, Sime, \& Mayes]{GSSM91}
Ganster, D.~C., Schaubroeck, J., Sime, W.~E., \& Mayes, B.~T. (1991).
 The nomological validity of the Type A personality among employed
 adults [Monograph]. \emph{Journal of Applied Psychology}, \emph{76},
 143--168.

\bibitem[Gilbert et~al.(2004)]{Gil04}
Gilbert, D.~G., McClernon, F.~J., Rabinovich, N.~E., Sugai, C., Plath,
 L.~C., Asgaard, G., \ldots Botros, N. (2004). Effects of quitting
 smoking on EEG activation and attention last for more than 31 days and
 are more severe with stress, dependence, DRD2 A1 allele, and depressive
 traits. \emph{Nicotine and Tobacco Research}, \emph{6}, 249--267.

\bibitem[Graham(2005)]{Gra05}
Graham, G. (2005). Behaviorism. In E.~N. Zalta (ed.), \emph{The
 Stanford encyclopedia of philosophy} (Fall 2007 ed.). Retrieved from
 http://plato.stanford.edu/entries/behaviorism

\bibitem[Haney \& Wiener(2004)]{Han04}
Haney, C., \& Wiener, R.~L. (Eds.). (2004). Capital punishment in the
 United States [Special issue]. \emph{Psychology, Public Policy, and
 Law}, \emph{10}(4).

\bibitem[Haybron(2008)]{Hay08}
Haybron, D.~M. (2008). Philosophy and the science of subjective well-
 being. In M. Eid \& R.~J. Larsen (Eds.), \emph{The science of
 subjective well-being} (pp. 17--43). New York, NY: Guilford Press.

\bibitem[Koch(1959-1963)]{Koc59}
Koch, S. (Ed.). (1959--1963). \emph{Psychology: A study of science}
 (Vols.~1--6). New York, NY: McGraw-Hill.

\bibitem[I. Light(2006)]{Lig06}
Light, I. (2006). \emph{Deflecting immigration: Networks, markets,
 and regulation in Los Angeles}. New York, NY: Russell Sage Foundation.

\bibitem[M.~A. Light \& Light(2008)]{Lig08}
Light, M.~A., \& Light, I.~H. (2008). The geographic expansion of
 Mexican immigration in the United States and its applications for
 local law enforcement. \emph{Law Enforcement Executive Forum
 Journal}, \emph{8}(1), 73--82.

\bibitem[Marshall-Pescini \& Whiten(2008)]{MPW08}
Marshall-Pescini, S., \& Whiten, A. (2008). Social learning of nut-
 cracking behavior in East African sanctuary-living chimpanzees
 (\emph{Pan troglodytes schweinfurthii}) [Supplemental material].
 \emph{Journal of Comparative Psychology}, \emph{122}, 186--194.

\bibitem[Nash(1993)]{Nas93}
Nash, M. (1993). Malay. In P.~Hockings (Ed.), \emph{Encyclopedia of
 world cultures} (Vol.~5, pp.~174--176). New York, NY: G.~K. Hall.

\bibitem[Oviedo(1995)]{Ovi95}
Oviedo, S. (1995). \emph{Adolescent pregnancy: Voices heard in the
 everyday lives of pregnant teenagers} (Unpublished master's thesis).
 University of North Texas, Denton, TX.

\bibitem[Piaget(1988)]{Pia88}
Piaget, J. (1988). Extracts from Piaget's theory (G.~Gellerier \&
 J.~Langer, Trans.). In K.~Richardson \& S.~Sheldon (Eds.),
 \emph{Cognitive development to adolescence: A reader} (pp. 3--18).
 Hillsdale, NJ: Erlbaum. (Reprinted from \emph{Manual of child
 psychology}, pp. 703--732, by P.~H. Mussen, Ed., 1970, New York,
 NY: Wiley)

\bibitem[Piaget \& Inhelder(1951)]{PI51}
Piaget, J., \& Inhelder, B. (1951). \emph{La gen{\`e}se de l'id{\'e}e
 de hasard chez l'enfant} [The origin of the idea of chance in the
 child]. Paris: Presses Universitaires de France.

\bibitem[Ritzmann(1974)]{Rit74}
Ritzmann, R.~E. (1974). \emph{The snapping mechanism of \emph{Alpheid}
 shrimp} (Unpublished doctoral dissertation). University of Virginia,
 Charlottesville, VA.

\bibitem[Schatz(2000)]{Sch00}
Schatz, B.~R. (2000, November 17). Learning by text or context?
 [Review of the book \emph{The social life of information}, by J.~S.
 Brown \& P. Duguid]. \emph{Science}, \emph{290}, 1304.

\bibitem[Schwartz(1993)]{Sch93}
Schwartz, J. (1993, September 30). Obesity affects economic, social
 status. \emph{The Washington Post}, pp.~A1, A4.

\bibitem[Von~Ledebur(2007)]{VL07}
Von~Ledebur, S.~C. (2007). Optimizing knowledge transfer by new
 employees in companies. \emph{Knowledge Management Research \&
 Practice}. Advance online publication. doi:10.1057/palgrave/kmrp.8500141

\end{thebibliography}
\end{verbatim}
\bigskip
\noindent Each entry takes the form:
\begin{verbatim}
\bibitem[short list of authors' surnames(date of publication)long list
 of authors' surnames]{key}
Bibliography entry
\end{verbatim}
where `\texttt{long list of authors' surnames}' is the \emph{optional} `long' list of three, four or five names which enables them all to appear where the \verb"bibitem" is first cited in the text (if the long list is missing, the short list will be used instead), and `\texttt{key}' is the tag that is to be used as an argument for the \verb"\cite" commands in the text of the article. `\texttt{Bibliography entry}' is the material that is to appear in the list of references, suitably formatted. The commands
\begin{verbatim}
\usepackage[longnamesfirst,sort]{natbib}
\bibpunct[, ]{(}{)}{;}{a}{,}{,}
\renewcommand\bibfont{\fontsize{10}{12}\selectfont}
\end{verbatim}
need to be included in the preamble of your .tex file in order to generate the citations and bibliography as described above.

Instead of typing the bibliography by hand, you may prefer to create the list of references using a \textsc{Bib}\TeX\ database. For this we suggest using Erik Meijer's \texttt{apacite} package, which is available via CTAN if you do not already have it. The \verb"apacite.sty", \verb"apacite.bst" and (if your paper is written in English) \verb"english.apc" files need to be in your working folder or an appropriate directory, the commands
\begin{verbatim}
\usepackage[natbibapa,nodoi]{apacite}
\setlength\bibhang{12pt}
\renewcommand\bibliographytypesize{\fontsize{10}{12}\selectfont}
\end{verbatim}
included in the preamble of your .tex file instead of the \verb"\usepackage[]{natbib}", \verb"\bibpunct" and \verb"\renewcommand\bibfont" commands described above, and the lines
\begin{verbatim}
\bibliographystyle{apacite}
\bibliography{interactapasample}
\end{verbatim}
included where the list of references is to appear, where \texttt{interactapasample.bib} is the bibliographic database included with the \textsf{Interact}-APA \LaTeX\ bundle (to be replaced with the name of your own .bib file). The \verb"[natbibapa]" option has to be added to \verb"\usepackage{apacite}" in order to enable citation commands of the type \verb"\citep" and \verb"\citet". \LaTeX/\textsc{Bib}\TeX\ will extract from your .bib file only those references that are cited in your .tex file and list them in the References section.

Please include a copy of your .bib file and/or the final generated .bbl file among your source files if your .tex file does not contain a reference list in a \texttt{thebibliography} environment.


\section{Appendices}

Any appendices should be placed after the list of references, beginning with the command \verb"\appendix" followed by the command \verb"\section" for each appendix title, e.g.
\begin{verbatim}
\appendix
\section{This is the title of the first appendix}
\section{This is the title of the second appendix}
\end{verbatim}
produces:\medskip

\noindent\textbf{Appendix A. This is the title of the first appendix}\medskip

\noindent\textbf{Appendix B. This is the title of the second appendix}\medskip

\noindent Subsections, equations, figures, tables, etc.\ within appendices will then be automatically numbered as appropriate. Some theorem-like environments may need to have their counters reset manually (e.g.\ if they are not numbered within sections in the main text). You can achieve this by using \verb"\numberwithin{remark}{section}" (for example) just after the \verb"\appendix" command.

Note that if the \verb"endfloat" package is used on a document containing any appendices, the \verb"\processdelayedfloats" command must be included immediately before the \verb"\appendix" command in order to ensure that the floats belonging to the main body of the text are numbered as such.

%\processdelayedfloats %%% See above for an explanation of why this command might be needed here.

\appendix

\section{Troubleshooting}

Authors may occasionally encounter problems with the preparation of a manuscript using \LaTeX. The appropriate action to take will depend on the nature of the problem:
\begin{enumerate}
\item[(i)] If the problem is with \LaTeX\ itself, rather than with the actual macros, please consult an appropriate \LaTeXe\ manual for initial advice. If the solution cannot be found, or if you suspect that the problem does lie with the macros, then please contact Taylor \& Francis for assistance (\texttt{latex.helpdesk@tandf.co.uk}), clearly stating the title of the journal to which you are submitting.
\item[(ii)] Problems with page make-up (e.g.\ occasional overlong lines of text; figures or tables appearing out of order): please do not try to fix these using `hard' page make-up commands -- the typesetter will deal with such problems. (You may, if you wish, draw attention to particular problems when submitting the final version of your manuscript.)
\item[(iii)] If a required font is not available on your system, allow \TeX\ to substitute the font and specify which font is required in a covering letter accompanying your files.
\end{enumerate}


\section{Obtaining the template and class file}

\subsection{Via the Taylor \& Francis website}

This article template and the \texttt{interact} class file may be obtained via the `Instructions for Authors' pages of selected Taylor \& Francis journals.

Please note that the class file calls up the open-source \LaTeX\ packages booktabs.sty, epsfig.sty and rotating.sty, which will, for convenience, unpack with the downloaded template and class file. The template optionally calls for natbib.sty and subfig.sty, which are also supplied for convenience.


\subsection{Via e-mail}

This article template, the \texttt{interact} class file and the associated open-source \LaTeX\ packages are also available via e-mail. Requests should be addressed to \texttt{latex.helpdesk@tandf.co.uk}, clearly stating for which journal you require the template and class file.

\end{document}
