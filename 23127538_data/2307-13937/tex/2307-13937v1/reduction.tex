\section{Reduction from Label-Cover}\label{sec: reduction}
We now prove  \Cref{thm2} by reducing Label Cover to a GMP instance. 

\medskip

{\bf Overview.}
Our construction builds on the ideas used by Lund and Yannakakis \cite{lund1994hardness} to show the $\Omega(\log n)$ hardness of set cover, using a gadget based on a natural $\Omega(\log n)$ integrality gap instance. We first give a rough sketch of their idea (see e.g.~\cite{arora1996hardness} for an excellent exposition), and then explain the additional steps needed in our setting and why we only get a $\log^\delta n$ hardness for some small $\delta>0$, despite the $\Omega(\log^{1/2}n)$ integrality gap instance above.

Consider a label cover instance $\mathcal{L} = ((U,V,E),L,\Pi)$, as defined in \Cref{def: label-cover}, with label set $[L]$ and vertex sets $U$ and $V$. The main idea in \cite{lund1994hardness} is the following. For each edge $e =(u,v)\in E$, one associates a $(n,m,\ell,\beta)$-system $I^e =(U^e; A_1^e \ldots, A_m^e)$  (with disjoint universes for each edge). The sets will be associated with labels for vertices (so that $L=m$) and we associate the set $A_{\pi_e(a)}^e$ with label $a$ for $u$ and $\overline{A}_b^e$ 
 with label $b$ for vertex $v$. The point is that in the completeness case, where the labels $a'$ and $b'$ for $u$ and $v$ satisfy $e$ (i.e.,~$\pi_e(a')=b'$), $U^{e}$ can be covered by just the two corresponding sets 
$A_{\pi_e(a')}^e$ and $\overline{A}_{b'}^e$. 
Conversely, in the soundness case, if $U^e$ is covered using less than $\ell$ sets, there must be a pair of sets that are complements of each other, which can be used to produce a good labeling for $\mathcal{L}$.

To adapt this to our setting, suppose we create a job for each element in $U^e$, and $m$ machines per vertex, one for each label in $[L]$. Also suppose that for labels $a,b$, we set the processing times of jobs in the set $A_{\pi_e(a)}^e$ to be finite on the $a$-th machine of vertex $u$ and the processing times of jobs in set $\overline{A_{b}^e}$ to be finite on the $b$-th machine of vertex $v$.
In the completeness case, any perfect labeling gives an assignment where jobs are assigned to exactly one out of the $m$ machines per vertex (similar to the value of the LP solution in the gap example in \Cref{sec:int-gap}).
However, the soundness argument fails, as a low makespan assignment can spread jobs from $U^e$ on multiple machines, and not give any information to recover a good labeling 
(this is similar to the reason we needed multiple size classes in the gap example in \Cref{sec:int-gap}).

To get around this, for each $e$, we will use 
$h$ different set systems $I^{e,1},\ldots,I^{e,h}$ (of geometrically increasing sizes) where the $s$-th set system is the following $I^{e,s} = (U^e(s); A_1^e(s), \ldots, A_m^e(s))$.
Each vertex will have $m$ machines, one for each label. The intended solution is that if vertex $u$ is assigned label $a$, then we pick the sets $A^e_{\pi_e(a)}(1),\ldots,A^e_{\pi_e(a)}(h)$, and assign the corresponding jobs to   the $m$ machines for $u$ with small makespan (this requires some care so that for any label $a$, jobs from different classes $s$ can be assigned to different machines, but we ignore this issue here).

Using the properties of the norm $\psi$ and the arguments in \Cref{sec:int-gap} for the integrality gap, one can show that if $h=\Omega(m)$, then given any schedule with low makespan,
one can construct a good labeling thereby proving soundness. A key new idea here beyond \cite{lund1994hardness} is to show that there is some fixed size class $s^*$, such that the assignment of jobs in class $s^*$ gives a small set of good candidate labels for a large fraction of edges.
However, as the hardness of Label-Cover is only $\Omega(L^c)$ for some small constant $c$ as a function of the number of labels, our resulting hardness is only $\Omega(\log^{c'}n)$ for some small $c'>0$, instead of $\Omega(\log^{1/2} n$).



\subsection{The Reduction}\label{subsec: reduction}
Suppose that we are given a label cover instance, $\mathcal{L} = ((U, V, E), \labels, \Pi)$ satisfying the properties of \Cref{lem: hardness-of-label-cover}, i.e., the number of vertices $|U| = |V| = \lsize$, the degree of every vertex is $d = O((\log \lsize)^{c_1})$ and the number of labels $L = \sqrt{\log \lsize}$. We now describe a polynomial-time (randomized) reduction from $\mathcal{L}$ to a GMP instance $\instance$ with machines ${M}$, jobs ${J}$ and assign processing times for each machine $i\in{M}$ and job $j\in {J}$. 

\medskip

\noindent\textbf{Machines.} For each vertex $w\in U\cup V$, we create $m=\labels = \sqrt{\log \lsize}$ machines. We denote the set of machines corresponding to vertex $w$ by $M_w = \{\mach{1}{w}, \ldots, \mach{m}{w}\}$. We denote the set of all machines by ${M}=\bigcup_{w\in U\cup V}M_w$. In total, we have $2N \sqrt{\log N}$ machines. 


\medskip

\noindent\textbf{Jobs.} For each edge $e \in E$, we create $O(N)$ jobs and partition them into $h=m/8$ size-classes  $U^e(1), U^e(2), \ldots, U^e(h)$ of geometrically increasing size. More precisely, we pick the number of jobs in the $s$-th set to be $|U^e(s)| = \sqrt{N} \cdot\exp(4s\sqrt{\log N})$ which is always $O(N)$ since $s \leq \sqrt{\log \lsize}/8$. Therefore, the total number $n$ of jobs created is $\poly(N)$. We also remark that these sets can be constructed efficiently by the randomized procedure described in \Cref{lem: mlb-set-system} and that this is the only randomized step of the reduction.

\medskip

\noindent\textbf{Processing times.} To assign processing times, for each edge $e\in E$ and  $s \in [h]$,  we construct a $(|U^e(s)|, m,\ell,\beta)$ set system $I^{e,s} = (U^e(s); A_1^e(s), \ldots, A_m^e(s))$ with $\ell = m/10$ and $\beta = \exp(-m)$. For every vertex $u\in U$, label $a\in [\labels]$ and $s\in [h]$, define the set
\begin{align}\label{eq: suas}
\Set{u}{a}{s} = \bigcup\limits_{e\in \delta(u)} A^e_{\pi_e(a)}(s)\text{.}
\end{align}
Similarly for every vertex $v\in V$, label $b\in [\labels]$ and $s\in [h]$, define the set
\begin{align}\label{eq: svas}
\Set{v}{b}{s} = \bigcup\limits_{e\in \delta(v)} \overline{A^e_{b}}(s)\text{.}
\end{align}
For every vertex $w\in U\cup V$, label $a\in [L]$ and size class $s\in [h]$, we choose the processing times of all the jobs in $\Set{w}{a}{s}$ to be ${p^{(s)} = \bt^{s-1}}$ on the machine $\mach{i}{w}$ if the index $i$ satisfies $i\equiv(a+s)\pmod m$. This way of assigning processing times ensures that for a fixed machine $\mach{i}{w}$ and a label $a$, there is at most one set of jobs among $S_{w,a,1}, S_{w,a,2}, \ldots, S_{w,a,h}$ that have finite processing time on $\mach{i}{w}$. This is a useful property to have when we prove the completeness of our reduction.

\medskip
{\noindent \bf Norm.} Define the norm $\psi = \sum_{s\in[h]} \term{s}$, where each $\term{s}$ is a scaled top-$k$ norm given by
\begin{equation}\label{eq:innernorm-final}
    \term{s}(\normvar) = \frac{\Top{\bt^2 n_s}(\normvar)}{(\bt^2 n_s p^{(s)})}
\end{equation}
where $n_s = \left\lvert\bigcup_{e\in\delta(w)}U^e(s)\right\rvert$ is the number of jobs of size class $s$ contained in edges incident to any vertex. Note that since the graph is $d$-regular this number is the same for each vertex.

The above reduction takes $\poly(N)$ time because there are only polynomially many jobs, machines and the $(n,m,\ell, \beta)$ set systems required can be constructed efficiently by \Cref{lem: mlb-set-system}.

\subsection{Analysis}
The set of jobs $\Set{w}{a}{s}$ for any vertex $w$, label $a$ and size class $s$, only contribute to the $s$-th top-k term of $\psi$ on machine $\mach{i}{w}$. Similar to \Cref{def:heavysets}, we define heavy and light size classes and state a lemma whose proof is analogous to that of \Cref{lem:heavy_sets_bound}.
\begin{definition}[Heavy Size Class]
   For a vertex $w$, label $a$ and size class $s$, consider the set $\Set{w}{a}{s}$ of jobs and the machine $\mach{i}{w}$ satisfying $i \equiv (a + s) \pmod m$. For an assignment $\schedule: {J} \rightarrow {M}$, we say that size class $s$ is heavy on machine $\mach{i}{w}$, if $\schedule$ assigns at least $\beta^2 n_s$ jobs from $\Set{w}{a}{s}$ to machine $\mach{i}{w}$; otherwise, we say the size class $s$ is light on machine $\mach{i}{w}$.
\end{definition}

\begin{lemma}\label{lem: norm-is-nice}
    For any vertex $w$, label $a$, size class $s$,  and a machine $\mach{i}{w}$ satisfying $i\equiv a+s\pmod m$, the norm $\psi_{\mach{i}{w}}(\Set{w}{a}{s}) = 1+o(1)$. Furthermore, for an assignment $\schedule:J\rightarrow M$, let $S$ be the set of jobs assigned to machine $\mach{i}{w}$. Then $\psi_{\mach{i}{w}}(S)$ is at least the number of heavy size classes heavy on $\mach{i}{w}$. 
\end{lemma}


\subsubsection{Completeness}
Given a labeling $\sigma$ for the label cover instance $\mathcal{L}$ which satisfies all the edges, we use it to construct an assignment of jobs $\schedule$ with a low makespan. 
\begin{lemma}\label{lem: completeness}
   If the label cover instance $\mathcal{L}$ satisfies $OPT(\mathcal{L}) = 1$, the  instance $\instance$ has an assignment $\schedule: {J} \rightarrow {M}$ with makespan less than $2$.
\end{lemma}
\begin{proof}
     Let $\sigma$ be the labeling of vertices that satisfies all edges in the label cover instance $\mathcal{L}$. Consider the assignment $\schedule$ of jobs to machines constructed using $\sigma$ in the following way: for every vertex $w\in U\cup V$, and size class $s\in[h]$, assign the jobs in  $\Set{w}{\sigma(w)}{s}$ to machine $\mach{i}{w}$ where the index $i$ satisfies $i\equiv\sigma(w)+s\pmod m$. 
     

    For an edge $e=(u,v)\in E$ we first show that each job in the sets  $U^e(1), U^e(2),\ldots, U^e(h)$ is assigned to some machine. For a size class $s\in[h]$, the jobs in  $\Set{u}{\sigma(u)}{s}$ are assigned to a machine in $M_u$, and similarly, the jobs in $\Set{v}{\sigma(v)}{s}$ are assigned to a machine in $M_v$. Since $\pi_e(\sigma(u)) = \sigma(v)$, we infer that $A^e_{\sigma(v)}(s) = A^e_{\pi_e(\sigma(u))}(s)  \subseteq \Set{u}{\sigma(u)}{s}$ and $\overline{A^e_{\sigma(v)}}(s) \subseteq \Set{v}{\sigma(v)}{s}$. It follows that each job in $U^e(s)$ is assigned to some machine. 


    We now bound the makespan of the assignment by showing that each machine is assigned jobs from at most one size class. For a vertex $w$ and $i \in [m]$ consider the machine $w_i$. Since $h < m$, there is at most one $s \in [h]$ for which $\sigma(w) + s \equiv i \pmod m$. Therefore, $\mach{i}{w}$ is assigned jobs from at most one of the sets $\Set{w}{\sigma(w)}{1}, \Set{w}{\sigma(w)}{2}, \cdots, \Set{w}{\sigma(w)}{h}$. Therefore, by \Cref{lem: norm-is-nice} its norm is at most $1+o(1)<2$.
\end{proof}
    
\subsubsection{Soundness}
Next, we show that if the instance $\instance$ has a makespan much less than $\ell$, then $OPT(\mathcal{L})$ is large. Towards this end, we first prove some useful lemmas. Consider an assignment $\schedule$ with makespan $T \leq \ell/100$. Call a size class $s$ to be \textit{good} for a vertex $w$ if it is heavy on at most $32T$ of the machines $\mach{1}{w}, \ldots, \mach{m}{w}$; if not define it to be bad. We first show that a large fraction of size classes are good for any vertex.
\begin{lemma}\label{lem: good-size-classes}
    There are at least $3h/4$ good size classes for each vertex.
\end{lemma}
\begin{proof}
    Let $b$ be the number of bad size classes for some vertex $w$.  
     By averaging over the $m$ machines on vertex $w$, there is a machine $\mach{i}{w}$ for which at least $(32Tb)/m$ size classes are heavy. By \Cref{lem: norm-is-nice},  $\mach{i}{w}$ has  norm at least $(32Tb)/m$. As any machine has norm at most $T$, we get $b \leq m/32 = h/4$ and hence the claim follows.
\end{proof} 


Call a size class {\em good} for an edge $(u,v)$ if it is good for both $u$ and $v$. By \Cref{lem: good-size-classes} each edge $e$ has at least $h/2$ good size classes. 
By averaging over the edges $e \in E$, there must exist a size class $s^* \in [h]$ which is good for at least $|E|/2$ edges.

We will fix the class $s^*$ henceforth, and use it to construct a good label cover solution by assigning a suitable label to each vertex. These labels will only depend on the class $s^*$.
    

\medskip

\noindent{\bf Constructing a good labeling.}
 For each vertex $w \in U \cup V$, we define $L(w)$ to be the set of all labels $a$ such that size-class $s^*$ is heavy on $\mach{i}{w}$ where $i \equiv (a + s) \pmod m$. If no such label exists, add an arbitrary label to $L(w)$.
 \begin{lemma}\label{lem: existence-of-matching-labels}
     Let $e = (u,v)$ be an edge for which $s^*$ is good. There exists $a \in L(u)$ and $b \in L(v)$ such that $\pi_e(a) = b$.
 \end{lemma}
 \begin{proof}
         Assume there are no such labels $a\in L(u)$ and $b\in L(v)$ for which  $\pi_e(a)=b$. Since $|L(u)|+|L(v)|\leq 64T <\ell$, then the union of all the sets $A_{\pi_e(a)}^e(s^*)$ and $\overline{A_b^e}(s^*)$ such that $a\in L(u)$ and $b\in L(v)$ covers at most $(1-\bt)|U^e(s^*)|$ from \Cref{def: mlb-set-system}. 
         
         From the definition of a light size class, for all labels $a\notin L(u)$ (resp. $b\notin L(v)$), at most $\bt^2n_{s^*} = \bt^2 (|U^e(s^*)|\cdot d)$ jobs from the sets $A_{\pi_e(a)}^e(s^*)$ (resp. $\overline{A_b^e}(s^*)$) are assigned to some machines on vertex $u$ (resp. $v$) . Notice that the degree of the graph $(U,V,E)$ is $d = O((\log N)^{c_1})$. So, the union of all the jobs assigned from the sets $A_{\pi_e(a)}^e(s^*)$ (resp. $\overline{A_b^e}(s^*)$) such that $a\notin L(u)$ (resp. $b\notin L(v)$) has at most $(2m\cdot\bt^2\cdot|U^e(s^*)|\cdot d) < \bt|U^e(s^*)|$ jobs which is a contradiction.
\end{proof}
  For each vertex $w$, set $\sigma(w)$ to be a label selected uniformly at random from $L(w)$. We show that $\sigma$ satisfies a large fraction of edges of $\mathcal{L}$ completing the proof of soundness.


\begin{lemma}
\label{lem: soundness}
   If  the  instance $\instance$ has an assignment $\schedule: {J} \rightarrow {M}$ with makespan $T \leq \ell/100$, then $OPT(\mathcal{L}) \geq 1/(2048T^2)$.
\end{lemma}
\begin{proof}
    Consider an edge $e = (u,v)$ for which size class $s^*$ is good. In this case, we have $|L(u)| \leq 32T$ and $|L(v)| \leq 32T$. Also, by \Cref{lem: existence-of-matching-labels} there exists $a \in L(u)$ and $b \in L(w)$ for which $\pi_e(a) = b$. Therefore, $\pi_e(\sigma(u)) = \sigma(v)$ with probability at least $1/(32T)^2$. We have by the analysis above that the number of edges for which $s^*$ is good is at least $|E|/2$. Therefore, the expected number of edges satisfied by $\sigma$ is at least $|E|/(2(32)^2T^2) = |E|/(2048 T^2)$ and the lemma follows.
\end{proof}

\hardnessresult*

\begin{proof}
    Suppose that we have an algorithm that in polynomial time can decide if the $GMP$ instance constructed has a makespan of at least $T$ or at most $2$. By the reduction above, from \Cref{lem: completeness,lem: soundness}  this algorithm can also distinguish between label cover instances that have value $1$ and those that have value at most $1/2048T^2$. Due to the hardness of label cover (\Cref{lem: hardness-of-label-cover}), this is not possible if $1/2048T^2 \geq 1/(\log N)^{\razconstant}$, i.e., if $T \leq O((\log N)^{\razconstant/2}) = O((\log n)^{\razconstant/2})$ since the number of jobs $n=\poly(N)$. This, in particular, implies that any approximation algorithm for GMP has an approximation ratio of at least $\Omega((\log n)^{\razconstant/2})$ provided that $\textsf{NP} \not\subset \textsf{ZTIME}(n^{O(\log \log n)}).$
\end{proof}
  
 
