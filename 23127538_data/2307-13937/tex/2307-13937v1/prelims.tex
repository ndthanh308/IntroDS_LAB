\section{Preliminaries}

\subsection{\texorpdfstring{$(n, m,\ell, \beta)$}{(n, m, l, beta)} Set-System}
The constructions of the integrality gap instance in \Cref{sec:int-gap} and the reduction from label cover to GMP presented in \Cref{sec: reduction} both use the following set system as a building block. 
\begin{definition}[$(n, \m,  \ell, \bt)$ Set-system] \label{def: mlb-set-system}Let $n, m,\ell$ be positive integers, $\beta \in (0,1)$, $U$ be a set with $|U| = n$, and $A_1, \ldots,  A_{\m}$ be subsets of $U$. The sets $(U; A_1, \ldots, A_m)$ form an $(n, \m,  \ell, \bt)$ set-system if for every set $I$ of at most $ \ell$ indices from $[\m]$, $\left\lvert\cup_{i \in I} B_i\right\rvert \leq (1 - \bt) |U|$, where $B_i$ is either $A_i$ or $\overline{A}_i$. 
\end{definition}

Intuitively, an $(n, m, \ell, \beta)$ set-system has the property that any set cover which uses at most $\ell$ subsets necessarily uses a complementary pair of subsets $A_i$ and $\overline{A}_i$. Moreover, any collection of at most $\ell$ subsets that do not contain any complementary pair can cover at most a $(1-\beta)$ fraction of the elements in $U$. The following lemma shows that for a particular choice of parameters $n, m,\ell, $ and $\beta$, there is a simple and efficient randomized construction of an $(n, m, \ell, \beta)$ set-system.

\begin{lemma}\label{lem: mlb-set-system}
    For a sufficiently large positive integer $n$ and a positive integer $m \in [\sqrt{\log n}, 2\sqrt{\log n}]$ there exists an $(n, \m, \ell,\bt)$ set-system $(U; A_1, \ldots, A_m)$
 with $\,\,\ell = m/10$ and $\bt = \exp(-m) = \exp(-O(\sqrt{\log n}))$. There is a polynomial-time algorithm that constructs such a set system with high probability.
\end{lemma}
\begin{proof}
    Let $U$ be a set of $n$ elements, and initialize $m$ empty sets $A_1, \ldots, A_m$. For each element $e \in U$, sample a random index set $J \subset [\m]$ of size exactly $m/2$ and add $e$ to sets $A_j$ for $j \in J$.

We show that this construction gives an $(n, m,  \ell, \bt)$ set-system with high probability.  
    Consider an index set $I \subset[m]$ with $|I| = \ell$ and a collection of sets $B_i$ for $i \in I$ such that each $B_i$ is either $A_i$ or $\overline{A}_i$. For a fixed $e \in U$, let $p$ denote the probability that $e$ is not contained in $\cup_{i \in I} B_i$, i.e., $p = \Pr[e \notin \cup_{i \in I} B_i]$. We have,
    \[
         p \geq \binom{m-\ell}{m/2}/\binom{m}{m/2}
        \geq \left(\frac{m-\ell}{m/2}\right)^{m/2}/\left(\frac{em}{m/2}\right)^{m/2} 
        \geq \left(\frac{m-\ell}{em} \right)^{m/2}
        \geq \exp(-0.6 m),
    \]
where the second inequality uses that  $\left(\frac{n}{k}\right)^k \leq \binom{n}{k} \leq \left(\frac{en}{k}\right)^k$.

The probability that  $\cup_{i \in I}B_i$  contains a fixed subset of cardinality greater than or equal to $(1-\beta)n$ is at most $(1 - p)^{(1-\beta)n}$. By a union bound over at most $ \binom{n}{\beta n}n$ possible subsets with cardinality at least $(1-\beta)n$,
    \begin{align*}\Pr\left[\lvert \cup_{i \in I}B_i \rvert \geq (1 - \bt) n\right] &\leq \binom{n}{\bt n} n \cdot (1 - p)^{(1-\bt)n}
    \leq \left(e/\bt\right)^{\bt n}n \cdot e^{-(1-\bt)np}\\
    &\leq \exp\left(3n\bt  \log(1/\bt) - np/2\right) \leq \exp(-np/4) \leq \exp(-n^{0.9}),
    \end{align*}
where in the second to last inequality we use that $3\bt \log(1/\bt) < p/4$.

A union bound over the at most $2^\ell \cdot \binom{m}{\ell} \leq \exp( \sqrt{\log n})$ possible choices to pick the $\ell$ sets  $B_i$, gives that with high probability, the union of any $\ell$ sets $B_i$ has cardinality less than $(1-\bt)n$. 
\end{proof}

\subsection{Label Cover}
In \Cref{sec: reduction}, we prove the hardness of approximation of GMP via a reduction from the standard label cover problem as defined below. 
\begin{definition} \label{def: label-cover}
A \emph{label cover} instance $\mathcal{L}$ is defined by a tuple $((U,V, E), \labels, \Pi)$. Here $(U,V,E)$ is a bipartite graph with vertices $U \cup V$ and edges $E \subseteq U \times V$; $\labels$ is a positive integer and $\Pi$ is a set of functions one for each edge $e \in E$ i.e., $\Pi = \{\pi_e: [\labels] \rightarrow [\labels] \,\, |\,\, e \in E \}$. A labeling of the vertices $\sigma: U \cup V \rightarrow [\labels]$ is said to satisfy an edge $e = (u,v)$ if $\pi_e(\sigma(u)) = \sigma(v)$. Given $\mathcal{L}$, the goal of the label cover problem is to find a labeling $\sigma^*$ that satisfies the maximum number of edges in $E$. We use $OPT(\mathcal{L})$ to denote the fraction of the edges in $E$ satisfied by $\sigma^*$.
\end{definition}

As we will need the explicit dependence between the number of labels and the soundness, for completeness we 
sketch below the precise gap version of the label cover problem that we will use. 
\begin{lemma}[Hardness of  Gap  Label Cover] \label{lem: hardness-of-label-cover}
Given a label cover instance $\mathcal{L} = ((U,V, E), \labels, \Pi)$ satisfying:
\begin{enumerate}
  \item[(i)] $|U| = |V| = \lsize$
  \item[(ii)]  The degree of every vertex in $U \cup V$ is $d = O((\log \lsize)^{c_1})$ for some constant $c_1$.
  \item[(iii)]
  $\labels = \sqrt{\log \lsize}$
\end{enumerate}
There is some constant $\razconstant > 0$, for which there is no polynomial-time algorithm to decide if $OPT(\mathcal{L}) = 1$ or $OPT(\mathcal{L}) \leq (\log \lsize)^{-\razconstant}$ provided that $\emph{\textsf{NP}} \not\subseteq \emph{\textsf{DTIME}}(\lsize^{O(\log \log \lsize)})$.
\end{lemma}
\begin{proof}
Using a standard argument (see for ex., \cite{feige1996threshold, arora1996hardness}) one can obtain a reduction from a 3SAT-5 instance $\phi$ with $\satvars$ variables to a Label Cover instance $\mathcal{L}_1 = ((U_1, V_1, E_1), 8, \Pi)$, where $|U_1| = |V_1|=O(\satvars)$ and the graph $(U_1, V_1, E)$ is $15$-regular. The  instance $\mathcal{L}_1$ has the following property: if $\phi$ has a satisfying assignment then $OPT(\mathcal L_1) = 1$; else if any assignment satisfies at most $(1-\epsilon)$ fraction of clauses in $\phi$, then $OPT(\mathcal{L}_1) \leq (1 - \Theta(\epsilon))$. By the PCP-theorem \cite{10.1145/278298.278306} it follows that, for some constant $\epsilon_0 > 0$, deciding if $OPT(\mathcal L_1) = 1$ or $OPT(\mathcal L_1) \leq 1 - \epsilon_0$ is \textsf{NP}-hard.


The following well-known construction \cite{arora1996hardness}
gives stronger inapproximability results for label cover. We define the $k$th power of the label cover instance $\mathcal{L}_k = ((U_k,V_k,E_k), 8^k, \Pi^k)$, where $U_k$, $V_k$ are $k$-tuples of vertices in $U_1$, $V_1$ respectively, $E_k$ is the set of all $k$-tuples of edges in $E_1$. The resulting graph has $N = \satvars^{O(k)}$ vertices and is $(15)^k$-regular. The new set of labels\footnote{The labels are essentially numbers from $1$ to $8^k$.} consist of $k$-tuples of $\{1,\ldots,8\}$.  For an edge $e = (e_1, \ldots, e_k) \in E_k$, we define the function $\pi_e^k(a_1, \ldots, a_k) = (\pi_{e_1}(a_1), \ldots, \pi_{e_k}(a_k))$. Raz's Parallel Repetition Theorem \cite{raz1995parallel}, shows that for the label cover instance constructed above, there exists a constant $\alpha$ such that $OPT(\mathcal L_k) \leq (OPT(\mathcal L_1))^{\alpha k}$. 

We now pick $k$ so that $L = \sqrt{\log N}$. Since $L = 8^k$ and $N = \satvars^{O(k)}$, this gives $k = \Theta(\log \log \satvars)$. This choice of $k$ ensures that $d = (15)^k =(\log N)^{c_1}$ for some constant $c_1$. Moreover, if $OPT(\mathcal{L}_1) \leq (1-\epsilon_0)$, then $OPT(\mathcal L_k) \leq (1-\epsilon_0)^{\alpha k} \leq (\log t)^{-c'} \leq (\log N)^{-\razconstant}$ for some positive constants $\razconstant, c'$.
\end{proof}
