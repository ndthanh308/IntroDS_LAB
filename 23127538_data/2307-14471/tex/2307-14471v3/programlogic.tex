\section{Program Logic for Location Virtualization}
\label{sec:logic}
% The predicate gen_heap_interp.
\newcommand{\gammaPred}{\delta}
\newcommand{\gammaPreds}{\delta\textsf{s}}
\newcommand{\rtv}{\textsf{rtv}}
\newcommand{\qone}{\texttt{q1}}
\newcommand{\qtwo}{\texttt{q2}}
\newcommand{\qthree}{\texttt{q3}}
\newcommand{\qfour}{\texttt{q4}}

\newcommand{\sumwalkabs}[3]{
  \ownGhost\gammaPred{\authfrag{\singletonMap{#1}{(#2, #3)}}}
}

\newcommand{\sumapaces}[2]{
  \ownGhost\gammaPreds{\authfrag{\singletonMap{#1}{#2}}}
}
\newcommand{\ptableabswalk}[1]{\mathcal{A}\textsf{bsPTableWalk}(#1)}
\newcommand{\ptablestore}{\theta}

We describe a program logic (a separation logic) along the lines suggested earlier, where every assertion is relative
to an address space in which it is interpreted, allowing us to define \emph{virtual points-to} assertions that make claims
about memory locations in a particular address space. Virtual addresses, and even virtual points-to assertions, 
are not tagged with their address spaces in any way. Memory access in this logic is validated through the use 
of virtual points-to assertions in pre-conditions, which guarantee that address translations succeed.
This supports rules for updating not only typical data in memory that happens to be subject to address translation, but
manipulation of the page tables themselves via virtual addresses (as demanded by all modern hardware), also
via virtual points-to assertions.
To support specifications that deal with multiple address spaces, our logic incorporates a hybrid-style modality
$[r](P)$ to state that an assertion is true in another (assertion-specified) address space rather than the address space
currently active in hardware, which is not only useful for virtual memory manager invariants but critical to reasoning
about change-of-address-space.
By developing this within the \iris framework, we obtain additional features (e.g., fractional permissions) that allow us to verify
some of the most subtle and technically challenging instruction sequences in an OS kernel (Section \ref{sec:experiment}).

% We derive a program logic (a separation logic) supporting the following stances and constraints:
% \begin{enumerate}
% \item \textit{address-spaces as modal contexts}: Assertions in our logic are context-dependent,
%   in the sense that their truth depends on which address space they are used in, due to the need to support virtual points-to assertions.
% \item \textit{sharing}: The physical location backing a virtual address's storage is located (during a page table walk) through a 
%       set of physical page-table (L4-L1 page-tables) acceses that are shared amongst different virtual addresses (specifically,
%       those on the same page of memory\footnote{of in the case of L2 or higher levels of tables, within a given broader region.}).
%       This sharing imposes constraints on defining points-to assertions
%       in terms of physical (L4-L1) page-table memory accesses
% \item \textit{context-agnostic-resources}: each virtual address is valid under a certain address-space, 
%       but it does not represent this \textit{knowledge} of its address-space. 
%       That is, assertions are not explicitly tagged with their address space validity
% \item \textit{updating address-space mappings}: We present logical abstractions to enable 
%       updating not only pages of typical data in memory, but also page tables themselves.\footnote{Prior work relied on unfolding operational semantics
%       to verify page table updates.}
% \item \textit{explicitly-modal assertions}: Our logic includes a means to talk about facts being true
%       in another address space
% \item \textit{address-space switch as changing the "World" of truth}: Switching from one address-space to another logically
%       becomes a simultaneous introduction-and-elimination of a pair of modal assertions (for different address spaces)
% \end{enumerate}

% The idea that the truth of an assertion is relative to an address space has far-reaching consequences.
To support making assertions depend on a choice of address space, we work entirely in a pointwise lifting of Iris's base BI logic,
essentially working with separation logic assertions indexed by a choice of page table root as a $\mathcal{W}_{64}$, which we call $\textsf{vProp }\Sigma$:\footnote{
  \iris experts may notice our \lstinline|-b>| resembles another variety of pointwise lifting that already exists in the main \iris
  distribution~\cite{dang2019rustbelt,dang2022compass}. This is essentially true, but the standard formalization does not work for index types which
  are indexed in the sense of \coq's metatheory, as our \lstinline|word n| types are.
}
\begin{lstlisting}[language=Coq]
  Definition vProp  $\Sigma$ : bi := word 64 -b> iPropI  $\Sigma$.
\end{lstlisting}
This is the (\textsc{Coq}) type of assertions in our logic.
Most constructs in Iris's base logic are defined with respect to any BI-algebra (of \textsc{Coq} type \lstinline|bi|), so automatically
carry over to our derived logic.
However, we must still build up from existing Iris primitives to provide new primitives that depend on the address space --- primarily the notion
of virtual points-to.

% % Figure environment removed

\subsection{Base Points-To Assertions}
\label{sec:pointsto}
We build up our context-dependent assertions from two basic types of assertions that do not care about address spaces.

\paragraph{Register points-to} 
The assertion $\textsf{r}\;\mapsto_{\textsf{r}}\{q\}\;\textsf{rv}$ ensures the ownership of the register $\rg$ naming the register value $\rv$. 
The fraction $\qfrac$ with value 1 asserts the unique ownership of the register mapping, and grants update permission on it, 
otherwise, any value $0 < \qfrac <1$ represents partial ownership granting readonly permission on the mapping.

\paragraph{Physical memory  points-to} The soundness proofs for our logic's rules largely center around 
proving that page-table-walk accesses as in Figure \ref{fig:pagetables} succeed, which requires assertions
dealing with physical memory locations.
We have two notions of physical points-to facts. The primitive notion closest to our machine model is captured by an assertion
$ \textsf{pfn} \ \sim \ \textsf{pageoff} \mapsto_{\textsf{a}} \; \{\textsf{q}\} \; \textsf{v} $, where \textsf{pfn} (a $\mathcal{W}_{52}$ \emph{page frame number}) essentially selects a 4KB page of physical memory,
and \textsf{pageoff} (a $\mathcal{W}_{12}$) is an offset within that page.
% could be an 52-bits masked address to level 4 table 
% ($\textsf{w1 } =( \textsf{ l4M52 maddr cr3val}) $),
%and, expectedly \textsf{w2} is an address computed by page-offset computation (e.g. $\textsf{l4off maddr cr3val}$). 
From this we can derive a more concise physical points-to when the split is unimportant:
% Giving a raw 64-bits memory pointsto assertion becomes
\[\textsf{w} \mapsto_{\textsf{p}} \{q\} \textsf{ v} \stackrel{\triangle}{=} (\textsf{drop 12}~w) \ \sim \ (\textsf{bottom 12}~w)\mapsto_{\textsf{a}} \; \{\textsf{q}\} \textsf{ v} \]
 %I put a newline in between the following two Definitions as the second Definition seems not proper without the newline
% Figure environment removed



\subsection{An Overly-Restrictive Definition for Virtual Memory Addressing}
\label{sec:overly-restrictive}
A natural definition for a virtual points-to that depends on the current address space is to
require that in order for a virtual address \textsf{va} to point to a value \textsf{v}, the assertion contains
partial ownership of the physical memory involved in the page table walk that would translate \textsf{va} to
its backing physical location --- with locations existentially quantified since a virtual points-to should not assert
\emph{which} locations are accessed in a page table walk.
Figure \ref{fig:strongvirtualpointsto} gives such a definition.
It asserts the existence of four page table entries, one at each translation level, and via \lstinline|L4_L1_PointsTo|
asserts from the current \lstinline|cr3| value, taking the offsets from \textsf{va} (as in Figure \ref{fig:pagetables})
to index each level, finding the existentially quantified entries in each level, 
ending with the L1 entry, whose physical page base is added to the page offset of \textsf{va} to obtain the physical address
 --- and there (\lstinline|addr_L1(va,l1e)|) in physical memory, is the value.
Most of the definition lives directly in \textsf{vProp}, using the separation logic structure lifted from Iris's \textsf{iProp}.

\lstinline|L4_L1_PointsTo| works by
chaining together the entries for each level, using the sequence of table offsets from the address being translated to index
each table level, and using the physical page address embedded in each entry.\footnote{
  The fractions \lstinline|q1| through \lstinline|q4| represent the fractional ownership of each entry based on how many
  word-aligned addresses might need to share the entry.
}
For example, 
  the first level address translation to get the L4 entry (\lstinline|l4e|) 
  uses the masks \textsf{l4M52} with the current \lstinline|cr3| to get the 
  physical address of the start of the L4 table,
  and \textsf{l4off} with the virtual address being translated to compute the correct offset within that table.
  Then at that physical location is the appropriate entry in the L4 table \textsf{l4e}.
  Subsequent levels of the page table walk work similarly.
The statement of these assertions is simplified by the use of our split physical points-to assertions, since
each level of tables is page-sized.\footnote{We do not address superpages and hugepages in this paper.}
This helper definition is also a more explicit \textsf{vProp}, explicitly binding a value for \lstinline|cr3| and using it
to start the translation process.


% Given the definition of physical page-pointsto assertion and the root address of virtual-address space as shown in Figure \ref{fig:pagetable}, one can build the physical address-translation for a virtual address (e.g. \textsf{va}) via abstracting the L4-L1 table traversal as the following:
% \begin{itemize}
%   \item Level-4 Translation (L4): Performs 
%   the first level address translation to get the L4 entry (L4E) in Figure \ref{fig:pagetables} by using the masks l4M52 and l4off with \textsf{rtv} virtual base address to get the starting address of the L4 table (L4T), and locate the entry amongst 512 (q1) ones respectively.
%     \begin{lstlisting}[language=Coq]
%       $\hbox{(\TirNameStyle{{L4translate}})} \quad$ (l4M52 maddr rtv) \$\sim$\ (l4off maddr rtv) $\mapsto_{a}$ {q1} l4e 
%     \end{lstlisting}
%  \item Level-3 Translation (L3): Performs the second level address translation to get the L3 entry (L3E) in Figure \ref{fig:pagetables} by using the masks l3M52 and l3off with L4 entry (l4e) to get the starting address of the L3 table (L3T), and locate the entry amongst 512 ones (q2) respectively.
%     \begin{lstlisting}[language=Coq]
%     $\hbox{(\TirNameStyle{{L3translate}})}$ (l3M52 maddr l4e) \$\sim$\ (l3off maddr l4e)  $\mapsto_{a}$ {q2} l3e
%     \end{lstlisting}
%   \item Level-2 Translation (L2): Performs the second level address translation to get the L2 entry (L2E) in Figure \ref{fig:pagetables} by using the masks l2M52 and l2off with L3 entry (l3e) to get the starting address of the L2 table (L2T), and locate the entry amongst 512 ones (q3) respectively.
% \begin{lstlisting}[language=Coq]
%     $\hbox{(\TirNameStyle{{L2translate}})}$ (l2M52 maddr l3e) \$\sim$\ (l2off maddr l3e)  $\mapsto_{a}$ {q3} l2e
%     \end{lstlisting}
%   \item Level-1 Translation (L1): Performs the second level address translation to get the L1 entry (L1E) in Figure \ref{fig:pagetables} by using the masks l1M52 and l1off with L2 entry (l2e) to get the starting address of the L1 table (L1T), and locate the entry amongst ones 512 (q4) respectively.
%    \begin{lstlisting}[language=Coq]
%     $\hbox{(\TirNameStyle{{L1translate}})}$ (l1M52 maddr l2e) \$\sim$\ (l1off maddr l2e)  $\mapsto_{a}$ {q4} l1e
%     \end{lstlisting}
%   \item Page Address Level Translation: Final computed physical page-address ($\textsf{addr\_L1}(\vaddr$,$\textsf{l1e}$)) points-to the value stored in the address ($\vpage$).
%    \begin{lstlisting}[language=Coq]
%     $\hbox{(\TirNameStyle{{PageLevelAccess}})} \qquad$ addr_L1($\vaddr,\textsf{l1e}$) $\mapsto_{p} \vpage$.
%     \end{lstlisting} 
% \end{itemize}

This solution is in fact very close to that of \citet{kolanski08vstte}, who define a separation logic from scratch in \textsc{Isabelle/HOL},
where the semantics of all assertions are functions from pairs of heaps and page table root values to booleans.\footnote{
  This was a typical explicit construction at the time; their work significantly predates Iris.
}
Our solution removes some restrictions in this definition by further abstracting the handling of address translation.

\subsection{Aliasing/Sharing Physical Pages}
  \label{sec:sharingpages}  
  The virtual points-to definition shown in Figure \ref{fig:strongvirtualpointsto} 
  is too strong to specify some operations a virtual memory manager may need to do, such as
  move one level of the page table to a different physical location while preserving all virtual-to-physical mappings.\footnote{
    x86-64 hardware, like other architectures, includes a feature (which we do not formalize assertions for) to
    replace an L1 page table address in an L2 entry with a pointer to a \emph{larger} 2MB page (called super-pages), 
    or replace an L2 page table address in an L3 entry with a pointer to a 1GB page (called huge-pages).
  }
  The use of $\textsf{L}_{4}\_\textsf{L}_{1}\_\textsf{PointsTo}$ in Figure \ref{fig:strongvirtualpointsto}'s
  virtual points-to definition'
   stores knowledge of the page table walk's details with ownership of the backing physical
  memory.
  Updating any of these mappings (e.g., moving the page tables in physical memory, as in coallescing for
  superpages or hugepages)
  would require explicitly collecting all virtual points-to facts that traverse affected entries.
  It is preferable to permit the page tables themselves to be updated independently of the virtual points-to assertions,
  so long as those updates preserve the same virtual-to-physical translations.

  % iris.sty lacks nice syntax for the ghost maps
  \newcommand{\ghostmaptoken}[3]{\ensuremath{#2\hookrightarrow^{#1}#3}}
  \newcommand{\fracghostmaptoken}[4]{\ensuremath{#2\hookrightarrow^{#1}_{#4}#3}}

\newcommand{\vale}{\textsf{val}}
% Figure environment removed  

  Intuitively, the definition in Figure \ref{fig:strongvirtualpointsto} is too strong because the virtual points-to
  assertion there tracks too much information: when writing programs that access memory via virtual addresses,
  most code does not care \emph{which physical memory locations are involved in address translation}: it only cares
  that virtual address translation would succeed. The necessary information about the physical page table walk
  must still be tracked, but can be tracked separately from the virtual points-to assertion itself.
  In practice the decisions about which virtual addresses are valid rest not with code posessing a virtual address, but with
  the virtual memory manager --- and its invariants.

  We separate the physical page-table-walk  
  from the virtual pointsto relation, replacing it with ghost state that merely guarantees the address translation would succeed.
  Iris includes a \emph{ghost map} construction which we use to track mappings from
  virtual addresses to the physical addresses they translate to.
  The implementation details in terms of named invariants and various resource algebras are not important to our presentation, so
  we treat the maps abstractly.
   The map includes for each key in the map (i.e.,
  each virtual address) a token $\ghostmaptoken{\gamma}{k}{v}$ whose ownership is required to update that key-value pair in the ghost map
  named $\gamma$. The existence of such a token implies that the actual map $\theta$ tracked by a corresponding $\mathsf{GhostMap}(\gamma,\theta)$
  resource indeed maps $k$ to $v$. These properties are captured by the Iris rules:\footnote{Iris ghost maps lack established notation in the literature,
  but the syntax we use captures the details of \texttt{iris.base\_logic.lib.ghost\_map}.}

  \begin{mathpar}
  \inferrule{ }{\mathsf{GhostMap}(\gamma,\theta) \ast \ghostmaptoken{\gamma}{\mathsf{pa}}{\mathsf{va}} \vs 
  \mathsf{GhostMap}(\gamma,\theta[\mathsf{pa}\mapsto\mathsf{va'}]) \ast \ghostmaptoken{\gamma}{\mathsf{pa}}{\mathsf{va}'} 
  }
  \and
  \inferrule{ }{\mathsf{GhostMap}(\gamma,\theta) \ast \ghostmaptoken{\gamma}{\mathsf{pa}}{\mathsf{va}} \wand \ulcorner \theta(\mathsf{pa})=\mathsf{Some}(\mathsf{va}) \urcorner }
  \end{mathpar}
  
  % Updating the virtual-to-physical (ghost) mapping for a virtual address $va$
  % requires full ownership of the token. 
  Then the \emph{virtual memory manager's invariant} ensures that for each $\ghostmaptoken{\gamma}{va}{pa}$ mapping
  in this map, there are \emph{physical} resources sufficient to ensure that address translation for $va$
  will resolve in the hardware to $pa$ via --- $\textsf{L}_{4}\_\textsf{L}_{1}\_\textsf{PointsTo}$.
  This allows modification of the page tables themselves as long as the changes do not invalidate overall virtual-to-physical
  translation.
  Consistency follows from replacing the page table walk with ownership of the ghost map token for the relevant ghost map
  entry, preventing the overall translation from being changed separately from the virtual points-to.
  \looseness=-1

  %Thus the specification of \emph{which} physical addresses support translation is separated from the virtual points-to.
  % But these physical resources, which specify \emph{which} physical locations
  %are involved in the page table walk, are now separated from but consistent with
  %the knowledge that such resources exist (which is embodied by the token for $va$, which tracks that $va$ maps to $pa$
  %in the ghost map). Thus we can store the token which summarizes the translation and ensures it exists in the virtual
  %points-to, and keep the ghost map and the invariant that every mapping in the ghost map has corresponding physical resources
  %for translation in a separate global invariant for each address space.

  % In Iris this is realized by using an authoritative resource algebra: there is a single \emph{authoritative} global copy of the (ghost)
  % map caching virtual-to-physical address translations, and for each entry a read only \emph{partial} ownership of that key-value pair.
  % The resource algebra itself is instantiated as:
  % \[\mathcal{A}\textsf{bsPTableWalk} \stackrel{\triangle}{=} \textsc{Auth} (\; \mathcal{W}_{64} \;\rightarrow_{\textrm{fin}} \;  ( (\textsc{Frac }, \mathord{+}) \times (\textsc{Agree } \Loc,\mathord{=}) ))\]
For clarity we refer to the ghost map summarizing virtual-to-physical translations by 
\[\mathcal{A}\textsf{bsPTableWalk}(\delta,\theta) \stackrel{\triangle}{=} \mathsf{GhostMap}(\delta,\theta)\]
(omitting $\delta$ for brevity when only one is in scope)
and keep this in a per-address-space invariant described shortly.
We then replace the physical
 $\textsf{L}_{4}\_\textsf{L}_{1}\_\textsf{PointsTo}$ in the virtual points-to definition
with ownership of the token \ghostmaptoken{\delta}{\mathsf{va}}{\mathsf{paddr}}, %ghost-map ($ \sumwalkabs\vaddr\qfrac\paddr$),
guaranteeing that ghost map contains an mapping from the virtual address ($\vaddr$) to a physical address ($\paddr$),
and thus that the per-address-space invariant contains the physical resources that guarantee the hardware resolves the translation.

We place the authorative ownership of the ghost translation $\mathcal{A}\textsf{PTableWalk}$ in a per-address-space invariant
$\mathcal{I}$\textsf{ASpace} (Figure \ref{fig:peraspaceinvariant}). 
$\mathcal{I}$\textsf{ASpace} alone allows changes to the page tables 
that preserve overall virtual-to-physical translations.
When combined with the
fragment stored in the virtual points-to (Figure \ref{fig:virtualpointstosharing}),
changes in the virtual-to-physical translations become possible.

  % Figure environment removed

Because our logic is the first to address the existence of multiple address spaces,
the ghost map of virtual-to-physical translations (identified by the ghost name $\delta$ in Figure \ref{fig:virtualpointstosharing}
 \emph{for each address space}) must be locateable from a representation of the specific address space:
% This means we also require a way to locate the map for the current address space from 
the current root page table pointer stored in \lstinline|cr3|.

Figure \ref{fig:peraspaceinvariant} describes the resources for each address space.
$\theta$ is the logical map from virtual addresses to the physical addresses they should translate to, 
for the \emph{current} address space, corresponding to the currently installed page tables (the ones
indicated by \lstinline|cr3|). In Figure \ref{fig:virtualpointstosharing}, the existentially-quantified $\delta$
is the ghost name for this map (again, for the current address space).
$m$ is the (full) logical map from the various page table roots to ghost names for the 
per-address-space mappings like $\theta$.
It is tracked by a ghost map named by the ghost name $\delta{}s$ in
Figure \ref{fig:virtualpointstosharing}; there is only one such map in the system.
\looseness=-1
% This is why Figure \ref{fig:virtualpointstosharing} asserts there is some ghost name in that (global) map
% for the current address space.
% The global map itself is tracked by a resource algebra 
% \[\mathcal{A}\textsf{spaces} \stackrel{\triangle}{=} \textsc{Auth} (\; \mathcal{W}_{64} \;\rightarrow_{\textrm{fin}} \;  ( (\textsc{Frac }, \mathord{+}) \times (\textsc{Agree } \textsc{GhostName},\mathord{=}) )) \]


$\mathcal{I}\textsf{ASpace}(\Theta,m)$ is then the resources for the current address space,
ensuring that the current address space is valid (has an entry in $m$ whose corresponding
authoritative map matches $\Theta$), and asserting ownership of the appropriate fractional ownership
of the physical page table walk for each virtual-to-physical translation in $\Theta$ (for the current address space).



  
% Figure \ref{fig:logicaladdrspace} depicts the relationships between these pieces, for a single
% address space.
% The invariant for the currently active address space (solid round square labelled as 
% $\mathcal{I}$\textsf{ASpace} in Figure \ref{fig:logicaladdrspace}) asserts authorative ownership (i.e. 
% update capability), $\ownGhost\gammaPred{\authfull{\ptableabswalk\ptablestore}}$, of the ghost map $\Theta$
% describing virtual-to-physical address translations
% -- shown as a solid-round-head arrow to $\mathcal{A}\textsf{PTableWalk}$.
% Each virtual points-to assertion (dashed round square labelled as \textsf{Virtual PointsTo}
%  in Figure \ref{fig:peraspaceinvariant})
% asserts fragmental ownership of the ghost map $\Theta$, $\sumwalkabs\vaddr\qfrac\paddr$,
% shown via the dashed-diamond-head arrow to $\mathcal{A}\textsf{PTableWalk}$. Specifically, it asserts that
% the virtual address the virtual points-to assertion concerns corresponds to some physical address.
% There exists only one persistent authorative ownership 
% ($\ownGhost\gammaPred{\authfull{\ptableabswalk\ptablestore}}$) which collects the fragmental 
% ghost page-table-walk mapping ($\sumwalkabs\vaddr\qfrac\paddr$),
%  shown as double-ended solid arrow 1 to N imposing the update protocol.

% % Figure environment removed

\paragraph{From A Single Address Space to Many}
In principle the resources from each individual address space should be collected into a single
shared invariant, and for each memory access,
a fragment of this corresponding to $\mathcal{I}\textsf{ASpace}(\Theta,m)$
for the current address space should be extracted from this global resource, used to prove correctness of the
memory access, and put back.
In this paper we focus on the middle section, explicity identifying resources for each
address space. The reason for this is that in practice, this global resource is also
the place where kernel-specific assumptions, such as guaranteeing that certain virtual address range
was mapped in all address spaces, would be enforced. This paper focuses on the
reasoning principles behind the virtual memory access, and we leave the use of this within a larger
kernel to future work. We do, however, deal with the existence of multiple address spaces in examples (Section \ref{sec:experiment}).

\subsection{Address-Space Management}
\label{sec:aspacemanagement}
So far, we have introduced logical abstractions for a single address space, but VMMs
 handle more than one address-space,
and doing so requires a way to talk about other address spaces, and means to switch address spaces.
% requiring a means to  which we abstract as a ghost map from root address of an address-space to a namespace labelling the address space

% \paragraph{Valid Context-Resources} Fragmental ownership of $\mathcal{A}\textsf{spaces}$ ($\sumapaces\rtv\gammaPred$) conveys the fact that a current address-space ($\gammaPred$) is the one selected, i.e. loaded into the $\rtv$. This simple fact becomes essential when we observe it inside virtual-pointsto definition in Figure \ref{fig:virtualpointstosharing} and pictorially Figure \ref{logicaladdrspace}. Existence of this fragmental ownership ensures that the contextual resource, i.e. a virtual pointsto assertion, resides under the correct address-space, i.e. modal-context, that makes it valid.

% \paragraph{Modal Resource Context\&Interaction with Ambient Logic \textsf{vProp}}
% \label{sec:resourcecontext}
% The relation between the address-space and and the validity of virtual-pointsto assertions exhibits itself as a contingency, i.e. \textit{"a certain virtual-pointsto facts whose validity is indexed by the root address of an address space"}. But, what sort of convenience this pattern provides when we do reasoning about more than one address-space.

% In Figure \ref{fig:addrswitch}, we see two different address-spaces rooted at R0 and R1 in the pre/post condition. To be put into words, while switching an address-space by executing \lstinline|mov_ctl|) updating the  \lstinline|cr3|, it is expected to mention resources of two different address-spaces in the specification. In the precondition,  \lstinline|cr3| is loaded with R0 and all the virtual-pointsto assertions $a_0 \mapsto_{\textsf{v}}p_0$ and $a_1 \mapsto_{\textsf{v}}p_1$ are in the current-view of the memory and valid in the ambient logic \textsf{vProp}. The other address-space with the root address R1 has its own resources ($b_0 \mapsto_{\textsf{v}}p_3$ and $b_1 \mapsto_{\textsf{v}}p_4$) which are invalid under the current address space rooted at R0. This pattern of specification requires resources being mentioned under the \textit{world} they are valid, i.e. a modal-framing within the specification. Once the current address-space becomes the one with root address R1, i.e.  \lstinline|cr3| is loaded with R1, the context-resources $b_0 \mapsto_{\textsf{v}}p_3$ and $b_1 \mapsto_{\textsf{v}}p_4$ are loaded into the current-view of the memory, and all the other resources that are valid under R0 are introduced into the modal context rooted at R0
% Figure environment removed
Figure \ref{fig:modaldef} gives the definition of our modal operator for asserting truth of a modal
(address-space-contingent) assertion \emph{in another address space}, which we call
the \emph{other-space} modality. The definition itself is not
particularly surprising --- as our modal assertions are semantically predicates on a page table root (physical)
address, the assertion $[r](P)$ is a modal assertion which ignores the (implicit) current page table root,
and evaluates the truth of $P$ as if $r$ were the page table root. 
The novelty here is not in the details of the definition, but in recognizing that this is the right way to deal with
multiple address spaces, and working out how to support interaction of multiple address spaces (discussed in the next section).
% This simple definition is 
% what we need in order for assertions in one address space to talk about assertions in another.
\looseness=-1

% The inspiration for this assertion comes from hybrid logic~\cite{blackburn1995hybrid,areces2001hybrid,goranko1996hierarchies,gargov1993modal},
% which as mentioned earlier uses as modalities the names of worlds in a Kripke model to state
% that an assertion is true in that specific world.
% Conceptually the address space rooted at some physical address (intended as the start of an L4 table)
% is a distinct world, and it makes sense to name it by referring to the root of the
% supporting page table structure. However, rather than explicitly building a Kripke model for our
% assertions, we inherit them by lifting the model of our ambient logic, Iris.

We can prove that this modality follows certain basic laws: that its truth is independent of the address
space in which it is considered,  it distributes over various logical connectives, and it follows the rule of
consequence.
We call \textsf{vProp} assertions whose truth is independent of the current address space
\textsf{Fact}s; these include other-space assertions, physical memory points-tos, and register assertions.
\textsf{Fact}s can move in and out of other-space modalities freely.



  
% % Figure environment removed

% \begin{remark}[The Choice of Modal Context, Contingency and  Pattern of Verification Context]
%   \label{remark:pattern}
% The truth on an address space, exhibits itself as a contingent truth: a location virtualization assertion (a virtual points-to in \ref{fig:virtualpointstosharing}),  happens to valid in the \textit{current world} (in the current address space), and switching address spaces pulls information out of one \textit{world} into the “current view” of memory, and leaves other assertions true relative to the previous address space.

% Therefore, an address-space, as an abstraction, can be treated as naming the memory state as a modal frame and the choice of page table root as a world in Kripke-style semantics. Not suprisingly, transition between two address-spaces, then, can be just an entailment relation \textit{alternating} the \textit{named-state}

% Being inspired by what hybrid logic \ref{} calls a satisfaction operator, which evaluates the truth of a predicate in a named alternative state (here, address space), we give a modal definition describing the truth of assertions for the resources (i.e. virtual pointsto relations) inside the address space. Ignoring the predicate types for a while, $[r]P$ in Figure \ref{fig:modaldef}, indicates that $P$ holds in the virtual address space rooted at $r$, the truth ($P$) on an address-space is indexed by the root page-table address $r$ of the address-space. In the rest of this section, we explain the structural aspects -- the modal resource context of address space modality -- , and how we lift the interaction of address-space modality with the ambient logic, i.e. separation logic as an entailment for specifying \textit{address-space-switch}.

% Pragmatically, 
%   \begin{enumerate}
%   \item a modal context with its context-resources and picked contingency defines a modality (e.g. a address-space modality $[r]P$). For the convenience of verification, it enables focusing on the certain facts in the interest of reasoning (e.g. virtual pointsto relations in the current address-space)
%   \item and, offloads the burden of individual bookkeeping of these facts (e.g. virtual pointsto relations per address-space) under different context (e.g. virtual pointsto relations in other address-spaces) via utilizing the \textit{summarization} aspect of the contingency it represents (e.g. other virtual pointsto relations are valid under other certain address-space page-table root addresseses).  
%   \end{enumerate}
% \end{remark}

\subsection{Subtleties of Changing Address Spaces Using Modalities}
% \subsection{Considering Hoare Doubles in the Context of Switching Address-Space}
% \subsubsection{Issues with Using Hoare Triples in Address-Space Specification}
\label{sec:issues}
Operating systems on x86-64 change address spaces by updating \lstinline|cr3|.
Conceptually, we expect a proof rule for such an update to take move resources in and out of modalities for
the old and new address spaces. This suggests a rule like the following, which captures the intuition of the
address space change, but is subtly unsound under common assumptions.
% An important subtlety arises with supporting \lstinline|mov|s into \lstinline|cr3|. Consider the hypothetical rule
\begin{mathpar}
\inferrule*[left=Broken]{ }{
  \{P \ast \textsf{cr3}\mapsto_{r} \;r_1 \ast r \mapsto_{r} \;r_2 \ast [r_2](Q)\}
  \texttt{mov}~\texttt{\%cr3},~r%\lstinline|mov %cr3, r| 
  \{[r_1](P) \ast \textsf{cr3} \mapsto_{r} r_2 \ast r \mapsto_{r} r_2 \ast Q\}
}
\end{mathpar}
This rule takes all assertions not under a modality ($P$), and moves them under a modality for the old address space
(which is where they were valid), while pulling assertions for the new, target address space out of a corresponding modality.
(For clarity, we also omit the address space invariants, which must also be present.)
% which is slightly simplified (by dropping per-address-space resources) for clarity in making the following point.
% This rule captures the intuitive change of address space in a Hoare triple.
Unfortunately, as stated, this interacts quite poorly with the traditional frame
rule.
If we abbreviate the pre- and post-conditions above as \textsf{Pre} and \textsf{Post}:
\begin{mathpar}
  \inferrule*[right=Frame]{
    \inferrule*[right=Broken]{ }{
    \{\textsf{Pre}\}
    \texttt{mov}~\texttt{\%cr3},~r%\lstinline|mov %cr3, r| 
    \{\textsf{Post}\}
    }
  }{
    \{a\mapsto_\mathsf{v} x \ast \textsf{Pre}\}
    \texttt{mov}~\texttt{\%cr3},~r%\lstinline|mov %cr3, r| 
    \{a\mapsto_\mathsf{v} x \ast \textsf{Post}\}
  }
\end{mathpar}
Notice that both the precondition and postcondition assert that $a\mapsto_\mathsf{v} x$ in the current address space, but we have no basis for concluding that address translation is preserved by the change of address space. So this derivation clearly leads to an unsound conclusion. 
The essential problem is that the frame rule is motivated by local reasoning about local updates, but
a switch of address space is a \emph{global} change that may invalidate information about virtual addresses.
Thus framing around \lstinline|cr3| updates is unsound, which has important impacts on our formal development.

Switching to Hoare doubles resolves this problem because an under-appreciated subtlety of Hoare doubles 
is that typically \emph{there is no primitive frame rule}. 
Instead each verification essentially includes a local frame that it passes to the next instructions
(think continuation-passing style), giving each overall rule a \emph{global} (rather than local) precondition.
For most rules this is not that important, but it does permit rules that have global effects on their preconditions.

% This is then how we justify our actual rule for \lstinline|cr3| updates, which is \emph{almost}
This leads to our actual rule for \lstinline|cr3| updates, which is \emph{almost}
\begin{mathpar}
\inferrule*[left=ChangeAddressSpace]{
  \{[r_1](P) \ast cr3 \mapsto_{r} r_2 \ast r \mapsto_{r} r_2 \ast Q\}\;\overline{is}
}{
  \{P \ast cr3 \mapsto_{r} r_1 \ast r \mapsto_{r} r_2 \ast [r_2](Q)\}\;
  \texttt{mov}~\texttt{\%cr3},~r;\;\overline{is}
  %\lstinline|mov %cr3, r| 
}
\end{mathpar}
(Again, the actual rule in Section \ref{sec:selected_rules} includes the address space invariants.)
Because the precondition on this rule is global, we avoid issues with framing. 
A more typical frame rule can be recovered from Hoare doubles in the typical way~\cite{Chlipala2013Bedrock} 
with the added restriction that \lstinline|cr3| is restored to its pre-framing value before un-framing (in this case, the framed-out assertion is effectively
stashed under an other-space assertion for the initial address space, and pulled out as part of restoring).

% If we wanted to consider a frame rule that would work for this logic, we could consider:
% \begin{mathpar}
%   \inferrule*[left=Cr3Frame]{
%     \{P\ast cr3=v\}\;C\;\{ Q \ast cr3=v\}
%   }{
%     \{R\ast P\ast cr3=v\}\;C\;\{ R \ast Q \ast cr3=v\}
%   }
% \end{mathpar}
% By demanding that \lstinline|cr3| be held constant (or rather, at least restored to its original value) we could frame almost traditionally. In particular, this rule would work with framing around calls that might lead to address space switches, such as calling blocking operations in the kernel.
% Attempting to derive a frame rule as some work using Hoare doubles does~\cite{Chlipala2013Bedrock}
% would enforce this restriction.

% Another useful perspective on this might be the dynamic frames~\cite{parkinson2011relationship} 
% view: virtual points-to assertions are self-stable \emph{except} for changes in \lstinline|cr3|, so framing would then naturally require other means of holding \lstinline|cr3| constant (or saving and restoring it).
% Virtual points-to assertions could be made self-stable by also giving them partial ownership over \lstinline|cr3| assertions, but this would require explicitly plumbing that ownership from \emph{all} assertions back to any place an address space change might occur; this would seem to be a far graver loss of modularity than this extra quirk in framing discussions.


\subsection{Selected Logical Rules}
\label{sec:selected_rules}
Per the discussion in Section \ref{sec:issues}, we define our logic using Hoare doubles:\footnote{This
omits some low-level Iris details (stuckness, observations) that play no meaningful
role in our development.}
  \[
  \begin{array}{l}
    %\textsf{wpd\_def e s E1 } \Phi \;\mathsf{ rtv } : \textsf{iProp }\Sigma := \\
    \{ \Phi \}_\mathsf{ rtv }\;\textsf{e} : \textsf{iProp }\Sigma := 
   % \qquad
   ((\textsf{cr3} \mapsto_{\textsf{r}} \textsf{rtv} \ast \Phi) \textsf{ rtv}) \wand \textsf{WP e } \{\_, \textsf{True} \}
    \end{array}
  \]
Our Hoare doubles $\{\Phi\}_\textsf{rtv}\;\textsf{e}$ state that the expression (i.e., sequence of instructions)
\textsf{e} are safe to execute (will not fault)
when executed with \textsf{vProp} precondition $\Phi\ast\textsf{cr3}\mapsto_{\textsf{r}} \textsf{rtv}$.
\textsf{WP} is Iris's own weakest precondition modality, unmodified~\cite{jung2018iris}.
Making \textsf{rtv} an explicit parameter to the double, rather than simply using register assertions
directly solves a technical problem with ensuring that the page table root used to evaluate
the \textsf{vProp} (i.e., evaluating the assertion in the \emph{current}) address space
is feasible.\footnote{Consider the difficulty of selecting the correct page table root value from an arbitrary
opaque $\Phi$, which may even existentially quantify the page table root. An alternative is to
require $\Phi$ to have a syntactic form where we can directly extract the value of \lstinline|cr3|,
but this makes using Iris Proof Mode (IPM)~\cite{Krebbers:2017:IPH:3009837.3009855} with \textsf{vProps}
  difficult; IPM works for any type matching the signature of an Iris \lstinline|bi|, which includes
  \textsf{vProp}s, but manually guiding IPM to put an assertion in a specific position over and over adds
  significant proof burden.
}

The rest of this section describes specifications of some selected \textsf{AMD64} instructions 
in our logic. 
These rules and others (e.g., including accessing memory at an instruction-specified offset from a register
value, which is common in most ISAs)
can be found in our artifact.
Each rule in Figure \ref{fig:wpdamd}, 
is annotated with a root (i.e., \lstinline|cr3|) address value (\textsf{rtv}), 
under which the resources mentioned in the specification are valid.
In general, we use metavariables $\textsf{r}_s$ and $\textsf{r}_d$ to specify source and destination registers
for each instruction, and prefix various register value variables with \textsf{rv}.
We sometimes use $\textsf{r}_a$ to emphasize when a register is expected to hold an address used
for memory access, though the figure also uses typical assembler conventions of specifying
memory access operands by bracketing the register holding the memory address.
Standard for Hoare doubles, there is a frame resource $P$ in each rule for passing resources
not used by the first instruction in sequence through to subsequent instructions.
Our rules include tracking of each instruction's memory address to track \lstinline|rip| updates, which is critical
for control transfer instructions. Our development also includes handling of the \lstinline|rflags| register updates from arithmetic instructions.
Most rules are otherwise standard (e.g., \lstinline|mov| between register, etc.), with Figure \ref{fig:wpdamd} showing the rules
most unique to our development.
\looseness=-1

% Figure environment removed


\paragraph{Accessing Virtual Addresses}
Figure \ref{fig:wpdamd} includes two  rules for accessing memory at an address stored in a register $r_a$. 
Ultimately, any memory access needs to ensure the relevant address translation would succeed,
which can be ensured by what we informally call the page-table-walk points-to collection
($\textsf{L}_{4}\_\textsf{L}_{1}\_\textsf{PointsTo}$ in Figure \ref{fig:strongvirtualpointsto}).
\TirNameStyle{WriteToRegFromVirtMem} and \TirNameStyle{WriteToVirtMemFromReg}
each use a virtual-pointsto assertion ($\textsf{vaddr} \mapsto_{\textsf{v}} \textsf{v}$),
and are nearly-standard (assembly) separation logic rules for memory accesses.
However, because we split the physical resources for the page table walk from the
virtual points-to itself (in order to permit the physical page tables to be freely modified
as long as they preserve virtual-to-physical translations), the rule requires $\mathcal{I}\textsf{ASpace}$
for the current address space to be carried through.
The soundness proofs for these rules extract
% the token ($\sumwalkabs\vaddr\qfrac\paddr$) from the virtual points-to
the token ($\fracghostmaptoken{\delta}{\vaddr}{\paddr}{\qfrac}$) from the virtual points-to,
use that to extract the physical page-table-traversal points-to collection describing
the page table walk for the relevant address ($\textsf{L}_{4}\_\textsf{L}_{1}\_\textsf{PointsTo}$)
from the invariant ($\mathcal{I}\mathsf{ASpace}$), prove that the page table walk succeeds and
memory or registers are updated appropriately, before re-packing the invariant and virtual points-to resources.

\paragraph{Updating \lstinline|cr3|} 
Unlike other rules, \TirNameStyle{WriteToRegCtlFromRegModal} updates the root address of the 
address-space determining the validity of resources, from $\rtv$ before the
\lstinline|mov| to $\textsf{rvs}$ afterwards. The global effects of this rule are reflected in moving
bare assertions under an other-space modality for $\rtv$, and moving the new address space's assertions out of
the corresponding modality.
% From that rule, with one step outside the \textsf{vProp} logic (so it is not quite a derived rule within
% our embedded logic), we can derive 
% \TirNameStyle{WriteToRegCtlFromRegModal},
% which is the (full) Hoare double version of the address space switch rule proposed in Section \ref{sec:issues}.

\subsection{Soundness}
Our rules from Figure \ref{fig:wpdamd} are proven sound in Iris against an assembly-level hardware model
implementing a fragment of x86-64 including 64-bit address translation with 4-level page tables.

%\todo[inline]{
Our rules for control transfers (\lstinline|jmp|, \lstinline|call|, and \lstinline|ret|) are currently
axiomatized (with completely standard specifications~\cite{ni2007contexts,Chlipala2013Bedrock}).% because Iris's built-in machinery does not provide
%convenient ways to discard the current continuation.
We are working to adapt existing machinery~\cite{de2023type} to prove their soundness.
Our soundness proofs for all other instructions (including the instructions above, arithmetic, other heap-accessing operations, push, and pop)
are axiom-free.
%}
% At the moment our proofs do rely on 13 small axioms of properties which should be provable, but
% are challenging to discharge due to some representation choices in our model.\footnote{See \lstinline|srx/x64/machine/current_axioms.v|}
