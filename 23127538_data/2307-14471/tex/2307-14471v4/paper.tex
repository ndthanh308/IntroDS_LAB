\pdfoutput=1
%%
%% This is file `sample-acmsmall-conf.tex',
%% generated with the docstrip utility.
%%
%% The original source files were:
%%
%% samples.dtx  (with options: `acmsmall-conf')
%% 
%% IMPORTANT NOTICE:
%% 
%% For the copyright see the source file.
%% 
%% Any modified versions of this file must be renamed
%% with new filenames distinct from sample-acmsmall-conf.tex.
%% 
%% For distribution of the original source see the terms
%% for copying and modification in the file samples.dtx.
%% 
%% This generated file may be distributed as long as the
%% original source files, as listed above, are part of the
%% same distribution. (The sources need not necessarily be
%% in the same archive or directory.)
%%
%%
%% Commands for TeXCount
%TC:macro \cite [option:text,text]
%TC:macro \citep [option:text,text]
%TC:macro \citet [option:text,text]
%TC:envir table 0 1
%TC:envir table* 0 1
%TC:envir tabular [ignore] word
%TC:envir displaymath 0 word
%TC:envir math 0 word
%TC:envir comment 0 0
%%
%%
%% The first command in your LaTeX source must be the \documentclass
%% command.
%%
%% For submission and review of your manuscript please change the
%% command to \documentclass[manuscript, screen, review]{acmart}.
%%
%% When submitting camera ready or to TAPS, please change the command
%% to \documentclass[sigconf]{acmart} or whichever template is required
%% for your publication.
%%
%%
\documentclass[acmsmall,screen]{acmart}
\newif\ifARXIV
\ARXIVfalse
\newif\ifShowRevisions
\ShowRevisionsfalse
\ifShowRevisions
\usepackage{soul}
\newcommand{\add}[1]{\textcolor{blue}{#1}}
% \newcommand{\del}[1]{\textcolor{red}{\st{#1}}}
\newcommand{\replace}[2]{\del{#1}\add{#2}}
\else
\newcommand{\add}[1]{{#1}}
% \newcommand{\del}[1]{}
\newcommand{\replace}[2]{\add{#2}}
\fi


\newif\ifPLDI
\PLDItrue

% \settopmatter{printacmref=false}
%%
%% \BibTeX command to typeset BibTeX logo in the docs
\AtBeginDocument{%
  \providecommand\BibTeX{{%
    Bib\TeX}}}

%% Rights management information.  This information is sent to you
%% when you complete the rights form.  These commands have SAMPLE
%% values in them; it is your responsibility as an author to replace
%% the commands and values with those provided to you when you
%% complete the rights form.
% \setcopyright{none}
% \setcopyright{rightsretained}
% \copyrightyear{2025}
% \acmYear{2025}
% \acmDOI{XXXXXXX.XXXXXXX}

% %% These commands are for a PROCEEDINGS abstract or paper.
% % \acmConference[Conference acronym 'XX]{Make sure to enter the correct
% %   conference title from your rights confirmation emai}{June 03--05,
% %   2018}{Woodstock, NY}
% \acmJournal{PACMPL}
% \acmVolume{9}
% \acmNumber{OOPSLA2}
% %%
% %%  Uncomment \acmBooktitle if the title of the proceedings is different
% %%  from ``Proceedings of ...''!
% %%
% %%\acmBooktitle{Woodstock '18: ACM Symposium on Neural Gaze Detection,
% %%  June 03--05, 2018, Woodstock, NY}
% \acmPrice{15.00}
% \acmISBN{978-1-4503-XXXX-X/18/06}

%\setcopyright{cc}
%\setcctype{by-nd}
%\acmJournal{PACMPL}
%\acmYear{2025} \acmVolume{9} \acmNumber{OOPSLA2} \acmArticle{356} \acmMonth{10} \acmPrice{}\acmDOI{10.1145/3763134}
%%% The following is specific to OOPSLA2 '25 and the paper
%%% 'Modal Abstractions for Virtualizing Memory Addresses'
%%% by Ismail Kuru and Colin S. Gordon.
%%%
\setcopyright{cc}
\setcctype{by-nd}
\acmDOI{10.1145/3763134}
\acmYear{2025}
\acmJournal{PACMPL}
\acmVolume{9}
\acmNumber{OOPSLA2}
\acmArticle{356}
\acmMonth{10}
\received{2025-03-25}
\received[accepted]{2025-08-12}


%%
%% Submission ID.
%% Use this when submitting an article to a sponsored event. You'll
%% receive a unique submission ID from the organizers
%% of the event, and this ID should be used as the parameter to this command.
%%\acmSubmissionID{123-A56-BU3}

%%
%% For managing citations, it is recommended to use bibliography
%% files in BibTeX format.
%%
%% You can then either use BibTeX with the ACM-Reference-Format style,
%% or BibLaTeX with the acmnumeric or acmauthoryear sytles, that include
%% support for advanced citation of software artefact from the
%% biblatex-software package, also separately available on CTAN.
%%
%% Look at the sample-*-biblatex.tex files for templates showcasing
%% the biblatex styles.
%%

%%
%% The majority of ACM publications use numbered citations and
%% references.  The command \citestyle{authoryear} switches to the
%% "author year" style.
%%
%% If you are preparing content for an event
%% sponsored by ACM SIGGRAPH, you must use the "author year" style of
%% citations and references.
%% Uncommenting
%% the next command will enable that style.
%%\citestyle{acmauthoryear}
%\usepackage{amsmath}
%\usepackage{amssymb}
%\usepackage{amsmath}
%\usepackage{amssymb}
\usepackage[skins,theorems]{tcolorbox}
\tcbset{highlight math style={enhanced jigsaw,
  colframe=red,colback=white,arc=0pt,boxrule=0.25pt,
  borderline={0.5mm}{0mm}{blue!70!white,dashed}}}
\usepackage{subfigure}
\usepackage{hyperref}
\usepackage{array}
%\usepackage{amsthm}
\usepackage{proof}
\usepackage{stmaryrd}
\usepackage{xspace}
%\usepackage{listings}
\usepackage{graphicx}

\usepackage{lipsum}
\usepackage[utf8]{inputenc}
%\usepackage{todonotes}
\usepackage{blindtext}
\usepackage{listings}
\lstset{
  columns=fullflexible,
  numbers=left,
  basicstyle=\ttfamily,
  keywordstyle=\color{blue}\bfseries,
  morekeywords={mov,add,call,and,or,xor,ret,push,pop,shl,shr,movabs,callq,jmp,jz,je,jlt,jge,jle,shiftr,shiftl,sub,cmp,jne},
  emph={rsp,rdx,rax,rbx,rbp,rsi,rdi,rcx,r8,r9,r10,r11,r12,r13,r14,r15,rflags,rip},
  emphstyle=\color{green},
  emph={[2]cr3},
  emphstyle={[2]\color{violet}},
  morecomment=[l]{;;},
  xleftmargin={2em},
}
% Hack for Colin to edit in low light: https://tex.stackexchange.com/questions/40495/invert-background-and-text-colours-across-whole-document-with-pdflatex
%Commands definitions
\newcommand{\setbackgroundcolour}{\pagecolor[rgb]{0.19,0.19,0.19}}  
\newcommand{\settextcolour}{\color[rgb]{0.77,0.77,0.77}}    
\newcommand{\invertbackgroundtext}{\setbackgroundcolour\settextcolour}
%Command execution. 
%If this line is commented, then the appearance remains as usual.
%\invertbackgroundtext
\usepackage{wrapfig}
\newtheorem*{remark}{Remark}

% Packages.
\usepackage{amsmath}

% If we need multiple references to a footnote, via \cref:
\usepackage{cleveref}
\crefformat{footnote}{#2\footnotemark[#1]#3}
\usepackage[T1]{fontenc}
\usepackage{fontawesome}
\usepackage[utf8]{inputenc}
\usepackage{iris}
\usepackage{mathpartir}
\renewcommand{\TirNameStyle}[1]{\hypertarget{#1}{\textsc{#1}}}
\newcommand{\RULE}[1]{\hyperlink{#1}{\textsc{#1}}\xspace}
\usepackage{xspace}
\newcommand\calF{\mathcal{F}}
\newcommand\calG{\mathcal{G}}
\newcommand\calM{\mathcal{M}}
\newcommand\calV{\mathcal{V}}
\newcommand\calU{\mathcal{U}}
\newcommand\calW{\mathcal{W}}
\newcommand\calP{\mathcal{P}}
\newcommand\calD{\mathbb{D}}
%%%%%%%%%%%%%%%%%
%% macros introduced by Luke 
\newcommand\mydef[1]{{\bf\em #1}}
%%%%%%%%%%%%%%%%%

\newcommand{\numviparams}{{| \lambda |}}
\newcommand{\scoreaccvars}[1]{s_1^{#1}, \ldots, s_{\numviparams}^{#1}}
\newcommand{\scoreaccvar}[2]{s_{#1}^{#2}}
\newcommand{\isdeterm}[1]{\text{Deterministic}({#1})}


\newcommand{\expect}[1]{\mathbb{E}\left[{#1}\right]}
\newcommand{\var}[1]{\mathbb{V}\left[ {#1} \right]}
\newcommand{\expectdist}[2]{\mathbb{E}_{#1}\left[ {#2} \right]}
\newcommand{\vardist}[2]{\mathbb{V}_{#1}\left[ {#2} \right]}
\newcommand{\cov}[2]{\mathbb{C}\text{ov}[{#1}][{#2}]}
\newcommand{\covv}[1]{\mathbb{C}\text{ov}[{#1}]}
\newcommand{\corr}[1]{\mathbb{C}\text{orr}[{#1}]}

\newcommand{\fix}[1]{\mathit{fix}\left({#1}\right)}
\newcommand{\sbr}[1]{\left\llbracket {#1} \right\rrbracket}
\newcommand{\ctxtype}[3]{{#1} \cong_\text{ctx} {#2} : {#3}}
\newcommand{\bigstep}[3]{{#1} \Downarrow_{#2} {#3}}


% PCF types
\newcommand{\bool}{\mathit{bool}}
\newcommand{\nat}{\mathit{nat}}

\newcommand{\ctx}[1]{\mathcal{C}\left[ {#1}\right] }
\newcommand{\pcft}[1]{\text{PCF}_{#1}}

\newcommand{\nfl}{\mathbb{N}_\bot}
\newcommand{\bfl}{\mathbb{B}_\bot}

% PCF constructs
\newcommand{\succc}[1]{\mathbf{succ}({#1})}
\newcommand{\succcn}[2]{\mathbf{succ}^{#1}({#2})}
\newcommand{\zero}{\mathbf{0}}
\newcommand{\zerotest}[1]{\mathbf{zero}\left({#1}\right)}
\newcommand{\pred}[1]{\mathbf{pred}\left( {#1} \right)}
\newcommand{\predn}[2]{\mathbf{pred}^{#1}\left( {#2} \right)}
\def\solvable{\#}

\newcommand{\true}{\mathbf{true}}
\newcommand{\false}{\mathbf{false}}
\newcommand{\pcffix}[1]{\mathbf{fix}\left({#1}\right)}
\newcommand{\pcffn}[3]{\mathbf{fn}~{#1}:{#2}\mathpunct{.}{#3}}
\newcommand{\pairtype}[2]{{#1} * {#2}}
\newcommand{\pairexp}[2]{\mathbf{pair}({#1}, {#2})}
\newcommand{\leftexp}[1]{\mathbf{left}({#1})}
\newcommand{\rightexp}[1]{\mathbf{right}({#1})}

\newcommand{\RationalPos}{\mathbb{Q}^{+}}

\newcommand{\meas}[1]{\mathbb{M}\left( {#1} \right) }
\newcommand{\integ}[1]{\sbr{#1}_I}

\newcommand{\notbigstep}[2]{{#1}~\cancel{\Downarrow}_{#2}}
\newcommand{\subtrace}[3]{{#1}^{{#2} \ldots {#3}}}
\newcommand{\supp}[1]{\textsf{supp}\left({#1}\right)}
\newcommand{\dom}[1]{\textsf{Dom}\left({#1}\right)}
\newcommand{\suppk}[2]{\textsf{Supp}^{#1}\left({#2}\right)}
\newcommand{\tracespace}{\bigcup_{n \in \mathbb{N}}[0, 1]^n}
\newcommand{\generictracespace}{\mathbb{T}}
\newcommand{\nnreals}{\mathbb{R}_{\geq 0}}
\newcommand{\posreals}{\mathbb{R}_{> 0}}
\newcommand{\reals}{\mathbb{R}}

\newcommand{\unrollkM}[2]{\textsf{unroll}_{#1}\left({#2}\right)}
\newcommand{\nphmcint}[5]{\Psi_\textsf{NP}\left({#1}, {#2}, {#3}, {#4}, {#5}\right)}

%SPCF constructs
\newcommand{\spcfvalues}{\Lambda^0_v}

\newcommand{\prevalueM}[1]{\textsf{value}^{-1}_{#1}(\spcfvalues{})}
\newcommand{\num}[1]{\underline{#1}}

% \theoremstyle{definition}
% \newtheorem{thm}{Theorem}
% \newtheorem{lem}{Lemma}
% \newtheorem{defn}{Definition}
% \newtheorem{conj}{Conjecture}
% \newtheorem{prop}{Proposition}

%\theoremstyle{definition}
%\newtheorem{defn}{Definition}[section]
%\newtheorem{example}[defn]{Example}
%
%
%\theoremstyle{plain}
%\newtheorem{thm}{Theorem}[section]
%\newtheorem{lem}[thm]{Lemma}
%\newtheorem{cor}[thm]{Corollary}
%\newtheorem{conj}[thm]{Conjecture}
%\newtheorem{prop}[thm]{Proposition}
%\newtheorem{remark}[thm]{Remark}

%% Proofs
%\let\oldproof\proof
%\renewcommand{\proof}{\color{blue}\oldproof}


\definecolor{codegreen}{rgb}{0,0.6,0}
\definecolor{codegray}{rgb}{0.5,0.5,0.5}
\definecolor{codepurple}{rgb}{0.58,0,0.82}
\definecolor{backcolour}{rgb}{0.95,0.95,0.92}

\lstdefinestyle{myStyle}{
    belowcaptionskip=1\baselineskip,
    breaklines=true,
    frame=none,
    basicstyle=\footnotesize\ttfamily,
    keywordstyle=\bfseries\color{green!40!black},
    commentstyle=\itshape\color{purple!40!black},
    identifierstyle=\color{blue},
    backgroundcolor=\color{gray!10!white},
    %backgroundcolor=\color{backcolour}, 
    numberstyle=\tiny\color{codegray},
    stringstyle=\color{codepurple},
    breakatwhitespace=false,                          
    keepspaces=true,                 
    numbers=left,       
    numbersep=5pt,                  
    showspaces=false,                
    showstringspaces=false,
    showtabs=false,                  
    tabsize=2,
}

% argmin/argmax
\DeclareMathOperator*{\argmax}{arg\,max}
\DeclareMathOperator*{\argmin}{arg\,min}

% Concatenation of lists
\newcommand\doubleplus{+\kern-1.3ex+\kern0.8ex}

% Program configurations
\newcommand{\tuple}[1]{\ensuremath{\langle #1 \rangle}}
% Rule based definitions
\newcommand{\Rule}[4][]{\ensuremath{\inferrule*[lab={\hypertarget{#2}{(\TirName{#2})}},#1]{#3}{#4}}}

% Calligraphic symbols
\newcommand{\calI}{{\mathcal I}} 
\newcommand{\calT}{{\mathcal T}}

%  Macro for new Y operator.
\newcommand{\yBounded}[3]{\mu^{#1}_{#2}\rvert_{#3}}

%%%%%%%%%%%%%%%%%
 
%%%%%%%%%%%%%%%%%

\newcommand{\expv}{\mathbb{E}}

\newcommand{\combTr}[2]{\left[\begin{matrix}
		#1\\
		#2
	\end{matrix} \right]}

\newcommand{\exType}[2]{\left\{\begin{matrix}
		#1\\
		#2
	\end{matrix} \right\}}
\newcommand{\myint}[1]{ [#1]}
\newcommand{\Uniform}{\ensuremath{\mathrm{Uniform}}}
\newcommand{\Normal}{\ensuremath{\mathrm{normal}}}
\DeclareMathOperator{\abs}{abs}
\DeclareMathOperator{\pdf}{pdf}

\newcommand{\intConf}[1]{\lceil#1\rceil}
\newcommand{\tr}{\boldsymbol{t}}

\newcommand{\sample}{\tt{sample}}
%\newcommand{\fix}{\texttt{fix}}
%\newcommand{\num}[1]{\underline{#1}}
\newcommand{\myif}{\texttt{if}}
\newcommand{\mylet}{\texttt{let} \, }
\newcommand{\myin}{\, \texttt{in} \,}
\newcommand{\mythen}{\, \texttt{then} \,}
\newcommand{\myelse}{\, \texttt{else} \,}
\newcommand{\score}{\tt{score}}
\newcommand{\tick}{\tt{tick}}

\newcommand{\term}{\tt{term}}
\newcommand{\pv}{\mathbf{v}}
\newcommand{\rv}{\mathbf{r}}

\newcommand{\interval}{\mathfrak{I}}

\newcommand{\typeReal}{\textbf{\textsf{R}}}

\newcommand{\symbolInt}{\myint{\cdot}}

\newcommand{\LambdaInterval}{\Lambda_{\interval}}
\newcommand{\LambdaSymbolic}{\Lambda_{\text{sym}}}

\newcommand{\toIntervalTerm}[1]{#1^{2\interval}}

%Others
\newcommand{\Sset}{\mathbb{S}}
\newcommand{\Iset}{\mathbb{I}}
\newcommand{\Rset}{\mathbb{R}}
\newcommand{\Nset}{\mathbb{N}}
\newcommand{\Zset}{\mathbb{Z}}

\newcommand{\Term}{\mathbb{T}}
\newcommand{\prob}{\mathbb{P}}
\newcommand{\expt}{\mathbb{E}}


\newcommand{\Leb}{\tt{Leb}}
\newcommand{\Red}{\tt{Red}}
\newcommand{\cost}{\text{cost}}

%\newcommand{\intervalab}[2]{\underline{[#1,#2]}}
\newcommand{\intervalab}{\underline{[a,b]}}
\newcommand{\interI}{\mathcal{I}}
\newcommand{\trans}{\mathcal{T}}

\newcommand{\iv}{\mathbb{I}}

% Programming language constructs
\newcommand{\lit}[1]{\underline{#1}}
\newcommand{\letIn}[1]{\mathsf{let}\,{#1}\,\mathsf{in}\,}
\newcommand{\fixLam}[2]{\mu {#1} {#2}.}
\newcommand{\ifElse}[3]{\mathsf{if} (#1 \le \num{0}) \, {#2} \,\mathsf{else}\, {#3}}

%%Basic notions
\newcommand{\pspace}{(\Omega,\mathcal{F},\probm)}
\newcommand{\probm}{\mathbb{P}}
\newcommand{\condexpv}[2]{{\expt}{\left[{#1} \mid {#2}\right]}}

\newcommand{\stdConf}[1]{(#1)}
%\newcommand{\intConf}[1]{\lceil#1\rceil}
%\newcommand{\intConf}[1]{(#1)}
%\newcommand{\symConf}[1]{\langle\!\langle  #1 \rangle\!\rangle}
%\newcommand\symPath[1]{(#1)}
\newcommand{\symPath}[1]{\langle\!\langle  #1 \rangle\!\rangle}
\newcommand\symConf[1]{(#1)}

\newcommand{\ifSimple}[3]{\mathsf{if}(#1, #2, #3)}
%\newcommand{\ifElse}[3]{\mathsf{if} (#1 \le 0) \, \allowbreak {#2} \, \allowbreak \mathsf{else}\, {#3}}
%\newcommand{\ifElse}[3]{\ifSimple{#1}{#2}{#3}}

%\newcommand{\trace}{\mathsf{s}}
%
%\newcommand\defn[1]{{\bf \em #1}}
\newcommand{\traces}{\mathbb{T}}
%
%\newcommand{\stdConf}[1]{(#1)}
%%\newcommand{\intConf}[1]{\lceil#1\rceil}
%\newcommand{\intConf}[1]{(#1)}
%%\newcommand{\symConf}[1]{\langle\!\langle  #1 \rangle\!\rangle}
%%\newcommand\symPath[1]{(#1)}
%\newcommand{\symPath}[1]{\langle\!\langle  #1 \rangle\!\rangle}
%\newcommand\symConf[1]{(#1)}

\newcommand{\valueSem}[1]{\mathsf{val}_{#1}} % value (semantics)
\newcommand{\weightSem}[1]{\mathsf{wt}_{#1}} % weight (semantics)
\newcommand{\measureSem}[1]{\llbracket #1 \rrbracket}
\newcommand{\posterior}{\mathsf{posterior}}


%%%%%%%%%
% 
%%%%%%%%
\newcommand{\loc}{\ell}
\newcommand{\locs}{\mathit{L}}
\newcommand{\blocs}{\mathit{L}_{\mathrm{b}}}

\newcommand{\iflocs}{\mathit{L}_{\mathrm{if}}}
\newcommand{\looplocs}{\mathit{L}_{\mathrm{while}}}

\newcommand{\alocs}{\mathit{L}_{\mathrm{a}}}
\newcommand{\wlocs}{\mathit{L}_{\mathrm{w}}}
\newcommand{\rlocs}{\mathit{L}_{\mathrm{r}}}
\newcommand{\Alocs}[1]{\mathit{L}_{\mathrm{A}}^{\mathsf{#1}}}
\newcommand{\Dlocs}{\mathit{L}_{\mathrm{nd}}}
\newcommand{\transitions}{{\rightarrow}}

%%% 
\newcommand{\plocs}{\mathit{L}_{\mathrm{p}}}
\newcommand{\tlocs}{\mathit{L}_{\mathrm{t}}}

\newcommand{\lin}{\loc_\mathrm{init}}
\newcommand{\lout}{\loc_\mathrm{out}}
\newcommand{\val}[1]{\mbox{\sl Val}_{#1}}

\newcommand{\pvars}{V_\mathrm{p}}
\newcommand{\rvars}{V_{\mathrm{r}}}
\newcommand{\pre}{\mathrm{pre}}

\newcommand{\sle}{\sqsubseteq}
\newcommand{\sge}{\sqsupseteq}

\newcommand{\lfp}{\mathrm{lfp}}
\newcommand{\gfp}{\mathrm{gfp}}

\newcommand{\rdvarjdis}{\mathcal D}
\newcommand{\sampset}{\textit{supp}}

\newcommand{\upd}{\mbox{\sl upd}}
\newcommand{\wet}{\mbox{\sl wt}}
\newcommand{\transset}{\mathfrak T}
\newcommand{\valin}{\pv_{\mathrm{init}}}
\newcommand{\ret}{\mbox{\sl ret}}

\newcommand{\win}{w_{\mathrm{init}}}

\newcommand{\sampdpd}{\overline{\Upsilon}}

\newcommand{\outmap}{\text{O}}
\newcommand{\sat}[1]{\langle #1 \rangle}
\newcommand{\monoid}{\mbox{\sl Monoid}}
\newcommand{\handelmanformat}{(\dagger)}

\newcommand{\trunc}{\mathcal{B}}

\newcommand{\ewt}{\mbox{\sl ewt}}
\newcommand{\statemap}{\text{St}}

\newcommand{\valrd}{{\mathbf{r}}}
\newcommand{\frmloc}{\ell^{\mathrm{src}}}
\newcommand{\toloc}{\ell^{\mathrm{dst}}}

\newcommand{\monomials}{\mathbf{M}}
\newcommand{\mytodo}[1]{% <==========================================
  \todo[linecolor=white, bordercolor=white, textcolor=white]{#1}%
}



%%
%% Submission ID.
%% Use this when submitting an article to a sponsored event. You'll
%% receive a unique submission ID from the organizers
%% of the event, and this ID should be used as the parameter to this command.
%%\acmSubmissionID{123-A56-BU3}

%%
%% For managing citations, it is recommended to use bibliography
%% files in BibTeX format.
%%
%% You can then either use BibTeX with the ACM-Reference-Format style,
%% or BibLaTeX with the acmnumeric or acmauthoryear sytles, that include
%% support for advanced citation of software artefact from the
%% biblatex-software package, also separately available on CTAN.
%%
%% Look at the sample-*-biblatex.tex files for templates showcasing
%% the biblatex styles.
%%

%%
%% The majority of ACM publications use numbered citations and
%% references.  The command \citestyle{authoryear} switches to the
%% "author year" style.
%%
%% If you are preparing content for an event
%% sponsored by ACM SIGGRAPH, you must use the "author year" style of
%% citations and references.
%% Uncommenting
%% the next command will enable that style.
% \citestyle{acmauthoryear}



%%
%% end of the preamble, start of the body of the document source.
\begin{document}

%%
%% The "title" command has an optional parameter,
%% allowing the author to define a "short title" to be used in page headers.
\title{Modal Abstractions for Virtualizing Memory Addresses}
%%
%% The "author" command and its associated commands are used to define
%% the authors and their affiliations.
%% Of note is the shared affiliation of the first two authors, and the
%% "authornote" and "authornotemark" commands
%% used to denote shared contribution to the research.
\author{Ismail Kuru}
\email{ik335@drexel.edu}
\author{Colin S. Gordon}
\email{csgordon@drexel.edu}
\affiliation{%
  \institution{Drexel University}
  \city{Philadelphia, PA}
  \country{USA}
}


%%
%% By default, the full list of authors will be used in the page
%% headers. Often, this list is too long, and will overlap
%% other information printed in the page headers. This command allows
%% the author to define a more concise list
%% of authors' names for this purpose.
\renewcommand{\shortauthors}{Kuru and Gordon}

%%
%% The abstract is a short summary of the work to be presented in the
%% article.
\begin{abstract}
\ifPLDI
\else
Operating system kernels employ virtual memory subsystems, which use a CPU's memory management units (MMUs) to virtualize the addresses of memory regions:
a logical (virtual) address is translated to a physical address in memory by the MMU based on
kernel-controlled page tables -- a hardware-defined sparse tree-map structure -- stored in memory, which itself is 
accessed by the kernel through virtual addresses.
Operating systems manipulate these virtualized memory mappings to isolate untrusted processes,
 restrict which memory is accessible to different processes, 
hide memory limits from user programs, 
ensure process isolation, implement demand-paging and copy-on-write behaviors for performance
and resource controls.
At the same time, misuse of MMU hardware can lead to kernel crashes.
\fi

Virtual memory management (VMM) code is a critical piece of general-purpose OS kernels, but verification of this functionality
is challenging due to the complexity of the hardware interface (the page tables are updated via writes to those
memory locations, using addresses which are themselves virtualized).
Prior work on verification of VMM code has either only handled a single address space, or trusted significant
pieces of assembly code.

In this paper, we introduce a modal abstraction to describe
the truth of assertions relative to a specific virtual address space: [r]P indicating that P holds in the
virtual address space rooted at r. Such modal assertions 
allow different address spaces to refer to each other, enabling complete verification of instruction sequences
manipulating multiple address spaces. Using them effectively requires working with other assertions,
% such as points-to assertions in our separation logic,  relative to a given address space.
such as points-to assertions about memory contents --- which implicitly depend on the address space
they are used in. 
We therefore define virtual points-to assertions to definitionally mimic hardware address translation,
relative to a page table root.
We demonstrate our approach with challenging fragments of VMM code showing that our approach
handles examples beyond what prior work can address, including reasoning about
a sequence of instructions as it changes address spaces.
\ifPLDI
Our results are formalized for a RISC-like fragment of x86-64 assembly in Rocq.
\looseness=-1
\else
All definitions and theorems mentioned in this paper including the operational model of a RISC-like fragment of x86-64, 
a simple language run on this operational model, and a logic as an instantiation of the Iris framework are mechanized 
inside Rocq.
\fi
\end{abstract}

%%
%% The code below is generated by the tool at http://dl.acm.org/ccs.cfm.
%% Please copy and paste the code instead of the example below.
%%

\begin{CCSXML}
<ccs2012>
   <concept>
       <concept_id>10011007.10010940.10010992.10010998.10010999</concept_id>
       <concept_desc>Software and its engineering~Software verification</concept_desc>
       <concept_significance>500</concept_significance>
       </concept>
   <concept>
       <concept_id>10011007.10010940.10010941.10010949.10010950.10010951</concept_id>
       <concept_desc>Software and its engineering~Virtual memory</concept_desc>
       <concept_significance>500</concept_significance>
       </concept>
 </ccs2012>
\end{CCSXML}

\ccsdesc[500]{Software and its engineering~Software verification}
% \ccsdesc[500]{Theory of computation~Modal and temporal logics}
\ccsdesc[500]{Software and its engineering~Virtual memory}

% \begin{CCSXML}
% <ccs2012>
%  <concept>
%   <concept_id>10010520.10010553.10010562</concept_id>
%   <concept_desc>Computer systems organization~Embedded systems</concept_desc>
%   <concept_significance>500</concept_significance>
%  </concept>
%  <concept>
%   <concept_id>10010520.10010575.10010755</concept_id>
%   <concept_desc>Computer systems organization~Redundancy</concept_desc>
%   <concept_significance>300</concept_significance>
%  </concept>
%  <concept>
%   <concept_id>10010520.10010553.10010554</concept_id>
%   <concept_desc>Computer systems organization~Robotics</concept_desc>
%   <concept_significance>100</concept_significance>
%  </concept>
%  <concept>
%   <concept_id>10003033.10003083.10003095</concept_id>
%   <concept_desc>Networks~Network reliability</concept_desc>
%   <concept_significance>100</concept_significance>
%  </concept>
% </ccs2012>
% \end{CCSXML}

% \ccsdesc[500]{Computer systems organization~Embedded systems}
% \ccsdesc[300]{Computer systems organization~Redundancy}
% \ccsdesc{Computer systems organization~Robotics}
% \ccsdesc[100]{Networks~Network reliability}

%%
%% Keywords. The author(s) should pick words that accurately describe
%% the work being presented. Separate the keywords with commas.
\keywords{program verification, virtual memory, modal logic}

\maketitle

{
\theoremstyle{acmdefinition}
\newtheorem{assumption}[theorem]{Assumption}
}

% Figure environment removed

\section{Introduction}
Automatic 3D reconstruction of clothed humans using image inputs has gained increasing significance due to its potential applications in a wide array of AR/VR scenarios. High-fidelity reconstructions typically depend on sophisticated capture systems, which are developed with dense camera arrays~\cite{collet2015high,joo2015panoptic,joo2018total}, programmable light-stages~\cite{Vlasic2009, guo2019relightables}, and depth sensors~\cite{newcombe2011kinectfusion,DoubleFusion,BodyFusion,dou2016fusion4d,newcombe2015dynamicfusion}. However, stringent capture environments equipped with complex hardware pose significant challenges for consumer-level applications.


In this context, considerable research effort has been dedicated to developing methods that allow for more flexible capture configurations, such as utilizing a few RGB inputs. Among these works, learning implicit functions \cite{iccv2020PIFu, saito2020pifuhd, hong2021stereopifu} has proven effective in achieving highly detailed reconstructions by integrating the advancements of deep neural networks. These methods employ large multi-layer perceptrons (MLPs) to predict the occupancy probability or truncated signed distance function (TSDF) value of every queried 3D point based on its associated local feature, which is extracted from images. They can recover a continuous surface at arbitrary resolutions without topology restrictions.


However, in typical MLP-based implicit networks, the occupancy or TSDF value at each location is solved independently with planar image features, rendering them less capable of addressing challenging cases such as occlusions. Consequently, these methods suffer from generalization and robustness issues, particularly when tackling strong occlusions caused by large motion or multiple interacting humans. 
Some follow-up studies  \cite{zheng2021deepmulticap,zheng2021pamir,huang2020arch} utilize an extra geometric model, SMPL~\cite{Loper2015}, to improve robustness by introducing strong shape priors. 
Their success typically relies on the assumption of geometrical similarity \cite{huang2020arch} between the shape prior and target reconstruction, making them intractable for handling complex cases with loose clothes and sensitive to errors in SMPL model fitting.



%\ping{this paragraph sounds like `TSDF is better than MLP/SMPL, and we use TSDF to solve the problem'. But in Sec 3, we are telling a different story, saying `MLP needs a 3D convolutional encoder'. We need to make these two sections consistent.}\sicong{I think in this paragraph we claim that the TSDF}


%We opt for Trucated Signed Distance Funtion (TSDF) volumetric representations as they are naturally suitable for convolution operations, which have shown remarkable performance for learning hierarchical features on 2D visual perception tasks \cite{SunXLW19}. 
%Meanwhile, TSDF also describes the gradual geometry change around shape surface, which is not reflected by occupancy volume. 

We instead revisit the 3D volumetric representation and resort to 3D convolutional neural networks (CNNs) for feature learning, due to their impressive performance in feature learning and the ability to incorporate spatial context. However, volumetric methods and 3D convolution involve discretization, which might raise concerns regarding whether a discretized volume can preserve subtle geometric details as continuous representations learned in implicit functions. We investigate the relationship between volume resolution and quantization error on synthetic data by converting target mesh objects to TSDF volumes, as shown in Figure~\ref{fig:quantization_error}. We observe that the quantization errors are significantly reduced by increasing volume resolution and become nearly negligible when reaching a relatively high resolution (e.g., 512 or higher). In other words, achieving fine-detailed reconstruction is not supposed to be restricted by the use of volume representations as long as a proper volume resolution is utilized. Therefore, we present a method with high-resolution feature volumes, e.g., 256 and 512, while traditional volumetric methods \cite{varol18_bodynet,gilbert2018volumetric} are often limited to much lower resolutions, such as 32 or 128.



On the other hand, an increase in volume resolution may lead to a cubic growth of memory overhead \cite{8100085}. Reducing memory costs while guaranteeing the granularity of volumetric representations is necessary for pursuing high-quality reconstruction. Thus, we adopt a coarse-to-fine approach and cull away irrelevant voxels to build a sparse high-resolution feature volume. At the coarse level, the network computes an initial TSDF by applying a U-Net with sparse 3D CNN \cite{3DSemanticSegmentationWithSubmanifoldSparseConvNet} on the sparse feature volume, which is carved by a visual hull. Through our experiments, it turns out that more than 95\% of the volume grids are discarded by the visual hull culling, making the sparse 3D CNN efficient. At the fine level, the network focuses on a narrow band near the zero-level set of the initial TSDF and discretizes the narrow band with smaller voxels. By employing this narrow-band culling, we further shrink the sampling space, resulting in a relatively small range of grid numbers (usually 300K--500K in our experiments) even with a high volume resolution of 512. The remaining voxels in the narrow band are associated with features that fuse high-frequency information from the computed normal maps upon the low-frequency shape from the coarse level to compute the TSDF at high resolution. The final mesh is then extracted from the TSDF using the Marching-Cube algorithm ~\cite{Lorensen87marchingcubes}.
% Different from the u-net sturcture to preserve global topology context, we then apply a shallow 3dcnn to compute the final TSDF $D_{final}$ which contain more local geometry detail.




% \ping{this paragraph can be expanded. It is an important contribution and often ignored by other works. stress on the novel idea of regressing blending weights instead of colors}

In addition to geometry, high-quality mesh texture is also a crucial factor contributing to visual appearance. Directly computing a color field in 3D space, as in \cite{iccv2020PIFu}, struggles to capture high-frequency texture details, while the neural radiance field (NeRF) \cite{yu2020pixelnerf} or the DoubleField~\cite{shao2022doublefield} require expensive per-instance optimization and are often unstable for sparse input images. In contrast, we adopt an image-based rendering approach to compute a texture atlas map, which is efficient and widely supported in existing computer graphics tools. 
Specifically, we compute a blending weight at each 3D point on the mesh surface to determine its color as a weighted average of the colors at its image projections. The blending weights can be computed at a relatively coarse resolution, e.g., 512 volume resolution in our case, and leave texture details to the high-resolution images, such as 1K or 2K. Unlike previous methods that generate blurry texturing results under sparse input, our method generalizes well on both synthetic and real data with just a few input views. 
Figure~\ref{fig:teaser} shows two examples reconstructed by our method. Despite the challenging garment, pose, and occlusion, our method recovers faithful shape, normal, and texture on the right.

%with a wide variety of poses and clothing styles, and it is also adaptive to handle input image with arbitrary resolutions.
%\sicong{For this concern we claim that when the resolution of dicretized volume meets certain threshold (which is 256 in our experiment), the quantization error can be neglected.} 



In summary, the main contributions of this paper are as follows:
\begin{itemize}
\vspace{-0.1in}
  \item 
  We revisit the 3D volumetric representation and demonstrate that it can support clothed human reconstruction with equal or even better performance compared to implicit representation. 
  \item 
  We develop a memory and computation-efficient method for high-resolution volumetric reconstruction using sophisticated sparse 3D CNN, coarse-to-fine estimation, and voxel culling by visual hull and narrow bands. 
  \item 
  We introduce a novel method to compute a texture atlas map, which captures rich appearance details from high-resolution input images.
  \item 
  We achieve impressive results on standard benchmark datasets Twindom and MultiHuman, significantly reducing the point-2-surface (P2S) precision to approximately 0.2cm from just six input views, with more than $50\%$ error reduction compared to the state-of-the-art methods, including DoubleField~\cite{shao2022doublefield} and PIFuHD~\cite{saito2020pifuhd}.
\end{itemize}
%\section{Motivation}
\label{sec:motivation}

IGNORE THIS FILE, WILL DO IN INTRO

\vspacebeforesection
\section{Background}
\label{sec:background}

In this section, we provide the necessary background information to ensure a comprehensive understanding of the attack described in this paper. We start with a description of the Distributed Hash Table (DHT) used by IPFS, followed by its content resolution mechanisms. We also detail techniques for network size estimation, necessary for our attack detection and mitigation mechanisms.

\vspacebeforesection
\subsection{IPFS DHT}
\label{sec:kad_dht}

We review the features of the Kademlia DHT~\cite{maymounkov2002kademlia} and its \texttt{libp2p} implementation~\cite{libp2p_github} that are the most relevant to our attack.
To participate in the DHT, each peer generates a public/private key pair and derives an identity $\peerid \in \{0,1\}^{256}$ as the hash of its public key.
Ideally, each peer generates a random key pair and, therefore, peer IDs are distributed uniformly and independently over the space $\{0,1\}^{256}$.
While honest nodes follow this rule, malicious nodes may generate and choose from an arbitrary number of key pairs.
Each peer maintains a routing table consisting of $m=256$ buckets.
The $i$-th bucket contains the addresses of up to $k=20$ peers whose peer IDs share a common prefix of exactly $i$ bits with the peer's own peer ID. 

%
A new participant node joins the IPFS network by contacting one of the hardcoded bootstrap nodes. This bootstrap node provides the new node with some initial peers allowing it to join the DHT. The new node uses this information to perform a walk through the DHT towards its own peer ID.
The walk allows to: \textit{(i)}~make sure that there is no other node in the network with the same ID; \textit{(ii)}~discover new peers and fill the newcomer's DHT routing table. At the same time, the newcomer establishes \bitswap~\cite{de2021accelerating} connections to a subset of encountered peers (usually around 300 of them). The core role of the \bitswap protocol is to enable bilateral content transfer and to play the role of a cache for recently-accessed content.

The main DHT operation $\Call{GetClosestPeers}{\key}$ returns the $k=20$ closest peers to $\key$. 
%
In Kademlia, the distance between two keys $x$ and $y$ in the key space is given by $x \oplus y \in \{0,...,2^{256}-1\}$, where $\oplus$ denotes the bitwise XOR operation on the keys; the resulting binary string is interpreted as an integer.
%
When a client wants to find the peers with IDs closest to $\key$, it sends a request to the $\alpha=3$ peers in its routing table whose peer IDs are closest to $\key$. Each of these peers returns the $k$ closest peers to $\key$ in its own routing table and the addresses of these peers. 
%
The client again sends a request to the $\alpha$ peers closest to $\key$, among peers in its routing table and those whose addresses it just received. This process repeats until the client does not find any more peers closer to $\key$.
Due to network churn and imperfect routing tables, we observed in our experiments that successive calls to $\Call{GetClosestPeers}{\key}$ do not always return the same set of $k=20$ peers (we provide more details in \Cref{sec:evaluation}, \Cref{fig:20closest}). This is an important limitation affecting our attack.

\vspacebeforesection
\subsection{Content Resolution in IPFS}
\label{sec:ipfs}

IPFS is a content-centric network.
It allows its participant to request files without specifying their location. 
%
Content is indexed by content IDs $\cid \in \{0,1\}^{256}$ that are derived from a hash of that content.
Both peer IDs and CIDs are used as keys in the DHT.
Each node can play the role of a \provider, \downloader, or \resolver. 
The process of content advertisement and resolution is illustrated in \Cref{fig:add_get_provider}.

%
When a \provider wishes to publish content with a given $\cid$ on IPFS, it creates a \emph{provider record} that contains $cid$ and the \provider's address.
During a $\Call{Provide}{\cid}$ operation, the \provider first uses $\Call{GetClosestPeers}{\cid}$ to locate the $k=20$ peers with their peer IDs closest to $\cid$, 
%
and then sends them a $\mathsf{PutProvider}$ message including the provider record (\Cref{fig:add_get_provider}(a)).
We call the peers that hold provider records for $\cid$ the \emph{resolvers} for $\cid$.

Each CID can have several \providers. In fact, by default, each IPFS client becomes a provider for each piece of content it downloads for a fixed amount of time (12h, 24h, or 48h depending on the client version or custom configuration). As a result, the system provides an auto-scaling feature with supply automatically rising with demand.

%
When a \downloader wishes to fetch a piece of content, it first sends a request to all its \bitswap peers. If none of them has the content, the \downloader uses the DHT-based resolution system. We stress that the \bitswap protocol plays the supporting role of a cache in the dissemination of popular files. However, the mechanism does not provide reliable content resolution, in particular for new or less popular content. %

When \bitswap unstructured search fails, the \downloader resolves $\cid$ using $\Call{FindProviders}{\cid}$. This operation uses a DHT walk identical to that of $\Call{GetClosestPeers}{\cid}$ to find $k$ \resolvers but also queries encountered nodes for a provider record for $\cid$ (\Cref{fig:add_get_provider}(b)). The process terminates when either 20 \providers have been found, or all \resolvers have been asked. Querying all encountered nodes (\ie, not only the designated \resolvers) is useful because some of the encountered nodes may have a provider record in their cache.
%

Upon receiving a provider record, the client connects to the address specified in the provider record to retrieve the actual content (\Cref{fig:add_get_provider}(c)).
Provider records are not authenticated, and therefore malicious \providers may respond with incorrect provider records (or may not respond at all). However, the integrity of the content is preserved because the hash of the retrieved content can be verified against its $\cid$.
%


%

\input{img/add_get_provider.tex}

\vspacebeforesection
\subsection{Network Size Estimator}
\label{sec:netsize}

The number of nodes in a decentralized system is generally unknown due to the avoidance of centralized membership management.
This number is nonetheless useful for optimizations, deciding on individual node configurations, or security mechanisms.
Various methods were proposed for the decentralized estimation of unstructured and structured networks~\cite{eli-sohl-dht-size-estimation,kostoulas2005decentralized, manku2003symphony}.
We use in this work a mechanism developed initially by Protocol Labs as part of a mechanism for decreasing the latency of publishing content in IPFS~\cite{network-size-estimation-notion,network-size-estimation-github-pr}.

%
%
%
%
%
%
%
%
%
%

Each node in the DHT refreshes its routing table periodically (every $10$ minutes in \texttt{libp2p}). 
For this, the node samples $m$ random keys (one for each bucket of its routing table)
%
and queries the DHT to obtain the $k=20$ closest peer IDs to each key.
Using these, the node then computes the average distance between each one of these keys $\key_j$ for $j=1,\dots,m$ and their $i$-th closest peer ID for $i=1,...,k$ (with $m=256$ and $k=20$).
\begin{equation}
    \label{equ:avg-dist}
    \overline{D}_i = \frac{1}{m} \sum_{j=1}^m \operatorname{dist}(\key_j, \peerid_{j}^{(i)})
\end{equation}
where $\peerid_{j}^{(i)}$ is the $i$-th closest peer ID to $\key_j$.
With $N$ peers in the DHT and peer IDs uniformly distributed in the hash space, the expected distance between a $\key$ and its $i$-th closest peer ID is $\frac{2^{256}i}{N+1}$. The node then runs a least square regression to compute the value of $N$ for which the expected distances best fit the empirical average distances, \ie,
\begin{equation}
    \label{equ:netsize-least-squares}
    \hat{N} = \arg\min_{N} \sum_{i=1}^k \left(\overline{D}_i - \frac{2^{256}i}{N+1}\right)^2.
\end{equation}
The resulting estimate $\hat{N}$ can be computed in closed form.
%

When a node starts running, it must perform DHT queries for a few random keys to initialize its network size estimate. 
Since a larger number of queries will result in higher accuracy, making more queries than what is needed to initialize one's routing table is recommended.
Thereafter, keeping the estimate up-to-date does not require any excess DHT queries beyond what is already used for refreshing the routing table as this is done frequently (every 10 minutes).

While the network size estimate has a stochastic variance resulting from the probability distribution of the honest peer IDs, it is hard for an attacker to bias the estimate significantly. Since the estimator uses the density of peer IDs around keys chosen uniformly at random, the adversary would require numerous Sybil nodes (on the order of the whole network size) to significantly affect the peer ID density around those keys.

\definecolor{dkgreen}{rgb}{0,0.6,0}
\definecolor{ltblue}{rgb}{0,0.4,0.4}
\definecolor{dkviolet}{rgb}{0.3,0,0.5}

% lstlisting coq style (inspired from a file of Assia Mahboubi)
\lstdefinelanguage{Coq}{ 
    % Anything betweeen $ becomes LaTeX math mode
    mathescape=true,
    % Comments may or not include Latex commands
    texcl=false, 
    % Vernacular commands
    morekeywords=[1]{Section, Module, End, Require, Import, Export,
        Variable, Variables, Parameter, Parameters, Axiom, Hypothesis,
        Hypotheses, Notation, Local, Tactic, Reserved, Scope, Open, Close,
        Bind, Delimit, Definition, Let, Ltac, Fixpoint, CoFixpoint, Add,
        Morphism, Relation, Implicit, Arguments, Unset, Contextual,
        Strict, Prenex, Implicits, Inductive, CoInductive, Record,
        Structure, Canonical, Coercion, Context, Class, Global, Instance,
        Program, Infix, Theorem, Lemma, Corollary, Proposition, Fact,
        Remark, Example, Proof, Goal, Save, Qed, Defined, Hint, Resolve,
        Rewrite, View, Search, Show, Print, Printing, All, Eval, Check,
        Projections, inside, outside, Def},
    % Gallina
    morekeywords=[2]{forall, exists, exists2, fun, fix, cofix, struct,
        match, with, end, as, in, return, let, if, is, then, else, for, of,
        nosimpl, when},
    % Sorts
    morekeywords=[3]{Type, Prop, Set, true, false, option},
    % Various tactics, some are std Coq subsumed by ssr, for the manual purpose
    morekeywords=[4]{pose, set, move, case, elim, apply, clear, hnf,
        intro, intros, generalize, rename, pattern, after, destruct,
        induction, using, refine, inversion, injection, rewrite, congr,
        unlock, compute, ring, field, fourier, replace, fold, unfold,
        change, cutrewrite, simpl, have, suff, wlog, suffices, without,
        loss, nat_norm, assert, cut, trivial, revert, bool_congr, nat_congr,
        symmetry, transitivity, auto, split, left, right, autorewrite},
    % Terminators
    morekeywords=[5]{by, done, exact, reflexivity, tauto, romega, omega,
        assumption, solve, contradiction, discriminate},
    % Control
    morekeywords=[6]{do, last, first, try, idtac, repeat},
    % Comments delimiters, we do turn this off for the manual
    morecomment=[s]{(*}{*)},
    % Spaces are not displayed as a special character
    showstringspaces=false,
    % String delimiters
    morestring=[b]",
    morestring=[d],
    % Size of tabulations
    tabsize=3,
    % Enables ASCII chars 128 to 255
    extendedchars=false,
    % Case sensitivity
    sensitive=true,
    % Automatic breaking of long lines
    breaklines=false,
    % Default style fors listings
    basicstyle=\small,
    % Position of captions is bottom
    captionpos=b,
    % flexible columns
    columns=[l]flexible,
    % Style for (listings') identifiers
    identifierstyle={\ttfamily\color{black}},
    % Style for declaration keywords
    keywordstyle=[1]{\ttfamily\color{dkviolet}},
    % Style for gallina keywords
    keywordstyle=[2]{\ttfamily\color{dkgreen}},
    % Style for sorts keywords
    keywordstyle=[3]{\ttfamily\color{ltblue}},
    % Style for tactics keywords
    keywordstyle=[4]{\ttfamily\color{dkblue}},
    % Style for terminators keywords
    keywordstyle=[5]{\ttfamily\color{dkred}},
    %Style for iterators
    %keywordstyle=[6]{\ttfamily\color{dkpink}},
    % Style for strings
    stringstyle=\ttfamily,
    % Style for comments
    commentstyle={\ttfamily\color{dkgreen}},
    %moredelim=**[is][\ttfamily\color{red}]{/&}{&/},
    literate=
    {\\forall}{{\color{dkgreen}{$\forall\;$}}}1
    {\\exists}{{$\exists\;$}}1
    {<-}{{$\leftarrow\;$}}1
    {=>}{{$\Rightarrow\;$}}1
    {==}{{\code{==}\;}}1
    {==>}{{\code{==>}\;}}1
    %    {:>}{{\code{:>}\;}}1
    {->}{{$\rightarrow\;$}}1
    {<->}{{$\leftrightarrow\;$}}1
    {<==}{{$\leq\;$}}1
    {\#}{{$^\star$}}1 
    {\\o}{{$\circ\;$}}1 
    {\@}{{$\cdot$}}1 
    {\/\\}{{$\wedge\;$}}1
    {\\\/}{{$\vee\;$}}1
    {++}{{\code{++}}}1
    {~}{{\ }}1
    {\@\@}{{$@$}}1
    {\\mapsto}{{$\mapsto\;$}}1
    {\\hline}{{\rule{\linewidth}{0.5pt}}}1
    %
}[keywords,comments,strings]
\section{Machine State and Semantics}
% \section{Machine State \& Syntax}
\label{sec:syntax}
To develop our core logical ideas, we instantiate \textsf{Iris} with a simple language for streams of instructions, 
and a logical machine model corresponding to execution of x86-64 assembly instructions with virtual memory enabled on the 
CPU.

% \subsection{Registers and Memory}
Programs we demonstrate in this paper requires accessing two types of computer resource: registers and memory.
A register identifier, $\reg$, is chosen from a fixed finite set of register identifiers, $\regset$. 
We use these identifiers $\reg$ for register names such as \texttt{rax}, \texttt{r8}, or \texttt{cr3}. Our model includes
all x86-64 integer registers (including stack and instruction pointers), as well as \texttt{cr3} (for page table roots) and \texttt{rflags} (for
flags set by comparison operations and inspected by conditional jumps).
% % Figure environment removed
For clarity and ease of representation, we use machine words, $\loc \in \Loc$, with the subscripts showing the number of bits in a word,
for memory addresses, values, and offsets, rather than distinct location types that wrap machine words.
For example, $\kw{w}_{12}$ is a 12-bit word, which can be obtained for example truncating away 52 bits of a 64-bit word ($\kw{w}_{64}$).

% \subsection{State}
\label{sec:state}
We represent the machine state mainly as a finite map of registers to register values and a map of word-aligned physical memory addresses 
to 64-bit physical memory values. 
Thus our states $\sigma$ include register maps $\sigma.\mathcal{R}: \kw{greg} \rightarrow_{\textrm{fin}} \kw{regval} $ and
memory maps $\sigma.\mathcal{M}: \Locft \rightharpoonup_{\textrm{fin}} (\Loctw \rightharpoonup_{\textrm{fin}} \Locsf )$.
Of particular note, \texttt{cr3}, the page table register, is included in the machine state.
% As one might have already anticipated from the syntax we introduce, we do not bind any value of an evaluated expression. All the indices and accessed values are treated as globally referenced. In align with this design choice, our expression is a stream of instructions, which is not evaluated to a value to be bound, but changes the machine state through the indices (e.g. $\kw{r}\in\kw{greg}$) -- to the global maps.


% \subsection{Instructions}
\label{sec:instructions}

% % Figure environment removed

Programs in our logic are instruction sequences \instrs, which are formed by either a basic instruction \iskip, or prefixing an existing instruction
sequence with an additional instruction (\iseq\instr\instrs).
We model (and later, give program logic rules for) instructions for basic register moves, and reading and writing memory.
The latter require page table walks.
% Figure \ref{fig:coq_addr_translation} gives a slightly simplified version of our address translation code in monadic style:
% starting from the root page table address \textsf{rtv}, the virtual address \textsf{w} is either resolved to a physical address or an error is returned.
% The full set of possible errors is not relevant to this paper; our logic is tailored for kernel code which should not fault,
% so our proof rules guarantee no memory failures occur --- that the page table walk for any dereferenced address will succeed.

% Page-table walk shown in Figure \ref{fig:pagetables} is realized with physical memory \texttt{load}s and \textsf{store}s for the entries in the 
% page-tables which are simply modelled as lookup and updates to an instance of a memory map with 64 bit entries. In the address translation,
% we see a physical memory $\mathsf{load}$ (i.e. physical memory map ($\sigma.\mathcal{M}$) for each level) for each level of page-table-walk 
% concluded with alignment of the returned address for each \textit{successful} page-table-walk traversal which is what our reasoning principles 
% only consider.

The most important instructions we model are memory-accessing variants of the x86-64 \lstinline|mov| instruction, which we format in Intel syntax
(destination on the left, source on the right):
\[
\begin{array}{l}
  \hbox{(\TirNameStyle{WriteToMemFromReg})} \qquad
  \textsf{mov}~[r_m]~r_r \\
  \hbox{(\TirNameStyle{WriteToRegFromMem})} \qquad 
  \textsf{mov}~r_r~[r_m]
  \end{array}
  \]
The semantics of each is realized by first translating the virtual-memory address of a memory location stored in $r_m$ to a physical memory location
per the description in Section \ref{sec:backgroundonmachinemodel},
then either updating those memory contents with the contents of register $r_r$ (\TirNameStyle{WriteToMemFromReg}) 
or loading the value at that physical memory location into the register $r_r$
(\TirNameStyle{WriteToRegFromMem}). 
Additional \lstinline|mov| variants (e.g., accessing memory at constant offsets from the base register, or moves between registers) are also treated.
Our formalization also includes basic integer and bitwise operations (\lstinline|add|,
\lstinline|and|, bit shifts, etc.) with their effects on \lstinline|rflags|, jumps and some (not all) conditional jumps, \lstinline|call|, \lstinline|ret|, \lstinline|push|,
and \lstinline|pop|.

\section{Program Logic for Location Virtualization}
\label{sec:logic}
% The predicate gen_heap_interp.
\newcommand{\gammaPred}{\delta}
\newcommand{\gammaPreds}{\delta\textsf{s}}
\newcommand{\rtv}{\textsf{rtv}}
\newcommand{\qone}{\texttt{q1}}
\newcommand{\qtwo}{\texttt{q2}}
\newcommand{\qthree}{\texttt{q3}}
\newcommand{\qfour}{\texttt{q4}}

\newcommand{\sumwalkabs}[3]{
  \ownGhost\gammaPred{\authfrag{\singletonMap{#1}{(#2, #3)}}}
}

\newcommand{\sumapaces}[2]{
  \ownGhost\gammaPreds{\authfrag{\singletonMap{#1}{#2}}}
}
\newcommand{\ptableabswalk}[1]{\mathcal{A}\textsf{bsPTableWalk}(#1)}
\newcommand{\ptablestore}{\theta}

\add{
  A program logic for reasoning about code which may work with (and possibly update) multiple address spaces requires dealing with
  several key challenges. It must ensure that reasoning about memory accesses only depends on assertions that hold
  in the active address space at the time of access. It must allow invariants for code or data structures to refer
  to other address spaces. These constraints mean it must also support reasoning about when the active address space
  changes, as this affects which memory assumptions are usable or not for memory access and thus which data structures are
  immediately directly accessible.
}
\add{
  Many approaches could handle these problems in principle, such as tagging pointers with the relevant address space.
  But such approaches introduce other complexities.
  Most code, even in an operating system kernel, only works with a single address space (the current one).
  Explicitly plumbing address space information through a simple linked list specification, simply because some other part of the program
  may manipulate other address spaces, adds significant specification burden.
  Specifications which need to talk about a particular invariant holding in a specific other address space (or multiple other address spaces) would need to quantify
  over \emph{functions} from address space identity to assertions rather than just assertions.
}

\add{
  Using a modal approach resolves all of these challenges cleanly and uniformly.
  Specifications that are not \emph{about} address space manipulation need not mention address spaces in assertions.
  One can use standard separation logic data structure specifications without restructuring or
  adding explicit address space
  tracking,
  but every assertion can be still stated for either the current or specific other address spaces as needed.
  In short, modalities make it possible for specifications to mention address spaces when
  it is important to the code, and not when it is unimportant to the code.
  \looseness=-1
}

We describe a program logic (a separation logic) along the lines suggested {above}, where every assertion is relative
to an address space in which it is interpreted, allowing us to define \emph{virtual points-to} assertions that make claims
about memory locations in a particular address space. Virtual addresses, and even virtual points-to assertions, 
are not tagged with their address spaces in any way. Memory access in this logic is validated through the use
of virtual points-to assertions in preconditions, which guarantee that address translations succeed.
This supports rules for updating not only typical data in memory that happens to be subject to address translation, but \add{also}
manipulation of the page tables themselves via virtual addresses (as demanded by all modern hardware) {and}
via virtual points-to assertions.
To support specifications that deal with multiple address spaces, our logic incorporates a hybrid-style modality
$[r](P)$ to state that an assertion is true in another (assertion-specified) address space rather than the address space
currently active in hardware, which is not only useful for virtual memory manager invariants, but \add{also} critical to reasoning
about change of address space.
By developing this within the \iris framework, we obtain additional features (e.g., fractional permissions) that allow us to verify
some of the most subtle and technically challenging instruction sequences in an OS kernel (Section \ref{sec:experiment}).



% We derive a program logic (a separation logic) supporting the following stances and constraints:
% \begin{enumerate}
% \item \textit{address spaces as modal contexts}: Assertions in our logic are context-dependent,
%   in the sense that their truth depends on which address space they are used in, due to the need to support virtual points-to assertions.
% \item \textit{sharing}: The physical location backing a virtual address's storage is located (during a page table walk) through a 
%       set of physical page-table (L4-L1 page-tables) acceses that are shared amongst different virtual addresses (specifically,
%       those on the same page of memory\footnote{of in the case of L2 or higher levels of tables, within a given broader region.}).
%       This sharing imposes constraints on defining points-to assertions
%       in terms of physical (L4-L1) page-table memory accesses
% \item \textit{context-agnostic-resources}: each virtual address is valid under a certain address space, 
%       but it does not represent this \textit{knowledge} of its address space. 
%       That is, assertions are not explicitly tagged with their address space validity
% \item \textit{updating address-space mappings}: We present logical abstractions to enable 
%       updating not only pages of typical data in memory, but also page tables themselves.\footnote{Prior work relied on unfolding operational semantics
%       to verify page table updates.}
% \item \textit{explicitly-modal assertions}: Our logic includes a means to talk about facts being true
%       in another address space
% \item \textit{address-space switch as changing the "World" of truth}: Switching from one address-space to another logically
%       becomes a simultaneous introduction-and-elimination of a pair of modal assertions (for different address spaces)
% \end{enumerate}

% The idea that the truth of an assertion is relative to an address space has far-reaching consequences.
To support making assertions depend on a choice of address space, we work entirely in a pointwise lifting of \iris's base BI logic,
essentially working with separation logic assertions indexed by a choice of page table root as a $\mathcal{W}_{64}$, which we call $\textsf{vProp }\Sigma$:\footnote{
  \iris experts may notice our \lstinline|-b>| resembles another pointwise lifting already in  \iris~\cite{dang2019rustbelt,dang2022compass}. 
  This similarity is real, but the existing lifting does not appear to work with indexed \coq types like our \lstinline|word n| as a domain.
}
\lstinline[language=Coq]|Definition vProp  $\Sigma$ : bi := word 64 -b> iPropI  $\Sigma$|.
This is the (\rocq) type of assertions in our logic.
Most constructs in \iris's base logic are defined with respect to any BI-algebra (of \coq type \lstinline|bi|), so \add{they} automatically
carry over to our derived logic.
However, we must still build up from existing \iris primitives to provide new primitives that depend on the address space --- primarily the notion
of virtual points-to.
To define and use virtual points-to assertions, we require two basic assertions that ignore
the current address space:

\paragraph{Register points-to} 
The assertion $\textsf{r}\;\mapsto_{\textsf{r}}\{q\}\;\textsf{rv}$ ensures the ownership of the register $\rg$ containing the
value of the register $\rv$.
The fraction $\qfrac$ with value 1 asserts the unique ownership of the register mapping and grants update permission {to} it;
otherwise, any value $0 < \qfrac <1$ represents partial ownership, granting read-only permission on the mapping.\footnote{
\add{We adopt the standard naming convention of $\qfrac$-related names representing fractional permission, with fractions
sometimes appearing in braces or as subscripts in various asertions.}}

\paragraph{Physical memory  points-to} The soundness proofs for our logic's rules largely center around
proving that page-table-walk accesses as in Figure \ref{fig:pagetables} succeed, which requires assertions
dealing with physical memory locations.
We have two notions of physical points-to facts. The primitive notion closest to our machine model is captured by an assertion
$ \textsf{pfn} \ \sim \ \textsf{pageoff} \mapsto_{\textsf{a}} \; \{\textsf{q}\} \; \textsf{v} $, where \textsf{pfn} (a $\mathcal{W}_{52}$ \emph{page frame number}) essentially selects a 4KB page of physical memory,
and \textsf{pageoff} (a $\mathcal{W}_{12}$) is an offset within that page.
% could be an 52-bits masked address to level 4 table 
% ($\textsf{w1 } =( \textsf{ l4M52 maddr cr3val}) $),
%and, expectedly \textsf{w2} is an address computed by page-offset computation (e.g. $\textsf{l4off maddr cr3val}$). 
From this we can derive a more concise physical points-to when the split is unimportant:
% Giving a raw 64-bits memory pointsto assertion becomes
{$\textsf{w} \mapsto_{\textsf{p}} \{q\} \textsf{ v} \stackrel{\triangle}{=} (\textsf{drop 12}~w) \ \sim \ (\textsf{bottom 12}~w)\mapsto_{\textsf{a}} \; \{\textsf{q}\} \textsf{ v} $}

 %I put a newline in between the following two Definitions as the second Definition seems not proper without the newline
% Figure environment removed



\subsection{An Overly-Restrictive Definition for Virtual Memory Addressing}
\label{sec:overly-restrictive}
A natural definition for a virtual points-to
asserting that virtual address \textsf{va} points to a value \textsf{v}
would 
% require that in order for a virtual address \textsf{va} to point to a value \textsf{v}, the assertion 
contain
partial ownership of the physical memory involved in the page table walk that would translate \textsf{va} to
its backing physical location --- with locations existentially quantified since a virtual points-to should not assert
\emph{which} locations are accessed in a page table walk, as in Figure \ref{fig:strongvirtualpointsto}.
It asserts the existence of four page-table entries, one at each translation level, and via \lstinline|L4_L1_PointsTo|
asserts that the physical page table walk (per Figure \ref{fig:pagetables}) succeeds in reaching the L1 entry,
which points to the page holding the physical memory backing the virtual address, which contains value \textsf{v}.
Most of the definition lives directly in \textsf{vProp}, using the separation logic structure lifted from \iris's \textsf{iProp}.
\looseness=-1

\lstinline|L4_L1_PointsTo| works by
chaining together the entries for each level, using the sequence of table offsets from the address being translated to index
each table level, and using the physical page address embedded in each entry.\footnote{
  The fractions \lstinline|q1| through \lstinline|q4| represent the fractional ownership of each entry based on how many
  word-aligned addresses might need to share the entry ---  $(\frac{1}{512})^n$ for each level $n$.
}
For example, the first-level address translation to get the L4 entry (\lstinline|l4e|) 
  uses the masks \textsf{l4M52} with the current \lstinline|cr3| to get the physical address of the start of the L4 table
  and \textsf{l4off} with the virtual address being translated to compute the correct \add{byte} offset within that table \add{just as in the first translation
  step of Figure \ref{fig:pagetables}}.\footnote{\add{Note the offsets mentioned in Figure \ref{fig:pagetables} are 9-bit indexes into the 512 entries; the byte offset is that times 8.}}
    Thus Line \ref{line:l4pointsto} asserts that the physical address built from the table base and offset points to the L4 entry \textsf{l4e}.
  Subsequent levels of the page table walk \add{assertion (Lines \ref{line:l3pointsto}--\ref{line:l1pointsto})} work similarly.
The statement of these assertions is simplified by the use of our split physical points-to assertions, since
each level of tables is page-sized. \footnote{We do not address superpages and hugepages in this paper.}
This helper definition is also more explicit in \textsf{vProp} which binds a value to \lstinline|cr3| and uses it to start the translation process.
\looseness=-1


% Given the definition of physical page-pointsto assertion and the root address of virtual-address space as shown in Figure \ref{fig:pagetable}, one can build the physical address-translation for a virtual address (e.g. \textsf{va}) via abstracting the L4-L1 table traversal as the following:
% \begin{itemize}
%   \item Level-4 Translation (L4): Performs 
%   the first level address translation to get the L4 entry (L4E) in Figure \ref{fig:pagetables} by using the masks l4M52 and l4off with \textsf{rtv} virtual base address to get the starting address of the L4 table (L4T), and locate the entry amongst 512 (q1) ones respectively.
%     \begin{lstlisting}[language=Coq]
%       $\hbox{(\TirNameStyle{{L4translate}})} \quad$ (l4M52 maddr rtv) \$\sim$\ (l4off maddr rtv) $\mapsto_{a}$ {q1} l4e 
%     \end{lstlisting}
%  \item Level-3 Translation (L3): Performs the second level address translation to get the L3 entry (L3E) in Figure \ref{fig:pagetables} by using the masks l3M52 and l3off with L4 entry (l4e) to get the starting address of the L3 table (L3T), and locate the entry amongst 512 ones (q2) respectively.
%     \begin{lstlisting}[language=Coq]
%     $\hbox{(\TirNameStyle{{L3translate}})}$ (l3M52 maddr l4e) \$\sim$\ (l3off maddr l4e)  $\mapsto_{a}$ {q2} l3e
%     \end{lstlisting}
%   \item Level-2 Translation (L2): Performs the second level address translation to get the L2 entry (L2E) in Figure \ref{fig:pagetables} by using the masks l2M52 and l2off with L3 entry (l3e) to get the starting address of the L2 table (L2T), and locate the entry amongst 512 ones (q3) respectively.
% \begin{lstlisting}[language=Coq]
%     $\hbox{(\TirNameStyle{{L2translate}})}$ (l2M52 maddr l3e) \$\sim$\ (l2off maddr l3e)  $\mapsto_{a}$ {q3} l2e
%     \end{lstlisting}
%   \item Level-1 Translation (L1): Performs the second level address translation to get the L1 entry (L1E) in Figure \ref{fig:pagetables} by using the masks l1M52 and l1off with L2 entry (l2e) to get the starting address of the L1 table (L1T), and locate the entry amongst ones 512 (q4) respectively.
%    \begin{lstlisting}[language=Coq]
%     $\hbox{(\TirNameStyle{{L1translate}})}$ (l1M52 maddr l2e) \$\sim$\ (l1off maddr l2e)  $\mapsto_{a}$ {q4} l1e
%     \end{lstlisting}
%   \item Page Address Level Translation: Final computed physical page-address ($\textsf{addr\_L1}(\vaddr$,$\textsf{l1e}$)) points-to the value stored in the address ($\vpage$).
%    \begin{lstlisting}[language=Coq]
%     $\hbox{(\TirNameStyle{{PageLevelAccess}})} \qquad$ addr_L1($\vaddr,\textsf{l1e}$) $\mapsto_{p} \vpage$.
%     \end{lstlisting} 
% \end{itemize}

This solution is in fact very close to that of \citet{kolanski08vstte}, who define a separation logic from scratch in \textsc{Isabelle/HOL},
where the semantics of all assertions are functions from pairs of heaps and page table root values to booleans.\footnote{
  This was a typical explicit construction at the time; their work significantly predates \iris.
}
Our solution in the next subsection improves on theirs, removing some restrictions in this definition by further abstracting the handling of address translation.
\looseness=-1

\subsection{Aliasing/Sharing Physical Pages}
  \label{sec:sharingpages}  
  The virtual points-to definition shown in Figure \ref{fig:strongvirtualpointsto} 
  is too strong to specify some operations that a virtual memory manager may need to do, such as move one level of the page table to a different physical location while preserving all virtual-to-physical mappings. %\footnote{
  %   x86-64 hardware, like other architectures, includes a feature (which we do not formalize assertions for) to
  %   replace an L1 page table address in an L2 entry with a pointer to a \emph{larger} 2MB page (called super-pages), 
  %   or replace an L2 page table address in an L3 entry with a pointer to a 1GB page (called huge-pages).
  % }
  The use of $\textsf{L}_{4}\_\textsf{L}_{1}\_\textsf{PointsTo}$ in Figure \ref{fig:strongvirtualpointsto}'s
  virtual points-to definition stores knowledge of the page table walk details with ownership of the backing physical memory.
  Updating any of these mappings (e.g., moving the page tables in physical memory, as in coalescing for superpages or hugepages)
  would require explicitly collecting all virtual points-to facts that traverse affected entries.
  It is preferable to permit the page tables themselves to be updated independently of the virtual points-to assertions,
  so long as those updates preserve the same virtual-to-physical translations.
  But this is not possible with Figure \ref{fig:strongvirtualpointsto}'s definition, which ties ownership of particular pieces of page table memory to the virtual points-to.


  % iris.sty lacks nice syntax for the ghost maps
  \newcommand{\ghostmaptoken}[3]{\ensuremath{#2\hookrightarrow^{#1}#3}}
  \newcommand{\fracghostmaptoken}[4]{\ensuremath{#2\hookrightarrow^{#1}_{#4}#3}}

\newcommand{\vale}{\textsf{val}}
% Figure environment removed  

  % Intuitively, the definition in Figure \ref{fig:strongvirtualpointsto} is too strong because the virtual points-to
  % assertion there tracks too much information: when writing programs that access memory via virtual addresses,
  % most code does not care \emph{which physical memory locations are involved in address translation}: it only cares
  % that virtual address translation would succeed. The necessary information about the physical page table walk
  % must still be tracked, but can be tracked separately from the virtual points-to assertion itself.
  % In practice the decisions about which virtual addresses are valid rest not with code posessing a virtual address, but with
  % the virtual memory manager --- and its invariants.

  % Figure environment removed

  We separate the physical page-table walk from the virtual points-to relation, replacing it with a ghost state that merely guarantees that the address translation would succeed.
  \iris includes a \emph{ghost map} construction, which we use to track mappings from virtual addresses to the physical addresses they translate to as a piece of ghost state.
  The map includes, for each key in the map (i.e.,
  each virtual address), a token $\ghostmaptoken{\gamma}{k}{v}$ whose ownership is required to update that key-value pair in the ghost map named $\gamma$. The existence of such a token implies that the actual map $\theta$ tracked by a corresponding $\mathsf{GhostMap}(\gamma,\theta)$
  resource indeed maps $k$ to $v$. These properties are captured by key \iris rules in Figure \ref{fig:ghostmaps}.\footnote{\iris ghost maps lack established notation\add{;}
   the syntax we use captures the details of \texttt{iris.base\_logic.lib.ghost\_map}.}
  There are other rules, but these two are most important for explaining ghost maps.
  \textsc{GhostMapUpdate} says that ownership of the actual ghost map with ghost name $\gamma$ and map contents $\theta$,
  and a token witnessing that $\theta$ maps \textsf{pa} to \textsf{va} permits an update to the ghost map's state,
  changing the map and replacing the token to represent the new value.
  \textsc{GhostMapLookup} allows using the same information to simply conclude that the mapping indicated by the token is true.
  
  The \emph{virtual memory manager's invariant} ensures that for each $\ghostmaptoken{\gamma}{\vaddr}{\paddr}$ mapping in this map, there are \emph{physical} resources sufficient to ensure that the address translation for $\vaddr$
will resolve on the hardware to $\paddr$ --- via $\textsf{L}_{4}\_\textsf{L}_{1}\_\textsf{PointsTo}$.
  \add{This kernel invariant turns out to be a key ingredient in supporting proofs of VMM functionality:
  in Section \ref{sec:experiment} we will see that separating the logical and physical virtual-to-physical mappings
  is what allows stating the global kernel invariants needed for software page traversals, which prior work did not (and could not) pursue.}
  \looseness=-1

  %Thus the specification of \emph{which} physical addresses support translation is separated from the virtual points-to.
  % But these physical resources, which specify \emph{which} physical locations
  %are involved in the page table walk, are now separated from but consistent with
  %the knowledge that such resources exist (which is embodied by the token for $va$, which tracks that $va$ maps to $pa$
  %in the ghost map). Thus we can store the token which summarizes the translation and ensures it exists in the virtual
  %points-to, and keep the ghost map and the invariant that every mapping in the ghost map has corresponding physical resources
  %for translation in a separate global invariant for each address space.

  % In Iris this is realized by using an authoritative resource algebra: there is a single \emph{authoritative} global copy of the (ghost)
  % map caching virtual-to-physical address translations, and for each entry a read only \emph{partial} ownership of that key-value pair.
  % The resource algebra itself is instantiated as:
  % \[\mathcal{A}\textsf{bsPTableWalk} \stackrel{\triangle}{=} \textsc{Auth} (\; \mathcal{W}_{64} \;\rightarrow_{\textrm{fin}} \;  ( (\textsc{Frac }, \mathord{+}) \times (\textsc{Agree } \Loc,\mathord{=}) ))\]
For clarity, we refer to the specific ghost map summarizing virtual-to-physical translations by 
\mbox{$\mathcal{A}\textsf{bsPTableWalk}(\delta,\theta) \stackrel{\triangle}{=} \mathsf{GhostMap}(\delta,\theta)$}
(omitting $\delta$ for brevity when only one is in scope)
and keep this in a per-address-space invariant described shortly.
We then replace the physical traversal $\textsf{L}_{4}\_\textsf{L}_{1}\_\textsf{PointsTo}$ in Figure \ref{fig:strongvirtualpointsto}'s virtual points-to definition
with ownership of the token \ghostmaptoken{\delta}{\vaddr}{\paddr}, %ghost-map ($ \sumwalkabs\vaddr\qfrac\paddr$),
yielding Figure \ref{fig:virtualpointstosharing}'s definition.
This new definition guarantees that the ghost map \add{$\theta$} maps the virtual address ($\vaddr$) to a physical address ($\paddr$),
and thus that the per-address-space invariant \add{described next} will contain the physical resources that guarantee \add{that} the hardware resolves the translation.
\looseness=-1

  % Figure environment removed

We place the authorative ownership of the ghost \add{map} translation $\mathcal{A}\textsf{PTableWalk}$ in a per-address-space invariant
$ I$\textsf{ASpace} (Figure \ref{fig:peraspaceinvariant}), 
{allowing} changes to the page tables
that preserve overall virtual-to-physical translations \add{in isolation},
\add{and also allowing changes to specific the virtual-to-physical translations}
when combined with the
{token} stored in the \add{relevant} virtual points-to (Figure \ref{fig:virtualpointstosharing}).
\looseness=-1

% \todo[inline,color=cyan]{Explain $\delta{s}$ (Iris ghost name) vs $m$ (logical contents of ghost map) in next 2 paragraphs}
% \todo[inline]{This next paragraph below explains details, but should first explain the big picture: the current address space
% is identified by a paddr, so to state the invariant for the address space named by that paddr we need to look up
% what invariants should hold, then assert that those invariants do hold.}
\add{We must also ensure that different address spaces can have independent ghost maps ---
which we resolve with an additional unique global ghost map (with ghost name $\delta{s}$) from address-space identifiers (page table roots
whose values are manipulated by the kernel code) to
the \textsc{Iris} ghost name for that address space's ghost map.
In Figure \ref{fig:virtualpointstosharing}, the extra ghost map token for $\delta{s}$ asserts that $\delta$ --- which is exisentially
quantified --- is the correct ghost name for the current address space. That is then the ghost map named
in the ghost virtual-to-physical translation token of Figure \ref{fig:virtualpointstosharing}.
}
\add{
Just as the ghost name $\delta$ names the ghost map with contents $\theta$,
$\delta{s}$ names a ghost map, whose contents appear as $m$ in Figure \ref{fig:peraspaceinvariant} (the association of $\delta{s}$ to $m$
is a global invariant not shown).
}
$ I\textsf{ASpace}(\theta,m)$ then \add{performs 3 roles:
it associates the current address space's root with an appropriate \textsc{Iris} ghost name $\delta$;
it tracks authoritatively that $\delta$'s logical contents match $\theta$; and it}
stores the physical resources for the current address space mappings \add{(via the iterated $\textsf{L}_{4}\_\textsf{L}_{1}\_\textsf{PointsTo}$)}.

  
\subsection{Address Space Management}
\label{sec:aspacemanagement}
% So far, we have introduced logical abstractions for a single address space, but VMMs
Real VMMs must
 handle more than one address space.
Doing so requires a way to talk about other address spaces, and means to switch address spaces.
% Figure environment removed
Figure \ref{fig:modaldef} gives the definition of our modal operator for asserting the truth of a modal
(address-space-contingent) assertion \emph{in another address space}, which we call
the \emph{other-space} modality. The definition itself is not
particularly surprising --- as our modal assertions are semantically predicates on a page table root (physical)
address, the assertion $[r](P)$ is a modal assertion that ignores the (implicit) current page table root,
and evaluates the truth of $P$ as if $r$ were the page table root. 
The novelty here is not in the details of the definition, but in recognizing that this is the right way to deal with
multiple address spaces, and working out how to support interaction of multiple address spaces (discussed in the next section).
\add{The modal assertions, together with the other-space modality, mean we can give generic definitions
of data structure assertions (e.g., linked lists, etc.) which do not need to track information
about their own address space. In fact, \emph{only} assertions that explicitly deal with multiple address
spaces need to mention address spaces at all (via the other-space modality).
}

We can prove that this modality follows certain basic laws, showing that its truth is independent of the address
space in which it is considered, \add{that} it distributes over various logical connectives, and \add{that} it follows the rule of
consequence.
We call \textsf{vProp} assertions whose truth is independent of the current address space
\textsf{Fact}s; these include other-space assertions, physical memory points-tos, and register assertions.
\textsf{Fact}s can \add{freely} move in and out of other address space modalities.
\looseness=-1

In general, per-address-space invariants should be collected in a larger
VMM invariant, with individual address spaces' invariants pulled out as needed, such as when proving
soundness of an individual virtual memory access.
However, such larger invariants would contain many kernel-specific properties that are orthogonal
to the fundamental reasoning principles that are the focus of this paper.
We leave such kernel-specific reasoning to future work, but our verification of task switching
(Section \ref{sec:experiment}) demonstrates support for managing multiple address spaces.
\looseness=-1



  
% % Figure environment removed

% \begin{remark}[The Choice of Modal Context, Contingency and  Pattern of Verification Context]
%   \label{remark:pattern}
% The truth on an address space, exhibits itself as a contingent truth: a location virtualization assertion (a virtual points-to in \ref{fig:virtualpointstosharing}),  happens to valid in the \textit{current world} (in the current address space), and switching address spaces pulls information out of one \textit{world} into the “current view” of memory, and leaves other assertions true relative to the previous address space.

% Therefore, an address-space, as an abstraction, can be treated as naming the memory state as a modal frame and the choice of page table root as a world in Kripke-style semantics. Not suprisingly, transition between two address-spaces, then, can be just an entailment relation \textit{alternating} the \textit{named-state}

% Being inspired by what hybrid logic \ref{} calls a satisfaction operator, which evaluates the truth of a predicate in a named alternative state (here, address space), we give a modal definition describing the truth of assertions for the resources (i.e. virtual pointsto relations) inside the address space. Ignoring the predicate types for a while, $[r]P$ in Figure \ref{fig:modaldef}, indicates that $P$ holds in the virtual address space rooted at $r$, the truth ($P$) on an address-space is indexed by the root page-table address $r$ of the address-space. In the rest of this section, we explain the structural aspects -- the modal resource context of address space modality -- , and how we lift the interaction of address-space modality with the ambient logic, i.e. separation logic as an entailment for specifying \textit{address-space-switch}.

% Pragmatically, 
%   \begin{enumerate}
%   \item a modal context with its context-resources and picked contingency defines a modality (e.g. a address-space modality $[r]P$). For the convenience of verification, it enables focusing on the certain facts in the interest of reasoning (e.g. virtual pointsto relations in the current address-space)
%   \item and, offloads the burden of individual bookkeeping of these facts (e.g. virtual pointsto relations per address-space) under different context (e.g. virtual pointsto relations in other address-spaces) via utilizing the \textit{summarization} aspect of the contingency it represents (e.g. other virtual pointsto relations are valid under other certain address-space page-table root addresseses).  
%   \end{enumerate}
% \end{remark}




\subsection{Selected Logical Rules}
\label{sec:selected_rules}
% Per the discussion in Section \ref{sec:issues}, w
As common for assembly-level verification~\add{\cite{Ni2006codeptrs,ni2007contexts}}, we define our logic using Hoare doubles:%\footnote{This
% omits some low-level Iris details (stuckness, observations) that play no meaningful
% role in our development.}
\\\centerline{$
  \begin{array}{l}
    %\textsf{wpd\_def e s E1 } \Phi \;\mathsf{ rtv } : \textsf{iProp }\Sigma := \\
    \{ \Phi \}_\mathsf{ rtv }\;\textsf{e} : \textsf{iProp }\Sigma := 
   % \qquad
   ((\textsf{cr3} \mapsto_{\textsf{r}} \textsf{rtv} \ast \Phi) \textsf{ rtv}) \wand \textsf{WP e } \{\_, \textsf{True} \}
    \end{array}
$}\\
Our Hoare doubles $\{\Phi\}_\textsf{rtv}\;\textsf{e}$ state that the expression (i.e., sequence of instructions)
\textsf{e} are safe to execute (will not fault)
when executed with \textsf{vProp} precondition $\Phi\ast\textsf{cr3}\mapsto_{\textsf{r}} \textsf{rtv}$.
\textsf{WP} is \iris's own weakest precondition modality, unmodified~\cite{jung2018iris}.
Making \textsf{rtv} a parameter to the double (vs.\ a simple register assertion)
makes it possible to ensure ownership of the \lstinline|cr3| register and its value is accounted for
while avoiding some technical headaches with trying to enforce that $\Phi$ itself contains that.
\looseness=-1
% solves a technical problem with ensuring that the page table root used to evaluate
% the \textsf{vProp} (i.e., evaluating the assertion in the \emph{current}) address space
% is feasible.
% \footnote{Consider the difficulty of selecting the correct page table root value from an arbitrary
% opaque $\Phi$, which may even existentially quantify the page table root. An alternative is to
% require $\Phi$ to have a syntactic form where we can directly extract the value of \lstinline|cr3|,
% but this makes using Iris Proof Mode (IPM)~\cite{Krebbers:2017:IPH:3009837.3009855} with \textsf{vProps}
%   difficult; IPM works for any type matching the signature of an Iris \lstinline|bi|, which includes
%   \textsf{vProp}s, but manually guiding IPM to put an assertion in a specific position over and over adds
%   significant proof burden.
% }

The rest of this section describes specifications of three key \textsf{AMD64} instructions 
in our logic. 
These rules and others (e.g., including accessing memory at an instruction-specified offset from a register
value, which is common in most ISAs)
can be found in our artifact.
% Each rule in Figure \ref{fig:wpdamd}, 
% is annotated with a root (i.e., \lstinline|cr3|) address value (\textsf{rtv}), 
% under which the resources mentioned in the specification are valid.
In general, we use metavariables $\textsf{r}_s$ and $\textsf{r}_d$ to specify source and destination registers
for each instruction, and prefix various register value variables with \textsf{rv}.
We sometimes use $\textsf{r}_a$ to emphasize when a register is expected to hold an address used
for memory access, though the figure also uses typical assembler conventions of specifying
memory access operands by bracketing the register holding the memory address.
Standard for Hoare doubles, there is a frame resource $P$ in each rule for passing resources
not used by the first instruction in sequence through to subsequent instructions.
Our rules include tracking of each instruction's memory address to track \lstinline|rip| updates, which is critical
for control transfer instructions. Our development also includes handling of the \lstinline|rflags| register updates from arithmetic instructions.
Most rules are otherwise standard (e.g., \lstinline|mov| between registers, etc.), with Figure \ref{fig:wpdamd} showing the rules
most unique to our development.
\add{As a reminder, in systems of Hoare doubles, an instruction's precondition appears in the conclusion of the rule,
and an instruction's ``postcondition'' appears as the precondition to subsequent instructions in the
antecedent of the rule.}
\looseness=-1

% Figure environment removed


\subsubsection{Accessing Virtual Addresses}
Figure \ref{fig:wpdamd} includes two  rules for accessing memory at an address stored in a register $r_a$. 
\add{Setting aside $P$, $ I\textsf{ASpace}$, and the instruction pointer \lstinline|rip|,}
\TirNameStyle{WriteToRegFromVirtMem} and \TirNameStyle{WriteToVirtMemFromReg}
are nearly-standard (assembly) separation logic rules for memory accesses~\add{\cite{Chlipala2013Bedrock,ni2007contexts}}.
\add{For example, \textsc{WriteToRegFromVirtMem}'s specification
reflects that it reads from the (virtual) memory address \textsf{vaddr} stored in the address
register $\textsf{r}_a$ --- and thus requires register points-to and virtual points-to assertions describing
that relationship and the assumed value \textsf{v} in memory in its precondition (the precondition of the rule's
conclusion). 
It reflects the load (\lstinline|mov|) of that memory value into the destination register $\textsf{r}_d$, with
the updated register points-to in the precondition for $\overline{is}$.
$P$ describes framed resources, which are passed along to the precondition of subsequent instructions,
as in any system of Hoare doubles~\cite{Chlipala2013Bedrock,ni2007contexts}.
\textsc{WriteToVirtMemFromReg} is analogous for writing to memory.
}
\add{There are only two changes specific to our approach.}
\looseness=-1

First,
because we split the physical resources for the page table walk from the
virtual points-to itself ({per the discussion of Section \ref{sec:sharingpages}}), the rule requires $ I\textsf{ASpace}$
for the current address space to be carried through.
The soundness proofs for these rules extract
the token ($\fracghostmaptoken{\delta}{\vaddr}{\paddr}{\qfrac}$) from the virtual points-to,
use that to extract the physical page-table-traversal points-to collection describing
the page table walk for the relevant address ($\textsf{L}_{4}\_\textsf{L}_{1}\_\textsf{PointsTo}$)
from the invariant ($ I\mathsf{ASpace}$), prove that the page table walk succeeds
and that memory or registers are updated appropriately, before re-packing the invariant and virtual points-to resources.
\looseness=-1

Second, the memory access rules --- as with all rules in our logic ---
{increment} the instruction pointer \lstinline|rip| \add{by the length of the encoded instruction.}
\textsf{MovLen} returns how long the instruction encoding for the corresponding \lstinline|mov| is;
x86-64 instruction encodings are often longer for instructions using registers that are absent from the 32-bit
x86 ISA that preceded x86-64.
\looseness=-1

\add{Note that the use of a modal abstraction of address space simplifies these rules.
The antecedents of \textsc{WriteToRegFromVirtMem} and \textsc{WriteToVirtMemFromReg}
only mention the address space in the index of the
Hoare double --- not in $P$, or the (virtual) points-to assertions.
There is no extra condition to discharge that the address being accessed is from
the current address space.
}

\subsubsection{Updating \lstinline|cr3|} 
Unlike other rules, \TirNameStyle{WriteToRegCtlFromRegModal} updates the root address of the 
address space determining the validity of resources, from $\rtv$ before the
\lstinline|mov| to $\textsf{rvs}$ afterwards. The global effects of this rule are reflected in moving
assertions \add{for the current address space ($P$ and $ I\mathsf{ASpace}$)} under an other-space modality for \add{the initial
(outgoing) address space} $\rtv$, and moving the new address space's assertions out of
the corresponding modality \add{(since after the \lstinline|mov|, those will hold in the
new current address space)}.
The \emph{global} aspect is important. A na\"ive frame rule would be unsound for \lstinline|cr3| updates:
one could frame out assertions in one address space, switch address spaces, and bring those assertions from the \emph{old}
address space back into the \emph{new} address space, where they may not hold. 
% Appendix \ref{sec:issues} gives more details.
It is often said that the frame rule (below) is one of the key pieces of separation logic.
\centerline{$
  \mbox{$\inferrule*[right=Frame]{
    \{P\}\;C\;\{Q\}
  }{
    \{P\ast R\}\;C\;\{Q\ast R\}
  }$}
  \qquad\begin{array}{c}\textrm{(unsound with address space changes)}\\\\\end{array}
$}\\
Such a rule is normally recoverable from Hoare doubles (see, e.g., \citet{Chlipala2011Bedrock,Chlipala2013Bedrock}).
However, in the presence of address space changes, the traditional frame rule is unsound.
Consider:\\
\centerline{$
  \inferrule*[right=Frame]{
    \inferrule{\ldots }{
    \{\textsf{Pre}\}
    \texttt{mov}~\texttt{\%cr3},~r%\lstinline|mov %cr3, r| 
    \{\textsf{Post}\}
    }
  }{
    \{a\mapsto_\mathsf{v} x \ast \textsf{Pre}\}
    \texttt{mov}~\texttt{\%cr3},~r%\lstinline|mov %cr3, r| 
    \{a\mapsto_\mathsf{v} x \ast \textsf{Post}\}
  }
$}\\
In this (broken!) hypothetical example,
both the precondition and the postcondition assert that $a\mapsto_\mathsf{v} x$ in the current address space, but
  the new address space may map $a$ to another value. So, this derivation clearly leads to an unsound conclusion. 
The essential problem is that the frame rule is motivated by local reasoning about local updates, but
a switch of address space is a \emph{global} change that may invalidate information about virtual addresses.
Thus, framing around arbitrary \lstinline|cr3| updates is unsound --- hence the \emph{global} nature of \textsc{WriteToRegCtlFromRegModal} ---
though a variant the ensures the same \lstinline|cr3| value is installed before and after the framing
can be recovered.


\subsection{Soundness}
\label{sec:soundness}
Our rules from Figure \ref{fig:wpdamd} are proven to be sound in \iris against an assembly-level hardware model
implementing a fragment of x86-64, including 64-bit address translation with 4-level page tables.
% \todo[inline]{
Our rules for control transfers (\lstinline|jne|, \lstinline|call|, and \lstinline|ret|) are currently
axiomatized (with completely standard specifications~\cite{ni2007contexts,Chlipala2013Bedrock})\footnote{
  \add{An assertion that code at some address is safe to call with a given precondition~\cite{Ni2006codeptrs}
  asserts that the address is mapped and that memory at that address decodes to an instruction sequence
  that is safe with that precondition.
  }
} 
because \iris's built-in machinery does not provide
convenient ways to discard the current continuation; adaptation of others'
approaches~\cite{de2023type} is future work.
Our soundness proofs for all other instructions (including, critically, all memory accesses)
are axiom-free.
% }
% At the moment our proofs do rely on 13 small axioms of properties which should be provable, but
% are challenging to discharge due to some representation choices in our model.\footnote{See \lstinline|srx/x64/machine/current_axioms.v|}
\looseness=-1

%\definecolor{dkgreen}{rgb}{0,0.6,0}
\definecolor{ltblue}{rgb}{0,0.4,0.4}
\definecolor{dkviolet}{rgb}{0.3,0,0.5}

% lstlisting coq style (inspired from a file of Assia Mahboubi)
\lstdefinelanguage{Coq}{ 
    % Anything betweeen $ becomes LaTeX math mode
    mathescape=true,
    % Comments may or not include Latex commands
    texcl=false, 
    % Vernacular commands
    morekeywords=[1]{Section, Module, End, Require, Import, Export,
        Variable, Variables, Parameter, Parameters, Axiom, Hypothesis,
        Hypotheses, Notation, Local, Tactic, Reserved, Scope, Open, Close,
        Bind, Delimit, Definition, Let, Ltac, Fixpoint, CoFixpoint, Add,
        Morphism, Relation, Implicit, Arguments, Unset, Contextual,
        Strict, Prenex, Implicits, Inductive, CoInductive, Record,
        Structure, Canonical, Coercion, Context, Class, Global, Instance,
        Program, Infix, Theorem, Lemma, Corollary, Proposition, Fact,
        Remark, Example, Proof, Goal, Save, Qed, Defined, Hint, Resolve,
        Rewrite, View, Search, Show, Print, Printing, All, Eval, Check,
        Projections, inside, outside, Def},
    % Gallina
    morekeywords=[2]{forall, exists, exists2, fun, fix, cofix, struct,
        match, with, end, as, in, return, let, if, is, then, else, for, of,
        nosimpl, when},
    % Sorts
    morekeywords=[3]{Type, Prop, Set, true, false, option},
    % Various tactics, some are std Coq subsumed by ssr, for the manual purpose
    morekeywords=[4]{pose, set, move, case, elim, apply, clear, hnf,
        intro, intros, generalize, rename, pattern, after, destruct,
        induction, using, refine, inversion, injection, rewrite, congr,
        unlock, compute, ring, field, fourier, replace, fold, unfold,
        change, cutrewrite, simpl, have, suff, wlog, suffices, without,
        loss, nat_norm, assert, cut, trivial, revert, bool_congr, nat_congr,
        symmetry, transitivity, auto, split, left, right, autorewrite},
    % Terminators
    morekeywords=[5]{by, done, exact, reflexivity, tauto, romega, omega,
        assumption, solve, contradiction, discriminate},
    % Control
    morekeywords=[6]{do, last, first, try, idtac, repeat},
    % Comments delimiters, we do turn this off for the manual
    morecomment=[s]{(*}{*)},
    % Spaces are not displayed as a special character
    showstringspaces=false,
    % String delimiters
    morestring=[b]",
    morestring=[d],
    % Size of tabulations
    tabsize=3,
    % Enables ASCII chars 128 to 255
    extendedchars=false,
    % Case sensitivity
    sensitive=true,
    % Automatic breaking of long lines
    breaklines=false,
    % Default style fors listings
    basicstyle=\small,
    % Position of captions is bottom
    captionpos=b,
    % flexible columns
    columns=[l]flexible,
    % Style for (listings') identifiers
    identifierstyle={\ttfamily\color{black}},
    % Style for declaration keywords
    keywordstyle=[1]{\ttfamily\color{dkviolet}},
    % Style for gallina keywords
    keywordstyle=[2]{\ttfamily\color{dkgreen}},
    % Style for sorts keywords
    keywordstyle=[3]{\ttfamily\color{ltblue}},
    % Style for tactics keywords
    keywordstyle=[4]{\ttfamily\color{dkblue}},
    % Style for terminators keywords
    keywordstyle=[5]{\ttfamily\color{dkred}},
    %Style for iterators
    %keywordstyle=[6]{\ttfamily\color{dkpink}},
    % Style for strings
    stringstyle=\ttfamily,
    % Style for comments
    commentstyle={\ttfamily\color{dkgreen}},
    %moredelim=**[is][\ttfamily\color{red}]{/&}{&/},
    literate=
    {\\forall}{{\color{dkgreen}{$\forall\;$}}}1
    {\\exists}{{$\exists\;$}}1
    {<-}{{$\leftarrow\;$}}1
    {=>}{{$\Rightarrow\;$}}1
    {==}{{\code{==}\;}}1
    {==>}{{\code{==>}\;}}1
    %    {:>}{{\code{:>}\;}}1
    {->}{{$\rightarrow\;$}}1
    {<->}{{$\leftrightarrow\;$}}1
    {<==}{{$\leq\;$}}1
    {\#}{{$^\star$}}1 
    {\\o}{{$\circ\;$}}1 
    {\@}{{$\cdot$}}1 
    {\/\\}{{$\wedge\;$}}1
    {\\\/}{{$\vee\;$}}1
    {++}{{\code{++}}}1
    {~}{{\ }}1
    {\@\@}{{$@$}}1
    {\\mapsto}{{$\mapsto\;$}}1
    {\\hline}{{\rule{\linewidth}{0.5pt}}}1
    %
}[keywords,comments,strings]

\section{Implementing Logical Machinery \& Soundness}
We build our program logic, as an instantiation of Iris~\cite{jung2018iris}.
\subsection{Soundness}
\label{sec:soundness}
Our logic, operates on the machine state, which means we do not need to augment the machine state. The invarian that we pick, \textit{central invariant} ($\textsf{x64\_h}$), is just semantic interpretation of stores in the machine state, $\sigma.\mathcal{R}$ and $\sigma.\mathcal{M}$. This semantic interpratation ensures the correct lifting of mappings in the machine state to the assertions that the client of our logic uses, i.e. points-to assertions that are defined as the ownership of a fragment of the logical state.

We prefer to skip explaining the steps used in instantiation of Iris because it is an almost standard procedure, has already been explained for many other logic \ref{}, and we are concerned with the page-count limitation. However, it is worh noting that once you instantiate Iris for your language, it comes with the semantic definition for the weakest-precondition which we can refactor into Hoare-Doubles to specify our  selected \textsf{AMD64} instructions shown in Figure \ref{fig:wpdamd}, and show that these triples are sound.
\subsection{The Soundness Statement}
\label{def:soundness:statement}
The operational semantics of our simple lang just executes sequences of instructions in our x86-64 model. Therefore, our soundness argument is to show that any execution composed of instructions in our machine model (some of which are shown in Figure \sref{sec:semantics}) does not end-up in a invalid state.
% Figure environment removed

\begin{theorem}[Soundness of the Logic]
  \label{th:adequacy}
 Together with the assumptions on the initial state hold,
 the execution of the instruction~$\instrs$, beginning with this initial state, cannot result in a configuration where the execution is stuck.
\end{theorem}
which states that if the program~$\instr$, with the given \textsf{valid\_init} asserting a valid state initialization, satisfies a semantic Hoare
triple, then this program cannot crash: by a direct consequence of Iris's adequacy theorem~\cite[\S6.4]{iris}.

Moreover we need to show the validity of each rules in Figures \fref{fig:reasoning} and \fref{fig:laws}.
\begin{theorem}[Validity of the Reasoning Rules]
\label{th:validity}
  Each of the rules in Figures~\ref{fig:wpdamd}
  and~\ref{fig:structural} is valid.
\end{theorem}
Due to the space limits in this paper, we do not mention each proof for the rules in Figures \ref{fig:wpdamd} and \ref{fig:structural} within this section, but we provide mechanized proofs for all these in Coq as a part of our artifact submission.
However, we would like to give the definitions and constructions used in our proofs, and would like to give an outline of paper proof for \TirNameStyle{WriteToRegFromVirtMem} in Figure \ref{fig:wpdamd} within this section.

Together, Theorems~\ref{th:adequacy} and~\ref{th:validity} guarantee that, if
the Hoare triple $\textsf{valid\_init }\instrs\;\iTrue$ can be obtained by applying
the reasoning rules of our logic, then the program~$\instrs$ is safe.

\subsection{Logical Constructions}
\label{sec:invariant}
We already have the physical and logical stores and a simple invariant between them. Now, we can rely on the following Assumption \ref{assumption} from Iris to utilize its logical constructions.
% The predicate gen_heap_interp.
\newcommand{\genheapinterp}[1]{\mathit{Heap}\;#1}
\newcommand{\genmemheapinterp}[1]{\mathit{MemHeap}\;#1}
% Our predicate pred (defined in ph.v), expanded.
\newcommand{\pred}[1]{\ownGhost\gammaPred{\authfull{(\mapone\predstore)}}}
% A notation for assigning fraction 1 to every element of \predstore.
\newcommand{\mapone}[1]{1.#1}
% The predicate mapsfrom_exact, expanded.
\newcommand{\mapsfromexact}[3]{
  \ownGhost\gammaPred{\authfrag{\singletonMap{#1}{(#2, #3)}}}
}
% A metavariable for a share.
\newcommand{\sh}{L'}
% The predicate mapsfrom, expanded.
\newcommand{\mapsfromdef}[3]{
  \exists\sh.\;
  \mapsfromexact{#1}{#2}{\sh} \star \pure{\sh \subseteq #3}
}

\begin{assumption}
\label{assumption}
Iris defines two pieces of ghost state
\begin{enumerate}
\item  defines a predicate $\textsf{to\_gen\_heap }\store.\mathcal{R}$
  that ties a store~$\store.\mathcal{R}$ to this ghost state,
  and defines the points-to assertion $\ppointsto\rg\rv\qfrac\rpts$
  in terms of this ghost state.
  This is visible in the paper~\cite[\S6.3.2]{iris}
  and in Iris's \texttt{gen\_heap} library~\cite{genheap}.
  %
  We re-use this machinery without change,
  so we do not repeat these definitions.
  We mention the predicate $\genheapinterp\!$
  in our own invariant (Definition~\ref{def:invariant}),
  where it is applied to the \logical store~$\store$.
\item unlike the register points-to relation obtained by direct interpretion of $\textsf{to\_gen\_heap }\store.\mathcal{R}$ using Iris, the existing \textsf{gen\_heap\_heap},
  does not directly helps introducing the ghost we define a new algebra (\textsf{gen\_mem\_UR}) for abstracting the nested maps due to different levels of masking in memory mappings and an interpretation for this algebra. 
\end{enumerate}
\end{assumption}

\begin{definition}[Ghost State - Memory with Nested Mappings]
We define our custom-tailored algebra 
\newcommand\fpfn{\rightarrow_{\textrm{fin}}}
\( \textsf{gen\_memUR} \stackrel{\triangle}{=}
  \authm(\;
  \Locft \;\fpfn\;
  (\Loctw \;\fpfn\;  (\textsc{Frac }, \mathord{+}) \times (\textsc{Agree } \Loc,\mathord{=}) )
  \)
  for our nested memory mapping abstracting two different machine word masking. To interpret this nested ghost map, i.e. obtain points-to assertions out of mappings in an ordinary \textsf{gmap}, we define \textsf{to\_gen\_mem}
  \begin{lstlisting}[language=Coq]
    Definition to_gen_mem : gmap L1 ( gmap L2 V) $\rightarrow$ gen_memUR L1 L2 V := fmap ($\lambda$ m . to_gen_heap m).
    Definition to_gen_heap : gmap L V $\rightarrow$ gen_heapUR L V :=  $\lambda$  v $\ldotp$ (1, to_agree (v :leibnizO V)).
  \end{lstlisting}
 throuh using \textsf{to\_gen\_heap} from previous Iris version. \todo[inline,color=red]{Ismail give exact version commit etc.}
\end{definition}

\begin{definition}[Ghost State - Register Mappings]
We allocate $\theta$ ghost cell which stores an
element of the monoid 
\newcommand\fpfn{\rightarrow_{\textrm{fin}}}
\(
  \authm(\;
    \regset \;\fpfn\;
    (\textsc{Frac}, \mathord{+})
    \times
    (\regvaltype, \mathord{=})
  \;)
\)
% \emph{authoritative camera}
\cite[\S6.3.3]{iris}.
\end{definition}

\begin{definition}[Central Invariant]
\label{def:invariant}
The central invariant of our logic is, due to lack of need for augmenting the machine state, simply the state interpretation: 
\[
\textsf{x64\_h}\;\store \triangleq
\def\arraystretch{1.2}
\begin{array}{l@{\quad\ast\quad}l@{\quad}l}
  \textsf{to\_gen\_heap} \;\store.\mathcal{R} & \textsf{to\_gen\_mem} \; \store.\mathcal{M}
\end{array}
\]
\end{definition}

As a last definition, we give the head step relation required for the Iris instantiation for our simple language in Figure \ref{} to sequence instructions. This relation allows lifting the program expression to enable the application of changes imposed by the operational semantics on the program state (for a certain cpu, a register map, and a selected memory for an address-space) when applied for an insturction (\textsf{i}).
\begin{lstlisting}[language=Coq]
Definition exec_step (i: instr) ($\sigma$:state) : option state :=  exec_instr i $\sigma.\mathcal{C}$ $\sigma.\mathcal{R}$ ((memToPhysMem $\sigma.\mathcal{M}$)  ($\sigma.\mathcal{R}$ !! cr3)).
\end{lstlisting}
\todo[inline,color=yellow]{Colin, in case needed,not proven, could you simply say that we assume these map equalities or a paper proof etc.}. Again, due to the page limits, we can only give an outline of proof for a selected instruction (\TirNameStyle{WriteToRegFromVirtMem}) from our \textsf{AMD64} model. However, mechanized soundness proofs of other instructions can be found as a part of our submission artifact.

After obtaining ownership (points-to) predicates from application of state interpretation, now, we have well-enough definition for giving an outline for the proof of \TirNameStyle{WriteToRegFromVirtMem}.
 \begin{lemma}[\textsc{\TirNameStyle{WriteToRegFromVirtMem}}]
   \label{lemma:unlink}
\begin{align*}
\inferrule{
  \{\mathsf{P} \ast \mathsf{r}_d \mapsto_{r}  \mathsf{v} \ast \mathsf{r}_a \mapsto_{r} \{\mathsf{q}\} \; \mathsf{ vaddr} \ast \mathsf{vaddr} \mapsto_{\mathsf{v}} \mathsf{v} \}_{\mathsf{rtv}}\;\overline{is}
}{
  \{\mathsf{P} \ast \mathsf{r}_d \mapsto_{r}  \mathsf{rvd} \ast \mathsf{r}_a \mapsto_{r} \{\mathsf{q}\} \;\mathsf{ vaddr} \ast \mathsf{vaddr} \mapsto_{\mathsf{v}} \mathsf{v} \}_{\mathsf{rtv}}
\; \mathsf{ mov}~\mathsf{r}_d~\mathsf{r}_a;\;\overline{is}
}
\end{align*}
 \end{lemma}
 
 \begin{proof}
   Assuming the inference rule realized with \textsf{wpd\_def}, we expand the precondition with $\textsf{cr3} \mapsto_{\textsf{r}} \rtv$.
   Then we do two proofs, one for head reducibility of atomic step \textsf{mov\_reg64\_mem64}, the second one for the executing the expression and obtaining the new state. Steps taken in the first one are subset of the second one, so we outline the second portion of the proof, but in case of an interest in details of the proof, Coq artifact can be consulted.

   \begin{itemize}
   \item Step 1: we apply the head step relation and obtain the current valid state $\sigma1$ interpretation : $\textsf{x64\_h}\;\store$
   \item Step 2: we unfold the virtual-pointsto ($\vaddr \mapsto_{\textsf{v,rtv}} \textsf{v}$) definition, and for an existential physical page address $\paddr$, we exchange our fragmental toke ($\sumwalkabs\vaddr\qfrac\paddr$) to obtain physical table-pointsto ($\textsf{L}_{4}\_\textsf{L}_{1}\_\textsf{PointsTo}$ in Figure \ref{fig:strongvirtualpointsto}) relation
   \item Step 3: for each physical pointsto inside $\textsf{L}_{4}\_\textsf{L}_{1}\_\textsf{PointsTo}$, there exists a \textsf{load}, i.e. a concrete memory lookup
   \item Step 4: use these concrete lookups to traverse the page tables to obtain the value \textsf{v}
     \item Step 5: do the map update the $\sigma.\mathcal{R}$ for relevant register mapping ($r_d$) with the value \textsf{v} 
   \end{itemize}
   
   \end{proof}

%$\assert{\ulcorner \mathsf{aligned maddr} \urcorner \ast \mathsf{r14} \mapsto_{\textsf{r}} \textsf{r14v} \ast \mathsf{r13} \mapsto_{\textsf{r}} \textsf{maddr} \ast \mathsf{rdi} \mapsto_{\textsf{r}} \textsf{rdiv} \ast \mathsf{rax} \mapsto_{\textsf{r}} \textsf{raxv}}$
%$\assert{\ulcorner \ptablestore !! \textsf{maddr} = \textsf{None} \urcorner \ast \ownGhost\gammaPred{\authfull{\ptableabswalk\ptablestore}} \ast \textsf{Pf}} $
\definecolor{main-color}{rgb}{0.6627, 0.7176, 0.7764}
\definecolor{back-color}{rgb}{0.1686, 0.1686, 0.1686}
\definecolor{string-color}{rgb}{0.3333, 0.5254, 0.345}
\definecolor{key-color}{rgb}{0.8, 0.47, 0.196}

\newcommand{\sumwalkabsent}{
  \ownGhost\gammaPred{\authfrag{\singletonMap{\texttt{entry+KERNBASE}}{(\textsf{qfrac}, \textsf{entry})}}}
}


\newcommand{\ventry}{\texttt{entry + KERNBASE}}
\newcommand{\entry}{\texttt{entry}}
\newcommand{\qfraczero}{\textsf{qfrac}}
\newcommand{\true}{\textsf{true}}
\tikzstyle{boxedassert_border} = [sharp corners,line width=0.2pt]
\NewDocumentCommand \boxedassertpv {O{} m o}{%
	\tikz[baseline=(m.base)]{
		%	  \node[rectangle, draw,inner sep=0.8pt,anchor=base,#1] (m) {${#2}\mathstrut$};
		\node[rectangle,inner sep=1.5pt,outer sep=0.2pt,anchor=base] (m) {${\,#2\,}\mathstrut$};
		\draw[#1,boxedassert_border] ($(m.south west)$) rectangle ($(m.north east)$);
	}\IfNoValueF{#3}{^{\,#3}}%
}
\newcommand*{\knowInvpv}[2]{\boxedassertpv{#2}[#1]}
\newcommand*{\ownGhostpv}[2]{\boxedassertpv[dash dot]{#2}[#1]}

\newcommand{\sumpv}[3]{
  \ownGhostpv\gammaPred{\authfrag{\singletonMap{#1}{(#2, #3)}}}
}

\newcommand{\pvmapping}[1]{\mathcal{A}\textsf{P2VMappings}(#1)}


\newcommand{\fpaddr}{\texttt{fpaddr}}
\newcommand{\specline}[1]{{\color{blue}\left\{#1\right\}}}
\newcommand{\sumapacesfull}[2]{
  \ownGhost\gammaPreds{\authfull{\singletonMap{#1}{#2}}}
}
\section{Experiments}
\label{sec:experiment}
%To both validate and demonstrate the value of the modal approach to reasoning about virtual memory management, 
% we study several
% We validate our logic by studying
% distillations of key VMM functionality.
% real concerns of virtual memory managers.
% Recall from Section \ref{sec:logic} that virtual points-to assertions work just like regular points-to assertions, by design.
In this section, we verify several critical and challenging pieces of VMM code.
First, we formally verify a switch into a new address space as part of a task switch,
the first such verification handling both old and new processes' assertions (in different address spaces) at the time of the switch.
Then, in several stages, we work up to mapping a new page in the current address space, addressing significantly more of this process than prior
work that included address translation in its hardware model.
This requires a number of independently challenging substeps: dynamically traversing a page table to find
the appropriate L1 entry to update; inserting additional levels of the page table if necessary (updating
the VMM invariants along the way);
converting the physical addresses found in intermediate entries into the corresponding virtual addresses
that can be used for memory access;
installing the new mapping;
and collecting sufficient resources to form a virtual points-to assertion.
Of these, only the second-to-last step (installing the correct mapping into the
current address space) has previously been formally verified with respect to a machine model with address translation.


% Figure environment removed

\subsection{Change of Address Space}
A critical piece of \emph{trusted} code in current verified OS kernels is the assembly code to change the current address space; current verified OS kernels currently
lack effective ways to specify and reason about this low-level operation, for reasons outlined in Section \ref{sec:relwork}.

Figure \ref{fig:swtchC} gives simplified code for a basic task switch, the heart of an OS scheduler implementation. This is code that saves the context (registers and stack)
of the running thread, and resumes execution of a previously-suspended thread of execution.
In C this code would be given the signature
\mbox{\lstinline[language=C]|void swtch(context_t* save, context_t* restore)|}.
Saving the context is a straightforward matter of storing each register into the \lstinline|save| context.
Restoring the \lstinline|restore| context is the tricky bit, because both the stack pointer and address space must be restored.
Confusingly, a single dynamic execution of this function begins execution in one thread,
and returns in another thread --- because the execution switches stacks, and thus returns on the second thread's stack.%\footnote{This is the function in UNIX 6th Edition 
% with the infamous ``You are not expected to understand this'' comment~\cite{lions1996lions},
% though Doeppner~\cite{doeppner2010operating} and others, offer detailed explanations.}
Hence this is used, for example, when the scheduler has chosen a new thread to execute for a voluntary (non-preempted)
context switch, and will call
the code with \lstinline|save| pointing to a reserved storage space for the current thread, and
\lstinline|restore| pointing to the context of the next thread to execute.
We will ignore non-integer registers; others are handled similarly.
In Figure \ref{fig:swtchC}, Lines \ref{line:start_save}--\ref{line:end_save} store the callee-save registers (per the System V AMD64 ABI calling conventions) of the calling
thread into the context data structure pointed to by \lstinline|rdi| (at virtual address $save$).
This is justified by the $\textsf{ContextAt}(save,\_)$ assertion in the precondition (Line \ref{line:end_swtch_pre}), which expands
into a full set of full-permission (thus writable) virtual points-to assertions for various offsets from $save$, one for each saved register.
Verification up through line \ref{line:end_save} is straightforward application of \textsc{WriteToVirtMemFromReg} (Figure \ref{fig:wpdamd}).

The code to restore the previously-saved context located at $restore$ (accessed via \lstinline|rsi|) in Lines
\ref{line:start_restore}--\ref{line:end_restore}
is where the proof becomes subtle, though our logic makes the construction of the proof feel similar to typical assembly-level verification
because most instructions are verified with rules that work very similarly to standard proof rules while being proven
sound against a machine model with address translation.
Similar to the precondition for the save context, the restore context has a corresponding $\mathsf{ContextAt}(restore,[\ldots])$
assertion expanding to virtual points-to assertions --- in the caller's adddress space.
The \lstinline|mov| instructions prior to Line \ref{line:end_restore} are each verified with a fixed-register-offset
variant of \textsc{WriteToRegFromVirtMem}, but
Line \ref{line:stack_switch}'s implications are subtle because it switches stacks by updating \lstinline|rsp|.
Because the new stack pointer may only be valid in the address space of the restored context, stack accesses at this point are unsafe.
Prior to Line \ref{line:end_restore}, we can see in code and invariants that the local registers are updated
to match the values populating the restore context, except for the page table root.
Line \ref{line:end_restore} itself is verified with a rule similar to \textsc{WriteToRegCtlFromRegModal} (but obviously
reading from a fixed offset of a register, as needed in Line \ref{line:end_restore}).
This rule also globally moves assertions into and out-of other-space assertions, to reflect that
assertions holding in the outgoing address space \rtv{} generally will not hold in the incoming address space $\rtv'$
and vice versa. Thus the precondition has assertions for the new thread under an other-space modality for the new address space,
and the postcondition has assertions for the old thread under an other-space modality for the old address space.%\footnote{
  % The \textsf{ContextAt} assertions \emph{both} end up under the other-space modality for the old thread.
  % A real kernel would want to transfer both out, but justifying this is highly kernel-specific,
  % and particularly post-Spectre-and-Meltdown is quite varied.
  %\looseness=-1
%}
Both \textsf{ContextAt} assertions end up under the old other-space modality, but in a real kernel would
transfer out based on kernel-specific invariants.
\looseness=-1

The specification above does not directly discuss the relationship between instruction pointers and registers --- and does not need to
because \textsf{P} and \textsf{POther} can be instantiated to capture that relationship with additional information about stack contents.
This code is meant to be called with a return address for the current thread stored on the current stack,
and a return address for the target thread on the target thread's stack.
% But the target thread's precondition is \emph{relative to its address space}, 
% not the address space of the calling thread! This is reflected by
% the other-space modality
% $[\rtv']( I\texttt{ASpace}(\theta,\Xi,m) \ast \texttt{Pother})$
% in the specfication. 
For a given call site, \textsf{P} would be instantiated to require that the initial stack pointer (before \lstinline|rsp| is updated)
has a return address expecting the then-current callee-save register values in the \emph{current} (initial) address space
to (together with other resources used to instantiate \textsf{P}) imply the precondition of the code at that return address.
The situation for the target thread is similar, but using \textsf{POther}, \emph{and using the other-space modality}
because the other thread's stack, code, and other relevant assertions may only be valid in the new address space.
Our logic's rules for updating the page table root, and thus moving assertions into and out of other-space modalities,
neatly manage which assertions are \emph{currently} valid, without the need to explicitly plumb address space labels through
every assertion in the larger proof.

% Immediately after the page table switch, assertions about the saved and restored contexts are
% guarded by a modality for the retiring
% address space \rtv{} (Line \ref{line:modality_switch}), per
% \textsc{WriteToRegCtlFromRegModal} (Figure \ref{fig:wpdamd}),
% because
% there is no guarantee that the data structures of the previous address space are mapped in the new address space.
% The ability to transfer that points-to information out of that modality is specific to a given kernel's design. 
% Kernels that map kernel memory into all address spaces would need invariants
% that justified moving those assertions out of the other-space modality.
% % Following Spectre and Meltdown, this kernel design became less prevalent because speculative execution of accesses to kernel addresses could leak information even if the access did eventually cause a fault (the user/kernel mode permission check was done after fetching data from memory). Thus many modern kernels have reverted to the older kernel design where the kernel inhabits its own unique address space, and user processes have only enough extra material mapped in their address spaces to switch into the kernel (CPUs do not speculate past updates to \texttt{cr3}).
% \looseness=-1

Although prior work has verified context switches within a single address space~\cite{ni2007contexts}, and address space switches
without any code before or after~\cite{syeda2020formal} (that is, not reasoning about the \emph{impact} of address space change
on what data were accessible), this is the first verification that handles both.
\looseness=-1



%\begin{comment}
%\todo[inline]{Identity mappings are difficult, and our current approach won't quite work. Consider trying to have a virtual pointsto for an actual page table entry (i.e., that one could use to update a page table mapping), while also having a virtual pointsto for an address that entry mapped. With the current (let's call it v1) solution, we can't actually have both of those simultaneously!  That's because the PTE pointsto will assert full ownership of the physical memory cell holding the PTE as its data value, while the virtual pointsto for the data mapped by that entry will \emph{also} assert (fractional) ownership of all entries a page table walk would traverse.
%}
%\todo[inline,color=violet]{This doesn't seem to cause issues with the mapping/unmapping examples, only with changing intermediate page table pointers. The mapping example requires a virtual pointsto for the blank PTE, and once filled in that ownership can be immediately split to create the 512 new virtual pointsto assertions for the newly mapped page. Conversely, for unmapping we'd assume ownership of all the relevant virtual pointsto assertions for the page we're unmapping, at which point we can (with a bit of work) show that they all correspond to the same L1 PTE, and extract the 512 fractional shares of that entry from the pointsto assertions.  But changing intermediate page tables, as one would do for coallescing or splitting a superpage while preserving the virtual-to-physical mappings, couldn't be done without some really complicated separating implication tricks.}
%\todo[inline,color=green]{One possible approach to resolving this, which we came up with in our Tuesday meeting, is to recognize that the current (v1) virtual points-to is too strong, because it really doesn't care about \emph{owning} those fractional resources, it only cares that \emph{something} ensures the correct page table walk exists. Iris has a ghost map resource where authoritative ownership of an individual key-value pair can be handled as a resource.  (Colin was using this in the filesystem cache.)
%We can use that mechanism to separate the virtual-to-physical translation from the physical memory involved (Kolanski and Klein may have done something similar for different reasons): (fractional) virtual points-to assertions can be defined in terms of (fractional) ownership of these authoritative ghost map entry assertions, plus sharing an invariant that the current installed page table respects all entries of the mapping. Unmapping collects the authoritative map kvpairs from collecting the assertions, and then can remove them from the ghost map and update the page tables. Critically, physical ownership of the page tables then lives in the invariant on the current page table, so some virtual pointsto assertions can refer to memory in those page tables.
%This still works with the modality, since that invariant is also semantically a predicate on a page table root.
%Let's call this v2.
%}
%\end{comment}
\subsection{Traversing Live Page Tables}
\label{sec:traversingC}
We build up to the main task of mapping a new page after traversing the page tables in the software.
This algorithm is complex and corresponds to a significant amount of assembly code.
To assist with readability, we present C code for this process, with assertions adjusted slightly to refer to
C program variables rather than registers. The actual verification was carried out on x86-64 assembly
\emph{generated from this source code}.
Listings of the assembly fragments with inline assertions appear in
\ifARXIV
Appendix \ref{sec:experiment_appendix}.
\else
our technical report~\cite{kuru2025modal}.
\fi
Whether in C or assembly,
the page table traversal involved in mapping a new page is very challenging functionality to verify.
Loading the current table root from \lstinline|cr3| is straightforward (a \lstinline|mov| instruction).
However, this produces the \emph{physical} address stored in \lstinline|cr3|, not a \emph{virtual} address the kernel code can use to access that memory.
This problem repeats at each level of the page table: assuming that the code has \emph{somehow} read the appropriate L4 (or L3, or L2) entry, those entries again
yield physical addresses, not virtual.
The only prior work to verify page mapping ignored the traversal and only verified mapping
assuming code \emph{already} had an appropriate virtual address for the L1 entry, where a physical
address could simply be stored. Our proof is the first to additionally deal with the critical code
leading up to that point.

\paragraph{Code Overview}
As described in Section \ref{sec:background}, mapping a new page consists of 
simulating the hardware address translation of Figure \ref{fig:pagetables}, but in software.
The code for this task takes three explicit parameters:
the root pointer (read from \lstinline|cr3| by earlier code),
the page-aligned \emph{virtual} address (\lstinline|va|) at which to make a new piece of memory accessible,
and the \emph{physical} address (\lstinline|fpaddr|) of the memory to map in that location.
The function we ultimately verify, \lstinline|vaspace_mappage| (Figure \ref{fig:mapping_codeC}),
relies primarily on a helper function already shown in Figure \ref{fig:pagetablescode}.
\lstinline|walkpgdir| finds the (virtual) address of the the correct L1 entry to translate \lstinline|va|,
by walking the page tables in software one level at a time.
\lstinline|vaspace_mappage| then uses the result to install the new entry.
\lstinline|walkpgdir| itself relies on its own helper function \lstinline|pte_get_next_table|, also shown in Figure \ref{fig:pagetablescode},
which implements a single-level of traversal from level $n+1$ to level $n$ (and whose specification and proof are therefore
parameterized by page table level), allocating additional levels as needed.

We organize our explanation of the proofs by essentially following execution from the start of \lstinline|walkpgdir|, through
execution of \lstinline|pte_get_next_table|, and out to its callsite in \lstinline|vaspace_mappage|.
While slightly awkward because we start in the middle of the mapping execution,
this ordering allows us to start with the simpler pieces of the proof, and incrementally explain the complexities
of the proofs and kernel invariant, before concluding with the top-level verification.


% Figure environment removed

\subsubsection{From L$n+1$ Entries to L$n$ Tables}
We discuss access to the level 4 table later (Section \ref{wlkpgdirC}). However, for subsequent levels, the base address of level $n$ must be
fetched from the appropriate entry in the table of level $n+1$.
This is the role of \lstinline|pte_get_next_table| (originally Figure \ref{fig:pagetablescode}, with proof details in Figure \ref{fig:calltopteinitializeC}).
It is passed the virtual address of the page table entry in level $n+1$, and should return the \emph{virtual} 
address of the \emph{base} of the level $n$ table
indicated by that entry.
If the entry is empty (i.e., this is a sparse part of the page table representation),
the code also allocates a page for the level $n$ table, installs it in the level $n+1$ entry, and establishes appropriate invariants.
Figure \ref{fig:calltopteinitializeC} presents the function with proof annotations that we will explain shortly, but we first explain the functionality.
\lstinline|pte_get_next_table| accepts a \emph{virtual} address \lstinline|entry| which points to a level-$n+1$
table entry.
\looseness=-1

On Line \ref{line:read_entry_contentsC}, the code checks the present bit of the entry.
If the bit is unset, there is no level-$n$ table, so one must be allocated via \lstinline|pte_initialize| (explained shortly,
but it essentially
allocates a fresh physical page, and initializes the memory pointed to by \lstinline|entry| with that physical address) and marked present.
By Line \ref{line:finalpieceS} the entry is known to be valid and contain the physical address of
the base of a level $n$ table. That address is then extracted (Line \ref{line:extract_pfn}),
converted to a virtual address (Line \ref{line:p2vC}), and returned to the caller.
We can now discuss \lstinline|pte_get_next_table|'s proof of correctness.
While at first glance this code may look like its subtlety is mostly care to distinguish physical and virtual addresses,
it has a highly nontrivial correctness argument, which depends critically on detailed invariants on how access to page table
entries is shared between parts of the kernel. No prior work has engaged with this problem.
\looseness=-1

For this C presentation of what is really an assembly-level proof, we abuse notation and
use our register points-to for C-level program variables. So on Line \ref{line:precondition_entry_out},
$\mathsf{entry} \mapsto_r \mathsf{entryp+KERNBASE}$ means that the \emph{register representing} the C program variable
\lstinline|entry| (per the calling convention, \lstinline|rdi|) holds the sum on the right (a constant offset added to the physical address $\mathsf{entryp}$ of the entry).
That particular value is one subtlety of the proof related to the aforementioned kernel invariant, and is explained in Section \ref{sec:p2vC}.
The \emph{virtual pte-points-to} from that virtual address (Line \ref{line:get_next_vpte_preconditionC}) indicates that it points to a value
$\mathsf{entryv}$, a (possibly-unpopulated) page table entry.
A virtual pte-points-to is defined just like the normal virtual points-to of Figure \ref{fig:virtualpointstosharing},
except the physical address (\textsf{entryp} on Line \ref{line:get_next_vpte_preconditionC}) is explicit in the assertion
rather than existentially quantified:\\
\centerline{$
    \vaddr\mapsto_{\textsf{vpte,q}} \; \paddr \; \vpage : \mathsf{vProp}~\Sigma \stackrel{\triangle}{=} 
    \exists \delta\ldotp
	(\lambda \mathit{cr3val}\ldotp
	\ghostmaptoken{\delta{}s}{\mathit{cr3val}}{\delta}) \ast 
  \fracghostmaptoken{\delta}{\vaddr}{\paddr}{\qfrac} \ast \paddr \mapsto_{\mathsf{p}} \vpage
$}\\
This supports memory access rules much like Figure \ref{fig:wpdamd}'s rules (which are proven
sound using the virtual pte-points-to rules as lemmas!)
while exposing the physical location being modified.
This is useful for page table modifications, which require knowing the physical location being changed.
They are used throughout the software page table walk because entries in any level may be initialized.
\looseness=-1

\subsubsection{Address Space Invariant: Identity Mappings and Conditional Page Table Ownership}
\label{subsec:identitymappingsC}
Assembly-level verification of compiler output from Figure \ref{fig:calltopteinitializeC} is verbose, but largely
similar to other assembly-level verification thanks to Section \ref{sec:logic}'s logic (including virtual pte-points-to
assertions),
but only after resolving two critical challenges.
Two key challenges stand out and end up affecting both the pre- and post-conditions, neither of which has been addressed by prior work.
First, the update to the memory at (virtual) address \lstinline|entry| depends on subtle ownership invariants:
if the entry is present then its fractional ownership is shared with a large number of $\textsf{L}_{4}\_\textsf{L}_{1}\_\textsf{PointsTo}$ assertions
from the address space invariant (Figure \ref{fig:peraspaceinvariant}),
but if the entry is absent the proof requires full ownership to update it. We resolve this by extending the address space invariant
to make the owned fraction of the entry's memory \emph{dependent on its own contents}.
Second, the conversion of physical addresses into a corresponding virtual address that can be used to modify the specific
physical location relies on subtle, never-before-formalized kernel invariants.
% \looseness=-1
%
% These two factors percolate to the precondition (for conditional fractional ownership) and postcondition
% (for physical-to-virtual conversions), as the caller essentially passes output from one call to \lstinline|pte_get_next_table|
% as input to a subsequent call to traverse multiple levels.
Since the key to solving these challenges is to extend the address space invariant, we
first discuss that invariant and the kernel designs it supports, before returning to the subtle details of
verifying lines \ref{line:install_new_entryC} and \ref{line:p2vC}.
The key idea is to establish extra invariants on physical addresses that are part of a page table ---
but to do so in a way that meshes with the existing invariants (in the informal sense) already preserved in most
unverified kernel designs.
Each of the above problems requires its own extension to the invariant, but we will discuss
physical-to-virtual conversion first, both because it dictates the organization of the invariant
and because when Line \ref{line:check_entry_present_jumpC}'s
conditional check is false that is all that is necessary for the proof; correctness of the conditional branch deals with \emph{both}
extensions.

\subsubsection{Physical-to-Virtual Mappings and \textsf{P2V}}
\label{sec:p2vC}
Kernels need to convert between physical and virtual addresses, in both directions.
Traversing the page tables in software is the simplest way to convert a virtual address to a physical address;
this is the context we are working up to.
However, implementing this virtual-to-physical (V2P) translation in software ironically requires physical-to-virtual (P2V) translation,
because the addresses stored in page table entries are physical, but memory accesses issued by the OS code use virtual addresses.
Because VMM operations are performance-critical for many workloads, most kernels
maintain invariants that enable very fast P2V conversions (rather than adding another data structure).
Specifically, many kernels maintain an invariant on their page tables that the virtual address of any page used for a page table 
% lives at a virtual address whose value 
is \emph{a constant offset from the physical address} --- a practice sometimes referred to as \emph{identity mapping} 
(even though the physical-to-virtual translation
is typically not literally the identity function, but adding a nonzero constant offset).\footnote{Some kernels do this for all physical memory on the machine, simplifying interaction
with DMA devices.
On newer platforms like RISC-V, this sometimes truly is an identity mapping ---
x86-64 machines are forced into offsets by backward compatibility with bootloaders that cannot access the full memory space of the
machine.
}
Thus \lstinline|P2V| on line \ref{line:p2vC} of Figure \ref{fig:calltopteinitializeC} is a macro for adding the fixed constant \lstinline|KERNBASE|.

Figure \ref{fig:peraspaceinvariant_with_p2v_extensionC} extends the per-address-space invariant  to also track which
addresses we can perform a P2V conversion on by adding a constant offset (i.e., the set of physical addresses which participate in page tables).
$\Xi$ is another ghost map, from physical addresses to the level of the page table they represent (1--4).
\emph{Only} physical addresses in $\Xi$ can undergo P2V conversion. 
Section \ref{sec:p2vC} describes the verification of an actual conversion,
but this invariant must be \emph{established} when adding a new page table level (notably on Line \ref{line:call_to_pte_initializeC},
hence the comment of Line \ref{line:now_we_know}).

% Figure environment removed

For each $\paddr\mapsto \textsf{v} \in\Xi$, the invariant contains a virtual points-to justifying that virtual address
$\paddr+\textsf{KERNBASE}$ maps to physical address $\paddr$
(\textcircled{1} in Figure \ref{fig:peraspaceinvariant_with_p2v_extensionC});
fractional ownership of the physical memory for that page table entry (\textcircled{2}, which together with \textcircled{1} is equivalent
to a virtual points-to);
and for valid entries (with the present bit set) above L1, ghost map tokens for $\Xi$ for every entry in the table pointed to by the entry, which can be used
to repeat the process one level down (\textcircled{4}). 
% (L1 entries point to data pages, whose physical memory ownership resides in some virtual points-to).
\textcircled{4} becomes part of the precondition to \lstinline|pte_get_next_table|:
Line \ref{line:precondition_conditional}) says that if the entry is valid (points to a next-level table)
then there are tokens for accessing $\Xi$ for every entry in the next-level table.
By Line \ref{line:finalpieceS} the entry is guaranteed to be valid so all tokens for converting the next table level's physical addresses to virtual
are available (in the form expressed by the assertion on Line \ref{line:available_child_tokens}).
\looseness=-1

As noted above, for the code path where the conditional does not execute (there was already a valid entry), this is all we need of the new
invariant to verify the end of the function from Line \ref{line:finalpieceS} onward.
By that point the invariant holds and applies to the definitely-valid entry,
so we can the physical address of the next-level table to a corresponding virtual address via the identity mappings just described.
Line \ref{line:extract_pfn} simply retrieves the physical address.
Line \ref{line:p2vC} is the critical piece, and arguably corresponds to the most subtle verification of an \lstinline|add| instruction
(\lstinline|add rax, KERNBASE|)
that we are aware of, and something no prior work on verified OS kernels has dealt with.

After Line \ref{line:extract_pfn}, it is already known that the present bit is set in the entry;
Line \ref{line:childrenC}'s assertion reflects that the tokens for $\Xi$ exist for each word-aligned
physical address in the next-level table.
However, note that no argument to this function specifies which virtual address is being accessed,
so \lstinline|pte_get_next_table| does not know which entry in the next table to retrieve.
Even if that address were passed, this function is used for each step-down, so the slice of the
virtual address (per Figure \ref{fig:pagetables}) is not fixed.
Thus Line \ref{line:p2vC} computes the virtual address of the \emph{base} of the next-level table,
and the postcondition includes a renamed version of the assertion on line \ref{line:childrenC},
for the \emph{caller} --- \lstinline|walkpgdir| (discussed next) to perform the conversion for
The caller determines which entry in that table must actually
be accessed --- by selecting the appropriate index into the 512 ghost map tokens returned in the postcondition,
and using the ghost translation and physical location portions of the invariant to assemble a vpte-pointsto
that justifies the caller's subsequent access to a particular entry of the returned table.
The postcondition also passes back the per-address-space invariant with the
identity mapping resources for \lstinline|entry| still pulled out (it was removed by the caller).
\looseness=-1



\subsubsection{Self-Conditional Fractional Ownership and Installing a New Table}
\label{sec:selfconditional}
The fractional ownership of the entry's physical memory is subtle.
As noted above, a \emph{valid} entry must coexist with the fractional ownership from
$\textsf{L}_{4}\_\textsf{L}_{1}\_\textsf{PointsTo}$ and therefore have less than full ownership,
but in the case where the entry is \emph{invalid}, Line \ref{line:call_to_pte_initializeC} must have full permissions in order
to populate the entry (i.e., to install a reference to a next-level table).
Fortunately, the entry can only be in use if its valid bit is set; if the valid bit is not set, we know
that no virtual points-to entry in $\delta$ or $\theta$ holds any partial ownership.
But determining this requires reading the very memory whose ownership is being determined.
We use the invariant portion annotated as ``Entry validity'' (\textcircled{3}) in Figure \ref{fig:peraspaceinvariant_with_p2v_extensionC} to capture this:
if the entry is invalid the invariant holds full ownership of the entry, so it can be updated;
while if the entry is valid, the invariant owns only a constant nonzero fraction sufficient to read but not modify the entry.
Since the fractional ownership is always non-zero, Line \ref{line:read_entry_contentsC} in Figure \ref{fig:calltopteinitializeC} can read the entry
(using a rule similar to \textsc{WriteToRegFromVirtMem}, tailored to virtual PTE-points-to assertions),
and if the entry is dynamically found to be invalid, the invariant is refined (Line \ref{line:refined_fractional_ownership})
to indicate full ownership, allowing updates.
Note that the caller is responsible for providing this conditional ownership, having pulled it out of the invariant earlier.
This is why the precondition (Line \ref{line:precondition_entry_out}) explicitly excludes the entry's physical address from the invariant ($\Xi\setminus\{\mathsf{entry}\}$) ---
its relevant assertions have already been borrowed by the caller.
\looseness=-1

% Figure environment removed

If the entry is not set, \lstinline|pte_initialize|  
allocates a physical page for use as the next-level table.
\lstinline|pte_initialize| (Figure \ref{pteinitializespecC}) calls
\lstinline|kalloc| to allocate a physical page (Figure \ref{pteinitializespecC} Line \ref{line:call_to_kallocC}),
and installs it into the entry (Line \ref{line:kalloc_install}).
The page-allocator's \textsf{kalloc}
is the only unverified (trusted) code in our case study.\footnote{
  This is an allocator for regions of pre-zeroed physical memory that is mapped, but not accessed by the allocator itself,
  as is typical for slab allocators~\cite{bonwick1994slab}.
  Its verification would be similar to verifying a usermode \textsf{malloc} verification ~\cite{Chlipala2013Bedrock,wickerson2010explicit},
  just with additional invariants on the memory pool.
} 
Since we are using \textsf{pte\_initialize} for page-table address allocation, we must relate this newly
allocated physical address to the identity mapping map $\Xi$ --- 
see Line \ref{line:page_of_capsC} in Figure \ref{fig:calltopteinitializeC}, where
\texttt{kalloc}'s specification guarantees it has returned memory from a designated memory
pool that is already mapped
\footnote{A reasonable reader might wonder where this pool
initially comes from and how it might grow when needed. Typically an initial mapping subject to this identity mapping
constraint is set up prior to transition to 64-bit kernel code (notably,
a page table must exist \emph{before} virtual memory is enabled during boot, as part of enabling it is setting
a page table root).
Growing this pool later requires cooperation of physical memory range allocation and virtual memory range allocation,
typically by starting general virtual address allocation at the highest physical memory address plus the identity mapping offset.
This reserves the virtual addresses corresponding to all physical addresses plus the offset for later use in this pool,
as needed.
} 
and satisfies the offset invariants (trivially, as the new page is zeroed).
The presence bit of the entry is \emph{not} set during \lstinline|pte_initialize|, but upon
return to \lstinline|pte_get_next_table|, where it will validate the conditional ownership discussed above.
\lstinline|pte_initialize| has a full-permission virtual pte-pointsto in its precondition.
Then the assertions that hold after Line \ref{line:now_we_know} of Figure \ref{fig:calltopteinitializeC}
are enough to establish the same page table invariants which hold in the case where the entry was already valid,
by updating the current address space's entry.



\subsubsection{The Specification of \lstinline|pte_get_next_table|}
Note that the specification does \emph{not} assume a specific page table level and is used
for all three level transitions (4 to 3, 3 to 2, 2 to 1).
The logical parameter \textsf{v} represents the level
of the entry passed as an argument (c.f. the token $\ghostmaptoken{\textsf{id}}{(\mathsf{entryp})}{\textsf{v}}$ witnessing
that \textsf{entryp} is part of the page table invariant on Line \ref{line:precondition_entry_out}).
This comes into play with a key subtlety of \lstinline|pte_get_next_table|'s
specification: its precondition
includes a virtual {pte-points-to} (discussed earlier, Line \ref{line:get_next_vpte_preconditionC})
but its postcondition does not yield new virtual points-to assertions!
It merely computes the base virtual address of the next table, and returns adequate tokens
% (discussed in Section \ref{subsec:identitymappingsC}, explicit on
(Line \ref{line:childrenC})
for the \emph{caller} to construct a vpte-points-to for any entry of the next table level.
\looseness=-1

\subsubsection{Walking The Page Tables: Calling \textsf{pte\_get\_next\_table} for Each Level}
\label{wlkpgdirC}
% Figure environment removed

% Figure environment removed

Implementing a software page table walk amounts to calling \textsf{pte\_get\_next\_table} for each level as shown in Figure \ref{walkpgdirC}. 
\lstinline|walkpgdir| traverses the page table anchored at \lstinline|l4| (the virtual address of the base of the L4 table)
and returns the virtual address of the L1 entry that should map the virtual address \lstinline|va|.
For each level, \lstinline|walkpgdir| locates the appropriate entry by using the level-specific slice of \lstinline|va| to index into
that table (simulating the hardware translation as in Figure \ref{fig:pagetables}), and passes the virtual address of that entry to
\lstinline|pte_get_next_table| to get the base of the next level down.
For example, Line \ref{line:start_l4_calcC} uses \lstinline|L4Offset| (a bit shifting and masking macro) to extract bits 39--47 of \lstinline|va|,
and uses that to find the address of the L4 entry that would map \lstinline|va| in the address space.
That is then passed to \lstinline|pte_get_next_table| on Line \ref{line:first_getnext_callC}, which
returns the virtual address of the base of the L3 table. This process repeats for 3-to-2 (Lines \ref{line:l3offset}--\ref{line:getnextl3}),
and 2-to-1 (Lines \ref{line:l2offset}--\ref{line:getnextl2}), after which Line \ref{line:returnl1entryaddr} returns
the virtual address of the appropriate L1 entry.

In Figures \ref{walkpgdirC} and \ref{fig:rwalkC}, there are four related concepts for each level.
\textsf{l4p} is the physical address of the L4 table base; \textsf{l4} is the corresponding virtual address
(using the same name for the value and the program variable name for brevity, since the variable is not reassigned);
\textsf{l4\_entry} is the virtual address of the L4 entry used to translate \textsf{va};
and \textsf{l4e\_val} is the value of that table entry.
Other levels are named consistently.
For each of the three level transitions, the main challenges for the proof are to
construct a virtual pte-points-to assertion for the entry in that level's table,
and pass the conditional assertion discussed in Section \ref{sec:p2vC} that if that entry is present then there are
identity map tokens for the the physical address of each entry in the subsequent level's table.
For traversing the L4 table, this proceeds by
exchanging the relevant identity map token provided in the precondition (Line \ref{line:walkpgdir_pre})
and pulling the resources for physical address $\mathsf{l4p}+8*\textsf{L4Offset}(\textsf{va})$
out of the identity map invariant: 
parts \textcircled{1} and \textcircled{2} of Figure \ref{fig:peraspaceinvariant_with_p2v_extensionC}) give a virtual pte-points-to,
and \textcircled{3} and \textcircled{4} satisfy other parts of \lstinline|pte_get_next_table|'s precondition.
This justifies the call on Line \ref{line:first_getnext_callC},
which returns the virtual address of the base of the appropriate L3 table and whose postcondition
includes 512 identity map tokens for each of those entries (anchored to the returned virtuall address minus \textsf{KERNBASE}).
That invariant (Line \ref{line:l3tokens}) is analogous to the one that justified the 4-3 step (Line \ref{line:walkpgdir_pre}), and the next two steps proceed the same way.

Even given the slight adaptation of our assembly-level proof for the C-level presentation in Figure \ref{walkpgdirC},
the proof outline in the figure omits some repeat intermediate assertions for readability.
But by Line \ref{line:ex_l3_vpteC}, it should be clear that the proof accumulates a set of similar assertions for each level.
Figure \ref{fig:rwalkC} expands the abbreviated postcondition to the full set of facts that are accumulated in this way.
$\mathsf{R}_\mathsf{walk}$ together with $\mathsf{R}_{\mathsf{l1e}}$
nearly entail \lstinline|L4_L1_PointsTo| (Figure \ref{fig:strongvirtualpointsto}) within the logic,
forming the basis of the construction of a new virtual points-to for virtual address \textsf{va}.
\lstinline|walkpgdir|'s execution observes most evidence of an address translation for \lstinline|va|,
at least down to a possibly-invalid L1 entry (which \lstinline|walkpgdir|'s caller, \lstinline|vaspace_mappage|, will check). 
Each virtual pte-points-to in $\mathsf{R}_\mathsf{walk}$ internally contains the physical points-to portion of one page table walk step for \lstinline|L4_L1_PointsTo|,
and the pure assertions in $\mathsf{R}_\mathsf{walk}$ ensure the address arithmetic works.
$\mathsf{R}_{\mathsf{l1e}}$ includes the self-conditional fractional ownership of the L1 entry (Section \ref{sec:selfconditional})
for the caller to initialize the entry if it is empty, so the caller can complete a virtual points-to assertions
for a newly-mapped page, as we discuss next.
\looseness=-1


% The key part of the specification and proof for a page table walk is accumulation of memory mappings for the page-table entries 
% visited and frame addresses for page-tables. 
% For example, Lines \ref{line:ex_l4_vpteC} and \ref{line:ex_l3_vpteC} in Figure \ref{walkpgdirC} show the virtual pte-pointsto assertions for L4 and L3 entries.
% In the final post-condition, we expect the accumulation of these resources from each level -- $\textsf{R}_{\textsf{walk}}$ -- 
% which allows us to construct and return the path to the L1 entry in the tree to insert a new page.  

% This is the code which performs most actual physical-to-virtual conversions using the identity mapping portion of the per-address-space invariant.
% \lstinline|walkpgdir| accepts a \emph{virtual} pointer to the base of the L4 table, and the address to translate.
% The precondition provides knowledge that the virtual base of the L4 is at the appropriate offset from the current \lstinline|cr3| value,
% but does not provide a virtual points-to assertion --- because the function must calculate (Lines \ref{line:start_l4_calcC}--\ref{line:end_l4_calcC})
% which entry it needs access.
% Instead, the precondition has 512 identity map tokens, guaranteeing that every entry on the page is subject to the identity mapping invariant.
% Line \ref{line:end_l4_calcC} calculates the virtual address of the relevant entry, and the subsequent view shift
% pulls that entry out of the identity mapping ($\Xi$) and fetches its corresponding resources as
% described by Figure \ref{fig:peraspaceinvariant_with_p2v_extensionC} and Section \ref{subsec:identitymappingsC}.
% The ghost translation and physical location are used to form the virtual pte-pointsto for the L4 entry
% (Line \ref{line:first_pte_pointstoC}), with entry validity and next-level indexing
% satisfying the rest of the precondition for \lstinline|pte_get_next_table|.
% Then, as described earlier, checks the valid bit in the indicated
% entry and either returns the (unconditional) tokens for the L3 entry physical addresses (if valid), or
% allocates into the entry and returns new (also unconditional) tokens for the L3 entry physical addresses.
% \lstinline|pte_get_next_table|'s first call (Line \ref{line:first_getnext_callC}) returns
% the virtual address of the base of the L3 table. Then the situation to move from that pointer to the base of the L2
% is just like the process just followed: the proof calculates the address of the relevant
% L3 entry, uses the appropriate L3 identity mapping token to construct a virtual pte-pointsto to that entry,
% and passes that along with additional resources pulled out of the invariant to another call to
% \lstinline|pte_get_next_table|. That call then returns the base of an L2 table, and the process
% repeats until the function returns the virtual address of the relevant L1 entry.
% That will then be used in the next section by the caller of \lstinline|walkpgdir|
% to install a new mapping.


% \textsf{walkpgdir}, as a client, holds the knowledge that there exists an identity mapping for the physical entry address (\textsf{entry})
%  in the root page table ($\textsf{L}_{4}$):  $\mathsf{entry} \mapsto_{\textsf{id}} \textsf{\_}$ in Specification Line 3 is a partially owned
%  token for accessing and looking up the resources in the identity map, $\Xi$, to construct the \textit{virtual-to-physical} pointsto relation 
% $\textsf{entry+KERNBASE} \mapsto_{\textsf{vpte,qfrac}} \textsf{entry \entry\_val}$ with the virtual address (\textsf{entry+KERNBASE}) obtained 
% by offsetting the physical address (\textsf{entry}). With this knowledge on the root-page-table-entry, we can start traversing the page-table 
% tree which requires locating the address of the next table -- a call to \textsf{pte\_get\_next\_table} shown in Figure \ref{fig:calltopteinitialize}. 
% Beyond a frame, the precondition before Line 15 requires the current address space invariant, and knowledge that \textsf{entry} is mapped to a 
% random entry value, subtly, 
% the operation also, at least, requires that the relevant table entry is readable, but the exact portion of ownership 
% returned must be determined by inspecting the valid bit
% of the value in memory --- so full ownership is returned only for unused entries.
% This is a simple piece of code whose functionality is critical and whose correctness is highly non-trivial. No prior work engages with this problem.



%% Figure environment removed



%\caption{Traversing page-tables, and allocating entries as needed while mapping-a-page in Figure \ref{fig:mappingcode}.}
% \citet{kolanski08vstte,kolanski09tphols} verified a single code block with their logic which was roughly Figure \ref{fig:mapping_code} for a 2-level ARM
% page table, but several critical complexities our work deals with were not addressed.
% First, beyond the limitations discussed in Section \ref{sec:overly-restrictive}, Kolanski and Klein assumed that virtual addresses
% for page tables at each level were given as parameters rather than verifying any conversion from physical addresses to virtual addresses (or even axiomatizing their lookup).
% In contrast, our verification articulates the address space invariant from which the physical-to-virtual translation can be implemented.
% Second, our proof deals with the construction of a valid virtual points-to \emph{to the PTE to update} in mapping, which Kolanski and Klein also
% assumed was given.
% \todo{some of this is really an argument for our verification being more thorough, rather than being about our logic}

% Reasoning about the page table walk in their logic would have required 
% could reason about the walk, but would need to explicitly prove that all other invariants
% of the kernel, the current address space, and all other address spaces of interest were preserved by each update, because their model
% only supports separation within a single address space. In our model, this follows for free from making
% our separation logic directly aware of address translation and internalizing assumptions about other address spaces as further separable assertions.
% Kolanski and Klein did address part of the walk information for a 2-level page table (a possible ARM configuration), but 

% \textsc{seL4} currently still trusts address translations; it models page tables as a data structure in regular memory, thus not capturing the possibility that even
% temporarily destroying the mappings and restoring them can actually crash the OS. \textsc{CertiKOS} papers share little in the way of precise details about
% their virtual memory management, but because their core technology is based on a fork of \textsc{CompCert}, whose model of memory is
% a set of unordered block allocations, we can infer their proofs must also trust these translations.


\subsection{Mapping a New Page}
\label{sec:mapnewC}
Finally we come to the top-level routine for mapping a new page of memory into an address space
by updating page tables --- the \lstinline|vaspace_mappage| function in Figure \ref{fig:mapping_codeC}.
Again, our verification was carried out at the assembly level, but presented as on the original C for readability,
and the proof outline omits all relevant facts in favor of the most critial assertions involved in the key parts of the proof.
\lstinline|vaspace_mappage| is typically called by a page fault handler, to map a previously-reserved but lazily-allocated page.
It is passed the virtual address of the L4 table base (\lstinline|l4|), the virtual address to map (\lstinline|va|),
and the \emph{physical} address of an empty memory page which should be used as backing memory for \lstinline|va| and its surrounding page.
It begins by calling \lstinline|walkpgdir| (Line \ref{line:call_walkpgdirC}) to return the virtual address of 
the L1 entry which corresponds to \lstinline|va| (allocating intermediate tables as needed).
It then checks if the entry is already initialized. If not, \lstinline|fpaddr| is installed
into the L1 entry, which is then marked valid (setting the present bit), and the page is mapped.
% To do so, with a given allocated fresh page (\textsf{fpaddr}), then calculate the appropriate
% known-valid page table walks (via \textsf{walkpgdir} Line \ref{line:call_walkpgdirC} in Figure \ref{fig:mapping_codeC}) and update
% the appropriate L1 page table entry (Line 35 in Figure \ref{fig:mapping_codeC});
Unmapping is the reverse of the logic we discuss here.
\looseness=-1
%\lstset{
%  columns=fullflexible,
%  numbers=left,
%  basicstyle=\ttfamily,
%  keywordstyle=\color{blue}\bfseries,
%  morekeywords={mov,add,call},
%  emph={rsp,rdx,rax,rbx,rbp,rsi,rdi,rcx,r8,r9,r10,r11,r12,r13,r14,r15},
%  emphstyle=\color{green},
%  emph={[2]cr3},
%  emphstyle={[2]\color{violet}},
%  morecomment=[l]{;;},
%  mathescape
%}
% Figure environment removed


For brevity this example is specialized to the case where $\vaddr$ is known to not be mapped:
the precondition on Line \ref{line:mappage_pre} includes $\theta \; !!\; \vaddr = \texttt{None}$;
generalization to returning an error if it is already mapped is straightforward.
% .\footnote{
%   The generalization to handling either case and returning an error is straightforward.
% }
The precondition on Line \ref{line:mappage_pre}, directly entails
the precondition of the \lstinline|walkpgdir| call.\footnote{The proof of \lstinline|vaspace_mappage|'s caller
would extract this single identity map token for the specific L4 entry from a set of 512 that are
part of the kernel invariant, as \lstinline|walkpgdir|'s proof does for lower levels.
}
\lstinline|walkpgdir|, as just discussed,
returns the virtual address of an allocated L1 entry and its postcondition contains almost all of the information
needed to construct a virtual points-to for \lstinline|va| --- except information about the L1 entry
being present and pointing to a data page.
We already discussed for the upper level page-tables how the entry-present checks are handled, and
Line \ref{line:mappage_pte_present_startC} is similar: $\mathsf{R}_{\textsf{l1e}}$ includes the self-conditional
fractional permission for the L1 entry, so as it is not present, by Line \ref{line:l1entry_storeC}
it is known that full permission is held to update that entry.
Lines \ref{line:l1entry_storeC} and \ref{line:l1entry_setpresent} are verified using the pte-points-to
memory store rule. In the assembly proof, this is a bitwise-and of the word-aligned \lstinline|fpaddr| with 1
(setting the present flag), yielding a single store.
\looseness=-1

% By incorporating verification of the
% \lstinline|ensure_L1| function (see Section \ref{sec:traversing}), our verification also directly handles several subtle aspects which
% were axiomatized in prior work.
\ifPLDI
\else
\subsection{Unmapping a Page}
\todo[inline]{update (esp. line refs) for new mapping code}
The reverse operation, unmapping a designated page that is currently mapped,
would essentially be the reverse of
the reasoning around line 22 above: given the virtual points-to assertions for all 512
machine words of memory that the L1 entry would map,
and information about the physical location, 
full permission on the L1 entry could be obtained, allowing the construction of a
full virtual PTE pointer for it, setting to 0, and reclaiming the now unmapped physical memory.
\fi


%\section{Discussion}
\label{sec: discussion}
\kmsdelete{In this work} We study \kmsreplace{Fairness-Aware PAC learning}{Fair-ERM} in the malicious noise model, and  in some cases allow 
the learner to maintain optimal overall accuracy despite the signal in Group $B$ being almost entirely washed out.
%when we allow learners to use the
%$\PQ$ randomized expansion of the hypothesis class $\mathcal{H}$
In particular we show that different fairness constraints have fundamentally different behavior in the presence of Malicious Noise, in terms of the amount of accuracy loss that a given level of Malicious Noise could cause a fairness-constrained learner to incur. 
The key to achieving our results, which are more optimistic than those in \cite{lampert}, is allowing for improper learners using the (P,Q)-randomized expansions of the given class $\mathcal{H}$.
%We \kmsreplace{present a picture of the}{prove upper and lower bounds on}
%accuracy loss for a range of fairness notions, given \kmsreplace{this simple randomization step.}{learning over $\PQ$.
%In general our results indicate Fair-ERM (given learning over $\PQ$) is more robust than claimed in \cite{lampert}.
The type of smoothness we create by using $\PQ$ seems to be a natural property that is likely shared by many natural hypothesis classes.

Fairness notions are motivated as a response to learned disparities when there is \kmsdelete{data corruption or} systemic error affecting \kmsdelete{the data for}
one group. 
Fairness notions are supposed to mitigate this by ruling out classifiers that have worse performance on a sub-group. 
This can peg both classifiers at a lower level of performance \kmsdelete{(e.g that the lower subgroup)} in order to \emph{motivate} \cite{hardt16} improving the data collection or labelling process to obtain more reliable performance. 
%So in \kmsreplace{some}{a} sense, sensitivity of the fairness notion to poor sub-group performance caused by malicious noise is the \textit{point} of fairness constraints! 
However, it also desirable that fairness constraints perform gracefully when subject to Malicious Noise because fairness constraints will be used in contexts where the data is unreliable and noisy and this might not be known to the learner.
This tension, exposed by our work, motivates 
%a revisiting of fairness notions from first principles approach and trying to axiomatize the 
%desired properties of a fairness intervention a la cryptography and privacy. \footnote{Work in multi-calibration \cite{multicalib} is a viable direction for this problem but it is unclear how 
%that and related notions behave with unreliable data. }
on going work studying the sensitivity level of fairness constraints. 
%If we we are to take a view, if a classifier is deployed 
 % not necessary
\section{Related Work}
\label{sec:relwork}

There has been relatively little prior work on formal verification of virtual memory.
Instead, much OS verification work has focused on minimizing reasoning about virtual memory management.
The original Verisoft project~\cite{alkassar2008verisoft,alkassar2010pervasive,alkassar2008formal,dalinger2005verification,hillebrand2005address,alkassar2008formal,starostin2010formal} relied on custom hardware which, among other things, always ran kernel code with virtual memory disabled, removing the circularity that is a key challenge of verifying actual virtual memory code: at that point page tables become a basic partial map data structure to represent user program address translations.
Other work on OS verification either never progressed far enough to address VMM verification (Verisoft XT~\cite{cohen2009vcc,cohen2010local,dahlweid2009vcc,cohen2013SOFSEM}), or uses memory-safe languages to enable safe co-habitation of a single address space by all processes (Singularity~\cite{Fahndrich2006language,Hunt2007singularity,Hunt2007sealing,Barnett2011specsharp}, Verve~\cite{Yang2010Verve}, and Tock~\cite{levy2017multiprogramming}).

The work that does address the core challenges of VMM verification is all associated with either \textsc{seL4} or \textsc{CertiKOS}.

\textsc{CertiKOS}~\cite{gu15,gu2016certikos,gu2018certikos,chen2016interrupts} is a microkernel intended for use as a hypervisor,
and its papers do not explicitly detail verification of the VMM, so we do not know the full space of which VMM functionality 
is verified, but we do know it includes the ability to map or unmap pages.
The work is clear, however, that it trusts low-level assembly fragments such as the instruction sequence which actually
switches address spaces, rather than verifying them.
The overall approach in that body of work is many layers of refinement proofs, using a
 proliferation of layers with small differences to keep most individual refinements tractable. In keeping with precursor work 
on the project from the same group~\cite{vaynberg2012compositional}, the purpose of some layers is to abstract away from 
virtual memory, so the proof is essentially a simulation proof covering for example a proof that execution with page-in on 
page faults is a valid refinement of an execution model where no paging occurs.
% Another key aspect of their approach is that the OS is written in Clight and compiled with \textsc{CompCert}~\cite{blazy2006formal,leroy2009formally,leroy2008formal}.
% CompCert's memory abstraction~\cite{leroy2008formal} assumes
% memory is a set of disjoint chunks of bytes with no overlap, so the lowest levels of CertiKOS must provide a matching 
% machine model as a layer. This prohibits virtual address aliasing, so CertiKOS cannot support simultaneous memory-mapped 
% (\texttt{mmap}) and stream-oriented (\texttt{read}/\texttt{write}) IO to a single file\todo{should we go into this detail?}, 
% and cannot use
% the common kernel design choice of mapping all physical memory into the bottom of the kernel's address space for direct access i
% while the kernel code is simultaneously mapped (and executed) at higher virtual addresses.
% This is not necessary for \textsc{CertiKOS}'s intended primary use case (a hypervisor), but means that \textsc{CertiKOS}'s
% approach cannot be used to support this functionality in other systems, without major surgery to \textsc{CompCert}.

\textsc{seL4}~\cite{Klein2009seL4,seL4TOCS,Sewell2013translation} is a formally verified L4 microkernel~\cite{Liedtke1995,Liedtke1996} (and the first verified OS kernel to run on real-world hardware), verified with a mix of refinement proofs and program logic reasoning down to the assembly level.
Because \textsc{seL4} is a microkernel, most VMM functionality actually lives in usermode and is unverified, and moreover, their hardware model omits address translation entirely and the MMU entirely~\cite{Klein2009seL4,seL4TOCS}. As a result, the limited page table management present in the microkernel treats page tables as idiosyncratic tree-maps, ignoring the risks posed by even transient inconsistencies that would crash the kernel on real hardware (like ``temporarily'' unmapping the kernel). This is mitigated primarily by manually identifying some trusted invariants (e.g., that the address range designated for the kernel is appropriately mapped) and setting up the proof to ensure those invariants are maintained (i.e., as an extra proof obligation not required by their hardware model).


One important outgrowth of the \textsc{seL4} project, not integrated into the main project's proof, was work by 
Kolanski and Klein which studied verification of code against a hardware model that \emph{did} include address translation
 --- the only work aside from ours to do so --- initially in terms of basic memory~\cite{kolanski08vstte} and subsequently 
integrating source-level types into the interpretation~\cite{kolanski09tphols}. 
They were the first work to model physical and virtual points-to assertions separately, defining virtual points-to assertions
in terms of physical points-to assertions mimicking page table walks, and defining all of their assertions as predicates on a
pair of (physical) machine memory and a page table root, an approach we improve on.

Their work has a number of significant limitations which our work addresses.
They also define their virtual points-to assertions such that a virtual points-to $p\mapsto_\mathsf{v} a$ owns the full 
lookup path to virtual address $p$. This means that given two virtual points-to assertions at the same time, such as 
$p\mapsto_\mathsf{v}a \ast p'\mapsto_\mathsf{v}b$, the memory locations traversed to translate $p$ and $p'$ must be disjoint. 
This means the logic has a peculiar limit on how many virtual points-to assertions can coexist in a proof. Since page tables 
fan out, the bottleneck is the number of entries in the root table. For their 32-bit ARMv6 example, the top-level address is 
still 4Kb (4096 bytes), and each entry (consumed entirely by a virtual points-to in their scheme) is 4 bytes, so they have a 
maximum of 1024 virtual points-tos in their ARMv6 configuration. Any assertion which implies more than that number
of virtual addresses are mapped implies false in their logic.
(They do formulate their logic over an abstract model, but every architecture would incur a similar limitation;
Na\"ively transferring their model to x86-64 4-level tables would yield a limit of 512 assertions (also a 4Kb root page, 
but 8-byte entries).

% Kolanski and Klein's points-to assertions do not model that page table entries for nearby addresses typically 
% \emph{share} entries in higher layers of the page tables --- a single L1 entry maps 4KB of memory on many architectures,
% but their logics avoid fractional ownership, so they can in fact only use a single memory location per page of memory.
% In fact, because their virtual points-to assertions contain \emph{full} ownership of all entries, even in the highest-level
% page table (L4 in our case), each entry can contribute to only a single mapped address. Thus any assertion
% in their logic that implies there are more virtual addresses mapped than entries in the top-level page table implies false.
Our definitions make use of fractional permissions throughout; Figure \ref{fig:strongvirtualpointsto}'s definition
of \lstinline|L4_L1_PointsTo| ellides the specific fractions used, but it in fact asserts 1/512 ownership of
the L1 entry, 1/($512^2$) of the L2 entry, and so on, so each entry may map the appropriate number of machine words.

As noted earlier, Kolanski and Klein's logics, by collocating both the physical ownership of the page table walk
as part of the virtual points-to itself, preempt support for changes to page tables which do not actually affect 
address translation.

The other major distinction is that Kolanski and Klein have no accounting for other address spaces.
Their logic does not deal with change of address space, and has no way to assert that certain facts hold
in another address space.
They verify only one address space manipulation: mapping a single unmapped page into the current address space (in both papers).
We verify this, as well as a change-of-address-space, which requires us to introduce assertions for talking
about other address spaces (we must know, for example, that the precondition of the code after the change must be true
in the \emph{other} address space), and to deal with the fact that the standard frame rule
for separation logic is unsound in the presence of address space changes and address-space-contingent assertions.
% The mapping write is verified in an ad hoc way by unfolding the machine semantics, because the logic lacks proper reusable rules for
% updating page tables.

Our approach in this paper uses modalities to distinguish virtual-address-based assertions that hold only in specific 
address spaces, making it possible to manipulate other address spaces, and equally critically, to \emph{change} address 
spaces while reasoning about correctness. 

Unlike our work, Kolanski and Klein prove very useful embedding theorems stating that code that does not modify page table 
entries can be verified in a VM-ignorant program logic, and that proofs in that logic can be embedded into the VM-aware logic 
(essentially by interpreting ``normal'' points-to relations as virtual points-to facts). While we have not proven such a result,
an analagous result {should} hold of our work: consider that the doubles for the \texttt{mov} instructions
that access memory behave just as one would expect for a VM-ignorant logic~\cite{Chlipala2013Bedrock}.
With our general approach to virtual points-to assertions being inspired by Kolanski and Klein, \emph{both}
 our approach and theirs could in principle be extended to account for pageable points-to assertions by adding additional 
disjunctions to an extended points-to definition; embedding ``regular'' separation logic into such a variant
is the appropriate next step to extend reasoning to usermode programs running with a kernel that may demand-page the program's
memory.

As noted throughout the paper, the inspiration for our other-space modality comes from hybrid logic~\cite{areces2001hybrid,blackburn1995hybrid,gargov1993modal,goranko1996hierarchies},
where modalities are indexed by \emph{nominals} which are names for specific individual states in a Kripke model.
We are aware of only two prior works combining hybrid logics with program logics specifically. 
Brotherston and Villard~\cite{brotherston2014parametric} demonstrated that may properties true of various 
separation logics are not definable in boolean \BI (\BBI), and showed that a hybrid extension \HyBBI allows
most such properties to be defined (e.g., the fact that separating conjunction is cancellative is unprovable 
in boolean \BI, but provable in \HyBBI). There, nominals named resources 
(roughly, but not exactly, heap fragments). 
Gordon~\cite{gordon2019modal} described a use of hybrid logic in the verification of actor programs, 
where nominals named the local state of individual actors (with such assertions stabilized with a 
rely/guarantee approach). Beyond these, there is limited work on the interaction of specifically 
\emph{hybrid logic} with substructural logics. 
Primarily there is a line of work on hybrid linear logic (\HyLL)~\cite{despeyroux2014hybrid}, 
originally used as a way to more conveniently express aspects of transition systems in linear logic. 
However, \HyLL's proof rules offer no non-trivial interactions with multiplicative connectives 
(every \HyLL proof can in fact be embedded into regular linear logic~\cite{chaudhuri2019hybrid}, 
unlike Brotherston and Villard's \HyBBI, which demonstrably increases expressive power over its base \BBI.

In both \HyLL and \HyBBI, nominals denote worlds with monoidal structure (as worlds in Kripke semantics
for either LL or \BBI necessarily have monoidal structure). Our nominals, by contrast, 
do not name worlds in the same sense with respect to Iris's CMRAs, 
but in fact \emph{classes} of worlds, because the names are locations 
(a means of \emph{selecting} resources) rather than resources.  
A key difference is that the use of nominals in those logics corresponds specifically to hypothetical 
reasoning about resources (until a nominal is connected to a current resource, in which case conclusions 
can be drawn about the current resource), which means the modalities themselves do not ``own'' resources. 
Instead, assertions under our other-space modality can and do
have resource footprints.
Pleasantly, we sidestep most of the metatheoretical complexity of those other substructural hybrid
systems by building our logic within a substructural metatheory (\iris).

\iris has been used to build other logics through pointwise lifting, notably logics that deal with weak
memory models~\cite{dang2019rustbelt,dang2022compass}. Those systems build a derived logic
whose lifting consists of functions from thread-local views of events (an operationalization of the release-acquire + nonatomic
portion of the repaired C11 memory model~\cite{lahav2017repairing}): there modalities $\Delta_\pi(P)$ and $\nabla_\pi(P)$
represent that $P$ held before or will hold after certain memory fence operations by thread $\pi$.
The definitions of those specific modalities existentially quantify over other views, related to the ``current'' view (the one where
the current thread's assertions are evaluated), and evaluate $P$ with respect to those other views. This approach to parameterizing
assertion semantics by a point of evaluation, and evaluating modalized assertions at other points, is what it means
to have a modality at all.
It is \emph{not}, however, an instance of hybrid logic, which is specifically demarcated by an assertion language where
\emph{assertions}, not their semantics, choose and name the evaluation points for modal assertions.
A hybrid extension of the aforementioned logics would include assertions which named specific views at which to evaluate
$P$, in the syntax of the assertion (e.g., $\Delta_\pi^v(P):=\lambda\_\ldotp (P\;v)$) rather than the 
$\Delta_\pi(P):= \lambda v\ldotp (\exists v_{rel}\ldotp \ownGhost{\pi}{\mathsf{RelV}(v_{rel})\;v} \ast (P\;v_{rel})))$ actually used.
Note the hybrid version takes the place to evaluate $P$ as a parameter, and therefore allows the \emph{derived} (modal) logic to explicitly
reason in terms of evaluation points, rather than hiding all points of evaluation in the internal definitions of modalities.



\section{Conclusions}
This paper advances the state of the art in formal verification of programs manipulating virtual memory mappings.
We treat assertions about virtual memory locations explicitly as assertions in a modal logic, where the notion of context
is a particular address space, named by the page table root.
We improved the modularity of our reasoning about virtual address translation and virtual points-to assertions
to permit page table modifications that
preserve mappings without collecting all affected virtual points-to assertions.
To specify of code involving other address spaces, we adapt 
hybrid logic's notion of \add{modalities} explicitly naming alternative conditions.
We implemented these ideas in a derived separation logic within \iris, and proved soundness of
the rules for essential memory- and address-space-change-related x86-64 instructions 
against a hardware model of 64-bit 4-level address translation.
Finally, we used our rules to verify the correctness of key VMM instruction sequences,
including the first assembly-level proof of correctness for a change
of address space expressing which assertions hold in which address space, 
the first physical-to-virtual translation proof,
\add{and the first verification of a software page table walk, all of which
are beyond reach of prior work}.
\looseness=-1

\begin{acks}
This work was supported in part by US NSF Award \#CCF-1844964.
\end{acks}


% In the future we plan to 
% extend our work in multiple directions, such as generalizing our assertions and proof rules
% to support large-page functionality common to virtual memory subsystems (superpages, hugepages),
% implementing 

% This is required for OOPSLA 2025
% \section*{Data Availability Statement}
% This work is backed by a substantial \rocq development, including
% (1) formalization of a subset of supervisor-mode x86-64 semantics;
% (2) specification and soundness proofs for the rules discussed in Section \ref{sec:logic}
% as well as additional rules covering arithmetic and bitwise operations,
% and fixed-offset variants of the memory access rules from Figure \ref{fig:wpdamd};
% and (3) proofs of the examples of Section \ref{sec:experiment}.

% The main limitation is \replace{the use of several axioms in the development, which are are currently
% in the process of removing. A}{that, a}s noted in Section \ref{sec:soundness}, control transfer instruction
% specifications are trusted (though again, entirely standard for assembly-level program logics~\cite{ni2007contexts,Ni2006codeptrs},
% aside from using our virtual points-to assertions in place of points-to assertions that assume flat memory).
% We also do not explicitly model instruction
% encoding and decoding, which is instead handled by an axiomatization of the relationship between
% byte sequences and instruction sequences, used in the definition of virtual code points-to
% assertions (i.e., assertions stating that a certain virtual address stores bytes
% encoding a valid instruction sequence with a given precondition, used for branches, call, and return instruction
% specifications). 
% \del{There are also a small number of axioms for a few bit-twiddling equivalences which are proving difficult to discharge
% with our XCAP-derived model of machine words.}

% % There are two main limitations.
% % The first is that our formalization does not explicitly model instruction
% % encoding and decoding, which is instead handled by an axiomatization of the relationship between
% % byte sequences and instruction sequences, used in the definition of virtual code points-to
% % assertions (i.e., assertions stating that a certain virtual address stores bytes
% % encoding a valid instruction sequence with a given precondition, used for branches, call, and return instruction
% % specifications). 
% % The second, as noted in Section \ref{sec:soundness}, is that control transfer
% % specifications are trusted (though again, entirely standard for assembly-level program logics~\cite{ni2007contexts,Ni2006codeptrs},
% % aside from using our virtual points-to assertions in place of points-to assertions that assume flat memory).
% \replace{We intend to submit this to artifact evaluation.
% We have attached a zip as supplementary material to the submission, though it does not
% currently include
% a guide for artifact evaluators.}{
%   Unfortunately we missed the line in the Major Revision notification that stated 
%   that we could submit an artifact this cycle. The notification also pointed to the call for papers
%   which still explicity states that Major Revisions were not allowed to submit artifacts.
%   If we are able to submit to a subsequent artifact evaluation cycle we will.
%   In either case, we will upload our artifact to Zenodo to obtain a DOI and archival storage,
%   and will cite it from the final version.
% }

%% The next two lines define the bibliography style to be used, and
%% the bibliography file.
\bibliographystyle{ACM-Reference-Format}
\bibliography{vmm}

%\ifARXIV
\appendix
\section{\add{Assembly-level Verification for Page Table Traversal and Mapping}}
\label{sec:experiment_appendix}
%To both validate and demonstrate the value of the modal approach to reasoning about virtual memory management, 
% we study several
% We validate our logic by studying
% distillations of key VMM functionality.
% real concerns of virtual memory managers.
% Recall from Section \ref{sec:logic} that virtual points-to assertions work just like regular points-to assertions, by design.

\replace{
In this section we verify several critical and challenging pieces of VMM code.
First, in several stages, we work up to mapping a new page in the current address space.
This requires a number of independently challenging substeps: dynamically traversing a page table to find
the appropriate L1 entry to update; inserting additional levels of the page table if necessary (updating
the VMM invariants along the way);
converting the physical addresses found in intermediate entries into the corresponding virtual addresses
that can be used for memory access;
installing the new mapping;
and collecting sufficient resources to form a virtual points-to assertion.
Of these, only the second-to-last step (installing the correct mapping into the
current address space) has previously been formally verified with respect to a machine model with address translation.
Second, we formally verify a switch into a new address space as part of a task switch,
the first such verification handling both old and new processes' assertions (in different address spaces) at the time of the switch.
}{
While our logic was developed and proven sound for x86-64 assembly,
Section \ref{sec:traversingC} described verification of software page table walking code (\lstinline|pte_get_next_table| and \lstinline|walkpgdir|)
as if at the level of C for improved readability.
This appendix describes the actual assembly-level verification carried out in Rocq.
Careful readers of both Section \ref{sec:traversingC} and this appendix will notice
strong similarities in the assertions and and reasoning, for good reason:
The C code in Section \ref{sec:traversingC} was the original kernel code that was compiled
(with no optimizations) to x86-64 assembly and verified with our logic, and the proof outlines
in that section largely back-port the assembly proofs back to C.
\looseness=-1
}

\add{
 This section describes the assembly proofs without reference to the C outlines given in Section \ref{sec:traversingC}.
 The main additional details of note at the assembly level are:
 \begin{itemize}
 \item Accurate treatment of register management (particularly the AMD64 System V calling convention) leading to more direct correspondence
       with our logic
 \item The assembly is naturally more verbose than the C, so the proof outlines are relatively more sparse, with assertions written
       only for key updates.
 \item Bitwise manipulations of page table entries are harder to follow than C's bitfield access support.
       Multiple manipulations which are each explicit in C become adjacent (or sometimes non-adjacent) bitwise operations.
       The critical ones are commented in the assembly figures.
 \item And compared to the C-based presentation earlier, there are differences in logical variable names. For example,
       the assembly proofs use \textsf{entry} as the name for the \emph{physical} address of the entry modified by
       \lstinline|pte_get_next_table| in Figure \ref{fig:calltopteinitialize}, whereas to make sense of the C code
       in Figure \ref{fig:calltopteinitializeC} we used \textsf{entry} consistently with the C variable name and introduced
       separate logical names for physical addresses. This propagates to figures presenting larger invariants separately,
       as they also refer to the logical names from the proofs.
 \end{itemize}
}

%\begin{comment}
%\todo[inline]{Identity mappings are difficult, and our current approach won't quite work. Consider trying to have a virtual pointsto for an actual page table entry (i.e., that one could use to update a page table mapping), while also having a virtual pointsto for an address that entry mapped. With the current (let's call it v1) solution, we can't actually have both of those simultaneously!  That's because the PTE pointsto will assert full ownership of the physical memory cell holding the PTE as its data value, while the virtual pointsto for the data mapped by that entry will \emph{also} assert (fractional) ownership of all entries a page table walk would traverse.
%}
%\todo[inline,color=violet]{This doesn't seem to cause issues with the mapping/unmapping examples, only with changing intermediate page table pointers. The mapping example requires a virtual pointsto for the blank PTE, and once filled in that ownership can be immediately split to create the 512 new virtual pointsto assertions for the newly mapped page. Conversely, for unmapping we'd assume ownership of all the relevant virtual pointsto assertions for the page we're unmapping, at which point we can (with a bit of work) show that they all correspond to the same L1 PTE, and extract the 512 fractional shares of that entry from the pointsto assertions.  But changing intermediate page tables, as one would do for coallescing or splitting a superpage while preserving the virtual-to-physical mappings, couldn't be done without some really complicated separating implication tricks.}
%\todo[inline,color=green]{One possible approach to resolving this, which we came up with in our Tuesday meeting, is to recognize that the current (v1) virtual points-to is too strong, because it really doesn't care about \emph{owning} those fractional resources, it only cares that \emph{something} ensures the correct page table walk exists. Iris has a ghost map resource where authoritative ownership of an individual key-value pair can be handled as a resource.  (Colin was using this in the filesystem cache.)
%We can use that mechanism to separate the virtual-to-physical translation from the physical memory involved (Kolanski and Klein may have done something similar for different reasons): (fractional) virtual points-to assertions can be defined in terms of (fractional) ownership of these authoritative ghost map entry assertions, plus sharing an invariant that the current installed page table respects all entries of the mapping. Unmapping collects the authoritative map kvpairs from collecting the assertions, and then can remove them from the ghost map and update the page tables. Critically, physical ownership of the page tables then lives in the invariant on the current page table, so some virtual pointsto assertions can refer to memory in those page tables.
%This still works with the modality, since that invariant is also semantically a predicate on a page table root.
%Let's call this v2.
%}
%\end{comment}
\subsection{Traversing Live Page Tables}
\label{sec:traversing}
We build up to the main task of mapping a new page after traversing page tables in software.
The mapping operation of Figure \ref{fig:mapping_code} assumes an operation \textsf{walkpgdir} which must traverse the page tables
in order to locate the address of the L1 entry to update --- 
% possibly allocating tables for levels 3, 2, and 1 in the process,
% installing them into levels 4, 3, and 2, along the way.
possibly allocating new L3, L2, and L1 tables as necessary.
Traversing the page tables is itself challenging functionality to verify: loading the current table root from \lstinline|cr3| is straightforward
(a \lstinline|mov| instruction), however this produces the physical address of \lstinline|cr3|, not the virtual address the kernel code would use to access that memory.
This problem repeats at each level of the page table: assuming the code has \emph{somehow} read the appropriate L4 (or L3, or L2) entry, those entries again
yield physical addresses, not virtual.

\subsubsection{Loading Page-Table Address Value}
We will discuss access to the level 4 table later (Section \ref{wlkpgdir}). But for subsequent levels, the base address of level $n$ must be
fetched from the appropriate entry in the level $n+1$ table.
This is the role of \lstinline|pte_get_next_table| (Figures \ref{fig:calltopteinitialize} and \ref{fig:p2v}):
it is passed the virtual address of the page table entry in level $n+1$, and should return the \emph{virtual} 
address of the \emph{base} of the level $n$ table
indicated by that entry.
If the entry is empty (i.e., this is a sparse part of the page table representation),
the code also allocates a page for the level $n$ table, installs it in the level $n+1$ entry, and establishes appropriate invariants.
Figure \ref{fig:calltopteinitialize} presents the initial part of the function, which performs the allocation if necessary.
Figure \ref{fig:p2v} (discussed in Section \ref{sec:p2v}) deals with the cases where no allocation is necessary \emph{or} the allocation has already
been performed by the code in this figure.
\looseness=-1

Note that the specification does \emph{not} assume a specific page table level --- logical parameter \textsf{v} represents the level
of the entry passed as an argument, and this code
is used for all three level transitions when traversing page tables (4 to 3, 3 to 2, 2 to 1).
This comes into play with a subtlety of the specification of \lstinline|pte_get_next_table| that we will
revisit several times: \lstinline|pte_get_next_table|'s specification
assumes it is given a virtual \emph{vpte-pointsto}
(a virtual points-to exposing the underlying physical address instead of existentially quantifying it;
 see Section \ref{sec:mapnew}) granting access to the specified entry,
but its postcondition does not yield new virtual points-to assertions!
Instead it merely computes the base virtual address of the next table, and returns adequate capabilities (discussed in Section \ref{subsec:identitymappings})
for the \emph{caller} to construct a vpte-pointsto for any entry of the next table level (if this is not an L1 entry ---
the caller knows which level of the table this is for).
\looseness=-1

Within \textsf{get\_next\_table}, after a standard function prologue, the code 
loads the entry pointed to by the argument (logical variable \textsf{entry} in the proof outline).
This is a page table entry: a 64-bit word divided into bit-fields for
the physical address of the next table, and control bits like the valid bit, as discussed in 
Section \ref{sec:backgroundonmachinemodel}.



\ifPLDI
Line \ref{line:mask_present} checks % In the condensed figure, it's all on one line
\else
Lines \ref{line:mask_present}--\ref{line:check_entry_present} check
\fi
if the entry's ``present'' bit is set.
If it is zero, a new page must be allocated for the next level of the table --- which is done by the fall-through
from Line \ref{line:check_entry_present_jump}'s conditional jump. Otherwise the code jumps ahead to
the case for the next level already existing, which is discussed in Section \ref{sec:p2v} and Figure \ref{fig:p2v}.
First, we must discuss another refinement of the address space invariant, establishing
enough structure on the page tables themselves to allow the traversal.
The code for allocating a new level of the page table must establish this extended invariant.

%wshiftll (wshiftll (natToWord 64 entry) (WordImpl.concat (WordImpl.zero 56) (WordImpl.from_nat 8 12 ^& WordImpl.concat (WordImpl.zero 2) WO~1~1~1~1~1~1)) ^& constf)
%(WordImpl.concat (WordImpl.zero 56) (natToWord 8 12 ^& WordImpl.concat (WordImpl.zero 2) WO~1~1~1~1~1~1))
%
%wshiftll
 %      (wshiftll
%          ((((natToWord 64 entry ^& WordImpl.concat (WordImpl.zero 32) consta ^| WordImpl.concat (WordImpl.zero 32) (natToWord 32 2))
%             ^& WordImpl.concat (WordImpl.zero 32) constb ^| WordImpl.concat (WordImpl.zero 32) (natToWord 32 4)) ^& constd
%            ^| wshiftll
%                 (wshiftll (nextpaddr ^+ ^~ (natToWord 64 KERNBASE))
%                    (WordImpl.concat (WordImpl.zero 56) (WordImpl.from_nat 8 12 ^& WordImpl.concat (WordImpl.zero 2) WO~1~1~1~1~1~1))
%                  ^& constf)
%                 (WordImpl.concat (WordImpl.zero 56) (WordImpl.from_nat 8 12 ^& WordImpl.concat (WordImpl.zero 2) WO~1~1~1~1~1~1)))
%           ^& WordImpl.concat (WordImpl.zero 32) conste ^| wone 64)
%          (WordImpl.concat (WordImpl.zero 56) (WordImpl.from_nat 8 12 ^& WordImpl.concat (WordImpl.zero 2) WO~1~1~1~1~1~1)) ^& constf)
%       (WordImpl.concat (WordImpl.zero 56) (natToWord 8 12 ^& WordImpl.concat (WordImpl.zero 2) WO~1~1~1~1~1~1)) 
% Figure environment removed

\subsubsection{Identity Mappings}
\label{subsec:identitymappings}
Kernels need to convert between physical and virtual addresses, in both directions.
Traversing the page tables in software is the simplest way to convert a virtual address to a physical address; this is the context we are working up to.
However, implementing this virtual-to-physical (V2P) translation in this way ironically requires physical-to-virtual (P2V) translation,
because the addresses stored in page table entries are physical, but memory accesses issued by the OS code use virtual addresses.
% There is no universal way to convert physical addresses to virtual --- doing so relies on the kernel maintaining careful invariants or
% additional data structures to enable P2V translation.
\looseness=-1

Because VMM operations are performance-critical for many workloads, most kernels 
maintain invariants that enable very fast P2V conversions (rather than adding another data structure).
Most kernels maintain an invariant on their page tables that the virtual address of any page used for a page table 
% lives at a virtual address whose value 
is \emph{a constant offset from the physical address} --- a practice sometimes referred to as \emph{identity mapping} 
(even though the physical-to-virtual translation
is typically not literally the identity function, but adding a non-zero constant offset).\footnote{Some kernels do this for all physical memory on the machine, simplifying interaction
with DMA devices.
On newer platforms like RISC-V, this sometimes truly is an identity mapping ---
x86-64 machines are forced into offsets by backwards compatibility with bootloaders that cannot access the full memory space of the
machine.
}

For this reason we extend the per-address-space invariant as in Figure \ref{fig:peraspaceinvariant_with_p2v_extension}, to also track which
addresses we can perform a P2V conversion on by a adding a constant offset.
$\Xi$ is another ghost map, from physical addresses to the level of the page table they represent (1--4).
\emph{Only} physical addresses in $\Xi$ can undergo P2V conversion. 
Section \ref{sec:p2v} describes the actual conversion,
but we describe the invariant here 
because adding new level 3/2/1 tables must maintain the invariant.

% Figure environment removed

For each $\paddr\mapsto \textsf{v} \in\Xi$, the invariant tracks a virtual points-to justifying that virtual address $\paddr+\textsf{KERNBASE}$ maps to physical address $\paddr$
(the ``Ghost translation'' in Figure \ref{fig:peraspaceinvariant_with_p2v_extension});
fractional ownership of the physical memory for that page table entry;
and for valid entries (with the present bit set) above L1, ghost map tokens for every entry in the table pointed to by the entry, which can be used
to repeat the process one level down. 
% (L1 entries point to data pages, whose physical memory ownership resides in some virtual points-to).
The assertion on Line \ref{line:conditional_children} of Figure \ref{fig:calltopteinitialize} comes from the invariant one level up; 
if the valid bit is set,
the code can return those child tokens without the conditional guard.
\looseness=-1

The fractional ownership of the entry's physical memory is subtle. Recall that $\textsf{L}_{4}\_\textsf{L}_{1}\_\textsf{PointsTo}$ retains some physical
ownership of each page table entry that is traversed (proportional to how many virtual addresses share the entry).
So in general the invariant cannot keep full permission to the memory in this part of the invariant, or it would overlap the page table walk for virtual points-to
assertions. But in the case where the entry is invalid, we may need to write to it (e.g., to install a reference to a next-level table, as we do in Figure \ref{fig:calltopteinitialize}),
which requires full permission. Fortunately, the entry can only be in use if its valid bit is set; if the valid bit is not set we know
that no virtual points-to entry in $\delta$/$\theta$ holds any partial ownership.
Thus we use the invariant portion annotated as ``Entry validity'' in Figure \ref{fig:peraspaceinvariant_with_p2v_extension} to capture this:
if the entry is invalid the invariant holds full ownership of the entry, so it can be updated; while if the entry is valid,
the invariant owns only a constant non-zero fragment sufficient to read the entry, but not modify it (which would invalidate some virtual points-to assertions):
\begin{equation*}
 \ulcorner \textsf{qfrac} = 1 \leftrightarrow \; \lnot\textsf{entry\_present }(\vale) \urcorner \tag{*}
\end{equation*}
Thus the fractional ownership of the physical location is enough for Line \ref{line:read_entry_contents} in Figure \ref{fig:calltopteinitialize} to access the entry, though in \lstinline|get_next_table|
the caller has pulled that piece of information out of the invariant and passed it for the entry at hand.
This removal appears explicitly in assertions,
as the argument to the invariant is $\Xi\setminus\{\mathsf{entry}\}$ (indexing by the set $\Xi$ allows us to borrow the physical resources
for a specific page table entry out of the invariant, and later put them back).
Line \ref{line:check_entry_present_jump}'s conditional then determines in the fall-through case that the bit is not set, which 
together with other facts entails $\textsf{qfrac} = 1$ at Line \ref{line:after_concluding_qfrac1},
and permits storing a new entry (in ellided code around Line \ref{line:install_new_entry}).
\looseness=-1

This seemingly-simple piece of code has a highly non-trivial correctness argument, which depends critically on detailed invariants on how access to page table
entries is shared between parts of the kernel. No prior work has engaged with this problem.

% Concretely speaking, going back to Line 15 in Figure \ref{fig:calltopteinitialize}, to read the value referenced by physical address \textsf{entry} while preserving the soundness of memory mappings, our extended invariant introduces the side condition (*)
% \begin{equation*}
%  \ulcorner \textsf{qfrac} = 1 \leftrightarrow \; \lnot\textsf{entry\_present }(\vale) \urcorner \tag{*}
% \end{equation*}
% assuring that looking the identity mapping for \textsf{entry} is safe under the subtle justification which equates the full ownership to the non/presence of the entry which can only be known when investigated in Line 21 in Figure \ref{fig:calltopteinitialize}.
\begin{comment}
 % Figure environment removed
\end{comment}


 \subsubsection{Installing a New Table}
 After obtaining the identity mapping for \textsf{entry}, we are able to load the \textsf{entry\_val} into \textsf{rdi}, and check the presence bit through
\ifPLDI
Line \ref{line:mask_present} % in condensed version, all on same line
\else
Lines \ref{line:mask_present}--\ref{line:check_entry_present} 
\fi
in Figure \ref{fig:calltopteinitialize}.
Accessing the presence bit and checking the value allows us to exploit the condition (*) that was just discussed when verifying the allocation
path (i.e., when the entry is invalid  and Lines \ref{line:alloc_path_start}--\ref{line:alloc_path_end} in Figure \ref{fig:calltopteinitialize}
must allocate the next level of tables).
This operation is subtle. To reiterate: the operation requires that the relevant table entry is readable, but the exact portion of ownership 
returned must be determined by inspecting the valid bit of the value in memory --- so full ownership is returned only for unused entries.
When the bit is not set, that entails full ownership of the entry's memory ($\textsf{qfrac} = 1$) and justifies writing to that memory.
Otherwise, the code jumps past the end of this listing, to the following code at the top of Figure \ref{fig:p2v} (which is also the
continuation of this code).

% Figure environment removed

If the entry is not set, \textsf{pte\_initialize} (Line \ref{line:call_to_pte_initialize} in Figure \ref{fig:calltopteinitialize}) 
allocates a physical page (internally utilizing the only unverified (trusted) code in our case studies, the page-allocator's \textsf{kalloc},\footnote{
  This is an allocator for regions of pre-zeroed physical memory that is mapped, but not accessed by the allocator itself,
  as is typical for slab allocators~\cite{bonwick1994slab}.
  Its verification would be similar to verifying a usermode \textsf{malloc} verifications~\cite{Chlipala2013Bedrock,wickerson2010explicit},
  just with additional invariants on the memory pool.
} 
on Line \ref{line:call_to_kalloc} in Figure \ref{pteinitializespec}). 
Since we are using \textsf{pte\_initialize} for page-table address allocation, we must relate this newly
allocated physical address to the identity mapping map $\Xi$ --- 
see Line \ref{line:page_of_caps} in Figure \ref{fig:calltopteinitialize}, where
\texttt{kalloc}'s specification guarantees it has returned memory from a designated memory
pool that is already mapped
\ifPLDI
\else
\footnote{A reasonable reader might wonder where this pool
initially comes from, and how it might grow when needed. Typically an initial mapping subject to this identity mapping
constraint is set up prior to transition to 64-bit kernel code (notably,
a page table must exist \emph{before} virtual memory is enabled during boot, as part of enabling it is setting
a page table root).
Growing this pool later requires cooperation of physical memory range allocation and virtual memory range allocation,
typically by starting general virtual address allocation at the highest physical memory address plus the identity mapping offset.
This reserves the virtual addresses corresponding to all physical addresses plus the offset for later use in this pool,
as needed.
} 
\fi
and satisfies the offset invariants.
% \todo[inline,color=blue]{colin frontier.
% Stuck with line 31 onwards in Figure 7. rax holds nextpaddr, but I think that should be entrypfn, and 
% the explicit entrypfn id token assertion should go away, as its covered by the forall assertion.
% then the postcondition for pte-initialize should have a specific level now for the entries,
% like 0, which can be updated in the view shift on line 42.
% }
% Focusing on the specification of \textsf{pte\_initialize} separately in Figure \ref{fig:pteinitializespec}, 
% we right immediately realize that instead of seeing see a physical pointsto for the fresly page-table address 
% (e.g. $\mathsf{nextpaddr} \mapsto_{\mathsf{p}} \mathsf{w64\_0}$) deliberately in the post-conditoin in Lines 15-16,
%  we observe a full-ownership token representing the knowledge that a frame and all the entries indexed from this 
% frame are freshly allocated with full-ownership to be a part of the identity map, $\Xi$. 
The soundness argument of this specification relies on the fact that these freshly allocated resources are part 
of an entry construction that has not been completed yet: the presence bit is set 
(Line \ref{line:install_new_entry} in Figure \ref{fig:calltopteinitialize}) after these freshly allocated resources are incorporated to the 
entry construction via the page-frame portion of the PTE. In other words, the side condition, (*),
 formalizes that any access to the entry with these resources is \textit{invalid} (in the sense of not necessarily
having accompanying resources) until the entry is marked present (and thus the memory returned from \textsf{kalloc}
moves into the page table invariant.

\add{Note that the C presentation in Figure \ref{fig:calltopteinitializeC}
omitted the precondition on the implication of Figure \ref{fig:calltopteinitialize}'s Line \ref{line:page_of_caps},
which is logically equivalent to \textsf{True} since \textsf{entry\_present} checks if the present bit is set in an entry,
and \textsf{pte\_initialize} sets that bit. The actual invariant has this form here, and in the postcondition
of \lstinline|pte_initialize| (Figure \ref{pteinitializespec}), to match the conditional form from earlier in
\lstinline|pte_get_next_table| (which is also provably true when the check of the present bit
determines that the entry was already valid/present).
Our proof discharges the conditional at the join point, rather than eagerly in each branch.
}

\subsubsection{Physical-to-Virtual Conversion with \textsf{P2V}}
\label{sec:p2v}
Once we know the entry refers to a physical address in the identity mapping range ($\Xi$)
(via the branch at Line \ref{line:check_entry_present_jump}, or  by allocating and installing a new entry
as just discussed for Lines \ref{line:check_entry_present_jump}--\ref{line:end_of_allocation_path}), 
we can convert this frame address to a corresponding virtual address via the identity mappings
discussed in Section \ref{subsec:identitymappings} and Figure \ref{fig:peraspaceinvariant_with_p2v_extension}.
in the last lines of \lstinline|pte_get_next_table| shown in Figure \ref{fig:p2v} (the continuation of Figure \ref{fig:calltopteinitialize}).
This is a critical piece of the full page table walk verification.
In our small kernel (Line \ref{line:p2v} in Figure \ref{fig:p2v}), as in larger kernels, the C macro \texttt{P2V} common to many kernels
is actually just addition by the constant offset mentioned in Section \ref{subsec:identitymappings}.
But the correctness of this simple instruction is quite subtle.
%  and cannot be proven 
% without the extended invariant (Figure \ref{fig:peraspaceinvariant_with_p2v_extension})
% worked out Section \ref{subsec:identitymappings}.

% Figure environment removed
Figure \ref{fig:p2v} shows the verification of the end of \lstinline|pte_get_next_table| specialized to the case where 
where no allocation was necessary (i.e., the conditional on Line \ref{line:check_entry_present} of Figure \ref{fig:calltopteinitialize} was taken).
In this case, the true present bit allows access to the child tokens from Line \ref{line:conditional_children} of Figure \ref{fig:calltopteinitialize},
which is then refined to the assertion on Line \ref{line:children} of Figure \ref{fig:p2v}.
The code loads \lstinline|rcx| with the offset value \textsf{KERNBASE}, which gives us the value of the virtual address ($\textsf{entry}_{\textsf{pfn}}$ \textsf{+KERNBASE})
of the \emph{base} of the next level of the page table.
% \todo[inline]{the next sentence depends on having figure 10 updated to reflect the page-worth of tokens}
While we could now convert this address to a virtual points-to, this is not necessarily the correct thing to do.
The caller \lstinline|walkpgdir| (discussed next) uses \lstinline|pte_get_next_table| to retrieve just the base address,
because only the caller knows which entry in the subsequent table will be accessed (it depends on the corresponding bits from the virtual
address being translated). So instead we pass back the per-address-space invariant with the identity mapping resources for \lstinline|entry|
pulled out. The caller determines which entry in that table must actually
be accessed --- by selecting the appropriate index into the 512 ghost map tokens returned in the postcondition,
and using the ghost translation and physical location portions of the invariant to assemble a vpte-pointsto
that justifies the caller's subsequent access to a particular entry of the returned table.
% in the identity map ($\Xi\setminus\{entry\}$) of the kernel invariant.
% the logical update in Specification  Lines 5-10 to 10-14 for obtaining virtual-pointsto resource for the frame 
% ($\textsf{entry}_{\textsf{pfn}}$) by removing it from the ghost map ($\Xi\setminus\{entry\}\cup \{\textsf{entry}_{\textsf{pfn}}) \}$) 
% in Line 5 and compute the identity mapping for this physical frame address in Line 13 in Figure \ref{fig:p2v}).

\subsubsection{Walking Page-Table Tree: Calling \textsf{pte\_get\_next\_table} for Each Level}
\label{wlkpgdir}
% Figure environment removed

% Figure environment removed
Implementing a software page-table walk amounts to calling \textsf{pte\_get\_next\_table} for each level as shown in Figure \ref{walkpgdir}. 
The key part of the specification and proof for a page table walk is accumulation of memory mappings for the page-table entries 
visited and frame addresses for page-tables. 
For example, Lines \ref{line:ex_l4_vpte} and \ref{line:ex_l3_vpte} in Figure \ref{walkpgdir} show the virtual pte-pointsto assertions for L4 and L3 entries.
In the final post-condition, we expect the accumulation of these resources from each level -- $\textsf{R}_{\textsf{walk}}$ -- 
which allows us to construct and return the path to the L1 entry in the tree to insert a new page.  

This is the code which performs most actual physical-to-virtual conversions using the identity mapping portion of the per-address-space invariant.
\lstinline|walkpgdir| accepts a \emph{virtual} pointer to the base of the L4 table, and the address to translate.
The precondition provides knowledge that the virtual base of the L4 is at the appropriate offset from the current \lstinline|cr3| value,
but does not provide a virtual points-to assertion --- because the function must calculate (Lines \ref{line:start_pml4_calc}--\ref{line:end_pml4_calc})
which entry it needs access to.
Instead the precondition has 512 identity map tokens, guaranteeing that every entry on the page is subject to the identity mapping invariant.
Line \ref{line:end_pml4_calc} calculates the virtual address of the relevant entry, and the subsequent view shift
pulls that entry out of the identity mapping ($\Xi$) and fetches its corresponding resources as
described by Figure \ref{fig:peraspaceinvariant_with_p2v_extension} and Section \ref{subsec:identitymappings}.
The ghost translation and physical location are used to form the virtual pte-pointsto for the L4 entry
(Line \ref{line:first_pte_pointsto}), with the entry validity and next-level indexing
satisfying the rest of the precondition for \lstinline|pte_get_next_table|.
\lstinline|pte_get_next_table| then, as described earlier, checks the valid bit in the indicated
entry and either returns the (unconditional) tokens for the L3 entry physical addresses (if valid), or
allocates into the entry and returns new (also unconditional) tokens for the L3 entry physical addresses.
\lstinline|pte_get_next_table|'s first call (Line \ref{line:first_getnext_call}) returns
the virtual address of the base of the L3 table (a \emph{page directory pointer}, so PDP, in official
x86-64 terminology). Then the situation to move from that pointer to the base of the L2
is just like the process just followed: the proof calculates the address of the relevant
L3 entry, uses the appropriate L3 identity mapping token to construct a virtual pte-pointsto to that entry,
and passes that along with additional resources pulled out of the invariant to another call to
\lstinline|pte_get_next_table|. That call then returns the base of an L2 table, and the process
repeats until the function returns the virtual address of the relevant L1 entry.
That will then be used in the next section by the caller of \lstinline|walkpgdir|
to install a new mapping.


% \textsf{walkpgdir}, as a client, holds the knowledge that there exists an identity mapping for the physical entry address (\textsf{entry})
%  in the root page table ($\textsf{L}_{4}$):  $\mathsf{entry} \mapsto_{\textsf{id}} \textsf{\_}$ in Specification Line 3 is a partially owned
%  token for accessing and looking up the resources in the identity map, $\Xi$, to construct the \textit{virtual-to-physical} pointsto relation 
% $\textsf{entry+KERNBASE} \mapsto_{\textsf{vpte,qfrac}} \textsf{entry \entry\_val}$ with the virtual address (\textsf{entry+KERNBASE}) obtained 
% by offsetting the physical address (\textsf{entry}). With this knowledge on the root-page-table-entry, we can start traversing the page-table 
% tree which requires locating the address of the next table -- a call to \textsf{pte\_get\_next\_table} shown in Figure \ref{fig:calltopteinitialize}. 
% Beyond a frame, the precondition before Line 15 requires the current address space invariant, and knowledge that \textsf{entry} is mapped to a 
% random entry value, subtly, 
% the operation also, at least, requires that the relevant table entry is readable, but the exact portion of ownership 
% returned must be determined by inspecting the valid bit
% of the value in memory --- so full ownership is returned only for unused entries.
% This is a simple piece of code whose functionality is critical and whose correctness is highly non-trivial. No prior work engages with this problem.



%% Figure environment removed



%\caption{Traversing page-tables, and allocating entries as needed while mapping-a-page in Figure \ref{fig:mappingcode}.}
% \citet{kolanski08vstte,kolanski09tphols} verified a single code block with their logic which was roughly Figure \ref{fig:mapping_code} for a 2-level ARM
% page table, but several critical complexities our work deals with were not addressed.
% First, beyond the limitations discussed in Section \ref{sec:overly-restrictive}, Kolanski and Klein assumed that virtual addresses
% for page tables at each level were given as parameters rather than verifying any conversion from physical addresses to virtual addresses (or even axiomatizing their lookup).
% In contrast, our verification articulates the address space invariant from which the physical-to-virtual translation can be implemented.
% Second, our proof deals with the construction of a valid virtual points-to \emph{to the PTE to update} in mapping, which Kolanski and Klein also
% assumed was given.
% \todo{some of this is really an argument for our verification being more thorough, rather than being about our logic}

% Reasoning about the page table walk in their logic would have required 
% could reason about the walk, but would need to explicitly prove that all other invariants
% of the kernel, the current address space, and all other address spaces of interest were preserved by each update, because their model
% only supports separation within a single address space. In our model, this follows for free from making
% our separation logic directly aware of address translation and internalizing assumptions about other address spaces as further separable assertions.
% Kolanski and Klein did address part of the walk information for a 2-level page table (a possible ARM configuration), but 

% \textsc{seL4} currently still trusts address translations; it models page tables as a data structure in regular memory, thus not capturing the possibility that even
% temporarily destroying the mappings and restoring them can actually crash the OS. \textsc{CertiKOS} papers share little in the way of precise details about
% their virtual memory management, but because their core technology is based on a fork of \textsc{CompCert}, whose model of memory is
% a set of unordered block allocations, we can infer their proofs must also trust these translations.


\subsection{Mapping a New Page}
\label{sec:mapnew}
One of the key tasks of a page fault handler in a general-purpose OS kernel is
to map new pages into an address space by writing into an existing page table via a call\\
\centerline{\textsf{vaspace\_mappage(pte\_t *pml4, void *va,uintptr\_t fpaddr)}}\\
in Figure \ref{fig:mapping_code}.
To do so, with a given allocated a fresh page (\textsf{fpaddr}), then calculate the appropriate
known-valid page table walks (via \textsf{walkpgdir} Line \ref{line:call_walkpgdir} in Figure \ref{fig:mapping_code})  and update 
the appropriate L1 page table entry (Line 35 in Figure \ref{fig:mapping_code});
unmapping is the reverse of the logic we discuss here.
\looseness=-1
%\lstset{
%  columns=fullflexible,
%  numbers=left,
%  basicstyle=\ttfamily,
%  keywordstyle=\color{blue}\bfseries,
%  morekeywords={mov,add,call},
%  emph={rsp,rdx,rax,rbx,rbp,rsi,rdi,rcx,r8,r9,r10,r11,r12,r13,r14,r15},
%  emphstyle=\color{green},
%  emph={[2]cr3},
%  emphstyle={[2]\color{violet}},
%  morecomment=[l]{;;},
%  mathescape
%}
% Figure environment removed

In Figure \ref{fig:mapping_code}, we see an address ($\vaddr$) currently not
mapped to a page ($\theta \; !!\; \vaddr = \texttt{None}$). Mapping a fresh
physical page to back the desired virtual page first requires ensuring
the existence of a memory location for an appropriate L1 table entry.
The code uses a helper function \lstinline{walkpgdir} (discussed again in Section \ref{sec:traversing}).
\textsf{walkpgdir}'s postcondition contains virtual \emph{PTE} pointsto assertions ($\mapsto_{\textsf{vpte}}$)
both for ensuring partial page table walk reaching the
L1 entry (l1e) by asserting that higher levels of the page table exist (R$_{\textsf{walk}}$ in Figure \ref{fig:rwalk}), 
and for allowing access to the memory of the L1 entry via virtual address (R$_{\textsf{l1e}}$ in Figure \ref{fig:rwalk}).

% After obtaining a virtual address \textsf{pte\_addr} in \textsf{rax} backed 
% by the physical memory for the L1 entry that will be used to translate the virtual addresses
% we are mapping, we save it to \textsf{r14} to be updated later in Line 9.

%In the precondition, we see Line 12 allocates a fresh page-aligned, zero-initialized page  (at \textsf{fpaddr}),
%returning a pre-filled PTE entry in \textsf{rax} ($+3$ sets the lower 2 bits).

% , to hold the freshly
% allocated physical page address (\textsf{fpaddr}) in Line X.

We already discussed for the upper level page-tables how the entry-present checks are handled.
However, for L1 entries this check is left to the caller of the 
traversal function \textsf{walkpgdir}. In other words, unlike what we see in R$_{\textsf{walk}}$ for the upper levels where all entry-present
checks have already been performed, the specification in R$_{\textsf{l1e}}$ ensures that page table entry for L1 needs to be checked at the caller site. 
By doing so, as we see in Figure \ref{fig:mapping_code}, the page reference \textsf{fpaddr} is linked to back the virtual address \textsf{va} 
only if it is not already referring to a physical resource (Lines \ref{line:mappage_pte_present_start}--\ref{line:mappage_pte_present_end} in Figure \ref{fig:mapping_code}). 

The crucial step in addition to traversing the page table in Figure \ref{walkpgdir} is actually updating the L1 entry (Line \ref{line:updatepfn} in Figure \ref{fig:mapping_code}),
via the virtual address (\textsf{pt\_entry+KERNBASE}) known to translate to the appropriate physical address, in our example the L1
table entry address ($\textsf{PTE\_ADDR\_TO\_PFN(fpaddr)}$).

Unlike the only prior work verifying analogous code for mapping a new page~\cite{kolanski08vstte,kolanski09tphols}, our proof above
does \emph{not} need to reason directly over the operational semantics,
making this the first verification we know of for mapping a virtual memory page that 
stays entirely at the program logic level.
\looseness=-1
% By incorporating verification of the
% \lstinline|ensure_L1| function (see Section \ref{sec:traversing}), our verification also directly handles several subtle aspects which
% were axiomatized in prior work.
\ifPLDI
\else
\subsection{Unmapping a Page}
\todo[inline]{update (esp. line refs) for new mapping code}
The reverse operation, unmapping a designated page that is currently mapped,
would essentially be the reverse of
the reasoning around line 22 above: given the virtual points-to assertions for all 512
machine words of memory that the L1 entry would map,
and information about the physical location, 
full permission on the L1 entry could be obtained, allowing the construction of a
full virtual PTE pointer for it, setting to 0, and reclaiming the now-unmapped physical memory.
\fi


% % Figure environment removed

% \subsection{Change of Address Space}
% A critical piece of \emph{trusted} code in current verified OS kernels is the assembly code to change the current address space; current verified OS kernels currently 
% lack effective ways to specify and reason about this low-level operation, for reasons outlined in Section \ref{sec:relwork}.

% Figure \ref{fig:swtch} gives simplified code for a basic task switch, the heart of an OS scheduler implementation. This is code that saves the context (registers and stack)
% of the running thread (here in a structure pointed to by \lstinline|rdi|'s value shown in Lines \ref{line:start_save}--\ref{line:end_save} of Figure \ref{fig:swtch}) and restores the context of 
% an existing thread (from \lstinline|rsi| shown in abbreviated Lines \ref{line:start_restore}--\ref{line:end_restore}), including the corresponding change of address space for a target thread in another process.
% This code assumes the System V AMD64 ABI calling convention, where the normal registers not mentioned are caller-save, and therefore saved on the stack of the thread
% that calls this code, as well as on the new stack of the thread that is restored, thus only the callee-save registers and \texttt{cr3} must be 
% restored.\footnote{We are simplifying by ignoring non-integer registers (e.g., floating point, vector registers),
% and the caller-save registers should be initialized to 0 to avoid leaking information across processes, but this captures the key challenges.}
% With the addition of a return instruction, this code would satisfy the C function signature\footnote{This is the function in UNIX 6th Edition 
% with the infamous ``You are not expected to understand this'' comment~\cite{lions1996lions}.}
% \centerline{\lstinline[language=C]|void swtch(context_t* save, context_t* restore);|}\\
% A call to this code begins executing one thread (until just before Line \ref{line:end_save}) in one address space ($\rtv$), whose information will be saved in a structure at address $old$,
% and finishes execution executing a different thread in a different address space (Line \ref{line:end_restore} on) whose information is initially in $new$.

% Because this code does not directly update the instruction pointer, it is worth explaining \emph{how} this switches threads: by switching address spaces and stacks. 
% This is meant to be called with a return address for the current thread stored on the current stack when called. 
% The precondition of the return address on the initial stack requires the callee-save register values at the time of the call: those stored in the first 
% half of the code.
% Likewise, part of the invariant of the stack of the second thread, the one being restored, is that the return address on \emph{that} stack requires the saved 
% callee-save registers stored in that context to be in registers as its precondition.

% The wrinkle, and the importance of the modal treatment of assertions, is that the target thread's precondition is \emph{relative to its address space}, 
% not the address space of the calling thread, which is reflected by
% the other-space modality 
% $[\rtv']( I\texttt{ASpace}(\theta,\Xi,m) \ast \texttt{Pother})$
% in the specfication. 
% The precondition of this code,
% in context, would include that the initial stack pointer (before \lstinline|rsp| is updated)
% has a return address expecting the then-current callee-save register values and 
% suitably updated (i.e., post-return) stack in the \emph{current} (initial) address space;
% this would be part of \textsf{P} in the precondition.
% The specification also requires that
% the stack pointer saved in the context to restore expects the same of the saved registers and stack 
% \emph{in the other address space}. 
% The other-space modality plays a critical role here; \textsf{Pother} would contain these assumptions in the other
% address space.
% \looseness=-1

% % Lines 10--16 save the current context into memory (in the current address space).
% % Line 22 saves the initial page table root.
% % Lines 33--38 begin restoring the target context, including the stack pointer (line 33),
% % which may not be mapped in the address space at that time: it is the stack for the context being
% % loaded into the CPU.
% % The actual address switch occurs on line 45, which is verified with our modal rule for updating \lstinline|cr3|,
% % and thus shifts resources in and out of other-space modalities as appropriate.

% The postcondition is analagous to the precondition, but interpreted \emph{in the new address space}: the then-current (updated) stack would have a return address expecting the new (restored) register values (again, in \textsf{Pother}),
% and the saved context's invariant captures the precondition for restoring its execution \emph{in the previous address space} (as part of \textsf{P}). 

% Immediately after the page table switch, assertions about the saved and restored contexts are
% guarded by a modality for the retiring
% address space \rtv{} (Line \ref{line:modality_switch}), per
% \textsc{WriteToRegCtlFromRegModal} (Figure \ref{fig:wpdamd}),
% because
% there is no guarantee that the data structures of the previous address space are mapped in the new address space.
% The ability to transfer that points-to information out of that modality is specific to a given kernel's design. 
% Kernels that map kernel memory into all address spaces would need invariants
% that justified moving those assertions out of the other-space modality.
% % Following Spectre and Meltdown, this kernel design became less prevalent because speculative execution of accesses to kernel addresses could leak information even if the access did eventually cause a fault (the user/kernel mode permission check was done after fetching data from memory). Thus many modern kernels have reverted to the older kernel design where the kernel inhabits its own unique address space, and user processes have only enough extra material mapped in their address spaces to switch into the kernel (CPUs do not speculate past updates to \texttt{cr3}).
% \looseness=-1

% While prior work has verified context switches within a single address space~\cite{ni2007contexts}, and context switches
% without any code before or after~\cite{syeda2020formal} (i.e., not reasoning about the impact of address space change
% on what data was accessible), this is the first verification handling both.
% \looseness=-1

% \begin{comment}
% \[  
% $\specline{\exists (\entryf ,\;\entrytr,\; \entrytw,\; \entryo,\;\textsf{pte\_addr },\paddr) \; \ldotp\textsf{P} \ast  I\texttt{ASpace}(\theta,m) \ast  \texttt{r14}\mapsto_{\textsf{r}} \_ \ast \texttt{rdi}\mapsto_{r} \vaddr \ast \texttt{rax}\mapsto_{\textsf{r}} \textsf{ pte\_addr} \; \ast }_{\rtv}$
% $\specline{ \ulcorner  \texttt{addr\_L1 }(\vaddr, \entryo) = \paddr \urcorner \ast \ulcorner \texttt{entry\_present } \entryf \land \texttt{entry\_present } \entrytr \land  \texttt{entry\_present } \entrytw \urcorner \; \ast}_{\rtv}$
% $\specline{\nfpointsto{\mask\vaddr\maskfour\rtv}{\mask\vaddr\maskfouroff\rtv}\entryf\qone\naddr \; \ast \nfpointsto{\mask\vaddr\maskthree\entryf}{\mask\vaddr\maskthreeoff\entryf}\entrytr\qtwo\naddr \ast}_{\rtv}$ 
% $\specline{  \nfpointsto{\mask\vaddr\masktwo\entrytr}{\mask\vaddr\masktwooff\entrytr}\paddr\qthree\entryo \;\ast \texttt{pte\_addr} \mapsto_{\texttt{vpte}} \paddr \;(\texttt{wzero 64}) \ast \texttt{rax}\mapsto_{\textsf{r}} \texttt{pte\_addr}  }_{\rtv}$
% mov r14, rax ;; Save that before another call
% $\specline{\textsf{P} \ast  I\texttt{ASpace}(\theta,m) \ast  \texttt{r14}\mapsto_{\textsf{r}} \texttt{pte\_addr} \ast \texttt{rdi}\mapsto_{\textsf{r}} \vaddr \ast \texttt{rax}\mapsto_{\textsf{r}} \textsf{ pte\_addr} \; \ast }_{\rtv}$
% $\specline{ \nfpointsto{\mask\vaddr\maskfour\rtv}{\mask\vaddr\maskfouroff\rtv}\entryf\qone\naddr \ast \ulcorner \texttt{entry\_present } \entryf \land \texttt{entry\_present } \entrytr \land  \texttt{entry\_present } \entrytw \urcorner \ast}_{\rtv}$ 
% $\specline{  \nfpointsto{\mask\vaddr\maskthree\entryf}{\mask\vaddr\maskthreeoff\entryf}\entrytr\qtwo\naddr \ast \nfpointsto{\mask\vaddr\masktwo\entrytr}{\mask\vaddr\masktwooff\entrytr}\paddr\qthree\entryo \;\ast}_{\rtv}$
% $\specline{\texttt{pte\_addr} \mapsto_{\texttt{vpte}} \paddr \;(\texttt{wzero 64}) \ast \texttt{rax}\mapsto_{\textsf{r}} \texttt{pte\_addr}  }_{\rtv}$
% call alloc_phys_page_or_panic
% $\specline{\textsf{P} \ast  I\texttt{ASpace}(\theta,m) \ast  \texttt{r14}\mapsto_{\textsf{r}} \texttt{pte\_addr} \ast \texttt{rdi}\mapsto_{\textsf{r}} \vaddr \;\ast \nfpointsto{\mask\vaddr\maskfour\rtv}{\mask\vaddr\maskfouroff\rtv}\entryf\qone\naddr \ast}_{\rtv}$ 
% $\specline{  \nfpointsto{\mask\vaddr\maskthree\entryf}{\mask\vaddr\maskthreeoff\entryf}\entrytr\qtwo\naddr \ast \nfpointsto{\mask\vaddr\masktwo\entrytr}{\mask\vaddr\masktwooff\entrytr}\paddr\qthree\naddr \;\ast}_{\rtv}$
% $\specline{\texttt{pte\_addr} \mapsto_{\texttt{vpte}} \paddr\; (\texttt{wzero 64})  \ast \ulcorner \texttt{entry\_present } \entryf \land \texttt{entry\_present } \entrytr \land  \texttt{entry\_present } \entrytw \urcorner}_{\rtv}$
% $\specline{\exists \texttt{ fpaddr} \ldotp \ulcorner \texttt{aligned fpaddr} \urcorner \ast \texttt{rax}\mapsto_{\textsf{r}} \texttt{fpaddr+3} \ast \texttt{fpaddr} \mapsto_{\textsf{p}} (\texttt{wzero 64}) \ast \ulcorner \texttt{entry\_present (fpaddr+3)}\urcorner}_{\rtv}$
% ;; Calculate new L1 entry
% mov [r14], rax ;; store the page table entry, mapping the page
% $\specline{\textsf{P} \ast  I\texttt{ASpace}(\theta,m) \ast  \texttt{r14}\mapsto_{\textsf{r}} \texttt{pte\_addr} \ast \texttt{rdi}\mapsto_{\textsf{r}} \vaddr \;\ast \nfpointsto{\mask\vaddr\maskfour\rtv}{\mask\vaddr\maskfouroff\rtv}\entryf\qone\naddr \ast}_{\rtv}$ 
% $\specline{  \nfpointsto{\mask\vaddr\maskthree\entryf}{\mask\vaddr\maskthreeoff\entryf}\entrytr\qtwo\naddr \ast \nfpointsto{\mask\vaddr\masktwo\entrytr}{\mask\vaddr\masktwooff\entrytr}\paddr\qthree\entryo \;\ast}_{\rtv}$
% $\specline{\texttt{pte\_addr} \mapsto_{\texttt{vpte}} \paddr \;(\texttt{fpaddr+3}) \; \ast \ulcorner \texttt{entry\_present } \entryf \land \texttt{entry\_present } \entrytr \land  \texttt{entry\_present } \entrytw \urcorner }_{\rtv}$
% $\specline{\ulcorner \texttt{aligned fpaddr} \urcorner \ast \texttt{rax}\mapsto_{\textsf{r}} \texttt{fpaddr+3} \ast \texttt{fpaddr} \mapsto_{\textsf{p}} (\texttt{wzero 64}) \ast \ulcorner \texttt{entry\_present fpaddr+3}\urcorner}_{\rtv}$
% $\;\;\;\;\;\;\;\;\;\;\;\;\;\;\;\;\;\;\;\;\;\;\;\;\;\;\;\;\;\;\;\;\;\;\;\;\;\;\;\;\;\;\;\; \sqsubseteq $
% $\specline{\textsf{P} \ast  I\texttt{ASpace}(\theta,m) \ast  \texttt{r14}\mapsto_{\textsf{r}} \texttt{pte\_addr} \ast \texttt{rdi}\mapsto_{\textsf{r}} \vaddr \ast }_{\rtv}$
% $\specline{\textsf{L}_{4}\_\textsf{L}_{1}\_\textsf{PointsTo}(\vaddr,\entryf,\entrytr,\entrytw,\fpaddr+3) \ast \ulcorner \theta \;!!\;\vaddr = \texttt{None}\urcorner \; \ast}_{\rtv}$
% $\specline{\ulcorner \texttt{aligned fpaddr} \urcorner \ast \texttt{rax}\mapsto_{\textsf{r}} \texttt{fpaddr+3} \ast \texttt{fpaddr} \mapsto_{\textsf{p}} (\texttt{wzero 64}) }_{\rtv}$
% $\;\;\;\;\;\;\;\;\;\;\;\;\;\;\;\;\;\;\;\;\;\;\;\;\;\;\;\;\;\;\;\;\;\;\;\;\;\;\;\;\;\;\;\; \sqsubseteq $
% $\specline{\textsf{P} \ast  I\texttt{ASpace} (<[\vaddr:=\texttt{fpaddr}]> \theta,m) \ast}_{\rtv}$
% $\specline{\ulcorner \texttt{aligned fpaddr} \urcorner \ast \texttt{fpaddr} \mapsto_{\textsf{p}} \textsf{ wzero 64} \ast \ghostmaptoken{\delta{}s}{\rtv}{\delta}  \ast\sumwalkabs\vaddr\qfrac\fpaddr}_{\rtv}$
% $\;\;\;\;\;\;\;\;\;\;\;\;\;\;\;\;\;\;\;\;\;\;\;\;\;\;\;\;\;\;\;\;\;\;\;\;\;\;\;\;\;\;\;\; \sqsubseteq $
%   $\specline{\textsf{P} \ast  I\texttt{ASpace} (<[\vaddr:=\texttt{fpaddr}]> \theta,m) \ast \vaddr \mapsto_{\textsf{vpte}}\; \{\qfrac\} \;\fpaddr \textsf{ wzero 64}}_{\rtv}$
% $\;\;\;\;\;\;\;\;\;\;\;\;\;\;\;\;\;\;\;\;\;\;\;\;\;\;\;\;\;\;\;\;\;\;\;\;\;\;\;\;\;\;\;\; \sqsubseteq $
% $\specline{\textsf{P} \ast  I\texttt{ASpace} (<[\vaddr:=\texttt{fpaddr}]> \theta,m) \ast \vaddr \mapsto_{\textsf{v}}\; \{\qfrac\} \textsf{wzero 64}}_{\rtv}$
% \end{comment}

%\fi

\end{document}
